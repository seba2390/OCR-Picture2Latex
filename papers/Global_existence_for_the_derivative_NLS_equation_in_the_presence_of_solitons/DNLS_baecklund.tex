% --------------------------------------------------------------
% Article Class (This is a LaTeX2e document)  ********************
% --------------------------------------------------------------
\documentclass[11pt,a4paper]{article}
\usepackage{etex}
\usepackage[english]{babel}
\usepackage{amsmath,amsthm}
\usepackage{amsfonts}
\usepackage{amssymb}
\usepackage{accents}
\usepackage{pstricks-add}
\usepackage{graphicx}
\usepackage{color}
\usepackage{framed}
\usepackage{mathptmx}
\usepackage{enumerate}
\usepackage[arrow, matrix, curve]{xy}
\usepackage{pstricks}
\usepackage{pstricks, pstricks-add, mathptmx, amsfonts, hyperref, color}
\usepackage[left=3cm, right=3cm, top=3cm, bottom=3cm]{geometry}
\usepackage{calrsfs}
\usepackage{multicol}
%\usepackage{mathtools}
%\mathtoolsset{showonlyrefs}
\DeclareMathAlphabet{\pazocal}{OMS}{zplm}{m}{n}
% THEOREMS -------------------------------------------------------
\newtheorem{thm}{Theorem}[section]
\newtheorem{cor}[thm]{Corollary}
\newtheorem{lem}[thm]{Lemma}
\newtheorem{prop}[thm]{Proposition}
\theoremstyle{definition}
\newtheorem{defn}[thm]{Definition}
\newtheorem{rhp}[thm]{Riemann--Hilbert problem}
\theoremstyle{remark}
\newtheorem{rem}[thm]{Remark}
\numberwithin{equation}{section}

\newcommand\pmtwo[4]{\left( \begin{array}{cc}#1&#2\\#3&#4\end{array} \right)}
\newcommand{\norm}[1]{\left\Vert#1\right\Vert}
\newcommand{\abs}[1]{\left\vert#1\right\vert}
\newcommand{\set}[1]{\left\{#1\right\}}
\newcommand{\Real}[0]{\mathbb R}
\newcommand{\Compl}[0]{\mathbb C}
\newcommand{\eps}[0]{\varepsilon}
\newcommand{\ran}{\text{ran }}
\newcommand{\gr}{\text{grad }}
\newcommand{\re}{\,\mathfrak{Re}\,}
\newcommand{\im}{\,\mathfrak{Im}\,}
\newcommand{\ord}{\,\text{ord}\,}
\newcommand{\supp}{\,\text{supp}\,}
\newcommand{\meas}{\,\text{meas}\,}
\newcommand{\tr}{\,\text{tr}\,}
\newcommand{\Div}{\,\text{div}\,}
\newcommand{\rank}{\,\text{rank}\,}
\newcommand{\id}{\,\text{id}\,}
\newcommand{\adj}{\,\text{adj}\,}
\newcommand{\pr}{\,\text{pr}\,}
\newcommand{\Gl}{\,\text{Gl}\,}
\newcommand{\Hess}{\,\text{Hess}}
\newcommand{\ii}{\mathrm{i}}
\newcommand{\sech}{\mathrm{sech}}
\newcommand{\Res}{\mathrm{Res}}
\newcommand{\ind}{\mathrm{ind}\,}
\newcommand{\Int}[1]{\accentset{\circ}{#1}}
\newcommand{\bigslant} [2]{{\raisebox{.2em}{$#1$} \left/\raisebox{-.2em}{$#2$}\right.}}
\newcommand{\dbar}{\overline{\partial}}
\newcommand{\F}{\mathfrak{F}}
\newcommand{\co}{\text{const. }}
\def\res{\mathop{Res}}
\newcommand\sol[2]{\text{sol}^{#1}_{#2}(x,t)}
\newcommand{\Pp}{\pazocal{P}^+}
\newcommand{\Pm}{\pazocal{P}^-}
\newcommand{\Ppm}{\pazocal{P}^{\pm}}
\newcommand{\rh}{Riemann--Hilbert problem }
\renewcommand\bf\bfseries
\begin{document}

\title{Global existence for the derivative NLS equation in the presence of solitons}
%\address{Mathematisches Institut\\
%  Universit\"{a}t zu K\"{o}ln\\
% 50931 K\"{o}ln, Germany}
%\email[A.~Saalmann]{asaalman@math.uni-koeln.de}
\author{Aaron Saalmann\thanks{A.S. gratefully acknowledges financial support from the projects ``Quantum Matter and Materials"
and SFB-TRR 191 ``Symplectic Structures in Geometry, Algebra and Dynamics" (Cologne University, Germany).}
\footnote{Mathematisches Institut, Universit\"{a}t zu K\"{o}ln, 50931 K\"{o}ln, Germany, e-mail: asaalman@math.uni-koeln.de}
}

\date{\today}
\maketitle
  In this paper, we explore the connection between secret key agreement and secure omniscience within the setting of the multiterminal source model with a wiretapper who has side information. While the secret key agreement problem considers the generation of a maximum-rate secret key through public discussion, the secure omniscience problem is concerned with communication protocols for omniscience that minimize the rate of information leakage to the wiretapper. The starting point of our work is a lower bound on the minimum leakage rate for omniscience, $\rl$, in terms of the wiretap secret key capacity, $\wskc$. Our interest is in identifying broad classes of sources for which this lower bound is met with equality, in which case we say that there is a duality between secure omniscience and secret key agreement. We show that this duality holds in the case of certain finite linear source (FLS) models, such as two-terminal FLS models and pairwise independent network models on trees with a linear wiretapper. Duality also holds for any FLS model in which $\wskc$ is achieved by a perfect linear secret key agreement scheme. We conjecture that the duality in fact holds unconditionally for any FLS model. On the negative side, we give an example of a (non-FLS) source model for which duality does not hold if we limit ourselves to communication-for-omniscience protocols with at most two (interactive) communications.  We also address the secure function computation problem and explore the connection between the minimum leakage rate for computing a function and the wiretap secret key capacity.
  
%   Finally, we demonstrate the usefulness of our lower bound on $\rl$ by using it to derive equivalent conditions for the positivity of $\wskc$ in the multiterminal model. This extends a recent result of Gohari, G\"{u}nl\"{u} and Kramer (2020) obtained for the two-user setting.
  
   
%   In this paper, we study the problem of secret key generation through an omniscience achieving communication that minimizes the 
%   leakage rate to a wiretapper who has side information in the setting of multiterminal source model.  We explore this problem by deriving a lower bound on the wiretap secret key capacity $\wskc$ in terms of the minimum leakage rate for omniscience, $\rl$. 
%   %The former quantity is defined to be the maximum secret key rate achievable, and the latter one is defined as the minimum possible leakage rate about the source through an omniscience scheme to a wiretapper. 
%   The main focus of our work is the characterization of the sources for which the lower bound holds with equality \textemdash it is referred to as a duality between secure omniscience and wiretap secret key agreement. For general source models, we show that duality need not hold if we limit to the communication protocols with at most two (interactive) communications. In the case when there is no restriction on the number of communications, whether the duality holds or not is still unknown. However, we resolve this question affirmatively for two-user finite linear sources (FLS) and pairwise independent networks (PIN) defined on trees, a subclass of FLS. Moreover, for these sources, we give a single-letter expression for $\wskc$. Furthermore, in the direction of proving the conjecture that duality holds for all FLS, we show that if $\wskc$ is achieved by a \emph{perfect} secret key agreement scheme for FLS then the duality must hold. All these results mount up the evidence in favor of the conjecture on FLS. Moreover, we demonstrate the usefulness of our lower bound on $\wskc$ in terms of $\rl$ by deriving some equivalent conditions on the positivity of secret key capacity for multiterminal source model. Our result indeed extends the work of Gohari, G\"{u}nl\"{u} and Kramer in two-user case.
\tableofcontents
\newpage
% !TEX root = ../arxiv.tex

Unsupervised domain adaptation (UDA) is a variant of semi-supervised learning \cite{blum1998combining}, where the available unlabelled data comes from a different distribution than the annotated dataset \cite{Ben-DavidBCP06}.
A case in point is to exploit synthetic data, where annotation is more accessible compared to the costly labelling of real-world images \cite{RichterVRK16,RosSMVL16}.
Along with some success in addressing UDA for semantic segmentation \cite{TsaiHSS0C18,VuJBCP19,0001S20,ZouYKW18}, the developed methods are growing increasingly sophisticated and often combine style transfer networks, adversarial training or network ensembles \cite{KimB20a,LiYV19,TsaiSSC19,Yang_2020_ECCV}.
This increase in model complexity impedes reproducibility, potentially slowing further progress.

In this work, we propose a UDA framework reaching state-of-the-art segmentation accuracy (measured by the Intersection-over-Union, IoU) without incurring substantial training efforts.
Toward this goal, we adopt a simple semi-supervised approach, \emph{self-training} \cite{ChenWB11,lee2013pseudo,ZouYKW18}, used in recent works only in conjunction with adversarial training or network ensembles \cite{ChoiKK19,KimB20a,Mei_2020_ECCV,Wang_2020_ECCV,0001S20,Zheng_2020_IJCV,ZhengY20}.
By contrast, we use self-training \emph{standalone}.
Compared to previous self-training methods \cite{ChenLCCCZAS20,Li_2020_ECCV,subhani2020learning,ZouYKW18,ZouYLKW19}, our approach also sidesteps the inconvenience of multiple training rounds, as they often require expert intervention between consecutive rounds.
We train our model using co-evolving pseudo labels end-to-end without such need.

\begin{figure}[t]%
    \centering
    \def\svgwidth{\linewidth}
    \input{figures/preview/bars.pdf_tex}
    \caption{\textbf{Results preview.} Unlike much recent work that combines multiple training paradigms, such as adversarial training and style transfer, our approach retains the modest single-round training complexity of self-training, yet improves the state of the art for adapting semantic segmentation by a significant margin.}
    \label{fig:preview}
\end{figure}

Our method leverages the ubiquitous \emph{data augmentation} techniques from fully supervised learning \cite{deeplabv3plus2018,ZhaoSQWJ17}: photometric jitter, flipping and multi-scale cropping.
We enforce \emph{consistency} of the semantic maps produced by the model across these image perturbations.
The following assumption formalises the key premise:

\myparagraph{Assumption 1.}
Let $f: \mathcal{I} \rightarrow \mathcal{M}$ represent a pixelwise mapping from images $\mathcal{I}$ to semantic output $\mathcal{M}$.
Denote $\rho_{\bm{\epsilon}}: \mathcal{I} \rightarrow \mathcal{I}$ a photometric image transform and, similarly, $\tau_{\bm{\epsilon}'}: \mathcal{I} \rightarrow \mathcal{I}$ a spatial similarity transformation, where $\bm{\epsilon},\bm{\epsilon}'\sim p(\cdot)$ are control variables following some pre-defined density (\eg, $p \equiv \mathcal{N}(0, 1)$).
Then, for any image $I \in \mathcal{I}$, $f$ is \emph{invariant} under $\rho_{\bm{\epsilon}}$ and \emph{equivariant} under $\tau_{\bm{\epsilon}'}$, \ie~$f(\rho_{\bm{\epsilon}}(I)) = f(I)$ and $f(\tau_{\bm{\epsilon}'}(I)) = \tau_{\bm{\epsilon}'}(f(I))$.

\smallskip
\noindent Next, we introduce a training framework using a \emph{momentum network} -- a slowly advancing copy of the original model.
The momentum network provides stable, yet recent targets for model updates, as opposed to the fixed supervision in model distillation \cite{Chen0G18,Zheng_2020_IJCV,ZhengY20}.
We also re-visit the problem of long-tail recognition in the context of generating pseudo labels for self-supervision.
In particular, we maintain an \emph{exponentially moving class prior} used to discount the confidence thresholds for those classes with few samples and increase their relative contribution to the training loss.
Our framework is simple to train, adds moderate computational overhead compared to a fully supervised setup, yet sets a new state of the art on established benchmarks (\cf \cref{fig:preview}).

\section{Direct scattering transform}\label{s scatt}
For a review of the scattering map for the DNLS equation we are going to follow closely \cite{Pelinovsky2016,Liu2016}. As pointed out in the pioneer work \cite{KaupNewell1978}, the DNLS equation is the compatibility condition for solutions $\psi\in\Compl^2$ of the linear system given by
\begin{equation}\label{e Lax1}
    \partial_x\psi=[-\ii\lambda^2\sigma_3+\lambda Q(u)]\psi
\end{equation}
and
\begin{equation}\label{e Lax2}
    \partial_t\psi=[-2\ii\lambda^4\sigma_3+ 2\lambda^3 Q(u)+\ii\lambda^2|u|^2\sigma_3- \lambda |u|^2Q(u) +\ii\lambda \sigma_3 Q(u_x)]\psi,
\end{equation}
where
\begin{equation*}
    Q(u)=
    \left[
      \begin{array}{cc}
        0 & u(x,t) \\
        -\overline{u}(x,t) & 0 \\
      \end{array}
    \right],\qquad
    \sigma_3=
    \left[
      \begin{array}{cc}
        1 & 0 \\
        0 & -1 \\
      \end{array}
    \right].
\end{equation*}
In this context the term \emph{compatibility condition} is chosen, because if the spectral parameter $\lambda$ is independent of $x$ and $t$, it can be shown that the formal equality of the mixed derivatives, $\partial_x\partial_t\psi=\partial_t\partial_x\psi$,  is equivalent to the statement that $u$ solves the DNLS equation (\ref{e dnls}).
\subsection{Jost functions}
It is natural to introduce solutions of (\ref{e Lax1}) which satisfy the same asymptotic behavior at infinity as solutions of the spectral problem (\ref{e Lax1}) in the case of vanishing potential $u\equiv 0$:
\begin{eqnarray*}
% \nonumber to remove numbering (before each equation)
  \psi^{(-)}_1(\lambda;x)\sim
  \left(
    \begin{array}{c}
      1 \\
      0 \\
    \end{array}
  \right)e^{-\ii x\lambda^2}
  ,\qquad
  \psi^{(-)}_2(\lambda;x)\sim
  \left(
    \begin{array}{c}
      0 \\
      1 \\
    \end{array}
  \right)e^{\ii x\lambda^2}
  && \text{ as } x\to -\infty\\
  \psi^{(+)}_1(\lambda;x)\sim
  \left(
    \begin{array}{c}
      1 \\
      0 \\
    \end{array}
  \right)e^{-\ii x\lambda^2}
  ,\qquad
  \psi^{(+)}_2(\lambda;x)\sim
  \left(
    \begin{array}{c}
      0 \\
      1 \\
    \end{array}
  \right)e^{\ii x\lambda^2}
  && \text{ as } x\to+\infty.
\end{eqnarray*}
In order to have constant boundary conditions we introduce the \emph{normalized Jost functions} by
\begin{equation*}
    \varphi_{\pm}(\lambda;x)=\psi^{(\pm)}_1(\lambda;x) e^{\ii x\lambda^2},\qquad \phi_{\pm}(\lambda;x)=\psi^{(\pm)}_2(\lambda;x) e^{-\ii x\lambda^2},
\end{equation*}
such that we have
\begin{equation}\label{e asymptotics psi}
    \lim_{x\to\pm\infty}\varphi_{\pm}(\lambda;x)=e_1\quad\text{ and } \quad\lim_{x\to\pm\infty}\phi_{\pm}(\lambda;x)=e_2,
\end{equation}
where $e_1=(1,0)^T$ and $e_2=(0,1)^T$. The Jost functions are solutions of the following Volterra's integral equations
\begin{equation}\label{e volterra phi varphi}
    \begin{aligned}
        \varphi_{\pm}(\lambda;x)&=e_1 + \lambda \int_{\pm\infty}^x
        \left[
          \begin{array}{cc}
            1 & 0 \\
            0 & e^{2\ii\lambda^2(x-y)} \\
          \end{array}
        \right]
        Q(u(y))\varphi_{\pm}(\lambda;y)dy,\\
        \phi_{\pm}(\lambda;x)&=e_2 + \lambda \int_{\pm\infty}^x
        \left[
          \begin{array}{cc}
            e^{-2\ii\lambda^2(x-y)} & 0 \\
            0 & 1\\
          \end{array}
        \right]
        Q(u(y))\phi_{\pm}(\lambda;y)dy.
    \end{aligned}
\end{equation}
It can be shown that (\ref{e volterra phi varphi}) admit solutions $\varphi_-(\lambda;x)$ and $\phi_+(\lambda;x)$ for $\im(\lambda^2)>0$ and $\varphi_+(\lambda;x)$ and $\phi_-(\lambda;x)$ for $\im(\lambda^2)<0$. Moreover the dependence of $\lambda$ is analytic in the corresponding domains where the Jost functions exist.
However, due to the presence of $\lambda$ that multiplies the matrix $Q(u)$ in the linear equation (\ref{e Lax1}), standard fixed point arguments for (\ref{e volterra phi varphi}) are not uniform in $\lambda$. Therefore, in \cite{Pelinovsky2016} the authors worked out a transformation of the Kaup-Newell type spectral problem (\ref{e Lax1}) to a linear equation of the Zakharov-Shabat type. The idea of that kind of transformation can already be found in \cite{KaupNewell1978}. In what follows we are going to present this transformation and set
\begin{equation}\label{e def T1 Q1}
    T_1(\lambda;x)=
    \left[
      \begin{array}{cc}
        1 & 0 \\
        -\overline{u}(x) & 2\ii\lambda \\
      \end{array}
    \right],\qquad
    Q_1(u)=\frac{1}{2\ii}
    \left[
      \begin{array}{cc}
        |u|^2 & u \\
        -2\ii\overline{u}_x -\overline{u}|u|^2& -|u|^2 \\
      \end{array}
    \right],
\end{equation}
and
\begin{equation}\label{e def T2 Q2}
    T_2(\lambda;x)=
    \left[
      \begin{array}{cc}
        2\ii\lambda &-u(x) \\
        0 & 1 \\
      \end{array}
    \right],\qquad
    Q_2(u)=\frac{1}{2\ii}
    \left[
      \begin{array}{cc}
        |u|^2 & -2\ii u_x -u|u|^2 \\
        -\overline{u}& -|u|^2 \\
      \end{array}
    \right].
\end{equation}
Then, it is elementary to check that $z=\lambda^2$ and
\begin{equation}\label{e def M N}
    M_{\pm}(z;x)=T_1(\lambda;x)\varphi_{\pm} (\lambda;x),\quad
    N_{\pm}(z;x)=T_2(\lambda;x)\phi_{\pm}(\lambda;x)
\end{equation}
make (\ref{e volterra phi varphi}) equivalent to
\begin{equation}\label{e volterra M N}
    \begin{aligned}
        M_{\pm}(z;x)&=e_1 +  \int_{\pm\infty}^x
        \left[
          \begin{array}{cc}
            1 & 0 \\
            0 & e^{2\ii z(x-y)} \\
          \end{array}
        \right]
        Q_1(u(y))M_{\pm}(z;y)dy,\\
        N_{\pm}(z;x)&=e_2 +  \int_{\pm\infty}^x
        \left[
          \begin{array}{cc}
            e^{-2\ii z (x-y)} & 0 \\
            0 & 1\\
          \end{array}
        \right]
        Q_2(u(y))N_{\pm}(z;y)dy.
    \end{aligned}
\end{equation}
Note that the symmetries
\begin{equation}\label{e symmetries phi varphi 1}
    \varphi_{\pm}(\lambda;x)=
    \left[
      \begin{array}{cc}
        1 & 0 \\
        0 & -1 \\
      \end{array}
    \right]
    \varphi_{\pm}(-\lambda;x),\qquad
    \phi_{\pm}(\lambda;x)=
    \left[
      \begin{array}{cc}
        -1 & 0 \\
        0 & 1 \\
      \end{array}
    \right]
    \phi_{\pm}(-\lambda;x)
\end{equation}
make sure that (\ref{e def M N}) is well-defined.
Equations (\ref{e volterra M N}) are analogues to the integral equations known from the forward scattering for the NLS equation (see, e.g., \cite{Ablowitz2004}). If $Q_{1,2}(u)\in L^1(\Real)$, then, (\ref{e volterra phi varphi}) admit solutions $M_-(z;x)$ and $N_+(z;x)$ for $\im(z)>0$ and $M_+(z;x)$ and $N_-(z;x)$ for $\im(z)<0$. Moreover the dependence on $z$ is analytic in the corresponding domains where the Jost functions exist.
\begin{rem}
    The assumption $u\in H^{1,1}(\Real)$ in Theorem \ref{t main} is chosen such that $Q_{1,2}(u)\in L^1(\Real)$.
\end{rem}
Compared to (\ref{e volterra phi varphi}), in (\ref{e volterra M N}) there is no $\lambda$ which multiplies the integral. As a result, the Neumann series for (\ref{e volterra M N}) converge uniformly in $z$.
By means of the asymptotic expansion for large $z$ of the Jost functions, the potential $u$ can be reconstructed from $M_{\pm}$ and $N_{\pm}$, respectively (see \cite[Lemma 2]{Pelinovsky2016}). Furthermore, regularity properties of $M_{\pm}$ and $N_{\pm}$ are used in \cite{Pelinovsky2016} to prove regularity of the reflection coefficient $r_+$ and $r_-$ which we will define in (\ref{e def r pm}) in the next subsection on the Scattering data.
%Therefore we obtain similar properties. In what follows we list the results of Pelinovsky and Shimabukuru (see \cite{Pelinovsky2016}) without proofs:
%\begin{lem}\label{l existence of M N}
%    For $u\in L^1(\Real)\cap L^3(\Real)$ and $u_x\in L^1(\Real)$ and every $z\in\Real$, equations (\ref{e volterra M N}) admit unique solutions $M_{\pm}(z;\cdot)\in L^{\infty}(\Real)$ and $N_{\pm}(z;\cdot)\in L^{\infty}(\Real)$. Moreover, $M_+(\cdot;x)$ and $N_-(\cdot;x)$ are continued analytically in $\Compl^+$, whereas $M_-(\cdot;x)$ and $N_+(\cdot;x)$ are continued analytically in $\Compl^-$. Finally, we have for all $z\in \Compl^{\pm}$
%    \begin{equation*}
%        \|M_{\mp}(z;\cdot)\|_{L^{\infty}}+ \|N_{\pm}(z;\cdot)\|_{L^{\infty}}\leq C,
%    \end{equation*}
%    where the positive constant does not depend on $z$.
%\end{lem}
%\begin{rem}
%    The assumptions of Lemma \ref{l existence of M N} are chosen such that $Q_{1,2}(u)\in L^1(\Real)$.
%\end{rem}
%By means of the asymptotic expansion for large $z$ of the Jost functions, the potential $u$ can be reconstructed from $M_{\pm}$ and $N_{\pm}$, respectively. See \cite{Pelinovsky2016} for the next Lemma.
%\begin{lem}\label{l expansion N M}
%    For $u\in L^1(\Real)\cap L^3(\Real)\cap C^1(\Real)$ and $u_x\in L^1(\Real)$ and every $x\in\Real$ the Jost functions $M_{\pm}(z;x)$ and $N_{\pm}(z;x)$ have the following expansion as $|\im(z)|\to\infty$ along a contour in the domains of their analyticity:
%    \begin{equation}\label{e expansion N M}
%    \begin{aligned}
%        M_{\pm}(z;x)&=&\!\!\!\!
%        \left(
%         \begin{array}{c}
%           \!\!\!\!M^{\infty}_{\pm}(x)\!\!\!\! \\
%           0 \\
%         \end{array}
%        \right)+\frac{1}{z}&
%        \left(
%        \begin{array}{c}
%          -\frac{1}{4}\, M^{\infty}_{\pm}(x) \int_{\pm\infty}^x \left\{u(y)\partial_y \overline{u}(y)+\frac{1}{2\ii}|u(y)|^4\right\} dy \vspace{2mm}\\
%          \frac{1}{2\ii}\partial_x \left(\overline{u}(x) \,M^{\infty}_{\pm}(x)\right) \\
%        \end{array}
%        \right)+\mathcal{O}\left(\frac{1}{z^{2}}\right), \\[8pt]
%        N_{\pm}(z;x)&=&\!\!\!\!
%        \left(
%         \begin{array}{c}
%           0 \\
%           \!\!\!\!N^{\infty}_{\pm}(x)\!\!\!\! \\
%         \end{array}
%        \right)+\frac{1}{z}&
%        \left(
%        \begin{array}{c}
%          -\frac{1}{2\ii}\partial_x \left(u(x) \,N^{\infty}_{\pm}(x)\right)\vspace{2mm} \\
%          \frac{1}{4}\, N^{\infty}_{\pm}(x) \int_{\pm\infty}^x \left\{\overline{u}(y)\partial_y u(y)-\frac{1}{2\ii}|u(y)|^4\right\} dy \\
%        \end{array}
%        \right)+\mathcal{O}\left(\frac{1}{z^{2}}\right),
%    \end{aligned}
%    \end{equation}
%    where
%    \begin{equation}\label{e def M N intfy}
%        M^{\infty}_{\pm}(x):= e^{\frac{1}{2\ii}\int_{\pm\infty}^x|u(y)|^2dy}, \quad N^{\infty}_{\pm}(x):= e^{-\frac{1}{2\ii}\int_{\pm\infty}^x|u(y)|^2dy}.
%    \end{equation}
%    For the zero order expansion, the requirement $u\in C^1(\Real)$ is not necessary.
%\end{lem}
%In order to estimate the scattering coefficients which will be defined in the following subsection, we need the following, \cite[Lemma 3]{Pelinovsky2016}:
%\begin{lem}\label{l reminder of M N}
%    If $u\in H^{1,1}(\Real)$, then for every  $x\in\Real^{\pm}$, we have
%    \begin{equation*}
%        M_{\pm}(\cdot;x)-M^{\infty}_{\pm}(x)e_1\in H^{1}(\Real),\quad
%        N_{\pm}(\cdot;x)-N^{\infty}_{\pm}(x)e_2\in H^{1}(\Real).
%    \end{equation*}
%    Moreover, if $u\in H^2(\Real)\cap H^{1,1}(\Real)$, then for every $x\in\Real$, we have
%    \begin{equation*}
%        z[M_{\pm}(\cdot;x)-M^{\infty}_{\pm}(x)e_1]-
%        \left(\!\!\!
%        \begin{array}{c}
%          -\frac{1}{4}\, M^{\infty}_{\pm}(x) \int_{\pm\infty}^x \left\{u(y)\partial_y \overline{u}(y)+\frac{1}{2\ii}|u(y)|^4 \right\} dy \vspace{2mm}\\
%          \frac{1}{2\ii}\partial_x \left(\overline{u}(x) \,M^{\infty}_{\pm}(x)\right) \\
%        \end{array}\!\!\!
%        \right)\in L^{2}_z(\Real)
%    \end{equation*}
%    and
%    \begin{equation*}
%        z[N_{\pm}(\cdot;x)-N^{\infty}_{\pm}(x)e_2]-
%        \left(
%        \begin{array}{c}
%          -\frac{1}{2\ii}\partial_x \left(u(x) \,N^{\infty}_{\pm}(x)\right)\vspace{2mm} \\
%          \frac{1}{4}\, N^{\infty}_{\pm}(x) \int_{\pm\infty}^x \left\{\overline{u}(y)\partial_y u(y)-\frac{1}{2\ii}|u(y)|^4\right\} dy \\
%        \end{array}
%        \right)\in L^{2}_z(\Real).
%    \end{equation*}
%\end{lem}
%We end this paragraph with transferring properties of $M_{\pm}(z;x)$ and $N_{\pm}(z;x)$ to properties of the original Jost functions $\varphi_{\pm}(\lambda;x)$ and $\phi_{\pm}(\lambda;x)$. The next concluding Corollary is obtained from Lemmas \ref{l existence of M N}, \ref{l expansion N M} and \ref{l reminder of M N} and the fact that  $(M_{\pm},N_{\pm})$ and $(\phi_{\pm},\varphi_{\pm})$ are related by the matrix transformation (\ref{e def M N}). We remark that $T_{1,2}(\lambda;x)$ are singular matrices for $\lambda=0$.
%\begin{cor}\label{c properties varphi phi}
%    Let $u\in H^{1,1}(\Real)$. Then for every $\lambda^2\in\Real\setminus\set{0}$, there exist unique solutions $\varphi_{\pm}(\lambda;\cdot)\in L^{\infty}(\Real)$ and $\phi_{\pm}(\lambda;\cdot)\in L^{\infty}(\Real)$ of the integral equations (\ref{e volterra phi varphi}) such that
%    \begin{equation*}
%        \lim_{x\to\pm\infty}\varphi_{\pm} (\lambda;x)=e_1, \quad\lim_{x\to\pm\infty}\phi_{\pm} (\lambda;x)=e_2.
%    \end{equation*}
%    Furthermore, for every $x\in\Real$, the Jost functions $\varphi_-(\cdot;x)$ and $\phi_+(\cdot;x)$ are analytic in the first and third quadrant of the $\lambda$ plane (where $\im(\lambda^2)>0$), whereas the functions $\varphi_+(\lambda;x)$ and $\phi_-(\lambda;x)$ are analytic in the second and fourth quadrant (where $\im(\lambda^2)<0$). Moreover, for $x\in\Real^{\pm}$, we have
%    \begin{equation*}
%    \begin{aligned}
%        \left[
%        \begin{array}{cc}
%           1 & 0 \\
%           0 & 2\ii\lambda \\
%        \end{array}
%        \right]\varphi_{\pm}(\lambda;x)- M_{\pm}^{\infty}(x)
%        \left(
%         \begin{array}{c}
%           \!\!1\!\! \\
%           \!\!\overline{u}(x)\!\! \\
%         \end{array}
%        \right)&\in H^1_z(\Real),\quad
%        \left[
%        \begin{array}{cc}
%           0 & 0 \\
%           0 & \lambda^{-1} \\
%        \end{array}
%        \right]\varphi_{\pm}(\lambda;x)&\in H^1_z(\Real)
%        \\
%        \left[
%        \begin{array}{cc}
%           2\ii\lambda & 0 \\
%           0 & 1 \\
%        \end{array}
%        \right]\phi_{\pm}(\lambda;x)-N_{\pm}^{\infty}(x)
%        \left(
%         \begin{array}{c}
%            \!\!u(x)\!\! \\
%            \!\!1\!\! \\
%         \end{array}
%        \right) &\in H^1_z(\Real),\quad
%        \left[
%        \begin{array}{cc}
%           \lambda^{-1} & 0 \\
%           0 & 0\\
%        \end{array}
%        \right]\phi_{\pm}(\lambda;x)&\in H^1_z(\Real),
%    \end{aligned}
%    \end{equation*}
%    where $M_{\pm}^{\infty}(x)$ and $N_{\pm}^{\infty}(x)$ are the same functions as defined in (\ref{e def M N intfy}).
%\end{cor}
%\begin{rem}
%    For $\lambda=0$, the Volterra's integral equations (\ref{e volterra phi varphi}) yield $\varphi_{\pm} (0;x)=e_1$ and $\phi_{\pm} (0;x)=e_2$. From the relation (\ref{e def M N}) we obtain for $z=0$,
%    \begin{equation*}
%        M_{\pm}(0;x)=
%        \left(\!\!
%         \begin{array}{c}
%            \!\!1\!\! \\
%            \!\!-\overline{u}(x)\!\! \\
%         \end{array}\!\!
%        \right),\quad
%        N_{\pm}(0;x)=
%        \left(\!\!
%         \begin{array}{c}
%            \!\!-u(x)\!\! \\
%            \!\!1\!\! \\
%         \end{array}\!\!
%        \right),
%    \end{equation*}
%    which are indeed solutions of the intergal equations (\ref{e volterra M N}).
%\end{rem} 
\subsection{Scattering data}\label{ss scattering data}
We recall that $\varphi_{\pm}(\lambda;x)e^{-\ii x\lambda^2}$ and $\phi_{\pm}(\lambda;x)e^{\ii x\lambda^2}$ are solutions of the spectral problem (\ref{e Lax1}) with boundary condition (\ref{e asymptotics psi}). Taking into account $\tr(\sigma_3)=\tr(Q)=0$ we find
\begin{equation}\label{e det phi psi =1}
    \lim_{x\to\pm\infty}\det[\varphi_{\pm} (\lambda;x)e^{-\ii x\lambda^2},\phi_{\pm}(\lambda;x)e^{+\ii x\lambda^2}]=1
\end{equation}
for all $\lambda^2\in\Real$ and $x\in\Real$. Thus, in particular $\varphi_{+}e^{-\ii x\lambda^2}$ and $\phi_{+}e^{\ii x\lambda^2}$ are linearly independent and by ODE theory they form a basis of the space of solutions of the spectral problem (\ref{e Lax1}). This enables us to express the "$-$" Jost functions in terms of the "$+$" Jost functions for every $\lambda^2\in\Real$ and $x\in\Real$. According to that, there exist coefficients $\alpha,\beta,\gamma,\delta$ which satisfy:
\begin{equation}\label{e scattering relation}
    \begin{aligned}
         \varphi_{-}(\lambda;x)e^{-\ii x\lambda^2}&&=&& \alpha(\lambda)\varphi_{+}(\lambda;x)e^{-\ii x\lambda^2} &&+&&\beta(\lambda) \phi_{+}(\lambda;x)e^{\ii x\lambda^2},\\
         \phi_{-}(\lambda;x)e^{\ii x\lambda^2}&&=&& \gamma(\lambda)\varphi_{+}(\lambda;x)e^{-\ii x\lambda^2} &&+&& \delta(\lambda) \phi_{+}(\lambda;x)e^{\ii x\lambda^2}.
    \end{aligned}
\end{equation}
The matrix
$
\left[
  \begin{array}{cc}
    \alpha & \beta \\
    \gamma & \delta \\
  \end{array}
\right]
$
is referred to as the \emph{transfer matrix} in the literature and (\ref{e scattering relation}) is called \emph{scattering relation}. By (\ref{e det phi psi =1}), we verify that the determinant of the transfer matrix equals one. By Cramer's rule we find
\begin{equation}\label{e def alpha beta}
    \begin{aligned}
        \alpha(\lambda)&=\det[\varphi_{-}(\lambda;x), \phi_{+}(\lambda;x)],\\[2pt]
        \beta(\lambda)&=\det[\varphi_{+} (\lambda;x)e^{-\ii x\lambda^2}, \varphi_{-}(\lambda;x)e^{-\ii x\lambda^2}].
    \end{aligned}
\end{equation}
Making again use of $\tr(\sigma_3)=\tr(Q)=0$, we justify that $\alpha$ and $\beta$ indeed do not depend on $x$. Moreover $\alpha$ can be analytically extended to the first and third quadrant, where $\im(\lambda^2)>0$, which follows from the analytic properties of the Jost functions $\varphi_{-}$, $\phi_{+}$ in this domain. Furthermore, from the symmetry
\begin{equation}\label{e symmetries phi varphi 2}
    \phi_{\pm}(\overline{\lambda};x)=
    \left[
      \begin{array}{cc}
        0 & -1 \\
        1 & 0 \\
      \end{array}
    \right]\overline{\varphi_{\pm}(\lambda;x)},
\end{equation}
which are direct consequences of integral equations (\ref{e volterra phi varphi}), we can derive from the scattering relation (\ref{e scattering relation}) the following conservation law:
\begin{equation}\label{e alpha^2+beta^2}
    \left\{
      \begin{array}{ll}
        |\alpha(\lambda)|^2+|\beta(\lambda)|^2=1, \quad\lambda\in\Real,\\
        |\alpha(\lambda)|^2-|\beta(\lambda)|^2=1, \quad\lambda\in\ii\Real.
      \end{array}
    \right.
\end{equation}
As pointed out in \cite{Pelinovsky2016} this is indicating that the DNLS equation combines elements of the focusing and as well of the defocusing cubic NLS equation.

%We can set $x=0$ in (\ref{e def alpha beta}) and find
%\begin{eqnarray*}
%% \nonumber to remove numbering (before each equation)
%  \alpha(\lambda) &=& \det\left\{
%  \left[
%    \begin{array}{cc}
%      1 & 0 \\
%      0 & 2\ii\lambda \\
%    \end{array}
%  \right]\varphi_{-}(\lambda;0),
%  \left[
%    \begin{array}{cc}
%      \frac{1}{2\ii\lambda} & 0 \\
%      0 & 1 \\
%    \end{array}
%  \right]\phi_{+}(\lambda;0)
%  \right\} \\
%   &=& \det\left\{
%  \left[
%    \begin{array}{cc}
%      1 & 0 \\
%      0 & 2\ii\lambda \\
%    \end{array}
%  \right]\varphi_{-}(\lambda;0)-M_{-}^{\infty}(0)
%  \left(
%    \begin{array}{c}
%      \!\!1\!\! \\
%      \!\!\overline{u}(0)\!\! \\
%    \end{array}
%  \right)
%  ,
%  \left[
%    \begin{array}{cc}
%      \frac{1}{2\ii\lambda} & 0 \\
%      0 & 1 \\
%    \end{array}
%  \right]\phi_{+}(\lambda;x)-N_{+}^{\infty}(0)
%  \left(
%    \begin{array}{c}
%      \!\!0\!\! \\
%      \!\!1\!\! \\
%    \end{array}
%  \right)
%  \right\}\\
%  &&\qquad+\det\left\{
%  \left[
%    \begin{array}{cc}
%      1 & 0 \\
%      0 & 2\ii\lambda \\
%    \end{array}
%  \right]\varphi_{-}(\lambda;0)-M_{-}^{\infty}(0)
%  \left(
%    \begin{array}{c}
%      \!\!1\!\! \\
%      \!\!\overline{u}(0)\!\! \\
%    \end{array}
%  \right)
%  ,
%  N_{+}^{\infty}
%  \left(
%    \begin{array}{c}
%      \!\!0\!\! \\
%      \!\!1\!\! \\
%    \end{array}
%  \right)\right\}\\
%  &&\qquad\qquad+\det\left\{
%  M_{-}^{\infty}(0)
%  \left(
%    \begin{array}{c}
%      \!\!1\!\! \\
%      \!\!\overline{u}(0)\!\! \\
%    \end{array}
%  \right)
%  ,
%  \left[
%    \begin{array}{cc}
%      \frac{1}{2\ii\lambda} & 0 \\
%      0 & 1 \\
%    \end{array}
%  \right]\phi_{+}(\lambda;x)-N_{+}^{\infty}(0)
%  \left(
%    \begin{array}{c}
%      \!\!0\!\! \\
%      \!\!1\!\! \\
%    \end{array}
%  \right)
%  \right\}+\alpha_{\infty},
%\end{eqnarray*}
%where
%\begin{equation*}
%    \alpha_{\infty}:=\det\left\{M_{-}^{\infty}(0)
%  \left(
%    \begin{array}{c}
%      \!\!1\!\! \\
%      \!\!\overline{u}(0)\!\! \\
%    \end{array}
%  \right)
%  ,N_{+}^{\infty}(0)
%  \left(
%    \begin{array}{c}
%      \!\!0\!\! \\
%      \!\!1\!\! \\
%    \end{array}
%  \right)
%  \right\}=e^{\frac{1}{2\ii}\int_{\Real}|u(y)|^2dy}.
%\end{equation*}
%Then, by similar expressions for $\lambda\beta(\lambda)$ and $\lambda^{-1}\beta(\lambda)$ we can deduce from Corollary \ref{c properties varphi phi} the following Lemma (see \cite{Pelinovsky2016}):
%\begin{lem}\label{l alpha in H^1}
%    If $u\in H^{1,1}(\Real)$, then
%    \begin{equation}\label{e alpha,beta in H^1}
%        \alpha(\lambda)-\alpha_{\infty},\; \lambda\beta(\lambda),\; \lambda^{-1}\beta(\lambda)\in H^1_z(\Real).
%    \end{equation}
%    Moreover, if $u\in H^2(\Real)\cap H^{1,1}(\Real)$, then
%    \begin{equation}\label{e beta in L^21}
%         \lambda\beta(\lambda),\; \lambda^{-1}\beta(\lambda)\in L^{2,1}_z(\Real).
%    \end{equation}
%\end{lem}
We now continue with the definition of the reflection coefficient:
\begin{equation}\label{e def r}
    r(\lambda)=\frac{\beta(\lambda)}{\alpha(\lambda)}.
\end{equation}
This definition makes sense for every $\lambda^2\in\Real$, if $\alpha$ admits no zeros on $\Real\cup\ii\Real$, but we can not expect generally that $\alpha$ behaves like that. Therefore we define the following set:
\begin{equation}\label{e def no resonances}
    \pazocal{R}:=\set{u\in H^{1,1}(\Real):\exists A>0,|\alpha(\lambda)|>A\text{ for every }\lambda\in\Real\cup\ii\Real}
\end{equation}
Zeroes $\lambda\in\Real\cup\ii\Real$ of $\alpha$ are called \emph{resonances} in \cite{Pelinovsky2016}. Hence, the set $\pazocal{R}$ consists of those potentials, which do not admit resonances of the linear equation (\ref{e Lax1}). Let us assume from now on that $u\in \pazocal{R}$. Then,  we can rewrite the scattering relation (\ref{e scattering relation}) in the following way:
\begin{equation}\label{e alternative scattering relation}
    \Phi_+(\lambda;x)=\Phi_-(\lambda;x)(1+S(\lambda;x)), \quad\lambda^2\in\Real,
\end{equation}
where the matrices $\Phi_{\pm}$ and $S$ are given by
\begin{equation}\label{e def Phi}
    \Phi_+(\lambda;x):=
    \left[\frac{\varphi_{-}(\lambda;x)} {\alpha(\lambda)},\phi_{+}(\lambda;x)
    \right],\quad
    \Phi_-(\lambda;x):=
    \left[\varphi_{_+}(\lambda;x) ,\frac{\phi_{-}(\lambda;x)} {\overline{\alpha(\overline{\lambda})}}
    \right],
\end{equation}
and
\begin{equation}\label{e def S}
    S(\lambda;x):=
    \left\{
      \begin{array}{ll}
        \left[
          \begin{array}{cc}
            |r(\lambda)|^2 & \overline{r(\lambda)} e^{-2\ii x\lambda^2} \\
            r(\lambda) e^{2\ii x\lambda^2} & 0 \\
          \end{array}
        \right]
        , & \hbox{for }\lambda\in\Real,\vspace{2mm} \\ \left[
          \begin{array}{cc}
            -|r(\lambda)|^2 & -\overline{r(\lambda)} e^{-2\ii x\lambda^2} \\
            r(\lambda) e^{2\ii x\lambda^2} & 0 \\
          \end{array}
        \right]
        , & \hbox{for }\lambda\in\ii\Real.
      \end{array}
    \right.
\end{equation}
It is clear from the representation (\ref{e def alpha beta}) that $\alpha$ has an analytic continuation in the first and third quadrants of the $\lambda$ plane. Therefore the function $\Phi_+$ defined in (\ref{e def Phi}) can be continued analytically in the first and third quadrants, as long as there are no zeros $\lambda_0$ of the continuation of $\alpha$ with $\im(\lambda_0^2)>0$. Under the same assumption, the function $\Phi_-$ in (\ref{e def Phi}) can be analytically continued in the second and fourth quadrant. From now on we want to allow that $\alpha(\lambda)$ has finite many simple zeroes. That is $\alpha(\lambda_k)=0$ and $\alpha'(\lambda_k)\neq0$ for a finite number of pairwise different $\lambda_1,...,\lambda_N$ which are assumed to lie in the first quadrant. Note that, if $\alpha(\lambda_k)=0$, then also $\alpha(-\lambda_k)=0$. Henceforth, the continuations of $\Phi_{\pm}$ are merely meromorphic. They admit simple poles at the zeros of $\alpha$, since  $\alpha'(\lambda_k)\neq0$ for $k=1,...,N$. The prime denotes the derivative with respect to $\lambda$. We find:
\begin{equation*}
    \res_{\lambda=\pm\lambda_k}
    \Phi_+(\lambda;x)=
    \left[\frac{\varphi_{-}(\pm\lambda_k;x)} {\pm\alpha'(\lambda_k)},\;0\;
    \right].
\end{equation*}
By (\ref{e def alpha beta}), the meaning of the zeros of $\alpha$ is the following. If $\alpha(\lambda_k)=0$, then by (\ref{e def alpha beta}) the $\Compl^2$ vectors $\varphi_{-}(\lambda_k;x)e^{-\ii x\lambda_k^2}$ and $\phi_{+}(\lambda_k;x)e^{\ii x\lambda_k^2}$ are linear dependent for every $x\in\Real$. Hence,
\begin{equation}\label{e def gamma}
    \varphi_{-}(\pm\lambda_k;x)=\pm\gamma_k\; e^{2\ii x\lambda_k^2}\; \phi_{+}(\pm\lambda_k;x)
\end{equation}
for some complex constant $\gamma_k\in\Compl\setminus\set{0}$. We will refer to $\gamma_k$ as the \emph{norming constant}. The norming constants do not depend on $x$. Indeed, differentiating (\ref{e def gamma}) with respect to $x$ and using the fact that  $\varphi_{-}(\lambda_k;x)e^{-\ii x\lambda_k^2}$ and $\phi_{+}(\lambda_k;x)e^{\ii x\lambda_k^2}$ are solutions of the spectral problem (\ref{e Lax1}), we easily obtain $\partial_x \gamma_k=0$. Note also that due to the symmetry (\ref{e symmetries phi varphi 1}) the cases $+\lambda_k$ and $-\lambda_k$ do have the same norming constants upon a minus sign.  Combining (\ref{e def gamma}) and the above residue calculation we find
\begin{equation}\label{e res Phi}
    \res_{\lambda=\pm\lambda_k}
    \Phi_+(\lambda;x)=
    \left[\frac{\pm\gamma_k \; e^{2\ii x\lambda_k^2}} {\alpha'(\pm\lambda_k)}\phi_{+}(\pm\lambda_k;x),\;0\;
    \right]=
    \lim_{\lambda\to\pm\lambda_k}\Phi_+(\lambda;x)
    \left[
      \begin{array}{cc}
        0 & 0\\
        \frac{\gamma_k \; e^{2\ii x\lambda_k^2}} {\alpha'(\lambda_k)} & 0 \\
      \end{array}
    \right].
\end{equation}
Correspondingly, we can compute an analogue relation for the residue of $\Phi_-$ at $\pm\overline{\lambda}_k$.\\
By a theorem of complex analysis (see, e.g., \cite[Theorem 3.2.8]{Ablowitz2003}), the zeroes of $\alpha$ must be isolated. In addition, by \cite[Lemma 4]{Pelinovsky2016} we know $\alpha(\lambda)\to\alpha_{\infty}\neq 0$ as $|\lambda|\to\infty$. Thus, we conclude that the zeroes of $\alpha(\lambda)$ in the first quadrant form a finite set $\set{\lambda_1,...,\lambda_N}$. But the essential assumption $\alpha'(\lambda_k)\neq 0$ is generally not expectable and give rise to the following definition:
\begin{equation}\label{e def no eigenvalues}
    \pazocal{E}:=\set{u\in H^{1,1}(\Real):\alpha'(\lambda_k)\neq 0\text{ for all zeroes }\lambda_k\text{ of }\alpha\text{ with } \im(\lambda_k^2)>0}
\end{equation}
From now on, additionally to $u\in\pazocal{R}$,  we assume $u\in\pazocal{G}:=\pazocal{R}\cap\pazocal{E}$. The elements of $\pazocal{G}$ are called \emph{generic potentials} according to the classical paper \cite{Beals1984}. As remarked by the authors in \cite[Remark 5]{Pelinovsky2016}, we have $u\in\pazocal{G}$ if
\begin{equation*}\label{e estimate for no eigenv and reso}
    \|u\|_{L^2}^2+\sqrt{\|u\|_{L^1}( 2\|\partial_x u\|_{L^1}+\|u\|_{L^3}^3)}<1.
\end{equation*}
The set $\pazocal{G}$ is open and, moreover, dense in $H^{1,1}(\Real)$. Due to the availability of the transformation (\ref{e def M N}), this can be deduced from \cite{Beals1984} as explained in \cite[Proposition 4]{PelinShimaSaal2017}. However, any soliton or multi soliton is contained in $\pazocal{G}$. For those explicit solutions, the expression
\begin{equation*}
    \|u\|_{L^2}^2+\sqrt{\|u\|_{L^1}( 2\|\partial_x u\|_{L^1}+\|u\|_{L^3}^3)}
\end{equation*}
can be arbitrary large.
\medskip\\
Using the transformation (\ref{e def M N}) it is shown in \cite{Pelinovsky2016} that for $u\in H^2(\Real)\cap H^{1,1}(\Real)$ the following holds.
\begin{equation}\label{e def Phi infty}
    \Phi_{\pm}(\lambda;x)\to\Phi_{\infty}(x):=
    \left[
      \begin{array}{cc}
        e^{-\frac{1}{2\ii}\int^{+\infty}_x|u(y)|^2dy} & 0 \\
        0 & e^{-\frac{1}{2\ii}\int_{-\infty}^x|u(y)|^2dy} \\
      \end{array}
    \right]\quad\text{as }|\lambda|\to\infty.
\end{equation}
The limit has to be taken along a contour  in the corresponding domain of analyticity.

The alternative scattering relation (\ref{e alternative scattering relation}), the residue condition (\ref{e res Phi}) and finally the asymptotic behavior (\ref{e def Phi infty}) set up a \rh.
%\begin{samepage}
%\begin{framed}
%\begin{rhp}\label{rhp phi}
%Find for each $x\in\Real$, a $2\times 2$-matrix valued function $\Compl\ni \lambda\mapsto \Phi(\lambda;x)$ which satisfies
%\begin{enumerate}[(i)]
%  \item $\Phi(\lambda;x)$ is meromorphic in $\Compl\setminus(\Real\cup\ii\Real)$ (with respect to the parameter $\lambda$).
%  \item $\Phi(\lambda;x)=\Phi_{\infty}(x) +\mathcal{O}\left(\frac{1}{\lambda}\right)$ as $|\lambda|\to\infty$.
%  \item The non-tangential boundary values $\Phi_{\pm}(z;x,t)$ exist for $\lambda^2\in\Real$ and satisfy the jump relation (\ref{e alternative scattering relation}).
%  \item $\Phi$ has simple poles at $\pm\lambda_1,...,\pm\lambda_N, \pm\overline{\lambda}_1,...,\pm \overline{\lambda}_N$ with
%      \begin{equation*}
%        \begin{aligned}
%          \res_{\lambda=\pm\lambda_k}
%          \Phi(\lambda;x)&=
%          \lim_{\lambda\to\pm\lambda_k}\Phi(\lambda;x)
%          \left[
%            \begin{array}{cc}
%              0 & 0\\
%              \frac{\gamma_k \; e^{2\ii x\lambda_k^2}} {\alpha'(\lambda_k)} & 0 \\
%            \end{array}
%          \right],\\
%          \res_{\lambda=\pm\overline{\lambda}_k}
%          \Phi(\lambda;x)&=
%          \lim_{\lambda\to\pm \overline{\lambda}_k}\Phi(\lambda;x)
%          \left[
%            \begin{array}{cc}
%              0 & \frac{-\overline{\gamma}_k \; e^{-2\ii x\overline{\lambda}_k^2}} {\overline{\alpha' (\overline{\lambda_k})}}\\
%              0 & 0 \\
%            \end{array}
%          \right].
%        \end{aligned}
%      \end{equation*}
%\end{enumerate}
%\end{rhp}
%\end{framed}
%\end{samepage}
Since that \rh is somewhat unsuitable to show the existence of the inverse Scattering map, we turn again to the Zhakarov-Shabat type Jost functions $M_{\pm}$ and $N_{\pm}$ (see (\ref{e def M N}), which are functions of $z$, where we recall $z=\lambda^2$. Due to $\alpha(\lambda)=\alpha(-\lambda)$, it is alowed to define $a(z):=\alpha(\lambda)$. Of course, if $\pm\lambda_k\neq0$ are (simple) zeroes of $\alpha$, then $z_k:=\lambda_k^2$ is a (simple) zero of $a$. In order to transfer the jump condition (\ref{e alternative scattering relation}) to the Jost functions $M_{\pm}$ and $N_{\pm}$, one more explicit definition is needed:
\begin{equation}\label{e def P}
    P_{\pm}(z;x):=\frac{1}{2\ii\lambda}
    T_1(\lambda;x)T_2^{-1}(\lambda;x)N_{\pm}(z;x)=
    -\frac{1}{4z}
    \left[
      \begin{array}{cc}
        1 & u(x) \\
        -\overline{u}(x) & -|u(x)|^2-4z \\
      \end{array}
    \right]N_{\pm}(z;x).
\end{equation}
In \cite[Lemma 5]{Pelinovsky2016} it is shown, that there is no singularity in (\ref{e def P}) and moreover, $P_{\pm}(z;x)$ satisfy the following limits as $|\im(z)|\to\infty$ along a contour in the domains of their analyticity:
\begin{equation*}
    \lim_{|z|\to\infty}P_{\pm}(z;x)=
    \left(
      \begin{array}{c}
        0\\
        N_{\pm}^{\infty}(x) \\
      \end{array}
    \right).
\end{equation*}
Now we are ready to define the analogue of (\ref{e def Phi}). Instead of $\lambda\in\Real\cup\ii\Real$, now we have $z\in\Real$ and set
\begin{equation}\label{e def pi}
    \pi_+(z;x):=
    \left[\frac{M_{-}(z;x)} {a(z)},P_{+}(z;x)
    \right],\quad
    \pi_-(z;x):=
    \left[M_{_+}(z;x) ,\frac{P_{-}(z;x)} {\overline{a(\overline{z})}}
    \right].
\end{equation}
These definitions entail the following analogue of (\ref{e alternative scattering relation}) which can be checked by elementary calculations:
\begin{equation}\label{e jump of pi}
    \pi_+(z;x)=\pi_-(z;x)(1+R(z;x)), \quad z\in\Real.
\end{equation}
Herein the new jump matrix $R$ which includes new reflection coefficients $r_{\pm}$, is defined by
\begin{equation*}
    R(z;x):=
        \left[
          \begin{array}{cc}
            \overline{r}_+(z)r_-(z) & e^{-2\ii xz}\overline{r}_+(z) \\
            e^{2\ii xz}r_-(z) & 0 \\
          \end{array}
        \right].
\end{equation*}
The new reflection coefficients are given by
\begin{equation}\label{e def r pm}
    r_{+}(z):=-\frac{\beta(\lambda)} {2\ii\lambda\alpha(\lambda)},\quad
    r_{-}(z):=\frac{2\ii\lambda\beta(\lambda)} {\alpha(\lambda)},\quad z\in\Real.
\end{equation}
We have the following Lemma \cite{Pelinovsky2016}.
\begin{lem}\label{l r pm in H1 and L^21}
    If $u\in H^2(\Real)\cap H^{1,1}(\Real)\cap\pazocal{R}$, then $r_{\pm}\in H^1(\Real) \cap L^{2,1}(\Real)$.
\end{lem}
Moreover, we found directly from the definition (\ref{e def r pm}) that $r_+$ and $r_-$ are connected by
\begin{equation}\label{e relation r+ r-}
    r_-(z)=4zr_+(z),\quad z\in\Real.
\end{equation}
Furthermore, $\overline{r}_+(z)r_-(z)=|r(\lambda)|^2$ if $z>0$, whereas $\overline{r}_+(z)r_-(z)=-|r(\lambda)|^2$ if $z<0$. Additionally, using (\ref{e alpha^2+beta^2}) we obtain $1-|r(\lambda)|^2=|\alpha(\lambda)|^{-2}$. Thus, we have
\begin{equation}\label{e r constraint}
   \left\{
     \begin{array}{ll}
       1+\overline{r}_+(z)r_-(z)\geq1, & z>0, \\
       1+\overline{r}_+(z)r_-(z)\geq c_0^2, & z<0,
     \end{array}
   \right.
\end{equation}
where $c_0^{-1}:=\sup_{\lambda\in\ii\Real}|\alpha(\lambda)|$. The constraint (\ref{e r constraint}) is used in \cite{Pelinovsky2016} to obtain a unique solution to the \rh \ref{rhp m} below.\par
Analytic continuations of $\pi_{\pm}$ in $\Compl^{\pm}$ exist if there is no $z\in\Compl$ such that $a(z)=0$. Otherwise we have analogously to (\ref{e def gamma})
\begin{equation*}
    M_{-}(z_k;x)=2\ii\lambda_k\gamma_k\; e^{2\ii x z_k}\; P_{+}(z_k;x)
\end{equation*}
with the same $\gamma_k$ as in (\ref{e def gamma}). Denoting the meromorphic continuations of $\pi_{\pm}(\cdot;x)$ with the same letters we can verify the following residue condition:
\begin{equation}\label{e res pi}
    \res_{z=z_k}\pi_+(z;x)=\lim_{z\to z_k}\pi_+(z;x)
          \left[
            \begin{array}{cc}
              0 & 0 \\
              2\ii\lambda_kc_k e^{2\ii xz_k} & 0
            \end{array}
          \right],
\end{equation}
where we set $c_k:=\gamma_k/a'(z_k)$. Correspondingly we can compute an analogue relation for the residuum of $\pi_-$ at $\overline{z}_k$. Next, we have
\begin{equation*}
    \pi_{\pm}(\lambda;x)\to\Phi_{\infty}(x)\quad\text{as }|\lambda|\to\infty,
\end{equation*}
similarly to (\ref{e def Phi infty}).
We obtain our final Riemann--Hilbert problem if we normalize the boundary condition at infinity:
\begin{equation}\label{e def m}
    m(z;x):=
    \left\{
      \begin{array}{ll}
        \,[\Phi_{\infty}(x)]^{-1}\pi_+(z;x), & z\in\Compl^+, \\
        \,[\Phi_{\infty}(x)]^{-1}\pi_-(z;x), & z\in\Compl^-.
      \end{array}
    \right.
\end{equation}
The multiplication from the left by the diagonal matrix $[\Phi_{\infty}(x)]^{-1}$ changes neither the analytic properties of $\pi_{\pm}$ nor the jump or residuum conditions. Therefore, the function $m$ defined in (\ref{e def m}) solves the following Riemann--Hilbert problem:
\begin{samepage}
\begin{framed}
\begin{rhp}\label{rhp m}
Find for each $x\in\Real$ a $2\times 2$-matrix valued function $\Compl\ni z\mapsto m(z;x)$ which satisfies
\begin{enumerate}[(i)]
  \item $m(z;x)$ is meromorphic in $\Compl\setminus\Real$ (with respect to the parameter $z$).
  \item $m(z;x)=1+\mathcal{O}\left(\frac{1}{z}\right)$ as $|z|\to\infty$.
  \item The non-tangential boundary values $m_{\pm}(z;x)$ exist for $z\in\Real$ and satisfy the jump relation
      \begin{equation}\label{e jump}
        m_+=m_-(1+R),\quad\text{where }
        R(z;x):=
        \left[
          \begin{array}{cc}
            \overline{r}_+(z)r_-(z) & e^{-2\ii xz}\overline{r}_+(z) \\
            e^{2\ii xz}r_-(z) & 0 \\
          \end{array}
        \right]
      \end{equation}
  \item $m$ has simple poles at $z_1,...,z_N,\overline{z}_1,...,\overline{z}_N$ with
      \begin{equation*}%\label{e Res}
        \begin{aligned}
          \res_{z=z_k}m(z;x)&=\lim_{z\to z_k}m(z;x)
          \left[
            \begin{array}{cc}
              0 & 0 \\
              2\ii\lambda_kc_k e^{2\ii xz_k} & 0
            \end{array}
          \right],\\
          \res_{z=\overline{z_k}}m(z;x)&=\lim_{z\to \overline{z}_k}m(z;x)
          \left[
            \begin{array}{cc}
              0 & \frac{-\overline{c}_k}{2\ii\lambda_k} e^{-2\ii x\overline{z}_k} \\
              0 & 0
            \end{array}
          \right].
        \end{aligned}
      \end{equation*}
\end{enumerate}
\end{rhp}
\end{framed}
\end{samepage}
We will use the notation
\begin{equation*}
    \mathcal{S}(u)=\set{r_{\pm};\lambda_1,... ,\lambda_N;c_1,...,c_N}
\end{equation*}
and call $\mathcal{S}$ the scattering data of $u$. They consist of the \emph{reflection coefficients} $r_{\pm}$ which satisfy the constraints (\ref{e relation r+ r-}) and (\ref{e r constraint}), the \emph{poles} $z_k:=\lambda_k^2$ and the \emph{norming constants} $c_k=\gamma_k/a'(z_k)$. $\mathcal{S}$ is all information we need to know about $u$ to formulate the \rh\ref{rhp m}. In the rest of this paper we treat the problem to define the inverse map $\set{r_{\pm};\lambda_1,... ,\lambda_N;c_1,...,c_N}\mapsto u$. Therefore we will solve \rh \ref{rhp m} and apply the following reconstruction formulas:
\begin{equation}\label{e rec 1}
    u(x)e^{\ii\int_{+\infty}^x|u(y)|^2dy}= -4\lim_{|z|\to\infty}z\;[m(z;x)]_{12}
\end{equation}
and
\begin{equation}\label{e rec 2}
    e^{-\frac{1}{2\ii}\int_{+\infty}^x|u(y)|^2dy} \partial_x\left(\overline{u}(x) e^{\frac{1}{2\ii}\int_{+\infty}^x|u(y)|^2dy}\right) =2\ii\lim_{|z|\to\infty}z\;[m(z;x)]_{21}.
\end{equation}
Both, (\ref{e rec 1}) and (\ref{e rec 2}), are justified in \cite{Pelinovsky2016} and the key of Inverse Scattering. By $[\cdot]_{ij}$ we denote the $i$-$j$-component of the matrix in the brackets.\\
The miraculous fact about the forward scattering is the trivial time evolution of the scattering data if the potential $u(x,t)$ evolves accordingly to the DNLS equation:
\begin{lem}\label{l time dependence scattering data}
    Under the assumption that an initial datum $u_0\in H^2(\Real)\cap H^{1,1}(\Real)\cap \pazocal{G}$ admits a (local) solution $u(\cdot,t)\in H^2(\Real)\cap H^{1,1}(\Real)$ to the Cauchy problem (\ref{e dnls}) for $t\in[0,T]$, the scattering data of $u(\cdot,t)$ are given by
    \begin{equation}\label{e time dependence scattering data}
        \mathcal{S}_t(u)=\set{r_{\pm}(z;t)=r_{\pm}(z;0) e^{4\ii z^2t};\lambda_1,... ,\lambda_N;c_1(0) e^{4\ii \lambda_1^4t},...,c_N(0) e^{4\ii \lambda_N^4t}},
    \end{equation}
    where
    \begin{equation*}
        \mathcal{S}_0(u)=\set{r_{\pm}(z;0);\lambda_1,... ,\lambda_N;c_1(0),...,c_N(0)}
    \end{equation*}
    are defined to be the scattering data of $u_0$. In particular, the set $\pazocal{G}$ is invariant under the flow of the DNLS equation, $r_{\pm}(\cdot;t)\in H^1(\Real)\cap L^{2,1}(\Real)$ for every $t\in[0,T]$, and, finally, (\ref{e relation r+ r-}) and (\ref{e r constraint}) remain valid.
\end{lem}
The proof of this Lemma is given in \cite[Section 5]{Pelinovsky2016} and we skip it here. Plugging the time dependence (\ref{e time dependence scattering data}) into the formulas of \rh \ref{rhp m} we obtain the dynamic Riemann--Hilbert problem for the DNLS equation.
\begin{samepage}
\begin{framed}
\begin{rhp}\label{rhp m dynamic}
Find for each $(x,t)\in\Real\times\Real$ a $2\times 2$-matrix valued function $\Compl\ni z\mapsto m(z;x,t)$ which satisfies
\begin{enumerate}[(i)]
  \item $m(z;x,t)$ is meromorphic in $\Compl\setminus\Real$ (with respect to the parameter $z$).
  \item $m(z;x,t)=1+\mathcal{O}\left(\frac{1}{z}\right)$ as $|z|\to\infty$.
  \item The non-tangential boundary values $m_{\pm}(z;x,t)$ exist for $z\in\Real$ and satisfy the jump relation
      \begin{equation*}%\label{e jump}
        m_+=m_-(1+R),\quad\text{where }
        R(z;x,t):=
        \left[
          \begin{array}{cc}
            \overline{r}_+(z)r_-(z) & e^{\overline{\phi}(z)}\overline{r}_+(z) \\
            e^{\phi(z)}r_-(z) & 0 \\
          \end{array}
        \right]
      \end{equation*}
      with $\phi(z):=2\ii xz+4\ii z^2t$.
  \item $m$ has simple poles at $z_1,...,z_N,\overline{z}_1,...,\overline{z}_N$ with
      \begin{equation}\label{e Res}
        \begin{aligned}
          \res_{z=z_k}m(z;x,t)&=\lim_{z\to z_k}m(z;x,t)
          \left[
            \begin{array}{cc}
              0 & 0 \\
              2\ii\lambda_kc_k e^{\phi_k} & 0
            \end{array}
          \right],\\
          \res_{z=\overline{z}_k}m(z;x,t)&=\lim_{z\to \overline{z}_k}m(z;x,t)
          \left[
            \begin{array}{cc}
              0 & \frac{-\overline{c}_k}{2\ii\lambda_k} e^{\overline{\phi}_k} \\
              0 & 0
            \end{array}
          \right],
        \end{aligned}
      \end{equation}
      where $\phi_k:=\phi(z_k)$.
\end{enumerate}
\end{rhp}
\end{framed}
\end{samepage}
\begin{rem}\label{r uniqueness + det=1}
    Without further theory we can observe that if \rh \ref{rhp m dynamic} is solvable, then the solution is unique. In order to show the uniqueness of solutions, we firstly find the following (trivial) Riemann Hilbert problem for the map $z\mapsto \det(m(z;x,t))$:
    \begin{equation*}
        \left\{
           \begin{array}{ll}
             \det(m(z;x,t))\text{ is an entire function with respect to the parameter }z,\\
             \det(m(z;x,t))\to 1,\text{ as }|z|\to\infty.
           \end{array}
         \right.
    \end{equation*}
    By Liouville's theorem we conclude
    \begin{equation}\label{e det m=1}
        \det(m(z;x,t))\equiv 1,\text{ for all }x,t\in\Real\text{ and }z\in \Compl.
    \end{equation}
    Hence, for a possible solution $m$ of \rh \ref{rhp m dynamic}, $[m(z;x,t)]^{-1}$ exists for all $x\in\Real$ and $z\in \Compl$. If we have a second solution $\widetilde{m}(z;x,t)$, the ratio $\widetilde{m}(z;x,t)[m(z;x,t)]^{-1}$ satisfies
    \begin{equation*}
        \left\{
           \begin{array}{ll}
             \widetilde{m}(z;x,t)[m(z;x,t)]^{-1}\text{ is an entire function with respect to the parameter }z,\\
             \widetilde{m}(z;x,t)[m(z;x,t)]^{-1}\to 1,\text{ as }|z|\to\infty,
           \end{array}
         \right.
    \end{equation*}
    such that $\widetilde{m}(z;x,t)[m(z;x,t)]^{-1}\equiv 1$.
\end{rem}
We end the subsection mentioning the following symmetry:
\begin{equation}\label{e symmetrie of m}
    m(z;x)=\frac{1}{4z}
    \left[
      \begin{array}{cc}
        w(x) & 1 \\
        -|w(x)|^2-4z & \overline{w}(x) \\
      \end{array}
    \right]
    \overline{m(\overline{z};x)}
    \left[
      \begin{array}{cc}
        0 & 1 \\
        4z & 0 \\
      \end{array}
    \right],
\end{equation}
where $w(x):= u(x)\,e^{\ii\int^x_{+\infty}|u(y)|^2dy}$. The symmetry (\ref{e symmetrie of m}) is obtained when one transfers the symmetry (\ref{e symmetries phi varphi 2}) to $\pi_{\pm}$ and $m$, respectively. 
%\input{g}




%$\lambda_k\in\Compl_{I}\footnote{$\Compl_{I}:=\set{\eta+\ii \zeta\in\Compl:\eta,\zeta>0}$}$ $(k=1,...,N)$, $z_k:=\lambda_k^2\in\Compl_+$, $c_k\in\Compl^*$, $r_{\pm}\in L^{2,1}(\Real)$
%\begin{samepage}
%\begin{framed}
%\textbf{RHP[DNLS]:}\\Find for each $(x,t)\in\Real\times\Real$ a $2\times 2$-matrix valued function $\Compl\ni z\mapsto m(z;x,t)$ which satisfies
%\begin{enumerate}[(i)]
%  \item $m(z;x,t)$ is meromorphic in $\Compl\setminus\Real$ (with respect to the parameter $z$).
%  \item $m(z;x,t)=1+\mathcal{O}\left(\frac{1}{z}\right)$ as $|z|\to\infty$.
%  \item The non-tangential boundary values $m_{\pm}(z;x,t)$ exist for $z\in\Real$ and satisfy the jump relation
%      \begin{equation}\label{e jump}
%        m_+=m_-(1+R),\quad\text{where }
%        R(z;x,t):=
%        \left(
%          \begin{array}{cc}
%            \overline{r}_+(z)r_-(z) & e^{\overline{\phi}(z)}\overline{r}_+(z) \\
%            e^{\phi(z)}r_-(z) & 0 \\
%          \end{array}
%        \right)
%      \end{equation}
%      with
%      \begin{equation}\label{e def phase}
%         \phi(z):=2\ii xz+4\ii z^2t.
%      \end{equation}
%  \item $m$ has simple poles at $z_1,...,z_N,\overline{z}_1,...,\overline{z}_N$ with
%      \begin{equation}\label{e Res}
%        \left[\begin{aligned}
%          \res_{z=z_k}m(z;x,t)&=\lim_{z\to z_k}m(z;x,t)
%          \left(
%            \begin{array}{cc}
%              0 & 0 \\
%              2\ii\lambda_kc_k e^{\phi_k} & 0
%            \end{array}
%          \right),\\
%          \res_{z=\overline{z_k}}m(z;x,t)&=\lim_{z\to \overline{z}_k}m(z;x,t)
%          \left(
%            \begin{array}{cc}
%              0 & \frac{-\overline{c}_k}{2\ii\lambda_k} e^{\overline{\phi}_k} \\
%              0 & 0
%            \end{array}
%          \right).
%        \end{aligned}\right.
%      \end{equation}
%\end{enumerate}
%\end{framed}
%\end{samepage}
%\begin{rem}\label{r uniqueness + det=1}
%    Without further theory we can observe that if \textbf{RHP[DNLS]} is solvable, then the solution is unique. In order to show the uniqueness of solutions, we firstly find the following (trivial) Riemann Hilbert problem for the map $z\mapsto \det(m(z;x))$:
%    \begin{equation*}
%        \left\{
%           \begin{array}{ll}
%             \det(m(z;x))\text{ is an entire function with respect to the parameter }z,\\
%             \det(m(z;x))\to 1,\text{ as }|z|\to\infty.
%           \end{array}
%         \right.
%    \end{equation*}
%    By Liouville's theorem we conclude
%    \begin{equation}\label{e det m=1}
%        \det(z;x)\equiv 1,\text{ for all }x\in\Real\text{ and }z\in \Compl.
%    \end{equation}
%    Hence, for a possible solution $m$ of \textbf{RHP[DNLS]}, $[m(z;x)]^{-1}$ exists for all $x\in\Real$ and $z\in \Compl$. If we have second solution $\widetilde{m}(z;x)$, the ratio $\widetilde{m}(z;x)[m(z;x)]^{-1}$ satisfies
%    \begin{equation*}
%        \left\{
%           \begin{array}{ll}
%             \widetilde{m}(z;x)[m(z;x)]^{-1}\text{ is an entire function with respect to the parameter }z,\\
%             \widetilde{m}(z;x)[m(z;x)]^{-1}\to 1,\text{ as }|z|\to\infty,
%           \end{array}
%         \right.
%    \end{equation*}
%    such that $\widetilde{m}(z;x)[m(z;x)]^{-1}\equiv 1$.
%\end{rem}




\section{Solitons}\label{s solitons}
This section is devoted to the exact solitary wave solutions of the DNLS equation (\ref{e dnls}) which are known since the 1970s (see, e.g., \cite{mjlhus1976} and \cite{KaupNewell1978}). Also more recent works are concerned with solitons. See for instance \cite{Colin2006}, where orbital stability of solitons is shown. The inverse scattering machinery admits a simple definition of $N$-solitons:
\begin{defn}
    (Global) solutions $u^{(N\text{-sol})}(x,t)$ of (\ref{e dnls}) such that the initial datum $u^{(N\text{-sol})}(\cdot,0)$ produces scattering data
    \begin{equation*}
        \mathcal{S} (u^{(N\text{-sol})})=\set{r_+\equiv r_-\equiv 0;\lambda_1,... ,\lambda_N;c_1,...,c_N},
    \end{equation*}
    are called $N$-\emph{solitons}. For $N=1$ we just say \emph{soliton}.
\end{defn}
In the case of $r_+\equiv r_-\equiv 0,$ the Riemann--Hilbert problem \rh \ref{rhp m dynamic} reads as follows:
\begin{samepage}
\begin{framed}
\begin{rhp}\label{rhp N sol}
Find for each $x\in\Real$ a $2\times 2$-matrix valued function $\Compl\ni z\mapsto m^{(N\text{-sol})} (z;x,t)$ which satisfies
\begin{enumerate}[(i)]
  \item $m^{(N\text{-sol})} (z;x,t)$ is meromorphic in $\Compl$ (with respect to the parameter $z$).
  \item $m^{(N\text{-sol})} (z;x,t)=1+\mathcal{O}\left(\frac{1}{z}\right)$ as $|z|\to\infty$.
  \item $m^{(N\text{-sol})} $ has simple poles at $z_1,...,z_N,\overline{z}_1,...,\overline{z}_N$ with
      \begin{equation*}%\label{e Res}
        \begin{aligned}
          \res_{z=z_k}m^{(N\text{-sol})} (z;x,t)&=\lim_{z\to z_k}m^{(N\text{-sol})} (z;x,t)
          \left[
            \begin{array}{cc}
              0 & 0 \\
              2\ii\lambda_kc_k e^{2\ii xz_k+4\ii t z_k^2} & 0
            \end{array}
          \right],\\
          \res_{z=\overline{z}_k}m^{(N\text{-sol})} (z;x,t)&=\lim_{z\to \overline{z}_k}m^{(N\text{-sol})} (z;x,t)
          \left[
            \begin{array}{cc}
              0 & \frac{-\overline{c}_k}{2\ii\lambda_k} e^{-2\ii x\overline{z}_k-4\ii t \overline{z}_k^2} \\
              0 & 0
            \end{array}
          \right].
        \end{aligned}
      \end{equation*}
\end{enumerate}
\end{rhp}
\end{framed}
\end{samepage}
Using the ansatz
\begin{equation*}
    m^{(N\text{-sol})} (z;x,t)=1+\sum_{k=1}^N\left\{ \frac{A_k(x,t)}{z-z_k}+ \frac{B_k(x,t)}{z-\overline{z}_k}\right\}
\end{equation*}
we can transfer \rh \ref{rhp N sol} into a purely algebraic system which can be solved explicitly. Then, the reconstruction formulas (\ref{e rec 1}) and (\ref{e rec 2}) yield explicit solutions of the DNLS equation, which are (multi) solitons. For the special case $N=1$ we find
\begin{equation}\label{e soliton}
    u_{\omega,v,x_0,\gamma}(x,t)=\phi_{\omega,v}(x-vt-x_0) e^{-\ii\gamma+\ii\omega t+\ii\frac{v}{2}(x-vt)- \frac{3}{4}\ii\int_{\infty}^{x-vt-x_0} |\phi_{\omega,v}(y)|^2dy},
\end{equation}
where
\begin{equation}\label{e sol ampl}
    \phi_{\omega,v}(x)=\left[\frac{\sqrt{\omega}}{4\omega-v^2} \left\{\cosh(\sqrt{4\omega-v^2}x)- \frac{v}{2\sqrt{\omega}}\right\}\right]^{-1/2}.
\end{equation}
The parameters $(\omega,v)\in \Real^2$ describe the speed and the width of the soliton and  are connected to the pole $z_1$ by
\begin{equation}\label{e omega v}
        \omega=4|z_1|^2,\qquad
         v=-4\re(z_1).
\end{equation}
Note that $v^2<4\omega$ is automatically fulfilled if $z_1\in\Compl_+$. The norming constant $c_1$ influences only the phase and the spatial position of the soliton. To be precise we have
\begin{equation}\label{e x_0 gamma}
    x_0=2\ln\left[\frac{|c_1|}{2\im(z_1)}\right] \left(\sqrt{4\omega-v^2}\right)^{-1},
    \quad
    \gamma=\arg(c_1)+\frac{\pi}{2}+\frac{1}{2}\arg(z_1).
\end{equation}
Expressions for $N$-solitons with $N\geq2$ are large and not presented here. If $\re(z_j)
\neq\re(z_k)$ for $j\neq k$, then for large $|t|$, $N$-solitons break up into $N$  individual solitons of the form (\ref{e soliton}):
\begin{equation}\label{e sol sep}
    u^{(N\text{-sol})}(x,t)\sim \sum_{k=1}^N u_{\omega_k,v_k,x_{0,k}^{\pm},\gamma_k^{\pm}}(x,t), \quad\text{as }t\to\pm\infty.
\end{equation}
If the real parts of two poles $z_j$ and $z_k$ coincide, we obtain a solution having two peaks traveling at the same speed and the separation (\ref{e sol sep}) will not occur. Instead, \emph{breather} phenomena will appear. 
\section{Inverse scattering without poles}\label{s inverse sc}
In this section we are dealing with \rh \ref{rhp m} in the case where $N=0$. Hence, $m$ has no pole in $\Compl\setminus\Real$ and is analytic in $\Compl\setminus\Real$. We recall the associated Riemann--Hilbert problem:\\
\begin{framed}
    \begin{rhp}\label{rhp m^0}
        Find for each $x\in\Real$ a $2\times 2$-matrix valued function $\Compl\ni z\mapsto m(z;x)$ which satisfies
        \begin{enumerate}[(i)]
          \item $m(z;x)$ is meromorphic in $\Compl\setminus\Real$ (with respect to the parameter $z$).
          \item $m(z;x)=1+\mathcal{O}\left(\frac{1}{z}\right)$ as $|z|\to\infty$.
          \item The non-tangential boundary values $m_{\pm}(z;x)$ exist for $z\in\Real$ and satisfy the jump relation
              \begin{equation}\label{e jump m^0}
                  m_+=m_-(1+R),\quad\text{where}\quad
                  R(z;x):=
                  \left(
                    \begin{array}{cc}
                       \overline{r}_+(z)r_-(z) & e^{-2\ii zx}\overline{r}_+(z) \\
                       e^{2\ii zx}r_-(z) & 0 \\
                    \end{array}
                  \right).
              \end{equation}
        \end{enumerate}
    \end{rhp}
\end{framed}
For any function $h\in L^p(\Real)$ with $1\leq p<\infty$, the Cauchy operator denoted by $\pazocal{C}$ is given by
\begin{equation*}
    \pazocal{C}(h)(z):=\frac{1}{2\pi\ii}\int_{\Real} \frac{h(s)}{s-z}ds,\quad z\in\Compl\setminus\Real.
\end{equation*}
When $z$ approaches to a point on the real line transversely from the upper and lower half planes, the Cauchy operator becomes the following projection operators:
\begin{equation*}
    \Ppm(h)(z):=\lim_{\eps\downarrow 0}\frac{1}{2\pi\ii}\int_{\Real} \frac{h(s)}{s-(z\pm\eps)}ds,\quad z\in\Real.
\end{equation*}
The following proposition summarizes all properties which are needed to establish the solvability of \rh \ref{rhp m^0} and furthermore to prove estimates on the solution.
\begin{prop}\label{p cauchy operator}
    \begin{enumerate}[(i)]
      \item For every $h\in L^p(\Real)$, $1\leq p<\infty$, the Cauchy operator $\pazocal{C}(h)$ is analytic off the real line.
      \item For $h\in L^1(\Real)$, $\pazocal{C}(h)(z)$ decays to zero as $|z|\to\infty$ and admits the asymptotic
          \begin{equation}\label{e lim z Ch(z)}
            \lim_{|z|\to\infty}z\pazocal{C}(h)(z)= -\frac{1}{2\pi\ii}\int_\Real h(s) ds,
          \end{equation}
          where the limit is taken either in $\Compl^+$ or $\Compl^-$.
      \item The projection operators $\Ppm$ are linear bounded operators $L^p(\Real)\to L^p(\Real)$ for each $p\in(1,\infty)$. For $p=2$ we have $\|\Ppm\|_{L^2\to L^2}=1$.
      \item For every $x_0\in\Real_+$ and every $r\in H^1(\Real)$, we have
          \begin{equation}\label{e sup P^pm r 1}
              \sup_{x\in(x_0,\infty)} \|\langle x\rangle\Ppm (r(z)e^{\mp2\ii zx})\|_{L^{2}_z(\Real)}\leq \|r\|_{H^1},
          \end{equation}
          where $\langle x\rangle:=\sqrt{1+|x|^2}$. In addition,
          \begin{equation}\label{e sup P^pm r 2}
              \sup_{x\in\Real} \|\Ppm (r(z)e^{\mp2\ii zx})\|_{L^{\infty}_z(\Real)}\leq \frac{1}{\sqrt{2}}\|r\|_{H^1}.
          \end{equation}
          Furthermore, if $r\in L^{2,1}(\Real)$, then
          \begin{equation}\label{e sup P^pm r 3}
              \sup_{x\in\Real}\|\Ppm (zr(z)e^{\mp2\ii zx})\|_{L^{\infty}_z(\Real)}\leq \frac{1}{\sqrt{2}}\|zr\|_{L^{2,1}}.
          \end{equation}
      \item (Sokhotski-Plemelj theorem) The following two identities hold:
      \begin{equation}\label{e Sokhotski-Plemelj}
        \begin{aligned}
            &\Pp-\Pm=\text{Id}_{L^p(\Real)},\\
            &\Pp+\Pm=-\ii\pazocal{H},
        \end{aligned}
      \end{equation}
      where $\pazocal{H}:L^p(\Real)\to L^p(\Real)$ is the Hilbert transform given by
      \begin{equation*}
        \pazocal{H}(h)(z):=\lim_{\eps\downarrow 0}\frac{1}{\pi}\left(\int_{-\infty}^{z-\eps}+ \int^{\infty}_{z+\eps}\right) \frac{h(s)}{s-z}ds,\quad z\in\Real.
      \end{equation*}
      \item Let $f_+$ and $f_-$ functions defined in the upper (lower) $\Compl$-plane. If $f_{\pm}$ is analytic in $\Compl^{\pm}$ and $f_{\pm}(z)\to 0$ as $|z|\to\infty$ for $\im(z)\gtrless0$, then
          \begin{equation}\label{e Ppm of analytic functions}
            \Ppm(f_{\mp})(z)=0,\qquad\Ppm(f_{\pm})(z)=\pm f_{\pm}(z),\quad z\in\Real.
          \end{equation}
    \end{enumerate}
\end{prop}
The Cauchy operator is useful to convert \rh \ref{rhp m^0} into an integral equation. Indeed, the jump condition (\ref{e jump m^0}) can be written as
\begin{equation*}
    (m_+(z;x)-1)-(m_-(z;x)-1)=m_-(z;x)R(z;x).
\end{equation*}
Applying $\Pp$ and $\Pm$ to this equation yields by (\ref{e Ppm of analytic functions}) the following integral equation
\begin{equation}\label{e integral equation for m+-}
    m_{\pm}(z;x)=1+\Ppm(m_-(\cdot;x)R(\cdot;x))(z),\quad z\in\Real,
\end{equation}
which represents the solution of \rh \ref{rhp m^0} on the real line. The following Lemma ensures the solvability of \rh \ref{rhp m} (see Corollary 6 and Lemma 9 in \cite{Pelinovsky2016}):
\begin{lem}\label{l solvability of RHP N=0}
    Let $r_{\pm}\in H^1(\Real)\cap L^{2,1}(\Real)$ such that the relation (\ref{e relation r+ r-}) and the constraint (\ref{e r constraint}) hold. Then there exists an unique solution $m_{\pm}$ of the system of integral equations (\ref{e integral equation for m+-}). Moreover there exists a positive constant $C$ that depends on $\|r_{\pm}\|_{L^{\infty}}$ only such that $m_{\pm}$ enjoys the estimate
    \begin{equation*}
        \|m_{\pm}(\cdot;x)-1\|_{L^2}\leq C(\|r_{+}\|_{L^2} +\|r_{-}\|_{L^2})
    \end{equation*}
    for every $x\in\Real$.
\end{lem}
This Lemma yields indeed a solution of \rh \ref{rhp m^0}, since the analytic continuation of $m_{\pm}$ is found by Proposition \ref{p cauchy operator} (ii):
\begin{equation}\label{e RHP solution formula}
    m(z;x)=1+\frac{1}{2\pi\ii}\int_{\Real} \frac{m_-(y;x)R(y;x)}{y-z}dy,\quad z\in\Compl\setminus \Real.
\end{equation}
Alternatively we can factorize $1+R=(1+R_+)(1+R_-)$ with
\begin{equation}\label{e def R+ and R-}
    R_+(z;x)=
    \left(
      \begin{array}{cc}
            0 & e^{\overline{\phi}(z)}\overline{r}_+(z) \\
            0 & 0 \\
      \end{array}
    \right),\qquad
    R_-(z;x)=
    \left(
      \begin{array}{cc}
            0 & 0 \\
            e^{\phi(z)}r_-(z) & 0 \\
      \end{array}
    \right).
\end{equation}
The jump relation (\ref{e jump}) then becomes $m_+ -m_-=m_-R_+ +m_+R_-$ and applying again $\Ppm$ to this equation yields us
\begin{equation}\label{e alternative RHP solution formula}
    m(z;x)=1+\frac{1}{2\pi\ii}\int_{\Real} \frac{m_-(y;x)R_+(y;x)+ m_+(y;x)R_-(y;x)}{y-z}dy.
\end{equation}
In component form, for the non-tangential limits $z\to\Real$, we find
\begin{equation}\label{e component RHP solution formula}
    m_{\pm}(z;x)=1+
    \left[
      \begin{array}{cc}
        \Ppm\left([m_+(z;x)]_{12} r_-(z) e^{2\ii zx}\right)(z) & \Ppm\left([m_-(z;x)]_{11} \overline{r}_+(z) e^{-2\ii zx}\right)(z) \\
        \Ppm\left([m_+(z;x)]_{22} r_-(z) e^{2\ii zx}\right)(z) & \Ppm\left([m_-(z;x)]_{21} \overline{r}_+(z) e^{-2\ii zx}\right)(z) \\
      \end{array}
    \right].
\end{equation}
In the further analysis of \rh \ref{rhp m} we will meet expressions of the form
\begin{equation}\label{e def I_1,2}
    \begin{aligned}
        &I_1(r)(x):=\frac{1}{2\pi\ii} \int_{\Real}[m_-(y;x)-1]_{11} r(y) e^{-2\ii yx}dy,\\
        &I_2(r)(x):=\frac{1}{2\pi\ii} \int_{\Real}[m_+(y;x)-1]_{22}r(y) e^{2\ii yx}dy,
    \end{aligned}
\end{equation}
where $m_{\pm}$ are the unique solutions of the system of integral equations (\ref{e integral equation for m+-}) and $r$ is some given function.
\begin{prop}\label{p bound <x>^2I}
    Suppose that the assumptions of Lemma \ref{l solvability of RHP N=0} are fulfilled and take $r\in H^1(\Real)\cap L^{2,1}(\Real)$. Then the functionals defined in (\ref{e def I_1,2}) satisfy the bound
    \begin{equation}\label{e bound <x>^2I}
        \begin{aligned}
            &\|I_1(r)\|_{H^1(\Real_+)\cap L^{2,1}(\Real_+)}\leq C\|r_-\|_{H^1\cap L^{2,1}}(\|r_+\|_{H^1\cap L^{2,1}}+\|r_-\|_{H^1\cap L^{2,1}})\|r\|_{H^1\cap L^{2,1}},\\
            &\|I_2(r)\|_{H^1(\Real_+)\cap L^{2,1}(\Real_+)}\leq C\|r_+\|_{H^1\cap L^{2,1}}(\|r_+\|_{H^1\cap L^{2,1}}+\|r_-\|_{H^1\cap L^{2,1}})\|r\|_{H^1\cap L^{2,1}}
        \end{aligned}
    \end{equation}
    where $C$ is a positive constant.
\end{prop}
\begin{proof}
    For the convenience of the reader we prove this proposition although it is already proven in \cite{Pelinovsky2016}. We find by (\ref{e component RHP solution formula}) and integrating by parts
    \begin{eqnarray*}
    % \nonumber to remove numbering (before each equation)
      I_1(r)(x) &=& \frac{1}{2\pi\ii} \int_{\Real}\Pm\left([m_+(z;x)]_{12} r_-(z) e^{2\ii zx}\right)\!(y)\: r(y) e^{-2\ii yx}dy\\
       &=&   \frac{-1}{2\pi\ii} \int_{\Real}[m_+(y;x)]_{12} r_-(y) e^{2\ii zx}\Pp\left(r(z) e^{-2\ii zx}\right)(y) dy.
    \end{eqnarray*}
    Using the H\"{o}lder inequality and the estimate (\ref{e sup P^pm r 1}), we arrive at
    \begin{equation*}
        \sup_{x\in(x_0,\infty)}|\langle x\rangle^2I_1(r)(x)|\leq \|r_-\|_{L^{\infty}}\|r\|_{H^1}\sup_{x\in(x_0,\infty)} \|\langle x\rangle[m_+(y;x)]_{12}\|_{L^2_z}.
    \end{equation*}
    We know $\sup_{x\in(x_0,\infty)} \|\langle x\rangle[m_+(y;x)]_{12}\|_{L^2_z}\leq C\|r_+\|_{H^1}$ by \cite[Lemma 10]{Pelinovsky2016} which completes the proof of $I_1(r)\in L^{2,1}(\Real_+)$. The assertion $\partial_x I_1(r)\in L^{2}(\Real_+)$ is established by using again the inhomogeneous equation (\ref{e component RHP solution formula}), its $x$ derivative, integration by parts, H\"{o}lder inequality, and in the end estimates (\ref{e sup P^pm r 1}) - (\ref{e sup P^pm r 3}),
    \begin{equation*}
        \sup_{x\in(x_0,\infty)} \|\langle x\rangle[m_+(y;x)]_{12}\|_{L^2_z}\leq C\|r_+\|_{H^1}
    \end{equation*}
    and
    \begin{equation*}
        \sup_{x\in\Real} \|[\partial_x m_+(y;x)]_{12}\|_{L^2_z}\leq C(\|r_+\|_{H^1\cap L^{2,1}}+\|r_-\|_{H^1\cap L^{2,1}}).
    \end{equation*}
    The latter statement can also be found in \cite[Lemma 10]{Pelinovsky2016}.
\end{proof}
The proposition above yields directly the following fundamental result (see Lemma 11 in \cite{Pelinovsky2016}):
\begin{cor}\label{c u in H^11 and H^2 (positive half line)}
   Fix $M>0$. Under the assumptions of Lemma \ref{l solvability of RHP N=0} and if $\|r_+\|_{H^1\cap L^{2,1}}+\|r_-\|_{H^1\cap L^{2,1}}\leq M$, the potential $u$ reconstructed from the solution $m$ of \rh \ref{rhp m^0} by using (\ref{e rec 1}) and (\ref{e rec 2}) lies in $H^2(\Real_+)\cap H^{1,1}(\Real_+)$. Moreover, it satisfies the bound
   \begin{equation}\label{e u in H^11 and H^2 (positive half line)}
       \|u\|_{H^2(\Real_+)\cap H^{1,1}(\Real_+)}\leq C_M,
   \end{equation}
   where the constant $C_M$ does not depend on $r_{\pm}$.
\end{cor}
\begin{proof}
    We set
    \begin{equation}\label{e def w}
        w(x):=u(x)e^{\ii\int_{+\infty}^x|u(y)|^2dy}
    \end{equation}
    and
    \begin{equation}\label{e def v}
        v(x):=\overline{u}(x)e^{-\frac{1}{2\ii} \int_{+\infty}^x|u(y)|^2dy},
    \end{equation}
    such that the following relations hold:
    \begin{equation}\label{e relations u v w}
        \begin{aligned}
            |u(x)|&=|v(x)|=|w(x)|\\
            |u_x(x)|&\leq |v_x(x)|+\frac{1}{2}|v(x)|^3
        \end{aligned}
    \end{equation}
    Using the reconstruction formulas (\ref{e rec 1}) and (\ref{e rec 2}), Proposition \ref{p cauchy operator} (ii) and the integral equation (\ref{e alternative RHP solution formula}) we immediately find
    \begin{equation*}
        w(x)=\frac{2}{\pi\ii}\int_{\Real}\overline{r}_+(z) e^{-2\ii zx}dz\;+\;4\,I_1(\overline{r}_+)(x)
    \end{equation*}
    and
    \begin{equation*}
        e^{-\frac{1}{2\ii} \int_{+\infty}^x|u(y)|^2dy}v_x(x)
        =-\frac{1}{\pi}\int_{\Real}r_-(z) e^{2\ii zx}dz\;-\;2\ii\:I_2(r_-)(x).
    \end{equation*}
    In each of these equations the first summand on the right hand side is controlled in $H^1\cap L^{2,1}$ since $r_{\pm}\in H^1\cap L^{2,1}$. Moreover Proposition \ref{p bound <x>^2I} yields directly $w\in L^{2,1}(\Real_+)$ and $v_x\in L^{2,1}(\Real_+)$ and thus finally by (\ref{e relations u v w}) $u\in H^{1,1}(\Real_+)$. Proposition \ref{p bound <x>^2I} also leads to
    \begin{equation*}
        \partial_x\left(e^{-\frac{1}{2\ii} \int_{+\infty}^x|u(y)|^2dy}v_x(x)\right)\in L^2_x(\Real_+).
    \end{equation*}
    By a straightforward calculation we conclude $u\in H^2(\Real_+)$. The bound (\ref{e u in H^11 and H^2 (positive half line)}) is obtained from application of (\ref{e bound <x>^2I}). The proof of the Corollary is now complete.
\end{proof}
With regard to the B\"{a}klund transformation which we intend to use in the following section in order to include solitons we need the following Lemma in addition to (\ref{e u in H^11 and H^2 (positive half line)}). The only purpose in the repeating of so many details of the inverse Scattering withour poles is to deduce this Lemma which can not be found in \cite{Pelinovsky2016}.
\begin{lem}\label{l m(z0)-1 in H^11 and H^2}
   Let the assumptions of Corollary \ref{c u in H^11 and H^2 (positive half line)} be valid and fix $z_0\in\Compl\setminus\Real$. Then for the solution $m(z;x)$ of \rh \ref{rhp m^0} we have $m(z_0;\cdot)-1\in H^1(\Real_+)\cap L^{2,1}(\Real_+)$ with the bound
   \begin{equation}\label{e m(z0)-1 in H^1 and L^2,1}
       \|m(z_0;\cdot)-1\|_{H^1(\Real_+)\cap L^{2,1}(\Real_+)}\leq C_{M},
   \end{equation}
   where the constant $C_M$ depends on $z_0$ and $M$ but not on $r_{\pm}$.
\end{lem}
\begin{proof}
    Fix $z_0\in\Compl\setminus\Real$. We use (\ref{e alternative RHP solution formula}) to find
    \begin{equation}\label{e m21 m12 in L^2,1}
        \begin{aligned}
            &[m(z_0;x)]_{12}= \frac{1}{2\pi\ii} \int_{\Real}\frac{[m_-(y;x)]_{11}\overline{r}_+(y) e^{-2\ii yx}}{y-z_0}dy= \frac{1}{2\pi\ii}\int_{\Real}\widetilde{r}_+(z) e^{-2\ii zx}dz\;+\;I_1(\widetilde{r}_+)(x),\\
            &[m(z_0;x)]_{21}= \frac{1}{2\pi\ii} \int_{\Real}\frac{[m_+(y;x)]_{22}r_-(y) e^{2\ii yx}}{y-z_0}dy= \frac{1}{2\pi\ii}\int_{\Real}\widetilde{r}_-(z) e^{2\ii zx}dz\;+\;I_2(\widetilde{r}_-)(x),
        \end{aligned}
    \end{equation}
    where $\widetilde{r}_-(z):={r}_-(z)/(z-z_0)$ and $\widetilde{r}_+(z):=\overline{r}_+(z)/(z-z_0)$, respectively. Due to the fact that $\|\widetilde{r}_{\pm}\|_{H^1\cap L^{2,1}}\leq c \|{r}_{\pm}\|_{H^1\cap L^{2,1}}$, where the constant $c>0$ depends on $z_0$ only, and using Proposition \ref{p bound <x>^2I} we end up with (\ref{e m(z0)-1 in H^1 and L^2,1}) for the non diagonal entries $m_{12}$ and $m_{21}$.
    Using again (\ref{e alternative RHP solution formula}) we obtain
    \begin{equation*}
        [m(z_0;x)]_{11}=1+\frac{1}{2\pi\ii} \int_{\Real} \frac{[m_+(y;x)]_{12}r_-(y) e^{2\ii yx}}{y-z_0}dy,
    \end{equation*}
    where we can insert $[m_+(y;x)]_{12}=\Pp ([m_-(z;x)]_{11} \overline{r}_+(z) e^{-2\ii zx})(y)$ from the integral equation (\ref{e integral equation for m+-}). Then we integrate by parts and obtain
    \begin{equation}\label{e formula for m(z_0;x)-1 in the proof of the lemma}
        [m(z_0;x)]_{11}=1-\frac{1}{2\pi\ii} \int_{\Real} [m_-(y;x)]_{11} \overline{r}_+(y) e^{-2\ii yx}\;\Pm (\widetilde{r}_-(z) e^{2\ii zx})(y)\:dy,
    \end{equation}
    where we put again $\widetilde{r}_-(z):={r}_-(z)/(z-z_0)$. Furthermore we set
    \begin{equation*}
        R_+(y):=\overline{r}_+(y)\;\Pm (\widetilde{r}_-(z) e^{2\ii zx})(y).
    \end{equation*}
    To prove $R_+\in H^1\cap L^{2,1}$ we recall the continuity property $\|\Ppm\|_{L^2\to L^2}=1$. One consequence is that $\|\Pm (\widetilde{r}_-(z) e^{2\ii zx})(\cdot)\|_{L^2} \leq c \|r_-\|_{L^2}$. Additionally, we find
    \begin{equation*}
        \|\partial_z \Pm (\widetilde{r}_-(z) e^{2\ii zx})(z)\|_{L^2_z}\leq \| \Pm (\widetilde{r}'_-(z) e^{2\ii zx})(\cdot)\|_{L^2_z} +\|2\ii x \Pm (\widetilde{r}_-(z) e^{2\ii zx})(\cdot)\|_{L^2_z},
    \end{equation*}
    where we can apply the bound (\ref{e sup P^pm r 1}) of Proposition \ref{p cauchy operator} and again $\|\Ppm\|_{L^2\to L^2}=1$. Thus, we are able to control  $\Pm (\widetilde{r}_-(z) e^{2\ii zx})(\cdot)$ in $H^1$ uniformly for $x>0$. Altogether, we have shown $\|R_+\|_ {H^1\cap L^{2,1}}\leq c \|r_+\|_ {H^1\cap L^{2,1}}\|r_-\|_ {H^1}$, which is needed because we want to apply Proposition \ref{p bound <x>^2I}. Therefore we write (\ref{e formula for m(z_0;x)-1 in the proof of the lemma}) in the form
    \begin{equation*}
        [m(z_0;x)]_{11}-1=\frac{1}{2\pi\ii} \int_{\Real} R_+(y)e^{-2\ii yx}\:dy+ \:I_1(R_+\!)(x).
    \end{equation*}
    Analogously, it can be carried out in a similar way, that for $R_-(y):=r_-(y)\;\Pp (\widetilde{r}_+(z) e^{-2\ii zx})(y)$,
    \begin{equation*}
        [m(z_0;x)]_{22}-1=\frac{1}{2\pi\ii} \int_{\Real} R_-(y)e^{2\ii yx}\:dy+ \:I_2(R_-\!)(x).
    \end{equation*}
    Combining Fourier theory and the bound (\ref{e bound <x>^2I}) we have now accomplished the proof of (\ref{e m(z0)-1 in H^1 and L^2,1}) also for the diagonal entries.
\end{proof}
Estimates on the negative half-line can be found by modifying the solution $m(z;x)$ of \rh \ref{rhp m^0} in the following way:
\begin{equation}\label{e def m delta}
    m_{\delta}(z;x):=m(z;x)
    \left[
      \begin{array}{cc}
        \delta^{-1}(z) & 0 \\
        0 & \delta(z) \\
      \end{array}
    \right],
\end{equation}
where
\begin{equation}\label{e delta}
    \delta(z)=\exp\left(\frac{1}{2\pi\ii} \int_{\Real}\frac{\log(1+ \overline{r}_+(y)r_-(y))}{y-z}dy\right).
\end{equation}
In Proposition 8 in \cite{Pelinovsky2016} it is shown that $\log(1+ \overline{r}_+r_-)\in L^2(\Real)$ due to (\ref{e r constraint}). Hence, the integral in (\ref{e delta}) is well-defined and $\delta$ solves the following RHP:
\begin{framed}
    \begin{rhp}\label{rhp delta}
        Find a scalar valued function $\Compl\ni z\mapsto \delta(z)$ which satisfies
        \begin{enumerate}[(i)]
          \item $\delta(z)$ is meromorphic in $\Compl\setminus\Real$.
          \item $\delta(z)=1+\mathcal{O} \left(\frac{1}{z}\right)$ as $|z|\to\infty$.
          \item The non-tangential boundary values $\delta_{\pm}(z)$ exist for $z\in\Real$ and satisfy the jump relation
              \begin{equation}\label{e jump delta}
                 \delta_+(z)=\left[1+ \phantom{\widehat{l}}\overline{r}_+(z)r_-(z) \right] \delta_-(z).
              \end{equation}
        \end{enumerate}
    \end{rhp}
\end{framed}
Using the symmetry $\delta(\overline{z})=\overline{\delta}^{-1}(z)$ and the jump condition (\ref{e jump delta}) it is an easy exercise to verify, that the function $m_{\delta}(z;x)$ defined in \ref{e def m delta} is a solution to the following \rh:
\begin{samepage}
\begin{framed}
    \begin{rhp}\label{rhp m^0 delta}
        Find for each $x\in\Real$ a $2\times 2$-matrix valued function $\Compl\ni z\mapsto m_{\delta}(z;x)$ which satisfies
        \begin{enumerate}[(i)]
          \item $m_{\delta}(z;x)$ is meromorphic in $\Compl\setminus\Real$ (with respect to the parameter $z$).
          \item $m_{\delta}(z;x)=1+\mathcal{O} \left(\frac{1}{z}\right)$ as $|z|\to\infty$.
          \item The non-tangential boundary values $m_{\pm,\delta}(z;x)$ exist for $z\in\Real$ and satisfy the jump relation
              \begin{equation}\label{e jump m^0 delta}
                  m_{+,\delta}=m_{-,\delta}(1+R_{\delta}) ,\quad\text{where}\quad
                  R_{\delta}(z;x):=
                  \left[
                    \begin{array}{cc}
                       0 & e^{-2\ii zx}\overline{r}_{+,\delta}(z) \\
                       e^{2\ii zx}r_{-,\delta}(z) & \overline{r}_{+,\delta}(z) r_{-,\delta}(z) \\
                    \end{array}
                  \right],
              \end{equation}
              and $r_{\pm,\delta}(z):= \overline{\delta}_+(z) \overline{\delta}_-(z)r_{\pm}(z)$.
        \end{enumerate}
    \end{rhp}
\end{framed}
\end{samepage}
The new jump matrix $R_{\delta}$ admits an factorization analogously to (\ref{e def R+ and R-}). For
\begin{equation}\label{e def R+ and R- delta}
    R_{+,\delta}(z;x):=
                  \left[
                    \begin{array}{cc}
                       0 & 0 \\
                       e^{2\ii zx}r_{-,\delta}(z) & 0 \\
                    \end{array}
                  \right],\qquad
    R_{-,\delta}(z;x):=
                  \left[
                    \begin{array}{cc}
                       0 & e^{-2\ii zx}\overline{r}_{+,\delta}(z) \\
                       0 & 0 \\
                    \end{array}
                  \right],
\end{equation}
we find
$m_{+,\delta} -m_{-,\delta}=m_{-,\delta}R_{+,\delta} +m_{+,\delta}R_{-,\delta}$ for $z\in \Real$, such that analogously to (\ref{e alternative RHP solution formula}),
\begin{equation*}
    m_{\delta}(z;x)=1+\frac{1}{2\pi\ii}\int_{\Real} \frac{m_{-,\delta}(y;x)R_{+,\delta}(y;x)+ m_{+,\delta}(y;x)R_{-,\delta}(y;x)}{y-z}dy.
\end{equation*}
The following exemplary calculation shows why \rh \ref{rhp m^0 delta} can be studied in order to extend Lemma \ref{l m(z0)-1 in H^11 and H^2} and Corollary \ref{c u in H^11 and H^2 (positive half line)} to the negative half-line. We have for $z_0\in\Compl\setminus\Real$
\begin{eqnarray*}
% \nonumber to remove numbering (before each equation)
  [m_{\delta}(z_0;x)]_{12}&=& \frac{1}{2\pi\ii} \int_{\Real}\frac{[m_{+,\delta}(y;x)]_{11} \overline{r}_{+,\delta}(y) e^{-2\ii yx}}{y-z_0}dy \\
   &=&  \frac{1}{2\pi\ii} \int_{\Real}\widetilde{r}_{+,\delta}(y) e^{-2\ii yx}dy+I_{1,\delta}(\widetilde{r}_{+,\delta})
\end{eqnarray*}
where $\widetilde{r}_{+,\delta}(z):= \overline{r}_{+,\delta}(z)/(z-z_0)$ and
\begin{equation*}
    I_{1,\delta}(r):=\frac{1}{2\pi\ii} \int_{\Real}[m_{+,\delta}(y;x)-1]_{11} r(y) e^{-2\ii yx}dy.
\end{equation*}
The functional $I_{1,\delta}(r)$ satisfies the same estimates as in Proposition \ref{p bound <x>^2I} with $\Real_+$ replaced by $\Real_-$ because the operators $\Pp$ and $\Pm$ swap their places in comparison with the integral equation (\ref{e integral equation for m+-}).
\begin{lem}\label{l m(z0)-1 in in H^1 and L^2,1 (delta)}
   Fix $M>0$ and $z_0\in\Compl\setminus\Real$ and let the assumptions of Lemma \ref{l solvability of RHP N=0} be valid. If in addition $\|r_+\|_{H^1\cap L^{2,1}}+\|r_-\|_{H^1\cap L^{2,1}}\leq M$, then for the solution $m_{\delta}(z;x)$ of \rh \ref{rhp delta} we have $m_{\delta}(z_0;\cdot)-1\in H^1(\Real_-)\cap L^{2,1}(\Real_-)$ with the bound
   \begin{equation*}%\label{e m(z0)-1 in H^1 and L^2,1}
       \|m_{\delta}(z_0;\cdot)-1\|_{H^1(\Real_-)\cap L^{2,1}(\Real_-)}\leq C_{M},
   \end{equation*}
   where the constant $C_M$ depends on $z_0$ and $M$, but not on $r_{\pm}$.
\end{lem}
With respect to the potential $u(x)$ the two Riemann--Hilbert problems \ref{rhp m^0} and \ref{rhp m^0 delta} are equivalent in the following sense:
\begin{equation}\label{e equivalent rhps}
    \begin{aligned}
        \lim_{|z|\to\infty}z\;[m(z;x)]_{12}= \lim_{|z|\to\infty}z\;[m_{\delta}(z;x)]_{12},\\
        \lim_{|z|\to\infty}z\;[m(z;x)]_{21}= \lim_{|z|\to\infty}z\;[m_{\delta}(z;x)]_{21}.
    \end{aligned}
\end{equation}
This observation follows directly from the definition (\ref{e def m delta}) and leads to the following extension of Corollary \ref{c u in H^11 and H^2 (positive half line)}.
\begin{cor}\label{c u in H^11 and H^2 (negative half line)}
   Fix $M>0$. Under the assumptions of Lemma \ref{l solvability of RHP N=0} and if $\|r_+\|_{H^1\cap L^{2,1}}+\|r_-\|_{H^1\cap L^{2,1}}\leq M$, the potential $u$ reconstructed from the solution $m$ of \rh \ref{rhp m^0} by using (\ref{e rec 1}) and (\ref{e rec 2}) lies in $H^2(\Real_-)\cap H^{1,1}(\Real_-)$ and satisfies the bound
   \begin{equation}\label{e u in H^11 and H^2 (negative half line)}
       \|u\|_{H^2(\Real_-)\cap H^{1,1}(\Real_-)}\leq C_M
   \end{equation}
   where the constant $C_M$ does not depend on $r_{\pm}$.
\end{cor}


\section{Adding a pole}\label{s adding a pole}
In this section we want to prove the solvability of \rh \ref{rhp m} if $N=1$. An auto-B\"{a}cklund transformation will establish a connection between the cases $N=1$ and $N=0$. All formulas were found in \cite{Deift2011} and \cite{Cuccagna2014}, where the B\"{a}cklund transformation was used in the context of the NLS equation. \\
Assume that a function $u^{(1)}\in H^2\cap H^{1,1}$ provides scattering data $\mathcal{S}^{(1)}=\set{r^{(1)}_{\pm};z_1;c_1}$. We recall the corresponding Riemann--Hilbert problem (without time dependence):
\begin{samepage}
\begin{framed}
    \begin{rhp}\label{rhp m^1}
        Find for each $x\in\Real$ a $2\times 2$-matrix valued function $\Compl\ni z\mapsto m^{(1)}(z;x)$ which satisfies
        \begin{enumerate}[(i)]
          \item $m^{(1)}(z;x)$ is meromorphic in $\Compl\setminus\Real$ (with respect to the parameter $z$).
          \item $m^{(1)}(z;x)=1+\mathcal{O}\left(\frac{1}{z}\right)$ as $|z|\to\infty$.
          \item The non-tangential boundary values $m^{(1)}_{\pm}(z;x)$ exist for $z\in\Real$ and satisfy the jump relation
              \begin{equation}\label{e jump m^1}
                  m^{(1)}_+=m^{(1)}_-(1+R), \quad\text{where}\quad
                  R(z;x):=
                  \left[
                    \begin{array}{cc}
                       \overline{r}_+(z)r_-(z) & e^{-2\ii zx}\overline{r}_+(z) \\
                       e^{2\ii zx}r_-(z) & 0 \\
                    \end{array}
                  \right].
              \end{equation}
          \item $m^{(1)}$ has simple poles at $z_1$ and $\overline{z}_1$ with
              \begin{equation}\label{e Res m^1}
                  \begin{aligned}
                     \res_{z=z_1}m^{(1)}(z;x)&=\lim_{z\to z_1}m^{(1)}(z;x)
                  \left[
                      \begin{array}{cc}
                        0 & 0 \\
                        2\ii\lambda_1c_1 e^{2\ii z_1x} & 0
                      \end{array}
                  \right],\\
                     \res_{z=\overline{z}_1} m^{(1)}(z;x)&=\lim_{z\to \overline{z}_1}m^{(1)}(z;x)
                  \left[
                      \begin{array}{cc}
                      0 & \frac{-\overline{c}_1 e^{-2\ii \overline{z}_1x}}{2\ii\lambda_1} \\
                      0 & 0
                      \end{array}
                  \right].
                  \end{aligned}
              \end{equation}
        \end{enumerate}
    \end{rhp}
\end{framed}
\end{samepage}
 By construction, the constraints (\ref{e relation r+ r-}) - (\ref{e r constraint}) hold. Now we change these data by removing the pole $z_1$ and modifying the reflection coefficient in the following way:
\begin{equation}\label{e r^0}
    r^{(0)}_{\pm}(z):=r^{(1)}_{\pm}(z) \frac{z-\overline{z}_1}{z-z_1}.
\end{equation}
Obviously, $r^{(0)}_{\pm}$ satisfy (\ref{e relation r+ r-}) -- (\ref{e r constraint}) and moreover, $r^{(1)}_{\pm}\in H^1\cap L^{2,1}$ implies $r^{(0)}_{\pm}\in H^1\cap L^{2,1}$. Hence, all assumptions of Lemma \ref{l solvability of RHP N=0} are satisfied and we get an unique solution $m^{(0)}(z;x)$ of \rh \ref{rhp m^0} with our new data $\mathcal{S}^{(0)}:=\set{r^{(0)}_{\pm}}$. This procedure defines a map $u^{(1)}(x)\mapsto u^{(0)}(x)$, where $u^{(0)}(x)$ is defined to be the pure radiation potential  which is associated to $m^{(0)}(z;x)$ by the reconstruction formulas (\ref{e rec 1}) and (\ref{e rec 1}), respectively.
\subsection{B\"{a}cklund transformation for $x>0$}
What we will do in this subsection is to explore the map $u^{(1)}\leftrightarrow u^{(0)}$ for $x>0$. Therefore we introduce the functions $w^{(j)}$, $v^{(j)}$ for $j=0,1$, which are related to $u^{(j)}$ by (\ref{e def w}) and (\ref{e def v}), respectively. Next we define the matrix
\begin{equation*}
    A(x)=
    \left[
      \begin{array}{cc}
        a_{11}(x) & a_{12}(x) \\
        a_{21}(x) & a_{22}(x) \\
      \end{array}
    \right]
\end{equation*}
by
\begin{equation*}
        \left(
          \begin{array}{c}
            a_{11}(x) \\
            a_{21}(x) \\
          \end{array}
        \right)
        :=m^{(0)}(z_1;x)
        \left(
          \begin{array}{c}
            1 \\
            - \frac{2\ii\lambda_1c_1 e^{2\ii z_1x}} {z_1-\overline{z}_1} \\
          \end{array}
        \right),
        \quad
%
        \left(
          \begin{array}{c}
            a_{12}(x) \\
            a_{22}(x) \\
          \end{array}
        \right)
        :=m^{(0)}(\overline{z}_1;x)
        \left(
          \begin{array}{c}
            \frac{\overline{c}_1 e^{-2\ii \overline{z}_1 x}} {2\ii\overline{\lambda}_1(\overline{z}_1-z_1)} \\
            1 \\
          \end{array}
        \right).
\end{equation*}
In order to define the B\"{a}cklund transformation it is necessary to know that there is no $x$ such that the determinant of $A(x)$ vanishes.
\begin{prop}\label{p A inverse}
    The matrix $A$ is invertible for all $x\in\Real$. Moreover, if $\|r^{(1)}_+\|_{H^1\cap L^{2,1}}+\|r^{(1)}_-\|_{H^1\cap L^{2,1}}\leq M$, then
    \begin{equation} \label{e lower bound for det A}
    |\det(A(x))|^{-1}\leq C_M,\quad\text{for all }x>0,
    \end{equation}
    where the constant $C_M$ does not depend on $x$ and $r_{\pm}$.
\end{prop}
\begin{proof}
    Using the symmetry (\ref{e symmetrie of m}) we find
    \begin{eqnarray*}
    % \nonumber to remove numbering (before each equation)
       \left(
          \begin{array}{c}
            a_{12}(x) \\
            a_{22}(x) \\
          \end{array}
       \right)
       &=&
       \left[
         \begin{array}{cc}
           w^{(0)}(x) & 1 \\
           -|w^{(0)}(x)|^2-4\overline{z}_1 & -\overline{w^{(0)}}(x) \\
         \end{array}
       \right]
       \overline{m^{(0)}}(z_1;x)
       \left[
          \begin{array}{cc}
            0 & \frac{-1}{4\overline{z}_1} \\
            1 & 0 \\
          \end{array}
       \right]
       \left(
          \begin{array}{c}
            \frac{\overline{c}_1 e^{-2\ii \overline{z}_1 x}} {2\ii\overline{\lambda}_1(\overline{z}_1-z_1)} \\
            1 \\
          \end{array}
       \right)
       \\
       &=&
       \frac{1}{4\overline{z}_1}
       \left[
         \begin{array}{cc}
           -w^{(0)}(x) & -1 \\
           |w^{(0)}(x)|^2+4\overline{z}_1 & \overline{w^{(0)}}(x) \\
         \end{array}
       \right]
       \overline{m^{(0)}}(z_1;x)
       \left(
          \begin{array}{c}
            1\\
            \frac{2\ii\overline{\lambda}_1\overline{c}_1 e^{-2\ii \overline{z}_1 x}} {(\overline{z}_1-z_1)} \\
          \end{array}
       \right)
       \\
       &=&
       \frac{1}{4\overline{z}_1}
       \left[
         \begin{array}{cc}
           -w^{(0)}(x) & -1 \\
           |w^{(0)}(x)|^2+4\overline{z}_1 & \overline{w^{(0)}}(x) \\
         \end{array}
       \right]
       \left(
          \begin{array}{c}
            \overline{a_{11}}(x) \\
            \overline{a_{21}}(x) \\
          \end{array}
       \right).
    \end{eqnarray*}
    It follows directly that
    \begin{equation*}
        \det(A(x))=|a_{11}(x)|^2 +\frac{1}{4\overline{z}_1} |\overline{w^{(0)}}(x)a_{11}(x)+a_{21}(x) |^2.
    \end{equation*}
    The case $\det(A(x))=0$ is impossible, since due to $\im(z_1)\neq 0$ it would follow that $a_{11}(x)=a_{21}(x)=0$ and hence $\left(1,
    - \frac{2\ii\lambda_1c_1 e^{2\ii z_1x}}{z_1-\overline{z}_1}\right)^T \in\ker[m^{(0)}]$. This contradicts $\det(m^{(0)}(z;x))\equiv 1$ (see Remark \ref{r uniqueness + det=1}). Now we turn to the proof of (\ref{e lower bound for det A}). For sake of contradiction we assume that for any $d>0$ we can find $x>0$ such that $|\det(A(x))|<d$. Due to $\im(z_1)\neq 0$ and $w\in L^{\infty}$ wa can assume w.l.o.g. $|a_{11}(x)|< d$ and $|a_{12}(x)|<d$. Using (\ref{e det m=1}) we find
    \begin{eqnarray*}
    % \nonumber to remove numbering (before each equation)
      1 &=& \left|[m^{(0)}(z_1;x)]_{11} [m^{(0)}(z_1;x)]_{22}- [m^{(0)}(z_1;x)]_{12} [m^{(0)}(z_1;x)]_{21}\right| \\
       &=& \left|\left\{a_{11}(x) +\frac{2\ii\lambda_1c_1 e^{2\ii z_1x}} {z_1-\overline{z}_1}[m^{(0)}(z_1;x)]_{12}\right\} [m^{(0)}(z_1;x)]_{22}\right.\\
         &&\,
         \left.-\,[m^{(0)}(z_1;x)]_{12} \left\{ a_{21}(x)+ \frac{2\ii\lambda_1c_1 e^{2\ii z_1x}} {z_1-\overline{z}_1}[m^{(0)}(z_1;x)]_{22}\right\} \right|\\
       &=& \left|a_{11}(x) [m^{(0)}(z_1;x)]_{22}- [m^{(0)}(z_1;x)]_{12} a_{21}(x)\right|\\
       &<&d\cdot   \left\{\left|[m^{(0)}(z_1;x)]_{22}\right|+ \left|[m^{(0)}(z_1;x)]_{12}\right|\right\}\\
       &\leq&d\cdot C\cdot  \|m^{(0)}(z_1;\cdot)-1\|_{H^1(\Real_+)\cap L^{2,1}(\Real_+)}\\
       &\leq& d\cdot C_M.
    \end{eqnarray*}
    Here $C_M$ is the constant in Lemma \ref{l m(z0)-1 in H^11 and H^2} and it follows that $d$ cannot be arbitrary small. In addition we also proved the bound (\ref{e lower bound for det A}).
\end{proof}
\begin{lem}\label{l solvability of RHP N=1}
    For any scattering data $\mathcal{S}^{(1)}=\{r^{(1)}_{\pm};z_1;c_1\}$ such that $r^{(1)}_{\pm}\in L^{2,1}\cap H^1$ satisfies (\ref{e relation r+ r-})-(\ref{e r constraint}), \rh \ref{rhp m^1} admits an unique solution $m^{(1)}(z;x)$. This solution can be obtained from $m^{(0)}(z;x)$ by the following:
    \begin{equation}\label{e B�cklund for m^1}
        m^{(1)}(z;x)=A(x)\mu(z) A^{-1}(x)m^{(0)}(z;x)\mu^{-1}(z),
    \end{equation}
    where
    \begin{equation*}
        \mu(z)=
        \left[
          \begin{array}{cc}
            z-z_1 & 0 \\
            0 & z-\overline{z}_1 \\
          \end{array}
        \right].
    \end{equation*}
\end{lem}
\begin{proof}
    Let us denote by $\widetilde{m}(z;x)$ the right hand side of (\ref{e B�cklund for m^1}) and set
    \begin{equation*}
        \left[
          \begin{array}{cc}
            \tau_{11}(z) & \tau_{12}(z) \\
            \tau_{21}(z) & \tau_{22}(z) \\
          \end{array}
        \right]:=A^{-1}(x)m^{(0)}(z;x).
    \end{equation*}
    We find
    \begin{equation}\label{e Res of m tilde}
        \begin{aligned}
          \res_{z=z_1}\widetilde{m}(z;x)&=
          A(x)
          \left[
            \begin{array}{cc}
              0 & 0 \\
              (z_1-\overline{z}_1)\tau_{21}(z_1) & 0 \\
            \end{array}
          \right]
          ,\\
          \res_{z=\overline{z}_1}\widetilde{m} (z;x)&=
          A(x)
          \left[
            \begin{array}{cc}
              0 & (\overline{z}_1-z_1) \tau_{12}(\overline{z}_1) \\
              0 & 0 \\
            \end{array}
          \right]
          ,
        \end{aligned}
    \end{equation}
    and
    \begin{equation}\label{e lim m tilde c}
        \begin{aligned}
        &\lim_{z\to z_1}\widetilde{m}(z;x)
          \left(
            \begin{array}{cc}
              0 & 0 \\
              2\ii\lambda_1c_1 e^{2\ii z_1x} & 0
            \end{array}
          \right)=
          A(x)
          \left[
            \begin{array}{cc}
              0 & 0 \\
              2\ii\lambda_1c_1 e^{2\ii z_1x}\tau_{22}(z_1) & 0 \\
            \end{array}
          \right]
          ,\\
          &\lim_{z\to \overline{z}_1}\widetilde{m}(z;x)
          \left(
            \begin{array}{cc}
              0 & \frac{-\overline{c}_1}{2\ii\lambda_1} e^{-2\ii \overline{z}_1x} \\
              0 & 0
            \end{array}
          \right)=
          A(x)
          \left[
            \begin{array}{cc}
              0 &\frac{-\overline{c}_1}{2\ii\lambda_1} \tau_{11}(\overline{z}_1) \\
              0 & 0 \\
            \end{array}
          \right].
        \end{aligned}
      \end{equation}
      Using $\det m^{(0)}\equiv 1 $ it is easy to obtain
      \begin{equation*}
        \tau_{21}(z_1)=\frac{1}{\det A(x)} \frac{2\ii\lambda_1c_1 e^{2\ii z_1x}}{z_1-\overline{z}_1},\quad
        \tau_{22}(z_1)=\frac{1}{\det A(x)},
      \end{equation*}
      and
      \begin{equation*}
        \tau_{11}(\overline{z}_1)=\frac{1}{\det A(x)},\quad\tau_{12}(\overline{z}_1)= \frac{-1}{\det A(x)} \frac{\overline{c}_1 e^{-2\ii \overline{z}_1x}}{2\ii\overline{\lambda}_1 (\overline{z}_1-z_1)},
      \end{equation*}
      and thus it follows from (\ref{e Res of m tilde}) and (\ref{e lim m tilde c}) that $\widetilde{m}$ satisfies (\ref{e Res m^1}). Now we proceed with the jump on the real axis and check if point (iii) of \rh \ref{rhp m^1} is satisfied. Using the jump condition of $m^{(0)}$ (see (\ref{e jump})) and the definition (\ref{e r^0}) of $r_{\pm}^{(0)}$ we find for $z\in\Real$
      \begin{eqnarray*}
      % \nonumber to remove numbering (before each equation)
        \widetilde{m}_+(z;x) &=& \widetilde{m}_-(z;x) \mu(z)\left(
          \begin{array}{cc}
            1+\overline{r}^{(0)}_+(z)r^{(0)}_-(z) & e^{-2\ii zx}\overline{r}^{(0)}_+(z) \\
            e^{2\ii zx}r^{(0)}_-(z) & 1 \\
          \end{array}
        \right)\mu^{-1}(z) \\
         &=&  \widetilde{m}_-(z;x)\left(
          \begin{array}{cc}
            1+\overline{r}^{(1)}_+(z)r^{(1)}_-(z) & e^{-2\ii zx}\overline{r}^{(1)}_+(z) \\
            e^{2\ii zx}r^{(1)}_-(z) & 1 \\
          \end{array}
        \right).
      \end{eqnarray*}
      Next we observe
      \begin{equation}\label{e expansion of m tilde}
        \widetilde{m}(z;x)= \left[1+\frac{A(x)\:\mu(0)\:A^{-1}(x)}{z}\right] m^{(0)}(z;x)
        \left[
          \begin{array}{cc}
            \frac{z}{z-z_1} & 0 \\
            0 & \frac{z}{z-\overline{z}_1} \\
          \end{array}
        \right].
      \end{equation}
      It follows that $\widetilde{m}$ behaves for $|z|\to\infty$ as required in the point (ii) of \rh \ref{rhp m^1}. Since also the point (i) of \rh \ref{rhp m^1} is true, we conclude by the uniqueness (see Remark \ref{r uniqueness + det=1}) that $m^{(1)}(z;x)\equiv\widetilde{m}(z;x)$.
\end{proof}
The B\"{a}cklund transformation formula (\ref{e B�cklund for m^1}) is an ideal expression to extend Corollary \ref{c u in H^11 and H^2 (positive half line)} and Lemma \ref{l m(z0)-1 in H^11 and H^2} to the case where the scattering data are involving one pole $z_1$.
\begin{cor}\label{c u in H^11 and H^2 (positive half line - 1 pole)}
   Under the assumptions of Lemma \ref{l solvability of RHP N=1} the potential $u^{(1)}(x)$ reconstructed from the solution $m^{(1)}(z;x)$ of \rh \ref{rhp m dynamic} by using (\ref{e rec 1}) and (\ref{e rec 2}) lies in $H^2(\Real_+)\cap H^{1,1}(\Real_+)$. Moreover, if $\|r^{(1)}_+\|_{H^1\cap L^{2,1}}+\|r^{(1)}_-\|_{H^1\cap L^{2,1}}+|c_1|\leq M$ for some fixed $M>0$, then $u^{(1)}$ satisfies the bound
   \begin{equation}\label{e u^1 in H^11 and H^2 (positive half line)}
       \|u^{(1)}\|_{H^2(\Real_+)\cap H^{1,1}(\Real_+)}\leq C_M
   \end{equation}
   where the constant $C_M$ depends on $M$ and $z_1$ but not on $r^{(1)}_{\pm}$ and $|c_1|$.
\end{cor}
\begin{proof}
    We use (\ref{e expansion of m tilde}) and
    the expansion
    \begin{equation*}
        \left[
          \begin{array}{cc}
            \frac{z}{z-z_1} & 0 \\
            0 & \frac{z}{z-\overline{z}_1} \\
          \end{array}
        \right]=1-\frac{\mu(0)}{z}+\mathcal{O}(z^{-2}), \quad\text{as }|z|\to\infty
    \end{equation*}
    in order to find
    \begin{equation*}
        \begin{aligned}
            \lim_{|z|\to\infty}z\; \left[m^{(1)}(z;x)\right]_{12}& = \lim_{|z|\to\infty}z\; \left[m^{(0)}(z;x)\right]_{12}+ \left[A(x)\:\mu(0)\:A^{-1}(x)\right]_{12},\\
            \lim_{|z|\to\infty}z\; \left[m^{(1)}(z;x)\right]_{21}& = \lim_{|z|\to\infty}z\; \left[m^{(0)}(z;x)\right]_{21}+ \left[A(x)\:\mu(0)\:A^{-1}(x)\right]_{21}.
        \end{aligned}
    \end{equation*}
    Using the notation (\ref{e def w}) and (\ref{e def v}), we find by the reconstruction formulas (\ref{e rec 1}) and (\ref{e rec 2})
    \begin{equation}\label{e decomposition w}
       w^{(1)}(x)=w^{(0)}(x)+B_1(x),\qquad B_1(x):=-\frac{8\ii \im(z_1)a_{11}(x)a_{12}(x)}{\det(A(x))}
    \end{equation}
    and
    \begin{multline}\label{e decomposition v}
        \qquad e^{-\frac{1}{2\ii} \int_{+\infty}^x|u^{(1)}(y)|^2dy}v^{(1)}_x(x)= e^{-\frac{1}{2\ii} \int_{+\infty}^x|u^{(0)}(y)|^2dy}v^{(0)}_x(x) +B_2(x),\\
        B_2(x):=\frac{4 \im(z_1)a_{21}(x)a_{22}(x)}{\det(A(x))}.\qquad
    \end{multline}
    As it is easily to derive from the definition of $A(x)$ and Lemma \ref{l m(z0)-1 in H^11 and H^2}, we have $(A(\cdot)-1)\in L^{2,1}(\Real_+)\cap H^1(\Real_+)$ (note that $\im(z_1)>0$ is necessary). In addition, $(\det(A(\cdot))-1)\in L^{2,1}(\Real_+)\cap H^1(\Real_+)$. These two facts and \ref{e lower bound for det A} yield $B_j(\cdot)\in L^{2,1}(\Real_+)\cap H^1(\Real_+)$ for $j=1,2$. If we apply Corollary \ref{c u in H^11 and H^2 (positive half line)} to $v^{(0)}$ and $w^{(0)}$, we end up with $w^{(1)}\in H^{1,1}(\Real_+)$ and
    \begin{equation*}
        \partial_x \left(e^{-\frac{1}{2\ii} \int_{+\infty}^x|u^{(1)}(y)|^2dy } v^{(1)}_x(x)\right)\in L^2_x(\Real_+),
    \end{equation*}
    which is sufficient to conclude $u^{(1)}\in H^2(\Real_+)\cap H^{1,1}(\Real_+)$.
\end{proof}
\begin{cor}\label{c m^1(z0)-1 in H^11 and H^2}
   Let the assumptions of Lemma \ref{l solvability of RHP N=1} be valid and fix $z_2\in\Compl\setminus(\Real\cup \set{z_1,\overline{z}_1})$. Then for the solution $m^{(1)}(z;x)$ of \rh \ref{rhp m dynamic} we have $m^{(1)}(z_2;\cdot)-1\in H^1(\Real_+)\cap L^{2,1}(\Real_+)$. Moreover, if $\|r^{(1)}_+\|_{H^1\cap L^{2,1}}+\|r^{(1)}_-\|_{H^1\cap L^{2,1}}\leq M$ for some fixed $M>0$, then we also have the bound
   \begin{equation}\label{e m^1(z0)-1 in H^1 and L^2,1}
       \|m^{(1)}(z_2;\cdot)-1\|_{H^1(\Real_+)\cap L^{2,1}(\Real_+)}\leq C_M,
   \end{equation}
   where the constant $C_M>0$ depends on $M$, $z_1$, $z_2$ and $|c_1|$ but not on $r^{(1)}_{\pm}$.
\end{cor}
\begin{proof}
    (\ref{e B�cklund for m^1}) can be written as
    \begin{multline*}
        m^{(1)}(z_2;x)=m^{(0)}(z_2;x)\\-2\ii\im(z_1) A(x)
        \left[
          \begin{array}{cc}
            0 & \frac{a_{21}(x)[m(z_2;x)]_{11}- a_{11}(x)[m(z_2;x)]_{21}} {(z_2-z_1)\det(A(x))} \\
            \frac{a_{22}(x)[m(z_2;x)]_{12}- a_{12}(x)[m(z_2;x)]_{22}} {(z_2-\overline{z}_1)\det(A(x))} & 0 \\
          \end{array}
        \right].
    \end{multline*}
    $m^{(1)}(z_2;\cdot)-1\in H^1(\Real_+)\cap L^{2,1}(\Real_+)$ is now a direct consequence of $m^{(0)}(z_2;\cdot)-1\in H^1(\Real_+)\cap L^{2,1}(\Real_+)$ (see Lemma \ref{l m(z0)-1 in H^11 and H^2}), $(A(\cdot)-1)\in L^{2,1}(\Real_+)\cap H^1(\Real_+)$, $(\det(A(\cdot))-1)\in L^{2,1}(\Real_+)\cap H^1(\Real_+)$ and (\ref{e lower bound for det A}).
\end{proof} 
\subsection{B\"{a}cklund transformation for $x<0$}
We consider the solution $m^{(1)}(z;x)$ of \rh \ref{rhp m^1} provided by Lemma \ref{l solvability of RHP N=1} and define
\begin{equation}\label{e def m^1 delta}
    m^{(1)}_{\delta}(z;x):=m^{(1)}(z;x)
    \left[
      \begin{array}{cc}
        \frac{z-z_1}{z-\overline{z}_1} & 0 \\
        0 & \frac{z-\overline{z}_1}{z-z_1} \\
      \end{array}
    \right]
    \left[
      \begin{array}{cc}
        \delta^{-1}(z) & 0 \\
        0 & \delta(z) \\
      \end{array}
    \right].
\end{equation}
The factor $\left(\frac{z-z_1}{z-\overline{z}_1}\right)^{\sigma_1}$ swaps the columns where the poles arise. The second factor $\delta^{-\sigma_1}$ has influence on the structure of the jump matrix. It can be shown by elementary calculations that (\ref{e def m^1 delta}) yields a solution of the following \rh.
\begin{samepage}
\begin{framed}
    \begin{rhp}\label{rhp m^1 delta}
        Find for each $x\in\Real$ a $2\times 2$-matrix valued function $\Compl\ni z\mapsto m_{\delta}^{(1)}(z;x)$ which satisfies
        \begin{enumerate}[(i)]
          \item $m_{\delta}^{(1)}(z;x)$ is meromorphic in $\Compl\setminus\Real$ (with respect to the parameter $z$).
          \item $m_{\delta}^{(1)}(z;x)=1+\mathcal{O}\left(\frac{1}{z}\right)$ as $|z|\to\infty$.
          \item The non-tangential boundary values $m^{(1)}_{\pm,\delta}(z;x)$ exist for $z\in\Real$ and satisfy the jump relation
              \begin{equation}\label{e jump m^1 delta}
                  m^{(1)}_{+,\delta}=m^{(1)}_{-,\delta} (1+R^{(1)}_{\delta}), \quad\text{where}\quad
                  R^{(1)}_{\delta}(z;x):=
                  \left[
                    \begin{array}{cc}
                       0 & e^{-2\ii zx}\overline{r}^{(1)}_{+,\delta}(z) \\
                       e^{2\ii zx}r^{(1)}_{-,\delta}(z) & \overline{r}^{(1)}_{+,\delta}(z) r_{-,\delta}^{(1)}(z) \\
                    \end{array}
                  \right],
              \end{equation}
              and $r^{(1)}_{\pm,\delta}(z):=r^{(1)}_{\pm} (z)\overline{\delta}_+(z) \overline{\delta}_-(z)\left( \frac{z-z_1}{z-\overline{z}_1}\right)^2$.
          \item $m_{\delta}^{(1)}$ has simple poles at $z_1$ and $\overline{z}_1$ with
              \begin{equation*}\label{e Res m^1 delta}
                  \begin{aligned}
                     \res_{z=z_1} m_{\delta}^{(1)}(z;x)&=\lim_{z\to z_1}m_{\delta}^{(1)}(z;x)
                  \left[
                      \begin{array}{cc}
                        0 & \frac{-e^{-2\ii z_1x} [2 \im(z_1)]^2} {\delta^{-2}(z_1)2\ii\lambda_1c_1} \\
                        0 & 0
                      \end{array}
                  \right],\\
                     \res_{z=\overline{z}_1} m_{\delta}^{(1)}(z;x)&=\lim_{z\to \overline{z}_1}m_{\delta}^{(1)}(z;x)
                  \left[
                      \begin{array}{cc}
                      0 & 0 \\
                      \frac{2\ii\lambda_1[2 \im(z_1)]^2e^{2\ii \overline{z}_1x}} {\delta^{2}(\overline{z}_1) \overline{c}_1} & 0
                      \end{array}
                  \right].
                  \end{aligned}
              \end{equation*}
        \end{enumerate}
    \end{rhp}
\end{framed}
\end{samepage}
Analogously to the previous subsection we set
\begin{equation*}
    r^{(0)}_{\pm,\delta}(z):= r^{(1 )}_{\pm,\delta}(z) \frac{z-\overline{z}_1}{z-z_1}= r^{(1)}_{\pm} (z)\overline{\delta}_+(z) \overline{\delta}_-(z) \frac{z-z_1}{z-\overline{z}_1},
\end{equation*}
and define $m^{(0)}_{\delta}(z;x)$ to be the unique solution
of \rh\ref{rhp m^0 delta} with data $\mathcal{S}^{(0)}_{\delta}:=\set{r^{(0)}_{\pm,\delta}}$. We have $r^{(0)}_{\pm,\delta}\in L^{2,1}\cap H^1$ and hence, the statements of Lemma \ref{l m(z0)-1 in in H^1 and L^2,1 (delta)} and Corollary \ref{c u in H^11 and H^2 (negative half line)} are available. Next we want to describe how the solutions $m^{(1)}_{\delta}(z;x)$ and $m^{(0)}_{\delta}(z;x)$ are connected by a B\"{a}cklund transformation of the form (\ref{e B�cklund for m^1}). For this purpose we define
\begin{align*}
        &\left(
          \begin{array}{c}
            \vspace{1mm} a^{(\delta)}_{11}(x) \\
            a^{(\delta)}_{21}(x) \\
          \end{array}
        \right)
        :=m^{(0)}_{\delta}(\overline{z}_1;x)
        \left(
          \begin{array}{c}
            1 \\
            \frac{2\ii\lambda_1[2 \im(z_1)]^2e^{2\ii \overline{z}_1x}} {\delta^{2}(\overline{z}_1) \overline{c}_1} \\
          \end{array}
        \right),\\
        &\left(
          \begin{array}{c}
            \vspace{1mm} a_{12}^{(\delta)}(x) \\
            a_{22}^{(\delta)}(x) \\
          \end{array}
        \right)
        :=m^{(0)}_{\delta}(z_1;x)
        \left(
          \begin{array}{c}
            \frac{-e^{-2\ii z_1x} [2 \im(z_1)]^2} {\delta^{-2}(z_1)2\ii\lambda_1c_1} \\
            1 \\
          \end{array}
        \right),
\end{align*}
and $A^{(\delta)}(x)=
    \left[
      \begin{array}{cc}
        a^{(\delta)}_{11}(x) & a^{(\delta)}_{12}(x) \\
        a^{(\delta)}_{21}(x) & a^{(\delta)}_{22}(x) \\
      \end{array}
    \right]$. It turns out that
\begin{equation}\label{e B�cklund for m^1 delta}
    m^{(1)}_{\delta}(z;x)=A^{(\delta)}(x)
    \left[
      \begin{array}{cc}
        z-\overline{z}_1 & 0 \\
        0 & z-z_1 \\
      \end{array}
    \right]
    \left[A^{(\delta)}(x)\right]^{-1}m^{(0)}_{\delta}(z;x)
    \left[
      \begin{array}{cc}
        \frac{1}{z-\overline{z}_1} & 0 \\
        0 & \frac{1}{z-z_1} \\
      \end{array}
    \right].
\end{equation}
Due to $\im(z_1)>0$ we have $e^{-2\ii z_1x}\in H^{1}_x(\Real_-)\cap L^{2,1}_x(\Real_-)$. Additionally, $(m^{(0)}_{\delta}(z_1;\cdot)-1)\in H^1(\Real_-)\cap L^{2,1}(\Real_-)$ and thus we find $(A^{(\delta)}(\cdot)-1)\in H^1(\Real_-)\cap L^{2,1}(\Real_-)$. These observation bring us in the position to extend the results of the previous subsection to the negative half-line.
\begin{cor}\label{c m^1(z0)-1 in L^21 and H^1 (delta)}
   Let the assumptions of Lemma \ref{l solvability of RHP N=1} be valid and fix $z_2\in\Compl\setminus(\Real\cup \set{z_1,\overline{z}_1})$. Then for the solution $m_{\delta}^{(1)}(z;x)$ of \rh\ref{rhp m^1 delta} we have $m_{\delta}^{(1)}(z_2;\cdot)-1\in H^1(\Real_-)\cap L^{2,1}(\Real_-)$. Moreover, if $\|r^{(1)}_+\|_{H^1\cap L^{2,1}}+\|r^{(1)}_-\|_{H^1\cap L^{2,1}}+|c_1|\leq M$ for some fixed $M>0$, then we also have the bound
   \begin{equation}\label{e m^1_delta(z0)-1 in H^1 and L^2,1}
       \|m_{\delta}^{(1)}(z_2;\cdot)-1\|_{H^1(\Real_-)\cap L^{2,1}(\Real_-)}\leq C_M,
   \end{equation}
   where the constant $C_M>0$ depends on $M$, $z_1$ and $z_2$ but not on $r^{(1)}_{\pm}$ and $|c_1|$.
\end{cor}
\begin{cor}\label{c u in H^11 and H^2 (negative half line - 1 pole)}
   Under the assumptions of Lemma \ref{l solvability of RHP N=1} the potential $u_{\delta}^{(1)}(x)$ reconstructed from the solution $m_{\delta}^{(1)}(z;x)$ of \rh\ref{rhp m^1 delta} by using (\ref{e rec 1}) and (\ref{e rec 2}) lies in $H^2(\Real_-)\cap H^{1,1}(\Real_-)$. Moreover, if $\|r^{(1)}_+\|_{H^1\cap L^{2,1}}+\|r^{(1)}_-\|_{H^1\cap L^{2,1}}+|c_1|\leq M$ for some fixed $M>0$, then we also have the bound
   \begin{equation}\label{e u^1 in H^11 and H^2 (negative half line)}
       \|u_{\delta}^{(1)}\|_{H^2(\Real_-)\cap H^{1,1}(\Real_-)}\leq C_M
   \end{equation}
   where the constant $C_M>0$ depends on $M$ and $z_1$, but not on $r^{(1)}_{\pm}$ and $|c_1|$.
\end{cor}
We finish the section with the following observation which is obvious from the definition:
\begin{equation*}
    \begin{aligned}
        \lim_{|z|\to\infty}z\;[m^{(1)}(z;x)]_{12}= \lim_{|z|\to\infty}z\;[m^{(1)}_{\delta}(z;x)]_{12},\\
        \lim_{|z|\to\infty}z\;[m^{(1)}(z;x)]_{21}= \lim_{|z|\to\infty}z\;[m^{(1)}_{\delta}(z;x)]_{21}.
    \end{aligned}
\end{equation*}
It follows that $u^{(1)}_{\delta}=u^{(1)}$. In conclusion the Corollaries \ref{c u in H^11 and H^2 (positive half line - 1 pole)} and \ref{c u in H^11 and H^2 (negative half line - 1 pole)} yield the existence of the mapping
\begin{equation}\label{e u in H^11 and H^2}
    H^1(\Real)\cap L^{2,1}(\Real)\ni (r^{(1)}_-,r^{(1)}_+)\mapsto u^{(1)}\in H^2(\Real)\cap H^{1,1}(\Real).
\end{equation}
 
\subsection{Proofs of \Cref{sec:ergodicity-hmc}}


% \begin{proof}
%\end{proof}

\subsubsection{Proof of \Cref{theo:irred_harris} }
\label{sec:proof-crefth-harris_0}
We first prove  \eqref{theo:irred_harris_a}.  Under the assumption that $\F$ is twice continuously
  differentiable, it follows by a straightforward induction, that for
  all $h >0$ and $q \in \rset^d$, $p \mapsto
  \Phiverletq[h][k](q,p)$, defined by  \eqref{eq:def_Phiverletq}, and $p \mapsto \gperthmc[k](q,p)$, defined by \eqref{eq:def_gperthmc}, are
  continuously differentiable and for all $(q,p) \in \rset^d \times
  \rset^d$,
\begin{equation}
  \Jac_{p,\gperthmc[T]}(q,p) =  \sum_{i=1}^{T-1}(T-i)\defEns{\nabla^2 \F \circ \Phiverletq[h][i](q,p)} \Jac_{p,\Phiverletq[h][i]}(q,p) \eqsp,
\end{equation}
where for all $q \in \rset^d$, $\Jac_{p,\gperthmc[k]}(q,p)$ ($\Jac_{p,\Phiverletq[i][h]}(q,p)$ respectively) is the Jacobian of the function $\tilde{p} \mapsto
\gperthmc[k](q,\tilde{p})$ ($\tilde{p} \mapsto
\Phiverletq[i][h](q,\tilde{p})$ respectively) at $p \in \rset^{d}$.


%  Under \Cref{assum:regOne}, $\sup_{x \in \rset^d} \normLigne{\nabla^2 \F(x)}
% \leq \constzero$ and by
% \Cref{lem:bound_first_iterate_leapfrog_a},
%  $ \sup_{(q,p) \in \rset^d \times \rset^d} \normLigne{\nabla_p \Phiverletq[h][i](q,p)} \leq (1+h \vartheta_1(h))^i$ for any $i \in \nsets$.
% Therefore for any $h >0$, $T \in \nsets$, setting $S = hT$ and using that $\tilde{h} \mapsto \vartheta_1(\tilde{h})$ is nondecreasing and greater than $1$ on $\rset_+^*$ and for any $u,s \geq 0$, $u \geq 1$, $(1+s/u)^{u-1} \leq \log(s+1) \rme^s$, we get that
Under \Cref{assum:regOne}, $\sup_{x \in \rset^d} \normLigne{\nabla^2 \F(x)}
 \leq \constzero$, therefore by \Cref{lem:inverse_1}, we have that for any $T \in \nsets$ and $h >0$,
\begin{equation}
  \label{eq:inverse_1}
 \sup_{(\q,\p) \in \rset^d \times \rset^d} \norm{\Jac_{p,\gperthmc[T]}(q,p)}
 \leq  T (\{1 + h \constzero^{1/2} \vartheta_1(h \constzero^{1/2})\}^T  -1) /h  \eqsp.
\end{equation}
%$\sup_{p \in \rset^d } \nabla_p \gperthmc[k](q,p) \leq C$.
% \begin{equation}
% \label{lem:inverse_1}
% \sup_{p \in \rset^d } \nabla_p \gperthmc[k](q,p) \leq C \eqsp.
% \end{equation}
% Then for all $q, p_1,p_2 \in \rset^d$,
% \begin{equation}
% \label{lem:inverse_1}
% \norm{\gperthmc[k](q,p_1) -\gperthmc[k](q,p_2)} \leq C \norm{p_1 - p_2} \eqsp.
% \end{equation}
For any $q \in \rset^d$, $T\in \nsets$ and $h >0$, define $\phia_{q,T,h}(p)$  for all  $p \in \rset^d$ by
\begin{equation}
  \phia_{q,T,h} (p) = p-(h/T) \gperthmc[T](q,p) \eqsp.
\end{equation}
It is a well known fact (see for example
\cite[Exercise 3.26]{duistermaat:kolk:2004}) that if
\begin{equation}
  \label{eq:inverse_1_2}
  \sup_{(q,p) \in \rset^d \times \rset^d} (h/T)\norm{ \Jac_{p,\gperthmc[T]}(q,p)} < 1 \eqsp,
\end{equation}
then for any $q \in \rset^d$, $\phia_{q,T,h}$ is a
diffeomorphism and  therefore by \eqref{eq:qk}, the same conclusion holds
for $p \mapsto \Phiverletq[h][T](q,p)$. Using \eqref{eq:inverse_1}, if $T \in \nsets$ and $h > 0$  satisfies \eqref{eq:condition-h,T-harris},
then the condition \eqref{eq:inverse_1_2} is verified and as a result \eqref{theo:irred_harris_a}.

Denoting for any $q \in \rset^d$ by $\Phiverletqi[h][T](q,\cdot) : \rset^d \to \rset$ the
continuously differentiable inverse of $p \mapsto
\Phiverletq[h][T](q,p)$ and using a change of variable with $\Phiverletqi[h][T](q,\cdot)$ in \eqref{eq:def_kernel_hmc} concludes the proof of \eqref{eq:def_kernel_hmc_false_density}.

We now show that $\Tker_{h,T}$ satisfies the condition which implies that $\Pkerhmc[h][T]$ is a \Tkernel. We first establish some estimates on the function $(q,p) \mapsto \Phiverletqi[h][T](q,p)$. By
\eqref{eq:inverse_1_2} and \eqref{eq:qk}, for any $q,p,v \in \rset^d$, there exists $\varepsilon \in \ooint{0,1}$ such that $  \normLigne{\Phiverletq[h][T](q,p)-\Phiverletq[h][T](q,v)} \geq (hT) \normLigne{\phi_{q,T,h}(p)-\phi_{q,T,h}(v)} \geq (hT) (1-\varepsilon)\norm{p-v}$ which implies that that there exists $C \geq 0$ satisfying
\begin{equation}
  \label{eq:regularity_phinverse1}
  \begin{aligned}
    \norm{\Phiverletqi[h][T](q,p)-\Phiverletqi[h][T](q,v)} &\leq (1-\varepsilon)^{-1} \norm{v-p}\eqsp, \\
    \norm{  \Phiverletqi[h][T](q,p)} &\leq C\defEns{\norm{\p} + \norm{\Phiverletq[h][T](q,0)}} \eqsp.
  \end{aligned}
\end{equation}
In addition, for $q,x,p \in \rset^d$, we have setting $\tilde{q} = \Phiverletqi[h][T](q,p)$ that
\begin{align}
  \nonumber
  \normLigne{\Phiverletqi[h][T](q,p) - \Phiverletqi[h][T](x,p)} &= \normLigne{\tilde{q} - \Phiverletqi[h][T](x, \Phiverletq[h][T](q,\tilde{q}))} \\
  \nonumber
                                                                &= \normLigne{\Phiverletqi[h][T](x, \Phiverletq[h][T](x,\tilde{q})) - \Phiverletqi[h][T](x, \Phiverletq[h][T](q,\tilde{q}))} \eqsp,
\end{align}
which implies by \eqref{eq:regularity_phinverse1} and \Cref{lem:bound_first_iterate_leapfrog_a}
that there exists $C \geq 0$ satisfying
\begin{equation}
  \label{eq:regularity_phinverse2}
  \norm{\Phiverletqi[h][T](q,p) - \Phiverletqi[h][T](x,p)} \leq C \norm{q-x} \eqsp.
\end{equation}


We now can prove that $\Tker_{h,T}$ is the continuous component of $\Pkerhmc[h][T]$. First by \eqref{eq:def_tker}, for all $\eventB \in \borelSet(\rset^d)$,
\begin{equation}
\label{eq:minoration_pseudo_density_P}
    \Tker_{h,T}(q, \eventB) \geq (2 \uppi)^{-d/2} \Leb(\eventB)
 \times \inf_{\bar{q} \in \eventB} \defEns{ \balphaacc(q,\bar{q}) \rme^{-\norm{\Phiverletqi_q(\bar{q})}^2/2}\detj_{\Phiverletqi[h][T](q,\cdot)}(\bar{q})} \eqsp,
\end{equation}
with the convention $0 \times \plusinfty = 0$ and
\begin{equation}
%  \label{eq:7}
  \balphaacc(q,\bar{q}) =  \alphaacc\defEns{(q,\Phiverletqi[h][T](q,\bar{q})),\Phiverlet[h][T](q,\Phiverletqi[h][T](q,\bar{q}))}\eqsp. 
\end{equation}
Since the function $  (q,p) \mapsto (\Phiverletq[h][T](q,p),\Phiverletqi[h][T](q,p), \detj_{\Phiverletqi[h][T](q,\cdot)}(p)) $
is continuous on $\rset^d\times \rset^d$ by \Cref{lem:bound_first_iterate_leapfrog_a}, \eqref{eq:regularity_phinverse1} and \eqref{eq:regularity_phinverse2}, and for any $q,p \in \rset^d$, $\Jac_{\Phiverletq[h][T](\q,\cdot)}(\Phiverletqi[h][T](q,p))
\Jac_{\Phiverletqi[h][t](q,\cdot)}(\p) = \operatorname{I}_n$, we get that  $\Tker_{h,T}(q,\eventB) >0$ for all $q \in \rset^d$ and all compact set $\eventB$ satisfying $\Leb(\eventB) > 0$. Therefore, using that the Lebesgue measure is regular which implies that for any $\msa \in \mcb(\rset^d)$ with $\Leb(\msa) >0$, there exists a compact set $\msb \subset\msa$, $\Leb(\msb)>0$, we can conclude that $\Pkerhmc[h][T]$ is irreducible with respect to the Lebesgue measure. In addition, we get  $\Tker_{h,T}(q,\rset^d) >0$, and therefore we obtain that $\Pkerhmc[h][T]$ is aperiodic.  Similarly we get that any compact set is $(1,\Leb)$-small.

It remains to show that for any $\eventB \in\mcb(\rset^d)$, $q \mapsto \Tker_{h,T}(q,\eventB)$ is lower semi-continuous which is a straightforward consequence of Fatou's Lemma and that for any $p \in \rset^d$, $q \mapsto (\Phiverlet[h][T](q,p), \Phiverletqi[h][T](q,p),\detj_{\Phiverletqi[h][T](q,\cdot)}(p))$ is continuous.

% We now show that all the compact sets are $(1,\Leb)$-small. Let $\eventB \subset \rset^d$ be compact.  Using
% \eqref{eq:inverse_1_2} there exists $C \geq 0$ such that for all
% $q,p,v \in \rset^d$, $ \norm{p-v} \leq C \normLigne{
%   \Phiverletq[h][T](q,p)- \Phiverletq[h][T](q,v)}$. It follows
% that for all $p \in \rset^d$, $\sup_{q \in \eventB} \normLigne{
%   \Phiverletqi[h][T](q,p)} \leq C\defEnsLigne{\norm{\p} + \sup_{q \in
%     \eventB} \normLigne{\Phiverletq[h][T](q,0)}}$. Using this upper
% bound and $\Jac_{\Phiverletq(\q,\cdot)}(\Phiverletqi_q(\p))
% \Jac_{\Phiverletqi_q}(\tilde{\p}) = \operatorname{I}_n$ in
% \eqref{eq:minoration_pseudo_density_P}, where $\operatorname{I}_n$ is
% the identity matrix, we deduce that there exists $\varepsilon >0$ such
% that for all $\eventA\in \borelSet(\rset^d)$, $\eventA \subset \eventB$,
% \begin{equation}
%   \inf_{q \in \eventB} \Pkerhmc[h][T](q, \eventA)  \geq \varepsilon \Leb(\eventA) \eqsp,
% \end{equation}
% and therefore $\eventB$ is small for $\Pkerhmc[h][T]$.
% \begin{equation}
% \norm{  \Phiverletqi_q(p_1)-  \Phiverletqi_q(p_2)} \leq C \norm{p_1-p_2} \eqsp,
% \end{equation}
% This result, \eqref{eq:def_acc_ratio}, \Cref{lem:bound_first_iterate_leapfrog} and \eqref{eq:minoration_pseudo_density_P} imply that $
% \Pkerhmc[h][T]$ is irreducible with respect to the Lebesgue measure
% and aperiodic.
% and any ball on $\rset^d$ is small.

% A straightforward adaptation of the proof of \cite[Corollary
% 2]{tierney:1994} shows that $ \Pkerhmc[h][T]$ is Harris recurrent, see \Cref{propo:harris_rec} in \Cref{sec:harr-recurr-metr}. The desired conclusion then follows from \cite[Theorem 13.0.1]{meyn:tweedie:2009}.
 % \Cref{theo:irred_harris} implies
% that for all $T \geq 0$, there exists $\hirr>0$ such that for all $h \in \ocintLigne{0,\hirr}$ and all $\q \in \rset^d$
%   \begin{equation}
% \lim_{n \to \plusinfty}    \tvnorm{\delta_\q \Pkerhmc[h][T]^n - \pi} = 0 \eqsp.
%   \end{equation}


% By \cite[Theorem 17.1.4, Theorem
% 17.1.7]{meyn:tweedie:2009}, it suffices the to prove that for all
% bounded harmonic function $\harmonic : \rset^d \to \rset$ satisfying
% \begin{equation}
%   \label{eq:def_harm}
%   \Pkerhmc[h][T]\harmonic = \harmonic \eqsp,
% \end{equation}
% %$\Pkerhmc[h][T]\harmonic = \harmonic$,
% are constant. First since $\Pkerhmc[h][T]$ is irreducible with respect
% to the Lebesgue measure and aperiodic, by \cite[Theorem
% 14.0.1]{meyn:tweedie:2009} for $\Leb$-almost all $q$ we get $\lim_{n
%   \to \plusinfty} \Pkerhmc[h][T]^n \harmonic(q) = \pi(\harmonic)$ and therefore by
% \eqref{eq:def_harm} $\harmonic(q) = \pi(\harmonic)$. Therefore we get that for all $q \in \rset^d$ by \eqref{eq:def_kernel_hmc_false_density},
% \begin{multline}
%    \Pkerhmc[h][T]^n \harmonic(q) = \pi(\harmonic)  \int_{\rset^d}  \alphaacc\defEns{(q,\tilde{p}),\Phiverlet[h][T](q,\tilde{p})} \rme^{-\norm{\tilde{p}}^2/2} \rmd \tilde{p} \\
% +   \harmonic(x) \int_{\rset^d}  \parentheseDeux{1-\alphaacc\defEns{(q,\tilde{p}),\Phiverlet[h][T](q,\tilde{p})}} \rme^{-\norm{\tilde{p}}^2/2} \rmd \tilde{p} \eqsp.
% \end{multline}
% Combining this result with \eqref{eq:def_harm}, we get for all $q \in \rset^d$
% \begin{equation}
% (\harmonic(q)-\pi(\harmonic)) \int_{\rset^d} \alphaacc\defEns{(q,\tilde{p}),\Phiverlet[h][T](q,\tilde{p})} \rme^{-\norm{\tilde{p}}^2/2} \rmd \tilde{p} = 0\eqsp.
% \end{equation}
% It follows from \Cref{lem:bound_first_iterate_leapfrog} and \eqref{eq:def_acc_ratio} that for all $q \in \rset^d$, $\harmonic(q) = \pi(\harmonic)$
% which concludes the proof.
% =======
% The proof of \ref{theo:irred_harris_b} using a change of variable with $\Phiverletqi[h][T](q,\cdot)$.

% We now show that $\Tker_{h,T}$ satisfies the condition which implies that $\Pkerhmc[h][T]$ is a \Tkernel.
% First, for all $\eventB \in \borelSet(\rset^d)$,
% \begin{equation}
% \label{eq:minoration_pseudo_density_P}
% \Tker_{h,T}(q, \eventB) \geq (2 \uppi)^{-d/2} \Leb(\eventB)
%  \times \inf_{\bar{q} \in \eventB} \defEns{\alphaacc\defEns{(q,\Phiverletqi[h][T](q,\bar{q})),\Phiverlet[h][T](q,\Phiverletqi[h][T](q,\bar{q}))} \rme^{-\norm{\Phiverletqi[h][T](q,\bar{q})}^2/2} \detj_{\Phiverletqi[h][T]}(q,\bar{q})} \eqsp,
% \end{equation}
% with the convention $0 \times \plusinfty = 0$. Since $\Phiverletqi[h][T](q,\cdot)$
% is a diffeomorphism on $\rset^d$, we get that  $
% \Tker_{h,T}(q,\eventB) >0$ for all $q \in \rset^d$ and all compact set $\eventB$ satisfying $\Leb(\eventB) > 0$. Since the Lebesgue measure is regular, this implies that $\Pkerhmc[h][T]$ is irreducible with respect to the Lebesgue measure and aperiodic.

% By Fatou's Lemma, for any $\eventB \in\mcb(\rset^d)$, $q \mapsto \Tker_{h,T}(q,\eventB)$ is lower semi-continuous.
% We now show that all the compact sets are small. Let $\eventB \subset \rset^d$ be compact.  Using
% \eqref{eq:inverse_1_2} there exists $C \geq 0$ such that for all
% $q,p,v \in \rset^d$, $ \norm{p-v} \leq C \normLigne{
%   \Phiverletq[h][T](q,p)- \Phiverletq[h][T](q,v)}$. It follows
% that for all $p \in \rset^d$, $\sup_{q \in \eventB} \normLigne{
%   \Phiverletqi[h][T](q,\p)} \leq C\defEnsLigne{\norm{\p} + \sup_{q \in
%     \eventB} \normLigne{\Phiverletq[h][T](q,0)}}$. Using this upper
% bound and $\Jac_{\Phiverletq(\q,\cdot)}(\Phiverletqi[h][T](q,\p))
% \Jac_{\Phiverletqi[h][T]}(q,\tilde{\p}) = \operatorname{I}_n$ in
% \eqref{eq:minoration_pseudo_density_P}, where $\operatorname{I}_n$ is
% the identity matrix, we deduce that there exists $\varepsilon >0$ such
% that for all $\eventA\in \borelSet(\rset^d)$, $\eventA \subset \eventB$,
% \begin{equation}
%   \inf_{q \in \eventB} \Pkerhmc[h][T](q, \eventA)  \geq \varepsilon \Leb(\eventA) \eqsp,
% \end{equation}
% and therefore $\eventB$ is small for $\Pkerhmc[h][T]$.
% >>>>>>> f8207bad5c0353bdfe37210ffc64a715e92e53ed

Finally, the last statements of \ref{theo:irred_harris_c} follows from \Cref{propo:harris_rec} in \Cref{sec:harr-recurr-metr} which implies that  $ \Pkerhmc[h][T]$ is Harris recurrent and  \cite[Theorem 13.0.1]{meyn:tweedie:2009} which implies  \eqref{eq:harris-theorem}.

\subsubsection{Proof of \Cref{theo:irred_D}}
\label{sec:proof-crefth_irred_D}
We use \Cref{coro:irred}. Indeed $\Pkerhmc[h][T]$ is
of form \eqref{eq:def_pkerb} and it is straightforward to check that it
satisfies \Cref{assumG:phi} (note that \Cref{lem:bound_first_iterate_leapfrog_a}
shows that $\Phiverlet[h][T]$ is a Lipshitz function on $\rset^{2d}$).

We now check that $\Pkerhmc[h][T]$ satisfies \Cref{assumG:irred_b}($\rassG,0,\MassG$) for all $\rassG,\MassG \in
\rset_+^*$ using \Cref{le:degree_application}.  By \eqref{eq:qk}, for all $T \in \nsets$, $h >0$, $q,p \in \rset^d$,
\begin{equation}
  \label{eq:phiverlet_gqth}
  \Phiverletq[h][T](q,p) = T
h p + g_{q,T,h}(p)
\end{equation}
where $g_{q,T,h}(p) = q - (Th^2/2) \nabla \F(q) -
h^2 \gperthmc[T](q,p)$ where $\gperthmc[T]$ is defined by \eqref{eq:def_gperthmc}. \Cref{lem:inverse_1} shows that for any $T \in \nsets$ and $h >0$, it holds that
\begin{equation}
    \label{eq:2:theo:irred_D}
\sup_{p,v,q \in  \rset^d} \frac{\norm{g_{q,T,h}(p)-g_{q,T,h}(v)}}{\norm{p - v}} \leq T h [\{1 + h \constzero^{1/2} \vartheta_1(h \constzero^{1/2} )\}^T-1] \eqsp,
\end{equation}
which implies that the condition
\Cref{le:degree_application}-\ref{propo:irred_b_item_i} is satisfied. To check that
condition  \Cref{le:degree_application}-\ref{propo:irred_b_item_ii} holds, we consider separately the two cases: $\beta <1$ and $\beta =1$.

\begin{enumerate}[label=$\bullet$, wide, labelwidth=!, labelindent=0pt]
\item Consider first the case $\beta <1$. By \Cref{assum:regOne}-\ref{assum:regOne_b},
for any $T \in \nsets$ and $h >0$, we get
\begin{equation}
\norm{\gperthmc[T](\q,\p)} \leq  T \sum_{i=1}^{T-1} \norm{\nabla \F \circ \Phiverletq[h][i](\q,\p)} \leq
\constzeroT T \sum_{i=1}^{T-1} \defEns{ 1 + \norm{\Phiverletq[h][i](\q,\p)}^{\expozero}}
 \eqsp.
\end{equation}
Hence, by \Cref{lem:bound_first_iterate_leapfrog_b}-\ref{lem:bound_first_iterate_leapfrog_1}
there exists $C \geq 0$ such that for all $R\in \rset_+^*$ and
$q,p \in \rset^d$, $\norm{q} \leq R$,
\begin{equation}
\label{eq:3:theo:irred_D}
\norm{g_{q,T,h}(p)} \leq C \defEns{1+R^{\beta} +\norm{p}^{\expozero}} \eqsp,
\end{equation}
which implies that condition \ref{propo:irred_b_item_ii} of \Cref{le:degree_application} holds for any $T \in \nsets$ and $h >0$.

\item Consider now the case $\beta =1$.  For any $T \in \nsets$, $h >0$,  $q,p \in \rset^d$ we get using \Cref{assum:regOne}-\ref{assum:regOne_a}
\begin{align}
  \norm{g_{q,T,h}(p)} &\leq \norm{q} + Th^2 \constzero  \norm{q} /2 + Th^2 \norm{\nabla U(0)} /2\\
  & \qquad \qquad +h^2 \norm{\gperthmc[T](q,p) - \gperthmc[T](q,0)} + h^2 \norm{ \gperthmc[T](q,0)} \eqsp.
\end{align}
Therefore using \Cref{lem:inverse_1}, for any $q,p \in \rset^d$, $\norm{q} \leq R$ for $R \geq 0$, for any $T \in \nsets$ and $h >0$ satisfying \eqref{eq:condition-h,T-harris}, there exists $C \geq 0$ such that
\begin{equation}
  \norm{g_{q,T,h}(p)} \leq C + h T  [ \{1+ h \constzero^{1/2} \vartheta_1(h\constzero^{1/2})\}^T-1]  \norm{p} \eqsp,
\end{equation}
showing that condition \ref{propo:irred_b_item_ii} of \Cref{le:degree_application} is satisfied.
\end{enumerate}

Therefore,  \Cref{le:degree_application} can be applied and for any $T \in \nsets$ and $h >0$ if $\beta <1$ and for any $h > 0$ and $T \in \nsets$ satisfying \eqref{eq:condition-h,T-harris} if $\beta =1$, $\Pkerhmc[h][T]$ satisfies \Cref{assumG:irred_b}($\rassG,0,\MassG$) for all $\rassG,\MassG \in
\rset_+^*$.  \Cref{coro:irred} concludes the proof of \ref{theo:irred_D_a} and \ref{theo:irred_D_b}.
The last statement then follows from   \cite[Theorem 14.0.1]{meyn:tweedie:2009}.

% Using this result and \Cref{theo:irred}, we get that for all $\rassG,\MassG  \in \rset_+^*$ there exists $\varepsilon >0$ such that
% for all $\q \in \ball{0}{\rassG}$ and $\eventA \in \borelSet(\rset^d)$,
% \begin{equation}
%   \Pkerhmc[h][T](q, \eventA) \geq \varepsilon \Leb(\eventA \cap \ball{0}{M}) \eqsp.
% \end{equation}
% \Cref{coro:irred} Combining this result and \eqref{eq:1:theo:irred_D} concludes the proof of \ref{theo:irred_D_a} and \ref{theo:irred_D_b}.

% The proof is a consequence of \Cref{lem:bound_first_iterate_leapfrog},
% \Cref{le:degree_application} and \Cref{theo:irred}.  \alain{give some
%   details}

%%% Local Variables:
%%% mode: latex
%%% TeX-master: "main"
%%% End:


\bibliographystyle{alpha}
\bibliography{lit}
\end{document} 