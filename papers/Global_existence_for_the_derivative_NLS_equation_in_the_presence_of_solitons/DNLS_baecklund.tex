% --------------------------------------------------------------
% Article Class (This is a LaTeX2e document)  ********************
% --------------------------------------------------------------
\documentclass[11pt,a4paper]{article}
\usepackage{etex}
\usepackage[english]{babel}
\usepackage{amsmath,amsthm}
\usepackage{amsfonts}
\usepackage{amssymb}
\usepackage{accents}
\usepackage{pstricks-add}
\usepackage{graphicx}
\usepackage{color}
\usepackage{framed}
\usepackage{mathptmx}
\usepackage{enumerate}
\usepackage[arrow, matrix, curve]{xy}
\usepackage{pstricks}
\usepackage{pstricks, pstricks-add, mathptmx, amsfonts, hyperref, color}
\usepackage[left=3cm, right=3cm, top=3cm, bottom=3cm]{geometry}
\usepackage{calrsfs}
\usepackage{multicol}
%\usepackage{mathtools}
%\mathtoolsset{showonlyrefs}
\DeclareMathAlphabet{\pazocal}{OMS}{zplm}{m}{n}
% THEOREMS -------------------------------------------------------
\newtheorem{thm}{Theorem}[section]
\newtheorem{cor}[thm]{Corollary}
\newtheorem{lem}[thm]{Lemma}
\newtheorem{prop}[thm]{Proposition}
\theoremstyle{definition}
\newtheorem{defn}[thm]{Definition}
\newtheorem{rhp}[thm]{Riemann--Hilbert problem}
\theoremstyle{remark}
\newtheorem{rem}[thm]{Remark}
\numberwithin{equation}{section}

\newcommand\pmtwo[4]{\left( \begin{array}{cc}#1&#2\\#3&#4\end{array} \right)}
\newcommand{\norm}[1]{\left\Vert#1\right\Vert}
\newcommand{\abs}[1]{\left\vert#1\right\vert}
\newcommand{\set}[1]{\left\{#1\right\}}
\newcommand{\Real}[0]{\mathbb R}
\newcommand{\Compl}[0]{\mathbb C}
\newcommand{\eps}[0]{\varepsilon}
\newcommand{\ran}{\text{ran }}
\newcommand{\gr}{\text{grad }}
\newcommand{\re}{\,\mathfrak{Re}\,}
\newcommand{\im}{\,\mathfrak{Im}\,}
\newcommand{\ord}{\,\text{ord}\,}
\newcommand{\supp}{\,\text{supp}\,}
\newcommand{\meas}{\,\text{meas}\,}
\newcommand{\tr}{\,\text{tr}\,}
\newcommand{\Div}{\,\text{div}\,}
\newcommand{\rank}{\,\text{rank}\,}
\newcommand{\id}{\,\text{id}\,}
\newcommand{\adj}{\,\text{adj}\,}
\newcommand{\pr}{\,\text{pr}\,}
\newcommand{\Gl}{\,\text{Gl}\,}
\newcommand{\Hess}{\,\text{Hess}}
\newcommand{\ii}{\mathrm{i}}
\newcommand{\sech}{\mathrm{sech}}
\newcommand{\Res}{\mathrm{Res}}
\newcommand{\ind}{\mathrm{ind}\,}
\newcommand{\Int}[1]{\accentset{\circ}{#1}}
\newcommand{\bigslant} [2]{{\raisebox{.2em}{$#1$} \left/\raisebox{-.2em}{$#2$}\right.}}
\newcommand{\dbar}{\overline{\partial}}
\newcommand{\F}{\mathfrak{F}}
\newcommand{\co}{\text{const. }}
\def\res{\mathop{Res}}
\newcommand\sol[2]{\text{sol}^{#1}_{#2}(x,t)}
\newcommand{\Pp}{\pazocal{P}^+}
\newcommand{\Pm}{\pazocal{P}^-}
\newcommand{\Ppm}{\pazocal{P}^{\pm}}
\newcommand{\rh}{Riemann--Hilbert problem }
\renewcommand\bf\bfseries
\begin{document}

\title{Global existence for the derivative NLS equation in the presence of solitons}
%\address{Mathematisches Institut\\
%  Universit\"{a}t zu K\"{o}ln\\
% 50931 K\"{o}ln, Germany}
%\email[A.~Saalmann]{asaalman@math.uni-koeln.de}
\author{Aaron Saalmann\thanks{A.S. gratefully acknowledges financial support from the projects ``Quantum Matter and Materials"
and SFB-TRR 191 ``Symplectic Structures in Geometry, Algebra and Dynamics" (Cologne University, Germany).}
\footnote{Mathematisches Institut, Universit\"{a}t zu K\"{o}ln, 50931 K\"{o}ln, Germany, e-mail: asaalman@math.uni-koeln.de}
}

\date{\today}
\maketitle
\begin{abstract}
\label{sec:abstract}

%% 1. what is the problem 
Scientific applications that run on leadership computing facilities often face the challenge 
of being unable to fit leading science cases onto accelerator devices due to memory constraints 
(memory-bound applications).
%
% 2. what is your solution 
In this work, the authors studied one such US Department of Energy mission-critical condensed matter 
physics application, Dynamical Cluster Approximation (DCA++), and this paper discusses how device memory-bound challenges were successfully reduced  by proposing an effective 
``all-to-all'' communication method---a ring communication algorithm. 
%
This implementation takes advantage of acceleration on GPUs and remote direct memory access (RDMA) for fast data exchange between GPUs. 
%
\\Additionally, the ring algorithm was optimized with sub-ring communicators
and multi-threaded support to further reduce communication overhead and 
expose more concurrency, respectively.
%
% 3. What's the cherry-picked evaluation result you want to mention
The computation and communication were also analyzed 
by using the Autonomic Performance Environment for Exascale 
(APEX) profiling tool,  and this paper further discusses the 
performance trade-off for the ring algorithm implementation. 
%
The memory analysis on the ring algorithm shows that the allocation size for the authors' most 
memory-intensive data structure per GPU is now reduced to $1/p$ of the original size, where $p$ is the number of GPUs in the ring communicator.
%
The communication analysis suggests that 
the distributed Quantum Monte Carlo execution time grows linearly as sub-ring size increases, and the cost of messages passing through the network interface connector could be a limiting factor.


%
% \todoRed{Ronnie: Next sentence needs rewrite, too much information about Green's function that no one knows in the abstract; recommend generalizing.} \emph {However, DCA++ is currently facing memory-bound challenge as 
% a larger device array $G_t$ is limited by device memory size, where
% $G_t$ is a two-particle Green's function that allows condensed matter
% scientists to explore larger and more complex (higher fidelity)
% physics cases.}

\end{abstract}

\keywords{DCA++, Quantum Monte Carlo, GPU Remote Direct Memory Access, memory-bound issue, exascale machines}

\tableofcontents
\newpage
Reinforcement learning has achieved great success in areas such as Game-playing \citep{silver2018general,vinyals2019grandmaster}, robotics \cite{kober2013reinforcement}, large language models \citep{ouyang2022training}, etc.
However, due to safety concerns or physical limitations, in some real-world reinforcement learning problems, we must consider additional constraints that may influence the optimal policy and the learning process \citep{garcia2015comprehensive}.
% For example, a robotic arm must not take actions that may cause harm to itself or the environments.
A standard framework to handle such cases is the constrained Markov Decision Process (CMDP) \citep{altman1999constrained}.
Within the CMDP framework, the agent has to maximize
the expected cumulative reward while
obeying a finite number of constraints, which are usually in the form of expected cumulative cost criteria.

However, we are sometimes concerned with the problem with a continuum of constraints.
For example,
the constraints we meet might be time-evolving or subject to uncertain parameters, which
cannot be formulated as an ordinary CMDP
(see Examples \ref{Example_Time_Evolving} and  \ref{Example_Uncertain}).
In this paper we would study a generalized CMDP  
to address the above problem.  Because the constraints are not only infinite-number but also lie
in a continuous set,
the generalization is not trivial. Fortunately, we find that we can borrow the idea behind semi-infinite programming (SIP) \citep{remez1934determination, hettich1993semi} to deal with the semi-infinite constraints.
Accordingly, we propose \emph{semi-infinitely constrained Markov decision processes} (SICMDPs)
as a novel complement to the ordinary CMDP framework.
%More specifically,  an SICMDP model %, we consider 
%contains a continuum of constraints whereas an ordinary CMDP contains a finite number of constraints. 

%This generalization is natural but not trivial. However, we can brows the idea  
%The idea is quite natural and can be backtracked
%to the practice of extending linear programming to linear semi-infinite programming (LSIP) %\cite{remez1934determination, GobernaLSIO1998}.
%In addition, 
%As a complementary approach to the ordinary CMDP framework, 
%SICMDP can be used to model these problems  which cannot be described by a finite number of constraints
%that are not covered by .
%For example,
%the restrictions we consider can be time-evolving or subject to uncertain parameters
%, thus
%cannot be described by a finite number of constraints but a continuum of constraints 
%(see Examples \ref{Example_Time_Evolving} and  \ref{Example_Uncertain}).

We also present two reinforcement learning algorithms to solve SICMDPs called SI-CRL and SI-CPO, respectively.
SI-CRL is a model-based reinforcement learning algorithm designed for tabular cases, and SI-CPO is a policy optimization algorithm for non-tabular cases.
% and analyze its performance both theoretically and empirically.
The main challenge is that we need to deal with a continuum of constraints, thus reinforcement learning algorithms for ordinary CMDPs do not work anymore.
In SI-CRL, we tackle this difficulty by first transforming the reinforcement learning problem to an equivalent LSIP problem, which can then be solved using methods in the LSIP literature like the dual exchange methods \citep{Hu1990,reemtsen1998numerical}.
In SI-CPO, we resort to the idea of cooperative stochastic approximation developed in \cite{lan2020algorithms, wei2020comirror}.
As far as we know, we are the first to introduce tools from semi-infinitely programming (SIP) into the reinforcement learning community for solving constrained reinforcement learning problems.

% To the best of our knowledge, we are the first to apply tools from semi-infinitely programming (SIP) to solve reinforcement learning problems.
Furthermore, we give theoretical analysis for both SI-CRL and SI-CPO.
We decompose the error of SI-CRL into two parts: the statistical error from approximating the true SICMDP with an offline dataset and the optimization error due to the fact that the solution of the LSIP problem obtained by the dual exchange method is inexact.
On the optimization side, we show that the iteration complexity of SI-CRL is $O\left(\left\{\mathrm{diam}(Y)L\sqrt{|\gS|^2|\gA|m}/\left[(1-\gamma)\epsilon\right]\right\}^m\right)$.
On the statistical side, we show that the sample complexity of SI-CRL is $\widetilde O\left(\frac{|S|^2|A|^2}{\epsilon^2(1-\gamma)^3}\right)$ if the offline dataset is generated by a generative model, and $\widetilde O\left(\frac{|S||A|}{\nu_{\min} \epsilon^2(1-\gamma)^3}\right)$ if the dataset is generated by a probability measure $\nu$ as considered in \cite{chen2019information}.
Here $\widetilde O$ means that all logarithm terms are discarded.
For SI-CPO, things become a little more complicated because other than the statistical error and the optimization error, we also need to consider the function approximation error, which comes from imperfect policy parametrizations.
It is shown if the function approximation error can be controlled to $O(\epsilon)$ order, the iteration complexity of SI-CPO is $\widetilde{O}\left(\frac{1}{\epsilon^2(1-\gamma)^6}\right)$ and the sample complexity of SI-CPO is $\widetilde{O}(\frac{1}{\epsilon^4(1-\gamma)^{10}})$.
Here our iteration complexity bound is equivalent to a typical $\widetilde O(1/\sqrt{T})$ global convergence rate.

We perform a set of numerical experiments to illustrate the SICMDP model and validate our proposed algorithms.
Specifically, we examine two numerical examples, namely the discharge of sewage and ship route planning.
Through the discharge of sewage example, we show the advantage of the SICMDP framework over the CMDP baseline obtained by naive discretization in modeling realistic sequential decision-making problems.
Moreover, we demonstrate the effectiveness of the SI-CRL and SI-CPO algorithms in such tabular environments. 
In the ship route planning example, we illustrate the benefits of the SICMDP framework and the ability of the SI-CPO algorithm to address complex continuous control tasks involving continuous state spaces with modern deep reinforcement learning techniques.

% In summary, our contributions are listed as follows.
% First, we present the SICMDP model, which can be viewed as a generalization of the ordinary CMDP model.
% Second, we propose an algorithm to perform reinforcement learning for SICMDPs, which is called SI-CRL, and we believe that we are the first to apply tools from SIP
% to solve reinforcement learning problems.
% Third, we give a theoretical analysis of SI-CRL and identify both its sample complexity and iteration complexity.
% In addition, we perform numerical experiments to illustrate the SICMDP model and validate the SI-CRL algorithm.
% \{This paragraph can be removed!!! \}





General results from scattering theory show that the operator $\hat\Omega^\pm(\nu)$ in \eqref{Omega_matrix} 
is related to the scattering matrix (see e.g. \cite[ Thm. 4]{BirmanEntina1967}).
Here, we give an elementary derivation.
To this end, we provide some scattering theoretic background for the one-dimensional
Schr\"odinger equation
\begin{equation}\label{st01}
  -\psi''(x) + V(x)\psi(x) = k^2 \psi(x),\ \im(k) \geq 0,
\end{equation}
both on the line and on the half-line.
We write $k^2$ instead of $z$ to stay closer to the usual notation in scattering theory.
For scattering on the line see, e.g., \cite[Sec. 2, \S 3, in particular pp. 145, 146]{DeiftTrubowitz1979}
and on the half-line see \cite[Ch. 4]{Yafaev2010}.

\section{Solitons}\label{s solitons}
This section is devoted to the exact solitary wave solutions of the DNLS equation (\ref{e dnls}) which are known since the 1970s (see, e.g., \cite{mjlhus1976} and \cite{KaupNewell1978}). Also more recent works are concerned with solitons. See for instance \cite{Colin2006}, where orbital stability of solitons is shown. The inverse scattering machinery admits a simple definition of $N$-solitons:
\begin{defn}
    (Global) solutions $u^{(N\text{-sol})}(x,t)$ of (\ref{e dnls}) such that the initial datum $u^{(N\text{-sol})}(\cdot,0)$ produces scattering data
    \begin{equation*}
        \mathcal{S} (u^{(N\text{-sol})})=\set{r_+\equiv r_-\equiv 0;\lambda_1,... ,\lambda_N;c_1,...,c_N},
    \end{equation*}
    are called $N$-\emph{solitons}. For $N=1$ we just say \emph{soliton}.
\end{defn}
In the case of $r_+\equiv r_-\equiv 0,$ the Riemann--Hilbert problem \rh \ref{rhp m dynamic} reads as follows:
\begin{samepage}
\begin{framed}
\begin{rhp}\label{rhp N sol}
Find for each $x\in\Real$ a $2\times 2$-matrix valued function $\Compl\ni z\mapsto m^{(N\text{-sol})} (z;x,t)$ which satisfies
\begin{enumerate}[(i)]
  \item $m^{(N\text{-sol})} (z;x,t)$ is meromorphic in $\Compl$ (with respect to the parameter $z$).
  \item $m^{(N\text{-sol})} (z;x,t)=1+\mathcal{O}\left(\frac{1}{z}\right)$ as $|z|\to\infty$.
  \item $m^{(N\text{-sol})} $ has simple poles at $z_1,...,z_N,\overline{z}_1,...,\overline{z}_N$ with
      \begin{equation*}%\label{e Res}
        \begin{aligned}
          \res_{z=z_k}m^{(N\text{-sol})} (z;x,t)&=\lim_{z\to z_k}m^{(N\text{-sol})} (z;x,t)
          \left[
            \begin{array}{cc}
              0 & 0 \\
              2\ii\lambda_kc_k e^{2\ii xz_k+4\ii t z_k^2} & 0
            \end{array}
          \right],\\
          \res_{z=\overline{z}_k}m^{(N\text{-sol})} (z;x,t)&=\lim_{z\to \overline{z}_k}m^{(N\text{-sol})} (z;x,t)
          \left[
            \begin{array}{cc}
              0 & \frac{-\overline{c}_k}{2\ii\lambda_k} e^{-2\ii x\overline{z}_k-4\ii t \overline{z}_k^2} \\
              0 & 0
            \end{array}
          \right].
        \end{aligned}
      \end{equation*}
\end{enumerate}
\end{rhp}
\end{framed}
\end{samepage}
Using the ansatz
\begin{equation*}
    m^{(N\text{-sol})} (z;x,t)=1+\sum_{k=1}^N\left\{ \frac{A_k(x,t)}{z-z_k}+ \frac{B_k(x,t)}{z-\overline{z}_k}\right\}
\end{equation*}
we can transfer \rh \ref{rhp N sol} into a purely algebraic system which can be solved explicitly. Then, the reconstruction formulas (\ref{e rec 1}) and (\ref{e rec 2}) yield explicit solutions of the DNLS equation, which are (multi) solitons. For the special case $N=1$ we find
\begin{equation}\label{e soliton}
    u_{\omega,v,x_0,\gamma}(x,t)=\phi_{\omega,v}(x-vt-x_0) e^{-\ii\gamma+\ii\omega t+\ii\frac{v}{2}(x-vt)- \frac{3}{4}\ii\int_{\infty}^{x-vt-x_0} |\phi_{\omega,v}(y)|^2dy},
\end{equation}
where
\begin{equation}\label{e sol ampl}
    \phi_{\omega,v}(x)=\left[\frac{\sqrt{\omega}}{4\omega-v^2} \left\{\cosh(\sqrt{4\omega-v^2}x)- \frac{v}{2\sqrt{\omega}}\right\}\right]^{-1/2}.
\end{equation}
The parameters $(\omega,v)\in \Real^2$ describe the speed and the width of the soliton and  are connected to the pole $z_1$ by
\begin{equation}\label{e omega v}
        \omega=4|z_1|^2,\qquad
         v=-4\re(z_1).
\end{equation}
Note that $v^2<4\omega$ is automatically fulfilled if $z_1\in\Compl_+$. The norming constant $c_1$ influences only the phase and the spatial position of the soliton. To be precise we have
\begin{equation}\label{e x_0 gamma}
    x_0=2\ln\left[\frac{|c_1|}{2\im(z_1)}\right] \left(\sqrt{4\omega-v^2}\right)^{-1},
    \quad
    \gamma=\arg(c_1)+\frac{\pi}{2}+\frac{1}{2}\arg(z_1).
\end{equation}
Expressions for $N$-solitons with $N\geq2$ are large and not presented here. If $\re(z_j)
\neq\re(z_k)$ for $j\neq k$, then for large $|t|$, $N$-solitons break up into $N$  individual solitons of the form (\ref{e soliton}):
\begin{equation}\label{e sol sep}
    u^{(N\text{-sol})}(x,t)\sim \sum_{k=1}^N u_{\omega_k,v_k,x_{0,k}^{\pm},\gamma_k^{\pm}}(x,t), \quad\text{as }t\to\pm\infty.
\end{equation}
If the real parts of two poles $z_j$ and $z_k$ coincide, we obtain a solution having two peaks traveling at the same speed and the separation (\ref{e sol sep}) will not occur. Instead, \emph{breather} phenomena will appear. 
\section{Inverse scattering without poles}\label{s inverse sc}
In this section we are dealing with \rh \ref{rhp m} in the case where $N=0$. Hence, $m$ has no pole in $\Compl\setminus\Real$ and is analytic in $\Compl\setminus\Real$. We recall the associated Riemann--Hilbert problem:\\
\begin{framed}
    \begin{rhp}\label{rhp m^0}
        Find for each $x\in\Real$ a $2\times 2$-matrix valued function $\Compl\ni z\mapsto m(z;x)$ which satisfies
        \begin{enumerate}[(i)]
          \item $m(z;x)$ is meromorphic in $\Compl\setminus\Real$ (with respect to the parameter $z$).
          \item $m(z;x)=1+\mathcal{O}\left(\frac{1}{z}\right)$ as $|z|\to\infty$.
          \item The non-tangential boundary values $m_{\pm}(z;x)$ exist for $z\in\Real$ and satisfy the jump relation
              \begin{equation}\label{e jump m^0}
                  m_+=m_-(1+R),\quad\text{where}\quad
                  R(z;x):=
                  \left(
                    \begin{array}{cc}
                       \overline{r}_+(z)r_-(z) & e^{-2\ii zx}\overline{r}_+(z) \\
                       e^{2\ii zx}r_-(z) & 0 \\
                    \end{array}
                  \right).
              \end{equation}
        \end{enumerate}
    \end{rhp}
\end{framed}
For any function $h\in L^p(\Real)$ with $1\leq p<\infty$, the Cauchy operator denoted by $\pazocal{C}$ is given by
\begin{equation*}
    \pazocal{C}(h)(z):=\frac{1}{2\pi\ii}\int_{\Real} \frac{h(s)}{s-z}ds,\quad z\in\Compl\setminus\Real.
\end{equation*}
When $z$ approaches to a point on the real line transversely from the upper and lower half planes, the Cauchy operator becomes the following projection operators:
\begin{equation*}
    \Ppm(h)(z):=\lim_{\eps\downarrow 0}\frac{1}{2\pi\ii}\int_{\Real} \frac{h(s)}{s-(z\pm\eps)}ds,\quad z\in\Real.
\end{equation*}
The following proposition summarizes all properties which are needed to establish the solvability of \rh \ref{rhp m^0} and furthermore to prove estimates on the solution.
\begin{prop}\label{p cauchy operator}
    \begin{enumerate}[(i)]
      \item For every $h\in L^p(\Real)$, $1\leq p<\infty$, the Cauchy operator $\pazocal{C}(h)$ is analytic off the real line.
      \item For $h\in L^1(\Real)$, $\pazocal{C}(h)(z)$ decays to zero as $|z|\to\infty$ and admits the asymptotic
          \begin{equation}\label{e lim z Ch(z)}
            \lim_{|z|\to\infty}z\pazocal{C}(h)(z)= -\frac{1}{2\pi\ii}\int_\Real h(s) ds,
          \end{equation}
          where the limit is taken either in $\Compl^+$ or $\Compl^-$.
      \item The projection operators $\Ppm$ are linear bounded operators $L^p(\Real)\to L^p(\Real)$ for each $p\in(1,\infty)$. For $p=2$ we have $\|\Ppm\|_{L^2\to L^2}=1$.
      \item For every $x_0\in\Real_+$ and every $r\in H^1(\Real)$, we have
          \begin{equation}\label{e sup P^pm r 1}
              \sup_{x\in(x_0,\infty)} \|\langle x\rangle\Ppm (r(z)e^{\mp2\ii zx})\|_{L^{2}_z(\Real)}\leq \|r\|_{H^1},
          \end{equation}
          where $\langle x\rangle:=\sqrt{1+|x|^2}$. In addition,
          \begin{equation}\label{e sup P^pm r 2}
              \sup_{x\in\Real} \|\Ppm (r(z)e^{\mp2\ii zx})\|_{L^{\infty}_z(\Real)}\leq \frac{1}{\sqrt{2}}\|r\|_{H^1}.
          \end{equation}
          Furthermore, if $r\in L^{2,1}(\Real)$, then
          \begin{equation}\label{e sup P^pm r 3}
              \sup_{x\in\Real}\|\Ppm (zr(z)e^{\mp2\ii zx})\|_{L^{\infty}_z(\Real)}\leq \frac{1}{\sqrt{2}}\|zr\|_{L^{2,1}}.
          \end{equation}
      \item (Sokhotski-Plemelj theorem) The following two identities hold:
      \begin{equation}\label{e Sokhotski-Plemelj}
        \begin{aligned}
            &\Pp-\Pm=\text{Id}_{L^p(\Real)},\\
            &\Pp+\Pm=-\ii\pazocal{H},
        \end{aligned}
      \end{equation}
      where $\pazocal{H}:L^p(\Real)\to L^p(\Real)$ is the Hilbert transform given by
      \begin{equation*}
        \pazocal{H}(h)(z):=\lim_{\eps\downarrow 0}\frac{1}{\pi}\left(\int_{-\infty}^{z-\eps}+ \int^{\infty}_{z+\eps}\right) \frac{h(s)}{s-z}ds,\quad z\in\Real.
      \end{equation*}
      \item Let $f_+$ and $f_-$ functions defined in the upper (lower) $\Compl$-plane. If $f_{\pm}$ is analytic in $\Compl^{\pm}$ and $f_{\pm}(z)\to 0$ as $|z|\to\infty$ for $\im(z)\gtrless0$, then
          \begin{equation}\label{e Ppm of analytic functions}
            \Ppm(f_{\mp})(z)=0,\qquad\Ppm(f_{\pm})(z)=\pm f_{\pm}(z),\quad z\in\Real.
          \end{equation}
    \end{enumerate}
\end{prop}
The Cauchy operator is useful to convert \rh \ref{rhp m^0} into an integral equation. Indeed, the jump condition (\ref{e jump m^0}) can be written as
\begin{equation*}
    (m_+(z;x)-1)-(m_-(z;x)-1)=m_-(z;x)R(z;x).
\end{equation*}
Applying $\Pp$ and $\Pm$ to this equation yields by (\ref{e Ppm of analytic functions}) the following integral equation
\begin{equation}\label{e integral equation for m+-}
    m_{\pm}(z;x)=1+\Ppm(m_-(\cdot;x)R(\cdot;x))(z),\quad z\in\Real,
\end{equation}
which represents the solution of \rh \ref{rhp m^0} on the real line. The following Lemma ensures the solvability of \rh \ref{rhp m} (see Corollary 6 and Lemma 9 in \cite{Pelinovsky2016}):
\begin{lem}\label{l solvability of RHP N=0}
    Let $r_{\pm}\in H^1(\Real)\cap L^{2,1}(\Real)$ such that the relation (\ref{e relation r+ r-}) and the constraint (\ref{e r constraint}) hold. Then there exists an unique solution $m_{\pm}$ of the system of integral equations (\ref{e integral equation for m+-}). Moreover there exists a positive constant $C$ that depends on $\|r_{\pm}\|_{L^{\infty}}$ only such that $m_{\pm}$ enjoys the estimate
    \begin{equation*}
        \|m_{\pm}(\cdot;x)-1\|_{L^2}\leq C(\|r_{+}\|_{L^2} +\|r_{-}\|_{L^2})
    \end{equation*}
    for every $x\in\Real$.
\end{lem}
This Lemma yields indeed a solution of \rh \ref{rhp m^0}, since the analytic continuation of $m_{\pm}$ is found by Proposition \ref{p cauchy operator} (ii):
\begin{equation}\label{e RHP solution formula}
    m(z;x)=1+\frac{1}{2\pi\ii}\int_{\Real} \frac{m_-(y;x)R(y;x)}{y-z}dy,\quad z\in\Compl\setminus \Real.
\end{equation}
Alternatively we can factorize $1+R=(1+R_+)(1+R_-)$ with
\begin{equation}\label{e def R+ and R-}
    R_+(z;x)=
    \left(
      \begin{array}{cc}
            0 & e^{\overline{\phi}(z)}\overline{r}_+(z) \\
            0 & 0 \\
      \end{array}
    \right),\qquad
    R_-(z;x)=
    \left(
      \begin{array}{cc}
            0 & 0 \\
            e^{\phi(z)}r_-(z) & 0 \\
      \end{array}
    \right).
\end{equation}
The jump relation (\ref{e jump}) then becomes $m_+ -m_-=m_-R_+ +m_+R_-$ and applying again $\Ppm$ to this equation yields us
\begin{equation}\label{e alternative RHP solution formula}
    m(z;x)=1+\frac{1}{2\pi\ii}\int_{\Real} \frac{m_-(y;x)R_+(y;x)+ m_+(y;x)R_-(y;x)}{y-z}dy.
\end{equation}
In component form, for the non-tangential limits $z\to\Real$, we find
\begin{equation}\label{e component RHP solution formula}
    m_{\pm}(z;x)=1+
    \left[
      \begin{array}{cc}
        \Ppm\left([m_+(z;x)]_{12} r_-(z) e^{2\ii zx}\right)(z) & \Ppm\left([m_-(z;x)]_{11} \overline{r}_+(z) e^{-2\ii zx}\right)(z) \\
        \Ppm\left([m_+(z;x)]_{22} r_-(z) e^{2\ii zx}\right)(z) & \Ppm\left([m_-(z;x)]_{21} \overline{r}_+(z) e^{-2\ii zx}\right)(z) \\
      \end{array}
    \right].
\end{equation}
In the further analysis of \rh \ref{rhp m} we will meet expressions of the form
\begin{equation}\label{e def I_1,2}
    \begin{aligned}
        &I_1(r)(x):=\frac{1}{2\pi\ii} \int_{\Real}[m_-(y;x)-1]_{11} r(y) e^{-2\ii yx}dy,\\
        &I_2(r)(x):=\frac{1}{2\pi\ii} \int_{\Real}[m_+(y;x)-1]_{22}r(y) e^{2\ii yx}dy,
    \end{aligned}
\end{equation}
where $m_{\pm}$ are the unique solutions of the system of integral equations (\ref{e integral equation for m+-}) and $r$ is some given function.
\begin{prop}\label{p bound <x>^2I}
    Suppose that the assumptions of Lemma \ref{l solvability of RHP N=0} are fulfilled and take $r\in H^1(\Real)\cap L^{2,1}(\Real)$. Then the functionals defined in (\ref{e def I_1,2}) satisfy the bound
    \begin{equation}\label{e bound <x>^2I}
        \begin{aligned}
            &\|I_1(r)\|_{H^1(\Real_+)\cap L^{2,1}(\Real_+)}\leq C\|r_-\|_{H^1\cap L^{2,1}}(\|r_+\|_{H^1\cap L^{2,1}}+\|r_-\|_{H^1\cap L^{2,1}})\|r\|_{H^1\cap L^{2,1}},\\
            &\|I_2(r)\|_{H^1(\Real_+)\cap L^{2,1}(\Real_+)}\leq C\|r_+\|_{H^1\cap L^{2,1}}(\|r_+\|_{H^1\cap L^{2,1}}+\|r_-\|_{H^1\cap L^{2,1}})\|r\|_{H^1\cap L^{2,1}}
        \end{aligned}
    \end{equation}
    where $C$ is a positive constant.
\end{prop}
\begin{proof}
    For the convenience of the reader we prove this proposition although it is already proven in \cite{Pelinovsky2016}. We find by (\ref{e component RHP solution formula}) and integrating by parts
    \begin{eqnarray*}
    % \nonumber to remove numbering (before each equation)
      I_1(r)(x) &=& \frac{1}{2\pi\ii} \int_{\Real}\Pm\left([m_+(z;x)]_{12} r_-(z) e^{2\ii zx}\right)\!(y)\: r(y) e^{-2\ii yx}dy\\
       &=&   \frac{-1}{2\pi\ii} \int_{\Real}[m_+(y;x)]_{12} r_-(y) e^{2\ii zx}\Pp\left(r(z) e^{-2\ii zx}\right)(y) dy.
    \end{eqnarray*}
    Using the H\"{o}lder inequality and the estimate (\ref{e sup P^pm r 1}), we arrive at
    \begin{equation*}
        \sup_{x\in(x_0,\infty)}|\langle x\rangle^2I_1(r)(x)|\leq \|r_-\|_{L^{\infty}}\|r\|_{H^1}\sup_{x\in(x_0,\infty)} \|\langle x\rangle[m_+(y;x)]_{12}\|_{L^2_z}.
    \end{equation*}
    We know $\sup_{x\in(x_0,\infty)} \|\langle x\rangle[m_+(y;x)]_{12}\|_{L^2_z}\leq C\|r_+\|_{H^1}$ by \cite[Lemma 10]{Pelinovsky2016} which completes the proof of $I_1(r)\in L^{2,1}(\Real_+)$. The assertion $\partial_x I_1(r)\in L^{2}(\Real_+)$ is established by using again the inhomogeneous equation (\ref{e component RHP solution formula}), its $x$ derivative, integration by parts, H\"{o}lder inequality, and in the end estimates (\ref{e sup P^pm r 1}) - (\ref{e sup P^pm r 3}),
    \begin{equation*}
        \sup_{x\in(x_0,\infty)} \|\langle x\rangle[m_+(y;x)]_{12}\|_{L^2_z}\leq C\|r_+\|_{H^1}
    \end{equation*}
    and
    \begin{equation*}
        \sup_{x\in\Real} \|[\partial_x m_+(y;x)]_{12}\|_{L^2_z}\leq C(\|r_+\|_{H^1\cap L^{2,1}}+\|r_-\|_{H^1\cap L^{2,1}}).
    \end{equation*}
    The latter statement can also be found in \cite[Lemma 10]{Pelinovsky2016}.
\end{proof}
The proposition above yields directly the following fundamental result (see Lemma 11 in \cite{Pelinovsky2016}):
\begin{cor}\label{c u in H^11 and H^2 (positive half line)}
   Fix $M>0$. Under the assumptions of Lemma \ref{l solvability of RHP N=0} and if $\|r_+\|_{H^1\cap L^{2,1}}+\|r_-\|_{H^1\cap L^{2,1}}\leq M$, the potential $u$ reconstructed from the solution $m$ of \rh \ref{rhp m^0} by using (\ref{e rec 1}) and (\ref{e rec 2}) lies in $H^2(\Real_+)\cap H^{1,1}(\Real_+)$. Moreover, it satisfies the bound
   \begin{equation}\label{e u in H^11 and H^2 (positive half line)}
       \|u\|_{H^2(\Real_+)\cap H^{1,1}(\Real_+)}\leq C_M,
   \end{equation}
   where the constant $C_M$ does not depend on $r_{\pm}$.
\end{cor}
\begin{proof}
    We set
    \begin{equation}\label{e def w}
        w(x):=u(x)e^{\ii\int_{+\infty}^x|u(y)|^2dy}
    \end{equation}
    and
    \begin{equation}\label{e def v}
        v(x):=\overline{u}(x)e^{-\frac{1}{2\ii} \int_{+\infty}^x|u(y)|^2dy},
    \end{equation}
    such that the following relations hold:
    \begin{equation}\label{e relations u v w}
        \begin{aligned}
            |u(x)|&=|v(x)|=|w(x)|\\
            |u_x(x)|&\leq |v_x(x)|+\frac{1}{2}|v(x)|^3
        \end{aligned}
    \end{equation}
    Using the reconstruction formulas (\ref{e rec 1}) and (\ref{e rec 2}), Proposition \ref{p cauchy operator} (ii) and the integral equation (\ref{e alternative RHP solution formula}) we immediately find
    \begin{equation*}
        w(x)=\frac{2}{\pi\ii}\int_{\Real}\overline{r}_+(z) e^{-2\ii zx}dz\;+\;4\,I_1(\overline{r}_+)(x)
    \end{equation*}
    and
    \begin{equation*}
        e^{-\frac{1}{2\ii} \int_{+\infty}^x|u(y)|^2dy}v_x(x)
        =-\frac{1}{\pi}\int_{\Real}r_-(z) e^{2\ii zx}dz\;-\;2\ii\:I_2(r_-)(x).
    \end{equation*}
    In each of these equations the first summand on the right hand side is controlled in $H^1\cap L^{2,1}$ since $r_{\pm}\in H^1\cap L^{2,1}$. Moreover Proposition \ref{p bound <x>^2I} yields directly $w\in L^{2,1}(\Real_+)$ and $v_x\in L^{2,1}(\Real_+)$ and thus finally by (\ref{e relations u v w}) $u\in H^{1,1}(\Real_+)$. Proposition \ref{p bound <x>^2I} also leads to
    \begin{equation*}
        \partial_x\left(e^{-\frac{1}{2\ii} \int_{+\infty}^x|u(y)|^2dy}v_x(x)\right)\in L^2_x(\Real_+).
    \end{equation*}
    By a straightforward calculation we conclude $u\in H^2(\Real_+)$. The bound (\ref{e u in H^11 and H^2 (positive half line)}) is obtained from application of (\ref{e bound <x>^2I}). The proof of the Corollary is now complete.
\end{proof}
With regard to the B\"{a}klund transformation which we intend to use in the following section in order to include solitons we need the following Lemma in addition to (\ref{e u in H^11 and H^2 (positive half line)}). The only purpose in the repeating of so many details of the inverse Scattering withour poles is to deduce this Lemma which can not be found in \cite{Pelinovsky2016}.
\begin{lem}\label{l m(z0)-1 in H^11 and H^2}
   Let the assumptions of Corollary \ref{c u in H^11 and H^2 (positive half line)} be valid and fix $z_0\in\Compl\setminus\Real$. Then for the solution $m(z;x)$ of \rh \ref{rhp m^0} we have $m(z_0;\cdot)-1\in H^1(\Real_+)\cap L^{2,1}(\Real_+)$ with the bound
   \begin{equation}\label{e m(z0)-1 in H^1 and L^2,1}
       \|m(z_0;\cdot)-1\|_{H^1(\Real_+)\cap L^{2,1}(\Real_+)}\leq C_{M},
   \end{equation}
   where the constant $C_M$ depends on $z_0$ and $M$ but not on $r_{\pm}$.
\end{lem}
\begin{proof}
    Fix $z_0\in\Compl\setminus\Real$. We use (\ref{e alternative RHP solution formula}) to find
    \begin{equation}\label{e m21 m12 in L^2,1}
        \begin{aligned}
            &[m(z_0;x)]_{12}= \frac{1}{2\pi\ii} \int_{\Real}\frac{[m_-(y;x)]_{11}\overline{r}_+(y) e^{-2\ii yx}}{y-z_0}dy= \frac{1}{2\pi\ii}\int_{\Real}\widetilde{r}_+(z) e^{-2\ii zx}dz\;+\;I_1(\widetilde{r}_+)(x),\\
            &[m(z_0;x)]_{21}= \frac{1}{2\pi\ii} \int_{\Real}\frac{[m_+(y;x)]_{22}r_-(y) e^{2\ii yx}}{y-z_0}dy= \frac{1}{2\pi\ii}\int_{\Real}\widetilde{r}_-(z) e^{2\ii zx}dz\;+\;I_2(\widetilde{r}_-)(x),
        \end{aligned}
    \end{equation}
    where $\widetilde{r}_-(z):={r}_-(z)/(z-z_0)$ and $\widetilde{r}_+(z):=\overline{r}_+(z)/(z-z_0)$, respectively. Due to the fact that $\|\widetilde{r}_{\pm}\|_{H^1\cap L^{2,1}}\leq c \|{r}_{\pm}\|_{H^1\cap L^{2,1}}$, where the constant $c>0$ depends on $z_0$ only, and using Proposition \ref{p bound <x>^2I} we end up with (\ref{e m(z0)-1 in H^1 and L^2,1}) for the non diagonal entries $m_{12}$ and $m_{21}$.
    Using again (\ref{e alternative RHP solution formula}) we obtain
    \begin{equation*}
        [m(z_0;x)]_{11}=1+\frac{1}{2\pi\ii} \int_{\Real} \frac{[m_+(y;x)]_{12}r_-(y) e^{2\ii yx}}{y-z_0}dy,
    \end{equation*}
    where we can insert $[m_+(y;x)]_{12}=\Pp ([m_-(z;x)]_{11} \overline{r}_+(z) e^{-2\ii zx})(y)$ from the integral equation (\ref{e integral equation for m+-}). Then we integrate by parts and obtain
    \begin{equation}\label{e formula for m(z_0;x)-1 in the proof of the lemma}
        [m(z_0;x)]_{11}=1-\frac{1}{2\pi\ii} \int_{\Real} [m_-(y;x)]_{11} \overline{r}_+(y) e^{-2\ii yx}\;\Pm (\widetilde{r}_-(z) e^{2\ii zx})(y)\:dy,
    \end{equation}
    where we put again $\widetilde{r}_-(z):={r}_-(z)/(z-z_0)$. Furthermore we set
    \begin{equation*}
        R_+(y):=\overline{r}_+(y)\;\Pm (\widetilde{r}_-(z) e^{2\ii zx})(y).
    \end{equation*}
    To prove $R_+\in H^1\cap L^{2,1}$ we recall the continuity property $\|\Ppm\|_{L^2\to L^2}=1$. One consequence is that $\|\Pm (\widetilde{r}_-(z) e^{2\ii zx})(\cdot)\|_{L^2} \leq c \|r_-\|_{L^2}$. Additionally, we find
    \begin{equation*}
        \|\partial_z \Pm (\widetilde{r}_-(z) e^{2\ii zx})(z)\|_{L^2_z}\leq \| \Pm (\widetilde{r}'_-(z) e^{2\ii zx})(\cdot)\|_{L^2_z} +\|2\ii x \Pm (\widetilde{r}_-(z) e^{2\ii zx})(\cdot)\|_{L^2_z},
    \end{equation*}
    where we can apply the bound (\ref{e sup P^pm r 1}) of Proposition \ref{p cauchy operator} and again $\|\Ppm\|_{L^2\to L^2}=1$. Thus, we are able to control  $\Pm (\widetilde{r}_-(z) e^{2\ii zx})(\cdot)$ in $H^1$ uniformly for $x>0$. Altogether, we have shown $\|R_+\|_ {H^1\cap L^{2,1}}\leq c \|r_+\|_ {H^1\cap L^{2,1}}\|r_-\|_ {H^1}$, which is needed because we want to apply Proposition \ref{p bound <x>^2I}. Therefore we write (\ref{e formula for m(z_0;x)-1 in the proof of the lemma}) in the form
    \begin{equation*}
        [m(z_0;x)]_{11}-1=\frac{1}{2\pi\ii} \int_{\Real} R_+(y)e^{-2\ii yx}\:dy+ \:I_1(R_+\!)(x).
    \end{equation*}
    Analogously, it can be carried out in a similar way, that for $R_-(y):=r_-(y)\;\Pp (\widetilde{r}_+(z) e^{-2\ii zx})(y)$,
    \begin{equation*}
        [m(z_0;x)]_{22}-1=\frac{1}{2\pi\ii} \int_{\Real} R_-(y)e^{2\ii yx}\:dy+ \:I_2(R_-\!)(x).
    \end{equation*}
    Combining Fourier theory and the bound (\ref{e bound <x>^2I}) we have now accomplished the proof of (\ref{e m(z0)-1 in H^1 and L^2,1}) also for the diagonal entries.
\end{proof}
Estimates on the negative half-line can be found by modifying the solution $m(z;x)$ of \rh \ref{rhp m^0} in the following way:
\begin{equation}\label{e def m delta}
    m_{\delta}(z;x):=m(z;x)
    \left[
      \begin{array}{cc}
        \delta^{-1}(z) & 0 \\
        0 & \delta(z) \\
      \end{array}
    \right],
\end{equation}
where
\begin{equation}\label{e delta}
    \delta(z)=\exp\left(\frac{1}{2\pi\ii} \int_{\Real}\frac{\log(1+ \overline{r}_+(y)r_-(y))}{y-z}dy\right).
\end{equation}
In Proposition 8 in \cite{Pelinovsky2016} it is shown that $\log(1+ \overline{r}_+r_-)\in L^2(\Real)$ due to (\ref{e r constraint}). Hence, the integral in (\ref{e delta}) is well-defined and $\delta$ solves the following RHP:
\begin{framed}
    \begin{rhp}\label{rhp delta}
        Find a scalar valued function $\Compl\ni z\mapsto \delta(z)$ which satisfies
        \begin{enumerate}[(i)]
          \item $\delta(z)$ is meromorphic in $\Compl\setminus\Real$.
          \item $\delta(z)=1+\mathcal{O} \left(\frac{1}{z}\right)$ as $|z|\to\infty$.
          \item The non-tangential boundary values $\delta_{\pm}(z)$ exist for $z\in\Real$ and satisfy the jump relation
              \begin{equation}\label{e jump delta}
                 \delta_+(z)=\left[1+ \phantom{\widehat{l}}\overline{r}_+(z)r_-(z) \right] \delta_-(z).
              \end{equation}
        \end{enumerate}
    \end{rhp}
\end{framed}
Using the symmetry $\delta(\overline{z})=\overline{\delta}^{-1}(z)$ and the jump condition (\ref{e jump delta}) it is an easy exercise to verify, that the function $m_{\delta}(z;x)$ defined in \ref{e def m delta} is a solution to the following \rh:
\begin{samepage}
\begin{framed}
    \begin{rhp}\label{rhp m^0 delta}
        Find for each $x\in\Real$ a $2\times 2$-matrix valued function $\Compl\ni z\mapsto m_{\delta}(z;x)$ which satisfies
        \begin{enumerate}[(i)]
          \item $m_{\delta}(z;x)$ is meromorphic in $\Compl\setminus\Real$ (with respect to the parameter $z$).
          \item $m_{\delta}(z;x)=1+\mathcal{O} \left(\frac{1}{z}\right)$ as $|z|\to\infty$.
          \item The non-tangential boundary values $m_{\pm,\delta}(z;x)$ exist for $z\in\Real$ and satisfy the jump relation
              \begin{equation}\label{e jump m^0 delta}
                  m_{+,\delta}=m_{-,\delta}(1+R_{\delta}) ,\quad\text{where}\quad
                  R_{\delta}(z;x):=
                  \left[
                    \begin{array}{cc}
                       0 & e^{-2\ii zx}\overline{r}_{+,\delta}(z) \\
                       e^{2\ii zx}r_{-,\delta}(z) & \overline{r}_{+,\delta}(z) r_{-,\delta}(z) \\
                    \end{array}
                  \right],
              \end{equation}
              and $r_{\pm,\delta}(z):= \overline{\delta}_+(z) \overline{\delta}_-(z)r_{\pm}(z)$.
        \end{enumerate}
    \end{rhp}
\end{framed}
\end{samepage}
The new jump matrix $R_{\delta}$ admits an factorization analogously to (\ref{e def R+ and R-}). For
\begin{equation}\label{e def R+ and R- delta}
    R_{+,\delta}(z;x):=
                  \left[
                    \begin{array}{cc}
                       0 & 0 \\
                       e^{2\ii zx}r_{-,\delta}(z) & 0 \\
                    \end{array}
                  \right],\qquad
    R_{-,\delta}(z;x):=
                  \left[
                    \begin{array}{cc}
                       0 & e^{-2\ii zx}\overline{r}_{+,\delta}(z) \\
                       0 & 0 \\
                    \end{array}
                  \right],
\end{equation}
we find
$m_{+,\delta} -m_{-,\delta}=m_{-,\delta}R_{+,\delta} +m_{+,\delta}R_{-,\delta}$ for $z\in \Real$, such that analogously to (\ref{e alternative RHP solution formula}),
\begin{equation*}
    m_{\delta}(z;x)=1+\frac{1}{2\pi\ii}\int_{\Real} \frac{m_{-,\delta}(y;x)R_{+,\delta}(y;x)+ m_{+,\delta}(y;x)R_{-,\delta}(y;x)}{y-z}dy.
\end{equation*}
The following exemplary calculation shows why \rh \ref{rhp m^0 delta} can be studied in order to extend Lemma \ref{l m(z0)-1 in H^11 and H^2} and Corollary \ref{c u in H^11 and H^2 (positive half line)} to the negative half-line. We have for $z_0\in\Compl\setminus\Real$
\begin{eqnarray*}
% \nonumber to remove numbering (before each equation)
  [m_{\delta}(z_0;x)]_{12}&=& \frac{1}{2\pi\ii} \int_{\Real}\frac{[m_{+,\delta}(y;x)]_{11} \overline{r}_{+,\delta}(y) e^{-2\ii yx}}{y-z_0}dy \\
   &=&  \frac{1}{2\pi\ii} \int_{\Real}\widetilde{r}_{+,\delta}(y) e^{-2\ii yx}dy+I_{1,\delta}(\widetilde{r}_{+,\delta})
\end{eqnarray*}
where $\widetilde{r}_{+,\delta}(z):= \overline{r}_{+,\delta}(z)/(z-z_0)$ and
\begin{equation*}
    I_{1,\delta}(r):=\frac{1}{2\pi\ii} \int_{\Real}[m_{+,\delta}(y;x)-1]_{11} r(y) e^{-2\ii yx}dy.
\end{equation*}
The functional $I_{1,\delta}(r)$ satisfies the same estimates as in Proposition \ref{p bound <x>^2I} with $\Real_+$ replaced by $\Real_-$ because the operators $\Pp$ and $\Pm$ swap their places in comparison with the integral equation (\ref{e integral equation for m+-}).
\begin{lem}\label{l m(z0)-1 in in H^1 and L^2,1 (delta)}
   Fix $M>0$ and $z_0\in\Compl\setminus\Real$ and let the assumptions of Lemma \ref{l solvability of RHP N=0} be valid. If in addition $\|r_+\|_{H^1\cap L^{2,1}}+\|r_-\|_{H^1\cap L^{2,1}}\leq M$, then for the solution $m_{\delta}(z;x)$ of \rh \ref{rhp delta} we have $m_{\delta}(z_0;\cdot)-1\in H^1(\Real_-)\cap L^{2,1}(\Real_-)$ with the bound
   \begin{equation*}%\label{e m(z0)-1 in H^1 and L^2,1}
       \|m_{\delta}(z_0;\cdot)-1\|_{H^1(\Real_-)\cap L^{2,1}(\Real_-)}\leq C_{M},
   \end{equation*}
   where the constant $C_M$ depends on $z_0$ and $M$, but not on $r_{\pm}$.
\end{lem}
With respect to the potential $u(x)$ the two Riemann--Hilbert problems \ref{rhp m^0} and \ref{rhp m^0 delta} are equivalent in the following sense:
\begin{equation}\label{e equivalent rhps}
    \begin{aligned}
        \lim_{|z|\to\infty}z\;[m(z;x)]_{12}= \lim_{|z|\to\infty}z\;[m_{\delta}(z;x)]_{12},\\
        \lim_{|z|\to\infty}z\;[m(z;x)]_{21}= \lim_{|z|\to\infty}z\;[m_{\delta}(z;x)]_{21}.
    \end{aligned}
\end{equation}
This observation follows directly from the definition (\ref{e def m delta}) and leads to the following extension of Corollary \ref{c u in H^11 and H^2 (positive half line)}.
\begin{cor}\label{c u in H^11 and H^2 (negative half line)}
   Fix $M>0$. Under the assumptions of Lemma \ref{l solvability of RHP N=0} and if $\|r_+\|_{H^1\cap L^{2,1}}+\|r_-\|_{H^1\cap L^{2,1}}\leq M$, the potential $u$ reconstructed from the solution $m$ of \rh \ref{rhp m^0} by using (\ref{e rec 1}) and (\ref{e rec 2}) lies in $H^2(\Real_-)\cap H^{1,1}(\Real_-)$ and satisfies the bound
   \begin{equation}\label{e u in H^11 and H^2 (negative half line)}
       \|u\|_{H^2(\Real_-)\cap H^{1,1}(\Real_-)}\leq C_M
   \end{equation}
   where the constant $C_M$ does not depend on $r_{\pm}$.
\end{cor}


\section{Adding a pole}\label{s adding a pole}
In this section we want to prove the solvability of \rh \ref{rhp m} if $N=1$. An auto-B\"{a}cklund transformation will establish a connection between the cases $N=1$ and $N=0$. All formulas were found in \cite{Deift2011} and \cite{Cuccagna2014}, where the B\"{a}cklund transformation was used in the context of the NLS equation. \\
Assume that a function $u^{(1)}\in H^2\cap H^{1,1}$ provides scattering data $\mathcal{S}^{(1)}=\set{r^{(1)}_{\pm};z_1;c_1}$. We recall the corresponding Riemann--Hilbert problem (without time dependence):
\begin{samepage}
\begin{framed}
    \begin{rhp}\label{rhp m^1}
        Find for each $x\in\Real$ a $2\times 2$-matrix valued function $\Compl\ni z\mapsto m^{(1)}(z;x)$ which satisfies
        \begin{enumerate}[(i)]
          \item $m^{(1)}(z;x)$ is meromorphic in $\Compl\setminus\Real$ (with respect to the parameter $z$).
          \item $m^{(1)}(z;x)=1+\mathcal{O}\left(\frac{1}{z}\right)$ as $|z|\to\infty$.
          \item The non-tangential boundary values $m^{(1)}_{\pm}(z;x)$ exist for $z\in\Real$ and satisfy the jump relation
              \begin{equation}\label{e jump m^1}
                  m^{(1)}_+=m^{(1)}_-(1+R), \quad\text{where}\quad
                  R(z;x):=
                  \left[
                    \begin{array}{cc}
                       \overline{r}_+(z)r_-(z) & e^{-2\ii zx}\overline{r}_+(z) \\
                       e^{2\ii zx}r_-(z) & 0 \\
                    \end{array}
                  \right].
              \end{equation}
          \item $m^{(1)}$ has simple poles at $z_1$ and $\overline{z}_1$ with
              \begin{equation}\label{e Res m^1}
                  \begin{aligned}
                     \res_{z=z_1}m^{(1)}(z;x)&=\lim_{z\to z_1}m^{(1)}(z;x)
                  \left[
                      \begin{array}{cc}
                        0 & 0 \\
                        2\ii\lambda_1c_1 e^{2\ii z_1x} & 0
                      \end{array}
                  \right],\\
                     \res_{z=\overline{z}_1} m^{(1)}(z;x)&=\lim_{z\to \overline{z}_1}m^{(1)}(z;x)
                  \left[
                      \begin{array}{cc}
                      0 & \frac{-\overline{c}_1 e^{-2\ii \overline{z}_1x}}{2\ii\lambda_1} \\
                      0 & 0
                      \end{array}
                  \right].
                  \end{aligned}
              \end{equation}
        \end{enumerate}
    \end{rhp}
\end{framed}
\end{samepage}
 By construction, the constraints (\ref{e relation r+ r-}) - (\ref{e r constraint}) hold. Now we change these data by removing the pole $z_1$ and modifying the reflection coefficient in the following way:
\begin{equation}\label{e r^0}
    r^{(0)}_{\pm}(z):=r^{(1)}_{\pm}(z) \frac{z-\overline{z}_1}{z-z_1}.
\end{equation}
Obviously, $r^{(0)}_{\pm}$ satisfy (\ref{e relation r+ r-}) -- (\ref{e r constraint}) and moreover, $r^{(1)}_{\pm}\in H^1\cap L^{2,1}$ implies $r^{(0)}_{\pm}\in H^1\cap L^{2,1}$. Hence, all assumptions of Lemma \ref{l solvability of RHP N=0} are satisfied and we get an unique solution $m^{(0)}(z;x)$ of \rh \ref{rhp m^0} with our new data $\mathcal{S}^{(0)}:=\set{r^{(0)}_{\pm}}$. This procedure defines a map $u^{(1)}(x)\mapsto u^{(0)}(x)$, where $u^{(0)}(x)$ is defined to be the pure radiation potential  which is associated to $m^{(0)}(z;x)$ by the reconstruction formulas (\ref{e rec 1}) and (\ref{e rec 1}), respectively.
\subsection{B\"{a}cklund transformation for $x>0$}
What we will do in this subsection is to explore the map $u^{(1)}\leftrightarrow u^{(0)}$ for $x>0$. Therefore we introduce the functions $w^{(j)}$, $v^{(j)}$ for $j=0,1$, which are related to $u^{(j)}$ by (\ref{e def w}) and (\ref{e def v}), respectively. Next we define the matrix
\begin{equation*}
    A(x)=
    \left[
      \begin{array}{cc}
        a_{11}(x) & a_{12}(x) \\
        a_{21}(x) & a_{22}(x) \\
      \end{array}
    \right]
\end{equation*}
by
\begin{equation*}
        \left(
          \begin{array}{c}
            a_{11}(x) \\
            a_{21}(x) \\
          \end{array}
        \right)
        :=m^{(0)}(z_1;x)
        \left(
          \begin{array}{c}
            1 \\
            - \frac{2\ii\lambda_1c_1 e^{2\ii z_1x}} {z_1-\overline{z}_1} \\
          \end{array}
        \right),
        \quad
%
        \left(
          \begin{array}{c}
            a_{12}(x) \\
            a_{22}(x) \\
          \end{array}
        \right)
        :=m^{(0)}(\overline{z}_1;x)
        \left(
          \begin{array}{c}
            \frac{\overline{c}_1 e^{-2\ii \overline{z}_1 x}} {2\ii\overline{\lambda}_1(\overline{z}_1-z_1)} \\
            1 \\
          \end{array}
        \right).
\end{equation*}
In order to define the B\"{a}cklund transformation it is necessary to know that there is no $x$ such that the determinant of $A(x)$ vanishes.
\begin{prop}\label{p A inverse}
    The matrix $A$ is invertible for all $x\in\Real$. Moreover, if $\|r^{(1)}_+\|_{H^1\cap L^{2,1}}+\|r^{(1)}_-\|_{H^1\cap L^{2,1}}\leq M$, then
    \begin{equation} \label{e lower bound for det A}
    |\det(A(x))|^{-1}\leq C_M,\quad\text{for all }x>0,
    \end{equation}
    where the constant $C_M$ does not depend on $x$ and $r_{\pm}$.
\end{prop}
\begin{proof}
    Using the symmetry (\ref{e symmetrie of m}) we find
    \begin{eqnarray*}
    % \nonumber to remove numbering (before each equation)
       \left(
          \begin{array}{c}
            a_{12}(x) \\
            a_{22}(x) \\
          \end{array}
       \right)
       &=&
       \left[
         \begin{array}{cc}
           w^{(0)}(x) & 1 \\
           -|w^{(0)}(x)|^2-4\overline{z}_1 & -\overline{w^{(0)}}(x) \\
         \end{array}
       \right]
       \overline{m^{(0)}}(z_1;x)
       \left[
          \begin{array}{cc}
            0 & \frac{-1}{4\overline{z}_1} \\
            1 & 0 \\
          \end{array}
       \right]
       \left(
          \begin{array}{c}
            \frac{\overline{c}_1 e^{-2\ii \overline{z}_1 x}} {2\ii\overline{\lambda}_1(\overline{z}_1-z_1)} \\
            1 \\
          \end{array}
       \right)
       \\
       &=&
       \frac{1}{4\overline{z}_1}
       \left[
         \begin{array}{cc}
           -w^{(0)}(x) & -1 \\
           |w^{(0)}(x)|^2+4\overline{z}_1 & \overline{w^{(0)}}(x) \\
         \end{array}
       \right]
       \overline{m^{(0)}}(z_1;x)
       \left(
          \begin{array}{c}
            1\\
            \frac{2\ii\overline{\lambda}_1\overline{c}_1 e^{-2\ii \overline{z}_1 x}} {(\overline{z}_1-z_1)} \\
          \end{array}
       \right)
       \\
       &=&
       \frac{1}{4\overline{z}_1}
       \left[
         \begin{array}{cc}
           -w^{(0)}(x) & -1 \\
           |w^{(0)}(x)|^2+4\overline{z}_1 & \overline{w^{(0)}}(x) \\
         \end{array}
       \right]
       \left(
          \begin{array}{c}
            \overline{a_{11}}(x) \\
            \overline{a_{21}}(x) \\
          \end{array}
       \right).
    \end{eqnarray*}
    It follows directly that
    \begin{equation*}
        \det(A(x))=|a_{11}(x)|^2 +\frac{1}{4\overline{z}_1} |\overline{w^{(0)}}(x)a_{11}(x)+a_{21}(x) |^2.
    \end{equation*}
    The case $\det(A(x))=0$ is impossible, since due to $\im(z_1)\neq 0$ it would follow that $a_{11}(x)=a_{21}(x)=0$ and hence $\left(1,
    - \frac{2\ii\lambda_1c_1 e^{2\ii z_1x}}{z_1-\overline{z}_1}\right)^T \in\ker[m^{(0)}]$. This contradicts $\det(m^{(0)}(z;x))\equiv 1$ (see Remark \ref{r uniqueness + det=1}). Now we turn to the proof of (\ref{e lower bound for det A}). For sake of contradiction we assume that for any $d>0$ we can find $x>0$ such that $|\det(A(x))|<d$. Due to $\im(z_1)\neq 0$ and $w\in L^{\infty}$ wa can assume w.l.o.g. $|a_{11}(x)|< d$ and $|a_{12}(x)|<d$. Using (\ref{e det m=1}) we find
    \begin{eqnarray*}
    % \nonumber to remove numbering (before each equation)
      1 &=& \left|[m^{(0)}(z_1;x)]_{11} [m^{(0)}(z_1;x)]_{22}- [m^{(0)}(z_1;x)]_{12} [m^{(0)}(z_1;x)]_{21}\right| \\
       &=& \left|\left\{a_{11}(x) +\frac{2\ii\lambda_1c_1 e^{2\ii z_1x}} {z_1-\overline{z}_1}[m^{(0)}(z_1;x)]_{12}\right\} [m^{(0)}(z_1;x)]_{22}\right.\\
         &&\,
         \left.-\,[m^{(0)}(z_1;x)]_{12} \left\{ a_{21}(x)+ \frac{2\ii\lambda_1c_1 e^{2\ii z_1x}} {z_1-\overline{z}_1}[m^{(0)}(z_1;x)]_{22}\right\} \right|\\
       &=& \left|a_{11}(x) [m^{(0)}(z_1;x)]_{22}- [m^{(0)}(z_1;x)]_{12} a_{21}(x)\right|\\
       &<&d\cdot   \left\{\left|[m^{(0)}(z_1;x)]_{22}\right|+ \left|[m^{(0)}(z_1;x)]_{12}\right|\right\}\\
       &\leq&d\cdot C\cdot  \|m^{(0)}(z_1;\cdot)-1\|_{H^1(\Real_+)\cap L^{2,1}(\Real_+)}\\
       &\leq& d\cdot C_M.
    \end{eqnarray*}
    Here $C_M$ is the constant in Lemma \ref{l m(z0)-1 in H^11 and H^2} and it follows that $d$ cannot be arbitrary small. In addition we also proved the bound (\ref{e lower bound for det A}).
\end{proof}
\begin{lem}\label{l solvability of RHP N=1}
    For any scattering data $\mathcal{S}^{(1)}=\{r^{(1)}_{\pm};z_1;c_1\}$ such that $r^{(1)}_{\pm}\in L^{2,1}\cap H^1$ satisfies (\ref{e relation r+ r-})-(\ref{e r constraint}), \rh \ref{rhp m^1} admits an unique solution $m^{(1)}(z;x)$. This solution can be obtained from $m^{(0)}(z;x)$ by the following:
    \begin{equation}\label{e B�cklund for m^1}
        m^{(1)}(z;x)=A(x)\mu(z) A^{-1}(x)m^{(0)}(z;x)\mu^{-1}(z),
    \end{equation}
    where
    \begin{equation*}
        \mu(z)=
        \left[
          \begin{array}{cc}
            z-z_1 & 0 \\
            0 & z-\overline{z}_1 \\
          \end{array}
        \right].
    \end{equation*}
\end{lem}
\begin{proof}
    Let us denote by $\widetilde{m}(z;x)$ the right hand side of (\ref{e B�cklund for m^1}) and set
    \begin{equation*}
        \left[
          \begin{array}{cc}
            \tau_{11}(z) & \tau_{12}(z) \\
            \tau_{21}(z) & \tau_{22}(z) \\
          \end{array}
        \right]:=A^{-1}(x)m^{(0)}(z;x).
    \end{equation*}
    We find
    \begin{equation}\label{e Res of m tilde}
        \begin{aligned}
          \res_{z=z_1}\widetilde{m}(z;x)&=
          A(x)
          \left[
            \begin{array}{cc}
              0 & 0 \\
              (z_1-\overline{z}_1)\tau_{21}(z_1) & 0 \\
            \end{array}
          \right]
          ,\\
          \res_{z=\overline{z}_1}\widetilde{m} (z;x)&=
          A(x)
          \left[
            \begin{array}{cc}
              0 & (\overline{z}_1-z_1) \tau_{12}(\overline{z}_1) \\
              0 & 0 \\
            \end{array}
          \right]
          ,
        \end{aligned}
    \end{equation}
    and
    \begin{equation}\label{e lim m tilde c}
        \begin{aligned}
        &\lim_{z\to z_1}\widetilde{m}(z;x)
          \left(
            \begin{array}{cc}
              0 & 0 \\
              2\ii\lambda_1c_1 e^{2\ii z_1x} & 0
            \end{array}
          \right)=
          A(x)
          \left[
            \begin{array}{cc}
              0 & 0 \\
              2\ii\lambda_1c_1 e^{2\ii z_1x}\tau_{22}(z_1) & 0 \\
            \end{array}
          \right]
          ,\\
          &\lim_{z\to \overline{z}_1}\widetilde{m}(z;x)
          \left(
            \begin{array}{cc}
              0 & \frac{-\overline{c}_1}{2\ii\lambda_1} e^{-2\ii \overline{z}_1x} \\
              0 & 0
            \end{array}
          \right)=
          A(x)
          \left[
            \begin{array}{cc}
              0 &\frac{-\overline{c}_1}{2\ii\lambda_1} \tau_{11}(\overline{z}_1) \\
              0 & 0 \\
            \end{array}
          \right].
        \end{aligned}
      \end{equation}
      Using $\det m^{(0)}\equiv 1 $ it is easy to obtain
      \begin{equation*}
        \tau_{21}(z_1)=\frac{1}{\det A(x)} \frac{2\ii\lambda_1c_1 e^{2\ii z_1x}}{z_1-\overline{z}_1},\quad
        \tau_{22}(z_1)=\frac{1}{\det A(x)},
      \end{equation*}
      and
      \begin{equation*}
        \tau_{11}(\overline{z}_1)=\frac{1}{\det A(x)},\quad\tau_{12}(\overline{z}_1)= \frac{-1}{\det A(x)} \frac{\overline{c}_1 e^{-2\ii \overline{z}_1x}}{2\ii\overline{\lambda}_1 (\overline{z}_1-z_1)},
      \end{equation*}
      and thus it follows from (\ref{e Res of m tilde}) and (\ref{e lim m tilde c}) that $\widetilde{m}$ satisfies (\ref{e Res m^1}). Now we proceed with the jump on the real axis and check if point (iii) of \rh \ref{rhp m^1} is satisfied. Using the jump condition of $m^{(0)}$ (see (\ref{e jump})) and the definition (\ref{e r^0}) of $r_{\pm}^{(0)}$ we find for $z\in\Real$
      \begin{eqnarray*}
      % \nonumber to remove numbering (before each equation)
        \widetilde{m}_+(z;x) &=& \widetilde{m}_-(z;x) \mu(z)\left(
          \begin{array}{cc}
            1+\overline{r}^{(0)}_+(z)r^{(0)}_-(z) & e^{-2\ii zx}\overline{r}^{(0)}_+(z) \\
            e^{2\ii zx}r^{(0)}_-(z) & 1 \\
          \end{array}
        \right)\mu^{-1}(z) \\
         &=&  \widetilde{m}_-(z;x)\left(
          \begin{array}{cc}
            1+\overline{r}^{(1)}_+(z)r^{(1)}_-(z) & e^{-2\ii zx}\overline{r}^{(1)}_+(z) \\
            e^{2\ii zx}r^{(1)}_-(z) & 1 \\
          \end{array}
        \right).
      \end{eqnarray*}
      Next we observe
      \begin{equation}\label{e expansion of m tilde}
        \widetilde{m}(z;x)= \left[1+\frac{A(x)\:\mu(0)\:A^{-1}(x)}{z}\right] m^{(0)}(z;x)
        \left[
          \begin{array}{cc}
            \frac{z}{z-z_1} & 0 \\
            0 & \frac{z}{z-\overline{z}_1} \\
          \end{array}
        \right].
      \end{equation}
      It follows that $\widetilde{m}$ behaves for $|z|\to\infty$ as required in the point (ii) of \rh \ref{rhp m^1}. Since also the point (i) of \rh \ref{rhp m^1} is true, we conclude by the uniqueness (see Remark \ref{r uniqueness + det=1}) that $m^{(1)}(z;x)\equiv\widetilde{m}(z;x)$.
\end{proof}
The B\"{a}cklund transformation formula (\ref{e B�cklund for m^1}) is an ideal expression to extend Corollary \ref{c u in H^11 and H^2 (positive half line)} and Lemma \ref{l m(z0)-1 in H^11 and H^2} to the case where the scattering data are involving one pole $z_1$.
\begin{cor}\label{c u in H^11 and H^2 (positive half line - 1 pole)}
   Under the assumptions of Lemma \ref{l solvability of RHP N=1} the potential $u^{(1)}(x)$ reconstructed from the solution $m^{(1)}(z;x)$ of \rh \ref{rhp m dynamic} by using (\ref{e rec 1}) and (\ref{e rec 2}) lies in $H^2(\Real_+)\cap H^{1,1}(\Real_+)$. Moreover, if $\|r^{(1)}_+\|_{H^1\cap L^{2,1}}+\|r^{(1)}_-\|_{H^1\cap L^{2,1}}+|c_1|\leq M$ for some fixed $M>0$, then $u^{(1)}$ satisfies the bound
   \begin{equation}\label{e u^1 in H^11 and H^2 (positive half line)}
       \|u^{(1)}\|_{H^2(\Real_+)\cap H^{1,1}(\Real_+)}\leq C_M
   \end{equation}
   where the constant $C_M$ depends on $M$ and $z_1$ but not on $r^{(1)}_{\pm}$ and $|c_1|$.
\end{cor}
\begin{proof}
    We use (\ref{e expansion of m tilde}) and
    the expansion
    \begin{equation*}
        \left[
          \begin{array}{cc}
            \frac{z}{z-z_1} & 0 \\
            0 & \frac{z}{z-\overline{z}_1} \\
          \end{array}
        \right]=1-\frac{\mu(0)}{z}+\mathcal{O}(z^{-2}), \quad\text{as }|z|\to\infty
    \end{equation*}
    in order to find
    \begin{equation*}
        \begin{aligned}
            \lim_{|z|\to\infty}z\; \left[m^{(1)}(z;x)\right]_{12}& = \lim_{|z|\to\infty}z\; \left[m^{(0)}(z;x)\right]_{12}+ \left[A(x)\:\mu(0)\:A^{-1}(x)\right]_{12},\\
            \lim_{|z|\to\infty}z\; \left[m^{(1)}(z;x)\right]_{21}& = \lim_{|z|\to\infty}z\; \left[m^{(0)}(z;x)\right]_{21}+ \left[A(x)\:\mu(0)\:A^{-1}(x)\right]_{21}.
        \end{aligned}
    \end{equation*}
    Using the notation (\ref{e def w}) and (\ref{e def v}), we find by the reconstruction formulas (\ref{e rec 1}) and (\ref{e rec 2})
    \begin{equation}\label{e decomposition w}
       w^{(1)}(x)=w^{(0)}(x)+B_1(x),\qquad B_1(x):=-\frac{8\ii \im(z_1)a_{11}(x)a_{12}(x)}{\det(A(x))}
    \end{equation}
    and
    \begin{multline}\label{e decomposition v}
        \qquad e^{-\frac{1}{2\ii} \int_{+\infty}^x|u^{(1)}(y)|^2dy}v^{(1)}_x(x)= e^{-\frac{1}{2\ii} \int_{+\infty}^x|u^{(0)}(y)|^2dy}v^{(0)}_x(x) +B_2(x),\\
        B_2(x):=\frac{4 \im(z_1)a_{21}(x)a_{22}(x)}{\det(A(x))}.\qquad
    \end{multline}
    As it is easily to derive from the definition of $A(x)$ and Lemma \ref{l m(z0)-1 in H^11 and H^2}, we have $(A(\cdot)-1)\in L^{2,1}(\Real_+)\cap H^1(\Real_+)$ (note that $\im(z_1)>0$ is necessary). In addition, $(\det(A(\cdot))-1)\in L^{2,1}(\Real_+)\cap H^1(\Real_+)$. These two facts and \ref{e lower bound for det A} yield $B_j(\cdot)\in L^{2,1}(\Real_+)\cap H^1(\Real_+)$ for $j=1,2$. If we apply Corollary \ref{c u in H^11 and H^2 (positive half line)} to $v^{(0)}$ and $w^{(0)}$, we end up with $w^{(1)}\in H^{1,1}(\Real_+)$ and
    \begin{equation*}
        \partial_x \left(e^{-\frac{1}{2\ii} \int_{+\infty}^x|u^{(1)}(y)|^2dy } v^{(1)}_x(x)\right)\in L^2_x(\Real_+),
    \end{equation*}
    which is sufficient to conclude $u^{(1)}\in H^2(\Real_+)\cap H^{1,1}(\Real_+)$.
\end{proof}
\begin{cor}\label{c m^1(z0)-1 in H^11 and H^2}
   Let the assumptions of Lemma \ref{l solvability of RHP N=1} be valid and fix $z_2\in\Compl\setminus(\Real\cup \set{z_1,\overline{z}_1})$. Then for the solution $m^{(1)}(z;x)$ of \rh \ref{rhp m dynamic} we have $m^{(1)}(z_2;\cdot)-1\in H^1(\Real_+)\cap L^{2,1}(\Real_+)$. Moreover, if $\|r^{(1)}_+\|_{H^1\cap L^{2,1}}+\|r^{(1)}_-\|_{H^1\cap L^{2,1}}\leq M$ for some fixed $M>0$, then we also have the bound
   \begin{equation}\label{e m^1(z0)-1 in H^1 and L^2,1}
       \|m^{(1)}(z_2;\cdot)-1\|_{H^1(\Real_+)\cap L^{2,1}(\Real_+)}\leq C_M,
   \end{equation}
   where the constant $C_M>0$ depends on $M$, $z_1$, $z_2$ and $|c_1|$ but not on $r^{(1)}_{\pm}$.
\end{cor}
\begin{proof}
    (\ref{e B�cklund for m^1}) can be written as
    \begin{multline*}
        m^{(1)}(z_2;x)=m^{(0)}(z_2;x)\\-2\ii\im(z_1) A(x)
        \left[
          \begin{array}{cc}
            0 & \frac{a_{21}(x)[m(z_2;x)]_{11}- a_{11}(x)[m(z_2;x)]_{21}} {(z_2-z_1)\det(A(x))} \\
            \frac{a_{22}(x)[m(z_2;x)]_{12}- a_{12}(x)[m(z_2;x)]_{22}} {(z_2-\overline{z}_1)\det(A(x))} & 0 \\
          \end{array}
        \right].
    \end{multline*}
    $m^{(1)}(z_2;\cdot)-1\in H^1(\Real_+)\cap L^{2,1}(\Real_+)$ is now a direct consequence of $m^{(0)}(z_2;\cdot)-1\in H^1(\Real_+)\cap L^{2,1}(\Real_+)$ (see Lemma \ref{l m(z0)-1 in H^11 and H^2}), $(A(\cdot)-1)\in L^{2,1}(\Real_+)\cap H^1(\Real_+)$, $(\det(A(\cdot))-1)\in L^{2,1}(\Real_+)\cap H^1(\Real_+)$ and (\ref{e lower bound for det A}).
\end{proof} 
\subsection{B\"{a}cklund transformation for $x<0$}
We consider the solution $m^{(1)}(z;x)$ of \rh \ref{rhp m^1} provided by Lemma \ref{l solvability of RHP N=1} and define
\begin{equation}\label{e def m^1 delta}
    m^{(1)}_{\delta}(z;x):=m^{(1)}(z;x)
    \left[
      \begin{array}{cc}
        \frac{z-z_1}{z-\overline{z}_1} & 0 \\
        0 & \frac{z-\overline{z}_1}{z-z_1} \\
      \end{array}
    \right]
    \left[
      \begin{array}{cc}
        \delta^{-1}(z) & 0 \\
        0 & \delta(z) \\
      \end{array}
    \right].
\end{equation}
The factor $\left(\frac{z-z_1}{z-\overline{z}_1}\right)^{\sigma_1}$ swaps the columns where the poles arise. The second factor $\delta^{-\sigma_1}$ has influence on the structure of the jump matrix. It can be shown by elementary calculations that (\ref{e def m^1 delta}) yields a solution of the following \rh.
\begin{samepage}
\begin{framed}
    \begin{rhp}\label{rhp m^1 delta}
        Find for each $x\in\Real$ a $2\times 2$-matrix valued function $\Compl\ni z\mapsto m_{\delta}^{(1)}(z;x)$ which satisfies
        \begin{enumerate}[(i)]
          \item $m_{\delta}^{(1)}(z;x)$ is meromorphic in $\Compl\setminus\Real$ (with respect to the parameter $z$).
          \item $m_{\delta}^{(1)}(z;x)=1+\mathcal{O}\left(\frac{1}{z}\right)$ as $|z|\to\infty$.
          \item The non-tangential boundary values $m^{(1)}_{\pm,\delta}(z;x)$ exist for $z\in\Real$ and satisfy the jump relation
              \begin{equation}\label{e jump m^1 delta}
                  m^{(1)}_{+,\delta}=m^{(1)}_{-,\delta} (1+R^{(1)}_{\delta}), \quad\text{where}\quad
                  R^{(1)}_{\delta}(z;x):=
                  \left[
                    \begin{array}{cc}
                       0 & e^{-2\ii zx}\overline{r}^{(1)}_{+,\delta}(z) \\
                       e^{2\ii zx}r^{(1)}_{-,\delta}(z) & \overline{r}^{(1)}_{+,\delta}(z) r_{-,\delta}^{(1)}(z) \\
                    \end{array}
                  \right],
              \end{equation}
              and $r^{(1)}_{\pm,\delta}(z):=r^{(1)}_{\pm} (z)\overline{\delta}_+(z) \overline{\delta}_-(z)\left( \frac{z-z_1}{z-\overline{z}_1}\right)^2$.
          \item $m_{\delta}^{(1)}$ has simple poles at $z_1$ and $\overline{z}_1$ with
              \begin{equation*}\label{e Res m^1 delta}
                  \begin{aligned}
                     \res_{z=z_1} m_{\delta}^{(1)}(z;x)&=\lim_{z\to z_1}m_{\delta}^{(1)}(z;x)
                  \left[
                      \begin{array}{cc}
                        0 & \frac{-e^{-2\ii z_1x} [2 \im(z_1)]^2} {\delta^{-2}(z_1)2\ii\lambda_1c_1} \\
                        0 & 0
                      \end{array}
                  \right],\\
                     \res_{z=\overline{z}_1} m_{\delta}^{(1)}(z;x)&=\lim_{z\to \overline{z}_1}m_{\delta}^{(1)}(z;x)
                  \left[
                      \begin{array}{cc}
                      0 & 0 \\
                      \frac{2\ii\lambda_1[2 \im(z_1)]^2e^{2\ii \overline{z}_1x}} {\delta^{2}(\overline{z}_1) \overline{c}_1} & 0
                      \end{array}
                  \right].
                  \end{aligned}
              \end{equation*}
        \end{enumerate}
    \end{rhp}
\end{framed}
\end{samepage}
Analogously to the previous subsection we set
\begin{equation*}
    r^{(0)}_{\pm,\delta}(z):= r^{(1 )}_{\pm,\delta}(z) \frac{z-\overline{z}_1}{z-z_1}= r^{(1)}_{\pm} (z)\overline{\delta}_+(z) \overline{\delta}_-(z) \frac{z-z_1}{z-\overline{z}_1},
\end{equation*}
and define $m^{(0)}_{\delta}(z;x)$ to be the unique solution
of \rh\ref{rhp m^0 delta} with data $\mathcal{S}^{(0)}_{\delta}:=\set{r^{(0)}_{\pm,\delta}}$. We have $r^{(0)}_{\pm,\delta}\in L^{2,1}\cap H^1$ and hence, the statements of Lemma \ref{l m(z0)-1 in in H^1 and L^2,1 (delta)} and Corollary \ref{c u in H^11 and H^2 (negative half line)} are available. Next we want to describe how the solutions $m^{(1)}_{\delta}(z;x)$ and $m^{(0)}_{\delta}(z;x)$ are connected by a B\"{a}cklund transformation of the form (\ref{e B�cklund for m^1}). For this purpose we define
\begin{align*}
        &\left(
          \begin{array}{c}
            \vspace{1mm} a^{(\delta)}_{11}(x) \\
            a^{(\delta)}_{21}(x) \\
          \end{array}
        \right)
        :=m^{(0)}_{\delta}(\overline{z}_1;x)
        \left(
          \begin{array}{c}
            1 \\
            \frac{2\ii\lambda_1[2 \im(z_1)]^2e^{2\ii \overline{z}_1x}} {\delta^{2}(\overline{z}_1) \overline{c}_1} \\
          \end{array}
        \right),\\
        &\left(
          \begin{array}{c}
            \vspace{1mm} a_{12}^{(\delta)}(x) \\
            a_{22}^{(\delta)}(x) \\
          \end{array}
        \right)
        :=m^{(0)}_{\delta}(z_1;x)
        \left(
          \begin{array}{c}
            \frac{-e^{-2\ii z_1x} [2 \im(z_1)]^2} {\delta^{-2}(z_1)2\ii\lambda_1c_1} \\
            1 \\
          \end{array}
        \right),
\end{align*}
and $A^{(\delta)}(x)=
    \left[
      \begin{array}{cc}
        a^{(\delta)}_{11}(x) & a^{(\delta)}_{12}(x) \\
        a^{(\delta)}_{21}(x) & a^{(\delta)}_{22}(x) \\
      \end{array}
    \right]$. It turns out that
\begin{equation}\label{e B�cklund for m^1 delta}
    m^{(1)}_{\delta}(z;x)=A^{(\delta)}(x)
    \left[
      \begin{array}{cc}
        z-\overline{z}_1 & 0 \\
        0 & z-z_1 \\
      \end{array}
    \right]
    \left[A^{(\delta)}(x)\right]^{-1}m^{(0)}_{\delta}(z;x)
    \left[
      \begin{array}{cc}
        \frac{1}{z-\overline{z}_1} & 0 \\
        0 & \frac{1}{z-z_1} \\
      \end{array}
    \right].
\end{equation}
Due to $\im(z_1)>0$ we have $e^{-2\ii z_1x}\in H^{1}_x(\Real_-)\cap L^{2,1}_x(\Real_-)$. Additionally, $(m^{(0)}_{\delta}(z_1;\cdot)-1)\in H^1(\Real_-)\cap L^{2,1}(\Real_-)$ and thus we find $(A^{(\delta)}(\cdot)-1)\in H^1(\Real_-)\cap L^{2,1}(\Real_-)$. These observation bring us in the position to extend the results of the previous subsection to the negative half-line.
\begin{cor}\label{c m^1(z0)-1 in L^21 and H^1 (delta)}
   Let the assumptions of Lemma \ref{l solvability of RHP N=1} be valid and fix $z_2\in\Compl\setminus(\Real\cup \set{z_1,\overline{z}_1})$. Then for the solution $m_{\delta}^{(1)}(z;x)$ of \rh\ref{rhp m^1 delta} we have $m_{\delta}^{(1)}(z_2;\cdot)-1\in H^1(\Real_-)\cap L^{2,1}(\Real_-)$. Moreover, if $\|r^{(1)}_+\|_{H^1\cap L^{2,1}}+\|r^{(1)}_-\|_{H^1\cap L^{2,1}}+|c_1|\leq M$ for some fixed $M>0$, then we also have the bound
   \begin{equation}\label{e m^1_delta(z0)-1 in H^1 and L^2,1}
       \|m_{\delta}^{(1)}(z_2;\cdot)-1\|_{H^1(\Real_-)\cap L^{2,1}(\Real_-)}\leq C_M,
   \end{equation}
   where the constant $C_M>0$ depends on $M$, $z_1$ and $z_2$ but not on $r^{(1)}_{\pm}$ and $|c_1|$.
\end{cor}
\begin{cor}\label{c u in H^11 and H^2 (negative half line - 1 pole)}
   Under the assumptions of Lemma \ref{l solvability of RHP N=1} the potential $u_{\delta}^{(1)}(x)$ reconstructed from the solution $m_{\delta}^{(1)}(z;x)$ of \rh\ref{rhp m^1 delta} by using (\ref{e rec 1}) and (\ref{e rec 2}) lies in $H^2(\Real_-)\cap H^{1,1}(\Real_-)$. Moreover, if $\|r^{(1)}_+\|_{H^1\cap L^{2,1}}+\|r^{(1)}_-\|_{H^1\cap L^{2,1}}+|c_1|\leq M$ for some fixed $M>0$, then we also have the bound
   \begin{equation}\label{e u^1 in H^11 and H^2 (negative half line)}
       \|u_{\delta}^{(1)}\|_{H^2(\Real_-)\cap H^{1,1}(\Real_-)}\leq C_M
   \end{equation}
   where the constant $C_M>0$ depends on $M$ and $z_1$, but not on $r^{(1)}_{\pm}$ and $|c_1|$.
\end{cor}
We finish the section with the following observation which is obvious from the definition:
\begin{equation*}
    \begin{aligned}
        \lim_{|z|\to\infty}z\;[m^{(1)}(z;x)]_{12}= \lim_{|z|\to\infty}z\;[m^{(1)}_{\delta}(z;x)]_{12},\\
        \lim_{|z|\to\infty}z\;[m^{(1)}(z;x)]_{21}= \lim_{|z|\to\infty}z\;[m^{(1)}_{\delta}(z;x)]_{21}.
    \end{aligned}
\end{equation*}
It follows that $u^{(1)}_{\delta}=u^{(1)}$. In conclusion the Corollaries \ref{c u in H^11 and H^2 (positive half line - 1 pole)} and \ref{c u in H^11 and H^2 (negative half line - 1 pole)} yield the existence of the mapping
\begin{equation}\label{e u in H^11 and H^2}
    H^1(\Real)\cap L^{2,1}(\Real)\ni (r^{(1)}_-,r^{(1)}_+)\mapsto u^{(1)}\in H^2(\Real)\cap H^{1,1}(\Real).
\end{equation}
 
\section{Influence in Completely Bounded Block-multilinear Forms}
\label{sec:proof}
\newcommand{\blocks}{\mathrm{blocks}}
\newcommand{\lt}{\mathrm{left}}
\newcommand{\rt}{\mathrm{right}}



In this section we prove the non-commutative root-influence inequality (\thmref{thm:bh-intro}),  the special case of the Aaronson-Ambainis conjecture given in \thmref{thm:aa}, and also briefly mention how the simulation result in \corref{cor:sim} follows from \thmref{thm:aa} and the results in \cite{AA14}. We first need some preliminaries from free probability theory. 



\subsection{Low-degree Polynomials of Haar Random Unitaries}

As discussed in the proof overview, we require bounds on the operator norm (as well as normalized trace) of low-degree polynomials of random unitaries and these follow from known results in free probability theory. Here we explain these connections and also prove some auxillary lemmas needed for the proof of \thmref{thm:bh-intro} and \thmref{thm:aa}. 



Let $z_{\ui}$ denote the non-commutative monomial $z_{i_1} z_{i_2} \cdots z_{i_d}$ for a $d$-tuple $\ui  = (i_1, \ldots, i_d) \in [t]^d$ and let $p(z_1, \ldots, z_t)$ be a non-commutative polynomial in the variables $z_1, \ldots, z_t$. We are interested in computing the operator norm $\|\cdot\|_{\op}$ and the normalized trace  $\tr_N$ of the polynomial $p(z_1, \ldots, z_t)$ (or its higher moments) when substituting $N \times N$ Haar random unitaries for the variables $z_i$.

As explained previously, the theory of free probability gives us tools that allow us to compute  the above in the limit $N \to \infty$. In particular, Voiculescu \cite{V98} showed that the  (normalized) trace of polynomials in Haar random unitaries and their conjugates converges to the trace of the same polynomial evaluated on certain infinite-dimensional operators called \emph{Haar unitaries} that satisfy a non-commutative notion of independence called \emph{free independence}. This was strengthened by Collins and Male \cite{CM11} who showed that such convergence also holds for the operator norm. A short primer on free probability is given in \appref{sec:free}, but for now one can think of $\CA$ as a self-adjoint algebra of bounded linear operators on a Hilbert space and $\phi$ as a trace functional for such operators in the statement given below.


\begin{theorem}[\cite{V98, CM11}] \label{thm:voiculescu}
    Let $p(z_1, \ldots, z_{2t})$ be a non-commutative polynomial in $\BR\langle z_1, \ldots, z_{2t}\rangle$. If $U_1, \ldots, U_t$ are $N \times N$ Haar random unitaries, then almost surely,
    \begin{align*}
     \ \tr_N[p(U_1, \ldots, U_t, U^*_1, \ldots, U_t^*)] &~\xrightarrow[N \to \infty]{}~ \phi[p(u_1, \ldots, u_t, u^*_1, \ldots, u^*_t)],\\
    \  \|p(U_1, \ldots, U_t, U^*_1, \ldots, U_t^*)\|_{\op} &~\xrightarrow[N \to \infty]{}~ \| p(u_1, \ldots, u_t, u^*_1, \ldots, u^*_t)\|,
    \end{align*}
    where $u_1, \ldots, u_t$ are free Haar unitaries in a $C^*$-probability space $(\CA, \phi)$ and $\|\cdot\|$ is the norm for the underlying $C^*$-algebra.
\end{theorem}




Using the above result it suffices to consider free Haar unitaries in a $C^*$-probability space to compute the operator norm and trace of polynomials of random unitaries. For a non-commutative polynomial $p(z_1, \ldots, z_t) = \sum_{|\ui|\le d} c_{\ui}z_{\ui}$, denoting by $\|p\|_2 =  \left(\sum_{|\ui| \le d} |c_{\ui}|^2\right)^{1/2}$, one can show the following easily using techniques from free probability. 

\begin{lemma} \label{thm:trace}
    Let $p(z_1, \ldots, z_t) = \sum_{|\ui|\le d} c_{\ui}z_{\ui} $ be a non-commutative degree-$d$ polynomial in $\R\langle z_1, \ldots, z_t\rangle$ and $u_1, \ldots, u_t$ be free Haar unitaries in a $C^*$-probability space $(\CA, \phi)$. Then, 
     \[ \phi[p(u_1, \ldots, u_t) (p(u_1, \ldots, u_t))^*] =  \|p\|_2^2.\]
\end{lemma}

The above implies that $\tr_N[p(U_1, \ldots, U_t) (p(U_1, \ldots, U_t))^*]$ converges to $\|p\|_2^2$ almost surely as $N \to \infty$. We shall defer the proof of \lref{thm:trace} to \appref{sec:app}, but to aid our intuition we note here that since the  $U_i$'s are independent $N \times N$ Haar random unitaries, the expected value

\[ \BE\left[\tr_N[p(U_1, \ldots, U_t) (p(U_1, \ldots, U_t))^*\right] = \|p\|_2^2,\] 
{and from concentration of measure, it is natural to expect that it converges to the above value}. 


Similarly, to compute the operator norm of $p(U_1, \ldots, U_t)$ for Haar random unitaries one can instead study the norm of the polynomial evaluated on free Haar unitaries. Such bounds are easier to prove using the trace method since free independence imposes strong restrictions on the non-commutative moments. For instance, if $U_1$ and $U_2$ are independent $N \times N$ Haar random matrices, then $\BE[\tr_N(U_1U_2U^*_1U_2^*)]$ is non-zero (albeit quite small), while the corresponding trace evaluated on free Haar unitaries $u_1$ and $u_2$ is zero, that is $\phi(u_1u_2u^*_1u_2^*) = 0$. Thus, computing the trace $\phi[p(u_1,\ldots, u_t, u^*_1, \ldots, u_t^*)]$ reduces to handling the combinatorics of the patterns of $u_i$'s and $u_i^*$'s. 

In particular, we will rely on the following result that follows from the work of Kemp and Speicher \cite{KS05}  who consider the operator norm of homogeneous polynomials evaluated on free $R$-diagonal operators, a class that includes free Haar unitaries as well. We also remark that a bound where the right-hand side below is worse by a multiplicative $O(d^{1/2})$ factor also follows from the work of Haagerup\footnote{We note that Haagerup considered the more general case of polynomials in both $u_i$'s and $u^*_i$'s.}\cite{H78} who proved it in another context, predating even the introduction of free probability theory. 


\begin{theorem}[\cite{KS05}]
\label{thm:kemp-speicher}
    Let $p(z_1, \ldots, z_t) = \sum_{|\ui| = d} c_{\ui}z_{\ui} $ be a homogeneous non-commutative degree-$d$ polynomial in $\R\langle z_1, \ldots, z_t\rangle$ and $u_1, \ldots, u_t$ be free Haar unitaries in a $C^*$-probability space. Then, 
    \[ 
    \|p(u_1, \ldots, u_t)\| \le \sqrt{e(d+1)} \cdot \|p\|_2,
    \]
    where the left-hand side denotes the norm in the underlying $C^*$-algebra. 
\end{theorem}

For completeness, we  introduce the necessary free probability background and some combinatorial details in \appref{sec:app}, and we present the fairly short proof of \thmref{thm:kemp-speicher} (from \cite{KS05}) there in a self-contained way. We shall need to extend the above bound to non-homogeneous polynomials. Let $p(z_1, \ldots, z_t) = \sum_{|\ui| \le d} c_{\ui}z_{\ui}$ and  let $p_k(z_1, \ldots, z_t) = \sum_{|\ui| = k} c_{\ui}z_{\ui}$ denote the degree-$k$ homogeneous part of $p$. Writing $p_k = p_k(u_1, \ldots, u_t)$ for $0 \le k  \le d$ and $p = p(u_1, \ldots, u_t)$, it follows from the triangle inequality,  \thmref{thm:kemp-speicher}, and Cauchy-Schwarz, that
    \begin{align*}
        \ \|p\| &\le \sum_{k=0}^d \|p_k\| 
        \le 
        \sum_{k=0}^d\sqrt{e(k+1)}\|p_k\|_2
        \le
       \sqrt{e}\left(\sum_{k=0}^d (k+1)\right)^{1/2} \left(\sum_{k=0}^d  \|p_k\|^2_2\right)^{1/2} \leq \sqrt{e}(d+1)  \cdot\|p\|_2.
    \end{align*}
Thus, we essentially get the same bound as in the homogeneous case, at the expense of an additional $O(d^{1/2})$ factor.



Collecting all the above we have the following as a direct consequence:

\begin{theorem} \label{thm:op-norm}
    Let $p(z_1, \ldots, z_t) = \sum_{|\ui|\le d} c_{\ui}z_{\ui} $ be a non-commutative degree-$d$ polynomial in $\R\langle z_1, \ldots, z_t\rangle$ and $U_1, \ldots, U_t$ be independent $N \times N$ Haar random unitaries. Then, as $N \to \infty$, the following holds almost surely, 
    \[ \tr_N[p(U_1, \ldots, U_t) (p(U_1, \ldots, U_t))^*] =  \|p\|_2^2,\]
    and
    \[ \|p(U_1, \ldots, U_t)\|_{\op} \le \sqrt{e}(d+1)  \cdot \|p\|_2,\]
    Moreover, the factor $(d+1)$ in the operator norm bound can be improved to $\sqrt{d+1}$ if the polynomial is homogeneous.
\end{theorem}

Based on the above theorem, we prove the following key lemma which captures the polar decomposition strategy mentioned in the earlier proof overview (\secref{sec:bh}). This will serve as the key ingredient in the proof of \thmref{thm:aa} and \thmref{thm:bh-intro}. 

\begin{lemma}\label{lem:polar}
    Let $p$ be a non-commutative degree-$d$ polynomial in $\R\langle y_1, \ldots, y_m, z_1, \ldots, z_t\rangle$ given by
    \[ p(y_1, \ldots, y_m, z_1, \ldots, z_t) = \sum_{i=1}^m y_i q_i(z_1, \ldots, z_t) + q_0(z_1, \ldots, z_t).\]
    Then, for every $\delta > 0$, there exist an integer $N$ and $N \times N$ unitaries $V_1,\ldots, V_m, W_1, \ldots, W_t$ such that 
    \[ \|p(V_1, \ldots, V_m, W_1, \ldots, W_t)\|_{\op} \ge \frac{1}{\sqrt{e}(d+1)} \sum_{i=1}^m \|q_i\|_2 - \delta.\]
    Moreover, the factor in front can be improved to $(e(d+1))^{-1/2}$ if $p$ is homogeneous. 
\end{lemma}

\begin{proof}[Proof of \lref{lem:polar}]
     For an arbitrary integer $N$, let us pick independent $N \times N$ Haar random unitaries $W_1, \ldots, W_t$ which we substitute for the variables $z_1,\ldots,z_t$, respectively, and let $M_i = q_i(W_1, \ldots, W_t)$ be the corresponding random matrices. Then, for any tuple of matrices $V_1, \ldots, V_m$ that we could substitute for the variables $y_1, \ldots, y_m$, we have that 
    \[ 
    p(V_1, \ldots, V_m, W_1, \ldots, W_t) = \sum_{i=1}^m V_i M_i + M_0.
    \] 
     \thmref{thm:op-norm} and union bound imply that as $N \to \infty$, with probability $1$ all the following events simultaneously hold: 
    \begin{itemize}
        \item $\|M_i\|_{\op} \le \sqrt{e}(d+1) \cdot \|q_i\|_2$ for each $i$,
        \item $\tr_N(M^*_iM_i) = \|q_i\|_2^2$ for each $i$, where $\tr_N(M)$ is the normalized trace.
    \end{itemize}
   To show that the operator norm must be large, let us fix a sufficiently large $N$ and a choice of $N\times N$ unitaries $W_1, \ldots, W_t$ such that $M_i$ satisfies $\|M_i\|_{\op} \le \sqrt{e}(d+1) \cdot \|q_i\|_2 + \epsilon$ and $\tr_N(M^*_iM_i) \ge \|q_i\|_2^2 - \epsilon$ for each $0\le i\le m$, where $\epsilon$ can be made arbitrarily small by increasing $N$. For $0 \leq i \leq m$, let $M_i = U_i P_i$ be the left polar decomposition of $M_i$, where $U_i$ is a unitary matrix and $P_i$ is a positive semidefinite matrix.
   
   We select the tuple of unitary matrices $V_1, \ldots, V_m$ that we substitute for the variables $y_1, \ldots, y_m$ to be $V_i = U_0U^*_i$ for $i \in [m]$. With this we have that $\|p(V_1, \ldots, V_m, W_1, \ldots, W_t)\|_{\op}$ is at least
    \begin{align*}
         \Big\|M_0 + \sum_{i=1}^m V_iM_i\Big\|_{\op} & = \Big\|U_0 P_0 + \sum_{i=1}^m U_0 U_i^* U_iP_i \Big\|_{\op} \\
        \ & =  \Big\|U_0 P_0 + \sum_{i=1}^m U_0 P_i\Big\|_{\op}  = \Big\| P_0 + \sum_{i=1}^m  P_i\Big\|_{\op}\ge \tr_N\Big(P_0 + \sum_{i=1}^m P_i\Big) \ge \tr_N\Big(\sum_{i=1}^m P_i\Big),
    \end{align*}
    where the last equality follows since the operator norm is unitarily invariant and the last two inequalities follow from the positive semidefiniteness of the $P_i$'s.

    For every positive semidefinite matrix $P$, we have that $\tr_N(P) \ge {\tr_N(P^2)}/{\|P\|_{\op}}$. 
  
    Hence,
     \[ \|p(V_1, \ldots, V_m, W_1, \ldots, W_t)\|_{\op} \ge \sum_{i=1}^m \frac{\tr_N(P_i^2)}{\|P_i\|_{\op}}.\]
     By our choice of $M_i$, we have that $\tr_N(P_i^2) = \tr_N(M_i^* M_i) \ge \|q_i\|_2^2 - \eps$ and $\|P_i\|_{\op} = \|M_i\|_{\op} \le \sqrt{e}(d+1)\|q_i\|_2 + \eps$. Since $\eps$ can be made arbitrarily small by increasing $N$, it follows that 
      \[ \|p(V_1, \ldots, V_m, W_1, \ldots, W_t)\|_{\op} \ge \frac1{\sqrt{e}(d+1)} \sum_{i=1}^m \|q_i\|_2 - \delta ,\]
     for large enough $N$. The improved bound for the homogeneous case follows directly by plugging the bound of \thmref{thm:op-norm} into the above proof.
\end{proof}





\subsection{Non-commutative root-influence inequality}
\label{sec:bh-proof}


For clarity in the proofs below, we remind our  convention that all tuples or blocks are denoted with boldface fonts (e.g. $\BU_1$ or $\BA$), while a single element is denoted without boldface (e.g. $U_1(i)$ or $A_i$ or $A$). Before proceeding with the proof, we restate the statement for convenience.

\bh*





\begin{proof}[Proof of \thmref{thm:bh-intro}] 
Since $f$ is homogeneous, we can write
   \begin{align*}
    f(\x_1,\ldots, \x_d) &= \sum_{i_1, \ldots, i_d \in [n]} \hf_{i_1, \ldots, i_d} ~x_1(i_1)x_2(i_2)\cdots x_d({i_d}) \\
    \ & = \sum_{i=1}^n  x_1(i) \underbrace{\left(\sum_{i_2,\ldots, i_d \in [n]} \hf_{i_1, \ldots, i_d} ~x_2(i_2)\cdots x_d({i_d})\right)}_{\textstyle := f_i(\x_2,\ldots, \x_d)}.
\end{align*}
 In this case, it follows from \eqref{eqn:inf-tensor} that for each $i \in [n]$, we have 
 \begin{equation}\label{eqn:var}
     \ \Var[f_i] = \|f_i\|^2_2 = \inf_{1,i}(f) \text{ and }  \Var[f] = \sum_{i=1}^n \inf_{1,i}(f).
 \end{equation}

  Let us denote the corresponding non-commutative block-multilinear polynomials by $f(\BU_1, \ldots, \BU_d)$ and $f_i(\BU_2, \ldots,\BU_d)$ where $\BU_b = (U_b(1), \ldots, U_b(n))$ denotes the $b^\text{th}$ block of non-commutative variables. To show a lower bound on $\cbnorm{f}$ it suffices to exhibit a collection of square matrices $\{U_b(i)\}_{b\in [d], i \in [n]}$ with operator norm at most~1, such that $\|f(\BU_1, \ldots, \BU_d)\|_{\op}$ is large. 
  
%  

Applying \lref{lem:polar} for the homogeneous case (with $p = f$, $q_i=f_i$ for $i \in [n]$, and $q_0=0)$, it follows that for every $\delta > 0$ there exists an integer $N$ and a choice of tuples of $N \times N$ unitaries $\BU_1, \ldots, \BU_d$ such that  
      \[ \cbnorm{f} \ge \|f(\BU_1, \ldots, \BU_d)\|_{\op} \ge \frac1{\sqrt{e(d+1)}} \sum_{i\in [n]} \|f_i\|_2  -\delta \stackrel{\eqref{eqn:var}}{\ge}  \frac{1}{\sqrt{e(d+1)}} \left(\sum_{i=1}^n \sqrt{\Inf_{1,i}(f)} \right) -\delta.\]
Taking $\delta \to 0$, we get the statement of the lemma. The proof for the inequality when $b=d$ is the last block follows similarly by using the right polar decomposition.
\end{proof}

\subsection{Aaronson-Ambainis Conjecture for non-homogeneous forms}

In this section, we prove \thmref{thm:aa}, which requires handling non-homogeneous forms. The proof will be similar to the proof of \thmref{thm:bh-intro} but we will need to be careful about certain details. 

\begin{proof}[Proof of \thmref{thm:aa}]
Any block-multilinear polynomial $f(x_1, \ldots, x_d)$ can be written as 
\begin{align*}
    f(\x_1,\ldots, \x_d) &= \BE f + \sum_{b\in [d]} f_b(\x_b, \x_{b+1}, \ldots, \x_d),
\end{align*}
where $f_b$ consists of all monomials of $f$ that start with a variable in the $b^\text{th}$ block $\x_b$. Note that $f_b$ depends only on the variables in blocks $\x_b, \x_{b+1},\ldots, \x_d$. Moreover, it follows from \eqref{eqn:inf-tensor} that 
 \begin{equation}\label{eqn:var-general}
     \ \Var[f] = \sum_{b \in [d]} \|f_b\|_2^2 = \sum_{b \in [d]} \Var[f_b],
 \end{equation}
so there exists a block $\beta \in [d]$ such that $\Var[f_{\beta}] \ge \frac{1}{d}\Var[f]$. 

Since $f_{\beta}$ contributes a lot to the variance, it is natural to try to find an influential variable in the block $\x_{\beta}$. Towards this end,  we pull out the variables $x_{\beta}(i)$ and write
\begin{align*}
    f_{\beta}(\x_{\beta},\ldots, \x_d) &= \sum_{i\in [n]} x_{\beta}(i) f_{\beta,i}(\x_{\beta+1}, \ldots, \x_d),
\end{align*}
for block-multilinear polynomials $f_{\beta,i}(\x_{\beta+1}, \ldots, \x_d)$. Note that some of the $f_{\beta,i}$'s could be identically zero, so let us define $S$ to be the set of those $i$ such that $f_{\beta,i}$ is non-zero. We note that
\begin{align} \label{eqn:part-inf}
  \|f_{\beta,i}\|_2^2  =  \Inf_{\beta,i}(f_{\beta}) \le \Inf_{\beta,i}(f)  
\end{align}
which implies that
\begin{align}\label{eqn:var-main}
    \frac{1}{d} \Var[f] \le \Var[f_{\beta}] = \sum_{i \in S}\|f_{\beta,i}\|_2^2 = \sum_{i \in S} \Inf_{\beta,i}(f_{\beta}).
\end{align}
\begin{sloppypar}
Denote the corresponding non-commutative block-multilinear polynomials by $f(\BU_1, \ldots, \BU_d)$,  $f_b(\BU_{b}, \ldots,\BU_d)$, and $f_{\beta}(\BU_{\beta+1}, \ldots,\BU_d)$ where $\BU_b = (U_b(1), \ldots, U_b(n))$ denotes the $b^\text{th}$ block of non-commutative variables. To show a lower bound on $\cbnorm{f}$ it suffices to exhibit a collection of square matrices $\{U_b(i)\}_{b\in [d], i \in [n]}$ with operator norm at most~1 such that $\|f(\BU_1, \ldots, \BU_d)\|_{\op}$ is large.
\end{sloppypar}
  
 We set the matrices in blocks $\BU_1, \ldots, \BU_{\beta-1}$ to be zero (that is, the all-zero matrix $\BZ$). Note that with this choice all polynomials $f_b(\U_b, \ldots, \U_d)$ where $b < \beta$ vanish and the non-commutative polynomial becomes 
 \[ f(\BZ, \ldots, \BZ, \BU_{\beta}, \BU_{\beta+1}, \ldots, \BU_d) = \sum_{i\in S} U_{\beta}(i) f_{\beta,i}(\BU_{\beta+1}, \ldots, \BU_d) + \sum_{b=\beta+1}^d f_b(\BU_b, \BU_{b+1}, \ldots, \BU_d) + \Ef,\]
  which is a non-commutative polynomial of the form considered in \lref{lem:polar} (with $m = |S|$, $q_i = f_{\beta,i}$ and $q_0 = \sum_{b=\beta+1}^d f_b + \Ef$). Thus, by \lref{lem:polar} for every small $\delta>0$ there exists an integer $N$ and a choice of $N \times N$ matrices for the blocks $\BU_{\beta},\ldots, \BU_d$ such that 
        \begin{align*}
             \ \cbnorm{f} & \ge \|f(\BZ, \ldots, \BZ, \BU_{\beta}, \BU_{\beta+1}, \ldots, \BU_d)\|_{\op} & \\
             \  & \ge \frac1{\sqrt{e}(d+1)} \sum_{i\in S} \|f_{\beta,i}\|_2 -\delta  \stackrel{\eqref{eqn:part-inf}}{=}  \frac{1}{\sqrt{e}(d+1)} \left(\sum_{i \in S} \sqrt{\Inf_{\beta,i}(f_{\beta})} \right) -\delta & \\
             \ &\stackrel{\eqref{eqn:var-main}}{\ge}  \frac{1}{\sqrt{e}(d+1)} \left( \frac{\sum_{i \in S} \Inf_{\beta,i}(f_{\beta})}{\sqrt{\maxinf(f)}} \right) -\delta  \stackrel{\eqref{eqn:part-inf}}{\ge}  \frac{1}{\sqrt{e}(d+1)^{2}} \left( \frac{\Var[f]}{ \sqrt{\maxinf(f)}} \right) -\delta
        \end{align*}
        Taking $\delta \to 0$ and using the assumption that $\|f\|_{\cb} \le 1$, we obtain the statement of the theorem:
     \[
     1\geq \cbnorm{f} \ge \frac{1}{\sqrt{e}(d+1)^{2}} \cdot \frac{\Var[f]}{\sqrt{\maxinf(f)}} \implies \maxinf(f) \ge  \frac{(\Var[f])^2}{e(d+1)^4}. \qedhere
     \]
\end{proof}
 

     
     

\subsection{Approximating completely bounded forms with decision trees}



In this section, we briefly mention how to obtain \corref{cor:sim}.
Aaronson and Ambainis \cite[Theorem 3.3]{AA14} showed that querying the most influential variable reduces the variance of the function~$f$, and if that influence is lower bounded by a polynomial in $\Var[f]/d$, then after $\poly(d)$ queries (the exact quantitative dependence can be read off from their proof), the variance of the function becomes small enough so that it can be approximated almost-everywhere by its expectation.  Since the family of degree-$d$ block-multilinear forms with completely bounded norm at most one is closed under restrictions, one can apply \thmref{thm:aa} repeatedly. This gives us \corref{cor:sim}.

\bibliographystyle{alpha}
\bibliography{lit}
\end{document} 