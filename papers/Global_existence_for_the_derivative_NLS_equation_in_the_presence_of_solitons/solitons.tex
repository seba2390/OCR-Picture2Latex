\section{Solitons}\label{s solitons}
This section is devoted to the exact solitary wave solutions of the DNLS equation (\ref{e dnls}) which are known since the 1970s (see, e.g., \cite{mjlhus1976} and \cite{KaupNewell1978}). Also more recent works are concerned with solitons. See for instance \cite{Colin2006}, where orbital stability of solitons is shown. The inverse scattering machinery admits a simple definition of $N$-solitons:
\begin{defn}
    (Global) solutions $u^{(N\text{-sol})}(x,t)$ of (\ref{e dnls}) such that the initial datum $u^{(N\text{-sol})}(\cdot,0)$ produces scattering data
    \begin{equation*}
        \mathcal{S} (u^{(N\text{-sol})})=\set{r_+\equiv r_-\equiv 0;\lambda_1,... ,\lambda_N;c_1,...,c_N},
    \end{equation*}
    are called $N$-\emph{solitons}. For $N=1$ we just say \emph{soliton}.
\end{defn}
In the case of $r_+\equiv r_-\equiv 0,$ the Riemann--Hilbert problem \rh \ref{rhp m dynamic} reads as follows:
\begin{samepage}
\begin{framed}
\begin{rhp}\label{rhp N sol}
Find for each $x\in\Real$ a $2\times 2$-matrix valued function $\Compl\ni z\mapsto m^{(N\text{-sol})} (z;x,t)$ which satisfies
\begin{enumerate}[(i)]
  \item $m^{(N\text{-sol})} (z;x,t)$ is meromorphic in $\Compl$ (with respect to the parameter $z$).
  \item $m^{(N\text{-sol})} (z;x,t)=1+\mathcal{O}\left(\frac{1}{z}\right)$ as $|z|\to\infty$.
  \item $m^{(N\text{-sol})} $ has simple poles at $z_1,...,z_N,\overline{z}_1,...,\overline{z}_N$ with
      \begin{equation*}%\label{e Res}
        \begin{aligned}
          \res_{z=z_k}m^{(N\text{-sol})} (z;x,t)&=\lim_{z\to z_k}m^{(N\text{-sol})} (z;x,t)
          \left[
            \begin{array}{cc}
              0 & 0 \\
              2\ii\lambda_kc_k e^{2\ii xz_k+4\ii t z_k^2} & 0
            \end{array}
          \right],\\
          \res_{z=\overline{z}_k}m^{(N\text{-sol})} (z;x,t)&=\lim_{z\to \overline{z}_k}m^{(N\text{-sol})} (z;x,t)
          \left[
            \begin{array}{cc}
              0 & \frac{-\overline{c}_k}{2\ii\lambda_k} e^{-2\ii x\overline{z}_k-4\ii t \overline{z}_k^2} \\
              0 & 0
            \end{array}
          \right].
        \end{aligned}
      \end{equation*}
\end{enumerate}
\end{rhp}
\end{framed}
\end{samepage}
Using the ansatz
\begin{equation*}
    m^{(N\text{-sol})} (z;x,t)=1+\sum_{k=1}^N\left\{ \frac{A_k(x,t)}{z-z_k}+ \frac{B_k(x,t)}{z-\overline{z}_k}\right\}
\end{equation*}
we can transfer \rh \ref{rhp N sol} into a purely algebraic system which can be solved explicitly. Then, the reconstruction formulas (\ref{e rec 1}) and (\ref{e rec 2}) yield explicit solutions of the DNLS equation, which are (multi) solitons. For the special case $N=1$ we find
\begin{equation}\label{e soliton}
    u_{\omega,v,x_0,\gamma}(x,t)=\phi_{\omega,v}(x-vt-x_0) e^{-\ii\gamma+\ii\omega t+\ii\frac{v}{2}(x-vt)- \frac{3}{4}\ii\int_{\infty}^{x-vt-x_0} |\phi_{\omega,v}(y)|^2dy},
\end{equation}
where
\begin{equation}\label{e sol ampl}
    \phi_{\omega,v}(x)=\left[\frac{\sqrt{\omega}}{4\omega-v^2} \left\{\cosh(\sqrt{4\omega-v^2}x)- \frac{v}{2\sqrt{\omega}}\right\}\right]^{-1/2}.
\end{equation}
The parameters $(\omega,v)\in \Real^2$ describe the speed and the width of the soliton and  are connected to the pole $z_1$ by
\begin{equation}\label{e omega v}
        \omega=4|z_1|^2,\qquad
         v=-4\re(z_1).
\end{equation}
Note that $v^2<4\omega$ is automatically fulfilled if $z_1\in\Compl_+$. The norming constant $c_1$ influences only the phase and the spatial position of the soliton. To be precise we have
\begin{equation}\label{e x_0 gamma}
    x_0=2\ln\left[\frac{|c_1|}{2\im(z_1)}\right] \left(\sqrt{4\omega-v^2}\right)^{-1},
    \quad
    \gamma=\arg(c_1)+\frac{\pi}{2}+\frac{1}{2}\arg(z_1).
\end{equation}
Expressions for $N$-solitons with $N\geq2$ are large and not presented here. If $\re(z_j)
\neq\re(z_k)$ for $j\neq k$, then for large $|t|$, $N$-solitons break up into $N$  individual solitons of the form (\ref{e soliton}):
\begin{equation}\label{e sol sep}
    u^{(N\text{-sol})}(x,t)\sim \sum_{k=1}^N u_{\omega_k,v_k,x_{0,k}^{\pm},\gamma_k^{\pm}}(x,t), \quad\text{as }t\to\pm\infty.
\end{equation}
If the real parts of two poles $z_j$ and $z_k$ coincide, we obtain a solution having two peaks traveling at the same speed and the separation (\ref{e sol sep}) will not occur. Instead, \emph{breather} phenomena will appear. 