\begin{abstract}
   We prove the existence of global solutions to the DNLS equation with initial data in a large subset of $H^2(\Real)\cap H^{1,1}(\Real)$ containing a neighborhood of all solitons. We use the inverse scattering transform method, which was recently developed by D. Pelinovsky and Y. Shimabukuro, and an auto-B\"{a}cklund transform in order to include solitons.
\end{abstract}
%\begin{abstract}
%    The derivative nonlinear Schr\"{o}dinger (DNLS) equation $ iu_t+u_{xx}+i(|u|^2u)_x=0$ models the propagation of circular polarized nonlinear Alfv\'{e}n waves in plasmas.
%    A local well-posedness theory for initial data $u_0\in H^2(\mathbb{R})$ was developed in 1980 by M. Tsutsumi and I. Fukuda. However, global existence was obtained by these authors only for initial data $u_0$ small in the $H^1$ norm.
%    This year, D. Pelinovsky and Y. Shimabakuro constructed a global solution in $H^2(\mathbb{R})\cap H^{1,1}(\mathbb{R})$ under some spectral restrictions on the initial data. They were using the central property of DNLS, discovered in 1978 by D. Kaup and A. Newell, that it is solvable through the inverse scattering mehtod. In my talk I will explain some details of the work by Pelinovsky and Shimabakuro and, furthermore, I would like to present a result, where an auto-B\"{a}cklund transformation is used to enlarge the class of possible initial data.
%\end{abstract} 