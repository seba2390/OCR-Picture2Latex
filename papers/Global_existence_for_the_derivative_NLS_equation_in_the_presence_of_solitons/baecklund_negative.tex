\subsection{B\"{a}cklund transformation for $x<0$}
We consider the solution $m^{(1)}(z;x)$ of \rh \ref{rhp m^1} provided by Lemma \ref{l solvability of RHP N=1} and define
\begin{equation}\label{e def m^1 delta}
    m^{(1)}_{\delta}(z;x):=m^{(1)}(z;x)
    \left[
      \begin{array}{cc}
        \frac{z-z_1}{z-\overline{z}_1} & 0 \\
        0 & \frac{z-\overline{z}_1}{z-z_1} \\
      \end{array}
    \right]
    \left[
      \begin{array}{cc}
        \delta^{-1}(z) & 0 \\
        0 & \delta(z) \\
      \end{array}
    \right].
\end{equation}
The factor $\left(\frac{z-z_1}{z-\overline{z}_1}\right)^{\sigma_1}$ swaps the columns where the poles arise. The second factor $\delta^{-\sigma_1}$ has influence on the structure of the jump matrix. It can be shown by elementary calculations that (\ref{e def m^1 delta}) yields a solution of the following \rh.
\begin{samepage}
\begin{framed}
    \begin{rhp}\label{rhp m^1 delta}
        Find for each $x\in\Real$ a $2\times 2$-matrix valued function $\Compl\ni z\mapsto m_{\delta}^{(1)}(z;x)$ which satisfies
        \begin{enumerate}[(i)]
          \item $m_{\delta}^{(1)}(z;x)$ is meromorphic in $\Compl\setminus\Real$ (with respect to the parameter $z$).
          \item $m_{\delta}^{(1)}(z;x)=1+\mathcal{O}\left(\frac{1}{z}\right)$ as $|z|\to\infty$.
          \item The non-tangential boundary values $m^{(1)}_{\pm,\delta}(z;x)$ exist for $z\in\Real$ and satisfy the jump relation
              \begin{equation}\label{e jump m^1 delta}
                  m^{(1)}_{+,\delta}=m^{(1)}_{-,\delta} (1+R^{(1)}_{\delta}), \quad\text{where}\quad
                  R^{(1)}_{\delta}(z;x):=
                  \left[
                    \begin{array}{cc}
                       0 & e^{-2\ii zx}\overline{r}^{(1)}_{+,\delta}(z) \\
                       e^{2\ii zx}r^{(1)}_{-,\delta}(z) & \overline{r}^{(1)}_{+,\delta}(z) r_{-,\delta}^{(1)}(z) \\
                    \end{array}
                  \right],
              \end{equation}
              and $r^{(1)}_{\pm,\delta}(z):=r^{(1)}_{\pm} (z)\overline{\delta}_+(z) \overline{\delta}_-(z)\left( \frac{z-z_1}{z-\overline{z}_1}\right)^2$.
          \item $m_{\delta}^{(1)}$ has simple poles at $z_1$ and $\overline{z}_1$ with
              \begin{equation*}\label{e Res m^1 delta}
                  \begin{aligned}
                     \res_{z=z_1} m_{\delta}^{(1)}(z;x)&=\lim_{z\to z_1}m_{\delta}^{(1)}(z;x)
                  \left[
                      \begin{array}{cc}
                        0 & \frac{-e^{-2\ii z_1x} [2 \im(z_1)]^2} {\delta^{-2}(z_1)2\ii\lambda_1c_1} \\
                        0 & 0
                      \end{array}
                  \right],\\
                     \res_{z=\overline{z}_1} m_{\delta}^{(1)}(z;x)&=\lim_{z\to \overline{z}_1}m_{\delta}^{(1)}(z;x)
                  \left[
                      \begin{array}{cc}
                      0 & 0 \\
                      \frac{2\ii\lambda_1[2 \im(z_1)]^2e^{2\ii \overline{z}_1x}} {\delta^{2}(\overline{z}_1) \overline{c}_1} & 0
                      \end{array}
                  \right].
                  \end{aligned}
              \end{equation*}
        \end{enumerate}
    \end{rhp}
\end{framed}
\end{samepage}
Analogously to the previous subsection we set
\begin{equation*}
    r^{(0)}_{\pm,\delta}(z):= r^{(1 )}_{\pm,\delta}(z) \frac{z-\overline{z}_1}{z-z_1}= r^{(1)}_{\pm} (z)\overline{\delta}_+(z) \overline{\delta}_-(z) \frac{z-z_1}{z-\overline{z}_1},
\end{equation*}
and define $m^{(0)}_{\delta}(z;x)$ to be the unique solution
of \rh\ref{rhp m^0 delta} with data $\mathcal{S}^{(0)}_{\delta}:=\set{r^{(0)}_{\pm,\delta}}$. We have $r^{(0)}_{\pm,\delta}\in L^{2,1}\cap H^1$ and hence, the statements of Lemma \ref{l m(z0)-1 in in H^1 and L^2,1 (delta)} and Corollary \ref{c u in H^11 and H^2 (negative half line)} are available. Next we want to describe how the solutions $m^{(1)}_{\delta}(z;x)$ and $m^{(0)}_{\delta}(z;x)$ are connected by a B\"{a}cklund transformation of the form (\ref{e B�cklund for m^1}). For this purpose we define
\begin{align*}
        &\left(
          \begin{array}{c}
            \vspace{1mm} a^{(\delta)}_{11}(x) \\
            a^{(\delta)}_{21}(x) \\
          \end{array}
        \right)
        :=m^{(0)}_{\delta}(\overline{z}_1;x)
        \left(
          \begin{array}{c}
            1 \\
            \frac{2\ii\lambda_1[2 \im(z_1)]^2e^{2\ii \overline{z}_1x}} {\delta^{2}(\overline{z}_1) \overline{c}_1} \\
          \end{array}
        \right),\\
        &\left(
          \begin{array}{c}
            \vspace{1mm} a_{12}^{(\delta)}(x) \\
            a_{22}^{(\delta)}(x) \\
          \end{array}
        \right)
        :=m^{(0)}_{\delta}(z_1;x)
        \left(
          \begin{array}{c}
            \frac{-e^{-2\ii z_1x} [2 \im(z_1)]^2} {\delta^{-2}(z_1)2\ii\lambda_1c_1} \\
            1 \\
          \end{array}
        \right),
\end{align*}
and $A^{(\delta)}(x)=
    \left[
      \begin{array}{cc}
        a^{(\delta)}_{11}(x) & a^{(\delta)}_{12}(x) \\
        a^{(\delta)}_{21}(x) & a^{(\delta)}_{22}(x) \\
      \end{array}
    \right]$. It turns out that
\begin{equation}\label{e B�cklund for m^1 delta}
    m^{(1)}_{\delta}(z;x)=A^{(\delta)}(x)
    \left[
      \begin{array}{cc}
        z-\overline{z}_1 & 0 \\
        0 & z-z_1 \\
      \end{array}
    \right]
    \left[A^{(\delta)}(x)\right]^{-1}m^{(0)}_{\delta}(z;x)
    \left[
      \begin{array}{cc}
        \frac{1}{z-\overline{z}_1} & 0 \\
        0 & \frac{1}{z-z_1} \\
      \end{array}
    \right].
\end{equation}
Due to $\im(z_1)>0$ we have $e^{-2\ii z_1x}\in H^{1}_x(\Real_-)\cap L^{2,1}_x(\Real_-)$. Additionally, $(m^{(0)}_{\delta}(z_1;\cdot)-1)\in H^1(\Real_-)\cap L^{2,1}(\Real_-)$ and thus we find $(A^{(\delta)}(\cdot)-1)\in H^1(\Real_-)\cap L^{2,1}(\Real_-)$. These observation bring us in the position to extend the results of the previous subsection to the negative half-line.
\begin{cor}\label{c m^1(z0)-1 in L^21 and H^1 (delta)}
   Let the assumptions of Lemma \ref{l solvability of RHP N=1} be valid and fix $z_2\in\Compl\setminus(\Real\cup \set{z_1,\overline{z}_1})$. Then for the solution $m_{\delta}^{(1)}(z;x)$ of \rh\ref{rhp m^1 delta} we have $m_{\delta}^{(1)}(z_2;\cdot)-1\in H^1(\Real_-)\cap L^{2,1}(\Real_-)$. Moreover, if $\|r^{(1)}_+\|_{H^1\cap L^{2,1}}+\|r^{(1)}_-\|_{H^1\cap L^{2,1}}+|c_1|\leq M$ for some fixed $M>0$, then we also have the bound
   \begin{equation}\label{e m^1_delta(z0)-1 in H^1 and L^2,1}
       \|m_{\delta}^{(1)}(z_2;\cdot)-1\|_{H^1(\Real_-)\cap L^{2,1}(\Real_-)}\leq C_M,
   \end{equation}
   where the constant $C_M>0$ depends on $M$, $z_1$ and $z_2$ but not on $r^{(1)}_{\pm}$ and $|c_1|$.
\end{cor}
\begin{cor}\label{c u in H^11 and H^2 (negative half line - 1 pole)}
   Under the assumptions of Lemma \ref{l solvability of RHP N=1} the potential $u_{\delta}^{(1)}(x)$ reconstructed from the solution $m_{\delta}^{(1)}(z;x)$ of \rh\ref{rhp m^1 delta} by using (\ref{e rec 1}) and (\ref{e rec 2}) lies in $H^2(\Real_-)\cap H^{1,1}(\Real_-)$. Moreover, if $\|r^{(1)}_+\|_{H^1\cap L^{2,1}}+\|r^{(1)}_-\|_{H^1\cap L^{2,1}}+|c_1|\leq M$ for some fixed $M>0$, then we also have the bound
   \begin{equation}\label{e u^1 in H^11 and H^2 (negative half line)}
       \|u_{\delta}^{(1)}\|_{H^2(\Real_-)\cap H^{1,1}(\Real_-)}\leq C_M
   \end{equation}
   where the constant $C_M>0$ depends on $M$ and $z_1$, but not on $r^{(1)}_{\pm}$ and $|c_1|$.
\end{cor}
We finish the section with the following observation which is obvious from the definition:
\begin{equation*}
    \begin{aligned}
        \lim_{|z|\to\infty}z\;[m^{(1)}(z;x)]_{12}= \lim_{|z|\to\infty}z\;[m^{(1)}_{\delta}(z;x)]_{12},\\
        \lim_{|z|\to\infty}z\;[m^{(1)}(z;x)]_{21}= \lim_{|z|\to\infty}z\;[m^{(1)}_{\delta}(z;x)]_{21}.
    \end{aligned}
\end{equation*}
It follows that $u^{(1)}_{\delta}=u^{(1)}$. In conclusion the Corollaries \ref{c u in H^11 and H^2 (positive half line - 1 pole)} and \ref{c u in H^11 and H^2 (negative half line - 1 pole)} yield the existence of the mapping
\begin{equation}\label{e u in H^11 and H^2}
    H^1(\Real)\cap L^{2,1}(\Real)\ni (r^{(1)}_-,r^{(1)}_+)\mapsto u^{(1)}\in H^2(\Real)\cap H^{1,1}(\Real).
\end{equation}
