\subsection{Jost functions}
It is natural to introduce solutions of (\ref{e Lax1}) which satisfy the same asymptotic behavior at infinity as solutions of the spectral problem (\ref{e Lax1}) in the case of vanishing potential $u\equiv 0$:
\begin{eqnarray*}
% \nonumber to remove numbering (before each equation)
  \psi^{(-)}_1(\lambda;x)\sim
  \left(
    \begin{array}{c}
      1 \\
      0 \\
    \end{array}
  \right)e^{-\ii x\lambda^2}
  ,\qquad
  \psi^{(-)}_2(\lambda;x)\sim
  \left(
    \begin{array}{c}
      0 \\
      1 \\
    \end{array}
  \right)e^{\ii x\lambda^2}
  && \text{ as } x\to -\infty\\
  \psi^{(+)}_1(\lambda;x)\sim
  \left(
    \begin{array}{c}
      1 \\
      0 \\
    \end{array}
  \right)e^{-\ii x\lambda^2}
  ,\qquad
  \psi^{(+)}_2(\lambda;x)\sim
  \left(
    \begin{array}{c}
      0 \\
      1 \\
    \end{array}
  \right)e^{\ii x\lambda^2}
  && \text{ as } x\to+\infty.
\end{eqnarray*}
In order to have constant boundary conditions we introduce the \emph{normalized Jost functions} by
\begin{equation*}
    \varphi_{\pm}(\lambda;x)=\psi^{(\pm)}_1(\lambda;x) e^{\ii x\lambda^2},\qquad \phi_{\pm}(\lambda;x)=\psi^{(\pm)}_2(\lambda;x) e^{-\ii x\lambda^2},
\end{equation*}
such that we have
\begin{equation}\label{e asymptotics psi}
    \lim_{x\to\pm\infty}\varphi_{\pm}(\lambda;x)=e_1\quad\text{ and } \quad\lim_{x\to\pm\infty}\phi_{\pm}(\lambda;x)=e_2,
\end{equation}
where $e_1=(1,0)^T$ and $e_2=(0,1)^T$. The Jost functions are solutions of the following Volterra's integral equations
\begin{equation}\label{e volterra phi varphi}
    \begin{aligned}
        \varphi_{\pm}(\lambda;x)&=e_1 + \lambda \int_{\pm\infty}^x
        \left[
          \begin{array}{cc}
            1 & 0 \\
            0 & e^{2\ii\lambda^2(x-y)} \\
          \end{array}
        \right]
        Q(u(y))\varphi_{\pm}(\lambda;y)dy,\\
        \phi_{\pm}(\lambda;x)&=e_2 + \lambda \int_{\pm\infty}^x
        \left[
          \begin{array}{cc}
            e^{-2\ii\lambda^2(x-y)} & 0 \\
            0 & 1\\
          \end{array}
        \right]
        Q(u(y))\phi_{\pm}(\lambda;y)dy.
    \end{aligned}
\end{equation}
It can be shown that (\ref{e volterra phi varphi}) admit solutions $\varphi_-(\lambda;x)$ and $\phi_+(\lambda;x)$ for $\im(\lambda^2)>0$ and $\varphi_+(\lambda;x)$ and $\phi_-(\lambda;x)$ for $\im(\lambda^2)<0$. Moreover the dependence of $\lambda$ is analytic in the corresponding domains where the Jost functions exist.
However, due to the presence of $\lambda$ that multiplies the matrix $Q(u)$ in the linear equation (\ref{e Lax1}), standard fixed point arguments for (\ref{e volterra phi varphi}) are not uniform in $\lambda$. Therefore, in \cite{Pelinovsky2016} the authors worked out a transformation of the Kaup-Newell type spectral problem (\ref{e Lax1}) to a linear equation of the Zakharov-Shabat type. The idea of that kind of transformation can already be found in \cite{KaupNewell1978}. In what follows we are going to present this transformation and set
\begin{equation}\label{e def T1 Q1}
    T_1(\lambda;x)=
    \left[
      \begin{array}{cc}
        1 & 0 \\
        -\overline{u}(x) & 2\ii\lambda \\
      \end{array}
    \right],\qquad
    Q_1(u)=\frac{1}{2\ii}
    \left[
      \begin{array}{cc}
        |u|^2 & u \\
        -2\ii\overline{u}_x -\overline{u}|u|^2& -|u|^2 \\
      \end{array}
    \right],
\end{equation}
and
\begin{equation}\label{e def T2 Q2}
    T_2(\lambda;x)=
    \left[
      \begin{array}{cc}
        2\ii\lambda &-u(x) \\
        0 & 1 \\
      \end{array}
    \right],\qquad
    Q_2(u)=\frac{1}{2\ii}
    \left[
      \begin{array}{cc}
        |u|^2 & -2\ii u_x -u|u|^2 \\
        -\overline{u}& -|u|^2 \\
      \end{array}
    \right].
\end{equation}
Then, it is elementary to check that $z=\lambda^2$ and
\begin{equation}\label{e def M N}
    M_{\pm}(z;x)=T_1(\lambda;x)\varphi_{\pm} (\lambda;x),\quad
    N_{\pm}(z;x)=T_2(\lambda;x)\phi_{\pm}(\lambda;x)
\end{equation}
make (\ref{e volterra phi varphi}) equivalent to
\begin{equation}\label{e volterra M N}
    \begin{aligned}
        M_{\pm}(z;x)&=e_1 +  \int_{\pm\infty}^x
        \left[
          \begin{array}{cc}
            1 & 0 \\
            0 & e^{2\ii z(x-y)} \\
          \end{array}
        \right]
        Q_1(u(y))M_{\pm}(z;y)dy,\\
        N_{\pm}(z;x)&=e_2 +  \int_{\pm\infty}^x
        \left[
          \begin{array}{cc}
            e^{-2\ii z (x-y)} & 0 \\
            0 & 1\\
          \end{array}
        \right]
        Q_2(u(y))N_{\pm}(z;y)dy.
    \end{aligned}
\end{equation}
Note that the symmetries
\begin{equation}\label{e symmetries phi varphi 1}
    \varphi_{\pm}(\lambda;x)=
    \left[
      \begin{array}{cc}
        1 & 0 \\
        0 & -1 \\
      \end{array}
    \right]
    \varphi_{\pm}(-\lambda;x),\qquad
    \phi_{\pm}(\lambda;x)=
    \left[
      \begin{array}{cc}
        -1 & 0 \\
        0 & 1 \\
      \end{array}
    \right]
    \phi_{\pm}(-\lambda;x)
\end{equation}
make sure that (\ref{e def M N}) is well-defined.
Equations (\ref{e volterra M N}) are analogues to the integral equations known from the forward scattering for the NLS equation (see, e.g., \cite{Ablowitz2004}). If $Q_{1,2}(u)\in L^1(\Real)$, then, (\ref{e volterra phi varphi}) admit solutions $M_-(z;x)$ and $N_+(z;x)$ for $\im(z)>0$ and $M_+(z;x)$ and $N_-(z;x)$ for $\im(z)<0$. Moreover the dependence on $z$ is analytic in the corresponding domains where the Jost functions exist.
\begin{rem}
    The assumption $u\in H^{1,1}(\Real)$ in Theorem \ref{t main} is chosen such that $Q_{1,2}(u)\in L^1(\Real)$.
\end{rem}
Compared to (\ref{e volterra phi varphi}), in (\ref{e volterra M N}) there is no $\lambda$ which multiplies the integral. As a result, the Neumann series for (\ref{e volterra M N}) converge uniformly in $z$.
By means of the asymptotic expansion for large $z$ of the Jost functions, the potential $u$ can be reconstructed from $M_{\pm}$ and $N_{\pm}$, respectively (see \cite[Lemma 2]{Pelinovsky2016}). Furthermore, regularity properties of $M_{\pm}$ and $N_{\pm}$ are used in \cite{Pelinovsky2016} to prove regularity of the reflection coefficient $r_+$ and $r_-$ which we will define in (\ref{e def r pm}) in the next subsection on the Scattering data.
%Therefore we obtain similar properties. In what follows we list the results of Pelinovsky and Shimabukuru (see \cite{Pelinovsky2016}) without proofs:
%\begin{lem}\label{l existence of M N}
%    For $u\in L^1(\Real)\cap L^3(\Real)$ and $u_x\in L^1(\Real)$ and every $z\in\Real$, equations (\ref{e volterra M N}) admit unique solutions $M_{\pm}(z;\cdot)\in L^{\infty}(\Real)$ and $N_{\pm}(z;\cdot)\in L^{\infty}(\Real)$. Moreover, $M_+(\cdot;x)$ and $N_-(\cdot;x)$ are continued analytically in $\Compl^+$, whereas $M_-(\cdot;x)$ and $N_+(\cdot;x)$ are continued analytically in $\Compl^-$. Finally, we have for all $z\in \Compl^{\pm}$
%    \begin{equation*}
%        \|M_{\mp}(z;\cdot)\|_{L^{\infty}}+ \|N_{\pm}(z;\cdot)\|_{L^{\infty}}\leq C,
%    \end{equation*}
%    where the positive constant does not depend on $z$.
%\end{lem}
%\begin{rem}
%    The assumptions of Lemma \ref{l existence of M N} are chosen such that $Q_{1,2}(u)\in L^1(\Real)$.
%\end{rem}
%By means of the asymptotic expansion for large $z$ of the Jost functions, the potential $u$ can be reconstructed from $M_{\pm}$ and $N_{\pm}$, respectively. See \cite{Pelinovsky2016} for the next Lemma.
%\begin{lem}\label{l expansion N M}
%    For $u\in L^1(\Real)\cap L^3(\Real)\cap C^1(\Real)$ and $u_x\in L^1(\Real)$ and every $x\in\Real$ the Jost functions $M_{\pm}(z;x)$ and $N_{\pm}(z;x)$ have the following expansion as $|\im(z)|\to\infty$ along a contour in the domains of their analyticity:
%    \begin{equation}\label{e expansion N M}
%    \begin{aligned}
%        M_{\pm}(z;x)&=&\!\!\!\!
%        \left(
%         \begin{array}{c}
%           \!\!\!\!M^{\infty}_{\pm}(x)\!\!\!\! \\
%           0 \\
%         \end{array}
%        \right)+\frac{1}{z}&
%        \left(
%        \begin{array}{c}
%          -\frac{1}{4}\, M^{\infty}_{\pm}(x) \int_{\pm\infty}^x \left\{u(y)\partial_y \overline{u}(y)+\frac{1}{2\ii}|u(y)|^4\right\} dy \vspace{2mm}\\
%          \frac{1}{2\ii}\partial_x \left(\overline{u}(x) \,M^{\infty}_{\pm}(x)\right) \\
%        \end{array}
%        \right)+\mathcal{O}\left(\frac{1}{z^{2}}\right), \\[8pt]
%        N_{\pm}(z;x)&=&\!\!\!\!
%        \left(
%         \begin{array}{c}
%           0 \\
%           \!\!\!\!N^{\infty}_{\pm}(x)\!\!\!\! \\
%         \end{array}
%        \right)+\frac{1}{z}&
%        \left(
%        \begin{array}{c}
%          -\frac{1}{2\ii}\partial_x \left(u(x) \,N^{\infty}_{\pm}(x)\right)\vspace{2mm} \\
%          \frac{1}{4}\, N^{\infty}_{\pm}(x) \int_{\pm\infty}^x \left\{\overline{u}(y)\partial_y u(y)-\frac{1}{2\ii}|u(y)|^4\right\} dy \\
%        \end{array}
%        \right)+\mathcal{O}\left(\frac{1}{z^{2}}\right),
%    \end{aligned}
%    \end{equation}
%    where
%    \begin{equation}\label{e def M N intfy}
%        M^{\infty}_{\pm}(x):= e^{\frac{1}{2\ii}\int_{\pm\infty}^x|u(y)|^2dy}, \quad N^{\infty}_{\pm}(x):= e^{-\frac{1}{2\ii}\int_{\pm\infty}^x|u(y)|^2dy}.
%    \end{equation}
%    For the zero order expansion, the requirement $u\in C^1(\Real)$ is not necessary.
%\end{lem}
%In order to estimate the scattering coefficients which will be defined in the following subsection, we need the following, \cite[Lemma 3]{Pelinovsky2016}:
%\begin{lem}\label{l reminder of M N}
%    If $u\in H^{1,1}(\Real)$, then for every  $x\in\Real^{\pm}$, we have
%    \begin{equation*}
%        M_{\pm}(\cdot;x)-M^{\infty}_{\pm}(x)e_1\in H^{1}(\Real),\quad
%        N_{\pm}(\cdot;x)-N^{\infty}_{\pm}(x)e_2\in H^{1}(\Real).
%    \end{equation*}
%    Moreover, if $u\in H^2(\Real)\cap H^{1,1}(\Real)$, then for every $x\in\Real$, we have
%    \begin{equation*}
%        z[M_{\pm}(\cdot;x)-M^{\infty}_{\pm}(x)e_1]-
%        \left(\!\!\!
%        \begin{array}{c}
%          -\frac{1}{4}\, M^{\infty}_{\pm}(x) \int_{\pm\infty}^x \left\{u(y)\partial_y \overline{u}(y)+\frac{1}{2\ii}|u(y)|^4 \right\} dy \vspace{2mm}\\
%          \frac{1}{2\ii}\partial_x \left(\overline{u}(x) \,M^{\infty}_{\pm}(x)\right) \\
%        \end{array}\!\!\!
%        \right)\in L^{2}_z(\Real)
%    \end{equation*}
%    and
%    \begin{equation*}
%        z[N_{\pm}(\cdot;x)-N^{\infty}_{\pm}(x)e_2]-
%        \left(
%        \begin{array}{c}
%          -\frac{1}{2\ii}\partial_x \left(u(x) \,N^{\infty}_{\pm}(x)\right)\vspace{2mm} \\
%          \frac{1}{4}\, N^{\infty}_{\pm}(x) \int_{\pm\infty}^x \left\{\overline{u}(y)\partial_y u(y)-\frac{1}{2\ii}|u(y)|^4\right\} dy \\
%        \end{array}
%        \right)\in L^{2}_z(\Real).
%    \end{equation*}
%\end{lem}
%We end this paragraph with transferring properties of $M_{\pm}(z;x)$ and $N_{\pm}(z;x)$ to properties of the original Jost functions $\varphi_{\pm}(\lambda;x)$ and $\phi_{\pm}(\lambda;x)$. The next concluding Corollary is obtained from Lemmas \ref{l existence of M N}, \ref{l expansion N M} and \ref{l reminder of M N} and the fact that  $(M_{\pm},N_{\pm})$ and $(\phi_{\pm},\varphi_{\pm})$ are related by the matrix transformation (\ref{e def M N}). We remark that $T_{1,2}(\lambda;x)$ are singular matrices for $\lambda=0$.
%\begin{cor}\label{c properties varphi phi}
%    Let $u\in H^{1,1}(\Real)$. Then for every $\lambda^2\in\Real\setminus\set{0}$, there exist unique solutions $\varphi_{\pm}(\lambda;\cdot)\in L^{\infty}(\Real)$ and $\phi_{\pm}(\lambda;\cdot)\in L^{\infty}(\Real)$ of the integral equations (\ref{e volterra phi varphi}) such that
%    \begin{equation*}
%        \lim_{x\to\pm\infty}\varphi_{\pm} (\lambda;x)=e_1, \quad\lim_{x\to\pm\infty}\phi_{\pm} (\lambda;x)=e_2.
%    \end{equation*}
%    Furthermore, for every $x\in\Real$, the Jost functions $\varphi_-(\cdot;x)$ and $\phi_+(\cdot;x)$ are analytic in the first and third quadrant of the $\lambda$ plane (where $\im(\lambda^2)>0$), whereas the functions $\varphi_+(\lambda;x)$ and $\phi_-(\lambda;x)$ are analytic in the second and fourth quadrant (where $\im(\lambda^2)<0$). Moreover, for $x\in\Real^{\pm}$, we have
%    \begin{equation*}
%    \begin{aligned}
%        \left[
%        \begin{array}{cc}
%           1 & 0 \\
%           0 & 2\ii\lambda \\
%        \end{array}
%        \right]\varphi_{\pm}(\lambda;x)- M_{\pm}^{\infty}(x)
%        \left(
%         \begin{array}{c}
%           \!\!1\!\! \\
%           \!\!\overline{u}(x)\!\! \\
%         \end{array}
%        \right)&\in H^1_z(\Real),\quad
%        \left[
%        \begin{array}{cc}
%           0 & 0 \\
%           0 & \lambda^{-1} \\
%        \end{array}
%        \right]\varphi_{\pm}(\lambda;x)&\in H^1_z(\Real)
%        \\
%        \left[
%        \begin{array}{cc}
%           2\ii\lambda & 0 \\
%           0 & 1 \\
%        \end{array}
%        \right]\phi_{\pm}(\lambda;x)-N_{\pm}^{\infty}(x)
%        \left(
%         \begin{array}{c}
%            \!\!u(x)\!\! \\
%            \!\!1\!\! \\
%         \end{array}
%        \right) &\in H^1_z(\Real),\quad
%        \left[
%        \begin{array}{cc}
%           \lambda^{-1} & 0 \\
%           0 & 0\\
%        \end{array}
%        \right]\phi_{\pm}(\lambda;x)&\in H^1_z(\Real),
%    \end{aligned}
%    \end{equation*}
%    where $M_{\pm}^{\infty}(x)$ and $N_{\pm}^{\infty}(x)$ are the same functions as defined in (\ref{e def M N intfy}).
%\end{cor}
%\begin{rem}
%    For $\lambda=0$, the Volterra's integral equations (\ref{e volterra phi varphi}) yield $\varphi_{\pm} (0;x)=e_1$ and $\phi_{\pm} (0;x)=e_2$. From the relation (\ref{e def M N}) we obtain for $z=0$,
%    \begin{equation*}
%        M_{\pm}(0;x)=
%        \left(\!\!
%         \begin{array}{c}
%            \!\!1\!\! \\
%            \!\!-\overline{u}(x)\!\! \\
%         \end{array}\!\!
%        \right),\quad
%        N_{\pm}(0;x)=
%        \left(\!\!
%         \begin{array}{c}
%            \!\!-u(x)\!\! \\
%            \!\!1\!\! \\
%         \end{array}\!\!
%        \right),
%    \end{equation*}
%    which are indeed solutions of the intergal equations (\ref{e volterra M N}).
%\end{rem} 