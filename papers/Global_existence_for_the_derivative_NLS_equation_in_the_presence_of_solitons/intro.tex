\section{Introduction}\label{s intro}
Consider the Cauchy problem for the derivative nonlinear Schr\"odinger (DNLS) equation
\begin{equation}\label{e dnls}
    \left\{
      \begin{array}{ll}
        \ii u_t+u_{xx}+\ii(|u|^2u)_x=0,\\
        u|_{t=0}=u_0,
      \end{array}
    \right.
\end{equation}
on $\Real$, where $u(x,t):\Real\times\Real\to\Compl$. Subscripts denote partial derivatives. In this paper we will prove the following global existence result:
\begin{thm}\label{t main}
    There exists an open subset $\pazocal{G}\subset H^{1,1}(\Real)$ such that if $u_0$ belongs to $H^2(\Real)\cap \pazocal{G}$, then there exists a unique global solution $u(\cdot,t)\in H^2(\Real)\cap \pazocal{G}$ of the Cauchy problem (\ref{e dnls}) for every $t\in\Real$.
\end{thm}
%\begin{rem}
%    The proof of the following statement is not worked out in the present paper. Nevertheless, there is no doubt about it. The map $H^2(\Real)\cap \pazocal{G}\ni u_0 \mapsto u\in C([-T,T];H^2(\Real)\cap \pazocal{G})$ is Lipschitz continuous for every $0<T<\infty$.
%\end{rem}
The spaces used in Theorem \ref{t main} are defined as follows.
\begin{equation*}
    H^k(\Real)=\set{u\in L^2(\Real),...,\partial_x^ku\in L^2(\Real)},\quad
    H^{1,1}(\Real)=\set{u\in L^{2,1}(\Real),\partial_xu\in L^{2,1}(\Real)},
\end{equation*}
where the weighted spaces $L^{2,s}(\Real)$ are defined by the norm
\begin{equation*}
    \|u\|_{L^{2,s}(\Real)}:=\left(\int_{\Real}\langle x\rangle^{2s}|u(x)|^2dx\right)^{1/2},\qquad  \langle x\rangle:=\sqrt{1+x^2}.
\end{equation*}
The question of global well-posedness of the DNLS equation (\ref{e dnls}) was an open problem for a long time. Local solvability in $H^s(\Real)$ with $s>3/2$ was shown in \cite{Tsutsumi1980}. Later in \cite{Tsutsumi1981}, the same authors presented a result on global solvability for $u_0\in H^2(\Real)$ under the assumption that the $H^1$ norm of $u_0$ is small. Similar global well-posedness results were proved in \cite{Hayashi1992,Hayashi1993},
where the authors work with $u_0\in H^1(\Real)$ and assume a small $L^2(\Real)$ norm. More than two decades later this upper bound on the $L^2(\Real)$ norm of the initial datum could be improved by \cite{Wu2013,Wu2015}. Only recently the authors of \cite{Hayashi2017} proved that there exist global solutions with any large $L^2(\Real)$ norm. They showed global existence of solutions for (\ref{e dnls}) with initial datum of the form $u_0=e^{icx}\psi$ where $\psi\in H^1(\Real)$ can be arbitrary and $c$ has to be chosen sufficiently large.
\medskip \\
None of the so far mentioned articles relies on the fact that the DNLS is formally solvable with the inverse scattering transform method. This structural property was discovered in \cite{KaupNewell1978}. The most extensive analysis of the Cauchy problem (\ref{e dnls}) using inverse scattering tools is certainly given by the series of papers \cite{Liu2016,LiuPerrySulem2017,LiuPerrySulem2017b}. In their first work the authors establish Lipschitz continuity of the direct and inverse scattering transform for the DNLS equation in appropriate function spaces and they prove global solvability for those initial data that are soliton-free. The second work is devoted to long-time behavior of solutions for soliton-free initial data. Therein it is proven that the amplitude of those solutions decays like $|t|^{-1/2}$ as $|t|\to\infty$. Using this dispersion result and including solitons the authors complete their studies in their third paper where they give a full description of the long-time behavior of the solutions. Moreover, the third paper contains a proof of global well-posedness under the same assumptions on the initial data as in our Theorem \ref{t main}. Other rigorous works on the inverse scattering transform in the context of the DNLS equation are given by \cite{Pelinovsky2016} (soliton-free case) and its complementing paper \cite{PelinShimaSaal2017} (finite number of eigenvalues). Whereas in \cite{Liu2016} a gauge equivalence of the DNLS with a related dispersive equation is used, in \cite{Pelinovsky2016} the direct scattering transformation is constructed for the DNLS equation itself. This technical difference leads to different spaces: $H^2(\Real)\cap H^{1,1}(\Real)$ is appropriate in \cite{Pelinovsky2016}, but in \cite{Liu2016} the space $H^{2,2}(\Real):=H^2(\Real)\cap L^{2,1}(\Real)$ is considered.
\medskip \\
In \cite{Liu2016,Pelinovsky2016} as well as in the present paper, the assumption $u_0\in H^2(\Real)\cap \pazocal{G}$ on the initial datum avoids resonances of the spectral problem (\ref{e Lax1}). But in contrast to the soliton-free case \cite{Liu2016,Pelinovsky2016}, the elements in $\pazocal{G}$ are allowed to admit eigenvalues of (\ref{e Lax1}). The set of eigenvalues
$\set{\lambda_1,...,\lambda_N}$ then corresponds to a particular multi-soliton, in whose neighborhood the solution $u(x,t)$ will be located. Since Theorem \ref{t main} is a natural extension of the main results in \cite{Liu2016,Pelinovsky2016} and, moreover, since our result is already covered by \cite{PelinShimaSaal2017} as well as by \cite{LiuPerrySulem2017b}, we cannot raise any claim of originality of the result itself. What makes this present paper new is the way how the existence of the inverse scattering map in the case of solitons is established. Whereas in \cite{LiuPerrySulem2017b} this technical issue is treated directly, we give a proof by adding successively more and more eigenvalues, see Lemma \ref{l solvability of RHP N=1}. For that purpose we use a B\"{a}cklund transformation found in \cite{Deift2011}, see (\ref{e B�cklund for m^1}), and show that this transformation can be applied to the rigorous treatment of the DNLS equation. Technical statements such as Lemma \ref{l m(z0)-1 in H^11 and H^2} and Proposition \ref{p A inverse} become necessary and constitute the most original parts of our proof.

It shall be mentioned that a B\"{a}cklund transformation is also used in \cite{PelinShimaSaal2017}. But therein the transformation is applied directly to the solution $u$ in order to remove solitons.  Then by the solvability results from \cite{Liu2016,Pelinovsky2016} and the invertibility of the B\"{a}cklund transformation, the global well-posedness result follows. Hence, compared to  \cite{PelinShimaSaal2017}, the present paper does not only construct global solutions  of the DNLS equation for a large class of initial data but also solves the inverse scattering problem for those initial data.
\medskip \\
The paper is organized as follows. Section \ref{s scatt} contains the construction of the Jost functions and the definition of the scattering data for an initial datum $u_0\in H^2(\Real)\cap \pazocal{G}$. This section does not contain new results but follows closely \cite{Pelinovsky2016}.  At the end of Subsection \ref{ss scattering data} we formulate the Riemann--Hilbert problem as the starting point for the inverse scattering which is treated in Section \ref{s inverse sc} and \ref{s adding a pole}. For the convenience of the reader we inserted Section \ref{s solitons} where we shortly describe the phenomenon of solitary waves. Whereas Section \ref{s inverse sc} handles pure radiation solutions, in Section \ref{s adding a pole} we add a pole and obtain solutions in a neighborhood of a soliton. We split this procedure into two subsections since the cases $x>0$ and $x<0$ require different Riemann--Hilbert problems. Finally, in Section \ref{s proof} we use the local well-posedness theory in \cite{Tsutsumi1980} and \cite{Hayashi1992} and our estimates for the continuity of the inverse scattering to show that local solutions can be continued for all times.
\medskip \\
\\ 