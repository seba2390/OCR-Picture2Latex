%**************************************************************************
%*
%*  Paper: ``INSTRUCTIONS FOR AUTHORS OF LATEX DOCUMENTS''
%*
%*  Publication: 2017 Winter Simulation Conference Author Kit
%*
%*  Filename: wsc17paper.tex
%*
%*  Date: January 30, 2017   Time:  5:30 PM
%*
%*  Word Processing System: TeXnicCenter and MiKTeX
%*
%*
%*  All files need the following
%
%
% File contains:
%   fancyhdr.sty
%   psfig.sty
%   url.sty
%   wsc.bst
%   wscpaperproc.cls
%   wscsetup.sty
%   wsc.sty
%   wscbib.tex
%%%%%%%%%%%%%%%%%%%%%%%%%%%%%%%%%%%%%%%%%%%%%%%%%%%
 %%%%%%%%%%%%%%%%%%%%%%%%%%%%%%%%%%%%%%%%%%%%%%%%%%%%%%%%%%%%%%%%%%%%%%%%%%%%%%%%%%%%%%%%%%%%%%%%%%%%%%%

\begin{filecontents*}{fancyhdr.sty}
% fancyhdr.sty version 3.2
% Fancy headers and footers for LaTeX.
% Piet van Oostrum,
% Dept of Computer and Information Sciences, University of Utrecht,
% Padualaan 14, P.O. Box 80.089, 3508 TB Utrecht, The Netherlands
% Telephone: +31 30 2532180. Email: piet@cs.uu.nl
% ========================================================================
% LICENCE:
% This file may be distributed under the terms of the LaTeX Project Public
% License, as described in lppl.txt in the base LaTeX distribution.
% Either version 1 or, at your option, any later version.
% ========================================================================
% MODIFICATION HISTORY:
% Sep 16, 1994
% version 1.4: Correction for use with \reversemargin
% Sep 29, 1994:
% version 1.5: Added the \iftopfloat, \ifbotfloat and \iffloatpage commands
% Oct 4, 1994:
% version 1.6: Reset single spacing in headers/footers for use with
% setspace.sty or doublespace.sty
% Oct 4, 1994:
% version 1.7: changed \let\@mkboth\markboth to
% \def\@mkboth{\protect\markboth} to make it more robust
% Dec 5, 1994:
% version 1.8: corrections for amsbook/amsart: define \@chapapp and (more
% importantly) use the \chapter/sectionmark definitions from ps@headings if
% they exist (which should be true for all standard classes).
% May 31, 1995:
% version 1.9: The proposed \renewcommand{\headrulewidth}{\iffloatpage...
% construction in the doc did not work properly with the fancyplain style.
% June 1, 1995:
% version 1.91: The definition of \@mkboth wasn't restored on subsequent
% \pagestyle{fancy}'s.
% June 1, 1995:
% version 1.92: The sequence \pagestyle{fancyplain} \pagestyle{plain}
% \pagestyle{fancy} would erroneously select the plain version.
% June 1, 1995:
% version 1.93: \fancypagestyle command added.
% Dec 11, 1995:
% version 1.94: suggested by Conrad Hughes <chughes@maths.tcd.ie>
% CJCH, Dec 11, 1995: added \footruleskip to allow control over footrule
% position (old hardcoded value of .3\normalbaselineskip is far too high
% when used with very small footer fonts).
% Jan 31, 1996:
% version 1.95: call \@normalsize in the reset code if that is defined,
% otherwise \normalsize.
% this is to solve a problem with ucthesis.cls, as this doesn't
% define \@currsize. Unfortunately for latex209 calling \normalsize doesn't
% work as this is optimized to do very little, so there \@normalsize should
% be called. Hopefully this code works for all versions of LaTeX known to
% mankind.
% April 25, 1996:
% version 1.96: initialize \headwidth to a magic (negative) value to catch
% most common cases that people change it before calling \pagestyle{fancy}.
% Note it can't be initialized when reading in this file, because
% \textwidth could be changed afterwards. This is quite probable.
% We also switch to \MakeUppercase rather than \uppercase and introduce a
% \nouppercase command for use in headers. and footers.
% May 3, 1996:
% version 1.97: Two changes:
% 1. Undo the change in version 1.8 (using the pagestyle{headings} defaults
% for the chapter and section marks. The current version of amsbook and
% amsart classes don't seem to need them anymore. Moreover the standard
% latex classes don't use \markboth if twoside isn't selected, and this is
% confusing as \leftmark doesn't work as expected.
% 2. include a call to \ps@empty in ps@@fancy. This is to solve a problem
% in the amsbook and amsart classes, that make global changes to \topskip,
% which are reset in \ps@empty. Hopefully this doesn't break other things.
% May 7, 1996:
% version 1.98:
% Added % after the line  \def\nouppercase
% May 7, 1996:
% version 1.99: This is the alpha version of fancyhdr 2.0
% Introduced the new commands \fancyhead, \fancyfoot, and \fancyhf.
% Changed \headrulewidth, \footrulewidth, \footruleskip to
% macros rather than length parameters, In this way they can be
% conditionalized and they don't consume length registers. There is no need
% to have them as length registers unless you want to do calculations with
% them, which is unlikely. Note that this may make some uses of them
% incompatible (i.e. if you have a file that uses \setlength or \xxxx=)
% May 10, 1996:
% version 1.99a:
% Added a few more % signs
% May 10, 1996:
% version 1.99b:
% Changed the syntax of \f@nfor to be resistent to catcode changes of :=
% Removed the [1] from the defs of \lhead etc. because the parameter is
% consumed by the \@[xy]lhead etc. macros.
% June 24, 1997:
% version 1.99c:
% corrected \nouppercase to also include the protected form of \MakeUppercase
% \global added to manipulation of \headwidth.
% \iffootnote command added.
% Some comments added about \@fancyhead and \@fancyfoot.
% Aug 24, 1998
% version 1.99d
% Changed the default \ps@empty to \ps@@empty in order to allow
% \fancypagestyle{empty} redefinition.
% Oct 11, 2000
% version 2.0
% Added LPPL license clause.
%
% A check for \headheight is added. An errormessage is given (once) if the
% header is too large. Empty headers don't generate the error even if
% \headheight is very small or even 0pt.
% Warning added for the use of 'E' option when twoside option is not used.
% In this case the 'E' fields will never be used.
%
% Mar 10, 2002
% version 2.1beta
% New command: \fancyhfoffset[place]{length}
% defines offsets to be applied to the header/footer to let it stick into
% the margins (if length > 0).
% place is like in fancyhead, except that only E,O,L,R can be used.
% This replaces the old calculation based on \headwidth and the marginpar
% area.
% \headwidth will be dynamically calculated in the headers/footers when
% this is used.
%
% Mar 26, 2002
% version 2.1beta2
% \fancyhfoffset now also takes h,f as possible letters in the argument to
% allow the header and footer widths to be different.
% New commands \fancyheadoffset and \fancyfootoffset added comparable to
% \fancyhead and \fancyfoot.
% Errormessages and warnings have been made more informative.
%
% Dec 9, 2002
% version 2.1
% The defaults for \footrulewidth, \plainheadrulewidth and
% \plainfootrulewidth are changed from \z@skip to 0pt. In this way when
% someone inadvertantly uses \setlength to change any of these, the value
% of \z@skip will not be changed, rather an errormessage will be given.

% March 3, 2004
% Release of version 3.0

% Oct 7, 2004
% version 3.1
% Added '\endlinechar=13' to \fancy@reset to prevent problems with
% includegraphics in header when verbatiminput is active.

% March 22, 2005
% version 3.2
% reset \everypar (the real one) in \fancy@reset because spanish.ldf does
% strange things with \everypar between << and >>.

\def\ifancy@mpty#1{\def\temp@a{#1}\ifx\temp@a\@empty}

\def\fancy@def#1#2{\ifancy@mpty{#2}\fancy@gbl\def#1{\leavevmode}\else
                                   \fancy@gbl\def#1{#2\strut}\fi}

\let\fancy@gbl\global

\def\@fancyerrmsg#1{%
        \ifx\PackageError\undefined
        \errmessage{#1}\else
        \PackageError{Fancyhdr}{#1}{}\fi}
\def\@fancywarning#1{%
        \ifx\PackageWarning\undefined
        \errmessage{#1}\else
        \PackageWarning{Fancyhdr}{#1}{}\fi}

% Usage: \@forc \var{charstring}{command to be executed for each char}
% This is similar to LaTeX's \@tfor, but expands the charstring.

\def\@forc#1#2#3{\expandafter\f@rc\expandafter#1\expandafter{#2}{#3}}
\def\f@rc#1#2#3{\def\temp@ty{#2}\ifx\@empty\temp@ty\else
                                    \f@@rc#1#2\f@@rc{#3}\fi}
\def\f@@rc#1#2#3\f@@rc#4{\def#1{#2}#4\f@rc#1{#3}{#4}}

% Usage: \f@nfor\name:=list\do{body}
% Like LaTeX's \@for but an empty list is treated as a list with an empty
% element

\newcommand{\f@nfor}[3]{\edef\@fortmp{#2}%
    \expandafter\@forloop#2,\@nil,\@nil\@@#1{#3}}

% Usage: \def@ult \cs{defaults}{argument}
% sets \cs to the characters from defaults appearing in argument
% or defaults if it would be empty. All characters are lowercased.

\newcommand\def@ult[3]{%
    \edef\temp@a{\lowercase{\edef\noexpand\temp@a{#3}}}\temp@a
    \def#1{}%
    \@forc\tmpf@ra{#2}%
        {\expandafter\if@in\tmpf@ra\temp@a{\edef#1{#1\tmpf@ra}}{}}%
    \ifx\@empty#1\def#1{#2}\fi}
%
% \if@in <char><set><truecase><falsecase>
%
\newcommand{\if@in}[4]{%
    \edef\temp@a{#2}\def\temp@b##1#1##2\temp@b{\def\temp@b{##1}}%
    \expandafter\temp@b#2#1\temp@b\ifx\temp@a\temp@b #4\else #3\fi}

\newcommand{\fancyhead}{\@ifnextchar[{\f@ncyhf\fancyhead h}%
                                     {\f@ncyhf\fancyhead h[]}}
\newcommand{\fancyfoot}{\@ifnextchar[{\f@ncyhf\fancyfoot f}%
                                     {\f@ncyhf\fancyfoot f[]}}
\newcommand{\fancyhf}{\@ifnextchar[{\f@ncyhf\fancyhf{}}%
                                   {\f@ncyhf\fancyhf{}[]}}

% New commands for offsets added

\newcommand{\fancyheadoffset}{\@ifnextchar[{\f@ncyhfoffs\fancyheadoffset h}%
                                           {\f@ncyhfoffs\fancyheadoffset h[]}}
\newcommand{\fancyfootoffset}{\@ifnextchar[{\f@ncyhfoffs\fancyfootoffset f}%
                                           {\f@ncyhfoffs\fancyfootoffset f[]}}
\newcommand{\fancyhfoffset}{\@ifnextchar[{\f@ncyhfoffs\fancyhfoffset{}}%
                                         {\f@ncyhfoffs\fancyhfoffset{}[]}}

% The header and footer fields are stored in command sequences with
% names of the form: \f@ncy<x><y><z> with <x> for [eo], <y> from [lcr]
% and <z> from [hf].

\def\f@ncyhf#1#2[#3]#4{%
    \def\temp@c{}%
    \@forc\tmpf@ra{#3}%
        {\expandafter\if@in\tmpf@ra{eolcrhf,EOLCRHF}%
            {}{\edef\temp@c{\temp@c\tmpf@ra}}}%
    \ifx\@empty\temp@c\else
        \@fancyerrmsg{Illegal char `\temp@c' in \string#1 argument:
          [#3]}%
    \fi
    \f@nfor\temp@c{#3}%
        {\def@ult\f@@@eo{eo}\temp@c
         \if@twoside\else
           \if\f@@@eo e\@fancywarning
             {\string#1's `E' option without twoside option is useless}\fi\fi
         \def@ult\f@@@lcr{lcr}\temp@c
         \def@ult\f@@@hf{hf}{#2\temp@c}%
         \@forc\f@@eo\f@@@eo
             {\@forc\f@@lcr\f@@@lcr
                 {\@forc\f@@hf\f@@@hf
                     {\expandafter\fancy@def\csname
                      f@ncy\f@@eo\f@@lcr\f@@hf\endcsname
                      {#4}}}}}}

\def\f@ncyhfoffs#1#2[#3]#4{%
    \def\temp@c{}%
    \@forc\tmpf@ra{#3}%
        {\expandafter\if@in\tmpf@ra{eolrhf,EOLRHF}%
            {}{\edef\temp@c{\temp@c\tmpf@ra}}}%
    \ifx\@empty\temp@c\else
        \@fancyerrmsg{Illegal char `\temp@c' in \string#1 argument:
          [#3]}%
    \fi
    \f@nfor\temp@c{#3}%
        {\def@ult\f@@@eo{eo}\temp@c
         \if@twoside\else
           \if\f@@@eo e\@fancywarning
             {\string#1's `E' option without twoside option is useless}\fi\fi
         \def@ult\f@@@lcr{lr}\temp@c
         \def@ult\f@@@hf{hf}{#2\temp@c}%
         \@forc\f@@eo\f@@@eo
             {\@forc\f@@lcr\f@@@lcr
                 {\@forc\f@@hf\f@@@hf
                     {\expandafter\setlength\csname
                      f@ncyO@\f@@eo\f@@lcr\f@@hf\endcsname
                      {#4}}}}}%
     \fancy@setoffs}

% Fancyheadings version 1 commands. These are more or less deprecated,
% but they continue to work.

\newcommand{\lhead}{\@ifnextchar[{\@xlhead}{\@ylhead}}
\def\@xlhead[#1]#2{\fancy@def\f@ncyelh{#1}\fancy@def\f@ncyolh{#2}}
\def\@ylhead#1{\fancy@def\f@ncyelh{#1}\fancy@def\f@ncyolh{#1}}

\newcommand{\chead}{\@ifnextchar[{\@xchead}{\@ychead}}
\def\@xchead[#1]#2{\fancy@def\f@ncyech{#1}\fancy@def\f@ncyoch{#2}}
\def\@ychead#1{\fancy@def\f@ncyech{#1}\fancy@def\f@ncyoch{#1}}

\newcommand{\rhead}{\@ifnextchar[{\@xrhead}{\@yrhead}}
\def\@xrhead[#1]#2{\fancy@def\f@ncyerh{#1}\fancy@def\f@ncyorh{#2}}
\def\@yrhead#1{\fancy@def\f@ncyerh{#1}\fancy@def\f@ncyorh{#1}}

\newcommand{\lfoot}{\@ifnextchar[{\@xlfoot}{\@ylfoot}}
\def\@xlfoot[#1]#2{\fancy@def\f@ncyelf{#1}\fancy@def\f@ncyolf{#2}}
\def\@ylfoot#1{\fancy@def\f@ncyelf{#1}\fancy@def\f@ncyolf{#1}}

\newcommand{\cfoot}{\@ifnextchar[{\@xcfoot}{\@ycfoot}}
\def\@xcfoot[#1]#2{\fancy@def\f@ncyecf{#1}\fancy@def\f@ncyocf{#2}}
\def\@ycfoot#1{\fancy@def\f@ncyecf{#1}\fancy@def\f@ncyocf{#1}}

\newcommand{\rfoot}{\@ifnextchar[{\@xrfoot}{\@yrfoot}}
\def\@xrfoot[#1]#2{\fancy@def\f@ncyerf{#1}\fancy@def\f@ncyorf{#2}}
\def\@yrfoot#1{\fancy@def\f@ncyerf{#1}\fancy@def\f@ncyorf{#1}}

\newlength{\fancy@headwidth}
\let\headwidth\fancy@headwidth
\newlength{\f@ncyO@elh}
\newlength{\f@ncyO@erh}
\newlength{\f@ncyO@olh}
\newlength{\f@ncyO@orh}
\newlength{\f@ncyO@elf}
\newlength{\f@ncyO@erf}
\newlength{\f@ncyO@olf}
\newlength{\f@ncyO@orf}
\newcommand{\headrulewidth}{0.4pt}
\newcommand{\footrulewidth}{0pt}
\newcommand{\footruleskip}{.3\normalbaselineskip}

% Fancyplain stuff shouldn't be used anymore (rather
% \fancypagestyle{plain} should be used), but it must be present for
% compatibility reasons.

\newcommand{\plainheadrulewidth}{0pt}
\newcommand{\plainfootrulewidth}{0pt}
\newif\if@fancyplain \@fancyplainfalse
\def\fancyplain#1#2{\if@fancyplain#1\else#2\fi}

\headwidth=-123456789sp %magic constant

% Command to reset various things in the headers:
% a.o.  single spacing (taken from setspace.sty)
% and the catcode of ^^M (so that epsf files in the header work if a
% verbatim crosses a page boundary)
% It also defines a \nouppercase command that disables \uppercase and
% \Makeuppercase. It can only be used in the headers and footers.
\let\fnch@everypar\everypar% save real \everypar because of spanish.ldf
\def\fancy@reset{\fnch@everypar{}\restorecr\endlinechar=13
 \def\baselinestretch{1}%
 \def\nouppercase##1{{\let\uppercase\relax\let\MakeUppercase\relax
     \expandafter\let\csname MakeUppercase \endcsname\relax##1}}%
 \ifx\undefined\@newbaseline% NFSS not present; 2.09 or 2e
   \ifx\@normalsize\undefined \normalsize % for ucthesis.cls
   \else \@normalsize \fi
 \else% NFSS (2.09) present
  \@newbaseline%
 \fi}

% Initialization of the head and foot text.

% The default values still contain \fancyplain for compatibility.
\fancyhf{} % clear all
% lefthead empty on ``plain'' pages, \rightmark on even, \leftmark on odd pages
% evenhead empty on ``plain'' pages, \leftmark on even, \rightmark on odd pages
\if@twoside
  \fancyhead[el,or]{\fancyplain{}{\sl\rightmark}}
  \fancyhead[er,ol]{\fancyplain{}{\sl\leftmark}}
\else
  \fancyhead[l]{\fancyplain{}{\sl\rightmark}}
  \fancyhead[r]{\fancyplain{}{\sl\leftmark}}
\fi
\fancyfoot[c]{\rm\thepage} % page number

% Use box 0 as a temp box and dimen 0 as temp dimen.
% This can be done, because this code will always
% be used inside another box, and therefore the changes are local.

\def\@fancyvbox#1#2{\setbox0\vbox{#2}\ifdim\ht0>#1\@fancywarning
  {\string#1 is too small (\the#1): ^^J Make it at least \the\ht0.^^J
    We now make it that large for the rest of the document.^^J
    This may cause the page layout to be inconsistent, however\@gobble}%
  \dimen0=#1\global\setlength{#1}{\ht0}\ht0=\dimen0\fi
  \box0}

% Put together a header or footer given the left, center and
% right text, fillers at left and right and a rule.
% The \lap commands put the text into an hbox of zero size,
% so overlapping text does not generate an errormessage.
% These macros have 5 parameters:
% 1. LEFTSIDE BEARING % This determines at which side the header will stick
%    out. When \fancyhfoffset is used this calculates \headwidth, otherwise
%    it is \hss or \relax (after expansion).
% 2. \f@ncyolh, \f@ncyelh, \f@ncyolf or \f@ncyelf. This is the left component.
% 3. \f@ncyoch, \f@ncyech, \f@ncyocf or \f@ncyecf. This is the middle comp.
% 4. \f@ncyorh, \f@ncyerh, \f@ncyorf or \f@ncyerf. This is the right component.
% 5. RIGHTSIDE BEARING. This is always \relax or \hss (after expansion).

\def\@fancyhead#1#2#3#4#5{#1\hbox to\headwidth{\fancy@reset
  \@fancyvbox\headheight{\hbox
    {\rlap{\parbox[b]{\headwidth}{\raggedright#2}}\hfill
      \parbox[b]{\headwidth}{\centering#3}\hfill
      \llap{\parbox[b]{\headwidth}{\raggedleft#4}}}\headrule}}#5}

\def\@fancyfoot#1#2#3#4#5{#1\hbox to\headwidth{\fancy@reset
    \@fancyvbox\footskip{\footrule
      \hbox{\rlap{\parbox[t]{\headwidth}{\raggedright#2}}\hfill
        \parbox[t]{\headwidth}{\centering#3}\hfill
        \llap{\parbox[t]{\headwidth}{\raggedleft#4}}}}}#5}

\def\headrule{{\if@fancyplain\let\headrulewidth\plainheadrulewidth\fi
    \hrule\@height\headrulewidth\@width\headwidth \vskip-\headrulewidth}}

\def\footrule{{\if@fancyplain\let\footrulewidth\plainfootrulewidth\fi
    \vskip-\footruleskip\vskip-\footrulewidth
    \hrule\@width\headwidth\@height\footrulewidth\vskip\footruleskip}}

\def\ps@fancy{%
\@ifundefined{@chapapp}{\let\@chapapp\chaptername}{}%for amsbook
%
% Define \MakeUppercase for old LaTeXen.
% Note: we used \def rather than \let, so that \let\uppercase\relax (from
% the version 1 documentation) will still work.
%
\@ifundefined{MakeUppercase}{\def\MakeUppercase{\uppercase}}{}%
\@ifundefined{chapter}{\def\sectionmark##1{\markboth
{\MakeUppercase{\ifnum \c@secnumdepth>\z@
 \thesection\hskip 1em\relax \fi ##1}}{}}%
\def\subsectionmark##1{\markright {\ifnum \c@secnumdepth >\@ne
 \thesubsection\hskip 1em\relax \fi ##1}}}%
{\def\chaptermark##1{\markboth {\MakeUppercase{\ifnum \c@secnumdepth>\m@ne
 \@chapapp\ \thechapter. \ \fi ##1}}{}}%
\def\sectionmark##1{\markright{\MakeUppercase{\ifnum \c@secnumdepth >\z@
 \thesection. \ \fi ##1}}}}%
%\csname ps@headings\endcsname % use \ps@headings defaults if they exist
\ps@@fancy
\gdef\ps@fancy{\@fancyplainfalse\ps@@fancy}%
% Initialize \headwidth if the user didn't
%
\ifdim\headwidth<0sp
%
% This catches the case that \headwidth hasn't been initialized and the
% case that the user added something to \headwidth in the expectation that
% it was initialized to \textwidth. We compensate this now. This loses if
% the user intended to multiply it by a factor. But that case is more
% likely done by saying something like \headwidth=1.2\textwidth.
% The doc says you have to change \headwidth after the first call to
% \pagestyle{fancy}. This code is just to catch the most common cases were
% that requirement is violated.
%
    \global\advance\headwidth123456789sp\global\advance\headwidth\textwidth
\fi}
\def\ps@fancyplain{\ps@fancy \let\ps@plain\ps@plain@fancy}
\def\ps@plain@fancy{\@fancyplaintrue\ps@@fancy}
\let\ps@@empty\ps@empty
\def\ps@@fancy{%
\ps@@empty % This is for amsbook/amsart, which do strange things with \topskip
\def\@mkboth{\protect\markboth}%
\def\@oddhead{\@fancyhead\fancy@Oolh\f@ncyolh\f@ncyoch\f@ncyorh\fancy@Oorh}%
\def\@oddfoot{\@fancyfoot\fancy@Oolf\f@ncyolf\f@ncyocf\f@ncyorf\fancy@Oorf}%
\def\@evenhead{\@fancyhead\fancy@Oelh\f@ncyelh\f@ncyech\f@ncyerh\fancy@Oerh}%
\def\@evenfoot{\@fancyfoot\fancy@Oelf\f@ncyelf\f@ncyecf\f@ncyerf\fancy@Oerf}%
}
% Default definitions for compatibility mode:
% These cause the header/footer to take the defined \headwidth as width
% And to shift in the direction of the marginpar area

\def\fancy@Oolh{\if@reversemargin\hss\else\relax\fi}
\def\fancy@Oorh{\if@reversemargin\relax\else\hss\fi}
\let\fancy@Oelh\fancy@Oorh
\let\fancy@Oerh\fancy@Oolh

\let\fancy@Oolf\fancy@Oolh
\let\fancy@Oorf\fancy@Oorh
\let\fancy@Oelf\fancy@Oelh
\let\fancy@Oerf\fancy@Oerh

% New definitions for the use of \fancyhfoffset
% These calculate the \headwidth from \textwidth and the specified offsets.

\def\fancy@offsolh{\headwidth=\textwidth\advance\headwidth\f@ncyO@olh
                   \advance\headwidth\f@ncyO@orh\hskip-\f@ncyO@olh}
\def\fancy@offselh{\headwidth=\textwidth\advance\headwidth\f@ncyO@elh
                   \advance\headwidth\f@ncyO@erh\hskip-\f@ncyO@elh}

\def\fancy@offsolf{\headwidth=\textwidth\advance\headwidth\f@ncyO@olf
                   \advance\headwidth\f@ncyO@orf\hskip-\f@ncyO@olf}
\def\fancy@offself{\headwidth=\textwidth\advance\headwidth\f@ncyO@elf
                   \advance\headwidth\f@ncyO@erf\hskip-\f@ncyO@elf}

\def\fancy@setoffs{%
% Just in case \let\headwidth\textwidth was used
  \fancy@gbl\let\headwidth\fancy@headwidth
  \fancy@gbl\let\fancy@Oolh\fancy@offsolh
  \fancy@gbl\let\fancy@Oelh\fancy@offselh
  \fancy@gbl\let\fancy@Oorh\hss
  \fancy@gbl\let\fancy@Oerh\hss
  \fancy@gbl\let\fancy@Oolf\fancy@offsolf
  \fancy@gbl\let\fancy@Oelf\fancy@offself
  \fancy@gbl\let\fancy@Oorf\hss
  \fancy@gbl\let\fancy@Oerf\hss}

\newif\iffootnote
\let\latex@makecol\@makecol
\def\@makecol{\ifvoid\footins\footnotetrue\else\footnotefalse\fi
\let\topfloat\@toplist\let\botfloat\@botlist\latex@makecol}
\def\iftopfloat#1#2{\ifx\topfloat\empty #2\else #1\fi}
\def\ifbotfloat#1#2{\ifx\botfloat\empty #2\else #1\fi}
\def\iffloatpage#1#2{\if@fcolmade #1\else #2\fi}

\newcommand{\fancypagestyle}[2]{%
  \@namedef{ps@#1}{\let\fancy@gbl\relax#2\relax\ps@fancy}}

\end{filecontents*}

 %%%%%%%%%%%%%%%%%%%%%%%%%%%%%%%%%%%%%%%%%%%%%%%%%%%%%%%%%%%%%%%%%%%%%%%%%%%%%%%%%%%%%%%%%%%%%%%%%%%%%%%

\begin{filecontents*}{psfig.sty}
% Psfig/TeX
\def\PsfigVersion{1.10}
\def\setDriver{\DvipsDriver} % \DvipsDriver or \OzTeXDriver
%
% All software, documentation, and related files in this distribution of
% psfig/tex are Copyright 1993 Trevor J. Darrell
%
% Permission is granted for use and non-profit distribution of psfig/tex
% providing that this notice is clearly maintained. The right to
% distribute any portion of psfig/tex for profit or as part of any commercial
% product is specifically reserved for the author(s) of that portion.
%
% To use with LaTeX, use \documentstyle[psfig,...]{...}
% To use with TeX, use \input psfig.sty
%
% Bugs and improvements to trevor@media.mit.edu.
%
% Thanks to Ned Batchelder, Greg Hager (GDH), J. Daniel Smith (JDS),
% Tom Rokicki (TR), Robert Russell (RR), George V. Reilly (GVR),
% Ken McGlothlen (KHC), Baron Grey (BG), Gerhard Tobermann (GT),
% Jairo R. Montoya-Torres (JRMT).
% and all others who have contributed code and comments to this project!
%
% ======================================================================
% Modification History:
%
%  9 Oct 1990   JDS used more robust bbox reading code from Tom Rokicki
% 29 Mar 1991   JDS implemented rotation= option
% 25 Jun 1991   RR  if bb specified on cmd line don't check
%           for .ps file.
%  3 Jul 1991   JDS check if file already read in once
%  4 Sep 1991   JDS fixed incorrect computation of rotated
%           bounding box
% 25 Sep 1991   GVR expanded synopsis of \psfig
% 14 Oct 1991   JDS \fbox code from LaTeX so \psdraft works with TeX
%           changed \typeout to \ps@typeout
% 17 Oct 1991   JDS added \psscalefirst and \psrotatefirst
% 23 Jun 1993   KHC     ``doclip'' must appear before ``rotate''
% 27 Oct 1993   TJD removed printing of filename to avoid
%           underscore problems. changed \frame to \fbox.
%           Added OzTeX support from BG. Added new
%           figure search path code from GT.
% 01 Feb 2010   JRMT implemented the use of .eps figure format
%
% ======================================================================
%
% Command synopsis:
%
% \psdraft  draws an outline box, but doesn't include the figure
%       in the DVI file.  Useful for previewing.
%
% \psfull   includes the figure in the DVI file (default).
%
% \psscalefirst width= or height= specifies the size of the figure
%       before rotation.
% \psrotatefirst (default) width= or height= specifies the size of the
%        figure after rotation.  Asymetric figures will
%        appear to shrink.
%
% \psfigurepath{dir:dir:...}  sets the path to search for the figure
%
% \psfig
% usage: \psfig{file=, figure=, height=, width=,
%           bbllx=, bblly=, bburx=, bbury=,
%           rheight=, rwidth=, clip=, angle=, silent=}
%
%   "file" is the filename.  If no path name is specified and the
%       file is not found in the current directory,
%       it will be looked for in directory \psfigurepath.
%   "figure" is a synonym for "file".
%   By default, the width and height of the figure are taken from
%       the BoundingBox of the figure.
%   If "width" is specified, the figure is scaled so that it has
%       the specified width.  Its height changes proportionately.
%   If "height" is specified, the figure is scaled so that it has
%       the specified height.  Its width changes proportionately.
%   If both "width" and "height" are specified, the figure is scaled
%       anamorphically.
%   "bbllx", "bblly", "bburx", and "bbury" control the PostScript
%       BoundingBox.  If these four values are specified
%               *before* the "file" option, the PSFIG will not try to
%               open the PostScript file.
%   "rheight" and "rwidth" are the reserved height and width
%       of the figure, i.e., how big TeX actually thinks
%       the figure is.  They default to "width" and "height".
%   The "clip" option ensures that no portion of the figure will
%       appear outside its BoundingBox.  "clip=" is a switch and
%       takes no value, but the `=' must be present.
%   The "angle" option specifies the angle of rotation (degrees, ccw).
%   The "silent" option makes \psfig work silently.
%
% ======================================================================
% check to see if macros already loaded in (maybe some other file says
% "\input psfig") ...
\ifx\undefined\psfig\else\endinput\fi
%
% from a suggestion by eijkhout@csrd.uiuc.edu to allow
% loading as a style file. Changed to avoid problems
% with amstex per suggestion by jbence@math.ucla.edu

\let\LaTeXAtSign=\@
\let\@=\relax
\edef\psfigRestoreAt{\catcode`\@=\number\catcode`@\relax}
%\edef\psfigRestoreAt{\catcode`@=\number\catcode`@\relax}
\catcode`\@=11\relax
\newwrite\@unused
\def\ps@typeout#1{{\let\protect\string\immediate\write\@unused{#1}}}

\def\DvipsDriver{
    \ps@typeout{psfig/tex \PsfigVersion -dvips}
\def\PsfigSpecials{\DvipsSpecials}  \def\ps@dir{/}
\def\ps@predir{} }
\def\OzTeXDriver{
    \ps@typeout{psfig/tex \PsfigVersion -oztex}
    \def\PsfigSpecials{\OzTeXSpecials}
    \def\ps@dir{:}
    \def\ps@predir{:}
    \catcode`\^^J=5
}

%% Here's how you define your figure path.  Should be set up with null
%% default and a user useable definition.

\def\figurepath{./:}
\def\psfigurepath#1{\edef\figurepath{#1:}}

%%% inserted for Searching Unixpaths
%%% (the path must end with :)
%%% (call: \DoPaths\figurepath )
%%%------------------------------------------------------
\def\DoPaths#1{\expandafter\EachPath#1\stoplist}
%
\def\leer{}
\def\EachPath#1:#2\stoplist{% #1 part of the list (delimiter :)
  \ExistsFile{#1}{\SearchedFile}
  \ifx#2\leer
  \else
    \expandafter\EachPath#2\stoplist
  \fi}
%
% exists the file (does not work for directories!)
%
\def\ps@dir{/}
\def\ExistsFile#1#2{%
   \openin1=\ps@predir#1\ps@dir#2
   \ifeof1
       \closein1
       %\ps@typeout{...not: \ps@predir#1\ps@dir#2}
   \else
       \closein1
       %\ps@typeout{...in:  \ps@predir#1\ps@dir#2}
        \ifx\ps@founddir\leer
          %\ps@typeout{set founddir #1}
           \edef\ps@founddir{#1}
        \fi
   \fi}
%------------------------------------------------------
%
% Get dir in path or error
%
\def\get@dir#1{%
  \def\ps@founddir{}
  \def\SearchedFile{#1}
  \DoPaths\figurepath
%  \fi
}
%------------------------------------------------------
%%% END of Searching Unixpaths


%
% @psdo control structure -- similar to Latex @for.
% I redefined these with different names so that psfig can
% be used with TeX as well as LaTeX, and so that it will not
% be vunerable to future changes in LaTeX's internal
% control structure,
%
\def\@nnil{\@nil}
\def\@empty{}
\def\@psdonoop#1\@@#2#3{}
\def\@psdo#1:=#2\do#3{\edef\@psdotmp{#2}\ifx\@psdotmp\@empty \else
    \expandafter\@psdoloop#2,\@nil,\@nil\@@#1{#3}\fi}
\def\@psdoloop#1,#2,#3\@@#4#5{\def#4{#1}\ifx #4\@nnil \else
       #5\def#4{#2}\ifx #4\@nnil \else#5\@ipsdoloop #3\@@#4{#5}\fi\fi}
\def\@ipsdoloop#1,#2\@@#3#4{\def#3{#1}\ifx #3\@nnil
       \let\@nextwhile=\@psdonoop \else
      #4\relax\let\@nextwhile=\@ipsdoloop\fi\@nextwhile#2\@@#3{#4}}
\def\@tpsdo#1:=#2\do#3{\xdef\@psdotmp{#2}\ifx\@psdotmp\@empty \else
    \@tpsdoloop#2\@nil\@nil\@@#1{#3}\fi}
\def\@tpsdoloop#1#2\@@#3#4{\def#3{#1}\ifx #3\@nnil
       \let\@nextwhile=\@psdonoop \else
      #4\relax\let\@nextwhile=\@tpsdoloop\fi\@nextwhile#2\@@#3{#4}}
%
% \fbox is defined in latex.tex; so if \fbox is undefined, assume that
% we are not in LaTeX.
% Perhaps this could be done better???
\ifx\undefined\fbox
% \fbox code from modified slightly from LaTeX
\newdimen\fboxrule
\newdimen\fboxsep
\newdimen\ps@tempdima
\newbox\ps@tempboxa
\fboxsep = 3pt \fboxrule = .4pt
\long\def\fbox#1{\leavevmode\setbox\ps@tempboxa\hbox{#1}\ps@tempdima\fboxrule
    \advance\ps@tempdima \fboxsep \advance\ps@tempdima \dp\ps@tempboxa
   \hbox{\lower \ps@tempdima\hbox
  {\vbox{\hrule height \fboxrule
          \hbox{\vrule width \fboxrule \hskip\fboxsep
          \vbox{\vskip\fboxsep \box\ps@tempboxa\vskip\fboxsep}\hskip
                 \fboxsep\vrule width \fboxrule}
                 \hrule height \fboxrule}}}}
\fi
%
%%%%%%%%%%%%%%%%%%%%%%%%%%%%%%%%%%%%%%%%%%%%%%%%%%%%%%%%%%%%%%%%%%%
% file reading stuff from epsf.tex
%   EPSF.TEX macro file:
%   Written by Tomas Rokicki of Radical Eye Software, 29 Mar 1989.
%   Revised by Don Knuth, 3 Jan 1990.
%   Revised by Tomas Rokicki to accept bounding boxes with no
%      space after the colon, 18 Jul 1990.
%   Portions modified/removed for use in PSFIG package by
%      J. Daniel Smith, 9 October 1990.
%
\newread\ps@stream
\newif\ifnot@eof       % continue looking for the bounding box?
\newif\if@noisy        % report what you're making?
\newif\if@atend        % %%BoundingBox: has (at end) specification
\newif\if@psfile       % does this look like a PostScript file?
%
% PostScript files should start with `%!'
%
{\catcode`\%=12\global\gdef\epsf@start{%!}}
\def\epsf@PS{PS}
%
\def\epsf@getbb#1{%
%
%   The first thing we need to do is to open the
%   PostScript file, if possible.
%
\openin\ps@stream=\ps@predir#1 \ifeof\ps@stream\ps@typeout{Error,
File #1 not found}\else
%
%   Okay, we got it. Now we'll scan lines until we find one that doesn't
%   start with %. We're looking for the bounding box comment.
%
   {\not@eoftrue \chardef\other=12
    \def\do##1{\catcode`##1=\other}\dospecials \catcode`\ =10
    \loop
       \if@psfile
      \read\ps@stream to \epsf@fileline
       \else{
      \obeyspaces
          \read\ps@stream to \epsf@tmp\global\let\epsf@fileline\epsf@tmp}
       \fi
       \ifeof\ps@stream\not@eoffalse\else
%
%   Check the first line for `%!'.  Issue a warning message if its not
%   there, since the file might not be a PostScript file.
%
       \if@psfile\else
       \expandafter\epsf@test\epsf@fileline:. \\%
       \fi
%
%   We check to see if the first character is a % sign;
%   if so, we look further and stop only if the line begins with
%   `%%BoundingBox:' and the `(atend)' specification was not found.
%   That is, the only way to stop is when the end of file is reached,
%   or a `%%BoundingBox: llx lly urx ury' line is found.
%
          \expandafter\epsf@aux\epsf@fileline:. \\%
       \fi
   \ifnot@eof\repeat
   }\closein\ps@stream\fi}%
%
% This tests if the file we are reading looks like a PostScript file.
%
\long\def\epsf@test#1#2#3:#4\\{\def\epsf@testit{#1#2}
            \ifx\epsf@testit\epsf@start\else
\ps@typeout{Warning! File does not start with `\epsf@start'.  It
may not be a PostScript file.}
            \fi
            \@psfiletrue} % don't test after 1st line
%
%   We still need to define the tricky \epsf@aux macro. This requires
%   a couple of magic constants for comparison purposes.
%
{\catcode`\%=12\global\let\epsf@percent=%\global\def\epsf@bblit{%BoundingBox}}
%
%
%   So we're ready to check for `%BoundingBox:' and to grab the
%   values if they are found.  We continue searching if `(at end)'
%   was found after the `%BoundingBox:'.
%
\long\def\epsf@aux#1#2:#3\\{\ifx#1\epsf@percent
   \def\epsf@testit{#2}\ifx\epsf@testit\epsf@bblit
    \@atendfalse
        \epsf@atend #3 . \\%
    \if@atend
       \if@verbose{
        \ps@typeout{psfig: found `(atend)'; continuing search}
       }\fi
        \else
        \epsf@grab #3 . . . \\%
        \not@eoffalse
        \global\no@bbfalse
        \fi
   \fi\fi}%
%
%   Here we grab the values and stuff them in the appropriate definitions.
%
\def\epsf@grab #1 #2 #3 #4 #5\\{%
   \global\def\epsf@llx{#1}\ifx\epsf@llx\empty
      \epsf@grab #2 #3 #4 #5 .\\\else
   \global\def\epsf@lly{#2}%
   \global\def\epsf@urx{#3}\global\def\epsf@ury{#4}\fi}%
%
% Determine if the stuff following the %%BoundingBox is `(atend)'
% J. Daniel Smith.  Copied from \epsf@grab above.
%
\def\epsf@atendlit{(atend)}
\def\epsf@atend #1 #2 #3\\{%
   \def\epsf@tmp{#1}\ifx\epsf@tmp\empty
      \epsf@atend #2 #3 .\\\else
   \ifx\epsf@tmp\epsf@atendlit\@atendtrue\fi\fi}


% End of file reading stuff from epsf.tex
%%%%%%%%%%%%%%%%%%%%%%%%%%%%%%%%%%%%%%%%%%%%%%%%%%%%%%%%%%%%%%%%%%%

%%%%%%%%%%%%%%%%%%%%%%%%%%%%%%%%%%%%%%%%%%%%%%%%%%%%%%%%%%%%%%%%%%%
% trigonometry stuff from "trig.tex"
\chardef\psletter = 11 % won't conflict with \begin{letter} now...
\chardef\other = 12

\newif \ifdebug %%% turn me on to see TeX hard at work ...
\newif\ifc@mpute %%% don't need to compute some values
\c@mputetrue % but assume that we do

\let\then = \relax
\def\r@dian{pt }
\let\r@dians = \r@dian
\let\dimensionless@nit = \r@dian
\let\dimensionless@nits = \dimensionless@nit
\def\internal@nit{sp }
\let\internal@nits = \internal@nit
\newif\ifstillc@nverging
\def \Mess@ge #1{\ifdebug \then \message {#1} \fi}

{ %%% Things that need abnormal catcodes %%%
    \catcode `\@ = \psletter
    \gdef \nodimen {\expandafter \n@dimen \the \dimen}
    \gdef \term #1 #2 #3%
           {\edef \t@ {\the #1}%%% freeze parameter 1 (count, by value)
        \edef \t@@ {\expandafter \n@dimen \the #2\r@dian}%
                   %%% freeze parameter 2 (dimen, by value)
        \t@rm {\t@} {\t@@} {#3}%
           }
    \gdef \t@rm #1 #2 #3%
           {{%
        \count 0 = 0
        \dimen 0 = 1 \dimensionless@nit
        \dimen 2 = #2\relax
        \Mess@ge {Calculating term #1 of \nodimen 2}%
        \loop
        \ifnum  \count 0 < #1
        \then   \advance \count 0 by 1
            \Mess@ge {Iteration \the \count 0 \space}%
            \Multiply \dimen 0 by {\dimen 2}%
            \Mess@ge {After multiplication, term = \nodimen 0}%
            \Divide \dimen 0 by {\count 0}%
            \Mess@ge {After division, term = \nodimen 0}%
        \repeat
        \Mess@ge {Final value for term #1 of
                \nodimen 2 \space is \nodimen 0}%
        \xdef \Term {#3 = \nodimen 0 \r@dians}%
        \aftergroup \Term
           }}
    \catcode `\p = \other
    \catcode `\t = \other
    \gdef \n@dimen #1pt{#1} %%% throw away the ``pt''
}

\def \Divide #1by #2{\divide #1 by #2} %%% just a synonym

\def \Multiply #1by #2%%% allows division of a dimen by a dimen
       {{%%% should really freeze parameter 2 (dimen, passed by value)
    \count 0 = #1\relax
    \count 2 = #2\relax
    \count 4 = 65536
    \Mess@ge {Before scaling, count 0 = \the \count 0 \space and
            count 2 = \the \count 2}%
    \ifnum  \count 0 > 32767 %%% do our best to avoid overflow
    \then   \divide \count 0 by 4
        \divide \count 4 by 4
    \else   \ifnum  \count 0 < -32767
        \then   \divide \count 0 by 4
            \divide \count 4 by 4
        \else
        \fi
    \fi
    \ifnum  \count 2 > 32767 %%% while retaining reasonable accuracy
    \then   \divide \count 2 by 4
        \divide \count 4 by 4
    \else   \ifnum  \count 2 < -32767
        \then   \divide \count 2 by 4
            \divide \count 4 by 4
        \else
        \fi
    \fi
    \multiply \count 0 by \count 2
    \divide \count 0 by \count 4
    \xdef \product {#1 = \the \count 0 \internal@nits}%
    \aftergroup \product
       }}

\def\r@duce{\ifdim\dimen0 > 90\r@dian \then   % sin(x+90) = sin(180-x)
        \multiply\dimen0 by -1
        \advance\dimen0 by 180\r@dian
        \r@duce
        \else \ifdim\dimen0 < -90\r@dian \then  % sin(-x) = sin(360+x)
        \advance\dimen0 by 360\r@dian
        \r@duce
        \fi
        \fi}

\def\Sine#1%
       {{%
    \dimen 0 = #1 \r@dian
    \r@duce
    \ifdim\dimen0 = -90\r@dian \then
       \dimen4 = -1\r@dian
       \c@mputefalse
    \fi
    \ifdim\dimen0 = 90\r@dian \then
       \dimen4 = 1\r@dian
       \c@mputefalse
    \fi
    \ifdim\dimen0 = 0\r@dian \then
       \dimen4 = 0\r@dian
       \c@mputefalse
    \fi
%
    \ifc@mpute \then
            % convert degrees to radians
        \divide\dimen0 by 180
        \dimen0=3.141592654\dimen0
%
        \dimen 2 = 3.1415926535897963\r@dian %%% a well-known constant
        \divide\dimen 2 by 2 %%% we only deal with -pi/2 : pi/2
        \Mess@ge {Sin: calculating Sin of \nodimen 0}%
        \count 0 = 1 %%% see power-series expansion for sine
        \dimen 2 = 1 \r@dian %%% ditto
        \dimen 4 = 0 \r@dian %%% ditto
        \loop
            \ifnum  \dimen 2 = 0 %%% then we've done
            \then   \stillc@nvergingfalse
            \else   \stillc@nvergingtrue
            \fi
            \ifstillc@nverging %%% then calculate next term
            \then   \term {\count 0} {\dimen 0} {\dimen 2}%
                \advance \count 0 by 2
                \count 2 = \count 0
                \divide \count 2 by 2
                \ifodd  \count 2 %%% signs alternate
                \then   \advance \dimen 4 by \dimen 2
                \else   \advance \dimen 4 by -\dimen 2
                \fi
        \repeat
    \fi
            \xdef \sine {\nodimen 4}%
       }}

% Now the Cosine can be calculated easily by calling \Sine
\def\Cosine#1{\ifx\sine\UnDefined\edef\Savesine{\relax}\else
                     \edef\Savesine{\sine}\fi
    {\dimen0=#1\r@dian\advance\dimen0 by 90\r@dian
     \Sine{\nodimen 0}
     \xdef\cosine{\sine}
     \xdef\sine{\Savesine}}}
% end of trig stuff
%%%%%%%%%%%%%%%%%%%%%%%%%%%%%%%%%%%%%%%%%%%%%%%%%%%%%%%%%%%%%%%%%%%%

\def\psdraft{
    \def\@psdraft{0}
    %\ps@typeout{draft level now is \@psdraft \space . }
}
\def\psfull{
    \def\@psdraft{100}
    %\ps@typeout{draft level now is \@psdraft \space . }
}

\psfull

\newif\if@scalefirst
\def\psscalefirst{\@scalefirsttrue}
\def\psrotatefirst{\@scalefirstfalse}
\psrotatefirst

\newif\if@draftbox
\def\psnodraftbox{
    \@draftboxfalse
}
\def\psdraftbox{
    \@draftboxtrue
} \@draftboxtrue

\newif\if@prologfile
\newif\if@postlogfile
\def\pssilent{
    \@noisyfalse
}
\def\psnoisy{
    \@noisytrue
} \psnoisy
%%% These are for the option list.
%%% A specification of the form a = b maps to calling \@p@@sa{b}
\newif\if@bbllx
\newif\if@bblly
\newif\if@bburx
\newif\if@bbury
\newif\if@height
\newif\if@width
\newif\if@rheight
\newif\if@rwidth
\newif\if@angle
\newif\if@clip
\newif\if@verbose
\def\@p@@sclip#1{\@cliptrue}
%
%
\newif\if@decmpr
%
\def\@p@@sfigure#1{\def\@p@sfile{null}\def\@p@sbbfile{null}\@decmprfalse
   % look directly for file (e.g. absolute path)
   \openin1=\ps@predir#1
   \ifeof1
    \closein1
    % failed, search directories for file
    \get@dir{#1}
    \ifx\ps@founddir\leer
        % failed, search directly for file.bb
        \openin1=\ps@predir#1.bb
        \ifeof1
            \closein1
            % failed, search directories for file.bb
            \get@dir{#1.bb}
            \ifx\ps@founddir\leer
                % failed, lose.
                \ps@typeout{Can't find #1 in \figurepath}
            \else
                % found file.bb in search dir
                \@decmprtrue
                \def\@p@sfile{\ps@founddir\ps@dir#1}
                \def\@p@sbbfile{\ps@founddir\ps@dir#1.bb}
            \fi
        \else
            \closein1
            %found file.bb directly
            \@decmprtrue
            \def\@p@sfile{#1}
            \def\@p@sbbfile{#1.bb}
        \fi
    \else
        % found file in search dir
        \def\@p@sfile{\ps@founddir\ps@dir#1}
        \def\@p@sbbfile{\ps@founddir\ps@dir#1}
    \fi
   \else
    % found file directly
    \closein1
    \def\@p@sfile{#1}
    \def\@p@sbbfile{#1}
   \fi
}
%
%
%
\def\@p@@sfile#1{\@p@@sfigure{#1}}
%
\def\@p@@sbbllx#1{
        %\ps@typeout{bbllx is #1}
        \@bbllxtrue
        \dimen100=#1
        \edef\@p@sbbllx{\number\dimen100}
}
\def\@p@@sbblly#1{
        %\ps@typeout{bblly is #1}
        \@bbllytrue
        \dimen100=#1
        \edef\@p@sbblly{\number\dimen100}
}
\def\@p@@sbburx#1{
        %\ps@typeout{bburx is #1}
        \@bburxtrue
        \dimen100=#1
        \edef\@p@sbburx{\number\dimen100}
}
\def\@p@@sbbury#1{
        %\ps@typeout{bbury is #1}
        \@bburytrue
        \dimen100=#1
        \edef\@p@sbbury{\number\dimen100}
}
\def\@p@@sheight#1{
        \@heighttrue
        \dimen100=#1
        \edef\@p@sheight{\number\dimen100}
        %\ps@typeout{Height is \@p@sheight}
}
\def\@p@@swidth#1{
        %\ps@typeout{Width is #1}
        \@widthtrue
        \dimen100=#1
        \edef\@p@swidth{\number\dimen100}
}
\def\@p@@srheight#1{
        %\ps@typeout{Reserved height is #1}
        \@rheighttrue
        \dimen100=#1
        \edef\@p@srheight{\number\dimen100}
}
\def\@p@@srwidth#1{
        %\ps@typeout{Reserved width is #1}
        \@rwidthtrue
        \dimen100=#1
        \edef\@p@srwidth{\number\dimen100}
}
\def\@p@@sangle#1{
        %\ps@typeout{Rotation is #1}
        \@angletrue
%       \dimen100=#1
        \edef\@p@sangle{#1} %\number\dimen100}
}
\def\@p@@ssilent#1{
        \@verbosefalse
}
\def\@p@@sprolog#1{\@prologfiletrue\def\@prologfileval{#1}}
\def\@p@@spostlog#1{\@postlogfiletrue\def\@postlogfileval{#1}}
\def\@cs@name#1{\csname #1\endcsname}
\def\@setparms#1=#2,{\@cs@name{@p@@s#1}{#2}}
%
% initialize the defaults (size the size of the figure)
%
\def\ps@init@parms{
        \@bbllxfalse \@bbllyfalse
        \@bburxfalse \@bburyfalse
        \@heightfalse \@widthfalse
        \@rheightfalse \@rwidthfalse
        \def\@p@sbbllx{}\def\@p@sbblly{}
        \def\@p@sbburx{}\def\@p@sbbury{}
        \def\@p@sheight{}\def\@p@swidth{}
        \def\@p@srheight{}\def\@p@srwidth{}
        \def\@p@sangle{0}
        \def\@p@sfile{} \def\@p@sbbfile{}
        \def\@p@scost{10}
        \def\@sc{}
        \@prologfilefalse
        \@postlogfilefalse
        \@clipfalse
        \if@noisy
            \@verbosetrue
        \else
            \@verbosefalse
        \fi
}
%
% Go through the options setting things up.
%
\def\parse@ps@parms#1{
        \@psdo\@psfiga:=#1\do
           {\expandafter\@setparms\@psfiga,}}
%
% Compute bb height and width
%
\newif\ifno@bb
\def\bb@missing{
    \if@verbose{
        \ps@typeout{psfig: searching \@p@sbbfile \space  for bounding box}
    }\fi
    \no@bbtrue
    \epsf@getbb{\@p@sbbfile}
        \ifno@bb \else \bb@cull\epsf@llx\epsf@lly\epsf@urx\epsf@ury\fi
}
\def\bb@cull#1#2#3#4{
    \dimen100=#1 bp\edef\@p@sbbllx{\number\dimen100}
    \dimen100=#2 bp\edef\@p@sbblly{\number\dimen100}
    \dimen100=#3 bp\edef\@p@sbburx{\number\dimen100}
    \dimen100=#4 bp\edef\@p@sbbury{\number\dimen100}
    \no@bbfalse
}
% rotate point (#1,#2) about (0,0).
% The sine and cosine of the angle are already stored in \sine and
% \cosine.  The result is placed in (\p@intvaluex, \p@intvaluey).
\newdimen\p@intvaluex
\newdimen\p@intvaluey
\def\rotate@#1#2{{\dimen0=#1 sp\dimen1=#2 sp
%               calculate x' = x \cos\theta - y \sin\theta
          \global\p@intvaluex=\cosine\dimen0
          \dimen3=\sine\dimen1
          \global\advance\p@intvaluex by -\dimen3
%       calculate y' = x \sin\theta + y \cos\theta
          \global\p@intvaluey=\sine\dimen0
          \dimen3=\cosine\dimen1
          \global\advance\p@intvaluey by \dimen3
          }}
\def\compute@bb{
        \no@bbfalse
        \if@bbllx \else \no@bbtrue \fi
        \if@bblly \else \no@bbtrue \fi
        \if@bburx \else \no@bbtrue \fi
        \if@bbury \else \no@bbtrue \fi
        \ifno@bb \bb@missing \fi
        \ifno@bb \ps@typeout{FATAL ERROR: no bb supplied or found}
            \no-bb-error
        \fi
        %
%\ps@typeout{BB: \@p@sbbllx, \@p@sbblly, \@p@sbburx, \@p@sbbury}
%
% store height/width of original (unrotated) bounding box
        \count203=\@p@sbburx
        \count204=\@p@sbbury
        \advance\count203 by -\@p@sbbllx
        \advance\count204 by -\@p@sbblly
        \edef\ps@bbw{\number\count203}
        \edef\ps@bbh{\number\count204}
        %\ps@typeout{ psbbh = \ps@bbh, psbbw = \ps@bbw }
        \if@angle
            \Sine{\@p@sangle}\Cosine{\@p@sangle}
                {\dimen100=\maxdimen\xdef\r@p@sbbllx{\number\dimen100}
                        \xdef\r@p@sbblly{\number\dimen100}
                                \xdef\r@p@sbburx{-\number\dimen100}
                        \xdef\r@p@sbbury{-\number\dimen100}}
%
% Need to rotate all four points and take the X-Y extremes of the new
% points as the new bounding box.
                        \def\minmaxtest{
               \ifnum\number\p@intvaluex<\r@p@sbbllx
                  \xdef\r@p@sbbllx{\number\p@intvaluex}\fi
               \ifnum\number\p@intvaluex>\r@p@sbburx
                  \xdef\r@p@sbburx{\number\p@intvaluex}\fi
               \ifnum\number\p@intvaluey<\r@p@sbblly
                  \xdef\r@p@sbblly{\number\p@intvaluey}\fi
               \ifnum\number\p@intvaluey>\r@p@sbbury
                  \xdef\r@p@sbbury{\number\p@intvaluey}\fi
               }
%           lower left
            \rotate@{\@p@sbbllx}{\@p@sbblly}
            \minmaxtest
%           upper left
            \rotate@{\@p@sbbllx}{\@p@sbbury}
            \minmaxtest
%           lower right
            \rotate@{\@p@sbburx}{\@p@sbblly}
            \minmaxtest
%           upper right
            \rotate@{\@p@sbburx}{\@p@sbbury}
            \minmaxtest
            \edef\@p@sbbllx{\r@p@sbbllx}\edef\@p@sbblly{\r@p@sbblly}
            \edef\@p@sbburx{\r@p@sbburx}\edef\@p@sbbury{\r@p@sbbury}
%\ps@typeout{rotated BB: \r@p@sbbllx, \r@p@sbblly, \r@p@sbburx, \r@p@sbbury}
        \fi
        \count203=\@p@sbburx
        \count204=\@p@sbbury
        \advance\count203 by -\@p@sbbllx
        \advance\count204 by -\@p@sbblly
        \edef\@bbw{\number\count203}
        \edef\@bbh{\number\count204}
        %\ps@typeout{ bbh = \@bbh, bbw = \@bbw }
}
%
% \in@hundreds performs #1 * (#2 / #3) correct to the hundreds,
%   then leaves the result in @result
%
\def\in@hundreds#1#2#3{\count240=#2 \count241=#3
             \count100=\count240    % 100 is first digit #2/#3
             \divide\count100 by \count241
             \count101=\count100
             \multiply\count101 by \count241
             \advance\count240 by -\count101
             \multiply\count240 by 10
             \count101=\count240    %101 is second digit of #2/#3
             \divide\count101 by \count241
             \count102=\count101
             \multiply\count102 by \count241
             \advance\count240 by -\count102
             \multiply\count240 by 10
             \count102=\count240    % 102 is the third digit
             \divide\count102 by \count241
             \count200=#1\count205=0
             \count201=\count200
            \multiply\count201 by \count100
            \advance\count205 by \count201
             \count201=\count200
            \divide\count201 by 10
            \multiply\count201 by \count101
            \advance\count205 by \count201
            %
             \count201=\count200
            \divide\count201 by 100
            \multiply\count201 by \count102
            \advance\count205 by \count201
            %
             \edef\@result{\number\count205}
}
\def\compute@wfromh{
        % computing : width = height * (bbw / bbh)
        \in@hundreds{\@p@sheight}{\@bbw}{\@bbh}
        %\ps@typeout{ \@p@sheight * \@bbw / \@bbh, = \@result }
        \edef\@p@swidth{\@result}
        %\ps@typeout{w from h: width is \@p@swidth}
}
\def\compute@hfromw{
        % computing : height = width * (bbh / bbw)
            \in@hundreds{\@p@swidth}{\@bbh}{\@bbw}
        %\ps@typeout{ \@p@swidth * \@bbh / \@bbw = \@result }
        \edef\@p@sheight{\@result}
        %\ps@typeout{h from w : height is \@p@sheight}
}
\def\compute@handw{
        \if@height
            \if@width
            \else
                \compute@wfromh
            \fi
        \else
            \if@width
                \compute@hfromw
            \else
                \edef\@p@sheight{\@bbh}
                \edef\@p@swidth{\@bbw}
            \fi
        \fi
}
\def\compute@resv{
        \if@rheight \else \edef\@p@srheight{\@p@sheight} \fi
        \if@rwidth \else \edef\@p@srwidth{\@p@swidth} \fi
        %\ps@typeout{rheight = \@p@srheight, rwidth = \@p@srwidth}
}
%
% Compute any missing values
\def\compute@sizes{
    \compute@bb
    \if@scalefirst\if@angle
% at this point the bounding box has been adjsuted correctly for
% rotation.  PSFIG does all of its scaling using \@bbh and \@bbw.  If
% a width= or height= was specified along with \psscalefirst, then the
% width=/height= value needs to be adjusted to match the new (rotated)
% bounding box size (specifed in \@bbw and \@bbh).
%    \ps@bbw       width=
%    -------  =  ----------
%    \@bbw       new width=
% so `new width=' = (width= * \@bbw) / \ps@bbw; where \ps@bbw is the
% width of the original (unrotated) bounding box.
    \if@width
       \in@hundreds{\@p@swidth}{\@bbw}{\ps@bbw}
       \edef\@p@swidth{\@result}
    \fi
    \if@height
       \in@hundreds{\@p@sheight}{\@bbh}{\ps@bbh}
       \edef\@p@sheight{\@result}
    \fi
    \fi\fi
    \compute@handw
    \compute@resv}
%
%
%
\def\OzTeXSpecials{
    \special{empty.ps /@isp {true} def}
    \special{empty.ps \@p@swidth \space \@p@sheight \space
            \@p@sbbllx \space \@p@sbblly \space
            \@p@sbburx \space \@p@sbbury \space
            startTexFig \space }
    \if@clip{
        \if@verbose{
            \ps@typeout{(clip)}
        }\fi
        \special{empty.ps doclip \space }
    }\fi
    \if@angle{
        \if@verbose{
            \ps@typeout{(rotate)}
        }\fi
        \special {empty.ps \@p@sangle \space rotate \space}
    }\fi
    \if@prologfile
        \special{\@prologfileval \space } \fi
    \if@decmpr{
        \if@verbose{
            \ps@typeout{psfig: Compression not available
            in OzTeX version \space }
        }\fi
    }\else{
        \if@verbose{
            \ps@typeout{psfig: including \@p@sfile \space }
        }\fi
        \special{epsf=\@p@sfile \space }
    }\fi
    \if@postlogfile
        \special{\@postlogfileval \space } \fi
    \special{empty.ps /@isp {false} def}
}
\def\DvipsSpecials{
    %
    \special{ps::[begin]    \@p@swidth \space \@p@sheight \space
            \@p@sbbllx \space \@p@sbblly \space
            \@p@sbburx \space \@p@sbbury \space
            startTexFig \space }
    \if@clip{
        \if@verbose{
            \ps@typeout{(clip)}
        }\fi
        \special{ps:: doclip \space }
    }\fi
    \if@angle
        \if@verbose{
            \ps@typeout{(clip)}
        }\fi
        \special {ps:: \@p@sangle \space rotate \space}
    \fi
    \if@prologfile
        \special{ps: plotfile \@prologfileval \space } \fi
    \if@decmpr{
        \if@verbose{
            \ps@typeout{psfig: including \@p@sfile.Z \space }
        }\fi
        \special{ps: plotfile "`zcat \@p@sfile.Z" \space }
    }\else{
        \if@verbose{
            \ps@typeout{psfig: including \@p@sfile \space }
        }\fi
        \special{ps: plotfile \@p@sfile \space }
    }\fi
    \if@postlogfile
        \special{ps: plotfile \@postlogfileval \space } \fi
    \special{ps::[end] endTexFig \space }
}
%
% \psfig
% usage : \psfig{file=, height=, width=, bbllx=, bblly=, bburx=, bbury=,
%           rheight=, rwidth=, clip=}
%
% "clip=" is a switch and takes no value, but the `=' must be present.
\def\psfig#1{\vbox {
    % do a zero width hard space so that a single
    % \psfig in a centering enviornment will behave nicely
    %{\setbox0=\hbox{\ }\ \hskip-\wd0}
    %
    \ps@init@parms
    \parse@ps@parms{#1}
    \compute@sizes
    %
    \ifnum\@p@scost<\@psdraft{
        \PsfigSpecials
        % Create the vbox to reserve the space for the figure.
        \vbox to \@p@srheight sp{
        % 1/92 TJD Changed from "true sp" to "sp" for magnification.
            \hbox to \@p@srwidth sp{
                \hss
            }
        \vss
        }
    }\else{
        % draft figure, just reserve the space and print the
        % path name.
        \if@draftbox{
            % Verbose draft: print file name in box
            % 10/93 TJD changed to fbox from frame
            \hbox{\fbox{\vbox to \@p@srheight sp{
            \vss
            \hbox to \@p@srwidth sp{ \hss
                    % 10/93 TJD deleted to avoid ``_'' problems
                % \@p@sfile
             \hss }
            \vss
            }}}
        }\else{
            % Non-verbose draft
            \vbox to \@p@srheight sp{
            \vss
            \hbox to \@p@srwidth sp{\hss}
            \vss
            }
        }\fi



    }\fi
}} \psfigRestoreAt \setDriver
\let\@=\LaTeXAtSign
\end{filecontents*}

 %%%%%%%%%%%%%%%%%%%%%%%%%%%%%%%%%%%%%%%%%%%%%%%%%%%%%%%%%%%%%%%%%%%%%%%%%%%%%%%%%%%%%%%%%%%%%%%%%%%%%%%

\begin{filecontents*}{url.sty}
% url.sty  ver 1.4    02-Mar-1999   Donald Arseneau   asnd@triumf.ca
% Copyright 1996-1999 Donald Arseneau,  Vancouver, Canada.
% This program can be used, distributed, and modified under the terms
% of the LaTeX Project Public License.
%
% A form of \verb that allows linebreaks at certain characters or
% combinations of characters, accepts reconfiguration, and can usually
% be used in the argument to another command.  It is intended for email
% addresses, hypertext links, directories/paths, etc., which normally
% have no spaces.  The font may be selected using the \urlstyle command,
% and new url-like commands can be defined using \urldef.
%
% Usage:    Conditions:
% \url{ }   If the argument contains any "%", "#", or "^^", or ends with
%           "\", it can't be used in the argument to another command.
%           The argument must not contain unbalanced braces.
% \url|  |  ...where "|" is any character not used in the argument and not
%           "{" or a space.  The same restrictions as above except that the
%           argument may contain unbalanced braces.
% \xyz      for "\xyz" a defined-url;  this can be used anywhere, no matter
%           what characters it contains.
%
% See further instructions after "\endinput"
%
\def\Url@ttdo{% style assignments for tt fonts or T1 encoding
\def\UrlBreaks{\do\.\do\@\do\\\do\/\do\!\do\_\do\|\do\%\do\;\do\>\do\]%
 \do\)\do\,\do\?\do\'\do\+\do\=}%
\def\UrlBigBreaks{\do\:\do@url@hyp}%
\def\UrlNoBreaks{\do\(\do\[\do\{\do\<}% (unnecessary)
\def\UrlSpecials{\do\ {\ }}%
\def\UrlOrds{\do\*\do\-\do\~}% any ordinary characters that aren't usually
}
\def\Url@do{% style assignments for OT1 fonts except tt
\def\UrlBreaks{\do\.\do\@\do\/\do\!\do\%\do\;\do\]\do\)\do\,\do\?\do\+\do\=}%
\def\UrlBigBreaks{\do\:\do@url@hyp}%
\def\UrlNoBreaks{\do\(\do\[\do\{}% prevents breaks after *next* character
\def\UrlSpecials{\do\<{\langle}\do\>{\mathbin{\rangle}}\do\_{\_%
 \penalty\@m}\do\|{\mid}\do\{{\lbrace}\do\}{\mathbin{\rbrace}}\do
 \\{\mathbin{\backslash}}\do\~{\raise.6ex\hbox{\m@th$\scriptstyle\sim$}}\do
 \ {\ }}%
\def\UrlOrds{\do\'\do\"\do\-}%
}
\def\url@ttstyle{%
\@ifundefined{selectfont}{\def\UrlFont{\tt}}{\def\UrlFont{\ttfamily}}\Url@ttdo
}
\def\url@rmstyle{%
\@ifundefined{selectfont}{\def\UrlFont{\rm}}{\def\UrlFont{\rmfamily}}\Url@do
}
\def\url@sfstyle{%
\@ifundefined{selectfont}{\def\UrlFont{\sf}}{\def\UrlFont{\sffamily}}\Url@do
}
\def\url@samestyle{\ifdim\fontdimen\thr@@\font=\z@ \url@ttstyle \else
  \url@rmstyle \fi \def\UrlFont{}}

\@ifundefined{strip@prefix}{\def\strip@prefix#1>{}}{}
\@ifundefined{verbatim@nolig@list}{\def\verbatim@nolig@list{\do\`}}{}

\def\Url{%
 \begingroup \let\url@moving\relax\relax \endgroup
 \ifmmode\@nomatherr$\fi
 \UrlFont $\fam\z@ \textfont\z@\font
 \let\do\@makeother \dospecials % verbatim catcodes
 \catcode`{\@ne \catcode`}\tw@ \catcode`\ 10 % except braces and spaces
 \medmuskip0mu \thickmuskip\medmuskip \thinmuskip\medmuskip
 \@tempcnta\fam\multiply\@tempcnta\@cclvi
 \let\do\set@mathcode \UrlOrds % ordinary characters that were special
 \advance\@tempcnta 8192 \UrlBreaks % bin
 \advance\@tempcnta 4096 \UrlBigBreaks % rel
 \advance\@tempcnta 4096 \UrlNoBreaks % open
 \let\do\set@mathact \UrlSpecials % active
 \let\do\set@mathnolig \verbatim@nolig@list % prevent ligatures
 \@ifnextchar\bgroup\Url@z\Url@y}

\def\Url@y#1{\catcode`{11 \catcode`}11
  \def\@tempa##1#1{\Url@z{##1}}\@tempa}
\def\Url@z#1{\def\@tempa{#1}\expandafter\expandafter\expandafter\Url@Hook
  \expandafter\strip@prefix\meaning\@tempa\UrlRight\m@th$\endgroup}
\def\Url@Hook{\UrlLeft}
\let\UrlRight\@empty
\let\UrlLeft\@empty

\def\set@mathcode#1{\count@`#1\advance\count@\@tempcnta\mathcode`#1\count@}
\def\set@mathact#1#2{\mathcode`#132768 \lccode`\~`#1\lowercase{\def~{#2}}}
\def\set@mathnolig#1{\ifnum\mathcode`#1<32768
   \lccode`\~`#1\lowercase{\edef~{\mathchar\number\mathcode`#1_{\/}}}%
   \mathcode`#132768 \fi}

\def\urldef#1#2{\begingroup \setbox\z@\hbox\bgroup
  \def\Url@z{\Url@def{#1}{#2}}#2}
\expandafter\ifx\csname DeclareRobustCommand\endcsname\relax
  \def\Url@def#1#2#3{\m@th$\endgroup\egroup\endgroup
    \def#1{#2{#3}}}
\else
  \def\Url@def#1#2#3{\m@th$\endgroup\egroup\endgroup
    \DeclareRobustCommand{#1}{#2{#3}}}
\fi

\def\urlstyle#1{\csname url@#1style\endcsname}

% Sample (and default) configuration:
%
\newcommand\url{\begingroup \Url}
%
% picTeX defines \path, so declare it optionally:
\@ifundefined{path}{\newcommand\path{\begingroup
\urlstyle{tt}\Url}}{}
%
% too many styles define \email like \address, so I will not define it.
% \newcommand\email{\begingroup \urlstyle{rm}\Url}

% Process LaTeX \package options
%
\urlstyle{tt}
\let\Url@sppen\@M
\def\do@url@hyp{}% by default, no breaks after hyphens

\@ifundefined{ProvidesPackage}{}{
  \ProvidesPackage{url}[1999/03/02 \space ver 1.4 \space
       Verb mode for urls, email addresses, and file names]
  \DeclareOption{hyphens}{\def\do@url@hyp{\do\-}}% allow breaks after hyphens
  \DeclareOption{obeyspaces}{\let\Url@Hook\relax}% a flag for later
  \DeclareOption{spaces}{\let\Url@sppen\relpenalty}
  \DeclareOption{T1}{\let\Url@do\Url@ttdo}
  \ProcessOptions
\ifx\Url@Hook\relax % [obeyspaces] was declared
  \def\Url@Hook#1\UrlRight\m@th{\edef\@tempa{\noexpand\UrlLeft
    \Url@retain#1\Url@nosp\, }\@tempa\UrlRight\m@th}
  \def\Url@retain#1 {#1\penalty\Url@sppen\ \Url@retain}
  \def\Url@nosp\,#1\Url@retain{}
\fi }

\edef\url@moving{\csname Url Error\endcsname}
\expandafter\edef\url@moving
 {\csname url used in a moving argument.\endcsname}
\expandafter\expandafter\expandafter \let \url@moving\undefined

\endinput
%
% url.sty  ver 1.4   02-Mar-1999   Donald Arseneau   asnd@reg.triumf.ca
%
% This package defines "\url", a form of "\verb" that allows linebreaks,
% and can often be used in the argument to another command.  It can be
% configured to print in different formats, and is particularly useful for
% hypertext links, email addresses, directories/paths, etc.  The font may
% be selected using the "\urlstyle" command and pre-defined text can be
% stored with the "\urldef" command. New url-like commands can be defined,
% and a "\path" command is provided this way.
%
% Usage:    Conditions:
% \url{ }   If the argument contains any "%", "#", or "^^", or ends with
%           "\", it can't be used in the argument to another command.
%           The argument must not contain unbalanced braces.
% \url|  |  ...where "|" is any character not used in the argument and not
%           "{" or a space.  The same restrictions as above except that the
%           argument may contain unbalanced braces.
% \xyz      for "\xyz" a defined-url;  this can be used anywhere, no matter
%           what characters it contains.
%
% The "\url" command is fragile, and its argument is likely to be very
% fragile, but a defined-url is robust.
%
% Package Option:  obeyspaces
% Ordinarily, all spaces are ignored in the url-text.  The "[obeyspaces]"
% option allows spaces, but may introduce spurious spaces when a url
% containing "\" characters is given in the argument to another command.
% So if you need to obey spaces you can say "\usepackage[obeyspaces]{url}",
% and if you need both spaces and backslashes, use a `defined-url' for
% anything with "\".
%
% Package Option:  hyphens
% Ordinarily, breaks are not allowed after "-" characters because this
% leads to confusion. (Is the "-" part of the address or just a hyphen?)
% The package option "[hyphens]" allows breaks after explicit hyphen
% characters.  The "\url" command will *never ever* hyphenate words.
%
% Package Option:  spaces
% Likewise, breaks are not usually allowed after spaces under the
% "[obeyspaces]" option, but giving the options "[obeyspaces,spaces]"
% will allow breaks at those spaces.
%
% Package Option:  T1
% This signifies that you will be using T1-encoded fonts which contain
% some characters missing from most older (OT1) encoded TeX fonts.  This
% changes the default definition for "\urlstyle{rm}".
%
% Defining a defined-url:
% Take for example the email address "myself%node@gateway.net" which could
% not be given (using "\url" or "\verb") in a caption or parbox due to the
% percent sign.  This address can be predefined with
%    \urldef{\myself}\url{myself%node@gateway.net}   or
%    \urldef{\myself}\url|myself%node@gateway.net|
% and then you may use "\myself" instead of "\url{myself%node@gateway.net}"
% in an argument, and even in a moving argument like a caption because a
% defined-url is robust.
%
% Style:
% You can switch the style of printing using "\urlstyle{tt}", where "tt"
% can be any defined style.  The pre-defined styles are "tt", "rm", "sf",
% and "same" which all allow the same linebreaks but different fonts --
% the first three select a specific font and the "same" style uses the
% current text font.  You can define your own styles with different fonts
% and/or line-breaking by following the explanations below.  The "\url"
% command follows whatever the currently-set style dictates.
%
% Alternate commands:
% It may be desireable to have different things treated differently, each
% in a predefined style; e.g., if you want directory paths to always be
% in tt and email addresses to be rm, then you would define new url-like
% commands as follows:
%
%    \newcommand\email{\begingroup \urlstyle{rm}\Url}
%    \newcommand\directory{\begingroup \urlstyle{tt}\Url}
%
% You must follow this format closely, and NOTE that the final command is
% "\Url", not "\url".  In fact, the "\directory" example is exactly the
% "\path" definition which is pre-defined in the package.  If you look
% above, you will see that "\url" is defined with
%    \newcommand\url{\begingroup \Url}
% I.e., using whatever url-style has been selected.
%
% You can make a defined-url for these other styles, using the usual
% "\urldef" command as in this example:
%
%    \urldef{\myself}{\email}{myself%node.domain@gateway.net}
%
% which makes "\myself" act like "\email{myself%node.domain@gateway.net}",
% if the "\email" command is defined as above.  The "\myself" command
% would then be robust.
%
% Defining styles:
% Before describing how to customize the printing style, it is best to
% mention something about the unusual implementation of "\url".  Although
% the material is textual in nature, and the font specification required
% is a text-font command, the text is actually typeset in *math* mode.
% This allows the context-sensitive linebreaking, but also accounts for
% the default behavior of ignoring spaces.  Now on to defining styles.
%
% To change the font or the list of characters that allow linebreaks, you
% could redefine the commands "\UrlFont", "\UrlBreaks", "\UrlSpecials" etc.
% directly in the document, but it is better to define a new `url-style'
% (following the example of "\url@ttstyle" and "\url@rmstyle") which defines
% all of "\UrlBigbreaks", "\UrlNoBreaks", "\UrlBreaks", "\UrlSpecials", and
% "\UrlFont".
%
% Changing font:
% The "\UrlFont" command selects the font.  The definition of "\UrlFont"
% done by the pre-defined styles varies to cope with a variety of LaTeX
% font selection schemes, but it could be as simple as "\def\UrlFont{\tt}".
% Depending on the font selected, some characters may need to be defined
% in the "\UrlSpecials" list because many fonts don't contain all the
% standard input characters.
%
% Changing linebreaks:
% The list of characters that allow line-breaks is given by "\UrlBreaks"
% and "\UrlBigBreaks", which have the format "\do\c" for character "c".
% The differences are that `BigBreaks' have a lower penalty and have
% different breakpoints when in sequence (as in "http://"): `BigBreaks'
% are treated as mathrels while `Breaks' are mathbins (see The TeXbook,
% p.170). In particular, a series of `BigBreak' characters will break at
% the end and only at the end; a series of `Break' characters will break
% after the first and after every following *pair*; there will be no
% break after a `Break' character if a `BigBreak' follows.  In the case
% of "http://" it doesn't matter whether ":" is a `Break' or `BigBreak' --
% the breaks are the same in either case; but for DECnet nodes with "::"
% it is important to prevent breaks *between* the colons, and that is why
% colons are `BigBreaks'.
%
% It is possible for characters to prevent breaks after the next following
% character (I use this for parentheses).  Specify these in "\UrlNoBreaks".
%
% You can do arbitrarily complex things with characters by making them
% active in math mode (mathcode hex-8000) and specifying the definition(s)
% in "\UrlSpecials".  This is used in the rm and sf styles for OT1 font
% encoding to handle several characters that are not present in those
% computer-modern style fonts.  See the definition of "\Url@do", which
% is used by both "\url@rmstyle" and "\url@sfstyle"; it handles missing
% characters via "\UrlSpecials".  The nominal format for setting each
% special character "c" is: "\do\c{<definition>}", but you can include
% other definitions too.
%
%
% If all this sounds confusing ... well, it is!  But I hope you won't need
% to redefine breakpoints -- the default assignments seem to work well for
% a wide variety of applications.  If you do need to make changes, you can
% test for breakpoints using regular math mode and the characters "+=(a".
%
% Yet more flexibility:
% You can also customize the verbatim text by defining "\UrlRight" and/or
% "\UrlLeft", e.g., for ISO formatting of urls surrounded by "<  >", define
%
%    \renewcommand\url{\begingroup \def\UrlLeft{<url: }\def\UrlRight{>}%
%        \urlstyle{tt}\Url}
%
% The meanings of "\UrlLeft" and "\UrlRight" are *not* reproduced verbatim.
% This lets you use formatting commands there, but you must be careful not
% to use TeX's special characters ("\^_%~#$&{}" etc.) improperly.
% You can also define "\UrlLeft" to reprocess the verbatim text, but the
% format of the definition is special:
%
%    \def\UrlLeft#1\UrlRight{ ... do things with #1 ... }
%
% Yes, that is "#1" followed by "\UrlRight" then the definition.  For
% example, to put a hyperTeX hypertext link in the DVI file:
%
%    \def\UrlLeft#1\UrlRight{\special{html:<a href="#1">}#1\special{html:</a>}}
%
% Using this technique, url.sty can provide a convenient interface for
% performing various operations on verbatim text.  You don't even need
% to print out the argument!  For greatest efficiency in such obscure
% applications, you can define a null url-style where all the lists like
% "\UrlBreaks" are empty.
%
% Revision History:
% ver 1.1 6-Feb-1996:
% Fix hyphens that wouldn't break and ligatures that weren't suppressed.
% ver 1.2 19-Oct-1996:
% Package option for T1 encoding; Hooks: "\UrlLeft" and "\UrlRight".
% ver 1.3 21-Jul-1997:
% Prohibit spaces as delimiter characters; change ascii tilde in OT1.
% ver 1.4 02-Mar-1999
% LaTeX license; moving-argument-error
% The End

Test file integrity:  ASCII 32-57, 58-126:  !"#$%&'()*+,-./0123456789
:;<=>?@ABCDEFGHIJKLMNOPQRSTUVWXYZ[\]^_`abcdefghijklmnopqrstuvwxyz{|}~
\end{filecontents*}

 %%%%%%%%%%%%%%%%%%%%%%%%%%%%%%%%%%%%%%%%%%%%%%%%%%%%%%%%%%%%%%%%%%%%%%%%%%%%%%%%%%%%%%%%%%%%%%%%%%%%%%%

\begin{filecontents*}{wsc.bst}
%%% ====================================================================
%%%  @BibTeX-style-file{
%%%     author          = "Glenn Paulley",
%%%     version         = "4",
%%%     date            = "28 August 1992",
%%%     time            = "10:24:38 199",
%%%     filename        = "chicagoa.bst",
%%%     address         = "Data Structuring Group
%%%                        Department of Computer Science
%%%                        University of Waterloo
%%%                        Waterloo, Ontario, Canada
%%%                        N2L 3G1",
%%%     telephone       = "(519) 885-1211",
%%%     FAX             = "(519) 885-1208",
%%%     checksum        = "62428 1684 5231 38311",
%%%     email           = "gnpaulle@bluebox.uwaterloo.ca",
%%%     codetable       = "ISO/ASCII",
%%%     keywords        = "",
%%%     supported       = "yes",
%%%     abstract        = "A BibTeX bibliography style that follows the
%%%                        `B' reference style of the 13th Edition of
%%%                        the Chicago Manual of Style. Supports annotated
%%%                        bibliographies through the BibTeX field
%%%                        `annotation'. A detailed feature list is given
%%%                        below.",
%%%     docstring       = "The checksum field above contains a CRC-16
%%%                        checksum as the first value, followed by the
%%%                        equivalent of the standard UNIX wc (word
%%%                        count) utility output of lines, words, and
%%%                        characters.  This is produced by Robert
%%%                        Solovay's checksum utility.",
%%%  }
%%% ====================================================================
%
% WSC bibtex style.  Modified from chicagoa.bst to conform to Winter
%  Simulation Conference standards.
% last modified: July 10, 2000, S. Chick
% last modified: Feb 13, 2001, S. Chick.  More compatible with WSC guidelines
% Last modified: Feb 09, 2011, J. Himmelspach. Some adaptions toward 16th edition of Chicago Style
% =====================================================================
%
%%% ====================================================================
%
% "Chicago" BibTeX style, chicagoa.bst - Chicago style with annotations
% =====================================================================
%
% BibTeX `chicagoa' style file for BibTeX version 0.99c, LaTeX version 2.09
% Place it in a file called chicagoa.bst in the BibTeX search path.
% You need to include chicago.sty as a \documentstyle option.
% (Placing it in the same directory as the LaTeX document should also work.)
% This "chicago" style is based on newapa.bst (American Psych. Assoc.)
% found at ymir.claremont.edu. Annotation support added 07/09/92.
% Modifications are loosely based on Chicago Manual of Style.
%
%   Citation format: (author-last-name year)
%             (author-last-name and author-last-name year)
%             (author-last-name, author-last-name, and author-last-name year)
%             (author-last-name et al. year)
%             (author-last-name)
%             author-last-name (year)
%             (author-last-name and author-last-name)
%             (author-last-name et al.)
%             (year) or (year,year)
%             year or year,year
%
%   Reference list ordering: alphabetical by author or whatever passes
%    for author in the absence of one.
%
% This BibTeX style has support for abbreviated author lists and for
%    year-only citations.  This is done by having the citations
%    actually look like
%
%    \citeauthoryear{full-author-info}{abbrev-author-info}{year}
%
% The LaTeX style has to have the following (or similar)
%
%     \let\@internalcite\cite
%     \def\fullcite{\def\citeauthoryear##1##2##3{##1, ##3}\@internalcite}
%     \def\fullciteA{\def\citeauthoryear##1##2##3{##1}\@internalcite}
%     \def\shortcite{\def\citeauthoryear##1##2##3{##2, ##3}\@internalcite}
%     \def\shortciteA{\def\citeauthoryear##1##2##3{##2}\@internalcite}
%     \def\citeyear{\def\citeauthoryear##1##2##3{##3}\@internalcite}
%
% These TeX macro definitions are found in chicago.sty. Additional
% commands to manipulate different components of a citation can be defined
% so that, for example, you can list author's names without parentheses
% if using a citation as a noun or object in a sentence.
%
% This file was originally copied from newapa.bst at ymir.claremont.edu.
%
% Features of chicago.bst:
% =======================
%
% - supports an "annotation" field for annotated bibliographies.
% - full names used in citations, but abbreviated citations are available
%   (see above)
% - if an entry has a "month", then the month and year are also printed
%   as part of that bibitem.
% - all conjunctions use "and" instead of "\&"
% - major modification from Chicago Manual of Style (13th ed.) is that
%   only the first author in a reference appears last name first-
%   additional authors appear as J. Q. Public.
% - pages are listed as "pp. xx-xx" in all entry types except
%   article entries.
% - book, inbook, and manual use "location: publisher" (or organization)
%   for address and publisher. All other types list publishers separately.
% - "pp." are used to identify page numbers for all entry types except
%   articles.
% - organization is used as a citation label if neither author nor editor
%   is present (for manuals).
% - "et al." is used for long author and editor lists, or when "others"
%   is used.
%
% Modifications and bug fixes from newapa.bst:
% ===========================================
%
%   - added month, year to bib entries if month is present
%   - fixed bug with In proceedings, added necessary comma after title
%   - all conjunctions changed to "and" from "\&"
%   - fixed bug with author labels in my.full.label: "et al." now is
%        generated when "others" is an author name
%   - major modification from Chicago Manual of Style (13th ed.) is that
%     only the first author in a reference appears last name first-
%     additional authors appear as J. Q. Public.
%   - pages are listed as "pp. xx-xx" in all entry types except
%     article entries. Unnecessary (IMHO) "()" around page numbers
%     were removed, and page numbers now don't end with a period.
%   - created chicago.sty for use with this bibstyle (required).
%   - fixed bugs in FUNCTION {format.vol.num.pages} for missing volume,
%     number, and /or pages. Renamed to format.jour.vol.
%   - fixed bug in formatting booktitles: additional period an error if
%     book has a volume.
%   - fixed bug: editors usually given redundant period before next clause
%     (format.editors.dot) removed.
%   - added label support for organizations, if both author and editor
%     are missing (from alpha.bst). If organization is too long, then
%     the key field is used for abbreviated citations.
%   - In proceedings or books of several volumes, no comma was written
%     between the "Volume x" and the page numbers (this was intentional
%     in newapa.bst). Fixed.
%   - Some journals may not have volumes/numbers, only month/year (eg.
%     IEEE Computer). Fixed bug in article style that assumed volume/number
%     was always present.
%
% Original documentation for newapa.sty:
% =====================================
%
% This version was made by modifying the master file made by
% Oren Patashnik (PATASHNIK@SCORE.STANFORD.EDU), and the 'named' BibTeX
% style of Peter F. Patel-Schneider.
%
% Copyright (C) 1985, all rights reserved.
% Copying of this file is authorized only if either
% (1) you make absolutely no changes to your copy, including name, or
% (2) if you do make changes, you name it something other than 'newapa.bst'.
% There are undoubtably bugs in this style.  If you make bug fixes,
% improvements, etc.  please let me know.  My e-mail address is:
%    spencer@cgrg.ohio.state.edu or 71160.3141@compuserve.com
%
% This style was made from 'plain.bst', 'named.bst', and 'apalike.bst',
% with lots of tweaking to make it look like APA style, along with tips
% from Young Ryu and Brian Reiser's modifications of 'apalike.bst'.

ENTRY
  { address
    annotation
    author
    booktitle
    chapter
    edition
    editor
    howpublished
    institution
    journal
    key
    month
    note
    number
    organization
    pages
    publisher
    school
    series
    title
    type
    volume
    year
  }
  {}
  { label.year extra.label sort.year sort.label }

INTEGERS { output.state before.all mid.sentence after.sentence
after.block }

FUNCTION {init.state.consts} { #0 'before.all :=
  #1 'mid.sentence :=
  #2 'after.sentence :=
  #3 'after.block :=
}

STRINGS { s t u }

FUNCTION {output.nonnull} { 's :=
  output.state mid.sentence =
    { ", " * write$ }
    { output.state after.block =
    { add.period$ write$
      newline$
      "\newblock " write$
    }
    { output.state before.all =
        'write$
        { add.period$ " " * write$ }
      if$
    }
      if$
      mid.sentence 'output.state :=
    }
  if$
  s
}

% Use a colon to separate output. Used only for address/publisher
% combination in book/inbook types, address/institution for manuals,
% and organization:publisher for proceedings (inproceedings).
%
FUNCTION {output.nonnull.colon} { 's :=
  output.state mid.sentence =
    { ": " * write$ }
    { output.state after.block =
    { add.period$ write$
      newline$
      "\newblock " write$
    }
    { output.state before.all =
        'write$
        { add.period$ " " * write$ }
      if$
    }
      if$
      mid.sentence 'output.state :=
    }
  if$
  s
}

FUNCTION {output} { duplicate$ empty$
    'pop$
    'output.nonnull
  if$
}

FUNCTION {output.colon} { duplicate$ empty$
    'pop$
    'output.nonnull.colon
  if$
}

FUNCTION {output.check} { 't :=
  duplicate$ empty$
    { pop$ "empty " t * " in " * cite$ * warning$ }
    'output.nonnull
  if$
}

FUNCTION {output.check.colon} { 't :=
  duplicate$ empty$
    { pop$ "empty " t * " in " * cite$ * warning$ }
    'output.nonnull.colon
  if$
}

%FUNCTION {output.year.check}
%{ year empty$
%     { "empty year in " cite$ * warning$ }
%     { write$
%        " (" year * extra.label *
%       month empty$
%          { ")" * }
%          { ", " * month * ")" * }
%       if$
%       mid.sentence 'output.state :=
%     }
%  if$
%}

FUNCTION {output.year.check} { year empty$
     { "empty year in " cite$ * warning$ }
     { write$
        " " year * extra.label *  % SEC
       month empty$
          { "." * }    % SEC
          { ", " * month * "." * }  % SEC
       if$
       mid.sentence 'output.state :=
     }
  if$
}


FUNCTION {fin.entry} { add.period$
  write$
  newline$
}

FUNCTION {new.block} { output.state before.all =
    'skip$
    { after.block 'output.state := }
  if$
}

FUNCTION {new.sentence} { output.state after.block =
    'skip$
    { output.state before.all =
    'skip$
    { after.sentence 'output.state := }
      if$
    }
  if$
}

FUNCTION {not} {   { #0 }
    { #1 }
  if$
}

FUNCTION {and} {   'skip$
    { pop$ #0 }
  if$
}

FUNCTION {or} {   { pop$ #1 }
    'skip$
  if$
}

FUNCTION {new.block.checka} { empty$
    'skip$
    'new.block
  if$
}

FUNCTION {new.block.checkb} { empty$
  swap$ empty$
  and
    'skip$
    'new.block
  if$
}

FUNCTION {new.sentence.checka} { empty$
    'skip$
    'new.sentence
  if$
}

FUNCTION {new.sentence.checkb} { empty$
  swap$ empty$
  and
    'skip$
    'new.sentence
  if$
}

FUNCTION {field.or.null} { duplicate$ empty$
    { pop$ "" }
    'skip$
  if$
}

%
% Emphasize the top string on the stack.
%
FUNCTION {emphasize} { duplicate$ empty$
    { pop$ "" }
    { "{\em " swap$ * "}" * }
  if$
}

%
% Emphasize the top string on the stack, but add a trailing space.
%
FUNCTION {emphasize.space} { duplicate$ empty$
    { pop$ "" }
    { "{\em " swap$ * "\/}" * }
  if$
}

INTEGERS { nameptr namesleft numnames }
%
% Format bibliographical entries with the first author last name first,
% and subsequent authors with initials followed by last name.
% All names are formatted in this routine.
%
FUNCTION {format.names} { 's :=
  #1 'nameptr :=               % nameptr = 1;
  s num.names$ 'numnames :=    % numnames = num.name$(s);
  numnames 'namesleft :=
    { namesleft #0 > }

    { nameptr #1 =
        {s nameptr "{vv~}{ll}{, jj}{, f.}" format.name$ 't := }
        {s nameptr "{f.~}{vv~}{ll}{, jj}" format.name$ 't := }
      if$
      nameptr #1 >
        { namesleft #1 >
              { ", " * t * }   % SEC was a comma
              { numnames #2 >
                  { "" * }   % SEC removed a comma
                  'skip$
                if$
                t "others" =
                    { " et~al." * }
                    { ", and " * t * "." *} % from Chicago Manual of Style % SEC added , and "." *
                  if$
               }
               if$
             }
            't
        if$
        nameptr #1 + 'nameptr :=          % nameptr += 1;
        namesleft #1 - 'namesleft :=      % namesleft =- 1;
    }

  while$
}

% Compute the label based on the names passed
FUNCTION {my.full.label} { 's :=
  #1 'nameptr :=               % nameptr = 1;
  s num.names$ 'numnames :=    % numnames = num.name$(s);
  numnames 'namesleft :=
    { namesleft #0 > }

    { s nameptr "{vv~}{ll}" format.name$ 't :=  % get the next name
      nameptr #1 >
        { namesleft #1 >
              { ", " * t * }
              { numnames #2 >
                  { "," * }
                  'skip$
                if$
                t "others" =
                    { " et~al." * }
                    { " and " * t * } % from Chicago Manual of Style
                  if$
               }
               if$
             }
            't
        if$
        nameptr #1 + 'nameptr :=          % nameptr += 1;
        namesleft #1 - 'namesleft :=      % namesleft =- 1;
    }
  while$
}

FUNCTION {format.names.fml}
%
% Format names in "familiar" format, with first initial followed by
% last name. Like format.names, ALL names are formatted.
%
{ 's :=
  #1 'nameptr :=               % nameptr = 1;
  s num.names$ 'numnames :=    % numnames = num.name$(s);
  numnames 'namesleft :=
    { namesleft #0 > }

    { s nameptr "{f.~}{vv~}{ll}{, jj}" format.name$ 't :=

      nameptr #1 >
        { namesleft #1 >
              { ", " * t * }
               { numnames #2 >
                    { "," * }
                    'skip$
                  if$
                  t "others" =
                        { " et~al." * }
                        { " and " * t * }
%                       { " \& " * t * }
                      if$
                }
               if$
             }
            't
        if$
        nameptr #1 + 'nameptr :=          % nameptr += 1;
        namesleft #1 - 'namesleft :=      % namesleft =- 1;
    }
  while$
}

FUNCTION {format.annotation} { annotation empty$
 { "" }
 { " \begin{quotation}\noindent "
   annotation
   * " \end{quotation} " *
 }
      if$
}

FUNCTION {format.authors} { author empty$
    { "" }
    { author format.names }
  if$
}


FUNCTION {format.misc.authors} { author empty$
    { "" }
    { author }
  if$
}


FUNCTION {format.key} { empty$
    { key field.or.null }
    { "" }
  if$
}

%
% Format editor names for use in the "in" types: inbook, incollection,
% inproceedings: first initial, then last names. When editors are the
% LABEL for an entry, then format.editor is used which lists editors
% by last name first.
%
%SEC: removed
%      editor num.names$ #1 >
%    { " (Eds.)" * }
%    { " (Ed.)" * }
%      if$
FUNCTION {format.editors.fml} { editor empty$
    { "" }
    { editor format.names.fml
    }
  if$
}

%
% Format editor names for use in labels, last names first.
%
FUNCTION {format.editors} { editor empty$
    { "" }
    { editor format.names
      editor num.names$ #1 >
    { " (Eds.)" * }
    { " (Ed.)" * }
      if$
    }
  if$
}

%FUNCTION {format.title} { title empty$
    %{ "" }
    %{ title "t" change.case$ }
  %if$
%}

FUNCTION {format.title} { title empty$
    { "" }
    { "``" title * "''" * } %  "`` " * "''" *     "t" change.case$
  if$
}

FUNCTION {n.dashify} { 't :=
  ""
    { t empty$ not }
    { t #1 #1 substring$ "-" =
    { t #1 #2 substring$ "--" = not
        { "--" *
          t #2 global.max$ substring$ 't :=
        }
        {   { t #1 #1 substring$ "-" = }
        { "-" *
          t #2 global.max$ substring$ 't :=
        }
          while$
        }
      if$
    }
    { t #1 #1 substring$ *
      t #2 global.max$ substring$ 't :=
    }
      if$
    }
  while$
}

FUNCTION {format.btitle} { edition empty$
  { title emphasize }
  { title empty$
    { title emphasize }
    { volume empty$     % gnp - check for volume, then don't need period
       { "{\em " title * "\/}. " * edition * " ed" * "." * }
       { "{\em " title * "\/}. " * edition * " ed" * }
%SEC       { "{\em " title * "\/} (" * edition * " ed.)" * "." * }
%SEC       { "{\em " title * "\/} (" * edition * " ed.)" * }
      if$
    }
    if$
  }
  if$
}

FUNCTION {format.emphasize.booktitle} { edition empty$
  { booktitle emphasize }
  { booktitle empty$
    { booktitle emphasize }
    { volume empty$    % gnp - extra period an error if book has a volume
        { "{\em " booktitle * "\/} (" * edition * " ed.)" * "." *}
        { "{\em " booktitle * "\/} (" * edition * " ed.)" * }
      if$
      }
    if$
    }
  if$
  }


FUNCTION {tie.or.space.connect} { duplicate$ text.length$ #3 <
    { "~" }
    { " " }
  if$
  swap$ * *
}

FUNCTION {either.or.check} { empty$
    'pop$
    { "can't use both " swap$ * " fields in " * cite$ * warning$ }
  if$
}

FUNCTION {format.bvolume} { volume empty$
    { "" }
    { "Volume" volume tie.or.space.connect % gnp - changed to mixed case
      series empty$
        'skip$
        { " of " * series emphasize * }
      if$
      "volume and number" number either.or.check
    }
  if$
}

FUNCTION {format.number.series} { volume empty$
    { number empty$
    { series field.or.null }
    { output.state mid.sentence =
        { "Number" } % gnp - changed to mixed case always
        { "Number" }
      if$
      number tie.or.space.connect
      series empty$
        { "there's a number but no series in " cite$ * warning$ }
        { " in " * series * }
      if$
    }
      if$
    }
    { "" }
  if$
}

INTEGERS { multiresult }

FUNCTION {multi.page.check} { 't :=
  #0 'multiresult :=
    { multiresult not
      t empty$ not
      and
    }
    { t #1 #1 substring$
      duplicate$ "-" =
      swap$ duplicate$ "," =
      swap$ "+" =
      or or
    { #1 'multiresult := }
    { t #2 global.max$ substring$ 't := }
      if$
    }
  while$
  multiresult
}

%SEC { "pp.\ " pages n.dashify tie.or.space.connect } % gnp - removed ()% S
%SEC { "pp.\ " pages tie.or.space.connect }
FUNCTION {format.pages} { pages empty$
  { "" }
  { pages multi.page.check
 { "" pages n.dashify tie.or.space.connect } % gnp - removed ()% S
 { "" pages tie.or.space.connect }
    if$
  }
  if$
}

% By Young (and Spencer)
% GNP - fixed bugs with missing volume, number, and/or pages
%
% Format journal, volume, number, pages for article types.
%
FUNCTION {format.jour.vol} { journal empty$
    { "no journal in " cite$ * warning$
      "" }
    { journal emphasize.space }
    if$
  number empty$
    { volume empty$
       { "no number and no volume in " cite$ * warning$
         "" * }
%SEC       { "~{\em " * Volume * "}" * }
       { "~" * Volume * }
      if$
    }
    { volume empty$
      {"no volume for " cite$ * warning$
       "~(" * number * ")" * }
      { "~" *
%SEC        volume emphasize.space
        volume
        " (" * number * ")" * * }   % added space before '(' per J. Wilson suggestion
      if$
    }
  if$
  pages empty$
    {"page numbers missing in " cite$ * warning$
     "" * } % gnp - place a null string on the stack for output
    { duplicate$ empty$
      { pop$ format.pages }
%      { ": " *  pages n.dashify * } % SEC: changed , to :.  Also, gnp - removed pp. for articles
   { number empty$     % SEC
        { ":" *  pages n.dashify * }  % SEC
        { ": " *  pages n.dashify * } % SEC
       if$   % SEC
   }    % SEC
      if$
    }
  if$
}

FUNCTION {format.chapter.pages} { chapter empty$
    'format.pages
    { type empty$
        { "Chapter" } % gnp - changed to mixed case
        { type "t" change.case$ }
      if$
      chapter tie.or.space.connect
      pages empty$
        {"page numbers missing in " cite$ * warning$} % gnp - added check
        { ", " * format.pages * }
      if$
    }
  if$
}

% SEC: was    { "In " format.editors.fml * ", " * format.emphasize.booktitle * }
FUNCTION {format.in.ed.booktitle} { booktitle empty$
  { "" }
  { editor empty$
    { "In " format.emphasize.booktitle * }
    { "In " format.emphasize.booktitle * ", edited by\ " * format.editors.fml * }
    if$
  }
  if$
}

FUNCTION {format.thesis.type} { type empty$
    'skip$
    { pop$
      type "t" change.case$
    }
  if$
}

FUNCTION {format.tr.number} { type empty$
    { "Technical Report" }
    'type
  if$
  number empty$
    { "t" change.case$ }
    { number tie.or.space.connect }
  if$
}

FUNCTION {format.article.crossref} { "See"
  "\citeN{" * crossref * "}" *
}

FUNCTION {format.crossref.editor} { editor #1 "{vv~}{ll}"
format.name$
  editor num.names$ duplicate$
  #2 >
    { pop$ " et~al." * }
    { #2 <
    'skip$
    { editor #2 "{ff }{vv }{ll}{ jj}" format.name$ "others" =
        { " et~al." * }
        { " and " * editor #2 "{vv~}{ll}" format.name$ * }
      if$
    }
      if$
    }
  if$
}

FUNCTION {format.book.crossref} { volume empty$
    { "empty volume in " cite$ * "'s crossref of " * crossref * warning$
      "In "
    }
    { "Volume" volume tie.or.space.connect % gnp - changed to mixed case
      " of " *
    }
  if$
  editor empty$
  editor field.or.null author field.or.null =
  or
    { key empty$
    { series empty$
        { "need editor, key, or series for " cite$ * " to crossref " *
          crossref * warning$
          "" *
        }
        { "{\em " * series * "\/}" * }
      if$
    }
    { key * }
      if$
    }
    { format.crossref.editor * }
  if$
  " \citeN{" * crossref * "}" *
}

FUNCTION {format.incoll.inproc.crossref} { "See"
  " \citeN{" * crossref * "}" *
}

% format.lab.names:
%
% determines "short" names for the abbreviated author information.
% "Long" labels are created in calc.label, using the routine my.full.label
% to format author and editor fields.
%
% There are 4 cases for labels.   (n=3 in the example)
% a) one author             Foo
% b) one to n               Foo, Bar and Baz
% c) use of "and others"    Foo, Bar et al.
% d) more than n            Foo et al.
%
FUNCTION {format.lab.names} { 's :=
  s num.names$ 'numnames :=
  numnames #2 >    % change number to number of others allowed before
     % forcing "et al".
    { s #1 "{vv~}{ll}" format.name$ " et~al." * }
    {
      numnames #1 - 'namesleft :=
      #2 'nameptr :=
      s #1 "{vv~}{ll}" format.name$
 { namesleft #0 > }
 { nameptr numnames =
     { s nameptr "{ff }{vv }{ll}{ jj}" format.name$ "others" =
  { " et~al." * }
  { " and " * s nameptr "{vv~}{ll}" format.name$ * }
       if$
     }
     { ", " * s nameptr "{vv~}{ll}" format.name$ * }
   if$
   nameptr #1 + 'nameptr :=
   namesleft #1 - 'namesleft :=
 }
      while$
    }
  if$
}

FUNCTION {author.key.label} { author empty$
    { key empty$
          { "no key, author in " cite$ * warning$
            cite$ #1 #3 substring$ }
         'key
      if$
    }
    { author format.lab.names }
  if$
}

FUNCTION {editor.key.label} { editor empty$
    { key empty$
          { "no key, editor in " cite$ * warning$
            cite$ #1 #3 substring$ }
          'key
        if$
     }
     { editor format.lab.names }
  if$
}

FUNCTION {author.key.organization.label}
%
% added - gnp. Provide label formatting by organization if author is null.
%
{ author empty$
    { organization empty$
 { key empty$
     { "no key, author or organization in " cite$ * warning$
              cite$ #1 #3 substring$ }
     'key
   if$
 }
        { organization }
      if$
    }
    { author format.lab.names }
  if$
}

FUNCTION {editor.key.organization.label}
%
% added - gnp. Provide label formatting by organization if editor is null.
%
{ editor empty$
    { organization empty$
 { key empty$
     { "no key, editor or organization in " cite$ * warning$
              cite$ #1 #3 substring$ }
     'key
   if$
 }
        { organization }
      if$
    }
    { editor format.lab.names }
  if$
}

FUNCTION {author.editor.key.label} { author empty$
    { editor empty$
          { key empty$
               { "no key, author, or editor in " cite$ * warning$
                 cite$ #1 #3 substring$ }
             'key
           if$
         }
          { editor format.lab.names }
      if$
    }
    { author format.lab.names }
  if$
}

FUNCTION {misc.author.label} { author empty$
    { key empty$
          { "no key, author in " cite$ * warning$
            cite$ #1 #3 substring$ }
          'key
        if$
     }
     {
		    author } %"test " author * "super " *} %format.lab.names ;
  if$
}

FUNCTION {calc.label}
%
% Changed - GNP. See also author.organization.sort, editor.organization.sort
% Form label for BibTeX entry. The classification of which fields are used
% for which type of entry (book, inbook, etc.) are taken from alpha.bst.
% The change here from newapa is to also include organization as a
% citation label if author or editor is missing.
%
{ % some entry types require a special treatment
   type$ "book" =
   type$ "inbook" =
   or
     'author.editor.key.label
		 %else (!= book | inbook)
   { type$ "proceedings" =
        'editor.key.organization.label
		 	  %else != proceedings
      { type$ "manual" =
           'author.key.organization.label
					{ type$ "misc" =
					    'misc.author.label
              'author.key.label
						if$
					}
         if$ % manual
      }
      if$ % type$ "proceedings" =
   }
   if$ % or


  author empty$  % generate the full label citation information.
    { editor empty$
        { organization empty$
           { "no author, editor, or organization in " cite$ * warning$
             "??" }
           { organization }
           if$
        }
        { editor my.full.label }
        if$
    }
    { author my.full.label }
  if$

% leave label on the stack, to be popped when required.

  "}{" * swap$ * "}{" *
%  year field.or.null purify$ #-1 #4 substring$ *
%
% save the year for sort processing afterwards (adding a, b, c, etc.)
%
  year field.or.null purify$ #-1 #4 substring$
  'label.year :=
}

FUNCTION {output.bibitem} { newline$

  "\bibitem[\protect\citeauthoryear{" write$
  calc.label write$
  sort.year write$
  "}]{" write$

  cite$ write$
  "}" write$
  newline$
  ""
  before.all 'output.state :=
}

FUNCTION {article} { output.bibitem
  format.authors
  "author" output.check
  author format.key output          % added
  output.year.check                 % added
  new.block
  format.title
  "title" output.check
  new.block
  crossref missing$
    { format.jour.vol output
    }
    { format.article.crossref output.nonnull
      format.pages output
    }
  if$
  new.block
  note output
  fin.entry
  format.annotation write$
  newline$
}

FUNCTION {book} { output.bibitem
  author empty$
    { format.editors
   "author and editor" output.check }
    { format.authors
   output.nonnull
      crossref missing$
     { "author and editor" editor either.or.check }
     'skip$
      if$
    }
  if$
  output.year.check       % added
  new.block
  format.btitle
  "title" output.check
  crossref missing$
    { format.bvolume output
      new.block
      format.number.series output
      new.sentence
      address output
      publisher "publisher" output.check.colon
    }
    { new.block
      format.book.crossref output.nonnull
    }
  if$
  new.block
  note output
  fin.entry
  format.annotation write$
  newline$
}

FUNCTION {booklet} { output.bibitem
  format.authors output
  author format.key output          % added
  output.year.check                 % added
  new.block
  format.title
  "title" output.check
  new.block
  howpublished output
  address output
  new.block
  note output
  fin.entry
  format.annotation write$
}

FUNCTION {inbook} { output.bibitem
  author empty$
    { format.editors
      "author and editor" output.check
    }
    { format.authors output.nonnull
      crossref missing$
    { "author and editor" editor either.or.check }
    'skip$
      if$
    }
  if$
  output.year.check                 % added
  new.block
  format.btitle
  "title" output.check
  crossref missing$
    { format.bvolume output
      format.chapter.pages
      "chapter and pages" output.check
      new.block
      format.number.series output
      new.sentence
      address output
      publisher
      "publisher" output.check.colon
    }
    { format.chapter.pages "chapter and pages" output.check
      new.block
      format.book.crossref output.nonnull
    }
  if$
  new.block
  note output
  fin.entry
  format.annotation write$
}

FUNCTION {incollection} { output.bibitem
  format.authors
  "author" output.check
  author format.key output       % added
  output.year.check              % added
  new.block
  format.title
  "title" output.check
  new.block
  crossref missing$
  { format.in.ed.booktitle
    "booktitle" output.check
    format.bvolume output
    format.number.series output
    format.chapter.pages output % gnp - was special.output.nonnull
%                                 left out comma before page numbers
    new.sentence
    address output
    publisher "publisher" output.check.colon
  }
  { format.incoll.inproc.crossref
 output.nonnull
    format.chapter.pages output
  }
  if$
  new.block
  note output
  fin.entry
  format.annotation write$
}

%SEC: modified this quite a bit.
FUNCTION {inproceedings} { output.bibitem
  format.authors
  "author" output.check
  author format.key output            % added
  output.year.check                   % added
  new.block
  format.title
  "title" output.check
  new.block
  crossref missing$
    { format.in.ed.booktitle
   "booktitle" output.check
      format.bvolume output
      format.number.series output
      format.pages output
  organization address new.block.checkb
% Reversed the order of "address" and "organization", added the ":".
  address output
  organization "organization" output.check.colon
      publisher output.colon
    }
    { format.incoll.inproc.crossref output.nonnull
      format.pages output
    }
  if$
  new.block
  note output
  fin.entry
  format.annotation write$
}


FUNCTION {conference} { inproceedings }

FUNCTION {manual} { output.bibitem
  author empty$
    { editor empty$
      { organization "organization" output.check
        organization format.key output }  % if all else fails, use key
      { format.editors "author and editor" output.check }
      if$
    }
    { format.authors output.nonnull }
    if$
  output.year.check                 % added
  new.block
  format.btitle
  "title" output.check
  organization address new.block.checkb
% Reversed the order of "address" and "organization", added the ":".
  address output
  organization "organization" output.check.colon
%  address output
%  ":" output
%  organization output
  new.block
  note output
  fin.entry
  format.annotation write$
}

FUNCTION {mastersthesis} { output.bibitem
  format.authors
  "author" output.check
  author format.key output          % added
  output.year.check                 % added
  new.block
  format.title
  "title" output.check
  new.block
  "Master's thesis" format.thesis.type output.nonnull
  school "school" output.check
  address output
  new.block
  note output
  fin.entry
  format.annotation write$
}

FUNCTION {misc} { output.bibitem
  format.misc.authors output                      % replaced: format.authors output
  author format.key output            % added
  output.year.check                   % added
  title howpublished new.block.checkb
  format.title output
  new.block
  howpublished output
  new.block
  note output
  fin.entry
  format.annotation write$
}

FUNCTION {phdthesis} { output.bibitem
  format.authors
  "author" output.check
  author format.key output            % added
  output.year.check                   % added
  new.block
  format.btitle
  "title" output.check
  new.block
  "Ph.\ D. thesis" format.thesis.type output.nonnull
  school "school" output.check
  address output
  new.block
  note output
  fin.entry
  format.annotation write$
}

FUNCTION {proceedings} { output.bibitem
  editor empty$
    { organization output
      organization format.key output }  % gnp - changed from author format.key
    { format.editors output.nonnull }
  if$
% author format.key output             % gnp - removed (should be either
%                                        editor or organization
  output.year.check                    % added (newapa)
  new.block
  format.btitle
  "title" output.check
  format.bvolume output
  format.number.series output
  address output
  new.sentence
  organization output
  publisher output.colon
  new.block
  note output
  fin.entry
  format.annotation write$
}

FUNCTION {techreport} { output.bibitem
  format.authors
  "author" output.check
  author format.key output             % added
  output.year.check                    % added
  new.block
  format.title
  "title" output.check
  new.block
  format.tr.number output.nonnull
  institution
  "institution" output.check
  address output
  new.block
  note output
  fin.entry
  format.annotation write$
}

FUNCTION {unpublished} { output.bibitem
  format.authors
  "author" output.check
  author format.key output              % added
  output.year.check                      % added
  new.block
  format.title
  "title" output.check
  new.block
  note "note" output.check
  fin.entry
  format.annotation write$
}

FUNCTION {default.type} { misc }

MACRO {jan} {"January"}

MACRO {feb} {"February"}

MACRO {mar} {"March"}

MACRO {apr} {"April"}

MACRO {may} {"May"}

MACRO {jun} {"June"}

MACRO {jul} {"July"}

MACRO {aug} {"August"}

MACRO {sep} {"September"}

MACRO {oct} {"October"}

MACRO {nov} {"November"}

MACRO {dec} {"December"}

MACRO {acmcs} {"ACM Computing Surveys"}

MACRO {acta} {"Acta Informatica"}

MACRO {ai} {"Artificial Intelligence"}

MACRO {cacm} {"Communications of the ACM"}

MACRO {ibmjrd} {"IBM Journal of Research and Development"}

MACRO {ibmsj} {"IBM Systems Journal"}

MACRO {ieeese} {"IEEE Transactions on Software Engineering"}

MACRO {ieeetc} {"IEEE Transactions on Computers"}

MACRO {ieeetcad}
 {"IEEE Transactions on Computer-Aided Design of Integrated Circuits"}

MACRO {ipl} {"Information Processing Letters"}

MACRO {jacm} {"Journal of the ACM"}

MACRO {jcss} {"Journal of Computer and System Sciences"}

MACRO {scp} {"Science of Computer Programming"}

MACRO {sicomp} {"SIAM Journal on Computing"}

MACRO {tocs} {"ACM Transactions on Computer Systems"}

MACRO {tods} {"ACM Transactions on Database Systems"}

MACRO {tog} {"ACM Transactions on Graphics"}

MACRO {toms} {"ACM Transactions on Mathematical Software"}

MACRO {toois} {"ACM Transactions on Office Information Systems"}

MACRO {tcs} {"Theoretical Computer Science"}

READ

FUNCTION {sortify} { purify$
  "l" change.case$
}

INTEGERS { len }

FUNCTION {chop.word} { 's :=
  'len :=
  s #1 len substring$ =
    { s len #1 + global.max$ substring$ }
    's
  if$
}



FUNCTION {sort.format.names} { 's :=
  #1 'nameptr :=
  ""
  s num.names$ 'numnames :=
  numnames 'namesleft :=
    { namesleft #0 > }
    { nameptr #1 >
          { "   " * }
         'skip$
      if$
      s nameptr "{vv{ } }{ll{ }}{  f{ }}{  jj{ }}" format.name$ 't :=
      nameptr numnames = t "others" = and
          { " et~al" * }
          { t sortify * }
      if$
      nameptr #1 + 'nameptr :=
      namesleft #1 - 'namesleft :=
    }
  while$
}

FUNCTION {sort.format.title} { 't :=
  "A " #2
    "An " #3
      "The " #4 t chop.word
    chop.word
  chop.word
  sortify
  #1 global.max$ substring$
}

FUNCTION {author.sort} { author empty$
    { key empty$
         { "to sort, need author or key in " cite$ * warning$
           "" }
         { key sortify }
      if$
    }
    { author sort.format.names }
  if$
}

FUNCTION {editor.sort} { editor empty$
    { key empty$
         { "to sort, need editor or key in " cite$ * warning$
           ""
         }
         { key sortify }
      if$
    }
    { editor sort.format.names }
  if$
}

FUNCTION {author.editor.sort} { author empty$
    { "missing author in " cite$ * warning$
      editor empty$
         { key empty$
             { "to sort, need author, editor, or key in " cite$ * warning$
               ""
             }
             { key sortify }
           if$
         }
         { editor sort.format.names }
      if$
    }
    { author sort.format.names }
  if$
}

FUNCTION {author.organization.sort}
%
% added - GNP. Stack author or organization for sorting (from alpha.bst).
% Unlike alpha.bst, we need entire names, not abbreviations
%
{ author empty$
    { organization empty$
 { key empty$
     { "to sort, need author, organization, or key in " cite$ * warning$
       ""
     }
     { key sortify }
   if$
 }
 { organization sortify }
      if$
    }
    { author sort.format.names }
  if$
}

FUNCTION {editor.organization.sort}
%
% added - GNP. Stack editor or organization for sorting (from alpha.bst).
% Unlike alpha.bst, we need entire names, not abbreviations
%
{ editor empty$
    { organization empty$
 { key empty$
     { "to sort, need editor, organization, or key in " cite$ * warning$
       ""
     }
     { key sortify }
   if$
 }
 { organization sortify }
      if$
    }
    { editor sort.format.names }
  if$
}

FUNCTION {presort}
%
% Presort creates the bibentry's label via a call to calc.label, and then
% sorts the entries based on entry type. Chicago.bst adds support for
% including organizations as the sort key; the following is stolen from
% alpha.bst.
%
{ calc.label sortify % recalculate bibitem label
  year field.or.null purify$ #-1 #4 substring$ * % add year
  "    "
  *
  type$ "book" =
  type$ "inbook" =
  or
    'author.editor.sort
    { type$ "proceedings" =
 'editor.organization.sort
 { type$ "manual" =
     'author.organization.sort
     'author.sort
   if$
 }
      if$
    }
  if$
  #1 entry.max$ substring$        % added for newapa
  'sort.label :=                  % added for newapa
  sort.label                      % added for newapa
  *
  "    "
  *
  title field.or.null
  sort.format.title
  *
  #1 entry.max$ substring$
  'sort.key$ :=
}

ITERATE {presort}

SORT             % by label, year, author/editor, title

STRINGS { last.label next.extra }

INTEGERS { last.extra.num }

FUNCTION {initialize.extra.label.stuff} { #0 int.to.chr$
'last.label :=
  "" 'next.extra :=
  #0 'last.extra.num :=
}

FUNCTION {forward.pass}
%
% Pass through all entries, comparing current entry to last one.
% Need to concatenate year to the stack (done by calc.label) to determine
% if two entries are the same (see presort)
%
{ last.label
  calc.label year field.or.null purify$ #-1 #4 substring$ * % add year
  #1 entry.max$ substring$ =     % are they equal?
     { last.extra.num #1 + 'last.extra.num :=
       last.extra.num int.to.chr$ 'extra.label :=
     }
     { "a" chr.to.int$ 'last.extra.num :=
       "" 'extra.label :=
       calc.label year field.or.null purify$ #-1 #4 substring$ * % add year
       #1 entry.max$ substring$ 'last.label := % assign to last.label
     }
  if$
}

FUNCTION {reverse.pass} { next.extra "b" =
    { "a" 'extra.label := }
     'skip$
  if$
  label.year extra.label * 'sort.year :=
  extra.label 'next.extra :=
}

EXECUTE {initialize.extra.label.stuff}

ITERATE {forward.pass}

REVERSE {reverse.pass}

FUNCTION {bib.sort.order} { sort.label
  "    "
  *
  year field.or.null sortify
  *
  "    "
  *
  title field.or.null
  sort.format.title
  *
  #1 entry.max$ substring$
  'sort.key$ :=
}

ITERATE {bib.sort.order}

SORT             % by sort.label, year, title --- giving final bib. order.

FUNCTION {begin.bib}

{ preamble$ empty$
    'skip$
    { preamble$ write$ newline$ }
  if$
  "\begin{thebibliography}{}" write$ newline$
}


EXECUTE {begin.bib}

EXECUTE {init.state.consts}

ITERATE {call.type$}

FUNCTION {end.bib} { newline$
  "\end{thebibliography}" write$ newline$
}

EXECUTE {end.bib}

\end{filecontents*}

 %%%%%%%%%%%%%%%%%%%%%%%%%%%%%%%%%%%%%%%%%%%%%%%%%%%%%%%%%%%%%%%%%%%%%%%%%%%%%%%%%%%%%%%%%%%%%%%%%%%%%%%

\begin{filecontents*}{wscpaperproc.cls}
% %%%%%%%%%%%%%%%%%%%%%%%%%%%%%%%%%%%%%%%%%%%%%%%%%%%%%%%%%%%%
% 					Class file for WSC proceedings paper
%
%  Effective with WSC 09, the two column format was replaced with
%  a single column format
%
% Parameters (like names of editors) have to be setup using the wscsetup.sty style file.
%
% All settings valid for papers and poster abstract should be placed into wsc.sty .
% %%%%%%%%%%%%%%%%%%%%%%%%%%%%%%%%%%%%%%%%%%%%%%%%%%%%%%%%%%%%
\ProvidesClass{wscpaperproc}
\NeedsTeXFormat{LaTeX2e}
\DeclareOption*{\PassOptionsToClass{\CurrentOption}{article}}
\ProcessOptions
\LoadClass[11pt, twoside, onecolumn, letterpaper]{article}

% --------------------------------- REQUIRED PACKAGES ---------------------------------
\RequirePackage{wsc}
\end{filecontents*}

 %%%%%%%%%%%%%%%%%%%%%%%%%%%%%%%%%%%%%%%%%%%%%%%%%%%%%%%%%%%%%%%%%%%%%%%%%%%%%%%%%%%%%%%%%%%%%%%%%%%%%%%

\begin{filecontents*}{wscsetup.sty}
%*******************************************************************************
%*
%* ***WSC Configuration File - Base for LaTeX Papers for the WinterSim ***
%*
%  Note to editors: The setup parameters section has to be changed for each conference!
%*
%
% ---------------------------------------------------------------------------------------------------------------------
% Setup parameters (new in '11 ) [single location, not in tex file to be edited by authors]
%
% How to use:
% Add a macro per year at the end of the list. The history can be of use to check the editor lists
% of WSC paper citations.
% - The macro definition has the form: \def\editorsYYYY{}
% - The new macro has to be used in the \currentEditors macro which is used by the style: so you
%   have to replace the \editorsYYYY macro used in there by the new one
% Update the \currentYear macro to contain the current year
%
% You can search for the places the updates have to be applied to by using the search term
% "EDITOR".



% -------------- history of editors ------------------

\def\editors2000{J.~A.~Joines, R.~R.~Barton, K.~Kang, and P.~A.~Fishwick}
\def\editors2001{B.~A.~Peters, J.~S.~Smith, D.~J.~Medeiros, and M.~W.~Rohrer}
\def\editors2002{E.~Y\"ucesan, C.-H.~Chen, J.~L.~Snowdon, and J.~M.~Charnes}
\def\editors2003{S.~Chick, P.~J.~S\`anchez, D.~Ferrin, and D.~J.~Morrice}
\def\editors2004{R.~G.~Ingalls, M.~D.~Rossetti, J.~S.~Smith, and B.~A.~Peters}
\def\editors2005{M.~E.~Kuhl, N.~M.~Steiger, F.~B.~Armstrong, and J.~A.~Joines}
\def\editors2006{L.~F.~Perrone, F.~P.~Wieland, J.~Liu, B.~G.~Lawson, D.~M.~Nicol, and R.~M.~Fujimoto}
\def\editors2007{S.~G.~Henderson, B.~Biller, M.-H.~Hsieh, J.~Shortle, J.~D.~Tew, and R.~R.~Barton}
\def\editors2008{S.~J.~Mason, R.~R.~Hill, L.~M\"onch, O.~Rose, T.~Jefferson, J.~W.~Fowler}
\def\editors2009{M.~D.~Rossetti, R.~R.~Hill, B.~Johansson, A.~Dunkin and R.~G.~Ingalls}
\def\editors2010{B.~Johansson, S.~Jain, J.~Montoya-Torres, J.~Hugan, and E.~Y\"ucesan}
\def\editors2011{S.~Jain, R.~R. Creasey, J.~Himmelspach, K.~P.~White, and M.~Fu}
\def\editors2012{C.~Laroque, J.~Himmelspach, R.~Pasupathy, O.~Rose, and A.~M.~Uhrmacher}
\def\editors2013{R.~Pasupathy, S.-H.~Kim, A.~Tolk, R.~Hill, and M.~E.~Kuhl}
\def\editors2014{A.~Tolk, S.~D. Diallo, I.~O. Ryzhov, L.~Yilmaz, S.~Buckley, and J.~A. Miller}
\def\editors2015{L.~Yilmaz, W.~K~V.~Chan, I.~Moon, T.~M.~K.~Roeder, C.~Macal, and M.~D.~Rossetti}
\def\editors2016{T.~M.~K.~Roeder, P.~I.~Frazier, R.~Szechtman, E.~Zhou, T.~Huschka, and S.~E.~Chick}
\def\editors2017{W.~K.~V.~Chan, A.~D'Ambrogio, G.~Zacharewicz, N.~Mustafee, G.~Wainer, and E.~Page}
% EDITOR: add new definition above this line




% -------------- settings for the current year ------------------
\def\currentYear {2017}               % EDITOR: update with current year
\def\currentEditors {\editors2017} % EDITOR: update by current editor definition macro
\def\currentCaption {Proceedings of the \currentYear{} Winter Simulation Conference}

%
% End of setup parameters
% % ---------------------------------------------------------------------------------------------------------------------
%
\end{filecontents*}

 %%%%%%%%%%%%%%%%%%%%%%%%%%%%%%%%%%%%%%%%%%%%%%%%%%%%%%%%%%%%%%%%%%%%%%%%%%%%%%%%%%%%%%%%%%%%%%%%%%%%%%%

\begin{filecontents*}{wsc.sty}
%*******************************************************************************
%*
%* ***WSC Style File - Base for LaTeX Papers for the WinterSim ***
%*
% Contains basic style elements valid for papers and for poster abstracts
%	
%*
%
% HISTORY
%
% WSC 2012
%  * fixed author font size bug
% WSC 2011 - created this style file based on the templates for WSC 2010
%  * outdated fancyheadings is from now on replaced by fancyhdr
%  * introduced sections, added comments
%  * includes wscsetup.sty -> contains year specific information (editors, etc ...)
%  * modified page layout, line distances, paragraphs, itemize environments, etc to get closer to word
%      template style
%  * introduced command WSCpagesetup (author names) which setups the running header and many
%      other things previously needed in each paper's first lines of document code
%
%
% --------------------------------- REQUIRED PACKAGES ---------------------------------
% This package is required for setting up the running page header (author names)
\RequirePackage{fancyhdr} 	

% WSC parameter setup (editors, year, proceedings name)
\RequirePackage{wscsetup} 	

% select times font (by including this here it will be hard for authors to change, font will be used in the end in anyway)
\RequirePackage{times}

% package for displaying urls
\RequirePackage{url}
% select times font for the url package
\urlstyle{rm}

% ?
\def\draft{\overfullrule=6pt}

% --------------------------------- SPECIAL COMMANDS ---------------------------------
%Print an email address in the required style
\def\email#1{\href{mailto://#1}{#1}}

% No one needs to use the following commands, but if they are used they help to avoid the problem of using figure, table, and section instead of Figure, Table, and Section

%Print Figure followed by the mark generated for the label passed
\def\reffig#1{Figure~\ref{#1}}

%Print Table followed by the mark generated for the label passed
\def\reftab#1{Table~\ref{#1}}

%Print Section followed by the mark generated for the label passed
\def\refsec#1{Section~\ref{#1}}


% --------------------------------- PAGE SETUP  ---------------------------------

%\RequirePackage[nofoot,letterpaper,lmargin=1in,tmargin=1in,rmargin=1in,bmargin=1in]{geometry}

\setlength\footskip{75\p@}

%The width of the text area is 6.5 inches (16.0 cm)
\setlength\textwidth{6.5in}

%new for WSC-10
\setlength\textheight{8.75in} %9.0125in
%\setlength\textheight{8.76in}  %
%\headsep=12pt
\headsep=12pt
%\headheight = 11pt

%left and right margins are 1 inch (2.54 cm)
\setlength\oddsidemargin{0in}%{-0.2in}
\setlength\evensidemargin{0in}%{-0.1in}
%\setlength\oddsidemargin{-.25in} \setlength\evensidemargin{-.25in}

\setlength\marginparwidth {\z@}

%top margin is set to be 1 inch (2.54 cm)
%\setlength\topmargin{2.54cm}%{-4pc}%
%\setlength\topmargin{-2pc}
%\setlength\topmargin{-19pt}
\setlength\topmargin{-21.89pt}   % JRW Edit 2/12/13
\setlength\columnsep{.375in}

%\pdfpagewidth 8.5in
%\pdfpageheight 11.0in

% --------------------------------- FLOAT SETUP  ---------------------------------
% Setting the spaces around float (i.e, Figures, Tables etc ...)


% distance between two floats
\setlength{\floatsep}{12pt}
% dynamic variant (more flexibel) \setlength\floatsep    {12\p@\@plus 2\p@ minus 2pt}

% distance between float and text (above & below)
\setlength{\textfloatsep}{6pt}
%\setlength\textfloatsep{12\p@ \@plus 2\p@minus 2pt}

%\textfloatsep - space between last top float or first bottom float and the text
% space left on top and bottom of an in-text float.
\setlength{\intextsep}{6pt}
% \setlength\intextsep   {12\p@ \@plus 2\p@ minus 2pt}

% space above caption
\setlength{\abovecaptionskip}{6pt}
% space below caption
\setlength{\belowcaptionskip}{6pt}

\setcounter{dbltopnumber}{4}

\setcounter{topnumber}{4}

% settings for two column mode - distance between floats
\setlength{\dblfloatsep}{6pt}
% settings for two column mode - distance between float and text
\setlength{\dbltextfloatsep}{6pt}

% --------------------------------- FONT SETUP  ---------------------------------

%\font\titlepageheadfont=tii at 10pt

% --------------------------------- NAMES SETUP  ---------------------------------
\renewcommand{\refname}{REFERENCES}
\newcommand\pagename{Page}

% --------------------------------- SPECIAL WORDS
\newcommand{\BibTeX}{B{\small IB}$\!$\TeX}

% --------------------------------- HEADER SETUP  ---------------------------------
% --- header of first page (including proceedings caption and names of editors) ---
% moving up 1st header by approx 6mm to match word template
\def\firstPageHead {
  %increase the head height for the two lines editors/proceedings header (to avoid warning by fancyhdr)
  \setlength{\headheight}{26pt}
	%setup the header (left, italics containing proceedings caption and editors in 2nd row)
  \lhead{
  \fancyplain{
	  \textit{\currentCaption\\
      \currentEditors , eds.}}{}}
} % end of firstPageHead definition

%****WSC setup for Running Heads of LaTeX Papers from second page on***
\def\WSCpagesetup#1{

% setting up general page style
\pagestyle{fancyplain}

% setting up page style of first page
\thispagestyle{plain}
\firstPageHead{}

% setting up running header (authors) of subsequent pages
\chead{\fancyplain{}{\itshape #1}}

% setting up seperation parameters
%\headsep=72pt
\rhead{}
\cfoot{}
\renewcommand{\headrulewidth}{0pt} % (renewcommand needed in fancyhdr to remove top decorative line)
%\headrulewidth=0pt  % ("setlength" needed in fancyheading to remove top decorative line)

%%%%%%%%%%%%%%%%%%%%%%%%%%%%%%%%%%%%%%%%%%%%%%%%%%%%%%%%%%%%%%%%%%%%%%%%%%%%%%
%                                                                            %
%     THESE COMMANDS ARE REQUIRED TO WORK WITH WSC.BST TO MAKE BIBLIO     %
%                                                                            %
%%%%%%%%%%%%%%%%%%%%%%%%%%%%%%%%%%%%%%%%%%%%%%%%%%%%%%%%%%%%%%%%%%%%%%%%%%%%%%
\makeatletter
\let\@internalcite\cite
%
\def\cite{\def\@citeseppen{-1000}%
    \def\@cite##1##2{(##1\if@tempswa , ##2\fi)}%
    \def\citeauthoryear##1##2##3{##1 ##3}\@internalcite}
\def\citeNP{\def\@citeseppen{-1000}%
    \def\@cite##1##2{##1\if@tempswa , ##2\fi}%
    \def\citeauthoryear##1##2##3{##1 ##3}\@internalcite}
\def\citeN{\def\@citeseppen{-1000}%
%  Pierre L'Ecuyer's fix for multiple cite bug
%  Added by Paul J Sanchez on 4 October 2001
%   \def\@cite##1##2{##1\if@tempswa , ##2)\else{)}\fi}%
%   \def\citeauthoryear##1##2##3{##1 (##3}\@citedata}
    \def\@cite##1##2{##1\if@tempswa, ##2)\else{}\fi}%
    \def\citeauthoryear##1##2##3{##1 (##3)}\@citedata}
\def\citeA{\def\@citeseppen{-1000}%
    \def\@cite##1##2{(##1\if@tempswa , ##2\fi)}%
    \def\citeauthoryear##1##2##3{##1}\@internalcite}
\def\citeANP{\def\@citeseppen{-1000}%
    \def\@cite##1##2{##1\if@tempswa , ##2\fi}%
    \def\citeauthoryear##1##2##3{##1}\@internalcite}
%
\def\shortcite{\def\@citeseppen{-1000}%
    \def\@cite##1##2{(##1\if@tempswa , ##2\fi)}%
    \def\citeauthoryear##1##2##3{##2 ##3}\@internalcite}
\def\shortciteNP{\def\@citeseppen{-1000}%
    \def\@cite##1##2{##1\if@tempswa , ##2\fi}%
    \def\citeauthoryear##1##2##3{##2 ##3}\@internalcite}
\def\shortciteN{\def\@citeseppen{-1000}%
%  Pierre L'Ecuyer's fix for multiple cite bug
%  Added by Paul J Sanchez on 2 September 2002
%  should have caught this last year...
%   \def\@cite##1##2{##1\if@tempswa , ##2)\else{)}\fi}%
%   \def\citeauthoryear##1##2##3{##2 (##3}\@citedata}
% Shane G. Henderson fix for extra right bracket at end of optional material June 8, 2005
%    \def\@cite##1##2{##1\if@tempswa, ##2)\else{}\fi}%
    \def\@cite##1##2{##1\if@tempswa, ##2\else{}\fi}%
    \def\citeauthoryear##1##2##3{##2 (##3)}\@citedata}
\def\shortciteA{\def\@citeseppen{-1000}%
    \def\@cite##1##2{(##1\if@tempswa , ##2\fi)}%
    \def\citeauthoryear##1##2##3{##2}\@internalcite}
\def\shortciteANP{\def\@citeseppen{-1000}%
    \def\@cite##1##2{##1\if@tempswa , ##2\fi}%
    \def\citeauthoryear##1##2##3{##2}\@internalcite}
%
\def\citeyear{\def\@citeseppen{-1000}%
    \def\@cite##1##2{(##1\if@tempswa , ##2\fi)}%
    \def\citeauthoryear##1##2##3{##3}\@citedata}
\def\citeyearNP{\def\@citeseppen{-1000}%
    \def\@cite##1##2{##1\if@tempswa , ##2\fi}%
    \def\citeauthoryear##1##2##3{##3}\@citedata}
%
% \@citedata and \@citedatax:
%
% Place commas in-between citations in the same \citeyear, \citeyearNP,
% \citeN, or \shortciteN command.
% Use something like \citeN{ref1,ref2,ref3} and \citeN{ref4} for a list.
%
\def\@citedata{%
    \@ifnextchar [{\@tempswatrue\@citedatax}%
                  {\@tempswafalse\@citedatax[]}%
}

\def\@citedatax[#1]#2{%
\if@filesw\immediate\write\@auxout{\string\citation{#2}}\fi%
  \def\@citea{}\@cite{\@for\@citeb:=#2\do%
    {\@citea\def\@citea{, }\@ifundefined% by Young
       {b@\@citeb}{{\bf ?}%
       \@warning{Citation `\@citeb' on page \thepage \space undefined}}%
{\csname b@\@citeb\endcsname}}}{#1}}%

% don't box citations, separate with ; and a space
% also, make the penalty between citations negative: a good place to break.
%
\def\@citex[#1]#2{%
\if@filesw\immediate\write\@auxout{\string\citation{#2}}\fi%
  \def\@citea{}\@cite{\@for\@citeb:=#2\do%
    {\@citea\def\@citea{; }\@ifundefined% by Young
       {b@\@citeb}{{\bf ?}%
       \@warning{Citation `\@citeb' on page \thepage \space undefined}}%
{\csname b@\@citeb\endcsname}}}{#1}}%

% (from apalike.sty)
% No labels in the bibliography.
%
\def\@biblabel#1{}
\makeatother

%\newlength{\bibhang}
%\setlength{\bibhang}{2em}

% Indent second and subsequent lines of bibliographic entries. Taken
% from openbib.sty: \newblock is set to {}.
% \renewcommand{\refname}{REFERENCES}

\newdimen\bibindent
\bibindent=0.0em
% SEC: was \def\thebibliography#1{\section*{\refname\@mkboth
% SEC: was   {\uppercase{\refname}}{\uppercase{\refname}}}\list
\def\thebibliography#1{\section*{\refname}\list
   {}{\settowidth\labelwidth{[#1]}
   \leftmargin\parindent
   \itemindent -\parindent
   \listparindent \itemindent
   \itemsep 0pt
   \parsep 0pt}
   \def\newblock{}
   \sloppy
   \sfcode`\.=1000\relax}


           % Set up BiBTeX macros

% needed to make the tex document look more like the word counterpart :-(
\setlength{\baselineskip}{12.7pt}
}

% --------------------------------- TITLE SETUP  ---------------------------------
% *** WSC formatting of the title (redefinition of title command)
\renewcommand\title[1]{\gdef\@title{\bf #1}}

%space between title and the abstract, this case has been set to be 3.4 inches (two rows of authors).
\newlength{\titlevboxsize}
\setlength\titlevboxsize {3.4in}

\def\authors#1{\gdef\theauthors{#1}}
\let\aauthor\authors

%definition of maketitle	

\def\maketitle{\par\global\titletrue
 \begingroup
   \renewcommand\thefootnote{\fnsymbol{footnote}}%
    \def\@makefnmark{\rlap{\@textsuperscript{\normalfont\@thefnmark}}}%
     \long\def\@makefntext##1{\parindent 1em\noindent
             \hb@xt@1.8em{%
                 \hss\@textsuperscript{\normalfont\@thefnmark}}##1}%
  % \twocolumn[\@maketitle]%
	
\@maketitle
   \@thanks
 \endgroup
 \setcounter{footnote}{0}%
 \let\maketitle\relax
 \let\@maketitle\relax
 \gdef\@thanks{}\gdef\@author{}\gdef\@title{}\let\thanks\relax}

\def\@maketitle{%
%\hbox to \titlevboxsize
% \vbox  to \titlevboxsize

   %Space between header and title
   %\vskip 2pc % 1 pc = 12 pt
{ \vspace*{1.5mm} %8.2mm
    \hsize\textwidth
    \linewidth\hsize
%    \vfil
    \centering
    { \@title \par}
    \vskip 24pt%Controls Text Between Title and Authors
    { \begin{tabular}[t]{@{}c@{}}\@author \\
\vrule width 3in height 0pt depth0pt\relax\end{tabular}\par %new for WSC-10: space between authors has been reduced to have the text of 6.5 inches

}\vskip 12pt} }
%\vrule width 3.5in height 0pt depth0pt\relax\end{tabular}\par

% definition of \and

\def\and{\\
 \vrule width 3in height 0pt depth0pt\relax %new for WSC-10: space between authors has been reduced to have the text of 6.5 inches
% \vrule width 3.5in height 0pt depth0pt\relax
\end {tabular}\hfill%
\begin{tabular}[t]{@{}c@{}}}

% --------------------------------- EASY AUTHOR BIOGRAPHIES ---------------------------------

\def\addBio#1#2{\noindent {\bf \uppercase{#1}} #2\\
}

% --------------------------------- SETTING UP COPYRIGHT SPACE ---------------------------------

\def\copyrightspace{%
  \footnotetext[0]{\mbox{}\vrule \@height 97\p@ \@width \z@}}
	
% Redefine the abstract environment to be a normal section (without number)
\renewenvironment{abstract}%
  {\section*{\abstractname}}
  {\par}


% --------------------------------- SETTING UP ITEM ENVIRONMENTS


\newdimen\labelwidthi
\newdimen\labelwidthii
\settowidth{\labelwidthi}{M}
\settowidth{\labelwidthii}{(d)}
%\leftmargini\labelwidthi    \advance\leftmargini\labelsep
%\leftmarginii\labelwidthii  \advance\leftmarginii\labelsep

%\leftmargin=.63cm
\leftmargini=1.2cm
\leftmarginii=4ex
\leftmarginii=4ex

% level 1
\def\@listi{\leftmargin\leftmargini
\rightmargin0pt
            \parsep 0\p@
            \topsep 10\p@
            \itemsep0\p@}
\let\@listI\@listi
\@listi

% level 2
\def\@listii{\leftmargin\leftmarginii
\rightmargin0pt
            \labelsep 1ex
            \parsep 0\p@
            \topsep 0\p@
            \itemsep0\p@}
\let\@listII\@listii
\@listii

% level 3
\def\@listiii{\leftmargin\leftmarginiii
\rightmargin0pt
            \parsep 0\p@
            \topsep 0\p@
            \itemsep0\p@}
\let\@listIII\@listiii
\@listiii

\labelsep=.25in
\setlength  \labelwidth{\leftmargini}
\addtolength\labelwidth{-\labelsep}

\renewcommand\labelenumi{\rlap{\rm\theenumi.}}

\renewcommand\labelitemi{\rlap{\textbullet}}


% SETTING UP SECTIONS

\def\@seccntformat#1{\hbox to .25in{\csname the#1\endcsname\hss}\relax}

\def\@sect#1#2#3#4#5#6[#7]#8{%
\ifnum #2>\c@secnumdepth
    \let\@svsec\@empty
  \else
    \refstepcounter{#1}%
    \protected@edef\@svsec{\@seccntformat{#1}\relax}%
  \fi
  \@tempskipa #5\relax
  \ifdim \@tempskipa>\z@
    \begingroup
      #6{%
        \@hangfrom{\hskip #3\relax\@svsec}%
          \interlinepenalty \@M #8\@@par}%
    \endgroup
    \csname #1mark\endcsname{#7}%
    \addcontentsline{toc}{#1}{%
      \ifnum #2>\c@secnumdepth \else
        \protect\numberline{\csname the#1\endcsname}%
      \fi
      #7}%
  \else
    \def\@svsechd{%
      #6{\hskip #3\relax
      \@svsec #8}%
      \csname #1mark\endcsname{#7}%
      \addcontentsline{toc}{#1}{%
        \ifnum #2>\c@secnumdepth \else
          \protect\numberline{\csname the#1\endcsname}%
        \fi
        #7}}%
  \fi
  \@xsect{#5}}

% XTABLE

\def\xtable{table}
\long\def\@makecaption#1#2{%
\ifx\@captype\xtable \vskip3pt \else
  \vskip\abovecaptionskip\fi
  \sbox\@tempboxa{#1: #2}%
  \ifdim \wd\@tempboxa >\hsize
    #1: #2\par
  \else
    \global \@minipagefalse
    \hb@xt@\hsize{\hfil\box\@tempboxa\hfil}%
  \fi
  \vskip\belowcaptionskip}

% REDEFINITION OF THE SECTION COMMANDS (spacing around them)

% new from WSC'11: 6 pts after Section

\renewcommand\section{\@startsection {section}{1}{\z@}%
                                  % {-3.5ex \@plus -1ex \@minus -.2ex}%
 {-12pt}%
                                   %{2.3ex \@plus.2ex}%
{6pt}%
                                   {\hyphenpenalty10000\normalfont\normalsize\bfseries}}
\renewcommand\subsection{\@startsection{subsection}{2}{\z@}%
                         %       {-3.25ex\@plus -1ex \@minus -.2ex}%
 {-12pt}%
       %                                     {1.5ex \@plus .2ex}%
{6pt}%
                                     {\normalfont\normalsize\hyphenpenalty10000\bfseries}}
\renewcommand\subsubsection{\@startsection{subsubsection}{3}{\z@}%
{-12pt}
      %                               {-3.25ex\@plus -1ex \@minus -.2ex}%
{6pt}
     %                                {1.5ex \@plus .2ex}%
                                     {\normalfont\normalsize\hyphenpenalty10000\bfseries}}

\let\savesubsub\subsubsection
\def\subsubsection#1{\savesubsub{\ \ #1}}

% REDEFINITION OF FURTHER DOCUMENT LAYOUT PARAMETERS

\parskip=0pt plus .01pt

%\baselineskip=0mm

\let\saveparagraph\paragraph
\def\paragraph#1{\vskip1sp
{\bf #1}\hskip1em\relax}


\raggedbottom

% hangref environment
\newenvironment{hangref}{\begin{list}{}{\setlength{\itemsep}{0pt}
\setlength{\parsep}{0pt}\setlength{\rightmargin}{0pt}
\setlength{\leftmargin}{+\parindent}
\setlength{\itemindent}{-\parindent}}}{\end{list}}


\newif\iftitle
\def\@oddhead{\iftitle\global\titlefalse
\vtop to0pt{\hbox to.9\textwidth{\titlepageheadfont
\currentCaption\hfill}%
\vskip2pt \hbox to .9\textwidth{\titlepageheadfont
\currentEditors , eds.\hfill}%
\vss} \else \hbox
to\textwidth{\titlepageheadfont\hfill\thetitle\hfill}\fi}

\def\@evenhead{\iftitle\global\titlefalse\fi%
\hbox to \textwidth{\hss\titlepageheadfont \theauthors\hss}}

\let\@oddfoot\relax
\let\@evenfoot\@oddfoot

\def\ttitle#1{\gdef\thetitle{#1}}

%\def\normalsize{%
%\@setfontsize \normalsize \@xpt \@xiipt \abovedisplayskip 12pt
%\abovedisplayshortskip 8pt \belowdisplayshortskip 8pt
%\belowdisplayskip=\abovedisplayskip
%\let \@listi \@listI}
%\normalsize

\def\sk{\vskip12pt}
\let\vs\sk
\def\bd{\vskip2pt\noindent} %belowdisplay

\spaceskip=3.5pt plus 2pt minus2pt

\parindent=.25in

\hfuzz=1pt

\widowpenalty=10000 \clubpenalty=10000

\def\verbatim{\spaceskip=0pt
\@verbatim \frenchspacing\@vobeyspaces \@xverbatim}

\newcommand\smscriptsize{\@setfontsize\scriptsize\@vipt\@viipt}

\def\nofloatfigure{\def\@captype{figure}}
\let\endnofloatfigure\relax

\def\nofloattable{\def\@captype{table}}
\let\endnofloattable\relax

\newcount\itemcount

\def\spenumerate{\bgroup\leftskip=.25in
\global\itemcount=0
\def\item{\global\advance\itemcount by 1
\vskip1sp \noindent\hskip-.25in\hbox to
.25in{\the\itemcount.\hss}}}

\def\endspenumerate{\vskip12pt\egroup}

\newif\ifnoindent
\def\@begintheorem#1#2{\vskip-12pt\vskip1sp
\trivlist
   \item[\ifnoindent\global\noindentfalse\else
\hskip.25in\fi\hskip \labelsep{\bfseries #1\ #2}]\itshape}

\def\@opargbegintheorem#1#2#3{\vskip-12pt\vskip1sp
\trivlist
      \item[\ifnoindent\global\noindentfalse\else\hskip.25in\fi%
\hskip \labelsep{\bfseries #1\ #2\ (#3)}]\itshape}

\def\@endtheorem{\vskip1sp}

% (from apalike.sty)
% Set length of hanging indentation for bibliography entries.
%
\newlength{\bibhang}
\setlength{\bibhang}{2em}

% Indent second and subsequent lines of bibliographic entries. Stolen
% from openbib.sty: \newblock is set to {}.

\newdimen\bibindent
\bibindent=.25in
\@ifundefined{refname}%
   {\@ifundefined{chapter}%
     {\newcommand{\refname}{References}}%
     {\newcommand{\refname}{Bibliography}}%
   }%
   {}%


\def\thebibliography#1{\section*{\refname\@mkboth
   {\uppercase{\refname}}{\uppercase{\refname}}}\list
   {[\arabic{enumi}]}{\settowidth\labelwidth{[#1]}
\rightmargin=0pt \leftmargin=0pt
   \leftmargin\labelwidth
   \advance\leftmargin\labelsep
   \advance\leftmargin\bibindent
\advance\leftmargin-24pt
   \itemindent -\bibindent
   \listparindent \itemindent
   \parsep \z@
   \usecounter{enumi}}
   \def\newblock{}
   \sloppy
   \sfcode`\.=1000\relax}

\endinput

\makeatletter
\def\@lbibitem[#1]#2{\item[]%[\@biblabel {#1}\hfill ]
\if@filesw {\let\protect\noexpand\immediate\write\@auxout{
\string\bibcite {#2}{#1}}}\fi\ignorespaces}

\def\@cite#1#2{{#1\if@tempswa , #2\fi}}
\makeatother



\def\Box{\vbox to 6pt{\hrule
\hbox{\vrule height 4.8pt  \hskip 4.8pt \vrule } \hrule}}

\end{filecontents*}

 %%%%%%%%%%%%%%%%%%%%%%%%%%%%%%%%%%%%%%%%%%%%%%%%%%%%%%%%%%%%%%%%%%%%%%%%%%%%%%%%%%%%%%%%%%%%%%%%%%%%%%%

\begin{filecontents*}{wscbib.tex}
%%%%%%%%%%%%%%%%%%%%%%%%%%%%%%%%%%%%%%%%%%%%%%%%%%%%%%%%%%%%%%%%%%%%%%%%%%%%%%
%                                                                            %
%     THESE COMMANDS ARE REQUIRED TO WORK WITH WSC.BST TO MAKE BIBLIO     %
%                                                                            %
%%%%%%%%%%%%%%%%%%%%%%%%%%%%%%%%%%%%%%%%%%%%%%%%%%%%%%%%%%%%%%%%%%%%%%%%%%%%%%
\makeatletter
\let\@internalcite\cite
%
\def\cite{\def\@citeseppen{-1000}%
    \def\@cite##1##2{(##1\if@tempswa , ##2\fi)}%
    \def\citeauthoryear##1##2##3{##1 ##3}\@internalcite}
\def\citeNP{\def\@citeseppen{-1000}%
    \def\@cite##1##2{##1\if@tempswa , ##2\fi}%
    \def\citeauthoryear##1##2##3{##1 ##3}\@internalcite}
\def\citeN{\def\@citeseppen{-1000}%
%  Pierre L'Ecuyer's fix for multiple cite bug
%  Added by Paul J Sanchez on 4 October 2001
%   \def\@cite##1##2{##1\if@tempswa , ##2)\else{)}\fi}%
%   \def\citeauthoryear##1##2##3{##1 (##3}\@citedata}
    \def\@cite##1##2{##1\if@tempswa, ##2)\else{}\fi}%
    \def\citeauthoryear##1##2##3{##1 (##3)}\@citedata}
\def\citeA{\def\@citeseppen{-1000}%
    \def\@cite##1##2{(##1\if@tempswa , ##2\fi)}%
    \def\citeauthoryear##1##2##3{##1}\@internalcite}
\def\citeANP{\def\@citeseppen{-1000}%
    \def\@cite##1##2{##1\if@tempswa , ##2\fi}%
    \def\citeauthoryear##1##2##3{##1}\@internalcite}
%
\def\shortcite{\def\@citeseppen{-1000}%
    \def\@cite##1##2{(##1\if@tempswa , ##2\fi)}%
    \def\citeauthoryear##1##2##3{##2 ##3}\@internalcite}
\def\shortciteNP{\def\@citeseppen{-1000}%
    \def\@cite##1##2{##1\if@tempswa , ##2\fi}%
    \def\citeauthoryear##1##2##3{##2 ##3}\@internalcite}
\def\shortciteN{\def\@citeseppen{-1000}%
%  Pierre L'Ecuyer's fix for multiple cite bug
%  Added by Paul J Sanchez on 2 September 2002
%  should have caught this last year...
%   \def\@cite##1##2{##1\if@tempswa , ##2)\else{)}\fi}%
%   \def\citeauthoryear##1##2##3{##2 (##3}\@citedata}
% Shane G. Henderson fix for extra right bracket at end of optional material June 8, 2005
%    \def\@cite##1##2{##1\if@tempswa, ##2)\else{}\fi}%
    \def\@cite##1##2{##1\if@tempswa, ##2\else{}\fi}%
    \def\citeauthoryear##1##2##3{##2 (##3)}\@citedata}
\def\shortciteA{\def\@citeseppen{-1000}%
    \def\@cite##1##2{(##1\if@tempswa , ##2\fi)}%
    \def\citeauthoryear##1##2##3{##2}\@internalcite}
\def\shortciteANP{\def\@citeseppen{-1000}%
    \def\@cite##1##2{##1\if@tempswa , ##2\fi}%
    \def\citeauthoryear##1##2##3{##2}\@internalcite}
%
\def\citeyear{\def\@citeseppen{-1000}%
    \def\@cite##1##2{(##1\if@tempswa , ##2\fi)}%
    \def\citeauthoryear##1##2##3{##3}\@citedata}
\def\citeyearNP{\def\@citeseppen{-1000}%
    \def\@cite##1##2{##1\if@tempswa , ##2\fi}%
    \def\citeauthoryear##1##2##3{##3}\@citedata}
%
% \@citedata and \@citedatax:
%
% Place commas in-between citations in the same \citeyear, \citeyearNP,
% \citeN, or \shortciteN command.
% Use something like \citeN{ref1,ref2,ref3} and \citeN{ref4} for a list.
%
\def\@citedata{%
    \@ifnextchar [{\@tempswatrue\@citedatax}%
                  {\@tempswafalse\@citedatax[]}%
}

\def\@citedatax[#1]#2{%
\if@filesw\immediate\write\@auxout{\string\citation{#2}}\fi%
  \def\@citea{}\@cite{\@for\@citeb:=#2\do%
    {\@citea\def\@citea{, }\@ifundefined% by Young
       {b@\@citeb}{{\bf ?}%
       \@warning{Citation `\@citeb' on page \thepage \space undefined}}%
{\csname b@\@citeb\endcsname}}}{#1}}%

% don't box citations, separate with ; and a space
% also, make the penalty between citations negative: a good place to break.
%
\def\@citex[#1]#2{%
\if@filesw\immediate\write\@auxout{\string\citation{#2}}\fi%
  \def\@citea{}\@cite{\@for\@citeb:=#2\do%
    {\@citea\def\@citea{, }\@ifundefined% by Young
       {b@\@citeb}{{\bf ?}%
       \@warning{Citation `\@citeb' on page \thepage \space undefined}}%
{\csname b@\@citeb\endcsname}}}{#1}}%

% (from apalike.sty)
% No labels in the bibliography.
%
\def\@biblabel#1{}
\makeatother

%\newlength{\bibhang}
%\setlength{\bibhang}{2em}

% Indent second and subsequent lines of bibliographic entries. Taken
% from openbib.sty: \newblock is set to {}.
% \renewcommand{\refname}{REFERENCES}

\newdimen\bibindent
\bibindent=0.0em
% SEC: was \def\thebibliography#1{\section*{\refname\@mkboth
% SEC: was   {\uppercase{\refname}}{\uppercase{\refname}}}\list
\def\thebibliography#1{\section*{\refname}\list
   {}{\settowidth\labelwidth{[#1]}
   \leftmargin\parindent
   \itemindent -\parindent
   \listparindent \itemindent
   \itemsep 0pt
   \parsep 0pt}
   \def\newblock{}
   \sloppy
   \sfcode`\.=1000\relax}


\end{filecontents*}
     % download from author kit.  Style files for wsc formatting. Don't remove this line - required for generating the final paper!

\documentclass{wscpaperproc}
\usepackage{latexsym}
%\usepackage{caption}
\usepackage{graphicx}
\usepackage{mathptmx}

\usepackage{subcaption}
\usepackage[pdftex,colorlinks=true,urlcolor=blue,citecolor=black,
anchorcolor=black,linkcolor=black]{hyperref}
%
%****************************************************************************
% AUTHOR: You may want to use some of these packages. (Optional)
\usepackage{amsmath}
\usepackage{amsfonts}
\usepackage{amssymb}
\usepackage{amsbsy}
\usepackage{amsthm}
%****************************************************************************



%
%****************************************************************************
% AUTHOR: If you do not wish to use hyperlinks, then just comment
% out the hyperref usepackage commands below.

%% This version of the command is used if you use pdflatex. In this case you
%% cannot use ps or eps files for graphics, but pdf, jpeg, png etc are fine.

\usepackage[pdftex,colorlinks=true,urlcolor=blue,citecolor=black,anchorcolor=black,linkcolor=black]{hyperref}

%% The next versions of the hyperref command are used if you adopt the
%% outdated latex-dvips-ps2pdf route in generating your pdf file. In
%% this case you can use ps or eps files for graphics, but not pdf, jpeg, png etc.
%% However, the final pdf file should embed all fonts required which means that you have to use file
%% formats which can embed fonts. Please note that the final PDF file will not be generated on your computer!
%% If you are using WinEdt or PCTeX, then use the following. If you are using
%% Y&Y TeX then replace "dvips" with "dvipsone"

%%\usepackage[dvips,colorlinks=true,urlcolor=blue,citecolor=black,%
%% anchorcolor=black,linkcolor=black]{hyperref}
%****************************************************************************



		



%
%****************************************************************************
%*
%* AUTHOR: YOUR CALL!  Document-specific macros can come here.
%*
%****************************************************************************

% If you use theoremes
\newtheoremstyle{wsc}% hnamei
{3pt}% hSpace abovei
{3pt}% hSpace belowi
{}% hBody fonti
{}% hIndent amounti1
{\bf}% hTheorem head fontbf
{}% hPunctuation after theorem headi
{.5em}% hSpace after theorem headi2
{}% hTheorem head spec (can be left empty, meaning `normal')i

\theoremstyle{wsc}
\newtheorem{theorem}{Theorem}
\renewcommand{\thetheorem}{ \arabic{theorem}}
\newtheorem{corollary}[theorem]{Corollary}
\renewcommand{\thecorollary}{\arabic{corollary}}
\newtheorem{definition}{Definition}
\renewcommand{\thedefinition}{\arabic{definition}}


%#########################################################
%*
%*  The Document.
%*
\begin{document}

%***************************************************************************
% AUTHOR: AUTHOR NAMES GO HERE
% FORMAT AUTHORS NAMES Like: Author1, Author2 and Author3 (last names)
%
%		You need to change the author listing below!
%               Please list ALL authors using last name only, separate by a comma except
%               for the last author, separate with "and"
%
\WSCpagesetup{Huang, Lam and Zhao}

% AUTHOR: Enter the title, all letters in upper case
\title{SEQUENTIAL EXPERIMENTATION TO EFFICIENTLY TEST AUTOMATED VEHICLES}
% SEQUENTIAL EXPERIMENTATION TO EVALUATE AUTOMATED VEHICLES

% AUTHOR: Enter the authors of the article, see end of the example document for further examples
\author{Zhiyuan Huang\\ Henry Lam\\ [12pt]
Department of Industrial and Operations Engineering \\
University of Michigan\\
1205 Beal Avenue\\
Ann Arbor, MI 48105, USA\\
% Multiple authors are entered as follows.
% You may also need to adjust the titlevbox size in the preamble - search for titlevboxsize
\and
Ding Zhao\\[12pt]
Department of Mechanical Engineering\\
University of Michigan\\
2901 Baxter Road\\
Ann Arbor, MI 48109, USA
}






\maketitle

\section*{ABSTRACT}
Automated vehicles have been under heavy developments in major auto and tech companies and are expected to release into market in the foreseeable future. However, the road safety of these vehicles remains a concern. One approach to evaluate their safety is via on-track experimentation, but this requires gigantic costs and time investments. This paper discusses a sequential learning approach based on kriging models to reduce the experimental runs and economize on-track experimentation. The approach relies on a heuristic simulation-based gradient descent procedure to search for the best next test scenario. We demonstrate our approach with some numerical test cases.


\section{INTRODUCTION}                                                                          
\subsection{Background of Automated Vehicles Evaluation}
While automated vehicles (AVs) are currently under intense developments by almost all major auto companies and tech giants, their safety has remained a concern, as reinforced by recent Tesla accidents involving self-driving systems \cite{teslanews}. The difficulty in evaluating AVs is that these vehicles are ``smart", in that they interact with their environments and prompt autonomous actions, and hence cannot be tested using existing standard approaches. 

For example, the so-called test matrix approach, adopted commonly in many vehicle testing procedures, uses fixed and predefined test scenarios to evaluate vehicles. However, an AV producer can tune the algorithm to excel in such test scenarios but fail on others, making the results of the test matrix invalid in capturing the actual risk \cite{Aust2012EvaluationSystems}. In the United States, there are currently no standards or protocols to test AVs with high degrees of automation (known as automation level 2 \cite{national2013preliminary} or higher). Most prospective AV manufacturers at present rely on Naturalist Field Operational Tests (N-FOT) \cite{festa2008festa} to evaluate AV safety, which means putting the vehicle prototypes on actual roads and collecting data from potential accidents or conflicts. Such tests, however, are both time-consuming and costly, as accidents are rare events that can only be assessed under statistical confidence with astronomical road miles driven by these prototypes. According to \shortciteN{akamatsu2013automotive}, an N-FOT ``cannot be conducted with less than \$10,000,000''.

As an alternative, researchers have explored the use of Monte Carlo simulation techniques. \citeN{yang2010development} and \citeN{lee2004longitudinal} evaluated collision avoidance systems by reusing existing N-FOT data, and \shortciteN{woodrooffe2014performance} used forward collision scenarios to evaluate collision warning and mitigation braking technologies on heavy trucks. \shortciteN{zhao2017accelerated}, \shortciteN{huang2017accelerated}, and \shortciteN{huang1001evaluation} applied importance sampling methods to evaluate car-following and lane change scenarios. However,  Monte-Carlo-based methods need to make assumptions on the control and dynamics of AVs. The lack of full knowledge in specifying these assumptions, complicated by the autonomous operations of AVs that are not publicly disclosed, remains one of the key difficulties in carrying out reliable Monte Carlo evaluation. On-track experiments to learn the behaviors of AVs is therefore a crucial step \cite{peng2012evaluation}. These behaviors, once accurately informed, can be used as inputs to the Monte Carlo evaluation. However, such experiments are only recently feasible \cite{mcity} and require huge cost and time investments. This motivates us to explore an adaptive approach to reduce the number of on-track experimental runs needed for the learning. 

% Since the exposure to safety critical scenarios is generally low, Monte Carlo simulations are time-consuming; therefore, variance reduction techniques have been considered. 

% Although variance reduction techniques largely reduced the required number of tests to a scale of thousand, the number is still not affordable for real world on-track experiments. The National Highway Traffic Safety Administration has the responsibility to guarantee the safety of vehicles in the market. On-track experiments with new vehicles are required for safety evaluation (Peng et al. 2012). For Automated Vehicles, the vast number of scenarios in the daily driving make it hard to evaluate. Considering the number of new types entering the market each year, the existing methods are inefficient for the tasks. 

\subsection{Outline of the Sequential Experimentation Approach}

The number of possible scenarios that an AV can react on, which collectively define the behavior of the AV, are typically infinite. This motivates us to consider a metamodel to make our learning feasible. Specifically, we use a kriging framework to model the unknown behavior of AVs, and investigate a myopic approach to sequentially select the next test scenario that can maximize the information gain (thereby reducing the runs needed to achieve a reasonable estimation accuracy). As the gain is in terms of the correctness of the Monte Carlo evaluation, finding the next test scenario generally requires simulation-based optimization. In particular, we investigate a heuristic use of stochastic gradient descent. The simulation is also used to make the final safety evaluation of the AV being considered.

% To overcome the number and cost of experimentation, this paper proposes a Kriging-based sampling approach. The proposed sampling procedure is based on an optimization problem that gives the sample point with the maximum impact to the event probability estimation. We validate the proposed model with numerical simulations and conclude from the results that the approach is an effective replacement for N-FOT and other burdensome evaluative techniques.

Our framework follows from the kriging technique originated from geology (e.g., \citeNP{chiles2009geostatistics}) and further developed in computer experiments (e.g., \shortciteNP{sacks1989design}). The primary use of kriging is to assimilate spatial data under correlation among different design points that is made computationally convenient through Gaussian process modeling. Our approach follows this framework by viewing the test scenarios as design points. In the static settings, the design points are typically selected using space-filling design (e.g., with Latin Hypercube Sampling; \citeNP{kleijnen2008design}). To reduce experimental costs with respect to a specified goal, one can sequentially select the design points, which is the approach we adopt. In particular, we follow the sequential sampling idea that has been applied to sensitivity analysis and optimization \cite{kleijnen2004application,kleijnen2009kriging}. Our work most closely follows the concept of knowledge gradient (e.g., \citeNP{powell2012optimal}) in the Bayesian setting. Other related literature includes the stream of study in stochastic kriging \shortcite{ankenman2010stochastic,staum2009better}, a generalization of the kriging technique to stochastic computer experiments. In this paper, however, we assume the on-track experimentation is error-free and hence relates more closely to the deterministic experimentation framework. On the other hand, stochasticity comes in the evaluation criterion and as a result, as discussed above, our sequential design point search will allude to the use of simulation optimization. 
% Finding the next design point according to this stochastic evaluation gives rise to a simulation optimization problem, for which we adopt an iterative gradient-based method via a heuristic use of stochastic approximation (

% The standard sampling methods are space-filling designs. The most popular method of this type is the Latin Hypercube Sampling (LHS) (Kleijnen 2008). The space-filling designs does not consider the goal of the experiments and hence is considered to be less efficient. Sequential sampling methods are developed for different goals of the experiments. studied sequential sampling methods for sensitivity analysis and  studied the sequential sampling methods for optimization. These approaches are efficient for the specific goals of the Kriging model. Huang et al. (2006a, 2006b) uses optimization in the sampling procedure. This approach used sequential sampling to maximize an augmented expected improvement function, which is connected with the evaluation costs. However, the optimization is not related to a specific use of the model and tries to balance the local and global search for design points. This paper proposes an sampling approach based on optimization which sequentially searches for samples with largest impacts on the event probability estimation. 
% Section \ref{sec:sampling_review} briefly reviews the range of Kriging sampling techniques.
The remainder of this paper is as follows.  Section \ref{sec:example} describes the basic setups in AV evaluation and casts our evaluation framework in the kriging setting. Section \ref{sec:optim} presents our optimization procedure to select test scenarios. Section \ref{sec:numerical} shows some numerical examples. 

% Section \ref{sec:conclusion} concludes this paper.



% \section{Review of Sampling Methods for Kriging} \label{sec:sampling_review}



\section{A KRIGING FRAMEWORK FOR AV EVALUATION} \label{sec:example}
We introduce our framework in two components. First, Section \ref{sec:task} describes the setting and the challenge of AV evaluation and gives a simple illustrative example. Section \ref{sec:kriging} then describes how we cast the AV evaluation task into a kriging-based learning model.

\subsection{The Task of AV Evaluation}\label{sec:task}
Evaluation of the road safety of AVs requires studying the risk arising from its interaction with the surrounding environments, such as other vehicles driven by human drivers, pedestrians etc. The risk can be measured by probabilistic quantities such as the chance of accidents (e.g., crashes) and conflicts (e.g., the AV and a front car within a dangerously short distance). For example, \shortciteN{zhao2017accelerated} demonstrate this calculation via Monte Carlo simulation with a lane change scenario. Figure \ref{fig:lane} describes this setting, where a human-controlled vehicle driving in front of the AV is cutting into the AV's lane. The AV has a built-in intelligent control system that is assumed deterministic, while the frontal vehicle is susceptible to noisy human behavior and hence is stochastic. 

A collision can occur when the gap is too short at any point of time. Consider a fixed period of time $T$ that represents the typical car-following duration. Denote $R_L(t,\omega)$ as the range between the AV and the human-driving vehicle and $\dot{R}_L(t,\omega)$ as its rate of change, which depend on the physical measurements of both vehicles including accelerations, velocities and positions. $\omega$ denotes the initial condition of the lane change scenario. We say a collision happens if the range at any point of time is too short, say within a threshold $b$. Then the collision probability is $P(R_L(t,\omega)<b\text{\ for some\ }t\in[0,T])$, or equivalently $P(\max_{t\in[0,T]}1/R_L(t,\omega)>1/b)$. 
% Since the lane change is initialized by the frontal human-controlled vehicle, we assume that the condition for the two vehicles are stochastic when the lane change starts. , where $R_L(t)$ denotes the gap between the human-driving vehicle and the AV, and $\gamma$. This quantity depends on the dynamic physical measurements We use the database to fit these three variables to probability distributions.

In general, the stochasticity of the human-driving vehicle, described by its acceleration etc., can be estimated from existing data. \shortciteN{zhao2017accelerated} for instance uses the naturalistic driving data among all the lane change scenarios extracted from the Safety Pilot Model Deployment (SPMD) database \cite{bezzina2014safety}. However, the AV control is typically not fully known to the tester. It could be known by the company that owns its production, but  due to commercial concern such knowledge is not revealed to governmental or public entities who conduct safety tests. So to carry out the Monte Carlo safety test, a governmental unit needs to learn the control system by carrying out its own on-track experiment. This experiment runs on a physical proving ground (e.g., \citeNP{mcity}) which, in the considered setting, can preset the configuration of the frontal vehicle to resemble an actual road condition. Observing how the AV reacts in these conditions provides some information on its underlying intelligent control.
% Fig. \ref{fig:lane} illustrates the scenario and notations of data. 
\begin{figure}[ht]
							\centering
							\includegraphics[width=0.5\linewidth]{oldfig1.png}
                            \caption{A lane change scenario.}
                            \label{fig:lane}
\end{figure}
            
%

Other scenarios can be evaluated similarly as above; see, e.g., \shortciteN{zhao2017accelerated} for a car-following setting. In the subsequent discussion we will focus on the lane change situation for illustration. 

\subsection{A Kriging Model}\label{sec:kriging}
% First developed by geostatisticians and further studied by mathematicians, the Kriging model is a powerful meta-modeling method for constructing response surface of outputs (Staum 2009). While the Kriging model is capable of modeling multivariate response, in this paper we use the univariate Kriging model.
We study a kriging-based learning approach to collect information about the AV from on-track experiments. Suppose we are interested in estimating $P(f(\omega)>\gamma)$, where $f:\mathcal X\to\mathbb R$ is an unknown function on $\mathcal X$, $\gamma$ is a given threshold, and $\omega\in\mathcal X$ is a random object under the probability $P$. For instance, $\gamma$ can be $1/b$ and $f(\omega)$ be $\max_{t\in[0,T]}1/R_L(t,\omega)$ in the example described in Section \ref{sec:task}, where $x$ here refers to the set of parameters that controls the human-driving vehicle, which is random and its distribution calibrated from the SPMD database.

To model how information on $f$ updates our estimate on $P(f(\omega)>\gamma)$, we view $f$ as a response surface on the domain $\mathcal X$. We model $f:\mathcal X\to\mathbb R$ as a Gaussian Random Field (GRF) \cite{rasmussen2006gaussian} that is independent of the stochasticity of $\omega$, denoted as\begin{equation*}
	f(\cdot) \sim GRF(\mu(\cdot),\sigma^2(\cdot,\cdot)),
\end{equation*}
where $\mu(\cdot)$ is the mean function and $\sigma^2(\cdot,\cdot)$ is the covariance function of the GRF. Given any fixed design points $x^1,\ldots,x^k\in \mathcal X$, $f(x^1),\ldots,f(x^k)$ comprises a Gaussian random vector with means $\mu(x^i)$ and covariances $\sigma^2(x^i,x^j)$. It is customary to assume that $\mu(x)=b(x)'\beta$ and $\sigma^2(x,\tilde x)=\tau^2 r(x,\tilde x)$ where the correlation function $r(x,\tilde x)$ implies stationary variance over $\mathcal X$ and depends on the design point pairs only through the value of $x-\tilde x$. For simplicity, we will further assume that $\mu(x)=\beta$ for some $\beta \in \mathbb{R}$, which represents a flat belief on $f(\cdot)$ over all the design points. We use the correlation function $r(x,\tilde x)=\exp\{\theta\|x-\tilde x\|^2\} $, where $\|\cdot\|$ denotes the Euclidean norm. This correlation function signifies a higher correlation for test scenarios that are closer to each other. Note that we have adopted intuitive choices for the mean and correlation functions here for convenience, but better ones (in the sense of better reflecting the prior belief on the vehicle behaviors under different test scenarios) should be used with the availability of expert knowledge.

Suppose that the parameters $\beta, \theta, \tau^2$ are known. Given some observations on the value of $f(x)$ at some points in $\mathcal X$, we can update the distribution of $f(\cdot)$ via conditioning. We denote $X$ as the observed design vector $(x^1,...,x^n)$ and $Y$ the associated response vector $(f(x^1),...,f(x^n))$. We define the matrix $\Sigma\in\mathbb R^{n\times n}$ such that its $(i,j)$th entry is $\Sigma_{ij}=\sigma^2(x^i,x^j)$, and define $R=\Sigma/\tau^2$ so that $R_{ij}=r(x^i,x^j)$, for $i=1,...,n$ and $j=1,...,n$. Given observations $(X,Y)$, for any fixed $x\in\mathcal X$, we have \begin{equation*}
	E[f(x)|X,Y]= \beta +r(x|X)'R^{-1}(Y-\beta)
% 	\label{eq:kriging_E}
\end{equation*}
and
\begin{equation*}
	Var(f(x)|X,Y)=\tau^2 (1- r(x|X)'R^{-1}r(x|X)),
% 	\label{eq:kriging_var}
\end{equation*}
where $r(x|X)\in\mathbb R^n$ is a vector with $r(x,x^i)$ as the $i$th element \cite{rasmussen2006gaussian}. Note that $f(x)|X,Y$ still follows a Gaussian distribution. For simplicity, we denote $\mu(x|X,Y)=E(f(x)|X,Y)$ and  $\sigma^2(x|X,Y)=Var(f(x)|X,Y)$. 

In practice, the parameters $\beta, \theta, \tau^2$ need to be either estimated (e.g., by using maximum likelihood) or assigned reasonable values according to expert knowledge. For more details on calibrating the parameters, see, e.g., \shortciteN{ankenman2010stochastic}. In our subsequent discussion, we assume these are given and unchanged throughout the learning process. 
% We also note that the mean and correlation functions do not necessarily abide to the forms we describe. 
% For estimation, the maximum likelihood estimation gives \begin{equation}
% \hat{\beta}= \frac{\sum_{i=1}^{n}y_i}{n}.
% \end{equation}
% For $\tau^2, \theta$, we maximize the log likelihood function \begin{equation}
% 	l(\tau^2,\theta)=-\frac{1}{2} \left( n \log(2\pi) +\log(|\Sigma|) + (Y-\beta)' \Sigma^{-1}  (Y-\beta)   \right).
%  \end{equation}



% \subsection{Using Kriging in Event Probability Estimation} \label{sec:prob_est}

% As mentioned, we want to estimate the probability of a particular event occurring in our Automated Vehicle experiments. Assuming that an Automated Vehicle is a deterministic system, the event or events we study depend only on the driving environment. In some cases, it is reasonable to model the environment as stochastic (Ding et al. 2017). In this paper, we focus on the simplest version of this type of events, in which the driving environment is modeled by random variables. 

% We use $x \in \mathcal{X}$ to denote the variable vector that represents the critical factors of the driving environment, where $\mathcal{X}$ is the feasible set. We define $\varepsilon \subset \mathcal{X}$ as the event set of interest. We write the event probability as $P(x\in \varepsilon)$. We define the event indicator function $I_\varepsilon(\cdot)$ on $\mathcal{X}$, which returns 1 if the variable $x$ is in the event set; 0 otherwise. We have \begin{equation}
% P(x\in \varepsilon)=E(I_\varepsilon(x))=\int I_\varepsilon(x) dF(x).
% \label{eq:obj_prob}
% \end{equation}
% We can use the crude Monte Carlo method for (\ref{eq:obj_prob}) to evaluate the event probability.

% We consider a special case where we have a performance function $f(\cdot)$ defined on $\mathcal{X}$ such that for $x \in \mathcal{X}$, $f(x)>\gamma$ indicates that $x \in \varepsilon$. $\gamma$ is a event threshold level. In this case, we have \begin{equation}
% P(x\in \varepsilon)=P(f(x)>\gamma).
% \label{eq:obj_prob2}
% \end{equation}
% To connect this with the general setting, we set the event indicator function as\begin{equation}
% I_\varepsilon(x)=\begin{cases} 1&f(x)>\gamma,\\ 0& otherwise. \end{cases}
% \end{equation}

% The crude Monte Carlo method requires a large amount of experiments to obtain an accurate estimation, especially when $P(x\in \varepsilon)$ is small. Knowing that our Automated Vehicle experiments and the use of importance sampling and cross entropy approaches are time consuming, we decide to use Kriging prediction. Assuming that we have observations $(x_i,f(x_i))$'s and the known parameters for the Kriging model, we construct the observation matrix $(X,Y)$ with the observations $(x_i,f(x_i))$'s. This yields a response surface $y(x)$ with mean and variance given by (\ref{eq:kriging_E}) and (\ref{eq:kriging_var}).

% To estimate the probability $P(x \in \varepsilon)$, we use the estimation indicator function \begin{equation}
% I_{\gamma}(x)=I\{y(x) \geq \gamma \}.
% \end{equation}
Under the GRF assumption and conditioning on $(X,Y)$, we now set our target quantity of interest as $P(f(\omega)>\gamma|X,Y)$, where $P$ now generates both the stochasticity in $\omega$ and the Gaussian uncertainty in $f$. Typically this probability is larger, i.e., more conservative, than when $f$ is completely known, because of the additional noise coming from the model uncertainty. We view this probability as a reasonable target, but clearly other formulations are plausible.

Note that we have
\begin{equation}
P(f(\omega)>\gamma|X,Y) = E_\omega[P(f(\omega)>\gamma |\omega,X,Y)]
\label{eq:ke_complex}
\end{equation}
where $E_\omega[\cdot]$ denotes the expectation taken with respect to the stochasticity of $\omega$. Since $f(x)|X,Y$ follows a Gaussian distribution with mean $\mu(x|X,Y)$ and variance $\sigma^2(x|X,Y)$, we can write \eqref{eq:ke_complex} further as
% We further expand the integrand as
% \begin{equation}
% E\left[I_{\gamma}(x) \right]=P(y(x)\geq \gamma)=1-\Phi(\frac{\gamma-\mu(x|X,Y)}{\sqrt{\sigma^2(x|X,Y)}}),
% \end{equation}
% Therefore, we write (\ref{eq:ke_complex}) as 
\begin{equation}
E_\omega\left[\bar\Phi\left(\frac{\gamma-\mu(\omega|X,Y)}{\sigma(\omega|X,Y)}\right)\right]
\label{main expression}
\end{equation}
where $\bar\Phi(\cdot)$ denotes the tail distribution function of a standard Gaussian distribution. 
% Now the problem is how to efficiently collect the observations. Since we assume that the experiments are expensive, we want to carefully design the experiments. (see Sections \ref{sec:optim} and \ref{sec:outline} for details.)


\section{SEQUENTIAL SELECTION OF TEST SCENARIOS VIA OPTIMIZATION} \label{sec:optim}

% Here, we assume that a initial observation set $(X_n,Y_n)$ with $n$ samples is given and we know the parameters for the Kriging model. Our goal is to design the next sampling point. The sampling scheme is based on an optimization problem which obtains the sample point with the maximum impact on the event probability estimation using the Kriging model. We measure the impact by the expected ``distance'' between the current model and the model with a new observation. The idea is similar to the knowledge gradient in optimal learning (Frazier et al. 2008).  

From \eqref{main expression}, we design a procedure to sequentially look for the next scenario, or design point, to test the value of $f$ that can in a sense maximize the information gain. We define information gain as the distance between the current estimate of $P(f(\omega)>\gamma|X,Y)$ and its update taking into account the outcome of the next test. We maximize the expected distance under the current posterior distribution. This framework follows generally from the concept of knowledge gradient \cite{powell2012optimal}, but here we are interested in a pure estimation problem instead of an optimization problem. Note that the distribution of $\omega$ is estimated from data, which can be parametrically modeled or fully data-driven, i.e., nonparametric. In general we need to run simulation to evaluate our target quantity, even though $f$ is highly structured.

We present some further notations. Let $(X_n,Y_n)$ be the vectors of historical design points and responses from $f$ collected up to step $n$. We denote $E_n[\cdot]=E[\cdot|X_n,Y_n]$. In particular, $f(x)$ under $E_n[\cdot]$ follows a Gaussian distribution with mean $\mu(x|X_n,Y_n)$ and variance $\sigma^2(x|X_n,Y_n)$. For simplicity, we write $\mu_n(\cdot)=\mu(\cdot|X_n,Y_n)$ and $\sigma^2_n(\cdot)=\sigma^2(\cdot|X_n,Y_n)$.

% $(X_{n+1},Y_{n+1})$ denotes the observation set containing the original sample set $(X_n,Y_n)$ and a new sample $(x,y)$, where $y$ is a realization of random variable $y(x)$. $E_{n+1}[\cdot]$ represents the expectation over $y(x)$, which follows a Gaussian distribution with mean $\mu(x|X_{n+1},Y_{n+1})$ and variance $\sigma^2(x|X_{n+1},Y_{n+1})$. We use $P_n$ to represent the event probability estimation using the Kriging model with the existing sample set $(X_n,Y_n)$. $P_{n+1}$ represents the probability estimation using the Kriging model with the new sample set $(X_{n+1},Y_{n+1})$. 

Let $P_n=P(f(\omega)>\gamma|X_n,Y_n)$ be the current target estimate, and $P_n(x,y)=P(f(\omega)>\gamma|(X_n,x),(Y_n,y))$ be the target estimate if one tests an additional design point $x$ and collects a response $y$. Let $d(\cdot,\cdot)$ be some distance criterion between two probabilities. Given $X_n,Y_n$, we search for the next design point by looking for $x\in\mathcal X$ that solves
% The optimization for searching design point $x$ is
\begin{equation}
	\max_{x \in \mathcal{X}} E_n \left[ d(P_n,P_n(x,f(x)))  \right].
    \label{obj}
\end{equation}
% where $\mathcal{X}$ is the feasible region of $x$ and $d$ is a distance metric. Note that the expectation is over the realization $y$ which follows a Gaussian distribution with mean $\mu(x|X_{n},Y_{n})$ and variance $\sigma^2(x|X_{n},Y_{n})$. Note that we have \begin{equation}
% 	P_n=\int E_n\left[I_{\gamma}(\omega) \right] dF(\omega)=1- \int \Phi(\frac{\gamma-\mu(x|X_{n},Y_{n})}{\sqrt{\sigma^2(x|X_{n},Y_{n})}})  dF(x)
%     \label{obj2_1}
% \end{equation}
% and 
% \begin{equation}
% P_{n+1}=\int E_{n+1}\left[I_{\gamma}(\omega) \right] dF(\omega)=1- \int \Phi(\frac{\gamma-\mu(x|X_{n+1},Y_{n+1})}{\sqrt{\sigma^2(x|X_{n+1},Y_{n+1})}})  dF(x).
% \label{obj2_2}
% \end{equation}

% For simplification, we denote $\mu(\cdot|X_{n},Y_{n})$ as $\mu_n(\cdot)$, $\sigma^2(x|X_{n},Y_{n})$ as $\sigma^2_n(\cdot)$. Similarly, we have $\mu_{n+1}(\cdot)$ and $\sigma^2_{n+1}(\cdot)$ for $\mu(x|X_{n+1},Y_{n+1})$ and $\sigma^2(x|X_{n+1},Y_{n+1})$, respectively. Then we have the objective as
A simple example of $d$ is the squared $L_2$-distance, which we adopt in the sequel. Optimization \eqref{obj} becomes
\begin{align}
&\max_{x \in \mathcal{X}} E_n \left[ (P_n-P_n(x,f(x)))^2  \right]\notag\\
=& \max_{x \in \mathcal{X}} \int (P_n-P_n(x,y))^2 d \Phi\left(\frac{y-\mu_n(x)}{\sigma_n(x)}\right)\notag\\
=& \max_{x \in \mathcal{X}} \int \left(\int \left(\Phi\left(\frac{\gamma-\mu_n(\omega|x,y)}{\sigma_n(\omega|x,y)}\right) -\Phi\left(\frac{\gamma-\mu_{n}(\omega)}{\sigma_{n}(\omega)}\right)\right) dF(\omega)\right)^2 d \Phi\left(\frac{y-\mu_n(x)}{\sigma_n(x)}\right)
\label{eq:obj}
\end{align}
where we denote $\mu_n(\cdot|x,y)=\mu(\cdot|(X_n,x),(Y_n,y))$, $\sigma_n^2(\cdot|x,y)=\sigma^2(\cdot|(X_n,x),(Y_n,y)), $ $F(\cdot)$ the distribution function of $\omega$, and $\Phi(\cdot)$ the standard Gaussian distribution function. 

Note that \eqref{eq:obj} generally does not support closed-form evaluation, and requires running simulation. If $\mathcal X$ is a discrete space, ranking and selection methods can be applied (an approach taken by \shortciteN{Huang2017kriging}). Here we focus on a continuous space for $\mathcal X\subset\mathcal R^d$. We use stochastic approximation (SA) \cite{kushner2003stochastic} to search for a local optimum for \eqref{eq:obj}. This approach follows from \shortciteN{wang2016parallel} that considers parallel Bayesian global optimization where the one-step optimum cannot be solved in closed-form under Gaussian process function models. Note that, like the setting in \shortciteN{wang2016parallel}, since there is no guarantee that the objective function in \eqref{eq:obj} is concave, we can only ensure that our SA converges to a local optimum under suitable conditions.

% The closed form for this objective function is not obvious. One simplified approach is to avoid optimizing the objective function in continuous space. By discretizing the feasible region $\mathcal{X}$, we can estimate the objective function by simulation. Ranking and selection methods can be applied for searching the optimal $x$ on the discretized space.

% We also consider using the stochastic gradient ascent method to search for the optimal value of $x$ on continuous space. We estimate the stochastic gradient using infinitesimal perturbation analysis (IPS). 
We describe our stochastic gradient estimator for the objective function \eqref{eq:obj}. Given i.i.d. samples $\omega_1,...,\omega_m$ drawn from $F$ and $z_1,\ldots,z_m$ drawn from a standard Gaussian distribution, our gradient estimator is a vector in $\mathbb R^d$ whose $j$-th element is given by
\begin{equation}
\frac{1}{m}\sum_{i=1}^{m} -2\left(P_n-P_{n}(x,\sqrt{\sigma_n^2(x)}z_i+\mu_n(x))\right)\frac{\partial}{\partial x_j} \Phi \left( \frac{\gamma-\mu_n(\omega_i|x,\sqrt{\sigma_n^2(x)}z_i+\mu_n(x))}{\sqrt{\sigma^2_n(\omega_i|x,\sqrt{\sigma_n^2(x)}z_i+\mu_n(x))}}\right) \label{grad estimate}
\end{equation}
where $\frac{\partial}{\partial x_j} \Phi \left( \frac{\gamma-\mu_n(\omega|x,y)}{\sqrt{\sigma^2_n(\omega|x,y))}}\right)$ is given by \begin{equation*}
\frac{\partial}{\partial x_j} \Phi \left(\frac{\gamma-\mu_n(\omega|x,y)}{\sqrt{\sigma^2_n(\omega|x,y)}}\right)=-\phi \left( \frac{\gamma-\mu_n(\omega|x,y)}{\sqrt{\sigma^2_n(\omega|x,y)}}\right)\left( \frac{1}{2} \sigma^2_n(\omega|x,y) ^{-3/2} \frac{\partial}{\partial x_j} \sigma^2_n(\omega|x,y)+\sigma^2_n(\omega|x,y) ^{-1/2} \frac{\partial}{\partial x_j} \mu_n(\omega|x,y)\right).
% \label{eq:derivative}
\end{equation*}
Here we have \begin{equation*}
\frac{\partial}{\partial x_j} \mu_n(\omega|x,y)=\frac{\partial}{\partial x_j} r_n(\omega|x) R_{n}(x)^{-1} [(Y_{n},y)-\beta]+ r_n(\omega|x) \frac{\partial}{\partial x_j}R_{n}(x)^{-1} [(Y_{n},y)-\beta]+ r_n(\omega|x) R_{n}(x)^{-1} \frac{\partial}{\partial x_j}(Y_{n},y)
\end{equation*} and \begin{equation*}
 \frac{\partial}{\partial x_j} \sigma^2_n(\omega|x,y) =-\tau^2 \left(2 \frac{\partial}{\partial x_j} r_n(\omega|x)'R_{n}(x)^{-1} r_n(\omega|x) +r_n(\omega|x)'\frac{\partial}{\partial x_j} R_{n}(x)^{-1} r_n(\omega|x) \right),
\end{equation*}
where we use $r_n(\omega|x)=r(\omega|(X_n,x))\in\mathbb R^{n+1}$ to denote the vector whose $i$th element is $r(\omega,x^i)$ for $i=1,\ldots,n$ and $(n+1)$th element is $r(\omega,x)$, $R_{n}(x)\in\mathbb R^{(n+1)\times(n+1)}$ to denote the matrix whose $(i,j)$th entry is $r(x^i,x^j)$ for $i=1,...,n$ and $j=1,...,n$, $(i,n+1)$th entry is $r(x^i,x)$ for $i=1,...,n$, $(n+1,j)$th entry is $r(x,x^j)$ for $j=1,...,n$, and $(n+1,n+1)$th entry is $r(x,x)$.

Furthermore, we have \begin{equation*}
\frac{\partial}{\partial x_j} R_{n}(x)^{-1}= R_{n}(x)^{-1} \frac{\partial}{\partial x_j} R_{n}(x) R_{n}(x)^{-1}.
\end{equation*}
The vector $\frac{\partial}{\partial x_j} r_n(\omega|x)$ has 0 in all entries but the last, which is equal to $\frac{\partial}{\partial x_j} r(\omega,x)$. $\frac{\partial}{\partial x_j} R_{n}(x)$ has 0 in all entries except the last row and column, where the $(i,n+1)$th entry and $(n+1,i)$th entry is equal to $\frac{\partial}{\partial x_j} r(x^i,x)$ for $i=1,...,n$ where $x^i$ denotes the $i$th observation. The vector $\frac{\partial}{\partial x_j}(Y_{n},y)$ has 0 in all entries but the last, which is equal to \begin{equation*}
\frac{1}{2}  \sigma^2_{n}(x) ^{-1/2} z \frac{\partial}{\partial x_j}\sigma^2_{n}(x)+\frac{\partial}{\partial x_j} \mu_{n}(x).
\end{equation*}
Lastly, we have \begin{equation*}
\frac{\partial}{\partial x_j} \mu_{n}(x)= \frac{\partial}{\partial x_j} r_{n} (x) R_{n}^{-1} [Y_{n}-\beta]
\end{equation*} and \begin{equation*}
\frac{\partial}{\partial x_j}\sigma^2_{n}(x)=-\tau^2 \left(2 \frac{\partial}{\partial x_j} r_{n}(x)'R_{n}^{-1} r_{n} (x) \right),
\end{equation*}
where \begin{equation*}
\frac{\partial}{\partial x_j} r_{n}(x)=\left[  {\begin{array}{c}
   \frac{\partial}{\partial x_j} r(x^1,x)  \\
   \frac{\partial}{\partial x_j} r(x^2,x) \\
   ...\\
   \frac{\partial}{\partial x_j} r(x^n,x)
  \end{array} }  \right]
\end{equation*}
and $R_n\in\mathbb R^{n\times n}$ is a matrix whose $(i,j)$th entry is $r(x^i,x^j)$ for $i=1,...,n$ and $j=1,...,n$, $r_n(x)\in\mathbb R^n$ is a vector whose $i$th element is $r(x^i,x)$ for $i=1,...,n$, and \begin{equation*}
\frac{\partial}{\partial x_j} r(x^i,x)=r(x^i,x)(-2 \theta(x^i_j-x_j)).
\end{equation*}
The above estimator is only a heuristic that roughly resembles an infinitesimal perturbation analysis. Upon closer inspection, one can see that the term $\sigma_n^2(\omega|x,y)$ in the denominator in the formulas above is close to 0 if $\omega$ approaches any of the observed design points, a consequence of the fact that the responses at those points are completely known. This may blow up the gradient estimate. This issue can potentially be addressed by adding artificial small noise to the kriging model to inflate the variance from zero at those positions. An alternative is to use the finite-difference method, although this will reduce the efficiency of the resulting gradient descent algorithm.

% as shown by the following: 
% \theorem Assume that the inner integral of \eqref{eq:obj} does not consider the neighborhood of $x$ and any element in $X_n$. For any $x$ such that $x \notin X_n$, the gradient estimation of the objective function \eqref{eq:obj} using infinitesimal perturbation analysis is unbiased.
% \proof 

% For simplification, we introduce a standard Gaussian random variable $z$ so that $y=\sqrt{\sigma_n^2(x)}z+\mu_n(x)$. By the proposition 
% 2.3 in \citeN{asmussen2007stochastic}, the IPS estimator is unbiased at $x$ if the integrand is differentiable at $x$ and satisfies the Lipschitz condition in a nonrandom neighborhood of $x$.
% We firstly show $(P_n-P_{n}(x,y))^2 $ is differentiable. Since composition of differentiable functions is differentiable, we can prove $(P_n-P_{n}(x,y))^2 $ is differentiable if
% \begin{equation}
% \label{eq:dif_pn+1}
% \int \Phi(\frac{\gamma-\mu_n(\omega|x,y)}{\sqrt{\sigma^2_n(\omega|x,y)}})  dF(\omega)
% \end{equation}
% is differentiable. To show  \eqref{eq:dif_pn+1} is differentiable, by Lebesgue dominated convergence theorem, we need to show the integrand is differentiable and the derivative is bounded by a integrable function $\omega$. Since we integrate on a probability measure $F(\cdot)$, it is sufficient to prove the derivative is bounded. Since $\Phi(\cdot)$ is differentiable and for \begin{equation}
% \frac{\gamma-\mu_n(\omega|x,y)}{\sqrt{\sigma^2_n(\omega|x,y)}},
% \end{equation}
% we need to show $\mu_n(\omega|x,y)$ and $\sigma^2_n(\omega|x,y)$ are differentiable. If we keep expand the formula, we find the key is to check whether $r(x,y)$ is differentiable. Finally, we have \begin{equation}
% \prod_{j=1}^{d} e^{- \theta((x+H)_j-y_j)^2}-\prod_{j=1}^{d} e^{- \theta(x_j-y_j)^2}=\prod_{j=1}^{d} e^{- \theta(x_j-y_j)^2}  e^{- \theta(h^2-2 h y_k+2x_k h)} ,
% \end{equation}
% where $H$ is 0 in all elements except $h$ in the $k$th element. Then we have \begin{equation}
% \lim_{h\rightarrow 0}  \frac{e^{- \theta(h^2-2 h y_k+2x_k h)}}{h} = \lim_{h\rightarrow 0} - \theta (2h-2 y_k+2x_k) e^{- \theta(h^2-2 h y_k+2x_k h)}=- \theta (2 y_k+2x_k),
% \end{equation} 
% which indicates that $r(x,y)$ is differentiable. So we have the integrand of \eqref{eq:dif_pn+1} is differentiable.

% Now we want to prove that the derivative of the integrand in \eqref{eq:dif_pn+1} is bounded. The derivative is given by \eqref{eq:derivative}. Since we have $x\notin X_n$ and the assumption makes $\omega \notin X_n$, any term in the derivatives is bounded. Therefore, we have \eqref{eq:dif_pn+1} differentiable and so that the integrand of \eqref{eq:obj} is differentiable.

% Since the derivative of \eqref{eq:dif_pn+1} is bounded, the derivative of the integrand of \eqref{eq:obj} is \begin{equation}
% 2(P_n-P_{n}(x,y)) \int - \frac{\partial}{\partial x_j} \Phi \left( \frac{\gamma-\mu_n(\omega|x,y)}{\sqrt{\sigma^2_n(\omega|x,y)}}\right) dF(\omega), 
% \end{equation}
% which is also bounded, since $(P_n-P_{n}(x,y))$ is bounded. The bounded derivative implies that the function satisfies Lipschitz condition. $\blacksquare$

% where the derivative of $\frac{\partial}{\partial x} \Phi \left( \frac{\gamma-\mu(\omega|(x,y))}{\sqrt{\sigma^2(x,y))}}\right)$ is given by \begin{equation}
% \frac{\partial}{\partial x} \Phi \left(\frac{\gamma-\mu_{n+1}(\omega)}{\sqrt{\sigma^2_{n+1}(\omega)}}\right)=-\phi \left( \frac{\gamma-\mu_{n+1}(\omega)}{\sqrt{\sigma^2_{n+1}(\omega)}}\right)\left( \frac{1}{2} \sigma^2_{n+1}(\omega) ^{-3/2} \frac{\partial}{\partial x} \sigma^2_{n+1}(\omega)+\sigma^2_{n+1}(\omega) ^{-1/2} \frac{\partial}{\partial x} \mu_{n+1}(\omega)\right).
% \label{eq:derivative}
% \end{equation}
% We have \begin{equation}
% \frac{\partial}{\partial x} \mu_{n+1}(\omega)=\frac{\partial}{\partial x} r_{n+1} (\omega) R_{n+1}^{-1} [Y_{n+1}-\beta]+ r_{n+1} (\omega) \frac{\partial}{\partial x}R_{n+1}^{-1} [Y_{n+1}-\beta]+ r_{n+1} (\omega) R_{n+1}^{-1} \frac{\partial}{\partial x}Y_{n+1}
% \end{equation} and \begin{equation}
%  \frac{\partial}{\partial x} \sigma^2_{n+1}(\omega)=-\tau^2 \left(2 \frac{\partial}{\partial x} r_{n+1}(\omega)'R_{n+1}^{-1} r_{n+1} (\omega) +r_{n+1}(\omega)'\frac{\partial}{\partial x} R_{n+1}^{-1} r_{n+1} (\omega) \right).
% \end{equation}
% Taking a further step, we have \begin{equation}
% \frac{\partial}{\partial x} R_{n+1}^{-1}= R_{n+1}^{-1} \frac{\partial}{\partial x} R_{n+1} R_{n+1}^{-1}.
% \end{equation}
% We know that $\frac{\partial}{\partial x} r_{n+1}(\omega)$ has 0 in all entries except the last element, which is equal to $\frac{\partial}{\partial x} r(\omega,x)$. $\frac{\partial}{\partial x} R_{n+1}$ has 0 in all entries except the last row and column, where the entry $i,(n+1)$ and $(n+1,i)$ is equal to $\frac{\partial}{\partial x} r(x_i,x)$ for $i=1,...,n$ and $x_i$ denotes the $i$th observation. $\frac{\partial}{\partial x}Y_{n+1}$ has 0 in all entries except the last element, which is equal to \begin{equation}
% \frac{1}{2}  \sigma^2_{n}(x) ^{-1/2} z \frac{\partial}{\partial x}\sigma^2_{n}(x)+\frac{\partial}{\partial x} \mu_{n}(x).
% \end{equation}
% We have \begin{equation}
% \frac{\partial}{\partial x} \mu_{n}(x)= \frac{\partial}{\partial x} r_{n} (x) R_{n}^{-1} [Y_{n}-\beta]
% \end{equation} and \begin{equation}
% \frac{\partial}{\partial x}\sigma^2_{n}(x)=-\tau^2 \left(2 \frac{\partial}{\partial x} r_{n}(x)'R_{n}^{-1} r_{n1} (x) \right),
% \end{equation}
% where \begin{equation}
% \frac{\partial}{\partial x} r_{n}(x)=\left[  {\begin{array}{c}
%    \frac{\partial}{\partial x} r(x_1,x)  \\
%    \frac{\partial}{\partial x} r(x_2,x) \\
%    ...\\
%    \frac{\partial}{\partial x} r(x_n,x)
%   \end{array} }  \right].
% \end{equation}
% Finally, we have \begin{equation}
% \frac{\partial}{\partial x^j} r(s,x)=r(s,x)(-2 \theta(s^j-x^j)),
% \end{equation}
% where $x^j$ is the $j$th element of $x$ and $s$ is arbitrary.

% \theorem Assume that the inner integral of \eqref{eq:obj} does not consider the neighborhood of $x$ and any element in $X_n$. For any $x$ such that $x \notin X_n$, the gradient estimation of the objective function \eqref{eq:obj} using infinitesimal perturbation analysis is unbiased.
% \proof By the proposition 
% 2.3 in Asmussen and Glynn (2007), the IPS estimator is unbiased at $x$ if the integrand is differentiable at $x$ and satisfies the Lipschitz condition in a nonrandom neighborhood of $x$.
% We firstly show $(P_n-P_{n+1})^2 $ is differentiable. Since composition of differentiable functions is differentiable, we can prove $(P_n-P_{n+1})^2 $ is differentiable if
% \begin{equation}
% \label{eq:dif_pn+1}
% \int \Phi(\frac{\gamma-\mu(\omega|(X_{n},x),(Y_{n},y))}{\sqrt{\sigma^2(\omega|(X_{n},x),(Y_{n},y))}})  dF(\omega)
% \end{equation}
% is differentiable. To show  \eqref{eq:dif_pn+1} is differentiable, by Lebesgue dominated convergence theorem, we need to show the integrand is differentiable and the derivative is bounded by a integrable function $\omega$. Since we integrate on a probability measure $F(\cdot)$, it is sufficient to prove the derivative is bounded. Since $\Phi(\cdot)$ is differentiable and we have \begin{equation}
% \frac{\gamma-\mu(\omega|(X_{n},x),(Y_{n},y))}{\sqrt{\sigma^2(\omega|(X_{n},x),(Y_{n},y))}},
% \end{equation}
% we need to show $\mu(\omega|(X_{n},x),(Y_{n},y))$ and $\sigma^2(\omega|X_{n+1},Y_{n+1})$ are differentiable. Only $r(x)$ and $Y_{n+1}$ contains $x$, and both terms contains $r(x,y)$. For $r(x,y)$, we have \begin{equation}
% \prod_{j=1}^{d} e^{- \theta((x+H)^j-y^j)^2}-\prod_{j=1}^{d} e^{- \theta(x^j-y^j)^2}=\prod_{j=1}^{d} e^{- \theta(x^j-y^j)^2}  e^{- \theta(h^2-2 h y^k+2x^k h)} ,
% \end{equation}
% where $H$ is 0 in all elements except $h$ in the $k$th element. Then we have \begin{equation}
% \lim_{h\rightarrow 0}  \frac{e^{- \theta(h^2-2 h y^k+2x^k h)}}{h} = \lim_{h\rightarrow 0} - \theta (2h-2 y^k+2x^k) e^{- \theta(h^2-2 h y^k+2x^k h)}=- \theta (2 y^k+2x^k),
% \end{equation} 
% which indicates that $r(x,y)$ is differentiable. So we have the integrand of \eqref{eq:dif_pn+1} is differentiable.

% Now we want to prove that the derivative of the integrand in \eqref{eq:dif_pn+1} is bounded. The derivative is given by \eqref{eq:derivative}. Since we have $x\notin X_n$ and the assumption makes $\omega \notin X_n$, any term in the derivatives is bounded. Therefore, we have \eqref{eq:dif_pn+1} differentiable and so that the integrand of \eqref{eq:obj} is differentiable.

% Since the derivative of \eqref{eq:dif_pn+1} is bounded, the derivative of the integrand of \eqref{eq:obj} is \begin{equation}
% 2(P_n-P_{n+1}) \int - \frac{\partial}{\partial x} \Phi \left( \frac{\gamma-\mu(\omega|X_{n+1},Y_{n+1})}{\sqrt{\sigma^2(\omega|X_{n+1},Y_{n+1})}}\right) dF(\omega), 
% \end{equation}
% which is also bounded, since $(P_n-P_{n+1})$ is bounded. The bounded derivative implies that the function satisfies Lipschitz condition. $\blacksquare$


% Further, we have\begin{equation}
% \frac{\partial}{\partial x} \mu_{n+1}(\omega)=\frac{\partial}{\partial x} r_{n+1} (\omega) R_{n+1}^{-1} [Y_{n+1}-\beta]+ r_{n+1} (\omega) \frac{\partial}{\partial x}R_{n+1}^{-1} [Y_{n+1}-\beta]+ r_{n+1} (\omega) R_{n+1}^{-1} \frac{\partial}{\partial x}Y_{n+1}
% \end{equation} and \begin{equation}
%  \frac{\partial}{\partial x} \sigma^2_{n+1}(\omega)=-\tau^2 \left(2 \frac{\partial}{\partial x} r_{n+1}(\omega)'R_{n+1}^{-1} r_{n+1} (\omega) +r_{n+1}(\omega)'\frac{\partial}{\partial x} R_{n+1}^{-1} r_{n+1} (\omega) \right).
% \end{equation}



% Because \eqref{eq:derivative} is an integral, we use Monte Carlo method to estimate the gradient, which we denote as $\nabla g(\cdot)$. 

With the gradient estimator, we iterate\begin{equation}
x^{(k+1)}=x^{(k)}+a_kg^{(k)}(x^{(k)})
 \label{eq:sa}
\end{equation}
where $g^{(k)}(x)$ denotes the gradient estimator in \eqref{grad estimate}, for $k=1,2,\ldots$ starting from an initial solution $x^{(0)}$ , to optimize the objective function in \eqref{eq:obj} according to a heuristic Robbins-Monro SA. The step size is taken as $a_k=a_0/k$. One may also apply the algorithm at multiple starting points in view of the non-convexity of the problem. 
% One can refer to \citeN{asmussen2007stochastic} Chapter VIII for the behaviors and features of SA using simulation-based gradient estimator. 

% update \cite{robbins1951stochastic}.
% \section{Outline of the Adaptive Sampling Scheme} \label{sec:outline}
% Note that the optimization objective function in Section \ref{sec:optim} is based on an initial observation set. To apply the optimal scheme, we need to design the initial experiments. Therefore, we use standard Kriging design method for the initial experiments. For example, we can use the Latin Hypercube Sampling method. Once we have the initial sample set, we can sequentially select the optimal sample points using the optimization function. 
Overall, to sequentially select the design points, the steps consist of:
\begin{enumerate}
	\item Use a small-sample space-filling design to build an initial observation set $(X_0,Y_0)$ and construct an initial kriging model. 
    \item Approximately solve (\ref{obj}) to select the next design point $x^*$. This involves recursion using \eqref{eq:sa} where $g^{(k)}(x^{(k)}),k=1,2,\ldots$ are estimated by generating i.i.d. samples $\omega_1,\ldots,\omega_m$ from $F$ and $z_1,\ldots,z_m$ from standard Gaussian as described above.  \label{item:step2}
    \item Conduct an experiment at $x^*$ and add $x^*$ and the associated experimental outcome to the observation set $(X_n,Y_n)$ to get $(X_{n+1},Y_{n+1})$.\label{item:step3} 
        \item Update the kriging model using the observation set $(X_{n+1},Y_{n+1})$. \label{item:step4}
	    \item Repeat steps \ref{item:step2}, \ref{item:step3} and \ref{item:step4} until the kriging model is acceptable.
\end{enumerate}

% We note that we can use the discretization approach or the stochastic gradient approach for step \ref{item:step3}. Using the discretization approach requires a decent method to discretize the sampling space and an appropriate ranking and selection scheme to find the optimal point. The stochastic gradient approach searches for the optimal sampling point on continuous space, while the algorithm cannot guarantee the convergence to optimal. The stochastic feature and the selection of step parameter introduce complexity to the implementation. 

% The algorithm might converge to a local optimum or to a non-optimal point. To tackle these problems, we need to carefully set up the algorithm. We can use multiple start points for the search algorithm and compare the obtained points. We evaluate the performance of the stochastic gradient approach in Section \ref{sec:numerical}.

\section{NUMERICAL EXAMPLES} \label{sec:numerical}
This section shows some numerics on our information criterion and simple illustrations of our procedure.

\subsection{Illustration of the Information Criterion}

% Once we have a Kriging model, we can use it to predict the outcome at any given point. Adding one more sample to the current Kriging model will change the prediction of the Kriging model and therefore change the probability estimation using the Kriging model. We defined the information gain in Section \ref{sec:optim} to search for design points. In this section, we want illustrate the interpretation of the information gain.

Here we present an example to illustrate the intuition behind the proposed information criterion. By contrasting with a simple alternative criterion, we demonstrate the relation between the proposed criterion and the underlying probability distribution.

Consider the generic target probability of interest $P(f(\omega)>\gamma)$, where $\omega \in \mathcal{X}$ is a random object with probability $P$. In addition to \eqref{obj}, we consider an alternative criterion to select the next design point by maximizing the pointwise variance of $I(f(x)>\gamma)$ over $x\in \mathcal{X}$ under the posterior distribution on $f$, namely $f(x)\sim N(\mu_n(x),\sigma_n^2(x)$), where $I(\cdot)$ is the indicator function. In other words, we maximize 
\begin{equation}
E_n\left( I\left(f(x) > \gamma\right)- P_n\left(f(x) > \gamma\right) \right)^2\label{new criterion}
\end{equation}
where $E_n[\cdot]$ and $P_n(\cdot)$ refer to the conditional distribution on $X_n,Y_n$ as before. Criterion \eqref{new criterion}, which we name the local prediction impact for convenience, does not depend on the distribution of $\omega$ but only measures the uncertainty (or confidence of our knowledge) on the values of the function $f(\cdot)$ at different points. This contrasts our suggestion in \eqref{obj} that accounts for both the uncertainty on $f$ and the distribution of $\omega$, and in this sense \eqref{new criterion} is a less comprehensive measure. In general, one would expect that the information gain measured by \eqref{obj} is large when the local prediction impact is large and the position of interest is ``important" according to the distribution of $\omega$. On the other hand, a position with a large local prediction impact may not necessarily be important in determining the estimate of $P(f(\omega)>\gamma)$, since the latter depends on the distribution of $\omega$. 

% lead to a gain On the other hand, a point with large local impact can have small information gain if the underlying density of the point is rare. Here, we use a lane change scenario example to show this interpretation.


% We interpret the information gain from two aspects: the local prediction impact and the underlying distribution of the design points. When
% First, we propose an intuitive measure for the impact on local prediction. We want to use the Kriging model to interpret the difference between the prediction and possible outcome at a given point. We consider that if we sample at an arbitrary point $x$ and we are interested in the outcome $I(f(x) > \gamma)$. Note that $f(x)|(X,Y)$ is random regarding the Kriging model, we can interpret that the outcome $I(f(x) > \gamma)$ of a sample at $x$   is a random variable. We want to measure the difference between this outcome and the prediction of the Kriging prediction $ P\left[f(x) > \gamma|(X,Y) \right]$ on $x$. We take the squared of the difference and integral over all possible $f(x)$. This gives us \begin{equation}
% E\left[ \left( I\left(f(x) > \gamma|(X,Y) \right)- E\left[f(x) > \gamma|(X,Y) \right] \right)^2 |(X,Y)\right]
% \end{equation}
% Note that this is the variance of $I\left(f(x) > \gamma|(X,Y) \right)$. Considering that a larger variance of $I\left(f(x) > \gamma|(X,Y) \right)$ indicates that the Kriging gives less assured prediction on $x$, it is reasonable to use the variance to measure the impact on a local point.

We demonstrate the two criteria with a study of lane change scenarios described in Section \ref{sec:task}. We assume that the AV uses a deterministic system with Adaptive Cruise Control (ACC) and Autonomous Emergency Braking (AEB) \shortcite{ulsoy2012automotive} (see Fig. \ref{fig:av_model}), but this is supposedly unknown to the tester. 
% With a random generated scenario, we can simulate the interaction of the vehicles in the lane change procedure and determine whether a crash will happen. This model allows us to use Monte Carlo simulation to estimate the probability of certain events (generally risk events) occcurring during the lane change scenario. Previously \shortciteN{zhao2017accelerated} and \shortciteN{huang2017accelerated}, we evaluated the probability of crash, conflict, and injury as the events of interest.
There are three key variables that constitute the scenario, namely the frontal vehicle's velocity $v$, range $R$ and time to collision $TTC$, where we define $TTC$ as
\begin{equation*}
	TTC=- \frac{R}{\dot{R}}.
\end{equation*}
As described in \shortciteN{zhao2017accelerated}, when the velocity $v$ is between 5 to 15 $m/s$, the other two variables $R^{-1}$ and $TTC^{-1}$ are independent of each other.  $TTC^{-1}$ can be modeled by an exponential distribution and $R^{-1}$ by a Pareto distribution. Here we define $\omega=[TTC^{-1},R^{-1}]$, and we are interested in estimating $P(\max_{t\in[0,T]}1/R_L(t)>1/2)$, i.e., the probability that the two vehicles have a minimum range smaller than 2 meters, when the velocity of the leading vehicle $v$ is set to lie in the aforementioned range. 
\begin{figure}[ht]
							\centering
							
							\begin{minipage}[b]{0.4\textwidth}
								\centering
								\includegraphics[width=\linewidth]{av_model.png}
								\caption{An example of AV control mechanism.}
								\label{fig:av_model}
							\end{minipage}
                            \begin{minipage}[b]{0.4\textwidth}
								\centering
                            \includegraphics[width=\linewidth]{problem_setting.png}
	\caption{Prediction of a kriging model in the lane change setting.}
	\label{fig:exp3_set}
							\end{minipage}%
			\end{figure}


We use a kriging model with parameters $\beta=0$, $\tau^2=0.01$ and $\theta=50$. We set the prediction threshold $\gamma=0.5$ for simplicity. The zero mean of $\beta$ is chosen to reflect the belief that the response of a scenario with no information is far from being a critical event. We choose the value of $\tau^2$ which intuitively puts $\gamma=0.5$ to be three standard deviations higher than the mean of a scenario $x$ that has no information (i.e., $f(x) \sim N(0,0.01)$). $\theta=50$ is selected to make the correlation between scenarios with distance $0.05$ (believed to represent initial conditions with different AV behaviors) to be small enough (less than $0.01$). 

We use 20 initial design points to build the model and its value of $I(f(x)>\gamma)$ is shown in Fig. \ref{fig:exp3_set}. The blue dots represent a sample distribution of the variable $\omega$. Red circles are existing design points with return 0 and red crosses are existing design points with return 1. We consider four arbitrarily picked new design points (which we call points A, B, C and D) shown by the red stars, whose coordinates are shown in the first row of Table \ref{table:exp3}. The local prediction impacts and the information gains depicted by the objective in \eqref{obj} of these design points are shown in the second and third rows respectively.
% The four target design points are represent by the red stars. 

We see that points B and C have smaller local prediction impacts than points A and D, which can be attributed to the vicinity of their positions to those of the historical data that subsequently reduces the uncertainty of $f$. This translates to a smaller variance of $I(f(x)>\gamma)$ and hence a smaller local prediction impacts. Relatedly, these points also have a low information gain measured by \eqref{obj}. However, point D, even though far away from the positions of the historical data, has an even lower information gain. This can be attributed to the tiny density of $\omega$ at this point, which makes the overall information gain low. In contrary, point A has a higher density of $\omega$ and consequently a higher information gain.

% the correlation is large between these points and the known data points. The variance of the predictions is low which indicates that these two points have relatively low impact on local prediction. We observe that both points A and D have large variance on $I(f(x)>\gamma)$, which means that sampling at this point has a high impact on local prediction.

% The information gains for points B and C are low. We interpret that the low information gain is caused by both low distribution density and local prediction impact. While points A and D have large impact on local prediction, since the distribution density for point D is too rare, the information gain is even smaller than points B and C. While point A has a large distribution density, so the information gain is larger than other points. 

\begin{table}[ht]
\centering
\caption{Local prediction impacts and information gains of 4 arbitrarily picked design points.}
\renewcommand{\arraystretch}{1.5}
\label{table:exp3}
%\resizebox{0.6\columnwidth}{!}
\end{table}

\subsection{Example of the Sequential Learning Approach}
To illustrate our sequential learning approach, we use a simple hypothetical problem where we define the probability of interest as $P(\omega_1 + \omega_2 > 2)$, with two random objects $\omega_1,\omega_2$ each following a standard Gaussian distribution (in the lane change scenario described before $\omega_1,\omega_2$ would correspond to the initial conditions such as frontal vehicle velocity, with a correspondingly more sophisticated $f$ function). The true probability is $1- \Phi (\frac{2}{\sqrt{2}}) \approx 0.0786$.

We use a kriging model with parameters $\beta=0$, $\tau^2=1$, and $\theta=1$. Here the parameters are arbitrarily chosen, as we assume that no prior information is available. We start with 20 initial design points. In the SA scheme, we use $a_k=a_0/k$ as the step size parameter with $a_0=20$, and we terminate the scheme after 50 iterations, at each new design point. The gradient estimator is averaged from 1,000 samples. For illustration, we use 10,000 samples to estimate the target probability under the kriging model to assess its error relative to the truth.  

Fig. \ref{fig:est_prob} shows that as we collect more observations to update the kriging model, the probability estimate gradually converges to the true probability. At each step, we start the SA from a randomly generated point using a standard Gaussian distribution. To illustrate the benefit from the optimization step, Fig. \ref{fig:compare} compares our approach with random sampling at each step, where this random sample is precisely the starting point of our SA scheme. We observe that our sequential learning approach converges to the true probability quickly in the first few steps, but the convergence slows down as the learning progresses. This may be caused by a saturation in terms of the highest accuracy affordable by the SA's noises, as well as the heuristic nature of our approach. Finally, Fig. \ref{fig:stationary1} and \ref{fig:stationary2} show the probability estimates when we use SA with starting points fixed at $(1,1)$ and $(0,0)$ respectively, at each learning step. We see that the probability estimates move towards the truth regardless of the starting points, giving a sign that the SA algorithm is at least working. Moreover, starting from $(1,1)$ appears to achieve faster convergence, which can be reasoned by the fact that $(1,1)$ is closer to the boundary of the event $\omega_1+\omega_2>2$ that facilitates the involved learning process. We note that the lines in the figures appear a bit fluctuant as they are illustrated in the scale of the probability estimates, which is small relative to the simulation replication size we use to generate them (i.e., $10,000$). Further investigation is clearly needed, but the above observations aim to show some preliminary insights on the behavior of our approach and confirm its potential.


\begin{figure}[ht]
							\centering
							\begin{minipage}[b]{0.5\textwidth}
								\centering
                                \captionsetup{width=.9\linewidth}
	\includegraphics[width=\linewidth]{Prob_est.png}
	\caption{Changes in probability estimates as new observations are collected to update the kriging model.}
	\label{fig:est_prob}
							\end{minipage}%
							\begin{minipage}[b]{0.5\textwidth}
								\centering
                                \captionsetup{width=.9\linewidth}
	\includegraphics[width=\linewidth]{compare.png}
	\caption{Comparison between probability estimates of the learning approach and random sampling without optimization.}
	\label{fig:compare}
							\end{minipage}
			\end{figure}
            
\begin{figure}[ht]
							\centering
							\begin{minipage}[b]{0.5\textwidth}
								\centering
                                \captionsetup{width=.9\linewidth}
	\includegraphics[width=\linewidth]{stationary.png}
	\caption{Sequential selection of design points using SA with starting point fixed at $(1,1)$.}
	\label{fig:stationary1}
							\end{minipage}%
							\begin{minipage}[b]{0.5\textwidth}
								\centering
                                \captionsetup{width=.9\linewidth}
	\includegraphics[width=\linewidth]{stationary2.png}
	\caption{Sequential selection of design points using SA with starting point fixed at $(0,0)$.}
	\label{fig:stationary2}
							\end{minipage}
			\end{figure}
            


% To further validate the proposed sampling approach, we fix the start point for the stochastic approximation in each iteration. Fig.We observe that $(1,1)$ is a better start point, since the convergence of estimation takes less steps then starting from $(0,0)$. This might because starting from $(1,1)$ gives better local optimum. The proposed approach is valid for both start points.



\section{CONCLUSION} \label{sec:conclusion}

This paper presents a sequential learning approach based on using kriging models to approximate AV behaviors, to reduce on-track experimentation for AV safety evaluation. The approach relies
on a heuristic simulation-based gradient descent procedure to search for the best next test scenario in terms of maximizing an information criterion regarding the accuracy of conflict probability evaluation. We derive a gradient estimator and investigate the performance of our procedure. Numerical examples show that our approach sequentially improves our probability estimate, and appears to perform better than simple strategies such as random scenario sampling. Future work includes the studies of further assumptions of the kriging models in the AV evaluation context and developments of scenario search procedures that are both more efficient and theoretically sound.

% However, the fluctuancy of the prediction indicates that the method is instable in selecting the ``optimal'' test scenario. This might be caused by the stochasticity or the blow-up issue of the gradient estimator.

\section*{ACKNOWLEDGMENTS}
We gratefully acknowledge support from the National Science Foundation under grants CMMI-1542020, CMMI-1523453 and CAREER CMMI-1653339.

% The authors acknowledge support from the University of Michigan Mobility Transformation Center under grant number N021552.




% Please don't exchange the bibliographystyle style
% \bibliographystyle{wsc}
% % AUTHOR: Include your bib file here
% \bibliography{Mendeley_MTC_AE}

\bibliographystyle{wsc}
 \bibliography{wsc_citation.bib}
 
 
% \section*{REFERENCES}

% \begin{hangref}
% \item National Highway Traffic Safety Administration.  2013. {\it Preliminary statement of policy concerning automated vehicles}.

% \item FESTA-Consortium. 2008. {\it FESTA Handbook Version 2 Deliverable T6. 4 of the Field opErational teSt supporT Action}, Brussels: European Commission.

% \item Akamatsu, M., Green, P. and Bengler, K., 2013. {\it Automotive technology and human factors research: Past, present, and future}, International journal of vehicular technology.

% \item Yang, H.H. and Peng, H., 2010. {\it Development and evaluation of collision warning/collision avoidance algorithms using an errable driver model.}, Vehicle system dynamics.

% \item Lee, K., 2004.  {\it  Longitudinal driver model and collision warning and avoidance algorithms based on human driving databases.}, University of Michigan.

% \item Woodrooffe, J., Blower, D., Bao, S., Bogard, S., Flannagan, C., Green, P.E. and LeBlanc, D., 2014. {\it  Performance characterization and safety effectiveness estimates of forward collision avoidance and mitigation systems for medium/heavy commercial vehicles}, UMTRI-2011-36.


% \item Zhao, D., Lam, H., Peng, H., Bao, S., LeBlanc, D.J., Nobukawa, K. and Pan, C.S., 2017. {\it Accelerated evaluation of automated vehicles safety in lane-change scenarios based on importance sampling techniques}, IEEE transactions on intelligent transportation systems, 18(3), pp.595-607.

% \item Huang, Z., Zhao, D., Lam, H. and LeBlanc, D.J., 2017. {\it Accelerated evaluation of automated vehicles using piecewise mixture models}, arXiv preprint arXiv:1701.08915.

% \item Peng, H. and Leblanc, D., 2012. {\it Evaluation of the performance and safety of automated vehicles}, White Pap. NSF Transp. CPS Work.
% \item Kleijnen, J.P., 2009. {\it Kriging metamodeling in simulation: A review}, European journal of operational research, 192(3), pp.707-716.


% \item Kleijnen, J.P., 2008.  {\it Design and analysis of simulation experiments}, New York: Springer.

% \item Kleijnen, J.P. and Van Beers, W.C., 2004. {\it Application-driven sequential designs for simulation experiments: Kriging metamodelling}, Journal of the Operational Research Society, 55(8), pp.876-883.

% \item Van Beers, W.C. and Kleijnen, J.P., 2008. {\it Customized sequential designs for random simulation experiments: Kriging metamodeling and bootstrapping}, European journal of operational research, 186(3), pp.1099-1113.

% \item Huang, D., Allen, T.T., Notz, W.I. and Zeng, N., 2006a. {\it Global optimization of stochastic black-box systems via sequential kriging meta-models}, Journal of global optimization, 34(3), pp.441-466.

% \item Huang, D., Allen, T.T., Notz, W.I. and Miller, R.A., 2006b. {\it Sequential kriging optimization using multiple-fidelity evaluations}, Structural and Multidisciplinary Optimization, 32(5), pp.369-382.

% \item Bezzina, D. and Sayer, J., 2014. {\it Safety pilot model deployment: Test conductor team report}, Report No. DOT HS, 812, p.171.

% \item Ulsoy, A.G., Peng, H. and Çakmakci, M., 2012. {\it Automotive control systems}, Cambridge University Press.

% \item Huang, Z., Zhao, D., Lam, H., LeBlanc, D.J. and Peng, H., 2017. {\it Evaluation of Automated Vehicles in the Frontal Cut-in Scenario-an Enhanced Approach using Piecewise Mixture Models}, Proceedings of IEEE International Conference on Robotics and Automation 2017.

% \item Rasmussen, C.E., 2006.  {\it Gaussian processes for machine learning}.

% \item Staum, J., 2009 {\it Better simulation metamodeling: The why, what, and how of stochastic kriging.}, Proceedings of the 2009 Winter Simulation Conference.


% \item Frazier, P.I., Powell, W.B. and Dayanik, S., 2008. {\it A knowledge-gradient policy for sequential information collection}, SIAM Journal on Control and Optimization, 47(5), pp.2410-2439.


% \item Robbins, H. and Monro, S., 1951. {\it A stochastic approximation method}, The annals of mathematical statistics, pp.400-407.

% \item Asmussen, S. and Glynn, P.W., 2007. {\it Stochastic simulation: algorithms and analysis (Vol. 57)}, Springer Science \& Business Media.

% \end{hangref}



\section*{AUTHOR BIOGRAPHIES}
\noindent {\bf ZHIYUAN HUANG} is a second-year Ph.D. student in the Department of Industrial and Operations Engineering at the University of Michigan, Ann Arbor. His research interests include simulation and stochastic optimization. His email address is \email{zhyhuang@umich.edu}.\\

\noindent {\bf HENRY LAM} is an Assistant Professor in the Department of Industrial and Operations Engineering at the University of Michigan, Ann Arbor. His research focuses on stochastic simulation, risk analysis, and
simulation optimization. His email address is \email{khlam@umich.edu}.\\




\noindent {\bf Ding Zhao} is a Assistant Research Scientist in the Department of Mechanical Engineering at the University of Michigan, Ann Arbor. His research focuses on Connected and Automated Vehicles (CAVs) using synthesized approaches rooted in advanced statistics, modeling, optimization, dynamic control, and big data analysis. His email address is \email{zhaoding@umich.edu}.\\






\end{document}

