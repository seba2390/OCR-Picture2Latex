% CVPR 2022 Paper Template
% based on the CVPR template provided by Ming-Ming Cheng (https://github.com/MCG-NKU/CVPR_Template)
% modified and extended by Stefan Roth (stefan.roth@NOSPAMtu-darmstadt.de)

\documentclass[10pt,twocolumn,letterpaper]{article}

%%%%%%%%% PAPER TYPE  - PLEASE UPDATE FOR FINAL VERSION
% \usepackage[review]{cvpr}      % To produce the REVIEW version
% \usepackage{cvpr}              % To produce the CAMERA-READY version
\usepackage[pagenumbers]{cvpr} % To force page numbers, e.g. for an arXiv version

% Include other packages here, before hyperref.
\usepackage{graphicx}
\usepackage{amsmath}
\usepackage{amssymb}
\usepackage{booktabs}
\usepackage{amsfonts}       % blackboard math symbols
\usepackage{nicefrac}       % compact symbols for 1/2, etc.
\usepackage{kotex}
\usepackage{microtype}      % microtypography
\usepackage{multirow}
\usepackage{makecell}
\usepackage{pifont}         % symbols
\usepackage{enumitem}
\usepackage{scalerel}
\usepackage{soul}
\usepackage{setspace}


% It is strongly recommended to use hyperref, especially for the review version.
% hyperref with option pagebackref eases the reviewers' job.
% Please disable hyperref *only* if you encounter grave issues, e.g. with the
% file validation for the camera-ready version.
%
% If you comment hyperref and then uncomment it, you should delete
% ReviewTempalte.aux before re-running LaTeX.
% (Or just hit 'q' on the first LaTeX run, let it finish, and you
%  should be clear).
\usepackage[pagebackref,breaklinks,colorlinks]{hyperref}


% Support for easy cross-referencing
\usepackage[capitalize]{cleveref}
\crefname{section}{Sec.}{Secs.}
\Crefname{section}{Section}{Sections}
\Crefname{table}{Table}{Tables}
\crefname{table}{Tab.}{Tabs.}


\begin{document}

%%%%%%%%% TITLE - PLEASE UPDATE
\title{Bridging the Gap between Classification and Localization\\ for Weakly Supervised Object Localization}

\author{Eunji Kim$^1$ ~~~~~~~ Siwon Kim$^1$ ~~~~~~~  Jungbeom Lee$^1$  ~~~~~~~  Hyunwoo Kim$^2$  ~~~~~~~  Sungroh Yoon$^{1, 3}$\thanks{Correspondence to: Sungroh Yoon (sryoon@snu.ac.kr).}\\
$^1$ Department of Electrical and Computer Engineering, Seoul National University ~
$^2$ LG AI Research\\
$^3$ Interdisciplinary Program in AI, AIIS, ASRI, INMC, and ISRC, Seoul National University\\
{\tt\small \{kce407, tuslkkk, jbeom.lee93\}@snu.ac.kr, hwkim@lgresearch.ai, sryoon@snu.ac.kr}}
\maketitle

\newcommand{\xmark}{\text{\ding{55}}}
\newcommand{\cmark}{\text{\ding{51}}}

%%%%%%%%% ABSTRACT
\begin{abstract}
Weakly supervised object localization aims to find a target object region in a given image with only weak supervision, such as image-level labels. Most existing methods use a class activation map (CAM) to generate a localization map; however, a CAM identifies only the most discriminative parts of a target object rather than the entire object region. In this work, we find the gap between classification and localization in terms of the misalignment of the directions between an input feature and a class-specific weight. We demonstrate that the misalignment suppresses the activation of CAM in areas that are less discriminative but belong to the target object. To bridge the gap, we propose a method to align feature directions with a class-specific weight. The proposed method achieves a state-of-the-art localization performance on the CUB-200-2011 and ImageNet-1K benchmarks.
\end{abstract}

%%%%%%%%% BODY TEXT
\section{Introduction}\label{introuction}
The number of videos available online is growing at an unprecedented speed.
Recent work~\cite{escorcia2019temporal,lei2020tvr} introduced the Video Corpus Moment Retrieval (VCMR) task: given a natural language query, a system needs to retrieve a short moment from a large video corpus. 
Figure~\ref{fig:data_example} shows a VCMR example.
Compared to the standard text-to-video retrieval task~\cite{xu2016msr,yu2018joint}, it allows more fine-grained moment-level retrieval, as it requires the system to not only retrieve the most relevant videos, but also localize the most relevant moments inside these videos. 
Various datasets~\cite{Krishna2017DenseCaptioningEI,anne2017localizing,gao2017tall,lei2020tvr} have been proposed or adapted for the task. 
However, they are all created for a single language (English), though the application could be useful for users speaking other languages as well. 
Besides, it is also unclear whether the progress and findings in one language generalizes to another language~\cite{bender2009linguistically}.
While there are multiple existing multilingual image datasets~\cite{gao2015you,elliott-etal-2016-multi30k,shimizu2018visual,pappas2016multilingual,lan2017fluency,li2019coco}, the availability of multilingual video datasets~\cite{Wang_2019_ICCV,chen2011collecting} is still limited.


\begin{figure}[!t]
\begin{center}
  \includegraphics[width=0.99\columnwidth]{res/example_vcmr.pdf}
  \vspace{-12pt}
  \caption{
  A \DsetName~example in the Video Corpus Moment Retrieval (VCMR) task. Ground truth moment is shown in \textit{\textcolor{green}{green}} box. Colors in the query text indicate whether the words are more related to video (\textcolor{orchid}{orchid}) or  subtitle (\textcolor{salmon}{salmon}) or both (\textcolor{orange}{orange}). 
  The query and the subtitle text are presented in both English and Chinese. 
  The video corpus typically contains thousands of videos, for brevity, we only show 3 videos here.
  }
  \label{fig:data_example}
  \end{center}
\end{figure}


Therefore, we introduce~\DsetName, a large-scale, multilingual moment retrieval dataset, with 218K human-annotated natural language queries in two languages, English and Chinese. 
\DsetName~extends the TVR~\cite{lei2020tvr} dataset by collecting paired Chinese queries and Chinese subtitle text (see Figure~\ref{fig:data_example}).
We choose TVR over other moment retrieval datasets~\cite{Krishna2017DenseCaptioningEI,anne2017localizing,gao2017tall} because TVR is the largest moment retrieval dataset, and also has the advantage of having dialogues (in the form of subtitle text) as additional context for retrieval, in contrast to pure video context in the other datasets.
We further propose \ModelName, a compact, multilingual model that learns jointly from both English and Chinese data for moment retrieval. 
Specifically, on top of the state-of-the-art monolingual moment retrieval model XML~\cite{lei2020tvr}, we enforce encoder parameter sharing~\cite{sachan2018parameter,dong2015multi} where the queries and subtitles from the two languages are encoded using shared encoders. 
We also incorporate a language neighborhood constraint~\cite{wang2018learning,kim2020mule} to the output query and subtitle embeddings. 
It encourages sentences of the same meaning in different languages to lie close to each other in the embedding space.
Compared to separately trained monolingual models, \ModelName~substantially reduces the total model size while improving retrieval performance (over monolingual models) as we show in Section~\ref{sec:experiments}. 
Detailed dataset analyses and model ablations are provided.







\section{Related Work}
The WSOL method trains a model to localize objects using image-level labels. Zhou~\etal~\cite{zhou2016learning} introduce a CAM to identify the location of a target object via GAP layer~\cite{lin2013network}. However, it fails to identify the entire object region.

Various methods have been proposed to activate the entire object region in a CAM. HaS~\cite{singh2017hide} trains a classifier using images that are erased with a random patch. ACoL~\cite{zhang2018adversarial} employs two parallel classifiers to identify complementary regions. ADL~\cite{choe2019attention,choe2020attention} stochastically drops out the attentive feature in a single forward pass. Ki~\etal~\cite{ki2020sample} introduced contrastive learning with foreground features and background features. EIL~\cite{mai2020erasing} adopts an additional forward pass to classify with the feature whose highly activated regions are erased. SPG~\cite{zhang2018self} utilizes a deep feature to guide a shallow feature and $\text{I}^2\text{C}$~\cite{zhang2020inter} uses pixel-level correlations between two different images. CutMix~\cite{yun2019cutmix} combines two patches from different images and assigns a new class label based on the area of each patch. DANet~\cite{xue2019danet} leverages divergent activations with the hierarchy of classification labels.

\begin{figure}[t]
	\centering
    \includegraphics[width=0.95\columnwidth]{figures/fig_cam_examples.pdf}
    \vspace{-0.5em}
    \caption{Examples of CAM and decomposed terms $\mathcal{F}$ and $\mathcal{S}$ from a vanilla model. The CAMs and $\mathcal{F}$ are normalized as in $[0, 1]$ for visualization. It shows the misalignment of the feature directions with the class-specific weights.
    }
    \label{fig:cam_norm_sim}
\end{figure}
\begin{figure*}[t]
	\centering
    \includegraphics[width=0.95\textwidth]{figures/fig_oveview.pdf}
    \vspace{-0.5em}
    \caption{Overview of the proposed method. It consists of two strategies: feature direction alignment and consistency with attentive dropout.}
    \label{fig:overall}
\end{figure*}

There have been attempts to obtain localization maps in different ways, pointing out the limitations of CAM-based methods. Pan~\etal~\cite{pan2021unveiling} proposed a method to utilize high-order point-wise correlation to generate localization maps. Kim~\etal~\cite{kim2021keep} proposed a CALM that learns to predict the location of the cue for recognition.

Several normalization methods have been proposed to obtain the bounding boxes around predicted object locations from a continuous localization map. Bae~\etal~\cite{bae2020rethinking} proposed several methods to address the bias in GAP, including a new normalization method, PaS, which restricts the maximum value of the activation map. IVR~\cite{kim2021normalization} is a normalization method that restricts the minimum value of the activation map.

Some works have adopted an auxiliary module for localization besides classification. GC-Net~\cite{lu2020geometry} adopts a separate detector for localization trained with a geometric constraint. FAM~\cite{meng2021foreground} generates a class-agnostic foreground map through a memory mechanism. ORNet~\cite{xie2021online} adopts an additional activation map generator and refines the activation map in an online manner. PSOL~\cite{zhang2020rethinking}, SLT-Net~\cite{guo2021strengthen}, and SPOL~\cite{wei2021shallow} use two separate networks for classification and localization.

Our method aims to address the gap between classification and localization without adopting any auxiliary module. The methods that adopt additional modules or even separate models use more parameters and computational resources.  Therefore, we compare our method mainly with the WSOL methods that use a single branch, for a fair comparison.
\section{Finding the Gap with CAM Decomposition}
Given an input image $x$ and a typical image classifier comprising convolutional layers and a GAP followed by an FC layer, a CAM for target class $c$ is computed as follows:
\begin{equation}\label{eq:cam}
\texttt{CAM}(x) = \mathbf{w}^\intercal_c F(x).
\end{equation}
$F(x)\in\mathbb{R}^{H \times W \times D}$ is the feature map before the GAP, and $\mathbf{w}_c\in\mathbb{R}^{D}$ is the weight of the FC layer connected to class $c$, where $H$, $W$, and $D$ are the height, width, and dimension, respectively.
Eq.~\ref{eq:cam} implies that the value of CAM at each spatial location is the dot product of two vectors, $\mathbf{w}_c$ and $F_u(x)$, where $u\in\{1, ..., HW\}$ is the index of spatial location.
It can be decomposed as follows:
\begin{equation}\label{eq:cam_each}
\begin{aligned}
\texttt{CAM}_u(x) = & \mathbf{w}_c \cdot F_u(x) \\
= & \|\mathbf{w}_c\|\|F_u(x)\| \underbrace{\frac{\mathbf{w}_c \cdot F_u(x)}{\|\mathbf{w}_c\|\|F_u(x)\|}}_{\textstyle S(\mathbf{w}_c,F_u(x))},
\end{aligned}
\end{equation}
where $S(\mathbf{a},\mathbf{b})$ is the cosine similarity between the two vectors, $\mathbf{a}$ and $\mathbf{b}$.
When generating a CAM, target class $c$ is fixed and $\|\mathbf{w}_c\|$ is the same for every $u$.
The CAM value at each position can now be interpreted as the product of the norm of the feature vector at the corresponding location and the similarity between the feature vector and class-specific weight vector.
Let $\mathcal{F}\in\mathbb{R}^{H \times W}$ and $\mathcal{S}\in\mathbb{R}^{H \times W}$ be the norm map and the similarity map, respectively, where $\mathcal{F}_u=\|F_u\|$ and $\mathcal{S}_u=S(\mathbf{w}_c, F_u(x))$. Subsequently, CAM can be rewritten as
\begin{equation}\label{eq:cam_abb}
\texttt{CAM}(x) = \|\mathbf{w}_c\|\cdot\mathcal{F}\odot\mathcal{S}.
\end{equation}
To localize the target object accurately, both $\mathcal{F}_u$ and $\mathcal{S}_u$ should be large for $u$ belonging to the object.

Likewise, the classification score can be interpreted with the output of the GAP, $f(x)=\text{GAP}(F(x))\in\mathbb{R}^{\rm{D}}$.
\begin{equation}\label{eq:logit}
\begin{aligned}
\texttt{logit}_c(x) = & \mathbf{w}_c \cdot f(x) \\
= & \|\mathbf{w}_c\|\left\lVert f(x)\right\rVert S\left( \mathbf{w}_c,f(x)\right).
\end{aligned}
\end{equation}
Because $\left\lVert f(x)\right\rVert$ is fixed for $x$, $\|\mathbf{w}_c\|$ and $S\left( \mathbf{w}_c,f(x)\right)$ determine the logit score of each class $c$.
The scale variation of $\|\mathbf{w}_c\|$ across classes is not very large.
Therefore, to classify $x$ correctly, $S(\mathbf{w}_c, f(x))$ must be large for the ground truth class $c$.
Here exists the gap between classification and localization.
The classifier is trained to increase $S(\mathbf{w}_c, f(x))$, not $S(\mathbf{w}_c, F_u(x))$ for $u$ belonging to an object region. Cosine similarity is interpreted as the degree of alignment between the directions of the two vectors, meaning that the input feature vector at the object region and class-specific weight vector are not ensured to be aligned with training only for classification.
This causes the model to fail to localize the entire object in a CAM.

Fig.~\ref{fig:cam_norm_sim} shows some examples of norm map $\mathcal{F}$, similarity map $\mathcal{S}$, and CAM from a vanilla model.
The less discriminative but object-belonging regions also have noticeably high activation in $\mathcal{F}$, including wings and bodies of birds. 
However, those regions are not activated in the final CAMs, due to the small values in $\mathcal{S}$.
Although $\mathcal{F}$ contains considerable information for localization, its effect diminishes because of the misalignment of the feature directions with the class-specific weight.

In the next section, we propose a method to bridge the gap between classification and localization by aligning feature directions: adjusting the cosine similarity between input features and class-specific weights. 


\section{Bridging the Gap through Alignment}
We describe how to align feature directions in Sec.~\ref{sec:feature_directions}. An additional strategy to enhance the effect of the feature direction alignment, consistency with attentive dropout, is introduced in Sec.~\ref{sec:consistency_drop}. In Sec.~\ref{sec:training_scheme}, we describe the overall training scheme.
Fig.~\ref{fig:overall} shows the overview of our proposed method.


\subsection{Alignment of Feature Directions}\label{sec:feature_directions}
To enhance the activation of the entire object region in CAM, we want the cosine similarity between $F_u$ and $\mathbf{w}_{c}$ to be high for $u$ belonging to the target object and low for the background region.
Because high activation in $\mathcal{F}$ implies that there is a cue for classification at the corresponding location, we divide the region of the feature map into coarse foreground region $\mathcal{R}^\text{norm}_\text{fg}$ and background region $\mathcal{R}^\text{norm}_\text{bg}$ based on a normalized $\mathcal{F}$.
\begin{equation}
\begin{aligned}
&\mathcal{R}^\text{norm}_\text{fg}=\{u|\hat{\mathcal{F}}_u>\tau_\text{fg}\},\\
&\mathcal{R}^\text{norm}_\text{bg}=\{u|\hat{\mathcal{F}}_u<\tau_\text{bg}\},\\
&\text{where}~\hat{\mathcal{F}}=\frac{\mathcal{F}-\min_{i}{\mathcal{F}_i}}{\max_{i}{\mathcal{F}_i}-\min_{i}{\mathcal{F}_i}}.
\end{aligned}
\end{equation}
$\tau_\text{fg}$ and $\tau_\text{bg}$ are constant thresholds that determine the foreground and background regions, respectively. Note that $\tau_\text{fg}$ and $\tau_\text{bg}$ are not the same; therefore, there is an unknown region that is not included in either $\mathcal{R}^\text{norm}_\text{fg}$ or $\mathcal{R}^\text{norm}_\text{bg}$.
To increase $\mathcal{S}_u$ in $\mathcal{R}^\text{norm}_\text{fg}$ and suppress it in $\mathcal{R}^\text{norm}_\text{bg}$, we define the similarity loss as follows:
\begin{equation}\label{eq:loss_sim}
\begin{aligned}
\mathcal{L}_\text{sim} = -\frac{1}{|\mathcal{R}^\text{norm}_\text{fg}|}\sum_{u\in \mathcal{R}^\text{norm}_\text{fg}}{\mathcal{S}_u} +\frac{1}{|\mathcal{R}^\text{norm}_\text{bg}|}\sum_{u\in \mathcal{R}^\text{norm}_\text{bg}}{\mathcal{S}_u}.
\end{aligned}
\end{equation}

There still remains a possibility that some parts of the object region have low activation in $\hat{\mathcal{F}}$.
In this case, $\mathcal{L}_\text{sim}$ may not be sufficient for the alignment.
Therefore, we introduce an additional loss term to increase $\hat{\mathcal{F}}$ in every candidate region belonging to the target object.
Because a positive $\mathcal{S}_u$ indicates that $u$ is making a positive contribution to increasing the classification logit, the regions with positive similarity can be treated as candidates for the object region. Therefore, we force this area to be activated.
We estimate the object region, $\mathcal{R}^\text{sim}_\text{fg}$, and background region, $\mathcal{R}^\text{sim}_\text{bg}$, based on $\mathcal{S}_u$ as
\begin{equation}
\begin{aligned}
&\mathcal{R}^\text{sim}_\text{fg}=\{u|\mathcal{S}_u>0\},\\
&\mathcal{R}^\text{sim}_\text{bg}=\{u|\mathcal{S}_u<0\}.
\end{aligned}
\end{equation}
With each estimated region, we define the norm loss in a manner similar to Eq.~\ref{eq:loss_sim}, as follows:
\begin{equation}\label{eq:loss_norm}
\begin{aligned}
\mathcal{L}_\text{norm} = -\frac{1}{|\mathcal{\mathcal{R}^\text{sim}_\text{fg}}|}\sum_{u\in \mathcal{R}^\text{sim}_\text{fg}}{\hat{\mathcal{F}}_u} +\frac{1}{|\mathcal{R}^\text{sim}_\text{bg}|}\sum_{u\in \mathcal{R}^\text{sim}_\text{bg}}{\hat{\mathcal{F}}_u}.
\end{aligned}
\end{equation}

For fine-grained classification, such as bird species classification, the object to be recognized is the same across classes. In this case, we define the region with a non-positive similarity with any class as $\mathcal{R}^\text{sim}_\text{bg}$ and the other as $\mathcal{R}^\text{sim}_\text{fg}$. In general, the regions $\mathcal{R}^\text{sim}_\text{bg}$ and $\mathcal{R}^\text{sim}_\text{fg}$ are defined with a similarity with a target class.

The two loss terms $\mathcal{L}_\text{sim}$ and $\mathcal{L}_\text{norm}$ operate complementary.
Through the minimization of $\mathcal{L}_\text{sim}$, the value of $\mathcal{S}$ in the region that is highly activated in $\hat{\mathcal{F}}$ increases.
Through the minimization of $\mathcal{L}_\text{norm}$, the value of $\hat{\mathcal{F}}$ in the region with high similarity increases.
After the joint minimization of $\mathcal{L}_\text{sim}$ and $\mathcal{L}_\text{norm}$, the activated region in $\hat{\mathcal{F}}$  and that in $\mathcal{S}$ become similar.


\subsection{Consistency with Attentive Dropout}\label{sec:consistency_drop}
We can expect the successful alignment by $\mathcal{L}_\text{sim}$ when the estimation of $\mathcal{R}^\text{norm}_\text{fg}$ and $\mathcal{R}^\text{norm}_\text{bg}$ is accurate: $\hat{\mathcal{F}}$ is consistently large over the entire object region and small over the background region.
\begin{figure}[t]
	\centering
    \includegraphics[width=\columnwidth]{figures/fig_dropconsistency.pdf}
    \vspace{-1.8em}
    \caption{Dropout mechanism of consistency with attentive dropout}
    \label{fig:dropconsistency}
\end{figure}
Because the value of $\mathcal{F}$ at the most discriminative region is significantly larger than that at the other region, the value of the normalized map $\hat{\mathcal{F}}$ at the less discriminative part but belonging to the object region becomes small.

We introduce consistency with attentive dropout, a method to distribute the activation to the target object region.
We adopt $L_1$ loss between the two feature maps $F$ and $F_\text{drop}$: $F$ is the feedforward result of an intermediate feature map $F'$, and $F_\text{drop}$ is the feedforward result of $F'_\text{drop}$ obtained by intentionally dropping large activations from $F'$.
Fig.~\ref{fig:dropconsistency} shows the overall process of obtaining $F'_\text{drop}$ for consistency with attentive dropout.
In $F'$, the activation at the spatial location whose channel-wise averaged activation is larger than $\gamma$ is dropped with probability $p$. The stochastic dropout prevents all information in the highly activated area from being eliminated. The loss for consistency with attentive dropout is as follows:
\begin{equation}\label{eq:loss_er}
\mathcal{L}_\text{drop} = \|F(x)- F_\text{drop}(x)\|_{1}.
\end{equation}

There have been several attempts that utilize a similar erasing mechanism~\cite{mai2020erasing,choe2019attention,zhang2018adversarial}.
They train a classifier to preserve the predicted labels before and after erasing highly activated features.
In contrast, our method explicitly regularizes a model to yield a similar feature map even after the highly activated features are dropped.
This decreases the dependency on the dropped features, resulting in more evenly distributed activation compared to the other methods.

\subsection{Training Scheme}\label{sec:training_scheme}
With cross-entropy loss for classification, $\mathcal{L}_\text{CE}$, the total cost function is defined as follows:
\begin{equation}\label{eq:loss_tot_step2}
\mathcal{L_\text{total}} = \mathcal{L}_\text{CE} + \lambda_\text{drop} \mathcal{L}_\text{drop} + \lambda_\text{sim} \mathcal{L}_\text{sim} + \lambda_\text{norm} \mathcal{L}_\text{norm},
\end{equation}
where $\lambda_\text{drop}$, $\lambda_\text{sim}$, and $\lambda_\text{norm}$ are hyperparameters for balancing the losses.
The feature direction alignment is better applied after training the classifier to some extent to obtain a suitable feature map for classification.
Thus, for the first few epochs (\ie, the warm stage), we train a model only with $\mathcal{L}_\text{CE}$ and $\mathcal{L}_\text{drop}$:
\vspace{-1pt}
\begin{equation}\label{eq:loss_tot_step1}
\mathcal{L_\text{warm}} = \mathcal{L}_\text{CE} + \lambda_\text{drop} \mathcal{L}_\text{drop}.
\end{equation}
\section{Experiments}

\subsection{Experimental Settings}
\noindent\textbf{Datasets.}
We evaluate our method on two popular benchmarks: CUB-200-2011~\cite{welinder2010caltech} and ImageNet-1K~\cite{russakovsky2015imagenet}.
In the CUB-200-2011 dataset, there are 5,994 images for training and 5,794 for testing from 200 bird species. In the ImageNet-1K, there are approximately 1.3 million images in the training set and 50,000 in the validation set from 1,000 different classes.

\noindent\textbf{Evaluation Metrics.}
Following the work of Russakovsky~\etal~\cite{russakovsky2015imagenet}, we use Top-1 localization accuracy (Top-1 Loc), Top-5 localization accuracy (Top-5 Loc), and localization accuracy with ground-truth class (GT Loc) as our evaluation metrics.
Top-$k$ Loc is the proportion of the images whose predicted bounding box has more than 50\% intersection over union (IoU) with the ground-truth bounding box and whose predicted top-$k$ classes include the ground-truth class.
GT Loc is the localization accuracy with the ground-truth class, which does not consider the classification result.
We also use \texttt{MaxBoxAccV2}~\cite{choe2020evaluation} to evaluate our method.
\texttt{MaxBoxAccV2}$(\delta)$ measures the localization accuracy with ground-truth class with multiple IoU thresholds $\delta\in\{0.3, 0.5, 0.7\}$.

\begin{figure*}[t]
	\centering
    \includegraphics[width=\textwidth]{figures/fig_comparison_cam.pdf}
    \vspace{-2em}
    \caption{Comparison of localization results from the vanilla method and our method on CUB-200-2011 and ImageNet-1K datasets, using VGG16 as a backbone. Blue boxes denote the ground truth bounding boxes and green boxes denote the predicted bounding boxes.}
    \label{fig:compare_cam}
\end{figure*}
\begin{table}[tbp]
\renewcommand{\arraystretch}{0.95}
  \centering
    \begin{tabular}{lccc}
    \Xhline{1pt}\\[-0.95em]
    Method & Top-1 & Top-5 & GT Loc \\
   \hline\hline
\multicolumn{2}{l}{Additional Branch}\\
SLT-Net~\cite{guo2021strengthen}$_{\text{~~CVPR '21}}$ & 67.8 & - & 87.6 \\
ORNet~\cite{xie2021online}$_{\text{~~ICCV '21}}$  &67.74 &80.77 &86.19 \\
FAM~\cite{meng2021foreground}$_{\text{~~ICCV '21}}$  &69.26 &- &89.26 \\
\midrule
\multicolumn{2}{l}{Single Branch}\\
CAM~\cite{zhou2016learning}$_{\text{~~CVPR '16}}$  &44.15 &52.16 &56.00 \\
ADL~\cite{choe2019attention}$_{\text{~~CVPR '19}}$  &52.36 &- & 75.41 \\
DANet~\cite{xue2019danet}$_{\text{~~ICCV '19}}$  &52.52 &61.96 &67.70 \\
EIL~\cite{mai2020erasing}$_{\text{~~CVPR '20}}$  &56.21 &- &- \\
MEIL~\cite{mai2020erasing}$_{\text{~~CVPR '20}}$  &57.46 &- &- \\
DGL~\cite{tan2020dual}$_{\text{~~ACMMM '20}}$ & 56.07 & 68.50	& 74.63 \\
Ki~\etal~\cite{ki2020sample}$_{\text{~~ACCV '20}}$  &57.50 &- &- \\
Bae~\etal~\cite{bae2020rethinking}$_{\text{~~ECCV '20}}$ & 58.96 &  - & 76.30\\
Pan~\etal~\cite{pan2021unveiling}$_{\text{~~CVPR '21}}$  &60.27 &72.45 &77.29 \\
Ours & \textbf{70.83} & \textbf{88.07} & 	\textbf{93.17}\\
    \Xhline{1pt}
    \end{tabular}%
    \vspace{-0.5em}
     \caption{Comparison of localization performance on the CUB-200-2011 test set, based on VGG16.}
  \label{tab:cub_top1loc_vgg}
\end{table}%

\noindent\textbf{Implementation Details.}
We evaluate our method using VGG16~\cite{simonyan2014very} and ResNet50~\cite{he2016deep} as backbone networks.
For VGG16, we adopt the GAP layer following the training settings of the previous work~\cite{zhou2016learning}.
For ResNet50, we set the stride of the third layer to 1.
The attentive dropout is applied before the last pooling layer in VGG16 and after the first block in the fourth layer in ResNet50.
We initialize the networks with the pretrained weights using ImageNet-1K~\cite{russakovsky2015imagenet}. We use a min-max normalization to draw the bounding box from the generated CAM.

\subsection{Comparison with State-of-the-art Methods}
We compare our method to the recent WSOL methods. For other WSOL methods, we report the localization performance of the original papers or that reproduced by \cite{choe2020evaluation,kim2021normalization,bae2020rethinking,tan2020dual}\footnote{https://github.com/clovaai/wsolevaluation}. Our method consistently outperforms existing WSOL methods using a single branch, across the datasets and the backbones by a large margin.

Tab.~\ref{tab:cub_top1loc_vgg} shows the localization performance on the CUB-200-2011~\cite{welinder2010caltech} test set, using VGG16 as a backbone. Our method achieves an 11.87\%p improvement in Top-1 Loc and a 16.87\%p improvement in GT Loc over the work of Bae~\etal~\cite{bae2020rethinking}, which is the state-of-the-art method among the CAM-based methods.
Furthermore, our method outperforms the methods adopting an additional branch for localization. Our method improves Top-1 Loc by 1.57\%p and GT-Loc by 3.91\%p improvement in GT Loc compared to FAM~\cite{meng2021foreground}.

Tab.~\ref{tab:cub_top1loc} shows the results using ResNet50 as a backbone. It shows that our method consistently outperforms the existing methods by a large margin ($>$13\%p), using a different backbone.
\textcolor{black}{Tab.~\ref{tab:imagenet_top1loc} shows the localization performance on the ImageNet-1K~\cite{russakovsky2015imagenet} validation set, based on VGG16 and ResNet50.
Our method achieves the state-of-the-art performance in the ImageNet-1K dataset regardless of the backbone, and only Top-1 Loc with ResNet50 is the second best after I$^2$C with a marginal difference.}

Additionally, we compare our \texttt{MaxBoxAccV2}~\cite{choe2020evaluation} scores with other state-of-the-art methods on the CUB-200-2011 and ImageNet-1K in Tab.~\ref{tab:total_maxbox}.
It shows that our method outperforms the most recent methods by a large margin for all IoU thresholds with various backbones and datasets. Especially, our method improves the score with IoU threshold of 0.7, which is strict accuracy, by 21.0\%p and 17.4\%p with VGG16 and ResNet50 on the CUB-200-2011 dataset, respectively, compared with the work of Ki~\etal~\cite{ki2020sample}.

\begin{table}[tbp]
\renewcommand{\arraystretch}{0.95}
  \centering
    \begin{tabular}{lccc}
    \Xhline{1pt}\\[-0.95em]
    Method & Top-1 & Top-5 & GT Loc \\
   \hline\hline
CAM~\cite{zhou2016learning}$_{\text{~~CVPR '16}}$  & 46.91 &53.57 & - \\
ADL~\cite{choe2019attention}$_{\text{~~CVPR '19}}$  & 57.40 &- & 71.99 \\
CutMix~\cite{yun2019cutmix}$_{\text{~~ICCV '19}}$  &54.81 &- &- \\
DGL~\cite{tan2020dual}$_{\text{~~ACMMM '20}}$ & 60.82 &70.50 & 74.65\\
Ki~\etal~\cite{ki2020sample}$_{\text{~~ACCV '20}}$  &56.10 &- &- \\
Bae~\etal~\cite{bae2020rethinking}$_{\text{~~ECCV '20}}$ &  59.53 &  - & 77.58\\
Ours & \textbf{73.16} &  \textbf{86.68} & \textbf{91.60}\\
    \Xhline{1pt}
    \end{tabular}%
    \vspace{-0.5em}
     \caption{Comparison of localization performance on the CUB-200-2011 test set, based on ResNet50.}
  \label{tab:cub_top1loc}
\end{table}%

Fig.~\ref{fig:compare_cam} shows some examples of localization results from the vanilla method~\cite{zhou2016learning} and from our method on the CUB-200-2011 and ImageNet-1K datasets. It shows that the model trained with our method captures the target object region more accurately than the vanilla model. On the CUB-200-2011 dataset, while the vanilla model fails to identify the tails, legs, and wings of birds, the classifier trained with our method successfully identifies them.

\setuldepth{53.87}
\begin{table}[tbp]
\renewcommand{\arraystretch}{0.95}
  \centering
    \begin{tabular}{lccc}
    \Xhline{1pt}\\[-0.95em]
    Method & Top-1 & Top-5 & GT Loc \\
    \hline\hline
     \multicolumn{2}{l}{Backbone: VGG16}\\
    CAM~\cite{zhou2016learning}$_{\text{~~CVPR '16}}$  & 42.80 &54.86 & - \\
    ACoL~\cite{zhang2018adversarial}$_{\text{~~CVPR '18}}$  & 45.83 &59.43 &62.96 \\
    ADL~\cite{choe2019attention}$_{\text{~~CVPR '19}}$  & 44.92 &- &- \\
    CutMix~\cite{yun2019cutmix}$_{\text{~~ICCV '19}}$  & 43.45 &- &- \\
    I$^{2}$C~\cite{zhang2020inter}$_{\text{~~ECCV '20}}$  & 47.41 &58.51 &63.90 \\
    EIL~\cite{mai2020erasing}$_{\text{~~CVPR '20}}$  &46.27 &- &- \\
    MEIL~\cite{mai2020erasing}$_{\text{~~CVPR '20}}$  &46.81 &- &- \\
    Ki~\etal~\cite{ki2020sample}$_{\text{~~ACCV '20}}$  & 47.20 &- &- \\
    DGL~\cite{tan2020dual}$_{\text{~~ACMMM '20}}$ &   47.66 &58.89 &64.78 \\
    Bae~\etal~\cite{bae2020rethinking}$_{\text{~~ECCV '20}}$ & 44.62 &  - & 60.73\\
    Pan~\etal~\cite{pan2021unveiling}$_{\text{~~CVPR '21}}$  & \ul{49.56} &\ul{61.32} &\ul{65.05} \\
    Ours & \textbf{49.94} & \textbf{63.25} & 	\textbf{68.92}\\
    \midrule
    \multicolumn{2}{l}{Backbone: ResNet50}\\
    ADL~\cite{choe2019attention}$_{\text{~~CVPR '19}}$  & 48.23 & - &61.04 \\
    CutMix~\cite{yun2019cutmix}$_{\text{~~ICCV '19}}$  &47.25 &- &- \\
    Ki~\etal~\cite{ki2020sample}$_{\text{~~ACCV '20}}$  & 48.40 &- &- \\
    Bae~\etal~\cite{bae2020rethinking}$_{\text{~~ECCV '20}}$ &  49.42 & - & 62.20 \\
     I$^{2}$C~\cite{zhang2020inter}$_{\text{~~ECCV '20}}$  & \textbf{54.83} &\ul{64.60} &68.50 \\
    DGL~\cite{tan2020dual}$_{\text{~~ACMMM '20}}$ &  53.41 &62.69 &\ul{69.34} \\
    Ours & \ul{53.76}	& \textbf{65.75}	& \textbf{69.89}\\
    \Xhline{1pt}
    \end{tabular}%
    \vspace{-0.5em}
     \caption{Comparison of localization performance on the ImageNet-1K validation set. The best performance is bold and the second best performance is underlined.}
  \label{tab:imagenet_top1loc}
\end{table}%

\begin{table*}[t]
  \centering
\setlength{\tabcolsep}{3.73pt}
\begin{tabular}{l|cccc|cccc|cccc|cccc}
\Xhline{1pt}
\multirow{4}{*}{Method} &\multicolumn{8}{c|}{CUB-200-2011} &\multicolumn{8}{c}{ImageNet-1K} \\
&\multicolumn{4}{c|}{VGG16} &\multicolumn{4}{c|}{ResNet50} &\multicolumn{4}{c|}{VGG16} &\multicolumn{4}{c}{ResNet50} \\
&\multicolumn{3}{c}{$\delta$} &\multirow{2}{*}{Mean} &\multicolumn{3}{c}{$\delta$} &\multirow{2}{*}{Mean} &\multicolumn{3}{c}{$\delta$} &\multirow{2}{*}{Mean} &\multicolumn{3}{c}{$\delta$} &\multirow{2}{*}{Mean} \\
&0.3 &0.5 &0.7 & &0.3 &0.5 &0.7 & &0.3 &0.5 &0.7 & &0.3 &0.5 &0.7 & \\
\hline\hline
CAM~\cite{zhou2016learning} &96.8 &73.1 &21.2 &63.7 &95.7 &73.3 &19.9 &63.0 &81.0 &62.0 &37.1 &60.0 &83.7 &65.7 &41.6 &63.7 \\
HaS~\cite{singh2017hide} &92.1 &69.9 &29.1 &63.7 &93.1 &72.2 &28.6 &64.6 &80.7 &62.1 &38.9 &60.6 &83.7 &65.2 &41.3 &63.4 \\
SPG~\cite{zhang2018self} &90.5 &61.0 &17.4 &56.3 &92.2 &68.2 &20.8 &60.4 &81.4 &62.0 &36.3 &59.9 &83.9 &65.4 &40.6 &63.3 \\
ADL~\cite{choe2019attention} &97.7 &78.1 &23.0 &66.3 &91.8 &64.8 &18.4 &58.3 &80.8 &60.9 &37.8 &59.9 &83.6 &65.6 &41.8 &63.7 \\
CutMix~\cite{yun2019cutmix} &91.1 &67.3 &28.6 &62.3 &94.3 &71.5 &22.5 &62.8 &80.3 &61.0 &37.1 &59.5 &83.7 &65.2 &41.0 &63.3 \\
Ki~\etal~\cite{ki2020sample} &96.2 &77.2 &26.8 &66.7 &96.2 &72.8 &20.6 &63.2 &81.5 &63.2 &39.4 &61.3 &84.3 &67.6 &43.6 &65.2 \\
HaS + PaS~\cite{bae2020rethinking} &- &- &- &61.2 &- &- &- &61.9 &- &- &- &62.1 &- &- &- &64.6 \\
CALM~\cite{kim2021keep} &- &- &- &64.8 &- &- &- &71.0 &- &- &- &62.8 &- &- &- & 63.4 \\
ADL + IVR~\cite{kim2021normalization} &- &- &- &71.5 &- &- &- &67.1 &- &- &- &63.7 &- &- &- &65.1 \\
Ours & \textbf{99.3} & 	\textbf{93.2} & \textbf {47.8} & \textbf{80.1} & \textbf{99.4} & \textbf{90.4} & \textbf{38.0} & \textbf{75.9} & \textbf{84.8} & \textbf{69.2} & \textbf{45.9} & \textbf{66.6} & \textbf{86.7} & \textbf{71.1}	& \textbf{48.3}	& \textbf{68.7}\\
    \Xhline{1pt}
    \end{tabular}%
    \vspace{-0.5em}
      \caption{Comparison of \texttt{MaxBoxAccV2} scores on the CUB-200-2011 and ImageNet-1K datasets using various backbones.}
  \label{tab:total_maxbox}%
\end{table*}%
\begin{figure}[t]
	\centering
    \includegraphics[width=0.88\columnwidth]{figures/fig_comparison_sim.pdf}
    \vspace{-0.6em}
    \caption{Comparisons of CAM, $\mathcal{F}$, and $\mathcal{S}$ between the vanilla method and our method on the CUB-200-2011 and ImageNet-1K datasets, using VGG16 as a backbone.}
    \label{fig:compare_sim}
\end{figure}

\subsection{Discussion}
\noindent\textbf{Feature Direction Alignment.}
Through the feature direction alignment, we force $\mathcal{S}$ and $\hat{\mathcal{F}}$ to be high in the object region and to be low in the background region. As Fig.~\ref{fig:compare_sim} shows, the classifier trained with our method yields $\mathcal{S}$ that has a high value in the object region and low value in the background region, different from the vanilla model. It also generates $\hat{\mathcal{F}}$ that has higher activation in less discriminative parts than the vanilla model does.
This makes CAM successfully identify the entire object region. As mentioned in Sec.~\ref{sec:feature_directions}, the feature direction alignment makes $\hat{\mathcal{F}}$ and $\mathcal{S}$ similar, resulting that CAM becomes also similar with them.
We generate a localization map with $\mathcal{F}$ and $\mathcal{S}$ and evaluate the localization performance for each case. We use a min-max normalization when drawing bounding boxes from $\mathcal{F}$. Since negative values in $\mathcal{S}$ denote the background region, we apply a max-normalization on $\mathcal{S}$. Tab.~\ref{tab:perf_sim_norm_cam} shows that the localization results with $\mathcal{F}$ and $\mathcal{S}$ also achieve similar localization performance with CAM. This proves the coincidence between CAM, $\mathcal{F}$, and $\mathcal{S}$ with our method.

Fig.~\ref{fig:hist}(a) shows the distribution of $\mathcal{S}_u$ inside the ground truth bounding boxes from the vanilla method and our method. Note that the bounding boxes include not only the target object but also the background region.
As the training progresses with our method, the similarity gradually splits into negative and large positive values.
This shows that our method effectively increases the similarity for the foreground region and decreases it for the background region.
In contrast, for the vanilla method, the similarity is clustered in small positive values, making no distinction between \mbox{the two}.

\begin{table}[t]
\normalsize
  \centering
  
    \begin{tabular}{cccc}
    \Xhline{1pt}
    Localization map & Top-1  & Top-5 & GT Loc \\
    \hline\hline
     \texttt{CAM}  & 70.83 & 88.07 & 93.17 \\
     $\mathcal{F}$  & 69.90 & 86.68 & 91.96 \\
     $\mathcal{S}$  & 70.38 & 87.64 & 93.13 \\
    \Xhline{1pt}
    \end{tabular}%
    \vspace{-0.5em}
    \caption{Localization performance with various localization maps on the CUB-200-2011 test set, based on VGG16.}
  \label{tab:perf_sim_norm_cam}%
\end{table}%


\begin{figure}[t]
	\centering
    \includegraphics[width=0.93\columnwidth]{figures/fig_hist1.pdf}
    \vspace{-0.7em}
    \caption{(a) Comparison of density histogram on $\mathcal{S}_u$ with the vanilla method and our method. (b) Comparison of density histogram on $\hat{\mathcal{F}}_u$ with the vanilla method, EIL, and consistency with attentive dropout. The analyzes are performed on the CUB-200-2011 test set using VGG16 as a backbone.}
    \label{fig:hist}
\end{figure}

\noindent\textbf{Consistency with Attentive Dropout.}
Fig.~\ref{fig:hist}(b) compares the effect of our consistency with attentive dropout on the distributions of $\hat{\mathcal{F}}_u$ with the vanilla method and EIL~\cite{mai2020erasing}, the state-of-the-art erasing WSOL method.
Here, the feature direction alignment with $\mathcal{L}_\text{sim}$ and $\mathcal{L}_\text{norm}$ is not applied.
With the vanilla training, most of $\hat{\mathcal{F}}_u$ are very low.
With EIL, overall $\hat{\mathcal{F}}_u$ increase compared with the vanilla method, implying that less discriminative parts become to be highly activated.
With consistency with attentive dropout, the distribution of $\hat{\mathcal{F}}_u$ shifts even more to the right.
This indirectly shows that our proposed method, consistency with attentive dropout, distributes the activation more over the target object region than the other methods. This results that the consistency with attentive dropout achieves higher performance than EIL when used along with feature direction alignment, as shown in Tab.~\ref{tab:compare_eil}. We provide a more detailed analysis in appendix.

\begin{table}[t]
  \centering
    \begin{tabular}{lccc}
    \Xhline{1pt}
    Method & Top-1  & Top-5 & GT Loc \\
    \hline
    \hline
    Align. & 62.27 & 77.48 & 81.93 \\
    EIL~[\textcolor{green}{15}] + Align. & 66.10 & 82.21 & 86.78 \\
    Attentive Dropout + Align. & \textbf{70.83} & \textbf{88.07} & \textbf{93.17} \\
    \Xhline{1pt}
    \end{tabular}%
    \vspace{-0.7em}
     \caption{Comparison of localization performance on the CUB-200-2011 dataset, based on VGG16. Align. denotes the feature direction alignment.}
  \label{tab:compare_eil}%
\end{table}%

\subsection{Ablation Study}\label{sec:ablation}
We perform a series of ablation studies on the CUB-200-2011 dataset using VGG16 as the backbone.

\noindent\textbf{Effect of Each Component.}
Tab.~\ref{tab:ablation} shows the localization performance of the classifier trained with and without each loss term.
Compared to the performance without the proposed loss terms, $\mathcal{L}_\text{drop}$ improves the Top-1 Loc by 7.4\%p and GT Loc by 14.32\%p.
The feature direction alignment using only $\mathcal{L}_\text{sim}$ improves the Top-1 Loc by 9.71\%p and GT Loc by 15.36\%p, which shows the largest improvement among the components.
Adopting $\mathcal{L}_\text{norm}$ improves all metrics more than 5\%p. The feature direction alignment using both $\mathcal{L}_\text{sim}$ and $\mathcal{L}_\text{norm}$ achieves 62.27\% of Top-1 Loc and 81.93\% of GT Loc, which is higher than the performance reported by Pan~\etal~\cite{pan2021unveiling}.
Adoption of all components shows the best performance in all metrics.

\noindent\textbf{Sensitivity to Hyperparameters.}
We analyze the effect of the balancing factors in the loss and the hyperparameters of each loss.

For the balancing factors in loss, we find the best localization performance at 0.5 for $\lambda_\text{sim}$, 0.15 for $\lambda_\text{norm}$, and 3 for $\lambda_\text{drop}$, respectively.
As shown in Fig.~\ref{fig:hyperparams}(a), the localization performance is most sensitively affected by $\lambda_\text{sim}$. $\lambda_\text{norm}$ insignificantly changes the performance.
The performance tends to decrease when the constraint with $\lambda_\text{drop}$ becomes too strong as 4.
\begin{table}[t]
  \centering
    \begin{tabular}{ccc|ccc}
    \Xhline{1pt}
    $\mathcal{L}_\text{drop}$ & $\mathcal{L}_\text{sim}$ & $\mathcal{L}_\text{norm}$ & Top-1  & Top-5 & GT Loc \\
    \hline\hline
    \textcolor{red}{\xmark} &  \textcolor{red}{\xmark} & \textcolor{red}{\xmark} &46.95 &57.23 &60.74 \\
    \textcolor{green}{\cmark}  &  \textcolor{red}{\xmark} &  \textcolor{red}{\xmark}    & 54.35 & 70.37 & 75.06 \\
    \textcolor{red}{\xmark}  &  \textcolor{green}{\cmark} &  \textcolor{red}{\xmark}    & 56.66 & 71.38 & 76.10 \\
    \textcolor{red}{\xmark} &  \textcolor{green}{\cmark}  &  \textcolor{green}{\cmark}& 62.27 &77.48 & 81.93 \\
    \textcolor{green}{\cmark}  &  \textcolor{green}{\cmark} &  \textcolor{red}{\xmark}  & 63.00 & 79.93 & 85.35\\
     \textcolor{green}{\cmark}  &  \textcolor{green}{\cmark} &  \textcolor{green}{\cmark}  & \textbf{70.83} & \textbf{88.07} & \textbf{93.17} \\
    \Xhline{1pt}
    \end{tabular}%
    \vspace{-0.7em}
    \caption{Ablations studies on the CUB-200-2011 test set, based on VGG16.}
  \label{tab:ablation}%
\end{table}%


\begin{figure}[t]
	\centering
    \includegraphics[width=0.99\columnwidth]{figures/fig_params.pdf}
    \vspace{-0.7em}
    \caption{Effect of (a) balancing factors for loss and (b) various hyperparameters.}
    \label{fig:hyperparams}
\end{figure}
For the hyperparameters of the feature direction alignment, we set $\tau_\text{fg}$ and $\tau_\text{bg}$ for $\mathcal{L}_\text{sim}$ to 0.6 and 0.1, respectively.
They determine the coarse foreground and background regions.
Fig.~\ref{fig:hyperparams}(b) shows that varying those thresholds has little effect on the performance.
The hyperparameters $\gamma$ and $p$ determine the drop of the activation in the intermediate feature map. $\gamma$ and $p$ for $\mathcal{L}_\text{drop}$ are set to 0.8 and 0.5, respectively.
When $\gamma$ is moderately large between 0.7 and 0.9, there is no significant change in the performance, but when $\gamma$ is too low, \ie, 0.6, the performance decreases.
From the results with various $p$, we observe that stochastic dropout produces little change of GT Loc regardless of the drop probability, but deterministic dropout with a probability of 1.0 yields a significant drop in the localization performance. This indicates that less but sufficient discriminative information should be maintained for a good localization performance.
\section{Conclusion}\label{conclusion}
In this work, we collect \DsetName, a new large-scale, multilingual moment retrieval dataset. 
It contains 218K queries in English and in Chinese from 21.8K video clips from 6 TV shows. 
We also propose a multilingual moment retrieval model \ModelName~as a strong baseline for the \DsetName~dataset. 
We show in experiments that \ModelName~outperforms monolingual models while using fewer parameters.


\section*{Acknowledgements}
We thank the reviewers for their helpful feedback. This research is supported by NSF Award \#1562098, DARPA KAIROS Grant \#FA8750-19-2-1004, and ARO-YIP Award \#W911NF-18-1-0336. The views contained in this article are those of the authors and not of the funding agency.


\bigskip
\vspace{-1pt}
\noindent\textbf{Acknowledgements:}
This work was supported by Institute of Information \& communications Technology Planning \& Evaluation (IITP) grant funded by the Korea government (MSIT) [NO.2021-0-01343, Artificial Intelligence Graduate School Program (Seoul National University)], LG AI Research, AIRS Company in Hyundai Motor and Kia through HMC/KIA-SNU AI Consortium Fund, and the BK21 FOUR program of the Education and Research Program for Future ICT Pioneers, Seoul National University in 2022.

%%%%%%%%% REFERENCES
{\small
\bibliographystyle{ieee_fullname}
\bibliography{egbib}
}


%%%%%%%%% APPENDIX
\clearpage
\noindent{\Large \textbf{Appendix}}

\setcounter{section}{0}
\renewcommand\thesection{\Alph{section}}
\setcounter{table}{0}
\renewcommand{\thetable}{A\arabic{table}}
\setcounter{figure}{0}
\renewcommand{\thefigure}{A\arabic{figure}}



\section{Details of Motivation Study}
As introduced in Section~\ref{section:intro}, we try to answer two questions: $(i)$ whether presenting a joint image-text data from non-parallel sources would improve the learned joint embedding space than alternatively presenting uni-modal data during pre-training. $(ii)$ If we fed joint image-text data to the model, how does its existing latent alignment affect the cross-modal representation learning. 

We conduct the unsupervised vision and language pre-training on Conceptual Captions (CC) by shuffling the image-text pairs. 
For pre-training objectives, we apply standard MLM + MRM. 
All other pre-training setup is the same as introduced in Section~\ref{sec:training_setup}. 
We first compare the round-robin and joint MLM + MRM pre-training, whose results are shown in Table~\ref{tab:data-fedding}.
We then evaluate how the alignment degree of the pre-training dataset affects the model performance, where the degree is controlled by the ratio of originally aligned image-text data in Conceptual Captions.
Table~\ref{tab:paired-ratio} shows the detailed results of each downstream task.
Their Meta-Ave scores are also plotted in Fig.~\ref{fig:intro}.
From these results, we obtained two important messages: 
$(i)$ joint image-and-text input is more optimal for UVLP than alternatively presenting uni-modal data from unparallel image and text corpus. 
$(ii)$ The more the latent semantic alignment exists in the image-text data the better the pre-trained model performs. 

We further explore the realistic unsupervised V+L pre-training, where the images and texts are from two different sources.
Specifically, we sample the images from Conceptual Captions and the texts from Book Corpus respectively.
Table~\ref{tab:bc_alignment} shows that the pre-trained model on our weakly aligned CC image and BC sentence corpus far outperforms that on random pairs, indicating it also holds that better latent image-text alignment leads to better pre-trained model's performance under realistic setting.
\begin{table}[!h]
\centering
\small
\tablestyle{5pt}{0.80}
\begin{tabular}{l|ccccc}\toprule
\multirow{2}{*}{} &VQA2 &NLVR2 &VE & RefCOCO+ & \multirow{2}{*}{ Meta-Ave } \\
&Test-Dev &Test-P &Test &Devs & \\\cmidrule{1-6}
random &70.3 &51.2 &75.3 &76.5 & 68.3 \\
proposed & \bf 71.2 & \bf 67.1 & \bf 77.1 & \bf 79.7 & \bf 73.8 \\
\bottomrule
\end{tabular}
\vspace{-0.3cm}
\caption{Pre-training on realistic CC + BC data}
\label{tab:bc_alignment}
\end{table}

\section{Effectiveness of Weighted ITM}
We compared the performance of pre-training our model with or without weighted ITM. 
The models are pre-trained on CC images and texts. 
As shown in Table~\ref{tab:WITM}, weighted ITM are consistently better than treating all the retrieved pairs with the same weight. 


\begin{table}[!htp]\centering
\footnotesize
\tablestyle{3pt}{0.80}
\begin{tabular}{l|ccccc}\toprule
\multirow{2}{*}{} &VQA2 &NLVR2 &VE & RefCOCO+ & \multirow{2}{*}{ Meta-Ave } \\
&Test-Dev &Test-P &Test &Devs & \\\cmidrule{1-6}
w/o $w_{\text{ITM}}$ &71.9 &72.6 &77.0 &79.7 & 75.3 \\
$w_{\text{ITM}}$ & \bf 72.1 & \bf 73.4 & \bf 77.3 & \bf 80.3 & \bf 75.8 \\
\bottomrule
\end{tabular}
\vspace{-0.3cm}
\caption{Ablation Study on weighted ITM}
\label{tab:WITM}
\end{table}


\begin{table*}[!ht]\centering
\small
\begin{tabular}{l|c|c|c|ccc|c}\toprule
\multirow{2}{*}{ Pre-training } &VQA2 &NLVR2 &VE & \multicolumn{3}{c|}{RefCOCO+} & \multirow{2}{*}{ Meta-Ave } \\
&Test-Dev &Test-P &Test &Dev &TestA &TestB & \\\cmidrule{1-8}
Round-Robin MLM+MRM &70.4 &51.1 &74.8 &73.3 &78.3 &\textbf{67.4} & 67.4 \\
Joint MLM+MRM &\textbf{70.6} &\textbf{52.4} &\textbf{74.9} &\textbf{74.5} &\textbf{79.4} &66.8 & \textbf{68.1} \\
\bottomrule
\end{tabular}
\vspace{-0.3cm}
\caption{Detailed evaluation results on four V+L downstream tasks with two different data feeding strategy for UVLP: (1) joint image-text data (joint MLM+MRM); (2) alternative uni-modal data (round-robin MLM+MRM).}
\label{tab:data-fedding}
\end{table*}

\begin{table*}[!ht]\centering
\small
\begin{tabular}{l|c|c|c|ccc|c}\toprule
\multirow{2}{*}{ Paired Ratio } &VQA2 &NLVR2 &VE & \multicolumn{3}{c|}{RefCOCO+} & \multirow{2}{*}{ Meta-Ave } \\
&Test-Dev &Test-P &Test &Dev &TestA &TestB & \\\cmidrule{1-8}
0\% &70.6 &52.4 &74.9 &74.5 &79.4 &66.8 & 68.1 \\
20\% &71.1 &70.0 &76.4 & 76.3 &80.3 &67.5 & 73.5 \\
40\% &71.4 &71.6 &77.2 &77.9 &82.4 &68.8 & 74.5 \\
60\% &71.9 &74.5 &77.8 &79.9 &84.4 &69.9 & 76.0 \\
80\% &72.2 &75.7 &78.4 &80.9 &85.7 &71.8 & 76.8 \\
100\% &72.5 &75.9 &78.7 &82.1 &86.6 &75.0 & 77.3 \\
\bottomrule
\end{tabular}
\vspace{-0.3cm}
\caption{Detailed evaluation results on four V+L downstream tasks with 6 sets of image and text corpus of different latent cross-modal alignment degree. The alignment degree is controlled by changing the ratio of original aligned image-text data from 0\% to 100\%.}
\label{tab:paired-ratio}
\end{table*}


\begin{figure*}[h!]
\centering
\includegraphics[width=14cm]{figures/Pos_Retrieve.png}
\vspace{-0.3cm}
\caption{Examples of retrieved text from both CC and BC. The covered grounded noun phrases in retrieved sentences are highlighted in green bar for positive examples.}
\label{fig:pos-ret}
\end{figure*}

\section{Downstream Task Details}
We describe the details of fine-tuning on the four different downstream tasks: Visual Question Answering (VQA2), Natural Language for Visual Reasoning (NLVR2), Visual Entailment (VE), and Referring Expression (RefCOCO+). We mainly follow the setup of UNITER\cite{chen2020uniter} for each downstream task with minor adjustments.  

\noindent\textbf{VQA2}
Given a question about an image, the task is to predict the answer to the question. Following \cite{yu2019mcan}, we take 3,129 most frequent answers as answer candidates. We use both training and validation sets from VQA 2.0 for fine-tuning. Following UNITER, we also leverage data from Visual Genome\cite{krishna2017visualgenome} to augment the best performance on the test-dev split. We fine-tune the model with a binary cross-entropy loss with a peak learning rate of $6\times10^{-5}$ for 20 epochs. The training batch size is set as 480. 

\noindent\textbf{NLVR2}
NLVR2 is a task for visual reasoning. The objective is to determine whether a natural language statement is true or not given a pair of input images. 
We follow UNITER's setup treating each data point as two text-image pairs with repeated text. 
The two [CLS] outputs from the model are then concatenated as the joint embedding for the example. We apply a multi-layer perceptron (MLP) classifier on top of this joint embedding for the final classification. Unlike~\cite{li2020unsupervised} that conducts additional ``pre-training" on NLVR2 datasets, we only fine-tune the model with the task-specific objective to maintain the same setting as all other downstream tasks. We train the model for 8 epochs with a batch size of 60 and a peak learning rate of $3\times10^{-5}$. 

\noindent\textbf{VE}
Visual Entailment is a task built on Flickr30k Images\cite{young-etal-2014-image}, where the goal is to determine the logical relationship between a natural language statement and an image. Similar to the Natural Language Inference problem in NLP, this task is formatted as a 3-way classification problem to predict if the language statement entails, contradicts, or is undetermined with respect to the given image. An MLP transformer classifier is applied to the output of the $\text{[CLS]}$ token to make the final prediction. The model is fine-tuned using cross-entropy loss. We set the batch size as 480 and the peak learning rate as $8\times10^{-5}$. The model is fine-tuned for 4 epochs for this downstream task. 

\noindent\textbf{RefCOCO+}
The referring expression task involves locating an image region given a natural language phrase. We use RefCOCO+ \cite{yu2016modeling} as the evaluation dataset. Bounding box proposals from VinVL object detectors are used for fine-tuning. A proposal box is considered correct if it has an IoU with a gold box larger than 0.5. We add an MLP layer on top of the region outputs from the Transformer to compute the alignment score between the language phrase and each proposed region. We fine-tune our model for 20 epochs with a peak-learning rate of $2\times10^{-4}$.


\begin{figure*}[h!]
\centering
\includegraphics[width=14cm]{figures/Neg_Retrieve.png}
\vspace{-0.1cm}
\caption{Examples of retrieved text from both CC and BC. The mistakenly covered grounded noun phrases in retrieved sentences are highlighted in red bar for negative examples.}
\label{fig:neg-ret}
\end{figure*}

\begin{figure}[h!]
\centering
\includegraphics[width=0.7\linewidth]{figures/attention_viz_1.jpg}
\vspace{-0.2cm}
\caption{Text-to-image attention given the aligned pair whose caption is ``person in a leather jacket riding a motorcycle on the road".}
\label{fig:attn_viz_1}
\end{figure}

\begin{figure}[h!]
\centering
\includegraphics[width=0.7\linewidth]{figures/attention_viz_2.jpg}
\vspace{-0.2cm}
\caption{Text-to-image attention given the aligned pair whose caption is ``girl in a blue kayak floating on the picturesque river at sunset".}
\label{fig:attn_viz_2}
\end{figure}

\begin{figure}[h!]
\centering
\includegraphics[width=0.7\linewidth]{figures/attention_viz_3.jpg}
\vspace{-0.2cm}
\caption{Text-to-image attention given the aligned pair whose caption is ``people walking by the christmas tree and stage area".}
\label{fig:attn_viz_3}
\end{figure}

\begin{figure}[h!]
\centering
\includegraphics[width=0.7\linewidth]{figures/attention_viz_4.jpg}
\vspace{-0.2cm}
\caption{Text-to-image attention given the aligned pair whose caption is ``single cowboy guiding a line of horses through the desert".}
\label{fig:attn_viz_4}
\end{figure}
\section{Additional Visualization}
We present additional examples of retrieved text from both CC and BookCorpus. Specifically, we demonstrate more positive examples in Fig \ref{fig:pos-ret} that covers the appropriate grounded noun phrases. We also share some negative examples in Fig \ref{fig:pos-ret}. As analyzed in the limitation section, the current language embedding model weighs all the object tags equally to generate the joint embedding representation. This can lead to mistakenly focused object tags when retrieving the text. In row 1 of Fig \ref{fig:neg-ret}, texts retrieved cover less important noun phrases such as ``finger" and ``hair" instead of the more important noun phrase "baby". Row 2 of Fig \ref{fig:neg-ret} demonstrate mistakenly retrieved texts due to the limitation of the pre-defined object categories in the object detector. In this example,  the students in the image are detected as ``person" or ``man", which leads to the failure of retrieving any valid text.    

We also demonstrate more examples on text-to-image attention between the pre-trained U-VisualBert and {\ModelName } on the Conceptual Captions Validation set in Fig \ref{fig:attn_viz_1}, \ref{fig:attn_viz_2}, \ref{fig:attn_viz_3}, \ref{fig:attn_viz_4}. These examples provide additional evidence on the better local alignment captured by \ModelName. 

\end{document}
