% CVPR 2022 Paper Template
% based on the CVPR template provided by Ming-Ming Cheng (https://github.com/MCG-NKU/CVPR_Template)
% modified and extended by Stefan Roth (stefan.roth@NOSPAMtu-darmstadt.de)

\documentclass[10pt,twocolumn,letterpaper]{article}

%%%%%%%%% PAPER TYPE  - PLEASE UPDATE FOR FINAL VERSION
% \usepackage[review]{cvpr}      % To produce the REVIEW version
% \usepackage{cvpr}              % To produce the CAMERA-READY version
\usepackage[pagenumbers]{cvpr} % To force page numbers, e.g. for an arXiv version

% Include other packages here, before hyperref.
\usepackage{graphicx}
\usepackage{amsmath}
\usepackage{amssymb}
\usepackage{booktabs}
\usepackage{amsfonts}       % blackboard math symbols
\usepackage{nicefrac}       % compact symbols for 1/2, etc.
\usepackage{kotex}
\usepackage{microtype}      % microtypography
\usepackage{multirow}
\usepackage{makecell}
\usepackage{pifont}         % symbols
\usepackage{enumitem}
\usepackage{scalerel}
\usepackage{soul}
\usepackage{setspace}


% It is strongly recommended to use hyperref, especially for the review version.
% hyperref with option pagebackref eases the reviewers' job.
% Please disable hyperref *only* if you encounter grave issues, e.g. with the
% file validation for the camera-ready version.
%
% If you comment hyperref and then uncomment it, you should delete
% ReviewTempalte.aux before re-running LaTeX.
% (Or just hit 'q' on the first LaTeX run, let it finish, and you
%  should be clear).
\usepackage[pagebackref,breaklinks,colorlinks]{hyperref}


% Support for easy cross-referencing
\usepackage[capitalize]{cleveref}
\crefname{section}{Sec.}{Secs.}
\Crefname{section}{Section}{Sections}
\Crefname{table}{Table}{Tables}
\crefname{table}{Tab.}{Tabs.}


\begin{document}

%%%%%%%%% TITLE - PLEASE UPDATE
\title{Bridging the Gap between Classification and Localization\\ for Weakly Supervised Object Localization}

\author{Eunji Kim$^1$ ~~~~~~~ Siwon Kim$^1$ ~~~~~~~  Jungbeom Lee$^1$  ~~~~~~~  Hyunwoo Kim$^2$  ~~~~~~~  Sungroh Yoon$^{1, 3}$\thanks{Correspondence to: Sungroh Yoon (sryoon@snu.ac.kr).}\\
$^1$ Department of Electrical and Computer Engineering, Seoul National University ~
$^2$ LG AI Research\\
$^3$ Interdisciplinary Program in AI, AIIS, ASRI, INMC, and ISRC, Seoul National University\\
{\tt\small \{kce407, tuslkkk, jbeom.lee93\}@snu.ac.kr, hwkim@lgresearch.ai, sryoon@snu.ac.kr}}
\maketitle

\newcommand{\xmark}{\text{\ding{55}}}
\newcommand{\cmark}{\text{\ding{51}}}

%%%%%%%%% ABSTRACT
\begin{abstract}

%ADMM is a powerful optimization technique that splits optimization problems into simpler sub-problems and solves them efficiently. 
In this paper, we propose Stochastic Block-ADMM as an approach to train deep neural networks in batch and online settings. Our method works by splitting neural networks into an arbitrary number of blocks and utilizes auxiliary variables to connect these blocks while optimizing with stochastic gradient descent. This allows training deep networks with non-differentiable constraints where conventional backpropagation is not applicable. An application of this is supervised feature disentangling, where our proposed DeepFacto inserts a  non-negative matrix factorization (NMF) layer into the network. Since backpropagation only needs to be performed within each block, our approach alleviates vanishing gradients and provides potentials for parallelization. We prove the convergence of our proposed method and justify its capabilities through experiments in supervised and weakly-supervised settings.

\end{abstract}

%%%%%%%%% BODY TEXT
\section{Introduction}
\label{sec:intro}
Object localization aims to find the area of a target object in a given image~\cite{ren2015faster,russakovsky2015imagenet,duan2019centernet,lin2017feature,tan2020efficientdet}. However, fully supervised approaches require accurate bounding box annotations, which require a tremendous cost. Weakly supervised object localization (WSOL) has been a great alternative because it requires only image-level labels to train a localization model~\cite{singh2017hide,choe2019attention,choe2020evaluation,pan2021unveiling, xue2019danet}.

The most commonly used approach for WSOL is a class activation map (CAM)~\cite{zhou2016learning}. CAM-based methods employ a global average pooling (GAP) layer~\cite{lin2013network} followed by a fully connected (FC) layer, and generate a CAM with the feature maps prior to the GAP layer.
A highly activated area in a CAM is predicted to be an object location.
However, it is widely observed that CAM identifies only the most discriminative parts of an object rather than the entire object area, resulting in low localization performance~\cite{mai2020erasing,lee2021anti,zhang2018adversarial}.

\begin{figure}[t]
	\centering
    \includegraphics[width=0.95\columnwidth]{figures/fig_first.pdf}
    \vspace{-0.5em}
    \caption{(a) Examples of CAM and decomposed terms from the classifier trained with the vanilla method~\cite{zhou2016learning} and with EIL~\cite{mai2020erasing}. (b) Visualization of the changes of CAM and decomposed terms as training with our method progresses.}
    \label{fig:first}
\end{figure}

We ask the question, ``\textit{Why does CAM generated from an accurate classifier fail to highlight the entire object area?}''
To answer this, we provide a new perspective of decomposing CAM into two terms: (1) activation in a feature map and (2) cosine similarity between the feature vector at each spatial location and the class-specific weight in the FC layer.
Fig.~\ref{fig:first}(a) shows that only the bird's body is highly activated in the CAM of the vanilla model, leaving the wing less activated. However, looking at the activation in the feature map, the wing as well as the body is highly activated.
The low similarity of the wing region offsets the activation in the feature map, making the region invisible in the CAM.
Here, we find that the low cosine similarity, \ie, misalignment of feature directions to the class-specific weights, prevents the less discriminative part belonging to a target object from being highly activated in a CAM.
This is because training for classification only considers the feature averaged over all locations, not the feature at each spatial location.
This brings the gap between classification and localization.

Although various approaches have been proposed to expand the activated region to the entire object area in a CAM~\cite{zhang2018adversarial,choe2019attention,mai2020erasing,yun2019cutmix,xue2019danet,zhang2018self}, none of them discovered or mitigated the misalignment. Fig.~\ref{fig:first}(a) shows that EIL~\cite{mai2020erasing}, one of those approaches, expands the activated region in the feature map. However, it fails to increase the similarity in the object region; hence, the expansion effect is not as large in the CAM as in the activation of the feature map.

To bridge the gap between classification and localization, we propose feature direction alignment, a method to enhance the alignment of feature directions in the entire object region to the directions of class-specific weights while discouraging the alignment in the background region.
We also introduce consistency with attentive dropout, which ensures that the target object region has uniformly high activation in the feature map.
Fig.~\ref{fig:first}(b) shows that our method gradually aligns the feature directions to the class-specific weight as the training progresses.
The alignment results in high activation of less discriminative regions, \eg, wing, in the CAM, enabling accurate localization of the entire object.
We evaluate our method on the most widely used WSOL benchmark datasets: CUB-200-2011~\cite{welinder2010caltech} and ImageNet-1K~\cite{russakovsky2015imagenet}.
Our method achieves a state-of-the-art localization performance for both datasets.

The contributions of this paper can be summarized as follows:
\begin{itemize}
\setlength{\itemsep}{2pt}
\vspace{-3pt}
	\item[$\bullet$] We interpret a CAM in terms of the degree of alignment between the direction of input features and the direction of class-specific vectors, and find the gap between classification and localization.
	\vspace{-2pt}
	\item[$\bullet$] We propose a method to bridge the gap between classification and localization by aligning feature directions with class-specific weights.
	\vspace{-2pt}
	\item[$\bullet$] We demonstrate that our proposed method outperforms other state-of-the-art WSOL methods on the CUB-200-2011 and ImageNet-1K datasets.
\end{itemize}
\section{Background and Related Work}

\subsection{Traditional Defect Prediction}
In traditional defect prediction, we measure the complexity of
software project using McCabe metrics, Halstead's effort metrics and  CK object-oriented code mertics~\cite{kafura1987use,chidamber1994metrics,mccabe1976complexity,halstead1977elements} at
a coarse granularity, like file or package level. It usually happens after the project has completed. With the collected data instances as well as the corresponding labels(defective or non-defective), we can  build defect prediction models using supervised machine learning algorithms such as Decision Tree, Random Forests, SVM, Naive Bayes and Logistic Regression\cite{khoshgoftaar2001modeling,khoshgoftaar2003software,khoshgoftaar2000balancing,menzies2007data,lessmann2008benchmarking,hall2012systematic}. After that, such trained defect predictor can be applied
to predict the defects on future projects. This type of defect prediction, where training data and testing data are from the same project, is referred as within-project defect prediction(WPDP).

However, for a new software project or a project with limited historical data, researchers
proposed to conduct cross-project defect prediction(CPDP), where the training data is from a different software project with the same metric measured as the testing project. Turhan et al. proposed the nearest neighbor method for CPDP~\cite{turhan2009relative}, where the most similar
training data to the testing data is select for building the model. However the performance of this method is still worse than WPDP. Transfer learning  techniques were proposed  in ~\cite{nam2013transfer, ma2012transfer}, both of these techniques achieve comparable performance with WPDP. Zhang et al.~\cite{zhang2014towards} proposed to build a universal model instead of building models
for each individual target project. They cluster projects based on the similarity of the distribution of 26 predictors, and derive the rank transformations using quantiles of predictors for a cluster. In this way, this universal model was fitted on data sets from 1398 open source projects. Their results show that the universal model obtains prediction performance comparable to WPDP models~\cite{zhang2014towards}. Recently, Nam et al.~\cite{nam2015clami} have proposed
to use unsupervised learning methods, CLA and CLAMI, for defect prediction  on unlabeled data sets. The key idea of
the CLA/CLAMI approaches is to label an unlabeled data set by using the magnitude of metric values~\cite{nam2015clami}. The intuition of this method is based on the empirical observation that higher complexity causes more defect-proneness~\cite{menzies2007data,rahman2013and}. They report that CLA/CLAMI acheive 0.636 and 0.723 F1-measure and AUC, respectively. 

To make defect prediction more feasible in practice, Nam et al.~\cite{nam2015heterogeneous}consider a situation where the data extracted from the
target project has different sets of metrics from any available training data. In this case,
current techniques for CPDP are difficult to apply across projects with heterogeneous metric sets. They proposed a  heterogeneous defect prediction method, which conducts metric selection and metric matching to build a prediction model between projects with heterogeneous metric sets. Experimental results  from 28 subjects show that about $68\%$ of predictions outperform or are comparable to WPDP with statistical significance~\cite{nam2015heterogeneous}.


\subsection{Just-In-Time Defect Prediction}
In traditional defect prediction has some drawbacks such as prediction at a coarse granularity and started at very late stage of software development circle~\cite{kamei2013large},wheras in JIT defect prediction paradigm, the models use change metrics generated from each change as the predictors, which
could easily help developers to narrow down the code for inspection and JIT defect prediction could be conducted right before developers commit the change. JIT defect prediction becomes more practical method for practitioners to carry out.

Mockus et al.~\cite{mockus2000predicting} conduct the first study to predict 
software failures on 5ESS software updates(a telecommunication system)
by using logistic regression on 
data sets consisted of change metrics of the project. Kim et al.
further evaluate the effectiveness of change metrics  on open source projects in ~\cite{kim2008classifying}, where they propose to apply support vector machine to build a defect predictor based on software change metrics, where they achieved $78\%$ accuracy and $60\%$ recall on average.Since the training data might
not available when building the defect predictor, Fukushima et al.~\cite{fukushima2014empirical} introduced
cross-project models into JIT defect prediction. Their results showed that using data from other projects to build JIT defect predictor is feasible. 

However, all the previous JIT defect prediction studies do not consider the efforts required to inspect the predicted defect-introducing changes. Kamei et al.~\cite{kamei2013large} take the effort into account and conduct a large-scale study on the effectiveness of JIT defect prediction, where they claim that using $
20\%$ of efforts required to inspect all changes, their modified linear regression model could
detect $35\%$ defect-introducing changes. Inspired by Menzies et al's ManualUp model(i.e., small size of modules inspected first)~\cite{menzies2010defect}, Yang et al. propose to build $12$ unsupervised learners by sorting the reciprocal values of $12$ different change metrics on each testing data set in descendant order. They
report that the experimental results from six open source projects
show that with $20\%$ efforts, many
unsupervised learners perform better than the state-of-art supervised learner.

Yang et al.'s method is so simple and might benefit a lot for defect prediction if working properly. However, there some issues
when applying to actual project. Firstly, since these unsupervised learners are built directly on testing data set, this indicates that there is no way to tell which learner(s) will perform better than supervised learners or all of them work equally same. 
If all of these learners have similar performance, then we have a strong confidence to select only one, instead of all of them, to apply on future projects. Otherwise, there should be a way to figure out which learner to select. Secondly, in yang's study, they only evaluate learners for two evaluation measures and there is no way to show how these unsupervised
learners perform in terms of different evaluation measures. Without any training data, how could we believe that those learners would meet our evaluation goals. Unfortunately, Yang et al. did not consider these issues in their work~\cite{yang2016effort}. 







% \vspace{-0.1in}

\section{Method} \label{sec:method}

%\subsection{ADMM Training of DNNs}\label{sec:admm_nn}

%Alternating Direction Method of Multipliers (ADMM) \cite{gabay1975dual,boyd2011distributed} is a class of optimization methods belonging to  \textit{operator splitting techniques} which borrows benefits from both dual decomposition and augmented Lagrangian methods for constrained optimization. To show the potentials of standard ADMM, we first revisit a general formulation of ADMM in DNN training, similar to those used in prior work. Then, we propose our stochastic block-ADMM in the next subsection.

%To formulate training an $L$-layer DNN in a general supervised setting, we would have the following non-convex constrained optimization problem \cite{zeng2018global}:
% \vspace{-0.1in}
%\begin{align} \label{eq:obj}
	%\minimize_{ \mathcal{W}, \mathcal{A}, \mathcal{Z}} \quad &\mathcal{J}\left(\mY, \mZ_{L} \right) + \sum_{\ell = 1}^{L} \lambda_{\ell}  {\bf r}_{\ell} (\mW_{\ell}) \\
	% {\rm subject~to} \quad & \mA_{\ell} - {\bm \phi}_{\ell } \left( \mZ_{\ell} \right) = {\bf 0}, \quad \ell = 1,\dots, L-1   \nonumber \\
	 %{\rm subject~to} \quad & \mZ_{\ell} - \mW_{\ell} \mA_{\ell-1} = {\bf 0}, \quad \ell = 1, \dots , L \nonumber 
%\end{align}
%where $\mathcal{J}$ is the main objective (\textit{e.g.}, cross-entropy, mean-squared-error loss functions) that needs to be minimized. The subscript $\ell$ denotes the $\ell$-th layer in the network. The optimization variables are $\mathcal{W} = \{ \mW_\ell\}_{\ell=1}^{L}$, $\mathcal{A} = \{ \mA_{\ell}\}_{\ell=1}^{L-1}$, and $\mathcal{Z} = \{ \mZ_{\ell}\}_{\ell=1}^{L}$ where $\mW_\ell$, $\mZ_{\ell}$, $\mA_\ell$, and ${\bm \phi}_\ell (.)$ are the weight matrix, output matrix, activation matrix, and the activation function (\textit{e.g.}, ReLU) at the $\ell$-th layer, respectively. Note that $\mA_{0} = \mX$ where $\mX = \{ \vx_1,\dots, \vx_N \} \in  \R^{M \times N}$ is the input data matrix containing $N$ samples with input dimensionality $M$; $\mY = \{\vy_1,\dots, \vy_N \} \in \R^{C \times N}$ is the target matrix pair comprised of $N$ one-hot vector label of dimension $C$, representing number of prediction classes. Also, ${\bf r(.)}$ is the regularization term with (\textit{e.g.}, Frobenius norm $\|.\|_F^2$) corresponding penalty weight $\lambda_{\ell}$. %Note that the regularization term can be simply ignored by setting $\lambda_\ell$ to zero. 
%In this formulation, the intercept in each layer is ignored for simplicity as it can be simply be added by slightly modifying the $\mW_\ell$ and the input to each layer. 
%The formulation in Eq. (\ref{eq:obj}) breaks the the conventional multi-layer backpropagation optimization of DNNs into simpler sub-problems that can be solved efficiently (e.g. reducing to least-squares problem). %This also facilitates training in a distributed manner --- as the layers of the DNN are decoupled and the variables can be updated in parallel across layers ($\mW_\ell$) and data points (\ $\mW_\ell, \mZ_\ell, \mA_\ell$).



%To enforce the constraints in problem (\ref{eq:obj}) and solve the optimization using ADMM, we would have the following augmented Lagrangian problem:

%\begin{eqnarray} \label{eq:augmented}
%	\minimize_{ \mathcal{W}, \mathcal{A}, \mathcal{Z}} \quad &\mathcal{J}\left(\mY, \mZ_{L} \right) + \sum_{\ell = 1}^{L} \lambda_{\ell}  {\bf r}_{\ell} (\mW_{\ell}) \\
%	& + \sum_{\ell=1}^{L} \frac{\beta_{\ell}}{2} \| \mZ_{\ell} - \mW_{\ell} \mA_{\ell-1} + \mU_{\ell}\|_{F}^{2} \nonumber\\
%	& + \sum_{\ell=1}^{L-1} \frac{\gamma_{\ell}}{2} \| \mA_{\ell} - {\bm \phi}_{\ell}(\mZ_{\ell}) + \mV_{\ell}\|_{F}^{2}\nonumber
%\end{eqnarray}
%where $\beta_{\ell}, \gamma_\ell >0$ are the step sizes, $\mU_{\ell}$ and $\mV_{\ell}$ are the \textit{(scaled) dual variables} \cite{boyd2011distributed} for the equality constraint at the layer $\ell$. 
%Algorithm \ref{alg:admm} shows a standard ADMM scheme for optimizing Eq. (\ref{eq:augmented}). Note, the parameters are updated in a closed-form as analytical solution can be simply derived. For simplicity of the equations, we denote $\gP_\ell (.) = \frac{\beta_{\ell}}{2} \| \mZ_{\ell} - \mW_{\ell} \mA_{\ell-1} + \mU_{\ell}\|_{F}^{2} $ and $\gQ_\ell (.) = \frac{\gamma_{\ell}}{2} \| \mA_{\ell} - {\bm \phi}_{\ell}(\mZ_{\ell}) + \mV_{\ell}\|_{F}^{2}$. 
%This is solved in \cite{taylor2016training,wang2019admm} with the difference that they only enforced the constraints on the last layer $L$ with dual variables while other constraints were only loosely enforced using quadratic penalty. 

%\begin{algorithm}[htb]
%   \caption{Standard ADMM for DNN Training}
%   \label{alg:admm}
%\begin{algorithmic}
%   {\STATE \scalebox{1}{\bfseries Input:} data $\mX$, labels $\mY$}
%   \STATE  \scalebox{1}{{\bfseries Params:} $\beta_\ell >0, \gamma_\ell >0,\lambda_\ell > 0$ }
%   \STATE  \scalebox{0.8}{{\bfseries Initialize:} $\{\mW_\ell^0\}_{\ell=1}^{L}, \{ \mU_\ell^0\}_{\ell=1}^{L}, \{ \mV_\ell^0\}_{\ell=1}^{L-1}, \{\mZ^0_\ell\}_{\ell=1}^{L}, \{\mA^0_\ell\}_{\ell=1}^{L-1}\; k \leftarrow 0$ }
%   \REPEAT
%   \FOR{$\ell=1$ {\bfseries to} $L$}
%   \STATE \scalebox{1}{$\mW_\ell^{k+1} \leftarrow \argmin\; \{ \gP_\ell (.) +  \lambda_{\ell}  {\bf r}_{\ell} (\mW_{\ell}^{k})\}$}
%   \ENDFOR
%   \FOR{$\ell=1$ {\bfseries to} $L-1$}
%   \STATE \scalebox{1}{ $\mZ_\ell^{k+1} \leftarrow \argmin\; \{ \gP_\ell (.) +  \gQ_\ell (.) \}$ }
%   \STATE \scalebox{1}{$\mA_\ell^{k+1} \leftarrow \argmin\; \{ \gP_{\ell+1} (.) +  \gQ_\ell (.) \} $}
%   \ENDFOR
%     \STATE \scalebox{1}{ $\mZ_{L}^{k+1} \leftarrow \argmin\; \{ \mathcal{J}\left(\mY, \mZ_{L}^{k} \right) + \gP_L (.) \}$ }
%   \FOR{$\ell=1$ {\bfseries to} $L-1$}
%   \STATE \scalebox{1}{$\mU_\ell^{k+1} \leftarrow \mU_\ell^{k} + \mZ_{\ell}^{k+1} - \mW_{\ell}^{k+1} \mA_{\ell-1}^{k+1}$}
%   \STATE \scalebox{1}{$\mV_\ell^{k+1} \leftarrow \mV_\ell^{k} + \mA_{\ell}^{k+1} - {\bm \phi}_{\ell}(\mZ_{\ell}^{k+1})$}
%   \ENDFOR
%   \STATE \scalebox{1}{$\mU_L^{k+1} \leftarrow \mU_L^{k} + \mZ_{L}^{k+1} - \mW_{L}^{k+1} \mA_{L-1}^{k+1}$}
%   \UNTIL{some stopping criterion is reached.}
% \end{algorithmic}
% \end{algorithm}


There were many hurdles in using ADMMs for deep learning --- the global convergence proof of the ADMM \cite{deng2016global} assumes that the optimization objective is deterministic and the global solution is calculated at each iteration of the cyclic parameter updates.
% and during each iteration of the cyclic parameter updates, all the data samples are visited.
This typically requires matrix inversion and makes standard ADMM computationally expensive thus impractical for training of many large-scale optimization problems. To see a formulation of standard ADMM for training DNNs refer to the supplementary materials \ref{sec:admm_nn}.  %Specifically, for  deep learning, this would impose a severe restriction on training set size when limited computational resources are available. 
%In addition, since the variable updates in standard ADMM require matrix inversion, the extent of its applications is limit to trivial tasks \cite{taylor2016training}, making it incompetent to perform on par with the recent complex architectures introduced in deep learning (e.g. \cite{he2016deep}).

In this section, we present stochastic Block-ADMM which does not require global solution as well as an online version which further reduces the communication load. We prove the convergence of these algorithms in Sec. \ref{sec:convergence} and present its application in supervised disentanglement in Sec.~\ref{sec:deepfacto}.

\begin{figure*}[t]
\begin{center}
\subfigure[] { \label{fig:block_admm}
\includegraphics[width=0.75\linewidth]{imgs/block_admm.pdf}
}
\subfigure[] { \label{fig:block}
\includegraphics[width=0.15\linewidth]{imgs/block.pdf}
}
\end{center}
% \vspace{-.2in}
\caption{\small a) General Architecture for training DNNs proposed in Stochastic block-ADMM. b) A few differential layers selected from a parent network are stacked inside a block. The parameters $\Theta_t$ are updated by SGD in a forward-backward pass.}
%  \vskip -0.1in
\end{figure*}




% \vspace{-0.025in}

%-----------------------------------------
\subsection{Stochastic Block-ADMM}\label{sec:block_admm}
% \vspace{-0.025in}

In this section, we introduce a novel variant of ADMM for training DNNs, the stochastic block-ADMM. We first split the conventional multi-layer network architectures into an arbitrary number of \emph{blocks}, each containing only a part of the network. To make the parameters of each block independent from its neighbors, \emph{decoupling variables} \{$\mZ_t, \; t=1, \dots, T$\} are introduced as shown in Fig.~\ref{fig:block_admm}. These variables pass the information forward and backward in the architecture to train blocks in a cyclic manner until consensus is reached. Each $block_t$  consists of one or multiple differentiable layers (e.g., convolutional layers, activation layers, etc.) that are detached from the rest of the network via coupling variables. Denote the set of all learnable parameters of each $block_t$ as $\Theta_t$. As an example, a $block_t$ wrapping multiple layers can be seen in Figure \ref{fig:block}. Our formulation is:
\begin{align} \label{eq:ourformulation}
	\minimize_{ {\bm \Theta}, \mathcal{Z}}\; &\mathcal{J}\left(\mY, \mZ_{T} \right) 
	 \\
 {\rm subject~to} ~ &\bm Z_t = \mathrm{block}_{\bm \Theta_t}(\bm Z_{t-1}), \quad \mZ_{0} = \mX \nonumber
\end{align}
where ${\bf \Theta} = \{\Theta_t\}_{t=1}^{T} \text{and } \mathcal{Z} = \{\mZ_t\}_{t=1}^{T}$. $\mathcal{J}$ is the desired cost to be minimized (\textit{e.g.}, cross-entropy loss), $T$ is the total number of blocks, $\mX = \{ \vx_1,\dots, \vx_N \} \in  \R^{M \times N}$ is the input data, and $\mY = \{\vy_1,\dots, \vy_N \} \in \R^{C \times N}$ is the target label -- for $C$ classes. Note that the number of blocks $T$ can be different than the number of layers in the network $L$.

To train DNNs with this new approach, 
%similar to problem (\ref{eq:augmented}), 
we would have the following augmented Lagrangian minimization problem to enforce the equality constraints needed for training,
\begin{align} \label{eq:block_admm_unconstrained}
	\min_{ {\bf \Theta}, \mathcal{Z}} \; &\mathcal{J}\left(\mY, \mZ_{T} \right) 
	+ \sum_{t=1}^{T} \frac{\beta_t}{2} \| \mZ_t - \mathrm{block}_{\Theta_t}(\mZ_{t-1}) + \mU_t\|_F^2 \nonumber \\
	& {\rm subject~to} \quad \mZ_{0} = \mX 
\end{align}
where $\beta_t$ and $\mU_t$ are the (scaled) step size  and the Lagrange multiplier corresponding to the $t$-th Block, respectively. Our proposed Stochastic block-ADMM method for training problem (\ref{eq:block_admm_unconstrained}) is presented in Algorithm \ref{alg:blockadmm}. %In stochastic block-ADMM%, parameters of the $t$-th block, $\Theta_t$ are updated using the \emph{Stochastic Gradient Descent} optimizer or its adaptive learning rate variants (e.g. \emph{Adam} \cite{kingma2014adam}). %or second-order optimizers including Newton's method and \emph{(L)BFGS}. 
%We have found Adam to consistently outperform other counterparts, particularly in updating the decoupling variables $\mZ_t$. 
$\zeta_t$ and $\eta_t$ are the learning rates in each update step for $\mZ_t$ and $\Theta_t$, respectively. Similar to training conventional neural networks, each block is updated by first going in a forward pass through the block and update the parameters using back-propagation. Update of the block parameters $\Theta_{t}$ is done using mini-batch stochastic gradient descent or Adam. The same goes for the decoupling variables $\mZ_t$. Note, in each cycle of the parameter update in Algorithm \ref{alg:blockadmm}, all the samples of $\mZ$ are updated, while $\Theta_{t}$ is updated stochastically. In addition, due to non-convexity of primal sub-problem (Eq. \ref{eq:primal}), one can perform the primal updates for multiple steps. %However, we found one step update to be sufficient in our experiments.
In Algorithm \ref{alg:blockadmm}, we take the reverse order for updating  the decoupling variables $\mZ_t$, which we have empirically found more efficient, as analogous to backpropagation where gradient flows backwards as well.
%Although any ordering should converge in theory, a few may converge faster in practice (e.g., in Algorithm \ref{alg:admm}, we found it more stable to update the  weights $\mW_\ell$ first). 

Note that in this formulation, %no gradient is backpropagated through the entire network. To be more precise, 
backpropagation stops at each auxiliary variable $\mZ_t$ . Hence, our method can readily mitigate the long-known vanishing gradient problem by splitting a conventional DNN into arbitrary sized blocks. 
%In section \ref{exp:mnist}, the results from our proposed method, Block-ADMM, compared with baselines in dealing with vanishing gradient and performance in supervised learning are presented. Keep in mind, the strategy of splitting a conventional networks into the Blocks is completely optional and relies on the task to be accomplished e.g., for the heterogeneous problem of activation factorization as illustrated in section \ref{exp:weakly}, the activation layer to be factorized is a splitting point, $\mZ_t$.\\
During testing time, one could follow Eq. (\ref{eq:block_admm_unconstrained}) to solve an optimization problem. But in practice, it suffices to use a straight-through estimator by removing the decoupling variables and simply pass the output of each layer to the next, equivalent of doing a forward pass in a conventional DNN. %We have taken this approach in our experiments as the error induced by ignoring the decoupling variables is negligible. 
%It should be noted that the update step for $\mZ_\ell$ is dependent on the adjacent blocks, hence can only be parallelized across the data points. However, fixing $\mZ$, all the block parameters $\Theta_t$ are independent of each other, hence can be updated in parallel across blocks as well as data points.


%---------------------------- algorithm block admm ------------------------------
%  \vspace{-0.05in}
\begin{algorithm}[htb]
   \caption{Stochastic Block-ADMM}
   \label{alg:blockadmm}
\begin{algorithmic}
   {\STATE \scalebox{1}{\bfseries Input:} data $\mX$, labels $\mY$}
   \STATE  \scalebox{1}{{\bfseries Params:} $\beta_t >0, \; \zeta_t >0, \eta_t >0$ }
   \STATE  {\bfseries Define:} \scalebox{0.80}{ $\mathcal{T}({\mZ_{t}, \mZ_{t-1}, \mU_{t}, \Theta_t}) = \frac{\beta_t}{2} \| \mZ_t - \mathrm{block}_{\Theta_t}(\mZ_{t-1}) + \mU_t\|_F^2$ }
   \STATE  \scalebox{1}{{\bfseries Initialize:} $\{{\Theta_t^0}\}_{t=1}^{T}, \{ \mU_t^0\}_{t=1}^{T} ,\; k \leftarrow 0$ }
   \STATE  \scalebox{1}{{\bfseries Initialize:} $\{\mZ_t\}_{t=1}^{T}$ in a forward pass. }
   \REPEAT
   \STATE \scalebox{1}{ $\mZ_{T}^{k+1} \leftarrow \mZ_{T}^{k} - \zeta_T \nabla_{\mZ_{T}^k} ( \mathcal{J}\left(\mY_{i}, \mZ_{T}^{k} \right)$ }
   \STATE \scalebox{1}{$ + \mathcal{T}({\mZ_{T}^k, \mZ_{T-1}^k, \mU_{T}^k, \Theta_L^k})) \;  $}
   %\forall i \in \{1,\dots,N\} $}
   \FOR{$t=T-1$ {\bfseries to} $1$}
   \STATE \scalebox{1}{ $\mZ_{t}^{k+1} \leftarrow\mZ_{t}^{k} - \zeta_t \nabla_{\mZ_{t}^k} ( \mathcal{T}({\mZ_{t}^k, \mZ_{t-1}^k, \mU_{t}^k, \Theta_{t}^k})$} \\
   \STATE  \scalebox{1}{$+ \mathcal{T}({\mZ_{t+1}^{k+1}, \mZ_{t}^k, \mU_{t+1}^k, \Theta_{t+1}^k})) \; $}
%   \forall i \in \{1,\dots,N\}  $}
   \ENDFOR
   \FOR{$t=1$ {\bfseries to} $T$}
%   \STATE{${\rm draw} \; i \subset \{1,\dots,N\}$}
   \STATE \scalebox{0.9}{${\Theta_t}^{k+1} \leftarrow {\Theta_t}^{k} - \eta_t  \nabla_{\Theta_t} \mathcal{T}({\mZ_{t,i}^{k+1}, \mZ_{t-1,i}^{k+1}, \mU_{t,i}^k, \Theta_{t}^k}),$}
   \STATE \scalebox{0.9}{$draw \; i \subset \{1,\dots,N\} \; $}
   \STATE \scalebox{1}{$\mU_t^{k+1} \leftarrow \mU_t^{k} + \mZ_{t}^{k+1} - \mathrm{block}_{\Theta_t}^{k+1}(\mZ_{t-1}^{k+1})$}
   \ENDFOR
   \UNTIL{some stopping criterion is reached.}
\end{algorithmic}
\end{algorithm}
% \vspace{-0.05in}


%----------------------------
\subsection{Online Stochastic Block-ADMM}\label{sec:onlineadmm}
% \vspace{-0.025in}

The stochastic block-ADMM formulation in section \ref{sec:block_admm} is still a batch mode algorithm, in the sense that the entire training set is updated at once. This imposes restrictions on the size of the input and the number of parameters in the network when limited resources are available. %As a result, the extension to extremely large datasets, often the case in deep learning applications, can be problematic. 
Also, it does not readily accommodate to settings where data is constantly changing, such as data augmentation on the input or reinforcement learning. To overcome such limitations, we propose an \textit{online} variant of the stochastic block-ADMM in Algorithm \ref{alg:online_admm} which alternatively solves the unconstrained problem, 
\begin{align}\label{eq:scalar_dual}
	\min_{ {\bf \Theta}, \mathcal{Z}} \; &\mathcal{J}\left(\vy, \vz_{T} \right) 
	+ \sum_{t=1}^{T} \frac{\beta_t}{2} \big( \|\vz_t - \mathrm{block}_{\Theta_t}(\vz_{t-1}) \|_F^2 + u_t\big) \nonumber\\
	& {\rm subject~to} \quad \vz_{0} = \vx 
\end{align}
Although similar to the Eq. (\ref{eq:block_admm_unconstrained}), the dual variable in the online Block-ADMM is a \textit{scalar}. The benefits of this are two-folded: First, this substantially reduces the memory size needed for storing the dual variables as the optimization proceeds. Second, this considerably reduces the variance in the gradient induced by re-initializing the auxiliary variables $\vz_{\ell,i}$ when updating the block parameters at each iteration. %On the other hand, when the dual variable has the same size as the auxiliary variables as in the batch stochastic block-ADMM, the algorithm easily diverges.

% In Algorithm \ref{alg:online_admm}, as the data pair $(\vx_i, \vy_i)$ comes, first, the auxiliary variables are initialized in a forward pass though the blocks. Then, one iteration of the alternating update by ADMM is performed.

%---------------------------- algorithm block admm ------------------------------
% \vspace{-0.1in}
\begin{algorithm}[htb]
   \caption{Online Stochastic Block-ADMM }
   \label{alg:online_admm}
\begin{algorithmic}
   {\STATE \scalebox{1}{\bfseries Input:} data $\mX$, labels $\mY$}
   \STATE  \scalebox{1}{{\bfseries Params:} $\beta_t >0, \; \zeta_t >0, \eta_t >0$ }
   \STATE  {\bfseries Define:} \scalebox{0.8}{ $\mathcal{T}({\vz_{t}, \vz_{t-1}, u_{t}, \Theta_t}) = \frac{\beta_t}{2} (\| \vz_t - \mathrm{block}_{\Theta_t}(\vz_{t-1}) \|_2 + u_t)^2$ }
   \STATE  \scalebox{1}{{\bfseries Initialize:} $\{{\Theta_t^0}\}_{t=1}^{T}, \{ u_t^0\}_{t=1}^{T} ,\; k \leftarrow 0$ }
%   \STATE  \scalebox{1}{{\bfseries Initialize:} $\{\mZ_t\}_{t=1}^{T}$ in a forward pass. }
   \REPEAT 
   \FOR{$(\vx_i, \vy_i) \text{\bfseries in} (\mX,\mY)$}
   \STATE \scalebox{1}{{\bfseries Initialize:} $\{\vz_{t,i}\}_{t=1}^{T}$ in a forward pass $(\vz_{0,i} = \vx_i)$.}
   \STATE \scalebox{1}{$\vz_{T,i} \leftarrow \vz_{T,i} - \zeta_T \nabla_{\vz_{T,i}} ( \mathcal{J}\left(\vy_{i}, \vz_{T,i} \right)$ }
   \STATE \scalebox{1}{$+\mathcal{T}({\vz_{T}, \vz_{T-1}, u_{T}^k, \Theta_T^k})) \;  $}
   %\forall i \in \{1,\dots,N\} $}
   \FOR{$t=T-1$ {\bfseries to} $1$}
   \STATE \scalebox{1}{ $\vz_{t,i} \leftarrow\vz_{t,i} - \zeta_t \nabla_{\vz_{t,i}} ( \mathcal{T}({\vz_{t,i}, \vz_{t-1,i}, u_{t}^k, \Theta_{t}^k})$} \\
   \STATE  \scalebox{1}{$+ \mathcal{T}({\vz_{t+1,i}, \vz_{t,i}, u_{t+1}^k, \Theta_{t+1}^k})) \; $}
%   \forall i \in \{1,\dots,N\}  $}
   \ENDFOR
   \FOR{$t=1$ {\bfseries to} $T$}
%   \STATE{${\rm draw} \; i \subset \{1,\dots,N\}$}
   \STATE \scalebox{1}{${\Theta_t}^{k+1} \leftarrow {\Theta_t}^{k} - \eta_t  \nabla_{\Theta_t} \mathcal{T}({\vz_{t,i}, \vz_{t-1,i}, u_{t}^k, \Theta_{t}})$}
%   \STATE \scalebox{0.9}{$draw \; i \subset \{1,\dots,N\} \; $}
   \STATE \scalebox{1}{$u_t^{k+1} \leftarrow u_t^{k} + \| \vz_{t}^{k} - \mathrm{block}_{\Theta_t^{k+1}}(\vz_{t-1,i})\|_2$}
   \ENDFOR
   \ENDFOR
   \UNTIL{some stopping criterion is reached.}
\end{algorithmic}
\end{algorithm}
% \vspace{-0.125in}




%----------------------------
\subsection{Convergence of the Algorithm}\label{sec:convergence}
Let us consider the following general problem:
\begin{align}\label{eq:main}
	\minimize_{\mathcal{Z},\bf \Theta} & f(\mathcal{Z})\\
	{\rm subject to} & h(\mathcal{Z},\bm \Theta)=\bm 0,\nonumber
\end{align}
where $\mathcal{Z}$ and $\bf \Theta$ are as defined in Sec.~\ref{sec:block_admm}, and $f(\cdot)$ represents the training objective, and $h(\cdot)$ represents the layer coupling equalities as in \eqref{eq:ourformulation}.
We also assume that both $f(\cdot)$ and $h(\cdot)$ are differentiable functions. Note that both $f$ and $h$ can be non-convex.

Let us consider the following augmented Lagrangian:
\[          {\cal L}_{\rho_k}(\mathcal{Z},{\bf \Theta},{\bm \lambda})=f(\mathcal{Z}) + \langle \bm \lambda, h(\mathcal{Z},{\bf \Theta})\rangle + \frac{1}{2\rho_k}\|h(\bm Z,\bf \Theta)\|_2^2,  \]
where $\bm \lambda$ collects all the dual variables $\bm U_1,\ldots,\bm U_T$ that correspond to different layers. The standard primal-dual updates can be summarized as follows:
\begin{subequations}\label{eq:stopdd}
\begin{align}
    (\bm Z^{k+1},\Theta^{k+1}) &\leftarrow  \arg\min_{\mathcal{Z},\bf \Theta}  {\cal L}_{\rho_k}(\mathcal{Z},\bf \Theta,\bm \lambda^k), \label{eq:primal}\\
    \bm \lambda^{k+1} &\leftarrow \bm \lambda^k + \frac{1}{\rho_k}h(\bm Z^{k+1}, \Theta^{k+1}),
\end{align}
\end{subequations}
%In our case, since the sub-problem in \eqref{eq:primal} is non-convex, exactly minimizing this function may not be possible. In the previous section, the primal update is carried out by stochastic optimization w.r.t. $\mathcal{Z}$ and $\mathbf{\Theta}$ in an alternating fashion---which is a computationally lightweight algorithm that converges to a stationary point of ${\cal L}_{\rho_k}(\mathcal{Z},\mathbf{ \Theta},\bm \lambda^k)$ under certain conditions \cite{bottou2012stochastic,xu2015block}.
%The convergence of this type of primal-dual algorithm with inexact stochastic solution for the primal problem is unclear. In this work, we offer convergence support for our designed deep network training algorithm. Our idea follows recent work in \cite{shi2017penalty} that handles deterministic primal problems under non-convex equality constraints. 
We employ the trick in \cite{shi2017penalty} for adaptively adjusting the parameter $\rho_k$. We assume that $\rho_k$ is adjusted by
\begin{align}\label{eq:rho}
    \rho_{k+1} \leftarrow \begin{cases}  \rho_k,&\quad \|h(\bm Z^{k},\bm \Theta^k)\|\leq \eta_k,\\
                                         c\rho_k,~0<c<1,&\quad {\rm o.w.}
    \end{cases}
\end{align}
where $\eta_k$ for $k=1,2,\ldots$ is a pre-specified sequence that bounds the equality-enforcing error.

Our analysis shows the following convergence result:
% \vspace{-0.1in}
\begin{Prop}\label{prop:convergence}
    Assume $h({\cal Z},\bm \Theta)=\bm 0$ satisfies the Robinson's condition.
    Also assume for each update in \eqref{eq:primal}, the sub-problem solution solved by stochastic alternating optimization satisfies
   \begin{equation}
       \mathbb{E}\left[ \left\| {\cal G}(\x^k) \right\|^2\right]\leq \varepsilon_k, ~ \mathbb{V}\left[  {\cal G}(\x^k) \right]\leq \sigma_k^2,
   \end{equation}  
   where $\x=({\cal Z},\bm \Theta)$ is a vector that collects all the optimization variables and ${\cal G}(\x^k)$ collects the stochastic gradients that we used for updating $({\cal Z},\bm \Theta)$.
   Assume that the stochastic gradient for the primal update is unbiased, i.e.,
   \begin{equation}\label{eq:unbiasedness}
       \mathbb{E}[{\cal G}(\x^k)] = \nabla {\cal L}_{\rho_k}(\x_k),~\forall k. 
   \end{equation}         
   Then, every limit point of the solution sequence produced by the algorithm in \eqref{eq:stopdd} converges to a KKT point of the problem in~\eqref{eq:main}, if $\eta_k\rightarrow 0$, $\sigma_k^2 \rightarrow 0$ and $\varepsilon_k\rightarrow 0$.	
\end{Prop}
% \vspace{-0.05in}
% {\it \bf Proof}: 
The proof for Proposition~\ref{prop:convergence} is presented in the supplementary materials \ref{sec:proof}.
Proposition~\ref{prop:convergence} asserts that the algorithm converges to a KKT point under some conditions. 
   There are a number of remarks regarding implementation.
   To begin with, the condition $\varepsilon_k\rightarrow 0$ means that the primal problem needs to be solved more and more accurately when $k$ grows, in terms of approaching the stationary point of the sub-problem using block stochastic gradient. This can be achieved via gradually increasing the number of iterations for the primal updates. Note that stochastic block gradient can provably attain $\mathbb{E}[\|{\cal G}(\X^k)\|^2]\leq \varepsilon_k$; see \cite{xu2015block}. 
%   \comment{
%   In addition, the condition $\sigma_k^2\rightarrow 0$ means that the variance of the stochastic gradient needs to shrink when $k$ increases. This can be achieved by increasing the batch size when $k$ grows; see discussions in \cite{xu2015block}. 
%   Hence, {\it in theory}, to satisfy both conditions, the complexity for carrying out each iteration $k$ may grow.
%   Nonetheless, our empirical experience shows that using a fixed number of iterations for stochastic primal optimization and a fixed batch size in general does not hurt convergence. We hypothesize that the momentum may play an important role in increasing the effective batch size since with momentum gradients from previous batches are remembered and utilized in the subsequent steps. We leave further analysis with momentum to future research. Another remark is that the unbiasedness of the primal stochastic gradient [cf. \eqref{eq:unbiasedness}] is not always easy to establish under block coordinate descent settings \cite{xu2015block}. Nonetheless, this can be fixed via a simple randomization strategy among the blocks \cite{fu2019block}. }
   
   %The third challenge is that the hyperparameter $c$ and the sequence $\{\eta_k\}$ are not necessarily easy to select in some cases~\cite{shi2017penalty,fu2018anchor}. These parameters control how quickly one should adjust the update settings to accommodate the current iteration, which varies from case to case. However, interestingly, we find that these hyperparameters are relatively easy to tune in our case. In particular, our extensive experiments show that fixed $\rho_k$ and $\eta_k$ work reasonably well for our deep learning problems.
   
   %The slightly stringent conditions in Proposition~\ref{prop:convergence} and the relatively `benign' convergence behavior observed in practice pose a gap between theory and practice---and an interesting direction for future research.



%----------------------------
\subsection{DeepFacto: Factorization of DNN Activations }\label{sec:deepfacto}

%To show the power and flexibility of our proposed method in training heterogeneous networks, we will 
Here, we investigate a task for supervised disentanglement, which can provide insights for explaining DNNs to humans. Supervised disentanglement aims to find disentangled factors that decide the CNN output, yet are human-understandable and distinct from each other. One approach to learn a disentangled representation is through adding  non-negative matrix factorization (NMF)\cite{lee1999learning} layers to the network \cite{collins2018deep}. Note that NMF imposes non-differentiable constraints into the network where conventional end-to-end training using backpropagation would not be applicable. Hence, prior work were mostly running NMF after the training, where the network might have already learned highly entangled features. In this work, aided with our stochastic block-ADMM, we attempt to perform training with NMF layers in the intermediate layers of DNNs.

Figure \ref{fig:deepfacto} shows an \emph{NMF module} with \emph{rank $r$} incorporated between two arbitrary neighboring blocks. The output from the $block_t$ is factorized into $\mM_t$ and $\mS_t$, namely, the basis and score matrices. In this configuration, only the score matrix $\mS_t$ is passed to the next blocks. The score matrix is low-rank, sparse and non-negative hence can possibly represent features that are more disentangled than the original network. 
Exploring this architecture is one attempt of us in making deep networks more explainable to humans. Humans would not be able to interpret conventional deep network weights which are both positive and negative and sometimes cancels out each other. The sparse and non-negative feature from NMF would be much more preferable to interpret~\cite{collins2018deep}.

However, the NMF module breaks the gradient path from $\mS_t$ to $Z_t$, hence conventional backpropagation would not be applicable in this problem. We extend the ADMM framework (\ref{eq:block_admm_unconstrained}) into having non-negative factorization constraints over its activations and formulate the following optimization problem:
\begin{eqnarray} \label{eq:block_admm_nmf}
	\min_{ {\bf \Theta}, \mathcal{Z}, \mS, \mM} \; &\mathcal{J}\left(\mY, \mZ_{T} \right) \nonumber \\
	+ & \sum_{k=1, k\neq t+1}^{T} \frac{\beta_k}{2} \| \mZ_k - block_k(\mZ_{k-1}) + \mU_k\|_F^2 \nonumber \\
	+ & \frac{\beta_{t+1}}{2} \| \mZ_{t+1} - block_{t+1}(\mS_{t}) + \mU_{t+1}\|_F^2 \nonumber \\
	+ &  {\frac{\gamma_t}{2} \| \mZ_t - \mM_t \mS_t + \mV_t\|_F^2} \nonumber \\
	  & {\forall i,j} \;  \mM_{\ell,ij} \ge 0,\; \mS_{\ell,ij} \ge 0 
\end{eqnarray}
where $\gamma_t$ is the step-size and $\mV_t$ is the corresponding multipliers to enforce the matrix factorization equality $\mZ_t = \mM_t \mS_t$. The NMF module adds a nonconvex term to the optimization. However, in the alternating optimization scheme, while keeping either $\mM_t$ or $\mS_t$ constant, solving for the other term would reduce to a normal convex least-squares problem. The rest of the updates are the same as in section \ref{sec:block_admm}. Note that, trivially to not change the input dimension of the next block after the NMF module, one can simply add an affine layer to increase the dimensions without changing the formulation.
% , as the linear layer can be easily regarded in the parameter space of the next block. 

At testing time, one only needs to perform a non-negative projection since the basis matrix $M$ will be given, which can be solved using a convex solver such as LBFGS. Note that for simplicity, we only formulated adding \emph{one} NMF module in the middle of the blocks. This can be simply extended to as many NMF modules as needed in the architecture.

%---------------------------- Figure admm nmf  ------------------------------
\begin{figure}[t!]
% \vskip -0.1in
\begin{center}
\centerline{
\includegraphics[width=1 \columnwidth]{imgs/block_admm_nmf.pdf}
}
%  \vskip -0.1in
 \caption{General architecture for Deepfacto: an NMF module with rank $r$ is added in the middle of two arbitrary blocks. Note, only $\mS_t$ is passed to the next blocks.}
%\vspace{-0.05in}
\label{fig:deepfacto}
\end{center}
%  \vskip -0.3in
\end{figure}

\section{Experiments}

\subsection{Experimental Settings}
\noindent\textbf{Datasets.}
We evaluate our method on two popular benchmarks: CUB-200-2011~\cite{welinder2010caltech} and ImageNet-1K~\cite{russakovsky2015imagenet}.
In the CUB-200-2011 dataset, there are 5,994 images for training and 5,794 for testing from 200 bird species. In the ImageNet-1K, there are approximately 1.3 million images in the training set and 50,000 in the validation set from 1,000 different classes.

\noindent\textbf{Evaluation Metrics.}
Following the work of Russakovsky~\etal~\cite{russakovsky2015imagenet}, we use Top-1 localization accuracy (Top-1 Loc), Top-5 localization accuracy (Top-5 Loc), and localization accuracy with ground-truth class (GT Loc) as our evaluation metrics.
Top-$k$ Loc is the proportion of the images whose predicted bounding box has more than 50\% intersection over union (IoU) with the ground-truth bounding box and whose predicted top-$k$ classes include the ground-truth class.
GT Loc is the localization accuracy with the ground-truth class, which does not consider the classification result.
We also use \texttt{MaxBoxAccV2}~\cite{choe2020evaluation} to evaluate our method.
\texttt{MaxBoxAccV2}$(\delta)$ measures the localization accuracy with ground-truth class with multiple IoU thresholds $\delta\in\{0.3, 0.5, 0.7\}$.

\begin{figure*}[t]
	\centering
    \includegraphics[width=\textwidth]{figures/fig_comparison_cam.pdf}
    \vspace{-2em}
    \caption{Comparison of localization results from the vanilla method and our method on CUB-200-2011 and ImageNet-1K datasets, using VGG16 as a backbone. Blue boxes denote the ground truth bounding boxes and green boxes denote the predicted bounding boxes.}
    \label{fig:compare_cam}
\end{figure*}
\begin{table}[tbp]
\renewcommand{\arraystretch}{0.95}
  \centering
    \begin{tabular}{lccc}
    \Xhline{1pt}\\[-0.95em]
    Method & Top-1 & Top-5 & GT Loc \\
   \hline\hline
\multicolumn{2}{l}{Additional Branch}\\
SLT-Net~\cite{guo2021strengthen}$_{\text{~~CVPR '21}}$ & 67.8 & - & 87.6 \\
ORNet~\cite{xie2021online}$_{\text{~~ICCV '21}}$  &67.74 &80.77 &86.19 \\
FAM~\cite{meng2021foreground}$_{\text{~~ICCV '21}}$  &69.26 &- &89.26 \\
\midrule
\multicolumn{2}{l}{Single Branch}\\
CAM~\cite{zhou2016learning}$_{\text{~~CVPR '16}}$  &44.15 &52.16 &56.00 \\
ADL~\cite{choe2019attention}$_{\text{~~CVPR '19}}$  &52.36 &- & 75.41 \\
DANet~\cite{xue2019danet}$_{\text{~~ICCV '19}}$  &52.52 &61.96 &67.70 \\
EIL~\cite{mai2020erasing}$_{\text{~~CVPR '20}}$  &56.21 &- &- \\
MEIL~\cite{mai2020erasing}$_{\text{~~CVPR '20}}$  &57.46 &- &- \\
DGL~\cite{tan2020dual}$_{\text{~~ACMMM '20}}$ & 56.07 & 68.50	& 74.63 \\
Ki~\etal~\cite{ki2020sample}$_{\text{~~ACCV '20}}$  &57.50 &- &- \\
Bae~\etal~\cite{bae2020rethinking}$_{\text{~~ECCV '20}}$ & 58.96 &  - & 76.30\\
Pan~\etal~\cite{pan2021unveiling}$_{\text{~~CVPR '21}}$  &60.27 &72.45 &77.29 \\
Ours & \textbf{70.83} & \textbf{88.07} & 	\textbf{93.17}\\
    \Xhline{1pt}
    \end{tabular}%
    \vspace{-0.5em}
     \caption{Comparison of localization performance on the CUB-200-2011 test set, based on VGG16.}
  \label{tab:cub_top1loc_vgg}
\end{table}%

\noindent\textbf{Implementation Details.}
We evaluate our method using VGG16~\cite{simonyan2014very} and ResNet50~\cite{he2016deep} as backbone networks.
For VGG16, we adopt the GAP layer following the training settings of the previous work~\cite{zhou2016learning}.
For ResNet50, we set the stride of the third layer to 1.
The attentive dropout is applied before the last pooling layer in VGG16 and after the first block in the fourth layer in ResNet50.
We initialize the networks with the pretrained weights using ImageNet-1K~\cite{russakovsky2015imagenet}. We use a min-max normalization to draw the bounding box from the generated CAM.

\subsection{Comparison with State-of-the-art Methods}
We compare our method to the recent WSOL methods. For other WSOL methods, we report the localization performance of the original papers or that reproduced by \cite{choe2020evaluation,kim2021normalization,bae2020rethinking,tan2020dual}\footnote{https://github.com/clovaai/wsolevaluation}. Our method consistently outperforms existing WSOL methods using a single branch, across the datasets and the backbones by a large margin.

Tab.~\ref{tab:cub_top1loc_vgg} shows the localization performance on the CUB-200-2011~\cite{welinder2010caltech} test set, using VGG16 as a backbone. Our method achieves an 11.87\%p improvement in Top-1 Loc and a 16.87\%p improvement in GT Loc over the work of Bae~\etal~\cite{bae2020rethinking}, which is the state-of-the-art method among the CAM-based methods.
Furthermore, our method outperforms the methods adopting an additional branch for localization. Our method improves Top-1 Loc by 1.57\%p and GT-Loc by 3.91\%p improvement in GT Loc compared to FAM~\cite{meng2021foreground}.

Tab.~\ref{tab:cub_top1loc} shows the results using ResNet50 as a backbone. It shows that our method consistently outperforms the existing methods by a large margin ($>$13\%p), using a different backbone.
\textcolor{black}{Tab.~\ref{tab:imagenet_top1loc} shows the localization performance on the ImageNet-1K~\cite{russakovsky2015imagenet} validation set, based on VGG16 and ResNet50.
Our method achieves the state-of-the-art performance in the ImageNet-1K dataset regardless of the backbone, and only Top-1 Loc with ResNet50 is the second best after I$^2$C with a marginal difference.}

Additionally, we compare our \texttt{MaxBoxAccV2}~\cite{choe2020evaluation} scores with other state-of-the-art methods on the CUB-200-2011 and ImageNet-1K in Tab.~\ref{tab:total_maxbox}.
It shows that our method outperforms the most recent methods by a large margin for all IoU thresholds with various backbones and datasets. Especially, our method improves the score with IoU threshold of 0.7, which is strict accuracy, by 21.0\%p and 17.4\%p with VGG16 and ResNet50 on the CUB-200-2011 dataset, respectively, compared with the work of Ki~\etal~\cite{ki2020sample}.

\begin{table}[tbp]
\renewcommand{\arraystretch}{0.95}
  \centering
    \begin{tabular}{lccc}
    \Xhline{1pt}\\[-0.95em]
    Method & Top-1 & Top-5 & GT Loc \\
   \hline\hline
CAM~\cite{zhou2016learning}$_{\text{~~CVPR '16}}$  & 46.91 &53.57 & - \\
ADL~\cite{choe2019attention}$_{\text{~~CVPR '19}}$  & 57.40 &- & 71.99 \\
CutMix~\cite{yun2019cutmix}$_{\text{~~ICCV '19}}$  &54.81 &- &- \\
DGL~\cite{tan2020dual}$_{\text{~~ACMMM '20}}$ & 60.82 &70.50 & 74.65\\
Ki~\etal~\cite{ki2020sample}$_{\text{~~ACCV '20}}$  &56.10 &- &- \\
Bae~\etal~\cite{bae2020rethinking}$_{\text{~~ECCV '20}}$ &  59.53 &  - & 77.58\\
Ours & \textbf{73.16} &  \textbf{86.68} & \textbf{91.60}\\
    \Xhline{1pt}
    \end{tabular}%
    \vspace{-0.5em}
     \caption{Comparison of localization performance on the CUB-200-2011 test set, based on ResNet50.}
  \label{tab:cub_top1loc}
\end{table}%

Fig.~\ref{fig:compare_cam} shows some examples of localization results from the vanilla method~\cite{zhou2016learning} and from our method on the CUB-200-2011 and ImageNet-1K datasets. It shows that the model trained with our method captures the target object region more accurately than the vanilla model. On the CUB-200-2011 dataset, while the vanilla model fails to identify the tails, legs, and wings of birds, the classifier trained with our method successfully identifies them.

\setuldepth{53.87}
\begin{table}[tbp]
\renewcommand{\arraystretch}{0.95}
  \centering
    \begin{tabular}{lccc}
    \Xhline{1pt}\\[-0.95em]
    Method & Top-1 & Top-5 & GT Loc \\
    \hline\hline
     \multicolumn{2}{l}{Backbone: VGG16}\\
    CAM~\cite{zhou2016learning}$_{\text{~~CVPR '16}}$  & 42.80 &54.86 & - \\
    ACoL~\cite{zhang2018adversarial}$_{\text{~~CVPR '18}}$  & 45.83 &59.43 &62.96 \\
    ADL~\cite{choe2019attention}$_{\text{~~CVPR '19}}$  & 44.92 &- &- \\
    CutMix~\cite{yun2019cutmix}$_{\text{~~ICCV '19}}$  & 43.45 &- &- \\
    I$^{2}$C~\cite{zhang2020inter}$_{\text{~~ECCV '20}}$  & 47.41 &58.51 &63.90 \\
    EIL~\cite{mai2020erasing}$_{\text{~~CVPR '20}}$  &46.27 &- &- \\
    MEIL~\cite{mai2020erasing}$_{\text{~~CVPR '20}}$  &46.81 &- &- \\
    Ki~\etal~\cite{ki2020sample}$_{\text{~~ACCV '20}}$  & 47.20 &- &- \\
    DGL~\cite{tan2020dual}$_{\text{~~ACMMM '20}}$ &   47.66 &58.89 &64.78 \\
    Bae~\etal~\cite{bae2020rethinking}$_{\text{~~ECCV '20}}$ & 44.62 &  - & 60.73\\
    Pan~\etal~\cite{pan2021unveiling}$_{\text{~~CVPR '21}}$  & \ul{49.56} &\ul{61.32} &\ul{65.05} \\
    Ours & \textbf{49.94} & \textbf{63.25} & 	\textbf{68.92}\\
    \midrule
    \multicolumn{2}{l}{Backbone: ResNet50}\\
    ADL~\cite{choe2019attention}$_{\text{~~CVPR '19}}$  & 48.23 & - &61.04 \\
    CutMix~\cite{yun2019cutmix}$_{\text{~~ICCV '19}}$  &47.25 &- &- \\
    Ki~\etal~\cite{ki2020sample}$_{\text{~~ACCV '20}}$  & 48.40 &- &- \\
    Bae~\etal~\cite{bae2020rethinking}$_{\text{~~ECCV '20}}$ &  49.42 & - & 62.20 \\
     I$^{2}$C~\cite{zhang2020inter}$_{\text{~~ECCV '20}}$  & \textbf{54.83} &\ul{64.60} &68.50 \\
    DGL~\cite{tan2020dual}$_{\text{~~ACMMM '20}}$ &  53.41 &62.69 &\ul{69.34} \\
    Ours & \ul{53.76}	& \textbf{65.75}	& \textbf{69.89}\\
    \Xhline{1pt}
    \end{tabular}%
    \vspace{-0.5em}
     \caption{Comparison of localization performance on the ImageNet-1K validation set. The best performance is bold and the second best performance is underlined.}
  \label{tab:imagenet_top1loc}
\end{table}%

\begin{table*}[t]
  \centering
\setlength{\tabcolsep}{3.73pt}
\begin{tabular}{l|cccc|cccc|cccc|cccc}
\Xhline{1pt}
\multirow{4}{*}{Method} &\multicolumn{8}{c|}{CUB-200-2011} &\multicolumn{8}{c}{ImageNet-1K} \\
&\multicolumn{4}{c|}{VGG16} &\multicolumn{4}{c|}{ResNet50} &\multicolumn{4}{c|}{VGG16} &\multicolumn{4}{c}{ResNet50} \\
&\multicolumn{3}{c}{$\delta$} &\multirow{2}{*}{Mean} &\multicolumn{3}{c}{$\delta$} &\multirow{2}{*}{Mean} &\multicolumn{3}{c}{$\delta$} &\multirow{2}{*}{Mean} &\multicolumn{3}{c}{$\delta$} &\multirow{2}{*}{Mean} \\
&0.3 &0.5 &0.7 & &0.3 &0.5 &0.7 & &0.3 &0.5 &0.7 & &0.3 &0.5 &0.7 & \\
\hline\hline
CAM~\cite{zhou2016learning} &96.8 &73.1 &21.2 &63.7 &95.7 &73.3 &19.9 &63.0 &81.0 &62.0 &37.1 &60.0 &83.7 &65.7 &41.6 &63.7 \\
HaS~\cite{singh2017hide} &92.1 &69.9 &29.1 &63.7 &93.1 &72.2 &28.6 &64.6 &80.7 &62.1 &38.9 &60.6 &83.7 &65.2 &41.3 &63.4 \\
SPG~\cite{zhang2018self} &90.5 &61.0 &17.4 &56.3 &92.2 &68.2 &20.8 &60.4 &81.4 &62.0 &36.3 &59.9 &83.9 &65.4 &40.6 &63.3 \\
ADL~\cite{choe2019attention} &97.7 &78.1 &23.0 &66.3 &91.8 &64.8 &18.4 &58.3 &80.8 &60.9 &37.8 &59.9 &83.6 &65.6 &41.8 &63.7 \\
CutMix~\cite{yun2019cutmix} &91.1 &67.3 &28.6 &62.3 &94.3 &71.5 &22.5 &62.8 &80.3 &61.0 &37.1 &59.5 &83.7 &65.2 &41.0 &63.3 \\
Ki~\etal~\cite{ki2020sample} &96.2 &77.2 &26.8 &66.7 &96.2 &72.8 &20.6 &63.2 &81.5 &63.2 &39.4 &61.3 &84.3 &67.6 &43.6 &65.2 \\
HaS + PaS~\cite{bae2020rethinking} &- &- &- &61.2 &- &- &- &61.9 &- &- &- &62.1 &- &- &- &64.6 \\
CALM~\cite{kim2021keep} &- &- &- &64.8 &- &- &- &71.0 &- &- &- &62.8 &- &- &- & 63.4 \\
ADL + IVR~\cite{kim2021normalization} &- &- &- &71.5 &- &- &- &67.1 &- &- &- &63.7 &- &- &- &65.1 \\
Ours & \textbf{99.3} & 	\textbf{93.2} & \textbf {47.8} & \textbf{80.1} & \textbf{99.4} & \textbf{90.4} & \textbf{38.0} & \textbf{75.9} & \textbf{84.8} & \textbf{69.2} & \textbf{45.9} & \textbf{66.6} & \textbf{86.7} & \textbf{71.1}	& \textbf{48.3}	& \textbf{68.7}\\
    \Xhline{1pt}
    \end{tabular}%
    \vspace{-0.5em}
      \caption{Comparison of \texttt{MaxBoxAccV2} scores on the CUB-200-2011 and ImageNet-1K datasets using various backbones.}
  \label{tab:total_maxbox}%
\end{table*}%
\begin{figure}[t]
	\centering
    \includegraphics[width=0.88\columnwidth]{figures/fig_comparison_sim.pdf}
    \vspace{-0.6em}
    \caption{Comparisons of CAM, $\mathcal{F}$, and $\mathcal{S}$ between the vanilla method and our method on the CUB-200-2011 and ImageNet-1K datasets, using VGG16 as a backbone.}
    \label{fig:compare_sim}
\end{figure}

\subsection{Discussion}
\noindent\textbf{Feature Direction Alignment.}
Through the feature direction alignment, we force $\mathcal{S}$ and $\hat{\mathcal{F}}$ to be high in the object region and to be low in the background region. As Fig.~\ref{fig:compare_sim} shows, the classifier trained with our method yields $\mathcal{S}$ that has a high value in the object region and low value in the background region, different from the vanilla model. It also generates $\hat{\mathcal{F}}$ that has higher activation in less discriminative parts than the vanilla model does.
This makes CAM successfully identify the entire object region. As mentioned in Sec.~\ref{sec:feature_directions}, the feature direction alignment makes $\hat{\mathcal{F}}$ and $\mathcal{S}$ similar, resulting that CAM becomes also similar with them.
We generate a localization map with $\mathcal{F}$ and $\mathcal{S}$ and evaluate the localization performance for each case. We use a min-max normalization when drawing bounding boxes from $\mathcal{F}$. Since negative values in $\mathcal{S}$ denote the background region, we apply a max-normalization on $\mathcal{S}$. Tab.~\ref{tab:perf_sim_norm_cam} shows that the localization results with $\mathcal{F}$ and $\mathcal{S}$ also achieve similar localization performance with CAM. This proves the coincidence between CAM, $\mathcal{F}$, and $\mathcal{S}$ with our method.

Fig.~\ref{fig:hist}(a) shows the distribution of $\mathcal{S}_u$ inside the ground truth bounding boxes from the vanilla method and our method. Note that the bounding boxes include not only the target object but also the background region.
As the training progresses with our method, the similarity gradually splits into negative and large positive values.
This shows that our method effectively increases the similarity for the foreground region and decreases it for the background region.
In contrast, for the vanilla method, the similarity is clustered in small positive values, making no distinction between \mbox{the two}.

\begin{table}[t]
\normalsize
  \centering
  
    \begin{tabular}{cccc}
    \Xhline{1pt}
    Localization map & Top-1  & Top-5 & GT Loc \\
    \hline\hline
     \texttt{CAM}  & 70.83 & 88.07 & 93.17 \\
     $\mathcal{F}$  & 69.90 & 86.68 & 91.96 \\
     $\mathcal{S}$  & 70.38 & 87.64 & 93.13 \\
    \Xhline{1pt}
    \end{tabular}%
    \vspace{-0.5em}
    \caption{Localization performance with various localization maps on the CUB-200-2011 test set, based on VGG16.}
  \label{tab:perf_sim_norm_cam}%
\end{table}%


\begin{figure}[t]
	\centering
    \includegraphics[width=0.93\columnwidth]{figures/fig_hist1.pdf}
    \vspace{-0.7em}
    \caption{(a) Comparison of density histogram on $\mathcal{S}_u$ with the vanilla method and our method. (b) Comparison of density histogram on $\hat{\mathcal{F}}_u$ with the vanilla method, EIL, and consistency with attentive dropout. The analyzes are performed on the CUB-200-2011 test set using VGG16 as a backbone.}
    \label{fig:hist}
\end{figure}

\noindent\textbf{Consistency with Attentive Dropout.}
Fig.~\ref{fig:hist}(b) compares the effect of our consistency with attentive dropout on the distributions of $\hat{\mathcal{F}}_u$ with the vanilla method and EIL~\cite{mai2020erasing}, the state-of-the-art erasing WSOL method.
Here, the feature direction alignment with $\mathcal{L}_\text{sim}$ and $\mathcal{L}_\text{norm}$ is not applied.
With the vanilla training, most of $\hat{\mathcal{F}}_u$ are very low.
With EIL, overall $\hat{\mathcal{F}}_u$ increase compared with the vanilla method, implying that less discriminative parts become to be highly activated.
With consistency with attentive dropout, the distribution of $\hat{\mathcal{F}}_u$ shifts even more to the right.
This indirectly shows that our proposed method, consistency with attentive dropout, distributes the activation more over the target object region than the other methods. This results that the consistency with attentive dropout achieves higher performance than EIL when used along with feature direction alignment, as shown in Tab.~\ref{tab:compare_eil}. We provide a more detailed analysis in appendix.

\begin{table}[t]
  \centering
    \begin{tabular}{lccc}
    \Xhline{1pt}
    Method & Top-1  & Top-5 & GT Loc \\
    \hline
    \hline
    Align. & 62.27 & 77.48 & 81.93 \\
    EIL~[\textcolor{green}{15}] + Align. & 66.10 & 82.21 & 86.78 \\
    Attentive Dropout + Align. & \textbf{70.83} & \textbf{88.07} & \textbf{93.17} \\
    \Xhline{1pt}
    \end{tabular}%
    \vspace{-0.7em}
     \caption{Comparison of localization performance on the CUB-200-2011 dataset, based on VGG16. Align. denotes the feature direction alignment.}
  \label{tab:compare_eil}%
\end{table}%

\subsection{Ablation Study}\label{sec:ablation}
We perform a series of ablation studies on the CUB-200-2011 dataset using VGG16 as the backbone.

\noindent\textbf{Effect of Each Component.}
Tab.~\ref{tab:ablation} shows the localization performance of the classifier trained with and without each loss term.
Compared to the performance without the proposed loss terms, $\mathcal{L}_\text{drop}$ improves the Top-1 Loc by 7.4\%p and GT Loc by 14.32\%p.
The feature direction alignment using only $\mathcal{L}_\text{sim}$ improves the Top-1 Loc by 9.71\%p and GT Loc by 15.36\%p, which shows the largest improvement among the components.
Adopting $\mathcal{L}_\text{norm}$ improves all metrics more than 5\%p. The feature direction alignment using both $\mathcal{L}_\text{sim}$ and $\mathcal{L}_\text{norm}$ achieves 62.27\% of Top-1 Loc and 81.93\% of GT Loc, which is higher than the performance reported by Pan~\etal~\cite{pan2021unveiling}.
Adoption of all components shows the best performance in all metrics.

\noindent\textbf{Sensitivity to Hyperparameters.}
We analyze the effect of the balancing factors in the loss and the hyperparameters of each loss.

For the balancing factors in loss, we find the best localization performance at 0.5 for $\lambda_\text{sim}$, 0.15 for $\lambda_\text{norm}$, and 3 for $\lambda_\text{drop}$, respectively.
As shown in Fig.~\ref{fig:hyperparams}(a), the localization performance is most sensitively affected by $\lambda_\text{sim}$. $\lambda_\text{norm}$ insignificantly changes the performance.
The performance tends to decrease when the constraint with $\lambda_\text{drop}$ becomes too strong as 4.
\begin{table}[t]
  \centering
    \begin{tabular}{ccc|ccc}
    \Xhline{1pt}
    $\mathcal{L}_\text{drop}$ & $\mathcal{L}_\text{sim}$ & $\mathcal{L}_\text{norm}$ & Top-1  & Top-5 & GT Loc \\
    \hline\hline
    \textcolor{red}{\xmark} &  \textcolor{red}{\xmark} & \textcolor{red}{\xmark} &46.95 &57.23 &60.74 \\
    \textcolor{green}{\cmark}  &  \textcolor{red}{\xmark} &  \textcolor{red}{\xmark}    & 54.35 & 70.37 & 75.06 \\
    \textcolor{red}{\xmark}  &  \textcolor{green}{\cmark} &  \textcolor{red}{\xmark}    & 56.66 & 71.38 & 76.10 \\
    \textcolor{red}{\xmark} &  \textcolor{green}{\cmark}  &  \textcolor{green}{\cmark}& 62.27 &77.48 & 81.93 \\
    \textcolor{green}{\cmark}  &  \textcolor{green}{\cmark} &  \textcolor{red}{\xmark}  & 63.00 & 79.93 & 85.35\\
     \textcolor{green}{\cmark}  &  \textcolor{green}{\cmark} &  \textcolor{green}{\cmark}  & \textbf{70.83} & \textbf{88.07} & \textbf{93.17} \\
    \Xhline{1pt}
    \end{tabular}%
    \vspace{-0.7em}
    \caption{Ablations studies on the CUB-200-2011 test set, based on VGG16.}
  \label{tab:ablation}%
\end{table}%


\begin{figure}[t]
	\centering
    \includegraphics[width=0.99\columnwidth]{figures/fig_params.pdf}
    \vspace{-0.7em}
    \caption{Effect of (a) balancing factors for loss and (b) various hyperparameters.}
    \label{fig:hyperparams}
\end{figure}
For the hyperparameters of the feature direction alignment, we set $\tau_\text{fg}$ and $\tau_\text{bg}$ for $\mathcal{L}_\text{sim}$ to 0.6 and 0.1, respectively.
They determine the coarse foreground and background regions.
Fig.~\ref{fig:hyperparams}(b) shows that varying those thresholds has little effect on the performance.
The hyperparameters $\gamma$ and $p$ determine the drop of the activation in the intermediate feature map. $\gamma$ and $p$ for $\mathcal{L}_\text{drop}$ are set to 0.8 and 0.5, respectively.
When $\gamma$ is moderately large between 0.7 and 0.9, there is no significant change in the performance, but when $\gamma$ is too low, \ie, 0.6, the performance decreases.
From the results with various $p$, we observe that stochastic dropout produces little change of GT Loc regardless of the drop probability, but deterministic dropout with a probability of 1.0 yields a significant drop in the localization performance. This indicates that less but sufficient discriminative information should be maintained for a good localization performance.
\section{Conclusion}
In this paper, we find the gap between classification and localization by decomposing CAM from a new perspective. We claim that the misalignment between the feature vector at each location and class-specific weight causes CAM to be activated only in a small discriminative region.
To bridge this gap, we propose a method of aligning feature directions with class-specific weights. We also introduce a strategy to enhance the effect of feature direction alignment.
Extensive experiments demonstrate the effectiveness of the proposed method, which outperforms existing WSOL methods by a large margin.

\noindent\textbf{Limitation.}
There are several hyperparameters to decide in our method. To alleviate the search burden, we discuss a rationale for hyperparameter selection.

\bigskip
\vspace{-1pt}
\noindent\textbf{Acknowledgements:}
This work was supported by Institute of Information \& communications Technology Planning \& Evaluation (IITP) grant funded by the Korea government (MSIT) [NO.2021-0-01343, Artificial Intelligence Graduate School Program (Seoul National University)], LG AI Research, AIRS Company in Hyundai Motor and Kia through HMC/KIA-SNU AI Consortium Fund, and the BK21 FOUR program of the Education and Research Program for Future ICT Pioneers, Seoul National University in 2022.

%%%%%%%%% REFERENCES
{\small
\bibliographystyle{ieee_fullname}
\bibliography{egbib}
}


%%%%%%%%% APPENDIX
\clearpage
\noindent{\Large \textbf{Appendix}}

\setcounter{section}{0}
\renewcommand\thesection{\Alph{section}}
\setcounter{table}{0}
\renewcommand{\thetable}{A\arabic{table}}
\setcounter{figure}{0}
\renewcommand{\thefigure}{A\arabic{figure}}


\appendix

\section{Appendix}

\paragraph{Data Analysis.}
In Table~\ref{tab:dataset_comparison} we show a comparison of \DsetName~with existing moment retrieval datasets and related video and language datasets. 
Compared to other moment retrieval datasets, \DsetName~is significantly larger in scale, and comes with query type annotations that allows in-depth analyses for the models trained on it.
Besides, it is also the only moment retrieval dataset with multilingual annotations, which is vital in studying the moment retrieval problem under the multilingual context. 
Compared to the existing multilingual video and language datasets, \DsetName~is unique as it has a more diverse set of context and annotations, i.e., dialogue, query type, and timestamps.


\paragraph{Training and Inference Details.}
In Figure~\ref{fig:mxml_overview} we show an overview of the \ModelName~model.
We compute video retrieval score as:
\begin{align}
    s^{vr} = \frac{1}{2}\sum_{m \in \{v, s\}} \mathrm{max}(\frac{H^{m}_{vr}}{\left\Vert H^{m}_{vr}\right\Vert} \frac{\boldsymbol{q}^{m}}{\left\Vert \boldsymbol{q}^{m}\right\Vert}).
\end{align}
The subscript $lang \in \{en, zh\}$ is omitted for simplicity.
It is optimized using a triplet loss similar to main text Equation (1).
For moment retrieval, we first compute the query-clip similarity scores $S^{q,c} \in \mathbb{R}^{l}$ as:
\begin{align}
    S^{q,c} = \frac{1}{2}(H^{s}_{mr}\boldsymbol{q}^{s} + H^{v}_{mr}\boldsymbol{q}^{v}).
\end{align}
Next, we apply Convolutional Start-End Detector (ConvSE module)~\cite{lei2020tvr} to obtain start, end probabilities $P_{st}, P_{ed} \in \mathbb{R}^{l}$. These scores are optimized using a cross-entropy loss. The single video moment retrieval score for moment $[t_{st}, t_{ed}]$ is computed as:
\begin{align}
    s^{mr}(t_{st}, t_{ed}) = P_{st}(t_{st}) P_{ed}(t_{ed}), \, t_{st} \leq t_{ed}.
\end{align}

\noindent
Given a query $q_i$, the retrieval score for moment [$t_{st}$:$t_{ed}$] in video $v_j$ is computed following the aggregation function as in~\cite{lei2020tvr}:
\begin{align}
    s^{vcmr}&(v_j,t_{st}, t_{ed}|q_i) = \nonumber \\ &s^{mr}(t_{st}, t_{ed})\mathrm{exp}(\alpha s^{vr}(v_j|q_i)),
\end{align}


\noindent
where $\alpha{=}20$ is used to assign higher weight to the video retrieval scores.
The overall loss is a simple summation of video and moment retrieval loss across the two languages, and the language neighborhood constraint loss. 








\paragraph{Implementation Details.}
\ModelName~is implemented in PyTorch~\cite{paszke2017automatic}.
We use Adam~\cite{kingma2014adam} with initial learning rate 1e-4, $\beta_1{=}0.9$, $\beta_2{=}0.999$, L2 weight decay 0.01, learning rate warm-up over the first 5 epochs. 
We train \ModelName~for at most 100 epochs at batch size 128, with early stop based on the sum of R@1 (IoU=0.7) scores for English and Chinese.
The experiments are conducted on a NVIDIA RTX 2080Ti GPU. 
Each run takes around 7 hours.



\begin{table}[!t]
\centering
\small
\setlength{\tabcolsep}{3.5pt}
\renewcommand{\arraystretch}{1.05}
\scalebox{1.0}{
\begin{tabular}{lcccc}
\toprule
& \multicolumn{2}{c}{English R@1} & \multicolumn{2}{c}{Chinese R@1} \\  \cmidrule(l){2-3} \cmidrule(l){4-5}
Setting & IoU=0.5 & IoU=0.7 & IoU=0.5 & IoU=0.7 \\
\midrule
unseen & 1.68 & 0.79 & 1 & 0.54 \\
seen & 4.82 & 2.79 & 4.18 & 2.32 \\
\bottomrule
\end{tabular}
}
\caption{\ModelName~performance on the \DsetName~val split \textit{Friends} examples, in both \textit{unseen} and \textit{seen} settings. 
}
\label{tab:ablation_unseen}
\end{table}



\begin{figure*}[!t]
  \includegraphics[width=\linewidth]{res/mXML_overview.pdf}
  \caption{
  \ModelName~overview. For brevity, we only show the modeling process for a single language (Chinese). The cross-language modifications, i.e., parameter sharing and neighborhood constraint are illustrated in Figure~\ref{fig:tvrm_encoding}. This figure is edited from the Figure 4 in~\citep{lei2020tvr}. 
  }
  \label{fig:mxml_overview}
\end{figure*}



\begin{table*}[ht]
\centering
\small
\setlength{\tabcolsep}{5pt}
\scalebox{0.96}{
\begin{tabular}{lcccccc}
\toprule
Dataset & Domain & \#Q/\#videos & Multilingual & Dialogue & QType & Timestamp \\
\midrule
\bf QA datasets with temporal annotation &  &  &  &  &  &  \\
TVQA~\cite{Lei2018TVQALC} & TV show & 152.5K/21.8K & - & \checkmark & - & \checkmark \\
How2QA~\cite{li2020hero} & Instructional & 44K/22K & - & \checkmark & - & \checkmark \\
\bf Multilingual video description datasets &  &  &  &  &  &  \\
MSVD~\cite{chen2011collecting} & Open & 70K/2K & \checkmark & - & - & - \\
VATEX~\cite{wang2019vatex} & Activity & 826K/41.3K & \checkmark & - & - & - \\
\bf Moment retrieval datasets &  &  &  &  &  &  \\
TACoS~\cite{regneri2013grounding} & Cooking & 16.2K/0.1K & - & - & - & \checkmark \\
DiDeMo~\cite{anne2017localizing} & Flickr & 41.2K/10.6K & - & - & - & \checkmark \\
ActivityNet Captions~\cite{Krishna2017DenseCaptioningEI} & Activity & 72K/15K & - & - & - & \checkmark \\
CharadesSTA~\cite{gao2017tall} & Activity & 16.1K/6.7K & - & - & - & \checkmark \\
How2R~\cite{li2020hero} & Instructional & 51K/24K & - & \checkmark & - & \checkmark \\
TVR~\cite{lei2020tvr} & TV show & 109K/21.8K & - & \checkmark & \checkmark & \checkmark \\
\midrule
\DsetName & TV show & 218K/21.8K & \checkmark & \checkmark & \checkmark & \checkmark \\ 
\bottomrule
\end{tabular}
}
\caption{
Comparison of~\DsetName~with related video and language datasets.   
}
\label{tab:dataset_comparison}
\end{table*}

 

\paragraph{Generalization to Unseen TV shows.} 
To investigate whether the learned model can be transferred to other TV shows, we conduct an experiment by using the TV show `\textit{Friends}' as an `\textit{unseen}' TV show for testing, and train the model on all the other 5 TV shows. 
For comparison, we also include a model trained on `\textit{seen}' setting, where we use all the 6 TV shows including \textit{Friends} for training. 
To ensure the models on these two settings are trained on the same number of examples, we downsample the examples in the \textit{seen} setting to match the \textit{unseen} setting.
The results are shown in Table~\ref{tab:ablation_unseen}.
We notice our \ModelName~achieves a reasonable performance even though it does see a single example from the TV show \textit{Friends}.
Meanwhile, the gap between \textit{unseen} and \textit{seen} settings are still large, we encourage future work to further explore this direction.


\paragraph{Prediction Examples}
We show \ModelName~prediction examples in Figure~\ref{fig:pred_examples}. 
We show both Chinese (\textit{top}) and English (\textit{bottom}) prediction examples, and correct (\textit{left}) and incorrect (\textit{right}) examples.


\begin{figure*}[!t]
  \includegraphics[width=\linewidth]{res/pred_examples.pdf}
  \caption{
  Qualitative examples of \ModelName. \textit{Top:} examples in Chinese. \textit{Bottom:} examples in English. \textit{Left:} correct predictions. \textit{Right:} incorrect predictions.
  We show top-3 retrieved moments for each query. \textcolor{salmon}{salmon bar} shows the predictions, \textcolor{ForestGreen}{green box} indicates the ground truth.
  }
  \label{fig:pred_examples}
\end{figure*}


\end{document}
