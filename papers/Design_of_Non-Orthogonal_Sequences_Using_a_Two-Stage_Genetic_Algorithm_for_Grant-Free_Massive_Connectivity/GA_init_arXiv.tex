\documentclass[journal]{IEEEtran}

\ifCLASSINFOpdf
\else
   \usepackage[dvips]{graphicx}
\fi
\usepackage{url}

\hyphenation{op-tical net-works semi-conduc-tor}

\usepackage{graphicx}



\usepackage{amssymb}
\usepackage{latexsym}
\usepackage{color}
\usepackage{verbatim}
\usepackage{boldline}
\usepackage{multirow}
\usepackage{amsmath}
%\usepackage{amstex}
%\usepackage{subfig}
\usepackage{amsfonts}

\newcommand{\K}{\ensuremath{\mathbb K}}
\newcommand{\F}{\ensuremath{\mathbb F}}
\newcommand{\E}{\ensuremath{\mathbb E}}
\newcommand{\Z}{\ensuremath{\mathbb Z}}
\newcommand{\C}{\ensuremath{\mathbb C}}
\newcommand{\R}{\ensuremath{\mathbb R}}
\newcommand{\G}{\ensuremath{\mathbb G}}
\newcommand{\X}{\ensuremath{\mathbb X}}
\newcommand{\J}{\ensuremath{\mathbb J}}

\newcommand{\mA}{\mathcal{A}}
\newcommand{\mB}{\mathcal{B}}
\newcommand{\mC}{\mathcal{C}}
\newcommand{\mD}{\mathcal{D}}
\newcommand{\mE}{\mathcal{E}}
\newcommand{\mH}{\mathcal{H}}
\newcommand{\mI}{\mathcal{I}}
\newcommand{\mJ}{\mathcal{J}}
\newcommand{\mK}{\mathcal{K}}
\newcommand{\mL}{\mathcal{L}}
\newcommand{\mM}{\mathcal{M}}
\newcommand{\mN}{\mathcal{N}}
\newcommand{\mO}{\mathcal{O}}
\newcommand{\mP}{\mathcal{P}}
\newcommand{\mQ}{\mathcal{Q}}
\newcommand{\mS}{\mathcal{S}}
\newcommand{\mT}{\mathcal{T}}
\newcommand{\mU}{\mathcal{U}}
\newcommand{\mV}{\mathcal{V}}

\newcommand{\abu}{{\bf a}}
\newcommand{\bbu}{{\bf b}}
\newcommand{\cbu}{{\bf c}}
\newcommand{\dbu}{{\bf d}}
\newcommand{\ebu}{{\bf e}}
\newcommand{\fbu}{{\bf f}}
\newcommand{\hbu}{{\bf h}}
\newcommand{\kbu}{{\bf k}}
\newcommand{\mbu}{{\bf m}}
\newcommand{\nbu}{{\bf n}}
\newcommand{\sbu}{{\bf s}}
\newcommand{\pbu}{{\bf p}}
\newcommand{\rbu}{{\bf r}}
\newcommand{\ubu}{{\bf u}}
\newcommand{\vbu}{{\bf v}}
\newcommand{\gbu}{{\bf g}}
\newcommand{\xbu}{{\bf x}}
\newcommand{\ybu}{{\bf y}}
\newcommand{\wbu}{{\bf w}}
\newcommand{\zbu}{{\bf z}}

\newcommand{\Abu}{{\bf A}}
\newcommand{\Bbu}{{\bf B}}
\newcommand{\Cbu}{{\bf C}}
\newcommand{\Dbu}{{\bf D}}
\newcommand{\Hbu}{{\bf H}}
\newcommand{\Fbu}{{\bf F}}
\newcommand{\Ibu}{{\bf I}}
\newcommand{\Jbu}{{\bf J}}
\newcommand{\Kbu}{{\bf K}}
\newcommand{\Mbu}{{\bf M}}
\newcommand{\Nbu}{{\bf N}}
\newcommand{\Pbu}{{\bf P}}
\newcommand{\Rbu}{{\bf R}}
\newcommand{\Sbu}{{\bf S}}
\newcommand{\Gbu}{{\bf G}}
\newcommand{\Tbu}{{\bf T}}
\newcommand{\Ubu}{{\bf U}}
\newcommand{\Vbu}{{\bf V}}
\newcommand{\Wbu}{{\bf W}}
\newcommand{\Xbu}{{\bf X}}
\newcommand{\Ybu}{{\bf Y}}
\newcommand{\Zbu}{{\bf Z}}
\newcommand{\Phibu}{{\bf \Phi}}
\newcommand{\bGamma}{{\bf \Gamma}}
\newcommand{\phibu}{{\boldsymbol \phi}}
\newcommand{\Psibu}{{\bf \Psi}}
\newcommand{\balp}{{\boldsymbol \alpha}}
\newcommand{\bsigma}{{\boldsymbol \sigma}}
\newcommand{\btheta}{{\boldsymbol \theta}}
\newcommand{\bSigma}{{\boldsymbol \Sigma}}
\newcommand{\bLambda}{{\boldsymbol \Lambda}}
\newcommand{\iproof}{{\noindent \textit{Proof}}}

\newcommand{\qed}{\hfill \ensuremath{\Box}}

\newtheorem{fact}{Fact}
\newtheorem{prop}{Proposition}
\newtheorem{df}{Definition}
\newtheorem{thr}{Theorem}
\newtheorem{lem}{Lemma}
\newtheorem{rem}{Remark}
\newtheorem{cor}{Corollary}
\newtheorem{const}{Construction}
\newtheorem{exa}{Example}
\newtheorem{crt}{Criterion}
\newtheorem{prot}{Property}
\newtheorem{assum}{Assumption}
\newtheorem{const2}{Construction}
\numberwithin{const2}{const}


\begin{document}

\title{Design of Non-Orthogonal Sequences Using a Two-Stage Genetic Algorithm for Grant-Free Massive Connectivity}

\author{Nam Yul Yu, \IEEEmembership{Senior Member,~IEEE}
%\thanks{%Manuscript received June 30, 2013; revised January 15, 2014; accepted March 2, 2014.
%This work was supported by the National Research Foundation of Korea (NRF) grant funded by the Korea Government (MSIT)
%(NRF-2017R1A2B4004405), and by the GIST Research Institute (GRI) grant.}
%\thanks{Copyright (c) 2017 IEEE. Personal use of this material is permitted.
%	However, permission to use this material for any other purposes
%	must be obtained from the IEEE by sending a request to pubs-permissions@ieee.org.
\thanks{This work has been submitted to the IEEE for possible publication. 
	Copyright may be transferred without notice, after which this version may no longer be accessible.
	This work was supported by the National Research Foundation of Korea (NRF) grant funded by the Korea Government (MSIT)
	(NRF-2021R1F1A1046282).}
\thanks{The author is with the School of
of Electrical Engineering and Computer Science (EECS), Gwangju Institute of Science and Technology (GIST), Korea.
(e-mail: nyyu@gist.ac.kr).}
}  % <-this % stops a space

%\markboth{Journal of \LaTeX\ Class Files, Vol. 14, No. 8, August 2015}
%{Shell \MakeLowercase{\textit{et al.}}: Bare Demo of IEEEtran.cls for IEEE Journals}
\maketitle

\begin{abstract}
In massive machine-type communications (mMTC),
grant-free access is a key enabler for a massive number of users
to be connected to a base station with low signaling overhead and low latency.
In this paper, a two-stage genetic algorithm (GA) is proposed 
to design a new set of user-specific, non-orthogonal, unimodular sequences
for uplink grant-free access.
The first-stage GA is
to find a subsampling index set for a partial unitary matrix 
that can be approximated to an equiangular tight frame.
Then in the second-stage GA, we try to find a sequence to be masked to each column
of the partial unitary matrix, 
in order to reduce the peak-to-average power ratio of the resulting columns
for multicarrier transmission.
Finally, the masked columns of the matrix are proposed as new non-orthogonal sequences  
for uplink grant-free access.
Simulation results 
demonstrate that the non-orthogonal sequences designed by our two-stage GA
exhibit excellent performance 
for compressed sensing based joint activity detection and channel estimation
in uplink grant-free access.
Compared to algebraic design,
this GA-based design 
can present a set of good non-orthogonal sequences of arbitrary length, 
which provides more flexibility for uplink grant-free access in mMTC.
\end{abstract}

\begin{IEEEkeywords}
Compressed sensing, genetic algorithm, grant-free access, machine-type communications, non-orthogonal multiple access, 
peak-to-average power ratio. 
\end{IEEEkeywords}


\IEEEpeerreviewmaketitle



\section{Introduction}


\IEEEPARstart{M}{assive} connectivity of wireless devices
is essential for industrial, commercial, and critical applications
of massive machine-type communications (mMTC)~\cite{Kunz:MTC, Bockel:mMTC},
which provides a concrete platform for the Internet of Things (IoT).
%by enabling a massive connection of devices with little or no human intervention~\cite{Kunz:MTC}.
Unlike human-type communications (HTC), 
mMTC is characterized by small size data, infrequent transmission, low cost devices,
low mobility, and so on~\cite{3gpp:22368}.
%Through a grant-free access,
In practice,
mMTC systems need to support a massive number of devices with low control overhead, low latency,
and low power consumption for delay-sensitive and energy efficient communications.


Non-orthogonal multiple access (NOMA)~\cite{Dai:noma, Dai:survey} has received a great deal of attention 
for massive connectivity in 5G wireless systems.
In code-domain NOMA, user-specific and
non-orthogonal spreading sequences
are assigned to users for their non-orthogonal multiplexing through common resources. 
%where active ones can reduce signaling overhead via uplink grant-free access. % in uplink MTC.
%There have been many research efforts for enabling massive connectivity with low complexity in code-domain NOMA. 
In sparse code multiple access (SCMA)~\cite{Baligh:SCMA}, 
sparse spreading sequences are assigned to users, % for spectrally efficient communications,
%is a
%well known research work for non-orthogonal multiple access of a massive number of devices.
%In SCMA, sparse spreading codes are assigned to devices,
where a message passing algorithm (MPA)~\cite{Zhang:MPA} and a list sphere decoding based MPA decoder~\cite{Wei:SCMA}
can be deployed for reliable multiuser detection with low complexity.  
Complex-valued spreading sequences are employed for 
multi-user shared access (MUSA)~\cite{Yuan:MUSA}, 
%as in a non-orthogonal, code-division multiple access (CDMA) style,
where the successive interference cancellation (SIC) can be performed for multiuser detection. 
Also, pattern division multiple access (PDMA)~\cite{Kang:PDMA} attempted
to enable massive connectivity with low complexity through an efficient pattern matrix design 
and a recursive approach of multiuser detection~\cite{Jamali:PDMA}.
For a survey on existing works of code-domain NOMA, readers are referred to~\cite{Dai:survey}.
Recently, the state-of-the-art technique of deep learning~\cite{Bengjo:DL} has been applied
for multiuser detection in uplink code-domain NOMA systems~\cite{Ye:deep}$-$\cite{Siva:dnn}.



%For fast and efficient connectivity,
Grant-free access is of tremendous interest to connect a massive number of users to mMTC systems
with low latency and low signaling overhead~\cite{Cirik:toward}.
In uplink grant-free access, 
%we assume that
active users send their data with no access-grant procedure.
%where the transmitted signals are spread onto multiple subcarriers by their own spreading sequences
%and superimposed at a base station (BS) receiver.
Then, a base station (BS) receiver has to identify active users with no aid of a grant procedure
and detect each active user's data from the superimposed signal.
The principle of 
compressed sensing (CS)~\cite{Eldar:CS} can be applied for multiuser detection in uplink grant-free access,  %~\cite{Du:joint}.,
exploiting the sparse activity that many users are present in a cell, % to be accommodated by a BS, 
but only a few of them are active at a time.
Many research articles~\cite{Abebe:iter}$-$\cite{Schober:meet} demonstrated that 
a CS-based detector can be successfully deployed at BS  
% exploiting the \emph{sparse} activity,
for joint activity detection, channel estimation, and/or data detection in uplink grant-free access.


%In uplink grant-free access, 
For non-orthogonal and grant-free access, %massive connectivity, %  through grant-free access, 
it is crucial to design a set of non-orthogonal sequences with low correlation,
% $M \times N$ sensing matrix $\Abu$,
constructively or algorithmically, which %in order to
ultimately guarantees reliable % performance of
CS-based detection at BS.
%joint channel estimation (CE)
%and multiuser detection (MUD).
Moreover, if the transmitted signals of active users are spread % of devices 
onto multiple subcarriers,
the high peak-to-average power ratio (PAPR) 
will cause signal distortion
%due to the non-linearity of power amplifiers~\cite{Litsyn:peak},
%which 
deteriorating all potential benefits of multicarrier communications~\cite{Litsyn:peak, No:papr}.
%Indeed, 
%the PAPR of transmitted signals has been an important research topic in uplink multicarrier transmission and 
Various reduction techniques~\cite{No:papr} have been proposed for mitigating the PAPR
of multicarrier transmitted signals. % in multicarrier transmission.
%Therefore, we have another requirement of PAPR reduction in spreading sequence design
%for uplink grant-free NOMA.
Recently, efforts have been made to reduce the PAPR of uplink multicarrier signals in SCMA~\cite{Yani:ra, Mherich:golden}.
In summary, we need to design a set of
non-orthogonal sequences with low correlation and low PAPR properties,
which ensures reliable and power efficient uplink grant-free access. % for massive connectivity. 
%reliable CS-based detection for BS receiver
%and exhibit low PAPR for multicarrier transmission at mMTC devices.
%each spreading sequence should have low PAPR for multicarrier transmission.
%in addition to low coherence of the spreading matrix.
%To sum up, % we believe that 
%spreading sequences with theoretically bounded low coherence and low PAPR 
%are essential for the success of
%will guarantee reliable performance of 
%CS-based joint CE and MUD in uplink grant-free NOMA. 
%of uplink grant-free NOMA.


In literature,
many constructive designs have been presented 
%many researchers have studied 
to provide a variety of pilot or spreading sequences for % non-orthogonal 
multiple access.
%Based on algebraic codes, 
In~\cite{Yang:qos}, quasi-orthogonal sequences have been introduced
to increase the system capacity of CDMA.
%to design sequences with low PAPR that result in a spreading matrix with low coherence.
%Taking each element from 
%the Gaussian distribution, 
Random sequences with the Gaussian distributed elements have been used in~\cite{Liu:mimo}$-$\cite{Jiang:noma}
%as pilot or spreading sequences
to theoretically guarantee reliable CS-based detection %joint CE and MUD 
for uplink access.
Also, the works of~\cite{Wang:struct}$-$\cite{Du:block} %$-$\cite{Du:efficient} 
used pseudo-random noise sequences 
for CS-based detection in uplink grant-free NOMA.
%However, random sequences may not be suitable for uplink NOMA, 
%since they generally have high PAPR in multicarrier transmission.
%through coding techniques and sequence design.
% multicarrier signals.
%Although these sequences %turned out to provide outstanding performance of CE and/or MUD 
%allow multiple access with low interference, 
%their PAPR properties 
%have not been discussed rigorously in the articles for multicarrier transmission.
%To obtain sequences with 
In multicarrier communications,
Golay complementary sequences and sets~\cite{Golay:series}$-$\cite{Paterson:gen} 
can be employed to provide theoretically bounded low PAPR.
%have attracted much attention 
%identified as a second-order coset of the first-order Reed-Muller (RM) code,
%have been attracted %for a coding scheme that
%Golay complementary sets~\cite{Paterson:gen}
In~\cite{Liu:FBMC}, Golay complementary sequences have been also applied for 
low PAPR preambles in the filter-bank multicarrier (FBMC) modulation.
Binary~\cite{Yu:binary} and non-binary~\cite{Yu:non} Golay spreading sequences have been employed for low PAPR in uplink grant-free NOMA.
Other complementary sequences have been studied in~\cite{Liu:comp}$-$\cite{Wu:Z}
for PAPR reduction. % in multicarrier transmission.
%Although they have desirable 
%PAPR properties, % for multicarrier transmission, 
%Golay and other complementary sequences have never been investigated
% CS-based joint CE and MUD 
%in uplink grant-free NOMA, 
%to connect a massive number of devices in MTC.
%support massive connectivity.
Zadoff-Chu (ZC) sequences~\cite{Chu:ZC}, also known as constant amplitude and zero autocorrelation (CAZAC) sequences,
have been adopted as preambles for random access
in 3GPP-LTE~\cite{3gpp:36.211},
providing low PAPR for multicarrier transmission.
%Recently, sparse spreading sequences are studied for low-density spreading aided OFDM and 
%sparse code multiple access (SCMA)~\cite{Dai:survey}.
%ZC sequences have been also studied in~\cite{Ding:anal}
%for random access with massive MIMO.
%However, the number of phases of ZC sequences grows as the sequence length increases,
%which may require higher implementation complexity than sequences with limited phases.
%Therefore, it is worth studying 
%\emph{binary} spreading sequences offering a
%low implementation cost for MTC devices, as well as providing low PAPR and 
%reliable CS-based joint CE and MUD for uplink grant-free NOMA.


Noting that a sensing matrix of CS is a collection of non-orthogonal column sequences,
%Besides the constructive approaches, 
we can find many algorithmic approaches %have been presented
for good sequences from the efforts of optimizing the sensing matrix.
%for design of good sensing matrices.
Elad~\cite{Elad:opt} launched an algorithmic design for a sensing matrix
by minimizing the average measure of the coherence iteratively.
In~\cite{Xu:opt}$-$\cite{Zhang:opt}, several algorithms have been proposed for
optimizing a sensing matrix, %$\Phibu$,
where each one attempts to approximate % the off-diagonal entries of 
its Gram matrix to that of an equiangular tight frame (ETF)~\cite{Kov:frames}. 
%the Welch bound equality~\cite{Welch:low}. 
In~\cite{Chen:utf}, Chen~\emph{et al.} demonstrated that
a unit-norm tight frame is a closest design of a nearly orthogonal matrix. % that is nearly orthogonal.
%$\Abu$ to minimize the Frobenius norm 
%of $\Abu^T\Abu - \Ibu$.
%Li \emph{at al.}~\cite{Li:opt} generalized from the result of \cite{Chen:utf}
%and proposed an iterative algorithm to optimize a sensing matrix.
%%by characterizing the best sensing matrix .
Other algorithms can be found in~\cite{Duarte:learn} and \cite{Lu:dir}. 
From these efforts, each sensing matrix optimized algorithmically can offer a set of 
non-orthogonal sequences % with its columns 
for reliable CS-based detection.
In~\cite{Chun:DL} and \cite{Kim:DL}, 
deep learning (DL) techniques have been also applied for pilot or spreading sequence design. 
In general, the non-orthogonal sequences obtained by algorithmic and DL-based designs 
can take arbitrary elements with no structure,
which may not be suitable for cost efficient implementation in mMTC devices.
% e.g., communications and wireless sensor networks (WSN).


Recently, the genetic algorithm (GA)~\cite{Holland:gen} has been applied 
for sensing matrix optimization in specific applications,
e.g., reducing the complexity of %specific applications. % in CS-based techniques.
radar imaging~\cite{Chen:ISAR}, allocating an optimized pilot pattern for channel estimation~\cite{Nie:maga}, 
and maximizing the energy efficiency of wireless sensor networks (WSN)~\cite{Leven:moga}.
%In~\cite{Chen:ISAR}, GA has been used to optimize a measurement matrix for radar imaging.
%Similarly, the modified adaptive GA has been proposed in~\cite{Nie:maga}
%for optimizing pilot allocation in sparse channel estimation.
%The multiple objective genetic algorithm 
%has been used to optimize a CS matrix for maximizing energy efficiency
%and minimizing reconstruction errors in wireless sensor networks (WSN)~\cite{Leven:moga}.
In particular, 
GA has been used
to find subsampling patterns to optimize partial Fourier matrices
with specific parameters~\cite{Chen:ISAR, Nie:maga}.
%which is also equivalent to finding a partial Fourier matrix minimizing the coherence.
%However, the above results are only about partial Fourier matrices
%with specific parameters for particular applications. 
%However, a more rigorous study is necessary
%for GA-based optimization of sensing matrices, 
This GA-based optimization motivates us to scrutinize the effectiveness of GA % with scrutiny
for optimizing a sensing matrix, which can ultimately present a set of 
non-orthogonal sequences for uplink grant-free access.
%In the end, we will be able to obtain a set of good spreading sequences
%from the optimized CS matrix.
%optimizing the subsampling patterns. %partial unitary matrices.








In this paper, 
we propose a two-stage genetic algorithm (GA) to design a new set of non-orthogonal sequences\footnote{
The resulting sequences from our design can be used as
spreading, pilot, or signature sequences, depending on specific access schemes.} 
%for guaranteeing reliable performance for CS-based detection 
for uplink grant-free access, %massive connectivity,
%with optimized subsampling, 
%where each element is restricted to a unimodular and complex-valued one of finite phase for
where each sequence has unimodular and complex-valued elements of finite phase for
cost efficient implementation in an mMTC device.
Each stage of GA makes an evolutionary approach to reach an optimized result by transforming
and improving the intermediate outcomes. 
The first-stage GA
is to find a subsampling index set to optimize\footnote{In this paper, 
`\emph{optimize}' does not mean to find a global optimum, since GA may converge to local optima.} 
a partial unitary matrix by
approximating it to an ETF,
where the evolutionary approach tries to minimize  
%reduce % minimize %the cost function of 
the average distance between the inner product of its column pair and the %of the partial unitary matrix and the 
Welch bound equality~\cite{Welch:low}.
%in an effort to approximate the optimized matrix to ETF.
%The cost function of GA is defined by the Frobenius norm of the upper triangular part of $\Abu^H \Abu - \Wbu $
%above the diagonal,
%where the off-diagonal entries of $\Wbu$ are the Welch bound equality (WBE).
Then, the second-stage GA tries to find a sequence to be commonly masked 
to each column of the partial unitary matrix from the first-stage,
in order to reduce the PAPR of the resulting columns.
%in an effort 
%reducing the top $30 \%$ average of PAPR of the resulting columns.
Note that masking each column with a common sequence does not change
the inner products and their distribution among the resulting column pairs.
Finally, the masked columns of the partial unitary matrix
are proposed as new non-orthogonal sequences with low correlation and low PAPR properties,
which can be uniquely assigned to users  
for uplink grant-free access.

Through simulations, the phase transition diagrams
%for several partial unitary matrices, %  with optimized subsampling, which 
reveal that the
partial Fourier and ZC-based matrices optimized by our first-stage GA guarantee more reliable CS reconstruction
than the randomly subsampled counterparts, respectively, % unitary matrices
over a wide range of compression and sparsity ratios.
%The phase transitions also show that the partial unitary matrices %obtained by our GA 
%are superior to those of GA attempting to minimize the mutual coherence of $\Abu$.
%As a result, %we demonstrate that %The phase transitions demonstrate that 
%non-orthogonal sequences from the partial unitary matrices
%can guarantee reliable CS-based detection, theoretically and numerically, for uplink grant-free access.
In addition, it turns out that the second-stage GA is effective to enhance
the PAPR properties of the resulting sequences, where % enhanced by the second-stage GA.
%The maximum PAPR of the sequences is not so low as that of
%ZC sequences of similar length, but 
the PAPR distributions appear to be acceptable for multicarrier transmission.
%In uplink grant-free access, 
In uplink grant-free access,
we show that the performance of the Fourier- and ZC-based sequences from our two-stage GA
%the performance of the proposed sequences 
is superior to that of random sequences, while comparable to that of ZC sequences of prime length,
for CS-based joint activity detection and channel estimation
Compared to algebraic design, we confirm that this GA-based design can present a new set of
non-orthogonal sequences of arbitrary length,
exhibiting acceptable PAPR distribution and 
guaranteeing reliable CS-based detection,
which can be more suitable for %uplink grant-free access in practical mMTC.
grant-free massive connectivity.

This paper is organized as follows.
Section II describes a system model of uplink grant-free access under consideration,
where a CS problem is formulated for joint activity detection and channel estimation.
Section III outlines a framework for non-orthogonal sequence design
using a two-stage GA. In each stage, we formulate the design goal by an optimization problem.
Section IV describes the evolution steps of each stage GA 
along with the cost function for the optimization problem.
Algorithms 1 and 2 summarize the two-stage GA.
Section V presents simulation results to demonstrate the effectiveness of each stage GA.
In addition, we evaluate the performance of the proposed sequences, comparing to other conventional ones,
in CS-based joint activity detection and channel estimation. 
Finally, concluding remarks will be given in Section IV.

\emph{Notations}: 
Throughout this paper, $\Z_N = \{0, \cdots, N-1\}$.
A matrix (or a vector) is represented by
a bold-face upper (or a lower) case letter.
$\Xbu^T$ denotes the transpose of a matrix $\Xbu$, 
while 
$\Xbu^*$ is its conjugate transpose. % of a matrix $\Xbu$.
The identity matrix is denoted by $\Ibu$, where the dimension is determined in the context.
%For a matrix $\Xbu = \{X_{i,j}\}$,
%${\rm abs} (\Xbu)$ takes the magnitude of each element of $\Xbu$, i.e.,
$ {\rm abs} (\Xbu) = [ | X_{i, j} | ]$
denotes a matrix taking the magnitude of each element of $\Xbu =  [X_{i,j}]$.
% denotes a matrix of % where each element is the magnitude of the respective one of $\Xbu$.
For a vector $\hbu $, 
%$\xbu^T$ denotes its transpose and
$\hbu_{\mS}$ is its subvector, indexed by an index set $\mS$, and
${\rm diag} (\hbu)$ is a diagonal matrix whose diagonal entries are from $\hbu$.
The inner product of vectors $\xbu$ and $\ybu$ is denoted by $\langle \xbu, \ybu \rangle$.
%The $l_p$-norm of a vector $\xbu  = (x_1, \cdots, x_N)$ is denoted by
%$ || \xbu ||_{p} = \left( \sum_{k=1} ^{N} |x_k|^p \right) ^{\frac{1}{p}} $ for
%$1 \leq p < \infty $. % If the context is clear, $|| \xbu ||$ denotes the $l_2$-norm of $\xbu$.
%$|| \xbu||_0  $ is the number of nonzero elements of $\xbu$.
The $l_2$-norm of a vector $\xbu  = (x_1, \cdots, x_N)$ is denoted by
$ || \xbu ||_{2} = \sqrt{ \sum_{k=1} ^{N} |x_k|^2 } $.
The Frobenius norm of a matrix $\Xbu = [X_{i,j}]$ is denoted by 
$\| \Xbu \|_F = \sqrt{\sum_{i, j} \left| X_{i,j} \right| ^2 }$.
Finally, $\hbu \sim \mathcal{CN} (\mathbf{m}, \mathbf{\Sigma})$ is a circularly symmetric complex Gaussian random vector
with mean $\bf m$ and covariance $\mathbf{\Sigma}$.
%$x \sim \mathcal{CN} (\mu, \sigma^2)$ is a circularly symmetric complex Gaussian random variable
%with mean $\mu$ and variance $\sigma^2$.




\section{System Model}
%In uplink grant-free NOMA, % this paper assumes 
%For uplink grant-free access,
In this paper, we consider a two-phase grant-free access scheme~\cite{Liu:mimo, Liu:massive}
for a single-cell massive connectivity. 
In an mMTC cell, a base station (BS) receiver equipped with $J$ antennas
accommodates total $N$ devices each of which transmits with a single antenna.
For a fully grant-free access, % of many devices,
we assume that devices are \emph{static}\footnote{
	If devices are moving from cell to cell, 
	it is hard to guarantee unique sequence assignment in a fully grant-free manner
	for all devices in a cell, and 
	some coordination may be required to assign unique sequences to devices.}
in a cell and thus BS accommodates a fixed set of devices
having their own user-specific sequences. 
In the first phase, each active device transmits its sequence as a dedicated pilot,
and the BS receiver then tries to identify active devices and estimate their channel profiles 
from the superimposed pilots.
Data can be directly transmitted in the second-phase from active devices with no grant from BS.
In this two-phase scheme, 
we assume that the channels and the device activity remain unchanged during $L$ slots for pilot 
and data transmissions.
%We also assume that devices are synchronized in a frame structure
%of $J$ time slots, where the activity of each device remains unchanged during an entire frame~\cite{Abebe:iter, Wang:struct}.
%only a few devices out of $N$ ones are assumed to be active in a frame.
%Each active device sends a pilot symbol for channel estimation
%in the first slot of a frame, and 
%subsequently sends data symbols for the following $J-1$ slots~\cite{Du:joint, Ahn:ep}.
Figure~\ref{fig:system} illustrates this system model.
%For an illustration of this system model, refer to Figure 1 of~\cite{Du:joint}.

\begin{figure}
	\centering
	\includegraphics[width=0.48\textwidth, angle=0]{system.eps}
	\caption{Two-phase grant-free access scheme with multiple receiver antennas.}
	\label{fig:system}
\end{figure}




%At a time slot $t$ of a frame,
%While a massive number of devices are present in a cell,
%For massive connectivity, we assume that $N$ is large, but
%only a few of them are active at a time. % , where the number of active devices is far less than $N$.
With sparse activity, 
each device is assumed to be active with probability $p_a$ in an i.i.d. manner,
where active devices are synchronized.
%The device activity remains unchanged over $L$ slot duration, 
In an access time, 
an activity indicator vector can be defined by $\balp = (\alpha_1, \cdots, \alpha_N)^T$ with
\begin{equation*}\label{eq:act}
	\alpha_n = \left\{ \begin{array}{ll} 1, & \mbox{ if device } n \mbox{ is active}, \\
	0, & \mbox{ otherwise}, \end{array} \right.
\end{equation*}
%From $\alpha_n$, 
where $\mS = \{ n \mid \alpha_n = 1 \}$ is a set of active devices and
the number of active devices is $|\mS| = \sum_{n=1} ^N \alpha_n = K \ll N$.

When device $n$ is active, 
it transmits its unique pilot
sequence $\sbu_n = (s_{1,n}, \cdots, s_{M,n})^T $ % as a dedicated pilot
over $M$ subcarriers for grant-free access, where $M < N$.
We consider a flat Rayleigh fading channel, 
where the channel gain remains unchanged during the coherence time interval of $L$ slots.
Let $\hbu_n = \left(h_n ^{(1)}, \cdots, h_n ^{(J)} \right)^T$, $ 1 \leq n \leq N$,
be a channel vector from device $n$, 
where $h_n ^{(t)}$ is the channel gain between device $n$ and BS receiver antenna $t$.
Assuming that the path loss and shadowing effects are known and can be removed by BS,
we have $\hbu_n \sim \mathcal{CN}(\bf{0}, \Ibu) $. 
%Also, the transmit signal power from each active device is assumed to be
%identical, due to perfect power control.
Then, the received signal at antenna $t$ can be represented % in frequency-domain 
by
\begin{equation}\label{eq:y}
	\ybu^{(t)} = \sum_{n=1} ^N \alpha_n  h_n ^{(t)} \sbu_n  + \wbu ^{(t)}
	=  \Sbu \xbu ^{(t)} + \wbu ^{(t)}, %\quad t = 1, \cdots, J,
	%= \Gbu^{(t)} \xbu^{(t)} + \wbu ^{(t)},
\end{equation}
where $\xbu ^{(t)} = \left(\alpha_1 h_1 ^{(t)}, \cdots, \alpha_N h_N ^{(t)} \right)^T$
for $ 1 \leq t \leq J$. 
In~\eqref{eq:y},
$\Sbu = [\sbu_1, \cdots, \sbu_N] \in \C^{M \times N} $ is a matrix of pilot sequences,
%$\hbu = (h_1, \cdots, h_N)^T \sim \mathcal{CN}(\bf{0}, \Ibu) $, 
%where $h_n$ is the channel gain between device $n$ and BS, and
and $\wbu^{(t)} \sim \mathcal{CN}(\textbf{0}, \sigma_n ^2 \Ibu) $ is the complex Gaussian noise vector
at antenna $t$. 


Collecting the $J$ received signals of~\eqref{eq:y}, 
we have a multiple measurement vector (MMV) % joint sparse 
model of
\begin{equation}\label{eq:mmv}
\Ybu = \Sbu \Xbu + \Wbu,
\end{equation}
where $\Ybu = \left[\ybu^{(1)}, \cdots, \ybu^{(J)} \right]$,
$\Xbu = \left[\xbu^{(1)}, \cdots, \xbu^{(J)} \right]$,
and $\Wbu = \left[\wbu^{(1)}, \cdots, \wbu^{(J)} \right]$, respectively.
Due to the activity indicator $\balp$, 
it is clear that $\Xbu$ has the row-wise sparsity with $K$ nonzero 
and $N-K$ zero rows.
%Note that the BS receiver attempts to detect the activity indicator $\balp$
%and estimate the channel vectors $\hbu_n$ for $n \in \mS$ 
Then, BS can apply
a joint sparse recovery algorithm to solve the MMV problem of~\eqref{eq:mmv},
in order to detect the activity indicator $\balp$ and estimate the channel vector $\hbu_n$ for $n \in \mS$.
%obtain $\widehat{\Xbu} ^{\widehat{\mS}}$
%by solving the MMV problem, 
%where $\widehat{\mS}$ is an estimate of $\mS$ and % set of active devices.
%$\widehat{\Xbu} ^{\widehat{\mS}}$ is an estimate of $\Xbu^\mS$, respectively.
If the nonzero rows of $\Xbu$ are estimated,
the row indices mean a detected index set of active devices, denoted by $\widehat{\mS}$, 
while the coefficients of each nonzero row give %$\widehat{\hbu}_n$,
an estimated channel vector $\widehat{\hbu}_n$ for $n \in \widehat{\mS}$. % between the active device and BS, where $n \in \widehat{\mS}$.
The CS-based joint active user detection (AUD) and channel estimation (CE)
complete the first phase of uplink grant-free access.
In the second phase, the BS receiver detects data from active devices
with the knowledge of device identity and channel profiles obtained from the first phase~\cite{Liu:mimo, Liu:massive}.
In this paper, we restrict our attention to joint AUD and CE in the first phase
via joint sparse recovery under the CS MMV model.

\begin{rem}\label{rm:one}
A CS MMV model can also be applied for \emph{one-shot} detection
in uplink grant-free NOMA~\cite{Du:joint, Yu:binary}.
%Similar to Fig.~\ref{fig:system}, 
In this system, each active device transmits its unique spreading sequence of length $M$,
spread onto $M$ subcarriers,  %  in uplink grant-free NOMA,
carrying its pilot and data over $J$ time slots.
Assuming that the channels and the device activity remain unchanged, % over $J$ slots,
the received signals over $J$ slots are also modeled by \eqref{eq:mmv}.
A BS receiver equipped with a single antenna
then conducts CS-based joint activity detection, channel estimation, and data detection,
by solving the MMV problem of \eqref{eq:mmv}.
Readers are referred to~\cite{Du:joint} and \cite{Yu:binary} for more details.
\end{rem}



\section{Framework for Sequence Design}
The goal of this paper is to present a set of non-orthogonal sequences
for grant-free massive connectivity.
In CS MMV model, the sequence set forms the matrix $\Sbu$ in \eqref{eq:mmv}, where
the problem of sequence design
boils down to designing a sensing matrix for reliable CS-based detection.
This section outlines a framework for sensing matrix design using the genetic algorithm (GA),
which ultimately provides a set of good non-orthogonal sequences for uplink grant-free access.

\subsection{Partial Unitary Matrices}

Compressed sensing (CS)~\cite{Eldar:CS} %\cite{Donoho:CS}$-$\cite{CanTao:univ}
is to reconstruct an $N$-dimensional sparse signal $\xbu$
from its underdetermined $M$-dimensional measurement $\ybu = \Abu \xbu$, 
where $M < N$.
The signal $\xbu $ is called \emph{$K$-sparse}
if it has at most $K$ nonzero elements,
%\footnote{Under the system model of this paper, it suffices to assume that a signal $\xbu$
%is sparse in the identity basis.},
where $K \ll N$. 
In CS techniques, it is essential to design a good $M \times N$ sensing matrix $\Abu$,
%constructively or algorithmically, 
in order to guarantee reliable reconstruction of sparse signals.
%In CS theory, it is well known that 
%if the sensing matrix $\Abu $ satisfies the \emph{restricted isometry property (RIP)}~\cite{Eldar:CS},
%stable and robust reconstruction of sparse signals can be guaranteed.
%In this letter, we assume that the sparsifying basis is identity, or $\Psibu = \Ibu$,
%which results in $\Abu = \Phibu$.

%When $\Abu$ is a purely random matrix with Gaussian distributed entries, %$\mathbf{\Gamma}=\mathbf{\Phi}\Psibu^H$
%it obeys the RIP with high probability, as long as $M\geq \mO (K \log (N/K))$~\cite{Eldar:CS, Baraniuk:rip}.
%%However, this matrix is computationally expensive, and
%In practice, however, a \emph{structured} random matrix can be more desirable
%than the computationally expensive random ones.
%%for the benefits of low complexity and simple implementation.
%A random or structural subsampling of unitary matrices~\cite{Duarte:struct}$-$\cite{Zhang:utf} 
%is a well known operation to achieve \emph{partial} unitary sensing matrices,
%which presents a theoretical recovery guarantee and allows fast and efficient processing.
%%A more general framework of modulated unit-norm tight frames
%%has been presented in~\cite{Zhang:utf}.
%A fully deterministic matrix, e.g., \cite{DeVore:det}$-$\cite{Yu:OOC},
%can also be employed for fast and efficient CS processing. 
%Deterministic construction presents a coherence-based recovery guarantee~\cite{Donoho:opt}, 
%which is however weaker than the RIP-based guarantee.
%Moreover, the available dimensions of deterministic matrices are limited. 
%A \emph{structured} random matrix is desirable for CS in practice,
%thanks to the computational efficiency.
%for the benefits of low complexity and simple implementation.
%By subsampling some rows of a unitary matrix,
Taking some rows out of a unitary matrix
%resulting in a \emph{partial} unitary matrix~\cite{Duarte:struct}, %$-$\cite{Zhang:utf},
%is a well known structured random matrix, obtained by taking some rows out of a unitary matrix,
is a well known operation to obtain a \emph{partial} unitary matrix~\cite{Duarte:struct} % a good sensing matrix 
that %\emph{structured} random matrices,
enjoys practical benefits as well as theoretical CS recovery guarantee.
A partial unitary matrix is formulated by
\begin{equation}\label{eq:S-single}
	\Abu  =\frac{1}{\sqrt{M}} \mathbf{R}_{\Omega}\Ubu \triangleq \Ubu_\Omega,
\end{equation}
%is an example of a structured random matrix,
where $\Ubu$ is an $N \times N$ unitary matrix of $\Ubu \Ubu^* = \Ubu^* \Ubu = N \Ibu$.
In~\eqref{eq:S-single}, $\mathbf{R}_\Omega$ is a subsampling operator selecting 
$M$ rows out of $N$ ones whose indices are specified by $\Omega \subset \{1, \cdots, N\}$, where $|\Omega| = M$. 
%Such measurement operators can be easily implemented by convolving $\mathbf{x}$ with a filter $\abu$
%followed by subsampling $\Rbu_\Omega$.
If the indices of $\Omega$ are selected randomly, %  subsampling operator, 
%it is known that 
$\Abu = \Ubu_\Omega$ %obeys the RIP with high probability,
guarantees reliable CS reconstruction % of $\xbu$ %from $\ybu$ 
theoretically with high probability,
provided that $M \geq \mO ( K \log^4 N)$~\cite{Rud:sparse}.
In practice, a partial unitary matrix allows fast and efficient measurement and reconstruction for CS,
thanks to the fast unitary transform, e.g., fast Fourier or Hadamard transform.
%provided that $M \geq \mO (\mu^2(\Ubu) K \log^4 N)$~\cite{Rud:sparse},
%where $\mu(\Ubu)$ denotes the maximum magnitude of the entries of $\Ubu$.
%Along with the theoretical recovery guarantee, 
%The partial unitary matrices are promising in several applications such as
%channel estimation, Fourier optics, coded aperture imaging
%and radar~\cite{Romb:rand}.

To design unimodular sequences, 
we begin with a unitary matrix $\Ubu$ whose elements take the magnitude of $1$.
%Given a unitary matrix $\Ubu$, 
Then, the first stage of sequence design 
attempts to find a subsampling index set $\Omega$
to optimize a partial unitary matrix $\Ubu_\Omega$ for reliable CS reconstruction.
%where %each column is a base sequence. 
%we assume that each element of $\Ubu$ is $1$ to obtain unimodular sequences in the end.



\subsection{PAPR Reduction}
%Let $\abu = (a_0, \cdots, a_{M-1})^T$ be a $q$-ary sequence of length $M$, where $a_i \in \Z_q$. %  \{ 0, 1, \cdots, q-1\}$.
%Then, its \emph{modulated} sequence is given by
%$\bbu = (b_0, \cdots, b_{M-1})^T$, %  be a $q$-ary modulated sequence of length $M$, 
%where $b_i = e^{ j \frac{2 \pi a_i}{q} }$. % $b_i= e^{j \frac{2 \pi a_i}{q}}$. 
In system model of Section II, 
if the sequence $\sbu_n $ %of device $n$ 
is transmitted through $M$ subcarriers,
the peak-to-average power ratio (PAPR) of its OFDM signal is determined by~\cite{Litsyn:peak}
\begin{equation}\label{eq:papr}
	{\rm PAPR} (\sbu_n)= \max_{t \in [0, 1) } 
	\frac{ \left| \sum_{i=1} ^{M} s_{m, n} e^{j 2 \pi (m-1) t} \right|^2 }{ M} ,
	%\frac{ \left| \sum_{i=0} ^{2^m-1} b_i e^{j 2 \pi i t} \right|^2 }{ \sum_{i=0} ^{2^m-1} |b_i |^2  } .
\end{equation}
where $j=\sqrt{-1}$.
In~\eqref{eq:papr}, we assumed that $\sbu_n $ is unimodular, i.e., $|s_{m, n}| =1$, for $m=1, \cdots, M$. 
%where $M=2^m$.
%In~\eqref{eq:papr},
%the computation of peak power containing a continuous-time signal
%can be approximated using the discrete Fourier transform (DFT) of the
%oversampled, discrete signal~\cite{Litsyn:peak}.
% It is well known that each modulated Golay complementary sequence has the PAPR of at most $2$~\cite{Jedwab:RM}.
%From the PAPR of all the sequences $\sbu_1, \cdots, \sbu_N$, 
%Then, the PAPR of the sequence set $\Sbu = [\sbu_1, \cdots, \sbu_N]$
%%in~\eqref{eq:y} 
%is defined by 
%\[
%{\rm PAPR} (\Sbu) = \max_{n = 1, \cdots,N} {\rm PAPR(\sbu_n)}.
%\]

Given a partial unitary matrix $\Ubu_\Omega$, 
we try to reduce the PAPR of
the column sequences\footnote{
If $\Ubu$ has a column of all ones, like the Fourier or Hadamard matrix,
the maximum PAPR of the column sequences of its partial unitary matrix $\Ubu_\Omega$ has the highest value of $M$,
regardless of $\Omega$.}.
%%i.e., ${\rm PAPR} (\Ubu_{\Omega}) = M$.}.
For PAPR reduction, we apply
%In the second design stage,
a unimodular and complex-valued sequence $\vbu = (v_1, \cdots, v_{M})^T$   
as a common mask to each column of $\Ubu_\Omega$, i.e.,
\begin{equation}\label{eq:mask}
\Ubu_{\Omega, \vbu}  
= {\rm diag} (\vbu) \cdot \Ubu_\Omega.
\end{equation}
In~\eqref{eq:mask}, we use
%we assume that 
a modulated $q$-ary sequence for $\vbu$, i.e., $v_m = e^{j \frac{2 \pi a_m}{q}}$,
where $a_m \in \Z_q$ for $m=1, \cdots, M$.
Then, it is clear that the inner product of a column pair in $\Ubu_{\Omega, \vbu}$
is identical to that of the corresponding pair in $\Ubu_\Omega$, since
$\Ubu_{\Omega, \vbu} ^* \Ubu_{\Omega, \vbu} = \Ubu_\Omega ^* \Ubu_\Omega$,
which suggests that the new matrix $\Ubu_{\Omega, \vbu}$ may exhibit 
the same performance of CS-based detection as the matrix $\Ubu_\Omega$.

In the second design stage, we search for a masking sequence $\vbu$
that allows the column sequences of $\Ubu_{\Omega, \vbu}$ to have a desired PAPR property,
maintaining the performance of reliable CS reconstruction from the first design stage.

%to enhance the PAPR property of the columns of $\Ubu_{\Omega, \vbu}$. % as low as possible.
%the PAPR of each column of $\Ubu_{\Omega, \vbu} $ as low as possible.



\subsection{Genetic Algorithm}
The genetic algorithm (GA) is an evolutionary technique
to solve an optimization problem that is computationally intractable~\cite{Holland:gen}.
Inspired by the evolutionary mechanism in nature,
GA transforms and evolves \emph{chromosomes} through crossover, mutation, selection, population updates, and so on.
Through a sufficient number of generations,
GA converges to a fittest chromosome, which can be a solution to the optimization problem. 
%By converging the local optima quickly,
Thanks to the fast convergence to local optima,
GA has attracted much attention in machine learning and data mining~\cite{Freitas:survey}$-$\cite{Viv:intel}.
Recently, GA has expanded its application to other areas, e.g., %communications and signal processing, including
channel coding~\cite{Hebbes:turbo}$-$\cite{brink:polar}, spreading code design~\cite{Dam:seq, Nat:evol},
CS recovery~\cite{Conde:sparse}$-$\cite{Erkoc:evol}
and matrix optimization~\cite{Chen:ISAR}$-$\cite{Leven:moga}, etc.
%While the GA has been used to optimize subsampling patterns for partial Fourier matrices
%in \cite{Chen:ISAR} and \cite{Nie:maga}, 



In Section III.A and III.B, 
we introduced two design stages to obtain a set of good sequences.
At each stage, the design goal can be specified by
an optimization problem that needs to be solved by GA. 
Given a unitary matrix $\Ubu$, 
the objective of the first design stage is to find a fittest chromosome 
or subsampling index set $\Omega$ in~\eqref{eq:S-single}. 
The optimization problem 
for this objective can be formulated by
\begin{equation}\label{eq:opt}
	\Omega_{\rm opt} = \underset{\Omega \subset \{1, \cdots, N\}, |\Omega| = M}{\mbox{argmin}} \ f_1 (\Ubu_\Omega), 
	%  \mbox{ subject to } \Ubu_\Omega = \frac{1}{\sqrt{M}} \Rbu_{\Omega} \Ubu,
\end{equation} 
where $f_1 (\Ubu_\Omega)$ is a cost function of the first-stage optimization.
%Different applications may define the cost function differently depending on the requirements. 
The cost function needs to be a good metric 
that reflects the performance of CS reconstruction
with the partial unitary matrix $\Ubu_{\Omega}$. 
The first-stage GA tries to minimize the cost function $f_1 (\Ubu_\Omega)$ through evolution steps,
in order to find a solution to~\eqref{eq:opt}.
%to find a fittest chromosome or subsampling index set. 
%For example, the GA tried to minimize the number of measurements~\cite{Chen:ISAR}, where 
%$f(\Ubu_\Omega) = |\Omega| $, while preserving the image quality with a punishment factor.
%In~\cite{Nie:maga}, the cost function was the mutual coherence of $\Abu$,
%i.e., $f(\Ubu_\Omega) = \mu(\Abu) = \max_{ 1 \leq i \neq j \leq N  }
%\frac{\left| \left \langle \abu_{i} ,   \abu_{j}  \right \rangle \right| }
%{\| \abu_{i} \|_2 \| \abu_{j} \|_2}$, 
%where $\abu_i$ and $\abu_j$ are the $i$th and the $j$th columns of $\Abu = \Ubu_\Omega$, respectively.


When the first-stage GA is completed, 
the partial unitary matrix $\Ubu_{\Omega_{\rm opt}}$ with an optimized subsampling index set $\Omega_{\rm opt}$ 
is available for the second design stage. 
Given $\Omega_{\rm opt}$, 
the second-stage GA tries to find a fittest chromosome or unimodular masking sequence $\vbu$ of length $M$,
which is a solution to another optimization problem of
\begin{equation}\label{eq:opt2}
	\vbu_{\rm opt} = \underset{\vbu \in \mV_{q, M}}{\mbox{argmin}} \ f_2 (\Ubu_{\Omega_{\rm opt}, \vbu}) ,
\end{equation}
where $\mV_{q, M}$ is a set of all modulated $q$-ary sequences of length $M$.
In~\eqref{eq:opt2}, $f_2 (\Ubu_{\Omega_{\rm opt}, \vbu})$ is a cost function of 
the second-stage optimization, which
%The cost function $f_2 (\Ubu_{\Omega_{\rm opt}, \vbu})$
should be a metric for the PAPR property of 
the column sequences of $\Ubu_{\Omega_{\rm opt}, \vbu} $.
The second-stage GA tries to enhance the PAPR property with $\vbu_{\rm opt}$,
maintaining the performance of $\Ubu_{\Omega_{\rm opt}}$ for reliable CS reconstruction.
 
Finally, if $\Omega_{\rm opt}$ and $\vbu_{\rm opt}$ are found by the two-stage GA,
we obtain the matrix $\Sbu = \Ubu_{\Omega_{\rm opt}, \vbu_{\rm opt}}$ in~\eqref{eq:mmv}, 
where the column sequences are proposed as non-orthogonal sequences with low PAPR  
for reliable CS-based detection in uplink grant-free access.
In next section, our two-stage GA will be described 
with the details % of our two-stage GA
to find $\Omega_{\rm opt}$ and $\vbu_{\rm opt}$, respectively.
%which is a proposed design process for spreading sequences of uplink grant-free NOMA.


%In this letter, 
%we employ the GA to find an optimized subsampling index set $\Omega_{\rm opt}$
%by solving \eqref{eq:opt}.
%We scrutinize the effectiveness of GA
%for optimizing various partial unitary matrices  
%over a wide range of compression and sparsity ratios.  


\section{Two-Stage Genetic Algorithm}
In this section, we describe the evolutionary steps of our two-stage GA. % the genetic algorithm
The first-stage GA is to find an optimized subsampling index set $\Omega_{\rm opt} \subset \{1, \cdots, N\}$ 
with $|\Omega_{\rm opt}| = M$, where $M$ and $N$ are fixed.
Given $\Omega_{\rm opt}$, 
the second-stage GA then attempts to find an optimized masking sequence $\vbu_{\rm opt} \in \mV_{q, M}$,
where $q = N$.



\subsection{Stage 1: Subsampling Optimization}

\subsubsection{Initialization}
A population $\mP_1$ is defined by a collection of $T_1$ subsampling index sets, i.e.,
$\mP_1 = (\Omega_1, \Omega_2, \cdots, \Omega_{T_1})$,
where $\Omega_t \subset \{1, \cdots, N\}$ with $|\Omega_t| = M$ for $t = 1, \cdots, T_1$.
Initially, the indices of $\Omega_t$ are selected randomly. % from $\{1, \cdots, N\}$.

\subsubsection{Cost Function}
%Instead of mutual coherence, 
In the first-stage, we propose the cost function for an index set $\Omega_t$ by
%the cost function is defined by
\begin{equation}\label{eq:cost}
	f_1(\Ubu_{\Omega_t}) = \frac{1}{\sqrt{N(N-1)}} \left \| {\rm abs} \left(\Ubu_{\Omega_t} ^* \Ubu_{\Omega_t} \right) - \Gbu_W \right \|_F,  
\end{equation}
where $\Gbu_W \in \R^{N \times N}$ is a matrix with diagonal entries of $1$ and off-diagonal entries 
of $\sqrt{\frac{N-M}{M(N-1)}}$.
Intuitively, the cost function of \eqref{eq:cost} represents
the average (rms-sense) distance between the inner product of a column pair of $\Ubu_{\Omega_t}$ 
and the Welch bound equality (WBE)~\cite{Welch:low}.
%By minimizing 
Attempting to minimize the cost function, 
the first-stage GA % attempts to %find an optimized index set % $\Omega_{\rm opt}$ 
makes the inner product of a column pair of $\Ubu_{\Omega_t}$ closer to the WBE,
which approximates the resulting matrix to an equiangular tight frame (ETF)~\cite{Kov:frames}.
%This cost function for a complex-valued $\Ubu_\Omega$ in general
The target matrix $\Gbu_W$ is similar to, but not the same as the one in  
the convex set (e.g. (12) in~\cite{Vahid:grad} and (14) in~\cite{Li:opt}) for 
optimizing CS matrices.
As remarked by \cite{Li:opt},
it is more reasonable to measure a distance from $\Gbu_W$ in~\eqref{eq:cost}, rather than $\Ibu$, 
which has been confirmed by the optimization of~\cite{Vahid:grad}. %  showed better performance with $\Wbu$.
%the former confirmed better performance of the optimization in \cite{Vahid:grad}. 

\subsubsection{Crossover}
%Note that each subsampling index set of a population $\mP$ is
%a chromosome in GA.
In population $\mP_1$,
let us 
consider a pair of index sets $\Omega_{t_1}$ and $\Omega_{t_2}$, $1 \leq t_1 \neq t_2 \leq T_1$, where
we assume
$f_1 \left(\Ubu_{\Omega_{t_1}} \right) < f_1 \left(\Ubu_{\Omega_{t_2}} \right)$. Then,
%$\mu \left(\Abu_{\Omega_{t_1}} \right) < \mu \left(\Abu_{\Omega_{t_2}} \right)$.
$d_1 = \lceil \beta_1 \cdot M \rceil$ and 
$d_2 = M - d_1$ indices are randomly selected from $\Omega_{t_1}$ and $\Omega_{t_2}$, respectively,
where $\beta_1 > 0.5$.
Finally, the selected indices, which should be all distinct, % so that 
are combined to generate a new index set through \emph{crossover}. 
In other words, we create a new subsampling index set by combining parents,
where a parent index set with a lower cost function
is more involved in creating its offspring.
We apply the crossover for every pair of parent index sets from $\mP_1$, % with $\beta_1 = 0.7$,
which yields a new population $\mC_1$ of size $ \binom{T_1}{2} = \frac{T_1(T_1-1)}{2}$ at each evolution step.



\subsubsection{Mutation}
In nature,
parts of a chromosome can be mutated in a generation,
which provides diversity for evolution.
In the first-stage GA, % a single index is 
$\mu_1$ indices are randomly selected from each index set % $\Omega_t$ 
in $\mP_1$,
which is then replaced by new (random) ones through \emph{mutation}.
At each evolution step, 
we apply the mutation to all index sets in $\mP_1$, % with $\mu_1 = 1$,
which yields a new population $\mM_1$ of size $T_1$.

\subsubsection{Population Update}
Through crossover and mutation,
we have a new, intermediate population $\mI_1 = \mP_1 + \mC_1 + \mM_1$,
%Even if there are identical chromosomes in $\mI_1$, the possibility will be extremely low, and 
%we treat them as separate ones.},
where the size\footnote{%The chromosomes in $\mI_1$ (or $\mI_2$) are highly likely to be distinct.
Identical chromosomes in $\mI_1$ (or $\mI_2$), if any, are treated as separate ones.} 
is $ |\mI_1| = T_1 + \binom{T_1}{2} + T_1 =  \frac{T_1(T_1+3)}{2}$.
From $\mI_1$, we select the $T_1$ index sets % from $\mS$ 
with the $T_1$ lowest cost functions of \eqref{eq:cost}.
%according to the cost function in \eqref{eq:cost}.
%In other words, we choose $T$ index sets that have the lowest .
The population $\mP_1$ is then updated by the $T_1$ fittest index sets
at each evolution step. 

\subsubsection{Iteration and Selection}
In the first-stage GA, 
the evolution steps of crossover, mutation, and population update
are repeated by a predefined number of iterations, denoted by $I_{\max, 1}$.
In the end, the fittest index set of $\mP_1$, which has the lowest cost function of \eqref{eq:cost},
will be selected as an optimized subsampling index set $\Omega_{\rm opt}$. 
 
Algorithm 1 describes the entire steps of the first-stage GA
to optimize a subsampling index set. % for partial unitary matrices.


\begin{table}[!t]
	\fontsize{8}{10pt}\selectfont
	\centering
	\begin{tabular}{l}
		\hlineB{2.5}
		\textbf{Algorithm 1} Genetic Algorithm for Subsampling Optimization \\
		\hline
		\textbf{Input:} Unitary matrix $\Ubu$, Number of measurements $M$, \\ 
		        \qquad \quad Population size $T_1$, Crossover rate $\beta_1$, Mutation rate $\mu_{1}$,  \\
   		        \qquad \quad Maximum number of iterations $I_{\max, 1}$. \\
		\emph{Initialization}: Create a population $\mP_1 = \{ \Omega_1, \cdots, \Omega_{T_1}\}$  \\
		\qquad \qquad \qquad of randomly selected index sets, where $|\Omega_t| = M$.  \\		
		\qquad \qquad \qquad Compute the cost function \eqref{eq:cost} 
		for each index set of $\mP_1$. \\ 
		%$f \left(\Ubu_{\Omega_t} \right)$ for $t = 1, \cdots, T$. \\
		\emph{Iteration}:  \\
		\textbf{for} $i=1$ to $I_{\max, 1}$ \textbf{do}  \\
		\quad \emph{Crossover}: Create a new population $\mC_1$ with index sets from $\mP_1$. \\
		\quad \emph{Mutation}: Create a new population $\mM_1$ with index sets from $\mP_1$. \\
		\quad \emph{Population update}: Compute the cost function \eqref{eq:cost} for each index set \\ %of \eqref{eq:cost} for \\
		\qquad \qquad \qquad \qquad \quad of $\mI_1 = \mP_1 + \mC_1+\mM_1$, select the $T_1$ index sets \\
		\qquad \qquad \qquad \qquad \quad from $\mI_1$ with the $T_1$ lowest cost functions, and \\
		\qquad \qquad \qquad \qquad \quad update $\mP_1$ with the $T_1$ fittest index sets. \\
		\textbf{end for} \\
		\emph{Selection}: Select the fittest index set $\Omega_{\rm opt}$ from $\mP_1$. \\ 
		\textbf{Output:} Optimized subsampling index set $\Omega_{\rm opt}$ \\
		\hline
	\end{tabular}
	\label{tb:ga}
\end{table}


\subsection{Stage 2: Masking Sequence Optimization}


\subsubsection{Initialization}
In the second-stage GA, a population consists of $T_2$ masking sequences, i.e.,
%is defined by %$\mP_2$ by a collection of $T$ candidates of subsampling index sets, i.e.,
$\mP_2 = (\vbu_1, \cdots, \vbu_{T_2})$, where
each element of $\vbu_t = (v_{1, t}, \cdots, v_{M, t})^T$ %   
is $ v_{m, t} = e^{j \frac{2 \pi a_{m, t}}{N}}$ for $m = 1, \cdots, M$ and $t = 1, \cdots, T_2$.
Initially, $a_{m, t} $ is randomly taken from $\Z_N$.


\subsubsection{Cost Function}
Note that the partial unitary matrix $\Ubu_{\Omega_{\rm opt}}$
is available %in the second-stage GA 
by the optimized subsampling index set $\Omega_{\rm opt}$ from the first-stage GA.
%Instead of mutual coherence, 
In the second-stage GA, the cost function for a masking sequence $\vbu_t$ is proposed by
%the cost function is defined by
\begin{equation}\label{eq:cost2}
	f_2(\Ubu_{\Omega_{\rm opt}, \vbu_t}) = \frac{1}{|\mK_\delta|}\sum_{\pbu_n \in \mK_\delta} {\rm PAPR} (\pbu_n)  ,
\end{equation}
where $\Ubu_{\Omega_{\rm opt}, \vbu_t} = {\rm diag}(\vbu_t) \cdot \Ubu_{\Omega_{\rm opt}}
= [\pbu_1, \cdots, \pbu_N] \triangleq \Pbu$.
%where $\pbu_n$ is a column of $\Pbu = \Ubu_{\Omega_{\rm opt}, \vbu} = [\pbu_1, \cdots, \pbu_N]$.
In~\eqref{eq:cost2}, $\mK_\delta$ is a set of columns in $\Pbu$
whose PAPR belong to the top $ \delta  \%$, where $|\mK_\delta| = \left\lfloor \frac{\delta N}{100} \right\rfloor$.
That is, $f_2(\Ubu_{\Omega_{\rm opt}, \vbu_t}) $ is the average of top $\delta \%$ PAPR
of $\Pbu$, which will be minimized % by the second-stage GA 
to enhance the PAPR distribution
of the columns of $\Pbu$. 



\subsubsection{Crossover}
%Note that each subsampling index set of a population $\mP$ is
%a chromosome in GA.
As in the first-stage GA,
%In the population $\mP_1$,
%let us 
we consider a pair of sequences $\vbu_{t_1}$ and $\vbu_{t_2}$ from $\mP_2$, $1 \leq t_1 \neq t_2 \leq T_2$, where
$f_2 \left(\Ubu_{\Omega_{\rm opt}, \vbu_{t_1}} \right) < f_2 \left(\Ubu_{\Omega_{\rm opt}, \vbu_{t_2}} \right)$.
%$\mu \left(\Abu_{\Omega_{t_1}} \right) < \mu \left(\Abu_{\Omega_{t_2}} \right)$.
Then, the first $d_1 = \lceil \beta_2 \cdot M \rceil$ elements from $\vbu_{t_1}$
and the last $d_2 = M - d_1 $ elements from $\vbu_{t_2}$ are combined to generate a new masking sequence,
where $\beta_2 > 0.5$.
%In other words, we create a new subsampling index set by combining parents,
%where a parent index set with the lower cost function %coherence 
%is more involved in creating its offspring.
Applying the crossover for every pair of sequences in $\mP_2$,
we have a new population $\mC_2$ of size $ \binom{T_2}{2} = \frac{T_2(T_2-1)}{2}$
at each evolution step.



\subsubsection{Mutation}
For mutation, the second-stage GA % a single element 
randomly selects $\mu_2$ elements from each sequence % $\vbu_t$ 
in $\mP_2$,
where each element is replaced by a new (random) $N$-ary modulated one. %$e^{j \frac{2 \pi a }{N}}$,
%where $a \in \Z_N$.
%At each evolution step, 
We obtain
a new population $\mM_2$ of size $T_2$ by applying the mutation
to all sequences in $\mP_2$.

\subsubsection{Population Update}
Through crossover and mutation,
we obtain % a new population 
$\mI_2 = \mP_2 + \mC_2 + \mM_2$,
where $ |\mI_2| = % T_2 + \binom{T_2}{2} + T_2 =  
\frac{T_2(T_2+3)}{2}$.
From $\mI_2$, we select the $T_2$ sequences % from $\mS$ 
with the $T_2$ lowest cost functions of \eqref{eq:cost2}.
%according to the cost function in \eqref{eq:cost}.
%In other words, we choose $T$ index sets that have the lowest .
The population $\mP_2$ is then updated by the $T_2$ fittest sequences
at each evolution step. 

\subsubsection{Iteration and Selection}
In the second-stage GA, %the evolutionary steps of 
crossover, mutation, and population update
are repeated by a predefined number of iterations, denoted by $I_{\max, 2}$.
Finally, the fittest sequence of $\mP_2$, which has the lowest cost function of \eqref{eq:cost2},
will be selected as an optimized mask $\vbu_{\rm opt}$. 



\begin{table}[!t]
	\fontsize{8}{10pt}\selectfont
	\centering
	\begin{tabular}{l}
		\hlineB{2.5}
		\textbf{Algorithm 2} Genetic Algorithm for Masking Sequence Optimization \\
		\hline
		\textbf{Input:} Partial unitary matrix $\Ubu_{\Omega_{\rm opt}}$, Number of measurements $M$, \\ 
		\qquad \quad Population size $T_2$, Crossover rate $\beta_2$, Mutation rate $\mu_{2}$,  \\
		\qquad \quad Maximum number of iterations $I_{\max, 2}$. \\
		\emph{Initialization}: Create a population $\mP_2 = \{ \vbu_1, \cdots, \vbu_{T_2}\}$    \\
		\qquad \qquad \qquad of random modulated $N$-ary sequences of length $M$.  \\		
		\qquad \qquad \qquad Compute the cost function \eqref{eq:cost2} 
		for each sequence of $\mP_2$. \\ 
		%$f \left(\Ubu_{\Omega_t} \right)$ for $t = 1, \cdots, T$. \\
		\emph{Iteration}:  \\
		\textbf{for} $i=1$ to $I_{\max, 2}$ \textbf{do}  \\
		\quad \emph{Crossover}: Create a new population $\mC_2$ with sequences from $\mP_2$. \\
		\quad \emph{Mutation}: Create a new population $\mM_2$ with sequences from $\mP_2$. \\
		\quad \emph{Population update}: Compute the cost function \eqref{eq:cost2} for each sequence \\ %of \eqref{eq:cost} for \\
		\qquad \qquad \qquad \qquad \quad of $\mI_2 = \mP_2 + \mC_2+\mM_2$, select the $T_2$ sequences \\
		\qquad \qquad \qquad \qquad \quad from $\mI_2$ with the $T_2$ lowest cost functions, and \\
		\qquad \qquad \qquad \qquad \quad update $\mP_2$ with the $T_2$ fittest sequences. \\
		\textbf{end for} \\
		\emph{Selection}: Select the fittest masking sequence $\vbu_{\rm opt}$ from $\mP_2$. \\ 
		\textbf{Output:} Optimized masking sequence $\vbu_{\rm opt}$ \\
		\hline
	\end{tabular}
	\label{tb:ga2}
\end{table}


Algorithm 2 describes the entire steps of the second-stage GA
to optimize a masking sequence.
Finally, a set of non-orthogonal sequences, or $\Sbu = \Ubu_{\Omega_{\rm opt}, \vbu_{\rm opt}}$, 
is provided by our two-stage GA of Algorithms 1 and 2,
as illustrated by Fig.~\ref{fig:twoGA}.


\begin{rem}\label{rm:gen}
%	As illustrated in Fig.~\ref{fig:twoGA},
	The set of non-orthogonal sequences designed by our two-stage GA
	can be represented by 
	\begin{equation}\label{eq:set}
	\Sbu =  \Ubu_{\Omega_{\rm opt}, \vbu_{\rm opt}} = \frac{1}{\sqrt{M}} {\rm diag} (\vbu_{\rm opt}) 
	\cdot \Rbu_{\Omega_{\rm opt}}  \Ubu.
 	\end{equation}	
	Given a unitary matrix $\Ubu$, \eqref{eq:set} means that the matrix
	$\Sbu$ can be generated by the operations of row selection specified by $\Omega_{\rm opt}$ and
	masking by $\vbu_{\rm opt}$.
	Therefore, a BS receiver can generate the sequence set $\Sbu$ easily with 
	the highly structured unitary matrix $\Ubu$ by storing $\Omega_{\rm opt}$ and $\vbu_{\rm opt}$.
	Moreover, %if $\Ubu$ allows the fast unitary transform, 
	CS-based detection can be carried out fast and efficiently at BS, exploiting the fast unitary transform by $\Ubu$.  
	Also, each mMTC device is able to generate its unique sequence on-the-fly 
	with a unique column structure of $\Ubu$ by
	storing $\Omega_{\rm opt}$ and $\vbu_{\rm opt}$, % with a unique column of $\Ubu$, %respectively, 
	which allows its cheap and efficient implementation. 	
\end{rem}



\begin{figure}
	\centering
	\includegraphics[width=0.46\textwidth, angle=0]{twoGA.eps}
	\caption{Two-stage GA for non-orthogonal sequence design.}
	\label{fig:twoGA}
\end{figure}


\section{Simulation Results}

In this section, 
%simulation results
%demonstrate the effectiveness of our two-stage GA to %optimize partial unitary matrices 
%obtain non-orthogonal sequences
%with low PAPR properties that can ultimately guarantee reliable CS-based detection 
%for uplink grant-free access.
%First of all, 
we first demonstrate the effectiveness of our two-stage GA for non-orthogonal sequence design.
%For this goal, we present the phase transitions of CS reconstruction performance 
%for several partial unitary matrices from the first-stage GA,
%and then examine the PAPR of the non-orthogonal sequences from the second-stage GA.
%with optimized subsampling index sets,
%Then, we examine the PAPR of the sequences obtained from the second-stage GA.
Then, we present simulation results of CS-based detection for uplink grant-free access,
%employing the non-orthogonal sequences designed by our two-stage GA.
which demonstrates the performance of non-orthogonal sequences designed by our two-stage GA.



For the unitary matrix $\Ubu$, % with $N=256$,
we use the $N \times N$ Fourier matrix $\Fbu = [F_{k,l}]$,
where
%each element is 
$F_{k,l} = e^{-\frac{j 2 \pi (k-1)(l-1)}{N}}$ 
for $1 \leq k, l \leq N$.
%while the Hadamard matrix $\Hbu$ is recursively generated.
Additionally, we consider another unitary matrix based on Zadoff-Chu (ZC) sequences~\cite{Chu:ZC}.
Each cyclic shift of the ZC sequence of even length %  $N=256$
becomes a column of a matrix $\Zbu = [Z_{k,l}]$,
where each element is given by
\[
Z_{k, l} = e^{\frac{-j \pi (k+N-l)^2}{N} }, \quad 1 \leq k, l \leq N.
\]
Due to the perfect auto-correlation~\cite{Chu:ZC} of ZC sequences,
it is clear that
$\Zbu$, called the ZC matrix, is also unitary.


Beginning with $\Fbu$ and $\Zbu$, 
our two-stage GA gives
$\Fbu_{\Omega_{\rm opt}, \vbu_{\rm opt}}$ and $\Zbu_{\Omega_{\rm opt}, \vbu_{\rm opt}}$, 
respectively, by Algorithms 1 and 2. %where the columns are
%the proposed non-orthogonal sequences.
Finally, their columns are
proposed as non-orthogonal sequences, called 
\emph{Fourier-based} and \emph{ZC-based}
sequences, respectively.

%Second, we use a binary $m$-sequence~\cite{GolGong:seq} of length $N-1=255$
%to define another unitary matrix $\Mbu = [M_{k,t}]$,
%where each element is % given by
%\begin{equation}\label{eq:mseq}
%	M_{k, t} = \left\{ \begin{array}{ll} 1, & \quad \mbox {if } k=1 \mbox{ or } t=1, \\  
%		a_{k-t}, & \quad \mbox{otherwise}, \end{array} \right. 
%\end{equation}
%for $1 \leq k, t \leq N$.
%In~\eqref{eq:mseq}, $(a_0, \cdots, a_{N-2})$ is a binary $m$-sequence taking $\pm 1$,
%%and $a_{k-t}$ is its cyclic shift, where 
%and the subscript $k-t$ is computed modulo $N-1$.
%Due to the ideal two-level auto-correlation~\cite{GolGong:seq} of the sequence,
%$\Mbu$ belongs to a class of the Hadamard matrices, called the Hadamard (m-seq) matrix in this paper.

\subsection{Effectiveness of Two-Stage GA}
In simulations, %we use $N \times N$ unitary matrices, where $N = 256$, and 
each stage GA has the population size of $T_1 = T_2 = 20$,
the crossover rate $\beta_1 = \beta_2 = 0.7$, and the mutation number $\mu_1 = \mu_2 = 1$, respectively.
%nd the evolutionary steps % of GA
%by $T = 20$ and $I_{\max}=500$, respectively. 
Algorithm 1 has $I_{\max, 1} = 500$,
whereas $I_{\max, 2} = 2000$ in Algorithm 2, as experiments showed that the cost function of Algorithm 2
converges slowly. % in experiments.
%experiments showed 
%the slower convergence of the cost function in Algorithm 2.
%$f_2(\Ubu_{\Omega_{\rm opt}, \vbu_t}) $ from experiments.
Finally, the cost function of Algorithm 2 computes %computes
top $\delta = 30 \%$ average of PAPR of $\Pbu= \Fbu_{\Omega_{\rm opt}, \vbu_t} $
(or $\Zbu_{\Omega_{\rm opt}, \vbu_t} $)
for a mask sequence $\vbu_t$.



\begin{figure}[!t]
	\centering
	\includegraphics[width=0.475\textwidth, angle=0]{ALL_Frob_N256-M80-I.eps}
	\caption{Evolution of the cost functions of Algorithms 1 and 2, where $N = 256$ and $M = 80$. 
		Each curve shows the cost function of the fittest chromosome at each iteration.}
	\label{fig:cost_iter}
\end{figure}




Fig.~\ref{fig:cost_iter} displays the evolution of the cost functions \eqref{eq:cost} and \eqref{eq:cost2} of the fittest chromosomes,
respectively, from $M \times N$ partial Fourier and ZC matrices,
where $N=256$ and $M = 80$.
The figure shows that each stage GA continues to reduce its cost function over the evolution steps.
%to reach an optimized target.
%to minimize it.
As mentioned above, we observed that 
the cost function of Algorithm 2 % the second-stage GA 
converges slowly, 
compared to that of Algorithm 1. % the first-stage GA.
It is because the search space size for the optimization problem \eqref{eq:opt2} is $N^M$,
which is much larger than that of \eqref{eq:opt}, or $\binom{N}{M} \leq \left( \frac{eN}{M} \right)^M$.
Fig.~\ref{fig:cost_iter} also reveals that the cost functions of partial Fourier and ZC matrices
converge to similar values,
%are higher than those of partial Hadamard matrices, 
which suggests that %the formers may exhibit more reliable performance of CS-based detection with lower PAPR properties.
their performance of CS reconstruction and PAPR property will be similar to each other.


\begin{figure}[!h]
	\centering
	\includegraphics[width=0.475\textwidth, angle=0]{FrZC_phase_N256-J8-snr20.eps}
	\caption{Phase transitions for partial Fourier % ($\Fbu_{\Omega_{\rm opt}}$) 
		and ZC %($\Zbu_{\Omega_{\rm opt}}$) %, and Hadamard ($\Hbu_\Omega$ and $\Mbu_\Omega$)
		matrices under MMV reconstruction (SOMP), % obtained by Algorithm 1, 
		where $N=256$ and $J=8$ at ${\rm SNR}= 20$ dB.}
	\label{fig:all_mmv}
\end{figure}

To investigate the effectiveness of the first-stage GA,
we sketch the phase transition diagrams for 
%evaluate the performance of 
CS reconstruction with partial Fourier and ZC matrices, respectively,
%where $\Omega_{\rm opt}$ is a subsampling index set 
obtained by Algorithm 1. 
%by sketching the phase transitions.
We consider an MMV problem $\Ybu = \Abu \Xbu + \Wbu$,
where $ \Abu =  \Fbu_{\Omega_{\rm opt}}$ (or $ \Zbu_{\Omega_{\rm opt}}$),
$\Xbu = (\xbu_1, \cdots, \xbu_J)$ is a jointly sparse matrix 
with common nonzero rows, and 
$\Ybu = (\ybu_1, \cdots, \ybu_J)$ is a collection of $J=8$ measurement vectors. 
The nonzero entries of $\Xbu$ are independently drawn from $\mathcal{CN} (0, 1)$,
where their row positions are uniformly distributed.
Also, each element of $\Wbu$ is the i.i.d. Gaussian noise from $\mathcal{CN} (0, \sigma_n ^2)$,
where the signal-to-noise ratio (SNR) is set to 
${\rm SNR} = \frac{\sum_{t=1} ^J|| \Abu \xbu_t ||_2 ^2}{J K M \sigma_n ^2} = 20$ dB.
%Also, $\nbu$ is the complex-valued Gaussian noise with each element drawn from $\mathcal{CN}(0, \sigma^2)$.
%Then, the signal-to-noise ratio (SNR) is set to ${\rm SNR} = \frac{|| \Abu \xbu ||_2 ^2}{K M \sigma^2} = 20$ dB.
In phase transition, we made $10^4$ trials of CS reconstruction
at each test point,
where the step sizes of $\frac{M}{N}$ and $\frac{K}{M}$ are $2^{-5}$ and $10^{-2}$, respectively.
The phase transition
indicates that the corresponding CS reconstruction is successful with 
probability exceeding $99 \%$ below the transition curve, where
a success is declared % for CS reconstruction
if an estimated $\widehat{\Xbu}$
achieves $\frac{|| \Xbu - \widehat{\Xbu}||_F ^2}{|| \Xbu||_F ^2} < 10^{-2}$.



%Figs.~\ref{fig:Fr_ZC} and \ref{fig:Hd_mH} depict the phase transitions
Fig.~\ref{fig:all_mmv} depicts the phase transitions
%for $\Fbu_{\Omega_{\rm opt}}$ and $\Zbu_{\Omega_{\rm opt}}$, respectively,
for partial Fourier and ZC matrices 
under MMV reconstruction by
the simultaneous orthogonal matching pursuit (SOMP)~\cite{Tropp:somp}, where 
the number of nonzero rows of $\Xbu$ is assumed to be known in advance.
In the figure, 
`GA (avg)' indicates the phase transition of $\Fbu_{\Omega_{\rm opt}}$ (or $\Zbu_{\Omega_{\rm opt}}$)
for which the cost function \eqref{eq:cost} has been minimized by Algorithm 1.
Meanwhile, `GA (coh)' corresponds to the case in which Algorithm 1 %optimizes the subsampling index sets 
changed its cost function with the mutual coherence, % of $\Ubu_\Omega$ 
%as the cost function, 
%the mutual coherence of $\Abu$,
i.e., $f_1(\Ubu_{\Omega_t}) %= \mu(\Abu) 
= \max_{ 1 \leq k \neq l \leq N  }
\frac{\left| \left \langle \abu_{k} ,  \abu_{l}  \right \rangle \right| }
{\| \abu_{k} \|_2 \| \abu_{l} \|_2}$, 
where $\abu_k$ and $\abu_l$ are the $k$th and the $l$th columns of $\Abu = \Ubu_{\Omega_t}$, respectively,
with $\Ubu = \Fbu$ or $\Zbu$.
Also, `random (coh)' and `random (avg)' show the phase transitions for
randomly subsampled Fourier (or ZC) matrices that %where the corresponding partial unitary matrices 
have the lowest coherence and the lowest cost function \eqref{eq:cost}, respectively,
out of $500$ trials.
%As a benchmark, we also sketch the phase transition for random Gaussian matrices.
Fig.~\ref{fig:all_mmv} shows that % for all tested partial unitary matrices,
the phase transition curves of `GA (avg)'
are higher than or equal to all the other ones over most compression ratios,
which demonstrates that the partial Fourier and ZC matrices optimized by Algorithm 1 
with the cost function \eqref{eq:cost}
present reliable MMV reconstruction %than with the other ones 
over a wide range of compression $\left( \frac{M}{N} \right)$ and sparsity $\left( \frac{K}{M} \right)$ ratios.



\begin{figure}[!t]
	\centering
	\includegraphics[width=0.475\textwidth, angle=0]{papr_FrZC_Frob_N256-M.eps}
	\caption{Maximum and top $30 \%$ average 
		PAPR of non-orthogonal sequences from partial Fourier and %($\Fbu_{\Omega_{\rm opt}, \vbu_{\rm opt}}$) and 
		ZC matrices, % ($\Zbu_{\Omega_{\rm opt}, \vbu_{\rm opt}}$) obtained by Algorithm 2
		where $N=256$.}
	\label{fig:all_papr}
\end{figure}




The effectiveness of the second-stage GA is verified by 
Fig.~\ref{fig:all_papr}, which sketches the maximum and top $ 30 \%$ average PAPR of the sequences
obtained by Algorithm 2. % from various partial unitary matrices.
In the figure, `avg' means that Algorithm 2 utilized 
the cost function of \eqref{eq:cost2},
while `max' indicates that the maximum PAPR of $\Ubu_{\Omega_{\rm opt}, \vbu_t}$
has been used as the cost function of Algorithm 2, where $\Ubu = \Fbu$ or $\Zbu$.
%instead of \eqref{eq:cost2}. % , in the evolution steps.
To demonstrate the PAPR improvement by Algorithm 2, 
we also sketch `no mask', which indicates the PAPR properties % maximum and top $30 \%$ average PAPR
of the sequences obtained by Algorithm 1 only, or column sequences of $\Fbu_{\Omega_{\rm opt}}$
(or $\Zbu_{\Omega_{\rm opt}}$).
Note that the maximum PAPR of partial Fourier matrices % $\Fbu_{\Omega_{\rm opt}}$ 
for `no mask' is 
outside the scope of this figure, taking the highest value of $M$ % is trivially high,
due to a column of all ones in $\Fbu_{\Omega_{\rm opt}}$.
Fig.~\ref{fig:all_papr} demonstrates that Algorithm 2 can significantly 
reduce the % lower 
maximum and top $30 \%$ average PAPR of 
the sequences from $\Fbu_{\Omega_{\rm opt}}$ and 
$\Zbu_{\Omega_{\rm opt}}$, respectively. 
Also, it 
reveals that using the cost function \eqref{eq:cost2} %leads to
is more effective for Algorithm 2 to enhance the % lower 
PAPR properties. % of the sequences.
%Meanwhile, the sequences from partial Hadamard matrices
%appear to be effective for reducing the top $30  \%$ average only.
%Moreover, Fig.~\ref{fig:all_papr} shows that the maximum and the top $30 \%$ average PAPR of
%the proposed sequences from partial Fourier and ZC matrices 
%are lower than those for sequences from partial Hadamard matrices.
%exhibit lower PAPR properties maximum and top $30  \%$ average.
%unless $M$ is too small.
%Compared to partial Hadamard matrices, 





\subsection{Performance of CS-based Detection}

Numerical experiments examine the performance of the proposed non-orthogonal sequences
for CS-based AUD and CE in uplink grant-free access.
Under the system model of Section II,
we assume that there are $N = 500$ devices in an mMTC cell,
where each one is assigned a unique non-orthogonal pilot sequence of length $M=80$. % for uplink grant-free access.
%Each active device sends its pilot for access attempt.
Reflecting sparse activity, each device sends its pilot with probability $p_a=0.1$
at each access time.
At BS, 
the received signal-to-noise ratio (SNR) per device is set to 
${\rm SNR} = \frac{1}{K} \cdot \frac{\sum_{t=1} ^J|| \Sbu \xbu^{(t)} ||_2 ^2}{J M \sigma_n ^2}$.



For CS-based AUD and CE, % joint activity detection and channel estimation,
a BS receiver deploys the SOMP algorithm that requires
no prior knowledge of the number of active devices\footnote{
%With no prior knowledge of the number of active devices, 
In simulations, this sparsity-blind SOMP stops its iteration empirically %  in simulations,
if the maximum signal proxy is less than $\sqrt{3\sigma_n ^2 J}$.}.
In AUD, both undetected and false-alarmed devices
are treated as errors.
Thus, the activity error rate (AER) is defined by the average of 
$\frac{|\mS \setminus \widehat{\mS} | + |\widehat{\mS} \setminus \mS  |}{|\mS \cup \widehat{\mS}|}$,
where $\mS$ and $\widehat{\mS}$ are true and detected sets of active devices, respectively.
Also, channel estimation errors are measured by the normalized 
mean squared errors (NMSE), or the average of
$\frac{|| \hbu_\mS - \widehat{\hbu}_{\mS} ||_2 ^2}{|| \hbu_\mS ||_2 ^2}$,
where $\hbu_\mS$ and $ \widehat{\hbu}_{\mS}$ are true and estimated channel vectors,
respectively, for truly active devices.
%To evaluate the performance of CS-based AUD and CE, 
In simulations, the averages for AER and NMSE are computed over $10^4$ access trials.




%In performance evaluation, we consider the %examine the AER and NMSE for the % proposed  %optimized by our two-stage GA.
To obtain the Fourier- and ZC-based sequences %designed by our two-stage GA
from 
$\Fbu_{\Omega_{\rm opt}, \vbu_{\rm opt}}$ and 
$\Zbu_{\Omega_{\rm opt}, \vbu_{\rm opt}}$, 
respectively,
Algorithms 1 and 2 use the same parameters as in Section V.A, but $I_{\max, 1} = 1000$ and $I_{\max, 2} = 4000$.
For comparison, 
%other non-orthogonal sequences 
%are also tested. % in performance evaluation.
%First of all, 
we generate complex-valued random Gaussian sequences of length $M$, 
where each element is drawn from the i.i.d. complex Gaussian distribution 
with zero mean and variance $1/M$~\cite{Liu:mimo}.
Also, we use the complex-valued MUSA spreading sequences of length $M$,
where each element is randomly taken from the 3-level signal constellation, i.e.,
$\frac{1}{\sqrt{12}} [{\pm 1} {\pm j} , \pm 1, \pm j, 0]$, in Fig.~2(b) of~\cite{Yuan:MUSA}.
Generating $N$ random Gaussian and $N$ %3-level 
MUSA sequences, we have
$M \times N$ matrices $\Gbu$ and $\Mbu$, respectively, 
where each one is a matrix with the lowest coherence 
among $1000$ trials.
%For each of the two random sequence sets, 


The last sequence set for comparison 
is obtained by cyclic shifts of the Zadoff-Chu (ZC) sequences with multiple roots,
where the sequence length $M_{\rm ZC}=79$ is a prime number closest to $M$.
In specific, we begin with an 
$M_{\rm ZC} \times M_{\rm ZC}$ matrix $\Cbu_u$
that consists of all cyclic shifts of 
a $u$th root ZC sequence~\cite{Chu:ZC} of length $M_{\rm ZC}$ with the $k$th element of 
$e^{ \frac{j \pi u k(k+1)}{M_{\rm ZC}} }$, where $u$
is a root number between $1$ and $ M_{\rm ZC}-1$.
Due to the perfect autocorrelation of the ZC sequence, 
$\Cbu_u$ is unitary for any $u$.
%all cyclic shifts of the sequence yield an 
%$M_{\rm ZC} \times M_{\rm ZC}$ unitary matrix $\Cbu_u$.
%When choosing $L = \lceil \frac{N}{M} \rceil$ root numbers,
For a set of sequences with low PAPR,
%optimize the PAPR of a set of resulting sequences, 
we then sort the root numbers $ u=1, \cdots, M_{\rm ZC}-1$
in ascending order of the maximum PAPR
that the column sequences of $\Cbu_u$ achieve.
Taking the first $L = \lceil \frac{N}{M_{\rm ZC}} \rceil$ root numbers, denoted by $u_1, \cdots, u_L$,
we produce a matrix $\Cbu' =  [\Cbu_{u_1}, \cdots, \Cbu_{u_L} ]$,
where the first $N$ columns are finally selected for
%to generate 
an $M_{\rm ZC} \times N$ matrix $\Cbu$, or 
%where
%the first $N$ columns are finally selected for 
a set of ZC sequences of prime length $M_{\rm ZC}$.
%selecting  from $\Cbu =  [\Cbu_{u_1}, \cdots, \Cbu_{u_L} ]$.
The coherence of $\Cbu$ is $\frac{1}{\sqrt{M_{\rm ZC}}}$, close to the Welch bound equality, 
due to the cross-correlation of ZC sequences with distinct roots~\cite{Sarwate:bound}. 
%i.e., $u_1, \cdots, u_L$, we obtain a set of $N$ sequences 
%from $\Zbu = [\Zbu_1, \cdots, \Zbu_L ']$,
%where $\Zbu_L '$ is the first $LM - N$ columns of $\Zbu_L$.
In simulations, 
the matrix $\Sbu$ in \eqref{eq:mmv} is determined by
the sequence sets under consideration, i.e.,
%Using the non-orthogonal sequences from our two-stage GA,
$\Sbu = \Fbu_{\Omega_{\rm opt}, \vbu_{\rm opt}}, \Zbu_{\Omega_{\rm opt}, \vbu_{\rm opt}}, 
\Gbu, \Mbu$, and $\Cbu$, respectively. 


%we construct the ZC spreading matrix by selecting the root numbers carefully.
%according to the suggestion from Reviewer 4.
%First of all, 



\begin{figure}[!t]
	\centering
	\includegraphics[width=0.365\textwidth, angle=0]{all_papr_d_N500-M80.eps}
	\caption{PAPR distribution (CCDF) of non-orthogonal sequences of length $M=80$ ($M_{ZC} = 79$), where $N=500$.}
	\label{fig:all_pd}
\end{figure}





Fig.~\ref{fig:all_pd} displays the complementary cumulative distribution function (CCDF) of 
PAPR of non-orthogonal sequences under consideration, 
where $N=500$ and $M=80~(M_{\rm ZC} = 79)$.
The ZC sequences of prime length $M_{\rm ZC}$, whose PAPR has been optimized as mentioned above,
exhibit the best PAPR distribution with maximum of $3.14 $ dB.
On the other hand, complex-valued random Gaussian and MUSA sequences have the poor distributions,
where the maximum PAPR are $10.86$ dB and $10.79$ dB, respectively.
% We observe that 
It is shown that the PAPR distributions of Fourier- and ZC-based sequences from our two-stage GA
are not so good as that of ZC sequences of prime length, but much better than those of the random sequences,
showing the maximum of $7.18$ dB and $7.66$ dB, respectively.
%Note that Algorithm 2 significantly reduced the maximum PAPR of Fourier- and ZC-based sequences,
%where they would be trivially high ($M$) with no masking sequence. 
As a result, the PAPR distributions of the proposed sequences appear to be acceptable 
for multicarrier transmission. % in uplink grant-free access.



\begin{figure}[!t]
	\centering
	\includegraphics[width=0.475\textwidth, angle=0]{ALL_N500-M80-J16-pa1-snr-new.eps}
	\caption{Performance of CS-based AUD and CE of non-orthogonal sequences over the received SNR per device, 
		where $N=500$, $M=80$ ($M_{ZC} = 79$), $J=16$, and $p_a = 0.1$.}
	\label{fig:all_snr}
\end{figure}




Fig.~\ref{fig:all_snr} shows the performance of CS-based AUD and CE %joint activity detection and channel estimation
over the received SNR per device.
%As can be seen from the figure,
In the figure, the AER and NMSE of Fourier- and ZC-based sequences %of length $M=80$ 
from our two-stage GA
are significantly lower than those of complex-valued random Gaussian and MUSA sequences.
The figure also shows that the proposed sequences slightly outperform the ZC sequences of prime length. % $M_{\rm ZC} = 79$.
In addition, Figs.~\ref{fig:all_J} and \ref{fig:all_M} depict the AER and NMSE over the number of BS antennas
and the sequence length, respectively,
which also confirm the excellent performance of Fourier- and ZC-based sequences.
%In Figs.~\ref{fig:all_snr} and \ref{fig:all_J}, 
%it appears that the Fourier- and ZC-based sequences of length $M=80$ outperform the ZC sequences of prime length $M_{\rm ZC} = 79$.
%However, Fig.~\ref{fig:all_M} 
%All the figures reveal that the performance gain is due to the use of one more subcarriers.
Taking into account the difference between $M$ and $M_{\rm ZC}$, %from the figures, 
we can say that %  Fig.~\ref{fig:all_M} reveals that
the AUD and CE performance of the proposed sequences are similar to
those of the ZC sequences of prime lengths.
%to that of the ZC sequences of prime lengths, while they show similar performance of CE (NMSE).


\begin{figure}[!t]
	\centering
	\includegraphics[width=0.475\textwidth, angle=0]{ALL_N500-M80-snr9-pa1-J.eps}
	\caption{Performance of CS-based AUD and CE of non-orthogonal sequences over the number of antennas, 
		where $N=500$, $M=80$ ($M_{ZC} = 79$), ${\rm SNR} = 9$ dB per device, and $p_a = 0.1$.}
	\label{fig:all_J}
\end{figure}

\begin{figure}[!t]
	\centering
	\includegraphics[width=0.475\textwidth, angle=0]{ALL_N500-snr5-J8-pa1-M.eps}
	\caption{Performance of CS-based AUD and CE of non-orthogonal sequences over the sequence length, 
		where $N=500$, $J=8$, ${\rm SNR} = 5$ dB per device, and $p_a = 0.1$. The prime lengths of ZC sequences are 
		$M_{\rm ZC} = 61, 71, 79, 89,$ and $101$, respectively.}
	\label{fig:all_M}
\end{figure}


\subsection{Discussion}
The simulation results of this section demonstrated that the Fourier- and ZC-based sequences
designed by our two-stage GA outperform complex-valued random Gaussian and MUSA sequences
for CS-based AUD and CE. % in uplink grant-free access.
Also, we observed that the performance of the proposed sequences is similar to that of   
the ZC sequences of prime lengths. 

In comparison to the ZC sequences, we would like to point out other potential
benefits of the proposed sequences.
First, the non-orthogonal sequences from our two-stage GA
have no limit to the granularity of sequence length, % inherited from arbitrary row selection of a unitary matrix,
which suggests that 
one can obtain the sequences of arbitrary length, 
according to %the number of available resource subcarriers.
the availability of resource subcarriers.
Meanwhile, the length of the ZC sequences should be odd prime only,
which is less flexible for managing resource subcarriers. % in sequence length.
Being able to take arbitrary sequence lengths,
the proposed sequences are expected to
facilitate mMTC systems to manage the resources more effectively for grant-free access.

Second, the proposed sequences constitute a partial unitary matrix with a column mask
through arbitrary row selection,
whereas the ZC sequences of prime length form a deterministic matrix with low coherence.
%As a \emph{structured} random matrix~\cite{Duarte:struct}, 
While the coherence-based recovery guarantee of the deterministic matrix 
is limited by its theoretical bottleneck~\cite{Eldar:CS},
our partial unitary matrix can present theoretical recovery guarantee with higher sparsity,
which ensures reliable CS-based detection theoretically for more active devices in mMTC.
%is another (theoretical) benefit of proposed sequences.
%Although Fig.~\ref{fig:all_papr} shows that the PAPR property 
%is worse than that of well optimized ZC sequences of prime length, 
 %the numerical results demonstrated 
In summary, the proposed non-orthogonal sequences % from our two-stage GA
can be a good option for uplink grant-free access,
supporting any number of subcarriers and % in uplink grant-free access,
% obtained by our two-stage GA
providing theoretically guaranteed performance for CS-based AUD and CE.
%find that the proposed sequences have many advantages over
%the ZC sequences of prime length that have been adopted in practice for the superb performance.
%Thanks to the benefits,
%which has been confirmed by numerical results.
%outperforming the algebraically designed ones that have been well known for the superb performance.
Remarkably, our GA-based design offers a new set of non-orthogonal sequences
with many advantages over the ZC sequences of prime length,
which are known for the superb performance in practice.












\section{Conclusion}
This paper has presented
a two-stage genetic algorithm (GA) to design new non-orthogonal sequences
%for guaranteeing reliable performance for CS-based detection 
for uplink grant-free access in mMTC. %massive connectivity,
The first-stage GA is to find a subsampling index set 
for a partial unitary matrix that can be approximated to an ETF.
%approximate the optimized matrix to ETF.
%The cost function of GA is defined by the Frobenius norm of the upper triangular part of $\Abu^H \Abu - \Wbu $
%above the diagonal,
%where the off-diagonal entries of $\Wbu$ are the Welch bound equality (WBE).
The second-stage GA then tries to find a masking sequence to be commonly applied  
to each column of the partial unitary matrix from the first-stage,
in order to enhance the PAPR property of the resulting columns.
In each stage GA, a new cost function has been elaborately proposed 
to improve the optimized result.
Finally, the masked columns of the partial unitary matrix
are proposed as new non-orthogonal sequences for uplink grant-free access.
To the best of our knowledge, this is the first effort to apply the GA technique
to non-orthogonal sequence design for achieving low correlation 
and low PAPR properties simultaneously.


Simulation results demonstrated that the
partial Fourier and ZC matrices optimized by our first-stage GA guarantee reliable CS reconstruction
% than randomly subsampled unitary matrices
over a wide range of compression and sparsity ratios.
Also, we observed that the second-stage GA produces
the Fourier- and ZC-based sequences that have acceptable PAPR distributions
for multicarrier transmission.
Finally, we demonstrated that %the proposed sequences %obtained by our two-stage GA 
%outperform some random sequences 
%in CS-based AUD and CE for uplink grant-free access.
%While their detection performance is similar to that of the ZC sequences of prime lengths,
the Fourier- and ZC-based sequences exhibit reliable performance of CS-based AUD and CE in uplink grant-free access,
which can be suitable for massive connectivity.
%taking arbitrary sequence length
%and guaranteeing the performance theoretically and numerically.

The main benefits of this GA-based design are summarized as follows.

\begin{itemize}
	\item[$\bullet$] The non-orthogonal sequences obtained by this GA-based design
	present theoretical recovery guarantee for CS reconstruction by 
	forming a partial unitary matrix through arbitrary row selection.  
	Simulation results confirmed that the Fourier- and ZC-based sequences from this design
	show excellent performance
	of CS-based AUD and CE in uplink grant-free access. %, comparable to algebraically designed ZC sequences of prime lengths.  
%	The two-stage GA presents a new set of non-orthogonal sequences for massive connectivity. 
%	To the best of our knowledge, it is the first effort to apply the GA technique
%	to non-orthogonal sequence design for achieving low correlation 
%	and low PAPR simultaneously.
%	Numerical results demonstrated that the non-orthogonal sequences exhibit acceptable PAPR property
%	and guarantee reliable CS-based detection for uplink grant-free access.
	\item[$\bullet$] This GA-based design is able to generate non-orthogonal sequences of arbitrary length,
	which can be a good choice for sequence lengths for which algebraically designed sequences with low correlation are unknown.
	In practice, the sequences of arbitrary length can be useful
	for mMTC systems to manage the resources effectively.
	\item[$\bullet$] Based on unitary matrices, this GA-based design
	can offer unimodular sequences of finite phase with rich structure, which 
	%Compared to other algorithmic and DL-based designs, 
%	The sequences with such structure
	are suitable for cost efficient implementation in mMTC devices.	
%	\item[$\bullet$] The two-stage GA can reach a good solution 
%	of a subsampling index set and a masking sequence. Applied to unitary matrices of high dimension, 
%	the solution can yield a relatively large number of sequences. %, although the solution may not be a global optimum.
%	Meanwhile, the number of sequences generated by a DL-based design 
%	may be limited in general, since all the sequence elements need to be obtained through training. 
%	Therefore, this GA-based design can provide a larger number of non-orthogonal sequences than DL-based ones,
%	to support highly overloaded mMTC devices. 	 
\end{itemize}
 
 
While this GA-based design successfully presented new 
non-orthogonal sequences  
%of arbitrary length that
%exhibits acceptable PAPR distribution and 
%guarantees reliable CS-based detection,
%that can be suitable 
for grant-free massive connectivity,
a further study will be necessary to enhance the design method.
First, the PAPR of the Fourier- and ZC-based sequences, although improved through evolution,
still needs to be reduced further.
It will be challenging, but necessary to devise 
a more elaborate method for further PAPR reduction
so that the PAPR distribution of the resulting sequences can be as good as that of the ZC sequences of prime lengths.
Second, this GA-based design using the Fourier or ZC matrices increases the number of phases of 
sequence elements as more sequences are required to support more mMTC devices.
To resolve this issue, we have employed the Hadamard matrices for our two-stage GA,
but found that the performance of CS-based detection is worse than that of Fourier or ZC matrices.
To obtain sequences of small phase,
we may need to study further with another unitary matrix 
of high dimension but with each element of smaller phase. 
Third, the proposed non-orthogonal sequences are for a single-cell massive connectivity,
but a further study will be necessary for this GA-based design % so that it can 
to provide multiple sets of non-orthogonal sequences 
for multi-cell environments.
Finally, %we may study a DL-based sequence design 
our two-stage GA can be considered 
as a component of DL-based sequence design, 
%for the state-of-the-art DL techniques 
%e.g., DL-based design, % replacing the CS-based detector,
%to design new non-orthogonal sequences, % provides the reference data sets, 
which is our ongoing research work.



%\section*{Acknowledgment}
%The preferred spelling of the word ``acknowledgment'' in American English is without an ``e'' after the ``g.'' Use the singular heading even if you have many %acknowledgments. Avoid expressions such as ``One of us (S.B.A.) would like to thank . . . .'' Instead, write “F. A. Author thanks ... .” In most cases, sponsor and %financial support acknowledgments are placed in the unnumbered footnote on the first page, not here.




%\section*{References}

\begin{thebibliography}{1}
	
	\bibitem{Kunz:MTC}
	T. Taleb and A. Kunz,
	``Machine type communications in 3GPP networks: Potential, challenges, and solutions,''
	\emph{IEEE Commun. Mag.}, vol. 50, no. 3, pp. 178-184, Mar. 2012.
	
	\bibitem{Bockel:mMTC}
	C.~Bockelmann \emph{et al.}, %  N.~Pratas, H.~Nikopour, K.~Au, T.~Svensson, C~Stefanovic, P.~Popovski, and A.~Dekorsy,
	``Massive machine-type communications in 5G: Physical and MAC-layer solutions,''
	\emph{IEEE Commun. Mag.}, vol. 54, no. 9, pp. 59-65, Sep. 2016.	
	
	\bibitem{3gpp:22368}
	3GPP TS 22.368, V13.1.0,
	\emph{Service Requirements for Machine-type Communications (MTC)}, Release 13, 2015.	
	
	\bibitem{Dai:noma}
	L. Dai, B. Wang, Y. Yuan, S. Han, C.-L. I, and Z. Wang,
	``Non-orthogonal multiple access for 5G: Solutions, challenges, opportunities, and future research trends,''
	\emph{IEEE Commun. Mag.}, vol. 53, no. 9, pp. 74-81, Sep. 2015.
	
	\bibitem{Dai:survey}
	L.~Dai, B.~Wang, Z.~Ding, Z.~Wang, S.~Chen, and L.~Hanzo,
	``A survey of non-orthogonal multiple access for 5G,''
	\emph{IEEE Commun. Surveys \& Tut.}, vol.~20, no.~3, pp.~2294-2323, 2018.
	
	\bibitem{Baligh:SCMA}
	H.~Nikopour and H.~Baligh,
	``Sparse code multiple access,''
	\emph{Proc. IEEE 24th Int. Symp. Pers. Indoor Mobile Radio Commun. (IEEE PIMRC)},
	London, U.K., Sep. 2013, pp.~332-336.
	
	\bibitem{Zhang:MPA} 
	S.~Zhang \emph{et. al.}, 
	``Sparse code multiple access: An energy efficient uplink approach for 5G wireless systems,''
	\emph{Proc. IEEE Glob. Commun. Conf. (IEEE GLOBECOM)},
	Austin, TX, USA, Dec. 2014, pp.~4782-4787.
	
	\bibitem{Wei:SCMA}
	F.~Wei and W.~Chen,
	``A low complexity SCMA decoder based on list sphere decoding,''
	\emph{Proc. IEEE Glob. Commun. Conf. (IEEE GLOBECOM)},
	Washington, DC, USA, Dec. 2016, pp.~1-6.
	
	\bibitem{Yuan:MUSA}
	Z.~Yuan, G.~Yu, W.~Li, Y.~Yuan, X.~Wang, and J.~Xu,
	``Multi-user shared access for Internet of Things,''
	\emph{IEEE 83rd Veh. Technol. Conf. (VTC Spring)}, pp.~1-5, China, May 15-18, 2016.
	
	\bibitem{Kang:PDMA} 
	S.~Kang, X.~Dai, and B.~Ren,
	``Pattern division multiple access for 5G,''
	\emph{Telecommun. Netw. Technol.}, vol.~5, no.~5, pp.~43-47, May 2015.
	
	\bibitem{Jamali:PDMA}
	M.~V.~Jamali and H.~Mahdavifar, 
	``A Low-complexity recursive approach toward code-domain NOMA for massive communications,''
	\emph{Proc. IEEE Glob. Commun. Conf. (IEEE GLOBECOM)},
	Abu Dhabi, United Arab Emirates, 2018, pp. 1-6.
	
	
	\bibitem{Bengjo:DL}
	I. Goodfellow, Y. Bengjo, and A. Courville, 
	\emph{Deep Learning}, Cambridge, MA, USA: MIT Press, 2016.
	
	
	\bibitem{Ye:deep}
	N.~Ye, X.~Li, H.~Yu, L.~Zhao, W.~Liu, and X.~Hou,
	``DeepNOMA: A unified framework for NOMA using deep multi-task learning,''
	\emph{IEEE Trans. Wireless Commun.}, vol. 19, no. 5, pp. 2208-2225, Apr. 2020.
	
	\bibitem{Kim:dnn}
	W.~Kim, Y.~Ahn, and B.~Shim,
	``Deep neural network-based active user detection for grant-free NOMA systems,''
	\emph{IEEE Trans. Commun.}, vol. 68, no. 4, pp. 2143-2155, Apr. 2020.
	
	\bibitem{Siva:dnn}
	T.~Sivalingam, S.~Ali, N.~H.~Mahmood, N.~Rajatheva, and M.~Latva-Aho,
	``Deep neural network-based blind multiple user detection for grant-free multi-user shared access,''
	\emph{arXiv:2106.11204v1 [cs.IT]}, Jun. 2021.	

	
	
	\bibitem{Cirik:toward}
	A.~C.~Cirik, N.~M.~Balasubramanya, L.~Lampe, G.~Vos, and S. Bennett, 
	``Toward the standardization of grant-free operation and the associated NOMA strategies in 3GPP,'' 
	\emph{IEEE Communications Standards Magazine}, vol. 3, no. 4, pp. 60-66, Dec. 2019.
	
	\bibitem{Eldar:CS}
	Y.~C.~Eldar and G.~Kutyniok,
	\emph{Compressed Sensing: Theory and Applications}, 
	Cambridge, UK: Cambridge Univ. Press, 2012. 
	
	
	\bibitem{Abebe:iter}
	A. T. Abebe and C. G. Kang,
	``Iterative order recursive least square estimation for exploiting frame-wise sparsity in compressive sensing-based MTC,''
	\emph{IEEE Commun. Lett.}, vol. 20, no. 5, pp. 1018-1021, May 2016.
	
	%\newpage
	

	
	\bibitem{Wei:amp}
	C. Wei, H. Liu, Z. Zhang, J. Dang, and L. Wu,
	``Approximate message passing-based joint user activity and data detection for NOMA,''
	\emph{IEEE Commun. Lett.}, vol. 21, no. 3, pp. 640-643, Mar. 2017.
	

	
	\bibitem{Cirik:admm}
	A. Cirik, N. M. Balasubramanya, and L. Lampe,
	``Multi-user detection using ADMM-based compressive sensing for uplink grant-free NOMA,''
	\emph{IEEE Wireless Commun. Lett.}, vol. 7, no. 1, pp. 46-49, Feb. 2018.
	
	\bibitem{Liu:mimo}
	L. Liu and W.~Yu, 
	``Massive connectivity with massive MIMO - part I: Device activity detection and channel estimation,'' 
	\emph{IEEE Trans. Signal Process.}, vol. 66, no. 11, pp. 2933-2946, June 2018.	
	
	\bibitem{Liu:massive}
	L. Liu, E. G. Larsson, W. Yu, P. Popovski, C. Stefanovic and E. de Carvalho, 
	``Sparse signal processing for grant-free massive connectivity: 
	A future paradigm for random access protocols in the internet of things,'' 
	\emph{IEEE Signal Process. Mag.}, vol. 35, no. 5, pp. 88-99, Sept. 2018.
	
	\bibitem{Ahn:ep}
	J. Ahn, B. Shim, and K. B. Lee,
	``EP-based joint active user detection and channel estimation for massive machine-type communications,''
	\emph{IEEE Trans. Commun.}, vol. 67, no. 7, pp. 5178-5189, July 2019.	
	
	\bibitem{Shao:dim}
	X. Shao, X. Chen and R. Jia, 
	``A dimension reduction-based joint activity detection and channel estimation algorithm for massive access,'' 
	\emph{IEEE Trans. Signal Process.}, vol. 68, pp. 420-435, 2020.
	
	\bibitem{Jiang:noma}
	S. Jiang, X. Yuan, X. Wang, C. Xu, and W. Yu,
	``Joint user identification, channel estimation, and signal detection for grant-free NOMA,''
	\emph{IEEE Trans. Wirel. Commun.}, vol. 19, no. 10, pp. 6960-6976, Oct. 2020.	
	
	
	\bibitem{Wang:struct}
	B. Wang, L. Dai, T. Mir, and Z. Wang,
	``Joint user activity and data detection based on structured compressive sensing for NOMA,''
	\emph{IEEE Commun. Lett.}, vol. 20, no. 7, pp. 1473-1476, July 2016.	
	
	\bibitem{Wang:dynamic}
	B. Wang, L. Dai, Y. Zhang, T. Mir and J. Li, 
	``Dynamic compressive sensing-based multi-user detection for uplink grant-free NOMA,''
	\emph{IEEE Commun. Lett.}, vol. 20, no. 11, pp. 2320-2323, Nov. 2016.
	
	\bibitem{Du:efficient}
	Y. Du \emph{et al.}, 
	``Efficient multi-user detection for uplink grant-free NOMA: 
	Prior-information aided adaptive compressive sensing perspective,''
	\emph{IEEE J. Sel. Areas Commun.}, vol. 35, no. 12, pp. 2812-2828, Dec. 2017.
	
	\bibitem{Du:joint}
	Y.~Du, B.~Dong, W.~Zhu, P.~Gao, Z.~Chen, X.~Wang, and J.~Fang,
	``Joint channel estimation and multiuser detection for uplink grant-free NOMA,''
	\emph{IEEE Wireless Commun. Lett.}, vol. 7, no. 4, pp. 682-685, Aug. 2018.
	
	\bibitem{Du:block}
	Y. Du, C. Cheng, B. Dong, Z. Chen, X. Wang, J. Fang, and S. Li,
	``Block-sparsity-based multiuser detection for uplink grant-free NOMA,''
	\emph{IEEE Trans. Wireless Commun.}, vol. 17, no. 12, pp. 7894-7909, Dec. 2018.	
	

		
	\bibitem{Yu:blind}
	N. Y. Yu,
	``Multiuser activity and data detection via sparsity-blind greedy recovery for uplink grant-free NOMA,''
	\emph{IEEE Commun. Lett.}, vol. 23, no. 11, pp. 2082-2085, Nov. 2019.
	

%	\bibitem{Quayum:pilot}
%	A.~Quayum, H.~Minn, and Y.~Kakishima,
%	``Non-orthogonal pilot designs for joint channel estimation and collision detection in grant-free access systems,''

%	\bibitem{Ding:anal}
%	J. Ding, D. Qu and J. Choi, 
%	``Analysis of non-orthogonal sequences for grant-gree RA with massive MIMO,'' 
%	\emph{IEEE Trans. Commun.}, vol. 68, no. 1, pp. 150-160, Jan. 2020.







	\bibitem{Schober:meet}
	M.~Ke, Z.~Gao, Y.~Wu, X.~Gao, and R.~Schober, 
	``Compressive sensing-based adaptive active user detection and channel estimation: Massive access meets massive MIMO,''
	\emph{IEEE Trans. Signal. Process.}, vol.~68, pp.~764-779, 2020.



	
	\bibitem{Litsyn:peak}
	S. Litsyn, 
	\emph{Peak Power Control in Multicarrier Communications},
	Cambridge University Press, 2007.
	
	
	\bibitem{No:papr}
	G. Wunder, R.~F.~H. Fischer, H. Boche, S. Litsyn and J.-S. No, 
	``The PAPR problem in OFDM transmission: New directions for a long-lasting problem''
	\emph{IEEE Signal Process. Mag.}, vol. 30, no. 6, pp. 130-144, Nov. 2013.
	
	\bibitem{Yani:ra}
	A.~S.Rajasekaran, M.~Vameghestahbanati, M.~Farsi, H.~Yanikomegoglu, and H.~Saeedi,
	``Resource allocation-based PAPR analysis in uplink SCMA-OFDM systems,''
	\emph{IEEE Access}, vol.~7, pp.~162803-162817, 2019.
	
	\bibitem{Mherich:golden}
	Z.~Mheich, L.~Wen, P.~Xiao, and A.~Maaref,
	``Design of SCMA codebooks based on golden angle modulation,''
	\emph{IEEE Trans. Veh. Technol.}, vol.~68, no.~2, pp.~1501-1509, Feb. 2019.		
	
	\bibitem{Yang:qos}
	K.~Yang, Y.-K. Kim, and P. V. Kumar,
	``Quasi-orthogonal sequences for code-division multiple-access system,''
	\emph{IEEE Trans. Inf. Theory}, vol.~46, no.~3, pp. 982-993, May 2000.


	
	%\bibitem{Apple:async}
	%L. Applebaum, W. U. Bajwa, M. F. Duarte, R. Calderbank,
	%``Asynchronous code-division random access using convex optimization,''
	%\emph{Physical Communication}, vol. 5, pp. 129-147, 2012.
	

	
	\bibitem{Golay:series}
	M.~J.~E.~Golay,
	``Complementary series,''
	\emph{IRE Trans. Inf. Theory}, vol.~IT-7, pp.~82-87, 1961.
	
	\bibitem{Jedwab:RM}
	J.~A.~Davis and J.~Jedwab,
	``Peak-to-mean power control in OFDM, Golay complementary sequences, and Reed-Muller codes,''
	\emph{IEEE Trans. Inf. Theory}, vol.~45, no.~7, pp.~2397-2417, Nov. 1999.
	
	\bibitem{Paterson:gen}
	K.~G.~Paterson,
	``Generalized Reed-Muller codes and power control in OFDM modulation,''
	\emph{IEEE Trans. Inf. Theory}, vol.~46, no.~1, pp.~104-120, Jan. 2000.
	
	\bibitem{Liu:FBMC}
	Z.~Liu, P.~Xiao, and S.~Hu,
	``Low-PAPR preamble design for FBMC systems,''
	\emph{IEEE Trans. Veh. Technol.}, vol. 68, no. 8, pp. 7869-7876, Aug. 2019.
	
	\bibitem{Yu:binary}
	N.~Y.~Yu,
	``Binary Golay spreading sequences and Reed-Muller codes for uplink grant-free NOMA,'' 
	\emph{IEEE Trans. Commun.}, vol. 69, no. 1, pp. 276-290, Jan. 2021.
	
	\bibitem{Yu:non}
	N. Y. Yu,
	``Non-orthogonal Golay-based spreading sequences for uplink grant-free access,''
	\emph{IEEE Commun. Lett.}, vol. 24, no. 10, pp. 2104-2108, Oct. 2020.	
	
	\bibitem{Liu:comp}
	Z.~Liu, Y.~L~Guan, and U.~Parampalli,
	``New complete complementary codes for peak-to-mean power control in multicarrier CDMA,''
	\emph{IEEE Trans. Commun.}, vol. 62, no. 3, pp. 1105-1113, Mar. 2014.
	
	\bibitem{Li:zcz}
	Y. Li and C. Xu, 
	``ZCZ aperiodic complementary sequence sets with low column sequence PMEPR,'' 
	\emph{IEEE Commun. Lett.}, vol. 19, no. 8, pp. 1303-1306, Aug. 2015.
	
	
	\bibitem{Jiang:novel}
	T. Jiang, C. Ni and Y. Xu, 
	``Novel 16-QAM and 64-QAM near-complementary sequences with low PMEPR in OFDM systems,''
	\emph{IEEE Trans. Commun.}, vol. 64, no. 10, pp. 4320-4330, Oct. 2016.
	
	
	\bibitem{Chen:comp}
	C. Chen, 
	``Complementary sets of non-power-of-two length for peak-to-average power ratio reduction in OFDM,''
	\emph{IEEE Trans. Inf. Theory}, vol. 62, no. 12, pp. 7538-7545, Dec. 2016.
	
	
	\bibitem{Wu:Z}
	S.~Wu and C.~Chen, 
	``Optimal Z-complementary sequence sets with good peak-to-average power-ratio property,''
	\emph{IEEE Signal Process. Lett.}, vol. 25, no. 10, pp. 1500-1504, Oct. 2018.
	
	\bibitem{Chu:ZC}
	C.~Chu,
	``Polyphase codes with good periodic correlation properties,''
	\emph{IEEE Trans. Inf. Theory}, vol.~18, no.~4, pp. 531-532, Jul. 1972.
	
	\bibitem{3gpp:36.211}
	3GPP TS 36.211 V13.1.0,
	\emph{Physical Channel and Modulation}, Mar. 2016.	
	
		
	\bibitem{Elad:opt}
	M.~Elad,
	``Optimized projections for compressed sensing,''
	\emph{IEEE Trans. Sig. Proc.}, vol.~55, no.~12, pp.~5695-5702, Dec. 2007.
	
	\bibitem{Xu:opt}
	J.~Xu, Y.~Pi, and Z.~Cao,
	``Optimized projection matrix for compressive sensing,''
	\emph{EURASIP J.~Adv.~Signal Process.}, vol.~2010, 2010.
	
	\bibitem{Vahid:grad}
	V.~Abolghasemi, S.~Ferdowsi, and S.~Sanei,
	``A gradient-based alternating minimization approach for optimization of the measurement matrix in compressive sensing,''
	\emph{Signal Process.}, vol.~92, pp.~999-1009, 2012.
	
	\bibitem{Li:opt}
	G.~Li, Z.~Zhu, D.~Yang, L.~Chang, and H.~Bai,
	``On projection matrix optmization for compressive sensing systems,''
	\emph{IEEE Trans. Sig. Proc.}, vol.~61, no.~11, pp.~2887-2898, Jun. 2013.	
	
	\bibitem{Zhang:opt}
	Q.~Xu, Z.~Sheng, Y.~Fang, and L.~Zhang,
	``Measurement matrix optimization for compressed sensing system with constructed dictionary via Takenaka-Malmquist functions,''
	\emph{Sensors} 21, no.~4: 1229, 2021. 
	
	
	\bibitem{Kov:frames}
	J.~Kovacevic and A.~Chebira,
	\emph{An Introduction to Frames}, Foundations and Trends in Signal Processing, now Publishers Inc., 2008.
	
	
	\bibitem{Chen:utf}
	W.~Chen, M.~R.~D.~Rodrigues, and I.~J.~Wassell,
	``On the use of unit-norm tight frames to improve the average MSE performance in compressive sensing applications,''
	\emph{IEEE Sig. Process. Lett.}, vol.~19, no.~1, pp.~8-11, Jan. 2012.	
	
	
	%	\bibitem{Eva:inc}
	%	E.~V.~Tsiligianni, L.~P.~Kondi, and ~A.~K.~Katsaggelos,
	%	``Construction of incoherent unit norm tight frames with application to compressed sensing,''
	%    \emph{IEEE Trans. Inf. Theory}, vol.~60, no.~4, pp.~2319-2330, Apr. 2014.
	
	\bibitem{Duarte:learn}
	J.~M.~Duarte-Carvajalino and G.~Saprio,
	``Learning to sense sparse signals: simultaneous sensing matrix and sparsifying dictionary optimization,''
	\emph{IEEE Trans. Imag. Process.}, vol.~18, no.~7, pp.~1935-1408, 2009.
	
	
	\bibitem{Lu:dir} 
	C.~Lu, H.~Li, and Z.~Lin,
	``Optimized projections for compressed sensing via direct mutual coherence minimization,''
	\emph{Signal Process.}, vol.~151, pp.~45-55, 2018.
	
	\bibitem{Chun:DL}
	C. Chun, J. Kang and I. Kim, 
	``Deep learning-based joint pilot design and channel estimation for multiuser MIMO channels," 
	\emph{IEEE Commun. Lett.}, vol. 23, no. 11, pp. 1999-2003, Nov. 2019.
	
	\bibitem{Kim:DL}
	N.~Kim, D.~Kim, B.~Shim and K.~B.~Lee, 
	``Deep learning-based spreading sequence design and active user detection for massive machine-type communications,'' 
	\emph{IEEE Wirel. Commun. Lett.}, Early Access, June 2021.
	
	\bibitem{Holland:gen}
	J.~H.~Holland, \emph{Adaptation in Natural and Artificial Systems: An Introductory Analysis
	With Applications to Biology, Control and Artificial Intelligence}, Cambridge, MA, USA: MIT Press, 1992.
	
	\bibitem{Chen:ISAR}
	Y.-J.~Chen, Q.~Zhang, Y.~Luo, and Y.-A.~Chen,
	``Measurement matrix optimization for ISAR sparse imaging based on genetic algorithm,''
	\emph{IEEE Geoscience Remote Sens. Lett.}, vol.~13, no.~12, pp.~1875-1879, Dec. 2016.
	
	\bibitem{Nie:maga}
	Y.~Nie, X~Yu, and Z.~Yang,
	``Deterministic pilot pattern allocation optimization for sparse channel estimation based on CS theory in OFDM system,''
	\emph{EURASIP J. Wirel. Commun. Netw.}, 2019:7, 2019.		
	
	\bibitem{Leven:moga}
	M.~A.~Mazaideh and J.~Levendovszky, 
	``A multi-hop routing algorithm for WSNs based on compressive sensing and multiple objective genetic algorithm,''
	\emph{Journ. of Commun. Netw.}, vol.~23, no.~2, pp.~138-147, Apr. 2021.	
	
	\bibitem{Welch:low}
	L.~R.~Welch,
	``Lower bounds on the maximum cross correlation of signals,''
	\emph{IEEE Trans. Inf. Theory}, vol.~IT-20, no.~3, pp.~397-399, May 1974.
	
	\newpage
	
%	\bibitem{Donoho:CS}
%	D.~L.~Donoho,
%	``Compressed sensing,''
%	\emph{IEEE Trans. Inf. Theory}, vol.~52, no.~4, pp.~1289-1306, Apr. 2006.	
%	
%	
%	\bibitem{CanRomTao:robust}
%	E.~J.~Candes, J.~Romberg, and T.~Tao,
%	``Robust uncertainty principles: Exact signal reconstruction from highly incomplete frequency information,''
%	\emph{IEEE Trans. Inf. Theory}, vol.~52, no.~2, pp.~489-509, Feb. 2006.
%	
%	
%	\bibitem{CanTao:univ}
%	E.~J.~Candes and T.~Tao,
%	``Near-optimal signal recovery from random projections: Universal encoding strategies?,''
%	\emph{IEEE Trans. Inf. Theory}, vol.~52, no.~12, pp.~5406-5425, Dec. 2006.
	

	
%	\bibitem{Baraniuk:rip}
%	R.~Baraniuk, M.~Davenport, R.~Devore, and M.~Wakin,
%	``A simple proof of the restricted isometry property for random matrices,''
%	\emph{Constructive Approximation}, vol.~28, no.~3, pp.~253-263, Dec. 2008.
	

	

%	
%	\bibitem{DeVore:det}
%	R.~A.~DeVore, 
%	``Deterministic constructions of compressed sensing matrices,''
%	\emph{J. Complexity}, vol. 23, pp. 918–925, 2007.
%	
%	\bibitem{Apple:chirp}
%	L.~Applebaum, S.~D.~Howard, S.~Searle, and R.~Calderbank, 
%	``Chirp sensing codes: Deterministic compressed sensing measurements for fast recovery,''
%	\emph{Appl. Comput. Harmon. Anal.}, vol. 26, pp. 283–290, 2009.
%	
%	\bibitem{Amini:block}
%	A.~Amini, V.~Montazerhodjat, and F.~Marvasti,
%	``Matrices with small coherence using $p$-ary block codes,''
%	\emph{IEEE Trans. Sig. Proc.}, vol.~60, no.~1, pp.~172-181, Jan. 2012.
%	
%    \bibitem{Yu:OOC}
%    N.~Y.~Yu and N.~Zhao,
%    ``Deterministic construction of real-valued ternary sensing matrices,''
%    \emph{IEEE Sig. Process. Lett.}, vol.~20, no.~11, pp.~1106-1109, Nov. 2013.
%    
%    \bibitem{Donoho:opt}
%    D.~Donoho and M.~Elad,
%    ``Optimally sparse representation in general (nonorthogonal) dictionaries via $l_2$ minimization,
%    \emph{Proc. Natl. Acad. Sci.}, vol.~100, no.~5, pp.~2197-2202, 2003. 


	
	\bibitem{Duarte:struct}
	M.~Duarte and Y.~C.~Eldar,
	``Structured compressed sensing: From theory to applications,''
	\emph{IEEE Trans. Signal Process.}, vol.~59, no.~9, pp.~4053-4085, Sep. 2011.	
	
	
%	\bibitem{Romb:rand}
%	J.~Romberg,
%	``Compressive sensing by random convolution,''
%	\emph{SIAM J.~Imag.~Sci.}, vol.~2, no.~4, pp.~1098-1128, Dec. 2009.
%	
%	\bibitem{Zhang:utf}
%	P.~Zhang, L.~Gan, S.~Sun, and C.~Ling,
%	``Modulated unit-norm tight frames for compressed sensing,''
%	\emph{IEEE Trans. Signal Process.}, vol.~63, no.~15, pp.~3974-3985, Aug. 2015.	
	
	
	\bibitem{Rud:sparse}
	M.~Rudelson and R.~Vershynin,
	``On sparse reconstruction from Fourier and Gaussian measurements,''
	\emph{Comm. Pure Appl. Math.}, vol.~61, no.~8, pp.~1025-1045, Aug. 2008.
		
%	\bibitem{Litsyn:maxima}
%	S. Litsyn and A. Yudin,
%	``Discrete and continuous maxima in multicarrier communication,''
%	\emph{IEEE Trans. Inf. Theory}, vol.~51, no.~3, pp.~919-928, Mar. 2005.
	

	
	\bibitem{Freitas:survey}
	A.~A.~Freitas,
	``A survey of evolutionary algorithms for data mining and knowledge discovery,''
	\emph{Advances in Evolution. Comput.}, Springer-Verlag, pp.~819-845, 2002.
	
	\bibitem{Yang:entropy}
	L.~Yang, D.~H.~Widyantoro, T.~Ioerger, and J.~Yen,
	``An entropy-based adaptive genetic algorithm for learning classification rules,''
	\emph{Proc. 2001 Congress Evolution. Comput.}, pp.~790-796, 2001.
	
	\bibitem{Viv:intel}
	P.~Vivekanandan, M.~Rajalakshmi, and R.~Nedunchezhian,
	``An intelligent genetic algorithm for mining classification rules in large data sets,''
	\emph{Computing and Informatics}, vol.~32, pp.~1-22, 2013.
	

	\bibitem{Hebbes:turbo}
	L.~Hebbes, R.~.R.~Malyan, and A.~P.~Lenaghan,
	``Genetic algorithms for turbo codes,''
	\emph{Proc. Int Conf. Comput. Tool.}, vol.~1, pp.~478-481, Nov. 2005.

    \bibitem{Maini:linear}
    H.~Maini, K.~Mehrotra, C.~Mohan, and S.~Ranka,
    ``Genetic algorithms for soft-decision decoding of linear block codes,''
    \emph{Evol. Comput.}, vol.~2, no.~2, pp.~145-164, Jun. 1994.
    
    \bibitem{scan:ldpc}
    A.~Scandura, A.~D.~Pra, L.~Arnone, L.~I.~Passoni, and J.~C.~Moreira,
    ``A genetic-algorithm based decoder for low density parity check codes,''
    \emph{Latin Amer. Appl. Res.}, vol.~36, no.~3, pp.~169-172, Jul. 2006.
    
    \bibitem{brink:polar}
    A.~Elkelesh, M.~Ebada, S.~Crammerer, and S.~ten Brink,
    ``Decoder-tailored polar code design using the genetic algorithm,''
   	\emph{IEEE Trans. Commun.}, vol.~67, no.~7, pp.~4521-4534, Jul. 2019.	
   	
   	\bibitem{Dam:seq}
   	H.~H.~Dam, H.-J.~Zepernick, and H.~Luders,
   	``On the design of complex-valued spreading sequences using a genetic algorithm,''
   	\emph{IEEE ISSTA}, pp.~704-707, Sydney, Australia, Aug., 2004.
   	
   	\bibitem{Nat:evol}
   	B.~Natarajan, S.~Das, and D.~Stevens,
   	``An evolutionary approach to designing complex spreading codes for DS-CDMA,''
   	\emph{IEEE Trans. Wirel. Commun.,}, vol.~4, no.~5, pp.~2051-2056, Sep. 2005.
   	
    \bibitem{Conde:sparse}
    M.~H.~Conde and O.~Loffeld,
    ``A genetic algorithm for compressive sensing sparse recovery,''   	
    \emph{IEEE Int. Symp. Signal Process. Inf. Technol.}, pp.~106-111, 2017.
    
    \bibitem{You:fusion}
    H.~You and J.~Zhu,
    ``A genetic approach to fusion of algorithms for compressive sensing,''
    \emph{Int. Symp. Neural Netw.}, \emph{Advances in Neural Networks-ISNN2017}, 
    pp.~371-379, Springer, 2017.
    
    \bibitem{Erkoc:evol}
    M.~E.~Erkoc and N.~Karaboga,
    ``Evolutionary algorithms for sparse signal reconstruction,''
    \emph{Signal, Image and Video Processing}, vol.~13, pp.~1293-1301, Springer-Verlag London Ltd., 2019.
	
		
%	\bibitem{GolGong:seq}
%	S.~W.~Golomb and G.~Gong, 
%	\emph{Signal Design for Good Correlation - for Wireless Communication, Cryptography, and Radar},
%	Cambridge University Press, 2005.
	
%	\bibitem{Tropp:omp}
%	J.~A.~Tropp and A.~C.~Gilbert,
%	``Signal recovery from random measurements via orthogonal matching pursuit,''
%	\emph{IEEE Trans. Inf. Theory}, vol.~53, no.~12, pp.~4655-4666, Dec. 2007.
	
%	\bibitem{Berg:spgl1}
%	E.~van den Berg and M.~P.~Friedlander,
%	``Probing the Pareto frontier for basis pursuit solutions,''
%	\emph{SIAM Journal on Scientific Computing}, vol.~31, no.~2, pp.~890-912, 2008.

%	\bibitem{Berg:solver}
%	E.~van den Berg and M.~P.~Friedlander,
%	``SPGL1: A solver for large-scaler sparse reconstruction'', Available in https://friedlander.io/spgl1. 

	\bibitem{Tropp:somp}
	J.~A.~Tropp, A.~C.~Gilbert, and M.~J.~Strauss,
	``Algorithms for simultaneous sparse approximation. Part I: Greedy pursuit,''
	\textit{Signal Process.}, vol.~86, pp.~572-588, Apr. 2006.	
	
	
	\bibitem{Sarwate:bound}
	D. V. Sarwate,
	``Bounds on crosscorrelation and autocorrelation of sequences,''
	\emph{IEEE Trans. Inf. Theory}, vol.~IT-25, no. 6, pp. 720-724, Nov. 1979.
	
	
	
	%\bibitem{CanTao:dec}
	%E.~J.~Candes and T.~Tao,
	%``Decoding by linear programming,''
	%{\em IEEE Trans. Inf. Theory}, vol. 51, no. 12, pp.~4203-4215, Dec. 2005.
	
%	
%	\bibitem{Felix:suprema}
%	F.~Krahmer, S.~Mendelson and H.~Rauhut,
%	``Suprema of chaos processes and the restricted isometry property,''
%	to appear in \emph{Communications on Pure and Applied Mathematics.} Available at
%	http://arxiv.org/abs/1207.0235.
%	
%	\bibitem{LiGan:conv}
%	K.~Li, L.~Gan, and C.~Ling
%	``Convolutional compressed sensing using deterministic sequences,''
%	\emph{IEEE Trans. Signal Process.}, vol.~61, no.~3, pp.~740-752, Feb. 2013.
%	
%	\bibitem{Chu:FZC}
%	C.~Chu,
%	``Polyphase codes with good periodic correlation properties,''
%	{\em IEEE Trans. Inf. Theory}, vol. 18, no. 4, pp.~531-532, Jul. 1972.
%	
%	
%	\bibitem{Golay}
%	M.~J.~E.~Golay,
%	``Golay complementary series,''
%	\emph{IEEE Trans. Inf. Theory}, vol.~7, no.~2, pp.~82-87, Apr. 1961.
%	
%	
%	
%	
%	
%	\bibitem{Sidel:org}
%	V.~M.~Sidelnikov,
%	``Some $k$-valued pseudo-random sequences and nearly equidistant codes,''
%	\emph{Probl. Inf. Transm.}, vol.~5, pp.~12-16, 1969.
%	

%	
%
%	
%	\bibitem{Candes:coh-l1}
%	E. J. Candes and Y. Plan,
%	``Near-ideal model selection by $l_1$ minimization,''
%	\emph{Ann. Statist.}, vol.~37, no.~5A, pp.~2145-2177, 2009.
%	
%	
%	\bibitem{mat-Pen}
%	Y. Hua and T. K. Sarkar,
%	``Matrix pencil method for estimating parameters of exponentially damped/undamped sinusoids in noise,''
%	\emph{IEEE Trans. Acoust., Speech, Signal Process.}, vol.~38, no.~5, pp.~814-824, May 1990.
%	
%	
%	\bibitem{weidai-SP}
%	W.~Dai and O.~Milenkovic,
%	``Subspace pursuit for compressive sensing signal reconstruction,"
%	\emph{IEEE Trans. Inf. Theory}, vol.~55, no.~5, pp.~2230-2249, May 2009.
%	
%	
%	\bibitem{Lidl:FF}
%	R.~Lidl and H.~Niederreiter,
%	\emph{Finite Fields,}
%	in \emph{Encyclopedia of Mathematics and Its Applications},
%	vol.~20,
%	Cambridge University Press, 1997.
%	
%	
%	\bibitem{Weil:BNT}
%	A.~Weil,
%	\emph{Basic Number Theory}, 3rd. Ed., Springer-Verlag, 1974.
%	
%	\bibitem{Wan:Weil}
%	D.~Wan,
%	``Generators and irreducible polynomials over finite fields,''
%	\emph{Math. Comput.}, vol.~66, no.~219, pp.~1195-1212, Jul. 1997.
%	
	%\bibitem{Schmidt:eq}
	%W.~M.~Schmidt,
	%\emph{Equations Over Finite Fields}, Lecture Notes in Mathematics, Springer-Verlag, 1976.
	
	
	%\bibitem{Lempel:binary}
	%A.~Lempel, M.~Cohn, W.~L.~Eastman,
	%``A class of binary sequences with optimal autocorrelation properties,''
	%\emph{IEEE Trans. Inf. Theory}, vol.~23, pp.~38-42, 1997.
	
	
	
	
	%\bibitem{Alltop}
	%W.~Alltop,
	%``Complex sequences with low periodic correlations,''
	%\emph{IEEE Trans. Inf. Theory}, vol.~26, no.~3, pp.~350-354, May 1980.
	
	
	%\bibitem{Garbor}
	%T.~Strohmer and R.~Heath,
	%``Grassmanian frames with applications to coding and communication,''
	%\emph{Appl. Comput. Harmon. Anal.}, vol.~14, no.~3, pp.~257-275, May 2003.
	
	
	
	%\bibitem{Cald:sub}
	%R.~Calderbank, S.~Howard, and S.~Jafarpour,
	%``A sublinear algorithm for sparse reconstruction with $l_2/l_2$ recovery guarantees,''
	%\emph{arXiv:0806.3799v2 [cs.IT]}, Oct. 2009.
	
	
	%\bibitem{Ailon:BCH}
	%N.~Ailon and E.~Liberty,
	%``Fast dimension reduction using Rademacher series on dual BCH codes,''
	%\emph{Annual ACM-SIAM Symposium on Discrete Algorithms (SODA)}, pp.~215-224, Jan.~2008.
	
	
	%\bibitem{Gur:osc}
	%S.~Gurevich, R.~Hadani, and N.~Sochen,
	%``On some determistic dictionaries supporting sparsity,''
	%\emph{J. Fourier Anal. Appl.}, vol.~14, no.~5-6, pp.~859-876, 2008.
	
	
	
	%\bibitem{Yu:mult}
	%N.~Y.~Yu,
	%''Deterministic compressed sensing matrices from multiplicative character sequences,''
	%\emph{45th Annual Conference on Information Sciences and Systems (CISS)}, Johns Hopkins University, Baltimore, MD, Mar. 2011.
	
	
	
	%\bibitem{HellKumar:low}
	%T. Helleseth and P. V. Kumar, ``Sequences with low correlation,''
	%a chapter
	%in \emph{Handbook of Coding Theory}. Edited by V.~Pless and C.~Huffman.
	%Elsevier Science Publishers, 1998.
	
	
	
	%\bibitem{Mixon:finger}
	%D.~G.~Mixon, C.~J.~Quinn, N.~Kiyavash, and M.~Fickus,
	%``Fingerprinting with equiangular tight frames,''
	%\emph{arXiv:1111.3376v1}, Nov. 2011.
	
	
	%\bibitem{SebYam:Had}
	%J.~Seberry and M.~Yamada,
	%``Hadamard matrices, sequences, and block designs,''
	%a chapter in \emph{Contemporary Design Theory: A Collection of Surveys},
	%J.~H.~Dinitz and D.~R.~Stinson, Eds. John Wiley \& Sons, Inc., 1992.
	
	%\bibitem{Mac:ECC}
	%F.~J.~MacWillams and N.~J.~A.~Sloane,
	%\emph{The Theory of Error Correcting Codes},
	%Amsterdam, The Netherlands: North Holland, 1986.
	
	%\bibitem{Golay}
	%M.~J.~E.~Golay,
	%``Complementary series,''
	%\emph{IRE Trans. Inf. Theory}, pp.~82-87, Apr. 1961.
	
	%\bibitem{ParkKen:GCS}
	%M.~G.~Parker, K.~G.~Paterson, and C.~Tellambura,
	%``Golay complementary sequences,''
	%\emph{Wiley Encyclopedia of Telecommunications}, Edited by J.~G.~Proakis, Wiley Interscience, 2002.
	
	%\bibitem{OchImai:ofdm}
	%H.~Ochiai and H.~Imai,
	%``OFDM-CDMA with peak power reduction based on the spreading sequences,''
	%\emph{IEEE International Conferences on Communications (ICC)}, vol.~3, pp.~1299-1303, Apr. 1998.
	
	
	
	
	
	
	%\bibitem{Do:samp}
	%T.~T.~Do, L~.Gan, N.~Nguyen and T.~D.~Tran,
	%``Sparsity adaptive matching pursuit algorithm for practical compressed sensing,''
	
	
	
	%\bibitem{YuFeng:ADS}
	%N.~Y.~Yu, K.~Feng, and A.~Zhang,
	%``A new class of near-optimal partial Fourier codebooks from an almost difference set,''
	%\emph{Des. Codes Cryptogr.}, Online first, Sep. 2012.
	
	
	
	
	
	%\bibitem{Abel:ooc}
	%R.~J.~R.~Abel and M.~Buratti,
	%``Some progress on $(v, 4, 1)$ difference families and optical orthogonal codes,''
	%\emph{J.~Combin.~Theory A}, pp.~59-75, 2004.
	
	%\bibitem{Kov:Intro}
	%J.~Kovacevic and A.~Chebira,
	%\emph{An Introduction to Frames}, Foundations and Trends in Signal Processing, vol.~2, no.~1, 2008.
	
	
	
	
	%\bibitem{Grass}
	%T.~Strohmer and R.~Heath,
	%``Grassmannian frames with applications to coding and communication,''
	%\emph{Appl. Comput. Harmon. Anal.}, vol.~14, no.~3, pp.~257-275, May 2003.
	
	
	
	%\bibitem{CanWak:intro}
	%E.~J.~Candes and M.~B.~Wakin,
	%``An introduction to compressive sampling,''
	%\emph{IEEE Sig. Proc. Mag.}, pp.~21-30, Mar. 2008.
	
	
\end{thebibliography}



\end{document}
