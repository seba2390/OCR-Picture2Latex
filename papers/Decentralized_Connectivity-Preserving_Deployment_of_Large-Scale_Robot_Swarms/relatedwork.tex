\section{Related Work}
\label{sec:relatedwork}

Extensive literature exists on methods for connectivity
preservation. Several recent works consist of motion control laws that
include an estimate of the Fielder value. Yang \emph{et
  al.}~\cite{Yang2010} introduced a decentralized algorithm to
estimate the Fiedler value and use it to maintain connectivity while
moving towards a target location. This algorithm was later refined by
Sabattini \emph{et al.}~\cite{Sabattini2011} and Williams \emph{et
  al.}~\cite{Williams2013}. Further extensions include inter-robot
collision avoidance~\cite{RobuffoGiordano2013} and multi-target
exploration~\cite{Nestmeyer2017}. The main advantage of this family of
approaches is that they allow navigation with arbitrary
topologies. However, accurate decentralized computation of the Fiedler
value is not easy in realistic settings in which messages might be
lost due to communication interference~\cite{DiLorenzo2013}. In
addition, computing the Fielder value in a decentralized manner
involves network-wide power iteration methods~\cite{Bertrand2013},
the slow convergence of which makes them suitable only for small teams of
robots~\cite{Sahai2012,Williams2013}. It should also be noted that all
of the above algorithms, with the exception
of~\cite{RobuffoGiordano2013}, have only been demonstrated in
simulated environments.

A second family of methods select a communication sub-graph and aim to
preserve its edges through some form of global consensus. Hsieh
\emph{et al.}~\cite{Hsieh2008} devised a reactive control law based on
radio signal and bandwidth estimation, in which links between robots
can be activated and deactivated as the topology changes over
time. Michael \emph{et al.}~\cite{Michael2009} employed distributed
consensus and auctions algorithms to establish which links to activate
and deactivate over time. Cornejo \emph{et
  al.}~\cite{Cornejo2009,Cornejo2012} proposed a distributed algorithm
for link selection in which the robots undergo a number of motion
rounds, during which the selected links must be preserved. Being based
on achieving global consensus before any topology modification can be
finalized, these algorithms are not scalable and work best when teams
involve a small number of robots.

A third class of connectivity-preserving algorithms assumes that a
certain structure is pre-existing. The dynamic structure is some form
of logical tree, dynamically built and updated over the physical links
of the robot network. Our work falls into this category. Krupke
\emph{et al.}~\cite{Krupke2015} employed a Steiner tree as a
pre-existing structure, and use spring-like virtual forces to balance
connectivity and cohesiveness while reaching distant targets. A number
of works, which constitute our main source of inspiration, utilized
minimum spanning trees as structures to preserve.  Aragues \emph{et
  al.}~\cite{Aragues2014} focused on a distributed coverage strategy
with connectivity constraints, and proposed a method based on
maintaining a network-wide minimum spanning tree. Analogously,
Soleymani et al.~\cite{Soleymani2015} proposed a distributed approach
that constructs and preserves a network-wide minimum spanning tree,
allowing for tree switching. Schuresko \emph{et
  al.}~\cite{Schuresko2012} studied a theoretical approach for
distributed and robust switching between minimum spanning trees.  All
these works were only demonstrated in numerical simulations. The main
advantage of these methods is the ease and speed with which spanning
trees can be built and updated in a distributed manner. However, as
discussed in this paper, spanning trees do not scale well with the
number of robots involved.

% \textit{Competition (to shorten)} A survey of existing methods
% \textit{can be found} in \cite{Zavlanos2011}. A majority of
% contributions can be grouped under spectral theory methods. These rely
% on the computation of eigenvalues of the Laplacian matrix $\Lagr$
% associated to the communication graph of the mobile ad-hoc network. In
% \cite{Yang2010}, the authors propose a decentralized power iteration
% procedure which produces the greatest eigenvalue of $\Lagr$ and its
% associated non-zero eigenvector. Each agent computes its estimate of
% the eigenvector and an averaging operation is carried out to compute
% the global estimate of the time derivative of this eigenvector. This
% averaging operation can be realized by the average consensus estimator
% from \cite{freeman2006}. Several works build on these contributions
% and propose decentralized control strategies relying on the estimation
% of the second smallest eigenvalue of $\Lagr$ which is also called
% Fiedler value (\cite{Sabattini2011} and~\cite{Zelazo2015}). These
% methods are computationally intensive and prone to noise. They
% converge slowly and therefore impose a limit on robot
% dynamics. Another more efficient heuristic method for estimating the
% Fiedler value is presented in~\cite{Bertrand2013}. It, however,
% assumes synchronicity and ideal communication between
% agents.~\cite{Williams2013} also present a more efficient method using
% inverse power iteration. In~\cite{Zareh}, the authors introduce a
% decentralized algorithm for simultaneously estimating all eigenvectors
% and eigenvalues of $\Lagr$ regardless of the multiplicity of the
% eigenvalues. The main idea in spectral theory methods is to estimate a
% global property or maintain a global structure. Because of the
% distributed nature of computations involved, scalability and
% intermittent communication pose a challenge.

% %but then virtual potential works with trees and in line with our work 
% Other methods aim to select a communication sub-graph and maintain its
% edges. In~\cite{Hsieh2008}, the authors propose a reactive control law
% which activates certain graph links and aims to maintain these links
% over time. In~\cite{Cornejo2009} and~\cite{Cornejo2012}, robots
% locally select a set of connectivity-preserving edges to maintain in
% the motion round.~\cite{Krupke2015} uses virtual forces to balance
% goals of connectivity, cohesiveness and adaptability to node failures
% and external forces. The problem with these methods is that they
% either require a global consensus or only ensure local
% connectivity. %\textit{The problem of dynamically growing a tree network of robots has been studied by which considered fixed access points and mobile node acting as communication relays for robots servicing tasks.} %\cite{Nestmeyer2017} Decentralized Simultaneous Multi-target Exploration using a Connected Network of Multiple Robots. \cite{Nouyan2008} Path formation in robot swarm}

% %In \cite{Bezzo2013}, a decentralized technique for estimating changes in the network topology 
% \textit{Similar works} : \cite{Nestmeyer2017} (very similar terminology, "connector", "prime traveler" but also includes motion planning). \cite{Nouyan2008} Path formation in robot swarm
% \cite{Kantaros2016}. Networks : survey paper with taxonomy in sigcomm CCR\cite{Keshav2006}
% % The dynamic growth of the network is achieved through motion planning strategies which consists in solving a constrained optimization problem. \cite{Bezzo2014} have considered the routing optimization of mobile relays to satisfy communication constraints.
% %\cite{Panerati}

% progressive deployment

% nouyan2008 is similar, but it assumes the robots already placed uniformly randomly, chains are probabilistic

%%% Local Variables:
%%% mode: latex
%%% TeX-master: "main"
%%% End:
