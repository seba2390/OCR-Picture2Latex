\section{Problem Statement}
\label{sec:probstat}

\subsection{Robot Dynamics}
\label{sec:robotdynamics}

We consider $N$ robots with linear discrete dynamics
$$
x_i(t+1) = A x_i(t) + B u_i(t)
$$
where $x_i(t) \in \mathbb{R}^{2M}$ is the state of robot $i$ at time
$t$, $u_i(t) \in \mathbb{R}^{2M}$ is the control signal, and
$A,B \in \mathbb{R}^{2M \times 2M}$. The state $x_i(t)$ is defined as
$\left[p_i(t), v_i(t)\right]$, where $p_i(t) \in \mathbb{R}^M$
designates the position of robot $i$ and $v_i(t) \in \mathbb{R}^M$ its
velocity. State and controls are subject to the convex constraints
$$
\forall t \ge 0 \quad x_i(t) \in \mathcal{X}_i \quad u_i(t) \in \mathcal{U}_i.
$$
In this work we focus on 2-dimensional navigation ($M = 2$).

\subsection{Robot Communication}
\label{sec:robotcommunication}

We assume that the robots are capable of \emph{situated
  communication}. This is a communication modality in which robots
broadcast data within a limited range $C$, and upon receiving data, a
robot is able to estimate the relative position of the data sender
with respect to its own local reference frame.

We define the \emph{communication graph}
$\mathcal{G}_C = (\mathcal{V}, \mathcal{E}_C)$, where $\mathcal{V}$ is
the set of robots $\left\{1, \dots, N\right\}$, and
$\mathcal{E}_C \subseteq \mathcal{V} \times \mathcal{V}$ is the set of edges
connecting the robots. An edge $(i,j)$ between two robots exists at
time $t$ if their distance is within their communication range $C$,
i.e.,  $\parallel p_i(t) - p_j(t) \parallel \le C$.

\begin{definition}[Graph connectivity]
  A graph is \emph{connected} is there exists a path between any two nodes.
\end{definition}

Graph connectivity can be verified through well-known concepts in
spectral graph theory. From the definition of the graph adjacency
matrix
$$
A_{ij} =
\begin{cases}
  1 & \text{if }(i,j) \in \mathcal{E}_C \\
  0 & \text{otherwise}
\end{cases}
$$
and of the graph degree matrix
$$
D_{ij} =
\begin{cases}
  \sum_k A_{ik} & \text{if }i = j\\
  0             & \text{otherwise}
\end{cases}
$$
we can derive the Laplacian matrix $L = D - A.$ The graph is connected
if and only if the second smallest eigenvalue of $L$ is greater than
0. For this reason, this eigenvalue is called \emph{algebraic
  connectivity} or \emph{Fiedler value}~\cite{Fiedler1973}. We will
employ algebraic connectivity as a performance measure in the
experiments of \sect{evaluation}.

\subsection{Objectives}
\label{sec:objectives}

The objective of this work can be stated as follows: we aim to create
a progressive deployment strategy that can reach an arbitrary number
of geographically distant tasks while satisfying connectivity
constraints. In particular, the final configuration of the network
topology must allow communication between any two target locations.

%%% Local Variables:
%%% mode: latex
%%% TeX-master: "main"
%%% End:
