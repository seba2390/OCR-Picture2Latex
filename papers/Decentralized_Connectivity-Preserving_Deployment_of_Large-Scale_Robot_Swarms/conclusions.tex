\section{Conclusions}
\label{sec:conclusions}

In this paper, we presented two algorithms to construct a long-range
communication backbone that connects multiple distant target
locations. The algorithms are decentralized and based on the idea of
constructing a logical tree over the set of physical network links.

We performed an extensive large set of experiments, both in simulation
and with real robots, to assess the performance of the algorithms
according to various experimental conditions. Our results show that,
in small-scale scenarios, \emph{outwards} tree growth, corresponding
to spanning tree formation, is a viable approach. However, as the
scale of the environment and the number of robots involved increase, a
more directed, \emph{inwards} growth from target locations towards the
tree root, is a preferable approach.

Our results also show that, as the number of unnecessary robots
increases, the benefit of redundancy is voided by the increased
physical interference in navigation. While a better spare robot
strategy could diminish this phenomenon, our results suggest that a
more progressive approach to deployment might be a better idea.

Nonetheless, the presence of a reasonable number of spare robots
offers the opportunity to tackle the problem of maintaining
\emph{persistent} long-range global connectivity despite individual
limitations in the energy supply of individual robots. We plan to
consider this scenario in future research.

In addition, possible extensions of our work include the presence of
moving targets, rather than static ones, and the presence of obstacles
in the environment.

%%% Local Variables:
%%% mode: latex
%%% TeX-master: "main"
%%% End:
