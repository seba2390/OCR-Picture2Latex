\section{Evaluation}
\label{sec:evaluation}

\subsection{Parameter Setting}
\label{sec:paramopt}

\begin{table*}[t]
  \centering
  \caption{Optimized Design Parameters}
  \label{tab:paramopt}
  \begin{tabular}{|c|c|l|c|c|c|}
    \hline
    Type & Symbol & Meaning & Outwards & Inwards & Unit\\
    \hline
    \hline
    \multirow{5}{*}{Motion}
    & $S$ & Safe range between parent and child & 138.93 & 135.25581 & cm \\
    \cline{2-6}
    & $A$ & Non-parent-child avoidance range & 43.16 & 40.99 & cm\\
    \cline{2-6}
    & $\delta$ & Ideal distance between parent and child & 190 & 154.0841 & cm\\
    \cline{2-6}
    & $\epsilon$ & Factor gain in parent-child interaction & 10 & 10 & \\
    \cline{2-6}
    & $\tau$ & Magnitude of attraction to target & 0.49 & 0.2539 & \\
    \hline
    \multirow{2}{*}{Tree Growth}
    & $R$ & Reconfiguration period & 38.8 & 44.0 & sec\\
    \cline{2-6}
    & $I$ & Information liveness period & 1.2 & 0.5 & sec\\
    \cline{2-6}
    \hline
    \multirow{2}{*}{Uncommitted Management}
    & $E$ & Distance threshold for \emph{spare} recruitment & 132.09 & 132.1353 & cm\\
    \cline{2-6}
    & $J$ & Distance threshold to switch to \emph{connector} & 9.79 & 6.6395 & cm\\
    \cline{2-6}
    \hline
  \end{tabular}
\end{table*}

The dynamics and the performance of our algorithms depends on the
design parameters reported in \tab{paramopt}. To set their value, we
used a genetic algorithm. We ran multiple instances of the
optimization process for both \emph{inwards} and \emph{outwards}, and
\tab{paramopt} reports the best values we found.

Every instance of the optimization was executed for 100
generations. We set this number as a reasonable margin after observing
that, across instances, after about 50 generations the optimization
process would find a plateau beyond which no improvement was found.

Every generation consisted of trials in which 9 Khepera IV
robots\footnote{\url{https://www.k-team.com/mobile-robotics-products/khepera-iv}}
were placed in the arena in a tight cluster. We configured two types
of trials:
\begin{itemize}
\item 2 target locations on a circle with a radius of \unit[2.3]{m} at
  $\unit[180]{^\circ}$ from each other;
\item 3 targets on a circle with a radius of \unit[1.6]{m} at
  $\unit[120]{^\circ}$ from each other.
\end{itemize}

We ran the trials in the ARGoS multi-robot
simulator~\cite{Pinciroli2012}, and maximized a two-step performance
function. The first step (performance 0 to 1) promoted connectivity
maintenance by penalizing the time spent with disconnected robots; the
second step (performance 1 to 2) was activated when no disconnections
occurred, and higher values corresponded to lower times to reach the
targets.

\subsection{Simulated Experiments}
\label{sec:simexperiments}

We tested the performance of the algorithms by varying three
parameters: the target radius, the redundancy factor, and number of
targets. We placed multiple targets on a circle with equal angles
between each other. The \emph{target radius} is the radius of the
circle. We chose radii of 3, 6 and 9 meters corresponding to small,
medium and large scales. The \emph{redundancy factor} is the factor by
which we multiply the minimum required number of robots needed to
reach all the targets given our communication range. We tested the
values of 2, 3 and 4 for this parameter. The \emph{number of targets}
was 2, 3, and 4. The largest configuration we considered involved 94
robots. Each scenario was executed with 50 different random seeds. We
ran all the experiments for both algorithms with and without
activating line-of-sight obstructions in the communication models of
ARGoS, to test the effect of this aspect.

\subsubsection{Simulation Time}
We studied the time performance of both algorithms, and declared an
experiment finished when all workers reach their targets. To compare
results across different scales, we normalized the mission duration by
the maximum allowed time. The maximum allowed time was computed by
considering the time for a robot to reach a target from the center of
the arena; this time was then multiplied by 10. The results are
reported in \fig{sim_time}. For small scales, the \emph{outwards}
algorithm outperforms the \emph{inwards} algorithm. However, as the
scale of the experiment is increased, the directed growth of the
\emph{inwards} algorithm is increasingly advantageous. In addition,
with the \emph{outwards} algorithm, some missions do not reach their
targets in the allotted time limits when higher redundancy factor is
employed. This is due to the increased interference that too many
useless branches create in robot navigation. This effect is not
prominent in the \emph{inwards} algorithm because the robots are added
to the tree only when it is necessary.
\begin{figure*}[t]
    \centering
    \includegraphics[width=0.3\textwidth]{simulation_time_graph_Small_}
    \includegraphics[width=0.3\textwidth]{simulation_time_graph_Medium_}
    \includegraphics[width=0.3\textwidth]{simulation_time_graph_Large_}
    \caption{Assessment of mission completion time in simulation.}
    \label{fig:sim_time}
\end{figure*}

\subsubsection{Disconnected Time}
We studied the ability to maintain connectivity by considering the
following metrics:
\begin{inparaenum}[(i)]
\item The \emph{disconnected time ratio}, defined as the number of
  time steps (over the total experiment time) with at least a broken
  edge in the tree;
\item The \emph{Fiedler value time ratio}, defined as the number of
  time steps (over the total experiment time) with swarm-wide Fiedler
  value lower than $10^{-3}$.
\end{inparaenum}
\begin{figure*}[t]
  \centering
  \includegraphics[width=0.3\textwidth]{disconn_time_graph_Small_}
  \includegraphics[width=0.3\textwidth]{disconn_time_graph_Medium_}
  \includegraphics[width=0.3\textwidth]{disconn_time_graph_Large_}
  \caption{Assessment of connectivity loss.}
  \label{fig:disc_time}
\end{figure*}
The results are reported in \fig{disc_time}. In small-scale scenarios,
in only two experiments out of 50 have positive disconnected time, and
the global communication graph always stays connected. In medium-scale
scenarios, larger numbers of redundant robots cause occasional
line-of-sight obstructions that delay messages exchanges, but
connectivity is generally maintained throughout the duration of the
experiment. In large-scale scenarios, the disruptive effect of a large
number of redundant robots is prominent for both algorithms. With
fewer robots, the inwards algorithm is capable of maintaining global
connectivity in all of the experiments, despite occasional breaking of
tree edges (in less than 5\% of the experiments).

% \subsubsection{Tree Selection Run-time}

% To understand the scalability of each algorithm we analyzed how long it takes build a tree using both algorithms.
% Figure \ref{fig:tree_selection_runtime} shows the line fitting of data points for the inwards algorithm.


% \begin{figure}
%     \centering
%     \includegraphics[height=5cm]{tree_selection_runtime_inwards_obstr_off}
%     \caption{Tree Selection Runtime}
%     \label{fig:tree_selection_runtime}
% \end{figure}

% TREE CONSTRUCTION ONLY ? 

% - graph showing runtime for each state with different number of robots 

% - graph showing nb steps for tree selection, count and centroid (run in parallel so show that total nb steps $<$ sum for each)

% FULL SIMULATION WITH DYNAMICS

% - graph with total time to reach targets (for different configs)

% - graph with tree metrics (Fiedler value, centrality? other ?) and progress to target (for 1 simulation ? how to agregate results? )

% - number of working robots over time ? 

% - maybe :
% show worst case (line and targets on one/both ends?) and best case ("disk" and distributed targets on circle?)? 

% - study precision and convergence of centroid?

\subsection{Real-Robot Validation}
To validate the simulated results simulations, we tested our
algorithms with 9 Khepera IV robots. A Vicon motion capture system was
used to track the position and orientation of the robots throughout the
duration of the experiments, and to simulate situated communication.
We employed 2 experimental scenarios:
\begin{inparaenum}[(i)]
\item 2 targets on a circle with a radius of 2.3 meters at
  approximately 180 degrees from each other;
\item 3 targets on a circle with a radius of 1.6 meters at
  approximately 120 degrees from each other.
\end{inparaenum}
We rescaled the distance-related parameters in \tab{paramopt} to fit
the arena and accommodate for the small number of robots involved.  We
repeated these experiments 15 times for setup (i) and 10 times for
setup (ii) with robots starting from the same positions and
orientations, to allow for better comparison. We also performed the
same experiments in simulation, with the same initial positions.

\fig{realexperiment} shows that real-robot and simulated experiments
follow analogous trends. In particular, we verified that for
small-scale experiments with low redundancy factor (in these
experiments it was set 1) the \emph{outwards} algorithm has better
performance than the \emph{inwards} algorithm.
\begin{figure}
    \centering
    \includegraphics[width=.3\textwidth]{realrobots_simulationtime}
    \includegraphics[width=.3\textwidth]{realrobots_disconnectedtime}
    \caption{Results of real-robot evaluation.}
    \label{fig:realexperiment}
\end{figure}

% % \begin{table}[]
% %    \centering
% %    \begin{tabular}{|c|c|c|c|c|}
% %        \hline
% %        \multicolumn{2}{|c|}{Experiment type} & \multirow{2}{*}{Scale} & \multirow{2}{*}{Data Type} & \multirow{2}{*}{p-value} \\
% %         \cline{1-2}
% %        Algorithm & Obstruction &&&\\
% %         \hline
% %        \multirow{18}{*}{Spanning tree}  & \multirow{9}{*}{On}  & \multirow{3}{*}{Small}& Disconnected time (tree)  & 0.00147 \\
% %        \cline{4-5}
% %        &  & & Disconnected time (graph)  & 0.31731 \\
% %        \cline{4-5}
% %        &  & & Simulation time  & 1.8e-75 \\
% %        \cline{3-5}
% %        &  & \multirow{3}{*}{Medium} & Disconnected time (tree) & 3e-23 \\
% %        \cline{4-5}
% %        &  & & Disconnected time (graph)   & 5.9e-05 \\
% %        \cline{4-5}
% %        &  & & Simulation time  & 1e-75 \\
% %        \cline{3-5}
% %        &  & \multirow{3}{*}{Large} & Disconnected time (tree) &  5.8e-50 \\
% %        \cline{4-5}
% %        &  & & Disconnected time (graph)   & 5.2e-14 \\
% %        \cline{4-5}
% %        &  & &Simulation time  & 2e-77 \\
% %        %  \hline
% %        \cline{2-5}

% %       %  \multirow{9}{*}{Spanning tree}
% %        & \multirow{9}{*}{Off}  & \multirow{3}{*}{Small} & Disconnected time (tree) &  0.06789 \\
% %        \cline{4-5}
% %        &  & &Disconnected time (graph)  & 0.31731 \\
% %        \cline{4-5}
% %        &  & &Simulation time  & 7e-27 \\
% %        \cline{3-5}
% %        &  & \multirow{3}{*}{Medium} & Disconnected time (tree)  & 2e-10 \\
% %        \cline{4-5}
% %        &  & & Disconnected time (graph)  & 0.00222 \\
% %        \cline{4-5}
% %        &  & & Simulation time  & 6e-27 \\
% %        \cline{3-5}
% %        &  & \multirow{3}{*}{Large} & Disconnected time (tree)  &  1e-18 \\
% %        \cline{4-5}
% %        &  & &Disconnected time (graph)  & 1.6e-8 \\
% %        \cline{4-5}
% %        &  & &Simulation time  & 2e-26 \\
% %        \hline

% %        \multirow{18}{*}{Sparse tree}  & \multirow{9}{*}{On}  & \multirow{3}{*}{Small} & Disconnected time (tree) &  2.5-06 \\
% %        \cline{4-5}
% %        &  & &Disconnected time (graph)  & 0.0 \\
% %        \cline{4-5}
% %        &  & &Simulation time  & 9.6e-70 \\
% %        \cline{3-5}
% %        &  & \multirow{3}{*}{Medium} & Disconnected time (tree)  & 1e-24 \\
% %        \cline{4-5}
% %        &  & &Disconnected time (graph)  & 0.17971 \\
% %        \cline{4-5}
% %        &  & &Simulation time  & 5.7e-67 \\
% %        \cline{3-5}
% %        &  & \multirow{3}{*}{Large} & Disconnected time (tree)  &  3.8e-49 \\
% %        \cline{4-5}
% %        &  & &Disconnected time (graph)  & 0.00098 \\
% %        \cline{4-5}
% %        &  & &Simulation time  & 1e-63 \\
% %        \cline{2-5}
% %       %  \hline

% %       %  \multirow{9}{*}{Sparse tree}
% %        & \multirow{9}{*}{Off}  & \multirow{3}{*}{Small} & Disconnected time (tree) &  7.7e-08 \\
% %        \cline{4-5}
% %        &  & &Disconnected time (graph)  & 0.17971 \\
% %        \cline{4-5}
% %        &  & &Simulation time  & 1.8e-75 \\
% %        \cline{3-5}
% %        &  & \multirow{3}{*}{Medium} & Disconnected time (tree)  & 2.6e-18 \\
% %        \cline{4-5}
% %        &  & &Disconnected time (graph)  & 0.00768 \\
% %        \cline{4-5}
% %        &  & &Simulation time  & 3.8e-75 \\
% %        \cline{3-5}
% %        &  & \multirow{3}{*}{Large} & Disconnected time (tree)  &  6.8e-45 \\
% %        \cline{4-5}
% %        &  & &Disconnected time (graph)  & 5.9e-05 \\
% %        \cline{4-5}
% %        &  & &Simulation time  & 9e-73 \\
% %        \hline
% %    \end{tabular}
% %    \caption{Wilcoxon Test}
% %    \label{tab:Wilcoxon Test }
% % \end{table}

% \subsection{Real-Robot Validation}
% \label{sec:realrobotvalid}



%%% Local Variables:
%%% mode: latex
%%% TeX-master: "main"
%%% End:
