
Among several definitions of integrability, we adopt the {\em Liouville-Arnold integrability}~\cite{Arnold-text,Perelomov-text,Arutyunov-text2019}.
Let $N$ be the number of particles and suppose that they move in one-dimensional space.
The dimension of the phase space is $2N$.
We here introduce involution: $N$ smooth functions of a Hamiltonian system denoted by $\{F_i\}_{i=1}^N$ are {\em in involution} if they satisfy the following condition:
\begin{align}
\{F_i, F_j\} = 0 \qquad i,j = 1, \ldots, N,
\end{align}
where $\{\cdot\}$ is the Poisson bracket:
\begin{align}
 \{A, B\} \defeq \sum_{i=1}^N
                 \left[\frac{\partial A}{\partial p_i}\frac{\partial B}{\partial q_i}
                 - \frac{\partial A}{\partial q_i}\frac{\partial B}{\partial p_i}\right].
\end{align}
Let us consider the $N$ smooth functions to be conserved quantities.
Obviously, the Hamiltonian can be one of the functions.
Then, the integrability is defined in the following way: {\em if a Hamiltonian system has $N$ conserved quantities in involution that are independent at every point of the phase space, the system is integrable}.
Here, the independence of the functions means that $dF_i$ are linearly independent, \ie the rank of the Jacobian of $F_i$ is $N$.

Our analysis is based on the property of the integrable system that the trajectory is diffeomorphic to the $N$-dimensional torus if the motion of the particles is bounded or periodic.
Let $\{\theta_i\}_{i=1}^{N}$ be angular coordinates of each dimension of the torus.
The motion of the integrable system is represented by
\begin{align}
   \frac{d\theta_i}{dt} & = \omega_i(F) \qquad i=1,\ldots,N,
\end{align}
where $\{\omega_i(F)\}_{i=1}^{N}$ are functions of $F=\{F_i\}_{i=1}^{N}$.
It is known that a set of variables $\{I_i\}_{i=1}^{N}$ can be constructed from $\{F_i\}_{i=1}^{N}$ such that $\{I_i,\theta_i\}_{i=1}^N$ are canonical coordinates~\cite{Arnold-text}: $I_i$ and $\theta_i$ are called the action and angle variables, respectively.
The equations of motion are given by
\begin{align}
 \begin{split}
   \frac{dI_i}{dt} & = 0, \\
   \frac{d\theta_i}{dt} & = \frac{\partial K(I)}{\partial I_i} = \omega_i(I), \qquad i=1,\ldots,N,
 \end{split} \label{eq:eom_of_action_angle_variabels}
\end{align}
where $K(I)$ is the Hamiltonian in latent space, which depends only on the action variables.
See Supplemental Material for some examples of the action-angle variables~\cite{our_supplemental_material}.

Equation~(\ref{eq:eom_of_action_angle_variabels}) implies that the system is integrable if the corresponding action-angle variables exist~\cite{Arnold-text}.
Is there any reasonable condition on the Hamiltonian? We seek systems that can be represented by a natural Hamiltonian~\cite{Cariglia-2014} in real space, which consists of the kinetic term and a potential function, that is,
\begin{align}
  H(p,q) = \sum_{i=1}^{N}\frac{p_i^2}{2} + V(q),\label{eq:natural_hamiltonian}
\end{align}
where $p=\{p_i\}_{i=1}^{N}$ are momenta, $q=\{q_i\}_{i=1}^{N}$ are positions or displacements, and $V(q)$ is a potential function.
An integrable system described by $K(I)$ is {\em physically reasonable} if the Hamiltonian can be represented by the natural form~(\ref{eq:natural_hamiltonian}) in real space.
In general, however, it is non-trivial to find an appropriate canonical transformation from a given $K(I)$ to a natural $H(p,q)$.
Thus, we propose a neural network approach to construction of the canonical transformation and the natural Hamiltonian simultaneously.
