
Integrable systems have provided deep understandings of physical phenomena and profound insight into the connection between physics and mathematics.
For example, the Toda lattice revealed the solution of the solitary wave called the soliton~\cite{Toda1967,Toda1967-2}, and Calogero-Moser systems---a group of integrable systems constructed by an ansatz for the Lax pair---unveiled the relation between integrability and motions in the Lie group~\cite{Calogero1975,Calogero1976,Moser1975}.
It is essential to find or build a new integrable model to advance the understanding of many-body systems further.
However, it is quite challenging as well: the integrability of correlated systems is highly non-trivial.
So far, the path to finding an integrable system is limited to the ingenious construction of constant motions (conserved quantities) and few ansatzes for the Lax pair.
Thus, it is desirable to establish a systematic way of finding integrable models.

In the meantime, machine learning techniques have been successfully applied to many physical systems~\cite{Carleo-etal2019}.
In particular, the neural network has proven powerful for learning complex transformations~\cite{Noe-etal2019,Shuo-Hui-etal2020,Bondesan-Lamacraft2019} and potential functions~\cite{Greydanus-Dzamba-Yosinki2019,Suwa2019}.
As a relevant application of unsupervised learning, the underlying system Hamiltonian can be rebuilt from the trajectory of particles~\cite{Greydanus-Dzamba-Yosinki2019}.
One of the most promising applications of the neural network is the canonical transformation of classical mechanical systems.
The complex distribution of correlated particles can be transformed into the Gaussian distribution of independent harmonic oscillators, enabling efficient sampling of equilibrium states~\cite{Noe-etal2019,Shuo-Hui-etal2020}.
In the classical integrable system, an extensive number of conserved quantities, namely action variables, exist.
The positions and momenta of the particles in real space are canonically transformed into the action-angle variables $(I,\theta)$ in latent space; accordingly, the Hamiltonian $H(p,q)$ becomes $K(I)$ in latent space through the transformation.
The latent-space Hamiltonian depends only on $I$, and thus the action variables are conserved obviously.
The neural network can reproduce the canonical transformation of some known integrable models~\cite{Bondesan-Lamacraft2019}.
Nevertheless, the application of machine learning to the canonical transformation has been limited to known integrable systems.

In this Letter, we propose a systematic way of constructing integrable systems without any prior knowledge about the canonical transformation or any ansatz for the Lax pair.
Training data in our approach are samples of the action-angle variables.
The action variables are sampled from the Boltzmann distribution, and the angle variables are sampled from the uniform distribution.
We assume the functional form of $K(I)$ or that samples of action-angle variables can be generated in some way.
Under this condition, we train neural networks in unsupervised learning such that the networks learn the canonical transformation and the Hamiltonian $H(p,q)$ simultaneously.
Our neural network seeks the natural Hamiltonian corresponding to the assumed $K(I)$, which consists of the kinetic term and the potential term.
While the potential function is represented using the residual neural network, the canonical transformation is implemented composing the RealNVP neural network, the symplectic linear transformation, and the discrete Hartley transformation.
We also use the symplectic integrator with the adjoint method for time evolution in the learning.
We here demonstrate successful learning for the Toda lattice.
The trained neural networks reproduce the true potential function and the exact canonical transformation with high accuracy.
The present approach enables us to find new integrable systems.
