% !TEX root = ../top.tex
% !TEX spellcheck = en-US

\begin{figure}[t]
\centering
\includegraphics[width=0.49\linewidth]{./fig/cube_problem/cube_problem.pdf}
\includegraphics[width=0.49\linewidth]{./fig/cube_problem/cube_problem2.pdf}
\begin{tabular}{C{0.45\linewidth}C{0.45\linewidth}}
{\scriptsize (a) Sensitivity to different keypoints} & {\scriptsize (b) Sensitivity to target positions}
\end{tabular}
\vspace{-5mm}
\caption{\small {\bf Problem with minimizing the 2D reprojection error.} {\bf (a)} The red lines denote the 2D reprojection errors for points ${\bf p}_1$ and ${\bf p}_2$. Because one is closer to the camera than the other, these 2D errors are of about the same magnitude even though the corresponding 3D errors, shown in blue, are very different. {\bf(b)} For the same object at different locations, the same 2D error can generate different 3D errors. This makes pose accuracy dependent on the relative position of the target to the camera. 
% The problem depicted by (a) has little impact as long as the size of the target is small with respect to its distance to the camera but worsens when the object comes closer. The exact opposite occurs in the case depicted by (b). \YH{these statements may be not very accurate, so I removed them to avoid potential criticizes.}
% \WJ{Lower figure shows $\mathbf{p}_1$ twice. This figure consumes a relatively large amount of space for the content, you could probably show (a) and (b) side by side if space is needed.}\YH{Fixed}
}
\label{fig:cube_problem}
\end{figure}