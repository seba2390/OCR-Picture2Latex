 % !TEX root = ../top.tex
% !TEX spellcheck = en-US

\begin{figure}[t]
\centering
\includegraphics[height=2.2cm]{fig/clearspace/agena2.png}
\includegraphics[height=2.2cm]{fig/clearspace/agena3.png}
% \includegraphics[height=2.2cm]{fig/clearspace/chaser.png}
\includegraphics[height=2.2cm]{fig/clearspace/capture.png}
\begin{small}
\begin{tabular}{C{0.3\linewidth}C{0.2\linewidth}C{0.4\linewidth}}
(a)&(b)&(c)
% (c)&(d)
\end{tabular}
\end{small}
\vspace{-6mm}
\caption{\small {\bf Docking and space cleaning.} {\bf (a, b)} Two different views of the Agena target vehicle during the first space docking. The appearance of Agena is strongly affected by the large scale and viewpoint changes, suggesting that different image features should be used for 6D pose estimation. In 1966, this docking procedure was controlled manually. {\bf (c)} In 2025, the ClearSpace One chaser satellite will be launched to retrieve and de-orbit a non-operational satellite, so as to showcase the feasibility of removing space debris. In this case, the capture will be fully automated. The synthetic image shown here highlights the challenges the algorithm will have to handle, such as reflections, over-exposure of some parts of the images, and lack of details in others. 
% \MS{I would tend to use the current image (d) in (c), and another image better showing these challenges, and possibly at a different scale, here.} \PF{Aren't images (a) and (b) challenging enough? You are looking at the same target in both.}
% \YH{Fixed, the main meanings are still the same but save more space.}
}
\label{fig:docking}
\end{figure}