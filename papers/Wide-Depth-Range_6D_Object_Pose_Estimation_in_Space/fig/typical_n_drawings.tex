 
 
% !TEX root = ../top.tex
% !TEX spellcheck = en-US

\begin{figure}[t]
    \begin{center}
    \includegraphics[width=0.49\linewidth]{./fig/typical_n_drawings/typical_n_drawings_1.pdf}
    \includegraphics[width=0.49\linewidth]{./fig/typical_n_drawings/typical_n_drawings_10.pdf}
    % \fbox{\rule{0pt}{2in} \rule{0.25\linewidth}{0pt}}
    \end{center}
    \vspace{-7mm}
    \caption{{\bf Sample count ${\cal N}_k$  at each pyramid level as a function of the object size ${\cal S}$.}
    Typically, when $\lambda > 20$, ${\cal N}_k$ degenerates to the simple ``hard'' assignment strategy that FPN adopts. Note that for a given object size, multiple ${\cal N}_k$s are non-zero, which translates to soft assignments to the different pyramid levels.
    % of ${\cal N}_k$ becomes ``soft'' across multiple pyramid levels in the left figure.
    }
    \label{fig:typical_k_drawings}
\end{figure}
