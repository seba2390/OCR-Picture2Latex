 
 
% !TEX root = ../top.tex
% !TEX spellcheck = en-US

\begin{figure}[t]
\centering
\includegraphics[width=0.42\linewidth]{./fig/sampling_demo/sampling_demo.pdf}
\hspace{2em}
\includegraphics[width=0.42\linewidth]{./fig/sampling_demo/sampling_demo_2.pdf}
\begin{small}
\begin{tabular}{C{0.45\linewidth}C{0.45\linewidth}}
(a) Standard strategy~\cite{Lin17e} & (b) Proposed strategy \\
\end{tabular}
\end{small}
\vspace{-6mm}
\caption{{\bf Sampling strategies during training.} 
    Let the circles denote all the training instances sorted in increasing order of depth from left to right. {\bf (a)} The traditional sampling strategy assigns each instance to a single pyramid level according to its size during training. For example, the red instance is fed only to pyramid level ${\cal F}_2$, thus encouraging only this level to yield a reasonable prediction for this sample. {\bf (b)} We propose to assign each instance to multiple pyramid levels, encouraging every pyramid level to produce a reasonable pose estimate for every instance.
    % {\bf (a)} The traditional sampling strategy assigns them to fixed pyramid levels based on thresholds related to the object size~\cite{Lin17e}. {\bf (b)} We propose to make different levels interactivable\WJ{(word does not exist)} during training and make the multi-scale architecture ensemble-ready for robust 6D pose estimation.\YH{Fixed}
    }
\label{fig:sampling_demo}
\end{figure}
