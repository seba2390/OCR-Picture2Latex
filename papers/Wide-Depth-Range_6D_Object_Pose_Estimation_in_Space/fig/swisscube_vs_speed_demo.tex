 % !TEX root = ../top.tex
% !TEX spellcheck = en-US


\begin{figure}[t]
\begin{minipage}{\linewidth}
\centering
\includegraphics[height=2.2cm,trim=450 350 380 50,clip]{fig/swisscube_vs_speed_demo/008982.jpg}
\hspace{1em}
\includegraphics[height=2.2cm,trim=300 450 350 400,clip]{fig/swisscube_vs_speed_demo/scene_000003_0050.jpg}
\begin{scriptsize}
\begin{tabular}{C{0.35\linewidth}C{0.6\linewidth}}
(a) The SPEED dataset & (b) The proposed SwissCube dataset
% (c)&(d)
\end{tabular}
\end{scriptsize}

\vspace{-6mm}
\caption{\small {\bf Comparison of datasets.} {\bf (a)} The SPEED dataset~\cite{Kisantal20} was generated with a non-physics-based renderer and only poorly reflects the complexity of illumination in space. {\bf (b)} We introduce a SwissCube dataset that was created via physics-based rendering. 
%The dataset name is temporarily anonymized as CubeSat for not revealing authorship.
}
\label{fig:swisscube_vs_speed_demo}
\end{minipage}
\end{figure}