% !TEX root = ../top.tex
% !TEX spellcheck = en-US

\section{Conclusion}
\label{sec:conclusion}

We have proposed to use a single hierarchical network to estimate the 6D pose of an object subject to large scale variations, as would be the case in a space scenario. Our experiments have evidenced that training the different level of the resulting pyramid for different object scales and fusing their predictions during inference improves accuracy and robustness. We have also introduced the SwissCube dataset, the first satellite dataset with an accurate 3D model, physically-based rendering, and physical simulations of the Sun, the Earth, and the stars. Our approach outperforms the state of the art in both the wide-depth-range scenario and the more classical Occluded-LINEMOD dataset. In the future, we will concentrate on other important aspects of 6D object pose estimation in space, such as removing jitter by 6D pose tracking, and training a usable model with fully-unsupervised real data.

\vspace{0.2em}
{\noindent \bf Acknowledgments.}
This work was supported by the Swiss Innovation Agency (Innosuisse). We would like to thank the EPFL Space Center (eSpace) for the data support.