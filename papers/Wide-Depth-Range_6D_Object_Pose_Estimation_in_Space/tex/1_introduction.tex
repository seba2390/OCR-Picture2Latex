% !TEX root = ../top.tex
% !TEX spellcheck = en-US

\section{Introduction}
\label{sec:introduction}

Reliable 6D pose estimation is key to automating many spatial maneuvers, such
as docking or capturing inert objects as shown in Fig.~\ref{fig:docking}. An
important consequence of such maneuvers is that they dramatically change the
scale and aspect of the observed target. Although 6D pose estimation is an
active area of research in computer vision and robotics, this important aspect
has not received significant attention thus far---for example, most benchmark
datasets~\cite{Hinterstoisser12b,Krull15,Xiang18b,Hodan18} feature objects whose depth varies within a limited range.
The lack of atmospheric scattering enabling observation from great
distances also leads to other challenges: harsh contrast, under- and
over-exposed areas, and significant specular reflections from reflective
materials used in space engineering (aluminium and carbon fiber panels, etc.).
% \WJ{(Many edits here, and in the abstract -- remove this comment, if they are okay.)}
% \YH{Seems great}

 % !TEX root = ../top.tex
% !TEX spellcheck = en-US

\begin{figure}[t]
\centering
\includegraphics[height=2.2cm]{fig/clearspace/agena2.png}
\includegraphics[height=2.2cm]{fig/clearspace/agena3.png}
% \includegraphics[height=2.2cm]{fig/clearspace/chaser.png}
\includegraphics[height=2.2cm]{fig/clearspace/capture.png}
\begin{small}
\begin{tabular}{C{0.3\linewidth}C{0.2\linewidth}C{0.4\linewidth}}
(a)&(b)&(c)
% (c)&(d)
\end{tabular}
\end{small}
\vspace{-6mm}
\caption{\small {\bf Docking and space cleaning.} {\bf (a, b)} Two different views of the Agena target vehicle during the first space docking. The appearance of Agena is strongly affected by the large scale and viewpoint changes, suggesting that different image features should be used for 6D pose estimation. In 1966, this docking procedure was controlled manually. {\bf (c)} In 2025, the ClearSpace One chaser satellite will be launched to retrieve and de-orbit a non-operational satellite, so as to showcase the feasibility of removing space debris. In this case, the capture will be fully automated. The synthetic image shown here highlights the challenges the algorithm will have to handle, such as reflections, over-exposure of some parts of the images, and lack of details in others. 
% \MS{I would tend to use the current image (d) in (c), and another image better showing these challenges, and possibly at a different scale, here.} \PF{Aren't images (a) and (b) challenging enough? You are looking at the same target in both.}
% \YH{Fixed, the main meanings are still the same but save more space.}
}
\label{fig:docking}
\end{figure}
% \section{Our Approach}
We formulate the problem as an anisotropic diffusion process and the diffusion tensor is learned through a deep CNN directly from the given image, which guides the refinement of the output.

\begin{figure}[t]
\includegraphics[width=1.0\textwidth]{fig/CSPN_SPN2.pdf}
\caption{Comparison between the propagation process in SPN~\cite{liu2017learning} and CPSN in this work.}
\label{fig:compare}
\end{figure}

\subsection{Convolutional Spatial Propagation Network}
% demonstrate the thereom is hold when turns to be convolution.
Given a depth map $D_o \in \spa{R}^{m\times n}$ that is output from a network, and image $\ve{X} \in \spa{R}^{m\times n}$, our task is to update the depth map to a new depth map $D_n$ within $N$ iteration steps, which first reveals more details of the image, and second improves the per-pixel depth estimation results. 

\figref{fig:compare}(b) illustrates our updating operation. Formally, without loss of generality, we can embed the $D_o$ to some hidden space $\ve{H} \in \spa{R}^{m \times n \times c}$. The convolutional transformation functional with a kernel size of $k$ for each time step $t$ could be written as,
\begin{align}
    \ve{H}_{i, j, t + 1} &= \sum\nolimits_{a,b = -(k-1)/2}^{(k-1)/2} \kappa_{i,j}(a, b) \odot \ve{H}_{i-a, j-b, t} \nonumber \\
\mbox{where,~~~~}
    \kappa_{i,j}(a, b) &= \frac{\hat{\kappa}_{i,j}(a, b)}{\sum_{a,b, a, b \neq 0} |\hat{\kappa}_{i,j}(a, b)|}, \nonumber\\
    \kappa_{i,j}(0, 0) &= 1 - \sum\nolimits_{a,b, a, b \neq 0}\kappa_{i,j}(a, b)
\label{eqn:cspn}
\end{align}
where the transformation kernel $\hat{\kappa}_{i,j} \in \spa{R}^{k\times k \times c}$ is the output from an affinity network, which is spatially dependent on the input image. The kernel size $k$ is usually set as an odd number so that the computational context surrounding pixel $(i, j)$ is symmetric.
$\odot$ is element-wise product. Following~\cite{liu2017learning}, we normalize kernel weights between range of $(-1, 1)$ so that the model can be stabilized and converged by satisfying the condition $\sum_{a,b, a,b \neq 0} |\kappa_{i,j}(a, b)| \leq 1$. Finally, we perform this iteration $N$ steps to reach a stationary distribution.

% theorem, it follows diffusion with PDE 
%\addlinespace
\noindent\textbf{Correspondence to diffusion process with a partial differential equation (PDE).} \\
Similar with~\cite{liu2017learning}, here we show that our CSPN holds all the desired properties of SPN.
Formally, we can rewrite the propagation in \equref{eqn:cspn} as a process of diffusion evolution by first doing column-first vectorization of feature map $\ve{H}$ to $\ve{H}_v \in \spa{R}^{\by{mn}{c}}$.
\begin{align}
     \ve{H}_v^{t+1} = 
     \begin{bmatrix}
    1-\lambda_{0, 0}  & \kappa_{0,0}(1,0) & \cdots & 0 \\
    \kappa_{1,0}(-1,0)   & 1-\lambda_{1, 0} & \cdots & 0 \\
    \vdots & \vdots & \ddots & \vdots \\
    \vdots & \cdots & \cdots & 1-\lambda_{m,n} \\
\end{bmatrix} = \ve{G}\ve{H}_v^{t}
\label{eqn:vector}
\end{align}
where $\lambda_{i, j} = \sum_{a,b}\kappa_{i,j}(a,b)$ and $\ve{G}$ is a $\by{mn}{mn}$ transformation matrix. The diffusion process expressed with a partial differential equation (PDE) is derived as follows, 
\begin{align}
     \ve{H}_v^{t+1} &= \ve{G}\ve{H}_v^{t} = (\ve{I} - \ve{D} + \ve{A})\ve{H}_v^{t} \nonumber\\
     \ve{H}_v^{t+1} - \ve{H}_v^{t} &= - (\ve{D} - \ve{A}) \ve{H}_v^{t} \nonumber\\
     \partial_t \ve{H}_v^{t+1} &= -\ve{L}\ve{H}_v^{t}
\label{eqn:proof}
\end{align}
where $\ve{L}$ is the Laplacian matrix, $\ve{D}$ is the diagonal matrix containing all the $\lambda_{i, j}$, and $\ve{A}$ is the affinity matrix which is the off diagonal part of $\ve{G}$.

In our formulation, different from~\cite{liu2017learning} which scans the whole image in four directions~(\figref{fig:compare}(a)) sequentially, CSPN propagates a local area towards all directions at each step~(\figref{fig:compare}(b)) simultaneously, \ie with~\by{k}{k} local context, while larger context is observed when recurrent processing is performed, and the context acquiring rate is in an order of $O(kN)$.

In practical, we choose to use convolutional operation due to that it can be efficiently implemented through image vectorization, yielding real-time performance in depth refinement tasks.

Principally, CSPN could also be derived from loopy belief propagation with sum-product algorithm~\cite{kschischang2001factor}. However, since our approach adopts linear propagation, which is efficient while just a special case of pairwise potential with L2 reconstruction loss in graphical models. Therefore, to make it more accurate, we call our strategy convolutional spatial propagation in the field of diffusion process.

\begin{figure}[t]
\centering
\includegraphics[width=0.9\textwidth]{fig/hist.pdf}
\caption {(a) Histogram of RMSE with depth maps from~\cite{Ma2017SparseToDense} at given sparse depth points.  (b) Comparison of gradient error between depth maps with sparse depth replacement (blue bars) and with ours CSPN (green bars), where ours is much smaller. Check~\figref{fig:gradient} for an example. Vertical axis shows the count of pixels.}
\label{fig:hist}
\end{figure}

\subsection{Spatial Propagation with Sparse Depth Samples}
In this application, we have an additional sparse depth map $D_s$ (\figref{fig:gradient}(b)) to help estimate a depth depth map from a RGB image. Specifically, a sparse set of pixels are set with real depth values from some depth sensors, which can be used to guide our propagation process. 

Similarly, we also embed the sparse depth map $D_s = \{d_{i,j}^s\}$ to a hidden representation $\ve{H}^s$,  and we can write the updating equation of $\ve{H}$ by simply adding a replacement step after performing \equref{eqn:cspn}, 
\begin{align}
    \ve{H}_{i, j, t+1} = (1 - m_{i, j}) \ve{H}_{i, j, t+1}  +  m_{i, j} \ve{H}_{i, j}^s 
\label{eqn:cspn_sp}
\end{align}
where $m_{i, j} = \spa{I}(d_{i, j}^s > 0)$ is an indicator for the availability of sparse depth at $(i, j)$. 

In this way, we guarantee that our refined depths have the exact same value at those valid pixels in sparse depth map. Additionally, we propagate the information from those sparse depth to its surrounding pixels such that the smoothness between the sparse depths and their neighbors are maintained. 
Thirdly, thanks to the diffusion process, the final depth map is well aligned with image structures. 
This fully satisfies the desired three properties for this task which is discussed in our introduction (\ref{sec:intro}). 

% it performs a non-symmetric propagation where the information can only be diffused from the given sparse depth to others, while not the other way around.

% still follows PDE
In addition, this process is still following the diffusion process with PDE, where the transformation matrix can be built by simply replacing the rows satisfying $m_{i, j} = 1$ in $\ve{G}$ (\equref{eqn:vector}), which are corresponding to sparse depth samples, by $\ve{e}_{i + j*m}^T$. Here $\ve{e}_{i + j*m}$ is an unit vector with the value at $i + j*m$ as 1.
Therefore, the summation of each row is still $1$, and obviously the stabilization still holds in this case.

\begin{figure}[t]
\centering
\includegraphics[width=0.95\textwidth]{fig/fig2.pdf}
\caption{Comparison of depth map~\cite{Ma2017SparseToDense} with sparse depth replacement and with our CSPN \wrt smoothness of depth gradient at sparse depth points. (a) Input image. (b) Sparse depth points. (c) Depth map with sparse depth replacement. (d) Depth map with our CSPN with sparse depth points. We highlight the differences in the red box.}
\label{fig:gradient}
\end{figure}

Our strategy has several advantages over the previous state-of-the-art sparse-to-dense methods~\cite{Ma2017SparseToDense,LiaoHWKYL16}.
In \figref{fig:hist}(a), we plot a histogram of depth displacement from ground truth at given sparse depth pixels from the output of Ma \etal~\cite{Ma2017SparseToDense}. It shows the accuracy of sparse depth points cannot preserved, and some pixels could have very large displacement (0.2m), indicating that directly training a CNN for depth prediction does not preserve the value of real sparse depths provided. To acquire such property, 
one may simply replace the depths from the outputs with provided sparse depths at those pixels, however, it yields non-smooth depth gradient \wrt surrounding pixels. 
In~\figref{fig:gradient}(c), we plot such an example, at right of the figure, we compute Sobel gradient~\cite{sobel2014history} of the depth map along x direction, where we can clearly see that the gradients surrounding pixels with replaced depth values are non-smooth.
We statistically verify this in \figref{fig:hist}(b) using 500 sparse samples, the blue bars are the histogram of gradient error  at sparse pixels by comparing the gradient of the depth map with sparse depth replacement and of ground truth depth map. We can see the difference is significant, 2/3 of the sparse pixels has large gradient error.
Our method, on the other hand, as shown with the green bars in \figref{fig:hist}(b), the average gradient error is much smaller, and most pixels have zero error. In\figref{fig:gradient}(d), we show the depth gradients surrounding sparse pixels are smooth and close to ground truth, demonstrating the effectiveness of our propagation scheme. 

% Finally, in our experiments~\ref{sec:exp}, we validate the number of iterations $N$ and kernel size $k$ used for doing the CSPN.


\subsection{Complexity Analysis}
\label{subsec:time}

As formulated in~\equref{eqn:cspn}, our CSPN takes the operation of convolution, where the complexity of using CUDA with GPU for one step CSPN is $O(\log_2(k^2))$, where $k$ is the kernel size. This is because CUDA uses parallel sum reduction, which has logarithmic complexity. In addition,  convolution operation can be performed parallel for all pixels and channels, which has a constant complexity of $O(1)$. Therefore, performing $N$-step propagation, the overall complexity for CSPN is $O(\log_2(k^2)N)$, which is irrelevant to image size $(m, n)$.

SPN~\cite{liu2017learning} adopts scanning row/column-wise propagation in four directions. Using $k$-way connection and running in parallel, the complexity for one step is $O(\log_2(k))$. The propagation needs to scan full image from one side to another, thus the complexity for SPN is $O(\log_2(k)(m + n))$. Though this is already more efficient than the densely connected CRF proposed by~\cite{philipp2012dense}, whose implementation complexity with permutohedral lattice is $O(mnN)$, ours $O(\log_2(k^2)N)$ is more efficient since the number of iterations $N$ is always much smaller than the size of image $m, n$. We show in our experiments (\secref{sec:exp}), with $k=3$ and $N=12$, CSPN already outperforms SPN with a large margin (relative $30\%$), demonstrating both efficiency and effectiveness of the proposed approach.


\subsection{End-to-End Architecture}
\label{subsec:unet}
\begin{figure}[t]
\centering
\includegraphics[width=0.95\textwidth,height=0.45\textwidth]{fig/framework2.pdf}
\caption{Architecture of our networks with mirror connections for  depth estimation via transformation kernel prediction with CSPN (best view in color). Sparse depth is an optional input, which can be embedded into the CSPN to guide the depth refinement.}
\label{fig:arch}
\end{figure}

We now explain our end-to-end network architecture to predict both the transformation kernel and the depth value, which are the inputs to CSPN for depth refinement.
 As shown in \figref{fig:arch}, our network has some similarity with that from Ma \etal~\cite{Ma2017SparseToDense}, with the final CSPN layer that outputs a dense depth map.  
 
For predicting the transformation kernel $\kappa$ in \equref{eqn:cspn}, 
rather than building a new deep network for learning affinity same as Liu \etal~\cite{liu2017learning}, we branch an additional output from the given network, which shares the same feature extractor with the depth network. This helps us to save memory and time cost for joint learning of both depth estimation and transformation kernels prediction. 

Learning of affinity is dependent on fine grained spatial details of the input image. However, spatial information is weaken or lost with the down sampling operation during the forward process of the ResNet in~\cite{laina2016deeper}. Thus, we add mirror connections similar with the U-shape network~\cite{ronneberger2015u} by directed concatenating the feature from encoder to up-projection layers as illustrated by ``UpProj$\_$Cat'' layer in~\figref{fig:arch}. Notice that it is important to carefully select the end-point of mirror connections. Through experimenting three possible positions to append the connection, \ie after \textit{conv}, after \textit{bn} and after \textit{relu} as shown by the ``UpProj'' layer in~\figref{fig:arch} , we found the last position provides the best results by validating with the NYU v2 dataset (\secref{subsec:ablation}). 
In doing so, we found not only the depth output from the network is better recovered, and the results after the CSPN is additionally refined, which we will show the experiment section~(\secref{sec:exp}).
Finally we adopt the same training loss as~\cite{Ma2017SparseToDense}, yielding an end-to-end learning system.



To address such challenges, the European Space Agency (ESA) and Stanford
University recently organized a satellite pose estimation challenge based on
the \emph{Spacecraft Pose Estimation Dataset} (SPEED)~\cite{Kisantal20}. The best-performing
methods in this competition use a two-step approach to handle large depth
variation: a detector finds an axis-aligned box bounding the target, which is
resampled to a uniform size and finally processed by a 6D pose estimator.

This approach is suboptimal in several ways. First, detection and pose
estimation are treated as separate processes, which precludes joint training.
Second, it provides supervisory signals only to the final layer of the
encoder-decoder architecture being used instead of to all levels of the
decoding pyramid, which would increase robustness. Third, many similar feature
extraction computations are performed by both processes, which results in an
unnecessary duplication of effort. Finally, these methods rely on the dominant approach
to deep learning based 6D object pose estimation~\cite{Rad17,Hu19a,Chen19DLR} consisting of
training a network to minimize the 2D reprojection error of predefined 3D
keypoints, which cannot cope with large depth range variations: As shown
in Fig.~\ref{fig:cube_problem}, reprojection error is strongly affected by the
distance of individual keypoints to the camera, and not explicitly taking this
into account degrades performance. 
% \WJ{Use of ``keypoint'' in this paragraph seems unnecessarily technical, why not ``position''? Also applies to Fig 2 caption.}\YH{It may be confused with other points on the object surface. We only use the 8 corners of the 3D object bounding box of the object. So we say they are ``key'' points. And most literature uses this terminology.}

To address these shortcomings, we introduce a single hierarchical end-to-end trainable network depicted by Fig.~\ref{fig:arch} that yields robust and scale-insensitive 6D poses. 
To use information across scales, it progressively downscales the learned features,  derives 3D-to-2D correspondences for each level of the resulting pyramid, and finally uses a RANSAC-based PnP strategy to infer a single reliable pose from these sets of correspondences. This is a departure from most networks that estimate pose only from the final layer. 
To address the issue in Fig.~\ref{fig:cube_problem}, we minimize a training loss based on 3D positions instead of 2D projections, making the method invariant to the target distance.
We use a Feature Pyramid Network (FPN)~\cite{Lin17e} as our backbone but, unlike in most approaches relying on such networks, we assign each training instance to multiple pyramid levels to promote the joint use of multi-scale information. 

 %
In short, our contribution is a new 6D pose estimation architecture that reliably handles large scale changes under challenging conditions. We will show that it outperforms all state-of-the-art methods on the established SPEED dataset while also being much faster. Furthermore, we introduce a larger-scale satellite pose estimation dataset featuring more realistic and more complex images than SPEED, and we show that our method delivers the same benefits in this more challenging scenario. Finally, we demonstrate that our method outperforms the state of the art even on images with smaller depth variations, such as those of the challenging Occluded LINEMOD dataset. Our code and new dataset will be publicly released.

% !TEX root = ../top.tex
% !TEX spellcheck = en-US

\begin{figure}[t]
\centering
\includegraphics[width=0.49\linewidth]{./fig/cube_problem/cube_problem.pdf}
\includegraphics[width=0.49\linewidth]{./fig/cube_problem/cube_problem2.pdf}
\begin{tabular}{C{0.45\linewidth}C{0.45\linewidth}}
{\scriptsize (a) Sensitivity to different keypoints} & {\scriptsize (b) Sensitivity to target positions}
\end{tabular}
\vspace{-5mm}
\caption{\small {\bf Problem with minimizing the 2D reprojection error.} {\bf (a)} The red lines denote the 2D reprojection errors for points ${\bf p}_1$ and ${\bf p}_2$. Because one is closer to the camera than the other, these 2D errors are of about the same magnitude even though the corresponding 3D errors, shown in blue, are very different. {\bf(b)} For the same object at different locations, the same 2D error can generate different 3D errors. This makes pose accuracy dependent on the relative position of the target to the camera. 
% The problem depicted by (a) has little impact as long as the size of the target is small with respect to its distance to the camera but worsens when the object comes closer. The exact opposite occurs in the case depicted by (b). \YH{these statements may be not very accurate, so I removed them to avoid potential criticizes.}
% \WJ{Lower figure shows $\mathbf{p}_1$ twice. This figure consumes a relatively large amount of space for the content, you could probably show (a) and (b) side by side if space is needed.}\YH{Fixed}
}
\label{fig:cube_problem}
\end{figure}
% !TEX root = ../top.tex
% !TEX spellcheck = en-US

\begin{figure}[t]
\centering
\includegraphics[width=0.99\linewidth]{./fig/arch/arch.pdf}
\vspace{-3mm}
    \caption{{\bf Our single-stage approach.} We use an encoder-decoder architecture to progressively downsample the image and then to re-expand it. At each level of the decoder, we establish 3D-to-2D correspondences. Finally, we use a RANSAC-based PnP strategy~\cite{Lepetit09} 
    % \WJ{citation?} \YH{Fixed} 
     to infer a single reliable pose from these sets of correspondences. 
    % \WJ{Impossible to read text in printout, please try to have text in the figure with a similar font size as the surrounding article.}\YH{Fixed}
    }
\label{fig:arch}
\end{figure}


%\PF{Based on our zoom meeting, I would recommend showing 
%\begin{itemize}
% \item quantitative results on SPEED. 
% \item qualitative results on VESPA to match Fig. 1. 
% \item qualitative results on real images of the space cube with a discussion about doing better in future work using DA. 
%\end{itemize}
%}
%
%\YH{Thank you. I will do it asap.}

%-------
% OLD
%-------

%Estimating 6D object pose has became an essential component of many real-world vision applications, including robotics,  machine perception, and augmented reality etc~\cite{xx}. On the other hand, estimating 6D pose of space objects, such as satellites and orbit debris, is drawing significant attention as the increasing congestion in Earth orbits~\cite{xx}.

%In a typical wide-depth-range scenario, estimating 6D object pose will be much more challenge due to the scaling problems as the estimation of farther samples will be dominated by closer ones~\cite{xx}. The most straightforward way to handle this probelm is to introduce a object detection network as a preprocessing component, and scale all the detected bounding boxes to the same size before feeding them into another pose regression network. However, this type of strategy equipping with two separated networks is heavy and suboptimal in practice~\cite{xx} (see Fig.~\ref{fig:pyramid_vs_twonetworks}).
