%%%%%%%%%%%%%%%%%%%%%%%%%%%%%%%%%%%%%%%%%%%%%%%%%%%%%%%%%%%%%%%%%%%%%%%%%%%%%%%%
%2345678901234567890123456789012345678901234567890123456789012345678901234567890
%        1         2         3         4         5         6         7         8

\documentclass[letterpaper, 10 pt, conference]{ieeeconf}  % Comment this line out if you need a4paper

%\documentclass[a4paper, 10pt, conference]{ieeeconf}      % Use this line for a4 paper

\IEEEoverridecommandlockouts                              % This command is only needed if 
                                                          % you want to use the \thanks command

\overrideIEEEmargins                                      % Needed to meet printer requirements.

%In case you encounter the following error:
%Error 1010 The PDF file may be corrupt (unable to open PDF file) OR
%Error 1000 An error occurred while parsing a contents stream. Unable to analyze the PDF file.
%This is a known problem with pdfLaTeX conversion filter. The file cannot be opened with acrobat reader
%Please use one of the alternatives below to circumvent this error by uncommenting one or the other
%\pdfobjcompresslevel=0
%\pdfminorversion=4

% See the \addtolength command later in the file to balance the column lengths
% on the last page of the document

% The following packages can be found on http:\\www.ctan.org
%\usepackage{graphics} % for pdf, bitmapped graphics files
%\usepackage{epsfig} % for postscript graphics files
%\usepackage{mathptmx} % assumes new font selection scheme installed
%\usepackage{times} % assumes new font selection scheme installed
%\usepackage{amsmath} % assumes amsmath package installed
%\usepackage{amssymb}  % assumes amsmath package installed

% \usepackage{cite}
\usepackage{amsmath,amssymb,amsfonts}
\usepackage{algorithmic}
\usepackage{graphicx}
\usepackage{braket}
\usepackage{textcomp}
\usepackage{xcolor}
\usepackage{subcaption}
\usepackage{float}
\usepackage{multicol}
\usepackage{caption}
% \setlength{\belowcaptionskip}{-10pt}
% \captionsetup{belowskip=0pt}
% \captionsetup[figure]{skip=0pt}
\usepackage{todonotes}
\usepackage{url}
\usepackage{graphicx}
\usepackage{xspace}
\usepackage{tikz,pgfplots}

\usepgfplotslibrary{statistics}


\newcommand{\system}{\textsc{Groot}\xspace}


\title{\LARGE \bf
Improving Grasp Stability with Rotation Measurement \\from Tactile Sensing
}


% \author{Albert Author$^{1}$ and Bernard D. Researcher$^{2}$% <-this % stops a space
% \thanks{*This work was not supported by any organization}% <-this % stops a space
% \thanks{$^{1}$Albert Author is with Faculty of Electrical Engineering, Mathematics and Computer Science,
%         University of Twente, 7500 AE Enschede, The Netherlands
%         {\tt\small albert.author@papercept.net}}%
% \thanks{$^{2}$Bernard D. Researcheris with the Department of Electrical Engineering, Wright State University,
%         Dayton, OH 45435, USA
%         {\tt\small b.d.researcher@ieee.org}}%
% }


\author{Raj Kolamuri$^{*,1}$, Zilin Si$^{*,2}$, Yufan Zhang$^{1}$, Arpit Agarwal$^{2}$ and Wenzhen Yuan$^{2}$% <-this % stops a space
\thanks{$^{*}$Authors with equal contribution.}%
\thanks{$^{1}$Raj Kolamuri and Yufan Zhang are with the Department of Mechanical Engineering, Carnegie Mellon University, 5000 Forbes Ave, Pittsburgh, PA 15213, USA
        {\tt\small \{rkolamur, yufanzha\}@andrew.cmu.edu}}%
\thanks{$^{2}$Zilin Si, Arpit Agarwal and Wenzhen Yuan are with the Robotics Institute, Carnegie Mellon University, 5000 Forbes Ave, Pittsburgh, PA 15213, USA
        {\tt\small \{zsi, arpita1, wenzheny\}@andrew.cmu.edu}}%
}

\begin{document}



\maketitle
\thispagestyle{empty}
\pagestyle{empty}


%%%%%%%%%%%%%%%%%%%%%%%%%%%%%%%%%%%%%%%%%%%%%%%%%%%%%%%%%%%%%%%%%%%%%%%%%%%%%%%%
\begin{abstract}
% We present a torsion detection algorithm and build a close-loop grasping framework based on tactile sensor, where we detect the failure in the process of grasping with large torsion and use it to guide the re-grasp. 
% A common case of grasping failure is when the object is grasped at the wrong location, causing rotational displacement between the gripper and object. We develop a model-based algorithm to detect the onset of rotation and the degree of rotation from the GelSight sensing, and further use it to guide the regrasp location adjustment until the stable grasp is reached. 
% We conduct offline experiments to test our torsion detection algorithm's accuracy with ground truth and online experiments to proof the efficacy of our system. 

% \arpit{Is this a large number for regrasp tasks? If yes, then the sentence could go like "We validate our grasp-regrasp system on over a 100 grasping attempts.}

Rotational displacement about the grasping point is a common grasp failure when an object is grasped at a location away from its center of gravity. Tactile sensors with soft surfaces, such as GelSight sensors, can detect the rotation patterns on the contacting surfaces when the object rotates. In this work, we propose a model-based algorithm that detects those rotational patterns and measures rotational displacement using the GelSight sensor. We also integrate the rotation detection feedback into a closed-loop regrasping framework, which detects the rotational failure of grasp in an early stage and drives the robot to a stable grasp pose. We validate our proposed rotation detection algorithm and grasp-regrasp system on self-collected dataset and online experiments to show how our approach accurately detects the rotation and increases grasp stability. 
% We perform experiments to show how the proposed algorithm can improve grasp stability by incorporating  

% From more than one hundred times grasping, we showed our system can efficiently finish the grasp-regrasp task.

\end{abstract}


%%%%%%%%%%%%%%%%%%%%%%%%%%%%%%%%%%%%%%%%%%%%%%%%%%%%%%%%%%%%%%%%%%%%%%%%%%%%%%%%
\section{INTRODUCTION}
\section{Introduction}  \label{sec:introduction}

\newcommand\inexpIntro[3]{#1?(#2,#3).}
\newcommand\rinexpIntro[3]{*#1?(#2,#3).}
\newcommand\outexpIntro[3]{#1!(#2,#3).}
\newcommand\outatomIntro[3]{#1!(#2,#3)}

We propose a fully automated method for proving termination of \(\pi\)-calculus processes.
Although there have been a lot of studies on termination analysis for the \(\pi\)-calculus
and related calculi~\cite{Deng06IC,Demangeon07,SangiorgiTermination,KobayashiHybrid,Yoshida04IC,DBLP:journals/jlp/DemangeonHS10,Venet98SAS}, most of them have been rather theoretical,
and there have been surprisingly little efforts in developing  fully automated termination
verification methods and tools based on them. To our knowledge,
Kobayashi's \typical{}~\cite{TyPiCal,KobayashiHybrid} is the only exception that
can prove termination of \(\pi\)-calculus processes (extended with natural numbers)
fully automatically, but its termination analysis is quite limited (see Section~\ref{sec:relatedwork}).

Our method is based on a reduction to termination analysis for sequential programs:
we translate a \(\pi\)-calculus process \(P\) to a sequential program \(S_P\), so that
if \(S_P\) is terminating, so is \(P\). The reduction allows us to use
powerful, mature methods and tools
for termination analysis of sequential programs~\cite{heizmann2016ultimate,freqterm,DBLP:conf/lics/PodelskiR04,Kuwahara2014Termination,DBLP:journals/cacm/CookPR11}.

The idea of the translation is to convert a chain of communications on replicated input
channels to a chain of recursive function calls of the target sequential program.
Let us consider the following Fibonacci process:
\begin{align*}
    & \rinexpIntro{\fib}{n}{r}
        \ifexp{n<2}{ \soutatom{r}{1} \\ &\quad}
                   { \nuexp{s_1} \nuexp{s_2} (\outatomIntro{\fib}{n-1}{s_1} \PAR \outatomIntro{\fib}{n-2}{s_2} \PAR \sinexp{s_1}{x}\sinexp{s_2}{y}\soutatom{r}{x+y}) \\}
    & \PAR \outatomIntro{\fib}{m}{r}
\end{align*}
Here, the process
$\rinexpIntro{\fib}{n}{r} \ldots$ is a function server that computes the \(n\)-th Fibonacci number
in parallel and returns the result to \(r\),
and $\outatom{\fib}{m}{r}$ sends a request for computing the \(m\)-th Fibonacci number;
those who are not familiar with the syntax of the \(\pi\)-calculus may wish to consult
Section~\ref{sec:targetlanguage} first.
To prove that the process above is terminating for any integer \(m\),
it suffices to show that there is no infinite chain of communications on $\fib$:
\[
    \fib(m,r) \to \fib(m_1,r_1) \to \fib(m_2,r_2) \to \cdots.
\]
We convert the process above to the following program:\footnote{The actual translation
  given later is a little more complex.}
\begin{verbatim}
 let rec fib(n) = if n<2 then () else (fib(n-1) [] fib(n-2)) in
 fib(m)
\end{verbatim}
Here, \texttt{[]} represents the non-deterministic choice.
Note that, although the calculation of Fibonacci numbers is not preserved,
for each chain of communications on \texttt{fib}, there is a corresponding
sequence of recursive calls:
\[
\mathtt{fib}(m) \to \mathtt{fib}(m_1) \to \mathtt{fib}(m_2) \to \cdots.
\]
Thus, the termination of the sequential program above implies the termination of
the original process.
As shown in the example above, (i) each communication on a replicated input channel
is converted to a function call, (ii) each communication on a non-replicated input
channel is just removed (or, in the actual translation, replaced by a call of
a trivial function defined by \(f(\seq{x})=(\,)\)), and (iii) parallel composition
is replaced by a non-deterministic choice.
We formalize the translation outlined above and prove its correctness.

The basic translation sketched above sometimes loses too much information.
For example, consider the following process:
\begin{align*}
    & \rinexpIntro{\pre}{n}{r} \soutatom{r}{n-1} \\
    & \PAR \rinexpIntro{f}{n}{r} \ifexp{n<0}{ \soutatom{r}{1} }
                                       { \nuexp{s} (\outatomIntro{\pre}{n}{s} \PAR \sinexp{s}{x}\outatomIntro{f}{x}{r}) } \\
    & \PAR \outatomIntro{f}{m}{r}
\end{align*}
The translation sketched above would yield:
\begin{verbatim}
  let pred(n) = n-1 in
  let rec f(n) = if n<0 then () else (pred(n) [] f(*)) in
  f(m)
\end{verbatim}
Here, \texttt{*} represents a non-deterministic integer: since we have removed
the input $\sinatom{s}{x}$, we do not have information about the value of \( x \).
As a result, the sequential program above is non-terminating, although the original
process is terminating.
To remedy this problem, we also refine the basic translation above by using a refinement
type system for the \(\pi\)-calculus. Using the refinement type system,
we can infer that the value of \(x\) in the original process is less than \(n\),
so that we can refine the definition of \texttt{f} to:
\begin{verbatim}
 let rec f(n) = ... else (pred(n) [] let x=* in assume(x<n);f(x))
\end{verbatim}
The target program is now terminating, from which
we can deduce that the original process is also terminating.
We have implemented an automated tool based on the refined translation above.

The contributions of this paper are summarized as follows.
\begin{itemize}
\item The formalization of the basic translation from the \(\pi\)-calculus
  (extended with integers) to sequential programs, and a proof of its correctness.
\item The formalization of a refined translation based on a refinement type system.
\item An implementation of the refined translation, including automated refinement type
  inference based on CHC solving, and experiments to evaluate the effectiveness of
  our method.
\end{itemize}

The rest of this paper is structured as follows.
Section~\ref{sec:targetlanguage} introduces the source and target languages
of our translation.
Section~\ref{sec:approach} 
formalizes the basic translation, and proves its correctness.
Section~\ref{sec:refinement} refines the basic translation by using a refinement type system.
Section~\ref{sec:implementation} reports an implementation and experiments.
Section~\ref{sec:relatedwork} discusses related work,
and Section~\ref{sec:conclusion} concludes the paper.


\section{RELATED WORK}
\subsection{Multitask Learning}

MTL has been succesfully used in different domains, including CV \cite{UberNet,MaskRCNN}. Some challenges appear when applying it \cite{Caruana}: \textit{learning speed} differences between tasks and  deciding \textit{what to share} according to the \textit{relatedness} between tasks in the multitask architecture \cite{Stitch, AdaptiveFeatureSharing}.

\subsection{Semantic Segmentation}

Semantic segmentation aims at partitioning parts of images belonging to the same semantic class, typically via pixel-wise classification. Fully convolutional networks (FCN) \cite{FCN} have improved both accuracy and speed for dense prediction problems by using only convolutional layers. Upsampling layers allow a segmentation output size equal to the input and skip connections add finer details. Other approaches add post-processing steps \cite{DeeplabCRF}, learnable \textit{deconvolution} layers \cite{ Deconv} or global context \cite{ParseNet}.

\subsection{Object Detection}

Object detection aims at finding in an image all instances of objects and classifying them in a number of classes. Faster R-CNN \cite{FasterRCNN} was the first to give close to real-time performance. YOLO \cite{YOLO} avoids the generation of region proposals for increased speed. SSD \cite{SSD} avoids fully-connected layers for speed and takes features at different levels for improved accuracy. 

%\cite{SpeedAccuracy} reviews the speed vs. accuracy trade-off for different object detectors.

\section{METHOD}
The proposed segmentation-by-detection framework, as depicted in Figure \ref{fig:framework}, consists of a detection module and a segmentation module.
In detection stage, 2D slices (layered box) from the input volume are fed to the RPN. Based on the region proposals obtained from RPN, an attention model (block in orange) is formed. The input volume as well as the attention model are further processed in segmentation stage to get the refined anatomical segmentation. 
\vspace{1em} 

\begin{figure}[t]
\centering
\includegraphics[width=0.95\linewidth]{fig/framework.pdf}
\caption{Schematic representation of the segmentation-by-detection framework. The left part is the detection module while the segmentation module is followed on the right. The blue block denotes the input volume which is 3D ultrasound scan of femoral head. The output segmentation is in red.}
\label{fig:framework}
\end{figure}
% dana could you improve the figure. we can try to think together of better ways 

\noindent\textbf{Detection Module:} 
% dana : here you have to make the clarification that you have ground truth on the boxes (in implementation part)
The detection module follows an RPN architecture, a fully convolutional network which takes image slice as input and outputs object region candidates. 
We use the VGG-16 model as the backbone \cite{simonyan2014very} to learn convolutional features and an $3 \times 3$ spatial window to generate region proposals. At each sliding-window location, 9 anchors are predicted associated with different scales and aspect ratios. The last layer consists of a box-regression (reg) layer and a box-classification (cls) layer in parallel. The reg layer outputs 4 regression offsets, $ t = (t_x,t_y,t_w,t_h)$, denoting a scale-invariant translation as well as log-space height and width shift, where $x,y,w$ and $h$ specify two coordinates of the box center, width and height. The cls layer outputs two scores by softmax, related to probabilities of object and background for each proposal. We assign a positive label (of being object) to candidate which has an Intersection-over-Union (IoU) ratio higher than 0.7 with ground truth box. Note that an image slice may contain multiple object regions or none. 

The loss function of RPN follows the multi-task loss \cite{ren2015faster} which is defined as $L = L_{reg} + L_{cls}$. The regression loss, $L_{reg} = -\log p_{obj}$ is log loss and the classification loss,
\begin{equation} \label{eq:loss}
L_{cls} = \sum_{i \in \{x,y,w,h\}} smooth_{L_1} (t_i - t_i^*)
\end{equation}
is smooth $L_1$ loss where $t_i^*$ denotes the ground truth box for the target object. 
\vspace{1em}

\noindent\textbf{Segmentation Module:}
3D U-Net \cite{cciccek20163d} is utilized in the segmentation module as its outstanding performance in medical image segmentation. The u-shaped architecture consists of two paths: a contracting path, where each layer contains two $3\times3\times3$ convolutions followed by a rectified linear unit (ReLU) and then a max pooling, provides high resolution features. While, the symmetric expanding path for semantically richer features replaces max pooling with a upconvolution $2\times2\times2$ with stride of 2 in each dimension, and then two $3\times3\times3$ convolutions each followed by a ReLU. Skip connections between layers of equal resolution in the contracting path and the expanding path enables context information as well as precise localization.

Different from 3D U-Net, to incorporate the attention model detected by the RPN, our architecture takes as input both the volumetric image data and the candidate RoIs proposed by the RPN, concatenated as 3D volume. 
% dana not sure what you like to say below
% densely annotated
The attention model makes the network to focus on the potential RoIs and can reduce the interference of the surrounding noise.
The anatomical segmentation is then generated from a $1\times1\times1$ convolution which reduces the number of feature maps to the number of labels.  The energy function is computed by a pixel-wise softmax combined with the cross entropy loss.
% dana equation ??

\subsection{System and implementation Details}
The segmentation-by-detection approach adopts a cascade structure with two stages: detection and segmentation. The two networks are trained separately in an end-to-end manner. All the new layers are randomly initialized from zero-mean Gaussian distribution with standard deviations 0.01. Biases are initialized to 0. We use Caffe \cite{jia2014caffe} for the implementation and an NVIDIA Titan X GPU for training.

In the detection stage, we initialize the VGG-16 model by the pre-trained model for ImageNet classification \cite{russakovsky2015imagenet} and further fine-tune the model for our detection task. The input fed to the network are image slices with a fixed size of $184\times96$ and the corresponding ground truth boxes are generated from the annotation in the format of tight bounding boxes surrounding the segmentation contour (as illustrated in Figure \ref{fig:hip} (b), the boundary of white area). To optimize the energy function, stochastic gradient descent (SGD) is used. The global learning rate is set to 0.001, while a momentum of 0.9 and a weight decay of 0.0005 are used. The batch size is set to 256 and each mini-batch only contains the positive anchors for training. The region proposals are obtained from the reg path for each image slice. The attention model is then formed by concatenating all the detected regions, as binary masks, into a volume.

In the segmentation stage, we use the Adam optimizer \cite{kingma2014adam} to learn the network parameters. A global learning rate is set to 0.001 while the two momentum coefficients are set to 0.9 and 0.999 respectively. A batch size of 1 is used due to the memory constraints of the GPU. The network takes the volume data as well as the attention model as input. We train the network for a maximum of 30K iterations and reserve the learned weights with the best performance from every 1K iterations. 
\vspace{1em}

\noindent\textbf{Inference:}
At test time, the 2D slices from an input volume are first fed to the detection module. The attention model is obtained based on the output. Then the volume data as well as the attention model are fed to the segmentation module to get the pixel-wise prediction.




\section{EXPERIMENTS}

\section{Experiments}\label{sec:experiments}
We validate our approach using multiple datasets containing real-life data from the fields of criminal risk assessment, credit, lending, and college admissions. In each of the datasets we select a binary feature and treat it as the protected attribute (e.g., race or gender), which is the feature we require our trained classifier to behave fairly upon. Our proposed method performs well on all of these datasets, succeeding in removing unfairness almost entirely, at a very modest price in terms of accuracy.


\begin{table*}[h]
\centering
\resizebox{\textwidth}{!}{
\def\arraystretch{1.2}

\begin{tabular}{c c c | c | c | c || c | c | c || c | c | c |}

\cline{4-12}
&&&
\multicolumn{9}{ c| }{\textbf{COMPAS Dataset}}
\\ \cline{4-12}
&&&
\multicolumn{3}{ c|| }{\textbf{FPR Considerations}}&
\multicolumn{3}{ c|| }{\textbf{FNR Considerations}}&
\multicolumn{3}{ c| }{\textbf{Both Considerations}}
\\ \cline{4-12}
&&&
 $\mathbf{Acc.}$ &  $\mathbf{D_{FPR}}$ &  $\mathbf{D_{FNR}}$ &  $\mathbf{Acc.}$ &  $\mathbf{D_{FPR}}$ &  $\mathbf{D_{FNR}}$ &  $\mathbf{Acc.}$ &  $\mathbf{D_{FPR}}$ &  $\mathbf{D_{FNR}}$
\\  \cline{4-12}
\vspace*{-0.5ex}
\\ \cline{1-2} \cline{4-12}
\multicolumn{1}{ |c  }{} &
\multicolumn{1}{ c|  }{  \textbf{Our Method (AVD Penalizers)}}  &&
$\mathbf{0.660}$    &  $\mathbf{0.01}$  &  $0.04$ &
$\mathbf{0.653}$    &  $0.02$   &  $\mathbf{0.04}$ &
$\mathbf{0.654}$    &  $\mathbf{0.02}$  &  $\mathbf{0.04}$
\\ \cline{1-2} \cline{4-12}
\multicolumn{1}{ |c  }{} &
\multicolumn{1}{ c|  }{  \textbf{Our Method (SD Penalizers)}}  &&
$\mathbf{0.664}$    &  $\mathbf{0.02}$  &  $0.09$ &
$\mathbf{0.661}$    &  $0.05$   &  $\mathbf{0.03}$ &
$\mathbf{0.661}$    &  $\mathbf{0.02}$  &  $\mathbf{0.03}$
\\ \cline{1-2} \cline{4-12}
\multicolumn{1}{ |c  }{} &
\multicolumn{1}{ c|  }{  Zafar et al.~(\citeyear{disparatemistreatment})}  &&
$0.660$    &   $0.06$    &   $0.14$  &
$0.662$    &   $0.03$    &   $0.10$  &
$0.661$    &   $0.03$    &   $0.11$
\\ \cline{1-2} \cline{4-12}
\multicolumn{1}{ |c  }{} &
\multicolumn{1}{ c|  }{  Zafar et al. Baseline~(\citeyear{disparatemistreatment})}  &&
$0.643$    &   $0.03$    &   $0.11$  &
$0.660$    &   $0.00$    &   $0.07$  &
$0.660$    &   $0.01$    &   $0.09$
\\ \cline{1-2} \cline{4-12}
\multicolumn{1}{ |c  }{} &
\multicolumn{1}{ c|  }{  Hardt et al.~(\citeyear{hardt})}  &&
$0.659$    &  $0.02$    &   $0.08$  &
$0.653$    &  $0.06$   &    $0.01$  &
$0.645$    &  $0.01$   &    $0.01$
\\ \cline{1-2} \cline{4-12}
\multicolumn{1}{ |c  }{} &
\multicolumn{1}{ c|  }{  \textbf{Vanilla Regularized Logistic Regression}}  &&
$\mathbf{0.672}$    &   $\mathbf{0.20}$    &   $\mathbf{0.30}$  &
$\mathbf{0.672}$    &   $\mathbf{0.20}$    &   $\mathbf{0.30}$  &
$\mathbf{0.672}$    &   $\mathbf{0.20}$    &   $\mathbf{0.30}$
\\ \cline{1-2} \cline{4-12}
\end{tabular}
}
\vspace{3mm}
\caption{Performance comparison on the COMPAS dataset. For the approaches in bold -- Accuracy, FPR difference and FNR difference are evaluated on the test set, averaging over five runs and using a 70-30 training/test split. The performance of the remaining three approaches is stated as reported in Zafar et al.~(\citeyear{disparatemistreatment}).} \label{table:comparison_results}
\end{table*}



\begin{figure*}[b]
  \includegraphics[scale=0.6]{compas0-400.png}
  \caption{COMPAS Dataset. Accuracy, FPR difference ($\mathbf{D_{FPR}}$), and FNR difference ($\mathbf{D_{FNR}}$) (all evaluated on the test set) of the learned classifier, as a function of the weight $c=c_1 = c_2 \geq 0$ placed on the fairness penalizer terms. On the left we use the Absolute Value Difference (AVD) penalizer, and the Squared Difference (SD) penalizer on the right, both as presented in Section~\ref{regularization}. ``Relaxed FPR/FNR Diff.'' plots the value of the relevant penalization term.} %In this particular run, parameters chosen for the absolute value relaxation were: $c=80, q_c=60$, and for the squared relaxation: $c=220, q_c=30$.}
  \label{fig:compas}
\end{figure*}


\subsection{Implementation}
\textbf{Our method} 
%We instantiate our method in the following way: Given dataset $Q$, we split it randomly into a training set $S$ (which we will use for learning) and a test set $T$ (which we will only use for reporting performance). 
For the purpose of comparison with  Zafar et al.~(\citeyear{disparatemistreatment}) and Hardt et al.~\cite{hardt} on the COMPAS data, we use a parameter $c$ to induce three possible combinations of weights on the FPR and FNR penalization terms: $c = c_1$ and $c_2 = 0$; $c_1 = 0$ and $c = c_2$; and $c = c_1 = c_2$. For the other three datasets, we consider only $c = c_1 = c_2$.\footnote{The reason for varying the values of $c$ in the training phase is since we shifted to a proxy problem, in which we rely on the distance from the decision boundary rather the actual classifications. 
%Our hope is that there is no need for a worst-case cross validation between all of the combinations of $c_1, c_2, c_3$, and that the training scheme we propose is sufficient. 
It is possible, of course, that even better results are attainable using our scheme with other combinations of $c_1, c_2$, and $q$.} To explore the accuracy/fairness trade-off curve for the relaxed optimization problem~(\ref{eq:2}), we train for different values of $c$, starting at $c=0$ (which is just standard logistic regression), and growing gradually.



Given a dataset $Q$ and fixing a $d_1, d_2 \in \{0, 1\}$ of interest, we use the following training scheme:
\begin{enumerate}
\item Split $Q$ at random into training set $S$ and test set $T$.
\item For each $c$, perform cross-validation on $S$ to select the corresponding best value $q_c$ for the regularization parameter.
\item For each $(c,q_c)$, let $\theta_c = \argmin\limits_{\theta} \text{Proxy}(\theta;S,c,c,q_c)$.
\item Select $\theta^* \in \argmin\limits_{\theta_c} \text{Objective}(\theta_c;S,d_1,d_2)$.
\item Evaluate performance using $\theta^*$ on test set $T$.
\end{enumerate}
We report the average of five such runs, each with a fresh training-test split.




%We instantiate our method by solving the relaxed optimization problem~(\ref{eq:2}), in place of the original, non-convex problem~(\ref{eq:1}).  
%We test our approach with three different combinations of weights on the penalization terms:
%\katrina{What are the $d$, and how are they related to the $c$s?}
%\begin{enumerate}
%\item FPR considerations only: $d_1 = 1, d_2 = 0$.
%\item FNR considerations only: $d_1 = 0, d_2 = 1$.
%\item Both FPR, FNR considerations, assigned similar significance: $d_1 = 1, d_2 = 1$.
%\end{enumerate}
%One could, of course, pick any other combination of the FPR and FNR penalty weights.

%\katrina{I don't understand how the below is distinct from the list above}
%Learning is done by training the parameters of a logistic regressor to solve~\ref{eq:2}, while picking the value of $c_1, %c_2$ as the following:
%\begin{enumerate}
%\item FPR considerations only: $c_1 = c \geq 0$, $c_2 = 0$.
%\item FNR considerations only: $c_1 = 0$, $c_2 = c \geq 0$.
%\item Both FPR, FNR considerations, assigned similar significance: $c_1 = c_2 = c \geq 0$
%\end{enumerate}



% We then cross-validate to pick the best $c_3$ (the weight on the standard $\ell_2$-regularization term) given $c$.\footnote{The reason for varying the values of $c$ in the training phase is since we shifted to a proxy problem, in which we rely on the distance from the decision boundary rather the actual classifications. 
%Our hope is that there is no need for a worst-case cross validation between all of the combinations of $c_1, c_2, c_3$, and that the training scheme we propose is sufficient. 
%It is possible, of course, that even better results are attainable using our scheme with other combinations of $c_1, c_2, c_3$.} For each such combination, we report results as the averages of multiple \katrina{how many?} different runs, each time splitting data randomly into training and test sets.
%\yahav{We need to shorten this description.}

We solve the relaxed convex optimization problem using the CVXPY solver. Due to stability issues with large training sets, we use a train/test split of 30-70 on the larger datasets, rather than 70-30 as on the COMPAS dataset\footnote{The code implementing our method can be found at https://github.com/jjgold012/lab-project-fairness}.

%
%
%We then report the results (as evaluated on the test set) attained by a regressor $\theta \in \mathbb{R}^d$ that minimizes (on the training set $S$) a weighted combination of the $0$-$1$ loss and the differences in FPR and FNR across populations:
%\begin{equation*}
%\begin{aligned}
%&\underset{\theta}{\text{argmin}}
%& & L_{S}^{0\text{-}1}(\theta) \\
%&&& + d_1|FPR_{A=0}(\theta;S)-FPR_{A=1}(\theta;S)| \\
%&&& + d_2|FNR_{A=0}(\theta;S)-FNR_{A=1}(\theta;S)|
%\end{aligned}
%\end{equation*}
%
%\katrina{What is $d_1$ vs. $c_1$ etc.?}



%For classification, we decided use a standard cut-off threshold of $c=0.5$. There are of course, further possible interactions between the FPR, FNR considerations, and picking a certain cut-off level. These are not straightforward, since  these interactions are data-specific. 



%allows for flexibility in picking the values of $c_1, c_2$, which reflect the significance we wish to place on the objectives of achieving accuracy, equal FPR, and equal FNR. As for $c_3$, we will want to find the value of it that achieves the best results, for any combined objective of accuracy and fairness defined by a specific selection of $c_1,c_2$. Therefore, given a specific selection of $c_1, c_2$, we apply cross-validation to select the value of $c_3$. 




We briefly describe the other algorithmic approaches to which we compare:\\
\textbf{Zafar et al.}~(\citeyear{disparatemistreatment}) performs optimization by considering a proxy for the bias: the covariance between the samples' sensitive attributes and the signed distance between the feature vectors of misclassified users and the classifier decision boundary.\\
\textbf{Zafar et al. Baseline}~(\citeyear{disparatemistreatment}) tries to enforce equal FP/FN rates on the different groups by introducing different penalties for misclassified data points with different sensitive attribute values during the training phase.\\
\textbf{Hardt et al.}~(\citeyear{hardt}) performs post-processing on a standard trained (unfair) logistic regressor, picking different decision thresholds for different groups, and possibly adding randomization.


\subsection{Experimental Results}

In what follows, we use the following notation, given a trained classifier $\hat{Y}$:
\begin{align*}
\mathbf{D_{FPR}}&=\left|FPR_{A=0}(\hat{Y})-FPR_{A=1}(\hat{Y})\right| \\ 
\mathbf{D_{FNR}}&=\left|FNR_{A=0}(\hat{Y})-FNR_{A=1}(\hat{Y})\right|
\end{align*}
The values $FPR_{A=0}(\hat{Y})$, $FPR_{A=1}(\hat{Y})$, $FNR_{A=0}(\hat{Y})$, $FNR_{A=1}(\hat{Y})$ are reported as evaluated on the test set.

\paragraph{The COMPAS Dataset\footnote{https://github.com/propublica/compas-analysis}} The Correctional Offender Management Profiling for Alternative Sanctions (COMPAS) records from Broward County, Florida 2013-2014, made available online by ProPublica, are perhaps the best-studied data in the context of fairness.  The goal in this scenario is to successfully predict recidivism within two years, based on features such as age, gender, race, number of prior offenses, and charge degree. The dataset contains 5,278 samples. The protected attribute in this scenario is race, where $A$ indicates black or white. We filtered the dataset using the same features as Zafar et al.~(\citeyear{disparatemistreatment}), to allow for comparison.

%\begin{table}[h]
%\centering
%\begin{tabularx}{\columnwidth}{c|c|c|c}
%\hline
%  &  Recid. ($y = 1$)        & No Recid.  ($y = 0$)       & Total \\ \hline
%Black &  $ 1661   $ & $ 1514 $ &  $ 3175 $ \\ \hline
%White &  $ 822   $  & $1281  $ &  $ 2103 $ \\ \hline
%Total &  $ 2483  $  & $2795 $ &  $ 5278 $ \\\hline
%\end{tabularx}
%\caption{Statistics of the ProPublica COMPAS data.} \label{table:compas-stats}
%\label{tab:stats}
%\end{table}
%\vspace{-1em}

%\begin{table}[h]
%\centering
%\begin{tabularx}{\columnwidth}{c|c}
%\hline
%Feature  &  Description \\ \hline
%Age Category &  $<25$, between $25$ and $45$, $>45$ \\
%Gender &  Male or Female \\
%Race &  White or Black \\
%Priors Count &  0--37 \\
%Charge Degree &  Misconduct or Felony \\
%\hline
%2-year-recid. & Whether or not the  \\
%(target feature)  & defendant recidivated within two years
%\end{tabularx}
%\caption{Description of features used from ProPublica COMPAS data.} \label{table:compas-features}
%\label{tab:features}
%\end{table}




\begin{table*}[t]
\centering
\caption{A description of the datasets used, along with parameters of the training procedure used for each.}
\label{table:datasets_description}
\begin{adjustbox}{max width=\textwidth}
\begin{tabular}{|l|l|l|l|l|l|l|l|}
\hline
\textbf{Dataset} & \textbf{No. Samples} & \textbf{No. Features} & \textbf{Train/Test Split} & \textbf{No. Repetitions} & \textbf{No. Folds in CV} & \textbf{Protected Feature} & \textbf{Target Variable} \\ \hline
COMPAS           & 5,278                     & 5                          & 70-30                     & 5                        & 5                                 & Race                       & 2-Year-Recidivism        \\ \hline
Adult            & 30,162                    & 10                         & 30-70                     & 5                        & 5                                 & Gender                     & Income Over/Under 50K    \\ \hline
Default          & 30,000                    & 23                         & 30-70                     & 5                        & 3                                 & Gender                     & Defaulting On Payments   \\ \hline
Admissions       & 20,839                    & 17                         & 30-70                     & 5                        & 3                                 & Race                       & Passing Bar Exam         \\ \hline
\end{tabular}
\end{adjustbox}
\end{table*}


\begin{table*}[t]
\centering
\resizebox{\textwidth}{!}{
\def\arraystretch{1.2}

\begin{tabular}{c c c | c | c | c || c | c | c || c | c | c |}

\cline{4-12}
&&&
\multicolumn{3}{ c|| }{\textbf{Adult Dataset}}&
\multicolumn{3}{ c|| }{\textbf{Default Dataset}}&
\multicolumn{3}{ c| }{\textbf{Admissions Dataset}}
\\ \cline{4-12}
%&&&
%\multicolumn{3}{ c|| }{\textbf{Both Considerations}}&
%\multicolumn{3}{ c|| }{\textbf{Both Considerations}}&
%\multicolumn{3}{ c| }{\textbf{Both Considerations}}
%\\ \cline{4-12}
&&&
 $\mathbf{Acc.}$ &  $\mathbf{D_{FPR}}$ &  $\mathbf{D_{FNR}}$ &  $\mathbf{Acc.}$ &  $\mathbf{D_{FPR}}$ &  $\mathbf{D_{FNR}}$ &  $\mathbf{Acc.}$ &  $\mathbf{D_{FPR}}$ &  $\mathbf{D_{FNR}}$
\\  \cline{4-12}
\vspace*{-0.5ex}
\\ \cline{1-2} \cline{4-12}
\multicolumn{1}{ |c  }{} &
\multicolumn{1}{ c|  }{  \textbf{Our Method (AVD Penalizers)}}  &&
$\mathbf{0.776}$    &  $\mathbf{0.00}$  &  $\mathbf{0.04}$ &
$\mathbf{0.807}$    &  $\mathbf{0.00}$   &  $\mathbf{0.01}$ &
$\mathbf{0.950}$    &  $\mathbf{0.01}$  &  $\mathbf{0.00}$
\\ \cline{1-2} \cline{4-12}
\multicolumn{1}{ |c  }{} &
\multicolumn{1}{ c|  }{  \textbf{Our Method (SD Penalizers)}}  &&
$\mathbf{0.783}$    &  $\mathbf{0.00}$  &  $\mathbf{0.09}$ &
$\mathbf{0.806}$    &  $\mathbf{0.01}$   &  $\mathbf{0.02}$ &
$\mathbf{0.950}$    &  $\mathbf{0.00}$  &  $\mathbf{0.00}$
\\ \cline{1-2} \cline{4-12}
\multicolumn{1}{ |c  }{} &
\multicolumn{1}{ c|  }{  \textbf{Vanilla Regularized Logistic Regression}}  &&
$\mathbf{0.800}$    &   $\mathbf{0.08}$    &   $\mathbf{0.39}$  &
$\mathbf{0.807}$    &   $\mathbf{0.01}$    &   $\mathbf{0.05}$  &
$\mathbf{0.951}$    &   $\mathbf{0.16}$    &   $\mathbf{0.02}$
\\ \cline{1-2} \cline{4-12}
\end{tabular}
}
\vspace{3mm}
\caption{Performance on the Adult, Loan Default, and Admissions datasets, penalizing for both FPR and FNR difference. Accuracy, FPR difference and FNR difference are evaluated on the test set, averaging over five runs and using a 30-70 training/test split.} \label{table:comparison_results_rest}
\end{table*}


In Table~\ref{table:comparison_results}, we compare the performance of our approach with that of three other techniques from the literature. Each method was trained based on logistic regression.  As a basis for comparison, we also present the performance of vanilla logistic regression, absent fairness considerations, with the regularization parameter selected via cross-validation.\footnote{Zafar et al.~(\citeyear{disparatemistreatment}) do not incorporate regularization in any of the approaches they report.}
%Results are reported as the averages of 5 different runs \katrina{Is that still correct?}, each time splitting data evenly and randomly into training and test sets. 
Results for Zafar et al., Zafar et al. baseline, and Hardt et al. appear here as reported in Zafar et al.~(\citeyear{disparatemistreatment}).\footnote{Our method selects the classifier based on the training set only and reports its performance over the test set. Results for the three other approaches, reported by Zafar et al.~(\citeyear{disparatemistreatment}), are based on tuning parameters after seeing the trade-off curve over the test set, and reporting according to the best selection of these parameters.}
%\katrina{Perhaps here is the right place for a footnote about the discrepancy with the Zafar baseline}

We find that the vanilla logistic regressor (absent fairness considerations) results in significant unfairness, as $\mathbf{D_{FPR}}=0.20$, and $\mathbf{D_{FNR}}=0.30$. The overall accuracy of this classifier measured on the test set was $0.672$.\footnote{Zafar et al.~(\citeyear{disparatemistreatment}) report a slightly different baseline of: Accuracy = 0.668, $\mathbf{D_{FPR}}=0.18$, $\mathbf{D_{FNR}}=0.30$.} Our SD penalization approach empirically achieves approximately the same accuracy as the Zafar et al.~(\citeyear{disparatemistreatment}) approach, with significantly better fairness. It is difficult to compare fairness-accuracy tradeoffs with the Hardt et al.~(\citeyear{hardt}) approach, since their accuracy is significantly lower than ours. A more direct comparison is possible by noting that our learned classifier can be post-processed to improve its fairness at a direct cost to accuracy. Hence, we can achieve accuracy of $0.659$ with $\mathbf{D_{FPR}} = \mathbf{D_{FNR}} = 0.01$, which compares very favorably with the Hardt et al. accuracy rate of 0.645 given the same FPR and FNR rates.\footnote{For completeness, we note that using a 50-50 training-test split (again not using the test set for parameter selection), our method (SD, both considerations) produces a classifier that provides: Accuracy = 0.659, $\mathbf{D_{FPR}} = 0.01, \mathbf{D_{FNR}} = 0.05$. This classifier can be post-processed to achieve rates of: Accuracy = 0.655, $\mathbf{D_{FPR}} = \mathbf{D_{FNR}} = 0.01$.}

Figure \ref{fig:compas} illustrates the accuracy/fairness trade-offs achievable using our scheme. Increasing the weight $c$ on the proxy fairness penalizers results in reducing their magnitude. The figure also illustrates how our relaxed penalizers succeed in tracking the real FPR and FNR differences. 
%
%
%\katrina{Must rewrite the following paragraph}
%We observe that our method succeeds in eliminating unfairness almost completely on the COMPAS dataset, while retaining most of the accuracy, when compared to the vanilla logistic regression. We achieve very low difference rates when penalizing for achieving each of the FPR and FNR criteria individually, and also for both. We achieve preferable results comparing to Zafar et al. and Zafar et al. baseline in all 3 scenarios, and also comparing to Hardt et al. in the settings of false positive/false negative considerations only. In the setting of both considerations - The Hardt et al. method removes a larger portion of the unfairness, however it results in major accuracy loss as it achieves accuracy rate of 0.645 in comparison to our method which results in accuracy of 0.665, retaining most of the original accuracy rate while removing most of the unfairness.




%The Hardt et al.~\cite{hardt} approach as reported removes a smaller portion of the bias in the different scenarios, however for FP/FN constraints alone, it provides higher accuracy rates. The Zafar et al.~(\citeyear{disparatemistreatment}) approach as reported retains significant bias (in most cases), but in some cases  achieves slightly superior accuracy rates to the methods above. 

%These performance comparisons are incomplete in the sense that each of the compared techniques has the potential to trade off between accuracy and fairness, using some degree of parameter tuning; what we report here is only one point on the achievable trade-off frontier for each algorithm. The ``correct'' trade-off, and, in particular, the best manner in which to weigh unfairness in the FPR against unfairness in the FNR, are matters of opinion. We have chosen to report our method's performance under parameters designed to very aggressively mitigate unfairness, at some cost to the accuracy.

%It would certainly be desirable to evaluate these and other approaches to fair learning on other datasets and on different tasks, particularly on larger datasets, which might afford both greater accuracy and better bias-reduction. The present empirical evaluations, however, suggest that our regularization-based approach provides a new tool worthy of consideration---we succeed in almost entirely eliminating bias on the hold-out set, at a modest price in terms of accuracy.

%Due to the fact that our true objective includes the original non-convex penalization terms, our approach does not carry any formal guarantees. However, the ease of implementation, generality, and empirical results are encouraging. Figure~\ref{fig:test1} illustrates the rate of convergence to a fair, accurate classifier on this dataset.
%In terms of computation costs, given that at each iteration we must calculate the gradient according to the FPR and FNR regularizers, we are required to predict the labels for the entire training set at each step. 
%However, this does not pose a computational burden, as it is already required by the (classic) gradient descent algorithm in our logistic regressor fitting scheme. Furthermore, when given a sufficiently large dataset (one or two orders of magnitude larger than the one currently available for the COMPAS scores data), this could be relaxed to sampling only a mini-batch of samples from the training data set at each iteration (much as is done in stochastic gradient descent).






\subsection{Additional Datasets}


Table~\ref{table:datasets_description} provides summary statistics on each of the datasets on which we tested our approach. We also briefly describe the datasets below. 


{\bf The Adult Dataset}\footnote{http://archive.ics.uci.edu/ml/datasets/Adult} is based on 1994 US Census data. The task we consider is to predict whether the income of each individual is over or under 50K dollars per year, based on features such as occupation, marital status, and education. The protected attribute selected in this task is gender. 

{\bf The Loan Default Dataset}\footnote{{\scriptsize https://archive.ics.uci.edu/ml/datasets/default+of+credit+card+clients}}
contains data regrading Taiwanese credit card users. The task we consider is to predict whether an individual will default on payments, based on features such as history of past payments, age, and the amount of given credit. The protected attribute is gender.

{\bf The Admissions Dataset}\footnote{http://www2.law.ucla.edu/sander/Systemic/Data.htm}
contains records of law school students who went on to take the bar exam. The task we consider is to predict whether a student will pass the exam based on features such as LSAT score, undergraduate GPA, and family income. The protected attribute is set to race.

Table~\ref{table:comparison_results_rest} describes the performance of our approach on these datasets, and Figures~\ref{fig:adult},~\ref{fig:default}, and~\ref{fig:lawschool} illustrate the fairness-accuracy trade-offs we achieve in each context. Overall, we see that unfairness is nearly eliminated while accuracy remains quite high. The dataset on which accuracy suffers most under our approach is the Adult dataset, which is also the dataset on which the vanilla regression is the most unfair.


\begin{figure*}[]
  \includegraphics[scale=0.6]{adult0-800.png}
  \caption{Adult Dataset. Fairness-Accuracy tradeoffs, as in Figure~\ref{fig:compas}.}
  \label{fig:adult}  
\end{figure*}



\begin{figure*}[]
  \includegraphics[scale=0.6]{default0-50.png}
  \caption{Loan Default Dataset. Fairness-Accuracy tradeoffs, as in Figure~\ref{fig:compas}.}
  \label{fig:default}
\end{figure*}



\begin{figure*}[]
  \includegraphics[scale=0.6]{admissions0-400.png}
  \caption{Admissions Dataset. Fairness-Accuracy tradeoffs, as in Figure~\ref{fig:compas}.}
  \label{fig:lawschool}
\end{figure*}




\section{CONCLUSIONS}
In this paper, 2D and 3D CNN models were used to generate pelvic sCTs from T1-weighted MR images. Our sCT generation methods were fully automated, requiring no deformable registration or manual segmentation of bone tissues. As shown in Figure~\ref{fig3}, the 2D and 3D CNN models generated high quality sCTs. MAE curves shown in Figure~\ref{fig4} indicated that both models could precisely estimate soft-tissue HU values but had difficulty in reproducing air and high-density bone tissues. 

The MAEs within the body contour across all patients were 40.5 $\pm$ 5.4 HU and 37.6 $\pm$ 5.1 HU for the 2D and 3D models, respectively. The time required for generating a pelvic sCT using our CNN models was about 5.5 s. Our MAE results are comparable to previous studies. Kim $et \ al.$\cite{RN41} presented a voxel-based weighted summation method that produced an MAE of 74.3 $\pm$ 3.9 HU. However, manual contouring of bone tissues required for this method can be tedious and time-consuming. An MAE of 40.5 $\pm$ 8.2 HU was achieved by Dowling $et \ al.$\cite{RN11} using an average MRI-CT atlas from 38 patients. Andreasen $et \ al.$\cite{RN42} reported an MAE of 54 $\pm$ 8 HU using an atlas-based method with pattern recognition, and its prediction time was about 20.8 min. Another random forest model proposed by Andreasen $et \ al.$\cite{RN43} generated sCTs with an MAE of 58 $pm$ 9 HU. A hybrid method suggested by Siversson $et \ al.$ \cite{RN45} obtained an MAE of 36.5 $\pm$ 4.1 HU when ignoring errors introduced by gas cavities. This hybrid method was implemented in the cloud-based commercial software MriPlanner (Spectronic Medical AB, Helsingborg, Sweden), which required 50 to 80 min to generate a sCT.\cite{RN45} The patch-based 3D context-aware generative adversarial network presented by Nie $et \ al.$\cite{RN26} achieved an MAE of 39.0 $\pm$ 4.6 HU. 

Our CNN models reproduced low-density bone as shown in Figure ~\ref{fig4}. The bone-region DSCs were 0.81 $\pm$ 0.04 and 0.82 $\pm$ 0.04 from the 2D and 3D models, respectively. These results are comparable to reported DSC results of 0.79 $\pm$ 0.12\cite{RN10} and 0.91$\pm$0.03{\cite{RN11}}, where the authors compared bone contours manually drawn on the sCT and CT.

It was feasible to train the proposed 3D model with 16 image volumes from scratch. Results of the Wilcoxon signed-rank tests shown in Table~\ref{tab1} demonstrated a statistically significant improvement in overall MAE, bone DSC, and bone precision of the 3D model compared to the 2D model. However, as shown in Figure~\ref{fig4}, the 2D model seemed to perform better in estimating the high-density bone HU values. It should be noted that smaller overall MAEs do not guarantee improved sCT dose calculation and patient positioning performance. While the models performed well, we will continue to acquire more patient data to potentially improve model accuracy and further test model differences.

As this was a retrospective study, the MR image voxel sizes were not matched, resulting in different voxel intensities between images. This may have affected the sCT generation accuracy although we applied intensity normalization. A potential study could examine how voxel size variations affects sCT estimation. 

The proposed 3D model can be implemented on a 12 GB GPU to process volumetric images with dimensions of 256 $\times$ 256 $\times$ 30. More GPU memory would be required to process higher resolution 3D images. Considering the limited access to multi-GPU systems, a 3D architecture with fewer convolutional layers could be considered to deal with higher resolutions. However, the performance could be affected by the reduced parameters and smaller receptive fields of the less complex model. Another approach would be to extract 30-slice sub-volumes from CT and MR images for training the 3D model. The sCT could then be generated by averaging 30-slice sCT sub-volumes produced by the model. 

A number of techniques could be investigated for improving model performance.  Nie $et \ al.$\cite{RN26} showed that introducing an additional adversarial discriminator improved overall sCT quality. The same approach could be adapted in our proposed 2D and 3D CNN models.  Non-rigid deformation\cite{RN44} could also be applied to both CT and MR images in the process of the on-the-fly data augmentation to produce more training pairs. Multiple MR images acquired with different sequences could be fed into models to provide more information for distinguishing different tissues. Multi-GPU systems with more memory would enable the exploration of larger batch sizes for training CNN models, which could reduce variances in gradient estimation and accelerate the training. 



% \addtolength{\textheight}{-10cm}   % This command serves to balance the column lengths
                                  % on the last page of the document manually. It shortens
                                  % the textheight of the last page by a suitable amount.
                                  % This command does not take effect until the next page
                                  % so it should come on the page before the last. Make
                                  % sure that you do not shorten the textheight too much.

%%%%%%%%%%%%%%%%%%%%%%%%%%%%%%%%%%%%%%%%%%%%%%%%%%%%%%%%%%%%%%%%%%%%%%%%%%%%%%%%



%%%%%%%%%%%%%%%%%%%%%%%%%%%%%%%%%%%%%%%%%%%%%%%%%%%%%%%%%%%%%%%%%%%%%%%%%%%%%%%%



%%%%%%%%%%%%%%%%%%%%%%%%%%%%%%%%%%%%%%%%%%%%%%%%%%%%%%%%%%%%%%%%%%%%%%%%%%%%%%%%
% \section*{APPENDIX}
% TODO

% \section*{ACKNOWLEDGMENT}
% TODO



%%%%%%%%%%%%%%%%%%%%%%%%%%%%%%%%%%%%%%%%%%%%%%%%%%%%%%%%%%%%%%%%%%%%%%%%%%%%%%%%
\bibliographystyle{bib/IEEEtran}
% \bibliographystyle{bib/IEEEtranN}
\bibliography{bib/IEEEabrv,bib/ref}

\end{document}
