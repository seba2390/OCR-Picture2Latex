% \wenzhen{If you don't have anything to discuss here, you should have 'discussion' in the title}
Rotational displacement during grasping is a common grasp failure, in which the robot grasps an object at a location far from its center of gravity. In this paper, we present an approach to detect rotational grasp failure from tactile sensing and a closed-loop regrasp 
% \wenzhen{you could use re-grasp or regrasp, but it should be consistent across the paper}
system to stabilize the grasp. The method is based on measuring the rotation angle of the soft tactile sensor's surface from tactile images, which directly correlates to the torque at the grasp surface. With the feedback of the rotation detection from tactile sensing, the robot attempts multiple regrasps and reaches a stable grasp location in the end to prevent huge rotation from happening.

% We have shown the success rate as 92.14\% in offline experiments and 90.8\% in regrasping experiments to detect rotation early and prevent grasp failures. \wenzhen{Those technical-level discussion should be in the experiment result part}However, there is space for improvement. The results show our approach's robustness to flat objects, but it is challenging to implement objects with complex shapes with curvatures. The curvatures can result in random marker motions on the gel, confusing the algorithm to classify the rotational and stable grasp.
% \Wtext{In future, we plan to extend our control framework to objects of irregular shapes and potentially use multi-fingered grippers and explore how they can give better tactile sensing capabilities. } 

In the future work, we plan to explore how learning based techniques can perform on the similar task of rotation measurement. In order for closed-loop regrasp control to work on objects of varied dimensions and shapes, we plan to include object detection and size estimation from external visual equipment and guide grasping. Also, we wish to explore control frameworks for multi-fingered robotic arms equipped with tactile sensors using our proposed method for measuring rotation. We believe higher camera fps and lower latency can help in scaling our proposed method to industrial applications. 

% \vspace{5mm}
%  \vspace{-2mm}
% In our future work, we will further analyze the generalization of rotation detection, such as with grasping complex-shaped objects or tracking loss situations.
% \wenzhen{Why do you say that? Does the current method fail with some type of objects?}And we will aim to improve the rotation detection's robustness to handle more general cases.

% Instead of tracking markers and analyzing markers' motion, we will consider the geometry of the contact surface and the surface's motion to capture more information about rotation patterns and improve rotation detection's robustness.