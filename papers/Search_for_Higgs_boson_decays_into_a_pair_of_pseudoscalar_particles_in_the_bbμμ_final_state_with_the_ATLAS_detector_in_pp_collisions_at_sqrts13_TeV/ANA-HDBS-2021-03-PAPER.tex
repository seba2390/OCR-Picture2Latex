\documentclass[cernpreprint, backref=false, texlive=2020, UKenglish, dvipsnames, block=none, texmf]{atlasdoc}
 
 
\usepackage{atlaspackage}
\PreprintIdNumber{CERN-EP-2021-157}
 
\interfootnotelinepenalty=10000
\usepackage{atlasbiblatex}
 
\usepackage{atlasphysics}
 
\addbibresource{ANA-HDBS-2021-03-PAPER.bib}
\addbibresource{ATLAS.bib}
\addbibresource{ATLAS-useful.bib}
\addbibresource{CMS.bib}
\addbibresource{ConfNotes.bib}
\addbibresource{PubNotes.bib}
 
\graphicspath{{logos/}{figures/}}
 
\usepackage{ANA-HDBS-2021-03-PAPER-defs}
\usepackage{atlassnippets}
\usepackage{atlasmisc}
\usepackage{atlasjetetmiss}
 
% The next lines are included from the .//ANA-HDBS-2021-03-PAPER-metadata.tex input file
 
\AtlasTitle{Search for Higgs boson decays into a pair of pseudoscalar particles in the $bb\mu\mu$ final state with the ATLAS detector in $pp$ collisions at $\sqrt{s}=13$~TeV}
 
\AtlasVersion{2.0}
\AtlasAbstract{
This paper presents a search for decays of the Higgs boson with a  mass of 125~GeV into a pair of new pseudoscalar particles, $H\rightarrow aa$, where one $a$-boson decays into a $b$-quark pair and the other into a muon pair.
The search uses 139~fb$^{-1}$ of proton--proton collision data at a center-of-mass energy of $\sqrt{s}=13$~TeV recorded between 2015 and 2018 by the ATLAS experiment at the LHC. A narrow dimuon resonance is searched for in the invariant mass spectrum between 16~GeV and 62~GeV.
The largest excess of events above the Standard Model backgrounds is observed at a dimuon invariant mass of 52~GeV and corresponds to a local (global)  significance of $3.3 \sigma$ ($1.7 \sigma$).
Upper limits at  95\% confidence level are placed on the
branching ratio of the Higgs boson to the $bb\mu\mu$ final state, $ \mathcal{B}(H\rightarrow aa\rightarrow bb\mu\mu)$, and are in the range $\text{(0.2--4.0)} \times10^{-4}$, depending on the signal mass hypothesis.
 
}
 
\AtlasJournal{PRD}
\author{The ATLAS Collaboration}
 
\AtlasRefCode{HDBS-2021-03}
 
 
 
 
 
 
 
 
 
 
 
 
\AtlasJournalRef{Phys. Rev. D 105 (2022) 012006}
\AtlasDOI{10.1103/PhysRevD.105.012006}
% End of text imported from the .//ANA-HDBS-2021-03-PAPER-metadata.tex input file
\hypersetup{pdftitle={ATLAS document},pdfauthor={The ATLAS Collaboration}}
 
\begin{document}
 
\maketitle
 
 
 
% The next lines are included from the .//Introduction.tex input file
\section{Introduction}
 
 
 
 
Light (pseudo) scalars that couple to the 125~\GeV\ Higgs boson~\cite{HIGG-2012-27,CMS-HIG-12-028} appear in many well-motivated extensions of the Standard Model (SM)~\cite{Dobrescu:2000yn,Ellwanger:2003jt,Dermisek:2005ar,Chang:2008cw,Morrissey:2008gm,Belanger:2013kya}.
These include models addressing the baryon asymmetry of the universe~\cite{Profumo:2007wc,Blinov:2015sna}, offering a solution to the naturalness problem~\cite{Craig:2015pha,Curtin:2015fna}, or providing insights into the nature of dark matter~\cite{Silveira:1985rk,Pospelov:2007mp,Draper:2010ew,Ipek:2014gua,Martin:2014sxa,Boehm:2014hva,GammaRayCascade}. Light bosons produced in Higgs boson decays could also be mediators to dark sectors that do not couple to the SM otherwise~\cite{Curtin:2013fra,TaniaTim1,TaniaTim2,TaniaTim3,ALPS}. Furthermore, pseudoscalar mediators appear in models, such as those described in Ref.~\cite{Liu_2019}, that were proposed to explain the anomalous muon magnetic moment~\cite{mug2_Fermi}.
A combination of ATLAS measurements of the Higgs boson production cross-sections and branching ratios
constrains the branching ratios  into invisible and undetected states to be  $\mathcal{B}(H\rightarrow\text{inv})<30\%$ and $\mathcal{B}(H\rightarrow\text{undetected})<21\%$, respectively,
whereas the overall branching fraction of the Higgs boson into beyond-the-SM (BSM) states is determined to be less than 47\% at 95\% confidence level (CL)~\cite{HIGG-2018-57}.
Combined measurements of Higgs boson couplings performed by the  CMS Collaboration set upper limits  of $\mathcal{B}(H\rightarrow\text{inv})<22\%$ and $\mathcal{B}(H\rightarrow\text{undetected})<38\%$ at 95\% CL~\cite{CMS-HIG-17-031}.
This motivates searches for light states in the Higgs boson decays that probe this potentially large $\mathcal{B}(H\rightarrow\text{BSM})$.
 
 
 
This paper presents a search for decays of the 125~\GeV\ Higgs boson into two pseudoscalars, denoted by $a$, in proton--proton ($pp$) collisions at the LHC~\cite{Evans:2008zzb}.
The search is performed in events where one $a$-boson decays into two $b$-quarks and the other into two muons, $H\rightarrow aa \rightarrow b\bar{b}\mu^+\mu^-$.\footnote{Denoted by $\haabbmm$ from now on for the rest of the paper.} The $a$-bosons are assumed to have  a decay width that is narrow compared to the detector resolution.
As pseudoscalar couplings are generally proportional to  mass,
which is for example the case in two-Higgs-doublet models~\cite{Curtin:2013fra,Branco:2011iw},
the $bb\mu\mu$ final state provides a good balance between a high branching ratio from the $a\to bb$  decay and a clean, high mass-resolution, dimuon resonance signature that is easy to trigger on from the $a\to\mu\mu$  decay.
In scenarios with enhanced lepton couplings,
the $a\to\mu\mu$ branching ratio can also be relatively large, resulting in $\mathcal{B}(H\rightarrow aa\rightarrow bb\mu\mu)/ \mathcal{B}(H\rightarrow aa )$ of  up to $ 0.16\%$~\cite{Curtin:2014pda}.
 
Light resonances in Higgs boson decays have been searched for by ATLAS and CMS in many different channels, i.e.\ in the  final states involving $4\mu$~\cite{CMS-HIG-18-003,EXOT-2016-22}, $2\mu2\tau$ or $4\tau$~\cite{CMS-HIG-18-024,CMS-HIG-18-006,CMS-HIG-17-029,CMS-HIG-16-015,HIGG-2014-02}, $2b2\tau$~\cite{CMS-HIG-17-024}, $4b$~\cite{HIGG-2017-05,HDBS-2018-47}, $4\gamma$~\cite{EXOT-2013-24} and $2\gamma+2$-jets~\cite{HIGG-2017-09}.
A search for a dimuon resonance produced in association with b-jets has been performed by CMS~\cite{CMS:2018lce} and a light resonance decaying to two muons has been searched for by LHCb~\cite{LHCb:2020ysn}.
CMS has performed a search for \haabbmm in 35.9~$\ifb$ of $pp$ collision data at a center-of-mass energy of $\sqrt{s}=13$~\TeV\ that sets upper limits on $\mathcal{B}(H \rightarrow aa \rightarrow bb\mu\mu)$ of $\text{(1--7)}\times 10^{-4}$ for $a$-boson masses ($m_a$) in the range $20 \leq m_a \leq 62.5$~\GeV~\cite{CMS-HIG-18-011}.
The ATLAS search  based on 36~$\ifb$ of Run~2 data~\cite{bbmm36ifb} sets upper limits on $\mathcal{B}(H \rightarrow aa \rightarrow bb\mu\mu)$ between $1.2\times10^{-4}$ and $8.4\times10^{-4}$ for $a$-boson masses  in the range $20 \leq m_a \leq 60$~\GeV.
In this paper, the full Run~2 dataset corresponding to an integrated luminosity of 139~$\ifb$ is used and the search is extended down to $m_a=16$~\GeV\ and up to $m_a=62$~\GeV.
Additionally,   boosted decision tree (BDT) techniques are used to improve the separation of  the signal from the SM backgrounds, increasing the analysis sensitivity, especially for higher $m_a$.
 
 
% End of text imported from the .//Introduction.tex input file
 
% The next lines are included from the .//Detector.tex input file
\section{ATLAS detector}
 
 
The ATLAS experiment~\cite{PERF-2007-01,PIX-2018-001} is a multipurpose particle detector
with a forward--backward symmetric cylindrical geometry and a nearly \(4\pi\) coverage in
solid angle.\footnote{ATLAS uses a right-handed coordinate system with its origin at the nominal interaction point (IP)
in the center of the detector and the \(z\)-axis along the beam pipe.
The \(x\)-axis points from the IP to the center of the LHC ring,
and the \(y\)-axis points upwards.
Cylindrical coordinates \((r,\phi)\) are used in the transverse plane,
\(\phi\) being the azimuthal angle around the \(z\)-axis.
The pseudorapidity is defined in terms of the polar angle \(\theta\) as \(\eta = -\ln \tan(\theta/2)\).
Angular distance is measured in units of \(\Delta R \equiv \sqrt{(\Delta\eta)^{2} + (\Delta\phi)^{2}}\).} It consists of an inner  detector (ID) surrounded by a thin superconducting solenoid
providing a \SI{2}{\tesla} axial magnetic field, electromagnetic (EM) and hadron calorimeters, and a muon spectrometer (MS).
The inner tracking detector covers the pseudorapidity range \(|\eta| < 2.5\).
It consists of silicon pixel, silicon microstrip, and transition radiation tracking detectors.
Lead/liquid-argon (LAr)  sampling calorimeters provide electromagnetic energy measurements
with high granularity.
A steel/scintillator-tile hadron calorimeter covers the central pseudorapidity range $|\eta| < 1.7$.
The endcap and forward regions are instrumented with LAr calorimeters
for both the EM and hadronic energy measurements up to \(|\eta| = 4.9\).
The MS surrounds the calorimeters and is based on
three large superconducting air-core toroidal magnets with eight coils each.
The field integral of the toroids ranges between \num{2.0}~Tm and \SI{6.0}Tm
across most of the detector.
The MS includes a system of precision tracking chambers and fast detectors for triggering.
A two-level trigger system is used to select events.
The first-level trigger is implemented in hardware and uses a subset of the detector information
to accept events at a rate below \SI{100}{\kHz}~\cite{TRIG-2016-01}.
This is followed by a software-based trigger that
reduces the accepted event rate to \SI{1}{\kHz} on average.
An extensive software suite~\cite{ATL-SOFT-PUB-2020-001} is used in the reconstruction and analysis of real and simulated data, in detector operations, and in the trigger and data acquisition systems of the experiment.
 
 
 
% End of text imported from the .//Detector.tex input file
% The next lines are included from the .//Samples.tex input file
\section{Dataset and simulated events}
 
The data used in this analysis were collected in Run~2 of the LHC during the 2015--2018 data-taking period with $pp$ collisions at a center-of-mass energy of $\sqrt{s} = 13$~\TeV. The dataset corresponds to an integrated luminosity of 139~$\text{fb}^{-1}$.
The lowest-threshold unprescaled single-muon and dimuon triggers are used to select the events~\cite{TRIG-2018-01}. Single-muon triggers require the transverse momentum (\pT) of the muon  to be above 20~\GeV\ or 26~\GeV, depending on the data-taking period, while the dimuon trigger requires both muons to have a \pT above 14~\GeV.
 
Simulated events are used in the estimation of the SM backgrounds. \textsc{Sherpa}~2.2.1~\cite{Gleisberg:2008ta,Bothmann:2019yzt} was used as the baseline generator for the Drell--Yan (DY)\,+\,jets, $W(\rightarrow \ell \nu)$+jets, diboson and triboson backgrounds. It is a multiparton matrix element and parton shower (PS) generator including hadronization~\cite{Schumann:2007mg,Hoeche:2011fd, Hoeche:2012yf, Catani:2001cc,Hoeche:2009rj}, with the \textsc{NNPDF3.0} parton distribution function (PDF) set at next-to-next-to-leading-order (NNLO) accuracy~\cite{Ball:2014uwa}.
The DY+jets and multiboson samples were generated with a minimum dilepton mass of 10~\GeV\ and 4~\GeV, respectively.
The \ttbar and single-top-quark samples were generated with \textsc{Powheg-Box~v2}~\cite{Frixione:2007vw,Nason:2004rx,Alioli:2010xd,Frixione:2007nw,Re:2010bp} using the \textsc{NNPDF3.0nlo} PDF in matrix element interfaced to \textsc{Pythia~8.230}~\cite{Sjostrand:2014zea} for the PS.
For the underlying-event description a set of tuned parameters called the \textsc{A14} tune~\cite{A14_article} was used, along with the \textsc{NNPDF2.3lo} PDF~\cite{Ball:2012cx}.
The \ttbar + vector-boson processes ($\ttbar{+}V$) were generated with \textsc{MadGraph5}\_aMC@NLO~2.3.3~\cite{Alwall:2014hca} interfaced to \textsc{Pythia}~8.210 for the PS. The underlying-event tune was the same as for the \ttbar sample.
\textsc{EvtGen}~\cite{Lange:2001uf} was used for the properties of the bottom and charm hadron decays in all simulated samples, except those simulated with \textsc{Sherpa}.
 
 
Higgs boson production through gluon--gluon fusion (ggF) was generated using the NNLOPS program~\cite{Hamilton:2012np,Hamilton:2015nsa} with \textsc{Powheg-Box~v2}~\cite{Frixione:2007vw,Alioli:2010xd, Campbell:2012am, Hamilton:2012rf}. The vector-boson fusion (VBF) processes were generated with \textsc{Powheg-Box~v2} at NLO accuracy~\cite{Nason:2009ai}. The Higgs boson mass was set to 125~\GeV. For both the ggF and VBF production processes, \textsc{Powheg-Box} was interfaced with \textsc{Pythia}~8.212 using  the AZNLO tune~\cite{STDM-2012-23}
for the simulation of the $H\rightarrow aa\rightarrow bb\mu\mu$ decays, where the $a$-boson is a pseudoscalar, as well
as for parton showering, hadronization and the underlying event.
The ggF Higgs boson production rate is normalized to the total cross-section predicted at
next-to-next-to-next-to-leading-order accuracy in QCD with NLO electroweak corrections applied~\cite{deFlorian:2016spz,Anastasiou:2015ema,Anastasiou:2016cez,Actis:2008ug,Anastasiou:2008tj} and amounts to 48.58~pb.
The VBF production rate is normalized to an approximate NNLO cross-section with the NLO electroweak corrections applied~\cite{Heinemeyer:2013tqa,Ciccolini2007,Ciccolini2008,BolzoniVBF}, which amounts to 3.8~pb.
The contribution from the associated production of a Higgs boson and a vector boson (VH) is calculated to be 3.5\% of the total  ggF+VBF cross-section and is accounted for by scaling the simulated ggF and VBF samples.
The contribution from Higgs boson production in association with a pair of top quarks is found to be negligible (below the percent level) and is neglected in the analysis.
Thirteen mass points were simulated for the ggF and VBF production modes, with the $a$-boson mass in the range $m_a = \text{16--62}$~\GeV.\footnote{More specifically, the simulated mass points are at $m_a = 16$, 18, 20, 25, 30, 35, 40, 45, 50, 52, 55, 60 and 62~\GeV.}
Below $m_a = 16$~\GeV\ the $b$-quarks coming from the decays of the $a$-boson tend to be so collimated due to its boost that they cannot be reconstructed as two separate $b$-jets (with a radius parameter of $R=0.4$). Another effect is that in the highly asymmetric decays of low-mass $a$-bosons, the subleading $b$-jet falls below the jet reconstruction threshold of 20~\GeV~\cite{PERF-2015-09}.
As a result, the signal acceptance falls below 0.2\% and the analysis loses sensitivity.
 
 
The effects of additional interactions in the same and neighboring beam-bunch crossings (pileup) were modeled for all simulated events by overlaying additional $pp$ collisions generated with \textsc{Pythia~8.186} using the \textsc{NNPDF2.3lo} PDF set and the A3 tune~\cite{Sjostrand:2007gs}. Simulated event samples are weighted to reproduce the distribution of the number of pileup interactions observed in the data. All the generated background and signal samples are processed through the  ATLAS detector simulation~\cite{SOFT-2010-01} based on $\GEANT4$~\cite{Agostinelli:2002hh} and reconstructed using the same software as for the data.
 
 
 
% End of text imported from the .//Samples.tex input file
% The next lines are included from the .//Selection.tex input file
\section{Event reconstruction and selection}
 
 
Muons are reconstructed by combining track information from the MS with tracks found in the ID \cite{MUON-2018-0333}. They also have to satisfy $\pt>5$~\GeV\ and $|\eta|<2.7$ (for $|\eta|>2.5$, only tracking information from the MS is used), and pass the $LowPt$ working point identification requirement defined in Ref.~\cite{MUON-2018-0333}.
Muon tracks must have a longitudinal impact parameter $z_0$ satisfying $|z_{0}\sin\theta|<0.5$~mm and a transverse impact parameter significance $|d_{0}|$/$\sigma_{d_{0}}<3$ relative to the primary interaction vertex, chosen as the reconstructed vertex with the highest sum of the $\pT^2$ of its associated tracks. Furthermore, muons are required to be isolated from the surrounding detector activity by requiring that the scalar sum of the \pt of additional inner detector tracks  and the sum of the  transverse momentum $E_\mathrm{T}$ of calorimeter energy deposits within a cone of size $\Delta R = 0.2$ around a muon be less than 15\% and 30\% of the muon \pt, respectively.
 
 
Jets are reconstructed using the anti-$k_t$ algorithm~\cite{Cacciari:2008gp} implemented in the \textsc{FastJet}\xspace package~\cite{Fastjet} with a radius parameter of $R = 0.4$. The inputs to the jet clustering are built by combining the information from both the calorimeters and the ID using a particle-flow algorithm~\cite{PERF-2015-09,JETM-2018-05}. Jets with $\pT < 60$~\GeV\ originating from pileup are suppressed with the jet-vertex-tagger (JVT)~\cite{ATLAS-CONF-2014-018}, a multivariate algorithm combining track-based variables. Selected jets are required to have $\pt > 20$~\GeV\ and $|\eta| <2.5$.
An algorithm (MV2c10) relying on multivariate techniques, taking as input the properties of displaced tracks and vertices reconstructed within a jet, is employed to identify (tag) jets containing $b$-hadrons~\cite{FTAG-2018-01}.
The MV2c10 tagger is used at 77\% $b$-jet identification efficiency, with an approximate misidentification probability of 25\% for jets arising from charm quarks, 6.3\% for hadronically decaying $\tau$-leptons, and 0.8\% for light-flavor jets as measured in simulated \ttbar events.
 
The missing transverse momentum (\met) is calculated as the magnitude of the negative vector sum of the transverse momenta of all the reconstructed and calibrated objects in the event, including a soft term that accounts for charged particles that are associated with the primary vertex, but not with any reconstructed object~\cite{PERF-2016-07,ATLAS-CONF-2018-023}.
 
 
The events selected for the analysis are required to have two muons of opposite charge, either with the leading and subleading muons satisfying $\pt^{\text{leading}}>27$~\GeV\ and $\pt^{\text{subleading}}>5$~\GeV, and the event being triggered by a single-muon trigger, or with both muons having $\pt >15$~\GeV, and the event being triggered by a dimuon trigger. The dimuon invariant mass, $\Mmm$, is required to be between 15~\GeV\ and 65~\GeV. Furthermore, the events must contain exactly two $b$-tagged jets with \pT above 20~\GeV.
 
 
A kinematic likelihood (KL) \cite{Erdmann:2013rxa} fit exploiting the equal invariant masses of the $bb$ and $\mu\mu$ systems in $\haa$ decays  is performed to improve the four-body invariant mass ($\Mbbmm$) resolution and  reduce the SM backgrounds. The same fit approach as considered in the previous ATLAS publication~\cite{bbmm36ifb} is used.
The dimuon invariant mass, $\Mmm$, is used to constrain the di-$b$-jet mass, as the former has a resolution approximately ten times better than the latter.
The $\Mmm$ resolution ranges between 0.4~\GeV\ at $m_a$ = 16~\GeV\ and 1.3~\GeV\ at $m_a = 62$~\GeV.
The fit maximizes the likelihood by shifting the $b$-jet energies within the resolution in order to satisfy the constraint $\Mmm \simeq \Mbb$.
The output of the fit is the logarithm of the maximum likelihood value, \LogL, which quantifies how well the event matches the $\Mmm = \Mbb$ hypothesis, characteristic of signal events.
The four-body invariant mass, recomputed after the KL fit, is denoted by $\KLMbbmm$ and is used for further event categorization.
 
Signal-like events are chosen by requiring that $110 <\KLMbbmm<140$~\GeV, and that $\LogL > -8$, which ensures that $\Mbb$ is compatible with $\Mmm$.
Finally, \met is required to be less than 60~\GeV\ to reduce the background from \ttbar events, which is one of the two major backgrounds and can contain large \met from neutrinos in top-quark decays. This  selection defines the ``inclusive'' signal region (SRincl) and is summarized in Table~\ref{selections}, along with the selection requirements for other analysis regions described later in the text.
 
A boosted decision tree (BDT) classifier implemented using the TMVA framework \cite{Speckmayer:2010zz} is employed to further reduce the SM backgrounds.
Its training is done in partially overlapping 8-\GeV-wide \Mmm windows centered at the $m_a$ values of each of the 12 generated signals,\footnote{One BDT was trained for each generated signal MC sample, except for $m_a = 52$~\GeV, as this sample was produced only at a later analysis stage.} in order to fully exploit their kinematic differences.
The background sample consists of \ttbar and DY+jets events, the two dominant backgrounds, combined in the proportions extracted from the background validation fit described in Section~\ref{sec:result}. The signal samples used for the training include ggF and VBF Higgs boson production samples combined according to their cross-sections.
The seven kinematic variables included in the training are:
\begin{itemize}
\item{$\Mbb$,}
\item{\LogL,}
\item{$\Delta R_{b_1b_2}$ (the angular distance between the two $b$-jets),}
\item{diff$\Delta R_bR_{\mu}=\Delta R_{b_1b_2}-\Delta R_{\mu_1\mu_2}$ (the difference between the angular separations between the two $b$-jets and the two muons),}
\item{$\Delta R_{bb\mu\mu}$ (the angular distance between the $bb$ and $\mu\mu$ systems),}
\item{$\overline{\Delta R_{b,\mu}} = \left[\Delta R_{b_1\mu_1}+\Delta R_{b_1\mu_2}+\Delta R_{b_2\mu_1}+\Delta R_{b_2\mu_2}\right]/4$ (the average angular distance  of all four combinations of a $b$-jet and a muon),}
\item{$\overline{m_{b\mu}}=\left[m_{b_1\mu_1}+m_{b_1\mu_2}+m_{b_2\mu_1}+m_{b_2\mu_2}\right]/4$ (the average mass of all four combinations of a $b$-jet and a muon).}
\end{itemize}
The distributions of these variables for the background and three representative signal masses are shown in Figure~\ref{BDTinputs}.
 
The $\Mbb$ variable helps separate the low-mass signal from the backgrounds, as $\Mbb$ peaks around 60~\GeV\ for the \ttbar and DY processes. The \LogL~peaks at higher values as the signal mass becomes smaller.
 
Due to a higher boost of a lighter $a$-boson, its decay products are collimated, resulting in $\Delta R_{b_1b_2}$ and  $\Delta R_{\mu_1\mu_2}$ being much smaller than for a signal from a heavier $a$-boson or for background processes.
As a consequence, diff$\Delta R_bR_{\mu}$ shows a narrow distribution centered around zero, while the background and a higher-mass signal exhibit a much broader diff$\Delta R_bR_{\mu}$ distribution.
 
 
The $\Delta R_{bb\mu\mu}$ variable helps enhance the sensitivity to higher signal masses. Heavier $a$-bosons are produced approximately at rest, resulting in the $\Delta R_{bb\mu\mu}$ distribution being relatively flat with a small peak at low values. As the signal mass decreases, the $\Delta R_{bb\mu\mu}$ distribution  transitions into a ``back-to-back'' topology, characteristic of both a low-mass signal and the background events.
 
 
Finally, the $\overline{\Delta R_{b,\mu}}$ and $\overline{m_{b,\mu}}$ variables provide another measure of how close the two $a$-bosons are in $\Delta R$.
In the ``back-to-back'' topology for lower signal masses, the muons are, on average, further away from the $b$-jets, while for heavier $a$-bosons produced approximately at rest, the average distance between the muons and the $b$-jets is smaller.
Consequently, $\overline{\Delta R_{b,\mu}}$ and $\overline{m_{b,\mu}}$ peak at high (low) values for low (high) signal masses, while the backgrounds peak somewhere between the two extreme signal topologies.
 
The output score of the BDT trained for a signal with mass $m_a$ is denoted by BDT$m_a$. The BDT$m_a$ distributions for $m_a$ = 20~\GeV, 40~\GeV\, and 60~\GeV\ are shown in Figure~\ref{BDToutputs}.
 
The final signal region (SR) bin for each signal mass is defined by imposing two requirements in addition to the SRincl selection: $m_a-X<\Mmm<m_a+X$ and  BDT$m_a>0.2$, where $X=1$~\GeV\ ($X=1.5$~\GeV) for $m_a\leq 45$~\GeV\ ($m_a>45$~\GeV). The widths of the SR bins and the BDT$m_a$ cut value are optimized to maximize the significance of signal over background events.
For masses at which no signal sample was generated, and, consequently, no BDT was trained, the BDT trained for the $m_a$ closest to the one being tested is used.
For example, when testing the  $m_a=32$~\GeV\ hypothesis, the requirement BDT$30>0.2$ is applied to select the events for the SR bin.
Signal yields for mass points where no signal sample was generated ($m_a=32$~\GeV\ in this example) are obtained by selecting events with BDT scores above 0.2 for the same BDT$m_a$ (BDT30 in this case) in all simulated mass points and interpolating using third-order splines.
To assess the uncertainty, the yields of the neighboring simulated mass points ($m_a=30$~\GeV\ and $m_a=35$~\GeV\ in this case) are interpolated using a linear function. The difference between the yields obtained using the splines and a linear function for the interpolation is assigned as a systematic uncertainty on the interpolated signal yield.
Using a BDT at a mass for which the training was not performed results in a negligible loss of significance relative to a BDT that was optimized for that mass point.
 
 
\begin{table}[h!]
\caption{Summary of the selection requirements for the  control (TCR and DYCR), validation (VR1 and VR2) and  inclusive signal (SRincl) regions in the analysis, as well as the final SR bins. The control and validation regions are defined in Section~\ref{sec:backgrounds}.}
\label{selections}
\begin{center}
\begin{tabular}{l|c|c|c|c|c}
\toprule
& TCR & DYCR & SRincl & VR1 & VR2 \\
\midrule
\midrule
\Mmm[\GeV] & \multicolumn{5}{c}{[15, 65]} \\
\midrule
\KLMbbmm[\GeV] & [110, 140] & [80, 110] or [140, 170]  & [110, 140]  & [170, 300] & [110, 140] \\
\midrule
\met[\GeV] &  $> 60$ & \multicolumn{4}{c}{$< 60$} \\
\midrule
\LogL & \multicolumn{4}{c|}{$> -8$} & [$-11$, $-8$] \\
\midrule
\midrule
SR bins & \multicolumn{5}{c}{  SRincl  \& BDT$m_a>0.2$ }  \\
& \multicolumn{5}{c}{ 2-\GeV-wide (3-\GeV-wide) \Mmm bins for $m_a\leq45$~\GeV\ ($m_a>45$~\GeV) }  \\
\bottomrule
\end{tabular}
\end{center}
\end{table}
 
 
The signal acceptance $\times$ efficiency varies between 0.3\% and 2.5\% for ggF Higgs boson production and between 0.2\% and 3.0\% for VBF production, where the lowest acceptance $\times$ efficiency is obtained for the lowest $m_a$, and grows as $m_a$ increases. The largest loss of acceptance occurs when requiring that there are two $b$-jets in the event, as one of the signal jets tends to fall below the reconstruction threshold of 20~\GeV. The fraction of signal events passing the two-$b$-jet requirement is less than 20\% for all mass points.
 
 
 
 
 
\begin{figure}
\centering
\includegraphics[width=0.45\hsize]{fig_01a.pdf}
\includegraphics[width=0.45\hsize]{fig_01b.pdf}
\includegraphics[width=0.45\hsize]{fig_01c.pdf}
\includegraphics[width=0.45\hsize]{fig_01d.pdf}
\includegraphics[width=0.45\hsize]{fig_01e.pdf}
\includegraphics[width=0.45\hsize]{fig_01f.pdf}
\includegraphics[width=0.45\hsize]{fig_01g.pdf}
\caption{Kinematic variables used as inputs to the BDT training. From top left to bottom right: $\Mbb$, \LogL, $\Delta R_{b_1b_2}$, diff$\Delta R_bR_{\mu}$, $\Delta R_{bb\mu\mu}$, $\overline{\Delta R_{b,\mu}}$, $\overline{m_{b,\mu}}$. The variables are plotted in SRincl.
All the distributions are normalized to unit area. The background histogram is the sum of the \ttbar and DY event templates, combined in the proportions extracted from the background validation fit described in Section~\ref{sec:result}.}
\label{BDTinputs}
\end{figure}
 
\begin{figure}
\centering
\includegraphics[width=0.45\hsize]{fig_02a.pdf}
\includegraphics[width=0.45\hsize]{fig_02b.pdf}
\includegraphics[width=0.45\hsize]{fig_02c.pdf}
\caption{Three BDT$m_a$ distributions, BDT20, BDT40, and BDT60, plotted in the \Mmm windows of SRincl, as indicated in the figures. The distributions are normalized to unit area. The background histogram is the sum of the \ttbar and DY event templates, combined in the proportions extracted from the background validation fit described in Section~\ref{sec:result}.}
\label{BDToutputs}
\end{figure}
 
 
 
 
 
 
 
 
 
 
 
 
 
% End of text imported from the .//Selection.tex input file
% The next lines are included from the .//Backgrounds.tex input file
\section{Background estimation}
\label{sec:backgrounds}
The dominant backgrounds in the analysis arise from the DY dimuon process in association with $b$-quarks and  pair production of top quarks ($\ttbar$) where each $W$ boson decays into a muon and a neutrino. These two backgrounds account for more than $96\%$ of background events in all analysis regions.
 
Two control regions are designed to constrain the  \ttbar and DY backgrounds. They are chosen so that they have negligible signal contamination, are kinematically as close as possible to SRincl, and maximize the contribution of one of the respective background processes.
A top-quark control region (TCR) is defined by inverting the $\met$ selection criterion in SRincl to $\met>60$~\GeV. This results in an event sample approximately 93\% pure in \ttbar events. The DY control region (DYCR) is defined in the 30 \GeV-wide \KLMbbmm sidebands of SRincl, i.e.\ by requiring $80 <\KLMbbmm<110$~\GeV\ or $140 <\KLMbbmm<170$~\GeV. Approximately 50\% of the events in DYCR originate from the DY process, whereas the rest mostly come from \ttbar production.
Two validation regions (VR1 and VR2) are used to validate the normalizations of the backgrounds. VR1 is defined in the $170 <\KLMbbmm<300$~\GeV\ range, while VR2 is obtained by inverting the \LogL\ selection criterion of SRincl to $-11< \LogL <-8$. All the analysis regions are summarized in Table~\ref{selections} and illustrated in Figure~\ref{regionsPic}.
 
The shapes of the $\ttbar$ kinematic variable  distributions are obtained from simulation, while the overall normalization is extracted from the fits described in Section~\ref{sec:result}.
The distributions for the DY background are taken from data templates because the limited sizes of the simulated event samples do not allow a reliable estimate. The template regions are defined in the same way as the analysis regions in Table~\ref{selections}, except that the two-$b$-tag requirement is replaced by a zero-$b$-tag requirement. The template regions are  ${>}95\%$ pure in DY events. Contributions from other processes, namely \ttbar, $W$+jets, diboson and single-top, are  subtracted using simulation.
Following the subtraction, the DY templates are corrected to account for kinematic differences between event samples dominated by jets originating from light quarks or gluons (template regions) and event samples dominated by $b$-jets (analysis regions). The correction is applied as a per-event weight, where the reweighting is derived from a comparison between two-$b$-tag and zero-$b$-tag kinematic distributions in simulated DY events.
Two sets of event weights are derived and applied sequentially.
First, the jet multiplicity of the zero-$b$-tag MC sample is reweighted to the one in the two-$b$-tag sample. It is the distribution with the largest difference between the zero- and two-$b$-tag samples and was hence corrected first.
Secondly, a BDT-based reweighting is employed to further correct the zero-$b$-tag template kinematics. A BDT is trained  on the zero-$b$-tag versus the two-$b$-tag simulated DY samples. The BDT input consists of kinematic properties and angular distributions of the $b$-jets, muons and the two corresponding $a$-boson candidates, as well as  $\met$ and  $\KLMbbmm$. The ratio of the BDT score distributions obtained for the two-b-tag and zero-b-tag simulated events is then applied as a weight to every event from the zero-b-tag DY template, as a function of its BDT score.
Following the BDT-based reweighting, the \KLMbbmm and $\met$ distributions are corrected by up to 20\%. The DY templates are normalized to data in the fits described in Section~\ref{sec:result}.
 
Minor backgrounds include diboson and single-top-quark production, production of a \ttbar\ pair in association with a vector boson, and $W$ boson production in association with $b$-jets.
The estimation of these minor backgrounds relies purely on simulation normalized to the best available theoretical prediction. The events where a jet is misidentified as a muon are taken into account as follows: non-prompt/misidentified muons in $W$+jets and \ttbar events are included in the analysis on the basis of simulation, any contribution of non-prompt/misidentified muons in the DY+jets component is accounted for by the data template, and the  potential contribution from multijet events is found to be negligible.
 
 
 
\begin{figure}
\centering
\includegraphics[width=0.5\hsize]{fig_03.pdf}
\caption{Illustration of the signal, control, and validation regions used in the analysis. VR2 (not shown) is defined by the same selection as SRincl, except that the requirement on \LogL\ is inverted to $-11 < \LogL < -8$.}
\label{regionsPic}
\end{figure}
% End of text imported from the .//Backgrounds.tex input file
% The next lines are included from the .//Systematics.tex input file
\section{Systematic uncertainties}
Systematic uncertainties in the analysis are divided into three
categories: experimental uncertainties affecting the simulated background and signal processes, uncertainties in the modeling of the DY template, and theoretical uncertainties of the simulated background and signal samples.
Table~\ref{tab:syst} shows a summary of the dominant systematic uncertainties in the total background and signal yields in the signal region bins, as resulting from the fits described in Section~\ref{sec:result} and hereafter denoted by "post-fit".
 
 
\begin{table}[!htbp]
\caption{Summary of the dominant post-fit systematic uncertainties in the background and signal yields. The uncertainties are expressed as a percentage of the total background and signal yields per $m_{\mu\mu}$ bin of the signal region.
Only uncertainties exceeding 2\% in at least one SR bin are shown.
\label{tab:syst}}
\begin{center}
\begin{tabular}{l|l|c|c}
\toprule
Category & Source &Total background [\%] & Signal [\%]  \\ \hline
 
\midrule
DY & BDT$m_a$ selection & ~~~~~7--14 &   -- \\
& normalization & ~~~~~5--10 &   -- \\
& \Mmm shape & ~~~1--8 &   -- \\
& kinematics & 0.3--6 &   --  \\
& background subtraction & 0.6--3  &   --  \\
\midrule
$\ttbar$ & hadronization/PS &   0.3--4 & --  \\
& hard-scatter generation  & 0.2--3 & --  \\
& normalization & 0.2--3 & --  \\
\midrule
Overall MC & Sample statistics &  ~~~~~8--40 &  1--2 \\
\midrule
Jets & $b$-tagging &  ~0.03--0.7 &  ~~9--10  \\
& Jet-energy resolution&  ~~~1--3 &  6--7 \\
& Jet-energy scale&  ~~~1--3 &  4--5  \\
\midrule
Signal & FSR & -- &  5 \\
& PS & -- & 4  \\
& VH contribution& -- & ~~~3.5  \\
& MPI & -- & 3  \\
& QCD scale & -- & 3  \\
& ISR & -- &  3 \\
& ggF cross-section & &   \\
& \hspace{1.2cm}- missing higher-order QCD  & -- & 5  \\
& \hspace{1.2cm}- PDF \&  $\alphas$  & -- & 3  \\
\bottomrule
\end{tabular}
\end{center}
\end{table}
 
 
 
Among the experimental uncertainties, the leading effects come from those associated with the
calibration and resolution of jet energies~\cite{PERF-2016-04}, and with the measurement of the $b$-tagging efficiency~\cite{FTAG-2018-01}. The impact of these uncertainties on the total background (signal) yields in the SR bins is as large as 3\% (10\%).
The uncertainty in the combined 2015--2018 integrated luminosity is 1.7\% \cite{ATLAS-CONF-2019-021}, obtained using the LUCID-2 detector \cite{LUCID2} for the primary luminosity measurement.
Other uncertainties, such as those arising from the muon identification efficiency, momentum scale and resolution~\cite{PERF-2015-10,MUON-2018-0333}, and pileup are found to have a negligible impact on the final yields.
 
The uncertainty arising from limited MC sample sizes ranges from 8\% to as large as 40\% in the low $m_a$ mass bins due to there being few \ttbar events in this region.
 
 
Five sources of uncertainty in the data-driven DY template are considered. The uncertainty in subtracting non-DY events from the non-reweighted template in the analysis regions is assessed by comparing the nominal template, for which the simulated non-DY backgrounds had been subtracted before reweighting, with an alternative template for which no subtraction had been performed.
The uncertainties in the template kinematics modeling
are derived by comparing the DY template with simulation in two key variables: \met and $\KLMbbmm$.
The ratios of the template to the simulated DY events are fit with linear functions and used in assigning uncertainties to the shapes of the \met and $\KLMbbmm$ distributions.
Similarly, the uncertainty in the  \Mmm template shape is assessed by comparing the template with the smoothed simulated sample and applying the observed difference  as a systematic uncertainty. 
The uncertainty in the normalization of the DY template is obtained from the fits to data.
Finally, the uncertainty in the efficiency of the BDT$m_a$ selection criteria is evaluated by taking the difference in the BDT$m_a$ cut efficiency, $N_{\text{DY events SR}}^{\text{BDTm}\mathrm{_a}\text{>0.2}}/N_{\text{DY events SR}}^{\text{no BDTm}\mathrm{_a} \text{ cut}}$, between the template and the simulation.  All one-sided DY template uncertainties are symmetrized around the nominal value.
 
 
To assess the uncertainties in the generation of the hard-scatter \ttbar process, the \textsc{Powheg} sample is compared with a sample generated using \textsc{MadGraph5}\_aMC@NLO~2.3.3. The hadronization and fragmentation uncertainties in the PS are evaluated by comparing the nominal sample showered by \textsc{Pythia~8.230} with an alternative sample generated by \textsc{Powheg} using the same PDF in matrix element as for the nominal sample, but showered with \textsc{Herwig}~7.0.4 \cite{Bahr:2008pv, Bellm:2015jjp}. The initial- and final-state radiation (ISR and FSR) uncertainties of the \ttbar sample are assessed by varying the internal \textsc{Pythia~8.230} showering parameters.
Finally, the uncertainties due to the PDF choice are evaluated using the internal variations of the nominal \textsc{PDF4LHC15\_nlo\_30} set \cite{Butterworth:2015oua}.
 
 
Uncertainties in the calculation of the ggF and VBF Higgs boson production cross-sections are assessed by following the recommendations of the LHC Higgs Working Group given in Refs.~\cite{deFlorian:2016spz, Heinemeyer:2013tqa}.
As no VH signal sample was generated, a conservative  100\%  uncertainty is assigned to the estimated VH yield.
To evaluate the  uncertainties due to the PDF choice, the yields obtained with the baseline \textsc{NNPDF30\_nlo\_as\_0118} set are compared with the yields obtained using the internal variations of \textsc{NNPDF30\_nlo\_as\_0118} and with the yields obtained with the nominal \textsc{MMHT2014nlo68clas118} \cite{Harland-Lang:2014zoa} and \textsc{CT14nlo}~\cite{Dulat:2015mca} sets. The largest difference is taken as the overall PDF uncertainty for all signal mass points. Furthermore, the effects of uncertainties in the ISR, FSR, multiparton interactions (MPI) in \textsc{Pythia}, parton showering, and renormalization and factorization scales are also  assessed. Uncertainties from these sources have an impact of 1--6\% on the signal yields, with the largest contributions arising from the ggF production cross-section and FSR uncertainties.
 
 
 
 
 
 
% End of text imported from the .//Systematics.tex input file
% The next lines are included from the .//Results.tex input file
\section{Analysis and results}
\label{sec:result}
The final background and signal estimates are obtained in a set of binned likelihood fits \cite{Cowan:2010js}  using the HistFitter~\cite{Baak:2014wma} package.
The likelihood is a product of Poisson probability functions, describing the observed and predicted numbers of events in each region, and Gaussian distributions that constrain the nuisance parameters associated with the systematic uncertainties.
In the background validation fit, the data in TCR and DYCR are used to extract the normalization of the \ttbar and DY backgrounds, respectively.
As the \ttbar sample in TCR is modeled very well, it is implemented as only one bin in the fit, whereas DYCR is divided into five equal-width bins in \Mmm to provide greater sensitivity to the DY template shape.
The purpose of this fit is to validate the modeling of the background in the control and validation regions and in SRincl. The fitted \ttbar normalization factor is $\mu_{\ttbar}=1.07^{+0.06}_{-0.07}$, while the value of $\mu_\text{DY}$ has no physical meaning because it is scaled from a template region and is thus not quoted. Figures~\ref{SRvalidation1} and \ref{SRvalidation2} show post-fit distributions of \KLMbbmm, \met, \LogL, and \Mmm spanning various analysis regions, while Figure~\ref{SRvalidation3} shows BDT20 and BDT50 in SRincl. Good agreement between the estimated backgrounds and the data is observed in the kinematic distributions.
In SRincl, 1185 events are observed, which is compatible with the total estimated background of $1155.3 \pm 13.6$.
The yields in several representative SR bins, i.e.\ \Mmm windows after applying the BDT selection, as obtained from the background validation fit above, are shown in Table~\ref{SRyields}. When comparing the systematic uncertainty with the statistical uncertainty, it can be seen that the analysis is clearly statistically limited.
Figure~\ref{SRpulls} shows the data and the estimated backgounds in all final SR bins. Due to the limited statistics of the background samples, the estimates are not perfectly smooth; however, the bin-to-bin fluctuations are much smaller than the statistical uncertainty of the data. Larger jumps, which occur at $m_a = 23$, 28, 33, 38~\GeV\ etc., appear when the BDT discriminant used for the selection changes from the one trained in the lower mass range to the one trained in the higher mass range.
 
 
\begin{table}[htb]
\caption{Total and individual background yields in six
representative $m_{\mu\mu}$ bins of the signal region after the BDT selection is applied. The yields
are the post-fit values as determined by the background validation fit.
The uncertainties shown include all systematic and statistical uncertainties. As the diboson, single top quark, $\ttbar V$, and $W$+jets contributions are very small, they are summed in the table under "Other".}
\begin{center}
\setlength{\tabcolsep}{0.5pc}{\small
\begin{tabular*}{\textwidth}{@{\extracolsep{\fill}}l|c|c|c|c|c|c}
\toprule
$m_{\mu\mu}$ bin [\GeV] &  [15--17]   & [24--26]   & [34--36]   & [44--46] &  [50.5--53.5]    & [60.5--63.5]               \\[-0.05cm]
\midrule
\midrule
Observed events          & $6$              & $9$              & $19$              & $17$              & $39$              & $8$                    \\
\noalign{\smallskip}\hline\noalign{\smallskip}
Total background         & $4.8 \pm 2.2$          & $9.0 \pm 1.8$          & $11.9 \pm 1.6$~~          & $15.5 \pm 2.0$~~          & $19.3 \pm 2.7$~~          & $9.3 \pm 1.7$              \\
\noalign{\smallskip}\hline\noalign{\smallskip}
DY         & $4.6 \pm 2.1$          & $6.4 \pm 1.5$          & $5.7 \pm 1.1$          & $6.4 \pm 1.5$          & $8.3 \pm 2.1$          & $5.3 \pm 1.4$              \\
\ttbar         & $0.2 \pm 0.1$          & $2.6 \pm 0.8$          & $6.0 \pm 1.1$          & $8.5 \pm 1.4$          & $10.4 \pm 2.4$~~          & $3.5 \pm 0.9$              \\
Other         & $0.03 \pm 0.01$          &   $0.03 \pm 0.00$          &    $0.24 \pm 0.12$          & $0.50 \pm 0.40$           & $0.50 \pm 0.12$          & $0.45 \pm 0.19$             \\
\bottomrule
\end{tabular*}
}
\end{center}
\label{SRyields}
\end{table}
 
 
\begin{figure}
\centering
\includegraphics[width=0.48\hsize]{fig_04a.pdf}
\includegraphics[width=0.48\hsize]{fig_04b.pdf}
\caption{Post-fit \KLMbbmm in DYCR, SRincl and VR1 (left); \met in SRincl and TCR (right). No selection based on the BDT discriminants is applied in the analysis regions shown in the figures. The signal distributions are normalized to the SM Higgs boson cross-section (including ggF, VBF, and VH production) and assume
$\mathcal{B}(H \rightarrow aa \rightarrow bb\mu\mu)$ as indicated in the legends of the figures (chosen to ensure good visibility in the plot). The hatched bands show the total post-fit statistical and systematic uncertainties of the backgrounds. The histogram labeled as ``Other'' in the legend includes the contributions from the diboson, single-top-quark, $\ttbar{+}V$ and $W$+jets backgrounds.}
\label{SRvalidation1}
\end{figure}
 
\begin{figure}
\centering
\includegraphics[width=0.48\hsize]{fig_05a.pdf}
\includegraphics[width=0.48\hsize]{fig_05b.pdf}
\caption{Post-fit \LogL~in VR2 and SRincl (left); \Mmm in SRincl (right). No selection based on the BDT discriminants is applied in the analysis regions shown in the figures. The signal distributions are normalized to the SM Higgs boson cross-section (including ggF, VBF, and VH production) and assume
$\mathcal{B}(H \rightarrow aa \rightarrow bb\mu\mu)$ as indicated in the legends of the figures (chosen to ensure good visibility in the plot). The hatched bands show the total post-fit statistical and systematic uncertainties of the backgrounds. The histogram labeled as ``Other'' in the legend includes the contributions from the diboson, single-top-quark, $\ttbar{+}V$ and $W$+jets backgrounds. }
\label{SRvalidation2}
\end{figure}
 
 
\begin{figure}
\centering
\includegraphics[width=0.48\hsize]{fig_06a.pdf}
\includegraphics[width=0.48\hsize]{fig_06b.pdf}
\caption{Post-fit BDT20 (left) and BDT50 (right) distributions in SRincl. The signal distributions are normalized to the SM Higgs boson cross-section (including ggF, VBF, and VH production) and assume
$\mathcal{B}(H \rightarrow aa \rightarrow bb\mu\mu)$ as indicated in the legends of the figures (chosen to ensure good visibility in the plot). The hatched bands show the total post-fit statistical and systematic uncertainties of the backgrounds. The histogram labeled as ``Other'' in the legend includes the contributions from the diboson, single-top-quark, $\ttbar{+}V$ and $W$+jets backgrounds. }
\label{SRvalidation3}
\end{figure}
 
 
 
\begin{figure}
\centering
\includegraphics[width=0.99\hsize]{fig_07.pdf}
\caption{Post-background-validation-fit number of  events in all SR bins (after applying the BDT selection) that are tested for the presence of signal. The bin widths are 2~\GeV\ (3~\GeV) in \Mmm for $m_a\leq45$~\GeV\ ($m_a>45$~\GeV). Neighboring bins partially overlap, hence they are not statistically independent.
The bottom panel shows the pull in each bin, defined as $(n_\mathrm{obs}-n_\mathrm{pred})/\sigma_\mathrm{tot}$, where $n_\mathrm{obs}$ is the number of events in the data, $n_\mathrm{pred}$ is the number of fitted background events and $\sigma_\mathrm{tot}$ is the total (systematic and statistical, added in quadrature) uncertainty in the fitted background yield.
Discontinuities in the background predictions appear when the BDT discriminant used for the selection changes from the one trained in the lower mass range to the one trained in the higher mass range. The histogram labeled as ``Other'' in the legend includes the contributions from the diboson, single-top-quark, $\ttbar{+}V$ and $W$+jets backgrounds.}
\label{SRpulls}
\end{figure}
 
 
To test for the presence of new phenomena, fits are performed for each of the 47 hypothesized signal masses in the range $16\leq\Mmm\leq62$~\GeV\ in 1~\GeV\ steps.
It was verified that the analysis is also sufficiently sensitive to a signal with \Mmm centered in between these 1~\GeV\ steps.
TCR, DYCR, and the respective SR bin are included in each fit in order to constrain  the backgrounds and the signal to the data.
 
 
A model-independent fit, i.e.\ not including any signal sample, is performed to test whether the data are compatible with the background-only hypothesis. The result  is a scan of $p_0$-values as shown in Figure~\ref{pvalue}. The largest discrepancy is found at $\Mmm=52$~\GeV, corresponding to a local (global) $p_0$-value of 0.00054 (0.048) and a local (global) significance of $3.3 \sigma$ ($1.7 \sigma$). The global significance was calculated from the asymptotic formulae in Refs.~\cite{Cowan:2010js,Gross_2010}.
 
 
Upper limits, derived using  the CL$_\text{s}$ technique \cite{Read:2002hq,Junk:1999kv}, are set on $\mathcal{B}(H \rightarrow aa \rightarrow bb\mu\mu)$ in a series of conditional fits, this time also including the signal samples.
The limits as a function of $m_a$ are shown in Figure~\ref{Limits}.
Uniform sensitivity is achieved for all masses above 18~\GeV, while for lower signal masses, $m_a \leq 18$~\GeV, the sensitivity of the analysis decreases due to $b$-jets falling below the reconstruction threshold or merging into one reconstructed jet.
Figure~\ref{SRwithSign1} shows \Mmm and BDT$m_a$ distributions  after the signal+background fit for two SR bins, $m_a=35$~\GeV\ and $m_a=52$~\GeV, where the two largest deviations from the background-only hypothesis are observed.
The signal in the plots is scaled to the best-fit value, corresponding to $\mathcal{B}(H \rightarrow aa \rightarrow bb\mu\mu)=6.4\times 10^{-5}$ ($1.9\times 10^{-4}$) for $m_a=35$~\GeV\ ($m_a=52$~\GeV).
 
The upper limits at 95\% CL on $\mathcal{B}(H\rightarrow aa\rightarrow bb\mu\mu)$ range between $0.2 \times 10^{-4}$ and $4.0 \times 10^{-4}$, depending on $m_a$.
These limits improve upon the previous ATLAS result based on 36~$\ifb$ of data~\cite{bbmm36ifb} by a factor of 2--5 over the full  \Mmm range. A factor of $\sim$2 improvement in sensitivity comes from the larger dataset, and a further factor of $\sim$2 is achieved thanks to the use of multivariate techniques to discriminate between the signal and the SM backgrounds. Due to small number of background events at lower signal masses $m_a$, the BDT training is less efficient in this region, and the gain from applying the BDT$m_a$ selection criteria is higher at higher $m_a$.
Taking as an example the favorable scenario with $\mathcal{B}(H\rightarrow aa\rightarrow bb\mu\mu)/ \mathcal{B}(H\rightarrow aa )= 0.16\%$, the analysis probes the Higgs boson branching fraction into pseudoscalars down to $\mathcal{B}(H\rightarrow aa) = 1.3\%$, much lower than the limits derived from combinations of the Higgs boson measurements.
 
So as not to restrict the analysis sensitivity solely to models where the $a$-particle is a pseudoscalar, upper limits obtained without employing the BDT discriminants are also derived as shown in Figure~\ref{Limits_noBDT}.
In addition to being less sensitive to the particle's CP properties, the limits in SRincl without the BDT selection also facilitate reinterpretations of the analysis.
These limits are derived in the same way as described above, i.e.\ by scanning the \Mmm windows of SRincl, but omitting the final selection on the BDT discriminants. The expected limits obtained when employing the baseline analysis strategy are also shown in Figure~\ref{Limits_noBDT} for comparison, illustrating the significant improvement in sensitivity to pseudoscalars when using the BDTs. The excess observed at $\Mmm=52$~\GeV\ in the BDT analysis is not supported by the limits derived without the BDTs.
Figure~\ref{SRpullsnoBDT} shows the data and the estimated backgrounds in all final SR bins, without applying the BDT selection.
 
 
 
 
 
\begin{figure}
\centering
\includegraphics[width=0.6\hsize]{fig_08.pdf}
\caption{The local $p_0$-values are quantified in standard deviations $\sigma$ and plotted as a function of the signal mass hypothesis. Between the points, the $p_0$-values are interpolated and may not be fully representative of the actual sensitivity.}
\label{pvalue}
\end{figure}
 
 
 
\begin{figure}
\centering
\includegraphics[width=0.6\hsize]{fig_09.pdf}
\caption{Upper limits on $\mathcal{B}(H \rightarrow aa \rightarrow bb\mu\mu)$ at 95\% CL, including the BDT selection, as a function of the signal mass hypothesis. Black and red dots show masses for which the hypothesis testing was done. Between these points, the limits are interpolated and may not be fully representative of the actual sensitivity.}
\label{Limits}
\end{figure}
 
 
 
 
 
\begin{figure}
\centering
\includegraphics[width=0.48\hsize]{fig_10a.pdf}
\includegraphics[width=0.48\hsize]{fig_10b.pdf}
\includegraphics[width=0.48\hsize]{fig_10c.pdf}
\includegraphics[width=0.48\hsize]{fig_10d.pdf}
\caption{\Mmm distributions in the SRincl after the BDT$35>0.2$ selection (top left) and BDT$50>0.2$ selection (bottom left), and BDT35 (top right) and BDT50 (bottom right) distributions in the SRincl in the \Mmm window 34--36~\GeV\ and 50.5--53.5~\GeV, respectively.
The signal is scaled to the best-fit value, $\mathcal{B}(H \rightarrow aa \rightarrow bb\mu\mu)=6.4\times 10^{-5}$ for the top plots, and $1.9\times 10^{-4}$ for the bottom plots, assuming the SM Higgs boson cross-section (including ggF, VBF, and VH production). The hatched bands show the total post-fit statistical and systematic uncertainties of the backgrounds and the signal. The histogram labeled as ``Other'' in the legend includes the contributions from the diboson, single-top-quark, $\ttbar{+}V$ and $W$+jets backgrounds.}
\label{SRwithSign1}
\end{figure}
 
 
\begin{figure}
\centering
\includegraphics[width=0.6\hsize]{fig_11.pdf}
\caption{Upper limits on $\mathcal{B}(H \rightarrow aa \rightarrow bb\mu\mu)$  at 95\% CL, with no BDT selection applied, as a function of the signal mass hypothesis. The dash-dotted blue line indicates the expected limit set in the analysis with the BDT selection. Black and red dots show masses for which the hypothesis testing was done. Between these points, the limits are interpolated and may not be fully representative of the actual sensitivity. }
\label{Limits_noBDT}
\end{figure}
 
 
\begin{figure}
\centering
\includegraphics[width=0.99\hsize]{fig_12.pdf}
\caption{Post-background-validation-fit number of  events in all SR bins (without applying the BDT selection) that are tested for the presence of signal. The bin widths are 2~\GeV\ (3~\GeV) in \Mmm for $m_a\leq45$~\GeV\ ($m_a>45$~\GeV). Neighboring bins partially overlap, hence they are not statistically independent.
The bottom panel shows the pull in each bin, defined as $(n_\mathrm{obs}-n_\mathrm{pred})/\sigma_\mathrm{tot}$, where $n_\mathrm{obs}$ is the number of events in the data, $n_\mathrm{pred}$ is the number of fitted background events and $\sigma_\mathrm{tot}$ is the total (systematic and statistical, added in quadrature) uncertainty in the fitted background yield. The discontinuity at $m_a=45$~\GeV\ appears where the \Mmm window size is changed. The histogram labeled as ``Other'' in the legend includes the contributions from the diboson, single-top-quark, $\ttbar{+}V$ and $W$+jets backgrounds.}
\label{SRpullsnoBDT}
\end{figure}
 
 
 
\FloatBarrier
 
% End of text imported from the .//Results.tex input file
 
% The next lines are included from the .//Conclusion.tex input file
\section{Conclusion}
 
A search for light pseudoscalar particles (denoted by $a$) in the decays of the 125~\GeV\ Higgs boson in the final state with two muons and two $b$-tagged jets, $H \rightarrow aa \rightarrow bb\mu\mu$, is presented.
The analysis is performed using 139~fb$^{-1}$ of $\sqrt{s}=13$~\TeV\ $pp$ collision data recorded by the ATLAS detector at the LHC between 2015 and 2018. A narrow resonance is searched for in the dimuon invariant mass spectrum in the range $16 \leq \Mmm \leq 62$~\GeV.
BDT classifiers are trained to distinguish the $H \rightarrow aa$ signal, where $a$ is a pseudoscalar, from the SM backgrounds.
Additionally, the result without selection on the BDT discriminants is also provided to ensure sensitivity to models where the $a$-particle is not necessarily a pseudoscalar, as well as  to facilitate reinterpretations of the analysis.
No significant excess of the data above the SM backgrounds is observed.
In the BDT analysis, the lowest local $p_0$-value of 0.00054 is observed at $\Mmm= 52$~\GeV\ and corresponds to a local significance of $3.3 \sigma$. The global significance of that excess is determined to be $1.7 \sigma$.
Upper limits at 95\%~CL including (excluding) the BDT selection criteria are set on $\mathcal{B}(H \rightarrow aa \rightarrow bb\mu\mu)$ and range between $0.2\times10^{-4}$ and $4.0\times10^{-4}$ ($0.5\times10^{-4}$ and $5.0\times10^{-4}$), depending on $m_a$.
The result including the BDT selection criteria improves upon previous ATLAS and CMS limits by about a factor of 2--5 for $m_a>20$~\GeV, while both results (with and without the BDT) extend the search down to $m_a$ values of 16~\GeV.
 
 
 
 
% End of text imported from the .//Conclusion.tex input file
 
\section*{Acknowledgments}
 
% The next lines are included from the .//acknowledgements/Acknowledgements.tex input file
 
 
We thank CERN for the very successful operation of the LHC, as well as the
support staff from our institutions without whom ATLAS could not be
operated efficiently.
 
We acknowledge the support of
ANPCyT, Argentina;
YerPhI, Armenia;
ARC, Australia;
BMWFW and FWF, Austria;
ANAS, Azerbaijan;
SSTC, Belarus;
CNPq and FAPESP, Brazil;
NSERC, NRC and CFI, Canada;
CERN;
ANID, Chile;
CAS, MOST and NSFC, China;
Minciencias, Colombia;
MSMT CR, MPO CR and VSC CR, Czech Republic;
DNRF and DNSRC, Denmark;
IN2P3-CNRS and CEA-DRF/IRFU, France;
SRNSFG, Georgia;
BMBF, HGF and MPG, Germany;
GSRI, Greece;
RGC and Hong Kong SAR, China;
ISF and Benoziyo Center, Israel;
INFN, Italy;
MEXT and JSPS, Japan;
CNRST, Morocco;
NWO, Netherlands;
RCN, Norway;
MEiN, Poland;
FCT, Portugal;
MNE/IFA, Romania;
JINR;
MES of Russia and NRC KI, Russian Federation;
MESTD, Serbia;
MSSR, Slovakia;
ARRS and MIZ\v{S}, Slovenia;
DSI/NRF, South Africa;
MICINN, Spain;
SRC and Wallenberg Foundation, Sweden;
SERI, SNSF and Cantons of Bern and Geneva, Switzerland;
MOST, Taiwan;
TAEK, Turkey;
STFC, United Kingdom;
DOE and NSF, United States of America.
In addition, individual groups and members have received support from
BCKDF, CANARIE, Compute Canada and CRC, Canada;
COST, ERC, ERDF, Horizon 2020 and Marie Sk{\l}odowska-Curie Actions, European Union;
Investissements d'Avenir Labex, Investissements d'Avenir Idex and ANR, France;
DFG and AvH Foundation, Germany;
Herakleitos, Thales and Aristeia programmes co-financed by EU-ESF and the Greek NSRF, Greece;
BSF-NSF and GIF, Israel;
Norwegian Financial Mechanism 2014-2021, Norway;
NCN and NAWA, Poland;
La Caixa Banking Foundation, CERCA Programme Generalitat de Catalunya and PROMETEO and GenT Programmes Generalitat Valenciana, Spain;
G\"{o}ran Gustafssons Stiftelse, Sweden;
The Royal Society and Leverhulme Trust, United Kingdom.
 
The crucial computing support from all WLCG partners is acknowledged gratefully, in particular from CERN, the ATLAS Tier-1 facilities at TRIUMF (Canada), NDGF (Denmark, Norway, Sweden), CC-IN2P3 (France), KIT/GridKA (Germany), INFN-CNAF (Italy), NL-T1 (Netherlands), PIC (Spain), ASGC (Taiwan), RAL (UK) and BNL (USA), the Tier-2 facilities worldwide and large non-WLCG resource providers. Major contributors of computing resources are listed in Ref.~\cite{ATL-SOFT-PUB-2021-003}.
 
 
% End of text imported from the .//acknowledgements/Acknowledgements.tex input file
 
 
 
\clearpage
 
 
\printbibliography
\clearpage
 
 
 
 
 
\clearpage % ATLAS Collaboration author list
% Reference date of HDBS-2021-03 is 2021-06-21
% Author list last updated on date 12-JAN-22
% Data extracted on 12-Jan-2022 for paper reference HDBS-2021-03
% at 4:09pm
 
\begin{flushleft}
\hypersetup{urlcolor=black}
{\Large The ATLAS Collaboration}

\bigskip

\AtlasOrcid[0000-0002-6665-4934]{G.~Aad}$^\textrm{\scriptsize 98}$,    
\AtlasOrcid[0000-0002-5888-2734]{B.~Abbott}$^\textrm{\scriptsize 123}$,    
\AtlasOrcid[0000-0002-7248-3203]{D.C.~Abbott}$^\textrm{\scriptsize 99}$,    
\AtlasOrcid[0000-0002-2788-3822]{A.~Abed~Abud}$^\textrm{\scriptsize 34}$,    
\AtlasOrcid[0000-0002-1002-1652]{K.~Abeling}$^\textrm{\scriptsize 51}$,    
\AtlasOrcid[0000-0002-2987-4006]{D.K.~Abhayasinghe}$^\textrm{\scriptsize 91}$,    
\AtlasOrcid[0000-0002-8496-9294]{S.H.~Abidi}$^\textrm{\scriptsize 27}$,    
\AtlasOrcid[0000-0002-9987-2292]{A.~Aboulhorma}$^\textrm{\scriptsize 33e}$,    
\AtlasOrcid[0000-0001-5329-6640]{H.~Abramowicz}$^\textrm{\scriptsize 156}$,    
\AtlasOrcid[0000-0002-1599-2896]{H.~Abreu}$^\textrm{\scriptsize 155}$,    
\AtlasOrcid[0000-0003-0403-3697]{Y.~Abulaiti}$^\textrm{\scriptsize 5}$,    
\AtlasOrcid[0000-0003-0762-7204]{A.C.~Abusleme~Hoffman}$^\textrm{\scriptsize 141a}$,    
\AtlasOrcid[0000-0002-8588-9157]{B.S.~Acharya}$^\textrm{\scriptsize 64a,64b,o}$,    
\AtlasOrcid[0000-0002-0288-2567]{B.~Achkar}$^\textrm{\scriptsize 51}$,    
\AtlasOrcid[0000-0001-6005-2812]{L.~Adam}$^\textrm{\scriptsize 96}$,    
\AtlasOrcid[0000-0002-2634-4958]{C.~Adam~Bourdarios}$^\textrm{\scriptsize 4}$,    
\AtlasOrcid[0000-0002-5859-2075]{L.~Adamczyk}$^\textrm{\scriptsize 81a}$,    
\AtlasOrcid[0000-0003-1562-3502]{L.~Adamek}$^\textrm{\scriptsize 161}$,    
\AtlasOrcid[0000-0002-2919-6663]{S.V.~Addepalli}$^\textrm{\scriptsize 24}$,    
\AtlasOrcid[0000-0002-1041-3496]{J.~Adelman}$^\textrm{\scriptsize 116}$,    
\AtlasOrcid[0000-0001-6644-0517]{A.~Adiguzel}$^\textrm{\scriptsize 11c,ac}$,    
\AtlasOrcid[0000-0003-3620-1149]{S.~Adorni}$^\textrm{\scriptsize 52}$,    
\AtlasOrcid[0000-0003-0627-5059]{T.~Adye}$^\textrm{\scriptsize 138}$,    
\AtlasOrcid[0000-0002-9058-7217]{A.A.~Affolder}$^\textrm{\scriptsize 140}$,    
\AtlasOrcid[0000-0001-8102-356X]{Y.~Afik}$^\textrm{\scriptsize 34}$,    
\AtlasOrcid[0000-0002-2368-0147]{C.~Agapopoulou}$^\textrm{\scriptsize 62}$,    
\AtlasOrcid[0000-0002-4355-5589]{M.N.~Agaras}$^\textrm{\scriptsize 12}$,    
\AtlasOrcid[0000-0002-4754-7455]{J.~Agarwala}$^\textrm{\scriptsize 68a,68b}$,    
\AtlasOrcid[0000-0002-1922-2039]{A.~Aggarwal}$^\textrm{\scriptsize 114}$,    
\AtlasOrcid[0000-0003-3695-1847]{C.~Agheorghiesei}$^\textrm{\scriptsize 25c}$,    
\AtlasOrcid[0000-0002-5475-8920]{J.A.~Aguilar-Saavedra}$^\textrm{\scriptsize 134f,134a,ab}$,    
\AtlasOrcid[0000-0001-8638-0582]{A.~Ahmad}$^\textrm{\scriptsize 34}$,    
\AtlasOrcid[0000-0003-3644-540X]{F.~Ahmadov}$^\textrm{\scriptsize 77}$,    
\AtlasOrcid[0000-0003-0128-3279]{W.S.~Ahmed}$^\textrm{\scriptsize 100}$,    
\AtlasOrcid[0000-0003-3856-2415]{X.~Ai}$^\textrm{\scriptsize 44}$,    
\AtlasOrcid[0000-0002-0573-8114]{G.~Aielli}$^\textrm{\scriptsize 71a,71b}$,    
\AtlasOrcid[0000-0003-2150-1624]{I.~Aizenberg}$^\textrm{\scriptsize 174}$,    
\AtlasOrcid[0000-0002-1681-6405]{S.~Akatsuka}$^\textrm{\scriptsize 83}$,    
\AtlasOrcid[0000-0002-7342-3130]{M.~Akbiyik}$^\textrm{\scriptsize 96}$,    
\AtlasOrcid[0000-0003-4141-5408]{T.P.A.~{\AA}kesson}$^\textrm{\scriptsize 94}$,    
\AtlasOrcid[0000-0002-2846-2958]{A.V.~Akimov}$^\textrm{\scriptsize 107}$,    
\AtlasOrcid[0000-0002-0547-8199]{K.~Al~Khoury}$^\textrm{\scriptsize 37}$,    
\AtlasOrcid[0000-0003-2388-987X]{G.L.~Alberghi}$^\textrm{\scriptsize 21b}$,    
\AtlasOrcid[0000-0003-0253-2505]{J.~Albert}$^\textrm{\scriptsize 170}$,    
\AtlasOrcid[0000-0001-6430-1038]{P.~Albicocco}$^\textrm{\scriptsize 49}$,    
\AtlasOrcid[0000-0003-2212-7830]{M.J.~Alconada~Verzini}$^\textrm{\scriptsize 86}$,    
\AtlasOrcid[0000-0002-8224-7036]{S.~Alderweireldt}$^\textrm{\scriptsize 48}$,    
\AtlasOrcid[0000-0002-1936-9217]{M.~Aleksa}$^\textrm{\scriptsize 34}$,    
\AtlasOrcid[0000-0001-7381-6762]{I.N.~Aleksandrov}$^\textrm{\scriptsize 77}$,    
\AtlasOrcid[0000-0003-0922-7669]{C.~Alexa}$^\textrm{\scriptsize 25b}$,    
\AtlasOrcid[0000-0002-8977-279X]{T.~Alexopoulos}$^\textrm{\scriptsize 9}$,    
\AtlasOrcid[0000-0001-7406-4531]{A.~Alfonsi}$^\textrm{\scriptsize 115}$,    
\AtlasOrcid[0000-0002-0966-0211]{F.~Alfonsi}$^\textrm{\scriptsize 21b}$,    
\AtlasOrcid[0000-0001-7569-7111]{M.~Alhroob}$^\textrm{\scriptsize 123}$,    
\AtlasOrcid[0000-0001-8653-5556]{B.~Ali}$^\textrm{\scriptsize 136}$,    
\AtlasOrcid[0000-0001-5216-3133]{S.~Ali}$^\textrm{\scriptsize 153}$,    
\AtlasOrcid[0000-0002-9012-3746]{M.~Aliev}$^\textrm{\scriptsize 160}$,    
\AtlasOrcid[0000-0002-7128-9046]{G.~Alimonti}$^\textrm{\scriptsize 66a}$,    
\AtlasOrcid[0000-0003-4745-538X]{C.~Allaire}$^\textrm{\scriptsize 34}$,    
\AtlasOrcid[0000-0002-5738-2471]{B.M.M.~Allbrooke}$^\textrm{\scriptsize 151}$,    
\AtlasOrcid[0000-0001-7303-2570]{P.P.~Allport}$^\textrm{\scriptsize 19}$,    
\AtlasOrcid[0000-0002-3883-6693]{A.~Aloisio}$^\textrm{\scriptsize 67a,67b}$,    
\AtlasOrcid[0000-0001-9431-8156]{F.~Alonso}$^\textrm{\scriptsize 86}$,    
\AtlasOrcid[0000-0002-7641-5814]{C.~Alpigiani}$^\textrm{\scriptsize 143}$,    
\AtlasOrcid{E.~Alunno~Camelia}$^\textrm{\scriptsize 71a,71b}$,    
\AtlasOrcid[0000-0002-8181-6532]{M.~Alvarez~Estevez}$^\textrm{\scriptsize 95}$,    
\AtlasOrcid[0000-0003-0026-982X]{M.G.~Alviggi}$^\textrm{\scriptsize 67a,67b}$,    
\AtlasOrcid[0000-0002-1798-7230]{Y.~Amaral~Coutinho}$^\textrm{\scriptsize 78b}$,    
\AtlasOrcid[0000-0003-2184-3480]{A.~Ambler}$^\textrm{\scriptsize 100}$,    
\AtlasOrcid[0000-0002-0987-6637]{L.~Ambroz}$^\textrm{\scriptsize 129}$,    
\AtlasOrcid{C.~Amelung}$^\textrm{\scriptsize 34}$,    
\AtlasOrcid[0000-0002-6814-0355]{D.~Amidei}$^\textrm{\scriptsize 102}$,    
\AtlasOrcid[0000-0001-7566-6067]{S.P.~Amor~Dos~Santos}$^\textrm{\scriptsize 134a}$,    
\AtlasOrcid[0000-0001-5450-0447]{S.~Amoroso}$^\textrm{\scriptsize 44}$,    
\AtlasOrcid{C.S.~Amrouche}$^\textrm{\scriptsize 52}$,    
\AtlasOrcid[0000-0003-3649-7621]{V.~Ananiev}$^\textrm{\scriptsize 128}$,    
\AtlasOrcid[0000-0003-1587-5830]{C.~Anastopoulos}$^\textrm{\scriptsize 144}$,    
\AtlasOrcid[0000-0002-4935-4753]{N.~Andari}$^\textrm{\scriptsize 139}$,    
\AtlasOrcid[0000-0002-4413-871X]{T.~Andeen}$^\textrm{\scriptsize 10}$,    
\AtlasOrcid[0000-0002-1846-0262]{J.K.~Anders}$^\textrm{\scriptsize 18}$,    
\AtlasOrcid[0000-0002-9766-2670]{S.Y.~Andrean}$^\textrm{\scriptsize 43a,43b}$,    
\AtlasOrcid[0000-0001-5161-5759]{A.~Andreazza}$^\textrm{\scriptsize 66a,66b}$,    
\AtlasOrcid[0000-0002-8274-6118]{S.~Angelidakis}$^\textrm{\scriptsize 8}$,    
\AtlasOrcid[0000-0001-7834-8750]{A.~Angerami}$^\textrm{\scriptsize 37}$,    
\AtlasOrcid[0000-0002-7201-5936]{A.V.~Anisenkov}$^\textrm{\scriptsize 117b,117a}$,    
\AtlasOrcid[0000-0002-4649-4398]{A.~Annovi}$^\textrm{\scriptsize 69a}$,    
\AtlasOrcid[0000-0001-9683-0890]{C.~Antel}$^\textrm{\scriptsize 52}$,    
\AtlasOrcid[0000-0002-5270-0143]{M.T.~Anthony}$^\textrm{\scriptsize 144}$,    
\AtlasOrcid[0000-0002-6678-7665]{E.~Antipov}$^\textrm{\scriptsize 124}$,    
\AtlasOrcid[0000-0002-2293-5726]{M.~Antonelli}$^\textrm{\scriptsize 49}$,    
\AtlasOrcid[0000-0001-8084-7786]{D.J.A.~Antrim}$^\textrm{\scriptsize 16}$,    
\AtlasOrcid[0000-0003-2734-130X]{F.~Anulli}$^\textrm{\scriptsize 70a}$,    
\AtlasOrcid[0000-0001-7498-0097]{M.~Aoki}$^\textrm{\scriptsize 79}$,    
\AtlasOrcid[0000-0001-7401-4331]{J.A.~Aparisi~Pozo}$^\textrm{\scriptsize 168}$,    
\AtlasOrcid[0000-0003-4675-7810]{M.A.~Aparo}$^\textrm{\scriptsize 151}$,    
\AtlasOrcid[0000-0003-3942-1702]{L.~Aperio~Bella}$^\textrm{\scriptsize 44}$,    
\AtlasOrcid[0000-0001-9013-2274]{N.~Aranzabal}$^\textrm{\scriptsize 34}$,    
\AtlasOrcid[0000-0003-1177-7563]{V.~Araujo~Ferraz}$^\textrm{\scriptsize 78a}$,    
\AtlasOrcid[0000-0001-8648-2896]{C.~Arcangeletti}$^\textrm{\scriptsize 49}$,    
\AtlasOrcid[0000-0002-7255-0832]{A.T.H.~Arce}$^\textrm{\scriptsize 47}$,    
\AtlasOrcid[0000-0001-5970-8677]{E.~Arena}$^\textrm{\scriptsize 88}$,    
\AtlasOrcid[0000-0003-0229-3858]{J-F.~Arguin}$^\textrm{\scriptsize 106}$,    
\AtlasOrcid[0000-0001-7748-1429]{S.~Argyropoulos}$^\textrm{\scriptsize 50}$,    
\AtlasOrcid[0000-0002-1577-5090]{J.-H.~Arling}$^\textrm{\scriptsize 44}$,    
\AtlasOrcid[0000-0002-9007-530X]{A.J.~Armbruster}$^\textrm{\scriptsize 34}$,    
\AtlasOrcid[0000-0001-8505-4232]{A.~Armstrong}$^\textrm{\scriptsize 165}$,    
\AtlasOrcid[0000-0002-6096-0893]{O.~Arnaez}$^\textrm{\scriptsize 161}$,    
\AtlasOrcid[0000-0003-3578-2228]{H.~Arnold}$^\textrm{\scriptsize 34}$,    
\AtlasOrcid{Z.P.~Arrubarrena~Tame}$^\textrm{\scriptsize 110}$,    
\AtlasOrcid[0000-0002-3477-4499]{G.~Artoni}$^\textrm{\scriptsize 129}$,    
\AtlasOrcid[0000-0003-1420-4955]{H.~Asada}$^\textrm{\scriptsize 112}$,    
\AtlasOrcid[0000-0002-3670-6908]{K.~Asai}$^\textrm{\scriptsize 121}$,    
\AtlasOrcid[0000-0001-5279-2298]{S.~Asai}$^\textrm{\scriptsize 158}$,    
\AtlasOrcid[0000-0001-8381-2255]{N.A.~Asbah}$^\textrm{\scriptsize 57}$,    
\AtlasOrcid[0000-0003-2127-373X]{E.M.~Asimakopoulou}$^\textrm{\scriptsize 166}$,    
\AtlasOrcid[0000-0001-8035-7162]{L.~Asquith}$^\textrm{\scriptsize 151}$,    
\AtlasOrcid[0000-0002-3207-9783]{J.~Assahsah}$^\textrm{\scriptsize 33d}$,    
\AtlasOrcid{K.~Assamagan}$^\textrm{\scriptsize 27}$,    
\AtlasOrcid[0000-0001-5095-605X]{R.~Astalos}$^\textrm{\scriptsize 26a}$,    
\AtlasOrcid[0000-0002-1972-1006]{R.J.~Atkin}$^\textrm{\scriptsize 31a}$,    
\AtlasOrcid{M.~Atkinson}$^\textrm{\scriptsize 167}$,    
\AtlasOrcid[0000-0003-1094-4825]{N.B.~Atlay}$^\textrm{\scriptsize 17}$,    
\AtlasOrcid{H.~Atmani}$^\textrm{\scriptsize 58b}$,    
\AtlasOrcid[0000-0002-7639-9703]{P.A.~Atmasiddha}$^\textrm{\scriptsize 102}$,    
\AtlasOrcid[0000-0001-8324-0576]{K.~Augsten}$^\textrm{\scriptsize 136}$,    
\AtlasOrcid[0000-0001-7599-7712]{S.~Auricchio}$^\textrm{\scriptsize 67a,67b}$,    
\AtlasOrcid[0000-0001-6918-9065]{V.A.~Austrup}$^\textrm{\scriptsize 176}$,    
\AtlasOrcid[0000-0003-1616-3587]{G.~Avner}$^\textrm{\scriptsize 155}$,    
\AtlasOrcid[0000-0003-2664-3437]{G.~Avolio}$^\textrm{\scriptsize 34}$,    
\AtlasOrcid[0000-0001-5265-2674]{M.K.~Ayoub}$^\textrm{\scriptsize 13c}$,    
\AtlasOrcid[0000-0003-4241-022X]{G.~Azuelos}$^\textrm{\scriptsize 106,aj}$,    
\AtlasOrcid[0000-0001-7657-6004]{D.~Babal}$^\textrm{\scriptsize 26a}$,    
\AtlasOrcid[0000-0002-2256-4515]{H.~Bachacou}$^\textrm{\scriptsize 139}$,    
\AtlasOrcid[0000-0002-9047-6517]{K.~Bachas}$^\textrm{\scriptsize 157}$,    
\AtlasOrcid[0000-0001-8599-024X]{A.~Bachiu}$^\textrm{\scriptsize 32}$,    
\AtlasOrcid[0000-0001-7489-9184]{F.~Backman}$^\textrm{\scriptsize 43a,43b}$,    
\AtlasOrcid[0000-0001-5199-9588]{A.~Badea}$^\textrm{\scriptsize 57}$,    
\AtlasOrcid[0000-0003-4578-2651]{P.~Bagnaia}$^\textrm{\scriptsize 70a,70b}$,    
\AtlasOrcid{H.~Bahrasemani}$^\textrm{\scriptsize 147}$,    
\AtlasOrcid[0000-0002-3301-2986]{A.J.~Bailey}$^\textrm{\scriptsize 168}$,    
\AtlasOrcid[0000-0001-8291-5711]{V.R.~Bailey}$^\textrm{\scriptsize 167}$,    
\AtlasOrcid[0000-0003-0770-2702]{J.T.~Baines}$^\textrm{\scriptsize 138}$,    
\AtlasOrcid[0000-0002-9931-7379]{C.~Bakalis}$^\textrm{\scriptsize 9}$,    
\AtlasOrcid[0000-0003-1346-5774]{O.K.~Baker}$^\textrm{\scriptsize 177}$,    
\AtlasOrcid[0000-0002-3479-1125]{P.J.~Bakker}$^\textrm{\scriptsize 115}$,    
\AtlasOrcid[0000-0002-1110-4433]{E.~Bakos}$^\textrm{\scriptsize 14}$,    
\AtlasOrcid[0000-0002-6580-008X]{D.~Bakshi~Gupta}$^\textrm{\scriptsize 7}$,    
\AtlasOrcid[0000-0002-5364-2109]{S.~Balaji}$^\textrm{\scriptsize 152}$,    
\AtlasOrcid[0000-0001-5840-1788]{R.~Balasubramanian}$^\textrm{\scriptsize 115}$,    
\AtlasOrcid[0000-0002-9854-975X]{E.M.~Baldin}$^\textrm{\scriptsize 117b,117a}$,    
\AtlasOrcid[0000-0002-0942-1966]{P.~Balek}$^\textrm{\scriptsize 137}$,    
\AtlasOrcid[0000-0001-9700-2587]{E.~Ballabene}$^\textrm{\scriptsize 66a,66b}$,    
\AtlasOrcid[0000-0003-0844-4207]{F.~Balli}$^\textrm{\scriptsize 139}$,    
\AtlasOrcid[0000-0002-7048-4915]{W.K.~Balunas}$^\textrm{\scriptsize 129}$,    
\AtlasOrcid[0000-0003-2866-9446]{J.~Balz}$^\textrm{\scriptsize 96}$,    
\AtlasOrcid[0000-0001-5325-6040]{E.~Banas}$^\textrm{\scriptsize 82}$,    
\AtlasOrcid[0000-0003-2014-9489]{M.~Bandieramonte}$^\textrm{\scriptsize 133}$,    
\AtlasOrcid[0000-0002-5256-839X]{A.~Bandyopadhyay}$^\textrm{\scriptsize 17}$,    
\AtlasOrcid[0000-0002-8754-1074]{S.~Bansal}$^\textrm{\scriptsize 22}$,    
\AtlasOrcid[0000-0002-3436-2726]{L.~Barak}$^\textrm{\scriptsize 156}$,    
\AtlasOrcid[0000-0002-3111-0910]{E.L.~Barberio}$^\textrm{\scriptsize 101}$,    
\AtlasOrcid[0000-0002-3938-4553]{D.~Barberis}$^\textrm{\scriptsize 53b,53a}$,    
\AtlasOrcid[0000-0002-7824-3358]{M.~Barbero}$^\textrm{\scriptsize 98}$,    
\AtlasOrcid{G.~Barbour}$^\textrm{\scriptsize 92}$,    
\AtlasOrcid[0000-0002-9165-9331]{K.N.~Barends}$^\textrm{\scriptsize 31a}$,    
\AtlasOrcid[0000-0001-7326-0565]{T.~Barillari}$^\textrm{\scriptsize 111}$,    
\AtlasOrcid[0000-0003-0253-106X]{M-S.~Barisits}$^\textrm{\scriptsize 34}$,    
\AtlasOrcid[0000-0002-5132-4887]{J.~Barkeloo}$^\textrm{\scriptsize 126}$,    
\AtlasOrcid[0000-0002-7709-037X]{T.~Barklow}$^\textrm{\scriptsize 148}$,    
\AtlasOrcid[0000-0002-5361-2823]{B.M.~Barnett}$^\textrm{\scriptsize 138}$,    
\AtlasOrcid[0000-0002-7210-9887]{R.M.~Barnett}$^\textrm{\scriptsize 16}$,    
\AtlasOrcid[0000-0001-7090-7474]{A.~Baroncelli}$^\textrm{\scriptsize 58a}$,    
\AtlasOrcid[0000-0001-5163-5936]{G.~Barone}$^\textrm{\scriptsize 27}$,    
\AtlasOrcid[0000-0002-3533-3740]{A.J.~Barr}$^\textrm{\scriptsize 129}$,    
\AtlasOrcid[0000-0002-3380-8167]{L.~Barranco~Navarro}$^\textrm{\scriptsize 43a,43b}$,    
\AtlasOrcid[0000-0002-3021-0258]{F.~Barreiro}$^\textrm{\scriptsize 95}$,    
\AtlasOrcid[0000-0003-2387-0386]{J.~Barreiro~Guimar\~{a}es~da~Costa}$^\textrm{\scriptsize 13a}$,    
\AtlasOrcid[0000-0002-3455-7208]{U.~Barron}$^\textrm{\scriptsize 156}$,    
\AtlasOrcid[0000-0003-2872-7116]{S.~Barsov}$^\textrm{\scriptsize 132}$,    
\AtlasOrcid[0000-0002-3407-0918]{F.~Bartels}$^\textrm{\scriptsize 59a}$,    
\AtlasOrcid[0000-0001-5317-9794]{R.~Bartoldus}$^\textrm{\scriptsize 148}$,    
\AtlasOrcid[0000-0002-9313-7019]{G.~Bartolini}$^\textrm{\scriptsize 98}$,    
\AtlasOrcid[0000-0001-9696-9497]{A.E.~Barton}$^\textrm{\scriptsize 87}$,    
\AtlasOrcid[0000-0003-1419-3213]{P.~Bartos}$^\textrm{\scriptsize 26a}$,    
\AtlasOrcid[0000-0001-5623-2853]{A.~Basalaev}$^\textrm{\scriptsize 44}$,    
\AtlasOrcid[0000-0001-8021-8525]{A.~Basan}$^\textrm{\scriptsize 96}$,    
\AtlasOrcid[0000-0002-1533-0876]{M.~Baselga}$^\textrm{\scriptsize 44}$,    
\AtlasOrcid[0000-0002-2961-2735]{I.~Bashta}$^\textrm{\scriptsize 72a,72b}$,    
\AtlasOrcid[0000-0002-0129-1423]{A.~Bassalat}$^\textrm{\scriptsize 62,ag}$,    
\AtlasOrcid[0000-0001-9278-3863]{M.J.~Basso}$^\textrm{\scriptsize 161}$,    
\AtlasOrcid[0000-0003-1693-5946]{C.R.~Basson}$^\textrm{\scriptsize 97}$,    
\AtlasOrcid[0000-0002-6923-5372]{R.L.~Bates}$^\textrm{\scriptsize 55}$,    
\AtlasOrcid{S.~Batlamous}$^\textrm{\scriptsize 33e}$,    
\AtlasOrcid[0000-0001-7658-7766]{J.R.~Batley}$^\textrm{\scriptsize 30}$,    
\AtlasOrcid[0000-0001-6544-9376]{B.~Batool}$^\textrm{\scriptsize 146}$,    
\AtlasOrcid{M.~Battaglia}$^\textrm{\scriptsize 140}$,    
\AtlasOrcid[0000-0002-9148-4658]{M.~Bauce}$^\textrm{\scriptsize 70a,70b}$,    
\AtlasOrcid[0000-0003-2258-2892]{F.~Bauer}$^\textrm{\scriptsize 139,*}$,    
\AtlasOrcid[0000-0002-4568-5360]{P.~Bauer}$^\textrm{\scriptsize 22}$,    
\AtlasOrcid{H.S.~Bawa}$^\textrm{\scriptsize 29}$,    
\AtlasOrcid[0000-0003-3542-7242]{A.~Bayirli}$^\textrm{\scriptsize 11c}$,    
\AtlasOrcid[0000-0003-3623-3335]{J.B.~Beacham}$^\textrm{\scriptsize 47}$,    
\AtlasOrcid[0000-0002-2022-2140]{T.~Beau}$^\textrm{\scriptsize 130}$,    
\AtlasOrcid[0000-0003-4889-8748]{P.H.~Beauchemin}$^\textrm{\scriptsize 164}$,    
\AtlasOrcid[0000-0003-0562-4616]{F.~Becherer}$^\textrm{\scriptsize 50}$,    
\AtlasOrcid[0000-0003-3479-2221]{P.~Bechtle}$^\textrm{\scriptsize 22}$,    
\AtlasOrcid[0000-0001-7212-1096]{H.P.~Beck}$^\textrm{\scriptsize 18,q}$,    
\AtlasOrcid[0000-0002-6691-6498]{K.~Becker}$^\textrm{\scriptsize 172}$,    
\AtlasOrcid[0000-0003-0473-512X]{C.~Becot}$^\textrm{\scriptsize 44}$,    
\AtlasOrcid[0000-0002-8451-9672]{A.J.~Beddall}$^\textrm{\scriptsize 11a}$,    
\AtlasOrcid[0000-0003-4864-8909]{V.A.~Bednyakov}$^\textrm{\scriptsize 77}$,    
\AtlasOrcid[0000-0001-6294-6561]{C.P.~Bee}$^\textrm{\scriptsize 150}$,    
\AtlasOrcid[0000-0001-9805-2893]{T.A.~Beermann}$^\textrm{\scriptsize 34}$,    
\AtlasOrcid[0000-0003-4868-6059]{M.~Begalli}$^\textrm{\scriptsize 78b}$,    
\AtlasOrcid[0000-0002-1634-4399]{M.~Begel}$^\textrm{\scriptsize 27}$,    
\AtlasOrcid[0000-0002-7739-295X]{A.~Behera}$^\textrm{\scriptsize 150}$,    
\AtlasOrcid[0000-0002-5501-4640]{J.K.~Behr}$^\textrm{\scriptsize 44}$,    
\AtlasOrcid[0000-0002-1231-3819]{C.~Beirao~Da~Cruz~E~Silva}$^\textrm{\scriptsize 34}$,    
\AtlasOrcid[0000-0001-9024-4989]{J.F.~Beirer}$^\textrm{\scriptsize 51,34}$,    
\AtlasOrcid[0000-0002-7659-8948]{F.~Beisiegel}$^\textrm{\scriptsize 22}$,    
\AtlasOrcid[0000-0001-9974-1527]{M.~Belfkir}$^\textrm{\scriptsize 4}$,    
\AtlasOrcid[0000-0002-4009-0990]{G.~Bella}$^\textrm{\scriptsize 156}$,    
\AtlasOrcid[0000-0001-7098-9393]{L.~Bellagamba}$^\textrm{\scriptsize 21b}$,    
\AtlasOrcid[0000-0001-6775-0111]{A.~Bellerive}$^\textrm{\scriptsize 32}$,    
\AtlasOrcid[0000-0003-2049-9622]{P.~Bellos}$^\textrm{\scriptsize 19}$,    
\AtlasOrcid[0000-0003-0945-4087]{K.~Beloborodov}$^\textrm{\scriptsize 117b,117a}$,    
\AtlasOrcid[0000-0003-4617-8819]{K.~Belotskiy}$^\textrm{\scriptsize 108}$,    
\AtlasOrcid[0000-0002-1131-7121]{N.L.~Belyaev}$^\textrm{\scriptsize 108}$,    
\AtlasOrcid[0000-0001-5196-8327]{D.~Benchekroun}$^\textrm{\scriptsize 33a}$,    
\AtlasOrcid[0000-0002-0392-1783]{Y.~Benhammou}$^\textrm{\scriptsize 156}$,    
\AtlasOrcid[0000-0001-9338-4581]{D.P.~Benjamin}$^\textrm{\scriptsize 27}$,    
\AtlasOrcid[0000-0002-8623-1699]{M.~Benoit}$^\textrm{\scriptsize 27}$,    
\AtlasOrcid[0000-0002-6117-4536]{J.R.~Bensinger}$^\textrm{\scriptsize 24}$,    
\AtlasOrcid[0000-0003-3280-0953]{S.~Bentvelsen}$^\textrm{\scriptsize 115}$,    
\AtlasOrcid[0000-0002-3080-1824]{L.~Beresford}$^\textrm{\scriptsize 34}$,    
\AtlasOrcid[0000-0002-7026-8171]{M.~Beretta}$^\textrm{\scriptsize 49}$,    
\AtlasOrcid[0000-0002-2918-1824]{D.~Berge}$^\textrm{\scriptsize 17}$,    
\AtlasOrcid[0000-0002-1253-8583]{E.~Bergeaas~Kuutmann}$^\textrm{\scriptsize 166}$,    
\AtlasOrcid[0000-0002-7963-9725]{N.~Berger}$^\textrm{\scriptsize 4}$,    
\AtlasOrcid[0000-0002-8076-5614]{B.~Bergmann}$^\textrm{\scriptsize 136}$,    
\AtlasOrcid[0000-0002-0398-2228]{L.J.~Bergsten}$^\textrm{\scriptsize 24}$,    
\AtlasOrcid[0000-0002-9975-1781]{J.~Beringer}$^\textrm{\scriptsize 16}$,    
\AtlasOrcid[0000-0003-1911-772X]{S.~Berlendis}$^\textrm{\scriptsize 6}$,    
\AtlasOrcid[0000-0002-2837-2442]{G.~Bernardi}$^\textrm{\scriptsize 130}$,    
\AtlasOrcid[0000-0003-3433-1687]{C.~Bernius}$^\textrm{\scriptsize 148}$,    
\AtlasOrcid[0000-0001-8153-2719]{F.U.~Bernlochner}$^\textrm{\scriptsize 22}$,    
\AtlasOrcid[0000-0002-9569-8231]{T.~Berry}$^\textrm{\scriptsize 91}$,    
\AtlasOrcid[0000-0003-0780-0345]{P.~Berta}$^\textrm{\scriptsize 137}$,    
\AtlasOrcid[0000-0002-3824-409X]{A.~Berthold}$^\textrm{\scriptsize 46}$,    
\AtlasOrcid[0000-0003-4073-4941]{I.A.~Bertram}$^\textrm{\scriptsize 87}$,    
\AtlasOrcid[0000-0003-2011-3005]{O.~Bessidskaia~Bylund}$^\textrm{\scriptsize 176}$,    
\AtlasOrcid[0000-0003-0073-3821]{S.~Bethke}$^\textrm{\scriptsize 111}$,    
\AtlasOrcid[0000-0003-0839-9311]{A.~Betti}$^\textrm{\scriptsize 40}$,    
\AtlasOrcid[0000-0002-4105-9629]{A.J.~Bevan}$^\textrm{\scriptsize 90}$,    
\AtlasOrcid[0000-0002-9045-3278]{S.~Bhatta}$^\textrm{\scriptsize 150}$,    
\AtlasOrcid[0000-0003-3837-4166]{D.S.~Bhattacharya}$^\textrm{\scriptsize 171}$,    
\AtlasOrcid{P.~Bhattarai}$^\textrm{\scriptsize 24}$,    
\AtlasOrcid[0000-0003-3024-587X]{V.S.~Bhopatkar}$^\textrm{\scriptsize 5}$,    
\AtlasOrcid{R.~Bi}$^\textrm{\scriptsize 133}$,    
\AtlasOrcid[0000-0001-7345-7798]{R.M.~Bianchi}$^\textrm{\scriptsize 133}$,    
\AtlasOrcid[0000-0002-8663-6856]{O.~Biebel}$^\textrm{\scriptsize 110}$,    
\AtlasOrcid[0000-0002-2079-5344]{R.~Bielski}$^\textrm{\scriptsize 126}$,    
\AtlasOrcid[0000-0003-3004-0946]{N.V.~Biesuz}$^\textrm{\scriptsize 69a,69b}$,    
\AtlasOrcid[0000-0001-5442-1351]{M.~Biglietti}$^\textrm{\scriptsize 72a}$,    
\AtlasOrcid[0000-0002-6280-3306]{T.R.V.~Billoud}$^\textrm{\scriptsize 136}$,    
\AtlasOrcid[0000-0001-6172-545X]{M.~Bindi}$^\textrm{\scriptsize 51}$,    
\AtlasOrcid[0000-0002-2455-8039]{A.~Bingul}$^\textrm{\scriptsize 11d}$,    
\AtlasOrcid[0000-0001-6674-7869]{C.~Bini}$^\textrm{\scriptsize 70a,70b}$,    
\AtlasOrcid[0000-0002-1492-6715]{S.~Biondi}$^\textrm{\scriptsize 21b,21a}$,    
\AtlasOrcid[0000-0002-1559-3473]{A.~Biondini}$^\textrm{\scriptsize 88}$,    
\AtlasOrcid[0000-0001-6329-9191]{C.J.~Birch-sykes}$^\textrm{\scriptsize 97}$,    
\AtlasOrcid[0000-0003-2025-5935]{G.A.~Bird}$^\textrm{\scriptsize 19,138}$,    
\AtlasOrcid[0000-0002-3835-0968]{M.~Birman}$^\textrm{\scriptsize 174}$,    
\AtlasOrcid{T.~Bisanz}$^\textrm{\scriptsize 34}$,    
\AtlasOrcid[0000-0001-8361-2309]{J.P.~Biswal}$^\textrm{\scriptsize 2}$,    
\AtlasOrcid[0000-0002-7543-3471]{D.~Biswas}$^\textrm{\scriptsize 175,j}$,    
\AtlasOrcid[0000-0001-7979-1092]{A.~Bitadze}$^\textrm{\scriptsize 97}$,    
\AtlasOrcid[0000-0003-3628-5995]{C.~Bittrich}$^\textrm{\scriptsize 46}$,    
\AtlasOrcid[0000-0003-3485-0321]{K.~Bj\o{}rke}$^\textrm{\scriptsize 128}$,    
\AtlasOrcid[0000-0002-6696-5169]{I.~Bloch}$^\textrm{\scriptsize 44}$,    
\AtlasOrcid[0000-0001-6898-5633]{C.~Blocker}$^\textrm{\scriptsize 24}$,    
\AtlasOrcid[0000-0002-7716-5626]{A.~Blue}$^\textrm{\scriptsize 55}$,    
\AtlasOrcid[0000-0002-6134-0303]{U.~Blumenschein}$^\textrm{\scriptsize 90}$,    
\AtlasOrcid[0000-0001-5412-1236]{J.~Blumenthal}$^\textrm{\scriptsize 96}$,    
\AtlasOrcid[0000-0001-8462-351X]{G.J.~Bobbink}$^\textrm{\scriptsize 115}$,    
\AtlasOrcid[0000-0002-2003-0261]{V.S.~Bobrovnikov}$^\textrm{\scriptsize 117b,117a}$,    
\AtlasOrcid[0000-0001-9734-574X]{M.~Boehler}$^\textrm{\scriptsize 50}$,    
\AtlasOrcid[0000-0003-2138-9062]{D.~Bogavac}$^\textrm{\scriptsize 12}$,    
\AtlasOrcid[0000-0002-8635-9342]{A.G.~Bogdanchikov}$^\textrm{\scriptsize 117b,117a}$,    
\AtlasOrcid{C.~Bohm}$^\textrm{\scriptsize 43a}$,    
\AtlasOrcid[0000-0002-7736-0173]{V.~Boisvert}$^\textrm{\scriptsize 91}$,    
\AtlasOrcid[0000-0002-2668-889X]{P.~Bokan}$^\textrm{\scriptsize 44}$,    
\AtlasOrcid[0000-0002-2432-411X]{T.~Bold}$^\textrm{\scriptsize 81a}$,    
\AtlasOrcid[0000-0002-9807-861X]{M.~Bomben}$^\textrm{\scriptsize 130}$,    
\AtlasOrcid[0000-0002-9660-580X]{M.~Bona}$^\textrm{\scriptsize 90}$,    
\AtlasOrcid[0000-0003-0078-9817]{M.~Boonekamp}$^\textrm{\scriptsize 139}$,    
\AtlasOrcid[0000-0001-5880-7761]{C.D.~Booth}$^\textrm{\scriptsize 91}$,    
\AtlasOrcid[0000-0002-6890-1601]{A.G.~Borbély}$^\textrm{\scriptsize 55}$,    
\AtlasOrcid[0000-0002-5702-739X]{H.M.~Borecka-Bielska}$^\textrm{\scriptsize 106}$,    
\AtlasOrcid[0000-0003-0012-7856]{L.S.~Borgna}$^\textrm{\scriptsize 92}$,    
\AtlasOrcid[0000-0002-4226-9521]{G.~Borissov}$^\textrm{\scriptsize 87}$,    
\AtlasOrcid[0000-0002-1287-4712]{D.~Bortoletto}$^\textrm{\scriptsize 129}$,    
\AtlasOrcid[0000-0001-9207-6413]{D.~Boscherini}$^\textrm{\scriptsize 21b}$,    
\AtlasOrcid[0000-0002-7290-643X]{M.~Bosman}$^\textrm{\scriptsize 12}$,    
\AtlasOrcid[0000-0002-7134-8077]{J.D.~Bossio~Sola}$^\textrm{\scriptsize 34}$,    
\AtlasOrcid[0000-0002-7723-5030]{K.~Bouaouda}$^\textrm{\scriptsize 33a}$,    
\AtlasOrcid[0000-0002-9314-5860]{J.~Boudreau}$^\textrm{\scriptsize 133}$,    
\AtlasOrcid[0000-0002-5103-1558]{E.V.~Bouhova-Thacker}$^\textrm{\scriptsize 87}$,    
\AtlasOrcid[0000-0002-7809-3118]{D.~Boumediene}$^\textrm{\scriptsize 36}$,    
\AtlasOrcid[0000-0001-9683-7101]{R.~Bouquet}$^\textrm{\scriptsize 130}$,    
\AtlasOrcid[0000-0002-6647-6699]{A.~Boveia}$^\textrm{\scriptsize 122}$,    
\AtlasOrcid[0000-0001-7360-0726]{J.~Boyd}$^\textrm{\scriptsize 34}$,    
\AtlasOrcid[0000-0002-2704-835X]{D.~Boye}$^\textrm{\scriptsize 27}$,    
\AtlasOrcid[0000-0002-3355-4662]{I.R.~Boyko}$^\textrm{\scriptsize 77}$,    
\AtlasOrcid[0000-0003-2354-4812]{A.J.~Bozson}$^\textrm{\scriptsize 91}$,    
\AtlasOrcid[0000-0001-5762-3477]{J.~Bracinik}$^\textrm{\scriptsize 19}$,    
\AtlasOrcid[0000-0003-0992-3509]{N.~Brahimi}$^\textrm{\scriptsize 58d,58c}$,    
\AtlasOrcid[0000-0001-7992-0309]{G.~Brandt}$^\textrm{\scriptsize 176}$,    
\AtlasOrcid[0000-0001-5219-1417]{O.~Brandt}$^\textrm{\scriptsize 30}$,    
\AtlasOrcid[0000-0003-4339-4727]{F.~Braren}$^\textrm{\scriptsize 44}$,    
\AtlasOrcid[0000-0001-9726-4376]{B.~Brau}$^\textrm{\scriptsize 99}$,    
\AtlasOrcid[0000-0003-1292-9725]{J.E.~Brau}$^\textrm{\scriptsize 126}$,    
\AtlasOrcid{W.D.~Breaden~Madden}$^\textrm{\scriptsize 55}$,    
\AtlasOrcid[0000-0002-9096-780X]{K.~Brendlinger}$^\textrm{\scriptsize 44}$,    
\AtlasOrcid[0000-0001-5791-4872]{R.~Brener}$^\textrm{\scriptsize 174}$,    
\AtlasOrcid[0000-0001-5350-7081]{L.~Brenner}$^\textrm{\scriptsize 34}$,    
\AtlasOrcid[0000-0002-8204-4124]{R.~Brenner}$^\textrm{\scriptsize 166}$,    
\AtlasOrcid[0000-0003-4194-2734]{S.~Bressler}$^\textrm{\scriptsize 174}$,    
\AtlasOrcid[0000-0003-3518-3057]{B.~Brickwedde}$^\textrm{\scriptsize 96}$,    
\AtlasOrcid[0000-0002-3048-8153]{D.L.~Briglin}$^\textrm{\scriptsize 19}$,    
\AtlasOrcid[0000-0001-9998-4342]{D.~Britton}$^\textrm{\scriptsize 55}$,    
\AtlasOrcid[0000-0002-9246-7366]{D.~Britzger}$^\textrm{\scriptsize 111}$,    
\AtlasOrcid[0000-0003-0903-8948]{I.~Brock}$^\textrm{\scriptsize 22}$,    
\AtlasOrcid[0000-0002-4556-9212]{R.~Brock}$^\textrm{\scriptsize 103}$,    
\AtlasOrcid[0000-0002-3354-1810]{G.~Brooijmans}$^\textrm{\scriptsize 37}$,    
\AtlasOrcid[0000-0001-6161-3570]{W.K.~Brooks}$^\textrm{\scriptsize 141d}$,    
\AtlasOrcid[0000-0002-6800-9808]{E.~Brost}$^\textrm{\scriptsize 27}$,    
\AtlasOrcid[0000-0002-0206-1160]{P.A.~Bruckman~de~Renstrom}$^\textrm{\scriptsize 82}$,    
\AtlasOrcid[0000-0002-1479-2112]{B.~Br\"{u}ers}$^\textrm{\scriptsize 44}$,    
\AtlasOrcid[0000-0003-0208-2372]{D.~Bruncko}$^\textrm{\scriptsize 26b}$,    
\AtlasOrcid[0000-0003-4806-0718]{A.~Bruni}$^\textrm{\scriptsize 21b}$,    
\AtlasOrcid[0000-0001-5667-7748]{G.~Bruni}$^\textrm{\scriptsize 21b}$,    
\AtlasOrcid[0000-0002-4319-4023]{M.~Bruschi}$^\textrm{\scriptsize 21b}$,    
\AtlasOrcid[0000-0002-6168-689X]{N.~Bruscino}$^\textrm{\scriptsize 70a,70b}$,    
\AtlasOrcid[0000-0002-8420-3408]{L.~Bryngemark}$^\textrm{\scriptsize 148}$,    
\AtlasOrcid[0000-0002-8977-121X]{T.~Buanes}$^\textrm{\scriptsize 15}$,    
\AtlasOrcid[0000-0001-7318-5251]{Q.~Buat}$^\textrm{\scriptsize 150}$,    
\AtlasOrcid[0000-0002-4049-0134]{P.~Buchholz}$^\textrm{\scriptsize 146}$,    
\AtlasOrcid[0000-0001-8355-9237]{A.G.~Buckley}$^\textrm{\scriptsize 55}$,    
\AtlasOrcid[0000-0002-3711-148X]{I.A.~Budagov}$^\textrm{\scriptsize 77}$,    
\AtlasOrcid[0000-0002-8650-8125]{M.K.~Bugge}$^\textrm{\scriptsize 128}$,    
\AtlasOrcid[0000-0002-5687-2073]{O.~Bulekov}$^\textrm{\scriptsize 108}$,    
\AtlasOrcid[0000-0001-7148-6536]{B.A.~Bullard}$^\textrm{\scriptsize 57}$,    
\AtlasOrcid[0000-0003-4831-4132]{S.~Burdin}$^\textrm{\scriptsize 88}$,    
\AtlasOrcid[0000-0002-6900-825X]{C.D.~Burgard}$^\textrm{\scriptsize 44}$,    
\AtlasOrcid[0000-0003-0685-4122]{A.M.~Burger}$^\textrm{\scriptsize 124}$,    
\AtlasOrcid[0000-0001-5686-0948]{B.~Burghgrave}$^\textrm{\scriptsize 7}$,    
\AtlasOrcid[0000-0001-6726-6362]{J.T.P.~Burr}$^\textrm{\scriptsize 44}$,    
\AtlasOrcid[0000-0002-3427-6537]{C.D.~Burton}$^\textrm{\scriptsize 10}$,    
\AtlasOrcid[0000-0002-4690-0528]{J.C.~Burzynski}$^\textrm{\scriptsize 99}$,    
\AtlasOrcid[0000-0003-4482-2666]{E.L.~Busch}$^\textrm{\scriptsize 37}$,    
\AtlasOrcid[0000-0001-9196-0629]{V.~B\"uscher}$^\textrm{\scriptsize 96}$,    
\AtlasOrcid[0000-0003-0988-7878]{P.J.~Bussey}$^\textrm{\scriptsize 55}$,    
\AtlasOrcid[0000-0003-2834-836X]{J.M.~Butler}$^\textrm{\scriptsize 23}$,    
\AtlasOrcid[0000-0003-0188-6491]{C.M.~Buttar}$^\textrm{\scriptsize 55}$,    
\AtlasOrcid[0000-0002-5905-5394]{J.M.~Butterworth}$^\textrm{\scriptsize 92}$,    
\AtlasOrcid[0000-0002-5116-1897]{W.~Buttinger}$^\textrm{\scriptsize 138}$,    
\AtlasOrcid{C.J.~Buxo~Vazquez}$^\textrm{\scriptsize 103}$,    
\AtlasOrcid[0000-0002-5458-5564]{A.R.~Buzykaev}$^\textrm{\scriptsize 117b,117a}$,    
\AtlasOrcid[0000-0002-8467-8235]{G.~Cabras}$^\textrm{\scriptsize 21b}$,    
\AtlasOrcid[0000-0001-7640-7913]{S.~Cabrera~Urb\'an}$^\textrm{\scriptsize 168}$,    
\AtlasOrcid[0000-0001-7808-8442]{D.~Caforio}$^\textrm{\scriptsize 54}$,    
\AtlasOrcid[0000-0001-7575-3603]{H.~Cai}$^\textrm{\scriptsize 133}$,    
\AtlasOrcid[0000-0002-0758-7575]{V.M.M.~Cairo}$^\textrm{\scriptsize 148}$,    
\AtlasOrcid[0000-0002-9016-138X]{O.~Cakir}$^\textrm{\scriptsize 3a}$,    
\AtlasOrcid[0000-0002-1494-9538]{N.~Calace}$^\textrm{\scriptsize 34}$,    
\AtlasOrcid[0000-0002-1692-1678]{P.~Calafiura}$^\textrm{\scriptsize 16}$,    
\AtlasOrcid[0000-0002-9495-9145]{G.~Calderini}$^\textrm{\scriptsize 130}$,    
\AtlasOrcid[0000-0003-1600-464X]{P.~Calfayan}$^\textrm{\scriptsize 63}$,    
\AtlasOrcid[0000-0001-5969-3786]{G.~Callea}$^\textrm{\scriptsize 55}$,    
\AtlasOrcid{L.P.~Caloba}$^\textrm{\scriptsize 78b}$,    
\AtlasOrcid[0000-0002-7668-5275]{S.~Calvente~Lopez}$^\textrm{\scriptsize 95}$,    
\AtlasOrcid[0000-0002-9953-5333]{D.~Calvet}$^\textrm{\scriptsize 36}$,    
\AtlasOrcid[0000-0002-2531-3463]{S.~Calvet}$^\textrm{\scriptsize 36}$,    
\AtlasOrcid[0000-0002-3342-3566]{T.P.~Calvet}$^\textrm{\scriptsize 98}$,    
\AtlasOrcid[0000-0003-0125-2165]{M.~Calvetti}$^\textrm{\scriptsize 69a,69b}$,    
\AtlasOrcid[0000-0002-9192-8028]{R.~Camacho~Toro}$^\textrm{\scriptsize 130}$,    
\AtlasOrcid[0000-0003-0479-7689]{S.~Camarda}$^\textrm{\scriptsize 34}$,    
\AtlasOrcid[0000-0002-2855-7738]{D.~Camarero~Munoz}$^\textrm{\scriptsize 95}$,    
\AtlasOrcid[0000-0002-5732-5645]{P.~Camarri}$^\textrm{\scriptsize 71a,71b}$,    
\AtlasOrcid[0000-0002-9417-8613]{M.T.~Camerlingo}$^\textrm{\scriptsize 72a,72b}$,    
\AtlasOrcid[0000-0001-6097-2256]{D.~Cameron}$^\textrm{\scriptsize 128}$,    
\AtlasOrcid[0000-0001-5929-1357]{C.~Camincher}$^\textrm{\scriptsize 170}$,    
\AtlasOrcid[0000-0001-6746-3374]{M.~Campanelli}$^\textrm{\scriptsize 92}$,    
\AtlasOrcid[0000-0002-6386-9788]{A.~Camplani}$^\textrm{\scriptsize 38}$,    
\AtlasOrcid[0000-0003-2303-9306]{V.~Canale}$^\textrm{\scriptsize 67a,67b}$,    
\AtlasOrcid[0000-0002-9227-5217]{A.~Canesse}$^\textrm{\scriptsize 100}$,    
\AtlasOrcid[0000-0002-8880-434X]{M.~Cano~Bret}$^\textrm{\scriptsize 75}$,    
\AtlasOrcid[0000-0001-8449-1019]{J.~Cantero}$^\textrm{\scriptsize 124}$,    
\AtlasOrcid[0000-0001-8747-2809]{Y.~Cao}$^\textrm{\scriptsize 167}$,    
\AtlasOrcid[0000-0002-3562-9592]{F.~Capocasa}$^\textrm{\scriptsize 24}$,    
\AtlasOrcid[0000-0002-2443-6525]{M.~Capua}$^\textrm{\scriptsize 39b,39a}$,    
\AtlasOrcid{A.~Carbone}$^\textrm{\scriptsize 66a,66b}$,    
\AtlasOrcid[0000-0003-4541-4189]{R.~Cardarelli}$^\textrm{\scriptsize 71a}$,    
\AtlasOrcid[0000-0002-6511-7096]{J.C.J.~Cardenas}$^\textrm{\scriptsize 7}$,    
\AtlasOrcid[0000-0002-4478-3524]{F.~Cardillo}$^\textrm{\scriptsize 168}$,    
\AtlasOrcid[0000-0002-4376-4911]{G.~Carducci}$^\textrm{\scriptsize 39b,39a}$,    
\AtlasOrcid[0000-0003-4058-5376]{T.~Carli}$^\textrm{\scriptsize 34}$,    
\AtlasOrcid[0000-0002-3924-0445]{G.~Carlino}$^\textrm{\scriptsize 67a}$,    
\AtlasOrcid[0000-0002-7550-7821]{B.T.~Carlson}$^\textrm{\scriptsize 133}$,    
\AtlasOrcid[0000-0002-4139-9543]{E.M.~Carlson}$^\textrm{\scriptsize 170,162a}$,    
\AtlasOrcid[0000-0003-4535-2926]{L.~Carminati}$^\textrm{\scriptsize 66a,66b}$,    
\AtlasOrcid[0000-0003-3570-7332]{M.~Carnesale}$^\textrm{\scriptsize 70a,70b}$,    
\AtlasOrcid[0000-0001-5659-4440]{R.M.D.~Carney}$^\textrm{\scriptsize 148}$,    
\AtlasOrcid[0000-0003-2941-2829]{S.~Caron}$^\textrm{\scriptsize 114}$,    
\AtlasOrcid[0000-0002-7863-1166]{E.~Carquin}$^\textrm{\scriptsize 141d}$,    
\AtlasOrcid[0000-0001-8650-942X]{S.~Carr\'a}$^\textrm{\scriptsize 44}$,    
\AtlasOrcid[0000-0002-8846-2714]{G.~Carratta}$^\textrm{\scriptsize 21b,21a}$,    
\AtlasOrcid[0000-0002-7836-4264]{J.W.S.~Carter}$^\textrm{\scriptsize 161}$,    
\AtlasOrcid[0000-0003-2966-6036]{T.M.~Carter}$^\textrm{\scriptsize 48}$,    
\AtlasOrcid[0000-0002-3343-3529]{D.~Casadei}$^\textrm{\scriptsize 31c}$,    
\AtlasOrcid[0000-0002-0394-5646]{M.P.~Casado}$^\textrm{\scriptsize 12,g}$,    
\AtlasOrcid{A.F.~Casha}$^\textrm{\scriptsize 161}$,    
\AtlasOrcid[0000-0001-7991-2018]{E.G.~Castiglia}$^\textrm{\scriptsize 177}$,    
\AtlasOrcid[0000-0002-1172-1052]{F.L.~Castillo}$^\textrm{\scriptsize 59a}$,    
\AtlasOrcid[0000-0003-1396-2826]{L.~Castillo~Garcia}$^\textrm{\scriptsize 12}$,    
\AtlasOrcid[0000-0002-8245-1790]{V.~Castillo~Gimenez}$^\textrm{\scriptsize 168}$,    
\AtlasOrcid[0000-0001-8491-4376]{N.F.~Castro}$^\textrm{\scriptsize 134a,134e}$,    
\AtlasOrcid[0000-0001-8774-8887]{A.~Catinaccio}$^\textrm{\scriptsize 34}$,    
\AtlasOrcid[0000-0001-8915-0184]{J.R.~Catmore}$^\textrm{\scriptsize 128}$,    
\AtlasOrcid{A.~Cattai}$^\textrm{\scriptsize 34}$,    
\AtlasOrcid[0000-0002-4297-8539]{V.~Cavaliere}$^\textrm{\scriptsize 27}$,    
\AtlasOrcid[0000-0002-1096-5290]{N.~Cavalli}$^\textrm{\scriptsize 21b,21a}$,    
\AtlasOrcid[0000-0001-6203-9347]{V.~Cavasinni}$^\textrm{\scriptsize 69a,69b}$,    
\AtlasOrcid[0000-0003-3793-0159]{E.~Celebi}$^\textrm{\scriptsize 11b}$,    
\AtlasOrcid[0000-0001-6962-4573]{F.~Celli}$^\textrm{\scriptsize 129}$,    
\AtlasOrcid[0000-0002-7945-4392]{M.S.~Centonze}$^\textrm{\scriptsize 65a,65b}$,    
\AtlasOrcid[0000-0003-0683-2177]{K.~Cerny}$^\textrm{\scriptsize 125}$,    
\AtlasOrcid[0000-0002-4300-703X]{A.S.~Cerqueira}$^\textrm{\scriptsize 78a}$,    
\AtlasOrcid[0000-0002-1904-6661]{A.~Cerri}$^\textrm{\scriptsize 151}$,    
\AtlasOrcid[0000-0002-8077-7850]{L.~Cerrito}$^\textrm{\scriptsize 71a,71b}$,    
\AtlasOrcid[0000-0001-9669-9642]{F.~Cerutti}$^\textrm{\scriptsize 16}$,    
\AtlasOrcid[0000-0002-0518-1459]{A.~Cervelli}$^\textrm{\scriptsize 21b}$,    
\AtlasOrcid[0000-0001-5050-8441]{S.A.~Cetin}$^\textrm{\scriptsize 11b}$,    
\AtlasOrcid[0000-0002-3117-5415]{Z.~Chadi}$^\textrm{\scriptsize 33a}$,    
\AtlasOrcid[0000-0002-9865-4146]{D.~Chakraborty}$^\textrm{\scriptsize 116}$,    
\AtlasOrcid[0000-0002-4343-9094]{M.~Chala}$^\textrm{\scriptsize 134f}$,    
\AtlasOrcid[0000-0001-7069-0295]{J.~Chan}$^\textrm{\scriptsize 175}$,    
\AtlasOrcid[0000-0003-2150-1296]{W.S.~Chan}$^\textrm{\scriptsize 115}$,    
\AtlasOrcid[0000-0002-5369-8540]{W.Y.~Chan}$^\textrm{\scriptsize 88}$,    
\AtlasOrcid[0000-0002-2926-8962]{J.D.~Chapman}$^\textrm{\scriptsize 30}$,    
\AtlasOrcid[0000-0002-5376-2397]{B.~Chargeishvili}$^\textrm{\scriptsize 154b}$,    
\AtlasOrcid[0000-0003-0211-2041]{D.G.~Charlton}$^\textrm{\scriptsize 19}$,    
\AtlasOrcid[0000-0001-6288-5236]{T.P.~Charman}$^\textrm{\scriptsize 90}$,    
\AtlasOrcid[0000-0003-4241-7405]{M.~Chatterjee}$^\textrm{\scriptsize 18}$,    
\AtlasOrcid[0000-0001-7314-7247]{S.~Chekanov}$^\textrm{\scriptsize 5}$,    
\AtlasOrcid[0000-0002-4034-2326]{S.V.~Chekulaev}$^\textrm{\scriptsize 162a}$,    
\AtlasOrcid[0000-0002-3468-9761]{G.A.~Chelkov}$^\textrm{\scriptsize 77,ae}$,    
\AtlasOrcid[0000-0001-9973-7966]{A.~Chen}$^\textrm{\scriptsize 102}$,    
\AtlasOrcid[0000-0002-3034-8943]{B.~Chen}$^\textrm{\scriptsize 156}$,    
\AtlasOrcid[0000-0002-7985-9023]{B.~Chen}$^\textrm{\scriptsize 170}$,    
\AtlasOrcid{C.~Chen}$^\textrm{\scriptsize 58a}$,    
\AtlasOrcid[0000-0003-1589-9955]{C.H.~Chen}$^\textrm{\scriptsize 76}$,    
\AtlasOrcid[0000-0002-5895-6799]{H.~Chen}$^\textrm{\scriptsize 13c}$,    
\AtlasOrcid[0000-0002-9936-0115]{H.~Chen}$^\textrm{\scriptsize 27}$,    
\AtlasOrcid[0000-0002-2554-2725]{J.~Chen}$^\textrm{\scriptsize 58c}$,    
\AtlasOrcid[0000-0003-1586-5253]{J.~Chen}$^\textrm{\scriptsize 24}$,    
\AtlasOrcid[0000-0001-7987-9764]{S.~Chen}$^\textrm{\scriptsize 131}$,    
\AtlasOrcid[0000-0003-0447-5348]{S.J.~Chen}$^\textrm{\scriptsize 13c}$,    
\AtlasOrcid[0000-0003-4977-2717]{X.~Chen}$^\textrm{\scriptsize 58c}$,    
\AtlasOrcid[0000-0003-4027-3305]{X.~Chen}$^\textrm{\scriptsize 13b}$,    
\AtlasOrcid[0000-0001-6793-3604]{Y.~Chen}$^\textrm{\scriptsize 58a}$,    
\AtlasOrcid[0000-0002-2720-1115]{Y-H.~Chen}$^\textrm{\scriptsize 44}$,    
\AtlasOrcid[0000-0002-4086-1847]{C.L.~Cheng}$^\textrm{\scriptsize 175}$,    
\AtlasOrcid[0000-0002-8912-4389]{H.C.~Cheng}$^\textrm{\scriptsize 60a}$,    
\AtlasOrcid[0000-0002-0967-2351]{A.~Cheplakov}$^\textrm{\scriptsize 77}$,    
\AtlasOrcid[0000-0002-8772-0961]{E.~Cheremushkina}$^\textrm{\scriptsize 44}$,    
\AtlasOrcid[0000-0002-3150-8478]{E.~Cherepanova}$^\textrm{\scriptsize 77}$,    
\AtlasOrcid[0000-0002-5842-2818]{R.~Cherkaoui~El~Moursli}$^\textrm{\scriptsize 33e}$,    
\AtlasOrcid[0000-0002-2562-9724]{E.~Cheu}$^\textrm{\scriptsize 6}$,    
\AtlasOrcid[0000-0003-2176-4053]{K.~Cheung}$^\textrm{\scriptsize 61}$,    
\AtlasOrcid[0000-0003-3762-7264]{L.~Chevalier}$^\textrm{\scriptsize 139}$,    
\AtlasOrcid[0000-0002-4210-2924]{V.~Chiarella}$^\textrm{\scriptsize 49}$,    
\AtlasOrcid[0000-0001-9851-4816]{G.~Chiarelli}$^\textrm{\scriptsize 69a}$,    
\AtlasOrcid[0000-0002-2458-9513]{G.~Chiodini}$^\textrm{\scriptsize 65a}$,    
\AtlasOrcid[0000-0001-9214-8528]{A.S.~Chisholm}$^\textrm{\scriptsize 19}$,    
\AtlasOrcid[0000-0003-2262-4773]{A.~Chitan}$^\textrm{\scriptsize 25b}$,    
\AtlasOrcid[0000-0002-9487-9348]{Y.H.~Chiu}$^\textrm{\scriptsize 170}$,    
\AtlasOrcid[0000-0001-5841-3316]{M.V.~Chizhov}$^\textrm{\scriptsize 77,s}$,    
\AtlasOrcid[0000-0003-0748-694X]{K.~Choi}$^\textrm{\scriptsize 10}$,    
\AtlasOrcid[0000-0002-3243-5610]{A.R.~Chomont}$^\textrm{\scriptsize 70a,70b}$,    
\AtlasOrcid[0000-0002-2204-5731]{Y.~Chou}$^\textrm{\scriptsize 99}$,    
\AtlasOrcid{Y.S.~Chow}$^\textrm{\scriptsize 115}$,    
\AtlasOrcid[0000-0002-2681-8105]{T.~Chowdhury}$^\textrm{\scriptsize 31f}$,    
\AtlasOrcid[0000-0002-2509-0132]{L.D.~Christopher}$^\textrm{\scriptsize 31f}$,    
\AtlasOrcid[0000-0002-1971-0403]{M.C.~Chu}$^\textrm{\scriptsize 60a}$,    
\AtlasOrcid[0000-0003-2848-0184]{X.~Chu}$^\textrm{\scriptsize 13a,13d}$,    
\AtlasOrcid[0000-0002-6425-2579]{J.~Chudoba}$^\textrm{\scriptsize 135}$,    
\AtlasOrcid[0000-0002-6190-8376]{J.J.~Chwastowski}$^\textrm{\scriptsize 82}$,    
\AtlasOrcid[0000-0002-3533-3847]{D.~Cieri}$^\textrm{\scriptsize 111}$,    
\AtlasOrcid[0000-0003-2751-3474]{K.M.~Ciesla}$^\textrm{\scriptsize 82}$,    
\AtlasOrcid[0000-0002-2037-7185]{V.~Cindro}$^\textrm{\scriptsize 89}$,    
\AtlasOrcid[0000-0002-9224-3784]{I.A.~Cioar\u{a}}$^\textrm{\scriptsize 25b}$,    
\AtlasOrcid[0000-0002-3081-4879]{A.~Ciocio}$^\textrm{\scriptsize 16}$,    
\AtlasOrcid[0000-0001-6556-856X]{F.~Cirotto}$^\textrm{\scriptsize 67a,67b}$,    
\AtlasOrcid[0000-0003-1831-6452]{Z.H.~Citron}$^\textrm{\scriptsize 174,k}$,    
\AtlasOrcid[0000-0002-0842-0654]{M.~Citterio}$^\textrm{\scriptsize 66a}$,    
\AtlasOrcid{D.A.~Ciubotaru}$^\textrm{\scriptsize 25b}$,    
\AtlasOrcid[0000-0002-8920-4880]{B.M.~Ciungu}$^\textrm{\scriptsize 161}$,    
\AtlasOrcid[0000-0001-8341-5911]{A.~Clark}$^\textrm{\scriptsize 52}$,    
\AtlasOrcid[0000-0002-3777-0880]{P.J.~Clark}$^\textrm{\scriptsize 48}$,    
\AtlasOrcid[0000-0003-3210-1722]{J.M.~Clavijo~Columbie}$^\textrm{\scriptsize 44}$,    
\AtlasOrcid[0000-0001-9952-934X]{S.E.~Clawson}$^\textrm{\scriptsize 97}$,    
\AtlasOrcid[0000-0003-3122-3605]{C.~Clement}$^\textrm{\scriptsize 43a,43b}$,    
\AtlasOrcid[0000-0002-4876-5200]{L.~Clissa}$^\textrm{\scriptsize 21b,21a}$,    
\AtlasOrcid[0000-0001-8195-7004]{Y.~Coadou}$^\textrm{\scriptsize 98}$,    
\AtlasOrcid[0000-0003-3309-0762]{M.~Cobal}$^\textrm{\scriptsize 64a,64c}$,    
\AtlasOrcid[0000-0003-2368-4559]{A.~Coccaro}$^\textrm{\scriptsize 53b}$,    
\AtlasOrcid{J.~Cochran}$^\textrm{\scriptsize 76}$,    
\AtlasOrcid[0000-0001-8985-5379]{R.F.~Coelho~Barrue}$^\textrm{\scriptsize 134a}$,    
\AtlasOrcid[0000-0001-5200-9195]{R.~Coelho~Lopes~De~Sa}$^\textrm{\scriptsize 99}$,    
\AtlasOrcid[0000-0002-5145-3646]{S.~Coelli}$^\textrm{\scriptsize 66a}$,    
\AtlasOrcid{H.~Cohen}$^\textrm{\scriptsize 156}$,    
\AtlasOrcid[0000-0003-2301-1637]{A.E.C.~Coimbra}$^\textrm{\scriptsize 34}$,    
\AtlasOrcid[0000-0002-5092-2148]{B.~Cole}$^\textrm{\scriptsize 37}$,    
\AtlasOrcid[0000-0002-9412-7090]{J.~Collot}$^\textrm{\scriptsize 56}$,    
\AtlasOrcid[0000-0002-9187-7478]{P.~Conde~Mui\~no}$^\textrm{\scriptsize 134a,134h}$,    
\AtlasOrcid[0000-0001-6000-7245]{S.H.~Connell}$^\textrm{\scriptsize 31c}$,    
\AtlasOrcid[0000-0001-9127-6827]{I.A.~Connelly}$^\textrm{\scriptsize 55}$,    
\AtlasOrcid[0000-0002-0215-2767]{E.I.~Conroy}$^\textrm{\scriptsize 129}$,    
\AtlasOrcid[0000-0002-5575-1413]{F.~Conventi}$^\textrm{\scriptsize 67a,ak}$,    
\AtlasOrcid[0000-0001-9297-1063]{H.G.~Cooke}$^\textrm{\scriptsize 19}$,    
\AtlasOrcid[0000-0002-7107-5902]{A.M.~Cooper-Sarkar}$^\textrm{\scriptsize 129}$,    
\AtlasOrcid[0000-0002-2532-3207]{F.~Cormier}$^\textrm{\scriptsize 169}$,    
\AtlasOrcid[0000-0003-2136-4842]{L.D.~Corpe}$^\textrm{\scriptsize 34}$,    
\AtlasOrcid[0000-0001-8729-466X]{M.~Corradi}$^\textrm{\scriptsize 70a,70b}$,    
\AtlasOrcid[0000-0003-2485-0248]{E.E.~Corrigan}$^\textrm{\scriptsize 94}$,    
\AtlasOrcid[0000-0002-4970-7600]{F.~Corriveau}$^\textrm{\scriptsize 100,y}$,    
\AtlasOrcid[0000-0002-2064-2954]{M.J.~Costa}$^\textrm{\scriptsize 168}$,    
\AtlasOrcid[0000-0002-8056-8469]{F.~Costanza}$^\textrm{\scriptsize 4}$,    
\AtlasOrcid[0000-0003-4920-6264]{D.~Costanzo}$^\textrm{\scriptsize 144}$,    
\AtlasOrcid[0000-0003-2444-8267]{B.M.~Cote}$^\textrm{\scriptsize 122}$,    
\AtlasOrcid[0000-0001-8363-9827]{G.~Cowan}$^\textrm{\scriptsize 91}$,    
\AtlasOrcid[0000-0001-7002-652X]{J.W.~Cowley}$^\textrm{\scriptsize 30}$,    
\AtlasOrcid[0000-0002-5769-7094]{K.~Cranmer}$^\textrm{\scriptsize 120}$,    
\AtlasOrcid[0000-0001-5980-5805]{S.~Cr\'ep\'e-Renaudin}$^\textrm{\scriptsize 56}$,    
\AtlasOrcid[0000-0001-6457-2575]{F.~Crescioli}$^\textrm{\scriptsize 130}$,    
\AtlasOrcid[0000-0003-3893-9171]{M.~Cristinziani}$^\textrm{\scriptsize 146}$,    
\AtlasOrcid[0000-0002-0127-1342]{M.~Cristoforetti}$^\textrm{\scriptsize 73a,73b,b}$,    
\AtlasOrcid[0000-0002-8731-4525]{V.~Croft}$^\textrm{\scriptsize 164}$,    
\AtlasOrcid[0000-0001-5990-4811]{G.~Crosetti}$^\textrm{\scriptsize 39b,39a}$,    
\AtlasOrcid[0000-0003-1494-7898]{A.~Cueto}$^\textrm{\scriptsize 34}$,    
\AtlasOrcid[0000-0003-3519-1356]{T.~Cuhadar~Donszelmann}$^\textrm{\scriptsize 165}$,    
\AtlasOrcid[0000-0002-9923-1313]{H.~Cui}$^\textrm{\scriptsize 13a,13d}$,    
\AtlasOrcid[0000-0002-7834-1716]{A.R.~Cukierman}$^\textrm{\scriptsize 148}$,    
\AtlasOrcid[0000-0001-5517-8795]{W.R.~Cunningham}$^\textrm{\scriptsize 55}$,    
\AtlasOrcid[0000-0002-8682-9316]{F.~Curcio}$^\textrm{\scriptsize 39b,39a}$,    
\AtlasOrcid[0000-0003-0723-1437]{P.~Czodrowski}$^\textrm{\scriptsize 34}$,    
\AtlasOrcid[0000-0003-1943-5883]{M.M.~Czurylo}$^\textrm{\scriptsize 59b}$,    
\AtlasOrcid[0000-0001-7991-593X]{M.J.~Da~Cunha~Sargedas~De~Sousa}$^\textrm{\scriptsize 58a}$,    
\AtlasOrcid[0000-0003-1746-1914]{J.V.~Da~Fonseca~Pinto}$^\textrm{\scriptsize 78b}$,    
\AtlasOrcid[0000-0001-6154-7323]{C.~Da~Via}$^\textrm{\scriptsize 97}$,    
\AtlasOrcid[0000-0001-9061-9568]{W.~Dabrowski}$^\textrm{\scriptsize 81a}$,    
\AtlasOrcid[0000-0002-7050-2669]{T.~Dado}$^\textrm{\scriptsize 45}$,    
\AtlasOrcid[0000-0002-5222-7894]{S.~Dahbi}$^\textrm{\scriptsize 31f}$,    
\AtlasOrcid[0000-0002-9607-5124]{T.~Dai}$^\textrm{\scriptsize 102}$,    
\AtlasOrcid[0000-0002-1391-2477]{C.~Dallapiccola}$^\textrm{\scriptsize 99}$,    
\AtlasOrcid[0000-0001-6278-9674]{M.~Dam}$^\textrm{\scriptsize 38}$,    
\AtlasOrcid[0000-0002-9742-3709]{G.~D'amen}$^\textrm{\scriptsize 27}$,    
\AtlasOrcid[0000-0002-2081-0129]{V.~D'Amico}$^\textrm{\scriptsize 72a,72b}$,    
\AtlasOrcid[0000-0002-7290-1372]{J.~Damp}$^\textrm{\scriptsize 96}$,    
\AtlasOrcid[0000-0002-9271-7126]{J.R.~Dandoy}$^\textrm{\scriptsize 131}$,    
\AtlasOrcid[0000-0002-2335-793X]{M.F.~Daneri}$^\textrm{\scriptsize 28}$,    
\AtlasOrcid[0000-0002-7807-7484]{M.~Danninger}$^\textrm{\scriptsize 147}$,    
\AtlasOrcid[0000-0003-1645-8393]{V.~Dao}$^\textrm{\scriptsize 34}$,    
\AtlasOrcid[0000-0003-2165-0638]{G.~Darbo}$^\textrm{\scriptsize 53b}$,    
\AtlasOrcid[0000-0002-9766-3657]{S.~Darmora}$^\textrm{\scriptsize 5}$,    
\AtlasOrcid[0000-0002-1559-9525]{A.~Dattagupta}$^\textrm{\scriptsize 126}$,    
\AtlasOrcid[0000-0003-3393-6318]{S.~D'Auria}$^\textrm{\scriptsize 66a,66b}$,    
\AtlasOrcid[0000-0002-1794-1443]{C.~David}$^\textrm{\scriptsize 162b}$,    
\AtlasOrcid[0000-0002-3770-8307]{T.~Davidek}$^\textrm{\scriptsize 137}$,    
\AtlasOrcid[0000-0003-2679-1288]{D.R.~Davis}$^\textrm{\scriptsize 47}$,    
\AtlasOrcid[0000-0002-4544-169X]{B.~Davis-Purcell}$^\textrm{\scriptsize 32}$,    
\AtlasOrcid[0000-0002-5177-8950]{I.~Dawson}$^\textrm{\scriptsize 90}$,    
\AtlasOrcid[0000-0002-5647-4489]{K.~De}$^\textrm{\scriptsize 7}$,    
\AtlasOrcid[0000-0002-7268-8401]{R.~De~Asmundis}$^\textrm{\scriptsize 67a}$,    
\AtlasOrcid[0000-0002-4285-2047]{M.~De~Beurs}$^\textrm{\scriptsize 115}$,    
\AtlasOrcid[0000-0003-2178-5620]{S.~De~Castro}$^\textrm{\scriptsize 21b,21a}$,    
\AtlasOrcid[0000-0001-6850-4078]{N.~De~Groot}$^\textrm{\scriptsize 114}$,    
\AtlasOrcid[0000-0002-5330-2614]{P.~de~Jong}$^\textrm{\scriptsize 115}$,    
\AtlasOrcid[0000-0002-4516-5269]{H.~De~la~Torre}$^\textrm{\scriptsize 103}$,    
\AtlasOrcid[0000-0001-6651-845X]{A.~De~Maria}$^\textrm{\scriptsize 13c}$,    
\AtlasOrcid[0000-0002-8151-581X]{D.~De~Pedis}$^\textrm{\scriptsize 70a}$,    
\AtlasOrcid[0000-0001-8099-7821]{A.~De~Salvo}$^\textrm{\scriptsize 70a}$,    
\AtlasOrcid[0000-0003-4704-525X]{U.~De~Sanctis}$^\textrm{\scriptsize 71a,71b}$,    
\AtlasOrcid[0000-0001-6423-0719]{M.~De~Santis}$^\textrm{\scriptsize 71a,71b}$,    
\AtlasOrcid[0000-0002-9158-6646]{A.~De~Santo}$^\textrm{\scriptsize 151}$,    
\AtlasOrcid[0000-0001-9163-2211]{J.B.~De~Vivie~De~Regie}$^\textrm{\scriptsize 56}$,    
\AtlasOrcid{D.V.~Dedovich}$^\textrm{\scriptsize 77}$,    
\AtlasOrcid[0000-0002-6966-4935]{J.~Degens}$^\textrm{\scriptsize 115}$,    
\AtlasOrcid[0000-0003-0360-6051]{A.M.~Deiana}$^\textrm{\scriptsize 40}$,    
\AtlasOrcid[0000-0001-7090-4134]{J.~Del~Peso}$^\textrm{\scriptsize 95}$,    
\AtlasOrcid[0000-0002-6096-7649]{Y.~Delabat~Diaz}$^\textrm{\scriptsize 44}$,    
\AtlasOrcid[0000-0003-0777-6031]{F.~Deliot}$^\textrm{\scriptsize 139}$,    
\AtlasOrcid[0000-0001-7021-3333]{C.M.~Delitzsch}$^\textrm{\scriptsize 6}$,    
\AtlasOrcid[0000-0003-4446-3368]{M.~Della~Pietra}$^\textrm{\scriptsize 67a,67b}$,    
\AtlasOrcid[0000-0001-8530-7447]{D.~Della~Volpe}$^\textrm{\scriptsize 52}$,    
\AtlasOrcid[0000-0003-2453-7745]{A.~Dell'Acqua}$^\textrm{\scriptsize 34}$,    
\AtlasOrcid[0000-0002-9601-4225]{L.~Dell'Asta}$^\textrm{\scriptsize 66a,66b}$,    
\AtlasOrcid[0000-0003-2992-3805]{M.~Delmastro}$^\textrm{\scriptsize 4}$,    
\AtlasOrcid[0000-0002-9556-2924]{P.A.~Delsart}$^\textrm{\scriptsize 56}$,    
\AtlasOrcid[0000-0002-7282-1786]{S.~Demers}$^\textrm{\scriptsize 177}$,    
\AtlasOrcid[0000-0002-7730-3072]{M.~Demichev}$^\textrm{\scriptsize 77}$,    
\AtlasOrcid[0000-0002-4028-7881]{S.P.~Denisov}$^\textrm{\scriptsize 118}$,    
\AtlasOrcid[0000-0002-4910-5378]{L.~D'Eramo}$^\textrm{\scriptsize 116}$,    
\AtlasOrcid[0000-0001-5660-3095]{D.~Derendarz}$^\textrm{\scriptsize 82}$,    
\AtlasOrcid[0000-0002-7116-8551]{J.E.~Derkaoui}$^\textrm{\scriptsize 33d}$,    
\AtlasOrcid[0000-0002-3505-3503]{F.~Derue}$^\textrm{\scriptsize 130}$,    
\AtlasOrcid[0000-0003-3929-8046]{P.~Dervan}$^\textrm{\scriptsize 88}$,    
\AtlasOrcid[0000-0001-5836-6118]{K.~Desch}$^\textrm{\scriptsize 22}$,    
\AtlasOrcid[0000-0002-9593-6201]{K.~Dette}$^\textrm{\scriptsize 161}$,    
\AtlasOrcid[0000-0002-6477-764X]{C.~Deutsch}$^\textrm{\scriptsize 22}$,    
\AtlasOrcid[0000-0002-8906-5884]{P.O.~Deviveiros}$^\textrm{\scriptsize 34}$,    
\AtlasOrcid[0000-0002-9870-2021]{F.A.~Di~Bello}$^\textrm{\scriptsize 70a,70b}$,    
\AtlasOrcid[0000-0001-8289-5183]{A.~Di~Ciaccio}$^\textrm{\scriptsize 71a,71b}$,    
\AtlasOrcid[0000-0003-0751-8083]{L.~Di~Ciaccio}$^\textrm{\scriptsize 4}$,    
\AtlasOrcid[0000-0001-8078-2759]{A.~Di~Domenico}$^\textrm{\scriptsize 70a,70b}$,    
\AtlasOrcid[0000-0003-2213-9284]{C.~Di~Donato}$^\textrm{\scriptsize 67a,67b}$,    
\AtlasOrcid[0000-0002-9508-4256]{A.~Di~Girolamo}$^\textrm{\scriptsize 34}$,    
\AtlasOrcid[0000-0002-7838-576X]{G.~Di~Gregorio}$^\textrm{\scriptsize 69a,69b}$,    
\AtlasOrcid[0000-0002-9074-2133]{A.~Di~Luca}$^\textrm{\scriptsize 73a,73b}$,    
\AtlasOrcid[0000-0002-4067-1592]{B.~Di~Micco}$^\textrm{\scriptsize 72a,72b}$,    
\AtlasOrcid[0000-0003-1111-3783]{R.~Di~Nardo}$^\textrm{\scriptsize 72a,72b}$,    
\AtlasOrcid[0000-0002-6193-5091]{C.~Diaconu}$^\textrm{\scriptsize 98}$,    
\AtlasOrcid[0000-0001-6882-5402]{F.A.~Dias}$^\textrm{\scriptsize 115}$,    
\AtlasOrcid[0000-0001-8855-3520]{T.~Dias~Do~Vale}$^\textrm{\scriptsize 134a}$,    
\AtlasOrcid[0000-0003-1258-8684]{M.A.~Diaz}$^\textrm{\scriptsize 141a}$,    
\AtlasOrcid[0000-0001-7934-3046]{F.G.~Diaz~Capriles}$^\textrm{\scriptsize 22}$,    
\AtlasOrcid[0000-0001-5450-5328]{J.~Dickinson}$^\textrm{\scriptsize 16}$,    
\AtlasOrcid[0000-0001-9942-6543]{M.~Didenko}$^\textrm{\scriptsize 168}$,    
\AtlasOrcid[0000-0002-7611-355X]{E.B.~Diehl}$^\textrm{\scriptsize 102}$,    
\AtlasOrcid[0000-0001-7061-1585]{J.~Dietrich}$^\textrm{\scriptsize 17}$,    
\AtlasOrcid[0000-0003-3694-6167]{S.~D\'iez~Cornell}$^\textrm{\scriptsize 44}$,    
\AtlasOrcid[0000-0002-0482-1127]{C.~Diez~Pardos}$^\textrm{\scriptsize 146}$,    
\AtlasOrcid[0000-0003-0086-0599]{A.~Dimitrievska}$^\textrm{\scriptsize 16}$,    
\AtlasOrcid[0000-0002-4614-956X]{W.~Ding}$^\textrm{\scriptsize 13b}$,    
\AtlasOrcid[0000-0001-5767-2121]{J.~Dingfelder}$^\textrm{\scriptsize 22}$,    
\AtlasOrcid[0000-0002-2683-7349]{I-M.~Dinu}$^\textrm{\scriptsize 25b}$,    
\AtlasOrcid[0000-0002-5172-7520]{S.J.~Dittmeier}$^\textrm{\scriptsize 59b}$,    
\AtlasOrcid[0000-0002-1760-8237]{F.~Dittus}$^\textrm{\scriptsize 34}$,    
\AtlasOrcid[0000-0003-1881-3360]{F.~Djama}$^\textrm{\scriptsize 98}$,    
\AtlasOrcid[0000-0002-9414-8350]{T.~Djobava}$^\textrm{\scriptsize 154b}$,    
\AtlasOrcid[0000-0002-6488-8219]{J.I.~Djuvsland}$^\textrm{\scriptsize 15}$,    
\AtlasOrcid[0000-0002-0836-6483]{M.A.B.~Do~Vale}$^\textrm{\scriptsize 142}$,    
\AtlasOrcid[0000-0002-6720-9883]{D.~Dodsworth}$^\textrm{\scriptsize 24}$,    
\AtlasOrcid[0000-0002-1509-0390]{C.~Doglioni}$^\textrm{\scriptsize 94}$,    
\AtlasOrcid[0000-0001-5821-7067]{J.~Dolejsi}$^\textrm{\scriptsize 137}$,    
\AtlasOrcid[0000-0002-5662-3675]{Z.~Dolezal}$^\textrm{\scriptsize 137}$,    
\AtlasOrcid[0000-0001-8329-4240]{M.~Donadelli}$^\textrm{\scriptsize 78c}$,    
\AtlasOrcid[0000-0002-6075-0191]{B.~Dong}$^\textrm{\scriptsize 58c}$,    
\AtlasOrcid[0000-0002-8998-0839]{J.~Donini}$^\textrm{\scriptsize 36}$,    
\AtlasOrcid[0000-0002-0343-6331]{A.~D'onofrio}$^\textrm{\scriptsize 13c}$,    
\AtlasOrcid[0000-0003-2408-5099]{M.~D'Onofrio}$^\textrm{\scriptsize 88}$,    
\AtlasOrcid[0000-0002-0683-9910]{J.~Dopke}$^\textrm{\scriptsize 138}$,    
\AtlasOrcid[0000-0002-5381-2649]{A.~Doria}$^\textrm{\scriptsize 67a}$,    
\AtlasOrcid[0000-0001-6113-0878]{M.T.~Dova}$^\textrm{\scriptsize 86}$,    
\AtlasOrcid[0000-0001-6322-6195]{A.T.~Doyle}$^\textrm{\scriptsize 55}$,    
\AtlasOrcid[0000-0002-8773-7640]{E.~Drechsler}$^\textrm{\scriptsize 147}$,    
\AtlasOrcid[0000-0001-8955-9510]{E.~Dreyer}$^\textrm{\scriptsize 147}$,    
\AtlasOrcid[0000-0002-7465-7887]{T.~Dreyer}$^\textrm{\scriptsize 51}$,    
\AtlasOrcid[0000-0003-4782-4034]{A.S.~Drobac}$^\textrm{\scriptsize 164}$,    
\AtlasOrcid[0000-0002-6758-0113]{D.~Du}$^\textrm{\scriptsize 58a}$,    
\AtlasOrcid[0000-0001-8703-7938]{T.A.~du~Pree}$^\textrm{\scriptsize 115}$,    
\AtlasOrcid[0000-0003-2182-2727]{F.~Dubinin}$^\textrm{\scriptsize 107}$,    
\AtlasOrcid[0000-0002-3847-0775]{M.~Dubovsky}$^\textrm{\scriptsize 26a}$,    
\AtlasOrcid[0000-0001-6161-8793]{A.~Dubreuil}$^\textrm{\scriptsize 52}$,    
\AtlasOrcid[0000-0002-7276-6342]{E.~Duchovni}$^\textrm{\scriptsize 174}$,    
\AtlasOrcid[0000-0002-7756-7801]{G.~Duckeck}$^\textrm{\scriptsize 110}$,    
\AtlasOrcid[0000-0001-5914-0524]{O.A.~Ducu}$^\textrm{\scriptsize 34,25b}$,    
\AtlasOrcid[0000-0002-5916-3467]{D.~Duda}$^\textrm{\scriptsize 111}$,    
\AtlasOrcid[0000-0002-8713-8162]{A.~Dudarev}$^\textrm{\scriptsize 34}$,    
\AtlasOrcid[0000-0003-2499-1649]{M.~D'uffizi}$^\textrm{\scriptsize 97}$,    
\AtlasOrcid[0000-0002-4871-2176]{L.~Duflot}$^\textrm{\scriptsize 62}$,    
\AtlasOrcid[0000-0002-5833-7058]{M.~D\"uhrssen}$^\textrm{\scriptsize 34}$,    
\AtlasOrcid[0000-0003-4813-8757]{C.~D{\"u}lsen}$^\textrm{\scriptsize 176}$,    
\AtlasOrcid[0000-0003-3310-4642]{A.E.~Dumitriu}$^\textrm{\scriptsize 25b}$,    
\AtlasOrcid[0000-0002-7667-260X]{M.~Dunford}$^\textrm{\scriptsize 59a}$,    
\AtlasOrcid[0000-0001-9935-6397]{S.~Dungs}$^\textrm{\scriptsize 45}$,    
\AtlasOrcid[0000-0003-2626-2247]{K.~Dunne}$^\textrm{\scriptsize 43a,43b}$,    
\AtlasOrcid[0000-0002-5789-9825]{A.~Duperrin}$^\textrm{\scriptsize 98}$,    
\AtlasOrcid[0000-0003-3469-6045]{H.~Duran~Yildiz}$^\textrm{\scriptsize 3a}$,    
\AtlasOrcid[0000-0002-6066-4744]{M.~D\"uren}$^\textrm{\scriptsize 54}$,    
\AtlasOrcid[0000-0003-4157-592X]{A.~Durglishvili}$^\textrm{\scriptsize 154b}$,    
\AtlasOrcid[0000-0001-7277-0440]{B.~Dutta}$^\textrm{\scriptsize 44}$,    
\AtlasOrcid[0000-0003-1464-0335]{G.I.~Dyckes}$^\textrm{\scriptsize 16}$,    
\AtlasOrcid[0000-0001-9632-6352]{M.~Dyndal}$^\textrm{\scriptsize 81a}$,    
\AtlasOrcid[0000-0002-7412-9187]{S.~Dysch}$^\textrm{\scriptsize 97}$,    
\AtlasOrcid[0000-0002-0805-9184]{B.S.~Dziedzic}$^\textrm{\scriptsize 82}$,    
\AtlasOrcid[0000-0003-0336-3723]{B.~Eckerova}$^\textrm{\scriptsize 26a}$,    
\AtlasOrcid{M.G.~Eggleston}$^\textrm{\scriptsize 47}$,    
\AtlasOrcid[0000-0001-5370-8377]{E.~Egidio~Purcino~De~Souza}$^\textrm{\scriptsize 78b}$,    
\AtlasOrcid[0000-0002-2701-968X]{L.F.~Ehrke}$^\textrm{\scriptsize 52}$,    
\AtlasOrcid[0000-0002-7535-6058]{T.~Eifert}$^\textrm{\scriptsize 7}$,    
\AtlasOrcid[0000-0003-3529-5171]{G.~Eigen}$^\textrm{\scriptsize 15}$,    
\AtlasOrcid[0000-0002-4391-9100]{K.~Einsweiler}$^\textrm{\scriptsize 16}$,    
\AtlasOrcid[0000-0002-7341-9115]{T.~Ekelof}$^\textrm{\scriptsize 166}$,    
\AtlasOrcid[0000-0001-9172-2946]{Y.~El~Ghazali}$^\textrm{\scriptsize 33b}$,    
\AtlasOrcid[0000-0002-8955-9681]{H.~El~Jarrari}$^\textrm{\scriptsize 33e}$,    
\AtlasOrcid[0000-0002-9669-5374]{A.~El~Moussaouy}$^\textrm{\scriptsize 33a}$,    
\AtlasOrcid[0000-0001-5997-3569]{V.~Ellajosyula}$^\textrm{\scriptsize 166}$,    
\AtlasOrcid[0000-0001-5265-3175]{M.~Ellert}$^\textrm{\scriptsize 166}$,    
\AtlasOrcid[0000-0003-3596-5331]{F.~Ellinghaus}$^\textrm{\scriptsize 176}$,    
\AtlasOrcid[0000-0003-0921-0314]{A.A.~Elliot}$^\textrm{\scriptsize 90}$,    
\AtlasOrcid[0000-0002-1920-4930]{N.~Ellis}$^\textrm{\scriptsize 34}$,    
\AtlasOrcid[0000-0001-8899-051X]{J.~Elmsheuser}$^\textrm{\scriptsize 27}$,    
\AtlasOrcid[0000-0002-1213-0545]{M.~Elsing}$^\textrm{\scriptsize 34}$,    
\AtlasOrcid[0000-0002-1363-9175]{D.~Emeliyanov}$^\textrm{\scriptsize 138}$,    
\AtlasOrcid[0000-0003-4963-1148]{A.~Emerman}$^\textrm{\scriptsize 37}$,    
\AtlasOrcid[0000-0002-9916-3349]{Y.~Enari}$^\textrm{\scriptsize 158}$,    
\AtlasOrcid[0000-0002-8073-2740]{J.~Erdmann}$^\textrm{\scriptsize 45}$,    
\AtlasOrcid[0000-0002-5423-8079]{A.~Ereditato}$^\textrm{\scriptsize 18}$,    
\AtlasOrcid[0000-0003-4543-6599]{P.A.~Erland}$^\textrm{\scriptsize 82}$,    
\AtlasOrcid[0000-0003-4656-3936]{M.~Errenst}$^\textrm{\scriptsize 176}$,    
\AtlasOrcid[0000-0003-4270-2775]{M.~Escalier}$^\textrm{\scriptsize 62}$,    
\AtlasOrcid[0000-0003-4442-4537]{C.~Escobar}$^\textrm{\scriptsize 168}$,    
\AtlasOrcid[0000-0001-8210-1064]{O.~Estrada~Pastor}$^\textrm{\scriptsize 168}$,    
\AtlasOrcid[0000-0001-6871-7794]{E.~Etzion}$^\textrm{\scriptsize 156}$,    
\AtlasOrcid[0000-0003-0434-6925]{G.~Evans}$^\textrm{\scriptsize 134a}$,    
\AtlasOrcid[0000-0003-2183-3127]{H.~Evans}$^\textrm{\scriptsize 63}$,    
\AtlasOrcid[0000-0002-4259-018X]{M.O.~Evans}$^\textrm{\scriptsize 151}$,    
\AtlasOrcid[0000-0002-7520-293X]{A.~Ezhilov}$^\textrm{\scriptsize 132}$,    
\AtlasOrcid[0000-0001-8474-0978]{F.~Fabbri}$^\textrm{\scriptsize 55}$,    
\AtlasOrcid[0000-0002-4002-8353]{L.~Fabbri}$^\textrm{\scriptsize 21b,21a}$,    
\AtlasOrcid[0000-0002-4056-4578]{G.~Facini}$^\textrm{\scriptsize 172}$,    
\AtlasOrcid[0000-0003-0154-4328]{V.~Fadeyev}$^\textrm{\scriptsize 140}$,    
\AtlasOrcid[0000-0001-7882-2125]{R.M.~Fakhrutdinov}$^\textrm{\scriptsize 118}$,    
\AtlasOrcid[0000-0002-7118-341X]{S.~Falciano}$^\textrm{\scriptsize 70a}$,    
\AtlasOrcid[0000-0002-2004-476X]{P.J.~Falke}$^\textrm{\scriptsize 22}$,    
\AtlasOrcid[0000-0002-0264-1632]{S.~Falke}$^\textrm{\scriptsize 34}$,    
\AtlasOrcid[0000-0003-4278-7182]{J.~Faltova}$^\textrm{\scriptsize 137}$,    
\AtlasOrcid[0000-0001-7868-3858]{Y.~Fan}$^\textrm{\scriptsize 13a}$,    
\AtlasOrcid[0000-0001-5140-0731]{Y.~Fang}$^\textrm{\scriptsize 13a}$,    
\AtlasOrcid[0000-0001-8630-6585]{Y.~Fang}$^\textrm{\scriptsize 13a}$,    
\AtlasOrcid[0000-0001-6689-4957]{G.~Fanourakis}$^\textrm{\scriptsize 42}$,    
\AtlasOrcid[0000-0002-8773-145X]{M.~Fanti}$^\textrm{\scriptsize 66a,66b}$,    
\AtlasOrcid[0000-0001-9442-7598]{M.~Faraj}$^\textrm{\scriptsize 58c}$,    
\AtlasOrcid[0000-0003-0000-2439]{A.~Farbin}$^\textrm{\scriptsize 7}$,    
\AtlasOrcid[0000-0002-3983-0728]{A.~Farilla}$^\textrm{\scriptsize 72a}$,    
\AtlasOrcid[0000-0003-3037-9288]{E.M.~Farina}$^\textrm{\scriptsize 68a,68b}$,    
\AtlasOrcid[0000-0003-1363-9324]{T.~Farooque}$^\textrm{\scriptsize 103}$,    
\AtlasOrcid[0000-0001-5350-9271]{S.M.~Farrington}$^\textrm{\scriptsize 48}$,    
\AtlasOrcid[0000-0002-4779-5432]{P.~Farthouat}$^\textrm{\scriptsize 34}$,    
\AtlasOrcid[0000-0002-6423-7213]{F.~Fassi}$^\textrm{\scriptsize 33e}$,    
\AtlasOrcid[0000-0003-1289-2141]{D.~Fassouliotis}$^\textrm{\scriptsize 8}$,    
\AtlasOrcid[0000-0003-3731-820X]{M.~Faucci~Giannelli}$^\textrm{\scriptsize 71a,71b}$,    
\AtlasOrcid[0000-0003-2596-8264]{W.J.~Fawcett}$^\textrm{\scriptsize 30}$,    
\AtlasOrcid[0000-0002-2190-9091]{L.~Fayard}$^\textrm{\scriptsize 62}$,    
\AtlasOrcid[0000-0002-1733-7158]{O.L.~Fedin}$^\textrm{\scriptsize 132,p}$,    
\AtlasOrcid[0000-0003-4124-7862]{M.~Feickert}$^\textrm{\scriptsize 167}$,    
\AtlasOrcid[0000-0002-1403-0951]{L.~Feligioni}$^\textrm{\scriptsize 98}$,    
\AtlasOrcid[0000-0003-2101-1879]{A.~Fell}$^\textrm{\scriptsize 144}$,    
\AtlasOrcid[0000-0001-9138-3200]{C.~Feng}$^\textrm{\scriptsize 58b}$,    
\AtlasOrcid[0000-0002-0698-1482]{M.~Feng}$^\textrm{\scriptsize 13b}$,    
\AtlasOrcid[0000-0003-1002-6880]{M.J.~Fenton}$^\textrm{\scriptsize 165}$,    
\AtlasOrcid{A.B.~Fenyuk}$^\textrm{\scriptsize 118}$,    
\AtlasOrcid[0000-0003-1328-4367]{S.W.~Ferguson}$^\textrm{\scriptsize 41}$,    
\AtlasOrcid[0000-0002-1007-7816]{J.~Ferrando}$^\textrm{\scriptsize 44}$,    
\AtlasOrcid[0000-0003-2887-5311]{A.~Ferrari}$^\textrm{\scriptsize 166}$,    
\AtlasOrcid[0000-0002-1387-153X]{P.~Ferrari}$^\textrm{\scriptsize 115}$,    
\AtlasOrcid[0000-0001-5566-1373]{R.~Ferrari}$^\textrm{\scriptsize 68a}$,    
\AtlasOrcid[0000-0002-5687-9240]{D.~Ferrere}$^\textrm{\scriptsize 52}$,    
\AtlasOrcid[0000-0002-5562-7893]{C.~Ferretti}$^\textrm{\scriptsize 102}$,    
\AtlasOrcid[0000-0002-4610-5612]{F.~Fiedler}$^\textrm{\scriptsize 96}$,    
\AtlasOrcid[0000-0001-5671-1555]{A.~Filip\v{c}i\v{c}}$^\textrm{\scriptsize 89}$,    
\AtlasOrcid[0000-0003-3338-2247]{F.~Filthaut}$^\textrm{\scriptsize 114}$,    
\AtlasOrcid[0000-0001-9035-0335]{M.C.N.~Fiolhais}$^\textrm{\scriptsize 134a,134c,a}$,    
\AtlasOrcid[0000-0002-5070-2735]{L.~Fiorini}$^\textrm{\scriptsize 168}$,    
\AtlasOrcid[0000-0001-9799-5232]{F.~Fischer}$^\textrm{\scriptsize 146}$,    
\AtlasOrcid[0000-0003-3043-3045]{W.C.~Fisher}$^\textrm{\scriptsize 103}$,    
\AtlasOrcid[0000-0002-1152-7372]{T.~Fitschen}$^\textrm{\scriptsize 19}$,    
\AtlasOrcid[0000-0003-1461-8648]{I.~Fleck}$^\textrm{\scriptsize 146}$,    
\AtlasOrcid[0000-0001-6968-340X]{P.~Fleischmann}$^\textrm{\scriptsize 102}$,    
\AtlasOrcid[0000-0002-8356-6987]{T.~Flick}$^\textrm{\scriptsize 176}$,    
\AtlasOrcid[0000-0002-1098-6446]{B.M.~Flierl}$^\textrm{\scriptsize 110}$,    
\AtlasOrcid[0000-0002-2748-758X]{L.~Flores}$^\textrm{\scriptsize 131}$,    
\AtlasOrcid[0000-0002-4462-2851]{M.~Flores}$^\textrm{\scriptsize 31f}$,    
\AtlasOrcid[0000-0003-1551-5974]{L.R.~Flores~Castillo}$^\textrm{\scriptsize 60a}$,    
\AtlasOrcid[0000-0003-2317-9560]{F.M.~Follega}$^\textrm{\scriptsize 73a,73b}$,    
\AtlasOrcid[0000-0001-9457-394X]{N.~Fomin}$^\textrm{\scriptsize 15}$,    
\AtlasOrcid[0000-0003-4577-0685]{J.H.~Foo}$^\textrm{\scriptsize 161}$,    
\AtlasOrcid{B.C.~Forland}$^\textrm{\scriptsize 63}$,    
\AtlasOrcid[0000-0001-8308-2643]{A.~Formica}$^\textrm{\scriptsize 139}$,    
\AtlasOrcid[0000-0002-3727-8781]{F.A.~F\"orster}$^\textrm{\scriptsize 12}$,    
\AtlasOrcid[0000-0002-0532-7921]{A.C.~Forti}$^\textrm{\scriptsize 97}$,    
\AtlasOrcid{E.~Fortin}$^\textrm{\scriptsize 98}$,    
\AtlasOrcid[0000-0002-0976-7246]{M.G.~Foti}$^\textrm{\scriptsize 129}$,    
\AtlasOrcid[0000-0002-9986-6597]{L.~Fountas}$^\textrm{\scriptsize 8}$,    
\AtlasOrcid[0000-0003-4836-0358]{D.~Fournier}$^\textrm{\scriptsize 62}$,    
\AtlasOrcid[0000-0003-3089-6090]{H.~Fox}$^\textrm{\scriptsize 87}$,    
\AtlasOrcid[0000-0003-1164-6870]{P.~Francavilla}$^\textrm{\scriptsize 69a,69b}$,    
\AtlasOrcid[0000-0001-5315-9275]{S.~Francescato}$^\textrm{\scriptsize 57}$,    
\AtlasOrcid[0000-0002-4554-252X]{M.~Franchini}$^\textrm{\scriptsize 21b,21a}$,    
\AtlasOrcid[0000-0002-8159-8010]{S.~Franchino}$^\textrm{\scriptsize 59a}$,    
\AtlasOrcid{D.~Francis}$^\textrm{\scriptsize 34}$,    
\AtlasOrcid[0000-0002-1687-4314]{L.~Franco}$^\textrm{\scriptsize 4}$,    
\AtlasOrcid[0000-0002-0647-6072]{L.~Franconi}$^\textrm{\scriptsize 18}$,    
\AtlasOrcid[0000-0002-6595-883X]{M.~Franklin}$^\textrm{\scriptsize 57}$,    
\AtlasOrcid[0000-0002-7829-6564]{G.~Frattari}$^\textrm{\scriptsize 70a,70b}$,    
\AtlasOrcid[0000-0003-4482-3001]{A.C.~Freegard}$^\textrm{\scriptsize 90}$,    
\AtlasOrcid{P.M.~Freeman}$^\textrm{\scriptsize 19}$,    
\AtlasOrcid[0000-0003-4473-1027]{W.S.~Freund}$^\textrm{\scriptsize 78b}$,    
\AtlasOrcid[0000-0003-0907-392X]{E.M.~Freundlich}$^\textrm{\scriptsize 45}$,    
\AtlasOrcid[0000-0003-3986-3922]{D.~Froidevaux}$^\textrm{\scriptsize 34}$,    
\AtlasOrcid[0000-0003-3562-9944]{J.A.~Frost}$^\textrm{\scriptsize 129}$,    
\AtlasOrcid[0000-0002-7370-7395]{Y.~Fu}$^\textrm{\scriptsize 58a}$,    
\AtlasOrcid[0000-0002-6701-8198]{M.~Fujimoto}$^\textrm{\scriptsize 121}$,    
\AtlasOrcid[0000-0003-3082-621X]{E.~Fullana~Torregrosa}$^\textrm{\scriptsize 168}$,    
\AtlasOrcid[0000-0002-1290-2031]{J.~Fuster}$^\textrm{\scriptsize 168}$,    
\AtlasOrcid[0000-0001-5346-7841]{A.~Gabrielli}$^\textrm{\scriptsize 21b,21a}$,    
\AtlasOrcid[0000-0003-0768-9325]{A.~Gabrielli}$^\textrm{\scriptsize 34}$,    
\AtlasOrcid[0000-0003-4475-6734]{P.~Gadow}$^\textrm{\scriptsize 44}$,    
\AtlasOrcid[0000-0002-3550-4124]{G.~Gagliardi}$^\textrm{\scriptsize 53b,53a}$,    
\AtlasOrcid[0000-0003-3000-8479]{L.G.~Gagnon}$^\textrm{\scriptsize 16}$,    
\AtlasOrcid[0000-0001-5832-5746]{G.E.~Gallardo}$^\textrm{\scriptsize 129}$,    
\AtlasOrcid[0000-0002-1259-1034]{E.J.~Gallas}$^\textrm{\scriptsize 129}$,    
\AtlasOrcid[0000-0001-7401-5043]{B.J.~Gallop}$^\textrm{\scriptsize 138}$,    
\AtlasOrcid[0000-0003-1026-7633]{R.~Gamboa~Goni}$^\textrm{\scriptsize 90}$,    
\AtlasOrcid[0000-0002-1550-1487]{K.K.~Gan}$^\textrm{\scriptsize 122}$,    
\AtlasOrcid[0000-0003-1285-9261]{S.~Ganguly}$^\textrm{\scriptsize 174}$,    
\AtlasOrcid[0000-0002-8420-3803]{J.~Gao}$^\textrm{\scriptsize 58a}$,    
\AtlasOrcid[0000-0001-6326-4773]{Y.~Gao}$^\textrm{\scriptsize 48}$,    
\AtlasOrcid[0000-0002-6082-9190]{Y.S.~Gao}$^\textrm{\scriptsize 29,m}$,    
\AtlasOrcid[0000-0002-6670-1104]{F.M.~Garay~Walls}$^\textrm{\scriptsize 141a}$,    
\AtlasOrcid[0000-0003-1625-7452]{C.~Garc\'ia}$^\textrm{\scriptsize 168}$,    
\AtlasOrcid[0000-0002-0279-0523]{J.E.~Garc\'ia~Navarro}$^\textrm{\scriptsize 168}$,    
\AtlasOrcid[0000-0002-7399-7353]{J.A.~Garc\'ia~Pascual}$^\textrm{\scriptsize 13a}$,    
\AtlasOrcid[0000-0002-5800-4210]{M.~Garcia-Sciveres}$^\textrm{\scriptsize 16}$,    
\AtlasOrcid[0000-0003-1433-9366]{R.W.~Gardner}$^\textrm{\scriptsize 35}$,    
\AtlasOrcid[0000-0001-8383-9343]{D.~Garg}$^\textrm{\scriptsize 75}$,    
\AtlasOrcid[0000-0002-2691-7963]{R.B.~Garg}$^\textrm{\scriptsize 148}$,    
\AtlasOrcid[0000-0003-4850-1122]{S.~Gargiulo}$^\textrm{\scriptsize 50}$,    
\AtlasOrcid{C.A.~Garner}$^\textrm{\scriptsize 161}$,    
\AtlasOrcid[0000-0001-7169-9160]{V.~Garonne}$^\textrm{\scriptsize 128}$,    
\AtlasOrcid[0000-0002-4067-2472]{S.J.~Gasiorowski}$^\textrm{\scriptsize 143}$,    
\AtlasOrcid[0000-0002-9232-1332]{P.~Gaspar}$^\textrm{\scriptsize 78b}$,    
\AtlasOrcid[0000-0002-6833-0933]{G.~Gaudio}$^\textrm{\scriptsize 68a}$,    
\AtlasOrcid[0000-0003-4841-5822]{P.~Gauzzi}$^\textrm{\scriptsize 70a,70b}$,    
\AtlasOrcid[0000-0001-7219-2636]{I.L.~Gavrilenko}$^\textrm{\scriptsize 107}$,    
\AtlasOrcid[0000-0003-3837-6567]{A.~Gavrilyuk}$^\textrm{\scriptsize 119}$,    
\AtlasOrcid[0000-0002-9354-9507]{C.~Gay}$^\textrm{\scriptsize 169}$,    
\AtlasOrcid[0000-0002-2941-9257]{G.~Gaycken}$^\textrm{\scriptsize 44}$,    
\AtlasOrcid[0000-0002-9272-4254]{E.N.~Gazis}$^\textrm{\scriptsize 9}$,    
\AtlasOrcid[0000-0003-2781-2933]{A.A.~Geanta}$^\textrm{\scriptsize 25b}$,    
\AtlasOrcid[0000-0002-3271-7861]{C.M.~Gee}$^\textrm{\scriptsize 140}$,    
\AtlasOrcid[0000-0002-8833-3154]{C.N.P.~Gee}$^\textrm{\scriptsize 138}$,    
\AtlasOrcid[0000-0003-4644-2472]{J.~Geisen}$^\textrm{\scriptsize 94}$,    
\AtlasOrcid[0000-0003-0932-0230]{M.~Geisen}$^\textrm{\scriptsize 96}$,    
\AtlasOrcid[0000-0002-1702-5699]{C.~Gemme}$^\textrm{\scriptsize 53b}$,    
\AtlasOrcid[0000-0002-4098-2024]{M.H.~Genest}$^\textrm{\scriptsize 56}$,    
\AtlasOrcid[0000-0003-4550-7174]{S.~Gentile}$^\textrm{\scriptsize 70a,70b}$,    
\AtlasOrcid[0000-0003-3565-3290]{S.~George}$^\textrm{\scriptsize 91}$,    
\AtlasOrcid[0000-0003-3674-7475]{W.F.~George}$^\textrm{\scriptsize 19}$,    
\AtlasOrcid[0000-0001-7188-979X]{T.~Geralis}$^\textrm{\scriptsize 42}$,    
\AtlasOrcid{L.O.~Gerlach}$^\textrm{\scriptsize 51}$,    
\AtlasOrcid[0000-0002-3056-7417]{P.~Gessinger-Befurt}$^\textrm{\scriptsize 34}$,    
\AtlasOrcid[0000-0003-3492-4538]{M.~Ghasemi~Bostanabad}$^\textrm{\scriptsize 170}$,    
\AtlasOrcid[0000-0002-4931-2764]{M.~Ghneimat}$^\textrm{\scriptsize 146}$,    
\AtlasOrcid[0000-0003-0819-1553]{A.~Ghosh}$^\textrm{\scriptsize 165}$,    
\AtlasOrcid[0000-0002-5716-356X]{A.~Ghosh}$^\textrm{\scriptsize 75}$,    
\AtlasOrcid[0000-0003-2987-7642]{B.~Giacobbe}$^\textrm{\scriptsize 21b}$,    
\AtlasOrcid[0000-0001-9192-3537]{S.~Giagu}$^\textrm{\scriptsize 70a,70b}$,    
\AtlasOrcid[0000-0001-7314-0168]{N.~Giangiacomi}$^\textrm{\scriptsize 161}$,    
\AtlasOrcid[0000-0002-3721-9490]{P.~Giannetti}$^\textrm{\scriptsize 69a}$,    
\AtlasOrcid[0000-0002-5683-814X]{A.~Giannini}$^\textrm{\scriptsize 67a,67b}$,    
\AtlasOrcid[0000-0002-1236-9249]{S.M.~Gibson}$^\textrm{\scriptsize 91}$,    
\AtlasOrcid[0000-0003-4155-7844]{M.~Gignac}$^\textrm{\scriptsize 140}$,    
\AtlasOrcid[0000-0001-9021-8836]{D.T.~Gil}$^\textrm{\scriptsize 81b}$,    
\AtlasOrcid[0000-0003-0731-710X]{B.J.~Gilbert}$^\textrm{\scriptsize 37}$,    
\AtlasOrcid[0000-0003-0341-0171]{D.~Gillberg}$^\textrm{\scriptsize 32}$,    
\AtlasOrcid[0000-0001-8451-4604]{G.~Gilles}$^\textrm{\scriptsize 115}$,    
\AtlasOrcid[0000-0003-0848-329X]{N.E.K.~Gillwald}$^\textrm{\scriptsize 44}$,    
\AtlasOrcid[0000-0002-2552-1449]{D.M.~Gingrich}$^\textrm{\scriptsize 2,aj}$,    
\AtlasOrcid[0000-0002-0792-6039]{M.P.~Giordani}$^\textrm{\scriptsize 64a,64c}$,    
\AtlasOrcid[0000-0002-8485-9351]{P.F.~Giraud}$^\textrm{\scriptsize 139}$,    
\AtlasOrcid[0000-0001-5765-1750]{G.~Giugliarelli}$^\textrm{\scriptsize 64a,64c}$,    
\AtlasOrcid[0000-0002-6976-0951]{D.~Giugni}$^\textrm{\scriptsize 66a}$,    
\AtlasOrcid[0000-0002-8506-274X]{F.~Giuli}$^\textrm{\scriptsize 71a,71b}$,    
\AtlasOrcid[0000-0002-8402-723X]{I.~Gkialas}$^\textrm{\scriptsize 8,h}$,    
\AtlasOrcid[0000-0003-2331-9922]{P.~Gkountoumis}$^\textrm{\scriptsize 9}$,    
\AtlasOrcid[0000-0001-9422-8636]{L.K.~Gladilin}$^\textrm{\scriptsize 109}$,    
\AtlasOrcid[0000-0003-2025-3817]{C.~Glasman}$^\textrm{\scriptsize 95}$,    
\AtlasOrcid[0000-0001-7701-5030]{G.R.~Gledhill}$^\textrm{\scriptsize 126}$,    
\AtlasOrcid{M.~Glisic}$^\textrm{\scriptsize 126}$,    
\AtlasOrcid[0000-0002-0772-7312]{I.~Gnesi}$^\textrm{\scriptsize 39b,d}$,    
\AtlasOrcid[0000-0002-2785-9654]{M.~Goblirsch-Kolb}$^\textrm{\scriptsize 24}$,    
\AtlasOrcid{D.~Godin}$^\textrm{\scriptsize 106}$,    
\AtlasOrcid[0000-0002-1677-3097]{S.~Goldfarb}$^\textrm{\scriptsize 101}$,    
\AtlasOrcid[0000-0001-8535-6687]{T.~Golling}$^\textrm{\scriptsize 52}$,    
\AtlasOrcid[0000-0002-5521-9793]{D.~Golubkov}$^\textrm{\scriptsize 118}$,    
\AtlasOrcid[0000-0002-8285-3570]{J.P.~Gombas}$^\textrm{\scriptsize 103}$,    
\AtlasOrcid[0000-0002-5940-9893]{A.~Gomes}$^\textrm{\scriptsize 134a,134b}$,    
\AtlasOrcid[0000-0002-8263-4263]{R.~Goncalves~Gama}$^\textrm{\scriptsize 51}$,    
\AtlasOrcid[0000-0002-3826-3442]{R.~Gon\c{c}alo}$^\textrm{\scriptsize 134a,134c}$,    
\AtlasOrcid[0000-0002-0524-2477]{G.~Gonella}$^\textrm{\scriptsize 126}$,    
\AtlasOrcid[0000-0002-4919-0808]{L.~Gonella}$^\textrm{\scriptsize 19}$,    
\AtlasOrcid[0000-0001-8183-1612]{A.~Gongadze}$^\textrm{\scriptsize 77}$,    
\AtlasOrcid[0000-0003-0885-1654]{F.~Gonnella}$^\textrm{\scriptsize 19}$,    
\AtlasOrcid[0000-0003-2037-6315]{J.L.~Gonski}$^\textrm{\scriptsize 37}$,    
\AtlasOrcid[0000-0001-5304-5390]{S.~Gonz\'alez~de~la~Hoz}$^\textrm{\scriptsize 168}$,    
\AtlasOrcid[0000-0001-8176-0201]{S.~Gonzalez~Fernandez}$^\textrm{\scriptsize 12}$,    
\AtlasOrcid[0000-0003-2302-8754]{R.~Gonzalez~Lopez}$^\textrm{\scriptsize 88}$,    
\AtlasOrcid[0000-0003-0079-8924]{C.~Gonzalez~Renteria}$^\textrm{\scriptsize 16}$,    
\AtlasOrcid[0000-0002-6126-7230]{R.~Gonzalez~Suarez}$^\textrm{\scriptsize 166}$,    
\AtlasOrcid[0000-0003-4458-9403]{S.~Gonzalez-Sevilla}$^\textrm{\scriptsize 52}$,    
\AtlasOrcid[0000-0002-6816-4795]{G.R.~Gonzalvo~Rodriguez}$^\textrm{\scriptsize 168}$,    
\AtlasOrcid[0000-0002-0700-1757]{R.Y.~González~Andana}$^\textrm{\scriptsize 141a}$,    
\AtlasOrcid[0000-0002-2536-4498]{L.~Goossens}$^\textrm{\scriptsize 34}$,    
\AtlasOrcid[0000-0002-7152-363X]{N.A.~Gorasia}$^\textrm{\scriptsize 19}$,    
\AtlasOrcid[0000-0001-9135-1516]{P.A.~Gorbounov}$^\textrm{\scriptsize 119}$,    
\AtlasOrcid[0000-0003-4362-019X]{H.A.~Gordon}$^\textrm{\scriptsize 27}$,    
\AtlasOrcid[0000-0003-4177-9666]{B.~Gorini}$^\textrm{\scriptsize 34}$,    
\AtlasOrcid[0000-0002-7688-2797]{E.~Gorini}$^\textrm{\scriptsize 65a,65b}$,    
\AtlasOrcid[0000-0002-3903-3438]{A.~Gori\v{s}ek}$^\textrm{\scriptsize 89}$,    
\AtlasOrcid[0000-0002-5704-0885]{A.T.~Goshaw}$^\textrm{\scriptsize 47}$,    
\AtlasOrcid[0000-0002-4311-3756]{M.I.~Gostkin}$^\textrm{\scriptsize 77}$,    
\AtlasOrcid[0000-0003-0348-0364]{C.A.~Gottardo}$^\textrm{\scriptsize 114}$,    
\AtlasOrcid[0000-0002-9551-0251]{M.~Gouighri}$^\textrm{\scriptsize 33b}$,    
\AtlasOrcid[0000-0002-1294-9091]{V.~Goumarre}$^\textrm{\scriptsize 44}$,    
\AtlasOrcid[0000-0001-6211-7122]{A.G.~Goussiou}$^\textrm{\scriptsize 143}$,    
\AtlasOrcid[0000-0002-5068-5429]{N.~Govender}$^\textrm{\scriptsize 31c}$,    
\AtlasOrcid[0000-0002-1297-8925]{C.~Goy}$^\textrm{\scriptsize 4}$,    
\AtlasOrcid[0000-0001-9159-1210]{I.~Grabowska-Bold}$^\textrm{\scriptsize 81a}$,    
\AtlasOrcid[0000-0002-5832-8653]{K.~Graham}$^\textrm{\scriptsize 32}$,    
\AtlasOrcid[0000-0001-5792-5352]{E.~Gramstad}$^\textrm{\scriptsize 128}$,    
\AtlasOrcid[0000-0001-8490-8304]{S.~Grancagnolo}$^\textrm{\scriptsize 17}$,    
\AtlasOrcid[0000-0002-5924-2544]{M.~Grandi}$^\textrm{\scriptsize 151}$,    
\AtlasOrcid{V.~Gratchev}$^\textrm{\scriptsize 132}$,    
\AtlasOrcid[0000-0002-0154-577X]{P.M.~Gravila}$^\textrm{\scriptsize 25f}$,    
\AtlasOrcid[0000-0003-2422-5960]{F.G.~Gravili}$^\textrm{\scriptsize 65a,65b}$,    
\AtlasOrcid[0000-0002-5293-4716]{H.M.~Gray}$^\textrm{\scriptsize 16}$,    
\AtlasOrcid[0000-0001-7050-5301]{C.~Grefe}$^\textrm{\scriptsize 22}$,    
\AtlasOrcid[0000-0002-5976-7818]{I.M.~Gregor}$^\textrm{\scriptsize 44}$,    
\AtlasOrcid[0000-0002-9926-5417]{P.~Grenier}$^\textrm{\scriptsize 148}$,    
\AtlasOrcid[0000-0003-2704-6028]{K.~Grevtsov}$^\textrm{\scriptsize 44}$,    
\AtlasOrcid[0000-0002-3955-4399]{C.~Grieco}$^\textrm{\scriptsize 12}$,    
\AtlasOrcid{N.A.~Grieser}$^\textrm{\scriptsize 123}$,    
\AtlasOrcid{A.A.~Grillo}$^\textrm{\scriptsize 140}$,    
\AtlasOrcid[0000-0001-6587-7397]{K.~Grimm}$^\textrm{\scriptsize 29,l}$,    
\AtlasOrcid[0000-0002-6460-8694]{S.~Grinstein}$^\textrm{\scriptsize 12,v}$,    
\AtlasOrcid[0000-0003-4793-7995]{J.-F.~Grivaz}$^\textrm{\scriptsize 62}$,    
\AtlasOrcid[0000-0002-3001-3545]{S.~Groh}$^\textrm{\scriptsize 96}$,    
\AtlasOrcid[0000-0003-1244-9350]{E.~Gross}$^\textrm{\scriptsize 174}$,    
\AtlasOrcid[0000-0003-3085-7067]{J.~Grosse-Knetter}$^\textrm{\scriptsize 51}$,    
\AtlasOrcid{C.~Grud}$^\textrm{\scriptsize 102}$,    
\AtlasOrcid[0000-0003-2752-1183]{A.~Grummer}$^\textrm{\scriptsize 113}$,    
\AtlasOrcid[0000-0001-7136-0597]{J.C.~Grundy}$^\textrm{\scriptsize 129}$,    
\AtlasOrcid[0000-0003-1897-1617]{L.~Guan}$^\textrm{\scriptsize 102}$,    
\AtlasOrcid[0000-0002-5548-5194]{W.~Guan}$^\textrm{\scriptsize 175}$,    
\AtlasOrcid[0000-0003-2329-4219]{C.~Gubbels}$^\textrm{\scriptsize 169}$,    
\AtlasOrcid[0000-0003-3189-3959]{J.~Guenther}$^\textrm{\scriptsize 34}$,    
\AtlasOrcid[0000-0001-8487-3594]{J.G.R.~Guerrero~Rojas}$^\textrm{\scriptsize 168}$,    
\AtlasOrcid[0000-0001-5351-2673]{F.~Guescini}$^\textrm{\scriptsize 111}$,    
\AtlasOrcid[0000-0002-4305-2295]{D.~Guest}$^\textrm{\scriptsize 17}$,    
\AtlasOrcid[0000-0002-3349-1163]{R.~Gugel}$^\textrm{\scriptsize 96}$,    
\AtlasOrcid[0000-0001-9021-9038]{A.~Guida}$^\textrm{\scriptsize 44}$,    
\AtlasOrcid[0000-0001-9698-6000]{T.~Guillemin}$^\textrm{\scriptsize 4}$,    
\AtlasOrcid[0000-0001-7595-3859]{S.~Guindon}$^\textrm{\scriptsize 34}$,    
\AtlasOrcid[0000-0001-8125-9433]{J.~Guo}$^\textrm{\scriptsize 58c}$,    
\AtlasOrcid[0000-0002-6785-9202]{L.~Guo}$^\textrm{\scriptsize 62}$,    
\AtlasOrcid[0000-0002-6027-5132]{Y.~Guo}$^\textrm{\scriptsize 102}$,    
\AtlasOrcid[0000-0003-1510-3371]{R.~Gupta}$^\textrm{\scriptsize 44}$,    
\AtlasOrcid[0000-0002-9152-1455]{S.~Gurbuz}$^\textrm{\scriptsize 22}$,    
\AtlasOrcid[0000-0002-5938-4921]{G.~Gustavino}$^\textrm{\scriptsize 123}$,    
\AtlasOrcid[0000-0002-6647-1433]{M.~Guth}$^\textrm{\scriptsize 52}$,    
\AtlasOrcid[0000-0003-2326-3877]{P.~Gutierrez}$^\textrm{\scriptsize 123}$,    
\AtlasOrcid[0000-0003-0374-1595]{L.F.~Gutierrez~Zagazeta}$^\textrm{\scriptsize 131}$,    
\AtlasOrcid[0000-0003-0857-794X]{C.~Gutschow}$^\textrm{\scriptsize 92}$,    
\AtlasOrcid[0000-0002-2300-7497]{C.~Guyot}$^\textrm{\scriptsize 139}$,    
\AtlasOrcid[0000-0002-3518-0617]{C.~Gwenlan}$^\textrm{\scriptsize 129}$,    
\AtlasOrcid[0000-0002-9401-5304]{C.B.~Gwilliam}$^\textrm{\scriptsize 88}$,    
\AtlasOrcid[0000-0002-3676-493X]{E.S.~Haaland}$^\textrm{\scriptsize 128}$,    
\AtlasOrcid[0000-0002-4832-0455]{A.~Haas}$^\textrm{\scriptsize 120}$,    
\AtlasOrcid[0000-0002-7412-9355]{M.~Habedank}$^\textrm{\scriptsize 17}$,    
\AtlasOrcid[0000-0002-0155-1360]{C.~Haber}$^\textrm{\scriptsize 16}$,    
\AtlasOrcid[0000-0001-5447-3346]{H.K.~Hadavand}$^\textrm{\scriptsize 7}$,    
\AtlasOrcid[0000-0003-2508-0628]{A.~Hadef}$^\textrm{\scriptsize 96}$,    
\AtlasOrcid[0000-0002-8875-8523]{S.~Hadzic}$^\textrm{\scriptsize 111}$,    
\AtlasOrcid[0000-0003-3826-6333]{M.~Haleem}$^\textrm{\scriptsize 171}$,    
\AtlasOrcid[0000-0002-6938-7405]{J.~Haley}$^\textrm{\scriptsize 124}$,    
\AtlasOrcid[0000-0002-8304-9170]{J.J.~Hall}$^\textrm{\scriptsize 144}$,    
\AtlasOrcid[0000-0001-7162-0301]{G.~Halladjian}$^\textrm{\scriptsize 103}$,    
\AtlasOrcid[0000-0001-6267-8560]{G.D.~Hallewell}$^\textrm{\scriptsize 98}$,    
\AtlasOrcid[0000-0002-0759-7247]{L.~Halser}$^\textrm{\scriptsize 18}$,    
\AtlasOrcid[0000-0002-9438-8020]{K.~Hamano}$^\textrm{\scriptsize 170}$,    
\AtlasOrcid[0000-0001-5709-2100]{H.~Hamdaoui}$^\textrm{\scriptsize 33e}$,    
\AtlasOrcid[0000-0003-1550-2030]{M.~Hamer}$^\textrm{\scriptsize 22}$,    
\AtlasOrcid[0000-0002-4537-0377]{G.N.~Hamity}$^\textrm{\scriptsize 48}$,    
\AtlasOrcid[0000-0002-1627-4810]{K.~Han}$^\textrm{\scriptsize 58a}$,    
\AtlasOrcid[0000-0003-3321-8412]{L.~Han}$^\textrm{\scriptsize 13c}$,    
\AtlasOrcid[0000-0002-6353-9711]{L.~Han}$^\textrm{\scriptsize 58a}$,    
\AtlasOrcid[0000-0001-8383-7348]{S.~Han}$^\textrm{\scriptsize 16}$,    
\AtlasOrcid[0000-0002-7084-8424]{Y.F.~Han}$^\textrm{\scriptsize 161}$,    
\AtlasOrcid[0000-0003-0676-0441]{K.~Hanagaki}$^\textrm{\scriptsize 79,t}$,    
\AtlasOrcid[0000-0001-8392-0934]{M.~Hance}$^\textrm{\scriptsize 140}$,    
\AtlasOrcid[0000-0002-4731-6120]{M.D.~Hank}$^\textrm{\scriptsize 35}$,    
\AtlasOrcid[0000-0003-4519-8949]{R.~Hankache}$^\textrm{\scriptsize 97}$,    
\AtlasOrcid[0000-0002-5019-1648]{E.~Hansen}$^\textrm{\scriptsize 94}$,    
\AtlasOrcid[0000-0002-3684-8340]{J.B.~Hansen}$^\textrm{\scriptsize 38}$,    
\AtlasOrcid[0000-0003-3102-0437]{J.D.~Hansen}$^\textrm{\scriptsize 38}$,    
\AtlasOrcid[0000-0002-8892-4552]{M.C.~Hansen}$^\textrm{\scriptsize 22}$,    
\AtlasOrcid[0000-0002-6764-4789]{P.H.~Hansen}$^\textrm{\scriptsize 38}$,    
\AtlasOrcid[0000-0003-1629-0535]{K.~Hara}$^\textrm{\scriptsize 163}$,    
\AtlasOrcid[0000-0001-8682-3734]{T.~Harenberg}$^\textrm{\scriptsize 176}$,    
\AtlasOrcid[0000-0002-0309-4490]{S.~Harkusha}$^\textrm{\scriptsize 104}$,    
\AtlasOrcid[0000-0001-5816-2158]{Y.T.~Harris}$^\textrm{\scriptsize 129}$,    
\AtlasOrcid{P.F.~Harrison}$^\textrm{\scriptsize 172}$,    
\AtlasOrcid[0000-0001-9111-4916]{N.M.~Hartman}$^\textrm{\scriptsize 148}$,    
\AtlasOrcid[0000-0003-0047-2908]{N.M.~Hartmann}$^\textrm{\scriptsize 110}$,    
\AtlasOrcid[0000-0003-2683-7389]{Y.~Hasegawa}$^\textrm{\scriptsize 145}$,    
\AtlasOrcid[0000-0003-0457-2244]{A.~Hasib}$^\textrm{\scriptsize 48}$,    
\AtlasOrcid[0000-0002-2834-5110]{S.~Hassani}$^\textrm{\scriptsize 139}$,    
\AtlasOrcid[0000-0003-0442-3361]{S.~Haug}$^\textrm{\scriptsize 18}$,    
\AtlasOrcid[0000-0001-7682-8857]{R.~Hauser}$^\textrm{\scriptsize 103}$,    
\AtlasOrcid[0000-0002-3031-3222]{M.~Havranek}$^\textrm{\scriptsize 136}$,    
\AtlasOrcid[0000-0001-9167-0592]{C.M.~Hawkes}$^\textrm{\scriptsize 19}$,    
\AtlasOrcid[0000-0001-9719-0290]{R.J.~Hawkings}$^\textrm{\scriptsize 34}$,    
\AtlasOrcid[0000-0002-5924-3803]{S.~Hayashida}$^\textrm{\scriptsize 112}$,    
\AtlasOrcid[0000-0001-5220-2972]{D.~Hayden}$^\textrm{\scriptsize 103}$,    
\AtlasOrcid[0000-0002-0298-0351]{C.~Hayes}$^\textrm{\scriptsize 102}$,    
\AtlasOrcid[0000-0001-7752-9285]{R.L.~Hayes}$^\textrm{\scriptsize 169}$,    
\AtlasOrcid[0000-0003-2371-9723]{C.P.~Hays}$^\textrm{\scriptsize 129}$,    
\AtlasOrcid[0000-0003-1554-5401]{J.M.~Hays}$^\textrm{\scriptsize 90}$,    
\AtlasOrcid[0000-0002-0972-3411]{H.S.~Hayward}$^\textrm{\scriptsize 88}$,    
\AtlasOrcid[0000-0003-2074-013X]{S.J.~Haywood}$^\textrm{\scriptsize 138}$,    
\AtlasOrcid[0000-0003-3733-4058]{F.~He}$^\textrm{\scriptsize 58a}$,    
\AtlasOrcid[0000-0002-0619-1579]{Y.~He}$^\textrm{\scriptsize 159}$,    
\AtlasOrcid[0000-0001-8068-5596]{Y.~He}$^\textrm{\scriptsize 130}$,    
\AtlasOrcid[0000-0003-2945-8448]{M.P.~Heath}$^\textrm{\scriptsize 48}$,    
\AtlasOrcid[0000-0002-4596-3965]{V.~Hedberg}$^\textrm{\scriptsize 94}$,    
\AtlasOrcid[0000-0002-7736-2806]{A.L.~Heggelund}$^\textrm{\scriptsize 128}$,    
\AtlasOrcid[0000-0003-0466-4472]{N.D.~Hehir}$^\textrm{\scriptsize 90}$,    
\AtlasOrcid[0000-0001-8821-1205]{C.~Heidegger}$^\textrm{\scriptsize 50}$,    
\AtlasOrcid[0000-0003-3113-0484]{K.K.~Heidegger}$^\textrm{\scriptsize 50}$,    
\AtlasOrcid[0000-0001-9539-6957]{W.D.~Heidorn}$^\textrm{\scriptsize 76}$,    
\AtlasOrcid[0000-0001-6792-2294]{J.~Heilman}$^\textrm{\scriptsize 32}$,    
\AtlasOrcid[0000-0002-2639-6571]{S.~Heim}$^\textrm{\scriptsize 44}$,    
\AtlasOrcid[0000-0002-7669-5318]{T.~Heim}$^\textrm{\scriptsize 16}$,    
\AtlasOrcid[0000-0002-1673-7926]{B.~Heinemann}$^\textrm{\scriptsize 44,ah}$,    
\AtlasOrcid[0000-0001-6878-9405]{J.G.~Heinlein}$^\textrm{\scriptsize 131}$,    
\AtlasOrcid[0000-0002-0253-0924]{J.J.~Heinrich}$^\textrm{\scriptsize 126}$,    
\AtlasOrcid[0000-0002-4048-7584]{L.~Heinrich}$^\textrm{\scriptsize 34}$,    
\AtlasOrcid[0000-0002-4600-3659]{J.~Hejbal}$^\textrm{\scriptsize 135}$,    
\AtlasOrcid[0000-0001-7891-8354]{L.~Helary}$^\textrm{\scriptsize 44}$,    
\AtlasOrcid[0000-0002-8924-5885]{A.~Held}$^\textrm{\scriptsize 120}$,    
\AtlasOrcid[0000-0002-2657-7532]{C.M.~Helling}$^\textrm{\scriptsize 140}$,    
\AtlasOrcid[0000-0002-5415-1600]{S.~Hellman}$^\textrm{\scriptsize 43a,43b}$,    
\AtlasOrcid[0000-0002-9243-7554]{C.~Helsens}$^\textrm{\scriptsize 34}$,    
\AtlasOrcid{R.C.W.~Henderson}$^\textrm{\scriptsize 87}$,    
\AtlasOrcid[0000-0001-8231-2080]{L.~Henkelmann}$^\textrm{\scriptsize 30}$,    
\AtlasOrcid{A.M.~Henriques~Correia}$^\textrm{\scriptsize 34}$,    
\AtlasOrcid[0000-0001-8926-6734]{H.~Herde}$^\textrm{\scriptsize 148}$,    
\AtlasOrcid[0000-0001-9844-6200]{Y.~Hern\'andez~Jim\'enez}$^\textrm{\scriptsize 150}$,    
\AtlasOrcid{H.~Herr}$^\textrm{\scriptsize 96}$,    
\AtlasOrcid[0000-0002-2254-0257]{M.G.~Herrmann}$^\textrm{\scriptsize 110}$,    
\AtlasOrcid[0000-0002-1478-3152]{T.~Herrmann}$^\textrm{\scriptsize 46}$,    
\AtlasOrcid[0000-0001-7661-5122]{G.~Herten}$^\textrm{\scriptsize 50}$,    
\AtlasOrcid[0000-0002-2646-5805]{R.~Hertenberger}$^\textrm{\scriptsize 110}$,    
\AtlasOrcid[0000-0002-0778-2717]{L.~Hervas}$^\textrm{\scriptsize 34}$,    
\AtlasOrcid[0000-0002-6698-9937]{N.P.~Hessey}$^\textrm{\scriptsize 162a}$,    
\AtlasOrcid[0000-0002-4630-9914]{H.~Hibi}$^\textrm{\scriptsize 80}$,    
\AtlasOrcid[0000-0002-5704-4253]{S.~Higashino}$^\textrm{\scriptsize 79}$,    
\AtlasOrcid[0000-0002-3094-2520]{E.~Hig\'on-Rodriguez}$^\textrm{\scriptsize 168}$,    
\AtlasOrcid{K.H.~Hiller}$^\textrm{\scriptsize 44}$,    
\AtlasOrcid[0000-0002-7599-6469]{S.J.~Hillier}$^\textrm{\scriptsize 19}$,    
\AtlasOrcid[0000-0002-8616-5898]{M.~Hils}$^\textrm{\scriptsize 46}$,    
\AtlasOrcid[0000-0002-5529-2173]{I.~Hinchliffe}$^\textrm{\scriptsize 16}$,    
\AtlasOrcid[0000-0002-0556-189X]{F.~Hinterkeuser}$^\textrm{\scriptsize 22}$,    
\AtlasOrcid[0000-0003-4988-9149]{M.~Hirose}$^\textrm{\scriptsize 127}$,    
\AtlasOrcid[0000-0002-2389-1286]{S.~Hirose}$^\textrm{\scriptsize 163}$,    
\AtlasOrcid[0000-0002-7998-8925]{D.~Hirschbuehl}$^\textrm{\scriptsize 176}$,    
\AtlasOrcid[0000-0002-8668-6933]{B.~Hiti}$^\textrm{\scriptsize 89}$,    
\AtlasOrcid{O.~Hladik}$^\textrm{\scriptsize 135}$,    
\AtlasOrcid[0000-0001-5404-7857]{J.~Hobbs}$^\textrm{\scriptsize 150}$,    
\AtlasOrcid[0000-0001-7602-5771]{R.~Hobincu}$^\textrm{\scriptsize 25e}$,    
\AtlasOrcid[0000-0001-5241-0544]{N.~Hod}$^\textrm{\scriptsize 174}$,    
\AtlasOrcid[0000-0002-1040-1241]{M.C.~Hodgkinson}$^\textrm{\scriptsize 144}$,    
\AtlasOrcid[0000-0002-2244-189X]{B.H.~Hodkinson}$^\textrm{\scriptsize 30}$,    
\AtlasOrcid[0000-0002-6596-9395]{A.~Hoecker}$^\textrm{\scriptsize 34}$,    
\AtlasOrcid[0000-0003-2799-5020]{J.~Hofer}$^\textrm{\scriptsize 44}$,    
\AtlasOrcid[0000-0002-5317-1247]{D.~Hohn}$^\textrm{\scriptsize 50}$,    
\AtlasOrcid[0000-0001-5407-7247]{T.~Holm}$^\textrm{\scriptsize 22}$,    
\AtlasOrcid[0000-0002-3959-5174]{T.R.~Holmes}$^\textrm{\scriptsize 35}$,    
\AtlasOrcid[0000-0001-8018-4185]{M.~Holzbock}$^\textrm{\scriptsize 111}$,    
\AtlasOrcid[0000-0003-0684-600X]{L.B.A.H.~Hommels}$^\textrm{\scriptsize 30}$,    
\AtlasOrcid[0000-0002-2698-4787]{B.P.~Honan}$^\textrm{\scriptsize 97}$,    
\AtlasOrcid[0000-0002-7494-5504]{J.~Hong}$^\textrm{\scriptsize 58c}$,    
\AtlasOrcid[0000-0001-7834-328X]{T.M.~Hong}$^\textrm{\scriptsize 133}$,    
\AtlasOrcid[0000-0003-4752-2458]{Y.~Hong}$^\textrm{\scriptsize 51}$,    
\AtlasOrcid[0000-0002-3596-6572]{J.C.~Honig}$^\textrm{\scriptsize 50}$,    
\AtlasOrcid[0000-0001-6063-2884]{A.~H\"{o}nle}$^\textrm{\scriptsize 111}$,    
\AtlasOrcid[0000-0002-4090-6099]{B.H.~Hooberman}$^\textrm{\scriptsize 167}$,    
\AtlasOrcid[0000-0001-7814-8740]{W.H.~Hopkins}$^\textrm{\scriptsize 5}$,    
\AtlasOrcid[0000-0003-0457-3052]{Y.~Horii}$^\textrm{\scriptsize 112}$,    
\AtlasOrcid[0000-0002-9512-4932]{L.A.~Horyn}$^\textrm{\scriptsize 35}$,    
\AtlasOrcid[0000-0001-9861-151X]{S.~Hou}$^\textrm{\scriptsize 153}$,    
\AtlasOrcid[0000-0002-0560-8985]{J.~Howarth}$^\textrm{\scriptsize 55}$,    
\AtlasOrcid[0000-0002-7562-0234]{J.~Hoya}$^\textrm{\scriptsize 86}$,    
\AtlasOrcid[0000-0003-4223-7316]{M.~Hrabovsky}$^\textrm{\scriptsize 125}$,    
\AtlasOrcid[0000-0002-5411-114X]{A.~Hrynevich}$^\textrm{\scriptsize 105}$,    
\AtlasOrcid[0000-0001-5914-8614]{T.~Hryn'ova}$^\textrm{\scriptsize 4}$,    
\AtlasOrcid[0000-0003-3895-8356]{P.J.~Hsu}$^\textrm{\scriptsize 61}$,    
\AtlasOrcid[0000-0001-6214-8500]{S.-C.~Hsu}$^\textrm{\scriptsize 143}$,    
\AtlasOrcid[0000-0002-9705-7518]{Q.~Hu}$^\textrm{\scriptsize 37}$,    
\AtlasOrcid[0000-0003-4696-4430]{S.~Hu}$^\textrm{\scriptsize 58c}$,    
\AtlasOrcid[0000-0002-0552-3383]{Y.F.~Hu}$^\textrm{\scriptsize 13a,13d,al}$,    
\AtlasOrcid[0000-0002-1753-5621]{D.P.~Huang}$^\textrm{\scriptsize 92}$,    
\AtlasOrcid[0000-0002-6617-3807]{X.~Huang}$^\textrm{\scriptsize 13c}$,    
\AtlasOrcid[0000-0003-1826-2749]{Y.~Huang}$^\textrm{\scriptsize 58a}$,    
\AtlasOrcid[0000-0002-5972-2855]{Y.~Huang}$^\textrm{\scriptsize 13a}$,    
\AtlasOrcid[0000-0003-3250-9066]{Z.~Hubacek}$^\textrm{\scriptsize 136}$,    
\AtlasOrcid[0000-0002-0113-2465]{F.~Hubaut}$^\textrm{\scriptsize 98}$,    
\AtlasOrcid[0000-0002-1162-8763]{M.~Huebner}$^\textrm{\scriptsize 22}$,    
\AtlasOrcid[0000-0002-7472-3151]{F.~Huegging}$^\textrm{\scriptsize 22}$,    
\AtlasOrcid[0000-0002-5332-2738]{T.B.~Huffman}$^\textrm{\scriptsize 129}$,    
\AtlasOrcid[0000-0002-1752-3583]{M.~Huhtinen}$^\textrm{\scriptsize 34}$,    
\AtlasOrcid[0000-0002-3277-7418]{S.K.~Huiberts}$^\textrm{\scriptsize 15}$,    
\AtlasOrcid[0000-0002-0095-1290]{R.~Hulsken}$^\textrm{\scriptsize 56}$,    
\AtlasOrcid[0000-0003-2201-5572]{N.~Huseynov}$^\textrm{\scriptsize 77,z}$,    
\AtlasOrcid[0000-0001-9097-3014]{J.~Huston}$^\textrm{\scriptsize 103}$,    
\AtlasOrcid[0000-0002-6867-2538]{J.~Huth}$^\textrm{\scriptsize 57}$,    
\AtlasOrcid[0000-0002-9093-7141]{R.~Hyneman}$^\textrm{\scriptsize 148}$,    
\AtlasOrcid[0000-0001-9425-4287]{S.~Hyrych}$^\textrm{\scriptsize 26a}$,    
\AtlasOrcid[0000-0001-9965-5442]{G.~Iacobucci}$^\textrm{\scriptsize 52}$,    
\AtlasOrcid[0000-0002-0330-5921]{G.~Iakovidis}$^\textrm{\scriptsize 27}$,    
\AtlasOrcid[0000-0001-8847-7337]{I.~Ibragimov}$^\textrm{\scriptsize 146}$,    
\AtlasOrcid[0000-0001-6334-6648]{L.~Iconomidou-Fayard}$^\textrm{\scriptsize 62}$,    
\AtlasOrcid[0000-0002-5035-1242]{P.~Iengo}$^\textrm{\scriptsize 34}$,    
\AtlasOrcid[0000-0002-0940-244X]{R.~Iguchi}$^\textrm{\scriptsize 158}$,    
\AtlasOrcid[0000-0001-5312-4865]{T.~Iizawa}$^\textrm{\scriptsize 52}$,    
\AtlasOrcid[0000-0001-7287-6579]{Y.~Ikegami}$^\textrm{\scriptsize 79}$,    
\AtlasOrcid[0000-0001-9488-8095]{A.~Ilg}$^\textrm{\scriptsize 18}$,    
\AtlasOrcid[0000-0003-0105-7634]{N.~Ilic}$^\textrm{\scriptsize 161}$,    
\AtlasOrcid[0000-0002-7854-3174]{H.~Imam}$^\textrm{\scriptsize 33a}$,    
\AtlasOrcid[0000-0002-3699-8517]{T.~Ingebretsen~Carlson}$^\textrm{\scriptsize 43a,43b}$,    
\AtlasOrcid[0000-0002-1314-2580]{G.~Introzzi}$^\textrm{\scriptsize 68a,68b}$,    
\AtlasOrcid[0000-0003-4446-8150]{M.~Iodice}$^\textrm{\scriptsize 72a}$,    
\AtlasOrcid[0000-0001-5126-1620]{V.~Ippolito}$^\textrm{\scriptsize 70a,70b}$,    
\AtlasOrcid[0000-0002-7185-1334]{M.~Ishino}$^\textrm{\scriptsize 158}$,    
\AtlasOrcid[0000-0002-5624-5934]{W.~Islam}$^\textrm{\scriptsize 175}$,    
\AtlasOrcid[0000-0001-8259-1067]{C.~Issever}$^\textrm{\scriptsize 17,44}$,    
\AtlasOrcid[0000-0001-8504-6291]{S.~Istin}$^\textrm{\scriptsize 11c,am}$,    
\AtlasOrcid[0000-0002-2325-3225]{J.M.~Iturbe~Ponce}$^\textrm{\scriptsize 60a}$,    
\AtlasOrcid[0000-0001-5038-2762]{R.~Iuppa}$^\textrm{\scriptsize 73a,73b}$,    
\AtlasOrcid[0000-0002-9152-383X]{A.~Ivina}$^\textrm{\scriptsize 174}$,    
\AtlasOrcid[0000-0002-9846-5601]{J.M.~Izen}$^\textrm{\scriptsize 41}$,    
\AtlasOrcid[0000-0002-8770-1592]{V.~Izzo}$^\textrm{\scriptsize 67a}$,    
\AtlasOrcid[0000-0003-2489-9930]{P.~Jacka}$^\textrm{\scriptsize 135}$,    
\AtlasOrcid[0000-0002-0847-402X]{P.~Jackson}$^\textrm{\scriptsize 1}$,    
\AtlasOrcid[0000-0001-5446-5901]{R.M.~Jacobs}$^\textrm{\scriptsize 44}$,    
\AtlasOrcid[0000-0002-5094-5067]{B.P.~Jaeger}$^\textrm{\scriptsize 147}$,    
\AtlasOrcid[0000-0002-1669-759X]{C.S.~Jagfeld}$^\textrm{\scriptsize 110}$,    
\AtlasOrcid[0000-0001-5687-1006]{G.~J\"akel}$^\textrm{\scriptsize 176}$,    
\AtlasOrcid[0000-0001-8885-012X]{K.~Jakobs}$^\textrm{\scriptsize 50}$,    
\AtlasOrcid[0000-0001-7038-0369]{T.~Jakoubek}$^\textrm{\scriptsize 174}$,    
\AtlasOrcid[0000-0001-9554-0787]{J.~Jamieson}$^\textrm{\scriptsize 55}$,    
\AtlasOrcid[0000-0001-5411-8934]{K.W.~Janas}$^\textrm{\scriptsize 81a}$,    
\AtlasOrcid[0000-0002-8731-2060]{G.~Jarlskog}$^\textrm{\scriptsize 94}$,    
\AtlasOrcid[0000-0003-4189-2837]{A.E.~Jaspan}$^\textrm{\scriptsize 88}$,    
\AtlasOrcid{N.~Javadov}$^\textrm{\scriptsize 77,z}$,    
\AtlasOrcid[0000-0002-9389-3682]{T.~Jav\r{u}rek}$^\textrm{\scriptsize 34}$,    
\AtlasOrcid[0000-0001-8798-808X]{M.~Javurkova}$^\textrm{\scriptsize 99}$,    
\AtlasOrcid[0000-0002-6360-6136]{F.~Jeanneau}$^\textrm{\scriptsize 139}$,    
\AtlasOrcid[0000-0001-6507-4623]{L.~Jeanty}$^\textrm{\scriptsize 126}$,    
\AtlasOrcid[0000-0002-0159-6593]{J.~Jejelava}$^\textrm{\scriptsize 154a,aa}$,    
\AtlasOrcid[0000-0002-4539-4192]{P.~Jenni}$^\textrm{\scriptsize 50,e}$,    
\AtlasOrcid[0000-0001-7369-6975]{S.~J\'ez\'equel}$^\textrm{\scriptsize 4}$,    
\AtlasOrcid[0000-0002-5725-3397]{J.~Jia}$^\textrm{\scriptsize 150}$,    
\AtlasOrcid[0000-0002-2657-3099]{Z.~Jia}$^\textrm{\scriptsize 13c}$,    
\AtlasOrcid{Y.~Jiang}$^\textrm{\scriptsize 58a}$,    
\AtlasOrcid[0000-0003-2906-1977]{S.~Jiggins}$^\textrm{\scriptsize 48}$,    
\AtlasOrcid[0000-0002-8705-628X]{J.~Jimenez~Pena}$^\textrm{\scriptsize 111}$,    
\AtlasOrcid[0000-0002-5076-7803]{S.~Jin}$^\textrm{\scriptsize 13c}$,    
\AtlasOrcid[0000-0001-7449-9164]{A.~Jinaru}$^\textrm{\scriptsize 25b}$,    
\AtlasOrcid[0000-0001-5073-0974]{O.~Jinnouchi}$^\textrm{\scriptsize 159}$,    
\AtlasOrcid[0000-0002-4115-6322]{H.~Jivan}$^\textrm{\scriptsize 31f}$,    
\AtlasOrcid[0000-0001-5410-1315]{P.~Johansson}$^\textrm{\scriptsize 144}$,    
\AtlasOrcid[0000-0001-9147-6052]{K.A.~Johns}$^\textrm{\scriptsize 6}$,    
\AtlasOrcid[0000-0002-5387-572X]{C.A.~Johnson}$^\textrm{\scriptsize 63}$,    
\AtlasOrcid[0000-0002-9204-4689]{D.M.~Jones}$^\textrm{\scriptsize 30}$,    
\AtlasOrcid[0000-0001-6289-2292]{E.~Jones}$^\textrm{\scriptsize 172}$,    
\AtlasOrcid[0000-0002-6427-3513]{R.W.L.~Jones}$^\textrm{\scriptsize 87}$,    
\AtlasOrcid[0000-0002-2580-1977]{T.J.~Jones}$^\textrm{\scriptsize 88}$,    
\AtlasOrcid[0000-0001-5650-4556]{J.~Jovicevic}$^\textrm{\scriptsize 51}$,    
\AtlasOrcid[0000-0002-9745-1638]{X.~Ju}$^\textrm{\scriptsize 16}$,    
\AtlasOrcid[0000-0001-7205-1171]{J.J.~Junggeburth}$^\textrm{\scriptsize 34}$,    
\AtlasOrcid[0000-0002-1558-3291]{A.~Juste~Rozas}$^\textrm{\scriptsize 12,v}$,    
\AtlasOrcid[0000-0003-0568-5750]{S.~Kabana}$^\textrm{\scriptsize 141c}$,    
\AtlasOrcid[0000-0002-8880-4120]{A.~Kaczmarska}$^\textrm{\scriptsize 82}$,    
\AtlasOrcid{M.~Kado}$^\textrm{\scriptsize 70a,70b}$,    
\AtlasOrcid[0000-0002-4693-7857]{H.~Kagan}$^\textrm{\scriptsize 122}$,    
\AtlasOrcid[0000-0002-3386-6869]{M.~Kagan}$^\textrm{\scriptsize 148}$,    
\AtlasOrcid{A.~Kahn}$^\textrm{\scriptsize 37}$,    
\AtlasOrcid[0000-0001-7131-3029]{A.~Kahn}$^\textrm{\scriptsize 131}$,    
\AtlasOrcid[0000-0002-9003-5711]{C.~Kahra}$^\textrm{\scriptsize 96}$,    
\AtlasOrcid[0000-0002-6532-7501]{T.~Kaji}$^\textrm{\scriptsize 173}$,    
\AtlasOrcid[0000-0002-8464-1790]{E.~Kajomovitz}$^\textrm{\scriptsize 155}$,    
\AtlasOrcid[0000-0002-2875-853X]{C.W.~Kalderon}$^\textrm{\scriptsize 27}$,    
\AtlasOrcid[0000-0002-7845-2301]{A.~Kamenshchikov}$^\textrm{\scriptsize 118}$,    
\AtlasOrcid[0000-0003-1510-7719]{M.~Kaneda}$^\textrm{\scriptsize 158}$,    
\AtlasOrcid[0000-0001-5009-0399]{N.J.~Kang}$^\textrm{\scriptsize 140}$,    
\AtlasOrcid[0000-0002-5320-7043]{S.~Kang}$^\textrm{\scriptsize 76}$,    
\AtlasOrcid[0000-0003-1090-3820]{Y.~Kano}$^\textrm{\scriptsize 112}$,    
\AtlasOrcid{J.~Kanzaki}$^\textrm{\scriptsize 79}$,    
\AtlasOrcid[0000-0002-4238-9822]{D.~Kar}$^\textrm{\scriptsize 31f}$,    
\AtlasOrcid[0000-0002-5010-8613]{K.~Karava}$^\textrm{\scriptsize 129}$,    
\AtlasOrcid[0000-0001-8967-1705]{M.J.~Kareem}$^\textrm{\scriptsize 162b}$,    
\AtlasOrcid[0000-0002-6940-261X]{I.~Karkanias}$^\textrm{\scriptsize 157}$,    
\AtlasOrcid[0000-0002-2230-5353]{S.N.~Karpov}$^\textrm{\scriptsize 77}$,    
\AtlasOrcid[0000-0003-0254-4629]{Z.M.~Karpova}$^\textrm{\scriptsize 77}$,    
\AtlasOrcid[0000-0002-1957-3787]{V.~Kartvelishvili}$^\textrm{\scriptsize 87}$,    
\AtlasOrcid[0000-0001-9087-4315]{A.N.~Karyukhin}$^\textrm{\scriptsize 118}$,    
\AtlasOrcid[0000-0002-7139-8197]{E.~Kasimi}$^\textrm{\scriptsize 157}$,    
\AtlasOrcid[0000-0002-0794-4325]{C.~Kato}$^\textrm{\scriptsize 58d}$,    
\AtlasOrcid[0000-0003-3121-395X]{J.~Katzy}$^\textrm{\scriptsize 44}$,    
\AtlasOrcid[0000-0002-7874-6107]{K.~Kawade}$^\textrm{\scriptsize 145}$,    
\AtlasOrcid[0000-0001-8882-129X]{K.~Kawagoe}$^\textrm{\scriptsize 85}$,    
\AtlasOrcid[0000-0002-9124-788X]{T.~Kawaguchi}$^\textrm{\scriptsize 112}$,    
\AtlasOrcid[0000-0002-5841-5511]{T.~Kawamoto}$^\textrm{\scriptsize 139}$,    
\AtlasOrcid{G.~Kawamura}$^\textrm{\scriptsize 51}$,    
\AtlasOrcid[0000-0002-6304-3230]{E.F.~Kay}$^\textrm{\scriptsize 170}$,    
\AtlasOrcid[0000-0002-9775-7303]{F.I.~Kaya}$^\textrm{\scriptsize 164}$,    
\AtlasOrcid[0000-0002-7252-3201]{S.~Kazakos}$^\textrm{\scriptsize 12}$,    
\AtlasOrcid[0000-0002-4906-5468]{V.F.~Kazanin}$^\textrm{\scriptsize 117b,117a}$,    
\AtlasOrcid[0000-0001-5798-6665]{Y.~Ke}$^\textrm{\scriptsize 150}$,    
\AtlasOrcid[0000-0003-0766-5307]{J.M.~Keaveney}$^\textrm{\scriptsize 31a}$,    
\AtlasOrcid[0000-0002-0510-4189]{R.~Keeler}$^\textrm{\scriptsize 170}$,    
\AtlasOrcid[0000-0001-7140-9813]{J.S.~Keller}$^\textrm{\scriptsize 32}$,    
\AtlasOrcid[0000-0002-2297-1356]{D.~Kelsey}$^\textrm{\scriptsize 151}$,    
\AtlasOrcid[0000-0003-4168-3373]{J.J.~Kempster}$^\textrm{\scriptsize 19}$,    
\AtlasOrcid[0000-0001-9845-5473]{J.~Kendrick}$^\textrm{\scriptsize 19}$,    
\AtlasOrcid[0000-0003-3264-548X]{K.E.~Kennedy}$^\textrm{\scriptsize 37}$,    
\AtlasOrcid[0000-0002-2555-497X]{O.~Kepka}$^\textrm{\scriptsize 135}$,    
\AtlasOrcid[0000-0002-0511-2592]{S.~Kersten}$^\textrm{\scriptsize 176}$,    
\AtlasOrcid[0000-0002-4529-452X]{B.P.~Ker\v{s}evan}$^\textrm{\scriptsize 89}$,    
\AtlasOrcid[0000-0002-8597-3834]{S.~Ketabchi~Haghighat}$^\textrm{\scriptsize 161}$,    
\AtlasOrcid[0000-0002-8785-7378]{M.~Khandoga}$^\textrm{\scriptsize 130}$,    
\AtlasOrcid[0000-0001-9621-422X]{A.~Khanov}$^\textrm{\scriptsize 124}$,    
\AtlasOrcid[0000-0002-1051-3833]{A.G.~Kharlamov}$^\textrm{\scriptsize 117b,117a}$,    
\AtlasOrcid[0000-0002-0387-6804]{T.~Kharlamova}$^\textrm{\scriptsize 117b,117a}$,    
\AtlasOrcid[0000-0001-8720-6615]{E.E.~Khoda}$^\textrm{\scriptsize 143}$,    
\AtlasOrcid[0000-0002-5954-3101]{T.J.~Khoo}$^\textrm{\scriptsize 17}$,    
\AtlasOrcid[0000-0002-6353-8452]{G.~Khoriauli}$^\textrm{\scriptsize 171}$,    
\AtlasOrcid[0000-0001-7400-6454]{E.~Khramov}$^\textrm{\scriptsize 77}$,    
\AtlasOrcid[0000-0003-2350-1249]{J.~Khubua}$^\textrm{\scriptsize 154b}$,    
\AtlasOrcid[0000-0003-0536-5386]{S.~Kido}$^\textrm{\scriptsize 80}$,    
\AtlasOrcid[0000-0001-9608-2626]{M.~Kiehn}$^\textrm{\scriptsize 34}$,    
\AtlasOrcid[0000-0003-1450-0009]{A.~Kilgallon}$^\textrm{\scriptsize 126}$,    
\AtlasOrcid[0000-0002-4203-014X]{E.~Kim}$^\textrm{\scriptsize 159}$,    
\AtlasOrcid[0000-0003-3286-1326]{Y.K.~Kim}$^\textrm{\scriptsize 35}$,    
\AtlasOrcid[0000-0002-8883-9374]{N.~Kimura}$^\textrm{\scriptsize 92}$,    
\AtlasOrcid[0000-0001-5611-9543]{A.~Kirchhoff}$^\textrm{\scriptsize 51}$,    
\AtlasOrcid[0000-0001-8545-5650]{D.~Kirchmeier}$^\textrm{\scriptsize 46}$,    
\AtlasOrcid[0000-0003-1679-6907]{C.~Kirfel}$^\textrm{\scriptsize 22}$,    
\AtlasOrcid[0000-0001-8096-7577]{J.~Kirk}$^\textrm{\scriptsize 138}$,    
\AtlasOrcid[0000-0001-7490-6890]{A.E.~Kiryunin}$^\textrm{\scriptsize 111}$,    
\AtlasOrcid[0000-0003-3476-8192]{T.~Kishimoto}$^\textrm{\scriptsize 158}$,    
\AtlasOrcid{D.P.~Kisliuk}$^\textrm{\scriptsize 161}$,    
\AtlasOrcid[0000-0003-4431-8400]{C.~Kitsaki}$^\textrm{\scriptsize 9}$,    
\AtlasOrcid[0000-0002-6854-2717]{O.~Kivernyk}$^\textrm{\scriptsize 22}$,    
\AtlasOrcid[0000-0003-1423-6041]{T.~Klapdor-Kleingrothaus}$^\textrm{\scriptsize 50}$,    
\AtlasOrcid[0000-0002-4326-9742]{M.~Klassen}$^\textrm{\scriptsize 59a}$,    
\AtlasOrcid[0000-0002-3780-1755]{C.~Klein}$^\textrm{\scriptsize 32}$,    
\AtlasOrcid[0000-0002-0145-4747]{L.~Klein}$^\textrm{\scriptsize 171}$,    
\AtlasOrcid[0000-0002-9999-2534]{M.H.~Klein}$^\textrm{\scriptsize 102}$,    
\AtlasOrcid[0000-0002-8527-964X]{M.~Klein}$^\textrm{\scriptsize 88}$,    
\AtlasOrcid[0000-0001-7391-5330]{U.~Klein}$^\textrm{\scriptsize 88}$,    
\AtlasOrcid[0000-0003-1661-6873]{P.~Klimek}$^\textrm{\scriptsize 34}$,    
\AtlasOrcid[0000-0003-2748-4829]{A.~Klimentov}$^\textrm{\scriptsize 27}$,    
\AtlasOrcid[0000-0002-9362-3973]{F.~Klimpel}$^\textrm{\scriptsize 34}$,    
\AtlasOrcid[0000-0002-5721-9834]{T.~Klingl}$^\textrm{\scriptsize 22}$,    
\AtlasOrcid[0000-0002-9580-0363]{T.~Klioutchnikova}$^\textrm{\scriptsize 34}$,    
\AtlasOrcid[0000-0002-7864-459X]{F.F.~Klitzner}$^\textrm{\scriptsize 110}$,    
\AtlasOrcid[0000-0001-6419-5829]{P.~Kluit}$^\textrm{\scriptsize 115}$,    
\AtlasOrcid[0000-0001-8484-2261]{S.~Kluth}$^\textrm{\scriptsize 111}$,    
\AtlasOrcid[0000-0002-6206-1912]{E.~Kneringer}$^\textrm{\scriptsize 74}$,    
\AtlasOrcid[0000-0003-2486-7672]{T.M.~Knight}$^\textrm{\scriptsize 161}$,    
\AtlasOrcid[0000-0002-1559-9285]{A.~Knue}$^\textrm{\scriptsize 50}$,    
\AtlasOrcid{D.~Kobayashi}$^\textrm{\scriptsize 85}$,    
\AtlasOrcid[0000-0002-0124-2699]{M.~Kobel}$^\textrm{\scriptsize 46}$,    
\AtlasOrcid[0000-0003-4559-6058]{M.~Kocian}$^\textrm{\scriptsize 148}$,    
\AtlasOrcid{T.~Kodama}$^\textrm{\scriptsize 158}$,    
\AtlasOrcid[0000-0002-8644-2349]{P.~Kodys}$^\textrm{\scriptsize 137}$,    
\AtlasOrcid[0000-0002-9090-5502]{D.M.~Koeck}$^\textrm{\scriptsize 151}$,    
\AtlasOrcid[0000-0002-0497-3550]{P.T.~Koenig}$^\textrm{\scriptsize 22}$,    
\AtlasOrcid[0000-0001-9612-4988]{T.~Koffas}$^\textrm{\scriptsize 32}$,    
\AtlasOrcid[0000-0002-0490-9778]{N.M.~K\"ohler}$^\textrm{\scriptsize 34}$,    
\AtlasOrcid[0000-0002-6117-3816]{M.~Kolb}$^\textrm{\scriptsize 139}$,    
\AtlasOrcid[0000-0002-8560-8917]{I.~Koletsou}$^\textrm{\scriptsize 4}$,    
\AtlasOrcid[0000-0002-3047-3146]{T.~Komarek}$^\textrm{\scriptsize 125}$,    
\AtlasOrcid[0000-0002-6901-9717]{K.~K\"oneke}$^\textrm{\scriptsize 50}$,    
\AtlasOrcid[0000-0001-8063-8765]{A.X.Y.~Kong}$^\textrm{\scriptsize 1}$,    
\AtlasOrcid[0000-0003-1553-2950]{T.~Kono}$^\textrm{\scriptsize 121}$,    
\AtlasOrcid{V.~Konstantinides}$^\textrm{\scriptsize 92}$,    
\AtlasOrcid[0000-0002-4140-6360]{N.~Konstantinidis}$^\textrm{\scriptsize 92}$,    
\AtlasOrcid[0000-0002-1859-6557]{B.~Konya}$^\textrm{\scriptsize 94}$,    
\AtlasOrcid[0000-0002-8775-1194]{R.~Kopeliansky}$^\textrm{\scriptsize 63}$,    
\AtlasOrcid[0000-0002-2023-5945]{S.~Koperny}$^\textrm{\scriptsize 81a}$,    
\AtlasOrcid[0000-0001-8085-4505]{K.~Korcyl}$^\textrm{\scriptsize 82}$,    
\AtlasOrcid[0000-0003-0486-2081]{K.~Kordas}$^\textrm{\scriptsize 157}$,    
\AtlasOrcid{G.~Koren}$^\textrm{\scriptsize 156}$,    
\AtlasOrcid[0000-0002-3962-2099]{A.~Korn}$^\textrm{\scriptsize 92}$,    
\AtlasOrcid[0000-0001-9291-5408]{S.~Korn}$^\textrm{\scriptsize 51}$,    
\AtlasOrcid[0000-0002-9211-9775]{I.~Korolkov}$^\textrm{\scriptsize 12}$,    
\AtlasOrcid{E.V.~Korolkova}$^\textrm{\scriptsize 144}$,    
\AtlasOrcid[0000-0003-3640-8676]{N.~Korotkova}$^\textrm{\scriptsize 109}$,    
\AtlasOrcid[0000-0001-7081-3275]{B.~Kortman}$^\textrm{\scriptsize 115}$,    
\AtlasOrcid[0000-0003-0352-3096]{O.~Kortner}$^\textrm{\scriptsize 111}$,    
\AtlasOrcid[0000-0001-8667-1814]{S.~Kortner}$^\textrm{\scriptsize 111}$,    
\AtlasOrcid[0000-0003-1772-6898]{W.H.~Kostecka}$^\textrm{\scriptsize 116}$,    
\AtlasOrcid[0000-0002-0490-9209]{V.V.~Kostyukhin}$^\textrm{\scriptsize 144,160}$,    
\AtlasOrcid[0000-0002-8057-9467]{A.~Kotsokechagia}$^\textrm{\scriptsize 62}$,    
\AtlasOrcid[0000-0003-3384-5053]{A.~Kotwal}$^\textrm{\scriptsize 47}$,    
\AtlasOrcid[0000-0003-1012-4675]{A.~Koulouris}$^\textrm{\scriptsize 34}$,    
\AtlasOrcid[0000-0002-6614-108X]{A.~Kourkoumeli-Charalampidi}$^\textrm{\scriptsize 68a,68b}$,    
\AtlasOrcid[0000-0003-0083-274X]{C.~Kourkoumelis}$^\textrm{\scriptsize 8}$,    
\AtlasOrcid[0000-0001-6568-2047]{E.~Kourlitis}$^\textrm{\scriptsize 5}$,    
\AtlasOrcid[0000-0003-0294-3953]{O.~Kovanda}$^\textrm{\scriptsize 151}$,    
\AtlasOrcid[0000-0002-7314-0990]{R.~Kowalewski}$^\textrm{\scriptsize 170}$,    
\AtlasOrcid[0000-0001-6226-8385]{W.~Kozanecki}$^\textrm{\scriptsize 139}$,    
\AtlasOrcid[0000-0003-4724-9017]{A.S.~Kozhin}$^\textrm{\scriptsize 118}$,    
\AtlasOrcid[0000-0002-8625-5586]{V.A.~Kramarenko}$^\textrm{\scriptsize 109}$,    
\AtlasOrcid[0000-0002-7580-384X]{G.~Kramberger}$^\textrm{\scriptsize 89}$,    
\AtlasOrcid[0000-0002-0296-5899]{P.~Kramer}$^\textrm{\scriptsize 96}$,    
\AtlasOrcid[0000-0002-6356-372X]{D.~Krasnopevtsev}$^\textrm{\scriptsize 58a}$,    
\AtlasOrcid[0000-0002-7440-0520]{M.W.~Krasny}$^\textrm{\scriptsize 130}$,    
\AtlasOrcid[0000-0002-6468-1381]{A.~Krasznahorkay}$^\textrm{\scriptsize 34}$,    
\AtlasOrcid[0000-0003-4487-6365]{J.A.~Kremer}$^\textrm{\scriptsize 96}$,    
\AtlasOrcid[0000-0002-8515-1355]{J.~Kretzschmar}$^\textrm{\scriptsize 88}$,    
\AtlasOrcid[0000-0002-1739-6596]{K.~Kreul}$^\textrm{\scriptsize 17}$,    
\AtlasOrcid[0000-0001-9958-949X]{P.~Krieger}$^\textrm{\scriptsize 161}$,    
\AtlasOrcid[0000-0002-7675-8024]{F.~Krieter}$^\textrm{\scriptsize 110}$,    
\AtlasOrcid[0000-0001-6169-0517]{S.~Krishnamurthy}$^\textrm{\scriptsize 99}$,    
\AtlasOrcid[0000-0002-0734-6122]{A.~Krishnan}$^\textrm{\scriptsize 59b}$,    
\AtlasOrcid[0000-0001-9062-2257]{M.~Krivos}$^\textrm{\scriptsize 137}$,    
\AtlasOrcid[0000-0001-6408-2648]{K.~Krizka}$^\textrm{\scriptsize 16}$,    
\AtlasOrcid[0000-0001-9873-0228]{K.~Kroeninger}$^\textrm{\scriptsize 45}$,    
\AtlasOrcid[0000-0003-1808-0259]{H.~Kroha}$^\textrm{\scriptsize 111}$,    
\AtlasOrcid[0000-0001-6215-3326]{J.~Kroll}$^\textrm{\scriptsize 135}$,    
\AtlasOrcid[0000-0002-0964-6815]{J.~Kroll}$^\textrm{\scriptsize 131}$,    
\AtlasOrcid[0000-0001-9395-3430]{K.S.~Krowpman}$^\textrm{\scriptsize 103}$,    
\AtlasOrcid[0000-0003-2116-4592]{U.~Kruchonak}$^\textrm{\scriptsize 77}$,    
\AtlasOrcid[0000-0001-8287-3961]{H.~Kr\"uger}$^\textrm{\scriptsize 22}$,    
\AtlasOrcid{N.~Krumnack}$^\textrm{\scriptsize 76}$,    
\AtlasOrcid[0000-0001-5791-0345]{M.C.~Kruse}$^\textrm{\scriptsize 47}$,    
\AtlasOrcid[0000-0002-1214-9262]{J.A.~Krzysiak}$^\textrm{\scriptsize 82}$,    
\AtlasOrcid[0000-0003-3993-4903]{A.~Kubota}$^\textrm{\scriptsize 159}$,    
\AtlasOrcid[0000-0002-3664-2465]{O.~Kuchinskaia}$^\textrm{\scriptsize 160}$,    
\AtlasOrcid[0000-0002-0116-5494]{S.~Kuday}$^\textrm{\scriptsize 3a}$,    
\AtlasOrcid[0000-0003-4087-1575]{D.~Kuechler}$^\textrm{\scriptsize 44}$,    
\AtlasOrcid[0000-0001-9087-6230]{J.T.~Kuechler}$^\textrm{\scriptsize 44}$,    
\AtlasOrcid[0000-0001-5270-0920]{S.~Kuehn}$^\textrm{\scriptsize 34}$,    
\AtlasOrcid[0000-0002-1473-350X]{T.~Kuhl}$^\textrm{\scriptsize 44}$,    
\AtlasOrcid[0000-0003-4387-8756]{V.~Kukhtin}$^\textrm{\scriptsize 77}$,    
\AtlasOrcid[0000-0002-3036-5575]{Y.~Kulchitsky}$^\textrm{\scriptsize 104,ad}$,    
\AtlasOrcid[0000-0002-3065-326X]{S.~Kuleshov}$^\textrm{\scriptsize 141b}$,    
\AtlasOrcid[0000-0003-3681-1588]{M.~Kumar}$^\textrm{\scriptsize 31f}$,    
\AtlasOrcid[0000-0001-9174-6200]{N.~Kumari}$^\textrm{\scriptsize 98}$,    
\AtlasOrcid[0000-0002-3598-2847]{M.~Kuna}$^\textrm{\scriptsize 56}$,    
\AtlasOrcid[0000-0003-3692-1410]{A.~Kupco}$^\textrm{\scriptsize 135}$,    
\AtlasOrcid{T.~Kupfer}$^\textrm{\scriptsize 45}$,    
\AtlasOrcid[0000-0002-7540-0012]{O.~Kuprash}$^\textrm{\scriptsize 50}$,    
\AtlasOrcid[0000-0003-3932-016X]{H.~Kurashige}$^\textrm{\scriptsize 80}$,    
\AtlasOrcid[0000-0001-9392-3936]{L.L.~Kurchaninov}$^\textrm{\scriptsize 162a}$,    
\AtlasOrcid[0000-0002-1281-8462]{Y.A.~Kurochkin}$^\textrm{\scriptsize 104}$,    
\AtlasOrcid[0000-0001-7924-1517]{A.~Kurova}$^\textrm{\scriptsize 108}$,    
\AtlasOrcid{M.G.~Kurth}$^\textrm{\scriptsize 13a,13d}$,    
\AtlasOrcid[0000-0002-1921-6173]{E.S.~Kuwertz}$^\textrm{\scriptsize 34}$,    
\AtlasOrcid[0000-0001-8858-8440]{M.~Kuze}$^\textrm{\scriptsize 159}$,    
\AtlasOrcid[0000-0001-7243-0227]{A.K.~Kvam}$^\textrm{\scriptsize 143}$,    
\AtlasOrcid[0000-0001-5973-8729]{J.~Kvita}$^\textrm{\scriptsize 125}$,    
\AtlasOrcid[0000-0001-8717-4449]{T.~Kwan}$^\textrm{\scriptsize 100}$,    
\AtlasOrcid[0000-0002-0820-9998]{K.W.~Kwok}$^\textrm{\scriptsize 60a}$,    
\AtlasOrcid[0000-0002-2623-6252]{C.~Lacasta}$^\textrm{\scriptsize 168}$,    
\AtlasOrcid[0000-0003-4588-8325]{F.~Lacava}$^\textrm{\scriptsize 70a,70b}$,    
\AtlasOrcid[0000-0002-7183-8607]{H.~Lacker}$^\textrm{\scriptsize 17}$,    
\AtlasOrcid[0000-0002-1590-194X]{D.~Lacour}$^\textrm{\scriptsize 130}$,    
\AtlasOrcid[0000-0002-3707-9010]{N.N.~Lad}$^\textrm{\scriptsize 92}$,    
\AtlasOrcid[0000-0001-6206-8148]{E.~Ladygin}$^\textrm{\scriptsize 77}$,    
\AtlasOrcid[0000-0001-7848-6088]{R.~Lafaye}$^\textrm{\scriptsize 4}$,    
\AtlasOrcid[0000-0002-4209-4194]{B.~Laforge}$^\textrm{\scriptsize 130}$,    
\AtlasOrcid[0000-0001-7509-7765]{T.~Lagouri}$^\textrm{\scriptsize 141c}$,    
\AtlasOrcid[0000-0002-9898-9253]{S.~Lai}$^\textrm{\scriptsize 51}$,    
\AtlasOrcid[0000-0002-4357-7649]{I.K.~Lakomiec}$^\textrm{\scriptsize 81a}$,    
\AtlasOrcid[0000-0003-0953-559X]{N.~Lalloue}$^\textrm{\scriptsize 56}$,    
\AtlasOrcid[0000-0002-5606-4164]{J.E.~Lambert}$^\textrm{\scriptsize 123}$,    
\AtlasOrcid{S.~Lammers}$^\textrm{\scriptsize 63}$,    
\AtlasOrcid[0000-0002-2337-0958]{W.~Lampl}$^\textrm{\scriptsize 6}$,    
\AtlasOrcid[0000-0001-9782-9920]{C.~Lampoudis}$^\textrm{\scriptsize 157}$,    
\AtlasOrcid[0000-0002-0225-187X]{E.~Lan\c{c}on}$^\textrm{\scriptsize 27}$,    
\AtlasOrcid[0000-0002-8222-2066]{U.~Landgraf}$^\textrm{\scriptsize 50}$,    
\AtlasOrcid[0000-0001-6828-9769]{M.P.J.~Landon}$^\textrm{\scriptsize 90}$,    
\AtlasOrcid[0000-0001-9954-7898]{V.S.~Lang}$^\textrm{\scriptsize 50}$,    
\AtlasOrcid[0000-0003-1307-1441]{J.C.~Lange}$^\textrm{\scriptsize 51}$,    
\AtlasOrcid[0000-0001-6595-1382]{R.J.~Langenberg}$^\textrm{\scriptsize 99}$,    
\AtlasOrcid[0000-0001-8057-4351]{A.J.~Lankford}$^\textrm{\scriptsize 165}$,    
\AtlasOrcid[0000-0002-7197-9645]{F.~Lanni}$^\textrm{\scriptsize 27}$,    
\AtlasOrcid[0000-0002-0729-6487]{K.~Lantzsch}$^\textrm{\scriptsize 22}$,    
\AtlasOrcid[0000-0003-4980-6032]{A.~Lanza}$^\textrm{\scriptsize 68a}$,    
\AtlasOrcid[0000-0001-6246-6787]{A.~Lapertosa}$^\textrm{\scriptsize 53b,53a}$,    
\AtlasOrcid[0000-0002-4815-5314]{J.F.~Laporte}$^\textrm{\scriptsize 139}$,    
\AtlasOrcid[0000-0002-1388-869X]{T.~Lari}$^\textrm{\scriptsize 66a}$,    
\AtlasOrcid[0000-0001-6068-4473]{F.~Lasagni~Manghi}$^\textrm{\scriptsize 21b}$,    
\AtlasOrcid[0000-0002-9541-0592]{M.~Lassnig}$^\textrm{\scriptsize 34}$,    
\AtlasOrcid[0000-0001-9591-5622]{V.~Latonova}$^\textrm{\scriptsize 135}$,    
\AtlasOrcid[0000-0001-7110-7823]{T.S.~Lau}$^\textrm{\scriptsize 60a}$,    
\AtlasOrcid[0000-0001-6098-0555]{A.~Laudrain}$^\textrm{\scriptsize 96}$,    
\AtlasOrcid[0000-0002-2575-0743]{A.~Laurier}$^\textrm{\scriptsize 32}$,    
\AtlasOrcid[0000-0002-3407-752X]{M.~Lavorgna}$^\textrm{\scriptsize 67a,67b}$,    
\AtlasOrcid[0000-0003-3211-067X]{S.D.~Lawlor}$^\textrm{\scriptsize 91}$,    
\AtlasOrcid[0000-0002-9035-9679]{Z.~Lawrence}$^\textrm{\scriptsize 97}$,    
\AtlasOrcid[0000-0002-4094-1273]{M.~Lazzaroni}$^\textrm{\scriptsize 66a,66b}$,    
\AtlasOrcid{B.~Le}$^\textrm{\scriptsize 97}$,    
\AtlasOrcid[0000-0003-1501-7262]{B.~Leban}$^\textrm{\scriptsize 89}$,    
\AtlasOrcid[0000-0002-9566-1850]{A.~Lebedev}$^\textrm{\scriptsize 76}$,    
\AtlasOrcid[0000-0001-5977-6418]{M.~LeBlanc}$^\textrm{\scriptsize 34}$,    
\AtlasOrcid[0000-0002-9450-6568]{T.~LeCompte}$^\textrm{\scriptsize 5}$,    
\AtlasOrcid[0000-0001-9398-1909]{F.~Ledroit-Guillon}$^\textrm{\scriptsize 56}$,    
\AtlasOrcid{A.C.A.~Lee}$^\textrm{\scriptsize 92}$,    
\AtlasOrcid[0000-0002-5968-6954]{G.R.~Lee}$^\textrm{\scriptsize 15}$,    
\AtlasOrcid[0000-0002-5590-335X]{L.~Lee}$^\textrm{\scriptsize 57}$,    
\AtlasOrcid[0000-0002-3353-2658]{S.C.~Lee}$^\textrm{\scriptsize 153}$,    
\AtlasOrcid[0000-0001-5688-1212]{S.~Lee}$^\textrm{\scriptsize 76}$,    
\AtlasOrcid[0000-0002-3365-6781]{L.L.~Leeuw}$^\textrm{\scriptsize 31c}$,    
\AtlasOrcid[0000-0001-8212-6624]{B.~Lefebvre}$^\textrm{\scriptsize 162a}$,    
\AtlasOrcid[0000-0002-7394-2408]{H.P.~Lefebvre}$^\textrm{\scriptsize 91}$,    
\AtlasOrcid[0000-0002-5560-0586]{M.~Lefebvre}$^\textrm{\scriptsize 170}$,    
\AtlasOrcid[0000-0002-9299-9020]{C.~Leggett}$^\textrm{\scriptsize 16}$,    
\AtlasOrcid[0000-0002-8590-8231]{K.~Lehmann}$^\textrm{\scriptsize 147}$,    
\AtlasOrcid[0000-0001-5521-1655]{N.~Lehmann}$^\textrm{\scriptsize 18}$,    
\AtlasOrcid[0000-0001-9045-7853]{G.~Lehmann~Miotto}$^\textrm{\scriptsize 34}$,    
\AtlasOrcid[0000-0002-2968-7841]{W.A.~Leight}$^\textrm{\scriptsize 44}$,    
\AtlasOrcid[0000-0002-8126-3958]{A.~Leisos}$^\textrm{\scriptsize 157,u}$,    
\AtlasOrcid[0000-0003-0392-3663]{M.A.L.~Leite}$^\textrm{\scriptsize 78c}$,    
\AtlasOrcid[0000-0002-0335-503X]{C.E.~Leitgeb}$^\textrm{\scriptsize 44}$,    
\AtlasOrcid[0000-0002-2994-2187]{R.~Leitner}$^\textrm{\scriptsize 137}$,    
\AtlasOrcid[0000-0002-1525-2695]{K.J.C.~Leney}$^\textrm{\scriptsize 40}$,    
\AtlasOrcid[0000-0002-9560-1778]{T.~Lenz}$^\textrm{\scriptsize 22}$,    
\AtlasOrcid[0000-0001-6222-9642]{S.~Leone}$^\textrm{\scriptsize 69a}$,    
\AtlasOrcid[0000-0002-7241-2114]{C.~Leonidopoulos}$^\textrm{\scriptsize 48}$,    
\AtlasOrcid[0000-0001-9415-7903]{A.~Leopold}$^\textrm{\scriptsize 130}$,    
\AtlasOrcid[0000-0003-3105-7045]{C.~Leroy}$^\textrm{\scriptsize 106}$,    
\AtlasOrcid[0000-0002-8875-1399]{R.~Les}$^\textrm{\scriptsize 103}$,    
\AtlasOrcid[0000-0001-5770-4883]{C.G.~Lester}$^\textrm{\scriptsize 30}$,    
\AtlasOrcid[0000-0002-5495-0656]{M.~Levchenko}$^\textrm{\scriptsize 132}$,    
\AtlasOrcid[0000-0002-0244-4743]{J.~Lev\^eque}$^\textrm{\scriptsize 4}$,    
\AtlasOrcid[0000-0003-0512-0856]{D.~Levin}$^\textrm{\scriptsize 102}$,    
\AtlasOrcid[0000-0003-4679-0485]{L.J.~Levinson}$^\textrm{\scriptsize 174}$,    
\AtlasOrcid[0000-0002-7814-8596]{D.J.~Lewis}$^\textrm{\scriptsize 19}$,    
\AtlasOrcid[0000-0002-7004-3802]{B.~Li}$^\textrm{\scriptsize 13b}$,    
\AtlasOrcid[0000-0002-1974-2229]{B.~Li}$^\textrm{\scriptsize 58b}$,    
\AtlasOrcid{C.~Li}$^\textrm{\scriptsize 58a}$,    
\AtlasOrcid[0000-0003-3495-7778]{C-Q.~Li}$^\textrm{\scriptsize 58c,58d}$,    
\AtlasOrcid[0000-0002-1081-2032]{H.~Li}$^\textrm{\scriptsize 58a}$,    
\AtlasOrcid[0000-0002-4732-5633]{H.~Li}$^\textrm{\scriptsize 58b}$,    
\AtlasOrcid[0000-0001-9346-6982]{H.~Li}$^\textrm{\scriptsize 58b}$,    
\AtlasOrcid[0000-0003-4776-4123]{J.~Li}$^\textrm{\scriptsize 58c}$,    
\AtlasOrcid[0000-0002-2545-0329]{K.~Li}$^\textrm{\scriptsize 143}$,    
\AtlasOrcid[0000-0001-6411-6107]{L.~Li}$^\textrm{\scriptsize 58c}$,    
\AtlasOrcid[0000-0003-4317-3203]{M.~Li}$^\textrm{\scriptsize 13a,13d}$,    
\AtlasOrcid[0000-0001-6066-195X]{Q.Y.~Li}$^\textrm{\scriptsize 58a}$,    
\AtlasOrcid[0000-0001-7879-3272]{S.~Li}$^\textrm{\scriptsize 58d,58c,c}$,    
\AtlasOrcid[0000-0001-7775-4300]{T.~Li}$^\textrm{\scriptsize 58b}$,    
\AtlasOrcid[0000-0001-6975-102X]{X.~Li}$^\textrm{\scriptsize 44}$,    
\AtlasOrcid[0000-0003-3042-0893]{Y.~Li}$^\textrm{\scriptsize 44}$,    
\AtlasOrcid[0000-0003-1189-3505]{Z.~Li}$^\textrm{\scriptsize 58b}$,    
\AtlasOrcid[0000-0001-9800-2626]{Z.~Li}$^\textrm{\scriptsize 129}$,    
\AtlasOrcid[0000-0001-7096-2158]{Z.~Li}$^\textrm{\scriptsize 100}$,    
\AtlasOrcid{Z.~Li}$^\textrm{\scriptsize 88}$,    
\AtlasOrcid[0000-0003-0629-2131]{Z.~Liang}$^\textrm{\scriptsize 13a}$,    
\AtlasOrcid[0000-0002-8444-8827]{M.~Liberatore}$^\textrm{\scriptsize 44}$,    
\AtlasOrcid[0000-0002-6011-2851]{B.~Liberti}$^\textrm{\scriptsize 71a}$,    
\AtlasOrcid[0000-0002-5779-5989]{K.~Lie}$^\textrm{\scriptsize 60c}$,    
\AtlasOrcid[0000-0003-0642-9169]{J.~Lieber~Marin}$^\textrm{\scriptsize 78b}$,    
\AtlasOrcid[0000-0002-2269-3632]{K.~Lin}$^\textrm{\scriptsize 103}$,    
\AtlasOrcid[0000-0002-4593-0602]{R.A.~Linck}$^\textrm{\scriptsize 63}$,    
\AtlasOrcid{R.E.~Lindley}$^\textrm{\scriptsize 6}$,    
\AtlasOrcid[0000-0001-9490-7276]{J.H.~Lindon}$^\textrm{\scriptsize 2}$,    
\AtlasOrcid[0000-0002-3961-5016]{A.~Linss}$^\textrm{\scriptsize 44}$,    
\AtlasOrcid[0000-0001-5982-7326]{E.~Lipeles}$^\textrm{\scriptsize 131}$,    
\AtlasOrcid[0000-0002-8759-8564]{A.~Lipniacka}$^\textrm{\scriptsize 15}$,    
\AtlasOrcid[0000-0002-1735-3924]{T.M.~Liss}$^\textrm{\scriptsize 167,ai}$,    
\AtlasOrcid[0000-0002-1552-3651]{A.~Lister}$^\textrm{\scriptsize 169}$,    
\AtlasOrcid[0000-0002-9372-0730]{J.D.~Little}$^\textrm{\scriptsize 7}$,    
\AtlasOrcid[0000-0003-2823-9307]{B.~Liu}$^\textrm{\scriptsize 13a}$,    
\AtlasOrcid[0000-0002-0721-8331]{B.X.~Liu}$^\textrm{\scriptsize 147}$,    
\AtlasOrcid[0000-0003-3259-8775]{J.B.~Liu}$^\textrm{\scriptsize 58a}$,    
\AtlasOrcid[0000-0001-5359-4541]{J.K.K.~Liu}$^\textrm{\scriptsize 35}$,    
\AtlasOrcid[0000-0001-5807-0501]{K.~Liu}$^\textrm{\scriptsize 58d,58c}$,    
\AtlasOrcid[0000-0003-0056-7296]{M.~Liu}$^\textrm{\scriptsize 58a}$,    
\AtlasOrcid[0000-0002-0236-5404]{M.Y.~Liu}$^\textrm{\scriptsize 58a}$,    
\AtlasOrcid[0000-0002-9815-8898]{P.~Liu}$^\textrm{\scriptsize 13a}$,    
\AtlasOrcid[0000-0003-1366-5530]{X.~Liu}$^\textrm{\scriptsize 58a}$,    
\AtlasOrcid[0000-0002-3576-7004]{Y.~Liu}$^\textrm{\scriptsize 44}$,    
\AtlasOrcid[0000-0003-3615-2332]{Y.~Liu}$^\textrm{\scriptsize 13c,13d}$,    
\AtlasOrcid[0000-0001-9190-4547]{Y.L.~Liu}$^\textrm{\scriptsize 102}$,    
\AtlasOrcid[0000-0003-4448-4679]{Y.W.~Liu}$^\textrm{\scriptsize 58a}$,    
\AtlasOrcid[0000-0002-5877-0062]{M.~Livan}$^\textrm{\scriptsize 68a,68b}$,    
\AtlasOrcid[0000-0003-1769-8524]{A.~Lleres}$^\textrm{\scriptsize 56}$,    
\AtlasOrcid[0000-0003-0027-7969]{J.~Llorente~Merino}$^\textrm{\scriptsize 147}$,    
\AtlasOrcid[0000-0002-5073-2264]{S.L.~Lloyd}$^\textrm{\scriptsize 90}$,    
\AtlasOrcid[0000-0001-9012-3431]{E.M.~Lobodzinska}$^\textrm{\scriptsize 44}$,    
\AtlasOrcid[0000-0002-2005-671X]{P.~Loch}$^\textrm{\scriptsize 6}$,    
\AtlasOrcid[0000-0003-2516-5015]{S.~Loffredo}$^\textrm{\scriptsize 71a,71b}$,    
\AtlasOrcid[0000-0002-9751-7633]{T.~Lohse}$^\textrm{\scriptsize 17}$,    
\AtlasOrcid[0000-0003-1833-9160]{K.~Lohwasser}$^\textrm{\scriptsize 144}$,    
\AtlasOrcid[0000-0001-8929-1243]{M.~Lokajicek}$^\textrm{\scriptsize 135}$,    
\AtlasOrcid[0000-0002-2115-9382]{J.D.~Long}$^\textrm{\scriptsize 167}$,    
\AtlasOrcid[0000-0002-0352-2854]{I.~Longarini}$^\textrm{\scriptsize 70a,70b}$,    
\AtlasOrcid[0000-0002-2357-7043]{L.~Longo}$^\textrm{\scriptsize 34}$,    
\AtlasOrcid[0000-0003-3984-6452]{R.~Longo}$^\textrm{\scriptsize 167}$,    
\AtlasOrcid[0000-0002-4300-7064]{I.~Lopez~Paz}$^\textrm{\scriptsize 12}$,    
\AtlasOrcid[0000-0002-0511-4766]{A.~Lopez~Solis}$^\textrm{\scriptsize 44}$,    
\AtlasOrcid[0000-0001-6530-1873]{J.~Lorenz}$^\textrm{\scriptsize 110}$,    
\AtlasOrcid[0000-0002-7857-7606]{N.~Lorenzo~Martinez}$^\textrm{\scriptsize 4}$,    
\AtlasOrcid[0000-0001-9657-0910]{A.M.~Lory}$^\textrm{\scriptsize 110}$,    
\AtlasOrcid[0000-0002-6328-8561]{A.~L\"osle}$^\textrm{\scriptsize 50}$,    
\AtlasOrcid[0000-0002-8309-5548]{X.~Lou}$^\textrm{\scriptsize 43a,43b}$,    
\AtlasOrcid[0000-0003-0867-2189]{X.~Lou}$^\textrm{\scriptsize 13a}$,    
\AtlasOrcid[0000-0003-4066-2087]{A.~Lounis}$^\textrm{\scriptsize 62}$,    
\AtlasOrcid[0000-0001-7743-3849]{J.~Love}$^\textrm{\scriptsize 5}$,    
\AtlasOrcid[0000-0002-7803-6674]{P.A.~Love}$^\textrm{\scriptsize 87}$,    
\AtlasOrcid[0000-0003-0613-140X]{J.J.~Lozano~Bahilo}$^\textrm{\scriptsize 168}$,    
\AtlasOrcid[0000-0001-8133-3533]{G.~Lu}$^\textrm{\scriptsize 13a}$,    
\AtlasOrcid[0000-0001-7610-3952]{M.~Lu}$^\textrm{\scriptsize 58a}$,    
\AtlasOrcid[0000-0002-8814-1670]{S.~Lu}$^\textrm{\scriptsize 131}$,    
\AtlasOrcid[0000-0002-2497-0509]{Y.J.~Lu}$^\textrm{\scriptsize 61}$,    
\AtlasOrcid[0000-0002-9285-7452]{H.J.~Lubatti}$^\textrm{\scriptsize 143}$,    
\AtlasOrcid[0000-0001-7464-304X]{C.~Luci}$^\textrm{\scriptsize 70a,70b}$,    
\AtlasOrcid[0000-0002-1626-6255]{F.L.~Lucio~Alves}$^\textrm{\scriptsize 13c}$,    
\AtlasOrcid[0000-0002-5992-0640]{A.~Lucotte}$^\textrm{\scriptsize 56}$,    
\AtlasOrcid[0000-0001-8721-6901]{F.~Luehring}$^\textrm{\scriptsize 63}$,    
\AtlasOrcid[0000-0001-5028-3342]{I.~Luise}$^\textrm{\scriptsize 150}$,    
\AtlasOrcid{L.~Luminari}$^\textrm{\scriptsize 70a}$,    
\AtlasOrcid{O.~Lundberg}$^\textrm{\scriptsize 149}$,    
\AtlasOrcid[0000-0003-3867-0336]{B.~Lund-Jensen}$^\textrm{\scriptsize 149}$,    
\AtlasOrcid[0000-0001-6527-0253]{N.A.~Luongo}$^\textrm{\scriptsize 126}$,    
\AtlasOrcid[0000-0003-4515-0224]{M.S.~Lutz}$^\textrm{\scriptsize 156}$,    
\AtlasOrcid[0000-0002-9634-542X]{D.~Lynn}$^\textrm{\scriptsize 27}$,    
\AtlasOrcid{H.~Lyons}$^\textrm{\scriptsize 88}$,    
\AtlasOrcid[0000-0003-2990-1673]{R.~Lysak}$^\textrm{\scriptsize 135}$,    
\AtlasOrcid[0000-0002-8141-3995]{E.~Lytken}$^\textrm{\scriptsize 94}$,    
\AtlasOrcid[0000-0002-7611-3728]{F.~Lyu}$^\textrm{\scriptsize 13a}$,    
\AtlasOrcid[0000-0003-0136-233X]{V.~Lyubushkin}$^\textrm{\scriptsize 77}$,    
\AtlasOrcid[0000-0001-8329-7994]{T.~Lyubushkina}$^\textrm{\scriptsize 77}$,    
\AtlasOrcid[0000-0002-8916-6220]{H.~Ma}$^\textrm{\scriptsize 27}$,    
\AtlasOrcid[0000-0001-9717-1508]{L.L.~Ma}$^\textrm{\scriptsize 58b}$,    
\AtlasOrcid[0000-0002-3577-9347]{Y.~Ma}$^\textrm{\scriptsize 92}$,    
\AtlasOrcid[0000-0001-5533-6300]{D.M.~Mac~Donell}$^\textrm{\scriptsize 170}$,    
\AtlasOrcid[0000-0002-7234-9522]{G.~Maccarrone}$^\textrm{\scriptsize 49}$,    
\AtlasOrcid[0000-0001-7857-9188]{C.M.~Macdonald}$^\textrm{\scriptsize 144}$,    
\AtlasOrcid[0000-0002-3150-3124]{J.C.~MacDonald}$^\textrm{\scriptsize 144}$,    
\AtlasOrcid[0000-0002-6875-6408]{R.~Madar}$^\textrm{\scriptsize 36}$,    
\AtlasOrcid[0000-0003-4276-1046]{W.F.~Mader}$^\textrm{\scriptsize 46}$,    
\AtlasOrcid[0000-0002-6033-944X]{M.~Madugoda~Ralalage~Don}$^\textrm{\scriptsize 124}$,    
\AtlasOrcid[0000-0001-8375-7532]{N.~Madysa}$^\textrm{\scriptsize 46}$,    
\AtlasOrcid[0000-0002-9084-3305]{J.~Maeda}$^\textrm{\scriptsize 80}$,    
\AtlasOrcid[0000-0003-0901-1817]{T.~Maeno}$^\textrm{\scriptsize 27}$,    
\AtlasOrcid[0000-0002-3773-8573]{M.~Maerker}$^\textrm{\scriptsize 46}$,    
\AtlasOrcid[0000-0003-0693-793X]{V.~Magerl}$^\textrm{\scriptsize 50}$,    
\AtlasOrcid[0000-0001-5704-9700]{J.~Magro}$^\textrm{\scriptsize 64a,64c}$,    
\AtlasOrcid[0000-0002-2640-5941]{D.J.~Mahon}$^\textrm{\scriptsize 37}$,    
\AtlasOrcid[0000-0002-3511-0133]{C.~Maidantchik}$^\textrm{\scriptsize 78b}$,    
\AtlasOrcid[0000-0001-9099-0009]{A.~Maio}$^\textrm{\scriptsize 134a,134b,134d}$,    
\AtlasOrcid[0000-0003-4819-9226]{K.~Maj}$^\textrm{\scriptsize 81a}$,    
\AtlasOrcid[0000-0001-8857-5770]{O.~Majersky}$^\textrm{\scriptsize 26a}$,    
\AtlasOrcid[0000-0002-6871-3395]{S.~Majewski}$^\textrm{\scriptsize 126}$,    
\AtlasOrcid[0000-0001-5124-904X]{N.~Makovec}$^\textrm{\scriptsize 62}$,    
\AtlasOrcid{V.~Maksimovic}$^\textrm{\scriptsize 14}$,    
\AtlasOrcid[0000-0002-8813-3830]{B.~Malaescu}$^\textrm{\scriptsize 130}$,    
\AtlasOrcid[0000-0001-8183-0468]{Pa.~Malecki}$^\textrm{\scriptsize 82}$,    
\AtlasOrcid[0000-0003-1028-8602]{V.P.~Maleev}$^\textrm{\scriptsize 132}$,    
\AtlasOrcid[0000-0002-0948-5775]{F.~Malek}$^\textrm{\scriptsize 56}$,    
\AtlasOrcid[0000-0002-3996-4662]{D.~Malito}$^\textrm{\scriptsize 39b,39a}$,    
\AtlasOrcid[0000-0001-7934-1649]{U.~Mallik}$^\textrm{\scriptsize 75}$,    
\AtlasOrcid[0000-0003-4325-7378]{C.~Malone}$^\textrm{\scriptsize 30}$,    
\AtlasOrcid{S.~Maltezos}$^\textrm{\scriptsize 9}$,    
\AtlasOrcid{S.~Malyukov}$^\textrm{\scriptsize 77}$,    
\AtlasOrcid[0000-0002-3203-4243]{J.~Mamuzic}$^\textrm{\scriptsize 168}$,    
\AtlasOrcid[0000-0001-6158-2751]{G.~Mancini}$^\textrm{\scriptsize 49}$,    
\AtlasOrcid[0000-0001-5038-5154]{J.P.~Mandalia}$^\textrm{\scriptsize 90}$,    
\AtlasOrcid[0000-0002-0131-7523]{I.~Mandi\'{c}}$^\textrm{\scriptsize 89}$,    
\AtlasOrcid[0000-0003-1792-6793]{L.~Manhaes~de~Andrade~Filho}$^\textrm{\scriptsize 78a}$,    
\AtlasOrcid[0000-0002-4362-0088]{I.M.~Maniatis}$^\textrm{\scriptsize 157}$,    
\AtlasOrcid[0000-0001-7551-0169]{M.~Manisha}$^\textrm{\scriptsize 139}$,    
\AtlasOrcid[0000-0003-3896-5222]{J.~Manjarres~Ramos}$^\textrm{\scriptsize 46}$,    
\AtlasOrcid[0000-0001-7357-9648]{K.H.~Mankinen}$^\textrm{\scriptsize 94}$,    
\AtlasOrcid[0000-0002-8497-9038]{A.~Mann}$^\textrm{\scriptsize 110}$,    
\AtlasOrcid[0000-0003-4627-4026]{A.~Manousos}$^\textrm{\scriptsize 74}$,    
\AtlasOrcid[0000-0001-5945-5518]{B.~Mansoulie}$^\textrm{\scriptsize 139}$,    
\AtlasOrcid[0000-0001-5561-9909]{I.~Manthos}$^\textrm{\scriptsize 157}$,    
\AtlasOrcid[0000-0002-2488-0511]{S.~Manzoni}$^\textrm{\scriptsize 115}$,    
\AtlasOrcid[0000-0002-7020-4098]{A.~Marantis}$^\textrm{\scriptsize 157,u}$,    
\AtlasOrcid[0000-0003-2655-7643]{G.~Marchiori}$^\textrm{\scriptsize 130}$,    
\AtlasOrcid[0000-0003-0860-7897]{M.~Marcisovsky}$^\textrm{\scriptsize 135}$,    
\AtlasOrcid[0000-0001-6422-7018]{L.~Marcoccia}$^\textrm{\scriptsize 71a,71b}$,    
\AtlasOrcid[0000-0002-9889-8271]{C.~Marcon}$^\textrm{\scriptsize 94}$,    
\AtlasOrcid[0000-0002-4468-0154]{M.~Marjanovic}$^\textrm{\scriptsize 123}$,    
\AtlasOrcid[0000-0003-0786-2570]{Z.~Marshall}$^\textrm{\scriptsize 16}$,    
\AtlasOrcid[0000-0002-3897-6223]{S.~Marti-Garcia}$^\textrm{\scriptsize 168}$,    
\AtlasOrcid[0000-0002-1477-1645]{T.A.~Martin}$^\textrm{\scriptsize 172}$,    
\AtlasOrcid[0000-0003-3053-8146]{V.J.~Martin}$^\textrm{\scriptsize 48}$,    
\AtlasOrcid[0000-0003-3420-2105]{B.~Martin~dit~Latour}$^\textrm{\scriptsize 15}$,    
\AtlasOrcid[0000-0002-4466-3864]{L.~Martinelli}$^\textrm{\scriptsize 70a,70b}$,    
\AtlasOrcid[0000-0002-3135-945X]{M.~Martinez}$^\textrm{\scriptsize 12,v}$,    
\AtlasOrcid[0000-0001-8925-9518]{P.~Martinez~Agullo}$^\textrm{\scriptsize 168}$,    
\AtlasOrcid[0000-0001-7102-6388]{V.I.~Martinez~Outschoorn}$^\textrm{\scriptsize 99}$,    
\AtlasOrcid[0000-0001-9457-1928]{S.~Martin-Haugh}$^\textrm{\scriptsize 138}$,    
\AtlasOrcid[0000-0002-4963-9441]{V.S.~Martoiu}$^\textrm{\scriptsize 25b}$,    
\AtlasOrcid[0000-0001-9080-2944]{A.C.~Martyniuk}$^\textrm{\scriptsize 92}$,    
\AtlasOrcid[0000-0003-4364-4351]{A.~Marzin}$^\textrm{\scriptsize 34}$,    
\AtlasOrcid[0000-0003-0917-1618]{S.R.~Maschek}$^\textrm{\scriptsize 111}$,    
\AtlasOrcid[0000-0002-0038-5372]{L.~Masetti}$^\textrm{\scriptsize 96}$,    
\AtlasOrcid[0000-0001-5333-6016]{T.~Mashimo}$^\textrm{\scriptsize 158}$,    
\AtlasOrcid[0000-0002-6813-8423]{J.~Masik}$^\textrm{\scriptsize 97}$,    
\AtlasOrcid[0000-0002-4234-3111]{A.L.~Maslennikov}$^\textrm{\scriptsize 117b,117a}$,    
\AtlasOrcid[0000-0002-3735-7762]{L.~Massa}$^\textrm{\scriptsize 21b}$,    
\AtlasOrcid[0000-0002-9335-9690]{P.~Massarotti}$^\textrm{\scriptsize 67a,67b}$,    
\AtlasOrcid[0000-0002-9853-0194]{P.~Mastrandrea}$^\textrm{\scriptsize 69a,69b}$,    
\AtlasOrcid[0000-0002-8933-9494]{A.~Mastroberardino}$^\textrm{\scriptsize 39b,39a}$,    
\AtlasOrcid[0000-0001-9984-8009]{T.~Masubuchi}$^\textrm{\scriptsize 158}$,    
\AtlasOrcid{D.~Matakias}$^\textrm{\scriptsize 27}$,    
\AtlasOrcid[0000-0002-6248-953X]{T.~Mathisen}$^\textrm{\scriptsize 166}$,    
\AtlasOrcid[0000-0002-2179-0350]{A.~Matic}$^\textrm{\scriptsize 110}$,    
\AtlasOrcid{N.~Matsuzawa}$^\textrm{\scriptsize 158}$,    
\AtlasOrcid[0000-0002-5162-3713]{J.~Maurer}$^\textrm{\scriptsize 25b}$,    
\AtlasOrcid[0000-0002-1449-0317]{B.~Ma\v{c}ek}$^\textrm{\scriptsize 89}$,    
\AtlasOrcid[0000-0001-8783-3758]{D.A.~Maximov}$^\textrm{\scriptsize 117b,117a}$,    
\AtlasOrcid[0000-0003-0954-0970]{R.~Mazini}$^\textrm{\scriptsize 153}$,    
\AtlasOrcid[0000-0001-8420-3742]{I.~Maznas}$^\textrm{\scriptsize 157}$,    
\AtlasOrcid[0000-0003-3865-730X]{S.M.~Mazza}$^\textrm{\scriptsize 140}$,    
\AtlasOrcid[0000-0003-1281-0193]{C.~Mc~Ginn}$^\textrm{\scriptsize 27}$,    
\AtlasOrcid[0000-0001-7551-3386]{J.P.~Mc~Gowan}$^\textrm{\scriptsize 100}$,    
\AtlasOrcid[0000-0002-4551-4502]{S.P.~Mc~Kee}$^\textrm{\scriptsize 102}$,    
\AtlasOrcid[0000-0002-1182-3526]{T.G.~McCarthy}$^\textrm{\scriptsize 111}$,    
\AtlasOrcid[0000-0002-0768-1959]{W.P.~McCormack}$^\textrm{\scriptsize 16}$,    
\AtlasOrcid[0000-0002-8092-5331]{E.F.~McDonald}$^\textrm{\scriptsize 101}$,    
\AtlasOrcid[0000-0002-2489-2598]{A.E.~McDougall}$^\textrm{\scriptsize 115}$,    
\AtlasOrcid[0000-0001-9273-2564]{J.A.~Mcfayden}$^\textrm{\scriptsize 151}$,    
\AtlasOrcid[0000-0003-3534-4164]{G.~Mchedlidze}$^\textrm{\scriptsize 154b}$,    
\AtlasOrcid{M.A.~McKay}$^\textrm{\scriptsize 40}$,    
\AtlasOrcid[0000-0001-5475-2521]{K.D.~McLean}$^\textrm{\scriptsize 170}$,    
\AtlasOrcid[0000-0002-3599-9075]{S.J.~McMahon}$^\textrm{\scriptsize 138}$,    
\AtlasOrcid[0000-0002-0676-324X]{P.C.~McNamara}$^\textrm{\scriptsize 101}$,    
\AtlasOrcid[0000-0001-9211-7019]{R.A.~McPherson}$^\textrm{\scriptsize 170,y}$,    
\AtlasOrcid[0000-0002-9745-0504]{J.E.~Mdhluli}$^\textrm{\scriptsize 31f}$,    
\AtlasOrcid[0000-0001-8119-0333]{Z.A.~Meadows}$^\textrm{\scriptsize 99}$,    
\AtlasOrcid[0000-0002-3613-7514]{S.~Meehan}$^\textrm{\scriptsize 34}$,    
\AtlasOrcid[0000-0001-8569-7094]{T.~Megy}$^\textrm{\scriptsize 36}$,    
\AtlasOrcid[0000-0002-1281-2060]{S.~Mehlhase}$^\textrm{\scriptsize 110}$,    
\AtlasOrcid[0000-0003-2619-9743]{A.~Mehta}$^\textrm{\scriptsize 88}$,    
\AtlasOrcid[0000-0003-0032-7022]{B.~Meirose}$^\textrm{\scriptsize 41}$,    
\AtlasOrcid[0000-0002-7018-682X]{D.~Melini}$^\textrm{\scriptsize 155}$,    
\AtlasOrcid[0000-0003-4838-1546]{B.R.~Mellado~Garcia}$^\textrm{\scriptsize 31f}$,    
\AtlasOrcid[0000-0002-3964-6736]{A.H.~Melo}$^\textrm{\scriptsize 51}$,    
\AtlasOrcid[0000-0001-7075-2214]{F.~Meloni}$^\textrm{\scriptsize 44}$,    
\AtlasOrcid[0000-0002-7616-3290]{A.~Melzer}$^\textrm{\scriptsize 22}$,    
\AtlasOrcid[0000-0002-7785-2047]{E.D.~Mendes~Gouveia}$^\textrm{\scriptsize 134a}$,    
\AtlasOrcid[0000-0001-6305-8400]{A.M.~Mendes~Jacques~Da~Costa}$^\textrm{\scriptsize 19}$,    
\AtlasOrcid{H.Y.~Meng}$^\textrm{\scriptsize 161}$,    
\AtlasOrcid[0000-0002-2901-6589]{L.~Meng}$^\textrm{\scriptsize 34}$,    
\AtlasOrcid[0000-0002-8186-4032]{S.~Menke}$^\textrm{\scriptsize 111}$,    
\AtlasOrcid[0000-0001-9769-0578]{M.~Mentink}$^\textrm{\scriptsize 34}$,    
\AtlasOrcid[0000-0002-6934-3752]{E.~Meoni}$^\textrm{\scriptsize 39b,39a}$,    
\AtlasOrcid[0000-0002-5445-5938]{C.~Merlassino}$^\textrm{\scriptsize 129}$,    
\AtlasOrcid[0000-0001-9656-9901]{P.~Mermod}$^\textrm{\scriptsize 52,*}$,    
\AtlasOrcid[0000-0002-1822-1114]{L.~Merola}$^\textrm{\scriptsize 67a,67b}$,    
\AtlasOrcid[0000-0003-4779-3522]{C.~Meroni}$^\textrm{\scriptsize 66a}$,    
\AtlasOrcid{G.~Merz}$^\textrm{\scriptsize 102}$,    
\AtlasOrcid[0000-0001-6897-4651]{O.~Meshkov}$^\textrm{\scriptsize 109,107}$,    
\AtlasOrcid[0000-0003-2007-7171]{J.K.R.~Meshreki}$^\textrm{\scriptsize 146}$,    
\AtlasOrcid[0000-0001-5454-3017]{J.~Metcalfe}$^\textrm{\scriptsize 5}$,    
\AtlasOrcid[0000-0002-5508-530X]{A.S.~Mete}$^\textrm{\scriptsize 5}$,    
\AtlasOrcid[0000-0003-3552-6566]{C.~Meyer}$^\textrm{\scriptsize 63}$,    
\AtlasOrcid[0000-0002-7497-0945]{J-P.~Meyer}$^\textrm{\scriptsize 139}$,    
\AtlasOrcid[0000-0002-3276-8941]{M.~Michetti}$^\textrm{\scriptsize 17}$,    
\AtlasOrcid[0000-0002-8396-9946]{R.P.~Middleton}$^\textrm{\scriptsize 138}$,    
\AtlasOrcid[0000-0003-0162-2891]{L.~Mijovi\'{c}}$^\textrm{\scriptsize 48}$,    
\AtlasOrcid[0000-0003-0460-3178]{G.~Mikenberg}$^\textrm{\scriptsize 174}$,    
\AtlasOrcid[0000-0003-1277-2596]{M.~Mikestikova}$^\textrm{\scriptsize 135}$,    
\AtlasOrcid[0000-0002-4119-6156]{M.~Miku\v{z}}$^\textrm{\scriptsize 89}$,    
\AtlasOrcid[0000-0002-0384-6955]{H.~Mildner}$^\textrm{\scriptsize 144}$,    
\AtlasOrcid[0000-0002-9173-8363]{A.~Milic}$^\textrm{\scriptsize 161}$,    
\AtlasOrcid[0000-0003-4688-4174]{C.D.~Milke}$^\textrm{\scriptsize 40}$,    
\AtlasOrcid[0000-0002-9485-9435]{D.W.~Miller}$^\textrm{\scriptsize 35}$,    
\AtlasOrcid[0000-0001-5539-3233]{L.S.~Miller}$^\textrm{\scriptsize 32}$,    
\AtlasOrcid[0000-0003-3863-3607]{A.~Milov}$^\textrm{\scriptsize 174}$,    
\AtlasOrcid{D.A.~Milstead}$^\textrm{\scriptsize 43a,43b}$,    
\AtlasOrcid{T.~Min}$^\textrm{\scriptsize 13c}$,    
\AtlasOrcid[0000-0001-8055-4692]{A.A.~Minaenko}$^\textrm{\scriptsize 118}$,    
\AtlasOrcid[0000-0002-4688-3510]{I.A.~Minashvili}$^\textrm{\scriptsize 154b}$,    
\AtlasOrcid[0000-0003-3759-0588]{L.~Mince}$^\textrm{\scriptsize 55}$,    
\AtlasOrcid[0000-0002-6307-1418]{A.I.~Mincer}$^\textrm{\scriptsize 120}$,    
\AtlasOrcid[0000-0002-5511-2611]{B.~Mindur}$^\textrm{\scriptsize 81a}$,    
\AtlasOrcid[0000-0002-2236-3879]{M.~Mineev}$^\textrm{\scriptsize 77}$,    
\AtlasOrcid{Y.~Minegishi}$^\textrm{\scriptsize 158}$,    
\AtlasOrcid[0000-0002-2984-8174]{Y.~Mino}$^\textrm{\scriptsize 83}$,    
\AtlasOrcid[0000-0002-4276-715X]{L.M.~Mir}$^\textrm{\scriptsize 12}$,    
\AtlasOrcid[0000-0001-7863-583X]{M.~Miralles~Lopez}$^\textrm{\scriptsize 168}$,    
\AtlasOrcid[0000-0001-6381-5723]{M.~Mironova}$^\textrm{\scriptsize 129}$,    
\AtlasOrcid[0000-0001-9861-9140]{T.~Mitani}$^\textrm{\scriptsize 173}$,    
\AtlasOrcid[0000-0002-1533-8886]{V.A.~Mitsou}$^\textrm{\scriptsize 168}$,    
\AtlasOrcid{M.~Mittal}$^\textrm{\scriptsize 58c}$,    
\AtlasOrcid[0000-0002-0287-8293]{O.~Miu}$^\textrm{\scriptsize 161}$,    
\AtlasOrcid[0000-0002-4893-6778]{P.S.~Miyagawa}$^\textrm{\scriptsize 90}$,    
\AtlasOrcid{Y.~Miyazaki}$^\textrm{\scriptsize 85}$,    
\AtlasOrcid[0000-0001-6672-0500]{A.~Mizukami}$^\textrm{\scriptsize 79}$,    
\AtlasOrcid[0000-0002-7148-6859]{J.U.~Mj\"ornmark}$^\textrm{\scriptsize 94}$,    
\AtlasOrcid[0000-0002-5786-3136]{T.~Mkrtchyan}$^\textrm{\scriptsize 59a}$,    
\AtlasOrcid[0000-0003-2028-1930]{M.~Mlynarikova}$^\textrm{\scriptsize 116}$,    
\AtlasOrcid[0000-0002-7644-5984]{T.~Moa}$^\textrm{\scriptsize 43a,43b}$,    
\AtlasOrcid[0000-0001-5911-6815]{S.~Mobius}$^\textrm{\scriptsize 51}$,    
\AtlasOrcid[0000-0002-6310-2149]{K.~Mochizuki}$^\textrm{\scriptsize 106}$,    
\AtlasOrcid[0000-0003-2135-9971]{P.~Moder}$^\textrm{\scriptsize 44}$,    
\AtlasOrcid[0000-0003-2688-234X]{P.~Mogg}$^\textrm{\scriptsize 110}$,    
\AtlasOrcid[0000-0002-5003-1919]{A.F.~Mohammed}$^\textrm{\scriptsize 13a}$,    
\AtlasOrcid[0000-0003-3006-6337]{S.~Mohapatra}$^\textrm{\scriptsize 37}$,    
\AtlasOrcid[0000-0001-9878-4373]{G.~Mokgatitswane}$^\textrm{\scriptsize 31f}$,    
\AtlasOrcid[0000-0003-1025-3741]{B.~Mondal}$^\textrm{\scriptsize 146}$,    
\AtlasOrcid[0000-0002-6965-7380]{S.~Mondal}$^\textrm{\scriptsize 136}$,    
\AtlasOrcid[0000-0002-3169-7117]{K.~M\"onig}$^\textrm{\scriptsize 44}$,    
\AtlasOrcid[0000-0002-2551-5751]{E.~Monnier}$^\textrm{\scriptsize 98}$,    
\AtlasOrcid{L.~Monsonis~Romero}$^\textrm{\scriptsize 168}$,    
\AtlasOrcid[0000-0002-5295-432X]{A.~Montalbano}$^\textrm{\scriptsize 147}$,    
\AtlasOrcid[0000-0001-9213-904X]{J.~Montejo~Berlingen}$^\textrm{\scriptsize 34}$,    
\AtlasOrcid[0000-0001-5010-886X]{M.~Montella}$^\textrm{\scriptsize 122}$,    
\AtlasOrcid[0000-0002-6974-1443]{F.~Monticelli}$^\textrm{\scriptsize 86}$,    
\AtlasOrcid[0000-0003-0047-7215]{N.~Morange}$^\textrm{\scriptsize 62}$,    
\AtlasOrcid[0000-0002-1986-5720]{A.L.~Moreira~De~Carvalho}$^\textrm{\scriptsize 134a}$,    
\AtlasOrcid[0000-0003-1113-3645]{M.~Moreno~Ll\'acer}$^\textrm{\scriptsize 168}$,    
\AtlasOrcid[0000-0002-5719-7655]{C.~Moreno~Martinez}$^\textrm{\scriptsize 12}$,    
\AtlasOrcid[0000-0001-7139-7912]{P.~Morettini}$^\textrm{\scriptsize 53b}$,    
\AtlasOrcid[0000-0002-7834-4781]{S.~Morgenstern}$^\textrm{\scriptsize 172}$,    
\AtlasOrcid[0000-0002-0693-4133]{D.~Mori}$^\textrm{\scriptsize 147}$,    
\AtlasOrcid[0000-0001-9324-057X]{M.~Morii}$^\textrm{\scriptsize 57}$,    
\AtlasOrcid[0000-0003-2129-1372]{M.~Morinaga}$^\textrm{\scriptsize 158}$,    
\AtlasOrcid[0000-0001-8715-8780]{V.~Morisbak}$^\textrm{\scriptsize 128}$,    
\AtlasOrcid[0000-0003-0373-1346]{A.K.~Morley}$^\textrm{\scriptsize 34}$,    
\AtlasOrcid[0000-0002-2929-3869]{A.P.~Morris}$^\textrm{\scriptsize 92}$,    
\AtlasOrcid[0000-0003-2061-2904]{L.~Morvaj}$^\textrm{\scriptsize 34}$,    
\AtlasOrcid[0000-0001-6993-9698]{P.~Moschovakos}$^\textrm{\scriptsize 34}$,    
\AtlasOrcid[0000-0001-6750-5060]{B.~Moser}$^\textrm{\scriptsize 115}$,    
\AtlasOrcid{M.~Mosidze}$^\textrm{\scriptsize 154b}$,    
\AtlasOrcid[0000-0001-6508-3968]{T.~Moskalets}$^\textrm{\scriptsize 50}$,    
\AtlasOrcid[0000-0002-7926-7650]{P.~Moskvitina}$^\textrm{\scriptsize 114}$,    
\AtlasOrcid[0000-0002-6729-4803]{J.~Moss}$^\textrm{\scriptsize 29,n}$,    
\AtlasOrcid[0000-0003-4449-6178]{E.J.W.~Moyse}$^\textrm{\scriptsize 99}$,    
\AtlasOrcid[0000-0002-1786-2075]{S.~Muanza}$^\textrm{\scriptsize 98}$,    
\AtlasOrcid[0000-0001-5099-4718]{J.~Mueller}$^\textrm{\scriptsize 133}$,    
\AtlasOrcid[0000-0002-5835-0690]{R.~Mueller}$^\textrm{\scriptsize 18}$,    
\AtlasOrcid[0000-0001-6223-2497]{D.~Muenstermann}$^\textrm{\scriptsize 87}$,    
\AtlasOrcid[0000-0001-6771-0937]{G.A.~Mullier}$^\textrm{\scriptsize 94}$,    
\AtlasOrcid{J.J.~Mullin}$^\textrm{\scriptsize 131}$,    
\AtlasOrcid[0000-0002-2567-7857]{D.P.~Mungo}$^\textrm{\scriptsize 66a,66b}$,    
\AtlasOrcid[0000-0002-2441-3366]{J.L.~Munoz~Martinez}$^\textrm{\scriptsize 12}$,    
\AtlasOrcid[0000-0002-6374-458X]{F.J.~Munoz~Sanchez}$^\textrm{\scriptsize 97}$,    
\AtlasOrcid[0000-0002-2388-1969]{M.~Murin}$^\textrm{\scriptsize 97}$,    
\AtlasOrcid[0000-0001-9686-2139]{P.~Murin}$^\textrm{\scriptsize 26b}$,    
\AtlasOrcid[0000-0003-1710-6306]{W.J.~Murray}$^\textrm{\scriptsize 172,138}$,    
\AtlasOrcid[0000-0001-5399-2478]{A.~Murrone}$^\textrm{\scriptsize 66a,66b}$,    
\AtlasOrcid[0000-0002-2585-3793]{J.M.~Muse}$^\textrm{\scriptsize 123}$,    
\AtlasOrcid[0000-0001-8442-2718]{M.~Mu\v{s}kinja}$^\textrm{\scriptsize 16}$,    
\AtlasOrcid[0000-0002-3504-0366]{C.~Mwewa}$^\textrm{\scriptsize 27}$,    
\AtlasOrcid[0000-0003-4189-4250]{A.G.~Myagkov}$^\textrm{\scriptsize 118,ae}$,    
\AtlasOrcid[0000-0003-1691-4643]{A.J.~Myers}$^\textrm{\scriptsize 7}$,    
\AtlasOrcid{A.A.~Myers}$^\textrm{\scriptsize 133}$,    
\AtlasOrcid[0000-0002-2562-0930]{G.~Myers}$^\textrm{\scriptsize 63}$,    
\AtlasOrcid[0000-0003-0982-3380]{M.~Myska}$^\textrm{\scriptsize 136}$,    
\AtlasOrcid[0000-0003-1024-0932]{B.P.~Nachman}$^\textrm{\scriptsize 16}$,    
\AtlasOrcid[0000-0002-2191-2725]{O.~Nackenhorst}$^\textrm{\scriptsize 45}$,    
\AtlasOrcid[0000-0001-6480-6079]{A.Nag~Nag}$^\textrm{\scriptsize 46}$,    
\AtlasOrcid[0000-0002-4285-0578]{K.~Nagai}$^\textrm{\scriptsize 129}$,    
\AtlasOrcid[0000-0003-2741-0627]{K.~Nagano}$^\textrm{\scriptsize 79}$,    
\AtlasOrcid[0000-0003-0056-6613]{J.L.~Nagle}$^\textrm{\scriptsize 27}$,    
\AtlasOrcid[0000-0001-5420-9537]{E.~Nagy}$^\textrm{\scriptsize 98}$,    
\AtlasOrcid[0000-0003-3561-0880]{A.M.~Nairz}$^\textrm{\scriptsize 34}$,    
\AtlasOrcid[0000-0003-3133-7100]{Y.~Nakahama}$^\textrm{\scriptsize 112}$,    
\AtlasOrcid[0000-0002-1560-0434]{K.~Nakamura}$^\textrm{\scriptsize 79}$,    
\AtlasOrcid[0000-0003-0703-103X]{H.~Nanjo}$^\textrm{\scriptsize 127}$,    
\AtlasOrcid[0000-0002-8686-5923]{F.~Napolitano}$^\textrm{\scriptsize 59a}$,    
\AtlasOrcid[0000-0002-8642-5119]{R.~Narayan}$^\textrm{\scriptsize 40}$,    
\AtlasOrcid[0000-0001-6042-6781]{E.A.~Narayanan}$^\textrm{\scriptsize 113}$,    
\AtlasOrcid[0000-0001-6412-4801]{I.~Naryshkin}$^\textrm{\scriptsize 132}$,    
\AtlasOrcid[0000-0001-9191-8164]{M.~Naseri}$^\textrm{\scriptsize 32}$,    
\AtlasOrcid[0000-0002-8098-4948]{C.~Nass}$^\textrm{\scriptsize 22}$,    
\AtlasOrcid[0000-0001-7372-8316]{T.~Naumann}$^\textrm{\scriptsize 44}$,    
\AtlasOrcid[0000-0002-5108-0042]{G.~Navarro}$^\textrm{\scriptsize 20a}$,    
\AtlasOrcid[0000-0002-4172-7965]{J.~Navarro-Gonzalez}$^\textrm{\scriptsize 168}$,    
\AtlasOrcid[0000-0001-6988-0606]{R.~Nayak}$^\textrm{\scriptsize 156}$,    
\AtlasOrcid[0000-0002-5910-4117]{P.Y.~Nechaeva}$^\textrm{\scriptsize 107}$,    
\AtlasOrcid[0000-0002-2684-9024]{F.~Nechansky}$^\textrm{\scriptsize 44}$,    
\AtlasOrcid[0000-0003-0056-8651]{T.J.~Neep}$^\textrm{\scriptsize 19}$,    
\AtlasOrcid[0000-0002-7386-901X]{A.~Negri}$^\textrm{\scriptsize 68a,68b}$,    
\AtlasOrcid[0000-0003-0101-6963]{M.~Negrini}$^\textrm{\scriptsize 21b}$,    
\AtlasOrcid[0000-0002-5171-8579]{C.~Nellist}$^\textrm{\scriptsize 114}$,    
\AtlasOrcid[0000-0002-5713-3803]{C.~Nelson}$^\textrm{\scriptsize 100}$,    
\AtlasOrcid[0000-0003-4194-1790]{K.~Nelson}$^\textrm{\scriptsize 102}$,    
\AtlasOrcid[0000-0001-8978-7150]{S.~Nemecek}$^\textrm{\scriptsize 135}$,    
\AtlasOrcid[0000-0001-7316-0118]{M.~Nessi}$^\textrm{\scriptsize 34,f}$,    
\AtlasOrcid[0000-0001-8434-9274]{M.S.~Neubauer}$^\textrm{\scriptsize 167}$,    
\AtlasOrcid[0000-0002-3819-2453]{F.~Neuhaus}$^\textrm{\scriptsize 96}$,    
\AtlasOrcid[0000-0002-8565-0015]{J.~Neundorf}$^\textrm{\scriptsize 44}$,    
\AtlasOrcid[0000-0001-8026-3836]{R.~Newhouse}$^\textrm{\scriptsize 169}$,    
\AtlasOrcid[0000-0002-6252-266X]{P.R.~Newman}$^\textrm{\scriptsize 19}$,    
\AtlasOrcid[0000-0001-8190-4017]{C.W.~Ng}$^\textrm{\scriptsize 133}$,    
\AtlasOrcid{Y.S.~Ng}$^\textrm{\scriptsize 17}$,    
\AtlasOrcid[0000-0001-9135-1321]{Y.W.Y.~Ng}$^\textrm{\scriptsize 165}$,    
\AtlasOrcid[0000-0002-5807-8535]{B.~Ngair}$^\textrm{\scriptsize 33e}$,    
\AtlasOrcid[0000-0002-4326-9283]{H.D.N.~Nguyen}$^\textrm{\scriptsize 106}$,    
\AtlasOrcid[0000-0002-2157-9061]{R.B.~Nickerson}$^\textrm{\scriptsize 129}$,    
\AtlasOrcid[0000-0003-3723-1745]{R.~Nicolaidou}$^\textrm{\scriptsize 139}$,    
\AtlasOrcid[0000-0002-9341-6907]{D.S.~Nielsen}$^\textrm{\scriptsize 38}$,    
\AtlasOrcid[0000-0002-9175-4419]{J.~Nielsen}$^\textrm{\scriptsize 140}$,    
\AtlasOrcid[0000-0003-4222-8284]{M.~Niemeyer}$^\textrm{\scriptsize 51}$,    
\AtlasOrcid[0000-0003-1267-7740]{N.~Nikiforou}$^\textrm{\scriptsize 10}$,    
\AtlasOrcid[0000-0001-6545-1820]{V.~Nikolaenko}$^\textrm{\scriptsize 118,ae}$,    
\AtlasOrcid[0000-0003-1681-1118]{I.~Nikolic-Audit}$^\textrm{\scriptsize 130}$,    
\AtlasOrcid[0000-0002-3048-489X]{K.~Nikolopoulos}$^\textrm{\scriptsize 19}$,    
\AtlasOrcid[0000-0002-6848-7463]{P.~Nilsson}$^\textrm{\scriptsize 27}$,    
\AtlasOrcid[0000-0003-3108-9477]{H.R.~Nindhito}$^\textrm{\scriptsize 52}$,    
\AtlasOrcid[0000-0002-5080-2293]{A.~Nisati}$^\textrm{\scriptsize 70a}$,    
\AtlasOrcid[0000-0002-9048-1332]{N.~Nishu}$^\textrm{\scriptsize 2}$,    
\AtlasOrcid[0000-0003-2257-0074]{R.~Nisius}$^\textrm{\scriptsize 111}$,    
\AtlasOrcid[0000-0002-9234-4833]{T.~Nitta}$^\textrm{\scriptsize 173}$,    
\AtlasOrcid[0000-0002-5809-325X]{T.~Nobe}$^\textrm{\scriptsize 158}$,    
\AtlasOrcid[0000-0001-8889-427X]{D.L.~Noel}$^\textrm{\scriptsize 30}$,    
\AtlasOrcid[0000-0002-3113-3127]{Y.~Noguchi}$^\textrm{\scriptsize 83}$,    
\AtlasOrcid[0000-0002-7406-1100]{I.~Nomidis}$^\textrm{\scriptsize 130}$,    
\AtlasOrcid{M.A.~Nomura}$^\textrm{\scriptsize 27}$,    
\AtlasOrcid[0000-0001-7984-5783]{M.B.~Norfolk}$^\textrm{\scriptsize 144}$,    
\AtlasOrcid[0000-0002-4129-5736]{R.R.B.~Norisam}$^\textrm{\scriptsize 92}$,    
\AtlasOrcid[0000-0002-3195-8903]{J.~Novak}$^\textrm{\scriptsize 89}$,    
\AtlasOrcid[0000-0002-3053-0913]{T.~Novak}$^\textrm{\scriptsize 44}$,    
\AtlasOrcid[0000-0001-6536-0179]{O.~Novgorodova}$^\textrm{\scriptsize 46}$,    
\AtlasOrcid[0000-0001-5165-8425]{L.~Novotny}$^\textrm{\scriptsize 136}$,    
\AtlasOrcid[0000-0002-1630-694X]{R.~Novotny}$^\textrm{\scriptsize 113}$,    
\AtlasOrcid{L.~Nozka}$^\textrm{\scriptsize 125}$,    
\AtlasOrcid[0000-0001-9252-6509]{K.~Ntekas}$^\textrm{\scriptsize 165}$,    
\AtlasOrcid{E.~Nurse}$^\textrm{\scriptsize 92}$,    
\AtlasOrcid[0000-0003-2866-1049]{F.G.~Oakham}$^\textrm{\scriptsize 32,aj}$,    
\AtlasOrcid[0000-0003-2262-0780]{J.~Ocariz}$^\textrm{\scriptsize 130}$,    
\AtlasOrcid[0000-0002-2024-5609]{A.~Ochi}$^\textrm{\scriptsize 80}$,    
\AtlasOrcid[0000-0001-6156-1790]{I.~Ochoa}$^\textrm{\scriptsize 134a}$,    
\AtlasOrcid[0000-0001-7376-5555]{J.P.~Ochoa-Ricoux}$^\textrm{\scriptsize 141a}$,    
\AtlasOrcid[0000-0001-5836-768X]{S.~Oda}$^\textrm{\scriptsize 85}$,    
\AtlasOrcid[0000-0002-1227-1401]{S.~Odaka}$^\textrm{\scriptsize 79}$,    
\AtlasOrcid[0000-0001-8763-0096]{S.~Oerdek}$^\textrm{\scriptsize 166}$,    
\AtlasOrcid[0000-0002-6025-4833]{A.~Ogrodnik}$^\textrm{\scriptsize 81a}$,    
\AtlasOrcid[0000-0001-9025-0422]{A.~Oh}$^\textrm{\scriptsize 97}$,    
\AtlasOrcid[0000-0002-8015-7512]{C.C.~Ohm}$^\textrm{\scriptsize 149}$,    
\AtlasOrcid[0000-0002-2173-3233]{H.~Oide}$^\textrm{\scriptsize 159}$,    
\AtlasOrcid[0000-0001-6930-7789]{R.~Oishi}$^\textrm{\scriptsize 158}$,    
\AtlasOrcid[0000-0002-3834-7830]{M.L.~Ojeda}$^\textrm{\scriptsize 44}$,    
\AtlasOrcid[0000-0003-2677-5827]{Y.~Okazaki}$^\textrm{\scriptsize 83}$,    
\AtlasOrcid{M.W.~O'Keefe}$^\textrm{\scriptsize 88}$,    
\AtlasOrcid[0000-0002-7613-5572]{Y.~Okumura}$^\textrm{\scriptsize 158}$,    
\AtlasOrcid{A.~Olariu}$^\textrm{\scriptsize 25b}$,    
\AtlasOrcid[0000-0002-9320-8825]{L.F.~Oleiro~Seabra}$^\textrm{\scriptsize 134a}$,    
\AtlasOrcid[0000-0003-4616-6973]{S.A.~Olivares~Pino}$^\textrm{\scriptsize 141c}$,    
\AtlasOrcid[0000-0002-8601-2074]{D.~Oliveira~Damazio}$^\textrm{\scriptsize 27}$,    
\AtlasOrcid[0000-0002-1943-9561]{D.~Oliveira~Goncalves}$^\textrm{\scriptsize 78a}$,    
\AtlasOrcid[0000-0002-0713-6627]{J.L.~Oliver}$^\textrm{\scriptsize 165}$,    
\AtlasOrcid[0000-0003-4154-8139]{M.J.R.~Olsson}$^\textrm{\scriptsize 165}$,    
\AtlasOrcid[0000-0003-3368-5475]{A.~Olszewski}$^\textrm{\scriptsize 82}$,    
\AtlasOrcid[0000-0003-0520-9500]{J.~Olszowska}$^\textrm{\scriptsize 82}$,    
\AtlasOrcid[0000-0001-8772-1705]{\"O.O.~\"Oncel}$^\textrm{\scriptsize 22}$,    
\AtlasOrcid[0000-0003-0325-472X]{D.C.~O'Neil}$^\textrm{\scriptsize 147}$,    
\AtlasOrcid[0000-0002-8104-7227]{A.P.~O'neill}$^\textrm{\scriptsize 129}$,    
\AtlasOrcid[0000-0003-3471-2703]{A.~Onofre}$^\textrm{\scriptsize 134a,134e}$,    
\AtlasOrcid[0000-0003-4201-7997]{P.U.E.~Onyisi}$^\textrm{\scriptsize 10}$,    
\AtlasOrcid{R.G.~Oreamuno~Madriz}$^\textrm{\scriptsize 116}$,    
\AtlasOrcid[0000-0001-6203-2209]{M.J.~Oreglia}$^\textrm{\scriptsize 35}$,    
\AtlasOrcid[0000-0002-4753-4048]{G.E.~Orellana}$^\textrm{\scriptsize 86}$,    
\AtlasOrcid[0000-0001-5103-5527]{D.~Orestano}$^\textrm{\scriptsize 72a,72b}$,    
\AtlasOrcid[0000-0003-0616-245X]{N.~Orlando}$^\textrm{\scriptsize 12}$,    
\AtlasOrcid[0000-0002-8690-9746]{R.S.~Orr}$^\textrm{\scriptsize 161}$,    
\AtlasOrcid[0000-0001-7183-1205]{V.~O'Shea}$^\textrm{\scriptsize 55}$,    
\AtlasOrcid[0000-0001-5091-9216]{R.~Ospanov}$^\textrm{\scriptsize 58a}$,    
\AtlasOrcid[0000-0003-4803-5280]{G.~Otero~y~Garzon}$^\textrm{\scriptsize 28}$,    
\AtlasOrcid[0000-0003-0760-5988]{H.~Otono}$^\textrm{\scriptsize 85}$,    
\AtlasOrcid[0000-0003-1052-7925]{P.S.~Ott}$^\textrm{\scriptsize 59a}$,    
\AtlasOrcid[0000-0001-8083-6411]{G.J.~Ottino}$^\textrm{\scriptsize 16}$,    
\AtlasOrcid[0000-0002-2954-1420]{M.~Ouchrif}$^\textrm{\scriptsize 33d}$,    
\AtlasOrcid[0000-0002-0582-3765]{J.~Ouellette}$^\textrm{\scriptsize 27}$,    
\AtlasOrcid[0000-0002-9404-835X]{F.~Ould-Saada}$^\textrm{\scriptsize 128}$,    
\AtlasOrcid[0000-0001-6818-5994]{A.~Ouraou}$^\textrm{\scriptsize 139,*}$,    
\AtlasOrcid[0000-0002-8186-0082]{Q.~Ouyang}$^\textrm{\scriptsize 13a}$,    
\AtlasOrcid[0000-0001-6820-0488]{M.~Owen}$^\textrm{\scriptsize 55}$,    
\AtlasOrcid[0000-0002-2684-1399]{R.E.~Owen}$^\textrm{\scriptsize 138}$,    
\AtlasOrcid[0000-0002-5533-9621]{K.Y.~Oyulmaz}$^\textrm{\scriptsize 11c}$,    
\AtlasOrcid[0000-0003-4643-6347]{V.E.~Ozcan}$^\textrm{\scriptsize 11c}$,    
\AtlasOrcid[0000-0003-1125-6784]{N.~Ozturk}$^\textrm{\scriptsize 7}$,    
\AtlasOrcid[0000-0001-6533-6144]{S.~Ozturk}$^\textrm{\scriptsize 11c}$,    
\AtlasOrcid[0000-0002-0148-7207]{J.~Pacalt}$^\textrm{\scriptsize 125}$,    
\AtlasOrcid[0000-0002-2325-6792]{H.A.~Pacey}$^\textrm{\scriptsize 30}$,    
\AtlasOrcid[0000-0002-8332-243X]{K.~Pachal}$^\textrm{\scriptsize 47}$,    
\AtlasOrcid[0000-0001-8210-1734]{A.~Pacheco~Pages}$^\textrm{\scriptsize 12}$,    
\AtlasOrcid[0000-0001-7951-0166]{C.~Padilla~Aranda}$^\textrm{\scriptsize 12}$,    
\AtlasOrcid[0000-0003-0999-5019]{S.~Pagan~Griso}$^\textrm{\scriptsize 16}$,    
\AtlasOrcid[0000-0003-0278-9941]{G.~Palacino}$^\textrm{\scriptsize 63}$,    
\AtlasOrcid[0000-0002-4225-387X]{S.~Palazzo}$^\textrm{\scriptsize 48}$,    
\AtlasOrcid[0000-0002-4110-096X]{S.~Palestini}$^\textrm{\scriptsize 34}$,    
\AtlasOrcid[0000-0002-7185-3540]{M.~Palka}$^\textrm{\scriptsize 81b}$,    
\AtlasOrcid[0000-0001-6201-2785]{P.~Palni}$^\textrm{\scriptsize 81a}$,    
\AtlasOrcid[0000-0001-5732-9948]{D.K.~Panchal}$^\textrm{\scriptsize 10}$,    
\AtlasOrcid[0000-0003-3838-1307]{C.E.~Pandini}$^\textrm{\scriptsize 52}$,    
\AtlasOrcid[0000-0003-2605-8940]{J.G.~Panduro~Vazquez}$^\textrm{\scriptsize 91}$,    
\AtlasOrcid[0000-0003-2149-3791]{P.~Pani}$^\textrm{\scriptsize 44}$,    
\AtlasOrcid[0000-0002-0352-4833]{G.~Panizzo}$^\textrm{\scriptsize 64a,64c}$,    
\AtlasOrcid[0000-0002-9281-1972]{L.~Paolozzi}$^\textrm{\scriptsize 52}$,    
\AtlasOrcid[0000-0003-3160-3077]{C.~Papadatos}$^\textrm{\scriptsize 106}$,    
\AtlasOrcid[0000-0003-1499-3990]{S.~Parajuli}$^\textrm{\scriptsize 40}$,    
\AtlasOrcid[0000-0002-6492-3061]{A.~Paramonov}$^\textrm{\scriptsize 5}$,    
\AtlasOrcid[0000-0002-2858-9182]{C.~Paraskevopoulos}$^\textrm{\scriptsize 9}$,    
\AtlasOrcid[0000-0002-3179-8524]{D.~Paredes~Hernandez}$^\textrm{\scriptsize 60b}$,    
\AtlasOrcid[0000-0001-8487-9603]{S.R.~Paredes~Saenz}$^\textrm{\scriptsize 129}$,    
\AtlasOrcid[0000-0001-9367-8061]{B.~Parida}$^\textrm{\scriptsize 174}$,    
\AtlasOrcid[0000-0002-1910-0541]{T.H.~Park}$^\textrm{\scriptsize 161}$,    
\AtlasOrcid[0000-0001-9410-3075]{A.J.~Parker}$^\textrm{\scriptsize 29}$,    
\AtlasOrcid[0000-0001-9798-8411]{M.A.~Parker}$^\textrm{\scriptsize 30}$,    
\AtlasOrcid[0000-0002-7160-4720]{F.~Parodi}$^\textrm{\scriptsize 53b,53a}$,    
\AtlasOrcid[0000-0001-5954-0974]{E.W.~Parrish}$^\textrm{\scriptsize 116}$,    
\AtlasOrcid[0000-0002-9470-6017]{J.A.~Parsons}$^\textrm{\scriptsize 37}$,    
\AtlasOrcid[0000-0002-4858-6560]{U.~Parzefall}$^\textrm{\scriptsize 50}$,    
\AtlasOrcid[0000-0003-4701-9481]{L.~Pascual~Dominguez}$^\textrm{\scriptsize 156}$,    
\AtlasOrcid[0000-0003-3167-8773]{V.R.~Pascuzzi}$^\textrm{\scriptsize 16}$,    
\AtlasOrcid[0000-0003-0707-7046]{F.~Pasquali}$^\textrm{\scriptsize 115}$,    
\AtlasOrcid[0000-0001-8160-2545]{E.~Pasqualucci}$^\textrm{\scriptsize 70a}$,    
\AtlasOrcid[0000-0001-9200-5738]{S.~Passaggio}$^\textrm{\scriptsize 53b}$,    
\AtlasOrcid[0000-0001-5962-7826]{F.~Pastore}$^\textrm{\scriptsize 91}$,    
\AtlasOrcid[0000-0003-2987-2964]{P.~Pasuwan}$^\textrm{\scriptsize 43a,43b}$,    
\AtlasOrcid[0000-0002-0598-5035]{J.R.~Pater}$^\textrm{\scriptsize 97}$,    
\AtlasOrcid[0000-0001-9861-2942]{A.~Pathak}$^\textrm{\scriptsize 175}$,    
\AtlasOrcid{J.~Patton}$^\textrm{\scriptsize 88}$,    
\AtlasOrcid[0000-0001-9082-035X]{T.~Pauly}$^\textrm{\scriptsize 34}$,    
\AtlasOrcid[0000-0002-5205-4065]{J.~Pearkes}$^\textrm{\scriptsize 148}$,    
\AtlasOrcid[0000-0003-4281-0119]{M.~Pedersen}$^\textrm{\scriptsize 128}$,    
\AtlasOrcid[0000-0003-3924-8276]{L.~Pedraza~Diaz}$^\textrm{\scriptsize 114}$,    
\AtlasOrcid[0000-0002-7139-9587]{R.~Pedro}$^\textrm{\scriptsize 134a}$,    
\AtlasOrcid[0000-0002-8162-6667]{T.~Peiffer}$^\textrm{\scriptsize 51}$,    
\AtlasOrcid[0000-0003-0907-7592]{S.V.~Peleganchuk}$^\textrm{\scriptsize 117b,117a}$,    
\AtlasOrcid[0000-0002-5433-3981]{O.~Penc}$^\textrm{\scriptsize 135}$,    
\AtlasOrcid[0000-0002-3451-2237]{C.~Peng}$^\textrm{\scriptsize 60b}$,    
\AtlasOrcid[0000-0002-3461-0945]{H.~Peng}$^\textrm{\scriptsize 58a}$,    
\AtlasOrcid[0000-0002-0928-3129]{M.~Penzin}$^\textrm{\scriptsize 160}$,    
\AtlasOrcid[0000-0003-1664-5658]{B.S.~Peralva}$^\textrm{\scriptsize 78a}$,    
\AtlasOrcid[0000-0003-3424-7338]{A.P.~Pereira~Peixoto}$^\textrm{\scriptsize 134a}$,    
\AtlasOrcid[0000-0001-7913-3313]{L.~Pereira~Sanchez}$^\textrm{\scriptsize 43a,43b}$,    
\AtlasOrcid[0000-0001-8732-6908]{D.V.~Perepelitsa}$^\textrm{\scriptsize 27}$,    
\AtlasOrcid[0000-0003-0426-6538]{E.~Perez~Codina}$^\textrm{\scriptsize 162a}$,    
\AtlasOrcid[0000-0003-3451-9938]{M.~Perganti}$^\textrm{\scriptsize 9}$,    
\AtlasOrcid[0000-0003-3715-0523]{L.~Perini}$^\textrm{\scriptsize 66a,66b}$,    
\AtlasOrcid[0000-0001-6418-8784]{H.~Pernegger}$^\textrm{\scriptsize 34}$,    
\AtlasOrcid[0000-0003-4955-5130]{S.~Perrella}$^\textrm{\scriptsize 34}$,    
\AtlasOrcid[0000-0001-6343-447X]{A.~Perrevoort}$^\textrm{\scriptsize 115}$,    
\AtlasOrcid[0000-0002-7654-1677]{K.~Peters}$^\textrm{\scriptsize 44}$,    
\AtlasOrcid[0000-0003-1702-7544]{R.F.Y.~Peters}$^\textrm{\scriptsize 97}$,    
\AtlasOrcid[0000-0002-7380-6123]{B.A.~Petersen}$^\textrm{\scriptsize 34}$,    
\AtlasOrcid[0000-0003-0221-3037]{T.C.~Petersen}$^\textrm{\scriptsize 38}$,    
\AtlasOrcid[0000-0002-3059-735X]{E.~Petit}$^\textrm{\scriptsize 98}$,    
\AtlasOrcid[0000-0002-5575-6476]{V.~Petousis}$^\textrm{\scriptsize 136}$,    
\AtlasOrcid[0000-0001-5957-6133]{C.~Petridou}$^\textrm{\scriptsize 157}$,    
\AtlasOrcid{P.~Petroff}$^\textrm{\scriptsize 62}$,    
\AtlasOrcid[0000-0002-5278-2206]{F.~Petrucci}$^\textrm{\scriptsize 72a,72b}$,    
\AtlasOrcid[0000-0003-0533-2277]{A.~Petrukhin}$^\textrm{\scriptsize 146}$,    
\AtlasOrcid[0000-0001-9208-3218]{M.~Pettee}$^\textrm{\scriptsize 177}$,    
\AtlasOrcid[0000-0001-7451-3544]{N.E.~Pettersson}$^\textrm{\scriptsize 34}$,    
\AtlasOrcid[0000-0002-0654-8398]{K.~Petukhova}$^\textrm{\scriptsize 137}$,    
\AtlasOrcid[0000-0001-8933-8689]{A.~Peyaud}$^\textrm{\scriptsize 139}$,    
\AtlasOrcid[0000-0003-3344-791X]{R.~Pezoa}$^\textrm{\scriptsize 141d}$,    
\AtlasOrcid[0000-0002-3802-8944]{L.~Pezzotti}$^\textrm{\scriptsize 34}$,    
\AtlasOrcid[0000-0002-6653-1555]{G.~Pezzullo}$^\textrm{\scriptsize 177}$,    
\AtlasOrcid[0000-0002-8859-1313]{T.~Pham}$^\textrm{\scriptsize 101}$,    
\AtlasOrcid[0000-0003-3651-4081]{P.W.~Phillips}$^\textrm{\scriptsize 138}$,    
\AtlasOrcid[0000-0002-5367-8961]{M.W.~Phipps}$^\textrm{\scriptsize 167}$,    
\AtlasOrcid[0000-0002-4531-2900]{G.~Piacquadio}$^\textrm{\scriptsize 150}$,    
\AtlasOrcid[0000-0001-9233-5892]{E.~Pianori}$^\textrm{\scriptsize 16}$,    
\AtlasOrcid[0000-0002-3664-8912]{F.~Piazza}$^\textrm{\scriptsize 66a,66b}$,    
\AtlasOrcid[0000-0001-5070-4717]{A.~Picazio}$^\textrm{\scriptsize 99}$,    
\AtlasOrcid[0000-0001-7850-8005]{R.~Piegaia}$^\textrm{\scriptsize 28}$,    
\AtlasOrcid[0000-0003-1381-5949]{D.~Pietreanu}$^\textrm{\scriptsize 25b}$,    
\AtlasOrcid[0000-0003-2417-2176]{J.E.~Pilcher}$^\textrm{\scriptsize 35}$,    
\AtlasOrcid[0000-0001-8007-0778]{A.D.~Pilkington}$^\textrm{\scriptsize 97}$,    
\AtlasOrcid[0000-0002-5282-5050]{M.~Pinamonti}$^\textrm{\scriptsize 64a,64c}$,    
\AtlasOrcid[0000-0002-2397-4196]{J.L.~Pinfold}$^\textrm{\scriptsize 2}$,    
\AtlasOrcid{C.~Pitman~Donaldson}$^\textrm{\scriptsize 92}$,    
\AtlasOrcid[0000-0001-5193-1567]{D.A.~Pizzi}$^\textrm{\scriptsize 32}$,    
\AtlasOrcid[0000-0002-1814-2758]{L.~Pizzimento}$^\textrm{\scriptsize 71a,71b}$,    
\AtlasOrcid[0000-0001-8891-1842]{A.~Pizzini}$^\textrm{\scriptsize 115}$,    
\AtlasOrcid[0000-0002-9461-3494]{M.-A.~Pleier}$^\textrm{\scriptsize 27}$,    
\AtlasOrcid{V.~Plesanovs}$^\textrm{\scriptsize 50}$,    
\AtlasOrcid[0000-0001-5435-497X]{V.~Pleskot}$^\textrm{\scriptsize 137}$,    
\AtlasOrcid{E.~Plotnikova}$^\textrm{\scriptsize 77}$,    
\AtlasOrcid[0000-0002-1142-3215]{P.~Podberezko}$^\textrm{\scriptsize 117b,117a}$,    
\AtlasOrcid[0000-0002-3304-0987]{R.~Poettgen}$^\textrm{\scriptsize 94}$,    
\AtlasOrcid[0000-0002-7324-9320]{R.~Poggi}$^\textrm{\scriptsize 52}$,    
\AtlasOrcid[0000-0003-3210-6646]{L.~Poggioli}$^\textrm{\scriptsize 130}$,    
\AtlasOrcid[0000-0002-3817-0879]{I.~Pogrebnyak}$^\textrm{\scriptsize 103}$,    
\AtlasOrcid[0000-0002-3332-1113]{D.~Pohl}$^\textrm{\scriptsize 22}$,    
\AtlasOrcid[0000-0002-7915-0161]{I.~Pokharel}$^\textrm{\scriptsize 51}$,    
\AtlasOrcid[0000-0001-8636-0186]{G.~Polesello}$^\textrm{\scriptsize 68a}$,    
\AtlasOrcid[0000-0002-4063-0408]{A.~Poley}$^\textrm{\scriptsize 147,162a}$,    
\AtlasOrcid[0000-0002-1290-220X]{A.~Policicchio}$^\textrm{\scriptsize 70a,70b}$,    
\AtlasOrcid[0000-0003-1036-3844]{R.~Polifka}$^\textrm{\scriptsize 137}$,    
\AtlasOrcid[0000-0002-4986-6628]{A.~Polini}$^\textrm{\scriptsize 21b}$,    
\AtlasOrcid[0000-0002-3690-3960]{C.S.~Pollard}$^\textrm{\scriptsize 129}$,    
\AtlasOrcid[0000-0001-6285-0658]{Z.B.~Pollock}$^\textrm{\scriptsize 122}$,    
\AtlasOrcid[0000-0002-4051-0828]{V.~Polychronakos}$^\textrm{\scriptsize 27}$,    
\AtlasOrcid[0000-0003-4213-1511]{D.~Ponomarenko}$^\textrm{\scriptsize 108}$,    
\AtlasOrcid[0000-0003-2284-3765]{L.~Pontecorvo}$^\textrm{\scriptsize 34}$,    
\AtlasOrcid[0000-0001-9275-4536]{S.~Popa}$^\textrm{\scriptsize 25a}$,    
\AtlasOrcid[0000-0001-9783-7736]{G.A.~Popeneciu}$^\textrm{\scriptsize 25d}$,    
\AtlasOrcid[0000-0002-9860-9185]{L.~Portales}$^\textrm{\scriptsize 4}$,    
\AtlasOrcid[0000-0002-7042-4058]{D.M.~Portillo~Quintero}$^\textrm{\scriptsize 162a}$,    
\AtlasOrcid[0000-0001-5424-9096]{S.~Pospisil}$^\textrm{\scriptsize 136}$,    
\AtlasOrcid[0000-0001-8797-012X]{P.~Postolache}$^\textrm{\scriptsize 25c}$,    
\AtlasOrcid[0000-0001-7839-9785]{K.~Potamianos}$^\textrm{\scriptsize 129}$,    
\AtlasOrcid[0000-0002-0375-6909]{I.N.~Potrap}$^\textrm{\scriptsize 77}$,    
\AtlasOrcid[0000-0002-9815-5208]{C.J.~Potter}$^\textrm{\scriptsize 30}$,    
\AtlasOrcid[0000-0002-0800-9902]{H.~Potti}$^\textrm{\scriptsize 1}$,    
\AtlasOrcid[0000-0001-7207-6029]{T.~Poulsen}$^\textrm{\scriptsize 44}$,    
\AtlasOrcid[0000-0001-8144-1964]{J.~Poveda}$^\textrm{\scriptsize 168}$,    
\AtlasOrcid[0000-0001-9381-7850]{T.D.~Powell}$^\textrm{\scriptsize 144}$,    
\AtlasOrcid[0000-0002-9244-0753]{G.~Pownall}$^\textrm{\scriptsize 44}$,    
\AtlasOrcid[0000-0002-3069-3077]{M.E.~Pozo~Astigarraga}$^\textrm{\scriptsize 34}$,    
\AtlasOrcid[0000-0003-1418-2012]{A.~Prades~Ibanez}$^\textrm{\scriptsize 168}$,    
\AtlasOrcid[0000-0002-2452-6715]{P.~Pralavorio}$^\textrm{\scriptsize 98}$,    
\AtlasOrcid[0000-0001-6778-9403]{M.M.~Prapa}$^\textrm{\scriptsize 42}$,    
\AtlasOrcid[0000-0002-0195-8005]{S.~Prell}$^\textrm{\scriptsize 76}$,    
\AtlasOrcid[0000-0003-2750-9977]{D.~Price}$^\textrm{\scriptsize 97}$,    
\AtlasOrcid[0000-0002-6866-3818]{M.~Primavera}$^\textrm{\scriptsize 65a}$,    
\AtlasOrcid[0000-0002-5085-2717]{M.A.~Principe~Martin}$^\textrm{\scriptsize 95}$,    
\AtlasOrcid[0000-0003-0323-8252]{M.L.~Proffitt}$^\textrm{\scriptsize 143}$,    
\AtlasOrcid[0000-0002-5237-0201]{N.~Proklova}$^\textrm{\scriptsize 108}$,    
\AtlasOrcid[0000-0002-2177-6401]{K.~Prokofiev}$^\textrm{\scriptsize 60c}$,    
\AtlasOrcid[0000-0001-6389-5399]{F.~Prokoshin}$^\textrm{\scriptsize 77}$,    
\AtlasOrcid[0000-0001-7432-8242]{S.~Protopopescu}$^\textrm{\scriptsize 27}$,    
\AtlasOrcid[0000-0003-1032-9945]{J.~Proudfoot}$^\textrm{\scriptsize 5}$,    
\AtlasOrcid[0000-0002-9235-2649]{M.~Przybycien}$^\textrm{\scriptsize 81a}$,    
\AtlasOrcid[0000-0002-7026-1412]{D.~Pudzha}$^\textrm{\scriptsize 132}$,    
\AtlasOrcid{P.~Puzo}$^\textrm{\scriptsize 62}$,    
\AtlasOrcid[0000-0002-6659-8506]{D.~Pyatiizbyantseva}$^\textrm{\scriptsize 108}$,    
\AtlasOrcid[0000-0003-4813-8167]{J.~Qian}$^\textrm{\scriptsize 102}$,    
\AtlasOrcid[0000-0002-6960-502X]{Y.~Qin}$^\textrm{\scriptsize 97}$,    
\AtlasOrcid[0000-0001-5047-3031]{T.~Qiu}$^\textrm{\scriptsize 90}$,    
\AtlasOrcid[0000-0002-0098-384X]{A.~Quadt}$^\textrm{\scriptsize 51}$,    
\AtlasOrcid[0000-0003-4643-515X]{M.~Queitsch-Maitland}$^\textrm{\scriptsize 34}$,    
\AtlasOrcid[0000-0003-1526-5848]{G.~Rabanal~Bolanos}$^\textrm{\scriptsize 57}$,    
\AtlasOrcid[0000-0002-4064-0489]{F.~Ragusa}$^\textrm{\scriptsize 66a,66b}$,    
\AtlasOrcid[0000-0002-5987-4648]{J.A.~Raine}$^\textrm{\scriptsize 52}$,    
\AtlasOrcid[0000-0001-6543-1520]{S.~Rajagopalan}$^\textrm{\scriptsize 27}$,    
\AtlasOrcid[0000-0003-3119-9924]{K.~Ran}$^\textrm{\scriptsize 13a,13d}$,    
\AtlasOrcid[0000-0002-5756-4558]{D.F.~Rassloff}$^\textrm{\scriptsize 59a}$,    
\AtlasOrcid[0000-0002-8527-7695]{D.M.~Rauch}$^\textrm{\scriptsize 44}$,    
\AtlasOrcid[0000-0002-0050-8053]{S.~Rave}$^\textrm{\scriptsize 96}$,    
\AtlasOrcid[0000-0002-1622-6640]{B.~Ravina}$^\textrm{\scriptsize 55}$,    
\AtlasOrcid[0000-0001-9348-4363]{I.~Ravinovich}$^\textrm{\scriptsize 174}$,    
\AtlasOrcid[0000-0001-8225-1142]{M.~Raymond}$^\textrm{\scriptsize 34}$,    
\AtlasOrcid[0000-0002-5751-6636]{A.L.~Read}$^\textrm{\scriptsize 128}$,    
\AtlasOrcid[0000-0002-3427-0688]{N.P.~Readioff}$^\textrm{\scriptsize 144}$,    
\AtlasOrcid[0000-0003-4461-3880]{D.M.~Rebuzzi}$^\textrm{\scriptsize 68a,68b}$,    
\AtlasOrcid[0000-0002-6437-9991]{G.~Redlinger}$^\textrm{\scriptsize 27}$,    
\AtlasOrcid[0000-0003-3504-4882]{K.~Reeves}$^\textrm{\scriptsize 41}$,    
\AtlasOrcid[0000-0001-5758-579X]{D.~Reikher}$^\textrm{\scriptsize 156}$,    
\AtlasOrcid{A.~Reiss}$^\textrm{\scriptsize 96}$,    
\AtlasOrcid[0000-0002-5471-0118]{A.~Rej}$^\textrm{\scriptsize 146}$,    
\AtlasOrcid[0000-0001-6139-2210]{C.~Rembser}$^\textrm{\scriptsize 34}$,    
\AtlasOrcid[0000-0003-4021-6482]{A.~Renardi}$^\textrm{\scriptsize 44}$,    
\AtlasOrcid[0000-0002-0429-6959]{M.~Renda}$^\textrm{\scriptsize 25b}$,    
\AtlasOrcid{M.B.~Rendel}$^\textrm{\scriptsize 111}$,    
\AtlasOrcid[0000-0002-8485-3734]{A.G.~Rennie}$^\textrm{\scriptsize 55}$,    
\AtlasOrcid[0000-0003-2313-4020]{S.~Resconi}$^\textrm{\scriptsize 66a}$,    
\AtlasOrcid[0000-0002-6777-1761]{M.~Ressegotti}$^\textrm{\scriptsize 53b,53a}$,    
\AtlasOrcid[0000-0002-7739-6176]{E.D.~Resseguie}$^\textrm{\scriptsize 16}$,    
\AtlasOrcid[0000-0002-7092-3893]{S.~Rettie}$^\textrm{\scriptsize 92}$,    
\AtlasOrcid{B.~Reynolds}$^\textrm{\scriptsize 122}$,    
\AtlasOrcid[0000-0002-1506-5750]{E.~Reynolds}$^\textrm{\scriptsize 19}$,    
\AtlasOrcid[0000-0002-3308-8067]{M.~Rezaei~Estabragh}$^\textrm{\scriptsize 176}$,    
\AtlasOrcid[0000-0001-7141-0304]{O.L.~Rezanova}$^\textrm{\scriptsize 117b,117a}$,    
\AtlasOrcid[0000-0003-4017-9829]{P.~Reznicek}$^\textrm{\scriptsize 137}$,    
\AtlasOrcid[0000-0002-4222-9976]{E.~Ricci}$^\textrm{\scriptsize 73a,73b}$,    
\AtlasOrcid[0000-0001-8981-1966]{R.~Richter}$^\textrm{\scriptsize 111}$,    
\AtlasOrcid[0000-0001-6613-4448]{S.~Richter}$^\textrm{\scriptsize 44}$,    
\AtlasOrcid[0000-0002-3823-9039]{E.~Richter-Was}$^\textrm{\scriptsize 81b}$,    
\AtlasOrcid[0000-0002-2601-7420]{M.~Ridel}$^\textrm{\scriptsize 130}$,    
\AtlasOrcid[0000-0003-0290-0566]{P.~Rieck}$^\textrm{\scriptsize 111}$,    
\AtlasOrcid[0000-0002-4871-8543]{P.~Riedler}$^\textrm{\scriptsize 34}$,    
\AtlasOrcid[0000-0002-9169-0793]{O.~Rifki}$^\textrm{\scriptsize 44}$,    
\AtlasOrcid[0000-0002-3476-1575]{M.~Rijssenbeek}$^\textrm{\scriptsize 150}$,    
\AtlasOrcid[0000-0003-3590-7908]{A.~Rimoldi}$^\textrm{\scriptsize 68a,68b}$,    
\AtlasOrcid[0000-0003-1165-7940]{M.~Rimoldi}$^\textrm{\scriptsize 44}$,    
\AtlasOrcid[0000-0001-9608-9940]{L.~Rinaldi}$^\textrm{\scriptsize 21b,21a}$,    
\AtlasOrcid[0000-0002-1295-1538]{T.T.~Rinn}$^\textrm{\scriptsize 167}$,    
\AtlasOrcid[0000-0003-4931-0459]{M.P.~Rinnagel}$^\textrm{\scriptsize 110}$,    
\AtlasOrcid[0000-0002-4053-5144]{G.~Ripellino}$^\textrm{\scriptsize 149}$,    
\AtlasOrcid[0000-0002-3742-4582]{I.~Riu}$^\textrm{\scriptsize 12}$,    
\AtlasOrcid[0000-0002-7213-3844]{P.~Rivadeneira}$^\textrm{\scriptsize 44}$,    
\AtlasOrcid[0000-0002-8149-4561]{J.C.~Rivera~Vergara}$^\textrm{\scriptsize 170}$,    
\AtlasOrcid[0000-0002-2041-6236]{F.~Rizatdinova}$^\textrm{\scriptsize 124}$,    
\AtlasOrcid[0000-0001-9834-2671]{E.~Rizvi}$^\textrm{\scriptsize 90}$,    
\AtlasOrcid[0000-0001-6120-2325]{C.~Rizzi}$^\textrm{\scriptsize 52}$,    
\AtlasOrcid[0000-0001-5904-0582]{B.A.~Roberts}$^\textrm{\scriptsize 172}$,    
\AtlasOrcid[0000-0003-4096-8393]{S.H.~Robertson}$^\textrm{\scriptsize 100,y}$,    
\AtlasOrcid[0000-0002-1390-7141]{M.~Robin}$^\textrm{\scriptsize 44}$,    
\AtlasOrcid[0000-0001-6169-4868]{D.~Robinson}$^\textrm{\scriptsize 30}$,    
\AtlasOrcid{C.M.~Robles~Gajardo}$^\textrm{\scriptsize 141d}$,    
\AtlasOrcid[0000-0001-7701-8864]{M.~Robles~Manzano}$^\textrm{\scriptsize 96}$,    
\AtlasOrcid[0000-0002-1659-8284]{A.~Robson}$^\textrm{\scriptsize 55}$,    
\AtlasOrcid[0000-0002-3125-8333]{A.~Rocchi}$^\textrm{\scriptsize 71a,71b}$,    
\AtlasOrcid[0000-0002-3020-4114]{C.~Roda}$^\textrm{\scriptsize 69a,69b}$,    
\AtlasOrcid[0000-0002-4571-2509]{S.~Rodriguez~Bosca}$^\textrm{\scriptsize 59a}$,    
\AtlasOrcid[0000-0002-1590-2352]{A.~Rodriguez~Rodriguez}$^\textrm{\scriptsize 50}$,    
\AtlasOrcid[0000-0002-9609-3306]{A.M.~Rodr\'iguez~Vera}$^\textrm{\scriptsize 162b}$,    
\AtlasOrcid{S.~Roe}$^\textrm{\scriptsize 34}$,    
\AtlasOrcid[0000-0001-5933-9357]{A.R.~Roepe}$^\textrm{\scriptsize 123}$,    
\AtlasOrcid[0000-0002-5749-3876]{J.~Roggel}$^\textrm{\scriptsize 176}$,    
\AtlasOrcid[0000-0001-7744-9584]{O.~R{\o}hne}$^\textrm{\scriptsize 128}$,    
\AtlasOrcid[0000-0002-6888-9462]{R.A.~Rojas}$^\textrm{\scriptsize 141d}$,    
\AtlasOrcid[0000-0003-3397-6475]{B.~Roland}$^\textrm{\scriptsize 50}$,    
\AtlasOrcid[0000-0003-2084-369X]{C.P.A.~Roland}$^\textrm{\scriptsize 63}$,    
\AtlasOrcid[0000-0001-6479-3079]{J.~Roloff}$^\textrm{\scriptsize 27}$,    
\AtlasOrcid[0000-0001-9241-1189]{A.~Romaniouk}$^\textrm{\scriptsize 108}$,    
\AtlasOrcid[0000-0002-6609-7250]{M.~Romano}$^\textrm{\scriptsize 21b}$,    
\AtlasOrcid[0000-0001-9434-1380]{A.C.~Romero~Hernandez}$^\textrm{\scriptsize 167}$,    
\AtlasOrcid[0000-0003-2577-1875]{N.~Rompotis}$^\textrm{\scriptsize 88}$,    
\AtlasOrcid[0000-0002-8583-6063]{M.~Ronzani}$^\textrm{\scriptsize 120}$,    
\AtlasOrcid[0000-0001-7151-9983]{L.~Roos}$^\textrm{\scriptsize 130}$,    
\AtlasOrcid[0000-0003-0838-5980]{S.~Rosati}$^\textrm{\scriptsize 70a}$,    
\AtlasOrcid[0000-0001-7492-831X]{B.J.~Rosser}$^\textrm{\scriptsize 131}$,    
\AtlasOrcid[0000-0001-5493-6486]{E.~Rossi}$^\textrm{\scriptsize 161}$,    
\AtlasOrcid[0000-0002-2146-677X]{E.~Rossi}$^\textrm{\scriptsize 4}$,    
\AtlasOrcid[0000-0001-9476-9854]{E.~Rossi}$^\textrm{\scriptsize 67a,67b}$,    
\AtlasOrcid[0000-0003-3104-7971]{L.P.~Rossi}$^\textrm{\scriptsize 53b}$,    
\AtlasOrcid[0000-0003-0424-5729]{L.~Rossini}$^\textrm{\scriptsize 44}$,    
\AtlasOrcid[0000-0002-9095-7142]{R.~Rosten}$^\textrm{\scriptsize 122}$,    
\AtlasOrcid[0000-0003-4088-6275]{M.~Rotaru}$^\textrm{\scriptsize 25b}$,    
\AtlasOrcid[0000-0002-6762-2213]{B.~Rottler}$^\textrm{\scriptsize 50}$,    
\AtlasOrcid[0000-0001-7613-8063]{D.~Rousseau}$^\textrm{\scriptsize 62}$,    
\AtlasOrcid[0000-0003-1427-6668]{D.~Rousso}$^\textrm{\scriptsize 30}$,    
\AtlasOrcid[0000-0002-3430-8746]{G.~Rovelli}$^\textrm{\scriptsize 68a,68b}$,    
\AtlasOrcid[0000-0002-0116-1012]{A.~Roy}$^\textrm{\scriptsize 10}$,    
\AtlasOrcid[0000-0003-0504-1453]{A.~Rozanov}$^\textrm{\scriptsize 98}$,    
\AtlasOrcid[0000-0001-6969-0634]{Y.~Rozen}$^\textrm{\scriptsize 155}$,    
\AtlasOrcid[0000-0001-5621-6677]{X.~Ruan}$^\textrm{\scriptsize 31f}$,    
\AtlasOrcid[0000-0002-6978-5964]{A.J.~Ruby}$^\textrm{\scriptsize 88}$,    
\AtlasOrcid[0000-0001-9941-1966]{T.A.~Ruggeri}$^\textrm{\scriptsize 1}$,    
\AtlasOrcid[0000-0003-4452-620X]{F.~R\"uhr}$^\textrm{\scriptsize 50}$,    
\AtlasOrcid[0000-0002-5742-2541]{A.~Ruiz-Martinez}$^\textrm{\scriptsize 168}$,    
\AtlasOrcid[0000-0001-8945-8760]{A.~Rummler}$^\textrm{\scriptsize 34}$,    
\AtlasOrcid[0000-0003-3051-9607]{Z.~Rurikova}$^\textrm{\scriptsize 50}$,    
\AtlasOrcid[0000-0003-1927-5322]{N.A.~Rusakovich}$^\textrm{\scriptsize 77}$,    
\AtlasOrcid[0000-0003-4181-0678]{H.L.~Russell}$^\textrm{\scriptsize 34}$,    
\AtlasOrcid[0000-0002-0292-2477]{L.~Rustige}$^\textrm{\scriptsize 36}$,    
\AtlasOrcid[0000-0002-4682-0667]{J.P.~Rutherfoord}$^\textrm{\scriptsize 6}$,    
\AtlasOrcid[0000-0002-6062-0952]{E.M.~R{\"u}ttinger}$^\textrm{\scriptsize 144}$,    
\AtlasOrcid[0000-0002-6033-004X]{M.~Rybar}$^\textrm{\scriptsize 137}$,    
\AtlasOrcid[0000-0001-7088-1745]{E.B.~Rye}$^\textrm{\scriptsize 128}$,    
\AtlasOrcid[0000-0002-0623-7426]{A.~Ryzhov}$^\textrm{\scriptsize 118}$,    
\AtlasOrcid[0000-0003-2328-1952]{J.A.~Sabater~Iglesias}$^\textrm{\scriptsize 44}$,    
\AtlasOrcid[0000-0003-0159-697X]{P.~Sabatini}$^\textrm{\scriptsize 168}$,    
\AtlasOrcid[0000-0002-0865-5891]{L.~Sabetta}$^\textrm{\scriptsize 70a,70b}$,    
\AtlasOrcid[0000-0003-0019-5410]{H.F-W.~Sadrozinski}$^\textrm{\scriptsize 140}$,    
\AtlasOrcid[0000-0002-9157-6819]{R.~Sadykov}$^\textrm{\scriptsize 77}$,    
\AtlasOrcid[0000-0001-7796-0120]{F.~Safai~Tehrani}$^\textrm{\scriptsize 70a}$,    
\AtlasOrcid[0000-0002-0338-9707]{B.~Safarzadeh~Samani}$^\textrm{\scriptsize 151}$,    
\AtlasOrcid[0000-0001-8323-7318]{M.~Safdari}$^\textrm{\scriptsize 148}$,    
\AtlasOrcid[0000-0003-3851-1941]{P.~Saha}$^\textrm{\scriptsize 116}$,    
\AtlasOrcid[0000-0001-9296-1498]{S.~Saha}$^\textrm{\scriptsize 100}$,    
\AtlasOrcid[0000-0002-7400-7286]{M.~Sahinsoy}$^\textrm{\scriptsize 111}$,    
\AtlasOrcid[0000-0002-7064-0447]{A.~Sahu}$^\textrm{\scriptsize 176}$,    
\AtlasOrcid[0000-0002-3765-1320]{M.~Saimpert}$^\textrm{\scriptsize 139}$,    
\AtlasOrcid[0000-0001-5564-0935]{M.~Saito}$^\textrm{\scriptsize 158}$,    
\AtlasOrcid[0000-0003-2567-6392]{T.~Saito}$^\textrm{\scriptsize 158}$,    
\AtlasOrcid{D.~Salamani}$^\textrm{\scriptsize 34}$,    
\AtlasOrcid[0000-0002-0861-0052]{G.~Salamanna}$^\textrm{\scriptsize 72a,72b}$,    
\AtlasOrcid[0000-0002-3623-0161]{A.~Salnikov}$^\textrm{\scriptsize 148}$,    
\AtlasOrcid[0000-0003-4181-2788]{J.~Salt}$^\textrm{\scriptsize 168}$,    
\AtlasOrcid[0000-0001-5041-5659]{A.~Salvador~Salas}$^\textrm{\scriptsize 12}$,    
\AtlasOrcid[0000-0002-8564-2373]{D.~Salvatore}$^\textrm{\scriptsize 39b,39a}$,    
\AtlasOrcid[0000-0002-3709-1554]{F.~Salvatore}$^\textrm{\scriptsize 151}$,    
\AtlasOrcid[0000-0001-6004-3510]{A.~Salzburger}$^\textrm{\scriptsize 34}$,    
\AtlasOrcid[0000-0003-4484-1410]{D.~Sammel}$^\textrm{\scriptsize 50}$,    
\AtlasOrcid[0000-0002-9571-2304]{D.~Sampsonidis}$^\textrm{\scriptsize 157}$,    
\AtlasOrcid[0000-0003-0384-7672]{D.~Sampsonidou}$^\textrm{\scriptsize 58d,58c}$,    
\AtlasOrcid[0000-0001-9913-310X]{J.~S\'anchez}$^\textrm{\scriptsize 168}$,    
\AtlasOrcid[0000-0001-8241-7835]{A.~Sanchez~Pineda}$^\textrm{\scriptsize 4}$,    
\AtlasOrcid[0000-0002-4143-6201]{V.~Sanchez~Sebastian}$^\textrm{\scriptsize 168}$,    
\AtlasOrcid[0000-0001-5235-4095]{H.~Sandaker}$^\textrm{\scriptsize 128}$,    
\AtlasOrcid[0000-0003-2576-259X]{C.O.~Sander}$^\textrm{\scriptsize 44}$,    
\AtlasOrcid[0000-0001-7731-6757]{I.G.~Sanderswood}$^\textrm{\scriptsize 87}$,    
\AtlasOrcid[0000-0002-6016-8011]{J.A.~Sandesara}$^\textrm{\scriptsize 99}$,    
\AtlasOrcid[0000-0002-7601-8528]{M.~Sandhoff}$^\textrm{\scriptsize 176}$,    
\AtlasOrcid[0000-0003-1038-723X]{C.~Sandoval}$^\textrm{\scriptsize 20b}$,    
\AtlasOrcid[0000-0003-0955-4213]{D.P.C.~Sankey}$^\textrm{\scriptsize 138}$,    
\AtlasOrcid[0000-0001-7700-8383]{M.~Sannino}$^\textrm{\scriptsize 53b,53a}$,    
\AtlasOrcid[0000-0002-9166-099X]{A.~Sansoni}$^\textrm{\scriptsize 49}$,    
\AtlasOrcid[0000-0002-1642-7186]{C.~Santoni}$^\textrm{\scriptsize 36}$,    
\AtlasOrcid[0000-0003-1710-9291]{H.~Santos}$^\textrm{\scriptsize 134a,134b}$,    
\AtlasOrcid[0000-0001-6467-9970]{S.N.~Santpur}$^\textrm{\scriptsize 16}$,    
\AtlasOrcid[0000-0003-4644-2579]{A.~Santra}$^\textrm{\scriptsize 174}$,    
\AtlasOrcid[0000-0001-9150-640X]{K.A.~Saoucha}$^\textrm{\scriptsize 144}$,    
\AtlasOrcid[0000-0001-7569-2548]{A.~Sapronov}$^\textrm{\scriptsize 77}$,    
\AtlasOrcid[0000-0002-7006-0864]{J.G.~Saraiva}$^\textrm{\scriptsize 134a,134d}$,    
\AtlasOrcid[0000-0002-6932-2804]{J.~Sardain}$^\textrm{\scriptsize 98}$,    
\AtlasOrcid[0000-0002-2910-3906]{O.~Sasaki}$^\textrm{\scriptsize 79}$,    
\AtlasOrcid[0000-0001-8988-4065]{K.~Sato}$^\textrm{\scriptsize 163}$,    
\AtlasOrcid{C.~Sauer}$^\textrm{\scriptsize 59b}$,    
\AtlasOrcid[0000-0001-8794-3228]{F.~Sauerburger}$^\textrm{\scriptsize 50}$,    
\AtlasOrcid[0000-0003-1921-2647]{E.~Sauvan}$^\textrm{\scriptsize 4}$,    
\AtlasOrcid[0000-0001-5606-0107]{P.~Savard}$^\textrm{\scriptsize 161,aj}$,    
\AtlasOrcid[0000-0002-2226-9874]{R.~Sawada}$^\textrm{\scriptsize 158}$,    
\AtlasOrcid[0000-0002-2027-1428]{C.~Sawyer}$^\textrm{\scriptsize 138}$,    
\AtlasOrcid[0000-0001-8295-0605]{L.~Sawyer}$^\textrm{\scriptsize 93}$,    
\AtlasOrcid{I.~Sayago~Galvan}$^\textrm{\scriptsize 168}$,    
\AtlasOrcid[0000-0002-8236-5251]{C.~Sbarra}$^\textrm{\scriptsize 21b}$,    
\AtlasOrcid[0000-0002-1934-3041]{A.~Sbrizzi}$^\textrm{\scriptsize 64a,64c}$,    
\AtlasOrcid[0000-0002-2746-525X]{T.~Scanlon}$^\textrm{\scriptsize 92}$,    
\AtlasOrcid[0000-0002-0433-6439]{J.~Schaarschmidt}$^\textrm{\scriptsize 143}$,    
\AtlasOrcid[0000-0002-7215-7977]{P.~Schacht}$^\textrm{\scriptsize 111}$,    
\AtlasOrcid[0000-0002-8637-6134]{D.~Schaefer}$^\textrm{\scriptsize 35}$,    
\AtlasOrcid[0000-0003-4489-9145]{U.~Sch\"afer}$^\textrm{\scriptsize 96}$,    
\AtlasOrcid[0000-0002-2586-7554]{A.C.~Schaffer}$^\textrm{\scriptsize 62}$,    
\AtlasOrcid[0000-0001-7822-9663]{D.~Schaile}$^\textrm{\scriptsize 110}$,    
\AtlasOrcid[0000-0003-1218-425X]{R.D.~Schamberger}$^\textrm{\scriptsize 150}$,    
\AtlasOrcid[0000-0002-8719-4682]{E.~Schanet}$^\textrm{\scriptsize 110}$,    
\AtlasOrcid[0000-0002-0294-1205]{C.~Scharf}$^\textrm{\scriptsize 17}$,    
\AtlasOrcid[0000-0001-5180-3645]{N.~Scharmberg}$^\textrm{\scriptsize 97}$,    
\AtlasOrcid[0000-0003-1870-1967]{V.A.~Schegelsky}$^\textrm{\scriptsize 132}$,    
\AtlasOrcid[0000-0001-6012-7191]{D.~Scheirich}$^\textrm{\scriptsize 137}$,    
\AtlasOrcid[0000-0001-8279-4753]{F.~Schenck}$^\textrm{\scriptsize 17}$,    
\AtlasOrcid[0000-0002-0859-4312]{M.~Schernau}$^\textrm{\scriptsize 165}$,    
\AtlasOrcid[0000-0003-0957-4994]{C.~Schiavi}$^\textrm{\scriptsize 53b,53a}$,    
\AtlasOrcid[0000-0002-6834-9538]{L.K.~Schildgen}$^\textrm{\scriptsize 22}$,    
\AtlasOrcid[0000-0002-6978-5323]{Z.M.~Schillaci}$^\textrm{\scriptsize 24}$,    
\AtlasOrcid[0000-0002-1369-9944]{E.J.~Schioppa}$^\textrm{\scriptsize 65a,65b}$,    
\AtlasOrcid[0000-0003-0628-0579]{M.~Schioppa}$^\textrm{\scriptsize 39b,39a}$,    
\AtlasOrcid[0000-0002-1284-4169]{B.~Schlag}$^\textrm{\scriptsize 96}$,    
\AtlasOrcid[0000-0002-2917-7032]{K.E.~Schleicher}$^\textrm{\scriptsize 50}$,    
\AtlasOrcid[0000-0001-5239-3609]{S.~Schlenker}$^\textrm{\scriptsize 34}$,    
\AtlasOrcid[0000-0003-1978-4928]{K.~Schmieden}$^\textrm{\scriptsize 96}$,    
\AtlasOrcid[0000-0003-1471-690X]{C.~Schmitt}$^\textrm{\scriptsize 96}$,    
\AtlasOrcid[0000-0001-8387-1853]{S.~Schmitt}$^\textrm{\scriptsize 44}$,    
\AtlasOrcid[0000-0002-8081-2353]{L.~Schoeffel}$^\textrm{\scriptsize 139}$,    
\AtlasOrcid[0000-0002-4499-7215]{A.~Schoening}$^\textrm{\scriptsize 59b}$,    
\AtlasOrcid[0000-0003-2882-9796]{P.G.~Scholer}$^\textrm{\scriptsize 50}$,    
\AtlasOrcid[0000-0002-9340-2214]{E.~Schopf}$^\textrm{\scriptsize 129}$,    
\AtlasOrcid[0000-0002-4235-7265]{M.~Schott}$^\textrm{\scriptsize 96}$,    
\AtlasOrcid[0000-0003-0016-5246]{J.~Schovancova}$^\textrm{\scriptsize 34}$,    
\AtlasOrcid[0000-0001-9031-6751]{S.~Schramm}$^\textrm{\scriptsize 52}$,    
\AtlasOrcid[0000-0002-7289-1186]{F.~Schroeder}$^\textrm{\scriptsize 176}$,    
\AtlasOrcid[0000-0002-0860-7240]{H-C.~Schultz-Coulon}$^\textrm{\scriptsize 59a}$,    
\AtlasOrcid[0000-0002-1733-8388]{M.~Schumacher}$^\textrm{\scriptsize 50}$,    
\AtlasOrcid[0000-0002-5394-0317]{B.A.~Schumm}$^\textrm{\scriptsize 140}$,    
\AtlasOrcid[0000-0002-3971-9595]{Ph.~Schune}$^\textrm{\scriptsize 139}$,    
\AtlasOrcid[0000-0002-6680-8366]{A.~Schwartzman}$^\textrm{\scriptsize 148}$,    
\AtlasOrcid[0000-0001-5660-2690]{T.A.~Schwarz}$^\textrm{\scriptsize 102}$,    
\AtlasOrcid[0000-0003-0989-5675]{Ph.~Schwemling}$^\textrm{\scriptsize 139}$,    
\AtlasOrcid[0000-0001-6348-5410]{R.~Schwienhorst}$^\textrm{\scriptsize 103}$,    
\AtlasOrcid[0000-0001-7163-501X]{A.~Sciandra}$^\textrm{\scriptsize 140}$,    
\AtlasOrcid[0000-0002-8482-1775]{G.~Sciolla}$^\textrm{\scriptsize 24}$,    
\AtlasOrcid[0000-0001-9569-3089]{F.~Scuri}$^\textrm{\scriptsize 69a}$,    
\AtlasOrcid{F.~Scutti}$^\textrm{\scriptsize 101}$,    
\AtlasOrcid[0000-0003-1073-035X]{C.D.~Sebastiani}$^\textrm{\scriptsize 88}$,    
\AtlasOrcid[0000-0003-2052-2386]{K.~Sedlaczek}$^\textrm{\scriptsize 45}$,    
\AtlasOrcid[0000-0002-3727-5636]{P.~Seema}$^\textrm{\scriptsize 17}$,    
\AtlasOrcid[0000-0002-1181-3061]{S.C.~Seidel}$^\textrm{\scriptsize 113}$,    
\AtlasOrcid[0000-0003-4311-8597]{A.~Seiden}$^\textrm{\scriptsize 140}$,    
\AtlasOrcid[0000-0002-4703-000X]{B.D.~Seidlitz}$^\textrm{\scriptsize 27}$,    
\AtlasOrcid[0000-0003-0810-240X]{T.~Seiss}$^\textrm{\scriptsize 35}$,    
\AtlasOrcid[0000-0003-4622-6091]{C.~Seitz}$^\textrm{\scriptsize 44}$,    
\AtlasOrcid[0000-0001-5148-7363]{J.M.~Seixas}$^\textrm{\scriptsize 78b}$,    
\AtlasOrcid[0000-0002-4116-5309]{G.~Sekhniaidze}$^\textrm{\scriptsize 67a}$,    
\AtlasOrcid[0000-0002-3199-4699]{S.J.~Sekula}$^\textrm{\scriptsize 40}$,    
\AtlasOrcid[0000-0002-8739-8554]{L.~Selem}$^\textrm{\scriptsize 4}$,    
\AtlasOrcid[0000-0002-3946-377X]{N.~Semprini-Cesari}$^\textrm{\scriptsize 21b,21a}$,    
\AtlasOrcid[0000-0003-1240-9586]{S.~Sen}$^\textrm{\scriptsize 47}$,    
\AtlasOrcid[0000-0001-7658-4901]{C.~Serfon}$^\textrm{\scriptsize 27}$,    
\AtlasOrcid[0000-0003-3238-5382]{L.~Serin}$^\textrm{\scriptsize 62}$,    
\AtlasOrcid[0000-0003-4749-5250]{L.~Serkin}$^\textrm{\scriptsize 64a,64b}$,    
\AtlasOrcid[0000-0002-1402-7525]{M.~Sessa}$^\textrm{\scriptsize 72a,72b}$,    
\AtlasOrcid[0000-0003-3316-846X]{H.~Severini}$^\textrm{\scriptsize 123}$,    
\AtlasOrcid[0000-0001-6785-1334]{S.~Sevova}$^\textrm{\scriptsize 148}$,    
\AtlasOrcid[0000-0002-4065-7352]{F.~Sforza}$^\textrm{\scriptsize 53b,53a}$,    
\AtlasOrcid[0000-0002-3003-9905]{A.~Sfyrla}$^\textrm{\scriptsize 52}$,    
\AtlasOrcid[0000-0003-4849-556X]{E.~Shabalina}$^\textrm{\scriptsize 51}$,    
\AtlasOrcid[0000-0002-2673-8527]{R.~Shaheen}$^\textrm{\scriptsize 149}$,    
\AtlasOrcid[0000-0002-1325-3432]{J.D.~Shahinian}$^\textrm{\scriptsize 131}$,    
\AtlasOrcid[0000-0001-9358-3505]{N.W.~Shaikh}$^\textrm{\scriptsize 43a,43b}$,    
\AtlasOrcid[0000-0002-5376-1546]{D.~Shaked~Renous}$^\textrm{\scriptsize 174}$,    
\AtlasOrcid[0000-0001-9134-5925]{L.Y.~Shan}$^\textrm{\scriptsize 13a}$,    
\AtlasOrcid[0000-0001-8540-9654]{M.~Shapiro}$^\textrm{\scriptsize 16}$,    
\AtlasOrcid[0000-0002-5211-7177]{A.~Sharma}$^\textrm{\scriptsize 34}$,    
\AtlasOrcid[0000-0003-2250-4181]{A.S.~Sharma}$^\textrm{\scriptsize 1}$,    
\AtlasOrcid[0000-0002-0190-7558]{S.~Sharma}$^\textrm{\scriptsize 44}$,    
\AtlasOrcid[0000-0001-7530-4162]{P.B.~Shatalov}$^\textrm{\scriptsize 119}$,    
\AtlasOrcid[0000-0001-9182-0634]{K.~Shaw}$^\textrm{\scriptsize 151}$,    
\AtlasOrcid[0000-0002-8958-7826]{S.M.~Shaw}$^\textrm{\scriptsize 97}$,    
\AtlasOrcid[0000-0002-6621-4111]{P.~Sherwood}$^\textrm{\scriptsize 92}$,    
\AtlasOrcid[0000-0001-9532-5075]{L.~Shi}$^\textrm{\scriptsize 92}$,    
\AtlasOrcid[0000-0002-2228-2251]{C.O.~Shimmin}$^\textrm{\scriptsize 177}$,    
\AtlasOrcid[0000-0003-3066-2788]{Y.~Shimogama}$^\textrm{\scriptsize 173}$,    
\AtlasOrcid[0000-0002-3523-390X]{J.D.~Shinner}$^\textrm{\scriptsize 91}$,    
\AtlasOrcid[0000-0003-4050-6420]{I.P.J.~Shipsey}$^\textrm{\scriptsize 129}$,    
\AtlasOrcid[0000-0002-3191-0061]{S.~Shirabe}$^\textrm{\scriptsize 52}$,    
\AtlasOrcid[0000-0002-4775-9669]{M.~Shiyakova}$^\textrm{\scriptsize 77}$,    
\AtlasOrcid[0000-0002-2628-3470]{J.~Shlomi}$^\textrm{\scriptsize 174}$,    
\AtlasOrcid[0000-0002-3017-826X]{M.J.~Shochet}$^\textrm{\scriptsize 35}$,    
\AtlasOrcid[0000-0002-9449-0412]{J.~Shojaii}$^\textrm{\scriptsize 101}$,    
\AtlasOrcid[0000-0002-9453-9415]{D.R.~Shope}$^\textrm{\scriptsize 149}$,    
\AtlasOrcid[0000-0001-7249-7456]{S.~Shrestha}$^\textrm{\scriptsize 122}$,    
\AtlasOrcid[0000-0001-8352-7227]{E.M.~Shrif}$^\textrm{\scriptsize 31f}$,    
\AtlasOrcid[0000-0002-0456-786X]{M.J.~Shroff}$^\textrm{\scriptsize 170}$,    
\AtlasOrcid[0000-0001-5099-7644]{E.~Shulga}$^\textrm{\scriptsize 174}$,    
\AtlasOrcid[0000-0002-5428-813X]{P.~Sicho}$^\textrm{\scriptsize 135}$,    
\AtlasOrcid[0000-0002-3246-0330]{A.M.~Sickles}$^\textrm{\scriptsize 167}$,    
\AtlasOrcid[0000-0002-3206-395X]{E.~Sideras~Haddad}$^\textrm{\scriptsize 31f}$,    
\AtlasOrcid[0000-0002-1285-1350]{O.~Sidiropoulou}$^\textrm{\scriptsize 34}$,    
\AtlasOrcid[0000-0002-3277-1999]{A.~Sidoti}$^\textrm{\scriptsize 21b}$,    
\AtlasOrcid[0000-0002-2893-6412]{F.~Siegert}$^\textrm{\scriptsize 46}$,    
\AtlasOrcid[0000-0002-5809-9424]{Dj.~Sijacki}$^\textrm{\scriptsize 14}$,    
\AtlasOrcid[0000-0002-5987-2984]{J.M.~Silva}$^\textrm{\scriptsize 19}$,    
\AtlasOrcid[0000-0003-2285-478X]{M.V.~Silva~Oliveira}$^\textrm{\scriptsize 34}$,    
\AtlasOrcid[0000-0001-7734-7617]{S.B.~Silverstein}$^\textrm{\scriptsize 43a}$,    
\AtlasOrcid{S.~Simion}$^\textrm{\scriptsize 62}$,    
\AtlasOrcid[0000-0003-2042-6394]{R.~Simoniello}$^\textrm{\scriptsize 34}$,    
\AtlasOrcid{N.D.~Simpson}$^\textrm{\scriptsize 94}$,    
\AtlasOrcid[0000-0002-9650-3846]{S.~Simsek}$^\textrm{\scriptsize 11b}$,    
\AtlasOrcid[0000-0002-5128-2373]{P.~Sinervo}$^\textrm{\scriptsize 161}$,    
\AtlasOrcid[0000-0001-5347-9308]{V.~Sinetckii}$^\textrm{\scriptsize 109}$,    
\AtlasOrcid[0000-0002-7710-4073]{S.~Singh}$^\textrm{\scriptsize 147}$,    
\AtlasOrcid[0000-0001-5641-5713]{S.~Singh}$^\textrm{\scriptsize 161}$,    
\AtlasOrcid[0000-0002-3600-2804]{S.~Sinha}$^\textrm{\scriptsize 44}$,    
\AtlasOrcid[0000-0002-2438-3785]{S.~Sinha}$^\textrm{\scriptsize 31f}$,    
\AtlasOrcid[0000-0002-0912-9121]{M.~Sioli}$^\textrm{\scriptsize 21b,21a}$,    
\AtlasOrcid[0000-0003-4554-1831]{I.~Siral}$^\textrm{\scriptsize 126}$,    
\AtlasOrcid[0000-0003-0868-8164]{S.Yu.~Sivoklokov}$^\textrm{\scriptsize 109}$,    
\AtlasOrcid[0000-0002-5285-8995]{J.~Sj\"{o}lin}$^\textrm{\scriptsize 43a,43b}$,    
\AtlasOrcid[0000-0003-3614-026X]{A.~Skaf}$^\textrm{\scriptsize 51}$,    
\AtlasOrcid[0000-0003-3973-9382]{E.~Skorda}$^\textrm{\scriptsize 94}$,    
\AtlasOrcid[0000-0001-6342-9283]{P.~Skubic}$^\textrm{\scriptsize 123}$,    
\AtlasOrcid[0000-0002-9386-9092]{M.~Slawinska}$^\textrm{\scriptsize 82}$,    
\AtlasOrcid[0000-0002-1201-4771]{K.~Sliwa}$^\textrm{\scriptsize 164}$,    
\AtlasOrcid{V.~Smakhtin}$^\textrm{\scriptsize 174}$,    
\AtlasOrcid[0000-0002-7192-4097]{B.H.~Smart}$^\textrm{\scriptsize 138}$,    
\AtlasOrcid[0000-0003-3725-2984]{J.~Smiesko}$^\textrm{\scriptsize 137}$,    
\AtlasOrcid[0000-0002-6778-073X]{S.Yu.~Smirnov}$^\textrm{\scriptsize 108}$,    
\AtlasOrcid[0000-0002-2891-0781]{Y.~Smirnov}$^\textrm{\scriptsize 108}$,    
\AtlasOrcid[0000-0002-0447-2975]{L.N.~Smirnova}$^\textrm{\scriptsize 109,r}$,    
\AtlasOrcid[0000-0003-2517-531X]{O.~Smirnova}$^\textrm{\scriptsize 94}$,    
\AtlasOrcid[0000-0001-6480-6829]{E.A.~Smith}$^\textrm{\scriptsize 35}$,    
\AtlasOrcid[0000-0003-2799-6672]{H.A.~Smith}$^\textrm{\scriptsize 129}$,    
\AtlasOrcid[0000-0002-3777-4734]{M.~Smizanska}$^\textrm{\scriptsize 87}$,    
\AtlasOrcid[0000-0002-5996-7000]{K.~Smolek}$^\textrm{\scriptsize 136}$,    
\AtlasOrcid[0000-0001-6088-7094]{A.~Smykiewicz}$^\textrm{\scriptsize 82}$,    
\AtlasOrcid[0000-0002-9067-8362]{A.A.~Snesarev}$^\textrm{\scriptsize 107}$,    
\AtlasOrcid[0000-0003-4579-2120]{H.L.~Snoek}$^\textrm{\scriptsize 115}$,    
\AtlasOrcid[0000-0001-8610-8423]{S.~Snyder}$^\textrm{\scriptsize 27}$,    
\AtlasOrcid[0000-0001-7430-7599]{R.~Sobie}$^\textrm{\scriptsize 170,y}$,    
\AtlasOrcid[0000-0002-0749-2146]{A.~Soffer}$^\textrm{\scriptsize 156}$,    
\AtlasOrcid[0000-0001-6959-2997]{F.~Sohns}$^\textrm{\scriptsize 51}$,    
\AtlasOrcid[0000-0002-0518-4086]{C.A.~Solans~Sanchez}$^\textrm{\scriptsize 34}$,    
\AtlasOrcid[0000-0003-0694-3272]{E.Yu.~Soldatov}$^\textrm{\scriptsize 108}$,    
\AtlasOrcid[0000-0002-7674-7878]{U.~Soldevila}$^\textrm{\scriptsize 168}$,    
\AtlasOrcid[0000-0002-2737-8674]{A.A.~Solodkov}$^\textrm{\scriptsize 118}$,    
\AtlasOrcid[0000-0002-7378-4454]{S.~Solomon}$^\textrm{\scriptsize 50}$,    
\AtlasOrcid[0000-0001-9946-8188]{A.~Soloshenko}$^\textrm{\scriptsize 77}$,    
\AtlasOrcid[0000-0002-2598-5657]{O.V.~Solovyanov}$^\textrm{\scriptsize 118}$,    
\AtlasOrcid[0000-0002-9402-6329]{V.~Solovyev}$^\textrm{\scriptsize 132}$,    
\AtlasOrcid[0000-0003-1703-7304]{P.~Sommer}$^\textrm{\scriptsize 144}$,    
\AtlasOrcid[0000-0003-2225-9024]{H.~Son}$^\textrm{\scriptsize 164}$,    
\AtlasOrcid[0000-0003-4435-4962]{A.~Sonay}$^\textrm{\scriptsize 12}$,    
\AtlasOrcid[0000-0003-1338-2741]{W.Y.~Song}$^\textrm{\scriptsize 162b}$,    
\AtlasOrcid[0000-0001-6981-0544]{A.~Sopczak}$^\textrm{\scriptsize 136}$,    
\AtlasOrcid{A.L.~Sopio}$^\textrm{\scriptsize 92}$,    
\AtlasOrcid[0000-0002-6171-1119]{F.~Sopkova}$^\textrm{\scriptsize 26b}$,    
\AtlasOrcid[0000-0002-1430-5994]{S.~Sottocornola}$^\textrm{\scriptsize 68a,68b}$,    
\AtlasOrcid[0000-0003-0124-3410]{R.~Soualah}$^\textrm{\scriptsize 64a,64c}$,    
\AtlasOrcid[0000-0002-2210-0913]{A.M.~Soukharev}$^\textrm{\scriptsize 117b,117a}$,    
\AtlasOrcid[0000-0002-8120-478X]{Z.~Soumaimi}$^\textrm{\scriptsize 33e}$,    
\AtlasOrcid[0000-0002-0786-6304]{D.~South}$^\textrm{\scriptsize 44}$,    
\AtlasOrcid[0000-0001-7482-6348]{S.~Spagnolo}$^\textrm{\scriptsize 65a,65b}$,    
\AtlasOrcid[0000-0001-5813-1693]{M.~Spalla}$^\textrm{\scriptsize 111}$,    
\AtlasOrcid[0000-0001-8265-403X]{M.~Spangenberg}$^\textrm{\scriptsize 172}$,    
\AtlasOrcid[0000-0002-6551-1878]{F.~Span\`o}$^\textrm{\scriptsize 91}$,    
\AtlasOrcid[0000-0003-4454-6999]{D.~Sperlich}$^\textrm{\scriptsize 50}$,    
\AtlasOrcid[0000-0002-9408-895X]{T.M.~Spieker}$^\textrm{\scriptsize 59a}$,    
\AtlasOrcid[0000-0003-4183-2594]{G.~Spigo}$^\textrm{\scriptsize 34}$,    
\AtlasOrcid[0000-0002-0418-4199]{M.~Spina}$^\textrm{\scriptsize 151}$,    
\AtlasOrcid[0000-0002-9226-2539]{D.P.~Spiteri}$^\textrm{\scriptsize 55}$,    
\AtlasOrcid[0000-0001-5644-9526]{M.~Spousta}$^\textrm{\scriptsize 137}$,    
\AtlasOrcid[0000-0002-6868-8329]{A.~Stabile}$^\textrm{\scriptsize 66a,66b}$,    
\AtlasOrcid[0000-0001-5430-4702]{B.L.~Stamas}$^\textrm{\scriptsize 116}$,    
\AtlasOrcid[0000-0001-7282-949X]{R.~Stamen}$^\textrm{\scriptsize 59a}$,    
\AtlasOrcid[0000-0003-2251-0610]{M.~Stamenkovic}$^\textrm{\scriptsize 115}$,    
\AtlasOrcid[0000-0002-7666-7544]{A.~Stampekis}$^\textrm{\scriptsize 19}$,    
\AtlasOrcid[0000-0002-2610-9608]{M.~Standke}$^\textrm{\scriptsize 22}$,    
\AtlasOrcid[0000-0003-2546-0516]{E.~Stanecka}$^\textrm{\scriptsize 82}$,    
\AtlasOrcid[0000-0001-9007-7658]{B.~Stanislaus}$^\textrm{\scriptsize 34}$,    
\AtlasOrcid[0000-0002-7561-1960]{M.M.~Stanitzki}$^\textrm{\scriptsize 44}$,    
\AtlasOrcid[0000-0002-2224-719X]{M.~Stankaityte}$^\textrm{\scriptsize 129}$,    
\AtlasOrcid[0000-0001-5374-6402]{B.~Stapf}$^\textrm{\scriptsize 44}$,    
\AtlasOrcid[0000-0002-8495-0630]{E.A.~Starchenko}$^\textrm{\scriptsize 118}$,    
\AtlasOrcid[0000-0001-6616-3433]{G.H.~Stark}$^\textrm{\scriptsize 140}$,    
\AtlasOrcid[0000-0002-1217-672X]{J.~Stark}$^\textrm{\scriptsize 98}$,    
\AtlasOrcid{D.M.~Starko}$^\textrm{\scriptsize 162b}$,    
\AtlasOrcid[0000-0001-6009-6321]{P.~Staroba}$^\textrm{\scriptsize 135}$,    
\AtlasOrcid[0000-0003-1990-0992]{P.~Starovoitov}$^\textrm{\scriptsize 59a}$,    
\AtlasOrcid[0000-0002-2908-3909]{S.~St\"arz}$^\textrm{\scriptsize 100}$,    
\AtlasOrcid[0000-0001-7708-9259]{R.~Staszewski}$^\textrm{\scriptsize 82}$,    
\AtlasOrcid[0000-0002-8549-6855]{G.~Stavropoulos}$^\textrm{\scriptsize 42}$,    
\AtlasOrcid[0000-0002-5349-8370]{P.~Steinberg}$^\textrm{\scriptsize 27}$,    
\AtlasOrcid[0000-0002-4080-2919]{A.L.~Steinhebel}$^\textrm{\scriptsize 126}$,    
\AtlasOrcid[0000-0003-4091-1784]{B.~Stelzer}$^\textrm{\scriptsize 147,162a}$,    
\AtlasOrcid[0000-0003-0690-8573]{H.J.~Stelzer}$^\textrm{\scriptsize 133}$,    
\AtlasOrcid[0000-0002-0791-9728]{O.~Stelzer-Chilton}$^\textrm{\scriptsize 162a}$,    
\AtlasOrcid[0000-0002-4185-6484]{H.~Stenzel}$^\textrm{\scriptsize 54}$,    
\AtlasOrcid[0000-0003-2399-8945]{T.J.~Stevenson}$^\textrm{\scriptsize 151}$,    
\AtlasOrcid[0000-0003-0182-7088]{G.A.~Stewart}$^\textrm{\scriptsize 34}$,    
\AtlasOrcid[0000-0001-9679-0323]{M.C.~Stockton}$^\textrm{\scriptsize 34}$,    
\AtlasOrcid[0000-0002-7511-4614]{G.~Stoicea}$^\textrm{\scriptsize 25b}$,    
\AtlasOrcid[0000-0003-0276-8059]{M.~Stolarski}$^\textrm{\scriptsize 134a}$,    
\AtlasOrcid[0000-0001-7582-6227]{S.~Stonjek}$^\textrm{\scriptsize 111}$,    
\AtlasOrcid[0000-0003-2460-6659]{A.~Straessner}$^\textrm{\scriptsize 46}$,    
\AtlasOrcid[0000-0002-8913-0981]{J.~Strandberg}$^\textrm{\scriptsize 149}$,    
\AtlasOrcid[0000-0001-7253-7497]{S.~Strandberg}$^\textrm{\scriptsize 43a,43b}$,    
\AtlasOrcid[0000-0002-0465-5472]{M.~Strauss}$^\textrm{\scriptsize 123}$,    
\AtlasOrcid[0000-0002-6972-7473]{T.~Strebler}$^\textrm{\scriptsize 98}$,    
\AtlasOrcid[0000-0003-0958-7656]{P.~Strizenec}$^\textrm{\scriptsize 26b}$,    
\AtlasOrcid[0000-0002-0062-2438]{R.~Str\"ohmer}$^\textrm{\scriptsize 171}$,    
\AtlasOrcid[0000-0002-8302-386X]{D.M.~Strom}$^\textrm{\scriptsize 126}$,    
\AtlasOrcid[0000-0002-4496-1626]{L.R.~Strom}$^\textrm{\scriptsize 44}$,    
\AtlasOrcid[0000-0002-7863-3778]{R.~Stroynowski}$^\textrm{\scriptsize 40}$,    
\AtlasOrcid[0000-0002-2382-6951]{A.~Strubig}$^\textrm{\scriptsize 43a,43b}$,    
\AtlasOrcid[0000-0002-1639-4484]{S.A.~Stucci}$^\textrm{\scriptsize 27}$,    
\AtlasOrcid[0000-0002-1728-9272]{B.~Stugu}$^\textrm{\scriptsize 15}$,    
\AtlasOrcid[0000-0001-9610-0783]{J.~Stupak}$^\textrm{\scriptsize 123}$,    
\AtlasOrcid[0000-0001-6976-9457]{N.A.~Styles}$^\textrm{\scriptsize 44}$,    
\AtlasOrcid[0000-0001-6980-0215]{D.~Su}$^\textrm{\scriptsize 148}$,    
\AtlasOrcid[0000-0002-7356-4961]{S.~Su}$^\textrm{\scriptsize 58a}$,    
\AtlasOrcid[0000-0001-7755-5280]{W.~Su}$^\textrm{\scriptsize 58d,143,58c}$,    
\AtlasOrcid[0000-0001-9155-3898]{X.~Su}$^\textrm{\scriptsize 58a}$,    
\AtlasOrcid[0000-0003-4364-006X]{K.~Sugizaki}$^\textrm{\scriptsize 158}$,    
\AtlasOrcid[0000-0003-3943-2495]{V.V.~Sulin}$^\textrm{\scriptsize 107}$,    
\AtlasOrcid[0000-0002-4807-6448]{M.J.~Sullivan}$^\textrm{\scriptsize 88}$,    
\AtlasOrcid[0000-0003-2925-279X]{D.M.S.~Sultan}$^\textrm{\scriptsize 52}$,    
\AtlasOrcid[0000-0002-0059-0165]{L.~Sultanaliyeva}$^\textrm{\scriptsize 107}$,    
\AtlasOrcid[0000-0003-2340-748X]{S.~Sultansoy}$^\textrm{\scriptsize 3c}$,    
\AtlasOrcid[0000-0002-2685-6187]{T.~Sumida}$^\textrm{\scriptsize 83}$,    
\AtlasOrcid[0000-0001-8802-7184]{S.~Sun}$^\textrm{\scriptsize 102}$,    
\AtlasOrcid[0000-0001-5295-6563]{S.~Sun}$^\textrm{\scriptsize 175}$,    
\AtlasOrcid[0000-0003-4409-4574]{X.~Sun}$^\textrm{\scriptsize 97}$,    
\AtlasOrcid[0000-0002-6277-1877]{O.~Sunneborn~Gudnadottir}$^\textrm{\scriptsize 166}$,    
\AtlasOrcid[0000-0001-7021-9380]{C.J.E.~Suster}$^\textrm{\scriptsize 152}$,    
\AtlasOrcid[0000-0003-4893-8041]{M.R.~Sutton}$^\textrm{\scriptsize 151}$,    
\AtlasOrcid[0000-0002-7199-3383]{M.~Svatos}$^\textrm{\scriptsize 135}$,    
\AtlasOrcid[0000-0001-7287-0468]{M.~Swiatlowski}$^\textrm{\scriptsize 162a}$,    
\AtlasOrcid[0000-0002-4679-6767]{T.~Swirski}$^\textrm{\scriptsize 171}$,    
\AtlasOrcid[0000-0003-3447-5621]{I.~Sykora}$^\textrm{\scriptsize 26a}$,    
\AtlasOrcid[0000-0003-4422-6493]{M.~Sykora}$^\textrm{\scriptsize 137}$,    
\AtlasOrcid[0000-0001-9585-7215]{T.~Sykora}$^\textrm{\scriptsize 137}$,    
\AtlasOrcid[0000-0002-0918-9175]{D.~Ta}$^\textrm{\scriptsize 96}$,    
\AtlasOrcid[0000-0003-3917-3761]{K.~Tackmann}$^\textrm{\scriptsize 44,w}$,    
\AtlasOrcid[0000-0002-5800-4798]{A.~Taffard}$^\textrm{\scriptsize 165}$,    
\AtlasOrcid[0000-0003-3425-794X]{R.~Tafirout}$^\textrm{\scriptsize 162a}$,    
\AtlasOrcid[0000-0001-7002-0590]{R.H.M.~Taibah}$^\textrm{\scriptsize 130}$,    
\AtlasOrcid[0000-0003-1466-6869]{R.~Takashima}$^\textrm{\scriptsize 84}$,    
\AtlasOrcid[0000-0002-2611-8563]{K.~Takeda}$^\textrm{\scriptsize 80}$,    
\AtlasOrcid[0000-0003-1135-1423]{T.~Takeshita}$^\textrm{\scriptsize 145}$,    
\AtlasOrcid[0000-0003-3142-030X]{E.P.~Takeva}$^\textrm{\scriptsize 48}$,    
\AtlasOrcid[0000-0002-3143-8510]{Y.~Takubo}$^\textrm{\scriptsize 79}$,    
\AtlasOrcid[0000-0001-9985-6033]{M.~Talby}$^\textrm{\scriptsize 98}$,    
\AtlasOrcid[0000-0001-8560-3756]{A.A.~Talyshev}$^\textrm{\scriptsize 117b,117a}$,    
\AtlasOrcid[0000-0002-1433-2140]{K.C.~Tam}$^\textrm{\scriptsize 60b}$,    
\AtlasOrcid{N.M.~Tamir}$^\textrm{\scriptsize 156}$,    
\AtlasOrcid[0000-0002-9166-7083]{A.~Tanaka}$^\textrm{\scriptsize 158}$,    
\AtlasOrcid[0000-0001-9994-5802]{J.~Tanaka}$^\textrm{\scriptsize 158}$,    
\AtlasOrcid[0000-0002-9929-1797]{R.~Tanaka}$^\textrm{\scriptsize 62}$,    
\AtlasOrcid{J.~Tang}$^\textrm{\scriptsize 58c}$,    
\AtlasOrcid[0000-0003-0362-8795]{Z.~Tao}$^\textrm{\scriptsize 169}$,    
\AtlasOrcid[0000-0002-3659-7270]{S.~Tapia~Araya}$^\textrm{\scriptsize 76}$,    
\AtlasOrcid[0000-0003-1251-3332]{S.~Tapprogge}$^\textrm{\scriptsize 96}$,    
\AtlasOrcid[0000-0002-9252-7605]{A.~Tarek~Abouelfadl~Mohamed}$^\textrm{\scriptsize 103}$,    
\AtlasOrcid[0000-0002-9296-7272]{S.~Tarem}$^\textrm{\scriptsize 155}$,    
\AtlasOrcid[0000-0002-0584-8700]{K.~Tariq}$^\textrm{\scriptsize 58b}$,    
\AtlasOrcid[0000-0002-5060-2208]{G.~Tarna}$^\textrm{\scriptsize 25b}$,    
\AtlasOrcid[0000-0002-4244-502X]{G.F.~Tartarelli}$^\textrm{\scriptsize 66a}$,    
\AtlasOrcid[0000-0001-5785-7548]{P.~Tas}$^\textrm{\scriptsize 137}$,    
\AtlasOrcid[0000-0002-1535-9732]{M.~Tasevsky}$^\textrm{\scriptsize 135}$,    
\AtlasOrcid[0000-0002-3335-6500]{E.~Tassi}$^\textrm{\scriptsize 39b,39a}$,    
\AtlasOrcid[0000-0003-3348-0234]{G.~Tateno}$^\textrm{\scriptsize 158}$,    
\AtlasOrcid[0000-0001-8760-7259]{Y.~Tayalati}$^\textrm{\scriptsize 33e}$,    
\AtlasOrcid[0000-0002-1831-4871]{G.N.~Taylor}$^\textrm{\scriptsize 101}$,    
\AtlasOrcid[0000-0002-6596-9125]{W.~Taylor}$^\textrm{\scriptsize 162b}$,    
\AtlasOrcid{H.~Teagle}$^\textrm{\scriptsize 88}$,    
\AtlasOrcid[0000-0003-3587-187X]{A.S.~Tee}$^\textrm{\scriptsize 175}$,    
\AtlasOrcid[0000-0001-5545-6513]{R.~Teixeira~De~Lima}$^\textrm{\scriptsize 148}$,    
\AtlasOrcid[0000-0001-9977-3836]{P.~Teixeira-Dias}$^\textrm{\scriptsize 91}$,    
\AtlasOrcid{H.~Ten~Kate}$^\textrm{\scriptsize 34}$,    
\AtlasOrcid[0000-0003-4803-5213]{J.J.~Teoh}$^\textrm{\scriptsize 115}$,    
\AtlasOrcid[0000-0001-6520-8070]{K.~Terashi}$^\textrm{\scriptsize 158}$,    
\AtlasOrcid[0000-0003-0132-5723]{J.~Terron}$^\textrm{\scriptsize 95}$,    
\AtlasOrcid[0000-0003-3388-3906]{S.~Terzo}$^\textrm{\scriptsize 12}$,    
\AtlasOrcid[0000-0003-1274-8967]{M.~Testa}$^\textrm{\scriptsize 49}$,    
\AtlasOrcid[0000-0002-8768-2272]{R.J.~Teuscher}$^\textrm{\scriptsize 161,y}$,    
\AtlasOrcid[0000-0003-1882-5572]{N.~Themistokleous}$^\textrm{\scriptsize 48}$,    
\AtlasOrcid[0000-0002-9746-4172]{T.~Theveneaux-Pelzer}$^\textrm{\scriptsize 17}$,    
\AtlasOrcid{O.~Thielmann}$^\textrm{\scriptsize 176}$,    
\AtlasOrcid{D.W.~Thomas}$^\textrm{\scriptsize 91}$,    
\AtlasOrcid[0000-0001-6965-6604]{J.P.~Thomas}$^\textrm{\scriptsize 19}$,    
\AtlasOrcid[0000-0001-7050-8203]{E.A.~Thompson}$^\textrm{\scriptsize 44}$,    
\AtlasOrcid[0000-0002-6239-7715]{P.D.~Thompson}$^\textrm{\scriptsize 19}$,    
\AtlasOrcid[0000-0001-6031-2768]{E.~Thomson}$^\textrm{\scriptsize 131}$,    
\AtlasOrcid[0000-0003-1594-9350]{E.J.~Thorpe}$^\textrm{\scriptsize 90}$,    
\AtlasOrcid[0000-0001-8739-9250]{Y.~Tian}$^\textrm{\scriptsize 51}$,    
\AtlasOrcid[0000-0002-9634-0581]{V.O.~Tikhomirov}$^\textrm{\scriptsize 107,af}$,    
\AtlasOrcid[0000-0002-8023-6448]{Yu.A.~Tikhonov}$^\textrm{\scriptsize 117b,117a}$,    
\AtlasOrcid{S.~Timoshenko}$^\textrm{\scriptsize 108}$,    
\AtlasOrcid[0000-0002-3698-3585]{P.~Tipton}$^\textrm{\scriptsize 177}$,    
\AtlasOrcid[0000-0002-0294-6727]{S.~Tisserant}$^\textrm{\scriptsize 98}$,    
\AtlasOrcid[0000-0002-4934-1661]{S.H.~Tlou}$^\textrm{\scriptsize 31f}$,    
\AtlasOrcid[0000-0003-2674-9274]{A.~Tnourji}$^\textrm{\scriptsize 36}$,    
\AtlasOrcid[0000-0003-2445-1132]{K.~Todome}$^\textrm{\scriptsize 21b,21a}$,    
\AtlasOrcid[0000-0003-2433-231X]{S.~Todorova-Nova}$^\textrm{\scriptsize 137}$,    
\AtlasOrcid{S.~Todt}$^\textrm{\scriptsize 46}$,    
\AtlasOrcid{M.~Togawa}$^\textrm{\scriptsize 79}$,    
\AtlasOrcid[0000-0003-4666-3208]{J.~Tojo}$^\textrm{\scriptsize 85}$,    
\AtlasOrcid[0000-0001-8777-0590]{S.~Tok\'ar}$^\textrm{\scriptsize 26a}$,    
\AtlasOrcid[0000-0002-8262-1577]{K.~Tokushuku}$^\textrm{\scriptsize 79}$,    
\AtlasOrcid[0000-0002-1027-1213]{E.~Tolley}$^\textrm{\scriptsize 122}$,    
\AtlasOrcid[0000-0002-1824-034X]{R.~Tombs}$^\textrm{\scriptsize 30}$,    
\AtlasOrcid[0000-0002-4603-2070]{M.~Tomoto}$^\textrm{\scriptsize 79,112}$,    
\AtlasOrcid[0000-0001-8127-9653]{L.~Tompkins}$^\textrm{\scriptsize 148}$,    
\AtlasOrcid[0000-0003-1129-9792]{P.~Tornambe}$^\textrm{\scriptsize 99}$,    
\AtlasOrcid[0000-0003-2911-8910]{E.~Torrence}$^\textrm{\scriptsize 126}$,    
\AtlasOrcid[0000-0003-0822-1206]{H.~Torres}$^\textrm{\scriptsize 46}$,    
\AtlasOrcid[0000-0002-5507-7924]{E.~Torr\'o~Pastor}$^\textrm{\scriptsize 168}$,    
\AtlasOrcid[0000-0001-9898-480X]{M.~Toscani}$^\textrm{\scriptsize 28}$,    
\AtlasOrcid[0000-0001-6485-2227]{C.~Tosciri}$^\textrm{\scriptsize 35}$,    
\AtlasOrcid[0000-0001-9128-6080]{J.~Toth}$^\textrm{\scriptsize 98,x}$,    
\AtlasOrcid[0000-0001-5543-6192]{D.R.~Tovey}$^\textrm{\scriptsize 144}$,    
\AtlasOrcid{A.~Traeet}$^\textrm{\scriptsize 15}$,    
\AtlasOrcid[0000-0002-0902-491X]{C.J.~Treado}$^\textrm{\scriptsize 120}$,    
\AtlasOrcid[0000-0002-9820-1729]{T.~Trefzger}$^\textrm{\scriptsize 171}$,    
\AtlasOrcid[0000-0002-8224-6105]{A.~Tricoli}$^\textrm{\scriptsize 27}$,    
\AtlasOrcid[0000-0002-6127-5847]{I.M.~Trigger}$^\textrm{\scriptsize 162a}$,    
\AtlasOrcid[0000-0001-5913-0828]{S.~Trincaz-Duvoid}$^\textrm{\scriptsize 130}$,    
\AtlasOrcid[0000-0001-6204-4445]{D.A.~Trischuk}$^\textrm{\scriptsize 169}$,    
\AtlasOrcid{W.~Trischuk}$^\textrm{\scriptsize 161}$,    
\AtlasOrcid[0000-0001-9500-2487]{B.~Trocm\'e}$^\textrm{\scriptsize 56}$,    
\AtlasOrcid[0000-0001-7688-5165]{A.~Trofymov}$^\textrm{\scriptsize 62}$,    
\AtlasOrcid[0000-0002-7997-8524]{C.~Troncon}$^\textrm{\scriptsize 66a}$,    
\AtlasOrcid[0000-0003-1041-9131]{F.~Trovato}$^\textrm{\scriptsize 151}$,    
\AtlasOrcid[0000-0001-8249-7150]{L.~Truong}$^\textrm{\scriptsize 31c}$,    
\AtlasOrcid[0000-0002-5151-7101]{M.~Trzebinski}$^\textrm{\scriptsize 82}$,    
\AtlasOrcid[0000-0001-6938-5867]{A.~Trzupek}$^\textrm{\scriptsize 82}$,    
\AtlasOrcid[0000-0001-7878-6435]{F.~Tsai}$^\textrm{\scriptsize 150}$,    
\AtlasOrcid[0000-0002-8761-4632]{A.~Tsiamis}$^\textrm{\scriptsize 157}$,    
\AtlasOrcid{P.V.~Tsiareshka}$^\textrm{\scriptsize 104,ad}$,    
\AtlasOrcid[0000-0002-6632-0440]{A.~Tsirigotis}$^\textrm{\scriptsize 157,u}$,    
\AtlasOrcid[0000-0002-2119-8875]{V.~Tsiskaridze}$^\textrm{\scriptsize 150}$,    
\AtlasOrcid{E.G.~Tskhadadze}$^\textrm{\scriptsize 154a}$,    
\AtlasOrcid[0000-0002-9104-2884]{M.~Tsopoulou}$^\textrm{\scriptsize 157}$,    
\AtlasOrcid[0000-0002-8965-6676]{I.I.~Tsukerman}$^\textrm{\scriptsize 119}$,    
\AtlasOrcid[0000-0001-8157-6711]{V.~Tsulaia}$^\textrm{\scriptsize 16}$,    
\AtlasOrcid[0000-0002-2055-4364]{S.~Tsuno}$^\textrm{\scriptsize 79}$,    
\AtlasOrcid{O.~Tsur}$^\textrm{\scriptsize 155}$,    
\AtlasOrcid[0000-0001-8212-6894]{D.~Tsybychev}$^\textrm{\scriptsize 150}$,    
\AtlasOrcid[0000-0002-5865-183X]{Y.~Tu}$^\textrm{\scriptsize 60b}$,    
\AtlasOrcid[0000-0001-6307-1437]{A.~Tudorache}$^\textrm{\scriptsize 25b}$,    
\AtlasOrcid[0000-0001-5384-3843]{V.~Tudorache}$^\textrm{\scriptsize 25b}$,    
\AtlasOrcid[0000-0002-7672-7754]{A.N.~Tuna}$^\textrm{\scriptsize 34}$,    
\AtlasOrcid[0000-0001-6506-3123]{S.~Turchikhin}$^\textrm{\scriptsize 77}$,    
\AtlasOrcid[0000-0002-0726-5648]{I.~Turk~Cakir}$^\textrm{\scriptsize 3a}$,    
\AtlasOrcid{R.J.~Turner}$^\textrm{\scriptsize 19}$,    
\AtlasOrcid[0000-0001-8740-796X]{R.~Turra}$^\textrm{\scriptsize 66a}$,    
\AtlasOrcid[0000-0001-6131-5725]{P.M.~Tuts}$^\textrm{\scriptsize 37}$,    
\AtlasOrcid[0000-0002-8363-1072]{S.~Tzamarias}$^\textrm{\scriptsize 157}$,    
\AtlasOrcid[0000-0001-6828-1599]{P.~Tzanis}$^\textrm{\scriptsize 9}$,    
\AtlasOrcid[0000-0002-0410-0055]{E.~Tzovara}$^\textrm{\scriptsize 96}$,    
\AtlasOrcid{K.~Uchida}$^\textrm{\scriptsize 158}$,    
\AtlasOrcid[0000-0002-9813-7931]{F.~Ukegawa}$^\textrm{\scriptsize 163}$,    
\AtlasOrcid[0000-0001-8130-7423]{G.~Unal}$^\textrm{\scriptsize 34}$,    
\AtlasOrcid[0000-0002-1646-0621]{M.~Unal}$^\textrm{\scriptsize 10}$,    
\AtlasOrcid[0000-0002-1384-286X]{A.~Undrus}$^\textrm{\scriptsize 27}$,    
\AtlasOrcid[0000-0002-3274-6531]{G.~Unel}$^\textrm{\scriptsize 165}$,    
\AtlasOrcid[0000-0003-2005-595X]{F.C.~Ungaro}$^\textrm{\scriptsize 101}$,    
\AtlasOrcid[0000-0002-2209-8198]{K.~Uno}$^\textrm{\scriptsize 158}$,    
\AtlasOrcid[0000-0002-7633-8441]{J.~Urban}$^\textrm{\scriptsize 26b}$,    
\AtlasOrcid[0000-0002-0887-7953]{P.~Urquijo}$^\textrm{\scriptsize 101}$,    
\AtlasOrcid[0000-0001-5032-7907]{G.~Usai}$^\textrm{\scriptsize 7}$,    
\AtlasOrcid[0000-0002-4241-8937]{R.~Ushioda}$^\textrm{\scriptsize 159}$,    
\AtlasOrcid[0000-0003-1950-0307]{M.~Usman}$^\textrm{\scriptsize 106}$,    
\AtlasOrcid[0000-0002-7110-8065]{Z.~Uysal}$^\textrm{\scriptsize 11d}$,    
\AtlasOrcid[0000-0001-9584-0392]{V.~Vacek}$^\textrm{\scriptsize 136}$,    
\AtlasOrcid[0000-0001-8703-6978]{B.~Vachon}$^\textrm{\scriptsize 100}$,    
\AtlasOrcid[0000-0001-6729-1584]{K.O.H.~Vadla}$^\textrm{\scriptsize 128}$,    
\AtlasOrcid[0000-0003-1492-5007]{T.~Vafeiadis}$^\textrm{\scriptsize 34}$,    
\AtlasOrcid[0000-0001-9362-8451]{C.~Valderanis}$^\textrm{\scriptsize 110}$,    
\AtlasOrcid[0000-0001-9931-2896]{E.~Valdes~Santurio}$^\textrm{\scriptsize 43a,43b}$,    
\AtlasOrcid[0000-0002-0486-9569]{M.~Valente}$^\textrm{\scriptsize 162a}$,    
\AtlasOrcid[0000-0003-2044-6539]{S.~Valentinetti}$^\textrm{\scriptsize 21b,21a}$,    
\AtlasOrcid[0000-0002-9776-5880]{A.~Valero}$^\textrm{\scriptsize 168}$,    
\AtlasOrcid[0000-0002-5510-1111]{L.~Val\'ery}$^\textrm{\scriptsize 44}$,    
\AtlasOrcid[0000-0002-6782-1941]{R.A.~Vallance}$^\textrm{\scriptsize 19}$,    
\AtlasOrcid[0000-0002-5496-349X]{A.~Vallier}$^\textrm{\scriptsize 98}$,    
\AtlasOrcid[0000-0002-3953-3117]{J.A.~Valls~Ferrer}$^\textrm{\scriptsize 168}$,    
\AtlasOrcid[0000-0002-2254-125X]{T.R.~Van~Daalen}$^\textrm{\scriptsize 143}$,    
\AtlasOrcid[0000-0002-7227-4006]{P.~Van~Gemmeren}$^\textrm{\scriptsize 5}$,    
\AtlasOrcid[0000-0002-7969-0301]{S.~Van~Stroud}$^\textrm{\scriptsize 92}$,    
\AtlasOrcid[0000-0001-7074-5655]{I.~Van~Vulpen}$^\textrm{\scriptsize 115}$,    
\AtlasOrcid[0000-0003-2684-276X]{M.~Vanadia}$^\textrm{\scriptsize 71a,71b}$,    
\AtlasOrcid[0000-0001-6581-9410]{W.~Vandelli}$^\textrm{\scriptsize 34}$,    
\AtlasOrcid[0000-0001-9055-4020]{M.~Vandenbroucke}$^\textrm{\scriptsize 139}$,    
\AtlasOrcid[0000-0003-3453-6156]{E.R.~Vandewall}$^\textrm{\scriptsize 124}$,    
\AtlasOrcid[0000-0001-6814-4674]{D.~Vannicola}$^\textrm{\scriptsize 156}$,    
\AtlasOrcid[0000-0002-9866-6040]{L.~Vannoli}$^\textrm{\scriptsize 53b,53a}$,    
\AtlasOrcid[0000-0002-2814-1337]{R.~Vari}$^\textrm{\scriptsize 70a}$,    
\AtlasOrcid[0000-0001-7820-9144]{E.W.~Varnes}$^\textrm{\scriptsize 6}$,    
\AtlasOrcid[0000-0001-6733-4310]{C.~Varni}$^\textrm{\scriptsize 16}$,    
\AtlasOrcid[0000-0002-0697-5808]{T.~Varol}$^\textrm{\scriptsize 153}$,    
\AtlasOrcid[0000-0002-0734-4442]{D.~Varouchas}$^\textrm{\scriptsize 62}$,    
\AtlasOrcid[0000-0003-1017-1295]{K.E.~Varvell}$^\textrm{\scriptsize 152}$,    
\AtlasOrcid[0000-0001-8415-0759]{M.E.~Vasile}$^\textrm{\scriptsize 25b}$,    
\AtlasOrcid{L.~Vaslin}$^\textrm{\scriptsize 36}$,    
\AtlasOrcid[0000-0002-3285-7004]{G.A.~Vasquez}$^\textrm{\scriptsize 170}$,    
\AtlasOrcid[0000-0003-1631-2714]{F.~Vazeille}$^\textrm{\scriptsize 36}$,    
\AtlasOrcid[0000-0002-5551-3546]{D.~Vazquez~Furelos}$^\textrm{\scriptsize 12}$,    
\AtlasOrcid[0000-0002-9780-099X]{T.~Vazquez~Schroeder}$^\textrm{\scriptsize 34}$,    
\AtlasOrcid[0000-0003-0855-0958]{J.~Veatch}$^\textrm{\scriptsize 51}$,    
\AtlasOrcid[0000-0002-1351-6757]{V.~Vecchio}$^\textrm{\scriptsize 97}$,    
\AtlasOrcid[0000-0001-5284-2451]{M.J.~Veen}$^\textrm{\scriptsize 115}$,    
\AtlasOrcid[0000-0003-2432-3309]{I.~Veliscek}$^\textrm{\scriptsize 129}$,    
\AtlasOrcid[0000-0003-1827-2955]{L.M.~Veloce}$^\textrm{\scriptsize 161}$,    
\AtlasOrcid[0000-0002-5956-4244]{F.~Veloso}$^\textrm{\scriptsize 134a,134c}$,    
\AtlasOrcid[0000-0002-2598-2659]{S.~Veneziano}$^\textrm{\scriptsize 70a}$,    
\AtlasOrcid[0000-0002-3368-3413]{A.~Ventura}$^\textrm{\scriptsize 65a,65b}$,    
\AtlasOrcid[0000-0002-3713-8033]{A.~Verbytskyi}$^\textrm{\scriptsize 111}$,    
\AtlasOrcid[0000-0001-8209-4757]{M.~Verducci}$^\textrm{\scriptsize 69a,69b}$,    
\AtlasOrcid[0000-0002-3228-6715]{C.~Vergis}$^\textrm{\scriptsize 22}$,    
\AtlasOrcid[0000-0001-8060-2228]{M.~Verissimo~De~Araujo}$^\textrm{\scriptsize 78b}$,    
\AtlasOrcid[0000-0001-5468-2025]{W.~Verkerke}$^\textrm{\scriptsize 115}$,    
\AtlasOrcid[0000-0002-8884-7112]{A.T.~Vermeulen}$^\textrm{\scriptsize 115}$,    
\AtlasOrcid[0000-0003-4378-5736]{J.C.~Vermeulen}$^\textrm{\scriptsize 115}$,    
\AtlasOrcid[0000-0002-0235-1053]{C.~Vernieri}$^\textrm{\scriptsize 148}$,    
\AtlasOrcid[0000-0002-4233-7563]{P.J.~Verschuuren}$^\textrm{\scriptsize 91}$,    
\AtlasOrcid[0000-0001-8669-9139]{M.~Vessella}$^\textrm{\scriptsize 99}$,    
\AtlasOrcid[0000-0002-6966-5081]{M.L.~Vesterbacka}$^\textrm{\scriptsize 120}$,    
\AtlasOrcid[0000-0002-7223-2965]{M.C.~Vetterli}$^\textrm{\scriptsize 147,aj}$,    
\AtlasOrcid[0000-0002-7011-9432]{A.~Vgenopoulos}$^\textrm{\scriptsize 157}$,    
\AtlasOrcid[0000-0002-5102-9140]{N.~Viaux~Maira}$^\textrm{\scriptsize 141d}$,    
\AtlasOrcid[0000-0002-1596-2611]{T.~Vickey}$^\textrm{\scriptsize 144}$,    
\AtlasOrcid[0000-0002-6497-6809]{O.E.~Vickey~Boeriu}$^\textrm{\scriptsize 144}$,    
\AtlasOrcid[0000-0002-0237-292X]{G.H.A.~Viehhauser}$^\textrm{\scriptsize 129}$,    
\AtlasOrcid[0000-0002-6270-9176]{L.~Vigani}$^\textrm{\scriptsize 59b}$,    
\AtlasOrcid[0000-0002-9181-8048]{M.~Villa}$^\textrm{\scriptsize 21b,21a}$,    
\AtlasOrcid[0000-0002-0048-4602]{M.~Villaplana~Perez}$^\textrm{\scriptsize 168}$,    
\AtlasOrcid{E.M.~Villhauer}$^\textrm{\scriptsize 48}$,    
\AtlasOrcid[0000-0002-4839-6281]{E.~Vilucchi}$^\textrm{\scriptsize 49}$,    
\AtlasOrcid[0000-0002-5338-8972]{M.G.~Vincter}$^\textrm{\scriptsize 32}$,    
\AtlasOrcid[0000-0002-6779-5595]{G.S.~Virdee}$^\textrm{\scriptsize 19}$,    
\AtlasOrcid[0000-0001-8832-0313]{A.~Vishwakarma}$^\textrm{\scriptsize 48}$,    
\AtlasOrcid[0000-0001-9156-970X]{C.~Vittori}$^\textrm{\scriptsize 21b,21a}$,    
\AtlasOrcid[0000-0003-0097-123X]{I.~Vivarelli}$^\textrm{\scriptsize 151}$,    
\AtlasOrcid{V.~Vladimirov}$^\textrm{\scriptsize 172}$,    
\AtlasOrcid[0000-0003-2987-3772]{E.~Voevodina}$^\textrm{\scriptsize 111}$,    
\AtlasOrcid[0000-0003-0672-6868]{M.~Vogel}$^\textrm{\scriptsize 176}$,    
\AtlasOrcid[0000-0002-3429-4778]{P.~Vokac}$^\textrm{\scriptsize 136}$,    
\AtlasOrcid[0000-0003-4032-0079]{J.~Von~Ahnen}$^\textrm{\scriptsize 44}$,    
\AtlasOrcid[0000-0001-8899-4027]{E.~Von~Toerne}$^\textrm{\scriptsize 22}$,    
\AtlasOrcid[0000-0001-8757-2180]{V.~Vorobel}$^\textrm{\scriptsize 137}$,    
\AtlasOrcid[0000-0002-7110-8516]{K.~Vorobev}$^\textrm{\scriptsize 108}$,    
\AtlasOrcid[0000-0001-8474-5357]{M.~Vos}$^\textrm{\scriptsize 168}$,    
\AtlasOrcid[0000-0001-8178-8503]{J.H.~Vossebeld}$^\textrm{\scriptsize 88}$,    
\AtlasOrcid[0000-0002-7561-204X]{M.~Vozak}$^\textrm{\scriptsize 97}$,    
\AtlasOrcid[0000-0003-2541-4827]{L.~Vozdecky}$^\textrm{\scriptsize 90}$,    
\AtlasOrcid[0000-0001-5415-5225]{N.~Vranjes}$^\textrm{\scriptsize 14}$,    
\AtlasOrcid[0000-0003-4477-9733]{M.~Vranjes~Milosavljevic}$^\textrm{\scriptsize 14}$,    
\AtlasOrcid{V.~Vrba}$^\textrm{\scriptsize 136,*}$,    
\AtlasOrcid[0000-0001-8083-0001]{M.~Vreeswijk}$^\textrm{\scriptsize 115}$,    
\AtlasOrcid[0000-0002-6251-1178]{N.K.~Vu}$^\textrm{\scriptsize 98}$,    
\AtlasOrcid[0000-0003-3208-9209]{R.~Vuillermet}$^\textrm{\scriptsize 34}$,    
\AtlasOrcid[0000-0003-3473-7038]{O.V.~Vujinovic}$^\textrm{\scriptsize 96}$,    
\AtlasOrcid[0000-0003-0472-3516]{I.~Vukotic}$^\textrm{\scriptsize 35}$,    
\AtlasOrcid[0000-0002-8600-9799]{S.~Wada}$^\textrm{\scriptsize 163}$,    
\AtlasOrcid{C.~Wagner}$^\textrm{\scriptsize 99}$,    
\AtlasOrcid[0000-0002-9198-5911]{W.~Wagner}$^\textrm{\scriptsize 176}$,    
\AtlasOrcid[0000-0002-6324-8551]{S.~Wahdan}$^\textrm{\scriptsize 176}$,    
\AtlasOrcid[0000-0003-0616-7330]{H.~Wahlberg}$^\textrm{\scriptsize 86}$,    
\AtlasOrcid[0000-0002-8438-7753]{R.~Wakasa}$^\textrm{\scriptsize 163}$,    
\AtlasOrcid[0000-0002-5808-6228]{M.~Wakida}$^\textrm{\scriptsize 112}$,    
\AtlasOrcid[0000-0002-7385-6139]{V.M.~Walbrecht}$^\textrm{\scriptsize 111}$,    
\AtlasOrcid[0000-0002-9039-8758]{J.~Walder}$^\textrm{\scriptsize 138}$,    
\AtlasOrcid[0000-0001-8535-4809]{R.~Walker}$^\textrm{\scriptsize 110}$,    
\AtlasOrcid{S.D.~Walker}$^\textrm{\scriptsize 91}$,    
\AtlasOrcid[0000-0002-0385-3784]{W.~Walkowiak}$^\textrm{\scriptsize 146}$,    
\AtlasOrcid[0000-0001-8972-3026]{A.M.~Wang}$^\textrm{\scriptsize 57}$,    
\AtlasOrcid[0000-0003-2482-711X]{A.Z.~Wang}$^\textrm{\scriptsize 175}$,    
\AtlasOrcid[0000-0001-9116-055X]{C.~Wang}$^\textrm{\scriptsize 58a}$,    
\AtlasOrcid[0000-0002-8487-8480]{C.~Wang}$^\textrm{\scriptsize 58c}$,    
\AtlasOrcid[0000-0003-3952-8139]{H.~Wang}$^\textrm{\scriptsize 16}$,    
\AtlasOrcid[0000-0002-5246-5497]{J.~Wang}$^\textrm{\scriptsize 60a}$,    
\AtlasOrcid[0000-0002-6730-1524]{P.~Wang}$^\textrm{\scriptsize 40}$,    
\AtlasOrcid[0000-0002-5059-8456]{R.-J.~Wang}$^\textrm{\scriptsize 96}$,    
\AtlasOrcid[0000-0001-9839-608X]{R.~Wang}$^\textrm{\scriptsize 57}$,    
\AtlasOrcid[0000-0001-8530-6487]{R.~Wang}$^\textrm{\scriptsize 116}$,    
\AtlasOrcid[0000-0002-5821-4875]{S.M.~Wang}$^\textrm{\scriptsize 153}$,    
\AtlasOrcid{S.~Wang}$^\textrm{\scriptsize 58b}$,    
\AtlasOrcid[0000-0002-1152-2221]{T.~Wang}$^\textrm{\scriptsize 58a}$,    
\AtlasOrcid[0000-0002-7184-9891]{W.T.~Wang}$^\textrm{\scriptsize 75}$,    
\AtlasOrcid[0000-0002-1444-6260]{W.X.~Wang}$^\textrm{\scriptsize 58a}$,    
\AtlasOrcid[0000-0002-6229-1945]{X.~Wang}$^\textrm{\scriptsize 13c}$,    
\AtlasOrcid[0000-0002-2411-7399]{X.~Wang}$^\textrm{\scriptsize 167}$,    
\AtlasOrcid[0000-0001-5173-2234]{X.~Wang}$^\textrm{\scriptsize 58c}$,    
\AtlasOrcid[0000-0003-2693-3442]{Y.~Wang}$^\textrm{\scriptsize 58a}$,    
\AtlasOrcid[0000-0002-0928-2070]{Z.~Wang}$^\textrm{\scriptsize 102}$,    
\AtlasOrcid[0000-0002-8178-5705]{C.~Wanotayaroj}$^\textrm{\scriptsize 34}$,    
\AtlasOrcid[0000-0002-2298-7315]{A.~Warburton}$^\textrm{\scriptsize 100}$,    
\AtlasOrcid[0000-0002-5162-533X]{C.P.~Ward}$^\textrm{\scriptsize 30}$,    
\AtlasOrcid[0000-0001-5530-9919]{R.J.~Ward}$^\textrm{\scriptsize 19}$,    
\AtlasOrcid[0000-0002-8268-8325]{N.~Warrack}$^\textrm{\scriptsize 55}$,    
\AtlasOrcid[0000-0001-7052-7973]{A.T.~Watson}$^\textrm{\scriptsize 19}$,    
\AtlasOrcid[0000-0002-9724-2684]{M.F.~Watson}$^\textrm{\scriptsize 19}$,    
\AtlasOrcid[0000-0002-0753-7308]{G.~Watts}$^\textrm{\scriptsize 143}$,    
\AtlasOrcid[0000-0003-0872-8920]{B.M.~Waugh}$^\textrm{\scriptsize 92}$,    
\AtlasOrcid[0000-0002-6700-7608]{A.F.~Webb}$^\textrm{\scriptsize 10}$,    
\AtlasOrcid[0000-0002-8659-5767]{C.~Weber}$^\textrm{\scriptsize 27}$,    
\AtlasOrcid[0000-0002-2770-9031]{M.S.~Weber}$^\textrm{\scriptsize 18}$,    
\AtlasOrcid[0000-0003-1710-4298]{S.A.~Weber}$^\textrm{\scriptsize 32}$,    
\AtlasOrcid[0000-0002-2841-1616]{S.M.~Weber}$^\textrm{\scriptsize 59a}$,    
\AtlasOrcid{C.~Wei}$^\textrm{\scriptsize 58a}$,    
\AtlasOrcid[0000-0001-9725-2316]{Y.~Wei}$^\textrm{\scriptsize 129}$,    
\AtlasOrcid[0000-0002-5158-307X]{A.R.~Weidberg}$^\textrm{\scriptsize 129}$,    
\AtlasOrcid[0000-0003-2165-871X]{J.~Weingarten}$^\textrm{\scriptsize 45}$,    
\AtlasOrcid[0000-0002-5129-872X]{M.~Weirich}$^\textrm{\scriptsize 96}$,    
\AtlasOrcid[0000-0002-6456-6834]{C.~Weiser}$^\textrm{\scriptsize 50}$,    
\AtlasOrcid[0000-0002-8678-893X]{T.~Wenaus}$^\textrm{\scriptsize 27}$,    
\AtlasOrcid[0000-0003-1623-3899]{B.~Wendland}$^\textrm{\scriptsize 45}$,    
\AtlasOrcid[0000-0002-4375-5265]{T.~Wengler}$^\textrm{\scriptsize 34}$,    
\AtlasOrcid[0000-0002-4770-377X]{S.~Wenig}$^\textrm{\scriptsize 34}$,    
\AtlasOrcid[0000-0001-9971-0077]{N.~Wermes}$^\textrm{\scriptsize 22}$,    
\AtlasOrcid[0000-0002-8192-8999]{M.~Wessels}$^\textrm{\scriptsize 59a}$,    
\AtlasOrcid[0000-0002-9383-8763]{K.~Whalen}$^\textrm{\scriptsize 126}$,    
\AtlasOrcid[0000-0002-9507-1869]{A.M.~Wharton}$^\textrm{\scriptsize 87}$,    
\AtlasOrcid[0000-0003-0714-1466]{A.S.~White}$^\textrm{\scriptsize 57}$,    
\AtlasOrcid[0000-0001-8315-9778]{A.~White}$^\textrm{\scriptsize 7}$,    
\AtlasOrcid[0000-0001-5474-4580]{M.J.~White}$^\textrm{\scriptsize 1}$,    
\AtlasOrcid[0000-0002-2005-3113]{D.~Whiteson}$^\textrm{\scriptsize 165}$,    
\AtlasOrcid[0000-0002-2711-4820]{L.~Wickremasinghe}$^\textrm{\scriptsize 127}$,    
\AtlasOrcid[0000-0003-3605-3633]{W.~Wiedenmann}$^\textrm{\scriptsize 175}$,    
\AtlasOrcid[0000-0003-1995-9185]{C.~Wiel}$^\textrm{\scriptsize 46}$,    
\AtlasOrcid[0000-0001-9232-4827]{M.~Wielers}$^\textrm{\scriptsize 138}$,    
\AtlasOrcid{N.~Wieseotte}$^\textrm{\scriptsize 96}$,    
\AtlasOrcid[0000-0001-6219-8946]{C.~Wiglesworth}$^\textrm{\scriptsize 38}$,    
\AtlasOrcid[0000-0002-5035-8102]{L.A.M.~Wiik-Fuchs}$^\textrm{\scriptsize 50}$,    
\AtlasOrcid{D.J.~Wilbern}$^\textrm{\scriptsize 123}$,    
\AtlasOrcid[0000-0002-8483-9502]{H.G.~Wilkens}$^\textrm{\scriptsize 34}$,    
\AtlasOrcid[0000-0002-7092-3500]{L.J.~Wilkins}$^\textrm{\scriptsize 91}$,    
\AtlasOrcid[0000-0002-5646-1856]{D.M.~Williams}$^\textrm{\scriptsize 37}$,    
\AtlasOrcid{H.H.~Williams}$^\textrm{\scriptsize 131}$,    
\AtlasOrcid[0000-0001-6174-401X]{S.~Williams}$^\textrm{\scriptsize 30}$,    
\AtlasOrcid[0000-0002-4120-1453]{S.~Willocq}$^\textrm{\scriptsize 99}$,    
\AtlasOrcid[0000-0001-5038-1399]{P.J.~Windischhofer}$^\textrm{\scriptsize 129}$,    
\AtlasOrcid[0000-0001-9473-7836]{I.~Wingerter-Seez}$^\textrm{\scriptsize 4}$,    
\AtlasOrcid[0000-0001-8290-3200]{F.~Winklmeier}$^\textrm{\scriptsize 126}$,    
\AtlasOrcid[0000-0001-9606-7688]{B.T.~Winter}$^\textrm{\scriptsize 50}$,    
\AtlasOrcid{M.~Wittgen}$^\textrm{\scriptsize 148}$,    
\AtlasOrcid[0000-0002-0688-3380]{M.~Wobisch}$^\textrm{\scriptsize 93}$,    
\AtlasOrcid[0000-0002-4368-9202]{A.~Wolf}$^\textrm{\scriptsize 96}$,    
\AtlasOrcid[0000-0002-7402-369X]{R.~W\"olker}$^\textrm{\scriptsize 129}$,    
\AtlasOrcid{J.~Wollrath}$^\textrm{\scriptsize 165}$,    
\AtlasOrcid[0000-0001-9184-2921]{M.W.~Wolter}$^\textrm{\scriptsize 82}$,    
\AtlasOrcid[0000-0002-9588-1773]{H.~Wolters}$^\textrm{\scriptsize 134a,134c}$,    
\AtlasOrcid[0000-0001-5975-8164]{V.W.S.~Wong}$^\textrm{\scriptsize 169}$,    
\AtlasOrcid[0000-0002-6620-6277]{A.F.~Wongel}$^\textrm{\scriptsize 44}$,    
\AtlasOrcid[0000-0002-3865-4996]{S.D.~Worm}$^\textrm{\scriptsize 44}$,    
\AtlasOrcid[0000-0003-4273-6334]{B.K.~Wosiek}$^\textrm{\scriptsize 82}$,    
\AtlasOrcid[0000-0003-1171-0887]{K.W.~Wo\'{z}niak}$^\textrm{\scriptsize 82}$,    
\AtlasOrcid[0000-0002-3298-4900]{K.~Wraight}$^\textrm{\scriptsize 55}$,    
\AtlasOrcid[0000-0002-3173-0802]{J.~Wu}$^\textrm{\scriptsize 13a,13d}$,    
\AtlasOrcid[0000-0001-5866-1504]{S.L.~Wu}$^\textrm{\scriptsize 175}$,    
\AtlasOrcid[0000-0001-7655-389X]{X.~Wu}$^\textrm{\scriptsize 52}$,    
\AtlasOrcid[0000-0002-1528-4865]{Y.~Wu}$^\textrm{\scriptsize 58a}$,    
\AtlasOrcid[0000-0002-5392-902X]{Z.~Wu}$^\textrm{\scriptsize 139,58a}$,    
\AtlasOrcid[0000-0002-4055-218X]{J.~Wuerzinger}$^\textrm{\scriptsize 129}$,    
\AtlasOrcid[0000-0001-9690-2997]{T.R.~Wyatt}$^\textrm{\scriptsize 97}$,    
\AtlasOrcid[0000-0001-9895-4475]{B.M.~Wynne}$^\textrm{\scriptsize 48}$,    
\AtlasOrcid[0000-0002-0988-1655]{S.~Xella}$^\textrm{\scriptsize 38}$,    
\AtlasOrcid[0000-0003-3073-3662]{L.~Xia}$^\textrm{\scriptsize 13c}$,    
\AtlasOrcid{M.~Xia}$^\textrm{\scriptsize 13b}$,    
\AtlasOrcid[0000-0002-7684-8257]{J.~Xiang}$^\textrm{\scriptsize 60c}$,    
\AtlasOrcid[0000-0002-1344-8723]{X.~Xiao}$^\textrm{\scriptsize 102}$,    
\AtlasOrcid[0000-0001-6707-5590]{M.~Xie}$^\textrm{\scriptsize 58a}$,    
\AtlasOrcid[0000-0001-6473-7886]{X.~Xie}$^\textrm{\scriptsize 58a}$,    
\AtlasOrcid{I.~Xiotidis}$^\textrm{\scriptsize 151}$,    
\AtlasOrcid[0000-0001-6355-2767]{D.~Xu}$^\textrm{\scriptsize 13a}$,    
\AtlasOrcid{H.~Xu}$^\textrm{\scriptsize 58a}$,    
\AtlasOrcid[0000-0001-6110-2172]{H.~Xu}$^\textrm{\scriptsize 58a}$,    
\AtlasOrcid[0000-0001-8997-3199]{L.~Xu}$^\textrm{\scriptsize 58a}$,    
\AtlasOrcid[0000-0002-1928-1717]{R.~Xu}$^\textrm{\scriptsize 131}$,    
\AtlasOrcid[0000-0002-0215-6151]{T.~Xu}$^\textrm{\scriptsize 58a}$,    
\AtlasOrcid[0000-0001-5661-1917]{W.~Xu}$^\textrm{\scriptsize 102}$,    
\AtlasOrcid[0000-0001-9563-4804]{Y.~Xu}$^\textrm{\scriptsize 13b}$,    
\AtlasOrcid[0000-0001-9571-3131]{Z.~Xu}$^\textrm{\scriptsize 58b}$,    
\AtlasOrcid[0000-0001-9602-4901]{Z.~Xu}$^\textrm{\scriptsize 148}$,    
\AtlasOrcid[0000-0002-2680-0474]{B.~Yabsley}$^\textrm{\scriptsize 152}$,    
\AtlasOrcid[0000-0001-6977-3456]{S.~Yacoob}$^\textrm{\scriptsize 31a}$,    
\AtlasOrcid[0000-0002-6885-282X]{N.~Yamaguchi}$^\textrm{\scriptsize 85}$,    
\AtlasOrcid[0000-0002-3725-4800]{Y.~Yamaguchi}$^\textrm{\scriptsize 159}$,    
\AtlasOrcid{M.~Yamatani}$^\textrm{\scriptsize 158}$,    
\AtlasOrcid[0000-0003-2123-5311]{H.~Yamauchi}$^\textrm{\scriptsize 163}$,    
\AtlasOrcid[0000-0003-0411-3590]{T.~Yamazaki}$^\textrm{\scriptsize 16}$,    
\AtlasOrcid[0000-0003-3710-6995]{Y.~Yamazaki}$^\textrm{\scriptsize 80}$,    
\AtlasOrcid{J.~Yan}$^\textrm{\scriptsize 58c}$,    
\AtlasOrcid[0000-0002-1512-5506]{S.~Yan}$^\textrm{\scriptsize 129}$,    
\AtlasOrcid[0000-0002-2483-4937]{Z.~Yan}$^\textrm{\scriptsize 23}$,    
\AtlasOrcid[0000-0001-7367-1380]{H.J.~Yang}$^\textrm{\scriptsize 58c,58d}$,    
\AtlasOrcid[0000-0003-3554-7113]{H.T.~Yang}$^\textrm{\scriptsize 16}$,    
\AtlasOrcid[0000-0002-0204-984X]{S.~Yang}$^\textrm{\scriptsize 58a}$,    
\AtlasOrcid[0000-0002-4996-1924]{T.~Yang}$^\textrm{\scriptsize 60c}$,    
\AtlasOrcid[0000-0002-1452-9824]{X.~Yang}$^\textrm{\scriptsize 58a}$,    
\AtlasOrcid[0000-0002-9201-0972]{X.~Yang}$^\textrm{\scriptsize 13a}$,    
\AtlasOrcid[0000-0001-8524-1855]{Y.~Yang}$^\textrm{\scriptsize 158}$,    
\AtlasOrcid[0000-0002-7374-2334]{Z.~Yang}$^\textrm{\scriptsize 102,58a}$,    
\AtlasOrcid[0000-0002-3335-1988]{W-M.~Yao}$^\textrm{\scriptsize 16}$,    
\AtlasOrcid[0000-0001-8939-666X]{Y.C.~Yap}$^\textrm{\scriptsize 44}$,    
\AtlasOrcid[0000-0002-4886-9851]{H.~Ye}$^\textrm{\scriptsize 13c}$,    
\AtlasOrcid[0000-0001-9274-707X]{J.~Ye}$^\textrm{\scriptsize 40}$,    
\AtlasOrcid[0000-0002-7864-4282]{S.~Ye}$^\textrm{\scriptsize 27}$,    
\AtlasOrcid[0000-0003-0586-7052]{I.~Yeletskikh}$^\textrm{\scriptsize 77}$,    
\AtlasOrcid[0000-0002-1827-9201]{M.R.~Yexley}$^\textrm{\scriptsize 87}$,    
\AtlasOrcid[0000-0003-2174-807X]{P.~Yin}$^\textrm{\scriptsize 37}$,    
\AtlasOrcid[0000-0003-1988-8401]{K.~Yorita}$^\textrm{\scriptsize 173}$,    
\AtlasOrcid[0000-0002-3656-2326]{K.~Yoshihara}$^\textrm{\scriptsize 76}$,    
\AtlasOrcid[0000-0001-5858-6639]{C.J.S.~Young}$^\textrm{\scriptsize 50}$,    
\AtlasOrcid[0000-0003-3268-3486]{C.~Young}$^\textrm{\scriptsize 148}$,    
\AtlasOrcid[0000-0002-8452-0315]{R.~Yuan}$^\textrm{\scriptsize 58b,i}$,    
\AtlasOrcid[0000-0001-6956-3205]{X.~Yue}$^\textrm{\scriptsize 59a}$,    
\AtlasOrcid[0000-0002-4105-2988]{M.~Zaazoua}$^\textrm{\scriptsize 33e}$,    
\AtlasOrcid[0000-0001-5626-0993]{B.~Zabinski}$^\textrm{\scriptsize 82}$,    
\AtlasOrcid[0000-0002-3156-4453]{G.~Zacharis}$^\textrm{\scriptsize 9}$,    
\AtlasOrcid{E.~Zaid}$^\textrm{\scriptsize 48}$,    
\AtlasOrcid[0000-0002-4961-8368]{A.M.~Zaitsev}$^\textrm{\scriptsize 118,ae}$,    
\AtlasOrcid[0000-0001-7909-4772]{T.~Zakareishvili}$^\textrm{\scriptsize 154b}$,    
\AtlasOrcid[0000-0002-4963-8836]{N.~Zakharchuk}$^\textrm{\scriptsize 32}$,    
\AtlasOrcid[0000-0002-4499-2545]{S.~Zambito}$^\textrm{\scriptsize 34}$,    
\AtlasOrcid[0000-0002-1222-7937]{D.~Zanzi}$^\textrm{\scriptsize 50}$,    
\AtlasOrcid[0000-0002-9037-2152]{S.V.~Zei{\ss}ner}$^\textrm{\scriptsize 45}$,    
\AtlasOrcid[0000-0003-2280-8636]{C.~Zeitnitz}$^\textrm{\scriptsize 176}$,    
\AtlasOrcid[0000-0002-2029-2659]{J.C.~Zeng}$^\textrm{\scriptsize 167}$,    
\AtlasOrcid[0000-0002-4867-3138]{D.T.~Zenger~Jr}$^\textrm{\scriptsize 24}$,    
\AtlasOrcid[0000-0002-5447-1989]{O.~Zenin}$^\textrm{\scriptsize 118}$,    
\AtlasOrcid[0000-0001-8265-6916]{T.~\v{Z}eni\v{s}}$^\textrm{\scriptsize 26a}$,    
\AtlasOrcid[0000-0002-9720-1794]{S.~Zenz}$^\textrm{\scriptsize 90}$,    
\AtlasOrcid[0000-0001-9101-3226]{S.~Zerradi}$^\textrm{\scriptsize 33a}$,    
\AtlasOrcid[0000-0002-4198-3029]{D.~Zerwas}$^\textrm{\scriptsize 62}$,    
\AtlasOrcid[0000-0002-9726-6707]{B.~Zhang}$^\textrm{\scriptsize 13c}$,    
\AtlasOrcid[0000-0001-7335-4983]{D.F.~Zhang}$^\textrm{\scriptsize 144}$,    
\AtlasOrcid[0000-0002-5706-7180]{G.~Zhang}$^\textrm{\scriptsize 13b}$,    
\AtlasOrcid[0000-0002-9907-838X]{J.~Zhang}$^\textrm{\scriptsize 5}$,    
\AtlasOrcid[0000-0002-9778-9209]{K.~Zhang}$^\textrm{\scriptsize 13a}$,    
\AtlasOrcid[0000-0002-9336-9338]{L.~Zhang}$^\textrm{\scriptsize 13c}$,    
\AtlasOrcid[0000-0001-8659-5727]{M.~Zhang}$^\textrm{\scriptsize 167}$,    
\AtlasOrcid[0000-0002-8265-474X]{R.~Zhang}$^\textrm{\scriptsize 175}$,    
\AtlasOrcid{S.~Zhang}$^\textrm{\scriptsize 102}$,    
\AtlasOrcid[0000-0003-4731-0754]{X.~Zhang}$^\textrm{\scriptsize 58c}$,    
\AtlasOrcid[0000-0003-4341-1603]{X.~Zhang}$^\textrm{\scriptsize 58b}$,    
\AtlasOrcid[0000-0002-7853-9079]{Z.~Zhang}$^\textrm{\scriptsize 62}$,    
\AtlasOrcid[0000-0003-0054-8749]{P.~Zhao}$^\textrm{\scriptsize 47}$,    
\AtlasOrcid[0000-0003-0494-6728]{Y.~Zhao}$^\textrm{\scriptsize 140}$,    
\AtlasOrcid[0000-0001-6758-3974]{Z.~Zhao}$^\textrm{\scriptsize 58a}$,    
\AtlasOrcid[0000-0002-3360-4965]{A.~Zhemchugov}$^\textrm{\scriptsize 77}$,    
\AtlasOrcid[0000-0002-8323-7753]{Z.~Zheng}$^\textrm{\scriptsize 148}$,    
\AtlasOrcid[0000-0001-9377-650X]{D.~Zhong}$^\textrm{\scriptsize 167}$,    
\AtlasOrcid{B.~Zhou}$^\textrm{\scriptsize 102}$,    
\AtlasOrcid[0000-0001-5904-7258]{C.~Zhou}$^\textrm{\scriptsize 175}$,    
\AtlasOrcid[0000-0002-7986-9045]{H.~Zhou}$^\textrm{\scriptsize 6}$,    
\AtlasOrcid[0000-0002-1775-2511]{N.~Zhou}$^\textrm{\scriptsize 58c}$,    
\AtlasOrcid{Y.~Zhou}$^\textrm{\scriptsize 6}$,    
\AtlasOrcid[0000-0001-8015-3901]{C.G.~Zhu}$^\textrm{\scriptsize 58b}$,    
\AtlasOrcid[0000-0002-5918-9050]{C.~Zhu}$^\textrm{\scriptsize 13a,13d}$,    
\AtlasOrcid[0000-0001-8479-1345]{H.L.~Zhu}$^\textrm{\scriptsize 58a}$,    
\AtlasOrcid[0000-0001-8066-7048]{H.~Zhu}$^\textrm{\scriptsize 13a}$,    
\AtlasOrcid[0000-0002-5278-2855]{J.~Zhu}$^\textrm{\scriptsize 102}$,    
\AtlasOrcid[0000-0002-7306-1053]{Y.~Zhu}$^\textrm{\scriptsize 58a}$,    
\AtlasOrcid[0000-0003-0996-3279]{X.~Zhuang}$^\textrm{\scriptsize 13a}$,    
\AtlasOrcid[0000-0003-2468-9634]{K.~Zhukov}$^\textrm{\scriptsize 107}$,    
\AtlasOrcid[0000-0002-0306-9199]{V.~Zhulanov}$^\textrm{\scriptsize 117b,117a}$,    
\AtlasOrcid[0000-0002-6311-7420]{D.~Zieminska}$^\textrm{\scriptsize 63}$,    
\AtlasOrcid[0000-0003-0277-4870]{N.I.~Zimine}$^\textrm{\scriptsize 77}$,    
\AtlasOrcid[0000-0002-1529-8925]{S.~Zimmermann}$^\textrm{\scriptsize 50,*}$,    
\AtlasOrcid{J.~Zinsser}$^\textrm{\scriptsize 59b}$,    
\AtlasOrcid[0000-0002-2891-8812]{M.~Ziolkowski}$^\textrm{\scriptsize 146}$,    
\AtlasOrcid[0000-0003-4236-8930]{L.~\v{Z}ivkovi\'{c}}$^\textrm{\scriptsize 14}$,    
\AtlasOrcid[0000-0002-0993-6185]{A.~Zoccoli}$^\textrm{\scriptsize 21b,21a}$,    
\AtlasOrcid[0000-0003-2138-6187]{K.~Zoch}$^\textrm{\scriptsize 52}$,    
\AtlasOrcid[0000-0003-2073-4901]{T.G.~Zorbas}$^\textrm{\scriptsize 144}$,    
\AtlasOrcid[0000-0003-3177-903X]{O.~Zormpa}$^\textrm{\scriptsize 42}$,    
\AtlasOrcid[0000-0002-0779-8815]{W.~Zou}$^\textrm{\scriptsize 37}$,    
\AtlasOrcid[0000-0002-9397-2313]{L.~Zwalinski}$^\textrm{\scriptsize 34}$.    
\bigskip
\\

$^{1}$Department of Physics, University of Adelaide, Adelaide; Australia.\\
$^{2}$Department of Physics, University of Alberta, Edmonton AB; Canada.\\
$^{3}$$^{(a)}$Department of Physics, Ankara University, Ankara;$^{(b)}$Istanbul Aydin University, Application and Research Center for Advanced Studies, Istanbul;$^{(c)}$Division of Physics, TOBB University of Economics and Technology, Ankara; Turkey.\\
$^{4}$LAPP, Univ. Savoie Mont Blanc, CNRS/IN2P3, Annecy ; France.\\
$^{5}$High Energy Physics Division, Argonne National Laboratory, Argonne IL; United States of America.\\
$^{6}$Department of Physics, University of Arizona, Tucson AZ; United States of America.\\
$^{7}$Department of Physics, University of Texas at Arlington, Arlington TX; United States of America.\\
$^{8}$Physics Department, National and Kapodistrian University of Athens, Athens; Greece.\\
$^{9}$Physics Department, National Technical University of Athens, Zografou; Greece.\\
$^{10}$Department of Physics, University of Texas at Austin, Austin TX; United States of America.\\
$^{11}$$^{(a)}$Bahcesehir University, Faculty of Engineering and Natural Sciences, Istanbul;$^{(b)}$Istanbul Bilgi University, Faculty of Engineering and Natural Sciences, Istanbul;$^{(c)}$Department of Physics, Bogazici University, Istanbul;$^{(d)}$Department of Physics Engineering, Gaziantep University, Gaziantep; Turkey.\\
$^{12}$Institut de F\'isica d'Altes Energies (IFAE), Barcelona Institute of Science and Technology, Barcelona; Spain.\\
$^{13}$$^{(a)}$Institute of High Energy Physics, Chinese Academy of Sciences, Beijing;$^{(b)}$Physics Department, Tsinghua University, Beijing;$^{(c)}$Department of Physics, Nanjing University, Nanjing;$^{(d)}$University of Chinese Academy of Science (UCAS), Beijing; China.\\
$^{14}$Institute of Physics, University of Belgrade, Belgrade; Serbia.\\
$^{15}$Department for Physics and Technology, University of Bergen, Bergen; Norway.\\
$^{16}$Physics Division, Lawrence Berkeley National Laboratory and University of California, Berkeley CA; United States of America.\\
$^{17}$Institut f\"{u}r Physik, Humboldt Universit\"{a}t zu Berlin, Berlin; Germany.\\
$^{18}$Albert Einstein Center for Fundamental Physics and Laboratory for High Energy Physics, University of Bern, Bern; Switzerland.\\
$^{19}$School of Physics and Astronomy, University of Birmingham, Birmingham; United Kingdom.\\
$^{20}$$^{(a)}$Facultad de Ciencias y Centro de Investigaci\'ones, Universidad Antonio Nari\~no, Bogot\'a;$^{(b)}$Departamento de F\'isica, Universidad Nacional de Colombia, Bogot\'a; Colombia.\\
$^{21}$$^{(a)}$Dipartimento di Fisica e Astronomia A. Righi, Università di Bologna, Bologna;$^{(b)}$INFN Sezione di Bologna; Italy.\\
$^{22}$Physikalisches Institut, Universit\"{a}t Bonn, Bonn; Germany.\\
$^{23}$Department of Physics, Boston University, Boston MA; United States of America.\\
$^{24}$Department of Physics, Brandeis University, Waltham MA; United States of America.\\
$^{25}$$^{(a)}$Transilvania University of Brasov, Brasov;$^{(b)}$Horia Hulubei National Institute of Physics and Nuclear Engineering, Bucharest;$^{(c)}$Department of Physics, Alexandru Ioan Cuza University of Iasi, Iasi;$^{(d)}$National Institute for Research and Development of Isotopic and Molecular Technologies, Physics Department, Cluj-Napoca;$^{(e)}$University Politehnica Bucharest, Bucharest;$^{(f)}$West University in Timisoara, Timisoara; Romania.\\
$^{26}$$^{(a)}$Faculty of Mathematics, Physics and Informatics, Comenius University, Bratislava;$^{(b)}$Department of Subnuclear Physics, Institute of Experimental Physics of the Slovak Academy of Sciences, Kosice; Slovak Republic.\\
$^{27}$Physics Department, Brookhaven National Laboratory, Upton NY; United States of America.\\
$^{28}$Departamento de F\'isica (FCEN) and IFIBA, Universidad de Buenos Aires and CONICET, Buenos Aires; Argentina.\\
$^{29}$California State University, CA; United States of America.\\
$^{30}$Cavendish Laboratory, University of Cambridge, Cambridge; United Kingdom.\\
$^{31}$$^{(a)}$Department of Physics, University of Cape Town, Cape Town;$^{(b)}$iThemba Labs, Western Cape;$^{(c)}$Department of Mechanical Engineering Science, University of Johannesburg, Johannesburg;$^{(d)}$National Institute of Physics, University of the Philippines Diliman (Philippines);$^{(e)}$University of South Africa, Department of Physics, Pretoria;$^{(f)}$School of Physics, University of the Witwatersrand, Johannesburg; South Africa.\\
$^{32}$Department of Physics, Carleton University, Ottawa ON; Canada.\\
$^{33}$$^{(a)}$Facult\'e des Sciences Ain Chock, R\'eseau Universitaire de Physique des Hautes Energies - Universit\'e Hassan II, Casablanca;$^{(b)}$Facult\'{e} des Sciences, Universit\'{e} Ibn-Tofail, K\'{e}nitra;$^{(c)}$Facult\'e des Sciences Semlalia, Universit\'e Cadi Ayyad, LPHEA-Marrakech;$^{(d)}$LPMR, Facult\'e des Sciences, Universit\'e Mohamed Premier, Oujda;$^{(e)}$Facult\'e des sciences, Universit\'e Mohammed V, Rabat; Morocco.\\
$^{34}$CERN, Geneva; Switzerland.\\
$^{35}$Enrico Fermi Institute, University of Chicago, Chicago IL; United States of America.\\
$^{36}$LPC, Universit\'e Clermont Auvergne, CNRS/IN2P3, Clermont-Ferrand; France.\\
$^{37}$Nevis Laboratory, Columbia University, Irvington NY; United States of America.\\
$^{38}$Niels Bohr Institute, University of Copenhagen, Copenhagen; Denmark.\\
$^{39}$$^{(a)}$Dipartimento di Fisica, Universit\`a della Calabria, Rende;$^{(b)}$INFN Gruppo Collegato di Cosenza, Laboratori Nazionali di Frascati; Italy.\\
$^{40}$Physics Department, Southern Methodist University, Dallas TX; United States of America.\\
$^{41}$Physics Department, University of Texas at Dallas, Richardson TX; United States of America.\\
$^{42}$National Centre for Scientific Research "Demokritos", Agia Paraskevi; Greece.\\
$^{43}$$^{(a)}$Department of Physics, Stockholm University;$^{(b)}$Oskar Klein Centre, Stockholm; Sweden.\\
$^{44}$Deutsches Elektronen-Synchrotron DESY, Hamburg and Zeuthen; Germany.\\
$^{45}$Fakult\"{a}t Physik , Technische Universit{\"a}t Dortmund, Dortmund; Germany.\\
$^{46}$Institut f\"{u}r Kern-~und Teilchenphysik, Technische Universit\"{a}t Dresden, Dresden; Germany.\\
$^{47}$Department of Physics, Duke University, Durham NC; United States of America.\\
$^{48}$SUPA - School of Physics and Astronomy, University of Edinburgh, Edinburgh; United Kingdom.\\
$^{49}$INFN e Laboratori Nazionali di Frascati, Frascati; Italy.\\
$^{50}$Physikalisches Institut, Albert-Ludwigs-Universit\"{a}t Freiburg, Freiburg; Germany.\\
$^{51}$II. Physikalisches Institut, Georg-August-Universit\"{a}t G\"ottingen, G\"ottingen; Germany.\\
$^{52}$D\'epartement de Physique Nucl\'eaire et Corpusculaire, Universit\'e de Gen\`eve, Gen\`eve; Switzerland.\\
$^{53}$$^{(a)}$Dipartimento di Fisica, Universit\`a di Genova, Genova;$^{(b)}$INFN Sezione di Genova; Italy.\\
$^{54}$II. Physikalisches Institut, Justus-Liebig-Universit{\"a}t Giessen, Giessen; Germany.\\
$^{55}$SUPA - School of Physics and Astronomy, University of Glasgow, Glasgow; United Kingdom.\\
$^{56}$LPSC, Universit\'e Grenoble Alpes, CNRS/IN2P3, Grenoble INP, Grenoble; France.\\
$^{57}$Laboratory for Particle Physics and Cosmology, Harvard University, Cambridge MA; United States of America.\\
$^{58}$$^{(a)}$Department of Modern Physics and State Key Laboratory of Particle Detection and Electronics, University of Science and Technology of China, Hefei;$^{(b)}$Institute of Frontier and Interdisciplinary Science and Key Laboratory of Particle Physics and Particle Irradiation (MOE), Shandong University, Qingdao;$^{(c)}$School of Physics and Astronomy, Shanghai Jiao Tong University, Key Laboratory for Particle Astrophysics and Cosmology (MOE), SKLPPC, Shanghai;$^{(d)}$Tsung-Dao Lee Institute, Shanghai; China.\\
$^{59}$$^{(a)}$Kirchhoff-Institut f\"{u}r Physik, Ruprecht-Karls-Universit\"{a}t Heidelberg, Heidelberg;$^{(b)}$Physikalisches Institut, Ruprecht-Karls-Universit\"{a}t Heidelberg, Heidelberg; Germany.\\
$^{60}$$^{(a)}$Department of Physics, Chinese University of Hong Kong, Shatin, N.T., Hong Kong;$^{(b)}$Department of Physics, University of Hong Kong, Hong Kong;$^{(c)}$Department of Physics and Institute for Advanced Study, Hong Kong University of Science and Technology, Clear Water Bay, Kowloon, Hong Kong; China.\\
$^{61}$Department of Physics, National Tsing Hua University, Hsinchu; Taiwan.\\
$^{62}$IJCLab, Universit\'e Paris-Saclay, CNRS/IN2P3, 91405, Orsay; France.\\
$^{63}$Department of Physics, Indiana University, Bloomington IN; United States of America.\\
$^{64}$$^{(a)}$INFN Gruppo Collegato di Udine, Sezione di Trieste, Udine;$^{(b)}$ICTP, Trieste;$^{(c)}$Dipartimento Politecnico di Ingegneria e Architettura, Universit\`a di Udine, Udine; Italy.\\
$^{65}$$^{(a)}$INFN Sezione di Lecce;$^{(b)}$Dipartimento di Matematica e Fisica, Universit\`a del Salento, Lecce; Italy.\\
$^{66}$$^{(a)}$INFN Sezione di Milano;$^{(b)}$Dipartimento di Fisica, Universit\`a di Milano, Milano; Italy.\\
$^{67}$$^{(a)}$INFN Sezione di Napoli;$^{(b)}$Dipartimento di Fisica, Universit\`a di Napoli, Napoli; Italy.\\
$^{68}$$^{(a)}$INFN Sezione di Pavia;$^{(b)}$Dipartimento di Fisica, Universit\`a di Pavia, Pavia; Italy.\\
$^{69}$$^{(a)}$INFN Sezione di Pisa;$^{(b)}$Dipartimento di Fisica E. Fermi, Universit\`a di Pisa, Pisa; Italy.\\
$^{70}$$^{(a)}$INFN Sezione di Roma;$^{(b)}$Dipartimento di Fisica, Sapienza Universit\`a di Roma, Roma; Italy.\\
$^{71}$$^{(a)}$INFN Sezione di Roma Tor Vergata;$^{(b)}$Dipartimento di Fisica, Universit\`a di Roma Tor Vergata, Roma; Italy.\\
$^{72}$$^{(a)}$INFN Sezione di Roma Tre;$^{(b)}$Dipartimento di Matematica e Fisica, Universit\`a Roma Tre, Roma; Italy.\\
$^{73}$$^{(a)}$INFN-TIFPA;$^{(b)}$Universit\`a degli Studi di Trento, Trento; Italy.\\
$^{74}$Institut f\"{u}r Astro-~und Teilchenphysik, Leopold-Franzens-Universit\"{a}t, Innsbruck; Austria.\\
$^{75}$University of Iowa, Iowa City IA; United States of America.\\
$^{76}$Department of Physics and Astronomy, Iowa State University, Ames IA; United States of America.\\
$^{77}$Joint Institute for Nuclear Research, Dubna; Russia.\\
$^{78}$$^{(a)}$Departamento de Engenharia El\'etrica, Universidade Federal de Juiz de Fora (UFJF), Juiz de Fora;$^{(b)}$Universidade Federal do Rio De Janeiro COPPE/EE/IF, Rio de Janeiro;$^{(c)}$Instituto de F\'isica, Universidade de S\~ao Paulo, S\~ao Paulo; Brazil.\\
$^{79}$KEK, High Energy Accelerator Research Organization, Tsukuba; Japan.\\
$^{80}$Graduate School of Science, Kobe University, Kobe; Japan.\\
$^{81}$$^{(a)}$AGH University of Science and Technology, Faculty of Physics and Applied Computer Science, Krakow;$^{(b)}$Marian Smoluchowski Institute of Physics, Jagiellonian University, Krakow; Poland.\\
$^{82}$Institute of Nuclear Physics Polish Academy of Sciences, Krakow; Poland.\\
$^{83}$Faculty of Science, Kyoto University, Kyoto; Japan.\\
$^{84}$Kyoto University of Education, Kyoto; Japan.\\
$^{85}$Research Center for Advanced Particle Physics and Department of Physics, Kyushu University, Fukuoka ; Japan.\\
$^{86}$Instituto de F\'{i}sica La Plata, Universidad Nacional de La Plata and CONICET, La Plata; Argentina.\\
$^{87}$Physics Department, Lancaster University, Lancaster; United Kingdom.\\
$^{88}$Oliver Lodge Laboratory, University of Liverpool, Liverpool; United Kingdom.\\
$^{89}$Department of Experimental Particle Physics, Jo\v{z}ef Stefan Institute and Department of Physics, University of Ljubljana, Ljubljana; Slovenia.\\
$^{90}$School of Physics and Astronomy, Queen Mary University of London, London; United Kingdom.\\
$^{91}$Department of Physics, Royal Holloway University of London, Egham; United Kingdom.\\
$^{92}$Department of Physics and Astronomy, University College London, London; United Kingdom.\\
$^{93}$Louisiana Tech University, Ruston LA; United States of America.\\
$^{94}$Fysiska institutionen, Lunds universitet, Lund; Sweden.\\
$^{95}$Departamento de F\'isica Teorica C-15 and CIAFF, Universidad Aut\'onoma de Madrid, Madrid; Spain.\\
$^{96}$Institut f\"{u}r Physik, Universit\"{a}t Mainz, Mainz; Germany.\\
$^{97}$School of Physics and Astronomy, University of Manchester, Manchester; United Kingdom.\\
$^{98}$CPPM, Aix-Marseille Universit\'e, CNRS/IN2P3, Marseille; France.\\
$^{99}$Department of Physics, University of Massachusetts, Amherst MA; United States of America.\\
$^{100}$Department of Physics, McGill University, Montreal QC; Canada.\\
$^{101}$School of Physics, University of Melbourne, Victoria; Australia.\\
$^{102}$Department of Physics, University of Michigan, Ann Arbor MI; United States of America.\\
$^{103}$Department of Physics and Astronomy, Michigan State University, East Lansing MI; United States of America.\\
$^{104}$B.I. Stepanov Institute of Physics, National Academy of Sciences of Belarus, Minsk; Belarus.\\
$^{105}$Research Institute for Nuclear Problems of Byelorussian State University, Minsk; Belarus.\\
$^{106}$Group of Particle Physics, University of Montreal, Montreal QC; Canada.\\
$^{107}$P.N. Lebedev Physical Institute of the Russian Academy of Sciences, Moscow; Russia.\\
$^{108}$National Research Nuclear University MEPhI, Moscow; Russia.\\
$^{109}$D.V. Skobeltsyn Institute of Nuclear Physics, M.V. Lomonosov Moscow State University, Moscow; Russia.\\
$^{110}$Fakult\"at f\"ur Physik, Ludwig-Maximilians-Universit\"at M\"unchen, M\"unchen; Germany.\\
$^{111}$Max-Planck-Institut f\"ur Physik (Werner-Heisenberg-Institut), M\"unchen; Germany.\\
$^{112}$Graduate School of Science and Kobayashi-Maskawa Institute, Nagoya University, Nagoya; Japan.\\
$^{113}$Department of Physics and Astronomy, University of New Mexico, Albuquerque NM; United States of America.\\
$^{114}$Institute for Mathematics, Astrophysics and Particle Physics, Radboud University/Nikhef, Nijmegen; Netherlands.\\
$^{115}$Nikhef National Institute for Subatomic Physics and University of Amsterdam, Amsterdam; Netherlands.\\
$^{116}$Department of Physics, Northern Illinois University, DeKalb IL; United States of America.\\
$^{117}$$^{(a)}$Budker Institute of Nuclear Physics and NSU, SB RAS, Novosibirsk;$^{(b)}$Novosibirsk State University Novosibirsk; Russia.\\
$^{118}$Institute for High Energy Physics of the National Research Centre Kurchatov Institute, Protvino; Russia.\\
$^{119}$Institute for Theoretical and Experimental Physics named by A.I. Alikhanov of National Research Centre "Kurchatov Institute", Moscow; Russia.\\
$^{120}$Department of Physics, New York University, New York NY; United States of America.\\
$^{121}$Ochanomizu University, Otsuka, Bunkyo-ku, Tokyo; Japan.\\
$^{122}$Ohio State University, Columbus OH; United States of America.\\
$^{123}$Homer L. Dodge Department of Physics and Astronomy, University of Oklahoma, Norman OK; United States of America.\\
$^{124}$Department of Physics, Oklahoma State University, Stillwater OK; United States of America.\\
$^{125}$Palack\'y University, Joint Laboratory of Optics, Olomouc; Czech Republic.\\
$^{126}$Institute for Fundamental Science, University of Oregon, Eugene, OR; United States of America.\\
$^{127}$Graduate School of Science, Osaka University, Osaka; Japan.\\
$^{128}$Department of Physics, University of Oslo, Oslo; Norway.\\
$^{129}$Department of Physics, Oxford University, Oxford; United Kingdom.\\
$^{130}$LPNHE, Sorbonne Universit\'e, Universit\'e de Paris, CNRS/IN2P3, Paris; France.\\
$^{131}$Department of Physics, University of Pennsylvania, Philadelphia PA; United States of America.\\
$^{132}$Konstantinov Nuclear Physics Institute of National Research Centre "Kurchatov Institute", PNPI, St. Petersburg; Russia.\\
$^{133}$Department of Physics and Astronomy, University of Pittsburgh, Pittsburgh PA; United States of America.\\
$^{134}$$^{(a)}$Laborat\'orio de Instrumenta\c{c}\~ao e F\'isica Experimental de Part\'iculas - LIP, Lisboa;$^{(b)}$Departamento de F\'isica, Faculdade de Ci\^{e}ncias, Universidade de Lisboa, Lisboa;$^{(c)}$Departamento de F\'isica, Universidade de Coimbra, Coimbra;$^{(d)}$Centro de F\'isica Nuclear da Universidade de Lisboa, Lisboa;$^{(e)}$Departamento de F\'isica, Universidade do Minho, Braga;$^{(f)}$Departamento de F\'isica Te\'orica y del Cosmos, Universidad de Granada, Granada (Spain);$^{(g)}$Dep F\'isica and CEFITEC of Faculdade de Ci\^{e}ncias e Tecnologia, Universidade Nova de Lisboa, Caparica;$^{(h)}$Instituto Superior T\'ecnico, Universidade de Lisboa, Lisboa; Portugal.\\
$^{135}$Institute of Physics of the Czech Academy of Sciences, Prague; Czech Republic.\\
$^{136}$Czech Technical University in Prague, Prague; Czech Republic.\\
$^{137}$Charles University, Faculty of Mathematics and Physics, Prague; Czech Republic.\\
$^{138}$Particle Physics Department, Rutherford Appleton Laboratory, Didcot; United Kingdom.\\
$^{139}$IRFU, CEA, Universit\'e Paris-Saclay, Gif-sur-Yvette; France.\\
$^{140}$Santa Cruz Institute for Particle Physics, University of California Santa Cruz, Santa Cruz CA; United States of America.\\
$^{141}$$^{(a)}$Departamento de F\'isica, Pontificia Universidad Cat\'olica de Chile, Santiago;$^{(b)}$Universidad Andres Bello, Department of Physics, Santiago;$^{(c)}$Instituto de Alta Investigaci\'on, Universidad de Tarapac\'a, Arica;$^{(d)}$Departamento de F\'isica, Universidad T\'ecnica Federico Santa Mar\'ia, Valpara\'iso; Chile.\\
$^{142}$Universidade Federal de S\~ao Jo\~ao del Rei (UFSJ), S\~ao Jo\~ao del Rei; Brazil.\\
$^{143}$Department of Physics, University of Washington, Seattle WA; United States of America.\\
$^{144}$Department of Physics and Astronomy, University of Sheffield, Sheffield; United Kingdom.\\
$^{145}$Department of Physics, Shinshu University, Nagano; Japan.\\
$^{146}$Department Physik, Universit\"{a}t Siegen, Siegen; Germany.\\
$^{147}$Department of Physics, Simon Fraser University, Burnaby BC; Canada.\\
$^{148}$SLAC National Accelerator Laboratory, Stanford CA; United States of America.\\
$^{149}$Department of Physics, Royal Institute of Technology, Stockholm; Sweden.\\
$^{150}$Departments of Physics and Astronomy, Stony Brook University, Stony Brook NY; United States of America.\\
$^{151}$Department of Physics and Astronomy, University of Sussex, Brighton; United Kingdom.\\
$^{152}$School of Physics, University of Sydney, Sydney; Australia.\\
$^{153}$Institute of Physics, Academia Sinica, Taipei; Taiwan.\\
$^{154}$$^{(a)}$E. Andronikashvili Institute of Physics, Iv. Javakhishvili Tbilisi State University, Tbilisi;$^{(b)}$High Energy Physics Institute, Tbilisi State University, Tbilisi; Georgia.\\
$^{155}$Department of Physics, Technion, Israel Institute of Technology, Haifa; Israel.\\
$^{156}$Raymond and Beverly Sackler School of Physics and Astronomy, Tel Aviv University, Tel Aviv; Israel.\\
$^{157}$Department of Physics, Aristotle University of Thessaloniki, Thessaloniki; Greece.\\
$^{158}$International Center for Elementary Particle Physics and Department of Physics, University of Tokyo, Tokyo; Japan.\\
$^{159}$Department of Physics, Tokyo Institute of Technology, Tokyo; Japan.\\
$^{160}$Tomsk State University, Tomsk; Russia.\\
$^{161}$Department of Physics, University of Toronto, Toronto ON; Canada.\\
$^{162}$$^{(a)}$TRIUMF, Vancouver BC;$^{(b)}$Department of Physics and Astronomy, York University, Toronto ON; Canada.\\
$^{163}$Division of Physics and Tomonaga Center for the History of the Universe, Faculty of Pure and Applied Sciences, University of Tsukuba, Tsukuba; Japan.\\
$^{164}$Department of Physics and Astronomy, Tufts University, Medford MA; United States of America.\\
$^{165}$Department of Physics and Astronomy, University of California Irvine, Irvine CA; United States of America.\\
$^{166}$Department of Physics and Astronomy, University of Uppsala, Uppsala; Sweden.\\
$^{167}$Department of Physics, University of Illinois, Urbana IL; United States of America.\\
$^{168}$Instituto de F\'isica Corpuscular (IFIC), Centro Mixto Universidad de Valencia - CSIC, Valencia; Spain.\\
$^{169}$Department of Physics, University of British Columbia, Vancouver BC; Canada.\\
$^{170}$Department of Physics and Astronomy, University of Victoria, Victoria BC; Canada.\\
$^{171}$Fakult\"at f\"ur Physik und Astronomie, Julius-Maximilians-Universit\"at W\"urzburg, W\"urzburg; Germany.\\
$^{172}$Department of Physics, University of Warwick, Coventry; United Kingdom.\\
$^{173}$Waseda University, Tokyo; Japan.\\
$^{174}$Department of Particle Physics and Astrophysics, Weizmann Institute of Science, Rehovot; Israel.\\
$^{175}$Department of Physics, University of Wisconsin, Madison WI; United States of America.\\
$^{176}$Fakult{\"a}t f{\"u}r Mathematik und Naturwissenschaften, Fachgruppe Physik, Bergische Universit\"{a}t Wuppertal, Wuppertal; Germany.\\
$^{177}$Department of Physics, Yale University, New Haven CT; United States of America.\\

$^{a}$ Also at Borough of Manhattan Community College, City University of New York, New York NY; United States of America.\\
$^{b}$ Also at Bruno Kessler Foundation, Trento; Italy.\\
$^{c}$ Also at Center for High Energy Physics, Peking University; China.\\
$^{d}$ Also at Centro Studi e Ricerche Enrico Fermi; Italy.\\
$^{e}$ Also at CERN, Geneva; Switzerland.\\
$^{f}$ Also at D\'epartement de Physique Nucl\'eaire et Corpusculaire, Universit\'e de Gen\`eve, Gen\`eve; Switzerland.\\
$^{g}$ Also at Departament de Fisica de la Universitat Autonoma de Barcelona, Barcelona; Spain.\\
$^{h}$ Also at Department of Financial and Management Engineering, University of the Aegean, Chios; Greece.\\
$^{i}$ Also at Department of Physics and Astronomy, Michigan State University, East Lansing MI; United States of America.\\
$^{j}$ Also at Department of Physics and Astronomy, University of Louisville, Louisville, KY; United States of America.\\
$^{k}$ Also at Department of Physics, Ben Gurion University of the Negev, Beer Sheva; Israel.\\
$^{l}$ Also at Department of Physics, California State University, East Bay; United States of America.\\
$^{m}$ Also at Department of Physics, California State University, Fresno; United States of America.\\
$^{n}$ Also at Department of Physics, California State University, Sacramento; United States of America.\\
$^{o}$ Also at Department of Physics, King's College London, London; United Kingdom.\\
$^{p}$ Also at Department of Physics, St. Petersburg State Polytechnical University, St. Petersburg; Russia.\\
$^{q}$ Also at Department of Physics, University of Fribourg, Fribourg; Switzerland.\\
$^{r}$ Also at Faculty of Physics, M.V. Lomonosov Moscow State University, Moscow; Russia.\\
$^{s}$ Also at Faculty of Physics, Sofia University, 'St. Kliment Ohridski', Sofia; Bulgaria.\\
$^{t}$ Also at Graduate School of Science, Osaka University, Osaka; Japan.\\
$^{u}$ Also at Hellenic Open University, Patras; Greece.\\
$^{v}$ Also at Institucio Catalana de Recerca i Estudis Avancats, ICREA, Barcelona; Spain.\\
$^{w}$ Also at Institut f\"{u}r Experimentalphysik, Universit\"{a}t Hamburg, Hamburg; Germany.\\
$^{x}$ Also at Institute for Particle and Nuclear Physics, Wigner Research Centre for Physics, Budapest; Hungary.\\
$^{y}$ Also at Institute of Particle Physics (IPP); Canada.\\
$^{z}$ Also at Institute of Physics, Azerbaijan Academy of Sciences, Baku; Azerbaijan.\\
$^{aa}$ Also at Institute of Theoretical Physics, Ilia State University, Tbilisi; Georgia.\\
$^{ab}$ Also at Instituto de Fisica Teorica, IFT-UAM/CSIC, Madrid; Spain.\\
$^{ac}$ Also at Istanbul University, Dept. of Physics, Istanbul; Turkey.\\
$^{ad}$ Also at Joint Institute for Nuclear Research, Dubna; Russia.\\
$^{ae}$ Also at Moscow Institute of Physics and Technology State University, Dolgoprudny; Russia.\\
$^{af}$ Also at National Research Nuclear University MEPhI, Moscow; Russia.\\
$^{ag}$ Also at Physics Department, An-Najah National University, Nablus; Palestine.\\
$^{ah}$ Also at Physikalisches Institut, Albert-Ludwigs-Universit\"{a}t Freiburg, Freiburg; Germany.\\
$^{ai}$ Also at The City College of New York, New York NY; United States of America.\\
$^{aj}$ Also at TRIUMF, Vancouver BC; Canada.\\
$^{ak}$ Also at Universita di Napoli Parthenope, Napoli; Italy.\\
$^{al}$ Also at University of Chinese Academy of Sciences (UCAS), Beijing; China.\\
$^{am}$ Also at Yeditepe University, Physics Department, Istanbul; Turkey.\\
$^{*}$ Deceased

\end{flushleft}

% Created with Glance <Atlas.Glance@cern.ch>

 
\end{document}
