\documentclass[cernpreprint, backref=false, texlive=2020, UKenglish, dvipsnames, block=none, texmf]{atlasdoc}
 
 
\usepackage{atlaspackage}
\PreprintIdNumber{CERN-EP-2021-157}
 
\interfootnotelinepenalty=10000
\usepackage{atlasbiblatex}
 
\usepackage{atlasphysics}
 
\addbibresource{ANA-HDBS-2021-03-PAPER.bib}
\addbibresource{ATLAS.bib}
\addbibresource{ATLAS-useful.bib}
\addbibresource{CMS.bib}
\addbibresource{ConfNotes.bib}
\addbibresource{PubNotes.bib}
 
\graphicspath{{logos/}{figures/}}
 
\usepackage{ANA-HDBS-2021-03-PAPER-defs}
\usepackage{atlassnippets}
\usepackage{atlasmisc}
\usepackage{atlasjetetmiss}
 
% The next lines are included from the .//ANA-HDBS-2021-03-PAPER-metadata.tex input file
 
\AtlasTitle{Search for Higgs boson decays into a pair of pseudoscalar particles in the $bb\mu\mu$ final state with the ATLAS detector in $pp$ collisions at $\sqrt{s}=13$~TeV}
 
\AtlasVersion{2.0}
\AtlasAbstract{
This paper presents a search for decays of the Higgs boson with a  mass of 125~GeV into a pair of new pseudoscalar particles, $H\rightarrow aa$, where one $a$-boson decays into a $b$-quark pair and the other into a muon pair.
The search uses 139~fb$^{-1}$ of proton--proton collision data at a center-of-mass energy of $\sqrt{s}=13$~TeV recorded between 2015 and 2018 by the ATLAS experiment at the LHC. A narrow dimuon resonance is searched for in the invariant mass spectrum between 16~GeV and 62~GeV.
The largest excess of events above the Standard Model backgrounds is observed at a dimuon invariant mass of 52~GeV and corresponds to a local (global)  significance of $3.3 \sigma$ ($1.7 \sigma$).
Upper limits at  95\% confidence level are placed on the
branching ratio of the Higgs boson to the $bb\mu\mu$ final state, $ \mathcal{B}(H\rightarrow aa\rightarrow bb\mu\mu)$, and are in the range $\text{(0.2--4.0)} \times10^{-4}$, depending on the signal mass hypothesis.
 
}
 
\AtlasJournal{PRD}
\author{The ATLAS Collaboration}
 
\AtlasRefCode{HDBS-2021-03}
 
 
 
 
 
 
 
 
 
 
 
 
\AtlasJournalRef{Phys. Rev. D 105 (2022) 012006}
\AtlasDOI{10.1103/PhysRevD.105.012006}
% End of text imported from the .//ANA-HDBS-2021-03-PAPER-metadata.tex input file
\hypersetup{pdftitle={ATLAS document},pdfauthor={The ATLAS Collaboration}}
 
\begin{document}
 
\maketitle
 
 
 
% The next lines are included from the .//Introduction.tex input file
\section{Introduction}
 
 
 
 
Light (pseudo) scalars that couple to the 125~\GeV\ Higgs boson~\cite{HIGG-2012-27,CMS-HIG-12-028} appear in many well-motivated extensions of the Standard Model (SM)~\cite{Dobrescu:2000yn,Ellwanger:2003jt,Dermisek:2005ar,Chang:2008cw,Morrissey:2008gm,Belanger:2013kya}.
These include models addressing the baryon asymmetry of the universe~\cite{Profumo:2007wc,Blinov:2015sna}, offering a solution to the naturalness problem~\cite{Craig:2015pha,Curtin:2015fna}, or providing insights into the nature of dark matter~\cite{Silveira:1985rk,Pospelov:2007mp,Draper:2010ew,Ipek:2014gua,Martin:2014sxa,Boehm:2014hva,GammaRayCascade}. Light bosons produced in Higgs boson decays could also be mediators to dark sectors that do not couple to the SM otherwise~\cite{Curtin:2013fra,TaniaTim1,TaniaTim2,TaniaTim3,ALPS}. Furthermore, pseudoscalar mediators appear in models, such as those described in Ref.~\cite{Liu_2019}, that were proposed to explain the anomalous muon magnetic moment~\cite{mug2_Fermi}.
A combination of ATLAS measurements of the Higgs boson production cross-sections and branching ratios
constrains the branching ratios  into invisible and undetected states to be  $\mathcal{B}(H\rightarrow\text{inv})<30\%$ and $\mathcal{B}(H\rightarrow\text{undetected})<21\%$, respectively,
whereas the overall branching fraction of the Higgs boson into beyond-the-SM (BSM) states is determined to be less than 47\% at 95\% confidence level (CL)~\cite{HIGG-2018-57}.
Combined measurements of Higgs boson couplings performed by the  CMS Collaboration set upper limits  of $\mathcal{B}(H\rightarrow\text{inv})<22\%$ and $\mathcal{B}(H\rightarrow\text{undetected})<38\%$ at 95\% CL~\cite{CMS-HIG-17-031}.
This motivates searches for light states in the Higgs boson decays that probe this potentially large $\mathcal{B}(H\rightarrow\text{BSM})$.
 
 
 
This paper presents a search for decays of the 125~\GeV\ Higgs boson into two pseudoscalars, denoted by $a$, in proton--proton ($pp$) collisions at the LHC~\cite{Evans:2008zzb}.
The search is performed in events where one $a$-boson decays into two $b$-quarks and the other into two muons, $H\rightarrow aa \rightarrow b\bar{b}\mu^+\mu^-$.\footnote{Denoted by $\haabbmm$ from now on for the rest of the paper.} The $a$-bosons are assumed to have  a decay width that is narrow compared to the detector resolution.
As pseudoscalar couplings are generally proportional to  mass,
which is for example the case in two-Higgs-doublet models~\cite{Curtin:2013fra,Branco:2011iw},
the $bb\mu\mu$ final state provides a good balance between a high branching ratio from the $a\to bb$  decay and a clean, high mass-resolution, dimuon resonance signature that is easy to trigger on from the $a\to\mu\mu$  decay.
In scenarios with enhanced lepton couplings,
the $a\to\mu\mu$ branching ratio can also be relatively large, resulting in $\mathcal{B}(H\rightarrow aa\rightarrow bb\mu\mu)/ \mathcal{B}(H\rightarrow aa )$ of  up to $ 0.16\%$~\cite{Curtin:2014pda}.
 
Light resonances in Higgs boson decays have been searched for by ATLAS and CMS in many different channels, i.e.\ in the  final states involving $4\mu$~\cite{CMS-HIG-18-003,EXOT-2016-22}, $2\mu2\tau$ or $4\tau$~\cite{CMS-HIG-18-024,CMS-HIG-18-006,CMS-HIG-17-029,CMS-HIG-16-015,HIGG-2014-02}, $2b2\tau$~\cite{CMS-HIG-17-024}, $4b$~\cite{HIGG-2017-05,HDBS-2018-47}, $4\gamma$~\cite{EXOT-2013-24} and $2\gamma+2$-jets~\cite{HIGG-2017-09}.
A search for a dimuon resonance produced in association with b-jets has been performed by CMS~\cite{CMS:2018lce} and a light resonance decaying to two muons has been searched for by LHCb~\cite{LHCb:2020ysn}.
CMS has performed a search for \haabbmm in 35.9~$\ifb$ of $pp$ collision data at a center-of-mass energy of $\sqrt{s}=13$~\TeV\ that sets upper limits on $\mathcal{B}(H \rightarrow aa \rightarrow bb\mu\mu)$ of $\text{(1--7)}\times 10^{-4}$ for $a$-boson masses ($m_a$) in the range $20 \leq m_a \leq 62.5$~\GeV~\cite{CMS-HIG-18-011}.
The ATLAS search  based on 36~$\ifb$ of Run~2 data~\cite{bbmm36ifb} sets upper limits on $\mathcal{B}(H \rightarrow aa \rightarrow bb\mu\mu)$ between $1.2\times10^{-4}$ and $8.4\times10^{-4}$ for $a$-boson masses  in the range $20 \leq m_a \leq 60$~\GeV.
In this paper, the full Run~2 dataset corresponding to an integrated luminosity of 139~$\ifb$ is used and the search is extended down to $m_a=16$~\GeV\ and up to $m_a=62$~\GeV.
Additionally,   boosted decision tree (BDT) techniques are used to improve the separation of  the signal from the SM backgrounds, increasing the analysis sensitivity, especially for higher $m_a$.
 
 
% End of text imported from the .//Introduction.tex input file
 
% The next lines are included from the .//Detector.tex input file
\section{ATLAS detector}
 
 
The ATLAS experiment~\cite{PERF-2007-01,PIX-2018-001} is a multipurpose particle detector
with a forward--backward symmetric cylindrical geometry and a nearly \(4\pi\) coverage in
solid angle.\footnote{ATLAS uses a right-handed coordinate system with its origin at the nominal interaction point (IP)
in the center of the detector and the \(z\)-axis along the beam pipe.
The \(x\)-axis points from the IP to the center of the LHC ring,
and the \(y\)-axis points upwards.
Cylindrical coordinates \((r,\phi)\) are used in the transverse plane,
\(\phi\) being the azimuthal angle around the \(z\)-axis.
The pseudorapidity is defined in terms of the polar angle \(\theta\) as \(\eta = -\ln \tan(\theta/2)\).
Angular distance is measured in units of \(\Delta R \equiv \sqrt{(\Delta\eta)^{2} + (\Delta\phi)^{2}}\).} It consists of an inner  detector (ID) surrounded by a thin superconducting solenoid
providing a \SI{2}{\tesla} axial magnetic field, electromagnetic (EM) and hadron calorimeters, and a muon spectrometer (MS).
The inner tracking detector covers the pseudorapidity range \(|\eta| < 2.5\).
It consists of silicon pixel, silicon microstrip, and transition radiation tracking detectors.
Lead/liquid-argon (LAr)  sampling calorimeters provide electromagnetic energy measurements
with high granularity.
A steel/scintillator-tile hadron calorimeter covers the central pseudorapidity range $|\eta| < 1.7$.
The endcap and forward regions are instrumented with LAr calorimeters
for both the EM and hadronic energy measurements up to \(|\eta| = 4.9\).
The MS surrounds the calorimeters and is based on
three large superconducting air-core toroidal magnets with eight coils each.
The field integral of the toroids ranges between \num{2.0}~Tm and \SI{6.0}Tm
across most of the detector.
The MS includes a system of precision tracking chambers and fast detectors for triggering.
A two-level trigger system is used to select events.
The first-level trigger is implemented in hardware and uses a subset of the detector information
to accept events at a rate below \SI{100}{\kHz}~\cite{TRIG-2016-01}.
This is followed by a software-based trigger that
reduces the accepted event rate to \SI{1}{\kHz} on average.
An extensive software suite~\cite{ATL-SOFT-PUB-2020-001} is used in the reconstruction and analysis of real and simulated data, in detector operations, and in the trigger and data acquisition systems of the experiment.
 
 
 
% End of text imported from the .//Detector.tex input file
% The next lines are included from the .//Samples.tex input file
\section{Dataset and simulated events}
 
The data used in this analysis were collected in Run~2 of the LHC during the 2015--2018 data-taking period with $pp$ collisions at a center-of-mass energy of $\sqrt{s} = 13$~\TeV. The dataset corresponds to an integrated luminosity of 139~$\text{fb}^{-1}$.
The lowest-threshold unprescaled single-muon and dimuon triggers are used to select the events~\cite{TRIG-2018-01}. Single-muon triggers require the transverse momentum (\pT) of the muon  to be above 20~\GeV\ or 26~\GeV, depending on the data-taking period, while the dimuon trigger requires both muons to have a \pT above 14~\GeV.
 
Simulated events are used in the estimation of the SM backgrounds. \textsc{Sherpa}~2.2.1~\cite{Gleisberg:2008ta,Bothmann:2019yzt} was used as the baseline generator for the Drell--Yan (DY)\,+\,jets, $W(\rightarrow \ell \nu)$+jets, diboson and triboson backgrounds. It is a multiparton matrix element and parton shower (PS) generator including hadronization~\cite{Schumann:2007mg,Hoeche:2011fd, Hoeche:2012yf, Catani:2001cc,Hoeche:2009rj}, with the \textsc{NNPDF3.0} parton distribution function (PDF) set at next-to-next-to-leading-order (NNLO) accuracy~\cite{Ball:2014uwa}.
The DY+jets and multiboson samples were generated with a minimum dilepton mass of 10~\GeV\ and 4~\GeV, respectively.
The \ttbar and single-top-quark samples were generated with \textsc{Powheg-Box~v2}~\cite{Frixione:2007vw,Nason:2004rx,Alioli:2010xd,Frixione:2007nw,Re:2010bp} using the \textsc{NNPDF3.0nlo} PDF in matrix element interfaced to \textsc{Pythia~8.230}~\cite{Sjostrand:2014zea} for the PS.
For the underlying-event description a set of tuned parameters called the \textsc{A14} tune~\cite{A14_article} was used, along with the \textsc{NNPDF2.3lo} PDF~\cite{Ball:2012cx}.
The \ttbar + vector-boson processes ($\ttbar{+}V$) were generated with \textsc{MadGraph5}\_aMC@NLO~2.3.3~\cite{Alwall:2014hca} interfaced to \textsc{Pythia}~8.210 for the PS. The underlying-event tune was the same as for the \ttbar sample.
\textsc{EvtGen}~\cite{Lange:2001uf} was used for the properties of the bottom and charm hadron decays in all simulated samples, except those simulated with \textsc{Sherpa}.
 
 
Higgs boson production through gluon--gluon fusion (ggF) was generated using the NNLOPS program~\cite{Hamilton:2012np,Hamilton:2015nsa} with \textsc{Powheg-Box~v2}~\cite{Frixione:2007vw,Alioli:2010xd, Campbell:2012am, Hamilton:2012rf}. The vector-boson fusion (VBF) processes were generated with \textsc{Powheg-Box~v2} at NLO accuracy~\cite{Nason:2009ai}. The Higgs boson mass was set to 125~\GeV. For both the ggF and VBF production processes, \textsc{Powheg-Box} was interfaced with \textsc{Pythia}~8.212 using  the AZNLO tune~\cite{STDM-2012-23}
for the simulation of the $H\rightarrow aa\rightarrow bb\mu\mu$ decays, where the $a$-boson is a pseudoscalar, as well
as for parton showering, hadronization and the underlying event.
The ggF Higgs boson production rate is normalized to the total cross-section predicted at
next-to-next-to-next-to-leading-order accuracy in QCD with NLO electroweak corrections applied~\cite{deFlorian:2016spz,Anastasiou:2015ema,Anastasiou:2016cez,Actis:2008ug,Anastasiou:2008tj} and amounts to 48.58~pb.
The VBF production rate is normalized to an approximate NNLO cross-section with the NLO electroweak corrections applied~\cite{Heinemeyer:2013tqa,Ciccolini2007,Ciccolini2008,BolzoniVBF}, which amounts to 3.8~pb.
The contribution from the associated production of a Higgs boson and a vector boson (VH) is calculated to be 3.5\% of the total  ggF+VBF cross-section and is accounted for by scaling the simulated ggF and VBF samples.
The contribution from Higgs boson production in association with a pair of top quarks is found to be negligible (below the percent level) and is neglected in the analysis.
Thirteen mass points were simulated for the ggF and VBF production modes, with the $a$-boson mass in the range $m_a = \text{16--62}$~\GeV.\footnote{More specifically, the simulated mass points are at $m_a = 16$, 18, 20, 25, 30, 35, 40, 45, 50, 52, 55, 60 and 62~\GeV.}
Below $m_a = 16$~\GeV\ the $b$-quarks coming from the decays of the $a$-boson tend to be so collimated due to its boost that they cannot be reconstructed as two separate $b$-jets (with a radius parameter of $R=0.4$). Another effect is that in the highly asymmetric decays of low-mass $a$-bosons, the subleading $b$-jet falls below the jet reconstruction threshold of 20~\GeV~\cite{PERF-2015-09}.
As a result, the signal acceptance falls below 0.2\% and the analysis loses sensitivity.
 
 
The effects of additional interactions in the same and neighboring beam-bunch crossings (pileup) were modeled for all simulated events by overlaying additional $pp$ collisions generated with \textsc{Pythia~8.186} using the \textsc{NNPDF2.3lo} PDF set and the A3 tune~\cite{Sjostrand:2007gs}. Simulated event samples are weighted to reproduce the distribution of the number of pileup interactions observed in the data. All the generated background and signal samples are processed through the  ATLAS detector simulation~\cite{SOFT-2010-01} based on $\GEANT4$~\cite{Agostinelli:2002hh} and reconstructed using the same software as for the data.
 
 
 
% End of text imported from the .//Samples.tex input file
% The next lines are included from the .//Selection.tex input file
\section{Event reconstruction and selection}
 
 
Muons are reconstructed by combining track information from the MS with tracks found in the ID \cite{MUON-2018-0333}. They also have to satisfy $\pt>5$~\GeV\ and $|\eta|<2.7$ (for $|\eta|>2.5$, only tracking information from the MS is used), and pass the $LowPt$ working point identification requirement defined in Ref.~\cite{MUON-2018-0333}.
Muon tracks must have a longitudinal impact parameter $z_0$ satisfying $|z_{0}\sin\theta|<0.5$~mm and a transverse impact parameter significance $|d_{0}|$/$\sigma_{d_{0}}<3$ relative to the primary interaction vertex, chosen as the reconstructed vertex with the highest sum of the $\pT^2$ of its associated tracks. Furthermore, muons are required to be isolated from the surrounding detector activity by requiring that the scalar sum of the \pt of additional inner detector tracks  and the sum of the  transverse momentum $E_\mathrm{T}$ of calorimeter energy deposits within a cone of size $\Delta R = 0.2$ around a muon be less than 15\% and 30\% of the muon \pt, respectively.
 
 
Jets are reconstructed using the anti-$k_t$ algorithm~\cite{Cacciari:2008gp} implemented in the \textsc{FastJet}\xspace package~\cite{Fastjet} with a radius parameter of $R = 0.4$. The inputs to the jet clustering are built by combining the information from both the calorimeters and the ID using a particle-flow algorithm~\cite{PERF-2015-09,JETM-2018-05}. Jets with $\pT < 60$~\GeV\ originating from pileup are suppressed with the jet-vertex-tagger (JVT)~\cite{ATLAS-CONF-2014-018}, a multivariate algorithm combining track-based variables. Selected jets are required to have $\pt > 20$~\GeV\ and $|\eta| <2.5$.
An algorithm (MV2c10) relying on multivariate techniques, taking as input the properties of displaced tracks and vertices reconstructed within a jet, is employed to identify (tag) jets containing $b$-hadrons~\cite{FTAG-2018-01}.
The MV2c10 tagger is used at 77\% $b$-jet identification efficiency, with an approximate misidentification probability of 25\% for jets arising from charm quarks, 6.3\% for hadronically decaying $\tau$-leptons, and 0.8\% for light-flavor jets as measured in simulated \ttbar events.
 
The missing transverse momentum (\met) is calculated as the magnitude of the negative vector sum of the transverse momenta of all the reconstructed and calibrated objects in the event, including a soft term that accounts for charged particles that are associated with the primary vertex, but not with any reconstructed object~\cite{PERF-2016-07,ATLAS-CONF-2018-023}.
 
 
The events selected for the analysis are required to have two muons of opposite charge, either with the leading and subleading muons satisfying $\pt^{\text{leading}}>27$~\GeV\ and $\pt^{\text{subleading}}>5$~\GeV, and the event being triggered by a single-muon trigger, or with both muons having $\pt >15$~\GeV, and the event being triggered by a dimuon trigger. The dimuon invariant mass, $\Mmm$, is required to be between 15~\GeV\ and 65~\GeV. Furthermore, the events must contain exactly two $b$-tagged jets with \pT above 20~\GeV.
 
 
A kinematic likelihood (KL) \cite{Erdmann:2013rxa} fit exploiting the equal invariant masses of the $bb$ and $\mu\mu$ systems in $\haa$ decays  is performed to improve the four-body invariant mass ($\Mbbmm$) resolution and  reduce the SM backgrounds. The same fit approach as considered in the previous ATLAS publication~\cite{bbmm36ifb} is used.
The dimuon invariant mass, $\Mmm$, is used to constrain the di-$b$-jet mass, as the former has a resolution approximately ten times better than the latter.
The $\Mmm$ resolution ranges between 0.4~\GeV\ at $m_a$ = 16~\GeV\ and 1.3~\GeV\ at $m_a = 62$~\GeV.
The fit maximizes the likelihood by shifting the $b$-jet energies within the resolution in order to satisfy the constraint $\Mmm \simeq \Mbb$.
The output of the fit is the logarithm of the maximum likelihood value, \LogL, which quantifies how well the event matches the $\Mmm = \Mbb$ hypothesis, characteristic of signal events.
The four-body invariant mass, recomputed after the KL fit, is denoted by $\KLMbbmm$ and is used for further event categorization.
 
Signal-like events are chosen by requiring that $110 <\KLMbbmm<140$~\GeV, and that $\LogL > -8$, which ensures that $\Mbb$ is compatible with $\Mmm$.
Finally, \met is required to be less than 60~\GeV\ to reduce the background from \ttbar events, which is one of the two major backgrounds and can contain large \met from neutrinos in top-quark decays. This  selection defines the ``inclusive'' signal region (SRincl) and is summarized in Table~\ref{selections}, along with the selection requirements for other analysis regions described later in the text.
 
A boosted decision tree (BDT) classifier implemented using the TMVA framework \cite{Speckmayer:2010zz} is employed to further reduce the SM backgrounds.
Its training is done in partially overlapping 8-\GeV-wide \Mmm windows centered at the $m_a$ values of each of the 12 generated signals,\footnote{One BDT was trained for each generated signal MC sample, except for $m_a = 52$~\GeV, as this sample was produced only at a later analysis stage.} in order to fully exploit their kinematic differences.
The background sample consists of \ttbar and DY+jets events, the two dominant backgrounds, combined in the proportions extracted from the background validation fit described in Section~\ref{sec:result}. The signal samples used for the training include ggF and VBF Higgs boson production samples combined according to their cross-sections.
The seven kinematic variables included in the training are:
\begin{itemize}
\item{$\Mbb$,}
\item{\LogL,}
\item{$\Delta R_{b_1b_2}$ (the angular distance between the two $b$-jets),}
\item{diff$\Delta R_bR_{\mu}=\Delta R_{b_1b_2}-\Delta R_{\mu_1\mu_2}$ (the difference between the angular separations between the two $b$-jets and the two muons),}
\item{$\Delta R_{bb\mu\mu}$ (the angular distance between the $bb$ and $\mu\mu$ systems),}
\item{$\overline{\Delta R_{b,\mu}} = \left[\Delta R_{b_1\mu_1}+\Delta R_{b_1\mu_2}+\Delta R_{b_2\mu_1}+\Delta R_{b_2\mu_2}\right]/4$ (the average angular distance  of all four combinations of a $b$-jet and a muon),}
\item{$\overline{m_{b\mu}}=\left[m_{b_1\mu_1}+m_{b_1\mu_2}+m_{b_2\mu_1}+m_{b_2\mu_2}\right]/4$ (the average mass of all four combinations of a $b$-jet and a muon).}
\end{itemize}
The distributions of these variables for the background and three representative signal masses are shown in Figure~\ref{BDTinputs}.
 
The $\Mbb$ variable helps separate the low-mass signal from the backgrounds, as $\Mbb$ peaks around 60~\GeV\ for the \ttbar and DY processes. The \LogL~peaks at higher values as the signal mass becomes smaller.
 
Due to a higher boost of a lighter $a$-boson, its decay products are collimated, resulting in $\Delta R_{b_1b_2}$ and  $\Delta R_{\mu_1\mu_2}$ being much smaller than for a signal from a heavier $a$-boson or for background processes.
As a consequence, diff$\Delta R_bR_{\mu}$ shows a narrow distribution centered around zero, while the background and a higher-mass signal exhibit a much broader diff$\Delta R_bR_{\mu}$ distribution.
 
 
The $\Delta R_{bb\mu\mu}$ variable helps enhance the sensitivity to higher signal masses. Heavier $a$-bosons are produced approximately at rest, resulting in the $\Delta R_{bb\mu\mu}$ distribution being relatively flat with a small peak at low values. As the signal mass decreases, the $\Delta R_{bb\mu\mu}$ distribution  transitions into a ``back-to-back'' topology, characteristic of both a low-mass signal and the background events.
 
 
Finally, the $\overline{\Delta R_{b,\mu}}$ and $\overline{m_{b,\mu}}$ variables provide another measure of how close the two $a$-bosons are in $\Delta R$.
In the ``back-to-back'' topology for lower signal masses, the muons are, on average, further away from the $b$-jets, while for heavier $a$-bosons produced approximately at rest, the average distance between the muons and the $b$-jets is smaller.
Consequently, $\overline{\Delta R_{b,\mu}}$ and $\overline{m_{b,\mu}}$ peak at high (low) values for low (high) signal masses, while the backgrounds peak somewhere between the two extreme signal topologies.
 
The output score of the BDT trained for a signal with mass $m_a$ is denoted by BDT$m_a$. The BDT$m_a$ distributions for $m_a$ = 20~\GeV, 40~\GeV\, and 60~\GeV\ are shown in Figure~\ref{BDToutputs}.
 
The final signal region (SR) bin for each signal mass is defined by imposing two requirements in addition to the SRincl selection: $m_a-X<\Mmm<m_a+X$ and  BDT$m_a>0.2$, where $X=1$~\GeV\ ($X=1.5$~\GeV) for $m_a\leq 45$~\GeV\ ($m_a>45$~\GeV). The widths of the SR bins and the BDT$m_a$ cut value are optimized to maximize the significance of signal over background events.
For masses at which no signal sample was generated, and, consequently, no BDT was trained, the BDT trained for the $m_a$ closest to the one being tested is used.
For example, when testing the  $m_a=32$~\GeV\ hypothesis, the requirement BDT$30>0.2$ is applied to select the events for the SR bin.
Signal yields for mass points where no signal sample was generated ($m_a=32$~\GeV\ in this example) are obtained by selecting events with BDT scores above 0.2 for the same BDT$m_a$ (BDT30 in this case) in all simulated mass points and interpolating using third-order splines.
To assess the uncertainty, the yields of the neighboring simulated mass points ($m_a=30$~\GeV\ and $m_a=35$~\GeV\ in this case) are interpolated using a linear function. The difference between the yields obtained using the splines and a linear function for the interpolation is assigned as a systematic uncertainty on the interpolated signal yield.
Using a BDT at a mass for which the training was not performed results in a negligible loss of significance relative to a BDT that was optimized for that mass point.
 
 
\begin{table}[h!]
\caption{Summary of the selection requirements for the  control (TCR and DYCR), validation (VR1 and VR2) and  inclusive signal (SRincl) regions in the analysis, as well as the final SR bins. The control and validation regions are defined in Section~\ref{sec:backgrounds}.}
\label{selections}
\begin{center}
\begin{tabular}{l|c|c|c|c|c}
\toprule
& TCR & DYCR & SRincl & VR1 & VR2 \\
\midrule
\midrule
\Mmm[\GeV] & \multicolumn{5}{c}{[15, 65]} \\
\midrule
\KLMbbmm[\GeV] & [110, 140] & [80, 110] or [140, 170]  & [110, 140]  & [170, 300] & [110, 140] \\
\midrule
\met[\GeV] &  $> 60$ & \multicolumn{4}{c}{$< 60$} \\
\midrule
\LogL & \multicolumn{4}{c|}{$> -8$} & [$-11$, $-8$] \\
\midrule
\midrule
SR bins & \multicolumn{5}{c}{  SRincl  \& BDT$m_a>0.2$ }  \\
& \multicolumn{5}{c}{ 2-\GeV-wide (3-\GeV-wide) \Mmm bins for $m_a\leq45$~\GeV\ ($m_a>45$~\GeV) }  \\
\bottomrule
\end{tabular}
\end{center}
\end{table}
 
 
The signal acceptance $\times$ efficiency varies between 0.3\% and 2.5\% for ggF Higgs boson production and between 0.2\% and 3.0\% for VBF production, where the lowest acceptance $\times$ efficiency is obtained for the lowest $m_a$, and grows as $m_a$ increases. The largest loss of acceptance occurs when requiring that there are two $b$-jets in the event, as one of the signal jets tends to fall below the reconstruction threshold of 20~\GeV. The fraction of signal events passing the two-$b$-jet requirement is less than 20\% for all mass points.
 
 
 
 
 
\begin{figure}
\centering
\includegraphics[width=0.45\hsize]{fig_01a.pdf}
\includegraphics[width=0.45\hsize]{fig_01b.pdf}
\includegraphics[width=0.45\hsize]{fig_01c.pdf}
\includegraphics[width=0.45\hsize]{fig_01d.pdf}
\includegraphics[width=0.45\hsize]{fig_01e.pdf}
\includegraphics[width=0.45\hsize]{fig_01f.pdf}
\includegraphics[width=0.45\hsize]{fig_01g.pdf}
\caption{Kinematic variables used as inputs to the BDT training. From top left to bottom right: $\Mbb$, \LogL, $\Delta R_{b_1b_2}$, diff$\Delta R_bR_{\mu}$, $\Delta R_{bb\mu\mu}$, $\overline{\Delta R_{b,\mu}}$, $\overline{m_{b,\mu}}$. The variables are plotted in SRincl.
All the distributions are normalized to unit area. The background histogram is the sum of the \ttbar and DY event templates, combined in the proportions extracted from the background validation fit described in Section~\ref{sec:result}.}
\label{BDTinputs}
\end{figure}
 
\begin{figure}
\centering
\includegraphics[width=0.45\hsize]{fig_02a.pdf}
\includegraphics[width=0.45\hsize]{fig_02b.pdf}
\includegraphics[width=0.45\hsize]{fig_02c.pdf}
\caption{Three BDT$m_a$ distributions, BDT20, BDT40, and BDT60, plotted in the \Mmm windows of SRincl, as indicated in the figures. The distributions are normalized to unit area. The background histogram is the sum of the \ttbar and DY event templates, combined in the proportions extracted from the background validation fit described in Section~\ref{sec:result}.}
\label{BDToutputs}
\end{figure}
 
 
 
 
 
 
 
 
 
 
 
 
 
% End of text imported from the .//Selection.tex input file
% The next lines are included from the .//Backgrounds.tex input file
\section{Background estimation}
\label{sec:backgrounds}
The dominant backgrounds in the analysis arise from the DY dimuon process in association with $b$-quarks and  pair production of top quarks ($\ttbar$) where each $W$ boson decays into a muon and a neutrino. These two backgrounds account for more than $96\%$ of background events in all analysis regions.
 
Two control regions are designed to constrain the  \ttbar and DY backgrounds. They are chosen so that they have negligible signal contamination, are kinematically as close as possible to SRincl, and maximize the contribution of one of the respective background processes.
A top-quark control region (TCR) is defined by inverting the $\met$ selection criterion in SRincl to $\met>60$~\GeV. This results in an event sample approximately 93\% pure in \ttbar events. The DY control region (DYCR) is defined in the 30 \GeV-wide \KLMbbmm sidebands of SRincl, i.e.\ by requiring $80 <\KLMbbmm<110$~\GeV\ or $140 <\KLMbbmm<170$~\GeV. Approximately 50\% of the events in DYCR originate from the DY process, whereas the rest mostly come from \ttbar production.
Two validation regions (VR1 and VR2) are used to validate the normalizations of the backgrounds. VR1 is defined in the $170 <\KLMbbmm<300$~\GeV\ range, while VR2 is obtained by inverting the \LogL\ selection criterion of SRincl to $-11< \LogL <-8$. All the analysis regions are summarized in Table~\ref{selections} and illustrated in Figure~\ref{regionsPic}.
 
The shapes of the $\ttbar$ kinematic variable  distributions are obtained from simulation, while the overall normalization is extracted from the fits described in Section~\ref{sec:result}.
The distributions for the DY background are taken from data templates because the limited sizes of the simulated event samples do not allow a reliable estimate. The template regions are defined in the same way as the analysis regions in Table~\ref{selections}, except that the two-$b$-tag requirement is replaced by a zero-$b$-tag requirement. The template regions are  ${>}95\%$ pure in DY events. Contributions from other processes, namely \ttbar, $W$+jets, diboson and single-top, are  subtracted using simulation.
Following the subtraction, the DY templates are corrected to account for kinematic differences between event samples dominated by jets originating from light quarks or gluons (template regions) and event samples dominated by $b$-jets (analysis regions). The correction is applied as a per-event weight, where the reweighting is derived from a comparison between two-$b$-tag and zero-$b$-tag kinematic distributions in simulated DY events.
Two sets of event weights are derived and applied sequentially.
First, the jet multiplicity of the zero-$b$-tag MC sample is reweighted to the one in the two-$b$-tag sample. It is the distribution with the largest difference between the zero- and two-$b$-tag samples and was hence corrected first.
Secondly, a BDT-based reweighting is employed to further correct the zero-$b$-tag template kinematics. A BDT is trained  on the zero-$b$-tag versus the two-$b$-tag simulated DY samples. The BDT input consists of kinematic properties and angular distributions of the $b$-jets, muons and the two corresponding $a$-boson candidates, as well as  $\met$ and  $\KLMbbmm$. The ratio of the BDT score distributions obtained for the two-b-tag and zero-b-tag simulated events is then applied as a weight to every event from the zero-b-tag DY template, as a function of its BDT score.
Following the BDT-based reweighting, the \KLMbbmm and $\met$ distributions are corrected by up to 20\%. The DY templates are normalized to data in the fits described in Section~\ref{sec:result}.
 
Minor backgrounds include diboson and single-top-quark production, production of a \ttbar\ pair in association with a vector boson, and $W$ boson production in association with $b$-jets.
The estimation of these minor backgrounds relies purely on simulation normalized to the best available theoretical prediction. The events where a jet is misidentified as a muon are taken into account as follows: non-prompt/misidentified muons in $W$+jets and \ttbar events are included in the analysis on the basis of simulation, any contribution of non-prompt/misidentified muons in the DY+jets component is accounted for by the data template, and the  potential contribution from multijet events is found to be negligible.
 
 
 
\begin{figure}
\centering
\includegraphics[width=0.5\hsize]{fig_03.pdf}
\caption{Illustration of the signal, control, and validation regions used in the analysis. VR2 (not shown) is defined by the same selection as SRincl, except that the requirement on \LogL\ is inverted to $-11 < \LogL < -8$.}
\label{regionsPic}
\end{figure}
% End of text imported from the .//Backgrounds.tex input file
% The next lines are included from the .//Systematics.tex input file
\section{Systematic uncertainties}
Systematic uncertainties in the analysis are divided into three
categories: experimental uncertainties affecting the simulated background and signal processes, uncertainties in the modeling of the DY template, and theoretical uncertainties of the simulated background and signal samples.
Table~\ref{tab:syst} shows a summary of the dominant systematic uncertainties in the total background and signal yields in the signal region bins, as resulting from the fits described in Section~\ref{sec:result} and hereafter denoted by "post-fit".
 
 
\begin{table}[!htbp]
\caption{Summary of the dominant post-fit systematic uncertainties in the background and signal yields. The uncertainties are expressed as a percentage of the total background and signal yields per $m_{\mu\mu}$ bin of the signal region.
Only uncertainties exceeding 2\% in at least one SR bin are shown.
\label{tab:syst}}
\begin{center}
\begin{tabular}{l|l|c|c}
\toprule
Category & Source &Total background [\%] & Signal [\%]  \\ \hline
 
\midrule
DY & BDT$m_a$ selection & ~~~~~7--14 &   -- \\
& normalization & ~~~~~5--10 &   -- \\
& \Mmm shape & ~~~1--8 &   -- \\
& kinematics & 0.3--6 &   --  \\
& background subtraction & 0.6--3  &   --  \\
\midrule
$\ttbar$ & hadronization/PS &   0.3--4 & --  \\
& hard-scatter generation  & 0.2--3 & --  \\
& normalization & 0.2--3 & --  \\
\midrule
Overall MC & Sample statistics &  ~~~~~8--40 &  1--2 \\
\midrule
Jets & $b$-tagging &  ~0.03--0.7 &  ~~9--10  \\
& Jet-energy resolution&  ~~~1--3 &  6--7 \\
& Jet-energy scale&  ~~~1--3 &  4--5  \\
\midrule
Signal & FSR & -- &  5 \\
& PS & -- & 4  \\
& VH contribution& -- & ~~~3.5  \\
& MPI & -- & 3  \\
& QCD scale & -- & 3  \\
& ISR & -- &  3 \\
& ggF cross-section & &   \\
& \hspace{1.2cm}- missing higher-order QCD  & -- & 5  \\
& \hspace{1.2cm}- PDF \&  $\alphas$  & -- & 3  \\
\bottomrule
\end{tabular}
\end{center}
\end{table}
 
 
 
Among the experimental uncertainties, the leading effects come from those associated with the
calibration and resolution of jet energies~\cite{PERF-2016-04}, and with the measurement of the $b$-tagging efficiency~\cite{FTAG-2018-01}. The impact of these uncertainties on the total background (signal) yields in the SR bins is as large as 3\% (10\%).
The uncertainty in the combined 2015--2018 integrated luminosity is 1.7\% \cite{ATLAS-CONF-2019-021}, obtained using the LUCID-2 detector \cite{LUCID2} for the primary luminosity measurement.
Other uncertainties, such as those arising from the muon identification efficiency, momentum scale and resolution~\cite{PERF-2015-10,MUON-2018-0333}, and pileup are found to have a negligible impact on the final yields.
 
The uncertainty arising from limited MC sample sizes ranges from 8\% to as large as 40\% in the low $m_a$ mass bins due to there being few \ttbar events in this region.
 
 
Five sources of uncertainty in the data-driven DY template are considered. The uncertainty in subtracting non-DY events from the non-reweighted template in the analysis regions is assessed by comparing the nominal template, for which the simulated non-DY backgrounds had been subtracted before reweighting, with an alternative template for which no subtraction had been performed.
The uncertainties in the template kinematics modeling
are derived by comparing the DY template with simulation in two key variables: \met and $\KLMbbmm$.
The ratios of the template to the simulated DY events are fit with linear functions and used in assigning uncertainties to the shapes of the \met and $\KLMbbmm$ distributions.
Similarly, the uncertainty in the  \Mmm template shape is assessed by comparing the template with the smoothed simulated sample and applying the observed difference  as a systematic uncertainty. 
The uncertainty in the normalization of the DY template is obtained from the fits to data.
Finally, the uncertainty in the efficiency of the BDT$m_a$ selection criteria is evaluated by taking the difference in the BDT$m_a$ cut efficiency, $N_{\text{DY events SR}}^{\text{BDTm}\mathrm{_a}\text{>0.2}}/N_{\text{DY events SR}}^{\text{no BDTm}\mathrm{_a} \text{ cut}}$, between the template and the simulation.  All one-sided DY template uncertainties are symmetrized around the nominal value.
 
 
To assess the uncertainties in the generation of the hard-scatter \ttbar process, the \textsc{Powheg} sample is compared with a sample generated using \textsc{MadGraph5}\_aMC@NLO~2.3.3. The hadronization and fragmentation uncertainties in the PS are evaluated by comparing the nominal sample showered by \textsc{Pythia~8.230} with an alternative sample generated by \textsc{Powheg} using the same PDF in matrix element as for the nominal sample, but showered with \textsc{Herwig}~7.0.4 \cite{Bahr:2008pv, Bellm:2015jjp}. The initial- and final-state radiation (ISR and FSR) uncertainties of the \ttbar sample are assessed by varying the internal \textsc{Pythia~8.230} showering parameters.
Finally, the uncertainties due to the PDF choice are evaluated using the internal variations of the nominal \textsc{PDF4LHC15\_nlo\_30} set \cite{Butterworth:2015oua}.
 
 
Uncertainties in the calculation of the ggF and VBF Higgs boson production cross-sections are assessed by following the recommendations of the LHC Higgs Working Group given in Refs.~\cite{deFlorian:2016spz, Heinemeyer:2013tqa}.
As no VH signal sample was generated, a conservative  100\%  uncertainty is assigned to the estimated VH yield.
To evaluate the  uncertainties due to the PDF choice, the yields obtained with the baseline \textsc{NNPDF30\_nlo\_as\_0118} set are compared with the yields obtained using the internal variations of \textsc{NNPDF30\_nlo\_as\_0118} and with the yields obtained with the nominal \textsc{MMHT2014nlo68clas118} \cite{Harland-Lang:2014zoa} and \textsc{CT14nlo}~\cite{Dulat:2015mca} sets. The largest difference is taken as the overall PDF uncertainty for all signal mass points. Furthermore, the effects of uncertainties in the ISR, FSR, multiparton interactions (MPI) in \textsc{Pythia}, parton showering, and renormalization and factorization scales are also  assessed. Uncertainties from these sources have an impact of 1--6\% on the signal yields, with the largest contributions arising from the ggF production cross-section and FSR uncertainties.
 
 
 
 
 
 
% End of text imported from the .//Systematics.tex input file
% The next lines are included from the .//Results.tex input file
\section{Analysis and results}
\label{sec:result}
The final background and signal estimates are obtained in a set of binned likelihood fits \cite{Cowan:2010js}  using the HistFitter~\cite{Baak:2014wma} package.
The likelihood is a product of Poisson probability functions, describing the observed and predicted numbers of events in each region, and Gaussian distributions that constrain the nuisance parameters associated with the systematic uncertainties.
In the background validation fit, the data in TCR and DYCR are used to extract the normalization of the \ttbar and DY backgrounds, respectively.
As the \ttbar sample in TCR is modeled very well, it is implemented as only one bin in the fit, whereas DYCR is divided into five equal-width bins in \Mmm to provide greater sensitivity to the DY template shape.
The purpose of this fit is to validate the modeling of the background in the control and validation regions and in SRincl. The fitted \ttbar normalization factor is $\mu_{\ttbar}=1.07^{+0.06}_{-0.07}$, while the value of $\mu_\text{DY}$ has no physical meaning because it is scaled from a template region and is thus not quoted. Figures~\ref{SRvalidation1} and \ref{SRvalidation2} show post-fit distributions of \KLMbbmm, \met, \LogL, and \Mmm spanning various analysis regions, while Figure~\ref{SRvalidation3} shows BDT20 and BDT50 in SRincl. Good agreement between the estimated backgrounds and the data is observed in the kinematic distributions.
In SRincl, 1185 events are observed, which is compatible with the total estimated background of $1155.3 \pm 13.6$.
The yields in several representative SR bins, i.e.\ \Mmm windows after applying the BDT selection, as obtained from the background validation fit above, are shown in Table~\ref{SRyields}. When comparing the systematic uncertainty with the statistical uncertainty, it can be seen that the analysis is clearly statistically limited.
Figure~\ref{SRpulls} shows the data and the estimated backgounds in all final SR bins. Due to the limited statistics of the background samples, the estimates are not perfectly smooth; however, the bin-to-bin fluctuations are much smaller than the statistical uncertainty of the data. Larger jumps, which occur at $m_a = 23$, 28, 33, 38~\GeV\ etc., appear when the BDT discriminant used for the selection changes from the one trained in the lower mass range to the one trained in the higher mass range.
 
 
\begin{table}[htb]
\caption{Total and individual background yields in six
representative $m_{\mu\mu}$ bins of the signal region after the BDT selection is applied. The yields
are the post-fit values as determined by the background validation fit.
The uncertainties shown include all systematic and statistical uncertainties. As the diboson, single top quark, $\ttbar V$, and $W$+jets contributions are very small, they are summed in the table under "Other".}
\begin{center}
\setlength{\tabcolsep}{0.5pc}{\small
\begin{tabular*}{\textwidth}{@{\extracolsep{\fill}}l|c|c|c|c|c|c}
\toprule
$m_{\mu\mu}$ bin [\GeV] &  [15--17]   & [24--26]   & [34--36]   & [44--46] &  [50.5--53.5]    & [60.5--63.5]               \\[-0.05cm]
\midrule
\midrule
Observed events          & $6$              & $9$              & $19$              & $17$              & $39$              & $8$                    \\
\noalign{\smallskip}\hline\noalign{\smallskip}
Total background         & $4.8 \pm 2.2$          & $9.0 \pm 1.8$          & $11.9 \pm 1.6$~~          & $15.5 \pm 2.0$~~          & $19.3 \pm 2.7$~~          & $9.3 \pm 1.7$              \\
\noalign{\smallskip}\hline\noalign{\smallskip}
DY         & $4.6 \pm 2.1$          & $6.4 \pm 1.5$          & $5.7 \pm 1.1$          & $6.4 \pm 1.5$          & $8.3 \pm 2.1$          & $5.3 \pm 1.4$              \\
\ttbar         & $0.2 \pm 0.1$          & $2.6 \pm 0.8$          & $6.0 \pm 1.1$          & $8.5 \pm 1.4$          & $10.4 \pm 2.4$~~          & $3.5 \pm 0.9$              \\
Other         & $0.03 \pm 0.01$          &   $0.03 \pm 0.00$          &    $0.24 \pm 0.12$          & $0.50 \pm 0.40$           & $0.50 \pm 0.12$          & $0.45 \pm 0.19$             \\
\bottomrule
\end{tabular*}
}
\end{center}
\label{SRyields}
\end{table}
 
 
\begin{figure}
\centering
\includegraphics[width=0.48\hsize]{fig_04a.pdf}
\includegraphics[width=0.48\hsize]{fig_04b.pdf}
\caption{Post-fit \KLMbbmm in DYCR, SRincl and VR1 (left); \met in SRincl and TCR (right). No selection based on the BDT discriminants is applied in the analysis regions shown in the figures. The signal distributions are normalized to the SM Higgs boson cross-section (including ggF, VBF, and VH production) and assume
$\mathcal{B}(H \rightarrow aa \rightarrow bb\mu\mu)$ as indicated in the legends of the figures (chosen to ensure good visibility in the plot). The hatched bands show the total post-fit statistical and systematic uncertainties of the backgrounds. The histogram labeled as ``Other'' in the legend includes the contributions from the diboson, single-top-quark, $\ttbar{+}V$ and $W$+jets backgrounds.}
\label{SRvalidation1}
\end{figure}
 
\begin{figure}
\centering
\includegraphics[width=0.48\hsize]{fig_05a.pdf}
\includegraphics[width=0.48\hsize]{fig_05b.pdf}
\caption{Post-fit \LogL~in VR2 and SRincl (left); \Mmm in SRincl (right). No selection based on the BDT discriminants is applied in the analysis regions shown in the figures. The signal distributions are normalized to the SM Higgs boson cross-section (including ggF, VBF, and VH production) and assume
$\mathcal{B}(H \rightarrow aa \rightarrow bb\mu\mu)$ as indicated in the legends of the figures (chosen to ensure good visibility in the plot). The hatched bands show the total post-fit statistical and systematic uncertainties of the backgrounds. The histogram labeled as ``Other'' in the legend includes the contributions from the diboson, single-top-quark, $\ttbar{+}V$ and $W$+jets backgrounds. }
\label{SRvalidation2}
\end{figure}
 
 
\begin{figure}
\centering
\includegraphics[width=0.48\hsize]{fig_06a.pdf}
\includegraphics[width=0.48\hsize]{fig_06b.pdf}
\caption{Post-fit BDT20 (left) and BDT50 (right) distributions in SRincl. The signal distributions are normalized to the SM Higgs boson cross-section (including ggF, VBF, and VH production) and assume
$\mathcal{B}(H \rightarrow aa \rightarrow bb\mu\mu)$ as indicated in the legends of the figures (chosen to ensure good visibility in the plot). The hatched bands show the total post-fit statistical and systematic uncertainties of the backgrounds. The histogram labeled as ``Other'' in the legend includes the contributions from the diboson, single-top-quark, $\ttbar{+}V$ and $W$+jets backgrounds. }
\label{SRvalidation3}
\end{figure}
 
 
 
\begin{figure}
\centering
\includegraphics[width=0.99\hsize]{fig_07.pdf}
\caption{Post-background-validation-fit number of  events in all SR bins (after applying the BDT selection) that are tested for the presence of signal. The bin widths are 2~\GeV\ (3~\GeV) in \Mmm for $m_a\leq45$~\GeV\ ($m_a>45$~\GeV). Neighboring bins partially overlap, hence they are not statistically independent.
The bottom panel shows the pull in each bin, defined as $(n_\mathrm{obs}-n_\mathrm{pred})/\sigma_\mathrm{tot}$, where $n_\mathrm{obs}$ is the number of events in the data, $n_\mathrm{pred}$ is the number of fitted background events and $\sigma_\mathrm{tot}$ is the total (systematic and statistical, added in quadrature) uncertainty in the fitted background yield.
Discontinuities in the background predictions appear when the BDT discriminant used for the selection changes from the one trained in the lower mass range to the one trained in the higher mass range. The histogram labeled as ``Other'' in the legend includes the contributions from the diboson, single-top-quark, $\ttbar{+}V$ and $W$+jets backgrounds.}
\label{SRpulls}
\end{figure}
 
 
To test for the presence of new phenomena, fits are performed for each of the 47 hypothesized signal masses in the range $16\leq\Mmm\leq62$~\GeV\ in 1~\GeV\ steps.
It was verified that the analysis is also sufficiently sensitive to a signal with \Mmm centered in between these 1~\GeV\ steps.
TCR, DYCR, and the respective SR bin are included in each fit in order to constrain  the backgrounds and the signal to the data.
 
 
A model-independent fit, i.e.\ not including any signal sample, is performed to test whether the data are compatible with the background-only hypothesis. The result  is a scan of $p_0$-values as shown in Figure~\ref{pvalue}. The largest discrepancy is found at $\Mmm=52$~\GeV, corresponding to a local (global) $p_0$-value of 0.00054 (0.048) and a local (global) significance of $3.3 \sigma$ ($1.7 \sigma$). The global significance was calculated from the asymptotic formulae in Refs.~\cite{Cowan:2010js,Gross_2010}.
 
 
Upper limits, derived using  the CL$_\text{s}$ technique \cite{Read:2002hq,Junk:1999kv}, are set on $\mathcal{B}(H \rightarrow aa \rightarrow bb\mu\mu)$ in a series of conditional fits, this time also including the signal samples.
The limits as a function of $m_a$ are shown in Figure~\ref{Limits}.
Uniform sensitivity is achieved for all masses above 18~\GeV, while for lower signal masses, $m_a \leq 18$~\GeV, the sensitivity of the analysis decreases due to $b$-jets falling below the reconstruction threshold or merging into one reconstructed jet.
Figure~\ref{SRwithSign1} shows \Mmm and BDT$m_a$ distributions  after the signal+background fit for two SR bins, $m_a=35$~\GeV\ and $m_a=52$~\GeV, where the two largest deviations from the background-only hypothesis are observed.
The signal in the plots is scaled to the best-fit value, corresponding to $\mathcal{B}(H \rightarrow aa \rightarrow bb\mu\mu)=6.4\times 10^{-5}$ ($1.9\times 10^{-4}$) for $m_a=35$~\GeV\ ($m_a=52$~\GeV).
 
The upper limits at 95\% CL on $\mathcal{B}(H\rightarrow aa\rightarrow bb\mu\mu)$ range between $0.2 \times 10^{-4}$ and $4.0 \times 10^{-4}$, depending on $m_a$.
These limits improve upon the previous ATLAS result based on 36~$\ifb$ of data~\cite{bbmm36ifb} by a factor of 2--5 over the full  \Mmm range. A factor of $\sim$2 improvement in sensitivity comes from the larger dataset, and a further factor of $\sim$2 is achieved thanks to the use of multivariate techniques to discriminate between the signal and the SM backgrounds. Due to small number of background events at lower signal masses $m_a$, the BDT training is less efficient in this region, and the gain from applying the BDT$m_a$ selection criteria is higher at higher $m_a$.
Taking as an example the favorable scenario with $\mathcal{B}(H\rightarrow aa\rightarrow bb\mu\mu)/ \mathcal{B}(H\rightarrow aa )= 0.16\%$, the analysis probes the Higgs boson branching fraction into pseudoscalars down to $\mathcal{B}(H\rightarrow aa) = 1.3\%$, much lower than the limits derived from combinations of the Higgs boson measurements.
 
So as not to restrict the analysis sensitivity solely to models where the $a$-particle is a pseudoscalar, upper limits obtained without employing the BDT discriminants are also derived as shown in Figure~\ref{Limits_noBDT}.
In addition to being less sensitive to the particle's CP properties, the limits in SRincl without the BDT selection also facilitate reinterpretations of the analysis.
These limits are derived in the same way as described above, i.e.\ by scanning the \Mmm windows of SRincl, but omitting the final selection on the BDT discriminants. The expected limits obtained when employing the baseline analysis strategy are also shown in Figure~\ref{Limits_noBDT} for comparison, illustrating the significant improvement in sensitivity to pseudoscalars when using the BDTs. The excess observed at $\Mmm=52$~\GeV\ in the BDT analysis is not supported by the limits derived without the BDTs.
Figure~\ref{SRpullsnoBDT} shows the data and the estimated backgrounds in all final SR bins, without applying the BDT selection.
 
 
 
 
 
\begin{figure}
\centering
\includegraphics[width=0.6\hsize]{fig_08.pdf}
\caption{The local $p_0$-values are quantified in standard deviations $\sigma$ and plotted as a function of the signal mass hypothesis. Between the points, the $p_0$-values are interpolated and may not be fully representative of the actual sensitivity.}
\label{pvalue}
\end{figure}
 
 
 
\begin{figure}
\centering
\includegraphics[width=0.6\hsize]{fig_09.pdf}
\caption{Upper limits on $\mathcal{B}(H \rightarrow aa \rightarrow bb\mu\mu)$ at 95\% CL, including the BDT selection, as a function of the signal mass hypothesis. Black and red dots show masses for which the hypothesis testing was done. Between these points, the limits are interpolated and may not be fully representative of the actual sensitivity.}
\label{Limits}
\end{figure}
 
 
 
 
 
\begin{figure}
\centering
\includegraphics[width=0.48\hsize]{fig_10a.pdf}
\includegraphics[width=0.48\hsize]{fig_10b.pdf}
\includegraphics[width=0.48\hsize]{fig_10c.pdf}
\includegraphics[width=0.48\hsize]{fig_10d.pdf}
\caption{\Mmm distributions in the SRincl after the BDT$35>0.2$ selection (top left) and BDT$50>0.2$ selection (bottom left), and BDT35 (top right) and BDT50 (bottom right) distributions in the SRincl in the \Mmm window 34--36~\GeV\ and 50.5--53.5~\GeV, respectively.
The signal is scaled to the best-fit value, $\mathcal{B}(H \rightarrow aa \rightarrow bb\mu\mu)=6.4\times 10^{-5}$ for the top plots, and $1.9\times 10^{-4}$ for the bottom plots, assuming the SM Higgs boson cross-section (including ggF, VBF, and VH production). The hatched bands show the total post-fit statistical and systematic uncertainties of the backgrounds and the signal. The histogram labeled as ``Other'' in the legend includes the contributions from the diboson, single-top-quark, $\ttbar{+}V$ and $W$+jets backgrounds.}
\label{SRwithSign1}
\end{figure}
 
 
\begin{figure}
\centering
\includegraphics[width=0.6\hsize]{fig_11.pdf}
\caption{Upper limits on $\mathcal{B}(H \rightarrow aa \rightarrow bb\mu\mu)$  at 95\% CL, with no BDT selection applied, as a function of the signal mass hypothesis. The dash-dotted blue line indicates the expected limit set in the analysis with the BDT selection. Black and red dots show masses for which the hypothesis testing was done. Between these points, the limits are interpolated and may not be fully representative of the actual sensitivity. }
\label{Limits_noBDT}
\end{figure}
 
 
\begin{figure}
\centering
\includegraphics[width=0.99\hsize]{fig_12.pdf}
\caption{Post-background-validation-fit number of  events in all SR bins (without applying the BDT selection) that are tested for the presence of signal. The bin widths are 2~\GeV\ (3~\GeV) in \Mmm for $m_a\leq45$~\GeV\ ($m_a>45$~\GeV). Neighboring bins partially overlap, hence they are not statistically independent.
The bottom panel shows the pull in each bin, defined as $(n_\mathrm{obs}-n_\mathrm{pred})/\sigma_\mathrm{tot}$, where $n_\mathrm{obs}$ is the number of events in the data, $n_\mathrm{pred}$ is the number of fitted background events and $\sigma_\mathrm{tot}$ is the total (systematic and statistical, added in quadrature) uncertainty in the fitted background yield. The discontinuity at $m_a=45$~\GeV\ appears where the \Mmm window size is changed. The histogram labeled as ``Other'' in the legend includes the contributions from the diboson, single-top-quark, $\ttbar{+}V$ and $W$+jets backgrounds.}
\label{SRpullsnoBDT}
\end{figure}
 
 
 
\FloatBarrier
 
% End of text imported from the .//Results.tex input file
 
% The next lines are included from the .//Conclusion.tex input file
\section{Conclusion}
 
A search for light pseudoscalar particles (denoted by $a$) in the decays of the 125~\GeV\ Higgs boson in the final state with two muons and two $b$-tagged jets, $H \rightarrow aa \rightarrow bb\mu\mu$, is presented.
The analysis is performed using 139~fb$^{-1}$ of $\sqrt{s}=13$~\TeV\ $pp$ collision data recorded by the ATLAS detector at the LHC between 2015 and 2018. A narrow resonance is searched for in the dimuon invariant mass spectrum in the range $16 \leq \Mmm \leq 62$~\GeV.
BDT classifiers are trained to distinguish the $H \rightarrow aa$ signal, where $a$ is a pseudoscalar, from the SM backgrounds.
Additionally, the result without selection on the BDT discriminants is also provided to ensure sensitivity to models where the $a$-particle is not necessarily a pseudoscalar, as well as  to facilitate reinterpretations of the analysis.
No significant excess of the data above the SM backgrounds is observed.
In the BDT analysis, the lowest local $p_0$-value of 0.00054 is observed at $\Mmm= 52$~\GeV\ and corresponds to a local significance of $3.3 \sigma$. The global significance of that excess is determined to be $1.7 \sigma$.
Upper limits at 95\%~CL including (excluding) the BDT selection criteria are set on $\mathcal{B}(H \rightarrow aa \rightarrow bb\mu\mu)$ and range between $0.2\times10^{-4}$ and $4.0\times10^{-4}$ ($0.5\times10^{-4}$ and $5.0\times10^{-4}$), depending on $m_a$.
The result including the BDT selection criteria improves upon previous ATLAS and CMS limits by about a factor of 2--5 for $m_a>20$~\GeV, while both results (with and without the BDT) extend the search down to $m_a$ values of 16~\GeV.
 
 
 
 
% End of text imported from the .//Conclusion.tex input file
 
\section*{Acknowledgments}
 
% The next lines are included from the .//acknowledgements/Acknowledgements.tex input file
 
 
We thank CERN for the very successful operation of the LHC, as well as the
support staff from our institutions without whom ATLAS could not be
operated efficiently.
 
We acknowledge the support of
ANPCyT, Argentina;
YerPhI, Armenia;
ARC, Australia;
BMWFW and FWF, Austria;
ANAS, Azerbaijan;
SSTC, Belarus;
CNPq and FAPESP, Brazil;
NSERC, NRC and CFI, Canada;
CERN;
ANID, Chile;
CAS, MOST and NSFC, China;
Minciencias, Colombia;
MSMT CR, MPO CR and VSC CR, Czech Republic;
DNRF and DNSRC, Denmark;
IN2P3-CNRS and CEA-DRF/IRFU, France;
SRNSFG, Georgia;
BMBF, HGF and MPG, Germany;
GSRI, Greece;
RGC and Hong Kong SAR, China;
ISF and Benoziyo Center, Israel;
INFN, Italy;
MEXT and JSPS, Japan;
CNRST, Morocco;
NWO, Netherlands;
RCN, Norway;
MEiN, Poland;
FCT, Portugal;
MNE/IFA, Romania;
JINR;
MES of Russia and NRC KI, Russian Federation;
MESTD, Serbia;
MSSR, Slovakia;
ARRS and MIZ\v{S}, Slovenia;
DSI/NRF, South Africa;
MICINN, Spain;
SRC and Wallenberg Foundation, Sweden;
SERI, SNSF and Cantons of Bern and Geneva, Switzerland;
MOST, Taiwan;
TAEK, Turkey;
STFC, United Kingdom;
DOE and NSF, United States of America.
In addition, individual groups and members have received support from
BCKDF, CANARIE, Compute Canada and CRC, Canada;
COST, ERC, ERDF, Horizon 2020 and Marie Sk{\l}odowska-Curie Actions, European Union;
Investissements d'Avenir Labex, Investissements d'Avenir Idex and ANR, France;
DFG and AvH Foundation, Germany;
Herakleitos, Thales and Aristeia programmes co-financed by EU-ESF and the Greek NSRF, Greece;
BSF-NSF and GIF, Israel;
Norwegian Financial Mechanism 2014-2021, Norway;
NCN and NAWA, Poland;
La Caixa Banking Foundation, CERCA Programme Generalitat de Catalunya and PROMETEO and GenT Programmes Generalitat Valenciana, Spain;
G\"{o}ran Gustafssons Stiftelse, Sweden;
The Royal Society and Leverhulme Trust, United Kingdom.
 
The crucial computing support from all WLCG partners is acknowledged gratefully, in particular from CERN, the ATLAS Tier-1 facilities at TRIUMF (Canada), NDGF (Denmark, Norway, Sweden), CC-IN2P3 (France), KIT/GridKA (Germany), INFN-CNAF (Italy), NL-T1 (Netherlands), PIC (Spain), ASGC (Taiwan), RAL (UK) and BNL (USA), the Tier-2 facilities worldwide and large non-WLCG resource providers. Major contributors of computing resources are listed in Ref.~\cite{ATL-SOFT-PUB-2021-003}.
 
 
% End of text imported from the .//acknowledgements/Acknowledgements.tex input file
 
 
 
\clearpage
 
 
\printbibliography
\clearpage
 
 
 
 
 
\clearpage % ATLAS Collaboration author list
% Data extracted on 17-Jul-2017 for paper reference HIGG-2016-14
%\documentclass[11pt]{article}
%\usepackage{a4wide}\begin{document}
\begin{flushleft}
{\Large The ATLAS Collaboration}

\bigskip

M.~Aaboud$^\textrm{\scriptsize 137d}$,
G.~Aad$^\textrm{\scriptsize 88}$,
B.~Abbott$^\textrm{\scriptsize 115}$,
O.~Abdinov$^\textrm{\scriptsize 12}$$^{,*}$,
B.~Abeloos$^\textrm{\scriptsize 119}$,
S.H.~Abidi$^\textrm{\scriptsize 161}$,
O.S.~AbouZeid$^\textrm{\scriptsize 139}$,
N.L.~Abraham$^\textrm{\scriptsize 151}$,
H.~Abramowicz$^\textrm{\scriptsize 155}$,
H.~Abreu$^\textrm{\scriptsize 154}$,
R.~Abreu$^\textrm{\scriptsize 118}$,
Y.~Abulaiti$^\textrm{\scriptsize 148a,148b}$,
B.S.~Acharya$^\textrm{\scriptsize 167a,167b}$$^{,a}$,
S.~Adachi$^\textrm{\scriptsize 157}$,
L.~Adamczyk$^\textrm{\scriptsize 41a}$,
J.~Adelman$^\textrm{\scriptsize 110}$,
M.~Adersberger$^\textrm{\scriptsize 102}$,
T.~Adye$^\textrm{\scriptsize 133}$,
A.A.~Affolder$^\textrm{\scriptsize 139}$,
T.~Agatonovic-Jovin$^\textrm{\scriptsize 14}$,
C.~Agheorghiesei$^\textrm{\scriptsize 28c}$,
J.A.~Aguilar-Saavedra$^\textrm{\scriptsize 128a,128f}$,
S.P.~Ahlen$^\textrm{\scriptsize 24}$,
F.~Ahmadov$^\textrm{\scriptsize 68}$$^{,b}$,
G.~Aielli$^\textrm{\scriptsize 135a,135b}$,
S.~Akatsuka$^\textrm{\scriptsize 71}$,
H.~Akerstedt$^\textrm{\scriptsize 148a,148b}$,
T.P.A.~{\AA}kesson$^\textrm{\scriptsize 84}$,
E.~Akilli$^\textrm{\scriptsize 52}$,
A.V.~Akimov$^\textrm{\scriptsize 98}$,
G.L.~Alberghi$^\textrm{\scriptsize 22a,22b}$,
J.~Albert$^\textrm{\scriptsize 172}$,
P.~Albicocco$^\textrm{\scriptsize 50}$,
M.J.~Alconada~Verzini$^\textrm{\scriptsize 74}$,
S.C.~Alderweireldt$^\textrm{\scriptsize 108}$,
M.~Aleksa$^\textrm{\scriptsize 32}$,
I.N.~Aleksandrov$^\textrm{\scriptsize 68}$,
C.~Alexa$^\textrm{\scriptsize 28b}$,
G.~Alexander$^\textrm{\scriptsize 155}$,
T.~Alexopoulos$^\textrm{\scriptsize 10}$,
M.~Alhroob$^\textrm{\scriptsize 115}$,
B.~Ali$^\textrm{\scriptsize 130}$,
M.~Aliev$^\textrm{\scriptsize 76a,76b}$,
G.~Alimonti$^\textrm{\scriptsize 94a}$,
J.~Alison$^\textrm{\scriptsize 33}$,
S.P.~Alkire$^\textrm{\scriptsize 38}$,
B.M.M.~Allbrooke$^\textrm{\scriptsize 151}$,
B.W.~Allen$^\textrm{\scriptsize 118}$,
P.P.~Allport$^\textrm{\scriptsize 19}$,
A.~Aloisio$^\textrm{\scriptsize 106a,106b}$,
A.~Alonso$^\textrm{\scriptsize 39}$,
F.~Alonso$^\textrm{\scriptsize 74}$,
C.~Alpigiani$^\textrm{\scriptsize 140}$,
A.A.~Alshehri$^\textrm{\scriptsize 56}$,
M.I.~Alstaty$^\textrm{\scriptsize 88}$,
B.~Alvarez~Gonzalez$^\textrm{\scriptsize 32}$,
D.~\'{A}lvarez~Piqueras$^\textrm{\scriptsize 170}$,
M.G.~Alviggi$^\textrm{\scriptsize 106a,106b}$,
B.T.~Amadio$^\textrm{\scriptsize 16}$,
Y.~Amaral~Coutinho$^\textrm{\scriptsize 26a}$,
C.~Amelung$^\textrm{\scriptsize 25}$,
D.~Amidei$^\textrm{\scriptsize 92}$,
S.P.~Amor~Dos~Santos$^\textrm{\scriptsize 128a,128c}$,
S.~Amoroso$^\textrm{\scriptsize 32}$,
G.~Amundsen$^\textrm{\scriptsize 25}$,
C.~Anastopoulos$^\textrm{\scriptsize 141}$,
L.S.~Ancu$^\textrm{\scriptsize 52}$,
N.~Andari$^\textrm{\scriptsize 19}$,
T.~Andeen$^\textrm{\scriptsize 11}$,
C.F.~Anders$^\textrm{\scriptsize 60b}$,
J.K.~Anders$^\textrm{\scriptsize 77}$,
K.J.~Anderson$^\textrm{\scriptsize 33}$,
A.~Andreazza$^\textrm{\scriptsize 94a,94b}$,
V.~Andrei$^\textrm{\scriptsize 60a}$,
S.~Angelidakis$^\textrm{\scriptsize 9}$,
I.~Angelozzi$^\textrm{\scriptsize 109}$,
A.~Angerami$^\textrm{\scriptsize 38}$,
A.V.~Anisenkov$^\textrm{\scriptsize 111}$$^{,c}$,
N.~Anjos$^\textrm{\scriptsize 13}$,
A.~Annovi$^\textrm{\scriptsize 126a,126b}$,
C.~Antel$^\textrm{\scriptsize 60a}$,
M.~Antonelli$^\textrm{\scriptsize 50}$,
A.~Antonov$^\textrm{\scriptsize 100}$$^{,*}$,
D.J.~Antrim$^\textrm{\scriptsize 166}$,
F.~Anulli$^\textrm{\scriptsize 134a}$,
M.~Aoki$^\textrm{\scriptsize 69}$,
L.~Aperio~Bella$^\textrm{\scriptsize 32}$,
G.~Arabidze$^\textrm{\scriptsize 93}$,
Y.~Arai$^\textrm{\scriptsize 69}$,
J.P.~Araque$^\textrm{\scriptsize 128a}$,
V.~Araujo~Ferraz$^\textrm{\scriptsize 26a}$,
A.T.H.~Arce$^\textrm{\scriptsize 48}$,
R.E.~Ardell$^\textrm{\scriptsize 80}$,
F.A.~Arduh$^\textrm{\scriptsize 74}$,
J-F.~Arguin$^\textrm{\scriptsize 97}$,
S.~Argyropoulos$^\textrm{\scriptsize 66}$,
M.~Arik$^\textrm{\scriptsize 20a}$,
A.J.~Armbruster$^\textrm{\scriptsize 32}$,
L.J.~Armitage$^\textrm{\scriptsize 79}$,
O.~Arnaez$^\textrm{\scriptsize 161}$,
H.~Arnold$^\textrm{\scriptsize 51}$,
M.~Arratia$^\textrm{\scriptsize 30}$,
O.~Arslan$^\textrm{\scriptsize 23}$,
A.~Artamonov$^\textrm{\scriptsize 99}$,
G.~Artoni$^\textrm{\scriptsize 122}$,
S.~Artz$^\textrm{\scriptsize 86}$,
S.~Asai$^\textrm{\scriptsize 157}$,
N.~Asbah$^\textrm{\scriptsize 45}$,
A.~Ashkenazi$^\textrm{\scriptsize 155}$,
L.~Asquith$^\textrm{\scriptsize 151}$,
K.~Assamagan$^\textrm{\scriptsize 27}$,
R.~Astalos$^\textrm{\scriptsize 146a}$,
M.~Atkinson$^\textrm{\scriptsize 169}$,
N.B.~Atlay$^\textrm{\scriptsize 143}$,
K.~Augsten$^\textrm{\scriptsize 130}$,
G.~Avolio$^\textrm{\scriptsize 32}$,
B.~Axen$^\textrm{\scriptsize 16}$,
M.K.~Ayoub$^\textrm{\scriptsize 119}$,
G.~Azuelos$^\textrm{\scriptsize 97}$$^{,d}$,
A.E.~Baas$^\textrm{\scriptsize 60a}$,
M.J.~Baca$^\textrm{\scriptsize 19}$,
H.~Bachacou$^\textrm{\scriptsize 138}$,
K.~Bachas$^\textrm{\scriptsize 76a,76b}$,
M.~Backes$^\textrm{\scriptsize 122}$,
M.~Backhaus$^\textrm{\scriptsize 32}$,
P.~Bagnaia$^\textrm{\scriptsize 134a,134b}$,
M.~Bahmani$^\textrm{\scriptsize 42}$,
H.~Bahrasemani$^\textrm{\scriptsize 144}$,
J.T.~Baines$^\textrm{\scriptsize 133}$,
M.~Bajic$^\textrm{\scriptsize 39}$,
O.K.~Baker$^\textrm{\scriptsize 179}$,
E.M.~Baldin$^\textrm{\scriptsize 111}$$^{,c}$,
P.~Balek$^\textrm{\scriptsize 175}$,
F.~Balli$^\textrm{\scriptsize 138}$,
W.K.~Balunas$^\textrm{\scriptsize 124}$,
E.~Banas$^\textrm{\scriptsize 42}$,
A.~Bandyopadhyay$^\textrm{\scriptsize 23}$,
Sw.~Banerjee$^\textrm{\scriptsize 176}$$^{,e}$,
A.A.E.~Bannoura$^\textrm{\scriptsize 178}$,
L.~Barak$^\textrm{\scriptsize 32}$,
E.L.~Barberio$^\textrm{\scriptsize 91}$,
D.~Barberis$^\textrm{\scriptsize 53a,53b}$,
M.~Barbero$^\textrm{\scriptsize 88}$,
T.~Barillari$^\textrm{\scriptsize 103}$,
M-S~Barisits$^\textrm{\scriptsize 32}$,
J.T.~Barkeloo$^\textrm{\scriptsize 118}$,
T.~Barklow$^\textrm{\scriptsize 145}$,
N.~Barlow$^\textrm{\scriptsize 30}$,
S.L.~Barnes$^\textrm{\scriptsize 36c}$,
B.M.~Barnett$^\textrm{\scriptsize 133}$,
R.M.~Barnett$^\textrm{\scriptsize 16}$,
Z.~Barnovska-Blenessy$^\textrm{\scriptsize 36a}$,
A.~Baroncelli$^\textrm{\scriptsize 136a}$,
G.~Barone$^\textrm{\scriptsize 25}$,
A.J.~Barr$^\textrm{\scriptsize 122}$,
L.~Barranco~Navarro$^\textrm{\scriptsize 170}$,
F.~Barreiro$^\textrm{\scriptsize 85}$,
J.~Barreiro~Guimar\~{a}es~da~Costa$^\textrm{\scriptsize 35a}$,
R.~Bartoldus$^\textrm{\scriptsize 145}$,
A.E.~Barton$^\textrm{\scriptsize 75}$,
P.~Bartos$^\textrm{\scriptsize 146a}$,
A.~Basalaev$^\textrm{\scriptsize 125}$,
A.~Bassalat$^\textrm{\scriptsize 119}$$^{,f}$,
R.L.~Bates$^\textrm{\scriptsize 56}$,
S.J.~Batista$^\textrm{\scriptsize 161}$,
J.R.~Batley$^\textrm{\scriptsize 30}$,
M.~Battaglia$^\textrm{\scriptsize 139}$,
M.~Bauce$^\textrm{\scriptsize 134a,134b}$,
F.~Bauer$^\textrm{\scriptsize 138}$,
H.S.~Bawa$^\textrm{\scriptsize 145}$$^{,g}$,
J.B.~Beacham$^\textrm{\scriptsize 113}$,
M.D.~Beattie$^\textrm{\scriptsize 75}$,
T.~Beau$^\textrm{\scriptsize 83}$,
P.H.~Beauchemin$^\textrm{\scriptsize 165}$,
P.~Bechtle$^\textrm{\scriptsize 23}$,
H.P.~Beck$^\textrm{\scriptsize 18}$$^{,h}$,
H.C.~Beck$^\textrm{\scriptsize 57}$,
K.~Becker$^\textrm{\scriptsize 122}$,
M.~Becker$^\textrm{\scriptsize 86}$,
M.~Beckingham$^\textrm{\scriptsize 173}$,
C.~Becot$^\textrm{\scriptsize 112}$,
A.J.~Beddall$^\textrm{\scriptsize 20e}$,
A.~Beddall$^\textrm{\scriptsize 20b}$,
V.A.~Bednyakov$^\textrm{\scriptsize 68}$,
M.~Bedognetti$^\textrm{\scriptsize 109}$,
C.P.~Bee$^\textrm{\scriptsize 150}$,
T.A.~Beermann$^\textrm{\scriptsize 32}$,
M.~Begalli$^\textrm{\scriptsize 26a}$,
M.~Begel$^\textrm{\scriptsize 27}$,
J.K.~Behr$^\textrm{\scriptsize 45}$,
A.S.~Bell$^\textrm{\scriptsize 81}$,
G.~Bella$^\textrm{\scriptsize 155}$,
L.~Bellagamba$^\textrm{\scriptsize 22a}$,
A.~Bellerive$^\textrm{\scriptsize 31}$,
M.~Bellomo$^\textrm{\scriptsize 154}$,
K.~Belotskiy$^\textrm{\scriptsize 100}$,
O.~Beltramello$^\textrm{\scriptsize 32}$,
N.L.~Belyaev$^\textrm{\scriptsize 100}$,
O.~Benary$^\textrm{\scriptsize 155}$$^{,*}$,
D.~Benchekroun$^\textrm{\scriptsize 137a}$,
M.~Bender$^\textrm{\scriptsize 102}$,
K.~Bendtz$^\textrm{\scriptsize 148a,148b}$,
N.~Benekos$^\textrm{\scriptsize 10}$,
Y.~Benhammou$^\textrm{\scriptsize 155}$,
E.~Benhar~Noccioli$^\textrm{\scriptsize 179}$,
J.~Benitez$^\textrm{\scriptsize 66}$,
D.P.~Benjamin$^\textrm{\scriptsize 48}$,
M.~Benoit$^\textrm{\scriptsize 52}$,
J.R.~Bensinger$^\textrm{\scriptsize 25}$,
S.~Bentvelsen$^\textrm{\scriptsize 109}$,
L.~Beresford$^\textrm{\scriptsize 122}$,
M.~Beretta$^\textrm{\scriptsize 50}$,
D.~Berge$^\textrm{\scriptsize 109}$,
E.~Bergeaas~Kuutmann$^\textrm{\scriptsize 168}$,
N.~Berger$^\textrm{\scriptsize 5}$,
J.~Beringer$^\textrm{\scriptsize 16}$,
S.~Berlendis$^\textrm{\scriptsize 58}$,
N.R.~Bernard$^\textrm{\scriptsize 89}$,
G.~Bernardi$^\textrm{\scriptsize 83}$,
C.~Bernius$^\textrm{\scriptsize 145}$,
F.U.~Bernlochner$^\textrm{\scriptsize 23}$,
T.~Berry$^\textrm{\scriptsize 80}$,
P.~Berta$^\textrm{\scriptsize 86}$,
C.~Bertella$^\textrm{\scriptsize 35a}$,
G.~Bertoli$^\textrm{\scriptsize 148a,148b}$,
F.~Bertolucci$^\textrm{\scriptsize 126a,126b}$,
I.A.~Bertram$^\textrm{\scriptsize 75}$,
C.~Bertsche$^\textrm{\scriptsize 45}$,
D.~Bertsche$^\textrm{\scriptsize 115}$,
G.J.~Besjes$^\textrm{\scriptsize 39}$,
O.~Bessidskaia~Bylund$^\textrm{\scriptsize 148a,148b}$,
M.~Bessner$^\textrm{\scriptsize 45}$,
N.~Besson$^\textrm{\scriptsize 138}$,
C.~Betancourt$^\textrm{\scriptsize 51}$,
A.~Bethani$^\textrm{\scriptsize 87}$,
S.~Bethke$^\textrm{\scriptsize 103}$,
A.J.~Bevan$^\textrm{\scriptsize 79}$,
J.~Beyer$^\textrm{\scriptsize 103}$,
R.M.~Bianchi$^\textrm{\scriptsize 127}$,
O.~Biebel$^\textrm{\scriptsize 102}$,
D.~Biedermann$^\textrm{\scriptsize 17}$,
R.~Bielski$^\textrm{\scriptsize 87}$,
K.~Bierwagen$^\textrm{\scriptsize 86}$,
N.V.~Biesuz$^\textrm{\scriptsize 126a,126b}$,
M.~Biglietti$^\textrm{\scriptsize 136a}$,
T.R.V.~Billoud$^\textrm{\scriptsize 97}$,
H.~Bilokon$^\textrm{\scriptsize 50}$,
M.~Bindi$^\textrm{\scriptsize 57}$,
A.~Bingul$^\textrm{\scriptsize 20b}$,
C.~Bini$^\textrm{\scriptsize 134a,134b}$,
S.~Biondi$^\textrm{\scriptsize 22a,22b}$,
T.~Bisanz$^\textrm{\scriptsize 57}$,
C.~Bittrich$^\textrm{\scriptsize 47}$,
D.M.~Bjergaard$^\textrm{\scriptsize 48}$,
C.W.~Black$^\textrm{\scriptsize 152}$,
J.E.~Black$^\textrm{\scriptsize 145}$,
K.M.~Black$^\textrm{\scriptsize 24}$,
R.E.~Blair$^\textrm{\scriptsize 6}$,
T.~Blazek$^\textrm{\scriptsize 146a}$,
I.~Bloch$^\textrm{\scriptsize 45}$,
C.~Blocker$^\textrm{\scriptsize 25}$,
A.~Blue$^\textrm{\scriptsize 56}$,
W.~Blum$^\textrm{\scriptsize 86}$$^{,*}$,
U.~Blumenschein$^\textrm{\scriptsize 79}$,
S.~Blunier$^\textrm{\scriptsize 34a}$,
G.J.~Bobbink$^\textrm{\scriptsize 109}$,
V.S.~Bobrovnikov$^\textrm{\scriptsize 111}$$^{,c}$,
S.S.~Bocchetta$^\textrm{\scriptsize 84}$,
A.~Bocci$^\textrm{\scriptsize 48}$,
C.~Bock$^\textrm{\scriptsize 102}$,
M.~Boehler$^\textrm{\scriptsize 51}$,
D.~Boerner$^\textrm{\scriptsize 178}$,
D.~Bogavac$^\textrm{\scriptsize 102}$,
A.G.~Bogdanchikov$^\textrm{\scriptsize 111}$,
C.~Bohm$^\textrm{\scriptsize 148a}$,
V.~Boisvert$^\textrm{\scriptsize 80}$,
P.~Bokan$^\textrm{\scriptsize 168}$$^{,i}$,
T.~Bold$^\textrm{\scriptsize 41a}$,
A.S.~Boldyrev$^\textrm{\scriptsize 101}$,
A.E.~Bolz$^\textrm{\scriptsize 60b}$,
M.~Bomben$^\textrm{\scriptsize 83}$,
M.~Bona$^\textrm{\scriptsize 79}$,
M.~Boonekamp$^\textrm{\scriptsize 138}$,
A.~Borisov$^\textrm{\scriptsize 132}$,
G.~Borissov$^\textrm{\scriptsize 75}$,
J.~Bortfeldt$^\textrm{\scriptsize 32}$,
D.~Bortoletto$^\textrm{\scriptsize 122}$,
V.~Bortolotto$^\textrm{\scriptsize 62a}$,
D.~Boscherini$^\textrm{\scriptsize 22a}$,
M.~Bosman$^\textrm{\scriptsize 13}$,
J.D.~Bossio~Sola$^\textrm{\scriptsize 29}$,
J.~Boudreau$^\textrm{\scriptsize 127}$,
J.~Bouffard$^\textrm{\scriptsize 2}$,
E.V.~Bouhova-Thacker$^\textrm{\scriptsize 75}$,
D.~Boumediene$^\textrm{\scriptsize 37}$,
C.~Bourdarios$^\textrm{\scriptsize 119}$,
S.K.~Boutle$^\textrm{\scriptsize 56}$,
A.~Boveia$^\textrm{\scriptsize 113}$,
J.~Boyd$^\textrm{\scriptsize 32}$,
I.R.~Boyko$^\textrm{\scriptsize 68}$,
J.~Bracinik$^\textrm{\scriptsize 19}$,
A.~Brandt$^\textrm{\scriptsize 8}$,
G.~Brandt$^\textrm{\scriptsize 57}$,
O.~Brandt$^\textrm{\scriptsize 60a}$,
U.~Bratzler$^\textrm{\scriptsize 158}$,
B.~Brau$^\textrm{\scriptsize 89}$,
J.E.~Brau$^\textrm{\scriptsize 118}$,
W.D.~Breaden~Madden$^\textrm{\scriptsize 56}$,
K.~Brendlinger$^\textrm{\scriptsize 45}$,
A.J.~Brennan$^\textrm{\scriptsize 91}$,
L.~Brenner$^\textrm{\scriptsize 109}$,
R.~Brenner$^\textrm{\scriptsize 168}$,
S.~Bressler$^\textrm{\scriptsize 175}$,
D.L.~Briglin$^\textrm{\scriptsize 19}$,
T.M.~Bristow$^\textrm{\scriptsize 49}$,
D.~Britton$^\textrm{\scriptsize 56}$,
D.~Britzger$^\textrm{\scriptsize 45}$,
F.M.~Brochu$^\textrm{\scriptsize 30}$,
I.~Brock$^\textrm{\scriptsize 23}$,
R.~Brock$^\textrm{\scriptsize 93}$,
G.~Brooijmans$^\textrm{\scriptsize 38}$,
T.~Brooks$^\textrm{\scriptsize 80}$,
W.K.~Brooks$^\textrm{\scriptsize 34b}$,
J.~Brosamer$^\textrm{\scriptsize 16}$,
E.~Brost$^\textrm{\scriptsize 110}$,
J.H~Broughton$^\textrm{\scriptsize 19}$,
P.A.~Bruckman~de~Renstrom$^\textrm{\scriptsize 42}$,
D.~Bruncko$^\textrm{\scriptsize 146b}$,
A.~Bruni$^\textrm{\scriptsize 22a}$,
G.~Bruni$^\textrm{\scriptsize 22a}$,
L.S.~Bruni$^\textrm{\scriptsize 109}$,
BH~Brunt$^\textrm{\scriptsize 30}$,
M.~Bruschi$^\textrm{\scriptsize 22a}$,
N.~Bruscino$^\textrm{\scriptsize 23}$,
P.~Bryant$^\textrm{\scriptsize 33}$,
L.~Bryngemark$^\textrm{\scriptsize 45}$,
T.~Buanes$^\textrm{\scriptsize 15}$,
Q.~Buat$^\textrm{\scriptsize 144}$,
P.~Buchholz$^\textrm{\scriptsize 143}$,
A.G.~Buckley$^\textrm{\scriptsize 56}$,
I.A.~Budagov$^\textrm{\scriptsize 68}$,
F.~Buehrer$^\textrm{\scriptsize 51}$,
M.K.~Bugge$^\textrm{\scriptsize 121}$,
O.~Bulekov$^\textrm{\scriptsize 100}$,
D.~Bullock$^\textrm{\scriptsize 8}$,
T.J.~Burch$^\textrm{\scriptsize 110}$,
S.~Burdin$^\textrm{\scriptsize 77}$,
C.D.~Burgard$^\textrm{\scriptsize 51}$,
A.M.~Burger$^\textrm{\scriptsize 5}$,
B.~Burghgrave$^\textrm{\scriptsize 110}$,
K.~Burka$^\textrm{\scriptsize 42}$,
S.~Burke$^\textrm{\scriptsize 133}$,
I.~Burmeister$^\textrm{\scriptsize 46}$,
J.T.P.~Burr$^\textrm{\scriptsize 122}$,
E.~Busato$^\textrm{\scriptsize 37}$,
D.~B\"uscher$^\textrm{\scriptsize 51}$,
V.~B\"uscher$^\textrm{\scriptsize 86}$,
P.~Bussey$^\textrm{\scriptsize 56}$,
J.M.~Butler$^\textrm{\scriptsize 24}$,
C.M.~Buttar$^\textrm{\scriptsize 56}$,
J.M.~Butterworth$^\textrm{\scriptsize 81}$,
P.~Butti$^\textrm{\scriptsize 32}$,
W.~Buttinger$^\textrm{\scriptsize 27}$,
A.~Buzatu$^\textrm{\scriptsize 35c}$,
A.R.~Buzykaev$^\textrm{\scriptsize 111}$$^{,c}$,
S.~Cabrera~Urb\'an$^\textrm{\scriptsize 170}$,
D.~Caforio$^\textrm{\scriptsize 130}$,
V.M.~Cairo$^\textrm{\scriptsize 40a,40b}$,
O.~Cakir$^\textrm{\scriptsize 4a}$,
N.~Calace$^\textrm{\scriptsize 52}$,
P.~Calafiura$^\textrm{\scriptsize 16}$,
A.~Calandri$^\textrm{\scriptsize 88}$,
G.~Calderini$^\textrm{\scriptsize 83}$,
P.~Calfayan$^\textrm{\scriptsize 64}$,
G.~Callea$^\textrm{\scriptsize 40a,40b}$,
L.P.~Caloba$^\textrm{\scriptsize 26a}$,
S.~Calvente~Lopez$^\textrm{\scriptsize 85}$,
D.~Calvet$^\textrm{\scriptsize 37}$,
S.~Calvet$^\textrm{\scriptsize 37}$,
T.P.~Calvet$^\textrm{\scriptsize 88}$,
R.~Camacho~Toro$^\textrm{\scriptsize 33}$,
S.~Camarda$^\textrm{\scriptsize 32}$,
P.~Camarri$^\textrm{\scriptsize 135a,135b}$,
D.~Cameron$^\textrm{\scriptsize 121}$,
R.~Caminal~Armadans$^\textrm{\scriptsize 169}$,
C.~Camincher$^\textrm{\scriptsize 58}$,
S.~Campana$^\textrm{\scriptsize 32}$,
M.~Campanelli$^\textrm{\scriptsize 81}$,
A.~Camplani$^\textrm{\scriptsize 94a,94b}$,
A.~Campoverde$^\textrm{\scriptsize 143}$,
V.~Canale$^\textrm{\scriptsize 106a,106b}$,
M.~Cano~Bret$^\textrm{\scriptsize 36c}$,
J.~Cantero$^\textrm{\scriptsize 116}$,
T.~Cao$^\textrm{\scriptsize 155}$,
M.D.M.~Capeans~Garrido$^\textrm{\scriptsize 32}$,
I.~Caprini$^\textrm{\scriptsize 28b}$,
M.~Caprini$^\textrm{\scriptsize 28b}$,
M.~Capua$^\textrm{\scriptsize 40a,40b}$,
R.M.~Carbone$^\textrm{\scriptsize 38}$,
R.~Cardarelli$^\textrm{\scriptsize 135a}$,
F.~Cardillo$^\textrm{\scriptsize 51}$,
I.~Carli$^\textrm{\scriptsize 131}$,
T.~Carli$^\textrm{\scriptsize 32}$,
G.~Carlino$^\textrm{\scriptsize 106a}$,
B.T.~Carlson$^\textrm{\scriptsize 127}$,
L.~Carminati$^\textrm{\scriptsize 94a,94b}$,
R.M.D.~Carney$^\textrm{\scriptsize 148a,148b}$,
S.~Caron$^\textrm{\scriptsize 108}$,
E.~Carquin$^\textrm{\scriptsize 34b}$,
S.~Carr\'a$^\textrm{\scriptsize 94a,94b}$,
G.D.~Carrillo-Montoya$^\textrm{\scriptsize 32}$,
D.~Casadei$^\textrm{\scriptsize 19}$,
M.P.~Casado$^\textrm{\scriptsize 13}$$^{,j}$,
M.~Casolino$^\textrm{\scriptsize 13}$,
D.W.~Casper$^\textrm{\scriptsize 166}$,
R.~Castelijn$^\textrm{\scriptsize 109}$,
V.~Castillo~Gimenez$^\textrm{\scriptsize 170}$,
N.F.~Castro$^\textrm{\scriptsize 128a}$$^{,k}$,
A.~Catinaccio$^\textrm{\scriptsize 32}$,
J.R.~Catmore$^\textrm{\scriptsize 121}$,
A.~Cattai$^\textrm{\scriptsize 32}$,
J.~Caudron$^\textrm{\scriptsize 23}$,
V.~Cavaliere$^\textrm{\scriptsize 169}$,
E.~Cavallaro$^\textrm{\scriptsize 13}$,
D.~Cavalli$^\textrm{\scriptsize 94a}$,
M.~Cavalli-Sforza$^\textrm{\scriptsize 13}$,
V.~Cavasinni$^\textrm{\scriptsize 126a,126b}$,
E.~Celebi$^\textrm{\scriptsize 20d}$,
F.~Ceradini$^\textrm{\scriptsize 136a,136b}$,
L.~Cerda~Alberich$^\textrm{\scriptsize 170}$,
A.S.~Cerqueira$^\textrm{\scriptsize 26b}$,
A.~Cerri$^\textrm{\scriptsize 151}$,
L.~Cerrito$^\textrm{\scriptsize 135a,135b}$,
F.~Cerutti$^\textrm{\scriptsize 16}$,
A.~Cervelli$^\textrm{\scriptsize 18}$,
S.A.~Cetin$^\textrm{\scriptsize 20d}$,
A.~Chafaq$^\textrm{\scriptsize 137a}$,
D.~Chakraborty$^\textrm{\scriptsize 110}$,
S.K.~Chan$^\textrm{\scriptsize 59}$,
W.S.~Chan$^\textrm{\scriptsize 109}$,
Y.L.~Chan$^\textrm{\scriptsize 62a}$,
P.~Chang$^\textrm{\scriptsize 169}$,
J.D.~Chapman$^\textrm{\scriptsize 30}$,
D.G.~Charlton$^\textrm{\scriptsize 19}$,
C.C.~Chau$^\textrm{\scriptsize 31}$,
C.A.~Chavez~Barajas$^\textrm{\scriptsize 151}$,
S.~Che$^\textrm{\scriptsize 113}$,
S.~Cheatham$^\textrm{\scriptsize 167a,167c}$,
A.~Chegwidden$^\textrm{\scriptsize 93}$,
S.~Chekanov$^\textrm{\scriptsize 6}$,
S.V.~Chekulaev$^\textrm{\scriptsize 163a}$,
G.A.~Chelkov$^\textrm{\scriptsize 68}$$^{,l}$,
M.A.~Chelstowska$^\textrm{\scriptsize 32}$,
C.~Chen$^\textrm{\scriptsize 67}$,
H.~Chen$^\textrm{\scriptsize 27}$,
J.~Chen$^\textrm{\scriptsize 36a}$,
S.~Chen$^\textrm{\scriptsize 35b}$,
S.~Chen$^\textrm{\scriptsize 157}$,
X.~Chen$^\textrm{\scriptsize 35c}$$^{,m}$,
Y.~Chen$^\textrm{\scriptsize 70}$,
H.C.~Cheng$^\textrm{\scriptsize 92}$,
H.J.~Cheng$^\textrm{\scriptsize 35a,35d}$,
A.~Cheplakov$^\textrm{\scriptsize 68}$,
E.~Cheremushkina$^\textrm{\scriptsize 132}$,
R.~Cherkaoui~El~Moursli$^\textrm{\scriptsize 137e}$,
E.~Cheu$^\textrm{\scriptsize 7}$,
K.~Cheung$^\textrm{\scriptsize 63}$,
L.~Chevalier$^\textrm{\scriptsize 138}$,
V.~Chiarella$^\textrm{\scriptsize 50}$,
G.~Chiarelli$^\textrm{\scriptsize 126a,126b}$,
G.~Chiodini$^\textrm{\scriptsize 76a}$,
A.S.~Chisholm$^\textrm{\scriptsize 32}$,
A.~Chitan$^\textrm{\scriptsize 28b}$,
Y.H.~Chiu$^\textrm{\scriptsize 172}$,
M.V.~Chizhov$^\textrm{\scriptsize 68}$,
K.~Choi$^\textrm{\scriptsize 64}$,
A.R.~Chomont$^\textrm{\scriptsize 37}$,
S.~Chouridou$^\textrm{\scriptsize 156}$,
Y.S.~Chow$^\textrm{\scriptsize 62a}$,
V.~Christodoulou$^\textrm{\scriptsize 81}$,
M.C.~Chu$^\textrm{\scriptsize 62a}$,
J.~Chudoba$^\textrm{\scriptsize 129}$,
A.J.~Chuinard$^\textrm{\scriptsize 90}$,
J.J.~Chwastowski$^\textrm{\scriptsize 42}$,
L.~Chytka$^\textrm{\scriptsize 117}$,
A.K.~Ciftci$^\textrm{\scriptsize 4a}$,
D.~Cinca$^\textrm{\scriptsize 46}$,
V.~Cindro$^\textrm{\scriptsize 78}$,
I.A.~Cioara$^\textrm{\scriptsize 23}$,
C.~Ciocca$^\textrm{\scriptsize 22a,22b}$,
A.~Ciocio$^\textrm{\scriptsize 16}$,
F.~Cirotto$^\textrm{\scriptsize 106a,106b}$,
Z.H.~Citron$^\textrm{\scriptsize 175}$,
M.~Citterio$^\textrm{\scriptsize 94a}$,
M.~Ciubancan$^\textrm{\scriptsize 28b}$,
A.~Clark$^\textrm{\scriptsize 52}$,
B.L.~Clark$^\textrm{\scriptsize 59}$,
M.R.~Clark$^\textrm{\scriptsize 38}$,
P.J.~Clark$^\textrm{\scriptsize 49}$,
R.N.~Clarke$^\textrm{\scriptsize 16}$,
C.~Clement$^\textrm{\scriptsize 148a,148b}$,
Y.~Coadou$^\textrm{\scriptsize 88}$,
M.~Cobal$^\textrm{\scriptsize 167a,167c}$,
A.~Coccaro$^\textrm{\scriptsize 52}$,
J.~Cochran$^\textrm{\scriptsize 67}$,
L.~Colasurdo$^\textrm{\scriptsize 108}$,
B.~Cole$^\textrm{\scriptsize 38}$,
A.P.~Colijn$^\textrm{\scriptsize 109}$,
J.~Collot$^\textrm{\scriptsize 58}$,
T.~Colombo$^\textrm{\scriptsize 166}$,
P.~Conde~Mui\~no$^\textrm{\scriptsize 128a,128b}$,
E.~Coniavitis$^\textrm{\scriptsize 51}$,
S.H.~Connell$^\textrm{\scriptsize 147b}$,
I.A.~Connelly$^\textrm{\scriptsize 87}$,
S.~Constantinescu$^\textrm{\scriptsize 28b}$,
G.~Conti$^\textrm{\scriptsize 32}$,
F.~Conventi$^\textrm{\scriptsize 106a}$$^{,n}$,
M.~Cooke$^\textrm{\scriptsize 16}$,
A.M.~Cooper-Sarkar$^\textrm{\scriptsize 122}$,
F.~Cormier$^\textrm{\scriptsize 171}$,
K.J.R.~Cormier$^\textrm{\scriptsize 161}$,
M.~Corradi$^\textrm{\scriptsize 134a,134b}$,
F.~Corriveau$^\textrm{\scriptsize 90}$$^{,o}$,
A.~Cortes-Gonzalez$^\textrm{\scriptsize 32}$,
G.~Cortiana$^\textrm{\scriptsize 103}$,
G.~Costa$^\textrm{\scriptsize 94a}$,
M.J.~Costa$^\textrm{\scriptsize 170}$,
D.~Costanzo$^\textrm{\scriptsize 141}$,
G.~Cottin$^\textrm{\scriptsize 30}$,
G.~Cowan$^\textrm{\scriptsize 80}$,
B.E.~Cox$^\textrm{\scriptsize 87}$,
K.~Cranmer$^\textrm{\scriptsize 112}$,
S.J.~Crawley$^\textrm{\scriptsize 56}$,
R.A.~Creager$^\textrm{\scriptsize 124}$,
G.~Cree$^\textrm{\scriptsize 31}$,
S.~Cr\'ep\'e-Renaudin$^\textrm{\scriptsize 58}$,
F.~Crescioli$^\textrm{\scriptsize 83}$,
W.A.~Cribbs$^\textrm{\scriptsize 148a,148b}$,
M.~Cristinziani$^\textrm{\scriptsize 23}$,
V.~Croft$^\textrm{\scriptsize 108}$,
G.~Crosetti$^\textrm{\scriptsize 40a,40b}$,
A.~Cueto$^\textrm{\scriptsize 85}$,
T.~Cuhadar~Donszelmann$^\textrm{\scriptsize 141}$,
A.R.~Cukierman$^\textrm{\scriptsize 145}$,
J.~Cummings$^\textrm{\scriptsize 179}$,
M.~Curatolo$^\textrm{\scriptsize 50}$,
J.~C\'uth$^\textrm{\scriptsize 86}$,
S.~Czekierda$^\textrm{\scriptsize 42}$,
P.~Czodrowski$^\textrm{\scriptsize 32}$,
G.~D'amen$^\textrm{\scriptsize 22a,22b}$,
S.~D'Auria$^\textrm{\scriptsize 56}$,
L.~D'eramo$^\textrm{\scriptsize 83}$,
M.~D'Onofrio$^\textrm{\scriptsize 77}$,
M.J.~Da~Cunha~Sargedas~De~Sousa$^\textrm{\scriptsize 128a,128b}$,
C.~Da~Via$^\textrm{\scriptsize 87}$,
W.~Dabrowski$^\textrm{\scriptsize 41a}$,
T.~Dado$^\textrm{\scriptsize 146a}$,
T.~Dai$^\textrm{\scriptsize 92}$,
O.~Dale$^\textrm{\scriptsize 15}$,
F.~Dallaire$^\textrm{\scriptsize 97}$,
C.~Dallapiccola$^\textrm{\scriptsize 89}$,
M.~Dam$^\textrm{\scriptsize 39}$,
J.R.~Dandoy$^\textrm{\scriptsize 124}$,
M.F.~Daneri$^\textrm{\scriptsize 29}$,
N.P.~Dang$^\textrm{\scriptsize 176}$,
A.C.~Daniells$^\textrm{\scriptsize 19}$,
N.S.~Dann$^\textrm{\scriptsize 87}$,
M.~Danninger$^\textrm{\scriptsize 171}$,
M.~Dano~Hoffmann$^\textrm{\scriptsize 138}$,
V.~Dao$^\textrm{\scriptsize 150}$,
G.~Darbo$^\textrm{\scriptsize 53a}$,
S.~Darmora$^\textrm{\scriptsize 8}$,
J.~Dassoulas$^\textrm{\scriptsize 3}$,
A.~Dattagupta$^\textrm{\scriptsize 118}$,
T.~Daubney$^\textrm{\scriptsize 45}$,
W.~Davey$^\textrm{\scriptsize 23}$,
C.~David$^\textrm{\scriptsize 45}$,
T.~Davidek$^\textrm{\scriptsize 131}$,
D.R.~Davis$^\textrm{\scriptsize 48}$,
P.~Davison$^\textrm{\scriptsize 81}$,
E.~Dawe$^\textrm{\scriptsize 91}$,
I.~Dawson$^\textrm{\scriptsize 141}$,
K.~De$^\textrm{\scriptsize 8}$,
R.~de~Asmundis$^\textrm{\scriptsize 106a}$,
A.~De~Benedetti$^\textrm{\scriptsize 115}$,
S.~De~Castro$^\textrm{\scriptsize 22a,22b}$,
S.~De~Cecco$^\textrm{\scriptsize 83}$,
N.~De~Groot$^\textrm{\scriptsize 108}$,
P.~de~Jong$^\textrm{\scriptsize 109}$,
H.~De~la~Torre$^\textrm{\scriptsize 93}$,
F.~De~Lorenzi$^\textrm{\scriptsize 67}$,
A.~De~Maria$^\textrm{\scriptsize 57}$,
D.~De~Pedis$^\textrm{\scriptsize 134a}$,
A.~De~Salvo$^\textrm{\scriptsize 134a}$,
U.~De~Sanctis$^\textrm{\scriptsize 135a,135b}$,
A.~De~Santo$^\textrm{\scriptsize 151}$,
K.~De~Vasconcelos~Corga$^\textrm{\scriptsize 88}$,
J.B.~De~Vivie~De~Regie$^\textrm{\scriptsize 119}$,
W.J.~Dearnaley$^\textrm{\scriptsize 75}$,
R.~Debbe$^\textrm{\scriptsize 27}$,
C.~Debenedetti$^\textrm{\scriptsize 139}$,
D.V.~Dedovich$^\textrm{\scriptsize 68}$,
N.~Dehghanian$^\textrm{\scriptsize 3}$,
I.~Deigaard$^\textrm{\scriptsize 109}$,
M.~Del~Gaudio$^\textrm{\scriptsize 40a,40b}$,
J.~Del~Peso$^\textrm{\scriptsize 85}$,
D.~Delgove$^\textrm{\scriptsize 119}$,
F.~Deliot$^\textrm{\scriptsize 138}$,
C.M.~Delitzsch$^\textrm{\scriptsize 7}$,
A.~Dell'Acqua$^\textrm{\scriptsize 32}$,
L.~Dell'Asta$^\textrm{\scriptsize 24}$,
M.~Dell'Orso$^\textrm{\scriptsize 126a,126b}$,
M.~Della~Pietra$^\textrm{\scriptsize 106a,106b}$,
D.~della~Volpe$^\textrm{\scriptsize 52}$,
M.~Delmastro$^\textrm{\scriptsize 5}$,
C.~Delporte$^\textrm{\scriptsize 119}$,
P.A.~Delsart$^\textrm{\scriptsize 58}$,
D.A.~DeMarco$^\textrm{\scriptsize 161}$,
S.~Demers$^\textrm{\scriptsize 179}$,
M.~Demichev$^\textrm{\scriptsize 68}$,
A.~Demilly$^\textrm{\scriptsize 83}$,
S.P.~Denisov$^\textrm{\scriptsize 132}$,
D.~Denysiuk$^\textrm{\scriptsize 138}$,
D.~Derendarz$^\textrm{\scriptsize 42}$,
J.E.~Derkaoui$^\textrm{\scriptsize 137d}$,
F.~Derue$^\textrm{\scriptsize 83}$,
P.~Dervan$^\textrm{\scriptsize 77}$,
K.~Desch$^\textrm{\scriptsize 23}$,
C.~Deterre$^\textrm{\scriptsize 45}$,
K.~Dette$^\textrm{\scriptsize 46}$,
M.R.~Devesa$^\textrm{\scriptsize 29}$,
P.O.~Deviveiros$^\textrm{\scriptsize 32}$,
A.~Dewhurst$^\textrm{\scriptsize 133}$,
S.~Dhaliwal$^\textrm{\scriptsize 25}$,
F.A.~Di~Bello$^\textrm{\scriptsize 52}$,
A.~Di~Ciaccio$^\textrm{\scriptsize 135a,135b}$,
L.~Di~Ciaccio$^\textrm{\scriptsize 5}$,
W.K.~Di~Clemente$^\textrm{\scriptsize 124}$,
C.~Di~Donato$^\textrm{\scriptsize 106a,106b}$,
A.~Di~Girolamo$^\textrm{\scriptsize 32}$,
B.~Di~Girolamo$^\textrm{\scriptsize 32}$,
B.~Di~Micco$^\textrm{\scriptsize 136a,136b}$,
R.~Di~Nardo$^\textrm{\scriptsize 32}$,
K.F.~Di~Petrillo$^\textrm{\scriptsize 59}$,
A.~Di~Simone$^\textrm{\scriptsize 51}$,
R.~Di~Sipio$^\textrm{\scriptsize 161}$,
D.~Di~Valentino$^\textrm{\scriptsize 31}$,
C.~Diaconu$^\textrm{\scriptsize 88}$,
M.~Diamond$^\textrm{\scriptsize 161}$,
F.A.~Dias$^\textrm{\scriptsize 39}$,
M.A.~Diaz$^\textrm{\scriptsize 34a}$,
E.B.~Diehl$^\textrm{\scriptsize 92}$,
J.~Dietrich$^\textrm{\scriptsize 17}$,
S.~D\'iez~Cornell$^\textrm{\scriptsize 45}$,
A.~Dimitrievska$^\textrm{\scriptsize 14}$,
J.~Dingfelder$^\textrm{\scriptsize 23}$,
P.~Dita$^\textrm{\scriptsize 28b}$,
S.~Dita$^\textrm{\scriptsize 28b}$,
F.~Dittus$^\textrm{\scriptsize 32}$,
F.~Djama$^\textrm{\scriptsize 88}$,
T.~Djobava$^\textrm{\scriptsize 54b}$,
J.I.~Djuvsland$^\textrm{\scriptsize 60a}$,
M.A.B.~do~Vale$^\textrm{\scriptsize 26c}$,
D.~Dobos$^\textrm{\scriptsize 32}$,
M.~Dobre$^\textrm{\scriptsize 28b}$,
C.~Doglioni$^\textrm{\scriptsize 84}$,
J.~Dolejsi$^\textrm{\scriptsize 131}$,
Z.~Dolezal$^\textrm{\scriptsize 131}$,
M.~Donadelli$^\textrm{\scriptsize 26d}$,
S.~Donati$^\textrm{\scriptsize 126a,126b}$,
P.~Dondero$^\textrm{\scriptsize 123a,123b}$,
J.~Donini$^\textrm{\scriptsize 37}$,
J.~Dopke$^\textrm{\scriptsize 133}$,
A.~Doria$^\textrm{\scriptsize 106a}$,
M.T.~Dova$^\textrm{\scriptsize 74}$,
A.T.~Doyle$^\textrm{\scriptsize 56}$,
E.~Drechsler$^\textrm{\scriptsize 57}$,
M.~Dris$^\textrm{\scriptsize 10}$,
Y.~Du$^\textrm{\scriptsize 36b}$,
J.~Duarte-Campderros$^\textrm{\scriptsize 155}$,
A.~Dubreuil$^\textrm{\scriptsize 52}$,
E.~Duchovni$^\textrm{\scriptsize 175}$,
G.~Duckeck$^\textrm{\scriptsize 102}$,
A.~Ducourthial$^\textrm{\scriptsize 83}$,
O.A.~Ducu$^\textrm{\scriptsize 97}$$^{,p}$,
D.~Duda$^\textrm{\scriptsize 109}$,
A.~Dudarev$^\textrm{\scriptsize 32}$,
A.Chr.~Dudder$^\textrm{\scriptsize 86}$,
E.M.~Duffield$^\textrm{\scriptsize 16}$,
L.~Duflot$^\textrm{\scriptsize 119}$,
M.~D\"uhrssen$^\textrm{\scriptsize 32}$,
M.~Dumancic$^\textrm{\scriptsize 175}$,
A.E.~Dumitriu$^\textrm{\scriptsize 28b}$,
A.K.~Duncan$^\textrm{\scriptsize 56}$,
M.~Dunford$^\textrm{\scriptsize 60a}$,
H.~Duran~Yildiz$^\textrm{\scriptsize 4a}$,
M.~D\"uren$^\textrm{\scriptsize 55}$,
A.~Durglishvili$^\textrm{\scriptsize 54b}$,
D.~Duschinger$^\textrm{\scriptsize 47}$,
B.~Dutta$^\textrm{\scriptsize 45}$,
D.~Duvnjak$^\textrm{\scriptsize 1}$,
M.~Dyndal$^\textrm{\scriptsize 45}$,
B.S.~Dziedzic$^\textrm{\scriptsize 42}$,
C.~Eckardt$^\textrm{\scriptsize 45}$,
K.M.~Ecker$^\textrm{\scriptsize 103}$,
R.C.~Edgar$^\textrm{\scriptsize 92}$,
T.~Eifert$^\textrm{\scriptsize 32}$,
G.~Eigen$^\textrm{\scriptsize 15}$,
K.~Einsweiler$^\textrm{\scriptsize 16}$,
T.~Ekelof$^\textrm{\scriptsize 168}$,
M.~El~Kacimi$^\textrm{\scriptsize 137c}$,
R.~El~Kosseifi$^\textrm{\scriptsize 88}$,
V.~Ellajosyula$^\textrm{\scriptsize 88}$,
M.~Ellert$^\textrm{\scriptsize 168}$,
S.~Elles$^\textrm{\scriptsize 5}$,
F.~Ellinghaus$^\textrm{\scriptsize 178}$,
A.A.~Elliot$^\textrm{\scriptsize 172}$,
N.~Ellis$^\textrm{\scriptsize 32}$,
J.~Elmsheuser$^\textrm{\scriptsize 27}$,
M.~Elsing$^\textrm{\scriptsize 32}$,
D.~Emeliyanov$^\textrm{\scriptsize 133}$,
Y.~Enari$^\textrm{\scriptsize 157}$,
O.C.~Endner$^\textrm{\scriptsize 86}$,
J.S.~Ennis$^\textrm{\scriptsize 173}$,
J.~Erdmann$^\textrm{\scriptsize 46}$,
A.~Ereditato$^\textrm{\scriptsize 18}$,
M.~Ernst$^\textrm{\scriptsize 27}$,
S.~Errede$^\textrm{\scriptsize 169}$,
M.~Escalier$^\textrm{\scriptsize 119}$,
C.~Escobar$^\textrm{\scriptsize 170}$,
B.~Esposito$^\textrm{\scriptsize 50}$,
O.~Estrada~Pastor$^\textrm{\scriptsize 170}$,
A.I.~Etienvre$^\textrm{\scriptsize 138}$,
E.~Etzion$^\textrm{\scriptsize 155}$,
H.~Evans$^\textrm{\scriptsize 64}$,
A.~Ezhilov$^\textrm{\scriptsize 125}$,
M.~Ezzi$^\textrm{\scriptsize 137e}$,
F.~Fabbri$^\textrm{\scriptsize 22a,22b}$,
L.~Fabbri$^\textrm{\scriptsize 22a,22b}$,
V.~Fabiani$^\textrm{\scriptsize 108}$,
G.~Facini$^\textrm{\scriptsize 81}$,
R.M.~Fakhrutdinov$^\textrm{\scriptsize 132}$,
S.~Falciano$^\textrm{\scriptsize 134a}$,
R.J.~Falla$^\textrm{\scriptsize 81}$,
J.~Faltova$^\textrm{\scriptsize 32}$,
Y.~Fang$^\textrm{\scriptsize 35a}$,
M.~Fanti$^\textrm{\scriptsize 94a,94b}$,
A.~Farbin$^\textrm{\scriptsize 8}$,
A.~Farilla$^\textrm{\scriptsize 136a}$,
C.~Farina$^\textrm{\scriptsize 127}$,
E.M.~Farina$^\textrm{\scriptsize 123a,123b}$,
T.~Farooque$^\textrm{\scriptsize 93}$,
S.~Farrell$^\textrm{\scriptsize 16}$,
S.M.~Farrington$^\textrm{\scriptsize 173}$,
P.~Farthouat$^\textrm{\scriptsize 32}$,
F.~Fassi$^\textrm{\scriptsize 137e}$,
P.~Fassnacht$^\textrm{\scriptsize 32}$,
D.~Fassouliotis$^\textrm{\scriptsize 9}$,
M.~Faucci~Giannelli$^\textrm{\scriptsize 49}$,
A.~Favareto$^\textrm{\scriptsize 53a,53b}$,
W.J.~Fawcett$^\textrm{\scriptsize 122}$,
L.~Fayard$^\textrm{\scriptsize 119}$,
O.L.~Fedin$^\textrm{\scriptsize 125}$$^{,q}$,
W.~Fedorko$^\textrm{\scriptsize 171}$,
S.~Feigl$^\textrm{\scriptsize 121}$,
L.~Feligioni$^\textrm{\scriptsize 88}$,
C.~Feng$^\textrm{\scriptsize 36b}$,
E.J.~Feng$^\textrm{\scriptsize 32}$,
H.~Feng$^\textrm{\scriptsize 92}$,
M.J.~Fenton$^\textrm{\scriptsize 56}$,
A.B.~Fenyuk$^\textrm{\scriptsize 132}$,
L.~Feremenga$^\textrm{\scriptsize 8}$,
P.~Fernandez~Martinez$^\textrm{\scriptsize 170}$,
S.~Fernandez~Perez$^\textrm{\scriptsize 13}$,
J.~Ferrando$^\textrm{\scriptsize 45}$,
A.~Ferrari$^\textrm{\scriptsize 168}$,
P.~Ferrari$^\textrm{\scriptsize 109}$,
R.~Ferrari$^\textrm{\scriptsize 123a}$,
D.E.~Ferreira~de~Lima$^\textrm{\scriptsize 60b}$,
A.~Ferrer$^\textrm{\scriptsize 170}$,
D.~Ferrere$^\textrm{\scriptsize 52}$,
C.~Ferretti$^\textrm{\scriptsize 92}$,
F.~Fiedler$^\textrm{\scriptsize 86}$,
A.~Filip\v{c}i\v{c}$^\textrm{\scriptsize 78}$,
M.~Filipuzzi$^\textrm{\scriptsize 45}$,
F.~Filthaut$^\textrm{\scriptsize 108}$,
M.~Fincke-Keeler$^\textrm{\scriptsize 172}$,
K.D.~Finelli$^\textrm{\scriptsize 152}$,
M.C.N.~Fiolhais$^\textrm{\scriptsize 128a,128c}$$^{,r}$,
L.~Fiorini$^\textrm{\scriptsize 170}$,
A.~Fischer$^\textrm{\scriptsize 2}$,
C.~Fischer$^\textrm{\scriptsize 13}$,
J.~Fischer$^\textrm{\scriptsize 178}$,
W.C.~Fisher$^\textrm{\scriptsize 93}$,
N.~Flaschel$^\textrm{\scriptsize 45}$,
I.~Fleck$^\textrm{\scriptsize 143}$,
P.~Fleischmann$^\textrm{\scriptsize 92}$,
R.R.M.~Fletcher$^\textrm{\scriptsize 124}$,
T.~Flick$^\textrm{\scriptsize 178}$,
B.M.~Flierl$^\textrm{\scriptsize 102}$,
L.R.~Flores~Castillo$^\textrm{\scriptsize 62a}$,
M.J.~Flowerdew$^\textrm{\scriptsize 103}$,
G.T.~Forcolin$^\textrm{\scriptsize 87}$,
A.~Formica$^\textrm{\scriptsize 138}$,
F.A.~F\"orster$^\textrm{\scriptsize 13}$,
A.~Forti$^\textrm{\scriptsize 87}$,
A.G.~Foster$^\textrm{\scriptsize 19}$,
D.~Fournier$^\textrm{\scriptsize 119}$,
H.~Fox$^\textrm{\scriptsize 75}$,
S.~Fracchia$^\textrm{\scriptsize 141}$,
P.~Francavilla$^\textrm{\scriptsize 83}$,
M.~Franchini$^\textrm{\scriptsize 22a,22b}$,
S.~Franchino$^\textrm{\scriptsize 60a}$,
D.~Francis$^\textrm{\scriptsize 32}$,
L.~Franconi$^\textrm{\scriptsize 121}$,
M.~Franklin$^\textrm{\scriptsize 59}$,
M.~Frate$^\textrm{\scriptsize 166}$,
M.~Fraternali$^\textrm{\scriptsize 123a,123b}$,
D.~Freeborn$^\textrm{\scriptsize 81}$,
S.M.~Fressard-Batraneanu$^\textrm{\scriptsize 32}$,
B.~Freund$^\textrm{\scriptsize 97}$,
D.~Froidevaux$^\textrm{\scriptsize 32}$,
J.A.~Frost$^\textrm{\scriptsize 122}$,
C.~Fukunaga$^\textrm{\scriptsize 158}$,
T.~Fusayasu$^\textrm{\scriptsize 104}$,
J.~Fuster$^\textrm{\scriptsize 170}$,
C.~Gabaldon$^\textrm{\scriptsize 58}$,
O.~Gabizon$^\textrm{\scriptsize 154}$,
A.~Gabrielli$^\textrm{\scriptsize 22a,22b}$,
A.~Gabrielli$^\textrm{\scriptsize 16}$,
G.P.~Gach$^\textrm{\scriptsize 41a}$,
S.~Gadatsch$^\textrm{\scriptsize 32}$,
S.~Gadomski$^\textrm{\scriptsize 80}$,
G.~Gagliardi$^\textrm{\scriptsize 53a,53b}$,
L.G.~Gagnon$^\textrm{\scriptsize 97}$,
C.~Galea$^\textrm{\scriptsize 108}$,
B.~Galhardo$^\textrm{\scriptsize 128a,128c}$,
E.J.~Gallas$^\textrm{\scriptsize 122}$,
B.J.~Gallop$^\textrm{\scriptsize 133}$,
P.~Gallus$^\textrm{\scriptsize 130}$,
G.~Galster$^\textrm{\scriptsize 39}$,
K.K.~Gan$^\textrm{\scriptsize 113}$,
S.~Ganguly$^\textrm{\scriptsize 37}$,
Y.~Gao$^\textrm{\scriptsize 77}$,
Y.S.~Gao$^\textrm{\scriptsize 145}$$^{,g}$,
F.M.~Garay~Walls$^\textrm{\scriptsize 49}$,
C.~Garc\'ia$^\textrm{\scriptsize 170}$,
J.E.~Garc\'ia~Navarro$^\textrm{\scriptsize 170}$,
J.A.~Garc\'ia~Pascual$^\textrm{\scriptsize 35a}$,
M.~Garcia-Sciveres$^\textrm{\scriptsize 16}$,
R.W.~Gardner$^\textrm{\scriptsize 33}$,
N.~Garelli$^\textrm{\scriptsize 145}$,
V.~Garonne$^\textrm{\scriptsize 121}$,
A.~Gascon~Bravo$^\textrm{\scriptsize 45}$,
K.~Gasnikova$^\textrm{\scriptsize 45}$,
C.~Gatti$^\textrm{\scriptsize 50}$,
A.~Gaudiello$^\textrm{\scriptsize 53a,53b}$,
G.~Gaudio$^\textrm{\scriptsize 123a}$,
I.L.~Gavrilenko$^\textrm{\scriptsize 98}$,
C.~Gay$^\textrm{\scriptsize 171}$,
G.~Gaycken$^\textrm{\scriptsize 23}$,
E.N.~Gazis$^\textrm{\scriptsize 10}$,
C.N.P.~Gee$^\textrm{\scriptsize 133}$,
J.~Geisen$^\textrm{\scriptsize 57}$,
M.~Geisen$^\textrm{\scriptsize 86}$,
M.P.~Geisler$^\textrm{\scriptsize 60a}$,
K.~Gellerstedt$^\textrm{\scriptsize 148a,148b}$,
C.~Gemme$^\textrm{\scriptsize 53a}$,
M.H.~Genest$^\textrm{\scriptsize 58}$,
C.~Geng$^\textrm{\scriptsize 92}$,
S.~Gentile$^\textrm{\scriptsize 134a,134b}$,
C.~Gentsos$^\textrm{\scriptsize 156}$,
S.~George$^\textrm{\scriptsize 80}$,
D.~Gerbaudo$^\textrm{\scriptsize 13}$,
A.~Gershon$^\textrm{\scriptsize 155}$,
G.~Ge\ss{}ner$^\textrm{\scriptsize 46}$,
S.~Ghasemi$^\textrm{\scriptsize 143}$,
M.~Ghneimat$^\textrm{\scriptsize 23}$,
B.~Giacobbe$^\textrm{\scriptsize 22a}$,
S.~Giagu$^\textrm{\scriptsize 134a,134b}$,
N.~Giangiacomi$^\textrm{\scriptsize 22a,22b}$,
P.~Giannetti$^\textrm{\scriptsize 126a,126b}$,
S.M.~Gibson$^\textrm{\scriptsize 80}$,
M.~Gignac$^\textrm{\scriptsize 171}$,
M.~Gilchriese$^\textrm{\scriptsize 16}$,
D.~Gillberg$^\textrm{\scriptsize 31}$,
G.~Gilles$^\textrm{\scriptsize 178}$,
D.M.~Gingrich$^\textrm{\scriptsize 3}$$^{,d}$,
N.~Giokaris$^\textrm{\scriptsize 9}$$^{,*}$,
M.P.~Giordani$^\textrm{\scriptsize 167a,167c}$,
F.M.~Giorgi$^\textrm{\scriptsize 22a}$,
P.F.~Giraud$^\textrm{\scriptsize 138}$,
P.~Giromini$^\textrm{\scriptsize 59}$,
G.~Giugliarelli$^\textrm{\scriptsize 167a,167c}$,
D.~Giugni$^\textrm{\scriptsize 94a}$,
F.~Giuli$^\textrm{\scriptsize 122}$,
C.~Giuliani$^\textrm{\scriptsize 103}$,
M.~Giulini$^\textrm{\scriptsize 60b}$,
B.K.~Gjelsten$^\textrm{\scriptsize 121}$,
S.~Gkaitatzis$^\textrm{\scriptsize 156}$,
I.~Gkialas$^\textrm{\scriptsize 9}$$^{,s}$,
E.L.~Gkougkousis$^\textrm{\scriptsize 139}$,
P.~Gkountoumis$^\textrm{\scriptsize 10}$,
L.K.~Gladilin$^\textrm{\scriptsize 101}$,
C.~Glasman$^\textrm{\scriptsize 85}$,
J.~Glatzer$^\textrm{\scriptsize 13}$,
P.C.F.~Glaysher$^\textrm{\scriptsize 45}$,
A.~Glazov$^\textrm{\scriptsize 45}$,
M.~Goblirsch-Kolb$^\textrm{\scriptsize 25}$,
J.~Godlewski$^\textrm{\scriptsize 42}$,
S.~Goldfarb$^\textrm{\scriptsize 91}$,
T.~Golling$^\textrm{\scriptsize 52}$,
D.~Golubkov$^\textrm{\scriptsize 132}$,
A.~Gomes$^\textrm{\scriptsize 128a,128b,128d}$,
R.~Gon\c{c}alo$^\textrm{\scriptsize 128a}$,
R.~Goncalves~Gama$^\textrm{\scriptsize 26a}$,
J.~Goncalves~Pinto~Firmino~Da~Costa$^\textrm{\scriptsize 138}$,
G.~Gonella$^\textrm{\scriptsize 51}$,
L.~Gonella$^\textrm{\scriptsize 19}$,
A.~Gongadze$^\textrm{\scriptsize 68}$,
S.~Gonz\'alez~de~la~Hoz$^\textrm{\scriptsize 170}$,
S.~Gonzalez-Sevilla$^\textrm{\scriptsize 52}$,
L.~Goossens$^\textrm{\scriptsize 32}$,
P.A.~Gorbounov$^\textrm{\scriptsize 99}$,
H.A.~Gordon$^\textrm{\scriptsize 27}$,
I.~Gorelov$^\textrm{\scriptsize 107}$,
B.~Gorini$^\textrm{\scriptsize 32}$,
E.~Gorini$^\textrm{\scriptsize 76a,76b}$,
A.~Gori\v{s}ek$^\textrm{\scriptsize 78}$,
A.T.~Goshaw$^\textrm{\scriptsize 48}$,
C.~G\"ossling$^\textrm{\scriptsize 46}$,
M.I.~Gostkin$^\textrm{\scriptsize 68}$,
C.A.~Gottardo$^\textrm{\scriptsize 23}$,
C.R.~Goudet$^\textrm{\scriptsize 119}$,
D.~Goujdami$^\textrm{\scriptsize 137c}$,
A.G.~Goussiou$^\textrm{\scriptsize 140}$,
N.~Govender$^\textrm{\scriptsize 147b}$$^{,t}$,
E.~Gozani$^\textrm{\scriptsize 154}$,
L.~Graber$^\textrm{\scriptsize 57}$,
I.~Grabowska-Bold$^\textrm{\scriptsize 41a}$,
P.O.J.~Gradin$^\textrm{\scriptsize 168}$,
J.~Gramling$^\textrm{\scriptsize 166}$,
E.~Gramstad$^\textrm{\scriptsize 121}$,
S.~Grancagnolo$^\textrm{\scriptsize 17}$,
V.~Gratchev$^\textrm{\scriptsize 125}$,
P.M.~Gravila$^\textrm{\scriptsize 28f}$,
C.~Gray$^\textrm{\scriptsize 56}$,
H.M.~Gray$^\textrm{\scriptsize 16}$,
Z.D.~Greenwood$^\textrm{\scriptsize 82}$$^{,u}$,
C.~Grefe$^\textrm{\scriptsize 23}$,
K.~Gregersen$^\textrm{\scriptsize 81}$,
I.M.~Gregor$^\textrm{\scriptsize 45}$,
P.~Grenier$^\textrm{\scriptsize 145}$,
K.~Grevtsov$^\textrm{\scriptsize 5}$,
J.~Griffiths$^\textrm{\scriptsize 8}$,
A.A.~Grillo$^\textrm{\scriptsize 139}$,
K.~Grimm$^\textrm{\scriptsize 75}$,
S.~Grinstein$^\textrm{\scriptsize 13}$$^{,v}$,
Ph.~Gris$^\textrm{\scriptsize 37}$,
J.-F.~Grivaz$^\textrm{\scriptsize 119}$,
S.~Groh$^\textrm{\scriptsize 86}$,
E.~Gross$^\textrm{\scriptsize 175}$,
J.~Grosse-Knetter$^\textrm{\scriptsize 57}$,
G.C.~Grossi$^\textrm{\scriptsize 82}$,
Z.J.~Grout$^\textrm{\scriptsize 81}$,
A.~Grummer$^\textrm{\scriptsize 107}$,
L.~Guan$^\textrm{\scriptsize 92}$,
W.~Guan$^\textrm{\scriptsize 176}$,
J.~Guenther$^\textrm{\scriptsize 65}$,
F.~Guescini$^\textrm{\scriptsize 163a}$,
D.~Guest$^\textrm{\scriptsize 166}$,
O.~Gueta$^\textrm{\scriptsize 155}$,
B.~Gui$^\textrm{\scriptsize 113}$,
E.~Guido$^\textrm{\scriptsize 53a,53b}$,
T.~Guillemin$^\textrm{\scriptsize 5}$,
S.~Guindon$^\textrm{\scriptsize 2}$,
U.~Gul$^\textrm{\scriptsize 56}$,
C.~Gumpert$^\textrm{\scriptsize 32}$,
J.~Guo$^\textrm{\scriptsize 36c}$,
W.~Guo$^\textrm{\scriptsize 92}$,
Y.~Guo$^\textrm{\scriptsize 36a}$$^{,w}$,
R.~Gupta$^\textrm{\scriptsize 43}$,
S.~Gupta$^\textrm{\scriptsize 122}$,
G.~Gustavino$^\textrm{\scriptsize 115}$,
P.~Gutierrez$^\textrm{\scriptsize 115}$,
N.G.~Gutierrez~Ortiz$^\textrm{\scriptsize 81}$,
C.~Gutschow$^\textrm{\scriptsize 81}$,
C.~Guyot$^\textrm{\scriptsize 138}$,
M.P.~Guzik$^\textrm{\scriptsize 41a}$,
C.~Gwenlan$^\textrm{\scriptsize 122}$,
C.B.~Gwilliam$^\textrm{\scriptsize 77}$,
A.~Haas$^\textrm{\scriptsize 112}$,
C.~Haber$^\textrm{\scriptsize 16}$,
H.K.~Hadavand$^\textrm{\scriptsize 8}$,
N.~Haddad$^\textrm{\scriptsize 137e}$,
A.~Hadef$^\textrm{\scriptsize 88}$,
S.~Hageb\"ock$^\textrm{\scriptsize 23}$,
M.~Hagihara$^\textrm{\scriptsize 164}$,
H.~Hakobyan$^\textrm{\scriptsize 180}$$^{,*}$,
M.~Haleem$^\textrm{\scriptsize 45}$,
J.~Haley$^\textrm{\scriptsize 116}$,
G.~Halladjian$^\textrm{\scriptsize 93}$,
G.D.~Hallewell$^\textrm{\scriptsize 88}$,
K.~Hamacher$^\textrm{\scriptsize 178}$,
P.~Hamal$^\textrm{\scriptsize 117}$,
K.~Hamano$^\textrm{\scriptsize 172}$,
A.~Hamilton$^\textrm{\scriptsize 147a}$,
G.N.~Hamity$^\textrm{\scriptsize 141}$,
P.G.~Hamnett$^\textrm{\scriptsize 45}$,
L.~Han$^\textrm{\scriptsize 36a}$,
S.~Han$^\textrm{\scriptsize 35a,35d}$,
K.~Hanagaki$^\textrm{\scriptsize 69}$$^{,x}$,
K.~Hanawa$^\textrm{\scriptsize 157}$,
M.~Hance$^\textrm{\scriptsize 139}$,
B.~Haney$^\textrm{\scriptsize 124}$,
P.~Hanke$^\textrm{\scriptsize 60a}$,
J.B.~Hansen$^\textrm{\scriptsize 39}$,
J.D.~Hansen$^\textrm{\scriptsize 39}$,
M.C.~Hansen$^\textrm{\scriptsize 23}$,
P.H.~Hansen$^\textrm{\scriptsize 39}$,
K.~Hara$^\textrm{\scriptsize 164}$,
A.S.~Hard$^\textrm{\scriptsize 176}$,
T.~Harenberg$^\textrm{\scriptsize 178}$,
F.~Hariri$^\textrm{\scriptsize 119}$,
S.~Harkusha$^\textrm{\scriptsize 95}$,
R.D.~Harrington$^\textrm{\scriptsize 49}$,
P.F.~Harrison$^\textrm{\scriptsize 173}$,
N.M.~Hartmann$^\textrm{\scriptsize 102}$,
M.~Hasegawa$^\textrm{\scriptsize 70}$,
Y.~Hasegawa$^\textrm{\scriptsize 142}$,
A.~Hasib$^\textrm{\scriptsize 49}$,
S.~Hassani$^\textrm{\scriptsize 138}$,
S.~Haug$^\textrm{\scriptsize 18}$,
R.~Hauser$^\textrm{\scriptsize 93}$,
L.~Hauswald$^\textrm{\scriptsize 47}$,
L.B.~Havener$^\textrm{\scriptsize 38}$,
M.~Havranek$^\textrm{\scriptsize 130}$,
C.M.~Hawkes$^\textrm{\scriptsize 19}$,
R.J.~Hawkings$^\textrm{\scriptsize 32}$,
D.~Hayakawa$^\textrm{\scriptsize 159}$,
D.~Hayden$^\textrm{\scriptsize 93}$,
C.P.~Hays$^\textrm{\scriptsize 122}$,
J.M.~Hays$^\textrm{\scriptsize 79}$,
H.S.~Hayward$^\textrm{\scriptsize 77}$,
S.J.~Haywood$^\textrm{\scriptsize 133}$,
S.J.~Head$^\textrm{\scriptsize 19}$,
T.~Heck$^\textrm{\scriptsize 86}$,
V.~Hedberg$^\textrm{\scriptsize 84}$,
L.~Heelan$^\textrm{\scriptsize 8}$,
S.~Heer$^\textrm{\scriptsize 23}$,
K.K.~Heidegger$^\textrm{\scriptsize 51}$,
S.~Heim$^\textrm{\scriptsize 45}$,
T.~Heim$^\textrm{\scriptsize 16}$,
B.~Heinemann$^\textrm{\scriptsize 45}$$^{,y}$,
J.J.~Heinrich$^\textrm{\scriptsize 102}$,
L.~Heinrich$^\textrm{\scriptsize 112}$,
C.~Heinz$^\textrm{\scriptsize 55}$,
J.~Hejbal$^\textrm{\scriptsize 129}$,
L.~Helary$^\textrm{\scriptsize 32}$,
A.~Held$^\textrm{\scriptsize 171}$,
S.~Hellman$^\textrm{\scriptsize 148a,148b}$,
C.~Helsens$^\textrm{\scriptsize 32}$,
R.C.W.~Henderson$^\textrm{\scriptsize 75}$,
Y.~Heng$^\textrm{\scriptsize 176}$,
S.~Henkelmann$^\textrm{\scriptsize 171}$,
A.M.~Henriques~Correia$^\textrm{\scriptsize 32}$,
S.~Henrot-Versille$^\textrm{\scriptsize 119}$,
G.H.~Herbert$^\textrm{\scriptsize 17}$,
H.~Herde$^\textrm{\scriptsize 25}$,
V.~Herget$^\textrm{\scriptsize 177}$,
Y.~Hern\'andez~Jim\'enez$^\textrm{\scriptsize 147c}$,
H.~Herr$^\textrm{\scriptsize 86}$,
G.~Herten$^\textrm{\scriptsize 51}$,
R.~Hertenberger$^\textrm{\scriptsize 102}$,
L.~Hervas$^\textrm{\scriptsize 32}$,
T.C.~Herwig$^\textrm{\scriptsize 124}$,
G.G.~Hesketh$^\textrm{\scriptsize 81}$,
N.P.~Hessey$^\textrm{\scriptsize 163a}$,
J.W.~Hetherly$^\textrm{\scriptsize 43}$,
S.~Higashino$^\textrm{\scriptsize 69}$,
E.~Hig\'on-Rodriguez$^\textrm{\scriptsize 170}$,
K.~Hildebrand$^\textrm{\scriptsize 33}$,
E.~Hill$^\textrm{\scriptsize 172}$,
J.C.~Hill$^\textrm{\scriptsize 30}$,
K.H.~Hiller$^\textrm{\scriptsize 45}$,
S.J.~Hillier$^\textrm{\scriptsize 19}$,
M.~Hils$^\textrm{\scriptsize 47}$,
I.~Hinchliffe$^\textrm{\scriptsize 16}$,
M.~Hirose$^\textrm{\scriptsize 51}$,
D.~Hirschbuehl$^\textrm{\scriptsize 178}$,
B.~Hiti$^\textrm{\scriptsize 78}$,
O.~Hladik$^\textrm{\scriptsize 129}$,
X.~Hoad$^\textrm{\scriptsize 49}$,
J.~Hobbs$^\textrm{\scriptsize 150}$,
N.~Hod$^\textrm{\scriptsize 163a}$,
M.C.~Hodgkinson$^\textrm{\scriptsize 141}$,
P.~Hodgson$^\textrm{\scriptsize 141}$,
A.~Hoecker$^\textrm{\scriptsize 32}$,
M.R.~Hoeferkamp$^\textrm{\scriptsize 107}$,
F.~Hoenig$^\textrm{\scriptsize 102}$,
D.~Hohn$^\textrm{\scriptsize 23}$,
T.R.~Holmes$^\textrm{\scriptsize 33}$,
M.~Homann$^\textrm{\scriptsize 46}$,
S.~Honda$^\textrm{\scriptsize 164}$,
T.~Honda$^\textrm{\scriptsize 69}$,
T.M.~Hong$^\textrm{\scriptsize 127}$,
B.H.~Hooberman$^\textrm{\scriptsize 169}$,
W.H.~Hopkins$^\textrm{\scriptsize 118}$,
Y.~Horii$^\textrm{\scriptsize 105}$,
A.J.~Horton$^\textrm{\scriptsize 144}$,
J-Y.~Hostachy$^\textrm{\scriptsize 58}$,
S.~Hou$^\textrm{\scriptsize 153}$,
A.~Hoummada$^\textrm{\scriptsize 137a}$,
J.~Howarth$^\textrm{\scriptsize 87}$,
J.~Hoya$^\textrm{\scriptsize 74}$,
M.~Hrabovsky$^\textrm{\scriptsize 117}$,
J.~Hrdinka$^\textrm{\scriptsize 32}$,
I.~Hristova$^\textrm{\scriptsize 17}$,
J.~Hrivnac$^\textrm{\scriptsize 119}$,
T.~Hryn'ova$^\textrm{\scriptsize 5}$,
A.~Hrynevich$^\textrm{\scriptsize 96}$,
P.J.~Hsu$^\textrm{\scriptsize 63}$,
S.-C.~Hsu$^\textrm{\scriptsize 140}$,
Q.~Hu$^\textrm{\scriptsize 36a}$,
S.~Hu$^\textrm{\scriptsize 36c}$,
Y.~Huang$^\textrm{\scriptsize 35a}$,
Z.~Hubacek$^\textrm{\scriptsize 130}$,
F.~Hubaut$^\textrm{\scriptsize 88}$,
F.~Huegging$^\textrm{\scriptsize 23}$,
T.B.~Huffman$^\textrm{\scriptsize 122}$,
E.W.~Hughes$^\textrm{\scriptsize 38}$,
G.~Hughes$^\textrm{\scriptsize 75}$,
M.~Huhtinen$^\textrm{\scriptsize 32}$,
P.~Huo$^\textrm{\scriptsize 150}$,
N.~Huseynov$^\textrm{\scriptsize 68}$$^{,b}$,
J.~Huston$^\textrm{\scriptsize 93}$,
J.~Huth$^\textrm{\scriptsize 59}$,
G.~Iacobucci$^\textrm{\scriptsize 52}$,
G.~Iakovidis$^\textrm{\scriptsize 27}$,
I.~Ibragimov$^\textrm{\scriptsize 143}$,
L.~Iconomidou-Fayard$^\textrm{\scriptsize 119}$,
Z.~Idrissi$^\textrm{\scriptsize 137e}$,
P.~Iengo$^\textrm{\scriptsize 32}$,
O.~Igonkina$^\textrm{\scriptsize 109}$$^{,z}$,
T.~Iizawa$^\textrm{\scriptsize 174}$,
Y.~Ikegami$^\textrm{\scriptsize 69}$,
M.~Ikeno$^\textrm{\scriptsize 69}$,
Y.~Ilchenko$^\textrm{\scriptsize 11}$$^{,aa}$,
D.~Iliadis$^\textrm{\scriptsize 156}$,
N.~Ilic$^\textrm{\scriptsize 145}$,
G.~Introzzi$^\textrm{\scriptsize 123a,123b}$,
P.~Ioannou$^\textrm{\scriptsize 9}$$^{,*}$,
M.~Iodice$^\textrm{\scriptsize 136a}$,
K.~Iordanidou$^\textrm{\scriptsize 38}$,
V.~Ippolito$^\textrm{\scriptsize 59}$,
M.F.~Isacson$^\textrm{\scriptsize 168}$,
N.~Ishijima$^\textrm{\scriptsize 120}$,
M.~Ishino$^\textrm{\scriptsize 157}$,
M.~Ishitsuka$^\textrm{\scriptsize 159}$,
C.~Issever$^\textrm{\scriptsize 122}$,
S.~Istin$^\textrm{\scriptsize 20a}$,
F.~Ito$^\textrm{\scriptsize 164}$,
J.M.~Iturbe~Ponce$^\textrm{\scriptsize 62a}$,
R.~Iuppa$^\textrm{\scriptsize 162a,162b}$,
H.~Iwasaki$^\textrm{\scriptsize 69}$,
J.M.~Izen$^\textrm{\scriptsize 44}$,
V.~Izzo$^\textrm{\scriptsize 106a}$,
S.~Jabbar$^\textrm{\scriptsize 3}$,
P.~Jackson$^\textrm{\scriptsize 1}$,
R.M.~Jacobs$^\textrm{\scriptsize 23}$,
V.~Jain$^\textrm{\scriptsize 2}$,
K.B.~Jakobi$^\textrm{\scriptsize 86}$,
K.~Jakobs$^\textrm{\scriptsize 51}$,
S.~Jakobsen$^\textrm{\scriptsize 65}$,
T.~Jakoubek$^\textrm{\scriptsize 129}$,
D.O.~Jamin$^\textrm{\scriptsize 116}$,
D.K.~Jana$^\textrm{\scriptsize 82}$,
R.~Jansky$^\textrm{\scriptsize 52}$,
J.~Janssen$^\textrm{\scriptsize 23}$,
M.~Janus$^\textrm{\scriptsize 57}$,
P.A.~Janus$^\textrm{\scriptsize 41a}$,
G.~Jarlskog$^\textrm{\scriptsize 84}$,
N.~Javadov$^\textrm{\scriptsize 68}$$^{,b}$,
T.~Jav\r{u}rek$^\textrm{\scriptsize 51}$,
M.~Javurkova$^\textrm{\scriptsize 51}$,
F.~Jeanneau$^\textrm{\scriptsize 138}$,
L.~Jeanty$^\textrm{\scriptsize 16}$,
J.~Jejelava$^\textrm{\scriptsize 54a}$$^{,ab}$,
A.~Jelinskas$^\textrm{\scriptsize 173}$,
P.~Jenni$^\textrm{\scriptsize 51}$$^{,ac}$,
C.~Jeske$^\textrm{\scriptsize 173}$,
S.~J\'ez\'equel$^\textrm{\scriptsize 5}$,
H.~Ji$^\textrm{\scriptsize 176}$,
J.~Jia$^\textrm{\scriptsize 150}$,
H.~Jiang$^\textrm{\scriptsize 67}$,
Y.~Jiang$^\textrm{\scriptsize 36a}$,
Z.~Jiang$^\textrm{\scriptsize 145}$,
S.~Jiggins$^\textrm{\scriptsize 81}$,
J.~Jimenez~Pena$^\textrm{\scriptsize 170}$,
S.~Jin$^\textrm{\scriptsize 35a}$,
A.~Jinaru$^\textrm{\scriptsize 28b}$,
O.~Jinnouchi$^\textrm{\scriptsize 159}$,
H.~Jivan$^\textrm{\scriptsize 147c}$,
P.~Johansson$^\textrm{\scriptsize 141}$,
K.A.~Johns$^\textrm{\scriptsize 7}$,
C.A.~Johnson$^\textrm{\scriptsize 64}$,
W.J.~Johnson$^\textrm{\scriptsize 140}$,
K.~Jon-And$^\textrm{\scriptsize 148a,148b}$,
R.W.L.~Jones$^\textrm{\scriptsize 75}$,
S.D.~Jones$^\textrm{\scriptsize 151}$,
S.~Jones$^\textrm{\scriptsize 7}$,
T.J.~Jones$^\textrm{\scriptsize 77}$,
J.~Jongmanns$^\textrm{\scriptsize 60a}$,
P.M.~Jorge$^\textrm{\scriptsize 128a,128b}$,
J.~Jovicevic$^\textrm{\scriptsize 163a}$,
X.~Ju$^\textrm{\scriptsize 176}$,
A.~Juste~Rozas$^\textrm{\scriptsize 13}$$^{,v}$,
M.K.~K\"{o}hler$^\textrm{\scriptsize 175}$,
A.~Kaczmarska$^\textrm{\scriptsize 42}$,
M.~Kado$^\textrm{\scriptsize 119}$,
H.~Kagan$^\textrm{\scriptsize 113}$,
M.~Kagan$^\textrm{\scriptsize 145}$,
S.J.~Kahn$^\textrm{\scriptsize 88}$,
T.~Kaji$^\textrm{\scriptsize 174}$,
E.~Kajomovitz$^\textrm{\scriptsize 48}$,
C.W.~Kalderon$^\textrm{\scriptsize 84}$,
A.~Kaluza$^\textrm{\scriptsize 86}$,
S.~Kama$^\textrm{\scriptsize 43}$,
A.~Kamenshchikov$^\textrm{\scriptsize 132}$,
N.~Kanaya$^\textrm{\scriptsize 157}$,
L.~Kanjir$^\textrm{\scriptsize 78}$,
V.A.~Kantserov$^\textrm{\scriptsize 100}$,
J.~Kanzaki$^\textrm{\scriptsize 69}$,
B.~Kaplan$^\textrm{\scriptsize 112}$,
L.S.~Kaplan$^\textrm{\scriptsize 176}$,
D.~Kar$^\textrm{\scriptsize 147c}$,
K.~Karakostas$^\textrm{\scriptsize 10}$,
N.~Karastathis$^\textrm{\scriptsize 10}$,
M.J.~Kareem$^\textrm{\scriptsize 57}$,
E.~Karentzos$^\textrm{\scriptsize 10}$,
S.N.~Karpov$^\textrm{\scriptsize 68}$,
Z.M.~Karpova$^\textrm{\scriptsize 68}$,
K.~Karthik$^\textrm{\scriptsize 112}$,
V.~Kartvelishvili$^\textrm{\scriptsize 75}$,
A.N.~Karyukhin$^\textrm{\scriptsize 132}$,
K.~Kasahara$^\textrm{\scriptsize 164}$,
L.~Kashif$^\textrm{\scriptsize 176}$,
R.D.~Kass$^\textrm{\scriptsize 113}$,
A.~Kastanas$^\textrm{\scriptsize 149}$,
Y.~Kataoka$^\textrm{\scriptsize 157}$,
C.~Kato$^\textrm{\scriptsize 157}$,
A.~Katre$^\textrm{\scriptsize 52}$,
J.~Katzy$^\textrm{\scriptsize 45}$,
K.~Kawade$^\textrm{\scriptsize 70}$,
K.~Kawagoe$^\textrm{\scriptsize 73}$,
T.~Kawamoto$^\textrm{\scriptsize 157}$,
G.~Kawamura$^\textrm{\scriptsize 57}$,
E.F.~Kay$^\textrm{\scriptsize 77}$,
V.F.~Kazanin$^\textrm{\scriptsize 111}$$^{,c}$,
R.~Keeler$^\textrm{\scriptsize 172}$,
R.~Kehoe$^\textrm{\scriptsize 43}$,
J.S.~Keller$^\textrm{\scriptsize 31}$,
E.~Kellermann$^\textrm{\scriptsize 84}$,
J.J.~Kempster$^\textrm{\scriptsize 80}$,
J~Kendrick$^\textrm{\scriptsize 19}$,
H.~Keoshkerian$^\textrm{\scriptsize 161}$,
O.~Kepka$^\textrm{\scriptsize 129}$,
B.P.~Ker\v{s}evan$^\textrm{\scriptsize 78}$,
S.~Kersten$^\textrm{\scriptsize 178}$,
R.A.~Keyes$^\textrm{\scriptsize 90}$,
M.~Khader$^\textrm{\scriptsize 169}$,
F.~Khalil-zada$^\textrm{\scriptsize 12}$,
A.~Khanov$^\textrm{\scriptsize 116}$,
A.G.~Kharlamov$^\textrm{\scriptsize 111}$$^{,c}$,
T.~Kharlamova$^\textrm{\scriptsize 111}$$^{,c}$,
A.~Khodinov$^\textrm{\scriptsize 160}$,
T.J.~Khoo$^\textrm{\scriptsize 52}$,
V.~Khovanskiy$^\textrm{\scriptsize 99}$$^{,*}$,
E.~Khramov$^\textrm{\scriptsize 68}$,
J.~Khubua$^\textrm{\scriptsize 54b}$$^{,ad}$,
S.~Kido$^\textrm{\scriptsize 70}$,
C.R.~Kilby$^\textrm{\scriptsize 80}$,
H.Y.~Kim$^\textrm{\scriptsize 8}$,
S.H.~Kim$^\textrm{\scriptsize 164}$,
Y.K.~Kim$^\textrm{\scriptsize 33}$,
N.~Kimura$^\textrm{\scriptsize 156}$,
O.M.~Kind$^\textrm{\scriptsize 17}$,
B.T.~King$^\textrm{\scriptsize 77}$,
D.~Kirchmeier$^\textrm{\scriptsize 47}$,
J.~Kirk$^\textrm{\scriptsize 133}$,
A.E.~Kiryunin$^\textrm{\scriptsize 103}$,
T.~Kishimoto$^\textrm{\scriptsize 157}$,
D.~Kisielewska$^\textrm{\scriptsize 41a}$,
V.~Kitali$^\textrm{\scriptsize 45}$,
K.~Kiuchi$^\textrm{\scriptsize 164}$,
O.~Kivernyk$^\textrm{\scriptsize 5}$,
E.~Kladiva$^\textrm{\scriptsize 146b}$,
T.~Klapdor-Kleingrothaus$^\textrm{\scriptsize 51}$,
M.H.~Klein$^\textrm{\scriptsize 92}$,
M.~Klein$^\textrm{\scriptsize 77}$,
U.~Klein$^\textrm{\scriptsize 77}$,
K.~Kleinknecht$^\textrm{\scriptsize 86}$,
P.~Klimek$^\textrm{\scriptsize 110}$,
A.~Klimentov$^\textrm{\scriptsize 27}$,
R.~Klingenberg$^\textrm{\scriptsize 46}$,
T.~Klingl$^\textrm{\scriptsize 23}$,
T.~Klioutchnikova$^\textrm{\scriptsize 32}$,
E.-E.~Kluge$^\textrm{\scriptsize 60a}$,
P.~Kluit$^\textrm{\scriptsize 109}$,
S.~Kluth$^\textrm{\scriptsize 103}$,
E.~Kneringer$^\textrm{\scriptsize 65}$,
E.B.F.G.~Knoops$^\textrm{\scriptsize 88}$,
A.~Knue$^\textrm{\scriptsize 103}$,
A.~Kobayashi$^\textrm{\scriptsize 157}$,
D.~Kobayashi$^\textrm{\scriptsize 159}$,
T.~Kobayashi$^\textrm{\scriptsize 157}$,
M.~Kobel$^\textrm{\scriptsize 47}$,
M.~Kocian$^\textrm{\scriptsize 145}$,
P.~Kodys$^\textrm{\scriptsize 131}$,
T.~Koffas$^\textrm{\scriptsize 31}$,
E.~Koffeman$^\textrm{\scriptsize 109}$,
N.M.~K\"ohler$^\textrm{\scriptsize 103}$,
T.~Koi$^\textrm{\scriptsize 145}$,
M.~Kolb$^\textrm{\scriptsize 60b}$,
I.~Koletsou$^\textrm{\scriptsize 5}$,
A.A.~Komar$^\textrm{\scriptsize 98}$$^{,*}$,
Y.~Komori$^\textrm{\scriptsize 157}$,
T.~Kondo$^\textrm{\scriptsize 69}$,
N.~Kondrashova$^\textrm{\scriptsize 36c}$,
K.~K\"oneke$^\textrm{\scriptsize 51}$,
A.C.~K\"onig$^\textrm{\scriptsize 108}$,
T.~Kono$^\textrm{\scriptsize 69}$$^{,ae}$,
R.~Konoplich$^\textrm{\scriptsize 112}$$^{,af}$,
N.~Konstantinidis$^\textrm{\scriptsize 81}$,
R.~Kopeliansky$^\textrm{\scriptsize 64}$,
S.~Koperny$^\textrm{\scriptsize 41a}$,
A.K.~Kopp$^\textrm{\scriptsize 51}$,
K.~Korcyl$^\textrm{\scriptsize 42}$,
K.~Kordas$^\textrm{\scriptsize 156}$,
A.~Korn$^\textrm{\scriptsize 81}$,
A.A.~Korol$^\textrm{\scriptsize 111}$$^{,c}$,
I.~Korolkov$^\textrm{\scriptsize 13}$,
E.V.~Korolkova$^\textrm{\scriptsize 141}$,
O.~Kortner$^\textrm{\scriptsize 103}$,
S.~Kortner$^\textrm{\scriptsize 103}$,
T.~Kosek$^\textrm{\scriptsize 131}$,
V.V.~Kostyukhin$^\textrm{\scriptsize 23}$,
A.~Kotwal$^\textrm{\scriptsize 48}$,
A.~Koulouris$^\textrm{\scriptsize 10}$,
A.~Kourkoumeli-Charalampidi$^\textrm{\scriptsize 123a,123b}$,
C.~Kourkoumelis$^\textrm{\scriptsize 9}$,
E.~Kourlitis$^\textrm{\scriptsize 141}$,
V.~Kouskoura$^\textrm{\scriptsize 27}$,
A.B.~Kowalewska$^\textrm{\scriptsize 42}$,
R.~Kowalewski$^\textrm{\scriptsize 172}$,
T.Z.~Kowalski$^\textrm{\scriptsize 41a}$,
C.~Kozakai$^\textrm{\scriptsize 157}$,
W.~Kozanecki$^\textrm{\scriptsize 138}$,
A.S.~Kozhin$^\textrm{\scriptsize 132}$,
V.A.~Kramarenko$^\textrm{\scriptsize 101}$,
G.~Kramberger$^\textrm{\scriptsize 78}$,
D.~Krasnopevtsev$^\textrm{\scriptsize 100}$,
M.W.~Krasny$^\textrm{\scriptsize 83}$,
A.~Krasznahorkay$^\textrm{\scriptsize 32}$,
D.~Krauss$^\textrm{\scriptsize 103}$,
J.A.~Kremer$^\textrm{\scriptsize 41a}$,
J.~Kretzschmar$^\textrm{\scriptsize 77}$,
K.~Kreutzfeldt$^\textrm{\scriptsize 55}$,
P.~Krieger$^\textrm{\scriptsize 161}$,
K.~Krizka$^\textrm{\scriptsize 33}$,
K.~Kroeninger$^\textrm{\scriptsize 46}$,
H.~Kroha$^\textrm{\scriptsize 103}$,
J.~Kroll$^\textrm{\scriptsize 129}$,
J.~Kroll$^\textrm{\scriptsize 124}$,
J.~Kroseberg$^\textrm{\scriptsize 23}$,
J.~Krstic$^\textrm{\scriptsize 14}$,
U.~Kruchonak$^\textrm{\scriptsize 68}$,
H.~Kr\"uger$^\textrm{\scriptsize 23}$,
N.~Krumnack$^\textrm{\scriptsize 67}$,
M.C.~Kruse$^\textrm{\scriptsize 48}$,
T.~Kubota$^\textrm{\scriptsize 91}$,
H.~Kucuk$^\textrm{\scriptsize 81}$,
S.~Kuday$^\textrm{\scriptsize 4b}$,
J.T.~Kuechler$^\textrm{\scriptsize 178}$,
S.~Kuehn$^\textrm{\scriptsize 32}$,
A.~Kugel$^\textrm{\scriptsize 60a}$,
F.~Kuger$^\textrm{\scriptsize 177}$,
T.~Kuhl$^\textrm{\scriptsize 45}$,
V.~Kukhtin$^\textrm{\scriptsize 68}$,
R.~Kukla$^\textrm{\scriptsize 88}$,
Y.~Kulchitsky$^\textrm{\scriptsize 95}$,
S.~Kuleshov$^\textrm{\scriptsize 34b}$,
Y.P.~Kulinich$^\textrm{\scriptsize 169}$,
M.~Kuna$^\textrm{\scriptsize 134a,134b}$,
T.~Kunigo$^\textrm{\scriptsize 71}$,
A.~Kupco$^\textrm{\scriptsize 129}$,
T.~Kupfer$^\textrm{\scriptsize 46}$,
O.~Kuprash$^\textrm{\scriptsize 155}$,
H.~Kurashige$^\textrm{\scriptsize 70}$,
L.L.~Kurchaninov$^\textrm{\scriptsize 163a}$,
Y.A.~Kurochkin$^\textrm{\scriptsize 95}$,
M.G.~Kurth$^\textrm{\scriptsize 35a,35d}$,
V.~Kus$^\textrm{\scriptsize 129}$,
E.S.~Kuwertz$^\textrm{\scriptsize 172}$,
M.~Kuze$^\textrm{\scriptsize 159}$,
J.~Kvita$^\textrm{\scriptsize 117}$,
T.~Kwan$^\textrm{\scriptsize 172}$,
D.~Kyriazopoulos$^\textrm{\scriptsize 141}$,
A.~La~Rosa$^\textrm{\scriptsize 103}$,
J.L.~La~Rosa~Navarro$^\textrm{\scriptsize 26d}$,
L.~La~Rotonda$^\textrm{\scriptsize 40a,40b}$,
F.~La~Ruffa$^\textrm{\scriptsize 40a,40b}$,
C.~Lacasta$^\textrm{\scriptsize 170}$,
F.~Lacava$^\textrm{\scriptsize 134a,134b}$,
J.~Lacey$^\textrm{\scriptsize 45}$,
D.P.J.~Lack$^\textrm{\scriptsize 87}$,
H.~Lacker$^\textrm{\scriptsize 17}$,
D.~Lacour$^\textrm{\scriptsize 83}$,
E.~Ladygin$^\textrm{\scriptsize 68}$,
R.~Lafaye$^\textrm{\scriptsize 5}$,
B.~Laforge$^\textrm{\scriptsize 83}$,
T.~Lagouri$^\textrm{\scriptsize 179}$,
S.~Lai$^\textrm{\scriptsize 57}$,
S.~Lammers$^\textrm{\scriptsize 64}$,
W.~Lampl$^\textrm{\scriptsize 7}$,
E.~Lan\c{c}on$^\textrm{\scriptsize 27}$,
U.~Landgraf$^\textrm{\scriptsize 51}$,
M.P.J.~Landon$^\textrm{\scriptsize 79}$,
M.C.~Lanfermann$^\textrm{\scriptsize 52}$,
V.S.~Lang$^\textrm{\scriptsize 60a}$,
J.C.~Lange$^\textrm{\scriptsize 13}$,
R.J.~Langenberg$^\textrm{\scriptsize 32}$,
A.J.~Lankford$^\textrm{\scriptsize 166}$,
F.~Lanni$^\textrm{\scriptsize 27}$,
K.~Lantzsch$^\textrm{\scriptsize 23}$,
A.~Lanza$^\textrm{\scriptsize 123a}$,
A.~Lapertosa$^\textrm{\scriptsize 53a,53b}$,
S.~Laplace$^\textrm{\scriptsize 83}$,
J.F.~Laporte$^\textrm{\scriptsize 138}$,
T.~Lari$^\textrm{\scriptsize 94a}$,
F.~Lasagni~Manghi$^\textrm{\scriptsize 22a,22b}$,
M.~Lassnig$^\textrm{\scriptsize 32}$,
P.~Laurelli$^\textrm{\scriptsize 50}$,
W.~Lavrijsen$^\textrm{\scriptsize 16}$,
A.T.~Law$^\textrm{\scriptsize 139}$,
P.~Laycock$^\textrm{\scriptsize 77}$,
T.~Lazovich$^\textrm{\scriptsize 59}$,
M.~Lazzaroni$^\textrm{\scriptsize 94a,94b}$,
B.~Le$^\textrm{\scriptsize 91}$,
O.~Le~Dortz$^\textrm{\scriptsize 83}$,
E.~Le~Guirriec$^\textrm{\scriptsize 88}$,
E.P.~Le~Quilleuc$^\textrm{\scriptsize 138}$,
M.~LeBlanc$^\textrm{\scriptsize 172}$,
T.~LeCompte$^\textrm{\scriptsize 6}$,
F.~Ledroit-Guillon$^\textrm{\scriptsize 58}$,
C.A.~Lee$^\textrm{\scriptsize 27}$,
G.R.~Lee$^\textrm{\scriptsize 133}$$^{,ag}$,
S.C.~Lee$^\textrm{\scriptsize 153}$,
L.~Lee$^\textrm{\scriptsize 59}$,
B.~Lefebvre$^\textrm{\scriptsize 90}$,
G.~Lefebvre$^\textrm{\scriptsize 83}$,
M.~Lefebvre$^\textrm{\scriptsize 172}$,
F.~Legger$^\textrm{\scriptsize 102}$,
C.~Leggett$^\textrm{\scriptsize 16}$,
G.~Lehmann~Miotto$^\textrm{\scriptsize 32}$,
X.~Lei$^\textrm{\scriptsize 7}$,
W.A.~Leight$^\textrm{\scriptsize 45}$,
M.A.L.~Leite$^\textrm{\scriptsize 26d}$,
R.~Leitner$^\textrm{\scriptsize 131}$,
D.~Lellouch$^\textrm{\scriptsize 175}$,
B.~Lemmer$^\textrm{\scriptsize 57}$,
K.J.C.~Leney$^\textrm{\scriptsize 81}$,
T.~Lenz$^\textrm{\scriptsize 23}$,
B.~Lenzi$^\textrm{\scriptsize 32}$,
R.~Leone$^\textrm{\scriptsize 7}$,
S.~Leone$^\textrm{\scriptsize 126a,126b}$,
C.~Leonidopoulos$^\textrm{\scriptsize 49}$,
G.~Lerner$^\textrm{\scriptsize 151}$,
C.~Leroy$^\textrm{\scriptsize 97}$,
A.A.J.~Lesage$^\textrm{\scriptsize 138}$,
C.G.~Lester$^\textrm{\scriptsize 30}$,
M.~Levchenko$^\textrm{\scriptsize 125}$,
J.~Lev\^eque$^\textrm{\scriptsize 5}$,
D.~Levin$^\textrm{\scriptsize 92}$,
L.J.~Levinson$^\textrm{\scriptsize 175}$,
M.~Levy$^\textrm{\scriptsize 19}$,
D.~Lewis$^\textrm{\scriptsize 79}$,
B.~Li$^\textrm{\scriptsize 36a}$$^{,w}$,
Changqiao~Li$^\textrm{\scriptsize 36a}$,
H.~Li$^\textrm{\scriptsize 150}$,
L.~Li$^\textrm{\scriptsize 36c}$,
Q.~Li$^\textrm{\scriptsize 35a,35d}$,
Q.~Li$^\textrm{\scriptsize 36a}$,
S.~Li$^\textrm{\scriptsize 48}$,
X.~Li$^\textrm{\scriptsize 36c}$,
Y.~Li$^\textrm{\scriptsize 143}$,
Z.~Liang$^\textrm{\scriptsize 35a}$,
B.~Liberti$^\textrm{\scriptsize 135a}$,
A.~Liblong$^\textrm{\scriptsize 161}$,
K.~Lie$^\textrm{\scriptsize 62c}$,
J.~Liebal$^\textrm{\scriptsize 23}$,
W.~Liebig$^\textrm{\scriptsize 15}$,
A.~Limosani$^\textrm{\scriptsize 152}$,
S.C.~Lin$^\textrm{\scriptsize 182}$,
T.H.~Lin$^\textrm{\scriptsize 86}$,
R.A.~Linck$^\textrm{\scriptsize 64}$,
B.E.~Lindquist$^\textrm{\scriptsize 150}$,
A.E.~Lionti$^\textrm{\scriptsize 52}$,
E.~Lipeles$^\textrm{\scriptsize 124}$,
A.~Lipniacka$^\textrm{\scriptsize 15}$,
M.~Lisovyi$^\textrm{\scriptsize 60b}$,
T.M.~Liss$^\textrm{\scriptsize 169}$$^{,ah}$,
A.~Lister$^\textrm{\scriptsize 171}$,
A.M.~Litke$^\textrm{\scriptsize 139}$,
B.~Liu$^\textrm{\scriptsize 153}$$^{,ai}$,
H.~Liu$^\textrm{\scriptsize 92}$,
H.~Liu$^\textrm{\scriptsize 27}$,
J.K.K.~Liu$^\textrm{\scriptsize 122}$,
J.~Liu$^\textrm{\scriptsize 36b}$,
J.B.~Liu$^\textrm{\scriptsize 36a}$,
K.~Liu$^\textrm{\scriptsize 88}$,
L.~Liu$^\textrm{\scriptsize 169}$,
M.~Liu$^\textrm{\scriptsize 36a}$,
Y.L.~Liu$^\textrm{\scriptsize 36a}$,
Y.~Liu$^\textrm{\scriptsize 36a}$,
M.~Livan$^\textrm{\scriptsize 123a,123b}$,
A.~Lleres$^\textrm{\scriptsize 58}$,
J.~Llorente~Merino$^\textrm{\scriptsize 35a}$,
S.L.~Lloyd$^\textrm{\scriptsize 79}$,
C.Y.~Lo$^\textrm{\scriptsize 62b}$,
F.~Lo~Sterzo$^\textrm{\scriptsize 153}$,
E.M.~Lobodzinska$^\textrm{\scriptsize 45}$,
P.~Loch$^\textrm{\scriptsize 7}$,
F.K.~Loebinger$^\textrm{\scriptsize 87}$,
A.~Loesle$^\textrm{\scriptsize 51}$,
K.M.~Loew$^\textrm{\scriptsize 25}$,
A.~Loginov$^\textrm{\scriptsize 179}$$^{,*}$,
T.~Lohse$^\textrm{\scriptsize 17}$,
K.~Lohwasser$^\textrm{\scriptsize 141}$,
M.~Lokajicek$^\textrm{\scriptsize 129}$,
B.A.~Long$^\textrm{\scriptsize 24}$,
J.D.~Long$^\textrm{\scriptsize 169}$,
R.E.~Long$^\textrm{\scriptsize 75}$,
L.~Longo$^\textrm{\scriptsize 76a,76b}$,
K.A.~Looper$^\textrm{\scriptsize 113}$,
J.A.~Lopez$^\textrm{\scriptsize 34b}$,
D.~Lopez~Mateos$^\textrm{\scriptsize 59}$,
I.~Lopez~Paz$^\textrm{\scriptsize 13}$,
A.~Lopez~Solis$^\textrm{\scriptsize 83}$,
J.~Lorenz$^\textrm{\scriptsize 102}$,
N.~Lorenzo~Martinez$^\textrm{\scriptsize 5}$,
M.~Losada$^\textrm{\scriptsize 21}$,
P.J.~L{\"o}sel$^\textrm{\scriptsize 102}$,
X.~Lou$^\textrm{\scriptsize 35a}$,
A.~Lounis$^\textrm{\scriptsize 119}$,
J.~Love$^\textrm{\scriptsize 6}$,
P.A.~Love$^\textrm{\scriptsize 75}$,
H.~Lu$^\textrm{\scriptsize 62a}$,
N.~Lu$^\textrm{\scriptsize 92}$,
Y.J.~Lu$^\textrm{\scriptsize 63}$,
H.J.~Lubatti$^\textrm{\scriptsize 140}$,
C.~Luci$^\textrm{\scriptsize 134a,134b}$,
A.~Lucotte$^\textrm{\scriptsize 58}$,
C.~Luedtke$^\textrm{\scriptsize 51}$,
F.~Luehring$^\textrm{\scriptsize 64}$,
W.~Lukas$^\textrm{\scriptsize 65}$,
L.~Luminari$^\textrm{\scriptsize 134a}$,
O.~Lundberg$^\textrm{\scriptsize 148a,148b}$,
B.~Lund-Jensen$^\textrm{\scriptsize 149}$,
M.S.~Lutz$^\textrm{\scriptsize 89}$,
P.M.~Luzi$^\textrm{\scriptsize 83}$,
D.~Lynn$^\textrm{\scriptsize 27}$,
R.~Lysak$^\textrm{\scriptsize 129}$,
E.~Lytken$^\textrm{\scriptsize 84}$,
F.~Lyu$^\textrm{\scriptsize 35a}$,
V.~Lyubushkin$^\textrm{\scriptsize 68}$,
H.~Ma$^\textrm{\scriptsize 27}$,
L.L.~Ma$^\textrm{\scriptsize 36b}$,
Y.~Ma$^\textrm{\scriptsize 36b}$,
G.~Maccarrone$^\textrm{\scriptsize 50}$,
A.~Macchiolo$^\textrm{\scriptsize 103}$,
C.M.~Macdonald$^\textrm{\scriptsize 141}$,
B.~Ma\v{c}ek$^\textrm{\scriptsize 78}$,
J.~Machado~Miguens$^\textrm{\scriptsize 124,128b}$,
D.~Madaffari$^\textrm{\scriptsize 170}$,
R.~Madar$^\textrm{\scriptsize 37}$,
W.F.~Mader$^\textrm{\scriptsize 47}$,
A.~Madsen$^\textrm{\scriptsize 45}$,
J.~Maeda$^\textrm{\scriptsize 70}$,
S.~Maeland$^\textrm{\scriptsize 15}$,
T.~Maeno$^\textrm{\scriptsize 27}$,
A.S.~Maevskiy$^\textrm{\scriptsize 101}$,
V.~Magerl$^\textrm{\scriptsize 51}$,
J.~Mahlstedt$^\textrm{\scriptsize 109}$,
C.~Maiani$^\textrm{\scriptsize 119}$,
C.~Maidantchik$^\textrm{\scriptsize 26a}$,
A.A.~Maier$^\textrm{\scriptsize 103}$,
T.~Maier$^\textrm{\scriptsize 102}$,
A.~Maio$^\textrm{\scriptsize 128a,128b,128d}$,
O.~Majersky$^\textrm{\scriptsize 146a}$,
S.~Majewski$^\textrm{\scriptsize 118}$,
Y.~Makida$^\textrm{\scriptsize 69}$,
N.~Makovec$^\textrm{\scriptsize 119}$,
B.~Malaescu$^\textrm{\scriptsize 83}$,
Pa.~Malecki$^\textrm{\scriptsize 42}$,
V.P.~Maleev$^\textrm{\scriptsize 125}$,
F.~Malek$^\textrm{\scriptsize 58}$,
U.~Mallik$^\textrm{\scriptsize 66}$,
D.~Malon$^\textrm{\scriptsize 6}$,
C.~Malone$^\textrm{\scriptsize 30}$,
S.~Maltezos$^\textrm{\scriptsize 10}$,
S.~Malyukov$^\textrm{\scriptsize 32}$,
J.~Mamuzic$^\textrm{\scriptsize 170}$,
G.~Mancini$^\textrm{\scriptsize 50}$,
I.~Mandi\'{c}$^\textrm{\scriptsize 78}$,
J.~Maneira$^\textrm{\scriptsize 128a,128b}$,
L.~Manhaes~de~Andrade~Filho$^\textrm{\scriptsize 26b}$,
J.~Manjarres~Ramos$^\textrm{\scriptsize 47}$,
K.H.~Mankinen$^\textrm{\scriptsize 84}$,
A.~Mann$^\textrm{\scriptsize 102}$,
A.~Manousos$^\textrm{\scriptsize 32}$,
B.~Mansoulie$^\textrm{\scriptsize 138}$,
J.D.~Mansour$^\textrm{\scriptsize 35a}$,
R.~Mantifel$^\textrm{\scriptsize 90}$,
M.~Mantoani$^\textrm{\scriptsize 57}$,
S.~Manzoni$^\textrm{\scriptsize 94a,94b}$,
L.~Mapelli$^\textrm{\scriptsize 32}$,
G.~Marceca$^\textrm{\scriptsize 29}$,
L.~March$^\textrm{\scriptsize 52}$,
L.~Marchese$^\textrm{\scriptsize 122}$,
G.~Marchiori$^\textrm{\scriptsize 83}$,
M.~Marcisovsky$^\textrm{\scriptsize 129}$,
M.~Marjanovic$^\textrm{\scriptsize 37}$,
D.E.~Marley$^\textrm{\scriptsize 92}$,
F.~Marroquim$^\textrm{\scriptsize 26a}$,
S.P.~Marsden$^\textrm{\scriptsize 87}$,
Z.~Marshall$^\textrm{\scriptsize 16}$,
M.U.F~Martensson$^\textrm{\scriptsize 168}$,
S.~Marti-Garcia$^\textrm{\scriptsize 170}$,
C.B.~Martin$^\textrm{\scriptsize 113}$,
T.A.~Martin$^\textrm{\scriptsize 173}$,
V.J.~Martin$^\textrm{\scriptsize 49}$,
B.~Martin~dit~Latour$^\textrm{\scriptsize 15}$,
M.~Martinez$^\textrm{\scriptsize 13}$$^{,v}$,
V.I.~Martinez~Outschoorn$^\textrm{\scriptsize 169}$,
S.~Martin-Haugh$^\textrm{\scriptsize 133}$,
V.S.~Martoiu$^\textrm{\scriptsize 28b}$,
A.C.~Martyniuk$^\textrm{\scriptsize 81}$,
A.~Marzin$^\textrm{\scriptsize 32}$,
L.~Masetti$^\textrm{\scriptsize 86}$,
T.~Mashimo$^\textrm{\scriptsize 157}$,
R.~Mashinistov$^\textrm{\scriptsize 98}$,
J.~Masik$^\textrm{\scriptsize 87}$,
A.L.~Maslennikov$^\textrm{\scriptsize 111}$$^{,c}$,
L.~Massa$^\textrm{\scriptsize 135a,135b}$,
P.~Mastrandrea$^\textrm{\scriptsize 5}$,
A.~Mastroberardino$^\textrm{\scriptsize 40a,40b}$,
T.~Masubuchi$^\textrm{\scriptsize 157}$,
P.~M\"attig$^\textrm{\scriptsize 178}$,
J.~Maurer$^\textrm{\scriptsize 28b}$,
S.J.~Maxfield$^\textrm{\scriptsize 77}$,
D.A.~Maximov$^\textrm{\scriptsize 111}$$^{,c}$,
R.~Mazini$^\textrm{\scriptsize 153}$,
I.~Maznas$^\textrm{\scriptsize 156}$,
S.M.~Mazza$^\textrm{\scriptsize 94a,94b}$,
N.C.~Mc~Fadden$^\textrm{\scriptsize 107}$,
G.~Mc~Goldrick$^\textrm{\scriptsize 161}$,
S.P.~Mc~Kee$^\textrm{\scriptsize 92}$,
A.~McCarn$^\textrm{\scriptsize 92}$,
R.L.~McCarthy$^\textrm{\scriptsize 150}$,
T.G.~McCarthy$^\textrm{\scriptsize 103}$,
L.I.~McClymont$^\textrm{\scriptsize 81}$,
E.F.~McDonald$^\textrm{\scriptsize 91}$,
J.A.~Mcfayden$^\textrm{\scriptsize 81}$,
G.~Mchedlidze$^\textrm{\scriptsize 57}$,
S.J.~McMahon$^\textrm{\scriptsize 133}$,
P.C.~McNamara$^\textrm{\scriptsize 91}$,
R.A.~McPherson$^\textrm{\scriptsize 172}$$^{,o}$,
S.~Meehan$^\textrm{\scriptsize 140}$,
T.J.~Megy$^\textrm{\scriptsize 51}$,
S.~Mehlhase$^\textrm{\scriptsize 102}$,
A.~Mehta$^\textrm{\scriptsize 77}$,
T.~Meideck$^\textrm{\scriptsize 58}$,
K.~Meier$^\textrm{\scriptsize 60a}$,
B.~Meirose$^\textrm{\scriptsize 44}$,
D.~Melini$^\textrm{\scriptsize 170}$$^{,aj}$,
B.R.~Mellado~Garcia$^\textrm{\scriptsize 147c}$,
J.D.~Mellenthin$^\textrm{\scriptsize 57}$,
M.~Melo$^\textrm{\scriptsize 146a}$,
F.~Meloni$^\textrm{\scriptsize 18}$,
A.~Melzer$^\textrm{\scriptsize 23}$,
S.B.~Menary$^\textrm{\scriptsize 87}$,
L.~Meng$^\textrm{\scriptsize 77}$,
X.T.~Meng$^\textrm{\scriptsize 92}$,
A.~Mengarelli$^\textrm{\scriptsize 22a,22b}$,
S.~Menke$^\textrm{\scriptsize 103}$,
E.~Meoni$^\textrm{\scriptsize 40a,40b}$,
S.~Mergelmeyer$^\textrm{\scriptsize 17}$,
P.~Mermod$^\textrm{\scriptsize 52}$,
L.~Merola$^\textrm{\scriptsize 106a,106b}$,
C.~Meroni$^\textrm{\scriptsize 94a}$,
F.S.~Merritt$^\textrm{\scriptsize 33}$,
A.~Messina$^\textrm{\scriptsize 134a,134b}$,
J.~Metcalfe$^\textrm{\scriptsize 6}$,
A.S.~Mete$^\textrm{\scriptsize 166}$,
C.~Meyer$^\textrm{\scriptsize 124}$,
J-P.~Meyer$^\textrm{\scriptsize 138}$,
J.~Meyer$^\textrm{\scriptsize 109}$,
H.~Meyer~Zu~Theenhausen$^\textrm{\scriptsize 60a}$,
F.~Miano$^\textrm{\scriptsize 151}$,
R.P.~Middleton$^\textrm{\scriptsize 133}$,
S.~Miglioranzi$^\textrm{\scriptsize 53a,53b}$,
L.~Mijovi\'{c}$^\textrm{\scriptsize 49}$,
G.~Mikenberg$^\textrm{\scriptsize 175}$,
M.~Mikestikova$^\textrm{\scriptsize 129}$,
M.~Miku\v{z}$^\textrm{\scriptsize 78}$,
M.~Milesi$^\textrm{\scriptsize 91}$,
A.~Milic$^\textrm{\scriptsize 161}$,
D.W.~Miller$^\textrm{\scriptsize 33}$,
C.~Mills$^\textrm{\scriptsize 49}$,
A.~Milov$^\textrm{\scriptsize 175}$,
D.A.~Milstead$^\textrm{\scriptsize 148a,148b}$,
A.A.~Minaenko$^\textrm{\scriptsize 132}$,
Y.~Minami$^\textrm{\scriptsize 157}$,
I.A.~Minashvili$^\textrm{\scriptsize 54b}$,
A.I.~Mincer$^\textrm{\scriptsize 112}$,
B.~Mindur$^\textrm{\scriptsize 41a}$,
M.~Mineev$^\textrm{\scriptsize 68}$,
Y.~Minegishi$^\textrm{\scriptsize 157}$,
Y.~Ming$^\textrm{\scriptsize 176}$,
L.M.~Mir$^\textrm{\scriptsize 13}$,
K.P.~Mistry$^\textrm{\scriptsize 124}$,
T.~Mitani$^\textrm{\scriptsize 174}$,
J.~Mitrevski$^\textrm{\scriptsize 102}$,
V.A.~Mitsou$^\textrm{\scriptsize 170}$,
A.~Miucci$^\textrm{\scriptsize 18}$,
P.S.~Miyagawa$^\textrm{\scriptsize 141}$,
A.~Mizukami$^\textrm{\scriptsize 69}$,
J.U.~Mj\"ornmark$^\textrm{\scriptsize 84}$,
T.~Mkrtchyan$^\textrm{\scriptsize 180}$,
M.~Mlynarikova$^\textrm{\scriptsize 131}$,
T.~Moa$^\textrm{\scriptsize 148a,148b}$,
K.~Mochizuki$^\textrm{\scriptsize 97}$,
P.~Mogg$^\textrm{\scriptsize 51}$,
S.~Mohapatra$^\textrm{\scriptsize 38}$,
S.~Molander$^\textrm{\scriptsize 148a,148b}$,
R.~Moles-Valls$^\textrm{\scriptsize 23}$,
R.~Monden$^\textrm{\scriptsize 71}$,
M.C.~Mondragon$^\textrm{\scriptsize 93}$,
K.~M\"onig$^\textrm{\scriptsize 45}$,
J.~Monk$^\textrm{\scriptsize 39}$,
E.~Monnier$^\textrm{\scriptsize 88}$,
A.~Montalbano$^\textrm{\scriptsize 150}$,
J.~Montejo~Berlingen$^\textrm{\scriptsize 32}$,
F.~Monticelli$^\textrm{\scriptsize 74}$,
S.~Monzani$^\textrm{\scriptsize 94a,94b}$,
R.W.~Moore$^\textrm{\scriptsize 3}$,
N.~Morange$^\textrm{\scriptsize 119}$,
D.~Moreno$^\textrm{\scriptsize 21}$,
M.~Moreno~Ll\'acer$^\textrm{\scriptsize 32}$,
P.~Morettini$^\textrm{\scriptsize 53a}$,
S.~Morgenstern$^\textrm{\scriptsize 32}$,
D.~Mori$^\textrm{\scriptsize 144}$,
T.~Mori$^\textrm{\scriptsize 157}$,
M.~Morii$^\textrm{\scriptsize 59}$,
M.~Morinaga$^\textrm{\scriptsize 157}$,
V.~Morisbak$^\textrm{\scriptsize 121}$,
A.K.~Morley$^\textrm{\scriptsize 32}$,
G.~Mornacchi$^\textrm{\scriptsize 32}$,
J.D.~Morris$^\textrm{\scriptsize 79}$,
L.~Morvaj$^\textrm{\scriptsize 150}$,
P.~Moschovakos$^\textrm{\scriptsize 10}$,
M.~Mosidze$^\textrm{\scriptsize 54b}$,
H.J.~Moss$^\textrm{\scriptsize 141}$,
J.~Moss$^\textrm{\scriptsize 145}$$^{,ak}$,
K.~Motohashi$^\textrm{\scriptsize 159}$,
R.~Mount$^\textrm{\scriptsize 145}$,
E.~Mountricha$^\textrm{\scriptsize 27}$,
E.J.W.~Moyse$^\textrm{\scriptsize 89}$,
S.~Muanza$^\textrm{\scriptsize 88}$,
F.~Mueller$^\textrm{\scriptsize 103}$,
J.~Mueller$^\textrm{\scriptsize 127}$,
R.S.P.~Mueller$^\textrm{\scriptsize 102}$,
D.~Muenstermann$^\textrm{\scriptsize 75}$,
P.~Mullen$^\textrm{\scriptsize 56}$,
G.A.~Mullier$^\textrm{\scriptsize 18}$,
F.J.~Munoz~Sanchez$^\textrm{\scriptsize 87}$,
W.J.~Murray$^\textrm{\scriptsize 173,133}$,
H.~Musheghyan$^\textrm{\scriptsize 32}$,
M.~Mu\v{s}kinja$^\textrm{\scriptsize 78}$,
A.G.~Myagkov$^\textrm{\scriptsize 132}$$^{,al}$,
M.~Myska$^\textrm{\scriptsize 130}$,
B.P.~Nachman$^\textrm{\scriptsize 16}$,
O.~Nackenhorst$^\textrm{\scriptsize 52}$,
K.~Nagai$^\textrm{\scriptsize 122}$,
R.~Nagai$^\textrm{\scriptsize 69}$$^{,ae}$,
K.~Nagano$^\textrm{\scriptsize 69}$,
Y.~Nagasaka$^\textrm{\scriptsize 61}$,
K.~Nagata$^\textrm{\scriptsize 164}$,
M.~Nagel$^\textrm{\scriptsize 51}$,
E.~Nagy$^\textrm{\scriptsize 88}$,
A.M.~Nairz$^\textrm{\scriptsize 32}$,
Y.~Nakahama$^\textrm{\scriptsize 105}$,
K.~Nakamura$^\textrm{\scriptsize 69}$,
T.~Nakamura$^\textrm{\scriptsize 157}$,
I.~Nakano$^\textrm{\scriptsize 114}$,
R.F.~Naranjo~Garcia$^\textrm{\scriptsize 45}$,
R.~Narayan$^\textrm{\scriptsize 11}$,
D.I.~Narrias~Villar$^\textrm{\scriptsize 60a}$,
I.~Naryshkin$^\textrm{\scriptsize 125}$,
T.~Naumann$^\textrm{\scriptsize 45}$,
G.~Navarro$^\textrm{\scriptsize 21}$,
R.~Nayyar$^\textrm{\scriptsize 7}$,
H.A.~Neal$^\textrm{\scriptsize 92}$,
P.Yu.~Nechaeva$^\textrm{\scriptsize 98}$,
T.J.~Neep$^\textrm{\scriptsize 138}$,
A.~Negri$^\textrm{\scriptsize 123a,123b}$,
M.~Negrini$^\textrm{\scriptsize 22a}$,
S.~Nektarijevic$^\textrm{\scriptsize 108}$,
C.~Nellist$^\textrm{\scriptsize 119}$,
A.~Nelson$^\textrm{\scriptsize 166}$,
M.E.~Nelson$^\textrm{\scriptsize 122}$,
S.~Nemecek$^\textrm{\scriptsize 129}$,
P.~Nemethy$^\textrm{\scriptsize 112}$,
M.~Nessi$^\textrm{\scriptsize 32}$$^{,am}$,
M.S.~Neubauer$^\textrm{\scriptsize 169}$,
M.~Neumann$^\textrm{\scriptsize 178}$,
P.R.~Newman$^\textrm{\scriptsize 19}$,
T.Y.~Ng$^\textrm{\scriptsize 62c}$,
T.~Nguyen~Manh$^\textrm{\scriptsize 97}$,
R.B.~Nickerson$^\textrm{\scriptsize 122}$,
R.~Nicolaidou$^\textrm{\scriptsize 138}$,
J.~Nielsen$^\textrm{\scriptsize 139}$,
V.~Nikolaenko$^\textrm{\scriptsize 132}$$^{,al}$,
I.~Nikolic-Audit$^\textrm{\scriptsize 83}$,
K.~Nikolopoulos$^\textrm{\scriptsize 19}$,
J.K.~Nilsen$^\textrm{\scriptsize 121}$,
P.~Nilsson$^\textrm{\scriptsize 27}$,
Y.~Ninomiya$^\textrm{\scriptsize 157}$,
A.~Nisati$^\textrm{\scriptsize 134a}$,
N.~Nishu$^\textrm{\scriptsize 35c}$,
R.~Nisius$^\textrm{\scriptsize 103}$,
I.~Nitsche$^\textrm{\scriptsize 46}$,
T.~Nitta$^\textrm{\scriptsize 174}$,
T.~Nobe$^\textrm{\scriptsize 157}$,
Y.~Noguchi$^\textrm{\scriptsize 71}$,
M.~Nomachi$^\textrm{\scriptsize 120}$,
I.~Nomidis$^\textrm{\scriptsize 31}$,
M.A.~Nomura$^\textrm{\scriptsize 27}$,
T.~Nooney$^\textrm{\scriptsize 79}$,
M.~Nordberg$^\textrm{\scriptsize 32}$,
N.~Norjoharuddeen$^\textrm{\scriptsize 122}$,
O.~Novgorodova$^\textrm{\scriptsize 47}$,
M.~Nozaki$^\textrm{\scriptsize 69}$,
L.~Nozka$^\textrm{\scriptsize 117}$,
K.~Ntekas$^\textrm{\scriptsize 166}$,
E.~Nurse$^\textrm{\scriptsize 81}$,
F.~Nuti$^\textrm{\scriptsize 91}$,
K.~O'connor$^\textrm{\scriptsize 25}$,
D.C.~O'Neil$^\textrm{\scriptsize 144}$,
A.A.~O'Rourke$^\textrm{\scriptsize 45}$,
V.~O'Shea$^\textrm{\scriptsize 56}$,
F.G.~Oakham$^\textrm{\scriptsize 31}$$^{,d}$,
H.~Oberlack$^\textrm{\scriptsize 103}$,
T.~Obermann$^\textrm{\scriptsize 23}$,
J.~Ocariz$^\textrm{\scriptsize 83}$,
A.~Ochi$^\textrm{\scriptsize 70}$,
I.~Ochoa$^\textrm{\scriptsize 38}$,
J.P.~Ochoa-Ricoux$^\textrm{\scriptsize 34a}$,
S.~Oda$^\textrm{\scriptsize 73}$,
S.~Odaka$^\textrm{\scriptsize 69}$,
A.~Oh$^\textrm{\scriptsize 87}$,
S.H.~Oh$^\textrm{\scriptsize 48}$,
C.C.~Ohm$^\textrm{\scriptsize 16}$,
H.~Ohman$^\textrm{\scriptsize 168}$,
H.~Oide$^\textrm{\scriptsize 53a,53b}$,
H.~Okawa$^\textrm{\scriptsize 164}$,
Y.~Okumura$^\textrm{\scriptsize 157}$,
T.~Okuyama$^\textrm{\scriptsize 69}$,
A.~Olariu$^\textrm{\scriptsize 28b}$,
L.F.~Oleiro~Seabra$^\textrm{\scriptsize 128a}$,
S.A.~Olivares~Pino$^\textrm{\scriptsize 34a}$,
D.~Oliveira~Damazio$^\textrm{\scriptsize 27}$,
A.~Olszewski$^\textrm{\scriptsize 42}$,
J.~Olszowska$^\textrm{\scriptsize 42}$,
A.~Onofre$^\textrm{\scriptsize 128a,128e}$,
K.~Onogi$^\textrm{\scriptsize 105}$,
P.U.E.~Onyisi$^\textrm{\scriptsize 11}$$^{,aa}$,
H.~Oppen$^\textrm{\scriptsize 121}$,
M.J.~Oreglia$^\textrm{\scriptsize 33}$,
Y.~Oren$^\textrm{\scriptsize 155}$,
D.~Orestano$^\textrm{\scriptsize 136a,136b}$,
N.~Orlando$^\textrm{\scriptsize 62b}$,
R.S.~Orr$^\textrm{\scriptsize 161}$,
B.~Osculati$^\textrm{\scriptsize 53a,53b}$$^{,*}$,
R.~Ospanov$^\textrm{\scriptsize 36a}$,
G.~Otero~y~Garzon$^\textrm{\scriptsize 29}$,
H.~Otono$^\textrm{\scriptsize 73}$,
M.~Ouchrif$^\textrm{\scriptsize 137d}$,
F.~Ould-Saada$^\textrm{\scriptsize 121}$,
A.~Ouraou$^\textrm{\scriptsize 138}$,
K.P.~Oussoren$^\textrm{\scriptsize 109}$,
Q.~Ouyang$^\textrm{\scriptsize 35a}$,
M.~Owen$^\textrm{\scriptsize 56}$,
R.E.~Owen$^\textrm{\scriptsize 19}$,
V.E.~Ozcan$^\textrm{\scriptsize 20a}$,
N.~Ozturk$^\textrm{\scriptsize 8}$,
K.~Pachal$^\textrm{\scriptsize 144}$,
A.~Pacheco~Pages$^\textrm{\scriptsize 13}$,
L.~Pacheco~Rodriguez$^\textrm{\scriptsize 138}$,
C.~Padilla~Aranda$^\textrm{\scriptsize 13}$,
S.~Pagan~Griso$^\textrm{\scriptsize 16}$,
M.~Paganini$^\textrm{\scriptsize 179}$,
F.~Paige$^\textrm{\scriptsize 27}$,
G.~Palacino$^\textrm{\scriptsize 64}$,
S.~Palazzo$^\textrm{\scriptsize 40a,40b}$,
S.~Palestini$^\textrm{\scriptsize 32}$,
M.~Palka$^\textrm{\scriptsize 41b}$,
D.~Pallin$^\textrm{\scriptsize 37}$,
E.St.~Panagiotopoulou$^\textrm{\scriptsize 10}$,
I.~Panagoulias$^\textrm{\scriptsize 10}$,
C.E.~Pandini$^\textrm{\scriptsize 126a,126b}$,
J.G.~Panduro~Vazquez$^\textrm{\scriptsize 80}$,
P.~Pani$^\textrm{\scriptsize 32}$,
S.~Panitkin$^\textrm{\scriptsize 27}$,
D.~Pantea$^\textrm{\scriptsize 28b}$,
L.~Paolozzi$^\textrm{\scriptsize 52}$,
Th.D.~Papadopoulou$^\textrm{\scriptsize 10}$,
K.~Papageorgiou$^\textrm{\scriptsize 9}$$^{,s}$,
A.~Paramonov$^\textrm{\scriptsize 6}$,
D.~Paredes~Hernandez$^\textrm{\scriptsize 179}$,
A.J.~Parker$^\textrm{\scriptsize 75}$,
M.A.~Parker$^\textrm{\scriptsize 30}$,
K.A.~Parker$^\textrm{\scriptsize 45}$,
F.~Parodi$^\textrm{\scriptsize 53a,53b}$,
J.A.~Parsons$^\textrm{\scriptsize 38}$,
U.~Parzefall$^\textrm{\scriptsize 51}$,
V.R.~Pascuzzi$^\textrm{\scriptsize 161}$,
J.M.~Pasner$^\textrm{\scriptsize 139}$,
E.~Pasqualucci$^\textrm{\scriptsize 134a}$,
S.~Passaggio$^\textrm{\scriptsize 53a}$,
Fr.~Pastore$^\textrm{\scriptsize 80}$,
S.~Pataraia$^\textrm{\scriptsize 86}$,
J.R.~Pater$^\textrm{\scriptsize 87}$,
T.~Pauly$^\textrm{\scriptsize 32}$,
B.~Pearson$^\textrm{\scriptsize 103}$,
S.~Pedraza~Lopez$^\textrm{\scriptsize 170}$,
R.~Pedro$^\textrm{\scriptsize 128a,128b}$,
S.V.~Peleganchuk$^\textrm{\scriptsize 111}$$^{,c}$,
O.~Penc$^\textrm{\scriptsize 129}$,
C.~Peng$^\textrm{\scriptsize 35a,35d}$,
H.~Peng$^\textrm{\scriptsize 36a}$,
J.~Penwell$^\textrm{\scriptsize 64}$,
B.S.~Peralva$^\textrm{\scriptsize 26b}$,
M.M.~Perego$^\textrm{\scriptsize 138}$,
D.V.~Perepelitsa$^\textrm{\scriptsize 27}$,
F.~Peri$^\textrm{\scriptsize 17}$,
L.~Perini$^\textrm{\scriptsize 94a,94b}$,
H.~Pernegger$^\textrm{\scriptsize 32}$,
S.~Perrella$^\textrm{\scriptsize 106a,106b}$,
R.~Peschke$^\textrm{\scriptsize 45}$,
V.D.~Peshekhonov$^\textrm{\scriptsize 68}$$^{,*}$,
K.~Peters$^\textrm{\scriptsize 45}$,
R.F.Y.~Peters$^\textrm{\scriptsize 87}$,
B.A.~Petersen$^\textrm{\scriptsize 32}$,
T.C.~Petersen$^\textrm{\scriptsize 39}$,
E.~Petit$^\textrm{\scriptsize 58}$,
A.~Petridis$^\textrm{\scriptsize 1}$,
C.~Petridou$^\textrm{\scriptsize 156}$,
P.~Petroff$^\textrm{\scriptsize 119}$,
E.~Petrolo$^\textrm{\scriptsize 134a}$,
M.~Petrov$^\textrm{\scriptsize 122}$,
F.~Petrucci$^\textrm{\scriptsize 136a,136b}$,
N.E.~Pettersson$^\textrm{\scriptsize 89}$,
A.~Peyaud$^\textrm{\scriptsize 138}$,
R.~Pezoa$^\textrm{\scriptsize 34b}$,
F.H.~Phillips$^\textrm{\scriptsize 93}$,
P.W.~Phillips$^\textrm{\scriptsize 133}$,
G.~Piacquadio$^\textrm{\scriptsize 150}$,
E.~Pianori$^\textrm{\scriptsize 173}$,
A.~Picazio$^\textrm{\scriptsize 89}$,
E.~Piccaro$^\textrm{\scriptsize 79}$,
M.A.~Pickering$^\textrm{\scriptsize 122}$,
R.~Piegaia$^\textrm{\scriptsize 29}$,
J.E.~Pilcher$^\textrm{\scriptsize 33}$,
A.D.~Pilkington$^\textrm{\scriptsize 87}$,
A.W.J.~Pin$^\textrm{\scriptsize 87}$,
M.~Pinamonti$^\textrm{\scriptsize 135a,135b}$,
J.L.~Pinfold$^\textrm{\scriptsize 3}$,
H.~Pirumov$^\textrm{\scriptsize 45}$,
M.~Pitt$^\textrm{\scriptsize 175}$,
L.~Plazak$^\textrm{\scriptsize 146a}$,
M.-A.~Pleier$^\textrm{\scriptsize 27}$,
V.~Pleskot$^\textrm{\scriptsize 86}$,
E.~Plotnikova$^\textrm{\scriptsize 68}$,
D.~Pluth$^\textrm{\scriptsize 67}$,
P.~Podberezko$^\textrm{\scriptsize 111}$,
R.~Poettgen$^\textrm{\scriptsize 84}$,
R.~Poggi$^\textrm{\scriptsize 123a,123b}$,
L.~Poggioli$^\textrm{\scriptsize 119}$,
I.~Pogrebnyak$^\textrm{\scriptsize 93}$,
D.~Pohl$^\textrm{\scriptsize 23}$,
G.~Polesello$^\textrm{\scriptsize 123a}$,
A.~Poley$^\textrm{\scriptsize 45}$,
A.~Policicchio$^\textrm{\scriptsize 40a,40b}$,
R.~Polifka$^\textrm{\scriptsize 32}$,
A.~Polini$^\textrm{\scriptsize 22a}$,
C.S.~Pollard$^\textrm{\scriptsize 56}$,
V.~Polychronakos$^\textrm{\scriptsize 27}$,
K.~Pomm\`es$^\textrm{\scriptsize 32}$,
D.~Ponomarenko$^\textrm{\scriptsize 100}$,
L.~Pontecorvo$^\textrm{\scriptsize 134a}$,
G.A.~Popeneciu$^\textrm{\scriptsize 28d}$,
S.~Pospisil$^\textrm{\scriptsize 130}$,
K.~Potamianos$^\textrm{\scriptsize 16}$,
I.N.~Potrap$^\textrm{\scriptsize 68}$,
C.J.~Potter$^\textrm{\scriptsize 30}$,
T.~Poulsen$^\textrm{\scriptsize 84}$,
J.~Poveda$^\textrm{\scriptsize 32}$,
M.E.~Pozo~Astigarraga$^\textrm{\scriptsize 32}$,
P.~Pralavorio$^\textrm{\scriptsize 88}$,
A.~Pranko$^\textrm{\scriptsize 16}$,
S.~Prell$^\textrm{\scriptsize 67}$,
D.~Price$^\textrm{\scriptsize 87}$,
M.~Primavera$^\textrm{\scriptsize 76a}$,
S.~Prince$^\textrm{\scriptsize 90}$,
N.~Proklova$^\textrm{\scriptsize 100}$,
K.~Prokofiev$^\textrm{\scriptsize 62c}$,
F.~Prokoshin$^\textrm{\scriptsize 34b}$,
S.~Protopopescu$^\textrm{\scriptsize 27}$,
J.~Proudfoot$^\textrm{\scriptsize 6}$,
M.~Przybycien$^\textrm{\scriptsize 41a}$,
A.~Puri$^\textrm{\scriptsize 169}$,
P.~Puzo$^\textrm{\scriptsize 119}$,
J.~Qian$^\textrm{\scriptsize 92}$,
G.~Qin$^\textrm{\scriptsize 56}$,
Y.~Qin$^\textrm{\scriptsize 87}$,
A.~Quadt$^\textrm{\scriptsize 57}$,
M.~Queitsch-Maitland$^\textrm{\scriptsize 45}$,
D.~Quilty$^\textrm{\scriptsize 56}$,
S.~Raddum$^\textrm{\scriptsize 121}$,
V.~Radeka$^\textrm{\scriptsize 27}$,
V.~Radescu$^\textrm{\scriptsize 122}$,
S.K.~Radhakrishnan$^\textrm{\scriptsize 150}$,
P.~Radloff$^\textrm{\scriptsize 118}$,
P.~Rados$^\textrm{\scriptsize 91}$,
F.~Ragusa$^\textrm{\scriptsize 94a,94b}$,
G.~Rahal$^\textrm{\scriptsize 181}$,
J.A.~Raine$^\textrm{\scriptsize 87}$,
S.~Rajagopalan$^\textrm{\scriptsize 27}$,
C.~Rangel-Smith$^\textrm{\scriptsize 168}$,
T.~Rashid$^\textrm{\scriptsize 119}$,
S.~Raspopov$^\textrm{\scriptsize 5}$,
M.G.~Ratti$^\textrm{\scriptsize 94a,94b}$,
D.M.~Rauch$^\textrm{\scriptsize 45}$,
F.~Rauscher$^\textrm{\scriptsize 102}$,
S.~Rave$^\textrm{\scriptsize 86}$,
I.~Ravinovich$^\textrm{\scriptsize 175}$,
J.H.~Rawling$^\textrm{\scriptsize 87}$,
M.~Raymond$^\textrm{\scriptsize 32}$,
A.L.~Read$^\textrm{\scriptsize 121}$,
N.P.~Readioff$^\textrm{\scriptsize 58}$,
M.~Reale$^\textrm{\scriptsize 76a,76b}$,
D.M.~Rebuzzi$^\textrm{\scriptsize 123a,123b}$,
A.~Redelbach$^\textrm{\scriptsize 177}$,
G.~Redlinger$^\textrm{\scriptsize 27}$,
R.~Reece$^\textrm{\scriptsize 139}$,
R.G.~Reed$^\textrm{\scriptsize 147c}$,
K.~Reeves$^\textrm{\scriptsize 44}$,
L.~Rehnisch$^\textrm{\scriptsize 17}$,
J.~Reichert$^\textrm{\scriptsize 124}$,
A.~Reiss$^\textrm{\scriptsize 86}$,
C.~Rembser$^\textrm{\scriptsize 32}$,
H.~Ren$^\textrm{\scriptsize 35a,35d}$,
M.~Rescigno$^\textrm{\scriptsize 134a}$,
S.~Resconi$^\textrm{\scriptsize 94a}$,
E.D.~Resseguie$^\textrm{\scriptsize 124}$,
S.~Rettie$^\textrm{\scriptsize 171}$,
E.~Reynolds$^\textrm{\scriptsize 19}$,
O.L.~Rezanova$^\textrm{\scriptsize 111}$$^{,c}$,
P.~Reznicek$^\textrm{\scriptsize 131}$,
R.~Rezvani$^\textrm{\scriptsize 97}$,
R.~Richter$^\textrm{\scriptsize 103}$,
S.~Richter$^\textrm{\scriptsize 81}$,
E.~Richter-Was$^\textrm{\scriptsize 41b}$,
O.~Ricken$^\textrm{\scriptsize 23}$,
M.~Ridel$^\textrm{\scriptsize 83}$,
P.~Rieck$^\textrm{\scriptsize 103}$,
C.J.~Riegel$^\textrm{\scriptsize 178}$,
J.~Rieger$^\textrm{\scriptsize 57}$,
O.~Rifki$^\textrm{\scriptsize 115}$,
M.~Rijssenbeek$^\textrm{\scriptsize 150}$,
A.~Rimoldi$^\textrm{\scriptsize 123a,123b}$,
M.~Rimoldi$^\textrm{\scriptsize 18}$,
L.~Rinaldi$^\textrm{\scriptsize 22a}$,
G.~Ripellino$^\textrm{\scriptsize 149}$,
B.~Risti\'{c}$^\textrm{\scriptsize 32}$,
E.~Ritsch$^\textrm{\scriptsize 32}$,
I.~Riu$^\textrm{\scriptsize 13}$,
F.~Rizatdinova$^\textrm{\scriptsize 116}$,
E.~Rizvi$^\textrm{\scriptsize 79}$,
C.~Rizzi$^\textrm{\scriptsize 13}$,
R.T.~Roberts$^\textrm{\scriptsize 87}$,
S.H.~Robertson$^\textrm{\scriptsize 90}$$^{,o}$,
A.~Robichaud-Veronneau$^\textrm{\scriptsize 90}$,
D.~Robinson$^\textrm{\scriptsize 30}$,
J.E.M.~Robinson$^\textrm{\scriptsize 45}$,
A.~Robson$^\textrm{\scriptsize 56}$,
E.~Rocco$^\textrm{\scriptsize 86}$,
C.~Roda$^\textrm{\scriptsize 126a,126b}$,
Y.~Rodina$^\textrm{\scriptsize 88}$$^{,an}$,
S.~Rodriguez~Bosca$^\textrm{\scriptsize 170}$,
A.~Rodriguez~Perez$^\textrm{\scriptsize 13}$,
D.~Rodriguez~Rodriguez$^\textrm{\scriptsize 170}$,
S.~Roe$^\textrm{\scriptsize 32}$,
C.S.~Rogan$^\textrm{\scriptsize 59}$,
O.~R{\o}hne$^\textrm{\scriptsize 121}$,
J.~Roloff$^\textrm{\scriptsize 59}$,
A.~Romaniouk$^\textrm{\scriptsize 100}$,
M.~Romano$^\textrm{\scriptsize 22a,22b}$,
S.M.~Romano~Saez$^\textrm{\scriptsize 37}$,
E.~Romero~Adam$^\textrm{\scriptsize 170}$,
N.~Rompotis$^\textrm{\scriptsize 77}$,
M.~Ronzani$^\textrm{\scriptsize 51}$,
L.~Roos$^\textrm{\scriptsize 83}$,
S.~Rosati$^\textrm{\scriptsize 134a}$,
K.~Rosbach$^\textrm{\scriptsize 51}$,
P.~Rose$^\textrm{\scriptsize 139}$,
N.-A.~Rosien$^\textrm{\scriptsize 57}$,
E.~Rossi$^\textrm{\scriptsize 106a,106b}$,
L.P.~Rossi$^\textrm{\scriptsize 53a}$,
J.H.N.~Rosten$^\textrm{\scriptsize 30}$,
R.~Rosten$^\textrm{\scriptsize 140}$,
M.~Rotaru$^\textrm{\scriptsize 28b}$,
J.~Rothberg$^\textrm{\scriptsize 140}$,
D.~Rousseau$^\textrm{\scriptsize 119}$,
A.~Rozanov$^\textrm{\scriptsize 88}$,
Y.~Rozen$^\textrm{\scriptsize 154}$,
X.~Ruan$^\textrm{\scriptsize 147c}$,
F.~Rubbo$^\textrm{\scriptsize 145}$,
F.~R\"uhr$^\textrm{\scriptsize 51}$,
A.~Ruiz-Martinez$^\textrm{\scriptsize 31}$,
Z.~Rurikova$^\textrm{\scriptsize 51}$,
N.A.~Rusakovich$^\textrm{\scriptsize 68}$,
H.L.~Russell$^\textrm{\scriptsize 90}$,
J.P.~Rutherfoord$^\textrm{\scriptsize 7}$,
N.~Ruthmann$^\textrm{\scriptsize 32}$,
Y.F.~Ryabov$^\textrm{\scriptsize 125}$,
M.~Rybar$^\textrm{\scriptsize 169}$,
G.~Rybkin$^\textrm{\scriptsize 119}$,
S.~Ryu$^\textrm{\scriptsize 6}$,
A.~Ryzhov$^\textrm{\scriptsize 132}$,
G.F.~Rzehorz$^\textrm{\scriptsize 57}$,
A.F.~Saavedra$^\textrm{\scriptsize 152}$,
G.~Sabato$^\textrm{\scriptsize 109}$,
S.~Sacerdoti$^\textrm{\scriptsize 29}$,
H.F-W.~Sadrozinski$^\textrm{\scriptsize 139}$,
R.~Sadykov$^\textrm{\scriptsize 68}$,
F.~Safai~Tehrani$^\textrm{\scriptsize 134a}$,
P.~Saha$^\textrm{\scriptsize 110}$,
M.~Sahinsoy$^\textrm{\scriptsize 60a}$,
M.~Saimpert$^\textrm{\scriptsize 45}$,
M.~Saito$^\textrm{\scriptsize 157}$,
T.~Saito$^\textrm{\scriptsize 157}$,
H.~Sakamoto$^\textrm{\scriptsize 157}$,
Y.~Sakurai$^\textrm{\scriptsize 174}$,
G.~Salamanna$^\textrm{\scriptsize 136a,136b}$,
J.E.~Salazar~Loyola$^\textrm{\scriptsize 34b}$,
D.~Salek$^\textrm{\scriptsize 109}$,
P.H.~Sales~De~Bruin$^\textrm{\scriptsize 168}$,
D.~Salihagic$^\textrm{\scriptsize 103}$,
A.~Salnikov$^\textrm{\scriptsize 145}$,
J.~Salt$^\textrm{\scriptsize 170}$,
D.~Salvatore$^\textrm{\scriptsize 40a,40b}$,
F.~Salvatore$^\textrm{\scriptsize 151}$,
A.~Salvucci$^\textrm{\scriptsize 62a,62b,62c}$,
A.~Salzburger$^\textrm{\scriptsize 32}$,
D.~Sammel$^\textrm{\scriptsize 51}$,
D.~Sampsonidis$^\textrm{\scriptsize 156}$,
D.~Sampsonidou$^\textrm{\scriptsize 156}$,
J.~S\'anchez$^\textrm{\scriptsize 170}$,
V.~Sanchez~Martinez$^\textrm{\scriptsize 170}$,
A.~Sanchez~Pineda$^\textrm{\scriptsize 167a,167c}$,
H.~Sandaker$^\textrm{\scriptsize 121}$,
R.L.~Sandbach$^\textrm{\scriptsize 79}$,
C.O.~Sander$^\textrm{\scriptsize 45}$,
M.~Sandhoff$^\textrm{\scriptsize 178}$,
C.~Sandoval$^\textrm{\scriptsize 21}$,
D.P.C.~Sankey$^\textrm{\scriptsize 133}$,
M.~Sannino$^\textrm{\scriptsize 53a,53b}$,
Y.~Sano$^\textrm{\scriptsize 105}$,
A.~Sansoni$^\textrm{\scriptsize 50}$,
C.~Santoni$^\textrm{\scriptsize 37}$,
H.~Santos$^\textrm{\scriptsize 128a}$,
I.~Santoyo~Castillo$^\textrm{\scriptsize 151}$,
A.~Sapronov$^\textrm{\scriptsize 68}$,
J.G.~Saraiva$^\textrm{\scriptsize 128a,128d}$,
B.~Sarrazin$^\textrm{\scriptsize 23}$,
O.~Sasaki$^\textrm{\scriptsize 69}$,
K.~Sato$^\textrm{\scriptsize 164}$,
E.~Sauvan$^\textrm{\scriptsize 5}$,
G.~Savage$^\textrm{\scriptsize 80}$,
P.~Savard$^\textrm{\scriptsize 161}$$^{,d}$,
N.~Savic$^\textrm{\scriptsize 103}$,
C.~Sawyer$^\textrm{\scriptsize 133}$,
L.~Sawyer$^\textrm{\scriptsize 82}$$^{,u}$,
J.~Saxon$^\textrm{\scriptsize 33}$,
C.~Sbarra$^\textrm{\scriptsize 22a}$,
A.~Sbrizzi$^\textrm{\scriptsize 22a,22b}$,
T.~Scanlon$^\textrm{\scriptsize 81}$,
D.A.~Scannicchio$^\textrm{\scriptsize 166}$,
M.~Scarcella$^\textrm{\scriptsize 152}$,
J.~Schaarschmidt$^\textrm{\scriptsize 140}$,
P.~Schacht$^\textrm{\scriptsize 103}$,
B.M.~Schachtner$^\textrm{\scriptsize 102}$,
D.~Schaefer$^\textrm{\scriptsize 32}$,
L.~Schaefer$^\textrm{\scriptsize 124}$,
R.~Schaefer$^\textrm{\scriptsize 45}$,
J.~Schaeffer$^\textrm{\scriptsize 86}$,
S.~Schaepe$^\textrm{\scriptsize 23}$,
S.~Schaetzel$^\textrm{\scriptsize 60b}$,
U.~Sch\"afer$^\textrm{\scriptsize 86}$,
A.C.~Schaffer$^\textrm{\scriptsize 119}$,
D.~Schaile$^\textrm{\scriptsize 102}$,
R.D.~Schamberger$^\textrm{\scriptsize 150}$,
V.A.~Schegelsky$^\textrm{\scriptsize 125}$,
D.~Scheirich$^\textrm{\scriptsize 131}$,
M.~Schernau$^\textrm{\scriptsize 166}$,
C.~Schiavi$^\textrm{\scriptsize 53a,53b}$,
S.~Schier$^\textrm{\scriptsize 139}$,
L.K.~Schildgen$^\textrm{\scriptsize 23}$,
C.~Schillo$^\textrm{\scriptsize 51}$,
M.~Schioppa$^\textrm{\scriptsize 40a,40b}$,
S.~Schlenker$^\textrm{\scriptsize 32}$,
K.R.~Schmidt-Sommerfeld$^\textrm{\scriptsize 103}$,
K.~Schmieden$^\textrm{\scriptsize 32}$,
C.~Schmitt$^\textrm{\scriptsize 86}$,
S.~Schmitt$^\textrm{\scriptsize 45}$,
S.~Schmitz$^\textrm{\scriptsize 86}$,
U.~Schnoor$^\textrm{\scriptsize 51}$,
L.~Schoeffel$^\textrm{\scriptsize 138}$,
A.~Schoening$^\textrm{\scriptsize 60b}$,
B.D.~Schoenrock$^\textrm{\scriptsize 93}$,
E.~Schopf$^\textrm{\scriptsize 23}$,
M.~Schott$^\textrm{\scriptsize 86}$,
J.F.P.~Schouwenberg$^\textrm{\scriptsize 108}$,
J.~Schovancova$^\textrm{\scriptsize 32}$,
S.~Schramm$^\textrm{\scriptsize 52}$,
N.~Schuh$^\textrm{\scriptsize 86}$,
A.~Schulte$^\textrm{\scriptsize 86}$,
M.J.~Schultens$^\textrm{\scriptsize 23}$,
H.-C.~Schultz-Coulon$^\textrm{\scriptsize 60a}$,
H.~Schulz$^\textrm{\scriptsize 17}$,
M.~Schumacher$^\textrm{\scriptsize 51}$,
B.A.~Schumm$^\textrm{\scriptsize 139}$,
Ph.~Schune$^\textrm{\scriptsize 138}$,
A.~Schwartzman$^\textrm{\scriptsize 145}$,
T.A.~Schwarz$^\textrm{\scriptsize 92}$,
H.~Schweiger$^\textrm{\scriptsize 87}$,
Ph.~Schwemling$^\textrm{\scriptsize 138}$,
R.~Schwienhorst$^\textrm{\scriptsize 93}$,
J.~Schwindling$^\textrm{\scriptsize 138}$,
A.~Sciandra$^\textrm{\scriptsize 23}$,
G.~Sciolla$^\textrm{\scriptsize 25}$,
M.~Scornajenghi$^\textrm{\scriptsize 40a,40b}$,
F.~Scuri$^\textrm{\scriptsize 126a,126b}$,
F.~Scutti$^\textrm{\scriptsize 91}$,
J.~Searcy$^\textrm{\scriptsize 92}$,
P.~Seema$^\textrm{\scriptsize 23}$,
S.C.~Seidel$^\textrm{\scriptsize 107}$,
A.~Seiden$^\textrm{\scriptsize 139}$,
J.M.~Seixas$^\textrm{\scriptsize 26a}$,
G.~Sekhniaidze$^\textrm{\scriptsize 106a}$,
K.~Sekhon$^\textrm{\scriptsize 92}$,
S.J.~Sekula$^\textrm{\scriptsize 43}$,
N.~Semprini-Cesari$^\textrm{\scriptsize 22a,22b}$,
S.~Senkin$^\textrm{\scriptsize 37}$,
C.~Serfon$^\textrm{\scriptsize 121}$,
L.~Serin$^\textrm{\scriptsize 119}$,
L.~Serkin$^\textrm{\scriptsize 167a,167b}$,
M.~Sessa$^\textrm{\scriptsize 136a,136b}$,
R.~Seuster$^\textrm{\scriptsize 172}$,
H.~Severini$^\textrm{\scriptsize 115}$,
T.~Sfiligoj$^\textrm{\scriptsize 78}$,
F.~Sforza$^\textrm{\scriptsize 32}$,
A.~Sfyrla$^\textrm{\scriptsize 52}$,
E.~Shabalina$^\textrm{\scriptsize 57}$,
N.W.~Shaikh$^\textrm{\scriptsize 148a,148b}$,
L.Y.~Shan$^\textrm{\scriptsize 35a}$,
R.~Shang$^\textrm{\scriptsize 169}$,
J.T.~Shank$^\textrm{\scriptsize 24}$,
M.~Shapiro$^\textrm{\scriptsize 16}$,
P.B.~Shatalov$^\textrm{\scriptsize 99}$,
K.~Shaw$^\textrm{\scriptsize 167a,167b}$,
S.M.~Shaw$^\textrm{\scriptsize 87}$,
A.~Shcherbakova$^\textrm{\scriptsize 148a,148b}$,
C.Y.~Shehu$^\textrm{\scriptsize 151}$,
Y.~Shen$^\textrm{\scriptsize 115}$,
N.~Sherafati$^\textrm{\scriptsize 31}$,
P.~Sherwood$^\textrm{\scriptsize 81}$,
L.~Shi$^\textrm{\scriptsize 153}$$^{,ao}$,
S.~Shimizu$^\textrm{\scriptsize 70}$,
C.O.~Shimmin$^\textrm{\scriptsize 179}$,
M.~Shimojima$^\textrm{\scriptsize 104}$,
I.P.J.~Shipsey$^\textrm{\scriptsize 122}$,
S.~Shirabe$^\textrm{\scriptsize 73}$,
M.~Shiyakova$^\textrm{\scriptsize 68}$$^{,ap}$,
J.~Shlomi$^\textrm{\scriptsize 175}$,
A.~Shmeleva$^\textrm{\scriptsize 98}$,
D.~Shoaleh~Saadi$^\textrm{\scriptsize 97}$,
M.J.~Shochet$^\textrm{\scriptsize 33}$,
S.~Shojaii$^\textrm{\scriptsize 94a}$,
D.R.~Shope$^\textrm{\scriptsize 115}$,
S.~Shrestha$^\textrm{\scriptsize 113}$,
E.~Shulga$^\textrm{\scriptsize 100}$,
M.A.~Shupe$^\textrm{\scriptsize 7}$,
P.~Sicho$^\textrm{\scriptsize 129}$,
A.M.~Sickles$^\textrm{\scriptsize 169}$,
P.E.~Sidebo$^\textrm{\scriptsize 149}$,
E.~Sideras~Haddad$^\textrm{\scriptsize 147c}$,
O.~Sidiropoulou$^\textrm{\scriptsize 177}$,
A.~Sidoti$^\textrm{\scriptsize 22a,22b}$,
F.~Siegert$^\textrm{\scriptsize 47}$,
Dj.~Sijacki$^\textrm{\scriptsize 14}$,
J.~Silva$^\textrm{\scriptsize 128a,128d}$,
S.B.~Silverstein$^\textrm{\scriptsize 148a}$,
V.~Simak$^\textrm{\scriptsize 130}$,
L.~Simic$^\textrm{\scriptsize 14}$,
S.~Simion$^\textrm{\scriptsize 119}$,
E.~Simioni$^\textrm{\scriptsize 86}$,
B.~Simmons$^\textrm{\scriptsize 81}$,
M.~Simon$^\textrm{\scriptsize 86}$,
P.~Sinervo$^\textrm{\scriptsize 161}$,
N.B.~Sinev$^\textrm{\scriptsize 118}$,
M.~Sioli$^\textrm{\scriptsize 22a,22b}$,
G.~Siragusa$^\textrm{\scriptsize 177}$,
I.~Siral$^\textrm{\scriptsize 92}$,
S.Yu.~Sivoklokov$^\textrm{\scriptsize 101}$,
J.~Sj\"{o}lin$^\textrm{\scriptsize 148a,148b}$,
M.B.~Skinner$^\textrm{\scriptsize 75}$,
P.~Skubic$^\textrm{\scriptsize 115}$,
M.~Slater$^\textrm{\scriptsize 19}$,
T.~Slavicek$^\textrm{\scriptsize 130}$,
M.~Slawinska$^\textrm{\scriptsize 42}$,
K.~Sliwa$^\textrm{\scriptsize 165}$,
R.~Slovak$^\textrm{\scriptsize 131}$,
V.~Smakhtin$^\textrm{\scriptsize 175}$,
B.H.~Smart$^\textrm{\scriptsize 5}$,
J.~Smiesko$^\textrm{\scriptsize 146a}$,
N.~Smirnov$^\textrm{\scriptsize 100}$,
S.Yu.~Smirnov$^\textrm{\scriptsize 100}$,
Y.~Smirnov$^\textrm{\scriptsize 100}$,
L.N.~Smirnova$^\textrm{\scriptsize 101}$$^{,aq}$,
O.~Smirnova$^\textrm{\scriptsize 84}$,
J.W.~Smith$^\textrm{\scriptsize 57}$,
M.N.K.~Smith$^\textrm{\scriptsize 38}$,
R.W.~Smith$^\textrm{\scriptsize 38}$,
M.~Smizanska$^\textrm{\scriptsize 75}$,
K.~Smolek$^\textrm{\scriptsize 130}$,
A.A.~Snesarev$^\textrm{\scriptsize 98}$,
I.M.~Snyder$^\textrm{\scriptsize 118}$,
S.~Snyder$^\textrm{\scriptsize 27}$,
R.~Sobie$^\textrm{\scriptsize 172}$$^{,o}$,
F.~Socher$^\textrm{\scriptsize 47}$,
A.~Soffer$^\textrm{\scriptsize 155}$,
A.~S{\o}gaard$^\textrm{\scriptsize 49}$,
D.A.~Soh$^\textrm{\scriptsize 153}$,
G.~Sokhrannyi$^\textrm{\scriptsize 78}$,
C.A.~Solans~Sanchez$^\textrm{\scriptsize 32}$,
M.~Solar$^\textrm{\scriptsize 130}$,
E.Yu.~Soldatov$^\textrm{\scriptsize 100}$,
U.~Soldevila$^\textrm{\scriptsize 170}$,
A.A.~Solodkov$^\textrm{\scriptsize 132}$,
A.~Soloshenko$^\textrm{\scriptsize 68}$,
O.V.~Solovyanov$^\textrm{\scriptsize 132}$,
V.~Solovyev$^\textrm{\scriptsize 125}$,
P.~Sommer$^\textrm{\scriptsize 51}$,
H.~Son$^\textrm{\scriptsize 165}$,
A.~Sopczak$^\textrm{\scriptsize 130}$,
D.~Sosa$^\textrm{\scriptsize 60b}$,
C.L.~Sotiropoulou$^\textrm{\scriptsize 126a,126b}$,
R.~Soualah$^\textrm{\scriptsize 167a,167c}$,
A.M.~Soukharev$^\textrm{\scriptsize 111}$$^{,c}$,
D.~South$^\textrm{\scriptsize 45}$,
B.C.~Sowden$^\textrm{\scriptsize 80}$,
S.~Spagnolo$^\textrm{\scriptsize 76a,76b}$,
M.~Spalla$^\textrm{\scriptsize 126a,126b}$,
M.~Spangenberg$^\textrm{\scriptsize 173}$,
F.~Span\`o$^\textrm{\scriptsize 80}$,
D.~Sperlich$^\textrm{\scriptsize 17}$,
F.~Spettel$^\textrm{\scriptsize 103}$,
T.M.~Spieker$^\textrm{\scriptsize 60a}$,
R.~Spighi$^\textrm{\scriptsize 22a}$,
G.~Spigo$^\textrm{\scriptsize 32}$,
L.A.~Spiller$^\textrm{\scriptsize 91}$,
M.~Spousta$^\textrm{\scriptsize 131}$,
R.D.~St.~Denis$^\textrm{\scriptsize 56}$$^{,*}$,
A.~Stabile$^\textrm{\scriptsize 94a}$,
R.~Stamen$^\textrm{\scriptsize 60a}$,
S.~Stamm$^\textrm{\scriptsize 17}$,
E.~Stanecka$^\textrm{\scriptsize 42}$,
R.W.~Stanek$^\textrm{\scriptsize 6}$,
C.~Stanescu$^\textrm{\scriptsize 136a}$,
M.M.~Stanitzki$^\textrm{\scriptsize 45}$,
B.S.~Stapf$^\textrm{\scriptsize 109}$,
S.~Stapnes$^\textrm{\scriptsize 121}$,
E.A.~Starchenko$^\textrm{\scriptsize 132}$,
G.H.~Stark$^\textrm{\scriptsize 33}$,
J.~Stark$^\textrm{\scriptsize 58}$,
S.H~Stark$^\textrm{\scriptsize 39}$,
P.~Staroba$^\textrm{\scriptsize 129}$,
P.~Starovoitov$^\textrm{\scriptsize 60a}$,
S.~St\"arz$^\textrm{\scriptsize 32}$,
R.~Staszewski$^\textrm{\scriptsize 42}$,
P.~Steinberg$^\textrm{\scriptsize 27}$,
B.~Stelzer$^\textrm{\scriptsize 144}$,
H.J.~Stelzer$^\textrm{\scriptsize 32}$,
O.~Stelzer-Chilton$^\textrm{\scriptsize 163a}$,
H.~Stenzel$^\textrm{\scriptsize 55}$,
G.A.~Stewart$^\textrm{\scriptsize 56}$,
M.C.~Stockton$^\textrm{\scriptsize 118}$,
M.~Stoebe$^\textrm{\scriptsize 90}$,
G.~Stoicea$^\textrm{\scriptsize 28b}$,
P.~Stolte$^\textrm{\scriptsize 57}$,
S.~Stonjek$^\textrm{\scriptsize 103}$,
A.R.~Stradling$^\textrm{\scriptsize 8}$,
A.~Straessner$^\textrm{\scriptsize 47}$,
M.E.~Stramaglia$^\textrm{\scriptsize 18}$,
J.~Strandberg$^\textrm{\scriptsize 149}$,
S.~Strandberg$^\textrm{\scriptsize 148a,148b}$,
M.~Strauss$^\textrm{\scriptsize 115}$,
P.~Strizenec$^\textrm{\scriptsize 146b}$,
R.~Str\"ohmer$^\textrm{\scriptsize 177}$,
D.M.~Strom$^\textrm{\scriptsize 118}$,
R.~Stroynowski$^\textrm{\scriptsize 43}$,
A.~Strubig$^\textrm{\scriptsize 49}$,
S.A.~Stucci$^\textrm{\scriptsize 27}$,
B.~Stugu$^\textrm{\scriptsize 15}$,
N.A.~Styles$^\textrm{\scriptsize 45}$,
D.~Su$^\textrm{\scriptsize 145}$,
J.~Su$^\textrm{\scriptsize 127}$,
S.~Suchek$^\textrm{\scriptsize 60a}$,
Y.~Sugaya$^\textrm{\scriptsize 120}$,
M.~Suk$^\textrm{\scriptsize 130}$,
V.V.~Sulin$^\textrm{\scriptsize 98}$,
DMS~Sultan$^\textrm{\scriptsize 162a,162b}$,
S.~Sultansoy$^\textrm{\scriptsize 4c}$,
T.~Sumida$^\textrm{\scriptsize 71}$,
S.~Sun$^\textrm{\scriptsize 59}$,
X.~Sun$^\textrm{\scriptsize 3}$,
K.~Suruliz$^\textrm{\scriptsize 151}$,
C.J.E.~Suster$^\textrm{\scriptsize 152}$,
M.R.~Sutton$^\textrm{\scriptsize 151}$,
S.~Suzuki$^\textrm{\scriptsize 69}$,
M.~Svatos$^\textrm{\scriptsize 129}$,
M.~Swiatlowski$^\textrm{\scriptsize 33}$,
S.P.~Swift$^\textrm{\scriptsize 2}$,
I.~Sykora$^\textrm{\scriptsize 146a}$,
T.~Sykora$^\textrm{\scriptsize 131}$,
D.~Ta$^\textrm{\scriptsize 51}$,
K.~Tackmann$^\textrm{\scriptsize 45}$,
J.~Taenzer$^\textrm{\scriptsize 155}$,
A.~Taffard$^\textrm{\scriptsize 166}$,
R.~Tafirout$^\textrm{\scriptsize 163a}$,
E.~Tahirovic$^\textrm{\scriptsize 79}$,
N.~Taiblum$^\textrm{\scriptsize 155}$,
H.~Takai$^\textrm{\scriptsize 27}$,
R.~Takashima$^\textrm{\scriptsize 72}$,
E.H.~Takasugi$^\textrm{\scriptsize 103}$,
T.~Takeshita$^\textrm{\scriptsize 142}$,
Y.~Takubo$^\textrm{\scriptsize 69}$,
M.~Talby$^\textrm{\scriptsize 88}$,
A.A.~Talyshev$^\textrm{\scriptsize 111}$$^{,c}$,
J.~Tanaka$^\textrm{\scriptsize 157}$,
M.~Tanaka$^\textrm{\scriptsize 159}$,
R.~Tanaka$^\textrm{\scriptsize 119}$,
S.~Tanaka$^\textrm{\scriptsize 69}$,
R.~Tanioka$^\textrm{\scriptsize 70}$,
B.B.~Tannenwald$^\textrm{\scriptsize 113}$,
S.~Tapia~Araya$^\textrm{\scriptsize 34b}$,
S.~Tapprogge$^\textrm{\scriptsize 86}$,
S.~Tarem$^\textrm{\scriptsize 154}$,
G.F.~Tartarelli$^\textrm{\scriptsize 94a}$,
P.~Tas$^\textrm{\scriptsize 131}$,
M.~Tasevsky$^\textrm{\scriptsize 129}$,
T.~Tashiro$^\textrm{\scriptsize 71}$,
E.~Tassi$^\textrm{\scriptsize 40a,40b}$,
A.~Tavares~Delgado$^\textrm{\scriptsize 128a,128b}$,
Y.~Tayalati$^\textrm{\scriptsize 137e}$,
A.C.~Taylor$^\textrm{\scriptsize 107}$,
G.N.~Taylor$^\textrm{\scriptsize 91}$,
P.T.E.~Taylor$^\textrm{\scriptsize 91}$,
W.~Taylor$^\textrm{\scriptsize 163b}$,
P.~Teixeira-Dias$^\textrm{\scriptsize 80}$,
D.~Temple$^\textrm{\scriptsize 144}$,
H.~Ten~Kate$^\textrm{\scriptsize 32}$,
P.K.~Teng$^\textrm{\scriptsize 153}$,
J.J.~Teoh$^\textrm{\scriptsize 120}$,
F.~Tepel$^\textrm{\scriptsize 178}$,
S.~Terada$^\textrm{\scriptsize 69}$,
K.~Terashi$^\textrm{\scriptsize 157}$,
J.~Terron$^\textrm{\scriptsize 85}$,
S.~Terzo$^\textrm{\scriptsize 13}$,
M.~Testa$^\textrm{\scriptsize 50}$,
R.J.~Teuscher$^\textrm{\scriptsize 161}$$^{,o}$,
T.~Theveneaux-Pelzer$^\textrm{\scriptsize 88}$,
F.~Thiele$^\textrm{\scriptsize 39}$,
J.P.~Thomas$^\textrm{\scriptsize 19}$,
J.~Thomas-Wilsker$^\textrm{\scriptsize 80}$,
P.D.~Thompson$^\textrm{\scriptsize 19}$,
A.S.~Thompson$^\textrm{\scriptsize 56}$,
L.A.~Thomsen$^\textrm{\scriptsize 179}$,
E.~Thomson$^\textrm{\scriptsize 124}$,
M.J.~Tibbetts$^\textrm{\scriptsize 16}$,
R.E.~Ticse~Torres$^\textrm{\scriptsize 88}$,
V.O.~Tikhomirov$^\textrm{\scriptsize 98}$$^{,ar}$,
Yu.A.~Tikhonov$^\textrm{\scriptsize 111}$$^{,c}$,
S.~Timoshenko$^\textrm{\scriptsize 100}$,
P.~Tipton$^\textrm{\scriptsize 179}$,
S.~Tisserant$^\textrm{\scriptsize 88}$,
K.~Todome$^\textrm{\scriptsize 159}$,
S.~Todorova-Nova$^\textrm{\scriptsize 5}$,
S.~Todt$^\textrm{\scriptsize 47}$,
J.~Tojo$^\textrm{\scriptsize 73}$,
S.~Tok\'ar$^\textrm{\scriptsize 146a}$,
K.~Tokushuku$^\textrm{\scriptsize 69}$,
E.~Tolley$^\textrm{\scriptsize 113}$,
L.~Tomlinson$^\textrm{\scriptsize 87}$,
M.~Tomoto$^\textrm{\scriptsize 105}$,
L.~Tompkins$^\textrm{\scriptsize 145}$$^{,as}$,
K.~Toms$^\textrm{\scriptsize 107}$,
B.~Tong$^\textrm{\scriptsize 59}$,
P.~Tornambe$^\textrm{\scriptsize 51}$,
E.~Torrence$^\textrm{\scriptsize 118}$,
H.~Torres$^\textrm{\scriptsize 144}$,
E.~Torr\'o~Pastor$^\textrm{\scriptsize 140}$,
J.~Toth$^\textrm{\scriptsize 88}$$^{,at}$,
F.~Touchard$^\textrm{\scriptsize 88}$,
D.R.~Tovey$^\textrm{\scriptsize 141}$,
C.J.~Treado$^\textrm{\scriptsize 112}$,
T.~Trefzger$^\textrm{\scriptsize 177}$,
F.~Tresoldi$^\textrm{\scriptsize 151}$,
A.~Tricoli$^\textrm{\scriptsize 27}$,
I.M.~Trigger$^\textrm{\scriptsize 163a}$,
S.~Trincaz-Duvoid$^\textrm{\scriptsize 83}$,
M.F.~Tripiana$^\textrm{\scriptsize 13}$,
W.~Trischuk$^\textrm{\scriptsize 161}$,
B.~Trocm\'e$^\textrm{\scriptsize 58}$,
A.~Trofymov$^\textrm{\scriptsize 45}$,
C.~Troncon$^\textrm{\scriptsize 94a}$,
M.~Trottier-McDonald$^\textrm{\scriptsize 16}$,
M.~Trovatelli$^\textrm{\scriptsize 172}$,
L.~Truong$^\textrm{\scriptsize 147b}$,
M.~Trzebinski$^\textrm{\scriptsize 42}$,
A.~Trzupek$^\textrm{\scriptsize 42}$,
K.W.~Tsang$^\textrm{\scriptsize 62a}$,
J.C-L.~Tseng$^\textrm{\scriptsize 122}$,
P.V.~Tsiareshka$^\textrm{\scriptsize 95}$,
G.~Tsipolitis$^\textrm{\scriptsize 10}$,
N.~Tsirintanis$^\textrm{\scriptsize 9}$,
S.~Tsiskaridze$^\textrm{\scriptsize 13}$,
V.~Tsiskaridze$^\textrm{\scriptsize 51}$,
E.G.~Tskhadadze$^\textrm{\scriptsize 54a}$,
K.M.~Tsui$^\textrm{\scriptsize 62a}$,
I.I.~Tsukerman$^\textrm{\scriptsize 99}$,
V.~Tsulaia$^\textrm{\scriptsize 16}$,
S.~Tsuno$^\textrm{\scriptsize 69}$,
D.~Tsybychev$^\textrm{\scriptsize 150}$,
Y.~Tu$^\textrm{\scriptsize 62b}$,
A.~Tudorache$^\textrm{\scriptsize 28b}$,
V.~Tudorache$^\textrm{\scriptsize 28b}$,
T.T.~Tulbure$^\textrm{\scriptsize 28a}$,
A.N.~Tuna$^\textrm{\scriptsize 59}$,
S.A.~Tupputi$^\textrm{\scriptsize 22a,22b}$,
S.~Turchikhin$^\textrm{\scriptsize 68}$,
D.~Turgeman$^\textrm{\scriptsize 175}$,
I.~Turk~Cakir$^\textrm{\scriptsize 4b}$$^{,au}$,
R.~Turra$^\textrm{\scriptsize 94a}$,
P.M.~Tuts$^\textrm{\scriptsize 38}$,
G.~Ucchielli$^\textrm{\scriptsize 22a,22b}$,
I.~Ueda$^\textrm{\scriptsize 69}$,
M.~Ughetto$^\textrm{\scriptsize 148a,148b}$,
F.~Ukegawa$^\textrm{\scriptsize 164}$,
G.~Unal$^\textrm{\scriptsize 32}$,
A.~Undrus$^\textrm{\scriptsize 27}$,
G.~Unel$^\textrm{\scriptsize 166}$,
F.C.~Ungaro$^\textrm{\scriptsize 91}$,
Y.~Unno$^\textrm{\scriptsize 69}$,
C.~Unverdorben$^\textrm{\scriptsize 102}$,
J.~Urban$^\textrm{\scriptsize 146b}$,
P.~Urquijo$^\textrm{\scriptsize 91}$,
P.~Urrejola$^\textrm{\scriptsize 86}$,
G.~Usai$^\textrm{\scriptsize 8}$,
J.~Usui$^\textrm{\scriptsize 69}$,
L.~Vacavant$^\textrm{\scriptsize 88}$,
V.~Vacek$^\textrm{\scriptsize 130}$,
B.~Vachon$^\textrm{\scriptsize 90}$,
K.O.H.~Vadla$^\textrm{\scriptsize 121}$,
A.~Vaidya$^\textrm{\scriptsize 81}$,
C.~Valderanis$^\textrm{\scriptsize 102}$,
E.~Valdes~Santurio$^\textrm{\scriptsize 148a,148b}$,
M.~Valente$^\textrm{\scriptsize 52}$,
S.~Valentinetti$^\textrm{\scriptsize 22a,22b}$,
A.~Valero$^\textrm{\scriptsize 170}$,
L.~Val\'ery$^\textrm{\scriptsize 13}$,
S.~Valkar$^\textrm{\scriptsize 131}$,
A.~Vallier$^\textrm{\scriptsize 5}$,
J.A.~Valls~Ferrer$^\textrm{\scriptsize 170}$,
W.~Van~Den~Wollenberg$^\textrm{\scriptsize 109}$,
H.~van~der~Graaf$^\textrm{\scriptsize 109}$,
P.~van~Gemmeren$^\textrm{\scriptsize 6}$,
J.~Van~Nieuwkoop$^\textrm{\scriptsize 144}$,
I.~van~Vulpen$^\textrm{\scriptsize 109}$,
M.C.~van~Woerden$^\textrm{\scriptsize 109}$,
M.~Vanadia$^\textrm{\scriptsize 135a,135b}$,
W.~Vandelli$^\textrm{\scriptsize 32}$,
A.~Vaniachine$^\textrm{\scriptsize 160}$,
P.~Vankov$^\textrm{\scriptsize 109}$,
G.~Vardanyan$^\textrm{\scriptsize 180}$,
R.~Vari$^\textrm{\scriptsize 134a}$,
E.W.~Varnes$^\textrm{\scriptsize 7}$,
C.~Varni$^\textrm{\scriptsize 53a,53b}$,
T.~Varol$^\textrm{\scriptsize 43}$,
D.~Varouchas$^\textrm{\scriptsize 119}$,
A.~Vartapetian$^\textrm{\scriptsize 8}$,
K.E.~Varvell$^\textrm{\scriptsize 152}$,
J.G.~Vasquez$^\textrm{\scriptsize 179}$,
G.A.~Vasquez$^\textrm{\scriptsize 34b}$,
F.~Vazeille$^\textrm{\scriptsize 37}$,
T.~Vazquez~Schroeder$^\textrm{\scriptsize 90}$,
J.~Veatch$^\textrm{\scriptsize 57}$,
V.~Veeraraghavan$^\textrm{\scriptsize 7}$,
L.M.~Veloce$^\textrm{\scriptsize 161}$,
F.~Veloso$^\textrm{\scriptsize 128a,128c}$,
S.~Veneziano$^\textrm{\scriptsize 134a}$,
A.~Ventura$^\textrm{\scriptsize 76a,76b}$,
M.~Venturi$^\textrm{\scriptsize 172}$,
N.~Venturi$^\textrm{\scriptsize 32}$,
A.~Venturini$^\textrm{\scriptsize 25}$,
V.~Vercesi$^\textrm{\scriptsize 123a}$,
M.~Verducci$^\textrm{\scriptsize 136a,136b}$,
W.~Verkerke$^\textrm{\scriptsize 109}$,
A.T.~Vermeulen$^\textrm{\scriptsize 109}$,
J.C.~Vermeulen$^\textrm{\scriptsize 109}$,
M.C.~Vetterli$^\textrm{\scriptsize 144}$$^{,d}$,
N.~Viaux~Maira$^\textrm{\scriptsize 34b}$,
O.~Viazlo$^\textrm{\scriptsize 84}$,
I.~Vichou$^\textrm{\scriptsize 169}$$^{,*}$,
T.~Vickey$^\textrm{\scriptsize 141}$,
O.E.~Vickey~Boeriu$^\textrm{\scriptsize 141}$,
G.H.A.~Viehhauser$^\textrm{\scriptsize 122}$,
S.~Viel$^\textrm{\scriptsize 16}$,
L.~Vigani$^\textrm{\scriptsize 122}$,
M.~Villa$^\textrm{\scriptsize 22a,22b}$,
M.~Villaplana~Perez$^\textrm{\scriptsize 94a,94b}$,
E.~Vilucchi$^\textrm{\scriptsize 50}$,
M.G.~Vincter$^\textrm{\scriptsize 31}$,
V.B.~Vinogradov$^\textrm{\scriptsize 68}$,
A.~Vishwakarma$^\textrm{\scriptsize 45}$,
C.~Vittori$^\textrm{\scriptsize 22a,22b}$,
I.~Vivarelli$^\textrm{\scriptsize 151}$,
S.~Vlachos$^\textrm{\scriptsize 10}$,
M.~Vogel$^\textrm{\scriptsize 178}$,
P.~Vokac$^\textrm{\scriptsize 130}$,
G.~Volpi$^\textrm{\scriptsize 126a,126b}$,
H.~von~der~Schmitt$^\textrm{\scriptsize 103}$,
E.~von~Toerne$^\textrm{\scriptsize 23}$,
V.~Vorobel$^\textrm{\scriptsize 131}$,
K.~Vorobev$^\textrm{\scriptsize 100}$,
M.~Vos$^\textrm{\scriptsize 170}$,
R.~Voss$^\textrm{\scriptsize 32}$,
J.H.~Vossebeld$^\textrm{\scriptsize 77}$,
N.~Vranjes$^\textrm{\scriptsize 14}$,
M.~Vranjes~Milosavljevic$^\textrm{\scriptsize 14}$,
V.~Vrba$^\textrm{\scriptsize 130}$,
M.~Vreeswijk$^\textrm{\scriptsize 109}$,
R.~Vuillermet$^\textrm{\scriptsize 32}$,
I.~Vukotic$^\textrm{\scriptsize 33}$,
P.~Wagner$^\textrm{\scriptsize 23}$,
W.~Wagner$^\textrm{\scriptsize 178}$,
J.~Wagner-Kuhr$^\textrm{\scriptsize 102}$,
H.~Wahlberg$^\textrm{\scriptsize 74}$,
S.~Wahrmund$^\textrm{\scriptsize 47}$,
J.~Wakabayashi$^\textrm{\scriptsize 105}$,
J.~Walder$^\textrm{\scriptsize 75}$,
R.~Walker$^\textrm{\scriptsize 102}$,
W.~Walkowiak$^\textrm{\scriptsize 143}$,
V.~Wallangen$^\textrm{\scriptsize 148a,148b}$,
C.~Wang$^\textrm{\scriptsize 35b}$,
C.~Wang$^\textrm{\scriptsize 36b}$$^{,av}$,
F.~Wang$^\textrm{\scriptsize 176}$,
H.~Wang$^\textrm{\scriptsize 16}$,
H.~Wang$^\textrm{\scriptsize 3}$,
J.~Wang$^\textrm{\scriptsize 45}$,
J.~Wang$^\textrm{\scriptsize 152}$,
Q.~Wang$^\textrm{\scriptsize 115}$,
R.~Wang$^\textrm{\scriptsize 6}$,
S.M.~Wang$^\textrm{\scriptsize 153}$,
T.~Wang$^\textrm{\scriptsize 38}$,
W.~Wang$^\textrm{\scriptsize 153}$$^{,aw}$,
W.~Wang$^\textrm{\scriptsize 36a}$$^{,ax}$,
Z.~Wang$^\textrm{\scriptsize 36c}$,
C.~Wanotayaroj$^\textrm{\scriptsize 118}$,
A.~Warburton$^\textrm{\scriptsize 90}$,
C.P.~Ward$^\textrm{\scriptsize 30}$,
D.R.~Wardrope$^\textrm{\scriptsize 81}$,
A.~Washbrook$^\textrm{\scriptsize 49}$,
P.M.~Watkins$^\textrm{\scriptsize 19}$,
A.T.~Watson$^\textrm{\scriptsize 19}$,
M.F.~Watson$^\textrm{\scriptsize 19}$,
G.~Watts$^\textrm{\scriptsize 140}$,
S.~Watts$^\textrm{\scriptsize 87}$,
B.M.~Waugh$^\textrm{\scriptsize 81}$,
A.F.~Webb$^\textrm{\scriptsize 11}$,
S.~Webb$^\textrm{\scriptsize 86}$,
M.S.~Weber$^\textrm{\scriptsize 18}$,
S.W.~Weber$^\textrm{\scriptsize 177}$,
S.A.~Weber$^\textrm{\scriptsize 31}$,
J.S.~Webster$^\textrm{\scriptsize 6}$,
A.R.~Weidberg$^\textrm{\scriptsize 122}$,
B.~Weinert$^\textrm{\scriptsize 64}$,
J.~Weingarten$^\textrm{\scriptsize 57}$,
M.~Weirich$^\textrm{\scriptsize 86}$,
C.~Weiser$^\textrm{\scriptsize 51}$,
H.~Weits$^\textrm{\scriptsize 109}$,
P.S.~Wells$^\textrm{\scriptsize 32}$,
T.~Wenaus$^\textrm{\scriptsize 27}$,
T.~Wengler$^\textrm{\scriptsize 32}$,
S.~Wenig$^\textrm{\scriptsize 32}$,
N.~Wermes$^\textrm{\scriptsize 23}$,
M.D.~Werner$^\textrm{\scriptsize 67}$,
P.~Werner$^\textrm{\scriptsize 32}$,
M.~Wessels$^\textrm{\scriptsize 60a}$,
T.D.~Weston$^\textrm{\scriptsize 18}$,
K.~Whalen$^\textrm{\scriptsize 118}$,
N.L.~Whallon$^\textrm{\scriptsize 140}$,
A.M.~Wharton$^\textrm{\scriptsize 75}$,
A.S.~White$^\textrm{\scriptsize 92}$,
A.~White$^\textrm{\scriptsize 8}$,
M.J.~White$^\textrm{\scriptsize 1}$,
R.~White$^\textrm{\scriptsize 34b}$,
D.~Whiteson$^\textrm{\scriptsize 166}$,
B.W.~Whitmore$^\textrm{\scriptsize 75}$,
F.J.~Wickens$^\textrm{\scriptsize 133}$,
W.~Wiedenmann$^\textrm{\scriptsize 176}$,
M.~Wielers$^\textrm{\scriptsize 133}$,
C.~Wiglesworth$^\textrm{\scriptsize 39}$,
L.A.M.~Wiik-Fuchs$^\textrm{\scriptsize 51}$,
A.~Wildauer$^\textrm{\scriptsize 103}$,
F.~Wilk$^\textrm{\scriptsize 87}$,
H.G.~Wilkens$^\textrm{\scriptsize 32}$,
H.H.~Williams$^\textrm{\scriptsize 124}$,
S.~Williams$^\textrm{\scriptsize 109}$,
C.~Willis$^\textrm{\scriptsize 93}$,
S.~Willocq$^\textrm{\scriptsize 89}$,
J.A.~Wilson$^\textrm{\scriptsize 19}$,
I.~Wingerter-Seez$^\textrm{\scriptsize 5}$,
E.~Winkels$^\textrm{\scriptsize 151}$,
F.~Winklmeier$^\textrm{\scriptsize 118}$,
O.J.~Winston$^\textrm{\scriptsize 151}$,
B.T.~Winter$^\textrm{\scriptsize 23}$,
M.~Wittgen$^\textrm{\scriptsize 145}$,
M.~Wobisch$^\textrm{\scriptsize 82}$$^{,u}$,
T.M.H.~Wolf$^\textrm{\scriptsize 109}$,
R.~Wolff$^\textrm{\scriptsize 88}$,
M.W.~Wolter$^\textrm{\scriptsize 42}$,
H.~Wolters$^\textrm{\scriptsize 128a,128c}$,
V.W.S.~Wong$^\textrm{\scriptsize 171}$,
S.D.~Worm$^\textrm{\scriptsize 19}$,
B.K.~Wosiek$^\textrm{\scriptsize 42}$,
J.~Wotschack$^\textrm{\scriptsize 32}$,
K.W.~Wozniak$^\textrm{\scriptsize 42}$,
M.~Wu$^\textrm{\scriptsize 33}$,
S.L.~Wu$^\textrm{\scriptsize 176}$,
X.~Wu$^\textrm{\scriptsize 52}$,
Y.~Wu$^\textrm{\scriptsize 92}$,
T.R.~Wyatt$^\textrm{\scriptsize 87}$,
B.M.~Wynne$^\textrm{\scriptsize 49}$,
S.~Xella$^\textrm{\scriptsize 39}$,
Z.~Xi$^\textrm{\scriptsize 92}$,
L.~Xia$^\textrm{\scriptsize 35c}$,
D.~Xu$^\textrm{\scriptsize 35a}$,
L.~Xu$^\textrm{\scriptsize 27}$,
T.~Xu$^\textrm{\scriptsize 138}$,
B.~Yabsley$^\textrm{\scriptsize 152}$,
S.~Yacoob$^\textrm{\scriptsize 147a}$,
D.~Yamaguchi$^\textrm{\scriptsize 159}$,
Y.~Yamaguchi$^\textrm{\scriptsize 120}$,
A.~Yamamoto$^\textrm{\scriptsize 69}$,
S.~Yamamoto$^\textrm{\scriptsize 157}$,
T.~Yamanaka$^\textrm{\scriptsize 157}$,
M.~Yamatani$^\textrm{\scriptsize 157}$,
K.~Yamauchi$^\textrm{\scriptsize 105}$,
Y.~Yamazaki$^\textrm{\scriptsize 70}$,
Z.~Yan$^\textrm{\scriptsize 24}$,
H.~Yang$^\textrm{\scriptsize 36c}$,
H.~Yang$^\textrm{\scriptsize 16}$,
Y.~Yang$^\textrm{\scriptsize 153}$,
Z.~Yang$^\textrm{\scriptsize 15}$,
W-M.~Yao$^\textrm{\scriptsize 16}$,
Y.C.~Yap$^\textrm{\scriptsize 83}$,
Y.~Yasu$^\textrm{\scriptsize 69}$,
E.~Yatsenko$^\textrm{\scriptsize 5}$,
K.H.~Yau~Wong$^\textrm{\scriptsize 23}$,
J.~Ye$^\textrm{\scriptsize 43}$,
S.~Ye$^\textrm{\scriptsize 27}$,
I.~Yeletskikh$^\textrm{\scriptsize 68}$,
E.~Yigitbasi$^\textrm{\scriptsize 24}$,
E.~Yildirim$^\textrm{\scriptsize 86}$,
K.~Yorita$^\textrm{\scriptsize 174}$,
K.~Yoshihara$^\textrm{\scriptsize 124}$,
C.~Young$^\textrm{\scriptsize 145}$,
C.J.S.~Young$^\textrm{\scriptsize 32}$,
J.~Yu$^\textrm{\scriptsize 8}$,
J.~Yu$^\textrm{\scriptsize 67}$,
S.P.Y.~Yuen$^\textrm{\scriptsize 23}$,
I.~Yusuff$^\textrm{\scriptsize 30}$$^{,ay}$,
B.~Zabinski$^\textrm{\scriptsize 42}$,
G.~Zacharis$^\textrm{\scriptsize 10}$,
R.~Zaidan$^\textrm{\scriptsize 13}$,
A.M.~Zaitsev$^\textrm{\scriptsize 132}$$^{,al}$,
N.~Zakharchuk$^\textrm{\scriptsize 45}$,
J.~Zalieckas$^\textrm{\scriptsize 15}$,
A.~Zaman$^\textrm{\scriptsize 150}$,
S.~Zambito$^\textrm{\scriptsize 59}$,
D.~Zanzi$^\textrm{\scriptsize 91}$,
C.~Zeitnitz$^\textrm{\scriptsize 178}$,
G.~Zemaityte$^\textrm{\scriptsize 122}$,
A.~Zemla$^\textrm{\scriptsize 41a}$,
J.C.~Zeng$^\textrm{\scriptsize 169}$,
Q.~Zeng$^\textrm{\scriptsize 145}$,
O.~Zenin$^\textrm{\scriptsize 132}$,
T.~\v{Z}eni\v{s}$^\textrm{\scriptsize 146a}$,
D.~Zerwas$^\textrm{\scriptsize 119}$,
D.~Zhang$^\textrm{\scriptsize 92}$,
F.~Zhang$^\textrm{\scriptsize 176}$,
G.~Zhang$^\textrm{\scriptsize 36a}$$^{,ax}$,
H.~Zhang$^\textrm{\scriptsize 35b}$,
J.~Zhang$^\textrm{\scriptsize 6}$,
L.~Zhang$^\textrm{\scriptsize 51}$,
L.~Zhang$^\textrm{\scriptsize 36a}$,
M.~Zhang$^\textrm{\scriptsize 169}$,
P.~Zhang$^\textrm{\scriptsize 35b}$,
R.~Zhang$^\textrm{\scriptsize 23}$,
R.~Zhang$^\textrm{\scriptsize 36a}$$^{,av}$,
X.~Zhang$^\textrm{\scriptsize 36b}$,
Y.~Zhang$^\textrm{\scriptsize 35a,35d}$,
Z.~Zhang$^\textrm{\scriptsize 119}$,
X.~Zhao$^\textrm{\scriptsize 43}$,
Y.~Zhao$^\textrm{\scriptsize 36b}$$^{,az}$,
Z.~Zhao$^\textrm{\scriptsize 36a}$,
A.~Zhemchugov$^\textrm{\scriptsize 68}$,
B.~Zhou$^\textrm{\scriptsize 92}$,
C.~Zhou$^\textrm{\scriptsize 176}$,
L.~Zhou$^\textrm{\scriptsize 43}$,
M.~Zhou$^\textrm{\scriptsize 35a,35d}$,
M.~Zhou$^\textrm{\scriptsize 150}$,
N.~Zhou$^\textrm{\scriptsize 35c}$,
C.G.~Zhu$^\textrm{\scriptsize 36b}$,
H.~Zhu$^\textrm{\scriptsize 35a}$,
J.~Zhu$^\textrm{\scriptsize 92}$,
Y.~Zhu$^\textrm{\scriptsize 36a}$,
X.~Zhuang$^\textrm{\scriptsize 35a}$,
K.~Zhukov$^\textrm{\scriptsize 98}$,
A.~Zibell$^\textrm{\scriptsize 177}$,
D.~Zieminska$^\textrm{\scriptsize 64}$,
N.I.~Zimine$^\textrm{\scriptsize 68}$,
C.~Zimmermann$^\textrm{\scriptsize 86}$,
S.~Zimmermann$^\textrm{\scriptsize 51}$,
Z.~Zinonos$^\textrm{\scriptsize 103}$,
M.~Zinser$^\textrm{\scriptsize 86}$,
M.~Ziolkowski$^\textrm{\scriptsize 143}$,
L.~\v{Z}ivkovi\'{c}$^\textrm{\scriptsize 14}$,
G.~Zobernig$^\textrm{\scriptsize 176}$,
A.~Zoccoli$^\textrm{\scriptsize 22a,22b}$,
R.~Zou$^\textrm{\scriptsize 33}$,
M.~zur~Nedden$^\textrm{\scriptsize 17}$,
L.~Zwalinski$^\textrm{\scriptsize 32}$.
\bigskip
\\
$^{1}$ Department of Physics, University of Adelaide, Adelaide, Australia\\
$^{2}$ Physics Department, SUNY Albany, Albany NY, United States of America\\
$^{3}$ Department of Physics, University of Alberta, Edmonton AB, Canada\\
$^{4}$ $^{(a)}$ Department of Physics, Ankara University, Ankara; $^{(b)}$ Istanbul Aydin University, Istanbul; $^{(c)}$ Division of Physics, TOBB University of Economics and Technology, Ankara, Turkey\\
$^{5}$ LAPP, CNRS/IN2P3 and Universit{\'e} Savoie Mont Blanc, Annecy-le-Vieux, France\\
$^{6}$ High Energy Physics Division, Argonne National Laboratory, Argonne IL, United States of America\\
$^{7}$ Department of Physics, University of Arizona, Tucson AZ, United States of America\\
$^{8}$ Department of Physics, The University of Texas at Arlington, Arlington TX, United States of America\\
$^{9}$ Physics Department, National and Kapodistrian University of Athens, Athens, Greece\\
$^{10}$ Physics Department, National Technical University of Athens, Zografou, Greece\\
$^{11}$ Department of Physics, The University of Texas at Austin, Austin TX, United States of America\\
$^{12}$ Institute of Physics, Azerbaijan Academy of Sciences, Baku, Azerbaijan\\
$^{13}$ Institut de F{\'\i}sica d'Altes Energies (IFAE), The Barcelona Institute of Science and Technology, Barcelona, Spain\\
$^{14}$ Institute of Physics, University of Belgrade, Belgrade, Serbia\\
$^{15}$ Department for Physics and Technology, University of Bergen, Bergen, Norway\\
$^{16}$ Physics Division, Lawrence Berkeley National Laboratory and University of California, Berkeley CA, United States of America\\
$^{17}$ Department of Physics, Humboldt University, Berlin, Germany\\
$^{18}$ Albert Einstein Center for Fundamental Physics and Laboratory for High Energy Physics, University of Bern, Bern, Switzerland\\
$^{19}$ School of Physics and Astronomy, University of Birmingham, Birmingham, United Kingdom\\
$^{20}$ $^{(a)}$ Department of Physics, Bogazici University, Istanbul; $^{(b)}$ Department of Physics Engineering, Gaziantep University, Gaziantep; $^{(d)}$ Istanbul Bilgi University, Faculty of Engineering and Natural Sciences, Istanbul; $^{(e)}$ Bahcesehir University, Faculty of Engineering and Natural Sciences, Istanbul, Turkey\\
$^{21}$ Centro de Investigaciones, Universidad Antonio Narino, Bogota, Colombia\\
$^{22}$ $^{(a)}$ INFN Sezione di Bologna; $^{(b)}$ Dipartimento di Fisica e Astronomia, Universit{\`a} di Bologna, Bologna, Italy\\
$^{23}$ Physikalisches Institut, University of Bonn, Bonn, Germany\\
$^{24}$ Department of Physics, Boston University, Boston MA, United States of America\\
$^{25}$ Department of Physics, Brandeis University, Waltham MA, United States of America\\
$^{26}$ $^{(a)}$ Universidade Federal do Rio De Janeiro COPPE/EE/IF, Rio de Janeiro; $^{(b)}$ Electrical Circuits Department, Federal University of Juiz de Fora (UFJF), Juiz de Fora; $^{(c)}$ Federal University of Sao Joao del Rei (UFSJ), Sao Joao del Rei; $^{(d)}$ Instituto de Fisica, Universidade de Sao Paulo, Sao Paulo, Brazil\\
$^{27}$ Physics Department, Brookhaven National Laboratory, Upton NY, United States of America\\
$^{28}$ $^{(a)}$ Transilvania University of Brasov, Brasov; $^{(b)}$ Horia Hulubei National Institute of Physics and Nuclear Engineering, Bucharest; $^{(c)}$ Department of Physics, Alexandru Ioan Cuza University of Iasi, Iasi; $^{(d)}$ National Institute for Research and Development of Isotopic and Molecular Technologies, Physics Department, Cluj Napoca; $^{(e)}$ University Politehnica Bucharest, Bucharest; $^{(f)}$ West University in Timisoara, Timisoara, Romania\\
$^{29}$ Departamento de F{\'\i}sica, Universidad de Buenos Aires, Buenos Aires, Argentina\\
$^{30}$ Cavendish Laboratory, University of Cambridge, Cambridge, United Kingdom\\
$^{31}$ Department of Physics, Carleton University, Ottawa ON, Canada\\
$^{32}$ CERN, Geneva, Switzerland\\
$^{33}$ Enrico Fermi Institute, University of Chicago, Chicago IL, United States of America\\
$^{34}$ $^{(a)}$ Departamento de F{\'\i}sica, Pontificia Universidad Cat{\'o}lica de Chile, Santiago; $^{(b)}$ Departamento de F{\'\i}sica, Universidad T{\'e}cnica Federico Santa Mar{\'\i}a, Valpara{\'\i}so, Chile\\
$^{35}$ $^{(a)}$ Institute of High Energy Physics, Chinese Academy of Sciences, Beijing; $^{(b)}$ Department of Physics, Nanjing University, Jiangsu; $^{(c)}$ Physics Department, Tsinghua University, Beijing 100084; $^{(d)}$ University of Chinese Academy of Science (UCAS), Beijing, China\\
$^{36}$ $^{(a)}$ Department of Modern Physics and State Key Laboratory of Particle Detection and Electronics, University of Science and Technology of China, Anhui; $^{(b)}$ School of Physics, Shandong University, Shandong; $^{(c)}$ Department of Physics and Astronomy, Key Laboratory for Particle Physics, Astrophysics and Cosmology, Ministry of Education; Shanghai Key Laboratory for Particle Physics and Cosmology, Shanghai Jiao Tong University, Shanghai(also at PKU-CHEP), China\\
$^{37}$ Universit{\'e} Clermont Auvergne, CNRS/IN2P3, LPC, Clermont-Ferrand, France\\
$^{38}$ Nevis Laboratory, Columbia University, Irvington NY, United States of America\\
$^{39}$ Niels Bohr Institute, University of Copenhagen, Kobenhavn, Denmark\\
$^{40}$ $^{(a)}$ INFN Gruppo Collegato di Cosenza, Laboratori Nazionali di Frascati; $^{(b)}$ Dipartimento di Fisica, Universit{\`a} della Calabria, Rende, Italy\\
$^{41}$ $^{(a)}$ AGH University of Science and Technology, Faculty of Physics and Applied Computer Science, Krakow; $^{(b)}$ Marian Smoluchowski Institute of Physics, Jagiellonian University, Krakow, Poland\\
$^{42}$ Institute of Nuclear Physics Polish Academy of Sciences, Krakow, Poland\\
$^{43}$ Physics Department, Southern Methodist University, Dallas TX, United States of America\\
$^{44}$ Physics Department, University of Texas at Dallas, Richardson TX, United States of America\\
$^{45}$ DESY, Hamburg and Zeuthen, Germany\\
$^{46}$ Lehrstuhl f{\"u}r Experimentelle Physik IV, Technische Universit{\"a}t Dortmund, Dortmund, Germany\\
$^{47}$ Institut f{\"u}r Kern-{~}und Teilchenphysik, Technische Universit{\"a}t Dresden, Dresden, Germany\\
$^{48}$ Department of Physics, Duke University, Durham NC, United States of America\\
$^{49}$ SUPA - School of Physics and Astronomy, University of Edinburgh, Edinburgh, United Kingdom\\
$^{50}$ INFN e Laboratori Nazionali di Frascati, Frascati, Italy\\
$^{51}$ Fakult{\"a}t f{\"u}r Mathematik und Physik, Albert-Ludwigs-Universit{\"a}t, Freiburg, Germany\\
$^{52}$ Departement  de Physique Nucleaire et Corpusculaire, Universit{\'e} de Gen{\`e}ve, Geneva, Switzerland\\
$^{53}$ $^{(a)}$ INFN Sezione di Genova; $^{(b)}$ Dipartimento di Fisica, Universit{\`a} di Genova, Genova, Italy\\
$^{54}$ $^{(a)}$ E. Andronikashvili Institute of Physics, Iv. Javakhishvili Tbilisi State University, Tbilisi; $^{(b)}$ High Energy Physics Institute, Tbilisi State University, Tbilisi, Georgia\\
$^{55}$ II Physikalisches Institut, Justus-Liebig-Universit{\"a}t Giessen, Giessen, Germany\\
$^{56}$ SUPA - School of Physics and Astronomy, University of Glasgow, Glasgow, United Kingdom\\
$^{57}$ II Physikalisches Institut, Georg-August-Universit{\"a}t, G{\"o}ttingen, Germany\\
$^{58}$ Laboratoire de Physique Subatomique et de Cosmologie, Universit{\'e} Grenoble-Alpes, CNRS/IN2P3, Grenoble, France\\
$^{59}$ Laboratory for Particle Physics and Cosmology, Harvard University, Cambridge MA, United States of America\\
$^{60}$ $^{(a)}$ Kirchhoff-Institut f{\"u}r Physik, Ruprecht-Karls-Universit{\"a}t Heidelberg, Heidelberg; $^{(b)}$ Physikalisches Institut, Ruprecht-Karls-Universit{\"a}t Heidelberg, Heidelberg, Germany\\
$^{61}$ Faculty of Applied Information Science, Hiroshima Institute of Technology, Hiroshima, Japan\\
$^{62}$ $^{(a)}$ Department of Physics, The Chinese University of Hong Kong, Shatin, N.T., Hong Kong; $^{(b)}$ Department of Physics, The University of Hong Kong, Hong Kong; $^{(c)}$ Department of Physics and Institute for Advanced Study, The Hong Kong University of Science and Technology, Clear Water Bay, Kowloon, Hong Kong, China\\
$^{63}$ Department of Physics, National Tsing Hua University, Taiwan, Taiwan\\
$^{64}$ Department of Physics, Indiana University, Bloomington IN, United States of America\\
$^{65}$ Institut f{\"u}r Astro-{~}und Teilchenphysik, Leopold-Franzens-Universit{\"a}t, Innsbruck, Austria\\
$^{66}$ University of Iowa, Iowa City IA, United States of America\\
$^{67}$ Department of Physics and Astronomy, Iowa State University, Ames IA, United States of America\\
$^{68}$ Joint Institute for Nuclear Research, JINR Dubna, Dubna, Russia\\
$^{69}$ KEK, High Energy Accelerator Research Organization, Tsukuba, Japan\\
$^{70}$ Graduate School of Science, Kobe University, Kobe, Japan\\
$^{71}$ Faculty of Science, Kyoto University, Kyoto, Japan\\
$^{72}$ Kyoto University of Education, Kyoto, Japan\\
$^{73}$ Research Center for Advanced Particle Physics and Department of Physics, Kyushu University, Fukuoka, Japan\\
$^{74}$ Instituto de F{\'\i}sica La Plata, Universidad Nacional de La Plata and CONICET, La Plata, Argentina\\
$^{75}$ Physics Department, Lancaster University, Lancaster, United Kingdom\\
$^{76}$ $^{(a)}$ INFN Sezione di Lecce; $^{(b)}$ Dipartimento di Matematica e Fisica, Universit{\`a} del Salento, Lecce, Italy\\
$^{77}$ Oliver Lodge Laboratory, University of Liverpool, Liverpool, United Kingdom\\
$^{78}$ Department of Experimental Particle Physics, Jo{\v{z}}ef Stefan Institute and Department of Physics, University of Ljubljana, Ljubljana, Slovenia\\
$^{79}$ School of Physics and Astronomy, Queen Mary University of London, London, United Kingdom\\
$^{80}$ Department of Physics, Royal Holloway University of London, Surrey, United Kingdom\\
$^{81}$ Department of Physics and Astronomy, University College London, London, United Kingdom\\
$^{82}$ Louisiana Tech University, Ruston LA, United States of America\\
$^{83}$ Laboratoire de Physique Nucl{\'e}aire et de Hautes Energies, UPMC and Universit{\'e} Paris-Diderot and CNRS/IN2P3, Paris, France\\
$^{84}$ Fysiska institutionen, Lunds universitet, Lund, Sweden\\
$^{85}$ Departamento de Fisica Teorica C-15, Universidad Autonoma de Madrid, Madrid, Spain\\
$^{86}$ Institut f{\"u}r Physik, Universit{\"a}t Mainz, Mainz, Germany\\
$^{87}$ School of Physics and Astronomy, University of Manchester, Manchester, United Kingdom\\
$^{88}$ CPPM, Aix-Marseille Universit{\'e} and CNRS/IN2P3, Marseille, France\\
$^{89}$ Department of Physics, University of Massachusetts, Amherst MA, United States of America\\
$^{90}$ Department of Physics, McGill University, Montreal QC, Canada\\
$^{91}$ School of Physics, University of Melbourne, Victoria, Australia\\
$^{92}$ Department of Physics, The University of Michigan, Ann Arbor MI, United States of America\\
$^{93}$ Department of Physics and Astronomy, Michigan State University, East Lansing MI, United States of America\\
$^{94}$ $^{(a)}$ INFN Sezione di Milano; $^{(b)}$ Dipartimento di Fisica, Universit{\`a} di Milano, Milano, Italy\\
$^{95}$ B.I. Stepanov Institute of Physics, National Academy of Sciences of Belarus, Minsk, Republic of Belarus\\
$^{96}$ Research Institute for Nuclear Problems of Byelorussian State University, Minsk, Republic of Belarus\\
$^{97}$ Group of Particle Physics, University of Montreal, Montreal QC, Canada\\
$^{98}$ P.N. Lebedev Physical Institute of the Russian Academy of Sciences, Moscow, Russia\\
$^{99}$ Institute for Theoretical and Experimental Physics (ITEP), Moscow, Russia\\
$^{100}$ National Research Nuclear University MEPhI, Moscow, Russia\\
$^{101}$ D.V. Skobeltsyn Institute of Nuclear Physics, M.V. Lomonosov Moscow State University, Moscow, Russia\\
$^{102}$ Fakult{\"a}t f{\"u}r Physik, Ludwig-Maximilians-Universit{\"a}t M{\"u}nchen, M{\"u}nchen, Germany\\
$^{103}$ Max-Planck-Institut f{\"u}r Physik (Werner-Heisenberg-Institut), M{\"u}nchen, Germany\\
$^{104}$ Nagasaki Institute of Applied Science, Nagasaki, Japan\\
$^{105}$ Graduate School of Science and Kobayashi-Maskawa Institute, Nagoya University, Nagoya, Japan\\
$^{106}$ $^{(a)}$ INFN Sezione di Napoli; $^{(b)}$ Dipartimento di Fisica, Universit{\`a} di Napoli, Napoli, Italy\\
$^{107}$ Department of Physics and Astronomy, University of New Mexico, Albuquerque NM, United States of America\\
$^{108}$ Institute for Mathematics, Astrophysics and Particle Physics, Radboud University Nijmegen/Nikhef, Nijmegen, Netherlands\\
$^{109}$ Nikhef National Institute for Subatomic Physics and University of Amsterdam, Amsterdam, Netherlands\\
$^{110}$ Department of Physics, Northern Illinois University, DeKalb IL, United States of America\\
$^{111}$ Budker Institute of Nuclear Physics, SB RAS, Novosibirsk, Russia\\
$^{112}$ Department of Physics, New York University, New York NY, United States of America\\
$^{113}$ Ohio State University, Columbus OH, United States of America\\
$^{114}$ Faculty of Science, Okayama University, Okayama, Japan\\
$^{115}$ Homer L. Dodge Department of Physics and Astronomy, University of Oklahoma, Norman OK, United States of America\\
$^{116}$ Department of Physics, Oklahoma State University, Stillwater OK, United States of America\\
$^{117}$ Palack{\'y} University, RCPTM, Olomouc, Czech Republic\\
$^{118}$ Center for High Energy Physics, University of Oregon, Eugene OR, United States of America\\
$^{119}$ LAL, Univ. Paris-Sud, CNRS/IN2P3, Universit{\'e} Paris-Saclay, Orsay, France\\
$^{120}$ Graduate School of Science, Osaka University, Osaka, Japan\\
$^{121}$ Department of Physics, University of Oslo, Oslo, Norway\\
$^{122}$ Department of Physics, Oxford University, Oxford, United Kingdom\\
$^{123}$ $^{(a)}$ INFN Sezione di Pavia; $^{(b)}$ Dipartimento di Fisica, Universit{\`a} di Pavia, Pavia, Italy\\
$^{124}$ Department of Physics, University of Pennsylvania, Philadelphia PA, United States of America\\
$^{125}$ National Research Centre "Kurchatov Institute" B.P.Konstantinov Petersburg Nuclear Physics Institute, St. Petersburg, Russia\\
$^{126}$ $^{(a)}$ INFN Sezione di Pisa; $^{(b)}$ Dipartimento di Fisica E. Fermi, Universit{\`a} di Pisa, Pisa, Italy\\
$^{127}$ Department of Physics and Astronomy, University of Pittsburgh, Pittsburgh PA, United States of America\\
$^{128}$ $^{(a)}$ Laborat{\'o}rio de Instrumenta{\c{c}}{\~a}o e F{\'\i}sica Experimental de Part{\'\i}culas - LIP, Lisboa; $^{(b)}$ Faculdade de Ci{\^e}ncias, Universidade de Lisboa, Lisboa; $^{(c)}$ Department of Physics, University of Coimbra, Coimbra; $^{(d)}$ Centro de F{\'\i}sica Nuclear da Universidade de Lisboa, Lisboa; $^{(e)}$ Departamento de Fisica, Universidade do Minho, Braga; $^{(f)}$ Departamento de Fisica Teorica y del Cosmos, Universidad de Granada, Granada; $^{(g)}$ Dep Fisica and CEFITEC of Faculdade de Ciencias e Tecnologia, Universidade Nova de Lisboa, Caparica, Portugal\\
$^{129}$ Institute of Physics, Academy of Sciences of the Czech Republic, Praha, Czech Republic\\
$^{130}$ Czech Technical University in Prague, Praha, Czech Republic\\
$^{131}$ Charles University, Faculty of Mathematics and Physics, Prague, Czech Republic\\
$^{132}$ State Research Center Institute for High Energy Physics (Protvino), NRC KI, Russia\\
$^{133}$ Particle Physics Department, Rutherford Appleton Laboratory, Didcot, United Kingdom\\
$^{134}$ $^{(a)}$ INFN Sezione di Roma; $^{(b)}$ Dipartimento di Fisica, Sapienza Universit{\`a} di Roma, Roma, Italy\\
$^{135}$ $^{(a)}$ INFN Sezione di Roma Tor Vergata; $^{(b)}$ Dipartimento di Fisica, Universit{\`a} di Roma Tor Vergata, Roma, Italy\\
$^{136}$ $^{(a)}$ INFN Sezione di Roma Tre; $^{(b)}$ Dipartimento di Matematica e Fisica, Universit{\`a} Roma Tre, Roma, Italy\\
$^{137}$ $^{(a)}$ Facult{\'e} des Sciences Ain Chock, R{\'e}seau Universitaire de Physique des Hautes Energies - Universit{\'e} Hassan II, Casablanca; $^{(b)}$ Centre National de l'Energie des Sciences Techniques Nucleaires, Rabat; $^{(c)}$ Facult{\'e} des Sciences Semlalia, Universit{\'e} Cadi Ayyad, LPHEA-Marrakech; $^{(d)}$ Facult{\'e} des Sciences, Universit{\'e} Mohamed Premier and LPTPM, Oujda; $^{(e)}$ Facult{\'e} des sciences, Universit{\'e} Mohammed V, Rabat, Morocco\\
$^{138}$ DSM/IRFU (Institut de Recherches sur les Lois Fondamentales de l'Univers), CEA Saclay (Commissariat {\`a} l'Energie Atomique et aux Energies Alternatives), Gif-sur-Yvette, France\\
$^{139}$ Santa Cruz Institute for Particle Physics, University of California Santa Cruz, Santa Cruz CA, United States of America\\
$^{140}$ Department of Physics, University of Washington, Seattle WA, United States of America\\
$^{141}$ Department of Physics and Astronomy, University of Sheffield, Sheffield, United Kingdom\\
$^{142}$ Department of Physics, Shinshu University, Nagano, Japan\\
$^{143}$ Department Physik, Universit{\"a}t Siegen, Siegen, Germany\\
$^{144}$ Department of Physics, Simon Fraser University, Burnaby BC, Canada\\
$^{145}$ SLAC National Accelerator Laboratory, Stanford CA, United States of America\\
$^{146}$ $^{(a)}$ Faculty of Mathematics, Physics {\&} Informatics, Comenius University, Bratislava; $^{(b)}$ Department of Subnuclear Physics, Institute of Experimental Physics of the Slovak Academy of Sciences, Kosice, Slovak Republic\\
$^{147}$ $^{(a)}$ Department of Physics, University of Cape Town, Cape Town; $^{(b)}$ Department of Physics, University of Johannesburg, Johannesburg; $^{(c)}$ School of Physics, University of the Witwatersrand, Johannesburg, South Africa\\
$^{148}$ $^{(a)}$ Department of Physics, Stockholm University; $^{(b)}$ The Oskar Klein Centre, Stockholm, Sweden\\
$^{149}$ Physics Department, Royal Institute of Technology, Stockholm, Sweden\\
$^{150}$ Departments of Physics {\&} Astronomy and Chemistry, Stony Brook University, Stony Brook NY, United States of America\\
$^{151}$ Department of Physics and Astronomy, University of Sussex, Brighton, United Kingdom\\
$^{152}$ School of Physics, University of Sydney, Sydney, Australia\\
$^{153}$ Institute of Physics, Academia Sinica, Taipei, Taiwan\\
$^{154}$ Department of Physics, Technion: Israel Institute of Technology, Haifa, Israel\\
$^{155}$ Raymond and Beverly Sackler School of Physics and Astronomy, Tel Aviv University, Tel Aviv, Israel\\
$^{156}$ Department of Physics, Aristotle University of Thessaloniki, Thessaloniki, Greece\\
$^{157}$ International Center for Elementary Particle Physics and Department of Physics, The University of Tokyo, Tokyo, Japan\\
$^{158}$ Graduate School of Science and Technology, Tokyo Metropolitan University, Tokyo, Japan\\
$^{159}$ Department of Physics, Tokyo Institute of Technology, Tokyo, Japan\\
$^{160}$ Tomsk State University, Tomsk, Russia\\
$^{161}$ Department of Physics, University of Toronto, Toronto ON, Canada\\
$^{162}$ $^{(a)}$ INFN-TIFPA; $^{(b)}$ University of Trento, Trento, Italy\\
$^{163}$ $^{(a)}$ TRIUMF, Vancouver BC; $^{(b)}$ Department of Physics and Astronomy, York University, Toronto ON, Canada\\
$^{164}$ Faculty of Pure and Applied Sciences, and Center for Integrated Research in Fundamental Science and Engineering, University of Tsukuba, Tsukuba, Japan\\
$^{165}$ Department of Physics and Astronomy, Tufts University, Medford MA, United States of America\\
$^{166}$ Department of Physics and Astronomy, University of California Irvine, Irvine CA, United States of America\\
$^{167}$ $^{(a)}$ INFN Gruppo Collegato di Udine, Sezione di Trieste, Udine; $^{(b)}$ ICTP, Trieste; $^{(c)}$ Dipartimento di Chimica, Fisica e Ambiente, Universit{\`a} di Udine, Udine, Italy\\
$^{168}$ Department of Physics and Astronomy, University of Uppsala, Uppsala, Sweden\\
$^{169}$ Department of Physics, University of Illinois, Urbana IL, United States of America\\
$^{170}$ Instituto de Fisica Corpuscular (IFIC), Centro Mixto Universidad de Valencia - CSIC, Spain\\
$^{171}$ Department of Physics, University of British Columbia, Vancouver BC, Canada\\
$^{172}$ Department of Physics and Astronomy, University of Victoria, Victoria BC, Canada\\
$^{173}$ Department of Physics, University of Warwick, Coventry, United Kingdom\\
$^{174}$ Waseda University, Tokyo, Japan\\
$^{175}$ Department of Particle Physics, The Weizmann Institute of Science, Rehovot, Israel\\
$^{176}$ Department of Physics, University of Wisconsin, Madison WI, United States of America\\
$^{177}$ Fakult{\"a}t f{\"u}r Physik und Astronomie, Julius-Maximilians-Universit{\"a}t, W{\"u}rzburg, Germany\\
$^{178}$ Fakult{\"a}t f{\"u}r Mathematik und Naturwissenschaften, Fachgruppe Physik, Bergische Universit{\"a}t Wuppertal, Wuppertal, Germany\\
$^{179}$ Department of Physics, Yale University, New Haven CT, United States of America\\
$^{180}$ Yerevan Physics Institute, Yerevan, Armenia\\
$^{181}$ Centre de Calcul de l'Institut National de Physique Nucl{\'e}aire et de Physique des Particules (IN2P3), Villeurbanne, France\\
$^{182}$ Academia Sinica Grid Computing, Institute of Physics, Academia Sinica, Taipei, Taiwan\\
$^{a}$ Also at Department of Physics, King's College London, London, United Kingdom\\
$^{b}$ Also at Institute of Physics, Azerbaijan Academy of Sciences, Baku, Azerbaijan\\
$^{c}$ Also at Novosibirsk State University, Novosibirsk, Russia\\
$^{d}$ Also at TRIUMF, Vancouver BC, Canada\\
$^{e}$ Also at Department of Physics {\&} Astronomy, University of Louisville, Louisville, KY, United States of America\\
$^{f}$ Also at Physics Department, An-Najah National University, Nablus, Palestine\\
$^{g}$ Also at Department of Physics, California State University, Fresno CA, United States of America\\
$^{h}$ Also at Department of Physics, University of Fribourg, Fribourg, Switzerland\\
$^{i}$ Also at II Physikalisches Institut, Georg-August-Universit{\"a}t, G{\"o}ttingen, Germany\\
$^{j}$ Also at Departament de Fisica de la Universitat Autonoma de Barcelona, Barcelona, Spain\\
$^{k}$ Also at Departamento de Fisica e Astronomia, Faculdade de Ciencias, Universidade do Porto, Portugal\\
$^{l}$ Also at Tomsk State University, Tomsk, and Moscow Institute of Physics and Technology State University, Dolgoprudny, Russia\\
$^{m}$ Also at The Collaborative Innovation Center of Quantum Matter (CICQM), Beijing, China\\
$^{n}$ Also at Universita di Napoli Parthenope, Napoli, Italy\\
$^{o}$ Also at Institute of Particle Physics (IPP), Canada\\
$^{p}$ Also at Horia Hulubei National Institute of Physics and Nuclear Engineering, Bucharest, Romania\\
$^{q}$ Also at Department of Physics, St. Petersburg State Polytechnical University, St. Petersburg, Russia\\
$^{r}$ Also at Borough of Manhattan Community College, City University of New York, New York City, United States of America\\
$^{s}$ Also at Department of Financial and Management Engineering, University of the Aegean, Chios, Greece\\
$^{t}$ Also at Centre for High Performance Computing, CSIR Campus, Rosebank, Cape Town, South Africa\\
$^{u}$ Also at Louisiana Tech University, Ruston LA, United States of America\\
$^{v}$ Also at Institucio Catalana de Recerca i Estudis Avancats, ICREA, Barcelona, Spain\\
$^{w}$ Also at Department of Physics, The University of Michigan, Ann Arbor MI, United States of America\\
$^{x}$ Also at Graduate School of Science, Osaka University, Osaka, Japan\\
$^{y}$ Also at Fakult{\"a}t f{\"u}r Mathematik und Physik, Albert-Ludwigs-Universit{\"a}t, Freiburg, Germany\\
$^{z}$ Also at Institute for Mathematics, Astrophysics and Particle Physics, Radboud University Nijmegen/Nikhef, Nijmegen, Netherlands\\
$^{aa}$ Also at Department of Physics, The University of Texas at Austin, Austin TX, United States of America\\
$^{ab}$ Also at Institute of Theoretical Physics, Ilia State University, Tbilisi, Georgia\\
$^{ac}$ Also at CERN, Geneva, Switzerland\\
$^{ad}$ Also at Georgian Technical University (GTU),Tbilisi, Georgia\\
$^{ae}$ Also at Ochadai Academic Production, Ochanomizu University, Tokyo, Japan\\
$^{af}$ Also at Manhattan College, New York NY, United States of America\\
$^{ag}$ Also at Departamento de F{\'\i}sica, Pontificia Universidad Cat{\'o}lica de Chile, Santiago, Chile\\
$^{ah}$ Also at The City College of New York, New York NY, United States of America\\
$^{ai}$ Also at School of Physics, Shandong University, Shandong, China\\
$^{aj}$ Also at Departamento de Fisica Teorica y del Cosmos, Universidad de Granada, Granada, Portugal\\
$^{ak}$ Also at Department of Physics, California State University, Sacramento CA, United States of America\\
$^{al}$ Also at Moscow Institute of Physics and Technology State University, Dolgoprudny, Russia\\
$^{am}$ Also at Departement  de Physique Nucleaire et Corpusculaire, Universit{\'e} de Gen{\`e}ve, Geneva, Switzerland\\
$^{an}$ Also at Institut de F{\'\i}sica d'Altes Energies (IFAE), The Barcelona Institute of Science and Technology, Barcelona, Spain\\
$^{ao}$ Also at School of Physics, Sun Yat-sen University, Guangzhou, China\\
$^{ap}$ Also at Institute for Nuclear Research and Nuclear Energy (INRNE) of the Bulgarian Academy of Sciences, Sofia, Bulgaria\\
$^{aq}$ Also at Faculty of Physics, M.V.Lomonosov Moscow State University, Moscow, Russia\\
$^{ar}$ Also at National Research Nuclear University MEPhI, Moscow, Russia\\
$^{as}$ Also at Department of Physics, Stanford University, Stanford CA, United States of America\\
$^{at}$ Also at Institute for Particle and Nuclear Physics, Wigner Research Centre for Physics, Budapest, Hungary\\
$^{au}$ Also at Giresun University, Faculty of Engineering, Turkey\\
$^{av}$ Also at CPPM, Aix-Marseille Universit{\'e} and CNRS/IN2P3, Marseille, France\\
$^{aw}$ Also at Department of Physics, Nanjing University, Jiangsu, China\\
$^{ax}$ Also at Institute of Physics, Academia Sinica, Taipei, Taiwan\\
$^{ay}$ Also at University of Malaya, Department of Physics, Kuala Lumpur, Malaysia\\
$^{az}$ Also at LAL, Univ. Paris-Sud, CNRS/IN2P3, Universit{\'e} Paris-Saclay, Orsay, France\\
$^{*}$ Deceased
\end{flushleft}

%\end{document}
% Created with xml2latex.py

 
\end{document}
