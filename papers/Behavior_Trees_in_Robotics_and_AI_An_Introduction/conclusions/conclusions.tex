
\chapter{Concluding Remarks}
\thispagestyle{empty}

In this book, we have tried to present a broad, unified picture of BTs.
We have covered the classical formulation of BTs, its extensions and its relation to other approaches.
We have provided theoretical results on efficiency, safety and robustness, using a new state space formalism,
as well as estimates on execution time and success probabilities using a stochastic framework.
We have described a number of practical design principles as well as connections between BTs and the important areas of planning and learning.

We believe that modularity is the main reason behind the huge success of BTs in the computer game AI community, and the growing popularity of BTs in robotics.
It is well known that modularity is a key enabler when designing
 complex, maintainable and reusable systems. Clear interfaces 
 reduce dependencies between components and makes development, testing, and reuse much simpler.
 BTs have such  interfaces, as each level of the tree has the same interface as a single action,
 and the internal nodes of the tree makes
 the implementation of an action  independent of the context and order in which the action is to be used. 
 Finally, these simple interfaces provide structures that are equally beneficial for both humans and machines. 
 In fact, they are vital to the ideas of all chapters, from state-space formalism and planning to design principles and machine learning.

 
 



Thus, BTs represent a promising control architecture in both computer game AI and robotics. However, the parallel development in the field has given rise to a set of different formulations and variations on the theme.  This book is an attempt to provide a unified view of a breadth of ideas, algoritms and applications.
There is still lots of work to be done, and we hope the reader has found this book helpful,
and perhaps inspiring, when continuing on the journey towards building better virtual agents and robots.

