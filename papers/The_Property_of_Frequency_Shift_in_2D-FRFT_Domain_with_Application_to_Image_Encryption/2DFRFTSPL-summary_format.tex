%% bare_jrnl.tex
%% V1.4b
%% 2015/08/26
%% by Michael Shell
%% see http://www.michaelshell.org/
%% for current contact information.
%%
%% This is a skeleton file demonstrating the use of IEEEtran.cls
%% (requires IEEEtran.cls version 1.8b or later) with an IEEE
%% journal paper.
%%
%% Support sites:
%% http://www.michaelshell.org/tex/ieeetran/
%% http://www.ctan.org/pkg/ieeetran
%% and
%% http://www.ieee.org/

%%*************************************************************************
%% Legal Notice:
%% This code is offered as-is without any warranty either expressed or
%% implied; without even the implied warranty of MERCHANTABILITY or
%% FITNESS FOR A PARTICULAR PURPOSE!
%% User assumes all risk.
%% In no event shall the IEEE or any contributor to this code be liable for
%% any damages or losses, including, but not limited to, incidental,
%% consequential, or any other damages, resulting from the use or misuse
%% of any information contained here.
%%
%% All comments are the opinions of their respective authors and are not
%% necessarily endorsed by the IEEE.
%%
%% This work is distributed under the LaTeX Project Public License (LPPL)
%% ( http://www.latex-project.org/ ) version 1.3, and may be freely used,
%% distributed and modified. A copy of the LPPL, version 1.3, is included
%% in the base LaTeX documentation of all distributions of LaTeX released
%% 2003/12/01 or later.
%% Retain all contribution notices and credits.
%% ** Modified files should be clearly indicated as such, including  **
%% ** renaming them and changing author support contact information. **
%%*************************************************************************


% *** Authors should verify (and, if needed, correct) their LaTeX system  ***
% *** with the testflow diagnostic prior to trusting their LaTeX platform ***
% *** with production work. The IEEE's font choices and paper sizes can   ***
% *** trigger bugs that do not appear when using other class files.       ***                          ***
% The testflow support page is at:
% http://www.michaelshell.org/tex/testflow/



\documentclass[journal]{IEEEtran}
%
% If IEEEtran.cls has not been installed into the LaTeX system files,
% manually specify the path to it like:
% \documentclass[journal]{../sty/IEEEtran}





% Some very useful LaTeX packages include:
% (uncomment the ones you want to load)


% *** MISC UTILITY PACKAGES ***
%
%\usepackage{ifpdf}
% Heiko Oberdiek's ifpdf.sty is very useful if you need conditional
% compilation based on whether the output is pdf or dvi.
% usage:
% \ifpdf
%   % pdf code
% \else
%   % dvi code
% \fi
% The latest version of ifpdf.sty can be obtained from:
% http://www.ctan.org/pkg/ifpdf
% Also, note that IEEEtran.cls V1.7 and later provides a builtin
% \ifCLASSINFOpdf conditional that works the same way.
% When switching from latex to pdflatex and vice-versa, the compiler may
% have to be run twice to clear warning/error messages.






% *** CITATION PACKAGES ***
%
%\usepackage{cite}
% cite.sty was written by Donald Arseneau
% V1.6 and later of IEEEtran pre-defines the format of the cite.sty package
% \cite{} output to follow that of the IEEE. Loading the cite package will
% result in citation numbers being automatically sorted and properly
% "compressed/ranged". e.g., [1], [9], [2], [7], [5], [6] without using
% cite.sty will become [1], [2], [5]--[7], [9] using cite.sty. cite.sty's
% \cite will automatically add leading space, if needed. Use cite.sty's
% noadjust option (cite.sty V3.8 and later) if you want to turn this off
% such as if a citation ever needs to be enclosed in parenthesis.
% cite.sty is already installed on most LaTeX systems. Be sure and use
% version 5.0 (2009-03-20) and later if using hyperref.sty.
% The latest version can be obtained at:
% http://www.ctan.org/pkg/cite
% The documentation is contained in the cite.sty file itself.






% *** GRAPHICS RELATED PACKAGES ***
%
\ifCLASSINFOpdf
  % \usepackage[pdftex]{graphicx}
  % declare the path(s) where your graphic files are
  % \graphicspath{{../pdf/}{../jpeg/}}
  % and their extensions so you won't have to specify these with
  % every instance of \includegraphics
  % \DeclareGraphicsExtensions{.pdf,.jpeg,.png}
\else
  % or other class option (dvipsone, dvipdf, if not using dvips). graphicx
  % will default to the driver specified in the system graphics.cfg if no
  % driver is specified.
  % \usepackage[dvips]{graphicx}
  % declare the path(s) where your graphic files are
  % \graphicspath{{../eps/}}
  % and their extensions so you won't have to specify these with
  % every instance of \includegraphics
  % \DeclareGraphicsExtensions{.eps}
\fi
% graphicx was written by David Carlisle and Sebastian Rahtz. It is
% required if you want graphics, photos, etc. graphicx.sty is already
% installed on most LaTeX systems. The latest version and documentation
% can be obtained at:
% http://www.ctan.org/pkg/graphicx
% Another good source of documentation is "Using Imported Graphics in
% LaTeX2e" by Keith Reckdahl which can be found at:
% http://www.ctan.org/pkg/epslatex
%
% latex, and pdflatex in dvi mode, support graphics in encapsulated
% postscript (.eps) format. pdflatex in pdf mode supports graphics
% in .pdf, .jpeg, .png and .mps (metapost) formats. Users should ensure
% that all non-photo figures use a vector format (.eps, .pdf, .mps) and
% not a bitmapped formats (.jpeg, .png). The IEEE frowns on bitmapped formats
% which can result in "jaggedy"/blurry rendering of lines and letters as
% well as large increases in file sizes.
%
% You can find documentation about the pdfTeX application at:
% http://www.tug.org/applications/pdftex





% *** MATH PACKAGES ***
%
%\usepackage{amsmath}
% A popular package from the American Mathematical Society that provides
% many useful and powerful commands for dealing with mathematics.
%
% Note that the amsmath package sets \interdisplaylinepenalty to 10000
% thus preventing page breaks from occurring within multiline equations. Use:
%\interdisplaylinepenalty=2500
% after loading amsmath to restore such page breaks as IEEEtran.cls normally
% does. amsmath.sty is already installed on most LaTeX systems. The latest
% version and documentation can be obtained at:
% http://www.ctan.org/pkg/amsmath





% *** SPECIALIZED LIST PACKAGES ***
%
%\usepackage{algorithmic}
% algorithmic.sty was written by Peter Williams and Rogerio Brito.
% This package provides an algorithmic environment fo describing algorithms.
% You can use the algorithmic environment in-text or within a figure
% environment to provide for a floating algorithm. Do NOT use the algorithm
% floating environment provided by algorithm.sty (by the same authors) or
% algorithm2e.sty (by Christophe Fiorio) as the IEEE does not use dedicated
% algorithm float types and packages that provide these will not provide
% correct IEEE style captions. The latest version and documentation of
% algorithmic.sty can be obtained at:
% http://www.ctan.org/pkg/algorithms
% Also of interest may be the (relatively newer and more customizable)
% algorithmicx.sty package by Szasz Janos:
% http://www.ctan.org/pkg/algorithmicx




% *** ALIGNMENT PACKAGES ***
%
%\usepackage{array}
% Frank Mittelbach's and David Carlisle's array.sty patches and improves
% the standard LaTeX2e array and tabular environments to provide better
% appearance and additional user controls. As the default LaTeX2e table
% generation code is lacking to the point of almost being broken with
% respect to the quality of the end results, all users are strongly
% advised to use an enhanced (at the very least that provided by array.sty)
% set of table tools. array.sty is already installed on most systems. The
% latest version and documentation can be obtained at:
% http://www.ctan.org/pkg/array


% IEEEtran contains the IEEEeqnarray family of commands that can be used to
% generate multiline equations as well as matrices, tables, etc., of high
% quality.




% *** SUBFIGURE PACKAGES ***
%\ifCLASSOPTIONcompsoc
%  \usepackage[caption=false,font=normalsize,labelfont=sf,textfont=sf]{subfig}
%\else
%  \usepackage[caption=false,font=footnotesize]{subfig}
%\fi
% subfig.sty, written by Steven Douglas Cochran, is the modern replacement
% for subfigure.sty, the latter of which is no longer maintained and is
% incompatible with some LaTeX packages including fixltx2e. However,
% subfig.sty requires and automatically loads Axel Sommerfeldt's caption.sty
% which will override IEEEtran.cls' handling of captions and this will result
% in non-IEEE style figure/table captions. To prevent this problem, be sure
% and invoke subfig.sty's "caption=false" package option (available since
% subfig.sty version 1.3, 2005/06/28) as this is will preserve IEEEtran.cls
% handling of captions.
% Note that the Computer Society format requires a larger sans serif font
% than the serif footnote size font used in traditional IEEE formatting
% and thus the need to invoke different subfig.sty package options depending
% on whether compsoc mode has been enabled.
%
% The latest version and documentation of subfig.sty can be obtained at:
% http://www.ctan.org/pkg/subfig




% *** FLOAT PACKAGES ***
%
%\usepackage{fixltx2e}
% fixltx2e, the successor to the earlier fix2col.sty, was written by
% Frank Mittelbach and David Carlisle. This package corrects a few problems
% in the LaTeX2e kernel, the most notable of which is that in current
% LaTeX2e releases, the ordering of single and double column floats is not
% guaranteed to be preserved. Thus, an unpatched LaTeX2e can allow a
% single column figure to be placed prior to an earlier double column
% figure.
% Be aware that LaTeX2e kernels dated 2015 and later have fixltx2e.sty's
% corrections already built into the system in which case a warning will
% be issued if an attempt is made to load fixltx2e.sty as it is no longer
% needed.
% The latest version and documentation can be found at:
% http://www.ctan.org/pkg/fixltx2e


%\usepackage{stfloats}
% stfloats.sty was written by Sigitas Tolusis. This package gives LaTeX2e
% the ability to do double column floats at the bottom of the page as well
% as the top. (e.g., "\begin{figure*}[!b]" is not normally possible in
% LaTeX2e). It also provides a command:
%\fnbelowfloat
% to enable the placement of footnotes below bottom floats (the standard
% LaTeX2e kernel puts them above bottom floats). This is an invasive package
% which rewrites many portions of the LaTeX2e float routines. It may not work
% with other packages that modify the LaTeX2e float routines. The latest
% version and documentation can be obtained at:
% http://www.ctan.org/pkg/stfloats
% Do not use the stfloats baselinefloat ability as the IEEE does not allow
% \baselineskip to stretch. Authors submitting work to the IEEE should note
% that the IEEE rarely uses double column equations and that authors should try
% to avoid such use. Do not be tempted to use the cuted.sty or midfloat.sty
% packages (also by Sigitas Tolusis) as the IEEE does not format its papers in
% such ways.
% Do not attempt to use stfloats with fixltx2e as they are incompatible.
% Instead, use Morten Hogholm'a dblfloatfix which combines the features
% of both fixltx2e and stfloats:
%
% \usepackage{dblfloatfix}
% The latest version can be found at:
% http://www.ctan.org/pkg/dblfloatfix




%\ifCLASSOPTIONcaptionsoff
%  \usepackage[nomarkers]{endfloat}
% \let\MYoriglatexcaption\caption
% \renewcommand{\caption}[2][\relax]{\MYoriglatexcaption[#2]{#2}}
%\fi
% endfloat.sty was written by James Darrell McCauley, Jeff Goldberg and
% Axel Sommerfeldt. This package may be useful when used in conjunction with
% IEEEtran.cls'  captionsoff option. Some IEEE journals/societies require that
% submissions have lists of figures/tables at the end of the paper and that
% figures/tables without any captions are placed on a page by themselves at
% the end of the document. If needed, the draftcls IEEEtran class option or
% \CLASSINPUTbaselinestretch interface can be used to increase the line
% spacing as well. Be sure and use the nomarkers option of endfloat to
% prevent endfloat from "marking" where the figures would have been placed
% in the text. The two hack lines of code above are a slight modification of
% that suggested by in the endfloat docs (section 8.4.1) to ensure that
% the full captions always appear in the list of figures/tables - even if
% the user used the short optional argument of \caption[]{}.
% IEEE papers do not typically make use of \caption[]'s optional argument,
% so this should not be an issue. A similar trick can be used to disable
% captions of packages such as subfig.sty that lack options to turn off
% the subcaptions:
% For subfig.sty:
% \let\MYorigsubfloat\subfloat
% \renewcommand{\subfloat}[2][\relax]{\MYorigsubfloat[]{#2}}
% However, the above trick will not work if both optional arguments of
% the \subfloat command are used. Furthermore, there needs to be a
% description of each subfigure *somewhere* and endfloat does not add
% subfigure captions to its list of figures. Thus, the best approach is to
% avoid the use of subfigure captions (many IEEE journals avoid them anyway)
% and instead reference/explain all the subfigures within the main caption.
% The latest version of endfloat.sty and its documentation can obtained at:
% http://www.ctan.org/pkg/endfloat
%
% The IEEEtran \ifCLASSOPTIONcaptionsoff conditional can also be used
% later in the document, say, to conditionally put the References on a
% page by themselves.




% *** PDF, URL AND HYPERLINK PACKAGES ***
%
%\usepackage{url}
% url.sty was written by Donald Arseneau. It provides better support for
% handling and breaking URLs. url.sty is already installed on most LaTeX
% systems. The latest version and documentation can be obtained at:
% http://www.ctan.org/pkg/url
% Basically, \url{my_url_here}.




% *** Do not adjust lengths that control margins, column widths, etc. ***
% *** Do not use packages that alter fonts (such as pslatex).         ***
% There should be no need to do such things with IEEEtran.cls V1.6 and later.
% (Unless specifically asked to do so by the journal or conference you plan
% to submit to, of course. )


% correct bad hyphenation here
\hyphenation{op-tical net-works semi-conduc-tor}
\usepackage{cite}
\usepackage{amsmath,amssymb,amsfonts}
\usepackage{algorithmic}
\usepackage{algorithm}
\usepackage{epsfig}
\usepackage{epstopdf}
\usepackage{graphicx}
\usepackage{textcomp}
\usepackage{xcolor}
\usepackage{lipsum}
\usepackage{stfloats}
\begin{document}
%
% paper title
% Titles are generally capitalized except for words such as a, an, and, as,
% at, but, by, for, in, nor, of, on, or, the, to and up, which are usually
% not capitalized unless they are the first or last word of the title.
% Linebreaks \\ can be used within to get better formatting as desired.
% Do not put math or special symbols in the title.
\title{The Property of Frequency Shift in 2D-FRFT Domain with Application to Image Encryption}
% author names and IEEE memberships
% note positions of commas and nonbreaking spaces ( ~ ) LaTeX will not break
% a structure at a ~ so this keeps an author's name from being broken across
% two lines.
% use \thanks{} to gain access to the first footnote area
% a separate \thanks must be used for each paragraph as LaTeX2e's \thanks
% was not built to handle multiple paragraphs
%

\author{Lei Gao,~\IEEEmembership{Member,~IEEE,}
        Lin Qi,~\IEEEmembership{Member,~IEEE,}
        Ling Guan,~\IEEEmembership{Fellow,~IEEE}% <-this % stops a space
\thanks{L. Gao and L. Guan are with the Department of Electrical, Computer and Biomedical Engineering, Ryerson University, Toronto, ON M5B 2K3, Canada (email:iegaolei@gmail.com; lguan@ee.ryerson.ca).}% <-this % stops a space
\thanks{L. Qi is with the School of Information Engineering, Zhengzhou University, Zhengzhou, China (email:ielqi@zzu.edu.cn).}}
%\thanks{Manuscript received April 19, 2005; revised August 26, 2015.}}

% note the % following the last \IEEEmembership and also \thanks -
% these prevent an unwanted space from occurring between the last author name
% and the end of the author line. i.e., if you had this:
%
% \author{....lastname \thanks{...} \thanks{...} }
%                     ^------------^------------^----Do not want these spaces!
%
% a space would be appended to the last name and could cause every name on that
% line to be shifted left slightly. This is one of those "LaTeX things". For
% instance, "\textbf{A} \textbf{B}" will typeset as "A B" not "AB". To get
% "AB" then you have to do: "\textbf{A}\textbf{B}"
% \thanks is no different in this regard, so shield the last } of each \thanks
% that ends a line with a % and do not let a space in before the next \thanks.
% Spaces after \IEEEmembership other than the last one are OK (and needed) as
% you are supposed to have spaces between the names. For what it is worth,
% this is a minor point as most people would not even notice if the said evil
% space somehow managed to creep in.



% The paper headers
%\markboth{Journal of \LaTeX\ Class Files,~Vol.~14, No.~8, August~2015}%
%{Shell \MakeLowercase{\textit{et al.}}: Bare Demo of IEEEtran.cls for IEEE Journals}
% The only time the second header will appear is for the odd numbered pages
% after the title page when using the twoside option.
%
% *** Note that you probably will NOT want to include the author's ***
% *** name in the headers of peer review papers.                   ***
% You can use \ifCLASSOPTIONpeerreview for conditional compilation here if
% you desire.




% If you want to put a publisher's ID mark on the page you can do it like
% this:
%\IEEEpubid{0000--0000/00\$00.00~\copyright~2015 IEEE}
% Remember, if you use this you must call \IEEEpubidadjcol in the second
% column for its text to clear the IEEEpubid mark.



% use for special paper notices
%\IEEEspecialpapernotice{(Invited Paper)}




% make the title area
\maketitle

% As a general rule, do not put math, special symbols or citations
% in the abstract or keywords.
\begin{abstract}
The Fractional Fourier Transform (FRFT) has been playing a unique and increasingly important role in signal and image processing. In this paper, we investigate the property of frequency shift in two-dimensional FRFT (2D-FRFT) domain. It is shown that the magnitude of image reconstruction from phase information is frequency shift-invariant in 2D-FRFT domain, enhancing the robustness of image encryption, an important multimedia security task. Experiments are conducted to demonstrate the effectiveness of this property against the frequency shift attack, improving the robustness of image encryption.
\end{abstract}
\begin{IEEEkeywords}
Frequency Shift, 2D-FRFT, image encryption.
\end{IEEEkeywords}
% For peer review papers, you can put extra information on the cover
% page as needed:
% \ifCLASSOPTIONpeerreview
% \begin{center} \bfseries EDICS Category: 3-BBND \end{center}
% \fi
%
% For peerreview papers, this IEEEtran command inserts a page break and
% creates the second title. It will be ignored for other modes.
\IEEEpeerreviewmaketitle
\section{Introduction}
% The very first letter is a 2 line initial drop letter followed
% by the rest of the first word in caps.
%
% form to use if the first word consists of a single letter:
% \IEEEPARstart{A}{demo} file is ....
%
% form to use if you need the single drop letter followed by
% normal text (unknown if ever used by the IEEE):
% \IEEEPARstart{A}{}demo file is ....
%
% Some journals put the first two words in caps:
% \IEEEPARstart{T}{his demo} file is ....
%
% Here we have the typical use of a "T" for an initial drop letter
% and "HIS" in caps to complete the first word.
\IEEEPARstart{T}{he} Fourier Transform (FT) is one of the most important analysis tools used in physical optics and signal processing [1-3]. As a generalization of the FT, Fractional Fourier Transform (FRFT) was introduced in 1980 [4-5]. Different from the FT, the FRFT of a signal is flexibly operated at any angle with respect to the time axis on the time-frequency plane, generating a versatile representation for time-frequency distributions (TFDS) of the Cohen class. In fact, the conventional FT is a special case of the FRFT, when the operation angle is 90 degree with respect to the time axis. FRFT provides a powerful tool to analyze signals in the time-frequency domain [6]. The FRFT has since drawn the attention of researchers in the signal processing communities, and fractional operations have been introduced [7-8]. Typical examples include the fractional convolution [9], the fractional correlation [10-11], and the fractional filter [12], which extend the original operations.\\\indent In the Fourier representation of signals, a widely accepted confidence is that amplitude and phase tend to play different roles. Hayes [3] demonstrated that phase was more important than amplitude by reconstructing a multidimensional sequence from the phase part of its FT. In the past couple of decades, extensive related works also have been presented on FRFT. Signal reconstruction from amplitude or phase information of one dimensional FRFT (1D-FRFT) has been extensively investigated. In-depth analysis on amplitude and phase in FRFT domain was presented in [13-15]. It demonstrated that phase played more important roles than amplitude in FRFT domain [13]. Thus, phase retrieval using the FRFT was introduced for image encryption and examination of sensitivities of the various encryption keys [16]. In addition, the methods of multiple-parameter FRFT [17-18] were proposed and applied to the image feature extraction and representation.\\\indent To study the 2D time varying signals, extension of FRFT to two dimensions has also been conducted, in both continuous and discrete domains [19], laying foundations for further investigations in two-dimensional FRFT (2D-FRFT). The properties of spatial shift [20] and rotation invariance [21] in 2D-FRFT were investigated with applications to moving target detection and watermarking respectively. As a result, it is an urgent priority in investigating the characteristics of phase information and amplitude information in 2D-FRFT domain. Nevertheless, as a standing problem, frequency shift can introduce interference into the phase information, leading to poor performance on related applications [22-23]. As far as we know, studies on frequency shift in 2D-FRFT are limited.\\\indent To address the aforementioned issues, in this letter, we present a study of the properties of frequency shift in 2D-FRFT from amplitude and phase information with mathematical verification and computer simulations. The main contributions are summarized as follows.\\
\textbf{1}. It is demonstrated that the magnitude of image reconstruction from phase information is frequency shift-invariant in 2D-FRFT domain while the magnitude of reconstruction from amplitude information does not possess this property.\\
\textbf{2}. In application, we show that the utilization of this property improves robustness of image encryption.\\\indent The remainder of this letter is organized as follows: Section II reviews related work. Section III introduces and verifies the property of frequency shift in 2D-FRFT domain. Section IV presents application examples and Section V draws conclusions.
\section{Related Work}
In this section, we will briefly present the existing fundamentals of FRFT and 2D-FRFT, respectively.
\subsection{FRFT}
The transform of a 1D signal \emph{h}(\emph{t}) by FRFT is written as
\begin{equation}
{H_\alpha }\left( u \right) = \left\{ {{F_\alpha }\left[ {h\left( t \right)} \right]} \right\}\left( u \right) = \int_{ - \infty }^\infty  {h\left( t \right){K_\alpha }\left( {t,u} \right)dt,}
\end{equation}
with the transform kernel $ {{K_\alpha }\left( {t,u} \right)}$, in the following form
\begin{equation}
{K_\alpha }(t,u) =\left\{{\begin{array}{*{20}{c}}{{k_\alpha}\cdot \exp \left( {\begin{array}{*{20}{c}}
{i\frac{{{t^2} + {u^2}}}{2}\cot \alpha  - itu\csc \alpha}  \\
\end{array}} \right), \quad \alpha  \ne n\pi }  \\\
{\delta (t - u),\quad \quad \quad \quad \quad \quad \quad \quad \quad \quad \quad \quad {\rm{  }}\alpha  = 2n\pi }  \\\
{\delta (t + u),\quad \;\quad \quad \quad \quad \quad \quad \quad \quad {\rm{  }}\alpha  = (2n \pm 1)\pi }  \\
\end{array}} \right.
\end{equation}
where ${k_\alpha } = \sqrt {1 - i\cot \alpha /2\pi }$ $ (i = \sqrt {-1})$ and $ \alpha $ is the rotation angle in FRFT.
\subsection{2D-FRFT}
With two rotation angles $\alpha$ and $\beta$, 2D-FRFT provides two degrees of freedom coping with signal and image processing problems. Analytically, the definition of 2D-FRFT to a 2D signal \emph{d}(\emph{s}, \emph{t}) is given as
\begin{equation}
{D_{\alpha ,\beta }}(u,v) = \left\{ {{F_\beta }\left\{ {{F_\alpha }\left[ {d(s,t)} \right]} \right\}(u,t)} \right\}(u,v).
\end{equation}
Let the size of a discrete 2D signal \emph{g(p, q)} be (\emph{P}, \emph{Q}). The forward and inverse 2D-FRFT to a 2D discrete signal \emph{g(p,q)} are expressed as in [19]:
\begin{equation}
{G_{\alpha ,\beta}}(m,n) = \sum\limits_{p = 0}^{P - 1} {\sum\limits_{q = 0}^{Q - 1} {g(p,q)} } {K_{\alpha ,\beta }}(p,q,m,n),
\end{equation}
\begin{equation}
g(p,q) = \sum\limits_{m = 0}^{P - 1} {\sum\limits_{n = 0}^{Q - 1} {{G_{\alpha ,\beta}}(m,n)} } {K_{ - \alpha , - \beta }}(p,q,m,n),
\end{equation}
where ${K_{\alpha ,\beta}}(p,q,m,n)$ and ${K_{-\alpha ,-\beta}}(p,q,m,n)$ are the forward and inverse 2D discrete transform kernels, respectively.
%\begin{figure*}[hb]
%\hrulefill
%\begin{scriptsize}
%\begin{align*} \label{equ:BigWrited LiZi}
%\begin{array}{l}
%\int\limits_{ - \infty }^{ + \infty } {\int\limits_{ - \infty }^{ + \infty } {{\Gamma _{\alpha ,\beta }}(x,y,u,v)\cdot\exp (i2\pi x\delta  + i2\pi y\varepsilon )dx} } dy \\
%= {k_\alpha }{k_\beta }\int\limits_{ - \infty }^{ + \infty } {\int\limits_{ - \infty }^{ + \infty } {\exp \left( {\begin{array}{*{20}{c}}
%   {i\frac{{{x^2} + {u^2}}}{2}\cot \alpha  - ixu\csc \alpha }  \\
%\end{array}} \right)\cdot\exp \left( {\begin{array}{*{20}{c}}
 %  {i\frac{{{y^2} + {v^2}}}{2}\cot \beta  - iyv\csc \beta }  \\
%\end{array}} \right)\cdot f(x,y)\cdot\exp (i2\pi x\delta  + i2\pi y\varepsilon )dx} } dy \\
 % = {k_\alpha }{k_\beta }\int\limits_{ - \infty }^{ + \infty } {\int\limits_{ - \infty }^{ + \infty } {\exp \left( {\begin{array}{*{20}{c}}
  % {i\frac{{{x^2} + {u^2}}}{2}\cot \alpha  - ixu\csc \alpha  + i2\pi x\delta }  \\
%\end{array}} \right)\cdot\exp \left( {\begin{array}{*{20}{c}}
 %  {i\frac{{{y^2} + {v^2}}}{2}\cot \beta  - iyv\csc \beta  + i2\pi y\varepsilon }  \\
%\end{array}} \right)\cdot f(x,y)dx} } dy \\
 % = {k_\alpha }{k_\beta }\int\limits_{ - \infty }^{ + \infty } {\exp \left( {\begin{array}{*{20}{c}}
  % {i\frac{{{y^2} + {v^2}}}{2}\cot \beta  - iyv\csc \beta  + i2\pi y\varepsilon }  \\
%\end{array}} \right)\cdot\{ \int\limits_{ - \infty }^{ + \infty } {\exp \left( {\begin{array}{*{20}{c}}
 %  {i(\frac{{{x^2} + {u^2}}}{2}\cot \alpha  - xu\csc \alpha  + 2\pi x\delta )}  \\
%\end{array}} \right)} f(x,y)dx\} dy}  \\
 %\end{array}
%\tag{15a}
%\end{align*}
%\end{scriptsize}
%\end{figure*}
%\begin{figure*}[hb]
%\begin{scriptsize}
%Let $\zeta (y,v,\varepsilon ,\beta ) = \exp \left( {\begin{array}{*{20}{c}}
%   {i\frac{{{y^2} + {v^2}}}{2}\cot \beta  - iyv\csc \beta  + i2\pi y\varepsilon }  \\
%\end{array}} \right)$ and ${u^*} = u - 2\pi \delta \sin \alpha$, and (15a) is rewritten in equation (15b)
%\begin{align*} \label{}
%\begin{array}{l}
% \int\limits_{ - \infty }^{ + \infty } {\int\limits_{ - \infty }^{ + \infty } {{\Gamma _{\alpha ,\beta }}(x,y,u,v)\cdot\exp (i2\pi x\delta  + i2\pi y\varepsilon )dx} } dy \\
%  = {k_\alpha }{k_\beta }\int\limits_{ - \infty }^{ + \infty } {\zeta (y,v,\varepsilon ,\beta )\cdot\{ \int\limits_{ - \infty }^{ + \infty } {\exp \left( {\begin{array}{*{20}{c}}
%   {i(\frac{{{x^2} + {u^2}}}{2}\cot \alpha  - xu\csc \alpha  + 2\pi x\delta )}  \\
%\end{array}} \right)} f(x,y)dx\} dy}  \\
%  = {k_\alpha }{k_\beta } \cdot \exp [i(2\pi u\delta \cos \alpha  - \pi \sin \alpha \cos \alpha (2\pi ){\delta ^2})]\int\limits_{ - \infty }^{ + \infty } {\zeta (y,v,\varepsilon ,\beta )\cdot\{ \int\limits_{ - \infty }^{ + \infty } {\exp \left( {\begin{array}{*{20}{c}}
 %  {i[(\frac{{{x^2} + {{({u^*})}^2}}}{2}\cot \alpha  - x({u^*})\csc \alpha )]}  \\
%\end{array}} \right)} f(x,y)dx\} dy}
%\end{array}
%\tag{15b}
%\end{align*}
%Since the item outside the integrations does not depend on $x$, we merge it into the second integration, leading to equation (15).
%\end{scriptsize}
%\end{figure*}
\section{Frequency Shift in 2D-FRFT Domain}
\begin{figure*}[t]
\centering
\includegraphics[height=3.6in,width=6.0in]{fig1.eps}\\ Figure. 1 The simulation results on image `Lena'.\\
\end{figure*}
\subsection{Mathematical Derivation}
Again, the 2D-FRFT to a 2D signal \emph{f}(\emph{$x$}, \emph{$y$}) is expressed as
\begin{equation}
{F_{\alpha ,\beta }}(u,v) = \int\limits_{ - \infty }^{ + \infty } {\int\limits_{ - \infty }^{ + \infty } {{K_\alpha }(x,u)\cdot{K_\beta }(y,v)\cdot f(x,y)\cdot dxdy} },
\end{equation}
where $ {{K_\alpha }\left( {x,u} \right)}$ and $ {{K_\beta }\left( {y,v} \right)}$ are transform kernel functions defined in equation (2). Equation (6) is rewritten equivalently as follows
\begin{equation}
\begin{array}{l}
 {F_{\alpha ,\beta }}(u,v) = {A_{\alpha ,\beta }}(u,v)\cdot\exp (i2\pi {\varphi _\alpha }(u) + i2\pi {\varphi _\beta }(v)), \\
 \end{array}
\end{equation}
where ${A_{\alpha ,\beta }}(u,v)$ and $\exp (i2\pi {\varphi _\alpha }(u) + i2\pi {\varphi _\beta }(v))$ represent the amplitude and phase components of equation (6). \\\indent Then, the amplitude part $ {f_{A}(x,y)} $ and the phase part $ {f_{\varphi}(x,y)} $ in the space domain are reconstructed from equation (7) by inverse 2D-FRFT transform [19], and are defined in equations (8) and (9), respectively.
\begin{equation}
{f_{A}}(x,y) = {F_{ - \alpha , - \beta }}({A_{\alpha ,\beta }}(u,v)),
\end{equation}
\begin{equation}
{f_{\varphi}}(x,y) = {F_{ - \alpha , - \beta }}(\exp (i2\pi {\varphi _\alpha }(u) + i2\pi {\varphi _\beta }(v))),
\end{equation}
where $F_{ - \alpha , - \beta }$ is the inverse 2D-FRFT transform with rotation angles $-\alpha$ and $-\beta$.\\\indent Set the horizontal and vertical frequency shifts as $\delta$ and $\varepsilon$, expressing in the forms of $ \exp (i2\pi x\delta ) $ and $\exp (i2\pi y\varepsilon ) $ in 2D-FRFT. The frequency shift operation ${F_{\alpha ,\beta }}^\sim(u,v)$ in 2D-FRFT is in the form of:
\begin{equation}
{F_{\alpha ,\beta }}^\sim(u,v) = \int\limits_{ - \infty }^{ + \infty } {\int\limits_{ - \infty }^{ + \infty } {{\Gamma _{\alpha ,\beta }}\cdot\exp (i2\pi x\delta + i2\pi y\varepsilon )dx} } dy,
\end{equation}
where ${\Gamma _{\alpha ,\beta }} = {K_\alpha }(x,u)\cdot{K_\beta }(y,v)\cdot f(x,y)$. Then, equation (10) can be equivalently written as follows
\begin{equation}
 {F_{\alpha ,\beta }}^\sim(u,v)  = {A_{\alpha ,\beta }}^\sim(u,v)\cdot\exp (i2\pi {\varphi _\alpha }^\sim(u) + i2\pi {\varphi _\beta }^\sim(v)),
\end{equation}
where ${A_{\alpha ,\beta }}^ \sim (u,v)$ and $\exp (i2\pi {\varphi _\alpha }^ \sim (u)+ i2\pi {\varphi _\beta }^ \sim (v))$ represent the amplitude component and phase component of equation (10).\\\indent ${A_{\alpha ,\beta }}^ \sim (u,v)$ is equivalently expressed as follows
\begin{equation}
{A_{\alpha ,\beta }}^\sim(u,v) = \left| {{F_{\alpha ,\beta }}^\sim(u,v)} \right|.
\end{equation}
Using an algebraic operation, equation (12) is further written as seen below
\begin{equation}
\begin{array}{l}
 {A_{\alpha ,\beta }}^\sim(u,v) = \left| {\int\limits_{ - \infty }^{ + \infty } {\int\limits_{ - \infty }^{ + \infty } {{\Gamma _{\alpha ,\beta }}\cdot\exp (i2\pi x\delta  + i2\pi y\varepsilon )dx} } dy} \right| \\
  = {A_{\alpha ,\beta }}(u - 2\pi \delta \sin \alpha ,v - 2\pi \varepsilon \sin \beta ). \\
 \end{array}
\end{equation}
The derivation of equation (13) is given in Appendix \emph{A} of the supporting document.\\\indent
Based on the separability of 2D-FRFT and algebraic operation, the phase parts $\exp (i2\pi {\varphi _\alpha }^ \sim (u))$ and $\exp(i2\pi {\varphi _\beta }^ \sim (v))$ in equation (11) are given as follows
\begin{small}
\begin{equation}
\begin{array}{l}
 \exp (i2\pi {\varphi _\alpha }^\sim(u)) \\
  = \exp \left\{ {i2\pi [{\varphi _\alpha }(u - 2\pi \delta \sin \alpha ) + (u\delta \cos \alpha )] - i\pi \sin \alpha \cos \alpha (2\pi ){\delta ^2}} \right\}, \\
 \end{array}
\end{equation}
\end{small}
\begin{small}
\begin{equation}
\begin{array}{l}
 \exp (i2\pi {\varphi _\beta }^\sim(v)) \\
  = \exp \left\{ {i2\pi [{\varphi _\beta }(v - 2\pi \varepsilon \sin \beta ) + (v\varepsilon \cos \beta )] - i\pi \sin \beta \cos \beta (2\pi ){\varepsilon ^2}} \right\}. \\
 \end{array}
\end{equation}
\end{small}
The derivation of equations (14) and (15) is given in Appendix \emph{B} of the supporting document.\\\indent
Let $\rho$ and $\lambda$ be horizontal and vertical spatial shifts of $f(x,y)$. The spatial shift ${F_{\alpha ,\beta }}^{'}(u,v)$ in 2D-FRFT possesses the following relation [20]:
\begin{equation}
{F_{\alpha ,\beta }}^{'}(u,v) = \int\limits_{ - \infty }^{ + \infty } {\int\limits_{ - \infty }^{ + \infty } {{\kappa _{\alpha ,\beta }}(x,y,u,v) \cdot f(x - \rho ,y - \lambda )dxdy} },
\end{equation}
where ${\kappa _{\alpha ,\beta }}(x,y,u,v) = {K_\alpha }(x,u)\cdot{K_\beta }(y,v)$. Equation (16) is equivalently given in the form of amplitude ${A_{\alpha ,\beta }}^{'}(u,v)$ and phase $\exp (i2\pi {\varphi ^{'}}_\alpha (u) + i2\pi {\varphi ^{'}}_\beta (v))$ in equation (17),
\begin{equation}
\begin{array}{l}
 {F_{\alpha ,\beta }}^{'}(u,v) = {A_{\alpha ,\beta }}^{'}(u,v) \cdot \exp (i2\pi {\varphi ^{'}}_\alpha (u) + i2\pi {\varphi ^{'}}_\beta (v)). \\
 \end{array}
\end{equation}
Using an algebraic method, ${A_{\alpha ,\beta }}^{'}(u,v)$ is equivalently written as follows
\begin{equation}
\begin{array}{l}
 {A_{\alpha ,\beta }}^{'}(u,v) = \left| {\int\limits_{ - \infty }^{ + \infty } {\int\limits_{ - \infty }^{ + \infty } {{\kappa _{\alpha ,\beta }}(x,y,u,v)\cdot f(x - \rho ,y - \lambda )dxdy} } } \right| \\
  = {A_{\alpha ,\beta }}(u - \rho \cos \alpha ,v - \lambda \cos \beta ). \\
 \end{array}
\end{equation}
The derivation of equation (18) is given in Appendix \emph{C} of the supporting document.\\\indent Since 2D-FRFT satisfies the inversed transform in equation (5), implementing the inverse 2D-FRFT w. r. t. $-\alpha$ and $-\beta$ representing the magnitude of the reconstructed amplitude component ${f_{A}}^\sim(x,y)$ from ${A_{\alpha ,\beta }}^ \sim (u,v)$ leads to the following expression,
\begin{equation}
 \left| {{f_{{A^{^\sim}}}}\left( {x,y} \right)} \right| = \left| {{F_{ - \alpha , - \beta }}({A_{\alpha ,\beta }}^\sim(u,v))} \right|.
\end{equation}
By using equation (13), equation (19) is rewritten as follows
\begin{equation}
\begin{array}{l}
 \left| {{f_A}^\sim(x,y)} \right| \\
  = \left| {{F_{ - \alpha , - \beta }}({A_{\alpha ,\beta }}^\sim(u,v))} \right| \\
  = \left| {{F_{ - \alpha , - \beta }}({A_{\alpha ,\beta }}(u - 2\pi \delta \sin \alpha ,v - 2\pi \varepsilon \sin \beta ))} \right|. \\
 \end{array}
\end{equation}
Based on equations (8) and (18), $| {{f_A}^\sim(x,y)}| $ is further expressed as
\begin{equation}
\begin{array}{l}
 \left| {{f_A}^\sim(x,y)} \right| \\
  = \left| {{F_{ - \alpha , - \beta }}({A_{\alpha ,\beta }}(u - 2\pi \delta \sin \alpha ,v - 2\pi \varepsilon \sin \beta ))} \right| \\
  = \left| {{f_A}\left( {x - 2\pi \delta \sin \alpha \cos \alpha ,y - 2\pi \varepsilon \sin \beta \cos \beta } \right)} \right| \\
  = \left| {{f_A}\left( {x - \pi \delta \sin 2\alpha ,y - \pi \varepsilon \sin 2\beta } \right)} \right|. \\
 \end{array}
\end{equation}
Similarly, the magnitude of the reconstructed phase component ${{f_{{\varphi ^\sim}}}(x,y)}$ from $\exp (i2\pi {\varphi _\alpha }^ \sim (u)+ i2\pi {\varphi _\beta }^ \sim (v))$ yields the following expression,
\begin{equation}
\begin{array}{l}
 \left| {{f_{{\varphi ^\sim}}}(x,y)} \right| \\
  = \left| {{F_{ - \alpha , - \beta }}(\exp (i2\pi {\varphi _\alpha }^\sim(u) + i2\pi {\varphi _\beta }^\sim(v)))} \right| \\
  = \left| {{F_{ - \alpha , - \beta }}(\exp (i2\pi {\varphi _\alpha }^\sim(u)) \cdot \exp (i2\pi {\varphi _\beta }^\sim(v)))} \right|. \\
 \end{array}
\end{equation}
Since 2D-FRFT is equivalent to apply FRFT on the two variables successively, mathematical manipulation of (22) yields
\begin{equation}
\begin{array}{l}
 \left| {{f_{{\varphi ^\sim}}}(x,y)} \right| \\
  = \left| {{F_{ - \alpha , - \beta }}(\exp (i2\pi {\varphi _\alpha }^\sim(u)) \cdot \exp (i2\pi {\varphi _\beta }^\sim(v)))} \right| \\
  = \left| {{F_{ - \alpha }}(\exp (i2\pi {\varphi _\alpha }^\sim(u))) \cdot {F_{ - \beta }}(\exp (i2\pi {\varphi _\beta }^\sim(v)))} \right|. \\
 \end{array}
\end{equation}
From equation (23), $|{{F_{ - \alpha }}(\exp (i2\pi {\varphi _\alpha }^\sim(u)))}|$ can be rewritten as follows
\begin{equation}
\begin{array}{l}
 \left| {{F_{ - \alpha }}(\exp (i2\pi {\varphi _\alpha }^\sim(u)))} \right| \\
  = \left| {{F_{ - \alpha }}(\exp \left\{ {i2\pi [{\varphi _\alpha }(u - 2\pi \delta \sin \alpha ) + (u\delta \cos \alpha )]} \right\})} \right| \\
  = \left| {{F_{ - \alpha }}(\exp (i2\pi {\varphi _\alpha }(u - 2\pi \delta \sin \alpha )) \cdot \exp (i2\pi u\delta \cos \alpha ))} \right|. \\
 \end{array}
\end{equation}
By definition,
\begin{equation}
{f_\varphi }(x) = {F_{ - \alpha }}(\exp (i2\pi {\varphi _\alpha }(u))).
\end{equation}
According to the separability of 2D-FRFT and equation (18), $| {{F_{ - \alpha }}(\exp (i2\pi {\varphi _\alpha }(u - 2\pi \delta \sin \alpha )))} | $ is further written as
\begin{equation}
\begin{array}{l}
 \left| {{F_{ - \alpha }}(\exp (i2\pi {\varphi _\alpha }(u - 2\pi \delta \sin \alpha )))} \right| \\
  = \left| {{f_\varphi }(x - 2\pi \delta \sin \alpha \cos ( - \alpha ))} \right| \\
  = \left| {{f_\varphi }(x - 2\pi \delta \sin \alpha \cos \alpha )} \right|. \\
 \end{array}
\end{equation}
Employing equation (13) and substituting (26) into (24) yields equation (27)
\begin{equation}
\begin{array}{l}
 \left| {{F_{ - \alpha }}(\exp (i2\pi {\varphi _\alpha }^\sim(u)))} \right| \\
  = \left| {{F_{ - \alpha }}(\exp (i2\pi {\varphi _\alpha }(u - 2\pi \delta \sin \alpha )) \cdot \exp (i2\pi u\delta \cos \alpha ))} \right| \\
  = \left| {{f_\varphi }(x - 2\pi \delta \sin \alpha \cos \alpha  - 2\pi \delta \cos \alpha \sin ( - \alpha ))} \right| \\
  = \left| {{f_\varphi }(x)} \right|. \\
 \end{array}
\end{equation}
The equivalence between $|{{F_{ - \beta }}(\exp (i2\pi {\varphi _\beta }^\sim(v)))}|$ and $|{{f_\varphi }(y)}|$ can be similarly verified, thus
\begin{equation}
\begin{array}{l}
 \left| {{F_{ - \beta }}(\exp (i2\pi {\varphi _\beta }^\sim(v)))} \right| \\
  = \left| {{F_{ - \beta }}(\exp (i2\pi {\varphi _\beta }(v - 2\pi \varepsilon \sin \beta ))\exp (i2\pi v\varepsilon \cos \beta ))} \right| \\
  = \left| {{F_{ - \beta }}(\exp (i2\pi {\varphi _\beta }(v)))} \right| \\
  = \left| {{f_\varphi }(y)} \right|.\\
 \end{array}
\end{equation}
Substituting equations (27) and (28) into (22) yields,
\begin{equation}
\begin{array}{l}
  \left| {{f_{{\varphi ^\sim}}}(x,y)} \right| \\
 =\left| {{F_{ - \alpha , - \beta }}(\exp (i2\pi {\varphi _\alpha }^\sim(u) + i2\pi {\varphi _\beta }^\sim(v)))} \right| \\
  = \left| {{F_{ - \alpha }}(\exp (i2\pi {\varphi _\alpha }(u)))} \cdot {{F_{ - \beta }}(\exp (i2\pi {\varphi _\beta }(v)))} \right| \\
  = \left|{F_{ - \alpha , - \beta }}(\exp (i2\pi {\varphi _\alpha }(u) + i2\pi {\varphi _\beta }(v)))\right|\\
  = \left| {{f_{\varphi}}(x,y)} \right|. \\
 \end{array}
\end{equation}
Equations (21) and (29) demonstrate that the magnitude of reconstruction from amplitude-only information in 2D-FRFT domain is due to the corresponding shift with respect to the frequency shift operations. Nevertheless, the magnitude of reconstruction from phase-only information in 2D-FRFT domain has no shift at all.\\\indent Moreover, since $x$ and $y$ are integers in the field of digital image processing, $ \exp (i2\pi x\delta ) $ and $\exp (i2\pi y\varepsilon ) $ change into periodic functions and the period is 1 for $\delta$ and $\varepsilon$, respectively. Therefore, during the following computer simulations, $\delta$ and $\varepsilon$ satisfy the relation (30)
\begin{equation}
\left\{ \begin{array}{l}
 \exp (i2\pi x\delta ) = \exp (i2\pi x(\delta  + 1)), \\
 \exp (i2\pi x\varepsilon ) = \exp (i2\pi x(\varepsilon  + 1)). \\
 \end{array} \right.
\end{equation}
\subsection{Simulations}
In this subsection, the impact of frequency shift on the amplitude and phase components in 2D-FRFT is shown by the following computer simulations. During the simulations, the rotation angles are selected as $\alpha$=$\beta$=36$^o$ and frequency shift parameters are set as ($\delta$ = 0.2, $\varepsilon$ = 0) and ($\delta$ = 10.2, $\varepsilon$ = 0) randomly. The simulation results on image `Lena' are illustrated in Fig. 1. From the simulation results, it is observed the phase information is frequency shift-invariant for image reconstruction in 2D-FRFT domain while the amplitude information does not possess this property. Moreover, experimental results on the periodic characteristics of $ \exp (i2\pi x\delta )  (\delta =0.2, \delta =10.2)$ in 2D-FRFT domain ($\alpha$=$\beta$=36$^o$) are shown in Fig. 1 (d) to Fig. 1 (i).
%\centerline{\includegraphics[height=2.0in,width=3.5in]{fig1.eps}}\\ { {Fig. 1 (a) Image `Lena'. (b) (c) Magnitude of image reconstructions from the amplitude and phase of (a), respectively, in 2D-FRFT with ($\alpha$=$\beta$=36$^o$). (d) The shift image of `Lena' ($\delta$ = 0.2, $\varepsilon$ = 0) in 2D-FRFT with ($\alpha$=$\beta$=36$^o$). (e) (f) Magnitude of image reconstructions from the amplitude and phase components of (d), respectively, in 2D-FRFT with ($\alpha$=$\beta$=36$^o$). (g) The shift image of `Lena' ($\delta$ = 10.2, $\varepsilon$ = 0) in 2D-FRFT with ($\alpha$=$\beta$=36$^o$). (h) (i) Magnitude of image reconstructions from the amplitude and phase components of (g), respectively, in 2D-FRFT with ($\alpha$=$\beta$=36$^o$).}}
\section{Applications}
As two degrees of freedom are provided in 2D-FRFT, raising the potential to generate more security [24], 2D-FRFT has been widely applied in the field of image encryption. In this section, we present utilization of the property of frequency shift in 2D-FRFT, which is expected to find applications in the aforementioned fields to improve the robustness.\\\indent Information processing in the encrypted domain has attracted considerable research interests [25-26]. In [27], a double random phase fractional order Fourier domain encoding scheme is proposed for image encryption to enhance the level of security. It demonstrated that the double random phase method is robust against attacks such as occlusion, crop, and so forth [27]. However, there is a chronic issue [28-29] that frequency shift can introduce interference into phase information and decrease the robustness of the double random phase encoding scheme. Since the image reconstruction from phase information satisfies the frequency shift-invariance property, it has potential to extract encryption information/data even when frequency shift attacks exist. In the following experiments, we will select the method of double random phase encoding used in [27] to demonstrate the effectiveness of the frequency shift-invariant property in image encryption.\\\indent In the double random phase encoding method, an independent random function $r(x,y)$ is uniformly distributed in the interval [0 2$\pi$] and the rotation angles are set as $\alpha$=$\beta$=9$^o$ randomly. Then, the method of random phase encoding on a two dimensional signal \emph{$I(x,y)$} is written as follows
\begin{equation}
g(\varsigma ,\eta ) = \int {\int {I(x,y)\cdot exp(2\pi i \cdot r(x,y))\cdot {\bf{\Phi }} \cdot dxdy} },
\end{equation}
\begin{equation}
i=(-1)^{1/2},
\end{equation}
where ${\bf{\Phi }}={K_{\alpha  = {9^o}^,\beta  = {9^o}}}(\varsigma  ,\eta , x , y )$ is the transform kernel function in 2D-FRFT, and the function $g(\varsigma  ,\eta )$ is the encrypted signal.\\\indent Since it satisfies the property of inverse in 2D-FRFT domain, the original signal \emph{$I(x,y)$} can be recovered with the correct independent random functions and rotation angles. Nevertheless, when the frequency shifts $exp(i 2\pi x \delta )$ and $ exp(i 2\pi y \varepsilon)$ are introduced, $\emph{I(x,y)}* exp(i 2\pi x \delta ) * exp(i 2\pi y \varepsilon)* exp(2\pi  * i * r(x,y))$ will replace $\emph{I(x,y)} * exp(2\pi  * i * r(x,y))$ in equation (31), resulting in failures of the encrypted information/data recovery even with the correct independent random functions and rotation angles. However, we can recover the encryption information/data successfully benefiting from the property of frequency shift-invariance from phase information in 2D-FRFT domain.\\\indent Experiments are provided to demonstrate the effectiveness of this property against the frequency shift attack with rotation angles ($\alpha$=$\beta$=9$^o$) in 2D-FRFT domain shown in Fig. 2. In Fig. 2, when there is no frequency shift in Fig. 2(b), the key image is successfully recovered straightforwardly in Fig. 2(c). When there is frequency shift existing in Fig. 2(d), we see the failure without using the frequency shift-invariant property shown in Fig. 2(e), and the success using the property shown in Fig. 2(f). \\\
\centerline{\includegraphics[height=4.2in,width=3.6in]{fig2.eps}}\\ { {Fig. 2 Experiments on image encryption with the frequency shift-invariance property}}
%(a) The original key image. (b) The encrypted key image with double random phase encoding method by the two independent random functions and rotation angles ($\alpha$=$\beta$=9$^o$). (c) The recovered key image from (b). (d) The encrypted key image with double random phase encoding method with frequency shifts ($\delta$ = 34.7, $\varepsilon$ = 0) and rotation angles ($\alpha$=$\beta$=9$^o$). (e) The recovered key image from (d) using correct independent random functions and rotation angles without the frequency shift-invariant property. (f) The recovered key image from (d) using correct independent random functions and rotation angles with the frequency shift-invariant property.
\section{Conclusions}
In this letter, the property of frequency shift operation, from the amplitude and phase information in 2D-FRFT domain, has been studied. It is demonstrated that the magnitude of image reconstruction from phase information is frequency shift-invariant while the property does not hold for the amplitude information. Experiments are provided, illustrating the effectiveness of the property in improving the robustness of image encryption.
%\appendices
%\section{Proof of the First Zonklar Equation}
%Appendix one text goes here.

% you can choose not to have a title for an appendix
% if you want by leaving the argument blank
%\section{}
%Appendix two text goes here.


% use section* for acknowledgment
%\section*{Acknowledgment}


%The authors would like to thank...


% Can use something like this to put references on a page
% by themselves when using endfloat and the captionsoff option.
\ifCLASSOPTIONcaptionsoff
  \newpage
\fi



% trigger a \newpage just before the given reference
% number - used to balance the columns on the last page
% adjust value as needed - may need to be readjusted if
% the document is modified later
%\IEEEtriggeratref{8}
% The "triggered" command can be changed if desired:
%\IEEEtriggercmd{\enlargethispage{-5in}}

% references section

% can use a bibliography generated by BibTeX as a .bbl file
% BibTeX documentation can be easily obtained at:
% http://mirror.ctan.org/biblio/bibtex/contrib/doc/
% The IEEEtran BibTeX style support page is at:
% http://www.michaelshell.org/tex/ieeetran/bibtex/
%\bibliographystyle{IEEEtran}
% argument is your BibTeX string definitions and bibliography database(s)
%\bibliography{IEEEabrv,../bib/paper}
%
% <OR> manually copy in the resultant .bbl file
% set second argument of \begin to the number of references
% (used to reserve space for the reference number labels box)
\begin{thebibliography}{1}


\bibitem{IEEEhowto:kopka}
F. Auger, E. Chassande-Mottin, and P. Flandrin. ``On phase-magnitude relationships in the short-time Fourier transform." \emph{IEEE Signal Processing Letters}, vol. 19, no. 5, pp. 267--270, 2010.
\bibitem{IEEEhowto:kopka}
S. So, and K.K. Paliwal. ``Reconstruction of a signal from the real part of its discrete Fourier transform [tips \& tricks]." \emph{IEEE Signal Processing Magazine}, vol. 35, no. 2, pp. 162--174, 2018.
\bibitem{IEEEhowto:kopka}
M. Hayes. ``The reconstruction of a multidimensional sequence from the phase or magnitude of its fourier transform." \emph{IEEE Transactions on Acoustics, Speech, and Signal Processing}, vol. 30, no. 2, pp. 140--154, 1982.
\bibitem{IEEEhowto:kopka}
V. Namias. ``The fractional order Fourier transform and its application to quantum mechanics." \emph{Journal of Applied Mathematics}, vol. 25, no. 3, pp. 241--265, 1980.
\bibitem{IEEEhowto:kopka}
 A.C. McBride, and F.H. Kerr. ``On Namias's fractional Fourier transforms." \emph{Journal of applied mathematics}, vol. 39, no. 2, pp. 159--175, 1987.
\bibitem{IEEEhowto:kopka}
H.M. Ozaktas, O. Arikan, M.A. Kutay, and G. Bozdagt. ``Digital computation of the fractional Fourier transform." \emph{IEEE Transactions on signal processing}, vol. 44, no. 9, pp. 2141--2150, 1996.
\bibitem{IEEEhowto:kopka}
H. Zhang, T. Shan, S. Liu, and R. Tao. ``Optimized sparse fractional Fourier transform: Principle and performance analysis." \emph{Signal Processing}, vol. 174, pp. 1--12, 2020.
\bibitem{IEEEhowto:kopka}
A. Serbes. ``Compact fractional Fourier domains." \emph{IEEE Signal Processing Letters}, vol. 24, no. 4, pp. 427--431, 2017.
\bibitem{IEEEhowto:kopka}
A.I. Zayed, ``A convolution and product theorem for the fractional Fourier transform." \emph{IEEE Signal processing letters}, vol. 5, no. 4, pp. 101--103, 1998.
\bibitem{IEEEhowto:kopka}
A.I. Zayed. ``On the relationship between the Fourier and fractional Fourier transforms." \emph{IEEE signal processing letters}, vol. 3, no. 12, pp. 310--311, 1996.
\bibitem{IEEEhowto:kopka}
J. Ma, R. Tao, Y. Li, and X. Kang. ``Fractional Power Spectrum and Fractional Correlation Estimations for Nonuniform Sampling." \emph{IEEE Signal Processing Letters (Early Access)}, 2020.
\bibitem{IEEEhowto:kopka}
J. Shi, Y. Chi, and N. Zhang. ``Multichannel sampling and reconstruction of bandlimited signals in fractional Fourier domain." \emph{IEEE Signal Processing Letters}, vol. 17, no. 11, pp. 909--912, 2010.
\bibitem{IEEEhowto:kopka}
S.C. Pei, J.J. Ding, ``Relations between fractional operations and time-frequency distributions and their applications." \emph{IEEE Transactions on Signal Processing}, vol. 49, no. 8, pp. 1638--1655, 2001.
\bibitem{IEEEhowto:kopka}
T. Alieva, M. L. Calvo, ``Importance of the phase and amplitude in the fractional fourier domain." \emph{Journal of the Optical Society of America A}, vol. 20, no. 3, pp. 533--541, 2003.
\bibitem{IEEEhowto:kopka}
T. Alieva, M. Calvo, ``Image reconstruction from amplitude-only and phase-only data in the fractional fourier domain." \emph{Optics and Spectroscopy}, vol.  95, no. 1, pp. 110--113, 2003.
\bibitem{IEEEhowto:kopka}
B. Hennelly, J.T. Sheridan. ``Fractional fourier transform-based image encryption: phase retrieval algorithm." \emph{Optics Communications}, vol. 226, no. 1, pp. 61--80, 2003.
\bibitem{IEEEhowto:kopka}
X. Kang, A. Ming and R. Tao, ``Reality-Preserving Multiple Parameter Discrete Fractional Angular Transform and Its Application to Color Image Encryption." \emph{IEEE Transactions on Circuits and Systems for Video Technology}, vol. 29, no. 6 pp. 1595--1607, 2018.
\bibitem{IEEEhowto:kopka}
X. Kang, F. Zhang, and R. Tao. ``Multichannel random discrete fractional Fourier transform." \emph{IEEE Signal Processing Letters}, vol. 22, no. 9, pp. 1340--1344, 2015.
\bibitem{IEEEhowto:kopka}
S.C. Pei, M.H. Yeh. ``Two dimensional discrete fractional fourier transform." \emph{Signal Processing}, vol. 67, no. 1, pp. 99--108, 1998.
\bibitem{IEEEhowto:kopka}
L. Gao, L. Qi, L. Guan. ``The spatial shift operations on image reconstruction from 2d-frft information with application to sar moving target detection." \emph{2013 IEEE China Summit \& International Conference on Signal and Information Processing (ChinaSIP)}, pp. 523--527, 2013.
\bibitem{IEEEhowto:kopka}
L. Gao, L. Qi, Y. Wang, E. Chen, S. Yang, L. Guan, ``Rotation invariance in 2D-frft with application to digital image watermarking." \emph{Journal of Signal Processing Systems}, vol. 72, no. 2, pp. 133--148, 2013.
\bibitem{IEEEhowto:kopka}
V.M. Kuzkin, and A.A. Lunkov. ``Frequency shifts of sound field maxima in oceanic waveguides." \emph{Acoustical Physics}, vol. 57, no. 5, pp. 667--671, 2011.
\bibitem{IEEEhowto:kopka}
V.M. Kuzkin, and S.A. Pereselkov. ``Methods of measuring frequency shifts in the interference structure of the sound field in oceanic waveguides." \emph{Acoustical Physics}, vol. 56, no. 4 pp. 514--524, 2010.
\bibitem{IEEEhowto:kopka}
X. Yang, and J. Hu. ``Anamorphic fractional Fourier transforms graded index lens designed using transformation optics." \emph{Optics express}, vol. 26, no. 21, pp. 27528--27544, 2018.
\bibitem{IEEEhowto:kopka}
P. Zheng and J. Huang. ``Efficient Encrypted Images Filtering and Transform Coding With Walsh-Hadamard Transform and Parallelization." \emph{IEEE Transactions on Image Processing}, vol. 27, no. 5, pp. 2541--2556, 2018.
\bibitem{IEEEhowto:kopka}
X. Kang, and R. Tao. ``Color image encryption using pixel scrambling operator and reality-preserving MPFRHT." \emph{IEEE Transactions on Circuits and Systems for Video Technology}, vol. 29, no. 7, pp. 1919--1932, 2019.
\bibitem{IEEEhowto:kopka}
N. Rawat, R. Kumar, and B.G. Lee. ``Implementing compressive fractional Fourier transformation with iterative kernel steering regression in double random phase encoding." \emph{Optik}, vol. 125, no. 18, pp. 5414--5417, 2015.
\bibitem{IEEEhowto:kopka}
V. Kuzkin, S. Pereselkov. ``Frequency shifts of the spatial interference structure of the sound field in shallow water." \emph{Acoustical Physics}, vol. 54, no. 3, pp. 375--381, 2008.
\bibitem{IEEEhowto:kopka}
M. Rivera, R. Bizuet, A. Martinez, J. A. Rayas. ``Half-quadratic cost function for computing arbitrary phase shifts and phase: Adaptive out of step phase shifting." \emph{Optics express}, vol. 14, no. 8, pp. 3204--3213, 2008.
\end{thebibliography}

% biography section
%
% If you have an EPS/PDF photo (graphicx package needed) extra braces are
% needed around the contents of the optional argument to biography to prevent
% the LaTeX parser from getting confused when it sees the complicated
% \includegraphics command within an optional argument. (You could create
% your own custom macro containing the \includegraphics command to make things
% simpler here.)
%\begin{IEEEbiography}[{\includegraphics[width=1in,height=1.25in,clip,keepaspectratio]{mshell}}]{Michael Shell}
% or if you just want to reserve a space for a photo:

%\begin{IEEEbiography}{Michael Shell}
%Biography text here.
%\end{IEEEbiography}

% if you will not have a photo at all:
%\begin{IEEEbiographynophoto}{John Doe}
%Biography text here.
%\end{IEEEbiographynophoto}

% insert where needed to balance the two columns on the last page with
% biographies
%\newpage

%\begin{IEEEbiographynophoto}{Jane Doe}
%Biography text here.
%\end{IEEEbiographynophoto}

% You can push biographies down or up by placing
% a \vfill before or after them. The appropriate
% use of \vfill depends on what kind of text is
% on the last page and whether or not the columns
% are being equalized.

%\vfill

% Can be used to pull up biographies so that the bottom of the last one
% is flush with the other column.
%\enlargethispage{-5in}
\end{document}


