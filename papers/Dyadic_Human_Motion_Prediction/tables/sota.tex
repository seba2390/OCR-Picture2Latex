

\begin{table*}[t]
	%\vspace{0.2cm}
	\centering
	\scalebox{1.0}{
		\begin{tabular}{lccccccccccc}
			\toprule
			milliseconds											&100	&200	&300	&400	&500	&600 &700 &800 &900 &1000 &Average  \\ 
			\midrule
			
			
			{TIM~\cite{Lebailly20}}				   &6.06 &12.39 &19.83 &29.35 &41.80 &56.91 &73.17 &89.23 &104.31 &118.20    &51.13 \\
			{MSR-GCN~\cite{Lingwei21}}		&9.02 &17.02 &24.79 &33.26 &43.69 &56.34 &70.49 &85.00  &98.37   &109.73 &51.11  \\	
			{HRI-Itr~\cite{Mao20}}				   &2.21 &4.94 &9.51 &17.71 &30.93 &49.66 &72.95 &98.39 &122.93 &144.24  &50.41\\
			{HRI~\cite{Mao20}}						&5.34 &9.95 &15.08 &22.19 &32.45 &45.82 &61.29 &77.40 &92.47 &105.15    &43.17 \\
			{Ours}		&\textbf{1.31} &\textbf{4.31} &\textbf{9.49} &\textbf{17.33} &\textbf{27.42} &\textbf{39.85} &\textbf{54.22} &\textbf{70.20} &\textbf{86.23} &\textbf{100.09} &\textbf{37.57}\\	
			\bottomrule 
		\end{tabular}
		
	}  \\
	\caption[Comparison of our dyadic motion prediction approach with the state-of-the-art methods on the \lindyhop{} dataset]{\textbf{Comparison of our dyadic motion prediction approach with the state-of-the-art single person methods on the \lindyhop{} dataset.} We present the MPJPE for short-term ($<$ 500ms) and long-term ($>$ 500ms) motion prediction in mm. Despite the fast-paced and nonrepetitive nature of the dance moves, our method outperforms all the baselines for both short-term and long-term prediction. The best results in each column are shown in bold.}
	\label{table:sota_lhop}
\end{table*}