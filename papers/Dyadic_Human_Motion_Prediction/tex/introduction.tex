\section{Introduction}

Forecasting future motion from observed past 3D poses has primarily been studied in a single-person setting~\cite{Li18k,Mao19,Mao20,Lebailly20,Lingwei21}. A naive way to extend these approaches to the multi-person case is to simply treat each subject independently. However, this fails to account for interactions that condition future behavior. Only in~\cite{Adeli20} is there an attempt to capture them via the use of social cues obtained by pooling the learned features  for each individual. While effective in the presence of weak social interactions, this approach is ill-suited to modeling the stronger dependencies that arise from two closely-interacting individuals whose movements are highly correlated.

In this paper, we therefore introduce an approach to dyadic, or pairwise, human motion prediction that more strongly models interactions. To this end, we develop an encoder-decoder architecture with both self- and pairwise attention modules. While self-attention captures the similarities between someone's present and past motions, pairwise attention models capture the dependencies between the pose histories of both subjects. Then, for each subject, the decoder takes as input the subject's own self-attention features and the pairwise attention ones, and outputs the future 3D pose sequence.

As there is no dyadic motion prediction benchmark with closely-interacting people, we build the \lindyhop{} dataset. It features Lindy Hop dancers performing  energetic moves, ranging from frenzied kicks to smooth and sophisticated body motions. The dancers synchronize their fast-paced steps with one another and the music. The standard footwork can be followed by infrequent twirls, which make the upcoming pose prediction hard without observing the highly correlated moves of the partner. The motion of one person gives significant clues about infrequent or subtle motion patterns of the other that cannot be easily inferred from the isolated individual motion.

Hence, our contributions are twofold.
%
\begin{compactitem} 

	\item We propose the first 3D motion prediction method that models the dyadic motion dependencies between two subjects.
	 
	\item We introduce a new dance dataset, \lindyhop{}, which consists of videos and 3D human body poses of dancers performing diverse swing motions.
	
\end{compactitem}
%
Our experiments on the \lindyhop{} dataset clearly demonstrate the benefits of our method. It outperforms both the state-of-the-art single person baselines and the use of weaker social cues~\cite{Adeli20}. Our results are especially promising in terms of long-term prediction. The proposed method models the motion dynamics much more reliably than the baselines. Our code and dataset will be made publicly available.
