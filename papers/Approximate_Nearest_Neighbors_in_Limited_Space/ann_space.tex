\documentclass[11pt]{article}

% Packages
\usepackage{amssymb,amsmath,amsthm,sectsty,url}
\usepackage[letterpaper,hmargin=1.0in,vmargin=1.0in]{geometry}
%\usepackage[dvips,colorlinks,linkcolor=blue,citecolor=blue,filecolor=blue,urlcolor=blue]{hyperref}
\usepackage[pdftex,colorlinks,linkcolor=blue,citecolor=blue,filecolor=blue,urlcolor=blue]{hyperref}
%\usepackage[dvips,colorlinks,linkcolor=black,citecolor=black,filecolor=black,urlcolor=black]{hyperref}
%\usepackage[colorlinks,linkcolor=black,citecolor=black,filecolor=black,urlcolor=black]{hyperref}
\usepackage{cleveref}
\usepackage{color}
\usepackage[boxed]{algorithm}
%\usepackage[margin=20pt,font=small,labelfont=bf]{caption}
\pagestyle{plain}

\usepackage{tikz}
\usepackage{graphicx}
\usepackage{nicefrac}
\usepackage{aliascnt}

%Environments
\newtheorem{theorem}{Theorem}[section]
\crefname{theorem}{Theorem}{Theorems}

\newaliascnt{lemma}{theorem}
\newtheorem{lemma}[lemma]{Lemma}
\aliascntresetthe{lemma}
\crefname{lemma}{Lemma}{Lemmas}

\newaliascnt{proposition}{theorem}
\newtheorem{proposition}[proposition]{Proposition}
\aliascntresetthe{proposition}
\crefname{proposition}{Proposition}{Propositions}

\newaliascnt{corollary}{theorem}
\newtheorem{corollary}[corollary]{Corollary}
\aliascntresetthe{corollary}
\crefname{corollary}{Corollary}{Corollaries}

\newaliascnt{fact}{theorem}
\newtheorem{fact}[fact]{Fact}
\aliascntresetthe{fact}
\crefname{fact}{Fact}{Facts}

\newaliascnt{definition}{theorem}
\newtheorem{definition}[definition]{Definition}
\aliascntresetthe{definition}
\crefname{definition}{Definition}{Definitions}

\newaliascnt{remark}{theorem}
\newtheorem{remark}[remark]{Remark}
\aliascntresetthe{remark}
\crefname{remark}{Remark}{Remarks}

\newaliascnt{conjecture}{theorem}
\newtheorem{conjecture}[conjecture]{Conjecture}
\aliascntresetthe{conjecture}
\crefname{conjecture}{Conjecture}{Conjectures}

\newaliascnt{claim}{theorem}
\newtheorem{claim}[claim]{Claim}
\aliascntresetthe{claim}
\crefname{claim}{Claim}{Claims}

\newaliascnt{question}{theorem}
\newtheorem{question}[question]{Question}
\aliascntresetthe{question}
\crefname{question}{Question}{Questions}

\newaliascnt{exercise}{theorem}
\newtheorem{exercise}[exercise]{Exercise}
\aliascntresetthe{exercise}
\crefname{exercise}{Exercise}{Exercises}

\newaliascnt{example}{theorem}
\newtheorem{example}[example]{Example}
\aliascntresetthe{example}
\crefname{example}{Example}{Examples}

\newaliascnt{notation}{theorem}
\newtheorem{notation}[notation]{Notation}
\aliascntresetthe{notation}
\crefname{notation}{Notation}{Notations}

\newaliascnt{problem}{theorem}
\newtheorem{problem}[problem]{Problem}
\aliascntresetthe{problem}
\crefname{problem}{Problem}{Problems}

\newcommand{\aref}[1]{\autoref{#1}}


%General
\newcommand{\abs}[1]{\left|#1\right|} 
\newcommand{\norm}[1]{\lVert#1\rVert}

%Probability
\def\E{\mathbb E}
\newcommand{\Exp}{\mathop{\mathrm E}\displaylimits} % expectation
\newcommand{\Var}{{\bf Var}}
\newcommand{\Cov}{{\bf Cov}}
\newcommand{\1}{{\rm 1\hspace*{-0.4ex}%

\rule{0.1ex}{1.52ex}\hspace*{0.2ex}}}

%Algebraic Structures
\newcommand{\U}{\mathbf U}
\newcommand{\N}{\mathbb N}
\newcommand{\R}{\mathbb R}
\newcommand{\Z}{\mathbb Z}
\newcommand{\C}{\mathbb C}
\newcommand{\F}{\mathbb F}
\newcommand{\GF}{\mathbb{GF}}
\newcommand{\B}{\{ 0,1 \}}
\newcommand{\BM}{\{ -1,1 \}}

% Misc
\newcommand\polylog[1]{\ensuremath{\mathrm{polylog}\left(#1\right)}}
\newcommand\argmin{\ensuremath{\mathrm{argmin}}}



%opening
\title{Approximate Nearest Neighbors in Limited  Space}
\author{
  Piotr Indyk\thanks{\texttt{indyk@mit.edu}} \\
  MIT \\
  \and
  Tal Wagner\thanks{\texttt{talw@mit.edu}} \\
  MIT \\
}

\begin{document}
\maketitle

\begin{abstract}
We consider the $(1+\epsilon)$-approximate nearest neighbor search problem: given a set $X$ of $n$ points in a $d$-dimensional space, build a data structure that, given any query point  $y$,  finds a point $x \in X$ whose distance to $y$ is at most $(1+\epsilon) \min_{x \in X} \|x-y\|$ for an accuracy parameter $\epsilon \in (0,1)$.  Our main result is a data structure that occupies only $O(\epsilon^{-2} n \log(n) \log(1/\epsilon))$
bits of space, assuming all point coordinates are integers in the range  $\{-n^{O(1)} \ldots n^{O(1)}\}$, i.e., the coordinates have $O(\log n)$ bits of precision. This improves over the best previously known space bound of         $O(\epsilon^{-2} n \log(n)^2)$, obtained via the randomized dimensionality reduction method of~\cite{johnson1984extensions}.  We also consider the more general problem of estimating all distances from a collection of query points to all data points $X$, and provide almost tight upper and lower bounds for the space complexity of this problem. 
\end{abstract}

% !TEX root = ../arxiv.tex

Unsupervised domain adaptation (UDA) is a variant of semi-supervised learning \cite{blum1998combining}, where the available unlabelled data comes from a different distribution than the annotated dataset \cite{Ben-DavidBCP06}.
A case in point is to exploit synthetic data, where annotation is more accessible compared to the costly labelling of real-world images \cite{RichterVRK16,RosSMVL16}.
Along with some success in addressing UDA for semantic segmentation \cite{TsaiHSS0C18,VuJBCP19,0001S20,ZouYKW18}, the developed methods are growing increasingly sophisticated and often combine style transfer networks, adversarial training or network ensembles \cite{KimB20a,LiYV19,TsaiSSC19,Yang_2020_ECCV}.
This increase in model complexity impedes reproducibility, potentially slowing further progress.

In this work, we propose a UDA framework reaching state-of-the-art segmentation accuracy (measured by the Intersection-over-Union, IoU) without incurring substantial training efforts.
Toward this goal, we adopt a simple semi-supervised approach, \emph{self-training} \cite{ChenWB11,lee2013pseudo,ZouYKW18}, used in recent works only in conjunction with adversarial training or network ensembles \cite{ChoiKK19,KimB20a,Mei_2020_ECCV,Wang_2020_ECCV,0001S20,Zheng_2020_IJCV,ZhengY20}.
By contrast, we use self-training \emph{standalone}.
Compared to previous self-training methods \cite{ChenLCCCZAS20,Li_2020_ECCV,subhani2020learning,ZouYKW18,ZouYLKW19}, our approach also sidesteps the inconvenience of multiple training rounds, as they often require expert intervention between consecutive rounds.
We train our model using co-evolving pseudo labels end-to-end without such need.

\begin{figure}[t]%
    \centering
    \def\svgwidth{\linewidth}
    \input{figures/preview/bars.pdf_tex}
    \caption{\textbf{Results preview.} Unlike much recent work that combines multiple training paradigms, such as adversarial training and style transfer, our approach retains the modest single-round training complexity of self-training, yet improves the state of the art for adapting semantic segmentation by a significant margin.}
    \label{fig:preview}
\end{figure}

Our method leverages the ubiquitous \emph{data augmentation} techniques from fully supervised learning \cite{deeplabv3plus2018,ZhaoSQWJ17}: photometric jitter, flipping and multi-scale cropping.
We enforce \emph{consistency} of the semantic maps produced by the model across these image perturbations.
The following assumption formalises the key premise:

\myparagraph{Assumption 1.}
Let $f: \mathcal{I} \rightarrow \mathcal{M}$ represent a pixelwise mapping from images $\mathcal{I}$ to semantic output $\mathcal{M}$.
Denote $\rho_{\bm{\epsilon}}: \mathcal{I} \rightarrow \mathcal{I}$ a photometric image transform and, similarly, $\tau_{\bm{\epsilon}'}: \mathcal{I} \rightarrow \mathcal{I}$ a spatial similarity transformation, where $\bm{\epsilon},\bm{\epsilon}'\sim p(\cdot)$ are control variables following some pre-defined density (\eg, $p \equiv \mathcal{N}(0, 1)$).
Then, for any image $I \in \mathcal{I}$, $f$ is \emph{invariant} under $\rho_{\bm{\epsilon}}$ and \emph{equivariant} under $\tau_{\bm{\epsilon}'}$, \ie~$f(\rho_{\bm{\epsilon}}(I)) = f(I)$ and $f(\tau_{\bm{\epsilon}'}(I)) = \tau_{\bm{\epsilon}'}(f(I))$.

\smallskip
\noindent Next, we introduce a training framework using a \emph{momentum network} -- a slowly advancing copy of the original model.
The momentum network provides stable, yet recent targets for model updates, as opposed to the fixed supervision in model distillation \cite{Chen0G18,Zheng_2020_IJCV,ZhengY20}.
We also re-visit the problem of long-tail recognition in the context of generating pseudo labels for self-supervision.
In particular, we maintain an \emph{exponentially moving class prior} used to discount the confidence thresholds for those classes with few samples and increase their relative contribution to the training loss.
Our framework is simple to train, adds moderate computational overhead compared to a fully supervised setup, yet sets a new state of the art on established benchmarks (\cf \cref{fig:preview}).


Along all this paper, when not specified, the homology is understood to be computed with constant coefficients over a ring $\ring$. 
Let $p$ be a prime or $0$ and let $\F$ be a field of characteristic $p$. 
We write $\md$ for the $\ring$-module of Laurent series $\ring[t, t^{-1}] $. 

%We assume $n$ to be an odd integer.

Let $\Ga{n-1}  = \Br_n$ be the classical braid group on $n$ strands and let $\Gb{n}$ be the 
Artin group of type $\mathrm{B}$.
%$\Gb{n} = \Gbn$.

We write $\confn$ for the configuration space of $n$ unordered points in the unitary disk  $\disk:= \{z \in \C | \, |z|  < 1\}$.
% $\conf_n = \confn$, 
A generic element of $\confn$ is an unordered set of $n$ distinct points $\P = \{x_1, \ldots, x_n\} \subset \disk$. In particular $\conf_1 = \disk$. The fundamental group of $\conf_n$ is the classical braid group on $n$ strands $\Ga{n-1}  = \Br_n$ and we recall that the space $\conf_n$ is a $K(\Br_n,1)$ (see \cite{fa_neu_62}).

Given an element $\P \in \confn$, we can consider the set of points $$\surf_n:= \{(z,y) \in  \disk \times \C | y^2 = (z-x_1)\cdots(z-x_n) \}.$$ This is a connected oriented surface with one boundary component for $n$ odd and with two boundary components if $n$ is even. The genus of $\surf_n$ is $g = \frac{n-1}{2}$ for odd $n$ and $g = \frac{n-2}{2}$ for $n$ even.

Hence we define the space
$$
\totsp_n := \{(\P, z,y ) \in \conf_n  \times \disk \times \C| y^2 = (z-x_1)\cdots(z-x_n) \}.
$$
Notice that $E_n$ has a natural projection $\pi:E_n\to \conf_n$ that maps $(\P, y, z) \mapsto \P$. The fiber of $\pi$ is the surface $\surf_n$.

It is natural to consider the complement of the $n$-points set in the disk: $\disk \setminus \P$. We have that $H_1(\disk \setminus \P)$ has rank $n$. The surface $\surf_n$ is a double covering of $\disk$ ramified along $\P$, hence it is natural to identify $\P$ as a subset of $\surf_n$. 
We define $\ddiskP:= \surf_n \setminus \P$ as the double covering of $\disk \setminus \P$ induced by $\surf_n \to \disk$.
Notice that for $n$ odd $H_1(\ddiskP)$ has rank $2n-1$. 

There is a projection 
$\totsp_n \stackrel{p}{\longrightarrow} \conf_n \times \disk 
$ given by $p:(\P,z,y) \mapsto (\P, z)$.
Hence $\totsp_n$ is a double covering of $\conf_n \times \disk$ ramified along the space $\confm{n-1} := \{ (\P, z) \in \conf_n \times \disk | z \in \P\}$. This is the configuration space of $n-1$ unordered distinct points in $\disk$ with one additional distinct marked point. In particular the complement of
$\confm{n-1} \subset \conf_n \times \disk$ is $\confm{n}$, so the complement of
$p^{-1}(\confm{n-1})$ in $\totsp_n$ is a double covering of $\confm{n}$ that we call $\confmd{n}$.
The fundamental group of $\confm{n}$ is the Artin groups $\Gb{n}$. Moreover (see for example \cite{bri_73}) the space $\confm{n}$ is a $K(\Gbn,1)$.

%Ramified double covering of $\C \setminus \P$: $\surf_n$. Notice that $\surf_n$ is an open surface of genus $g= (n-1)/2$ and one boundary component and $H_1(\surf_n)$ has rank $2g = n-1$.

%Configurations space of $n$ points in $\disk$ with an additional market point: $\confm{n} = \confmn$.

%Double covering of $\confmn: \confmd{n} = \confmdn$.

%Double covering of $\confmn$ with ramification points: $\totsp_n$.

%Fibration: $\surf_n \to \totsp_n \stackrel{\pi}{\to} \conf_n$. 
\begin{rem}\label{rem:globalsection}
Notice that the covering $\surf_n \into \totsp_n \stackrel{\pi}{\to} \conf_n$ admits a global section (see Definition \ref{def:section}) and hence $H_*(\totsp_n) = H_*(\totsp_n,\conf_n) \oplus H_*(\conf_n)$ and $$H_i(\totsp_n, \conf_n) = H_{i-1}(\conf_n; H_1(\surf_n)).$$
\end{rem}


We recall that the $\Z$-module $H_1(\surf_n)$  is endowed with a symplectic form given by the cap product. Moreover the action of $\pi_1(\conf_n)$ on $H_1(\surf_n)$  associated to the covering $\pi$ preserves this form. This monodromy representation is induced by the embedding of the braid group  $\pi_1(\conf_n)$  in the mapping class group of the surface $\surf_n$. Such a monodromy representation maps the standard genyerators of the braid groups to Dehn twists and is called geometric monodromy (see \cite{per_van_92, waj_99}).
Hence we can consider $H_1(\surf_n)$ as a $\pi_1(\conf_n)= \Br_n$-representation; we write also $\sym{g} :=H_1(\surf_n)$, where $g =\frac{n-1}{2}$ for $n$ odd, and $g = \frac{n-2}{2}$ for $n$ even.

The braid group $\Br_n = \Ga{n-1}$ maps on the permutation group $\perm_n$ on $n$ letters. Hence the group $\Br_n $ has a natural representation on $\Z^n$ by permuting cohordinates. We write $\st_n$ for this representation of $\Br_n$.




\section{Basic Sketch}\label{sec:sketch}

In this section we describe the basic data structure (generated by Alice) used for all of our results.
% Below we describe Alice's sketching algorithm. Assume w.l.o.g.~that the minimum distance within $X\cup Y$ is $1$ and the diameter is $\Phi$. 
The data structure augments the representation from~\cite{indyk2017near}, which we will now reproduce.
For the sake of readability, the notions from the latter paper (tree construction via hierarchical clustering, centers, ingresses and surrogates) are interleaved with the new ideas introduced in this paper (top-out compression, grid quantization and surrogate hashing).
Proofs in this section are deferred to Appendix~\ref{sec:sketch_proofs}.

%\textcolor{blue}{Changes from the SODA construction are colored in blue.}

\subsection{Hierarchical Clustering Tree}
The sketch consists of an annotated hierarchical clustering tree, which we now describe with our modified ``top-out compression'' step.

\paragraph{Tree construction}
We construct the inter-link hierarchical clustering tree of $X$: In the bottom level (numbered $0$) every point is a singleton cluster, and level $\ell>0$ is formed from level $\ell-1$ by recursively merging any two clusters whose distance is at most $2^\ell$, until no two such clusters are present.
We repeat this until level $\lceil\log(2\sqrt d\Phi)\rceil$, even if all points in $X$ are already joined in one cluster at a lower level.
%Note that every level in the tree induces a partition of $X$ into the clusters associated with the nodes at that level.
The following observation is immediate.
\begin{lemma}\label{lmm:separation}
If $x,x'\in X$ are in different clusters at level $\ell$, then $\norm{x-x'}\geq2^\ell$.
\end{lemma}

\paragraph{Notation}
Let $T^*$ denote the tree.
For every tree node $v$, we denote its level by $\ell(v)$, its associated cluster by $C(v)\subset X$, and its cluster diameter by $\Delta(v)$. For a point $x_i\in X$, let $\mathrm{leaf}(x_i)$ denote the tree leaf whose associated cluster is $\{x_i\}$.


\paragraph{Top-out compression}
The~\emph{degree} of a node in $T^*$ is its number of children.
A~\emph{$1$-path with $k$ edges} in  $T^*$ is a downward path $u_0,u_1,\ldots,u_k$, such that (i) each of the nodes $u_0,\ldots,u_{k-1}$ has degree $1$, (ii) $u_k$ has degree either $0$ or more than $1$, (iii) if $u_0$ is not the root of $T^*$, then its ancestor has degree more than $1$.

For every node $v$ denote $\Lambda(v) := \log(\Delta(v)/(2^{\ell(v)}\epsilon))$. 
If $v$ is the bottom of a $1$-path with more than $\Lambda(v)$ edges, we replace all but the bottom $\Lambda(v)$ edges with a~\emph{long edge}, and annotate it by the length of the path it represents.
More precisely, if the downward $1$-path is $u_0,\ldots,u_k=v$ and $k>\Lambda(v)$, then we connect $u_0$ directly to $u_{k-\Lambda(v)}$ by the long edge, and the nodes $u_1,\ldots,u_{k-\Lambda(v)-1}$ are removed from the tree, and the long edge is annotated with length $k-\Lambda(v)$.

\begin{lemma}\label{lmm:tree_size}
The compressed tree has $O(n\log(1/\epsilon))$ nodes.
\end{lemma}


We henceforth refer only to the compressed tree, and denote it by $T$.
However, for every node $v$ in $T$, $\ell(v)$ continues to denote its level before compression (i.e., the level where the long edges are counted according to their lengths).
We partition $T$ into~\emph{subtrees} by removing the long edges.
Let $\mathcal F(T)$ denote the set of subtrees.

\begin{lemma}\label{lmm:subtree_root}
Let $v$ be the bottom node of a long edge, and $x,x'\in C(v)$. Then $\norm{x-x'}\leq2^{\ell(v)}\epsilon$.
\end{lemma}

\begin{lemma}\label{lmm:subtree_leaf}
Let $u$ be a leaf of a subtree in $\mathcal F(T)$, and $x,x'\in C(u)$. Then $\norm{x-x'}\leq2^{\ell(u)}\epsilon$.
\end{lemma}


\subsection{Surrogates}
The purpose of annotating the tree is to be able to recover a list of~\emph{surrogates} for every point in $X$.
A surrogate is a point whose location approximates $x$.
Since we will need to compare $x$ to a new query point, which is unknown during sketching, we define the surrogates to encompass a certain amount information about the absolute point location, by hashing a coarsened grid quantization of a representative point in each subtree.

\paragraph{Centers}
With every tree node $v$ we associate an index $c(v)\in[n]$ such that $x_{c(v)}\in C(v)$, and we call $x_{c(v)}$ the~\emph{center} of $C(v)$.
The centers are chosen bottom-up in $T$ as follows.
For a leaf $v$, $C(v)$ contains a single point $x_i\in X$, and we set $c(v)=i$. 
For a non-leaf $v$ with children $u_1,\ldots,u_k$, we set $c(v)=\min\{c(u_i):i\in[k]\}$.

\paragraph{Ingresses}
Fix a subtree $T'\in\mathcal F(T)$.
To every node $u$ in $T'$, except the root, we will now assign an~\emph{ingress} node, denoted $\mathrm{in(u)}$.
Intuitively this is a node in the same subtree whose center is close to $u$, and the purpose is to store the location of $u$ by its quantized displacement from that center (whose location will have been already stored, by induction).

We will now assign ingresses to all children of a given node $v$. (Doing this for every $v$ in $T'$ defines ingresses for all nodes in $T'$ except its root.) Let $u_1,\ldots,u_k$ be the children of $v$, and w.l.o.g.~$c(v)=c(u_1)$. Consider the graph $H_v$ whose nodes are $u_1,\ldots,u_k$, and $u_i,u_j$ are neighbors if there are points $x\in C(u_i)$ and $x'\in C(u_j)$ such that $\norm{x-x'}\leq2^{\ell(v)}$. By the tree construction, $H_v$ is connected. We fix an arbitrary spanning tree $\tau(v)$ of $H_v$ which is rooted at $u_1$.

For $u_1$ we set $\mathrm{in}(u_1):=v$. For $u_i$ with $i>1$, let $u_j$ be its (unique) direct ancestor in the tree $\tau(v)$. Let $x\in C(u_j)$ be the closest point to $C(u_i)$ in $C(u_j)$. Note that in $T$ there is a downward path from $u_j$ to $\mathrm{leaf}(x)$. Let $u_x$ be the bottom node in that path that belongs to $T'$. (Equivalently, $u_x$ is the bottom node on that downward path that is reachable from $u$ without traversing a long edge.) We set $\mathrm{in}(u_i):=u_x$.

\paragraph{Grid net quantization}
Assume w.l.o.g.~that $\Phi$ is a power of $2$.
We define a hierarchy of grids aligned with $\{-\Phi \ldots \Phi\}^d$ as follows.
We begin with the single hypercube whose corners are $(\pm\Phi, \ldots, \pm\Phi)^d$.
We generate the next grid by halving along each dimension, and so on.
For every $\gamma>0$, let $\mathcal{N}_\gamma$ be the coarsest grid generated, whose cell side is at most $\gamma/\sqrt{d}$. Note that every cell in $\mathcal{N}_\gamma$ has diameter at most $\gamma$.
For a point $x\in\R^d$, we denote by $\mathcal{N}_\gamma[x]$ the closest corner of the grid cell containing it.

We will rely on the following fact about the intersection size of a grid and a ball; see, for example, \cite{har2012approximate}.

\begin{claim}\label{clm:gridball}
For every $\gamma>0$, the number of points  in $\mathcal N_\gamma$ at distance at most $2\gamma$ from any given point, is at most $O(1)^d$.
\end{claim}


\paragraph{Surrogates}
Fix a subtree $T'\in\mathcal F(T)$. With every node $v$ in $T'$ we will now associate a~\emph{surrogate} $s^*(v)\in\R^d$.
Define the following for every node $v$ in $T'$:
\[
  \gamma(v) = 
  \begin{cases}
  \left(5 + \lceil\frac{\Delta(v)}{2^{\ell(v)}}\rceil\right)^{-1}\cdot\epsilon & \text{if $v$ is a leaf in $T'$,}\\
  \left(5 + \lceil\frac{\Delta(v)}{2^{\ell(v)}}\rceil\right)^{-1} & \text{otherwise.}
  \end{cases}
\]
The surrogates are defined by induction on the ingresses.
%Note that we have defined an ingress node for every node in $T'$ except its root.

Induction base: For the root $v$ of $T'$ we set $s^*(v) := \mathcal N_{2^{\ell(v)}}[x_{c(v)}]$.

Induction step: For a non-root $v$ we denote the quantized displacement of $c(v)$ from its ingress by $\eta(v)=\mathcal N_{\gamma(v)}\left[\frac{\gamma(v)}{2^{\ell(v)}}(x_{c(v)}-s^*(in(v)))\right]$, and set
$s^*(v) := s^*(in(v)) + \frac{2^{\ell(v)}}{\gamma(v)}\cdot\eta(v)$.

\begin{lemma}\label{lmm:surrogates}
For every node $v$, $\norm{x_{c(v)}-s^*(v)}\leq2^{\ell(v)}$.
Furthermore if $v$ is a leaf of a subtree in $\mathcal F(T)$, then $\norm{x_{c(v)}-s^*(v)}\leq2^{\ell(v)}\epsilon$.
\end{lemma}
%\begin{proof}
%By induction on the ingresses.
%In the base case we use that $\norm{x_{c(v)}-s^*(v)}\leq2^{\ell(v)}$ by the choice of grid net.
%The induction step is identical to~\cite{indyk2017near}.
%\end{proof}

\paragraph{Hash functions}
For every level $\ell$ in the tree, we pick a hash function $H_\ell:\mathcal N_{2^\ell}\rightarrow[m]$, from a universal family (\cite{carter1979universal}), where $m=O(1)^d\cdot\log(2\sqrt d\Phi)\cdot q/\delta$.
The $O(1)$ term is the same constant from~\Cref{clm:gridball} above.
For every subtree root $v$, we store its hashed surrogate $H_{\ell(v)}(\mathcal N_{2^{\ell(v)}}[x_{c(v)}])$.
We also store the description of each hash function $H_\ell$ for every level $\ell$.

\subsection{Sketch Size}
The sketch contains the tree $T$, with each node $v$ annotated by its center $c(v)$, ingress $\mathrm{in(u)}$, precision $\gamma(v)$ and quantized displacement $\eta(v)$ (if applicable). For subtree roots we store their hashed surrogate, and for long edges we store their length. We also store the hash functions $\{H_\ell\}$.
\begin{lemma}\label{lmm:sketch_size}
The total sketch size is
\[
  O \left( n\left((d+\log n)\log(1/\epsilon) + \log\log\Phi + \log\frac{q}{\delta}\right) + d\log\Phi \right)
  \;\; \text{bits.}
\]
\end{lemma}
%\begin{proof}
%The sketch of~\cite{indyk2017near} stores the compressed tree $T'$, with each node annotated by its center $c(v)$, ingress $\mathrm{in}(v)$, precision $\gamma(v)$ and quantized displacement $\eta(v)$.
%Every long edge is annotated by its length.
%They show this takes $O\left( n\left((d+\log n)\log(1/\epsilon) + \log\log\Phi\right)\right)$ bits;
%note that by Lemma~\ref{lmm:tree_size}, top-out compression did not effect this bound.
%
%We additionally store the hashed surrogates of subtree roots.
%There are $O(n)$ subtrees,\footnote{By construction, the tree of subtrees in $T$ has no degree-$1$ nodes. Since $T$ has $n$ leaves, there are at most $2n-1$ subtrees.} and each hash takes $\log m$ bits to store, which adds $O(n(d+\log\log\Phi+\log(q/\delta)))$ bits to the above.
%Finally, we store the hash functions $H_\ell$ for every $\ell$.
%The domain of each $H_\ell$ is $N_{2^\ell}$, which is a subset of $\{-\Phi \ldots \Phi\}^d$, and hence $H_\ell$ can be specified by $O(\log(\Phi^d))$ random bits (\cite{carter1979universal}).
%Since we do not require independence between hash functions of different levels, we can use the same random bits for all hash functions, adding a total of $O(d\log\Phi)$ bits to the sketch.
%\end{proof}

As a preprocessing step, Alice can reduce the dimension of her points to $O(\epsilon^{-2}\log(qn/\delta))$ by a Johnson-Lindenstrauss projection.
She then augments the sketch with the projection, in order for Bob to be able to project his points as well.
By~\cite{kane2011almost}, the projection can be stored with $O(\log d + \log(q/\delta)\cdot\log\log((q/\delta)/\epsilon))$ bits.
This yields the sketch size stated in Theorem~\ref{thm:ann_ub}.


\paragraph{Remark} Both the hash functions and the projection map can be sampled using public randomness.
If one is only interested in the communication complexity, one can use the general reduction from public to private randomness due to~\cite{newman1991private}, which replaces the public coins by augmenting $O(\log(nd\Phi))$ bits to the sketch (since Alice's input has size $O(nd\Phi)$ bits).
The bound in~\Cref{thm:ann_ub} then improves to $O\left( n\left(\frac{\log n\cdot\log(1/\epsilon)}{\epsilon^2} + \log\log\Phi + \log\left(\frac{q}{\delta}\right)\right) + \log\Phi \right)$ bits, and the bound in~\Cref{thm:distances_ub} improves to $O\left(\frac{n}{\epsilon^2}\left(\log n\cdot\log(1/\epsilon) + \log(d\Phi)\log\left(\frac{q}{\delta}\right)\right) \right)$ bits.
However, that reduction is non-constructive; we state our bounds so as to describe explicit sketches.

\section{Approximate Nearest Neighbor Search}\label{sec:ann}

We now describe our approximate nearest neighbor search query procedure, and prove~\Cref{thm:ann_ub}.
Suppose Bob wants to report a $(1+\epsilon)$-approximate nearest neighbor in $X$ for a point $y\in Y$.

\paragraph{Algorithm Report Nearest Neighbor:}
\begin{enumerate}
  \item Start at the subtree $T'\in\mathcal F(T)$ that contains the root of $T$.
  \item Recover all surrogates $\{s^*(v):v\in T'\}$, by the subroutine below.
  \item Let $v$ be the leaf of $T'$ that minimizes $\norm{y-s^*(v)}$.
  \item If $v$ is the head of a long edge, recurse on the subtree under that long edge. Otherwise $v$ is a leaf in $T$, and in that case return $c(v)$.
\end{enumerate}

\paragraph{Subroutine Recover Surrogates:}
This is a subroutine that attempts to recover all surrogates $\{s^*(v):v\in T'\}$ in a given subtree $T'\in\mathcal F(T)$, using both Alice's sketch and Bob's point $y$.

Observe that to this end, the only information missing from the sketch is the root surrogate $s^*(r)$, which served as the induction base for defining the rest of the surrogates. The induction steps are fully defined by $\ell(v)$, $\mathrm{in}(v)$, $\gamma(v)$, and $\eta(v)$, which are stored in the sketch for every node $v\neq r$ in the subtree.
The missing root surrogate was defined as $s^*(r)=\mathcal N_{2^{\ell(r)}}[x_{c(r)}]$.
Instead, the sketch stores its hashed value $H_{\ell(r)}(\mathcal N_{2^{\ell(r)}}[x_{c(r)}])$ and the hash function $H_{\ell(r)}$.\footnote{Note that fully storing the root surrogates is prohibitive: $\mathcal N_{2^{\ell(r)}}$ has $\Theta(2\sqrt{d}\Phi/2^{\ell(r)})^d$ cells, hence storing a cell ID takes $\Omega(d\log d)$ bits, and since there can be $\Omega(n)$ subtree roots, this would bring the total sketch size to $\Omega(nd\log d)$.}

The subroutine attempts to reverse the hash.
It enumerates over all points $p\in \mathcal N_{2^{\ell(r)}}$ such that $\norm{p-y}\leq2\cdot2^{\ell(r)}$.
For each $p$ it computes $H_{\ell(r)}(p)$.
If $H_{\ell(r)}(x_{c(r)})=H_{\ell(r)}(p)$ then it sets $s^*(r)=p$ and recovers all surrogates accordingly.
If either no $p$, or more than one $p$, satisfy $H_{\ell(r)}(x_{c(r)})=H_{\ell(r)}(p)$, then it proceeds with $s^*(r)$ set to an arbitrary point (say, the origin in $\R^d$).

\paragraph{Analysis.} Let $r_0,r_1,\ldots$ be the roots of the subtrees traversed on the algorithm.
Note that they reside on a downward path in $T$.

\begin{claim}
$\norm{x_{c(r_0)}-y} \leq 2^{\ell(r_0)}$.
\end{claim}
\begin{proof}
Since $X\cup Y\subset\{-\Phi \ldots \Phi\}^d$, we have $\norm{x_{c(r_0)}-y} \leq 2\sqrt{d}\Phi\leq2^{\lceil\log(2\sqrt d\Phi)\rceil}=2^{\ell(r_0)}$.
%By the tree construction, $\ell(r_0)=\lceil\log(2\sqrt d\Phi)\rceil$, and the claim follows.
\end{proof}

Let $t$ be the smallest such that $r_t$ satisfies $\norm{x_{c(r_t)}-y}>2^{\ell(r_t)}$.
(The algorithm does not identify $t$, but we will use it for the analysis.)

\begin{lemma}\label{lmm:hashes}
With probability $1-\delta/q$, for every $i=0,\ldots,t-1$ simultaneously,
the subroutine recovers $s^*(r_i)$ correctly as $\mathcal N_{2^{\ell(r)}}[x_{c(r)}]$.
(Consequently, all surrogates in the subtree rooted by $r_i$ are also recovered correctly.)
\end{lemma}
\begin{proof}
Fix a subtree $T'\in\mathcal F(T)$ rooted in $r$, that satisfies $\norm{y-x_{c(r)}}\leq2^{\ell(r)}$.
Since $\norm{x_{c(r)}-s^*(r)}\leq2^{\ell(r)}$ (by Lemma~\ref{lmm:surrogates}), we have $\norm{y-s^*(r)}\leq2\cdot2^{\ell(r)}$.
Hence the surrogate recovery subroutine tries $s^*(r)$ as one of the hash pre-image candidates, and will identify that $H_{\ell(r)}(s^*(r))$ matches the hash stored in the sketch.
Furthermore, by~\Cref{clm:gridball}, the number of candidates is at most $O(1)^d$.
Since the range of $H_{\ell(r)}$ has size $m=O(1)^d\cdot\log(2\sqrt d\Phi)\cdot q/\delta$, then with probability $1-\delta/(q\log(2\sqrt d\Phi))$ there are no collisions, and $s^*(r)$ is recovered correctly.
The lemma follows by taking a union bound over the first $t$ subtrees traversed by the algorithm, i.e.~those rooted by $r_i$ for $i=0,1,\ldots,t-1$. Noting that $t$ is upper-bounded by the number of levels in the tree, $\log(2\sqrt d\Phi)$, we get that all the $s^*(r_i)$'s are recovered correctly simultaneously with probability $1-\delta/q$.
\end{proof}

From now on we assume that the event in Lemma~\ref{lmm:hashes} succeeds, meaning in steps $0,1,\ldots,t-1$, the algorithm recovers all surrogates correctly. We henceforth prove that under this event, the algorithm returns a $(1+\epsilon)$-approximate nearest neighbor of $y$.
In what follows, let $x^*\in X$ be a fixed true nearest neighbor of $y$ in $X$.

\begin{lemma}\label{lmm:annrounds}
Let $T'\in\mathcal F(T)$ be a subtree rooted in $r$, such that $x^*\in C(r)$.
Let $v$ a leaf of $T'$ that minimizes $\norm{y-s^*(v)}$.
Then either $x^*\in C(v)$,
or every $z\in C(v)$ is a $(1+O(\epsilon))$-approximate nearest neighbor of $y$.
\end{lemma}
\begin{proof}
Suppose w.l.o.g.~by scaling that $\epsilon<1/6$.
If $x^*\in C(v)$ then we are done. Assume now that $x^*\in C(u)$ for a leaf $u\neq v$ of $T'$.
Let $\ell:=\max\{\ell(v),\ell(u)\}$. We start by showing that $\norm{y-x^*}>\frac{1}{4}\cdot 2^\ell$. Assume by contradiction this is not the case. Since $u$ is a subtree leaf and $x^*\in C(u)$, we have $\norm{x^*-x_{c(u)}}\leq2^{\ell}\epsilon$ by Lemma~\ref{lmm:subtree_leaf}.
We also have $\norm{x_{c(u)}-s^*(u)}\leq2^{\ell}\epsilon$ by Lemma~\ref{lmm:surrogates}. Together, $\norm{y-s^*(u)}\leq(\frac{1}{4}+2\epsilon)2^\ell$. On the other hand, by the triangle inequality,
$\norm{y-s^*(v)} \geq \norm{x^*-x_{c(v)}} - \norm{y-x^*} - \norm{x_{c(v)}-s^*(v)}$.
Noting that $\norm{x^*-x_{c(v)}}\geq2^\ell$ (by Lemma~\ref{lmm:separation}, since $x^*$ and $x_{c(v)}$ are separated at level $\ell$), $\norm{y-x^*}\leq\frac{1}{4}\cdot 2^\ell$ (by the contradiction hypothesis) and $\norm{x_{c(v)}-s^*(v)}\leq2^\ell\epsilon$ (by Lemma~\ref{lmm:surrogates}), we get $\norm{y-s^*(v)}\geq(\frac{3}{4}-\epsilon)2^\ell>(\frac{1}{4}+2\epsilon)2^\ell\geq\norm{y-s^*(u)}$. This contradicts the choice of $v$.

The lemma now follows because for every $z\in C(v)$,
\begin{align}
\norm{y-z} &\leq \norm{y-s^*(v)} + \norm{s^*(v)-x_{c(v)}} + \norm{x_{c(v)}-z} \label{ineq1} \\
&\leq \norm{y-s^*(u)} + \norm{s^*(v)-x_{c(v)}} + \norm{x_{c(v)}-z} \label{ineq2} \\
&\leq \norm{y-x^*} + \norm{x^*-x_{c(u)}} + \norm{x_{c(u)}-s^*(u)} +\norm{s^*(v)-x_{c(v)}} + \norm{x_{c(v)}-z} \label{ineq3} \\
&\leq \norm{y-x^*} + 4\cdot2^\ell\epsilon \label{ineq4} \\
&\leq (1+16\epsilon)\norm{y-x^*}, \label{ineq5}
\end{align}
where~(\ref{ineq1}) and (\ref{ineq3}) are by the triangle inequality, (\ref{ineq2}) is since $\norm{y-s^*(v)}\leq\norm{y-s^*(u)}$ by choice of $v$, (\ref{ineq4}) is by Lemmas~\ref{lmm:subtree_leaf} and~\ref{lmm:surrogates}, and~(\ref{ineq5}) is since we have shown that $\norm{y-x^*}>\frac{1}{4}\cdot 2^\ell$.
Therefore $z$ is a $(1+16\epsilon)$-approximate nearest neighbor of $y$.
\end{proof}

\paragraph{Proof of~\Cref{thm:ann_ub}.}
We may assume w.l.o.g.~that $\epsilon$ is smaller than a sufficiently small constant.
Suppose that the event in Lemma~\ref{lmm:hashes} holds, hence all surrogates in the subtrees rooted by $r_0,r_1,\ldots,r_{t-1}$ are recovered correctly.
We consider two cases. In the first case, $x^*\notin C(r_t)$.
Let $i\in\{1,\ldots,t\}$ be the smallest such that $x^*\notin C(r_i)$.
By applying Lemma~\ref{lmm:annrounds} on $r_{i-1}$, we have that every point in $C(r_i)$ is a $(1+O(\epsilon))$-approximate nearest neighbor of $y$. After reaching $r_i$, the algorithm would return the center of some leaf reachable from $r_i$, and it would be a correct output.

In the second case, $x^*\in C(r_t)$. We will show that every point in $C(r_t)$ is a $(1+O(\epsilon))$-approximate nearest neighbor of $y$, so once again, once the algorithm arrives at $r_t$ it can return anything.
By Lemma~\ref{lmm:subtree_root}, every $x\in C(r_t)$ satisfies
\begin{equation}\label{eq:ann_endgame}
\norm{x-x^*} \leq  2^{\ell(r_t)}\epsilon .
\end{equation}
In particular, $\norm{x_{c(r_t)}-x^*} \leq 2^{\ell(r_t)}\epsilon$.
By definition of $t$ we have $\norm{x_{c(r_t)}-y}>2^{\ell(r_t)}$.
Combining the two yields $\norm{y-x^*} \geq \norm{y-x_{c(r_t)}} - \norm{x_{c(r_t)}-x^*} > (1-\epsilon)2^{\ell(r_t)}$. 
Combining this with~\cref{eq:ann_endgame}, we find that every $x\in C(r_t)$ satisfies $\norm{x-x^*}\leq\frac{\epsilon}{1-\epsilon}\norm{y-x^*}$, and hence $\norm{y-x}\leq(1+2\epsilon)\norm{y-x^*}$ (for $\epsilon\leq1/2$). Hence $x$ is a $(1+2\epsilon)$-nearest neighbor of $y$.

The proof assumes the event in Lemma~\ref{lmm:hashes}, which occurs with probability $1-\delta/q$.
By a union bound, the simultaneous success probability of the $q$ query points of Bob is $1-\delta$ as required. \qed

%Let $x^*\in C(r_t)$ be a $(1+\epsilon)$-approximate nearest neighbor of $y$. Note that by Lemma~\ref{lmm:annrounds}, such $x^*$ indeed exists in $C(r_t)$. 
%By Lemma~\ref{lmm:subtree_root}, every $x\in C(r_t)$ satisfies
%\begin{equation}\label{eq:ann_endgame}
%\norm{x-x^*} \leq  2^{\ell(r_t)}\epsilon .
%\end{equation}
%In particular, $\norm{x_{c(r_t)}-x^*} \leq 2^{\ell(r_t)}\epsilon$.
%By definition of $t$ we have $\norm{x_{c(r_t)}-y}>2^{\ell(r_t)}$.
%Combining the two yields $\norm{y-x^*} \geq \norm{y-x_{c(r_t)}} - \norm{x_{c(r_t)}-x^*} > (1-\epsilon)2^{\ell(r_t)}$. 
%Combining this with~\cref{eq:ann_endgame}, we find that every $x\in C(r_t)$ satisfies $\norm{x-x^*}\leq\frac{\epsilon}{1-\epsilon}\norm{y-x^*}$, and hence $\norm{y-x}\leq(1+2\epsilon)\norm{y-x^*}$. The fact that $x^*$ is a $(1+\epsilon)$-approximate nearest neighbor of $y$ now implies that $x$ is a $(1+3\epsilon)$-nearest neighbor of $y$.
%
%The proof assumes the event in Lemma~\ref{lmm:hashes}, which occurs with probability $1-\delta/q$.
%By a union bound, the simultaneous success probability of the $q$ query points of Bob is $1-\delta$ as required. \qed



\section{Distance Estimation}\label{sec:dist}
We now prove~\cref{thm:distances_ub}.
To this end, we augment the basic sketch from Section~\ref{sec:sketch} with additional information, relying on the following distance sketches due to~\cite{achlioptas2001database} (following~\cite{johnson1984extensions}) and~\cite{kushilevitz2000efficient}.

\begin{lemma}[\cite{achlioptas2001database}]\label{lmm:binaryjl}
Let $\epsilon,\delta'>0$.
Let $d'=c\epsilon^{-2}\log(1/\delta')$ for a sufficiently large constant $c>0$.
Let $M$ be a random $d'\times d$ matrix in which every entry is chosen independently uniformly at random from $\{-1/\sqrt{d'},1/\sqrt{d'}\}$.
Then for every $x,y\in\R^d$, with probability $1-\delta'$, $\norm{Mx-My}=(1\pm\epsilon)\norm{x-y}$.
\end{lemma}

\begin{lemma}[\cite{kushilevitz2000efficient}]\label{lmm:kor}
Let $R>0$ be fixed and let $\epsilon,\delta'>0$. There is a randomized map $\mathrm{sk}_R$ of vectors in $\R^d$ into $O(\epsilon^{-2}\log(1/\delta'))$ bits, with the following guarantee.
For every $x,y\in\R^d$, given $\mathrm{sk}_R(x)$ and $\mathrm{sk}_R(y)$, one can output the following with probability $1-\delta'$:
\begin{itemize}
  \item If $R\leq\norm{x-y}\leq 2R$, output a $(1+\epsilon)$-estimate of $\norm{x-y}$.
  \item If $\norm{x-y}\leq (1-\epsilon)R$, output ``Small''.
  \item If $\norm{x-y}\geq (1+\epsilon)R$, output ``Large''.  
\end{itemize} 
\end{lemma}

%\begin{lemma}\label{lmm:kor_onescale}
%Let $R>0$ be fixed. There is a randomized map $\mathrm{sk}_R$ of vectors in $\R^d$ into $O(\epsilon^{-2}\log(1/\delta))$ bits, and a boolean predicate $P_R(\cdot,\cdot)$, such that for every $x,y\in\R^d$, with probability $1-\delta$,
%\begin{itemize}
%  \item If $\norm{x-y}<(1-\epsilon)R$ then $P_R(\mathrm{sk}_R(x),\mathrm{sk}_R(y))$ is true.
%  \item If $\norm{x-y}>R$ then $P_R(\mathrm{sk}_R(x),\mathrm{sk}_R(y))$ is false.
%\end{itemize}
%\end{lemma}
%
%\begin{lemma}\label{lmm:kor_multiscale}
%There is a randomized map $\mathrm{sk}^*$ of vectors in $\R^d$ into $O(\epsilon^{-2}\log(1/\delta)\log\Phi)$ bits, and a map $D^*(\cdot,\cdot)$, such that for every $x,y\in\R^d$, with probability $1-\delta$,
%$D^*(\mathrm{sk}^*(x),\mathrm{sk}^*(y))=(1\pm\epsilon)\norm{x-y}$.
%\end{lemma}


We augment the basic sketch from Section~\ref{sec:sketch} as follows.
We sample a matrix $M$ from Lemma~\ref{lmm:binaryjl}, with $\delta'=\delta/q$.
In addition, for every level $\ell$ in the tree $T$, we sample a map $\mathrm{sk}_{2^\ell}$ from Lemma~\ref{lmm:kor}, with $\delta'=\delta/(q\log(2\sqrt{d}\Phi))$.
For every subtree root $r$ in $T$, we store $Mx_{c(r)}$ and $\mathrm{sk}_{2^{\ell(r)}}(x_{c(r)})$ in the sketch.
Let us calculate the added size to the sketch:

\begin{itemize}
  \item Since $x_{c(r)}$ has $d$ coordinates of magnitude $O(\Phi)$ each, $Mx_{c(r)}$ has $d'$ coordinates of magnitude $O(d\Phi)$ each. Since there are $O(n)$ subtree roots (cf.~Lemma~\ref{lmm:sketch_size}), storing $Mx_{c(r)}$ for every $r$ adds $O(nd'd\Phi)=O(\epsilon^{-2}n\log(q/\delta)\log(d\Phi))$ bits to the sketch. In addition we store the matrix $M$, which takes $O(d'd)$ bits to store, which is dominated by the previous term.
  \item By Lemma~\ref{lmm:kor}, each $\mathrm{sk}_{2^{\ell(r)}}(x_{c(r)})$ adds $O(\epsilon^{-2}\log(q\log(2\sqrt{d}\Phi)/\delta))$ bits to the sketch, and as above there are $O(n)$ of these. In addition we store the map $\mathrm{sk}_{2^{\ell(r)}}$ for every $\ell$. Each map takes $\mathrm{poly}(d,\log\Phi,\log(q/\delta),1/\epsilon)$ bits to store.
\end{itemize}
In total, we get the sketch size stated in~\Cref{thm:distances_ub}.
Next we show how to compute all distances from a new query point $y$.

\paragraph{Query algorithm.}
Given the sketch, an index $k\in[n]$ of a point in $X$, and a new query point $y$, the algorithm needs to estimate $\norm{y-x_k}$ up to $1\pm O(\epsilon)$ distortion. It proceeds as follows.
\begin{enumerate}
  \item Perform the approximate nearest neighbor query algorithm from Section~\ref{sec:ann}. Let $r_0,r_1,\ldots$ be the downward sequence of subtree roots traversed by it.
  \item For each $r_j$, estimate from the sketch whether $\norm{y-x_{c(r_j)}}\leq 2^{\ell(r_j)}$.
  This can be done by Lemma~\ref{lmm:kor}, since the sketch stores $\mathrm{sk}_{2^{\ell(r_j)}}(x_{c(r_j)})$ and also the map $\mathrm{sk}_{2^{\ell(r_j)}}$, with which we can compute $\mathrm{sk}_{2^{\ell(r_j)}}(y)$.
  
   \item Let $t$ be the smallest $j$ that satisfies $\norm{y-x_{c(r_j)}} > 2^{\ell(r_j)}$ according the estimates of Lemma~\ref{lmm:kor}.
  (This attempts to recover from the sketch the same $t$ as defined in the analysis in Section~\ref{sec:ann}.)
  \item Let $t_k\in\{0,\ldots,t\}$ be the maximal such that $x_k\in C(r_{t_k})$.
  
  (In words, $r_{t_k}$ is the root of the subtree in which $x_k$ and $y$ ``part ways''.)
  \item If $t_k=t$, return $\norm{My-Mx_{c(r_t)}}$. Note that $M$ and $Mx_{c(r_t)}$ are stored in the sketch.
  \item If $t_k<t$, let $v_k$ be the bottom node on the downward path from $r_{t_k}$ to $\mathrm{leaf}(x_k)$ that does not traverse a long edge. Return $\norm{y-s^*(v_k)}$.
\end{enumerate}

\paragraph{Analysis.}
Fix a query point $y$.
Define the ``good event'' $\mathcal A(y)$ as the intersection of the following:
%We specify a ``good event'' $\mathcal A(y)$ as the intersection of the following events:
% for $y$. It is the intsersection of following events:
\begin{enumerate}
  \item For every subtree root $r_j$ traversed by the query algorithm above, the invocation of Lemma~\ref{lmm:kor} on $\mathrm{sk}_{2^{\ell(r_j)}}(x_{c(r_j)})$ and $\mathrm{sk}_{2^{\ell(r_j)}}(y)$ succeeds in deciding whether $\norm{y-x_{c(r_j)}}\leq 2^{\ell(r_j)}$.
  Specifically, this ensures that $\norm{y-x_{c(r_j)}}\leq 2^{\ell(r_j)}$ for every $j<t$, and $\norm{y-x_{c(r_t)}}\geq (1-\epsilon)2^{\ell(r_t)}$.
  Recalling that we invoked the lemma with $\delta'=\delta/(q\log(2\sqrt{d}\Phi))$, we can take a union bound and succeed in all levels simultaneously with probability $1-\delta/q$.

  \item $\norm{My-Mx_{c(r_t)}}=(1\pm\epsilon)\norm{y-x_{c(r_t)}}$. By Lemma~\ref{lmm:binaryjl} this holds with probability $1-\delta/q$.
\end{enumerate}
Altogether, $\mathcal A(y)$ occurs with probability $1-O(\delta/q)$.

\begin{lemma}
Conditioned on $\mathcal A(y)$ occuring, with probabiliy $1-\delta/q$, Lemma~\ref{lmm:hashes} holds. Namely, the query algorithm correctly recovers all surrogrates in the subtrees rooted by $r_j$ for $j=0,1,\ldots,t-1$.
\end{lemma}
\begin{proof}
The proof of Lemma~\ref{lmm:hashes} in Section~\ref{sec:ann} relied on having $\norm{y-x_{c(r_j)}}\leq2^{\ell(r_j)}$ for every $j<t$. Conditioning on $\mathcal A(y)$ ensures this holds.
\end{proof}

\paragraph{Proof of~\Cref{thm:distances_ub}.}
Let $\mathcal A^*(y)$ denote the event in which both $\mathcal A(y)$ occurs and the conclusion of Lemma~\ref{lmm:hashes} occurs. By the above lemma, $\mathcal A^*(y)$ happens with probability $1-O(\delta/q)$.
From now on we will assume that $\mathcal A^*(y)$ occurs, and conditioned on this, we will show that the distance from $y$ to any data point can be deterministically estimated correctly.
%
To this end, fix $k\in[n]$ and suppose our goal is to estimate $\norm{y-x_k}$. Let $t_k$ and $v_k$ be as defined by the distance query algorithm above.
We handle the two cases of the algorithm separately.

\textbf{Case I:} $t_k=t$. This means $x_k\in C(r_t)$. By Lemma~\ref{lmm:subtree_root} we have $\norm{x_k-x_{c(r_t)}}\leq2^{\ell(r_t)}\epsilon$. By the occurance of $\mathcal A^*(y)$ we have $\norm{y-x_{c(r_t)}}>(1-\epsilon)2^{\ell(r_t)}$. Together,
%\[
  $\norm{y-x_k} =
  \norm{y-x_{c(r_t)}} \pm \norm{x_k-x_{c(r_t)}} =
  (1\pm2\epsilon)\norm{y-x_{c(r_t)}}$.
%\]
This means that $\norm{y-x_{c(r_t)}}$ is a good estimate for $\norm{y-x_k}$. Since $\mathcal A^*(y)$ occurs, it holds that $\norm{My-Mx_{c(r_t)}}=(1\pm\epsilon)\norm{y-x_{c(r_t)}}$, hence $\norm{My-Mx_{c(r_t)}}$ is also a good estimate for $\norm{y-x_k}$, and this is what the algorithm returns.

\textbf{Case II:} $t_k<t$. Let $T_{t_k}$ be the subtree rooted by $r_{t_k}$. 
By the occurance of $\mathcal A^*(y)$, all surrogates in $T_{t_k}$ are recovered correctly, and in particular $s^*(v_k)$ is recovered correctly.
By Lemma~\ref{lmm:surrogates} we have $\norm{x_{c(v_k)}-s^*(v_k)}ֿ\leq2^{\ell(v_k)}\epsilon$,
and by Lemma~\ref{lmm:subtree_leaf} (noting that $x_k\in C(v_k)$ by choice of $v_k$) we have $\norm{x_k-x_{c(v_k)}}\leq2^{\ell(v_k)}\epsilon$. Together,
$\norm{x_k-s^*(v_k)}ֿ\leq2\cdot2^{\ell(v_k)}\epsilon$.
%\begin{equation}\label{eq:xk_vk}
%\norm{x_k-s^*(v_k)}ֿ\leq2\cdot2^{\ell(v_k)}\epsilon.
%\end{equation}

Let $v$ be the leaf in $T_{t_k}$ that minimizes $\norm{y-s^*(v)}$ (over all leaves of $T_{t_k}$). Equivalently, $v$ is the top node of the long edge whose bottom node is $r_{t_k+1}$.
Let $\ell:=\max\{\ell(v),\ell(v_k)\}$.
By choice of $t_k$ we have $v\neq v_k$, hence the centers of these two leaves are separated already at level $\ell$, hence $\norm{x_{c(v_k)}-x_{c(v)}}\geq2^{\ell}$ by Lemma~\ref{lmm:separation}.
By two applications of Lemma~\ref{lmm:surrogates} we have $\norm{x_{c(v_k)}-s^*(v_k)}ֿ\leq2^{\ell}\epsilon$ and $\norm{x_{c(v)}-s^*(v)}ֿ\leq2^{\ell}\epsilon$.
Together, $\norm{s^*(v_k)-s^*(v)}\geq(1-2\epsilon)\cdot2^{\ell}$.
Since $y$ is closer to $s^*(v)$ than to $s^*(v_k)$ (by choice of $v$), we have
$\norm{y-s^*(v_k)} \geq \frac12\cdot\norm{s^*(v_k)-s^*(v)} \geq \left(\frac12-\epsilon\right)\cdot2^{\ell}$.
%\begin{equation}\label{eq:y_vk}
%\norm{y-s^*(v_k)} \geq \frac12\cdot\norm{s^*(v_k)-s^*(v)} \geq \left(\frac12-\epsilon\right)\cdot2^{\ell} .
%\end{equation}
%Combining~\cref{eq:y_vk} and~\cref{eq:xk_vk} yields
Combining this with $\norm{x_k-s^*(v_k)}ֿ\leq2\cdot2^{\ell(v_k)}\epsilon$, which was shown above, yields
$\norm{x_k-s^*(v_k)}ֿ\leq\epsilon\cdot\frac{1}{1/2-\epsilon}\cdot\norm{y-s^*(v_k)}=O(\epsilon)\cdot\norm{y-s^*(v_k)}$.
Therefore,
$\norm{y-x_k} = \norm{y-s^*(v_k)} \pm \norm{x_k-s^*(v_k)} = (1\pm O(\epsilon))\cdot\norm{y-s^*(v_k)}$,
%\[
%  \norm{y-x_k} =
%  \norm{y-s^*(v_k)} \pm \norm{x_k-s^*(v_k)} =
%  (1\pm O(\epsilon))\cdot\norm{y-s^*(v_k)},
%\]
which means $\norm{y-s^*(v_k)}$ is a good estimate for $\norm{y-x_k}$, and this is what the algorithm returns.

\textbf{Conclusion:} Combining both cases, we have shown that for any query point $y$, all distances from $y$ to $X$ can be estimated correctly with probability $1-O(\delta/q)$. Taking a union bound over $q$ queries, and scaling $\delta$ and $\epsilon$ appropriately by a constant, yields the theorem. \qed

\paragraph{Acknowledgments}
This work was supported by grants from the MITEI-Shell program, Amazon Research Award and Simons Investigator Award.

\bibliographystyle{amsalpha}
\bibliography{sublinear_space_nn}

\newpage
\appendix
\section{Deferred Proofs from Section~\ref{sec:sketch}}\label{sec:sketch_proofs}

\begin{proof}[Proof of Lemma~\ref{lmm:tree_size}]
Charging the degree-$1$ nodes along every maximal $1$-path to its bottom (non-degree-$1$) node,
the total number of nodes after top-out compression is bounded by
\[ \sum_{v:\mathrm{deg}(v)\neq1}\Lambda(v) . \]
\cite{indyk2017near} show this is at most $O(n\log(1/\epsilon))$.
The difference is that their compression replaces summands larger than $\Lambda(v)$ by zero, while our (top-out) compression trims them to $\Lambda(v)$.
%The upper bound holds in both cases.
\end{proof}


\begin{proof}[Proof of Lemma~\ref{lmm:subtree_root}]
By top-out compression, $v$ is the top of a downward $1$-path of length $\Lambda(v')$ whose bottom node is $v'$. Since no clusters are joined along a $1$-path, we have $C(v')=C(v)$, hence $x,x'\in C(v')$ and hence $\norm{x-x'} \leq \Delta(v')$. 
Noting that $\ell(v)=\ell(v')+\Lambda(v')=\ell(v')+\log(\Delta(v')/(2^{\ell(v')}\epsilon))=\log(\Delta(v')/\epsilon)$ and rearranging, we find $\Delta(v')=2^{\ell(v)}\epsilon$, which yields the claim.
\end{proof}


\begin{proof}[Proof of Lemma~\ref{lmm:subtree_leaf}]
If $u$ is a leaf in $T$ then $C(u)$ is a singleton cluster, hence $x=x'$.
Otherwise $u$ is the top node of a long edge, and the claim follows by Lemma~\ref{lmm:subtree_root} on the bottom node of that long edge.
\end{proof}


\begin{proof}[Proof of Lemma~\ref{lmm:surrogates}]
The first part of the lemma (where $v$ is any node, not necessarily a subtree leaf) is proved by induction on the ingresses.
In the base case we use that $\norm{x_{c(v)}-s^*(v)}\leq2^{\ell(v)}$ by the choice of grid net.
The induction step is identical to~\cite{indyk2017near}.
The ``furthermore'' part of the lemma then follows as a corollary due to the refined definition of $\gamma(v)$ for a subtree leaf $v$, in the induction step leading to it.
\end{proof}



\begin{proof}[Proof of Lemma~\ref{lmm:sketch_size}]
The sketch of~\cite{indyk2017near} stores the compressed tree $T'$, with each node annotated by its center $c(v)$, ingress $\mathrm{in}(v)$, precision $\gamma(v)$ and quantized displacement $\eta(v)$.
Every long edge is annotated by its length.
They show this takes
\[ O\left( n\left((d+\log n)\log(1/\epsilon) + \log\log\Phi\right)\right) \]
bits;
note that by Lemma~\ref{lmm:tree_size}, top-out compression did not effect this bound.

We additionally store the hashed surrogates of subtree roots.
There are $O(n)$ subtrees,\footnote{By construction, the tree of subtrees in $T$ has no degree-$1$ nodes. Since $T$ has $n$ leaves, there are at most $2n-1$ subtrees.} and each hash takes $\log m$ bits to store, which adds $O(n(d+\log\log\Phi+\log(q/\delta)))$ bits to the above.
Finally, we store the hash functions $H_\ell$ for every $\ell$.
The domain of each $H_\ell$ is $N_{2^\ell}$, which is a subset of $\{-\Phi \ldots \Phi\}^d$, and hence $H_\ell$ can be specified by $O(\log(\Phi^d))$ random bits (\cite{carter1979universal}).
Since we do not require independence between hash functions of different levels, we can use the same random bits for all hash functions, adding a total of $O(d\log\Phi)$ bits to the sketch.
\end{proof}


\section{Approximate Nearest Neighbor Sketching Lower Bound}\label{sec:nnlower}

\begin{theorem}\label{thm:ann_lb}
Suppose that $d\geq\Omega(\epsilon^{-2}\log n)$, $\Phi\geq1/\epsilon$, and $1/n^{0.5-\beta}\leq\epsilon\leq\epsilon_0$ for a constant $\beta>0$ and a sufficiently small constant $\epsilon_0$.
Suppose also that $\delta<1/n^2$.
Then, for the all-nearest-neighbors problem, Alice must use a sketch of at least $\Omega(\beta\epsilon^{-2}n\log n)$ bits.
\end{theorem}
\begin{proof}
We start with dimension $d=n+1+\log n$; it can then be reduced by standard dimension reduction.
Fix $k=1/\epsilon^2$ and assume w.l.o.g.~that $k$ is a square integer (by taking $\epsilon$ to be appropriately small). Note that since $\epsilon>1/\sqrt{n}$ we have $k\leq n$, and that since $\Phi\geq1/\epsilon$ we have $\sqrt{k}\leq\Phi$. 

The data set will consist of $2n$ points, $x_1,\ldots,x_n$ and $z_1,\ldots,z_n$.
Let $i\in[n]$. We choose the first $n$ coordinates of $x_i$ to be an arbitrary $k$-sparse vector, in which each nonzero coordinate equals $1/\sqrt{k}$. Note that the norm of this part is $1$. The $(n+1)$th coordinate of $x_i$ is set to $0$. The remaining $\log n$ coordinates encode the binary encoding of $i$, with each coordinate multiplied by $10$.

Next we define $z_i$. The first $n$ coordinates are $0$. The $(n+1)$th coordinate equals $\sqrt{1-\epsilon}$. The remaining $\log n$ coordinates encode $i$ similarly to $x_i$.

The number of different choices for $\{x_1,\ldots,x_n\}$ is ${n\choose k}^n$.
Therefore if we show that one can fully recover $x_1,\ldots,x_n$ from a given all-nearest-neighbor sketch of the dataset, we would get the desired lower bound
\[
  \log\left({n\choose k}^n\right) \geq
   nk\log(n/k) = \epsilon^{-2}n\log(\epsilon^2n) =\epsilon^{-2}n\log(n^{2\beta}) = 2\beta\epsilon^{-2}n\log n .
\]

Suppose we have such a sketch.
For given $i,j\in[n]$ we now show how to recover the $j$th coordinate of $x_i$, denoted $x_i(j)$, with a single approximate nearest neighbor query.
Let $y_{ij}$ be the following vector in $\R^d$: The first $n+1$ coordinates are all zeros, except for the $j$th coordinate which is set to $1$. The last $\log n$ coordinates encode $i$ similary to $x_i$ and $z_i$.

Consider the distances from $y_{ij}$ to all data points. We start with $x_i$. It is identical to $y_i$ in the last $\log n+1$ coordinates, so we will restrict both to the first $n$ coordinates and denote the restricted vectors by $x_i^{:n}$ and $y_{ij}^{:n}$. $x_i^{:n}$ is a $k$-sparse vector with nonzero entries equal to $1/\sqrt{k}$, hence $\norm{x_i^{:n}}=1$. $y_{ij}^{:n}$ is just the standard basis vector $e_j$ in $\R^n$. Hence,
\[
  \norm{x_i-y_{ij}}^2 =
  \norm{x_i^{:n}-y_{ij}^{:n}}^2 =
  \norm{x_i^{:n}}^2 + \norm{y_{ij}^{:n}}^2 - 2(x_i^{:n})^\top y_{ij}^{:n} =
  2-2x_i(j).
\]
This equals $2$ if $x_i(j)=0$ and $2-2/\sqrt k=2-2\epsilon$ if $x_i(j)=1/\sqrt k$.

Next consider $z_i$. It is identical to $y_{ij}$ in all except the $j$th coordinate, which is $0$ in $z_i$ and $1$ in $y_{ij}$, and the $(n+1)$th coordinate, which is $0$ for $y_{ij}$ and $\sqrt{1-\epsilon}$ for $z_i$. Therefore, $\norm{z_i-y_{ij}}^2=2-\epsilon$.

Finally, for every $i'\neq i$, both $x_{i'}$ and $z_{i'}$ are at distance at least $10$ from $y_{ij}$ due to the encoding of $i$ (as binary multiplied by $10$) in the last $\log n$ coordinates.

In summation we have established the following:
\begin{itemize}
  \item If $x_i(j)\neq0$, then the closest point to $y_{ij}$ in the dataset is $x_i$ at distance $\sqrt{2-2\epsilon}$, and the next closest point is $z_i$ at distance $\sqrt{2-\epsilon}$.
  \item If $x_i(j)=0$, then the closest point to $y_{ij}$ in the dataset is $z_i$ at distance $\sqrt{2-\epsilon}$, and the next closest point is $x_i$ at distance $2$.
\end{itemize}
Therefore, if the sketch supports $(1+\frac{1}{8}\epsilon)$-approximate nearest neighbors, we can recover the true nearest neighbor of $y_{ij}$ and thus recover $x_i(j)$. By hypothesis, the query succeeds with probability $\delta<1/n^2$.
By a union bound over all $i,j\in[n]$ we can recover all of $x_1,\ldots,x_n$ simultaneously, and the theorem follows.
\end{proof}

\section{Lower Bound for Distance Estimation}\label{sec:dist_lb}
In this section we prove~\Cref{thm:distances_lb}.
We handle the two terms in the lower bound separately.

\subsection{First Lower Bound Term}
\begin{lemma}\label{lmm:lb1}
Suppose that $d\geq\Omega(\epsilon^{-2}\log n)$; $\Phi\geq1/\epsilon$; $\epsilon$ is at most a sufficiently small constant; and $\epsilon\geq1/n^{0.5-\rho'}$ for a constant $\rho'>0$.
Then, for the all-cross-distances problem, Alice must use a sketch of at least $\Omega(\rho'\epsilon^{-2}n\log n)$ bits.
\end{lemma}
\begin{proof}
Consider the following problem: Given a dataset $X\subset\{-\Phi \ldots \Phi\}^d$ consisting of $n$ points, we need to produce a sketch, from which we can recover (deterministically) all distances $\{\norm{x-x'}:x,x'\in X\}$ up to distortion $1\pm\epsilon$.
This is the problem considered in~\cite{indyk2017near}, and they show a lower bound of $\Omega(\rho'\epsilon^{-2}n\log n)$ under the asssumptions of the current lemma. We will obtain the same lower bound by reducing this problem to all-cross-distances.

To this end, suppose we have a given sketching procedure for the all-cross-distances problem that uses $s=s(n,d,\Phi,1,\epsilon,\delta)$ amortized bits per point.
We invoke it on $X$ and denote the resulting sketch by $S_0$. For every point $y\in\{-\Phi \ldots \Phi\}^d$, with probabiliy $1-\delta$, all distances $\{\norm{x-y}:x\in X\}$ can be recovered from $S_0$. In particular, this holds in expectation for $(1-\delta)n$ of the points in $X$.
By Markov's inequality, this holds for $\frac12(1-\delta)n>\frac14n$ of the points in $X$ with probability at least $1/2$.
We proceed by recursion on the remaining $\frac34n$ points in $X$.
The sketch produced in the $i$th step of the recursion is denoted by $S_i$ and has total size $(\frac34)^ins$ bits.
After $t=O(\log n)$ steps, with nonzero probability $1/n^{O(1)}$, we have produced a sequence of sketches $S_0,\ldots,S_t$ from which every distance in $\{\norm{x-x'}:x,x'\in X\}$ can be recovered, with a total size of $O(\sum_{i=0}^t(\frac34)^ins)=O(ns)$ bits. This yields the desired lower bound.
\end{proof}

%%% DIRECT PROOF that needs to assume $\delta<1/n$
%\begin{proof}
%%Since $q$ does not appear in the claimed lower bound, we may take $q=n$.
%The proof is essentially identical to the lower bound proof in~\cite{indyk2017near}.
%Let $k=1/\epsilon^2$ and assume w.l.o.g.~that $k$ is a square integer (by taking $\epsilon$ to be appropriately small). Note that since $\epsilon>1/\sqrt{n}$ we have $k\leq n$, and that since $\Phi\geq1/\epsilon$ we have $\sqrt{k}\leq\Phi$. 
%Let $X$ be an arbitrary set of $k$-sparse vectors in $\R^n$ (with repetitions), in which each non-zero coordinate equals $1/\sqrt{k}$. For $i\in[n]$, denote by $x(i)$ the $i$th coordinate of $x\in X$.
%Let $Y=\{e_1,\ldots,e_n\}$ be the set of standard basis vectors in $\R^n$.
%These defines the hard family of inputs $X,Y$ for Alice and Bob in the all-cross-distances problem.
%
%Let $x\in X$ and $e_i\in Y$. Note that $\norm{x}=\norm{e_i}=1$ and hence $\norm{x-e_i}=2-2x^\top e_i=2-2x(i)$.
%This equals $2$ if $x(i)=0$ and $2-2/\sqrt{k}=2-2\epsilon$ if $x(i)=1/\sqrt{k}$.
%Therefore approximating the distance $\norm{x-e_i}$ up to distortion $1\pm\epsilon/2$ is sufficient to recover $x(i)$.
%If Bob can estimate all cross distances between $X$ and $Y$, he can thus recover all vectors in $X$.
%Therefore, the number of bits needed to specify $X$ is a lower bound on Alice's sketch size.
%The number of choices for $X$ is ${n\choose k}^n$, and hence the resulting lower bound is
%\[
%  \log\left({n\choose k}^n\right) \geq
%   nk\log(n/k) = \epsilon^{-2}n\log(\epsilon^2n) =\epsilon^{-2}n\log(n^{2\rho'}) = 2\rho'\epsilon^{-2}n\log n,
%\]
%where we have used the hypothesis $\epsilon\geq1/n^{0.5-\rho'}$.
%
%Finally, note that we proved the lower bound in the setting $d=n$.
%However, we can embed the hard inputs into dimension as low as $d\geq\Omega((\beta\epsilon)^{-2}\log n)$ by the Johnson-Lindenstrauss theorem, and distort the distances by only $(1\pm\beta\epsilon)$, which for a sufficiently small constant $\beta$ would not effect the above arguments.
%\end{proof}

\subsection{Second Lower Bound Term}

\begin{lemma}\label{lmm:lb2}
Suppose that $d^{1-\rho}\geq\epsilon^{-2}\log(q/\delta)$ for a constant $\rho>0$, and $\epsilon$ is at most a sufficiently small constant.
Then, for the all-cross-distances problem, Alice must use a sketch of at least $\Omega(\epsilon^{-2}n\log(d\Phi)\log(q/\delta))$ bits.
\end{lemma}

The proof is by adapting the framework of~\cite{molinaro2013beating}, who proved (among other results) this statement for the case $q=n$.
We describe the adaption and refer to~\cite{molinaro2013beating} for missing details that remain similar.

\subsubsection{Preliminaries}
We follow the approach of~\cite{jayram2013optimal}, of proving one-way communication lower bounds by reduction to variants of the augmented indexing problem, defined next.
\begin{definition}[Augmented Indexing]\label{def:augind}
In the Augmented Indexing problem $AugInd(k,\delta)$, Alice gets a vector $A$ with $k$ entries, whose elements are entries of a universe of size $20/\delta$.
Bob gets an index $i\in[k]$, an element $e$, and the elements $A(i')$ for every $i'<i$.
Bob needs to decide whether $e=A(i)$, and succeed with probability $1-\delta$.
\end{definition}

\cite{jayram2013optimal} give a one-way communication lower bound of $\Omega(k\log(1/\delta))$ for this problem.
The main component in~\cite{molinaro2013beating} is a modified one-way communication model, in which the protocol is allowed to abort with a substantially larger (constant) probability than it is allowed to err.
We will refer to it simply as the~\emph{abortion model} and refer to~\cite{molinaro2013beating} for the exact definition (which we will not require).
They prove the same lower bound for Augmented Indexing.
\begin{lemma}[informal]\label{lmm:augind}
In the abortion model, the one-way communication complexity of $AugInd(k,\delta)$ is $\Omega(k\log(1/\delta))$.
\end{lemma}

\subsubsection{Variants of Augmented Indexing}

We start by defining a variant of augmented indexing that will be suitable for out purpose.

\begin{definition}
In the Matrix Augmented Indexing problem $MatAugInd(k,m,\delta)$, Alice gets a matrix $A$ of order $k\times m$, whose entries are elements of a universe of size $1/\delta$.
Bob gets indices $i\in[k]$ and $j\in[m]$, an element $e$, and the elements $A(i,j')$ for every $j<j'$.
Bob needs to decide whether $e=A(i,j)$, and succeed with probability $1-\delta$.
\end{definition}

This problem is clearly at least as difficult as $AugInd(km,\delta)$ from Definition~\ref{def:augind}, since in the latter Bob gets more information (namely, if we arrange the vector $A$ in $AugInd(km,\delta)$ as a $k\times m$ matrix, then 
Bob gets all entries of $A$ which lexicographically precede $A(i,j)$).
We get the following immediate corollary from Lemma~\ref{lmm:augind}.

\begin{corollary}\label{cor:mataugind}
In the abortion model,
the one-way communication complexity of $MatAugInd(k,m,\delta)$ is $\Omega(km\log(1/\delta))$.
\end{corollary}

\cite{molinaro2013beating} reformulate Augmented Indexing so that Alice's input is a set instead of vector.
Similary, we reformulate Matrix Augmented Indexing as follows.

\begin{definition}
Let $m>0$ and $k>0$ be integers, and $\delta\in(0,1)$.
Partition the interval $[m/\delta]$ into $m$ intervals $I_1,\ldots,I_m$ of size $1/\delta$ each.

In the Augmented Set List problem $AugSetList(k,m,\delta)$, Alice gets a list of subsets $S_1,\ldots,S_k\subset [m/\delta]$, such that each $S_i$ has size exactly $m$ and contains exactly one element from each interval $I_1,\ldots,I_m$.
Bob gets an index $i\in[k]$, an element $e\in[m/\delta]$ and a subset $T$ of $S_i$ that contains exactly the elements of $S_i$ that are smaller than $e$.
Bob needs to decide whether $e\in S_i$, and succeed with probability at least $1-\delta$.
\end{definition}

The equivalence to Matrix Augmented Indexing is not hard to show; the details are similar to~\cite{molinaro2013beating} and we omit them here. By the equivalence, we get the following corollary from Corollary~\ref{cor:mataugind}.
\begin{corollary}\label{cor:setlist}
In the abortion model,
the one-way communication complexity of $AugSetList(k,m,\delta)$ is $\Omega(km\log(1/\delta))$.
\end{corollary}

Next we define the $q$-fold version of the same problem.
\begin{definition}\label{def:qaugsetlist}
In the problem $q$-$AugSetList(k,m,\delta)$, Alice and Bob get $q$ instances of AugSetList$(k,m,\delta/q)$, and Bob needs to answer correctly on all of them simoultaneously with probability at least $1-\delta$.
\end{definition}

The main tehcnical result of~\cite{molinaro2013beating} is, loosely speaking, a direct-sum theorem which lifts a lower bound in the abortion model to a $q$-fold lower bound in the usual model.
Applying their theorem to Corollary~\ref{cor:setlist}, we obtain the following.
\begin{corollary}\label{cor:augsetlist}
The one-way communication complexity of $q$-$AugSetList(k,m,\delta)$ is $\Omega(qkm\log(q/\delta))$.
\end{corollary}

Finally, we construct a ``generalized augmented indexing'' problem over $r$ copies of the above problem.
\begin{definition}
In the problem $r$-$Ind(q$-$AugSetList(k,m,\delta))$,
Alice gets $r$ instances $A_1,\ldots,A_r$ of $q$-$AugSetList(k,m,\delta)$.
Bob gets an index $j\in[r]$, his part $B_j$ of instance $j$, and Alice's instances $A_1,\ldots,A_{j-1}$.
Bob needs to solve instance $j$ with success probability at least $1-\delta$.
\end{definition}

By standard direct sum results in communication complexity (reproduced in~\cite{molinaro2013beating}) we obtain from Corollary~\ref{cor:augsetlist} the final lower bound we need.
\begin{proposition}\label{prp:lb}
The one-way communication complexity of $r$-$Ind(q$-$AugSetList(k,m,\delta))$ is $\Omega(rqkm\log(q/\delta))$.
\end{proposition}

\subsubsection{Reductions to All-Cross-Distances}

We now prove Lemma~\ref{lmm:lb2}, by reducing $r$-$Ind(q$-$AugSetList(k,m,\delta))$ to the all-cross-distances problem.
We will use two reductions, to get a lower bound once in terms of $d$ and once in terms of $\Phi$.
Specifically, in the first reduction
we will set $m=1/\epsilon^2$, $k=n/q$ and $r=\rho\log d$ (where $\rho$ is the constant from the statement of Lemma~\ref{lmm:lb2}). Then the lower bound we would get by Proposition~\ref{prp:lb} is $\Omega\left(\epsilon^{-2}n\log d\log(q/\delta)\right)$.
In the second reduction we will set $r=\rho\log\Phi$, yielding the lower bound $\Omega\left(\epsilon^{-2}n\log\Phi\log(q/\delta)\right)$.
Together they lead to Lemma~\ref{lmm:lb2}.

In both settings, recall we are reducing to the following problem: For dimension $d=\Omega(\epsilon^{-2}\log(q/\delta))$ and aspect ratio $\Phi$, Alice gets $n$ points, Bob gets $q$ points, and Bob needs to estimate all cross-distances up to distortion $1\pm\epsilon$.

Consider an instance of $r$-$Ind(q$-$AugSetList(k,m,\delta))$.
It can be visualized as follows: Alice gets a matrix $S$ with $n=qk$ rows and $r$ columns, where each entry contains a set of size $m$.
Bob gets an index $j\in[r]$, indices $i_1,\ldots,i_q\in[k]$, elements $e_1,\ldots,e_q$, subsets $T_1\subset S(i_1,j),\ldots, T_q\subset S(i_q,j)$, and the first $j-1$ columns of the matrix $S$.

We now use the encoding scheme of~\cite{jayram2013optimal}, in the set formulation which was given in~\cite{molinaro2013beating}. We restate the result.
\begin{lemma}[\cite{jayram2013optimal}]\label{lmm:jw}
Let $m=1/\epsilon^2$ and $0<\eta<1$. Suppose we have the following setting:
\begin{itemize}
  \item Alice has subsets $S_1,\ldots,S_r$ of $[m/\eta]$.
  \item Bob has an index $j\in[r]$, an element $e\in[m/\eta]$, the subset $T\subset S_j$ of elements smaller than $e$, and the sets $S_1,\ldots,S_{j-1}$.
\end{itemize}
There is a shared-randomness mapping of their inputs into points $v_A,v_B$ and a scale $\Psi>0$ (the scale is known to both), such that
\begin{enumerate}
  \item $v_A,v_B\in\B^D$ for $D=O(\epsilon^{-2}\log(\frac1\eta)\exp(r))$.
  \item If $e\in S_j$ (YES instance) then w.p.~$1-\eta$, $\norm{v_A-v_B}^2 \leq (1-2\epsilon)\Psi$.
  \item If $e\notin S_j$ (NO instance) then w.p.~$1-\eta$, $\norm{v_A-v_B}^2 \geq (1-\epsilon)\Psi$
\end{enumerate}
\end{lemma}

\subsubsection{Lower Bound in terms of $d$}
We start with the first reduction that yields a lower bound in terms of $d$.
\begin{lemma}\label{lmm:lb2a}
Under the assumptions of Lemma~\ref{lmm:lb2}, for the all-cross-distances problem, Alice must use a sketch of at least $\Omega(\epsilon^{-2}n\log(d)\log(q/\delta))$ bits.
\end{lemma}

\begin{proof}
We invoke Lemma~\ref{lmm:jw} with $r=\rho\log d$ and $\eta=\delta/q$.
Note that the latter is the desired success probability in each instance of $q$-$AugSetList(k,m,\delta)$ (cf.~Definition~\ref{def:qaugsetlist}).
Alice encodes each row of the matrix, $(S(i,1),\ldots,S(i,r))$, into a point $x_i$, thus $n$ points $x_1,\ldots,x_n$.
Bob encodes $(S(i,1),\ldots,S(i_z,j-1),T_i,j,e_z)$ for each $z\in[q]$ into a point $y_z$, thus $q$ points $y_1,\ldots,y_z$.
For every $z\in[q]$, the problem represented by row $i_z$ in the matrix $S$ is reduced by Lemma~\ref{lmm:jw} to estimating the distance $\norm{x_{i_z}-y_z}$.
By Item 1 of Lemma~\ref{lmm:jw}, the points $\{x_i\}_{i\in[n]}$, $\{y_z\}_{z\in[q]}$ have binary coordinates and dimension $D=O(\epsilon^{-2}\log(q/\delta)d^\rho)$. 
By the hypothesis $d^{1-\rho}\geq\epsilon^{-2}\log(q/\delta)$ of Lemma~\ref{lmm:lb2}, $D=O(d)$.
Therefore Alice and Bob can now feed them into a given black-box solution of the all-cross-distances problem, which estimates all the required distances and solves $r$-$Ind(q$-$AugSetList(k,m,\delta))$.

Let us establish the success probability of the reduction.
Since we set $\eta=\delta/q$ in Lemma~\ref{lmm:jw}, it preserves each distance $\norm{x_{i_z}-y_z}$ for $z\in[q]$ with probability $1-\delta/q$. By a union bound, it preserves all of them simultaneously with probability $1-\delta$.
The success probability of the all-cross-distances problem, simultaneously on all query points $\{\tilde y_z:z\in[q]\}$, is again $1-\delta$. Altogether, the reduction succeeds with probability $1-O(\delta)$. As a result, the all-cross-distances problem solves the given instance of  $r$-$Ind(q$-$AugSetList(k,m,\delta))$, and Lemma~\ref{lmm:lb2a} follows.
\end{proof}

\subsubsection{Lower Bound in terms of $\Phi$}
We proceed to the second reduction that would yield a lower bound in terms of $\Phi$.
\begin{lemma}\label{lmm:lb2b}
Under the assumptions of Lemma~\ref{lmm:lb2}, for the all-cross-distances problem, Alice must use a sketch of at least $\Omega(\epsilon^{-2}n\log(\Phi)\log(q/\delta))$ bits.
\end{lemma}
\begin{proof}
We may assume that $\Phi\geq d$ since otherwise Lemma~\ref{lmm:lb2b} already follows from Lemma~\ref{lmm:lb2a}. Therefore $\Phi^{1-\rho}\geq\epsilon^{-2}\log(q/\delta)$.

The reduction is very similar to the one in Lemma~\ref{lmm:lb2a}.
Again we evoke Lemma~\ref{lmm:jw} with $\eta=\delta/q$, but this time we set $r=\rho\log\Phi$.
Again we denote Alice's encoded points by $x_1,\ldots ,x_n$, and Bob's by $y_1,\ldots,y_q$.
By Item 1 of Lemma~\ref{lmm:jw}, the points have binary coordinates and dimension $D=O(\epsilon^{-2}\log(q/\delta)\Phi^\rho)$. The difference from Lemma~\ref{lmm:lb2a} is that since it is possible that $\Phi\gg d$, the dimension $D$ is too large for the given black-box solution of the all-cross-distances problem (which is limited to dimension $O(d)$).

To solve this, Alice and Bob project their points into dimension $D'=O(\epsilon^{-2}\log(q/\delta))$ by a Johnson-Lindenstrauss transform, using shared randomness.
Let $\tilde x_1,\ldots,\tilde x_n$ and $\tilde y_1,\ldots,\tilde y_z$ denote the projected points.
After the projection each coordinate has magnitude at most $O(\epsilon^{-2}\log(q/\delta)\Phi^\rho)$.
By our assumption $\Phi^{1-\rho}\geq\Omega(\epsilon^{-2}\log(q/\delta))$, this is at most $O(\Phi)$.
Since the dimension $D'$ is $O(d)$, Alice and Bob can now feed $\tilde x_1,\ldots,\tilde x_n$ and $\tilde y_1,\ldots,\tilde y_z$ into a given black-box solution of the all-cross-distances problem with dimension $O(d)$ and aspect ratio $O(\Phi)$.

Let us establish the success probability of the reduction.
As before, Lemma~\ref{lmm:jw} preserves all the required distances, $\norm{x_{i_z}-y_z}$ for $z\in[q]$, with probability $1-\delta$.
The Johnson-Lindenstrauss transform into dimension $D'$ preserves each distance as $\norm{\tilde x_{i_z}-\tilde y_z}$ with probability at least $1-\delta$, since we picked the dimension to be $D'=O(\epsilon^{-2}\log(\frac{q}{\delta}))$.
The success probability of the all-cross-distances problem simultaneously is again $1-\delta$. Altogether, the reduction succeeds with probability $1-O(\delta)$. As a result, the all-cross-distances problem solves the given instance of  $r$-$Ind(q$-$AugSetList(k,m,\delta))$, and Lemma~\ref{lmm:lb2b} follows.
\end{proof}

\subsubsection{Conclusion}
Lemmas~\ref{lmm:lb2a} and~\ref{lmm:lb2b} together imply Lemma~\ref{lmm:lb2}.
The latter, together with Lemma~\ref{lmm:lb1}, implies~\Cref{thm:distances_lb}. \qed

\section{Practical Variant}\label{sec:middleout}
\cite{indyk2017practical} presented a simplified version of the sketch of~\cite{indyk2017near}, which is lossier by a factor $O(\log\log n)$ in the size bound (more precisely it uses $O(\epsilon^{-2}\log(n)(\log\log(n) + \log(1/\epsilon))+\log\log\Phi)$ bits per point; compare this to Table~\ref{tbl:sketches_related_work}), but on the other hand is practical to implement and was shown to work well empirically.
Both variants do not provably support out-of-sample queries.

In the main part of this work, we showed how to adapt the framework of~\cite{indyk2017near} to support out-of-sample queries with nearly optimal size bounds.
The goal of this section is to show that our techniques can also be applied in a simplified way to~\cite{indyk2017practical} in order to obtain a~\emph{practical} algorithm.
Specifically, focusing on the all-nearest-neighbors problem, we will show that a slight modification to~\cite{indyk2017practical} yields provable support in out-of-sample approximate nearest neighbor queries, with a size bound that is the same as in Theorem~\ref{thm:ann_ub} plus an additive $O(\epsilon^{-2}\log(n)\log\log(n))$ term.

\paragraph{Technique: Middle-out compression}
In~\cite{indyk2017practical}, every $1$-path is pruned (i.e.~replaced by a long edge) except for its top $\Lambda$ nodes, where $\Lambda$ is an integer parameter.
Combining this ``bottom-out'' compression with the ``top-out'' compression which was introduced in Section~\ref{sec:sketch}, we obtain~\emph{middle-out compression}: every long $1$-path longer than $2\Lambda$ is replaced by a long edge, except for its top and bottom $\Lambda$ nodes.
As we will show in the remainder of this section, applying this pruning rule to the quadtree of~\cite{indyk2017practical} (instead of their ``bottom-out'' rule) is sufficient to obtain a sketch that provably supports out-of-sample approximate nearest neighbor queries. Thus, the sketching algorithm is nearly unchanged.

We remark that in Section~\ref{sec:sketch} we introduced two additional modifications: \emph{grid-net quantization} and~\emph{surrogate hashing}. These were required in order to prove Theorems~\ref{thm:ann_ub} and~\ref{thm:distances_ub}, but in the framework of~\cite{indyk2017practical} they turn out to be unnecessary: grid-net quantization is already organically built into the quadtree approach of~\cite{indyk2017practical}, and surrogate hashing only served to avoid a $O(\log\log n)$ factor in the sketch size (see footnote 4), but in~\cite{indyk2017practical} this factor is tolerated anyway.

\subsection{Sketching Algorithm Recap}
For completeness, let us briefly describe the sketching algorithm of~\cite{indyk2017practical} (the reader is referred to that paper for more formal details), with our modification.
To this end, set
\[ \Lambda = \lceil\log\left(\frac{16d^{1.5}\log\Phi}{\epsilon\delta}\right)\rceil . \]
Suppose w.l.o.g.~that $\Phi$ is a power of $2$. The sketching algorithm proceeds in three steps:
\begin{enumerate}
  \item\emph{Random shifted grids:} Impose a randomly shifted enclosing hypercube on the data points $X$. More precisely, choose a uniformly random shift $\sigma\in\{-\Phi,\ldots,\Phi\}^d$, and set the enclosing hypercube to be $H=[-2\Phi+\sigma_1,2\Phi+\sigma_1]\times[-2\Phi+\sigma_2,2\Phi+\sigma_2]\times\ldots\times[-2\Phi+\sigma_d,2\Phi+\sigma_d]$. Since $X\subset\{-\Phi,\ldots,\Phi\}^d$, it is indeed enclosed by $H$. We then half $H$ along every dimension to create a finer grid with $2^d$ cells, and proceed so (recursively halving every cell along every dimension) to create a hierarchy of nested grids, with $\log(4\Phi)+\Lambda$ hierarchy levels. The top level is numbered $\Phi+2$, which is the log the side length of $H$, and the next levels are decrementing, so  that the grid cells in level $\ell$ have side length $2^\ell$.
  \item\emph{Quadtree construction:} Construct the quadtree which is naturally associated with the nested grids: the root corresponds to $H$, its children correspond to the non-empty cells of the next grid in the hierarchy (a cell is non-empty if it contains a point in $X$), and so on.
Each tree edge is annotated by a bitstring of length $d$, that marks whether the child cell coincides with the bottom half (bit $0$) or the top half (bit $1$) of the parent cell in each dimension.
  \item\emph{Middle-out compression:} For every path of degree-$1$ tree nodes whose length is more than $2\Lambda$, we keep its top $\Lambda$ and bottom $\Lambda$, and replace its remaining middle portion by a long edge. This removes the edge annotations of the middle section (this achieving compression). We label each long edge with the length of the path it replaces.
\end{enumerate}

In the remainder of this section we prove the following.
\begin{theorem}\label{thm:ann_practical}
The above algorithm, with the above setting of $\Lambda$, runs in time $\tilde O(nd(\log\Phi+\Lambda))$ and produces a sketch for the all-nearest-neighbors problem, whose size in bits is
\[
  O\left( n\left(\frac{\log n\cdot(\log\log n + \log(1/\epsilon))}{\epsilon^2} + \log\log\Phi + \log\left(\frac{q}{\delta}\right)\right) + d\log\Phi  + \log\left(\frac{q}{\delta}\right)\log\left(\frac{\log(q/\delta)}{\epsilon}\right) \right).
\]
\end{theorem}

The sketch size is the same as in~\cite{indyk2017practical}, except that we keep at most $2\Lambda$ instead of $\Lambda$ nodes per $1$-path, which increases the sketch size by only a factor of $2$.

\subsection{Basic lemmas}
We start with some useful properties of the above sketch, which are analogous to lemmas from in Section~\ref{sec:sketch}. In the notation below, for a node $v$ in the quadtree, $C(v)$ denotes the subset of points in $X$ that are contained in the grid cell associated with $v$. As in Section~\ref{sec:sketch}, the quadtree is partitioned into a set $\mathcal F(T)$ of~\emph{subtrees} by removing the long edges.

\begin{lemma}[analog of Lemma~\ref{lmm:separation}]\label{lmm:separation_quadtree}
For every point $x\in X$, with probability $1-\delta$, the following holds.
If $z\in\R^d$ is any point outside the grid cell that contains $x$ in level $\ell$ of the quadtree, then $\norm{x-x'}\geq 8\epsilon^{-1}\cdot2^{\ell-\Lambda}\sqrt{d}$.
\end{lemma}
\begin{proof}
The setting of $\Lambda$ is such that with probability $1-\delta$, in every level $\ell$ of the quadtree, the grid cell that contains $x$ also contains the ball at radius $8\epsilon^{-1}\cdot2^{\ell-\Lambda}\sqrt{d}$ around $x$. (This property is known as ``padding''.) The lemma is just a restatement of this property.
See Lemma 1 and Equation (1) in~\cite{indyk2017practical} for details.
\end{proof}

\begin{lemma}[analog of Lemmas~\ref{lmm:subtree_root}, \ref{lmm:subtree_leaf}, \ref{lmm:surrogates}]\label{lmm:samecell}
Let $v$ be a node in the quadtree, and $x,x'\in\R^d$ points contained in the grid cell associated with $v$. Then $\norm{x-x'}\leq2^{\ell(v)}\sqrt{d}$.
\end{lemma}
\begin{proof}
The grid cell associated with $v$ is a hypercube with side $2^{\ell(v)}$ and diameter $2^{\ell(v)}\sqrt{d}$.
\end{proof}

Before proceeding let us make the following point about the quadtree.
\begin{claim}\label{clm:quadtree_levels}
For every leaf $v$ of the quadtree, $C(v)$ contains a single point of $X$, and $v$ is the bottom of a $1$-path of length at least $\Lambda$.
\end{claim}
\begin{proof}
Refining the quadtree grid hierarchy for $\log(4\Phi)$ levels ensures that each grid cell contains at most one point from $X$, and refining for $\Lambda$ additional levels ensures that each leaf is the bottom of a $1$-path of length at least $\Lambda$.
\end{proof}

\begin{claim}\label{clm:leaves}
Every subtree leaf in the quadtree is the bottom of a $1$-path of length at least $\Lambda$.
\end{claim}
\begin{proof}
If $v$ is a leaf of the quadtree, this follows from Claim~\ref{clm:quadtree_levels}. Otherwise this follows from middle-out compression.
\end{proof}

Next we define centers and surrogates.
Centers $c(v)$ are chosen similarly to Section~\ref{sec:sketch}.
The surrogate $s^*(v)$ of every tree node $v$ is simply defined to be the ``bottom-left'' (i.e.~minimal in all dimensions) corner of the grid cell associated with $v$.

\subsection{Approximate Nearest Neighbor Search}
Finally, we can describe the query algorithm and complete its analysis.
Let $y$ be a query point for which we need to report an approximate nearest neighbor from the sketch. The query algorithm is the same as in Section~\ref{sec:ann}: starting with the subtree that contains the quadtree root, it recovers the surrogates in the current subtree and chooses the subtree $v$ whose surrogate is the closest to $y$. If $v$ is a quadtree leaf, its center is returned as the approximate nearest neighbor. Otherwise, the algorithm proceeds by recursion on the subtree under $v$.

\paragraph{Surrogate recovery}
The difference is in the way we recover the surrogates of a given subtree. In Section~\ref{sec:ann} this was done using the surrogate hashes. Here we will use a simpler, deterministic surrogate recovery subroutine.
Let $s^*(H)\in\R^d$ the surrogate of the quadtree root. (We store this point explicitly in the sketch, and it will be convenient to think of it w.l.o.g.~as the the origin in $\R^d$.) As observed in~\cite{indyk2017practical}, for every tree node $v$, if we concatenate the bits annotating the edges on the path from the root to $v$, we get the binary expansion of the point $s^*(H)+s^*(v)$. Therefore, we can recover $s^*(v)$ from the sketch, as long as the path from the root to $v$ does not traverse a long edge.

If the path to $v$ contains long edges (and thus missing bits in the binary expansion of $s^*(v)$), the algorithm completes these bits from the binary expansion of $y$.
Let $r_0,r_1,\ldots$ be the subtree roots traversed by the algorithm, and let $T_0,T_1,\ldots$ be the corresponding subtrees. Let $t$ be the smallest such that the algorithm does not recover the surrogates in $T_t$ correctly (because the bits missing on the long edge connecting $T_{t-1}$ to $T_t$ are not truly equal to those of $y$). As in Section~\ref{sec:ann}, the query algorithm does not know $t$ (it simply always assumes that the bits of $y$ are the correct missing ones), but we will use it for analysis.
Note that by definition of $t$, all surrogates in the subtrees rooted at $r_0,\ldots,r_{t-1}$ are recovered correctly. Thus, the event from Lemma~\ref{lmm:hashes} holds deterministically.

\paragraph{Proof of Theorem~\ref{thm:ann_practical}}
Let $x^*\in X$ be a fixed true nearest neighbor of $y$ in $X$ (chosen arbitrarily if there is more than one). We shall assume that the event in Lemma~\ref{lmm:separation_quadtree} occurs for $x^*$.

\begin{lemma}[analog of Lemma~\ref{lmm:annrounds}]\label{lmm:annrounds_quadtree}
Let $T'\in\mathcal F(T)$ be a subtree rooted in $r$, such that $x^*\in C(r)$.
Let $v$ a leaf of $T'$ that minimizes $\norm{y-s^*(v)}$.
Then either $x^*\in C(v)$,
or every $z\in C(v)$ is a $(1+O(\epsilon))$-approximate nearest neighbor of $y$.
\end{lemma}
\begin{proof}
If $x^*\in C(v)$ then we are done. Assume now that $x^*\in C(u)$ for a leaf $u\neq v$ of $T'$.
Let $\ell:=\max\{\ell(v),\ell(u)\}$. We start by showing that $\norm{y-x^*}>\epsilon^{-1}2^\ell\sqrt{d}$.
Assume by contradiction this is not the case.
Since $x^*\in C(u)$ we have $\norm{x^*-x_{c(u)}}\leq2^{\ell}\sqrt{d}$ by Lemma~\ref{lmm:samecell}, and similarly $\norm{x_{c(u)}-s^*(u)}\leq2^{\ell}\sqrt{d}$. Together, $\norm{y-s^*(u)}\leq(\epsilon^{-1}+2)2^\ell\sqrt{d}$.
On the other hand, by the triangle inequality,
$\norm{y-s^*(v)} \geq \norm{x^*-x_{c(v)}} - \norm{y-x^*} - \norm{x_{c(v)}-s^*(v)}$.
By Claim~\ref{clm:leaves}, both $v$ and $u$ are the bottom of $1$-paths of length at least $\Lambda$, This means that $x^*$ and $x_{c(v)}$ are separated already at level $\ell+\Lambda$, and by Lemma~\ref{lmm:separation_quadtree} this implies $\norm{x^*-x_{c(v)}}\geq8\epsilon^{-1}\cdot2^{\ell}\sqrt{d}$. By the contradiction hypothesis we have $\norm{y-x^*}\leq\epsilon^{-1}2^\ell\sqrt{d}$, and by Lemma~\ref{lmm:samecell}, $\norm{x_{c(v)}-s^*(v)}\leq2^\ell\sqrt{d}$. 
Putting these together yields $\norm{y-s^*(v)}\geq8\epsilon^{-1}\cdot2^{\ell}\sqrt{d}-\epsilon^{-1}2^\ell\sqrt{d}-2^\ell\sqrt{d} > (\epsilon^{-1}+2)2^\ell\sqrt{d}\geq \norm{x_{c(u)}-s^*(u)}$. This contradicts the choice of $v$.

The lemma now follows because for every $z\in C(v)$,
\begin{align}
\norm{y-z} &\leq \norm{y-s^*(v)} + \norm{s^*(v)-x_{c(v)}} + \norm{x_{c(v)}-z} \label{ineq1b} \\
&\leq \norm{y-s^*(u)} + \norm{s^*(v)-x_{c(v)}} + \norm{x_{c(v)}-z} \label{ineq2b} \\
&\leq \norm{y-x^*} + \norm{x^*-x_{c(u)}} + \norm{x_{c(u)}-s^*(u)} +\norm{s^*(v)-x_{c(v)}} + \norm{x_{c(v)}-z} \label{ineq3b} \\
&\leq \norm{y-x^*} + 4\cdot2^\ell\sqrt{d} \label{ineq4b} \\
&\leq (1+4\epsilon)\norm{y-x^*}, \label{ineq5b}
\end{align}
where~(\ref{ineq1b}) and (\ref{ineq3b}) are by the triangle inequality, (\ref{ineq2b}) is since $\norm{y-s^*(v)}\leq\norm{y-s^*(u)}$ by choice of $v$, (\ref{ineq4b}) is by applying Lemma~\ref{lmm:samecell} to each of the last four summands, and~(\ref{ineq5b}) is since we have shown that $\norm{y-x^*}>\epsilon^{-1}2^\ell\sqrt{d}$.
Therefore $z$ is a $(1+4\epsilon)$-approximate nearest neighbor of $y$.
\end{proof}

Now we prove that the query algorithm returns an approximate nearest neighbor for $y$.
We may assume w.l.o.g.~that $\epsilon$ is smaller than a sufficiently small constant.
We consider two cases. In the first case, $x^*\notin C(r_t)$.
Let $i\in\{1,\ldots,t\}$ be the smallest such that $x^*\notin C(r_i)$.
By applying Lemma~\ref{lmm:annrounds_quadtree} on $r_{i-1}$, we have that every point in $C(r_i)$ is a $(1+O(\epsilon))$-approximate nearest neighbor of $y$. After reaching $r_i$, the algorithm would return the center of some leaf reachable from $r_i$, and it would be a correct output.

In the second case, $x^*\in C(r_t)$ contains a true nearest neighbor of $y$. We will show that every point in $C(r_t)$ is a $(1+O(\epsilon))$-approximate nearest neighbor of $y$, so once again, once the algorithm arrives at $r_t$ it can return anything.
By definition of $t$, we know that $y$ does not reside in the grid cell associated with $r_t$. Since $y$ does reside in that cell, we have $\norm{y-x^*}\geq8\epsilon^{-1}2^{\ell(r_t)-\Lambda}\sqrt{d}$ by Lemma~\ref{lmm:separation_quadtree}. On the other hand, by Claim~\ref{clm:leaves}, $r_t$ is the bottom of a $1$-path of length at least $\Lambda$, and therefore any two points in $C(r_t)$ are contained in the same grid cell at level $\ell(r_t)-\Lambda$, whose diameter is $2^{\ell(r_t)-\Lambda}\sqrt{d}$. In particular, for every $x\in C(r_t)$ we have $\norm{x-x^*}\leq2^{\ell(r_t)-\Lambda}\sqrt{d}\leq\frac{1}{8}\epsilon\norm{y-x^*}$. Altogether we get $\norm{y-x} \leq \norm{y-x^*} + \norm{x^*-x} \leq (1+\frac{1}{8}\epsilon)\norm{y-x^*}$, so every $x\in C(r_t)$ is a $(1+\epsilon)$-approximate nearest neighbor of $y$ in $X$.

The proof assumes the event in Lemma~\ref{lmm:separation_quadtree} holds for $x^*$, which happens with probability $1-\delta$. To handle $q$ queries, we can scale $\delta$ down to $\delta/q$ and take a union bound over the $q$ nearest neighbors of the $q$ query points. \qed


\end{document}
