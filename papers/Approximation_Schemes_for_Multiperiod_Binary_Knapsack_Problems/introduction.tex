\label{sec:typesetting-summary}
Knapsack problems are a classical category of combinatorial optimization problems, and have been studied for more than a century~\citep{mathews1896partition}. They have found wide applications in various fields~\citep{strusevich2005knapsack}, such as selection of investments and portfolios, selection of assets, finding the least wasteful way to cut raw materials, etc.
One of the most commonly studied problem is the so-called \emph{0-1 knapsack problem}, where a set of $n$ items are given, each with a reward and a size, and the goal is to select a subset of these items to maximize the total reward, subject to the constraint that the total size may not exceed some knapsack capacity. It is well-known that the 0-1 knapsack problem is NP-complete. However, the problem was shown to possess \emph{fully polynomial-time approximation schemes (FPTASs)}, i.e., there are algorithms that achieve $(1+\epsilon)$ factor of the optimal value for any $\epsilon\in (0,1)$, and take polynomial time in $n$ and $1/\epsilon$.

The first published FPTAS for the 0-1 knapsack problem was due to~\cite{ibarra1975fast} where they achieve a time complexity $\widetilde{\Ocal}\left(n+(1/\epsilon^4)\right)$ by dividing the items into a class of ``large'' items and a class of ``small'' items. The problem is first solved for large items only, using the dynamic program approach, with rewards rounded down using some discretization quantum (chosen in advance), and the small items are added later. \cite{lawler1979fast} proposed a more nuanced discretization method to improve the polylogarithmic factors. Since then, improvements have been made on the dynamic program for large items~\citep{kellerer2004improved,rhee2015faster}. Most recently, the FPTAS has been improved to $\widetilde{\Ocal}\left(n + (1/\epsilon)^{9/4}\right)$ in~\cite{jin:LIPIcs:2019:10652}.

In this paper, we study {  three} extensions of the 0-1 knapsack problem. First, we consider a multiperiod version of the 0-1 knapsack problem, which we call the \emph{multiperiod binary knapsack problem (MPBKP)}. 
There is a horizon length $T$ and a vector of knapsack sizes $(c_1,\ldots,c_T)$, where $c_t$ is the cumulative size for periods $1,\ldots,t$ and is non-decreasing in $t$. We are also given a list of $n$ items, each associated with a triple $(r, q, d)$ where $r$ denotes the reward or value of the item, $q$ its size, and $d$ denotes its time index (or, deadline).
The goal is to choose a reward maximizing set of items to include such that for any $t=1,\ldots,T$, the total size of selected items with deadlines at most $t$ does not exceed the cumulative capacity of the knapsack up to time $t$. The application that motivates this problem is a seller who produces $(c_t - c_{t-1})$ units of a good in time period $t$, and can store unsold goods for selling later. The seller is offered a set of bids, where each bid includes a price ($r$), a quantity demanded ($q$), and a time at which this quantity is needed. The problem of deciding the revenue maximizing subset of bids to accept is exactly MPBKP.

The second extension we consider is the \emph{multiperiod binary knapsack problem with soft capacity constraints (MPBKP-S)} where {  at each period the capacity constraint is allowed to be violated by paying a penalty that is linear in the violation.} The goal of MPBKP-S is then to maximize the total profit, which is the total reward of the selected items less the total penalty. In this case, the seller can procure goods from outside at a certain rate if his supply is not enough to fulfill the bids he accepts, and wants to maximize his profit.

{ 
The third extension we consider is the \emph{multiperiod binary knapsack problem with soft stochastic capacity constraints (MPBKP-SS)} where the non-decreasing vector of knapsack sizes $(c_1, \ldots, c_T)$ follows some arbitrary joint distribution given as the set of sample paths of the possible realizations and their probabilities. We select the items \emph{before} realizations of any of these random incremental capacities to maximize the total \emph{expected} profit, which is the total reward of selected items less the total expected penalty. In this case, the production of the seller at each time is random, but he has to select a subset of bids before realizing his supply. Again, the seller can procure capacity from outside at a certain rate if his realized supply is not enough to fulfill the bids he accepts, and wants to maximize his expected profit.
}

We note that MPBKP is also related to a number of other multiperiod versions of the knapsack problem in literature. The multiperiod knapsack problem (MPKP) proposed by~\cite{faaland1981multiperiod} has the same structure as  MPBKP, except that in~\cite{faaland1981multiperiod}, each item can be repeated multiple times, i.e., the decision variables for each item is not binary, but any nonnegative integer (in the single-period case, this is called the unbounded knapsack problem~\citep{andonov2000unbounded}). To the best of our knowledge, there has been no further studies on MPKP since~\cite{faaland1981multiperiod}. In the multiple knapsack problem (MKP), there are $m$ knapsacks, each with a different capacity, and items can be inserted to any knapsacks (subject to its capacity constraints). It has been shown in~\cite{chekuri2005polynomial} that MKP does not admit an FPTAS, but an efficient polynomial time approximation scheme (EPTAS) has been found in~\cite{10.1007/978-3-642-27660-6_26}, with runtime depending polynomially on $n$ but exponentially on $1/\epsilon$. The incremental knapsack problem (IKP) is another multiperiod version of the knapsack problem~\citep{10.1007/11764298_4}, where the knapsack capacity increases over time, and each selected item generates a reward on every period after its insertion, but this reward is discounted over time. Unlike MPBKP, items do not have deadlines and can be selected anytime throughout the $T$ periods. A PTAS for the IKP when the discount factor is~$1$ (time invariant, referred to as IIKP) and $T=\Ocal\left(\sqrt{\log n}\right)$ has been found in~\cite{bienstock2013approximation}, and it has been shown that IIKP is strongly NP-hard. Later,~\cite{faenza2018ptas} proposed the first PTAS for IIKP regardless of $T$, and~\cite{della2019approximating} proposed an PTAS for IKP when $T$ is a constant. Most recent developments of IKP include~\cite{aouad2020approximate,faenza2020approximation}. Other similar problems and/or further extensions include the multiple-choice multiperiod knapsack problem~\citep{randeniya1994multiple,lin2004multiple,lin2010dynamic}, the multiperiod multi-dimensional knapsack problem~\citep{lau2004multi}, the multiperiod precedence-constrained knapsack problem~\citep{moreno2010large,samavati2017methodology}, to name a few.

Our main contributions of this paper are two-fold. First, from the perspective of model formulation, we propose the MPBKP and its generalized versions MPBKP-S and MPBKP-SS. {  Despite the fact that there are a number of multiperiod/multiple versions of knapsack problems, including those mentioned above (many of which are strongly NP-hard), the MPBKP and MPBKP-S we proposed here are the first to admit an FPTAS among any multiperiod versions of the classical knapsack problem since their initiation back in 1980s.} With these results, it is thus interesting to see where the boundary lies between these multiperiod problems that admit an FPTAS and those problems that do not admit an FPTAS. { Second, the algorithms we propose for both MPBKP and MPBKP-S are generalized from the ideas of solving 0-1 knapsack problems, but with nontrivial modifications as we will address in the following sections. For MPBKP-SS, we propose a greedy algorithm that achieves $2$-approximation for the special case when all items have the same size.}
\begin{comment}
For MPBKP, we adopt the ``functional'' approach as used in~\cite{chan:OASIcs:2018:8299,jin:LIPIcs:2019:10652}. Roughly speaking, for each period $t$, we approximate the function that gives the maximum reward on every capacity, by selecting items with deadline $t$. Then, we conduct the $(\max,+)$-convolution on the truncated version of these functions, where the resulting function values are then rounded down to some powers of $(1+\epsilon)$. After $T$ periods, we obtain an approximation factor of $(1+\epsilon)^T$ and we adjust $\epsilon$ correspondingly to achieve $(1+\epsilon)$ approximation in $\tilde{\Ocal}\left(n+\frac{T^{3.25}}{\epsilon^{2.25}}\right)$. In Appendix, we also provide another FPTAS with runtime $\tilde{\Ocal}\left(n+\frac{T^2}{\epsilon^3}\right)$. For MPBKP-S, however,  this approach would not work for some technical reasons (details provided in section~\ref{sec:approx2}). Our algorithm for MPBKP-S is motivated by the earlier technique that originated from~\cite{ibarra1975fast}, but with significant modifications. We partition the set of items into ``large'' items and ``small'' items according to a suitably chosen criterion, and during each time period, we use dynamic programming for picking large items and a greedy heuristic for picking small items. The partial solutions obtained at the end of each time $t$ are cleverly carried to the next time $t+1$ for selecting items with deadline $t+1$. We prove that the discretization involved in dynamic programming for large items results in a $\frac{1}{2}\epsilon$ approximation error, and the greedy heuristic for small items results in a $\frac{1}{2T}\epsilon$ approximation error per time period, for a total of $\epsilon$ approximation error overall. Our results show that a~$(1+\epsilon)$ approximation factor can be achieved in $\Ocal\left(\frac{n\log n}{\epsilon}\cdot\min\left\{\frac{T}{\epsilon},n\right\}\right)$.
\end{comment}
 %This is also the first time in literature that the knapsack capacities are turned from ``hard'' constraints to ``soft'' constraints.

The rest of this paper is organized as follows. In Section~\ref{sec:form} we formally write the three problems in mathematical programming form. The FPTAS for MPBKP is proposed in Section~\ref{sec:MPBKP} and the FPTAS for MPBKP-S is proposed in Section~\ref{sec:approx2}. Alternative algorithms for both problems are also provided in Apendix. A greedy algorithm for a special case of MPBKP-SS is proposed in Section~\ref{sec:unit-MPBKP-SS}. All proofs are left to Appendix but we provide proof ideas in the main body.