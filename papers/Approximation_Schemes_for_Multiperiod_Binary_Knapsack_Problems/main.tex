\documentclass[11pt]{article}
\usepackage{empheq}
\usepackage{xcolor}


%\usepackage[OT1,T1]{fontenc}

\usepackage[numbers,sort&compress]{natbib}
\renewcommand{\bibfont}{\footnotesize}
%\usepackage{cite}
%\usepackage{mystyle}
%%%%%%%%%%%%%%%%%%%%%%%%%%%%%%%%%%%%
\makeatletter

\usepackage{etex}

%%% Review %%%

\usepackage{zref-savepos}

\newcounter{mnote}%[page]
\renewcommand{\themnote}{p.\thepage\;$\langle$\arabic{mnote}$\rangle$}

\def\xmarginnote{%
  \xymarginnote{\hskip -\marginparsep \hskip -\marginparwidth}}

\def\ymarginnote{%
  \xymarginnote{\hskip\columnwidth \hskip\marginparsep}}

\long\def\xymarginnote#1#2{%
\vadjust{#1%
\smash{\hbox{{%
        \hsize\marginparwidth
        \@parboxrestore
        \@marginparreset
\footnotesize #2}}}}}

\def\mnoteson{%
\gdef\mnote##1{\refstepcounter{mnote}\label{##1}%
  \zsavepos{##1}%
  \ifnum20432158>\number\zposx{##1}%
  \xmarginnote{{\color{blue}\bf $\langle$\arabic{mnote}$\rangle$}}% 
  \else
  \ymarginnote{{\color{blue}\bf $\langle$\arabic{mnote}$\rangle$}}%
  \fi%
}
  }
\gdef\mnotesoff{\gdef\mnote##1{}}
\mnoteson
\mnotesoff








%%% Layout %%%

% \usepackage{geometry} % override layout
% \geometry{tmargin=2.5cm,bmargin=m2.5cm,lmargin=3cm,rmargin=3cm}
% \setlength{\pdfpagewidth}{8.5in} % overrides default pdftex paper size
% \setlength{\pdfpageheight}{11in}

\newlength{\mywidth}

%%% Conventions %%%

% References
\newcommand{\figref}[1]{Fig.~\ref{#1}}
\newcommand{\defref}[1]{Definition~\ref{#1}}
\newcommand{\tabref}[1]{Table~\ref{#1}}
% general
%\usepackage{ifthen,nonfloat,subfigure,rotating,array,framed}
\usepackage{framed}
%\usepackage{subfigure}
\usepackage{subcaption}
\usepackage{comment}
%\specialcomment{nb}{\begingroup \noindent \framed\textbf{n.b.\ }}{\endframed\endgroup}
%%\usepackage{xtab,arydshln,multirow}
% topcaption defined in xtab. must load nonfloat before xtab
%\PassOptionsToPackage{svgnames,dvipsnames}{xcolor}
\usepackage[svgnames,dvipsnames]{xcolor}
%\definecolor{myblue}{rgb}{.8,.8,1}
%\definecolor{umbra}{rgb}{.8,.8,.5}
%\newcommand*\mybluebox[1]{%
%  \colorbox{myblue}{\hspace{1em}#1\hspace{1em}}}
\usepackage[all]{xy}
%\usepackage{pstricks,pst-node}
\usepackage{tikz}
\usetikzlibrary{positioning,matrix,through,calc,arrows,fit,shapes,decorations.pathreplacing,decorations.markings,decorations.text}

\tikzstyle{block} = [draw,fill=blue!20,minimum size=2em]

% allow prefix to scope name
\tikzset{%
	prefix node name/.code={%
		\tikzset{%
			name/.code={\edef\tikz@fig@name{#1 ##1}}
		}%
	}%
}


\@ifpackagelater{tikz}{2013/12/01}{
	\newcommand{\convexpath}[2]{
		[create hullcoords/.code={
			\global\edef\namelist{#1}
			\foreach [count=\counter] \nodename in \namelist {
				\global\edef\numberofnodes{\counter}
				\coordinate (hullcoord\counter) at (\nodename);
			}
			\coordinate (hullcoord0) at (hullcoord\numberofnodes);
			\pgfmathtruncatemacro\lastnumber{\numberofnodes+1}
			\coordinate (hullcoord\lastnumber) at (hullcoord1);
		}, create hullcoords ]
		($(hullcoord1)!#2!-90:(hullcoord0)$)
		\foreach [evaluate=\currentnode as \previousnode using \currentnode-1,
		evaluate=\currentnode as \nextnode using \currentnode+1] \currentnode in {1,...,\numberofnodes} {
			let \p1 = ($(hullcoord\currentnode) - (hullcoord\previousnode)$),
			\n1 = {atan2(\y1,\x1) + 90},
			\p2 = ($(hullcoord\nextnode) - (hullcoord\currentnode)$),
			\n2 = {atan2(\y2,\x2) + 90},
			\n{delta} = {Mod(\n2-\n1,360) - 360}
			in 
			{arc [start angle=\n1, delta angle=\n{delta}, radius=#2]}
			-- ($(hullcoord\nextnode)!#2!-90:(hullcoord\currentnode)$) 
		}
	}
}{
	\newcommand{\convexpath}[2]{
		[create hullcoords/.code={
			\global\edef\namelist{#1}
			\foreach [count=\counter] \nodename in \namelist {
				\global\edef\numberofnodes{\counter}
				\coordinate (hullcoord\counter) at (\nodename);
			}
			\coordinate (hullcoord0) at (hullcoord\numberofnodes);
			\pgfmathtruncatemacro\lastnumber{\numberofnodes+1}
			\coordinate (hullcoord\lastnumber) at (hullcoord1);
		}, create hullcoords ]
		($(hullcoord1)!#2!-90:(hullcoord0)$)
		\foreach [evaluate=\currentnode as \previousnode using \currentnode-1,
		evaluate=\currentnode as \nextnode using \currentnode+1] \currentnode in {1,...,\numberofnodes} {
			let \p1 = ($(hullcoord\currentnode) - (hullcoord\previousnode)$),
			\n1 = {atan2(\x1,\y1) + 90},
			\p2 = ($(hullcoord\nextnode) - (hullcoord\currentnode)$),
			\n2 = {atan2(\x2,\y2) + 90},
			\n{delta} = {Mod(\n2-\n1,360) - 360}
			in 
			{arc [start angle=\n1, delta angle=\n{delta}, radius=#2]}
			-- ($(hullcoord\nextnode)!#2!-90:(hullcoord\currentnode)$) 
		}
	}
}

% circle around nodes

% typsetting math
\usepackage{qsymbols,amssymb,mathrsfs}
\usepackage{amsmath}
\usepackage[standard,thmmarks]{ntheorem}
\theoremstyle{plain}
\theoremsymbol{\ensuremath{_\vartriangleleft}}
\theorembodyfont{\itshape}
\theoremheaderfont{\normalfont\bfseries}
\theoremseparator{}
\newtheorem{Claim}{Claim}
\newtheorem{Subclaim}{Subclaim}
\theoremstyle{nonumberplain}
\theoremheaderfont{\scshape}
\theorembodyfont{\normalfont}
\theoremsymbol{\ensuremath{_\blacktriangleleft}}
\newtheorem{Subproof}{Proof}

\theoremnumbering{arabic}
\theoremstyle{plain}
\usepackage{latexsym}
\theoremsymbol{\ensuremath{_\Box}}
\theorembodyfont{\itshape}
\theoremheaderfont{\normalfont\bfseries}
\theoremseparator{}
\newtheorem{Conjecture}{Conjecture}

\theorembodyfont{\upshape}
\theoremprework{\bigskip\hrule}
\theorempostwork{\hrule\bigskip}
\newtheorem{Condition}{Condition}%[section]


%\RequirePckage{amsmath} loaded by empheq
\usepackage[overload]{empheq} % no \intertext and \displaybreak
%\usepackage{breqn}

\let\iftwocolumn\if@twocolumn
\g@addto@macro\@twocolumntrue{\let\iftwocolumn\if@twocolumn}
\g@addto@macro\@twocolumnfalse{\let\iftwocolumn\if@twocolumn}

%\empheqset{box=\mybluebox}
%\usepackage{mathtools}      % to polish math typsetting, loaded
%                                % by empeq
\mathtoolsset{showonlyrefs=false,showmanualtags}
\let\underbrace\LaTeXunderbrace % adapt spacing to font sizes
\let\overbrace\LaTeXoverbrace
\renewcommand{\eqref}[1]{\textup{(\refeq{#1})}} % eqref was not allowed in
                                       % sub/super-scripts
\newtagform{brackets}[]{(}{)}   % new tags for equations
\usetagform{brackets}
% defined commands:
% \shortintertext{}, dcases*, \cramped, \smashoperator[]{}

\usepackage[Smaller]{cancel}
\renewcommand{\CancelColor}{\color{Red}}
%\newcommand\hcancel[2][black]{\setbox0=\hbox{#2}% colored horizontal cross
%  \rlap{\raisebox{.45\ht0}{\color{#1}\rule{\wd0}{1pt}}}#2}



\usepackage{graphicx,psfrag}
\graphicspath{{figure/}{image/}} % Search path of figures

% for tabular
\usepackage{diagbox} % \backslashbox{}{} for slashed entries
%\usepackage{threeparttable} % threeparttable, \tnote{},
                                % tablenotes, and \item[]
%\usepackage{colortab} % \cellcolor[gray]{0.9},
%\rowcolor,\columncolor,
%\usepackage{colortab} % \LCC \gray & ...  \ECC \\

% typesetting codes
%\usepackage{maple2e} % need to use \char29 for ^
\usepackage{algorithm2e}
\usepackage{listings} 
\lstdefinelanguage{Maple}{
  morekeywords={proc,module,end, for,from,to,by,while,in,do,od
    ,if,elif,else,then,fi ,use,try,catch,finally}, sensitive,
  morecomment=[l]\#,
  morestring=[b]",morestring=[b]`}[keywords,comments,strings]
\lstset{
  basicstyle=\scriptsize,
  keywordstyle=\color{ForestGreen}\bfseries,
  commentstyle=\color{DarkBlue},
  stringstyle=\color{DimGray}\ttfamily,
  texcl
}
%%% New fonts %%%
\DeclareMathAlphabet{\mathpzc}{OT1}{pzc}{m}{it}
\usepackage{upgreek} % \upalpha,\upbeta, ...
%\usepackage{bbold}   % blackboard math
\usepackage{dsfont}  % \mathds

%%% Macros for multiple definitions %%%

% example usage:
% \multi{M}{\boldsymbol{#1}}  % defines \multiM
% \multi ABC.                 % defines \MA \MB and \MC as
%                             % \boldsymbol{A}, \boldsymbol{B} and
%                             % \boldsymbol{C} respectively.
% 
%  The last period '.' is necessary to terminate the macro expansion.
%
% \multi*{M}{\boldsymbol{#1}} % defines \multiM and \M
% \M{A}                       % expands to \boldsymbol{A}

\def\multi@nostar#1#2{%
  \expandafter\def\csname multi#1\endcsname##1{%
    \if ##1.\let\next=\relax \else
    \def\next{\csname multi#1\endcsname}     
    %\expandafter\def\csname #1##1\endcsname{#2}
    \expandafter\newcommand\csname #1##1\endcsname{#2}
    \fi\next}}

\def\multi@star#1#2{%
  \expandafter\def\csname #1\endcsname##1{#2}
  \multi@nostar{#1}{#2}
}

\newcommand{\multi}{%
  \@ifstar \multi@star \multi@nostar}

%%% new alphabets %%%

\multi*{rm}{\mathrm{#1}}
\multi*{mc}{\mathcal{#1}}
\multi*{op}{\mathop {\operator@font #1}}
% \multi*{op}{\operatorname{#1}}
\multi*{ds}{\mathds{#1}}
\multi*{set}{\mathcal{#1}}
\multi*{rsfs}{\mathscr{#1}}
\multi*{pz}{\mathpzc{#1}}
\multi*{M}{\boldsymbol{#1}}
\multi*{R}{\mathsf{#1}}
\multi*{RM}{\M{\R{#1}}}
\multi*{bb}{\mathbb{#1}}
\multi*{td}{\tilde{#1}}
\multi*{tR}{\tilde{\mathsf{#1}}}
\multi*{trM}{\tilde{\M{\R{#1}}}}
\multi*{tset}{\tilde{\mathcal{#1}}}
\multi*{tM}{\tilde{\M{#1}}}
\multi*{baM}{\bar{\M{#1}}}
\multi*{ol}{\overline{#1}}

\multirm  ABCDEFGHIJKLMNOPQRSTUVWXYZabcdefghijklmnopqrstuvwxyz.
\multiol  ABCDEFGHIJKLMNOPQRSTUVWXYZabcdefghijklmnopqrstuvwxyz.
\multitR   ABCDEFGHIJKLMNOPQRSTUVWXYZabcdefghijklmnopqrstuvwxyz.
\multitd   ABCDEFGHIJKLMNOPQRSTUVWXYZabcdefghijklmnopqrstuvwxyz.
\multitset ABCDEFGHIJKLMNOPQRSTUVWXYZabcdefghijklmnopqrstuvwxyz.
\multitM   ABCDEFGHIJKLMNOPQRSTUVWXYZabcdefghijklmnopqrstuvwxyz.
\multibaM   ABCDEFGHIJKLMNOPQRSTUVWXYZabcdefghijklmnopqrstuvwxyz.
\multitrM   ABCDEFGHIJKLMNOPQRSTUVWXYZabcdefghijklmnopqrstuvwxyz.
\multimc   ABCDEFGHIJKLMNOPQRSTUVWXYZabcdefghijklmnopqrstuvwxyz.
\multiop   ABCDEFGHIJKLMNOPQRSTUVWXYZabcdefghijklmnopqrstuvwxyz.
\multids   ABCDEFGHIJKLMNOPQRSTUVWXYZabcdefghijklmnopqrstuvwxyz.
\multiset  ABCDEFGHIJKLMNOPQRSTUVWXYZabcdefghijklmnopqrstuvwxyz.
\multirsfs ABCDEFGHIJKLMNOPQRSTUVWXYZabcdefghijklmnopqrstuvwxyz.
\multipz   ABCDEFGHIJKLMNOPQRSTUVWXYZabcdefghijklmnopqrstuvwxyz.
\multiM    ABCDEFGHIJKLMNOPQRSTUVWXYZabcdefghijklmnopqrstuvwxyz.
\multiR    ABCDEFGHIJKL NO QR TUVWXYZabcd fghijklmnopqrstuvwxyz.
\multibb   ABCDEFGHIJKLMNOPQRSTUVWXYZabcdefghijklmnopqrstuvwxyz.
\multiRM   ABCDEFGHIJKLMNOPQRSTUVWXYZabcdefghijklmnopqrstuvwxyz.
\newcommand{\RRM}{\R{M}}
\newcommand{\RRP}{\R{P}}
\newcommand{\RRe}{\R{e}}
\newcommand{\RRS}{\R{S}}
%%% new symbols %%%

%\newcommand{\dotgeq}{\buildrel \textstyle  .\over \geq}
%\newcommand{\dotleq}{\buildrel \textstyle  .\over \leq}
\newcommand{\dotleq}{\buildrel \textstyle  .\over {\smash{\lower
      .2ex\hbox{\ensuremath\leqslant}}\vphantom{=}}}
\newcommand{\dotgeq}{\buildrel \textstyle  .\over {\smash{\lower
      .2ex\hbox{\ensuremath\geqslant}}\vphantom{=}}}

\DeclareMathOperator*{\argmin}{arg\,min}
\DeclareMathOperator*{\argmax}{arg\,max}

%%% abbreviations %%%

% commands
\newcommand{\esm}{\ensuremath}

% environments
\newcommand{\bM}{\begin{bmatrix}}
\newcommand{\eM}{\end{bmatrix}}
\newcommand{\bSM}{\left[\begin{smallmatrix}}
\newcommand{\eSM}{\end{smallmatrix}\right]}
\renewcommand*\env@matrix[1][*\c@MaxMatrixCols c]{%
  \hskip -\arraycolsep
  \let\@ifnextchar\new@ifnextchar
  \array{#1}}



% sets of number
\newqsymbol{`N}{\mathbb{N}}
\newqsymbol{`R}{\mathbb{R}}
\newqsymbol{`P}{\mathbb{P}}
\newqsymbol{`Z}{\mathbb{Z}}

% symbol short cut
\newqsymbol{`|}{\mid}
% use \| for \parallel
\newqsymbol{`8}{\infty}
\newqsymbol{`1}{\left}
\newqsymbol{`2}{\right}
\newqsymbol{`6}{\partial}
\newqsymbol{`0}{\emptyset}
\newqsymbol{`-}{\leftrightarrow}
\newqsymbol{`<}{\langle}
\newqsymbol{`>}{\rangle}

%%% new operators / functions %%%

\newcommand{\sgn}{\operatorname{sgn}}
\newcommand{\Var}{\op{Var}}
\newcommand{\diag}{\operatorname{diag}}
\newcommand{\erf}{\operatorname{erf}}
\newcommand{\erfc}{\operatorname{erfc}}
\newcommand{\erfi}{\operatorname{erfi}}
\newcommand{\adj}{\operatorname{adj}}
\newcommand{\supp}{\operatorname{supp}}
\newcommand{\E}{\opE\nolimits}
\newcommand{\T}{\intercal}
% requires mathtools
% \abs,\abs*,\abs[<size_cmd:\big,\Big,\bigg,\Bigg etc.>]
\DeclarePairedDelimiter\abs{\lvert}{\rvert} 
\DeclarePairedDelimiter\norm{\lVert}{\rVert}
\DeclarePairedDelimiter\ceil{\lceil}{\rceil}
\DeclarePairedDelimiter\floor{\lfloor}{\rfloor}
\DeclarePairedDelimiter\Set{\{}{\}}
\newcommand{\imod}[1]{\allowbreak\mkern10mu({\operator@font mod}\,\,#1)}

%%% new formats %%%
\newcommand{\leftexp}[2]{{\vphantom{#2}}^{#1}{#2}}


% non-floating figures that can be put inside tables
\newenvironment{nffigure}[1][\relax]{\vskip \intextsep
  \noindent\minipage{\linewidth}\def\@captype{figure}}{\endminipage\vskip \intextsep}

\newcommand{\threecols}[3]{
\hbox to \textwidth{%
      \normalfont\rlap{\parbox[b]{\textwidth}{\raggedright#1\strut}}%
        \hss\parbox[b]{\textwidth}{\centering#2\strut}\hss
        \llap{\parbox[b]{\textwidth}{\raggedleft#3\strut}}%
    }% hbox 
}

\newcommand{\reason}[2][\relax]{
  \ifthenelse{\equal{#1}{\relax}}{
    \left(\text{#2}\right)
  }{
    \left(\parbox{#1}{\raggedright #2}\right)
  }
}

\newcommand{\marginlabel}[1]
{\mbox[]\marginpar{\color{ForestGreen} \sffamily \small \raggedright\hspace{0pt}#1}}


% up-tag an equation
\newcommand{\utag}[2]{\mathop{#2}\limits^{\text{(#1)}}}
\newcommand{\uref}[1]{(#1)}


% Notation table

\newcommand{\Hline}{\noalign{\vskip 0.1in \hrule height 0.1pt \vskip
    0.1in}}
  
\def\Malign#1{\tabskip=0in
  \halign to\columnwidth{
    \ensuremath{\displaystyle ##}\hfil
    \tabskip=0in plus 1 fil minus 1 fil
    &
    \parbox[t]{0.8\columnwidth}{##}
    \tabskip=0in
    \cr #1}}


%%%%%%%%%%%%%%%%%%%%%%%%%%%%%%%%%%%%%%%%%%%%%%%%%%%%%%%%%%%%%%%%%%%
% MISCELLANEOUS

% Modification from braket.sty by Donald Arseneau
% Command defined is: \extendvert{ }
% The "small versions" use fixed-size brackets independent of their
% contents, whereas the expand the first vertical line '|' or '\|' to
% envelop the content
\let\SavedDoubleVert\relax
\let\protect\relax
{\catcode`\|=\active
  \xdef\extendvert{\protect\expandafter\noexpand\csname extendvert \endcsname}
  \expandafter\gdef\csname extendvert \endcsname#1{\mskip-5mu \left.%
      \ifx\SavedDoubleVert\relax \let\SavedDoubleVert\|\fi
     \:{\let\|\SetDoubleVert
       \mathcode`\|32768\let|\SetVert
     #1}\:\right.\mskip-5mu}
}
\def\SetVert{\@ifnextchar|{\|\@gobble}% turn || into \|
    {\egroup\;\mid@vertical\;\bgroup}}
\def\SetDoubleVert{\egroup\;\mid@dblvertical\;\bgroup}

% If the user is using e-TeX with its \middle primitive, use that for
% verticals instead of \vrule.
%
\begingroup
 \edef\@tempa{\meaning\middle}
 \edef\@tempb{\string\middle}
\expandafter \endgroup \ifx\@tempa\@tempb
 \def\mid@vertical{\middle|}
 \def\mid@dblvertical{\middle\SavedDoubleVert}
\else
 \def\mid@vertical{\mskip1mu\vrule\mskip1mu}
 \def\mid@dblvertical{\mskip1mu\vrule\mskip2.5mu\vrule\mskip1mu}
\fi

%%%%%%%%%%%%%%%%%%%%%%%%%%%%%%%%%%%%%%%%%%%%%%%%%%%%%%%%%%%%%%%%

\makeatother

%%%%%%%%%%%%%%%%%%%%%%%%%%%%%%%%%%%%

\usepackage{ctable}
\usepackage{fouridx}
%\usepackage{calc}
\usepackage{framed}
\usetikzlibrary{positioning,matrix}

\usepackage{paralist}
%\usepackage{refcheck}
\usepackage{enumerate}

\usepackage[normalem]{ulem}
\newcommand{\Ans}[1]{\uuline{\raisebox{.15em}{#1}}}



\numberwithin{equation}{section}
\makeatletter
\@addtoreset{equation}{section}
\renewcommand{\theequation}{\arabic{section}.\arabic{equation}}
\@addtoreset{Theorem}{section}
\renewcommand{\theTheorem}{\arabic{section}.\arabic{Theorem}}
\@addtoreset{Lemma}{section}
\renewcommand{\theLemma}{\arabic{section}.\arabic{Lemma}}
\@addtoreset{Corollary}{section}
\renewcommand{\theCorollary}{\arabic{section}.\arabic{Corollary}}
\@addtoreset{Example}{section}
\renewcommand{\theExample}{\arabic{section}.\arabic{Example}}
\@addtoreset{Remark}{section}
\renewcommand{\theRemark}{\arabic{section}.\arabic{Remark}}
\@addtoreset{Proposition}{section}
\renewcommand{\theProposition}{\arabic{section}.\arabic{Proposition}}
\@addtoreset{Definition}{section}
\renewcommand{\theDefinition}{\arabic{section}.\arabic{Definition}}
\@addtoreset{Claim}{section}
\renewcommand{\theClaim}{\arabic{section}.\arabic{Claim}}
\@addtoreset{Subclaim}{Theorem}
\renewcommand{\theSubclaim}{\theTheorem\Alph{Subclaim}}
\makeatother

\newcommand{\Null}{\op{Null}}
%\newcommand{\T}{\op{T}\nolimits}
\newcommand{\Bern}{\op{Bern}\nolimits}
\newcommand{\odd}{\op{odd}}
\newcommand{\even}{\op{even}}
\newcommand{\Sym}{\op{Sym}}
\newcommand{\si}{s_{\op{div}}}
\newcommand{\sv}{s_{\op{var}}}
\newcommand{\Wtyp}{W_{\op{typ}}}
\newcommand{\Rco}{R_{\op{CO}}}
\newcommand{\Tm}{\op{T}\nolimits}
\newcommand{\JGK}{J_{\op{GK}}}

\newcommand{\diff}{\mathrm{d}}

\newenvironment{lbox}{
  \setlength{\FrameSep}{1.5mm}
  \setlength{\FrameRule}{0mm}
  \def\FrameCommand{\fboxsep=\FrameSep \fcolorbox{black!20}{white}}%
  \MakeFramed {\FrameRestore}}%
{\endMakeFramed}

\newenvironment{ybox}{
	\setlength{\FrameSep}{1.5mm}
	\setlength{\FrameRule}{0mm}
  \def\FrameCommand{\fboxsep=\FrameSep \fcolorbox{black!10}{yellow!8}}%
  \MakeFramed {\FrameRestore}}%
{\endMakeFramed}

\newenvironment{gbox}{
	\setlength{\FrameSep}{1.5mm}
\setlength{\FrameRule}{0mm}
  \def\FrameCommand{\fboxsep=\FrameSep \fcolorbox{black!10}{green!8}}%
  \MakeFramed {\FrameRestore}}%
{\endMakeFramed}

\newenvironment{bbox}{
	\setlength{\FrameSep}{1.5mm}
\setlength{\FrameRule}{0mm}
  \def\FrameCommand{\fboxsep=\FrameSep \fcolorbox{black!10}{blue!8}}%
  \MakeFramed {\FrameRestore}}%
{\endMakeFramed}

\newenvironment{yleftbar}{%
  \def\FrameCommand{{\color{yellow!20}\vrule width 3pt} \hspace{10pt}}%
  \MakeFramed {\advance\hsize-\width \FrameRestore}}%
 {\endMakeFramed}

\newcommand{\tbox}[2][\relax]{
 \setlength{\FrameSep}{1.5mm}
  \setlength{\FrameRule}{0mm}
  \begin{ybox}
    \noindent\underline{#1:}\newline
    #2
  \end{ybox}
}

\newcommand{\pbox}[2][\relax]{
  \setlength{\FrameSep}{1.5mm}
 \setlength{\FrameRule}{0mm}
  \begin{gbox}
    \noindent\underline{#1:}\newline
    #2
  \end{gbox}
}

\newcommand{\gtag}[1]{\text{\color{green!50!black!60} #1}}
\let\labelindent\relax
\usepackage{enumitem}

%%%%%%%%%%%%%%%%%%%%%%%%%%%%%%%%%%%%
% fix subequations
% http://tex.stackexchange.com/questions/80134/nesting-subequations-within-align
%%%%%%%%%%%%%%%%%%%%%%%%%%%%%%%%%%%%

\usepackage{etoolbox}

% let \theparentequation use the same definition as equation
\let\theparentequation\theequation
% change every occurence of "equation" to "parentequation"
\patchcmd{\theparentequation}{equation}{parentequation}{}{}

\renewenvironment{subequations}[1][]{%              optional argument: label-name for (first) parent equation
	\refstepcounter{equation}%
	%  \def\theparentequation{\arabic{parentequation}}% we patched it already :)
	\setcounter{parentequation}{\value{equation}}%    parentequation = equation
	\setcounter{equation}{0}%                         (sub)equation  = 0
	\def\theequation{\theparentequation\alph{equation}}% 
	\let\parentlabel\label%                           Evade sanitation performed by amsmath
	\ifx\\#1\\\relax\else\label{#1}\fi%               #1 given: \label{#1}, otherwise: nothing
	\ignorespaces
}{%
	\setcounter{equation}{\value{parentequation}}%    equation = subequation
	\ignorespacesafterend
}

\newcommand*{\nextParentEquation}[1][]{%            optional argument: label-name for (first) parent equation
	\refstepcounter{parentequation}%                  parentequation++
	\setcounter{equation}{0}%                         equation = 0
	\ifx\\#1\\\relax\else\parentlabel{#1}\fi%         #1 given: \label{#1}, otherwise: nothing
}

% hyperlink
\PassOptionsToPackage{breaklinks,letterpaper,hyperindex=true,backref=false,bookmarksnumbered,bookmarksopen,linktocpage,colorlinks,linkcolor=BrickRed,citecolor=OliveGreen,urlcolor=Blue,pdfstartview=FitH}{hyperref}
\usepackage{hyperref}

% makeindex style
\newcommand{\indexmain}[1]{\textbf{\hyperpage{#1}}}



\includeversion{paperonly}
\excludeversion{techreportonly}
%\includeversion{techreportonly}
%\excludeversion{paperonly}

\begin{document}
	\title{Approximation Schemes for Multiperiod Binary Knapsack Problems} 
	\author{Zuguang Gao, John R. Birge, and Varun Gupta\footnote{All authors are with the University of Chicago. Emails: \{zuguang.gao, john.birge, varun.gupta\}@chicagobooth.edu.}}
	\date{}

	\maketitle
	
	\begin{abstract}
		\begin{onehalfspace} 
			An instance of the multiperiod binary knapsack problem (MPBKP) is given by a horizon length $T$, a non-decreasing vector of knapsack sizes $(c_1, \ldots, c_T)$ where $c_t$ denotes the cumulative size for periods $1,\ldots,t$, and a list of $n$ items. Each item is a triple $(r, q, d)$ where $r$ denotes the reward or value of the item, $q$ its size, and $d$ denotes its time index (or, deadline). The goal is to choose, for each deadline $t$, which items to include to maximize the total reward, subject to the constraints that for all $t=1,\ldots,T$, the total size of selected items with deadlines at most $t$ does not exceed the cumulative capacity of the knapsack up to time $t$. We also consider the multiperiod binary knapsack problem with soft capacity constraints (MPBKP-S) where the capacity constraints are allowed to be violated by paying a penalty that is linear in the violation. The goal of MPBKP-S is to maximize the total profit, which is the total reward of the selected items less the total penalty. Finally, we consider the multiperiod binary knapsack problem with soft stochastic capacity constraints (MPBKP-SS), where the non-decreasing vector of knapsack sizes $(c_1, \ldots, c_T)$ follow some arbitrary joint distribution but we are given access to the profit as an oracle, and we must choose a subset of items to maximize the total expected profit, which is the total reward less the total expected penalty.

%			In the multiperiod binary knapsack problem (MPBKP), there are $T$ time periods. In each period, there are a number of items, each with a reward and a size. The goal is to choose in each period which items to include to maximize the total reward, subject to the constraints that for any $t=1,\ldots,T$, the total size of selected items from period~$1$ to period~$t$ cannot exceed the capacity of the knapsack at time $t$. In the multiperiod binary knapsack problem with soft capacity constriants (MPBKP-SS), the capacity constraints are allowed to be violated by paying some penalty for each unit size that goes beyond the capacity. The goal of MPBKP-SS is then to maximize the total profit, which is the total reward of the selected items deducted by the total penalty.

For MPBKP, we exhibit a fully polynomial-time approximation scheme that achieves $(1+\epsilon)$ approximation with runtime $\tilde{\mathcal{O}}\left(\min\left\{n+\frac{T^{3.25}}{\epsilon^{2.25}},n+\frac{T^{2}}{\epsilon^{3}},\frac{nT}{\epsilon^2},\frac{n^2}{\epsilon}\right\}\right)$; for MPBKP-S, the $(1+\epsilon)$ approximation can be achieved in $\mathcal{O}\left(\frac{n\log n}{\epsilon}\cdot\min\left\{\frac{T}{\epsilon},n\right\}\right)$. To the best of our knowledge, our algorithms are the first FPTAS for any multiperiod version of the Knapsack problem since its study began in 1980s. For MPBKP-SS, we prove that a natural greedy algorithm is a $2$-approximation when items have the same size. Our algorithms also provide insights on how other multiperiod versions of the knapsack problem may be approximated.

%\keywords{approximation algorithms \and knapsack problem \and optimization.}
		\end{onehalfspace}
	\end{abstract}
	
	%\tableofcontents
	\thispagestyle{empty}
	\setlength{\parskip}{.1in}
	\maketitle

	\clearpage
	\setcounter{page}{1}

	\section{Introduction}
	
	% \leavevmode
% \\
% \\
% \\
% \\
% \\
\section{Introduction}
\label{introduction}

AutoML is the process by which machine learning models are built automatically for a new dataset. Given a dataset, AutoML systems perform a search over valid data transformations and learners, along with hyper-parameter optimization for each learner~\cite{VolcanoML}. Choosing the transformations and learners over which to search is our focus.
A significant number of systems mine from prior runs of pipelines over a set of datasets to choose transformers and learners that are effective with different types of datasets (e.g. \cite{NEURIPS2018_b59a51a3}, \cite{10.14778/3415478.3415542}, \cite{autosklearn}). Thus, they build a database by actually running different pipelines with a diverse set of datasets to estimate the accuracy of potential pipelines. Hence, they can be used to effectively reduce the search space. A new dataset, based on a set of features (meta-features) is then matched to this database to find the most plausible candidates for both learner selection and hyper-parameter tuning. This process of choosing starting points in the search space is called meta-learning for the cold start problem.  

Other meta-learning approaches include mining existing data science code and their associated datasets to learn from human expertise. The AL~\cite{al} system mined existing Kaggle notebooks using dynamic analysis, i.e., actually running the scripts, and showed that such a system has promise.  However, this meta-learning approach does not scale because it is onerous to execute a large number of pipeline scripts on datasets, preprocessing datasets is never trivial, and older scripts cease to run at all as software evolves. It is not surprising that AL therefore performed dynamic analysis on just nine datasets.

Our system, {\sysname}, provides a scalable meta-learning approach to leverage human expertise, using static analysis to mine pipelines from large repositories of scripts. Static analysis has the advantage of scaling to thousands or millions of scripts \cite{graph4code} easily, but lacks the performance data gathered by dynamic analysis. The {\sysname} meta-learning approach guides the learning process by a scalable dataset similarity search, based on dataset embeddings, to find the most similar datasets and the semantics of ML pipelines applied on them.  Many existing systems, such as Auto-Sklearn \cite{autosklearn} and AL \cite{al}, compute a set of meta-features for each dataset. We developed a deep neural network model to generate embeddings at the granularity of a dataset, e.g., a table or CSV file, to capture similarity at the level of an entire dataset rather than relying on a set of meta-features.
 
Because we use static analysis to capture the semantics of the meta-learning process, we have no mechanism to choose the \textbf{best} pipeline from many seen pipelines, unlike the dynamic execution case where one can rely on runtime to choose the best performing pipeline.  Observing that pipelines are basically workflow graphs, we use graph generator neural models to succinctly capture the statically-observed pipelines for a single dataset. In {\sysname}, we formulate learner selection as a graph generation problem to predict optimized pipelines based on pipelines seen in actual notebooks.

%. This formulation enables {\sysname} for effective pruning of the AutoML search space to predict optimized pipelines based on pipelines seen in actual notebooks.}
%We note that increasingly, state-of-the-art performance in AutoML systems is being generated by more complex pipelines such as Directed Acyclic Graphs (DAGs) \cite{piper} rather than the linear pipelines used in earlier systems.  
 
{\sysname} does learner and transformation selection, and hence is a component of an AutoML systems. To evaluate this component, we integrated it into two existing AutoML systems, FLAML \cite{flaml} and Auto-Sklearn \cite{autosklearn}.  
% We evaluate each system with and without {\sysname}.  
We chose FLAML because it does not yet have any meta-learning component for the cold start problem and instead allows user selection of learners and transformers. The authors of FLAML explicitly pointed to the fact that FLAML might benefit from a meta-learning component and pointed to it as a possibility for future work. For FLAML, if mining historical pipelines provides an advantage, we should improve its performance. We also picked Auto-Sklearn as it does have a learner selection component based on meta-features, as described earlier~\cite{autosklearn2}. For Auto-Sklearn, we should at least match performance if our static mining of pipelines can match their extensive database. For context, we also compared {\sysname} with the recent VolcanoML~\cite{VolcanoML}, which provides an efficient decomposition and execution strategy for the AutoML search space. In contrast, {\sysname} prunes the search space using our meta-learning model to perform hyperparameter optimization only for the most promising candidates. 

The contributions of this paper are the following:
\begin{itemize}
    \item Section ~\ref{sec:mining} defines a scalable meta-learning approach based on representation learning of mined ML pipeline semantics and datasets for over 100 datasets and ~11K Python scripts.  
    \newline
    \item Sections~\ref{sec:kgpipGen} formulates AutoML pipeline generation as a graph generation problem. {\sysname} predicts efficiently an optimized ML pipeline for an unseen dataset based on our meta-learning model.  To the best of our knowledge, {\sysname} is the first approach to formulate  AutoML pipeline generation in such a way.
    \newline
    \item Section~\ref{sec:eval} presents a comprehensive evaluation using a large collection of 121 datasets from major AutoML benchmarks and Kaggle. Our experimental results show that {\sysname} outperforms all existing AutoML systems and achieves state-of-the-art results on the majority of these datasets. {\sysname} significantly improves the performance of both FLAML and Auto-Sklearn in classification and regression tasks. We also outperformed AL in 75 out of 77 datasets and VolcanoML in 75  out of 121 datasets, including 44 datasets used only by VolcanoML~\cite{VolcanoML}.  On average, {\sysname} achieves scores that are statistically better than the means of all other systems. 
\end{itemize}


%This approach does not need to apply cleaning or transformation methods to handle different variances among datasets. Moreover, we do not need to deal with complex analysis, such as dynamic code analysis. Thus, our approach proved to be scalable, as discussed in Sections~\ref{sec:mining}.
	
	
%	\subsection{Notation}	
%	\[ \Qcal(\Scal) = \sum_{i \in \Scal} q_i \]
%	\[ \Rcal(\Scal) = \sum_{i \in \Scal} r_i \]
%	\[ \hat{\Rcal}(\Scal) = \sum_{i \in \Scal} \hat{r}_i \]
%	\[ \Pcal(\Scal) = \Rcal(\Scal) - \sum_{t=1}^T B_t \left[\sum_{j\in  \Scal : d_j = t } q_j - \max_{0 \leq t' < t}\left\{ c_t - c_{t'} - \sum_{j\in \Scal : t'+1 \leq d_j \leq  t-1}q_j \right\}\right]^+ \]
	
%	For a solution $\Scal =  \Scal(1) \cup \Scal(2) \cup \cdots \Scal(T)$ with $\Scal(t) = (i^{(t)}_1, \ldots, i^{(t)}_{I_t})$ denoting an indexing of items in the solution $\Scal$ with deadline $t$, we define the leftover capacity for serving items in $\Scal$ via the Lindley-type recursion:
%	\begin{align*}
%	W^{(1)}_1 &= c_1 \\
%	W^{(t)}_1 &= \left( W^{(t-1)}_{I_{t-1}} - q_{i^{(t-1)}_{I_{t-1}}}\right)^+  + (c_t-c_{t-1})   & (t \geq 2)\\
%	W^{(t)}_j &= \left( W^{(t)}_{j-1} - q_{i^{(t)}_{j-1}} \right)^+   & (j \geq 2).
%	\end{align*}
	
%	Based on $W^{(t)}_j$ defined above, we then define the rounded profit of the set $\Scal$ as the sum of the rounded (down) reward of each item in $\Scal$ minus the rounded (up) penalty for each item in $\Scal$ (the discretization quantum $\kappa$ will be clear from the context and hence we suppress the dependence of $\hat{\Pcal}$ on it):
%	
%	\[ \hat{\Pcal}(\Scal) = \hat{
%		\Rcal}(\Scal) - \sum_{t=1}^T \sum_{j =1 }^{I_t} \left\lceil B_t \cdot \left( q^{(t)}_j -  W^{(t)}_j \right)^+ \right \rceil_\kappa . \]
	
	%\clearpage
\section{Problem Formulation and Main Results}\label{sec:form}
In this section, we formally introduce the Multiperiod Binary Knapsack Problem (MPBKP), as well as the generalized versions: the Multiperiod Binary Knapsack Problem with Soft capacity constraints (MPBKP-S), {  and Multiperiod Binary Knapsack Problem with Soft Stochastic Capacity constraints (MPBKP-SS)}.

%	\varun{Maybe we can mention that intuitively this is $T$ knapsack problems with capacity $c_t-c_{t-1}$ for the $t$th problem but where we can (a) carry forward unused capacity, or (b) additionally buy extra capacity at cost $B_t$ (and optionally carry that forward too) }

\subsection{Multiperiod binary Knapsack problem (MPBKP)}

An instance of MPBKP is given by a set of $n$ items, each associated with a triple $(r_i,q_i,d_i)$, and a sequence of knapsack capacities $\{c_1,\ldots,c_T\}$. For each item $i$, we get reward $r_i$ if and only if $i$ is included in the knapsack by time $d_i$. We assume that $r_i\in \NN$, $q_i\in\NN$ and $d_i\in [T]:=\{1,\ldots, T\}$. The knapsack capacity at time $t$ is $c_t$, and by convention $c_0=0$. The MPBKP can be written in the integer program (IP) form:
\begin{subequations}\label{MPBKP}
	\begin{align}
	&\max_x z = \sum_{i=1}^{n} r_ix_i\\
	&\text{ s.t. } \sum_{j: d_j\le t} q_jx_j\le c_{t},\quad \forall t=1,\ldots, T\\
	&\qquad x_i\in\{0,1\},\quad \forall i=1,\ldots,n
	\end{align}
\end{subequations}
where $x_i$'s are binary decision variables, i.e., $x_i$ is~$1$ if item $i$ is included in the knapsack and is~$0$ otherwise. In~\eqref{MPBKP}, we aim to pick a subset of items to maximize the objective function, which is the total reward of picked items, subject to the constraints that by each time $t$, the total size of picked items with deadlines up to $t$ does not exceed the knapsack capacity at time $t$, which is $c_{t}$. 
For each $t\in [T]$, let $\mathcal{I}(t):=\{i\in [n]\mid d_i=t\}$ denote the set of items with deadline $t$. Note that without loss of generality, we may assume that $\mathcal{I}(t)\ne\emptyset, \forall t$ and $c_t>0$.
%for some $t\in [T]$, i.e., $d_i\ne t$ for all $i\in [n]$, then we can eliminate this $t$ by redefining $t:= \min_{j: d_j> t} d_j$. Thus, we can without loss of generality assume that $\mathcal{I}(t)\ne \emptyset$ for all $t\in [T]$, which also implies that $T\le n$. 
We further note that the decision variables $x_i$'s in~\eqref{MPBKP} are binary, but if we relax this to any nonnegative integers, the problem becomes the so-called multiperiod knapsack problem (MPKP) as in~\cite{faaland1981multiperiod}. %As we will see in the next subsection, MPBKP can be viewed as a special case of MPBKP-S, and thus is not further discussed in this paper. 
%\begin{comment}
Our first main result is the following theorem on MPBKP.
\begin{theorem}\label{mainthm1}
	An FPTAS exists for MPBKP. Specifically, there exists a deterministic algorithm that achieves $(1+\epsilon)$-approximation in $\tilde{\mathcal{O}}\left(\min\left\{n+\frac{T^{3.25}}{\epsilon^{2.25}},n+\frac{T^{2}}{\epsilon^{3}},\frac{nT}{\epsilon^2},\frac{n^2}{\epsilon}\right\}\right)$.
\end{theorem}
%\end{comment}
As we will see shortly, MPBKP can be viewed as a special case of MPBKP-S. In Section~\ref{sec:MPBKP}, we will provide an approximation algorithm for MPBKP with runtime $\tilde{\mathcal{O}}\left(n+\frac{T^{3.25}}{\epsilon^{2.25}}\right)$. An alternative algorithm with runtime $\tilde{\Ocal}\left(n+\frac{T^{2}}{\epsilon^{3}}\right)$ is provided in Appendix~\ref{appT2}. In Section~\ref{sec:approx2}, we will provide an approximation algorithm for MPBKP-S with runtime $\tilde{O}\left(\frac{nT}{\epsilon^2}\right)$, which is also applicable to MPBKP.

\subsection{Multiperiod binary Knapsack problem with soft capacity constraints (MPBKP-S)}
In MPBKP-S, the capacity constraints in~\eqref{MPBKP} no longer exist, i.e., the total size of selected items at each time step is allowed to be greater than the total capacity up to that time, however, there is a penalty rate $B_t\in\NN$ for each unit of overflow at period $t$. We assume that $B_t>\max_{i\in[n]:d_i\le t}\frac{r_i}{q_i}$ to avoid trivial cases (any item with $\frac{r_i}{q_i}\ge B_t$ and $d_i\le t$ will always be added to generate more profit). In the IP form, MPBKP-S can be written as 
\begin{comment}
\begin{equation}\label{MPBKP-S}
\begin{aligned}
\max_{x \in \{0,1\}^n} z(x) :=&\sum_{i=1}^nr_ix_i - B\cdot \Bigg\{\bigg[\sum_{j\in \mathcal{I}(1)}q_jx_j-c_1\bigg]^+ \\&+ \bigg[\sum_{j\in \mathcal{I}(2)}q_jx_j-\Big(c_1-\sum_{j\in \mathcal{I}(1)}q_jx_j\Big)^+-(c_2-c_1)\bigg]^++\cdots\Bigg\}
%\\
%&\text{ s.t. } x_i\in \{0,1\}, \quad\forall i\in[n]
\end{aligned}
\end{equation}
where $\left[a\right]^+$ is the maximum of $a$ and $0$. In the objective function, $\sum_{j\in \mathcal{I}(1)}q_jx_j$ is the total size of selected items with deadline~$1$, and $c_1$ is the capacity for time $1$, thus $B\cdot \left[\sum_{j\in \mathcal{I}(1)}q_jx_j-c_1\right]^+ $ is the penalty generated at time~$1$. Similarly, $\sum_{j\in \mathcal{I}(2)}q_jx_j$ is the total size of selected items with deadline~$2$, $c_2-c_1$ is the incremental capacity from time~$1$ to time~$2$, and $\left(c_1-\sum_{j\in \mathcal{I}(1)}q_jx_j\right)^+$ is the leftover capacity (if any) carried from time~$1$, thus $B\cdot \left[\sum_{j\in \mathcal{I}(2)}q_jx_j-\left(c_1-\sum_{j\in \mathcal{I}(1)}q_jx_j\right)^+-(c_2-c_1)\right]^+$ is the penalty generated at time~$2$. We continue this pattern and write out the penalties generated at each time.



An equivalent expression
of~\eqref{MPBKP-S} is the following.
\begin{equation}\label{MPBKP-S2}
\begin{aligned}
&\max_{x \in\{0,1\}^n} z(x) :=\sum_{i=1}^nr_ix_i - B\cdot \sum_{t=1}^T\left[\sum_{j\in \mathcal{I}(t)}q_jx_j-\max_{0\leq t' < t}\left\{ c_t - c_{t'}-\sum_{j  \in \Scal : t'+1 \leq d_j \leq t-1} q_j x_j\right\}\right]^+ .
%\\
%&\text{ s.t. } x_i\in \{0,1\}, \quad\forall i\in[n]
\end{aligned}
\end{equation}
Further, if we add decision variables $y_t, t=1,\ldots, T$, which represents the overflow at time $t$, then the problem can be written as
\end{comment}
\begin{subequations}\label{MPBKP-S3}
	\begin{align}
	&\max_{x,y} \sum_{i\in[n]}r_ix_i - \sum_{t=1}^TB_ty_t\\
%	&\text{s.t. } \sum_{i\in\Ical(1)\cup\cdots\cup\Ical(t)}q_ix_i - \sum_{s=1}^ty_s \le \sum_{s=1}^ta_t = c_t,\quad \forall t: 1\le t\le T\\
	&\text{s.t. } \sum_{i\in\Ical(1)\cup\cdots\cup\Ical(t)}q_ix_i - \sum_{s=1}^ty_s \le c_t,\quad \forall t: 1\le t\le T\\
	&\qquad x_i\in\{0,1\},\quad y_t\ge 0,
	\end{align}
\end{subequations}
where the  decision variables $y_t, t=1,\ldots, T$ represent the units of overflow at time~$t$, and {  $c_t-c_{t-1}$} is the incremental capacity at time~$t$. The objective is to choose a subset of the $n$ items to maximize the total profit, which is the sum of the rewards of the selected items  minus the sum of penalty paid at each period, and the constraints enforce that the total size of accepted items by the end of each period must not exceed the sum of the cumulative capacity and the units of overflow. Our second main result is the following theorem on MPBKP-S.
\begin{theorem}\label{mainthm2}
	An FPTAS exists for MPBKP-S. Specifically, there exists an algorithm which achieves $(1+\epsilon)$-approximation in  ${\mathcal{O}}\left(\frac{n\log n}{\epsilon}\cdot \min\left\{\frac{T}{\epsilon}, n\right\}\right)$.
\end{theorem}

In section~\ref{sec:approx2} we will present an approximation algorithm for solving MPBKP-S with time complexity $\mathcal{O}\left(\frac{nT\log n}{\epsilon^2}\right)$. An alternative FPTAS with runtime $\Ocal\left(\frac{n^2}{\epsilon}\right)$ is provided in Appendix~\ref{simple-MPBKP-S}. 
For the ease of presentation, our algorithms and analysis are presented for the case $B_t=B$, but they can be generalized to the heterogeneous $\{B_1,\ldots,B_T\}$ in a straightforward manner. It is worth noting that the algorithm for MPBKP that we introduce in section~\ref{sec:MPBKP} does not extend to MPBKP-S, and we will make this clear in the beginning of section~\ref{sec:approx2}.

{ 
\subsection{Multiperiod Binary Knapsack Problem with Soft Stochastic Capacity Constraints (MPBKP-SS)}
The MPBKP-SS formulation is similar to~\eqref{MPBKP-S3}, except that the vector of knapsack sizes $(c_1, \ldots, c_T)$ follows some arbitrary joint distribution given to the algorithm as the set of possible sample path (realization) of knapsack sizes and the probability of each sample path. We use $\omega$ to index sample paths which we denote by $\{c_t(\omega)\}$, $p(\omega)$ as the probability of sample path $\omega$, and $\Omega$ as the set of possible sample paths. The goal is to pick a subset of items before the realization of $\omega$ so as to maximize the expected total profit, which is the sum of the rewards of the selected items deducted by the total (expected) penalty. For a sample $\omega\in\Omega$ let $y_t(\omega)$ be the overflow at time $t$. Then, we can write the problem in IP form as:
\begin{subequations}\label{MPBKP-SS}
\begin{align}
&\max_{x,y} \sum_{i\in[n]}r_ix_i - \mathbb{E}_\omega\left[B_t\cdot \sum_{t=1}^Ty_t(\omega)\right]\\
&\text{s.t. } \sum_{i\in\Ical(1)\cup\cdots\cup\Ical(t)}q_ix_i - \sum_{s=1}^ty_s(\omega) \le   c_t(\omega),\quad \forall \omega\in\Omega, 1\le t\le T\\
&\qquad x_i\in\{0,1\},\quad y_t\ge 0
\end{align}
\end{subequations}

Our third main result is the following theorem on MPBKP-SS, which asserts a greedy algorithm for the special case when all items are of the same size. Details will be provided in Section~\ref{sec:unit-MPBKP-SS}.
\begin{theorem}\label{mainthm3}
	If $q_i=q$ for all $i\in[n]$, then there exists a greedy algorithm that achieves $2$-approximation for MPBKP-SS in $\Ocal\left(n^2T|\Omega|\right)$.
\end{theorem}

We further note that both MPBKP-S and MPBKP-SS are special cases of non-monotone submodular maximization which is \emph{not} non-negative, for which not many general approximations are known. In that sense, studying these problems would be an interesting direction to develop techniques for it.
}
	
	
\section{FPTAS for MPBKP}\label{sec:MPBKP}
In this section, we provide an FPTAS for the MPBKP with time complexity $\tilde{\mathcal{O}}\left(n+\frac{T^{3.25}}{\epsilon^{2.25}}\right)$. We will apply the ``functional approach'' as used in~\cite{chan:OASIcs:2018:8299}. The main idea is to use the results on function approximations~\citep{chan:OASIcs:2018:8299,jin:LIPIcs:2019:10652} as building blocks -- for each period we approximate one function that gives, for every choice of available capacity, the maximum reward obtainable by selecting items in that period. We then combine ``truncated'' version of these functions using $(\max,+)$-convolution. This idea, despite its simplicity, allows us to obtain an FPTAS for MPBKP. Such a result should not be taken as granted -- as we will see in the next section, this method does not apply for MPBKP-S, even though it is just a slight generalization of MPBKP. 

We begin with some preliminary definitions and notations. For a given set of item rewards and sizes, $\Ical = \{(r_1,q_1), \ldots, (r_{n'}, q_{n'})\}$, define the function
\begin{align}\label{eqn:func}
f_\Ical(c) := \max_{x_1,\ldots,x_{n'}}\left\{\sum_{i\in\Ical}r_ix_i\ :\ \sum_{i\in\Ical}q_ix_i\le c, \ x_1,\ldots,x_{n'}\in\{0,1\}\right\}
\end{align}
for all $c\ge 0$, and $f_\Ical(c) := -\infty$ for $c<0$. The function $f_\Ical$ is a nondecreasing step function, and the number of steps is called the \emph{complexity} of that function. Further, for any $\Ical = \Ical_1\sqcup \Ical_2$, i.e., $\Ical$ being a disjoint union of $\Ical_1$ and $\Ical_2$, we have that $f_\Ical = f_{\Ical_1}\oplus f_{\Ical_2}$, where $\oplus$ denotes the $(\max,+)$-\emph{convolution}: $(f\oplus g)(c) = \max_{c'\in\mathbb{R}}\left(f(c')+g(c-c')\right)$.

We define the \emph{truncated function} $f_\Ical^{c'}$ as follows:
\begin{align}
f_\Ical^{c'}(c) = \begin{cases}
f_\Ical(c) &  c\le c',\\
-\infty & c>c'.
\end{cases}
\end{align}
Recall that we denote the set of items with deadline $t$ by $\Ical(t)$. We next define the function $f_t$ as follows: 
\begin{align}\label{eqn:ft}
f_t := \begin{cases}
f_{\Ical(1)}^{c_1} & t=1,\\
\left(f_{t-1}\oplus f_{\Ical(t)}\right)^{c_t} & t\ge 2.
\end{cases}
\end{align}
%In other words, $f_t$'s are defined recursively: for $t=1$, let $f_1 := f_{\Ical(1)}^{c_1}$; for $t\ge 2$, we define $f_t = \left(f_{t-1}\oplus f_{\Ical(t)}\right)^{c_t}$.
%, we write $f_t$ for $f_{\Ical(t)}$. 
In words, each function value of $f_t(c)$ corresponds to a feasible, in fact an optimal, solution $x$ for items with deadline at most $t$ as the next proposition shows.
\begin{proposition}\label{prop:optimalfunc}
Let $x^*$ be the optimal solution for MPBKP~\eqref{MPBKP}. We have that
the optimal value of~\eqref{MPBKP}, $\sum_{i\in[n]}r_ix_i^*$, satisfies
$
\sum_{i\in[n]}r_ix_i^*  = f_T(c_T).
$
\end{proposition}

Proposition~\ref{prop:optimalfunc} implies that, to obtain an approximately optimal solution for MPBKP~\eqref{MPBKP}, it is sufficient to have a good approximation for the function
\begin{align}
f_T = \left(\cdots\left(\left(f_{\Ical(1)}^{c_1}\oplus f_{\Ical(2)}\right)^{c_2}\oplus f_{\Ical(3)}\right)^{c_3}\cdots\oplus f_{\Ical(T)}\right)^{c_{T}}.
\end{align}

We say that a function $\tilde{f}$ approximates the nonnegative function $f$ with factor $1+\epsilon$ if $\tilde{f}(c)\le f(c)\le (1+\epsilon)\tilde{f}(c)$ for all $c\in\mathbb{R}$. It should be clear that if $\tilde{f}$ approximates $f$ with factor $1+\epsilon$ and $\tilde{g}$ approximates $g$ with factor $1+\epsilon$, then $\tilde{f}\oplus\tilde{g}$ approximates $f\oplus g$ with factor $1+\epsilon$. We then introduce the following result from~\cite{jin:LIPIcs:2019:10652} for 0-1 Knapsack problem. 
\begin{lemma}[\cite{jin:LIPIcs:2019:10652}]\label{lem:01}
Given a set $\Ical=\{(r_1,q_1),\ldots,(r_n,q_n)\}$, we can obtain $\tilde{f}_{\Ical}$ that approximates $f_\Ical$ (defined in~\eqref{eqn:func}) with factor $1+\epsilon$ and complexity $\tilde{O}\left(\frac{1}{\epsilon}\right)$ in $\tilde{O}\left(n+\left(1/\epsilon\right)^{2.25}\right)$.
\end{lemma}
%Suppose that $\tilde{f}_\Ical$ has complexity $l$, then, we denote by $(C_k,R_k)$ as the ``steps'' of function $\tilde{f}_\Ical$, i.e., for $k=1,\ldots,l$, we have that $\tilde{f}_\Ical(C_k) = R_k$ and $C_k = \min_{\tilde{f}_{\Ical}(c)=R_k}c$. 
With the above lemma, we present Algorithm~\ref{alg:MPBKP} for MPBKP.
\begin{algorithm}[ht]
\footnotesize
\caption{FPTAS for MPBKP}
\label{alg:MPBKP}
\algsetblock[Name]{Parameters}{}{0}{}
\algsetblock[Name]{Initialize}{}{0}{}
\algsetblock[Name]{Define}{}{0}{}
\begin{algorithmic}[1]
	\Statex \textbf{Input:} $\epsilon, [n], c_1,\ldots, c_T$  \Comment {Set of items to be packed, cumulative capacities}
	\Statex \textbf{Output:} $\tilde{f}_t$ \Comment Approximation of function $f_t$
	\State Discard all items with $r_i\le \frac{\epsilon}{n}\max_jr_j$ and relabel the items 
	\State $r_0\gets \min_ir_i$ \Comment Lower bound of solution value
	\State $m\gets \left\lceil\log_{1+\epsilon}\frac{n^2}{\epsilon}\right\rceil$ \Comment number of distinct rewards to be considered, each in the form $r_0\cdot(1+\epsilon)^k$
	%		\State Initialize $\tilde{A}(0,r) = \begin{cases}
	%		0 & r = 0,\\
	%		-\infty & r > 0.
	%		\end{cases}$
	\State Obtain $\tilde{f}_{\Ical(1)}$ that approximates $f_{\Ical(1)}$ with factor $(1+\epsilon)$ using Lemma~\ref{lem:01}
	\State $\tilde{f}_1:= \tilde{f}_{\Ical(1)}^{c_1}$ \Comment $\tilde{f}_1$ has complexity at most $m=\tilde{\mathcal{O}}\left(\frac{1}{\epsilon}\right)$
	\For {$t=2,\ldots, T$}
	\State Obtain $\tilde{f}_{\Ical(t)}$ that approximates $f_{\Ical(t)}$ with factor $(1+\epsilon)$ using Lemma~\ref{lem:01}
	\State $l\gets$ complexity of $\tilde{f}_{\Ical(t)}$ \Comment $l=\tilde{\mathcal{O}}\left(\frac{1}{\epsilon}\right)$
	\State Compute (all breakpoints and their values of) $\hat{f}_{t}:= \left(\tilde{f}_{t-1}\oplus \tilde{f}_{\Ical(t)}\right)^{c_t}$, taking $m\cdot l$ time
\Comment $\hat{f}_t$ has complexity $\tilde{\mathcal{O}}\left(\frac{1}{\epsilon^2}\right)$
	\State $\tilde{f}_t := r_0\cdot (1+\epsilon)^{\left\lfloor\log_{1+\epsilon}\left(\frac{\hat{f}_t}{r_0}\right)\right\rfloor}$ \Comment Round $\hat{f}_t$ down to the nearest $r_0\cdot (1+\epsilon)^k$. Now $\tilde{f}_t$ has complexity at most $m=\tilde{\mathcal{O}}\left(\frac{1}{\epsilon}\right)$
	\EndFor
\end{algorithmic}
\end{algorithm}

We now describe the intuition behind Algorithm~\ref{alg:MPBKP}. We first discard all items with reward $r_i\le \frac{\epsilon}{n}\max_jr_j$. The maximum we could lose is $n\cdot \frac{\epsilon}{n}\max_jr_j = \epsilon\max_jr_j$, which is at most $\epsilon$ fraction of the optimal value. We next obtain all $\tilde{f}_{\Ical(t)}$, for all $t=1,\ldots, T$, that approximate $f_{\Ical(t)}$ (as defined in~\eqref{eqn:func}) within a $(1+\epsilon)$ factor. These functions $\tilde{f}_{\Ical(t)}$ have complexity $\tilde{\Ocal}\left(\frac{1}{\epsilon}\right)$. We start with combining the functions of period~$1$ and period~$2$ using $(\max,+)$-convolution. To enforce the constraint that the total size of selected items in period~$1$ does not exceed the capacity of period~$1$, we truncate $\tilde{f}_{\Ical(1)}$ by $c_1$ (so that any solution using more capacity in period~$1$ results in $-\infty$ reward) and do the convolution on the truncated function $\tilde{f}_1$. Since both functions are step functions with complexity $\tilde{\Ocal}\left(\frac{1}{\epsilon}\right)$, the $(\max,+)$ convolution can be done in time $\Ocal\left(\frac{1}{\epsilon^2}\right)$. The resulting function $\hat{f}_2$ would have complexity $\Ocal\left(\frac{1}{\epsilon^2}\right)$. To avoid inflating the complexity throughout different periods (which increases computation complexity), the function $\hat{f}_2$ is rounded down to the nearest $r_0\cdot (1+\epsilon)^k$, where $r_0:=\min_jr_j$ and $k$ is some nonnegative integer. Note that $r_0$ is a lower bound of any solution value. After discarding small-reward items, we have that $\frac{\max_jr_j}{r_0}\le \frac{n}{\epsilon}$, which implies that $n\max_jr_j = \frac{n^2}{\epsilon}r_0$ is an upper bound for the optimal solution value. Therefore, after rounding down the function values of $\hat{f}_2$ and obtaining $\tilde{f}_2$, there are at most $\log_{1+\epsilon}\frac{n^2}{\epsilon}\approx \frac{1}{\epsilon}\log\frac{n^2}{\epsilon}$ different values on $\tilde{f}_2$. Now we have brought down the complexity of $\tilde{f}_2$ again to $\tilde{\Ocal}\left(\frac{1}{\epsilon}\right)$, at an additional $(1+\epsilon)$ factor loss in the approximation error. We then move to period~$3$ and continue this pattern of $(\max,+)$-convolution, truncation, and rounding down. In the end when we reach period $T$, $\tilde{f}_T$ will only contain feasible solutions to~\eqref{MPBKP}, and approximate $f_T$ with total approximation factor of $(1+\epsilon)^T\approx (1+T\epsilon)$. Formally, we have the following lemma which shows the approximation factor of $\tilde{f}_t$ for $f_t$.
\begin{lemma}\label{lem:fapprox}
Let $\tilde{f}_t$ be the functions obtained from Algorithm~\ref{alg:MPBKP}, and let $f_t$ be defined as in~\eqref{eqn:ft}. Then, $\tilde{f}_t$ approximates $f_t$ with factor $(1+\epsilon)^t$, i.e., $\tilde{f}_t(c)\le f_t(c)\le (1+\epsilon)^t\tilde{f}_t(c)$ for all $0\le c\le c_t$.
\end{lemma}



Lemma~\ref{lem:fapprox} and Proposition~\ref{prop:optimalfunc} together imply that $\tilde{f}_T(c_T)$, obtained from Algorithm~\ref{alg:MPBKP}, approximates the optimal value of MPBKP~\eqref{MPBKP} by a factor of $(1+\epsilon)^T \approx (1+T\epsilon)$. In Algorithm~\ref{alg:MPBKP}, obtaining $\tilde{f}_{\Ical(t)}$ for all $t=1,\ldots,T$ takes time $\tilde{O}\left(n+{T}/{\epsilon^{2.25}}\right)$; computing the $(\max,+)$-convolution on $\tilde{f}_{t-1}\oplus \tilde{f}_{\Ical(t)}$ for all $t$ take time $T\cdot m\cdot l = \tilde{O}\left(T/\epsilon^2\right)$. Therefore, Algorithm~\ref{alg:MPBKP} has runtime $\tilde{O}\left(n+T/\epsilon^{2.25}\right)$. As a result, we have the following proposition. 
\begin{proposition}
Taking $\epsilon' = T\epsilon$, Algorithm~\ref{alg:MPBKP} achieves $(1+\epsilon')$-approximation for MPBKP in $\tilde{O}\left(n+\frac{T^{3.25}}{{\epsilon'}^{2.25}}\right)$.
\end{proposition}


\section{FPTAS for MPBKP-S}\label{sec:approx2}

In this section, we provide an FPTAS for the MPBKP-S with time complexity $\mathcal{O}\left(\frac{Tn\log n}{\epsilon^2}\right)$. An alternative FPTAS with time complexity $\mathcal{O}\left(\frac{n^2\log n}{\epsilon}\right)$ is provided in Appendix~\ref{simple-MPBKP-S}. Combining the two, we show that our algorithms achieve $(1+\epsilon)$ approximation ratio in time $\mathcal{O}\left(\frac{n\log n}{\epsilon}\cdot\min\left\{\frac{T}{\epsilon},n\right\}\right)$, which proves Theorem~\ref{mainthm2}. 
We should note that the algorithm in the previous section does not apply here: we could similarly define a function which gives the maximum profit ($=$reward$-$penalty) under a given capacity constraint, but the main obstacle is on the $(\max,+)$-convolution because profit does not ``add up''. In other words, the total profit we earn by selecting items in the set $\Scal_1\cup\Scal_2$ is not the sum of the profits we earned by selecting $\Scal_1$ and $\Scal_2$ separately. For this reason, we can no longer rely on the techniques used in function approximation and $(\max,+)$-convolution as in~\cite{chan:OASIcs:2018:8299,jin:LIPIcs:2019:10652}. Instead, our main idea is motivated by the techniques that originated from earlier papers~\citep{ibarra1975fast,lawler1979fast}, but adapting their technique to MPBKP-S requires significant modifications as we show in this section. %To a large extent the algorithms and proofs follow the structure of Section~\ref{sec:approx} and for succinctness we only emphasize the modifications necessary. 
We restrict our presentation to the case $B_t =B$ for readability, but our algorithms and analysis generalize in a straightforward manner when the penalties for buying capacity are heterogeneous $\{B_1, \ldots, B_T\}$ (by replacing $B$ with $\min_{\tau\le t}B_\tau$ in the calculations of profit/penalty at period $t$ on line 7 of Algorithm~\ref{alg:FPTAS_nTlogn_large2}).

%The main idea is motivated by the technique that originated from~\cite{ibarra1975fast}, but adapting their technique to MPBKP requires significant modifications as we show in this Section. 

%\subsubsection*{Preparation}
\noindent \textbf{Preliminaries:} We first introduce some notation. From now on, let $\Rcal(\Scal):=\sum_{i\in\Scal}r_i$. The optimal solution set to~\eqref{MPBKP-S3} is denoted by $\mathcal{S}^*$. The total profit earned can be expressed as a function of the solution set $\mathcal{S}$:
\begin{align}
\Pcal(\mathcal{S}) = \Rcal(\Scal) - B\cdot \sum_{t=1}^T\left[\sum_{j\in \Scal\cap\mathcal{I}(t)}q_j-\max\left\{c_t-\sum_{j\in \Scal , d_j \leq t }q_j,\ c_t-c_{t-1}\right\}\right]^+.
\end{align}
Let $p_i$ be the profit of item $i$, which is defined as the profit earned if we select only $i$, i.e., $p_i = r_i-B \cdot \left(q_i-c_{d_i}\right)^+$. Without loss of generality, we assume that each item $i$ is by itself profitable, i.e., $p_i\ge 0$, so one profitable solution would be $\{i\}$. % This assumption is natural as otherwise there exists some item $i$ that will only bring down the profit if included in any solution, in which case we may simply discard that item when solving for the problem.
Let $P:=\max_{i}p_i$ and $\bar{P}:=\sum_{i\in[n]}p_i$. %Then since each item $i$ is profitable by itself, 
The following bounds on $\Pcal(\Scal^*)$  follow: 
\begin{align}\label{upperP}
P \leq \Pcal(\mathcal{S}^*)\le \bar{P} \leq nP.
\end{align}

%\subsubsection*{Partition of items}
\noindent \textbf{Partition of items:} We partition the set of items $[n]$ into two sets: a set of ``large'' items $\Ical_L$ and a set of ``small'' items $\Ical_S$ such that we can bound the number of large items in any optimal solution. The main idea is to use dynamic programming to pick the large items in the solution, and a greedy heuristic for `padding' this partial solution with small items.
The criterion for small and large items is based on balancing the permissible error $\epsilon \Pcal(\Scal^*)$ equally in filling large items and filling small items. Instead of first packing all large items and then all small items, we consider items in the order of their deadlines, and for each deadline $t$, the large items are selected first and then the small items are selected greedily in order of their reward densities. As a result, the approximation error due to large items overall will be $\frac{1}{2}\epsilon \Pcal(\Scal^*)$ and the error due to the small items with each deadline will be $\frac{1}{2T}\epsilon \Pcal(\Scal^*)$. This gives a total approximation error of $\frac{1}{2}\epsilon \Pcal(\Scal^*) + T\cdot \frac{1}{2T}\epsilon \Pcal(\Scal^*) = \epsilon \Pcal(\Scal^*)$.
%\subsection{An alternative FPTAS: separation of items}\label{sec:alter2}
%As in Section~\ref{sec:alter}, we propose in this subsection another FPTAS with time complexity $\mathcal{O}\left(\frac{Tn\log n}{\epsilon^2}\right)$. The algorithms we propose will be similar to Algorithms~\ref{alg:FPTAS_nTlogn_large},~\ref{alg:FPTAS_nTlogn_small},~\ref{alg:FPTAS_nTlogn}, and~\ref{alg:FPTAS_enumerate}. 

Suppose that we can find some $P_0$ that satisfies~\eqref{P0}. \begin{align}\label{P0}
P_0\le \Pcal(\mathcal{S}^*)\le 2P_0.
\end{align}
Then, the set of items is partitioned as follows.
\begin{equation}\label{div2}
\begin{aligned}
\Ical_L := \left\{i\in[n]\mid p_i\ge \frac{1}{2T}\epsilon P_0\right\}; \qquad 
\Ical_S := \left\{i\in[n]\mid p_i< \frac{1}{2T}\epsilon P_0\right\}.
\end{aligned}
\end{equation}

This partition is computed in $\mathcal{O}(n)$ time and is not the dominant term in time complexity. Let $n_L = |\Ical_L|$ and $n_S=|\Ical_S|$, so that $n_L+n_S=n$. 
Further, let 
\begin{align*}
\mathcal{I}_L(t) := \left\{i\in \Scal_L\mid d_i = t\right\}, \quad  \mbox{and} \quad 
\mathcal{I}_S(t) := \left\{i\in \Scal_S\mid d_i = t\right\}
\end{align*}
denote the set of large and small items, respectively, with deadline $t$. 
We will assume that the items in $\Ical_L$ are indexed in non-decreasing order of their deadlines, i.e., $\forall i,j\in \Ical_L$ such that $j\ge i$, we have that $ d_i\le d_j$. Denote by $I_L(t)$ as the index of the last item with deadline $t$, i.e., $I_L(t):= \max_{i\in \Scal_L\cap \mathcal{I}_L(t)} i$. 
For each time $t$, we will also sort the small items in $\mathcal{I}_S(t)$ according to their reward densities, i.e., $\forall i<j$ and $i,j\in \mathcal{I}_S(t)$, $\frac{r_i}{q_i}\ge \frac{r_j}{q_j}$. This sorting only takes place once for each guess $P_0$, and does not affect our overall time complexity result. \\



\begin{algorithm}[ht]
\footnotesize
\caption{DP on large items for MPBKP-S}
\label{alg:FPTAS_nTlogn_large2}
\algsetblock[Name]{Parameters}{}{0}{}
\algsetblock[Name]{Initialize}{}{0}{}
\algsetblock[Name]{Define}{}{0}{}
\begin{algorithmic}[1]
	\Statex \textbf{Input:} $\ \Ical_L, \Delta c,$  \Comment Set of (large) items to be packed, additional capacity available for packing
	\Statex \hspace{0.35in}	$\widetilde{A}(p)$ for all $p = \left\{ 0, 1, \ldots,\left\lceil\frac{16T}{\epsilon^2}\right\rceil \right\} \cdot \kappa $ \Comment A set of partial solutions 
	\Statex \textbf{Output:} $\hat{A}(I_L,p)$ for all $p = \left\{ 0, 1, \ldots,\left\lceil\frac{16T}{\epsilon^2}\right\rceil \right\} \cdot \kappa $ \Comment Set of partial solutions after packing $\Ical_L$
	\State Initialize $\forall p \ : \ \hat{A}(0,p) := \widetilde{A}(p) + \Delta c$	
	\For {$i=1,\ldots, I_L$}
	\For {$p = \left\{ 0, 1, \ldots,\left\lceil\frac{16T}{\epsilon^2}\right\rceil \right\} \cdot \kappa $}
	\State $\hat{A}(i,p) := \hat{A}(i-1, p)$ 
	\Comment If reject item $i$
	\EndFor
	\For {$\bar{p} = \left\{ 0, 1, \ldots,\left\lceil\frac{16T}{\epsilon^2}\right\rceil \right\} \cdot \kappa $}
	\State ${p} = \bar{p} + \hat{r}_i - \left\lceil B\left(q_i - {\color{black}\max\left\{0, \hat{A}(i-1,\bar{p})\right\}}\right)^+\right\rceil_{\kappa}$
	\State $\hat{A}(i, p) = \max\left\{ \hat{A}(i,p ), \hat{A}(i-1,\bar{p}) - q_i \right\}$		\Comment Accept $i$
	\EndFor
	\For{$p = \left\{\left\lceil\frac{16T}{\epsilon^2}\right\rceil,\left\lceil\frac{16T}{\epsilon^2}\right\rceil-1,\ldots,1  \right\}\cdot \kappa$}
	\vspace{0.1cm}
	\If {$\hat{A}(i,p-\kappa)<\hat{A}(i,p)$}
	\vspace{0.1cm}
	\State $\hat{A}(i,p-\kappa) = \hat{A}(i,p)$
	\EndIf
	\EndFor
	\EndFor
\end{algorithmic}
\end{algorithm}

\begin{algorithm}[h]
\footnotesize
\caption{Greedy on small items for MPBKP-S}
\label{alg:FPTAS_nTlogn_small2}
\algsetblock[Name]{Parameters}{}{0}{}
\algsetblock[Name]{Initialize}{}{0}{}
\algsetblock[Name]{Define}{}{0}{}
\begin{algorithmic}[1]
	\Statex \textbf{Input:} $\ \Ical_S$, $\hat{A}(p)$ for all $p = \left\{ 0, 1, \ldots,\left\lceil\frac{16T}{\epsilon^2}\right\rceil \right\} \cdot \kappa $.  \Comment Set of (small) items to be packed, a set of partial solutions
	\Statex \textbf{Output:} $\widetilde{A}(p)$ for all $p = \left\{ 0, 1, \ldots,\left\lceil\frac{16T}{\epsilon^2}\right\rceil \right\} \cdot \kappa $ \Comment Set of partial solutions after packing $\Ical_S$
	\State Initialize $\forall p \ : \ \widetilde{A}(p) = \hat{A}(p)$ 
	\For {$\bar{p}=\left\{ 0, 1, \ldots,\left\lceil\frac{16T}{\epsilon^2}\right\rceil \right\} \cdot \kappa$}
	\Statex  \texttt{// Filter out small items with size larger than $\hat{A}(p)$ } 
	%\State Sort the items in $\mathcal{I}_S(t)$ in decreasing order of reward density $r_i/q_i$
	\State $\widetilde{\mathcal{I}}_S\gets\emptyset$
	\For {$i\in \mathcal{I}_S$}
	\If {$\hat{A}(\bar{p}) \geq q_i$}
	%\State $\Delta p_{i} = r_{i} - B\cdot \left(q_{i}-\hat{A}(I_L(t),p)\right)$
	\State $\widetilde{\mathcal{I}}_S\gets \widetilde{\mathcal{I}}_S\cup \{i\}$
	\EndIf
	\EndFor
	
	
	
	\State $\tilde{\Rcal}_{0'} = 0, \tilde{q}_{0'} = 0$, and relabel the items in $\widetilde{\mathcal{I}}_S$ as $\left\{1',\ldots,|\widetilde{\mathcal{I}}_S|'\right\}$ (in decreasing order of reward density)
	\For {$i' = 1',\ldots, |\widetilde{\mathcal{I}}_S|'$}
	\State  $\tilde{\Rcal}_{i'} = \tilde{\Rcal}_{(i-1)'} + r_{i'}$
	\State  $\tilde{q}_{i'} = \tilde{q}_{(i-1)'} + q_{i'}$
	\EndFor
	
	\State {\texttt{// Add small items using Greedy algorithm}}
	\For {$i' = 1',\ldots, |\widetilde{\mathcal{I}}_S|'$}
	\If {$\tilde{q}_{i'} \leq \hat{A}(\bar{p})$}
	\State $p = \left\lfloor \bar{p} + \tilde{\Rcal}_{i'}\right\rfloor_\kappa$
	\State $\widetilde{A}({p}) = \max\left\{\widetilde{A}(p) , \hat{A}(\bar{p}) - \tilde{q}_{i'}  \right\}$
	\EndIf
	\EndFor
	\EndFor 
\end{algorithmic}
\end{algorithm}


\begin{algorithm}[ht]
\footnotesize
\caption{DP on large items and Greedy on small items for MPBKP-S}
\label{alg:FPTAS_nTlogn2}
\algsetblock[Name]{Parameters}{}{0}{}
\algsetblock[Name]{Initialize}{}{0}{}
\algsetblock[Name]{Define}{}{0}{}
\begin{algorithmic}[1]
	\Define \ $\kappa = \frac{\epsilon^2 P_0 }{8T}$
	\Define \ $\hat{r}_i = \floor{ {r_i} }_\kappa$ \Comment Round down reward
	\Statex  \texttt{// $\widetilde{A}_t(p)=$ leftover capacity for the algorithm's partial solution when earning (rounded) profit $p$ using items with deadlines at most $t$ (small and large)} %and rounded down supply
	\Statex  \texttt{// $\hat{A}_t(p)=$ capacity left for the algorithm's partial solution when earning (rounded) profit $p$ by selecting large items in $\mathcal{I}_L(t)$ with rounded down rewards $\hat{r}$, given the partial solutions $\widetilde{A}_{t-1}(p)$}
	\State Initialize $\hat{A}(0,p) = \widetilde{A}_0(p) = \begin{cases}
	0 & p = 0,\\
	-\infty & p > 0.
	\end{cases}$
	\For {$t=1,\ldots, T$}
	\State Run Algorithm~\ref{alg:FPTAS_nTlogn_large2} with $\Ical_L = \Ical_L(t), \Delta c = c_t - c_{t-1}$, and $\widetilde{A}(p)=\widetilde{A}_{t-1}(p)$ for all $p = \left\{ 0, 1, \ldots,\left\lceil\frac{16T}{\epsilon^2}\right\rceil \right\} \cdot \kappa $, and obtain $\hat{A}_t(p):= \hat{A}(I_L,p)$ for all $p$.
	\State Run Algorithm~\ref{alg:FPTAS_nTlogn_small2} with $\Ical_S=\Ical_S(t)$ and $\hat{A}(p) = \hat{A}(I_L(t),p)$ for all $p = \left\{ 0, 1, \ldots,\left\lceil\frac{16T}{\epsilon^2}\right\rceil \right\} \cdot \kappa $, and obtain $\widetilde{A}_{t}(p):=\widetilde{A}(p)$ for all $p$.
	\EndFor
\end{algorithmic}
\end{algorithm}

\begin{algorithm}[ht]
\footnotesize
\caption{FPTAS for MPBKP-S in $\mathcal{O}(Tn\log n/\epsilon^2)$}
\label{alg:SC_FPTAS}
\algsetblock[Name]{Parameters}{}{0}{}
\algsetblock[Name]{Initialize}{}{0}{}
\algsetblock[Name]{Define}{}{0}{}
\begin{algorithmic}[1]
	\State $P_0\gets {\bar{P}}$
	\State $p^*\gets 0$
	\While {$p^*<(1-\epsilon)P_0$}
	\vspace{0.1cm}
	\State $P_0\gets \frac{P_0}{2}$
	\vspace{0.1cm}
	\State	Run Algorithm~\ref{alg:FPTAS_nTlogn2} with the current $P_0$.
	\State $p^*\gets \max_{\left\{\substack{p\in \left\{ 0,\ldots,\left\lceil\frac{16T}{\epsilon^2}\right\rceil \right\} \cdot \kappa\\ \widetilde{A}_T(p)> -\infty}\right\}}p$
	\EndWhile
\end{algorithmic}
\end{algorithm}
%\subsubsection*{Overview of the algorithm}


\noindent \textbf{Algorithm overview:} Our FPTAS algorithm is given in Algorithm~\ref{alg:SC_FPTAS} which uses a doubling trick to guess the value of $P_0$ satisfying \eqref{P0}, and for each guess uses Algorithm~\ref{alg:FPTAS_nTlogn2} as a subroutine. Algorithm~\ref{alg:FPTAS_nTlogn2} is the main  algorithm for MPBKP-S, which first selects the items with deadline~$1$, then the items with deadline~$2$, and so on. 
For each deadline $t$, we maintain two sets of partial solutions: the first, $\widetilde{A}_t(p)$, corresponds to an approximately optimal (in terms of leftover capacity carried forward to time $t+1$) subset of large and small items with deadline at most $t$ and some \emph{rounded profit} $p$ %(precise definition of rounded profit will be given in~\eqref{Ptilde})
; and the second $\hat{A}_t(p)$ corresponds to the optimal appending of large items with deadline $t$ to the approximately optimal set of solutions corresponding to $\widetilde{A}_{t-1}$.

Given $\widetilde{A}_{t-1}$, we first select large items from $\mathcal{I}_L(t)$ using dynamic programming to obtain $\hat{A}_t$, which is done in Algorithm~\ref{alg:FPTAS_nTlogn_large2}. In other words, {\it given} the partial solutions $\widetilde{A}_{t-1}(\bar{p})$ for all $\bar{p} \in \left\{ 0, 1, \ldots,\left\lceil\frac{16T}{\epsilon^2}\right\rceil \right\} \cdot \kappa$, $\hat{A}_t(p)$ is the maximum capacity left when earning \emph{rounded profit} (precise definition given in~\eqref{Ptilde}) $p$ by adding items in $\mathcal{I}_L(t)$. We then use a greedy heuristic to pick small items from $\mathcal{I}_S(t)$ to obtain $\widetilde{A}_t$, which is done in Algorithm~\ref{alg:FPTAS_nTlogn_small2}. Specifically, our goal in Algorithm~\ref{alg:FPTAS_nTlogn_small2} is to obtain the partial solutions $\widetilde{A}_t(\cdot)$ given the partial solutions $\hat{A}_t(\cdot)$ by packing the small items $\Ical_S(t)$. We initialize $\widetilde{A}_t(\bar{p})$ with $\hat{A}_t(\bar{p})$, and for each $\bar{p}$ we try to augment the solution corresponding to $\hat{A}_t(\bar{p})$ using a subset $\widetilde{\mathcal{I}}_S(t) \subseteq \Ical_S(t)$ defined as $$\widetilde{\mathcal{I}}_S(t):= \{i\in \mathcal{I}_S(t)\mid q_i\le \hat{A}_t(\bar{p})\}.$$  The small items in $\widetilde{\mathcal{I}}_S(t)$ are sorted according to their reward densities, and are added to the solution of $\hat{A}_t(\bar{p})$ one by one. After each addition of a small item, if the new total rounded reward is ${p}$, we compare the leftover capacity with current $\widetilde{A}_t({p})$, and update $\widetilde{A}_t({p})$ with the new solution if it has more leftover capacity. We continue this add-and-compare (and possibly update) until we reach the situation where adding the next small item overflows the available capacity.  

\begin{comment}
We now give intuition behind Algorithms~\ref{alg:FPTAS_nTlogn_large2} and~\ref{alg:FPTAS_nTlogn_small2}, with the rigorous proofs left to the end of this section.
%Similar to the previous section, by letting $\kappa:= \frac{\epsilon^2P_0}{8T}$ and $\hat{r}_i := \kappa\left\lfloor\frac{r_i}{\kappa}\right\rfloor$, we have that $\Pcal(\Scal^*)\le 2P_0\le \left\lceil\frac{16}{\epsilon^2}\right\rceil\kappa$. Since the large items are selected using dynamic program (Algorithm~\ref{alg:FPTAS_nTlogn_large2}), it is straightforward to prove (as we do later) that {\it given} the partial solutions $\widetilde{T}_{t-1}(\bar{p})$ for all $\bar{p} \in \left\{ 0, 1, \ldots,\left\lceil\frac{16T}{\epsilon^2}\right\rceil \right\} \cdot \kappa$, $\hat{A}_t(p)$ is the maximum capacity left when earning rounded profit $p$ by adding items in $\mathcal{I}_L(t)$. 
The intuition for Algorithm~\ref{alg:FPTAS_nTlogn_large2} (selecting large items using dynamic program) is the following: {\it given} the partial solutions $\widetilde{A}_{t-1}(\bar{p})$ for all $\bar{p} \in \left\{ 0, 1, \ldots,\left\lceil\frac{16T}{\epsilon^2}\right\rceil \right\} \cdot \kappa$, $\hat{A}_t(p)$ is the maximum capacity left when earning some \emph{rounded profit} (precise definition given in~\eqref{Ptilde}) $p$ by adding items in $\mathcal{I}_L(t)$.

The intuition behind Algorithm~\ref{alg:FPTAS_nTlogn_small2} for packing small items is similar: Our goal is to obtain the partial solutions $\widetilde{A}_t(\cdot)$ given the partial solutions $\hat{A}_t(\cdot)$ by packing the small items $\Ical_S(t)$. We initialize $\widetilde{A}_t(\bar{p})$ with $\hat{A}_t(\bar{p})$, and for each $\bar{p}$ we try to augment the solution corresponding to $\hat{A}_t(\bar{p})$ using a subset $\widetilde{\mathcal{I}}_S(t) \subseteq \Ical_S(t)$ defined as $$\widetilde{\mathcal{I}}_S(t):= \{i\in \mathcal{I}_S(t)\mid q_i\le \hat{A}_t(\bar{p})\}.$$  The small items in $\widetilde{\mathcal{I}}_S(t)$ are sorted according to their reward densities, and are added to the solution of $\hat{A}_t(\bar{p})$ one by one. After each addition of a small item, if the new total rounded reward is ${p}$, we compare the leftover capacity with current $\widetilde{A}_t({p})$, and update $\widetilde{A}_t({p})$ with the new solution if it has more leftover capacity. We continue this add-and-compare (and possibly update) until we reach the situation where adding the next small item overflows the available capacity.  
\end{comment}

%We further note that these small items have to be added one by one to the solutions of $\hat{A}_t(\bar{p})$, which in the worst case takes $\Ocal\left(\frac{nT}{\epsilon^2}\right)$. We cannot first group the small items into sets of partial solutions and do $(\max,+)$ convolution with solution sets of $\hat{A}_t(p)$, which would take $\Ocal\left(\frac{T^2}{\epsilon^4}\right)$ (similar to~\cite{ibarra1975fast,lawler1979fast}), because again the profits of two sets do not add up when we take the union of these two sets. For this reason, although we could further bound the number of large items from $\Ocal(n)$ to $\Ocal\left(\frac{T}{\epsilon^2}\right)$ in a similar manner as~\cite{lawler1979fast}, we do not adopt that method as it does not improve the overall time complexity (since the bottleneck is on packing small items).

Intuitively, for any amount of capacity available to be filled by small items, and a minimum increase in profit, the optimal solution either packs a single item from $\Ical_S(t) \setminus \widetilde{\Ical}_S(t)$ in which case the loss by ignoring items in this set is bounded by the maximum reward of any small item, or the optimal solution only contains items from $\widetilde{\Ical}_S(t)$ in which case the space used by this optimal set of items is lower bounded by the a fractional packing of the highest density items in  $\widetilde{I}_S(t)$. During Algorithm~\ref{alg:FPTAS_nTlogn_small2}, one of the solutions we would consider would be the integral items of this fractional solution, and lose at most $\frac{1}{2T}\epsilon P_0$ in profit, and obtain a solution with still smaller space used (more leftover capacity) than the fractional solution. Accumulation of these errors for $t$ periods then will give us the invariant: the partial solution $\widetilde{A}_t(p)$ obtained as above has more leftover capacity than any solution obtained by selecting items from $\cup_{t'=1}^t \mathcal{I}_L(t')$ with rounded rewards and rounded penalties, and items from $\cup_{t'=1}^t \mathcal{I}_S(t')$ with original (unrounded) rewards such that the rounded total profit is at least $p+\frac{1}{2T}\epsilon P_0t+\kappa t$.


%Recall that we have assumed $R_0\le \Rcal(\Scal^*)\le 2R_0$. To find such an $R_0$, we would again enumerate $R_0$ from $\bar{R}/2, \bar{R}/4, \bar{R}/8,\ldots$, and one of them must satisfy~\eqref{R0}. This is done in Algorithm~\ref{alg:FPTAS_enumerate}. 
Our main theorem for the approximation ratio for MPBKP follows.

\begin{theorem}[Partially restating Theorem~\ref{mainthm2}]\label{main:MPBKP-S}
Algorithm~\ref{alg:SC_FPTAS} is a fully polynomial approximation scheme for the MPBKP-S, which achieves $(1+\epsilon)$ approximation ratio with running time $\mathcal{O}\left(\frac{Tn\log n}{\epsilon^2}\right)$.
\end{theorem}
\begin{comment}
\begin{remark}
Theorem~\ref{main:MPBKP-S}, together with Theorem~\ref{thm:FPTAS2}, implies that we can obtain a $(1-\epsilon)$ approximate solution for the MPBKP-S in $\mathcal{O}\left(\frac{n\log n}{\epsilon}\cdot\min\left\{\frac{T}{\epsilon},n\right\}\right)$, where Algorithm~\ref{alg:FPTAS2} is used when $T/\epsilon \gg n$ and Algorithm~\ref{alg:SC_FPTAS} is used when $T/\epsilon \ll n$.
\end{remark}


\begin{remark}
One may question if it is possible to achieve $\tilde{\Ocal}\left(n+T^\alpha/\epsilon^\beta\right)$ for some $\alpha,\beta$, as in the 0-1 Knapsack problem. We note that using a finer rounding technique as in~\cite{lawler1979fast}, the number of large items can be further bounded from $\Ocal(n)$ to $\Ocal\left(\frac{T}{\epsilon^2}\right)$, which would reduce the runtime of the DP (for large items) from $\Ocal\left(nT/\epsilon^2\right)$ to $\tilde{\Ocal}\left(n+T^2/\epsilon^4\right)$. However, the small items still have to be added one by one to the solutions of $\hat{A}_t(\bar{p})$, which in the worst case takes $\Ocal\left(\frac{nT}{\epsilon^2}\right)$. We cannot first group the small items into sets of partial solutions and do $(\max,+)$ convolution with solution sets of $\hat{A}_t(p)$, which would take $\Ocal\left(\frac{T^2}{\epsilon^4}\right)$ (similar to~\cite{ibarra1975fast,lawler1979fast}), because again the profits of two sets do not add up when we take the union of these two sets. Therefore, we do not further bound the number of large items as it does not improve the overall asymptotic time complexity (since the bottleneck is on packing small items).
\end{remark}
\end{comment}


\section{A greedy algorithm for a special case of MPBKP-SS}\label{sec:unit-MPBKP-SS}
In this subsection, we consider the special case of MPBKP-SS when all items have the same size, i.e., $q_i=q,\forall i\in[n]$. We again only present for the case $B_t = B,\forall t\in[T]$. We note that in the deterministic problems (MPBKP or MPBKP-S), when items all have the same size, greedily adding items one by one in decreasing order of their rewards leads to the optimal solution. For MPBKP-SS, as the capacities are now stochastic, we wonder if there is any greedy algorithm performs well. We propose Algorithm~\ref{alg:unitq-greedybyprofit}, where we start with an empty set, and greedily insert the item that brings the maximum increment on expected profit, and we stop if adding any of the remaining items does not increase the expected profit.

\begin{algorithm}[h]
	\footnotesize
	\caption{Greedy algorithm according to profit change}
	\label{alg:unitq-greedybyprofit}
	\algsetblock[Name]{Parameters}{}{0}{}
	\algsetblock[Name]{Initialize}{}{0}{}
	\algsetblock[Name]{Define}{}{0}{}
	\begin{algorithmic}[1]
		\State $\Scal\gets \emptyset$
		\State $s\gets 1$
		\While {$s == 1$}
		\State $i^*\gets \argmax_{i \notin\Scal}\left\{\Pcal(\Scal\cup\{i\})-\Pcal(\Scal)\right\}$
		\If {$\Pcal(\Scal\cup\{i^*\})-\Pcal(\Scal)\ge 0$}
		\State $\Scal\gets \Scal\cup\{i^*\}$
		\Else 
		\State $s\gets 0$
		\EndIf
		\EndWhile
		\State $\Scal_{p}\gets \Scal$
		\State {\bf Return} $\Scal_p$
	\end{algorithmic}
\end{algorithm}

Let $\Scal^*$ be an optimal solution, i.e.,
%\begin{align}\label{obj:unitsize}
$\Scal^* \in \arg\max_{\Scal\subseteq [n]} \Pcal(\Scal) := \Rcal(\Scal) - B\cdot \Phi(\Scal)$,
%\end{align}
where $$\Phi(\Scal) := \mathbb{E}\left\{\sum_{t=1}^T\left[\sum_{j\in \mathcal{I}(t)\cap\Scal}q_j-\max_{0\leq t' < t}\left\{ c_t - c_{t'}-\sum_{j  \in \Scal : t'+1 \leq d_j \leq t-1} q_j\right\}\right]^+ \right\}$$ is the expected quantity of overflow on set $\Scal$, and let $\Scal_p$ be the set output by Algorithm~\ref{alg:unitq-greedybyprofit}.
Then, we have the following theorem. 
\begin{theorem}[Restating Theorem~\ref{mainthm3}]\label{thm:GRprofit}
	Algorithm~\ref{alg:unitq-greedybyprofit} achieves $2$-approximation factor for MPBKP-SS when items have the same size, i.e., $\Pcal(\Scal_p) \ge \frac{1}{2}\Pcal(\Scal^*)$  in $\Ocal\left(n^2T|\Omega|\right)$.
\end{theorem}

The proof of the $2$-approximation could be more nontrivial than one may think. The idea is to look at the greedy solution set $\Scal_p$ and the optimal solution set $\Scal^*$,  where we will use the dual to characterize the optimal solution on each sample path. By swapping each item in $\Scal_p$ to $\Scal^*$ in replacement of the same item or two other items, we construct a sequence of partial solutions of the greedy algorithm as well as modified optimal solution set, while maintaining the invariant that the profit of $\Scal^*$ is bounded by the sum of two times the profit of items in $\Scal_p$ swapped into $\Scal^*$ so far and the additional profit of remaining items in the modified optimal solution set. We leave the formal proof of Theorem~\ref{thm:GRprofit} to Appendix~\ref{appc-unit}.


\section{Comments and Future Directions}\label{sec:conc}
The current work represents to the best of our knowledge the first FPTAS  for the two multi-period variants of the classical knapsack problem. For MPBKP, we obtained the runtime $\tilde{\Ocal}\left(n+(T^{3.25}/\epsilon^{2.25})\right)$. This was done via the function approximation approach, where a function is approximated for each period. The runtime increases in $T$ since we conduct $T$ number of rounding downs, one after each $(\max,+)$-convolution. An alternative algorithm with runtime $\tilde{\Ocal}\left(n+\frac{T^{2}}{\epsilon^{3}}\right)$ is also provided in Appendix~\ref{appT2}. Note that the function we approximated is in the same form as used in the 0-1 knapsack problem~\citep{chan:OASIcs:2018:8299}. It is thus interesting to ask if we could instead directly approximate the following function:
$$
f_{\Ical}(c) = \max_{x}\left\{\sum_{i\in\Ical}r_ix_i\ :\ \sum_{i\in\cup_{t'=1}^t\Ical(t')}q_ix_i\le c_t,\forall t\in[T],\ x\in\{0,1\}^n\right\},
$$
where $\Ical = \cup_{t=1}^T\Ical(t)$ and $c=\{c_1,\ldots,c_T\}$ is a $T$-dimensional vector. Here we impose all $T$ constraints in the function. The hope is that, if the above function could be approximated, and if we could properly define the $(\max,+)$-convolution on $T$ dimensional vectors (and have a fairly easy computation of it), then we may get an algorithm that depends more mildly on~$T$.

For MPBKP-S and MPBKP-SS, there seems to be less we can do without further assumptions. One direction to explore is  parameterized approximation schemes: assuming that in the optimal solution, the total (expected) penalty is at most~$\beta$ fraction of the total reward. Then we may just focus on rewards. Our ongoing work suggests that an approximation factor of $\left(1+\frac{\epsilon}{1-\beta}\right)$ may be achieved in $\tilde{\Ocal}\left(n+(T^{3.25}/\epsilon^{2.25})\right)$ for MPBKP-S, and the same approximation factor in $\tilde{\Ocal}\left(n+\frac{1}{\epsilon^{T}}\right)$ for MPBKP-SS. 

We further note that the objective function for the three multiperiod variants are in fact submodular (but not non-negative, or monotone). Whether we can get a constant competitive solution in time $\widetilde{\Ocal}(n)$, using approaches in submodular function maximization, is also an intriguing open problem. 

Finally, motivated by applications, one natural extension that the authors are working on now is when there is a general non-decreasing cost function $\phi_t(\Delta c)$ for procuring capacity $\Delta c$ at time $t$, and the goal is to admit a profit maximizing set of items when the unused capacity can be carried forward. Another extension is when there is a bound on the leftover capacity that can be carried forward.
	
 
	
	
	{\small
		\begin{spacing}{1.2}
			\bibliographystyle{apalike}
			\bibliography{references}
		\end{spacing}
	}
	
	
	\begin{appendix}
		\chapter{Supplementary Material}
\label{appendix}

In this appendix, we present supplementary material for the techniques and
experiments presented in the main text.

\section{Baseline Results and Analysis for Informed Sampler}
\label{appendix:chap3}

Here, we give an in-depth
performance analysis of the various samplers and the effect of their
hyperparameters. We choose hyperparameters with the lowest PSRF value
after $10k$ iterations, for each sampler individually. If the
differences between PSRF are not significantly different among
multiple values, we choose the one that has the highest acceptance
rate.

\subsection{Experiment: Estimating Camera Extrinsics}
\label{appendix:chap3:room}

\subsubsection{Parameter Selection}
\paragraph{Metropolis Hastings (\MH)}

Figure~\ref{fig:exp1_MH} shows the median acceptance rates and PSRF
values corresponding to various proposal standard deviations of plain
\MH~sampling. Mixing gets better and the acceptance rate gets worse as
the standard deviation increases. The value $0.3$ is selected standard
deviation for this sampler.

\paragraph{Metropolis Hastings Within Gibbs (\MHWG)}

As mentioned in Section~\ref{sec:room}, the \MHWG~sampler with one-dimensional
updates did not converge for any value of proposal standard deviation.
This problem has high correlation of the camera parameters and is of
multi-modal nature, which this sampler has problems with.

\paragraph{Parallel Tempering (\PT)}

For \PT~sampling, we took the best performing \MH~sampler and used
different temperature chains to improve the mixing of the
sampler. Figure~\ref{fig:exp1_PT} shows the results corresponding to
different combination of temperature levels. The sampler with
temperature levels of $[1,3,27]$ performed best in terms of both
mixing and acceptance rate.

\paragraph{Effect of Mixture Coefficient in Informed Sampling (\MIXLMH)}

Figure~\ref{fig:exp1_alpha} shows the effect of mixture
coefficient ($\alpha$) on the informed sampling
\MIXLMH. Since there is no significant different in PSRF values for
$0 \le \alpha \le 0.7$, we chose $0.7$ due to its high acceptance
rate.


% \end{multicols}

\begin{figure}[h]
\centering
  \subfigure[MH]{%
    \includegraphics[width=.48\textwidth]{figures/supplementary/camPose_MH.pdf} \label{fig:exp1_MH}
  }
  \subfigure[PT]{%
    \includegraphics[width=.48\textwidth]{figures/supplementary/camPose_PT.pdf} \label{fig:exp1_PT}
  }
\\
  \subfigure[INF-MH]{%
    \includegraphics[width=.48\textwidth]{figures/supplementary/camPose_alpha.pdf} \label{fig:exp1_alpha}
  }
  \mycaption{Results of the `Estimating Camera Extrinsics' experiment}{PRSFs and Acceptance rates corresponding to (a) various standard deviations of \MH, (b) various temperature level combinations of \PT sampling and (c) various mixture coefficients of \MIXLMH sampling.}
\end{figure}



\begin{figure}[!t]
\centering
  \subfigure[\MH]{%
    \includegraphics[width=.48\textwidth]{figures/supplementary/occlusionExp_MH.pdf} \label{fig:exp2_MH}
  }
  \subfigure[\BMHWG]{%
    \includegraphics[width=.48\textwidth]{figures/supplementary/occlusionExp_BMHWG.pdf} \label{fig:exp2_BMHWG}
  }
\\
  \subfigure[\MHWG]{%
    \includegraphics[width=.48\textwidth]{figures/supplementary/occlusionExp_MHWG.pdf} \label{fig:exp2_MHWG}
  }
  \subfigure[\PT]{%
    \includegraphics[width=.48\textwidth]{figures/supplementary/occlusionExp_PT.pdf} \label{fig:exp2_PT}
  }
\\
  \subfigure[\INFBMHWG]{%
    \includegraphics[width=.5\textwidth]{figures/supplementary/occlusionExp_alpha.pdf} \label{fig:exp2_alpha}
  }
  \mycaption{Results of the `Occluding Tiles' experiment}{PRSF and
    Acceptance rates corresponding to various standard deviations of
    (a) \MH, (b) \BMHWG, (c) \MHWG, (d) various temperature level
    combinations of \PT~sampling and; (e) various mixture coefficients
    of our informed \INFBMHWG sampling.}
\end{figure}

%\onecolumn\newpage\twocolumn
\subsection{Experiment: Occluding Tiles}
\label{appendix:chap3:tiles}

\subsubsection{Parameter Selection}

\paragraph{Metropolis Hastings (\MH)}

Figure~\ref{fig:exp2_MH} shows the results of
\MH~sampling. Results show the poor convergence for all proposal
standard deviations and rapid decrease of AR with increasing standard
deviation. This is due to the high-dimensional nature of
the problem. We selected a standard deviation of $1.1$.

\paragraph{Blocked Metropolis Hastings Within Gibbs (\BMHWG)}

The results of \BMHWG are shown in Figure~\ref{fig:exp2_BMHWG}. In
this sampler we update only one block of tile variables (of dimension
four) in each sampling step. Results show much better performance
compared to plain \MH. The optimal proposal standard deviation for
this sampler is $0.7$.

\paragraph{Metropolis Hastings Within Gibbs (\MHWG)}

Figure~\ref{fig:exp2_MHWG} shows the result of \MHWG sampling. This
sampler is better than \BMHWG and converges much more quickly. Here
a standard deviation of $0.9$ is found to be best.

\paragraph{Parallel Tempering (\PT)}

Figure~\ref{fig:exp2_PT} shows the results of \PT sampling with various
temperature combinations. Results show no improvement in AR from plain
\MH sampling and again $[1,3,27]$ temperature levels are found to be optimal.

\paragraph{Effect of Mixture Coefficient in Informed Sampling (\INFBMHWG)}

Figure~\ref{fig:exp2_alpha} shows the effect of mixture
coefficient ($\alpha$) on the blocked informed sampling
\INFBMHWG. Since there is no significant different in PSRF values for
$0 \le \alpha \le 0.8$, we chose $0.8$ due to its high acceptance
rate.



\subsection{Experiment: Estimating Body Shape}
\label{appendix:chap3:body}

\subsubsection{Parameter Selection}
\paragraph{Metropolis Hastings (\MH)}

Figure~\ref{fig:exp3_MH} shows the result of \MH~sampling with various
proposal standard deviations. The value of $0.1$ is found to be
best.

\paragraph{Metropolis Hastings Within Gibbs (\MHWG)}

For \MHWG sampling we select $0.3$ proposal standard
deviation. Results are shown in Fig.~\ref{fig:exp3_MHWG}.


\paragraph{Parallel Tempering (\PT)}

As before, results in Fig.~\ref{fig:exp3_PT}, the temperature levels
were selected to be $[1,3,27]$ due its slightly higher AR.

\paragraph{Effect of Mixture Coefficient in Informed Sampling (\MIXLMH)}

Figure~\ref{fig:exp3_alpha} shows the effect of $\alpha$ on PSRF and
AR. Since there is no significant differences in PSRF values for $0 \le
\alpha \le 0.8$, we choose $0.8$.


\begin{figure}[t]
\centering
  \subfigure[\MH]{%
    \includegraphics[width=.48\textwidth]{figures/supplementary/bodyShape_MH.pdf} \label{fig:exp3_MH}
  }
  \subfigure[\MHWG]{%
    \includegraphics[width=.48\textwidth]{figures/supplementary/bodyShape_MHWG.pdf} \label{fig:exp3_MHWG}
  }
\\
  \subfigure[\PT]{%
    \includegraphics[width=.48\textwidth]{figures/supplementary/bodyShape_PT.pdf} \label{fig:exp3_PT}
  }
  \subfigure[\MIXLMH]{%
    \includegraphics[width=.48\textwidth]{figures/supplementary/bodyShape_alpha.pdf} \label{fig:exp3_alpha}
  }
\\
  \mycaption{Results of the `Body Shape Estimation' experiment}{PRSFs and
    Acceptance rates corresponding to various standard deviations of
    (a) \MH, (b) \MHWG; (c) various temperature level combinations
    of \PT sampling and; (d) various mixture coefficients of the
    informed \MIXLMH sampling.}
\end{figure}


\subsection{Results Overview}
Figure~\ref{fig:exp_summary} shows the summary results of the all the three
experimental studies related to informed sampler.
\begin{figure*}[h!]
\centering
  \subfigure[Results for: Estimating Camera Extrinsics]{%
    \includegraphics[width=0.9\textwidth]{figures/supplementary/camPose_ALL.pdf} \label{fig:exp1_all}
  }
  \subfigure[Results for: Occluding Tiles]{%
    \includegraphics[width=0.9\textwidth]{figures/supplementary/occlusionExp_ALL.pdf} \label{fig:exp2_all}
  }
  \subfigure[Results for: Estimating Body Shape]{%
    \includegraphics[width=0.9\textwidth]{figures/supplementary/bodyShape_ALL.pdf} \label{fig:exp3_all}
  }
  \label{fig:exp_summary}
  \mycaption{Summary of the statistics for the three experiments}{Shown are
    for several baseline methods and the informed samplers the
    acceptance rates (left), PSRFs (middle), and RMSE values
    (right). All results are median results over multiple test
    examples.}
\end{figure*}

\subsection{Additional Qualitative Results}

\subsubsection{Occluding Tiles}
In Figure~\ref{fig:exp2_visual_more} more qualitative results of the
occluding tiles experiment are shown. The informed sampling approach
(\INFBMHWG) is better than the best baseline (\MHWG). This still is a
very challenging problem since the parameters for occluded tiles are
flat over a large region. Some of the posterior variance of the
occluded tiles is already captured by the informed sampler.

\begin{figure*}[h!]
\begin{center}
\centerline{\includegraphics[width=0.95\textwidth]{figures/supplementary/occlusionExp_Visual.pdf}}
\mycaption{Additional qualitative results of the occluding tiles experiment}
  {From left to right: (a)
  Given image, (b) Ground truth tiles, (c) OpenCV heuristic and most probable estimates
  from 5000 samples obtained by (d) MHWG sampler (best baseline) and
  (e) our INF-BMHWG sampler. (f) Posterior expectation of the tiles
  boundaries obtained by INF-BMHWG sampling (First 2000 samples are
  discarded as burn-in).}
\label{fig:exp2_visual_more}
\end{center}
\end{figure*}

\subsubsection{Body Shape}
Figure~\ref{fig:exp3_bodyMeshes} shows some more results of 3D mesh
reconstruction using posterior samples obtained by our informed
sampling \MIXLMH.

\begin{figure*}[t]
\begin{center}
\centerline{\includegraphics[width=0.75\textwidth]{figures/supplementary/bodyMeshResults.pdf}}
\mycaption{Qualitative results for the body shape experiment}
  {Shown is the 3D mesh reconstruction results with first 1000 samples obtained
  using the \MIXLMH informed sampling method. (blue indicates small
  values and red indicates high values)}
\label{fig:exp3_bodyMeshes}
\end{center}
\end{figure*}

\clearpage



\section{Additional Results on the Face Problem with CMP}

Figure~\ref{fig:shading-qualitative-multiple-subjects-supp} shows inference results for reflectance maps, normal maps and lights for randomly chosen test images, and Fig.~\ref{fig:shading-qualitative-same-subject-supp} shows reflectance estimation results on multiple images of the same subject produced under different illumination conditions. CMP is able to produce estimates that are closer to the groundtruth across different subjects and illumination conditions.

\begin{figure*}[h]
  \begin{center}
  \centerline{\includegraphics[width=1.0\columnwidth]{figures/face_cmp_visual_results_supp.pdf}}
  \vspace{-1.2cm}
  \end{center}
	\mycaption{A visual comparison of inference results}{(a)~Observed images. (b)~Inferred reflectance maps. \textit{GT} is the photometric stereo groundtruth, \textit{BU} is the Biswas \etal (2009) reflectance estimate and \textit{Forest} is the consensus prediction. (c)~The variance of the inferred reflectance estimate produced by \MTD (normalized across rows).(d)~Visualization of inferred light directions. (e)~Inferred normal maps.}
	\label{fig:shading-qualitative-multiple-subjects-supp}
\end{figure*}


\begin{figure*}[h]
	\centering
	\setlength\fboxsep{0.2mm}
	\setlength\fboxrule{0pt}
	\begin{tikzpicture}

		\matrix at (0, 0) [matrix of nodes, nodes={anchor=east}, column sep=-0.05cm, row sep=-0.2cm]
		{
			\fbox{\includegraphics[width=1cm]{figures/sample_3_4_X.png}} &
			\fbox{\includegraphics[width=1cm]{figures/sample_3_4_GT.png}} &
			\fbox{\includegraphics[width=1cm]{figures/sample_3_4_BISWAS.png}}  &
			\fbox{\includegraphics[width=1cm]{figures/sample_3_4_VMP.png}}  &
			\fbox{\includegraphics[width=1cm]{figures/sample_3_4_FOREST.png}}  &
			\fbox{\includegraphics[width=1cm]{figures/sample_3_4_CMP.png}}  &
			\fbox{\includegraphics[width=1cm]{figures/sample_3_4_CMPVAR.png}}
			 \\

			\fbox{\includegraphics[width=1cm]{figures/sample_3_5_X.png}} &
			\fbox{\includegraphics[width=1cm]{figures/sample_3_5_GT.png}} &
			\fbox{\includegraphics[width=1cm]{figures/sample_3_5_BISWAS.png}}  &
			\fbox{\includegraphics[width=1cm]{figures/sample_3_5_VMP.png}}  &
			\fbox{\includegraphics[width=1cm]{figures/sample_3_5_FOREST.png}}  &
			\fbox{\includegraphics[width=1cm]{figures/sample_3_5_CMP.png}}  &
			\fbox{\includegraphics[width=1cm]{figures/sample_3_5_CMPVAR.png}}
			 \\

			\fbox{\includegraphics[width=1cm]{figures/sample_3_6_X.png}} &
			\fbox{\includegraphics[width=1cm]{figures/sample_3_6_GT.png}} &
			\fbox{\includegraphics[width=1cm]{figures/sample_3_6_BISWAS.png}}  &
			\fbox{\includegraphics[width=1cm]{figures/sample_3_6_VMP.png}}  &
			\fbox{\includegraphics[width=1cm]{figures/sample_3_6_FOREST.png}}  &
			\fbox{\includegraphics[width=1cm]{figures/sample_3_6_CMP.png}}  &
			\fbox{\includegraphics[width=1cm]{figures/sample_3_6_CMPVAR.png}}
			 \\
	     };

       \node at (-3.85, -2.0) {\small Observed};
       \node at (-2.55, -2.0) {\small `GT'};
       \node at (-1.27, -2.0) {\small BU};
       \node at (0.0, -2.0) {\small MP};
       \node at (1.27, -2.0) {\small Forest};
       \node at (2.55, -2.0) {\small \textbf{CMP}};
       \node at (3.85, -2.0) {\small Variance};

	\end{tikzpicture}
	\mycaption{Robustness to varying illumination}{Reflectance estimation on a subject images with varying illumination. Left to right: observed image, photometric stereo estimate (GT)
  which is used as a proxy for groundtruth, bottom-up estimate of \cite{Biswas2009}, VMP result, consensus forest estimate, CMP mean, and CMP variance.}
	\label{fig:shading-qualitative-same-subject-supp}
\end{figure*}

\clearpage

\section{Additional Material for Learning Sparse High Dimensional Filters}
\label{sec:appendix-bnn}

This part of supplementary material contains a more detailed overview of the permutohedral
lattice convolution in Section~\ref{sec:permconv}, more experiments in
Section~\ref{sec:addexps} and additional results with protocols for
the experiments presented in Chapter~\ref{chap:bnn} in Section~\ref{sec:addresults}.

\vspace{-0.2cm}
\subsection{General Permutohedral Convolutions}
\label{sec:permconv}

A core technical contribution of this work is the generalization of the Gaussian permutohedral lattice
convolution proposed in~\cite{adams2010fast} to the full non-separable case with the
ability to perform back-propagation. Although, conceptually, there are minor
differences between Gaussian and general parameterized filters, there are non-trivial practical
differences in terms of the algorithmic implementation. The Gauss filters belong to
the separable class and can thus be decomposed into multiple
sequential one dimensional convolutions. We are interested in the general filter
convolutions, which can not be decomposed. Thus, performing a general permutohedral
convolution at a lattice point requires the computation of the inner product with the
neighboring elements in all the directions in the high-dimensional space.

Here, we give more details of the implementation differences of separable
and non-separable filters. In the following, we will explain the scalar case first.
Recall, that the forward pass of general permutohedral convolution
involves 3 steps: \textit{splatting}, \textit{convolving} and \textit{slicing}.
We follow the same splatting and slicing strategies as in~\cite{adams2010fast}
since these operations do not depend on the filter kernel. The main difference
between our work and the existing implementation of~\cite{adams2010fast} is
the way that the convolution operation is executed. This proceeds by constructing
a \emph{blur neighbor} matrix $K$ that stores for every lattice point all
values of the lattice neighbors that are needed to compute the filter output.

\begin{figure}[t!]
  \centering
    \includegraphics[width=0.6\columnwidth]{figures/supplementary/lattice_construction}
  \mycaption{Illustration of 1D permutohedral lattice construction}
  {A $4\times 4$ $(x,y)$ grid lattice is projected onto the plane defined by the normal
  vector $(1,1)^{\top}$. This grid has $s+1=4$ and $d=2$ $(s+1)^{d}=4^2=16$ elements.
  In the projection, all points of the same color are projected onto the same points in the plane.
  The number of elements of the projected lattice is $t=(s+1)^d-s^d=4^2-3^2=7$, that is
  the $(4\times 4)$ grid minus the size of lattice that is $1$ smaller at each size, in this
  case a $(3\times 3)$ lattice (the upper right $(3\times 3)$ elements).
  }
\label{fig:latticeconstruction}
\end{figure}

The blur neighbor matrix is constructed by traversing through all the populated
lattice points and their neighboring elements.
% For efficiency, we do this matrix construction recursively with shared computations
% since $n^{th}$ neighbourhood elements are $1^{st}$ neighborhood elements of $n-1^{th}$ neighbourhood elements. \pg{do not understand}
This is done recursively to share computations. For any lattice point, the neighbors that are
$n$ hops away are the direct neighbors of the points that are $n-1$ hops away.
The size of a $d$ dimensional spatial filter with width $s+1$ is $(s+1)^{d}$ (\eg, a
$3\times 3$ filter, $s=2$ in $d=2$ has $3^2=9$ elements) and this size grows
exponentially in the number of dimensions $d$. The permutohedral lattice is constructed by
projecting a regular grid onto the plane spanned by the $d$ dimensional normal vector ${(1,\ldots,1)}^{\top}$. See
Fig.~\ref{fig:latticeconstruction} for an illustration of the 1D lattice construction.
Many corners of a grid filter are projected onto the same point, in total $t = {(s+1)}^{d} -
s^{d}$ elements remain in the permutohedral filter with $s$ neighborhood in $d-1$ dimensions.
If the lattice has $m$ populated elements, the
matrix $K$ has size $t\times m$. Note that, since the input signal is typically
sparse, only a few lattice corners are being populated in the \textit{slicing} step.
We use a hash-table to keep track of these points and traverse only through
the populated lattice points for this neighborhood matrix construction.

Once the blur neighbor matrix $K$ is constructed, we can perform the convolution
by the matrix vector multiplication
\begin{equation}
\ell' = BK,
\label{eq:conv}
\end{equation}
where $B$ is the $1 \times t$ filter kernel (whose values we will learn) and $\ell'\in\mathbb{R}^{1\times m}$
is the result of the filtering at the $m$ lattice points. In practice, we found that the
matrix $K$ is sometimes too large to fit into GPU memory and we divided the matrix $K$
into smaller pieces to compute Eq.~\ref{eq:conv} sequentially.

In the general multi-dimensional case, the signal $\ell$ is of $c$ dimensions. Then
the kernel $B$ is of size $c \times t$ and $K$ stores the $c$ dimensional vectors
accordingly. When the input and output points are different, we slice only the
input points and splat only at the output points.


\subsection{Additional Experiments}
\label{sec:addexps}
In this section, we discuss more use-cases for the learned bilateral filters, one
use-case of BNNs and two single filter applications for image and 3D mesh denoising.

\subsubsection{Recognition of subsampled MNIST}\label{sec:app_mnist}

One of the strengths of the proposed filter convolution is that it does not
require the input to lie on a regular grid. The only requirement is to define a distance
between features of the input signal.
We highlight this feature with the following experiment using the
classical MNIST ten class classification problem~\cite{lecun1998mnist}. We sample a
sparse set of $N$ points $(x,y)\in [0,1]\times [0,1]$
uniformly at random in the input image, use their interpolated values
as signal and the \emph{continuous} $(x,y)$ positions as features. This mimics
sub-sampling of a high-dimensional signal. To compare against a spatial convolution,
we interpolate the sparse set of values at the grid positions.

We take a reference implementation of LeNet~\cite{lecun1998gradient} that
is part of the Caffe project~\cite{jia2014caffe} and compare it
against the same architecture but replacing the first convolutional
layer with a bilateral convolution layer (BCL). The filter size
and numbers are adjusted to get a comparable number of parameters
($5\times 5$ for LeNet, $2$-neighborhood for BCL).

The results are shown in Table~\ref{tab:all-results}. We see that training
on the original MNIST data (column Original, LeNet vs. BNN) leads to a slight
decrease in performance of the BNN (99.03\%) compared to LeNet
(99.19\%). The BNN can be trained and evaluated on sparse
signals, and we resample the image as described above for $N=$ 100\%, 60\% and
20\% of the total number of pixels. The methods are also evaluated
on test images that are subsampled in the same way. Note that we can
train and test with different subsampling rates. We introduce an additional
bilinear interpolation layer for the LeNet architecture to train on the same
data. In essence, both models perform a spatial interpolation and thus we
expect them to yield a similar classification accuracy. Once the data is of
higher dimensions, the permutohedral convolution will be faster due to hashing
the sparse input points, as well as less memory demanding in comparison to
naive application of a spatial convolution with interpolated values.

\begin{table}[t]
  \begin{center}
    \footnotesize
    \centering
    \begin{tabular}[t]{lllll}
      \toprule
              &     & \multicolumn{3}{c}{Test Subsampling} \\
       Method  & Original & 100\% & 60\% & 20\%\\
      \midrule
       LeNet &  \textbf{0.9919} & 0.9660 & 0.9348 & \textbf{0.6434} \\
       BNN &  0.9903 & \textbf{0.9844} & \textbf{0.9534} & 0.5767 \\
      \hline
       LeNet 100\% & 0.9856 & 0.9809 & 0.9678 & \textbf{0.7386} \\
       BNN 100\% & \textbf{0.9900} & \textbf{0.9863} & \textbf{0.9699} & 0.6910 \\
      \hline
       LeNet 60\% & 0.9848 & 0.9821 & 0.9740 & 0.8151 \\
       BNN 60\% & \textbf{0.9885} & \textbf{0.9864} & \textbf{0.9771} & \textbf{0.8214}\\
      \hline
       LeNet 20\% & \textbf{0.9763} & \textbf{0.9754} & 0.9695 & 0.8928 \\
       BNN 20\% & 0.9728 & 0.9735 & \textbf{0.9701} & \textbf{0.9042}\\
      \bottomrule
    \end{tabular}
  \end{center}
\vspace{-.2cm}
\caption{Classification accuracy on MNIST. We compare the
    LeNet~\cite{lecun1998gradient} implementation that is part of
    Caffe~\cite{jia2014caffe} to the network with the first layer
    replaced by a bilateral convolution layer (BCL). Both are trained
    on the original image resolution (first two rows). Three more BNN
    and CNN models are trained with randomly subsampled images (100\%,
    60\% and 20\% of the pixels). An additional bilinear interpolation
    layer samples the input signal on a spatial grid for the CNN model.
  }
  \label{tab:all-results}
\vspace{-.5cm}
\end{table}

\subsubsection{Image Denoising}

The main application that inspired the development of the bilateral
filtering operation is image denoising~\cite{aurich1995non}, there
using a single Gaussian kernel. Our development allows to learn this
kernel function from data and we explore how to improve using a \emph{single}
but more general bilateral filter.

We use the Berkeley segmentation dataset
(BSDS500)~\cite{arbelaezi2011bsds500} as a test bed. The color
images in the dataset are converted to gray-scale,
and corrupted with Gaussian noise with a standard deviation of
$\frac {25} {255}$.

We compare the performance of four different filter models on a
denoising task.
The first baseline model (`Spatial' in Table \ref{tab:denoising}, $25$
weights) uses a single spatial filter with a kernel size of
$5$ and predicts the scalar gray-scale value at the center pixel. The next model
(`Gauss Bilateral') applies a bilateral \emph{Gaussian}
filter to the noisy input, using position and intensity features $\f=(x,y,v)^\top$.
The third setup (`Learned Bilateral', $65$ weights)
takes a Gauss kernel as initialization and
fits all filter weights on the train set to minimize the
mean squared error with respect to the clean images.
We run a combination
of spatial and permutohedral convolutions on spatial and bilateral
features (`Spatial + Bilateral (Learned)') to check for a complementary
performance of the two convolutions.

\label{sec:exp:denoising}
\begin{table}[!h]
\begin{center}
  \footnotesize
  \begin{tabular}[t]{lr}
    \toprule
    Method & PSNR \\
    \midrule
    Noisy Input & $20.17$ \\
    Spatial & $26.27$ \\
    Gauss Bilateral & $26.51$ \\
    Learned Bilateral & $26.58$ \\
    Spatial + Bilateral (Learned) & \textbf{$26.65$} \\
    \bottomrule
  \end{tabular}
\end{center}
\vspace{-0.5em}
\caption{PSNR results of a denoising task using the BSDS500
  dataset~\cite{arbelaezi2011bsds500}}
\vspace{-0.5em}
\label{tab:denoising}
\end{table}
\vspace{-0.2em}

The PSNR scores evaluated on full images of the test set are
shown in Table \ref{tab:denoising}. We find that an untrained bilateral
filter already performs better than a trained spatial convolution
($26.27$ to $26.51$). A learned convolution further improve the
performance slightly. We chose this simple one-kernel setup to
validate an advantage of the generalized bilateral filter. A competitive
denoising system would employ RGB color information and also
needs to be properly adjusted in network size. Multi-layer perceptrons
have obtained state-of-the-art denoising results~\cite{burger12cvpr}
and the permutohedral lattice layer can readily be used in such an
architecture, which is intended future work.

\subsection{Additional results}
\label{sec:addresults}

This section contains more qualitative results for the experiments presented in Chapter~\ref{chap:bnn}.

\begin{figure*}[th!]
  \centering
    \includegraphics[width=\columnwidth,trim={5cm 2.5cm 5cm 4.5cm},clip]{figures/supplementary/lattice_viz.pdf}
    \vspace{-0.7cm}
  \mycaption{Visualization of the Permutohedral Lattice}
  {Sample lattice visualizations for different feature spaces. All pixels falling in the same simplex cell are shown with
  the same color. $(x,y)$ features correspond to image pixel positions, and $(r,g,b) \in [0,255]$ correspond
  to the red, green and blue color values.}
\label{fig:latticeviz}
\end{figure*}

\subsubsection{Lattice Visualization}

Figure~\ref{fig:latticeviz} shows sample lattice visualizations for different feature spaces.

\newcolumntype{L}[1]{>{\raggedright\let\newline\\\arraybackslash\hspace{0pt}}b{#1}}
\newcolumntype{C}[1]{>{\centering\let\newline\\\arraybackslash\hspace{0pt}}b{#1}}
\newcolumntype{R}[1]{>{\raggedleft\let\newline\\\arraybackslash\hspace{0pt}}b{#1}}

\subsubsection{Color Upsampling}\label{sec:color_upsampling}
\label{sec:col_upsample_extra}

Some images of the upsampling for the Pascal
VOC12 dataset are shown in Fig.~\ref{fig:Colour_upsample_visuals}. It is
especially the low level image details that are better preserved with
a learned bilateral filter compared to the Gaussian case.

\begin{figure*}[t!]
  \centering
    \subfigure{%
   \raisebox{2.0em}{
    \includegraphics[width=.06\columnwidth]{figures/supplementary/2007_004969.jpg}
   }
  }
  \subfigure{%
    \includegraphics[width=.17\columnwidth]{figures/supplementary/2007_004969_gray.pdf}
  }
  \subfigure{%
    \includegraphics[width=.17\columnwidth]{figures/supplementary/2007_004969_gt.pdf}
  }
  \subfigure{%
    \includegraphics[width=.17\columnwidth]{figures/supplementary/2007_004969_bicubic.pdf}
  }
  \subfigure{%
    \includegraphics[width=.17\columnwidth]{figures/supplementary/2007_004969_gauss.pdf}
  }
  \subfigure{%
    \includegraphics[width=.17\columnwidth]{figures/supplementary/2007_004969_learnt.pdf}
  }\\
    \subfigure{%
   \raisebox{2.0em}{
    \includegraphics[width=.06\columnwidth]{figures/supplementary/2007_003106.jpg}
   }
  }
  \subfigure{%
    \includegraphics[width=.17\columnwidth]{figures/supplementary/2007_003106_gray.pdf}
  }
  \subfigure{%
    \includegraphics[width=.17\columnwidth]{figures/supplementary/2007_003106_gt.pdf}
  }
  \subfigure{%
    \includegraphics[width=.17\columnwidth]{figures/supplementary/2007_003106_bicubic.pdf}
  }
  \subfigure{%
    \includegraphics[width=.17\columnwidth]{figures/supplementary/2007_003106_gauss.pdf}
  }
  \subfigure{%
    \includegraphics[width=.17\columnwidth]{figures/supplementary/2007_003106_learnt.pdf}
  }\\
  \setcounter{subfigure}{0}
  \small{
  \subfigure[Inp.]{%
  \raisebox{2.0em}{
    \includegraphics[width=.06\columnwidth]{figures/supplementary/2007_006837.jpg}
   }
  }
  \subfigure[Guidance]{%
    \includegraphics[width=.17\columnwidth]{figures/supplementary/2007_006837_gray.pdf}
  }
   \subfigure[GT]{%
    \includegraphics[width=.17\columnwidth]{figures/supplementary/2007_006837_gt.pdf}
  }
  \subfigure[Bicubic]{%
    \includegraphics[width=.17\columnwidth]{figures/supplementary/2007_006837_bicubic.pdf}
  }
  \subfigure[Gauss-BF]{%
    \includegraphics[width=.17\columnwidth]{figures/supplementary/2007_006837_gauss.pdf}
  }
  \subfigure[Learned-BF]{%
    \includegraphics[width=.17\columnwidth]{figures/supplementary/2007_006837_learnt.pdf}
  }
  }
  \vspace{-0.5cm}
  \mycaption{Color Upsampling}{Color $8\times$ upsampling results
  using different methods, from left to right, (a)~Low-resolution input color image (Inp.),
  (b)~Gray scale guidance image, (c)~Ground-truth color image; Upsampled color images with
  (d)~Bicubic interpolation, (e) Gauss bilateral upsampling and, (f)~Learned bilateral
  updampgling (best viewed on screen).}

\label{fig:Colour_upsample_visuals}
\end{figure*}

\subsubsection{Depth Upsampling}
\label{sec:depth_upsample_extra}

Figure~\ref{fig:depth_upsample_visuals} presents some more qualitative results comparing bicubic interpolation, Gauss
bilateral and learned bilateral upsampling on NYU depth dataset image~\cite{silberman2012indoor}.

\subsubsection{Character Recognition}\label{sec:app_character}

 Figure~\ref{fig:nnrecognition} shows the schematic of different layers
 of the network architecture for LeNet-7~\cite{lecun1998mnist}
 and DeepCNet(5, 50)~\cite{ciresan2012multi,graham2014spatially}. For the BNN variants, the first layer filters are replaced
 with learned bilateral filters and are learned end-to-end.

\subsubsection{Semantic Segmentation}\label{sec:app_semantic_segmentation}
\label{sec:semantic_bnn_extra}

Some more visual results for semantic segmentation are shown in Figure~\ref{fig:semantic_visuals}.
These include the underlying DeepLab CNN\cite{chen2014semantic} result (DeepLab),
the 2 step mean-field result with Gaussian edge potentials (+2stepMF-GaussCRF)
and also corresponding results with learned edge potentials (+2stepMF-LearnedCRF).
In general, we observe that mean-field in learned CRF leads to slightly dilated
classification regions in comparison to using Gaussian CRF thereby filling-in the
false negative pixels and also correcting some mis-classified regions.

\begin{figure*}[t!]
  \centering
    \subfigure{%
   \raisebox{2.0em}{
    \includegraphics[width=.06\columnwidth]{figures/supplementary/2bicubic}
   }
  }
  \subfigure{%
    \includegraphics[width=.17\columnwidth]{figures/supplementary/2given_image}
  }
  \subfigure{%
    \includegraphics[width=.17\columnwidth]{figures/supplementary/2ground_truth}
  }
  \subfigure{%
    \includegraphics[width=.17\columnwidth]{figures/supplementary/2bicubic}
  }
  \subfigure{%
    \includegraphics[width=.17\columnwidth]{figures/supplementary/2gauss}
  }
  \subfigure{%
    \includegraphics[width=.17\columnwidth]{figures/supplementary/2learnt}
  }\\
    \subfigure{%
   \raisebox{2.0em}{
    \includegraphics[width=.06\columnwidth]{figures/supplementary/32bicubic}
   }
  }
  \subfigure{%
    \includegraphics[width=.17\columnwidth]{figures/supplementary/32given_image}
  }
  \subfigure{%
    \includegraphics[width=.17\columnwidth]{figures/supplementary/32ground_truth}
  }
  \subfigure{%
    \includegraphics[width=.17\columnwidth]{figures/supplementary/32bicubic}
  }
  \subfigure{%
    \includegraphics[width=.17\columnwidth]{figures/supplementary/32gauss}
  }
  \subfigure{%
    \includegraphics[width=.17\columnwidth]{figures/supplementary/32learnt}
  }\\
  \setcounter{subfigure}{0}
  \small{
  \subfigure[Inp.]{%
  \raisebox{2.0em}{
    \includegraphics[width=.06\columnwidth]{figures/supplementary/41bicubic}
   }
  }
  \subfigure[Guidance]{%
    \includegraphics[width=.17\columnwidth]{figures/supplementary/41given_image}
  }
   \subfigure[GT]{%
    \includegraphics[width=.17\columnwidth]{figures/supplementary/41ground_truth}
  }
  \subfigure[Bicubic]{%
    \includegraphics[width=.17\columnwidth]{figures/supplementary/41bicubic}
  }
  \subfigure[Gauss-BF]{%
    \includegraphics[width=.17\columnwidth]{figures/supplementary/41gauss}
  }
  \subfigure[Learned-BF]{%
    \includegraphics[width=.17\columnwidth]{figures/supplementary/41learnt}
  }
  }
  \mycaption{Depth Upsampling}{Depth $8\times$ upsampling results
  using different upsampling strategies, from left to right,
  (a)~Low-resolution input depth image (Inp.),
  (b)~High-resolution guidance image, (c)~Ground-truth depth; Upsampled depth images with
  (d)~Bicubic interpolation, (e) Gauss bilateral upsampling and, (f)~Learned bilateral
  updampgling (best viewed on screen).}

\label{fig:depth_upsample_visuals}
\end{figure*}

\subsubsection{Material Segmentation}\label{sec:app_material_segmentation}
\label{sec:material_bnn_extra}

In Fig.~\ref{fig:material_visuals-app2}, we present visual results comparing 2 step
mean-field inference with Gaussian and learned pairwise CRF potentials. In
general, we observe that the pixels belonging to dominant classes in the
training data are being more accurately classified with learned CRF. This leads to
a significant improvements in overall pixel accuracy. This also results
in a slight decrease of the accuracy from less frequent class pixels thereby
slightly reducing the average class accuracy with learning. We attribute this
to the type of annotation that is available for this dataset, which is not
for the entire image but for some segments in the image. We have very few
images of the infrequent classes to combat this behaviour during training.

\subsubsection{Experiment Protocols}
\label{sec:protocols}

Table~\ref{tbl:parameters} shows experiment protocols of different experiments.

 \begin{figure*}[t!]
  \centering
  \subfigure[LeNet-7]{
    \includegraphics[width=0.7\columnwidth]{figures/supplementary/lenet_cnn_network}
    }\\
    \subfigure[DeepCNet]{
    \includegraphics[width=\columnwidth]{figures/supplementary/deepcnet_cnn_network}
    }
  \mycaption{CNNs for Character Recognition}
  {Schematic of (top) LeNet-7~\cite{lecun1998mnist} and (bottom) DeepCNet(5,50)~\cite{ciresan2012multi,graham2014spatially} architectures used in Assamese
  character recognition experiments.}
\label{fig:nnrecognition}
\end{figure*}

\definecolor{voc_1}{RGB}{0, 0, 0}
\definecolor{voc_2}{RGB}{128, 0, 0}
\definecolor{voc_3}{RGB}{0, 128, 0}
\definecolor{voc_4}{RGB}{128, 128, 0}
\definecolor{voc_5}{RGB}{0, 0, 128}
\definecolor{voc_6}{RGB}{128, 0, 128}
\definecolor{voc_7}{RGB}{0, 128, 128}
\definecolor{voc_8}{RGB}{128, 128, 128}
\definecolor{voc_9}{RGB}{64, 0, 0}
\definecolor{voc_10}{RGB}{192, 0, 0}
\definecolor{voc_11}{RGB}{64, 128, 0}
\definecolor{voc_12}{RGB}{192, 128, 0}
\definecolor{voc_13}{RGB}{64, 0, 128}
\definecolor{voc_14}{RGB}{192, 0, 128}
\definecolor{voc_15}{RGB}{64, 128, 128}
\definecolor{voc_16}{RGB}{192, 128, 128}
\definecolor{voc_17}{RGB}{0, 64, 0}
\definecolor{voc_18}{RGB}{128, 64, 0}
\definecolor{voc_19}{RGB}{0, 192, 0}
\definecolor{voc_20}{RGB}{128, 192, 0}
\definecolor{voc_21}{RGB}{0, 64, 128}
\definecolor{voc_22}{RGB}{128, 64, 128}

\begin{figure*}[t]
  \centering
  \small{
  \fcolorbox{white}{voc_1}{\rule{0pt}{6pt}\rule{6pt}{0pt}} Background~~
  \fcolorbox{white}{voc_2}{\rule{0pt}{6pt}\rule{6pt}{0pt}} Aeroplane~~
  \fcolorbox{white}{voc_3}{\rule{0pt}{6pt}\rule{6pt}{0pt}} Bicycle~~
  \fcolorbox{white}{voc_4}{\rule{0pt}{6pt}\rule{6pt}{0pt}} Bird~~
  \fcolorbox{white}{voc_5}{\rule{0pt}{6pt}\rule{6pt}{0pt}} Boat~~
  \fcolorbox{white}{voc_6}{\rule{0pt}{6pt}\rule{6pt}{0pt}} Bottle~~
  \fcolorbox{white}{voc_7}{\rule{0pt}{6pt}\rule{6pt}{0pt}} Bus~~
  \fcolorbox{white}{voc_8}{\rule{0pt}{6pt}\rule{6pt}{0pt}} Car~~ \\
  \fcolorbox{white}{voc_9}{\rule{0pt}{6pt}\rule{6pt}{0pt}} Cat~~
  \fcolorbox{white}{voc_10}{\rule{0pt}{6pt}\rule{6pt}{0pt}} Chair~~
  \fcolorbox{white}{voc_11}{\rule{0pt}{6pt}\rule{6pt}{0pt}} Cow~~
  \fcolorbox{white}{voc_12}{\rule{0pt}{6pt}\rule{6pt}{0pt}} Dining Table~~
  \fcolorbox{white}{voc_13}{\rule{0pt}{6pt}\rule{6pt}{0pt}} Dog~~
  \fcolorbox{white}{voc_14}{\rule{0pt}{6pt}\rule{6pt}{0pt}} Horse~~
  \fcolorbox{white}{voc_15}{\rule{0pt}{6pt}\rule{6pt}{0pt}} Motorbike~~
  \fcolorbox{white}{voc_16}{\rule{0pt}{6pt}\rule{6pt}{0pt}} Person~~ \\
  \fcolorbox{white}{voc_17}{\rule{0pt}{6pt}\rule{6pt}{0pt}} Potted Plant~~
  \fcolorbox{white}{voc_18}{\rule{0pt}{6pt}\rule{6pt}{0pt}} Sheep~~
  \fcolorbox{white}{voc_19}{\rule{0pt}{6pt}\rule{6pt}{0pt}} Sofa~~
  \fcolorbox{white}{voc_20}{\rule{0pt}{6pt}\rule{6pt}{0pt}} Train~~
  \fcolorbox{white}{voc_21}{\rule{0pt}{6pt}\rule{6pt}{0pt}} TV monitor~~ \\
  }
  \subfigure{%
    \includegraphics[width=.18\columnwidth]{figures/supplementary/2007_001423_given.jpg}
  }
  \subfigure{%
    \includegraphics[width=.18\columnwidth]{figures/supplementary/2007_001423_gt.png}
  }
  \subfigure{%
    \includegraphics[width=.18\columnwidth]{figures/supplementary/2007_001423_cnn.png}
  }
  \subfigure{%
    \includegraphics[width=.18\columnwidth]{figures/supplementary/2007_001423_gauss.png}
  }
  \subfigure{%
    \includegraphics[width=.18\columnwidth]{figures/supplementary/2007_001423_learnt.png}
  }\\
  \subfigure{%
    \includegraphics[width=.18\columnwidth]{figures/supplementary/2007_001430_given.jpg}
  }
  \subfigure{%
    \includegraphics[width=.18\columnwidth]{figures/supplementary/2007_001430_gt.png}
  }
  \subfigure{%
    \includegraphics[width=.18\columnwidth]{figures/supplementary/2007_001430_cnn.png}
  }
  \subfigure{%
    \includegraphics[width=.18\columnwidth]{figures/supplementary/2007_001430_gauss.png}
  }
  \subfigure{%
    \includegraphics[width=.18\columnwidth]{figures/supplementary/2007_001430_learnt.png}
  }\\
    \subfigure{%
    \includegraphics[width=.18\columnwidth]{figures/supplementary/2007_007996_given.jpg}
  }
  \subfigure{%
    \includegraphics[width=.18\columnwidth]{figures/supplementary/2007_007996_gt.png}
  }
  \subfigure{%
    \includegraphics[width=.18\columnwidth]{figures/supplementary/2007_007996_cnn.png}
  }
  \subfigure{%
    \includegraphics[width=.18\columnwidth]{figures/supplementary/2007_007996_gauss.png}
  }
  \subfigure{%
    \includegraphics[width=.18\columnwidth]{figures/supplementary/2007_007996_learnt.png}
  }\\
   \subfigure{%
    \includegraphics[width=.18\columnwidth]{figures/supplementary/2010_002682_given.jpg}
  }
  \subfigure{%
    \includegraphics[width=.18\columnwidth]{figures/supplementary/2010_002682_gt.png}
  }
  \subfigure{%
    \includegraphics[width=.18\columnwidth]{figures/supplementary/2010_002682_cnn.png}
  }
  \subfigure{%
    \includegraphics[width=.18\columnwidth]{figures/supplementary/2010_002682_gauss.png}
  }
  \subfigure{%
    \includegraphics[width=.18\columnwidth]{figures/supplementary/2010_002682_learnt.png}
  }\\
     \subfigure{%
    \includegraphics[width=.18\columnwidth]{figures/supplementary/2010_004789_given.jpg}
  }
  \subfigure{%
    \includegraphics[width=.18\columnwidth]{figures/supplementary/2010_004789_gt.png}
  }
  \subfigure{%
    \includegraphics[width=.18\columnwidth]{figures/supplementary/2010_004789_cnn.png}
  }
  \subfigure{%
    \includegraphics[width=.18\columnwidth]{figures/supplementary/2010_004789_gauss.png}
  }
  \subfigure{%
    \includegraphics[width=.18\columnwidth]{figures/supplementary/2010_004789_learnt.png}
  }\\
       \subfigure{%
    \includegraphics[width=.18\columnwidth]{figures/supplementary/2007_001311_given.jpg}
  }
  \subfigure{%
    \includegraphics[width=.18\columnwidth]{figures/supplementary/2007_001311_gt.png}
  }
  \subfigure{%
    \includegraphics[width=.18\columnwidth]{figures/supplementary/2007_001311_cnn.png}
  }
  \subfigure{%
    \includegraphics[width=.18\columnwidth]{figures/supplementary/2007_001311_gauss.png}
  }
  \subfigure{%
    \includegraphics[width=.18\columnwidth]{figures/supplementary/2007_001311_learnt.png}
  }\\
  \setcounter{subfigure}{0}
  \subfigure[Input]{%
    \includegraphics[width=.18\columnwidth]{figures/supplementary/2010_003531_given.jpg}
  }
  \subfigure[Ground Truth]{%
    \includegraphics[width=.18\columnwidth]{figures/supplementary/2010_003531_gt.png}
  }
  \subfigure[DeepLab]{%
    \includegraphics[width=.18\columnwidth]{figures/supplementary/2010_003531_cnn.png}
  }
  \subfigure[+GaussCRF]{%
    \includegraphics[width=.18\columnwidth]{figures/supplementary/2010_003531_gauss.png}
  }
  \subfigure[+LearnedCRF]{%
    \includegraphics[width=.18\columnwidth]{figures/supplementary/2010_003531_learnt.png}
  }
  \vspace{-0.3cm}
  \mycaption{Semantic Segmentation}{Example results of semantic segmentation.
  (c)~depicts the unary results before application of MF, (d)~after two steps of MF with Gaussian edge CRF potentials, (e)~after
  two steps of MF with learned edge CRF potentials.}
    \label{fig:semantic_visuals}
\end{figure*}


\definecolor{minc_1}{HTML}{771111}
\definecolor{minc_2}{HTML}{CAC690}
\definecolor{minc_3}{HTML}{EEEEEE}
\definecolor{minc_4}{HTML}{7C8FA6}
\definecolor{minc_5}{HTML}{597D31}
\definecolor{minc_6}{HTML}{104410}
\definecolor{minc_7}{HTML}{BB819C}
\definecolor{minc_8}{HTML}{D0CE48}
\definecolor{minc_9}{HTML}{622745}
\definecolor{minc_10}{HTML}{666666}
\definecolor{minc_11}{HTML}{D54A31}
\definecolor{minc_12}{HTML}{101044}
\definecolor{minc_13}{HTML}{444126}
\definecolor{minc_14}{HTML}{75D646}
\definecolor{minc_15}{HTML}{DD4348}
\definecolor{minc_16}{HTML}{5C8577}
\definecolor{minc_17}{HTML}{C78472}
\definecolor{minc_18}{HTML}{75D6D0}
\definecolor{minc_19}{HTML}{5B4586}
\definecolor{minc_20}{HTML}{C04393}
\definecolor{minc_21}{HTML}{D69948}
\definecolor{minc_22}{HTML}{7370D8}
\definecolor{minc_23}{HTML}{7A3622}
\definecolor{minc_24}{HTML}{000000}

\begin{figure*}[t]
  \centering
  \small{
  \fcolorbox{white}{minc_1}{\rule{0pt}{6pt}\rule{6pt}{0pt}} Brick~~
  \fcolorbox{white}{minc_2}{\rule{0pt}{6pt}\rule{6pt}{0pt}} Carpet~~
  \fcolorbox{white}{minc_3}{\rule{0pt}{6pt}\rule{6pt}{0pt}} Ceramic~~
  \fcolorbox{white}{minc_4}{\rule{0pt}{6pt}\rule{6pt}{0pt}} Fabric~~
  \fcolorbox{white}{minc_5}{\rule{0pt}{6pt}\rule{6pt}{0pt}} Foliage~~
  \fcolorbox{white}{minc_6}{\rule{0pt}{6pt}\rule{6pt}{0pt}} Food~~
  \fcolorbox{white}{minc_7}{\rule{0pt}{6pt}\rule{6pt}{0pt}} Glass~~
  \fcolorbox{white}{minc_8}{\rule{0pt}{6pt}\rule{6pt}{0pt}} Hair~~ \\
  \fcolorbox{white}{minc_9}{\rule{0pt}{6pt}\rule{6pt}{0pt}} Leather~~
  \fcolorbox{white}{minc_10}{\rule{0pt}{6pt}\rule{6pt}{0pt}} Metal~~
  \fcolorbox{white}{minc_11}{\rule{0pt}{6pt}\rule{6pt}{0pt}} Mirror~~
  \fcolorbox{white}{minc_12}{\rule{0pt}{6pt}\rule{6pt}{0pt}} Other~~
  \fcolorbox{white}{minc_13}{\rule{0pt}{6pt}\rule{6pt}{0pt}} Painted~~
  \fcolorbox{white}{minc_14}{\rule{0pt}{6pt}\rule{6pt}{0pt}} Paper~~
  \fcolorbox{white}{minc_15}{\rule{0pt}{6pt}\rule{6pt}{0pt}} Plastic~~\\
  \fcolorbox{white}{minc_16}{\rule{0pt}{6pt}\rule{6pt}{0pt}} Polished Stone~~
  \fcolorbox{white}{minc_17}{\rule{0pt}{6pt}\rule{6pt}{0pt}} Skin~~
  \fcolorbox{white}{minc_18}{\rule{0pt}{6pt}\rule{6pt}{0pt}} Sky~~
  \fcolorbox{white}{minc_19}{\rule{0pt}{6pt}\rule{6pt}{0pt}} Stone~~
  \fcolorbox{white}{minc_20}{\rule{0pt}{6pt}\rule{6pt}{0pt}} Tile~~
  \fcolorbox{white}{minc_21}{\rule{0pt}{6pt}\rule{6pt}{0pt}} Wallpaper~~
  \fcolorbox{white}{minc_22}{\rule{0pt}{6pt}\rule{6pt}{0pt}} Water~~
  \fcolorbox{white}{minc_23}{\rule{0pt}{6pt}\rule{6pt}{0pt}} Wood~~ \\
  }
  \subfigure{%
    \includegraphics[width=.18\columnwidth]{figures/supplementary/000010868_given.jpg}
  }
  \subfigure{%
    \includegraphics[width=.18\columnwidth]{figures/supplementary/000010868_gt.png}
  }
  \subfigure{%
    \includegraphics[width=.18\columnwidth]{figures/supplementary/000010868_cnn.png}
  }
  \subfigure{%
    \includegraphics[width=.18\columnwidth]{figures/supplementary/000010868_gauss.png}
  }
  \subfigure{%
    \includegraphics[width=.18\columnwidth]{figures/supplementary/000010868_learnt.png}
  }\\[-2ex]
  \subfigure{%
    \includegraphics[width=.18\columnwidth]{figures/supplementary/000006011_given.jpg}
  }
  \subfigure{%
    \includegraphics[width=.18\columnwidth]{figures/supplementary/000006011_gt.png}
  }
  \subfigure{%
    \includegraphics[width=.18\columnwidth]{figures/supplementary/000006011_cnn.png}
  }
  \subfigure{%
    \includegraphics[width=.18\columnwidth]{figures/supplementary/000006011_gauss.png}
  }
  \subfigure{%
    \includegraphics[width=.18\columnwidth]{figures/supplementary/000006011_learnt.png}
  }\\[-2ex]
    \subfigure{%
    \includegraphics[width=.18\columnwidth]{figures/supplementary/000008553_given.jpg}
  }
  \subfigure{%
    \includegraphics[width=.18\columnwidth]{figures/supplementary/000008553_gt.png}
  }
  \subfigure{%
    \includegraphics[width=.18\columnwidth]{figures/supplementary/000008553_cnn.png}
  }
  \subfigure{%
    \includegraphics[width=.18\columnwidth]{figures/supplementary/000008553_gauss.png}
  }
  \subfigure{%
    \includegraphics[width=.18\columnwidth]{figures/supplementary/000008553_learnt.png}
  }\\[-2ex]
   \subfigure{%
    \includegraphics[width=.18\columnwidth]{figures/supplementary/000009188_given.jpg}
  }
  \subfigure{%
    \includegraphics[width=.18\columnwidth]{figures/supplementary/000009188_gt.png}
  }
  \subfigure{%
    \includegraphics[width=.18\columnwidth]{figures/supplementary/000009188_cnn.png}
  }
  \subfigure{%
    \includegraphics[width=.18\columnwidth]{figures/supplementary/000009188_gauss.png}
  }
  \subfigure{%
    \includegraphics[width=.18\columnwidth]{figures/supplementary/000009188_learnt.png}
  }\\[-2ex]
  \setcounter{subfigure}{0}
  \subfigure[Input]{%
    \includegraphics[width=.18\columnwidth]{figures/supplementary/000023570_given.jpg}
  }
  \subfigure[Ground Truth]{%
    \includegraphics[width=.18\columnwidth]{figures/supplementary/000023570_gt.png}
  }
  \subfigure[DeepLab]{%
    \includegraphics[width=.18\columnwidth]{figures/supplementary/000023570_cnn.png}
  }
  \subfigure[+GaussCRF]{%
    \includegraphics[width=.18\columnwidth]{figures/supplementary/000023570_gauss.png}
  }
  \subfigure[+LearnedCRF]{%
    \includegraphics[width=.18\columnwidth]{figures/supplementary/000023570_learnt.png}
  }
  \mycaption{Material Segmentation}{Example results of material segmentation.
  (c)~depicts the unary results before application of MF, (d)~after two steps of MF with Gaussian edge CRF potentials, (e)~after two steps of MF with learned edge CRF potentials.}
    \label{fig:material_visuals-app2}
\end{figure*}


\begin{table*}[h]
\tiny
  \centering
    \begin{tabular}{L{2.3cm} L{2.25cm} C{1.5cm} C{0.7cm} C{0.6cm} C{0.7cm} C{0.7cm} C{0.7cm} C{1.6cm} C{0.6cm} C{0.6cm} C{0.6cm}}
      \toprule
& & & & & \multicolumn{3}{c}{\textbf{Data Statistics}} & \multicolumn{4}{c}{\textbf{Training Protocol}} \\

\textbf{Experiment} & \textbf{Feature Types} & \textbf{Feature Scales} & \textbf{Filter Size} & \textbf{Filter Nbr.} & \textbf{Train}  & \textbf{Val.} & \textbf{Test} & \textbf{Loss Type} & \textbf{LR} & \textbf{Batch} & \textbf{Epochs} \\
      \midrule
      \multicolumn{2}{c}{\textbf{Single Bilateral Filter Applications}} & & & & & & & & & \\
      \textbf{2$\times$ Color Upsampling} & Position$_{1}$, Intensity (3D) & 0.13, 0.17 & 65 & 2 & 10581 & 1449 & 1456 & MSE & 1e-06 & 200 & 94.5\\
      \textbf{4$\times$ Color Upsampling} & Position$_{1}$, Intensity (3D) & 0.06, 0.17 & 65 & 2 & 10581 & 1449 & 1456 & MSE & 1e-06 & 200 & 94.5\\
      \textbf{8$\times$ Color Upsampling} & Position$_{1}$, Intensity (3D) & 0.03, 0.17 & 65 & 2 & 10581 & 1449 & 1456 & MSE & 1e-06 & 200 & 94.5\\
      \textbf{16$\times$ Color Upsampling} & Position$_{1}$, Intensity (3D) & 0.02, 0.17 & 65 & 2 & 10581 & 1449 & 1456 & MSE & 1e-06 & 200 & 94.5\\
      \textbf{Depth Upsampling} & Position$_{1}$, Color (5D) & 0.05, 0.02 & 665 & 2 & 795 & 100 & 654 & MSE & 1e-07 & 50 & 251.6\\
      \textbf{Mesh Denoising} & Isomap (4D) & 46.00 & 63 & 2 & 1000 & 200 & 500 & MSE & 100 & 10 & 100.0 \\
      \midrule
      \multicolumn{2}{c}{\textbf{DenseCRF Applications}} & & & & & & & & &\\
      \multicolumn{2}{l}{\textbf{Semantic Segmentation}} & & & & & & & & &\\
      \textbf{- 1step MF} & Position$_{1}$, Color (5D); Position$_{1}$ (2D) & 0.01, 0.34; 0.34  & 665; 19  & 2; 2 & 10581 & 1449 & 1456 & Logistic & 0.1 & 5 & 1.4 \\
      \textbf{- 2step MF} & Position$_{1}$, Color (5D); Position$_{1}$ (2D) & 0.01, 0.34; 0.34 & 665; 19 & 2; 2 & 10581 & 1449 & 1456 & Logistic & 0.1 & 5 & 1.4 \\
      \textbf{- \textit{loose} 2step MF} & Position$_{1}$, Color (5D); Position$_{1}$ (2D) & 0.01, 0.34; 0.34 & 665; 19 & 2; 2 &10581 & 1449 & 1456 & Logistic & 0.1 & 5 & +1.9  \\ \\
      \multicolumn{2}{l}{\textbf{Material Segmentation}} & & & & & & & & &\\
      \textbf{- 1step MF} & Position$_{2}$, Lab-Color (5D) & 5.00, 0.05, 0.30  & 665 & 2 & 928 & 150 & 1798 & Weighted Logistic & 1e-04 & 24 & 2.6 \\
      \textbf{- 2step MF} & Position$_{2}$, Lab-Color (5D) & 5.00, 0.05, 0.30 & 665 & 2 & 928 & 150 & 1798 & Weighted Logistic & 1e-04 & 12 & +0.7 \\
      \textbf{- \textit{loose} 2step MF} & Position$_{2}$, Lab-Color (5D) & 5.00, 0.05, 0.30 & 665 & 2 & 928 & 150 & 1798 & Weighted Logistic & 1e-04 & 12 & +0.2\\
      \midrule
      \multicolumn{2}{c}{\textbf{Neural Network Applications}} & & & & & & & & &\\
      \textbf{Tiles: CNN-9$\times$9} & - & - & 81 & 4 & 10000 & 1000 & 1000 & Logistic & 0.01 & 100 & 500.0 \\
      \textbf{Tiles: CNN-13$\times$13} & - & - & 169 & 6 & 10000 & 1000 & 1000 & Logistic & 0.01 & 100 & 500.0 \\
      \textbf{Tiles: CNN-17$\times$17} & - & - & 289 & 8 & 10000 & 1000 & 1000 & Logistic & 0.01 & 100 & 500.0 \\
      \textbf{Tiles: CNN-21$\times$21} & - & - & 441 & 10 & 10000 & 1000 & 1000 & Logistic & 0.01 & 100 & 500.0 \\
      \textbf{Tiles: BNN} & Position$_{1}$, Color (5D) & 0.05, 0.04 & 63 & 1 & 10000 & 1000 & 1000 & Logistic & 0.01 & 100 & 30.0 \\
      \textbf{LeNet} & - & - & 25 & 2 & 5490 & 1098 & 1647 & Logistic & 0.1 & 100 & 182.2 \\
      \textbf{Crop-LeNet} & - & - & 25 & 2 & 5490 & 1098 & 1647 & Logistic & 0.1 & 100 & 182.2 \\
      \textbf{BNN-LeNet} & Position$_{2}$ (2D) & 20.00 & 7 & 1 & 5490 & 1098 & 1647 & Logistic & 0.1 & 100 & 182.2 \\
      \textbf{DeepCNet} & - & - & 9 & 1 & 5490 & 1098 & 1647 & Logistic & 0.1 & 100 & 182.2 \\
      \textbf{Crop-DeepCNet} & - & - & 9 & 1 & 5490 & 1098 & 1647 & Logistic & 0.1 & 100 & 182.2 \\
      \textbf{BNN-DeepCNet} & Position$_{2}$ (2D) & 40.00  & 7 & 1 & 5490 & 1098 & 1647 & Logistic & 0.1 & 100 & 182.2 \\
      \bottomrule
      \\
    \end{tabular}
    \mycaption{Experiment Protocols} {Experiment protocols for the different experiments presented in this work. \textbf{Feature Types}:
    Feature spaces used for the bilateral convolutions. Position$_1$ corresponds to un-normalized pixel positions whereas Position$_2$ corresponds
    to pixel positions normalized to $[0,1]$ with respect to the given image. \textbf{Feature Scales}: Cross-validated scales for the features used.
     \textbf{Filter Size}: Number of elements in the filter that is being learned. \textbf{Filter Nbr.}: Half-width of the filter. \textbf{Train},
     \textbf{Val.} and \textbf{Test} corresponds to the number of train, validation and test images used in the experiment. \textbf{Loss Type}: Type
     of loss used for back-propagation. ``MSE'' corresponds to Euclidean mean squared error loss and ``Logistic'' corresponds to multinomial logistic
     loss. ``Weighted Logistic'' is the class-weighted multinomial logistic loss. We weighted the loss with inverse class probability for material
     segmentation task due to the small availability of training data with class imbalance. \textbf{LR}: Fixed learning rate used in stochastic gradient
     descent. \textbf{Batch}: Number of images used in one parameter update step. \textbf{Epochs}: Number of training epochs. In all the experiments,
     we used fixed momentum of 0.9 and weight decay of 0.0005 for stochastic gradient descent. ```Color Upsampling'' experiments in this Table corresponds
     to those performed on Pascal VOC12 dataset images. For all experiments using Pascal VOC12 images, we use extended
     training segmentation dataset available from~\cite{hariharan2011moredata}, and used standard validation and test splits
     from the main dataset~\cite{voc2012segmentation}.}
  \label{tbl:parameters}
\end{table*}

\clearpage

\section{Parameters and Additional Results for Video Propagation Networks}

In this Section, we present experiment protocols and additional qualitative results for experiments
on video object segmentation, semantic video segmentation and video color
propagation. Table~\ref{tbl:parameters_supp} shows the feature scales and other parameters used in different experiments.
Figures~\ref{fig:video_seg_pos_supp} show some qualitative results on video object segmentation
with some failure cases in Fig.~\ref{fig:video_seg_neg_supp}.
Figure~\ref{fig:semantic_visuals_supp} shows some qualitative results on semantic video segmentation and
Fig.~\ref{fig:color_visuals_supp} shows results on video color propagation.

\newcolumntype{L}[1]{>{\raggedright\let\newline\\\arraybackslash\hspace{0pt}}b{#1}}
\newcolumntype{C}[1]{>{\centering\let\newline\\\arraybackslash\hspace{0pt}}b{#1}}
\newcolumntype{R}[1]{>{\raggedleft\let\newline\\\arraybackslash\hspace{0pt}}b{#1}}

\begin{table*}[h]
\tiny
  \centering
    \begin{tabular}{L{3.0cm} L{2.4cm} L{2.8cm} L{2.8cm} C{0.5cm} C{1.0cm} L{1.2cm}}
      \toprule
\textbf{Experiment} & \textbf{Feature Type} & \textbf{Feature Scale-1, $\Lambda_a$} & \textbf{Feature Scale-2, $\Lambda_b$} & \textbf{$\alpha$} & \textbf{Input Frames} & \textbf{Loss Type} \\
      \midrule
      \textbf{Video Object Segmentation} & ($x,y,Y,Cb,Cr,t$) & (0.02,0.02,0.07,0.4,0.4,0.01) & (0.03,0.03,0.09,0.5,0.5,0.2) & 0.5 & 9 & Logistic\\
      \midrule
      \textbf{Semantic Video Segmentation} & & & & & \\
      \textbf{with CNN1~\cite{yu2015multi}-NoFlow} & ($x,y,R,G,B,t$) & (0.08,0.08,0.2,0.2,0.2,0.04) & (0.11,0.11,0.2,0.2,0.2,0.04) & 0.5 & 3 & Logistic \\
      \textbf{with CNN1~\cite{yu2015multi}-Flow} & ($x+u_x,y+u_y,R,G,B,t$) & (0.11,0.11,0.14,0.14,0.14,0.03) & (0.08,0.08,0.12,0.12,0.12,0.01) & 0.65 & 3 & Logistic\\
      \textbf{with CNN2~\cite{richter2016playing}-Flow} & ($x+u_x,y+u_y,R,G,B,t$) & (0.08,0.08,0.2,0.2,0.2,0.04) & (0.09,0.09,0.25,0.25,0.25,0.03) & 0.5 & 4 & Logistic\\
      \midrule
      \textbf{Video Color Propagation} & ($x,y,I,t$)  & (0.04,0.04,0.2,0.04) & No second kernel & 1 & 4 & MSE\\
      \bottomrule
      \\
    \end{tabular}
    \mycaption{Experiment Protocols} {Experiment protocols for the different experiments presented in this work. \textbf{Feature Types}:
    Feature spaces used for the bilateral convolutions, with position ($x,y$) and color
    ($R,G,B$ or $Y,Cb,Cr$) features $\in [0,255]$. $u_x$, $u_y$ denotes optical flow with respect
    to the present frame and $I$ denotes grayscale intensity.
    \textbf{Feature Scales ($\Lambda_a, \Lambda_b$)}: Cross-validated scales for the features used.
    \textbf{$\alpha$}: Exponential time decay for the input frames.
    \textbf{Input Frames}: Number of input frames for VPN.
    \textbf{Loss Type}: Type
     of loss used for back-propagation. ``MSE'' corresponds to Euclidean mean squared error loss and ``Logistic'' corresponds to multinomial logistic loss.}
  \label{tbl:parameters_supp}
\end{table*}

% \begin{figure}[th!]
% \begin{center}
%   \centerline{\includegraphics[width=\textwidth]{figures/video_seg_visuals_supp_small.pdf}}
%     \mycaption{Video Object Segmentation}
%     {Shown are the different frames in example videos with the corresponding
%     ground truth (GT) masks, predictions from BVS~\cite{marki2016bilateral},
%     OFL~\cite{tsaivideo}, VPN (VPN-Stage2) and VPN-DLab (VPN-DeepLab) models.}
%     \label{fig:video_seg_small_supp}
% \end{center}
% \vspace{-1.0cm}
% \end{figure}

\begin{figure}[th!]
\begin{center}
  \centerline{\includegraphics[width=0.7\textwidth]{figures/video_seg_visuals_supp_positive.pdf}}
    \mycaption{Video Object Segmentation}
    {Shown are the different frames in example videos with the corresponding
    ground truth (GT) masks, predictions from BVS~\cite{marki2016bilateral},
    OFL~\cite{tsaivideo}, VPN (VPN-Stage2) and VPN-DLab (VPN-DeepLab) models.}
    \label{fig:video_seg_pos_supp}
\end{center}
\vspace{-1.0cm}
\end{figure}

\begin{figure}[th!]
\begin{center}
  \centerline{\includegraphics[width=0.7\textwidth]{figures/video_seg_visuals_supp_negative.pdf}}
    \mycaption{Failure Cases for Video Object Segmentation}
    {Shown are the different frames in example videos with the corresponding
    ground truth (GT) masks, predictions from BVS~\cite{marki2016bilateral},
    OFL~\cite{tsaivideo}, VPN (VPN-Stage2) and VPN-DLab (VPN-DeepLab) models.}
    \label{fig:video_seg_neg_supp}
\end{center}
\vspace{-1.0cm}
\end{figure}

\begin{figure}[th!]
\begin{center}
  \centerline{\includegraphics[width=0.9\textwidth]{figures/supp_semantic_visual.pdf}}
    \mycaption{Semantic Video Segmentation}
    {Input video frames and the corresponding ground truth (GT)
    segmentation together with the predictions of CNN~\cite{yu2015multi} and with
    VPN-Flow.}
    \label{fig:semantic_visuals_supp}
\end{center}
\vspace{-0.7cm}
\end{figure}

\begin{figure}[th!]
\begin{center}
  \centerline{\includegraphics[width=\textwidth]{figures/colorization_visuals_supp.pdf}}
  \mycaption{Video Color Propagation}
  {Input grayscale video frames and corresponding ground-truth (GT) color images
  together with color predictions of Levin et al.~\cite{levin2004colorization} and VPN-Stage1 models.}
  \label{fig:color_visuals_supp}
\end{center}
\vspace{-0.7cm}
\end{figure}

\clearpage

\section{Additional Material for Bilateral Inception Networks}
\label{sec:binception-app}

In this section of the Appendix, we first discuss the use of approximate bilateral
filtering in BI modules (Sec.~\ref{sec:lattice}).
Later, we present some qualitative results using different models for the approach presented in
Chapter~\ref{chap:binception} (Sec.~\ref{sec:qualitative-app}).

\subsection{Approximate Bilateral Filtering}
\label{sec:lattice}

The bilateral inception module presented in Chapter~\ref{chap:binception} computes a matrix-vector
product between a Gaussian filter $K$ and a vector of activations $\bz_c$.
Bilateral filtering is an important operation and many algorithmic techniques have been
proposed to speed-up this operation~\cite{paris2006fast,adams2010fast,gastal2011domain}.
In the main paper we opted to implement what can be considered the
brute-force variant of explicitly constructing $K$ and then using BLAS to compute the
matrix-vector product. This resulted in a few millisecond operation.
The explicit way to compute is possible due to the
reduction to super-pixels, e.g., it would not work for DenseCRF variants
that operate on the full image resolution.

Here, we present experiments where we use the fast approximate bilateral filtering
algorithm of~\cite{adams2010fast}, which is also used in Chapter~\ref{chap:bnn}
for learning sparse high dimensional filters. This
choice allows for larger dimensions of matrix-vector multiplication. The reason for choosing
the explicit multiplication in Chapter~\ref{chap:binception} was that it was computationally faster.
For the small sizes of the involved matrices and vectors, the explicit computation is sufficient and we had no
GPU implementation of an approximate technique that matched this runtime. Also it
is conceptually easier and the gradient to the feature transformations ($\Lambda \mathbf{f}$) is
obtained using standard matrix calculus.

\subsubsection{Experiments}

We modified the existing segmentation architectures analogous to those in Chapter~\ref{chap:binception}.
The main difference is that, here, the inception modules use the lattice
approximation~\cite{adams2010fast} to compute the bilateral filtering.
Using the lattice approximation did not allow us to back-propagate through feature transformations ($\Lambda$)
and thus we used hand-specified feature scales as will be explained later.
Specifically, we take CNN architectures from the works
of~\cite{chen2014semantic,zheng2015conditional,bell2015minc} and insert the BI modules between
the spatial FC layers.
We use superpixels from~\cite{DollarICCV13edges}
for all the experiments with the lattice approximation. Experiments are
performed using Caffe neural network framework~\cite{jia2014caffe}.

\begin{table}
  \small
  \centering
  \begin{tabular}{p{5.5cm}>{\raggedright\arraybackslash}p{1.4cm}>{\centering\arraybackslash}p{2.2cm}}
    \toprule
		\textbf{Model} & \emph{IoU} & \emph{Runtime}(ms) \\
    \midrule

    %%%%%%%%%%%% Scores computed by us)%%%%%%%%%%%%
		\deeplablargefov & 68.9 & 145ms\\
    \midrule
    \bi{7}{2}-\bi{8}{10}& \textbf{73.8} & +600 \\
    \midrule
    \deeplablargefovcrf~\cite{chen2014semantic} & 72.7 & +830\\
    \deeplabmsclargefovcrf~\cite{chen2014semantic} & \textbf{73.6} & +880\\
    DeepLab-EdgeNet~\cite{chen2015semantic} & 71.7 & +30\\
    DeepLab-EdgeNet-CRF~\cite{chen2015semantic} & \textbf{73.6} & +860\\
  \bottomrule \\
  \end{tabular}
  \mycaption{Semantic Segmentation using the DeepLab model}
  {IoU scores on the Pascal VOC12 segmentation test dataset
  with different models and our modified inception model.
  Also shown are the corresponding runtimes in milliseconds. Runtimes
  also include superpixel computations (300 ms with Dollar superpixels~\cite{DollarICCV13edges})}
  \label{tab:largefovresults}
\end{table}

\paragraph{Semantic Segmentation}
The experiments in this section use the Pascal VOC12 segmentation dataset~\cite{voc2012segmentation} with 21 object classes and the images have a maximum resolution of 0.25 megapixels.
For all experiments on VOC12, we train using the extended training set of
10581 images collected by~\cite{hariharan2011moredata}.
We modified the \deeplab~network architecture of~\cite{chen2014semantic} and
the CRFasRNN architecture from~\cite{zheng2015conditional} which uses a CNN with
deconvolution layers followed by DenseCRF trained end-to-end.

\paragraph{DeepLab Model}\label{sec:deeplabmodel}
We experimented with the \bi{7}{2}-\bi{8}{10} inception model.
Results using the~\deeplab~model are summarized in Tab.~\ref{tab:largefovresults}.
Although we get similar improvements with inception modules as with the
explicit kernel computation, using lattice approximation is slower.

\begin{table}
  \small
  \centering
  \begin{tabular}{p{6.4cm}>{\raggedright\arraybackslash}p{1.8cm}>{\raggedright\arraybackslash}p{1.8cm}}
    \toprule
    \textbf{Model} & \emph{IoU (Val)} & \emph{IoU (Test)}\\
    \midrule
    %%%%%%%%%%%% Scores computed by us)%%%%%%%%%%%%
    CNN &  67.5 & - \\
    \deconv (CNN+Deconvolutions) & 69.8 & 72.0 \\
    \midrule
    \bi{3}{6}-\bi{4}{6}-\bi{7}{2}-\bi{8}{6}& 71.9 & - \\
    \bi{3}{6}-\bi{4}{6}-\bi{7}{2}-\bi{8}{6}-\gi{6}& 73.6 &  \href{http://host.robots.ox.ac.uk:8080/anonymous/VOTV5E.html}{\textbf{75.2}}\\
    \midrule
    \deconvcrf (CRF-RNN)~\cite{zheng2015conditional} & 73.0 & 74.7\\
    Context-CRF-RNN~\cite{yu2015multi} & ~~ - ~ & \textbf{75.3} \\
    \bottomrule \\
  \end{tabular}
  \mycaption{Semantic Segmentation using the CRFasRNN model}{IoU score corresponding to different models
  on Pascal VOC12 reduced validation / test segmentation dataset. The reduced validation set consists of 346 images
  as used in~\cite{zheng2015conditional} where we adapted the model from.}
  \label{tab:deconvresults-app}
\end{table}

\paragraph{CRFasRNN Model}\label{sec:deepinception}
We add BI modules after score-pool3, score-pool4, \fc{7} and \fc{8} $1\times1$ convolution layers
resulting in the \bi{3}{6}-\bi{4}{6}-\bi{7}{2}-\bi{8}{6}
model and also experimented with another variant where $BI_8$ is followed by another inception
module, G$(6)$, with 6 Gaussian kernels.
Note that here also we discarded both deconvolution and DenseCRF parts of the original model~\cite{zheng2015conditional}
and inserted the BI modules in the base CNN and found similar improvements compared to the inception modules with explicit
kernel computaion. See Tab.~\ref{tab:deconvresults-app} for results on the CRFasRNN model.

\paragraph{Material Segmentation}
Table~\ref{tab:mincresults-app} shows the results on the MINC dataset~\cite{bell2015minc}
obtained by modifying the AlexNet architecture with our inception modules. We observe
similar improvements as with explicit kernel construction.
For this model, we do not provide any learned setup due to very limited segment training
data. The weights to combine outputs in the bilateral inception layer are
found by validation on the validation set.

\begin{table}[t]
  \small
  \centering
  \begin{tabular}{p{3.5cm}>{\centering\arraybackslash}p{4.0cm}}
    \toprule
    \textbf{Model} & Class / Total accuracy\\
    \midrule

    %%%%%%%%%%%% Scores computed by us)%%%%%%%%%%%%
    AlexNet CNN & 55.3 / 58.9 \\
    \midrule
    \bi{7}{2}-\bi{8}{6}& 68.5 / 71.8 \\
    \bi{7}{2}-\bi{8}{6}-G$(6)$& 67.6 / 73.1 \\
    \midrule
    AlexNet-CRF & 65.5 / 71.0 \\
    \bottomrule \\
  \end{tabular}
  \mycaption{Material Segmentation using AlexNet}{Pixel accuracy of different models on
  the MINC material segmentation test dataset~\cite{bell2015minc}.}
  \label{tab:mincresults-app}
\end{table}

\paragraph{Scales of Bilateral Inception Modules}
\label{sec:scales}

Unlike the explicit kernel technique presented in the main text (Chapter~\ref{chap:binception}),
we didn't back-propagate through feature transformation ($\Lambda$)
using the approximate bilateral filter technique.
So, the feature scales are hand-specified and validated, which are as follows.
The optimal scale values for the \bi{7}{2}-\bi{8}{2} model are found by validation for the best performance which are
$\sigma_{xy}$ = (0.1, 0.1) for the spatial (XY) kernel and $\sigma_{rgbxy}$ = (0.1, 0.1, 0.1, 0.01, 0.01) for color and position (RGBXY)  kernel.
Next, as more kernels are added to \bi{8}{2}, we set scales to be $\alpha$*($\sigma_{xy}$, $\sigma_{rgbxy}$).
The value of $\alpha$ is chosen as  1, 0.5, 0.1, 0.05, 0.1, at uniform interval, for the \bi{8}{10} bilateral inception module.


\subsection{Qualitative Results}
\label{sec:qualitative-app}

In this section, we present more qualitative results obtained using the BI module with explicit
kernel computation technique presented in Chapter~\ref{chap:binception}. Results on the Pascal VOC12
dataset~\cite{voc2012segmentation} using the DeepLab-LargeFOV model are shown in Fig.~\ref{fig:semantic_visuals-app},
followed by the results on MINC dataset~\cite{bell2015minc}
in Fig.~\ref{fig:material_visuals-app} and on
Cityscapes dataset~\cite{Cordts2015Cvprw} in Fig.~\ref{fig:street_visuals-app}.


\definecolor{voc_1}{RGB}{0, 0, 0}
\definecolor{voc_2}{RGB}{128, 0, 0}
\definecolor{voc_3}{RGB}{0, 128, 0}
\definecolor{voc_4}{RGB}{128, 128, 0}
\definecolor{voc_5}{RGB}{0, 0, 128}
\definecolor{voc_6}{RGB}{128, 0, 128}
\definecolor{voc_7}{RGB}{0, 128, 128}
\definecolor{voc_8}{RGB}{128, 128, 128}
\definecolor{voc_9}{RGB}{64, 0, 0}
\definecolor{voc_10}{RGB}{192, 0, 0}
\definecolor{voc_11}{RGB}{64, 128, 0}
\definecolor{voc_12}{RGB}{192, 128, 0}
\definecolor{voc_13}{RGB}{64, 0, 128}
\definecolor{voc_14}{RGB}{192, 0, 128}
\definecolor{voc_15}{RGB}{64, 128, 128}
\definecolor{voc_16}{RGB}{192, 128, 128}
\definecolor{voc_17}{RGB}{0, 64, 0}
\definecolor{voc_18}{RGB}{128, 64, 0}
\definecolor{voc_19}{RGB}{0, 192, 0}
\definecolor{voc_20}{RGB}{128, 192, 0}
\definecolor{voc_21}{RGB}{0, 64, 128}
\definecolor{voc_22}{RGB}{128, 64, 128}

\begin{figure*}[!ht]
  \small
  \centering
  \fcolorbox{white}{voc_1}{\rule{0pt}{4pt}\rule{4pt}{0pt}} Background~~
  \fcolorbox{white}{voc_2}{\rule{0pt}{4pt}\rule{4pt}{0pt}} Aeroplane~~
  \fcolorbox{white}{voc_3}{\rule{0pt}{4pt}\rule{4pt}{0pt}} Bicycle~~
  \fcolorbox{white}{voc_4}{\rule{0pt}{4pt}\rule{4pt}{0pt}} Bird~~
  \fcolorbox{white}{voc_5}{\rule{0pt}{4pt}\rule{4pt}{0pt}} Boat~~
  \fcolorbox{white}{voc_6}{\rule{0pt}{4pt}\rule{4pt}{0pt}} Bottle~~
  \fcolorbox{white}{voc_7}{\rule{0pt}{4pt}\rule{4pt}{0pt}} Bus~~
  \fcolorbox{white}{voc_8}{\rule{0pt}{4pt}\rule{4pt}{0pt}} Car~~\\
  \fcolorbox{white}{voc_9}{\rule{0pt}{4pt}\rule{4pt}{0pt}} Cat~~
  \fcolorbox{white}{voc_10}{\rule{0pt}{4pt}\rule{4pt}{0pt}} Chair~~
  \fcolorbox{white}{voc_11}{\rule{0pt}{4pt}\rule{4pt}{0pt}} Cow~~
  \fcolorbox{white}{voc_12}{\rule{0pt}{4pt}\rule{4pt}{0pt}} Dining Table~~
  \fcolorbox{white}{voc_13}{\rule{0pt}{4pt}\rule{4pt}{0pt}} Dog~~
  \fcolorbox{white}{voc_14}{\rule{0pt}{4pt}\rule{4pt}{0pt}} Horse~~
  \fcolorbox{white}{voc_15}{\rule{0pt}{4pt}\rule{4pt}{0pt}} Motorbike~~
  \fcolorbox{white}{voc_16}{\rule{0pt}{4pt}\rule{4pt}{0pt}} Person~~\\
  \fcolorbox{white}{voc_17}{\rule{0pt}{4pt}\rule{4pt}{0pt}} Potted Plant~~
  \fcolorbox{white}{voc_18}{\rule{0pt}{4pt}\rule{4pt}{0pt}} Sheep~~
  \fcolorbox{white}{voc_19}{\rule{0pt}{4pt}\rule{4pt}{0pt}} Sofa~~
  \fcolorbox{white}{voc_20}{\rule{0pt}{4pt}\rule{4pt}{0pt}} Train~~
  \fcolorbox{white}{voc_21}{\rule{0pt}{4pt}\rule{4pt}{0pt}} TV monitor~~\\


  \subfigure{%
    \includegraphics[width=.15\columnwidth]{figures/supplementary/2008_001308_given.png}
  }
  \subfigure{%
    \includegraphics[width=.15\columnwidth]{figures/supplementary/2008_001308_sp.png}
  }
  \subfigure{%
    \includegraphics[width=.15\columnwidth]{figures/supplementary/2008_001308_gt.png}
  }
  \subfigure{%
    \includegraphics[width=.15\columnwidth]{figures/supplementary/2008_001308_cnn.png}
  }
  \subfigure{%
    \includegraphics[width=.15\columnwidth]{figures/supplementary/2008_001308_crf.png}
  }
  \subfigure{%
    \includegraphics[width=.15\columnwidth]{figures/supplementary/2008_001308_ours.png}
  }\\[-2ex]


  \subfigure{%
    \includegraphics[width=.15\columnwidth]{figures/supplementary/2008_001821_given.png}
  }
  \subfigure{%
    \includegraphics[width=.15\columnwidth]{figures/supplementary/2008_001821_sp.png}
  }
  \subfigure{%
    \includegraphics[width=.15\columnwidth]{figures/supplementary/2008_001821_gt.png}
  }
  \subfigure{%
    \includegraphics[width=.15\columnwidth]{figures/supplementary/2008_001821_cnn.png}
  }
  \subfigure{%
    \includegraphics[width=.15\columnwidth]{figures/supplementary/2008_001821_crf.png}
  }
  \subfigure{%
    \includegraphics[width=.15\columnwidth]{figures/supplementary/2008_001821_ours.png}
  }\\[-2ex]



  \subfigure{%
    \includegraphics[width=.15\columnwidth]{figures/supplementary/2008_004612_given.png}
  }
  \subfigure{%
    \includegraphics[width=.15\columnwidth]{figures/supplementary/2008_004612_sp.png}
  }
  \subfigure{%
    \includegraphics[width=.15\columnwidth]{figures/supplementary/2008_004612_gt.png}
  }
  \subfigure{%
    \includegraphics[width=.15\columnwidth]{figures/supplementary/2008_004612_cnn.png}
  }
  \subfigure{%
    \includegraphics[width=.15\columnwidth]{figures/supplementary/2008_004612_crf.png}
  }
  \subfigure{%
    \includegraphics[width=.15\columnwidth]{figures/supplementary/2008_004612_ours.png}
  }\\[-2ex]


  \subfigure{%
    \includegraphics[width=.15\columnwidth]{figures/supplementary/2009_001008_given.png}
  }
  \subfigure{%
    \includegraphics[width=.15\columnwidth]{figures/supplementary/2009_001008_sp.png}
  }
  \subfigure{%
    \includegraphics[width=.15\columnwidth]{figures/supplementary/2009_001008_gt.png}
  }
  \subfigure{%
    \includegraphics[width=.15\columnwidth]{figures/supplementary/2009_001008_cnn.png}
  }
  \subfigure{%
    \includegraphics[width=.15\columnwidth]{figures/supplementary/2009_001008_crf.png}
  }
  \subfigure{%
    \includegraphics[width=.15\columnwidth]{figures/supplementary/2009_001008_ours.png}
  }\\[-2ex]




  \subfigure{%
    \includegraphics[width=.15\columnwidth]{figures/supplementary/2009_004497_given.png}
  }
  \subfigure{%
    \includegraphics[width=.15\columnwidth]{figures/supplementary/2009_004497_sp.png}
  }
  \subfigure{%
    \includegraphics[width=.15\columnwidth]{figures/supplementary/2009_004497_gt.png}
  }
  \subfigure{%
    \includegraphics[width=.15\columnwidth]{figures/supplementary/2009_004497_cnn.png}
  }
  \subfigure{%
    \includegraphics[width=.15\columnwidth]{figures/supplementary/2009_004497_crf.png}
  }
  \subfigure{%
    \includegraphics[width=.15\columnwidth]{figures/supplementary/2009_004497_ours.png}
  }\\[-2ex]



  \setcounter{subfigure}{0}
  \subfigure[\scriptsize Input]{%
    \includegraphics[width=.15\columnwidth]{figures/supplementary/2010_001327_given.png}
  }
  \subfigure[\scriptsize Superpixels]{%
    \includegraphics[width=.15\columnwidth]{figures/supplementary/2010_001327_sp.png}
  }
  \subfigure[\scriptsize GT]{%
    \includegraphics[width=.15\columnwidth]{figures/supplementary/2010_001327_gt.png}
  }
  \subfigure[\scriptsize Deeplab]{%
    \includegraphics[width=.15\columnwidth]{figures/supplementary/2010_001327_cnn.png}
  }
  \subfigure[\scriptsize +DenseCRF]{%
    \includegraphics[width=.15\columnwidth]{figures/supplementary/2010_001327_crf.png}
  }
  \subfigure[\scriptsize Using BI]{%
    \includegraphics[width=.15\columnwidth]{figures/supplementary/2010_001327_ours.png}
  }
  \mycaption{Semantic Segmentation}{Example results of semantic segmentation
  on the Pascal VOC12 dataset.
  (d)~depicts the DeepLab CNN result, (e)~CNN + 10 steps of mean-field inference,
  (f~result obtained with bilateral inception (BI) modules (\bi{6}{2}+\bi{7}{6}) between \fc~layers.}
  \label{fig:semantic_visuals-app}
\end{figure*}


\definecolor{minc_1}{HTML}{771111}
\definecolor{minc_2}{HTML}{CAC690}
\definecolor{minc_3}{HTML}{EEEEEE}
\definecolor{minc_4}{HTML}{7C8FA6}
\definecolor{minc_5}{HTML}{597D31}
\definecolor{minc_6}{HTML}{104410}
\definecolor{minc_7}{HTML}{BB819C}
\definecolor{minc_8}{HTML}{D0CE48}
\definecolor{minc_9}{HTML}{622745}
\definecolor{minc_10}{HTML}{666666}
\definecolor{minc_11}{HTML}{D54A31}
\definecolor{minc_12}{HTML}{101044}
\definecolor{minc_13}{HTML}{444126}
\definecolor{minc_14}{HTML}{75D646}
\definecolor{minc_15}{HTML}{DD4348}
\definecolor{minc_16}{HTML}{5C8577}
\definecolor{minc_17}{HTML}{C78472}
\definecolor{minc_18}{HTML}{75D6D0}
\definecolor{minc_19}{HTML}{5B4586}
\definecolor{minc_20}{HTML}{C04393}
\definecolor{minc_21}{HTML}{D69948}
\definecolor{minc_22}{HTML}{7370D8}
\definecolor{minc_23}{HTML}{7A3622}
\definecolor{minc_24}{HTML}{000000}

\begin{figure*}[!ht]
  \small % scriptsize
  \centering
  \fcolorbox{white}{minc_1}{\rule{0pt}{4pt}\rule{4pt}{0pt}} Brick~~
  \fcolorbox{white}{minc_2}{\rule{0pt}{4pt}\rule{4pt}{0pt}} Carpet~~
  \fcolorbox{white}{minc_3}{\rule{0pt}{4pt}\rule{4pt}{0pt}} Ceramic~~
  \fcolorbox{white}{minc_4}{\rule{0pt}{4pt}\rule{4pt}{0pt}} Fabric~~
  \fcolorbox{white}{minc_5}{\rule{0pt}{4pt}\rule{4pt}{0pt}} Foliage~~
  \fcolorbox{white}{minc_6}{\rule{0pt}{4pt}\rule{4pt}{0pt}} Food~~
  \fcolorbox{white}{minc_7}{\rule{0pt}{4pt}\rule{4pt}{0pt}} Glass~~
  \fcolorbox{white}{minc_8}{\rule{0pt}{4pt}\rule{4pt}{0pt}} Hair~~\\
  \fcolorbox{white}{minc_9}{\rule{0pt}{4pt}\rule{4pt}{0pt}} Leather~~
  \fcolorbox{white}{minc_10}{\rule{0pt}{4pt}\rule{4pt}{0pt}} Metal~~
  \fcolorbox{white}{minc_11}{\rule{0pt}{4pt}\rule{4pt}{0pt}} Mirror~~
  \fcolorbox{white}{minc_12}{\rule{0pt}{4pt}\rule{4pt}{0pt}} Other~~
  \fcolorbox{white}{minc_13}{\rule{0pt}{4pt}\rule{4pt}{0pt}} Painted~~
  \fcolorbox{white}{minc_14}{\rule{0pt}{4pt}\rule{4pt}{0pt}} Paper~~
  \fcolorbox{white}{minc_15}{\rule{0pt}{4pt}\rule{4pt}{0pt}} Plastic~~\\
  \fcolorbox{white}{minc_16}{\rule{0pt}{4pt}\rule{4pt}{0pt}} Polished Stone~~
  \fcolorbox{white}{minc_17}{\rule{0pt}{4pt}\rule{4pt}{0pt}} Skin~~
  \fcolorbox{white}{minc_18}{\rule{0pt}{4pt}\rule{4pt}{0pt}} Sky~~
  \fcolorbox{white}{minc_19}{\rule{0pt}{4pt}\rule{4pt}{0pt}} Stone~~
  \fcolorbox{white}{minc_20}{\rule{0pt}{4pt}\rule{4pt}{0pt}} Tile~~
  \fcolorbox{white}{minc_21}{\rule{0pt}{4pt}\rule{4pt}{0pt}} Wallpaper~~
  \fcolorbox{white}{minc_22}{\rule{0pt}{4pt}\rule{4pt}{0pt}} Water~~
  \fcolorbox{white}{minc_23}{\rule{0pt}{4pt}\rule{4pt}{0pt}} Wood~~\\
  \subfigure{%
    \includegraphics[width=.15\columnwidth]{figures/supplementary/000008468_given.png}
  }
  \subfigure{%
    \includegraphics[width=.15\columnwidth]{figures/supplementary/000008468_sp.png}
  }
  \subfigure{%
    \includegraphics[width=.15\columnwidth]{figures/supplementary/000008468_gt.png}
  }
  \subfigure{%
    \includegraphics[width=.15\columnwidth]{figures/supplementary/000008468_cnn.png}
  }
  \subfigure{%
    \includegraphics[width=.15\columnwidth]{figures/supplementary/000008468_crf.png}
  }
  \subfigure{%
    \includegraphics[width=.15\columnwidth]{figures/supplementary/000008468_ours.png}
  }\\[-2ex]

  \subfigure{%
    \includegraphics[width=.15\columnwidth]{figures/supplementary/000009053_given.png}
  }
  \subfigure{%
    \includegraphics[width=.15\columnwidth]{figures/supplementary/000009053_sp.png}
  }
  \subfigure{%
    \includegraphics[width=.15\columnwidth]{figures/supplementary/000009053_gt.png}
  }
  \subfigure{%
    \includegraphics[width=.15\columnwidth]{figures/supplementary/000009053_cnn.png}
  }
  \subfigure{%
    \includegraphics[width=.15\columnwidth]{figures/supplementary/000009053_crf.png}
  }
  \subfigure{%
    \includegraphics[width=.15\columnwidth]{figures/supplementary/000009053_ours.png}
  }\\[-2ex]




  \subfigure{%
    \includegraphics[width=.15\columnwidth]{figures/supplementary/000014977_given.png}
  }
  \subfigure{%
    \includegraphics[width=.15\columnwidth]{figures/supplementary/000014977_sp.png}
  }
  \subfigure{%
    \includegraphics[width=.15\columnwidth]{figures/supplementary/000014977_gt.png}
  }
  \subfigure{%
    \includegraphics[width=.15\columnwidth]{figures/supplementary/000014977_cnn.png}
  }
  \subfigure{%
    \includegraphics[width=.15\columnwidth]{figures/supplementary/000014977_crf.png}
  }
  \subfigure{%
    \includegraphics[width=.15\columnwidth]{figures/supplementary/000014977_ours.png}
  }\\[-2ex]


  \subfigure{%
    \includegraphics[width=.15\columnwidth]{figures/supplementary/000022922_given.png}
  }
  \subfigure{%
    \includegraphics[width=.15\columnwidth]{figures/supplementary/000022922_sp.png}
  }
  \subfigure{%
    \includegraphics[width=.15\columnwidth]{figures/supplementary/000022922_gt.png}
  }
  \subfigure{%
    \includegraphics[width=.15\columnwidth]{figures/supplementary/000022922_cnn.png}
  }
  \subfigure{%
    \includegraphics[width=.15\columnwidth]{figures/supplementary/000022922_crf.png}
  }
  \subfigure{%
    \includegraphics[width=.15\columnwidth]{figures/supplementary/000022922_ours.png}
  }\\[-2ex]


  \subfigure{%
    \includegraphics[width=.15\columnwidth]{figures/supplementary/000025711_given.png}
  }
  \subfigure{%
    \includegraphics[width=.15\columnwidth]{figures/supplementary/000025711_sp.png}
  }
  \subfigure{%
    \includegraphics[width=.15\columnwidth]{figures/supplementary/000025711_gt.png}
  }
  \subfigure{%
    \includegraphics[width=.15\columnwidth]{figures/supplementary/000025711_cnn.png}
  }
  \subfigure{%
    \includegraphics[width=.15\columnwidth]{figures/supplementary/000025711_crf.png}
  }
  \subfigure{%
    \includegraphics[width=.15\columnwidth]{figures/supplementary/000025711_ours.png}
  }\\[-2ex]


  \subfigure{%
    \includegraphics[width=.15\columnwidth]{figures/supplementary/000034473_given.png}
  }
  \subfigure{%
    \includegraphics[width=.15\columnwidth]{figures/supplementary/000034473_sp.png}
  }
  \subfigure{%
    \includegraphics[width=.15\columnwidth]{figures/supplementary/000034473_gt.png}
  }
  \subfigure{%
    \includegraphics[width=.15\columnwidth]{figures/supplementary/000034473_cnn.png}
  }
  \subfigure{%
    \includegraphics[width=.15\columnwidth]{figures/supplementary/000034473_crf.png}
  }
  \subfigure{%
    \includegraphics[width=.15\columnwidth]{figures/supplementary/000034473_ours.png}
  }\\[-2ex]


  \subfigure{%
    \includegraphics[width=.15\columnwidth]{figures/supplementary/000035463_given.png}
  }
  \subfigure{%
    \includegraphics[width=.15\columnwidth]{figures/supplementary/000035463_sp.png}
  }
  \subfigure{%
    \includegraphics[width=.15\columnwidth]{figures/supplementary/000035463_gt.png}
  }
  \subfigure{%
    \includegraphics[width=.15\columnwidth]{figures/supplementary/000035463_cnn.png}
  }
  \subfigure{%
    \includegraphics[width=.15\columnwidth]{figures/supplementary/000035463_crf.png}
  }
  \subfigure{%
    \includegraphics[width=.15\columnwidth]{figures/supplementary/000035463_ours.png}
  }\\[-2ex]


  \setcounter{subfigure}{0}
  \subfigure[\scriptsize Input]{%
    \includegraphics[width=.15\columnwidth]{figures/supplementary/000035993_given.png}
  }
  \subfigure[\scriptsize Superpixels]{%
    \includegraphics[width=.15\columnwidth]{figures/supplementary/000035993_sp.png}
  }
  \subfigure[\scriptsize GT]{%
    \includegraphics[width=.15\columnwidth]{figures/supplementary/000035993_gt.png}
  }
  \subfigure[\scriptsize AlexNet]{%
    \includegraphics[width=.15\columnwidth]{figures/supplementary/000035993_cnn.png}
  }
  \subfigure[\scriptsize +DenseCRF]{%
    \includegraphics[width=.15\columnwidth]{figures/supplementary/000035993_crf.png}
  }
  \subfigure[\scriptsize Using BI]{%
    \includegraphics[width=.15\columnwidth]{figures/supplementary/000035993_ours.png}
  }
  \mycaption{Material Segmentation}{Example results of material segmentation.
  (d)~depicts the AlexNet CNN result, (e)~CNN + 10 steps of mean-field inference,
  (f)~result obtained with bilateral inception (BI) modules (\bi{7}{2}+\bi{8}{6}) between
  \fc~layers.}
\label{fig:material_visuals-app}
\end{figure*}


\definecolor{city_1}{RGB}{128, 64, 128}
\definecolor{city_2}{RGB}{244, 35, 232}
\definecolor{city_3}{RGB}{70, 70, 70}
\definecolor{city_4}{RGB}{102, 102, 156}
\definecolor{city_5}{RGB}{190, 153, 153}
\definecolor{city_6}{RGB}{153, 153, 153}
\definecolor{city_7}{RGB}{250, 170, 30}
\definecolor{city_8}{RGB}{220, 220, 0}
\definecolor{city_9}{RGB}{107, 142, 35}
\definecolor{city_10}{RGB}{152, 251, 152}
\definecolor{city_11}{RGB}{70, 130, 180}
\definecolor{city_12}{RGB}{220, 20, 60}
\definecolor{city_13}{RGB}{255, 0, 0}
\definecolor{city_14}{RGB}{0, 0, 142}
\definecolor{city_15}{RGB}{0, 0, 70}
\definecolor{city_16}{RGB}{0, 60, 100}
\definecolor{city_17}{RGB}{0, 80, 100}
\definecolor{city_18}{RGB}{0, 0, 230}
\definecolor{city_19}{RGB}{119, 11, 32}
\begin{figure*}[!ht]
  \small % scriptsize
  \centering


  \subfigure{%
    \includegraphics[width=.18\columnwidth]{figures/supplementary/frankfurt00000_016005_given.png}
  }
  \subfigure{%
    \includegraphics[width=.18\columnwidth]{figures/supplementary/frankfurt00000_016005_sp.png}
  }
  \subfigure{%
    \includegraphics[width=.18\columnwidth]{figures/supplementary/frankfurt00000_016005_gt.png}
  }
  \subfigure{%
    \includegraphics[width=.18\columnwidth]{figures/supplementary/frankfurt00000_016005_cnn.png}
  }
  \subfigure{%
    \includegraphics[width=.18\columnwidth]{figures/supplementary/frankfurt00000_016005_ours.png}
  }\\[-2ex]

  \subfigure{%
    \includegraphics[width=.18\columnwidth]{figures/supplementary/frankfurt00000_004617_given.png}
  }
  \subfigure{%
    \includegraphics[width=.18\columnwidth]{figures/supplementary/frankfurt00000_004617_sp.png}
  }
  \subfigure{%
    \includegraphics[width=.18\columnwidth]{figures/supplementary/frankfurt00000_004617_gt.png}
  }
  \subfigure{%
    \includegraphics[width=.18\columnwidth]{figures/supplementary/frankfurt00000_004617_cnn.png}
  }
  \subfigure{%
    \includegraphics[width=.18\columnwidth]{figures/supplementary/frankfurt00000_004617_ours.png}
  }\\[-2ex]

  \subfigure{%
    \includegraphics[width=.18\columnwidth]{figures/supplementary/frankfurt00000_020880_given.png}
  }
  \subfigure{%
    \includegraphics[width=.18\columnwidth]{figures/supplementary/frankfurt00000_020880_sp.png}
  }
  \subfigure{%
    \includegraphics[width=.18\columnwidth]{figures/supplementary/frankfurt00000_020880_gt.png}
  }
  \subfigure{%
    \includegraphics[width=.18\columnwidth]{figures/supplementary/frankfurt00000_020880_cnn.png}
  }
  \subfigure{%
    \includegraphics[width=.18\columnwidth]{figures/supplementary/frankfurt00000_020880_ours.png}
  }\\[-2ex]



  \subfigure{%
    \includegraphics[width=.18\columnwidth]{figures/supplementary/frankfurt00001_007285_given.png}
  }
  \subfigure{%
    \includegraphics[width=.18\columnwidth]{figures/supplementary/frankfurt00001_007285_sp.png}
  }
  \subfigure{%
    \includegraphics[width=.18\columnwidth]{figures/supplementary/frankfurt00001_007285_gt.png}
  }
  \subfigure{%
    \includegraphics[width=.18\columnwidth]{figures/supplementary/frankfurt00001_007285_cnn.png}
  }
  \subfigure{%
    \includegraphics[width=.18\columnwidth]{figures/supplementary/frankfurt00001_007285_ours.png}
  }\\[-2ex]


  \subfigure{%
    \includegraphics[width=.18\columnwidth]{figures/supplementary/frankfurt00001_059789_given.png}
  }
  \subfigure{%
    \includegraphics[width=.18\columnwidth]{figures/supplementary/frankfurt00001_059789_sp.png}
  }
  \subfigure{%
    \includegraphics[width=.18\columnwidth]{figures/supplementary/frankfurt00001_059789_gt.png}
  }
  \subfigure{%
    \includegraphics[width=.18\columnwidth]{figures/supplementary/frankfurt00001_059789_cnn.png}
  }
  \subfigure{%
    \includegraphics[width=.18\columnwidth]{figures/supplementary/frankfurt00001_059789_ours.png}
  }\\[-2ex]


  \subfigure{%
    \includegraphics[width=.18\columnwidth]{figures/supplementary/frankfurt00001_068208_given.png}
  }
  \subfigure{%
    \includegraphics[width=.18\columnwidth]{figures/supplementary/frankfurt00001_068208_sp.png}
  }
  \subfigure{%
    \includegraphics[width=.18\columnwidth]{figures/supplementary/frankfurt00001_068208_gt.png}
  }
  \subfigure{%
    \includegraphics[width=.18\columnwidth]{figures/supplementary/frankfurt00001_068208_cnn.png}
  }
  \subfigure{%
    \includegraphics[width=.18\columnwidth]{figures/supplementary/frankfurt00001_068208_ours.png}
  }\\[-2ex]

  \subfigure{%
    \includegraphics[width=.18\columnwidth]{figures/supplementary/frankfurt00001_082466_given.png}
  }
  \subfigure{%
    \includegraphics[width=.18\columnwidth]{figures/supplementary/frankfurt00001_082466_sp.png}
  }
  \subfigure{%
    \includegraphics[width=.18\columnwidth]{figures/supplementary/frankfurt00001_082466_gt.png}
  }
  \subfigure{%
    \includegraphics[width=.18\columnwidth]{figures/supplementary/frankfurt00001_082466_cnn.png}
  }
  \subfigure{%
    \includegraphics[width=.18\columnwidth]{figures/supplementary/frankfurt00001_082466_ours.png}
  }\\[-2ex]

  \subfigure{%
    \includegraphics[width=.18\columnwidth]{figures/supplementary/lindau00033_000019_given.png}
  }
  \subfigure{%
    \includegraphics[width=.18\columnwidth]{figures/supplementary/lindau00033_000019_sp.png}
  }
  \subfigure{%
    \includegraphics[width=.18\columnwidth]{figures/supplementary/lindau00033_000019_gt.png}
  }
  \subfigure{%
    \includegraphics[width=.18\columnwidth]{figures/supplementary/lindau00033_000019_cnn.png}
  }
  \subfigure{%
    \includegraphics[width=.18\columnwidth]{figures/supplementary/lindau00033_000019_ours.png}
  }\\[-2ex]

  \subfigure{%
    \includegraphics[width=.18\columnwidth]{figures/supplementary/lindau00052_000019_given.png}
  }
  \subfigure{%
    \includegraphics[width=.18\columnwidth]{figures/supplementary/lindau00052_000019_sp.png}
  }
  \subfigure{%
    \includegraphics[width=.18\columnwidth]{figures/supplementary/lindau00052_000019_gt.png}
  }
  \subfigure{%
    \includegraphics[width=.18\columnwidth]{figures/supplementary/lindau00052_000019_cnn.png}
  }
  \subfigure{%
    \includegraphics[width=.18\columnwidth]{figures/supplementary/lindau00052_000019_ours.png}
  }\\[-2ex]




  \subfigure{%
    \includegraphics[width=.18\columnwidth]{figures/supplementary/lindau00027_000019_given.png}
  }
  \subfigure{%
    \includegraphics[width=.18\columnwidth]{figures/supplementary/lindau00027_000019_sp.png}
  }
  \subfigure{%
    \includegraphics[width=.18\columnwidth]{figures/supplementary/lindau00027_000019_gt.png}
  }
  \subfigure{%
    \includegraphics[width=.18\columnwidth]{figures/supplementary/lindau00027_000019_cnn.png}
  }
  \subfigure{%
    \includegraphics[width=.18\columnwidth]{figures/supplementary/lindau00027_000019_ours.png}
  }\\[-2ex]



  \setcounter{subfigure}{0}
  \subfigure[\scriptsize Input]{%
    \includegraphics[width=.18\columnwidth]{figures/supplementary/lindau00029_000019_given.png}
  }
  \subfigure[\scriptsize Superpixels]{%
    \includegraphics[width=.18\columnwidth]{figures/supplementary/lindau00029_000019_sp.png}
  }
  \subfigure[\scriptsize GT]{%
    \includegraphics[width=.18\columnwidth]{figures/supplementary/lindau00029_000019_gt.png}
  }
  \subfigure[\scriptsize Deeplab]{%
    \includegraphics[width=.18\columnwidth]{figures/supplementary/lindau00029_000019_cnn.png}
  }
  \subfigure[\scriptsize Using BI]{%
    \includegraphics[width=.18\columnwidth]{figures/supplementary/lindau00029_000019_ours.png}
  }%\\[-2ex]

  \mycaption{Street Scene Segmentation}{Example results of street scene segmentation.
  (d)~depicts the DeepLab results, (e)~result obtained by adding bilateral inception (BI) modules (\bi{6}{2}+\bi{7}{6}) between \fc~layers.}
\label{fig:street_visuals-app}
\end{figure*}
 
		
	\end{appendix}
	
	
	
	%\end{document}  % This is where a 'short' article might terminate
	%\addtolength{\partopsep}{3mm}
	
	
	%\thispagestyle{empty}
	
	%\bibliography{sigproc}  % sigproc.bib is the name of the Bibliography in this case
	% You must have a proper ".bib" file
	%  and remember to run:
	% latex bibtex latex latex
	% to resolve all references
	%
	% ACM needs 'a single self-contained file'!
	%
	%APPENDICES are optional
	
\end{document}