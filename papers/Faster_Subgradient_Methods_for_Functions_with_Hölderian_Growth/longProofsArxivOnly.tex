\section{Proof of Theorems \ref{thmDimSum}, \ref{ThmLargeTheta}, and \ref{thmD1}}\label{secProofDecay}
\subsection{Preliminaries}\label{secProofDecay1}
In order to determine the convergence rate of the recursion (\ref{ABiggy}) derived in Prop. \ref{Prop_keyRecur} under generic nonsummable stepsizes, we need two Lemmas. We start with a result from \cite{PolyakIntro} which considers (\ref{ABiggy}) when $\theta<\frac{1}{2}$ without the nuisance term $\alpha_k^2 G^2$. 
\begin{lemma}
Suppose
$$
0\leq u_{k+1}\leq u_k - \gamma_k u_k^{1+q}
$$
for $k=0,1,\ldots$ where $\gamma_k\geq 0$ and $q>0$. Then 
$$
u_k
\leq
u_0\left(1+qu_0^q\sum_{i=0}^{k-1}\gamma_i\right)^{-\frac{1}{q}}.
$$\label{PolyakLemma2}
\end{lemma} 
\begin{proof}
\cite[Lemma 6 pp. 46]{PolyakIntro}.
\end{proof} 
We will also use the following estimates for the sum of stepsizes $\sum_{i=k_0}^k\alpha_i$. 
\begin{lemma}\label{sumSteps}
Let $k\geq k_0\geq 1$.
\begin{enumerate}
\item If $p\in(0,1)$
\begin{eqnarray*}
\sum_{i=k_0}^k i^{-p}\geq \frac{(k+1)^{1-p}-k_0^{1-p}}{1-p}.
\end{eqnarray*}
\item 
If $p=1$ 
\begin{eqnarray*}
\sum_{i=k_0}^ki^{-p}\geq \ln\frac{k+1}{k_0}.
\end{eqnarray*}
\end{enumerate} 	
\end{lemma}
\begin{proof}
A straightforward integral test.
\end{proof}


\subsection{Main Proof for Theorems \ref{thmDimSum} and \ref{ThmLargeTheta}}
Continuing with the main analysis, the goal is to derive convergence rates for a sequence $e_k$ satisfying (\ref{ABiggy}). To this end, let
\begin{eqnarray}\label{defI}
I=\{k: \alpha_k G^2\geq  c e_k^\gamma\}.
\end{eqnarray}
Recall the notation $\gamma=1/(2\theta)$. 
We will consider three types of iterates and bound the convergence rate in each case. First, for those iterates $k\in I$ it is easy to derive the convergence rate. Second, we will bound the rate for an iterate in $I^c$ when the previous iterate is in $I$. Finally we will consider $s$ consecutive iterates in $I^c$, for which we can use the inequality in (\ref{defI}) to simplify recursion (\ref{ABiggy}). Note that $s$ can be arbitrarily large. In particular when $I$ is finite there are an unbounded number of consecutive iterates in $I^c$. Together these three cases cover all possible iterates. 

First for, $k\in I$ and $\alpha_k>0$
\begin{eqnarray*}
	\alpha_k c e_k^\gamma \leq \alpha_k^2 G^2\implies e_k\leq \left(\frac{\alpha_k G^2}{c}\right)^{\frac{1}{\gamma}}.
\end{eqnarray*}
Thus the rate of $e_k$ is $O\left(\alpha_k^{\frac{1}{\gamma}}\right)$ for $k\in I$. In particular since $\alpha_k=\alpha_1k^{-p}$, then for $k\in I$ and $\alpha_1>0$
\begin{eqnarray}\label{firstCase}
e_k\leq \left(\frac{\alpha_1 G^2}{c}\right)^{2\theta} k^{-2 p\theta}.
\end{eqnarray}

Now assume $k\in I$ and $k+1\in I^c$. Then
\begin{eqnarray}\label{z}
	e_{k+1}\leq e_k + \alpha_k^2 G^2 \leq \left(\frac{\alpha_k G^2}{c}\right)^{\frac{1}{\gamma}}+ \alpha_k^2 G^2.
\end{eqnarray}
Now since $\frac{1}{\gamma}=2\theta\in(0,2)$, for $k\geq 1$
\begin{eqnarray*}
k^{-2p\theta}\geq k^{-2p}.
\end{eqnarray*}
Therefore (\ref{z}) implies that for $k\in I$, $k+1\in I^c$, and $k\geq 1$,
\begin{eqnarray}\label{case3}\label{case33}
e_{k+1}
&\leq&
C_1(k+1)^{-2p\theta}
\end{eqnarray}
where
\begin{eqnarray*}
C_1 = 
2^{2p\theta}
\left(
\left(\frac{\alpha_1 G^2}{c}\right)^{\frac{1}{\gamma}}
+
\alpha_1^2 G^2
\right).
\end{eqnarray*}

Next assume $k\in I$, $k+1\in I^c$, and $k+i\in I^c$ for $i=2,\ldots s$ for some $s\geq 2$. Then for $i=2,\ldots s$
\begin{eqnarray}\label{bigrecursion}
e_{k+i}< e_{k+i-1}-\alpha_kc e_{k+i-1}^\gamma.
\end{eqnarray}
To analyze the recursion (\ref{bigrecursion}) we consider $\theta<\frac{1}{2}$ and $\theta\geq \frac{1}{2}$ separately.

\noindent 
{\bf \underline{Case 1}: $\boldsymbol{\theta<\frac{1}{2}}$.} 

\noindent
 Now since $\gamma> 1$ we can apply Lemma \ref{PolyakLemma2} along with Lemma \ref{sumSteps} to (\ref{bigrecursion}) and derive for $i=2,\ldots,s$
\begin{eqnarray}
e_{k+i}
&\leq& 
e_{k+1}
\left[
1+\frac{1-2\theta}{2\theta}e_{k+1}^{\frac{1-2\theta}{2\theta}}\sum_{j=1}^{i-1}\alpha_{k+j}
\right]^{\frac{2\theta}{2\theta-1}}
\nonumber\\\label{flow}
&\leq&
e_{k+1}
\left[
1+\frac{\alpha_1(1-2\theta)}{2\theta(1-p)}e_{k+1}^{\frac{1-2\theta}{2\theta}}
\left(
(k+i)^{1-p}
-(k+1)^{1-p}
\right)
\right]^{\frac{2\theta}{2\theta-1}}.
\end{eqnarray}
Now consider the condition given in (\ref{cc}).
%\begin{eqnarray*}
%C_1
%\leq
%\left(\frac{2\theta(1-d)}{1-2\theta}\right)^{\frac{2\theta}{1-2\theta}}(k+1)
%^{\frac{2\theta(2d(1-\theta)-1)}{1-2\theta}}.
%\end{eqnarray*}
Note that since $p$ satisfies (\ref{dcond}), if (\ref{cc}) holds for $k=k_0$, it holds for all $k>k_0$. In particular if it holds for $k=0$, then it holds for all $k$. Continuing, 
if (\ref{cc}) holds then for all $k>k_0$
\begin{eqnarray}
	1-\frac{\alpha_1(1-2\theta)}{2\theta(1-p)}e_{k+1}^{\frac{1-2\theta}{2\theta}}(k+1)^{1-p}\geq 0\label{gg}
\end{eqnarray}
where we have used the fact that $k+1\in I^c$. Therefore since (\ref{gg}) holds we can simplify (\ref{flow}) to say that for $k\in I$ and $k+i\in I^c$ for $i=2,3,\ldots,s$, and $k>k_0$,
\begin{eqnarray}
e_{k+i}&\leq& e_{k+1}
\left[\frac{\alpha_1(1-2\theta)}{2\theta(1-p)}e_{k+1}^{\frac{1-2\theta}{2\theta}}
(k+i)^{1-p}
\right]^{\frac{2\theta}{2\theta-1}}
\nonumber\\
&\leq&
\left(\frac{\alpha_1(1-2\theta)}{2\theta(1-p)}\right)^{\frac{2\theta}{2\theta-1}}
(k+i)^{\frac{2\theta(1-p)}{2\theta-1}}
\label{late}.
\end{eqnarray}
\begin{comment}
Now since $k+1\in I^c$, 
\begin{eqnarray*}
e_{k+1}\geq 
\left(
\frac{\alpha_{k+1} G^2}{c}
\right)^{\frac{1}{\gamma}}
=
\left(
\frac{\alpha_1 G^2}{c}
\right)^{2\theta}
(k+1)^{-2d\theta}.
\end{eqnarray*}
Combining this with (\ref{case33}) and plugging into (\ref{late}) yields
\begin{eqnarray}
e_{k+i}&\leq& C_1(k+1)^{-2d\theta}
\left[\frac{1}{2}+C_2(k+1)^{-d(1-2\theta)}
(k+i)^{1-d}
\right]^{\frac{2\theta}{2\theta-1}}
\nonumber\\\nonumber
&=&
C_1
\left[\frac{1}{2}(k+1)^{d(1-2\theta)}+C_2
(k+i)^{1-d}
\right]^{\frac{2\theta}{2\theta-1}}
\\\label{NewCase}
&\leq&
C_1
\left[\frac{1}{2}+C_2
(k+i)^{1-d}
\right]^{\frac{2\theta}{2\theta-1}}
\end{eqnarray}
where
\begin{eqnarray*}
C_2=\left(
\frac{G^2}{c}
\right)^{2\theta}\frac{\alpha_1^{2(1-\theta)}(1-2\theta)}{2\theta(1-d)}
\end{eqnarray*}
\end{comment}
The final case to consider is when $i=1,2,\ldots,s$ are in $I^c$. In this case, the same bound (\ref{flow}) can be derived but with $e_{1}$ replacing $e_{k+1}$. Thus for $i=2,3,\ldots s$ in $I$
\begin{eqnarray}
e_{i}
\leq
e_1
\left[1+\frac{\alpha_1(1-2\theta)}{2\theta(1-p)}e_{1}^{\frac{1-2\theta}{2\theta}}
\left(
i^{1-p}-1
\right)
\right]^{\frac{2\theta}{2\theta-1}}\label{firstI}.
\end{eqnarray}
Thus if $\alpha_1$ is chosen to satisfy (\ref{cc2}) then
\begin{eqnarray}
e_{i}\label{prettyPleaseFinal}
\leq
\left(\frac{\alpha_1(1-2\theta)}{2\theta(1-p)}\right)^{\frac{2\theta}{2\theta-1}}
i^{\frac{2\theta(1-p)}{2\theta-1}}.
\end{eqnarray} 
Combining (\ref{firstCase}), (\ref{case3}), (\ref{late}), and (\ref{prettyPleaseFinal}) establishes (\ref{lowTheta1}) and concludes the proof of Theorem \ref{thmDimSum}.
\begin{comment}

On the other hand, since $k+i\in I^c$ for $i=1,2,\ldots, s$, $e_{k+i}\leq e_{k+1}$ for all such $i$. Therefore using (\ref{case3}) for $i=2,\ldots, s$, and letting 
$$
C_{\max} = \max\left\{
\frac{c^{\frac{2\theta}{2\theta-1}}}{1-d},
2\alpha_1^{2\theta}\left(\frac{G^2}{c}\right)^{-2d\theta}
\right\}
$$
\begin{eqnarray*}
e_{k+i}\leq C_{\max}\min
\left\{
k^{-2d\theta},
\left[
(k+i)^{1-d}-k^{1-d}
\right]^{\frac{2\theta}{2\theta-1}}
\right\}.
\end{eqnarray*}
In general we have for any $\rho\in[0,1]$
\begin{eqnarray}
&&\nonumber\min\{k^{-2d\theta},((k+i)^{1-d}-k^{1-d})^{-\frac{2\theta}{1-2\theta}}\}
\\
&=&
\min\left\{k^{-2d\theta},2^{-\frac{2\theta}{1-2\theta}}\left(\frac{1}{2}(k+i)^{1-d}+\frac{1}{2}(-k^{1-d})\right)^{-\frac{2\theta}{1-2\theta}}\right\}
\nonumber\\
&\leq&
\rho k^{-2d\theta}
\nonumber\\
&&\label{hols}
+
(1-\rho)2^{-\frac{2\theta}{1-2\theta}}
\left(
\frac{1}{2}(k+i)^{-\frac{2\theta(1-d)}{1-2\theta}}
-
\frac{1}{2}k^{-\frac{2\theta(1-d)}{1-2\theta}}
\right)
\end{eqnarray}
by convexity of $t^{-\frac{2\theta}{1-2\theta}}$ for $0<\theta<\frac{1}{2}$. Take 
$$
\rho\leq (1+2^{\frac{1}{2\theta-1}})^{-1}
$$
and since $d\geq\frac{1}{2(1-\theta)}$ then (\ref{hols}) implies 
\begin{eqnarray}
e_{k+i}&\leq& C_{\max}\min\{k^{-2d\theta},((k+i)^{1-d}-k^{1-d})^{-\frac{2\theta}{1-2\theta}}\}
\nonumber\\\label{\alpha_1case}
&\leq& 
C_{\max}(1-\rho)2^{\frac{1}{2\theta-1}}(k+i)^{-\frac{2\theta(1-d)}{1-2\theta}}
\end{eqnarray}
\end{comment}

\noindent 
{\bf\underline{Case 2}: $\boldsymbol{\theta\geq\frac{1}{2}}$}

\noindent
Next we consider the case where $\frac{1}{2}\leq\theta\leq 1$ which will finish the proof of Theorem \ref{ThmLargeTheta}. Before commencing we introduce the following Lemma which allows us to bound a decaying exponential by an appropriately scaled decaying polynomial of any degree.
\begin{lemma}\label{expToPoly}
Suppose $\delta>0$, then if $C_\delta\geq e^{-\delta}\delta^\delta$,
\begin{eqnarray}\label{ap}
\exp(-x)\leq C_\delta x^{-\delta}\quad\forall x>0.
\end{eqnarray}

\end{lemma}
\begin{proof}
Taking logs of both sides of (\ref{ap}) yields
\begin{eqnarray*}
-x\leq -\delta\ln x + \beta_\delta\quad\forall x>0
\end{eqnarray*}
where $\beta_\delta=\ln C_\delta$. Therefore
\begin{eqnarray*}
\beta_\delta
&\geq& \delta\ln x - x\quad\forall x>0
\end{eqnarray*}
which implies
\begin{eqnarray*}
\beta_\delta
&\geq& 
\max_{x>0}\{\delta\ln x - x\}.
\end{eqnarray*}
The right hand side is a smooth concave coercive maximization problem which therefore has a unique solution given by $x^*=\delta$. Hence
\begin{eqnarray*}
\beta_{\delta}\geq \delta\ln\delta - \delta
\end{eqnarray*}
which implies the Lemma. 
\end{proof}

Continuing, we consider $k\in I$, $k+1\in I^c$, and $k+i\in I^c$ for $i=2\ldots,s$ in the case where $\theta\geq\frac{1}{2}$, so $\gamma\leq 1$. Then since $k+i\in I^c$ for $i=2,\ldots s$,
\begin{eqnarray*}
0\leq\frac{e_{k+i-1}}{e_{k+1}}\leq 1\implies \left(\frac{e_{k+i-1}}{e_{k+1}}\right)^{\gamma}\geq \frac{e_{k+i-1}}{e_{k+1}}
\implies
e_{k+i-1}^\gamma \geq e_{k+1}^{\gamma-1}e_{k+i-1}.
\end{eqnarray*} 

\begin{comment}
Then by (\ref{case3}) if
\begin{eqnarray}
k>k_2 = 
\left\{
\begin{array}{c}
C^{2 d\theta}\\
0:\quad \theta=\frac{1}{2}
\end{array}
\right.
\end{eqnarray}
\end{comment}
Thus for $k\in I$, $k+1\in I^c$, and $k+i\in I^c$ for $i=2,\ldots, s$ for some $s\geq 2$
\begin{eqnarray}
e_{k+i}&\leq& e_{k+i-1}-\alpha_{k+i-1} c e_{k+i-1}^\gamma
\nonumber\\\label{innovate}
&\leq& 
e_{k+i-1}-\alpha_{k+i-1} e_{k+1}^{\gamma-1} c  e_{k+i-1}.
\end{eqnarray}
Now taking logs and using $\log(1-x)\leq-x$, 
\begin{eqnarray*}
\ln e_{k+i}
&\leq&
\ln e_{k+i-1} + \ln(1-e_{k+1}^{\gamma-1} c \alpha_{k+i-1} )
\\
&\leq&
\ln e_{k+i-1} - e_{k+1}^{\gamma-1} c\alpha_{k+i-1} .
\end{eqnarray*}
Now summing and using Lemma \ref{sumSteps} 
\begin{eqnarray*}
\ln e_{k+i}
&\leq&
\ln e_{k+1}-\alpha_1  e_{k+1}^{\gamma-1}c\sum_{i=k+1}^{k+i-1}i^{-p}
\\
&\leq&
\ln e_{k+1}-\frac{\alpha_1 e_{k+1}^{\gamma-1}c }{1-p}\left(
(k+i)^{1-p}-(k+1)^{1-p}
\right).
\end{eqnarray*}
This leads to
\begin{eqnarray}
e_{k+i}
&\leq&
e_{k+1}
\exp
\left\{-\frac{\alpha_1 e_{k+1}^{\gamma-1} c}{1-p}\left((k+i)^{1-p}-(k+1)^{1-p}\right)
\right\}\label{expForm}
\\\nonumber
&=&
\exp
\left\{-\frac{\alpha_1 e_{k+1}^{\gamma-1} c(k+i)^{1-p}}{1-p}\left(1-\left(\frac{k+1}{k+i}\right)^{1-p}\right)
\right\}.
\end{eqnarray}
We further consider two possible cases. If $i\geq k$, then 
\begin{eqnarray*}
\frac{k+1}{k+i}
\leq
\frac{k+1}{k+k}
=
\frac{1}{2}+\frac{1}{2k}
\end{eqnarray*}
therefore by concavity of $t^{1-p}$
\begin{eqnarray*}
\left(
\frac{k+1}{k+i}
\right)^{1-p}
\leq
2^{p-1}\left[1+\frac{1-p}{k}\right].
\end{eqnarray*}
Take 
$
k>3
$
so that
\begin{eqnarray*}
\frac{2^{p-1}(1-p)}{k}\leq \frac{1-2^{p-1}}{2}.
\end{eqnarray*}
Hence
\begin{eqnarray*}
1-\left(\frac{k+1}{k+i}\right)^{1-p}
\geq 
1-2^{p-1}\left[1+\frac{1-p}{k}\right]
\geq
1-2^{p-1}-\frac{2^{p-1}(1-p)}{k}
\geq 
 \frac{1-2^{p-1}}{2}.
\end{eqnarray*}
Hence if $3< k\leq i$ then 
\begin{eqnarray}
e_{k+i}&\leq& e_{k+1}\exp\left(
-\frac{(1-2^{p-1})\alpha_1  e_{k+1}^{\gamma-1} c}{2(1-p)}(k+i)^{1-p}
\right).
\end{eqnarray}
Now by Lemma \ref{expToPoly} for any $\delta_1>0$,
\begin{eqnarray*}
&&\exp\left\{-\frac{\alpha_1(1-2^{p-1})c e_{k+1}^{\gamma-1}}{2(1-p)}(k+i)^{1-p}\right\}
\\
&\leq&
\delta_1^{\delta_1}e^{-\delta_1} e_{k+1}^{1+\delta_1(1-\gamma)}
\left(
\frac{\alpha_1(1-2^{p-1})c}{2(1-p)}(k+i)^{1-p}
\right)^{-\delta_1}.
\end{eqnarray*}
Therefore using (\ref{case3}) for any $k\leq i$ and $k>3$ 
\begin{eqnarray}
e_{k+i}&\leq& 
\delta_1^{\delta_1}
C_1^{1+\delta_1(1-\gamma)}
\left(\frac{\alpha_1(1-2^{p-1}) c e }{2(1-p)}\right)^{-\delta_1}
(k+i)^{-\delta_1(1-p)}.
\label{precase}
\end{eqnarray}
Taking $\delta_1=\frac{2 p\theta}{1-p}$ and simplifying (\ref{precase}) yields
\begin{eqnarray}
\label{somecase}
e_{k+i}&\leq& 
C_1^{\frac{1+2p(\theta-1)}{1-p}}
\left(
\frac{\alpha_1(1-2^{p-1}) c e}{4p\theta}
\right)^{-\frac{2 p\theta}{1-p}}
(k+i)^{-2 p\theta}.
\end{eqnarray}



Next consider $k\geq i>1$. Now
\begin{eqnarray}
(k+i)^{1-p} - (k+1)^{1-p}
&=&
(k+i)^{1-p}
\left(
1-
\left(
\frac{k+1}{k+i}
\right)^{1-p}
\right)
\nonumber\\
&=&
(k+i)^{1-p}
\left(
1-
\left(
1-\frac{i-1}{k+i}
\right)^{1-p}
\right)
\nonumber\\
&\geq&
(k+i)^{1-p}
\left(
1-
\left(1-
\frac{i-1}{2k}
\right)^{1-p}
\right)
\nonumber\\
&\geq&
\frac{
(1-p)(k+i)^{1-p} 
(i-1)
}{2k}
\label{conc}\\\nonumber
&\geq&
\frac{1-p}{2}
k^{-p}
(i-1)
\end{eqnarray}
where in (\ref{conc}) we used the concavity of $t^{1-p}$.
\begin{comment}
We now use the assumption that $1-d=2^{-n}\triangleq \alpha$. So that $2^n \alpha = 1$. For example $d=1-\frac{1}{16}$. 
Then we use
\begin{eqnarray*}
(k+i)^\alpha - (k+1)^\alpha
&=&
\frac{(k+i)^{2\alpha} - (k+1)^{2\alpha}}{(k+i)^\alpha + (k+1)^\alpha}
\\
&\geq&
\frac{(k+i)^{2\alpha} - (k+1)^{2\alpha}}{2\cdot (2k)^{\alpha}}.
\end{eqnarray*}
\alpha_1pplying this recursively we get
\begin{eqnarray*}
(k+i)^\alpha - (k+1)^\alpha
&\geq&
\frac{i-1}
{2\cdot(2k)^\alpha\cdot 2(2k^{2\alpha})\cdots 2(2k)^{\frac{1}{2}}}
\\
&=&
\frac{i-1}
{2^n(2k)^{d}}.
\end{eqnarray*}
\end{comment}
Thus substituting this into (\ref{expForm}) implies for $k\geq i$
\begin{eqnarray*}
e_{k+i}
\leq 
e_{k+1}
\exp\left(
\frac{-\alpha_1  e_{k+1}^{\gamma-1} c(i-1)}
{2k^p}
\right).
\end{eqnarray*}
Therefore for all $\delta_2\geq 0$
it follows Lemma \ref{expToPoly} that
\begin{eqnarray}
e_{k+i}
&\leq&
e_{k+1}\exp\left(
\frac{-\alpha_1  e_{k+1}^{\gamma-1}c(i-1)}
{2k^{p}}
\right)
\nonumber\\
&\leq&
\delta_2^{\delta_2} e_{k+1}
\left(
\frac{\alpha_1  e_{k+1}^{\gamma-1} c(i-1)e}
{2k^{p}}
\right)^{-\delta_2}
\nonumber\\\label{step}
&\leq&
 C_1^{1+\delta_2(1-\gamma)}
\left(\frac{4\delta_2}{c \alpha_1 e}\right)^{\delta_2}  
k^{-2p\theta(1+\delta_2(1-\gamma))} k^{p\delta_2}i^{-\delta_2}
\end{eqnarray}
where we used $e_{k+1}\leq C_1 k^{-2p\theta}$ and $(i-1)^{-\delta_2}\leq 2^{\delta_2} i^{-\delta_2}$.
Now if we choose 
\begin{eqnarray}\label{optDelta2}
\delta_2=2\theta
\end{eqnarray}
then (\ref{step}) implies
\begin{eqnarray}\label{steps}
e_{k+i}\leq C_4 i^{-2\theta}
\end{eqnarray}
where
\begin{eqnarray}
C_4 = \left(\frac{8\theta C_1}{c \alpha_1 e}\right)^{2\theta}.  
\end{eqnarray}
Thus combining $e_{k+i}\leq e_{k+1}\leq C_1k^{-2p\theta}$ and (\ref{steps}) implies that for $i\leq k$
\begin{eqnarray*}
e_{k+i}\leq \max\{C_1,C_4\}\min\{k^{-2p\theta},i^{-2\theta}\}.
\end{eqnarray*}
Now since $-2\theta<-2p\theta$,
\begin{eqnarray*}
e_{k+i}\leq \max\{C_1,C_4\}\min\{k^{-2p\theta},i^{-2p\theta}\}\leq \frac{\max\{C_1,C_4\}}{\max\{k^{2p\theta},i^{2p\theta}\}}.
\end{eqnarray*}
If $2p\theta\geq 1$ then by convexity of $t^{2p\theta}$
\begin{eqnarray}
\max\{k^{2p\theta},i^{2p\theta}\}
\geq
\frac{1}{2}
\left(
k^{2p\theta}+i^{2p\theta}
\right)
\geq 
2^{-2p\theta}\left(k+i\right)^{2p\theta}.\label{aas}
\end{eqnarray}
On the other hand if $2p\theta<1$ then because $t^{2p\theta}$ is subadditive
\begin{eqnarray}\label{bees}
\max\{k^{2p\theta},i^{2p\theta}\}
\geq
\frac{1}{2}
\left(
k^{2p\theta}+i^{2p\theta}
\right)
\geq 
\frac{1}{2}\left(k+i\right)^{2p\theta}.
\end{eqnarray}
Combining (\ref{aas}) and (\ref{bees}) gives
\begin{eqnarray}\label{TheFinalCase}
e_{k+i}
\leq
4 \max\{C_1,C_4\}(k+i)^{-2p\theta}.
\end{eqnarray}
\begin{comment}
To conclude the analysis of the $\theta>\frac{1}{2}$ case we note that if $I$ is finite and $k_1$ is the largest element of $I$ and $k_1<k_2$ defined in (\ref{k2Def}), then there exists some $k_3\geq k_1$ such that for $k>k_3$ $e_k\leq 1$. Therefore for $k>k_3$, $e_k^\gamma\geq e_k$ and 
\begin{eqnarray*}
e_{k+1}&\leq& e_k-\alpha_k c e_k^\gamma
\\
&\leq&
e_k(1-\alpha_k c).
\end{eqnarray*}
Without repeating the analysis, this implies that for all $k>k_3$ and $\delta_3>0$
\begin{eqnarray}
e_{k}&\leq& e_{k_3}\exp\left(\frac{-c \alpha_1}{1-d}\left((k+1)^{1-d}-k_2^{1-d}\right)\right)
\nonumber\\
&\leq&
 \delta_3^{\delta_3}e^{-\delta_3}e_{k_3}
 \left(\frac{c \alpha_1}{1-d}\left((k+1)^{1-d}-k_2^{1-d}\right)\right)^{-\delta_3}
 \\
 &\leq&
  \delta_3^{\delta_3}e^{-\delta_3}e_{k_3}
  \left(\frac{c \alpha_1}{1-d}\left((k+1)^{1-d}-k_2^{1-d}\right)\right)^{-\delta_3}
 \label{caseCase}
\end{eqnarray}
Choosing $\delta_3=\frac{2d\theta}{1-d}$ finishes the proof.
\end{comment}
Finally we consider the case where the first $s$ iterates belong to $I^c$. Therefore, using (\ref{expForm}), for $i=1,2,\ldots, s$
\begin{eqnarray}
e_{i}
&\leq&
e_{1}
\exp
\left\{-\frac{\alpha_1 e_{1}^{\gamma-1} c}{1-p}\left(i^{1-p}-1\right)
\right\}.\label{firstI2}
\end{eqnarray}
Now since for $x\geq 1$, $x-1\geq \frac{x}{2}$, this implies that
\begin{eqnarray*}
e_{i}
&\leq&
e_{1}
\exp
\left\{-\frac{\alpha_1 e_{1}^{\gamma-1} c}{2(1-p)}i^{1-p}
\right\}.
\end{eqnarray*}
Using Lemma \ref{expToPoly} this implies that for any $\delta_3>0$
\begin{eqnarray}
e_{i}\label{cases}
&\leq&
e^{-\delta_3}\delta_3^{\delta_3}
e_{1}
\left(\frac{\alpha_1 e_{1}^{\gamma-1} c}{2(1-p)}i^{1-p}
\right)^{-\delta_3}.
\end{eqnarray}
and we will use $\delta_3=\frac{2p\theta}{1-p}$.

Combining (\ref{firstCase}), (\ref{case3}), (\ref{somecase}), (\ref{TheFinalCase}), and (\ref{cases}) yields the desired result (\ref{BigThetaResult}) and concludes the proof of Theorem \ref{ThmLargeTheta}.
 
\subsection{Proof of Theorem \ref{thmD1}}

The format of the proof is identical to Theorems \ref{thmDimSum} and \ref{ThmLargeTheta}. As before it is based on the set $I$ defined in (\ref{defI}) and we consider three types of iterates. First we bound the convergence rate for iterates in $I$, second for iterates in $I^c$ when the previous iterate is in $I$. And finally for $s$ consecutive iterates in $I^c$ where $s$ may be unbounded. 

If $k\in I$ then repeating (\ref{case3}) yields 
\begin{eqnarray}\label{newcase}
e_k\leq \frac{\alpha_1 G^2}{c} k^{-1}.
\end{eqnarray}
Similarly for $k\in I$ and $k+1\in I^c$, 
\begin{eqnarray}\label{newcase2}
e_{k+1}\leq \frac{2\alpha_1 G^2}{c} (k+1)^{-1}.
\end{eqnarray}
Finally for $k\in I$, $k+1\in I^c$, and $k+i\in I^c$, for $i=2,\ldots, s$, then repeating (\ref{innovate}) but with $\gamma=1$ this time,
\begin{eqnarray*}
e_{k+i}\leq e_{k+i-1}(1-c \alpha_{k+i-1}).
\end{eqnarray*}
Taking logs, using $\log(1-x)\leq -x$ and summing yields
\begin{eqnarray*}
	\log e_{k+i}
	&\leq& 
	\log e_{k+1} - c \alpha_1 \sum_{j=k+1}^{k+i-1} j^{-1}
	\\
	&\leq&
	\log e_{k+1}-c \alpha_1 \left(\log(k+i)-\log(k+1)\right)
\end{eqnarray*}
where we applied Lemma \ref{sumSteps} in the second inequality. This yields for all $k\in I$ and $k+i\in I^c$ for $i=2,3,\ldots, s$ for some $s\in \mathbb{N}$
\begin{eqnarray}
e_{k+i}&\leq& e_{k+1}\left(\frac{k+i}{k+1}\right)^{-c \alpha_1 }.
\label{need}
\end{eqnarray}
Using (\ref{newcase2}) yields 
\begin{eqnarray}
e_{k+i}&\leq&\frac{2 \alpha_1 G^2}{c} (k+1)^{-1}(k+1)^{c \alpha_1 }
(k+i)^{-c \alpha_1}
\nonumber\\
&\leq&
\frac{2 \alpha_1 G^2}{c}
(k+i)^{-c \alpha_1}
\label{thisCase}
\end{eqnarray}

Finally we consider the case where the initial iterates $i=1,2,\ldots,s$ are in $I^c$. Therefore repeating (\ref{need}) with $k=0$ gives
\begin{eqnarray}\label{fin}
e_{i}&\leq& e_{1} i^{-c \alpha_1 }
\end{eqnarray}

Combining (\ref{newcase}), (\ref{newcase2}), (\ref{thisCase}), and (\ref{fin}) yields (\ref{ThmD1Result}) and concludes the proof of Theorem \ref{thmD1}.  

\subsection{Proof of Proposition \ref{PropThta05}}\label{sec:PropProof}
As previously mentioned, this argument is a direct extension of \cite[Thm. 4]{karimi2016linear}. For $\theta=\frac{1}{2}$, (\ref{ABiggy}) reads as 
\begin{eqnarray*}
e_{k+1}\leq (1-2\alpha_k c)e_k+\alpha_k^2 G^2.
\end{eqnarray*}
We consider the choice $\alpha_k =\frac{2k+1}{2c(k+1)^2}$. Then
\begin{eqnarray*}
e_{k+1}
\leq
\left(
1-\frac{2k+1}{(k+1)^2}
\right)e_k+
\frac{G^2(2k+1)^2}{4 c^2(k+1)^4}.
\end{eqnarray*}
Multiplying both sides by $(k+1)^2$ yields
\begin{eqnarray*}
(k+1)^2e_{k+1}
&\leq&
k^2 e_k
+
\frac{G^2(2k+1)^2}{4 c^2(k+1)^2}
\\
&\leq&
k^2 e_k
+
\frac{G^2}{c^2}
\\
&\leq&
e_1+\frac{G^2}{c^2}k.
\end{eqnarray*}
Therefore
\begin{eqnarray*}
e_{k+1}
\leq
\frac{e_1}{(k+1)^2}
+
\frac{G^2}{c^2(k+1)}.
\end{eqnarray*}




 
 
{\bf Acknowledgments.} We thank Prof. Niao He for many illuminating and important discussions. 
