\begin{proof}
 Recall our notation $e_k=d(x_k,\calX_h)^2$ and let $\gamma=\frac{1}{2\theta}$. 
 Returning to the main recursion (\ref{ABiggy}) derived in Prop. \ref{Prop_keyRecur} and replacing the stepsize with a constant yields
 \begin{eqnarray}\label{fixedRecurs}
 0\leq e_{k+1}\leq e_k - 2\alpha c e_k^\gamma + \alpha^2 G^2,
 \end{eqnarray}
 where $\gamma\geq \frac{1}{2}$. 
  The key to understanding the behavior of this recursion is to write it as
 \begin{eqnarray}
 e_{k+1}-e_*\leq e_k - e_* - 2\alpha c (e_k^\gamma - e_*^\gamma)\label{one}
 \end{eqnarray}
 where $e_* = \left(\frac{\alpha G^2}{2c}\right)^{\frac{1}{\gamma}}$.
  We will show that $e_k - e_*$ must go to $0$ and derive the convergence rate.

 
 \noindent
 {\bf \underline{Boundedness:}}

 \noindent 
 We first prove (\ref{eqBounded}), which says that $e_k$ is bounded. Considering (\ref{one}) we see that since $\alpha>0$ and $c>0$, if $e_k\geq e_*$ then $e_{k+1}\leq e_k$. On the other hand, if $e_k\leq e_*$, then (\ref{fixedRecurs}) yields $e_{k+1}\leq e_k+\alpha^2 G^2\leq e_*+\alpha^2 G^2$. Therefore
 \begin{eqnarray*}
 e_{k+1}\leq\max\{e_k,e_*+\alpha^2 G^2\}\leq\max\{e_1,e_*+\alpha^2 G^2\}.
 \end{eqnarray*}
 
 \noindent
 {\bf\underline{Case 1}: $\boldsymbol{\theta\leq\frac{1}{2}}$.}
 
 \noindent
 For $\theta\leq\frac{1}{2}$, $\gamma\geq 1$ and by the convexity of $t^\gamma$,
 \begin{eqnarray*}
 e_k^\gamma - e_*^\gamma &\geq& \gamma e_*^{\gamma-1}(e_k-e_*).
 \end{eqnarray*}
 Using this in (\ref{one}) along with the facts that $\alpha>0$ and $c>0$ yields 
 \begin{eqnarray*}
 e_{k+1}-e_*\leq  (1-2\alpha c\gamma e_*^{\gamma-1})(e_k - e_*).
 \end{eqnarray*}
 Thus so long as 
 \begin{eqnarray}
 1-2\alpha c\gamma e_*^{\gamma-1}\geq 0,\label{two}
 \end{eqnarray}
 we have $q_1\geq 0$ where $q_1$ is defined in \eqref{fixed2} and
 \begin{align*}
 e_{k+1}-e^*\leq q_1(e_k - e^*)
 \leq
 q_1^k(e_1 - e^*)
% q_1\max\{e_k - e^*,0\}
% \leq q_1^k\max\{e_1 - e^*,0\}.
 \end{align*}
where the second inequality comes from recursing. This proves \eqref{fixed1}.
%Suppose $e_{k+1}-e^*\geq 0$, then 
%$$
%\max\{e_{k+1}-e^*,0\} = e_{k+1}-e^*\leq q_1^k\max\{e_{k+1}-e^*,0\}.
%$$ Otherwise, $e_{k+1}-e^*\leq 0$ and 
%$$
%\max\{e_{k+1}-e^*,0\}=0\leq q_1^k\max\{e_{k+1}-e^*,0\}.
%$$
%Together this implies \eqref{fixed1}.

Simplifying (\ref{two}) yields
 \begin{eqnarray*}
 2\alpha c \gamma e_*^{\gamma-1}
 &\leq& 1
 \\
 &\implies&
\alpha c \gamma  \left(\frac{\alpha G^2}{2c}\right)^{\frac{\gamma-1}{\gamma}}
 \leq 2^{-1}
 \\
 &\implies&
 \alpha
 \leq 
 \left(
 \frac{1}{\gamma}G^{\frac{2(1-\gamma)}{\gamma}}2^{-\frac{1}{\gamma}}
 c^{-\frac{1}{\gamma}}
 \right)^{\frac{\gamma}{2\gamma-1}}
 \end{eqnarray*}
 which  is equivalent to  (\ref{fixed0}).
 
 \noindent{\bf\underline{Case 2}: $\boldsymbol{\theta\geq\frac{1}{2}}$.}
 
 \noindent
 For $\theta\in[\frac{1}{2},1]$, $\gamma\in[\frac{1}{2},1]$, which implies by concavity
 \begin{eqnarray*}
 e_*^\gamma- e_k^\gamma
 \leq 
 \gamma e_k^{\gamma-1}(e_*-e_k).
 \end{eqnarray*}
 Therefore
 \begin{eqnarray*}
 e_k^\gamma-e_*^{\gamma}
 \geq
 \gamma  e_k^{\gamma-1}
 (e_k-e_*).
 \end{eqnarray*}
 Substituting this inequality into (\ref{one}) and again using $\alpha>0$ and $c>0$ yields
 \begin{eqnarray*}
 e_{k+1}-e_*&\leq& 
 e_k - e_* - 2\alpha c\gamma e_k^{\gamma-1} (e_k - e_*).
 \end{eqnarray*}
  Now if $e_*\leq e_k$, then since $e_k\leq D$,
 \begin{eqnarray*}\label{pas}
 e_{k+1}-e_*&\leq& 
 (1 - 2\alpha c\gamma D^{\gamma-1}) (e_k - e_*) = q_2(e_k - e_*).
 \end{eqnarray*}
 So long as 
 \begin{eqnarray*}
 1>1-2\alpha c\gamma  D^{\gamma-1}\geq 0
 \end{eqnarray*}
(which is implied by (\ref{linConv})), we have $q_2\in [0,1)$. On the other hand if $e_k\leq e_*$ then, using (\ref{fixedRecurs}), $e_{k+1}\leq e_*+ \alpha^2 G^2$. Thus for all $k\geq 1$
\begin{eqnarray*}
e_{k+1}-e_*\leq 
\max\left\{
q_2(e_k-e_*),\alpha^2 G^2
\right\}
.
\end{eqnarray*}
 Iterating this recursion and using the fact that $q_2\in[0,1)$ yields (\ref{ghh}).
 
\end{proof}