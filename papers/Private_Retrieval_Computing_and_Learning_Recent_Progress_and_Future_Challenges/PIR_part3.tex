
PIR holds particular significance as a point of convergence of complementary perspectives. 
It is well known that PIR shares intimate connections to prominent problems in theoretical computer science and cryptography, communication and information theory, and coding and signal processing. PIR protocols are often used as essential ingredients of oblivious transfer \cite{SymPIR}, instance hiding \cite{Hide, Hide_one, Hide_multiple},  multiparty computation \cite{Local_random}, secret sharing schemes \cite{Shamir, Beimel_Ishai_Kushilevitz_Orlov} and locally decodable codes \cite{YekhaninPhd}. Through the  topics of locally decodable, recoverable, repairable and correctable codes \cite{Gopalan_Huang_Simitci_Yekhanin}, PIR connects to  distributed data storage repair \cite{Dimakis_survey}, index coding \cite{Birk_Kol_Trans} and the entire umbrella of  network coding \cite{Ahlswede_Cai_etal} in general. PIR schemes are essentially interference alignment schemes \cite{Jafar_FnT} as the downloads comprise a mix of desired messages with undesired messages (interference). Efficient retrieval requires the alignment of interference dimensions across the downloads from different servers while keeping desired signals resolvable. It is not surprising then that interference alignment has been used implicitly in  PIR and index coding long before its applications in wireless networks \cite{Jafar_FnT}. Various equivalence results have been established between PIR and blind interference alignment (BIA) \cite{Sun_Jafar_PIR, Sun_Jafar_BIAPIR}; BIA and topological interference management (TIM) \cite{Jafar_TIM}; TIM and index coding \cite{Jafar_TIM}; index coding and locally repairable codes \cite{Shanmugam_Dimakis_LRC,Mazumdar_LRC}; locally repairable and locally decodable codes \cite{Gopalan_Huang_Simitci_Yekhanin}; and between locally decodable codes and PIR \cite{YekhaninPhd}. Add to this the equivalence between index coding and network coding  \cite{Rouayheb_Sprintson_Georghiades, Effros_Rouayheb_Langberg}, storage capacity and index coding \cite{Mazumdar}, index coding and hat guessing \cite{Riis_Hat}, or the application of asymptotic interference alignment schemes originally developed for wireless interference networks \cite{Cadambe_Jafar_int} to distributed storage exact repair \cite{Cadambe_Jafar_Maleki_Ramchandran_Suh}, and it becomes evident that discoveries in PIR have the potential for a ripple effect in their impact on a number of related problems.

% Within the scope of this article, the connection between PIR, secure distributed computing, and private federated learning is  exemplified by the idea of cross-subspace alignment (CSA) which extends to all three domains.  CSA codes originated in \cite{Jia_Sun_Jafar_XSTPIR} as a solution to XS-TPIR, i.e., the problem of $T$-private information retrieval from $N$ servers that store $K$ messages in an $X$-secure fashion. CSA codes then found applications in private secure coded computation \cite{Kim_Lee_PSCC, Chang_Tandon_PSDMM, Jia_Jafar_MDSXSTPIR}, and in particular secure distributed matrix multiplication (SDMM) \cite{Chang_Tandon_SDMMOS}. CSA codes were first applied to SDMM by Kakar et al. in \cite{Kakar_Ebadifar_Sezgin_CSA}, and subsequently applied to  secure distributed \emph{batch} matrix multiplication  (SDBMM) by Jia et al. in \cite{Jia_Jafar_SDBMM}. These works produced sharp capacity\footnote{Analogous to PIR, the capacity of SDBMM is defined as the supremum of the ratio of the number of bits of desired information (the desired matrix products), to the total number of bits downloaded  from the $N$ servers.} characterizations for various cases. For example, in \cite{Jia_Jafar_SDBMM} the capacity for $X$-secure distributed computation by $N$ servers of a batch of outer products of two vectors is shown to be $(1-X/N)^+$,  the capacity for computing the inner product of two  length-$K$ vectors is  $\frac{1}{K}(1-\frac{X}{N})^+$ when $N\leq 2X$, and for long vectors $(K\rightarrow\infty)$ the capacity of computing inner products is shown to be $(1-2X/N)^+$.  While LCC (Lagrange Coded Computing) \cite{pmlr-v89-yu19b} codes and CSA codes originated independently in seemingly unrelated contexts of distributed secure computing and secure PIR around the same time,   their  connection is evident from the observation that for secure multiparty/distributed batch matrix multiplications, CSA codes yield LCC codes as a special case\cite{Jia_Jafar_CSA_MM, Chen_Jia_Wang_Jafar}. The generalization inherent in CSA codes is beneficial primarily in download-limited settings, where CSA codes are able to strictly outperform LCC codes. Finally, in the domain of federated learning, CSA codes were applied in \cite{Jia_Jafar_XSTPFSL} to find a solution to the problem of $X$-secure $T$-private federated submodel learning. Fundamentally, this is a problem of privately reading from and privately writing to a database comprised of $K$ files (messages/submodels) that are stored across $N$ distributed servers in an $X$-secure fashion. The CSA read-write scheme of \cite{Jia_Jafar_XSTPFSL}  is able to fully update the storage at all $N$ servers after each write operation even if some of the servers (up to a specified threshold value) are inaccessible, and  achieves a synergistic gain from the joint design of private-read and private-write operations. Intuitively, the connection between these problems arises because the operation required at each server for many (but not all) PIR (and private write) schemes can be interpreted as a \emph{matrix multiplication} between a threshold-$T$ secret-shared query vector/matrix (polynomial encoded for $T$-privacy) and a threshold-$X$ secret-shared data vector/matrix (polynomial encoded for $X$-security), which produces various desired and undesired products. CSA codes are characterized by a Cauchy-Vandermonde structure that facilitates interference alignment of undesired products along the Vandermonde terms, while the desired products remain separable along the Cauchy terms. This alignment structure allows efficient downloads by reducing interference dimensions. Therefore, to the extent that a multiplication of polynomial encoded matrices is involved, and download efficiency is of concern, the same Cauchy-Vandermonde alignment structure facilitated by CSA codes turns out to be useful across these problems. It is also noteworthy that applications of CSA codes generalize naturally beyond matrix products, to tensor products, as seen in Double Blind Private Information Retrieval ($M$-way blind PIR in general) \cite{Lu_Jia_Jafar_DBTPIR}. 


The remainder of this article explores in greater depth the topics of secure distributed computing and private federated learning.