\section{Multi-Goal Audio-Visual Navigation}

\subsection{Task Definitions}


本論文で提案するMulti-Goal Audio-Visual Navigationとは、一つのエピソードで複数のゴールへナビゲーションを行う視聴覚ナビゲーションである(図\ref{fig:multi_goal_av_nav})。
したがって、エージェントは複数の音が重なった音を観測し、それぞれの音源に対して定位しなければならない。
また、聴覚の情報を用いること以外の、マルチオブジェクトナビゲーションとの大きな違いとして、各ゴールにナビゲーションを行う順番が指定されていないことが挙げられる。
よって、どのような順番で各音源へナビゲーションを行うと効率的であるのかを、エージェント自身が考えなければならない。
以下では、マルチゴール視聴覚ナビゲーションについてより詳細な説明をする。

形式的には、一つのエピソードは、組$\{E, \bm{p}_s, \theta_s, \bm{p}_{g_1}, ..., \bm{p}_{g_n}, S_{g_1}, ..., S_{g_n}\}$で定義される。
ここで、$E$はエピソードで用いるシーン環境を表し、$\bm{p}_s \in \mathbb{R}^3$と$\theta_s \in [0, 2\pi]$はそれぞれ、エージェントの開始時の位置と向いている方向を表す。
また、$n \in \mathbb{N}$はゴールの数を表しており、各$i \in \{1, ..., n\}$に対し、$\bm{p}_{g_i} \in \mathbb{R}^3$と$S_{g_i} \in SoundCategory$はそれぞれ、ゴール$g_i$の位置と音源のカテゴリを表している。
ここで、$SoundCategory$は音のカテゴリの集合を表し、本研究では合計$91$種類の音のカテゴリを用いている。
以上より、マルチゴール視聴覚ナビゲーションとは、環境$E$において、スタート$(\bm{p}_s, \theta_s)$から出発し、$S_{g_1}, ..., S_{g_n}$の重なった音を聞きながら、ゴール位置$\bm{p}_{g_1}, ..., \bm{p}_{g_n}$を推定し、到達していくことであるということができる。

今回、マルチゴール視聴覚ナビゲーションでは、AudioGoal~\cite{chen2020soundspaces}とよばれるゴール形式を採用している。
AudioGoalとは、ゴールは音源であり、周期的に発生するその音の情報から位置を推測しなければならないというものである。
また、ゴールは視覚的には示されていないため、聴覚の情報のみを頼りにゴール位置を推定する必要がある。

また、エージェントがゴールに到達すると、そのゴールの音源はそれ以降音を発しない。
つまり、エージェントがゴール$g_i$に到達すると、それ以降エージェントは$S_{g_i}$の音を観測しなくなる。
よって、現在エージェントが観測している音は、まだ到達していない音源から発せられていることになる。
したがって、エージェントは現在観測している音が、すでに到達した音源から発せられているものであるのか、ないしはまだ到達していない音源から発せられているものであるのかを判定する必要はない。



\subsection{Action Space}

マルチゴール視聴覚ナビゲーションにおける、エージェントの行動空間は
\[ 
\mathcal{A} = \{MoveForward, TurnLeft, TurnRight, Found\} \label{eq:action_space}
\]
である。
$MoveForward$を選択すると、エージェントは環境を$0.25\ \mathrm{m}$進む。
$TurnLeft$または$TurnRight$を選択すると、それぞれ左または右に$10^\circ$回転する。
また、ゴールである音源の半径$1\ \mathrm{m}$未満で$Found$を選択することができれば、その音源に到達したことを意味する。
以上の具体的な数値は、SoundSpaces 2.0のデフォルトの設定である。

エピソードが終了するのは、以下の三つの条件のうち、いずれかが満たされたときである。
一つ目は、全てのゴールに到達することである。
二つ目は、ゴールから半径1以上離れた場所で$Found$を選択することである。
三つ目は、合計の行動回数が2,500を超えることである。
この行動回数の上限は先行研究\cite{wani2020multion}を参考にして設定した。

\subsection{Observation Space}

マルチゴール視聴覚ナビゲーションでエージェントが観測することができるのは、視覚の情報と聴覚の情報である。
視覚の情報としては、一人称視点の$128 \times 128$のRGBD画像を用いている。
また、聴覚の情報としては、$257 \times 69$のスペクトログラムを用いている。
音の方式は、先行研究に倣って2チャネルのバイノーラル音を用いている。

本研究におけるスペクトログラム作成の手順は、以下のとおりである。
まず、サンプリング周波数44,100$\ \mathrm{Hz}$でサンプリングした$0.25$秒分の離散的な音の時系列のデータを獲得する。
次に、その時系列データに対して短時間フーリエ変換を行い、各時間の各周波数の成分の振幅を求める。
ここで、窓関数(Window Function)はハニング窓(Hanning Window)、フレームサイズは$512$、フレームシフトは$160$である。
最後に、各値に対して$1$を足して対数をとった結果を各成分の強さとして、スペクトログラムを作成する。


\subsection{Metrics}


本研究では、$SUCCESS$、$SPL$、$PROGRESS$、$PPL$の四つの評価指標を用いた。
以下では、これらの詳細を説明する。

$SUCCESS$は、テストしたエピソードの数を$N$とすると、以下で表される。
\[
SUCCESS = \frac{1}{N} \sum_{i = 1}^N S_i
\]
ここで、$S_i \in \{0, 1\}$はエピソード$i \in \{1, ..., N\}$において全てのゴールに到達できたかどうかのバイナリーの値である。
つまり、$i$番目のエピソードにおいて、エージェントが全てのゴールに到達できたら$S_i = 1$であり、一つでも到達できなければ$S_i = 0$となる。

$SPL$は以下で表される\cite{anderson2018evaluation}。
\[
SPL = \frac{1}{N} \sum_{i = 1}^N S_i \frac{l_i}{\max(l^A_i, l_i)}
\]
ここで、$l^A_i \in \mathbb{R}$はエージェントが実際に進んだ経路の長さ、$l_i \in \mathbb{R}$は最短経路の長さを表している。
つまり、たとえエージェントが全てのゴールに到達できても、最短経路に近い経路で進まなければ、$SPL$は高い値にはならない。

$PROGRESS$は以下で表される\cite{wani2020multion}。
\[
PROGRESS = \frac{1}{N} \sum_{i = 1}^N \frac{n^A_i}{n}
\]
ここで、$n \in \mathbb{N}$はゴールの数、$n^A_i \in \mathbb{N}$はエージェントがエピソード$i$で到達したゴールの数を表している。
つまり、$SUCCESS$とは異なり、全てのゴールに到達しなくても、一つのゴールにさえ到達していれば値が$0$になることはない。

$PPL$は以下で表される\cite{wani2020multion}。
\[
PPL = \frac{1}{N} \sum_{i = 1}^N \frac{n^A_i}{n} \frac{l^\mathrm{MG}_i}{\max(l^A_i, l^\mathrm{MG}_i)}
\]
ここで、$l^\mathrm{MG}_i \in \mathbb{R}$は、スタート地点からエージェントが到達した全てのゴール地点を通るための最短経路の長さを表す。
つまり$PPL$は、より多くのゴールへ、より最短経路に近い経路で到達することで値が高くなる。

ただし、Waniら\cite{wani2020multion}では到達するべき順番が決められている一方で、本研究では到達するべき順番が決められていない。
そのため、$l^\mathrm{MG}_i$の計算方法は異なる。
仮に、エピソード$i$においてエージェントが$g_1, ..., g_{n^A_i}$の順でゴールに到達したとし、スタートの位置が$\bm{p}_s$であったとする。
また、点$\bm{p}, \bm{q}$間の最短経路の長さは$d_{\mathrm{min}}(\bm{p}, \bm{q})$で表すことができるとする。
このとき、Waniら\cite{wani2020multion}では$l^\mathrm{MG}_i = d_{\mathrm{min}}(\bm{p}_{s}, \bm{p}_{g_1}) + \sum_{j=2}^{n^A_i} d_{\mathrm{min}}(\bm{p}_{g_{j-1}}, \bm{p}_{g_j})$と表される。
しかし、本研究では到達するべき順番が決められていないため、この計算によって正しく$l^\mathrm{MG}_i$が計算できるとは限らない。
代わりに、以下のようにして計算する必要がある。
\[
l^\mathrm{MG}_i = \min_{\sigma \in T_{n^A_i}} \left\{ d_{\mathrm{min}}(\bm{p}_{s}, \bm{p}_{g_{\sigma(1)}}) +  \sum_{j=2}^{n^A_i} d_{\mathrm{min}}(\bm{p}_{g_{\sigma(j-1)}}, \bm{p}_{g_{\sigma(j)}}) \right\}
\]
ここで、$T_{n^A_i}$とは、$n^A_i$個の数字$1, 2, ..., n^A_i$の置換全体の集合である。