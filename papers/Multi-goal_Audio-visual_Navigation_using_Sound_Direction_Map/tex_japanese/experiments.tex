\section{Experiments}

\subsection{Implementation Details}

\subsubsection{シミュレーション}
エージェントの学習およびテストは、SoundSpaces~\cite{chen2020soundspaces, chen2022soundspaces}というシミュレーター上で、Replica~\cite{straub2019replica}というシーンデータセットを用いて行った。
本研究で用いたSoundSpaces 2.0~\cite{chen2022soundspaces}は、視覚レンダリングを行うシミュレーターHabitat-Sim~\cite{savva2019habitat}に、音響伝搬エンジンRLR-Audio-Propagationを統合することで、Habitat-Simを拡張したシミュレーターである。
また、Replicaは18種類のアパート・オフィス・部屋・ホテルのシーンで構成されるシーンデータセットである。

学習・評価・テスト用のシーンは先行研究\cite{chen2020soundspaces}と同じように分けている。
したがって、テスト時には学習時には用いていないシーンを用いている。
また、今回は評価用のデータセットは用いておらず、一番最後のパラメータ更新によって得られたパラメータによってテストを行なった。
ここで、テストの試行回数は全ての実験において1,000である。



\subsubsection{エピソード生成}

本研究では、簡単すぎるエピソードと難しすぎるエピソードを排除するために、$\boldsymbol{p}_s, \theta_s, \boldsymbol{p}_{g_1}, ..., \boldsymbol{p}_{g_n}$の生成時にいくつか制約をかけている。

まず、簡単すぎるエピソードを排除するために、各地点間の距離は$1\ \mathrm{m}$以上離れるようにし、かつ測地線距離に対するユークリッド距離の比率が1.1より大きくなるようにした。
二つ目の制約の理由は、ほとんど直線のみでゴールに到達できる場合を排除するためである。
ただし、room 2とoffice 1はこれらの制約を満たすことが難しかったため、各地点間の距離は$0.6\ \mathrm{m}$以上離れ、かつ測地線距離に対するユークリッド距離の比率が1.001より大きくなれば良いことにした。

また、簡単すぎるエピソードを排除するために、各地点間の高さは$0.3\ \mathrm{m}$未満になるようにし、かつapartment 0において各地点間の距離が長くなりにくくなるようにした。
具体的には、二点間の距離が$10\ \mathrm{m}$以上なら確率$1.0$、$6\ \mathrm{m}$以上なら確率$0.7$、$5\ \mathrm{m}$以上なら確率$0.6$、$4\ \mathrm{m}$以上なら確率$0.5$、$3\ \mathrm{m}$以上なら確率$0.4$、$3\ \mathrm{m}$未満なら確率$0$で、却下されるというものである。
この制約の理由は、この制約をかけなければapartment 0における性能だけが著しく低くなり、学習曲線が不安定になることがわかったためである。
この問題の本質的な解決には、カリキュラムラーニングの導入が必要であると考えている。


\subsubsection{音源}
特に断りのない限り、先行研究\cite{chen2020soundspaces}にならって、ゴールの音源は訓練には73種類、テストには18種類用いている。
ここで、訓練とテストに重複するような音源はない。
したがって、テストでは訓練時には聞いたことのない音に対する汎化性能を評価していることになる。

各音源は全て、サンプリング周波数44,100$\ \mathrm{Hz}$でサンプリングされた$1$秒分の音データである。
特に断りのない限り、各エピソードでは、音データは$0$秒から$1$秒の間のランダムな時間から再生が開始されるようにしている。

\subsection{ゴール数による比較}

まず、ベースラインとなる手法で、ゴール数のみを変えて実験を行った。
この実験の目的は、ゴール数による難易度の違いを調査することである。
ゴール数$n$は、$n=1,2,3$の三通りで行った。
テスト結果は、表\ref{tab:n_goal_results}のとおりである。


\begin{table}[tb]
    \setlength{\tabcolsep}{4pt}
    \centering
    \caption{
        ゴール数による比較。
        $n$はゴール数を表している。
        ここで、ゴール数が$n$であるとは、訓練とテストの全てのエピソードにおいて、ゴールが$n$個存在しているということを表している
    }
    \label{tab:n_goal_results}
    \begin{tabular}{@{}cccccc@{}}
    \toprule
        手法 & $n$ & $SUCCESS$ & $SPL$ & $PROGRESS$ & $PPL$  \\ \midrule
        ランダム & 1 & \textbf{0.432} & \textbf{0.141} & \textbf{0.432} & \textbf{0.141} \\
        & 2 & 0.167 & 0.048 & 0.377 & 0.070 \\
        & 3 & 0.053 & 0.017 & 0.317 & 0.055 \\ \midrule
        AV-Nav~\cite{chen2020soundspaces} & 1 & \textbf{0.503} & \textbf{0.323} & \textbf{0.503} & \textbf{0.323} \\
        & 2 & 0.179 & 0.119 & 0.229 & 0.142 \\
        & 3 & 0.107 & 0.071 & 0.292 & 0.160 \\ \midrule
        SAVi~\cite{chen2021semantic} & 1 & \textbf{0.647} & \textbf{0.336} & \textbf{0.647} & \textbf{0.336} \\
        & 2 & 0.032 & 0.022 & 0.184 & 0.108 \\
        & 3 & 0.013 & 0.010 & 0.162 & 0.108 \\ \bottomrule
        \end{tabular}
\end{table}


ゴール数を増加させることで、精度が極端に下がる傾向があることがわかった。
$SUCCESS$が大きく下がるのは二つの理由があると考えている。
一つ目の理由は$SUCCESS$は指数関数的に劣化するためである。
これは、スタートから一つのゴールまでに到達する確率が$p$だとすれば、$n$個のゴールに到達できる確率は$p^n$になるからである。
二つ目の理由は、音源分離の必要性が出てくるからであると考えているが、より詳細な議論は次節で行う。
また、$PROGRESS$が下がるのは、エピソードの終了条件が原因であると考えている。
今回の設定では、途中のゴールで失敗してしまえば、そのエピソードは終了する。
したがって、ゴールが複数残っている場合、一つのゴールの失敗は複数のゴールの失敗となる。
これにより、ゴールへの到達率が下がり、$PROGRESS$が下がったと考えられる。


\subsection{複数の音源ゴールがある場合の難点調査}


\subsubsection{大きい音と小さい音}
  

音の大きさによって、つまり音量によって難易度が変化するのかを調査した。
ここでは、小さい音と大きい音へ到達率の違いを比較している。
ただし、これらの音は全て訓練では用いていない音から選択されている。
また、小さい音の集合の音の鳴る長さの平均と大きい音の集合の音の鳴る長さの平均は近くなるように、これらの音は選択されている。

\begin{table}[tb]
    \caption{
        小さい音(top)と大きい音(bottom)への到達率の比較。
        $n$-smallとは、小さい音の集合から$n$個選択されたものがゴールになることを表す。
        また、$n$-bigとは大きい音の集合から$n$個選択されたものがゴールになることを表す。
        ただし、学習時には訓練用の73種類全ての音源を用いている。
    }
    \label{tab:big_and_small}
    \centering
    %% \begin{tabular}{@{}ccc@{}}
    %% \toprule
    %%     & ゴール形式 & 到達率 \\ \midrule
    %%     小さい音 & 1-small & 0.441 \\
    %%     & 2-small & 0.232 \\
    %%     & 1-small-1-big & 0.187 \\ \midrule
    %%     大きい音 & 1-big & 0.449 \\
    %%     & 2-big & 0.209 \\
    %%     & 1-smal-1-big & 0.253 \\ \bottomrule
    %% \end{tabular}
    \begin{tabular}{@{}ccccccc@{}}
      \toprule
      & \multicolumn{2}{c}{1-goal task} & \multicolumn{3}{c}{2-goal task} \\
      \cmidrule(lr){2-3} \cmidrule(lr){4-6} 
      & 1-small & 1-big & 2-small & 2-big & 1-small-1-big \\
      \midrule
      small & 0.441 & N/A & 0.232 & N/A & 0.187 \\
      big & N/A & 0.449 & N/A & 0.209 & 0.253 \\
      \bottomrule
    \end{tabular}
\end{table}

表\ref{tab:big_and_small}はその結果である。
この結果から、大きい音も小さい音も、同時に大きい音が鳴ると精度が下がることがわかった。
これは、大きい音は遠くにいても聞こえるため、ノイズになりやすいからであると考えている。


\subsubsection{長い音と短い音}

音の鳴る時間の長さによって、難易度が変化するのかを調査した。
ここでは、短い音と長い音へ到達率の違いを比較している。
ただし、これらの音は全て訓練では用いていない音から選択されている。
また、短い音の集合の最大音量の平均と、長い音の集合の最大音量の平均は近くなるようにこれらの音は選択されている。

\begin{table}[tb]
    \centering
    \caption[短い音と長い音への到達率]
    {
        短い音(top)と長い音(bottom)への到達率の比較。
        $n$-shortとは、短い音の集合から$n$個選択されたものがゴールになることを表す。
        また、$n$-longとは長い音の集合から$n$個選択されたものがゴールになることを表す。
        ただし、学習時には訓練用の73種類全ての音源を用いている。
    }
    \label{tab:long_and_short}
    %% \begin{tabular}{@{}ccc@{}}
    %% \toprule
    %%     & ゴール形式 & 到達率 \\ \midrule
    %%     短い音 & 1-short & 0.235 \\
    %%     & 2-short & 0.160 \\
    %%     & 1-short-1-long & 0.141 \\ \midrule
    %%     長い音 & 1-long & 0.632 \\
    %%     & 2-long & 0.255 \\
    %%     & 1-short-1-long & 0.262 \\ \bottomrule
    %% \end{tabular}
    \begin{tabular}{@{}ccccccc@{}}
      \toprule
      & \multicolumn{2}{c}{1-goal task} & \multicolumn{3}{c}{2-goal task} \\
      \cmidrule(lr){2-3} \cmidrule(lr){4-6} 
      & 1-short & 1-long & 2-short & 2-long & 1-short-1-long \\
      \midrule
      short & 0.235 & N/A & 0.160 & N/A & 0.141 \\
      long & N/A & 0.632 & N/A & 0.255 & 0.262 \\
      \bottomrule
    \end{tabular}
\end{table}

表\ref{tab:long_and_short}はその結果である。
この結果から、長い音も短い音も、同時に長い音が鳴ると精度が下がることがわかった。
これは、長い音は常になり続けているため、もう一方の音のノイズになりやすいからであると考えている。



\subsubsection{同一音と不同音}

複数の音源がゴールとなる場合に、それらの音の種類が同じであるのか、それとも異なるのかによって、難易度が変化するのかを調査した。
ここでは、学習時には重複も許してランダムに二つの種類の音源を用いており、テスト時には、同じ二つの種類・異なる二つの種類・重複も許してランダムに二つの種類を用いる三つの状況で行った。
結果は表\ref{tab:same_and_different_sound}のようになった。

\begin{table}[tb]
    \centering
    \caption{
        同じ二つの音が鳴る場合と異なる二つの音が鳴る場合の比較。
        ゴールは二つあり、手法はAV-Nav\cite{chen2020soundspaces}である。
    }
    \label{tab:same_and_different_sound}
    \begin{tabular}{@{}ccccc@{}}
    \toprule
        & $SUCCESS$ & $SPL$ & $PROGRESS$ & $PPL$  \\ \midrule
        同じ音 & \textbf{0.193} & \textbf{0.131} & \textbf{0.244} & \textbf{0.155} \\
        異なる音 & 0.181 & 0.120 & 0.232 & 0.145 \\
        ランダム & 0.179 & 0.119 & 0.229 & 0.142 \\ \bottomrule
        \end{tabular}
\end{table}

異なる音が鳴っている方が、精度が低くなることがわかった。
これは、大きい音と小さい音が鳴っているときは小さい音への、長い音と短い音が鳴っているときは短い音への到達率が下がっているように、異なる性質の音が鳴ることで、一方の音への到達率が下がるからではないかと考えている。
異なっているときは一方がかき消されることが多くなり、到達率が下がるのではないかと考えている。
これらの結果は、音源分離の重要性を示している。




\subsubsection{鳴るタイミング}

複数の音の鳴るタイミングによって、難しさが変化するのかを調査した。
結果は、表\ref{tab:sound_timing}のようになった。

\begin{table}[tb]
    \centering
    \caption{
        二つの音が鳴るタイミングによる比較。
        ゴールは二つあり、手法はAV-Nav\cite{chen2020soundspaces}である。
        この実験では、音源は電話の音のみが用いられている。
        また、学習はランダムの設定で行なっている。
    }
    \label{tab:sound_timing}
    \begin{tabular}{@{}ccccc@{}}
    \toprule
        & $SUCCESS$ & $SPL$ & $PROGRESS$ & $PPL$ \\ \midrule
        重なる & \textbf{0.792} & \textbf{0.430} & \textbf{0.868} & \textbf{0.444} \\
        重ならない & 0.735 & 0.410 & 0.811 & 0.424 \\
        ランダム & 0.736 & 0.413 & 0.820 & 0.440 \\ \bottomrule
    \end{tabular}
\end{table}

同じ音が二つ鳴っている場合、音に重複がない方が精度が低くなるという結果を得た。
これは、交互に同じ音が鳴ると音源の定位が難しくなるからであると考えている。
エージェントには音源の数は与えられていないため、同じ音が交互に別の場所で鳴ると、一つの音が行き来していると判断してしまう可能性がある。
このことから、動的複数音源定位の履歴を表現する手法が重要であると考えられる。



\subsection{Sound Direction Map}

\subsubsection{定量評価}

SDMの有用性を示すために、ベースラインにおいてSDMを用いた場合と用いなかった場合の性能を比較した。
表\ref{tab:dm_results}はその結果である。

\begin{table}[tb]
    \setlength{\tabcolsep}{3pt}
    \centering
    \caption{
        SDMの定量評価。
        $n$はゴール数を表す。
    }
    \label{tab:dm_results}
    \begin{tabular}{@{}llcccc@{}}
    \toprule
        $n$~ & 手法 & $SUCCESS$ & $SPL$ & $PROGRESS$ & $PPL$ \\ \midrule
        1 & AV-Nav~\cite{chen2020soundspaces} & 0.503 & 0.323 & 0.503 & 0.323  \\
        & SAVi~\cite{chen2021semantic} & 0.647 & 0.336 & 0.647 & 0.336 \\
        & AV-Nav~\cite{chen2020soundspaces} w/ SDM & 0.610 & 0.354 & 0.610 & 0.354 \\
        & SAVi~\cite{chen2021semantic} w/ SDM & \textbf{0.874} & \textbf{0.574} & \textbf{0.874} & \textbf{0.574} \\ \midrule
        2 & AV-Nav~\cite{chen2020soundspaces} & 0.179 & 0.119 & 0.229 & 0.142  \\
        & SAVi~\cite{chen2021semantic} & 0.032 & 0.022 & 0.184 & 0.108 \\
        & AV-Nav~\cite{chen2020soundspaces} w/ SDM & \textbf{0.332} & \textbf{0.172} & \textbf{0.506} & \textbf{0.232} \\
        & SAVi~\cite{chen2021semantic} w/ SDM & 0.183 & 0.109 & 0.400 & 0.206 \\ \midrule
        3 & AV-Nav~\cite{chen2020soundspaces} & 0.107 & 0.071 & 0.292 & 0.160 \\
        & SAVi~\cite{chen2021semantic} & 0.013 & 0.010 & 0.162 & 0.108 \\
        & AV-Nav~\cite{chen2020soundspaces} w/ SDM & 0.174 & 0.101 & 0.368 & 0.186 \\
        & SAVi~\cite{chen2021semantic} w/ SDM & \textbf{0.335} & \textbf{0.221} & \textbf{0.571} & \textbf{0.336} \\ \bottomrule
    \end{tabular}
\end{table}

全てのゴール数と全てのベースラインにおいて、SDMを用いることで性能を向上させることができた。
さらに、SDMはゴール数増加による劣化を抑える傾向があることもわかった。
このような結果を得た理由は、記憶を有効に活用し、複数の音源についての定位がより正確にできるようになったからであると考えている。

\subsubsection{定性評価}

\begin{figure*}[t]
    \begin{center}
        \centering
        \includegraphics[scale=0.7]{fig/2g-nav-traj.pdf}
        \caption{
            ナビゲーションの軌跡比較。
            上段がAV-Nav w/o SDMで、下段がAV-Nav w/ SDMである。
            また、Pathの色はステップの経過を表している。
            ステップが経過するにつれ、青から赤へ変化する。
        }
        \label{fig:qualitative_eval}
    \end{center}
\end{figure*}


図\ref{fig:qualitative_eval}はSDMを用いた場合と用いなかった場合の、エージェントの軌跡の比較である。
SDMを用いることで無駄な行動が減っていることがわかる。
これは、音源定位がより正確にできるようになったからではないかと考えている。
また、AV-Nav w/o SDMの失敗例では、長く彷徨っている例が見られた。
これは、音源定位が正確にできず、どこにゴールがあるかを判断できずにいるからではないかと考えている。
また、w/ SDMでは、障害物に引っかかり、失敗してしまうケースが見られた。
これは、SDMにより、障害物の反対側に音源があると判断されてしまっているからであると考えている。

