\section{INTRODUCTION}

深層強化学習エージェントによる室内環境の視覚ナビゲーションタスクはここ10年で特に注目されている研究分野である。
ベーシックな視覚ナビゲーションは、一人称視点の画像という視覚情報のみを用いて、ある一つのゴールまでナビゲーションを行うものである。
近年では、より発展的なタスクが生まれており、視覚だけでなく聴覚の情報も同時に用いて音源へナビゲーションを行うAudio-Visual Navigation~\cite{chen2020soundspaces}や、一つではなく複数のゴールへナビゲーションを行うMulti-Object Navigation (MultiON)~\cite{wani2020multion}とよばれるタスクも提案されている。
一方、視覚の情報と聴覚の情報の二つを用いて、複数の音源ゴールへナビゲーションを行うタスク、すなわち、先に挙げた視聴覚ナビゲーションとマルチオブジェクトナビゲーションを組み合わせたようなタスクはいまだに提案されていない。
実応用を考えると、人命救助や鳥獣害駆除といった、聴覚情報が有効で、かつ必ずしもゴールが一つとは限らないようなタスクは多く存在する。
%むしろこれらのタスクでは、複数のゴールがあると考えるのが自然である。

\begin{figure}[t]
    \begin{center}
        \centering
        \includegraphics[width=\linewidth]{fig/multi-goal-audio-visual-navigation.pdf}
        \caption{
            マルチゴール視聴覚ナビゲーションの様子。
            ここでは、室内環境において三つの異なる音源へナビゲーションを行っている。
            エージェントは、一人称視点の視覚の情報と三つの異なる音源の音が重なった聴覚の情報を観測し、適切な行動選択を行わなければならない。
        }
        \label{fig:multi_goal_av_nav}
    \end{center}
\end{figure}

%したがって、こうした応用先を念頭におき、視覚の情報と聴覚の情報の二つを用いた複数の音源ゴールへのナビゲーションタスクについて研究することは、意義があると考えられる。視聴覚ナビゲーションだけでは、ある一つのゴールに到達するまでに得た有効な情報を捨ててしまうことになる。マルチオブジェクトナビゲーションだけでは、聴覚情報という有効な情報を捨ててしまうことになる。しかし、効率的なナビゲーションを行えるようにするためには、可能な限り、得ることのできる有効な情報は捨てずにあつかうべきである。
本研究では、視聴覚ナビゲーションとマルチオブジェクトナビゲーションを組み合わせた新しいタスクであるMulti-Goal Audio-Visual Navigationを提案する(図\ref{fig:multi_goal_av_nav})。
マルチゴール視聴覚ナビゲーションを解くためには、三つの重要な要素がある。
それは、音源分離・記憶・行動計画である。
まず、強化学習エージェントは、複数の音が重なった音を観測する。
そのため、正確な音源分離は、各音源に対する音源定位の精度向上に重要な役割を果たす。
また、ある一つのゴールへ到達するまでに取得した情報を記憶しておくことで、次のゴールへ効率的にナビゲーションを行えるようになることが期待できる。
さらに、そもそもどの音源を次のゴールにすると効率的な経路計画となるのかを推論しなくてはいけないという点において、行動計画は重要である。

本研究には、二つの目的がある。
第一の目的は、マルチゴール視聴覚ナビゲーションの難しさがどこにあるのかを明らかにすることである。
先にも述べたが、複数の音源ゴールがある場合の視聴覚ナビゲーションに対する先行研究はない。
本研究では、多様な状況下でマルチゴール視聴覚ナビゲーションを行うことで、どこに難しさがあるのかを明らかにする。
第二の目的は、マルチゴール視聴覚ナビゲーションをより高精度に解くための手法を提案することである。
本研究では、先述の三つの課題を解決するために、implicitな動的複数音源定位に基づく手法Sound Direction Map (SDM)を提案する。
SDMは、複数の音源を同時に定位することで経路計画を助ける。
また、SDMは記憶を有効に活用することで動的に更新される。
この動的な更新手法により、過去の音源定位情報を活用することで音源分離の性能を向上させ、その結果、音源定位の性能をさらに高める可能性がある。

本論文の主な貢献は以下の三つである。

\begin{itemize}[leftmargin=*]
\item 新しいナビゲーションフレームワークであるマルチゴール視聴覚ナビゲーションを提案し、様々な状況下でテストすることで、このタスクの難しさがどこにあるのかを検証した。
\item マルチゴール視聴覚ナビゲーションを効率的に解く手法として、implicitな動的複数音源定位の履歴を表現するSound Direction Map (SDM)を提案した。
\item SoundSpaces 2.0において、提案するSDMを用いることで、複数のベースライン手法に対する精度向上を示した。
\end{itemize}