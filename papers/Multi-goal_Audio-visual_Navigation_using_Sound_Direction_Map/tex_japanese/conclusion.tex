\section{CONCLUSION}
\label{conclusion}

本論文では、まず、複数の音源がゴールとなるマルチゴール視聴覚ナビゲーションを提案し、ゴール数が増加することによるナビゲーションへの影響を調査した。
調査の結果、ゴール数が増加することで、極端に性能が劣化する傾向があることがわかった。
それに続く難点調査では、マルチゴール視聴覚ナビゲーションにおける音源分離の重要性が示され、さらに記憶を有効に活用することの重要性も示唆された。

次に、マルチゴール視聴覚ナビゲーションをより高精度に解くための手法、Sound Direction Map (SDM)を提案した。
SDMは、記憶を活かし、動的に複数の音源について定位を行う手法である。
実験により、SDMは全てのゴール数で、全てのベースラインで有用であることが示された。
