%% 
%% Copyright 2007, 2008, 2009 Elsevier Ltd
%% 
%% This file is part of the 'Elsarticle Bundle'.
%% ---------------------------------------------
%% 
%% It may be distributed under the conditions of the LaTeX Project Public
%% License, either version 1.2 of this license or (at your option) any
%% later version.  The latest version of this license is in
%%    http://www.latex-project.org/lppl.txt
%% and version 1.2 or later is part of all distributions of LaTeX
%% version 1999/12/01 or later.
%% 
%% The list of all files belonging to the 'Elsarticle Bundle' is
%% given in the file `manifest.txt'.
%% 
%% Template article for Elsevier's document class `elsarticle'
%% with harvard style bibliographic references
%% SP 2008/03/01

%\documentclass[preprint,12pt,authoryear]{elsarticle}

%% Use the option review to obtain double line spacing
%% \documentclass[authoryear,preprint,review,12pt]{elsarticle}

%% Use the options 1p,twocolumn; 3p; 3p,twocolumn; 5p; or 5p,twocolumn
%% for a journal layout:
% \documentclass[final,1p,times,num]{elsarticle}
\documentclass[final,3p,times, num,sort&compress]{elsarticle}
 % \documentclass[final,1p,times,twocolumn,authoryear]{elsarticle}
 %\documentclass[final,3p,times,authoryear]{elsarticle}
%% \documentclass[final,3p,times,twocolumn,authoryear]{elsarticle}
%% \documentclass[final,5p,times,authoryear]{elsarticle}
%% \documentclass[final,5p,times,twocolumn,authoryear]{elsarticle}

%% For including figures, graphicx.sty has been loaded in
%% elsarticle.cls. If you prefer to use the old commands
%% please give \usepackage{epsfig}

%% The amssymb package provides various useful mathematical symbols
\usepackage{amssymb}
\usepackage{amsmath}
\usepackage{multibib}
\usepackage{mathtools}
\usepackage{caption}
\usepackage{lineno}
\usepackage{subcaption}
\usepackage{float}
\numberwithin{equation}{section}
\graphicspath{{Results/}}
\usepackage{soul}
\usepackage{xcolor}
\usepackage{hyperref}
%% The amsthm package provides extended theorem environments
%% \usepackage{amsthm}

%% The lineno packages adds line numbers. Start line numbering with
%% \begin{linenumbers}, end it with \end{linenumbers}. Or switch it on
%% for the whole article with \linenumbers.
%\numberwithin{equation}{section}
%% The amsthm package provides extended theorem environments
%% \usepackage{amsthm}

%% The lineno packages adds line numbers. Start line numbering with
%% \begin{linenumbers}, end it with \end{linenumbers}. Or switch it on
%% for the whole article with \linenumbers.
 \usepackage{lineno}
\newcommand{\dx}{{\Delta{x}}}
\newcommand{\dt}{{\Delta{t}}}
\newcommand{\eps}{\epsilon}
\journal{Computational Physics}
\begin{document}
\begin{frontmatter}

%% Title, authors and addresses

%% use the tnoteref command within \title for footnotes;
%% use the tnotetext command for theassociated footnote;
%% use the fnref command within \author or \address for footnotes;
%% use the fntext command for theassociated footnote;
%% use the corref command within \author for corresponding author footnotes;
%% use the cortext command for theassociated footnote;
%% use the ead command for the email address,
%% and the form \ead[url] for the home page:
%% \title{Title\tnoteref{label1}}
%% \tnotetext[label1]{}
%% \author{Name\corref{cor1}\fnref{label2}}
%% \ead{email address}
%% \ead[url]{home page}
%% \fntext[label2]{}
%% \cortext[cor1]{}
%% \address{Address\fnref{label3}}
%% \fntext[label3]{}

%\title{Modified Equation Analysis for Harmonic and Arithmetic Averaging of a Continuous Nonlinear Class of Diffusion Coefficients for the Nonlinear Heat Equation}
\title{Numerical Artifacts in the Discontinuous Generalized Porous Medium Equation: How to Avoid Spurious Temporal Oscillations}
%% use optional labels to link authors explicitly to addresses:
%% \author[label1,label2]{}
%% \address[label1]{}
%% \address[label2]{}

\author[label1]{Danielle C. Maddix}
\author[label2]{Luiz Sampaio}
\author[label1,label2]{Margot Gerritsen}
%\author[label3]{Anna Nissen}
\address[label1]{Institute of Computational and Mathematical Engineering}%, \\ Stanford University}
\address[label2]{Energy Resources Engineering \\Stanford University}
%\address[label3]{Dept. of Mathematics, University of Bergen, Norway}

\begin{abstract} %CHNAGE ABSTRACT-porous media finite volume doing this up front
Numerical discretizations of the Generalized Porous Medium Equation (GPME) with discontinuous coefficients are analyzed with respect to the formation of numerical artifacts.  In addition to the degeneracy and self-sharpening of the GPME with continuous coefficients, detailed in \cite{maddix_pme}, increased numerical challenges occur in the discontinuous coefficients case.  These numerical challenges manifest themselves in spurious temporal oscillations in second order finite volume discretizations with both arithmetic and harmonic averaging.  The integral average, developed in \cite{vandermeer2016}, leads to improved solutions with monotone and reduced amplitude temporal oscillations.  In this paper, we propose a new method called the Shock-Based Averaging Method (SAM) that incorporates the shock position into the numerical scheme.  The shock position is numerically calculated by discretizing the theoretical speed of the front from the GPME theory.  The speed satisfies the jump condition for integral conservation laws.  SAM results in a non-oscillatory temporal profile, producing physically valid numerical results.   We use SAM to demonstrate that the choice of averaging alone is not the cause of the oscillations, and that the shock position must be a part of the numerical scheme to avoid the artifacts.%Fix this part! cite Jakolein internal average and say monotone oscillations
%A dynamic average known as the integral average is formulated in terms of well-known numerical schemes.  want to discuss how to form in terms of other discretizations
%It is shown to be derived from a condition of flux equality in and out of the shock cell.  This leads to monotonic oscillations.  
%We then propose the Shock-Based Averaging Method (SAM) that incorporates the shock position into the numerical scheme.  % and removes the flux equality constraint at the shock cell.   
\end{abstract}

\begin{keyword}
discontinuous generalized porous medium equation  \sep Stefan problem  \sep nonlinear degenerate parabolic equations \sep temporal oscillations \sep numerical shock detection \sep jump condition  %\sep jump condition
% keywords here, in the form: keyword \sep keywordg

%% PACS codes here, in the form: \PACS code \sep code

%% MSC codes here, in the form: \MSC code \sep code
%% or \MSC[2008] code \sep code (2000 is the default)

\end{keyword}

\end{frontmatter}

%% \linenumbers

%% main text
%OUTLINE:
\section{Introduction}
\label{intro}
%We discuss how to discretize the self-sharpening Generalized Porous Medium Equation (GPME) without the temporal oscillations, the locking, and the lagging reported in previous works \cite{lipnikov2016, inl2008, nie2013, vandermeer2016}.  In these works, the numerical artifacts have been attributed to harmonic averaging, and arithmetic averaging has been proposed to resolve them.  The exact causes of the artifacts were not understood.  We demonstrate that harmonic averaging is not solely to blame and that these artifacts also depend on both the spatial and temporal discretizations.  We propose a modified numerical approach that removes the artifacts and leads to sharper solutions than those given by the arithmetic average.  

%Foam? numerical artifacts vague and then show temporal oscillations-no removes doesn't occur fix wording of first paragraph
%We propose a method that removes the numerical artifacts reported in literature \cite{vandermeer2016, lipnikov2016} present in numerical discretizations of the Generalized Porous Medium Equation (GPME) with discontinuous coefficients.  The governing equation is given by 
The purpose of the paper is to identify the cause of the numerical artifacts reported in the literature \cite{maddix_pme, lipnikov2016,vandermeer2016} for second order finite volume discretizations of the Generalized Porous Medium Equation (GPME) with discontinuous coefficients, and to suggest a numerical approach that does not have these problems.
The GPME, commonly known as the Filtration Equation, can be expressed in both % is known in a variety of ways throughout the literature \cite{vazquez2007}.  
% for its use in filtration models
%The 
conservative and integral forms as: % given here for discontinuous coefficients:%of the discontinuous GPME are given as,
\begin{equation}
\begin{aligned}
p_t &= \nabla \cdot (k(p) \nabla p) %\\& = \nabla \cdot (\Phi_p \nabla p)
 \\ &= \Delta \Phi (p),
\hspace{.25cm} \text{where} \\
\vspace{1cm}
\Phi(p) &= \int_0^p k(\tilde p) d\tilde p, \hspace{.25cm}  k(p) = \Phi'(p).
 \end{aligned}
 \label{eq:GPME}
\end{equation}
In the discontinuous coefficients case, $k(p)$ is given as:
\begin{equation}
\begin{aligned}
	k(p)& = 
	\begin{dcases}
  		k_{\max},           & p \ge p^* \\
    		k_{\min},              &p^* > p, \hspace{.25cm} \text{where}
	\end{dcases}
\label{eq:discont_k}
\end{aligned}
\end{equation} 
$k_{\max}$, $k_{\min}$ and $p^*$ are real positive constants.

In this paper, we are interested in a subclass of the GPME known as the Stefan problem \cite{stefan91, rubinstein71, meirmanov92, vazquez2007}, for which
	\begin{equation}
				\Phi(p) = \begin{dcases}
  									c_1(p - c_3)_+,          & \text{if} \hspace{.1cm} p \ge 0, \\
    									c_2p ,             & \text{otherwise},
						\end{dcases}
					%\nonumber
					\label{eq:stef_gen}
	\end{equation}
for arbitrary $c_1, c_2, c_3 \in \mathbb{R}$.  We look at a particular Stefan problem, where $p \ge 0, c_1 = k_{\max}$ and $c_3 = p^*$.  %Eqn. \eqref{eq:stef_sp} can be interpreted as a special class of Stefan problem, where $c_1 = k_{\max}$, $c_3 = p^*$.
%eqn 3 move 
%By the fundamental theorem of calculus and the chain rule, Eqn. \eqref{eq:GPME} can be expressed in integral form %This form is useful for the central difference interpretation reformulation same with different Phi
 %Strefan problem paragraph early? talk about porous medium equation and superslow diffusion now gives arbitrary k(p) and then talk about discontinuous coeff  
 %illsutrates the numerical challenges
  %Following this paper, we will refer to $k(p)$ as the permeability and $p$ as the pressure.  %possibly remove
Then, $\Phi(p)$ in Eqn. \eqref{eq:GPME} can be expressed as the positive part function
	\begin{equation}
		\Phi(p) = k_{\max}(p-p^*)_+ = \begin{dcases}
  											k_{\max}(p-p^*),          & \text{if} \hspace{.1cm} p \ge p^*,  \\
    											0,              & \text{otherwise},
									\end{dcases}
		\label{eq:stef_sp}
	\end{equation}
	and $k(p)$ is given by Eqn. \eqref{eq:discont_k} with $k_{\min} = 0$.
%It is clear that $\Phi(p)$ is continuous with a kink at $p^*$.  
%Eqn. \eqref{eq:GPME} can then be formulated as a Stefan Problem.  
	%Third paragraph on the speed!	
	
 The Stefan problem can also be formulated in its classical form \cite{sethian88,osher97} as  
 \[
\begin{aligned}
	%\begin{dcases}
  		\frac{\partial p}{\partial t}= k_{\max} \Delta p,   \hspace{0.25cm}        & p \ge p^*, \\
	%\end{dcases}
\end{aligned}
\]
\[
\begin{aligned}
	%\begin{dcases}
    		\frac{\partial p}{\partial t} = k_{\min} \Delta p,      \hspace{0.25cm}        & p < p^*. \\       %  &p^* > p,
	%\end{dcases}
\end{aligned}
\]
In the classical Stefan problem, the above two parabolic equations are defined on domains that are separated by a moving interface $x^*(t)$.
The Stefan condition is given on the interface as
\[
	(p_L -p_R)\frac{dx^*(t)}{dt} = -k_{\max}\frac{\partial p_L}{dx}+k_{\min}\frac{\partial p_R}{dx},
\]
where $p_L \equiv \lim_{x \rightarrow x^*(t)^-}p(x,t)$ and $p_R \equiv \lim_{x \rightarrow x^*(t)^+}p(x,t)$.
The Stefan condition can be derived by using the Rankine-Hugoniot jump condition for this conservation law \cite{myers15}.

The Stefan Problem has been key to both the numerical and theoretical developments of the GPME.   The Stefan Problem is used in modeling phase transitions, and was developed to study the evolution of a medium of two phases, water and ice \cite{vazquez2007}.  \citet{brattkus92} use the Stefan problem to model crystal growth.  \citet{sethian92} and \citet{osher97} develop a modified Stefan problem to model crystal growth as well as dendritic solidification.  Eqn. \eqref{eq:discont_k} is often used to illustrate the numerical challenges present in more complex porous media applications. 
For example, in \citet{vandermeer2016} a foam model prototype is developed with $k_{\max} = 1$, $k_{\min} = \eps \rightarrow 0$ and $p^* = 0.5$. %introduce reduction to stefan
 
  The continuous GPME already poses numerical challenges, caused by self-sharpening and degeneracy for near-zero $k(p)$ \cite{maddix_pme}.  %The discontinuous GPME in Eqn. \eqref{eq:discont_k} poses further numerical challenges.  
  Due to this self-sharpening and degeneracy in the continuous case, it is not just the discontinuity that poses the numerical challenges in the Stefan problem in Eqn. \eqref{eq:stef_sp}.  % poses severe numerical challenges.  As we saw before, it is not just the discontinuity that causes these issues, but the discontinuity makes it more complicated.  
The discontinuity in $k(p)$ does make the challenges more severe and because of this discontinuity, the Modified Equation Analysis approach in \cite{maddix_pme} for the continuous GPME is not applicable. Here, we look at the discontinuous GPME, and also refer to subclasses of the continuous GPME.  The Porous Medium Equation (PME) subclass, where 
$
k(p) = p^m \text{ and }  m \ge 1,
$
is used to model gas flow through a porous medium \cite{maddix_pme, vazquez2007, ngo2016}.  Another application in thermodynamics is the superslow diffusion equation \cite{vazquez2007, maddix_pme}, where
$
	k(p)  = \exp(-1/p)$.  Further applications of the GPME for continuous $k(p)$ are detailed in \cite{maddix_pme}.  %In \cite{maddix_pme}, we propose alternate approaches to overcome these numerical difficulties based on Modified Equation Analysis.  The same approach does not apply here because of the discontinuity in $k(p)$.
 %Many other applications of the GPME can be found for continuous $k(p)$.  
%Further applications of the GPME for continuous $k(p)$ are detailed in \cite{maddix_pme}.  

%with the following step-function coefficient

%One of the most common applications of the GPME occurs in its Porous Medium Equation (PME) subclass.  The PME is used to model gas flow through a porous medium \cite{maddix_pme, vazquez2007, ngo2016}.  Further applications of the GPME for continuous $k(p)$ are detailed in \cite{maddix_pme}.  We will discuss in more detail the applications for discontinuous $k(p)$ that pertain to this paper.


\subsection{Understanding the Behavior of the GPME} %Relevant Theoretical Properties in constant to the infinite speed of 
% shows the time evolution of the exact solution to the GPME with discontinuous coefficients.  The solution is a rightward moving shock with interface $\Gamma(t)$.  The shock position $x^*(t)$ is unknown and increases in time.  The pressure value at the shock is known and fixed at $p^* = 0.5$.  applications some problem set up

The GPME in Eqn. \eqref{eq:GPME} at first appears like a heat equation.  Contrary to solutions of the heat equation, where the propagation speed is infinite, solutions of the GPME are known to have a finite speed of propagation \cite{vazquez2007}.  This results from the degeneracy of the GPME for compactly supported initial data, and is a property that distinguishes the GPME from classical parabolic theory.  This degeneracy leads to self-sharpening and moving interface solutions, as illustrated in Figure \ref{exactsol_time}.  In addition, for certain compactly supported smooth initial data, the waiting time phenomenon can occur.  This is discussed in the literature \cite{fischer15, angenent88} \cite[Chapter~3]{antontsev15} and is illustrated in Figure \ref{wait_time}.  The term waiting time refers to the fact that an interface moves only after sufficient sharpening.

\label{rh_cond}
\begin{figure}[H]
		\center
		\includegraphics[width =0.49\textwidth]{exactsol_time1}
		\caption{The exact solution of Eqn. \eqref{eq:GPME} with $k(p)$ given by Eqn. \eqref{eq:discont_k}, where $k_{\max} = 1$ and $k_{\min} = 0$, evolves as a rightward moving shock over time.  The moving shock position is given by $x^*(t)$ and the fixed $p$ value at the shock is given by $p^* = 0.5$.}
		\label{exactsol_time}
\end{figure}


The theoretical propagation speed for compactly supported initial data is known and is given by Darcy's Law \cite{vazquez2007} as
\[
	V = -\hspace{-.2cm}\lim_{x \rightarrow x^*(t)^-}\nabla v,
\] where
\begin{equation}	%need to introduce fluxes here INTRODUCE FLUXES
	v = \int_0^p \frac{\Phi'(\tilde p)}{\tilde p} d\tilde p ,
	\label{eq:vel}
\end{equation} and $x^*(t)$ is the shock position.  Combining these expressions gives	\begin{equation}
		V = -\hspace{-.2cm}\lim_{x \rightarrow x^*(t)^-}\frac{\Phi'(p) \nabla p}{p} =  -\hspace{-.2cm}\lim_{x \rightarrow x^*(t)^-}\frac{k(p) \nabla p}{p}.
		\label{eq:vel_final}
	\end{equation}
	%cite other paper for RH condition
	%remove pme part
	%added in limit check notation
	
Eqn. \eqref{eq:vel_final} holds for any $k(p)$.  %For the PME in Eqn. \eqref{PME}, $k(p) = p^m$, and then 
%\[
%	V_{\text{PME}} %= -\frac{p^m \nabla p}{p} = -p^{m-1} 
%\nabla p 
%= -\hspace{-.2cm}\lim_{x \rightarrow x^*(t)^-}\frac{\nabla p^m}{m}.
%\]  
In the particular case of the Stefan problem in Eqn. \eqref{eq:stef_sp}, %$p$ is monotone non-increasing, and so
 $k(p) = k_{\max} = 1.0$ to the left of the shock.  Substituting the $k(p)$ limit into Eqn. \eqref{eq:vel_final} gives %can fill in with kmax
\begin{equation}
	V = -\hspace{-.2cm}\lim_{x \rightarrow x^*(t)^-}\frac{\nabla p}{p} = -\hspace{-.2cm}\lim_{x \rightarrow x^*(t)^-}\nabla \log(p).
	\label{eq:front_speed}
\end{equation}	
%say this theoretically is the ranking hungiot condition just how to approximate numerically FIX!!! %Pg. 198 new vazquez
%CITE NEW vazquez paper!!

%fix this part express in terms of fluxes!!
 The expression for the velocity $V$ can also be expressed in terms of fluxes.  The flux $F$ for the integral conservation law in Eqn. \eqref{eq:GPME} is given by 
\begin{equation}
	F(p) = -k(p) \nabla p =  -\nabla \Phi(p).
	\label{eq:anal_flux}
\end{equation}
Then, from Eqn. \eqref{eq:vel_final}, 
\begin{equation}
 	V = \frac{F(p_L)}{p_L},
	\label{eq:RH_pre}
\end{equation}
 where $p_L \equiv \lim_{x \rightarrow x^*(t)^-}p(x,t)$.  For a compactly supported initial condition and $k_{\min} = 0$, the flux and $p$ values to the right of the shock are zero.  
In this case, the velocity in Eqn. \eqref{eq:RH_pre} can be expressed in terms of the familiar jump condition for integral conservation laws as\begin{equation}
 	V = \frac{F(p_L) - F(p_R)}{p_L - p_R},
	\label{eq:RH}
\end{equation}
where $p_R \equiv \lim_{x \rightarrow x^*(t)^+}p(x,t)$ \cite{vazquez05, vazquez2007}.  The velocity in Eqn. \eqref{eq:RH} holds in general for any $k_{\min}, p_R \ge 0$, by the Rankine-Hugoniot condition \cite{lax57}.
%The expression for $V$ is the same as the Rankine-Hugoniot jump condition \cite{lax57} for the shock speed of hyperbolic conservation laws
%The expression for $\hat{V}$ is similar to approximating the Rankine-Hugoniot expression for the shock speed of hyperbolic conservation laws, namely   %say the same and cite paper
 %discuss more about the flux jump desirable condition
%The Rankine-Hugoniot condition shows that the analytical fluxes are discontinuous, and provides the exact expression for the flux jump across the interface 
	%modeled instead of formulated  
%move stefan problem and later references for applications here
%%uncomment if want foam model
%combine Stefan paragraph and then other applications?

%application paragraph. other applications: crystal growth and dendritic solidification
%will see why in application section care about stedan problem


%Since this model belongs to this GPME class, the problem is well-posed. Well-posed?
%changed from foam model
%check this
%In typical foam models as detailed in \cite{vandermeer2016}, $k_{\max} = 1$, $k_{\min} = 0$ and $p^* = 0.5$.  The permeability is a step function representing two separate materials.    %A physical interpretation in terms of the fluxes is presented.   The fluxes provide the intuition and understanding behind our new method.  Smooth flux functions are needed, where the out flux from the shock cell is larger than the in flux of this control volume.   
%where to place?!?

%use labeling from PME paper
\subsection{Numerical Methods to Approach this Problem} %refer ahead instead of plots in intro
\label{lit}
The degeneracy, self-sharpening and nonlinearity of the GPME pose interesting numerical challenges.  The numerical approaches used by practitioners vary based on the field.  In the porous media communities, central flux-based finite volume methods are widely employed to solve the related variable coefficients problem
$p_t  = \nabla \cdot (k(x) \nabla p)$.
The coefficient $k(x)$ is defined at the cell-centers, and harmonic and arithmetic averaging are commonly used to compute the coefficient, which may be discontinuous, at the cell interface.  Harmonic averaging is often preferred in these problems because it leads to more physical solutions.  This common finite volume averaged-based approach has also been extended to the nonlinear GPME in Eqn. \eqref{eq:GPME}.  For our case, the selection of the $k(p)$ averaging is not as straightforward as in the variable coefficient case.     %use same averaging framework that the codes are built on
After carefully comparing arithmetic to harmonic averaging, \citet{lipnikov2016} prefer arithmetic averaging over harmonic for the continuous PME with near-zero $k(p) = p^m$.  An integral average is developed in \citet{vandermeer2016} for the discontinuous GPME in Eqn. \eqref{eq:discont_k}.  With these average-based approaches, it is difficult to satisfy the jump condition in Eqn. \eqref{eq:RH}. Violation of this condition manifests itself in locking and spurious temporal oscillations, as illustrated in Figures \ref{fig:intro} and \ref{wait_time} and further discussed in Section \ref{static_avg}.   
%This is the focus of the paper
%In  \cite{vandermeer2016}, the authors also conclude that the temporal oscillations are present, regardless of the discretization.   %possibly delete
%Averaged-based approaches for variable coefficient $k(x)$ problems have been applied to the nonlinear GPME.  
 %This problem deals with small and even zero permeability values for $k_{\min} = 0$ in Eqn. \eqref{eq:gen_eq}.
%The small permeabilities reveal the degenerate parabolic character of the equation, as detailed in \citet{maddix_pme}.  
%STOPPED HERE!!!
%Figure \ref{art_space} illustrates the locking numerical artifact associated with harmonic averaging, and that arithmetic averaging is diffusive. %From the spatial plots in Section \ref{static_avg}, we see that the scheme with arithmetic averaging appears to work well.  
%The temporal plots in Figures \ref{art_time} and \ref{art_timezoom} reveal the non-physical behavior of the numerical solution with arithmetic averaging, as manifested in the temporal oscillations.  
%The temporal plots provide an important metric, and demonstrate that changing from harmonic to arithmetic averaging is not a solution for this problem.  %arithmetic 
  %The integral average incorporates the pressure value at the shock into the scheme.  The numerical shock position is not utilized.  
%In Figure \ref{art_timezoom}, we see that the high-frequency oscillations around $p^* = 0.5$ present with arithmetic averaging are removed. 
% Monotonic grid-dependent temporal oscillations remain that do not vanish upon grid refinement. 
%By allowing unequal shock cell fluxes for a problem with discontinuities, we show in Section \ref{sam_exact}, that the temporal oscillations can be removed. %refer ahead to section with plots
In other fields, adaptive and moving mesh approaches have been applied to this problem also with increased refinement near the shock.  In \cite{ngo2016}, for example, a finite element moving mesh method for the PME is developed.  Adaptive Mesh Refinement (AMR) will be discussed further in Section \ref{static_avg}. %Similar to moving meshes,  Adaptive Mesh Refinement (AMR) \cite{berger84} can also be applied.   With AMR, the amplitude of the temporal oscillations is diminished, but the grid-dependent oscillations still persist  (see section \ref{static_avg}).   %SAM is more robust than AMR-based methods, and is accurate on coarse meshes without refinement needed.
 
 Variational particle schemes have been developed for the PME in \cite{wilkening10, otto01,villani03}, where a variational principle is defined and steepest descent is applied to optimize the corresponding energy function.  Variational principles have also been defined for the Stefan problem in \cite{rossi04} and a similar gradient flow method can be applied for the discontinuous GPME.%gradient flow
 
%Rose method was unconditionally stable can change interface for boundary
%%FILL IN HERE!! combine into 1 paragraph and just give references or describe? list to reference therein
In the crystal application communities, numerical work has been done on solving the Stefan problem, as detailed in \cite{berger79, kharab86, Rizwan98, caldwell2002} and the references therein.  In \cite{rose90, rose93}, an enthalpy scheme for Stefan problems is introduced.  This method consists of a compact finite difference stencil with an implicit temporal scheme.  The shock position is not incorporated into the scheme, and the error is observed to be concentrated at the moving interface.  \citet{brattkus92} show that finite difference and finite element methods that do not take special care at the interface result in a significant error of $\mathcal{O}(\sqrt \dx)$.  They propose an effective method based on an integral equation formulation that %involves computationally efficient integration of the memory integral,  
 requires an additional integral evaluation.
 %  A disadvantage of this method is that to maintain a constant operation count at each time step, the memory integral is evaluated on a spatial grid, in addition to solving the governing equation on that grid. %fix introduction on spatial grid as well as solve equation
%mentions that fd or fem applied without special take of interface leads to large errors on the order of sqrt(dx) important paper!
%that standard difference formulas (or finite element schemes that do not track the interface) applied directly to (4)-(10) result in significant errors unless special care is takenwhendifferencingaboutthepointz �(t).Thisisduetothejumpdiscontinuity in the gradient of T(z, t) as we approach �(t) from either side. Typically, schemes that do not track the interface result in errors of O(sqrtdx), where dx is a typical meshsize for the difference scheme.

  \begin{figure}[H]
		\center
		\begin{subfigure}[H]{.33\textwidth}  
			\includegraphics[width =\textwidth]{art.eps}
			\caption{Spatial profiles at time $t = 0.05$.}
			\label{art_space}
		\end{subfigure}
		\begin{subfigure}[H]{.33\textwidth}  
			\includegraphics[width =\textwidth]{art_t.eps}
			\caption{Temporal profiles at position $x = 0.32$.}
			\label{art_time}
		\end{subfigure}
		\begin{subfigure}[H]{.33\textwidth}  
			\includegraphics[width =\textwidth]{art_tzoom.eps}
			\caption{Zoomed in temporal profiles.}
			\label{art_timezoom}
		\end{subfigure}
		\caption{Comparison of various averages, where $k_{\max} = 1$, $k_{\min} = 0$ and $p^* = 0.5$.  The spatial step size $\dx = 0.04$ and the time step size $\dt = \dx^2/32$.}
		\label{fig:intro}
		\end{figure}

\begin{figure}[H]
		\center
		\begin{subfigure}[H]{0.33\textwidth}  
			\includegraphics[width =\textwidth]{plinearic_posintro100.eps}
			\caption{Spatial profiles}
		\end{subfigure}
		\begin{subfigure}[H]{0.33\textwidth}  
			\includegraphics[width =\textwidth]{pclinic_intro.eps}
			\caption{Temporal profiles at position $x = 0.32$.}
		\end{subfigure}
		\begin{subfigure}[H]{0.33\textwidth}  
			\includegraphics[width =\textwidth]{pclinic_zoomintro.eps}
			\caption{Zoomed in temporal profiles.}
		\end{subfigure}
			\caption{Comparison of various averages, where $k_{\max} = 1$, $k_{\min} = 0$ and $p^* = 0.5$ for a waiting time phenomenon example.  %Waiting time phenomenon exhibited for the GPME with $N = 100$ grid points.  
The piecewise linear initial condition self-sharpens until it is sharp enough for the support interface position to move rightward at later times past the initial interface position at $x = 0.5$.  The spatial step size $\dx = 0.01$ and the time step size $\dt = \dx^2/32$.}
			\label{wait_time}
\end{figure}

%and for a problem with discontinuities, we must enforce that the shock cell fluxes are not equal.    In Section \ref{sam_exact}, we show that this is not the case.  %With SAM, we show that with the exact shock position or numerically detected shock position in the scheme, the oscillations are resolved.  

%remove theta method
%The temporal oscillations remain with the above averaging schemes, even with implicit and higher order total variation diminishing (TVD) temporal schemes.  Backward Euler and the TVD RK2 (Shu) scheme \cite{shu98} have been implemented and tested.  As observed in \cite{maddix_pme}, changing to these time-stepping schemes is not sufficient to remove the temporal oscillations.  It will be shown in Section \ref{sam_exact} that the spatial scheme needs to include information about the shock location to remove these temporal oscillations.
%implemented on the test problem with various initial conditions and boundary conditions given in Eqn. \eqref{eq:bc}.  Changing to a higher order or TVD temporal scheme is not enough \cite{maddix_pme} because the spatial scheme needs to include information about the shock location to remove these temporal oscillations.

%	Based on the hyperbolic conservation law formulation in Eqn. \eqref{eq:hyperbolic} of the governing equation
%	\begin{equation}
%	p_t +  \nabla \cdot F = 0,
%	\label{eq:hyperbolic}
%\end{equation} 
%for $F = -\nabla\Phi = -k(p) \nabla p$, higher order spatial methods for hyperbolic conservation laws were also implemented and tested.  These methods include the Monotonic Upstream-Centered Scheme for Conservative Laws (MUSCL) \cite{vanleer79}, symmetric limited positive (SLIP) and upstream limited positive (USLIP) \cite{jameson95} schemes with slope limiters, such as minmod, vanleer and superbee.  The Jameson-Schmidt Turkel (JST) scheme, as detailed in \cite{jameson15} is another effective method for shock detection in hyperbolic conservation laws.  The JST scheme is higher order accurate away from the shock and first order accurate at the shock.  For these methods, the temporal oscillations still persist, with monotonic temporal profiles similar to the results with the integral average in Figure \ref{fig:int_time}.  A challenge to this approach is that the equation is not truly hyperbolic and with small coefficients, it is degenerate parabolic.  In addition, the flux function ($F = -\nabla \Phi$) is a derivative, and must be appropriately discretized. 

%The Modified Equation Analysis approach for the PME from \cite{maddix_pme} is not applicable for the governing equation because the derivatives of the discontinuous $k(p)$ do not exist.  An approach was implemented, where $k(p)$ was made to be continuous with various types of ramps, such as arctangent, sine and linear ramps.  The problem is that the derivatives are inversely proportional to the ramp width.  As the ramp width tends towards zero, these derivatives become very large, resulting in blow-up.  To be added explicitly, extremely small time steps are required, which is not practical. Implicit methods would require a nonlinear solve, which is also computationally expensive to do at each time step.  Even with the ramp, the derivatives are too sharp and large in magnitude for the Modified Equation Analysis to hold.  Extremely fine grids are needed to reach the asymptotic region. 

In two-phase flow applications of the Stefan problem, level set methods \cite{sethian88} could also be useful in tracking the evolving interface.  Thus far, they have not been directly applied to Eqn. \eqref{eq:stef_sp}. Related problems with Stefan boundary conditions are discussed in \citet{sethian92} and \citet{osher97}.  More recent approaches using the level set method in the solidification community are detailed in \cite{kim00,tan07}.  An advantage of the level set approach over traditional front-tracking approaches \cite{rawahi02, li03} is that the interface is treated implicitly.  In doing so, the level set method can be more easily extended to higher dimensions, and is robust under complex topologies.  %use the level set method to solve a modified Stefan problem used in modeling crystal growth and dendritic solidification.
%second order central finite differences are used away from the interface and the interface itself is evolved by solving the level set equation.  
Numerical methods combining the level set approach with the Extended Finite Element Method (X-FEM) have also been developed to solve these Stefan problems \cite{bernauer12, salvatori09}.
 A modified Stefan problem is also solved in \citet{zhao2016} using a phase-field method.  Phase-field methods converge as the interface thickness parameter tends to zero.  The use of phase-field approaches in solidification and the effect of the interface thickness parameter have also been studied in \cite{yang2015, han2013}.  While these papers in the solidification and phase change communities provide approaches to solve the Stefan problem, they do not give details on understanding why the numerical artifacts occur with finite volume average-based approaches.  This leads us to the main goal of this paper.
 
  %A disadvantage is that in practice these methods are known to have difficulties in resolving the interface, and level set methods can be more accurate \cite{osher97}. %converge as the interface thickness tend, disadvantage in tuning interface thickness param

%None of the approaches have worked sufficiently well for the solving Eqn. \eqref{eq:GPME} and/or have been tried on simplified problems.
%
%Osher use \phi to interpolate polynomials approx T so satisy Gibbs Thomson relation
% In both of these papers, the permeabilities known as thermal diffusivities in this context are set to be unity.  Without the small permeabilities, the degenerate%permeabilities are set to 1 so reduces to laplacian and standard heat equation
%The equations of motions are transformed into a single boundary integral equation on the moving boundary, which is evolved by solving the level set equation.  %rather than marker particle methods Only Sethian paper uses boundary integral formulation and Osher reinitializes level set to be signed at every step. Get laplacian of T out because assuming
%adapted methods from classical stefan theory for unstable solidification

%Similarly, \citet{brattkus92} uses a finite difference discretization of Stefan problem to model crystal growth. This method is based on an integral equation formulation of the equation that involves integration over the entire time history.

%Caldwell considers 1D Stefan problem describing the melting process and uses two independent methods, the nodal integral and FD approaches to determine evolution of temperature and phase boundary. Rizwan-uddin also uses nodal integral approach

%Bratkus: direct calculation of memory integral over entree period of the growth is costly as time increases, so indirect method presented that has fixed operation per tiemstep with memory efficient version operation count constant for any given time step

%khabab main advantage of boundary-fixing method to solve free-boundary problems is to avoid difficulties arising from having unknown domain.  Turns moving boundary into fixed boundary problem.  Main disadvantage is the highly nonlinear structure of transformed diff eqtns

%front-tracking method proposed by Marshall introduced a variable time step procedure in conduction with a predictor-corrector scheme. Comparison by Furzeland

%Rizwan cites front tacking method for 1D problem as proposed by Marshall using variable-timestep procedure in conjunction with predictor corrector comparison of num methods by furzeland.

%5 references from front-tracking to nodal integral methods (caldwell/rizwan) to finite difference and fixed boundary methods (kharab) and predictor corrector method(kharab)

 %For this problem, the anti-diffusive term in the Modified Equation from \cite{maddix_pme} is not the cause of the temporal oscillations.  The cause is due to the uncertainty in the shock position for schemes that do not detect or incorporate it.  

%We see that is is also an intrinsic challenge of the foam model that when the pressure increases above $p^*$ the permeability increases.  This increase in permeability results in a decrease in pressure and so a decrease in permeability.  The decrease in permeability increases the pressure and the cycle continues.
%Connections among the various schemes
%higher order finite difference stencils
%\subsection{Various Interpretations of the Governing Equation and Numerical Schemes} %change title and move Stefan part up?
%\subsection{Main Questions Leading to the Proposed Numerical Method (SAM)}
\subsection{Main Goal of this Paper}
The main goal of this paper is to shed light on the origin of the numerical problems reported in the literature for finite volume averaged-based approaches of the discontinuous GPME.   
Before we discuss the cause of the artifacts, we first introduce an alternate numerical method %The first goal of this paper is to devise an accurate numerical method 
for Eqn. \eqref{eq:stef_sp} that does not exhibit the numerical errors observed in the literature.  This newly developed Shock-Based Averaging Method (SAM) is discussed in Section \ref{sam_exact}.  We discuss this first because it shows that incorporation of the shock position in the scheme is key, and also helps with the understanding of the artifacts in later sections.  Section \ref{exact_shock} discusses the derivation of SAM when the shock position is known.  The shock position can also be approximated using the jump condition, which is detailed in Section \ref{shock_speed}.  In Section \ref{static_avg}, we cast SAM in the finite volume framework to help identify what has been lacking in the other approaches.  The main issue is that these discretizations do not contain enough information about the shock.

\section{Proposed Numerical Method: Shock-Based Averaging Method (SAM)}
\label{sam_exact}
Due to the discontinuities in $k(p)$ and $p$, we adopt a finite volume approach to the integral form of the governing equation: %For the governing equation with $k_{\max} = 1$, the pressure is assumed to be monotone non-increasing and so $k(p) = k_{\max}$ to the left of the shock.  
%For the finite volume discretization and notation see \citet{maddix_pme}.  
\begin{equation}
	\begin{aligned}
	p_t &= \nabla \cdot (k(p) \nabla p), \hspace{.25cm} \text{where}
\\
	k(p)& = 
	\begin{dcases}
  		k_{\max},           & p \ge p^* \\
    		k_{\min},              &p^* > p,
	\end{dcases}
\label{eq:prob_def}
\end{aligned}
\end{equation}
where $p^*$ is the left limiting value.
We first define a finite volume grid with cell-centers $x_j$ for $j = 1,\dots,N + 1$. The boundaries of the domain are at $x_1$ and $x_{N+1}$, respectively and the corresponding unknowns $p$ are fixed by Dirichlet boundary conditions.  The remaining $N\!-\!1$ degrees of freedom $p_j$, and the corresponding coefficients $k_j$, are defined at the nodes $x_j$ for $j = 2,\dots,N$.  We define control volumes $CV_j$ with width $\dx_j$.  %Each cell-center is separated by a spatial step size $\dx_j$, and the cell faces are at a distance of $\dx_j/2$ from the cell-centers.  
The cell faces $x_{j+1/2}$ of each $CV_j$ are at a distance of $\dx_j/2$ from the cell-centers $x_j$.
%The unknowns defined at each node $x_j$ are $p_j$ for $j = 2,\dots,N$.  
We assume that the solution is monotone and non-increasing, and that the shock is located between $x_i$ and $x_{i+1}$, such that $ p_i \ge p^* \ge p_{i+1}$.  %The unknowns $p_j$ and the corresponding coefficients $k_j$ are defined at the nodes.  %We assume that $p$ is piecewise constant in each control volume $CV_j$ with nodal value $p_j$ and area $\dx_j$.  

The semi-discrete numerical discretization for $CV_j$ with volume $\dx_j$ is given by
\begin{equation}	
	%\int_{CV_j} \frac{dp}{dt} \hspace{.1cm}dx 
	\dx_j\frac{dp_j}{dt}= F_j^- - F_j^+,%F_{j-1/2} - F_{j+1/2}, %change to \dx for the correct area
	\label{eq:fv_discret}
\end{equation}%give formulation and definitions
where $F_j^-$ %$F_{j-1/2}$ 
represents the in-flux and $F_j^+$ %$F_{j+1/2}$ 
represents the out-flux of $CV_j$.    
%or just say refer to paper 1 for the notation
The numerical fluxes  
\begin{equation}
	 F_{j}^+ = -k_{j+1/2}\frac{p_{j+1}-p_j}{\dx}, \hspace{.25cm} j \ne i,
	\label{eq:stand_outflux}
\end{equation}
and
\begin{equation}
	F_{j}^- = -k_{j-1/2}\frac{p_{j}-p_{j-1}}{\dx}, \hspace{.25cm} j \ne i+1,
	\label{eq:stand_influx}
\end{equation}
are defined at all faces away from the shock cell with the standard two-point flux approximation, where $\dx \equiv \dx_j$.  The coefficient $k_{j+1/2}$ at the cell face represents a local average of its neighboring coefficients.  Then, for any two-neighbor average away from the shock cell, the coefficient is constant and given by
%For any two-neighbor average away from the shock cell, where $p_j \ge p^* \ge p_{q+1}$ for monotone non-increasing $p$, the coefficient is constant and given by
\begin{equation} %without requiring an average it is known
k_{j+1/2} =
	\begin{dcases}
  		 k_{\max},           & 1 \le j \le i - 1,   \\
		  k_{\min}, & i+1 \le j \le N.
	\end{dcases}
	\label{eq:gen_coeff}
\end{equation}
Analogous definitions are used for $j\!-\!1/2$.
Because of the discontinuity, Eqns. \eqref{eq:stand_outflux}-\eqref{eq:stand_influx} cannot be used for the flux at the cell interface $x_{i+1/2}$.
%The gradient of $p$ is also well-defined in these two regions. 
%replace discontinuity with jump

\subsection{Formulation of the Fluxes near the Shock}
\label{aux_grid}
%More discussion about where this comes from in hyperbolic conservation laws!!
%is not defined with the discontinuity in $k(p)$ and $p$.  
To estimate the fluxes out of $CV_i$ and into $CV_{i+1}$, we borrow ideas from hyperbolic systems \cite{rao09} and place a control volume around the discontinuity.  %we define a modified finite volume grid.  
%changed modified to auxiliary 
Figure \ref{fig:FV} shows the auxiliary finite volume grid with the additional control volume $CV_*$ around the physical shock position $x^*(t)$, where $p(x^*(t),t) \equiv p^*$ is defined.      %Let $\dx_j \equiv \dx$ for all $j \ne i$. 
Figure \ref{fig:FV} also shows $\dx^*(t)$ as the distance between $x_i$ and $x^*(t)$, where $0 \le \dx^*(t) \le \dx$.  We then remove the cell face at position $x_i + \dx/2$ and add the cell faces of $CV_*$.  %We choose is as central discretization.  %However, could choose in different ways. (ADD discussion) 
The new cell faces are chosen to be centrally located at a distance of 
\[
	\frac{|x^*(t) - x_i| }{2} \equiv \frac{\dx^*}{2},
\] from $x_i$ and $x^*(t)$, and a distance of 
\[
	\frac{|x^*(t) - x_{i+1}|} { 2} \equiv \frac{\dx - \dx^*}{2},
\]
 from $x^*(t)$ and $x_{i+1}$.  
 %accuracy
 %removed degree of freedom
 We have effectively added an additional degree of freedom $p^*$ at $x^*(t)$.  In Eqn. \eqref{eq:prob_def}, $p^*$ is known, and it does not need to be computed.  We present the approach in its general form for future extensions.%We define control volumes $CV_j$ with cell-center $i$, $CV_{q+1}$ with cell-center $i+1$ and $CV_*$ with cell-center $*$.  Let $\dx^*$ be the distance from $x_*$ to $x_j$ and $\dx$ is distance from $x_{q+1}$ to $x_j$.  
%So $\frac{\dx^*}{2}$ is the distance to the new face between $i$ and $*$.  
%Paragraph write scheme
% Recall that $\dx^* = x^*(t) - x_j$ and so 
%\[
%	\frac{d{\dx^*}}{dt} = \frac{d{x^*(t)}}{dt},
%\] since the fixed grid point ($x_j$) does not change in time. 
\begin{figure}[H]
		\center 
		\includegraphics[width =.49\textwidth]{FVdiagram_new.eps}
		\caption{Illustration of the additional control volume $CV_*$ around the shock position $x^*(t).$  The grid points $x_i$ and $x_{i+1}$ are depicted by the blue dots, and $x^*(t)$ is depicted by the red star.   The dashed line represents the deleted cell face between $CV_i$ and $CV_{i+1}$.  The new cell faces are located at a distance $\dx^*(t)/2$ from $x_i$ and $[\dx - \dx^*(t)]/2$ from $x_{i+1}$.}
		\label{fig:FV}
\end{figure}

% With the additional control volume, we can now define the necessary out-flux $F_i^+ \equiv F_*^-$ of $CV_i$ and in-flux $F_{i+1}^- \equiv F_*^+$ of $CV_{i+1}$. 
 %namely the flux out of $CV_j$ and the flux into $CV_{q+1}$, where $F_j^+$ is defined at the cell face between $x_j$ and $x_*$ and $F_{q+1}^-$ is defined as the cell between between $x_*$ and $x_{q+1}$.  
 With the additional control volume, we can now define the necessary out-flux $F_i^+$ of $CV_i$ and in-flux $F_{i+1}^-$ of $CV_{i+1}$. 
 The coefficient at each of the new cell faces is known.  By the monotonicity of $p$, for any $p(x,t)$ to the left of the interface, $k(p(x,t)) = k_{\max}$.  Similarly, for any $p(x,t)$ to the right of the interface, $k(p(x,t)) = k_{\min}$.  %use assumption in ext section motivation why want k_{min} to be 0 or just assume it from beginning from physical models
%Since all the other fluxes will be the same regardless of the choice of average on one side of the shock for piecewise constant $k$, we only need to modify the expressions for $F_j^+$ and $F_{q+1}^-$.  
For this piecewise constant coefficient, any two-neighbor average on this auxiliary grid gives the same coefficient value at the faces.  Based on these known coefficient values at the faces and the face-centered grid, the second-order fluxes are given by %fix change correct 
\begin{equation}
	F_{i}^+ = -k_{\max}\frac{p^*- p_i}{\dx^*}, %add (t) defined at different points
	\label{eq:fluxi}
\end{equation}
at the cell face between $x_i$ and $x^*$
and
\begin{equation}
F_{i+1}^- = -k_{\min}\frac{p_{i+1}-p^*}{\dx-\dx^*}, %relate to stefan problem and jump across flux RH
\label{eq:fluxiplus}
\end{equation} %exactly what we expect from Stefan problem relationship
 at the cell face between $x^*$ and $x_{i+1}$. 
 The resulting scheme is conservative because $F_i^+$ is equal to the in-flux $F_*^-$ of $CV_*$, by definition and similarly $F_{i+1}^-$ is equal to the out-flux $F_*^+$ of $CV_*$.  The fluxes are then substituted into the semi-discrete equation \eqref{eq:fv_discret}, where the cell volumes are now given by $\dx_i = (\dx+ \dx^*)/2$ and $\dx_{i+1} = \dx- \dx^*/2$, respectively.
 
 %are key in preventing the temporal oscillations. 
 %With this choice, we have discontinuous fluxes, $F_{i}^+  \neq F_{i+1}^-$ that are key in preventing the temporal oscillations, and are desirable to satisfy the jump condition, as discussed in Section \ref{rh_cond}.  %break up into parts and do each flux on its own
 %relate back to intro section Rh condition
 
For $k_{\min} = 0$, we can relate the expression for $F_{i}^+$ in Eqn. \eqref{eq:fluxi} to the expression for $\Phi(p) = k_{\max}(p-p^*)_+$ in Eqn. \eqref{eq:stef_sp} for the Stefan problem.  The analytical flux $F(p_L)$ is given by $-\nabla \Phi(p_L)$ in Eqn. \eqref{eq:anal_flux}.  The numerical flux $F_{i}^+$ can be interpreted as approximating this gradient with an upwind discretization as 
\[
	-\frac{\Phi(p^*)-\Phi(p_i)}{\dx^*} = -k_{\max}\frac{p^*-p_i}{\dx^*} = F_{i}^+.
\]
%We see that the numerical flux jump is a first-order approximation of the jump condition in Eqn. \eqref{eq:RH}.  
We see that the above formula is a first order approximation to the flux at $x^*(t)$.  From the jump condition in Eqn. \eqref{eq:RH_pre}, the velocity can then be computed.

 %Remove?
Round-off errors can arise when the shock location approaches the grid points $x_i$ and $x_{i+1}$, and the denominators in Eqn. \eqref{eq:fluxi} and Eqn. \eqref{eq:fluxiplus} approach zero.  To avoid problems, we specify a tolerance of $\eps$ that is proportional to $\dx$.  If $\dx^* \le \eps$, we neglect the leftmost portion near $x_i$, and set $k_{i+1/2} = k_{\min}$ for $CV_i$.  Similarly, if $\dx - \dx^* \le \eps$, we ignore the rightmost portion near $x_{i+1}$, and set $k_{i+1/2} = k_{\max}$ for $CV_{i+1}$.  We then use these coefficients in the standard two-point flux approximation in Eqns. \eqref{eq:stand_outflux} and \eqref{eq:stand_influx}.  %With decreased $\dx$, the grid cell size is smaller, and $\eps$ also decreases.

We will refer to the above approach as the Shock Based-Averaging Method (SAM).

\section{SAM Numerical Results: Exact Shock Location}
\label{exact_shock}
In the numerical results presented throughout the paper, the test problem is given by Eqn. \eqref{eq:prob_def} with %unless stated otherwise 
$k_{\max}=1.0$, $k_{\min}=0.0$ and $p^* = 0.5$.  The choice of $k_{\max}$ and $p^*$ is arbitrary, while $k_{\min}$ is set to zero to test the behavior of the numerical method in the degenerate case.  The algorithm also works for arbitrarily small $k_{\min}$. 
The fixed Dirichlet boundary conditions are given by
	\begin{equation}
		\begin{aligned}
			p(0,t) &= 1.0, \hspace{.2cm} \forall t \ge 0, \\
			p(1,t) &= 0.0,\hspace{.2cm} \forall t \ge 0.
		\end{aligned}
		\label{eq:bc}
	\end{equation}
We first present results for the initial condition displayed in Figure \ref{IC}.  This initial condition is formed as the exact solution to Eqn. \eqref{eq:prob_def} at a later time.  The same parameters for $k_{\max}$ and $p^*$, as given above for the test problem, are used to generate the initial condition, whereas $k_{\min} = 0.01$ for some smoothness at the bottom of the front.  The coefficient value $k_{\min} = 0$ throughout the simulations.%rather than a sharp corner
%We use a different initial condition than the one presented in \cite{vandermeer2016} to see whether the discontinuous initial condition was causing the observed temporal oscillations in that work. %add more to this
\begin{figure}[H]
		\center
		\includegraphics[width =0.49\textwidth]{IC-eps-converted-to.pdf}
		\caption{Spatial profile at time $t = 0$.}
		\label{IC}
\end{figure}
For the representative test problem, an analytical solution exists,   
%The analytical solution of the test problem is known, and 
and is used to verify the implementation through detailed convergence studies. %\cite{Kraaijevanger2011,brattkus92},  
The derivation of the exact solution is generalized for arbitrary $k_{\max}$ in \ref{exact}.  In this section,  we use the exact shock position to compute $\dx^*(t) \equiv |x^*(t) - x_i|$, where 
\[
	x^*(t) = \alpha \sqrt{t}, \hspace{.5cm} \alpha  = 2\sqrt{k_{\max}}z_1,
\]
and $z_1$ is the solution to the nonlinear equation in Eqn. \eqref{eq:alpha}.  In the following section, we will show how the shock location can be approximated, if it is not available.

After numerical stability and accuracy tests, the time step is selected to be $\dt  = \dx^2/32$ for the explicit Forward Euler method.  The discontinuity in the coefficient and the degeneracy make the time step criterion be more restrictive than it is for classical parabolic equations.  If stability is the only interest, and not accuracy, the coefficient in the time step can be increased, but the time step must still be on the same order of $\mathcal{O}(\dx^2)$.
\begin{figure}[H]
	\center
	\begin{subfigure}[H]{0.49\textwidth}  
			\includegraphics[width =\textwidth]{time_exactsam50.eps}
		\end{subfigure}
		\begin{subfigure}[H]{0.49\textwidth}  
			\includegraphics[width =\textwidth]{time_exactsam100.eps}
		\end{subfigure}
		\begin{subfigure}[H]{0.49\textwidth}  
			\includegraphics[width =\textwidth]{time_exactsam50_zoom.eps}
			\caption{$N = 50$ grid points}
		\end{subfigure}
		\begin{subfigure}[H]{0.49\textwidth}  
			\includegraphics[width =\textwidth]{time_exactsam100_zoom.eps}
			\caption{$N = 100$ grid points}
		\end{subfigure}
		\caption {Temporal profiles at position $x = 0.32$ with $\dt  = \dx^2/32$.}
		\label{fig:sam_exact_time}
\end{figure}
In Figure \ref{fig:sam_exact_time}, the numerical solution is plotted at the arbitrary position $x = 0.32$ over time.  By self-similarity of the solution \cite{vazquez2007, maddix_pme, ngo2016},  the temporal plots have the same profile for any $x$-coordinate that the front has passed through.  Figure \ref{fig:sam_exact_time} illustrates that the numerical solution with SAM has an accurate and non-oscillatory temporal profile, even on a coarse grid.  Figure \ref{fig:sam_exact_space} reveals that the shock location is accurate in space for this moving front problem.  The self-sharpening nature of the GPME \cite{maddix_pme, ngo2016, vazquez2007} is depicted, by the smooth lower corner in the initial condition in Figure \ref{IC} evolving into a sharp corner in Figure \ref{fig:sam_exact_space}.  Figure \ref{fig:sam_exact_space} also illustrates the sharp capture of the shock.  

%do not appear
%The results in this section show that the numerical artifacts discussed in Section \ref{lit} are not present with SAM.

			\begin{figure}[H]
		\center
		\begin{subfigure}[H]{.49\textwidth}  
			\includegraphics[width =\textwidth]{pos_exactsam50.eps}
			\caption{$N = 50$ grid points}
		\end{subfigure}
		\begin{subfigure}[H]{.49\textwidth}  
			\includegraphics[width =\textwidth]{pos_exactsam100.eps}
			\caption{$N = 100$ grid points}
		\end{subfigure}
		\caption{Spatial profiles at time $t = 0.05$ with $\dt  = \dx^2/32$.}
		\label{fig:sam_exact_space}
\end{figure}

	
%LAST SECTION
\section{SAM Numerical Results: Approximate Shock Location}
\label{shock_speed}
The method in the prior section can be extended to more general GPME problems, where the exact shock position is not known.
The finite speed of propagation property and theoretical speed of the front for the GPME, discussed in Section \ref{rh_cond}, can be utilized to numerically approximate the shock location. 
Again the fluxes in Eqns. \eqref{eq:fluxi} and \eqref{eq:fluxiplus} are used with the difference that $\dx^*(t)$ is numerically calculated.  In this section, we show that this approximation does not introduce numerical artifacts, such as temporal oscillations.  The convergence results are provided in \ref{app:conv}. %the shock position can be numerically approximated without the occurrence of temporal oscillations in the SAM solution. 
%Since the governing equation is of the GPME class, the theoretical speed of the front for $k_{\max} = 1$ and $k_{\min} = 0$ is given by the expression in Eqn. \eqref{eq:front_speed}.  Since the governing equation is of the GPME class, we may utilize the expression for the theoretical speed of the front from \cite{vazquez2007}.  This will be discretized and integrated in Section \ref{shock_speed} to approximate the front location.  
%Identifying the model problem of the GPME class provides the exact form of the shock velocity to discretize.


 We discretize $V$ in Eqn. \eqref{eq:front_speed} with upwinding for the derivative as 
\begin{equation}
	\hat{V} = -\frac{p_i - p_{i-1}}{\dx p_i}, 
	\label{eq:num_speed}
\end{equation}
 where $i$ is the index, such that $p_{i} \ge p^*  \ge p_{i+1}$.  Eqn. \eqref{eq:num_speed} can be interpreted as an approximation of the jump condition in Eqn. \eqref{eq:RH_pre}, where $p_L$ is approximated by $p_i$ and $p_R = 0$.  For problems where $p_R$ is not initially zero, such as the waiting time problem in Section \ref{lit}, we can also use a discrete approximation to Eqn. \eqref{eq:RH}.  Care must be taken numerically when approximating $p_R$ in Eqn. \eqref{eq:RH}, since $p_{i+1}$ is not guaranteed to be zero: all that is guaranteed is that $0 \le p_{i+1} \le p^*$.  In this case, we use $p_{i+2}$ to approximate $p_R$.
 
To obtain the numerical shock position, a simple time integration is implemented.  We let $\xi^n$ represent the approximate $x^*(t)$ at time step $n \ge 0$ and substitute 
\begin{equation}
	\dx^*(t) \approx \xi^n - x_i,
	\label{eq:deltax_star}
\end{equation}
into the expressions for the fluxes in Eqns. \eqref{eq:fluxi} and \eqref{eq:fluxiplus}.  We integrate the approximate shock speed in Eqn. \eqref{eq:num_speed} in time, using the following update
\begin{equation}
			\xi^{n+1} = \xi^n + \dt \hat{V}.
			\label{eq:xi}
\end{equation}
We assume that the initial position $\xi^0$ is known from the problem definition or can also be approximated. 

We can also approximate $\dx^*(t)$ using a level set method \cite{sethian88}. %The case of interest in the test problem, where $k(p)$ tends to zero is not addressed.  %  In these papers, the coefficient $k(p)$, representing the thermal diffusivity in these applications, is set to unity.  The case of interest where the coefficient tends toward zero is not addressed.  
In the level set method, the interface $x^*(t)$ is represented by the zero level set $\{x \hspace{.1cm}	 | \, \phi(x,t) = 0\}$ of a signed distance function $\phi(x,t)$.  %, such that $\phi(x,t)$.  %The signed distance function $\phi(x,t)$ is initialized the initial signed distance to $\Gamma$, negative inside a phase and positive outside of it.   
Using the velocity $V$ of the interface in Eqn. \eqref{eq:front_speed}, the evolution equation can be written in terms of $\phi(x,t)$ as
\begin{equation}
	\phi_t + V \cdot \nabla \phi = 0.
	\label{eq:levelset}
\end{equation}
The velocity can be extended to the entire domain or in a neighborhood of the interface by using velocity extension methods based on the Fast Marching Method \cite{chopp09}.  The level set equation \eqref{eq:levelset} can be solved numerically using Forward Euler in time and upwinding for the gradient, where $V$ is discretized as $\hat{V}$ in Eqn. \eqref{eq:num_speed}.  Since $\dx^*(t)$ is defined as the distance from the shock or interface $x^*(t)$ to the grid point $x_i$, $\dx^*(t)$ can be calculated by $|\phi(x_i)|$.   Solving Eqn. \eqref{eq:levelset} for $\phi_i^n$ and setting $\dx^*(t) \approx |\phi_i^n|$ gives the same results as solving Eqns. \eqref{eq:deltax_star}-\eqref{eq:xi}.  %derived by chain rule can use normal velocity to get 


\begin{figure}[H]
		\center
		\begin{subfigure}[H]{.49\textwidth}  
			\includegraphics[width =\textwidth]{shockpos_new.eps}
			\caption{Shock position}
			\label{fig:shock_pos}
		\end{subfigure}
		\begin{subfigure}[H]{.49\textwidth}  
			\includegraphics[width =\textwidth]{shockpos_err.eps}
			\caption{Shock position error relative to $\dx$}
			\label{fig:shock_pos_err}
		\end{subfigure}
		\caption{Comparison of the temporal profiles of the shock position and relative error for $\dx = 0.04$ and $\dt  = \dx^2/32$.}
		\end{figure}

Figures \ref{fig:shock_pos} and  \ref{fig:shock_pos_err} illustrate the long-time behavior of the numerical shock position evolution.  Figure \ref{fig:shock_pos} displays that the numerical shock position aligns with the exact shock position as a function of time, and that the shock position evolution is accurately captured without oscillations.  Both plots verify the numerical implementation, and show that there is no significant accuracy loss in estimating the shock position using Eqn. \eqref{eq:xi}.
  

\begin{figure}[H]
		\center
		\begin{subfigure}[H]{.49\textwidth}  
			\includegraphics[width =\textwidth]{pos_sam50.eps}
			\caption{$N = 50$ grid points}
		\end{subfigure}
		\begin{subfigure}[H]{.49\textwidth}  
			\includegraphics[width =\textwidth]{pos_sam100.eps}
			\caption{$N = 100$ grid points}
		\end{subfigure}
		\caption{Spatial profiles at time $t = 0.05$ with $\dt  = \dx^2/32$.}
		\label{fig:sam_space}
		\end{figure}
  The results in Figures \ref{fig:sam_space} and \ref{fig:sam_time} on the same test case from the prior section show that there is no significant change in the behavior of the SAM solution with the approximate shock speed than with the exact shock speed in the prior section.  %Figure \ref{fig:sam_space} shows that the shock position capture is still accurate, and Figure \ref{fig:sam_time} shows that there are no spurious temporal oscillations nor any other numerical artifacts.  	
		\begin{figure}[H]
		\center
		\begin{subfigure}[H]{0.49\textwidth}  
			\includegraphics[width =\textwidth]{time_sam50.eps}
		\end{subfigure}
		\begin{subfigure}[H]{0.49\textwidth}  
			\includegraphics[width =\textwidth]{time_sam100.eps}
		\end{subfigure}
		\begin{subfigure}[H]{0.49\textwidth}  
			\includegraphics[width =\textwidth]{time_sam50_zoom.eps}
			\caption{$N = 50$ grid points}
		\end{subfigure}
		\begin{subfigure}[H]{0.49\textwidth}  
			\includegraphics[width =\textwidth]{time_sam100_zoom.eps}
			\caption{$N = 100$ grid points}
		\end{subfigure}
		\caption {Temporal profiles at position $x = 0.32$.  with $\dt  = \dx^2/32$. }
		\label{fig:sam_time}
\end{figure}

  
%SAM with an approximate shock position via front-tracking results in a fully numerical method, without any prior knowledge of an analytical solution.

Figures \ref{wait_space}-\ref{wait_timezoom} show results for SAM for the waiting phenomenon problem discussed in Section \ref{rh_cond}.  For this initial condition, the exact solution is not known.  The figures show that the SAM solution is accurate and does not possess the numerical artifacts shown in Figure \ref{wait_time}.      %Approximating the shock position with SAM results in a numerical solution, without any prior knowledge of an analytical solution.	
 %   ADD: description of test problem with different initial condition exact solution not known use jump condition $F(p_R)$ is not 0 at early times when sharpens
    
  \begin{figure}[H]
		\center
		\begin{subfigure}[H]{0.33\textwidth}  
			\includegraphics[width =\textwidth]{plinearic_possam100.eps}
			\caption{Spatial profiles}
			\label{wait_space}
		\end{subfigure}
		\begin{subfigure}[H]{0.33\textwidth}  
			\includegraphics[width =\textwidth]{pclinic_sam.eps}
			\caption{Temporal profiles at position $x = 0.32$.}
			\label{wait_time2}
		\end{subfigure}
		\begin{subfigure}[H]{0.33\textwidth}  
			\includegraphics[width =\textwidth]{pclinic_samzoom.eps}
			\caption{Zoomed in temporal profiles.}
			\label{wait_timezoom}
		\end{subfigure} %how to change caption!
		\caption{SAM numerical solution with $N = 100$ grid points and $\dt = \dx^2/32$ for the waiting time phenomenon example from Figure \ref{wait_time}.}
\end{figure}


%\cite{brattkus92}%lead into plots %no significant change from the prior section

%can add in exact shock location plot

  %This reveals that removing the temporal oscillations requires $F_j^+ \neq F_{q+1}^-$ and that it is sufficient to approximate the shock location.


%SAM as an average-based approach
\section{Average-based approaches} %summary: static averages no information about shock does not change as shock moves through cell
\label{static_avg} %add harmonic part up here or just say confirmed results with van der meer
%In Section \ref{sam_exact}, we formulated SAM as a shock tracking method.  In this section, we show results for commonly used averaging methods from the literature, discussed in Section \ref{lit}.  We also show how SAM can be expressed in this averaging framework to explain the cause of the temporal oscillations seen in the porous media literature.  %from the literature and discussed in
In the prior sections, we showed that we developed an accurate method, SAM, for the Stefan problem.  %One speculation we have is that SAM and level set methods in other applications work well because the shock location is integrated into the scheme.  
The main goal of the paper is to understand why the artifacts are occurring with finite volume averaged-based methods from the literature.  We will now recast SAM in this framework to shed light on what is happening in these methods that are used frequently in the porous media community.

We implement the finite volume average-based methods using the Forward in Time, Central in Space (FTCS) discretization on a uniform Cartesian grid, as is commonly done in the porous media literature.  The numerical fluxes are given by 
\begin{equation}
	 F_{j}^+ = -k_{j+1/2}\frac{p_{j+1}-p_j}{\dx},
	 \label{eq:out_flux_arith}
\end{equation}
and
\begin{equation}
	F_{j}^- = -k_{j-1/2}\frac{p_{j}-p_{j-1}}{\dx},
	\label{eq:in_flux_arith}
\end{equation}
%are given byEqns. \eqref{eq:stand_outflux} -  \eqref{eq:stand_influx} 
for an arbitrary average $k_{j+1/2}$ of the neighboring coefficients at the cell face $x_{j+1/2}$ for all $j = 1, \dots, N$.  For verification, we successfully repeated the test case presented for the arithmetic and integral average in \citet{vandermeer2016}.  %In this section, we again use the test case in Figure \ref{IC} to see whether a smoother initial condition than the one in \cite{vandermeer2016} would still result in artifacts.

%SAVE FOR thesis!!!
%In \cite{vandermeer2016}, the results for the harmonic average are not shown, since the authors claim that the harmonic average is not suitable for the governing equation with near-zero coefficients \cite{lipnikov2016}.  
%For the results shown in this section, the discontinuous initial condition in \cite{vandermeer2016} is replaced with the curve in Figure \ref{IC} to confirm that the discontinuity in the initial condition is not the source of the temporal oscillations.

%add results verified with Jakolein's in her paper for arithmetic and integral average
	%with Forward in Time, Central in Space (FTCS) discretizations.
	%The effect of the averages on temporal oscillations is examined.  
%, which are of interest to us.  We will show/ using?

\subsection{Arithmetic and Harmonic Averages}
In the FTCS finite volume scheme with arithmetic %and arithmetic averaging contains no information about the shock location.  
%The FTCS scheme with harmonic %paragraph with formula and main results
 \begin{equation}
 	k^A_{j+1/2} = \frac{k_j + k_{j+1}}{2}, %matches Jakolein's case
	\label{eq:arith}
\end{equation}
 and harmonic
\begin{equation}
	k^H_{j+1/2} = \frac{2k_jk_{j+1}}{k_j + k_{j+1}},
	\label{eq:harm}
\end{equation}
averaging, %, no special care is taken to compute the pressure gradient across the interface.   %not correct because should be 0 or 1 picewise constant so now fluxes to left or right of shock-wrong not correct k(p) value known if know shock position to left and right of shock
the fluxes in and out of the control volumes surrounding the shock are discretized using Eqns. \eqref{eq:out_flux_arith}-\eqref{eq:in_flux_arith}. 
The corresponding arithmetic fluxes are given by
\begin{equation}
	{F^+_{i}}^A =  {F^-_{i+1}}^A = -k^A_{i+1/2} \frac{p_{i+1} - p_i}{\dx}.
	\label{eq:arith_flux}
\end{equation}
The analogous expression holds for the harmonic fluxes ${F^+_{i}}^H$ and ${F^-_{i+1}}^H$, where the arithmetic average $k^A_{i+1/2}$ in Eqn. \eqref{eq:arith_flux} is replaced with the harmonic average $k^H_{i+1/2}$.
 
%harmonic
 The temporal plots at $x = 0.32$ in Figure \ref{fig:arith_time} reveal that the numerical interface with harmonic averaging is locked, that is, does not move at all, and so does not advance to this position.  Figure \ref{fig:space} shows the spatial evolution and we again see the locking with harmonic averaging.  This numerical solution in Figure \ref{fig:space} evolves to a step function with values at 1 and 0.  Eqn. \eqref{eq:harm} explains this locking numerical artifact, since $k^H_{j+1/2} = 0$ when either coefficient is zero.  %This locking property of the harmonic average is also known as blocking.  
 Although here locking is seen as a drawback, there are situations, where it can be desirable, such as in variable coefficient problems, $p_t  = \nabla \cdot (k(x) \nabla p)$, when the interface separates a permeable and impermeable material.  The arithmetic average is known to cause leakage across such interfaces.  We could relax $k_{\min} = 0$ to $k_{\min} = \eps$ for some small $\eps > 0$.  Figure \ref{spatial_01} shows that for $\eps = 0.01$, the numerical solution with harmonic averaging no longer completely locks, but is still lagging behind the true front location.  From Eqn. \eqref{eq:harm}, we see that $k^H_{i+1/2}$ favors the smaller coefficient and is approximately equal to $2\eps$, as the shock moves through the interval $[x_i,x_{i+1}]$.  The constant and small averaged $k$ value at the cell face results in a solution whose numerical speed is too slow.  Grid dependent oscillations with much larger amplitudes than those in the solution with arithmetic averaging are present in its temporal profile in Figures \ref{time_01}-\ref{time_01_zoom}.  In \cite{vandermeer2016}, the harmonic average is not considered, since it is known in the literature to behave poorly for the GPME with near-zero coefficients.  In the rest of this subsection, we focus on the behavior of the solution with arithmetic averaging.
 \begin{figure}[H]
	\center
	\begin{subfigure}[H]{0.49\textwidth}  
			\includegraphics[width =\textwidth]{time50.eps}
		\end{subfigure}
		\begin{subfigure}[H]{0.49\textwidth}  
			\includegraphics[width =\textwidth]{time100.eps}
		\end{subfigure}
		\begin{subfigure}[H]{0.49\textwidth}  
			\includegraphics[width =\textwidth]{time50_zoom.eps}
			\caption{$N = 50$ grid points}
			\label{fig:arith_res50}
		\end{subfigure}
		\begin{subfigure}[H]{0.49\textwidth}  
			\includegraphics[width =\textwidth]{time100_zoom.eps}
			\caption{$N = 100$ grid points}
			\label{fig:arith_res100}
		\end{subfigure}
		\caption{Temporal profiles at position $x = 0.32$  with $\dt  = \dx^2/32$.} %The numerical solutions with $N = 50$ and $N = 100$ grid points are in the first and second column, respectively.}
		\label{fig:arith_time}
\end{figure}
The numerical solutions depicted in Figures \ref{fig:arith_time} and \ref{fig:space} show results for the arithmetic average.  We see that the numerical solution with arithmetic averaging does not sharply capture the shock.  Figure \ref{fig:arith_time} shows spurious oscillations of low and high frequency.  The low frequency oscillations are grid dependent.  The high frequency oscillations are apparent in the zoomed in region around the solution shock value of $p^* = 0.5$.  Figure \ref{fig:space} shows that the solution with arithmetic averaging does not have spatial oscillations.  In the temporal view, oscillations are present.  %The numerical solutions satisfy the total variation diminishing (TVD) property \cite{shu98} that was designed to eliminate spurious spatial oscillations from dispersion in hyperbolic conservation laws.  The total variation $TV(p^n) = \sum_i |p^n_{i+1} - p^n_i| = 1$ for all $n$ for both the arithmetic and harmonic solutions, matching that of the true solution.  To determine the presence of the temporal oscillations, another metric must therefore be used. %also explains why tad schemes must not eliminate the temporal oscillations
  \begin{figure}[H]
		\center
		\begin{subfigure}[H]{.49\textwidth}  
			\includegraphics[width =\textwidth]{pos50.eps}
			\caption{$N = 50$ grid points} %FIX CAPTION
		\end{subfigure}
		\begin{subfigure}[H]{.49\textwidth}  
			\includegraphics[width =\textwidth]{pos100.eps}
			\caption{$N = 100$ grid points} %FIX CAPTION
		\end{subfigure}
				\caption{Spatial profiles at time $t = 0.05$ with $\dt  = \dx^2/32.$} %The numerical solutions with $N = 50$ and $N = 100$ grid points are in the first and second column, respectively.}
		\label{fig:space} %must be after caption
\end{figure} %arithmetic

\begin{figure}[H]
		\center
		\begin{subfigure}[H]{.32\textwidth}  
			\includegraphics[width =\textwidth]{pos200_kmin01.eps}
			\caption{Spatial profiles at $t = 0.05$}
			\label{spatial_01}
		\end{subfigure}
		\begin{subfigure}[H]{.32\textwidth}  
			\includegraphics[width =\textwidth]{time200_kmin01.eps}
			\caption{Temporal profile at $x = 0.32$}
			\label{time_01}
		\end{subfigure}
		\begin{subfigure}[H]{0.32\textwidth}  
			\includegraphics[width =\textwidth]{time200_zoomkmin01.eps}
			\caption{Zoomed in temporal profiles}
			\label{time_01_zoom}
		\end{subfigure}
	\caption{Solution profiles with $N = 200$ grid point and $\dt  = \dx^2/32$.}
	\label{fig:time_01}
\end{figure}

There are two main differences in the artifacts for the continuous and discontinuous GPME.  Here, for the discontinuous case, there are temporal oscillations of low and high frequencies for the arithmetic average.  For the continuous GPME, temporal oscillations were only observed in solutions with harmonic averaging  \cite{maddix_pme}. %The main difference in the results in this paper is the presence of the temporal oscillations in numerical solutions with arithmetic averaging.  
The lower frequency grid dependent oscillations were analyzed in \cite{maddix_pme} for discretizations with harmonic averaging.  The high frequency oscillations are an additional numerical artifact with arithmetic averaging that were not observed for the harmonic average in \cite{maddix_pme}.  %In \citet{maddix_pme}, we looked at numerical artifacts associated with arithmetic and harmonic averaging for the GPME with continuous coefficients.  
 %Modified Equation Analysis was utilized to identify an anti-diffusive term contributing to the temporal oscillations in solutions with harmonic averaging.  This approach is not applicable to the foam model prototype discussed in this paper, since $k(p)$ in Eqn. \eqref{eq:GPME} is not differentiable with respect to $p$.  %In  paper 1 , smooth problem could explain with MEA here main difference is permeability discontinuous with respect to p.  discussed in intro see if should repeat

%cite rossen citation
The low frequency temporal oscillations have been studied in the literature \cite{maddix_pme, vandermeer2016, zanganeh14, rossen99}.  It has been shown in these works that the temporal oscillations occur as the front crosses a grid cell.  The spurious oscillations then do not vanish with grid refinement.  The frequency increases and the amplitude decreases, as the number of grid points increases.  The differences in frequency and amplitude are illustrated by the solutions for $N = 50$ in Figure \ref{fig:arith_res50} and $N = 100$ in Figure \ref{fig:arith_res100}.  Another observed characteristic of these oscillations in \cite{vandermeer2016, zanganeh14, rossen99} is that the amplitude decreases as shock moves further away from the solution probe point.  This is because the possible error in the shock position relative to the distance from the probe point to the shock decreases.  %claims harmonic not suitable small eps and independent of choice of flux discretization, this will lead to oscillatory behavior.

The high frequency oscillations were also observed in \cite{vandermeer2016}.  Figures \ref{fig:arith_oscill_50}-\ref{fig:arith_oscill_100} illustrate the temporal profile on a shorter time interval, where the high frequency oscillations occur for $p$ values near $p^* = 0.5$.  We see that $p_{i+1}$ is slowly increasing, until it crosses the threshold at $p^* = 0.5$.  The corresponding $k_{i+1}$ then jumps from $k_{\min} = 0$ to $k_{\max} = 1$, according to the model for $k(p)$ in Eqn. \eqref{eq:discont_k}.  The arithmetic average $k^A_{i+3/2}$ in Eqn. \eqref{eq:arith} jumps from 0 to 0.5, as illustrated in Figures \ref{fig:arith_oscill_50}-\ref{fig:arith_oscill_100}.  The increase in $k^A_{i+3/2}$ causes $p$ to drop below $p^* = 0.5$ and $k^A_{i+3/2}$ to jump back down to 0 at the next time step.  The cycle then repeats itself.

%The linear increase in the pressure, observed in Figures \ref{fig:arith_oscill_50} and \ref{fig:arith_oscill_100} can also be explained by Eqn. \eqref{eq:scheme}.  %The first term in Eqn. \eqref{eq:scheme} is positive and the second term is 0, since $k^A_{i+3/2} = 0$ at these times.  
 \begin{figure}[H]
		\center
		\begin{subfigure}[H]{0.49\textwidth}  
			\includegraphics[width =\textwidth]{arith_oscill50.eps}
			\caption{$\dt  = \dx^2/32$}
			\label{fig:arith_oscill_50}
		\end{subfigure}
		\begin{subfigure}[H]{0.49\textwidth}  
			\includegraphics[width =\textwidth]{arith_oscill50dthalf.eps}
			\caption{$\dt  = \dx^2/64$}
			\label{fig:arith_oscill_100}
		\end{subfigure}
						\caption{Zoomed in region of the high frequency temporal oscillations near $p^* = 0.5$ in the left cell $p_{i+1}$ for the arithmetic average $k^A_{i+3/2}$ with $N = 50$ grid points at position $x = 0.32$.} 
\end{figure}
\begin{figure}[H]
		\center
		\begin{subfigure}[H]{0.49\textwidth}  
			\includegraphics[width =\textwidth]{arith_oscill50left.eps}
			\caption{$\dt  = \dx^2/32$}
			\label{fig:arith_oscill_50left}
		\end{subfigure}
		\begin{subfigure}[H]{0.49\textwidth}  
			\includegraphics[width =\textwidth]{arith_oscill50leftdthalf.eps}
			\caption{$\dt  = \dx^2/64$}
			\label{fig:arith_oscill_100left}
		\end{subfigure}
						\caption{Zoomed in region of the solution in the right cell $p_{i+2}$ for the arithmetic average $k^A_{i+3/2}$ with $N = 50$ grid points at position $x = 0.34$.} %The numerical solutions with $N = 50$ and $N = 100$ grid points are in the first and second column, respectively.}
		 %must be after caption
\end{figure}

We use the FTCS discretization to explain the effect of the jump in $k^A_{i+3/2}$ on the $p$ value at the next time step.  We discretize the semi-discrete equation \eqref{eq:fv_discret} with Forward Euler in time and substitute in the arithmetic fluxes to obtain
 \begin{equation}
 	\begin{aligned}
%	p_{i+1}^{n+1} &=  %p^n_{i+1} + \frac{\dt}{\dx}\bigg(F_{i+1}^- -F_{i+1}^+\bigg) 
	p^{n+1}_{i+1} = p^n_{i+1} + \frac{\dt}{\dx}\bigg(-k^A_{i+1/2}\Big[\frac{p^n_{i+1}-p^n_{i}}{\dx}\Big] + k^A_{i+3/2}\Big[\frac{p^n_{i+2}-p^n_{i+1}}{\dx}\Big]\bigg),
		\label{eq:scheme}
	\end{aligned}
\end{equation}
where $n$ is a time step before the solution drop and $p^n_{i+1} \ge 0.5$.
The first term in Eqn. \eqref{eq:scheme} is small in magnitude, since $p^n_i \approx 0.5$.  The second term in Eqn. \eqref{eq:scheme} has a larger gradient of $p$, since $p^n_{i+2} \approx 0$.  It is also negative, and so the increase in $k^A_{i+3/2}$ results in the drop in $p^{n+1}_{i+1}$.  At the following time steps, $k^A_{i+3/2} = 0$, until the solution crosses the threshold again.  %matches pressure drop in cell where outflow increase and makes pressure increase in cell where influx increases


%The numerical solutions with $N = 50$ and $N = 100$ grid points are in the first and second column, respectively.}
As $p_{i+1}$ is increasing and is below the threshold, $k^A_{i+1/2}$ is fixed at $0.5$, regardless of the shock position in the cell.  %Eqn. \eqref{eq:scheme} reduces to a linear equation with respect to $\dt$ in these time intervals.  
From Eqn. \eqref{eq:scheme}, we see that the positive quantity $\dt/(2\dx^2)[p_i^n - p_{i+1}^n]$ is added to the current $p$ value at each time step.  The constant arithmetic average at the interface $x_{i+1/2}$ allows the solution to artificially increase above $p^* = 0.5$, resulting in the high frequency oscillations. %The first term is positive and $k^A_{i+1/2} = 0.5$, and so we have linear growth of the pressure until it reaches $p^*$ and the cycle then repeats itself.  

Figures \ref{fig:arith_oscill_50}-\ref{fig:arith_oscill_100} also show that the high frequency oscillations are dependent on the time step size.  As $\dt$ is decreased by half, the number of oscillations in Figure \ref{fig:arith_oscill_100} doubles from those in Figure \ref{fig:arith_oscill_50}.  The amplitude of the high frequency oscillations also decreases, as $\dt$ decreases.  The amplitude decrease is expected from Eqn. \eqref{eq:scheme}, since the additional term is proportional to $\dt$.  In \cite{vandermeer2016}, it is observed that as the time step is decreased, the arithmetic average solution in the high frequency region reaches a constant state at $p^* = 0.5$, rather then converging to the true solution. 

Figures \ref{fig:arith_oscill_50}-\ref{fig:arith_oscill_100} display the profile of $p_{i+1}$ over a time window when the shock is in the interval $[x_i,x_{i+1}]$.  In Figures  \ref{fig:arith_oscill_50left}-\ref{fig:arith_oscill_100left}, we now look at the profile of $p_{i+2}$ over the same time window to see the effect of the jump in $k^A_{i+3/2}$ on the neighboring cell.  Figures \ref{fig:arith_oscill_50left}-\ref{fig:arith_oscill_100left} show that the profile of $p_{i+2}$ is an increasing piecewise constant.  Since $p_{i+2}$ and $p_{i+3}$ are both less than $p^*$, $k^A_{i+5/2} = 0$ and the solution update for $p^{n+1}_{i+2}$ reduces to
 \begin{equation}
 	\begin{aligned}
%	p_{i+1}^{n+1} &=  %p^n_{i+1} + \frac{\dt}{\dx}\bigg(F_{i+1}^- -F_{i+1}^+\bigg) 
%	p^n_{i+1} + \frac{\dt}{\dx}\bigg(-k^A_{i+1/2}\Big[\frac{p^n_{i+1}-p^n_{i}}{\dx}\Big] + k^A_{i+3/2}\Big[\frac{p^n_{i+2}-p^n_{i+1}}{\dx}\Big]\bigg), 
%p_{i+2}^{n+1} &=  %p^n_{i+1} + \frac{\dt}{\dx}\bigg(F_{i+1}^- -F_{i+1}^+\bigg) 
%	p^n_{i+2} + \frac{\dt}{\dx}\bigg(-k^A_{i+3/2}\Big[\frac{p^n_{i+2}-p^n_{i+1}}{\dx}\Big] + k^A_{i+5/2}\Big[\frac{p^n_{i+3}-p^n_{i+2}}{\dx}\Big]\bigg), 
p_{i+2}^{n+1} &=  %p^n_{i+1} + \frac{\dt}{\dx}\bigg(F_{i+1}^- -F_{i+1}^+\bigg) 
	p^n_{i+2} + k^A_{i+3/2}\bigg(p^n_{i+1}-p^n_{i+2}\bigg)\frac{\dt}{\dx^2}.
		\label{eq:scheme2}
	\end{aligned}
\end{equation}
Eqn. \eqref{eq:scheme2} shows that at the times when $p^n_{i+1} < p^* = 0.5$, $k^A_{i+3/2} = 0$, and so $p_{i+2}^{n+1} = p_{i+2}^{n}$.  Otherwise, at the times when $k^A_{i+3/2}$ jumps to 0.5, the solution increases proportional to $\dt$.  As expected, the increase in out-flux causes the solution to decrease in the left cell (Figures \ref{fig:arith_oscill_50}-\ref{fig:arith_oscill_100}) and the same increase in in-flux causes the solution in the right cell to increase (Figures \ref{fig:arith_oscill_50left}-\ref{fig:arith_oscill_100left}). %causes the pressure to increase artificially early

%probably cut below
%The results in this subsection demonstrate that standard use of the harmonic and arithmetic averages, commonly used in finite volume methods with variable coefficients, give undesirable numerical artifacts. %do not perform well for this degenerate parabolic problem. %increasing too early

%cite Figur encumber for this
%Figure \ref{eq:stand_fluxes} shows that the numerical solution with arithmetic averaging aligns with the reference solution in space, 
 %The grid size dependent temporal oscillations occur even when the coefficient is nonzero.  
 %Unlike for the PME in \cite{maddix_pme}, the discretization with arithmetic averaging leads to high frequency oscillations around the pressure value at the shock ($p^* = 0.5$).  



%\vspace{-.5cm}
%This is an inherent challenge of the model that the averaging scheme cannot resolve on its own without shock detection built into the scheme. 

%combine paragraphs and cut

%Since the numerical solution with harmonic averaging locks for $k_{\min} = 0$, we also test the solution with $k_{\min} = \eps \rightarrow 0$ for $\eps = 0.01$ to observe the temporal oscillations present in time.  The results in Figures \ref{N=200} and \ref{N=200_time} show that the numerical solution with harmonic averaging is lagging  behind the true front location.  This reveals that even in the less extreme case where $k_{\min}$ is not equal to 0, both numerical solutions with harmonic and arithmetic averaging have temporal oscillations. 
%%possibly group into one image
%\begin{figure}[H]
%		\center
%		\begin{subfigure}[H]{.49\textwidth}  
%			\includegraphics[width =\textwidth]{Results/pos200_kmin01.eps}
%		\end{subfigure}
%			\caption {Spatial profiles at time $t = 0.05$ with $\dt  = \dx^2/32$ and $N = 200$ grid points.}
%			\label{N=200}
%\end{figure}
%\begin{figure}[H]
%		\center
%		\begin{subfigure}[H]{.49\textwidth}  
%			\includegraphics[width =\textwidth]{Results/time200_kmin01.eps}
%		\end{subfigure}
%		\begin{subfigure}[H]{0.49\textwidth}  
%			\includegraphics[width =\textwidth]{Results/time200_zoomkmin01.eps}
%		\end{subfigure}
%	\caption{Temporal profiles at position $x = 0.32$ with $\dt  = \dx^2/32$ and $N = 200$ grid points.}
%	\label{N=200_time}
%\end{figure}
%remove summary too early?
%can move in between images

%The problem occurs in the assumption that it is all about the choice of average, rather than incorporating the shock position detection.

%Subsection on integral average: finished arithmetic and harmonic!-integral average is monotonic oscillations contrast to arithmetic and harmonic
%of second derivative? make title
\subsection{Integral Average}
\citet{vandermeer2016} developed the integral average 
\begin{equation}
	k^{I}_{j+1/2} =  \frac{ \int_{p_j}^{p_{j+1}} k(\tilde{p})d\tilde{p}}{p_{j+1}-p_j},
	\label{eq:int_avg}
\end{equation}
which is effective in reducing the numerical artifacts.    The integral average is derived by expressing the coefficient $k(p) = \Phi_p$ in Eqn. \eqref{eq:GPME} to obtain %explain a bit more how derived in paper
\[
	p_t =  \nabla \cdot (\Phi_p \nabla p).
	\label{eq:int_form}
\] %talk about central 1-2 1 from GPME equation
	%\begin{equation}
		%		k^{I}_{q+1/2} = \frac{\Phi(p_{q+1}) - \Phi(p_j)}{p_{q+1}-p_j} %= \frac{ \int_0^{p_{q+1}} k(\tilde{p})d\tilde{p} -  \int_0^{p_j} k(\tilde{p})d\tilde{p}}{p_{q+1}-p_j}  
	%			= \frac{ \int_{p_j}^{p_{q+1}} k(\tilde{p})d\tilde{p}}{p_{q+1}-p_j},
	%\end{equation}
Discretizing $\Phi_p$ directly with central differences at the cell face $x_{j+1/2}$ gives 
%Central differences are used to compute the derivative of $\Phi$ with respect to $p$ at the cell face $x_{j+1/2}$, which gives 
\begin{equation}
	k^{I}_{j+1/2} = \frac{\Phi(p_{j+1}) - \Phi(p_j)}{p_{j+1} - p_{j}}.
	\label{eq:int_central}
\end{equation} %more details on how described in Jakoleins
%Hence, the integral average derives its name from the fact that it is defined by a definite integral. 
Using the definition of $\Phi(p) = \int_0^p k(\tilde{p})d\tilde{p}$, Eqn. \eqref{eq:int_central} simplifies to Eqn. \eqref{eq:int_avg}.

%As a side note, we show that we can express 
%The scheme with integral averaging can also be derived in terms of a well-known discretization to provide directions for further development.  We substitute the integral average $k^I_{j+1/2}$ in Eqn. \eqref{eq:int_avg} into the finite volume semi-discrete discretization to obtain
%\begin{equation}
%	\begin{aligned}
%	\dx\frac{dp_j}{dt} 
%					&= -\frac{ \int_{p_{j-1}}^{p_{j}} k(\tilde{p})d\tilde{p}}{p_j-p_{j-1}}\frac{p_{j} - p_{j-1}}{\dx} + \frac{ \int_{p_{j}}^{p_{j+1}} k(\tilde{p})d\tilde{p}}{p_{j+1}-p_{j}}\frac{p_{j+1} - p_{j}}{\dx} \\
%					&= \frac{\int_0^{p_{j-1}} k(\tilde{p})d\tilde{p} - 2 \int_0^{p_j} k(\tilde{p})d\tilde{p} +\int_0^{p_{j+1}} k(\tilde{p})d\tilde{p}}{\dx} \\
%					& = \frac{\Phi(p_{j-1}) - 2\Phi(p_j) + \Phi(p_{j+1})}{\dx}, %\iff \\
%		\end{aligned}
%		\nonumber
%	\end{equation}
% for $2 \le j \le N$.
%Dividing both sides by $\dx$ recovers the second order central discretization of the one-dimensional laplacian on a uniform Cartesian grid.

The numerical solution with the integral average is provided in Figures \ref{fig:int_time} and \ref{fig:int_space}.  The improvement with the integral average is clear.  It does not introduce high frequency oscillations near $p^*$.  %The high frequency oscillations near $p^*$ present with the arithmetic average in Figure \ref{fig:arith_time} are not present with the integral average in Figure \ref{fig:int_time}.  
Although the low frequency, grid-dependent oscillations remain, the amplitude of these oscillations is smaller than those with arithmetic averaging.  The numerical solution is now monotonically increasing in time \cite{vandermeer2016}, matching the behavior of the true solution.  %Figure \ref{fig:sam_exact_time} shows that the SAM numerical solution sharply captures the discontinuity in the temporal plot, whereas the numerical solution with integral averaging in Figure \ref{fig:int_time} is sloped.  
The diffusive shock profile with the integral average is also illustrated in the spatial results in Figure \ref{fig:int_space}.%the integral average can remove the high frequency oscillations but not get rid of the low frequency oscillations.
\begin{figure}[H]
\center
	\begin{subfigure}[H]{0.49\textwidth}  
			\includegraphics[width =\textwidth]{time_int50.eps}
		\end{subfigure}
		\begin{subfigure}[H]{0.49\textwidth}  
			\includegraphics[width =\textwidth]{time_int100.eps}
		\end{subfigure}
		\begin{subfigure}[H]{0.49\textwidth}  
			\includegraphics[width =\textwidth]{time_int50_zoom.eps}
			\caption{$N = 50$ grid points}
		\end{subfigure}
		\begin{subfigure}[H]{0.49\textwidth}  
			\includegraphics[width =\textwidth]{time_int100_zoom.eps}
			\caption{$N = 100$ grid points}
		\end{subfigure}
		\caption{Temporal profiles at position $x = 0.32$  with $\dt  = \dx^2/32$.}
		\label{fig:int_time}
\end{figure} 

To explain why the integral average outperforms the arithmetic and harmonic averages, we write it in an alternate form than presented in \cite{vandermeer2016}.  The integral average $k^I_{i+1/2}$ in Eqn. \eqref{eq:int_avg} can be broken up at $p^*$ into two separate integrals to obtain %.  The precise $k(p)$ value is known for each of these integrals and for the shock cell we have %The values for the integral average for the governing equation defined at each cell face are given below as
\begin{equation}
k^I_{i+1/2} =
	%\begin{dcases}
  		% k_{\max},           & 1 \le j \le i - 1,  \\
    		% \frac{ \int_{p_{q+1}}^{p^*} k(\tilde{p})d\tilde{p}  +  \int_{p^*}^{p_j} k(\tilde{p})d\tilde{p} }{p_j - p_{q+1}} = 
		\frac{k_{\min}(p_{i+1}-p^*) + k_{\max}(p^*-p_i)}{p_{i+1}-p_i}, %& j = i, \\
		%  k_{\min}, & i+1 \le j \le N.
	%\end{dcases}
	\label{eq:int_avg_val} %distance to shock is dx
\end{equation}
in the shock interval.  
It then contains some information about the shock, as encoded in the bounds of the integral. 
Unlike the harmonic and arithmetic averages, the integral average monotonically increases as the shock advances through the interval $[x_i, x_{i+1}]$.  

The corresponding continuous flux is given by 
%n particular, 
\begin{equation}
	%F_j^+ = F_{q+1}^- 
	{F^+_{i}}^I = {F^-_{i+1}}^I = -\frac{k_{\min}(p_{i+1}-p^*) + k_{\max}(p^*-p_i)}{\dx}.
	\label{eq:int_flux}
\end{equation} %The distance from $x_i$ to the shock is then assumed to be fixed of size $\dx$. 
%no special care is taken to compute the pressure gradient across the interface.
Using this description, we can see that by utilizing $p^*$, the integral flux avoids computing the undefined gradient of $p$ across the jump, as done in the schemes with arithmetic and harmonic averaging.  % where no special care is taken to compute this gradient across the interface..  
It is this incorporation of $p^*$ into the scheme that prevents the high frequency oscillations seen with arithmetic averaging, which explains the improved behavior.
\begin{figure}[H]
		\center
		\begin{subfigure}[H]{.49\textwidth}  
			\includegraphics[width =\textwidth]{pos_int50.eps}
			\caption{$N = 50$ grid points}
		\end{subfigure}
		\begin{subfigure}[H]{.49\textwidth}  
			\includegraphics[width =\textwidth]{pos_int100.eps}
			\caption{$N = 100$ grid points}
		\end{subfigure}
		\caption{Spatial profiles at time $t = 0.05$ with $\dt  = \dx^2/32$.}
		\label{fig:int_space}
\end{figure} 

As seen, the integral average does not remove the low frequency oscillations.  %not all the temporal oscillations are removed, as illustrated by the presence of the grid-dependent temporal oscillations in Figure \ref{fig:int_time}.  
This is because although $p^*$ is incorporated in the flux in Eqn. \eqref{eq:int_flux}, $x^*(t)$ is not.  In the case, where $k_{\min} = 0$, ${F^+_{i}}^I = {F^-_{i+1}}^I$ in Eqn. \eqref{eq:int_flux} appears to be similar to $F_{i}^+$ in Eqn. \eqref{eq:fluxi}.  The difference occurs in the denominator, where the relative shock position $\dx^*(t)$ detected in SAM is replaced with $\dx$ in the integral flux.  The integral flux is then assuming that the distance to the shock is fixed of size $\dx$.


 \begin{figure}[H]
		\center
		\begin{subfigure}[H]{0.49\textwidth}  
			\includegraphics[width =\textwidth]{time_intamr100.eps}
		\end{subfigure}
		\begin{subfigure}[H]{0.49\textwidth}  
			\includegraphics[width =\textwidth]{time_intamr100_zoom.eps}
		\end{subfigure}
		\caption{Temporal profiles at position $x = 0.32$ with $\dt  = \dx^2/32$ and $N = 100$ coarse grid points and 10 inner grid points.}
		\label{fig:AMR}
\end{figure}
To see what happens with the temporal oscillations under grid refinement, we also implement the scheme with integral averaging and Adaptive Mesh Refinement (AMR) \cite{berger84}.  We define a coarse mesh away from the shock and a fine mesh near the shock.  The fine mesh travels with the shock as it moves.  In Figure \ref{fig:AMR}, a coarse mesh size of $\dx = 1/100$ is utilized.  Within the coarse mesh cells surrounding the shock, a fine inner mesh size of $\dx_{\text{inner}} = 1/10$ is defined. 
Figure \ref{fig:AMR} displays that oscillations are present with a smaller period and damped amplitude.  %The presence of the oscillations can be explained by the fact that the oscillations occur as the front crosses a grid cell. 
The zoomed in results in Figure \ref{fig:AMR} illustrate that there are $N_{\text{inner}} \equiv 1 / \dx_{\text{inner}}$ oscillations in between the coarse grid cells, as expected as the front crosses the $N_{\text{inner}} = 10$ inner grid cells for $\dx_{\text{inner}} = 1/10$.  The temporal oscillations are spatially dependent, and applying AMR does not remove them.%AMR provides limited improvement in resolving the temporal oscillations. 

\subsection{SAM in Finite Volume Average Form and its Connection to the Integral Average}
\label{sam_discont}
 Writing SAM as an averaged-based method on a uniform Cartesian grid of spatial step size $\dx$ provides additional insight.  In particular, we will see another explanation of the temporal oscillations.  For $j \ne i$, we have defined $k^{SAM}_{j+1/2}$ by Eqn. \eqref{eq:gen_coeff} in Section \ref{sam_exact}.  The Cartesian and auxiliary grids are the same in the parabolic regions away from the shock.  %The difference between the grids occurs in the interval $[x_i, x_{i+1}]$ containing the shock. %For the cells away from the shock, the coefficient values are known by the monotonicity of the solution.  We then define $k^{SAM}_{j+1/2}$ as in Eqn. \eqref{eq:gen_coeff}.
%\begin{equation}
%k^{SAM}_{j-1/2} =
%	\begin{dcases}
%  		 k_{\max},           & 1 \le j \le i,   \\
%		 k_{\min}\frac{\dx}{\dx - \dx^*}\frac{p^*-p_{q+1}}{p_j - p_{q+1}}, &j = i + 1, \\
%		  k_{\min}, & i+2 \le j \le N,
%	\end{dcases}
%	\hspace{.5cm}
%	k^{SAM}_{j+1/2} =
%	\begin{dcases}
%  		 k_{\max},           & 1 \le j \le i - 1,  \\
%    		% k_{\max} \frac{\dx}{\dx^*}\frac{p_j-p^*}{p_{i}-p_{q+1}}, & j = i,\\
%		  k_{\min}, & i+1 \le j \le N.
%	\end{dcases}
%	\label{eq:pc}
%\end{equation}
%Now, we will look at how we can write SAM as an average in the shock cell.%use k^+ k^- notation
Using Eqns. \eqref{eq:fluxi} - \eqref{eq:fluxiplus}, we can write the out-flux $F_{i}^+$ of $CV_i$ and the in-flux $F_{i+1}^-$ of $CV_{i+1}$ for SAM in finite volume average form as
\begin{equation}%to identify the corresponding newly developed average ($k^{SAM}_{q+1/2}$) as
	\begin{aligned}
	\Huge F_{i}^+ &=   -\Bigg(\underbrace{k_{\max} \frac{\dx}{\dx^*}\frac{p^*-p_i}{p_{i+1}-p_i}}_{\mbox{\normalsize ${k^{SAM}_{i+1/2}}^+$ }}\Bigg)\frac{p_{i+1}-p_i}{\dx}, 
	\label{eq:sam_avgplus}
	\end{aligned}
\end{equation}
and
\begin{equation}
	\begin{aligned}
	F_{i+1}^- &= -\Bigg(\underbrace{k_{\min}\frac{\dx}{\dx - \dx^*}\frac{p_{i+1}-p^*}{p_{i+1}-p_i}}_{\mbox{\normalsize ${k^{SAM}_{i+1/2}}^-$ }}\Bigg)\frac{p_{i+1}-p_i}{\dx},
	\label{eq:sam_avgminus}
	\end{aligned}
\end{equation} %explain why makes sense
respectively.  As opposed to the other averaging approaches, now $F_i^+ \ne F_{i+1}^-$: conservation is honored on the auxiliary grid, defined in Section \ref{aux_grid}.

We can express these SAM averages as a weighted linear combination  %We see that $k^{SAM}_{q+1/2} $ in the interval containing the shock is a weighted linear combination of the terms in the 
 %integral average $k^{I}_{i+1/2} $ in Eqn. \eqref{eq:int_avg_val}, namely 
\[
	c_1\frac{k_{\min}(p_{i+1}-p^*)}{p_{i+1}-p_i} + c_2\frac{k_{\max}(p^*-p_i)}{p_{i+1}-p_i},
\]  
of the integral average $k^{I}_{i+1/2} $ in Eqn. \eqref{eq:int_avg_val}.  
For  $c_1 = 0$ and $c_2 = \dx/\dx^*$, ${k^{SAM}_{i+1/2}}^+$ in Eqn. \eqref{eq:sam_avgplus} is recovered, while for $c_1 = \dx/(\dx - \dx^*)$ and $c_2 = 0$, ${k^{SAM}_{i+1/2}}^-$ in Eqn. \eqref{eq:sam_avgminus} is recovered.  Conversely, we can also express the integral flux in terms of the SAM fluxes surrounding the shock cell. %to explain why the temporal oscillations occur with the integral average.  
Rewriting Eqn. \eqref{eq:int_flux}, we have
\begin{equation}
\begin{aligned}
	{F^+_{i}}^I &=  - \Big[k_{\min}\frac{p_{i+1}-p^*}{\dx-\dx^*}\frac{\dx-\dx^*}{\dx} + k_{\max}\frac{p^*-p_i}{\dx^*}\frac{\dx^*}{\dx}\Big] \\
	&=  F_{i+1}^-(1-y) + F_i^+y, 
	\end{aligned}
	\label{eq:int_anal}
\end{equation}
where $y \equiv \dx^*/\dx$.  The shock moves from left to right in the interval $[x_i,x_{i+1}]$, and $0 < y < 1$.  %Below, we use a representative example for a specific $y$ to show that the integral flux does not converge to the analytical flux.

As discussed in Section \ref{aux_grid}, the SAM flux is a first order approximation of the analytical flux, and the jump condition is satisfied.  The integral flux, on the other hand, does not converge to the analytical flux as $\dx \rightarrow 0$, and does not satisfy the jump condition.  Without loss of generality, we let the shock be to the right of the cell face $x_{i+1/2}$ by $\eps\dx$ for some small $\eps > 0$.  Then $y = 1/2 + \eps$.  Using Taylor series expansion in the parabolic region to the left of the shock gives 
\begin{equation}
	F_i^+ = F^L[p(x_{i+1/2+\eps})] + \mathcal{O}(\dx),
	\label{eq:int_analplus}
\end{equation}
where $F^L[p(x)]= -k_{\max}\nabla p(x)$.
Similarly,  Taylor series expansion in the parabolic region to the right of the shock gives
\begin{equation}
	F_{i+1}^- = F^R[p(x_{i+1/2 + \eps})] + \mathcal{O}(\dx),
	\label{eq:int_analminus}
\end{equation}
where $F^R[p(x)] = -k_{\min} \nabla p(x)$.
Substituting Eqns. \eqref{eq:int_analplus}-\eqref{eq:int_analminus} into Eqn. \eqref{eq:int_anal} gives the following 
\[
	{F^+_{i}}^I = F^R[p(x_{i+1/2+\eps})](0.5-\eps) +  F^L[p(x_{i+1/2+\eps})](0.5+\eps) + \mathcal{O}(\dx).
\]
Taking the limit as $\eps \rightarrow 0$, we obtain 
\[
	{F^+_{i}}^I = \frac{F^R[p(x_{i+1/2})] +  F^L[p(x_{i+1/2})]}{2} + \mathcal{O}(\dx),
\]
as the integral flux at the interface $x_{i+1/2}$.  Since $x_{i+1/2}$ is to the left of the shock, the analytical flux at $x_{i+1/2}$ is given by the left flux $F^{\text{L}}[p(x_{i+1/2})]$.  The integral flux is then averaging the left and right fluxes across the jump.  The right flux $F^R[p(x_{i+1/2})]$ should not have any impact on the value across the jump.  This jump condition violation is exactly what results in the temporal oscillations.  SAM does satisfy the jump condition and as the results in Sections \ref{exact_shock} and \ref{shock_speed} show, it does remove the artifacts entirely.  %The integral flux does not converge to the analytical flux $F^L[p(x_{i+1/2})]$ at the interface, while the SAM flux in Eqn. \eqref{eq:int_analplus} is a first-order approximation to this analytical flux.

\label{flux}

\section{Conclusions and Future Work} %add front tracking to conclusion
%PME without known p^* can estimate and level set for higher dim
%This paper demonstrates that the choice of averaging is not responsible for the temporal oscillations.  mention GPME
This paper explains what causes the mysterious artifacts in discretizations of the discontinuous Generalized Porous Medium Equation (GPME), and suggests an alternative method that results in numerical solutions without these artifacts.  The FTCS scheme with integral averaging performs better than the schemes with harmonic and arithmetic averaging because it has information about the shock value $p^*$.  By rewriting the integral average in the shock cell, it can be seen that full removal of the numerical artifacts requires more than $p^*$, and that the shock location $x^*(t)$ must also be included in the numerical scheme.  %With the integral average, the jump condition for integral conservation laws is not fully satisfied, leading to grid size dependent monotonic temporal oscillations.
%In the proposed Shock-Based Averaging Method (SAM) tracking the front position, using the jump condition is the key to a numerical solution without artifacts.  
The Shock-Based Averaging Method (SAM) incorporates both $x^*(t)$ and $p^*$, and satisfies the jump condition for integral conservation laws to result in numerical solutions with accurate and smooth temporal profiles.  Casting SAM in the finite-volume framework helps provide understanding on why the numerical artifacts were occurring with finite volume averaged-based methods.
%fix last paragraph and assumptions

Future work includes extensions to higher dimensions and to the Porous Medium Equation (PME) subclass.  Since the velocity for the GPME is given for arbitrary dimensions, it can be used as input to a level set implementation for extensions to higher dimensions, as discussed in Section \ref{shock_speed}.  %The level set method is a desirable approach because it is robust, and handles sharp interfaces and topological changes. 
Another future direction would be to combine the approach from this paper with our previous paper \cite{maddix_pme}, discussing the PME subclass of the GPME with continuous coefficients.  %For these problems, as the exponent is increased, the solutions become less regular and front formation occurs.  
For the PME, there is no known specified $p$ value at the shock, as given by $p^*$ in this paper.  %This would require more advanced shock detection methods.  
With the additional control volume around the shock in the auxiliary finite volume grid from Section \ref{sam_exact}, $p^*(t)$ can be solved for as an additional degree of freedom, and an approach similar to SAM can be taken.% since the speed of the front is known. %can solve for p* but coupled
%\label{a}

%% The Appendices part is started with the command \appendix;
%% appendix sections are then done as normal sections

%% \section{}
%% \label{}

%% If you have bibdatabase file and want bibtex to generate the
%% bibitems, please use
%%
%%  \bibliographystyle{elsarticle-harv} 
%%  \bibliography{<your bibdatabase>}

%% else use the following coding to input the bibitems directly in the
%% TeX file.

\appendix
\section{Exact Solution Derivation for arbitrary parameters}
\label{exact}
For the test problem with the initial condition in Figure \ref{IC}, the exact solution is known and is used for testing the numerical methods.  We give the full derivation, which is similar to \cite{Kraaijevanger2011}, and is generalized for arbitrary $k_{\max}$.  The domain is partitioned into $\Omega_1 = (0, x^*(t)]$ and $\Omega_2 = [x^*(t), \infty)$.  The problem can be subdivided into two constant coefficient heat equations  \cite{Kraaijevanger2011, crank1984}.  The solution is monotonically non-increasing and so in $\Omega_1, p_1(x,t) > p^*$ and in $\Omega_2, p_2(x,t) \le p^*$.  It can be verified that $p_1(x,t) = 1 - c_1\Phi\Big(x/(2\sqrt{k_{\max}t})\Big)$ and $p_2(x,t) = c_2\Big(1 - \Phi\Big(x/(2\sqrt{k_{\min}t})\Big)\Big)$ for some constants $c_1, c_2$ to be determined.  $\Phi(x) = \text{erf}(z) = \int_0^z \phi(y) dy$ is the standard Gaussian error function, where $\phi(y) = \frac{2}{\sqrt{\pi}} \exp(-y^2)$. %check bc, ic

The unknown shock location ($x^*(t)$) needs to be computed.  To do so, two additional boundary conditions are required at the shock $x = x^*(t)$: 
\begin{enumerate}
	\item $p_1(x^*(t), t) = p_2(x^*(t),t) = p^*, \hspace{.25cm} \forall t$ (Continuity)
	\item  $k_{\max}(\partial{p_1(x^*(t), t)}/\partial{x}) = k_{\min}(\partial{p_2(x^*(t), t)}/\partial{x}), \hspace{.25cm}  \forall t$ (Flux Continuity).
\end{enumerate}

Since Condition (1) must hold for all $t$, $x^*(t) = \alpha \sqrt{t}$, 
\[
	c_1 = \frac{1-p^*}{\Phi\Big(\alpha/(2\sqrt{k_{\max}})\Big)}, \hspace{.5cm} c_2 = \frac{p^*}{1-\Phi\Big(\alpha / (2\sqrt{k_{\min}})\Big)}.
\] 

Condition (2) is used to derive a nonlinear solve for the remaining unknown, $\alpha$.  Substituting in the expressions for the derivatives and simplifying leads to
\begin{equation}
	c_1\sqrt{k_{\max}}\phi\Big(\frac{\alpha}{2\sqrt{k_{\max}}}\Big) = c_2\sqrt{k_{\min}}\phi\Big(\frac{\alpha}{2\sqrt{k_{\min}}}\Big).
	\label{eq:flux_eq_bc}
\end{equation}
Let $z_1 = \alpha/(2\sqrt{k_{\max}})$ and $z_2 = \alpha/(2\sqrt{k_{\min}})$.  We substitute the expressions for $c_1$ and $c_2$ into Eqn. \eqref{eq:flux_eq_bc}, multiply both sides by  $\frac{\alpha}{2}$ and simplify to obtain
\begin{equation}
	\begin{aligned}
		\frac{1-p^*}{\Phi(z_1)} \sqrt{k_{\max}}\phi(z_1) &=  \frac{p^*}{1-\Phi(z_2)}\sqrt{k_{\min}}\phi(z_2) \iff \\
								(1-p^*)(1-\Phi(z_2))\exp(z_2^2)\frac{\alpha}{2\sqrt{k_{\min}}} &=  p^*\frac{\alpha}{2\sqrt{k_{\max}}}\Phi(z_1)\exp(z_1^2) \iff \\
								(1-p^*)(1-\Phi(z_2))\exp(z_2^2)z_2 &=  p^*\Phi(z_1)z_1\exp(z_1^2) .
		\label{eq:flux_eq_long}
	\end{aligned}
\end{equation}
By a series expansion from integration by parts, 

\[
	1-\Phi(z_2) = \frac{\exp(-z_2^2)}{z_2\sqrt{\pi}}\Bigg(1 - \frac{1}{2z_2^2} + \frac{3}{4z_2^4} - \dots\Bigg).
\] Using the above expression, Eqn. \eqref{eq:flux_eq_long} simplifies to
					\begin{equation}
					\frac{1 - p^*}{\sqrt{\pi}}\Bigg(1 - \frac{1}{2z_2^2} + \frac{3}{4z_2^4} - \dots\Bigg) =  p^*\Phi(z_1)z_1\exp(z_1^2).
					\label{eq:limit_jbp}
					\end{equation}

%Note that the constant from $\phi$ cancels from each side.
In our application and most of the numerical tests, $k_{\min} = 0$.  This implies that $z_2 \rightarrow \infty$.  The limit as $z_2 \rightarrow \infty$ of the left hand side of Eqn. \eqref{eq:limit_jbp} is simply $(1-p^*)/\sqrt{\pi}$.  This is an advantage of computing the series form, since we can easily take this limit as $k_{\min} \rightarrow 0$.   This form is also preferred numerically when $k_{\min}$ is small to avoid multiplication of the large $\exp(z_2^2)z_2$ terms.  For the $k_{\min} = 0$ case, the unknown ($\alpha$) only appears in $z_1$ on the right hand side of the equation, given by 
\begin{equation}
	\frac{1 - p^*}{\sqrt{\pi}}=  p^*\Phi(z_1)z_1\exp(z_1^2).
\label{eq:alpha}
\end{equation}
A simple one-dimensional nonlinear equation solver can be used to solve this equation for $z_1$, where $\alpha = 2\sqrt{k_{\max}}z_1$.
%NSF acknowledgement
%\begin{thebibliography}{00}

%% \bibitem[Author(year)]{label}
%% Text of bibliographic item


\section{Error Tables and Convergence Study}
\label{app:conv}
This appendix contains the convergence results for the model problem tested with the same time step size $\dt = \dx^2/32$ as in the prior sections.  The $l_2$ and $l_\infty$ error norms are calculated with respect to the exact solution on the corresponding grid.  The error is measured globally, and includes the points surrounding the shock.  We compare the errors of the FTCS with arithmetic and integral averaging to those of SAM.  The errors of the FTCS with harmonic averaging are not shown, since the solution locks and does not converge.  For the methods with arithmetic and integral averaging, even on finer grids, the asymptotic region is not yet reached.  Tables \ref{table:timing}-\ref{table:timing2} show that SAM has approximately first order convergence and errors that are orders of magnitude lower than those for the averaged-based methods.  The first order convergence of SAM is also shown in Figures \ref{fig:log1}-\ref{fig:log2}.  We do not consider higher order methods because of the dominance of the shock.%for these problems involving shocks, very fine grids are required to reach the asymptotic region and higher order methods also reduce to first order accuracy at the shock.}
\begin{table}[H]
\begin{center}
\caption{$l_2$ norm errors}
\small
\begin{tabular}{|c|c|c|c|c|c|c|}  % there will be four vertical lines with three centered columns
\hline 
& $N = 25$ & $N = 50$ & $N = 100$ &$N = 200$  &Order \\  \hline
Arithmetic			&3.2335e-02         &6.8351e-03& 3.4431e-02 & 1.9773e-02 &N/A  \\ \hline
Integral       &7.4250e-03        & 7.0866e-03& 2.4898e-02  &1.1974e-02  & N/A   \\  \hline
SAM  		&8.4813e-04 &4.7762e-04 &2.0583e-04 &6.6594e-05 & 1.2229
   \\ \hline
\end{tabular}
\label{table:timing}
\end{center}
\end{table} 

\begin{table}[H]
\begin{center}
\caption{$l_{\infty}$ norm errors}
\small
\begin{tabular}{|c|c|c|c|c|c|c|}  % there will be four vertical lines with three centered columns
\hline 
& $N = 25$ & $N = 50$ & $N = 100$ &$N = 200$  &Order \\  \hline
Arithmetic		&4.8237e-02         &1.5223e-02 &3.5742e-02 & 2.0394e-02&N/A \\ \hline
Integral	      &1.4104e-02          &9.1485e-03 &2.9971e-02 &1.6252e-02 &N/A     \\  \hline
SAM	  		&2.1859e-03  &1.7763e-03 &1.0586e-03  & 4.5134e-04	
 & 1.1336   \\ \hline
\end{tabular}
\label{table:timing2}
\end{center}
\end{table} 

\begin{figure}[H]
		\center
		\begin{subfigure}[H]{.49\textwidth}  
			\includegraphics[width =\textwidth]{l2_error.eps}
			\caption {$l_2$ norm}
			\label{fig:log1}
			\end{subfigure}
			\begin{subfigure}[H]{.49\textwidth}  
			\includegraphics[width =\textwidth]{l_inf_error.eps}
			\caption {$l_\infty$ norm}
			\label{fig:log2}
			\end{subfigure}
		\caption{Loglog convergence plots for the Stefan problem.}
\end{figure}

%\bibitem[ ()]{}
\section*{Acknowledgements}
This material is based upon work supported by the National Science Foundation Graduate Research Fellowship under Grant No. DGE - 114747.  We would also like to thank Giuseppe Domenico Cervelli for his guidance.
%\end{thebibliography}
%\section*{References}
\bibliographystyle{elsarticle-num-names}
 \bibliography{Foamreferences}
\end{document}

\endinput
%%
%% End of file `elsarticle-template-harv.tex'.
