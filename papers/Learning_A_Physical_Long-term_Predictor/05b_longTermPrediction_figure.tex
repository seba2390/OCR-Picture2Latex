\iftrue
\begin{figure*}[t]
    \vspace{-0.1in}
    \centering\resizebox{0.9\textwidth}{!}{
    \begin{overpic}[width=\linewidth]{images/result_graphs/graph_all}
    %\includegraphics[width=\linewidth]{images/result_graphs/graph_all.pdf}
    \put(18,-0.1){(a)}
    \put(50.5,-0.1){(b)}
    \put(83,-0.1){(c)}
    \end{overpic}
    }
    \vspace{-0.1in}
    \caption{\textbf{Long term prediction evolutions}. Comparison of network trained using sequences of length 20. Error bars denote $25^{th}$ and $75^{th}$ percentiles of the $L^2$ loss in pixels.}\label{fig:result_graphs}
        
    % The baselines (Linear and Quadratic) were fit to the first 10 frames, all other networks had to make predictions after having seen the first $\ninputs=4$ frames. Note, how deterministic predictors cannot handle situations with unforeseen circumstances (\textit{i.e.}, heterogeneous friction of the slope). The training set consisted of 8750 scenes. Error bars denote $25^{th}$ and $75^{th}$ percentiles of the $L^2$ loss in pixels (the image size was $128\times128$) evaluated over 3750 test cases.

    %\vspace{-0.2in}
\end{figure*}
\else
\begin{figure}[h]
    \centering
    \includegraphics[width=\linewidth]{images/result_graphs/graph_S0}
    \caption{
    Comparison of long-term prediction accuracy of baselines and proposed networks in the least complex scenario \szero.
    %Comparison of prediction accuracy of baseline predictors using $\ninputs$ images as input, and predicting $16,~23,~26, \ldots$ timesteps ahead. Units (pixels?). Mark $\ninputs$. Log plot?
    }
    \label{fig:result_s0}
\end{figure}