\section{Conclusions}
\label{sec:conclusions}

In this paper we explored the possibility of using a single neural network for long-term prediction of mechanical phenomena. We considered in particular the problem of predicting the long-term motion of a cuboid sliding down a slope of unknown inclination and heterogeneous friction. Differently from many other approaches, we use the network {\em not} to predict some physical quantities to be integrated by a simulator, but to directly predict the complete trajectory of the object end-to-end.

Our results, obtained from extensive synthetic simulation, indicate that deep neural networks can successfully predict long-term trajectories without requiring explicit modeling of the underlying physics. They can also reliably estimate a distribution over such predictions to account for uncertainty in the data. Remarkably, these models are competitive with alternative predictors that have access to the ground-truth physical simulator, and outperform them when some of the physical parameters are not observable or known \emph{a-priori}. However, neural networks exhibit a limited capability to perform predictions outside the physical regimes observed during training. In other words, the internal representation of physics learned by such model is not as general as standard physical laws.

Several future directions remain to be explored. Given the accuracy of mechanical simulators, synthetic experiments are sufficient to assess the capability of networks to learn mechanical phenomena. However, the obvious next phase will be to test the framework on video footage obtained from real-world data in order to assess the ability to do so from visual data affected by real nuisance factors. The other important generalization is to consider more complex physical phenomena, including multiple sliding objects with possible interactions, rolling motion, and sliding over non-flat surfaces.
