%\sisetup{input-ignore={,},input-decimal-markers={.},group-separator={,}}
\begin{table*}
%\begin{minipage}[b]{0.68\linewidth}x
\footnotesize
\setlength{\tabcolsep}{1pt}
%\begin{tabular}{|c|ccc|cccc|cccc|cccc|}
\sisetup{detect-weight=true,detect-inline-weight=math}
\begin{tabular}{|c|ccc|*2S[table-format=-2.2]|*2S[table-format=-2.2]|*2S[table-format=-2.2]|}
\hline
& & & &
\multicolumn{2}{c|}{\szero} &
\multicolumn{2}{c|}{\sone} &
\multicolumn{2}{c|}{\stwo}
\\
Method & Feat.\ & Prop.\ & T.~Obj.\ & 
\multicolumn{2}{c|}{$L^2$ (ln perplexity)} &
\multicolumn{2}{c|}{$L^2$ (ln perplexity)} &  
\multicolumn{2}{c|}{$L^2$ (ln perplexity)} \\
& & & & {20} & {40} & {20} & {40} & {20} & {40} 
\\
\hline
\text{Linear}    & \noentry & \noentry & \noentry & 12.62 & 49.07 & 11.81 & 42.58 &  11.86 & 42.37 \\
Quadratic & \noentry & \noentry & \noentry & 6.67 & 37.95 & 8.21 & 46.34 & 8.35  & 47.29 \\
\SimNet & \vect & \noentry & \noentry  & 1.21 & \bfseries 1.89 & 1.93 & \bfseries 5.16  & 5.67 & 24.69 \\
\NetOne   & \vect & LSTM & L2  & \bfseries 0.31 & 25.37 & \bfseries 1.52 & 30.84 &  \noentry & \noentry \\
\NetTwo   & \tensor & conv & L2  & 0.77 & 10.68 & 1.91 & 34.46 & \bfseries 1.95 & 13.24  \\
\NetThree & \tensor & conv & GL & 0.55 & 15.74  & 2.26 & 26.04 &  \noentry & \noentry  \\
& & & &   {(0.97)} & {(385)} &  {(4.08)} & {(36)}&  \noentry &   \noentry
\\
\NetFour  & \tensor & conv & SM& 0.56 & 24.59 & 2.13 & 19.54  & 3.62 & \bfseries 9.55   \\
& & & &   {\bfseries (0.49)} & {\bfseries (0.96)} &    {\bfseries (3.19)} & {\bfseries (17.2)} & {(5.24)}  & {(11.47)} \\
\hline
\end{tabular}\hfill%
%\end{minipage}%
\begin{minipage}[c]{0.32\linewidth}%
\includegraphics[width=\textwidth,trim=0 30pt 0 20pt]{images/result_graphs/graph_S2_truncx10}
\end{minipage}
\caption{\textbf{Long term predictions.} \vect\ and \tensor\ refers to the dimensionality of the internal state representations (vector and tensor respectively). We expect \tensor\ to maintain a 2D spatial model which leads to higher accuracy. The \SimNet\ and all \textit{MechaNet} models observed the $\ninputs=4$ first frames as input. All networks have been trained to predict the $T=20$ first positions, except in \mbox{Scenario~\stwo}, where \NetOne\ and \NetFour\ have been trained to predict $T=30$ frames in order to experience enough variation in the underlying physical conditions, \textit{i.e.}, changing friction. Perplexity ($\log_e$ values shown in the table) is defined as $2^{-\mathbb{E}[\log_2(p(x))]}$ where $p$ is the estimated posterior distribution. \emph{Right:} error evolution on experiment S2 for all time steps up to 40. Error bars denote $25^{th}$ and $75^{th}$ percentiles of the $L^2$ loss in pixels.}
\label{tab:results}
\end{table*}