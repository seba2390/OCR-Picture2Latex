%% ****** Start of file apstemplate.tex ****** %
%%
%%
%%   This file is part of the APS files in the REVTeX 4 distribution.
%%   Version 4.1r of REVTeX, August 2010
%%
%%
%%   Copyright (c) 2001, 2009, 2010 The American Physical Society.
%%
%%   See the REVTeX 4 README file for restrictions and more information.
%%
%
% This is a template for producing manuscripts for use with REVTEX 4.0
% Copy this file to another name and then work on that file.
% That way, you always have this original template file to use.
%
% Group addresses by affiliation; use superscriptaddress for long
% author lists, or if there are many overlapping affiliations.
% For Phys. Rev. appearance, change preprint to twocolumn.
% Choose pra, prb, prc, prd, pre, prl, prstab, prstper, or rmp for journal
%  Add 'draft' option to mark overfull boxes with black boxes
%  Add 'showpacs' option to make PACS codes appear
%  Add 'showkeys' option to make keywords appear
\documentclass[aps,prb,onecolumn,superscriptaddress]{revtex4-1}
%\documentclass[aps,prl,preprint,superscriptaddress]{revtex4-1}
%\documentclass[aps,prl,reprint,groupedaddress]{revtex4-1}


\usepackage{physics}
\usepackage{bbold}
\usepackage{lgrind}        % convert program listings to a form includable in a LaTeX document
\usepackage{natbib}    % allows a bibliography for each chapter (each labguide has it's own)
\usepackage{color}         % produces boxes or entire pages with colored backgrounds
%\usepackage{graphics}      % standard graphics specifications
\usepackage[pdftex]{graphicx}      % alternative graphics specifications
\usepackage{longtable}     % helps with long table options
\usepackage{epsf}          % old package handles encapsulated post script issues
\usepackage{bm}            % special 'bold-math' package
\usepackage{amsmath}
%\usepackage{asymptote}     % For typesetting of mathematical illustrations
\usepackage{thumbpdf}
\usepackage{esint}
\usepackage{subfigure}
\usepackage[colorlinks=true]{hyperref} 
% You should use BibTeX and apsrev.bst for references
% Choosing a journal automatically selects the correct APS
% BibTeX style file (bst file), so only uncomment the line
% below if necessary.
\bibliographystyle{apsrev4-1}

\begin{document}

% Use the \preprint command to place your local institutional report
% number in the upper righthand corner of the title page in preprint mode.
% Multiple \preprint commands are allowed.
% Use the 'preprintnumbers' class option to override journal defaults
% to display numbers if necessary
%\preprint{}

%Title of paper
\title{Supplementary Material for Prediction of $\mathrm{TiRhAs}$ as a Dirac Nodal Line Semimetal via First-Principles Calculations}

% repeat the \author .. \affiliation  etc. as needed
% \email, \thanks, \homepage, \altaffiliation all apply to the current
% author. Explanatory text should go in the []'s, actual e-mail
% address or url should go in the {}'s for \email and \homepage.
% Please use the appropriate macro foreach each type of information

% \affiliation command applies to all authors since the last
% \affiliation command. The \affiliation command should follow the
% other information
% \affiliation can be followed by \email, \homepage, \thanks as well.
\author{Sophie F. Weber}
\affiliation{Department of Physics, University of California, Berkeley, CA 94720, USA}
\affiliation{Molecular Foundry, Lawrence Berkeley National Laboratory, Berkeley, CA 94720, USA}
\author{Ru Chen}
\affiliation{Department of Physics, University of California, Berkeley, CA 94720, USA}
\affiliation{Molecular Foundry, Lawrence Berkeley National Laboratory, Berkeley, CA 94720, USA}
\author{Qimin Yan}
\affiliation{Department of Physics, Temple University, Philadelphia, PA 19122, USA}
\author{Jeffrey B. Neaton}
\affiliation{Department of Physics, University of California, Berkeley, CA 94720, USA}
\affiliation{Molecular Foundry, Lawrence Berkeley National Laboratory, Berkeley, CA 94720, USA}
%\email[]{Your e-mail address}
%\homepage[]{Your web page}
%\thanks{}

%Collaboration name if desired (requires use of superscriptaddress
%option in \documentclass). \noaffiliation is required (may also be
%used with the \author command).
%\collaboration can be followed by \email, \homepage, \thanks as well.
%\collaboration{}
%\noaffiliation

\date{\today}















% Put \label in argument of \section for cross-referencing
%\section{\label{}}

% tables should appear as floats within the text


% Specify following sections are appendices. Use \appendix* if there
% only one appendix.


\maketitle



% If you have acknowledgments, this puts in the proper section head.
%\begin{acknowledgments}
% put your acknowledgments here.
%\end{acknowledgments}

% Create the reference section using BibTeX:

\section{HSE06 band structures}
As mentioned in the main text, GGA is expected to overestimate the inversion of conduction and valence bands necessary for a nodal line compound\cite{Vidal2011}. To confirm our prediction of $\mathrm{TiRhAs}$ as a DNL system we repeat our bulk calculations using the hybrid density functional HSE06, which uses a fraction of screened Hartree-Fock (HF) exchange to correct for self-energy errors inherent in GGA\cite{Heyd2003}. Our energy cutoff and reciprocal grid are identical to our GGA parameters ($300$ eV and $8\times6\times6$ respectively). The result is shown in Figure \ref{fig:HSE}, confirming that the DNL persists and is not a false positive of the GGA functional. HSE06 yields an even more attractive result than GGA in terms of experimental implications. While the GGA band structure is overall free from trivial bands near the Fermi surface, there is one band located at $\Gamma$ which is almost exactly at $E_f$. Such a feature is potentially problematic because it makes experimental probing and manipulation of the DNL itself difficult, an issue explicitly mentioned for the recent example of synthesized $\mathrm{PbTaSe_2}$\cite{Neupane2016}.  However, we see in Figure \ref{fig:HSE} that the HSE06 functional pushes this band down away from the DNL, suggesting that experimental studies of the topological properties in $\mathrm{TiAsRh}$ should be relatively straightforward to implement and interpret.

\begin{figure}
\includegraphics{HSE_plot.png}
\caption{\label{fig:HSE}$\mathrm{TiRhAs}$ band structure computed using the HSE06 hybrid functional method.}
\end{figure}

\section{Berry phase argument for surface states}
Surface states in a DNL system are topologically guaranteed by a nonzero one-dimensional Berry phase\cite{Chan2016,Huang2016}. For a given value $\mathbf{k}_{\parallel}$ of crystal momentum parallel to the crystalline surface in question, the Berry phase $\mathcal{\theta}(\mathbf{k}_{\parallel})$ is given by
\begin{equation}
\mathcal{\theta}(\mathbf{k}_{\parallel})=-i\sum_{E_i<E_F}\int_{-\pi}^{\pi} \bra{u_n(\mathbf{k})}\partial_{k_{\perp}}\ket{u_n(\mathbf{k})}dk_{\perp},
\label{eq:1dBphase}
\end{equation}
where $k_{\perp}$ is the crystal momentum in the direction perpendicular to the surface. Vanderbilt et al. showed that $\mathcal{\theta}(\mathbf{k}_{\parallel})$ is related to the charge $q_{end}$ at the end of a one-dimensional system\cite{Vanderbilt1993},
\begin{equation}
q_{end}=\frac{e}{2\pi}\mathcal{\theta}(\mathbf{k}_{\parallel}),
\label{eq:surfcharge}
\end{equation}
modulo $e$. Thus in the case $\mathcal{\theta}(\mathbf{k}_{\parallel})\neq{0}$, a surface state \emph{must} occur at $\mathbf{k}_{\parallel}$. In the case of $\mathrm{TiRhAs}$, two symmetries force $\mathcal{\theta}(\mathbf{k}_{\parallel})$ to be either $0$ or $\pi$; the composite $\mathcal{P}{\mathcal{T}}$, and the reflection symmetry $\mathcal{R}_x$. Therefore, if we calculate values of $\mathcal{\theta}(\mathbf{k}_{\parallel})$ as $\mathbf{k}_{\parallel}$ moves from the outside to the inside of the projected nodal line, we expect the phase to jump from $0$ to $\pi$ as the surface states appear. We confirm this by using WannierTools\cite{Wu2017} to track the 1d hybrid Wannier charge centers (WCCs). Hybrid Wannier functions are analogous to the maximally localized Wannier functions (MLWFs) mentioned in the main text except that they are Wannier-like in only one direction while they remain delocalized and Bloch-like in the other two directions\cite{Gresch2017}. For the example of Wannierization in the $[100]$ direction, as required for $\mathrm{TiRhAs}$, they are defined as
\begin{equation}
\ket{n;l_x,k_y,k_z}=\frac{a}{2\pi}\int_{-\pi/a}^{\pi/a} e^{ik_xl_xa}\ket{\psi_{n\mathbf{k}}}dk_x,
\label{eq:hybridwan}
\end{equation}
where $l_x$ is an integer and $a$ is the lattice constant along $[100]$. The charge center is then given by 
\begin{equation}
\bar{x}_n(k_y,k_z)=\bra{n;l_x,k_y,k_z}\mathbf{x}\ket{n;l_x,k_y,k_z}
\label{eq:wcc}
\end{equation}
The sum of charge centers $\sum_n \bar{x}(\mathbf{k_{\parallel}})$ for all occupied hybrid Wannier functions is equal, up to a factor of $2\pi$, to the Berry phase in Equation \ref{eq:1dBphase} (This is evident by changing the position operator for $\hat{x}$ to the Bloch representation $\propto \partial_{k_x}$)\cite{Cohen2016}. So for $\mathrm{TiRhAs}$, since the surface states appear inside the nodal line, we expect $\sum_n \bar{x}(\mathbf{k_{\parallel}})=0.5$ for values of $\mathrm{k}_{\parallel}$ inside the DNL and $\sum_n \bar{x}(\mathbf{k_{\parallel}})=0$ (modulo a lattice vector) for values outside the node. By wannierizing in the $[100]$ direction and varying $k_z$ as we move across the $k_y=0$ plane we obtain exactly this result, shown in Figure \ref{fig:wcc}.

\begin{figure}
\includegraphics{wcc_plot.png}
\caption{\label{fig:wcc}Sum of charge centers $\sum_n \bar{x}(k_z)$ for all occupied Wannier functions $\ket{n;l_x,k_y,k_z}$ as a function of $k_z$ in the $k_y=0$ plane. In moving from the outside to the inside of the DNL at $k_z=0.31$, $\sum_n \bar{x}(k_z)$ jumps abruptly to a nonzero value of $\frac{1}{2}$, a topological guarantee for the surface states we observe.}
\label{fig:top}
\end{figure}

\section{Surface states with SOC}
With the inclusion of SOC, the DNL in $\mathrm{TiRhAs}$ develops a continuous gap. The compound with SOC has topological indices $(\nu_0;\nu_1,\nu_2\nu_3)=(1;000)$, equivalent to those for the DNL compound with no SOC. To study the effect of SOC on the surface spectrum we construct a tight-binding model from DFT-PBE-SOC calculations, again using MLWFs as our basis states.  The band structure along the $\bar{Y}-\bar{\Gamma}-\bar{Z}$ direction is plotted in Figure \ref{fig:surf_soc}. Overall the band structure is nearly identical to the case with no SOC as in Figure 4 of the main text. This is not surprising given that the gap induced by SOC is very small. However, the key difference is that the nearly flat, completely degenerate drumhead states on the two surfaces of the DNL compound without SOC have evolved into an extremely shallow Dirac cone at $\Gamma$, characteristic of a topological insulator (TI) with band inversion at $\Gamma$ \cite{Zhang2009}. The two branches of the cone from $\bar{\Gamma}-\bar{Z}$ are very nearly degenerate; we provide a zoomed-in plot around $\Gamma$ in Figure \ref{fig:surf_zoom}. The surface state does not cross the Fermi level, unlike the general case for a TI, merely because the DNL without SOC is not pinned to the Fermi energy, and thus the gap, upon inclusion of SOC, does not cut through the Fermi level everywhere in the BZ.

\begin{figure}
\subfigure[]{\label{fig:surf_soc}}\includegraphics{surface_SOC.png}
\subfigure[]{\label{fig:surf_zoom}}\includegraphics{surface_SOC_zoom.png}
\caption{(a)DFT-PBE-SOC tight-binding band structure for the $(100)$ surface plotted along the $\bar{Y}-\bar{\Gamma}-\bar{Z}$ direction. Note that the flat, drumhead states on the two surfaces shown in the main text without SOC have evolved into a shallow Dirac cone, characteristic of a TI. (b) Zoomed-in plot of the portion of (a) bordered by the rectangle.}
\label{fig:SOC}
\end{figure}

\section{WIEN2k calculation details}
The WIEN2k calculations were carried out using the standard generalized gradient approximation (GGA)\cite{Perdew1996}. An RKmax parameter of $7.0$ was chosen and the wave functions were expanded in spherical harmonics up to $l_{wf}^{max} = 10$ inside the atomic spheres and $l_{pot}^{max} = 4$ for non-muffin tins. The experimental lattice parameters were used for the GGA calculations and $\mathbf{k}$-point mesh was set to $12\times9\times9$.

\section{Table of parity eigenvalues at Time-reversal invariant momenta }
In Table \ref{tab: parity}, we give the product of parity eigenvalues $\epsilon_i=\prod_{n_{occ}}\epsilon_n(\Gamma_i)$ at the eight time-reversal invariant momenta (TRIM) in $\mathrm{TiRhAs}$.

 \begin{table}[htb] %add [H] placement to break table across pages
 \caption{\label{tab: parity}Product of parity eigenvalues of occupied Bloch functions at each of the time-reversal invariant momenta (TRIM)}
 \begin{ruledtabular}
 \begin{tabular}{| c | c |}
 \hline
Symmetry label& Parity\\ \hline
$\Gamma$ & $-1$ \\ \hline
$X$ & $+1$\\ \hline
$Y$ & $+1$\\ \hline
 $Z$ & $+1$\\ \hline
$U$ & $+1$\\ \hline
$S$ &  $+1$\\ \hline
$T$ & $+1$\\ \hline
$R$ & $+1$\\ 
\hline
\end{tabular}
\end{ruledtabular}
\end{table}


%merlin.mbs apsrev4-1.bst 2010-07-25 4.21a (PWD, AO, DPC) hacked
%Control: key (0)
%Control: author (72) initials jnrlst
%Control: editor formatted (1) identically to author
%Control: production of article title (-1) disabled
%Control: page (0) single
%Control: year (1) truncated
%Control: production of eprint (0) enabled
\begin{thebibliography}{11}%
\makeatletter
\providecommand \@ifxundefined [1]{%
 \@ifx{#1\undefined}
}%
\providecommand \@ifnum [1]{%
 \ifnum #1\expandafter \@firstoftwo
 \else \expandafter \@secondoftwo
 \fi
}%
\providecommand \@ifx [1]{%
 \ifx #1\expandafter \@firstoftwo
 \else \expandafter \@secondoftwo
 \fi
}%
\providecommand \natexlab [1]{#1}%
\providecommand \enquote  [1]{``#1''}%
\providecommand \bibnamefont  [1]{#1}%
\providecommand \bibfnamefont [1]{#1}%
\providecommand \citenamefont [1]{#1}%
\providecommand \href@noop [0]{\@secondoftwo}%
\providecommand \href [0]{\begingroup \@sanitize@url \@href}%
\providecommand \@href[1]{\@@startlink{#1}\@@href}%
\providecommand \@@href[1]{\endgroup#1\@@endlink}%
\providecommand \@sanitize@url [0]{\catcode `\\12\catcode `\$12\catcode
  `\&12\catcode `\#12\catcode `\^12\catcode `\_12\catcode `\%12\relax}%
\providecommand \@@startlink[1]{}%
\providecommand \@@endlink[0]{}%
\providecommand \url  [0]{\begingroup\@sanitize@url \@url }%
\providecommand \@url [1]{\endgroup\@href {#1}{\urlprefix }}%
\providecommand \urlprefix  [0]{URL }%
\providecommand \Eprint [0]{\href }%
\providecommand \doibase [0]{http://dx.doi.org/}%
\providecommand \selectlanguage [0]{\@gobble}%
\providecommand \bibinfo  [0]{\@secondoftwo}%
\providecommand \bibfield  [0]{\@secondoftwo}%
\providecommand \translation [1]{[#1]}%
\providecommand \BibitemOpen [0]{}%
\providecommand \bibitemStop [0]{}%
\providecommand \bibitemNoStop [0]{.\EOS\space}%
\providecommand \EOS [0]{\spacefactor3000\relax}%
\providecommand \BibitemShut  [1]{\csname bibitem#1\endcsname}%
\let\auto@bib@innerbib\@empty
%</preamble>
\bibitem [{\citenamefont {Vidal}\ \emph {et~al.}(2011)\citenamefont {Vidal},
  \citenamefont {Zhang}, \citenamefont {Yu}, \citenamefont {Luo},\ and\
  \citenamefont {Zunger}}]{Vidal2011}%
  \BibitemOpen
  \bibfield  {author} {\bibinfo {author} {\bibfnamefont {J.}~\bibnamefont
  {Vidal}}, \bibinfo {author} {\bibfnamefont {X.}~\bibnamefont {Zhang}},
  \bibinfo {author} {\bibfnamefont {L.}~\bibnamefont {Yu}}, \bibinfo {author}
  {\bibfnamefont {J.~W.}\ \bibnamefont {Luo}}, \ and\ \bibinfo {author}
  {\bibfnamefont {A.}~\bibnamefont {Zunger}},\ }\href {\doibase
  10.1103/PhysRevB.84.041109} {\bibfield  {journal} {\bibinfo  {journal}
  {Physical Review B - Condensed Matter and Materials Physics}\ }\textbf
  {\bibinfo {volume} {84}},\ \bibinfo {pages} {1} (\bibinfo {year} {2011})},\
  \Eprint {http://arxiv.org/abs/1101.3734} {arXiv:1101.3734} \BibitemShut
  {NoStop}%
\bibitem [{\citenamefont {Heyd}\ \emph {et~al.}(2003)\citenamefont {Heyd},
  \citenamefont {Scuseria},\ and\ \citenamefont {Ernzerhof}}]{Heyd2003}%
  \BibitemOpen
  \bibfield  {author} {\bibinfo {author} {\bibfnamefont {J.}~\bibnamefont
  {Heyd}}, \bibinfo {author} {\bibfnamefont {G.~E.}\ \bibnamefont {Scuseria}},
  \ and\ \bibinfo {author} {\bibfnamefont {M.}~\bibnamefont {Ernzerhof}},\
  }\href {\doibase 10.1063/1.1564060} {\bibfield  {journal} {\bibinfo
  {journal} {Journal of Chemical Physics}\ }\textbf {\bibinfo {volume} {118}},\
  \bibinfo {pages} {8207} (\bibinfo {year} {2003})}\BibitemShut {NoStop}%
\bibitem [{\citenamefont {Neupane}\ \emph {et~al.}(2016)\citenamefont
  {Neupane}, \citenamefont {Belopolski}, \citenamefont {Hosen}, \citenamefont
  {Sanchez}, \citenamefont {Sankar}, \citenamefont {Szlawska}, \citenamefont
  {Xu}, \citenamefont {Dimitri}, \citenamefont {Dhakal}, \citenamefont
  {Maldonado}, \citenamefont {Oppeneer}, \citenamefont {Kaczorowski},
  \citenamefont {Chou}, \citenamefont {Hasan},\ and\ \citenamefont
  {Durakiewicz}}]{Neupane2016}%
  \BibitemOpen
  \bibfield  {author} {\bibinfo {author} {\bibfnamefont {M.}~\bibnamefont
  {Neupane}}, \bibinfo {author} {\bibfnamefont {I.}~\bibnamefont {Belopolski}},
  \bibinfo {author} {\bibfnamefont {M.~M.}\ \bibnamefont {Hosen}}, \bibinfo
  {author} {\bibfnamefont {D.~S.}\ \bibnamefont {Sanchez}}, \bibinfo {author}
  {\bibfnamefont {R.}~\bibnamefont {Sankar}}, \bibinfo {author} {\bibfnamefont
  {M.}~\bibnamefont {Szlawska}}, \bibinfo {author} {\bibfnamefont {S.~Y.}\
  \bibnamefont {Xu}}, \bibinfo {author} {\bibfnamefont {K.}~\bibnamefont
  {Dimitri}}, \bibinfo {author} {\bibfnamefont {N.}~\bibnamefont {Dhakal}},
  \bibinfo {author} {\bibfnamefont {P.}~\bibnamefont {Maldonado}}, \bibinfo
  {author} {\bibfnamefont {P.~M.}\ \bibnamefont {Oppeneer}}, \bibinfo {author}
  {\bibfnamefont {D.}~\bibnamefont {Kaczorowski}}, \bibinfo {author}
  {\bibfnamefont {F.}~\bibnamefont {Chou}}, \bibinfo {author} {\bibfnamefont
  {M.~Z.}\ \bibnamefont {Hasan}}, \ and\ \bibinfo {author} {\bibfnamefont
  {T.}~\bibnamefont {Durakiewicz}},\ }\href {\doibase
  10.1103/PhysRevB.93.201104} {\bibfield  {journal} {\bibinfo  {journal}
  {Physical Review B}\ }\textbf {\bibinfo {volume} {93}},\ \bibinfo {pages} {1}
  (\bibinfo {year} {2016})},\ \Eprint {http://arxiv.org/abs/1604.00720}
  {arXiv:1604.00720} \BibitemShut {NoStop}%
\bibitem [{\citenamefont {Chan}\ \emph {et~al.}(2016)\citenamefont {Chan},
  \citenamefont {Chiu}, \citenamefont {Chou},\ and\ \citenamefont
  {Schnyder}}]{Chan2016}%
  \BibitemOpen
  \bibfield  {author} {\bibinfo {author} {\bibfnamefont {Y.~H.}\ \bibnamefont
  {Chan}}, \bibinfo {author} {\bibfnamefont {C.~K.}\ \bibnamefont {Chiu}},
  \bibinfo {author} {\bibfnamefont {M.~Y.}\ \bibnamefont {Chou}}, \ and\
  \bibinfo {author} {\bibfnamefont {A.~P.}\ \bibnamefont {Schnyder}},\ }\href
  {\doibase 10.1103/PhysRevB.93.205132} {\bibfield  {journal} {\bibinfo
  {journal} {Physical Review B}\ }\textbf {\bibinfo {volume} {93}},\ \bibinfo
  {pages} {1} (\bibinfo {year} {2016})},\ \Eprint
  {http://arxiv.org/abs/1510.02759} {arXiv:1510.02759} \BibitemShut {NoStop}%
\bibitem [{\citenamefont {Huang}\ \emph {et~al.}(2016)\citenamefont {Huang},
  \citenamefont {Liu}, \citenamefont {Vanderbilt},\ and\ \citenamefont
  {Duan}}]{Huang2016}%
  \BibitemOpen
  \bibfield  {author} {\bibinfo {author} {\bibfnamefont {H.}~\bibnamefont
  {Huang}}, \bibinfo {author} {\bibfnamefont {J.}~\bibnamefont {Liu}}, \bibinfo
  {author} {\bibfnamefont {D.}~\bibnamefont {Vanderbilt}}, \ and\ \bibinfo
  {author} {\bibfnamefont {W.}~\bibnamefont {Duan}},\ }\href {\doibase
  10.1103/PhysRevB.93.201114} {\bibfield  {journal} {\bibinfo  {journal}
  {Physical Review B}\ }\textbf {\bibinfo {volume} {93}},\ \bibinfo {pages} {1}
  (\bibinfo {year} {2016})},\ \Eprint {http://arxiv.org/abs/1605.04050}
  {arXiv:1605.04050} \BibitemShut {NoStop}%
\bibitem [{\citenamefont {Vanderbilt}\ and\ \citenamefont
  {King-Smith}(1993)}]{Vanderbilt1993}%
  \BibitemOpen
  \bibfield  {author} {\bibinfo {author} {\bibfnamefont {D.}~\bibnamefont
  {Vanderbilt}}\ and\ \bibinfo {author} {\bibfnamefont {R.~D.}\ \bibnamefont
  {King-Smith}},\ }\href {\doibase 10.1103/PhysRevB.48.4442} {\bibfield
  {journal} {\bibinfo  {journal} {Physical Review B}\ }\textbf {\bibinfo
  {volume} {48}},\ \bibinfo {pages} {4442} (\bibinfo {year} {1993})},\ \Eprint
  {http://arxiv.org/abs/arXiv:1011.1669v3} {arXiv:arXiv:1011.1669v3}
  \BibitemShut {NoStop}%
\bibitem [{\citenamefont {Wu}\ \emph {et~al.}(2017)\citenamefont {Wu},
  \citenamefont {Zhang}, \citenamefont {Song}, \citenamefont {Troyer},\ and\
  \citenamefont {Soluyanov}}]{Wu2017}%
  \BibitemOpen
  \bibfield  {author} {\bibinfo {author} {\bibfnamefont {Q.}~\bibnamefont
  {Wu}}, \bibinfo {author} {\bibfnamefont {S.}~\bibnamefont {Zhang}}, \bibinfo
  {author} {\bibfnamefont {H.-F.}\ \bibnamefont {Song}}, \bibinfo {author}
  {\bibfnamefont {M.}~\bibnamefont {Troyer}}, \ and\ \bibinfo {author}
  {\bibfnamefont {A.~A.}\ \bibnamefont {Soluyanov}},\ }\href
  {http://arxiv.org/abs/1703.07789{\%}5Cnhttp://www.arxiv.org/pdf/1703.07789.pdf}
  {\bibfield  {journal} {\bibinfo  {journal} {arXiv:1703.07789 [cond-mat,
  physics:physics]}\ } (\bibinfo {year} {2017})},\ \Eprint
  {http://arxiv.org/abs/1703.07789} {arXiv:1703.07789} \BibitemShut {NoStop}%
\bibitem [{\citenamefont {Gresch}\ \emph {et~al.}(2017)\citenamefont {Gresch},
  \citenamefont {Aut{\`{e}}s}, \citenamefont {Yazyev}, \citenamefont {Troyer},
  \citenamefont {Vanderbilt}, \citenamefont {Bernevig},\ and\ \citenamefont
  {Soluyanov}}]{Gresch2017}%
  \BibitemOpen
  \bibfield  {author} {\bibinfo {author} {\bibfnamefont {D.}~\bibnamefont
  {Gresch}}, \bibinfo {author} {\bibfnamefont {G.}~\bibnamefont {Aut{\`{e}}s}},
  \bibinfo {author} {\bibfnamefont {O.~V.}\ \bibnamefont {Yazyev}}, \bibinfo
  {author} {\bibfnamefont {M.}~\bibnamefont {Troyer}}, \bibinfo {author}
  {\bibfnamefont {D.}~\bibnamefont {Vanderbilt}}, \bibinfo {author}
  {\bibfnamefont {B.~A.}\ \bibnamefont {Bernevig}}, \ and\ \bibinfo {author}
  {\bibfnamefont {A.~A.}\ \bibnamefont {Soluyanov}},\ }\href {\doibase
  10.1103/PhysRevB.95.075146} {\bibfield  {journal} {\bibinfo  {journal}
  {Physical Review B}\ }\textbf {\bibinfo {volume} {95}},\ \bibinfo {pages} {1}
  (\bibinfo {year} {2017})},\ \Eprint {http://arxiv.org/abs/1610.08983}
  {arXiv:1610.08983} \BibitemShut {NoStop}%
\bibitem [{\citenamefont {Cohen}\ and\ \citenamefont
  {Louie}(2016)}]{Cohen2016}%
  \BibitemOpen
  \bibfield  {author} {\bibinfo {author} {\bibfnamefont {M.~L.}\ \bibnamefont
  {Cohen}}\ and\ \bibinfo {author} {\bibfnamefont {S.~G.}\ \bibnamefont
  {Louie}},\ }\href@noop {} {\emph {\bibinfo {title} {{Fundamentals of
  Condensed Matter Physics}}}}\ (\bibinfo  {publisher} {Cambridge University
  Press},\ \bibinfo {year} {2016})\BibitemShut {NoStop}%
\bibitem [{\citenamefont {Zhang}\ \emph {et~al.}(2009)\citenamefont {Zhang},
  \citenamefont {Liu}, \citenamefont {Qi}, \citenamefont {Dai}, \citenamefont
  {Fang},\ and\ \citenamefont {Zhang}}]{Zhang2009}%
  \BibitemOpen
  \bibfield  {author} {\bibinfo {author} {\bibfnamefont {H.}~\bibnamefont
  {Zhang}}, \bibinfo {author} {\bibfnamefont {C.-X.}\ \bibnamefont {Liu}},
  \bibinfo {author} {\bibfnamefont {X.-L.}\ \bibnamefont {Qi}}, \bibinfo
  {author} {\bibfnamefont {X.}~\bibnamefont {Dai}}, \bibinfo {author}
  {\bibfnamefont {Z.}~\bibnamefont {Fang}}, \ and\ \bibinfo {author}
  {\bibfnamefont {S.-C.}\ \bibnamefont {Zhang}},\ }\href {\doibase
  10.1038/nphys1270} {\bibfield  {journal} {\bibinfo  {journal} {Nature
  Physics}\ }\textbf {\bibinfo {volume} {5}},\ \bibinfo {pages} {438} (\bibinfo
  {year} {2009})},\ \Eprint {http://arxiv.org/abs/1405.2036} {arXiv:1405.2036}
  \BibitemShut {NoStop}%
\bibitem [{\citenamefont {Perdew}\ \emph {et~al.}(1996)\citenamefont {Perdew},
  \citenamefont {Burke},\ and\ \citenamefont {Ernzerhof}}]{Perdew1996}%
  \BibitemOpen
  \bibfield  {author} {\bibinfo {author} {\bibfnamefont {J.~P.}\ \bibnamefont
  {Perdew}}, \bibinfo {author} {\bibfnamefont {K.}~\bibnamefont {Burke}}, \
  and\ \bibinfo {author} {\bibfnamefont {M.}~\bibnamefont {Ernzerhof}},\ }\href
  {\doibase 10.1103/PhysRevLett.77.3865} {\bibfield  {journal} {\bibinfo
  {journal} {Physical Review Letters}\ }\textbf {\bibinfo {volume} {77}},\
  \bibinfo {pages} {3865} (\bibinfo {year} {1996})},\ \Eprint
  {http://arxiv.org/abs/0927-0256(96)00008} {arXiv:0927-0256(96)00008
  [10.1016]} \BibitemShut {NoStop}%
\end{thebibliography}%


\end{document}
%
% ****** End of file apstemplate.tex ******

