\section{System Architecture}
As depicted in Figure~\ref{Design2}, the System architecture consists of 3 main components: Application, Coalescer and Stratagem. Applications are lightweight and containerized units which are deployed to achieve a particular sub-task. The main focus in this research is to choose updates efficiently when a group of different applications coordinate to achieve a common objective. Such a group of applications working towards a common objective is called "Swarm"~\cite{swarm_docker}. 


The single point of contact for multiple applications is the "Coalescer". Coalescer is the orchestrator unit in our design which helps in coordinating multiple applications to achieve a particular goal. A Coalescer has multiple functionalities: It tracks application changes, it processes migration request, it tracks progress and performance per application. If there is performance degradation in any of the application(s), Coalescer makes sure the expected performance of application would be restored. Coalescer handles coordination of multiple applications, updates to application and makes sure overall performance of the system is preserved. Stratagem is a component which records application changes and updates the application with suitable code/model in order to satisfy performance criteria. Stratagem prioritizes updates considering different performance metrics and migrates the required difference, between source code/model and updated code/model, to the appropriate application.

\begin{figure}
    \centering
    \includegraphics[width=8.5cm]{figures/Microservice.pdf}
    \caption{Interaction of different nodes in DSOC}
    \label{figX}
\end{figure}
Figure~\ref{figX} explains the logical components that are used to build Coalescer and Stratagem. "Manager" is a logical component present in Coalescer. A Manager node tracks one or more applications and makes sure the performance of those applications are optimum. "Worker" is a logical component present in Stratagem. A Manager node creates multiple worker nodes to track the deployed applications. A manager creates a worker node per application to track changes and makes sure the performance, progress expectation of that application is being met over time.

As Autonomous systems such as Self driving cars, Aerial Vehicles, Smart traffic system, smart restaurants are becoming increasingly popular~\cite{swarm_docker}, they have not focused on building DSOC type application which efficiently progresses towards the goal, using a strategy which is easy to deploy and maintain. An autonomous application which is deployed in production will comprise of several smaller applications coordinating together to achieve an end goal~\cite{IOT_growth}. This is a Microservices based architecture where each independent component would have a logical function and contributes towards the end goal~\cite{micro_article}. Building such an efficient system which tracks and makes timely progress towards end goal is really crucial. During deployment, there might be updates to individual applications which improves their performance. If all the update hungry applications are allowed to update their model, it would lead to increased throughput and bandwidth degradation. There should be an effective method of prioritizing updates taking several factors such as latency, progress, cpu utilization, memory, accuracy when multiple applications are seeking an update. The implementation section discusses details about prioritizing updates and using the framework from ~\cite{SoftwarePilot}. Using a DSOC approach would give greater control over applications and ensure performance of the system is maintained. Using DSOC, critical concerns like code update, Machine learning model update, performance based progress towards end goal are carefully considered.



