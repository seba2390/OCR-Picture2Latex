\section{Background}

\subsection{Docker and Docker Swarm}
Docker Containers are best suited for Microservices. Docker provides lightweight encapsulation of each application enabling independent deployment and scaling of each microservice. Docker is a composing Engine for Linux containers, an OS-level virtualization technique, which uses namespace and cgroups to isolate applications in the same linux kernel. Control group (abbreviated as cgroup) is a collection of processes that are associated with a set of parameters. cgroup ensures that the specified resources are actually available for a container. Namespace isolation is another such feature where groups of processes are separated such that each group cannot see the resources used by other groups. The kernel resources are partitioned such that a set of processes sees a set of resources while another set of processes see a different set of resources~\cite{docker_article}. 

Docker uses copy on write (COW) and layered storage within images and containers. A Docker image is a read-only template, which references a list of read-only storage layers, used to build a linux container. Docker container is a standard software unit that packages up code and dependencies so that the application runs quickly and can be shipped reliably from one computing environment to another. Docker images become Docker containers at run time when they run on Docker Engine. The layered storage allows fast packaging and shipping of an application as a lightweight container by sharing common layers between images and container. By using Docker, there is potential for faster deployment time and faster model updates~\cite{docker_article}.

When you have a lot of containerized applications running, there should be a mechanism to make them all work towards a common goal. One method to achieve this is using Docker Swarm. Docker Swarm is a group of machines that are joined together as a cluster and commands are executed by swarm manager to control the group of machines. Each machine in swarm is called a node which can be physical or virtual Machine. Applications can be defined using a manifest file and easily deployed using Docker commands~\cite{swarm_docker}.

\subsection{Worker and Manager Nodes}
Manager Nodes control the cluster with tasks such as maintaining the state of the cluster, dispatching tasks to worker and providing fault tolerance to the system. Currently, Docker supports using multiple Manager nodes where only one manager would be elected as leader and it performs all the responsibilities of a manager. The other manager nodes are standby managers which receive updated information about state of the system and may be chosen as leader when leader node goes down. Using multiple managers is fairly new experimental feature in Docker which would be explored in this research. Worker Nodes are instances which accept tasks from Manager Nodes and execute them as containers. Worker Nodes will not share their state information with other worker nodes and do not make scheduling decisions~\cite{swarm_docker}.

\subsection{Scale-in and Scale-out of applications}
When a Microservice application is deployed, we might need to increase the number of microservice components (scale-out) or decrease the number of microservice components (scale-in) based on the user demand and progress towards end goal. This calculation should happen automatically and applications need to be re-scaled based on workload and progress towards end goal~\cite{Venugopal_2017}.

\begin{figure}
    \centering
    \includegraphics[width=8.5cm]{figures/Design2.pdf}
    \caption{System Architecture diagram}
    \label{Design2}
\end{figure}
