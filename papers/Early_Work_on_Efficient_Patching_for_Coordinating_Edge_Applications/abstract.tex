%%% Things to talk about in abstract:-
%  Current AVs and where research is heading
% 1. What is the new technology (DONE)
% 2. Why is it important? (DONE)
% 3. What is the problem? (DONE)
% 4. How much does the problem inhibit technology (DONE)
% 5. What is the expected type of solution? (DONE)
% 6  Why is it hard to achieve? (DONE)
% 7. What is your solution? (DONE)
% 8. How does your solution get around the hard problem in (quesn6)
% 9. What will I do (Since this is an idea paper)
% 10. Why will all of this matter?
Multiple applications running on Edge computers can be orchestrated to achieve the desired goal. Orchestration of  applications is prominent when working with Internet of Things based applications, Autonomous driving and Autonomous Aerial vehicles. As the applications receive modified classifiers/code, there will be multiple applications that need to be updated. If all the classifiers are synchronously updated there would be increased throughput and bandwidth degradation. On the other hand, delaying updates of applications which need immediate update hinders performance and delays progress towards end goal. The updates of applications should be prioritized and updates should happen according to this priority. This paper explores the setup and benchmarks to understand the impact of updates when multiple applications working to achieve same objective are orchestrated with prioritized updates. We discuss methods to build a distributed, reliable and scalable system called "DSOC"(Docker Swarm Orchestration Component).


%--  Think of this like a Micro service architecture, each component logically separate and performing their functionality. When we have such Docker containerized application running and they need model updates, there should be an efficient way to track each application and prioritize which application model need to be updated. There should be an engine which tracks the performance of system and decide which application need to be updated with a newer model.