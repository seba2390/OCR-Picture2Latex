\section{INTRODUCTION}

%%% Things to talk about in introduction:-
%  Current AVs and where research is heading
% 1. What is the new technology (DONE)
% 2. Why is it important? (DONE)
% 3. What is the problem? (DONE)
% 4. How much does the problem inhibit technology (DONE)

% Make sure the above points are strong enough in abstract and expand each point in the introduction.....

% Draw a graph explaining how current trend of application dependence on smaller components lead to end goal.
% 1. How many applications coordinate together towards end goal
% 2. How many such co-ordinating application have updates?
% Prioritize the updates: Critically important, major update, minor update and unimportant --- Green, Yellow, blue and Red

Autonomous systems like Self-driving cars, autonomous aerial systems, smart restaurants and smart traffic lights have attracted a lot of interest from both academia and Industry~\cite{transportresearch}~\cite{boubin2019managingdeprecated}. Many top companies like Google, Uber, Intel, Apple, Tesla, Amazon have significantly invested in researching and building Autonomous Systems~\cite{selfdriving}. An autonomous system is a critical decision making system which makes decisions without human intervention. An autonomous system is comprised of complex technologies which learns the environment, makes decision and accomplishes the goal~\cite{transportresearch}. In this paper, we focus on "Microservices Orchestration" for coordinating multiple autonomous applications that are working towards a common objective.
 
Microservices is an architectural style which structures an application as a collection of different services that are loosely coupled, independently deployable and highly maintainable. Large, complex applications can be deployed using Microservices where each service will have different logical function and contribute to the bigger application~\cite{micro_article}. %There is high Complexity coordinating Microservices based application~\cite{micro_article}.
When working towards a particular goal, we might need to deploy multiple applications which need to take up a sub task and coordinate with other applications in order to efficiently complete the task at hand. Efficient coordination of multiple different applications is really crucial for building fully autonomous systems.% Handling interaction between applications, error handling and failure recovery is resilient. 
Configuring, controlling and keeping track of each microservice would be really hard~\cite{orchestrate}. An efficient way to track and manage multiple applications would be using an orchestrator.  Orchestration is a process which involves automated configuration, management and coordination of multiple computer systems and application software~\cite{it_orchestrate}. There are various orchestration tools for Microservices like ansible~\cite{ansible}, Kubernetes~\cite{kubernetes}, Docker Swarm~\cite{swarm_docker}. 

%############# Insert one citation above and search for different types of orchestrator. ############
%####### add 2-3 sentences also

%%%%%%%%%% Please insert graph1 here.
\begin{figure}
    \centering
    \includegraphics[width=8cm]{figures/intro_graph.pdf}
    \caption{Why Coordination and Patching?}
    \label{fig:introduc}
\end{figure}

When working with Artificial Intelligence based applications, the performance of each microservice may degrade over time and there needs to be an updated code or Machine Learning model in order to restore the performance~\cite{naveeniotonline}. When multiple applications seek an update, allowing all updates would degrade bandwidth, increase throughput and may not yield much performance gain~\cite{naveeniotonline}. If any microservice application need an update, it would be a tedious task to identify individual application and perform the update. While performing such updates we need to consider individual application performance, progress towards end goal and system performance~\cite{7886112}. It is practically impossible to consider every application's performance parameters and pick the model to be updated at run-time~\cite{poster_Aiiot}.

A Patch could be a code update that fixes a bug or yields performance improvement, Machine Learning model update and would be referred as "Classifier". The term Classifier is used in rest of this paper. Figure~\ref{fig:introduc} approximates the usage of Classifiers in AI-based applications and patching in real world Autonomous Systems such as Self Driving cars, Smart Traffic Systems, Aerial vehicles and Smart surveillance. These autonomous applications make use of nearly 40-140 total classifiers out of which at least 40 percent have frequent classifier update to improve performance~\cite{naveeniotonline}. The frequency of update in individual application is calculated by performing a literature survey of updates using incremental software releases~\cite{autopilot}~\cite{dji}~\cite{trlights}.
%%%%% ############ Try to add a couple of citations above.
Out of the total updates, at least 50 percent of them are correlated updates. For example, an update to an application's model would impact the performance of another interdependent model or code fragment. If multiple applications are coordinating with one another towards a common objective, the choice of update significantly impacts the performance of the system and rate of completion towards the end goal.
% The performance can be measured in terms of accuracy, latency and execution time of the application. 

This paper proposes Docker Swarm Orchestration Component called "DSOC" which is responsible for orchestrating multiple applications and efficiently prioritizing classifier updates. To the best of our knowledge, this is the first work to propose an efficient method of using Docker Swarm for multiple AI based application coordination involving classifier updates. 
%The expected type of solution would be to regularly store the performance metrics of individual microservice application. When multiple applications seek an update, a program logic can be run to choose to rank the applications based on importance of update. This would incur a lot of processing at the time of performing update and there might be deadlock situation when only few updates can be performed synchronously but many applications are striving for immediate update [5].
 %An orchestrator would track and record the status and performance of multiple coordinating applications. The orchestrator would take care of tracking different applications and take complete responsibility of seamlessly updating the applications.

% DONEEEEE --- Inserting figure is causing all sorts of alignment issue.
%****** Draw a graph explaining what the current trend is and how an orchestrator like K8 or Docker swarm would help while running dynamic application which are working towards the same objective.
%********

