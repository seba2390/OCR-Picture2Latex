\section{Conclusion and future work}

%%%%%%% REWORK, Don't go into specifics ... Make this a point. %%%%%%
Autonomous systems are evolving at a very fast pace and moving towards achieving full autonomy~\cite{transportresearch}. The industry and research community need to focus on coordinating multiple applications that work closely together towards achieving an end goal. Containers are increasing in popularity for building and shipping applications efficiently~\cite{docker_article}. This paper is an early work which focuses on building an orchestrator component using Docker Swarm mode to coordinate multiple applications that are working together towards an end goal. The orchestrator component is responsible for tracking and choosing the model updates leading to performance improvement. The updates would be prioritized considering system performance and individual application performance. Currently, we have a framework and infrastructure setup to deploy, track and update an application. The future plan is to build several different applications using different Machine Learning techniques. We plan to build applications such as intruder detection, simple face recognition, obstacle detection, mission planner which can work collectively towards safely reaching the destination from a source point. During the mission, we try to run different workload by constraining the mission to measure performance of system and record how DSOC efficiently reaches the end goal.

{\small \noindent {\bf Acknowledgments:} This work was funded in part by NSF Grants 1749501 and 1350941 with support from NSF CENTRA collaborations (grant 1550126). This was an IDEA and Early work paper submitted to ICAC 2019 (now known as ASCOS).}