In this section, we discuss several limitations of our VA system that could be regarded as improvement ideas for the future.

\textbf{Generalization.} Since \textsc{VisRuler} mainly focuses on the exploration of decision paths, our approach is applicable to any tree-based algorithm. If we follow the same methodology, RF could be changed to extremely randomized trees~\cite{Geurts2006Extremely} (known as extra trees) or other bagging algorithms; while instead of AB, gradient boosting~\cite{Ke2017LightGBM,Chen2016XGBoost} or any boosting algorithm can be employed. An extension of our approach could be to include new supervised ML methods such as rule-based algorithms, or even association rule learning.~\cite{Song2016Research} However, such algorithms consist of two parts: the antecedent (IF) and the consequent (THEN), which are complicated because there could be multiple if statements that bind many times the values using the same features. Therefore, a single decision path is non-trivial to be extracted, as in our case. One potential enhancement would be to support rule-based algorithms to encode these reoccuring information, which implies an order of consecutive events. 
%We could work on finding out more ways to address this issue in the future.

As shown in the paper, \textsc{VisRuler} works with both binary and multi-class classification problems. Since humans may have problems in perceiving more than ten categorical colors~\cite{Ware2019Information} at the same time, \textsc{VisRuler} utilizes all of them as well as possible. Nonetheless, a limitation is the extensive (but unavoidable) use of color that might hinder our tool from operating with more than a few classes. Therefore, applying the one-vs-rest strategy is one possible idea to solve multi-class classification problems with several classes. This strategy translates to either designating one class as the positive class and all others as the negative class or choosing classes and gradually examining them.

At last, although \textsc{VisRuler} concentrates on the interpretation of rules and extraction of manual decisions driven by the experienced domain experts, it can also be used by ML experts to tune the hyperparameters of models and eventually debug RF and AB models. We acknowledge that \textsc{VisRuler} takes premature steps toward this direction, but this concept appears an exciting research opportunity for the future.

\textbf{Scalability.} Similar to many VA tools/systems,~\cite{Robertson2009Scale} a major challenge we considered when designing \textsc{VisRuler} is scalability.
%
In the \emph{models overview} panel, the number of trained models is limited to 20 RF and AB models in total (or 10 for each algorithm). However, in the backend users can set different hyperparameters and perform hyperparameter search with various automatic hyperparameter tuning approaches~\cite{Claesen2015Hyperparameter,Claesen2014Easy} or VA tools for this same goal.~\cite{Li2018HyperTuner,Chatzimparmpas2021VisEvol} \textsc{VisRuler} utilizes Random search for this purpose due to several benefits identified by Bergstra and Bengio.~\cite{Bergstra2012Random} The end result is to visualize several robust and diverse models in our tool, which can be deemed an adequate number of models. Also, the two PCPs for training new RF and AB models allow users to explore more models progressively, especially since the provided hyperparameters' ranges are also easily modifiable through the code.

In the \emph{decisions space} view, circles may overlap when the number is large, as with any dimensionality reduction technique presented in a scatter plot. Although \textsc{VisRuler} can visualize thousands of decision paths (illustrated by the usage scenario in Section~\nameref{sec:case}), the cluttering of the low-dimensional embedding could be considered as an intrinsic difficulty. To address this issue, we first adopt a filtering approach to remove irrelevant or even overfitting decisions (e.g., see Figure~\ref{fig:use_case1_broderline}(a)), and second we enable users to partially hide out impure decisions (cf. Figure~\ref{fig:use_case1_safe}(a)). Users may also utilize interactions like panning and zooming to focus on certain regions of circles. A potential future idea is to enhance the current scatter plot with approaches designed to reduce overlapping.~\cite{Mayorga2013Splatterplots,Hilasaca2019Overlap}

In the \emph{manual decisions} view while trying to get an overview, the vertical PCP may be challenging to interpret because it requires users to scroll through a list of decisions that expands by the number of features. Despite that, we have implemented multiple layout treatments (e.g., filtering out decisions visible in Figure~\ref{fig:use_case1_broderline}(b)) and interaction possibilities (e.g., the comparison mode for juxtaposing groups of decisions shown in Figure~\ref{fig:use_case1_outlier}(b)) for users to partially overcome these challenges. Furthermore, the most common scenario is to explore regions of decisions paths and focus on specific test instances which by default drastically limits the number of decision rules. In summary, the benefits of this tweaked visualization are many, since users can directly compare a test case with training instances for the various rules applicable and all features of a data set. The vertical PCP can help domain experts to externalize their domain knowledge because it serves as a root for a discussion between experts and the general public. One potential update is to try out alternative PCP designs that could boost the scalability of this view, such as the proposal from Wu et al.~\cite{Wu2017Making}

\textbf{Efficiency.} The performance of \textsc{VisRuler} could pose problems if numerous models are simultaneously active and produce too many decisions. Indeed, the excessive computational time required for exploring thousands of decisions paths along with the initial training of those ML algorithms can be a root cause for further troubles. Using distributed computation processes on performant cloud servers can be one solution for scaling \textsc{VisRuler} to enormous data sets. In the future, we believe that the improvement in high-performance hardware as well as progressive VA approaches~\cite{Stolper2014Progressive,Turkay2018Progressive} will also benefit \textsc{VisRuler}.

The use case, usage scenario, and user study were performed on a MacBook Pro 2019 with a 2.6 GHz (6-Core) Intel Core i7 CPU, an AMD Radeon Pro 5300M 4 GB GPU, 16 GB of DDR4 RAM at 2667 Mhz, running macOS Monterey, and with Chrome (version 99) as the browser. The system can perform interactively after the model training stage is over, which may take a few minutes for the data sets used in the use case of Section~\nameref{sec:overview} and the usage scenario of Section~\nameref{sec:case}. However, we cache the results to speed-up drastically the future executions for the same data sets.

\textbf{Complexity.} Compared to several VA tools/systems,~\cite{Chatzimparmpas2020The} \textsc{VisRuler} is no exception in terms of the high cognitive load that could overwhelm users. Despite that, the proposed workflow of \textsc{VisRuler} (see Figure~\ref{fig:workflow-diagram} and read Section~\nameref{sec:overview}) is mainly linear. Furthermore, the participants of our user study (cf. Section~\nameref{sec:eval}) correctly performed most of the provided tasks after a specific training period, which is indicative of the gradual learning curve of our tool. However, in a future iteration of the tool, we plan to implement a hiding functionality for the \emph{models overview} panel after the initial phase, involving mainly an ML expert selecting powerful and diverse models, is over. Another incremental improvement could be to increase the size of the symbols for the three validation metrics present in the line chart of Figure~\ref{fig:teaser}.

\textbf{Evaluation.} While we already conducted a task-based user study with 12 participants that tested the applicability and effectiveness of \textsc{VisRuler}, additional review sessions with experts could help us to validate our tool further. However, as illustrated in Figure~\ref{fig:collaboration-diagram}, our VA system is designed to be operated with a single workflow for two experts that most of the time are set apart and work independently. The prior knowledge and expertise of each group of experts is useful in specific steps of the collaboration schema, especially since they meet only in step 4, related to the \emph{decisions space} exploration. A threat in this case is the overconfidence effect and overinterpretation of the models' capabilities by both domain-specific and ML experts, especially in noisy data scenarios. Despite that, we believe our first user study was an appropriate choice of method to understand preliminarily if \textsc{VisRuler} is usable and effective. In the future, we could further evaluate the particular designs of this multi-component system with both ML and domain experts.