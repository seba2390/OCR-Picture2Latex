We conducted a user study to evaluate our tool's effectiveness in supporting decision-making. % based on many alternative decision paths. 
As in prior works,~\cite{Ming2019RuleMatrix,Neto2021Explainable} we created five questions (Qs) %that cover \textsc{VisRuler}'s different views (Figure~\ref{fig:results}),
%
that cover \textsc{VisRuler}'s different views,
%For the descriptions as shown to the participants, please refer to the supplemental materials. 
focusing on appraising the goals described in Section~\nameref{sec:goals} with the use case outlined in Section~\nameref{sec:overview} as the GT (see Table~\ref{fig:questions}). Note that this user study was based on a slightly different arrangement of visualizations for the \emph{models overview} panel, but all the rest of the visual representations and in-depth functionalities remained the same. In particular during the study, the line chart and the confusion plot were not aligned with the bar charts below, and the legends of this panel were positioned at the very top instead of beside the visualizations (as in Figure~\ref{fig:teaser}(a)).

\textbf{Demographics.} Figure~\ref{fig:demographics} contains general information about the attendees of the user study. Seven male and five female volunteers aged 23 to 49 (mean: $\approx$33) participated in our study, all with at least an MSc degree (and two PhD's). None of them knew the data set used, and no colorblindness issues were reported. Four of the participants were highly knowledgeable in visualization and seven in ML, while the rest had limited knowledge regarding all aspects. Additionally, four of them had never worked with any EL method. All participants were researchers that have to frequently work with ML and tune ML models for various data tasks, such as image classification and natural language processing.

\begin{figure}[h]
  \centering
  \includegraphics[width=\linewidth]{demographics-crop.pdf}
  \caption{General information on the participants of our user study.}
  \label{fig:demographics}
\end{figure}

\textbf{Methodology and Instructions.} Initially, participants watched an $\approx$18-minute video tutorial about bagging and boosting concepts, \textsc{VisRuler}'s goals, and how to work with our tool to analyze decision paths, using the Iris data set.~\cite{Fisher1936The} The participants experimented for five minutes with Iris, which concludes the required training for utilizing the capabilities of our tool. Then, they proceeded to use the data set described in Section~\nameref{sec:overview}. They were asked to answer five questions (cf. Table~\ref{fig:questions} for a summary). The original document containing the questions and instructions as shown by the attendees is available in the supplemental material accompanying this paper. Finally, the participants were requested to provide qualitative feedback via the ICE-T questionnaire.~\cite{wall2019aheuristic}
%(cf. Figure~\ref{results}) and provide qualitative feedback through the ICE-T questionnaire~\cite{wall2019aheuristic}.
%
%and provide qualitative feedback via the ICE-T questionnaire~\cite{wall2019aheuristic}. 

\textbf{Question-related Results.} The completion time it took the users to respond to each question and their answers to the multiple choice questions are shown in Figure~\ref{fig:results}. After the initial setting shown in Figure~\ref{fig:use_case1_model}(a), all participants decided to exclude \emph{Generosity} in Q1 (Answer: d, Q1), which happened in 2.03 minutes on average. For Q2, 9 participants followed our GT (Answer: a, Q2), as described in Figure~\ref{fig:use_case1_model}(c). The remaining attendees selected AB10 instead of AB8 (Answer: b, Q2). This action led to 5 test instances in conflict (Answer: b, Q3) compared to 3 (Answer: d, Q3) in our analysis (Figure~\ref{fig:use_case1_outlier}(c) presents a single case). This result could be a strong indication that our approach is essential for making such decisions. To respond in Q2 and Q3, participants took 4.04 and 2.58 minutes on average, respectively. The most time-consuming question was Q4 with an average response time of 6.15 minutes (but with very accurate results (Answer: d, Q4), see  Figure~\ref{fig:use_case1_safe}(b)).
%
%Finally, a confused participant answered wrongly in Q5 (in comparison to Figure~\ref{fig:use_case1_outlier}(b)). 
The average time taken for Q5 was 6.07 minutes, with all correct answers (Answer: c, Q5) except for one (Answer: a, Q5). The participant that responded incorrectly chose \emph{Freedom} because it was at the bottom of the vertical PCP (cf. Figure~\ref{fig:use_case1_outlier}(b)).

\begin{figure}[h]
  \includegraphics[width=\linewidth]{Time_Results-crop.pdf} 
  \caption{The question-related results of the user experiment. The top row presents the \emph{completion time} for every question of the study separately, and the bottom row comprises the histograms of the participants' answers in all questions.}
  \label{fig:results}
\end{figure} 

\textbf{Qualitative Results.} 
%A key component of our study is the qualitative user feedback acquired by following the ICE-T methodology~\cite{wall2019aheuristic}. 
In Table~\ref{fig:ICET}, the mean scores of the ICE-T components~\cite{wall2019aheuristic} for each participant are displayed along with the two-tailed 95\% confidence intervals (CIs) per component ($t^* = 2.201$, $N=12$). Higher values in green indicate good results, as opposed to red. \textsc{VisRuler} has received a few 7.0 scores, and most are at least 6.0 and above (the lowest score is 4.67). Essence, Insight, and Time received large scores which means users found our tool competent in portraying decisions, guiding users to come up with fundamental questions, and performing these discoveries quickly. Confidence was lower, with a mean value of 5.87. However, this value still makes \textsc{VisRuler} a reliable and trustworthy VA tool according to Wall et al.~\cite{wall2019aheuristic}

\begin{table}[h]
\captionsetup{justification=justified}
\centering
\caption{Analyzed results from the ICE-T feedback.~\cite{wall2019aheuristic}}
\includegraphics[width=\columnwidth]{ICET-crop.pdf}
\label{fig:ICET}
\end{table}

%Please refer to the handout provided in the supplemental materials for the complete description, as seen by the participants.