\section{Current Workforce Development Initiatives: Shaping the Talent Pipeline}\label{sec4:workforce}
To build a secure semiconductor supply chain as a part of onshoring the U.S. semiconductor industry, a stable talented workforce is needed, especially as CHIPS Act fuels new facilities and fabs constructions across the country. A talented workforce is equipped with the skills to either design advanced chips through R\&D or through monitoring fab operations to ensure maximum production yield and quality, depending on which educational spectrum the individual is from. All of these factors call for urgent measure to establish WFD pipeline initiative that provides K-12 and college students with experiential learning opportunities through internships/apprenticeships, modernized academic curriculum with skilled instructors, and up-to-date research facilities and equipment. The amalgamation of degree programs offered by both community colleges and universities is best complemented with hands-on-experience to build and retain a resilient workforce \cite{fueling_american}.

This urgent talent gap has compelled extensive collaboration between various industry sectors, government, academic institutions, consortiums and organizations, to set the framework for the next workforce ecosystem, with the goal of creating diverse pathways to build and support a domestic workforce in microelectronics and advanced packaging to meet both immediate and future demands of the industry. The talent pipeline targets the entire educational continuum, including K-12 schools, community colleges, universities, and veterans—for a vast array of workforce development programs to reinforce industry awareness and train/onboard students to be well-prepared in their career endeavors. With the rise of the CHIPS Act, the new fabs being built will generates thousands of open positions for skilled technicians and engineers to fulfill. Companies are racing to partner with community colleges and universities to establish strategic partnerships in creating apprenticeship and internships programs across the entire academic spectrum. One notable case is Intel, with plans to build multiple fabs across Ohio, has promised to fund up to \$50 million to various institutions across Ohio that, other than sponsoring direct internship and research opportunities, will help to revamp the current curriculum, hire skilled faculty to teach advanced courses, and supply new lab equipment \cite{Patel2023_ref14}. Figure 5 provides a list of WFD initiatives dedicated to each education/skills domain, which are currently shaping the talent pipeline to establish defined pathway entries to the workforce.


\begin{figure*}
    \centering
    \includegraphics[width=0.99\textwidth]{images/fig5.png}
    \caption{WFD initiatives to create a robust pipeline of talent – from core to advanced pathway entries to the workforce \cite{Zhou2015Industry4T,chapterMAPT,Kindling_undated-br}.}
    \label{fig:5}
\end{figure*}

\subsection{Initiatives within K-12 Education}\label{sec4_1:k_12_and_up}
Current downward trajectory of undergraduate retention rates within engineering programs has initiated substantial strategic efforts pivoted towards drawing pre-college students towards STEM, particularly electronics, as early as elementary and middle school. Since the average American school curriculum leaves out critical engineering disciplines with the exception of computer science courses, many talented students gravitate towards software, which leaves them alienated from the microelectronics field. As a response, companies and organizations are teaming with schools in taking remarkable initiatives to cultivate meaningful connections with K-12 students, such as STEM@GF, an online platform by Global Foundries (GF) that provides guided science projects for kids to develop core engineering skills \cite{STEM_GF}. The following initiatives introduce core hardware concepts, and familiarize students with careers opportunities \cite{Alam2023-hc}, \cite{Nathan_J_Edwards_Carter_Grizzle_Vaanathi_Sekar_Brett_Meadows_Michael_McGivern_Steven_Kiss_Asher_Edwards_John_Branning_Mohamed_Kassem2023-ec}:
\begin{itemize}
    \item {Gamified learning projects}
    \item {STEM outreach programs, summer camps, and extracurriculars sponsored by industrial partners}
    \item {Facility visits and invited professional as guest lecturers}
    \item {Mentorships in career exploration and research}
    \item {On-campus (high school) career fairs with industry visits to classrooms}
    \item {Industry-specific STEM camps or industry-led short courses with involved projects for direct workforce preparation}
\end{itemize}

Notable efforts are put forth by Samsung \cite{samsung}, which offers middle and high school outreach programs, known as the Semiconductor STEM Academy Experience, dedicated to exposing 6-12\textsuperscript{th} grade students to careers path in the semiconductor and manufacturing industry. Either students themselves or the school district can request a tour of the facility and time with subject matter experts such as the engineers/technicians or inviting Samsung for school events and presentations. Micron offers similar initiates such as the Chip Camp in select U.S. middle school districts, where students get to explore key processing steps in chip production (processing silicon wafers, photolithography, etching, and ion implanting), various STEM activities, as well as receiving mentorship by both Micron team members and engineering intern students. Notable student group projects include designing a solid-state drive using Lego parts under multiple constraints to stimulate innate creativity. For high school graduates, Samsung provides high school graduates a 10-week internship in a technician career track, enabling them experience working in the industry. The first two weeks involve a mentor guiding the interns to help them determine if this is a suitable career choice for them. The final week consists of a showcase event with intern presentations, and select students will be invited to transition into their Fab Apprentice Program, an earn-as-you-learn initiative. This program allows the aspiring technicians the opportunity to work with 2 days a week while completing their associates degree in engineering technology or manufacturing, and upon graduation, they will be eligible to receive a full-time technician role at Samsung. Engineering students also have the opportunity to intern as a semiconductor engineering intern, a 3-month internship they are assigned a project in an array of areas such as analytics, automation, diffusion, etch, photolithography, etc. with the opportunity to transition to a full-time engineering position as well.

\subsection{Continuing Technical Education Programs}\label{sec4_2:continue_teched}
A particular focus of WFD programs also consist of boot-camp courses offered by various community colleges to train future technicians, such as the community college system of Arizona \cite{Maricopa_Corporate_College_undated-zr}, one of the key states that is projected to become the next technological hub with the ongoing construction of fabs by both Taiwan Semiconductor Manufacturing Company(TSMC) and Intel. The routine operation of these fabs relies on the hands of clean-room technicians, who facilitate production manufacturing processes, oversee the quality and flow of equipment functions, and handle general maintenance. Candidates don't require advanced degrees to take on these jobs, merely needing a short-term training course to meet the position qualifications. Such programs are targeted towards skilling and hiring equipment maintenance technicians. Particularly, this includes graduates of technical Associate degree programs, veterans entering the workforce, and skilled workers from other sectors of manufacturing (with transferable skills), as all are credible sources of talent in for such positions.

A case is The Registered Apprenticeship at GF, an apprenticeship program that utilizes the learn-and-earn approach through hands-on training and classroom instructions for full-time technician roles. The training is free, targeted for all backgrounds with a high school diploma. Additionally, program graduates have the option to pursue an Associate degree afterwards. Through this program, all three stakeholders benefit within the partnership: the industry receives a constant flow of employees, each program graduate gets immediate employment with new skills, and the training program receives a pipeline of student participants to optimize new content delivery modalities for adult learners \cite{McCaughey2023RegisteredAF}.

To further expand educational and training initiatives, Purdue University established the Semiconductor Degrees Program's interdisciplinary track to offer undergraduate, masters', and postgraduate courses that encompass all key steps in chip production, such as chemical and materials processing, manufacturing, packaging, and supply chain management, as well as associates degrees through the nearby community college. 

In brief, the CHIPS Act has compelled key pathway agreements between 2-year and 4-year institutions to increase the flow of student transfers, new multidisciplinary educational and training programs (internships/apprenticeships) to expand and diversify the workforce, and strategic industry partnerships with leading semiconductor companies and organizations \cite{Weller2023-vi}.  

