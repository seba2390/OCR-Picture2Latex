\section{Introduction}\label{sec1:introduction}
%\Kang: it feels like many citations are missing here...
The advancement of semiconductor technology, design, and performance is the cornerstone for transformations across all facets of the high-tech market, including high-performance computing, 5G telecommunication, Internet of Things(IoT), Artificial Intelligence(AI)/Machine Learning(ML) applications, aerospace/defense, and automotive electrification. The global sale of semiconductor chips hit an astonishing record of \$602 billion in 2022, as evident by the increasing digitization of industries around the world \cite{semiconductor_report}. Historically, the U.S. has been leading in the microelectronics revolution, starting from the advent of the transistor in 1947 to the new era of nanoelectronics—accelerated through Moore’s Law by the shift to FinFet and nano-ribbon technology. The progression of semiconductor manufacturing has allowed for substantial chip functionality, such as heterogenous integration in advanced packaging techniques of 2.5D and 3D chiplet integration for maximum chip-to-chip communication, which ultimately combines various functionalities into one package for a smaller device footprint. The myriad of rapid innovations necessitates more construction and operation of chip Fabrication Facilities (fabs). Since building advanced fabs accrues around \$20 billion in construction and chip manufacturing presents significant complexities, only Integrated Device Manufacturer companies (IDMs) and foundries within the US are able to fabricate their own chips, so majority of U.S. semiconductor firms maintain a fabless model. To keep up the pace, majority of U.S.-based semiconductor companies outsource their front-end and back-end manufacturing processing to overseas foundries for wafer fabrication and Outsourced Semiconductor Assembly and Testing (OSATs) for Assembly/Testing/Packaging (ATP). The lower construction and operating costs, along with presence of OSATs, drives companies to build fabs abroad; consequently, close to 80\% of fabs are based overseas \cite{fueling_american}. Similarly, at least 81\% of OSATs are also overseas, with Amkor and ASE being the major ATP powerhouses. According to Semiconductor Industry Association (SIA), the US semiconductor industry makes up 50\% of annual global market share, but constitutes only 12\% of the global manufacturing capacity. Furthermore, merely 6\% of development in manufacturing is located within the US. Consequently, the U.S. is generations behind the leading-edge chip manufacturers in Taiwan, South Korea, and China \cite{Shivakumar2022}. A confluence of factors, such as political conflicts and instabilities, national security concerns, as well as increasing supply chain vulnerabilities has driven policy makers to become wary of the geopolitical and economic threats associated with off-shored manufacturing in weakening the U.S. supply chain. In response, the U.S. Congress officially enacted the Creating Helpful Incentives to Produce Semiconductors (CHIPS) for America Act in October 2022 \cite{sia_2022}, which invests \$280 billion to bolster the research, development, and domestic fabrication/manufacturing of semiconductor chips in the U.S. Figure1 breaks down the CHIPS Act funding allocations, while Table1 provides the specific subsets allocated \cite{Badlam2022,chips_act2022}.


\begin{figure}[!hbp]
    \includegraphics[width=\linewidth]{images/pie_chart.png}
    \caption{CHIPS ACT funding breakdown (in billions)}\label{fig:1}
\end{figure}

\begin{center}
\begin{table*}
\caption{CHIPS ACT Funding Allocations} \label{table-1}
\label{table:1}
{\renewcommand{\arraystretch}{2}%
\begin{tabularx}{0.99\textwidth}{|X|X|}
%\begin{tabularx}{0.99\textwidth}{|X|X|X|X|X|X|X|X|X|X|X|X|}
%\begin{tabular}{ | m{5em} | m{1cm}| m{1cm} | m{5em} | m{1cm}| m{1cm} | m{5em} | m{1cm}| m{1cm} | m{5em} | m{1cm}| m{1cm} |} 
  \hline
  Subsets of CHIPS Funding & 
  Purpose\\
  \hline
  A: Scientific Research and Development(R\&D) and WFD initiatives permitted by National Science Foundation(NSF), Department of Energy(DOE), Department of Commerce(DOC) & 
  Dedicated towards funding research, workforce, and economic development programs\\
  \hline
  B: CHIPS for America Fund & 
  Construction and expansion of domestic manufacturing facilities\\
  \hline
  C: CHIPS advanced manufacturing tax credit & 
  A 25\% investment tax credit for advanced manufacturing and new facilities\\
  \hline
  D: CHIPS Defense Fund for Department of Defense(DoD) & 
  To implement the Microelectronics Commons, a network to onshore the prototyping and lab-to-fab transition of semiconductor technologies for WFD in advanced defense systems\\
  \hline
  E: CHIPS for America International Technology Security and Innovation Fund & 
  To support the development and adoption of secure telecommunications technologies and semiconductors\\
  \hline
  F: National Semiconductor Technology Center(NSTC) &
  To conduct advanced semiconductor manufacturing R\&D and prototyping for innovative technologies and advance workforce training\\
  \hline
  G: National Advanced Packaging Manufacturing Program & 
  A federal program dedicated to R\&D in advanced assembly, test, and packaging (ATP) capabilities through NSTC\\
  \hline
  H: Microelectronics R\&D Manufacturing USA Institute &
  Collaboration between industry, academia, and government to research virtualization of semiconductor machinery, optimizing ATP, and WFD\\
  \hline
  I: National Institute of standards and Technology(NIST) semiconductor programs &
  Various NIST programs to advance material characterization, instrumentation, testing, and manufacturing capabilities\\
  \hline
  J: Public Wireless Supply Chain Innovation Fund &
  To innovate the architecture of software-based wireless technologies, in the U.S. mobile broadband market\\
  \hline
  K: CHIPS for America Workforce and Education Fund &
  Initiate domestic WFD to resolve labor shortages through NSF\\ 
  \hline
\end{tabularx}}
\end{table*}
\end{center}

The U.S. is leading the market share globally in research and development (chip IP design) and equipment manufacturing \cite{fueling_american}. However, any effort by the CHIPS Act directed towards strengthening domestic manufacturing is undermined by the major talent shortage plaguing the entire spectrum, starting from a lack of technicians to design and operations engineers. This has alarming implications, as it keeps the U.S. from maintaining a secure manufacturing capacity in wafer fabrication and ATP, without a talented workforce driving it— which is paramount for innovating the next generation of semiconductor chips for advanced U.S. defense systems, automated machinery, as well as quantum computing. The key industry workforce development challenges that the CHIPS Act targets include \cite{fueling_american}:
\begin{itemize}    

    \item \textbf{Lack of brand awareness associated with semiconductor companies – } Semiconductor companies have a poor brand image and overall low recognition compared to other consumer-facing tech companies (hyper-scalers including Google, Microsoft, Amazon).

    \item \textbf{Industry going obsolete – }Decades of offshore manufacturing and fabrication has rendered this sector of the industry obsolete in the United States.

    \item \textbf{A lack of student interest in hardware electronics in comparison with software –} Starting as early as middle school, U.S. students have little to no exposure to basic electronics and hardware-oriented projects, a critical time for developing academic interests in exploring career opportunities. The focus is more on software skills, with computer science becoming one of the most popular college majors within Science, Technology, Engineering, and Mathematics(STEM) as students hope to find themselves working at big tech companies. As a result, majority of K-12 students remain unaware of the various technical opportunities that offer direct pathways to the semiconductor industry.
    
    \item \textbf{Outdated microelectronics curriculum – }As a result of misalignment between the high tech industry and education system, the current academic curriculum offer little to no exposure to modern semiconductor manufacturing and packaging techniques. This also includes K-12, where technical STEM exposure does not go beyond computer science fundamentals. Subsequently, there exists a wide gap between the entire talent spectrum, leaving a mismatch between K-12 STEM and college graduates’ skills and employer expectations within the industry .

    \item \textbf{Aging faculty and infrastructure – }A shortage of skilled faculty and outdated infrastructure calls for programs directed towards training instructors on state-of-the-art equipment for them to eventually integrate the gained knowledge and experience into their classrooms.
    
\end{itemize}

With \$39 billion dollars being funneled into funding new infrastructure, semiconductor companies are racing to build fabs, which ultimately creates thousands of job opportunities—however, the biggest hindrance will be finding workers to fill the gap with the onset of the talent shortage. There is a strong momentum to develop a secure semiconductor ecosystem in order to achieve long-term self-sufficiency in fabrication, manufacturing, and ATP—and the key in making this attainable is a stable and expanding workforce. While various new educational, training programs and stakeholder partnerships are targeted towards building the next talent pipeline, it is lengthy process with no guarantees. 

In this paper will explore the impacting parameters affecting semiconductor workforce development and proposes two Industry 4.0 capabilities: automation and AR/VR. It is structured as the following: section I. provides a high-level introduction to complex workforce issues that the CHIPS Act targets, as well as the motivation and opportunities from the enactment. Section II presents challenges the semiconductor industry is facing in establishing a domestic workforce and proposes Industry 4.0 solutions. A shift in technological paradigms in presented in section III, which illustrates the evolution of Industry 1.0 to 4.0 paradigms. Section IV introduces the Work Force Development (WFD) pipeline initiative model. Section V  discusses the challenges facing the WFD model as a result of fab working conditions.  The two technologies and future roadmaps are explored in section VI and VII through real-life use cases and unique research case studies conducted by previous literature. The ultimate goal is to showcase Industry 4.0 technologies as promising solutions in workforce education, training, and manufacturing to complement ongoing workforce development efforts in parallel and increase education/workplace desirability.

