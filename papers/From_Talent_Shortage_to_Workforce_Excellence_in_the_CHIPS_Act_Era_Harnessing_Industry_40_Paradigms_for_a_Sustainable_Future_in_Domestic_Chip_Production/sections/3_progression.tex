\section{Industry 4.0 Applications in the Semiconductor Industry}\label{sec3:progression}
    The Industrial Revolution has shifted from mechanizing production to narrowing the gap between AI and brain power, through emulation of human thinking/decision-making by a computer. As highlighted throughput this section and Figure 3, each period has advanced by pushing scientific and technological limits to extend beyond human capabilities.
\subsection{Industry 1.0}\label{sec3_1:industry_1}
The roots of Industry 4.0 stems from the Industrial Revolution, which was propelled by the rapid development in science and technology in the 18\textsuperscript{th} century to streamline production and transportation through mechanization, with the emergence of the steam engine and the spinning wheel. Before Industry 1.0, iron production was arduous, which required workers to manually heat the iron ores with charcoal through a furnace. Eventually, this led to a timber shortage and called for a more efficient process and accessible fuel source. The discovery of coal revolutionized the industry as it fueled both the steam engine to provide mechanical power and revamped iron production \cite{industrial_rev}. By controlling the carbon content present in iron during smelting, the Bessemer process enabled the mass production of steel, which had superior properties to the brittle wrought iron. This transformed the global economy and infrastructure through the development of industrial machinery and complex railroad systems. 


\begin{figure}
    \includegraphics[width=\linewidth]{images/fig3.png}
    \caption{Industrial Evolution}\label{fig:3}
\end{figure}

\subsection{Industry 2.0}\label{sec3_2:industry_2}
The growth of the steel industry paved the way for Industry 2.0, which further modernized industrialization through electricity, assembly lines, and conveyor belts. Henry Ford pioneered the idea of mass production through the innovative combination of the assembly line to produce his affordable Ford Model T vehicles, which ushered in rapid technological globalization by revolutionizing transportation and general manufacturing \cite{model_2020}. Before his unique invention, only one station sustained the full assembly, which then transitioned to a distributed labor system divided into organized steps, inspired by the conveyor belts at that time. This incited the development of industrial-scale assembly lines, which deployed a novel systematic method composed of a rotating belt that transported parts for each worker to install, leading to an equal division of labor. As a result, the industry experienced increased worker productivity and decreased costs, ultimately simplifying and maximizing through-put for factories. 


The key hallmark of Industry 1.0 was ``steam-power”, which harnessed the potential of mechanization to replace the ``muscle-power” endured from manual labor \cite{Wan2022TheRT}, while Industry 2.0 realized the colossal power of sophisticated machinery and electricity in further bolstering the economy.

\begin{figure*}
    \centering
    \includegraphics[width=0.85\textwidth]{images/fig4.png}
    \caption{Applying the Industry 4.0 Paradigm in the Semiconductor Industry}
    \label{fig:4}
\end{figure*}

\subsection{Industry 3.0 evolving towards Industry 4.0}\label{sec3_3:industry_3_to_4}
In contrast, Industry 3.0’s main pivot was directed towards improvising production systems through microelectronics and Information Technology(IT), therefore introducing industrial digitization with advent of the computer. The digitization of factories is rooted in embedded programmable logic controllers incorporated into machinery to allow partial automation and streamlined data collection \cite{industry_4}. Industry 3.0 proliferated the power of automation via the rapid evolution of technological systems, as it currently exists in various aspects of manufacturing, including material tracking, equipment control, and product flow management to advanced process control. The benefits of automated metrology, decision-making, and analysis extends onto faster time-to-data, increased productivity and throughput, as it presents in Industry 4.0.

\subsection{Industry 4.0’s Unique Paradigms}\label{sec3_4:industry_4}
Industry 4.0 is characterized by full automation, intelligent tools with decision-making power, and big data analytics \cite{Butte2016-ii} to efficiently maximize productivity across the entire value chain. It is an amalgamation of recent innovations, such as: IoT, collaborative robots, digital twin modeling, AI, AR/VR, and cloud computing \cite{Zhou2015Industry4T, Cemernek2017-cz}. The seamless cooperation between each domain creates a reconfigurable and adaptive cyber-physical realm that aims to digitize manufacturing. Wan et al. \cite{Wan2022TheRT} defines Industry 4.0 as a fusion of three prime technological domains aimed at developing ingenious dynamics within modern manufacturing, known as the CIOT collaboration. The domains are Collaboration Technology (CT), Information Technology (IT), and the Operational Technology (OT)\cite{Wan2022TheRT}. 

\begin{itemize}
   \item \textbf{Collaboration Domain - }The exchange of big data is conceded through the CT domain, as it encompasses a wide range of communication protocols, including wireless networks such as Bluetooth and Zigbee as well as high-caliber cellular networks \cite{Wan2022TheRT}. 
     \item \textbf{Information Technology Domain - }To provide an intricate classification of OT data, the IT framework addresses the enhanced processing of complex datasets collected by humans and smart sensors through the aid of digital twin modeling, IoT, big data analytics, cloud computing, and artificial intelligence. With the current trajectory towards advanced manufacturing, traditional data analysis and synthesis techniques are projected to be overwhelmed by the enormous datasets that come from complex industrial processes \cite{Kumar2019-mt}, such as in 300 mm wafer fabs, especially as time progresses with growing production demands. 
   \item \textbf{Operational Technology Domain - }The OT serves to create a sustainable industrial operation/production ecosystem through effective management for maintenance of factories and creates a simpler interaction between employees and industrial equipment/materials. 
  
    
\end{itemize}
\subsection{Industry 4.0 in the Semiconductor Industry}\label{sec3_5:industry_4_semiconductor_industry}
\textbf{All three ubiquitous domains form the basis for four potential innovative solutions applicable towards the semiconductor industry:} AR/VR, remote servicing, Computer Vision(CV)/ML enabled automated inspections, and predictive maintenance, as illustrated in Figure 4.
\begin{enumerate}
    \item \textbf{AR/VR }for remote planning and digital twin modeling. Employees can congregate through the CT domain to layout plans through AR/VR and visualize ideas, as well as exchange assistance. Furthermore, AR/VR can be used to simulate the 3D digital twin model of various fab scenarios for optimized production and reduce delays in congregating collaborative meetings through a virtual setting.
    \item \textbf{Remote servicing }is functional throughout all industrial settings. In midst of natural disasters or travel barriers, AR/VR can aid in remote servicing, where a group of experts can join to combat issues, such as equipment failure. It facilitates real-time interaction between the equipment manufacturers and clients by overcoming geographical barriers.
    \item \textbf{Predictive Maintenance }enabled through a network of IoT sensors. The data captured through the sensors is processed through mechanisms of big data analytics. ML algorithms can be developed to analyze the data, which workers can monitor to stay attentive of probable equipment failures that can halt production and create intervention plans ahead of time. With the tool to tool communications, there is not just predictive maintenance, but also  quality bound optimization and automated control adjustments between processes, such as course correction versus the compilation of tolerance failures during manufacturing. The aim is to reduce time spent by workers on diagnosing unexpected failures and repairs.
    \item \textbf{CV/ML enabled automated inspections  }to monitor production quality through analysis of visual data from the production line to spot possible defects and faults. The trained algorithms can streamline failure analysis to ensure maximum yield without needing manual quality inspection (which carries a higher chance of missed defects due to human error).
\end{enumerate}
Adoption of this integrated framework offers multiple unique venues that all play a crucial role in both front-end and back-end semiconductor manufacturing. Every century has carried innovations from all scientific domains to develop technology to supplement human productivity,which serves to optimize yield and maximize efficiency. The rest of this paper analyzes the feasibility of current WFD initiatives and its the pressing challenges, including talent retention and workplace desirability. 