\section{Harnessing AR/VR Technologies in the Semiconductor Industry}\label{sec6:ar_vr}
    Both augmented and virtual reality are becoming growing areas of innovation across various sectors, with ubiquitous functionalities in the education, training, and professional space using VR head-mounted displays, as shown in detail in Figure 7. The main contrast of augmented reality is it combines the physical and virtual realms for a seamlessly interactive environment, in which simulated 3D objects are blended in the real world, while virtual reality is pure simulated replicas of the real-world. The rapid advances in both allows for an enhanced learning experience through realistic practice, which gives users a very high level of immersion and contributes to long-term learning sustainability. As a result, both AR/VR express a promising potential to revolutionize skill acquisition in all domains.

\paragraph {AR/VR in Learning \& Training}
\begin{itemize}
    \item \textbf{Engineering Education:} Both can also transform traditional education by enhancing student comprehension of core engineering subjects and improving performance through gamified simulations, done in parallel with experiential learning in physical laboratories.  Since current microelectronics training programs are faced with aging faculty and a lack of state-of-the-art equipment for lab operations and design software, AR/VR simulations be leveraged to provide a realistic view to clean-room and fabrication equipment through training modules, which is a limited ability for many programs with physical laboratories due to geographical barriers \cite{fueling_american}. Swansea University in Wales has realized this by launching a VR application that introduces the semiconductor industry to school children through gamification for an interactive and intuitive learning experience, as a part of its efforts to expand the industry regionally. The provided virtual interface takes the users through a virtual journey on clean room gowning, lithography, and highlights the importance role of semiconductors as the backbone of modern technology \cite{Thomas2023-ba}. To summarize, AR/VR increases interest in quality learning outside of a mundane environment and equips students with enhanced problem-solving skills to tackle real-world problems, since it gives them a chance to actively exercise their skills in a simulated world \cite{Makarova2015-dk}.
    
    \item \textbf{Employee Training:} AR/VR also supports training novice technicians in realistic work settings and maintenance, which accelerates knowledge acquisition and reduces repair operations (without relying on manuals only). Both serve as innovative training strategies to overcome skill barriers and provide flexible on-the-job training in fast-changing manufacturing environments \cite{Ulmer2020-xk}, which ultimately raises job satisfaction. This was realized through Ford Motor Company, where technicians were trained to diagnose and fix issues of the new e-vehicle's battery systems through modules using a VR headset. VR has the ability to attract new employees, which companies can leverage to brand themselves as high-tech and forward thinking, providing efficiency through accessibility \cite{Dearborn2020-mc}. Additionally, AR technology even shortens the learning curve in medical training by providing doctors with precise virtual surgical simulations. Without having to rely on surgical dummies, it allows doctors to practice anywhere and anytime through a VR headset and a haptic device \cite{Mileva2019-mz}. 
\end{itemize}

The demonstrated use cases illustrate the benefits and outcomes of using AR/VR technology, which can be applied to generate student interest in technical manufacturing as well as to train semiconductor technicians in complex tasks to increase employee engagement in interactive learning.
Interviews were conducted with Research Microscopy Solutions (RMS) experts to provide a Subject Matter Expert (SME) perspective on the use of VR within the industry, specifically for the microscopy community and service engineers. As shown in detail in Figure 8, the general consensus from both experts is that VR provides an immersive learning experience that is sustainable for long-term information retention, as it allows direct hands-on learning through a simulated but realistic environment. While it cannot completely replace traditional learning formats such as physical hands-on experience, it provides a complimentary learning mechanism to foster rapid learning/training with a realistic sense in scenarios that would otherwise be deemed unfeasible.

\paragraph {AR/VR Use Within the Industry}
Other than learning and training, it provides 3D prototyping and remote servicing. 3D-models are converted into Computer-Aided Design(CAD) data files as replicas to construct the virtual environment as a part of the advanced manufacturing mechanism. This allows companies to optimize product quality through enhanced design and modeling, as well as to decrease production costs and the design-to-assembly time\cite{Hamid2014VirtualRA}. Enabling the manipulation of the manufacturing models in the computer, helps users to be immersed in digital environment and improvise each step of the manufacturing process. Furthermore, the interactive environment supports the remote sharing of production data between workers, where one provides feedback and guidance on the operation flow via an established communication network that is attributed to the CT domain.

\begin{itemize}
    \item \textbf{Manufacturing/Assembly:} AR/VR are gaining significant momentum in the assembly and manufacturing spaces with the onset of Industry 4.0. Specifically, AR provides simulated path planning of robotics, maintenance, and assembly, as it allows workers to control quality during production with hands-free access. Spitzer et al. \cite{Spitzer2020-df} provides two cases for industrial uses of AR, one of which is an automated assembly manual system.  In this case study, AR is used to visualize changes during production, from the CAD engineer making design changes, to the assembly planners translating technical documents to instruction manuals and the shop-floor workers performing the updated assembly with it. The AR system adapts and visualizes any edits/updates made by the CAD engineer within the design, which eliminates the time-consuming process of making visual changes within the physical manual and cuts down production time. This increases worker productivity and job satisfaction as it allows the employees to make better decisions with a smoother operation, when the visualized information is embedded within the AR system. Botto et al. \cite{Botto2020AugmentedRF}  further supports AR for assembly use, in a  user study of 21 volunteers using AR on a tablet for assembling given models. The result was a reduced error rate during assembly, with a higher completion time, which is attributed to the limited background knowledge of the users in both AR and manufacturing.
    \item \textbf{Virtual Engineering:} Inspection and maintenance of equipment is one of the core activities in manufacturing. Remote inspection  approaches provide the opportunity to optimize the efficiency of maintenance processes without any time or geographical barriers or physical presence, as it allows technicians and engineers the flexibility to collaborate virtually from anywhere to perform the needed procedures \cite{Linn2017-uc}. 
Also known as virtual engineering, this concept was notably realized by ASML, a leading global supplier of lithography machines for semiconductor industry, with virtual engineers-in-charge. It unveiled an intuitive use of AR during the COVID-19 pandemic \cite{asml2020}, where by a task force of 100 SME across the world gathered remotely to service the lithography machines, which could’ve resulted in million-dollar implications for ASML and its customers in midst of the travel restrictions. The team rapidly developed an AR solution and after quick factory testing at the ASML site in Netherlands, the team fully embedded the AR platform on a smartphone as well as a Microsoft HoloLens headset to complete the necessary service actions at the customer site. The SMEs virtually entered the cleanrooms in customer fabs and completed the servicing via providing real-time instructions to the walk the on-site engineer through performing the troubleshooting and the actual service action— a innovative application of AR successfully applied to a dynamically complex scheme that was resolved through interactive communication converged on a virtual bridge between experts to accomplish the necessitated tasks. AR has immense capabilities in the instant deployment of expertise from anywhere across the globe to surmount any potential barriers.  
    
    \item \textbf{Digital Twin Modeling:} Another dynamic implementation is digital twin modeling for smart manufacturing. By recreating a physical system in a cyber realm, manufacturers can realize potential optimizations, achieve a greater increase in capacity, and maximize production efficiency without the risks associated in physical implementation, which was applied successfully to layout fab planning through the digital twin modeling and simulation of an automated 300mm Infineon Technologies fab \cite{Heinrich2012RulesOA}.
\end{itemize}

Nevertheless, widespread adaptation of AR/VR within both educational and industrial domains is a complex process. Addressing these limitations necessitates a comprehensive framework according to each domain's use standards, which is beyond the scope of this paper. 

\begin{figure}
    \includegraphics[width=\linewidth]{images/ar.png}
    \caption{Applications of AR/VR}\label{fig:7}
\end{figure}

\begin{figure*}
    \centering
    \includegraphics[width=0.99\textwidth]{images/venn.png}
    \caption{Industry perspective from experts on using VR for training}
    \label{fig:8}
\end{figure*}


