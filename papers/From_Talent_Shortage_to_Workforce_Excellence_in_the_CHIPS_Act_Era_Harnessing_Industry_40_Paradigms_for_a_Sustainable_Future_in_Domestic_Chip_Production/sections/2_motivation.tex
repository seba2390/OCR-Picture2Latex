\section{Background and Motivation}\label{sec2:motivation}
\subsection{CHIPS Act and Workforce Development Challenges}\label{sec2_1:challenges_of_chips_act}
Semiconductor manufacturing encompasses a diverse set of expertise from various talent pools, including fab operators and quality technicians to electrical and materials science engineering PhDs performing specialized silicon design. A skills gap exists across the entire value chain within the industry. The main challenges the semiconductor industry is facing is explored, which is also summarized in Figure 2.

\begin{figure*}
    \centering
    \includegraphics[width=0.8\textwidth]{images/image.png}
    \caption{Challenges the semiconductor industry is facing in establishing a domestic workforce.}
    \label{fig:2}
\end{figure*}


\begin{enumerate}
\item \textbf{It is a career path that very few Americans are familiar with due to decades of outsourcing and doesn’t carry the same prestige as software jobs in big tech.} In a survey of 18 industry executives from various member companies of SEMI, 60\% felt that the entire semiconductor sector has a poor perception compared to other tech companies \cite{deloitte}, which tend to be more attractive in brand name. There is also a lack of awareness amongst employees in how dynamic the industry is, bearing the ability to switch to between roles as well upskilling for more lucrative opportunities. 

Even if the US merely focuses on maintaining domestic production for critical applications, this would entail up to 90,000 talented personnel required to staff new fabs \cite{Alam2023-hc}. Recruiting the right talent is a prevailing challenge despite extensive private and public investments towards reshoring semiconductor production. Furthermore, the U.S. has access to the largest talent pool across the globe, as it offers the best system of research institutions worldwide with endless opportunities. As a result, majority of international STEM students choose American colleges and universities to pursue their graduate degrees. 

\item \textbf{Subsequently, only 46\% and 36\% of engineering Master's and PhD students are U.S. citizens, respectively.} Limited number of domestic students end up pursuing advanced engineering degrees compared to international students, which spans a multitude of social and economic factors. For majority of domestic students, a Bachelor’s degree in any engineering discipline earned from U.S. universities is sufficient for a rewarding career, particularly when considering the time and capital investment spent towards graduate degree programs and a general lack of awareness about various scholarship opportunities to fund graduate programs.

To further accelerate the growth of the semiconductor 
ecosystem, there are three key strategies to remain
competitive in the talent war: reskilling, automation, and
expanding the pipeline through workforce development programs \cite{Alam2023-hc}. Leading Integrated Device Manufacturer(IDM) companies such as Intel, Micron, and TI have made notable strides towards reskilling for both current and future workforce through rotational and educational programs. Nonetheless, deploying all these efforts is a long-term investment that requires major time and capital to produce a substantial outcome, which makes it inefficient in accruing enough staff to operationalize the new fabs. Additionally, the mere investments in building fabrications plants will do little to offset the talent shortage, as it projected that by 2030, the U.S. will experience a shortage of 300,000 engineers and 90,000 technicians \cite{Alam2023-hc}. 

\item \textbf{Talent retention issues.} Many jobs within the industry, particularly fab roles, experience low employee retention due to impacting parameters affecting workplace desirability, such as lack of remote option, flexibility, salary, and compensation.

\item \textbf{To meet global chip demands for all market sectors, there is a lengthy learning curve for new or retrained employees to properly execute all the complex steps involved in the fabrication process within a fast-paced fab environment.} Even though large portion of semiconductor jobs are manufacturing intensive, majority remain unfilled with high attrition rates compared to design and analytic positions. 

\item \textbf{There is an income discrepancy between design engineers and quality engineers/manufacturing technicians working in fabs,} as the former is isolated from the hazards of working around risky machinery and chemicals as well as having the ability to maintain a flexible work schedule. The new generation of talent is placing a higher emphasis on hybrid/remote work options, especially after the 2020 pandemic. This is a concern of industry executives as fab roles happen to be less flexible than design counterparts and it is where the acute talent shortage is. However, the industry cannot expect their existing engineers and those in early stages of the value chain to give up better paying, lower stress, flexible desk jobs to be reskilled for demanding fab roles, which can further exasperate this issue as it leads to more attrition and dissatisfaction \cite{Alam2023-hc}.

\end{enumerate}

\subsection{Harnessing Industry 4.0 to Tackle Challenges }\label{sec2_2:harnessing_industry_4.0_to_tackle_challenges}
Ultimately, leading-edge innovations in Integrated Circuit(IC) chip packaging increases manufacturing complexity, which necessitates more process automation along with expanding the US capacity to hire more workers familiar with automation tools amid a competitive talent war. This specifically applies to technicians who are responsible for monitoring equipment and running recalibrations, especially with advancements towards microscale tool operations. The high capital investment that goes into recruiting, training, onboarding, and retaining employees requires dedicated training tools/programs and automation of repetitive tasks as a suitable option to re-structure the workforce, as it can alleviate the demands for scarce talent in the ongoing competitive war while maximizing workplace desirability. Traditional levers of automation include Automated Material Handling System (AMHS), and advanced dispatching and scheduling, which are older automation mechanisms commonly used in semiconductor fabrication facilities (fabs) processes. Moreover, the increased design and packaging complexities, along with outsourced fabrication and ATP, leaves the chip in immense vulnerability, creating the opportunity for potential adversaries to perform malicious activities by tampering with the chips, such as through Trojan insertion. This presents a need for rigorous testing, assurance, and inspection standards to detect and classify defects and Trojans—which can be a mostly automated process if embedded properly. Since a major target of the CHIPS Act is on chip design and wafer fabrication, automation in general can also garner support for increased funding and investment towards a domestic advanced packaging ecosystem since it reduces labor rates, increases supply chain security, and stimulates economic gains of equipment manufacturers by establishing ATP facilities \cite{Ver_Wey2022-qv}. However, the entire value-chain, especially backend manufacturing, have yet to reap the full benefits of Industry 4.0’s digitization for streamlined manufacturing/ATP. 

AR/VR is another Industry 4.0 technology that is sporadically used in the industry, which is becoming more accessible and widespread in various fields. The users are immersed within a digital environment that provides them with a real-world sense of simulated realms, allowing them to see and work with digital objects. The application of AR/VR has a transformative capacity in semiconductor education, workforce training/onboarding, and maintenance/manufacturing/assembly.

\subsection{Introducing the Roadmap to Automation \& AR/VR}\label{sec2_3:roadmap_automation_VR_AR}

The intricate chip design and fabrication flow of advanced packaging and technology necessitates a talented interdisciplinary workforce with extensive knowledge, skills, and abilities in various STEM domains. Nonetheless, U.S. community colleges, technical/vocational schools, and colleges/universities are not supplying enough talent to meet industry demands. Current estimates predict that workforce needs will double as the CHIPS Act, both directly and indirectly, generates thousands of job opportunities within the next couple years \cite{sia_2022}, but there is not enough talent to fill them.

The promising capabilities of automation and AR/VR, especially when used in parallel, increases workplace desirability, decreases manufacturing defects, and optimizes defect detection through recent advancements in AI/ML, ultimately maximizing yields in production.  Villani et al. \cite{Villani2021TheIS} provides a high-level analysis of industrial automation through case studies, which briefly demonstrates that despite rapid progression in modern production systems, human workers remain critical in industrial workplaces and that automation features as a complementing element by handling complex machinery operations to reduce heavy workloads on human workers. Leveraging automation mechanisms serves to improve desirability in work environments as it streamlines tedious and repetitive tasks.

