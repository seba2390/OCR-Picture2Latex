\section{Automation: Current Case Studies and Future Road map}\label{sec7:case_studies}
\subsection{Current Automation Use Cases }\label{sec6_1:current_automation}

The process of semiconductor manufacturing is broadly split across three mains areas: 
wafer manufacturing, chip manufacturing (verification of internal circuitry), and product manufacturing (assembling and testing of the full chip) \cite{Liao2010-rz}. Automation within the semiconductor value-chain spans across a continuum, as the high-volume manufacturing environment is highly dynamic and errors of any degree within any step of production can drastically affect the quality and yield. 

Historically, front-end operations within 200mm fabs were manually handled by clean room technicians, such as material transportation, which affected production throughput due to more human error and higher risk of contamination during wafer processing—this was replaced by the AMHS, which reasonably reduced human action during manufacturing, as well as advanced process control, and production planning. As a result, current fabs are highly automated up to Industry 3.0 level.

Despite the evolved techniques of Industry 4.0, the semiconductor sectors, specifically back-end manufacturing ATP, do not fully harness the power of smart manufacturing, which restricts the valuable time of existing workers spent towards manual and repetitive tasks. Amid the talent shortage, automation releases the pressure the industry is facing by optimizing workplace desirability through automating repetitive and/or hazardous tasks and increasing worker productivity, as well as maximizing profitability in parallel, as automation tools streamline time-consuming processing, assembly, and quality control flow \cite{Alam2023-hc}.

 The SEMI Foundation \cite{Hanny_undated-fl} provides a spectrum that presents the progressing stages of automation for fabs, from level 0 (zero automation) to level 5 (full automation). Level 1 and 2 consists of Industry 3.0 techniques. For level 1, it consists of AMHS, defect control for yield tracking, and limited preventative maintenance, all of which aid the operator at a subsystem level. Advanced Process Control (APC) is introduced at level 2, which involves Run-to-Run (R2R) processing and Fault Defect Classification (FDC). Broader Industry 4.0 applications, such as AI/ML, are introduced in level 3 to provide conditional automation, in providing preventative maintenance techniques and digital twin modeling on cloud-based platforms. Level 4 and 5 further advance level 3 as with the move towards full automation, where AI/ML provides high-level cognitive abilities for operational decision-making.

In addition, back-end semiconductor manufacturing ATP, which is currently present across US-based foundries and IDMs, remains relatively labor-intensive as it has not fully embraced advanced automation technologies for its existing tasks. With the growing use of advanced packaging techniques, there is a call for more Industry 4.0 automation tools, since more manual tasks increase error rates and reduce worker productivity during the execution of complex procedures. The amount of time a worker spends handling various materials and machinery accounts for 30 to 50 percent of all labor, in addition to time spent idle for production cycle completion \cite{De_Backer2017-qd}.  

While the construction of new fabs across the U.S. will open up numerous potential jobs, Accenture presents that certain lower-level fab roles, extensively involved in laborious tasks, will be fading away by 2030 due to automation superseding manual labor across all segments of the industry, as forecasted by the U.S. Bureau of Labor \cite{Alam2023-hc}. A series of case studies will be presented that realize the potential in automating fab roles such as: equipment engineering and technician, quality engineering, characterization, reliability tests engineer, and process engineering. 
 
\subsection{Future Road map and Need for Automation }\label{sec7_2:cases}
Smart manufacturing, within the context of Industry 4.0, is a broad amalgamation of the following innovations and their role: IoT provides data collection platforms and communication mediums, cyber-physical systems programmed with predictive algorithms allow the physical and virtual systems to interact through a IoT platform which provides predictive equipment maintenance capabilities, and cloud computing serves as a network storage base for cloud manufacturing services such as product design and testing simulations \cite{Lin2017DevelopmentOA}. Large datasets of critical processing information is collected and analyzed by an ML algorithm to derive hidden patterns \cite{Khakifirooz2017BayesianIF})

 
 A particular use case within front-end manufacturing is Automated Visual Inspection (AVI) of wafers aims to reduce manual labor spent on quality control while increasing production yield. The general procedure of AVI consists of inspecting the device by sensors and processing the collected data through multitude of CV or ML techniques, which provides feedback throughout manufacturing flow for further optimization \cite{Huang2015AutomatedVI}. Any detected abnormalities notify the quality or process engineer in-charge for further fault assessment, only if a back-up Out-of-Control Action Plan is not implemented to automatically halt the operation to prevent potential yield loss \cite{Barar2020TakingEA}. As a result, this eliminated the need for manual data tracking and analysis.
\subsubsection{Wafer Quality Control} 
In regards to automating quality control during semiconductor manufacturing, Azamfar et al. \cite{Azamfar2020DeepLD} proposes an adaptable deep learning algorithm that’s applicable to a wide range of manufacturing settings that employ varying operating conditions, so it offers a generalized fault detection mechanism to monitor optimal wafer quality, as experimental dataset is obtained from actual semiconductor manufacturing. Using this method allows for predicting the quality of wafers without tedious manual inspection, ultimately reducing human intervention in quality control of fab processes. Since deep learning algorithms require supervised training using enormous datasets to ensure a well-trained model, the same prototype can’t be carried onto the manufacturing industry as it is not feasible to maintain such large database of labeled data that also accounts for varying operating conditions when considering the non-linear nature of manufacturing. This study overcomes this constraint by using cross-domain training for a domain-invariant Convolutional Neural Network (CNN) model, meaning that it’s applicable to various manufacturing processing conditions for reliable quality control. By using data obtained from multiple sensor sources, the trained model was tested with an unlabeled target data that’s also captured from various sensor sources. The model performance is evaluated using datasets obtained during real etching process in wafer processing, which produced favorable outcomes in demonstrating a generalized deep learning model that’s applicable for all scenarios in manufacturing.  The ultimate goal is to lower reliance on human expertise and lessen manual inspection, which also lowers risk of human error and increases worker efficiency by automating a repetitive task for process engineers \cite{Azamfar2020DeepLD}.

Defects occurring in semiconductor devices take on a variety of visual shapes and textures as the fabrication process is highly complex. As a result, manual inspection technique and classification remains dependent on the background of the experts performing the inspects. Imoto et al. \cite{Imoto2018ACT}  introduces a CNN-based transfer learning method for automatic defect classification that overcomes any ambiguity associated with manual classification, by assisting engineers in their tasks. The process consists of classifying images captured by the Scanning Electron Microscope (SEM) system, observing the frequency of each defect type and categorizing it for deeper failure analysis. Deep-learning CNN algorithms necessitate a large labeled dataset of training data, which is not feasible in large-scale manufacturing and various design IPs involved. Transfer learning serves as a solution to this issue through re-use of previous learned tasks on a limited database. At an actual manufacturing site, wafer surface SEM images of various defects were sampled for performance assessment. As a result, the labor for manual inspection was reduced by approximately 2/3 compared to commercially-used automatic defect classification methods.

\subsubsection{AI-Controlled Real-Time Cluster Tool Scheduler for Wafer Processing}
Cluster tools are an automated manufacturing mechanism that consists of multiple computer-controlled process units, a wafer-handling robot, and loadlocks (LLs) for loading and unloading wafers, as a part of the wafer fabrication process \cite{Pan2018-ua}. It allows for reducing the number of equipment needed in wafer processing steps by implementing various process chambers with the same recipe on one platform. Various wafer recipes rotating through the chamber on  different sequences, combined with the overall non-linearity of the manufacturing process, raises the processing complexity and inefficiency. Suerich et al. 
 \cite{Suerich2022ArtificialIF} deploys an AI-controlled cluster tool scheduler that actively collects real-time data and creates optimal scheduling algorithms, which are adaptable to any changes made in the tool processing times, without the need for human intervention. This was realized through virtual tests of various scheduling scenarios on the experimental digital twin models of the equipment (another Industry 4.0 technique) without risking machine downtime. The sensors and automation software is already embedded within the cluster tool system, where the AI algorithm is able to adapt the schedule based on any changes detected from collected data. The main limitation of this model is the time the AI spends searching for the most optimal scheduling solution, which calls for further improvising through offline optimization.

\subsubsection{IC \& System-Level Failure Analysis and Physical Assurance}
While front-end semiconductor manufacturing processes are in a better state in regards to automation, failure analysis and physical assurance of ICs and PCBs remain cases where the potential of intelligent mechanism is yet to be fully leveraged. Traditional inspections techniques have relied on human input to validate and check electronic components for both defects and malicious tampering, which is a time-consuming process prone to human error as the dataset becomes larger \cite{Zhao2022-ee}. As IC packaging opens the avenue for evolved systems through 2.5D and 3D chiplet integration, it also becomes more sophisticated and reliability becomes a much bigger priority, especially as the technologies gain widespread use in critical applications including aerospace, defense, and healthcare. The physical inspection techniques in validating IC packaging and PCBs are categorized into either destructive reverse engineering (mechanical/chemical cross-sectioning) or non-destructive methods through imaging (optical microscopy, x-ray, SEM). The ultimate goal is to streamline the detection and analysis of defects/faults using non-destructive methods as it is not sustainable for to utilize destructive reverse engineering methods in manufacturing setting. Particularly, as advanced IC packages possess highly intricate interconnections, it necessitates a comprehensive workflow with automated image acquisition, data extraction, and analysis to ensure structural reliability and streamline the repetitive task of technicians and engineers involved in manufacturing lines \cite{Yang2021-cp}.

Optical inspection remains the gold standard for detecting any external or internal abnormalities for quality control in part of minimizing yield loss, as it is versatile for all types of electric components without needing destructive sample preparation. Defect classification of cross-sectional images obtained from optical, x-ray, or SEM tools is an arduous task when done manually and necessitate use of virtual metrology and advanced defect classification algorithms. Additionally, miniaturized features in recent nodes may result in noisier images being produced, thus leaving failure analysis heavily dependent on SME input without extensive image processing techniques. However, rapid progressions in machine learning algorithms have allowed for image recognition to become an automated task \cite{Silva2019-bw}, particularly with the advances in ML, particularly CNN, a subset of Deep-Learning Neural Network (DNN), designed for processing high volumes of data. Since defect detection is highly complex process, CV solutions alone cannot provide full assurance. Subsequently, optical inspection solutions are scaling towards ML/DL methods for a reconfigurable approach to failure analysis, which also reduce the need for preprocessing the image \cite{Jessurun2022-vr, Yang2021-bz}. Both techniques can be combined with traditional CV algorithms to complement newer ML-based techniques by extracting parametric features to reduce the amount of data needed to achieve high accuracy when training ML models \cite{Zhao2022-ee}. This reduces the burden on fab engineers that utilize defect information to optimize production lines \cite{Peters2022-wd}.

\begin{itemize}
 \item \textbf{Automated Via Detection for PCBs:}
While this case study's main focus is on Printed Circuit Board (PCB) assurance, IC chips constitute a major part of any electronic system within a PCB, and both chips and PCBs happen to be the most counterfeited component as a result of outsourced manufacturing \cite{Sathiaseelan2021-zh}. Non-destructive reverse engineering of PCBs using X-ray computed tomography remains a 
manual process with a heavy reliance on SME presence, which requires comprehensive inspection to ensure system-level integrity of electronic devices. Botero et al. \cite{Botero2020-pl} explores the automation of non-destructive PCB reverse engineering by
presenting automatic detection of vias, which are small openings that connect the circuity of an IC device and vary a lot depending on
the size and imaging conditions. The via detection framework uses image processing techniques to first detect the via, using 3D constructed images captured from x-ray computed tomography, and then iterates through an unsupervised multi-step process to remove false-positive via detection. The results were confirmed using a deep learning CNN-based algorithm. The aim of the study was to present an automated non-destructive reverse engineering techniques that minimizes human     intervention for validation and verification of  hardware boards.
\end{itemize}

\begin{itemize}
    \item \textbf{Isolating IC TSV Defects:}
    \begin{itemize}
    \item Kim et al. \cite{Kim2018DeepLB} proposes a deep neural network (CNN) architecture to be leveraged for Through-Silicon Via (TSV) defect classification in 3D-ICs through comprehensive analysis in feature extraction and pattern identification. The methodology consisted of training the CNN model with a dataset of TSV defect images from x-ray microscopy and SEM, which outperformed traditional image processing techniques by reducing human dependance throughout the process by as much as 78.6\%, but nevertheless requires a sharper classification accuracy for size-dependent classification if it is to be deployed in fab testing and packaging processes.
    \item Wolz et al. \cite{Wolz2022XrayMA} presents a more detailed view into using x-ray microscopy and deep learning algorithms for precise isolation of wall delamination defects in micrometer-sized Cu-lined TSVs with a low depth-to-diameter ratio, to be used in quality control and process efficiency within R\&D settings. This study visualized the delamination defects by scanning both the whole IC device and the TSV through varying resolutions, with the latter providing better image quality down to sub-micron scale. To establish precise feature extraction and characterization, the database consisted of 30,000 objects to provide a fully automated defect analysis.
    \end{itemize}    

    \item \textbf{Detecting flip-chip C4 bump defects through SAM:}
    Scanning Acoustic Microscope (SAM) is a fast and non-destructive failure analysis technique to identify and inspect faults and isolate defects within microelectronics opaque internal structure, as well as ghost markings, using multiple types of material characterization scans to validate the integrity and authenticity of the chip \cite{Johnson2021-qd}. The transducer sends ultrasonic signals of a certain frequency into the sample, and any reflected signal is a result of changes in material density, voids/cracks, and delamination. The reflected signal is detected by the same transducer and converted back to an electric signal, which is combined to construct an image of the internal structure \cite{YazdanMehr2015AnOO}. Since a majority of automated classification methodologies use either X-ray, optical microscopy, and SEM images to form the image-based algorithms, a large set of high-quality data is required for training. Nair et al. \cite{Nair2022AutomatedDC} proposes a systematic deep-learning approach to interpret and analyze the detected ultrasonic signals autonomously using semi-supervised training with a limited labeled dataset. The model was tested to categorize flip-chip C4 bumps into defect and non-defect sets using the A-scan mode through a CNN algorithm, which outperformed other models in precise classification. 
    
    While further optimization of the current model’s accuracy and algorithm is needed to achieve full-scale industry use, this proposed model resulted in a favorable outcome in automating defect detection for maximum production yield through locating and eliminating a variety of faults while minimizing human expertise in signal interpretation.  
\end{itemize}

\subsubsection{Advanced Packaging Manufacturing}
    \begin{itemize}
    \item \textbf{Screening On-Package Faulty Passive Device:}
    Integrated Passive Devices (IPDs) have become increasingly prevalent in semiconductor advanced packaging, enhancing power integrity, and impedance matching. The demand for guaranteeing signal and power integrity in chips used in safety-critical applications like automotive, aviation, industrial, and defense systems has increased. IPD is used in various analog and Radio-Frequency Integrated Circuit(RFIC) for enabling on-package passive devices. IPDs significantly improve the quality and reliability of these chips and their packaging. As a result, conducting comprehensive testing and screening of IPDs becomes essential. It is crucial to note that replacing faulty IPDs far exceeds the manufacturing cost, emphasizing the significance of screening defective IPDs before their installation. Chuang et al. \cite{IPD_detection} proposed a machine learning-based screening method to detect faulty IPD, specifically capacitors with potential reliability issues. Using parametric data gathered from 360,000 integrated passive devices (IPDs) during the wafer probing test, their developed ML model’s primary objective is identifying IPDs with low breakdown voltage, which signifies reduced reliability. Using the ML algorithm, this method can successfully detect and eliminate 6 to 15 times more faulty dies, significantly improving the screening process. 
    Automating the detection of faulty on-package passive devices will result in an increased overall chip manufacturing yield rate. However, this works only focuses on detecting the capacitor. Further automation is needed to increase reliability to detect faulty on-package passive devices like inductors and resistors.

    \item \textbf{Optimizing Manufacturing Process for Reliability:}
    As advanced packaging is increasingly getting complex due to the introduction of 2.5D or 3D packaging, the reliability of semiconductor packaging is becoming an issue. Adapdix\cite{adapdix_manufacturing} developed an ML algorithm to optimize the manufacturing process to ensure reliability in semiconductor packaging manufacturing. An example showcases utilizing an AI/ML platform that operates directly at the machine's edge, leveraging real-time machine data and operational data. This platform integrates with various machine data sources such as sensors, actuators, PLC, and log data, utilizing appropriate communication protocols. This integration enables access to a wide range of real-time data, including parameters like acceleration, rotation, conveyor speed, gripper position, bonding head motion, epoxy dispense pressure, and other relevant machine part parameters. These real-time data points are critical in evaluating machine and process performance. This foundation facilitates the development of ML models, effectively harnessing the power of the collected data. Numerous prominent global enterprises have successfully employed these capabilities in semiconductor fabs, advanced packaging, fiber optics for transceivers, and precision placement for Surface Mount Technology (SMT). Remarkable enhancements have been observed, including improved alignment accuracy, increased process yield, reduced cycle time, and minimized machine downtime. Using automated data acquisition and analysis with AI/ML will effectively minimize the human expertise to detect and predicts faults manually. As semiconductor packaging continues evolving and becoming more complex, there are future scopes for the researcher to improve automation and optimization of semiconductor manufacturing to solve the reliability problem. 

    \item \textbf{Automated Thermal Characterization of IC Packaging:} Proper thermal characterization of packages  ensures maximum performance and reliability of ICs for power electronic applications \cite{noauthor_2007-vf}. To maintain the IC junction temperature at a optimal level, there needs to be effective thermal management of heat flow through the system's electrical paths, from the IC to the substrate. One of the roles of a packaging engineer is to predict and classify the patterns of IC package substrates' thermal conductivity for a reliable end device. However, this is becoming a labor-intensive task with potential human errors due to increasing complexities in advanced packaging designs. Kim et al. \cite{Kim2022-qm} proposes a CNN-based algorithm for an automated prediction algorithm that determines the Effective Thermal Conductivity (ETC) of packaging substrates using CNN for heat transfer optimization. The CNN algorithm is trained to recognize and extract patterns from the imaged substrate layers for precise prediction by dividing the layer-pattern images into single unit cells. From there, the local ETC is determined from each cell, with all of them re-grouped into a larger unit  afterwards for a comprehensive thermal analysis. This result were confirmed through finite element simulations of the package CAD file. The ETC serves as a critical parameter of temperature distribution since the IC constitutes the main heat source in the device system, and a stable ETC provides optimal device performance. The goal of this study was to reduce human labor in thermal characterization of packaging, which ultimately decreases risk of human error as well. 


        \item \textbf{ML-based Classifiers for Predicting IC Yield in 3D Packaging:} ML algorithms can be used to predict results of final package device tests, with the goal of spotting bad dies coming from front-end fabs before they are packaged, using dataset obtained from early stage wafer fabrication testing for each production unit. This is especially critical for 3D IC packaging devices, where one bad die means the whole package fails. As a result, this leads to profit loss and puts extra burden on packaging engineers. Chen et al. \cite{Chen2017-xb} overcomes this issue by presenting a ML classifier model to maximize IC yields using a batch (to overcome imbalances in the dataset), online (the model is updated when a new dataset is received), and incremental learning (adapting the classifier to changes in data distribution as a result of non-linear manufacturing conditions) framework. This approach is tested on 3D flash memory chips using real data from an industry partner, where the ultimate goal is to separate bad dies, which undergo a lower cost testing and packaging process to be used in a low-end product. Predicting and isolating the good dies beforehand increases efficiency as the the packaging engineer is able to spend more time on packaging and memory testing the good dies, which also maximizes profits as the number of high-end products is increased. This developed algorithm resulted in a 3.4\% yield improvement in a 16 die stack case, which shows a promising ability to streamline the packaging process in a back-end manufacturing setting.

    
    \end{itemize}



