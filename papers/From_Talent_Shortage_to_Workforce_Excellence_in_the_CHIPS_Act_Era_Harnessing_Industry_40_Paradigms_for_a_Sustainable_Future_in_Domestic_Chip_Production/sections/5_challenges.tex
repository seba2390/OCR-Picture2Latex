\section{Challenges to Development of a Secure Workforce}\label{sec5:challenges}
However, while the WFD talent pipeline initiative has a promising vision on securing and expanding a robust workforce, significant time commitments and the current working conditions within the industry bring forth major challenges to onboard and retain a stable workforce, particularly for existing fab roles. Successful recruitment and hire of new technicians and engineers may not securely close the existing gap if the impacting parameters that affect workplace desirability of microelectronics jobs are not addressed, as a high attrition rate remains a major issue.

\begin{figure*}
    \centering
    \includegraphics[width=0.85\textwidth]{images/fig6.png}
    \caption{Key industry workforce development challenges}
    \label{fig:6}
\end{figure*}

Since majority of critical jobs are manufacturing-intensive, there is a lack of talent inflow into those roles. 
The various parameters, which are also highlighted in Figure 6, include:
\begin{enumerate} 
    \item Long overtime hours spent in fabs
    \item Lack of industry preparation and limited onboarding
    \item Less competitive salaries compared to other sectors
   \item Lack of remote options within manufacturing roles
\end{enumerate}
All of these synergistically contribute to a work-life imbalance. As the aging workforce retires, recent college graduates view manufacturing roles as reserved for those that did not pursue a college degree, as they also hold contrasting perceptions compared to previous generations, such as placing higher emphasis on flexible working hours to fit their lifestyles \cite{Deichler2021-gv}. Having the option to work remotely from home is another motivating factor that’s not feasible in the manufacturing realm. Within specific engineering roles at semiconductor companies, it is common for process, quality, or equipment engineers and technicians operating the fabs to endure prolonged working hours for weeks compared to those working in R\&D, design, and analyst positions in a corporate setting. This is highlighted with the multitudes of issues that are already starting to surface with the clash between U.S. employees and TSMC's working conditions. 

\subsection{Case Study: TSMC}\label{sec5_1:tsmc}
TSMC, a global leading manufacturer of semiconductor chips, is spending \$40 billion to build two major semiconductor foundries in Arizona, based on the 3nm and 4nm process, with plans to produce the most advanced chips in building a resilient U.S. supply chain that meets annual chip demands. To do so, it will need approximately 4,500 new hires to operate the two fabs, but has already earned a poor reputation amongst both current and prospective employees. On the company’s Glassdoor page, employees anonymously reveal their experiences about the brutal working environment: cultural clashes, high-stress fab work, misaligned management, 12-hour work days, and a cultural rift \cite{Lau2023-du}. Meanwhile, U.S. workers are accustomed to standard working conditions with expectations of a competitive salary and reasonable compensation. As a part of the onboarding process, U.S. staff are required to train at the company’s main hub in Taiwan for 12-18 months, which is a considerable downside for potential graduates entering the workforce. The next issue is salary, which is comparatively low compared to big tech and even rivaling companies in the industry, notably Intel, Micron, and GlobalFoundries—both of which still experience similar challenges in staffing their news plants despite a well-established reputation within the U.S. technological industry. 

Decades of offshoring has not only eroded the domestic supply chain and experienced workers, but the entire talent ecosystem and expertise as well, including instructors, consultants, and other support systems needed to sustain the semiconductor industry and even extends to general high tech manufacturing. Regardless of the current manufacturing gap and staffing challenges, doubled by the hurdles in initiating new plant operations, the potential geopolitical and economic gains far outweigh the risks for leading companies \cite{Lau2023-du}. 

\ To summarize, rebuilding a successful workforce with continuous output of talent while improving workplace desirability for the next generation is a part of CHIPS Act’s objectives coming to fruition. To keep U.S. competitive in the manufacturing race calls for measures in investing in two innovative solutions to complement WFD efforts: Automation and AR/VR. To revitalize domestic chip production, the semiconductor industry immensely benefits from leveraging digitization, realized through Deep Learning (DL) (a subset of ML), along with Computer Vision(CV), in broader areas of manufacturing, packaging, and failure analysis/detection \cite{Lee2021-lo, Nakazawa2019-ki, Kim2020-pw, Batool2021-yv, Lu2022-mt, Adly2014-nk, Hsu2021-qt, Ma2019-tv, Richter2019-pg, Ghosh2021-lg, cryptoeprint:2022/924, Asadizanjani2017-xe}. This is coupled with operation processes becoming more sophisticated with heterogeneous integration in the context of advanced packaging techniques. The following section introduces various opportunities in automation and AR/VR to streamline various fab roles to support worker productivity and maximize workplace desirability.

