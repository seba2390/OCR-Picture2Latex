\documentclass[final,a4paper,3p,times]{elsarticle}
\usepackage{fancyhdr}
\newcommand\hmmax{0} %This is added to avoid an error for too many math alphabets
\newcommand\bmmax{0}
%\usepackage{authblk}
\usepackage{array}
\usepackage{gensymb}
\usepackage[section]{placeins}
\usepackage{caption}
\usepackage{float}
\usepackage{breqn}

\makeatletter
\def\ps@pprintTitle{%
  \let\@oddhead\@empty
  \let\@evenhead\@empty
  \def\@oddfoot{\reset@font\hfil\footnotesize{\emph{\today}}}
  \let\@evenfoot\@oddfoot
}
\makeatother

\usepackage{siunitx}
\usepackage{inputenc}
\renewcommand{\thefigure}{S\arabic{figure}}
\renewcommand{\theequation}{S\arabic{equation}}
\usepackage{subfigure}
\usepackage[abs]{overpic}
\usepackage{booktabs}
%\usepackage[square,sort&compress,comma,numbers]{natbib}
\usepackage{textcomp}
\usepackage{multirow}
%\usepackage{a4wide}
\usepackage{times}
% \usepackage{lineno}
% \usepackage[sort&compress,numbers]{natbib}
%\usepackage[square,sort&compress,comma,numbers]{natbib}
\bibliographystyle{elsarticle-num-names}
% \setcitestyle{square, comma, numbers,sort&compress, super}
% \usepackage[style=numeric-comp,sortcites,natbib]{biblatex}
\newcolumntype{L}[1]{>{\raggedright\let\newline\\\arraybackslash\hspace{0pt}}m{#1}}
\newcolumntype{C}[1]{>{\centering\let\newline\\\arraybackslash\hspace{0pt}}m{#1}}
\newcolumntype{R}[1]{>{\raggedleft\let\newline\\\arraybackslash\hspace{0pt}}m{#1}}
\usepackage[bottom]{footmisc}
\renewcommand{\thefootnote}{\fnsymbol{footnote}}
\usepackage{hyperref}
\usepackage{tabu}
\usepackage{enumitem}
\usepackage{scrextend}
%\usepackage[a4paper, total={6in, 8in}]{geometry}
\def\changemargin#1#2{\list{}{\rightmargin#2\leftmargin#1}\item[]}
\let\endchangemargin=\endlist 
\usepackage{booktabs, siunitx}
\usepackage[svgnames,table]{xcolor}
\usepackage[tableposition=below]{caption}
\usepackage{gensymb}
\usepackage{tikz}
\usetikzlibrary{arrows,calc,positioning}
\tikzstyle{B1}=[draw,text centered,minimum size=2.3 em,text width=4 cm,text height=0.1 cm]
\tikzstyle{B2}=[draw,text centered,minimum size=2.3 em,text width=1.5 cm,text height=0.1 cm]
\tikzstyle{B3}=[draw,text centered,minimum size=2.5 em,text width=2.5 cm,text height=0.1 cm]
\definecolor{mycolor}{RGB}{170,255,255}
\usepackage{graphicx}
\usepackage{amsmath}
\usepackage[nameinlink]{cleveref}
\crefname{figure}{Fig.}{Figs.}
\crefname{equation}{Eq.}{Eqs.}
\crefname{table}{Table}{Tables}
\crefname{section}{Section}{Sections}
\Crefname{figure}{Figure}{Figures}
\Crefname{equation}{Equation}{Equations}
\Crefname{table}{Table}{Tables}
\Crefname{section}{Section}{Sections}
\usepackage{amssymb}
\usepackage{setspace}
\usepackage{geometry}
\usepackage{longtable}
\usepackage{mathtools}
\usepackage{physics}
\usepackage{siunitx}
%\renewcommand{\thesection}{S\arabic{section}.}
\usepackage{hyperref}
\hypersetup{
  %  backref =       true,
   % pagebackref  =  true,
    colorlinks =    true,
    linkcolor =     [rgb]{0.0,0.0,1.0},
    anchorcolor =   [rgb]{0.0,0.0,1.0},
    citecolor =     [rgb]{0.0,0.0,1.0},
    filecolor =     [rgb]{0.0,0.0,1.0},
    urlcolor =      [rgb]{0.0,0.0,1.0},
    pdftitle=       {Title},
    pdfsubject=     {Title},
    pdfauthor=      {A. Uthor}
}
\newcolumntype{R}{>{\centering\arraybackslash}m{0.12\textwidth}}


\makeatletter
\long\def\MaketitleBox{%
  \resetTitleCounters
  \def\baselinestretch{1}%
  \begin{center}%
   \def\baselinestretch{1}%
    \Large\@title\par\vskip18pt
    \normalsize\elsauthors\par\vskip10pt
    \footnotesize\itshape\elsaddress\par\vskip36pt
%    \hrule\vskip12pt
%    \ifvoid\absbox\else\unvbox\absbox\par\vskip10pt\fi
%    \ifvoid\keybox\else\unvbox\keybox\par\vskip10pt\fi
%    \hrule\vskip12pt
    \end{center}%
  }
\makeatother
%\title{Supplementary Material:\\Design of auxetic cellular structures for in-plane response through out-of-plane actuation of stimuli-responsive bridge films}
% \date{}
%\date{\vspace{-5ex}}
\usepackage{etoolbox}
\preto\thesection{S}

%\renewcommand{\thesubfigure}{S\arabic{subfigure}}
\begin{document}
\begin{frontmatter}
%%
% \newcolumntype{R}{>{\centering\arraybackslash}m{0.12\textwidth}}
\title{Supplementary Material:\\Design of auxetic cellular structures for in-plane response through out-of-plane actuation of stimuli-responsive bridge films}
%\date{\today}
%\maketitle
%\doublespacing
%\singlespacing
\author[inst1]{Anirudh Chandramouli\corref{cor2}}
\author[inst2]{Sri Datta Rapaka\corref{cor2}}
\author[inst1]{Ratna Kumar Annabattula}
\cortext[cor2]{Equal contribution}
\affiliation[inst1]{organization={Department of Mechanical Engineering},%Department and Organization
addressline={Indian Institute of Technology Madras}, 
city={Chennai},
postcode={600036}, 
state={Tamil Nadu},
country={India}}
\affiliation[inst2]{organization={Haas Formula One},
city={Banbury},
postcode={OX16 4PN},
country={UK}}
\end{frontmatter}
\hrule
\section{Derivation of equivalent stiffness of torsion spring for a hexagonal honeycomb unit cell}
The analytical study of the coupled response of a unit cell and an attached stimuli-responsive bridge film to an external stimuli field in this work involved approximating the unit cell response to a linear spring and two torsion springs. While the derivation of equivalent linear response to forces is available in the literature \citep{gibson_ashby_1997}, the study of its response to moments at its vertex is not available; therefore, the equivalent torsional spring stiffness is not readily known. Therefore, in this section, the derivation of this stiffness is discussed\footnote{The MATHEMATICA code used for the derivation of an expression for $\tau$ can be downloaded from \href{https://github.com/AnirudhChandramouli/HexagonalUnitCell_Stiffness.git}{https://github.com/AnirudhChandramouli/HexagonalUnitCell\_Stiffness.git}}. 
\begin{figure}[htpb]
\begin{center}
\subfigure[]{
    % \includegraphics[width=0.3\textwidth]{Images_SupplementaryA/CaseA_trimmed.pdf}
    \includegraphics[width=0.38\textwidth]{Images_SupplementaryA/TorsionSchematic-cropped.pdf}
    \label{fig:Schem1}
    } \hfill
\subfigure[]{
    % \includegraphics[width=0.3\textwidth]{Images_SupplementaryA/CaseB_trimmed.pdf}
    \includegraphics[width=0.53\textwidth]{Images_SupplementaryA/3PartSchematic_4-cropped.pdf}
    \label{fig:Schem2}
    } 
\end{center}
    \caption{(a) Hexagonal unit cell subject to moments at the vertices due to the bending of the attached stimuli responsive bridge film. The red and blue planes indicate planes of symmetry. (b) Simplified model of the response of a quarter section of the bridge film to an external moment about the $X$-axis at the vertex.}
    \label{fig:Schematic}
\end{figure}

Consider the unit cell subject to a moment at its vertices, arising from the bending of the bridge film, as shown in \cref{fig:Schem1}. Note that the resulting moment acts only at the vicinity of the vertex and thus cannot be analyzed as a simple bending problem of the entire unit cell section. Consequently, to derive its stiffness with respect to this moment, the system is idealized as presented in \cref{fig:Schem2}. The vertical side walls show negligible deformations, and the inclined walls are split into three sections. Section II is the section that is directly attached to the bridge film, and sections I and III are mirror symmetric. The following assumptions are made in the derivations which follow
\begin{itemize}
    \item The total stiffness of the unit cell is seen as the stiffness of the three sections acting in parallel (additive).
    \item The wall thickness is small, and thin plate approximations hold.
\end{itemize}
Since the moment acts across the entire cross-section of II, the stiffness of II can be calculated as $\tau_{II}=2EI/w$, where $E$, $I$, and $w/2$ are the Young's modulus, area moment of inertia and length of section II respectively. The moment on section I acts only at the corner (refer to point 2 in \cref{fig:Schem2}). To derive its stiffness, we consider this section in local coordinates ($xyz$) with its origin at 4, wherein the section lies horizontally along the $xy$ plane. The deformation of the plate due to the moment is approximated through suitable polynomials. The vertical displacement ($u_z$) along the $z$-axis is assumed to be of the form
\begin{equation}
    u_z=(\alpha+\beta y^2+\gamma y^4)(\mu x + \nu x^3),
\end{equation}
where $\alpha$, $\beta$, $\gamma$, $\mu$ and $\nu$ are constants. Here, only even terms of the polynomial are considered along $y$ to make the function even (due to mirror symmetry - refer to the red plane in \cref{fig:Schem1}) while it is assumed to be odd along $x$. The boundary condition on 3-4 implies that $(\partial u_z/\partial x)|_{x=0}=0$, implying that $\mu=0$. Therefore, with $a=\alpha\nu$, $b=\beta\nu$ and $a=\gamma\nu$, we have
\begin{equation} \label{eqn:w}
    u_z=(a+b y^2 + c y^4)(x^3)
\end{equation}
Further, since section II is assumed to be narrow, the deformation of section I must be such that 2-5 has 0 displacement along the X-axis. In other words, the deformation of 2-5 must occur along a plane parallel to the $YZ$ plane and containing 2-5. Note that in the local coordinate, this is a plane containing 2-5 and at an inclination of $\theta$ (about $x$) from the $yz$ plane. This implies that the displacement ($u_x$) along $x$ at $x=l$ must be
\begin{equation}
    u_x|_{x=l}=(u_z|_{x=l})\tan{\theta}
\end{equation}
We assume the variation of $u_x$ along $x$ to follow $u_z$ and therefore, we have
\begin{equation} \label{eqn:u}
    u_x=(u_z|_{x=l})\tan{\theta}(x^3/l^3)=u_z\tan{\theta}
\end{equation}
Now, assume that a moment $M$ at 2 (refer \cref{fig:Schem2}) causes an inclination of $\psi$ at 2 about the $X$-axis. Therefore, we have 
\begin{equation} \label{eqn:slopecondn}
    \frac{\partial u_z/\partial y}{\cos{\theta}}\bigg|_{(x=l,y=w/2)}=\psi.
\end{equation}
\Cref{eqn:slopecondn} can be used to eliminate a constant ($b$) from \cref{eqn:w} to yield
\begin{equation} \label{eqn:wf}
    u_z=(a+ \left(\dfrac{\psi\cos{\theta}}{w l^3}-\dfrac{c w^2}{2}\right) y^2 + c y^4)x^3
\end{equation}
Finally, the displacement ($u_y$) along $y$ is assumed to be linear along $y$ and follow a cubic variation similar to $u_z$ and $u_x$ along $x$ to give
\begin{equation} \label{eqn:v}
u_y=f y x^3,    
\end{equation}
where $f$ is a constant. The strain in the plate is given by
\begin{subequations} \label{eqn:epsilon}
\begin{equation}
    \epsilon_{xx}=\pdv{u_x}{x}-z\pdv[2]{u_z}{x}.
\end{equation}
\begin{equation}
    \epsilon_{yy}=\pdv{u_y}{y}-z\pdv[2]{u_z}{y}.
\end{equation}
\begin{equation}
    \epsilon_{xy}=\left(\pdv{u_x}{y}+\pdv{u_y}{x}\right)/2-z\pdv{u_z}{x}{y}.
\end{equation}   
\end{subequations}
Using \cref{eqn:u,eqn:v,eqn:wf} in \cref{eqn:epsilon}, we write the strains in the plate as follows
\begin{align}
    \epsilon_{xx}&=\frac{3 x (x \tan{\theta}-2 z) \left(l^3 w \left(2 a-c w^2 y^2+2 c y^4\right)+2 \psi  y^2 \cos{\theta}\right)}{2 l^3 w} \label{eqn:exx}\\
    \epsilon_{yy}&=x^3 \left(c z \left(w^2-12 y^2\right)+f-\frac{2 \psi  z \cos{\theta}}{l^3 w}\right) \label{eqn:eyy}\\
    \epsilon_{xy}&=\frac{x^2 y \left(l^3 w \left(-c x \tan{\theta} \left(w^2-4 y^2\right)+6 c z \left(w^2-4 y^2\right)+3 f\right)+2 x \psi  \sin{\theta}-12 \psi  z \cos{\theta}\right)}{2 l^3 w} \label{eqn:exy}
\end{align}
Neglecting the effect from the Poisson's ratio, the corresponding stresses are given by
% \begin{align}
%     \sigma_{xx}&=E \epsilon_{xx}\\
%     \sigma_{yy}&=E \epsilon_{yy}\\
%     \sigma_{xy}&=(E/2) \epsilon_{xy}
% \end{align}
\begin{subequations} \label{eqn:sigma}
\begin{equation}
     \sigma_{xx}=E \epsilon_{xx}.
\end{equation}
\begin{equation}
    \sigma_{yy}=E \epsilon_{yy}.
\end{equation}
\begin{equation}
    \sigma_{xy}=(E/2) \epsilon_{xy}.
\end{equation}   
\end{subequations}
Furthermore, the energy of an infinitesimal element of the section is given by
\begin{equation} \label{eqn:dut}
    dU=\frac{1}{2}\left(\sigma_{xx}\epsilon_{xx}+\sigma_{yy}\epsilon_{yy}+\sigma_{xy}\epsilon_{xy}\right)dxdydz.
\end{equation}
Therefore, using \cref{eqn:sigma} and \cref{eqn:exx,eqn:eyy,eqn:exy} in \cref{eqn:dut}, we obtain
\begin{dmath} \label{eqn:du}
    dU=\frac{E x^2}{16 l^6 w^2}\bigg(18 (x \tan{\theta}-2 z)^2 (l^3 w (2 a-c w^2 y^2+2 c y^4)+2 \psi  y^2 \cos{\theta} )^2
    +x^2 y^2 (l^3 w (c x \tan{\theta} (w^2-4 y^2)-3 (2 c z (w^2-4 y^2)+f))
    -2 x \psi  \sin{\theta}+12 \psi  z \cos{\theta})^2
    +8 x^4 (l^3 w (c z (w^2-12 y^2)+f)-2 \psi  z \cos{\theta})^2\bigg)dxdydz.
\end{dmath} 
The total energy ($U$) is found by integrating \cref{eqn:du} through the volume as $U=\int^{t_w/2}_{-t_w/2}{\int^{w/2}_0{\int^l_0{dU}}}$ to give
\begin{dmath} \label{eqn:U}
    U=\frac{E t_w}{11289600 l^3 w} \bigg(3 l^2 w^2 \big(\tan ^2{\theta} \big(1693440 a^2 l^6-98784 a c l^6 w^4+160 c^2 l^8 w^6+2247 c^2 l^6 w^8+10584 w^2 \psi ^2 \cos (2 \theta )+10584 w^2 \psi ^2\big)-56 l^3 w \tan{\theta} (70 c f l^4 w^3-9 \psi  \sin{\theta} (560 a-27 c w^4))+160 l^8 (840 f^2-c^2 w^6)+20 l^2 \sec ^2{\theta} \big(8 c^2 l^6 w^6+28 c l^3 w^3 \psi  \cos (3 \theta )+441 f^2 l^4 w^2 \cos (2 \theta )+441 f^2 l^4 w^2+245 f l^2 w \psi  \sin (3 \theta )-35 \psi ^2 \cos (4 \theta )+35 \psi ^2\big)-140 l^4 w \psi  \sec{\theta} (4 c l w^2-35 f \tan{\theta})\big)+7 t_w^2 \big(403200 a^2 l^6 w^2-23520 a c l^6 w^6+24 l^3 w^3 \psi  \cos{\theta} (2800 a-3 c w^2 (56 l^2+45 w^2))+3840 c^2 l^{10} w^6+576 c^2 l^8 w^8+535 c^2 l^6 w^{10}+9600 l^4 \psi ^2+5040 l^2 w^2 \psi ^2+120 \psi ^2 \cos (2 \theta ) (80 l^4+42 l^2 w^2+21 w^4)+2520 w^4 \psi ^2\big)\bigg)
\end{dmath}
Finally, the constants $a$, $c$, $f$, and $\psi$ are obtained through energy minimization with respect to these constants. In other words, the following set of equations is solved simultaneously
\begin{subequations} \label{eqn:MinEqns}
\begin{equation}
    \pdv{U}{a}=0
\end{equation} 
\begin{equation}
    \pdv{U}{c}=0
\end{equation} 
\begin{equation}
    \pdv{U}{f}=0
\end{equation} 
\begin{equation}
    \pdv{U}{\psi}=M
\end{equation} 
\end{subequations}
Using \cref{eqn:U} in \cref{eqn:MinEqns} and solving simultaneously using the symbolic computing toolbox in MATHEMATICA\textsuperscript{\textregistered} yields
\begin{dgroup}
    \begin{dmath}
        a= 525 M w^2 \sec{\theta} \Big(3 l^2 w^2 \tan ^2{\theta} (28800 l^4+48965 l^2 w^2+7056 w^4)
        -28 (160 l^2+21 w^2) (2400 l^4-81 l^2 w^2-20 w^4) t_w^2\Big)/G,
    \end{dmath}
    \begin{dmath}
        c=882000 M \sec{\theta} \Big(4 (63 l^2+20 w^2) (160 l^2+21 w^2) t_w^2
        +l^2 \tan ^2{\theta} (9600 l^4+23075 l^2 w^2+3024 w^4)\Big)/G,
    \end{dmath}
    \begin{dmath}
        f=1176000 M w^2 \tan{\theta} \sec{\theta} \Big(-7 (600 l^5+27 l^3 w^2+10 l w^4) t_w^2\\
        -9 l^3 w^2 \tan ^2{\theta} (5 l^2+14 w^2)\Big)/G,
    \end{dmath}
    \begin{dmath} \label{eqn:psi}
        \psi=25200 l^3 M \sec ^2{\theta}\Big(840 w (160 l^2+21 w^2) (20 l^4+3 l^2 w^2+w^4) t_w^2+l^2 w^3 \tan ^2{\theta} (96000 l^4+245945 l^2 w^2+31752 w^4)\Big)/G,
    \end{dmath}    
\end{dgroup}
where $G$ is given as
\begin{dmath*}
    G=E t_w \Big(9 l^4 w^4 \tan ^4{\theta} \left(240000 l^6+672875 l^4 w^2+215180 l^2 w^4+21168 w^6\right)+5 l^2 w^2 \tan ^2{\theta} \left(51840000 l^8+101026800 l^6 w^2+18697875 l^4 w^4+1043252 l^2 w^6+42336 w^8\right) t_w^2+140 \left(160 l^2+21 w^2\right) \left(72000 l^8+48600 l^6 w^2+13701 l^4 w^4+630 l^2 w^6+20 w^8\right) t_w^4\Big).
\end{dmath*}
It is evident from \cref{eqn:psi} that $\psi$ is linear with respect to M (Note that small strain assumptions have been made). Consequently, the stiffness of section I (and III equivalently) can thus be found as $\tau_{I}=M/\psi$, which yields
\begin{dmath} \label{eqn:tau1}
        \tau_I=G/\Big(25200 l^3 \sec ^2{\theta}\big(840 w (160 l^2+21 w^2) (20 l^4+3 l^2 w^2+w^4) t_w^2+l^2 w^3 \tan ^2{\theta} (96000 l^4+245945 l^2 w^2+31752 w^4)\big)\Big).
    \end{dmath}
Therefore, using \cref{eqn:tau1}, the total stiffness of the equivalent torsion spring can thus be written as
\begin{dmath} \label{eqn:tau}
        \tau=\tau_{II}+2G/\Big(25200 l^3 \sec ^2{\theta}\big(840 w (160 l^2+21 w^2) (20 l^4+3 l^2 w^2+w^4) t_w^2+l^2 w^3 \tan ^2{\theta} (96000 l^4+245945 l^2 w^2+31752 w^4)\big)\Big),
    \end{dmath}
where $\tau_{II}$ is given by
\begin{equation}
    \tau_{II}=\frac{E \left(d^3 \tan ^2{\theta} \sec{\theta} t_w+4 d \cos{\theta} t_w^3\right)}{24 w},
\end{equation}
where $d$ is the width of the bridge film. The above expression is found by evaluating the area moment of inertia of II as twice that of a rectangular section rotated through $\theta$. 

The stiffness of the equivalent torsion spring can thus be approximated by \cref{eqn:tau}. It is to be noted that the above expression has been derived assuming a certain polynomial form for the displacements. Consequently, it is found that the \cref{eqn:tau} holds only within a certain domain. Accordingly, the above expression is found to be valid for $(10\leq l/t_w\leq 60)$. A large class of hexagonal cellular structures fall within this domain, and the above formula can be used to approximate their stiffness. On the other hand, higher accuracies and larger domains of validity require higher-order approximations.
\bibliography{references}

\end{document}