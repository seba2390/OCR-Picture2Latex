\documentclass[11pt,a4paper]{article}
\usepackage[hyperref]{tacl2018v2}

\usepackage[T1]{fontenc}
\usepackage{times}
\usepackage{plex-mono}

\usepackage{latexsym}
\usepackage{amsmath,amssymb}
\usepackage{dsfont}

\usepackage{algorithm}
\usepackage{algpseudocode}


%%%% table stuff
\usepackage{array,etoolbox}
\preto\tabular{\setcounter{magicrownumbers}{0}}
\newcounter{magicrownumbers}
\def\rownumber{}

%%%% graphics
\usepackage{tikz}
\usetikzlibrary{arrows,matrix,positioning,fit,calc,shapes,decorations.pathreplacing,svg.path,shadows,patterns}
\pgfdeclarelayer{background}
\pgfdeclarelayer{backgroundoverlay}
\pgfsetlayers{background,backgroundoverlay,main}
\usepackage{pgfplots}
\pgfplotsset{compat=1.13}
\usepackage{adjustbox}
\usepackage{ocr}
\newcommand{\circnum}[1]{\raisebox{.5pt}{\textcircled{\raisebox{-.5pt}{\scalebox{.8}{\(#1\)}}}}}
\newcommand{\ofw}[1]{\ensuremath{{#1}\raisebox{.06em}{\scalebox{.7}{$(w)$}}}}
\newcommand{\charword}[1]{\scalebox{.9}{\ocrfamily #1}}
\newcommand{\spelling}{s}
\newcommand{\wughighlight}[1]{{\ocrfamily\color{black} #1}}
%%%% end of graphics

\renewcommand{\UrlFont}{\ttfamily\small}

\usepackage{dirtytalk}
\usepackage{microtype}

\usepackage{booktabs}
\usepackage{arydshln}
\usepackage{multirow}
\usepackage{array}
\newcolumntype{H}{>{\setbox0=\hbox\bgroup}c<{\egroup}@{}}

\usepackage{bm}

\newcommand{\tokdetok}[0]{\textsc{XRayEmb}}
\newcommand{\tok}[0]{\textsc{XR-Enc}}
\newcommand{\detok}[0]{\textsc{XR-Dec}}
\newcommand{\tdloop}[0]{E$\rightarrow$D}
\newcommand{\dtloop}[0]{D$\rightarrow$E}
% \newcommand{\tokdetok}[0]{\textsc{TokDetok}}
% \newcommand{\tok}[0]{\textsc{Tok}}
% \newcommand{\detok}[0]{\textsc{Detok}}
\newcommand{\marco}[0]{\textsc{MarcoQA}}
\newcommand{\unilm}[0]{Unigram LM}
\newcommand{\llm}[0]{LPLM}
\newcommand{\wpc}[0]{WordPiece}
\newcommand{\ppt}[0]{\textsc{2pt}}
\newcommand{\mmod}[0]{\textsc{$M$}}
\newcommand{\gen}[0]{\textsc{Pred}}
\DeclareMathOperator*{\agg}{agg}

\renewcommand{\vec}[1]{\bm{#1}}
\newcommand{\tribracket}[1]{\textless#1\textgreater}
\newcommand{\vE}{\ensuremath \vec{E}}
\newcommand{\vI}{\ensuremath \vec{I}}
\newcommand{\vc}{\ensuremath \vec{c}}
\newcommand{\vh}{\ensuremath \vec{h}}
\newcommand{\vb}{\ensuremath \vec{b}}
\newcommand{\ve}{\ensuremath \vec{e}}
\newcommand{\vr}{\ensuremath \vec{r}}
\newcommand{\vw}{\ensuremath \vec{w}}
\newcommand{\vx}{\ensuremath \vec{x}}
\newcommand{\vy}{\ensuremath \vec{y}}
\newcommand{\vf}{\ensuremath \vec{f}}


\taclpubformattrue % Uncomment this line to deanonymize

\setlength\titlebox{8cm} % comment out for review version

\title{Learning to Look Inside: Augmenting Token-Based Encoders with Character-Level Information}

\author{Yuval Pinter\thanks{Work done as an intern at Bloomberg LP and as a Bloomberg PhD Data Science Fellow.} \\
  School of Interactive Computing \\
  Georgia Institute of Technology \\
  Atlanta, GA, USA \\
  \texttt{uvp@gatech.edu} \\\And
  Amanda Stent \\
  Bloomberg \\
  New York, NY, USA \\
  \texttt{astent@bloomberg.net} \\\AND 
  Mark Dredze \\
  Bloomberg \\
  Department of Computer Science \\
  Johns Hopkins University \\
  \texttt{mdredze@cs.jhu.edu} \\\And
  Jacob Eisenstein \\
  Google Research \\
  \texttt{jeisenstein@google.com} \\}

\date{}

\begin{document}
\maketitle
\begin{abstract}
    Commonly-used transformer language models depend on a tokenization schema which sets an unchangeable subword vocabulary prior to pre-training, destined to be applied to all downstream tasks regardless of domain shift, novel word formations, or other sources of vocabulary mismatch.
    Recent work has shown that ``token-free'' models can be trained directly on characters or bytes, but training these models from scratch requires substantial computational resources, and this implies discarding the many domain-specific models that were trained on tokens. In this paper, we present \tokdetok{}, a method for retrofitting existing token-based models with character-level information.  
    \tokdetok{} is composed of a character-level ``encoder'' that computes vector representations of character sequences, and a generative component that decodes from the internal representation to a character sequence.
    We show that incorporating \tokdetok{}'s learned vectors into sequences of pre-trained token embeddings helps performance on both autoregressive and masked pre-trained transformer architectures and on both sequence-level and sequence tagging tasks, particularly on non-standard English text.
\end{abstract}


\section{Introduction}
\label{sec:intro}


% To insert a figure: % $Id: template.tex 11 2007-04-03 22:25:53Z jpeltier $
\documentclass{vgtc}                          % final (conference style)
%\documentclass[review]{vgtc}                 % review
%\documentclass[widereview]{vgtc}             % wide-spaced review
%\documentclass[preprint]{vgtc}               % preprint
%\documentclass[electronic]{vgtc}             % electronic version

%% Uncomment one of the lines above depending on where your paper is
%% in the conference process. ``review'' and ``widereview'' are for review
%% submission, ``preprint'' is for pre-publication, and the final version
%% doesn't use a specific qualifier. Further, ``electronic'' includes
%% hyperreferences for more convenient online viewing.

%% Please use one of the ``review'' options in combination with the
%% assigned online id (see below) ONLY if your paper uses a double blind
%% review process. Some conferences, like IEEE Vis and InfoVis, have NOT
%% in the past.

%% Figures should be in CMYK or Grey scale format, otherwise, colour 
%% shifting may occur during the printing process.

%% These few lines make a distinction between latex and pdflatex calls and they
%% bring in essential packages for graphics and font handling.
%% Note that due to the \DeclareGraphicsExtensions{} call it is no longer necessary
%% to provide the the path and extension of a graphics file:
%% \includegraphics{diamondrule} is completely sufficient.
%%
\ifpdf%                                % if we use pdflatex
  \pdfoutput=1\relax                   % create PDFs from pdfLaTeX
  \pdfcompresslevel=9                  % PDF Compression
  \pdfoptionpdfminorversion=7          % create PDF 1.7
  \ExecuteOptions{pdftex}
  \usepackage{graphicx}                % allow us to embed graphics files
  \DeclareGraphicsExtensions{.pdf,.png,.jpg,.jpeg} % for pdflatex we expect .pdf, .png, or .jpg files
\else%                                 % else we use pure latex
  \ExecuteOptions{dvips}
  \usepackage{graphicx}                % allow us to embed graphics files
  \DeclareGraphicsExtensions{.eps}     % for pure latex we expect eps files
\fi%

%% it is recomended to use ``\autoref{sec:bla}'' instead of ``Fig.~\ref{sec:bla}''
\graphicspath{{figures/}{pictures/}{images/}{./}} % where to search for the images
\usepackage{microtype}                 % use micro-typography (slightly more compact, better to read)
\PassOptionsToPackage{warn}{textcomp}  % to address font issues with \textrightarrow
\usepackage{textcomp}                  % use better special symbols
\usepackage{mathptmx}                  % use matching math font
\usepackage{times}                     % we use Times as the main font
\renewcommand*\ttdefault{txtt}         % a nicer typewriter font
\usepackage{cite}                      % needed to automatically sort the references
\usepackage{tabu}                      % only used for the table example
\usepackage{booktabs}   % only used for the table example
\usepackage{caption} 
\captionsetup[table]{skip=10pt}
\usepackage{color}


%%% 
%%% Meta notes
%%%
\newlength{\dummylen}
\newcommand{\NOTE}[1]{\setlength{\dummylen}{\fboxrule}\setlength{\fboxrule}{2pt}%
            \vspace{1ex}\noindent\hfill%
            \fbox{\begin{minipage}{.96\columnwidth}#1\end{minipage}}%
            \setlength{\fboxrule}{\dummylen}\hfill{}\vspace{1ex}}

%\renewcommand{\NOTE}[1]{\ignorespaces}



%% We encourage the use of mathptmx for consistent usage of times font
%% throughout the proceedings. However, if you encounter conflicts
%% with other math-related packages, you may want to disable it.


%% If you are submitting a paper to a conference for review with a double
%% blind reviewing process, please replace the value ``0'' below with your
%% OnlineID. Otherwise, you may safely leave it at ``0''.

\onlineid{0}

%% declare the category of your paper, only shown in review mode
\vgtccategory{Research}

%% allow for this line if you want the electronic option to work properly
\vgtcinsertpkg

%% In preprint mode you may define your own headline.
%\preprinttext{To appear in an IEEE VGTC sponsored conference.}

%% Paper title.

\title{Recognizing Handwritten Source Code}
%\title{Writing Source Code (literally!)
%% This is how authors are specified in the conference style

%% Author and Affiliation (single author).
%%\author{Roy G. Biv\thanks{e-mail: roy.g.biv@aol.com}}
%%\affiliation{\scriptsize Allied Widgets Research}

%Author and Affiliation (multiple authors with single affiliations).
\author{Qiyu Zhi\thanks{e-mail: qzhi@nd.edu} %
\and Ronald Metoyer\thanks{e-mail:rmetoyer@nd.edu}}
\affiliation{\scriptsize University of Notre Dame}

%% Author and Affiliation (multiple authors with multiple affiliations)
% \author{Roy G. Biv\thanks{e-mail: roy.g.biv@aol.com}\\ %
%         \scriptsize Starbucks Research %
% \and Ed Grimley\thanks{e-mail:ed.grimley@aol.com}\\ %
%      \scriptsize Grimley Widgets, Inc. %
% \and Martha Stewart\thanks{e-mail:martha.stewart@marthastewart.com}\\ %
%      \parbox{1.4in}{\scriptsize \centering Martha Stewart Enterprises \\ Microsoft Research}}

%% A teaser figure can be included as follows, but is not recommended since
%% the space is now taken up by a full width abstract.
% \teaser{
%  \includegraphics[width=1.5in]{sample.eps}
%  \caption{Lookit! Lookit!}
% }

%% Abstract section.
\abstract{
%Programming requires textual input and typically requires typing which is not always the %optimal input method for all users - especialy on touchscreen devices. 
Supporting programming on touchscreen devices requires effective text input and editing methods.  Unfortunately, the virtual keyboard can be inefficient and uses valuable screen space on already small devices.  Recent advances in stylus input make handwriting a potentially viable text input solution for programming on touchscreen devices.  
The primary barrier, however, is that handwriting
recognition systems are built to take advantage of the rules of natural language, not those of a programming language. In this paper, we explore this particular problem of handwriting recognition for source code.  
We collect and make publicly available a dataset of handwritten \textit{Python} code samples from 15 participants and we characterize the typical recognition errors for this handwritten \textit{Python} source code when using a state-of-the-art handwriting recognition tool.  We present an approach to improve the recognition accuracy by augmenting a handwriting recognizer with the programming language grammar rules. Our experiment on the collected dataset shows an 8.6\% word error rate and a 3.6\% character error rate which outperforms standard handwriting recognition systems and compares favorably to typing source code on virtual keyboards.
} % end of abstract

%% ACM Computing Classification System (CCS). 
%% See <http://www.acm.org/class/1998/> for details.
%% The ``\CCScat'' command takes four arguments.


\keywords{programming, handwriting recognition, touch screen, source code, python.}
\CCScatlist{ 
  \CCScat{H.1.2.}{User/Machine Systems}{Human information processing}{};
  \CCScat{H.5.2.}{User Interfaces}{Input devices and strategies (e.g., mouse, touchscreen)}{}
}


%% Copyright space is enabled by default as required by guidelines.
%% It is disabled by the 'review' option or via the following command:
% \nocopyrightspace

%%%%%%%%%%%%%%%%%%%%%%%%%%%%%%%%%%%%%%%%%%%%%%%%%%%%%%%%%%%%%%%%
%%%%%%%%%%%%%%%%%%%%%% START OF THE PAPER %%%%%%%%%%%%%%%%%%%%%%
%%%%%%%%%%%%%%%%%%%%%%%%%%%%%%%%%%%%%%%%%%%%%%%%%%%%%%%%%%%%%%%%%

\begin{document}

%% The ``\maketitle'' command must be the first command after the
%% ``\begin{document}'' command. It prepares and prints the title block.

%% the only exception to this rule is the \firstsection command
%\firstsection{Introduction}

\maketitle

\section{Introduction} %for journal use above \firstsection{..} instead
%Supporting programming on mobile devices is not a novel idea. 
With the rapid technology shift in current computing devices, high-quality low-cost mobile devices such as tablets and smartphones are being increasingly used in everyday activities. Many tasks that previously required a PC are now feasible on mobile devices. For example, tablets are typically equipped with powerful batteries, advanced graphic processors, high-resolution screens and fast processors, making writing and compiling code on them completely plausible. TouchDevelop, for example, is a novel programming environment, language and code editor for mobile devices  \cite{tillmann2011touchdevelop}. Furthermore, Tillmann et al. predict that programming on mobile devices will be widely used for teaching programming \cite{tillmann2012future}.  However, mobile devices are also inherently restricted by their limitations such as small screens and the clumsy virtual keyboard.  Entering and editing large amounts of text for programming tasks can quickly become difficult and time consuming with these virtual keyboards because they are notoriously difficult to use when compared to a physical keyboard and they consume valuable screen space\cite{raab2013refactorpad}.

%that is difficult to use and that takes up valuable screen space - posing challenges for a programmer who wishes to enter or edit source code.


% comparing handwriting and typing in different scenarios to motivate the work. 
While keyboards have been the primary input device for entering computer programs since the computer was invented\cite{gordon2013improving}, this predominant mechanism is not ideal for all programming situations. For example, software developers that suffer from repetitive strain injuries (RSI) and related disabilities may find typing on a keyboard difficult or impossible \cite{begelprogramming}.
%people suffering from repetitive strain injuries (RSI), carpal tunnel syndrome, or other motor impairments may experience tremendous difficulty using a keyboard and mouse \cite{arnold2000programming}.
Instead, handwriting with a stylus may be a preferred input mechanism for some of these users \cite{mankoff1998cirrin}.
%In addition, entering and editing large amounts of text for programming tasks can quickly become difficult and time consuming with
%\emph{virtual} keyboards, for any user, because they are notoriously difficult to use when compared to a physical keyboard and they consume valuable screen space\cite{raab2013refactorpad}.
%People with disabilities  (e.g. limb loss) or who do not have full mobility may also find typing on a keyboard  (physical or virtual) is difficult or impossible \cite{begelprogramming}. 
In addition, some physical configurations (e.g. seated on a plane) may simply be more suited to the writing posture than a typing posture for many users.

Handwriting has also been shown to have potential cognitive benefits\cite{alonso2015metacognition}.  In particular, Mueller and Oppenheimer found that students who took longhand notes performed better on conceptual questions than those that typed notes on a laptop \cite{mueller2014pen}.  Given these findings, and that fact that many programmers write pseudocode by hand before typing, it is reasonable to consider that handwriting may provide cognitive benefits for programming, especially on mobile devices. 
Furthermore, recent advances in pen-based input and handwriting recognition technology are quickly making handwriting a viable alternative to typing.
%Entering and editing large amounts of text for
%programming tasks can quickly become difficult and %time consuming
%with \emph{virtual} keyboards that are notoriously %difficult to use when compared to a physical %keyboard and consume valuable screen %space\cite{raab2013refactorpad}. 
%It is, therefore, important that we explore handwriting as an input option to support these users and their needs.
%keyboards are not conveniently available on many of today's touchscreen devices or come in the form of \emph{virtual} keyboards that are notoriously difficult to use and require the use of valuable screen space \cite{}.  


% One alternative to typing is to use speech (via dictation) to create the text of a program. Desilets et al. \cite{desilets2006voicecode} propose VoiceCode, which translates the pronounced syntax into native syntax in the current programming language to support programming by speech. Gordon \cite{gordon2013improving} employs language design and incorporates dynamic context for this purpose. Speech, however, is not always an appropriate option given social conventions and privacy issues.
% %While many alternatives have been explored, few have thoroughly examined the use of handwriting as a means for textual input, especially when considering source code. 
% Given the recent advances in pen-based input, however, handwriting is a potentially viable alternative to typing and speech.
%other input forms, such as speech, which are not always appropriate.

%Given recent advances in pen-based input, handwriting code is becoming a potentially viable alternative to typing.


In this paper, we explore the use of handwriting as a means for source code text input.  There are two ways to approach this problem.  One alternative is to develop or modify a handwriting recognition engine to take source code directly into account.  Given that source code often includes English language words, another alternative is to leverage the capabilities of an existing English language handwriting recognition engine. We explore this latter option.  First, we collect and present a publicly available dataset of handwritten \textit{Python} source code for use in handwriting recognition research.  Second, we explore the use of the state-of-the-art recognition system, MyScript \cite{myscript} for recognizing \textit{Python} source code.  We characterize the errors made by the MyScript engine and present a method for post-processing the engine's results to improve recognition performance on handwritten \textit{Python} source code.   

After presenting related work and necessary background information, we describe our data collection process and the resulting publicly available dataset. We then describe the performance of MyScript on recognizing the handwritten source code and present our algorithm for leveraging the MyScript engine to produce improved results.  We discuss those results in Section \ref{results} and conclude with avenues of future work.


%leveraging programming language grammar and lexicon information into a current handwriting recognition system, MyScript \cite{myscript}. Our primary contributions include a handwritten source code dataset, a characterization of errors made when using a general handwriting recognition system on the source code, and an algorithm to leverage general handwriting recognition systems to produce improved results for handwritten source code.


% overview of our work

%Another possible situation is when keyboards are not available, which is becoming a common scenario because touchscreen devices such as the advanced iPad and Android-based tablet is sufficient for daily use.

\section{Background and Related Work}
% Handwriting source code
% For each paper, what did they do  (few sentences), and what did they find, how are we going to be different.

% Handwriting recognition


Many alternatives to typed source code have been considered, typically in the context of making programming more accessible.   Most of those appoaches fall in the realm of speech-based programming \cite{desilets2006voicecode, gordon2013improving}, however speech is not always an acceptable solution, especially in quiet environments or with applications that require privacy.
%One approach is to use human speech (via dictation) to create the text of a program. Desilets et al. \cite{desilets2006voicecode} propose VoiceCode, which translates the pronounced syntax into native syntax in the current programming language to support programming by speech. Gordon \cite{gordon2013improving} employs language design and incorporates
%dynamic context for this purpose.   
Given the recent advances in pen-based input, handwriting is a potentially viable alternative.  The pendragon supports people who are unable to use
a keyboard and seeks to find new interaction techniques for input which may improve communication speed \cite{Pendragon}. Mankoff et al.also suggest that word prediction, sentence completion and the syntax of programming languages could be used for handwriting source code \cite{mankoff1998cirrin}. 
Most closely related to our work is a programming IDE integrated with a handwriting area in which the handwritten code is recognized by an enhanced handwriting recognition system \cite{frye2008pdp}.  This work, however, does not present an evaluation of the recognition engine or guidance for how to improve general handwriting recognition engines for application to source code recognition.
%However, they failed to present a clear evaluation for the handwritten source code recognition system.
In this paper, we focus on the recognition step of handwritten source code as the most important step for developing an effective handwriting interface for source code input and editing.

Research in handwriting recognition has a long history dating back to the 1960s~\cite{tappert1990state}. Hidden Markov Model (HMM) based handwriting recognition~\cite{hu1996hmm,kundu1988recognition,nag1986script} is one of the 
most widely used approaches while neural networks are gaining in popularity~\cite{jaeger2001online}. Some approaches also leverage additional constraints for recognizing handwriting in specific domains such as postal addresses \cite{srihari1993recognition, srihari1993interpretation} and banking checks \cite{gorski1999a2ia, agarwal1997bank}. These handwriting recognition systems are developed to take advantage of the English language \cite{van2003using}, which is intrinsically different from source code. For instance, variable names are often created from
concatenated words (e.g. camelCase or underscore naming), which poses a problem for the traditional handwriting
recognition system as it expects spaces to appear between words
contained within its dictionary.   We do not aim to contribute to the extensive literature in handwriting recognition, but rather, we intend to examine how we can leverage this existing work for application to handwritten source code recognition interfaces.  




\section{Data Collection}

Our first contribution in this paper is a data collection study designed to generate a sample set of handwritten source code for research purposes.  The first question to consider is what programming language to study.  We decided to collect handwritten \textit{Python} source code because of the current popularity of \textit{Python}\footnote{\url{http://www.tiobe.com/tiobe-index/}} and its projected growth rate 
%at some high-profile
%organizations, such as Google, Disney and the US
%Government 
\cite{radenski2006python}. 
We chose a ``copying task'', where three code samples are provided for every participant to copy on the tablet using the stylus. While we understand that a ``copying task'' may be cognitively quite different from other writing tasks that require synthesis, we sought to eliminate sources of cognitive load that could impact timing as well as writing quality for the purposes of this data collection task. The three shared code samples allow for comparison across participants. To broaden our dataset of unique handwritten source code samples, we also randomly selected a fourth source code sample function (per participant) that was unique to that participant. In this section, we describe our data collection process and the resulting database of handwritten \textit{Python} samples.


%its current popularity as a programming language and it's growth trajectory.  



%In order to test the recognition system, we collected a dataset of handwritten \textit{Python} code samples written by students from the [anonymized for review] campus. We chose \textit{Python} because of its current popularity as a programming language. 

\subsection{Participants}
We recruited 15 participants (9 females) from the University of Notre Dame for our study. Thirteen of the participants were computer science majors and all participants had at least two semesters of programming experience. Their ages ranged from 19 to 29 (mean = 22.3). Two participants were left-handed. Eight participants had used a pen/stylus for handwriting on a touchscreen device and only one participant had used a tablet for inputting source code (via the virtual keyboard). All participants were compensated \$5 for the study which took approximately 30 minutes each.

\subsection{Apparatus and Software}

\begin{figure}[tb]
 \centering % avoid the use of \begin{center}...\end{center} and use \centering instead (more compact)
 \includegraphics[width=\columnwidth]{datacollection}
 \caption{Screen shot of our data collection web application.  Participants entered their name and code sample number into the boxes in the upper left.  They then entered the code sample using the Apple Pencil and selected `Save' when finished.}
 \label{fig:webpage}
\end{figure}


We used a 12.9 inch iPad Pro with a 2732-by-2048 screen resolution at 264 pixels per inch (PPI) and fingerprint-resistant oleophobic coating. Participants used the Apple Pencil as the stylus device. We implemented a web application with a writing area to record user input. This application was responsible for converting touch points of the stylus into handwriting strokes and saving strokes to a JSON file. Each stroke consists of the coordinates of the sampled points and time-stamp information for each coordinate. The writing area in this application measured 795 * 805 pixels with subtle lines on the background to provide guides for the participants (See \autoref{fig:webpage}). We also implemented functions to undo or redo the previous stroke as well as clear the writing area of all strokes. 

\subsection{Representative Source Code Material}
Our goal was to create a database of representative samples of handwritten \textit{Python} source code for use in evaluating the performance of a handwriting recognition system. Because different \textit{Python} samples contain different language elements, there is no single representative corpus\cite{almusaly2015syntax}. Ideally, representative code samples should contain a variety of language constructs and not be restricted to a single project. 

Our process for choosing source code samples is based on that used by McMillan et al. \cite{mcmillan2012exemplar, rodeghero2014improving}. First, we selected six popular \textit{Python} projects on Github. \autoref{table:project} summarizes the details of these projects. We then extracted all functions from the project source code and eliminated comments in order to focus solely on the source code of the samples.  Next, to obtain functions that were sufficiently long to collect a substantial amount of handwriting, but not so long as to require multiple pages of handwriting, we filtered the functions to those with between 9 and 18 lines of source code and those with no lines greater than 60 characters (to eliminate long, wrapping lines).  We also manually filtered out highly repetitive functions, such as a function that includes only assignment statements for variables. The result was 1324 eligible functions. We randomly selected the three shared test code samples from this set for use by all participants and one additional unique code sample to be entered by each participant in the study. Although \textit{Python} syntax considers whitespace, we decided to ignore indentation for the purposes of focusing purely on handwriting recognition. 
%We are currently in the process of augmenting the collected samples to produce a second identical dataset with indentation inserted.


%We collected this unique code sample for each participant to broaden the coverage of code samples and characterize the common recognition errors from the common recognition system as well.


\subsection{Procedure}

For every participant, we began our data collection with an informed consent process.  Each participant then filled out a pre-study questionnaire about demographics and experience using touchscreen devices and a stylus. Participants were given a practice task to familiarize them with the process.  For each of the four tasks, participants were given a sheet of paper with the sample typed \textit{Python} source code.   Participants entered their name and code sample number into the web application and then entered the code sample using the Apple Pencil and selected `Save' when finished.  After completing all four input tasks, participants were compensated and the session ended.


\subsection{Data Collection Results}
The final dataset includes stroke data for four code samples for each of 15 participants resulting in a total of 60 handwritten source code samples. So, for each of 3 given source code input samples, we have 15 copies of handwritten source code (for a total of 45 handwritten source code samples). The remaining 15 are handwritten samples of unique input source code examples from each participant. 
The handwritten source code data can be downloaded at \newline \url{http://www.purl.org/recognizinghandwrittencode/data}.
 
\begin{table}
  \centering
  \begin{tabular}{l r r r}
    % \toprule
    {Project}
    & {Lines}
      & {Fuctions}
    & {Eligible Functions} \\
    \midrule
    AlphaGo & 1,963 & 151 & 1 \\
    Bittorrent & 7,164 &  570 & 39 \\
    Blender & 265,684  & 12,774 & 1,126 \\
    Instagram & 1,265 & 145 & 8 \\
    Requests & 14,009  & 862 & 84 \\
    Webpy & 10,199 & 1,029 & 66 \\
    % \bottomrule
  \end{tabular}
  \caption{\textit{Python} projects used for selecting code samples}~\label{table:project}
\end{table}



\section{Source Code Recognition Errors}

Current commercial handwriting recognition systems are built to take advantage of the rules of the English language as opposed to that of a programming language, therefore it is not surprising that these systems might perform poorly on source code recognition \cite{frye2008pdp}. There is, however, no previous research that evaluates how well existing state-of-the-art handwriting recognition systems perform on handwritten source code. Here we describe the state-of-the-art handwriting recognition system we employed and characterize the errors based on the dataset we collected.

\subsection{State-of-the-art: Myscript}
Automatic recognition of handwriting is now a mature discipline that has found many commercial uses\cite{plamondon2000online}. MyScript\cite{myscript} is an online handwriting recognition engine that supports more than 80 languages and achieved the best recognition rate in the International Conference on Document Analysis and Recognition competition\cite{el2011line}.
Here we use the MyScript engine as our baseline for comparison and study the typical recognition errors produced when applied to handwritten source code to better understand the complexities introduced by \textit{Python} source code and source code in general.


\subsection{Data Pre-Processing}

In order to use MyScript efficiently and to make a fair comparison between its performance and our algorithm, we apply two simple pre-processing steps to the data. First, we provide MyScript with \textit{Python} specific context through the Subset Knowledge (SK) facility and a custom lexicon. SK is a MyScript feature for telling the recognizer that we only want it to enable recognition of certain characters. For example, for a phone number field, we may want only digits to be recognized.   We created an SK resource in MyScript to allow only characters that can legally appear in \textit{Python} source code.  We also provide the legal \textit{Python} keywords through a user-defined lexicon.

%We used two main MyScript resources to process the data we collected: 
%The MyScript Cloud Development Kit (CDK) and Subset knowledge(SK).
The MyScript Cloud Development Kit (CDK) is an HTTP-based set of services that take handwritten strokes as input and produce potential recognition results as output. To use the CDK in experiments, we must send strokes to the recognizer at some level of granularity (e.g. single character, whole word, whole line, etc).
%The data processing step is to process the data for submission to the MyScript Cloud Development Kit (CDK). Pre-processing includes character filtration and sentence separation. 
%Character filtration is to restrict the recognition result to certain characters. We used the Subset knowledge(SK) feature by creating an SK resource in MyScript to allow only characters that can legally appear in \textit{Python} source code. In sentence separation, 
We chose to break the stroke data into lines assuming developers might write a statement at a time on a single line. To do so, we analyze stroke coordinates and create a new line each time the user moves to a new vertical position. Each line is then sent one at a time to the MyScript CDK.  This simulates a developer writing one programming statement (one line) at a time, pausing at the end of each line.  Alternative pre-processing is possible given the raw stroke data and timestamp information (e.g. sending incomplete lines when a participant pauses).

\subsection{Characterizing errors}
\label{sec:characterizing}
We processed all of the handwritten data as described above to collect baseline recognition results for all handwritten samples in our dataset.  We then set out to understand the types of recognition errors that were present in the final recognized text. We identified three major types of recognition errors: word errors, symbol errors, and space errors. 

Word errors occur when MyScript simply incorrectly recognizes a written word. This is typically due to poor writing and can occur for keywords as well as non-keywords.
For example, when the handwritten word `self' is recognized as `silt', we characterize this as a word error. 
Symbol errors represent incorrect recognition of symbols or non alpha-numeric characters.
For example, an `\_' (underscore) is often recognized as a `-' (dash). 
Finally, a space error results when the system inserts an unexpected space.
For example, when `ConflictError' is recognized as `Conflict Error', we characterize it as a space error. 

Most of the word errors and symbol errors can be attributed to poor writing or cursive writing (characters are written joined together in a flowing manner) which is inherently more difficult for MyScript to recognize than block writing (characters are written separately). Space errors, on the other hand, appear to depend on the language model of the recognizer, which most likely does not include training on CamelCase\footnote{\url{https://en.wikipedia.org/wiki/Camel\_case}} or proper English words separated by dot notation (e.g. student.name).  The result is that MyScript inserts space at these word and dot notation separators.

In summary, from the statistical results for each type of error presented in \autoref{fig:errors}, space errors, mainly caused by the internal mechanism of English handwriting recognition system, represent the most prevalent recognition error. In addition, poor writing and the tendency to return an English word for a non-English word in the source code lead to word errors, which also represents a significant portion of all errors. Symbol errors are also a prevalent error type. This makes sense given that MyScript is designed to recognize general words, however, symbols, dot notation, and combinations of symbols and words are typically not present in general text, especially in the way that they are used in source code. For example, the most problematic symbols includes underscore `\_', parentheses `( )' and equal `='.


\begin{figure}[t!]
\centering
\includegraphics[width=3.16in, height = 2.2in]{errors}
\caption{Average error numbers of all participants for each code sample from MyScript general handwriting recognition engine}
\label{fig:errors}
\end{figure}


\section{Handwritten Source Code Recognition Pipeline}



A programming language is governed by grammar rules, which stipulate the positions of keywords and symbols. For example, in \textit{Python}, a \textit{def} sentence must end with a `:'. However, handwritten symbols are often problematic.  For example, colons `:' are sometimes recognized as semicolons `;'. In addition to grammar rules, programming languages are highly repetitive with predictable properties\cite{hindle2012naturalness}. 
Function names and variable names are the most common repetitive words in a single source code project. If a function name appears more than once in the same handwritten code sample, however, it is impossible for users to hand write the \textit{exact} same strokes for this function name, which makes different recognition results of the same handwritten function name a possibility that we must account for.

In this section, we present an approach to improve the recognition rate for handwritten source code by addressing these issues as well as those common errors characterized in Section \ref{sec:characterizing}. We leverage what we know about the predictability and structure of source code to improve recognition results beyond that of the state-of-the-art recognizer. 


The general premise of our approach is that state-of-the-art engines can produce excellent results given good writing and the absence of symbols and programming practices like camelCase.  Our framework, illustrated in \autoref{fig:overview}, is therefore aimed at analyzing and post-processing the recognition results produced from MyScript to utilize its recognition capabilities but correct for those common errors. This framework can be divided into four parts: statement classification, statement parsing, token processing, and statement concatenation. 
%The various parts are presented in the following sub-sections.
The source code for this post-processing algorithm can be found at \url{http://www.purl.org/recognizinghandwrittencode/code}.

\begin{figure}[h!]
\centering
\includegraphics[width=0.5\textwidth]{overview}
\caption{Framework for augmenting MyScript to correct for common recognition errors in handwritten source code.}
\label{fig:overview}
\end{figure}

\begin{figure*}[h]
 \center
  \includegraphics[width=2.2in, height = 1.32in]{picture1}
  \includegraphics[width=2.2in, height = 1.32in]{picture2}
  \includegraphics[width=2.2in, height = 1.32in]{picture3}
  \caption{Average recognition error rate of MyScript and our augmented MyScript system for three test code samples}
  \label{result}
\end{figure*}



\subsection{Statement Classification}
As we mentioned before, we process the handwritten source code data considering each statement as a unit. According to the \textit{Python} grammar specification, we can restrict \textit{Python} source code statements into a limited number of classes, each of which has specified structure rules\footnote{\url{https://docs.python.org/2/reference/grammar.html}}. Here we use the first token in the statement as the symbol for classification. For example, a `def' statement starts with `def' and its structure is defined as `def' + `function name' + `(parameters0, parameters1 ...):'. We define 14 classes for \textit{Python} code statements, including an `assignment' statement, which means the first word in this statement is not a keyword but rather a variable name. In \autoref{fig:overview}, the recognition result is classified as an `if' statement. \autoref{table:statement} presents statistics for the various statement classes in the three code samples.



\begin{table}
  \centering
  \begin{tabular}{l r r r}
    % \toprule
    {Class}
    & {Frequency}
      & {Class}
    & {Frequency} \\
    \midrule
    def & 3 & except & 1  \\
    if & 7 &  while & 1 \\
    for & 3  & try & 1 \\
    raise & 2 & break & 1 \\
    return & 2  & else & 1 \\
    yield & 2 & assignment & 13 \\
    % \bottomrule
  \end{tabular}
  \caption{Frequency for each statement class in three test code samples}~\label{table:statement}
\end{table}



\subsection{Statement Parsing}
After classifying the statement, we need to break it down into independent parts according to the grammar rules.
Similar to a recursive-descent parser \cite{van1993recursive}, our system consists of a series of functions, each of which is responsible for one class of statement. Each function includes a set of mutually recursive procedures where each such procedure implements one of the productions of the grammar as a regular expression. We implement a top-down LL parser to parse the input from left to right and perform a leftmost derivation \cite{fernau1998regulated} of the statement. As a result, a statement is parsed into a list of single tokens and/or characters. For example, the statement in \autoref{fig:overview} is parsed into five individual tokens. Specifically, `if' is a keyword token; `Cookie. name' is a variable token; `==' is a symbol token; `naue' is a variable token; `;' is the last symbol token.



\subsection{Token Processing}
The previous stage results in a list of single tokens and/or characters that make up the statement.
%In this step, we process the list of the single words or characters received from the last step. 
We assume all non-keywords are properly recognized and add them to the lexicon assuming they are \emph{variable} names.
%We build a non-keyword lexicon to save all non-keywords in the first sentence. 
Then for all non-keywords in each statement that follows, we first compare the token to all the words in the non-keyword lexicon. If a `similar' token already exists in the lexicon, we replace it with the `similar' token in the lexicon. For example, in \autoref{fig:overview}, `naue' is very similar to `name', which is already in the lexicon, so we just replace the token `naue' with `name'. If there is no `similar' token in the lexicon, we accept this token as it is and add it to the lexicon. We calculate similarity using the Levenshtein distance \cite{levenshtein1966binary} with a threshold of 0.7, determined empirically.

\subsection{Statement concatenation}
After processing all tokens, we remove all extra spaces in any single token, then concatenate each token with a single space between them to reconstruct the final statement.
Additionally, we ensure that the last recognized character of a statement is a ':'.
For example, in \autoref{fig:overview}, we first remove the space in `cookie. name' and then replace the last character `;' with `:'.





\section{Evaluation}


%define WER and CER
To assess the performance of our system, we measure the Character Error Rate (CER) and Word Error Rate (WER). WER and CER are percentages obtained from the Levenshtein distance between the recognized sequence and the corresponding ground truth. They are calculated as
\[ \frac{D+I+S}{L} \times 100\% \]
where D is the number of deleted units, I is the number of inserted
units, S is the number of substituted units, and L is the total number of
units in the ground truth transcriptions. A unit is a word for WER or a
character for CER.


We evaluate our recognition approach by applying our framework to the 45 code samples in our database. In the following section, we compare the results of our enhanced recognizer to the results of using MyScript alone.

\section{Results} \label{results}

\begin{table}
  \centering
  \begin{tabular}{l r r}
    % \toprule
    {}
    & {t-test score ($t_{14}$)}
      & {P-value} \\
    \midrule
    WER on sample 1 & -9.02 & $P < 0.00001$ \\
    WER on sample 2 & -8.29 & $P < 0.00001$ \\
    WER on sample 3 & -6.57 &  $P < 0.00001$  \\
    CER on sample 1 & -3.88  & $P < 0.001$  \\
    CER on sample 2 & -5.45 & $P < 0.0001$ \\
    CER on sample 3 & -6.13 & $P < 0.0001$ \\
    % \bottomrule
  \end{tabular}
  \caption{Statistical evidence (T-test and P-value) for WER and CER on three code samples}~\label{table:significance}
\end{table}


%result
As shown in \autoref{result}, 
our augmented recognition approach results in an 8.6\% word error rate and 3.6\% character error rate, on average, over the three code samples, which outperforms the original MyScript recognizer with 31.31\%  and 9.24\% in word and character error rate respectively. We also find statistical evidence for an effect of our augmented recognition approach on both WER and CER (See \autoref{table:significance}). 

%on WER on sample \#1 ($t_{14} = -9.02, P < 0.00001$), on CER on sample #1 ($t_{14} = -3.88, P < 0.001$), WER on sample 2 ($t_{14} = -8.29, P < 0.00001$), CER on sample 2 ($t_{14} = -5.45, P < 0.0001$), WER on sample 3 ($t_{14} = -6.57, P < 0.00001$), and CER on sample3 ($t_{14} = -6.13, P < 0.0001$).


%compare to English handwriting recognition
%%Hi Ron, these four systems has lowest WER and CER, the 97% and 95% I said before is not English recognition, it's for arabic. 
\begin{table}
  \centering
  \begin{tabular}{l r r}
    % \toprule
    {System}
    & {WER(\%)}
      & {CER(\%)} \\
    \midrule
    \textcolor{red}{Augmented MyScript}  & \textcolor{red}{8.6} & \textcolor{red}{3.6} \\
    Kozielski et al. \cite{doetsch2013improvements} & 9.5 & 2.7 \\
    Keysers et al. \cite{keysers2016multi} & 10.4 &  4.3  \\
    Zamora et al. \cite{zamora2014neural} & 16.1  & 7.6  \\
    Poznanski et al. \cite{poznanski2016cnn} & 6.45 & 3.44 \\
    % \bottomrule
  \end{tabular}
  \caption{Performance of our system compared to handwritten English recognition systems on the IAM dataset}~\label{table:iam}
\end{table}

%To examine how our source code recognition rates compare
%to acceptable recognition rates in general handwriting recognizers,
%we present the recognition rate comparison between our augmented
%MyScript recognition system (on source code) to four other state-ofthe-art
%general handwritten English recognition systems

Since there is no existing handwriting source code recognizer for comparison, we compare the recognition rate of our our augmented MyScript recognition system (on source code) to that of four state-of-the-art general handwritten English recognition systems (on general text). 
%Since there is no existing handwriting source code recognizer for comparison, we compare our recognition rates on source code to acceptable recognition rates of general handwriting recognizers on general text.  Specifically, we compare the recognition rate of our our augmented MyScript recognition system (on source code) to that of four state-of-the-art general handwritten English recognition systems on general text. 
The IAM handwriting database \cite{marti2002iam} consists of 9,285 lines of general handwritten text
written by approximately 400 writers with no restrictions on style or writing tool. This database has been widely used to evaluate English handwriting recognition systems. The four systems in \autoref{table:iam} were tested based on this IAM handwriting database. \autoref{table:significance} shows that the WER and CER of our augmented source code recognition system are comparable with other state-of-the-art handwritten English recognition systems on general handwritten text.

%Finally, we acknowledge the importance of indentation in \textit{Python} source code.  We chose to ignore indentation for the purposes of this project to focus solely on recognition and would argue that recognition results will not be affected by indentation considerations.  We are in the process of introducing indentation into the collected data to create an identical set of source code samples with appropriate indentation.

%Because typing on a virtual keyboard is the standard input method on touchscreen devices, it's useful to examine how virtual keyboard typing error rates compare to those of handwritten source code recognition.
% %Regular typing is also compared with handwriting source code on a virtual keyboard. 
% For example, Almusaly et al. report a 7.81\% total error rate (TER) for typing java programs on a standard virtual keyboard as measured from 32 participants \cite{almusaly2015syntax}. TER, similar to CER, is a measure of the total number of errors (i.e., omissions, substitutions, and insertions) and corrections that are made in the resulting typed text. 


%explain how our system fix errors in characterization and existing errors

\section{Discussion}


%The improvement of recognition rate embodied in fixing all the three error types. 


Our approach achieved an 8.6\% word error rate and a 3.6\% character error rate on the collected dataset by taking the language grammar rules into account. Overall, improvement of our recognition pipeline over the baseline MyScript recognition engine can be attributed to addressing the three main error types identified in Section \ref{sec:characterizing}.  After statement concatenation, all unnecessary space errors in a single token are removed. Ensuring the last character of a statement eliminates 32\% of the symbol errors. Token processing fixes around 78\% of the word errors. 

Recognition results, however, are still not 100\% accurate. Initial inspection indicates that this is mainly due to the illegible or cursive handwriting of the participants and the incorrect recognition of symbols. Also, since one of our lexicons is dependent on the non-keywords already recognized in the code, incorrectly recognized words will also be added to the lexicon, thereby corrupting the lexicon and preventing it from enhancing the recognition of the following words. Additionally, it is difficult to identify incorrectly recognized symbols; for example, if `(' appearing in the middle of the text is recognized as `l', it becomes impossible to rectify it using our approach. 
%errors still existing, develop widget for errors, what will improve
%Our approach achieved an 8.6\% word error rate and a 3.6\% character error rate on the collected dataset by taking the language grammar rules into account. 
%It should be noted that there are still identified errors remaining to be corrected. 
Errors like unmatched `(' and `)' in a statement can be detected, but not reliably corrected. For example, `(name' can be recognized as `cname', but we have no evidence to correct `cname' to `(name'. Two methods can be employed to resolve remaining errors such as this. The first is to develop a widget in the handwriting interface to highlight all errors that are identified but can't be corrected and let users correct them manually. Another option is to train a language model to identify words that do not exist \cite{zamora2014neural}. 


Because typing on a virtual keyboard is the standard input method on touchscreen devices, it is useful to examine how virtual keyboard typing error rates compare to those of handwritten source code recognition.
% %Regular typing is also compared with handwriting source code on a virtual keyboard. 
Almusaly et al. report a 7.81\% total error rate (TER) for typing \textit{Java} programs on a standard virtual keyboard as measured from 32 participants \cite{almusaly2015syntax}. TER, similar to CER, is a measure of the total number of errors (i.e., omissions, substitutions, and insertions) and corrections that are made in the resulting typed text.  Our handwriting results are comparable.

%generalize to other languages
This approach can also be generalized to other programming languages with strict grammar rules. For instance, one can define statement classes for \textit{Java} according to the first word in the statement and then replace the regular expressions with productions of \textit{Java} grammar rules.  Algorithms for searching and replacing similar words can be kept unchanged. Other heuristic steps like concatenating tokens are also trivial to implement for new languages. 
%Due to the uncertainty of the program structure, however, our approach cannot easily be generalized to non-strict programming languages.

%\section{Discussion and Implication for HCI}
\section{Conclusion and Future Work}
%%mobile programming is widely used, and its limitation.
%Supporting programming on mobile devices is not a novel idea. With the rapid technology shift in current computing devices, high-quality low-cost mobile devices such as tablets and smartphones are being increasingly used in everyday activities. Many tasks that previously required a PC are now feasible on mobile devices. For example, tablets are typically equipped with powerful batteries, advanced graphic processors, high-resolution screens and fast processors, making writing and compiling code on them completely plausible. TouchDevelop \cite{tillmann2011touchdevelop}, for example, is a novel programming environment, language and code editor for mobile devices and Tilman et al. \cite{tillmann2012future} predict that programming on mobile devices will be widely used for teaching programming.  However, mobile devices are also inherently restricted by their limintations such as small screens and the clumsy virtual keyboard that is difficult to use and that takes up valuable screen space - posing challenges for a programmer who wishes to enter or edit source code.
%For this to happen, however, input must be made more intuitive.
%Alternatives to typing have been considered for a long time. One is to use speech (via dictation) to create the text of a program. Desilets et al. \cite{desilets2006voicecode} propose VoiceCode, which translates the pronounced syntax into native syntax in the current programming language to support programming by speech. Gordon \cite{gordon2013improving} employs language design and incorporates dynamic context for this purpose. Speech, however, is not always an appropriate option given social conventions and privacy issues. Given the advances of pen-based input technology, we chose to explore handwriting input in this paper.  

 
%conclusion
The keyboard is not an ideal input mechanism for every person and situation.
Alternatives to typing, such as speech, have been considered in the past \cite{desilets2006voicecode, gordon2013improving}. However, speech is not always an appropriate option given social conventions and privacy issues.
Given advances in pen-based technology that provides an opportunity for users to engage with devices in a potentially more `natural' way than that supported by a virtual keyboard, handwriting input is a viable alternative to virtual keyboard input. In this paper, we have explored handwriting recognition specifically for source code with the ultimate goal of supporting handwriting as a means for programming.
%Handwriting is a viable alternative to a keyboard given recent technological advances in pen-based technology.
%for programming can be used for people with disabilities and people suffering RSI. %In this paper, we focus on supporting handwriting recognition for the particular domain of source code text input. 
We collect and present a small database of publicly available handwritten source code samples and we propose an approach to recognize handwritten source code by leveraging a commercial handwriting recognition system. Experiments on the data collected from 15 participants show our framework has an average 8.6\% word error rate and 3.6\% character error rate which outperforms the baseline recognition system and produces rates comparable to the recognition of general handwritten English text.  We are encouraged by these initial results but believe there are several avenues of future work.


%Given the advances of pen-based input technology, we chose to explore handwriting input in this paper. %%writing is new viable, stylus technique
%Input via stylus is becoming more precise and the familiar writing action makes handwriting a viable alternative to typing code on mobile devices. 
%Given advances in pen-based technology that provides an opportunity for users to engage with devices in a potentially more `natural' way than that supported by a virtual keyboard, we have explored handwriting recognition for source code with the ultimate goal of supporting handwriting as a means for programming.
%Additionally, research suggests that handwriting can lead to cognitive, memory, and creativity enhancements \cite{alonso2015metacognition}. Alternative input mechanisms are also important for people with RSI or other motor impairments who find typing difficult, and in general, for those who may wish to carry out simple source code editing and entry tasks in mobile situations. 
%While handwriting without recognition produces `digital ink' that is appropriate for applications like annotation and graphic design, 
%For handwriting to be used in scenarios like programming, however, applications must be equipped with recognition technology to support translation to searchable and editable digital text. To effectively incorporate the handwriting experience for source code entry and editing, we must first address the source-code recognition problem.




%importance


  

% Summarize what we've done
%We have presented an initial study to collect data on handwritten source code and explored the use of a state-of-the-art recognition system for recognizing handwritten source code. 
%In this section, we talk about why it is important to HCI and how HCI community could benefit from our work.


%%writing is new viable, stylus technique
%Input via stylus is becoming more precise and the writing action is very similar to writing on paper. Handwriting is therefore a viable alternative to typing code on mobile devices. Moreover, handwriting is an acceptable input method for people with RSI or other motor impairments who find typing difficult. 
%Thus studying handwriting input method will be a benefit for a large group of users. 
%While handwriting without recognition produces `digital ink' that is appropriate for applications like annotation and graphic design, for handwriting to be used in scenarios like programming, it must be equipped with recognition technology to support translation to searchable and editable digital text.

%%implication and importance for HCI
From the view-point of human-computer interaction, usability and user satisfaction is critical. For handwriting text input, users expect recognition technology with a low error rate and responsive recognition speed. LaLomia et al. \cite{lalomia1994user} reported that users are willing to accept a recognition error rate of only 3\% (a 97\% recognition rate), although Frankish et al. \cite{frankish1995recognition} concluded that users will accept higher error rates depending on the text-editing task. It would not be surprising, therefore, if higher error rates were acceptable for source code entry and editing which is inherently difficult due primarily to the use of symbols. Input speed is another concern with respect to handwriting. Modest touch typing speeds on a virtual keyboard in the range of 20 to 40 words per minute (wpm) are achievable.
Handwriting speeds are commonly in the 15 to 25 wpm range \cite{card1983psychology,devoe1967alternatives,dunlop2009pickup}. We suspect that this decrease in speed, however, will be acceptable to the particular groups for whom handwriting is the most viable input option. Additionally, in professional programming, most of the code that developers
write involves reuse of existing example code and libraries \cite{bellon2007comparison}. This `reuse' typically amounts to editing existing code to suit a
new context or problem and generally provides benefits to developers in terms of time and error reduction \cite{ko2011state}.
%is able to save time and
%avoid the risk of writing erroneous new code \cite{ko2011state}.
For these reasons, we envision our system as being particularly useful in the code editing domain as opposed to writing extensive source code from scratch.  Studying how the algorithms perform in editing tasks is left as future work.

%a recognizer used primarily for code edits as opposed to being used to write an entire program from scratch.


% How does the HCI community benefit from this work
%% Dataset for testing

While databases exist for research in general handwritten text recognition \cite{marti2002iam, grosicki1rimes}, there is no such dataset for handwritten source code.  This paper represents the first such contribution of a handwritten source code dataset consisting of 555 lines of  \textit{Python} code written by 15 participants. While we recognize that using the same three code samples for all users and employing a ``copy task'' may lessen the generality of the dataset, we sought to eliminate all effects of cognitive complexity (e.g. actually solving programming problems) to focus solely on the handwritten source code quality.  Collecting data for other programming languages and for actual programming tasks is left as future work.

%A standard database is needed to facilitate research in handwritten source code recognition. For general handwritten text recognition, the IAM database \cite{marti2002iam} and the RIMES database \cite{grosicki1rimes} are widely used for research purposes. Based on Lancaster-Oslo/Bergen (LOB) corpus, the IAM database \cite{marti2002iam} consists of 9,285 lines of handwriting text from 400 writers. The RIMES Database comes from the ICDAR 2011 block-recognition competition and consists of 1,500 paragraphs of the handwritten French text.
%Unfortunately, there is no such dataset for handwritten source code - this paper represents the first such contribution.  We collected only a small handwritten source code dataset consisting of 555 lines written by 15 participants. While we recognize that using the same three code samples for all users and employing a ``copy task'' may lessen the generality of the dataset, we sought to eliminate all effects of cognitive complexity (e.g. actually solving programming problems) to focus solely on the handwritten source code quality.  Collecting data from actual programming tasks and for additional programming languages is left as future work.
%and affect the writing organization.   but cognitive load and possible errors are avoided. In addition, t
%It is expected that the database would be particularly useful for further handwritten source code recognition research using \textit{Python} as the language of choice.  More data on additional languages will be necessary to further investigate handwriting as a viable input mechanism for source code.



%% Approach based on state-of-the-art recognizer
%Our handwritten source code recognition framework is implemented by leveraging the programming language grammar information to augment an existing handwriting recognition system. By replacing the grammar rules in the framework, the system can be generalized to other programming languages. It should be noted, however, that using an existing handwriting recognition system designed for natural language is not tackling the problem at its source. Rebuilding the core of a recognition system based on properties of source code is thus an alternative approach that should be explored.  We leave this for future work.






%conclusion
%The keyboard is not an ideal input mechanism for everyone.
%Handwriting as an alternative to a keyboard for programming can be used for people with disabilities and people suffering RSI. In this paper, we focus on supporting the recognition aspect of handwriting for source code text input. We collect and present a small database of publicly available handwritten source code samples and we propose an approach to recognize handwritten source code by leveraging a commercial handwriting recognition system. Experiments on the data collected from 15 participants shows our framework has an average 8.6\% word error rate and 3.6\% character error rate which outperforms the baseline recognition system and produces rates comparable to recognition of general handwritten English text.

%future work

%%from scratch
%%IDE
%Clearly, the current work is limited in both scope and depth and 
%We are encouraged by these initial results but believe there are several avenue of future work. 

The next most obvious area of future work is to develop a handwritten source code recognition system from scratch instead of augmenting the results produced by an existing system.  We suspect this approach would lead to comparable and most likely improved recognition rates. Building a universal handwritten source code reading system could employ deep learning techniques such as Concurrent Neural Networks \cite{poznanski2016cnn} or neural network language models \cite{zamora2014neural} trained purely on the source code. 

Additionally, there are several opportunities to explore the integration of handwriting recognition into source code IDEs \cite{frye2008pdp}.  For example, how do we now integrate source code completion into a handwriting-based interaction?   Can we integrate elements such as syntax insertion and highlighting?  Exploring the affordances of handwriting in the context of an IDE is an exciting area of future work that is enabled by these initial findings.

%Similar to an IDE for handwriting C\# code \cite{frye2008pdp}, integrating handwriting source code to current programming IDE or building a programming IDE solely with handwriting as an input method is also valuable to explore.

%%speed, auto completion
%Furthermore, to address concerns pertain to handwriting speed, it may be possible to add auto-completion feature into handwriting source code interface. Next word or character suggestion is also helpful to facilitate inputting. In addition, combing voice input and handwriting input may also improve the inputting speed.

Multimodal methods present another area of future work.  Perhaps the combination of handwriting and speech input or handwriting and occasional keyboard input \cite{mueller2014pen} begin to produce interaction experiences that rival those of typed source code input.  

Finally, we will never reach a perfect recognition rate for handwritten text (general or source code).  How do we effectively support efficient editing of the recognized text so that users can quickly correct mistakes? Natural and effective text entry and editing is an interesting topic for future studies.

%%texue of future work.t editing
%, our research projeng avenct aims to develop a programming interface to support handwriting, editing and recognizing source code. We also intend to explore the potential of adding editing techniques such as selecting, deleting, copying, and pasting.






%% if specified like this the section will be committed in review mode
\acknowledgments{
The authors wish to thank all the study participants as well as Poorna Talkad Sukumar, Jason Liu, and Suwen Lin for their valuable discussions and input.}

%\bibliographystyle{abbrv}
\bibliographystyle{abbrv-doi}
%\bibliographystyle{abbrv-doi-narrow}
%\bibliographystyle{abbrv-doi-hyperref}
%\bibliographystyle{abbrv-doi-hyperref-narrow}

\bibliography{template}
\end{document}

% Or table: % $Id: template.tex 11 2007-04-03 22:25:53Z jpeltier $
\documentclass{vgtc}                          % final (conference style)
%\documentclass[review]{vgtc}                 % review
%\documentclass[widereview]{vgtc}             % wide-spaced review
%\documentclass[preprint]{vgtc}               % preprint
%\documentclass[electronic]{vgtc}             % electronic version

%% Uncomment one of the lines above depending on where your paper is
%% in the conference process. ``review'' and ``widereview'' are for review
%% submission, ``preprint'' is for pre-publication, and the final version
%% doesn't use a specific qualifier. Further, ``electronic'' includes
%% hyperreferences for more convenient online viewing.

%% Please use one of the ``review'' options in combination with the
%% assigned online id (see below) ONLY if your paper uses a double blind
%% review process. Some conferences, like IEEE Vis and InfoVis, have NOT
%% in the past.

%% Figures should be in CMYK or Grey scale format, otherwise, colour 
%% shifting may occur during the printing process.

%% These few lines make a distinction between latex and pdflatex calls and they
%% bring in essential packages for graphics and font handling.
%% Note that due to the \DeclareGraphicsExtensions{} call it is no longer necessary
%% to provide the the path and extension of a graphics file:
%% \includegraphics{diamondrule} is completely sufficient.
%%
\ifpdf%                                % if we use pdflatex
  \pdfoutput=1\relax                   % create PDFs from pdfLaTeX
  \pdfcompresslevel=9                  % PDF Compression
  \pdfoptionpdfminorversion=7          % create PDF 1.7
  \ExecuteOptions{pdftex}
  \usepackage{graphicx}                % allow us to embed graphics files
  \DeclareGraphicsExtensions{.pdf,.png,.jpg,.jpeg} % for pdflatex we expect .pdf, .png, or .jpg files
\else%                                 % else we use pure latex
  \ExecuteOptions{dvips}
  \usepackage{graphicx}                % allow us to embed graphics files
  \DeclareGraphicsExtensions{.eps}     % for pure latex we expect eps files
\fi%

%% it is recomended to use ``\autoref{sec:bla}'' instead of ``Fig.~\ref{sec:bla}''
\graphicspath{{figures/}{pictures/}{images/}{./}} % where to search for the images
\usepackage{microtype}                 % use micro-typography (slightly more compact, better to read)
\PassOptionsToPackage{warn}{textcomp}  % to address font issues with \textrightarrow
\usepackage{textcomp}                  % use better special symbols
\usepackage{mathptmx}                  % use matching math font
\usepackage{times}                     % we use Times as the main font
\renewcommand*\ttdefault{txtt}         % a nicer typewriter font
\usepackage{cite}                      % needed to automatically sort the references
\usepackage{tabu}                      % only used for the table example
\usepackage{booktabs}   % only used for the table example
\usepackage{caption} 
\captionsetup[table]{skip=10pt}
\usepackage{color}


%%% 
%%% Meta notes
%%%
\newlength{\dummylen}
\newcommand{\NOTE}[1]{\setlength{\dummylen}{\fboxrule}\setlength{\fboxrule}{2pt}%
            \vspace{1ex}\noindent\hfill%
            \fbox{\begin{minipage}{.96\columnwidth}#1\end{minipage}}%
            \setlength{\fboxrule}{\dummylen}\hfill{}\vspace{1ex}}

%\renewcommand{\NOTE}[1]{\ignorespaces}



%% We encourage the use of mathptmx for consistent usage of times font
%% throughout the proceedings. However, if you encounter conflicts
%% with other math-related packages, you may want to disable it.


%% If you are submitting a paper to a conference for review with a double
%% blind reviewing process, please replace the value ``0'' below with your
%% OnlineID. Otherwise, you may safely leave it at ``0''.

\onlineid{0}

%% declare the category of your paper, only shown in review mode
\vgtccategory{Research}

%% allow for this line if you want the electronic option to work properly
\vgtcinsertpkg

%% In preprint mode you may define your own headline.
%\preprinttext{To appear in an IEEE VGTC sponsored conference.}

%% Paper title.

\title{Recognizing Handwritten Source Code}
%\title{Writing Source Code (literally!)
%% This is how authors are specified in the conference style

%% Author and Affiliation (single author).
%%\author{Roy G. Biv\thanks{e-mail: roy.g.biv@aol.com}}
%%\affiliation{\scriptsize Allied Widgets Research}

%Author and Affiliation (multiple authors with single affiliations).
\author{Qiyu Zhi\thanks{e-mail: qzhi@nd.edu} %
\and Ronald Metoyer\thanks{e-mail:rmetoyer@nd.edu}}
\affiliation{\scriptsize University of Notre Dame}

%% Author and Affiliation (multiple authors with multiple affiliations)
% \author{Roy G. Biv\thanks{e-mail: roy.g.biv@aol.com}\\ %
%         \scriptsize Starbucks Research %
% \and Ed Grimley\thanks{e-mail:ed.grimley@aol.com}\\ %
%      \scriptsize Grimley Widgets, Inc. %
% \and Martha Stewart\thanks{e-mail:martha.stewart@marthastewart.com}\\ %
%      \parbox{1.4in}{\scriptsize \centering Martha Stewart Enterprises \\ Microsoft Research}}

%% A teaser figure can be included as follows, but is not recommended since
%% the space is now taken up by a full width abstract.
% \teaser{
%  \includegraphics[width=1.5in]{sample.eps}
%  \caption{Lookit! Lookit!}
% }

%% Abstract section.
\abstract{
%Programming requires textual input and typically requires typing which is not always the %optimal input method for all users - especialy on touchscreen devices. 
Supporting programming on touchscreen devices requires effective text input and editing methods.  Unfortunately, the virtual keyboard can be inefficient and uses valuable screen space on already small devices.  Recent advances in stylus input make handwriting a potentially viable text input solution for programming on touchscreen devices.  
The primary barrier, however, is that handwriting
recognition systems are built to take advantage of the rules of natural language, not those of a programming language. In this paper, we explore this particular problem of handwriting recognition for source code.  
We collect and make publicly available a dataset of handwritten \textit{Python} code samples from 15 participants and we characterize the typical recognition errors for this handwritten \textit{Python} source code when using a state-of-the-art handwriting recognition tool.  We present an approach to improve the recognition accuracy by augmenting a handwriting recognizer with the programming language grammar rules. Our experiment on the collected dataset shows an 8.6\% word error rate and a 3.6\% character error rate which outperforms standard handwriting recognition systems and compares favorably to typing source code on virtual keyboards.
} % end of abstract

%% ACM Computing Classification System (CCS). 
%% See <http://www.acm.org/class/1998/> for details.
%% The ``\CCScat'' command takes four arguments.


\keywords{programming, handwriting recognition, touch screen, source code, python.}
\CCScatlist{ 
  \CCScat{H.1.2.}{User/Machine Systems}{Human information processing}{};
  \CCScat{H.5.2.}{User Interfaces}{Input devices and strategies (e.g., mouse, touchscreen)}{}
}


%% Copyright space is enabled by default as required by guidelines.
%% It is disabled by the 'review' option or via the following command:
% \nocopyrightspace

%%%%%%%%%%%%%%%%%%%%%%%%%%%%%%%%%%%%%%%%%%%%%%%%%%%%%%%%%%%%%%%%
%%%%%%%%%%%%%%%%%%%%%% START OF THE PAPER %%%%%%%%%%%%%%%%%%%%%%
%%%%%%%%%%%%%%%%%%%%%%%%%%%%%%%%%%%%%%%%%%%%%%%%%%%%%%%%%%%%%%%%%

\begin{document}

%% The ``\maketitle'' command must be the first command after the
%% ``\begin{document}'' command. It prepares and prints the title block.

%% the only exception to this rule is the \firstsection command
%\firstsection{Introduction}

\maketitle

\section{Introduction} %for journal use above \firstsection{..} instead
%Supporting programming on mobile devices is not a novel idea. 
With the rapid technology shift in current computing devices, high-quality low-cost mobile devices such as tablets and smartphones are being increasingly used in everyday activities. Many tasks that previously required a PC are now feasible on mobile devices. For example, tablets are typically equipped with powerful batteries, advanced graphic processors, high-resolution screens and fast processors, making writing and compiling code on them completely plausible. TouchDevelop, for example, is a novel programming environment, language and code editor for mobile devices  \cite{tillmann2011touchdevelop}. Furthermore, Tillmann et al. predict that programming on mobile devices will be widely used for teaching programming \cite{tillmann2012future}.  However, mobile devices are also inherently restricted by their limitations such as small screens and the clumsy virtual keyboard.  Entering and editing large amounts of text for programming tasks can quickly become difficult and time consuming with these virtual keyboards because they are notoriously difficult to use when compared to a physical keyboard and they consume valuable screen space\cite{raab2013refactorpad}.

%that is difficult to use and that takes up valuable screen space - posing challenges for a programmer who wishes to enter or edit source code.


% comparing handwriting and typing in different scenarios to motivate the work. 
While keyboards have been the primary input device for entering computer programs since the computer was invented\cite{gordon2013improving}, this predominant mechanism is not ideal for all programming situations. For example, software developers that suffer from repetitive strain injuries (RSI) and related disabilities may find typing on a keyboard difficult or impossible \cite{begelprogramming}.
%people suffering from repetitive strain injuries (RSI), carpal tunnel syndrome, or other motor impairments may experience tremendous difficulty using a keyboard and mouse \cite{arnold2000programming}.
Instead, handwriting with a stylus may be a preferred input mechanism for some of these users \cite{mankoff1998cirrin}.
%In addition, entering and editing large amounts of text for programming tasks can quickly become difficult and time consuming with
%\emph{virtual} keyboards, for any user, because they are notoriously difficult to use when compared to a physical keyboard and they consume valuable screen space\cite{raab2013refactorpad}.
%People with disabilities  (e.g. limb loss) or who do not have full mobility may also find typing on a keyboard  (physical or virtual) is difficult or impossible \cite{begelprogramming}. 
In addition, some physical configurations (e.g. seated on a plane) may simply be more suited to the writing posture than a typing posture for many users.

Handwriting has also been shown to have potential cognitive benefits\cite{alonso2015metacognition}.  In particular, Mueller and Oppenheimer found that students who took longhand notes performed better on conceptual questions than those that typed notes on a laptop \cite{mueller2014pen}.  Given these findings, and that fact that many programmers write pseudocode by hand before typing, it is reasonable to consider that handwriting may provide cognitive benefits for programming, especially on mobile devices. 
Furthermore, recent advances in pen-based input and handwriting recognition technology are quickly making handwriting a viable alternative to typing.
%Entering and editing large amounts of text for
%programming tasks can quickly become difficult and %time consuming
%with \emph{virtual} keyboards that are notoriously %difficult to use when compared to a physical %keyboard and consume valuable screen %space\cite{raab2013refactorpad}. 
%It is, therefore, important that we explore handwriting as an input option to support these users and their needs.
%keyboards are not conveniently available on many of today's touchscreen devices or come in the form of \emph{virtual} keyboards that are notoriously difficult to use and require the use of valuable screen space \cite{}.  


% One alternative to typing is to use speech (via dictation) to create the text of a program. Desilets et al. \cite{desilets2006voicecode} propose VoiceCode, which translates the pronounced syntax into native syntax in the current programming language to support programming by speech. Gordon \cite{gordon2013improving} employs language design and incorporates dynamic context for this purpose. Speech, however, is not always an appropriate option given social conventions and privacy issues.
% %While many alternatives have been explored, few have thoroughly examined the use of handwriting as a means for textual input, especially when considering source code. 
% Given the recent advances in pen-based input, however, handwriting is a potentially viable alternative to typing and speech.
%other input forms, such as speech, which are not always appropriate.

%Given recent advances in pen-based input, handwriting code is becoming a potentially viable alternative to typing.


In this paper, we explore the use of handwriting as a means for source code text input.  There are two ways to approach this problem.  One alternative is to develop or modify a handwriting recognition engine to take source code directly into account.  Given that source code often includes English language words, another alternative is to leverage the capabilities of an existing English language handwriting recognition engine. We explore this latter option.  First, we collect and present a publicly available dataset of handwritten \textit{Python} source code for use in handwriting recognition research.  Second, we explore the use of the state-of-the-art recognition system, MyScript \cite{myscript} for recognizing \textit{Python} source code.  We characterize the errors made by the MyScript engine and present a method for post-processing the engine's results to improve recognition performance on handwritten \textit{Python} source code.   

After presenting related work and necessary background information, we describe our data collection process and the resulting publicly available dataset. We then describe the performance of MyScript on recognizing the handwritten source code and present our algorithm for leveraging the MyScript engine to produce improved results.  We discuss those results in Section \ref{results} and conclude with avenues of future work.


%leveraging programming language grammar and lexicon information into a current handwriting recognition system, MyScript \cite{myscript}. Our primary contributions include a handwritten source code dataset, a characterization of errors made when using a general handwriting recognition system on the source code, and an algorithm to leverage general handwriting recognition systems to produce improved results for handwritten source code.


% overview of our work

%Another possible situation is when keyboards are not available, which is becoming a common scenario because touchscreen devices such as the advanced iPad and Android-based tablet is sufficient for daily use.

\section{Background and Related Work}
% Handwriting source code
% For each paper, what did they do  (few sentences), and what did they find, how are we going to be different.

% Handwriting recognition


Many alternatives to typed source code have been considered, typically in the context of making programming more accessible.   Most of those appoaches fall in the realm of speech-based programming \cite{desilets2006voicecode, gordon2013improving}, however speech is not always an acceptable solution, especially in quiet environments or with applications that require privacy.
%One approach is to use human speech (via dictation) to create the text of a program. Desilets et al. \cite{desilets2006voicecode} propose VoiceCode, which translates the pronounced syntax into native syntax in the current programming language to support programming by speech. Gordon \cite{gordon2013improving} employs language design and incorporates
%dynamic context for this purpose.   
Given the recent advances in pen-based input, handwriting is a potentially viable alternative.  The pendragon supports people who are unable to use
a keyboard and seeks to find new interaction techniques for input which may improve communication speed \cite{Pendragon}. Mankoff et al.also suggest that word prediction, sentence completion and the syntax of programming languages could be used for handwriting source code \cite{mankoff1998cirrin}. 
Most closely related to our work is a programming IDE integrated with a handwriting area in which the handwritten code is recognized by an enhanced handwriting recognition system \cite{frye2008pdp}.  This work, however, does not present an evaluation of the recognition engine or guidance for how to improve general handwriting recognition engines for application to source code recognition.
%However, they failed to present a clear evaluation for the handwritten source code recognition system.
In this paper, we focus on the recognition step of handwritten source code as the most important step for developing an effective handwriting interface for source code input and editing.

Research in handwriting recognition has a long history dating back to the 1960s~\cite{tappert1990state}. Hidden Markov Model (HMM) based handwriting recognition~\cite{hu1996hmm,kundu1988recognition,nag1986script} is one of the 
most widely used approaches while neural networks are gaining in popularity~\cite{jaeger2001online}. Some approaches also leverage additional constraints for recognizing handwriting in specific domains such as postal addresses \cite{srihari1993recognition, srihari1993interpretation} and banking checks \cite{gorski1999a2ia, agarwal1997bank}. These handwriting recognition systems are developed to take advantage of the English language \cite{van2003using}, which is intrinsically different from source code. For instance, variable names are often created from
concatenated words (e.g. camelCase or underscore naming), which poses a problem for the traditional handwriting
recognition system as it expects spaces to appear between words
contained within its dictionary.   We do not aim to contribute to the extensive literature in handwriting recognition, but rather, we intend to examine how we can leverage this existing work for application to handwritten source code recognition interfaces.  




\section{Data Collection}

Our first contribution in this paper is a data collection study designed to generate a sample set of handwritten source code for research purposes.  The first question to consider is what programming language to study.  We decided to collect handwritten \textit{Python} source code because of the current popularity of \textit{Python}\footnote{\url{http://www.tiobe.com/tiobe-index/}} and its projected growth rate 
%at some high-profile
%organizations, such as Google, Disney and the US
%Government 
\cite{radenski2006python}. 
We chose a ``copying task'', where three code samples are provided for every participant to copy on the tablet using the stylus. While we understand that a ``copying task'' may be cognitively quite different from other writing tasks that require synthesis, we sought to eliminate sources of cognitive load that could impact timing as well as writing quality for the purposes of this data collection task. The three shared code samples allow for comparison across participants. To broaden our dataset of unique handwritten source code samples, we also randomly selected a fourth source code sample function (per participant) that was unique to that participant. In this section, we describe our data collection process and the resulting database of handwritten \textit{Python} samples.


%its current popularity as a programming language and it's growth trajectory.  



%In order to test the recognition system, we collected a dataset of handwritten \textit{Python} code samples written by students from the [anonymized for review] campus. We chose \textit{Python} because of its current popularity as a programming language. 

\subsection{Participants}
We recruited 15 participants (9 females) from the University of Notre Dame for our study. Thirteen of the participants were computer science majors and all participants had at least two semesters of programming experience. Their ages ranged from 19 to 29 (mean = 22.3). Two participants were left-handed. Eight participants had used a pen/stylus for handwriting on a touchscreen device and only one participant had used a tablet for inputting source code (via the virtual keyboard). All participants were compensated \$5 for the study which took approximately 30 minutes each.

\subsection{Apparatus and Software}

\begin{figure}[tb]
 \centering % avoid the use of \begin{center}...\end{center} and use \centering instead (more compact)
 \includegraphics[width=\columnwidth]{datacollection}
 \caption{Screen shot of our data collection web application.  Participants entered their name and code sample number into the boxes in the upper left.  They then entered the code sample using the Apple Pencil and selected `Save' when finished.}
 \label{fig:webpage}
\end{figure}


We used a 12.9 inch iPad Pro with a 2732-by-2048 screen resolution at 264 pixels per inch (PPI) and fingerprint-resistant oleophobic coating. Participants used the Apple Pencil as the stylus device. We implemented a web application with a writing area to record user input. This application was responsible for converting touch points of the stylus into handwriting strokes and saving strokes to a JSON file. Each stroke consists of the coordinates of the sampled points and time-stamp information for each coordinate. The writing area in this application measured 795 * 805 pixels with subtle lines on the background to provide guides for the participants (See \autoref{fig:webpage}). We also implemented functions to undo or redo the previous stroke as well as clear the writing area of all strokes. 

\subsection{Representative Source Code Material}
Our goal was to create a database of representative samples of handwritten \textit{Python} source code for use in evaluating the performance of a handwriting recognition system. Because different \textit{Python} samples contain different language elements, there is no single representative corpus\cite{almusaly2015syntax}. Ideally, representative code samples should contain a variety of language constructs and not be restricted to a single project. 

Our process for choosing source code samples is based on that used by McMillan et al. \cite{mcmillan2012exemplar, rodeghero2014improving}. First, we selected six popular \textit{Python} projects on Github. \autoref{table:project} summarizes the details of these projects. We then extracted all functions from the project source code and eliminated comments in order to focus solely on the source code of the samples.  Next, to obtain functions that were sufficiently long to collect a substantial amount of handwriting, but not so long as to require multiple pages of handwriting, we filtered the functions to those with between 9 and 18 lines of source code and those with no lines greater than 60 characters (to eliminate long, wrapping lines).  We also manually filtered out highly repetitive functions, such as a function that includes only assignment statements for variables. The result was 1324 eligible functions. We randomly selected the three shared test code samples from this set for use by all participants and one additional unique code sample to be entered by each participant in the study. Although \textit{Python} syntax considers whitespace, we decided to ignore indentation for the purposes of focusing purely on handwriting recognition. 
%We are currently in the process of augmenting the collected samples to produce a second identical dataset with indentation inserted.


%We collected this unique code sample for each participant to broaden the coverage of code samples and characterize the common recognition errors from the common recognition system as well.


\subsection{Procedure}

For every participant, we began our data collection with an informed consent process.  Each participant then filled out a pre-study questionnaire about demographics and experience using touchscreen devices and a stylus. Participants were given a practice task to familiarize them with the process.  For each of the four tasks, participants were given a sheet of paper with the sample typed \textit{Python} source code.   Participants entered their name and code sample number into the web application and then entered the code sample using the Apple Pencil and selected `Save' when finished.  After completing all four input tasks, participants were compensated and the session ended.


\subsection{Data Collection Results}
The final dataset includes stroke data for four code samples for each of 15 participants resulting in a total of 60 handwritten source code samples. So, for each of 3 given source code input samples, we have 15 copies of handwritten source code (for a total of 45 handwritten source code samples). The remaining 15 are handwritten samples of unique input source code examples from each participant. 
The handwritten source code data can be downloaded at \newline \url{http://www.purl.org/recognizinghandwrittencode/data}.
 
\begin{table}
  \centering
  \begin{tabular}{l r r r}
    % \toprule
    {Project}
    & {Lines}
      & {Fuctions}
    & {Eligible Functions} \\
    \midrule
    AlphaGo & 1,963 & 151 & 1 \\
    Bittorrent & 7,164 &  570 & 39 \\
    Blender & 265,684  & 12,774 & 1,126 \\
    Instagram & 1,265 & 145 & 8 \\
    Requests & 14,009  & 862 & 84 \\
    Webpy & 10,199 & 1,029 & 66 \\
    % \bottomrule
  \end{tabular}
  \caption{\textit{Python} projects used for selecting code samples}~\label{table:project}
\end{table}



\section{Source Code Recognition Errors}

Current commercial handwriting recognition systems are built to take advantage of the rules of the English language as opposed to that of a programming language, therefore it is not surprising that these systems might perform poorly on source code recognition \cite{frye2008pdp}. There is, however, no previous research that evaluates how well existing state-of-the-art handwriting recognition systems perform on handwritten source code. Here we describe the state-of-the-art handwriting recognition system we employed and characterize the errors based on the dataset we collected.

\subsection{State-of-the-art: Myscript}
Automatic recognition of handwriting is now a mature discipline that has found many commercial uses\cite{plamondon2000online}. MyScript\cite{myscript} is an online handwriting recognition engine that supports more than 80 languages and achieved the best recognition rate in the International Conference on Document Analysis and Recognition competition\cite{el2011line}.
Here we use the MyScript engine as our baseline for comparison and study the typical recognition errors produced when applied to handwritten source code to better understand the complexities introduced by \textit{Python} source code and source code in general.


\subsection{Data Pre-Processing}

In order to use MyScript efficiently and to make a fair comparison between its performance and our algorithm, we apply two simple pre-processing steps to the data. First, we provide MyScript with \textit{Python} specific context through the Subset Knowledge (SK) facility and a custom lexicon. SK is a MyScript feature for telling the recognizer that we only want it to enable recognition of certain characters. For example, for a phone number field, we may want only digits to be recognized.   We created an SK resource in MyScript to allow only characters that can legally appear in \textit{Python} source code.  We also provide the legal \textit{Python} keywords through a user-defined lexicon.

%We used two main MyScript resources to process the data we collected: 
%The MyScript Cloud Development Kit (CDK) and Subset knowledge(SK).
The MyScript Cloud Development Kit (CDK) is an HTTP-based set of services that take handwritten strokes as input and produce potential recognition results as output. To use the CDK in experiments, we must send strokes to the recognizer at some level of granularity (e.g. single character, whole word, whole line, etc).
%The data processing step is to process the data for submission to the MyScript Cloud Development Kit (CDK). Pre-processing includes character filtration and sentence separation. 
%Character filtration is to restrict the recognition result to certain characters. We used the Subset knowledge(SK) feature by creating an SK resource in MyScript to allow only characters that can legally appear in \textit{Python} source code. In sentence separation, 
We chose to break the stroke data into lines assuming developers might write a statement at a time on a single line. To do so, we analyze stroke coordinates and create a new line each time the user moves to a new vertical position. Each line is then sent one at a time to the MyScript CDK.  This simulates a developer writing one programming statement (one line) at a time, pausing at the end of each line.  Alternative pre-processing is possible given the raw stroke data and timestamp information (e.g. sending incomplete lines when a participant pauses).

\subsection{Characterizing errors}
\label{sec:characterizing}
We processed all of the handwritten data as described above to collect baseline recognition results for all handwritten samples in our dataset.  We then set out to understand the types of recognition errors that were present in the final recognized text. We identified three major types of recognition errors: word errors, symbol errors, and space errors. 

Word errors occur when MyScript simply incorrectly recognizes a written word. This is typically due to poor writing and can occur for keywords as well as non-keywords.
For example, when the handwritten word `self' is recognized as `silt', we characterize this as a word error. 
Symbol errors represent incorrect recognition of symbols or non alpha-numeric characters.
For example, an `\_' (underscore) is often recognized as a `-' (dash). 
Finally, a space error results when the system inserts an unexpected space.
For example, when `ConflictError' is recognized as `Conflict Error', we characterize it as a space error. 

Most of the word errors and symbol errors can be attributed to poor writing or cursive writing (characters are written joined together in a flowing manner) which is inherently more difficult for MyScript to recognize than block writing (characters are written separately). Space errors, on the other hand, appear to depend on the language model of the recognizer, which most likely does not include training on CamelCase\footnote{\url{https://en.wikipedia.org/wiki/Camel\_case}} or proper English words separated by dot notation (e.g. student.name).  The result is that MyScript inserts space at these word and dot notation separators.

In summary, from the statistical results for each type of error presented in \autoref{fig:errors}, space errors, mainly caused by the internal mechanism of English handwriting recognition system, represent the most prevalent recognition error. In addition, poor writing and the tendency to return an English word for a non-English word in the source code lead to word errors, which also represents a significant portion of all errors. Symbol errors are also a prevalent error type. This makes sense given that MyScript is designed to recognize general words, however, symbols, dot notation, and combinations of symbols and words are typically not present in general text, especially in the way that they are used in source code. For example, the most problematic symbols includes underscore `\_', parentheses `( )' and equal `='.


\begin{figure}[t!]
\centering
\includegraphics[width=3.16in, height = 2.2in]{errors}
\caption{Average error numbers of all participants for each code sample from MyScript general handwriting recognition engine}
\label{fig:errors}
\end{figure}


\section{Handwritten Source Code Recognition Pipeline}



A programming language is governed by grammar rules, which stipulate the positions of keywords and symbols. For example, in \textit{Python}, a \textit{def} sentence must end with a `:'. However, handwritten symbols are often problematic.  For example, colons `:' are sometimes recognized as semicolons `;'. In addition to grammar rules, programming languages are highly repetitive with predictable properties\cite{hindle2012naturalness}. 
Function names and variable names are the most common repetitive words in a single source code project. If a function name appears more than once in the same handwritten code sample, however, it is impossible for users to hand write the \textit{exact} same strokes for this function name, which makes different recognition results of the same handwritten function name a possibility that we must account for.

In this section, we present an approach to improve the recognition rate for handwritten source code by addressing these issues as well as those common errors characterized in Section \ref{sec:characterizing}. We leverage what we know about the predictability and structure of source code to improve recognition results beyond that of the state-of-the-art recognizer. 


The general premise of our approach is that state-of-the-art engines can produce excellent results given good writing and the absence of symbols and programming practices like camelCase.  Our framework, illustrated in \autoref{fig:overview}, is therefore aimed at analyzing and post-processing the recognition results produced from MyScript to utilize its recognition capabilities but correct for those common errors. This framework can be divided into four parts: statement classification, statement parsing, token processing, and statement concatenation. 
%The various parts are presented in the following sub-sections.
The source code for this post-processing algorithm can be found at \url{http://www.purl.org/recognizinghandwrittencode/code}.

\begin{figure}[h!]
\centering
\includegraphics[width=0.5\textwidth]{overview}
\caption{Framework for augmenting MyScript to correct for common recognition errors in handwritten source code.}
\label{fig:overview}
\end{figure}

\begin{figure*}[h]
 \center
  \includegraphics[width=2.2in, height = 1.32in]{picture1}
  \includegraphics[width=2.2in, height = 1.32in]{picture2}
  \includegraphics[width=2.2in, height = 1.32in]{picture3}
  \caption{Average recognition error rate of MyScript and our augmented MyScript system for three test code samples}
  \label{result}
\end{figure*}



\subsection{Statement Classification}
As we mentioned before, we process the handwritten source code data considering each statement as a unit. According to the \textit{Python} grammar specification, we can restrict \textit{Python} source code statements into a limited number of classes, each of which has specified structure rules\footnote{\url{https://docs.python.org/2/reference/grammar.html}}. Here we use the first token in the statement as the symbol for classification. For example, a `def' statement starts with `def' and its structure is defined as `def' + `function name' + `(parameters0, parameters1 ...):'. We define 14 classes for \textit{Python} code statements, including an `assignment' statement, which means the first word in this statement is not a keyword but rather a variable name. In \autoref{fig:overview}, the recognition result is classified as an `if' statement. \autoref{table:statement} presents statistics for the various statement classes in the three code samples.



\begin{table}
  \centering
  \begin{tabular}{l r r r}
    % \toprule
    {Class}
    & {Frequency}
      & {Class}
    & {Frequency} \\
    \midrule
    def & 3 & except & 1  \\
    if & 7 &  while & 1 \\
    for & 3  & try & 1 \\
    raise & 2 & break & 1 \\
    return & 2  & else & 1 \\
    yield & 2 & assignment & 13 \\
    % \bottomrule
  \end{tabular}
  \caption{Frequency for each statement class in three test code samples}~\label{table:statement}
\end{table}



\subsection{Statement Parsing}
After classifying the statement, we need to break it down into independent parts according to the grammar rules.
Similar to a recursive-descent parser \cite{van1993recursive}, our system consists of a series of functions, each of which is responsible for one class of statement. Each function includes a set of mutually recursive procedures where each such procedure implements one of the productions of the grammar as a regular expression. We implement a top-down LL parser to parse the input from left to right and perform a leftmost derivation \cite{fernau1998regulated} of the statement. As a result, a statement is parsed into a list of single tokens and/or characters. For example, the statement in \autoref{fig:overview} is parsed into five individual tokens. Specifically, `if' is a keyword token; `Cookie. name' is a variable token; `==' is a symbol token; `naue' is a variable token; `;' is the last symbol token.



\subsection{Token Processing}
The previous stage results in a list of single tokens and/or characters that make up the statement.
%In this step, we process the list of the single words or characters received from the last step. 
We assume all non-keywords are properly recognized and add them to the lexicon assuming they are \emph{variable} names.
%We build a non-keyword lexicon to save all non-keywords in the first sentence. 
Then for all non-keywords in each statement that follows, we first compare the token to all the words in the non-keyword lexicon. If a `similar' token already exists in the lexicon, we replace it with the `similar' token in the lexicon. For example, in \autoref{fig:overview}, `naue' is very similar to `name', which is already in the lexicon, so we just replace the token `naue' with `name'. If there is no `similar' token in the lexicon, we accept this token as it is and add it to the lexicon. We calculate similarity using the Levenshtein distance \cite{levenshtein1966binary} with a threshold of 0.7, determined empirically.

\subsection{Statement concatenation}
After processing all tokens, we remove all extra spaces in any single token, then concatenate each token with a single space between them to reconstruct the final statement.
Additionally, we ensure that the last recognized character of a statement is a ':'.
For example, in \autoref{fig:overview}, we first remove the space in `cookie. name' and then replace the last character `;' with `:'.





\section{Evaluation}


%define WER and CER
To assess the performance of our system, we measure the Character Error Rate (CER) and Word Error Rate (WER). WER and CER are percentages obtained from the Levenshtein distance between the recognized sequence and the corresponding ground truth. They are calculated as
\[ \frac{D+I+S}{L} \times 100\% \]
where D is the number of deleted units, I is the number of inserted
units, S is the number of substituted units, and L is the total number of
units in the ground truth transcriptions. A unit is a word for WER or a
character for CER.


We evaluate our recognition approach by applying our framework to the 45 code samples in our database. In the following section, we compare the results of our enhanced recognizer to the results of using MyScript alone.

\section{Results} \label{results}

\begin{table}
  \centering
  \begin{tabular}{l r r}
    % \toprule
    {}
    & {t-test score ($t_{14}$)}
      & {P-value} \\
    \midrule
    WER on sample 1 & -9.02 & $P < 0.00001$ \\
    WER on sample 2 & -8.29 & $P < 0.00001$ \\
    WER on sample 3 & -6.57 &  $P < 0.00001$  \\
    CER on sample 1 & -3.88  & $P < 0.001$  \\
    CER on sample 2 & -5.45 & $P < 0.0001$ \\
    CER on sample 3 & -6.13 & $P < 0.0001$ \\
    % \bottomrule
  \end{tabular}
  \caption{Statistical evidence (T-test and P-value) for WER and CER on three code samples}~\label{table:significance}
\end{table}


%result
As shown in \autoref{result}, 
our augmented recognition approach results in an 8.6\% word error rate and 3.6\% character error rate, on average, over the three code samples, which outperforms the original MyScript recognizer with 31.31\%  and 9.24\% in word and character error rate respectively. We also find statistical evidence for an effect of our augmented recognition approach on both WER and CER (See \autoref{table:significance}). 

%on WER on sample \#1 ($t_{14} = -9.02, P < 0.00001$), on CER on sample #1 ($t_{14} = -3.88, P < 0.001$), WER on sample 2 ($t_{14} = -8.29, P < 0.00001$), CER on sample 2 ($t_{14} = -5.45, P < 0.0001$), WER on sample 3 ($t_{14} = -6.57, P < 0.00001$), and CER on sample3 ($t_{14} = -6.13, P < 0.0001$).


%compare to English handwriting recognition
%%Hi Ron, these four systems has lowest WER and CER, the 97% and 95% I said before is not English recognition, it's for arabic. 
\begin{table}
  \centering
  \begin{tabular}{l r r}
    % \toprule
    {System}
    & {WER(\%)}
      & {CER(\%)} \\
    \midrule
    \textcolor{red}{Augmented MyScript}  & \textcolor{red}{8.6} & \textcolor{red}{3.6} \\
    Kozielski et al. \cite{doetsch2013improvements} & 9.5 & 2.7 \\
    Keysers et al. \cite{keysers2016multi} & 10.4 &  4.3  \\
    Zamora et al. \cite{zamora2014neural} & 16.1  & 7.6  \\
    Poznanski et al. \cite{poznanski2016cnn} & 6.45 & 3.44 \\
    % \bottomrule
  \end{tabular}
  \caption{Performance of our system compared to handwritten English recognition systems on the IAM dataset}~\label{table:iam}
\end{table}

%To examine how our source code recognition rates compare
%to acceptable recognition rates in general handwriting recognizers,
%we present the recognition rate comparison between our augmented
%MyScript recognition system (on source code) to four other state-ofthe-art
%general handwritten English recognition systems

Since there is no existing handwriting source code recognizer for comparison, we compare the recognition rate of our our augmented MyScript recognition system (on source code) to that of four state-of-the-art general handwritten English recognition systems (on general text). 
%Since there is no existing handwriting source code recognizer for comparison, we compare our recognition rates on source code to acceptable recognition rates of general handwriting recognizers on general text.  Specifically, we compare the recognition rate of our our augmented MyScript recognition system (on source code) to that of four state-of-the-art general handwritten English recognition systems on general text. 
The IAM handwriting database \cite{marti2002iam} consists of 9,285 lines of general handwritten text
written by approximately 400 writers with no restrictions on style or writing tool. This database has been widely used to evaluate English handwriting recognition systems. The four systems in \autoref{table:iam} were tested based on this IAM handwriting database. \autoref{table:significance} shows that the WER and CER of our augmented source code recognition system are comparable with other state-of-the-art handwritten English recognition systems on general handwritten text.

%Finally, we acknowledge the importance of indentation in \textit{Python} source code.  We chose to ignore indentation for the purposes of this project to focus solely on recognition and would argue that recognition results will not be affected by indentation considerations.  We are in the process of introducing indentation into the collected data to create an identical set of source code samples with appropriate indentation.

%Because typing on a virtual keyboard is the standard input method on touchscreen devices, it's useful to examine how virtual keyboard typing error rates compare to those of handwritten source code recognition.
% %Regular typing is also compared with handwriting source code on a virtual keyboard. 
% For example, Almusaly et al. report a 7.81\% total error rate (TER) for typing java programs on a standard virtual keyboard as measured from 32 participants \cite{almusaly2015syntax}. TER, similar to CER, is a measure of the total number of errors (i.e., omissions, substitutions, and insertions) and corrections that are made in the resulting typed text. 


%explain how our system fix errors in characterization and existing errors

\section{Discussion}


%The improvement of recognition rate embodied in fixing all the three error types. 


Our approach achieved an 8.6\% word error rate and a 3.6\% character error rate on the collected dataset by taking the language grammar rules into account. Overall, improvement of our recognition pipeline over the baseline MyScript recognition engine can be attributed to addressing the three main error types identified in Section \ref{sec:characterizing}.  After statement concatenation, all unnecessary space errors in a single token are removed. Ensuring the last character of a statement eliminates 32\% of the symbol errors. Token processing fixes around 78\% of the word errors. 

Recognition results, however, are still not 100\% accurate. Initial inspection indicates that this is mainly due to the illegible or cursive handwriting of the participants and the incorrect recognition of symbols. Also, since one of our lexicons is dependent on the non-keywords already recognized in the code, incorrectly recognized words will also be added to the lexicon, thereby corrupting the lexicon and preventing it from enhancing the recognition of the following words. Additionally, it is difficult to identify incorrectly recognized symbols; for example, if `(' appearing in the middle of the text is recognized as `l', it becomes impossible to rectify it using our approach. 
%errors still existing, develop widget for errors, what will improve
%Our approach achieved an 8.6\% word error rate and a 3.6\% character error rate on the collected dataset by taking the language grammar rules into account. 
%It should be noted that there are still identified errors remaining to be corrected. 
Errors like unmatched `(' and `)' in a statement can be detected, but not reliably corrected. For example, `(name' can be recognized as `cname', but we have no evidence to correct `cname' to `(name'. Two methods can be employed to resolve remaining errors such as this. The first is to develop a widget in the handwriting interface to highlight all errors that are identified but can't be corrected and let users correct them manually. Another option is to train a language model to identify words that do not exist \cite{zamora2014neural}. 


Because typing on a virtual keyboard is the standard input method on touchscreen devices, it is useful to examine how virtual keyboard typing error rates compare to those of handwritten source code recognition.
% %Regular typing is also compared with handwriting source code on a virtual keyboard. 
Almusaly et al. report a 7.81\% total error rate (TER) for typing \textit{Java} programs on a standard virtual keyboard as measured from 32 participants \cite{almusaly2015syntax}. TER, similar to CER, is a measure of the total number of errors (i.e., omissions, substitutions, and insertions) and corrections that are made in the resulting typed text.  Our handwriting results are comparable.

%generalize to other languages
This approach can also be generalized to other programming languages with strict grammar rules. For instance, one can define statement classes for \textit{Java} according to the first word in the statement and then replace the regular expressions with productions of \textit{Java} grammar rules.  Algorithms for searching and replacing similar words can be kept unchanged. Other heuristic steps like concatenating tokens are also trivial to implement for new languages. 
%Due to the uncertainty of the program structure, however, our approach cannot easily be generalized to non-strict programming languages.

%\section{Discussion and Implication for HCI}
\section{Conclusion and Future Work}
%%mobile programming is widely used, and its limitation.
%Supporting programming on mobile devices is not a novel idea. With the rapid technology shift in current computing devices, high-quality low-cost mobile devices such as tablets and smartphones are being increasingly used in everyday activities. Many tasks that previously required a PC are now feasible on mobile devices. For example, tablets are typically equipped with powerful batteries, advanced graphic processors, high-resolution screens and fast processors, making writing and compiling code on them completely plausible. TouchDevelop \cite{tillmann2011touchdevelop}, for example, is a novel programming environment, language and code editor for mobile devices and Tilman et al. \cite{tillmann2012future} predict that programming on mobile devices will be widely used for teaching programming.  However, mobile devices are also inherently restricted by their limintations such as small screens and the clumsy virtual keyboard that is difficult to use and that takes up valuable screen space - posing challenges for a programmer who wishes to enter or edit source code.
%For this to happen, however, input must be made more intuitive.
%Alternatives to typing have been considered for a long time. One is to use speech (via dictation) to create the text of a program. Desilets et al. \cite{desilets2006voicecode} propose VoiceCode, which translates the pronounced syntax into native syntax in the current programming language to support programming by speech. Gordon \cite{gordon2013improving} employs language design and incorporates dynamic context for this purpose. Speech, however, is not always an appropriate option given social conventions and privacy issues. Given the advances of pen-based input technology, we chose to explore handwriting input in this paper.  

 
%conclusion
The keyboard is not an ideal input mechanism for every person and situation.
Alternatives to typing, such as speech, have been considered in the past \cite{desilets2006voicecode, gordon2013improving}. However, speech is not always an appropriate option given social conventions and privacy issues.
Given advances in pen-based technology that provides an opportunity for users to engage with devices in a potentially more `natural' way than that supported by a virtual keyboard, handwriting input is a viable alternative to virtual keyboard input. In this paper, we have explored handwriting recognition specifically for source code with the ultimate goal of supporting handwriting as a means for programming.
%Handwriting is a viable alternative to a keyboard given recent technological advances in pen-based technology.
%for programming can be used for people with disabilities and people suffering RSI. %In this paper, we focus on supporting handwriting recognition for the particular domain of source code text input. 
We collect and present a small database of publicly available handwritten source code samples and we propose an approach to recognize handwritten source code by leveraging a commercial handwriting recognition system. Experiments on the data collected from 15 participants show our framework has an average 8.6\% word error rate and 3.6\% character error rate which outperforms the baseline recognition system and produces rates comparable to the recognition of general handwritten English text.  We are encouraged by these initial results but believe there are several avenues of future work.


%Given the advances of pen-based input technology, we chose to explore handwriting input in this paper. %%writing is new viable, stylus technique
%Input via stylus is becoming more precise and the familiar writing action makes handwriting a viable alternative to typing code on mobile devices. 
%Given advances in pen-based technology that provides an opportunity for users to engage with devices in a potentially more `natural' way than that supported by a virtual keyboard, we have explored handwriting recognition for source code with the ultimate goal of supporting handwriting as a means for programming.
%Additionally, research suggests that handwriting can lead to cognitive, memory, and creativity enhancements \cite{alonso2015metacognition}. Alternative input mechanisms are also important for people with RSI or other motor impairments who find typing difficult, and in general, for those who may wish to carry out simple source code editing and entry tasks in mobile situations. 
%While handwriting without recognition produces `digital ink' that is appropriate for applications like annotation and graphic design, 
%For handwriting to be used in scenarios like programming, however, applications must be equipped with recognition technology to support translation to searchable and editable digital text. To effectively incorporate the handwriting experience for source code entry and editing, we must first address the source-code recognition problem.




%importance


  

% Summarize what we've done
%We have presented an initial study to collect data on handwritten source code and explored the use of a state-of-the-art recognition system for recognizing handwritten source code. 
%In this section, we talk about why it is important to HCI and how HCI community could benefit from our work.


%%writing is new viable, stylus technique
%Input via stylus is becoming more precise and the writing action is very similar to writing on paper. Handwriting is therefore a viable alternative to typing code on mobile devices. Moreover, handwriting is an acceptable input method for people with RSI or other motor impairments who find typing difficult. 
%Thus studying handwriting input method will be a benefit for a large group of users. 
%While handwriting without recognition produces `digital ink' that is appropriate for applications like annotation and graphic design, for handwriting to be used in scenarios like programming, it must be equipped with recognition technology to support translation to searchable and editable digital text.

%%implication and importance for HCI
From the view-point of human-computer interaction, usability and user satisfaction is critical. For handwriting text input, users expect recognition technology with a low error rate and responsive recognition speed. LaLomia et al. \cite{lalomia1994user} reported that users are willing to accept a recognition error rate of only 3\% (a 97\% recognition rate), although Frankish et al. \cite{frankish1995recognition} concluded that users will accept higher error rates depending on the text-editing task. It would not be surprising, therefore, if higher error rates were acceptable for source code entry and editing which is inherently difficult due primarily to the use of symbols. Input speed is another concern with respect to handwriting. Modest touch typing speeds on a virtual keyboard in the range of 20 to 40 words per minute (wpm) are achievable.
Handwriting speeds are commonly in the 15 to 25 wpm range \cite{card1983psychology,devoe1967alternatives,dunlop2009pickup}. We suspect that this decrease in speed, however, will be acceptable to the particular groups for whom handwriting is the most viable input option. Additionally, in professional programming, most of the code that developers
write involves reuse of existing example code and libraries \cite{bellon2007comparison}. This `reuse' typically amounts to editing existing code to suit a
new context or problem and generally provides benefits to developers in terms of time and error reduction \cite{ko2011state}.
%is able to save time and
%avoid the risk of writing erroneous new code \cite{ko2011state}.
For these reasons, we envision our system as being particularly useful in the code editing domain as opposed to writing extensive source code from scratch.  Studying how the algorithms perform in editing tasks is left as future work.

%a recognizer used primarily for code edits as opposed to being used to write an entire program from scratch.


% How does the HCI community benefit from this work
%% Dataset for testing

While databases exist for research in general handwritten text recognition \cite{marti2002iam, grosicki1rimes}, there is no such dataset for handwritten source code.  This paper represents the first such contribution of a handwritten source code dataset consisting of 555 lines of  \textit{Python} code written by 15 participants. While we recognize that using the same three code samples for all users and employing a ``copy task'' may lessen the generality of the dataset, we sought to eliminate all effects of cognitive complexity (e.g. actually solving programming problems) to focus solely on the handwritten source code quality.  Collecting data for other programming languages and for actual programming tasks is left as future work.

%A standard database is needed to facilitate research in handwritten source code recognition. For general handwritten text recognition, the IAM database \cite{marti2002iam} and the RIMES database \cite{grosicki1rimes} are widely used for research purposes. Based on Lancaster-Oslo/Bergen (LOB) corpus, the IAM database \cite{marti2002iam} consists of 9,285 lines of handwriting text from 400 writers. The RIMES Database comes from the ICDAR 2011 block-recognition competition and consists of 1,500 paragraphs of the handwritten French text.
%Unfortunately, there is no such dataset for handwritten source code - this paper represents the first such contribution.  We collected only a small handwritten source code dataset consisting of 555 lines written by 15 participants. While we recognize that using the same three code samples for all users and employing a ``copy task'' may lessen the generality of the dataset, we sought to eliminate all effects of cognitive complexity (e.g. actually solving programming problems) to focus solely on the handwritten source code quality.  Collecting data from actual programming tasks and for additional programming languages is left as future work.
%and affect the writing organization.   but cognitive load and possible errors are avoided. In addition, t
%It is expected that the database would be particularly useful for further handwritten source code recognition research using \textit{Python} as the language of choice.  More data on additional languages will be necessary to further investigate handwriting as a viable input mechanism for source code.



%% Approach based on state-of-the-art recognizer
%Our handwritten source code recognition framework is implemented by leveraging the programming language grammar information to augment an existing handwriting recognition system. By replacing the grammar rules in the framework, the system can be generalized to other programming languages. It should be noted, however, that using an existing handwriting recognition system designed for natural language is not tackling the problem at its source. Rebuilding the core of a recognition system based on properties of source code is thus an alternative approach that should be explored.  We leave this for future work.






%conclusion
%The keyboard is not an ideal input mechanism for everyone.
%Handwriting as an alternative to a keyboard for programming can be used for people with disabilities and people suffering RSI. In this paper, we focus on supporting the recognition aspect of handwriting for source code text input. We collect and present a small database of publicly available handwritten source code samples and we propose an approach to recognize handwritten source code by leveraging a commercial handwriting recognition system. Experiments on the data collected from 15 participants shows our framework has an average 8.6\% word error rate and 3.6\% character error rate which outperforms the baseline recognition system and produces rates comparable to recognition of general handwritten English text.

%future work

%%from scratch
%%IDE
%Clearly, the current work is limited in both scope and depth and 
%We are encouraged by these initial results but believe there are several avenue of future work. 

The next most obvious area of future work is to develop a handwritten source code recognition system from scratch instead of augmenting the results produced by an existing system.  We suspect this approach would lead to comparable and most likely improved recognition rates. Building a universal handwritten source code reading system could employ deep learning techniques such as Concurrent Neural Networks \cite{poznanski2016cnn} or neural network language models \cite{zamora2014neural} trained purely on the source code. 

Additionally, there are several opportunities to explore the integration of handwriting recognition into source code IDEs \cite{frye2008pdp}.  For example, how do we now integrate source code completion into a handwriting-based interaction?   Can we integrate elements such as syntax insertion and highlighting?  Exploring the affordances of handwriting in the context of an IDE is an exciting area of future work that is enabled by these initial findings.

%Similar to an IDE for handwriting C\# code \cite{frye2008pdp}, integrating handwriting source code to current programming IDE or building a programming IDE solely with handwriting as an input method is also valuable to explore.

%%speed, auto completion
%Furthermore, to address concerns pertain to handwriting speed, it may be possible to add auto-completion feature into handwriting source code interface. Next word or character suggestion is also helpful to facilitate inputting. In addition, combing voice input and handwriting input may also improve the inputting speed.

Multimodal methods present another area of future work.  Perhaps the combination of handwriting and speech input or handwriting and occasional keyboard input \cite{mueller2014pen} begin to produce interaction experiences that rival those of typed source code input.  

Finally, we will never reach a perfect recognition rate for handwritten text (general or source code).  How do we effectively support efficient editing of the recognized text so that users can quickly correct mistakes? Natural and effective text entry and editing is an interesting topic for future studies.

%%texue of future work.t editing
%, our research projeng avenct aims to develop a programming interface to support handwriting, editing and recognizing source code. We also intend to explore the potential of adding editing techniques such as selecting, deleting, copying, and pasting.






%% if specified like this the section will be committed in review mode
\acknowledgments{
The authors wish to thank all the study participants as well as Poorna Talkad Sukumar, Jason Liu, and Suwen Lin for their valuable discussions and input.}

%\bibliographystyle{abbrv}
\bibliographystyle{abbrv-doi}
%\bibliographystyle{abbrv-doi-narrow}
%\bibliographystyle{abbrv-doi-hyperref}
%\bibliographystyle{abbrv-doi-hyperref-narrow}

\bibliography{template}
\end{document}



Text-based video editing~\cite{tuneavideo,zhao2023controlvideo,vid2vid,qi2023fatezero,liu2023video} has recently garnered significant interest as a versatile tool for multimedia content manipulation. However, existing approaches present several limitations that undermine their practical utility. Firstly, traditional methods typically require per-video-per-model finetuning, which imposes a considerable computational burden. Furthermore, current methods require users to describe both the original and the target video~\cite{tuneavideo,zhao2023controlvideo,vid2vid,qi2023fatezero,liu2023video}. This requirement is counterintuitive, as users generally only want to specify what edits they desire, rather than providing a comprehensive description of the original content. Moreover, these methods are constrained to individual video clips; if a video is too long to fit into model, these approaches fail to ensure transfer consistency across different clips.

To overcome these limitations, we introduce a novel method with several distinctive features. 
Firstly, our approach offers a universal one-model-all-video transfer, freeing the process from per-video-per-model finetuning. Moreover, our model simplifies user interaction by only necessitating an intuitive editing prompt, rather than detailed descriptions of both the original and target videos, to carry out desired alterations. Secondly, we develop a synthetic dataset precisely crafted for video-to-video transfer tasks. Through rigorous pairing of text and video components, we establish an ideal training foundation for our models. Lastly, we introduce a sampling method specifically tailored for generating longer videos. By using the transferred results from preceding batches as a reference, we achieve consistent transfers across extended video sequences. 

We introduce Instruct Video-to-Video (\ours), a diffusion-based model that enables video editing using only an editing instruction, eliminating the need for per-video-per-model tuning. This capability is inspired by Instruct Pix2Pix~\cite{brooks2023instructpix2pix}, which similarly allows for arbitrary image editing through textual instructions. A significant challenge in training such a model is the scarcity of naturally occurring paired video samples that can reflect an editing instruction. Such video pairs are virtually nonexistent in the wild, motivating us to create a synthetic dataset for training.

Our synthetic video generation pipeline builds upon a large language model (LLM) and the Prompt-to-Prompt~\cite{prompt2prompt} method that is initially designed for image editing tasks (\Cref{fig:synthetic_dataset}). We use an example-driven in-context learning approach to guide the LLM to produce these paired video descriptions. Additionally, we adapt the Prompt-to-Prompt (PTP) method to the video domain by substituting the image diffusion model with a video counterpart~\cite{ho2204video}. This modification enables the generation of paired samples that consist of an input video and its edited version, precisely reflecting the relationships delineated by the editing prompts.

In addressing the limitations of long video editing in conventional video editing methods, we introduce Long Video Sampling Correction (LVSC). This technique mitigates challenges arising from fixed frame limitations and ensures seamless transitions between separately processed batches of a lengthy video. LVSC employs the final frames of the previous batch as a reference to guide the generation of subsequent batches, thereby maintaining visual consistency across the entire video. We also tackle issues related to global or holistic camera motion by introducing a motion compensation feature that uses optical flow. Our empirical evaluations confirm the effectiveness of LVSC and motion compensation in enhancing video quality and consistency.



\section{\tokdetok}
\label{sec:model}

We focus on character-informed representations for \llm{}s which use the transformer architecture~\cite{vaswani2017attention}, such as BERT~\cite{devlin-etal-2019-bert}, RoBERTa~\cite{liu2019roberta}, 
and GPT~\cite{radford2018improving}.
A Transformer \llm{} is composed around a core module \mmod{} parameterized as multi-head self-attention layers, which accepts a sequence $\{\ve_1, \dots, \ve_l\}$ of embedding vectors corresponding to a list of token indices $\{t_1, \dots, t_l\}$ from a vocabulary $\mathcal{V}$,
and outputs a sequence of contextualized vectors $\{\vh_1, \dots, \vh_l\}$, of which a subset $\{\vh_i\}_{i\in\mathcal{I}}$ correspond to masked tokens in positions $\mathcal{I}\subseteq [l]$ selected by a boolean masking operator $m$.\footnote{In the case of an autoregressive \llm{} like GPT, $\mathcal{I} = [l]$ but each vector $\vh_i$ is only dependent on the tokens preceding its position, $\{t_1, \dots, t_{i-1}\}$, and token-wise prediction is performed for each position assuming no knowledge of future tokens.}
The token index list supplied to \mmod{} is the output of a tokenization function $\tau$ operating on a sequence $X$ followed by lookup in an embedding table $\vE$,
while its output vectors $\{\vh_1, \dots, \vh_l\}$ serve as input to a prediction module \gen{}, which outputs a distribution $D_i$ over $\mathcal{V}$ for each masked position $i$.
All embeddings $\{\ve_i\},\{\vh_i\}$ live in a shared space $\mathds{R}^d$.
Together, the components described so far operate in the following manner:

\begin{equation}
    \{D_i(\mathcal{V})\}_{i\in\{\mathcal{I}\}} = 
    \text{\gen{}}\left( \text{\mmod{}} \left( \vE\left[ \tau(X) \right] \right) \right),
\end{equation}
where square bracketing denotes elementwise table lookup.

In the considered \llm{} architectures, the input $X$, which is atomically made up of a sequence of characters $\{c_1, \dots, c_n\}\in\Sigma^{n}$, is broken down by $\tau$ to provide the tokenization of length $l\leq n$,
and the prediction/generation operator \gen{} accepts the contextualized outputs $\{\vh_1, \dots, \vh_l\}$ and implements prediction by means of a softmax distribution which is based on scores obtained via dot-product against an output embedding table, which is usually the same as $\vE$.

The \tokdetok{} model makes no adjustments to \mmod{} itself, and only offers conditional replacements for $\tau$ and \gen{}.
An \textbf{encoding policy} $\pi^t$ selects a subset of tokens from the original sequence $\mathcal{J}^t\subseteq [l]$ to be represented by \tok{} instead of $\vE \circ \tau$.
\tok{} has access to the part of the character sequence $X$ which underly the tokens selected by $\pi^t$, and produces input embeddings directly from the character level.
These alternate embeddings must agree in dimension with those of each $\ve$, but a single embedding may be used to replace multiple base tokens (usually when all tokens corresponding to a single out-of-$\vE$ word are replaced), resulting in a shorter input sequence for \mmod{}.
A separate \textbf{generation policy} $\pi^g$ selects a subset of the original sequence $\mathcal{J}^g\subseteq [l]$ to be generated from the output vectors $\vh$ by \detok{} instead of \gen{}.
A high-level schematic depicting this framework is presented in \autoref{fig:algo}.

\begin{table*}
    \centering
    \tiny
    \begin{tabular}{ll}
        \toprule
        % Original & \texttt{Approx~~~~~~~~~~distance in miles from Bruges~~~~~~~to Amsterdam is 107 miles or 172.16 KMS~~~~.} \\
        % Tokenization & \texttt{A \#\#pp \#\#ro \#\#x distance in miles from B \#\#rug \#\#es to Amsterdam is 107 miles or 172 . 16 K \#\#MS .}  \\
        {\small Original} & \texttt{He was emphatically~~~~~a modern gentleman,~~of scrupulous~~~~~~~courtesy,~~sportive~~~gaiety,} \\ % acquainted with what was going on in the world.} \\
        {\small Word pieces} & \texttt{He was em \#pha \#tically a modern gentleman , of s \#c \#rup \#ulous courtesy , sport \#ive g \#ai \#ety ,} \\
        \midrule
        % \multirow{2}{*}{\small Random-20\%} & {\Monospace .9~.75~.15~~~~~~~~~~~~~.3~.05~~~~.55~~~~~~.6~.4~.7~~~~~~~~~~~~~~~.1~~~~~~.8~.25~~~~~~~~.45~~~~~~~.3} \\
        {\small Random-20\%} & \texttt{He was [TOK]~~~~~~~~~~~~a [TOK]~~gentleman , of s \#c \#rup \#ulous~[TOK]~~~~, sport \#ive g \#ai \#ety ,} \\
        {\small All-multi} & \texttt{He was [TOK]~~~~~~~~~~~~a modern gentleman , of [TOK]~~~~~~~~~~~~courtesy , [TOK]~~~~~~[TOK]~~~~~~,} \\
        {\small \textsc{Suffixes}} & \texttt{He was [TOK]~~~~~~~~~~~~a modern gentleman , of [TOK]~~~~~~~~~~~~courtesy , sport \#ive [TOK]~~~~~~,} \\
        \bottomrule
    \end{tabular}
    \caption{Example of token input selection by different policies,
    % (randomly-generated numbers determine the random policy) 
    where \texttt{[TOK]} signifies a word to be replaced by its character-based representation from \tok{}.
    Sentence fragment taken from the MS-MARCO QA dataset (see \S\ref{sec:tasks}) and tokenized using BERT-cased (\#\# replaced with \# to fit paper width).}
    \label{tab:policies}
\end{table*}



The specific policies in a given application may be defined based on the model's use case.
For example, in text classification no generation is required, and so $\pi^g$ will return $\emptyset$ for all sequences; $\pi^t$ can be tuned for a task based on known features of the base model (BERT/GPT etc.) and of the domain text, some examples including:
tokens corresponding to all words that are not in a pre-determined vocabulary;
all words in the sequence;
all words assigned more than one token by the base tokenizer $\tau$;
a random sample of words in the sequence;
or all words including characters that are not lowercase English characters.
One particular policy we hypothesize could be useful is one that affords the tokenizer slack in detecting a single simple derivational or inflectional suffix: all words which are single-token or whose second-and-final token is in the list \textsc{Suffixes} are left for $\vE \circ \tau$; the rest are represented using \tok.\footnote{\textsc{Suffixes}, compiled by manually examining a list of most common second-and-final tokens in a large corpus under GPT-2's tokenization, = $\{$\emph{s}, \emph{ed}, \emph{es}, \emph{ing}, \emph{ly}, \emph{al}, \emph{ally}, \emph{'m}, \emph{'re}, \emph{'ve}, \emph{y}, \emph{ive}, \emph{er}, \emph{'t}, \emph{'ll}, \emph{an}, \emph{ers}$\}$.
Similar policies can be hand-crafted for different languages, based on learner-level knowledge of the language and minimal preliminary analysis of sample tokenizations. Our results show they are not strictly necessary.}
Different policies may be applied in training settings as well, for example in order to \say{familiarize} the heavily-parametrized \mmod{} with the inputs from \tok{}.
Three example policies are illustrated in~\autoref{tab:policies}.



\subsection{Second Pre-training}
\label{ssec:pretr}

A typical \llm{} is initiated through a computationally-intensive pre-training step, iterating over a large corpus in batches of sequences and backpropagating a cross-entropy loss calculated over the prediction layer's output into all of its components, through \gen{} to \mmod{} to $\vE$.
In order to train \tokdetok{}, we introduce a second pre-training step we term \ppt{}, where the \llm{} continues to update its parameters for a (possibly different) corpus, but is supplemented with the \tokdetok{} elements in order to \say{acclimatize} the \mmod{} components to outputs from \tok{}.\footnote{Within the taxonomy of phases between pre-training and task training~\cite{ruder2021lmfinetuning}, this is closest to \textbf{Adaptive Fine-Tuning}. However, due to our added modules we opt to assign it a new name.}
In addition, \tok{} is also trained through a lower-level objective requiring it to approximate the outputs of $\vE \circ \tau$ which it is replacing, and \detok{} is trained to sequentially produce the correct character sequence from \mmod{}'s outputs.
Together, a batch of text sequences in a \ppt{} step produces the following loss elements which are backpropagated into the unified model:
\begin{itemize}
    \item A \textbf{language modeling loss} from the softmax operation over masked tokens, updating \mmod{}'s and $\vE$'s parameters, as well as \tok{}'s for tokens selected by a usage policy $\pi^t_{(u)}$ to be used in \mmod{}'s input;
    \item An \textbf{embedding loss} for the \tok{} component, computed against $\vE \circ \tau$'s token embeddings, over a set selected by a policy $\pi^t_{(l)}$ which may or may not equal $\pi^t_{(u)}$.
    This loss can be computed, e.g., as the euclidean distance between the output and the target.
    When the target corresponds to multiple token counts, an aggregation pooling function $\agg:\mathds{R}^{d \times \mathds{N}^{+}}\rightarrow\mathds{R}^d$ needs to be defined over the embeddings, for example taking their dimension-wise mean, taking the leftmost token's embedding, or taking the dimension-wise $\max$;
    \item A \textbf{generation loss} for words generated by \detok{} from \mmod{}'s output vectors, according to a $\pi^g$ policy.
    This loss is the character-level cross-entropy for autoregressive sequence generation.
    Note that in order to only generate full words, $\pi^g$ must align with $\pi^t_{(u)}$ so that no multi-token words left as input to \mmod{} are also selected for generation.
\end{itemize}

As a form of regularization within the \tokdetok{} components, we introduce additional training batches we call \textbf{cycle dependency loops}.
In such a batch, \tok{} and \detok{} act in succession, starting from one of the spaces they operate in, with the goal of arriving at the same point after cycling through both components.
An \textbf{\tdloop} loop thus starts at a character sequence $\vc\in{\Sigma^*}$, runs it through \tok{} to obtain a vector $\ve=\tok(\vc)$, and runs \detok{} in an attempt to return to the original sequence $\hat{\vc}=\detok(\tok(\vc))\approx \vc$.
Analogously, a \textbf{\dtloop} loop starts at a vector $\tilde{\ve}\in\mathds{R}^{d}$ and targets $\hat{\ve}=\tok(\detok(\tilde{\ve}))\approx \tilde{\ve}$.
In this loop, loss is only backpropagated as far as \tok{}, since backpropagating through a generative model's decision component introduces discrete steps which must be smoothed or approximated (see discussion in \newcite{peng-etal-2018-backpropagating}).




\section{Tasks}
\label{sec:tasks}

To evaluate the advantages of character-sequence awareness in large-scale transformers, we chose a diverse set of datasets which reflect unedited user-generated language in English, as well as its interaction with edited text.
We report results on a sequence classification task (emoji prediction), a sequence tagging task (named entity recognition, or NER) in both an in-domain (Twitter NER) and cross-domain (emerging entities NER) setting, a sequence ranking task based on information retrieval in a hybrid edited-unedited textual setting (MARCO-QA ranking), and a task where a single word's class is predicted within a sequence (NYTWIT).

\paragraph{Emoji prediction.}
The English portion of the Multilingual Emoji Prediction dataset \cite{barbieri-etal-2018-semeval,ma-etal-2020-multi} is composed of tweets containing one of the twenty most common emoji symbols.
For the task, the emoji are stripped from the tweet text and the system is asked to predict which one appeared in the original tweet, allowing it to be construed as a self-annotated, fine-grained sentiment analysis task.
All tweets are identified as geographically originating in the US.
We note that there is a significant qualitative difference between the training (+ development) set, and the test set of this corpus, hinted at by one of the participant teams in the original task~\cite{chen-etal-2018-peperomia} but not explored.\footnote{Another team identified a seasonal shift affecting the distribution of the Christmas tree 
% {\NotoEmoji\symbol{"1F384}} 
emoji~\cite{coster-etal-2018-hatching}.}
The partitions were extracted based on a temporal split, with all test set tweets post-dating all training set tweets by at least three months. %(February 2017 -- May 2017).
More crucially, the test set dates (May 2017 -- January 2018) overlap with Twitter's increase of the tweet character limit from 140 to 280 characters, phased in mostly during November 2017.
As a result, test set tweets have on average $\sim$10\% more words, increasing the amount of information within and diverging their textual feature distribution from that of the training set by more than is customary in NLP tasks.
The label distribution remains more or less the same.

\paragraph{Twitter NER.}
The Twitter NER dataset \cite{strauss-etal-2016-results} is a prime example of sequence tagging in noisy user-generated data settings.
It is composed of randomly sampled English tweets from 2016, annotated for ten different entity types, including music artists and sports teams, thus representing topics prevalent in social media.

\paragraph{Emerging entities recognition.}
The emerging entities dataset \cite{derczynski-etal-2017-results} shares most of its training set with the same Twitter source as the Twitter NER dataset, but as a domain-adaptation setup it includes a more ambitious evaluation fold: the development set is extracted from YouTube comments, and the test set from StackExchange and Reddit.
The taskmasters claim that the dataset contains mostly rare and hitherto-unseen entities, but do not provide exact statistics.

\paragraph{Community question answering.}
The MS-MARCO question answering dataset~\cite[\marco;][]{nguyen2016ms} was collected by mining a commercial search engine log for user queries and asking humans to answer them, supplying the answer writers with a set of passages retrieved automatically from edited text by the search engine which the answer writers then marked as \say{selected} if they helped them formulate the answer.
We recast this dataset into a selection / ranking problem, not pursued in the original challenge: given the query and the set of possibly helpful passages, which is the passage the answer writer selected?
To this end, we filtered out queries with more or fewer than one selected passage, and evaluated each system based on the mean reciprocal ranks (MRR) of the true selected passages in the rankings it produced.
Due to its size, we uniformly sampled 10\% of the queries and associated passage collections from each partition in this dataset, and ran all analyses and experiments on this new sampled dataset.

\paragraph{Novel word classification.}
The NYTWIT dataset~\cite{pinter-etal-2020-nytwit} includes passages in the New York Times where a word appears which has not appeared in the publication before.
The system is tasked to classify the novel word into one of eighteen types of novelty sources, such as \say{inflection of known word} or \say{lexical blend}.
The dataset is not partitioned into train/dev/test, and so we use the 10-fold partition from \newcite{pinter-etal-2020-nytwit} and report accuracy results aggregated over all instances, each from a model trained on the other nine folds.


\begin{table*}
    \centering
    \small
    \begin{tabular}{lcccccc} \toprule
        Dataset & Instances & Tokens & Types & TTR & Multitoks & Token mass \\
         & & \multicolumn{2}{c}{(Space-delimited)} & & & increase \\
        \midrule
        CoNLL-2003 NER & 14,986 & 204,563 & 23,624 & .115 & 16.08\% & 29.50\% \\
        Wiki1 & --- & 204,564 & 28,092 & .137 & 8.64\% & 12.85\% \\
        \midrule
        NYTWIT & 1,903 & 94,403 & 26,028 & .276 & 24.04\% & 38.04\% \\
        Wiki2 & --- & 94,403 & 16,903 & .179 & 9.11\% & 13.75\% \\
        \midrule
        Twitter NER & 2,394 & 46,469 & 10,586 & .228 & 17.23\% & 43.21\% \\
        Wiki3 & --- & 46,509 & 10,855 & .233 & 9.02\% & 13.77\% \\
        \midrule
        Emerging NER & 3,394 & 62,730 & 14,878 & .237 & 19.15\% & 54.25\% \\
        Wiki4 & --- & 62,757 & 13,392 & .213 & 9.05\% & 13.74\% \\
        \midrule
        Emoji Prediction & 427,458 & 4,973,813 & 504,644 & .101 & 32.96\% & 64.90\% \\ % test set: 21.53\% & 63.49\%
        Wiki5 & --- & 4,973,813 & 194,240 & .039 & 9.07\% & 13.58\% \\ % 95.52\% of types are multitoken! % prev version had 440,689 types, idk why
        \midrule
        \marco{} (10\% sample) & 80,704 & 46,956,674 & 1,355,049 & .029 & 20.58\% & 31.91\% \\ % 97.78\% of types are multitoken!
        Wiki6 & --- & 46,956,703 & 818,712 & .017 & 9.48\% & 14.37\% \\
        \bottomrule
    \end{tabular}
    \caption{Surface statistics for the datasets used for evaluation (training sets), against Wikipedia text of comparable token count sizes. Tokenization performed with GPT-2.}
    \label{tab:stats}
\end{table*}



\subsection{Analysis}
\label{ssec:stats}

We begin with an analysis of the datasets and their subword properties in order to gauge the direct effects of tokenization on the dataset's genre and domain.
In \autoref{tab:stats}, we present surface-level statistics reflecting the challenges posed by the datasets' sources, comparing each task's training set with comparably-sized sets of sentences sampled uniformly from English Wikipedia.
We use the GPT-2 tokenizer, which boasts the largest subword vocabulary of all models considered in this work, to obtain a lower bound on the added token mass expected by our models on these datasets, and report the following measures: the number of unique word types and type/word ratio, the percent of types which are subword-tokenized by GPT-2 (\textit{Multitoks}), and the overall increase in number of tokens over the corpus compared to a single-token-per-word representation (i.e., strictest application of \tokdetok{} considered in this work).

We find a striking disparity between the well-edited Wikipedia corpora, themselves far from being completely in-vocabulary, and the datasets at hand.
Wikipedia itself proves to be scale-invariant on the metrics that are not TTR, maintaining a token-OOV rate of roughly 9\% and a tokenization overhead of $\sim$13\%.\footnote{This number is very close to the one found by \newcite{acs2021subword} using multilingual transformer models; other languages' corpora boast overheads ranging from 28\% (French) to 95\% (Japanese).}
A control dataset in the form of CoNLL-2003's English NER portion~\cite{tjong-kim-sang-de-meulder-2003-introduction}, extracted from newswire text, exhibits a marked increase in complex words, possibly mostly named entities, or time-sensitive terms compared to the tokenizer's training set, alongside a decrease in type count which reflects its narrow source domain.
NYTWIT, despite also being extracted from newswire text, is sourced from news items that contain novel forms and so are often written in a high register or involve niche domains; as a result, it contains a larger overall token mass, split rather evenly across individual words (i.e., unknown words have fairly \say{regular} structure).
The Twitter NER datasets, exhibiting tweet language, do not exceed CoNLL's multitoken type proportion by much, but its OOVs tend to be completely unexpected forms, leading to a much higher raw post-tokenization count.
In the emoji dataset, which has not been pre-processed according to NER standards and instead was directly scraped off Twitter, almost a third of all unique forms are multi-token, and their presence enlarges the total token count by nearly two thirds.
\marco{} data, most of which is text from highly-ranked web pages, but also including user-generated queries, assumes a middle position between these two extremes.







\section{Experiments}
\label{sec:exp}


We evaluate the effect of \tokdetok{} when included in various \llm{}s. 
We choose \textsc{BERT-base-cased}~\cite{devlin-etal-2019-bert}, \textsc{GPT2-small}~\cite{radford2019language}, and \textsc{RoBERTa-base}~\cite{liu2019roberta} as the base \llm{}s to be manipulated.\footnote{Models loaded from the Huggingface repository~\cite{wolf-etal-2020-transformers}.}
The former contains roughly 108M parameters, and the latter two roughly 125M, a difference accounted for by their larger subword vocabulary (50k vs. 29k) and the resulting larger embedding table.
All models are case sensitive, but they differ in their strategy for preserving the original space-delimited word sequence: BERT's tokenizer marks word-non-initial tokens with a \say{\#\#} string, while GPT-2's tokenizer marks spaces with a special underline character and appends them to the following word.
The difference manifests itself in sequence-initial words, whose initial token in the GPT-2 representation shares the form of sequence-medial word-medial tokens, rather than that of sequence-medial word-initial tokens; and in symbols pre-tokenized without a preceding space, such as punctuation and apostrophes.
RoBERTa's tokenizer adopts the GPT-2 marking strategy but avoids the first pitfall by internally prepending all input sequences with a space character.
We perform the \ppt{} phase for each model on a collection of English tweets from 2016 obtained from the Firehose and preprocessed to replace all `@'-mentions with \texttt{@user}.
We later ablate this domain change effect by training models with \ppt{} text from the English Wikipedia March 2019 dump (see \S\ref{ssec:abl}).
We sample both resources to create pre-training corpora of roughly 725MB (unzipped), several orders of magnitude smaller than what contemporary models use for the first pre-training phase, and train for a single epoch.

In preliminary experiments on several character-level \tok{} architectures, we found that a convolutional net outperforms bidirectional LSTMs and small transformer stacks.
We pass the input characters through three separate convolution layers of width 2, 3, and 4 (characters), then pass the outputs through max-pooling layers and a \texttt{ReLU} activation, and finally project the concatenation of the results onto the base models' embedding dimension.
% \todo{Add figure if feeling masochistic.}
This \tok{} component contains $\sim$1M parameters, negligible compared to the transformers' parameter count.\footnote{The Wikipedia-trained models contain $\sim$2.8M parameters, the difference owing to language-ID filtering performed on the Twitter data, leading to a much smaller character set.} % \detok{}, which uses a 2-layer MLP with hidden layer same-dim as output dim, has 81M params for Wiki and 540K params for Twitter (but we don't use it in our tasks).
% exact numbers: 1071760, 2819360, 81470895, 539539
We implement \detok{} as a 2-layer unidirectional LSTM whose hidden layer is initialized by projecting the context vector $\vh_i$ output from \mmod{} into the hidden dimension.
Characters are generated by projecting the LSTM's output through two linear layers with a \texttt{tanh} activation.

During \ppt{}, we insert a cycle dependency batch every 5,000 LM steps.
For the replacement policies we choose $\pi^t_{(l)}$ to sample uniformly random sets of tokens representing 15\% of space-delimited words to pass as a target loss
for \tok{}, and $\pi^t_{(u)}$ to replace the embedding input to \mmod{} with \tok{}'s output for all multi-token words.
$\pi^g$ selects all tokens to pass as target losses for \detok{}, calculated as a sum of the cross-entropy loss for each character in the target sequence.
Cycle batches consist of sampling $k$ words out of the $K$ most frequent words from the training corpus, with replacement, and $k$ vectors from a Gaussian distribution blown to concentrate around the surface of the unit sphere in hidden-dimension space:
\[ \tilde{\ve} \sim \frac{1}{\sqrt{d}}\cdot\mathcal{N}(0,\vI^{(d)}). \]
\tdloop{} loops are optimized for a
character-level cross-entropy
loss, whereas \dtloop{} loops target a
euclidean distance loss.
When a \tok{} embedding correspond to multiple $\tau$ tokens, the learning target is created by max-pooling their embeddings.\footnote{This $\agg$ function outperformed average-pooling and first-token selection in preliminary experiments.}
We set $K=$25,000 and $k=$1,000.


\subsection{Intrinsic Assessment}

Across all base models and both \ppt{} corpora selected for our experiments, we observed a steady decrease in the \llm{} models' built-in loss metrics (masked prediction / autoregressive prediction) until stabilizing at roughly half the initial value before the end of the \ppt{} epoch.
This indicated that the transformer layers are able to process inputs from both the embedding table and \tok{} and reconcile them.
\autoref{fig:updates} depicts the parameter updates in RoBERTa by parameter type, across layers, comparing parameter values before and after the \ppt{} phase on Twitter data.
It shows that the change along the model layers is fairly stable, with mildly more extensive updates in the bottom and top layers.
The former is to be expected given the introduction of inputs from \tok{}; the latter can also be influenced by encountering Twitter data, which is substantially different than what RoBERTa is \say{used to}.

\begin{figure}
    \centering
    \includegraphics[width=8.25cm]{figures/layer_updates.png}
    \caption{Euclidean distance between RoBERTa weight parameter values before and after a \ppt{} training phase on the Twitter corpus.}
    \label{fig:updates}
\end{figure}

\begin{table*}
    \centering
    \footnotesize
    \begin{tabular}{lHcccccc}
        \toprule
        Model & Uncased BERT & BERT & \multicolumn{2}{c}{GPT-2} & \multicolumn{2}{c}{RoBERTa} \\ % 202102052207 & 202103171733 & 202104091857 & 202102260408 & 202103192237 & 202103192237 \\
        Corpus & Wikipedia & Wikipedia & Wikipedia & Twitter & Twitter & Twitter \\
        Steps & 76,000 & 9,000 & 13,500 & 21,000 & 7,500 & 57,500 \\
        Sequences ($10^3$) & 3,648 & 5,184 & 7,776 & 10,080 & 2,160 & 16,560 \\
         % 48x & 576x & 576x & 480x & 288x & 288x \\
         \midrule
        & ppercented & proming & crordman & d & orereren & everyone \\
        & or & dy & sssion & . & ant & kerned \\
        & ter & deded & gental & the & re & levernger \\
        & peprepored & terse & 2 & @ & cerent & and \\
        & essed & h & ther & ==666!!!!!!!!!!!!!!! & ennte & ed \\
         \bottomrule
    \end{tabular}
    \caption{Example generated words from random locations near the surface of the unit sphere in $\mathds{R}^{768}$.}
    \label{tab:gen}
\end{table*}


% More, from RoBERTa: Wiki-5\% after step 6,300 (576x):
% .<e> reenter<e> rencement<e> o<e> e<e> ĵ<e> ,<e> reciention<e> underesting<e> and<e>

% RoBERTa 13,000 Twitter-7.5\% (576x): reverally<e>	the<e>	revering<e>	.<e>	prot<e>	handers<e>	a<e>	there<e>	.<e>	@<e>





Another artifact of \ppt{} is the outputs of the \detok{} module.
While monitoring the training procedure, we periodically sample words from random locations centered around the surface of the unit sphere of the embedding space, to see what \say{priors} the generative net is learning from the vectors encountered during training.
\autoref{tab:gen} presents some of these samples for different models and different corpora at different points in the training phase.
Masked models appear to be learning well-formed fractions of English or pseudo-English words at early stages of the training phase from both Wikipedia and Twitter data.
As training continues, fewer sequences containing repetitions are observed, fewer generations occur across  samples, and more in-vocabulary words appear, suggesting convergence of the vector space towards representing well-formed and diverse English vocabulary.
The autoregressive GPT-2, on the other hand, struggles to produce meaningful sequences beyond short words and punctuation symbols when trained on the informal Twitter input, suggesting a difficulty in learning a mapping of language from vector space without availability of a two-sided context.



\subsection{Downstream Evaluation}
\label{ssec:finetune}

During task fine-tuning and inference,
since we do not evaluate on generative tasks, we do not use \detok{}.
We perform minimal hyperparameter search for each base model + task combination, and fine-tune model parameters during downstream training in all tasks but NYTWIT.
For these tasks, we also experimented with setups where \mmod{} and \tok{} are used as feature extractors only, and where \tok{} training is supplemented by an additional embedding loss (as described in \S\ref{ssec:pretr} for the \ppt{} phase) computed against embeddings of the task input.
Neither setup provided improvement on any task during our tuning experiments (see \S\ref{ssec:abl}).

Downstream models are implemented as follows: for sequence classification and ranking, a two-layer perceptron with \texttt{ReLU} activation is trained to make the prediction from the top-layer representation of the initial \texttt{[CLS]} token (in BERT and RoBERTa models) or of the final token in the sequence (in GPT-2).
For NER, an LSTM is run over the sequence of each word's top-layer representation, followed by a single linear layer which makes the prediction.
For NYTWIT, \mmod{}'s contextualized vector for the target word is used as input for a single logistic layer.
In cases of multi-token words, the prediction from the first token is selected.
We set \tok{}'s character embedding dimension to 200 and the convolutional layers to 256 channels.
Following \newcite{sun2019fine}, we set the maximum learning rate to $10^{-3}$ for the task models and $2\times 10^{-5}$ for fine-tuning, and perform warm-up for 10\% of the total expected training steps before linearly decaying the rates to zero.
All parameters are optimized using Adam~\cite{adam} with default settings.
We run all NER models for twenty epochs and sequence-level task models for three, evaluating on the validation set after each epoch using the metrics reported below, and stopping early if performance has not improved for four epochs.
In order to avoid unfairly favorable conditions for \tokdetok{} models, task hyperparameters are all tuned on the base models, with \tokdetok{} models using only values on which their base equivalents have also been evaluated.

We evaluate the effectiveness of \tokdetok{}'s concepts and components by comparing the following setups:\footnote{A setup where all words are represented by \tok{} obtained noncompetitive results on all tasks.}
\begin{itemize}
    \item \textsc{\textbf{None}} uses only the base model;
    \item \textsc{\textbf{None+\ppt{}}} uses a version of the base model that was further pre-trained on the same Twitter corpus on which \tokdetok{} is trained (adaptive fine-tuning~\cite{ruder2021lmfinetuning}), in order to control for the increase in total unlabeled text seen by the model;
    \item \textsc{\textbf{Scaffolding}} is a model trained with \tokdetok{} in a \ppt{} phase, but only using base model embeddings during task fine-tuning;
    \item \textsc{\textbf{Stochastic}} samples 10\% of the words in the downstream datasets, calling \tok{} on their character sequences while using the base model's embedding(s) for the remaining 90\%;
    \item \textsc{\textbf{All no-suff}} calls \tok{} on all multi-token words which are not of the form \texttt{[token suff]}, where \texttt{suff} is a member of the \textsc{Suffixes} set described in \S\ref{sec:model}, and uses the base model's embedding on the rest.
\end{itemize}

\begin{table}
\scriptsize
\caption{Ranking performance when trained using content-based sources (NYT and Wiki). Significant differences compared to the baselines ([B]M25, [W]T10, [A]OL) are indicated with $\uparrow$ and $\downarrow$ (paired t-test, $p<0.05$).}\label{tab.results}
\vspace{-1em}
\begin{tabular}{llrrrrrr}
\toprule
&&\multicolumn{3}{c}{nDCG@20} \\\cmidrule(lr){3-5}
        Model &   Training & WT12 & WT13 & WT14 \\
\midrule


\multicolumn{2}{l}{BM25 (tuned w/~\cite{Yang2017AnseriniET})} & 0.1087  & 0.2176  & 0.2646 \\
\midrule
PACRR & WT10 & B$\uparrow$ 0.1628  & 0.2513  & 0.2676 \\
 & AOL & 0.1910  & 0.2608  & 0.2802 \\
\greyrule
 & NYT & \bf W$\uparrow$ B$\uparrow$ 0.2135  & \bf A$\uparrow$ W$\uparrow$ B$\uparrow$ 0.2919  & \bf W$\uparrow$ 0.3016 \\
 & Wiki & W$\uparrow$ B$\uparrow$ 0.1955  & A$\uparrow$ B$\uparrow$ 0.2881  & W$\uparrow$ 0.3002 \\
\midrule
Conv-KNRM & WT10 & B$\uparrow$ 0.1580  & 0.2398  & B$\uparrow$ 0.3197 \\
 & AOL & 0.1498  & 0.2155  & 0.2889 \\
\greyrule
 & NYT & \bf A$\uparrow$ B$\uparrow$ 0.1792  & \bf A$\uparrow$ W$\uparrow$ B$\uparrow$ 0.2904  & \bf B$\uparrow$ 0.3215 \\
 & Wiki & 0.1536  & A$\uparrow$ 0.2680  & B$\uparrow$ 0.3206 \\
\midrule
KNRM & WT10 & B$\uparrow$ 0.1764  & \bf 0.2671  & 0.2961 \\
 & AOL & \bf B$\uparrow$ 0.1782  & 0.2648  & \bf 0.2998 \\
\greyrule
 & NYT & W$\downarrow$ 0.1455  & A$\downarrow$ 0.2340  & 0.2865 \\
 & Wiki & A$\downarrow$ W$\downarrow$ 0.1417  & 0.2409  & 0.2959 \\


\bottomrule
\end{tabular}
\vspace{-2em}
\end{table}










\renewcommand{\arraystretch}{1}









We present the results of the downstream prediction tasks in \autoref{tab:main_results}.
All transformer models except for \textsc{None} were 2nd-phase pre-trained three separate times using different random seeds, and the mean results are reported.

The first observation we make is the dominance of RoBERTa, a thoroughly optimized masked language model, over the other models on the NER datasets in all its variants.
This suggests RoBERTa has captured fine-grained information about individual words that it was able to retain in its representations for them; the struggle of the \tokdetok{} model to provide improvement over the \textsc{None} versions of the model strengthens this hypothesis.
GPT-2, despite having access to only left-side context of each word, still outperforms the basic BERT model on most setups in the NER tasks, perhaps due to its larger subword vocabulary size.
At the same time, its gains from the \tok{} representations are much more considerable, suggesting that its left-context-only inference may in fact be detrimental to the its \mmod{}'s performance as a whole.
In general, models perform better on Emerging NER than on Twitter NER, which we attribute to several possible causes or their combination: first, the source shift in the Emerging NER test set from Twitter to Reddit is meant to encumber the models, but given the extensive pre-training they undergo they may actually benefit from the fact that test sequences are longer on average than those in the Twitter NER dataset;
second, more prosaically, the Emerging NER dataset contains fewer entity types (6 vs. 10), making the task itself somewhat easier.

The other word-level task, NYTWIT, demonstrates substantial gains made by the \tokdetok{} training regime: the \textsc{Scaffolding} setup preforms best in all three base models, and in both masked langugage models the \textsc{Stochastic} setup outperforms both \textsc{None} variants.
Together with the NER results, this suggests that \tokdetok{} succeeds in providing transformer models with \textbf{word-level} representations that allow coarse-grained classifications (such as named entity type or novel word origin), better than default subword segmentations, for both edited and user-generated text.

We find that \tokdetok{} is less successful in improving sequence classification and ranking performance than on word-level tasks.
The \textsc{None+\ppt} setup obtains the best results in most models on the Emoji and QA datasets, suggesting the improvements seen in \tokdetok{}-based models is mostly attributable to the domain shift introduced by the Twitter pre-training corpus.\footnote{We note the complete inconsistency in both model performance and comparative model ranking present in the Emoji dataset, which calls back the systemic issue we identified in~\S\ref{sec:tasks}:
The time span over which the test set was collected contained a fundamental shift in Twitter's properties --- doubling the tweet length limit from 140 to 280 characters --- and so exhibits an unpredictable corpus incompatible with the train and dev partitions.
As a result, model performance over the dev set does not predict the test set results (a large difference on performance in this task between dev and test sets was also observed in macro-F1 scores by~\newcite{barbieri-etal-2020-tweeteval}).
In fact, under such specific data shift circumstances, it could be the case that the simpler base model is better equipped for facing longer test data, which scales generalizations made over the training and dev sets, as opposed to \tok{}-augmented models which have more levels of generalization to acquire during training and cannot anticipate the scale change.
Post-hoc inspection of specific model outputs resulted in some more concrete hypotheses for the cause of discrepancy in the major categories of confusion, for example the distributions of \say{@} presence in heart emoji
%{\NotoEmoji\symbol{"2764}}
tweets and heart-eyes emoji
%{\NotoEmoji\symbol{"1F60D}}
tweets shifted to a degree which could explain the models' growing confusion between the two, but no signals accounted for the entire difference in performance and we conclude that the main reason remains the change in sequence length distributions.}
This suggests that \tok{} may be a good learner for word-internal phenomena, picking up structural cues as to their roles within or without context, making it more useful \textbf{locally} than subword tokens' uninformed embeddings, while not being strong enough to provide a better semantic prior for \mmod{} to aggregate together with surrounding well-formed words, making its \textbf{global} utility limited.


\begin{table}
    \centering
    \small
    \begin{tabular}{lrrr}
        \toprule
        Ablation & BERT & GPT2 & RoBERTa \\
        \midrule
        Full & 37.17 & 38.27 & 41.97 \\
        No FT & $-$0.31 & $-$0.25 & $-$0.80 \\
        Wiki \ppt{} & $-$3.11 & $-$2.67 & $-$2.17 \\
        % Wiki+Emb loss &  & $-$4.83 &  \\
        No loops & $+$0.82 & $+$0.26 & $-$1.12 \\
        All multi & $-$1.39 & $+$0.44 & $-$1.81 \\	
        \bottomrule
    \end{tabular}
    \caption{Dev set average effect of model variants compared with the \textit{All no-suff} condition: % on a single random seed:
    \say{No-FT} --- pre-trained model used only for feature extraction;
    \say{Wiki} --- \tokdetok{} trained on Wikipedia data instead of Twitter;
    % \say{Emb loss} --- single-token words in downstream data fed for training \tok{};
    \say{No loops} --- trained without the cycle dependency loops;
    \say{All multi} --- common-suffix words also inferred using \tok{}.}
    \label{tab:ablations}
\end{table}

% \begin{table}
%     \centering
%     \small
%     \begin{tabular}{lrr}
%         \toprule
%         Model / Noise & Random case & Repeat \\
%         \midrule
%         Base & 42.73 & 43.57 \\
%         Base + \ppt{} & 43.54 & 45.07 \\
%         Scaffolding & 43.15 & 44.31 \\
%         Stochastic & 42.54 & 43.92 \\
%         All-no-suff & 41.72 & 42.33 \\
%         \bottomrule
%     \end{tabular}
%     \caption{Noising experiments on RoBERTa models on the QA dataset.}
%     \label{tab:ablations}
% \end{table}




\subsection{Ablations}
\label{ssec:abl}
We compare the dev set results of the \textsc{All no-suff} condition on several modified versions of the model, presented in Table~\ref{tab:ablations}.
First, we find that fine-tuning \mmod{} and \tokdetok{} parameters during downstream task application is beneficial for results across base models, indicating susceptibility of \tok{}'s network to tune itself on task data and not solely on LM and the vectorization signal.
Next, and most substantially, we note the vast improvement of Twitter-trained models compared with \ppt{} performed over a Wikipedia corpus with comparable size;
even though some tasks are on edited text, the overall effect of domain change during second pre-training is apparent (and, indeed, least impactful on the NYTWIT task which features the best-edited text).
Other decisions made during the \ppt{} phase appear to be less decisive: removing the dependency loops helps performance on BERT and GPT2, but makes a large dent in the best-performing RoBERTa, indicating potential gains to be made by applying \detok{} in generative tasks not pursued within our scope;
the suffix-based dialing down of inference in pre-training helps the masked models but hurts GPT2 performance, possibly because its autoregressive application prohibits it from looking at a simply-inflected word's suffix when processing its stem, a problem not incurred in masked modeling.




\section{Related Work}
\label{sec:related}

CharBERT~\cite{ma-etal-2020-charbert} is a method which incorporates character-level encodings in \llm{}s, while re-designing the transformer stack to include a \say{character channel} distinct from the parallel token channel.
Char2Subword~\cite{aguilar2020char2subword} integrates multiple losses in a system trained to predict subword tokens from the character level, all originating in distance metrics between target and prediction.
CharacterBERT~\cite{el-boukkouri-etal-2020-characterbert} uses an ELMo-style character convolutional network to encode input into a transformer MLM.
CANINE~\cite{clark2021canine} is a method for training \llm{}s which removes the need for a subword tokenizer, by utilizing character-level representations which are pooled into inputs for the main transformer module.
ByT5~\cite{xue2021byt5} ventures yet deeper by training a T5-architecture model~\cite{raffel2020exploring} over byte representations, reducing the required embedding table size by one more order of magnitude.
Unlike \tokdetok{}, these systems all require training an \llm{} from scratch and do not support using a \ppt{} step for tuning existing large models.
In addition, with the exception of CANINE, they all resort to softmax prediction over a subword vocabulary for the generative portion of the pre-training phase, and do not offer a character-level decoder which can also be tied to the encoding component.





\section{Conclusion}
We have presented a novel tokenization scheme for Vision Transformers, replacing the standard uniform patch grid with a mixed-resolution sequence of tokens, where each token represents a patch of arbitrary size. We integrated the Quadtree algorithm with a novel feature-based saliency scorer to create mixed-resolution patch mosaics, making this work the first to use the Quadtree representations of RGB images as inputs for a neural network.

Through experiments in image classification, we have shown the capacity of standard Vision Transformer models to adapt to mixed-resolution tokenization via fine-tuning. Our Quadformer models achieve substantial accuracy gains compared to vanilla ViTs when controlling for the number of patches or GMACs. Although we do not use dedicated tools for accelerated inference, Quadformers also show gains when controlling for inference speed.

We believe that future work could successfully apply mixed-resolution ViTs to other computer vision tasks, especially those that involve large images with heterogeneous information densities, including tasks with dense outputs such as image generation and segmentation.



\bibliography{anthology,tdt}
\bibliographystyle{acl_natbib}

\clearpage

% \appendix

% \input{99_appendix}

\end{document}
