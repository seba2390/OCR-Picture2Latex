\documentclass[11pt,a4paper]{article}
\usepackage[hyperref]{tacl2018v2}

\usepackage[T1]{fontenc}
\usepackage{times}
\usepackage{plex-mono}

\usepackage{latexsym}
\usepackage{amsmath,amssymb}
\usepackage{dsfont}

\usepackage{algorithm}
\usepackage{algpseudocode}


%%%% table stuff
\usepackage{array,etoolbox}
\preto\tabular{\setcounter{magicrownumbers}{0}}
\newcounter{magicrownumbers}
\def\rownumber{}

%%%% graphics
\usepackage{tikz}
\usetikzlibrary{arrows,matrix,positioning,fit,calc,shapes,decorations.pathreplacing,svg.path,shadows,patterns}
\pgfdeclarelayer{background}
\pgfdeclarelayer{backgroundoverlay}
\pgfsetlayers{background,backgroundoverlay,main}
\usepackage{pgfplots}
\pgfplotsset{compat=1.13}
\usepackage{adjustbox}
\usepackage{ocr}
\newcommand{\circnum}[1]{\raisebox{.5pt}{\textcircled{\raisebox{-.5pt}{\scalebox{.8}{\(#1\)}}}}}
\newcommand{\ofw}[1]{\ensuremath{{#1}\raisebox{.06em}{\scalebox{.7}{$(w)$}}}}
\newcommand{\charword}[1]{\scalebox{.9}{\ocrfamily #1}}
\newcommand{\spelling}{s}
\newcommand{\wughighlight}[1]{{\ocrfamily\color{black} #1}}
%%%% end of graphics

\renewcommand{\UrlFont}{\ttfamily\small}

\usepackage{dirtytalk}
\usepackage{microtype}

\usepackage{booktabs}
\usepackage{arydshln}
\usepackage{multirow}
\usepackage{array}
\newcolumntype{H}{>{\setbox0=\hbox\bgroup}c<{\egroup}@{}}

\usepackage{bm}

\newcommand{\tokdetok}[0]{\textsc{XRayEmb}}
\newcommand{\tok}[0]{\textsc{XR-Enc}}
\newcommand{\detok}[0]{\textsc{XR-Dec}}
\newcommand{\tdloop}[0]{E$\rightarrow$D}
\newcommand{\dtloop}[0]{D$\rightarrow$E}
% \newcommand{\tokdetok}[0]{\textsc{TokDetok}}
% \newcommand{\tok}[0]{\textsc{Tok}}
% \newcommand{\detok}[0]{\textsc{Detok}}
\newcommand{\marco}[0]{\textsc{MarcoQA}}
\newcommand{\unilm}[0]{Unigram LM}
\newcommand{\llm}[0]{LPLM}
\newcommand{\wpc}[0]{WordPiece}
\newcommand{\ppt}[0]{\textsc{2pt}}
\newcommand{\mmod}[0]{\textsc{$M$}}
\newcommand{\gen}[0]{\textsc{Pred}}
\DeclareMathOperator*{\agg}{agg}

\renewcommand{\vec}[1]{\bm{#1}}
\newcommand{\tribracket}[1]{\textless#1\textgreater}
\newcommand{\vE}{\ensuremath \vec{E}}
\newcommand{\vI}{\ensuremath \vec{I}}
\newcommand{\vc}{\ensuremath \vec{c}}
\newcommand{\vh}{\ensuremath \vec{h}}
\newcommand{\vb}{\ensuremath \vec{b}}
\newcommand{\ve}{\ensuremath \vec{e}}
\newcommand{\vr}{\ensuremath \vec{r}}
\newcommand{\vw}{\ensuremath \vec{w}}
\newcommand{\vx}{\ensuremath \vec{x}}
\newcommand{\vy}{\ensuremath \vec{y}}
\newcommand{\vf}{\ensuremath \vec{f}}


\taclpubformattrue % Uncomment this line to deanonymize

\setlength\titlebox{8cm} % comment out for review version

\title{Learning to Look Inside: Augmenting Token-Based Encoders with Character-Level Information}

\author{Yuval Pinter\thanks{Work done as an intern at Bloomberg LP and as a Bloomberg PhD Data Science Fellow.} \\
  School of Interactive Computing \\
  Georgia Institute of Technology \\
  Atlanta, GA, USA \\
  \texttt{uvp@gatech.edu} \\\And
  Amanda Stent \\
  Bloomberg \\
  New York, NY, USA \\
  \texttt{astent@bloomberg.net} \\\AND 
  Mark Dredze \\
  Bloomberg \\
  Department of Computer Science \\
  Johns Hopkins University \\
  \texttt{mdredze@cs.jhu.edu} \\\And
  Jacob Eisenstein \\
  Google Research \\
  \texttt{jeisenstein@google.com} \\}

\date{}

\begin{document}
\maketitle
\begin{abstract}
    Commonly-used transformer language models depend on a tokenization schema which sets an unchangeable subword vocabulary prior to pre-training, destined to be applied to all downstream tasks regardless of domain shift, novel word formations, or other sources of vocabulary mismatch.
    Recent work has shown that ``token-free'' models can be trained directly on characters or bytes, but training these models from scratch requires substantial computational resources, and this implies discarding the many domain-specific models that were trained on tokens. In this paper, we present \tokdetok{}, a method for retrofitting existing token-based models with character-level information.  
    \tokdetok{} is composed of a character-level ``encoder'' that computes vector representations of character sequences, and a generative component that decodes from the internal representation to a character sequence.
    We show that incorporating \tokdetok{}'s learned vectors into sequences of pre-trained token embeddings helps performance on both autoregressive and masked pre-trained transformer architectures and on both sequence-level and sequence tagging tasks, particularly on non-standard English text.
\end{abstract}


\section{Introduction}
Federated learning (FL)~\citep{mcmahan2017fedavg} represents a significant advancement in machine learning, emphasizing decentralized training while preserving data privacy. FL enhances data privacy and collaboration compared to traditional machine learning by enabling model training across multitudes of decentralized devices without direct data sharing. However, challenges like non-identical independent distribution (non-IID) data and heterogeneous computational resources among devices present potential training failures.


Concurrently, transformer-based foundation models (FMs) \citep{Bommasani2021FoundationModels}, typified by GPT~\citep{radford2019gpt2,brown2020gpt3,openai2023gpt4}, BERT~\citep{kenton2019bert}, and ViT~\citep{dosovitskiy2020vit}, pre-trained on large-scale datasets, have revolutionized AI research. 
FMs leverage their inherent pre-trained knowledge, achieving exceptional performance across multiple domains in downstream tasks even with limited fine-tuning data.


Given the superior strengths of FMs in few-shot transfer learning, they appear well-suited for non-IID FL environments~\citep{yu2023ffm,zhuang2023foundationfl}. However, seamlessly integrating FMs into FL presents significant challenges. The substantial size and intensive resource demands of FMs make their deployment on resource-constrained FL edge devices problematic. Furthermore, the uneven distribution of computational resources within FL increases the difficulty of existing challenges. A resource-limited device must first satisfy the resource requirements for FM optimization, despite the presence of more capable devices within the same FL network, leading to high system requirements overall.
Additionally, fine-tuning FMs typically requires approximately seven times the resources compared to inference. This disparity means that FL often faces resource-hungry during model training while leaving resources underutilized during inference.

We propose a framework, Resource-aware Federated Foundation Models (RaFFM), to address the resource utilization challenges in FL. RaFFM uses specialized transformer-based FM compression algorithms tailored for FL-edge environments, and dynamically deploys resource-aware scaled FMs to local clients. 
Since FMs are over-parameterized, a subset of salient parameters is more impactful to model performance. RaFFM first identifies and prioritizes the salient parameters in the given FM before FL communication starts. The salient parameter prioritization can ease the resource-aware sub-model extractions from FMs and model fusion in FL global updates via parallel matrix operations.
Then, RaFFM generates resource-aware sub-models with salient weights so clients can proceed with the FL fine-tuning cycle.
Post-training, given that model inference is more resource-friendly than training, RaFFM can deploy larger models, ensuring optimized performance and consistent resource allocation based on the client's capabilities.

In essence, RaFFM brings forth the following key contributions:
\begin{itemize}
    \item Designed specialized FM compression algorithms tailored for edge FL.
    \item Proposed specialized salient parameter prioritization strategy for transformer-based FMs.
    \item Enhanced resource utilization throughout the training and deployment stages of FL.
    \item Significant reduction in communication overhead between edge devices and central servers.
\end{itemize}

% \begin{itemize}
%     \item Introducing specialized FM compression algorithms ideal for edge FL environments.
%     \item Proposing a unique salient parameter prioritization strategy tailored for transformer-based FMs.
%     \item Optimizing resource utilization across FL's training and deployment phases.
%     \item Achieving a substantial reduction in communication overhead between edge devices and central servers.
% \end{itemize}






\section{\tokdetok}
\label{sec:model}

We focus on character-informed representations for \llm{}s which use the transformer architecture~\cite{vaswani2017attention}, such as BERT~\cite{devlin-etal-2019-bert}, RoBERTa~\cite{liu2019roberta}, 
and GPT~\cite{radford2018improving}.
A Transformer \llm{} is composed around a core module \mmod{} parameterized as multi-head self-attention layers, which accepts a sequence $\{\ve_1, \dots, \ve_l\}$ of embedding vectors corresponding to a list of token indices $\{t_1, \dots, t_l\}$ from a vocabulary $\mathcal{V}$,
and outputs a sequence of contextualized vectors $\{\vh_1, \dots, \vh_l\}$, of which a subset $\{\vh_i\}_{i\in\mathcal{I}}$ correspond to masked tokens in positions $\mathcal{I}\subseteq [l]$ selected by a boolean masking operator $m$.\footnote{In the case of an autoregressive \llm{} like GPT, $\mathcal{I} = [l]$ but each vector $\vh_i$ is only dependent on the tokens preceding its position, $\{t_1, \dots, t_{i-1}\}$, and token-wise prediction is performed for each position assuming no knowledge of future tokens.}
The token index list supplied to \mmod{} is the output of a tokenization function $\tau$ operating on a sequence $X$ followed by lookup in an embedding table $\vE$,
while its output vectors $\{\vh_1, \dots, \vh_l\}$ serve as input to a prediction module \gen{}, which outputs a distribution $D_i$ over $\mathcal{V}$ for each masked position $i$.
All embeddings $\{\ve_i\},\{\vh_i\}$ live in a shared space $\mathds{R}^d$.
Together, the components described so far operate in the following manner:

\begin{equation}
    \{D_i(\mathcal{V})\}_{i\in\{\mathcal{I}\}} = 
    \text{\gen{}}\left( \text{\mmod{}} \left( \vE\left[ \tau(X) \right] \right) \right),
\end{equation}
where square bracketing denotes elementwise table lookup.

In the considered \llm{} architectures, the input $X$, which is atomically made up of a sequence of characters $\{c_1, \dots, c_n\}\in\Sigma^{n}$, is broken down by $\tau$ to provide the tokenization of length $l\leq n$,
and the prediction/generation operator \gen{} accepts the contextualized outputs $\{\vh_1, \dots, \vh_l\}$ and implements prediction by means of a softmax distribution which is based on scores obtained via dot-product against an output embedding table, which is usually the same as $\vE$.

The \tokdetok{} model makes no adjustments to \mmod{} itself, and only offers conditional replacements for $\tau$ and \gen{}.
An \textbf{encoding policy} $\pi^t$ selects a subset of tokens from the original sequence $\mathcal{J}^t\subseteq [l]$ to be represented by \tok{} instead of $\vE \circ \tau$.
\tok{} has access to the part of the character sequence $X$ which underly the tokens selected by $\pi^t$, and produces input embeddings directly from the character level.
These alternate embeddings must agree in dimension with those of each $\ve$, but a single embedding may be used to replace multiple base tokens (usually when all tokens corresponding to a single out-of-$\vE$ word are replaced), resulting in a shorter input sequence for \mmod{}.
A separate \textbf{generation policy} $\pi^g$ selects a subset of the original sequence $\mathcal{J}^g\subseteq [l]$ to be generated from the output vectors $\vh$ by \detok{} instead of \gen{}.
A high-level schematic depicting this framework is presented in \autoref{fig:algo}.

\begin{table*}
    \centering
    \tiny
    \begin{tabular}{ll}
        \toprule
        % Original & \texttt{Approx~~~~~~~~~~distance in miles from Bruges~~~~~~~to Amsterdam is 107 miles or 172.16 KMS~~~~.} \\
        % Tokenization & \texttt{A \#\#pp \#\#ro \#\#x distance in miles from B \#\#rug \#\#es to Amsterdam is 107 miles or 172 . 16 K \#\#MS .}  \\
        {\small Original} & \texttt{He was emphatically~~~~~a modern gentleman,~~of scrupulous~~~~~~~courtesy,~~sportive~~~gaiety,} \\ % acquainted with what was going on in the world.} \\
        {\small Word pieces} & \texttt{He was em \#pha \#tically a modern gentleman , of s \#c \#rup \#ulous courtesy , sport \#ive g \#ai \#ety ,} \\
        \midrule
        % \multirow{2}{*}{\small Random-20\%} & {\Monospace .9~.75~.15~~~~~~~~~~~~~.3~.05~~~~.55~~~~~~.6~.4~.7~~~~~~~~~~~~~~~.1~~~~~~.8~.25~~~~~~~~.45~~~~~~~.3} \\
        {\small Random-20\%} & \texttt{He was [TOK]~~~~~~~~~~~~a [TOK]~~gentleman , of s \#c \#rup \#ulous~[TOK]~~~~, sport \#ive g \#ai \#ety ,} \\
        {\small All-multi} & \texttt{He was [TOK]~~~~~~~~~~~~a modern gentleman , of [TOK]~~~~~~~~~~~~courtesy , [TOK]~~~~~~[TOK]~~~~~~,} \\
        {\small \textsc{Suffixes}} & \texttt{He was [TOK]~~~~~~~~~~~~a modern gentleman , of [TOK]~~~~~~~~~~~~courtesy , sport \#ive [TOK]~~~~~~,} \\
        \bottomrule
    \end{tabular}
    \caption{Example of token input selection by different policies,
    % (randomly-generated numbers determine the random policy) 
    where \texttt{[TOK]} signifies a word to be replaced by its character-based representation from \tok{}.
    Sentence fragment taken from the MS-MARCO QA dataset (see \S\ref{sec:tasks}) and tokenized using BERT-cased (\#\# replaced with \# to fit paper width).}
    \label{tab:policies}
\end{table*}



The specific policies in a given application may be defined based on the model's use case.
For example, in text classification no generation is required, and so $\pi^g$ will return $\emptyset$ for all sequences; $\pi^t$ can be tuned for a task based on known features of the base model (BERT/GPT etc.) and of the domain text, some examples including:
tokens corresponding to all words that are not in a pre-determined vocabulary;
all words in the sequence;
all words assigned more than one token by the base tokenizer $\tau$;
a random sample of words in the sequence;
or all words including characters that are not lowercase English characters.
One particular policy we hypothesize could be useful is one that affords the tokenizer slack in detecting a single simple derivational or inflectional suffix: all words which are single-token or whose second-and-final token is in the list \textsc{Suffixes} are left for $\vE \circ \tau$; the rest are represented using \tok.\footnote{\textsc{Suffixes}, compiled by manually examining a list of most common second-and-final tokens in a large corpus under GPT-2's tokenization, = $\{$\emph{s}, \emph{ed}, \emph{es}, \emph{ing}, \emph{ly}, \emph{al}, \emph{ally}, \emph{'m}, \emph{'re}, \emph{'ve}, \emph{y}, \emph{ive}, \emph{er}, \emph{'t}, \emph{'ll}, \emph{an}, \emph{ers}$\}$.
Similar policies can be hand-crafted for different languages, based on learner-level knowledge of the language and minimal preliminary analysis of sample tokenizations. Our results show they are not strictly necessary.}
Different policies may be applied in training settings as well, for example in order to \say{familiarize} the heavily-parametrized \mmod{} with the inputs from \tok{}.
Three example policies are illustrated in~\autoref{tab:policies}.



\subsection{Second Pre-training}
\label{ssec:pretr}

A typical \llm{} is initiated through a computationally-intensive pre-training step, iterating over a large corpus in batches of sequences and backpropagating a cross-entropy loss calculated over the prediction layer's output into all of its components, through \gen{} to \mmod{} to $\vE$.
In order to train \tokdetok{}, we introduce a second pre-training step we term \ppt{}, where the \llm{} continues to update its parameters for a (possibly different) corpus, but is supplemented with the \tokdetok{} elements in order to \say{acclimatize} the \mmod{} components to outputs from \tok{}.\footnote{Within the taxonomy of phases between pre-training and task training~\cite{ruder2021lmfinetuning}, this is closest to \textbf{Adaptive Fine-Tuning}. However, due to our added modules we opt to assign it a new name.}
In addition, \tok{} is also trained through a lower-level objective requiring it to approximate the outputs of $\vE \circ \tau$ which it is replacing, and \detok{} is trained to sequentially produce the correct character sequence from \mmod{}'s outputs.
Together, a batch of text sequences in a \ppt{} step produces the following loss elements which are backpropagated into the unified model:
\begin{itemize}
    \item A \textbf{language modeling loss} from the softmax operation over masked tokens, updating \mmod{}'s and $\vE$'s parameters, as well as \tok{}'s for tokens selected by a usage policy $\pi^t_{(u)}$ to be used in \mmod{}'s input;
    \item An \textbf{embedding loss} for the \tok{} component, computed against $\vE \circ \tau$'s token embeddings, over a set selected by a policy $\pi^t_{(l)}$ which may or may not equal $\pi^t_{(u)}$.
    This loss can be computed, e.g., as the euclidean distance between the output and the target.
    When the target corresponds to multiple token counts, an aggregation pooling function $\agg:\mathds{R}^{d \times \mathds{N}^{+}}\rightarrow\mathds{R}^d$ needs to be defined over the embeddings, for example taking their dimension-wise mean, taking the leftmost token's embedding, or taking the dimension-wise $\max$;
    \item A \textbf{generation loss} for words generated by \detok{} from \mmod{}'s output vectors, according to a $\pi^g$ policy.
    This loss is the character-level cross-entropy for autoregressive sequence generation.
    Note that in order to only generate full words, $\pi^g$ must align with $\pi^t_{(u)}$ so that no multi-token words left as input to \mmod{} are also selected for generation.
\end{itemize}

As a form of regularization within the \tokdetok{} components, we introduce additional training batches we call \textbf{cycle dependency loops}.
In such a batch, \tok{} and \detok{} act in succession, starting from one of the spaces they operate in, with the goal of arriving at the same point after cycling through both components.
An \textbf{\tdloop} loop thus starts at a character sequence $\vc\in{\Sigma^*}$, runs it through \tok{} to obtain a vector $\ve=\tok(\vc)$, and runs \detok{} in an attempt to return to the original sequence $\hat{\vc}=\detok(\tok(\vc))\approx \vc$.
Analogously, a \textbf{\dtloop} loop starts at a vector $\tilde{\ve}\in\mathds{R}^{d}$ and targets $\hat{\ve}=\tok(\detok(\tilde{\ve}))\approx \tilde{\ve}$.
In this loop, loss is only backpropagated as far as \tok{}, since backpropagating through a generative model's decision component introduces discrete steps which must be smoothed or approximated (see discussion in \newcite{peng-etal-2018-backpropagating}).




\section{Tasks}
\label{sec:tasks}

To evaluate the advantages of character-sequence awareness in large-scale transformers, we chose a diverse set of datasets which reflect unedited user-generated language in English, as well as its interaction with edited text.
We report results on a sequence classification task (emoji prediction), a sequence tagging task (named entity recognition, or NER) in both an in-domain (Twitter NER) and cross-domain (emerging entities NER) setting, a sequence ranking task based on information retrieval in a hybrid edited-unedited textual setting (MARCO-QA ranking), and a task where a single word's class is predicted within a sequence (NYTWIT).

\paragraph{Emoji prediction.}
The English portion of the Multilingual Emoji Prediction dataset \cite{barbieri-etal-2018-semeval,ma-etal-2020-multi} is composed of tweets containing one of the twenty most common emoji symbols.
For the task, the emoji are stripped from the tweet text and the system is asked to predict which one appeared in the original tweet, allowing it to be construed as a self-annotated, fine-grained sentiment analysis task.
All tweets are identified as geographically originating in the US.
We note that there is a significant qualitative difference between the training (+ development) set, and the test set of this corpus, hinted at by one of the participant teams in the original task~\cite{chen-etal-2018-peperomia} but not explored.\footnote{Another team identified a seasonal shift affecting the distribution of the Christmas tree 
% {\NotoEmoji\symbol{"1F384}} 
emoji~\cite{coster-etal-2018-hatching}.}
The partitions were extracted based on a temporal split, with all test set tweets post-dating all training set tweets by at least three months. %(February 2017 -- May 2017).
More crucially, the test set dates (May 2017 -- January 2018) overlap with Twitter's increase of the tweet character limit from 140 to 280 characters, phased in mostly during November 2017.
As a result, test set tweets have on average $\sim$10\% more words, increasing the amount of information within and diverging their textual feature distribution from that of the training set by more than is customary in NLP tasks.
The label distribution remains more or less the same.

\paragraph{Twitter NER.}
The Twitter NER dataset \cite{strauss-etal-2016-results} is a prime example of sequence tagging in noisy user-generated data settings.
It is composed of randomly sampled English tweets from 2016, annotated for ten different entity types, including music artists and sports teams, thus representing topics prevalent in social media.

\paragraph{Emerging entities recognition.}
The emerging entities dataset \cite{derczynski-etal-2017-results} shares most of its training set with the same Twitter source as the Twitter NER dataset, but as a domain-adaptation setup it includes a more ambitious evaluation fold: the development set is extracted from YouTube comments, and the test set from StackExchange and Reddit.
The taskmasters claim that the dataset contains mostly rare and hitherto-unseen entities, but do not provide exact statistics.

\paragraph{Community question answering.}
The MS-MARCO question answering dataset~\cite[\marco;][]{nguyen2016ms} was collected by mining a commercial search engine log for user queries and asking humans to answer them, supplying the answer writers with a set of passages retrieved automatically from edited text by the search engine which the answer writers then marked as \say{selected} if they helped them formulate the answer.
We recast this dataset into a selection / ranking problem, not pursued in the original challenge: given the query and the set of possibly helpful passages, which is the passage the answer writer selected?
To this end, we filtered out queries with more or fewer than one selected passage, and evaluated each system based on the mean reciprocal ranks (MRR) of the true selected passages in the rankings it produced.
Due to its size, we uniformly sampled 10\% of the queries and associated passage collections from each partition in this dataset, and ran all analyses and experiments on this new sampled dataset.

\paragraph{Novel word classification.}
The NYTWIT dataset~\cite{pinter-etal-2020-nytwit} includes passages in the New York Times where a word appears which has not appeared in the publication before.
The system is tasked to classify the novel word into one of eighteen types of novelty sources, such as \say{inflection of known word} or \say{lexical blend}.
The dataset is not partitioned into train/dev/test, and so we use the 10-fold partition from \newcite{pinter-etal-2020-nytwit} and report accuracy results aggregated over all instances, each from a model trained on the other nine folds.


\begin{table*}
    \centering
    \small
    \begin{tabular}{lcccccc} \toprule
        Dataset & Instances & Tokens & Types & TTR & Multitoks & Token mass \\
         & & \multicolumn{2}{c}{(Space-delimited)} & & & increase \\
        \midrule
        CoNLL-2003 NER & 14,986 & 204,563 & 23,624 & .115 & 16.08\% & 29.50\% \\
        Wiki1 & --- & 204,564 & 28,092 & .137 & 8.64\% & 12.85\% \\
        \midrule
        NYTWIT & 1,903 & 94,403 & 26,028 & .276 & 24.04\% & 38.04\% \\
        Wiki2 & --- & 94,403 & 16,903 & .179 & 9.11\% & 13.75\% \\
        \midrule
        Twitter NER & 2,394 & 46,469 & 10,586 & .228 & 17.23\% & 43.21\% \\
        Wiki3 & --- & 46,509 & 10,855 & .233 & 9.02\% & 13.77\% \\
        \midrule
        Emerging NER & 3,394 & 62,730 & 14,878 & .237 & 19.15\% & 54.25\% \\
        Wiki4 & --- & 62,757 & 13,392 & .213 & 9.05\% & 13.74\% \\
        \midrule
        Emoji Prediction & 427,458 & 4,973,813 & 504,644 & .101 & 32.96\% & 64.90\% \\ % test set: 21.53\% & 63.49\%
        Wiki5 & --- & 4,973,813 & 194,240 & .039 & 9.07\% & 13.58\% \\ % 95.52\% of types are multitoken! % prev version had 440,689 types, idk why
        \midrule
        \marco{} (10\% sample) & 80,704 & 46,956,674 & 1,355,049 & .029 & 20.58\% & 31.91\% \\ % 97.78\% of types are multitoken!
        Wiki6 & --- & 46,956,703 & 818,712 & .017 & 9.48\% & 14.37\% \\
        \bottomrule
    \end{tabular}
    \caption{Surface statistics for the datasets used for evaluation (training sets), against Wikipedia text of comparable token count sizes. Tokenization performed with GPT-2.}
    \label{tab:stats}
\end{table*}



\subsection{Analysis}
\label{ssec:stats}

We begin with an analysis of the datasets and their subword properties in order to gauge the direct effects of tokenization on the dataset's genre and domain.
In \autoref{tab:stats}, we present surface-level statistics reflecting the challenges posed by the datasets' sources, comparing each task's training set with comparably-sized sets of sentences sampled uniformly from English Wikipedia.
We use the GPT-2 tokenizer, which boasts the largest subword vocabulary of all models considered in this work, to obtain a lower bound on the added token mass expected by our models on these datasets, and report the following measures: the number of unique word types and type/word ratio, the percent of types which are subword-tokenized by GPT-2 (\textit{Multitoks}), and the overall increase in number of tokens over the corpus compared to a single-token-per-word representation (i.e., strictest application of \tokdetok{} considered in this work).

We find a striking disparity between the well-edited Wikipedia corpora, themselves far from being completely in-vocabulary, and the datasets at hand.
Wikipedia itself proves to be scale-invariant on the metrics that are not TTR, maintaining a token-OOV rate of roughly 9\% and a tokenization overhead of $\sim$13\%.\footnote{This number is very close to the one found by \newcite{acs2021subword} using multilingual transformer models; other languages' corpora boast overheads ranging from 28\% (French) to 95\% (Japanese).}
A control dataset in the form of CoNLL-2003's English NER portion~\cite{tjong-kim-sang-de-meulder-2003-introduction}, extracted from newswire text, exhibits a marked increase in complex words, possibly mostly named entities, or time-sensitive terms compared to the tokenizer's training set, alongside a decrease in type count which reflects its narrow source domain.
NYTWIT, despite also being extracted from newswire text, is sourced from news items that contain novel forms and so are often written in a high register or involve niche domains; as a result, it contains a larger overall token mass, split rather evenly across individual words (i.e., unknown words have fairly \say{regular} structure).
The Twitter NER datasets, exhibiting tweet language, do not exceed CoNLL's multitoken type proportion by much, but its OOVs tend to be completely unexpected forms, leading to a much higher raw post-tokenization count.
In the emoji dataset, which has not been pre-processed according to NER standards and instead was directly scraped off Twitter, almost a third of all unique forms are multi-token, and their presence enlarges the total token count by nearly two thirds.
\marco{} data, most of which is text from highly-ranked web pages, but also including user-generated queries, assumes a middle position between these two extremes.







\section{Experiments}
\label{sec:exp}


We evaluate the effect of \tokdetok{} when included in various \llm{}s. 
We choose \textsc{BERT-base-cased}~\cite{devlin-etal-2019-bert}, \textsc{GPT2-small}~\cite{radford2019language}, and \textsc{RoBERTa-base}~\cite{liu2019roberta} as the base \llm{}s to be manipulated.\footnote{Models loaded from the Huggingface repository~\cite{wolf-etal-2020-transformers}.}
The former contains roughly 108M parameters, and the latter two roughly 125M, a difference accounted for by their larger subword vocabulary (50k vs. 29k) and the resulting larger embedding table.
All models are case sensitive, but they differ in their strategy for preserving the original space-delimited word sequence: BERT's tokenizer marks word-non-initial tokens with a \say{\#\#} string, while GPT-2's tokenizer marks spaces with a special underline character and appends them to the following word.
The difference manifests itself in sequence-initial words, whose initial token in the GPT-2 representation shares the form of sequence-medial word-medial tokens, rather than that of sequence-medial word-initial tokens; and in symbols pre-tokenized without a preceding space, such as punctuation and apostrophes.
RoBERTa's tokenizer adopts the GPT-2 marking strategy but avoids the first pitfall by internally prepending all input sequences with a space character.
We perform the \ppt{} phase for each model on a collection of English tweets from 2016 obtained from the Firehose and preprocessed to replace all `@'-mentions with \texttt{@user}.
We later ablate this domain change effect by training models with \ppt{} text from the English Wikipedia March 2019 dump (see \S\ref{ssec:abl}).
We sample both resources to create pre-training corpora of roughly 725MB (unzipped), several orders of magnitude smaller than what contemporary models use for the first pre-training phase, and train for a single epoch.

In preliminary experiments on several character-level \tok{} architectures, we found that a convolutional net outperforms bidirectional LSTMs and small transformer stacks.
We pass the input characters through three separate convolution layers of width 2, 3, and 4 (characters), then pass the outputs through max-pooling layers and a \texttt{ReLU} activation, and finally project the concatenation of the results onto the base models' embedding dimension.
% \todo{Add figure if feeling masochistic.}
This \tok{} component contains $\sim$1M parameters, negligible compared to the transformers' parameter count.\footnote{The Wikipedia-trained models contain $\sim$2.8M parameters, the difference owing to language-ID filtering performed on the Twitter data, leading to a much smaller character set.} % \detok{}, which uses a 2-layer MLP with hidden layer same-dim as output dim, has 81M params for Wiki and 540K params for Twitter (but we don't use it in our tasks).
% exact numbers: 1071760, 2819360, 81470895, 539539
We implement \detok{} as a 2-layer unidirectional LSTM whose hidden layer is initialized by projecting the context vector $\vh_i$ output from \mmod{} into the hidden dimension.
Characters are generated by projecting the LSTM's output through two linear layers with a \texttt{tanh} activation.

During \ppt{}, we insert a cycle dependency batch every 5,000 LM steps.
For the replacement policies we choose $\pi^t_{(l)}$ to sample uniformly random sets of tokens representing 15\% of space-delimited words to pass as a target loss
for \tok{}, and $\pi^t_{(u)}$ to replace the embedding input to \mmod{} with \tok{}'s output for all multi-token words.
$\pi^g$ selects all tokens to pass as target losses for \detok{}, calculated as a sum of the cross-entropy loss for each character in the target sequence.
Cycle batches consist of sampling $k$ words out of the $K$ most frequent words from the training corpus, with replacement, and $k$ vectors from a Gaussian distribution blown to concentrate around the surface of the unit sphere in hidden-dimension space:
\[ \tilde{\ve} \sim \frac{1}{\sqrt{d}}\cdot\mathcal{N}(0,\vI^{(d)}). \]
\tdloop{} loops are optimized for a
character-level cross-entropy
loss, whereas \dtloop{} loops target a
euclidean distance loss.
When a \tok{} embedding correspond to multiple $\tau$ tokens, the learning target is created by max-pooling their embeddings.\footnote{This $\agg$ function outperformed average-pooling and first-token selection in preliminary experiments.}
We set $K=$25,000 and $k=$1,000.


\subsection{Intrinsic Assessment}

Across all base models and both \ppt{} corpora selected for our experiments, we observed a steady decrease in the \llm{} models' built-in loss metrics (masked prediction / autoregressive prediction) until stabilizing at roughly half the initial value before the end of the \ppt{} epoch.
This indicated that the transformer layers are able to process inputs from both the embedding table and \tok{} and reconcile them.
\autoref{fig:updates} depicts the parameter updates in RoBERTa by parameter type, across layers, comparing parameter values before and after the \ppt{} phase on Twitter data.
It shows that the change along the model layers is fairly stable, with mildly more extensive updates in the bottom and top layers.
The former is to be expected given the introduction of inputs from \tok{}; the latter can also be influenced by encountering Twitter data, which is substantially different than what RoBERTa is \say{used to}.

\begin{figure}
    \centering
    \includegraphics[width=8.25cm]{figures/layer_updates.png}
    \caption{Euclidean distance between RoBERTa weight parameter values before and after a \ppt{} training phase on the Twitter corpus.}
    \label{fig:updates}
\end{figure}

\begin{table*}
    \centering
    \footnotesize
    \begin{tabular}{lHcccccc}
        \toprule
        Model & Uncased BERT & BERT & \multicolumn{2}{c}{GPT-2} & \multicolumn{2}{c}{RoBERTa} \\ % 202102052207 & 202103171733 & 202104091857 & 202102260408 & 202103192237 & 202103192237 \\
        Corpus & Wikipedia & Wikipedia & Wikipedia & Twitter & Twitter & Twitter \\
        Steps & 76,000 & 9,000 & 13,500 & 21,000 & 7,500 & 57,500 \\
        Sequences ($10^3$) & 3,648 & 5,184 & 7,776 & 10,080 & 2,160 & 16,560 \\
         % 48x & 576x & 576x & 480x & 288x & 288x \\
         \midrule
        & ppercented & proming & crordman & d & orereren & everyone \\
        & or & dy & sssion & . & ant & kerned \\
        & ter & deded & gental & the & re & levernger \\
        & peprepored & terse & 2 & @ & cerent & and \\
        & essed & h & ther & ==666!!!!!!!!!!!!!!! & ennte & ed \\
         \bottomrule
    \end{tabular}
    \caption{Example generated words from random locations near the surface of the unit sphere in $\mathds{R}^{768}$.}
    \label{tab:gen}
\end{table*}


% More, from RoBERTa: Wiki-5\% after step 6,300 (576x):
% .<e> reenter<e> rencement<e> o<e> e<e> ĵ<e> ,<e> reciention<e> underesting<e> and<e>

% RoBERTa 13,000 Twitter-7.5\% (576x): reverally<e>	the<e>	revering<e>	.<e>	prot<e>	handers<e>	a<e>	there<e>	.<e>	@<e>





Another artifact of \ppt{} is the outputs of the \detok{} module.
While monitoring the training procedure, we periodically sample words from random locations centered around the surface of the unit sphere of the embedding space, to see what \say{priors} the generative net is learning from the vectors encountered during training.
\autoref{tab:gen} presents some of these samples for different models and different corpora at different points in the training phase.
Masked models appear to be learning well-formed fractions of English or pseudo-English words at early stages of the training phase from both Wikipedia and Twitter data.
As training continues, fewer sequences containing repetitions are observed, fewer generations occur across  samples, and more in-vocabulary words appear, suggesting convergence of the vector space towards representing well-formed and diverse English vocabulary.
The autoregressive GPT-2, on the other hand, struggles to produce meaningful sequences beyond short words and punctuation symbols when trained on the informal Twitter input, suggesting a difficulty in learning a mapping of language from vector space without availability of a two-sided context.



\subsection{Downstream Evaluation}
\label{ssec:finetune}

During task fine-tuning and inference,
since we do not evaluate on generative tasks, we do not use \detok{}.
We perform minimal hyperparameter search for each base model + task combination, and fine-tune model parameters during downstream training in all tasks but NYTWIT.
For these tasks, we also experimented with setups where \mmod{} and \tok{} are used as feature extractors only, and where \tok{} training is supplemented by an additional embedding loss (as described in \S\ref{ssec:pretr} for the \ppt{} phase) computed against embeddings of the task input.
Neither setup provided improvement on any task during our tuning experiments (see \S\ref{ssec:abl}).

Downstream models are implemented as follows: for sequence classification and ranking, a two-layer perceptron with \texttt{ReLU} activation is trained to make the prediction from the top-layer representation of the initial \texttt{[CLS]} token (in BERT and RoBERTa models) or of the final token in the sequence (in GPT-2).
For NER, an LSTM is run over the sequence of each word's top-layer representation, followed by a single linear layer which makes the prediction.
For NYTWIT, \mmod{}'s contextualized vector for the target word is used as input for a single logistic layer.
In cases of multi-token words, the prediction from the first token is selected.
We set \tok{}'s character embedding dimension to 200 and the convolutional layers to 256 channels.
Following \newcite{sun2019fine}, we set the maximum learning rate to $10^{-3}$ for the task models and $2\times 10^{-5}$ for fine-tuning, and perform warm-up for 10\% of the total expected training steps before linearly decaying the rates to zero.
All parameters are optimized using Adam~\cite{adam} with default settings.
We run all NER models for twenty epochs and sequence-level task models for three, evaluating on the validation set after each epoch using the metrics reported below, and stopping early if performance has not improved for four epochs.
In order to avoid unfairly favorable conditions for \tokdetok{} models, task hyperparameters are all tuned on the base models, with \tokdetok{} models using only values on which their base equivalents have also been evaluated.

We evaluate the effectiveness of \tokdetok{}'s concepts and components by comparing the following setups:\footnote{A setup where all words are represented by \tok{} obtained noncompetitive results on all tasks.}
\begin{itemize}
    \item \textsc{\textbf{None}} uses only the base model;
    \item \textsc{\textbf{None+\ppt{}}} uses a version of the base model that was further pre-trained on the same Twitter corpus on which \tokdetok{} is trained (adaptive fine-tuning~\cite{ruder2021lmfinetuning}), in order to control for the increase in total unlabeled text seen by the model;
    \item \textsc{\textbf{Scaffolding}} is a model trained with \tokdetok{} in a \ppt{} phase, but only using base model embeddings during task fine-tuning;
    \item \textsc{\textbf{Stochastic}} samples 10\% of the words in the downstream datasets, calling \tok{} on their character sequences while using the base model's embedding(s) for the remaining 90\%;
    \item \textsc{\textbf{All no-suff}} calls \tok{} on all multi-token words which are not of the form \texttt{[token suff]}, where \texttt{suff} is a member of the \textsc{Suffixes} set described in \S\ref{sec:model}, and uses the base model's embedding on the rest.
\end{itemize}

\begin{table*}
    \centering
    \small
    \begin{tabular}{llccccccccc}
        \toprule
        Base & \tokdetok{} & \multicolumn{2}{c}{Emoji} & \multicolumn{2}{c}{Twitter NER} & \multicolumn{2}{c}{Emerging NER} & QA & NYT- & Avg. \\
        & & Dev & Test & Dev & Test & Dev & Test &  & WIT & Test \\
        \midrule
        BERT & None & 24.30 & 37.29 & 29.06 & 28.66 & 38.14 & 29.01 & 46.78 & 45.21 & 37.39 \\
        & None+\ppt{} & \underline{\textbf{29.25}} & \textbf{39.40} & 32.05 & 29.50 & 39.79 & \textbf{29.25} & \underline{\textbf{50.02}} & 44.87 & 38.61 \\
        & Scaffolding & 28.17 & 38.60 & 32.45 & \textbf{31.38} & \textbf{41.10} & \textbf{29.25} & 49.50 & \textbf{53.62} & \textbf{40.47} \\
        & Stochastic & 27.54 & 37.41 & 30.77 & 27.51 & 37.99 & 27.98 & 48.12 & 49.80 & 38.16 \\
        & All no-suff & 27.12 & 38.63 & \textbf{34.09} & 29.59 & 39.47 & 28.43 & 48.64 & 34.76 & 36.01 \\
        \midrule
        GPT2 & None & 24.89 & 38.72 & 30.25 & 26.70 & 40.06 & 28.40 & 44.90 & 47.25 & 37.19 \\
        & None+\ppt{} & 25.47 & \textbf{40.95} & 32.00 & 28.51 & 40.68 & 29.90 & \textbf{47.18} & 47.44 & 38.80 \\
        & Scaffolding & 25.29 & 40.70 & 30.46 & 28.97 & 41.16 & 28.23 & 46.99 & \textbf{50.99} & \textbf{39.18} \\
        & Stochastic & 25.23 & 38.55 & 32.20 & 28.10 & 39.58 & 28.35 & 46.11 & 46.37 & 37.50 \\
        & All no-suff & \textbf{26.30} & 34.27 & \textbf{36.27} & \textbf{32.48} & \textbf{49.20} & \textbf{34.87} & 44.98 & 33.21 & 35.96 \\
        \midrule
        RoBERTa & None & 25.07 & 39.50 & 48.57 & 44.86 & 56.43 & \underline{\textbf{46.22}} & 45.12 & 48.31 & 44.80 \\
        & None+\ppt{} & \textbf{27.04} & \underline{\textbf{42.82}} & 47.39 & 43.84 & 57.13 & 45.12 & \textbf{48.98} & 48.43 & 45.84 \\
        & Scaffolding & 25.87 & 41.12 & \underline{\textbf{49.71}} & \underline{\textbf{45.51}} & 56.28 & 44.60 & 48.26 & \underline{\textbf{53.98}} & \underline{\textbf{46.69}} \\
        & Stochastic & 26.38 & 40.09 & 49.57 & 44.82 & \underline{\textbf{58.21}} & 45.26 & 48.54 & 51.80 & 46.10 \\
        & All no-suff & 26.82 & 33.07 & 46.55 & 42.72 & 55.07 & 43.91 & 47.45 & 35.65 & 40.56 \\
        \midrule
        \multicolumn{2}{l}{SOTA (reported)} & & 47.46 & & 52.4\phantom{0} & & 49.6\phantom{0} & & 48.4\phantom{0} & \\
        \bottomrule
    \end{tabular}
    \caption{Results on all models (Emoji, NER, NYTWIT: Micro-F1 $\times$ 100; QA: MRR $\times$ 100), all results except for \textsc{None} are averaged over three models initialized on different random seeds.
    Best result for each base model in bold, best across all underlined.
    % \todo{Report intervals somehow, maybe in appendix.}
    % \todo{Or maybe report median?}
    }
    \label{tab:main_results}
\end{table*}






We present the results of the downstream prediction tasks in \autoref{tab:main_results}.
All transformer models except for \textsc{None} were 2nd-phase pre-trained three separate times using different random seeds, and the mean results are reported.

The first observation we make is the dominance of RoBERTa, a thoroughly optimized masked language model, over the other models on the NER datasets in all its variants.
This suggests RoBERTa has captured fine-grained information about individual words that it was able to retain in its representations for them; the struggle of the \tokdetok{} model to provide improvement over the \textsc{None} versions of the model strengthens this hypothesis.
GPT-2, despite having access to only left-side context of each word, still outperforms the basic BERT model on most setups in the NER tasks, perhaps due to its larger subword vocabulary size.
At the same time, its gains from the \tok{} representations are much more considerable, suggesting that its left-context-only inference may in fact be detrimental to the its \mmod{}'s performance as a whole.
In general, models perform better on Emerging NER than on Twitter NER, which we attribute to several possible causes or their combination: first, the source shift in the Emerging NER test set from Twitter to Reddit is meant to encumber the models, but given the extensive pre-training they undergo they may actually benefit from the fact that test sequences are longer on average than those in the Twitter NER dataset;
second, more prosaically, the Emerging NER dataset contains fewer entity types (6 vs. 10), making the task itself somewhat easier.

The other word-level task, NYTWIT, demonstrates substantial gains made by the \tokdetok{} training regime: the \textsc{Scaffolding} setup preforms best in all three base models, and in both masked langugage models the \textsc{Stochastic} setup outperforms both \textsc{None} variants.
Together with the NER results, this suggests that \tokdetok{} succeeds in providing transformer models with \textbf{word-level} representations that allow coarse-grained classifications (such as named entity type or novel word origin), better than default subword segmentations, for both edited and user-generated text.

We find that \tokdetok{} is less successful in improving sequence classification and ranking performance than on word-level tasks.
The \textsc{None+\ppt} setup obtains the best results in most models on the Emoji and QA datasets, suggesting the improvements seen in \tokdetok{}-based models is mostly attributable to the domain shift introduced by the Twitter pre-training corpus.\footnote{We note the complete inconsistency in both model performance and comparative model ranking present in the Emoji dataset, which calls back the systemic issue we identified in~\S\ref{sec:tasks}:
The time span over which the test set was collected contained a fundamental shift in Twitter's properties --- doubling the tweet length limit from 140 to 280 characters --- and so exhibits an unpredictable corpus incompatible with the train and dev partitions.
As a result, model performance over the dev set does not predict the test set results (a large difference on performance in this task between dev and test sets was also observed in macro-F1 scores by~\newcite{barbieri-etal-2020-tweeteval}).
In fact, under such specific data shift circumstances, it could be the case that the simpler base model is better equipped for facing longer test data, which scales generalizations made over the training and dev sets, as opposed to \tok{}-augmented models which have more levels of generalization to acquire during training and cannot anticipate the scale change.
Post-hoc inspection of specific model outputs resulted in some more concrete hypotheses for the cause of discrepancy in the major categories of confusion, for example the distributions of \say{@} presence in heart emoji
%{\NotoEmoji\symbol{"2764}}
tweets and heart-eyes emoji
%{\NotoEmoji\symbol{"1F60D}}
tweets shifted to a degree which could explain the models' growing confusion between the two, but no signals accounted for the entire difference in performance and we conclude that the main reason remains the change in sequence length distributions.}
This suggests that \tok{} may be a good learner for word-internal phenomena, picking up structural cues as to their roles within or without context, making it more useful \textbf{locally} than subword tokens' uninformed embeddings, while not being strong enough to provide a better semantic prior for \mmod{} to aggregate together with surrounding well-formed words, making its \textbf{global} utility limited.


\begin{table}
    \centering
    \small
    \begin{tabular}{lrrr}
        \toprule
        Ablation & BERT & GPT2 & RoBERTa \\
        \midrule
        Full & 37.17 & 38.27 & 41.97 \\
        No FT & $-$0.31 & $-$0.25 & $-$0.80 \\
        Wiki \ppt{} & $-$3.11 & $-$2.67 & $-$2.17 \\
        % Wiki+Emb loss &  & $-$4.83 &  \\
        No loops & $+$0.82 & $+$0.26 & $-$1.12 \\
        All multi & $-$1.39 & $+$0.44 & $-$1.81 \\	
        \bottomrule
    \end{tabular}
    \caption{Dev set average effect of model variants compared with the \textit{All no-suff} condition: % on a single random seed:
    \say{No-FT} --- pre-trained model used only for feature extraction;
    \say{Wiki} --- \tokdetok{} trained on Wikipedia data instead of Twitter;
    % \say{Emb loss} --- single-token words in downstream data fed for training \tok{};
    \say{No loops} --- trained without the cycle dependency loops;
    \say{All multi} --- common-suffix words also inferred using \tok{}.}
    \label{tab:ablations}
\end{table}

% \begin{table}
%     \centering
%     \small
%     \begin{tabular}{lrr}
%         \toprule
%         Model / Noise & Random case & Repeat \\
%         \midrule
%         Base & 42.73 & 43.57 \\
%         Base + \ppt{} & 43.54 & 45.07 \\
%         Scaffolding & 43.15 & 44.31 \\
%         Stochastic & 42.54 & 43.92 \\
%         All-no-suff & 41.72 & 42.33 \\
%         \bottomrule
%     \end{tabular}
%     \caption{Noising experiments on RoBERTa models on the QA dataset.}
%     \label{tab:ablations}
% \end{table}




\subsection{Ablations}
\label{ssec:abl}
We compare the dev set results of the \textsc{All no-suff} condition on several modified versions of the model, presented in Table~\ref{tab:ablations}.
First, we find that fine-tuning \mmod{} and \tokdetok{} parameters during downstream task application is beneficial for results across base models, indicating susceptibility of \tok{}'s network to tune itself on task data and not solely on LM and the vectorization signal.
Next, and most substantially, we note the vast improvement of Twitter-trained models compared with \ppt{} performed over a Wikipedia corpus with comparable size;
even though some tasks are on edited text, the overall effect of domain change during second pre-training is apparent (and, indeed, least impactful on the NYTWIT task which features the best-edited text).
Other decisions made during the \ppt{} phase appear to be less decisive: removing the dependency loops helps performance on BERT and GPT2, but makes a large dent in the best-performing RoBERTa, indicating potential gains to be made by applying \detok{} in generative tasks not pursued within our scope;
the suffix-based dialing down of inference in pre-training helps the masked models but hurts GPT2 performance, possibly because its autoregressive application prohibits it from looking at a simply-inflected word's suffix when processing its stem, a problem not incurred in masked modeling.




\section{Related Work}
\label{sec:related}

CharBERT~\cite{ma-etal-2020-charbert} is a method which incorporates character-level encodings in \llm{}s, while re-designing the transformer stack to include a \say{character channel} distinct from the parallel token channel.
Char2Subword~\cite{aguilar2020char2subword} integrates multiple losses in a system trained to predict subword tokens from the character level, all originating in distance metrics between target and prediction.
CharacterBERT~\cite{el-boukkouri-etal-2020-characterbert} uses an ELMo-style character convolutional network to encode input into a transformer MLM.
CANINE~\cite{clark2021canine} is a method for training \llm{}s which removes the need for a subword tokenizer, by utilizing character-level representations which are pooled into inputs for the main transformer module.
ByT5~\cite{xue2021byt5} ventures yet deeper by training a T5-architecture model~\cite{raffel2020exploring} over byte representations, reducing the required embedding table size by one more order of magnitude.
Unlike \tokdetok{}, these systems all require training an \llm{} from scratch and do not support using a \ppt{} step for tuning existing large models.
In addition, with the exception of CANINE, they all resort to softmax prediction over a subword vocabulary for the generative portion of the pre-training phase, and do not offer a character-level decoder which can also be tied to the encoding component.





\section{Discussion}
\label{sec:discuss}

Our approach presents a system to identify and assess the ambiguity of cluster structure in scatterplots. We provided preliminary steps toward a comprehensive automation system %as a measure of proxy 
to simulate human judgments 
%and evaluate scatterplot ambiguity. 
in scatterplots. In this section, we discuss the implications of our measure design and experiments 
%then explore the limitations and future works. 
including opportunities for future work. 


% effectiveness (\autoref{}) and reliability (\autoref{}) of our measure design, then discuss the findings about ambiguity
% 
% In this section, we discuss the development of an automated proxy for human perception, the validation of its credibility, the ambiguity of determining cluster ambiguity, and future work to improve the model and broaden its concept of ambiguity.
% 
% \begin{itemize}[noitemsep]
%     \item \textbf{Proxy for Human Perception:}  We replicate prior results that indicate cluster perception can be measured to inform design and analysis practice.
%     \item \textbf{Data-Driven Model to Deal with Wide Range of Scatterplot Patterns:} We develop \measure that deals with diverse dataset characteristics to measure and rank cluster ambiguity.
%     \item \textbf{Scalable Model to Rank Ambiguity:} With proposed applications and evaluations with available methods, we demonstrated the scalability and applicability of our \measure in a data analysis environment.
% \end{itemize}


\subsection{Proxy for Human Perception}
In this work, we provided preliminary steps toward developing the model to identify and rank intrinsic variability in visual perception. 
%, in our case conducting visual clustering. 
Evaluating variability based on human participation is not always feasible, scalable, or robust considering 
%wide diverse data characteristics. 
the diversity of data characteristics, visualization designs, and other factors that might influence people's perceptions. There are simply too many sources of variance to account for.
% 
\measure, which automatically generates a score simulating human judgments, offers a scalable and robust alternative to an approach that attempts to account for every possible source of variance. 
% such a human-based approach. 
% 
Our work demonstrates the promise of using statistical modeling to infer patterns over a corpus of human estimates.  
% 
%our applications (\autoref{sec:appl}) demonstrate the benefit of the automation. 
Extending our 
%Applying the findings as an actionable and 
approach to other perceptual tasks (e.g., outlier detection)
would 
%support us to inject 
bring the notion of perceptual variability to more complex and practical applications \cite{moritz2018formalizing}.

% as we have seen in available systems \cite{moritz2018formalizing}.

Toward this end, we provide a \href{http://www.clusterambiguity.dev.s3-website.ap-northeast-2.amazonaws.com/}{demo interface} that measures and explains cluster ambiguity. 
We believe that the interface will enhance the usability of \measure and moreover serve as a preliminary step toward future applications considering perceptual variability.

\subsection{Credibility of the User Study-Based Design Process}

\label{sec:credibility}

While designing \measure, we identified several factors that play a vital role in cluster perception through a user study (\autoref{sec:preexp}), and built a regression module estimating human cluster perception based on study findings (\autoref{tab:feature_eng}). These factors support building a robust module that accurately reflects how people identify clusters in data analysis (\autoref{sec:regmodel}).
Our ablation study (\autoref{sec:regmodeleval}) verifies the importance of these factors, validating the credibility of using human strategies to inform model parameters. 
Moreover, the main study (\autoref{sec:mainstudy}) showed that the measure built upon such a design process reliably estimates the ambiguity of scatterplots with a wide range of cluster patterns.

Our study asked participants about how the characteristics of datasets influence the ambiguity of a scatterplot (see \autoref{sec:mainstudy}). The findings from the interview validate the reliability of our design process. As shown in \autoref{fig:qual_study}, participants identified the factors that we have considered in our model (see \autoref{tab:feature_eng}), such as density, proximity, shape, and how clusters are located and related to each other (i.e., distribution). 
Note that such results also match well with prior findings~\cite{sedlmair2015data, sadahiro1997cluster,quadri21tvcg}. 

\subsection{\textit{Ambiguity} of Ambiguity}

While our interview revealed common factors that influence cluster perception (\autoref{sec:credibility}), it also showed that the perceived influence of different factors on ambiguity varies among participants. In other words, the definition of cluster ambiguity is \textit{ambiguous} to each participant. 
This observation depends on what factors are given more importance by different participants when determining a scatterplot's ambiguity.
For example, P01 said, \textit{"I find the shape of the group of points to be the main reason in identifying and separating clustering. If they have salient separable shapes, they are more clear"}, and P07 noted, \textit{"Proximity and concentration of the points help me separate the clusters quickly"}. 
\autoref{fig:qual_study} also supports this finding, showing that no single factor received complete agreement from the participants (see the white transparent bars).
Considering the variability across participants,
we conclude that ambiguity cannot be determined by a single person but must instead reflect a population of individuals. 
The fact that the performance of human annotators (i.e., participants) in predicting ground truth ambiguity varied in our main study (\autoref{sec:mainstudy}) supports this claim.
This observation indicates that, inevitably, multiple subjects are required for assessing ambiguity through human resources, thereby underscoring the importance of our automated solution.

\subsection{Limitations and Future Work}
\measure outperformed computational approaches and more than 50\% of participants (\autoref{sec:mainstudy}) in precisely estimating cluster ambiguity. However, there 
%is plenty of room 
are several opportunities to improve \measure.
For example, we found that \measure erroneously considers scatterplots with more numbers to have less ambiguity, as seen in \autoref{fig:topbottom}---the top eight clear scatterplots (top row) generally have more clusters than the top eight ambiguous ones (bottom row). We quantitatively demonstrate this bias in Appendix D.
\rev{We can also improve the method of aggregating pairwise ambiguity scores, which is currently a na{\"i}ve average. Considering cluster topology or their pairwise distances during the aggregation may better reflect human visual perception.}

\rev{
Another open direction is to generalize \measure. We can extend \measure to consider visual encoding (e.g., size, shape, and color of marks) of scatterplots by revising the features feed into our regression module. We may also improve our measure to deal with nested clusters by adopting hierarchical clustering algorithms \cite{mullner11arxiv} in place of GMM. We are also interested in
}
broadening the concept of ambiguity to encompass general visualization. 
For example, we can model the perceptual variability that may arise when examining cluster structures of high-dimensional data via scatterplot matrices or parallel coordinates. Moreover, we aim to develop a model estimating ambiguity in various perceptual tasks, including outlier detection and trend analysis.


\section{Conclusion}

We introduce \measure, a VQM for estimating the cluster ambiguity of a monochrome scatterplot, which originally required expensive human resources to compute. 
To serve as a proxy for human perception across a wide range of cluster patterns, our measure is designed based on a qualitative user study and is trained over perceptual data.
Through quantitative evaluations, we verified that \measure outperforms automatic competitors while showing competitive performance with human annotators. Our research findings not only demonstrate current applications but also invite discourse on potential future applications that capitalize on the concept of ambiguity.
In summary, our work represents a \rev{significant} advancement toward developing a comprehensive framework that elucidates the phenomenon of ambiguity in visualization. 

\newpage

% \vspace{5pt}
% \noindent \textbf{Interactive Demo of} \measure: \texttt{\href{http://clams-demo.s3-website.ap-northeast-2.amazonaws.com/}{clams-demo.com}.}

% Our approach is trained over perceptual data and works as a proxy for human perception for a wide range of scatterplots. 


% In conclusion, we have introduced CLAMS, a VQM for estimating the cluster ambiguity of a monochrome scatterplot. Our approach is trained over perceptual data and works as a proxy for human perception. By decomposing the input scatterplot using the Gaussian mixture model (GMM) and measuring ambiguity in a component-pairwise manner, our measure can deal with a wide range of cluster patterns while maintaining scalability. Our method can be used to improve the performance of clustering algorithms and aid in data analysis tasks. As future work, we plan to extend our approach to handle color scatterplots and explore its application in other domains.






% \subsection{Benchmarking Outlier Detection \fix{(Hyeon)}}
% \subsection{Computational Complexity and Scalability}



% searching for such defects and improving \measure will be an interesting future work; we discuss this issue in \autoref{sec:discuss}.


\bibliography{anthology,tdt}
\bibliographystyle{acl_natbib}

\clearpage

% \appendix

% \input{99_appendix}

\end{document}
