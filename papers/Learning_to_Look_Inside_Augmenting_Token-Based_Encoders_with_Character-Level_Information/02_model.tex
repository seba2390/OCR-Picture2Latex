\section{\tokdetok}
\label{sec:model}

We focus on character-informed representations for \llm{}s which use the transformer architecture~\cite{vaswani2017attention}, such as BERT~\cite{devlin-etal-2019-bert}, RoBERTa~\cite{liu2019roberta}, 
and GPT~\cite{radford2018improving}.
A Transformer \llm{} is composed around a core module \mmod{} parameterized as multi-head self-attention layers, which accepts a sequence $\{\ve_1, \dots, \ve_l\}$ of embedding vectors corresponding to a list of token indices $\{t_1, \dots, t_l\}$ from a vocabulary $\mathcal{V}$,
and outputs a sequence of contextualized vectors $\{\vh_1, \dots, \vh_l\}$, of which a subset $\{\vh_i\}_{i\in\mathcal{I}}$ correspond to masked tokens in positions $\mathcal{I}\subseteq [l]$ selected by a boolean masking operator $m$.\footnote{In the case of an autoregressive \llm{} like GPT, $\mathcal{I} = [l]$ but each vector $\vh_i$ is only dependent on the tokens preceding its position, $\{t_1, \dots, t_{i-1}\}$, and token-wise prediction is performed for each position assuming no knowledge of future tokens.}
The token index list supplied to \mmod{} is the output of a tokenization function $\tau$ operating on a sequence $X$ followed by lookup in an embedding table $\vE$,
while its output vectors $\{\vh_1, \dots, \vh_l\}$ serve as input to a prediction module \gen{}, which outputs a distribution $D_i$ over $\mathcal{V}$ for each masked position $i$.
All embeddings $\{\ve_i\},\{\vh_i\}$ live in a shared space $\mathds{R}^d$.
Together, the components described so far operate in the following manner:

\begin{equation}
    \{D_i(\mathcal{V})\}_{i\in\{\mathcal{I}\}} = 
    \text{\gen{}}\left( \text{\mmod{}} \left( \vE\left[ \tau(X) \right] \right) \right),
\end{equation}
where square bracketing denotes elementwise table lookup.

In the considered \llm{} architectures, the input $X$, which is atomically made up of a sequence of characters $\{c_1, \dots, c_n\}\in\Sigma^{n}$, is broken down by $\tau$ to provide the tokenization of length $l\leq n$,
and the prediction/generation operator \gen{} accepts the contextualized outputs $\{\vh_1, \dots, \vh_l\}$ and implements prediction by means of a softmax distribution which is based on scores obtained via dot-product against an output embedding table, which is usually the same as $\vE$.

The \tokdetok{} model makes no adjustments to \mmod{} itself, and only offers conditional replacements for $\tau$ and \gen{}.
An \textbf{encoding policy} $\pi^t$ selects a subset of tokens from the original sequence $\mathcal{J}^t\subseteq [l]$ to be represented by \tok{} instead of $\vE \circ \tau$.
\tok{} has access to the part of the character sequence $X$ which underly the tokens selected by $\pi^t$, and produces input embeddings directly from the character level.
These alternate embeddings must agree in dimension with those of each $\ve$, but a single embedding may be used to replace multiple base tokens (usually when all tokens corresponding to a single out-of-$\vE$ word are replaced), resulting in a shorter input sequence for \mmod{}.
A separate \textbf{generation policy} $\pi^g$ selects a subset of the original sequence $\mathcal{J}^g\subseteq [l]$ to be generated from the output vectors $\vh$ by \detok{} instead of \gen{}.
A high-level schematic depicting this framework is presented in \autoref{fig:algo}.

\begin{table*}
    \centering
    \tiny
    \begin{tabular}{ll}
        \toprule
        % Original & \texttt{Approx~~~~~~~~~~distance in miles from Bruges~~~~~~~to Amsterdam is 107 miles or 172.16 KMS~~~~.} \\
        % Tokenization & \texttt{A \#\#pp \#\#ro \#\#x distance in miles from B \#\#rug \#\#es to Amsterdam is 107 miles or 172 . 16 K \#\#MS .}  \\
        {\small Original} & \texttt{He was emphatically~~~~~a modern gentleman,~~of scrupulous~~~~~~~courtesy,~~sportive~~~gaiety,} \\ % acquainted with what was going on in the world.} \\
        {\small Word pieces} & \texttt{He was em \#pha \#tically a modern gentleman , of s \#c \#rup \#ulous courtesy , sport \#ive g \#ai \#ety ,} \\
        \midrule
        % \multirow{2}{*}{\small Random-20\%} & {\Monospace .9~.75~.15~~~~~~~~~~~~~.3~.05~~~~.55~~~~~~.6~.4~.7~~~~~~~~~~~~~~~.1~~~~~~.8~.25~~~~~~~~.45~~~~~~~.3} \\
        {\small Random-20\%} & \texttt{He was [TOK]~~~~~~~~~~~~a [TOK]~~gentleman , of s \#c \#rup \#ulous~[TOK]~~~~, sport \#ive g \#ai \#ety ,} \\
        {\small All-multi} & \texttt{He was [TOK]~~~~~~~~~~~~a modern gentleman , of [TOK]~~~~~~~~~~~~courtesy , [TOK]~~~~~~[TOK]~~~~~~,} \\
        {\small \textsc{Suffixes}} & \texttt{He was [TOK]~~~~~~~~~~~~a modern gentleman , of [TOK]~~~~~~~~~~~~courtesy , sport \#ive [TOK]~~~~~~,} \\
        \bottomrule
    \end{tabular}
    \caption{Example of token input selection by different policies,
    % (randomly-generated numbers determine the random policy) 
    where \texttt{[TOK]} signifies a word to be replaced by its character-based representation from \tok{}.
    Sentence fragment taken from the MS-MARCO QA dataset (see \S\ref{sec:tasks}) and tokenized using BERT-cased (\#\# replaced with \# to fit paper width).}
    \label{tab:policies}
\end{table*}



The specific policies in a given application may be defined based on the model's use case.
For example, in text classification no generation is required, and so $\pi^g$ will return $\emptyset$ for all sequences; $\pi^t$ can be tuned for a task based on known features of the base model (BERT/GPT etc.) and of the domain text, some examples including:
tokens corresponding to all words that are not in a pre-determined vocabulary;
all words in the sequence;
all words assigned more than one token by the base tokenizer $\tau$;
a random sample of words in the sequence;
or all words including characters that are not lowercase English characters.
One particular policy we hypothesize could be useful is one that affords the tokenizer slack in detecting a single simple derivational or inflectional suffix: all words which are single-token or whose second-and-final token is in the list \textsc{Suffixes} are left for $\vE \circ \tau$; the rest are represented using \tok.\footnote{\textsc{Suffixes}, compiled by manually examining a list of most common second-and-final tokens in a large corpus under GPT-2's tokenization, = $\{$\emph{s}, \emph{ed}, \emph{es}, \emph{ing}, \emph{ly}, \emph{al}, \emph{ally}, \emph{'m}, \emph{'re}, \emph{'ve}, \emph{y}, \emph{ive}, \emph{er}, \emph{'t}, \emph{'ll}, \emph{an}, \emph{ers}$\}$.
Similar policies can be hand-crafted for different languages, based on learner-level knowledge of the language and minimal preliminary analysis of sample tokenizations. Our results show they are not strictly necessary.}
Different policies may be applied in training settings as well, for example in order to \say{familiarize} the heavily-parametrized \mmod{} with the inputs from \tok{}.
Three example policies are illustrated in~\autoref{tab:policies}.



\subsection{Second Pre-training}
\label{ssec:pretr}

A typical \llm{} is initiated through a computationally-intensive pre-training step, iterating over a large corpus in batches of sequences and backpropagating a cross-entropy loss calculated over the prediction layer's output into all of its components, through \gen{} to \mmod{} to $\vE$.
In order to train \tokdetok{}, we introduce a second pre-training step we term \ppt{}, where the \llm{} continues to update its parameters for a (possibly different) corpus, but is supplemented with the \tokdetok{} elements in order to \say{acclimatize} the \mmod{} components to outputs from \tok{}.\footnote{Within the taxonomy of phases between pre-training and task training~\cite{ruder2021lmfinetuning}, this is closest to \textbf{Adaptive Fine-Tuning}. However, due to our added modules we opt to assign it a new name.}
In addition, \tok{} is also trained through a lower-level objective requiring it to approximate the outputs of $\vE \circ \tau$ which it is replacing, and \detok{} is trained to sequentially produce the correct character sequence from \mmod{}'s outputs.
Together, a batch of text sequences in a \ppt{} step produces the following loss elements which are backpropagated into the unified model:
\begin{itemize}
    \item A \textbf{language modeling loss} from the softmax operation over masked tokens, updating \mmod{}'s and $\vE$'s parameters, as well as \tok{}'s for tokens selected by a usage policy $\pi^t_{(u)}$ to be used in \mmod{}'s input;
    \item An \textbf{embedding loss} for the \tok{} component, computed against $\vE \circ \tau$'s token embeddings, over a set selected by a policy $\pi^t_{(l)}$ which may or may not equal $\pi^t_{(u)}$.
    This loss can be computed, e.g., as the euclidean distance between the output and the target.
    When the target corresponds to multiple token counts, an aggregation pooling function $\agg:\mathds{R}^{d \times \mathds{N}^{+}}\rightarrow\mathds{R}^d$ needs to be defined over the embeddings, for example taking their dimension-wise mean, taking the leftmost token's embedding, or taking the dimension-wise $\max$;
    \item A \textbf{generation loss} for words generated by \detok{} from \mmod{}'s output vectors, according to a $\pi^g$ policy.
    This loss is the character-level cross-entropy for autoregressive sequence generation.
    Note that in order to only generate full words, $\pi^g$ must align with $\pi^t_{(u)}$ so that no multi-token words left as input to \mmod{} are also selected for generation.
\end{itemize}

As a form of regularization within the \tokdetok{} components, we introduce additional training batches we call \textbf{cycle dependency loops}.
In such a batch, \tok{} and \detok{} act in succession, starting from one of the spaces they operate in, with the goal of arriving at the same point after cycling through both components.
An \textbf{\tdloop} loop thus starts at a character sequence $\vc\in{\Sigma^*}$, runs it through \tok{} to obtain a vector $\ve=\tok(\vc)$, and runs \detok{} in an attempt to return to the original sequence $\hat{\vc}=\detok(\tok(\vc))\approx \vc$.
Analogously, a \textbf{\dtloop} loop starts at a vector $\tilde{\ve}\in\mathds{R}^{d}$ and targets $\hat{\ve}=\tok(\detok(\tilde{\ve}))\approx \tilde{\ve}$.
In this loop, loss is only backpropagated as far as \tok{}, since backpropagating through a generative model's decision component introduces discrete steps which must be smoothed or approximated (see discussion in \newcite{peng-etal-2018-backpropagating}).


