\documentclass[prb,reprint,twocolumn,preprintnumbers,amsmath,amssymb,showpacs,nofootinbib,superscriptaddress]{revtex4-2}

\usepackage[titletoc,toc,title]{appendix}
\usepackage{braket}
\usepackage[final]{feynmp}
\usepackage{ifpdf}
\usepackage{comment}
\usepackage{amsmath,bbold}
\usepackage{mathrsfs}
\usepackage{color}
\usepackage{bm}

\bibliographystyle{apsrev4-2}

\newenvironment{rcases}
  {\left.\begin{aligned}}
  {\end{aligned}\right\rbrace}
  
\DeclareGraphicsRule{*}{mps}{*}{}
\makeatletter
\def\endfmffile{%
	\fmfcmd{\p@rcent\space the end.^^J%
			end.^^J%
			endinput;}%
	\if@fmfio
		\immediate\closeout\@outfmf
	\fi
	\ifnum\pdfshellescape=\@ne
		\immediate\write18{mpost \thefmffile}%
	\fi}
\makeatother

\frenchspacing

\usepackage{graphicx}% Include figure files
\usepackage{dcolumn}% Align table columns on decimal point
\usepackage{bm}% bold math
\usepackage{hyperref}% add hypertext capabilities

\newcommand{\bs}[1]{\boldsymbol{#1}}

\begin{document}

\title{Supplementary Material for ``Dislocation Non-Hermitian Skin Effect"}

\author{Frank Schindler}
\affiliation{Princeton Center for Theoretical Science, Princeton University, Princeton, NJ 08544, USA}

\author{Abhinav Prem}
\affiliation{Princeton Center for Theoretical Science, Princeton University, Princeton, NJ 08544, USA}

\maketitle

%%%%%%%%%%%%%%%%%%%%%%%%%%%%%%%%%
%%%%%%%%%%%%%%%%%%%%%%%%%%%%%%%%%

%\newcommand{\citenumfont}[1]{S#1}
%\renewcommand{\citenumfont}[1]{S#1}
\renewcommand{\bibnumfmt}[1]{[S#1]}
\renewcommand{\theequation}{S\arabic{equation}}
\renewcommand{\thefigure}{S\arabic{figure}}
\renewcommand{\thetable}{S.\Roman{table}}

\setcounter{equation}{0}
\setcounter{figure}{0}
\setcounter{table}{0}
%\setcounter{citation}{0}

%\clearpage
%\appendix
\onecolumngrid

%%%%%%%%%%%%%%%%%%%%%%%%%%%%%%%%%
%%%%%%%%%%%%%%%%%%%%%%%%%%%%%%%%%

\tableofcontents

%%%%%%%%%%%%%%%%%%%%%%%%%%%%%%%%%
%%%%%%%%%%%%%%%%%%%%%%%%%%%%%%%%%

\section{Scaling behaviour of the dislocation non-Hermitian skin effect}
\label{sec:scaling}

The skin effect is a well-established signature of topologically non-trivial nH systems~\cite{zhang2020correspondence,TopoSkin20}, wherein $\mathcal{O}(L^{d-1})$ eigenstates localise on the $(d-1)$-dimensional boundary of a $d$-dimensional nH system. For a $d$ dimensional system, if we open boundaries along the $j^{th}$ direction, the skin effect presents at one (both) boundaries for a non-reciprocal (reciprocal) system. Note that $L$ denotes the linear extent of the system. 

Here, we introduce a pair of dislocations with Burgers vector $\bs{B}=\pm\hat{y}$ into each of the three models considered in the main text and explicitly show that the dislocation non-Hermitian skin effect (DNHSE) corresponds to $\mathcal{O}(L)$ eigenstates localising at the dislocation core(s). Specifically, we calculate the total spectral weight of eigenstates $\Psi_i$
\begin{equation}
    S(A) = \sum_{{\bs x} \in A} s({\bs x}), \quad s({\bs x}) = \sum_i |\Psi_i({\bs x})|^2 ,
\end{equation}
over some region $A$ localized near the dislocation core(s) and study it as a function of system size $L$. For each of the three models in the main text, the results are shown in Fig.~\ref{fig:scaling}, with the corresponding region $A$ depicted in Figs.~2~c),~3~c), and~4~d). Note that we keep the size of the region $A$ fixed even as we change the system size. 

\begin{figure}[t]
    \centering
    \includegraphics[width=\textwidth]{scaling.pdf}
    \caption{Scaling of the total spectral weight $S(A)$ in a region $A$ localised around the dislocation core(s) for each of the three models considered in the main text. Open dots are results obtained numerically while dashed lines illustrate the linear scaling. For each $L$, we place the system on a square geometry ($L \times L$ sites) with periodic boundary conditions in both $x$ and $y$ directions and introduce a pair of dislocations separated by $\left \lfloor{L/2}\right \rfloor$ and with $\bs{B} = \pm \hat{y}$.}
    \label{fig:scaling}
\end{figure}

\begin{figure}[t]
    \centering
    \includegraphics[width=0.4\textwidth]{epdist.pdf}
    \caption{Total spectral weight $S(A)$ as a function of the distance between $k_y = \pi$ and the EP at $k_y = \Delta/2$ for the model in Eq.~(9). We place the system on a square geometry ($60 \times 60$ sites) with periodic boundary conditions in both $x$ and $y$ directions and with a pair of dislocations separated by 30 sites. The area $A$ is shown in Fig.~4~d).}
    \label{fig:EPdist}
\end{figure}

\begin{figure}[t]
\centering
\includegraphics[width=\textwidth]{epfig_sm.pdf}
\caption{Tight-binding model exhibiting stable exceptional points (EPs) and a dislocation non-Hermitian skin effect in the presence of an $\mathcal{O}(1)$ non-Hermitian perturbation [$\delta h$ in Eq.~\eqref{eq:EPHampert}]. (a)~PBC spectrum for the Hamiltonian in Eq.~\eqref{eq:EPHampert} for the parameter choice $t_x=t_y=t=1,m=2,\delta = -1.2$. (b)~PBC spectrum on a square geometry of $60 \times 60$ sites in presence of a pair of dislocations separated by $30$ sites. States are colored by their weight in the dislocation region $A$ [see panel c)]. Importantly, only the $k_y = \pi$ point gaps [indicated in red in panel a)] participate in the response to a dislocation with Burgers vector $\bs{B} = \hat{y}$. (c)~Local density of states. The accumulation at only one of the dislocations signals the non-Hermitian skin effect.}
\label{fig:EPpert}
\end{figure}

As discussed in the main text, for the model with weak nH topology [Eq.~(1)], the DNHSE can be simply understood as a consequence of the conventional skin effect in the one-dimensional Hatano-Nelson model. Specifically, when $t_y = t_d = 0$, Eq.~(1) reduces to a decoupled stack of one-dimensional Hatano-Nelson chains, and introducing a pair of dislocations with $\bs{B} = \pm \hat{y}$ is equivalent to opening the boundary conditions for a single Hatano-Nelson chain at a fixed $y$ coordinate. Thus, since only a single chain contributes to the DNHSE in the decoupled limit, we expect $S(A) \sim L/2$; this is confirmed numerically as shown in Fig.~\ref{fig:scaling}~a). Once we turn on hopping along the $y$ direction, the other one-dimensional chains also contribute to the DNHSE so that $S(A)$ still scales linearly with $L$ albeit with a parameter dependent slope.

Since the model in Eq.~(5) with strong nH topology is reciprocal, eigenstates localise at both of the dislocation cores and we calculate the combined spectral weight from each of the defects [i.e., $A = A_l \cup A_r$ as shown in Fig.~3~c)]. Fig.~\ref{fig:scaling}~b) clearly shows the $\mathcal{O}(L)$ scaling, consistent with the presence of a DNHSE. This scaling persists as long as there is a non-trivial $\mathbb{Z}_2$ winding along the BZ line at $k_y = \pi$. 

Finally, let us consider the model Eq.~(9) with $m=2, t=1$ which hosts bulk EPs for any $\delta \neq 0$. As a consequence of the non-trivial winding $w_x(\pm E^*,\pi) = 1$ (see main text), we find the presence of a DNHSE at one of the dislocation cores in this model. Surprisingly, we find that the total spectral weight localised at the defect shows linear scaling with system size even in the absence of a bulk gap. While for small system sizes we observe deviation from linear scaling as a consequence of finite-size effects, Fig.~\ref{fig:scaling}~c) clearly demonstrates that $S(A) \sim L$ at larger system sizes. To verify that the non-trivial winding of the $k_y = \pi$ BZ submanifold around $E = \pm E^*$ is indeed responsible for the robust dislocation response, we plot the total spectral weight $S(A)$ as a function of the distance $\pi - \Delta/2$ between $k_y = \pi$ and the EPs at $k_y = \pm \Delta/2$, where $\Delta = \cos^{-1}\left[1/2 (2-\delta^2) \right]$. As shown in Fig.~\ref{fig:EPdist}, we see that $S(A)$ is largely independent of this distance when the two EPs are not too close to $k_y=0$ (in this limit, where $\delta = 0$, the DNHSE must vanish because the system becomes Hermitian); this supports our claim that the non-trivial topological winding of the $k_y = \pi$ BZ submanifold is responsible for the DNHSE even in the absence of a bulk-gap.

%%%%%%%%%%%%%%%%%%%%%%%%%%%%%%%%%
%%%%%%%%%%%%%%%%%%%%%%%%%%%%%%%%%

\section{Stability of the dislocation non-Hermitian skin effect to large non-Hermitian perturbations}
\label{sec:stability}

Here, we demonstrate the robustness of the DNHSE to generic non-Hermitian perturbations in the model of Eq.~(9) with bulk EPs. In particular, we consider the following perturbed Hamiltonian
\begin{equation}
\label{eq:EPHampert}
    H_{\mathrm{pert}}(\bs{k}) = H(\bs{k}) + \delta h, 
\end{equation}
where $H(\bs{k})$ is given by Eq.~(9) with $t_x = t_y = t = 1, m=2$ and $\delta h$ is a random $2\times 2$ nH matrix. As discussed in the main text, the unperturbed Hamiltonian exhibits topologically non-trivial EPs, which do not rely upon any symmetry/fine tuning. That is, EPs are topologically stable band degeneracies which are robust against any continuous changes in the Hamiltonian as long as they do not cause the system to undergo a phase transition where EPs are created or annihilated~\cite{kawabata2019semi}. Thus, we expect that a generic nH perturbation $\delta h$ will only move the location of the EPs in the BZ but will not cause the pair to annihilate.

Concurrently, the point-gap for the $k_y = \pi$ line will remain open for generic perturbations since the system is unlikely to undergo a phase transition in the 2D BZ. We hence expect that adding a nH perturbation will simply move the location of the EPs and of the energies around which the the $k_y = \pi$ BZ line winds, but will not change the topological winding invariant of the $k_y = \pi$ line. As a consequence, the DNHSE is robust to generic (non fine-tuned) nH perturbations. Note crucially that, unlike in Hermitian systems, we do not impose any requirement on the strength of these perturbations or place any symmetry constraints: the DNHSE in gapless nH systems hence represents a topological phenomenon with no Hermitian counterpart. As shown in Fig.~\ref{fig:EPpert}, we find that the DNHSE is indeed robust to $\mathcal{O}(1)$ (compared to the unperturbed part of the Hamiltonian) nH perturbations.

%%%%%%%%%%%%%%%%%%%%%%%%%%%%%%%%%
%%%%%%%%%%%%%%%%%%%%%%%%%%%%%%%%%

\section{Dislocation non-Hermitian skin effect under open boundary conditions }
\label{sec:OBC}

For completeness, here we consider the DNHSE for each of the three models considered in the main text in the presence of open boundary conditions (OBC). 

\begin{figure}[t]
    \centering
    \includegraphics[width=0.4\textwidth]{OBCWeak.pdf}
    \caption{Local density of states for the Hamiltonian in Eq.~\eqref{eq:weakRS} on a square geometry of $60 \times 60$ sites in presence of a pair of dislocations separated by $30$ sites and with OBC. The parameters are the same as those in Fig.~(2) in the main text: $t_r = 3/2, t_l = t_u = 1/2, t_d = 0$.}
    \label{fig:OBCWeak}
\end{figure}

\begin{itemize}

\item The real space Hamiltonian for the weak topological phase, corresponding to Eq.~(1) in the main text, is 
\begin{equation}
\label{eq:weakRS}
    H = \sum_{\bm{r}} \left[ t_l c_{\bm{r}}^\dagger c_{\bm{r} + \bm{\hat{x}}} + t_r c_{\bm{r}+\bm{\hat{x}}}^\dagger c_{\bm{r}} + t_d c_{\bm{r}}^\dagger c_{\bm{r} + \bm{\hat{y}}} + t_u c_{\bm{r} + \bm{\hat{y}}}^\dagger c_{\bm{r}} \right].
\end{equation}
We study this system on a square geometry with OBC in both directions and in the presence of a pair of dislocations. As is clear from Fig.~\ref{fig:OBCWeak}, we see that the DNHSE is greatly suppressed. Indeed, this is exactly what one expects since the model Eq.~\eqref{eq:weakRS} exhibits the conventional skin effect under OBCs and we see clearly that modes are localized around the right boundary. Note that the localization is asymmetric along the $y$-axis since $t_u \neq t_d$ and the modes would be symmetrically localized along the right boundary if $t_u = t_d$.

\item The real space Hamiltonian for the strong topological phase, corresponding to Eq.~(5) in the main text, is 
\begin{equation}
\label{eq:strongRS}
\begin{aligned}
    H = \sum_{\bm{r}} \bigg[&\frac{t_x}{2 i} \left(c_{\bm{r}+\bm{\hat{x}}}^\dagger \sigma_x c_{\bm{r}} - c_{\bm{r}}^\dagger \sigma_x c_{\bm{r}+\bm{\hat{x}}} \right) + \frac{t_y}{2 i} \left(c_{\bm{r}+\bm{\hat{y}}}^\dagger \sigma_y c_{\bm{r}} - c_{\bm{r}}^\dagger \sigma_y c_{\bm{r}+\bm{\hat{y}}} \right) \\
    &+ \frac{i \gamma}{2} \left(c_{\bm{r} + \bm{\hat{x}}}^\dagger c_{\bm{r}} + c_{\bm{r}}^\dagger c_{\bm{r} + \bm{\hat{x}}}  \right) + \frac{i \gamma}{2} \left(c_{\bm{r} + \bm{\hat{y}}}^\dagger c_{\bm{r}} + c_{\bm{r}}^\dagger c_{\bm{r} + \bm{\hat{y}}}  \right) \bigg].
\end{aligned}
\end{equation}
For this model, which exhibits strong nH topology, we see from Fig.~\ref{fig:OBCStrong} that the DNHSE persists and modes remain localized in the vicinity of the dislocation cores. However, we also observe additional modes that are localized at the corners of the sample -- since this model is known not to exhibit the conventional skin effect for full OBC~\cite{okuma2020}, these corner modes are spurious and, unlike the DNHSE, not topologically enforced.

\begin{figure}[t]
    \centering
    \includegraphics[width=0.4\textwidth]{OBCStrong.pdf}
    \caption{Local density of states for the Hamiltonian in Eq.~\eqref{eq:strongRS} on a square geometry of $60 \times 60$ sites in presence of a pair of dislocations separated by $30$ sites and with OBC. The parameters are the same as those in Fig.~(3) in the main text: $t_x = t_y = \gamma = 1$.}
    \label{fig:OBCStrong}
\end{figure}

\item The real space Hamiltonian for the exceptional point Hamiltonian, corresponding to Eq.~(9) in the main text, is
\begin{equation}
    \label{eq:EPRS}
    H = \sum_{\bm{r}} \left[ c_{\bm{r}}^\dagger \left(m \sigma_z + i \delta \sigma_x \right)c_{\bm{r}} \right] +  \sum_{\bm{r}} \sum_{j=x,y} \left[ c_{\bm{r}}^\dagger \left(\frac{i}{2}t_j \sigma_j -\frac{1}{2}t \sigma_z \right) c_{\bm{r}+ \bm{\hat{j}}}+ \text{h.c.} \right]
\end{equation}


\begin{figure}[t]
    \centering
    \includegraphics[width=0.4\textwidth]{OBCEP.pdf}
    \caption{Local density of states for the Hamiltonian in Eq.~\eqref{eq:EPRS} on a square geometry of $60 \times 60$ sites in presence of a pair of dislocations separated by $30$ sites and with OBC. The parameters are the same as those in Fig.~(4) in the main text: $m=2, t_x = t_y = t = 1, \delta=-1.2$.}
    \label{fig:OBCEP}
\end{figure}
As discussed in Ref.~\cite{zhang2021universal}, nH systems with EPs can exhibit the conventional skin effect. This is also the case for the model Eq.~\eqref{eq:EPRS} when placed on a sample with OBC in both directions. As a consequence, we see in Fig.~\ref{fig:OBCEP} that the DNHSE is greatly suppressed due to the competition with the conventional skin effect. 

\end{itemize}

%%%%%%%%%%%%%%%%%%%%%%%%%%%%%%%%%
%%%%%%%%%%%%%%%%%%%%%%%%%%%%%%%%%

\section{Proof that nontrivial point-gap topology of Brillouin zone submanifolds effects a dislocation non-Hermitian skin effect}
\label{sec:proof}

We here derive necessary and sufficient conditions for the DNHSE in presence of a nH point gap. We focus on 2D nH insulators in two simple symmetry classes: (1) no symmetry and (2) reciprocity. The conditions derived here were exemplified by the models with weak and strong topology described in the main text, but hold more generally in that they can be used to predict the DNHSE for arbitrary nH band structures in the same symmetry class.
Specifically, we prove that the $\mathbb{Z}$ ($\mathbb{Z}_2$) nH winding number of 1D BZ loops in 2D non-reciprocal (reciprocal) nH insulators determines the presence or absence of a $\mathbb{Z}$ ($\mathbb{Z}_2$) DNHSE, via the if-and-only-if relations
\begin{equation} \label{eq: nonhermcorrespondence}
\begin{rcases}
  \nu^{\mathbb{Z}}_{\gamma(\bs{B})}(E) \neq 0 \quad &\leftrightarrow \quad \mathbb{Z} \, \text{DNHSE} \\
  \nu^{\mathbb{Z}_2}_{\gamma(\bs{B})}(E) = 1 \quad &\leftrightarrow \quad \mathbb{Z}_2 \, \text{DNHSE}
\end{rcases} \text{occurs for dislocations with Burgers vector } \bs{B}.
\end{equation}
Here, the 1D BZ loop $\gamma(\bs{B})$ in the Brillouin zone BZ is determined by the condition 
\begin{equation}
\gamma(\bs{B}) = \{\bs{k}\in \mathrm{BZ}|\bs{B} \cdot \bs{k} \mod 2\pi = \pi \}, 
\end{equation}
Moreover, the $\mathbb{Z}$ and $\mathbb{Z}_2$ nH winding numbers with respect to a point gap at complex energy $E$ are defined as
\begin{equation}
    \nu^{\mathbb{Z}}_{\gamma(\bs{B})}(E) = \oint_{\gamma(\bs{B})} \frac{d\bs{k}}{2\pi \mathrm{i}} \cdot \bs{\nabla_k} \log \det \left[H(\bs{k})-E\right] \quad \in \mathbb{Z},
\end{equation}
and
\begin{equation}
    (-1)^{\nu^{\mathbb{Z}_2}_{\gamma(\bs{B})}(E)} = \mathrm{sgn} \Bigg\{\frac{\mathrm{Pf}[Q(\bs{k}_f)]}{\mathrm{Pf}[Q(\bs{k}_i)]} \exp \bigg[-\frac{1}{2} \int_{\bs{k}_i}^{\bs{k}_f} d\bs{k} \cdot \mathrm{Tr}\left[Q(\bs{k})^{-1} \bs{\nabla_k} Q(\bs{k})\right]\Bigg\},
\end{equation}
respectively. In the last expression, we have used $Q(\bs{k}) = [H(\bs{k})-E]T$, where $T$ is the reciprocity operator (see below or main text), $\mathrm{Pf}(M)$ is the Pfaffian of the anti-symmetric matrix $M$, and we have introduced a reciprocal decomposition of the BZ loop $\gamma(\bs{B}) = \{\bs{k}_i \rightarrow \bs{k}_f\} \cup \{-\bs{k}_f \rightarrow -\bs{k}_i\}$ where $\bs{k}_i = -\bs{k}_i+\bs{G}$ and $\bs{k}_f = -\bs{k}_f+\bs{G}$ must hold for a 2D reciprocal lattice vector $\bs{G}$. For consistency, we note that $w_x(E,k_y)$ as defined in Eq.~(11) of the main text satisfies $w_x(E,\pi) = \nu^{\mathbb{Z}}_{\gamma(\hat{y})}(E)$, where the Burgers vector $\hat{y}$ is a unit vector in $y$-direction. Moreover, $v_x(E,k_y)$ as defined in Eq.~(7) of the main text satisfies $v_x(E,\pi)=(-1)^{\nu^{\mathbb{Z}_2}_{\gamma(\hat{y})}(E)}$.

In order to prove Eq.~\eqref{eq: nonhermcorrespondence}, we begin by recalling that a nH Bloch Hamiltonian $H(\bs{k})$ naturally gives rise to an extended Hermitian Hamiltonian $\tilde{H}(\bs{k})$. Given a point gap at complex energy $E$, such an extended Hamiltonian is defined via
\begin{equation} \label{eq: extendedH}
    \tilde{H}(\bs{k}) = \begin{pmatrix}
    0 & H(\bs{k})-E \\
    H(\bs{k})^\dagger -E^* & 0
    \end{pmatrix}.
\end{equation}
Irrespective of $E$, the extended Hamiltonian automatically satisfies a chiral (anti-)symmetry
\begin{equation} \label{eq: chiralsym}
C\tilde{H}(\bs{k}) C^\dagger = - \tilde{H}(\bs{k}), \quad C= \begin{pmatrix} \mathbb{1} & 0 \\ 0 & - \mathbb{1} \end{pmatrix},
\end{equation}
which implies that its eigenvalues are either exactly zero or must come in pairs $(\epsilon,-\epsilon)$. Hence, when $H(\bs{k})$ has no special symmetry itself, the corresponding $\tilde{H}(\bs{k})$ lies in Altland-Zirnbauer symmetry class AIII. On the other hand, if $H(\bs{k})$ is reciprocal so that
\begin{equation} \label{eq:reciprocity}
T H(\bs{k})^{\mathrm{T}} T^\dagger = H(-\bs{k}),
\end{equation}
where $T$ is a unitary operator that satisfies $T T^* = -1$, then the extended Hermitian Hamiltonian has a conventional time-reversal symmetry
\begin{equation}
\tilde{T}\tilde{H}(\bs{k}) \tilde{T}^\dagger = \tilde{H}(-\bs{k}), \quad \tilde{T} = \begin{pmatrix} 0 & T \\ T & 0 \end{pmatrix} \mathit{K}.
\end{equation}
Here, $\mathit{K}$ denotes complex conjugation, so that $\tilde{T}$ is anti-unitary as required for time-reversal. Since $\tilde{T}$ and $C$ anti-commute, any $\tilde{H}(\bs{k})$ associated with a reciprocal nH Hamiltonian lies in Altland-Zirnbauer symmetry class DIII.

Importantly, because the mapping in Eq.~\eqref{eq: extendedH} is on-site in real space, inserting a pair of dislocations into a nH insulator described by $H(\bs{k})$ translates into inserting a pair of dislocations into a Hermitian insulator described by $\tilde{H}(\bs{k})$. The dislocation response of the nH Hamiltonian $H(\bs{k})$ can therefore be induced from two ingredients: (1) the response of the Hermitian Hamiltonian $\tilde{H}(\bs{k})$ and (2) a mapping between the response of $\tilde{H}(\bs{k})$ and the response of $H(\bs{k})$. We discuss these two ingredients in the following.

(1) \emph{Dislocation response of $\tilde{H}(\bs{k})$}. We observe that the invariants $\nu^{\mathbb{Z}}_{\gamma(\bs{B})}(0)$ and $\nu^{\mathbb{Z}_2}_{\gamma(\bs{B})}(0)$ are simply the 1D invariants of Hermitian topological insulators in Altland-Zirnbauer symmetry classes AIII and DIII, respectively: When we identify $\gamma(\bs{B})$ as the Brillouin zone of a 1D Hermitian wire with chiral symmetry, then $\nu^{\mathbb{Z}}_{\gamma(\bs{B})}(0)$ counts the number of protected zero-energy end states, and $\nu^{\mathbb{Z}_2}_{\gamma(\bs{B})}(0)$ counts the parity of gapless Kramers pair end states in presence of time-reversal symmetry. (Here, we have set the reference energy to $E=0$, it is only well-defined relative to a given nH Hamiltonian.) Refs.~\onlinecite{TeoKaneDefect,ran2010weak} related these invariants to the presence of dislocation-localized midgap states in 2D Hermitian topological insulators in the Altland-Zirnbauer symmetry classes AIII and DIII. Specifically, the following correspondence holds:
\begin{equation} \label{eq: hermcorrespondenceran}
\begin{rcases}
  \nu^{\mathbb{Z}}_{\gamma(\bs{B})}(0) \neq 0 \quad &\leftrightarrow \quad \nu^{\mathbb{Z}}_{\gamma(\bs{B})}(0) \, \text{zero-modes} \\
    \nu^{\mathbb{Z}_2}_{\gamma(\bs{B})}(0) = 1 \quad &\leftrightarrow \quad \text{two Kramers-paired zero-modes}
\end{rcases} \text{are bound to dislocations with Burgers vector } \bs{B}.
\end{equation}

(2) \emph{Relationship between $\tilde{H}(\bs{k})$ and $H(\bs{k})$}. To connect the dislocation response of the extended Hermitian Hamiltonian $\tilde{H}(\bs{k})$ with that of the corresponding nH Hamiltonian $H(\bs{k})$, we first note that $E$ can be chosen freely in Eq.~\eqref{eq: extendedH}, as long as it lies in a point gap of $H(\bs{k})$ -- this requirement translates to a finite energy gap of $\tilde{H}(\bs{k})$. Hence, since $\nu^{\mathbb{Z}}_{\gamma(\bs{B})}(E)$ and $\nu^{\mathbb{Z}_2}_{\gamma(\bs{B})}(E)$ are independent of the reference energy $E$ as long as it lies in the point gap, the dislocation response of $\tilde{H}(\bs{k})$ is also independent of $E$. We can now employ the results of Refs.~\onlinecite{TopoSkin20,okuma2020,okuma2021}, where it was shown that the presence of \emph{exact} zero-energy states (Kramers pairs) in the spectrum of the extended Hermitian Hamiltonian for all $E$ inside the point gap is equivalent to a $\mathbb{Z}$ ($\mathbb{Z}_2$) skin effect of the non-Hermitian Hamiltonian. Here, by \emph{exact}, we mean that these zero-modes must be eigenstates of the chiral symmetry in Eq.~\eqref{eq: chiralsym}, as is the case for the protected 0D zero-modes implied in Eq.~\eqref{eq: hermcorrespondenceran}. Hence, the conditions in Eq.~\eqref{eq: hermcorrespondenceran} imply a skin effect in the nH model. Since the system with dislocation is indistinguishable from the pristine crystal described by $H(\bs{k})$ except at the dislocations themselves, this skin effect can only locate at the dislocations cores. Correspondingly, we deduce that $H(\bs{k})$, when put on a lattice hosting a pair of dislocations, exhibits a $\mathbb{Z}$ ($\mathbb{Z}_2$) DNHSE as long as the first (second) condition in Eq.~\eqref{eq: hermcorrespondenceran} is fulfilled. This concludes our proof of Eq.~\eqref{eq: nonhermcorrespondence}.

%%%%%%%%%%%%%%%%%%%%%%%%%%%%%%%%%
%%%%%%%%%%%%%%%%%%%%%%%%%%%%%%%%%

%---------------------------------------
%SUPPLEMENTAL BIBLIOGRAPHY
%---------------------------------------

\bibliography{refs}

%---------------------------------------
%---------------------------------------

\end{document}