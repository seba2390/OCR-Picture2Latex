\documentclass[sigconf]{acmart}

\usepackage{booktabs} % For formal tables
\usepackage{amsmath,amsfonts,amssymb}
\usepackage{eqnarray}
\usepackage{mathtools}
\usepackage{balance} 
\usepackage{mathrsfs}
\usepackage{multirow}
\usepackage{csquotes}
\usepackage{bbm}
\usepackage{enumitem} 
\usepackage{amsmath}
\usepackage{colortbl}
%\usepackage{arydshln}

% \usepackage{algorithmicx}
% \usepackage{algpseudocode}
% \usepackage{algorithm}
\usepackage[]{algorithm2e}
\usepackage{graphics}
\usepackage{subfigure}
\usepackage{units}
\usepackage{csquotes}
\usepackage{todonotes}

\newcommand{\paragraphHdNospace}[1] {\noindent\textbf{#1.}} % for initial headings (no extra spacing)
\newcommand{\paragraphHd}[1] {\vspace{1.2mm}\noindent\textbf{#1.}} % for subsequent headings (small space added)
\newcommand{\ra}{\renewcommand{\arraystretch}{1.2}}

\DeclareMathOperator*{\argmax}{arg\!\max}
\DeclareMathOperator*{\argmin}{arg\!\min}
\DeclareMathOperator{\Tr}{\mathit{Tr}}
\makeatletter
\def\@fnsymbol#1{\ensuremath{\ifcase#1\or *\or \dagger\or \ddagger\or
   \mathsection\or \mathparagraph\or \|\or **\or \dagger\dagger
   \or \ddagger\ddagger \else\@ctrerr\fi}}
\makeatother
 
\newcommand{\specialcell}[2][c]{%
  \begin{tabular}[#1]{@{}c@{}}#2\end{tabular}}
\newcommand{\up}{$^\blacktriangle$}
\newcommand{\dn}{$^\blacktriangledown$}
\newcommand{\greyrule}{\arrayrulecolor{black!30}\midrule\arrayrulecolor{black}}

\makeatletter
\renewcommand*\env@matrix[1][*\c@MaxMatrixCols c]{%
  \hskip -\arraycolsep
  \let\@ifnextchar\new@ifnextchar
  \array{#1}}
\makeatother

% Copyright
\copyrightyear{2019}
\acmYear{2019}
\setcopyright{acmcopyright}
\acmConference[SIGIR '19]{Proceedings of the 42nd International ACM SIGIR Conference on Research and Development in Information Retrieval}{July 21--25, 2019}{Paris, France}
\acmBooktitle{Proceedings of the 42nd International ACM SIGIR Conference on Research and Development in Information Retrieval (SIGIR '19), July 21--25, 2019, Paris, France}
\acmPrice{15.00}
\acmDOI{10.1145/3331184.3331316}
\acmISBN{978-1-4503-6172-9/19/07}
\fancyhead{}

% \settopmatter{printacmref=false}

% These commands are optional
%\acmBooktitle{Transactions of the ACM Woodstock conference}

\begin{document}

% TODO title
% or? Improving Weak Supervision of Neural Information Retrieval Models by Modeling Query-Document Interactions
% ? Enabling the use of content-based weak supervision through interaction filtering [needs to mention NIR though?]
% ? Filtering Interactions to Improve Content-Based Weak Supervision for Neural Information Retrieval Models
% ? Improving Content-Based Weak Supervision for Neural Models through Interaction Filtering
\title{Content-Based Weak Supervision for Ad-Hoc Re-Ranking}

\author{Sean MacAvaney}
\affiliation{%
  \institution{IRLab, Georgetown University}
}
\email{sean@ir.cs.georgetown.edu}

\author{Andrew Yates}
\affiliation{%
  \institution{Max Planck Institute for Informatics}
}
\email{ayates@mpi-inf.mpg.de}

\author{Kai Hui}
\authornote{Work conducted while the author was at the Max Planck Institute for Informatics.}
\affiliation{%
  \institution{Amazon}
}
\email{kaihuibj@amazon.com}

\author{Ophir Frieder}
\affiliation{%
  \institution{IRLab, Georgetown University}
}
\email{ophir@ir.cs.georgetown.edu}

% The default list of authors is too long for headers.
% \renewcommand{\shortauthors}{S. MacAvaney et al.}

\begin{abstract}
One challenge with neural ranking is the need for a large amount of manually-labeled relevance judgments for training. In contrast with prior work, we examine the use of weak supervision sources for training that yield pseudo query-document \textit{pairs} that already exhibit relevance (e.g., newswire headline-content pairs and encyclopedic heading-paragraph pairs). We also propose filtering techniques to eliminate training samples that are too far out of domain using two techniques: a heuristic-based approach and novel supervised filter that re-purposes a neural ranker. Using several leading neural ranking architectures and multiple weak supervision datasets, we show that these sources of training pairs are effective on their own (outperforming prior weak supervision techniques), and that filtering can further improve performance.
\end{abstract}







%
% The code below should be generated by the tool at
% http://dl.acm.org/ccs.cfm
% Please copy and paste the code instead of the example below. 
%
%\begin{CCSXML}
%<ccs2012>
%<concept>
%<concept_id>10002951.10003317</concept_id>
%<concept_desc>Information systems~Information retrieval</concept_desc>
%<concept_significance>500</concept_significance>
%</concept>
%<concept>
%<concept_id>10002951.10003317.10003338.10003342</concept_id>
%<concept_desc>Information systems~Similarity measures</concept_desc>
%<concept_significance>500</concept_significance>
%</concept>
%<concept>
%<concept_id>10010147.10010257.10010258.10010260</concept_id>
%<concept_desc>Computing methodologies~Unsupervised learning</concept_desc>
%<concept_significance>500</concept_significance>
%</concept>
%</ccs2012>
%\end{CCSXML}

%\ccsdesc[500]{Information systems~Information retrieval}
%\ccsdesc[500]{Information systems~Similarity measures}
%\ccsdesc[500]{Computing methodologies~Unsupervised learning}


%\keywords{, Weak supervision}


\maketitle

% Introduction
\IEEEraisesectionheading{\section{Introduction}}

\IEEEPARstart{V}{ision} system is studied in orthogonal disciplines spanning from neurophysiology and psychophysics to computer science all with uniform objective: understand the vision system and develop it into an integrated theory of vision. In general, vision or visual perception is the ability of information acquisition from environment, and it's interpretation. According to Gestalt theory, visual elements are perceived as patterns of wholes rather than the sum of constituent parts~\cite{koffka2013principles}. The Gestalt theory through \textit{emergence}, \textit{invariance}, \textit{multistability}, and \textit{reification} properties (aka Gestalt principles), describes how vision recognizes an object as a \textit{whole} from constituent parts. There is an increasing interested to model the cognitive aptitude of visual perception; however, the process is challenging. In the following, a challenge (as an example) per object and motion perception is discussed. 



\subsection{Why do things look as they do?}
In addition to Gestalt principles, an object is characterized with its spatial parameters and material properties. Despite of the novel approaches proposed for material recognition (e.g.,~\cite{sharan2013recognizing}), objects tend to get the attention. Leveraging on an object's spatial properties, material, illumination, and background; the mapping from real world 3D patterns (distal stimulus) to 2D patterns onto retina (proximal stimulus) is many-to-one non-uniquely-invertible mapping~\cite{dicarlo2007untangling,horn1986robot}. There have been novel biology-driven studies for constructing computational models to emulate anatomy and physiology of the brain for real world object recognition (e.g.,~\cite{lowe2004distinctive,serre2007robust,zhang2006svm}), and some studies lead to impressive accuracy. For instance, testing such computational models on gold standard controlled shape sets such as Caltech101 and Caltech256, some methods resulted $<$60\% true-positives~\cite{zhang2006svm,lazebnik2006beyond,mutch2006multiclass,wang2006using}. However, Pinto et al.~\cite{pinto2008real} raised a caution against the pervasiveness of such shape sets by highlighting the unsystematic variations in objects features such as spatial aspects, both between and within object categories. For instance, using a V1-like model (a neuroscientist's null model) with two categories of systematically variant objects, a rapid derogate of performance to 50\% (chance level) is observed~\cite{zhang2006svm}. This observation accentuates the challenges that the infinite number of 2D shapes casted on retina from 3D objects introduces to object recognition. 

Material recognition of an object requires in-depth features to be determined. A mineralogist may describe the luster (i.e., optical quality of the surface) with a vocabulary like greasy, pearly, vitreous, resinous or submetallic; he may describe rocks and minerals with their typical forms such as acicular, dendritic, porous, nodular, or oolitic. We perceive materials from early age even though many of us lack such a rich visual vocabulary as formalized as the mineralogists~\cite{adelson2001seeing}. However, methodizing material perception can be far from trivial. For instance, consider a chrome sphere with every pixel having a correspondence in the environment; hence, the material of the sphere is hidden and shall be inferred implicitly~\cite{shafer2000color,adelson2001seeing}. Therefore, considering object material, object recognition requires surface reflectance, various light sources, and observer's point-of-view to be taken into consideration.


\subsection{What went where?}
Motion is an important aspect in interpreting the interaction with subjects, making the visual perception of movement a critical cognitive ability that helps us with complex tasks such as discriminating moving objects from background, or depth perception by motion parallax. Cognitive susceptibility enables the inference of 2D/3D motion from a sequence of 2D shapes (e.g., movies~\cite{niyogi1994analyzing,little1998recognizing,hayfron2003automatic}), or from a single image frame (e.g., the pose of an athlete runner~\cite{wang2013learning,ramanan2006learning}). However, its challenging to model the susceptibility because of many-to-one relation between distal and proximal stimulus, which makes the local measurements of proximal stimulus inadequate to reason the proper global interpretation. One of the various challenges is called \textit{motion correspondence problem}~\cite{attneave1974apparent,ullman1979interpretation,ramachandran1986perception,dawson1991and}, which refers to recognition of any individual component of proximal stimulus in frame-1 and another component in frame-2 as constituting different glimpses of the same moving component. If one-to-one mapping is intended, $n!$ correspondence matches between $n$ components of two frames exist, which is increased to $2^n$  for one-to-any mappings. To address the challenge, Ullman~\cite{ullman1979interpretation} proposed a method based on nearest neighbor principle, and Dawson~\cite{dawson1991and} introduced an auto associative network model. Dawson's network model~\cite{dawson1991and} iteratively modifies the activation pattern of local measurements to achieve a stable global interpretation. In general, his model applies three constraints as it follows
\begin{inlinelist}
	\item \textit{nearest neighbor principle} (shorter motion correspondence matches are assigned lower costs)
	\item \textit{relative velocity principle} (differences between two motion correspondence matches)
	\item \textit{element integrity principle} (physical coherence of surfaces)
\end{inlinelist}.
According to experimental evaluations (e.g.,~\cite{ullman1979interpretation,ramachandran1986perception,cutting1982minimum}), these three constraints are the aspects of how human visual system solves the motion correspondence problem. Eom et al.~\cite{eom2012heuristic} tackled the motion correspondence problem by considering the relative velocity and the element integrity principles. They studied one-to-any mapping between elements of corresponding fuzzy clusters of two consecutive frames. They have obtained a ranked list of all possible mappings by performing a state-space search. 



\subsection{How a stimuli is recognized in the environment?}

Human subjects are often able to recognize a 3D object from its 2D projections in different orientations~\cite{bartoshuk1960mental}. A common hypothesis for this \textit{spatial ability} is that, an object is represented in memory in its canonical orientation, and a \textit{mental rotation} transformation is applied on the input image, and the transformed image is compared with the object in its canonical orientation~\cite{bartoshuk1960mental}. The time to determine whether two projections portray the same 3D object
\begin{inlinelist}
	\item increase linearly with respect to the angular disparity~\cite{bartoshuk1960mental,cooperau1973time,cooper1976demonstration}
	\item is independent from the complexity of the 3D object~\cite{cooper1973chronometric}
\end{inlinelist}.
Shepard and Metzler~\cite{shepard1971mental} interpreted this finding as it follows: \textit{human subjects mentally rotate one portray at a constant speed until it is aligned with the other portray.}



\subsection{State of the Art}

The linear mapping transformation determination between two objects is generalized as determining optimal linear transformation matrix for a set of observed vectors, which is first proposed by Grace Wahba in 1965~\cite{wahba1965least} as it follows. 
\textit{Given two sets of $n$ points $\{v_1, v_2, \dots v_n\}$, and $\{v_1^*, v_2^* \dots v_n^*\}$, where $n \geq 2$, find the rotation matrix $M$ (i.e., the orthogonal matrix with determinant +1) which brings the first set into the best least squares coincidence with the second. That is, find $M$ matrix which minimizes}
\begin{equation}
	\sum_{j=1}^{n} \vert v_j^* - Mv_j \vert^2
\end{equation}

Multiple solutions for the \textit{Wahba's problem} have been published, such as Paul Davenport's q-method. Some notable algorithms after Davenport's q-method were published; of that QUaternion ESTimator (QU\-EST)~\cite{shuster2012three}, Fast Optimal Attitude Matrix \-(FOAM)~\cite{markley1993attitude} and Slower Optimal Matrix Algorithm (SOMA)~\cite{markley1993attitude}, and singular value decomposition (SVD) based algorithms, such as Markley’s SVD-based method~\cite{markley1988attitude}. 

In statistical shape analysis, the linear mapping transformation determination challenge is studied as Procrustes problem. Procrustes analysis finds a transformation matrix that maps two input shapes closest possible on each other. Solutions for Procrustes problem are reviewed in~\cite{gower2004procrustes,viklands2006algorithms}. For orthogonal Procrustes problem, Wolfgang Kabsch proposed a SVD-based method~\cite{kabsch1976solution} by minimizing the root mean squared deviation of two input sets when the determinant of rotation matrix is $1$. In addition to Kabsch’s partial Procrustes superimposition (covers translation and rotation), other full Procrustes superimpositions (covers translation, uniform scaling, rotation/reflection) have been proposed~\cite{gower2004procrustes,viklands2006algorithms}. The determination of optimal linear mapping transformation matrix using different approaches of Procrustes analysis has wide range of applications, spanning from forging human hand mimics in anthropomorphic robotic hand~\cite{xu2012design}, to the assessment of two-dimensional perimeter spread models such as fire~\cite{duff2012procrustes}, and the analysis of MRI scans in brain morphology studies~\cite{martin2013correlation}.

\subsection{Our Contribution}

The present study methodizes the aforementioned mentioned cognitive susceptibilities into a cognitive-driven linear mapping transformation determination algorithm. The method leverages on mental rotation cognitive stages~\cite{johnson1990speed} which are defined as it follows
\begin{inlinelist}
	\item a mental image of the object is created
	\item object is mentally rotated until a comparison is made
	\item objects are assessed whether they are the same
	\item the decision is reported
\end{inlinelist}.
Accordingly, the proposed method creates hierarchical abstractions of shapes~\cite{greene2009briefest} with increasing level of details~\cite{konkle2010scene}. The abstractions are presented in a vector space. A graph of linear transformations is created by circular-shift permutations (i.e., rotation superimposition) of vectors. The graph is then hierarchically traversed for closest mapping linear transformation determination. 

Despite of numerous novel algorithms to calculate linear mapping transformation, such as those proposed for Procrustes analysis, the novelty of the presented method is being a cognitive-driven approach. This method augments promising discoveries on motion/object perception into a linear mapping transformation determination algorithm.




% Background
\section{Background and Related Work}~\label{sec:background}
%This section presents the background on MDE, ML, and MDE for systems with ML components. We further present the related work on the existing secondary and relevant studies.
\subsection{Model-driven Engineering}~\label{subsec:MDEBackground}
%The word \textit{model} originates from the Latin word \textit{modulus}, which means a measure, pattern, or example to follow~\cite{ludewig2003models}. 
%While modeling is relatively new to software engineering, it has been successfully applied for a long time in several traditional engineering domains~\cite{selic2012will,bucchiarone2020grand}. 
Model-driven Engineering (MDE) is a software development methodology that relies on models as the primary artifacts that drive the development process~\cite{ciccozzi2019execution, almonte2021recommender,hutchinson2011model}. This differs from traditional software development processes such as waterfall and agile, where the focus is on development phases like requirements engineering, design, and implementation, and models are only used as auxiliary artifacts to support these activities and serve as documentation~\cite{ciccozzi2019execution}. 
%In contrast to traditional software engineering using waterfall or agile methodology, where the focus is on the different phases of development, e.g., requirements engineering, design, implementation, and quality assurance, and the models are used to aid in requirements analysis or design, in MDE models are the primary artifact. 
The focus of MDE is on the continual refinement and transformation of models, beginning with computation-independent models (CIMs), to platform-independent models (PIMs) and then platform-specific models (PSMs)~\cite{brambilla2017model}. Finally, these models are transformed into code, documentation, configurations, and tests for the software system.

MDE relies on two key aspects: abstraction and automation~\cite{mohagheghi2009mde}. Models are abstractions of complex entities; they hide unwanted information so modelers can easily focus on areas of interest~\cite{schmidt2006model, brambilla2017model}. 
%Currently, MDE is the state-of-the-art in software abstraction~\cite{hutchinson2011model} by reducing complexity and offering a more intuitive and natural way to define software compared to programming languages~\cite{ciccozzi2019execution}. 
In MDE, models are automatically transformed into artifacts such as code, documentation, and other models to achieve various goals such as merging, translation, refinement, refactoring, or alignment~\cite{brambilla2017model}. These transformations help reduce developers' manual effort and production time by generating executable artifacts -- leading to improved software quality, reduced complexity, and decreased development time and effort~\cite{kelly2008domain}. There are two types of transformations in MDE: 1) Model-to-Text (M2T) transformations, for a given input model a M2T transformation produces a textual artifact such as code or documentation as output; and ) Model-to-model (M2M) transformations, for a given input model an M2M transformation produces a different kind of model, for example translating a model from one language to another~\cite{brambilla2017model}.

A model is created in a modeling language, conforming to a meta-model that defines the syntax and semantics of that language. There are two types of modeling languages: general-purpose languages (GPL) and domain-specific languages (DSL). GPLs are intended for modeling generic concepts applicable to multiple domains; some examples include the Unified Modeling Language (UML)~\cite{eriksson2003uml}, Petri-nets~\cite{peterson1977petri} and finite state machines~\cite{wagner2006modeling}. On the other hand, a DSL has modeling concepts tailored to a specific domain or context, like SysML for embedded systems, HTML for web page development, and SQL for database queries~\cite{brambilla2017model}.

While exploring the literature, one encounters terms similar to MDE: examples include model-driven architecture (MDA), model-driven development (MDD), and model-based engineering (MBE). MDA is an architectural standard~\cite{mda} developed by the Object Management Group (OMG) \cite{omg} for MDD. MDD refers to automatically generating artifacts from models, whereas MDE has a broader scope and includes analysis, validation~\cite{almonte2021recommender}, interoperability of artifacts and reverse engineering \cite{brambilla2017model}. MBE is a lighter version of MDE, where models are not necessarily the central focus of the engineering process; however, they provide critical support~\cite{brambilla2017model}. This SLR primarily focuses on MDE.

\subsection{Machine Learning}
Machine Learning (ML) is a branch of Artificial Intelligence (AI) that enables machines to learn patterns from data without being explicitly programmed~\cite{samuel1959machine}. ML algorithms are fed with existing data to \textit{train} them and produce an ML model. This trained ML model then has the capability to \textit{infer}, i.e., predict outcomes for new data inputs or also commonly known as \emph{ML model inference}~\cite{mueller2021machine}. For example, an ML model trained on stock prices for a company till September 2023 can predict stock prices in the following months. ML is preferable when solving problems that would require very complex and difficult-to-maintain traditional algorithms~\cite{geron2022hands}. Since ML algorithms can learn autonomously, they reduce complexity and facilitate easier maintenance~\cite{geron2022hands}. This ability of ML to minimise complexity, learn from changing data, and make future predictions is immensely valuable for businesses~\cite{lee2020machine}. According to a recent survey~\cite{rackspace2023report}, organizations report that applying ML increases employee efficiency by 20\%, innovation by 17\%, and lowers costs by 16\% -- leading to increased adoption of ML in practical settings~\cite{rackspace2023report}.

ML can further be divided into three broad categories: supervised learning, unsupervised learning, and reinforcement learning. The most suitable ML approach depends on the specific problem and data.
%
Supervised learning is when an ML algorithm is trained on a labeled dataset that has labels to define the meaning of data~\cite{mueller2021machine}. For example, a dataset with images labeled as ``cat'' or ``not cat'' images. Supervised learning algorithms learn to make classifications or predictions by learning patterns and relationships in labeled data~\cite{lee2020machine,mueller2021machine}. When the labels are discrete, this is known as \textit{classification} and when labels are continuous, this is known as \textit{regression}~\cite{mueller2021machine}. Once the algorithm is trained, the performance is evaluated on unseen or test data. Some popular supervised learning algorithms include linear regression, decision trees, naive Bayes classifier, support vector machines (SVM), random forest, and artificial neural networks (ANNs)~\cite{lee2020machine}. Supervised model applications include fraud detection and recommender systems~\cite{mueller2021machine}. 

Unsupervised learning is when an ML algorithm is trained on an unlabeled dataset with few or no labels to define the meaning of data~\cite{mueller2021machine,lee2020machine}. Unsupervised learning algorithms attempt to understand hidden patterns in data and group similar data together creating a classification of the data~\cite{mueller2021machine}. Unsupervised learning works without any guidance, hence it is most suitable for large volumes of data when classifications are unknown and data cannot be labeled~\cite{mueller2021machine}. Evaluating the performance of such algorithms can be challenging due to the lack of ground truth. Some popular unsupervised techniques include clustering, k-means, principal component analysis, and association rules~\cite{lee2020machine}. Applications of unsupervised models include customer segmentation and clustering user reviews~\cite{mueller2021machine}.

 Reinforcement learning is when an ML algorithm receives feedback on actions to guide the behavior toward an optimal outcome~\cite{mueller2021machine, lee2020machine}. Reinforcement learning algorithms are not trained with datasets; instead, they learn from trial and error in a simulated environment or a real-world environment~\cite{mueller2021machine}. Desired behaviors are rewarded and reinforcement learning algorithms attempt to maximize rewards through successful decisions~\cite{lee2020machine,mueller2021machine}. These algorithms are most suitable when sequential decision-making is required, interaction with an environment is possible and feedback is available. %However, reinforcement learning can be expensive since the algorithms require a large number of interactions with the environment to learn effectively. 
 Some popular reinforcement learning algorithms are Q-learning, temporal difference learning, hierarchal reinforcement learning, and policy gradient~\cite{lee2020machine}. Applications of reinforcement learning include robotics, self-driving cars, and game playing~\cite{lee2020machine}.
 
\subsection{Model-driven Engineering for Machine Learning (MDE4ML)}
%Models are a significant element of both MDE and ML. In MDE, models describe software systems in all phases of their life-cycle: requirements, design, implementation, testing and evolution~\cite{moin2022model}. ML models are mathematical models that learn patterns in data to make predictions~\cite{moin2022model}. 
Developing and managing systems with ML models and components is challenging. 
Some aspects of this complexity are immature requirements specification~\cite{kuwajima2020engineering, ahmad2023requirements}, constantly evolving data~\cite{baumann2022dynamic}, lack of ML domain knowledge~\cite{yohannis2022towards}, integration with traditional software \cite{atouani2021artifact}, responsible use of ML~\cite{yohannis2022towards}, and deployment and maintenance of ML models~\cite{kourouklidis2021model, langford2021modalas}. 

These complexities introduce several challenges. For example, Nils Baumann et al.~\cite{baumann2022dynamic} describe how challenging it is to handle changing datasets; ML engineers have to manually merge new and old datasets and re-train the entire ML model;  Benjamin Jahi et al. \cite{jahic2023semkis} point out how challenging it is to describe the dataset and neural network requirements to satisfy customer expectations;  Benjamin Benni et al. \cite{benni2019devops} state how the development of a correct ML pipeline is a highly demanding task, data scientists must have knowledge and experience to go through numerous data pre-processing and ML models to select the best one; and Kaan Koseler et al. \cite{koseler2019realization} mention the difficulties developers face when attempting to use ML techniques with big data, developers need to acquire knowledge of the problem space, domain and ML concepts. There is a need for solutions to efficiently and effectively address these challenges~\cite{raedler2023model}.
%All these challenges point towards the need for a technique that can efficiently and effectively address them.

A synergy between MDE and ML development exists, where software models are leveraged to drive the development and management of ML components~\cite{safdar2022modlf, yohannis2022towards, kourouklidis2021model}. This should not to be confused with AI or ML for MDE (AI4MDE), where intelligent agents and recommenders support users in modeling and related activities \cite{almonte2021recommender, gil2021artificial, boubekeur2020towards, saini2019teaching}. The application of MDE for ML-based systems (MDE4ML) offers many potential benefits to developers, such as reduced complexity~\cite{kourouklidis2021model, bucchiarone2020grand}, development effort, and time~\cite{yohannis2022towards,gatto2019modeling}. Domain experts, software engineers and ML novices can also take advantage of ML through the abstraction and automation of MDE \cite{shi2022feature,moin2022supporting, bucchiarone2020grand}. Additionally, MDE can also improve the quality of the ML-based system through easier maintainability, scalability~\cite{selic2003pragmatics}, reusability, and interoperability~\cite{brambilla2017model}.

\subsection{Key MDE4ML Related Work}
MDE4ML has received growing attention from researchers in recent years. We found six relevant secondary studies comprising SLRs, scoping reviews, and surveys. In their SLR \cite{raedler2023model}, the authors identify 15 primary studies on MDE for AI and analyze them with respect to MDE practices for the development of AI systems and the stages of AI development aligned with CRoss Industry Standard Process for Data Mining (CRISP-DM) \cite{wirth2000crisp} methodology. However, this study only considers a small subset of studies and performs a shallow analysis with no details about goals, end-users, types of models, implemented tools, and evaluation. A second SLR \cite{zafar2017systematic} reviews 24 papers on MDE for ML in the context of Big data analytics. This study has a narrower scope compared to ours and provides only a brief overview of the models, approaches, tools, and frameworks in the studies. In a third SLR \cite{li2022can}, 31 studies on no/low code platforms for ML applications are reviewed. This study is limited to no/low code approaches and therefore misses out on many other MDE for ML studies. A scoping review is presented in \cite{mardani2023model} on MDE for ML in IoT applications. The study examines 68 studies in depth; however, the review focuses more on MDE for IoT applications and only four of the selected studies apply ML techniques. A preliminary survey on DSLs for ML in Big data is presented in \cite{portugal2016preliminary}, with an extended version in \cite{portugal2016survey}. These surveys do not follow a systematic review process, include studies only for big data, and briefly highlight the DSLs and frameworks in the studies. From the analysis of existing literature, we found that the available secondary studies consist of limited subsets of papers on MDE for ML, lack analysis of key areas like goals, end-users, ML aspects, MDE approach details, evaluation methods, and limitations, and often do not follow a systematic and rigorous review process. Therefore, we aim to address these gaps in this SLR.



% Method
\section{Method}\label{sec:method}
%

\subsection{Interaction-aware Human-Object Capture}\label{sec:human_capture}
Classical multi-view stereo reconstruction approaches \citep{Furukawa2013,Strecha2008,Newcombe2011,collet2015high} and recent neural rendering approaches \citep{Wu_2020_CVPR,NeuralVolumes,nerf} rely on multi-view dome based setup to achieve high-fidelity reconstruction and rendering results.
%
However, they suffer from both sparse-view inputs and occlusion of objects.
%
To this end, we propose a novel implicit human-object capture scheme to model the mutual influence between human and object from only sparse-view RGB inputs.

\noindent{\textbf{(a) Occlusion-aware Implicit Human Reconstruction.}}
For the human reconstruction, we perform a neural implicit geometry generation to jointly utilize both the pixel-aligned image features and global human motion priors with the aid of an occlusion-aware training data augmentation.
% 

Without dense RGB cameras and depth cameras, traditional multi-view stereo approaches \citep{collet2015high,motion2fusion} and depth-fusion approaches \citep{KinectFusion,UnstructureLan,robustfusion} can hardly reconstruct high-quality human meshes.
%
With implicit function approaches \citep{PIFU_2019ICCV,PIFuHD}, we can generate fine-detailed human meshes with sparse-view RGB inputs.
%
However, the occlusion from human-object interaction can still cause severe artifacts.
%
To end this, we thus utilize the pixel-aligned image features and global human motion priors.

% 
Specifically, we adopt the off-the-shelf instance segmentation approach \citep{Bolya_2019_ICCV} to obtain human and object masks, thus distinguishing the human and object separately from the sparse-view RGB input streams.
%
Meanwhile, we apply the parametric model estimation to provide human motion priors for our implicit human reconstruction.
%
We voxelized the mesh of this estimated human model to represent it with a volume field.

We give both the pixel-aligned image features and global human motion priors in volume representation to two different encoders of our implicit function, as shown in Fig. \ref{fig:pipeline} (a).
%
Different from \cite{2020phosa_Arrangements} with only a single RGB input, we use pixel-aligned image features from the multi-view inputs and concatenate them with our encoded voxel-aligned features.
%
We finally decode the pixel-aligned and voxel-aligned feature to occupancy values with a multilayer perceptron (MLP).

For each query 3D point $P$ on the volume grid, we follow PIFu \citep{saito2019pifu} to formulate the implicit function $f$ as:
\begin{align}
	f( \Phi(P),\Psi(P),Z(P)) & = \sigma : \sigma \in [0.0, 1.0],             \\
	\Phi(P)                  & = \frac{1}{n} \sum_{i}^{n}F_{I_{i}}(\pi_{i}(P)), \\
	\Psi(P)                  & =  G(F_{V},P),
\end{align}
where $p = \pi_{i}(P)$ denotes the projection of 3D point to camera view $i$, $F_{I_{i}}(x)= g(I_{i}(p))$ is the image feature at $p$.
%
$\Psi(P) = G(F_{V},P)$ denotes the voxel aligned features at $P$, $F_{V}$ is the voxel feature.
%
To better deal with occlusion, we introduce an occlusion-aware reconstruction loss to enhance the prediction at the occluded part of human.
% 
It is formulated as:
	\begin{align}
		 & \mathcal{L}_{\sigma} = \lambda_{occ}\sum_{t=1}^T \left\| \sigma_{occ}^{gt} - \sigma_{occ}^{pred} \right\|_2^2 + \lambda_{vis}\sum_{t=1}^T \left\| \sigma_{vis}^{gt} - \sigma_{vis}^{pred} \right\|_2^2.
	\end{align}

% 
Here, $\lambda_{occ}$ and $\lambda_{vis}$ represent the weight of occlusion points and visible points, respectively.
%
$\sigma_{occ}$ and $\sigma_{vis}$ are the training sampling points at the occlusion area and visible area.


\begin{figure}[t]
    \centering
    \includegraphics[width=\linewidth]{figures/data_augmentation}
    \vspace{-10pt}
    \caption{Illustration of our synthetic 3D data with both human and objects.}
    % \vspace{-1mm}
    \vspace{-15pt}
    \label{fig:DataAugmentation}
\end{figure}

\begin{figure*}[t]
	\centering
	\includegraphics[width=\linewidth]{figures/pipeline_net}
	\caption{Illustration of our layered human-object rendering approach, which not only includes a direction-aware neural texture blending scheme to encode the occlusion information explicitly but also adopts a spatial-temporal texture completion for the occluded regions based on the human motion priors.}
	\vspace{-10pt}
	\label{fig:pipeline_net}
\end{figure*}

For the detail of the parametric model estimation, we fit the parametric human model, SMPL \citep{SMPL2015}, to capture occluded human with the predicted 2D keypoints.
%
Specifically, we use Openpose \citep{Openpose} as our joint detector to estimate 2D human keypoints from sparse-view RGB inputs.
%
To estimate the pose/shape parameters of SMPL as our human prior for occluded human, we formulate the energy function $\boldsymbol{E}_{\mathrm{prior}}$ of this optimization as:
\begin{align} \label{eq:opt}
	\boldsymbol{E}_{\mathrm{prior}}(\boldsymbol{\theta}_t, \boldsymbol{\beta}) = \boldsymbol{E}_{\mathrm{2D}} + \lambda_{\mathrm{T}}\boldsymbol{E}_{\mathrm{T}}
\end{align}
% 
Here, $\boldsymbol{E}_{\mathrm{2D}}$ represents the re-projection constraint on 2D keypoints detected from sparse-view RGB inputs, while $\boldsymbol{E}_{\mathrm{T}}$ enforces the final pose and shape to be temporally smooth.
%
$\boldsymbol{\theta}_t$ is the pose parameters of frame $t$, while $\boldsymbol{\beta}$ is the shape parameters.
%
Note that this temporal smoothing enables globally consistent capture during the whole sequence, and benefits the parametric model estimation when some part of the body is gradually occluded.
%
We follow \cite{he2021challencap} to formulate the 2D term $\boldsymbol{E}_{\mathrm{2D}}$ and the temporal term $\boldsymbol{E}_{\mathrm{T}}$ under the sparse-view setting.
%

Moreover, we apply an occlusion-aware data augmentation to reduce the domain gap between our training set and the challenging human-object interaction testing set.
%
Specially, we randomly sample some objects from ShapeNet dataset~\cite{chang2015shapenet}.
%
We then randomly rotate and place them around human before training, as shown in Fig. \ref{fig:DataAugmentation}.
%
By simulating the occlusion of human-object interaction, our network is more robust to occluded human features.

With both the pixel-aligned image features and the statistical human motion priors under this occlusion-aware data augmentation training, our implicit function generates high-quality human meshes with only spare RGB inputs and occlusions from human-object interaction.

\noindent{\textbf{(b) Human-aware Object Tracking.}}
%
For the objects around the human, people recover them from depth maps~\cite{new2011kinect}, implicit fields~\cite{mescheder2019occupancy}, or semantic parts~\cite{chen2018autosweep}. We perform a template-based object alignment for the first frame and human-aware tracking to maintain temporal consistency and prevent the segmentation uncertainty caused by interaction. With the inspiration of PHOSA \cite{2020phosa_Arrangements}, we consider each object as a rigid body mesh.


To faithfully and robustly capture object in 3D space as time going, we introduce a human-aware tracking method.
%
Expressly, we assume objects are rigid bodies and transforming rigidly in the human-object interaction activities.
%
So the object mesh $O_{t}$ at frame $t$ can be represented as: $O_{t} = R_{t}O_{t-1}+T_{t}$.
% 
Based on the soft rasterization rendering~\cite{ravi2020accelerating}, the rotation $R_{t}$ and the translation $T_{t}$ can be naively optimized by comparing $\mathcal{L}_{2}$ norm between the rendered silhouette $S_{t}^{i}$ and object mask $\mathcal{M}o_{t}^{i}$.
%

Human is also an important cue to locate the object position.
%
From the 2D perspective, when objects are occluded by the human at a camera view, the  $\mathcal{L}_{2}$ loss between rendered silhouette and occluded mask will lead to the wrong object location due to the wrong guidelines at the occluded area.
%
So we remove the occluded area affected by human mask $\mathcal{M}h_{t}^{i}$ when computing the $\mathcal{L}_{2}$ loss.
%
From the 3D perspective, human can not interpenetrate an rigid object, so we also add an interpenetration loss $\mathcal{L}_{P}$~\cite{jiang2020mpshape} to regularize optimization. Our total object tracking loss is:
\begin{align}
	\mathcal{L}_{track} = \lambda_{1}\sum_{i=0}^{n}\| \mathcal{B}(\mathcal{M}h_{t}^{i}==0) \odot  S_{t}^{i} - \mathcal{M}o_{t}^{i}  \|  + \lambda_{2}\mathcal{L}_{P},
\end{align}
where n denotes view numbers, $\lambda_{1}$ denotes weight of silhouette loss, $\lambda_{2}$ denotes weight of interpenetration loss, $\mathcal{B}$ represents an binary operation, it returns 0 when the condition is true, else 1.
% 	}

Our implicit human-object capture utilizes both the pixel-aligned image features and global human motion priors with the aid of an occlusion-aware training data augmentation, and captures objects with the template-based alignment and the human-aware tracking to maintain temporal consistency and prevent the segmentation uncertainty caused by interaction. Thus, our approaches can generate high-quality human-object geometry with sparse inputs and occlusions.

\subsection{Layered Human-Object Rendering}\label{sec:rendering}
We introduce a neural human-object rendering pipeline to encode local fine-detailed human geometry and texture features from adjacent input views, so as to produce photo-realistic layered output in the target view, as illustrated in Fig. \ref{fig:pipeline_net}.

\begin{figure*}[t]
	\centering
	\includegraphics[width=\linewidth]{figures/gallery}
	\vspace{-20pt}
	\caption{The geometry and texture results of our proposed approach, which generates photo-realistic rendering results and high fidelity geometry on a various of sequences, such as rolling a box, playing with balls.}
    \vspace{-10pt}
	\label{fig:gallery}
\end{figure*}

\noindent{\textbf{(c) Direction-aware Neural Texture Blending.}} \label{sec:neuralBlending}
%
While traditional image-based rendering approaches always show the artifacts with the sparse-view texture blending, we follow \cite{NeuralHumanFVV2021CVPR} to propose a direction-aware neural texture blending approach to render photo-realistic human in the novel view.
% %
For a novel view image $I_{n}$, the linear combination of two source view $I_{1}$ and $I_{2}$ with blending weight map $W$ is formulate as:
\begin{align}
	I_{n} = W \cdot I_{1} + (1 - W) \cdot I_{2}.
\end{align}
However, in the sparse-view setting, the neural blending approach \citep{NeuralHumanFVV2021CVPR} can still generate unsmooth results. As the reason of these artifacts, the imbalance of angles between two source views with a novel view will lead to the imbalance wrapped image quality.
%

Different from \citet{NeuralHumanFVV2021CVPR}, we thus propose a direction-aware neural texture blending to eliminate such artifacts, as shown in Fig. \ref{fig:pipeline_net}.
%
The direction and angle between the two source views and target view will be an important cue for neural rendering quality, especially under occluded scenarios. 
%
Given novel view depth $D_{n}$ and source view depth $D_{1}$, $D_{2}$, we wrap them to the novel view $D_{1}^{n}$ and $D_{2}^{n}$, then compute the occlusion map $O_{i} = D_{n}- D_{i}^{n} (i=1,2)$.
%
Then, we unproject $D_{i}$ to point-clouds.
%
For each point $P$, we compute the cosine value between $\overrightarrow{c_{i}P}$ and $\overrightarrow{c_{n}P}$ to get angle map $A_{i}$, where $c_{i}$ denotes the optical center of source camera $i$, $c_{n}$ denotes the optical center of novel view camera.
%
Thus, we introduce a direction-aware blending network $\Theta_{DAN}$ to utilize global feature from image and local feature from human geometry to generate the blending weight map $W$, which can be formulated as:
\begin{align}
	W = \Theta_{DAN}(I_{1},O_{1},A_{1},I_{2},O_{2},A_{2}),
\end{align}
% 

\noindent{\textbf{(d) Spatial-temporal Texture Completion.}}
%
While human-object interaction activities consistently lead to occlusion, the missing texture on human, therefore, leads to severe artifacts for free-viewpoint rendering.
%
To end this, we propose a spatial-temporal texture completion method to generate a texture-completed proxy in the canonical human space.
%
We use the temporal and spatial information to complete the missing texture at view $i$ and time $t$ from different times and different views.

Specifically, we first use the non-rigid deformation to register an up-sampled SMPL model (41330 vertices) with the captured human meshes.
%	
Then, for each point on the proxy, we find the nearest visible points along with all views and all frames, then assign an interpolation color to this point.
% 
We thus generate a canonical human space with the fused texture.
%
For the occluded part of human in novel view, we render the texture-completed image and blend it with the neural rendering results in Sec. \ref{sec:neuralBlending} (c).

We utilize a layered human-object rendering strategy to render human-object together with the reconstruction and tracking of object.
%
For each frame, we render human with our novel neural texture blending while rendering objects through a classical graphics pipeline with color correction matrix (CCM).
%
To combine human and object rendering results, we utilize the depth buffer from the geometry of our human-object capture.
%

\noindent{\textbf{Training Strategy.}} To enable our sparse-view neural human performance rendering under human-object interaction, we need to train the direction-aware blending network $\Theta_{DAN}$ properly.
% 

We follow \citet{NeuralHumanFVV2021CVPR} to utilize 1457 scans from the Twindom dataset \cite{Twindom} to train our DAN $\Theta_{DAN}$ properly.
%
Differently, we randomly place the performers inside the virtual camera views and augment this dataset by randomly placing some objects from ShapeNet dataset~\cite{chang2015shapenet}.
%
By simulating the occlusion of human-object interaction, we make our network more robust to occluded human.
%
Our training dataset contains RGB images, depth maps and angle maps for all the views and models.

For the training of our direction-aware blending network $\Theta_{DAN}$, we set out to apply the following learning scheme to enable more robust blending weight learning.
%
The appearance loss function with the perceptual term ~\cite{Johnson2016Perceptual} is to make the blended texture as close as possible to the ground truth, which is formulated as:
\begin{align}
	\left.\mathcal{L}_{r g b}=\frac{1}{n} \sum_{j}^{n}
	\left(
	\left\|I_{r}^{j}-I_{g t}^{j}\right\|_{2}^{2}
% 	\right)
	+\left\|\varphi
	\left(
	I_{r}^{j}
	\right)
	-\varphi
	\left(
	I_{g t}^{j}
	\right)
	\right\|_{2}^{2}
	\right) \right.
\end{align}
where $I_{g t}$ is the ground truth RGB images; $\varphi(\cdot)$ denotes the output features of the third-layer of pre-trained VGG-19.

With the aid of such occlusion analysis, our texturing scheme maps the input adjacent images into a photo-realistic texture output of human-object activities in the target view through efficient blending weight learning, without requiring further per-scene training.

% % 
%Evaluation
\section{Evaluation} 
\label{sec.evaluation}

\subsection{Experimental setup}\label{sec.expsetting}

\textbf{Training sources.} We use the following four sources of training data to verify the effectiveness of our methods:
\begin{itemize}[leftmargin=*]
\item[-] \textbf{Query Log (AOL, ranking-based, $100k$ queries).}
This source uses the AOL query log~\cite{pass2006} as the basis for a ranking-based source, following the approach of~\cite{dehghani2017neural}.\footnote{
Distinct non-navigational queries from the AOL query log from March 1, 2006 to May 31, 2006 are selected. We randomly sample $100k$ of queries with length of at least 4. While \citeauthor{dehghani2017neural}~\cite{dehghani2017neural} used a larger number of queries to train their model, the state-of-the-art relevance matching models we evaluate do not learn term embeddings (as \cite{dehghani2017neural} does) and thus converge with fewer than $100k$ training samples.}
We retrieve ClueWeb09 documents for each query using the Indri\footnote{https://www.lemurproject.org/indri/} query likelihood~(QL) model. We fix $c^+=1$ and $c^-=10$ due to the expense of sampling documents from ClueWeb.

\item[-] \textbf{Newswire (NYT, content-based, $1.8m$ pairs).} 
We use the New York Times corpus~\cite{sandhaus2008new} as a content-based source, using headlines as pseudo queries and the corresponding content as pseudo relevant documents. We use BM25 to select the negative articles, retaining top $c^-=100$ articles for individual headlines.

\item[-] 
\textbf{Wikipedia (Wiki, content-based, $1.1m$ pairs).} 
Wikipedia article heading hierarchies and their corresponding paragraphs have been employed as a training set for the \textsc{Trec} Complex Answer Retrieval (CAR) task~\cite{Nanni2017BenchmarkFC,macavaney2018overcoming}.
We use these pairs as a content-based source, assuming that the hierarchy of headings is a relevant query for the paragraphs under the given heading.
Heading-paragraph pairs from train fold 1 of the \textsc{Trec} CAR dataset~\cite{Dietz2017} (v1.5) are used. We generate negative heading-paragraph pairs for each heading using BM25 ($c^-=100$).

\item[-] \textbf{Manual relevance judgments (WT10).}
We compare the ranking-based and content-based sources with a data source that consists of relevance judgments generated by human assessors. In particular, manual judgments from 2010 \textsc{Trec} Web Track ad-hoc task (WT10) are employed, which includes $25k$ manual relevance judgments ($5.2k$ relevant) for 50 queries (topics + descriptions, in line with~\cite{hui2017pacrr,guo2016deep}). This setting represents a new target domain, with limited (yet still substantial) manually-labeled data.

\end{itemize}


\textbf{Training neural IR models.}
We test our method using several state-of-the-art neural IR models (introduced in Section~\ref{sec.background.nir}):
PACRR~\cite{hui2017pacrr},
Conv-KNRM~\cite{convknrm}, and
KNRM~\cite{xiong2017end}.\footnote{By using these stat-of-the-art architectures, we are using stronger baselines than those used in~\cite{dehghani2017neural,Li2018JointLF}.}
We use the model architectures and hyper-parameters (e.g., kernel sizes) from the best-performing configurations presented in the original papers for all models.
All models are trained using pairwise loss for 200 iterations with 512 training samples each iteration.
We use Web Track 2011 (WT11) manual relevance judgments as validation data to select the best iteration via nDCG@20. This acts as a way of fine-tuning the model to the particular domain, and is the only place that manual relevance judgments are used during the weak supervision training process. At test time, we re-rank the top 100 Indri QL results for each query.

\textbf{Interaction filters.}
We use the 2-maximum and discriminator filters for each ranking architecture to evaluate the effectiveness of the interaction filters.
We use queries from the target domain (\textsc{Trec} Web Track 2009--14) to generate the template pair set for the target domain $T_D$.
To generate pairs for $T_D$, the top 20 results from query likelihood (QL) for individual queries on ClueWeb09 and ClueWeb12\footnote{\url{https://lemurproject.org/clueweb09.php}, \url{https://lemurproject.org/clueweb12.php}} are used to construct query-document pairs.
Note that this approach makes no use of manual relevance judgments because only query-document pairs from the QL search results are used (without regard for relevance).
We do not use query-document pairs from the target year to avoid any latent query signals from the test set. The supervised discriminator filter is validated using a held-out set of 1000 pairs. To prevent overfitting the training data, we reduce the convolutional filter sizes of PACRR and ConvKNRM to 4 and 32, respectively. We tune $c_{max}$ with the validation dataset (WT11) for each model ($100k$ to $900k$, $100k$ intervals).

\textbf{Baselines and benchmarks.}
As baselines, we use the AOL ranking-based source as a weakly supervised baseline~\cite{dehghani2017neural}, WT10 as a manual relevance judgment baseline, and BM25 as an unsupervised baseline. The two supervised baselines are trained using the same conditions as our approach, and the BM25 baselines is tuned on each testing set with Anserini~\cite{Yang2017AnseriniET}, representing the best-case performance of BM25.\footnote{Grid search: $b\in[0.05,1]$ (0.05 interval), and $k_1\in[0.2,4]$ (0.2 interval)}
We measure the performance of the models using
the \textsc{Trec} Web Track 2012--2014 (WT12--14) queries (topics + descriptions) and manual relevance judgments. These cover two target collections: ClueWeb09 and ClueWeb12.
Akin to~\cite{dehghani2017neural}, the trained models are used to
re-rank the top 100 results from a query-likelihood model (QL, Indri~\cite{strohman2005indri} version).
Following the \textsc{Trec} Web Track, we use
nDCG@20 and ERR@20 for evaluation.

\begin{table}
\scriptsize
\caption{Ranking performance when trained using content-based sources (NYT and Wiki). Significant differences compared to the baselines ([B]M25, [W]T10, [A]OL) are indicated with $\uparrow$ and $\downarrow$ (paired t-test, $p<0.05$).}\label{tab.results}
\vspace{-1em}
\begin{tabular}{llrrrrrr}
\toprule
&&\multicolumn{3}{c}{nDCG@20} \\\cmidrule(lr){3-5}
        Model &   Training & WT12 & WT13 & WT14 \\
\midrule


\multicolumn{2}{l}{BM25 (tuned w/~\cite{Yang2017AnseriniET})} & 0.1087  & 0.2176  & 0.2646 \\
\midrule
PACRR & WT10 & B$\uparrow$ 0.1628  & 0.2513  & 0.2676 \\
 & AOL & 0.1910  & 0.2608  & 0.2802 \\
\greyrule
 & NYT & \bf W$\uparrow$ B$\uparrow$ 0.2135  & \bf A$\uparrow$ W$\uparrow$ B$\uparrow$ 0.2919  & \bf W$\uparrow$ 0.3016 \\
 & Wiki & W$\uparrow$ B$\uparrow$ 0.1955  & A$\uparrow$ B$\uparrow$ 0.2881  & W$\uparrow$ 0.3002 \\
\midrule
Conv-KNRM & WT10 & B$\uparrow$ 0.1580  & 0.2398  & B$\uparrow$ 0.3197 \\
 & AOL & 0.1498  & 0.2155  & 0.2889 \\
\greyrule
 & NYT & \bf A$\uparrow$ B$\uparrow$ 0.1792  & \bf A$\uparrow$ W$\uparrow$ B$\uparrow$ 0.2904  & \bf B$\uparrow$ 0.3215 \\
 & Wiki & 0.1536  & A$\uparrow$ 0.2680  & B$\uparrow$ 0.3206 \\
\midrule
KNRM & WT10 & B$\uparrow$ 0.1764  & \bf 0.2671  & 0.2961 \\
 & AOL & \bf B$\uparrow$ 0.1782  & 0.2648  & \bf 0.2998 \\
\greyrule
 & NYT & W$\downarrow$ 0.1455  & A$\downarrow$ 0.2340  & 0.2865 \\
 & Wiki & A$\downarrow$ W$\downarrow$ 0.1417  & 0.2409  & 0.2959 \\


\bottomrule
\end{tabular}
\vspace{-2em}
\end{table}










\renewcommand{\arraystretch}{1}








\subsection{Results}\label{sec.results}
In Table~\ref{tab.results}, we present the performance of the rankers when trained using content-based sources without filtering.
In terms of absolute score, we observe that the two n-gram models (PACRR and ConvKNRM) always perform better when trained on content-based sources than when trained on the limited sample of in-domain data. When trained on NYT, PACRR performs significantly better. KNRM performs worse when trained using the content-based sources, sometimes significantly. These results suggest that these content-based training sources contain relevance signals where n-grams are useful, and it is valuable for these models to see a wide variety of n-gram relevance signals when training. The n-gram models also often perform significantly better than the ranking-based AOL query log baseline. This makes sense because BM25's rankings do not consider term position, and thus cannot capture this important indicator of relevance. This provides further evidence that content-based sources do a better job providing samples that include various notions of relevance than ranking-based sources.

When comparing the performance of the content-based training sources, we observe that the NYT source usually performs better than Wiki. We suspect that this is due to the web domain being more similar to the newswire domain than the complex answer retrieval domain. For instance, the document lengths of news articles are more similar to web documents, and precise term matches are less common in the complex answer retrieval domain~\cite{macavaney2018overcoming}.

We present filtering performance on NYT and Wiki for each ranking architecture in Table~\ref{tab:filter_results}. In terms of absolute score, the filters almost always improve the content-based data sources, and in many cases this difference is statistically significant. The one exception is for Conv-KNRM on NYT. One possible explanation is that the filters caused the training data to become too homogeneous, reducing the ranker's ability to generalize. We suspect that Conv-KNRM is particularly susceptible to this problem because of language-dependent convolutional filters; the other two models rely only on term similarity scores. We note that Wiki tends to do better with the 2max filter, with significant improvements seen for Conv-KNRM and KNRM. In thse models, the discriminator filter may be learning surface characteristics of the dataset,
rather than more valuable notions of relevance. We also note that $c_{max}$ is an important (yet easy) hyper-parameter to tune, as the optimal value varies considerably between systems and datasets.


%Related Work
%\input{Relatedwork}

% %  
%Conclusion
\section{Conclusion}
We have presented a neural performance rendering system to generate high-quality geometry and photo-realistic textures of human-object interaction activities in novel views using sparse RGB cameras only. 
%
Our layer-wise scene decoupling strategy enables explicit disentanglement of human and object for robust reconstruction and photo-realistic rendering under challenging occlusion caused by interactions. 
%
Specifically, the proposed implicit human-object capture scheme with occlusion-aware human implicit regression and human-aware object tracking enables consistent 4D human-object dynamic geometry reconstruction.
%
Additionally, our layer-wise human-object rendering scheme encodes the occlusion information and human motion priors to provide high-resolution and photo-realistic texture results of interaction activities in the novel views.
%
Extensive experimental results demonstrate the effectiveness of our approach for compelling performance capture and rendering in various challenging scenarios with human-object interactions under the sparse setting.
%
We believe that it is a critical step for dynamic reconstruction under human-object interactions and neural human performance analysis, with many potential applications in VR/AR, entertainment,  human behavior analysis and immersive telepresence.





\bibliographystyle{ACM-Reference-Format}
\bibliography{main}

\end{document}
