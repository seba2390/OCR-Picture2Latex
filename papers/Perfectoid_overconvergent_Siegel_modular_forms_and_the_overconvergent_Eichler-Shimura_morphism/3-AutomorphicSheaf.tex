\section{Constructions of the overconvergent automorphic sheaf}\label{section:constructionsheaf}
In this section, we construct the overconvergent automorphic sheaves using the geometric objects introduced in the previous section. In particular, we generalise the ``perfectoid method'' which was originally adopted by Chojecki--Hansen--Johansson in \cite{CHJ-2017} to handle the compact Shimura curves over $\Q$. Notice that overconvergent automorphic sheaves are first introduced by Andreatta--Iovita--Pilloni in \cite{AIP-2015} using a different approach. At the end of the section we shall compare the two constructions (when $p>2g$).

\subsection{The perfectoid construction}\label{subsection: the perfectoid construction}
Let $\Alg_{(\Z_p, \Z_p)}$ be the category of complete sheafy $(\Z_p, \Z_p)$-algebras. We consider the functor $$\Alg_{(\Z_p, \Z_p)}\rightarrow \Sets, \quad (R, R^+)\mapsto \Hom_{\Groups}^{\cts}(T_{\GL_g, 0}, R^{\times}),$$ which is represented by the $(\Z_p, \Z_p)$-algebra $(\Z_p\llbrack T_{\GL_g,0}\rrbrack, \Z_p\llbrack T_{\GL_g, 0}\rrbrack)$. The \textit{\textbf{weight space}} is $$\calW:=\Spa(\Z_p\llbrack T_{\GL_g,0}\rrbrack,\Z_p\llbrack T_{\GL_g,0}\rrbrack)^{\rig},$$ where the superscript ``rig'' stands for taking the generic fibre. Every continuous group homomorphism $\kappa: T_{\GL_g, 0}\rightarrow R^{\times}$ can be expressed as $\kappa=(\kappa_{1}, ..., \kappa_{g})$ where each $\kappa_{i}:\Z_p^\times\rightarrow R^\times$ is a continuous group homomorphism. We write $\kappa^{\vee}:=(-\kappa_{g}, ..., -\kappa_{1})$ where $-\kappa_i$ is the inverse of $\kappa_i$. 

We adapt the terminologies of \say{small weights} and \say{affinoid weights} introduced in \cite{CHJ-2017} to our setting:

\begin{Definition}\label{Definition: weights}
\begin{enumerate}
    \item[(i)] A \textbf{small $\Z_p$-algebra} is a $p$-torsion free reduced ring which is also a finite $\Z_p\llbrack T_1, ..., T_d\rrbrack$-algebra for some $d\in \Z_{\geq 0}$. In particular, a small $\Z_p$-algebra is equipped with a canonical adic profinite topology and is complete with respect to the $p$-adic topology.
    \item[(ii)] A \textbf{small weight} is a pair $(R_{\calU}, \kappa_{\calU})$ where $R_{\calU}$ is a small $\Z_p$-algebra and $\kappa_{\calU}:T_{\GL_g,0}\rightarrow R^{\times}_{\calU}$ is a continuous group homomorphism such that $\kappa_{\calU}((1+p)\one_g)-1$ is topologically nilpotent in $R_{\calU}$ with respect to the $p$-adic topology. By the universal property of the weight space, we obtain a natural morphism
    \[\Spa(R_{\calU}, R_{\calU})^{\rig}\rightarrow \calW.\]
Occasionally, we abuse the terminology and call $\calU:=\Spa(R_{\calU}, R_{\calU})$ a small weight. For later use, we write $R_{\calU}^+:=R_{\calU}$ in this situation.
    \item[(iii)] An \textbf{affinoid weight} is a pair $(R_{\calU}, \kappa_{\calU})$ where $R_{\calU}$ is a reduced Tate algebra topologically of finite type over $\Q_p$ and $\kappa_{\calU}:T_{\GL_g,0}\rightarrow R^{\times}_{\calU}$ is a continuous group homomorphism. By the universal property of weight space, we obtain a natural morphism
     \[\Spa(R_{\calU}, R^{\circ}_{\calU})\rightarrow \calW.\]
Ocassionally, we abuse the terminology and call $\calU:=\Spa(R_{\calU}, R^{\circ}_{\calU})$ an affinoid weight. For later use, we write $R_{\calU}^+=R_{\calU}^{\circ}$ in this situation.
    \item[(iv)] By a \textbf{weight}, we shall mean either a small weight or an affinoid weight.
\end{enumerate}
\end{Definition}

\begin{Remark}\label{Remark: convention on weights}
\normalfont For any $n\in \Z_{\geq 0}$, we view $n$ as a weight by identifying it with the character $$T_{\GL_g, 0}\rightarrow \Z_p^\times, \quad \diag(\bftau_1, ..., \bftau_g)\mapsto \prod_{i=1}^g\bftau_i^n.$$ Moreover, for any weight $\kappa=(\kappa_1, ..., \kappa_g)$, we write $\kappa+n $ for the weight $ (\kappa_1+n, ..., \kappa_g+n)$ defined by $$\diag(\bftau_1, ..., \bftau_g)\mapsto \prod_{i=1}^g\kappa_i(\bftau_i)\bftau_i^n.$$
\end{Remark}

We adopt the notation of \say{mixed completed tensor} used in \cite{CHJ-2017}:

\begin{Definition}\label{Definition: unadorned completed tensor}
Let $R$ be a small $\Z_p$-algebra.
\begin{enumerate}
\item[(i)] For any $\Z_p$-module $M$, we define \footnote{Our notation $\widehat{\otimes}'$ corresponds to the notation $\widehat{\otimes}$ in \cite[Definition 6.3]{CHJ-2017}. We make this change to distinguish from the one in Definition \ref{Definition: unadorned completed tensor} (ii).}
$$M\widehat{\otimes}' R:=\varprojlim _{j\in J} (M\otimes_{\Z_p}R/I_j)$$
where $\{I_j: j\in J\}$ runs through a cofinal system of neighborhood of $0$ consisting of $\Z_p$-submodules of $R$. If, in addition, $M$ is a $\Z_p$-algebra, then $M\widehat{\otimes}' R$ is also a $\Z_p$-algebra.
\item[(ii)] Let $B$ be a $\Q_p$-Banach space and let $B_0$ be an open and bounded $\Z_p$-submodule. We define the \textbf{mixed completed tensor}
$$B\widehat{\otimes}R:=(B_0\widehat{\otimes}'R)[\frac{1}{p}].$$
which is in fact independent of the choice of $B_0$. 
\end{enumerate}
\end{Definition}

\begin{Definition}\label{Definition: mixed completed tensor}
Let $(R_{\calU}, \kappa_{\calU})$ be a weight.
\begin{enumerate}
\item[(i)] For any $\Z_p$-module $M$, the term $M\widehat{\otimes}R^+_{\calU}$ will either stand for $M\widehat{\otimes}' R_{\calU}$ in the case of small weights (notice that $R_{\calU}=R^+_{\calU}$ in this case), or stand for the $p$-adically completed tensor over $\Z_p$ in the case of affinoid weights.
\item[(ii)]
For any $\Q_p$-Banach space $B$, the term $B\widehat{\otimes} R_{\calU}$ will either stand for the mixed completed tensor in the case of small weights, or stand for the usual $p$-adically completed tensor over $\Q_p$ in the case of affinoid weights. 
\end{enumerate}
\end{Definition}

\begin{Remark}\label{Remark: uniform Banach algebra structure 1}\normalfont 
For a uniform Banach $\Q_p$-algebra $B$ and any weight $(R_{\calU}, \kappa_{\calU})$, the tensor product $B\widehat{\otimes}R_{\calU}$ also admits a structure of uniform $\Q_p$-Banach algebra. In particular,
if $B^{\circ}$ is the unit ball of $B$ (with respect to the unique power-multiplicative Banach algebra norm), then the unit ball in $B\widehat{\otimes}R_{\calU} = B\widehat{\otimes}_{\Q_p}R_{\calU}^+[1/p]$ is given by $B^{\circ}\widehat{\otimes} R^+_{\calU}$. Here, note that $R_{\calU}^+[1/p]$ has a structure of a uniform Banach $\Q_p$-algebra given by the corresponding spectral norm (see \cite[pp. 202]{CHJ-2017}).
\end{Remark}

Next, we introduce the notion of ``$r$-analytic functions''.

\begin{Definition}\label{Definition: w-analytic functions} Let $r\in \Q_{>0}$ and $n\in \Z_{\geq 1}$. Let $B$ be a uniform $\C_p$-Banach algebra and let $B^{\circ}$ be the corresponding unit ball.
\begin{enumerate}
\item[(i)] A function $f: \Z_p^n\rightarrow B$ (resp., a function $f:(\Z_p^{\times})^n\rightarrow B$) is called \textbf{$r$-analytic} if for every $\underline{a}=(a_1, \ldots, a_n)\in \Z_p^n$ (resp., every $\underline{a}=(a_1, \ldots, a_n)\in (\Z_p^{\times})^n$), there exists a power series $f_{\underline{a}}\in B\llbrack T_1, \ldots, T_n\rrbrack$ which converges on the $n$-dimensional closed unit ball $\mathbf{B}^n(0, p^{-r})\subset \C_p^n$ of radius $p^{-r}$ such that 
$$f(x_1+a_1, \ldots, x_n+a_n)=f_{\underline{a}}(x_1, \ldots, x_n)$$
for all $x_i\in p^{\lceil r\rceil}\Z_p$, $i=1,\ldots, n$. Here $\lceil r\rceil$ stands for the smallest integer that is greater or equal to $r$.
\item[(ii)] Let $C^{r-\an}(\Z_p^n, B)$ (resp., $C^{r-\an}((\Z_p^{\times})^n, B)$) denote the set of $r$-analytic functions from $\Z_p^n$ (resp., $(\Z_p^{\times})^n$) to $B$.
\item[(iii)] Let $C^{r-\an}(\Z_p^n, B^{\circ})$ (resp., $C^{r-\an}((\Z_p^{\times})^n, B^{\circ})$) denote the subset of $C^{r-\an}(\Z_p^n, B)$ (resp., $C^{r-\an}((\Z_p^{\times})^n, B)$) consisting of those functions with value in $B^{\circ}$.
\end{enumerate}
\end{Definition}

\begin{Remark}\label{Remark: uniform Banach algebra structure 2}
\normalfont We claim that $C^{r-\an}(\Z_p^n, B)$ (resp., $C^{r-\an}((\Z_p^{\times})^n, B)$) admits a natural structure of uniform $\C_p$-Banach algebra. Indeed, express $\Z_p^n$ as the disjoint union of $p^{n\lceil r\rceil}$ closed balls of radius $p^{\lceil r\rceil}$, labelled by an index set $A$ of size $p^{n\lceil r\rceil}$. Then, for every $f\in \calC^{r-\an}(\Z_p^n, B)$, the restriction of $f$ on each closed ball (with label $a\in A$) is given by a power series $$f_a\in B\langle \frac{T_1}{p^r}, \ldots, \frac{T_n}{p^r}\rangle$$
where $B\langle \frac{T_1}{p^r}, \ldots, \frac{T_n}{p^r}\rangle$ stands for the subset of $B\llbrack T_1, \ldots, T_n\rrbrack$ which converges on the $n$-dimensional closed unit ball $\mathbf{B}^n(0, p^{-r})\subset \C_p^n$. Let $|\bullet|_B$ be the unique power-multipicative norm on $B$. We can equip $B\langle \frac{T_1}{p^r}, \ldots, \frac{T_n}{p^r}\rangle$ with the following norm: for every $g=\sum_{\nu\in \Z_{\geq 0}^n} b_{\nu}T^{\nu}$, we put 
$$|g|:=\sup_{\nu\in \Z_{\geq 0}^n}|b_{\nu}|_B\cdot p^{-r |\nu|}.$$
Finally, if $f\in C^{r-\an}(\Z_p^n, B)$ is represented by $\{f_a\}_{a\in A}$, we put $|f|:=\sup_{a\in A} |f_a|$. This is indeed a uniform Banach norm with unit ball $\calC^{r-\an}(\Z_p^n, B^{\circ})$.
\end{Remark}

\begin{Definition}\label{Definition: w-analytic weight}
\begin{enumerate}
\item[(i)] A weight $(R_{\calU}, \kappa_{\calU})$ is called \textbf{$r$-analytic} if it is $r$-analytic when viewed as a function
$$\kappa_{\calU}:(\Z_p^{\times})^g\rightarrow R_{\calU}^{\times}\subset \C_p\widehat{\otimes}R_{\calU}$$
via the identification $T_{\GL_g,0}\cong (\Z_p^{\times})^g$.
\item[(ii)] For a weight $(R_{\calU}, \kappa_{\calU})$, we write $r_{\calU}$ for the smallest positive integer $r$ such that the weight is $r$-analytic.
\end{enumerate}
\end{Definition}

\begin{Remark}
\normalfont \begin{enumerate}
\item[(i)] It is well-known that every continuous character $\Z_p^{\times}\rightarrow R_{\calU}^{\times}$ is $r$-analytic for sufficiently large $r$. Moreover, if such a character is $r$-analytic, it necessarily extends to a character
$$\Z_p^{\times}(1+p^{r+1}\calO_{\C_p})\rightarrow (\calO_{\C_p}\widehat{\otimes}R_{\calU}^+)^{\times}\subset \C_p\widehat{\otimes}R_{\calU}.$$ See, for example, \cite[Proposition 2.6]{CHJ-2017}. 
\item[(ii)] If we write $\kappa_{\calU}=(\kappa_{\calU, 1}\ldots, \kappa_{\calU, g})$ with components $\kappa_{\calU,i}:\Z_p^{\times}\rightarrow R_{\calU}^{\times}$, then $\kappa_{\calU}$ is $r$-analytic if and only if all $\kappa_{\calU,i}$'s are $r$-analytic.
In this case, for any $w\in \Q_{>0}$ with $w>1+r_{\calU}$, $\kappa_{\calU}$ extends to a character
$$\kappa_{\calU}: T^{(w)}_{\GL_g, 0}\rightarrow (\calO_{\C_p}\widehat{\otimes}R_{\calU}^+)^{\times}\subset \C_p\widehat{\otimes}R_{\calU}.$$
\end{enumerate}
\end{Remark}

\begin{Definition}
Let $B$ be a uniform $\C_p$-Banach algebra.
\begin{enumerate}
\item[(i)] A function $\psi: U_{\GL_g, 1}^{\opp}\rightarrow B$ is called \textbf{$r$-analytic} if, under the (topological) identification
$$U^{\opp}_{\GL_g, 1}=\begin{pmatrix}1&&&\\p\Z_p&1&&\\\vdots&&\ddots&\\p\Z_p&\ldots&p\Z_p&1\end{pmatrix}\simeq \Z_p^{\frac{(g-1)g}{2}},$$
the function $$\psi:\Z_p^{\frac{(g-1)g}{2}}\rightarrow B$$
is $r$-analytic. Let $C^{r-\an}(U^{\opp}_{\GL_g,1}, B)$ denote the space of such functions. 
\item[(ii)] Let $(R_{\calU}, \kappa_{\calU})$ be an $r$-analytic weight. Using the decomposition $B_{\GL_g,0}=T_{\GL_g,0}U_{\GL_g,0}$, we extend $\kappa_{\calU}$ to a group homomorphism $\kappa_{\calU}: B_{\GL_g,0}\rightarrow R_{\calU}^{\times}$ by setting $\kappa_{\calU}|_{U_{\GL_g,0}}=1$. Define 
$$C^{r-\an}_{\kappa_{\calU}}(\Iw_{\GL_g}, B):=\left\{f: \Iw_{\GL_g}\rightarrow B: \begin{array}{l}
    f(\bfgamma\bfbeta)=\kappa_{\calU}(\bfbeta)f(\bfgamma),\,\,\forall \bfbeta\in B_{\GL_g, 0},\,\,\bfgamma \in \Iw_{\GL_g}\\
    f|_{U_{\GL_g, 1}^{\opp}}\text{ is $r$-analytic} 
\end{array}\right\}.$$ 
\item[(iii)] Let $C^{r-\an}_{\kappa_{\calU}}(\Iw_{\GL_g}, B^{\circ})$ denote the subset of $C^{r-\an}_{\kappa_{\calU}}(\Iw_{\GL_g}, B)$ consisting of those functions with value in $B^{\circ}$.
\end{enumerate}
\end{Definition}

\begin{Remark}\label{Remark: uniform Banach algebra structure 3}
\normalfont According to Remark \ref{Remark: uniform Banach algebra structure 2}, $C^{r-\an}(U^{\opp}_{\GL_g,1}, B)$ admits a structure of uniform $\C_p$-Banach algebra. Notice that an element in $C^{r-\an}_{\kappa_{\calU}}(\Iw_{\GL_g}, B)$ is determined by its restriction on $U^{\opp}_{\GL_g, 1}$. Consequently, $C^{r-\an}_{\kappa_{\calU}}(\Iw_{\GL_g}, B)$ admits a structure of uniform $\C_p$-Banach algebra via the identification $$C^{r-\an}_{\kappa_{\calU}}(\Iw_{\GL_g}, B)\cong C^{r-\an}(U^{\opp}_{\GL_g,1}, B).$$ In particular, $C^{r-\an}_{\kappa_{\calU}}(\Iw_{\GL_g}, B^{\circ})$ is the corresponding unit ball in $C^{r-\an}_{\kappa_{\calU}}(\Iw_{\GL_g}, B)$.
\end{Remark}

\begin{Remark}\label{Remark: extend to w-analytic nbhd}
\normalfont Let $\kappa_{\calU}$ be a weight and let $w\in \Q_{>0}$ with $w>r_{\calU}+1$. Recall that we have a decomposition $B_{\GL_g, 0}^{(w)}=T_{\GL_g,0}^{(w)}U_{\GL_g,0}^{(w)}$. Since $w>1+r_{\calU}$, $\kappa_{\calU}$ extends to a character on $T_{\GL_g,0}^{(w)}$, and hence to a character on $B_{\GL_g,0}^{(w)}$ by setting $\kappa_{\calU}|_{U_{\GL_g,0}^{(w)}}=0$.

We claim that every element $f$ in $C^{w-\an}_{\kappa_{\calU}}(\Iw_{\GL_g}, B)$ (resp., $C^{w-\an}_{\kappa_{\calU}}(\Iw_{\GL_g}, B^{\circ})$) naturally extends to a function $$f:\Iw_{\GL_g}^{(w)}\rightarrow B\quad(\text{resp., }\,\,f:\Iw_{\GL_g}^{(w)}\rightarrow B^{\circ})$$ such that $f(\bfgamma\bfbeta)=\kappa_{\calU}(\bfbeta)f(\bfgamma)$ for all $\bfbeta\in B_{\GL_g, 0}^{(w)}$ and $\bfgamma\in \Iw_{\GL_g}^{(w)}$. Indeed, we have decomposition $$\Iw_{\GL_g}^{(w)}=U_{\GL_g, 1}^{\opp,(w)}T_{\GL_g,0}^{(w)}U_{\GL_g,0}^{(w)}.$$ Then for every $\bfnu\in U^{\opp,(w)}_{\GL_g,1}$, $\bftau\in T_{\GL_g,0}^{(w)}$, and $\bfnu'\in U_{\GL_g,0}^{(w)}$, we put $$f(\bfnu\bftau\bfnu')=\kappa_{\calU}(\bftau)f(\bfnu).$$ It is straightforward to check that $f$ is well-defined and satisfies the required condition. 
\end{Remark}

\begin{Definition}\label{Definition: strict Iwahori action on w-analytic representation for IwGL_g}
As a consequence of Remark \ref{Remark: extend to w-analytic nbhd}, given $w\in \Q_{>0}$ with $w>1+r_{\calU}$, there is a natural left action of $\Iw_{\GL_g}^{+, (w)}$ on $C^{w-\an}_{\kappa_{\calU}}(\Iw_{\GL_g}, B)$ and $C^{w-\an}_{\kappa_{\calU}}(\Iw_{\GL_g}, B^{\circ})$ (hence also a left action of $\Iw_{\GL_g}^+$) given by $$
    (\bfgamma\cdot f)(\bfgamma')=f(\trans\bfgamma\bfgamma')
$$
for all $\bfgamma\in \Iw^{+, (w)}_{\GL_g}$, $\bfgamma'\in \Iw_{\GL_g}$, and $f\in C^{w-\an}_{\kappa_{\calU}}(\Iw_{\GL_g}, B)$ (resp., $C^{w-\an}_{\kappa_{\calU}}(\Iw_{\GL_g}, B^{\circ})$). This left action is denoted by $\rho_{\kappa_{\calU}}: \Iw^{+,(w)}_{\GL_g}\rightarrow \Aut(C^{w-\an}_{\kappa_{\calU}}(\Iw_{\GL_g}, B))$ (resp., $\rho_{\kappa_{\calU}}: \Iw^{+,(w)}_{\GL_g}\rightarrow \Aut(C^{w-\an}_{\kappa_{\calU}}(\Iw_{\GL_g}, B^{\circ}))$).
\end{Definition}

We are ready to define the sheaf of overconvergent Siegel modular forms. 

\begin{Definition}\label{Definition: the sheaf of overconvergent Siegel forms}Let $(R_{\calU}, \kappa_{\calU})$ be a weight and let $w\in \Q_{>0}$ with $w> 1+r_{\calU}$.
\begin{enumerate}
\item[(i)] Let $\scrO_{\overline{\calX}_{\Gamma(p^{\infty}), w}}\widehat{\otimes}R_{\calU}$ be the sheaf on $\overline{\calX}_{\Gamma(p^{\infty}), w}$ given by $$\calY\mapsto \scrO_{\overline{\calX}_{\Gamma(p^{\infty}), w}}(\calY)\widehat{\otimes}R_{\calU}$$ for every affinoid open subset $\calY\subset \overline{\calX}_{\Gamma(p^{\infty}), w}$. This is in fact a sheaf of uniform $\C_p$-Banach algebra; i.e., $(\scrO_{\overline{\calX}_{\Gamma(p^{\infty}), w}}\widehat{\otimes}R_{\calU})(\calY)$ is a uniform $\C_p$-Banach algebra for every affinoid open $\calY$.

Similarly, let $\scrO^+_{\overline{\calX}_{\Gamma(p^{\infty}), w}}\widehat{\otimes}R^+_{\calU}$ be the sheaf on $\overline{\calX}_{\Gamma(p^{\infty}), w}$ given by $$\calY\mapsto \scrO^+_{\overline{\calX}_{\Gamma(p^{\infty}), w}}(\calY)\widehat{\otimes}R^+_{\calU}$$ for every affinoid open subset $\calY\subset \overline{\calX}_{\Gamma(p^{\infty}), w}$.
\item[(ii)] For any $r\in \Q_{>0}$ with $r>1+ r_{\calU}$, let $\scrC^{r-\an}_{\kappa_{\calU}}(\Iw_{\GL_g}, \scrO_{\overline{\calX}_{\Gamma(p^{\infty}), w}}\widehat{\otimes}R_{\calU})$ denote the sheaf on $\overline{\calX}_{\Gamma(p^{\infty}), w}$ given by 
$$\calY\mapsto C^{r-\an}_{\kappa_{\calU}}(\Iw_{\GL_g}, \scrO_{\overline{\calX}_{\Gamma(p^{\infty}), w}}(\calY)\widehat{\otimes}R_{\calU})$$ for every affinoid open subset $\calY\subset \overline{\calX}_{\Gamma(p^{\infty}), w}$. This is also a sheaf of uniform $\C_p$-Banach algebra.

The sheaf $\scrC^{r-\an}_{\kappa_{\calU}}(\Iw_{\GL_g}, \scrO^+_{\overline{\calX}_{\Gamma(p^{\infty}), w}}\widehat{\otimes}R^+_{\calU})$ is defined in the same way.

\item[(iii)] The \textbf{sheaf of $w$-overconvergent Siegel modular forms of strict Iwahori level and weight $\kappa_{\calU}$} is a subsheaf $\underline{\omega}_w^{\kappa_{\calU}}$ of $h_{\Iw^+, *}\scrC^{w-\an}_{\kappa_{\calU}}(\Iw_{\GL_g}, \scrO_{\overline{\calX}_{\Gamma(p^{\infty}), w}}\widehat{\otimes}R_{\calU})$ defined as follows. For every affinoid open subset $\calV\subset \overline{\calX}_{\Iw^+, w}$ with $\calV_{\infty} = h_{\Iw^+}^{-1}(\calV)$, we put $$\underline{\omega}_w^{\kappa_{\calU}}(\calV):=\left\{f\in C^{w-\an}_{\kappa_{\calU}}(\Iw_{\GL_g}, \scrO_{\overline{\calX}_{\Gamma(p^{\infty}), w}}(\calV_{\infty})\widehat{\otimes}R_{\calU}): \bfgamma^*f=\rho_{\kappa_{\calU}}(\bfgamma_a+\frakz\bfgamma_c)^{-1} f,\,\,\forall \bfgamma=\begin{pmatrix}\bfgamma_a & \bfgamma_b\\ \bfgamma_c & \bfgamma_d\end{pmatrix}\in \Iw_{\GSp_{2g}}^+\right\}.$$ 
Here, $\bfgamma^*f$ stands for the left action of $\bfgamma$ on $\scrO_{\overline{\calX}_{\Gamma(p^{\infty}), w}}$ induced by the natural right $\Iw_{\GSp_{2g}}^+$-action on $\overline{\calX}_{\Gamma(p^{\infty}), w}$ defined in \S \ref{subsection: Hodge--Tate period map and the w-ordinary locus}. 
    
Similarly, the \textbf{sheaf of integral $w$-overconvergent Siegel modular forms of strict Iwahori level and weight $\kappa_{\calU}$} is a subsheaf $\underline{\omega}_w^{\kappa_{\calU},+}$ of $h_{\Iw^+, *}\scrC^{w-\an}_{\kappa_{\calU}}(\Iw_{\GL_g}, \scrO^+_{\overline{\calX}_{\Gamma(p^{\infty}), w}}\widehat{\otimes}R^+_{\calU})$ defined as follows. For every affinoid open subset $\calV\subset \overline{\calX}_{\Iw^+, w}$ with $\calV_{\infty} = h_{\Iw^+}^{-1}(\calV)$, we put $$\underline{\omega}_w^{\kappa_{\calU},+}(\calV):=\left\{f\in C^{w-\an}_{\kappa_{\calU}}(\Iw_{\GL_g}, \scrO^+_{\overline{\calX}_{\Gamma(p^{\infty}), w}}(\calV_{\infty})\widehat{\otimes}R^+_{\calU}): \bfgamma^*f=\rho_{\kappa_{\calU}}(\bfgamma_a+\frakz\bfgamma_c)^{-1} f,\,\,\forall \bfgamma=\begin{pmatrix}\bfgamma_a & \bfgamma_b\\ \bfgamma_c & \bfgamma_d\end{pmatrix}\in \Iw_{\GSp_{2g}}^+\right\}.$$

\item[(iv)] The \textbf{space of $w$-overconvergent Siegel modular forms of strict Iwahori level and weight $\kappa_{\calU}$} is defined to be $$M_{\Iw^+, w}^{\kappa_{\calU}}:=H^0(\overline{\calX}_{\Iw^+, w},\, \underline{\omega}_w^{\kappa_{\calU}}).$$ We similarly define the \textbf{space of integral $w$-overconvergent Siegel modular forms of strict Iwahori level and weight $\kappa_{\calU}$} to be $$M_{\Iw^+, w}^{\kappa_{\calU},+}:=H^0(\overline{\calX}_{\Iw^+, w}, \,\underline{\omega}_w^{\kappa_{\calU}, +}).$$
    
\item[(v)] Taking limit with respect to $w$, the \textbf{space of overconvergent Siegel modular forms of strict Iwahori level and weight $\kappa_{\calU}$} is 
$$M_{\Iw^+}^{\kappa_{\calU}}:=\lim_{w\rightarrow\infty}M^{\kappa_{\calU}}_{\Iw, w}.$$
Similarly, the \textbf{space of integral overconvergent Siegel modular forms of strict Iwahori level and weight $\kappa_{\calU}$} is 
$$M^{\kappa_{\calU}, +}_{\Iw^+}:=\lim_{w\rightarrow\infty}M^{\kappa_{\calU},+}_{\Iw^+, w}.$$

\item[(vi)] Recall that $\calZ_{\Iw^+} = \overline{\calX}_{\Iw^+}\smallsetminus \calX_{\Iw^+}$ is the boundary divisor. The \textbf{sheaf of $w$-overconvergent Siegel cuspforms of strict Iwahori level and weight $\kappa_{\calU}$} is defined to be the subsheaf $\underline{\omega}_{w, \cusp}^{\kappa_{\calU}} = \underline{\omega}_{w}^{\kappa_{\calU}}(-\calZ_{\Iw^+})$ of $\underline{\omega}_{w}^{\kappa_{\calU}}$ consisting of sections that vanish along $\calZ_{\Iw^+}$.

A $w$-overconvergent Siegel modular form of strict Iwahori level and weight $\kappa_{\calU}$ is called \textbf{cuspidal} if it is an element of  \[
        S_{\Iw^+, w}^{\kappa_{\calU}}:= H^0(\overline{\calX}_{\Iw^+, w},\, \underline{\omega}_{w, \cusp}^{\kappa_{\calU}}).
\] 

Moreover, by taking limit with respect to $w$, the \textbf{space of overconvergent Siegel cuspforms of strict Iwahori level and weight $\kappa_{\calU}$} is defined to be $$S^{\kappa_{\calU}}_{\Iw^+}:=\lim_{w\rightarrow\infty}S^{\kappa_{\calU}}_{\Iw^+, w}.$$
\end{enumerate}
\end{Definition}

\begin{Remark}\label{Remark: well-defined}
\normalfont Notice that, in Definition \ref{Definition: the sheaf of overconvergent Siegel forms} (iii), for every $\bfitx\in \overline{\calX}_{\Gamma(p^{\infty}), w}(\C_p, \calO_{\C_p})$ and any $\begin{pmatrix}\bfgamma_a & \bfgamma_b \\ \bfgamma_c & \bfgamma_d\end{pmatrix}\in \Iw_{\GSp_{2g}}^+$, we have $\bfgamma_a+\frakz(\bfitx)\bfgamma_c\in \Iw^{+, (w)}_{\GL_g}$. Hence, $\rho_{\kappa_{\calU}}(\bfgamma_a+\frakz\bfgamma_c)$ is well-defined.
\end{Remark}

\begin{Remark}\label{Remark: analogy to the complex version}
\normalfont The definition above yields an analogue to the complex theory, which we describe in the following.

Suppose $k=(k_1, \ldots, k_g)\in \Z_{\geq 0}^g$ is a dominant weight for $\GL_g$ and let $\rho_k:\GL_g(\C)\rightarrow \GL(\V_k)$ be the corresponding irreducible representation of $\GL_g$ of highest weight $k$. Let $\bbH_g^+$ be the (complex) Siegel upper-half space. Then a classical Siegel modular form of weight $k$ and level $\Gamma$ is a holomorphic function $f:\bbH_g^+\rightarrow \V_k$ such that
$$f(\bfgamma\cdot \bfitx)=\rho_k(\bfgamma_c\bfitx+\bfgamma_d)f(\bfitx)$$
for all $\bfitx\in \bbH_g^+$ and $\bfgamma=\begin{pmatrix}\bfgamma_a & \bfgamma_b\\ \bfgamma_c & \bfgamma_d\end{pmatrix}\in \Gamma\subset \GSp_{2g}(\Z)$.

In our case, a $w$-overconvergent Siegel modular form $f\in M_{\Iw^+, w }^{\kappa_{\calU}}$ can be viewed as a function 
$$f: \overline{\calX}_{\Gamma(p^{\infty}), w}\rightarrow C^{w-\an}_{\kappa_{\calU}}(\Iw_{\GL_g}, \C_p\widehat{\otimes}R_{\calU})$$ 
satisfying $$f(\bfitx\cdot \bfgamma)=\rho_{\kappa_{\calU}}(\bfgamma_a + \frakz\bfgamma_c)^{-1}f(\bfitx)$$
for all $\bfitx\in \overline{\calX}_{\Gamma(p^{\infty}), w}$ and $\bfgamma=\begin{pmatrix}\bfgamma_a & \bfgamma_b\\ \bfgamma_c & \bfgamma_d\end{pmatrix}\in \Iw_{\GSp_{2g}}^+\subset \GSp_{2g}(\Z_p)$. Notice that $C^{w-\an}_{\kappa_{\calU}}(\Iw_{\GL_g}, \C_p\widehat{\otimes}R_{\calU})$ is an analytic analogue of the algebraic representation $\V_k$.
\end{Remark}

To simplify the notation, we defined a ``twisted'' left action of $\Iw^+_{\GSp_{2g}}$ on $\scrC^{w-\an}_{\kappa_{\calU}}(\Iw_{\GL_g}, \scrO_{\overline{\calX}_{\Gamma(p^{\infty}),w}}\widehat{\otimes}R_{\calU})$ by
$$\bfgamma. f:= \rho_{\kappa_{\calU}}(\bfgamma_a+\frakz \bfgamma_c)\bfgamma^*f.$$
Then sections of $\underline{\omega}^{\kappa_{\calU}}_w$ are precisely the $\Iw^+_{\GSp_{2g}}$-invariant sections of $h_{\Iw^+, *}\scrC^{w-\an}_{\kappa_{\calU}}(\Iw_{\GL_g}, \scrO_{\overline{\calX}_{\Gamma(p^{\infty}),w}}\widehat{\otimes}R_{\calU})$ under the twisted action.

\begin{Remark}
\normalfont The sheaf $\underline{\omega}_w^{\kappa_{\calU}}$ is functorial in the weight $(R_{\calU}, \kappa_{\calU})$. Given a map of weights $R_{\calU} \rightarrow R_{\calU'}$ and $w > \mbox{max}\left\{1+r_{\calU}, 1+r_{\calU'}\right\}$, we obtain a natural map $\underline{\omega}_w^{\kappa_{\calU}} \rightarrow \underline{\omega}_w^{\kappa_{\calU'}}$ induced from
\[C^{w-\an}_{\kappa_{\calU}}(\Iw_{\GL_g}, \scrO_{\overline{\calX}_{\Gamma(p^{\infty}), w}}\widehat{\otimes}R_{\calU}) \rightarrow C^{w-\an}_{\kappa_{\calU'}}(\Iw_{\GL_g}, \scrO_{\overline{\calX}_{\Gamma(p^{\infty}), w}}\widehat{\otimes}R_{\calU'}).\]  
\end{Remark}


\subsection{Hecke operators}\label{subsection: Hecke operators on the overconvergent automorphic forms}
In this subsection, we spell out how the Hecke operators act on the overconvergent Siegel modular forms. The Hecke operators at the primes dividing the tame level $N$ are not considered in this paper.

Through out this subsection, let $(R_{\calU}, \kappa_{\calU})$ be a weight and $w>1+r_{\calU}$.

\paragraph{Hecke operators outside $Np$.} We define the Hecke operators outside $Np$ using correspondences. Let $\ell$ be a rational prime that does not divide $Np$. For every $\bfgamma\in \GSp_{2g}(\Q_{\ell})\cap M_{2g}(\Z_{\ell})$, consider the moduli space $X_{\bfgamma, \Iw^+}$ over $X_{\Iw^+}$ parameterising \textit{isogenies of type} $\bfgamma$. More precisely, $X_{\bfgamma, \Iw^+}$ is the moduli space of sextuples $$(A, \lambda, \psi_N, \Fil_{\bullet}A[p], \{C_i: i=1, \ldots, g\}, L)$$ where $(A, \lambda, \psi_N, \Fil_{\bullet}A[p],\{C_i: i=1, \ldots, g\})\in X_{\Iw^+}$ and $L\subset A$ is a subgroup of finite order such that the isogeny $(A, \lambda)\rightarrow (A/L, \lambda')$ is of type $\bfgamma$ in the sense of \cite[Chapter VII, \S 3]{Faltings-Chai}, where $\lambda'$ stands for the induced principal polarisation. According to \emph{loc. cit.}, for every isogeny of type $\bfgamma$, its dual isogeny is also of type $\bfgamma$. In particular, the assignment $$(A, \lambda, \psi_N, \Fil_{\bullet}A[p], \{C_i: i=1, \ldots, g\}, L)\mapsto (A'=A/L, \lambda', \psi'_N, \Fil_{\bullet}A'[p], \{C'_i: i=1, \ldots, g\}, L')$$ defines an isomorphism $\Phi_{\bfgamma}: X_{\bfgamma, \Iw^+}\xrightarrow[]{\sim}X_{\bfgamma, \Iw^+}$, where 
\begin{itemize}
\item $\lambda'$ is the induced polarisation on $A'$;
\item $\psi'_N$, $\Fil_{\bullet}A'[p]$, and $C'_i$'s are induced from $\psi_N$, $\Fil_{\bullet}A[p]$, and $C_i$'s, respectively, via the isomorphisms $A[N]\simeq A'[N]$ and $A[p]\simeq A'[p]$;
\item $L'$ is defined by the dual isogeny of $(A,\lambda)\rightarrow (A',\lambda')$.
\end{itemize}

There are two finite \'etale projections 
$$
\begin{tikzcd}
& X_{\bfgamma, \Iw^+}\arrow[ld, "\pr_1"']\arrow[rd, "\pr_2"] \\
X_{\Iw^+} & & X_{\Iw^+}
\end{tikzcd}$$ 
where $\pr_1$ is the forgetful map and $\pr_2$ sends the sextuple $(A, \lambda, \psi_N, \Fil_{\bullet}A[p], \{C_i: i=1, \ldots, g\}, L)$ to the quintuple $(A'=A/L, \lambda', \psi'_N, \Fil_{\bullet}A'[p], \{C'_i: i=1, \ldots, g\})$ described as above. Clearly, we have $\pr_1=\pr_2\circ \Phi_{\bfgamma}$.

Let $\calX_{\bfgamma, \Iw^+}$ be the adic space associated with $X_{\bfgamma, \Iw^+}$ by taking analytification. We obtain finite \'etale morphisms $\pr_1, \pr_2:\calX_{\bfgamma, \Iw^+}\rightrightarrows \calX_{\Iw^+}$ as well as an isomorphism $\Phi_{\bfgamma}:\calX_{\bfgamma, \Iw^+}\rightarrow \calX_{\bfgamma, \Iw^+}$. We further pass to the $w$-ordinary locus. More precisely, let $\calX_{\bfgamma, \Iw^+, w}$ denote the preimage of $\calX_{\Iw^+, w}$ under the projection $\pr_1$. Notice that $\Phi_{\bfgamma}$ preserves $\calX_{\bfgamma, \Iw^+, w}$ as the isogeny $(A, \lambda)\rightarrow (A', \lambda')$ induces a symplectic isomorphism $T_pA\cong T_pA'$. Hence, we obtain finite \'etale morphisms 
\begin{equation}\label{eq: correspondence-strict-Iw-level}
\begin{tikzcd}
& \calX_{\bfgamma, \Iw^+,w}\arrow[ld, "\pr_1"']\arrow[rd, "\pr_2"] \\
\calX_{\Iw^+,w} & & \calX_{\Iw^+,w}
\end{tikzcd}
\end{equation}
and an isomorphism $\Phi_{\bfgamma}:\calX_{\bfgamma,\Iw^+,w}\xrightarrow[]{\sim} \calX_{\bfgamma,\Iw^+,w}$. We still have $\pr_1=\pr_2\circ \Phi_{\bfgamma}$. 

In order to define the Hecke operator, we shall first construct a natural isomorphism
$$\varphi_{\bfgamma}: \pr_2^*\underline{\omega}^{\kappa_{\calU}}_w\xrightarrow[]{\sim}\pr_1^*\underline{\omega}^{\kappa_{\calU}}_w.$$
Here we have abused the notation and still write $\underline{\omega}^{\kappa_{\calU}}_w$ for its restriction to $\calX_{\Iw^+, w}$.

Indeed, pulling back the diagram (\ref{eq: correspondence-strict-Iw-level}) along the projection $h_{\Iw^+}:\calX_{\Gamma(p^{\infty}), w}\rightarrow \calX_{\Iw^+, w}$, we obtain finite \'etale morphisms
$$
\begin{tikzcd}
& \calX_{\bfgamma, \Gamma(p^{\infty}), w}\arrow[ld, "\pr_{1, \infty}"']\arrow[rd, "\pr_{2, \infty}"] \\
\calX_{\Gamma(p^{\infty}),w} & & \calX_{\Gamma(p^{\infty}),w}
\end{tikzcd}
$$
between perfectoid spaces and an $\Iw_{\GSp_{2g}}^+$-equivariant isomorphism $\Phi_{\bfgamma, \infty}:\calX_{\bfgamma,\Gamma(p^{\infty}),w}\xrightarrow[]{\sim} \calX_{\bfgamma,\Gamma(p^{\infty}),w}$. The isomorphism $\Phi_{\bfgamma, \infty}$ induces an isomorphism
$$\Phi^*_{\bfgamma, \infty}:\pr_{2,\infty}^*\scrO_{\calX_{\Gamma(p^{\infty})}, w}\xrightarrow[]{\sim}\pr_{1,\infty}^*\scrO_{\calX_{\Gamma(p^{\infty})}, w}.$$
It then induces an isomorphism
$$\Phi^*_{\bfgamma, \infty}:\scrC^{w-\an}_{\kappa_{\calU}}(\Iw_{\GL_g}, \pr_{2,\infty}^*\scrO_{\calX_{\Gamma(p^{\infty})}, w}\widehat{\otimes}R_{\calU})\xrightarrow[]{\sim}\scrC^{w-\an}_{\kappa_{\calU}}(\Iw_{\GL_g}, \pr_{1,\infty}^*\scrO_{\calX_{\Gamma(p^{\infty})}, w}\widehat{\otimes}R_{\calU})$$
by taking the identity on $R_{\calU}$.

Recall that $\frakz$ is the pullback of the coordinate $\bfitz$ via the Hodge--Tate period map $\pi_{\HT}: \calX_{\Gamma(p^{\infty}), w}\rightarrow \adicFL^{\times}_w$. Let $\frakz':=\pr^*_{1, \infty}\frakz$ and $\frakz'':=\pr^*_{2, \infty}\frakz$. Since $\Phi_{\bfgamma, \infty}$ induces an isomorphism on the $p$-adic Tate module, we have $\frakz'=\frakz''$. Consequently, a section $f$ of $\scrC^{w-\an}_{\kappa_{\calU}}(\Iw_{\GL_g}, \pr_{2,\infty}^*\scrO_{\calX_{\Gamma(p^{\infty})}, w}\widehat{\otimes}R_{\calU})$ satisfies 
$$\bfgamma^*f=\rho_{\kappa_{\calU}}(\bfgamma_a+\frakz''\bfgamma_c)^{-1} f$$ for all $\bfgamma=\begin{pmatrix}\bfgamma_a & \bfgamma_b\\ \bfgamma_c & \bfgamma_d\end{pmatrix}\in \Iw_{\GSp_{2g}}^+$, if and only if the section $\Phi^*_{\bfgamma, \infty}(f)$ of $\scrC^{w-\an}_{\kappa_{\calU}}(\Iw_{\GL_g}, \pr_{1,\infty}^*\scrO_{\calX_{\Gamma(p^{\infty})}, w}\widehat{\otimes}R_{\calU})$ satisfies
$$\bfgamma^*(\Phi^*_{\bfgamma, \infty}(f))=\rho_{\kappa_{\calU}}(\bfgamma_a+\frakz'\bfgamma_c)^{-1} (\Phi^*_{\bfgamma, \infty}(f))$$ for all $\bfgamma\in \Iw^+_{\GSp_{2g}}$. This yields the desired isomorphism
$$\varphi_{\bfgamma}: \pr_2^*\underline{\omega}^{\kappa_{\calU}}_w\xrightarrow[]{\sim}\pr_1^*\underline{\omega}^{\kappa_{\calU}}_w.$$

Given this, we consider the composition 
$$
    \begin{tikzcd}
    T_{\bfgamma}: & H^0(\calX_{\Iw^+, w}, \underline{\omega}_w^{\kappa_{\calU}})\arrow[r, "\pr_2^*"] & H^0(\calX_{\bfgamma, \Iw^+, w}, \pr_2^*\underline{\omega}_w^{\kappa_{\calU}}) \arrow[ld, out=-10, in=170, "\varphi_{\bfgamma}"']\\
    & H^0(\calX_{\bfgamma, \Iw^+, w}, \pr_1^*\underline{\omega}_w^{\kappa_{\calU}}) \arrow[r, "\Tr\pr_1"] & H^0(\calX_{\Iw^+, w}, \underline{\omega}_w^{\kappa_{\calU}}). \end{tikzcd}
$$

Finally, we have to extend the construction to the boundary. In fact, we shall prove that the sections of $\underline{\omega}^{\kappa_{\calU}}_w$ on $\overline{\calX}_{\Iw^+,w}$ are precisely the \emph{bounded} sections of $\underline{\omega}^{\kappa_{\calU}}_w$ over the open part $\calX_{\Iw^+,w}$. By definition, a section of $\underline{\omega}^{\kappa_{\calU}}_w$ on $\overline{\calX}_{\Iw^+,w}$ can be viewed a section of $\scrC^{w-\an}_{\kappa_{\calU}}(\Iw_{\GL_g}, \scrO_{\overline{\calX}_{\Gamma(p^{\infty}),w}}\widehat{\otimes}R_{\calU})$, which is a sheaf of uniform $\C_p$-Banach algebra by Definition \ref{Definition: the sheaf of overconvergent Siegel forms} (ii). In particular, the notion of bounded section of $\underline{\omega}^{\kappa_{\calU}}_w$ on $\overline{\calX}_{\Iw^+,w}$ is well-defined.

\begin{Lemma}\label{Lemma: extend to the boundary}
\begin{enumerate}
\item[(i)] Every bounded section of $\scrO_{\calX_{\Gamma(p^{\infty}),w}}$ uniquely extends to a section of $\scrO_{\overline{\calX}_{\Gamma(p^{\infty}),w}}$.
\item[(ii)] Every bounded section of $\underline{\omega}^{\kappa_{\calU}}_w$ on $\calX_{\Iw^+,w}$ uniquely extends to a section of $\overline{\calX}_{\Iw^+,w}$. 
\end{enumerate}
\end{Lemma}

\begin{proof}
\begin{enumerate}
\item[(i)]
The assertion follows from that every bounded section of $\scrO_{\calX_{\Gamma(p^{\infty})}}$ uniquely extends to a section of $\scrO_{\overline{\calX}_{\Gamma(p^{\infty})}}$. To show this, let $\calX_{\Gamma(p^{\infty})}^*$ be the minimal compactification of $\calX_{\Gamma(p^{\infty})}$ considered in \cite{Scholze-2015} and let \[
    \psi_{\Gamma(p^{\infty})}: \overline{\calX}_{\Gamma(p^{\infty})} \rightarrow \calX_{\Gamma(p^{\infty})}^*
\] be the natural map. By the proof of \cite[Proposition 1.11]{Pilloni-Stroh-CoherentCohomologyandGaloisRepresentations}, we know that $\psi_{\Gamma(p^{\infty})}$ is a surjection and the natural map \[
    \scrO_{\calX_{\Gamma(p^{\infty})}^*} \rightarrow \psi_{\Gamma(p^{\infty}), *}\scrO_{\overline{\calX}_{\Gamma(p^{\infty})}}
\] is an injection. Hence, it is enough to show that every bounded section of $\scrO_{\calX_{\Gamma(p^{\infty})}}$ extends to a section in $\scrO_{\calX_{\Gamma(p^{\infty})}^*}$. However, this follows from the fact that $\calX_{\Gamma(p^{\infty})}^*$ is covered by translates of the anticanonical locus (\cite[Lemma 3.3.11]{Scholze-2015}) and the assertion holds for the anticanonical locus ([\emph{op. cit.}, Theorem 3.2.36]).

\item[(ii)] For simplicity, let $m=\frac{g(g-1)}{2}$. According to Remark \ref{Remark: uniform Banach algebra structure 2} and Remark \ref{Remark: uniform Banach algebra structure 3}, every section $f$ in $\scrC^{w-\an}_{\kappa_{\calU}}(\Iw_{\GL_g}, \scrO_{\overline{\calX}_{\Gamma(p^{\infty}),w}}\widehat{\otimes}R_{\calU})$ is given by an $p^{m\lceil w\rceil}$-tuple $\{f_a\}_{a\in A}$ where each $$f_a\in \scrO_{\overline{\calX}_{\Gamma(p^{\infty}),w}}\widehat{\otimes}R_{\calU}\langle \frac{T_1}{p^w}, \ldots, \frac{T_m}{p^w}\rangle.$$ By choosing a pseudo-basis $\{e_{\sigma}:\sigma\in \Sigma\}$ of $R_{\calU}$ as in \cite[Proposition 6.2]{CHJ-2017}, we can identify $R_{\calU}\simeq \prod_{\sigma\in \Sigma} \Z_p e_{\sigma}$, equipped with the product topology. Under this identification, each $f_a$ can be viewed as an element $$f_a=\{f_{a,\sigma}\}_{\sigma\in \Sigma}\in\prod_{\sigma\in \Sigma} \scrO_{\overline{\calX}_{\Gamma(p^{\infty}),w}}\langle \frac{T_1}{p^w}, \ldots, \frac{T_m}{p^w}\rangle.$$ If we express each $f_{a,\sigma}$ in terms of power series expansion $$f_{a,\sigma}=\sum_{\nu\in \Z_{\geq 0}^m} g_{a,\sigma,\nu} T^{\nu}$$
with $g_{a,\sigma,\nu}\in \scrO_{\calX_{\Gamma(p^{\infty}), w}}$, it is clear that $f$ is bounded if and only if each $g_{a,\sigma,\nu}$ is bounded. In this case, by (i), each $g_{a,\sigma,\nu}$ extends uniquely to a section of $\scrO_{\overline{\calX}_{\Gamma(p^{\infty}), w}}$. One checks that the resulting extension of each $f_a$ still satisfies the desired convergence condition. Hence, $f=\{f_a\}_{a\in A}$ extends to a section of $\scrC^{w-\an}_{\kappa_{\calU}}(\Iw_{\GL_g}, \scrO_{\overline{\calX}_{\Gamma(p^{\infty}),w}}\widehat{\otimes}R_{\calU})$. Finally, by continuity, $f$ satisfies the condition 
$$\bfgamma^*f=\rho_{\kappa_{\calU}}(\bfgamma_a+\frakz\bfgamma_c)^{-1} f$$ for all $\bfgamma=\begin{pmatrix}\bfgamma_a & \bfgamma_b\\ \bfgamma_c & \bfgamma_d\end{pmatrix}\in \Iw_{\GSp_{2g}}^+$, and hence defines an element of $\underline{\omega}^{\kappa_{\calU}}_w$ on $\overline{\calX}_{\Iw^+,w}$. 
\end{enumerate}
\end{proof}

Thanks to Lemma \ref{Lemma: extend to the boundary} (ii), and observe that 
$$\Phi^*_{\bfgamma, \infty}:\scrC^{w-\an}_{\kappa_{\calU}}(\Iw_{\GL_g}, \pr_{2,\infty}^*\scrO_{\calX_{\Gamma(p^{\infty})}, w}\widehat{\otimes}R_{\calU})\xrightarrow[]{\sim}\scrC^{w-\an}_{\kappa_{\calU}}(\Iw_{\GL_g}, \pr_{1,\infty}^*\scrO_{\calX_{\Gamma(p^{\infty})}, w}\widehat{\otimes}R_{\calU})$$
sends bounded sections to bounded sections, we know that $T_{\bfgamma}$ extends to the boundary. We arrive at the Hecke operator $$T_{\bfgamma}: M_{\Iw^+, w}^{\kappa_{\calU}}=H^0(\overline{\calX}_{\Iw^+,w}, \underline{\omega}^{\kappa_{\calU}}_w)\rightarrow H^0(\overline{\calX}_{\Iw^+,w}, \underline{\omega}^{\kappa_{\calU}}_w)= M_{\Iw^+, w}^{\kappa_{\calU}}.$$


\paragraph{Hecke operators at $p$.} For $1\leq i\leq g$, we consider matrices $\bfu_{p,i}\in \GSp_{2g}(\Q_p)\cap M_{2g}(\Z_p)$ defined by
$$
\bfu_{p,i}:=\begin{pmatrix} \one_i \\ & p\one_{g-i}\\ & & p\one_{g-i}\\ & & & p^2\one_i\end{pmatrix}$$
for $1\leq i\leq g-1$, and
$$\bfu_{p,g}:=\begin{pmatrix}\one_g\\ & p\one_g\end{pmatrix}.$$
For later use, we write $$\bfu_{p, i}=\begin{pmatrix}\bfu_{p,i}^{\square} & \\ & \bfu_{p, i}^{\blacksquare}\end{pmatrix}$$ where $\bfu_{p, i}^{\square}$ and $\bfu_{p, i}^{\blacksquare}$ are the corresponding $g\times g$ diagonal matrices.

Notice that the $\bfu_{p,i}$-action on $\overline{\calX}_{\Gamma(p^{\infty})}$ preserves $\overline{\calX}_{\Gamma(p^{\infty}),w}$. This can be checked at the infinite level via local coordinates; \emph{i.e.,} the action of $\bfu_{p, i}$ on $\bfitz$ is given by $$\bfitz.\bfu_{p,i}=\bfu_{p,i}^{\square, -1}\bfitz\bfu_{p,i}^{\blacksquare}=\left\{\begin{array}{cl}
    \begin{pmatrix}
    p\bfitz_{1,1} & \cdots & p\bfitz_{1,g-i} & p^2\bfitz_{1,g+1-i} & \cdots & p^2\bfitz_{1,g}\\
    \vdots & & \vdots & \vdots && \vdots\\
    p\bfitz_{i,1} & \cdots & p\bfitz_{i,g-i} & p^2\bfitz_{i,g+1-i} & \cdots & p^2\bfitz_{i,g}\\
    \bfitz_{i+1,1} & \cdots & \bfitz_{i+1,g-i} & p\bfitz_{i+1,g+1-i} & \cdots & p\bfitz_{i+1,g}\\
    \vdots && \vdots & \vdots && \vdots\\
    \bfitz_{g,1} & \cdots & \bfitz_{g,g-i} & p\bfitz_{g,g+1-i} & \cdots & p\bfitz_{g,g}
    \end{pmatrix}, & \text{ if }i=1, ..., g-1 \\ \\
    p\bfitz, & \text{ if }i=g
\end{array}\right..$$ In particular, when $i=g$, the $\bfu_{p,g}$-action actually sends $\overline{\calX}_{\Gamma(p^{\infty}), w}$ into $\overline{\calX}_{\Gamma(p^{\infty}), w+1}$.

Recall the twisted left action of $\Iw^+_{\GSp_{2g}}$ on $\scrC^{w-\an}_{\kappa_{\calU}}(\Iw_{\GL_g}, \scrO_{\overline{\calX}_{\Gamma(p^{\infty}),w}}\widehat{\otimes}R_{\calU})$, given by the formula
$$\bfgamma. f:= \rho_{\kappa_{\calU}}(\bfgamma_a+\frakz \bfgamma_c)\bfgamma^*f.$$

\begin{Definition}\label{Definition:Upi}
\begin{enumerate}
\item[(i)] For $f\in \scrC^{w-\an}_{\kappa_{\calU}}(\Iw_{\GL_g}, \scrO_{\overline{\calX}_{\Gamma(p^{\infty}),w}}\widehat{\otimes}R_{\calU})$, we define $\bfu_{p,i}.f\in \scrC^{w-\an}_{\kappa_{\calU}}(\Iw_{\GL_g}, \scrO_{\overline{\calX}_{\Gamma(p^{\infty}),w}}\widehat{\otimes}R_{\calU})$ by
$$\bfu_{p,i}.f(\bfgamma'):=\bfu_{p,i}^*f(\bfu_{p,i}^{\square}\bfgamma'_0\bfu_{p,i}^{\square,-1}\bfbeta'_0)$$ where $\bfgamma'=\bfgamma'_0\bfbeta'_0\in \Iw_{\GL_g}$ with $\bfgamma'_0\in U^{\opp}_{\GL_g,1}$ and $\bfbeta'_0\in B_{\GL_g,0}$.
\item[(ii)] Suppose $f\in \scrC^{w-\an}_{\kappa_{\calU}}(\Iw_{\GL_g}, \scrO_{\overline{\calX}_{\Gamma(p^{\infty}),w}}\widehat{\otimes}R_{\calU})$ satisfies $$\bfgamma^*f=\rho_{\kappa_{\calU}}(\bfgamma_a+\frakz\bfgamma_c)^{-1}f$$ for all $\bfgamma\in \Iw^+_{\GSp_{2g}}$; i.e., $\bfgamma.f=f$. Pick a decomposition of the double coset
$$\Iw^+_{\GSp_{2g}}\bfu_{p,i}\Iw^+_{\GSp_{2g}}=\bigsqcup_{j=1}^m \bfdelta_{ij}\bfu_{p,i}\Iw^+_{\GSp_{2g}}$$
with $\bfdelta_{i,j}\in \Iw^+_{\GSp_{2g}}$. Define
$$U_{p,i}(f):=\sum_{j=1}^m \bfdelta_{i,j}.(\bfu_{p,i}.f)\in \scrC^{w-\an}_{\kappa_{\calU}}(\Iw_{\GL_g}, \scrO_{\overline{\calX}_{\Gamma(p^{\infty}),w}}\widehat{\otimes}R_{\calU}).$$
\end{enumerate}
\end{Definition}

Of course, we have to verify that $U_{p,i}(f)$ is independent to the choice of the representatives $\bfdelta_{ij}$'s. Suppose $\{\bfdelta'_{i,j}\}_{j=1}^m$ is another set of representatives. Up to re-labelling, we may assume that
$$\bfdelta'_{ij}\bfu_{p,i}\Iw^+_{\GSp_{2g}}=\bfdelta_{ij}\bfu_{p,i}\Iw^+_{\GSp_{2g}}$$
for every $j=1,\ldots, m$. Write $\bfdelta'_{ij}\bfu_{p,i}=\bfdelta_{ij}\bfu_{p,i}\bfgamma_j$ for some $\bfgamma_j\in \Iw^+_{\GSp_{2g}}$. We have to check that $\bfdelta_{ij}.(\bfu_{p,i}.f)=\bfdelta_{ij}.(\bfu_{p,i}.f)$. Indeed, if we write $\bfdelta_{ij}=\begin{pmatrix}\bfdelta_{ija}& \bfdelta_{ijb}\\ \bfdelta_{ijc}& \bfdelta_{ijd}\end{pmatrix}$, $\bfdelta'_{ij}=\begin{pmatrix}\bfdelta'_{ija}& \bfdelta'_{ijb}\\ \bfdelta'_{ijc}& \bfdelta'_{ijd}\end{pmatrix}$, and $\bfgamma_{j}=\begin{pmatrix}\bfgamma_{ja}& \bfgamma_{jb}\\ \bfgamma_{jc}& \bfgamma_{jd}\end{pmatrix}$, then for every $\bfgamma'\in \Iw_{\GL_g}$, we have
\begin{align*}
\bfdelta'_{ij}.(\bfu_{p,i}.f)(\bfgamma')
=&\rho_{\kappa_{\calU}}(\bfdelta'_{ija}+\frakz\bfdelta'_{ijc})\bfdelta_{ij}^*(\bfu_{p,i}^*f)(\bfu_{p,i}^{\square}\bfgamma'_0\bfu_{p,i}^{\square,-1}\bfbeta'_0)\\
=& (\bfdelta'_{ij}\bfu_{p,i})^*f(\bfu_{p,i}^{\square}\trans(\bfdelta'_{ija}+\frakz\bfdelta'_{ijc})\bfgamma'_0\bfu_{p,i}^{\square,-1}\bfbeta'_0)\\
=&(\bfdelta'_{ij}\bfu_{p,i})^*f(\trans(\bfgamma_{ja}+\frakz\bfgamma_{jc})\bfu_{p,i}^{\square}\trans(\bfdelta_{ija}+\frakz\bfdelta_{ijc})\bfgamma'_0\bfu_{p,i}^{\square,-1}\bfbeta'_0)\\
=&(\bfdelta'_{ij}\bfu_{p,i})^*(\rho_{\kappa_{\calU}}(\bfgamma_{ja}+\frakz\bfgamma_{jc})f)(\bfu_{p,i}^{\square}\trans(\bfdelta_{ija}+\frakz\bfdelta_{ijc})\bfgamma'_0\bfu_{p,i}^{\square,-1}\bfbeta'_0)\\
=&(\bfdelta'_{ij}\bfu_{p,i})^*(\bfgamma_j^{-1, *}f)(\bfu_{p,i}^{\square}\trans(\bfdelta_{ija}+\frakz\bfdelta_{ijc})\bfgamma'_0\bfu_{p,i}^{\square,-1}\bfbeta'_0)\\
=&(\bfdelta_{ij}\bfu_{p,i})^*f(\bfu_{p,i}^{\square}\trans(\bfdelta_{ija}+\frakz\bfdelta_{ijc})\bfgamma'_0\bfu_{p,i}^{\square,-1}\bfbeta'_0)\\
=&\rho_{\kappa_{\calU}}(\bfdelta_{ija}+\frakz\bfdelta_{ijc})(\bfdelta_{ij}\bfu_{p,i})^*f(\bfu_{p,i}^{\square}\bfgamma'_0\bfu_{p,i}^{\square,-1}\bfbeta'_0)\\
=&\bfdelta_{ij}.(\bfu_{p,i}.f)(\bfgamma')
\end{align*}
as desired. Here, the third equality follows from the identity
$$\bfu_{p,i}^{\square}\trans(\bfdelta'_{ija}+\frakz\bfdelta'_{ijc})=\trans(\bfgamma_{ja}+\frakz\bfgamma_{jc})\bfu_{p,i}^{\square}\trans(\bfdelta_{ija}+\frakz\bfdelta_{ijc}).$$

\begin{Lemma}
Suppose $f\in \scrC_{\kappa_{\calU}}^{w-\an}(\Iw_{\GL_g}, \scrO_{\overline{\calX}_{\Gamma(p^{\infty}), w}}\widehat{\otimes} R_{\calU})$ such that $\bfgamma . f = f$. Then, the section $U_{p,i}(f)\in \scrC^{w-\an}_{\kappa_{\calU}}(\Iw_{\GL_g}, \scrO_{\overline{\calX}_{\Gamma(p^{\infty}),w}}\widehat{\otimes}R_{\calU})$ satisfies $\bfgamma.(U_{p,i}(f))=U_{p,i}(f)$ for all $\bfgamma\in \Iw^+_{\GSp_{2g}}$.
\end{Lemma}

\begin{proof}
We have $$\bfgamma.(U_{p,i}(f))=\sum_{j=1}^m \bfgamma.(\bfdelta_{ij}.(\bfu_{p,i}(f)))=\sum_{j=1}^m (\bfgamma\bfdelta_{ij}).(\bfu_{p,i}(f)).$$
The last term indeed computes $U_{p,i}(f)$ because $\{\bfgamma\bfdelta_{ij}: 1\leq j \leq m\}$ is also a valid set of representatives.
\end{proof}

Consequently, we arrive at the Hecke operator $$U_{p,i}: M_{\Iw^+, w}^{\kappa_{\calU}}=H^0(\overline{\calX}_{\Iw^+,w}, \underline{\omega}^{\kappa_{\calU}}_w)\rightarrow H^0(\overline{\calX}_{\Iw^+,w}, \underline{\omega}^{\kappa_{\calU}}_w)= M_{\Iw^+, w}^{\kappa_{\calU}}.$$

\begin{Definition}\label{Definition: Hecke algebra}
The \textbf{Hecke algebra outside $Np$} is defined to be
$$\bbT^{Np}:=\Z_p\left[T_{\bfgamma};  \bfgamma\in \GSp_{2g}(\Q_{\ell})\cap M_{2g}(\Z_{\ell}), \,\,\ell\nmid Np\right]$$
and the \textbf{total Hecke algebra} is defined to be
$$\bbT:= \bbT^{Np}\otimes_{\Z_p}\Z_p[U_{p,i}; i=1, \ldots, g].$$
\end{Definition}
We now define  $U_{p}:=\prod_{i=1}^{g}U_{p,i}$ and conclude with the following proposition
\begin{Proposition} The operator $U_p$  is a compact operator on $M_{\Iw^+, w}^{\kappa_{\calU}}$. 
\end{Proposition}
\begin{proof}
Note that the action of $\bfu_{p,g} $ on $\bfitz$ is given by $p \bfitz$ and that, by definition, the action of $\prod_{i=1}^{g}\bfu_{p,i}$ on $ \scrC^{w-\an}_{\kappa_{\calU}}(\Iw_{\GL_g}, \scrO_{\overline{\calX}_{\Gamma(p^{\infty}),w}}\widehat{\otimes}R_{\calU})$ factors through the inclusion  \[
 \scrC^{(w-1)-\an}_{\kappa_{\calU}}(\Iw_{\GL_g}, \scrO_{\overline{\calX}_{\Gamma(p^{\infty}),w}}\widehat{\otimes}R_{\calU}) \hookrightarrow \scrC_{\kappa_{\calU}}^{w-an}(\Iw_{\GL_g}, \scrO_{\overline{\calX}_{\Gamma(p^{\infty}), w}}\widehat{\otimes}R_{\calU}).
\]  This means that $U_{p}$ factors as 
\[
U_{p}: H^0(\overline{\calX}_{\Iw^+,w}, \underline{\omega}^{\kappa_{\calU}}_w)\rightarrow H^0(\overline{\calX}_{\Iw^+,w+1}, \underline{\omega}^{\kappa_{\calU}}_w){\rightarrow}   H^0(\overline{\calX}_{\Iw^+,w+1}, \underline{\omega}^{\kappa_{\calU}}_{w-1}){\rightarrow} H^0(\overline{\calX}_{\Iw^+,w}, \underline{\omega}^{\kappa_{\calU}}_w),
\]
where the first arrow is the natural restriction map. 

To show the desired result, note that it is known that restrictions of the structure sheaf of $\overline{\calX}_{\Iw^+,w}$ are compact operators. Moreover, by the discussion in \cite[\S 2.2]{Hansen-PhD}, the injection of $(w-1)$-analytic functions into $w$-analytic functions is compact. The assertion then follows by combining these two facts.
\end{proof}

\begin{Remark}\label{Remark: cuspforms are stable under Hecke actions}
\normalfont Note that the subspace $S_{\Iw^+, w}^{\kappa_{\calU}}\subset M_{\Iw^+, w}^{\kappa_{\calU}}$ of $w$-overconvergent Siegel cuspforms of weight $\kappa_{\calU}$ is stable under the action of $\bbT$. Moreover, as $U_p$ is a compact operator on $M_{\Iw^+, w}^{\kappa_{\calU}}$, it is also a compact operator on $S_{\Iw^+, w}^{\kappa_{\calU}}$.
\end{Remark}


\subsection{Admissibility}\label{subsection: admissibility}
Throughout this subsection, let $(R_{\calU}, \kappa_{\calU})$ be a small weight and $w>1+r_{\calU}$. We fix an ideal $\fraka_{\calU}\subset R_{\calU}$ defining the profinite adic topology on $R_{\calU}$. In addition, we assume $p\in \fraka_{\calU}$.

The purpose of this subsection is to show that the overconvergent Siegel modular sheaf $\underline{\omega}^{\kappa_{\calU}}_w$ can be identified with the $G$-invariants of an \emph{admissible Kummer \'etale Banach sheaf} in the sense of Definition \ref{Definition: admissible Banach sheaf}, where $G$ is a finite group. Such a description allows us to apply Corollary \ref{Corollary: generalised projection formula with invariants} to the sheaf $\underline{\omega}^{\kappa_{\calU}}_w$. This will be used in \S \ref{subsection: OES}.

Firstly, we introduce the notion of $w$-compatibility inspired by \cite[\S 4.5]{AIP-2015}.

\begin{Definition}\label{Definition: w-compatible}
Let $R$ be a flat $\calO_{\C_p}$-algebra and suppose $M$ is a free $R$-module of rank $g$. We write $R_w:=R\otimes_{\calO_{\C_p}}\calO_{\C_p}/p^w$ and $M_w:=M\otimes_R R_w$. Let $\underline{\bfitm}:=(\bfitm_1, \ldots, \bfitm_g)$ be an $R_w$-basis for $M_w$. We denote by $\Fil_{\bullet}^{\underline{\bfitm}}$ the full flag
$$0\subset \langle\bfitm_1\rangle\subset \langle \bfitm_1, \bfitm_2\rangle\subset\cdots\subset\langle \bfitm_1, \ldots, \bfitm_g\rangle$$
of the free $R_w$-module $M_w$. Namely, $\Fil_i^{\underline{\bfitm}}=\langle \bfitm_1, \ldots, \bfitm_i\rangle$ for all $i=1, \ldots, g$.
\begin{enumerate}
\item[(i)] A full flag $\Fil_{\bullet}$ of the free $R$-module $M$ is called \textbf{\textit{$w$-compatible with $\underline{\bfitm}$}} if 
$$\Fil_{i}\otimes_{R}R_w=\Fil_i^{\underline{\bfitm}}$$
for all $i=1, \ldots, g$.
\item[(ii)] Suppose $\Fil_{\bullet}$ is a $w$-compatible full flag as in (i). Consider a collection $\{w_i: i=1, \ldots, g\}$ where each $w_i$ is an $R$-basis for $\Fil_i/\Fil_{i-1}$. Then $\{w_i: i=1, \ldots, g\}$ is called \textbf{\textit{$w$-compatible with $\underline{\bfitm}$}} if 
$$w_i\mod (p^wM+\Fil_{i-1})=\bfitm_i\mod \Fil_{i-1}^{\underline{\bfitm}}$$
for all $i=1, \ldots, g$.
\end{enumerate}
\end{Definition} 

Pick a positive integer $n>\sup\{w, \frac{g}{p-1}\}$. Recall from \S \ref{subsection: perfectoid Siegel modular variety} the locally free $\scrO^+_{\overline{\calX}_{\Gamma(p^n)}}$-module $\underline{\omega}^{\mathrm{mod},+}_{\Gamma(p^n)}$ over $\overline{\calX}_{\Gamma(p^n)}$. Also recall the Hodge--Tate map
$$\HT_{\Gamma(p^n)}:V\otimes_{\Z}(\Z/p^n\Z)\rightarrow \underline{\omega}^{\mathrm{mod}, +}_{\Gamma(p^n)}/p^n\underline{\omega}^{\mathrm{mod}, +}_{\Gamma(p^n)}$$
over $\overline{\calX}_{\Gamma(p^n)}$. Restricting to the $w$-ordinary locus $\overline{\calX}_{\Gamma(p^n),w}$ and composing with a natural projection, we obtain
$$\HT_{\Gamma(p^n),w}:V\otimes_{\Z}(\Z/p^n\Z)\rightarrow \underline{\omega}^{\mathrm{mod}, +}_{\Gamma(p^n),w}/p^n\underline{\omega}^{\mathrm{mod}, +}_{\Gamma(p^n),w}\twoheadrightarrow \underline{\omega}^{\mathrm{mod}, +}_{\Gamma(p^n),w}/p^w\underline{\omega}^{\mathrm{mod}, +}_{\Gamma(p^n),w}$$
where $\underline{\omega}^{\mathrm{mod}, +}_{\Gamma(p^n),w}$ is the restriction of $\underline{\omega}^{\mathrm{mod}, +}_{\Gamma(p^n)}$ on $\overline{\calX}_{\Gamma(p^n),w}$.

\begin{Lemma}
The sheaf $\underline{\omega}^{\mathrm{mod}, +}_{\Gamma(p^n),w}/p^w\underline{\omega}^{\mathrm{mod}, +}_{\Gamma(p^n),w}$ is a free $\scrO^+_{\overline{\calX}_{\Gamma(p^n),w}}/p^w$-module of rank $g$ generated by the basis $\HT_{\Gamma(p^n),w}(e_{g+1})$, ..., $\HT_{\Gamma(p^n),w}(e_{2g})$.
\end{Lemma}

\begin{proof}
Notice that $\underline{\omega}^{\mathrm{mod}, +}_{\Gamma(p^n),w}/p^w\underline{\omega}^{\mathrm{mod}, +}_{\Gamma(p^n),w}$ is locally free of rank $g$. It follows from the definition of $w$-ordinary locus that $\HT_{\Gamma(p^n),w}(e_{g+1})$, ..., $\HT_{\Gamma(p^n),w}(e_{2g})$ span $\underline{\omega}^{\mathrm{mod}, +}_{\Gamma(p^n),w}/p^w\underline{\omega}^{\mathrm{mod}, +}_{\Gamma(p^n),w}$. Hence they must form a set of free generators.
\end{proof}

We consider an adic space $\adicIW^+_w$ over $\overline{\calX}_{\Gamma(p^n),w}$ parameterising certain $w$-compatible objects. More precisely, for every affinoid open subset $\Spa(R, R^+)\subset \overline{\calX}_{\Gamma(p^n),w}$, the set $\adicIW^+_w(R, R^+)$ consists of triples $$(\bfsigma, \Fil_{\bullet}, \{w_i: i=1, \ldots, g\})$$ where 
\begin{enumerate}
\item[(i)] $\bfsigma$ is a matrix in $\Iw^+_{\GL_g}(\Z/p^n\Z)$ where $\Iw^+_{\GL_g}(\Z/p^n\Z)$ is the preimage of $T_{\GL_g}(\Z/p\Z)$ under the surjection $\GL_g(\Z/p^n\Z)\xrightarrow[]{\mathrm{mod}\,\,p} \GL_g(\Z/p\Z)$; 
\item[(ii)] both $\Fil_{\bullet}$ and $\{w_i: i=1, \ldots, g\}$ are $w$-compatible with $\left(\HT_{\Gamma(p^n),w}(e_{g+1}),\ldots, \HT_{\Gamma(p^n),w}(e_{2g})\right) \cdot\bfsigma$.
\end{enumerate}

Let $\pi: \adicIW^+_w\rightarrow \overline{\calX}_{\Gamma(p^n),w}$ denote the natural projection. To further understand $\adicIW^+_w$, we consider the following group objects in adic spaces.

\begin{Definition}\label{definition: torsors}
Let $\mathbf{B}(0,1)=\Spa(\C_p\langle X\rangle, \calO_{\C_p}\langle X\rangle)$ be the closed unit ball.
\begin{enumerate}
\item[(i)]
Define
$$\calB^{\opp}_w=\begin{pmatrix}1+p^w\mathbf{B}(0,1) \\ p^w\mathbf{B}(0,1) & 1+p^w\mathbf{B}(0,1)\\ \vdots & \vdots &  \ddots\\ p^w\mathbf{B}(0,1) & p^w\mathbf{B}(0,1) & \cdots & 1+p^w\mathbf{B}(0,1)\end{pmatrix}$$
In particular, the underlying adic space is isomorphic to a $\frac{1}{2}g(g+1)$-dimension ball of radius $p^{-w}$.
\item[(ii)]
Define
\begin{align*}
\calT_{\GL_g,0}^{(w)}=&\begin{pmatrix}\Z_p^{\times}+p^w\mathbf{B}(0,1) \\  & \Z_p^{\times}+p^w\mathbf{B}(0,1)\\  &  &  \ddots\\  &  & & \Z_p^{\times}+p^w\mathbf{B}(0,1)\end{pmatrix}\\
=\bigsqcup_{h_1, \ldots, h_g\in S}&\begin{pmatrix}h_1+p^w\mathbf{B}(0,1) \\  & h_2+p^w\mathbf{B}(0,1)\\  &  &  \ddots\\  &  & & h_g+p^w\mathbf{B}(0,1)\end{pmatrix}
\end{align*}
where $S\subset \Z_p^{\times}$ is a set of representatives of $\Z_p^{\times}/(1+p^n\Z_p)$. In particular, the underlying adic space is isomorphic to the disjoint union of $(p-1)^gp^{g(n-1)}$ copies of $g$-dimension ball of radius $p^{-w}$.
\item[(iii)]
Define
\begin{align*}
\calU^{\opp, (w)}_{\GL_g,1}=&\begin{pmatrix}1 \\ p\Z_p+p^w\mathbf{B}(0,1) & 1\\ \vdots & \vdots &  \ddots\\ p\Z_p+p^w\mathbf{B}(0,1) & p\Z_p+p^w\mathbf{B}(0,1) & \cdots & 1\end{pmatrix}\\
=\bigsqcup_{\substack{h_{i,j}\in S\\1\leq j<i\leq g}}&\begin{pmatrix}1 \\ h_{2,1}+p^w\mathbf{B}(0,1) & 1\\ \vdots & \vdots &  \ddots\\ h_{g,1}+p^w\mathbf{B}(0,1) & h_{g,2}+p^w\mathbf{B}(0,1) & \cdots & 1\end{pmatrix}
\end{align*}
where $S\subset \Z_p^{\times}$ is a set of representatives of $p\Z_p/p^n\Z_p$. In particular, the underlying adic space is isomorphic to the disjoint union of $p^{\frac{1}{2}g(g-1)(n-1)}$ copies of $\frac{1}{2}g(g-1)$-dimension ball of radius $p^{-w}$.

\end{enumerate}
All of the group structures are given by the matrix multiplications.
\end{Definition}

\begin{Remark}
\normalfont The $(\C_p, \calO_{\C_p})$-points of $\calB^{\opp}_w$, $\calT_{\GL_g,0}^{(w)}$, and $\calU^{\opp, (w)}_{\GL_g,1}$ coincide with the groups $B^{\opp}_w$, $T^{(w)}_{\GL_g, 0}$, and $U^{\opp, (w)}_{\GL_g,1}$, respectively. This justifies the notations.
\end{Remark}

\begin{Lemma}
$\adicIW^+_w$ is a $\calU^{\opp, (w)}_{\GL_g,1}\times\calT_{\GL_g,0}^{(w)}\times U_{\GL_g,1}(\Z/p^n\Z)$-torsor over $\overline{\calX}_{\Gamma(p^n),w}$ where $U_{\GL_g,1}(\Z/p^n\Z)$ is the kernel of the natural surjection $U_{\GL_g}(\Z/p^n\Z)\xrightarrow[]{\mathrm{mod}\,\,p}U_{\GL_g}(\Z/p\Z)$. Namely, locally on $\overline{\calX}_{\Gamma(p^n),w}$, we have an identification
$$\adicIW^+_w\simeq \overline{\calX}_{\Gamma(p^n),w}\times_{\Spa(\C_p, \calO_{\C_p})}\left(\calU^{\opp, (w)}_{\GL_g,1}\times\calT_{\GL_g,0}^{(w)}\times U_{\GL_g,1}(\Z/p^n\Z)\right)$$ where $\calU^{\opp, (w)}_{\GL_g,1}\times\calT_{\GL_g,0}^{(w)}\times U_{\GL_g,1}(\Z/p^n\Z)$ acts from the right by matrix multiplication.
\end{Lemma}

\begin{proof}
This is clear from the construction. 
\end{proof}

Using the adic space $\adicIW^+_w$, we construct two auxiliary sheaves $\widetilde{\underline{\omega}}^{\kappa_{\calU},+}_{n,w}$ and $\widetilde{\underline{\omega}}^{\kappa_{\calU}}_{n,w}$. 

\begin{Definition}\label{definition: auxiliary sheaves}
\begin{enumerate}
\item[(i)] The sheaf $\widetilde{\underline{\omega}}^{\kappa_{\calU},+}_{n,w}$ over $\overline{\calX}_{\Gamma(p^n),w}$ is defined to be
$$\widetilde{\underline{\omega}}^{\kappa_{\calU},+}_{n,w}:=\left(\pi_*\scrO^+_{\adicIW^+_w}\widehat{\otimes}R_{\calU}\right)[\kappa_{\calU}^{\vee}];$$
i.e., the subsheaf of $\pi_*\scrO^+_{\adicIW^+_w}\widehat{\otimes}R_{\calU}$ consisting of those sections on which $T_{\GL_g,0}$-acts through the character $\kappa_{\calU}^{\vee}$ and $U_{\GL_g,1}(\Z/p^n\Z)$ acts trivially.
\item[(ii)] The sheaf
$$\widetilde{\underline{\omega}}^{\kappa_{\calU}}_{n,w}:=\left(\pi_*\scrO_{\adicIW^+_w}\widehat{\otimes}R_{\calU}\right)[\kappa_{\calU}^{\vee}]$$
is defined similarly.
\end{enumerate}
\end{Definition}

\begin{Remark}\label{remark: no difference}
\normalfont Since $\kappa_{\calU}$ is $w$-analytic, the character $\kappa_{\calU}^{\vee}:T_{\GL_g,0}\rightarrow R_{\calU}^{\times}$ extends to a character on $\calT^{(w)}_{\GL_g,0}$; namely,
a character 
$$\kappa_{\calU}^{\vee}: \calT^{(w)}_{\GL_g,0}(R, R^+)\rightarrow (R^+\widehat{\otimes}R_{\calU})^{\times}$$
for every affinoid $(\C_p, \calO_{\C_p})$-algebra $(R, R^+)$. It turns out, in Definition \ref{definition: auxiliary sheaves}, there is no difference between taking $\kappa_{\calU}^{\vee}$-eigenspaces with respect to $T_{\GL_g,0}$- or $\calT^{(w)}_{\GL_g,0}$-actions.
\end{Remark}

\begin{Lemma}\label{Lemma: omega is projective Banach sheaf}
The sheaf $\widetilde{\underline{\omega}}^{\kappa_{\calU}}_{n,w}$ is a projective Banach sheaf of $\scrO_{\overline{\calX}_{\Gamma(p^n),w}}\widehat{\otimes}R_{\calU}$-modules in the sense of Definition \ref{Definition: Banach sheaf} (ii). Moreover, $\widetilde{\underline{\omega}}^{\kappa_{\calU},+}_{n,w}$ is an integral model of $\widetilde{\underline{\omega}}^{\kappa_{\calU}}_{n,w}$ in the sense of Definition \ref{Definition: Banach sheaf} (iv).
\end{Lemma}

\begin{proof}
Let $\{\calV_{n, i} : i\in I\}$ be an affinoid open covering of $\overline{\calX}_{\Gamma(p^n), w}$ such that $\underline{\omega}_{\Gamma(p^n)}^{\mathrm{mod}, +}|_{\calV_{n, i}}$ is free, for every $i\in I$. By choosing a basis for $\underline{\omega}_{\Gamma(p^n)}^{\mathrm{mod}, +}|_{\calV_{n, i}}$, we can identify \[
    \widetilde{\underline{\omega}}_{n, w}^{\kappa_{\calU}, +}|_{\calV_{n, i}} \simeq \scrO^+_{\calV_{n, i}} \langle T_{st}: 1\leq s<t\leq g\rangle \widehat{\otimes} R_{\calU}
\] 
which is the $p$-adic completion of a free $\scrO^+_{\calV_{n, i}} \widehat{\otimes} R_{\calU}$-module, as desired.
\end{proof}

We also consider the associated $p$-adically completed sheaves on the Kummer \'etale site.

\begin{Definition}
Let 
$$
\widetilde{\underline{\omega}}^{\kappa_{\calU},+}_{n,w,\ket}:= \varprojlim_m\left(\widetilde{\underline{\omega}}^{\kappa_{\calU},+}_{n,w} \bigotimes_{\scrO^+_{\overline{\calX}_{\Gamma(p^n),w}}}\scrO^+_{\overline{\calX}_{\Gamma(p^n),w}, \ket}/p^m\right)
$$
and let
$$
\widetilde{\underline{\omega}}^{\kappa_{\calU}}_{n,w,\ket}:=\widetilde{\underline{\omega}}^{\kappa_{\calU},+}_{n,w,\ket}[\frac{1}{p}].
$$
\end{Definition}

By Lemma \ref{Lemma: omega is projective Banach sheaf} and Corollary \ref{Corollary: Banach induce Kummer etale Banach}, $\widetilde{\underline{\omega}}^{\kappa_{\calU}}_{n,w,\ket}$ is a projective Kummer \'etale Banach sheaf of $\scrO_{\overline{\calX}_{\Gamma(p^n),w}, \ket}\widehat{\otimes}R_{\calU}$-modules in the sense of Definition \ref{Definition: Kummer etale Banach sheaf} (ii). Moreover, $\widetilde{\underline{\omega}}^{\kappa_{\calU},+}_{n,w,\ket}$ is an integral model of $\widetilde{\underline{\omega}}^{\kappa_{\calU}}_{n,w,\ket}$ in the sense of Definition \ref{Definition: Kummer etale Banach sheaf} (iv).

In fact, we show that the Kummer \'etale Banach sheaf $\widetilde{\underline{\omega}}^{\kappa_{\calU}}_{n,w,\ket}$ is \emph{admissible}.

\begin{Lemma}\label{Lemma: omega is admissible Kummer etale Banach sheaf}
The sheaf $\widetilde{\underline{\omega}}^{\kappa_{\calU}}_{n,w,\ket}$ is an admissible Kummer \'etale Banach sheaf of $\scrO_{\overline{\calX}_{\Gamma(p^n),w,\ket}}\widehat{\otimes}R_{\calU}$-modules (in the sense of Definition \ref{Definition: admissible Banach sheaf}) with integral model $\widetilde{\underline{\omega}}^{\kappa_{\calU},+}_{n,w,\ket}$.
\end{Lemma}

\begin{proof}
The proof is inspired by the discussion in \cite[\S 8.1]{AIP-2015}. We provide a sketch of proof. 

To simplify the notation, we write $\scrF^+=\widetilde{\underline{\omega}}_{n, w, \ket}^{\kappa_{\calU}, +}$ and $\scrF=\widetilde{\underline{\omega}}_{n, w, \ket}^{\kappa_{\calU}}$. We also write $\scrF^+_m:=\scrF^+/\fraka_{\calU}^m$, for every $m\in \Z_{\geq 1}$.

Let $\mathfrak{U}=\{\calV_{n, i} : i\in I\}$ be an open affinoid covering for $\overline{\calX}_{\Gamma(p^n), w}$ such that $\underline{\omega}_{\Gamma(p^n)}^{\mathrm{mod}, +}|_{\calV_{n, i}}$ is free, for every $i\in I$. We equip each $\calV_{n,i}$ the induced log structure from $\overline{\calX}_{\Gamma(p^n), w}$. By choosing a basis for $\underline{\omega}_{\Gamma(p^n)}^{\mathrm{mod}, +}|_{\calV_{n, i}}$, we can identify \[
    \scrF^+|_{\calV_{n, i}} \simeq \scrO^+_{\calV_{n, i}}\langle T_{st}: 1\leq s<t\leq g\rangle \widehat{\otimes} R_{\calU}
\] 
which is the $p$-adic completion of a free $\scrO^+_{\calV_{n, i}} \widehat{\otimes} R_{\calU}$-module.
Modulo $\fraka_{\calU}^m$, we obtain
\[
\scrF^+_m|_{\calV_{n, i}} \simeq \left(\scrO^+_{\calV_{n, i}} \otimes_{\Z_p} (R_{\calU}/\fraka_{\calU}^m)\right)[T_{st}: 1\leq s<t\leq g].
\]
For any $d\in \Z_{\geq 0}$, consider the subsheaf $(\scrF^+_m|_{\calV_{n,i}})^{\leq d}\subset \scrF^+_m|_{\calV_{n,i}}$ consisting of those polynomials of degree $\leq d$, and consider
$$
\scrF^+_{m,d}:=\ker\left( \prod_{i\in I}(\scrF^+_m|_{\calV_{n,i}})^{\leq d}\rightarrow\prod_{i, j\in I} \scrF^+_m|_{\calV_{n,i}\cap \calV_{n,j}}\right).
$$
Then each $\scrF^+_{m,d}$ is a coherent $\scrO^+_{\overline{\calX}_{\Gamma(p^n), w, \ket}}\otimes_{\Z_p} (R_{\calU}/\fraka_{\calU}^m)$-module and we have $\scrF^+_m=\varinjlim_d \scrF^+_{m,d}$, as desired.
\end{proof}

Next, we are going to relate the overconvergent Siegel modular sheaves $\underline{\omega}^{\kappa_{\calU},+}_w$ and $\underline{\omega}^{\kappa_{\calU}}_w$ with the auxiliary sheaves $\widetilde{\underline{\omega}}^{\kappa_{\calU},+}_{n,w}$ and $\widetilde{\underline{\omega}}^{\kappa_{\calU}}_{n,w}$. To this end, we need two intermediate sheaves $\underline{\omega}_{n, w}^{\kappa_{\calU},+}$ and $\underline{\omega}_{n, w}^{\kappa_{\calU}}$ over $\overline{\calX}_{\Gamma(p^n),w}$.

\begin{Definition}\label{Definition: overconvergent automorphic sheaf at Gamma(p^n)}
Let $h_{\Gamma(p^n)}: \overline{\calX}_{\Gamma(p^{\infty}),w}\rightarrow \overline{\calX}_{\Gamma(p^n),w}$ be the natural projection. 
\begin{enumerate}
\item[(i)] The subsheaf $\underline{\omega}_{n,w}^{\kappa_{\calU}}$ of $h_{\Gamma(p^n), *}\scrC^{w-\an}_{\kappa_{\calU}}(\Iw_{\GL_g}, \scrO_{\overline{\calX}_{\Gamma(p^{\infty}), w}}\widehat{\otimes}R_{\calU})$ is defined as follows. For every affinoid open subset $\calV\subset \overline{\calX}_{\Gamma(p^n), w}$ with $\calV_{\infty} = h_{\Gamma(p^n)}^{-1}(\calV)$, we put $$\underline{\omega}_{n,w}^{\kappa_{\calU}}(\calV):=\left\{f\in C^{w-\an}_{\kappa_{\calU}}(\Iw_{\GL_g}, \scrO_{\overline{\calX}_{\Gamma(p^{\infty}), w}}(\calV_{\infty})\widehat{\otimes}R_{\calU}): \bfgamma^*f=\rho_{\kappa_{\calU}}(\bfgamma_a+\frakz\bfgamma_c)^{-1} f,\,\,\forall \bfgamma=\begin{pmatrix}\bfgamma_a & \bfgamma_b\\ \bfgamma_c & \bfgamma_d\end{pmatrix}\in \Gamma(p^n)\right\}.$$ 
\item[(ii)] The subsheaf $\underline{\omega}_{n,w}^{\kappa_{\calU},+}$ of $h_{\Gamma(p^n), *}\scrC^{w-\an}_{\kappa_{\calU}}(\Iw_{\GL_g}, \scrO^+_{\overline{\calX}_{\Gamma(p^{\infty}), w}}\widehat{\otimes}R_{\calU})$ is defined as follows. For every affinoid open subset $\calV\subset \overline{\calX}_{\Gamma(p^n), w}$ with $\calV_{\infty} = h_{\Gamma(p^n)}^{-1}(\calV)$, we put $$\underline{\omega}_{n,w}^{\kappa_{\calU},+}(\calV):=\left\{f\in C^{w-\an}_{\kappa_{\calU}}(\Iw_{\GL_g}, \scrO^+_{\overline{\calX}_{\Gamma(p^{\infty}), w}}(\calV_{\infty})\widehat{\otimes}R_{\calU}): \bfgamma^*f=\rho_{\kappa_{\calU}}(\bfgamma_a+\frakz\bfgamma_c)^{-1} f,\,\,\forall \bfgamma=\begin{pmatrix}\bfgamma_a & \bfgamma_b\\ \bfgamma_c & \bfgamma_d\end{pmatrix}\in \Gamma(p^n)\right\}.$$ 
\end{enumerate}
\end{Definition}

\begin{Remark}\label{remark: invariants}
\normalfont Let $h_n: \overline{\calX}_{\Gamma(p^n),w}\rightarrow \overline{\calX}_{\Iw^+,w}$ denote the natural projection. Then the overconvergent Siegel modular sheaf $\underline{\omega}^{\kappa_{\calU}}_w$ can be identified as the $\Iw^+_{\GSp_{2g}}/\Gamma(p^n)$-invariants of the sheaf $h_{n,*}\underline{\omega}_{n,w}^{\kappa_{\calU}}$ with respect to the ``twisted'' action
$\bfgamma.f:=\rho_{\kappa_{\calU}}(\bfgamma_a+\frakz\bfgamma_c) \bfgamma^*f$
for every $\bfgamma=\begin{pmatrix}\bfgamma_a & \bfgamma_b\\ \bfgamma_c & \bfgamma_d\end{pmatrix}\in \Gamma(p^n)$ and $f\in \underline{\omega}_{n,w}^{\kappa_{\calU}}$. Similar result holds for the integral sheaf $\underline{\omega}^{\kappa_{\calU},+}_w$.
\end{Remark}

\begin{Proposition}\label{Proposition: omega is admissible}
There is a natural isomorphism of $\scrO^+_{\overline{\calX}_{\Gamma(p^n),w}}\widehat{\otimes}R_{\calU}$-modules $\Psi^+: \underline{\omega}^{\kappa_{\calU},+}_{n,w}\simeq \widetilde{\underline{\omega}}^{\kappa_{\calU},+}_{n,w}$. Inverting $p$, we obtain a natural isomorphism of $\scrO_{\overline{\calX}_{\Gamma(p^n),w}}\widehat{\otimes}R_{\calU}$-modules $\Psi: \underline{\omega}^{\kappa_{\calU}}_{n,w}\simeq \widetilde{\underline{\omega}}^{\kappa_{\calU}}_{n,w}$.
\end{Proposition}

\begin{proof}As a preparation, consider the pullback diagram \[
    \begin{tikzcd}
        \adicIW_{w, \infty}^+ \arrow[r]\arrow[d, "\pi_{\infty}"'] & \adicIW_w^+\arrow[d, "\pi"]\\
        \overline{\calX}_{\Gamma(p^{\infty}),w} \arrow[r, "h_{\Gamma(p^n)}"] & \overline{\calX}_{\Gamma(p^n),w}
    \end{tikzcd}
\]
in the category of adic spaces. To show the existence of such a pullback, it suffices to check this locally on $\overline{\calX}_{\Gamma(p^n),w}$. Recall that $\adicIW_w^+$ is a $\calU_{\GL_g, 1}^{\opp, (w)}\times \calT_{\GL_g,0}^{(w)}\times U_{\GL_g,1}(\Z/p^n\Z)$-torsor over $\overline{\calX}_{\Gamma(p^n),w}$, and $\calU_{\GL_g, 1}^{\opp, (w)}\times \calT_{\GL_g,0}^{(w)}$ is isomorphic to finitely many copies of $\mathbf{B}(0,1)^{\frac{g(g+1)}{2}}$. It remains to show that the fibre product $\overline{\calX}_{\Gamma(p^{\infty}),w}\times_{\Spa(\C_p, \calO_{\C_p})} \mathbf{B}(0,1)^{\frac{g(g+1)}{2}}$ exists. Indeed, by \cite[Proposition 6.3.3 (3)]{Scholze-Weinstein-Berkeley}, such a fibre product exists and is a sousperfectoid space. In addition, we know that the pullback $\adicIW_{w, \infty}^+$ is likewise a $\calU_{\GL_g, 1}^{\opp, (w)}\times \calT_{\GL_g,0}^{(w)}\times U_{\GL_g,1}(\Z/p^n\Z)$-torsor over $\overline{\calX}_{\Gamma(p^{\infty}),w}$. 

For every affinoid open $\calV\subset \overline{\calX}_{\Gamma(p^n),w}$ and $\calV_{\infty}:=h_{\Gamma(p^n)}^{-1}\calV$, the desired isomorphism $\Psi^+$ will be established via a sequence of isomorphisms 
$$ \Psi^+: \underline{\omega}_{n,w}^{\kappa_{\calU},+}(\calV) \xrightarrow[\Psi_1]{\sim} \omega^{(1)} \xrightarrow[\Psi_2]{\sim} \omega^{(2)}\xrightarrow[\Psi_3]{\sim} \widetilde{\underline{\omega}}_{n,w}^{\kappa_{\calU},+}(\calV),
$$
where 
$$
\omega^{(1)} := \left\{ f\in C_{\kappa_{\calU}^{\vee}}^{w-\an}(\Iw_{\GL_g}, \scrO^+_{\calV_{\infty}}(\calV_{\infty})\widehat{\otimes}R_{\calU}): \bfgamma^* f = \rho_{\kappa_{\calU}^{\vee}}(\bfgamma_a^{\ddagger} + \frakz \bfgamma_c^{\ddagger}) f, \quad \forall \bfgamma = \begin{pmatrix} \bfgamma_a & \bfgamma_b \\ \bfgamma_c & \bfgamma_d\end{pmatrix} \in \Gamma(p^n)\right\}
$$
and
$$
\omega^{(2)} := \left\{f\in \pi_{\infty, *}\scrO^+_{\adicIW_{w, \infty}^+}(\calV_{\infty})\widehat{\otimes}R_{\calU}: \begin{array}{l}
        \bfgamma^* f = f, \quad \bftau^*f = \kappa_{\calU}^{\vee}(\bftau)f, \quad \bfnu^* f = f  \\
        \forall (\bfgamma, \bftau, \bfnu)\in \Gamma(p^n)\times T_{\GL_g, 0}\times U_{\GL_g,1}(\Z/p^n\Z) 
    \end{array}\right\}.
$$
Here, for any $\bfdelta \in M_g$, we write $\bfdelta^{\ddagger} := \oneanti_g \trans\bfdelta \oneanti_g$, which can be viewed as the ``transpose with respect to the anti-diagonal''. Notice that $\frakz^{\ddagger} = \frakz$. 

\paragraph{Construction of $\Psi_1$.}
Observe that there is an isomorphism \[
    \Psi_1: C_{\kappa_{\calU}}^{w-\an}(\Iw_{\GL_g}, \scrO^+_{\calV_{\infty}}(\calV_{\infty})\widehat{\otimes} R_{\calU}) \rightarrow C_{\kappa_{\calU}^{\vee}}^{w-\an}(\Iw_{\GL_g}, \scrO^+_{\calV_{\infty}}(\calV_{\infty})\widehat{\otimes} R_{\calU})
\] 
defined by 
$$\Psi_1(f)(\bfgamma'):=f(\oneanti_g \trans\bfgamma'^{-1}\oneanti_g)$$
for all $f\in C_{\kappa_{\calU}}^{w-\an}(\Iw_{\GL_g}, \scrO^+_{\calV_{\infty}}(\calV_{\infty})\widehat{\otimes} R_{\calU})$ and $\bfgamma'\in \Iw_{\GL_g}$.

We claim that $\Psi_1$ induces an isomorphism $\underline{\omega}_{n,w}^{\kappa_{\calU},+}(\calV)\simeq \omega^{(1)}$. It suffices to check that if $\bfgamma^* f = \rho_{\kappa_{\calU}}(\bfgamma_a + \frakz\bfgamma_c)^{-1} f$ for every $\bfgamma = \begin{pmatrix}\bfgamma_a & \bfgamma_b\\ \bfgamma_c & \bfgamma_d\end{pmatrix} \in \Gamma(p^n)$, then $\bfgamma^*(\Psi_1(f)) = \rho_{\kappa_{\calU}^{\vee}}(\bfgamma_a^{\ddagger} + \frakz\bfgamma_c^{\ddagger})\Psi_1(f)$. Indeed, for any $\bfgamma'\in \Iw_{\GL_g}$, we have \begin{align*}
    \bfgamma^*(\Psi_1(f))(\bfgamma') & = \rho_{\kappa_{\calU}}(\bfgamma_a + \frakz\bfgamma_c)^{-1} f(\oneanti_g \trans\bfgamma'^{-1}\oneanti_g)\\
    & = f\left(\trans(\bfgamma_a + \frakz \bfgamma_c)^{-1}\oneanti_g \trans\bfgamma'^{-1}\oneanti_g\right)\\
    & = f\left(\oneanti_g \oneanti_g\trans(\bfgamma_a + \frakz \bfgamma_c)^{-1}\oneanti_g \trans\bfgamma'^{-1}\oneanti_g\right)\\
    & = f\left(\oneanti_g \trans(\oneanti_g \bfgamma_a \oneanti_g + \oneanti_g\frakz\oneanti_g \oneanti_g \bfgamma_c \oneanti_g)^{-1}\trans\bfgamma'^{-1}\oneanti_g\right)\\
    & = f\left(\oneanti_g \trans(\trans(\bfgamma_a^{\ddagger}+\frakz\bfgamma_c^{\ddagger})\bfgamma')^{-1}\oneanti_g\right)\\
    & = \rho_{\kappa_{\calU}^{\vee}}(\bfgamma_a^{\ddagger} + \frakz\bfgamma_c^{\ddagger}) \Psi_1(f)(\bfgamma').
\end{align*}

\paragraph{Construction of $\Psi_2$.}
To construct $\Psi_2$, consider $\fraks^{\ddagger} = \begin{pmatrix}\fraks_g & \cdots & \fraks_1\end{pmatrix}\in \underline{\omega}_{\Gamma(p^{\infty})}(\calV_{\infty})^g$. Recall that $\fraks = \begin{pmatrix}\fraks_1\\ \vdots \\ \fraks_g\end{pmatrix}$ and thus \[
    \fraks^{\ddagger} = \trans\fraks\oneanti_g.
\] Moreover, for any $\bfgamma = \begin{pmatrix}\bfgamma_a & \bfgamma_b \\ \bfgamma_c & \bfgamma_d\end{pmatrix} \in \Gamma(p^n)$, we have $\bfgamma^* \fraks = \trans(\bfgamma_a + \frakz\bfgamma_c)\fraks$ by (\ref{eq: action on fake Hasse invariants}). Hence $$
    \bfgamma^* \fraks^{\ddagger}  = \trans(\bfgamma^* \fraks) \oneanti_g = \trans(\trans(\bfgamma_a + \frakz \bfgamma_c)\fraks)\oneanti_g= \trans\fraks \oneanti_g \oneanti_g (\bfgamma_a + \frakz \bfgamma_c)\oneanti_g = (\trans\fraks \oneanti_g) \trans(\bfgamma_a^{\ddagger} + \frakz \bfgamma_c^{\ddagger})= \fraks^{\ddagger} \trans(\bfgamma_a^{\ddagger} + \frakz \bfgamma_c^{\ddagger}).
$$

Let $\Fil_{\bullet}^{\ddagger}$ be the full flag of the free $\scrO^+_{\calV_{\infty}}(\calV_{\infty})$-module $\underline{\omega}_{\Gamma(p^{\infty})}(\calV_{\infty})$ given by \[
    \Fil_{\bullet}^{\ddagger} = 0 \subset \langle \fraks_g \rangle \subset \langle \fraks_g, \fraks_{g-1} \rangle\subset \cdots \langle \fraks_g, \ldots, \fraks_1\rangle
\] and let $w_i^{\ddagger}$ be the image of $\fraks_{g+1-i}$ in $\Fil_{i}^{\ddagger}/\Fil_{i-1}^{\ddagger}$, for all $i=1, \ldots, g$. Then the triple $(\one_g, \Fil_{\bullet}^{\ddagger}, \{w_i^{\ddagger}\})$ defines a section of the $\calU_{\GL_g, 1}^{\opp, (w)}\times \calT_{\GL_g, 0}^{(w)}\times U_{\GL_g}(\Z/p^n\Z)$-torsor $\pi_{\infty}^{-1}(\calV_{\infty}) \rightarrow \calV_{\infty}$. Consequently, one obtains an isomorphism \[
    \calU_{\GL_g, 1}^{\opp, (w)}\times \calT_{\GL_g, 0}^{(w)} \times U_{\GL_g,1}(\Z/p^n\Z) \xrightarrow{\sim} \pi_{\infty}^{-1}(\calV_{\infty}), \quad \bfgamma' \mapsto (\one_g, \Fil_{\bullet}^{\ddagger}, \{w_i^{\ddagger}\}) \cdot \bfgamma'
\] and thus an isomorphism \begin{align*}
    \Phi: \pi_{\infty, *}\scrO^+_{\adicIW_{w, \infty}^+}(\calV_{\infty})\widehat{\otimes} R^+_{\calU} & \xrightarrow{\sim} \left\{
        \text{ analytic functions }
        U_{\GL_g, 1}^{\opp, (w)}\times T_{\GL_g, 0}^{(w)}\times U_{\GL_g,1}(\Z/p^n\Z) \rightarrow \scrO^+_{\calV_{\infty}}(\calV_{\infty})\widehat{\otimes}R_{\calU} 
\right\}\\
    f & \mapsto \left(\bfgamma'\mapsto f((\one_g, \Fil_{\bullet}^{\ddagger}, \{w_i^{\ddagger}\}) \cdot \bfgamma')\right).
\end{align*} We claim that if $\bfgamma^* f = f$ for any $\bfgamma = \begin{pmatrix}\bfgamma_a & \bfgamma_b \\ \bfgamma_c & \bfgamma_d\end{pmatrix}\in \Gamma(p^n)$, then $\bfgamma^*\Phi(f) = \rho_{\kappa_{\calU}^{\vee}}(\bfgamma_a^{\ddagger} + \frakz\bfgamma_c^{\ddagger})\Phi(f)$. Indeed, for any $\bfgamma' \in U_{\GL_g, 1}^{\opp, (w)}\times T_{\GL_g, 0}^{(w)}\times U_{\GL_g,1}(\Z/p^n\Z)$, we have \begin{align*}
    (\bfgamma^*\Phi(f)) (\bfgamma') & = (\bfgamma^* f)(\bfgamma^*(\psi^{\ddagger}, \Fil_{\bullet}^{\ddagger}, \{w_i^{\ddagger}\}) \cdot \bfgamma')\\ 
    & = f\left((\psi^{\ddagger}, \Fil_{\bullet}^{\ddagger}, \{w_i^{\ddagger}\})\cdot  \trans(\bfgamma_a^{\ddagger} + \frakz\bfgamma_c^{\ddagger})\cdot\bfgamma'\right)\\
    & = \rho_{\kappa_{\calU}^{\vee}}(\bfgamma_a^{\ddagger} + \frakz\bfgamma_c^{\ddagger}) \Phi(f)(\bfgamma'),
\end{align*} where the second equation follows from the identity $\bfgamma^* \fraks^{\ddagger} = \fraks^{\ddagger} \trans(\bfgamma_a^{\ddagger} + \frakz\bfgamma_c^{\ddagger})$.

On the other hand, we can identify $\omega^{(1)}$ with the set of analytic functions
$$f: U_{\GL_g, 1}^{\opp, (w)}\times T_{\GL_g, 0}^{(w)}\times U_{\GL_g,1}(\Z/p^n\Z) \rightarrow \scrO^+_{\calV_{\infty}}(\calV_{\infty})\widehat{\otimes}R_{\calU} $$
satisfying
\begin{itemize}
\item $f(\bfupsilon \bftau \bfnu) = \kappa_{\calU}^{\vee}(\bftau)f(\bfupsilon)$ for all $(\bfupsilon, \bftau, \bfnu)\in U_{\GL_g, 1}^{\opp, (w)}\times T_{\GL_g, 0}^{(w)}\times U_{\GL_g}(\Z/p^n\Z)$;
\item $\bfgamma^* f = \rho_{\kappa_{\calU}^{\vee}}(\bfgamma_a^{\ddagger}+ \frakz\bfgamma_c^{\ddagger})f$ for all $\bfgamma = \begin{pmatrix}\bfgamma_a & \bfgamma_b \\ \bfgamma_c & \bfgamma_d\end{pmatrix}\in \Gamma(p^n)$.
\end{itemize}

Therefore, putting $\Psi_2 := \Phi^{-1}$, one obtains the desired isomorphism \[
    \Psi_2: \omega^{(1)} \xrightarrow{\sim} \omega^{(2)}.
\]

\paragraph{Construction of $\Psi_3$.} By the construction of $\underline{\omega}_{n,w}^{\kappa_{\calU}, +}$ and Lemma \ref{Lemma: structure sheaves at the infinite level and the structure sheaves at the Iwahori level}, one immediately obtains an identification of $\omega^{(2)}$ with $\widetilde{\underline{\omega}}_{n,w}^{\kappa_{\calU},+}(\calV)$. We simply take $\Psi_3$ to be this identification. 

Putting everything together, the composition $\Psi^+=\Psi_3\circ\Psi_2\circ\Psi_1$ yields an isomorphism
$$\Psi^+:\underline{\omega}_{n,w}^{\kappa_{\calU},+}(\calV)\xrightarrow{\sim} \widetilde{\underline{\omega}}_{n,w}^{\kappa_{\calU},+}(\calV).$$
It is also straightforward to check that the construction is functorial in $\calV$. By gluing, we arrive at an isomorphism
$$\Psi^+:\underline{\omega}_{n,w}^{\kappa_{\calU},+}\xrightarrow{\sim} \widetilde{\underline{\omega}}_{n,w}^{\kappa_{\calU},+}.$$
\end{proof}

Consider the $p$-adically completed sheaf of $\scrO_{\overline{\calX}_{\Iw^+,w,\ket}}$-modules associated with $\underline{\omega}^{\kappa_{\calU}}_w$; namely, let
$$
\underline{\omega}^{\kappa_{\calU},+}_{w,\ket}:=\varprojlim_m\left(\underline{\omega}^{\kappa_{\calU},+}_{w}\otimes_{\scrO^+_{\overline{\calX}_{\Iw^+,w}}}\scrO^+_{\overline{\calX}_{\Iw^+,w,\ket}}/p^m\right)
$$
and
$$\underline{\omega}^{\kappa_{\calU}}_{w,\ket}:=\underline{\omega}^{\kappa_{\calU},+}_{w,\ket}[\frac{1}{p}].$$

By Remark \ref{remark: invariants} and Proposition \ref{Proposition: omega is admissible}, $\underline{\omega}^{\kappa_{\calU}}_w$ can be identified with the sheaf of $\Iw^+_{\GSp_{2g}}/\Gamma(p^n)$-invariants of $h_{n,*}\widetilde{\underline{\omega}}_{n,w}^{\kappa_{\calU}}$. Hence, $\underline{\omega}^{\kappa_{\calU}}_{w,\ket}$ can be identified with the sheaf of $\Iw^+_{\GSp_{2g}}/\Gamma(p^n)$-invariants of $h_{n,*}\widetilde{\underline{\omega}}_{n,w, \ket}^{\kappa_{\calU}}$, later of which is an admissible Kummer \'etale Banach sheaf of $\scrO_{\overline{\calX}_{\Iw^+,w,\ket}}\widehat{\otimes}R_{\calU}$-modules by Lemma \ref{Lemma: omega is admissible Kummer etale Banach sheaf} and Lemma \ref{Lemma: pushforward along finite Kummer etale map}. Consequently, such a description allows us to apply Corollary \ref{Corollary: generalised projection formula with invariants} to the sheaf $\underline{\omega}^{\kappa_{\calU}}_{w,\ket}$. This will be used in the proof of Lemma \ref{Lemma: Leray spectral sequence for the automorphic sheaf}.


\subsection{Classical Siegel modular forms}\label{subsection: classical forms}
In this subsection, we show that the space of $w$-overconvergent Siegel modular forms does contain all of the classical Siegel modular forms.

Let $k = (k_1, ..., k_g)\in \Z_{\geq 0}^g$ be a dominant weight and consider $k^{\vee}=(-k_g, \ldots, -k_1)$. Let \[
    \calM:= \Isom_{\overline{\calX}_{\Iw^+}}(\scrO^g_{\overline{\calX}_{\Iw^+}}, \underline{\omega}_{\Iw^+})
\]be the $\GL_g$-torsor over $\overline{\calX}_{\Iw^+}$ together with the structure morphism $\vartheta: \calM \rightarrow \overline{\calX}_{\Iw^+}$. Then the sheaf $\underline{\omega}_{\Iw^+}^{k}$ of \textit{\textbf{classical Siegel modular forms of weight $k$ (of strict Iwahori level)}} is defined to be 
\[
    \underline{\omega}_{\Iw^+}^k := \vartheta_*\scrO_{\calM}[k^{\vee}];
\] 
namely, the subsheaf of $\vartheta_*\scrO_{\calM}$ on which $T_{\GL_g}$ acts through the character $k^{\vee}$. The \textit{\textbf{space of classical Siegel modular forms of weight $k$ (of strict Iwahori level)}} is defined to be
$$M^{k, \mathrm{cl}}_{\Iw^+}:=H^0(\overline{\calX}_{\Iw^+}, \underline{\omega}_{\Iw^+}^k)$$
equipped with naturally defined Hecke operators.

\begin{Remark}\label{Remark: integral classical sheaf}
\normalfont One can also define the sheaf of integral classical Siegel modular forms by
$$\underline{\omega}_{\Iw^+}^{k,+} := \vartheta_*\scrO^+_{\calM}[k^{\vee}].$$
But we do not need this in the current subsection.
\end{Remark}

Restricting to the $w$-ordinary locus, we may consider the sheaf $\underline{\omega}_{\Iw^+}^{k}|_{\overline{\calX}_{\Iw^+, w}}$. Repeating the strategy as in the proof of Proposition \ref{Proposition: omega is admissible}, we arrive at the following explicit description of $\underline{\omega}_{\Iw^+}^{k}|_{\overline{\calX}_{\Iw^+, w}}$.

\begin{Definition}\label{Definition: algebraic functions}
\begin{enumerate}
\item[(i)] Let $P(\GL_g, \mathbb{A}^1)$ denote the $\Q_p$-vector space of maps $\GL_g\rightarrow \mathbb{A}^1$ between algebraic varieties over $\Q_p$.
\item[(ii)] For every uniform $\C_p$-Banach algebra $B$, define $$P(\GL_g, B):=P(\GL_g, \mathbb{A}^1)\widehat{\otimes}_{\Q_p}B$$
and let $P_k(\GL_g, B)$ denote the subspace of $P(\GL_g, B)$ consisting of those $f:\GL_g \rightarrow B$ such that $f(\bfgamma\bfbeta) =k(\bfbeta)f(\bfgamma)$ for all $\bfgamma\in \GL_{g}$ and  $\bfbeta\in B_{\GL_{g}}$.
\item[(iii)] There is a natural left action of $\GL_g$ on $P_k(\GL_g, B)$ given by
$$(\bfgamma.f)(\bfgamma')=f(\trans \bfgamma\bfgamma')$$
for all $\bfgamma, \bfgamma'\in \GL_g$ and $f\in P_k(\GL_g, B)$. This left action is denoted by $$\rho_k: \GL_g\rightarrow \Aut(P_k(\GL_g, B)).$$
\end{enumerate}
\end{Definition}

\begin{Proposition}\label{Proposition: explicit description of classical modular sheaf}
For any affinoid open $\calV\subset \overline{\calX}_{\Iw^+, w}$ with preimage $\calV_{\infty}$ in $\overline{\calX}_{\Gamma(p^{\infty}), w}$, we have a natural identification
$$
        \underline{\omega}_{\Iw^+}^{k}(\calV) = \left\{f \in P_k(\GL_g, \scrO_{\overline{\calX}_{\Gamma(p^{\infty}), w}}(\calV_{\infty})) :  \bfgamma^* f = \rho_{k}(\bfgamma_a+\frakz\bfgamma_c)^{-1}f, \,\,\,\forall \bfgamma = \begin{pmatrix}\bfgamma_a & \bfgamma_b\\ \bfgamma_c & \bfgamma_d\end{pmatrix} \in \Iw_{\GSp_{2g}}^+ 
        \right\}.
$$
In particular, there is a natural injection
\end{Proposition}
\begin{equation}\label{eq: alg. sheaf into overconvergent sheaf}
    \underline{\omega}_{\Iw^+}^k|_{\overline{\calX}_{\Iw^+, w}} \hookrightarrow \underline{\omega}_w^{k}.
\end{equation} 

\begin{proof}
For the first statement, the strategy in the proof of Proposition \ref{Proposition: omega is admissible} applies verbatim, except that we consider the torsor $\calM$ in place of $\adicIW^+_w$. The details are left to the reader. The inclusion $\underline{\omega}_{\Iw^+}^k|_{\overline{\calX}_{\Iw^+, w}} \hookrightarrow \underline{\omega}_w^{k}$ follows from the natural inclusion $P_k(\GL_g, \scrO_{\overline{\calX}_{\Gamma(p^{\infty}), w}}(\calV_{\infty}))\hookrightarrow C^{w-\an}_k(\Iw_{\GL_g}, \scrO_{\overline{\calX}_{\Gamma(p^{\infty}), w}}(\calV_{\infty}))$.
\end{proof}

The following result shows that the space of classical forms naturally injects into the space of overconvergent modular forms.

\begin{Lemma}\label{Lemma: injection of classical forms}
The Hecke-equivariant composition of maps
$$M^{k, \mathrm{cl}}_{\Iw^+}=H^0(\overline{\calX}_{\Iw^+}, \underline{\omega}_{\Iw^+}^k)\xrightarrow[]{\Res} H^0(\overline{\calX}_{\Iw^+, w}, \underline{\omega}_{\Iw^+}^k) \hookrightarrow M_{\Iw^+, w}^{k}
$$
is injective. 
\end{Lemma}

\begin{proof}
It suffices to show that 
$$\Res: H^0(\overline{\calX}_{\Iw^+}, \underline{\omega}_{\Iw^+}^k)\rightarrow H^0(\overline{\calX}_{\Iw^+, w}, \underline{\omega}_{\Iw^+}^k)$$
is injective; namely, given any global section $f$ of $\underline{\omega}_{\Iw^+}^k$ that vanishes on $\overline{\calX}_{\Iw^+, w}$, we have to show that $f = 0$ on every irreducible component of $\overline{\calX}_{\Iw^+}$.

For every algebraic variety $Y$ over $\C_p$, we know that $Y$ is irreducible if and only if the associated adic space $\calY$ over $\Spa(\C_p, \calO_{\C_p})$ is irreducible (see \cite[Theorem 2.3.1]{conrad-conn} and \cite[\S 1.1.11.(c)]{Huber-2013}). In particular, the irreducible components of $\overline{\calX}_{\Iw^+}$ coincide with the irreducible components of $\overline{X}_{\Iw^+}^{\tor}$. As $\overline{X}_{\Iw^+}^{\tor}$ is a compactification of $X_{\Iw^+}$, its irreducible components correspond to the irreducible components of $X_{\Iw^+}$. Under the identification \[
    X_{\Iw^+}(\C) = \GSp_{2g}(\Q)\backslash \GSp_{2g}(\A_f)\times \bbH_g/\Iw_{\GSp_{2g}}^+\Gamma(N),
\] \cite[\S 2]{Deligne-Shimura} provides the following description of the irreducible components of $\overline{\calX}_{\Iw^+}$:
\[
\pi_0(\overline{\calX}_{\Iw^+}) =  {\Q_{>0}}\backslash \mathbb{G}_m(\A_f)/ \varsigma\left(\Gamma(N) \Iw_{\GSp_{2g}}^+\right).
\]
where $\varsigma$ is the character of similitude involved in the definition of $\GSp_{2g}$. There is a similar description for $\pi_0(\overline{\calX})$. Note that $\pi_0(\overline{\calX}_{\Iw^+})$ is the same as  $\pi_0(\overline{\calX})$ because $\Iw_{\GSp_{2g}}^+$ and $\GSp_{2g}(\Z_p)$ have the same image via $\varsigma$. In particular, since every irreducible component in $\pi_0(\overline{\calX})$ contains an ordinary point, every irreducible component of $\overline{\calX}_{\Iw^+}$ intersects $\overline{\calX}_{\Iw^+, w}$.

By definition, $f$ can be viewed as a global section of the structure sheaf of $\calM$. Let $\calC$ be any irreducible component of $\overline{\calX}_{\Iw^+}$, it remains to show that $f$ vanishes on $\calM \times_{\overline{\calX}_{\Iw^+}}\calC$. Indeed, observe that $\mathcal{M}\times_{\overline{\calX}_{\Iw^+}} {\mathcal{C}} $ is irreducible and $f$ vanishes on $\calM \times_{\overline{\calX}_{\Iw}}(\calC \cap \overline{\calX}_{\Iw^+, w})$. Hence, the desired vanishing follows from \cite[Proposition 0.1.13]{Berthelot-rigid_cohomology} which states that a rigid analytic function vanishing on an open subset of an irreducible rigid analytic variety is identically zero.
\end{proof}


\subsection{The construction \`{a} la Andreatta--Iovita--Pilloni}\label{subsection: construction of AIP}
The sheaves $\underline{\omega}^{\kappa_{\calU}}_w$ constructed in \S \ref{subsection: the perfectoid construction} are analogues of the overconvergent automorphic sheaves constructed by Andreatta-Iovita-Pilloni in \cite{AIP-2015}. It is a natural question whether these two constructions coincide. In this subsection, we recall the construction in \cite{AIP-2015}. Later in \S \ref{subsection:comparison sheaf aip}, we will present a comparison result.

Choose $v\in \Q_{>0}\cap [0,\frac{1}{2})$ and let $n$ be a positive integer such that $v<\frac{1}{2p^{n-1}}$. Consider the open subset $$\overline{\calX}(v):=\{\bfitx\in \overline{\calX}: |\widetilde{\Ha}(\bfitx)|\geq p^{-v}\}\subset \overline{\calX},$$ where $\widetilde{\Ha}$ is a fixed lift of the Hasse invariant. (We point out that, for those $\bfitx$ at the boundary,  the Hasse invariant of $\bfitx$ is defined to be the Hasse invariant of the abelian part of the semiabelian scheme associated with $\bfitx$.) Thanks to \cite[Proposition 4.1.3]{AIP-2015}, for every $1\leq m\leq n$, there is a universal canonical subgroup $\calH_m$ of level $m$ of the tautological semiabelian variety over $\overline{\calX}(v)$. Let $\underline{\omega}_v$ denote the restriction of $\underline{\omega}$ on $\overline{\calX}(v)$.

We also consider the following finite covers of $\overline{\calX}(v)$:
\begin{enumerate}
\item[$\bullet$] Let $$\overline{\calX}_1(p^n)(v):=\Isom_{\overline{\calX}(v)}((\Z/p^n\Z)^g, \calH_n^{\vee})$$ be the adic space over $\overline{\calX}(v)$ which parameterises trivialisations of $\calH_n^{\vee}$. Notice that the group $\GL_g(\Z/p^n\Z)$ naturally acts on $\overline{\calX}_1(p^n)(v)$ from the right by permuting the trivialisations. 
\item[$\bullet$] Let $$\overline{\calX}_1(v):=\Isom_{\overline{\calX}(v)}((\Z/p\Z)^g, \calH_1^{\vee})$$ be the adic space over $\overline{\calX}(v)$ which parameterises trivialisations of $\calH_1^{\vee}$. 
\item[$\bullet$] The group $\GL_g(\Z/p\Z)$ naturally acts on $\overline{\calX}_1(v)$ from the right by permuting the trivialisations. By taking the quotient
$$\overline{\calX}_{\Iw}(v):=\overline{\calX}_1(v)/B_{\GL_g}(\Z/p\Z),$$ we obtain an adic space $\overline{\calX}_{\Iw}(v)$ over $\overline{\calX}(v)$ which parameterises full flags $\Fil_{\bullet}\calH_1^{\vee}$ of $\calH_1^{\vee}$.
\item[$\bullet$] Let $\underline{\omega}_{n,v}$ be the pullback of $\underline{\omega}_v$ along $\overline{\calX}_1(p^n)(v)\rightarrow \overline{\calX}(v)$. 
\item[$\bullet$] Let $\underline{\omega}_{\Iw,v}$ be the pullback of $\underline{\omega}_v$ along $\overline{\calX}_{\Iw}(v)\rightarrow \overline{\calX}(v)$.
\end{enumerate}

In order to proceed, we need to introduce formal models of aforementioned geometric objects: 
\begin{enumerate}
\item[$\bullet$] Recall that $\overline{\frakX}^{\tor}$ is the formal completion of $\overline{X}_0^{\tor}$ along the special fibre. Let $\widetilde{\frakX}^{\tor}(v)$ be the blowup of $\overline{\frakX}^{\tor}$ along the ideal $(\widetilde{\Ha}, p^v)$. Let $\overline{\frakX}^{\tor}(v)$ be the $p$-adic completion of the normalisation of the largest open formal subscheme of $\widetilde{\frakX}^{\tor}(v)$ where the ideal $(\widetilde{\Ha}, p^v)$ is generated by $\widetilde{\Ha}$. Then $\overline{\frakX}^{\tor}(v)$ is a formal model of $\overline{\calX}(v)$. 
\item[$\bullet$] Let $\overline{\frakX}^{\tor}_1(p^n)(v)$ be the normalisation of $\overline{\frakX}^{\tor}(v)$ in $\overline{\calX}_1(p^n)(v)$. The group $\GL_g(\Z/p^n\Z)$ naturally acts on $\overline{\frakX}^{\tor}_1(p^n)(v)$.
\item[$\bullet$] Let $\overline{\frakX}^{\tor}_1(v)$ be the normalisation of $\overline{\frakX}^{\tor}(v)$ in $\overline{\calX}_1(v)$. The group $\GL_g(\Z/p\Z)$ naturally acts on $\overline{\frakX}^{\tor}_1(v)$.
\item[$\bullet$] Let $\overline{\frakX}_{\Iw}^{\tor}(v)$ be the normalisation of $\overline{\frakX}^{\tor}(v)$ in $\overline{\calX}_{\Iw}(v)$. We can identify $\overline{\frakX}_{\Iw}^{\tor}(v)$ with the quotient $\overline{\frakX}_1^{\tor}(v)/B_{\GL_g}(\Z/p\Z)$.
\item[$\bullet$] Let $\frakG_v^{\univ}$ be the tautological semiabelian scheme over $\overline{\frakX}^{\tor}(v)$ with the structure morphism 
$$\pi: \frakG_v^{\univ}\rightarrow \overline{\frakX}^{\tor}(v).$$
Define 
$$\underline{\Omega}_v:=\pi_*\Omega^1_{\frakG_v^{\univ}/\overline{\frakX}^{\tor}(v)}.$$
\item[$\bullet$] Let $\underline{\Omega}_{n,v}$ be the pullback of $\underline{\Omega}_v$ along $\overline{\frakX}^{\tor}_1(p^n)(v)\rightarrow \overline{\frakX}^{\tor}(v)$. 
\item[$\bullet$] Let $\underline{\Omega}_{\Iw,v}$ be the pullback of $\underline{\Omega}_v$ along $\overline{\frakX}^{\tor}_{\Iw}(v)\rightarrow \overline{\frakX}^{\tor}(v)$.
\end{enumerate}

Now suppose $w\in \Q_{>0}$ lies in the interval $\left(n-1+\frac{v}{p-1}, n-\frac{vp^n}{p-1}\right]$. Let $$\psi_n^{\univ}: (\Z/p^n\Z)^g\cong \calH_n^{\vee}$$ denote the universal trivialisation of $\calH_n^{\vee}$ over $\overline{\frakX}^{\tor}_1(p^n)(v)$. Then \cite[Proposition 4.3.1]{AIP-2015} yields a locally free $\scrO_{\overline{\frakX}^{\tor}_1(p^n)(v)}$-submodule $\scrF\subset \underline{\Omega}_{n,v}$ of rank $g$, equipped with a map
$$\HT_{n,v,w}:  (\Z/p^n\Z)^g\overset{\psi_n^{\univ}}{\cong} \calH_n^{\vee}\rightarrow \scrF\otimes_{\scrO_{\overline{\frakX}_1^{\tor}(p^n)(v)}}\scrO_{\overline{\frakX}_1^{\tor}(p^n)(v)}/p^w$$
which induces an isomorphism
$$\HT_{n,v,w}\otimes\id: (\Z/p^n\Z)^g\otimes_{\Z}\scrO_{\overline{\frakX}_1^{\tor}(p^n)(v)}/p^w\cong \scrF\otimes_{\scrO_{\overline{\frakX}_1^{\tor}(p^n)(v)}}\scrO_{\overline{\frakX}_1^{\tor}(p^n)(v)}/p^w.$$
More precisely, locally on $\overline{\frakX}_1^{\tor}(p^n)(v)$, consider the family version of the Hodge-Tate map 
$$\HT_n: (\Z/p^n\Z)^g\overset{\psi_n^{\univ}}{\cong}\calH_n^{\vee}\rightarrow \omega_{\calH_n}$$
studied in \cite[\S 4]{AIP-2015}. Let $\epsilon_1, \ldots, \epsilon_g$ be the standard $(\Z/p^n\Z)$-basis for $(\Z/p^n\Z)^g$ and let $\widetilde{\HT}_n(\epsilon_i)$ be lifts of $\HT_n(\epsilon_i)$ from $\omega_{\calH_n}$ to $\underline{\Omega}_{n,v}$. Then $\scrF$ is generated by $\widetilde{\HT}_n(\epsilon_1), \ldots, \widetilde{\HT}_n(\epsilon_g)$. It turns out this local construction glues to a locally free $\scrO_{\overline{\frakX}^{\tor}_1(p^n)(v)}$-module of rank $g$.

In \cite[\S 4.5]{AIS-2015}, Andreatta--Iovita--Pilloni constructs a formal scheme $\mathfrak{IW}^+_{w,v}$ over $\overline{\frakX}^{\tor}_1(p^n)(v)$ which parameterises certain $w$-compatible objects. More precisely, $\mathfrak{IW}^+_{w,v}$ is the formal schemes over $\overline{\frakX}^{\tor}_1(p^n)(v)$ such that for every affine open subset $\Spf R\subset \overline{\frakX}^{\tor}_1(p^n)(v)$ on which $\scrF$ is free, $\mathfrak{IW}^+_{w,v}(R)$ consists of pairs $(\Fil_{\bullet}, \{w_i: i=1,\ldots, g\})$ where both $\Fil_{\bullet}$ and $\{w_i: i=1, \ldots, g\}$ are $w$-compatible with $\HT_n(\epsilon_1), \ldots, \HT_n(\epsilon_g)$ in the sense of Definition \ref{Definition: w-compatible}.

Now we go back to the generic fibres. Let $\adicIW_{w,v}^+$ be the adic space associated with the formal scheme $\mathfrak{IW}^+_{w,v}$ over $\Spa(\C_p, \calO_{\C_p})$. Then we have a chain of morphisms of adic spaces
$$\pi^{\AIP}: \adicIW_{w,v}^+\rightarrow \overline{\calX}_1(p^n)(v)\rightarrow\overline{\calX}_1(v)\rightarrow \overline{\calX}_{\Iw}(v).$$

Recall the group adic spaces $\calB^{\opp}_w$, $\calT_{\GL_g,0}^{(w)}$, and $\calU^{\opp, (w)}_{\GL_g,1}$ defined in Definition \ref{definition: torsors}.

\begin{Lemma}
\begin{enumerate} 
\item[(i)] $\adicIW_{w,v}^+$ is a $\calB^{\opp}_w$-torsor over $\overline{\calX}_1(p^n)(v)$. Namely, locally on $\overline{\calX}_1(p^n)(v)$, we have identification
$$\adicIW_{w,v}^+\simeq \overline{\calX}_1(p^n)(v)\times_{\Spa(\C_p, \calO_{\C_p})}\calB^{\opp}_w$$
where $\calB^{\opp}_w$ permutes the points $\left(\Fil_{\bullet}, \{w_i\}\right)$ from the right. 

\item[(ii)] Similarly, $\adicIW_{w,v}^+$ is a $\calU^{\opp, (w)}_{\GL_g,1}\times\calT_{\GL_g,0}^{(w)}\times U_{\GL_g}(\Z/p^n\Z)$-torsor over $\overline{\calX}_{\Iw}(v)$. 
\end{enumerate}
\end{Lemma}

\begin{proof}
These are clear from the construction. 
\end{proof}

Finally, we are ready to define the overconvergent automorphic sheaves of Andreatta--Iovita--Pilloni.

\begin{Definition}\label{definition: automorphic sheaves of AIP}
Let $(R_{\calU}, \kappa_{\calU})$ be a $w$-analytic weight. 
\begin{enumerate}
\item[(i)] Andreatta--Iovita--Pilloni's \textbf{sheaf of $w$-analytic $v$-overconvergent Siegel modular forms of weight $\kappa_{\calU}$ (of Iwahori level)} is defined to be \footnote{The notations $\underline{\omega}_{w, v}^{\kappa_{\calU}, \AIP}$, $M^{\kappa_{\calU}, \AIP}_{\Iw, w,v}$, and $M^{\kappa_{\calU}, \AIP}_{\Iw}$ correspond to the notations $\omega^{\dagger\, \kappa_{\calU}}_w$, $M^{\dagger\,\kappa_{\calU}}_w(\calX_{\Iw}(p)(v))$, and $M^{\dagger\,\kappa_{\calU}}(\calX_{\Iw}(p))$, respectively, in \cite{AIP-2015}.}
$$\underline{\omega}_{w, v}^{\kappa_{\calU}, \AIP}:=\pi_{*}^{\AIP}\scrO_{\adicIW_{w,v}^+}[\kappa_{\calU}^{\vee}],$$ 
where $\pi_{*}^{\AIP}\scrO_{\adicIW_{w,v}^+}[\kappa_{\calU}^{\vee}]$ stands for the subsheaf of $\pi_{*}^{\AIP}(\scrO_{\adicIW_{w,v}^+}\widehat{\otimes}R_{\calU})$ consisting of sections on which $T_{\GL_g,0}$ acts via the character $\kappa_{\calU}^{\vee}$ and $U_{\GL_g}(\Z/p^n\Z)$ acts trivially. 
\item[(ii)] Andreatta--Iovita--Pilloni's \textbf{space of $w$-analytic $v$-overconvergent Siegel modular forms of weight $\kappa_{\calU}$ (of Iwahori level)} is 
$$M^{\kappa_{\calU}, \AIP}_{\Iw, w,v}:=H^0(\overline{\calX}_{\Iw}(v), \underline{\omega}_{w, v}^{\kappa_{\calU}, \AIP}).$$
\item[(iii)]The \textbf{space of locally analytic overconvergent Siegel modular forms of weight $\kappa_{\calU}$ (of Iwahori level)} is
$$M^{\kappa_{\calU}, \AIP}_{\Iw}:=\lim_{\substack{v\rightarrow 0\\ w\rightarrow\infty}}M^{\kappa_{\calU}, \AIP}_{\Iw, w, v}.$$
\item[(iv)] Recall that $\calZ_{\Iw} = \overline{\calX}_{\Iw}\smallsetminus \calX_{\Iw}$ is the boundary divisor. Andreatta--Iovita--Pilloni's \textbf{sheaf of $w$-analytic $v$-overconvergent Siegel cuspforms of weight $\kappa_{\calU}$ (of Iwahori level)} is defined to be the subseaf $\underline{\omega}_{w, v, \cusp}^{\kappa_{\calU}, \AIP} = \underline{\omega}_{w, v}^{\kappa_{\calU}, \AIP}(-\calZ_{\Iw})$ of $\underline{\omega}_{w, v}^{\kappa_{\calU}, \AIP}$ consisting of sections that vanish along $\calZ_{\Iw}$. 

Andreatta--Iovita--Pilloni's \textbf{space of $w$-analytic $v$-overconvergent Siegel cuspforms of weight $\kappa_{\calU}$ (of Iwahori level)} is defined to be \[
    S_{\Iw, w, v}^{\kappa_{\calU}, \AIP} := H^0(\overline{\calX}_{\Iw}(v), \underline{\omega}_{w, v, \cusp}^{\kappa_{\calU}, \AIP}),
\] and the \textbf{space of locally analytic overconvergent Siegel cuspforms of weight $\kappa_{\calU}$ (of Iwahori level)} is \[
    S_{\Iw}^{\kappa_{\calU}, \AIP} := \lim_{\substack{v\rightarrow 0 \\ w \rightarrow \infty}} S_{\Iw, w, v}^{\kappa_{\calU}, \AIP}.
\]
\end{enumerate}
\end{Definition}

\begin{Remark}\label{remark: no difference 2}
\normalfont Similar to Remark \ref{remark: no difference}, in Definition \ref{definition: automorphic sheaves of AIP} (i), there is no difference between taking $\kappa_{\calU}^{\vee}$-eigenspaces with respect to $T_{\GL_g,0}$- or $\calT^{(w)}_{\GL_g,0}$-actions.
\end{Remark}


\subsection{The \texorpdfstring{$w$}{w}-ordinary loci and the (pseudo-)canonical subgroups}\label{subsection: pseudocanonical subgroups}
In \S \ref{subsection:comparison sheaf aip}, we will prove the comparison between our perfectoid construction of the overconvergent Siegel modular forms and the construction of Andreatta--Iovita--Piloni. Immediate from the definitions, one observes the incompatibility of the underlying adic spaces used in the two constructions. That is, we employ the $w$-ordinary locus in the perfectoid construction while the authors of \cite{AIP-2015} make use of the ``$v$-locus'' $\overline{\calX}_{\Iw}(v)$. Therefore, as a preparation for the comparison result, we have to first compare these two different loci. Due to technical reasons, we assume $p>2g$ in this subsection. 

We start with the definition of the $v$-locus at the strict Iwahori level. Recall from \S \ref{subsection: construction of AIP} that, for any $v\in \Q_{>0}\cap [0,\frac{1}{2})$, $\overline{\calX}_1(p^n)(v)$ (resp., $\overline{\calX}_1(v)$; resp., $\overline{\calX}_{\Iw}(v)$) is the adic space over $\overline{\calX}(v)$ which parameterises trivialisations of $\calH_n^{\vee}$ (resp., trivialisations of $\calH_1^{\vee}$; resp., full flags of $\calH_1^{\vee}$). In particular, $\overline{\calX}_1(v)$ is equipped with a natural right action of $\GL_g(\Z/p\Z)$ permuting the trivialisations. Consider the quotient
$$\overline{\calX}_{\Iw^+}(v):=\overline{\calX}_1(v)/T_{\GL_g}(\Z/p\Z)$$
which is an adic space over $\overline{\calX}(v)$ parametersing the ``strict Iwahori structures'' of $\calH_1^{\vee}$; namely, it parameterises full flags $\Fil_{\bullet}\calH_1^{\vee}$ of $\calH_1^{\vee}$ together with a collection of subgroups $\{D_i: i=1, \ldots, g\}$ of $\calH_1^{\vee}$ of order $p$ such that 
$$\Fil_i\calH_1^{\vee}=\langle D_1, \ldots, D_i\rangle$$
for all $i=1, \ldots, g$. There is a chain of natural projections among these $v$-loci
$$\overline{\calX}_1(p^n)(v)\rightarrow\overline{\calX}_1(v)\rightarrow\overline{\calX}_{\Iw^+}(v)\rightarrow\overline{\calX}_{\Iw}(v)\rightarrow\overline{\calX}(v).$$ One can identify $\overline{\calX}_{\Iw}(v)$ as the quotient of $\overline{\calX}_{\Iw^+}(v)$ by the finite group $U_{\GL_g}(\Z/p\Z)$.

The main result of this subsection is the following:

\begin{Theorem}\label{Theorem: cofinal system of w- and v-loci}
Given $\Gamma \in \{\Iw^+, \Iw\}$, the system of $w$-ordinary loci $\{\overline{\calX}_{\Gamma, w} : w\in \Q_{>0}\}$ and the system of $v$-loci $\{\overline{\calX}_{\Gamma}(v): v\in \Q_{>0}\cap [0, 1/2)\}$ are mutually cofinal. More precisely, \begin{enumerate}
    \item[(i)] For any given $v\in \Q_{>0}\cap [0, 1/2)$, there exists sufficiently large $w\in \Q_{>0}$ such that $\overline{\calX}_{\Gamma, w}\subset \overline{\calX}_{\Gamma}(v)$.
    \item[(ii)] For any given $w\in \Q_{>0}$, there exists sufficiently small $v\in \Q_{>0}\cap [0, 1/2)$ such that $\overline{\calX}_{\Gamma}(v) \subset \overline{\calX}_{\Gamma, w}$.
\end{enumerate}
\end{Theorem}

To prove Theorem \ref{Theorem: cofinal system of w- and v-loci}, we follow the strategy in \cite[\S 2.3]{CHJ-2017}. However, we have to generalise their study of pseudocanonical subgroups to the case of semiabelian schemes with constant toric rank.

Let $C$ be an algebraically closed complete nonarchimedean field containing $\Q_p$ and let $\calO_C$ be its ring of integers. Suppose the valuation $v_p$ on $C$ is normalised so that $v_p(p)=1$. Let $G$ be a semiabelian scheme over $\calO_C$ of dimension $g$ with constant toric rank $r \leq g$. That is, $G$ sits inside an extension \[
    0 \rightarrow T \rightarrow G \rightarrow A \rightarrow 0,
\] where $T$ is a torus of rank $r$ over $\calO_C$ and $A$ is an abelian scheme of dimension $g-r$ over $\calO_C$. (We say that $G$ is \textit{\textbf{principally polarised}} if $A$ is principally polarised.) One sees that the $p$-adic Tate module $T_p G := \varprojlim_{n} G[p^n](C)$ is isomorphic to $\Z_p^{2g-r}$. 

Recall the Hodge--Tate complex over $\calO_C$ \[
    0 \rightarrow \Lie G \rightarrow T_p G\otimes_{\Z_p}\calO_C \rightarrow \omega_{G^{\vee}} \rightarrow 0,
\]  where ${\omega}_{G^{\vee}}$ is the dual of the Lie algebra $\Lie G^{\vee}$ of the dual semiabelian scheme $G^{\vee}$, and the second last map is induced from the Hodge--Tate map $\HT_G:T_pG\rightarrow \omega_{G^{\vee}}$. By \cite[Th\'eor\`eme II. 1.1]{Fargues-Genestier-Lafforgue}, the cohomology of this complex is killed by $p^{1/(p-1)}$.  

\begin{Definition}\label{Definition: subspace of Vp}
Recall that $V_p=V\otimes_{\Z} \Z_p\simeq \Z_p^{2g}$ is equipped with the standard basis $e_1, \ldots, e_{2g}$ together with a symplectic pairing. For every $0\leq r \leq g$, let $V_{p,r}$ denote the $\Z_p$-submodule of $V_p$ spanned by $e_{r+1}, e_{r+2}, \ldots, e_{2g-r}$, equipped with the induced symplectic pairing. We also write $V'_{p,r}$ to be $\Z_p$-submodule of $V_p$ spanned by $e_1, \ldots, e_{2g-r}$ and write $W_{p,r}$ to be the one spanned by $e_1, \ldots, e_r$. There is an obvious split exact sequence
$$0\rightarrow W_{p,r}\rightarrow V'_{p,r}\rightarrow V_{p,r}\rightarrow 0.$$
\end{Definition}

\begin{Definition}\label{Definition: w-ordinary semiabelian scheme with constant toric rank}
Let $G$ be a principally polarised semiabelian scheme over $\calO_C$ of dimension $g$ with constant toric rank $r \leq g$. \begin{enumerate}
    \item[(i)] An isomorphism $\alpha: V'_{p,r} \xrightarrow[]{\sim} T_p G$ is called a \textbf{trivialisation} of $T_p G$ if it is part of a commutative diagram \[
        \begin{tikzcd}
            V_{p,r}\arrow[r,"\sim"] & T_pA\\
            V'_{p,r} \arrow[r,"\sim"]\arrow[u, two heads] & T_pG\arrow[u, two heads]\\
            W_{p,r} \arrow[u, hook]\arrow[r,"\sim"] & T_p T\arrow[u, hook]
        \end{tikzcd}
        \] where \begin{itemize}
            \item the vertical arrows on the left are the ones as in Definition \ref{Definition: subspace of Vp};
            \item the vertical arrows on the right are induced from the exact sequence $0\rightarrow T\rightarrow G\rightarrow A\rightarrow 0$;
            \item the top arrow preserves the symplectic pairings.
        \end{itemize}
    \item[(ii)] A trivialisation $\alpha: V'_{p,r} \rightarrow T_p G$ is \textbf{$w$-ordinary} if $\HT_G(\alpha(e_i))\in p^w {\omega}_{G^{\vee}}$ for all $i=1, ..., g$. 
    \item[(iii)] We say that $G$ is \textbf{$w$-ordinary} if it admits a $w$-ordinary trivialisation. 
\end{enumerate}
\end{Definition}

\begin{Remark}\label{Remark: w-ordinary semiabelian schemes}
\normalfont From the definition, if $G$ is $w$-ordinary, it is $w'$-ordinary for any $w'>w$. It is also clear that $G$ is ordinary if and only if it is $w$-ordinary for all $w\in \Q_{>0}$.  
\end{Remark}

\begin{Lemma}\label{Lemma: pseudocanonical subgroup}
Let $G$ be a $w$-ordinary semiabelian scheme (of dimension $g$ with constant toric rank $r$) over $\calO_C$ and let $n\in \Z_{\geq 1}$ such that $n < w+1$. The Hodge-Tate map $\HT_G$ induces a map
$$G[p^n](C) \rightarrow (\image\HT_{G})/p^{\min\{n, w\}} (\image \HT_{G}).
$$ Then the schematic closure of the kernel of this map defines a flat subgroup scheme $H_n \subset G[p^n]$ whose generic fibre is isomorphic to $(\Z/p^n\Z)^{g}$. Moreover, if $\alpha$ is a $w$-ordinary trivialisation of $T_p G$, then $H_n(C)$ is generated by $\alpha(e_1)$, ..., $\alpha(e_g)$. Here we have abused the notations and still use $\alpha(e_i)$'s to denote their images in $G[p^n](C)$.
\end{Lemma}

\begin{proof}
Since the Hodge--Tate complex is exact after inverting $p$, the image of $\Lie G$ in $T_p G\otimes_{\Z_p}\calO_C$ is a rank $g$ sub-lattice in the kernel of $T_pG\otimes_{\Z_p}\calO_C\rightarrow \omega_{G^{\vee}}$. Hence, the kernel of $\HT_{G}: T_p G \rightarrow {\omega}_{G^{\vee}}$ has rank at most $g$.

On the other hand, there is a commutative diagram \[
    \begin{tikzcd}
        T_p G\arrow[d]\arrow[r, "\HT_{G}"] &  {\omega}_{G^{\vee}}\arrow[d]\\
        G[p^n](C)\arrow[r, "\HT_{G[p^n]}"] & {\omega}_{G[p^n]^{\vee}}
    \end{tikzcd},
\] where the right vertical arrow is induced from the natural identification ${\omega}_{G[p^n]^{\vee}} = {\omega}_{G^{\vee}}/p^n{\omega}_{G^{\vee}}$. Consequently, $\ker\HT_{G[p^n]}$ also has rank at most $g$.

Let $\alpha$ be a $w$-ordinary trivialisation of $T_p G$. Since $n<w+1$, the kernel of the composition \[
    T_p G \xrightarrow{\HT_{G}} {\omega}_{G^{\vee}} \rightarrow {\omega}_{G^{\vee}}/p^n{\omega}_{G^{\vee}}
\] necessarily contains $\alpha(e_i)$, for all $i=1, ..., g$. Since $\alpha(e_i)$'s are $\Z_p$-linearly independent, their images in $G[p^n](C)$ are $(\Z/p^n\Z)$-linearly independent and hence generate $\ker\HT_{G[p^n]}$. Consequently, $H_n$ is precisely the schematic closure in $G[p^n]$ of the subgroup of $G[p^n](C)$ generated by $\{\alpha(e_i): i=1, ..., g\}$. Flatness of $H_n$ follows from the flatness of $G$.
\end{proof}

\begin{Definition}
The subgroup scheme $H_n$ defined in Lemma \ref{Lemma: pseudocanonical subgroup} is called the \textbf{pseudocanonical subgroup of level $n$}. When $n= 1$, we simply call $H_1$ the \textbf{pseudocanonical subgroup} of $G$.
\end{Definition}

\begin{Lemma}\label{Lemma: pseudocanonical subgroups behave like canonical subgroups}
Let $m\leq n$ be positive integers and let $w\in \Q_{>0}$ such that $w>n$. Let $G$ be a semiabelian scheme (of dimension $g$ with constant toric rank $r$) over $\calO_C$. Suppose $G$ is $w$-ordinary. Then, $G/H_{m}$ is $(w-m)$-ordinary, and for any $m'\in \Z$ with $m < m' \leq  n$, we have $H'_{m'-m} = H_{m'}/H_{m}$, where $H'_{m'-m}$ is the pseudocanonical subgroup of $G/H_{m}$ of level $m'-m$.
\end{Lemma}

\begin{proof}
The proof is the same as in \cite[Lemma 2.11]{CHJ-2017} as long as we use the matrix $\diag(p^m\one_{g}, \one_{g-r})$ in place of $\diag(1, p^m)$. Notice that the ``$p^m$'' factor appears at the bottom right corner in \textit{loc. cit.} because they work with a slightly different action of $\GL_2(\Q_p)$. 
\end{proof}

Before stating the next lemma, let us recall the notion of the \emph{degree} of a finite flat group scheme over $\calO_C$ studied in \cite{Fargues-canonical}. If $M$ is a $p$-power torsion $\calO_C$-module of finite presentation, we can write $$M\simeq \bigoplus_{i=1}^l \calO_C/a_i\calO_C$$
for some $a_i\in \calO_C$, $i=1,\ldots, l$. Then the degree of $M$ is defined to be $\deg M:= \sum_{i=1}^{l}v_p(a_i)$. Now, if $H$ is a finite flat group scheme over $\calO_C$ and let $\omega_H$ denote the $\calO_C$-module of invariant differentials on $H$, then we define the \textbf{\textit{degree}} of $H$ to be $\deg H:=\deg \omega_H$.

\begin{Lemma}\label{Lemma: Oort--Tate theory}
Let $G$ be a $w$-ordinary semiabelian scheme (of dimension $g$ with constant toric rank $r$) over $\calO_C$ and let $\alpha$ be a $w$-ordinary trivialisation. Let ${\omega}_{H_1}$ be the dual of $\Lie H_1$ and let $\omega_{H_1^{\vee}}$ be the dual of $\Lie H_1^{\vee}$. For $i=1, ..., g$, let $H_{1, i}$ be the schematic closure in $H_1$ of the subgroup generated by $\alpha(e_i)$. Then \begin{enumerate}
    \item[(i)] Each $H_{1, i}$ is isomorphic to $\Spec(\calO_C[X]/(X^p - a_{i}X))$ for some $a_i\in \calO_C$. The dual $H_{1, i}^{\vee}$ is isomorphic to $\Spec(\calO_C[X]/(X^p-b_iX))$ with $a_ib_i = p$. 
    \item[(ii)] We have isomorphisms ${\omega}_{H_1} \simeq \bigoplus_{i=1}^g \calO_C/a_i\calO_C$ and 
${\omega}_{H_1^{\vee}} \simeq \bigoplus_{i=1}^g \calO_C/b_i\calO_C$. In particular, we have $\deg H_1 = \sum_{i=1}^{g}v_p(a_i)$ and $\deg H_1^{\vee}=\sum_{i=1}^{g}v_p(b_i)=g-\sum_{i=1}^{g}v_p(a_i)$.
    \item[(iii)] Under the identification ${\omega}_{H_1^{\vee}} \simeq \bigoplus_{i=1}^g \calO_C/b_i\calO_C$, the image of the (linearised) Hodge--Tate map \[
        H_1(C) \otimes_{\Z_p} \calO_C \rightarrow {\omega}_{H_1^{\vee}}
    \] is equal to $\bigoplus_{i=1}^g c_i\calO_C/b_i\calO_C$ for some $c_i\in \calO_C$ such that $v_p(c_i) = v_p(a_i)/(p-1)$, $i=1, \ldots, g$.
\end{enumerate}
\end{Lemma}
\begin{proof}
Since each $H_{1, i}$ is a finite flat group scheme over $\calO_{C}$ of degree $p$, the assertion follows from classical Oort--Tate theory. See, for example, \cite[\S 6.5, Lemme 9]{Fargues-canonical}. 
\end{proof}

Recall from \cite[\S 3.1]{AIP-2015} that the \emph{Hodge height} of $G$ is defined to be the ``truncated'' $p$-adic valuation of the Hasse invariant of $G$. See \emph{loc. cit.} for details.

\begin{Lemma}\label{Lemma: pseudocanonical = canonical}
Let $G$ be a $w$-ordinary semiabelian scheme (of dimension $g$ with constant toric rank $r$) over $\calO_C$. Suppose $\frac{(2g-1)p}{2g(p-1)} < w \leq 1$. \footnote{The inequalities are valid because of the assumption $p>2g$ at the beginning of the subsection.} Then $H_1$ coincides with the canonical subgroup of $G$. Moreover, the Hodge height of $G$ is smaller than $1/2$. 
\end{Lemma}

\begin{proof}
We follow the strategy of the proof of \cite[Lemma 2.14]{CHJ-2017}. Consider the commutative diagram \[
    \begin{tikzcd}
        0 \arrow[r] & H_1(C) \arrow[r]\arrow[d, "\HT_{H_1}"] & G[p](C)\arrow[d, "\HT_{G[p]}"]\\
        0 \arrow[r] & {\omega}_{H_1^{\vee}} \arrow[r] & {\omega}_{G[p]^{\vee}}
    \end{tikzcd}
\]  with exact rows. Notice that we have an identification ${\omega}_{G[p]^{\vee}} = {\omega}_{G^{\vee}}/p{\omega}_{G^{\vee}}$. Let $\alpha$ be a $w$-ordinary trivialisation of $T_pG$. According to Lemma \ref{Lemma: pseudocanonical subgroup}, $\alpha(e_1), \ldots, \alpha(e_g)$ form a basis for $H_1(C)$. Also, by definition, we have $\HT_{G[p]}(\alpha(e_i))\in p^w {\omega}_{G[p]^{\vee}}$. 

Now, with respect to the generators $\alpha(e_1) \ldots, \alpha(e_g)$ of $H_1(C)$, the map $\omega_{H_1^{\vee}} \rightarrow \omega_{G[p]^{\vee}}$ can be identified with the inclusion \[
    \bigoplus_{i=1}^g \calO_C/b_i\calO_C \rightarrow (\calO_C/p\calO_C)^g, \quad (x_1, ..., x_g)\mapsto (a_1x_1, ..., a_gx_g).
\] 
Therefore, we see that \[
    a_i\HT_{H_1}(\alpha(e_i)) = \HT_{G[p]}(\alpha(e_i)) \in p^w {\omega}_{G[p]^{\vee}}.
\]  
By Lemma \ref{Lemma: Oort--Tate theory} (iii), we know that $\HT_{H_1}(\alpha(e_i))$ has valuation $v_p(a_i)/(p-1)$. This implies \[
    w \leq v_p(a_i) + \frac{v_p(a_i)}{p-1} = \frac{p v_p(a_i)}{p-1}. 
\] Consequently, we have \[
    \deg H_1 = \sum_{i=1}^g v_p(a_i) \geq \frac{gw(p-1)}{p} > \frac{g(p-1)}{p}\cdot \frac{(2g-1)p}{2g(p-1)} = \frac{2g-1}{2} = g - \frac{1}{2}.
\] It follows from \cite[Proposition 3.1.2]{AIP-2015} that $H_1$ is exactly the canonical subgroup of $G$ and the Hodge height of $G$ is less than $\frac{1}{2}$.
\end{proof}

\begin{Remark}
\normalfont The lemma might hold without the assumption $p>2g$ as long as one can produce finer estimates on the degree and the Hodge height. However, we do not attempt to find these better estimates.
\end{Remark}

\begin{Proposition}\label{Proposition: pseudocanonical = canonical}
Let $G$ be a $w$-ordinary semiabelian scheme (of dimension $g$ with constant toric rank $r$) over $\calO_C$. Suppose $\frac{(2g-1)p}{2g(p-1)}+n-1 < w \leq n$, then $H_{n}$ coincides with the canonical subgroup of $G$ of level $n$. In this case, the Hodge height of $G$ is less than $\frac{1}{2p^{n-1}}$.
\end{Proposition}
\begin{proof}
The proof follows from induction. The case for $n=1$ is precisely Lemma \ref{Lemma: pseudocanonical = canonical}. 

Assume that the statement is affirmative for $n-1$. By Lemma \ref{Lemma: pseudocanonical subgroups behave like canonical subgroups}, $G/H_1$ is $(w-1)$-ordinary and we have $\frac{(2g-1)p}{2g(p-1)}+n-2 < w-1 \leq n-1$. The induction hypothesis implies that the pseudocanonical subgroup $H_{n}/H_1$ of of level $n-1$ of $G/H_1$ is the canonical subgroup of level $n-1$ and that the Hodge height of $G/H_1$ is less than $\frac{1}{2p^{n-2}}$.

However, $H_1$ coincides with the canonical subgroup of $G$ by Lemma \ref{Lemma: pseudocanonical = canonical}. Hence, by \cite[Th\'eor\`em 6 (4)]{Fargues-canonical} (see also \cite[Theorem 3.1.1 (5)]{AIP-2015}), we see that the Hodge height of $G$ is bounded by $\frac{1}{2p^{n-1}}$ and that $H_n$ is the canonical subgroup of level $n$ of $G$.
\end{proof}

\begin{Corollary}\label{Corollary: w-locus injects into v-locus}
Let $n\in \Z_{\geq 1}$ and suppose $w\in \Q_{>0}$ such that $\frac{(2g-1)p}{2g(p-1)}<w\leq n$. Then there exists $v\in \Q_{>0}\cap [0, \frac{1}{2p^{n-1}})$ and a natural inclusion $\overline{\calX}_w\hookrightarrow \overline{\calX}(v)$.
\end{Corollary}

\begin{proof}
It suffices to work with $(C, \calO_C)$-points for algebraically closed complete nonarchimedean field $C$ containing $\Q_p$. (Notice that the classical points determine these adic spaces by \cite[(1.1.11)]{Huber-2013}). Let $\bfitx\in \overline{\calX}_w(C, \calO_C)$. By the properness of $\overline{\calX}$, the point $\bfitx$ extends to an $\calO_C$-point $\tilde{\bfitx}$ of $\overline{\frakX}^{\tor}$. One can associate with $\tilde{\bfitx}$ a 1-motive $\widetilde{M}_{\tilde{\bfitx}}=[Y\rightarrow \widetilde{G}_{\tilde{\bfitx}}]$ where $\widetilde{G}_{\tilde{\bfitx}}$ is a semiabelian scheme (of dimension $g$ with constant toric rank) over $\calO_C$ and $Y$ is a free $\Z$-module of finite rank (see, for example, \cite{Stroh-TorComp}). 

From the definition of the Hodge--Tate period map (see \S \ref{subsection: perfectoid Siegel modular variety} for a quick review), we see that $\widetilde{G}_{\tilde{\bfitx}}$ is $w$-ordinary. By Proposition \ref{Proposition: pseudocanonical = canonical}, the Hodge height of $\widetilde{G}_{\tilde{\bfitx}}$ is smaller than $\frac{1}{2p^{n-1}}$. This means $\bfitx\in \overline{\calX}(v)(C, \calO_C)$ for some $v<\frac{1}{2p^{n-1}}$ and so we are done. %Finally, by the compactness of $\overline{\calX}_w$, we know that $\overline{\calX}_w\hookrightarrow \overline{\calX}(v)$ for some $v<\frac{1}{2p^{n-1}}$.
\end{proof}

Recall that, for any $v\in \Q_{>0}\cap [0, \frac{1}{2})$, $\calH_1$ is the universal canonical subgroup of the tautological semiabelian variety over $\overline{\calX}(v)$. Let $w>\frac{(2g-1)p}{2g(p-1)}$ and pick $v$ so that $\overline{\calX}_w\hookrightarrow \overline{\calX}(v)$ as in Corollary \ref{Corollary: w-locus injects into v-locus}. We still write $\calH_1$ for its pullback to $\overline{\calX}_w$.

In this case, consider
$$
   \overline{\calX}_{1, w} := \Isom_{\overline{\calX}_w}((\Z/p\Z)^g, \calH_1^{\vee});
$$
namely, the adic space over $\overline{\calX}_w$ which parameterises trivialisations of $\calH_1^{\vee}$. The group $\GL_g(\Z/p\Z)$ naturally acts on $\overline{\calX}_{1, w}$ by permuting the trivialisations.

\begin{Lemma}\label{Lemma: alternative definition of w-locus of (strict) Iwahori level variety}
For $w>\frac{(2g-1)p}{2g(p-1)}$, there are natural identifications
\[
    \overline{\calX}_{1, w}/B_{\GL_g}(\Z/p\Z) = \overline{\calX}_{\Iw, w}\quad \text{ and }\quad \overline{\calX}_{1, w}/T_{\GL_g}(\Z/p\Z) = \overline{\calX}_{\Iw^+, w}.
\]
\end{Lemma}
\begin{proof}
We only give the proof for the first identity. The second one is similar and left to the readers.

We first focus on the part away from the boundary. Let $\calX_{w} = \overline{\calX}_w \cap \calX$ and let $\calA_{w}^{\univ}$ be the universal abelian variety over $\calX_{w}$. 

The key observation is that any trivialisation $\psi:(\Z/p\Z)^g \rightarrow \calH_{1}^{\vee}$ induces a full flag $\Fil_{\bullet}^{\psi} \calA^{\univ}_{w}[p]$ on $\calA^{\univ}_{w}[p]$. Indeed, let $\epsilon_1, \ldots, \epsilon_g$ denote the standard basis for $(\Z/p\Z)^g$ and let $\Fil_{\bullet}^{\psi}\calH_1^{\vee}$ be the full flag of $\calH_1^{\vee}$ given by \[
    0\subset \langle \psi(\epsilon_1) \rangle \subset \langle \psi(\epsilon_1), \psi(\epsilon_2) \rangle \subset \cdots \subset \langle \psi(\epsilon_1), ..., \psi(\epsilon_g)\rangle.
\] 
Consider the natural projection \[
    \pr: \calA_w^{\univ}[p] \xrightarrow{\sim} \calA_w^{\univ}[p]^{\vee} \twoheadrightarrow \calH_1^{\vee}
\] where the first isomorphism is induced from the principal polarisation. Then the desired full flag $\Fil_{\bullet}^{\psi}\calA_w^{\univ}[p]$ is given by \[
    \Fil_{i}^{\psi}\calA_w^{\univ}[p] := \left\{\begin{array}{ll}
        \pr^{-1}\Fil_{i-g}^{\psi}\calH_1^{\vee}, & i > g \\
        (\pr^{-1}\Fil_{g-i}^{\psi}\calH_{1}^{\vee})^{\perp}, & i \leq g 
    \end{array}\right..
\] 
Moreover, if two such $\psi$'s induce the same $\Fil_{\bullet}^{\psi}\calH_1^{\vee}$, then the associated $\Fil_{\bullet}^{\psi}\calA_w^{\univ}[p]$ coincide. Hence, the assignment $\psi\mapsto \Fil_{\bullet}^{\psi} \calA^{\univ}_{w}[p]$ induces a natural inclusion $\calX_{1, w}/B_{\GL_g}(\Z/p\Z) \subset \calX_{\Iw, w}$ away from the boundary.

Conversely, using the $w$-ordinarity, one sees that the universal full flag $\Fil_{\bullet}\calA_w^{\univ}[p]$ on $\calX_{\Iw, w}$ induces a full flag $\Fil_{\bullet}\calH_1^{\vee}$ of $\calH_1^{\vee}$ given by 
\[
    \Fil_{i}\calH_{1}^{\vee} = \pr\left((\Fil_{g-i}\calA_{w}^{\univ}[p])^{\perp}\right)
\] for $i=1, ..., g$.
This yields the opposite inclusion away from the boundary. 

In order to extend to the boundary, one considers the 1-motives on the boundary strata and same argument as above applies verbatim. The details are left to the reader.
\end{proof}

Finally, we prove Theorem \ref{Theorem: cofinal system of w- and v-loci}.

\begin{proof}[Proof of Theorem \ref{Theorem: cofinal system of w- and v-loci}] 
\begin{enumerate}
    \item[(i)] We may assume $v=\frac{1}{2p^{n-1}}$ for some sufficiently large $n$. In this case, we can take any $\frac{(2g-1)p}{2g(p-1)}+n-1 < w \leq n$. Indeed, by Corollary \ref{Corollary: w-locus injects into v-locus}, we have a Cartesian diagram
     \[
        \begin{tikzcd}
            \overline{\calX}_{1, w}\arrow[r, hook]\arrow[d] & \overline{\calX}_1(v)\arrow[d]\\
        \overline{\calX}_{w}\arrow[r, hook] & \overline{\calX}(v)
        \end{tikzcd}
    \]
    where the top arrow is equivariant under the action of $\GL_g(\Z/p\Z)$. Taking the quotient by either $B_{\GL_g}(\Z/p\Z)$ or $T_{\GL_g}(\Z/p\Z)$, and applying Lemma \ref{Lemma: alternative definition of w-locus of (strict) Iwahori level variety}, we obtain the desired inclusions.
        
    \item[(ii)] We may assume $n-1<w<n$ for some sufficiently large $n$. Pick $v\in \Q_{>0}\cap [0, \frac{1}{2p^{n-1}})$ such that $w\in \left(n-1+\frac{v}{p-1}, n-\frac{vp^n}{p-1}\right]$. Applying \cite[Proposition 3.2.1]{AIP-2015}, on the level of classical points, we obtain a natural inclusion $\overline{\calX}(v)(C, \calO_C) \hookrightarrow \overline{\calX}_w(C,\calO_C)$ and hence an inclusion $\overline{\calX}(v)\hookrightarrow \overline{\calX}_w$. There is a Cartesian diagram
\[
        \begin{tikzcd}
            \overline{\calX}_1(v)\arrow[r, hook]\arrow[d] & \overline{\calX}_{1, w}\arrow[d]\\
            \overline{\calX}(v)\arrow[r, hook] & \overline{\calX}_w
        \end{tikzcd}
    \] 
    Once again, applying Lemma \ref{Lemma: alternative definition of w-locus of (strict) Iwahori level variety} and taking the corresponding quotients yield the desired inclusions.
\end{enumerate}
\end{proof}


\subsection{Comparison of the two constructions}\label{subsection:comparison sheaf aip} 
In this section, we still assume $p>2g$. The aim of this subsection is to prove the following theorem which compares the overconvergent automorphic sheaf $\underline{\omega}_w^{\kappa_{\calU}}$ constructed in \S \ref{subsection: the perfectoid construction} and the sheaf $\underline{\omega}_{w, v}^{\kappa_{\calU}, \AIP}$ of Andreatta--Iovita--Pilloni.

For any $v\in \Q_{>0}\cap [0, \frac{1}{2})$, let $h_{\diamond}:\overline{\calX}_{\Iw^+}(v)\rightarrow \overline{\calX}_{\Iw}(v)$ denote the natural projection.

\begin{Theorem}\label{Theorem: comparison with AIP}
Suppose $n>\frac{g}{p-1}$ and let $v\in \Q_{>0}\cap [0, \frac{1}{2p^{n-1}})$, $w\in\Q_{>0}\cap (n-1+\frac{v}{p-1}, n-\frac{vp^n}{p-1}]$. (In particular, by the proof of Theorem \ref{Theorem: cofinal system of w- and v-loci} (ii), there are natural inclusions $\overline{\calX}_{\Iw}(v)\hookrightarrow \overline{\calX}_{\Iw, w}$ and $\overline{\calX}_{\Iw^+}(v) \hookrightarrow \overline{\calX}_{\Iw^+, w}$.) Let $(R_{\calU}, \kappa_{\calU})$ be a weight such that $w>1+r_{\calU}$. Then, over $\overline{\calX}_{\Iw^+}(v)$, there is a canonical isomorphism of sheaves $$\Psi:\underline{\omega}_w^{\kappa_{\calU}}|_{\overline{\calX}_{\Iw^+}(v)}\xrightarrow{\sim} h_{\diamond}^*\underline{\omega}_{w, v}^{\kappa_{\calU}, \AIP}$$
\end{Theorem}

Recall that the space of overconvergent Siegel modular forms of weight $\kappa_{\calU}$ of strict Iwahori level (see Definition \ref{Definition: the sheaf of overconvergent Siegel forms} (v)) is defined to be
$$M^{\kappa_{\calU}}_{\Iw^+}=\varinjlim_{w\rightarrow \infty} M^{\kappa_{\calU}}_{\Iw^+,w}$$
where $$M^{\kappa_{\calU}}_{\Iw^+,w}=H^0(\overline{\calX}_{\Iw^+, w}, \,\,\underline{\omega}_w^{\kappa_{\calU}}).$$
We can also extend the notion of overconvergent Siegel modular forms of Andreatta--Iovita--Pilloni to the case of strict Iwahori level.

\begin{Definition}\label{Definition: AIP's Siegel modular forms for strict Iwahori level}
Let $(R_{\calU}, \kappa_{\calU})$ be a weight.
\begin{enumerate}
\item[(i)] Let $v\in \Q_{>0}\cap [0, 1/2)$ and $w\in \Q_{>0}$. Suppose $\kappa_{\calU}$ is $w$-analytic. The \textbf{space of $w$-analytic $v$-overconvergent Siegel modular forms of weight $\kappa_{\calU}$ (of strict Iwahori level)} of Andreatta--Iovita--Pilloni is defined to be
$$M^{\kappa_{\calU}, \AIP}_{\Iw^+,w,v}:=H^0(\overline{\calX}_{\Iw^+}(v), h_{\diamond}^*\underline{\omega}_{w, v}^{\kappa_{\calU}, \AIP}).$$
\item[(ii)] The \textbf{space of locally analytic overconvergent Siegel modular forms of weight $\kappa_{\calU}$ (of strict Iwahori level)} of Andreatta--Iovita--Pilloni is defnied to be
$$M^{\kappa_{\calU}, \AIP}_{\Iw^+}:=\lim_{\substack{v\rightarrow 0\\ w\rightarrow\infty}}M^{\kappa_{\calU}, \AIP}_{\Iw^+,w,v}.$$
\item[(iii)] Similarly, the \textbf{space of $w$-analytic $v$-overconvergent Siegel cuspforms of weight $\kappa_{\calU}$ (of strict Iwahori level)} of Andreatta--Iovita--Pilloni is defined to be \[
    S_{\Iw^+, w, v}^{\kappa_{\calU}, \AIP} := H^0(\overline{\calX}_{\Iw^+}(v), h_{\diamond}^* \underline{\omega}_{w, v, \cusp}^{\kappa_{\calU}, \AIP}),
\] and the \textbf{space of locally analytic overconvergent Siegel cuspforms of weight $\kappa_{\calU}$ (of strict Iwahori level)} of Andreatta--Iovita--Pilloni is defined to be \[
    S_{\Iw^+}^{\kappa_{\calU}, \AIP} := \lim_{\substack{v\rightarrow 0\\ w\rightarrow\infty}}S^{\kappa_{\calU}, \AIP}_{\Iw^+,w,v}.
\]
\end{enumerate}
\end{Definition}

Then we have the following immediate corollary of Theorem \ref{Theorem: comparison with AIP} and Theorem \ref{Theorem: cofinal system of w- and v-loci}.

\begin{Corollary}\label{Corollary: comparison with AIP}
There are canonical isomorphisms
$$M^{\kappa_{\calU}}_{\Iw^+}\cong M^{\kappa_{\calU}, \AIP}_{\Iw^+}\quad \text{ and }\quad S_{\Iw^+}^{\kappa_{\calU}} \cong S_{\Iw^+}^{\kappa_{\calU}, \AIP}.$$
\end{Corollary}

\begin{Remark}
\normalfont In fact, it will follow from the construction of $\Psi$ that the isomorphisms in Corollary \ref{Corollary: comparison with AIP} is also Hecke-equivariant, except that the $U_p$-operators are only equivariant up to $p$-power scalars. More precisely, the authors of \cite{AIP-2015} normalise their $U_p$-operators $U^{\mathrm{AIP}}_{p,i}$ by dividing by a certain power of $p$. Therefore, for all $i=1,\ldots,g-1$, our $U_{p,i}$ acts as $p^{i(g+1)}U^{\mathrm{AIP}}_{p,i}$, and $U_{p,g}$ acts as $p^{g(g+1)/2}U^{\mathrm{AIP}}_{p,g}$.
\end{Remark}

The rest of the subsection is dedicated to the proof of Theorem \ref{Theorem: comparison with AIP}. 

Let $n$, $v$, $w$, and $(R_{\calU}, \kappa_{\calU})$ be as in Theorem \ref{Theorem: comparison with AIP}. Recall that the $\scrO_{\overline{\calX}_{\Iw^+}(v)}$-module (resp., $\scrO_{\overline{\calX}_{\Iw}(v)}$-module) $\underline{\omega}_{\Iw^+, v}$ (resp., $\underline{\omega}_{\Iw, v}$) is locally free of rank $g$. Let $\calV' \subset \overline{\calX}_{\Iw}(v)$ be an affinoid open subset such that $\underline{\omega}_{\Iw, v}|_{\calV'}$ is free, and let $\calV \subset \overline{\calX}_{\Iw^+}(v)$ be the preimage of $\calV'$. To construct $\Psi$, it suffices to establish a canonical isomorphism
$$\Psi:\underline{\omega}_w^{\kappa_{\calU}}(\calV)\xrightarrow{\sim} h_{\diamond}^*\underline{\omega}_{w, v}^{\kappa_{\calU}, \AIP}(\calV)$$
for every such $\calV$, which is also functorial in $\calV$.

As a preparation, consider the pullback diagram \[
    \begin{tikzcd}
        \adicIW_{w,v, \infty}^+ \arrow[r]\arrow[d, "\pi_{\infty}^{\AIP}"'] & \adicIW_{w,v}^+\arrow[d, "\pi^{\AIP}"]\\
        \overline{\calX}_{\Gamma(p^{\infty})}(v) \arrow[r, "h_{\Iw}"] & \overline{\calX}_{\Iw}(v)
    \end{tikzcd}
\] where $\overline{\calX}_{\Gamma(p^{\infty})}(v)$ is the preimage of $\overline{\calX}_{\Iw}(v)$ under the natural morphism $h_{\Iw}: \overline{\calX}_{\Gamma(p^{\infty}), w} \rightarrow \overline{\calX}_{\Iw, w}$. The existence of the pullback follows from the same argument as in the proof of Proposition \ref{Proposition: omega is admissible}. For later usage, we denote by $\calV_{\infty}$ (resp., $\calV^+_{\infty}$) the preimage of $\calV'$ in $\overline{\calX}_{\Gamma(p^{\infty})}(v)$ (resp., in $\adicIW_{w,v, \infty}^+$) under the projection $h_{\Iw}$ (resp., $h_{\Iw}\circ \pi_{\infty}^{\AIP}$).

Since $\adicIW_{w,v}^+$ is a $\calU_{\GL_g, 1}^{\opp, (w)}\times \calT_{\GL_g,0}^{(w)}\times U_{\GL_g}(\Z/p^n\Z)$-torsor over $\overline{\calX}_{\Iw}$, we know that $\adicIW_{w,v, \infty}^+$ is likewise a $\calU_{\GL_g, 1}^{\opp, (w)}\times \calT_{\GL_g,0}^{(w)}\times U_{\GL_g}(\Z/p^n\Z)$-torsor over $\overline{\calX}_{\Gamma(p^{\infty})}(v)$. In what follows, we provide an explicit moduli interpretation of this torsor, in three steps.

\paragraph{Step 1.} Observe that the natural projection $h_{\Iw}:\overline{\calX}_{\Gamma(p^{\infty})}(v) \rightarrow \overline{\calX}_{\Iw}(v)$ factors as \[
    h_{\Iw}: \overline{\calX}_{\Gamma(p^{\infty})}(v) \xrightarrow{h_1} \overline{\calX}_1(p^n)(v) \rightarrow \overline{\calX}_{\Iw}(v).
\] Indeed, away from the boundary, the map $h_1$ can be described as follows. Let $\calX_{\Gamma(p^{\infty})}(v)$ be the part of $\overline{\calX}_{\Gamma(p^{\infty})}(v)$ away from the boundary. For every point $(A, \lambda, \psi_N, \psi_{p^{\infty}})\in \calX_{\Gamma(p^{\infty})}(v)$, consider the dual trivialisation $$\psi_{p^{\infty}}^{\vee}: V^{\vee}_p\xrightarrow[]{\sim} T_pA^{\vee}.$$
Modulo $p^n$, we obtain a symplectic isomorphism $$\psi_{p^n}^{\vee}:V_p^{\vee}\otimes_{\Z_p}(\Z/p^n\Z)\xrightarrow[]{\sim} A[p^n]^{\vee}.$$
Then $h_1$ sends $(A, \lambda, \psi_N, \psi_{p^{\infty}})$ to $(A, \lambda, \psi_N, \psi)$ where $\psi$ is the composition $$\psi:(\Z/p^n\Z)^g\hookrightarrow V_p^{\vee}\otimes_{\Z_p}(\Z/p^n\Z)\xrightarrow[]{\psi_{p^n}^{\vee}} A[p^n]^{\vee}\twoheadrightarrow H_n^{\vee}$$ with the first arrow sending $\epsilon_i$ to $e^{\vee}_{g+1-i}\otimes 1$, for all $i=1,\ldots, g$, and the last arrow being the natural surjection. From the proof of Lemma \ref{Lemma: alternative definition of w-locus of (strict) Iwahori level variety}, we see that $\psi$ is indeed a trivialisation of $H_n^{\vee}$.

Using the language of 1-motives, this description of $h_1$ also extends to the boundary. The details are left to the readers.

\paragraph{Step 2.} Recall that, in \S \ref{subsection: construction of AIP}, we defined a locally free $\scrO_{\overline{\frakX}^{\tor}_1(p^n)(v)}$-submodule $\scrF\subset \underline{\Omega}_{n, v}$ on $\overline{\frakX}^{\tor}_{1}(p^n)(v)$. Passing to the adic generic fibre, let $\underline{\omega}_{n, v}^+$ denote the sheaf of $\scrO^+_{\overline{\calX}_{1}(p^n)(v)}$-module on $\overline{\calX}_{1}(p^n)(v)$ associated with $\underline{\Omega}_{n, v}$. Then $\scrF$ can be identified with a locally free $\scrO_{\overline{\calX}_{1}(p^n)(v)}^+$-submodule of $\underline{\omega}_{n, v}^+$, which is still denoted by $\scrF$. Moreover, let $\scrF_{\infty}$ be the pullback of $\scrF$ to $\overline{\calX}_{\Gamma(p^{\infty})}(v)$ along $h_1$.

Recall as well the $\scrO_{\overline{\frakX}^{\tor}_{\Gamma(p^n)}}$-modules $\underline{\Omega}_{\Gamma(p^n)}^{\mathrm{mod}}\subset\underline{\Omega}_{\Gamma(p^n)}$ constructed in \S \ref{subsection: perfectoid Siegel modular variety}. Passing to the adic generic fibre, they induce
$\scrO^+_{\overline{\calX}_{\Gamma(p^{n})}}$-modules $\underline{\omega}_{\Gamma(p^n)}^{\mathrm{mod}, +} \subset \underline{\omega}_{\Gamma(p^{n})}^+$ on $\overline{\calX}_{\Gamma(p^{n})}$. Let $\underline{\omega}_{\Gamma(p^{\infty})}^{\mathrm{mod}, +}\subset\underline{\omega}_{\Gamma(p^{\infty})}^+$ be their pullbacks to $\overline{\calX}_{\Gamma(p^{\infty})}$ and let $\underline{\omega}_{\Gamma(p^{\infty}), v}^{\mathrm{mod}, +}\subset\underline{\omega}_{\Gamma(p^{\infty}), v}^+$ be their restrictions on $\overline{\calX}_{\Gamma(p^{\infty})}(v)$.

We claim that there is a natural inclusion \[
    \scrF_{\infty} \subset \underline{\omega}_{\Gamma(p^{\infty}), v}^{\mathrm{mod}, +}.
\] 
Indeed, recall the map 
$$\HT_n:(\Z/p^n\Z)^g\rightarrow \omega_{\calH_n}$$
on $\overline{\calX}_1(p^n)(v)$ constructed in \S \ref{subsection: construction of AIP}. Pulling back to $\overline{\calX}_{\Gamma(p^\infty)}(v)$, we obtain a map
$$\HT_{n,\infty}: (\Z/p^n\Z)^g\rightarrow \omega_{\calH_{n,\infty}}$$
where $\calH_{n,\infty}$ is the pullback of $\calH_n$ along the projection $\overline{\calX}_{\Gamma(p^\infty)}(v)\rightarrow \overline{\calX}_1(p^n)(v)$.
On the other hand, recall the map $\HT_{\Gamma(p^{\infty})}$ on $\overline{\calX}_{\Gamma(p^\infty)}$ constructed in \S \ref{subsection: perfectoid Siegel modular variety}. Restricting to $\overline{\calX}_{\Gamma(p^\infty)}(v)$ and modulo $p^n$, we obtain a map
$$\HT_{\Gamma(p^{\infty}),n,v}: V\otimes_{\Z}(\Z/p^n\Z)\rightarrow \underline{\omega}^{\mathrm{mod}, +}_{\Gamma(p^{\infty}),v}/p^n \underline{\omega}^{\mathrm{mod}, +}_{\Gamma(p^{\infty}),v}.$$

These maps fit into a commutative diagram
\[
    \begin{tikzcd}
        &&\underline{\omega}^{\mathrm{mod},+}_{\Gamma(p^{\infty}),v} \arrow[dd, two heads] \arrow[rr, hook] && \underline{\omega}_{\Gamma(p^{\infty}), v}^+\arrow[d, two heads]\\
        (\Z/p^n\Z)^{g}\arrow[rrrr, crossing over, "\HT_{n, \infty}"]\arrow[d, hook] &&& & \omega_{\calH_{n, \infty}}\arrow[d, two heads]\\
        V\otimes_{\Z}(\Z/p^n\Z)\arrow[rr, "\HT_{\Gamma(p^{\infty}),n,v}"] &&\underline{\omega}^{\mathrm{mod},+}_{\Gamma(p^{\infty}),v}/p^n \arrow[r, hook] & \underline{\omega}_{\Gamma(p^{\infty}), v}^+/p^n \arrow[r, equal] & \omega_{\calH_{n, \infty}}/p^n
    \end{tikzcd}.
\] where the left inclusion sends $\epsilon_i$ to $e_{2g+1-i}\otimes 1$, for all $i=1, \ldots, g$. The equality at the bottom right corner follows from \cite[Proposition 3.2.1]{AIP-2015}. By definition, $\scrF_{\infty}$ is generated by the lifts of $\HT_{n,\infty}(\epsilon_i)$'s from $\omega_{\calH_{n,\infty}}$ to $\underline{\omega}_{\Gamma(p^{\infty}), v}^+$ and hence the desired inclusion follows.

\paragraph{Step 3.} We are now able to describe the torsor. Recall that there is a universal full flag $\Fil^{\univ}_{\bullet}\calH_1^{\vee}$ of $\calH_1^{\vee}$ on $\overline{\calX}_{\Iw}(v)$. Pulling back to $\overline{\calX}_{\Gamma(p^{\infty})}(v)$, we obtain universal full flag $\Fil^{\univ}_{\bullet}\calH_{1, \infty}^{\vee}$ of $\calH_{1, \infty}^{\vee}$. There is a natural projection $\Theta: \calH_{n,\infty}^{\vee}\rightarrow \calH_{1,\infty}^{\vee}$. Moreover, the Hodge--Tate map on $\calH_{n,\infty}^{\vee}$ induces a map
$$\HT_{\calH_{n,\infty}^{\vee}}: \calH_{n,\infty}^{\vee}\rightarrow \scrF_{\infty}\otimes_{\scrO^+_{\overline{\calX}_{\Gamma(p^{\infty})}(v)}}\scrO^+_{\overline{\calX}_{\Gamma(p^{\infty})}(v)}/p^w.$$

Then, for every affinoid open $\calY = \Spa(R, R^+)\subset \overline{\calX}_{\Gamma(p^{\infty})}(v)$, the sections $\adicIW_{w, \infty}^+(\calY)$ parametrise triples $(\psi, \Fil_{\bullet}, \{w_i: i=1,\ldots, g\})$ where
\begin{itemize}
\item $\psi: (\Z/p^n\Z)^g\xrightarrow{\sim} \calH_{n, \infty}^{\vee}|_{\calY}$ is a trivialisation such that
$$\psi\langle \epsilon_1, \ldots, \epsilon_i\rangle=\Theta(\Fil_i^{\univ}\calH_{1,\infty}^{\vee})$$ for all $i=1, \ldots, g$.
\item $\Fil_{\bullet}$ is a full flag of the free $R^+$-module $\scrF_{\infty}(\calY)$, which is $w$-compatible with $\HT_{\calH_{n,\infty}^{\vee}}(\psi(\epsilon_1)), \ldots, \HT_{\calH_{n,\infty}^{\vee}}(\psi(\epsilon_g))$ in the sense of Definition \ref{Definition: w-compatible} (i).
\item Each $w_i$ is an $R^+$-basis for $\Fil_i/\Fil_{i-1}$, which is $w$-compatible with $\HT_{\calH_{n,\infty}^{\vee}}(\psi(\epsilon_1)), \ldots, \HT_{\calH_{n,\infty}^{\vee}}(\psi(\epsilon_g))$ in the sense of Definition \ref{Definition: w-compatible} (ii).
\end{itemize}
Moreover, $\calU_{\GL_g, 1}^{\opp, (w)}\times \calT_{\GL_g,0}^{(w)}\times U_{\GL_g}(\Z/p^n\Z)$ permutes these triples by right multiplication.

We are now ready to prove Theorem \ref{Theorem: comparison with AIP}.

\begin{proof}[Proof of Theorem \ref{Theorem: comparison with AIP}]
The construction of $\Psi$ is similar to the proof of Proposition \ref{Proposition: omega is admissible}. We only give a sketch of the proof. Indeed, the isomorphism $\Psi$ is established via a sequence of isomorphisms 
$$ \Psi: \underline{\omega}_{w}^{\kappa_{\calU}}(\calV) \xrightarrow[\Psi_1]{\sim} \omega^{(1)} \xrightarrow[\Psi_2]{\sim} \omega^{(2)}\xrightarrow[\Psi_3]{\sim} h_{\diamond}^*\underline{\omega}_{w, v}^{\kappa_{\calU}, \AIP}(\calV),
$$
where 
$$
\omega^{(1)} := \left\{ f\in C_{\kappa_{\calU}^{\vee}}^{w-\an}(\Iw_{\GL_g}, \scrO_{\calV_{\infty}}(\calV_{\infty})\widehat{\otimes}R_{\calU}): \bfgamma^* f = \rho_{\kappa_{\calU}^{\vee}}(\bfgamma_a^{\ddagger} + \frakz \bfgamma_c^{\ddagger}) f, \quad \forall \bfgamma = \begin{pmatrix} \bfgamma_a & \bfgamma_b \\ \bfgamma_c & \bfgamma_d\end{pmatrix} \in \Iw_{\GSp_{2g}}^+\right\}
$$
and
$$
\omega^{(2)} := \left\{f\in \pi^{\AIP}_{\infty, *}\scrO_{\adicIW_{w, \infty}^+}(\calV_{\infty})\widehat{\otimes}R_{\calU}: \begin{array}{l}
        \bfgamma^* f = f, \quad \bftau^*f = \kappa_{\calU}^{\vee}(\bftau)f, \quad \bfnu^* f = f  \\
        \forall (\bfgamma, \bftau, \bfnu)\in \Iw_{\GSp_{2g}}^+\times T_{\GL_g, 0}\times U_{\GL_g}(\Z/p^n\Z) 
    \end{array}\right\}.
$$

The construction of $\Psi_1$ and $\Psi_3$ follows verbatim as in Proposition \ref{Proposition: omega is admissible}. To construct $\Psi_2$, consider $\fraks^{\ddagger} = \begin{pmatrix}\fraks_g & \cdots & \fraks_1\end{pmatrix}\in \scrF_{\infty}(\calV_{\infty})^g$. Let $\Fil_{\bullet}^{\ddagger}$ be the full flag of the free $\scrO^+_{\calV_{\infty}}(\calV_{\infty})$-module $\scrF_{\infty}(\calV_{\infty})$ given by \[
    \Fil_{\bullet}^{\ddagger} = 0 \subset \langle \fraks_g \rangle \subset \langle \fraks_g, \fraks_{g-1} \rangle\subset \cdots \langle \fraks_g, \ldots, \fraks_1\rangle
\] and let $w_i^{\ddagger}$ be the image of $\fraks_{g+1-i}$ in $\Fil_{i}^{\ddagger}/\Fil_{i-1}^{\ddagger}$, for all $i=1, \ldots, g$. Moreover, consider the trivialisation \[
    \psi^{\ddagger}: (\Z/p^n\Z)^g \xrightarrow{\sim} \calH_{n,\infty}^{\vee}\] 
    obtained by pulling back the universal trivialisation of $\calH_n^{\vee}$ on $\overline{\calX}_1(p^n)(v)$ along $h_1:\overline{\calX}_{\Gamma(p^{\infty})}(v)\rightarrow \overline{\calX}_1(p^n)(v)$. Then the triple $(\psi^{\ddagger}, \Fil_{\bullet}^{\ddagger}, \{w_i^{\ddagger}\})$ defines a section of the $\calU_{\GL_g, 1}^{\opp, (w)}\times \calT_{\GL_g, 0}^{(w)}\times U_{\GL_g}(\Z/p^n\Z)$-torsor $\pi_{\infty}^{\AIP}: \calV^+_{\infty} \rightarrow \calV_{\infty}$. Consequently, one obtains an isomorphism \[
    \calU_{\GL_g, 1}^{\opp, (w)}\times \calT_{\GL_g, 0}^{(w)} \times U_{\GL_g}(\Z/p^n\Z) \xrightarrow{\sim} \calV^+_{\infty}, \quad \bfgamma' \mapsto (\psi^{\ddagger}, \Fil_{\bullet}^{\ddagger}, \{w_i^{\ddagger}\}) \cdot \bfgamma'
\] and thus an isomorphism \begin{align*}
    \Phi: \pi_{\infty, *}^{\AIP}\scrO_{\adicIW_{w, \infty}^+}(\calV_{\infty})\widehat{\otimes} R_{\calU} & \xrightarrow{\sim} \left\{
        \text{ analytic functions }
        U_{\GL_g, 1}^{\opp, (w)}\times T_{\GL_g, 0}^{(w)}\times U_{\GL_g}(\Z/p^n\Z) \rightarrow \scrO_{\calV_{\infty}}(\calV_{\infty})\widehat{\otimes}R_{\calU} 
\right\}\\
    f & \mapsto \left(\bfgamma'\mapsto f((\psi^{\ddagger}, \Fil_{\bullet}^{\ddagger}, \{w_i^{\ddagger}\}) \cdot \bfgamma')\right).
\end{align*} 
By the same calculation as in Proposition \ref{Proposition: omega is admissible}, one sees that, if $\bfgamma^* f = f$ for any $\bfgamma = \begin{pmatrix}\bfgamma_a & \bfgamma_b \\ \bfgamma_c & \bfgamma_d\end{pmatrix}\in \Iw_{\GSp_{2g}}^+$, then $\bfgamma^*\Phi(f) = \rho_{\kappa_{\calU}^{\vee}}(\bfgamma_a^{\ddagger} + \frakz\bfgamma_c^{\ddagger})\Phi(f)$. This induces an isomorphsm
$\Phi: \omega^{(2)}\xrightarrow[]{\sim}\omega^{(1)}$. Taking $\Psi_2=\Phi^{-1}$ does the job.
\end{proof}

\begin{Remark}\label{Remark: comparison of the integral sheaf}
\normalfont Notice that $\fraks_i$'s are, in fact, integral. Hence, by pulling back the formal scheme $\formalIW_{w}^+$ to the modified integral model, the method above provides a strategy to compare our integral sheaf $\underline{\omega}_{w}^{\kappa_{\calU}, +}$ with the integral overconvergent automorphic sheaf constructed in \cite{AIP-2015}. 
\end{Remark}