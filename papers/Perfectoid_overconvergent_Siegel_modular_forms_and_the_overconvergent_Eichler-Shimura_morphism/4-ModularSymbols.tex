\section{Overconvergent cohomology groups}\label{section:overconvergentcohomologies} 
In this section, we introduce the overconvergent cohomology groups that will later appear on the other side of the overconvergent Eichler--Shimura morphism. Our construction follows the standard constructions in the literature (see, for example, \cite{Hansen-PhD} and \cite{Johansson-Newton}). 

\subsection{Analytic functions and analytic distributions}\label{subsection:continuousfunctions}
Consider $$\T_0:=\left\{(\bfgamma, \bfupsilon)\in \Iw_{\GL_g}\times M_g(p\Z_p): \trans\bfgamma\oneanti_g\bfupsilon= \trans\bfupsilon\oneanti_g\bfgamma\right\}.$$ Notice that a pair $(\bfgamma, \bfupsilon)\in \Iw_{\GL_g}\times M_g(p\Z_p)$ lies in $\T_0$ if and only if there exist $\bfalpha_b, \bfalpha_d\in M_g(\Z_p)$ such that \[\begin{pmatrix}\bfgamma & \bfalpha_b\\ \bfupsilon & \bfalpha_d\end{pmatrix}\in \GSp_{2g}(\Q_p)\cap M_{2g}(\Z_p).\] 
In fact, there is a natural embedding
$$\T_0\hookrightarrow \Iw_{\GSp_{2g}}, \quad (\bfgamma, \bfupsilon)\mapsto \begin{pmatrix}\bfgamma & \\ \bfupsilon & \oneanti_g\trans\bfgamma^{-1}\oneanti_g\end{pmatrix}.$$
Also consider the subset $\T_{00}$ of $\T_0$ defined by $$\T_{00}:=\left\{(\bfgamma, \bfupsilon)\in \T_0: \bfgamma\in U_{\GL_g,1}^{\opp}\right\}.$$ We can identify $\T_{00}$ with $U_{\GSp_{2g}, 1}^{\opp}$ through the bijection $$\T_{00}\rightarrow U_{\GSp_{2g}, 1}^{\opp}, \quad (\bfgamma, \bfupsilon)\mapsto \begin{pmatrix}\bfgamma & \\ \bfupsilon & \oneanti_g\trans\bfgamma^{-1}\oneanti_g\end{pmatrix}.$$ Observe that $\T_0$ admits two natural actions:\begin{enumerate}
    \item[(i)] There is a right action of $\Iw_{\GL_g}$ given by $$\T_0\times \Iw_{\GL_g}\rightarrow \T_0, \quad ((\bfgamma, \bfupsilon), \bfgamma')\mapsto (\bfgamma\bfgamma', \bfupsilon\bfgamma').$$ To see that this is indeed a right action, we embed $\Iw_{\GL_g}$ into $\Iw_{\GSp_{2g}}$ through $\bfgamma'\mapsto \begin{pmatrix}\bfgamma' & \\ & \oneanti_g\trans\bfgamma'^{-1}\oneanti_g \end{pmatrix}$ and verify that 
    \[\begin{pmatrix}\bfgamma & * \\ \bfupsilon & *\end{pmatrix}\begin{pmatrix}\bfgamma' & \\  & \oneanti_g\trans\bfgamma'^{-1}\oneanti_g\end{pmatrix}=\begin{pmatrix}\bfgamma\bfgamma' & * \\ \bfupsilon\bfgamma' & *\end{pmatrix}\]
    
     \item[(ii)] There is a left action of $\Xi:=\begin{pmatrix}\Iw_{\GL_g} & M_g(\Z_p)\\ M_g(p\Z_p) & M_g(\Z_p)\end{pmatrix}\cap \GSp_{2g}(\Q_p)$ given by $$\Xi\times \T_0\rightarrow \T_0, \quad \left( \begin{pmatrix}\bfalpha_a & \bfalpha_b\\ \bfalpha_c & \bfalpha_d\end{pmatrix}, (\bfgamma, \bfupsilon)\right)\mapsto (\bfalpha_a\bfgamma+\bfalpha_b\bfupsilon, \bfalpha_c\bfgamma+\bfalpha_d\bfupsilon).$$ To see this is indeed a left action, it suffices to observe that $$\begin{pmatrix}\bfalpha_a & \bfalpha_b\\ \bfalpha_c & \bfalpha_d\end{pmatrix} \begin{pmatrix}\bfgamma & *\\ \bfupsilon & *\end{pmatrix}=\begin{pmatrix}\bfalpha_a\bfgamma+\bfalpha_b\bfupsilon & *\\ \bfalpha_c\bfgamma+\bfalpha_d\bfupsilon & *\end{pmatrix}.$$
\end{enumerate} 
Since $\Iw_{\GSp_{2g}}^+$ is a subset of $\Xi$, we also obtain a natural left action of $\Iw_{\GSp_{2g}}^+$ on $\T_0$.\\ 

Let $r\in \Z_{\geq 1}$ and let $(R_{\calU}, \kappa_{\calU})$ be an $r$-analytic weight. In what follows, we will study ``$r$-analytic'' functions on $U_{\GSp_{2g}, 1}^{\opp}$, $\T_{00}$, and $\T_0$. Let us fix a (topological) isomorphism $$\Z_p^{g^2}\simeq U_{\GSp_{2g}, 1}^{\opp}.$$ 

\begin{Definition}
\begin{enumerate}
\item[(i)] We say that a function $f: U_{\GSp_{2g}, 1}^{\opp}\rightarrow R^+_{\calU}$ is \textbf{$r$-analytic} if the composition
$$\Z_p^{g^2}\simeq U_{\GSp_{2g}, 1}^{\opp}\xrightarrow[]{f}R_{\calU}^+\hookrightarrow \C_p\widehat{\otimes}R_{\calU}$$
is $r$-analytic in the sense of Definition \ref{Definition: w-analytic functions} (i). 
\item[(ii)] We say that a function $f: \T_{00}\rightarrow R^+_{\calU}$ is \textbf{$r$-analytic} if it is $r$-analytic viewed as a function on $U^{\opp}_{\GSp_{2g}, 1}$, via the identification $\T_{00}\cong U^{\opp}_{\GSp_{2g}, 1}$.
\end{enumerate}
\end{Definition}

Write $$A^{r, \circ}(\T_{00}, R_{\calU}):=\left\{f: \T_{00}\rightarrow R_{\calU}^+: f \text{ is $r$-analytic} \right\}$$ 
and 
$$A^r(\T_{00}, R_{\calU}):=A^{r, \circ}(\T_{00}, R_{\calU})[\frac{1}{p}].$$ 

On the other hand, define 
$$A^{r, \circ}_{\kappa_{\calU}}(\T_0, R_{\calU}):=\left\{f:\T_0\rightarrow R_{\calU}^+: \begin{array}{l}
    f(\bfgamma\bfbeta, \bfupsilon\bfbeta)=\kappa_{\calU}(\bfbeta)f(\bfgamma, \bfupsilon)\,,\,\forall (\bfgamma, \bfupsilon)\in \T_0, \,\,\bfbeta\in B_{\GL_g, 0}\\
    f|_{\T_{00}} \text{ is $r$-analytic}
\end{array}\right\}$$
and
$$A^{r}_{\kappa_{\calU}}(\T_0, R_{\calU}):=A^{r, \circ}_{\kappa_{\calU}}(\T_0, R_{\calU})[\frac{1}{p}].$$
There is an identification $$A_{\kappa_{\calU}}^{r, \circ}(\T_0, R_{\calU})\xrightarrow{\sim} A^{r, \circ}(\T_{00}, R_{\calU}), \quad f\mapsto f|_{\T_{00}}.$$ 
Taking duals, we obtain the corresponding spaces of $r$-analytic distributions
$$
    D_{\kappa_{\calU}}^{r, \circ}(\T_0, R_{\calU}):=\Hom_{R_{\calU}^+}^{\cts}(A_{\kappa_{\calU}}^{r, \circ}(\T_0, R_{\calU}), R_{\calU}^+)
$$
and
$$
D_{\kappa_{\calU}}^{r}(\T_0, R_{\calU}):=D_{\kappa_{\calU}}^{r, \circ}(\T_0, R_{\calU})[\frac{1}{p}].
$$
 The right action of $\Iw_{\GL_g}$ and the left action of $\Xi$ on $\T_0$ then induce, respectively, a right action of $\Iw_{\GL_g}$ and a left action of $\Xi$ on both $D_{\kappa_{\calU}}^{r, \circ}(\T_0, R_{\calU})$ and $D_{\kappa_{\calU}}^{r}(\T_0, R_{\calU})$. Furthermore, if $r'\geq r$, there is a natural injection $A_{\kappa_{\calU}}^{r, \circ}(\T_0, R_{\calU})\hookrightarrow A_{\kappa_{\calU}}^{r', \circ}(\T_0, R_{\calU})$ which induces injections (see \cite[\S 2.2]{Hansen-PhD}) \begin{align*}
    D_{\kappa_{\calU}}^{r', \circ}(\T_0, R_{\calU})\hookrightarrow D_{\kappa_{\calU}}^{r, \circ}(\T_0, R_{\calU})\quad \text{ and }\quad D_{\kappa_{\calU}}^{r'}(\T_0, R_{\calU})\hookrightarrow D_{\kappa_{\calU}}^{r}(\T_0, R_{\calU}).
\end{align*} We then write \[
    D_{\kappa_{\calU}}^{\dagger}(\T_0, R_{\calU}) := \varprojlim_{r} D_{\kappa_{\calU}}^{r}(\T_0, R_{\calU}).
\]

Suppose now that $(R_{\calU}, \kappa_{\calU})$ is a small weight and take $r>1+r_{\calU}$ (see Definition \ref{Definition: w-analytic weight}). Fix an ideal $\fraka_{\calU}$ of $R_{\calU}$ defining the profinite topology on $R_{\calU}$. Similar to \cite[Proposition 3.1]{CHJ-2017}, $D_{\kappa_{\calU}}^{r, \circ}(\T_0, R_{\calU})$ admits a decreasing $\Xi$- and $\Iw_{\GL_g}$-stable filtration $\Fil^{\bullet} D_{\kappa_{\calU}}^{r, \circ}(\T_0, R_{\calU})$ defined by $$\Fil^jD_{\kappa_{\calU}}^{r, \circ}(\T_0, R_{\calU}):=\ker\left(D_{\kappa_{\calU}}^{r, \circ}(\T_0, R_{\calU})\rightarrow D_{\kappa_{\calU}}^{r-1, \circ}(\T_0, R_{\calU})/\fraka_{\calU}^jD_{\kappa_{\calU}}^{r-1, \circ}(\T_0, R_{\calU})\right).$$ Write $$D_{\kappa_{\calU}, j}^{r, \circ}(\T_0, R_{\calU}):=D_{\kappa_{\calU}}^{r, \circ}(\T_0, R_{\calU})/\Fil^j D_{\kappa_{\calU}}^{r, \circ}(\T_0, R_{\calU})$$ for every $j\in \Z_{\geq 1}$. One sees that $$D_{\kappa_{\calU}}^{r, \circ}(\T_0, R_{\calU})=\varprojlim_{j} D_{\kappa_{\calU}, j}^{r, \circ}(\T_0, R_{\calU}),$$ making $D_{\kappa_{\calU}}^{r, \circ}(\T_0, R_{\calU})$ a profinite flat $\Z_p$-module in the sense of \cite[Definition 6.1]{CHJ-2017}.

\begin{Remark}
\normalfont Note that we used the fact that $\kappa_{\calU}$ is a small weight so that each $D_{\kappa_{\calU}, j}^{r, \circ}(\T_0, R_{\calU})$ is a finitely generated $(\Z/p^j\Z)$-module. We do not know if a similar statement holds for affinoid weights.
\end{Remark}


\subsection{Overconvergent cohomology groups}\label{subsection: overconvergent cohomology groups}
Fix a small weight $(R_{\calU}, \kappa_{\calU})$ and we consider the \'{e}tale site $\calX_{\Iw^+, \et}$. Recall that, for every $n\in\Z_{\geq 1}$, $\calX_{\Gamma(p^n)}$ is a finite \'{e}tale Galois cover over $\calX_{\Iw^+}$ with Galois group $\Iw_{\GSp_{2g}}^+/\Gamma(p^n)$, and hence $\varprojlim_n\calX_{\Gamma(p^n)}$ is a pro-\'{e}tale Galois cover of $\calX_{\Iw^+}$ with Galois group $\Iw_{\GSp_{2g}}^+$. For each $j\in \Z_{\geq 1}$, let $\scrD_{\kappa_{\calU}, j}^{r, \circ}$ be the locally constant sheaf on $\calX_{\Iw^+, \et}$ associated with $D_{\kappa_{\calU}, j}^{r, \circ}(\T_0, R_{\calU})$ via $$\pi_1^{\et}(\calX_{\Iw^+})\rightarrow \Iw_{\GSp_{2g}}^+\rightarrow \Aut\left(D_{\kappa_{\calU}, j}^{r, \circ}(\T_0, R_{\calU})\right).$$ We obtain an inverse system of \'etale locally constant sheaves $(\scrD_{\kappa_{\calU}, j}^{r, \circ})_{j\in \Z_{\geq 1}}$ on $\calX_{\Iw^+, \et}$. This allows us to consider the \'{e}tale cohomology groups \begin{align*}
    & H_{\et}^t(\calX_{\Iw^+}, \scrD_{\kappa_{\calU}}^{r, \circ}):=\varprojlim_{j}H_{\et}^t(\calX_{\Iw^+}, \scrD_{\kappa_{\calU}, j}^{r, \circ}),\\
    & H_{\et}^t(\calX_{\Iw^+}, \scrD_{\kappa_{\calU}}^{r}):=H_{\et}^t(\calX_{\Iw^+}, \scrD_{\kappa_{\calU}}^{r, \circ})[\frac{1}{p}]
\end{align*} for every $t\in \Z_{\geq 0}$. 

\begin{Remark}
\normalfont On the algebraic variety $X_{\Iw^+}$, one can define locally constant sheaves $\scrD_{\kappa_{\calU}, j}^{r, \circ}$ and \'etale cohomology groups $H_{\et}^t(X_{\Iw^+}, \scrD_{\kappa_{\calU}}^{r, \circ})$ and $H_{\et}^t(X_{\Iw^+}, \scrD_{\kappa_{\calU}}^{r})$ in the same way.
\end{Remark}

Recall the identification \[
    X_{\Iw^+}(\C) = \GSp_{2g}(\Q)\backslash \GSp_{2g}(\A_f)\times \bbH_g/\Iw_{\GSp_{2g}}^+\Gamma(N).
\] By taking the trivial $\GSp_{2g}(\Z_{\ell})$-action on $D_{\kappa_{\calU}}^r(\T_0, R_{\calU})$ for every prime number $\ell\neq p$ and letting $\Iw_{\GSp_{2g}}^+$ act on $D_{\kappa_{\calU}}^r(\T_0, R_{\calU})$ via the left action of $\Xi$, we see that $D_{\kappa_{\calU}}^r(\T_0, R_{\calU})$ defines a local system on the locally symmetric space $X_{\Iw^+}(\C)$. In particular, for every $t\in \Z_{\geq 0}$, we can consider the Betti cohomology group \[
    H^t(X_{\Iw^+}(\C), D_{\kappa_{\calU}}^r(\T_0, R_{\calU})).
\]

\begin{Proposition}\label{Proposition: Comparison theorem of cohomologies}
For every $t\in \Z_{\geq 0}$, there is a natural isomorphism $$H_{\et}^t(\calX_{\Iw^+}, \scrD_{\kappa_{\calU}}^r)\cong H^t(X_{\Iw^+}(\C), D_{\kappa_{\calU}}^r(\T_0, R_{\calU})).$$\end{Proposition}
\begin{proof}
The assertion follows from the isomorphisms $$H_{\et}^t(\calX_{\Iw^+}, \scrD_{\kappa_{\calU}}^r)\cong H_{\et}^t(X_{\Iw^+}, \scrD_{\kappa_{\calU}}^r)\cong H^t(X_{\Iw^+}(\C), D_{\kappa_{\calU}}^r(\T_0, R_{\calU})),$$ where \begin{enumerate}
    \item[$\bullet$] the first isomorphism follows from the comparison isomorphism between the \'{e}tale cohomology groups of an algebraic variety and the ones on the corresponding adic spaces (see \cite[Theorem 3.8.1]{Huber-2013}); and
    \item[$\bullet$] the second isomorphism follows from the fact that $\Iw_{\GSp_{2g}}^+$ acts continuously on $D_{\kappa_{\calU}}^{r, \circ}(\T_0, R_{\calU})$ and the well-known Artin comparison between the \'{e}tale cohomology of a complex algebraic variety and the Betti cohomology of the associated complex manifold. 
\end{enumerate} Note that we have used the algebraic isomorphism $\C_p\simeq \C$ fixed at the beginning of the paper.
\end{proof}

Finally, we define the Hecke operators acting on $H_{\et}^t(\calX_{\Iw^+},\scrD_{\kappa_{\calU}}^r)$. We begin with a brief recollection of the Hecke operators on $H^t(X_{\Iw^+}(\C), D_{\kappa_{\calU}}^r(\T_0, R_{\calU}))$ studied in \cite{Hansen-PhD}. We refer the readers to \textit{loc. cit.} for a more detailed discussion.

\paragraph{Hecke operators outside $pN$.} Let $\ell$ be a prime number not dividing $pN$. For any $\bfgamma \in \GSp_{2g}(\Q_{\ell})\cap M_{2g}(\Z_{\ell})$, consider a double coset decomposition \[
    \GSp_{2g}(\Z_{\ell}) \bfgamma \GSp_{2g}(\Z_{\ell}) = \bigsqcup_j \bfdelta_{j}\bfgamma\GSp_{2g}(\Z_{\ell})
\] for some $\bfdelta_j\in \GSp_{2g}(\Z_{\ell})$. If we take the trivial $\GSp_{2g}(\Q_{\ell})$-action on $D_{\kappa_{\calU}}^r(\T_0, R_{\calU})$, then the natural left action of $\GSp_{2g}(\Q_{\ell})$ on $X_{\Iw^+}(\C)$ induces the Hecke operator 
\begin{equation}\label{eq: Hecke on coh. away from Np}
    T_{\bfgamma}: H^t(X_{\Iw^+}(\C), D_{\kappa_{\calU}}^r(\T_0, R_{\calU})) \rightarrow H^t(X_{\Iw^+}(\C), D_{\kappa_{\calU}}^r(\T_0, R_{\calU})), \quad [\mu]\mapsto \sum_{j} (\bfdelta_j\bfgamma) .  [\mu].
\end{equation}

\paragraph{Hecke operators at $p$.} For the Hecke operators at $p$, recall from \S \ref{subsection: Hecke operators on the overconvergent automorphic forms} the matrices \begin{align*}
    \bfu_{p,i} = \left\{\begin{array}{cc}
        \begin{pmatrix} \one_i \\ & p\one_{g-i}\\ & & p\one_{g-i}\\ & & & p^2\one_i\end{pmatrix}, & 1\leq i \leq g-1 \\
        \\
        \begin{pmatrix}\one_g\\ & p\one_g\end{pmatrix}, & i=g
    \end{array}\right.
\end{align*} and we write \[
    \bfu_{p,i} = \begin{pmatrix}\bfu_{p,i}^{\square} & \\ & \bfu_{p,i}^{\blacksquare}\end{pmatrix}.
\] 

For every $i=1, \ldots, g$, consider a $\bfu_{p,i}$-action on $\T_0$ defined as follows: for every $(\bfgamma, \bfupsilon)\in \T_0$, we put
\[
\bfu_{p,i} . (\bfgamma, \bfupsilon) = (\bfu_{p,i}^{\square} \bfgamma_0 \bfu_{p,i}^{\square, -1}, \bfu_{p,i}^{\blacksquare} \bfupsilon_0 \bfu_{p,i}^{\square, -1})\bfbeta
\]
where we write $(\bfgamma, \bfupsilon) = (\bfgamma_0, \bfupsilon_0)\bfbeta$ with $\bfgamma_0\in U_{\GL_g, 1}^{\opp}$ and $\bfbeta\in B_{\GL_g, 0}$. This then induces a $\bfu_{p,i}$-action on $D_{\kappa_{\calU}}^{r}(\T_0, R_{\calU})$.

Similar to \S \ref{subsection: Hecke operators on the overconvergent automorphic forms}, for every $i=1, \ldots, g$, choose a double coset decomposition \[
    \Iw_{\GSp_{2g}}^+ \bfu_{p,i} \Iw_{\GSp_{2g}}^+ = \bigsqcup_{j}\bfdelta_{ij}\bfu_{p,i}\Iw_{\GSp_{2g}}^+.
\]
with $\bfdelta_{ij}\in \Iw_{\GSp_{2g}}^+$. The natural left action of $\GSp_{2g}(\Q_p)$ on $X_{\Iw^+}(\C)$ together with the actions of $\Iw_{\GSp_{2g}}^+$ and $\bfu_{p,i}$ on $D_{\kappa_{\calU}}^r(\T_0, R_{\calU})$ induce the Hecke operator \begin{equation}\label{eq: Hecek on coh. at p}
    U_{p,i}: H^t(X_{\Iw^+}(\C), D_{\kappa_{\calU}}^r(\T_0, R_{\calU})) \rightarrow H^t(X_{\Iw^+}(\C), D_{\kappa_{\calU}}^r(\T_0, R_{\calU})), \quad [\mu]\mapsto \sum_{j} \bfdelta_{ij} . (\bfu_{p,i} . [\mu]).
\end{equation}

\begin{Definition}\label{Definition: Hecke operators on overconvergent cohomology}
\begin{enumerate}
    \item[(i)] The Hecke operators $T_{\bfgamma}$ (for $\bfgamma\in \GSp_{2g}(\Q_{\ell})\cap M_{2g}(\Z_{\ell})$ with $\ell\nmid Np$) and $U_{p,i}$ (for $i=1, ..., g$) acting on the overconvergent cohomology groups $H_{\et}^t(\calX_{\Iw^+}, \scrD_{\kappa_{\calU}}^r)$ are defined to be the operators $T_{\bfgamma}$ and $U_{p,i}$ acting on $H^t(X_{\Iw^+}(\C), D_{\kappa_{\calU}}^r(\T_0, R_{\calU}))$ via the isomorphism in Proposition \ref{Proposition: Comparison theorem of cohomologies}.
    \item[(ii)] We define the operator $U_{p}$ as the composition $U_p = \prod_{i=1}^gU_{p,i}$.
\end{enumerate}
\end{Definition}