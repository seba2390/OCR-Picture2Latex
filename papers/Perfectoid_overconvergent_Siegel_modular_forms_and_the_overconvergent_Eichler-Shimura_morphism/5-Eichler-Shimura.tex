\section{The overconvergent Eichler--Shimura morphism}\label{section:EichlerShimura}
In this section, we establish the second main result of this paper; {\it i.e.,} the construction of the overconvergent Eichler--Shimura morphism for Siegel modular forms. Our approach is similar to the one in \cite{CHJ-2017}. However, one major difference between our situation and the one in \emph{loc. cit.} is that the Siegel modular variety is non-compact. As a remedy, we apply the theory of (pro-)Kummer \'etale topology on log adic spaces developed in \cite{Diao} and \cite{Diao-Lan-Liu-Zhu} to handle the boundaries of the compactifications. (See \S \ref{subsection: review of Kummer etale and pro-Kummer etale sites} for a brief review.)


\subsection{The Kummer \'{e}tale and the pro-Kummer \'{e}tale cohomology groups}\label{subsection: pro-Kummer etale cohomology groups}

Recall from \S \ref{subsection: Siegel modular varieties} that $\overline{\calX}_{\Gamma(p^n)}$, $\overline{\calX}_{\Iw^+}$, and $\overline{\calX}$ are endowed with the divisorial log structures defined by the boundary divisors. The corresponding sheaves of monoids are denoted by $\scrM_n$, $\scrM_{\Iw^+}$, and $\scrM$, respectively. In what follows, we shall construct a sheaf $\sheafOD_{\kappa_{\calU}}^r$ on the pro-Kummer \'{e}tale site $\overline{\calX}_{\Iw^+, \proket}$ which computes the overconvergent cohomology groups introduced in \S \ref{subsection: overconvergent cohomology groups}. 

Consider the natural morphism of sites \[\jmath_{\ket}: \calX_{\Iw^+, \et}\rightarrow \overline{\calX}_{\Iw^+, \ket}.\]
Recall that, for every small weight $(R_{\calU}, \kappa_{\calU})$ and any integer $r\geq 1+r_{\calU}$, there is an inverse system of \'{e}tale locally constant sheaves $(\scrD_{\kappa_{\calU}, j}^{r, \circ})_{j\in \Z_{\geq 1}}$ on $\calX_{\Iw^+, \et}$. Applying \cite[Corollary 4.6.7]{Diao}, we obtain an isomorphism $$\varprojlim_{j}H_{\et}^t(\calX_{\Iw^+}, \scrD_{\kappa_{\calU}, j}^{r, \circ})\cong \varprojlim_{j} H_{\ket}^{t}(\overline{\calX}_{\Iw^+}, \jmath_{\ket, *}\scrD_{\kappa_{\calU}, j}^{r, \circ})$$ for every $t\in \Z_{\geq 0}$. Write \[H_{\ket}^t(\overline{\calX}_{\Iw^+}, \scrD_{\kappa_{\calU}}^r):=\varprojlim_{j} H_{\ket}^{t}(\overline{\calX}_{\Iw^+}, \jmath_{\ket, *}\scrD_{\kappa_{\calU}, j}^{r, \circ})[1/p].\] By Proposition \ref{Proposition: Comparison theorem of cohomologies}, we arrive at isomorphisms $$H_{\ket}^t(\overline{\calX}_{\Iw^+}, \scrD_{\kappa_{\calU}}^{r})\cong H_{\et}^t(\calX_{\Iw^+}, \scrD_{\kappa_{\calU}}^{r})\cong H^t(X_{\Iw^+}(\C), D_{\kappa_{\calU}}^{r}(\T_0, R_{\calU})).$$ To simplify the notation, we introduce the following abbreviations. 

\begin{Definition}\label{Definition: simple notations for overconvergent cohomologies}
Let $(R_{\calU}, \kappa_{\calU})$ be a small weight and let $r\geq 1+r_{\calU}$. We set \begin{align*}
    \OC_{\kappa_{\calU}}^{r, \circ} & := \varprojlim_{j} H_{\ket}^{n_0}(\overline{\calX}_{\Iw^+}, \jmath_{\ket, *}\scrD_{\kappa_{\calU}, j}^{r, \circ}),\\
    \OC_{\kappa_{\calU}}^{r} & := \OC_{\kappa_{\calU}}^{r, \circ}[\frac{1}{p}]=H_{\ket}^{n_0}(\overline{\calX}_{\Iw^+}, \scrD_{\kappa_{\calU}}^r),\\
    \OC_{\kappa_{\calU}, \calO_{\C_p}}^{r, \circ} & := \varprojlim_{j}\left(H_{\ket}^{n_0}(\overline{\calX}_{\Iw^+}, \jmath_{\ket, *}\scrD_{\kappa_{\calU}, j}^{r, \circ})\otimes_{\Z_p}\calO_{\C_p}\right),\\
    \OC_{\kappa_{\calU}, \C_p}^{r} & := \OC_{\kappa_{\calU}, \calO_{\C_p}}^{r, \circ}[\frac{1}{p}].
\end{align*}
where $n_0=\dim_{\C_p}\calX_{\Iw^+}$.
\end{Definition}

Let \[\nu: \overline{\calX}_{\Iw^+, \proket}\rightarrow \overline{\calX}_{\Iw^+, \ket}\] be the natural projection of sites. Consider the sheaf $\sheafOD_{\kappa_{\calU}}^r$ on the pro-Kummer \'{e}tale site $\overline{\calX}_{\Iw^+, \proket}$ defined by $$\sheafOD_{\kappa_{\calU}}^r:=\left(\varprojlim_{j}\left(\nu^{-1}\jmath_{\ket, *}\scrD_{\kappa_{\calU}, j}^{r, \circ}\otimes_{\Z_p}\scrO_{\overline{\calX}_{\Iw^+, \proket}}^+\right)\right)[\frac{1}{p}].$$ 

\begin{Proposition}\label{Proposition: overconvergent cohomology computed by the pro-Kummer etale cohomology}
There is a $G_{\Q_p}$-equivariant isomorphism $$\OC_{\kappa_{\calU}, \C_p}^r\cong H_{\proket}^{n_0}(\overline{\calX}_{\Iw^+}, \sheafOD_{\kappa_{\calU}}^r).$$
\end{Proposition}
\begin{proof}
By \cite[Theorem 6.2.1 \& Corollary 6.3.4]{Diao}, there is an almost isomorphism $$\left(H^{n_0}_{\ket}(\overline{\calX}_{\Iw^+}, \jmath_{\ket, *}\scrD_{\kappa_{\calU}, j}^{r, \circ})\otimes_{\Z_p}\calO_{\C_p}\right)^a\cong H_{\proket}^{n_0}(\overline{\calX}_{\Iw^+}, \nu^{-1}\jmath_{\ket, *}\scrD_{\kappa_{\calU}, j}^{r, \circ}\otimes_{\Z_p}\scrO_{\overline{\calX}_{\Iw^+, \proket}}^+)^a.$$ It remains to establish an almost isomorphism $$\varprojlim_{j} H_{\proket}^{n_0}\left(\overline{\calX}_{\Iw^+}, \nu^{-1}\jmath_{\ket, *}\scrD_{\kappa_{\calU}, j}^{r, \circ}\otimes_{\Z_p}\scrO_{\overline{\calX}_{\Iw^+, \proket}}^+\right)^a\cong H_{\proket}^{n_0}\left(\overline{\calX}_{\Iw^+}, \varprojlim_j\left(\nu^{-1}\jmath_{\ket, *}\scrD_{\kappa_{\calU}, j}^{r, \circ}\otimes_{\Z_p}\scrO_{\overline{\calX}_{\Iw^+, \proket}}^+\right)\right)^a.$$ Indeed, observe that the higher inverse limit $R^i \varprojlim_j\left(\nu^{-1}\jmath_{\ket, *}\scrD_{\kappa_{\calU}, j}^{r, \circ}\otimes_{\Z_p}\scrO_{\overline{\calX}_{\Iw^+, \proket}}^+\right)$ almost vanishes for $i\geq 1$ by an almost version of \cite[Lemma 3.18]{Scholze_2013} and \cite[Proposition 6.1.11]{Diao}. This then allows us to commute the inverse limit with taking cohomology, hence the result.
\end{proof}

Thanks to Proposition \ref{Proposition: overconvergent cohomology computed by the pro-Kummer etale cohomology}, $H_{\proket}^{n_0}(\overline{\calX}_{\Iw^+}, \sheafOD_{\kappa_{\calU}}^r)$ inherits actions of the Hecke operators $T_{\bfgamma}$ and $U_{p,i}$ from $H^{n_0}_{\et}(\calX_{\Iw^+}, \scrD^r_{\kappa_{\calU}})$. On the other hand, thanks to the $\scrO_{\overline{\calX}_{\Iw^+, \proket}}$-module structure on $\sheafOD_{\kappa_{\calU}}^r$, there is an alternative way to define the Hecke operators $T_{\bfgamma}$'s using correspondences. More precisely, for any prime number $\ell\nmid Np$ and any $\bfgamma\in \GSp_{2g}(\Q_{\ell})\cap M_{2g}(\Z_{\ell})$, consider the correspondence \[
    \begin{tikzcd}
        & \calX_{\bfgamma, \Iw^+}\arrow[dl, "\pr_1"']\arrow[rd, "\pr_2"]\\
        \calX_{\Iw^+} && \calX_{\Iw^+}
    \end{tikzcd},
\] studied in \S \ref{subsection: Hecke operators on the overconvergent automorphic forms}. Similar to the construction in \S \ref{subsection: Hecke operators on the overconvergent automorphic forms}, one obtains an isomorphism \[
    \varphi_{\bfgamma} : \pr_2^*\sheafOD_{\kappa_{\calU}}^r|_{\calX_{\Iw^+}}  \xrightarrow{\sim} \pr_1^*\sheafOD_{\kappa_{\calU}}^r|_{\calX_{\Iw^+}} .
\] Consider the composition \[
    \begin{tikzcd}
    T_{\bfgamma}': &  H^{n_0}_{\proet}(\calX_{\Iw^+}, \sheafOD_{\kappa_{\calU}}^r|_{\calX_{\Iw^+}} )\arrow[r, "\pr_2^*"] & H^{n_0}_{\proet}(\calX_{\bfgamma, \Iw^+}, \pr_2^*\sheafOD_{\kappa_{\calU}}^r|_{\calX_{\Iw^+}} )\arrow[ld, out=-10, in=170, "\varphi_{\bfgamma}"']\\
    &  H^{n_0}_{\proet}(\calX_{\Iw^+}, \pr_1^*\sheafOD_{\kappa_{\calU}}^r|_{\calX_{\Iw^+}} )\arrow[r, "\Tr_{\pr_1}"] & H_{\proet}^{n_0}(\calX_{\Iw^+}, \sheafOD_{\kappa_{\calU}}^r|_{\calX_{\Iw^+}})
    \end{tikzcd}.
\] 
However, since $H_{\et}^{n_0}(\calX_{\Iw^+}, \scrD_{\kappa_{\calU}, j}^{r, \circ})\cong H_{\ket}^{n_0}(\overline{\calX}_{\Iw^+}, \jmath_{\ket, *}\scrD_{\kappa_{\calU}, j}^{r, \circ})$ for every $j$, we have an identification
$$H^{n_0}_{\proet}(\calX_{\Iw^+}, \sheafOD_{\kappa_{\calU}}^r|_{\calX_{\Iw^+}} )\cong H_{\proket}^{n_0}(\overline{\calX}_{\Iw^+}, \sheafOD_{\kappa_{\calU}}^r)$$ and hence an operator $T_{\bfgamma}'$ on $H_{\proket}^{n_0}(\overline{\calX}_{\Iw^+}, \sheafOD_{\kappa_{\calU}}^r)$. One checks that $T_{\bfgamma}$ coincides with $T_{\bfgamma}'$.


\subsection{The overconvergent Eichler--Shimura morphism}\label{subsection: OES}
In this subsection, we construct the overconvergent Eichler--Shimura morphism by first constructing a morphism between sheaves on the pro-Kummer \'{e}tale site $\overline{\calX}_{\Iw^+, w, \proket}$.

Let $(R_{\calU}, \kappa_{\calU})$ be a small weight and let $w \geq r\geq 1+r_{\calU}$. Recall that we have defined a sheaf $\sheafOD_{\kappa_{\calU}}^{r}$ on the pro-Kummer \'{e}tale site $\overline{\calX}_{\Iw^+, \proket}$ in \S \ref{subsection: pro-Kummer etale cohomology groups}. The following lemma is an analogue of \cite[Lemma 4.5]{CHJ-2017}.

\begin{Lemma}\label{Lemma: Explicit description of the sheaf OD}
Let $\calV=\varprojlim_{n}\calV_n\rightarrow \overline{\calX}_{\Iw^+}$ be a pro-Kummer \'{e}tale presentation of a log affinoid perfectoid object in $\overline{\calX}_{\Iw^+, \proket}$. Let $\calV_{\infty}:=\calV\times_{\overline{\calX}_{\Iw^+}}\overline{\calX}_{\Gamma(p^{\infty})}$. (Here we have abused the notation and identify $\overline{\calX}_{\Gamma(p^{\infty})}$ with the object $\varprojlim_n \overline{\calX}_{\Gamma(p^n)}$ in $\overline{\calX}_{\Iw^+, \proket}$.) Then there is a natural isomorphism $$\sheafOD_{\kappa_{\calU}}^r(\calV)\cong\left(D_{\kappa_{\calU}}^{r, \circ}(\T_0, R_{\calU})\widehat{\otimes}_{\Z_p}\widehat{\scrO}_{\overline{\calX}_{\Iw^+, \proket}}(\calV_{\infty})\right)^{\Iw^+_{\GSp_{2g}}}.$$
\end{Lemma}
\begin{proof}
Recall that $\scrD_{\kappa_{\calU}, j}^{r, \circ}$ is the locally constant sheaf on $\calX_{\Iw^+, \et}$ induced by $$\pi_1^{\et}(\calX_{\Iw^+})\rightarrow \Iw^+_{\GSp_{2g}}\rightarrow \Aut\left(D_{\kappa_{\calU}, j}^{r, \circ}(\T_0, R_{\calU})\right).$$ 
Since $\overline{\calX}_{\Gamma(p^{\infty})}$ is a profinite Galois cover of $\overline{\calX}_{\Iw^+}$ with Galois group $\Iw^+_{\GSp_{2g}}$, one sees that $\nu^{-1}\jmath_{\ket, *}\scrD_{\kappa_{\calU}, j}^{r, \circ}$ becomes the constant local system associated with $D_{\kappa_{\calU}, j}^{r, \circ}(\T_0, R_{\calU})$ after restricting to the localised site $\overline{\calX}_{\Iw^+, \proket}/\overline{\calX}_{\Gamma(p^{\infty})}$. 

Applying \cite[Theorem 5.4.3]{Diao}, we obtain an almost isomorphism $$\left(D_{\kappa_{\calU}, j}^{r, \circ}(\T_0, R_{\calU})\otimes_{\Z_p}\widehat{\scrO}_{\overline{\calX}_{\Iw^+, \proket}}^{+}(\calV_{\infty})\right)^a\cong \left(\left(\nu^{-1}\jmath_{\ket,*}\scrD_{\kappa_{\calU}, j}^{r, \circ}\otimes_{\Z_p}\scrO_{\overline{\calX}_{\Iw^+, \proket}}^+\right)(\calV_{\infty})\right)^a.$$ By taking $\Iw^+_{\GSp_{2g}}$-invariants, we obtain almost isomorphisms \begin{align*}
    \left(\left(D_{\kappa_{\calU}, j}^{r, \circ}(\T_0, R_{\calU})\otimes_{\Z_p}\widehat{\scrO}_{\overline{\calX}_{\Iw^+, \proket}}^{+}(\calV_{\infty})\right)^{\Iw^+_{\GSp_{2g}}}\right)^a & \cong \left(\left(\left(\nu^{-1}\jmath_{\ket,*}\scrD_{\kappa_{\calU}, j}^{r, \circ}\otimes_{\Z_p}\scrO_{\overline{\calX}_{\Iw^+, \proket}}^+\right)(\calV_{\infty})\right)^{\Iw^+_{\GSp_{2g}}}\right)^a\\
    & = \left(\left(\nu^{-1}\jmath_{\ket,*}\scrD_{\kappa_{\calU}, j}^{r, \circ}\otimes_{\Z_p}\scrO_{\overline{\calX}_{\Iw^+, \proket}}^+\right)(\calV)\right)^{a}.
\end{align*} Finally, taking inverse limit over $j$ and inverting $p$, we conclude that \begin{align*}
    \sheafOD_{\kappa_{\calU}}^r(\calV) & = \left(\varinjlim_{j}\left(\nu^{-1}\jmath_{\ket, *}\scrD_{\kappa_{\calU}, j}^{r, \circ}\otimes_{\Z_p}\scrO_{\overline{\calX}_{\Iw^+, \proket}}^+\right)(\calV)\right)[\frac{1}{p}]\\
    & \cong \left(D_{\kappa_{\calU}}^{r, \circ}(\T_0, R_{\calU})\widehat{\otimes}_{\Z_p}\widehat{\scrO}_{\overline{\calX}_{\Iw^+, \proket}}(\calV_{\infty})\right)^{\Iw^+_{\GSp_{2g}}}.
\end{align*}
\end{proof}

To deal with the overconvergent automorphic sheaves, we recall the Kummer \'etale sheaves $\underline{\omega}^{\kappa_{\calU},+}_{w,\ket}$ and $\underline{\omega}^{\kappa_{\calU}}_{w,\ket}$ associated with $\underline{\omega}^{\kappa_{\calU},+}_{w}$ and $\underline{\omega}^{\kappa_{\calU}}_{w}$ defined by the end of \S \ref{subsection: admissibility}. Then we consider the $p$-adically completed pullback of them to the pro-Kummer \'etale site; namely,
\[
\widehat{\underline{\omega}}_{w}^{\kappa_{\calU},+}:=\varprojlim_m\left(\underline{\omega}^{\kappa_{\calU},+}_{w,\ket}\bigotimes_{\scrO^+_{\overline{\calX}_{\Iw^+, w,\ket}}}\scrO^+_{\overline{\calX}_{\Iw^+, w, \proket}}/p^m\right)
\]
and \[\widehat{\underline{\omega}}_{w}^{\kappa_{\calU}}:=\widehat{\underline{\omega}}_{w}^{\kappa_{\calU},+}[\frac{1}{p}].\]

\begin{Lemma}\label{Lemma: Leray spectral sequence for the automorphic sheaf}
There is a canonical $G_{\Q_p}$-equivariant morphism $$H_{\proket}^{n_0}(\overline{\calX}_{\Iw^+, w}, \widehat{\underline{\omega}}_w^{\kappa_{\calU}}) \rightarrow H^0(\overline{\calX}_{\Iw^+, w}, \underline{\omega}_w^{\kappa_{\calU}+g+1})(-n_0).$$
\end{Lemma}
\begin{proof}
By the discussion at the end of \S \ref{subsection: admissibility}, we have seen that $\underline{\omega}^{\kappa_{\calU}}_{w,\ket}$ can be identified with the sheaf of $\Iw^+_{\GSp_{2g}}/\Gamma(p^n)$-invariants of an admissible Kummer \'etale Banach sheaf of $\scrO_{\overline{\calX}_{\Iw^+, w,\ket}}\widehat{\otimes}R_{\calU}$-modules. Corollary \ref{Corollary: generalised projection formula with invariants} then yields a canonical isomorphism
$$\underline{\omega}_{w, \ket}^{\kappa_{\calU}}\otimes_{\scrO_{\overline{\calX}_{\Iw^+, w,\ket}}}R^i\nu_*\widehat{\scrO}_{\overline{\calX}_{\Iw^+, w, \proket}}\xrightarrow{\sim}R^i\nu_*\widehat{\underline{\omega}}_{w}^{\kappa_{\calU}}$$ for every $i\in \Z_{\geq 0}$. 
On the other hand, by Proposition \ref{Proposition: compatibility with completed tensor}, we have a canonical isomorphism $$R^i\nu_*\widehat{\scrO}_{\overline{\calX}_{\Iw^+, w, \proket}}\cong \Omega^{\log, i}_{\overline{\calX}_{\Iw^+, w, \ket}}(-i).$$ Combining the two isomorphisms, we obtain
$$
R^i\nu_*\widehat{\underline{\omega}}_{w}^{\kappa_{\calU}}\cong \underline{\omega}_{w, \ket}^{\kappa_{\calU}}\otimes_{\scrO_{\overline{\calX}_{\Iw^+, w,\ket}}} \Omega^{\log, i}_{\overline{\calX}_{\Iw^+, w, \ket}}(-i).
$$

Moreover, there is a Leray spectral sequence $$E_2^{j, i}=H_{\ket}^{j}(\overline{\calX}_{\Iw^+, w}, R^i\nu_*\widehat{\underline{\omega}}_w^{\kappa_{\calU}})\Rightarrow H_{\proket}^{j+i}(\overline{\calX}_{\Iw^+, w}, \widehat{\underline{\omega}}_w^{\kappa_{\calU}}).$$ The edge map yields a Galois-equivariant morphism 
$$H^{n_0}_{\proket}(\overline{\calX}_{\Iw^+, w}, \widehat{\underline{\omega}}_w^{\kappa_{\calU}})\rightarrow H^0_{\ket}(\overline{\calX}_{\Iw^+, w}, R^{n_0}\nu_*\widehat{\underline{\omega}}_w^{\kappa_{\calU}})\cong H^{0}_{\ket}(\overline{\calX}_{\Iw^+, w}, \underline{\omega}_{w, \ket}^{\kappa_{\calU}}\otimes_{\scrO_{\overline{\calX}_{\Iw^+, w,\ket}}} \Omega^{\log, n_0}_{\overline{\calX}_{\Iw^+, w, \ket}})(-n_0).$$ 

Finally, let $\pi_{\Iw^+}:\calG_{\Iw^+, w}^{\univ}\rightarrow \overline{\calX}_{\Iw^+, w}$ denote the universal semiabelian variety over $\overline{\calX}_{\Iw^+, w}$ and let $$\underline{\omega}_{\Iw^+, w}:=\pi_{\Iw^+, *}\Omega^1_{\calG_{\Iw^+, w}^{\univ}/\overline{\calX}_{\Iw^+ ,w}}.$$ 
Note that $\underline{\omega}_{\Iw^+, w}$ agrees with $\underline{\omega}^k_{\Iw^+}|_{\overline{\calX}_{\Iw^+ ,w}}$ studied in \S \ref{subsection: classical forms} for $k=(1,0,\ldots, 0)$. The Kodaira--Spencer isomorphism \cite[Theorem 1.41 (4)]{LanKS} yields an isomorphism $$\Sym^2\underline{\omega}_{\Iw^+, w}\cong \Omega_{\overline{\calX}_{\Iw^+, w}}^{\log, 1}.$$ 
Hence, 
$$\Omega_{\overline{\calX}_{\Iw^+, w}}^{\log, n_0}\cong \bigwedge^{n_0}\left(\Sym^2\underline{\omega}_{\Iw^+, w}\right)=\underline{\omega}_{\Iw^+, w}^{g+1}\subset \underline{\omega}_w^{g+1}$$
where the last inclusion follows from Lemma \ref{Lemma: injection of classical forms}. We obtain an injection $$
H^{0}_{\ket}(\overline{\calX}_{\Iw^+, w}, \underline{\omega}_{w, \ket}^{\kappa_{\calU}}\otimes_{\scrO_{\overline{\calX}_{\Iw^+, w,\ket}}} \Omega^{\log, n_0}_{\overline{\calX}_{\Iw^+, w, \ket}})(-n_0) \hookrightarrow H_{\ket}^0(\overline{\calX}_{\Iw^+, w}, \underline{\omega}_{w,\ket}^{\kappa_{\calU}+g+1})(-n_0)=H^0(\overline{\calX}_{\Iw^+, w}, \underline{\omega}_w^{\kappa_{\calU}+g+1})(-n_0).$$
\end{proof}

For any matrix $\bfsigma\in M_g(\calO_{\C_p})$ and $\mu\in D_{\kappa_{\calU}}^{r}(\T_0, R_{\calU})$, we define a function $f_{\mu, \bfsigma}\in C_{\kappa_{\calU}}^{w-\an}(\Iw_{\GL_g}, \C_p\widehat{\otimes}R_{\calU})$ as follows. For any $\bfgamma'\in \Iw_{\GL_g}$, we define
$$f_{\mu, \bfsigma}(\bfgamma'):= \int_{(\bfgamma, \bfupsilon)\in \T_0}e_{\kappa_{\calU}}^{\hst}(\trans\bfgamma'(\bfgamma+\bfsigma\bfupsilon))\quad d\mu,$$ where $e_{\kappa_{\calU}}^{\hst}$ sends a matrix $X=(X_{ij})_{1\leq i,j\leq g}$ in $\Iw^{(w)}_{\GL_g}$ to 
$$    e_{\kappa_{\calU}}^{\hst}(X)=\frac{\kappa_{\calU, 1}(X_{11})}{\kappa_{\calU, 2}(X_{11})}\times \frac{\kappa_{\calU, 2}(\det((X_{ij})_{1\leq i,j\leq 2}))}{\kappa_{\calU, 3}(\det((X_{ij})_{1\leq i,j\leq 2}))}\times \cdots \times \kappa_{\calU, g}(\det(X)).
$$

The following lemma justifies this definition.
\begin{Lemma}
\begin{enumerate}
    \item[(i)] For every $\bfsigma\in M_g(\calO_{\C_p})$ and $\bfgamma'\in \Iw_{\GL_g}$, the assignment $$(\bfgamma, \bfupsilon)\mapsto e_{\kappa_{\calU}}^{\hst}(\trans\bfgamma'(\bfgamma+\bfsigma\bfupsilon))$$ defines an element in $A^r_{\kappa_{\calU}}(\T_0, R_{\calU})$.
    \item[(ii)] For every $\bfgamma'\in \Iw_{\GL_g}$ and $\bfbeta\in B_{\GL_g, 0}$, we have $$f_{\mu, \bfsigma}(\bfgamma'\bfbeta)=\kappa_{\calU}(\bfbeta)f_{\mu, \bfsigma}(\bfgamma').$$
\end{enumerate} 
\end{Lemma}

\begin{proof}
This is straightforward.
\end{proof}

\begin{Remark}
\normalfont The function $e^{\hst}_{\kappa_{\calU}}$ is an analogue of the highest weight vector in an algebraic representation of $\GL_g$. Moreover, $e^{\hst}_{\kappa_{\calU}}$ has the following alternative interpretation: for every $\bfgamma\in \Iw^{(w)}_{\GL_g}$, if we write $\bfgamma=\bfnu\bftau\bfnu'$ with $\bfnu\in U^{\opp,(w)}_{\GL_g, 1}$, $\bftau\in T^{(w)}_{\GL_g, 0}$, and $\bfnu'\in U^{(w)}_{\GL_g, 0}$, then we have $e^{\hst}_{\kappa_{\calU}}(\bfgamma)=\kappa_{\calU}(\bftau)$.
\end{Remark}

We are ready to construct the desired morphism $\eta_{\kappa_{\calU}}:\sheafOD_{\kappa_{\calU}}^r\rightarrow\widehat{\underline{\omega}}_{w}^{\kappa_{\calU}}$ between sheaves on the pro-Kummer \'{e}tale site $\overline{\calX}_{\Iw^+, w, \proket}$. Indeed, it suffices to construct a map $\sheafOD_{\kappa_{\calU}}^r(\calV)\rightarrow \widehat{\underline{\omega}}_{w}^{\kappa_{\calU}}(\calV)$ for every log affinoid perfectoid object $\calV$ in $\overline{\calX}_{\Iw^+, \proket}$. By Lemma \ref{Lemma: Explicit description of the sheaf OD}, we have
$$\sheafOD_{\kappa_{\calU}}^r(\calV)\simeq\left(D_{\kappa_{\calU}}^{r, \circ}(\T_0, R_{\calU})\widehat{\otimes}_{\Z_p}\widehat{\scrO}_{\overline{\calX}_{\Iw^+, \proket}}(\calV_{\infty})\right)^{\Iw^+_{\GSp_{2g}}}$$
where $\calV_{\infty}:=\calV\times_{\overline{\calX}_{\Iw^+}}\overline{\calX}_{\Gamma(p^{\infty})}$. 

On the other hand, by definition, we know that $\widehat{\underline{\omega}}_{w}^{\kappa_{\calU}}(\calV)$ consists of $f\in C^{w-\an}_{\kappa_{\calU}}(\Iw_{\GL_g}, \widehat{\scrO}_{\overline{\calX}_{\Iw^+, \proket}}(\calV_{\infty})\widehat{\otimes}R_{\calU})$ satisfying $\bfalpha^*f=\rho_{\kappa_{\calU}}(\bfalpha_a+\frakz\bfalpha_c)^{-1}f$, for all $\bfalpha=\begin{pmatrix}\bfalpha_a & \bfalpha_b\\ \bfalpha_c & \bfalpha_d\end{pmatrix}\in \Iw_{\GSp_{2g}}^+$. This is equivalent to saying that $\widehat{\underline{\omega}}_{w}^{\kappa_{\calU}}(\calV)$ consists of $\Iw_{\GSp_{2g}}^+$-invariant elements $f\in C^{w-\an}_{\kappa_{\calU}}(\Iw_{\GL_g}, \widehat{\scrO}_{\overline{\calX}_{\Iw^+, \proket}}(\calV_{\infty})\widehat{\otimes}R_{\calU})$ with respect to the twisted $\Iw_{\GSp_{2g}}^+$-action 
$$\bfalpha.f:=\rho_{\kappa_{\calU}}(\bfalpha_a+\frakz\bfalpha_c)(\bfalpha^*f).$$

Consider the map
$$D_{\kappa_{\calU}}^{r, \circ}(\T_0, R_{\calU})\widehat{\otimes}_{\Z_p}\widehat{\scrO}_{\overline{\calX}_{\Iw^+, \proket}}(\calV_{\infty})\rightarrow C^{w-\an}_{\kappa_{\calU}}(\Iw_{\GL_g}, \widehat{\scrO}_{\overline{\calX}_{\Iw^+, \proket}}(\calV_{\infty})\widehat{\otimes}R_{\calU}), \quad \mu\otimes \delta\mapsto \delta f_{\mu, \frakz}.$$
We claim that this map is $\Iw_{\GSp_{2g}}^+$-equivariant, and hence taking the $\Iw_{\GSp_{2g}}^+$-invariants yields the desired map $\sheafOD_{\kappa_{\calU}}^r(\calV)\rightarrow\widehat{\underline{\omega}}_{w}^{\kappa_{\calU}}(\calV)$. Indeed, for any $\bfalpha=\begin{pmatrix}\bfalpha_a & \bfalpha_b\\ \bfalpha_c & \bfalpha_d\end{pmatrix}\in \Iw_{\GSp_{2g}}^+$ and any $\bfgamma'\in \Iw_{\GL_g}$, we have 
\begin{align*}
(\bfalpha^* \delta) f_{\bfalpha\cdot\mu, \frakz}(\bfgamma') & =  (\bfalpha^* \delta)\left(\int_{\T_0} e_{\kappa_{\calU}}^{\hst}(\trans\bfgamma'(\bfgamma+\frakz\bfupsilon))\quad d\bfalpha\cdot \mu\right) \\
    & = (\bfalpha^* \delta)\left(\int_{\T_0}e_{\kappa_{\calU}}^{\hst}\left(\trans\bfgamma'\left((\bfalpha_a\bfgamma+\bfalpha_b\bfupsilon)+\frakz(\bfalpha_c\bfgamma+\bfalpha_d\bfupsilon)\right)\right)\quad d\mu\right)\\
    & = (\bfalpha^* \delta) \left(\int_{\T_0}e_{\kappa_{\calU}}^{\hst}\left(\trans\bfgamma'((\bfalpha_a+\frakz\bfalpha_c)\bfgamma+(\bfalpha_b+\frakz\bfalpha_d)\bfupsilon)\right)\quad d\mu\right)\\
    & = (\bfalpha^* \delta)\left(\int_{\T_0}e_{\kappa_{\calU}}^{\hst}\left(\trans\bfgamma'(\bfalpha_a+\frakz\bfalpha_c)(\bfgamma+(\bfalpha_a+\frakz\bfalpha_c)^{-1}(\bfalpha_b+\frakz\bfalpha_d)\bfupsilon)\right)\quad d\mu\right) \\
    & = (\bfalpha^* \delta)\left(\int_{\T_0}e_{\kappa_{\calU}}^{\hst}\left(\trans(\trans(\bfalpha_a+\frakz\bfalpha_c)\bfgamma')(\bfgamma+(\frakz\cdot \bfalpha)\bfupsilon)\right)\quad d\mu\right) \\
    & = (\bfalpha^* \delta) \left(\rho_{\kappa_{\calU}}(\bfalpha_a+\frakz\bfalpha_c) \int_{\T_0}e_{\kappa_{\calU}}^{\hst}\left(\trans\bfgamma'(\bfgamma+(\frakz\cdot\bfalpha)\bfupsilon)\right)\quad d\mu\right)\\
    &= \bfalpha.(\delta f_{\mu, \frakz})(\bfgamma')
\end{align*}
as desired.

Putting everything together, we consider the composition
\begin{align*}
 \OC_{\kappa_{\calU}, \C_p}^{r} \cong & H_{\proket}^{n_0}(\overline{\calX}_{\Iw^+}, \sheafOD_{\kappa_{\calU}}^r)\xrightarrow[]{\Res}  H_{\proket}^{n_0}(\overline{\calX}_{\Iw^+,w}, \sheafOD_{\kappa_{\calU}}^r)\\
  \xrightarrow[]{\eta_{\kappa_{\calU}}} & H_{\proket}^{n_0}(\overline{\calX}_{\Iw^+, w}, \widehat{\underline{\omega}}_w^{\kappa_{\calU}}) \rightarrow H^0(\overline{\calX}_{\Iw^+, w}, \underline{\omega}_w^{\kappa_{\calU}+g+1})(-n_0)=M_{\Iw^+, w}^{\kappa_{\calU}+g+1}(-n_0)
 \end{align*}
where the second last morphism is given by Lemma \ref{Lemma: Leray spectral sequence for the automorphic sheaf}. We arrive at the \textbf{\textit{overconvergent Eichler--Shimura morphism}}
\begin{align*}
    \ES_{\kappa_{\calU}}:\OC_{\kappa_{\calU}, \C_p}^{r}\rightarrow M_{\Iw^+, w}^{\kappa_{\calU}+g+1}(-n_0).
\end{align*} 

\begin{Proposition}\label{Proposition: OES for sheaves on the pro-Kummer etale site}
The overconvergent Eichler--Shimura morphism\[
    \ES_{\kappa_{\calU}}: \OC_{\kappa_{\calU}, \C_p}^{r} \rightarrow M_{\Iw^+, w}^{\kappa_{\calU+g+1}}(-n_0)
\] is Hecke- and $G_{\Q_p}$-equivariant. 
\end{Proposition}
\begin{proof}
The Galois-equivariance follows immediately from Lemma \ref{Lemma: Leray spectral sequence for the automorphic sheaf}. For Hecke operators away from $Np$, notice that the operators $T_{\bfgamma}$'s on both sides are defined in the same way using correspondences. Hence, it is straightforward to verify the $T_{\bfgamma}$-equivariances. It remains to check the $U_{p,i}$-equivariance for all $i=1, ..., g$.

To this end, due to the $\Iw_{\GSp_{2g}}^+$-equivariance of $\eta_{\kappa_{\calU}}$, we only have to check the $\bfu_{p,i}$-equivariance. Indeed, for every $\bfgamma'=\bfgamma_0'\bfbeta_0'\in \Iw_{\GL_g}$ with $\bfgamma'_0\in U^{\opp}_{\GL_g,1}$ and $\bfbeta'_0\in B_{\GL_g,0}$, we have  
\begin{align*}
    (\bfu_{p,i}^*\delta) f_{\bfu_{p,i}\cdot\mu, \frakz}(\bfgamma') & = (\bfu_{p,i}^*\delta)\left(\kappa_{\calU}(\bfbeta_0')\int_{\T_0}e_{\kappa_{\calU}}^{\hst}(\trans\bfgamma'_0(\bfgamma+\frakz\bfupsilon))\quad d\bfu_{p,i}\cdot\mu\right) \\
    & = (\bfu_{p,i}^*\delta)\left(\kappa_{\calU}(\bfbeta_0')\int_{\T_0} \kappa_{\calU}(\bfbeta)e_{\kappa_{\calU}}^{\hst}\left(\trans\bfgamma_0'(\bfgamma_0+\frakz\bfupsilon_0)\right)\quad d\bfu_{p,i}\cdot\mu\right)  \\
    & = (\bfu_{p,i}^*\delta) \left(\kappa_{\calU}(\bfbeta_0')\int_{\T_0}\kappa_{\calU}(\bfbeta)e_{\kappa_{\calU}}^{\hst}\left(\trans\bfgamma_0'(\bfu_{p,i}^{\square}\bfgamma_0\bfu_{p,i}^{\square,-1}+\frakz\bfu_{p,i}^{\blacksquare}\bfupsilon_0\bfu_{p,i}^{\square, -1})\right)\quad d\mu\right) \\
    & = (\bfu_{p,i}^*\delta)\left(\kappa_{\calU}(\bfbeta_0')\int_{\T_0}\kappa_{\calU}(\bfbeta)e_{\kappa_{\calU}}^{\hst}\left(\trans\bfgamma_0'\bfu_{p,i}^{\square}(\bfgamma_0+\bfu_{p,i}^{\square, -1}\frakz\bfu_{p,i}^{\blacksquare}\bfupsilon_0)\bfu_{p,i}^{\square, -1}\right)\quad d\mu\right) \\
    & = (\bfu_{p,i}^*\delta)\left(\kappa_{\calU}(\bfbeta_0')\int_{\T_0}\kappa_{\calU}(\bfbeta) e_{\kappa_{\calU}}^{\hst}\left(\trans\bfgamma_0'\bfu_{p,i}^{\square}(\bfgamma_0+(\frakz\cdot\bfu_{p,i})\bfupsilon_0)\bfu_{p,i}^{\square, -1}\right)\quad d\mu\right) \\
    & = (\bfu_{p,i}^*\delta)\left(\kappa_{\calU}(\bfbeta_0')\int_{\T_0} \kappa_{\calU}(\bfbeta) e_{\kappa_{\calU}}^{\hst}\left(\bfu_{p,i}^{\square, -1}\trans\bfgamma'_0\bfu_{p, i}^{\square}(\bfgamma_0+(\frakz\cdot\bfu_{p,i})\bfupsilon_0)\right)\quad d\mu \right) \\
    & = (\bfu_{p,i}^*\delta)\left(\kappa_{\calU}(\bfbeta_0')\int_{\T_0}\kappa_{\calU}(\bfbeta) e_{\kappa_{\calU}}^{\hst}\left(\trans(\bfu_{p,i}^{\square}\bfgamma_0'\bfu_{p,i}^{\square, -1})(\bfgamma_0+(\frakz\cdot\bfu_{p,i})\bfupsilon_0)\right)\quad d\mu\right) \\
    & = \bfu_{p,i} . (\delta f_{\mu, \frakz}), 
\end{align*} 
where we have written $(\bfgamma, \bfupsilon)=(\bfgamma_0, \bfupsilon_0)\bfbeta$ for $(\bfgamma_0, \bfupsilon_0)\in \T_{00}$ and $\bfbeta\in B_{\GL_g,0}$.
The antepenultimate equation follows from the property of matrix determinants. 
\end{proof}

\begin{Remark}\label{Remark: OES for cuspforms}
\normalfont There is an analogue for compactly supported cohomology groups and overconvergent cuspforms. Let $r$, $w$, and $(R_{\calU}, \kappa_{\calU})$ be the same as before. On one hand, consider
$$\sheafOD_{\kappa_{\calU}}^{r, \cusp}:=\left(\varprojlim_{j}\left(\nu^{-1}\jmath_{\ket, !}\scrD_{\kappa_{\calU}, j}^{r, \circ}\otimes_{\Z_p}\scrO_{\overline{\calX}_{\Iw^+, \proket}}^+\right)\right)[\frac{1}{p}].$$
Since \[H_{\ket}^{n_0}(\overline{\calX}_{\Iw^+}, \jmath_{\ket, !}\scrD_{\kappa_{\calU}, j}^{r, \circ})=H^{n_0}_{\et, c}(\calX_{\Iw^+}, \scrD_{\kappa_{\calU}, j}^{r, \circ}),\] an analogue of Proposition \ref{Proposition: overconvergent cohomology computed by the pro-Kummer etale cohomology} implies that $\sheafOD_{\kappa_{\calU}}^{r, \cusp}$ computes \[
    \OC_{\kappa_{\calU}, \C_p}^{r, c}:=\left(\varprojlim_{j}H_{\et, c}^{n_0}(\calX_{\Iw^+}, \scrD_{\kappa_{\calU}, j}^{r, \circ})\otimes_{\Z_p}\calO_{\C_p}\right)[\frac{1}{p}].
\]

On the other hand, recall the sheaf $\underline{\omega}_{w, \cusp}^{\kappa_{\calU}}$ of $w$-overconvergent Siegel cuspforms of weight $\kappa_{\calU}$ and consider the $p$-adically completed pullback $\widehat{\underline{\omega}}_{w, \cusp}^{\kappa_{\calU}}$ to the pro-Kummer \'etale site. Repeating the construction above, we obtain a morphism $\eta_{\kappa_{\calU}}^{\cusp}: \sheafOD_{\kappa_{\calU}}^{r, \cusp}\rightarrow \widehat{\underline{\omega}}_{w, \cusp}^{\kappa_{\calU}}$ which induces a morphism
\[\ES_{\kappa_{\calU}}^{\cusp}:\OC_{\kappa_{\calU}, \C_p}^{r, c}\rightarrow H^0(\overline{\calX}_{\Iw^+, w}, \underline{\omega}_{w, \cusp}^{\kappa_{\calU}+g+1})(-n_0)
\]
rendering the following Galois- and Hecke-equivariant diagram commutative:
$$\begin{tikzcd}
\OC_{\kappa_{\calU}, \C_p}^{r}\arrow[r, "\ES_{\kappa_{\calU}}"] & H^0(\overline{\calX}_{\Iw^+, w}, \underline{\omega}_{w}^{\kappa_{\calU}+g+1})(-n_0)\\
\OC_{\kappa_{\calU}, \C_p}^{r, c}\arrow[r, "\ES_{\kappa_{\calU}}^{\cusp}"]\arrow[u] & H^0(\overline{\calX}_{\Iw^+, w}, \underline{\omega}_{w, \cusp}^{\kappa_{\calU}+g+1})(-n_0)\arrow[u, hook]
\end{tikzcd},$$ where the vertical arrow on the left is the natural map from the compactly supported cohomology group to the usual cohomology group. Let \[\OC_{\kappa_{\calU}, \C_p}^{r, \cusp}:=\image\left(\OC_{\kappa_{\calU}, \C_p}^{r, c}\rightarrow \OC_{\kappa_{\calU}, \C_p}^{r}\right).\] We arrive at the \textbf{\textit{overconvergent Eichler--Shimura morphism for overconvergent Siegel cuspforms}} $$\ES_{\kappa_{\calU}}^{\cusp}: \OC_{\kappa_{\calU}, \C_p}^{r, \cusp}\rightarrow S_{\Iw^+, w}^{\kappa_{\calU}+g+1}(-n_0),$$ where
\[S_{\Iw^+, w}^{\kappa_{\calU}+g+1}:=H^0(\overline{\calX}_{\Iw^+, w}, \underline{\omega}_{w, \cusp}^{\kappa_{\calU}+g+1})\]
is the space of $w$-overconvergent Siegel cuspforms of strict Iwahori level and weight $\kappa_{\calU}+g+1$.

Lastly, we point out that, by construction, both $\ES_{\kappa_{\calU}}^{\cusp}$ and $\ES_{\kappa_{\calU}}$ are functorial in the small weights $(R_{\calU}, \kappa_{\calU})$.
\end{Remark}


\subsection{The image of the overconvergent Eichler--Shimura morphism at classical weights}\label{subsection:imageOES}

The aim of this last part of the section is to describe the image of the overconvergent Eichler--Shimura morphism at classical algebraic weights. Let $k=(k_1,\ldots, k_g) \in \Z_{\geq 0}^g$ be a dominant weight. Note that the character group of $T_{\GSp_{2g}}$ is isomorphic to $\Z^{g+1}$ through $$\Z^{g+1}\times T_{\GSp_{2g}}\rightarrow \bbG_m, \quad ((k_1, ..., k_g; k_0), \diag(\bftau_1, ..., \bftau_g, \bftau_0\bftau_g^{-1}, ..., \bftau_0\bftau_1^{-1}))\mapsto \prod_{i=0}^{g}\bftau_i^{k_i}.$$ We view $k=(k_1,..., k_g)\in \Z^g$ as a character of $T_{\GSp_{2g}}$ via $$\Z^g\hookrightarrow\Z^{g+1}, \quad (k_1, ..., k_g)\mapsto (k_1, ..., k_g;0).$$

We first introduce some algebraic representations. Consider the algebraic representation $$\V_{\GSp_{2g}, k}^{\alg}:=\left\{\phi: \GSp_{2g}\rightarrow \bbA^1: \begin{array}{l}
    \phi\text{ is a morphism of algebraic varieties over $\Q_p$ such that}  \\
    \phi(\bfgamma\bfbeta)=k(\bfbeta)\phi(\bfgamma),\,\,\forall \bfgamma\in \GSp_{2g} , \bfbeta\in B_{\GSp_{2g}}  
\end{array}\right\}$$ 
equipped with a left $\GSp_{2g}$-action given by
$$(\bfgamma\cdot\phi)(\bfgamma') = \phi(\trans\bfgamma\bfgamma')$$ for any $\bfgamma, \bfgamma'\in \GSp_{2g}$ and $\phi\in \V_{\GSp_{2g}, k}^{\alg}$. 

Let $\V_{\GSp_{2g}, k}^{\alg, \vee} $ be the linear dual of $\V_{\GSp_{2g}, k}^{\alg}$. We equip $\V_{\GSp_{2g}, k}^{\alg, \vee}$ with a left $\GSp_{2g}$-action induced from the following right $\GSp_{2g}$-action on $\V_{\GSp_{2g}, k}^{\alg}$:
$$(\phi\cdot\bfgamma)(\bfgamma') = \phi(\bfgamma\bfgamma')$$ for any $\bfgamma, \bfgamma'\in \GSp_{2g}$ and $\phi\in \V_{\GSp_{2g}, k}^{\alg}$. Observe that there is a natural morphism of $\GSp_{2g}$-representations 
$$
    \beta:\V_{\GSp_{2g}, k}^{\alg, \vee} \rightarrow \V_{\GSp_{2g}, k}^{\alg}, \quad \mu \mapsto \left(\bfgamma \mapsto \int_{\bfalpha\in \GSp_{2g}}e_{k}^{\hst}(\trans\bfgamma \bfalpha)\quad d\mu\right),
$$where $e_k^{\hst}$ is the ``highest weight vector'' inside $\V_{\GSp_{2g}, k}^{\alg}$; namely, for any matrix $X=(X_{ij})_{1\leq i,j\leq 2g}\in \GSp_{2g}$, we define
$$e^{\hst}_k(X)=X_{11}^{k_1-k_2}\times \det((X_{ij})_{1\leq i,j\leq 2})^{k_2-k_3}\times\cdots \times\det((X_{ij})_{1\leq i,j\leq g})^{k_g}.$$

Secondly, we consider the cohomology groups induced by these algebraic representations. Notice that the left $\GSp_{2g}$-actions on $\V_{\GSp_{2g}, k}^{\alg}$ and $\V_{\GSp_{2g}, k}^{\alg, \vee}$ induce \'etale $\Q_p$-local systems on $\calX_{\Iw^+}$ which we still denote by the same symbols. In particular, we can consider the cohomology group $H_{\et}^{n_0}(\calX_{\Iw^+}, \V_{\GSp_{2g}, k}^{\alg, \vee})$. Similar to \S \ref{subsection: OES}, we introduce the sheaves $\sheafOV_k$ and $\sheafOV_k^{\vee}$ on $\overline{\calX}_{\Iw^+, \proket}$ defined by
$$
 \sheafOV_{k} := \nu^{-1}j_{\ket, *}\V_{\GSp_{2g}, k}^{\alg}\otimes_{\Q_p} \widehat{\scrO}_{\overline{\calX}_{\Iw^+, \proket}}
$$
and
$$
 \sheafOV_{k}^{\vee} := \nu^{-1}j_{\ket, *}\V_{\GSp_{2g}, k}^{\alg, \vee}\otimes_{\Q_p} \widehat{\scrO}_{\overline{\calX}_{\Iw^+, \proket}}.
$$
By the same argument as in Proposition \ref{Proposition: overconvergent cohomology computed by the pro-Kummer etale cohomology}, we obtain a natural identification
$$
H_{\et}^{n_0}(\calX_{\Iw^+}, \V_{\GSp_{2g}, k}^{\alg, \vee})\otimes_{\Q_p}\C_p \cong H_{\proket}^{n_0}(\overline{\calX}_{\Iw^+}, \sheafOV_k^{\vee}).
$$
Moreover, if $\calV = \varprojlim_{n}\calV_n \rightarrow \overline{\calX}_{\Iw^+}$ is a pro-Kummer \'etale presentation of a log affinoid perfectoid object in $\overline{\calX}_{\Iw^+, \proket}$ and let $\calV_{\infty} := \calV \times_{\overline{\calX}_{\Iw^+}}\overline{\calX}_{\Gamma(p^{\infty})}$, then, following the same argument as in the proof of Lemma \ref{Lemma: Explicit description of the sheaf OD}, we obtain identifications 
$$
\sheafOV_{k}(\calV) = \left(\V_{\GSp_{2g}, k}^{\alg}\otimes_{\Q_p} \widehat{\scrO}_{\overline{\calX}_{\Iw^+, \proket}}(\calV_{\infty})\right)^{\Iw_{\GSp_{2g}}^+}
$$
and
$$
\sheafOV_{k}^{\vee}(\calV) = \left(\V_{\GSp_{2g}, k}^{\alg, \vee}\otimes_{\Q_p} \widehat{\scrO}_{\overline{\calX}_{\Iw^+, \proket}}(\calV_{\infty})\right)^{\Iw_{\GSp_{2g}}^+}.
$$

\begin{Remark}\label{Remark: Hecke operators for alg. rep.}
\normalfont There are naturally defined Hecke operators on $H_{\et}^{n_0}(\calX_{\Iw^+}, \V_{\GSp_{2g}, k}^{\alg, \vee})$. More precisely, similar to Proposition \ref{Proposition: Comparison theorem of cohomologies}, we have \[
    H_{\et}^{n_0}(\calX_{\Iw^+}, \V_{\GSp_{2g}, k}^{\alg, \vee}) \cong H^{n_0}(X_{\Iw^+}(\C), \V_{\GSp_{2g}, k}^{\alg, \vee}),
\] where the right-hand side is the Betti cohomology of $X_{\Iw^+}(\C)$ with coefficients in $\V_{\GSp_{2g}, k}^{\alg, \vee}$. Hence, it suffices to define the Hecke operators acting on $H^{n_0}(X_{\Iw^+}(\C), \V_{\GSp_{2g}, k}^{\alg, \vee})$. They are defined as follows. \begin{enumerate}
    \item[$\bullet$] For any Hecke operator $T_{\bfgamma}$ away from $Np$, its action on $H^{n_0}(X_{\Iw^+}(\C), \V_{\GSp_{2g}, k}^{\alg, \vee})$ is defined by the same formula as (\ref{eq: Hecke on coh. away from Np}).
    \item[$\bullet$] For the $U_{p,i}$-action, let $\bfu_{p,i}$ act on $\GSp_{2g}(\Q_p)$ via conjugation \[
        \bfu_{p,i} . \bfgamma = \bfu_{p,i} \bfgamma \bfu_{p,i}^{-1}.
    \] Observe that if $\bfgamma\in B_{\GSp_{2g}}(\Q_p)$, then $\bfu_{p,i} . \bfgamma \in B_{\GSp_{2g}}(\Q_p)$ and the diagonal entries of $\bfgamma$ coincide with the diagonal entries of $\bfu_{p,i} . \bfgamma$. This action then induces a left $\bfu_{p,i}$-action on $\V_{\GSp_{2g}, k}^{\alg, \vee}$. The operator $U_{p,i}$ acting on $H^{n_0}(X_{\Iw^+}(\C), \V_{\GSp_{2g}, k}^{\alg, \vee})$ is defined by the same formula as (\ref{eq: Hecek on coh. at p}). 
\end{enumerate} 
\end{Remark}

We also consider the $p$-adically completed automorphic sheaf $\widehat{\underline{\omega}}_{\Iw^+}^{k}$ on $\overline{\calX}_{\Iw^+, \proket}$ defined by
$$
        \widehat{\underline{\omega}}_{\Iw^+}^{k} := \varprojlim_m\left(\underline{\omega}_{\Iw^+}^{k,+}\otimes_{\scrO^+_{\overline{\calX}_{\Iw^+}}}\scrO^+_{\overline{\calX}_{\Iw^+, \proket}}/p^m\right)[\frac{1}{p}]
$$
where $\underline{\omega}_{\Iw^+}^{k,+}$ is defined in Remark \ref{Remark: integral classical sheaf}. It follows from Proposition \ref{Proposition: explicit description of classical modular sheaf} that 
$$
        \widehat{\underline{\omega}}_{\Iw^+}^{k}(\calV) = \left\{f \in P_k(\GL_g, \widehat{\scrO}_{\overline{\calX}_{\Iw^+, w, \proket}}(\calV_{\infty})) :  \bfgamma^* f = \rho_{k}(\bfgamma_a+\frakz\bfgamma_c)^{-1}f, \,\,\,\forall \bfgamma = \begin{pmatrix}\bfgamma_a & \bfgamma_b\\ \bfgamma_c & \bfgamma_d\end{pmatrix} \in \Iw_{\GSp_{2g}}^+ 
        \right\}.
$$
for any log affinoid perfectoid object $\calV\in \overline{\calX}_{\Iw^+, \proket}$ and $\calV_{\infty} = \calV \times_{\overline{\calX}_{\Iw^+}}\overline{\calX}_{\Gamma(p^{\infty})}$.

\paragraph{The algebraic Eichler-Shimura morphism}
Recall the Hodge--Tate morphism \[
    \HT_{\Gamma(p^{\infty})}: V_p \rightarrow \underline{\omega}_{\Gamma(p^{\infty})}
\] from \S \ref{subsection: perfectoid Siegel modular variety}. It follows from the definition that 
$$
        \underline{\omega}_{\Iw^+}^k = (\Sym^{k_1-k_2}\underline{\omega}_{\Iw^+})\otimes_{\scrO_{\overline{\calX}_{\Iw^+}}} (\Sym^{k_2-k_3}\wedge^2 \underline{\omega}_{\Iw^+}) \otimes_{\scrO_{\overline{\calX}_{\Iw^+}}} \cdots \otimes_{\scrO_{\overline{\calX}_{\Iw^+}}} (\Sym^{k_g}\det \underline{\omega}_{\Iw^+})
$$
and hence
$$
        \underline{\omega}_{\Gamma(p^{\infty})}^k = (\Sym^{k_1-k_2}\underline{\omega}_{\Gamma(p^{\infty})})\otimes_{\scrO_{\overline{\calX}_{\Gamma(p^{\infty})}}} (\Sym^{k_2-k_3}\wedge^2 \underline{\omega}_{\Gamma(p^{\infty})}) \otimes_{\scrO_{\overline{\calX}_{\Gamma(p^{\infty})}}} \cdots \otimes_{\scrO_{\overline{\calX}_{\Gamma(p^{\infty})}}} (\Sym^{k_g}\det \underline{\omega}_{\Gamma(p^{\infty})}).
$$

Let $V_{\mathrm{std}}$ denote the standard representation of $\GSp_{2g}$ over $\Q_p$, with standard basis $x_1, \ldots, x_{2g}$. There is an isomorphism of $\GSp_{2g}(\Q_p)$-representations $V_{\mathrm{std}}\simeq V_{\Q_p}:=V_p\otimes_{\Z_p}\Q_p$ sending $x_i$ to $e_{2g+1-i}$, for $i=1, \ldots, g$, and sending $x_i$ to $-e_{2g+1-i}$, for $i=g+1, \ldots, 2g$. If we write
$$
 V_{\mathrm{std}}^{k}:=(\Sym^{k_1-k_2}V_{\mathrm{std}}) \otimes_{\Q_p} (\Sym^{k_2-k_3}(\wedge^2 V_{\mathrm{std}}))\otimes_{\Q_p} \cdots \otimes_{\Q_p} (\Sym^{k_g}(\wedge^gV_{\mathrm{std}}))
$$
and
$$
 V_{\Q_p}^{k}:=(\Sym^{k_1-k_2}V_{\Q_p}) \otimes_{\Q_p} (\Sym^{k_2-k_3}(\wedge^2 V_{\Q_p}))\otimes_{\Q_p} \cdots \otimes_{\Q_p} (\Sym^{k_g}(\wedge^gV_{\Q_p})),
$$
the Hodge-Tate map induces a map $V_{\mathrm{std}}^{k}\simeq V_{\Q_p}^{k}\rightarrow \underline{\omega}_{\Iw^+}^k$. Moreover, it is well-known that $\V_{\GSp_{2g}, k}^{\alg}$ is an irreducible $\GSp_{2g}$-subrepresentation of $V_{\mathrm{std}}^{k}$ (see for example \cite[Lecture 17]{Fulton-Harris}). In particular, the highest weight vector $e^{\hst}_k$ in $\V_{\GSp_{2g}, k}^{\alg}$ corresponds to the element
$$
x_1^{k_1-k_2}\otimes (x_1\wedge x_2)^{k_2-k_3}\otimes\cdots \otimes (x_1\wedge \cdots \wedge x_g)^{k_g}
$$
in $V_{\mathrm{std}}^k$.

The composition
$$\V_{\GSp_{2g}, k}^{\alg, \vee}\xrightarrow[]{\beta}\V_{\GSp_{2g}, k}^{\alg}\hookrightarrow V_{\mathrm{std}}^k\simeq V_{\Q_p}^{k}\rightarrow  \underline{\omega}_{\Iw^+}^k$$ then induces a map \[
    \eta_{k}^{\alg}: \sheafOV_k^{\vee} \rightarrow \widehat{\underline{\omega}}_{\Iw^+}^k.
\] Eventually, we arrive at the \textbf{\textit{algebraic Eichler-Shimura morphism of weight $k$}} \begin{align*}
    \ES_k^{\alg}: H_{\et}^{n_0}(\calX_{\Iw^+}, \V_{\GSp_{2g}, k}^{\alg, \vee}) \otimes_{\Q_p}\C_p \simeq H_{\proket}^{n_0}(\overline{\calX}_{\Iw^+}, \sheafOV_k^{\vee}) \xrightarrow{\eta_k^{\alg}} H_{\proket}^{n_0}(\overline{\calX}_{\Iw^+}, \widehat{\underline{\omega}}_{\Iw^+}^k) \rightarrow H^0(\overline{\calX}_{\Iw^+}, \underline{\omega}_{\Iw^+}^{k+g+1})(-n_0),
\end{align*} where the last map follows from the same argument as in the proof of Lemma \ref{Lemma: Leray spectral sequence for the automorphic sheaf}. We remark that $\ES_k^{\alg}$ coincides with the one induced from \cite[Theorem VI. 6.2]{Faltings-Chai}. It is Hecke- and $G_{\Q_p}$-equivariant, and also surjective. 

In fact, over the $w$-ordinary locus $\overline{\calX}_{\Iw^+, w}$, the map $\eta_{k}^{\alg}$ has the following explicit description. 

\begin{Lemma}
\begin{enumerate}
\item[(i)] Let $\calV = \varprojlim_n \calV_n$ be a pro-Kummer \'etale presentation of a log affinoid perfectoid object in $\overline{\calX}_{\Iw^+, w, \proket}$ and let $\calV_{\infty} = \calV\times_{\overline{\calX}_{\Iw^+, w}}\overline{\calX}_{\Gamma(p^{\infty}), w}$. There is a well-defined $\GSp_{2g}(\Q_p)$-equivariant map
$$
\widetilde{\eta}_k^{\alg}: \V_{\GSp_{2g}, k}^{\alg, \vee} \otimes_{\Q_p} \widehat{\scrO}_{\overline{\calX}_{\Iw^+, w, \proket}}(\calV_{\infty})\rightarrow P_k(\GL_g, \widehat{\scrO}_{\overline{\calX}_{\Iw^+, w, \proket}}(\calV_{\infty}))
$$
defined by $\mu\otimes\delta \mapsto \delta f_{\mu, \frakz}^{\alg}$ where 
$$ f_{\mu, \frakz}^{\alg}(\bfgamma')=\int_{\bfalpha\in \GSp_{2g}} e_{k}^{\hst}\left(\begin{pmatrix}\trans\bfgamma' & \\ & \oneanti_g \bfgamma'^{-1}\oneanti_g\end{pmatrix}\begin{pmatrix}\one_g & \frakz\\ & \one_g\end{pmatrix} \bfalpha\right)\quad d\mu.$$
Here, the $\GSp_{2g}(\Q_p)$-action on the right hand side is given by
$$
\bfgamma.f:=\rho_k(\bfgamma_a+\frakz\bfgamma_c)(\bfgamma^*f)
$$
for every $\bfgamma=\begin{pmatrix} \bfgamma_a & \bfgamma_b\\ \bfgamma_c & \bfgamma_d\end{pmatrix}\in \GSp_{2g}(\Q_p)$ and $f\in P_k(\GL_g, \widehat{\scrO}_{\overline{\calX}_{\Iw^+, w, \proket}}(\calV_{\infty}))$.
\item[(ii)] The map $\eta_k^{\alg}$ is obtained from $\widetilde{\eta}^{\alg}_k$ by taking $\Iw^+_{\GSp_{2g}}$-invariants on both sides.
\end{enumerate}
\end{Lemma}

\begin{proof}
\begin{enumerate}
\item[(i)] Notice that $\widetilde{\eta}^{\alg}_k$ is the composition of $\beta$ with the map 
$$
\xi_k^{\alg}: \V_{\GSp_{2g}, k}^{\alg} \otimes_{\Q_p} \widehat{\scrO}_{\overline{\calX}_{\Iw^+, w, \proket}}(\calV_{\infty})\rightarrow P_k(\GL_g, \widehat{\scrO}_{\overline{\calX}_{\Iw^+, w, \proket}}(\calV_{\infty}))
$$
defined by $\phi\otimes \delta \mapsto \delta g_{\phi,\frakz}$
where $$g_{\phi,\frakz}(\bfgamma')=\phi\left(\begin{pmatrix}\one_g& \\ \trans\frakz&\one_g\end{pmatrix}
\begin{pmatrix}\bfgamma' & \\ & \oneanti_g \trans(\bfgamma')^{-1}\oneanti_g\end{pmatrix}\right)$$
for all $\bfgamma'\in \GL_g(\C_p)$.

Recall that $\beta$ is $\GSp_{2g}(\Q_p)$-equivariant. It remains to check that $\xi^{\alg}_k$ is $\GSp_{2g}(\Q_p)$-equivariant, which follows from a straightforward calculation.

\item[(ii)] It suffices to check that the $\Iw^+_{\GSp_{2g}}$-invariance of $\xi^{\alg}_k$ coincides with the map induced from the composition $\V_{\GSp_{2g}, k}^{\alg}\hookrightarrow V_{\mathrm{std}}^k\simeq V_{\Q_p}^{k}\rightarrow  \underline{\omega}_{\Iw^+}^k$. Notice that $\V_{\GSp_{2g}, k}^{\alg}$ is spanned by $\GSp_{2g}$-translations of the highest weight vector $e^{\hst}_k$. Therefore, we only need to check that $\xi^{\alg}_k(e^{\hst}_k\otimes 1)$ gives the correct element in $\widehat{\underline{\omega}}_{\Iw^+}^k$.

Indeed, since the Hodge--Tate map $V_p\rightarrow \underline{\omega}_{\Iw^+}$ sends $e_{2g+1-i}$ to $\fraks_i$, for $i=1, \ldots, g$, we see that the composition $\V_{\GSp_{2g}, k}^{\alg}\hookrightarrow V_{\mathrm{std}}^k\simeq V_{\Q_p}^{k}\rightarrow  \underline{\omega}_{\Iw^+}^k$ sends the highest weight vector $e^{\hst}_k$ to 
$$\fraks_1^{k_1-k_2}\otimes (\fraks_1\wedge \fraks_2)^{k_2-k_3}\otimes \cdots\otimes (\fraks_1\wedge\cdots\wedge \fraks_g)^{k_g}.$$

On the other hand, notice that the element $\fraks_1\wedge\cdots \wedge \fraks_t$ corresponds to the function $X = (X_{ij})_{1\leq i,j\leq g}\mapsto \det((X_{ij})_{1\leq i,j\leq t})$ in $P_k(\GL_g, \widehat{\scrO}_{\overline{\calX}_{\Iw^+, w, \proket}}(\calV_{\infty}))$. Therefore, $e^{\hst}_k$ is sent to the function
$$X \mapsto X_{11}^{k_1-k_2}\times \det((X_{ij})_{1\leq i,j\leq 2})^{k_2-k_3}\times\cdots \times\det((X_{ij})_{1\leq i,j\leq g})^{k_g}$$
in $P_k(\GL_g, \widehat{\scrO}_{\overline{\calX}_{\Iw^+, w, \proket}}(\calV_{\infty}))$. This element coincides with $\xi^{\alg}_k(e^{\hst}_k\otimes 1)$, as desired.
\end{enumerate}
\end{proof}

Recall the natural inclusion $M^{k, \mathrm{cl}}_{\Iw^+}=H^0(\overline{\calX}_{\Iw^+}, \underline{\omega}_{\Iw^+}^k)\hookrightarrow H^0(\overline{\calX}_{\Iw^+, w}, \underline{\omega}_w^k) = M_{\Iw^+, w}^{k}$ from Lemma \ref{Lemma: injection of classical forms}. The main result of this subsection is the following.

\begin{Theorem}\label{thm:imageclassicalweight}
Let $k = (k_1, ..., k_g) \in \Z_{\geq 0}^g$ be a dominant weight. Then the image of
$$\ES_{k}: \OC^r_{k, \C_p}  \longrightarrow M^{k+g+1}_{\Iw^+, w}(-n_0)$$
is contained in the space of the classical forms $M^{k+g+1, \cl}_{\Iw^+}(-n_0)$.
\end{Theorem}
\begin{proof}
Firstly, there is a natural map \[
\V_{\GSp_{2g}, k}^{\alg} \rightarrow A^{r}_{k}(\T_0, \Q_p)
\]
induced by the inclusion \[
    \T_0 \rightarrow \GSp_{2g}(\Q_p), \quad (\bfgamma, \bfupsilon)\mapsto \begin{pmatrix}\bfgamma & \\ \bfupsilon & \oneanti_g \trans\bfgamma^{-1}\oneanti_g\end{pmatrix}.
\] The dual of this map gives
\[
D_k^r(\T_0, \Q_p)\rightarrow \V^{\alg, \vee}_{\GSp_{2g},k}
\]
which then induces a map of sheaves \[
    \sheafOD_{k}^r \rightarrow \sheafOV_{k}^{\vee}
\] over $\overline{\calX}_{\Iw^+, w, \proket}$. Hence, the theorem follows once we show that the following diagram commutes
$$
\begin{tikzcd}
H_{\proket}^{n_0}(\overline{\calX}_{\Iw^+}, \sheafOD_{k}^r)\arrow[d]\arrow[r, "\ES_{k}"]  & H^0(\overline{\calX}_{\Iw^+, w}, \underline{\omega}_w^{k+g+1})(-n_0)   \\
 H_{\proket}^{n_0}(\overline{\calX}_{\Iw^+}, \sheafOV_{k}^{\vee}) \arrow[r, "\ES_k^{\alg}"] & H^0(\overline{\calX}_{\Iw^+}, \underline{\omega}_{\Iw^+}^{ k+g+1})(-n_0) \arrow[u]
\end{tikzcd}.
$$

Over $\overline{\calX}_{\Iw^+, w, \proket}$, it follows from the construction that we have a commutative diagram \[
    \begin{tikzcd}
        \sheafOD_{k}^r\arrow[r, "\eta_k"]\arrow[d] & \widehat{\underline{\omega}}_w^{k}\\
        \sheafOV_k^{\vee}\arrow[r, "\eta_k^{\alg}"] & \widehat{\underline{\omega}}_{\Iw^+}^{k}\arrow[u, hook]
    \end{tikzcd},
\] where the inclusion on the right-hand side is given by the inclusion (\ref{eq: alg. sheaf into overconvergent sheaf}). Consequently, there is a commutative diagram on the cohomology groups \[
    \begin{tikzcd}
        H_{\proket}^{n_0}(\overline{\calX}_{\Iw^+}\sheafOD_{k}^r)\arrow[r]\arrow[d, "\mathrm{Res}"] \arrow[ddd, bend right = 80, "\ES_{k}"'] & H_{\proket}^{n_0}(\overline{\calX}_{\Iw^+}, \sheafOV_{k}^{\vee})\arrow[d, "\mathrm{Res}"] \arrow[dddr, bend left = 40, "\ES_k^{\alg}"]\\
        H_{\proket}^{n_0}(\overline{\calX}_{\Iw^+, w}, \sheafOD_{k}^r)\arrow[r]\arrow[d, "\eta_k"] & H_{\proket}^{n_0}(\overline{\calX}_{\Iw^+, w}, \sheafOV_k^{\vee})\arrow[d, "\eta_{k}^{\alg}"]\\
        H_{\proket}^{n_0}(\overline{\calX}_{\Iw^+, w}, \widehat{\underline{\omega}}_{w}^{k})\arrow[d] & H_{\proket}^{n_0}(\overline{\calX}_{\Iw^+, w}, \widehat{\underline{\omega}}_{\Iw^+}^{k})\arrow[d]\arrow[l, hook']\\
        H^0(\overline{\calX}_{\Iw^+, w}, \underline{\omega}_{w}^{k+g+1})(-n_0) & H^0(\overline{\calX}_{\Iw^+, w}, \underline{\omega}_{\Iw^+}^{k+g+1})(-n_0)\arrow[l, hook'] & H^0(\overline{\calX}_{\Iw^+}, \underline{\omega}_{\Iw^+}^{k})(-n_0)\arrow[l, hook', "\mathrm{Res}"]
    \end{tikzcd}.
\] This finishes the proof.
\end{proof}