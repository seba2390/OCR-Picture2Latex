\section{The overconvergent Eichler--Shimura morphism on the cuspidal eigenvariety }\label{section:oncuspidaleigenvariety}
In this section, we glue the overconvergent Eichler--Shimura morphism over a suitable eigenvariety $\calE_0$. We begin with some preliminaries on the overconvergent cohomology groups in \S \ref{subsection: preliminaries on overconvergent cohomology}. The eigenvariety $\calE_{0}$ is constructed in \S \ref{subsection: cuspidal eigenvariety}. Finally in \S \ref{subsection: sheaves on the cuspidal eigenvariety}, we show that the overconvergent Eichler--Shimura morphism spreads out over $\calE_0$.

Throughout this section, we assume $p>2g$ so that we can apply results in \cite{AIP-2015} via the comparison in \S \ref{subsection:comparison sheaf aip}. On the other hand, we believe that the results in this section hold for smaller primes as well. In order to deal with these smaller primes, one would have to reprove several results in \cite{AIP-2015} in our context; \emph{e.g.}, the classicality result and the fact that $S_{\Iw^+}^{\kappa_{\calU}}$ has property (Pr) in the sense of \cite{Buzzard_2007}. We decide to leave these generalities to the reader in order to keep this paper within a reasonable length.


\subsection{Some preliminaries on overconvergent cohomology groups}\label{subsection: preliminaries on overconvergent cohomology} The purpose of this subsection is to review the basic constructions and properties of the overconvergent cohomology groups needed in latter subsections. Most of the materials are recorded from \cite{Hansen-PhD, CHJ-2017}. We do not claim any originality here. 

\begin{Definition}\label{Definition: open weights}
\begin{enumerate}
    \item[(i)] Let $(R_{\calU}, \kappa_{\calU})$ be a small weight. We say it is \textbf{open} if the natural map \[
        \calU^{\rig} = \Spa(R_{\calU}, R_{\calU})^{\rig} \rightarrow \calW
    \] is an open immersion.
    \item[(ii)] Let $(R_{\calU}, \kappa_{\calU})$ be an affinoid weight. We say it is \textbf{open} if the natural map \[
        \calU^{\rig} = \calU = \Spa(R_{\calU}, R_{\calU}^{\circ}) \rightarrow \calW
    \] is an open immersion. 
    \item[(iii)] A weight $(R_{\calU}, \kappa_{\calU})$ is called an \textbf{open weight} if it is either an small open weight or an affinoid open weight. 
\end{enumerate}
\end{Definition}

Given an open weight $(R_{\calU}, \kappa_{\calU})$ and an integer $r>1+r_{\calU}$, one considers the so called \emph{Borel--Serre chain complex} $C_{\bullet}(\Iw_{\GSp_{2g}}^+, A_{\kappa_{\calU}}^{r}(\T_0, R_{\calU}))$ (resp., \emph{Borel--Serre cochain complex} $C^{\bullet}(\Iw_{\GSp_{2g}}^+, D_{\kappa_{\calU}}^{r}(\T_0, R_{\calU}))$) which computes the Betti homolgy groups $H_t(X_{\Iw^+}(\C), A_{\kappa_{\calU}}^r(\T_0, R_{\calU}))$ (resp., Betti cohomology groups $H^t(X_{\Iw^+}(\C), D_{\kappa_{\calU}}^r(\T_0, R_{\calU}))$). The Borel--Serre chain complex is a finite complex as it is constructed by a fixed triangulation on the Borel--Serre compactification of the locally symmetric space $X_{\Iw^+}(\C)$. We write \begin{align*}
    & C_{\tol}^{\kappa_{\calU}, r} := \bigoplus_{t}C_t(\Iw^+_{\GSp_{2g}}, A_{\kappa_{\calU}}^r(\T_0, R_{\calU})),\\
    & C_{\kappa_{\calU}, r}^{\tol} := \bigoplus_{t}C^t(\Iw_{\GSp_{2g}}^+, D_{\kappa_{\calU}}^r(\T_0, R_{\calU})).
\end{align*} Then $C^{\kappa_{\calU}, r}_{\tol}$ is an ON-able $R_{\calU}[1/p]$-module as $A_{\kappa_{\calU}}^r(\T_0, R_{\calU})$ is ON-able (see \cite[\S 2.2, Remarks]{Hansen-PhD}). Moreover, there are naturally defined Hecke operators on $C_{\tol}^{\kappa_{\calU}, r}$ and the action of $U_p$ is compact (see [\textit{op. cit.}, \S 2.2]). We define $F_{\kappa_{\calU}, r}^{\oc}\in R_{\calU}[1/p]\llbrack T\rrbrack$ to be the Fredholm determinant of $U_p$ acting on $C^{\kappa_{\calU}, r}_{\tol}$. Notice that, for any $h\in \Q_{\geq 0}$, the existence of a slope-$\leq h$ decomposition of $C_{\tol}^{\kappa_{\calU}, r}$ is equivalent to the existence of a slope-$\leq h$ factorisation of $F_{\kappa_{\calU},r}^{\oc}$. Moreover, by [\textit{op. cit.}, Proposition 3.1.2], if $C^{\kappa_{\calU},r}_{\tol, \leq h}$ is the slope-$\leq h$ submodule of $C^{\kappa_{\calU},r}_{\tol}$ and suppose $\calU'=(R_{\calU'}, \kappa_{\calU'})$ is another open weight such that $\calU'^{\rig} \subset \calU^{\rig}$, there is a canonical isomorphism \[
    C_{\tol, \leq h}^{\kappa_{\calU}, r}\otimes_{R_{\calU}[\frac{1}{p}]}R_{\calU'}[\frac{1}{p}] \cong C_{\tol, \leq h}^{\kappa_{\calU'}, r}.
\]

\begin{Definition}
Let $\calU=(R_{\calU}, \kappa_{\calU})$ be an open weight and let $h\in \Q_{\geq 0}$. The pair $(\calU, h)$ is called a \textbf{slope datum} if $F_{\kappa_{\calU},r}^{\oc}$ admits a slope-$\leq h$ factorisation.
\end{Definition}

\begin{Proposition}\label{Proposition: functoriality of BS chain complex under slope decomposition} Let $(\calU, h)$ be a slope datum and let $(R_{\calU'}, \kappa_{\calU'})$ be an affinoid open weight such that $\calU' \subset \calU^{\rig}$. 
\begin{enumerate}
    \item[(i)] There is a canonical isomorphism \[
        H_{t}(X_{\Iw^+}(\C), A_{\kappa_{\calU}}^r(\T_0, R_{\calU}))_{\leq h} \otimes_{R_{\calU}[\frac{1}{p}]} R_{\calU'} \cong H_{t}(X_{\Iw^+}(\C), A_{\kappa_{\calU'}}^r(\T_0, R_{\calU'}))_{\leq h}
    \] for all $t\in \Z$, where the subscript ``$\leq h$'' stands for the slope-$\leq h$ submodule.
    \item[(ii)] The cochain complex $C_{\kappa_{\calU}, r}^{\tol}$ and the cohomology groups $H^t(X_{\Iw^+}(\C), D_{\kappa_{\calU}}^r(\T_0, R_{\calU}))$ admit slope-$\leq h$ decompositions. The corresponding slope-$\leq h$ submodules are denoted by $C_{\kappa_{\calU}, r}^{\tol, \leq h}$ and $H^t(X_{\Iw^+}(\C), D_{\kappa_{\calU}}^r(\T_0, R_{\calU}))^{\leq h}$, respectively. 
    \item[(iii)] There are canonical isomorphisms$$
        C_{\kappa_{\calU}, r}^{\tol, \leq h}\otimes_{R_{\calU}[\frac{1}{p}]}R_{\calU'}\cong C_{\kappa_{\calU'}, r}^{\tol, \leq h}$$
        and 
        $$ H^t(X_{\Iw^+}(\C), D_{\kappa_{\calU}}^r(\T_0, R_{\calU}))^{\leq h}\otimes_{R_{\calU}[\frac{1}{p}]}R_{\calU'} \cong H^t(X_{\Iw^+}(\C), D_{\kappa_{\calU'}}^r(\T_0, R_{\calU'}))^{\leq h}.$$
\end{enumerate}
\end{Proposition}
\begin{proof}
The proof follows verbatim as in the proofs of \cite[Proposition 3.3 \& Proposition 3.4]{CHJ-2017}.
\end{proof}

Notice that, if we vary the open weights $\calU$, the Fredholm determinants glue into a power series $F_{\calW}^{\oc}\in \scrO_{\calW}(\calW)\{\{T\}\}$. Here we drop the subscript ``$r$'' because the Fredholm determinant does not depend on $r$ according to \cite[Proposition 3.1.1]{Hansen-PhD}. 


\subsection{The cuspidal eigenvariety}\label{subsection: cuspidal eigenvariety}
\paragraph{The spectral variety.}
In the previous subsection, we obtained the Fredholm determinant $F_{\calW}^{\oc}\in \scrO_{\calW}(\calW)\{\{T\}\}$. On the other hand, given an affinoid weight $(R_{\calU}, \kappa_{\calU})$ and $w>1+r_{\calU}$, by \cite[Proposition 8.1.3.1]{AIP-2015} and Theorem \ref{Theorem: comparison with AIP} (see also [\textit{op. cit.}, Proposition 8.2.3.3]), the space of cuspforms $S_{\Iw^+,w}^{\kappa_{\calU}}=H^0(\overline{\calX}_{\Iw^+, w}, \underline{\omega}_{w, \cusp}^{\kappa_{\calU}})$ has property (Pr) in the sense of \cite{Buzzard_2007}; namely, it is a direct summand of a potentially ON-able $\C_p\widehat{\otimes}R_{\calU}$-Banach space. Also recall that $U_p$ acts compactly on the space of overconvergent Siegel modular forms. Therefore, we can define the Fredholm determinant $F_{\kappa_{\calU}, w}^{\mf}$ of $U_p$ acting on $S_{\Iw^+, w}^{\kappa_{\calU}}$. When we vary the affinoid weights, the Fredholm determinants glue together. After further taking the inductive limit over $w$, we arrive at a power series $F_{\calW}^{\mf}\in \scrO_{\calW}(\calW)\{\{T\}\}\widehat{\otimes}_{\Q_p}\C_p$.

Define $$F_{\calW}:=F_{\calW}^{\mf}F_{\calW}^{\oc}\in \scrO_{\calW}(\calW)\{\{T\}\}\widehat{\otimes}_{\Q_p}\C_p,$$ which is still a Fredholm series. Let $\mathbb{A}_{\C_p}^1$ denote the adic affine line over $(\C_p, \calO_{\C_p})$ with coordinate function $T$ and let $\bbA_{\calW}^1:=\calW\times_{\Spa(\Q_p, \Z_p)}\mathbb{A}_{\C_p}^1$. The \textit{\textbf{spectral variety}} $\calS$ is defined to be the zero locus of $F_{\calW}$ in $\bbA_{\calW}^1$.

\paragraph{The eigenvarieties.} 

\begin{Definition}\label{Defintiion: slope data}
Let $\calU$ be an open weight so that $\calU^{\rig}\subset\calW$. Let $\calU^{\rig}_{\C_p}$ denote the base change of $\calU^{\rig}$ to $\Spa(\C_p, \calO_{\C_p})$ and consider the adic affine line $\bbA_{\calU}^1:=\calU^{\rig}\times_{\Spa(\Q_p, \Z_p)}\mathbb{A}_{\C_p}^1$ over $\calU^{\rig}_{\C_p}$. Moreover, for any $h\in \Q_{> 0}$, consider the closed ball $\mathbf{B}(0,p^h)$ of radius $p^h$ over $\C_p$ and define $\B_{\calU, h}:= \calU^{\rig} \times_{\Spa(\Q_p, \Z_p)}\mathbf{B}(0, p^h)$. Let $\calS_{\calU, h}:=\calS\cap \B_{\calU, h}$. We say that the pair $(\calU, h)$ is \textbf{slope-adapted} if  the natural map $\calS_{\calU, h}\rightarrow \calU^{\rig}_{\C_p}$ is finite flat.
\end{Definition}

Consider the collection $$\Cov(\calS)=\{\calS_{\calU, h}: (\calU, h)\text{ is slope-adapted}\}.$$ Let $\Cov_{\mathrm{aff}}(\calS)$ be a subcollection of $\Cov(\calS)$, consisting of those $\calS_{\calU, h}$ with $\calU$ being an affinoid weight. By \cite[Theorem 4.6]{Buzzard_2007} (see also \cite[Proposition 4.1.4]{Hansen-PhD}), $\Cov_{\mathrm{aff}}(\calS)$ forms an open covering for $\calS$ (and thus so is $\Cov(\calS)$). Using this covering, we define the following two coherent sheaves on $\calS$.

\begin{Definition}
\begin{enumerate}
\item[(i)] Recall from \S \ref{subsection:continuousfunctions} that $D_{\kappa_{\calU}}^{\dagger}(\T_0, R_{\calU})$ is defined to be the inverse limit of $D_{\kappa_{\calU}}^r(\T_0, R_{\calU})$ with respect to $r$. The coherent sheaf $\scrH_{\Par}^{\tol}$ on $\calS$ is defined by
$$\scrH_{\Par}^{\tol}(\calS_{\calU, h}):=\image\left(\bigoplus_{t}H_{c}^t(X_{\Iw^+}(\C), D_{\kappa_{\calU}}^{\dagger}(\T_0, R_{\calU}))^{\leq h}\rightarrow \bigoplus_{t}H^t(X_{\Iw^+}(\C), D_{\kappa_{\calU}}^{\dagger}(\T_0, R_{\calU}))^{\leq h}\right)\widehat{\otimes}_{\Q_p} \C_p$$
for all $\calS_{\calU, h}\in\Cov_{\mathrm{aff}}(\calS)$, where the map $$H_{c}^t(X_{\Iw^+}(\C), D_{\kappa_{\calU}}^{\dagger}(\T_0, R_{\calU}))^{\leq h}\rightarrow H^t(X_{\Iw^+}(\C), D_{\kappa_{\calU}}^{\dagger}(\T_0, R_{\calU}))^{\leq h}$$ is induced from the natural map from the compactly supported cohomology groups to the usual ones.

\item[(ii)] The coherent sheaf $\scrS_{\Iw^+}^{\dagger}$ on $\calS$ is defined by
$$\scrS_{\Iw^+}^{\dagger}(\calS_{\calU, h}):=S_{\Iw^+}^{ \kappa_{\calU}+g+1, \leq h}$$
for all $\calS_{\calU, h}\in\Cov_{\mathrm{aff}}(\calS)$, where the superscript ``$\leq h$'' stands for the slope-$\leq h$ part with respect to the $U_p$-operator.
\end{enumerate}
\end{Definition}

These are indeed well-defined coherent sheaves (see, for example, \cite[\S 4.3]{Hansen-PhD} and \cite[\S 8.1]{AIP-2015}, respectively). The Hecke algebra $\bbT$ acts on both of the coherent sheaves. The eigenvarieties we are interested in are the following.

\begin{Definition}
\begin{enumerate}
\item[(i)] For every $\calS_{\calU, h}\in \Cov_{\mathrm{aff}}(\calS)$, let $\bbT_{\calU, h}^{\oc}$ be the reduced $\scrO_{\calS_{\calU, h}}(\calS_{\calU, h})$-algebra generated by the image of $\bbT\rightarrow \End\left(\scrH_{\Par}^{\tol}(\calS_{\calU, h})\right)$. Let $\bbT_{\calU, h}^{\oc, \circ}$ be the integral closure of $\scrO_{\calS_{\calU, h}}(\calS_{\calU, h})^{\circ}$ inside $\bbT_{\calU, h}^{\oc}$. 
\item[(ii)] Let $\scrT_{\oc}$ be the coherent sheaf on $\calS$ defined by $\scrT_{\oc}(\calS_{\calU, h}):=\bbT_{\calU, h}^{\oc}$, and let $\scrT_{\oc}^{\circ}$ be the subsheaf of $\scrT_{\oc}$ defined by $\scrT_{\oc}^{\circ}(\calS_{\calU, h}):=\bbT_{\calU, h}^{\oc, \circ}$.
\item[(iii)] The \textbf{reduced cuspidal eigenvariety} $\calE_{0}^{\oc}$ is defined to be the relative adic space $\Spa_{\calS}(\scrT_{\oc}, \scrT_{\oc}^{\circ})$. 
\end{enumerate}
\end{Definition}

\begin{Definition}
\begin{enumerate}
\item[(i)] For every $\calS_{\calU, h}\in \Cov_{\mathrm{aff}}(\calS)$, let $\bbT_{\calU, h}^{\mf}$ be the reduced $\scrO_{\calS_{\calU, h}}(\calS_{\calU, h})$-algebra generated by the image of $\bbT\rightarrow \End\left(\scrS_{\Iw^+}^{\dagger}(\calS_{\calU, h})\right)$. Let $\bbT_{\calU, h}^{\mf,\circ}$ be the integal closure of $\scrO_{\calS_{\calU, h}}(\calS_{\calU, h})^{\circ}$ inside $\bbT_{\calU, h}^{\mf}$.
\item[(ii)] Let $\scrT_{\mf}$ be the coherent sheaf on $\calS$ defined by $\scrT_{\mf}(\calS_{\calU, h}):=\bbT_{\calU, h}^{\mf}$. Let $\scrT_{\mf}^{\circ}$ be the subsheaf of $\scrT_{\mf}$ defined by $\scrT_{\mf}^{\circ}(\calS_{\calU, h}):=\bbT_{\calU, h}^{\mf, \circ}$. 
\item[(iii)] The \textbf{equidimensional cuspidal eigenvariety} $\calE_0^{\mf}$ is defined to be the equidimensional locus of the  relative adic space $\Spa_{\calS}(\scrT_{\mf}, \scrT_{\mf}^{\circ})$.
\end{enumerate}
\end{Definition}

\begin{Remark}\label{Remark: eigenvarieties}
\normalfont Notice that $\calE_0^{\mf}$ is (the stricit Iwahori version of) the equidimensional cuspidal eigenvariety constructed in \cite{AIP-2015} after base change to $\C_p$. On the other hand, $\calE_{0}^{\oc}$ is the reduced cuspidal eigenvariety considered in \cite{Wu-2020} after base change to $\C_p$. We also point out that the cuspidal eigenvariety considered in \textit{op. cit.} is the cuspidal part of the eigenvariety for $\GSp_{2g}$ constructed in \cite{Johansson-Newton} (see also \cite{Hansen-PhD}).
\end{Remark}

\begin{Proposition}\label{Proposition: comparison of eigenvarieties}
There is a natural closed immersion $\calE_{0}^{\mf}\hookrightarrow \calE_{0}^{\oc}$.
\end{Proposition}
\begin{proof}
The strategy is to apply \cite[Theorem 5.1.2]{Hansen-PhD}. To this end, we need to find a \textit{very Zariski-dense} subset $\calS^{\cl}$ of $\calS$ such that for every $\bfitx\in \calS^{\cl}$ with dominate algebraic weight $k=(k_1, ..., k_g)\in \Z_{\geq 0}^{g}$ and any $Y\in \bbT$, we have $$\det\left(1-TY|\scrS_{\Iw^+, \bfitx}^{\dagger}\right)\,\,|\,\,\det\left(1-TY|\scrH_{\Par, \bfitx}^{\tol}\right).$$ 

By \cite[Theorem 4.3.2]{Wu-2020}, there exists an $h_{k}\in \R_{>0}$ such that for all $h\in \Q\cap (0, h_k]$, the canonical map $$H^{n_0}_{\Par}(X_{\Iw^+}(\C), D_{k}^{\dagger}(\T_0, \Q_p))^{\leq h}\rightarrow H_{\Par}^{n_0}(X_{\Iw^+}(\C), \V_{\GSp_{2g}, k}^{\alg, \vee})^{\leq h}$$ is an isomorphism. On the other hand, let 
$$\underline{\omega}_{\Iw^+, \cusp}^{k}:=\underline{\omega}_{\Iw^+}^{k}\otimes_{\scrO_{\overline{\calX}_{\Iw^+}}}\scrO_{\overline{\calX}_{\Iw^+}}(-\calZ_{\Iw^+})$$ be the sheaf of classical cuspidal Siegel modular forms of weight $k$ on $\overline{\calX}_{\Iw^+}$. The classicality theorem \cite[Theorem 7.1.1]{AIP-2015} provides an $a_k\in \Q_{>0}$ such that for all $h\in\Q\cap(0,a_k]$, the slope-$\leq h$ overconvergent Siegel cuspforms of weight $k$ are classical; namely, $$H^0(\overline{\calX}_{\Iw^+, w}, \underline{\omega}_{w, \cusp}^{k})^{\leq h}\subset H^0(\overline{\calX}_{\Iw^+}, \underline{\omega}_{\Iw^+, \cusp}^{k}).$$ Now, let $\ell_{k}=\min\{h_k, a_k\}$ and take $h\leq \ell_k$. Applying the \emph{generalised Eichler--Shimura morphism} in \cite[Theorem 3.8]{Hida_2002}, we obtain an injection from the space of slope-$\leq h$ overconvergent Siegel cuspforms of classical weight into the slope-$\leq h$ cohomology group with coefficient in the algebraic representation. Consequently, the desired very Zariski-dense subset of $\calS$ can be taken to be $$\calS^{\cl}=\bigcup_{\calS_{\calU, h}\in \Cov_{\mathrm{aff}}(\calS)}\{\bfitx\in \calS_{\calU, h}: \bfitx\text{ has classical weight }k\in \Z_{\geq 0}^{g}\text{ and } h\leq \ell_k\}$$
Finally, \cite[Theorem 5.1.2]{Hansen-PhD} yields the result. 
\end{proof}

Given Proposition \ref{Proposition: comparison of eigenvarieties}, we may identify $\calE_0^{\mf}$ with its image in $\calE_0^{\oc}$ and denote it by $\calE_0$ for simplicity. We have a diagram $$\begin{tikzcd}
\calE_0\arrow[r, "\pi"]\arrow[rr, bend right = 30, "\wt"'] & \calS\arrow[r, "\wt_{\calS}"] & \calW
\end{tikzcd}.$$ 


\subsection{Sheaves on the cuspidal eigenvariety}\label{subsection: sheaves on the cuspidal eigenvariety}

Given a weight $(R_{\calU}, \kappa_{\calU})$ and an integer $r>1+r_{\calU}$, we write \[
    \OC_{\kappa_{\calU}, \C_p}^{r, \cusp} = \left\{\begin{array}{ll}
        \OC_{\kappa_{\calU}, \C_p}^{r, \cusp}, & \text{if $\calU$ is a small weight} \\
        \image\left( H_{c}^{n_0}(X_{\Iw^+}(\C), D_{\kappa_{\calU}}^{r}(\T_0, R_{\calU}))\rightarrow H^{n_0}(X_{\Iw^+}(\C), D_{\kappa_{\calU}}^{r}(\T_0, R_{\calU}))\right)\widehat{\otimes}_{\Q_p} \C_p & \text{if $\calU$ is an affinoid weight}
    \end{array}\right.,
\] where $H_{c}^{n_0}(X_{\Iw^+}(\C), D_{\kappa_{\calU}}^{r}(\T_0, R_{\calU}))\rightarrow H^{n_0}(X_{\Iw^+}(\C), D_{\kappa_{\calU}}^{r}(\T_0, R_{\calU}))$ is the natural map from the compactly supported Betti cohomology group to the usual one. We also write \[
    \OC_{\kappa_{\calU}, \C_p}^{\dagger, \cusp} := \varprojlim_{r} \OC_{\kappa_{\calU}, \C_p}^{r, \cusp}.
\]

Suppose that $(R_{\calU}, \kappa_{\calU})$ is an small open weight and recall the overconvergent Eichler--Shimura morphism for overconvergent Siegel cuspforms \[
    \ES_{\kappa_{\calU}}^{\cusp}: \OC_{\kappa_{\calU}, \C_p}^{r, \cusp} \rightarrow S_{\Iw^+, w}^{\kappa_{\calU}+g+1}(-n_0).
\] If $(\calU, h)$ slope-adapted, then the Hecke-equivariance of $\ES_{\kappa_{\calU}}^{\cusp}$ induces a $\C_p\widehat{\otimes}R_{\calU}$-linear map \[
   \ES_{\kappa_{\calU}}^{\cusp, \leq h}: \OC_{\kappa_{\calU}, \C_p}^{r \cusp, \leq h}\rightarrow S_{\Iw^+, w}^{\kappa_{\calU}+g+1,\leq h}(-n_0). 
\] of finite projective $\C_p\widehat{\otimes}R_{\calU}$-modules. 

Now, if $\calU'\subset \calU^{\rig}$ is an affinoid weight, the $\C_p\widehat{\otimes}R_{\calU'}$-linear map $\ES_{\kappa_{\calU'}}^{\cusp}$ is defined to be the composition \begin{equation}\label{eq: ES for affinoid weights}
    \ES_{\kappa_{\calU'}}^{\cusp, \leq h}: \OC_{\kappa_{\calU'}, \C_p}^{r, \cusp, \leq h} \cong \OC_{\kappa_{\calU}, \C_p}^{r, \cusp, \leq h}\otimes_{R_{\calU}[\frac{1}{p}]}R_{\calU'} \rightarrow S_{\Iw^+, w}^{\kappa_{\calU}+g+1, \leq h}(-n_0) \rightarrow S_{\Iw^+, w}^{\kappa_{\calU'}+g+1, \leq h}(-n_0),
\end{equation} where the first isomorphism follows from Proposition \ref{Proposition: functoriality of BS chain complex under slope decomposition}. 

Recall the natural map $\pi:\calE_0\rightarrow\calS$ and let $\calE_{\calU, h}$ be the preimage of $\calS_{\calU, h}$. On the cuspidal eigenvariety $\calE_0$, we consider two coherent sheaves $\sheafOC^{\dagger}_{\cusp}$ and $\scrS_{\Iw^+}^{\dagger}(-n_0)$ given by 
$$
    \sheafOC^{\dagger}_{\cusp}(\calE_{\calU, h}):=\OC_{\kappa_{\calU}, \C_p}^{\dagger, \cusp, \leq h}
    $$
and    
$$\scrS_{\Iw^+}^{\dagger}(-n_0)(\calE_{\calU, h}):= S_{\Iw^+}^{\kappa_{\calU}+g+1, \leq h}(-n_0),
$$
for all $\calS_{\calU, h}\in \Cov_{\mathrm{aff}}(\calS)$. 

\begin{Theorem}\label{Theorem: OES over the eigenvariety}
There exists a morphism \[
    \EScal: \sheafOC_{\cusp}^{\dagger}\rightarrow \scrS_{\Iw^+}^{\dagger}(-n_0)
\] of coherent sheaves over $\calE_0$ such that if $(\calU, h)$ is a slope-adapted pair, then $\EScal(\calE_{\calU, h})$ is exactly the overconvergent Eichler--Shimura morphism for overconvergent Siegel cuspforms \[
    \ES_{\kappa_{\calU}}^{\cusp, \leq h}: \OC_{\kappa_{\calU}, \C_p}^{\dagger, \cusp, \leq h} \rightarrow S_{\Iw^+}^{\kappa_{\calU}+g+1, \leq h}(-n_0).
\]
\end{Theorem}
\begin{proof}
It follows from (\ref{eq: ES for affinoid weights}) and the functoriality of $\ES_{\kappa_{\calU}}^{\cusp}$ in small open weights $\calU$.
\end{proof}

Denote by $\sheafIm$ and $\sheafKer$ the image and the kernel of $\EScal$, respectively. We obtain a short exact sequence of sheaves on $\calE_0$ $$0\rightarrow \sheafKer\rightarrow \sheafOC_{\cusp}^{\dagger}\rightarrow \sheafIm\rightarrow 0.$$ We remind the readers that this short exact sequence is Galois- and Hecke-equivariant. 

\begin{Theorem}\label{Theorem: local spliting of OES} Let $\calE_0^{\loc}$ be the locus of $\calE_0$ where $\sheafKer$, $\sheafOC_{\cusp}^{\dagger}$, and $\sheafIm$ are locally free. Then the short exact sequence $$0\rightarrow \sheafKer\rightarrow \sheafOC_{\cusp}^{\dagger}\rightarrow \sheafIm\rightarrow 0$$ over $\calE_0^{\loc}$ splits locally.
\end{Theorem}
\begin{proof}
We follow the same strategy as in \cite[Theorem 6.1(c)]{AIS-2015}. For any sufficiently small open affinoid subset $\calV=\Spa(R_{\calV}, R_{\calV}^+)\subset \calE_0^{\loc}$ on which the three sheaves are free and the sequence $$0\rightarrow \sheafKer(\calV)\rightarrow \sheafOC_{\cusp}^{\dagger}(\calV)\rightarrow \sheafIm(\calV)\rightarrow 0$$ is exact, consider $$\scrH(\calV):=\Hom_{R_{\calV}}(\sheafIm(\calV), \sheafKer(\calV)).$$ The $G_{\Q_p}$-equivariance of the short exact sequence defines a class in $H^1(G_{\Q_p}, \scrH(\calV))\cong \text{Ext}^1_{R_{\calV}[G_{\Q_p}]}(\sheafIm(\calV), \sheafKer(\calV))$.

Let $\varphi_{\Sen}\in \End(\scrH(\calV))$ denote the Sen operator associated with $\scrH(\calV)$. Such an operator was introduced in \cite{Sen-analytic} (see also \cite{Kisin-2003}). Then by \cite[Proposition 2.3]{Kisin-2003}, $\det(\varphi_{\Sen})\in R_{\calV}$ kills the cohomology group $H^1(G_{\Q_p}, \scrH(\calV))$. On the other hand, $\det(\varphi_{\Sen})$ is non-zero. Therefore, after localising at this element, the short exact sequence splits as a sequence of semilinear $G_{\Q_p}$-representations. Since the Galois-action commutes with the Hecke-actions, the splitting can be chosen to be Hecke-equivariant. 
\end{proof}