\section{Toroidal compactifications of the Siegel modular varieties}\label{section: boundary}
In this section, we study the toroidal compactifications of the Siegel modular varieties following \cite{Stroh-TorComp} and \cite{Pilloni-Stroh-CoherentCohomologyandGaloisRepresentations}. In particular, Pilloni and Stroh construct the (toroidally compactified) perfectoid Siegel modular variety of infinite level (\`a la Scholze in \cite{Scholze-2015}) by introducing the \emph{modified integral structures} of the toroidal compactifications on the finite levels.

This section is organised as follows. In \S \ref{subsection: boundary strata}, we study the notion of toroidal compactification of Siegel modular varieties at finite level. Then, in \S \ref{subsection: perfectoid Siegel modular variety}, we recall the construction of the perfectoid Siegel modular variety of infinite level and the associated Hodge--Tate period map by following \cite{Pilloni-Stroh-CoherentCohomologyandGaloisRepresentations}. We also show that this perfectoid object serves as a pro-Kummer \'etale Galois cover over the Siegel modular varieties at the finite levels. In order to be consistent with the notations in the main text of this paper, our notations are slightly different from the ones in \cite{Stroh-TorComp} and \cite{Pilloni-Stroh-CoherentCohomologyandGaloisRepresentations}. 

Throughout this section, we fix the following notations: \begin{enumerate}
    \item[$\bullet$] Let $V=\Z^{2g}$ and $V_p=\Z_p^{2g}$, equipped with the symplectic pairings defined in \S \ref{subsection: Algebraic and p-adic groups}. We denote by $\frakC$ the collection of all totally isotropic direct summands of $V$.
    \item[$\bullet$] For any totally isotropic direct summand $V'\subset V$, let $C(V/V'^{\perp})$ denote the cone of symmetric bilinear forms on $(V/V'^{\perp})\otimes_{\Z}\R$ which are positive semi-definite and whose kernel is defined over $\Q$.
    \item[$\bullet$] Observe that if $V', V''\in \frakC$ such that $V'\subset V''$, there is a natural inclusion $C(V/V'^{\perp})\subset C(V/V''^{\perp})$. We define $$\calC:=\big(\bigsqcup_{V'\in \frakC}C(V/V'^{\perp})\big)/\sim$$ where the equivalence relation is given by the aforementioned inclusions.
    \item[$\bullet$] Let $\frakS$ be a fixed $\GSp_{2g}(\Z)$-admissible smooth rational polyhedral cone decomposition of $\calC$ (see \cite[Definition 3.2.3.1]{Stroh-TorComp}). This means $\frakS$ consists of a smooth rational polyhedral cone decomposition of $C(V/V'^{\perp})$ (in the sense of \cite[Chapter IV, \S 2]{Faltings-Chai}) for every $V'\in \frakC$ such that 
    \begin{enumerate}
    \item[(i)] The decomposition of $C(V/V'^{\perp})$ coincides with the restriction of the decomposition of $C(V/V''^{\perp})$ whenever $V'\subset V''$, and
    \item[(ii)] $\frakS$ is $\GSp_{2g}(\Z)$-invariant and $\frakS/\GSp_{2g}(\Z)$ is a finite set.
    \end{enumerate}
    \item[$\bullet$] For every $n\in \Z_{\geq 1}$, let 
    $$\Gamma(p^n)=\{\bfgamma\in \GSp_{2g}(\Z_p): \bfgamma\equiv \one_{2g}\mod p^n\}$$
    as in \S \ref{subsection: Algebraic and p-adic groups}. Let us abuse the notation and write $\Gamma(p^0):=\GSp_{2g}(\Z_p)$. 
    \item[$\bullet$] For simplicity, let $\Iw$ and $\Iw^+$ denote the $p$-adic groups $\Iw_{\GSp_{2g}}$ and $\Iw^+_{\GSp_{2g}}$ as in \S \ref{subsection: Algebraic and p-adic groups}, respectively.
    \item[$\bullet$] For the rest of the section, let $\Gamma$ denote either $\Gamma(p^n)$ (for some $n\geq 0$), $\Iw$, or $\Iw^+$ which indicates the level structures of the Siegel modular varieties that we concern. We also write $$\widetilde{\Gamma}:=\GSp_{2g}(\Z)\cap\Gamma.$$ 
    \end{enumerate}


\subsection{Toroidal compactifications and boundary strata}\label{subsection: boundary strata}
Let $N\geq 3$ be a fixed integer coprime to $p$. Let $X_0$ be the moduli scheme over $\calO_{\C_p}$ of principally polarised abelian schemes of dimension $g$ equipped with a principal $N$-level structure. The fixed choice of polyhedral cone decomposition $\frakS$ gives rise to a toroidal compactification $\overline{X}_0^{\tor}$ (see, for example, \cite[Chapter IV, \S 4]{Faltings-Chai} or \cite[\S 3.2]{Stroh-TorComp}). Let $X$ and $\overline{X}^{\tor}$ be the base change of $X_0$ and $\overline{X}_0^{\tor}$ to $\C_p$, respectively. We view $\overline{X}^{\tor}$ as an fs log scheme equipped with the divisorial log structure defined by the boundary divisor.

Let $\Gamma$ denote either $\Gamma(p^n)$ (for some $n\geq 0$), $\Iw$, or $\Iw^+$. Let $X_{\Gamma}$ be the finite \'etale cover of $X$ parameterising $\Gamma$-level structure, as defined in Definition \ref{Definition: Siegel modular varieties of (strict) Iwahoris level}. More precisely, 
\begin{enumerate}
\item[(i)] $X_{\Gamma(p^n)}$ parameterises $(A, \lambda, \psi_N, \psi_{p^n})$ where $(A, \lambda)$ is a principally polarised abelian variety over $\C_p$ and $$\psi_N:V\otimes_{\Z}(\Z/N\Z)\xrightarrow[]{\sim} A[N]$$ and $$\psi_{p^n}:V\otimes_{\Z}(\Z/p^n\Z)\xrightarrow[]{\sim}A[p^n]$$
are symplectic isomorphisms.
\item[(ii)] $X_{\Iw}$ parameterises $(A, \lambda, \psi_N, \Fil_{\bullet}A[p])$ where $(A, \lambda, \psi_N)$ is as in (i) and $\Fil_{\bullet}A[p]$ is a full flag of $A[p]$ that satisfies
$$(\Fil_{\bullet}A[p])^{\perp}\cong \Fil_{2g-\bullet}A[p]$$
with respect to the Weil pairing.
\item[(iii)] $X_{\Iw^+}$ parameterises $(A, \lambda, \psi_N, \Fil_{\bullet}A[p], \{C_i:i=1, \ldots, g\})$ where $(A, \lambda, \psi_N, \Fil_{\bullet}A[p])$ is as in (ii) and $\{C_i:i=1, \ldots, g\}$ is a collection of subgroups $C_i\subset A[p]$ of order $p$ such that
$$\Fil_iA[p]=\langle C_1, \ldots, C_i\rangle$$
for all $i=1, \ldots, g$.
\end{enumerate}

We know that $X_{\Gamma(p^n)} \rightarrow X$ (resp., $X_{\Gamma(p)} \rightarrow X_{\Iw}$; resp., $X_{\Gamma(p)} \rightarrow X_{\Iw^+}$) is Galois with Galois group $\GSp_{2g}(\Z/p^n\Z)$ (resp., $B_{\GSp_{2g}}(\Z/p\Z)$; resp., $B_{\GSp_{2g}}^+(\Z/p\Z) = \left\{ \left(\substack{\bfgamma_a \,\, \bfgamma_b\\ \quad \,\, \bfgamma_d}\right)\in B_{\GSp_{2g}}(\Z/p\Z): \bfgamma_a \text{ is diagonal}\right\}$). The goal of this subsection is to construct the \emph{toroidal compactification} $\overline{X}^{\tor}_{\Gamma}$ of $X_{\Gamma}$ determined by the fixed polyhedral decomposition $\frakS$. It is an fs log scheme satisfying the following properties:
\begin{enumerate}
    \item[(Tor1)] $\overline{X}^{\tor}_{\Gamma}$ is finite Kummer \'etale over $\overline{X}^{\tor}$; 
    \item[(Tor2)] There is a cartesian diagram \[
        \begin{tikzcd}
            X_{\Gamma} \arrow[r, hook]\arrow[d] & \overline{X}^{\tor}_{\Gamma}\arrow[d]\\
            X\arrow[r, hook] & \overline{X}^{\tor}
        \end{tikzcd}
    \] and that the log structure on $\overline{X}^{\tor}_{\Gamma}$ is the divisorial log structure defined by the divisor $Z_{\Gamma}:=\overline{X}^{\tor}_{\Gamma}\smallsetminus X_{\Gamma}$;
    \item[(Tor3)] \begin{enumerate}
        \item[(i)] If $\Gamma = \Gamma(p^n)$, then \[
            \overline{X}^{\tor}_{\Gamma} \rightarrow \overline{X}^{\tor}
        \] is Galois with Galois group $\GSp_{2g}(\Z/p^n\Z)$. 
        \item[(ii)] If $\Gamma = \Iw$, then \[
            \overline{X}^{\tor}_{\Gamma(p)} \rightarrow \overline{X}^{\tor}_{\Iw}
        \] is Galois with Galois group $B_{\GSp_{2g}}(\Z/p\Z)$. 
        \item[(iii)] If $\Gamma = \Iw^+$, then \[
            \overline{X}^{\tor}_{\Gamma(p)} \rightarrow \overline{X}^{\tor}_{\Iw^+}
        \] is Galois with Galois group $B^+_{\GSp_{2g}}(\Z/p\Z)$.
    \end{enumerate}
\end{enumerate}

The construction of the toroidal compactification in the case $\Gamma=\Gamma(p^n)$ is well-known. For completeness, we briefly review the construction of $\overline{X}_{\Gamma(p^n)}^{\tor}$ following \cite{Pilloni-Stroh-CoherentCohomologyandGaloisRepresentations}.

Notice that every $\sigma\in \frakS$ necessarily lives in the interior of $C(V/V'^{\perp})$ for a unique $V'\in \frakC$ of some rank $r\leq g$. We have the following diagram from \cite[4.1.A]{Pilloni-Stroh-CoherentCohomologyandGaloisRepresentations}: 
$$
\begin{tikzcd}
M_{V', n}\arrow[r]\arrow[rd] & M_{V', n, \sigma}\arrow[r]\arrow[d] & M_{V', n, \frakS}\arrow[ld]\\
& B_{V', n}\arrow[d]\\ & X_{V', n}
\end{tikzcd}.
$$ 
We briefly describe the objects in the diagram and refer to \cite[Appendice A]{Pilloni-Stroh-CoherentCohomologyandGaloisRepresentations} for details:\begin{enumerate}
\item[$\bullet$] Let $X_{0,V'}$ be the moduli scheme parameterising principally polarised abelian schemes over $\calO_{\C_p}$ of dimension $g-r$ equipped with a principal $N$-level structure. Let $X_{V'}$ denote the base change of $X_{0,V'}$ to $\C_p$.
\item[$\bullet$] Let $X_{V', n}$ be the finite \'etale cover of $X_{V'}$ parameterising principal $p^n$-level structures. Over $X_{V', n}$, there is a universal abelian variety $A_{V'}$. 
\item[$\bullet$] Roughly speaking, the algebraic variety $B_{V', n}$ over $X_{V', n}$ parameterises semiabelian varieties with ``principal $N$- and $p^n$-level structures'' where the semiabelian variety is an extension of $A_{V'}$ by the torus $T_{V'}:=V'\otimes_{\Z}\mathbb{G}_m$. In particular, over $B_{V', n}$, there is a universal semiabelian variety
$$0\rightarrow T_{V'}\rightarrow G_{V'}\rightarrow A_{V'}\rightarrow 0$$
together with a universal isogeny of semiabelian varieties
$$\begin{tikzcd}
T_{V'}\arrow[d, "\text{id}"] \arrow[r]& G_{V'}\arrow[d] \arrow[r] & A_{V'}\arrow[d, "p^n"]\\
T_{V'}\arrow[r] &G_{V'} \arrow[r] & A_{V'}
\end{tikzcd}$$
whose kernel induces a natural inclusion $A_{V'}[p^n]\subset G_{V'}[p^n]$. This yields a decomposition
$$G_{V'}[p^n]\simeq (V'/p^nV'\otimes\mu_{p^n})\oplus A_{V'}[p^n].$$

\item[$\bullet$] Roughly speaking, the algebraic variety $M_{V',n}$ over $B_{V', n}$ parameterises principally polarised 1-motives of type $[V/V'^{\perp}\rightarrow G_{V'}]$ together with a ``principal $p^n$-level structure''. In particular, over $M_{V',n}$, there is a universal 1-motive
$$\widetilde{M}_{V'}=[V/V'^{\perp}\rightarrow G_{V'}]$$ 
together with a universal decomposition
$$\widetilde{M}_{V'}[p^n]\simeq (V'/p^nV'\otimes\mu_{p^n})\oplus A_{V'}[p^n]\oplus (V/V'^{\perp}\otimes \Z/p^n\Z).$$
It turns out $M_{V', n}$ is a torus over $B_{V', n}$ with the torus $$\Hom\left(\frac{1}{Np^n}\Sym^2(V/V'^{\perp}), \bbG_m\right).$$ 

\item[$\bullet$] The morphism $M_{V', n}\rightarrow M_{V', n, \sigma}$ is the affine toroidal embedding attached to the cone $\sigma\in C(V/V'^{\perp})$. Let $Z_{V', n, \sigma}:=M_{V', n, \sigma}\smallsetminus M_{V',n}$ denote the closed stratum of $M_{V', n, \sigma}$. Since $\sigma$ uniquely determines $V'$, we might simply write $Z_{n, \sigma}$.
    \item[$\bullet$] The morphism $M_{V', n}\rightarrow M_{V', n, \frakS}$ is the toroidal embedding attached to the polyhedral decomposition $\frakS$. Let $Z_{V', n, \frakS}:=M_{V', n, \frakS}\smallsetminus M_{V', n}$ denote the closed stratum of $M_{V', n, \frakS}$.
\end{enumerate} 

\begin{Theorem}[$\text{\cite[Th\'{e}or\`{e}me 4.1]{Pilloni-Stroh-CoherentCohomologyandGaloisRepresentations}}$]
We have 
\begin{enumerate}
    \item[(i)] The toroidal compactification $\overline{X}_{\Gamma(p^n)}^{\tor}$ admits a stratification indexed by the finite set $\frakS/\widetilde{\Gamma}(p^n)$. For any $\sigma\in \frakS$, the corresponding stratum in $\overline{X}_{\Gamma(p^n)}^{\tor}$ is isomorphic to $Z_{V', n, \sigma}$.
    \item[(ii)] The boundary $\overline{X}_{\Gamma(p^n)}^{\tor}\smallsetminus X_{\Gamma(p^n)}$ is given by a normal crossing divisor. The codimension-one strata $Z_{V', n, \sigma}$ are in bijection with the irreducible components of the normal crossing divisor. Such $V'$ necessarily has rank 1.
    \item[(iii)] The toroidal compactification is compatible with change of levels. In particular, there are natural finite morphisms $\overline{X}_{\Gamma(p^n)}^{\tor}\rightarrow \overline{X}_{\Gamma(p^m)}^{\tor}$ for $n\geq m$. 
    \item[(iv)] There is a natural action of $\GSp_{2g}(\Z_p)/\Gamma(p^n)$ on $\overline{X}_{\Gamma(p^n)}^{\tor}$. It permutes the boundary strata accordingly.
   \end{enumerate}
\end{Theorem}

On the other hand, the case for $\Gamma = \Iw$ is carefully studied in \cite{Stroh-TorComp}. However, instead of following \emph{loc. cit.}, we propose an alternative way to obtain $\overline{X}_{\Gamma}^{\tor}$ with the desired properties (Tor1), (Tor2) and (Tor3). To this end, we recall a theorem of K. Fujiwara and K. Kato (\cite[Theorem 7.6]{Illusie}):

\begin{Theorem}[Fujiwara--Kato]\label{Theorem: Fujiwara--Kato}
Let $Y$ be a regular scheme, $D$ an effective divisor of $Y$ with normal crossing and $U := Y \smallsetminus D$. Equip $Y$ with the divisorial log structure defined by $D$. Then, the restriction functor \[
    \left[\begin{array}{c}
        \text{finite Kummer \'etale}  \\
        \text{cover over $Y$} 
    \end{array}\right] \rightarrow \left[\begin{array}{c}
        \text{finite \'etale}  \\
         \text{cover over $U$}
    \end{array}\right], \quad T \mapsto T\times_Y U
\] if fully faithful. The essential image of this functor consisting of those finite \'etale covers over $U$ which are tamely ramified along $D$.
\end{Theorem}

In particular, when $Y$ is further a variety over a field of characteristic $0$, every finite \'etale cover over $U$ is tamely ramified along $D$. That is, one obtains an isomorphism between the finite Kummer \'etale site $Y_{\fket}$ and the finite \'etale site $U_{\fet}$.

\begin{Proposition}\label{Proposition: toridal compactification of algebraic Siegel varieties}
Let $\Gamma$ denote either $\Gamma(p^n)$ (for some $n>0$), $\Iw$, or $\Iw^+$. There exists a unique fs log scheme $\overline{X}^{\tor}_{\Gamma}$ over $\overline{X}^{\tor}$ satisfying (Tor1),  (Tor2), and (Tor3).
\end{Proposition}
\begin{proof}
Recall that $\overline{X}^{\tor}$ is equipped with the divisorial log structure given by the boundary divisor 
$Z = \overline{X}^{\tor}\smallsetminus X$ of normal crossing (by \cite[Chapter IV, Theorem 6.7 (1)]{Faltings-Chai}). Theorem \ref{Theorem: Fujiwara--Kato} yields a unique log scheme $\overline{X}_{\Gamma}$, which is finite Kummer \'etale over $\overline{X}^{\tor}$, extending the finite \'etale morphism $X_{\Gamma} \rightarrow X$. This shows that $\overline{X}_{\Gamma}$ satisfies (Tor1) and (Tor2). Finally, by applying a scheme-theoretic version of Lemma \ref{Kummer etale Galois cover}, we conclude that $\overline{X}_{\Gamma}$ also satisfies (Tor3). 
\end{proof}

\begin{Remark}\label{Remark: comparison of constructions of toroidal compactification}
\normalfont When $\Gamma \in \{\Gamma(p^n), \Iw\}$, one should ask whether our construction of $\overline{X}^{\tor}_{\Gamma}$ coincides with the ones constructed in \cite{Pilloni-Stroh-CoherentCohomologyandGaloisRepresentations} and \cite{Stroh-TorComp}. The answer to this question is affirmative. When $\Gamma = \Gamma(p^n)$, \cite[Chapter IV, Theorem 6.7(6)]{Faltings-Chai} implies that $\overline{X}_{\Gamma(p^n)}^{\tor}$ is finite Kummer \'etale over $\overline{X}^{\tor}$ with Galois group $\GSp_{2g}(\Z/p^n\Z)$. The uniqueness of $\overline{X}^{\tor}_{\Gamma}$ then yields the identification. For $\Gamma = \Iw$, it follows similarly by applying \cite[Théorème 3.2.7.1]{Stroh-TorComp}. 
\end{Remark}

To wrap up the subsection, we pass to the realm of adic spaces. Let $\calX_{\Gamma}$ (resp., $\overline{\calX}_{\Gamma}$) denote the adic space over $\Spa(\C_p, \calO_{\C_p})$ associated with $X_{\Gamma}$ (resp., $\overline{X}^{\tor}_{\Gamma}$). In particular, we refer $\overline{\calX}_{\Gamma}$ as the \emph{toroidal compactification} of $\calX_{\Gamma}$ determined by the fixed polyheral decomposition $\frakS$. It satisfies the following analogues of (Tor1), (Tor2), and (Tor3):

\begin{enumerate}
    \item[(Tor1')] The log adic space $\overline{\calX}_{\Gamma}$, equipped with the divisorial log structure given by the boundary divisor $\calZ_{\Gamma} = \overline{\calX}_{\Gamma}\smallsetminus \calX_{\Gamma}$, is finite Kummer \'etale over $\overline{\calX}$; 
    \item[(Tor2')] There is a cartesian diagram \[
        \begin{tikzcd}
            \calX_{\Gamma} \arrow[r, hook]\arrow[d] & \overline{\calX}_{\Gamma}\arrow[d]\\
            \calX\arrow[r, hook] & \overline{\calX}
        \end{tikzcd}
    \]
    \item[(Tor3')] \begin{enumerate}
        \item[(i)] If $\Gamma = \Gamma(p^n)$, then \[
            \overline{\calX}_{\Gamma} \rightarrow \overline{\calX}
        \] is Galois with Galois group $\GSp_{2g}(\Z/p^n\Z)$. 
        \item[(ii)] If $\Gamma = \Iw$, then \[
            \overline{\calX}_{\Gamma(p)} \rightarrow \overline{\calX}_{\Iw}
        \] is Galois with Galois group $B_{\GSp_{2g}}(\Z/p\Z)$. 
        \item[(iii)] If $\Gamma = \Iw^+$, then \[
            \overline{\calX}_{\Gamma(p)} \rightarrow \overline{\calX}_{\Iw^+}
        \] is Galois with Galois group $B^+_{\GSp_{2g}}(\Z/p\Z)$.
    \end{enumerate}
\end{enumerate}


\subsection{The perfectoid Siegel modular variety}\label{subsection: perfectoid Siegel modular variety}
Let $\frakX$ (resp., $\overline{\frakX}^{\tor}$) be the formal completion of $X_0$ (resp., $\overline{X}_0^{\tor}$) along its special fibre. Let $\frakX_{\Gamma(p^n)}$ (resp., $\overline{\frakX}^{\tor}_{\Gamma(p^n)}$) be the normalisation of $\frakX$ (resp., $\overline{\frakX}^{\tor}$) inside the rigid analytic space associated with $X_{\Gamma(p^n)}$ (resp., $\overline{X}^{\tor}_{\Gamma(p^n)}$). 

In order to work with the toroidal compactification at the infinite level, the authors of \cite{Pilloni-Stroh-CoherentCohomologyandGaloisRepresentations} consider modified versions $\overline{\frakX}_{\Gamma(p^n)}^{\tor-\text{mod}}$ of the formal schemes $\overline{\frakX}_{\Gamma(p^n)}^{\tor}$, which we briefly recall. 

Let $n\in \Z_{\geq 0}$ and let $\frakG$ be the tautological semiabelian scheme over $\overline{\frakX}^{\tor}_{\Gamma(p^n)}$. Let $$\pi:\frakG\rightarrow \overline{\frakX}^{\tor}_{\Gamma(p^n)}$$
be the natural projection and let
$$\underline{\Omega}_{\Gamma(p^n)}:=\pi_*\Omega^1_{\frakG/\overline{\frakX}^{\tor}_{\Gamma(p^n)}}. \footnote{The sheaf $\underline{\Omega}_{\Gamma(p^n)}$ is denoted by $\omega_A$ in \cite{Pilloni-Stroh-CoherentCohomologyandGaloisRepresentations}.}$$
Over $\frakX_{\Gamma(p^n)}$, composing the universal trivialisation
$$\psi_{p^n}: V\otimes_{\Z}(\Z/p^n\Z)\rightarrow \frakG[p^n]$$
(which becomes isomorphism on the rigid generic fibre), the duality $\frakG[p^n]\cong \frakG[p^n]^{\vee}$, and the Hodge--Tate map
$$\frakG[p^n]^{\vee}\rightarrow \underline{\Omega}_{\Gamma(p^n)}/p^n\underline{\Omega}_{\Gamma(p^n)}$$
we obtain
$$\HT_{\Gamma(p^n)}: V\otimes_{\Z}(\Z/p^n\Z)\rightarrow \underline{\Omega}_{\Gamma(p^n)}/p^n\underline{\Omega}_{\Gamma(p^n)}$$ 
which induces
$$\HT_{\Gamma(p^n)}\otimes \id: \big(V\otimes_{\Z}(\Z/p^n\Z)\big)\otimes_{\Z} \scrO_{\frakX_{\Gamma(p^n)}}\rightarrow \underline{\Omega}_{\Gamma(p^n)}/p^n\underline{\Omega}_{\Gamma(p^n)}.$$ 
According to \cite[Proposition 1.2]{Pilloni-Stroh-CoherentCohomologyandGaloisRepresentations}, this map extends to the toroidal compactification:
\begin{equation}\label{eq: extended HT map}
    \HT_{\Gamma(p^n)}\otimes \id: \big(V\otimes_{\Z}(\Z/p^n\Z)\big)\otimes_{\Z} \scrO_{\overline{\frakX}^{\tor}_{\Gamma(p^n)}}\rightarrow \underline{\Omega}_{\Gamma(p^n)}/p^n\underline{\Omega}_{\Gamma(p^n)}.
\end{equation}
More precisely, in terms of the explicit description in \S \ref{subsection: boundary strata}, \'etale locally at the boundary stratum, there is a universal semiabelian scheme $G_{V'}$ with constant toric rank sitting in an exact sequence \[
    0 \rightarrow T_{V'} \rightarrow G_{V'} \rightarrow A_{V'} \rightarrow 0
\] as well as a principally polarised 1-motive $\widetilde{M}_{V'} = [V'^{\perp}/V' \rightarrow G_{V'}]$. We consider the composition \[
    \widetilde{M}_{V'}[p^n] \cong \widetilde{M}_{V'}[p^n]^{\vee} \twoheadrightarrow G_{V'}[p^n]^{\vee} \xrightarrow{\HT_{G_{V'}[p^n]^{\vee}}} \underline{\omega}_{G_{V'}}/p^n,
\] where the first isomorphism is given by the principal polarisation on $\widetilde{M}_{V'}$. Composing with the universal trivialisation of $\widetilde{M}_{V'}[p^n]$ and tensoring with the structure sheaf, we arrive at the desired morphism (\ref{eq: extended HT map}).

Consider the image of $\HT_{\Gamma(p^n)}\otimes \id$ and then consider its preimage inside $\underline{\Omega}_{\Gamma(p^n)}$. This yields a subsheaf $\underline{\Omega}_{\Gamma(p^n)}^{\text{mod}}\subset \underline{\Omega}_{\Gamma(p^n)}$. In fact, $\underline{\Omega}_{\Gamma(p^n)}^{\text{mod}}$ does not depend on $n$; \emph{i.e.,} if $n\geq m$ and $\overline{\frakX}_{\Gamma(p^n)}^{\tor} \rightarrow \overline{\frakX}_{\Gamma(p^m)}^{\tor}$ is the natural projection, then the pullback of $\underline{\Omega}_{\Gamma(p^m)}^{\text{mod}}$ coincides with $\underline{\Omega}_{\Gamma(p^n)}^{\text{mod}}$.

Now, let $n$ be any positive integer greater than $\frac{g}{p-1}$. Consider ideals $\scrI_1, \ldots, \scrI_g\subset \scrO_{\overline{\frakX}^{\tor}_{\Gamma(p^n)}}$ generated by the lifts of the determinants of the minors of rank $g,\ldots, 1$ of the map
$$\HT_{\Gamma(p^n)}\otimes \id: \big(V\otimes_{\Z}(\Z/p^n\Z)\big)\otimes_{\Z} \scrO_{\overline{\frakX}^{\tor}_{\Gamma(p^n)}}\rightarrow \underline{\Omega}_{\Gamma(p^n)}/p^n\underline{\Omega}_{\Gamma(p^n)}.$$ 
Notice that these ideals are invertible on the rigid generic fibre. Let $\widetilde{\frakX}^{\tor}_{\Gamma(p^n)}$ be the formal scheme obtained by consecutive formal blowups of $\overline{\frakX}^{\tor}_{\Gamma(p^n)}$ along these ideals. In particular, $\underline{\Omega}_{\Gamma(p^n)}^{\text{mod}}$ becomes locally free over $\widetilde{\frakX}^{\tor}_{\Gamma(p^n)}$.  

Let $\overline{\frakX}_{\Gamma(p^n)}^{\tor-\text{mod}}$ be the normalisation of $\widetilde{\frakX}^{\tor}_{\Gamma(p^n)}$ inside its adic generic fibre. We remark that the adic generic fibre of $\overline{\frakX}_{\Gamma(p^n)}^{\tor-\text{mod}}$ coincides with the one of $\overline{\frakX}_{\Gamma(p^n)}^{\tor}$. For any $m\geq n>\frac{g}{p-1}$, there is a natural finite morphism
$$\overline{\frakX}_{\Gamma(p^{m})}^{\tor-\text{mod}}\rightarrow \overline{\frakX}_{\Gamma(p^n)}^{\tor-\text{mod}}.$$

Notice that the adic generic fibre of $\overline{\frakX}_{\Gamma(p^n)}^{\tor-\text{mod}}$ coincides with $\overline{\calX}_{\Gamma(p^n)}$. The locally free sheaf $\underline{\Omega}_{\Gamma(p^n)}^{\text{mod}}$ gives rise to a locally free $\scrO^+_{\overline{\calX}_{\Gamma(p^n)}}$-module $\underline{\omega}^{\text{mod},+}_{\Gamma(p^n)}$ on $\overline{\calX}_{\Gamma(p^n)}$. Inverting $p$, we obtain the locally free $\scrO_{\overline{\calX}_{\Gamma(p^n)}}$-module $\underline{\omega}_{\Gamma(p^n)}$. Notice that $\underline{\omega}_{\Gamma(p^n)}$ is just the usual sheaf of invariant differentials defined using the universal semiabelian varieties.

Consider the projective limit
$$\overline{\frakX}_{\Gamma(p^{\infty})}^{\tor-\text{mod}}:=\varprojlim \overline{\frakX}_{\Gamma(p^n)}^{\tor-\text{mod}}$$
in the category of $p$-adic formal schemes. Let $\overline{\calX}_{\Gamma(p^{\infty})}$ be its adic generic fibre in the sense of \cite{Scholze-Weinstein}.

\begin{Proposition}[$\text{\cite[Proposition 4.9 \& Corollaire 4.14]{Pilloni-Stroh-CoherentCohomologyandGaloisRepresentations}}$]\label{Proposition: perfectoid toroidal compactification}
 We have \begin{enumerate}
   \item[(i)] The adic generic fibre $\overline{\calX}_{\Gamma(p^{\infty})}$ is a perfectoid space such that  $$\overline{\calX}_{\Gamma(p^{\infty})}\sim\varprojlim_{n}\overline{\calX}_{\Gamma(p^n)}$$ in the sense of \cite[Definition 2.4.1]{Scholze-Weinstein}.
   \item[(ii)] For every $n\in \Z_{\geq 0}$, the natural morphism
   $$\overline{\calX}_{\Gamma(p^{\infty})}\rightarrow \overline{\calX}_{\Gamma(p^n)}$$ is a pro-Kummer \'{e}tale Galois cover with Galois group $\Gamma(p^n)$. (Here we have abused the notation and identify $\overline{\calX}_{\Gamma(p^{\infty})}$ with the object $\varprojlim_{n}\overline{\calX}_{\Gamma(p^n)}$ in the pro-Kummer \'etale site.) Simiarly, the natural morphism
   $$\overline{\calX}_{\Gamma(p^{\infty})}\rightarrow \overline{\calX}_{\Iw} \quad (\text{resp., }\overline{\calX}_{\Gamma(p^{\infty})}\rightarrow \overline{\calX}_{\Iw^+})$$ is a pro-Kummer \'{e}tale Galois cover with Galois group $\Iw_{\GSp_{2g}}$ (resp., $\Iw_{\GSp_{2g}}^+$).
    \end{enumerate}
\end{Proposition}

\begin{Remark}
\normalfont Induced from the stratification on the finite levels, the perfectoid Siegel modular variety $\overline{\calX}_{\Gamma(p^{\infty})}$ admits a stratification by the profinite set
$$\hat{\frakS}:=\varprojlim_n \frakS/\widetilde{\Gamma}(p^n).$$
For each $\hat{\sigma}=(\sigma_n)_{n\geq 0}\in \hat{\frakS}$, the $\hat{\sigma}$-stratum is canonically isomorphic to 
$$\calZ_{\infty, \hat{\sigma}}:=\varprojlim_n\calZ_{n, \sigma_n}$$
where $\calZ_{n,\sigma_n}$ is the adic spaces given by the analytification of $Z_{n,\sigma_n}$.
\end{Remark}

Finally, we recall the construction of the Hodge--Tate period map. By definition of $\underline{\omega}^{\text{mod},+}_{\Gamma(p^n)}$, the Hodge--Tate map $\HT_{\Gamma(p^n)}$ induces a map (which we abuse the notation and still denote by $\HT_{\Gamma(p^n)}$)
$$\HT_{\Gamma(p^n)}: V\otimes_{\Z}(\Z/p^n\Z)\rightarrow \underline{\omega}^{\text{mod},+}_{\Gamma(p^n)}/p^n\underline{\omega}^{\text{mod},+}_{\Gamma(p^n)}.$$ 
Let $\underline{\omega}^{\text{mod},+}_{\Gamma(p^{\infty})}$ and $\underline{\omega}_{\Gamma(p^{\infty})}$ denote the pullbacks of $\underline{\omega}^{\text{mod},+}_{\Gamma(p^n)}$ and $\underline{\omega}_{\Gamma(p^n)}$, respectively, to $\overline{\calX}_{\Gamma(p^{\infty})}$. Pulling back $\HT_{\Gamma(p^n)}$ to the infinite level and taking inverse limit, we obtain 
$$\HT_{\Gamma(p^{\infty})}: V_p\rightarrow \underline{\omega}^{\text{mod},+}_{\Gamma(p^{\infty})}$$
which induces a surjection
$$\HT_{\Gamma(p^{\infty})}\otimes \id: V_p\otimes_{\Z_p} \scrO^+_{\overline{\calX}_{\Gamma(p^{\infty})}}\rightarrow \underline{\omega}^{\text{mod},+}_{\Gamma(p^{\infty})}.$$
Finally, inverting $p$, the surjection
$$\HT_{\Gamma(p^{\infty})}\otimes \id: V_p\otimes_{\Z_p} \scrO_{\overline{\calX}_{\Gamma(p^{\infty})}}\rightarrow \underline{\omega}_{\Gamma(p^{\infty})}$$
induces the \textbf{\textit{Hodge--Tate period map}}
$$\pi_{\HT}: \overline{\calX}_{\Gamma(p^{\infty})}\rightarrow \adicFL$$
where $\adicFL$ is the (adic) flag variety parameterising the maximal lagrangians of $V_p$.