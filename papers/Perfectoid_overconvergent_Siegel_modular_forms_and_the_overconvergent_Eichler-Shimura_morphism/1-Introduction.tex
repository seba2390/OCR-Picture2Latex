\section{Introduction}
A classical result, due to M. Eichler and G. Shimura, states that the first cohomology of the complex modular curve with coefficients in $\mathrm{Sym}^{k-2}\Q^2$, after scalar extension to $\C$, admits a Hecke-equivariant decomposition as the direct sum of the space of weight $k$ holomorphic modular forms and the space of weight $k$ anti-holomorphic cusp forms.

G. Faltings establishes a $p$-adic analogue of the Eichler--Shimura decomposition. Let $p$ be a prime number and let $\C_p$ be the completion of a fixed algebraic closure $\overline{\Q}_p$ of $\Q_p$. Suppose $X$ is the modular curve (of some tame level $N$) over $\Q_p$ and let $\overline{X}$ be its compactification. Let $\pi: E\rightarrow \overline{X}$ be the universal generalised elliptic curve and let $\underline{\omega}:=\pi_*\Omega^1_{E/\overline{X}}$. In \cite{FaltingsHT}, Faltings constructs a Hecke- and $\mathrm{Gal}(\overline{\Q}_p/\Q_p)$-equivariant decomposition
\[
H_{\et}^1(X_{\overline{\Q}_p},\mathrm{Sym}^{k}\Q_p^2)\otimes_{\Q_p} \C_p(1) \simeq \big(H^0(\overline{X},\underline{\omega}^{k+2})\otimes_{\Q_p} \C_p\big) \oplus \big(H^1(\overline{X},\underline{\omega}^{-k})\otimes_{\Q_p} \C_p (k+1)\big) 
\]
where the Galois actions on the coherent cohomology groups are trivial. 

This type of results have been generalised to families of overconvergent modular forms by F. Andreatta, A. Iovita and G. Stevens in \cite{AIS-2015} and later extended to the case of compact Shimura curves over $\Q$ by P. Chojecki, D. Hansen and C. Johansson in \cite{CHJ-2017}. The novelty of the second work consists of a ``perfectoid construction'' of overconvergent families of automorphic forms as well as a succinct calculation of overconvergent cohomology groups on the pro-\'etale site. The same method has also been adapted to the case of elliptic and Hilbert modular forms in \cite{BHW-2019}.

There are two main goals we would like to achieve in this paper. We first present a construction of automorphic sheaves for overconvergent Siegel modular forms (of genus $g$) using the perfectoid method. Then we follow the spirit of Chojecki--Hansen--Johansson to establish an overconvergent Eichler--Shimura morphism for Siegel modular forms. 

One key ingredient for such a perfectoid construction is the (toroidally compactified) perfectoid Siegel modular variety $\overline{\calX}_{\Gamma(p^{\infty})}$ studied in \cite{Pilloni-Stroh-CoherentCohomologyandGaloisRepresentations}. This perfectoid space comes equipped with the \emph{Hodge--Tate period map} $\pi_{\HT}: \overline{\calX}_{\Gamma(p^{\infty})}\rightarrow \adicFL$ where the adic flag variety $\adicFL$ parameterises maximal lagrangian subspaces of a fixed symplectic space of rank $2g$. We also consider the Siegel modular variety $\overline{\calX}_{\Iw^+}$ for the finite \emph{strict Iwahori level} (see Definition \ref{Definition: Siegel modular varieties of (strict) Iwahoris level} for details), viewed as an adic space. Roughly speaking, the strict Iwahori level serves as a deeper level compared with the usual Iwahori level. There is a natural projection $\overline{\calX}_{\Gamma(p^{\infty})}\rightarrow \overline{\calX}_{\Iw^+}$ from the Siegel modular variety at infinite level to the one at finite level.

In order to investigate the overconvergent Siegel modular forms, we restrict our attentions to certain open subspaces $\overline{\calX}_{\Gamma(p^{\infty}), w}$ and $\overline{\calX}_{\Iw^+, w}$ of $\overline{\calX}_{\Gamma(p^{\infty})}$ and $\overline{\calX}_{\Iw^+}$, respectively. They are referred as the \emph{$w$-ordinary loci} (see Definition \ref{Definition: w-ordinary}) and they contain the usual ordinary locus.

Let $(R_{\calU}, \kappa_{\calU})$ be a weight (see Definition \ref{Definition: weights}) and let $r_{\calU}$ be the integer as defined in Definition \ref{Definition: w-analytic weight}. For any $w\in \Q_{>0}$ with $w>1+r_{\calU}$, we construct a sheaf $\underline{\omega}_w^{\kappa_{\calU}}$ over $\overline{\calX}_{\Iw^+, w}$ whose global sections are exactly the $w$-overconvergent Siegel modular forms of strict Iwahori level and weight $\kappa_{\calU}$. Roughly speaking, sections of $\underline{\omega}_w^{\kappa_{\calU}}$ consist of functions $f$ on $\overline{\calX}_{\Gamma(p^{\infty}),w}$ which 
\begin{enumerate}
    \item[$\bullet$] take value in a certain weight-$\kappa_{\calU}$ analytic representation $C_{\kappa_{\calU}}^{w-\an}(\Iw_{\GL_g}, \C_p\widehat{\otimes}R_{\calU})$ of the Iwahori subgroup of $\GL_g(\Z_p)$, and
    \item[$\bullet$] satisfy the following formula regarding the natural action of the stricit Iwahori subgroup $\Iw_{\GSp_{2g}}^+$ of $\GSp_{2g}(\Z_p)$:
    \[
        \bfgamma^* f=\rho_{\kappa_{\calU}}(\bfgamma_a+\frakz\bfgamma_c)^{-1}f\qquad\text{for any} \qquad\bfgamma=\begin{pmatrix}\bfgamma_a & \bfgamma_b\\ \bfgamma_c & \bfgamma_d\end{pmatrix}\in \Iw^+_{\GSp_{2g}},
    \]
    where $\frakz$ stands for the pullback of the coordinate function on the flag variety and  $\rho_{\kappa_{\calU}}(\bfgamma_a+\frakz\bfgamma_c)$ stands for a certain automorphism on $C_{\kappa_{\calU}}^{w-\an}(\Iw_{\GL_g}, \C_p\widehat{\otimes}R_{\calU})$. 
\end{enumerate}
When $p>2g$, this sheaf coincides with (the pullback to the strict Iwahori level of) the automorphic sheaf constructed in \cite{AIP-2015}. See \S \ref{section:constructionsheaf} for a complete story.

The next step is to look at the overconvergent cohomology groups. Suppose the weight $(R_{\calU}, \kappa_{\calU})$ is \emph{small} in the sense of Definition \ref{Definition: weights}. A typical example of a small weight is a \emph{wide-open} disc whose ring of functions is isomorphic to $\Z_p\llbrack  T_1, \ldots, T_g \rrbrack$. For such a weight, we consider a space $D^r_{\kappa}(\T_0, R_{\calU})$ of analytic distributions on certain $p$-adic manifold $\T_0$. (See \S \ref{subsection:continuousfunctions} for the precise definitions.) This space of distributions is equipped with a natural action of the strict Iwahori subgroup and hence induces a sheaf $\scrD_{\kappa_{\calU}}^r$ on the Siegel modular variety $\overline{\calX}_{\Iw^+}$. The cohomology groups of $\scrD_{\kappa_{\calU}}^r$ are precisely the \emph{overconvergent modular symbols} studied by A. Ash and G. Stevens, among others (see for example \cite{Ash-Stevens, Hansen-PhD, Johansson-Newton}).

Our goal is to explicitly construct a morphism that relates the automorphic sheaves with the overconvergent cohomology groups. To this end, we will compute the objects on both sides on the \emph{pro-Kummer \'etale site} $\overline{\calX}_{\Iw^+, \proket}$. On one hand, the overconvergent cohomology groups can be calculated in terms of the cohomology groups of some sheaf of distributions $\sheafOD_{\kappa_{\calU}}^r$ over the pro-Kummer \'etale site. On the other hand, let $\widehat{\underline{\omega}}_{w}^{\kappa_{\calU}}$ be the ``completed'' pull-back of the automorphic sheaf $\underline{\omega}_w^{\kappa_{\calU}}$ to this site. The key step is then Proposition \ref{Proposition: OES for sheaves on the pro-Kummer etale site} where we construct a morphism of sheaves
\[
\sheafOD_{\kappa_{\calU}}^r\rightarrow \widehat{\underline{\omega}}_{w}^{\kappa_{\calU}}.
\]
This morphism can be thought of as an analytic analogue of the ``projection onto the highest weight vector''. We remark that the construction involves taking transposes of matrices while the usual Iwahori subgroup is not closed under taking transposes. This is why we have to work with a smaller subgroup---the strict Iwahori subgroup---as a compromise.

Consequently, we obtain the \emph{overconvergent Eichler-Shimura morphism}. It is a natural Hecke- and Galois-equivariant morphism from the (family of) oveconvergent cohomology groups to the (family of) overconvergent Siegel modular forms
\[
    \ES_{\kappa_{\calU}}: H_{\proket}^{n_0}(\overline{\calX}_{\Iw^+}, \sheafOD_{\kappa_{\calU}}^r) \rightarrow H^0(\overline{\calX}_{\Iw^+, w}, \,\underline{\omega}_w^{\kappa_{\calU}+g+1})(-n_0),
\] where $n_0 = \dim_{\C_p} \overline{\calX}_{\Iw^+}$ and ``$(-n_0)$'' stands for the Tate twist.

We list some properties of this map. First of all, we show that it is compatible with base change on the weights. Secondly, we are able to control its image when specialising to a dominant algebraic weight $k\in \Z_{\geq 0}^{g}$. In particular, we show in Theorem \ref{thm:imageclassicalweight} that the image of $\ES_{k}$ is contained in the space of classical algebraic Siegel modular forms. The proof uses the fact that when the weight is a dominant algebraic weight, the ``highest weight vector'' is an algebraic function. Lastly, we conclude the paper by showing that $\ES_{\kappa_{\calU}}$ can be glued to a morphism of sheaves on a suitable cuspidal eigenvariety.

There are many things we don't do: we don't control the cokernel of the map $\ES_{\kappa_{\calU}}$ in general and we do not investigate the kernel of this map at all. A natural expectation is that one could construct a full filtration of $H_{\proket}^{n_0}(\overline{\calX}_{\Iw^+}, \sheafOD_{\kappa_{\calU}}^r) $ using higher Coleman theory, recently developed by G. Boxer and V. Pilloni \cite{Boxer--Pilloni--higherColeman}, in terms of information on suitable strata of $\overline{\calX}_{\Iw^+}$. When $g=1$, such a result is recently announced by J. E. Rodr\'iguez Camargo (\cite{Rodriguez-dualOES}).

We also remark that while preparing this paper, Andreatta and Iovita have announced the article \cite{AI-2020}, in which they upgrade their previous work \cite{AIS-2015} to an ``overconvergent de Rham Eichler--Shimura morphism'', meaning that their Eichler--Shimura map has as source the overconvergent cohomology groups tensored with $\bbB_{\mathrm{dR}}$.
They have also announced upcoming work concerning this type of de Rham Eichler--Shimura morphisms for overconvergent Siegel modular forms of genus $2$. This suggests that the results in this paper can be upgraded to investigate finer $p$-adic Hodge theoretic properties of overconvergent cohomology groups; for example, the construction of de Rham (or even crystalline) periods in $p$-adic families. We shall leave this to future studies. 

The paper is organised as follows. In \S \ref{section:PerfectoidSMV}, we define the main geometric objects of interest, including the Siegel modular varieties for various level structures, the perfectoid Siegel modular variety at infinite level, the flag variety, and the Hodge--Tate period map. The next section, \S \ref{section:constructionsheaf}, contributes to the construction of the overconvergent automorphic sheaves. In particular, when $p>2g$, we show that our construction coincides with the (pullback to the strict Iwahori level of the) automorphic sheaves of Andreatta--Iovita--Pilloni. We warn the reader that \S \ref{subsection: admissibility}-\ref{subsection:comparison sheaf aip} is the most technical part of the paper. These subsections can be skipped on a first reading. In \S \ref{section:overconvergentcohomologies}, we study the space of analytic distributions and the overconvergent cohomology groups. Finally, in \S \ref{section:EichlerShimura} and \S \ref{section:oncuspidaleigenvariety}, we construct the Eichler--Shimura morphism as well as its upgraded version on the cuspidal eigenvariety. 

In \S \ref{Section: Kummer etale and pro-Kummer etale sites of log adic spaces}, we recall the basics of logarithmic adic spaces and their Kummer pro-\'etale topology following \cite{Diao}. Also in the same section, we include some technical calculations of the derived functor $R^i \nu_*$ and a generalised projection formula, both of which are used in the main text. In \S \ref{section: boundary}, we recall the basics of the toroidal compactifications of Siegel modular varieties. We also review the ``modified integral structures'' studied in \cite{Pilloni-Stroh-CoherentCohomologyandGaloisRepresentations} as well as the construction of the perfectoid Siegel modular variety at the infinite level.

\paragraph{Acknowledgement.} The authors would like to express their gratitude to Przemy{\l}aw Chojecki, David Hansen, and Christian Johansson for discussions on the perfectoid construction of sheaves of families of Siegel modular forms and for pointing out an inaccuracy in the first version of this paper. G.R. would like to thank Daniel Barrera-Salazar and Riccardo Brasca. J.-F. W. would like to thank Ulrich G\"ortz for helpful conversations about toroidal compactifications of Siegel modular varieties and $p$-divisible groups. He would also like to thank Marc Levine for a succinct introduction on $1$-motives during a casual conversation. 


\paragraph{Conventions and notations.} Through out this paper, we fix the following: \begin{enumerate}
    \item[$\bullet$] $g\in \Z_{\geq 1}$.
    \item[$\bullet$] $p\in \Z_{> 0}$ is an odd prime number. Due to certain technicality, we will have to assume $p>2g$ at some places. Such an assumption shall be clear in the context. 
    \item[$\bullet$] $N\in \Z_{\geq 3}$ is an integer coprime to $p$.
    \item[$\bullet$] We fix once and forever an algebraic closure $\overline{\Q}_p$ of $\Q_p$ and an algebraic isomorphism $\C_p\simeq \C$, where $\C_p$ is the $p$-adic completion of $\overline{\Q}_p$. We write $G_{\Q_p}$ for the absolute Galois group $\Gal(\overline{\Q}_p/\Q_p)$. We also fix the $p$-adic absolute value on $\C_p$ so that $|p|=p^{-1}$.
    \item[$\bullet$] For any $w\in \Q_{>0}$, we denote by ``$p^w$'' an element in $\C_p$ with absolute value $p^{-w}$. All constructions in the paper will not depend on such choices.
    \item[$\bullet$] We adopt the language of almost mathematics. In particular, for an $\calO_{\C_p}$-module $M$, we denote by $M^a$ for the associated almost $\calO_{\C_p}$-module.
    \item[$\bullet$] For $n\in \Z_{\geq 1}$ and any set $R$, we denote by $M_n(R)$ the set of $n$ by $n$ matrices with coefficients in $R$.
    \item[$\bullet$] The transpose of a matrix $\bfalpha$ is denoted by $\trans\bfalpha$.
    \item[$\bullet$] For any $n\in \Z_{\geq 1}$, we denote by $\one_n$ the $n\times n$ identity matrix and denote by $\oneanti_n$ the $n\times n$ anti-diagonal matrix whose non-zero entries are $1$; \emph{i.e.,} $$\one_n=\begin{pmatrix} 1& & \\ & \ddots & \\ & &1\end{pmatrix}\quad\text{ and }\quad\oneanti_n=\begin{pmatrix} & & 1\\ & \iddots & \\ 1 & &\end{pmatrix}$$
    \item[$\bullet$] We use $\cong$ to denote canonical isomorphisms and $\simeq$ to denote non-canonical ones.
    \item[$\bullet$] In principle (except for \S \ref{Section: Kummer etale and pro-Kummer etale sites of log adic spaces}), symbols in Gothic font (\emph{e.g.} $\frakX, \frakY, \frakZ$) stand for formal schemes; symbols in calligraphic font (\emph{e.g.} $\calX, \calY, \calZ$) stand for adic spaces; and symbols in script font (\emph{e.g.} $\scrO, \scrF, \scrE$) stand for sheaves (over various geometric objects). 
\end{enumerate}