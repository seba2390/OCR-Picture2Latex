\section{Siegel modular varieties and the Hodge--Tate period map}\label{section:PerfectoidSMV}
In this section, we introduce the Siegel modular varieties for various level structures, viewed as adic spaces, as well as their toroidal compactifications. We also recall the perfectoid Siegel modular variety introduced in \cite{Pilloni-Stroh-CoherentCohomologyandGaloisRepresentations} together with the Hodge--Tate period map. The notion of perfectoid Siegel modular variety has its root in \cite{Scholze-2015}. However, we point out that the author of \textit{loc. cit.} only considers minimal compactifications while it is important for us to work with the toroidal compactifications.

\subsection{\texorpdfstring{Algebraic and $p$-adic groups}{Algebraic and p-adic groups}}\label{subsection: Algebraic and p-adic groups}
We start with a list of algebraic and $p$-adic groups that will appear repeatedly throughout the paper. 

Let $V:=\Z^{2g}$ equipped with an alternating pairing $$\bla\cdot, \cdot\bra:V\times V\rightarrow \Z, \quad (\vec{v}, \vec{v}')\mapsto \trans\vec{v}\begin{pmatrix} & -\oneanti_g\\ \oneanti_g & \end{pmatrix}\vec{v}',$$ where we view elements in $V$ as column vectors. In particular, if $e_1, ..., e_{2g}$ is the standard basis for $V$, then $$\bla e_i, e_j\bra=\left\{\begin{array}{ll}
    -1, & \text{if }i<j\text{ and } j=2g+1-i\\
    1, & \text{if }i>j\text{ and }j=2g+1-i\\
    0, & \text{else}
\end{array}\right. .$$ We define the algebraic group $\GSp_{2g}$ to be the subgroup of $\GL_{2g}$ that preserves this pairing up to a unit. In other words, for any ring $R$, $$\GSp_{2g}(R):=\left\{\bfgamma\in \GL_{2g}(R): \trans\bfgamma \begin{pmatrix} & -\oneanti_g\\ \oneanti_g\end{pmatrix}\bfgamma = \varsigma(\bfgamma)\begin{pmatrix} & -\oneanti_g\\ \oneanti_g\end{pmatrix}\text{ for some }\varsigma(\bfgamma)\in R^{\times}\right\}.$$ Equivalently, for any $\bfgamma=\begin{pmatrix}\bfgamma_a & \bfgamma_b\\ \bfgamma_c & \bfgamma_d\end{pmatrix}\in \GL_{2g}$, $\bfgamma\in \GSp_{2g}$ if and only if 
$$\trans\bfgamma_a\oneanti_g\bfgamma_c=\trans\bfgamma_c\oneanti_g\bfgamma_a, \quad \trans\bfgamma_b\oneanti_g\bfgamma_d=\trans\bfgamma_d\oneanti_g\bfgamma_b, \text{ and }\trans\bfgamma_a\oneanti_g\bfgamma_d-\trans\bfgamma_c\oneanti_g\bfgamma_b=\varsigma(\bfgamma)\oneanti_g$$ for some $\varsigma(\bfgamma)\in \bbG_m$.

Taking base change to $\Z_p$, we consider $V_p:=V\otimes_{\Z}\Z_p$, equipped with the induced alternating pairing
$$\bla\cdot, \cdot\bra:V_p\times V_p\rightarrow \Z_p, \quad (\vec{v}, \vec{v}')\mapsto \trans\vec{v}\begin{pmatrix} & -\oneanti_g\\ \oneanti_g & \end{pmatrix}\vec{v}'.$$ Let $e_1, \ldots, e_{2g}$ be the standard basis for $V_p$ and let $\Fil^{\std}_{\bullet}$ denote the standard increasing filtration on $V_p$ defined by $\Fil^{\std}_0=0$ and $$\Fil^{\std}_i=\langle e_1, \ldots, e_i\rangle,$$ for $i=1, \ldots, 2g$.

The algebraic and $p$-adic subgroups of $\GL_g$ and $\GSp_{2g}$ considered in the present paper are the following:\begin{enumerate}
    \item[$\bullet$] For every $m\in \Z_{\geq 1}$, we write $$\Gamma(p^m):=\ker\left(\GSp_{2g}(\Z_p)\xrightarrow{\mod p^m}\GSp_{2g}(\Z/p^m\Z)\right).$$

    \item[$\bullet$] The Borel subgroups are \begin{align*}
        B_{\GL_g} & := \text{the Borel subgroup of upper triangular matrices in $\GL_g$},\\
        B_{\GSp_{2g}} & := \text{the Borel subgroup of upper triangular matrices in $\GSp_{2g}$.}
    \end{align*}
    
    \item[$\bullet$] The corresponding unipotent radicals are \begin{align*}
        U_{\GL_g} & := \text{ the upper triangular $g\times g$ matrices whose diagonal entries are all $1$},\\
        U_{\GSp_{2g}} & := \text{ the upper triangular $2g\times 2g$ matrices in $\GSp_{2g}$ whose diagonal entries are all $1$}.
    \end{align*} 
    
    \item[$\bullet$] The corresponding maximal tori for $\GL_g$ and $\GSp_{2g}$ are the maximal tori of diagonal matrices, denoted by $T_{\GL_g}$ and $T_{\GSp_{2g}}$, respectively. The Levi decomposition then yields $$B_{\GL_g}=T_{\GL_g}U_{\GL_g}\quad \text{ and }\quad B_{\GSp_{2g}}=T_{\GSp_{2g}}U_{\GSp_{2g}}.$$
    
    \item[$\bullet$] Let $B_{\GL_g}^{\opp}$ and $B_{\GSp_{2g}}^{\opp}$ be the opposite Borel subgroups of $B_{\GL_g}$ and $B_{\GSp_{2g}}$, respectively. They consist of lower triangular matrices of the corresponding algebraic groups. Similarly, $U_{\GL_g}^{\opp}$ and $U_{\GSp_{2g}}^{\opp}$ stand for the opposite unipotent radicals of $U_{\GL_g}$ and $U_{\GSp_{2g}}$, respectively. 
    
    \item[$\bullet$] To simplify the notations, we write $$T_{\GL_g, 0}=T_{\GL_g}(\Z_p),\quad U_{\GL_g, 0}=U_{\GL_g}(\Z_p), \quad B_{\GL_g, 0}=B_{\GL_g}(\Z_p),$$
$$T_{\GSp_{2g}, 0}=T_{\GSp_{2g}}(\Z_p),\quad U_{\GSp_{2g}, 0}=U_{\GSp_{2g}}(\Z_p),\quad B_{\GSp_{2g}, 0}=B_{\GSp_{2g}}(\Z_p).$$ 

The subgroups $B^{\opp}_{\GL_g,0}$, $B^{\opp}_{\GSp_{2g},0}$, $U^{\opp}_{\GL_g,0}$, and $U^{\opp}_{\GSp_{2g},0}$ are defined similarly.
 
For every $s\in \Z_{\geq 1}$, define \begin{align*}
        T_{\GL_g, s}:=\ker(T_{\GL_g}(\Z_p)\rightarrow T_{\GL_g}(\Z/p^s\Z)), & \qquad  T_{\GSp_{2g}, s}:=\ker(T_{\GSp_{2g}}(\Z_p)\rightarrow T_{\GSp_{2g}}(\Z/p^s\Z)),\\
        U_{\GL_g, s}:=\ker(U_{\GL_g}(\Z_p)\rightarrow U_{\GL_g}(\Z/p^s\Z)), & \qquad U_{\GSp_{2g}, s}:=\ker(U_{\GSp_{2g}}(\Z_p)\rightarrow U_{\GSp_{2g}}(\Z/p^s\Z)),\\
        B_{\GL_g, s}:=\ker(B_{\GL_g}(\Z_p)\rightarrow B_{\GL_g}(\Z/p^s\Z)), & \qquad  B_{\GSp_{2g}, s}:=\ker(B_{\GSp_{2g}}(\Z_p)\rightarrow B_{\GSp_{2g}}(\Z/p^s\Z)),
    \end{align*} where all of the maps are reduction modulo $p^s$. 
    
    The subgroups $B^{\opp}_{\GL_g,s}$, $B^{\opp}_{\GSp_{2g},s}$, $U^{\opp}_{\GL_g,s}$, and $U^{\opp}_{\GSp_{2g},s}$ are defined similarly.
    
    \item[$\bullet$] The Iwahori subgroups of $\GL_g(\Z_p)$ and $\GSp_{2g}(\Z_p)$ are \begin{align*}
        \Iw_{\GL_g} & := \text{ the preimage of $B_{\GL_g}(\F_p)$ under the reduction map $\GL_g(\Z_p)\rightarrow \GL_g(\F_p)$},\\
        \Iw_{\GSp_{2g}} & :=\text{ the preimage of $B_{\GSp_{2g}}(\F_p)$ under the reduction map $\GSp_{2g}(\Z_p)\rightarrow \GSp_{2g}(\F_p)$}.
    \end{align*} The Iwahori decomposition yields $$\Iw_{\GL_g}= U_{\GL_g, 1}^{\opp}T_{\GL_g, 0}U_{\GL_g, 0}\quad \text{ and }\quad \Iw_{\GSp_{2g}} =  U_{\GSp_{2g}, 1}^{\opp}T_{\GSp_{2g}, 0}U_{\GSp_{2g}, 0}.$$
    
    \item[$\bullet$] We consider the \textbf{\textit{strict Iwahori subgroups}} of $\GL_g(\Z_p)$ and $\GSp_{2g}(\Z_p)$ defined as \begin{align*}
        \Iw_{\GL_g}^+ & := \text{ the preimage of $T_{\GL_g}(\F_p)$ under the reduction map $\GL_g(\Z_p)\rightarrow \GL_g(\F_p)$}\\
        \Iw_{\GSp_{2g}}^{+} & := \left\{\bfgamma\in \GSp_{2g}(\Z_p): \bfgamma\equiv \left(\begin{array}{ccc|ccc}
            * &&  & * &\cdots & * \\ & \ddots & & \vdots & & \vdots \\ && * & * & \cdots & *\\ \hline &&& * &  &  \\ &&&& \ddots &  \\ &&&&& * 
        \end{array}\right)\mod p\right\},
    \end{align*} where the blocks in the definition of $\Iw_{\GSp_{2g}}^+$ are all $(g\times g)$-matrices.  
    
    Clearly, $\Iw_{\GL_g}^+\subset \Iw_{\GL_g}$ and $\Iw_{\GSp_{2g}}^+\subset \Iw_{\GSp_{2g}}$. Also observe that, for any $\bfgamma= \begin{pmatrix}\bfgamma_a & \bfgamma_b\\ \bfgamma_c & \bfgamma_d\end{pmatrix}\in \Iw_{\GSp_{2g}}^+$, we have $\bfgamma_a\in \Iw_{\GL_g}^+$. Moreover, the Iwahori decomposition induces decompositions 
    \[
    \Iw_{\GL_g}^+ = U_{\GL_g, 1}^{\opp}T_{\GL_g, 0}U_{\GL_g, 1} \quad \text{ and }\quad \Iw_{\GSp_{2g}}^+ = U_{\GSp_{2g}, 1}^{\opp} T_{\GSp_{2g}, 0} U_{\GSp_{2g}, 0}^+,
    \] where \( U_{\GSp_{2g}, 0}^+ := U_{\GSp_{2g}, 0}\cap \Iw_{\GSp_{2g}}^+\).

    \item[$\bullet$] Finally, we introduce the notion of ``$w$-neighbourhood'' of some aforementioned $p$-adic groups. For any $w\in \Q_{>0}$, define
\[T^{(w)}_{\GL_g, 0}:=\left\{ \bflambda=(\bflambda_{ij})_{i,j}\in T_{\GL_g}(\calO_{\C_p}):|\bflambda_{ij}-\bflambda'_{ij}|\leq p^{-w}\,\,\textrm{for some }\bflambda'=(\bflambda'_{ij})_{i,j}\in T_{\GL_g, 0}\right\},\]
\[U^{(w)}_{\GL_g, 0}:=\left\{ \bflambda=(\bflambda_{ij})_{i,j}\in U_{\GL_g}(\calO_{\C_p}):|\bflambda_{ij}-\bflambda'_{ij}|\leq p^{-w}\,\,\textrm{for some }\bflambda'=(\bflambda'_{ij})_{i,j}\in U_{\GL_g, 0}\right\},\]
\[B^{(w)}_{\GL_g, 0}:=\left\{ \bflambda=(\bflambda_{ij})_{i,j}\in B_{\GL_g}(\calO_{\C_p}):|\bflambda_{ij}-\bflambda'_{ij}|\leq p^{-w}\,\,\textrm{for some }\bflambda'=(\bflambda'_{ij})_{i,j}\in B_{\GL_g, 0}\right\}.\]
The groups $B^{\opp, (w)}_{\GL_g, 0}$ and $U^{\opp, (w)}_{\GL_g, 0}$ are defined similarly.

For every $s\in \Z_{\geq 1}$, define
\[T^{(w)}_{\GL_g, s}:=\left\{ \bflambda=(\bflambda_{ij})_{i,j}\in T_{\GL_g}(\calO_{\C_p}):|\bflambda_{ij}-\bflambda'_{ij}|\leq p^{-w}\,\,\textrm{for some }\bflambda'=(\bflambda'_{ij})_{i,j}\in T_{\GL_g, s}\right\},\]
\[U^{(w)}_{\GL_g, s}:=\left\{ \bflambda=(\bflambda_{ij})_{i,j}\in U_{\GL_g}(\calO_{\C_p}):|\bflambda_{ij}-\bflambda'_{ij}|\leq p^{-w}\,\,\textrm{for some }\bflambda'=(\bflambda'_{ij})_{i,j}\in U_{\GL_g, s}\right\},\]
\[B^{(w)}_{\GL_g, s}:=\left\{ \bflambda=(\bflambda_{ij})_{i,j}\in B_{\GL_g}(\calO_{\C_p}):|\bflambda_{ij}-\bflambda'_{ij}|\leq p^{-w}\,\,\textrm{for some }\bflambda'=(\bflambda'_{ij})_{i,j}\in B_{\GL_g, s}\right\}.\]
The groups $U^{\opp, (w)}_{\GL_g, s}$ and $B^{\opp, (w)}_{\GL_g, s}$ are defined similarly.

Define
\[\Iw^{(w)}_{\GL_g}:=\left\{ \bflambda=(\bflambda_{ij})_{i,j}\in \GL_g(\calO_{\C_p}):|\bflambda_{ij}-\bflambda'_{ij}|\leq p^{-w}\,\,\textrm{for some }\bflambda'=(\bflambda'_{ij})_{i,j}\in \Iw_{\GL_g}\right\}\]
and
\[\Iw^{+, (w)}_{\GL_g}:=\left\{ \bflambda=(\bflambda_{ij})_{i,j}\in \GL_g(\calO_{\C_p}):|\bflambda_{ij}-\bflambda'_{ij}|\leq p^{-w}\,\,\textrm{for some }\bflambda'=(\bflambda'_{ij})_{i,j}\in \Iw^+_{\GL_g}\right\}.\]
The Iwahori decomposition induces
$$\Iw_{\GL_g}^{(w)}=U_{\GL_g, 1}^{\opp,(w)}T_{\GL_g,0}^{(w)}U_{\GL_g,0}^{(w)}$$
and
$$\Iw_{\GL_g}^{+, (w)}=U_{\GL_g, 1}^{\opp,(w)}T_{\GL_g,0}^{(w)}U_{\GL_g,1}^{(w)}.$$

We also define
$$T_w=\ker(T_{\GL_g}(\calO_{\C_p})\rightarrow T_{\GL_g}(\calO_{\C_p}/p^w)),$$
$$U_w=\ker(U_{\GL_g}(\calO_{\C_p})\rightarrow U_{\GL_g}(\calO_{\C_p}/p^w)),$$
$$B_w=\ker(B_{\GL_g}(\calO_{\C_p})\rightarrow B_{\GL_g}(\calO_{\C_p}/p^w)).$$
The groups $U_w^{\opp}$ and $B_w^{\opp}$ are defined similarly.

Then we have 
$$T_{\GL_g,0}^{(w)}=T_{\GL_g,0}T_w,\quad U_{\GL_g,0}^{(w)}=U_{\GL_g,0}U_w,\quad B_{\GL_g,0}^{(w)}=B_{\GL_g,0}B_w.$$
There are similarly identities for $U_{\GL_g,0}^{\opp, (w)}$ and $B_{\GL_g,0}^{\opp, (w)}$.
\end{enumerate}


\subsection{Siegel modular varieties}\label{subsection: Siegel modular varieties} 
We consider Siegel modular varieties of genus $g$ (of principal tame level $N$) for various level structures at $p$.
\begin{Definition}\label{Definition: Siegel modular varieties of (strict) Iwahoris level}
\begin{enumerate}
\item[(i)] The \textbf{Siegel modular scheme} is the scheme $X_0$ over $\calO_{\C_p}$ that parameterises triples $(A, \lambda, \psi_N)$, where $(A, \lambda)$ is a principally polarised abelian scheme over $\calO_{\C_p}$ and $$\psi_N: V\otimes_{\Z}(\Z/N\Z)\xrightarrow[]{\sim} A[N]$$ 
is a symplectic isomorphism with respect to the pairing induced by $\bla\cdot, \cdot\bra$ on the left and the Weil pairing on the right. Let $X$ denote the base change of $X_0$ to $\C_p$.
\item[(ii)] For every $n\in \Z_{\geq 1}$, the \textbf{Siegel modular variety of principal $p^n$-level} is the algebraic variety $X_{\Gamma(p^n)}$ over $\C_p$ that parameterises quadruples $(A, \lambda, \psi_N, \psi_{p^n})$, where $(A, \lambda)$ is a principally polarised abelian variety over $\C_p$ and $$\psi_N: V\otimes_{\Z}(\Z/N\Z)\xrightarrow[]{\sim} A[N]$$ 
and
$$\psi_{p^n}:V\otimes_{\Z}(\Z/p^n\Z)\xrightarrow[]{\sim} A[p^n]$$
are symplectic isomorphisms.
 \item[(iii)] The \textbf{Siegel modular variety of Iwahori level} is the algebraic variety $X_{\Iw}$ over $\C_p$ that parameterises quadruples $(A, \lambda, \psi_N, \Fil_{\bullet}A[p])$, where $(A, \lambda, \psi_N)$ is as in (ii) and $\Fil_{\bullet}A[p]$ is a full flag of $A[p]$ that satisfies $$(\Fil_{\bullet}A[p])^{\perp}\cong \Fil_{2g-\bullet}A[p]$$ with respect to the Weil pairing.
    \item[(iv)] The \textbf{Siegel modular variety of strict Iwahori level} is the algebraic variety $X_{\Iw^+}$ over $\C_p$ that parameterises quintuples $(A, \lambda, \psi_N, \Fil_{\bullet}A[p], \{C_i:i=1, ..., g\})$, where $(A, \lambda, \psi_N, \Fil_{\bullet}A[p])$ is as in (iii) and $\{C_i: i=1, ..., g\}$ is a collection of subgroups $C_i\subset A[p]$ of order $p$ such that $$\Fil_iA[p]=\langle C_1, \ldots, C_i\rangle$$
for all $i=1, \ldots, g$.
\end{enumerate}
\end{Definition}

For $\Gamma \in \{\Gamma(p^n), \Iw^+ = \Iw_{\GSp_{2g}}^+, \Iw = \Iw_{\GSp_{2g}}, \emptyset\}$, it is well-known that the $\C$-points of the algebraic variety $X_{\Gamma}$ can be identified with the locally symmetric space \[
    X_{\Gamma}(\C) = \GSp_{2g}(\Q)\backslash \GSp_{2g}(\A_f)\times \bbH_g/\Gamma(N)\cdot\Gamma,
\] where \begin{enumerate}
    \item[$\bullet$] $\A_f$ is the ring of finite ad\`eles of $\Q$;
    \item[$\bullet$] $\bbH_g$ is the disjoint union of the Siegel upper- and lower-half spaces;
    \item[$\bullet$] $\Gamma(N) = \{\bfgamma \in \GSp_{2g}(\widehat{\Z}): \bfgamma \equiv 1 \mod N\}$.
\end{enumerate} Here, we use the fix isomorphism $\C_p \simeq \C$ to view $\C$ as a $\C_p$-algebra.

Moreover, we have a chain of forgetful maps
$$X_{\Gamma(p^n)}\rightarrow X_{\Gamma(p)}\rightarrow X_{\Iw^+}\rightarrow X_{\Iw}\rightarrow X$$
with the arrows described as follows:
\begin{enumerate}
\item[$\bullet$] The first arrow sends $(A, \lambda, \psi_N, \psi_{p^n})$ to $(A, \lambda, \psi_N, p^{n-1}\psi_{p^n})$.
\item[$\bullet$] The second arrow sends $(A, \lambda, \psi_N, \psi_p)$ to $(A, \lambda, \psi_N, \Fil_{\bullet}^{\psi_p}A[p], \{\langle \psi_p(e_i)\rangle: i=1, ..., g\})$ where $\Fil_{\bullet}^{\psi_p}A[p]$ stands for the full flag $$0\subset \langle \psi_p(e_1)\rangle\subset\langle \psi_p(e_1), \psi_p(e_2)\rangle\subset \cdots\subset \langle\psi_p(e_1), \ldots, \psi_p(e_{2g})\rangle.$$
\item[$\bullet$] The third arrow sends $(A, \lambda, \psi_N, \Fil_{\bullet}A[p], \{C_i: i=1, ..., g\})$ to $(A, \lambda, \psi_N, \Fil_{\bullet}A[p])$.
\item[$\bullet$] The fourth arrow sends $(A, \lambda, \psi_N, \Fil_{\bullet}A[p])$ to $(A, \lambda, \psi_N)$.
\end{enumerate}

For $\Gamma \in \{\Gamma(p^n), \Iw^+, \Iw, \emptyset\}$, let $\calX_{\Gamma}$ be the adic space over $\Spa(\C_p, \calO_{\C_p})$ associated with $X_{\Gamma}$. The analytifications of the forgetful maps yield 
\begin{equation}\label{eq: chain of adic Siegel varieties}
    \calX_{\Gamma(p^n)} \rightarrow \calX_{\Gamma(p)} \rightarrow \calX_{\Iw^+} \rightarrow \calX_{\Iw} \rightarrow \calX.
\end{equation} By fixing a $\GSp_{2g}(\Z)$-admissible polyhedral cone decomposition as in \S \ref{section: boundary}, we show in \S \ref{subsection: boundary strata} that the chain (\ref{eq: chain of adic Siegel varieties}) extends to a chain of log adic spaces\footnote{For a quick review of log adic spaces and the pro-Kummer \'etale site, see \S \ref{Section: Kummer etale and pro-Kummer etale sites of log adic spaces}.} \[
    \overline{\calX}_{\Gamma(p^n)} \rightarrow \overline{\calX}_{\Gamma(p)} \rightarrow \overline{\calX}_{\Iw^+} \rightarrow \overline{\calX}_{\Iw} \rightarrow \overline{\calX},
\] where, for each $\Gamma \in \{\Gamma(p^n), \Iw^+, \Iw, \emptyset\}$, 
\begin{enumerate}
    \item[$\bullet$] $\overline{\calX}_{\Gamma}$ is the adic space over $\Spa(\C_p, \calO_{\C_p})$ associated with the toroidal compactification $\overline{X}_{\Gamma}$ of $X_{\Gamma}$, determined by the fixed polyhedral cone decomposition; 
    \item[$\bullet$] the log structure on $\overline{\calX}_{\Gamma}$ is the divisorial log structure associated with the boundary divisor $\calZ_{\Gamma} := \overline{\calX}_{\Gamma}\smallsetminus \calX_{\Gamma}$; namely, the corresponding sheaf of monoids $\scrM_{\Gamma}$ on $\overline{\calX}_{\Gamma, \et}$ consists of sections of $\scrO_{\overline{\calX}_{\Gamma,\et}}$ that are invertible on the locus away from the boundary divisor;
    \item[$\bullet$] $\overline{\calX}_{\Gamma}$ is finite Kummer \'etale over $\overline{\calX}$ and \begin{enumerate}
        \item[(i)] $\overline{\calX}_{\Gamma(p^n)}\rightarrow \overline{\calX}$ is Galois with Galois group $\GSp_{2g}(\Z/p^n\Z)$;
        \item[(ii)] $\overline{\calX}_{\Gamma(p)}\rightarrow \overline{\calX}_{\Iw}$ is Galois with Galois group $B_{\GSp_{2g}}(\Z/p\Z)$;
        \item[(iii)] $\overline{\calX}_{\Gamma(p)} \rightarrow \overline{\calX}_{\Iw^+}$ is Galois with Galois group $$B^+_{\GSp_{2g}}(\Z/p\Z) := \left\{\begin{pmatrix}\bfgamma_a &\bfgamma_b\\ &\bfgamma_d\end{pmatrix}\in B_{\GSp_{2g}}(\Z/p\Z): \bfgamma_a\text{ is diagonal}\right\}.$$
    \end{enumerate}
\end{enumerate}
We call $\overline{\calX}_{\Gamma}$ the \textbf{\textit{toroidal compactification}} of $\calX_{\Gamma}$ (determined by the fixed polyhedral cone decomposition). 

Furthermore, we have the \emph{perfectoid Siegel modular variety of infinite level} constructed in \cite{Pilloni-Stroh-CoherentCohomologyandGaloisRepresentations}. See \S \ref{subsection: perfectoid Siegel modular variety} for details.

\begin{Theorem}[\text{\cite[Corollaire 4.14]{Pilloni-Stroh-CoherentCohomologyandGaloisRepresentations}}]\label{Theorem: perfectoid toroidally compactified Siegel modular variety}
There exists a perfectoid space $\overline{\calX}_{\Gamma(p^{\infty})}$ such that $$\overline{\calX}_{\Gamma(p^{\infty})}\sim \varprojlim_{n}\overline{\calX}_{\Gamma(p^n)},$$ where ``$\sim$'' is in the sense of \cite[Definition 2.4.1]{Scholze-Weinstein}.
\end{Theorem}

\begin{Remark}
\normalfont The perfectoid Siegel modular variety is constructed by introducing certain \emph{modified integral structures} at  finite levels. More precisely, consider the toroidal compactificaction $\overline{X}_0^{\tor}$ of $X_0$ and let $\overline{\frakX}^{\tor}$ denote the formal scheme obtained by taking completion along the special fibre of $\overline{\frakX}_0^{\tor}$. Let $\overline{\frakX}^{\tor}_{\Gamma(p^n)}$ denote the normalisation of $\overline{X}^{\tor}$ inside $\overline{\calX}_{\Gamma(p^n)}$. In \cite{Pilloni-Stroh-CoherentCohomologyandGaloisRepresentations}, the authors consider the modified formal schemes $\overline{\frakX}^{\tor-\text{mod}}_{\Gamma(p^n)}$ obtained by taking certain admissible formal blowups from $\overline{\frakX}^{\tor}_{\Gamma(p^n)}$ and then consider the projective limit
$$\overline{\frakX}^{\tor-\text{mod}}_{\Gamma(p^{\infty})}=\varprojlim_n \overline{\frakX}^{\tor-\text{mod}}_{\Gamma(p^n)}$$
in the category of $p$-adic formal schemes. Finally, $\overline{\calX}_{\Gamma(p^{\infty})}$ is defined to be the adic generic fibre of $\overline{\frakX}^{\tor-\text{mod}}_{\Gamma(p^{\infty})}$.
\end{Remark}

We summarise the discussion above in the following commutative diagram $$\begin{tikzcd}
\calX_{\Gamma(p^\infty)}\arrow[r, hook]\arrow[d] & \overline{\calX}_{\Gamma(p^\infty)}\arrow[d]\\
\calX_{\Gamma(p^n)}\arrow[r, hook]\arrow[d] & \overline{\calX}_{\Gamma(p^n)}\arrow[d]\\
\calX_{\Gamma(p)}\arrow[r, hook]\arrow[d] & \overline{\calX}_{\Gamma(p)}\arrow[d]\\
\calX_{\Iw^+}\arrow[r, hook]\arrow[d] & \overline{\calX}_{\Iw^+}\arrow[d]\\
\calX_{\Iw}\arrow[r, hook]\arrow[d] & \overline{\calX}_{\Iw}\arrow[d]\\
\calX \arrow[r, hook] & \overline{\calX}\\
\end{tikzcd}.$$ 
where $\calX_{\Gamma(p^\infty)}$ is the part of $\overline{\calX}_{\Gamma(p^\infty)}$ away from the boundary.

There is a natural $\GSp_{2g}(\Z_p)$-action on $\overline{\calX}_{\Gamma(p^{\infty})}$ permuting the $p$-power level structures. In particular, the chain of natural projections
$$\overline{\calX}_{\Gamma(p^{\infty})}\rightarrow \overline{\calX}_{\Gamma(p^n)}\rightarrow \overline{\calX}_{\Gamma(p)}\rightarrow \overline{\calX}_{\Iw^+}\rightarrow \overline{\calX}_{\Iw}\rightarrow  \overline{\calX}$$
is $\GSp_{2g}(\Z_p)$-equivariant. According to Proposition \ref{Proposition: perfectoid toroidal compactification} (ii), the projection $h_{\Gamma(p^n)}:\overline{\calX}_{\Gamma(p^{\infty})}\rightarrow \overline{\calX}_{\Gamma(p^n)}$ (resp., $h_{\Iw^+}:\overline{\calX}_{\Gamma(p^{\infty})}\rightarrow \overline{\calX}_{\Iw^+}$, $h_{\Iw}:\overline{\calX}_{\Gamma(p^{\infty})}\rightarrow \overline{\calX}_{\Iw}$) is a pro-Kummer \'{e}tale Galois cover with Galois group $\Gamma(p^n)$ (resp., $\Iw^+_{\GSp_{2g}}$, $\Iw_{\GSp_{2g}}$). \footnote{Here we have abused the notation and identify the perfectoid space $\overline{\calX}_{\Gamma(p^{\infty})}$ with the object $\varprojlim_n \overline{\calX}_{\Gamma(p^n)}$ in the pro-Kummer \'{e}tale site $\overline{\calX}_{\proket}$.}

\begin{Lemma}\label{Lemma: structure sheaves at the infinite level and the structure sheaves at the Iwahori level}
We have the following identities of sheaves \begin{align*}
    \scrO_{\overline{\calX}_{\Iw}}^{+}=\left(h_{\Iw, *}\scrO_{\overline{\calX}_{\Gamma(p^{\infty})}}^+\right)^{\Iw_{\GSp_{2g}}}, &  \quad \scrO_{\overline{\calX}_{\Iw}}=\left(h_{\Iw, *}\scrO_{\overline{\calX}_{\Gamma(p^{\infty})}}\right)^{\Iw_{\GSp_{2g}}}\\
    \scrO_{\overline{\calX}_{\Iw^+}}^{+}=\left(h_{\Iw^+, *}\scrO_{\overline{\calX}_{\Gamma(p^{\infty})}}^+\right)^{\Iw^+_{\GSp_{2g}}}, &  \quad \scrO_{\overline{\calX}_{\Iw^+}}=\left(h_{\Iw, *}\scrO_{\overline{\calX}_{\Gamma(p^{\infty})}}\right)^{\Iw^+_{\GSp_{2g}}}.
\end{align*}
\end{Lemma}
\begin{proof}
We give the proof of the first pair of identities. The second pair can be proven by the same argument.

It suffices to prove the first identity. For any affinoid open $\calV\subset \overline{\calX}_{\Iw}$ with preimages $\calV_{n}\subset \overline{\calX}_{\Gamma(p^{n})}$ for $n\in \Z_{>0}\cup \{\infty\}$ such that \[
    \scrO_{\overline{\calX}_{\Gamma(p^{\infty})}}^+(\calV_{\infty}) = \left(\varinjlim_{n}\scrO_{\overline{\calX}_{\Gamma(p^n)}}^+(\calV_n)\right)^{\wedge},
\] we have to show $$\scrO_{\overline{\calX}_{\Iw}}^+(\calV)=\left(\scrO_{\overline{\calX}_{\Gamma(p^{\infty})}}^+(\calV_{\infty})\right)^{\Iw_{\GSp_{2g}}}.$$ Here,
{``$\wedge$''} stands for the $p$-adic completion.
Consider the object $\widetilde{\calV}_{\infty}:=\varprojlim_{n}\calV_n$ in the pro-Kummer \'etale site $\overline{\calX}_{\Iw, \proket}$. By Lemma \ref{Kummer etale Galois cover}, each $\calV_n$ is finite Kummer \'{e}tale over $\calV$ with Galois group $G_n:=\Iw_{\GSp_{2g}}/\Gamma(p^n)$. Thus, $$\scrO_{\overline{\calX}_{\Gamma(p^{\infty})}}^+(\calV_{\infty})=\left(\varinjlim_{n}\scrO_{\overline{\calX}_{\Gamma(p^n)}}^+(\calV_n)\right)^{\wedge}=\left(\scrO_{\overline{\calX}_{\Iw, \proket}}^{+}(\widetilde{\calV}_{\infty})\right)^{\wedge}.$$ By \cite[Lemma 4.1.7 \& Corollary 4.4.13]{Diao}, we know $$\left(\scrO_{\overline{\calX}_{\Gamma(p^n)}}^+(\calV_n)/p^m\right)^{\Iw_{\GSp_{2g}}}=\left(\scrO_{\overline{\calX}_{\Gamma(p^n)}}^+(\calV_n)/p^m\right)^{G_n}=\scrO_{\overline{\calX}_{\Iw}}^+(\calV)/p^m$$ for every $m\in \Z_{\geq 1}$. This implies $$\left(\scrO_{\overline{\calX}_{\Iw, \proket}}^+(\widetilde{\calV}_{\infty})/p^m\right)^{\Iw_{\GSp_{2g}}}=\scrO_{\overline{\calX}_{\Iw}}^+(\calV)/p^m.$$ Consequently, we have \begin{align*}
    \left(\scrO_{\overline{\calX}_{\Gamma(p^{\infty})}}^+(\calV_{\infty})\right)^{\Iw_{\GSp_{2g}}} & = \left(\left(\scrO_{\overline{\calX}_{\Iw, \proket}}^{+}(\widetilde{\calV}_{\infty})\right)^{\wedge}\right)^{\Iw_{\GSp_{2g}}}
    = \left(\varprojlim_{m}\left(\scrO_{\overline{\calX}_{\Iw, \proket}}^{+}(\widetilde{\calV}_{\infty})/p^m\right)\right)^{\Iw_{\GSp_{2g}}}\\
    & = \varprojlim_{m}\left(\left(\scrO_{\overline{\calX}_{\Iw, \proket}}^{+}(\widetilde{\calV}_{\infty})/p^m\right)^{\Iw_{\GSp_{2g}}}\right)
    = \varprojlim_{m} \scrO_{\overline{\calX}_{\Iw}}^+(\calV)/p^m
    = \scrO_{\overline{\calX}_{\Iw}}^+(\calV).
\end{align*}
\end{proof}

\begin{Remark}\label{Remark: why strict Iwahori level}
\normalfont We point out that the main geometric object studied in \cite{AIP-2015} is the (toroidally compactified) Siegel modular variety of Iwahori level while ours is of strict Iwahori level. We introduce this deeper level to deal with certain technical issue involved in the construction of the overconvergent Eichler--Shimura morphism in \S \ref{subsection: OES}. 
\end{Remark}


\subsection{The flag variety}\label{subsection: flag varieties}
The Hodge--Tate period map is a $\GSp_{2g}(\Z_p)$-equivariant morphism from the perfectoid Siegel modular variety to certain flag variety. In this section, let us first describe the target flag variety (and its variants) carefully. 

Recall that $V_p=V\otimes_{\Z}\Z_p$ is the standard symplectic space of rank $2g$ over $\Z_p$. Let $P_{\Siegel}$ be the (opposite) Siegel parabolic subgroup of $\GSp_{2g}$ defined by \[P_{\Siegel}:=\begin{pmatrix}\GL_g \\ M_g & \GL_g\end{pmatrix}\cap \GSp_{2g}.\] Let $\Fl:=P_{\Siegel}\backslash\GSp_{2g}$ be the flag variety over $\Z_p$, parameterising the maximal lagrangians $W\subset V_p$. There is a natural action of $\GSp_{2g}$ on $\Fl$ by right multiplication. Let $\adicFL$ be the associated adic space of $\Fl$ over $\Spa(\Q_p, \Z_p)$, equipped with the induced right action of $\GSp_{2g}(\Q_p)$. Hence, for any $p$-adically complete sheafy $(\Q_p, \Z_p)$-algebra $(R, R^+)$, $\adicFL(R, R^+)$ parameterises maximal lagrangians $W\subset V_p\otimes_{\Z_p}R$. Consider the open subset $\adicFL^{\times}\subset \adicFL$ whose $(R, R^+)$-points are $$\adicFL^{\times}(R, R^+)=\left\{(W\subset V_p\otimes_{\Z_p}R)\in \adicFL(R, R^+):\begin{array}{l}
    \text{there exists a basis $\{w_i\}$ of $W$ such that}  \\
    \text{the matrix $(\bla w_i, e_{2g+1-j}\bra)_{1\leq i,j\leq g}$ is invertible}     
\end{array} \right\}.$$ For any $\bfitx_W=(W\subset V_p\otimes_{\Z_p}R)\in \adicFL^\times(R, R^+)$, there exists a unique basis $\{w_i^{\square}\}$ of $W$ such that \[(\bla w_i^{\square}, e_{2g+1-j}\bra)_{1\leq i,j\leq g}=\one_g.\] Therefore, there exist global sections $\bfitz_{i,j}\in \scrO_{\adicFL^\times}(\adicFL^\times)$ such that for any $\bfitx_{W}\in \adicFL^\times(R, R^+)$, $$w_i^{\square}=e_{i}+\sum_{j=1}^{g}\bfitz_{i,j}(\bfitx_W)e_{g+j}.$$ Since $\bla w_i^{\square}, w_j^{\square}\bra= 0$, we have \begin{align*}
    0 & = \bla w_i^{\square}, w_j^{\square}\bra\\
    & = \bla e_{i}, \sum_{k=1}^g\bfitz_{j,k}(\bfitx_W)e_{g+k}\bra + \bla \sum_{k=1}^g\bfitz_{i,k}(\bfitx_W)e_{g+k}, e_{j}\bra\\ 
    & = \bfitz_{j,g+1-i}(\bfitx_W)-\bfitz_{i,g+1-j}(\bfitx_W).
\end{align*} That is, the matrix $$\bfitz:=\begin{pmatrix}
\bfitz_{1,1} & \cdots & \bfitz_{1,g}\\
\vdots & & \vdots\\
\bfitz_{g,1} & \cdots & \bfitz_{g,g}
\end{pmatrix}$$ 
is symmetric with respect to the anti-diagonal. Moreover, we may use the matrix $\begin{pmatrix}\one_g & \bfitz(\bfitx_W)\end{pmatrix}$ (or just the matrix $\bfitz(\bfitx_W)$) to represent the element $\bfitx_W\in \adicFL^\times(R, R^+)$ because the basis $\{w_i^{\square}\}$ is represented by the matrix $$\begin{pmatrix}
1 & & & \bfitz_{1,1}(\bfitx_W) & \cdots & \bfitz_{1,g}(\bfitx_W)\\
& \ddots & & \vdots & & \vdots\\
& & 1 & \bfitz_{g,1}(\bfitx_W) & \cdots & \bfitz_{g,g}(\bfitx_W)
\end{pmatrix}=\begin{pmatrix}\one_g & \bfitz(\bfitx_W)\end{pmatrix}$$ with respect to the standard basis $e_1, \ldots, e_{2g}$ of $V_p$. 

In the rest of the paper, we take base change of the adic spaces $\adicFL$ and $\adicFL^{\times}$ to $\Spa(\C_p, \calO_{\C_p})$.

For every $w\in \Q_{>0}$, consider an open adic subspace $\adicFL_w^\times\subset \adicFL^\times$ defined by $$\adicFL^\times_w:=\left\{\bfitx\in \adicFL^\times: \max_{i, j}\inf_{h\in \Z_p}\{|\bfitz_{i,j}(\bfitx)-h|\}\leq p^{-w}\right\}.$$ 
For any algebraically closed complete nonarchimedean field $C$ containing $\Q_p$, let
$$\GSp_{2g, w}(C):=\left\{\bfgamma=\begin{pmatrix}\bfgamma_a & \bfgamma_b\\ \bfgamma_c & \bfgamma_d\end{pmatrix}\in\GSp_{2g}(C):\begin{array}{l}
    \bfgamma_a\in \GL_g(C),\text{ and}\\
    \max_{i, j}\inf_{h\in \Z_p}\{|(\bfgamma_a^{-1}\bfgamma_b)_{ij}-h|\}\leq p^{-w}
\end{array} \right\}$$
where $(\bfgamma_a^{-1}\bfgamma_b)_{ij}$ is the $(i,j)$-th entry of the matrix $\bfgamma_a^{-1}\bfgamma_b$. Then the $(C, \calO_C)$-points of $\adicFL^\times_w$ can be identified with the quotient $$\adicFL^\times_w(C, \calO_C)=P_{\Siegel}(C)\backslash\GSp_{2g, w}(C)$$
so that the natural inclusion $\adicFL^\times_w(C, \calO_C)\subset \adicFL(C, \calO_C)$ is induced by
$$\adicFL^\times_w(C, \calO_C)=P_{\Siegel}(C)\backslash\GSp_{2g, w}(C)\hookrightarrow P_{\Siegel}(C)\backslash\GSp_{2g}(C)=\adicFL(C, \calO_C).$$

Recall that there is a natural right action of $\GSp_{2g}(\Q_p)$ on $\adicFL$. The following lemma shows that $\adicFL_w^\times$ is stable under the action of the subgroup $\Iw_{\GSp_{2g}}\subset \GSp_{2g}(\Q_p)$.

\begin{Lemma}\label{Lemma: Iw_GSp stabilises FL_w^times}
The adic space $\adicFL_w^\times$ is stable under the right action of $\Iw_{\GSp_{2g}}$. Coordinate-wise, the action is given by $$\adicFL_w^{\times}\times \Iw_{\GSp_{2g}}\rightarrow \adicFL_w^{\times}, \quad \left(\bfitz,\begin{pmatrix}\bfgamma_a & \bfgamma_b\\ \bfgamma_c & \bfgamma_d\end{pmatrix} \right)\mapsto (\bfgamma_a+\bfitz\bfgamma_c)^{-1} (\bfgamma_b+\bfitz\bfgamma_d).$$
In particular, $\adicFL_w^{\times}$ is also stable under the right action of the subgroup $\Iw_{\GSp_{2g}}^+$.
\end{Lemma}
\begin{proof}
It follows from the definition that the right action of $\bfgamma\in\Iw_{\GSp_{2g}}$ indeed sends $\begin{pmatrix}\one_g & \bfitz(\bfitx_W)\end{pmatrix}$ to 
$$\begin{pmatrix}\one_g &  (\bfgamma_a+\bfitz(\bfitx_W)\bfgamma_c)^{-1} (\bfgamma_b+\bfitz(\bfitx_W)\bfgamma_d)\end{pmatrix} =(\bfgamma_a+\bfitz(\bfitx_W)\bfgamma_c)^{-1} \begin{pmatrix}\one_g & \bfitz(\bfitx_W)\end{pmatrix} \begin{pmatrix}\bfgamma_a & \bfgamma_b\\ \bfgamma_c & \bfgamma_d\end{pmatrix}.$$
It remains to show that, for every $\bfitx_W\in \adicFL^{\times}_w$, the matrix $(\bfgamma_a+\bfitz(\bfitx_W)\bfgamma_c)^{-1} (\bfgamma_b+\bfitz(\bfitx_W)\bfgamma_d)$ lands in $\adicFL^{\times}_w$. But this is straightforward.
\end{proof}


\subsection{Vector bundles on the flag variety}\label{subsection: vector bundles on the flag variety}
Let $\scrW_{\Fl}\subset\scrO_{\Fl}^{2g}$ be the universal maximal lagrangian over $\Fl$. The total space of $\scrW_{\Fl}$ can be naturally identified with $$\scrW_{\Fl}\simeq P_{\Siegel}\backslash (\bbA^g\times \GSp_{2g})$$ where \begin{enumerate}
    \item[$\bullet$] by viewing elements $\vec{v}\in \bbA^g$ as column vectors, $P_{\Siegel}$ acts on $\bbA^g$ from the left via $ \bfgamma * \vec{v}  = \trans\bfgamma_a^{-1}\vec{v}$, for any $\bfgamma=\begin{pmatrix}\bfgamma_a & \bfgamma_b\\ \bfgamma_c & \bfgamma_d\end{pmatrix}\in P_{\Siegel}$;
    \item[$\bullet$] $P_{\Siegel}$ acts on $\GSp_{2g}$ via the left multiplication. 
\end{enumerate} 

Similarly, consider the linear dual $\scrW_{\Fl}^{\vee}$ of $\scrW_{\Fl}$. Then the total space of $\scrW_{\Fl}^{\vee}$ can be naturally identified with $$\scrW_{\Fl}^{\vee}\simeq P_{\Siegel}\backslash (\bbA^g\times \GSp_{2g})$$ where, by viewing elements $\vec{v}\in \bbA^g$ as row vectors, $P_{\Siegel}$ acts on $\bbA^g$ from the left via $ \bfgamma * \vec{v}  = \vec{v}\trans\bfgamma_a$, for any $\bfgamma=\begin{pmatrix}\bfgamma_a & \bfgamma_b\\ \bfgamma_c & \bfgamma_d\end{pmatrix}\in P_{\Siegel}$. Under this identification, global sections of $\scrW_{\Fl}^{\vee}$ are identified with $$\left\{\text{algebraic functions }\phi: \GSp_{2g}\rightarrow \bbA^g: \phi(\bfgamma\bfalpha)=\phi(\bfalpha)\cdot\trans\bfgamma_a,\,\,\forall \bfgamma\in P_{\Siegel},\,\, \bfalpha\in\GSp_{2g}\right\}.$$
For every $i=1,\ldots, g$, we consider a global section $\bfits_i$ of $\scrW_{\Fl}^{\vee}$ defined by $$\bfits_i(\bfalpha):= \text{the $i$-th row of }\trans\bfalpha_a$$ for all $\bfalpha=\begin{pmatrix}\bfalpha_a & \bfalpha_b\\ \bfalpha_c & \bfalpha_d\end{pmatrix}\in \GSp_{2g}$. If we write 
$$\bfits:=\begin{pmatrix}\bfits_1\\ \vdots\\ \bfits_g\end{pmatrix}\in (\scrW_{\Fl}^{\vee})^g$$ 
then we have $\bfits(\bfalpha)=\trans\bfalpha_a$.

By passing to the adic space $\adicFL$ and restricting to $\adicFL_w^{\times}$, the (algebraic) sheaves $\scrW_{\Fl}$ and $\scrW_{\Fl}^{\vee}$ yield (analytic) sheaves $\scrW_{\adicFL_w^{\times}}$ and $\scrW_{\adicFL_w^{\times}}^{\vee}$ on $\adicFL_w^\times$. We still use $\bfits_i$'s to denote the restrictions on $\adicFL_w^{\times}$ of the corresponding algebraic sections. By definition, the sections $\bfits_i$'s are non-vanishing on $\adicFL_w^{\times}$ and hence $\bfits_i^{\vee}$'s are well-defined sections on $\scrW_{\adicFL_w^{\times}}$. We similarly write
$$\bfits^{\vee}:=\begin{pmatrix}\bfits_1^{\vee} & \cdots &  \bfits_g^{\vee}\end{pmatrix}\in (\scrW_{\adicFL_w^{\times}})^g.$$
Moreover, the right action of $\Iw_{\GSp_{2g}}$ on $\adicFL_w^\times$ induces a right action of $\Iw_{\GSp_{2g}}$ on $\scrW_{\adicFL_w^{\times}}$. For later use, we would like to understand the behavior of $\bfits_i^{\vee}$'s under this action.

\begin{Lemma}\label{Lemma: invariance of fake Hasse invariants}
For any $\bfgamma=\begin{pmatrix}\bfgamma_a &\bfgamma_b\\ \bfgamma_c & \bfgamma_d\end{pmatrix}\in \Iw_{\GSp_{2g}}$, we have
$$ \bfgamma^*(\bfits^{\vee}) = \bfits^{\vee} \cdot \trans(\bfgamma_a + \bfitz\bfgamma_c)^{-1}.$$ The right-hand side means the right multiplication of matrices, where we view $\bfits^{\vee}$ as a $(1\times g)$-matrix with entries $\bfits_i^{\vee}$.
\end{Lemma}
\begin{proof}
To prove the identity, it suffices to check on the level of $(C, \calO_C)$-points. Using the identification
$$\adicFL^\times_w(C, \calO_C)=P_{\Siegel}(C)\backslash\GSp_{2g, w}(C),$$
the sections of $\scrW_{\adicFL_w^{\times}}$ can be identified with
$$\left\{\text{analytic functions }\phi: \GSp_{2g,w}\rightarrow C^g: \phi(\bfgamma\bfalpha)=\trans\bfgamma_a^{-1}\cdot\phi(\bfalpha),\,\,\forall \bfgamma\in P_{\Siegel}(C),\,\, \bfalpha\in\GSp_{2g,w}(C)\right\}.$$
Here, elements in $C^g$ are viewed as column vectors. Under this identification, $\bfits^{\vee}$ sends $\bfalpha\in \GSp_{2g,w}(C)$ to $\trans\bfalpha_a^{-1}$. Notice that a section $\phi:\GSp_{2g,w}(C)\rightarrow C^g$ of $\scrW_{\adicFL_w^{\times}}$ is determined by its restriction on $$\left\{\begin{pmatrix}\one_g & \bfitz\\ & \one_g\end{pmatrix}: \trans\bfitz\oneanti_g=\oneanti_g\bfitz,\,\,\max_{i, j}\inf_{h\in \Z_p}\{|\bfitz_{i,j}(\bfitx)-h|\}\leq p^{-w} \right\}.$$ Let $\bfalpha=\begin{pmatrix}\one_g & \bfitz\\ & \one_g\end{pmatrix}$. Then $\bfits^{\vee}(\bfalpha)=\one_g$ and $$(\bfgamma^*(\bfits^{\vee}))(\bfalpha)=\bfits^{\vee}(\bfalpha\bfgamma)=\bfits^{\vee} \left(\begin{pmatrix}\bfgamma_a+ \bfitz\bfgamma_c & \bfgamma_b+\bfitz\bfgamma_d\\ \bfgamma_c & \bfgamma_d\end{pmatrix}\right)= \trans(\bfgamma_a+\bfitz\bfgamma_c)^{-1}=\bfits^{\vee}(\bfalpha) \cdot \trans(\bfgamma_a+\bfitz\bfgamma_c)^{-1}$$ as desired.
\end{proof}

An immediate corollary of the lemma above is the following:

\begin{Corollary}\label{Corollary: Iw-action on s}
For any $\bfgamma = \begin{pmatrix}\bfgamma_a & \bfgamma_b\\ \bfgamma_c & \bfgamma_d\end{pmatrix}\in \Iw_{\GSp_{2g}}$, we have \[
    \bfgamma^*(\bfits) =  \trans(\bfgamma_a + \bfitz \bfgamma_c) \cdot \bfits.
\] 
\end{Corollary}


\subsection{The Hodge--Tate period map and the \texorpdfstring{$w$}{w}-ordinary locus}\label{subsection: Hodge--Tate period map and the w-ordinary locus}

We briefly recall the well-known Hodge--Tate period map in the setup of (toroidally compactified) Siegel modular variety.

The Hodge--Tate period map (see \cite[\S 1]{Pilloni-Stroh-CoherentCohomologyandGaloisRepresentations} and \S \ref{subsection: perfectoid Siegel modular variety}) is a morphism of adic spaces
$$\pi_{\HT}:\overline{\calX}_{\Gamma(p^{\infty})}\rightarrow \adicFL.$$
On the level of points, and away from the boundary, the Hodge--Tate period map has the following explicit description. Suppose 
$C$ is an algebraically closed and complete extension of $\Q_p$ and $(A, \lambda)$ is a principally polarised abelian variety over $C$. The Hodge--Tate sequence of $A$ is
\[0\rightarrow \Lie A\rightarrow T_pA\otimes_{\Z_p}C\rightarrow \omega_{A^{\vee}}\rightarrow 0,\] where ${\omega}_{A^{\vee}}$ is the dual of the Lie algebra of the dual abelian variety $A^{\vee}$ and the second last map is induced from the Hodge--Tate map $\HT_A:T_pA\rightarrow \omega_{A^{\vee}}$. 
Here, we ignore the Tate twist by fixing a compatible system of $p$-power roots of unity $(\zeta_{p^n})_{n\in \Z_{\geq 1}}$ in $\C_p$. Notice that every point $\bfitx\in \calX_{\Gamma(p^{\infty})}(C, \calO_C)$ corresponds to a quadruple $(A, \lambda, \psi_N, \psi)$ where $(A, \lambda, \psi_N)$ is a principally polarised abelian variety over $C$ with a principal level $N$ structure and $\psi$ is a symplectic isomorphism $\psi: V_p\simeq T_p A$. Then $\pi_{\HT}$ sends $\bfitx$ to the maximal lagrangian $$\Lie A\subset T_pA\otimes_{\Z_p}C\overset{\psi^{-1}}{\cong} V_p\otimes_{\Z_p} C.$$

One can extend such an explicit description to the boundary points as well using the language of 1-motives. The details are left to the interested readers.

\begin{Remark}\label{Remark: GSp2g-equivariance}
\normalfont There are right $\GSp_{2g}(\Q_p)$-actions on both sides of the Hodge--Tate map. The $\GSp_{2g}(\Q_p)$-action on $\adicFL$ is given in \S \ref{subsection: flag varieties}. As for the $\GSp_{2g}(\Q_p)$-action on $\overline{\calX}_{\Gamma(p^{\infty})}$, here we only describe the action away from the boundary. (A similar description applies to the boundary points as well, using the language of 1-motives.) Let $\bfgamma\in \GSp_{2g}(\Q_p)$ and let $m\in\Z$ such that $p^m\bfgamma\in M_{2g}(\Z_p)$ and $p^{m-1}\bfgamma\not\in M_{2g}(\Z_p)$. Choose $k\in \Z_{\geq 0}$ sufficiently large such that the kernel of $p^m\bfgamma: A[p^k]\rightarrow A[p^k]$ stabilises. Let $H\subset A[p^k]$ denote the corresponding kernel. Then $\bfgamma$ sends $(A, \lambda, \psi_N, \psi)$ to $(A'=A/H, \lambda', \psi'_N, \psi')$ where
\begin{itemize}
\item $\lambda'$ is the induced polarisation on $A'$;
\item $\psi'_N$ is induced from $\psi_N$ via the isomorphism $A[N]\simeq A'[N]$;
\item $\psi'$ is given by the composition
$$V_p\rightarrow V_p\otimes_{\Z_p}\Q_p\xrightarrow[]{\psi}T_pA\otimes_{\Z_p}\Q_p\rightarrow T_pA'\otimes_{\Z_p}\Q_p$$
with the first map $V_p\rightarrow V_p\otimes_{\Z_p}\Q_p$ sending $\vec{v}$ to $(p^m\bfgamma)^{-1}\vec{v}$.
\end{itemize}
One checks that $\pi_{\HT}$ is equivariant with respect to these $\GSp_{2g}(\Q_p)$-actions.
\end{Remark}

Let $\calG^{\univ}$ be the tautological semiabelian variety over $\overline{\calX}$ extending the universal abelian variety $\calA^{\univ}$ over $\calX$. Let $\pi: \calG^{\univ} \rightarrow \overline{\calX}$ be the structure morphism and let $$\underline{\omega}:=\pi_*\Omega^1_{\calG^{\univ}/\overline{\calX}}$$ which is a vector bundle of rank $g$ over $\overline{\calX}$. Pulling back along the projection $h: \overline{\calX}_{\Gamma(p^{\infty})}\rightarrow \overline{\calX}$, we obtain a vector bundle $$\underline{\omega}_{\Gamma(p^{\infty})}:=h^*\underline{\omega}$$ over $\overline{\calX}_{\Gamma(p^{\infty})}$.

\begin{Proposition}\label{Proposition: fake Hasse invariants and the coherent automorphic sheaf}
There is a natural isomorphism
$$\pi_{\HT}^*\scrW_{\adicFL}^{\vee}\cong \underline{\omega}_{\Gamma(p^{\infty})}.$$
\end{Proposition}
\begin{proof}
Let $\calA^{\univ}_{\Gamma(p^{\infty})}$ be the pullback of $\calA^{\univ}$ to $\calX_{\Gamma(p^{\infty})}$. Away from the boundary, we have a universal trivialisation $\psi^{\univ}:V_p\cong T_p\calA^{\univ}_{\Gamma(p^{\infty})}$. Let $\psi^{\univ, \vee}:V_p^{\vee}\cong T_p\calA^{\univ, \vee}_{\Gamma(p^{\infty})}$ be the dual trivialisation. The Hodge--Tate map on the universal abelian variety $\calA^{\univ}_{\Gamma(p^{\infty})}$ induces a map
$$\HT_{\Gamma(p^{\infty})}:V_p^{\vee}\overset{\psi^{\univ,\vee}}{\cong} T_p\calA^{\univ, \vee}_{\Gamma(p^{\infty})}\rightarrow \underline{\omega}_{\Gamma(p^{\infty})}|_{\calX_{\Gamma(p^{\infty})}}$$
which induces a surjection
$$\HT_{\Gamma(p^{\infty})}\otimes \id:V_p^{\vee}\otimes_{\Z_p}\scrO_{\calX_{\Gamma(p^{\infty})}}\twoheadrightarrow \underline{\omega}_{\Gamma(p^{\infty})}|_{\calX_{\Gamma(p^{\infty})}}.$$
According to \S \ref{subsection: perfectoid Siegel modular variety},
this surjection extends to a surjection \footnote{The map $\HT_{\Gamma(p^{\infty})}\otimes \id$ here coincides with the map $\HT_{\Gamma(p^{\infty})}\otimes \id:V_p\otimes_{\Z_p}\scrO_{\overline{\calX}_{\Gamma(p^{\infty})}}\rightarrow \underline{\omega}_{\Gamma(p^{\infty})}$ in \S \ref{subsection: perfectoid Siegel modular variety} via the symplectic isomorphism $V_p\simeq V_p^{\vee}$ sending $e_i$ to $-e_{2g+1-i}^{\vee}$, for $i=1, \ldots, g$, and sending $e_i$ to $e_{2g+1-i}^{\vee}$, for $i=g+1, \ldots, 2g$.}
$$\HT_{\Gamma(p^{\infty})}\otimes \id:V_p^{\vee}\otimes_{\Z_p}\scrO_{\overline{\calX}_{\Gamma(p^{\infty})}}\twoheadrightarrow \underline{\omega}_{\Gamma(p^{\infty})}$$
on the entire perfectoid Siegel modular variety.

Consequently, the sheaf $\pi_{\HT}^* \scrW_{\adicFL}^{\vee}$, being the universal maximal Lagrangian quotient of $V_p^{\vee}\otimes_{\Z_p} \scrO_{\overline{\calX}_{\Gamma(p^{\infty})}}$, coincides with $\underline{\omega}_{\Gamma(p^{\infty})}$.
\end{proof}

Recall the sections $\bfits_i$ of $\scrW_{\adicFL}^{\vee}$ defined in \S \ref{subsection: vector bundles on the flag variety}. We define sections $\fraks_i\in \underline{\omega}_{\Gamma(p^{\infty})}$ by \begin{equation}\label{eq: fake Hasse invariants}
    \fraks_i:=\pi_{\HT}^*\bfits_i.
\end{equation} 
From the construction, one sees that
\begin{equation}\label{eq:fake Hasse invariants basis}
\fraks_i=\HT_{\Gamma(p^{\infty})}(e_i^{\vee})
\end{equation}
for all $i=1, \ldots, g$. These $\fraks_i$'s are examples of \textit{fake Hasse invariants} studied in \cite{Scholze-2015}. We also write $$\fraks:=\begin{pmatrix}\fraks_1\\ \vdots\\ \fraks_g\end{pmatrix}=\pi_{\HT}^*\bfits.$$ 

To wrap up the section, we introduce the notion of ``$w$-ordinary locus'' of the perfectoid Siegel modular variety. In particular, it is an open subset of $\overline{\calX}_{\Gamma(p^{\infty})}$ which contains the usual ordinary locus.

\begin{Definition}\label{Definition: w-ordinary}
For every $w\in \Q_{>0}$, define
\[\overline{\calX}_{\Gamma(p^{\infty}), w}:=\pi_{\HT}^{-1}(\adicFL_{w}^\times).\]
We also define
\[\overline{\calX}_{\Gamma(p^n), w}:=h_n(\overline{\calX}_{\Gamma(p^{\infty}), w}), \quad\overline{\calX}_{\Iw^+, w}:=h_{\Iw^+}(\overline{\calX}_{\Gamma(p^{\infty}), w}),\quad \overline{\calX}_{\Iw, w}:=h_{\Iw}(\overline{\calX}_{\Gamma(p^{\infty}), w}), \quad \text{and }\quad \overline{\calX}_w:=h(\overline{\calX}_{\Gamma(p^{\infty}), w}),\]
where $h_n: \overline{\calX}_{\Gamma(p^{\infty})} \rightarrow \overline{\calX}_{\Gamma(p^{n})}$, $h_{\Iw^+}: \overline{\calX}_{\Gamma(p^{\infty})}\rightarrow \overline{\calX}_{\Iw^+}$,  $h_{\Iw}: \overline{\calX}_{\Gamma(p^{\infty})}\rightarrow \overline{\calX}_{\Iw}$, and $h: \overline{\calX}_{\Gamma(p^{\infty})}\rightarrow \overline{\calX}$ are the natural projections. The subspaces $\overline{\calX}_{\Gamma(p^{\infty}), w}$, $\overline{\calX}_{\Gamma(p^n), w}$, $\overline{\calX}_{\Iw^+, w}$, $\overline{\calX}_{\Iw, w}$, and $\overline{\calX}_w$ are called the \textit{\textbf{$w$-ordinary loci}} of $\overline{\calX}_{\Gamma(p^{\infty})}$, $\overline{\calX}_{\Gamma(p^n)}$, $\overline{\calX}_{\Iw^+}$, $\overline{\calX}_{\Iw}$, and $\overline{\calX}$, respectively.
\end{Definition}

We still denote by
$$\pi_{\HT}:\overline{\calX}_{\Gamma(p^{\infty}), w}\rightarrow \adicFL_{w}^\times$$
the restriction of the Hodge--Tate period map on the $w$-ordinary locus. It is equivariant under the right $\Iw_{\GSp_{2g}}$-actions on both sides.

Denote by $\frakz_{ij}:=\pi_{\HT}^*\bfitz_{ij}$ and $\frakz:=(\frakz_{i,j})_{1\leq i,j\leq g}=\pi_{\HT}^*\bfitz$. Let $\fraks_i^{\vee}:=\pi_{\HT}^*(\bfits_i^{\vee})$ and $$\fraks^{\vee}:=\begin{pmatrix}\fraks_1^{\vee} &  \cdots &  \fraks_g^{\vee}\end{pmatrix}=\pi_{\HT}^*(\bfits^{\vee}).$$
By Lemma \ref{Lemma: invariance of fake Hasse invariants} and Corollary \ref{Corollary: Iw-action on s}, we have
$$\bfgamma^*(\fraks^{\vee})  = \fraks^{\vee} \cdot \trans(\bfgamma_a + \frakz\bfgamma_c)^{-1}$$ 
and 
\begin{equation}\label{eq: action on fake Hasse invariants}
    \bfgamma^*\fraks =  \trans(\bfgamma_a + \frakz\bfgamma_c)\cdot \fraks
\end{equation}
for all $\bfgamma=\begin{pmatrix}\bfgamma_a &\bfgamma_b\\ \bfgamma_c & \bfgamma_d\end{pmatrix}\in \Iw_{\GSp_{2g}}$. We will need these sections $\fraks_i$'s and $\fraks_i^{\vee}$'s in \S \ref{subsection: admissibility} and \S \ref{subsection:comparison sheaf aip}.