\section{Kummer \'{e}tale and pro-Kummer \'{e}tale sites of log adic spaces}\label{Section: Kummer etale and pro-Kummer etale sites of log adic spaces}
In order to study the boundaries of various toroidal compactifications of Siegel varieties, we adopt the language of \emph{logarithmic adic spaces} established in \cite{Diao}. The purpose of \S\ref{subsection: review of Kummer etale and pro-Kummer etale sites} is to review the basic notions of log adic spaces, as well as their Kummer \'etale and pro-Kummer \'etale topologies, for convenience of the readers who are not familiar with the language. In \S\ref{subsection: main result}, we present an explicit calculation of the sheaf $R^i\nu_*\widehat{\scrO}_{X_{\proket}}$ which plays an essential role in the construction of the overconvergent Eichler--Shimura morphisms in \S\ref{subsection: OES}. Finally, in \S \ref{subsection: generalised projective formula}, we introduce the notion of \emph{Kummer \'etale Banach sheaves} and prove a (generalised) projection formula for those Kummer \'etale Banach sheaves that are \emph{admissible}. 

\paragraph{Notation.} We warn the reader that, in this section, (log) adic spaces will no longer be written in calligraphic font as we deal with more general (log) adic spaces, not only those studied in the main body of the text. 

\subsection{Review of log adic spaces}\label{subsection: review of Kummer etale and pro-Kummer etale sites}
Let $k$ be a nonarchimedean field (\emph{i.e.,} a field complete with respect to a nonarchimedean norm $|\cdot|:k\rightarrow \R_{\geq 0}$) and let $\calO_k=\{x\in k\,:\, |x|\leq 1\}$.  

\begin{Definition}
Let $X$ be a locally noetherian adic space over $\Spa(k, \calO_k)$.
\begin{enumerate}
\item[(i)] A \textbf{pre-log structure} on $X$ is a pair $(\scrM_{X}, \alpha)$ where $\scrM_{X}$ is a sheaf of monoids on $X_{\et}$ and $\alpha:\scrM_{X}\rightarrow \scrO_{X_{\et}}$ is a morphism of sheaves of (multiplicative) monoids. It is called a \textbf{log structure} if the induced morphism $\alpha^{-1}(\scrO^{\times}_{X_{\et}})\rightarrow \scrO^{\times}_{X_{\et}}$ is an isomorphism. In this case, the triple $(X, \scrM_{X}, \alpha)$ is called a \textbf{log adic space}. If the context is clear, we simply say that $X$ is a log adic space. 
\item[(ii)] For a pre-log structure $(\scrM_{X}, \alpha)$ on $X$, the \textbf{associated log structure} is $({}^{a}\!\!\!\!\scrM_X, {}^{a}\alpha)$ where ${}^{a}\!\!\!\!\scrM_X$ is given by the pushout 
$$
\begin{tikzcd}
\alpha^{-1}(\scrO^{\times}_{X_{\et}})\arrow[r] \arrow[d] & \scrM_X \arrow[d]  \\
       \scrO^{\times}_{X_{\et}}  \arrow[r] & {}^{a}\!\!\!\!\scrM_X \arrow[ul, phantom, "\ulcorner", very near start]
\end{tikzcd}
$$
and ${}^{a}\alpha:{}^{a}\!\!\!\!\scrM_{X}\rightarrow \scrO_{X_{\et}}$ is the induced morphism.
\item[(iii)] A \textbf{morphism} $f: (Y, \scrM_Y, \alpha_Y)\rightarrow (X, \scrM_X, \alpha_X)$ of log adic spaces is a morphism $f: Y\rightarrow X$ of adic spaces together with a morphism of sheaves of monoids $f^{\sharp}:f^{-1}\scrM_X\rightarrow \scrM_Y$ such that the diagram
$$
\begin{tikzcd}
{f^{-1}\scrM_X} \arrow[r, "f^{\sharp}"] \arrow[d, "f^{-1}\alpha_X"'] & \scrM_Y \arrow[d, "\alpha_Y"]  \\
        {f^{-1}\scrO_{X_{\et}}} \arrow[r] & {\scrO_{Y_{\et}}} 
\end{tikzcd}
$$
commutes. Moreover, the log structure associated with the pre-log structure $f^{-1}\scrM_X\rightarrow f^{-1}\scrO_{X_{\et}}\rightarrow \scrO_{Y_{\et}}$ is called the \textbf{pullback log structure}, denoted by $f^*\scrM_X$. We say that $f$ is \textbf{strict} if $f^*\scrM_X\xrightarrow[]{\sim} \scrM_Y$.
\end{enumerate}
\end{Definition}

\begin{Definition}
\begin{enumerate}
\item[(i)] Let $(X, \scrM_{X}, \alpha)$ be a locally noetherian log adic space as above. Let $P$ be a monoid and let $P_{X}$ denote the associated constant sheaf of monoids on $X_{\et}$. A \textbf{chart of $X$ modeled on $P$} is a morphism of sheaves of monoids $\theta: P_{X}\rightarrow \scrM_{X}$ such that $\alpha(\theta(P_{X}))\subset \scrO^+_{X_{\et}}$ and such that the log structure associated with the pre-log structure $\alpha\circ\theta:P_{X}\rightarrow \scrO_{X_{\et}}$ is isomorphic to $\scrM_{X}$. We say that the chart is \textbf{fs} if $P$ is fine and saturated. 
\item[(ii)] A locally noetherian log adic space is called an \textbf{fs log adic space} if it \'etale locally admits charts modeled on fs monoids.
\item[(iii)] Let $f: (Y, \scrM_{Y}, \alpha_{Y})\rightarrow (X, \scrM_{X}, \alpha_{X})$ be a morphism between locally noetherian log adic spaces. A \textbf{chart} of $f$ consists of charts $\theta_{X}:P_{X}\rightarrow \scrM_{X}$ and $\theta_{Y}: Q_{Y}\rightarrow \scrM_{Y}$ and a homomorphism $u:P\rightarrow Q$ such that the diagram
 \[
        \xymatrix{ {P_{Y}} \ar^-u[r] \ar^-{\theta_{X}}[d] & {Q_{Y}} \ar^-{\theta_{Y}}[d] \\
        {f^{-1}\scrM_{X}} \ar^-{f^\sharp}[r] & {\scrM_{Y}} }
    \]
commutes. We say that the chart is \textbf{fs} if both $P$ and $Q$ are fs. When the context is clear, we simply say that $u:P\rightarrow Q$ is a chart of $f$.
\end{enumerate}
\end{Definition}

Below are two typical examples of locally noetherian fs log adic spaces. In this paper, all of the toroidally compactified Siegel varieties (equipped with logarithmic structures associated with the boundary divisors) have the form as in Example \ref{Example: normal crossings}.

\begin{Example}\label{Example: unit disc}
\normalfont Let $n>0$ be an integer. Consider the \textit{\textbf{$n$-dimensional unit disc}}
\[\bbD^n:=\Spa(k\langle T_1, \ldots, T_n\rangle, \calO_k\langle T_1, \ldots, T_n\rangle),\]
equipped with the log structure associated with the pre-log structure induced by \[\Z^n_{\geq 0}\rightarrow k\langle T_1, \ldots, T_n\rangle,\,\,\,\, (a_1, \ldots, a_n)\mapsto T_1^{a_1}\cdots T_n^{a_n}.\] Clearly, $\bbD^n$ is modeled on the fs chart $\Z^n_{\geq 0}$.\qed
\end{Example}

\begin{Example}\label{Example: normal crossings}
\normalfont Let $X$ be a smooth rigid analytic variety over $k$, viewed as an adic space over $\Spa(k, \calO_k)$ via \cite[(1.1.11)]{Huber-2013}. Let $D\subset X$ be a \textit{\textbf{normal crossings divisor}} in the sense of \cite[Example 2.3.17]{Diao}. Namely, $\iota:D\hookrightarrow X$ is a closed immersion such that, analytic locally, $X$ and $D$ are of the form $S\times \bbD^n$ and $S\times \{T_1\cdots T_n=0\}$, where $S$ is a smooth connected rigid analytic variety and $\iota$ is the pullback of the natural inclusion $\{T_1\cdots T_n=0\}\hookrightarrow \bbD^n$. We equip $X$ with the log structure 
\[\scrM_{X}=\{f\in \scrO_{X_{\et}}\,|\, f \textrm{ is invertible on }X\smallsetminus D\}\]
with $\alpha:\scrM_{X}\rightarrow \scrO_{X_{\et}}$ being the natural inclusion. This is the \textit{\textbf{divisorial log structure}} associated with the divisor $D$. This log structure agrees with the pullback of the log structure on $\bbD^n$ constructed in Example \ref{Example: unit disc}. \qed
\end{Example}

Log adic spaces in Example \ref{Example: normal crossings} are special cases of \emph{log smooth} ones. For later use, we recall the definition of log smoothness.

For any monoid $P$ and any commutative ring $T$, we write $T[P]$ for the associated monoid algebra.

\begin{Definition}
Let $X$ be a locally noetherian adic space over $\Spa(k, \calO_k)$ and let $P$ be a finitely generated monoid. For any affinoid open subspace $\Spa(R, R^+)\subset X$, let $(R\langle P\rangle, R^+\langle P\rangle)$ be the completion of $(R[P], R^+[P])$. By gluing the morphisms $\Spa(R\langle P\rangle, R^+\langle P\rangle)\rightarrow \Spa(R, R^+)$, we obtain a morphism $X\langle P\rangle\rightarrow X$. Moreover, we equip $X\langle P\rangle$ with the log structure modeled on the chart $P$; \emph{i.e.}, the one locally induced by $P\rightarrow R\langle P\rangle$.
\end{Definition}

\begin{Definition}\label{Definition: log smooth}
Let $f: Y\rightarrow X$ be a morphism between locally noetherian fs log adic spaces. We say that $f$ is \textbf{log smooth} if \'etale locally $f$ admits an fs chart $u:P\rightarrow Q$ such that
\begin{enumerate}
\item[(i)] the kernel and the torsion part of the cokernel of $u^{\mathrm{gp}}:P^{\mathrm{gp}}\rightarrow Q^{\mathrm{gp}}$ are finite groups of order invertible in $\scrO_X$; and
\item[(ii)] $f$ and $u$ induce a morphism $Y\rightarrow X\times_{X\langle P\rangle}X\langle Q\rangle$ of log adic spaces whose underlying morphism of adic spaces is \'etale.
\end{enumerate}
A locally noetherian fs log adic space $X$ is \textbf{log smooth} if the structure morphism $X\rightarrow \Spa(k, \calO_k)$ is log smooth, where $\Spa(k, \calO_k)$ is equipped with the trivial log structure.
\end{Definition}

Now we introduce the notion of Kummer \'etale morphisms and the Kummer \'etale site.

\begin{Definition}\label{Definition: Kummer etale morphism}
\begin{enumerate}
\item[(i)] An injective homomorphism $u:P\rightarrow Q$ of fs monoids is called \textbf{Kummer} if for every $a\in Q$, there exists some integer $n >0$ such that $na\in u(P)$.
\item[(ii)] A morphism (resp., finite morphism) $f: Y\rightarrow X$ of locally noetherian fs log adic spaces is called \textbf{Kummer \'etale} (resp., \textbf{finite Kummer \'etale}) if \'etale locally on $X$ and $Y$, $f$ admits an fs chart $u:P\rightarrow Q$ which is Kummer with $|Q^{\mathrm{gp}}/u^{\mathrm{gp}}(P^{\mathrm{gp}})|$ invertible on $\scrO_Y$, and such that $f$ and $u$ induce a morphism $Y\rightarrow X\times_{X\langle P\rangle} X\langle Q\rangle$ of log adic spaces whose underlying morphism of adic spaces is \'etale (resp., finite \'etale).
\item[(iii)] If a Kummer \'etale (resp., finite Kummer \'etale) morphism is strict, we say it is \textbf{strictly \'etale} (resp., \textbf{strictly finite \'etale}).
\end{enumerate}
\end{Definition}

\begin{Remark}
\normalfont By \cite[Lemma 4.1.10]{Diao}, if $f:Y\rightarrow X$ is a Kummer \'etale morphism between locally noetherian fs log adic spaces and if $X$ admits a chart modeled on a sharp fs monoid $P$, then, \'etale locally on $X$ and $Y$, the morphism $f$ admits a Kummer fs chart $P\rightarrow Q$ with $Q$ being sharp.
\end{Remark}

\begin{Definition} 
Let $X$ be a locally noetherian fs log adic space. The \textbf{Kummer \'etale site} $X_{\ket}$ (resp., \textbf{finite Kummer \'etale site} $X_{\textrm{fket}}$) of $X$ is defined as follows. The underlying category is the full subcategory of the category of locally noetherian fs log adic spaces consisting of objects that are Kummer \'etale (resp., finite Kummer \'etale) over $X$. The coverings are given by the topological coverings. 

The \textbf{structure sheaf} $\scrO_{X_{\ket}}$ (resp., \textbf{integral structure sheaf} $\scrO_{X_{\ket}}^{+})$ on $X_{\ket}$ is defined to be the presheaf sending $U\mapsto \scrO_U(U)$ (resp., $U\mapsto \scrO_U^+(U)$). We also write $\scrM_{X_{\ket}}$ for the presheaf sending $U\mapsto \scrM_U(U)$. By \cite[Theorem 4.3.1, Proposition 4.3.4]{Diao}, these presheaves are indeed sheaves. 
\end{Definition}

\begin{Proposition}[Corollary 4.4.18, \cite{Diao}]\label{Proposition: Kummer etale fundamental group}
Let $X$ be a connected locally noetherian fs log adic space and let $\xi$ be a log geometric point (see \cite[Definition 4.4.2]{Diao}). Then there is an equivalence of categories
\[F_{X}: X_{\text{fk\'et}}\xrightarrow[]{\sim} \pi_1^{\ket}(X, \xi)-\FSets\] sending $Y\mapsto Y_{\xi}:=\Hom_{X}(\xi, Y)$, where the $\pi_1^{\ket}(X, \xi)-\FSets$ denotes the category of finite sets equipped with a continuous action of the \textbf{Kummer \'etale fundamental group} $\pi_1^{\ket}(X, \xi)$.

For any two log geometric points $\xi$ and $\xi'$, the fundamental groups $\pi_1^{\ket}(X, \xi)$ and $\pi_1^{\ket}(X, \xi')$ are isomorphic. Hence, we may omit ``$\xi$'' from the notation whenever the context is clear.
\end{Proposition}

\begin{Lemma}\label{Kummer etale Galois cover}
Assume $k$ is of characteristic $0$. Let $X$ and $Y$ be locally noetherian fs log adic spaces whose underlying adic spaces are smooth connected rigid analytic varieties over $k$. Suppose the log structures on $X$ and $Y$ are the divisorial log structures associated with the normal crossing divisors $D\subset X$ and $E\subset Y$ as in Example \ref{Example: normal crossings}. Let $U=X\smallsetminus D$ and $V=Y\smallsetminus E$. Suppose we have a finite Kummer \'etale surjective morphism $f:Y\rightarrow X$ such that $f^{-1}(U)=V$ and that $f|_{V}:V\rightarrow U$ is a finite \'etale Galois cover with Galois group $G$. Then $f$ is a finite Kummer \'etale Galois cover with Galois group $G$.
\end{Lemma}

\begin{proof}
According to Proposition \ref{Proposition: Kummer etale fundamental group}, we have equivalences of categories
\[F_X:X_{\text{fk\'et}}\xrightarrow[]{\sim} \pi_1^{\ket}(X)-\FSets\]
and 
\[F_U: U_{\text{f\'et}}\xrightarrow[]{\sim} \pi_1^{\et}(U)-\FSets.\]
We have to show that $G$ is a finite quotient of $\pi_1^{\ket}(X)$ and, under the equivalence $F_X$, $Y$ corresponds to the finite set $G$ equipped with the natural $\pi_1^{\ket}(X)$-action.

By \cite[Proposition 4.2.1]{Diao} and \cite[Theorem 1.6]{Hansen-2020}, we have an equivalence of categories between $X_{\text{fk\'et}}$ and $U_{\text{f\'et}}$, under which $Y$ corresponds to $V$. It also induces a natural isomorphism $\pi_1^{\ket}(X)\cong \pi_1^{\et}(U)$ making the following diagram commutative. 
 \[
        \xymatrix{ {X_{\text{fk\'et}}} \ar^-{\sim}[r] \ar^{F_X}_{\sim}[d] & {U_{\text{f\'et}}} \ar^-{F_U}_{\sim}[d] \\
        {\pi_1^{\ket}(X)-\FSets} \ar^-{\sim}[r] & {\pi_1^{\et}(U)-\FSets} }
    \]
Since $V$ corresponds to the finite set $G$ under the equivalence $F_U$, we are done.
\end{proof}

Finally, we introduce the pro-Kummer \'etale site. For the rest of \S \ref{subsection: review of Kummer etale and pro-Kummer etale sites}, the nonarchimedean field $k$ is assumed to be an extension of $\Q_p$. 
\begin{Definition}
Let $X$ be a locally noetherian fs log adic space over $\Spa(k,\calO_k)$.
\begin{enumerate}
\item[(i)] The \textbf{pro-Kummer \'{e}tale site} $X_{\proket}$ of $X$ is defined as follows. The underlying category is the full subcategory of $\textrm{pro-}X_{\ket}$ consisting of cofiltered inverse limit $Y=\varprojlim_{i\in I}Y_i$ with $Y_i\in X_{\ket}$ such that the transition morphisms $Y_i\rightarrow Y_j$ are finite Kummer \'{e}tale and are surjective for sufficiently large $i$. Such an inverse limit if called a \textbf{pro-Kummer \'etale presentation} of $Y$. As for the coverings, we refer the readers to \cite[Definition 5.1.1, 5.1.2]{Diao} for details.
\item[(ii)] There is a natural projection of sites \[\nu: X_{\proket}\rightarrow X_{\ket}.\] The \textbf{structure sheaves} on $X_{\proket}$ are given by
$$
\scrO_{X_{\proket}}^+:=\nu^{-1}\scrO_{X_{\ket}}^+, \quad \scrO_{X_{\proket}}:=\nu^{-1}\scrO_{X_{\ket}}$$
and the \textbf{completed structure sheaves} are given by
$$
\quad \widehat{\scrO}_{X_{\proket}}^+:=\varprojlim_{n}\left(\scrO_{X_{\proket}}/p^n\right), \quad \widehat{\scrO}_{X_{\proket}}:=\widehat{\scrO}_{X_{\proket}}^+[1/p].
$$
We also write $\scrM_{X_{\proket}}:=\nu^{-1}(\scrM_{\ket})$ together with a natural morphism $\alpha:\scrM_{\proket}\rightarrow \scrO_{X_{\proket}}$.
\end{enumerate}
\end{Definition}
The pro-Kummer \'etale topology admits a convenient basis consisting of the \emph{log affinoid perfectoid objects}.

\begin{Definition}
An object $U$ in $X_{\proket}$ is called \textbf{log affinoid perfectoid} if it admits a pro-Kummer \'etale presentation $U = \varprojlim_{i \in I} U_i$ such that
\begin{enumerate}
 \item[(i)] There is an initial object $0 \in I$;
 \item[(ii)] Each $U_i= (\Spa(R_i, R_i^+)$ is affinoid and admits a chart modeled on a sharp fs monoid $P_i$ such that each transition morphism $U_j \rightarrow U_i$ is modeled on a Kummer chart $P_i \to P_j$;
\item[(iii)] The affinoid algebra $(R, R^+) := \big(\varinjlim_{i \in I} \, (R_i, R_i^{+})\big)^\wedge$ is a perfectoid affinoid algebra, where the completion is with respect to the $p$-adic topology;
\item[(iv)] The monoid $P := \varinjlim_{i \in I} P_i$ is \textbf{$n$-divisible}, for all $n \in \Z_{\geq 1}$. Namely, the $n$-th multiple map $[n]:P\rightarrow P$ is surjective for all $n\in\Z_{\geq 1}$.
    \end{enumerate}
    Such a presentation $U = \varprojlim_{i\in I} U_i$ is called a \textbf{perfectoid presentation} of $U$.
\end{Definition}

\begin{Proposition}[Proposition 5.3.12, \cite{Diao}]
The log affinoid perfectoid objects in $X_{\proket}$ form a basis of the pro-Kummer \'etale site.
\end{Proposition}

\begin{Proposition}[Theorem 5.4.3, \cite{Diao}]\label{Proposition: almost vanishing}
Let $U\in X_{\proket}$ be a log affinoid perfectoid object, with the associated perfectoid space $\widehat{U}=\Spa(R, R^+)$. Then
\begin{enumerate}
\item[(i)] For each $n\in \Z_{\geq 1}$, we have $\scrO^+_{X_{\proket}}(U)/p^n\cong R^+/p^n$, and it is canonically almost isomorphic to $(\scrO^+_{X_{\proket}}/p^n)(U)$.
\item[(ii)] For each $n\in \Z_{\geq 1}$ and $i\in \Z_{\geq 1}$, $H^i(U, \scrO^+_{X_{\proket}}/p^n)$ is almost equal to zero. Consequently, $H^i(U, \widehat{\scrO}^+_{X_{\proket}})$ is almost equal to zero.
\item[(iii)] We have $\widehat{\scrO}^+_{X_{\proket}}(U)\cong R^+$ and $\widehat{\scrO}_{X_{\proket}}(U)\cong R$. Moreover, $\widehat{\scrO}^+_{X_{\proket}}(U)$ is canonically isomorphic to the $p$-adic completion of $\scrO^+_{X_{\proket}}(U)$.
\end{enumerate}
\end{Proposition}

\begin{Example}\label{Example: basic example of profinite Galois cover}
\normalfont We recall the following example from \cite[\S 6]{Diao}. Let $P$ be a sharp fs monoid. Consider
\[\mathbb{E}:=\Spa(\C_p\langle P\rangle, \calO_{\C_p}\langle P\rangle)\]
equipped with the natural log structure modeled on chart $P$. (If $P=\Z_{\geq 0}^n$, then $\mathbb{E}$ is just the $n$-dimensional unit disc in Example \ref{Example: unit disc}.) For each $m\in \Z_{>0}$, let $\frac{1}{m}P$ denote the sharp fs monoid containing $P$ such that the inclusion $P\hookrightarrow \frac{1}{m}P$ is isomorphic to the $m$-th multiple map $[m]:P\rightarrow P$. Define
\[\mathbb{E}_m:=\Spa(\C_p\langle \frac{1}{m}P\rangle, \calO_{\C_p}\langle \frac{1}{m}P\rangle)\]
equipped with the natural log structure modeled on the chart $\frac{1}{m}P$.  If $m|m'$, there is a natural finite Kummer \'etale morphism $\mathbb{E}_{m'}\rightarrow \mathbb{E}_{m}$ modeled on the chart $\frac{1}{m}P\hookrightarrow \frac{1}{m'}P$. According to \cite[Proposition 4.1.6]{Diao}, the morphism $\mathbb{E}_m\rightarrow \mathbb{E}$ is actually finite Kummer \'etale Galois with Galois group 
$$
\Gamma_{/m}:=\Hom\big((\frac{1}{m}P)^{\mathrm{gp}}/P^{\mathrm{gp}}, \boldsymbol{\mu}_{\infty}\big),
$$
where $\boldsymbol{\mu}_{\infty}$ denotes the group of all roots of unity in $\C_p$. Let $P_{\Q_{\geq 0}}:=\varinjlim_m (\frac{1}{m}P)$. It turns out
\[\widetilde{\mathbb{E}}:=\varprojlim_m \mathbb{E}_m\in \mathbb{E}_{\proket}\] is a log affinoid perfectoid object, with associated perfectoid space
\[\widehat{\widetilde{\mathbb{E}}}=\Spa(\C_p\langle P_{\Q_{\geq 0}}\rangle, \calO_{\C_p}\langle P_{\Q_{\geq 0}}\rangle).\]\qed
\end{Example}

Following \cite[Definition 6.1.2]{Diao}, a pro-Kummer \'{e}tale cover $Y\rightarrow X$ is called a \textit{\textbf{Galois cover with (profinite) Galois group $G$}} if there exists a presentation $Y=\varprojlim_{i} Y_i$ such that each $Y_i\rightarrow X$ is a finite Kummer \'{e}tale cover with Galois group $G_i$ and $G\cong \varprojlim_i G_i$.

For example, $\widetilde{\mathbb{E}}$ is a Galois cover over $\mathbb{E}$ with profinite Galois group \[\Gamma\cong \varprojlim_m \Gamma_{/m}=\varprojlim_m \Hom((\frac{1}{m}P)^{\mathrm{gp}}/P^{\mathrm{gp}}, \boldsymbol{\mu}_{\infty}) \cong\Hom(P^{\mathrm{gp}}_{\Q\geq 0}/P^{\mathrm{gp}}, \boldsymbol{\mu}_{\infty})\]
(cf. \cite[(6.1.4)]{Diao}).


\subsection{Calculation of \texorpdfstring{$R^i\nu_*\widehat{\scrO}_{X_{\proket}}$}{Lg}}\label{subsection: main result}
In this subsection, we study the sheaves $R^i\nu_*\widehat{\scrO}_{X_{\proket}}$ following the calculations in \cite{Scholze-perfectoid-survey},  \cite{Scholze_2013}, and \cite{CHJ-2017}, but in the context of log adic spaces. Throughout this subsection, $X$ is an fs log adic space that is log smooth over $\Spa(\C_p, \calO_{\C_p})$ (cf. Definition \ref{Definition: log smooth}). We will omit the subscript ``prok\'et'' from $\scrO_{X_{\proket}}$, $\scrO^+_{X_{\proket}}$, $\widehat{\scrO}_{X_{\proket}}$, $\widehat{\scrO}^+_{X_{\proket}}$, etc., whenever this causes no confusion.

By definition, $R^i\nu_*\widehat{\scrO}_{X}$ (resp., $R^i\nu_*\widehat{\scrO}^+_{X}$) is the sheaf on $X_{\ket}$ associated with the presheaf 
\[U\mapsto H^i(U_{\proket}, \widehat{\scrO}_{X})\,\,\,\,\,\,\,\,\big(\textrm{resp., }U\mapsto H^i(U_{\proket}, \widehat{\scrO}^+_{X})\big).\]

For every $U\in X_{\ket}$, $U$ is log smooth over $\Spa(\C_p, \calO_{\C_p})$. By \cite[Proposition 3.1.10]{Diao}, \'etale locally on $U$ there exists a \textit{\textbf{toric chart}} $U\rightarrow \mathbb{E}=\Spa(\C_p\langle P\rangle, \calO_{\C_p}\langle P\rangle)$ for some sharp fs monoid $P$,  {\it i.e.,} a strictly \'etale morphism $U\rightarrow \mathbb{E}=\Spa(\C_p\langle P\rangle, \calO_{\C_p}\langle P\rangle)$ that is a composition of rational localisations and finite \'etale morphisms. For such toric charts, we are able to calculate $H^i(U_{\proket}, \widehat{\scrO}^+_{X})$ and $H^i(U_{\proket}, \widehat{\scrO}_{X})$ in an explicit way.

\begin{Lemma}\label{Lemma: Cartan-Leray}
Suppose $U\in X_{\ket}$ is equipped with a toric chart $U\rightarrow \mathbb{E}$ as above. 
\begin{enumerate}
\item[(i)] For every $i\in \Z_{\geq 0}$ and $m\in\Z_{\geq 1}$, there is a natural injection
\[H^i_{\cts}(\Gamma, \scrO^+_{X_{\ket}}(U)/p^m)^a\hookrightarrow H^i(U_{\proket}, \scrO^+_{X}/p^m)^a\]
with cokernel killed by $p$, where $\Gamma$ is equipped with the profinite topology, $\scrO^+_{X_{\ket}}(U)/p^m$ is equipped with the discrete topology, and $\Gamma$ acts trivially on $\scrO^+_{X_{\ket}}(U)/p^m$.
\item[(ii)] 
For every $i\in\Z_{\geq 0}$, there is a natural injection
\[H^i_{\cts}(\Gamma, \scrO^+_{X_{\ket}}(U))^a\hookrightarrow H^i(U_{\proket}, \widehat{\scrO}^+_{X})^a\]
with cokernel killed by $p$, where $\Gamma$ is equipped with the profinite topology, $\scrO^+_{X_{\ket}}(U)$ is equipped with the $p$-adic topology, and $\Gamma$ acts trivially on $\scrO^+_{X_{\ket}}(U)$. By inverting $p$, we obtain an isomorphism
\[H^i_{\cts}(\Gamma, \scrO_{X_{\ket}}(U))\xrightarrow[]{\sim} H^i(U_{\proket}, \widehat{\scrO}_{X}).\]
\item[(iii)]
For every $i\in \Z_{\geq 0}$ and $m\in \Z_{\geq 1}$, by choosing an isomorphism $P^{\mathrm{gp}}\simeq \Z^n$, there is a natural almost injection $$\bigwedge^i (\scrO^+_{U_{\ket}}/p^m)^n\hookrightarrow R^i\nu_* (\scrO^+_{U_{\proket}}/p^m)$$ whose cokernel is killed by $p$. This induces a natural almost injection $$\bigwedge^i (\scrO^+_{U_{\ket}})^n\hookrightarrow R^i\nu_* \widehat{\scrO}^+_{U_{\proket}}$$
with cokernel killed by $p$. Inverting $p$, we obtain an isomorphism
$$\bigwedge^i (\scrO_{U_{\ket}})^n\simeq R^i\nu_* \widehat{\scrO}_{U_{\proket}}.$$ In particular, the sheaf $R^i\nu_*\widehat{\scrO}_{X}$ is a locally free $\scrO_{X_{\ket}}$-module. 
\end{enumerate}
\end{Lemma}

\begin{proof}
\begin{enumerate}
\item[(i)] Recall the log affinoid perfectoid Galois cover $\widetilde{\mathbb{E}}\rightarrow \mathbb{E}$ (with profinite Galois group $\Gamma$) constructed in Example \ref{Example: basic example of profinite Galois cover}. Consider \[\widetilde{U}:=U\times_{\mathbb{E}}\widetilde{\mathbb{E}}\in X_{\proket}.\]
By \cite[Lemma 5.3.8]{Diao}, $\widetilde{U}$ is also log affinoid perfectoid and $\widetilde{U}\rightarrow U$ is a Galois cover with the same Galois group. We obtain the Cartan--Leray spectral sequence (see \cite[Remark 2.25]{CHJ-2017}) \[E_2^{i,j}= H_{\cts}^i(\Gamma, H^j(\widetilde{U}, \scrO_{X}^+/p^m))\Rightarrow H^{i+j}(U_{\proket}, \scrO_{X}^+/p^m).\] By Proposition \ref{Proposition: almost vanishing} (ii), $H^j(\widetilde{U}, \scrO_{X}^+/p^m)$ is almost zero for all $j\in \Z_{\geq 1}$. Therefore, we have an almost isomorphism 
\begin{equation}\label{eq: almost isomorphism Cartain-Leray}
H^i(U_{\proket}, \scrO_{X}^+/p^m)^a\simeq H_{\cts}^i(\Gamma, (\scrO_{X}^+/p^m)(\widetilde{U}))^a.
\end{equation}

On the other hand, by \cite[Lemma 6.1.7, Remark 6.1.8]{Diao}, the natural morphism 
\[H^i(\Gamma, (\scrO_{X_{\ket}}^+/p^m)(U))\rightarrow H^i(\Gamma, (\scrO_{X}^+/p^m)(\widetilde{U}))\]
is injective with cokernel killed by $p$ for all $i\in \Z_{\geq 0}$. Combining this with the almost isomorphism (\ref{eq: almost isomorphism Cartain-Leray}), we obtain the desired almost injection.

\item[(ii)] By an almost version of \cite[Lemma 3.18]{Scholze_2013} and Proposition \ref{Proposition: almost vanishing} (i) (ii), we see that the inverse system $\{\scrO_{X}^+/p^m: m\in \Z_{\geq 1}\}$ has almost vanishing higher inverse limits on the pro-Kummer \'{e}tale site. Therefore, we obtain almost isomorphisms
\[H^i(U_{\proket}, \widehat{\scrO}_{X}^+)^a\cong \varprojlim_m H^i(U_{\proket}, \scrO_{X}^+/p^m)^a \simeq \varprojlim_m H_{\cts}^i(\Gamma, (\scrO_{X}^+/p^m)(\widetilde{U}))^a.\]

On the other hand, for every $i\in \Z_{\geq 0}$, we claim that there is a natural isomorphism
\[H_{\cts}^i(\Gamma, \scrO_{X_{\ket}}^+(U))\cong \varprojlim_m H_{\cts}^i(\Gamma, (\scrO_{X_{\ket}}^+/p^m)(U)).\] Indeed, by the same arguments as in the proof of \cite[Theorem 2.7.5]{NSW-cohomology}, there is a short exact sequence \[0\rightarrow R^1\varprojlim_{m} H_{\cts}^{i-1}(\Gamma, (\scrO_{X_{\ket}}^+/p^m)(U))\rightarrow H_{\cts}^{i}(\Gamma, \scrO_{X_{\ket}}^+(U))\rightarrow \varprojlim_m H_{\cts}^{i}(\Gamma, (\scrO_{X_{\ket}}^+/p^m)(U))\rightarrow 0.\] 
It suffices to show that 
\[R^1\varprojlim_m H_{\cts}^{i-1}(\Gamma, (\scrO_{X_{\ket}}^+/p^m)(U))=0.\] 
Notice that $P^{\mathrm{gp}}$ is a finitely generated torsion-free abelian group. By choosing a $\Z$-basis of $P^{\mathrm{gp}}$, we obtain an isomorphism $\Gamma\cong \widehat{\Z}(1)^n$ of profinite groups which induces an isomorphism
\[
H^{i-1}_{\cts}(\Gamma, (\scrO^+_{X_{\ket}}/p^m)(U))\simeq \bigwedge^{i-1}(\scrO^+_{X_{\ket}}(U)/p^m)^n.
\]
Thus, for every $m'>m$, the transition map \[H_{\cts}^{i-1}(\Gamma, (\scrO_{X_{\ket}}^+/p^{m'})(U))\rightarrow H_{\cts}^{i-1}(\Gamma, (\scrO_{X_{\ket}}^+/p^{m})(U))\] is a surjection. Hence, the inverse system $\{H_{\cts}^{i-1}(\Gamma, (\scrO_{X_{\ket}}^+/p^m)(U)): m\in \Z_{>0}\}$ satisfies the Mittag-Leffler condition which yields the desired vanishing of $R^1\lim$.

Putting everything together, we obtain a natural injection
$$
H_{\cts}^i(\Gamma, \scrO_{X_{\ket}}^+(U))^a\cong \varprojlim_m H_{\cts}^i(\Gamma, (\scrO_{X_{\ket}}^+/p^m)(U))^a\hookrightarrow \varprojlim_m H_{\cts}^i(\Gamma, (\scrO_{X}^+/p^m)(\widetilde{U}))^a\cong H^i(U_{\proket}, \widehat{\scrO}_{X}^+)^a
$$
whose cokernel is killed by $p$, as desired.

\item[(iii)] We show that the restriction of $R^i\nu_*\widehat{\scrO}_X$ on $U_{\ket}$ is isomorphic to the free $\scrO_{U_{\ket}}$-module $\bigwedge^i (\scrO_{U_{\ket}})^n$. In fact, as a byproduct of the computation above, we have isomorphisms (depending on the fixed choice of the identification $\Gamma\cong \widehat{\Z}(1)^n$)
$$H^i_{\cts}(\Gamma, \scrO^+_{X_{\ket}}(U))\cong \varprojlim_m H_{\cts}^i(\Gamma, (\scrO_{X_{\ket}}^+/p^m)(U))\simeq \varprojlim_m\bigwedge^i(\scrO^+_{X_{\ket}}(U)/p^m)^n=\bigwedge^i(\scrO^+_{X_{\ket}}(U))^n.$$
Inverting $p$, we obtain an isomorphism
$$H^i(U_{\proket}, \widehat{\scrO}_X)\simeq H^i_{\cts}(\Gamma, \scrO_{X_{\ket}}(U))\simeq \bigwedge^i(\scrO_{X_{\ket}}(U))^n.$$

Consider $V\in U_{\ket}$ such that $V\rightarrow U$ admits a chart $P\rightarrow P'$ and such that $V\rightarrow U$ factors as 
$$V\rightarrow U'\times_{U'\langle P\rangle}U'\langle P'\rangle \rightarrow U' \rightarrow U$$
where 
\begin{itemize}
\item $U'\subset U$ is a strictly \'etale morphism which is a composition of finite \'etale morphisms and rational localisations;
\item $P\rightarrow P'$ is isomorphic to the $m$-th multiple map $[m]: P\rightarrow P$;
\item $V\rightarrow U'\times_{U'\langle P\rangle}U'\langle P'\rangle$ is a strictly \'etale morphism which is a composition of finite \'etale morphisms and rational localisations.
\end{itemize}

Notice that such a $V$ admits a toric chart $V\rightarrow \mathbb{E}'$ where $\mathbb{E}'=\Spa(\C_p\langle P'\rangle, \calO_{\C_p}\langle P'\rangle)$. Repeating the argument above, we arrive at an isomorphism
$$H^i_{\cts}(\Gamma', \scrO_{X_{\ket}}(V))\simeq H^i(V_{\proket}, \widehat{\scrO}_X)$$
where 
$$\Gamma':= \Hom (P'^{\mathrm{gp}}_{\Q\geq 0}/P'^{\mathrm{gp}}, \boldsymbol{\mu}_{\infty}).$$

In addition, the injection $P\rightarrow P'$ induces an injection $\Gamma'\rightarrow \Gamma$ which is isomorphic to multiplication by $m$. The fixed identification $\Gamma\cong \widehat{\Z}(1)^n$ then identifies $\Gamma'\rightarrow \Gamma$ with the $m$-th multiple map $[m]: \widehat{\Z}(1)^n\rightarrow \widehat{\Z}(1)^n$. We arrive at the following commutative diagram
$$\begin{tikzcd}
\bigwedge^i(\scrO_{X_{\ket}}(U))^n \arrow[d] \arrow[r, "\simeq"] & H^i_{\cts}(\Gamma, \scrO_{X_{\ket}}(U))\arrow[d] \arrow[r, "\simeq"] &H^i(U_{\proket}, \widehat{\scrO}_X) \arrow[d]
\\
\bigwedge^i(\scrO_{X_{\ket}}(V))^n \arrow[r, "\simeq"] & H^i_{\cts}(\Gamma', \scrO_{X_{\ket}}(V))\arrow[r, "\simeq"] &H^i(V_{\proket}, \widehat{\scrO}_X)
 \end{tikzcd}$$
 
Finally, let $\mathcal{B}_U$ denote the collection of such $V$'s. Notice that every Kummer map $P\rightarrow Q$ between sharp fs monoids factors through $[m]:P\rightarrow P$ for some $m\in \Z_{\geq 1}$. Hence, every $W\in U_{\ket}$ is covered by elements in $\mathcal{B}_U$. This is enough to conclude that the sheafification of the presheaf $W\rightarrow H^i(W_{\proket}, \widehat{\scrO}_X)$ on $U_{\ket}$ is isomorphic to the free sheaf $\bigwedge^i(\scrO_{U_{\ket}})^n$. This completes the proof. 
\end{enumerate}
\end{proof}

We also provide a coordinate-free description of $R^i\nu_*\widehat{\scrO}_{X}$. The following result is a logarithmic version of \cite[Proposition 3.23, Lemma 3.24]{Scholze-perfectoid-survey}.

\begin{Lemma}\label{Lemma: log analogue of Lemma 3.24 in Scholze's survey paper}
For every $n\in\Z_{\geq 1}$, let $\mu_{p^n}$ be the sheaf of $p^n$-th roots of unity on $X_{\ket}$. Consider the $\Z_p$-local system $\Z_p(1):=\varprojlim_{n}\mu_{p^n}$ on $X_{\ket}$ and let $\widehat{\Z}_p(1):=\nu^{-1}\Z_p(1)$ be the associated $\widehat{\Z}_p$-local system on $X_{\proket}$ (cf. \cite[Definition 6.3.2]{Diao}). The short exact sequence \[0\rightarrow \widehat{\Z}_p(1)\rightarrow \varprojlim_{x\mapsto x^p}\scrM_{X_{\proket}}\rightarrow \scrM_{X_{\proket}}\rightarrow 0\] induces a boundary map \[\scrM_{X_{\ket}}=\nu_*\scrM_{X_{\proket}}\rightarrow R^1\nu_*\widehat{\Z}_p(1).\] Then, there exists a unique $\scrO_{X_{\ket}}$-linear morphism $\Omega_{X_{\ket}}^{\log, 1}\rightarrow R^1\nu_*\widehat{\scrO}_{X}(1)$ such that the diagram 
  \[
        \xymatrix{{\scrM_{X_{\ket}}} \ar^-{}[r] \ar^{\mathrm{dlog}}[d] & {R^1\nu_*\widehat{\Z}_p(1)} \ar^-{}[d] \\
        {\Omega_{X_{\ket}}^{\log, 1}} \ar^-{}[r] & {R^1\nu_*\widehat{\scrO}_{X}(1)} }
    \]
is commutative, where $\Omega_{X_{\ket}}^{\log, 1}$  is the sheaf of log differentials defined in \cite[Definition 3.2.25]{Diao}.

Moreover, this morphism is an isomorphism. As a corollary, by taking cup product and exterior product, we obtain a canonical isomorphism $R^i\nu_*\widehat{\scrO}_{X}\cong \Omega_{X_{\ket}}^{\log,i}(-i)$ for every $i\geq 1$.
\end{Lemma}
\begin{proof}
The proof follows almost verbatim from the proof of \cite[Lemma 3.24]{Scholze-perfectoid-survey}, except that we have to replace the short exact sequence in \cite[Corollary 6.14]{Scholze_2013} by the short exact sequence in \cite[Corollary 2.4.5]{Diao-Lan-Liu-Zhu}. Here we only give a sketch of the proof.

Firstly, since the question is \'etale local, we may assume that $X$ admits a toric chart $X\rightarrow \Spa(\C_p\langle P\rangle, \calO_{\C_p}\langle P\rangle)$ for some sharp fs monoid $P$. Secondly, the desired $\scrO_{X_{\ket}}$-linear morphism, if exists, must be unique because $\Omega^{\log, 1}_{X_{\ket}}$ is a locally free $\scrO_{X_{\ket}}$-module generated by the image of $\textrm{dlog}$. It remains to show the existence.

Consider the map of short exact sequences
 \[
        \xymatrix{ 0 \ar^-{}[r] & {\widehat{\Z}_p(1)} \ar^-{}[r] \ar^{}[d] & {\varprojlim_{x\mapsto x^p}\scrM_{X_{\proket}}} \ar^-{}[r] \ar^-{}[d] & {\scrM_{X_{\proket}}}  \ar^-{}[r] \ar^-{\textrm{dlog}}[d] & 0 \\
        0 \ar^-{}[r] & {\widehat{\scrO}_{X}(1)} \ar^-{}[r]  & {\textrm{gr}^1 \scrO\!\mathbb{B}^+_{\textrm{dR}, \log,X}} \ar^-{}[r] & {\widehat{\scrO}_{X}\otimes_{\scrO_{X_{\ket}}} \Omega_{X_{\ket}}^{\log, 1}}  \ar^-{}[r]  & 0 }
    \]
    where the lower sequence is from \cite[Corollary 2.4.5]{Diao-Lan-Liu-Zhu}. The vertical map in the middle sends an element $a\in \varprojlim_{x\mapsto x^p}\calM_{X_{\proket}}$ to 
\[\log(\mathbf{e}^a):=-\sum_{n=1}^{\infty}\frac{1}{n}(1-\mathbf{e}^a)^n\in \textrm{Fil}^1\scrO\!\mathbb{B}^+_{\textrm{dR}, \log,X}\] and hence maps to $\textrm{gr}^1 \scrO\!\mathbb{B}^+_{\textrm{dR}, \log,X}$ (see \cite[\S 2.2]{Diao-Lan-Liu-Zhu} for the definition of $\mathbf{e}^a$). It is straightforward to check the commutativity of the diagram. This diagram then induces a diagram of boundary maps
  \[
        \xymatrix{{H^0(X_{\proket}, \scrM_{X_{\proket}})} \ar^-{}[r] \ar^{\textrm{dlog}}[d] & {H^1(X_{\proket}, \widehat{\Z}_p(1))} \ar^-{}[d] \\
        {H^0(X_{\proket}, \widehat{\scrO}_{X}\otimes_{\scrO_{X_{\ket}}} \Omega_{X_{\ket}}^{\log, 1})} \ar^-{}[r] & {H^1(X_{\proket}, \widehat{\scrO}_{X}(1))} }
    \]
    which provides the desired morphism. 
    
    To check that this map is an isomorphism, we fix an identification $P^{\mathrm{gp}}\simeq \Z ^n=\bigoplus_{j=1}^n \Z e_j$ which induces an isomorphism
 $$\Gamma:=\Hom(P^{\mathrm{gp}}_{\Q\geq 0}/P^{\mathrm{gp}}, \boldsymbol{\mu}_{\infty})\cong \widehat{\Z}(1)^n.$$ 
Notice that each $e_j$ can be written as a $\Z$-linear combination of elements in $P$; \emph{i.e.}, $e_j=\sum_{t=1}^m a_t p_t$ for some $a_t\in \Z$ and $p_t\in P$. We define $\mathrm{dlog}(e_j):=\sum_{t=1}^m a_t\mathrm{dlog}(p_t)$ where we have identify $p_t$ with its image in $\scrM_{X_{\ket}}$. One checks that $\mathrm{dlog}(e_j)$ is independent of the choice of the $\Z$-linear combination and the $\mathrm{dlog}(e_j)$'s form a basis for the free $\scrO_{X_{\ket}}$-module $\Omega^{\log, 1}_{X_{\ket}}$.
 
 On the other hand, by the computation in the proof of Lemma \ref{Lemma: Cartan-Leray}, the identification $\Gamma\simeq \widehat{\Z}(1)^n$ induces an isomorphism $R^1\nu_* \widehat{\scrO}_X\simeq \scrO_{X_{\ket}}^n=\bigoplus_{j=1}^n \scrO_{X_{\ket}}\epsilon_j$. Direct computation shows that the map $\Omega^{\log,1}_{X_{\ket}}\rightarrow R^1\nu_* \widehat{\scrO}_X$ sends $\mathrm{dlog}(e_j)$ to $\epsilon_j$, for every $j=1, \ldots, n$. This finishes the proof.
\end{proof}

To wrap up this subsection, we include a logarithmic analogue of \cite[Proposition 6.8]{CHJ-2017} which suggests that the calculation of $R^i\nu_*\widehat{\scrO}_X$ is compatible with the ``mixed completed tensor''. 

\begin{Proposition}\label{Proposition: compatibility with completed tensor}
Let $M$ be a profinite flat $\calO_K$-module in the sense of \cite[Definition 6.1]{CHJ-2017}. Then there is a canonical isomorphism
\[R^i\nu_*(\widehat{\scrO}_{X}\widehat{\otimes}M)\cong (R^i\nu_*\widehat{\scrO}_{X})\widehat{\otimes} M,\]
where $\widehat{\otimes}$ stands for the ``mixed completed tensor'' in the sense of \cite[Definition 6.6]{CHJ-2017}. Here, the mixed completed tensor on the right hand side is with respect to the subsheaf $\mathrm{Im}(R^i\nu_*\widehat{\scrO}^+_X\rightarrow R^i\nu_*\widehat{\scrO}_{X})\subset R^i\nu_*\widehat{\scrO}_{X}$.

Consequently, by Lemma \ref{Lemma: log analogue of Lemma 3.24 in Scholze's survey paper}, we have
\[R^i\nu_*(\widehat{\scrO}_{X}\widehat{\otimes}M)\cong \Omega^{\log, i}_{X_{\ket}}(-i)\widehat{\otimes}M.\]
\end{Proposition}

\begin{proof}
The proof follows verbatim as in the proof of \cite[Proposition 6.8]{CHJ-2017} as long as we replace \cite[Lemma 6.11(1)(2)]{CHJ-2017} by Lemma \ref{Lemma: Cartan-Leray}.
\end{proof}


\subsection{Banach sheaves and a (generalised) projection formula}\label{subsection: generalised projective formula} 
In this subsection, we introduce the notion of \emph{Banach sheaves} on the Kummer \'etale topology of a log adic space, generalising the ones studied in \cite[\S A]{AIP-2015} and \cite[\S 2]{Boxer--Pilloni--higherColeman}. Then, for certain \emph{admissible} Banach sheaves, we prove a projection formula which will be used in the main body of the paper.

Recall from Definition \ref{Definition: weights} that a \emph{small $\Z_p$-algebra} is a $p$-torsion free reduced ring $R$ which is also a finite $\Z_p\llbrack T_1, ..., T_d\rrbrack$-algebra for some $d\in \Z_{\geq 0}$. It is a profinite flat $\Z_p$-module in the sense of \cite[Definition 6.1]{CHJ-2017}. In particular, there exists a set of elements $\{e_\sigma: \sigma\in \Sigma\}$ in $R$ such that $R\simeq \prod_{\sigma\in \Sigma}\Z_pe_\sigma$ equipped with the product topology. This set of elements $\{e_{\sigma}: \sigma\in \Sigma\}$ is called a \emph{pseudo-basis} for $R$. Moreover, $R$ is equipped with an adic profinite topology and is complete with respect to the $p$-adic topology. 

Throughout this subsection, we keep the following notations:
\begin{itemize}
\item Let $R$ be a fixed small $\Z_p$-algebra and let $\fraka$ be a fixed ideal of definition containing $p$.
\item All (log) adic spaces are assumed to be reduced and quasi-separated. In particular, $X$ either stands for a locally noetherian reduced adic space over $(\C_p, \calO_{\C_p})$ or a locally noetherian reduced fs log adic space over $(\C_p, \calO_{\C_p})$. In the second case, we use $X_{\an}$ to denote the underlying adic space of $X$.
\item We adopt the notation of ``mixed completed tensors'' $-\,\widehat{\otimes}'R$ and $-\,\widehat{\otimes}R$ as in Definition \ref{Definition: unadorned completed tensor}.
\end{itemize}

\begin{Lemma}\label{Lemma: structure sheaf mixed completed tensor}
\begin{itemize}
\item[(i)] Let $X$ be a locally noetherian adic space over $(\C_p, \calO_{\C_p})$. Then the presheaf $\scrO^+_X\widehat{\otimes}' R$ (resp., $\scrO_X\widehat{\otimes} R$) sending any quasi-compact open subset $U\subset X$ to $\scrO^+_X(U)\widehat{\otimes}'R$ (resp., $\scrO_X(U)\widehat{\otimes}R$) is a sheaf. In particular, $\scrO_X\widehat{\otimes} R$ is a sheaf of Banach $\C_p$-algebras.
\item[(ii)] Let $X$ be a locally noetherian fs log adic space over $(\C_p, \calO_{\C_p})$. Then the presheaf $\scrO^+_{X_{\ket}}\widehat{\otimes}' R$ (resp., $\scrO_{X_{\ket}}\widehat{\otimes} R$) sending any quasi-compact $U\in X_{\ket}$ to $\scrO^+_{X_{\ket}}(U)\widehat{\otimes}'R$ (resp., $\scrO_{X_{\ket}}(U)\widehat{\otimes}R$) is a sheaf. In particular, $\scrO_{X_{\ket}}\widehat{\otimes} R$ is a sheaf of Banach $\C_p$-algebras.
\item[(iii)] Let $X$ be a locally noetherian fs log adic space over $(\C_p, \calO_{\C_p})$. Then the presheaf $\widehat{\scrO}^+_{X_{\proket}}\widehat{\otimes}' R$ (resp., $\widehat{\scrO}_{X_{\proket}}\widehat{\otimes} R$) sending any qcqs $U\in X_{\proket}$ to $\widehat{\scrO}^+_{X_{\proket}}(U)\widehat{\otimes}'R$ (resp., $\widehat{\scrO}_{X_{\proket}}(U)\widehat{\otimes}R$) is a sheaf. In particular, $\widehat{\scrO}_{X_{\proket}}\widehat{\otimes} R$ is a sheaf of Banach $\C_p$-algebras.
\end{itemize}
\end{Lemma}

\begin{proof}
Choosing a presentation $R\simeq \prod_{\sigma\in \Sigma} \Z_p e_\sigma$ and using \cite[Proposition 6.4]{CHJ-2017}, the statements reduce to the sheafiness of the corresponding structure presheaves.
\end{proof}

\begin{Definition}\label{Definition: Banach modules}
Let $B$ be a Banach $\Q_p$-algebra and let $B_0$ be an open and bounded $\Z_p$-submodule. 
\begin{enumerate}
\item[(i)] A topological $B$-module $M$ is called a \textbf{Banach $B$-module} if there exists an open bounded $B_0$-submodule $M_0$ which is $p$-adically complete and separated such that $M=M_0[1/p]$.
\item[(ii)] Let $J$ be an index set. Consider the $B$-module $B(J)$ consisting of sequences $\{b_j: j\in J\}$ which converge to 0 with respect to the filter in $J$ of the complement of the finite subsets of $J$.
Then $B(J)$ is a Banach $B$-module. Indeed, let $B_0(J)$ be the $p$-adic completion of the free $B_0$-module $\bigoplus_{j\in J} B_0$. Then we have $B(J)\simeq B_0(J)[1/p]$.
\item[(iii)] A topological $B$-module $M$ is called an \textbf{orthonormalisable Banach $B$-module} (or, \textbf{ON-able Banach $B$-module} for short) if there exists a topological isomorphism $M\simeq B(J)$ for some index set $J$. A topological $B$-module $M$ is called a \textbf{projective Banach $B$-module} if it is a direct summand (as a topological $B$-module) inside an orthonormalisable Banach $B$-module.
\end{enumerate}
\end{Definition}

\begin{Definition}\label{Definition: Banach sheaf}
Let $X$ be a locally noetherian adic space over $(\C_p, \calO_{\C_p})$.
\begin{enumerate}
\item[(i)] A sheaf of topological $\scrO_{X}\widehat{\otimes} R$-modules $\scrF$ is called a \textbf{Banach sheaf of $\scrO_{X}\widehat{\otimes} R$-modules} if 
\begin{itemize}
\item for every quasi-compact open subset $U\subset X$, $\scrF(U)$ is a Banach $\scrO_X(U)\widehat{\otimes} R$-module;
\item there exists an affinoid open covering $\mathfrak{U}=\{U_i: i\in I\}$ of $X$ such that for every $i\in I$ and every affinoid open subset $V\subset U_i$, the continuous restriction map
\[\scrF(U_i)\otimes_{\scrO_X(U_i)}\scrO_X(V)\rightarrow \scrF(V)\]
induces a topological isomorphism
\[\scrF(U_i)\widehat{\otimes}_{\scrO_X(U_i)}\scrO_X(V)\rightarrow \scrF(V)\]
\end{itemize}
where the completion is with respect to the $p$-adic topology. Such a covering $\mathfrak{U}$ is called an \textbf{atlas} of $\scrF$.
\item[(ii)] A sheaf $\scrF$ as in (i) is called a \textbf{projective Banach sheaf of $\scrO_X\widehat{\otimes} R$-modules} if there exists an atlas $\mathfrak{U}=\{U_i: i\in I\}$ such that $\scrF(U_i)$'s are projective Banach $\scrO_X(U_i)\widehat{\otimes}R$-modules. 
\item[(iii)] A morphism between Banach sheaves of $\scrO_X\widehat{\otimes}R$-modules is a continuous map of sheaves of topological $\scrO_X\widehat{\otimes}R$-modules.
\item[(iv)] Let $\scrF$ be a Banach sheaf of $\scrO_X\widehat{\otimes}R$-modules as in (i). An \textbf{integral model} of $\scrF$ is a subsheaf $\scrF^+$ of $\scrO^+_X\widehat{\otimes}'R$-modules such that
\begin{itemize}
\item for every quasi-compact open $U\subset X$, $\scrF^+(U)$ is open and bounded in $\scrF(U)$;
\item $\scrF=\scrF^+[1/p]$;
\item there exists an atlas $\mathfrak{U}=\{U_i:i\in I\}$ of $\scrF$ such that, for every $i\in I$ and every affinoid open subset $V\subset U_i$, the canonical map
\[\scrF^+(U_i)\widehat{\otimes}_{\scrO^+_X(U_i)}\scrO^+_X(V)\rightarrow \scrF^+(V)\]
is an isomorphism, where the completion is with respect to the $p$-adic topology.
\end{itemize}
\end{enumerate}
\end{Definition}

We are also interested in a Kummer \'etale version of Banach sheaves.

\begin{Definition}\label{Definition: Kummer etale Banach sheaf}
Let $X$ be a locally noetherian fs log adic space of $(\C_p, \calO_{\C_p})$.
\begin{enumerate}
\item[(i)] A sheaf of topological $\scrO_{X_{\ket}}\widehat{\otimes}R$-modules $\scrF$ is called a \textbf{Kummer \'etale Banach sheaf of} $\scrO_{X_{\ket}}\widehat{\otimes}R$-\textbf{modules} if
\begin{itemize}
\item for every quasi-compact open $U\in X_{\ket}$, $\scrF(U)$ is a Banach $\scrO_{X_{\ket}}(U)\widehat{\otimes} R$-module;
\item there exists an Kummer \'etale covering $\mathfrak{U}=\{U_i: i\in I\}$ of $X$ by affinoid $U_i$'s such that for every Kummer \'etale map $V\rightarrow U_i$ with affinoid $V$, the continuous restriction map
\[\scrF(U_i)\otimes_{\scrO_{X_{\ket}}(U_i)}\scrO_{X_{\ket}}(V)\rightarrow \scrF(V)\]
induces a topological isomorphism
\[\scrF(U_i)\widehat{\otimes}_{\scrO_{X_{\ket}}(U_i)}\scrO_{X_{\ket}}(V)\rightarrow \scrF(V)\]
\end{itemize}
where the completion is with respect to the $p$-adic topology. Such a covering $\mathfrak{U}$ is called a \textbf{Kummer \'etale atlas} of $\scrF$.

\item[(ii)] A sheaf as in (i) is called a \textbf{projective Kummer \'etale Banach sheaf of} $\scrO_{X_{\ket}}\widehat{\otimes} R$-\textbf{modules} if there exists a Kummer \'etale atlas $\mathfrak{U}=\{U_i: i\in I\}$ such that $\scrF(U_i)$'s are projective Banach $\scrO_{X_{\ket}}(U_i)\widehat{\otimes}R$-modules.
\item[(iii)] A morphism between Kummer \'etale Banach sheaves of $\scrO_{X_{\ket}}\widehat{\otimes}R$-modules is a continuous map of topological $\scrO_{X_{\ket}}\widehat{\otimes}R$-modules.
\item[(iv)] Let $\scrF$ be a Kummer \'etale Banach sheaf of $\scrO_{X_{\ket}}\widehat{\otimes}R$-modules as in (i). An \textbf{integral model} of $\scrF$ is a subsheaf $\scrF^+$ of $\scrO^+_{X_{\ket}}\widehat{\otimes}'R$-modules such that
\begin{itemize}
\item for every quasi-compact $U\in X_{\ket}$, $\scrF^+(U)$ is open and bounded in $\scrF(U)$;
\item $\scrF=\scrF^+[1/p]$;
\item there exists a Kummer \'etale atlas $\mathfrak{U}=\{U_i:i\in I\}$ of $\scrF$ such that, for every $i\in I$ and every affinoid $V\in U_{i,\ket}$, the canonical map
\[\scrF^+(U_i)\widehat{\otimes}_{\scrO^+_{X_{\ket}}(U_i)}\scrO^+_{X_{\ket}}(V)\rightarrow \scrF^+(V)\]
is an isomorphism, where the completion is with respect to the $p$-adic topology.
\end{itemize}
\end{enumerate}
\end{Definition}

Clearly, an analytic refinement of an atlas (resp., a Kummer \'etale refinement of a Kummer \'etale atlas) is also an atlas (resp., a Kummer \'etale atlas). Also notice that it is not true that a Banach sheaf (resp., Kummer \'etale Banach sheaf) on an affinoid adic space (resp., affinoid log adic space) is the sheaf associated with its global section. Nonetheless, we have the following result.

\begin{Lemma}\label{Lemma: Banach sheaf associated with global section}
Let $(A, A^+)$ be a complete reduced Tate algebra over $(\C_p, \calO_{\C_p})$ and let $M$ be a projective Banach $A\widehat{\otimes}R$-module.
\begin{enumerate}
\item[(i)] Let $X=\Spa(A, A^+)$ be the associated adic space. Then the presheaf $M\widehat{\otimes}_A \scrO_X$ sending an affinoid open subset $\Spa(B, B^+)\subset X$ to $M\widehat{\otimes}_A B$ is a sheaf.
\item[(ii)] Suppose $X=\Spa(A, A^+)$ is equipped with an fs log structure. Then the presheaf $M\widehat{\otimes}_A \scrO_{X_{\ket}}$ sending an affinoid open subset $\Spa(B, B^+)\in X_{\ket}$ to $M\widehat{\otimes}_A B$ is a sheaf.
\end{enumerate}
\end{Lemma}

\begin{proof}
It immediately reduces to the case where $M$ is an orthonormalisable Banach $A\widehat{\otimes}R$-module; \emph{i.e.}, $M\simeq (A\widehat{\otimes}R)(J)$ for some index set $J$. It then reduces to the case where $|J|=1$. Then the lemma follows from Lemma \ref{Lemma: structure sheaf mixed completed tensor}.
\end{proof}

As a corollary, one can associate a projective Kummer \'etale Banach sheaf with every projective Banach sheaf.

\begin{Corollary}\label{Corollary: Banach induce Kummer etale Banach}
Let $X$ be a locally noetherian fs log adic space over $(\C_p, \calO_{\C_p})$ and let $\scrF$ be a projective Banach sheaf of $\scrO_{X_{\an}}\widehat{\otimes}R$-modules with atlas $\mathfrak{U}=\{U_i: i\in I\}$. Suppose $\scrF$ admits an integral model $\scrF^+$. Consider the $p$-adically completed sheaf of $\scrO_{X_{\ket}}$-modules $\scrF_{\ket}$ associated with $\scrF$; namely, 
\[\scrF_{\ket}:=\left(\varprojlim_m \scrF^+\otimes_{\scrO^+_{X_{\an}}}\scrO^+_{X_{\ket}}/p^m\right)[\frac{1}{p}].\]
Then $\scrF_{\ket}$ is a projective Kummer \'etale Banach sheaf of $\scrO_{X_{\ket}}\widehat{\otimes}R$-modules with Kummer \'etale atlas $\mathfrak{U}=\{U_i: i\in I\}$, where each $U_i$ is equipped with the induced log structure from $X$. Moreover, for every affinoid $V\in U_{i, \ket}$, we have
\[\scrF_{\ket}(V)\cong \scrF(U_i)\widehat{\otimes}_{\scrO_{X_{\an}}(U_i)}\scrO_{X_{\ket}}(V).\]
\end{Corollary}

We need an easy lemma.

\begin{Lemma}\label{Lemma: a convenient basis}
Let $X$ be a locally noetherian fs log adic space over $(\C_p, \calO_{\C_p})$ and let $\mathfrak{U}=\{U_i:i\in I\}$ be a Kummer \'etale covering of $X$ by affinoid $U_i$'s. Consider the full subcategory $\calB_{\mathfrak{U}}$ of $X_{\ket}$ consisting of those affinoid $V\in X_{\ket}$ such that the map $V\rightarrow X$ factors through $V\rightarrow U_i \rightarrow X$ for some $i\in I$. Then $\calB_{\mathfrak{U}}$ forms a basis for the site $X_{\ket}$.
\end{Lemma}

\begin{proof}
We have to prove that every $U\in X_{\ket}$ admits a covering by such $V$'s and that $\calB_{\mathfrak{U}}$ is closed under fibred products. Both statements are clear.
\end{proof}

\begin{proof}[Proof of Corollary \ref{Corollary: Banach induce Kummer etale Banach}]
Let $\calB_{\mathfrak{U}}$ be the basis of $X_{\ket}$ as in Lemma \ref{Lemma: a convenient basis} associated with the covering $\mathfrak{U}=\{U_i: i\in I\}$. It suffices to show that the assignment $$V\mapsto \scrF(U_i)\widehat{\otimes}_{\scrO_{X_{\an}}(U_i)}\scrO_{X_{\ket}}(V),$$
for every $V\in \calB_{\mathfrak{U}}$ which factors through $V\rightarrow U_i\rightarrow X$, defines a sheaf on $\calB_{\mathfrak{U}}$. (Notice that this assigment is independent of the choice of $i$ and hence well-defined.) The sheafiness of this assignment follows from Lemma \ref{Lemma: Banach sheaf associated with global section} and the sheafiness of $\scrF$.
\end{proof}

In what follows, we are interested in those Kummer \'etale Banach sheaves that are ``admissible''. Let us first recall the notion of coherent sheaves on a ringed site.

\begin{Definition}\label{Definition: coherent sheaf on ringed space}
Let $(Z, \scrO_Z)$ be a ringed site. A sheaf of $\scrO_Z$-modules $\scrF$ is called a \textbf{coherent $\scrO_Z$-module} if there exists a covering $\mathfrak{U}=\{U_i: i\in I\}$ for $Z$ such that for every $i\in I$, there exist positive integers $m$, $n$, and an exact sequence of $\scrO_Z|_{U_i}$-modules
\[\bigoplus_{j=1}^m\scrO_Z|_{U_i}\rightarrow \bigoplus_{k=1}^n \scrO_Z|_{U_i}\rightarrow \scrF|_{U_i}\rightarrow 0.\]
In this situation, we say that $\scrF$ is a coherent $\scrO_Z$-module \textbf{subject to the covering} $\mathfrak{U}$.
\end{Definition}

We will apply this definition to the ringed site $(X_{\ket}, \scrO^+_{X_{\ket}}\otimes_{\Z_p} (R/\fraka^m))$.

\begin{Definition}\label{Definition: admissible Banach sheaf}
Let $X$ be a locally noetherian fs log adic space over $(\C_p, \calO_{\C_p})$ and let $\scrF$ be a projective Kummer \'etale Banach sheaf of $\scrO_{X_{\ket}}\widehat{\otimes}R$-modules. Suppose it admits an integral model $\scrF^+$ and, for every $m\in \Z_{\geq 1}$, we write $\scrF^+_m:=\scrF^+/\fraka^m$. We say that $\scrF$ is \textbf{admissible} if there exist
\begin{itemize}
\item a Kummer \'etale atlas $\mathfrak{U}=\{U_i: i\in I\}$ of $X$ such that each $\scrF^+(U_i)$ is the $p$-adic completion of a free $\scrO^+_{X_{\ket}}\widehat{\otimes}'R$-module; and
\item for every $m\in \Z_{\geq 1}$ and $d\in \Z_{\geq 1}$, a subsheaf $\scrF^+_{m,d}\subset \scrF^+_m$ which is a coherent $\scrO^+_{X_{\ket}}\otimes_{\Z_p} (R/\fraka^m)$-module subject to the covering $\mathfrak{U}$,
\end{itemize}
such that we have $\scrF^+\cong\varprojlim_m \scrF^+_m$ and $\scrF_m^+\cong \varinjlim_d \scrF^+_{m,d}$ for every $m\in \Z_{\geq 1}$.

Such a Kummer \'etale atlas is called an \textbf{admissible atlas} for $\scrF$.
\end{Definition}

\begin{Lemma}\label{Lemma: pushforward along finite Kummer etale map}
Let $h: Y\rightarrow X$ be a finite Kummer \'etale morphism between locally noetherian fs log adic spaces over $(\C_p, \calO_{\C_p})$. Suppose $\scrF$ is an admissible Kummer \'etale Banach sheaf of $\scrO_{Y_{\ket}}\widehat{\otimes}R$-modules. Then $h_*\scrF$ is an admissible Kummer \'etale Banach sheaf of $\scrO_{X_{\ket}}\widehat{\otimes}R$-modules.
\end{Lemma}

\begin{proof}
Suppose $\mathfrak{U}=\{U_i:i\in I\}$ is an admissible atlas for $\scrF$ on $Y$. By Definition \ref{Definition: Kummer etale morphism} and \cite[Proposition 4.1.6]{Diao}, the finite Kummer \'etale morphism $h:Y\rightarrow X$ is, Kummer \'etale locally on $X$, isomorphic to a direct sum of isomorphisms. Therefore, one can find an affinoid Kummer \'etale covering $\{V_j: j\in J\}$ of $X$ such that, for every $i\in I$ and $j\in J$, $U_i\times_X V_j$ is isomorphic to a disjoint union of finite copies of $U_i$'s. Consequently, the Kummer \'etale covering $\mathfrak{V}=\{U_i\times_X V_j: i\in I, j\in J\}$ is a desired admissible atlas for $h_*\scrF$.
\end{proof}

If $\scrF$ is a Kummer \'etale Banach sheaf of $\scrO_{X_{\ket}}\widehat{\otimes}R$-modules with an integral structure $\scrF^+$. We write
\[\widehat{\scrF}^+:=\varprojlim_m \left(\scrF^+\otimes_{\scrO^+_{X_{\ket}}}\scrO^+_{X_{\proket}}/p^m\right)\cong \varprojlim_m \left(\scrF^+\otimes_{(\scrO^+_{X_{\ket}}\widehat{\otimes}'R)}(\scrO^+_{X_{\proket}}\widehat{\otimes}'R)/p^m\right)\]
and $\widehat{\scrF}:=\widehat{\scrF}^+[1/p]$. They are sheaves of $\widehat{\scrO}^+_{X_{\proket}}\widehat{\otimes}'R$-modules and $\widehat{\scrO}_{X_{\proket}}\widehat{\otimes}R$-modules, respectively. 

Recall the natural projection of sites $\nu: X_{\proket}\rightarrow X_{\ket}$. The main result of this subsection is the following.

\begin{Proposition}[Generalised projection formula]\label{Proposition: generalised projection formula}
Let $X$ be a locally noetherian fs log adic space which is log smooth over $(\C_p, \calO_{\C_p})$ and let $\scrF$ be a projective Kummer \'etale Banach sheaf of $\scrO_{X_{\ket}}\widehat{\otimes}R$-modules. Suppose $\scrF$ is admissible. Then, for every $j\in \Z_{\geq 0}$, there is a natural isomorphism of Kummer \'etale Banach sheaves of $\scrO_{X_{\ket}}\widehat{\otimes} R$-modules
\[\scrF\otimes_{\scrO_{X_{\ket}}} R^j\nu_*\widehat{\scrO}_{X_{\proket}}\xrightarrow{\sim}R^j\nu_*\widehat{\scrF}.\]
\end{Proposition}

To prove the proposition, we need some preparations. 

\begin{Lemma}\label{Lemma: proj. formula lemma 1}
Let $X$ be a locally noetherian fs log adic space over $(\C_p, \calO_{\C_p})$. Let $\scrH$ be an $\widehat{\scrO}_{X_{\proket}}^+ \widehat{\otimes} R$-module and let $\scrH_m:= \scrH/\fraka^m$ for every $m\in \Z_{\geq 1}$. Suppose 
\begin{itemize}
    \item $\scrH = \varprojlim_m \scrH_m$; and
    \item for every $m\in \Z_{\geq 1}$, there exists a sequence of finite free $\widehat{\scrO}_{X_{\proket}}^+\otimes_{\Z_p} (R/\fraka^m)$-submodules $\{\scrH_{m, d}: d\in \Z_{\geq 0}\}$ of $\scrH_m$ such that $\scrH_m\cong \varinjlim_d \scrH_{m,d}$.
\end{itemize}
   Then, for every $j\in \Z_{\geq 0}$, the natural map \[
    R^j\nu_{*}\scrH \rightarrow \varprojlim_m R^j\nu_{*} \scrH_{m}
\] is an almost isomorphism.
\end{Lemma}
\begin{proof}
We have to show the almost vanishing of the higher inverse limit $R^j\varprojlim_{m}\scrH_m$. Applying an almost version of \cite[Lemma 3.18]{Scholze_2013}, it suffices to show that, for every log affinoid perfectoid object $U\in X_{\proket}$, there are almost isomorphisms 
$$R^1\varprojlim_{m} \scrH_m(U)^a = 0 
$$
and 
$$ H^j(U, \scrH_m)^a = 0$$
for every $j\in \Z_{\geq 0}$. The first almost vanishing follows from the Mittag-Leffler condition. To obtain the second almost isomorphism, observe that \[
    H^j(U, \scrH_m) \cong \varinjlim_{d} H^j(U, \scrH_{m, d}).
\] and each $H^j(U, \scrH_{m, d})$ is almost zero by \cite[Theorem 5.4.3]{Diao}.
\end{proof}

\begin{Lemma}\label{Lemma: proj. formula lemma 2}
Let $X$ be a locally noetherian fs log adic space which is log smooth over $(\C_p, \calO_{\C_p})$. If $\scrG$ is an projective Kummer \'etale Banach sheaf of $\scrO_{X_{\ket}}\widehat{\otimes}R$-modules, then, for every $j\in \Z_{\geq 0}$, the sheaf $R^j\nu_*\widehat{\scrG}$ is also a projective Kummer \'etale Banach sheaf of $\scrO_{X_{\ket}}\widehat{\otimes}R$-modules.
\end{Lemma}

\begin{proof}
By considering a Kummer \'etale atlas for $\scrG$ and writing $R\simeq \prod_{\sigma\in \Sigma} \Z_p e_{\sigma}$, we immediately reduce to the case where 
\begin{itemize}
\item $X$ is affinoid and admits a toric chart $X\rightarrow \mathbb{E}=\Spa(\C_p\langle P\rangle, \calO_{\C_p}\langle P\rangle)$ for some sharp fs monoid $P$; 
\item $R=\Z_p$ and $\fraka=(p)$;
\item $\scrG$ is globally projective; \emph{i.e.}, $\scrG(X)$ is a projective Banach $\scrO_{X_{\ket}}(X)$-module and for every 
affinoid $U\in X_{\ket}$, we have a natural isomorphism
\[\scrG(X)\widehat{\otimes}_{\scrO_{X_{\ket}}(X)}\scrO_{X_{\ket}}(U)\xrightarrow[]{\sim} \scrG(U).\]
\end{itemize}
We further reduce to the case where $\scrG$ is globally orthonormalisable; namely, $\scrG\simeq \scrO_{X_{\ket}}(J)$ for some index set $J$. Let $\scrG^+$ be the $p$-adic completion of the free $\scrO^+_{X_{\ket}}$-module $\bigoplus_{J}\scrO^+_{X_{\ket}}$ and let $\scrG^+_m:=\scrG^+/p^m\simeq \bigoplus_{J} \scrO^+_{X_{\ket}}/p^m$. By Lemma \ref{Lemma: proj. formula lemma 1}, we have a natural almost isomorphism
$$R^j\nu_*\widehat{\scrG}^+\xrightarrow[]{\sim} \varprojlim_m R^j \nu_*\widehat{\scrG}^+_m$$
where $\widehat{\scrG}^+_m=\widehat{\scrG}^+/p^m\simeq \bigoplus_{J} \scrO^+_{X_{\proket}}/p^m$.

We claim that, in this case, the sheaf $R^j\nu_*\widehat{\scrG}$ is isomorphic to $\left(\bigwedge^j (\scrO_{X_{\ket}})^n\right)(J)$ for some $n\in \Z_{\geq 1}$. For this, we follow the strategy as in the proof of Lemma \ref{Lemma: Cartan-Leray}.

In order to be consistent with the notations in Lemma \ref{Lemma: Cartan-Leray}, we write $U=X$. Consider the collection $\mathcal{B}_U$ used in the proof of Lemma \ref{Lemma: Cartan-Leray}. In particular, for every $V\in \mathcal{B}_U$, the map $V\rightarrow U$ admits a Kummer chart $P\rightarrow P'$ which is isomorphic to the $m$-th multiple map $[m]:P\rightarrow P$. Moreover, the injection $P\rightarrow P'$ induces an injection $\Gamma'\rightarrow \Gamma$. If we fix an identification $\Gamma\cong \widehat{\Z}(1)^n$, the injection $\Gamma'\rightarrow \Gamma$ can be identified with the $m$-th multiple map $[m]:\widehat{\Z}(1)^n\rightarrow \widehat{\Z}(1)^n$.

By the calculation in Lemma \ref{Lemma: Cartan-Leray}, we obtain an almost injection
$$\left(\bigwedge^j (\scrO^+_{X_{\ket}}/p^m(V))^n\right)^a \simeq H^j(\Gamma, \scrO^+_{X_{\ket}}/p^m(V))^a\hookrightarrow H^j_{\proket}(V, \scrO^+_X/p^m)^a$$
with cokernel killed by $p$. Taking direct sum and then inverse limit, we obtain an almost injection
$$
\varprojlim_m \bigoplus_J \left(\bigwedge^j (\scrO^+_{X_{\ket}}/p^m(V))^n\right)^a \hookrightarrow \varprojlim_m \bigoplus_J H^j_{\proket}(V, \scrO^+_X/p^m)
$$
with cokernel killed by $p$. Inverting $p$, we obtain an isomorphism
$$
\left(\bigwedge^j (\scrO^+_{X_{\ket}}(V))^n\right)(J)\simeq \varprojlim_m \bigoplus_J H^j_{\proket}(V, \scrO^+_X/p^m).
$$
However, note that the sheaf $$R^j\nu_*\widehat{\scrG}\cong \left(\varprojlim_m R^j \nu_*\widehat{\scrG}^+_m\right)[\frac{1}{p}]$$
is just the sheafification of $W\mapsto \varprojlim_m \bigoplus_J H^j_{\proket}(W, \scrO^+_X/p^m)$. Consequently, $R^j\nu_*\widehat{\scrG}$ coincides with the sheaf $\left(\bigwedge^j (\scrO^+_{X_{\ket}})^n\right)(J)$ which is clearly an ON-able Banach sheaf of $\scrO_{X_{\ket}}\widehat{\otimes} R$-modules.
\end{proof}

\begin{proof}[Proof of Proposition \ref{Proposition: generalised projection formula}]
We split the proof into three steps.

\paragraph{Step 1.} We first verify that both $\scrF\otimes_{\scrO_{X_{\ket}}} R^j\nu_*\widehat{\scrO}_{X_{\proket}}$ and $R^j\nu_* \widehat{\scrF}$ are projective Kummer \'etale Banach sheaf of $\scrO_{X_{\ket}}\widehat{\otimes} R$-modules.

Indeed, the statement for $\scrF\otimes_{\scrO_{X_{\ket}}} R^j\nu_*\widehat{\scrO}_{X_{\proket}}$ follows from the locally finite free-ness of $R^j\nu_*\widehat{\scrO}_{X_{\proket}}$ (cf. Lemma \ref{Lemma: Cartan-Leray}) and the statement for $R^j\nu_* \widehat{\scrF}$ follows from Lemma \ref{Lemma: proj. formula lemma 2}. In fact, we can be more precise. Consider an affinoid Kummer \'etale covering $\mathfrak{U}=\{U_i: i\in I\}$ satisfying:
\begin{itemize}
\item $\mathfrak{U}$ is an admissible atlas of $\scrF$;
\item each $U_i$ admits a toric chart $U_i\rightarrow \Spa(\C_p\langle P_i\rangle, \calO_{\C_p}\langle P_i\rangle)$ for some sharp fs monoid.
\end{itemize}
Then, by the proof of Lemma \ref{Lemma: Cartan-Leray} and Lemma \ref{Lemma: proj. formula lemma 2}, we see that $\mathfrak{U}$ is a Kummer \'etale atlas for both $\scrF\otimes_{\scrO_{X_{\ket}}} R^j\nu_*\widehat{\scrO}_{X_{\proket}}$ and $R^j\nu_* \widehat{\scrF}$. (In fact, they are both orthonormalisable on each $U_i$.) For the rest of the proof, we fix such a cover $\mathfrak{U}$.

\paragraph{Step 2.} We construct two natural morphisms
$$
\Psi: \scrF^+\bigotimes_{\scrO^+_{X_{\ket}}\widehat{\otimes}'R} R^j\nu_*\left(\widehat{\scrO}^+_{X_{\proket}}\widehat{\otimes}'R\right)\rightarrow \varprojlim_{m}R^j\nu_{*}\widehat{\scrF}_m^+
$$
and
$$
 \Theta: R^j\nu_*\widehat{\scrF}^+ \rightarrow \varprojlim_{m}R^j\nu_{*}\widehat{\scrF}_m^+.
$$
where $$\widehat{\scrF}^+_m=\scrF^+_m\otimes_{\scrO^+_{X_{\ket}}}\widehat{\scrO}^+_{X_{\proket}}=\widehat{\scrF}^+/\fraka^m.$$
There is clearly such a map $\Theta$. It remains to construct $\Psi$.

For every $m\in \Z_{\geq 1}$ and $d\in \Z_{\geq 0}$, we write \[
    \widehat{\scrF}_{m, d}^+ := \scrF_{m, d}^+ \bigotimes_{\scrO_{X_{\ket}}^+\otimes_{\Z_p} (R/\fraka^m)} \left(\widehat{\scrO}_{X_{\proket}}^+ \otimes_{\Z_p} ( R/\fraka^m)\right).
\] 
By the usual projection formula for ringed sites (see, for example, \cite[01E6]{stacks-project}), we obtain a canonical morphism
$$
    \Psi_{m, d}: \scrF^+_{m, d} \bigotimes_{\scrO_{X_{\ket}}^+\otimes_{\Z_p} (R/\fraka^m)} R^j\nu_* \left(\widehat{\scrO}_{X_{\proket}}^+ \otimes_{\Z_p} ( R/\fraka^m)\right) \rightarrow R^j\nu_* \widehat{\scrF}_{m, d}^+.
$$
Taking direct limit with respect to $d$, followed by taking inverse limit with respect to $m$, we obtain a canonical morphism \[
    \Psi': \varprojlim_m\left(\scrF^+_m \bigotimes_{\scrO_{X_{\ket}}^+\otimes_{\Z_p} (R/\fraka^m)} R^j\nu_* \left(\widehat{\scrO}_{X_{\proket}}^+ \otimes_{\Z_p} (R/\fraka^m)\right) \right) \rightarrow \varprojlim_mR^j\nu_* \widehat{\scrF}_{m}^+.
\]
On the other hand, we have natural morphisms
\begin{align*}
\Psi'': &\scrF^+\bigotimes_{\scrO^+_{X_{\ket}}\widehat{\otimes}'R} R^j\nu_*\left(\widehat{\scrO}^+_{X_{\proket}}\widehat{\otimes}'R\right) \rightarrow \scrF^+\bigotimes_{\scrO^+_{X_{\ket}}\widehat{\otimes}'R} \varprojlim_m \left(R^j\nu_*\left(\widehat{\scrO}^+_{X_{\proket}}\otimes_{\Z_p} (R/\fraka^m)\right)\right)\\
=& (\varprojlim_m\scrF^+_m)\bigotimes_{\scrO^+_{X_{\ket}}\widehat{\otimes}'R} \varprojlim_m \left(R^j\nu_*\left(\widehat{\scrO}^+_{X_{\proket}}\otimes_{\Z_p} (R/\fraka^m)\right)\right)\rightarrow \varprojlim_m\left(\scrF^+_m\bigotimes_{\scrO^+_{X_{\ket}}\otimes_{\Z_p} (R/\fraka^m)}  R^j\nu_*\left(\widehat{\scrO}^+_{X_{\proket}}\otimes_{\Z_p} (R/\fraka^m)\right)\right)
\end{align*}
Composing with $\Psi'$, we obtain the desired morphism
$$
\Psi: \scrF^+\bigotimes_{\scrO^+_{X_{\ket}}\widehat{\otimes}'R} R^j\nu_*\left(\widehat{\scrO}^+_{X_{\proket}}\widehat{\otimes}'R\right)\rightarrow \varprojlim_{m}R^j\nu_{*}\widehat{\scrF}_m^+.
$$

\paragraph{Step 3.} For simplicity, we write $\scrG_i$, $\scrG^+_i$, and $\scrG^+_{i,m}$ for $\scrF|_{U_i}$, $\scrF^+|_{U_i}$, and $\scrF^+_{i,m}|_{U_i}$, respectively. Since $\scrG^+_{i,m}$ is a free $\scrO^+_{U_{i,\ket}}\otimes (R/\fraka^m)$-module, we can express $\scrG^+_{i,m}$ as a filtered direct limit of finite free submodules $\scrG^+_{i,m,\alpha}$.

We can repeat the construction in Step 2 to $\scrG^+_i$, $\scrG^+_{i,m}$, and $\scrG^+_{i,m,\alpha}$. In particular, we obtain maps
$$ \Psi_i: \scrG^+_i \bigotimes_{\scrO_{U_{i,\ket}}^+ \widehat{\otimes}' R} R^j\nu_*\left(\widehat{\scrO}_{U_{i,\proket}}^+\widehat{\otimes}' R\right) \rightarrow \varprojlim_{m} R^j\nu_*\widehat{\scrG}_{i,m}^+$$
and
$$\Theta_i: R^j\nu_*\widehat{\scrG}^+_i \rightarrow \varprojlim_{m}R^j\nu_{*}\widehat{\scrG}_{i,m}^+$$
where $$\widehat{\scrG}^+_{i,m}=\scrG^+_{i,m}\otimes_{\scrO^+_{U_{i,\ket}}}\widehat{\scrO}^+_{U_{i,\proket}}=\widehat{\scrG}^+_i/\fraka^m.$$

Moreover, we have a commutative diagram
$$ \begin{tikzcd}
        \scrF^+|_{U_i} \bigotimes_{(\scrO^+_{U_{i, \ket}}\widehat{\otimes}' R)}R^j\nu_*\left(\widehat{\scrO}_{U_{i, \proket}}^+\widehat{\otimes}' R\right)\arrow[r, "\Psi|_{U_i}"]\arrow[d, "\cong "'] & \varprojlim_{m} R^j\nu_* \widehat{\scrF}_m^+|_{U_i}\arrow[d, "\cong"] & R^j\nu_*\widehat{\scrF}^+|_{U_i} \arrow[l, "\Theta|_{U_i}"']\arrow[d, "\cong"]\\
        \scrG_{i}^+ \bigotimes_{(\scrO^+_{U_{i, \ket}}\widehat{\otimes}' R)}R^j\nu_*\left(\widehat{\scrO}_{U_{i, \proket}}^+\widehat{\otimes}' R\right)\arrow[r, "\Psi_i"] & \varprojlim_{m} R^j\nu_* \widehat{\scrG}_{i,m}^+ & R^j\nu_*\widehat{\scrG}_i^+\arrow[l, "\Theta_i"']
    \end{tikzcd}
    $$
   The square on the left is commutative because the cofiltered systems $\{\scrF^+_{m,d}|_{U_i}\}$ and $\{\scrG^+_{i,m,\alpha}\}$ are cofinal to each other. By Lemma \ref{Lemma: proj. formula lemma 1}, $\Theta_i=\Theta|_{U_i}$ is an almost isomorphism. This implies that $\Theta[1/p]$ is an isomorphism of projective Kummer \'etale Banach sheaves of $\scrO_{X_{\ket}}\widehat{\otimes}R$-modules.

We claim that $\Psi_i$ also becomes an isomorphism after inverting $p$. By construction, $\Psi_i$ factors as the composition of
$$
\Psi''_i:\scrG_{i}^+ \bigotimes_{\scrO^+_{U_{i, \ket}}\widehat{\otimes}' R}R^j\nu_*\left(\widehat{\scrO}^+_{U_{i, \proket}}\widehat{\otimes}' R\right)\rightarrow \varprojlim_m\left(\scrG^+_{i,m}\bigotimes_{\scrO^+_{U_{i,\ket}}\otimes_{\Z_p}(R/\fraka^m)}\left(R^j\nu_*\widehat{\scrO}^+_{U_{i,\proket}}\otimes_{\Z_p} (R/\fraka^m)\right)\right)
$$
and a canonical isomorphism
\begin{align*}
\Psi'_i: & \varprojlim_m\left(\scrG^+_{i,m}\bigotimes_{\scrO^+_{U_{i,\ket}}\otimes_{\Z_p}(R/\fraka^m)}\left(R^j\nu_*\widehat{\scrO}^+_{U_{i,\proket}}\otimes_{\Z_p} (R/\fraka^m)\right)\right)\\ = &\varprojlim_m\left(\varinjlim_{\alpha}\scrG^+_{i,m,\alpha}\bigotimes_{\scrO^+_{U_{i,\ket}}\otimes_{\Z_p}(R/\fraka^m)}\left(R^j\nu_*\widehat{\scrO}^+_{U_{i,\proket}}\otimes_{\Z_p} (R/\fraka^m)\right)\right)\\
\cong & \varprojlim_m\varinjlim_{\alpha}R^j\nu_*\left(\scrG^+_{i,m,\alpha}\bigotimes_{\scrO^+_{U_{i,\ket}}\otimes_{\Z_p}(R/\fraka^m)}\left(\widehat{\scrO}^+_{U_{i,\proket}}\otimes_{\Z_p} (R/\fraka^m)\right)\right)\\
=&\varprojlim_m\varinjlim_{\alpha} R^j\nu_* \widehat{\scrG}^+_{i,m,\alpha}=\varprojlim_mR^j\nu_*\widehat{\scrG}^+_{i,m}
\end{align*}
where the second isomorphism follows from the fact that each $\scrG^+_{i,m,\alpha}$ is a finite free $\scrO^+_{U_{i,\ket}}\otimes_{\Z_p} (R/\fraka^m)$-module.

It remains to prove that $\Psi''_i$ becomes an isomorphism after inverting $p$. Recall that $U_i$ admits a toric chart $U_i\rightarrow \Spa(\C_p\langle P\rangle, \calO_{\C_p}\langle P\rangle)$ for some sharp fs monoid $P$. By choosing an identification $\Gamma:=\Hom(P^{\mathrm{gp}}_{\Q_{\geq 0}}/P^{\mathrm{gp}}, \boldsymbol{\mu}_{\infty})\simeq \widehat{\Z}(1)^n$, Lemma \ref{Lemma: Cartan-Leray} yields an isomorphism $R^j\nu_* \widehat{\scrO}_{U_{i,\proket}}\simeq \bigwedge^j (\scrO_{U_{i,\ket}})^n$.

On one hand, by Proposition \ref{Proposition: compatibility with completed tensor}, we have
\begin{align*}
\scrG_{i}^+ \bigotimes_{\scrO^+_{U_{i, \ket}}\widehat{\otimes}' R}R^j\nu_*\left(\widehat{\scrO}^+_{U_{i, \proket}}\widehat{\otimes}' R\right)[\frac{1}{p}] & =\scrG_{i} \bigotimes_{\scrO_{U_{i, \ket}}\widehat{\otimes} R}R^j\nu_*\left(\widehat{\scrO}_{U_{i, \proket}}\widehat{\otimes} R\right)\\
& \simeq \scrG_{i} \bigotimes_{\scrO_{U_{i, \ket}}\widehat{\otimes} R}\left((R^j\nu_*\widehat{\scrO}_{U_{i, \proket}})\widehat{\otimes} R\right)\\
& =\scrG_i \bigotimes_{\scrO_{U_{i,\ket}}}R^j\nu_*\widehat{\scrO}_{U_{i,\proket}}\simeq \scrG_i \bigotimes_{\scrO_{U_{i,\ket}}} \bigwedge^j (\scrO_{U_{i,\ket}})^n.
\end{align*}

On the other hand, if we write $R/\fraka^m\simeq \bigoplus_{\sigma\in \Sigma_m} \Z/p^{\sigma}$, then we have
$$R^j\nu_*\widehat{\scrO}^+_{U_{i,\proket}}\otimes_{\Z_p} (R/\fraka^m)\simeq \bigoplus_{\sigma\in \Sigma_m} R^j\nu_* (\scrO^+_{U_{i,\proket}}/p^{\sigma}).$$
By Lemma \ref{Lemma: Cartan-Leray} (iii), there is an almost injection
$$\bigwedge^j (\scrO^+_{U_{i,\ket}}/p^{\sigma})^n\hookrightarrow R^j\nu_*(\scrO^+_{U_{i,\proket}}/p^{\sigma})$$ 
whose cokernel is killed by $p$. This yields an almost injection
\begin{align*}
&\scrG^+_i\bigotimes_{\scrO^+_{U_{i,\ket}}}\left(\bigwedge^j(\scrO^+_{U_{i,\ket}})^n\right)=\varprojlim_m\scrG^+_{i,m}\bigotimes_{\scrO^+_{U_{i,\ket}}\otimes_{\Z_p}(R/\fraka^m)}\left(\bigwedge^j (\scrO^+_{U_{i,\ket}}\otimes_{\Z_p}(R/\fraka^m))^n\right)\\
=& \varprojlim_m\scrG^+_{i,m}\bigotimes_{\scrO^+_{U_{i,\ket}}\otimes_{\Z_p}(R/\fraka^m)}\left(\bigoplus_{\sigma\in \Sigma_m}\bigwedge^j (\scrO^+_{U_{i,\ket}}/p^{\sigma})^n\right)\hookrightarrow \varprojlim_m\scrG^+_{i,m}\bigotimes_{\scrO^+_{U_{i,\ket}}\otimes_{\Z_p}(R/\fraka^m)}\left(\bigoplus_{\sigma\in \Sigma_m} R^j\nu_*(\scrO^+_{U_{i,\proket}}/p^{\sigma})\right)\\
\simeq & \varprojlim_m\scrG^+_{i,m}\bigotimes_{\scrO^+_{U_{i,\ket}}\otimes_{\Z_p}(R/\fraka^m)}\left(R^j\nu_*\widehat{\scrO}^+_{U_{i,\proket}}\otimes_{\Z_p} (R/\fraka^m)\right)
\end{align*}
with cokernel killed by $p$. 

Consequently, both sides of $\Psi''_i$ are isomorphic to $\scrG_i\bigotimes_{\scrO_{U_{i,\ket}}}\left(\bigwedge^j(\scrO_{U_{i,\ket}})^n\right)$ after inverting $p$, and one checks that $\Psi''_i[1/p]$ is just the identity map on $\scrG_i\bigotimes_{\scrO_{U_{i,\ket}}}\left(\bigwedge^j(\scrO_{U_{i,\ket}})^n\right)$. This finishes the proof.
\end{proof}

\begin{Corollary}\label{Corollary: generalised projection formula with invariants}
Let $X$ be a locally noetherian fs log adic space which is log smooth over $(\C_p, \calO_{\C_p})$. Let $\scrF$ be an admissible projective Kummer \'etale Banach sheaf of $\scrO_{X_{\ket}}\widehat{\otimes} R$-modules, with the corresponding integral structure $\scrF^+$. Suppose $\scrF^+$ is equipped with an $\scrO^+_{X_{\ket}}\widehat{\otimes}'R$-linear action of a finite group $G$. This induces an $\scrO_{X_{\ket}}\widehat{\otimes}R$-linear action of $G$ on $\scrF$. Then the subsheaf of $G$-invariants $\scrF^G$ also satisfies the generalised projection formula. More precisely, we have a natural isomorphism\[
    \scrF^G \otimes_{\scrO_{X_{\ket}}}R^i\nu_*\widehat{\scrO}_{X_{\proket}}\xrightarrow{\sim} R^i\nu_*\widehat{\scrF^G}
\]
\end{Corollary}
\begin{proof}
By Proposition \ref{Proposition: generalised projection formula}, we have an isomorphism \[
    \scrF\otimes_{\scrO_{X_{\ket}}}R^i\nu_*\widehat{\scrO}_{X_{\proket}}\xrightarrow{\sim}R^i\nu_*\widehat{\scrF}.
\] Taking the $G$-invariants, we obtain an isomorphism
\[
    \scrF^G\otimes_{\scrO_{X_{\ket}}}R^i\nu_*\widehat{\scrO}_{X_{\proket}}\xrightarrow{\sim} \left(R^i\nu_*\widehat{\scrF}\right)^G.
\] %The reason why it is still an isomorphism as is that $G$ is a finite group, so $G$-invariants is an exact functor; we can check this for every $U \in X_{\ket}$: \[ H^i(U_{\proket},{\widehat{\scrF}})^G={\left(\scrF(U) \widehat{\otimes}_{(\scrO_{X_{\ket}}\widehat{\otimes} R)} \left(H^i(U_{\proket},\widehat{\scrO}_{X_{\proket}}\widehat{\otimes}R) \right) \right)}^G.\]

It remains to show $\left(R^i\nu_*{\widehat{\scrF}}\right)^G \cong  R^i\nu_*{\widehat{\scrF^G}}$. Indeed, consider the following commutative diagram
$$ \begin{tikzcd}
      \scrO_{X_{\proket}}[G]-\Mod \arrow[r, "\nu_*"]\arrow[d, "(-)^G"'] &  \scrO_{X_{\ket}}[G]-\Mod\arrow[d, "(-)^G"] \\
       \scrO_{X_{\proket}}-\Mod \arrow[r, "\nu_*"] & \scrO_{X_{\ket}}-\Mod
    \end{tikzcd}
    $$
Notice that the higher right derived functors of both of the vertical arrows vanish as $G$ is a finite group and the base field is of characteristic zero. Now, applying the standard Grothendieck spectral sequence argument to both compositions $\nu_*\circ (-)^G$ and $(-)^G\circ \nu_*$, we obtain the desired commutativity of $R^i\nu_*$ and $(-)^G$.
\end{proof}