\section{Programming examples}\label{sect:examples}

In this section we illustrate the {expressivity of the type system by more significant} examples. 
For sake of readability, we use additional constructs, such as {operators of} primitive types, static methods{, private fields,} and while loops. Moreover, we generally omit the
brackets of the outermost block, and abbreviate $\Block{\Dec{\T}{\x}{\e}}{\e'}$ by $\Sequence{\e}{\e'}$ when $\x\not\in\FV{\e'}$, with $\FV{\e}$ the set of the free variables
 of $\e$. 
  
\paragraph{Capsules and swapping}
{The following example illustrates} how a programmer can declare lent references to achieve {recovery}.
The class \Q@CustomerReader@ below models reading information about customers out of a text file formatted as shown in the example:

\begin{lstlisting}
Bob
1 500 2 1300
Mark
42 8 99 100
\end{lstlisting}

In even lines we have customer names, in odd lines we have a shop history: a sequence of product codes.

\begin{lstlisting}
class CustomerReader {
  static capsule Customer readCustomer(lent Scanner s){
      mut Customer c=new Customer(s.nextLine())
      while(s.hasNextNum()){
        c.addShopHistory(s.nextNum())
      }
      return c //ok, capsule recovery
  }
}
\end{lstlisting}
The method \Q@CustomerReader.readCustomer@ takes a \Q@lent Scanner@, assumed to be a class similar to the one in Java,
for reading a file and extracting different kinds of data.
A \Q@Customer@ object is {created reading its name} from the file, and then its shop history is added.
Since the scanner is declared $\lent$, and there are no other parameters, by  {recovery} the result can be declared $\capsule$. {That is, we can statically ensure that the data of the scanner are not mixed with the result.}
Previous work offers cumbersome solutions requiring the programmer to manually handle multiple initialization phases like ``raw'' and ``cooked''~\cite{ZibinEtAl10}.

{The following method \Q@update@ illustrates how we can ``open'' capsules, modify their values and then recover the original \Q@capsule@ guarantee. The} 
method  takes an old customer (as capsule) and a \Q@lent Scanner@ as before.
{%\scriptsize
\begin{lstlisting}
class CustomerReader {...//as before
  static capsule Customer update(capsule Customer old,
                                  lent Scanner s){
    mut Customer c=old//we open the capsule `old'
    while(s.hasNextNum()){
      c.addShopHistory(s.nextNum());
    }recovery
    return c; //ok, capsule recovery
  }
}
\end{lstlisting}
}
Every method {with no} mutable parameters can use the pattern illustrated above: one (or many) capsule parameters are opened  (that is, assigned to mutable local variables) and, in the end, the result is guaranteed to be again a capsule. {This mechanism is not possible in previous work \cite{Almeida97,ClarkeWrigstad03,DietlEtAl07}. The type systems of Gordon et al.~\cite{GordonEtAl12} and Pony \cite{ClebschEtAl15} lack borrowed references in the formalization, but could support the variant where the scanner is isolated ($\capsule$ in our teminology) without losing isolation of the scanner or customer (the
 scanner could remain isolated throughout, relying on their support for dispatching appropriate
 methods on isolated references without explicit unpacking)}.


{In the former example, explicit \capsule\ opening and recovery take place \emph{inside the method body}.
Consider the following alternative version:}
\begin{lstlisting}
class CustomerReader {...//as before
  static mut Customer update(mut Customer c,lent Scanner s){
    while(s.hasNextNum()){
      c.addShopHistory(s.nextNum());
    }
    return c;
  }
}
\end{lstlisting}
{This method {takes now a \mutable\ object and returns} another \mutable\ one, as it usually happens in imperative programming.
Surprisingly enough, this method turns out to be just a more flexible version of the former one. Indeed:
\begin{itemize}
\item If called on \mutable\ data, then it returns \mutable\ data, while a call of the former method was ill-typed.
\item If called on \capsule\ data, then capsule opening implicitly takes place during method call execution, and recovery can be achieved for the method call expression, returning a \capsule\ as for the former method.
\end{itemize}
That is, recovery happens \emph{on the client side}\footnote{{The type systems  of Gordon et al.~\cite{GordonEtAl12} and Pony \cite{ClebschEtAl15} could support a variant analogously to the case above.}}. However, a client does not need to worry about this, and can simply call the method.
In a sense, this version of \Q@update@ is polymorphic: it can be used as either
$\mutable\rightarrow\mutable$ or as $\capsule\rightarrow\capsule$.}

{This pattern can be used in combination with function composition. That is, a chain of $\mutable\rightarrow\mutable$ methods can be called, and, if we start from a \capsule, we will get a \capsule\ in the end.
As shown in our example, these methods can have additional (non \mutable) parameters.}

{Moreover, this method can be trasparently used as $\lent\rightarrow\lent$. That is, if called on  $\lent$ data, then it returns $\lent$ data, by applying rule \rn{t-swap} to the method call expression. Again this can be extended to chains of methods which may have additional (non \mutable) parameters. }

Finally, we show the code of the \Q@Scanner@ itself, and how swapping can be used to update {the
fields of a $\lent$ scanner in a safe way.}

\begin{lstlisting}
class Scanner{
  mut InputStream stream;
  imm String nextLine(mut/*=this qualifier*/){...}
  int nextNum(mut/*=this qualifier*/){...}
  bool hasNextNum(read/*=this qualifier*/){...}
}

lent Scanner s=...
mut InputStream stream1=...
capsule InputStream stream2 = ... 
//s.stream=stream1 //(a)wrong
s.stream=new InputStream("...")//(b)ok, swapping 
s.stream=stream2 //(c)ok, swapping 
\end{lstlisting}

In our type system,  a $\lent$ reference can be regarded as mutable if all the mutable references are regarded as $\lent$, as formally modeled by rule \rn{t-swap}.
% {This mechanism is similar to \emph{viewpoint adaptation}
%as in \cite{DietlEtAl07}. 
This can be trivially applied to line (b), where \Q@s@ is the only free variable, and to  line (c), where the other free variable is declared $\capsule$. 
In line (a), instead, swapping would assign a $\lent$ type to \lstinline{stream1}.

{The $\this$ qualifiers reflect the fact that the first two methods modify, whereas the third only reads, the scanner's state. Note that, as in the previous example, the first two methods can be safely applied to $\lent$ scanners by swapping (in this case the result type, being not mutable, remains the same). Note also that such methods (as any method with no parameters and non mutable result type) could be equivalently have $\this$ qualifier $\lent$, since, intuitively, no aliasing can be introduced for $\this$ (formally, we can apply rule \rn{t-swap} to the method body). On the other hand, the first two methods can be invoked on $\lent$ scanners by by applying rule \rn{t-swap} to the method call expression.}

\paragraph{Readable and immutable references}
We illustrate now the \Q@read@ and \Q@imm@ qualifiers by the example of a \Q@Person@ class with a list of friends.
\begin{lstlisting}
class Person{  
  private mut PersonList friends;  
  read PersonList readFriends (read/*=this qualifier*/) {
    return this.friends;
  }
}
\end{lstlisting}
To give access to the private field, we declare a method {which is like a usual getter, except} that it returns a \Q@read PersonList@ reference. 
In this way, a client can only read the list of friends obtained through an invocation \lstinline{person.readFriends()}{}, with \lstinline{person} of class \lstinline{Person}.
Note that, since the $\this$ qualifier of the method is \lstinline{read}{}, which is the top of the subtyping hierarchy, it can be invoked whichever is the qualifier of \lstinline{person}.
Moreover, in the case \lstinline{person} is \lstinline{imm}, we get an \Q@imm PersonList@ back, by {recovering immutability for the \Q@read@ result}.
Indeed, in this case, we can apply rule \rn{t-imm} to the invocation \lstinline{person.readFriends()}, since the only free variable \Q@person@ is immutable, so no variable needs to be {restricted}.
{This is another case where the method is polymorphic: it can be used as either
$\readable\rightarrow\readable$ or as $\imm\rightarrow\imm$, and a client does not need to worry about, and can simply call the method.}

Assume now that we want, instead, to give permission to a client to modify the list of friends.
In this case, we can declare a getter method with different type annotations:\label{exposer}
\begin{lstlisting}
mut PersonList getFriends(mut/*=this qualifier*/) {
  return this.friends;
}
\end{lstlisting}
This method takes a mutable \Q@Person@ and returns a mutable \Q@PersonList@ reference.
Hence, it cannot be invoked on \Q@read@ or \Q@imm@ objects.
However, this $\mutable$ getter can be invoked on a lent \lstinline{person}{}:
in \Q@person.getFriends()@  the only free variable \Q@person@ can be seen as \Q@mut@.
The result of the method is turned in \Q@lent PersonList@ by the \rn{t-swap} rule {(formally, swapping the singleton {lent} group \Q@person@ with the empty set).}

{Our approach forces the programmer to explicitly choose either $\readable$ or $\mutable$ getters.}
Other works permits the developer to use
either multiple getters or polymorphic type qualifiers, for instance
the type annotation \lstinline{PolyRead} of Javari~\cite{TschantzErnst05} allows one
to keep a single method, providing an additional design choice for programmers.
However, we argue that forcing programmers to consider the two operations as logically different  can be a 
simpler and more explicit programming pattern.
 (In \refToSection{related} we discuss also a third variant, with return type $\lent$.)
%The getter only gives access to the field, so that the data can be read by a client.
%The exposer, instead, gives permission to a client to modify the reachable object graph of the receiver by returning a mutable reference to the field.
In a language supporting many levels of visibility (like protected, package, friend, ...) a programmer may choose a restricted visibility for $\mutable$ getters and a more permissive visibility for $\readable$ getters.
Collection classes also can declare $\readable$ and $\mutable$ getters, as in the following example.
\begin{lstlisting}
class PersonList{...
 void add(mut, mut Person elem){...}
 read Person readElem (read, int index){...}
 mut Person getElem(mut, int index){...}
}
\end{lstlisting}
Finally, we show how we can create mutable objects, mutate them for a while, and then {recover their immutabiity}:
\begin{lstlisting}
class C{
  static mut Person lonelyPersonFactory(){
    return new Person(new PersonList());
  }
  static imm Person happyPersonFactory(){
    mut Person fred=C.lonelyPersonFactory();
    mut Person barney=C.lonelyPersonFactory();
    fred.getFriends().add(barney);
    barney.getFriends().add(fred);
    return fred; //mut Person recovered to be imm 
    //now fred and barney are friends forever!
  }
}
\end{lstlisting}
Here \Q@lonelyPersonFactory()@  creates lonely \Q@Person@s, that have no friends.
However, there is still hope, since they are mutable:
\Q@happyPersonFactory@ creates two lonely people, \Q@fred@ and \Q@barney@, and makes them friends.
Then the function returns \Q@fred@ (and, indirectly, also \Q@barney@ that is now in the reachable object graph of \Q@fred@).
The function body does not use any free variable, so we can {recover the capsule property of} the result, hence return it as \Q@imm@.

\paragraph{Qualifiers are deep} Note that recovery work since qualifiers have a \emph{deep/full} interpretation, in the sense that
they are propagated to the whole reachable object graph. In
a shallow interpretation, instead, it is possible, for instance,
to reach a mutable object from an immutable object. The
advantage of such interpretation is that libraries can declare strong intentions in a coherent and uniform way, independently of the concrete representation of the user input
(that, with the use of interfaces, could be unknown to the
library). On the other side, providing (only) deep/full qualifiers
means that we do not offer any language support for (as an
example) an immutable list of mutable objects.

\paragraph{Nested recovery and groups}
We conclude the section by an example showing that {recoveries} can be nested, {and, to ensure that all are safe, distinct lent groups must be kept.}
Consider the following code, where implementation of \lstinline{A}{} is omitted to emphasize that only type information provided by qualifiers is significant.
%\newpage
\begin{lstlisting}
class A{...
  mut A mix(mut, mut A a){...}
  //inserts a  in the reachable object graph 
    //of the receiver and returns a
  capsule A clone (read){...} 
  //returns a capsule clone of the receiver
  static mut A parse(){...} //reads an A from input
}

mut A a1= A.parse() //outside of recovery
capsule A outerA={//outer recovery
  mut A a2= A.parse()//inside outer recovery
  capsule A nestedA={//nested recovery
    mut A a3= A.parse()//inside nested recovery
    mut A res= ...
    res.mix(a3)
    //this is promoted and assigned to nestedA
  }
  nestedA.mix(a2)
  //this is promoted and assigned to outerA
}
//program continues, using outerA as capsule
\end{lstlisting}

The question is, what can we write instead of the dots, and why.
Clearly, (1) \Q@a3@ is allowed, {since the result of the inner block will be only connected to a local reference}, while
(2) \Q@a1@ and \Q@a2@ are not, since it will be connected to an external mutable reference. 
However, (3) \Q@a1.clone()@ and \Q@a2.clone()@ are allowed, since the method \lstinline{clone} returns a capsule.
In the same way,
(4) \Q@a2.mix(a2).clone()@ is allowed, as well as
 \Q@a1.mix(a1).clone()@.

However, when we start mixing different variables, things become trickier.
 For example, (5) \lstinline{a2.mix(a1).clone()}{} is not well-typed in our type system.
  Indeed, even though, thanks to cloning, mixing \Q@a2@
 and \Q@a1@ does not compromise the capsule well-formedness of \lstinline{nestedA}{} (that is, the nested recovery can be safely applied), the fact that \Q@a2@ and 
 \Q@a1@ are mixed could compromise the capsule well-formedness of \Q@outerA@ when \Q@outerA@ is computed (that is, the outer recovery would be unsafe).

In summary, mixing \Q@a@s {lent groups introduced} for different {recoveries} must be avoided.
Rule \rn{t-swap} swaps one group with another,
thus keeping them distinct.

This last example \label{comparison} illustrates many of the differences w.r.t. {the type system proposed in \cite{GordonEtAl12}, whose notion of \emph{recovery} is less expressive}.
Their system allows (1), and rejects (2) and (5), as ours. However, they conservatively rejects (3) and (4), since 
 the flow is not tracked at a fine enough granularity.
Depending on the concrete application, programmers may need to work around the
limitations of \cite{GordonEtAl12} by reordering local variables,
 introducing stricter type qualifiers or, in general, re-factoring their code.
 However, there may be cases where there is no possible reordering.