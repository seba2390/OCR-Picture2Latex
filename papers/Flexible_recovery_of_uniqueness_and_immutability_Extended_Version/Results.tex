% !TEX root =main.tex
\section{ {Type safety and properties of the type system}}\label{sect:results}
 {In this section we present  {the results}. We first give a characterisation of the values in terms of the
properties of their free variables, which  {correspond to} their reachable graph. Then we show the soundness
of the type system for the operational semantics, and finally we formalize the expected behaviour of capsule and
immutable references.}

 {In the following, we denote by $\deriv:\TypeCheckShort{\Delta}{\e}{\T}$ a derivation tree for the judgement 
 $\TypeCheckShort{\Delta}{\e}{\T}$. 
Moreover, we call the rules where expressions in the premises are the direct subterms of the expression in the consequent,  \emph{structural}; the others, that is, \rn{t-capsule}, \rn{t-imm}, \rn{t-swap}, \rn{t-unrst}, and \rn{t-sub},  \emph{non-structural}.

\subsection{Canonical Forms}
In a type derivation, given a construct, % (in the case of right-values we have constructors or blocks), 
in addition to the corresponding structural rule we can have applications of non-structural rules.  
In this section we first present some results exploring the effect of non-structural rules on the lent groups and the mutable groups of variables, and on the modifier derived for
the expression, then we give an inversion lemma for blocks (the construct which is relevant
for the analysis of right-values), and finally we present the Canonical Form theorem with its proof. 

Given a type judgement $\AuxTypeCheck{\Gamma}{\LentLocked}{\MutGroup}{\StronglyLocked}{\e}{\Type{\mu}{\C}}$,  {application of non-structural rules can only modify the lent groups and the mutable group by swapping, hence leading to a permutation $\LentLocked'\, \MutGroup$ of $\LentLocked\, \MutGroup$. In other terms,} the equivalence relation on $\domMut{\Gamma}$
 {they induce} is preserved,
as the following lemma shows.
\begin{lemma} [Non-structural rules]\label{lemma:nonStructural}
If $\deriv:\AuxTypeCheck{\Gamma}{\LentLocked}{\MutGroup}{\StronglyLocked}{\e}{\Type{\mu}{\C}}$, then 
there is a sub-derivation $\deriv'$ of $\deriv$ such that $\deriv':\AuxTypeCheck{\Gamma}{\LentLocked'}{\MutGroup'}{\StronglyLocked'}{\e}{\Type{\mu'}{\C}}$ ends with the application of a structural rule,  $\LentLocked\ \MutGroup=\LentLocked'\ \MutGroup'$, and $\mu\not=\imm$, $\mu\not=\capsule$ implies $\mu'\leq\mu$.
\end{lemma}
\begin{proof}
The proof is in~\ref{sect:proof-cf}.\qed
\end{proof}
The application of non-structural rules is finalized to the recovery of capsule and immutable properties
as expressed by the following lemma.
\begin{lemma}\label{lemma:typeStruct} 
Let $\deriv:\TypeCheck{\Gamma}{\LentLocked}{\StronglyLocked}{{\e}}{\Type{\mu}{\C}}$. 
\begin{enumerate}
  \item If $\mu=\mutable$, then the last rule applied in $\deriv$ cannot be \rn{t-swap} or \rn{t-unrst} or \rn{t-capsule} or  \rn{t-imm}.
  \item If $\mu=\imm$ or $\mu=\capsule$ and the last rule applied in $\deriv$ is  \rn{t-swap} or \rn{t-unrst}, then
  $\TypeCheck{\Gamma}{\LentLocked'}{\StronglyLocked'}{{\e}}{\Type{\mu}{\C}}$, for some $\LentLocked'$ and 
  $\StronglyLocked'$.
\end{enumerate}
\end{lemma}
\begin{proof}\
\begin{enumerate}
  \item Rule \rn{t-unrst} is only applicable when the type derived for the expression in the premise of the rule, $\Type{\mu'}{\C}$, is such that $\mu'\leq \imm$. For the rule \rn{t-swap}, when the type derived for the expression in the premise of the rule is $\Type{\mutable}{\C}$, then the the type of the expression in the consequent is  $\Type{\lent}{\C}$. Moreover, the two  {recovery} rules do not derive types with the $\mutable$ modifier.
  \item Immediate.
\end{enumerate}\qed
\end{proof}
From the previous lemma we derive that, if $\TypeCheck{\Gamma}{\LentLocked}{\StronglyLocked}{{\e}}{\Type{\mu}{\C}}$ with $\mu\not\geq\lent$, then without loss of generality we can assume that the last rule applied in the derivation is either a structural rule followed by a \rn{t-sub}, or a  {recovery rule} in case $\mu=\capsule$ or $\mu=\imm$. 
Moreover, rules \rn{t-swap} and \rn{t-unrst} cannot be used to derive the premise of rule \rn{t-capsule}.

 {If we can derive a type for a variable, then such type depends on the type assignment and the lent groups (if the variable is restricted no type can be derived).
If the variable is declared with modifier $\imm$ or $\capsule$, then its type depends only on the type assignment.}
\begin{lemma}\label{lemma:typeVars}
If $\TypeCheck{\Gamma}{\LentLocked}{\StronglyLocked}{{\x}}{\T}$ then $\Gamma(x)\leq\T$. Moreover, if 
$\T=\Type{\mu}{\C}$ with $\mu\leq\imm$, then $\TypeCheck{\Gamma}{\LentLocked'}{\StronglyLocked'}{{\x}}{\T}$ for all 
$\LentLocked'$ and $\StronglyLocked'$ such that  $\WellFormedTypeCtx{\Gamma};{\LentLocked'};\StronglyLocked'$.
\end{lemma}
\begin{proof}
The proof is in~\ref{sect:proof-cf}.\qed
\end{proof}
The block construct is central to our recovery technique. New variable are defined, and the lent and mutable
groups of the free variables of the block may be modified by the introduction of the newly defined
variables. However, as for the non-structural rules, the equivalence relation between the mutable 
variables induced by the partition determined by the lent and   mutable groups is preserved.

We first define a judgment asserting when a declaration of a variable is well typed and then give a lemma
relating the type judgement for a  block with the type judgements for the expressions
associated to its local variables by the declarations and for its body. The expression associated
with a variable declared with modifier $\lent$ has to be well typed taking as mutable the variable in its
group, whereas the others have the same mutable group of the body of the expression.  This expresses
the fact that variables declared with modifier $\lent$ should not be connected to the result of the block
(its body).
\begin{definition}\label{def:wellTypedDefsNew}
Define $\AuxDecsOK{\Gamma}{\LentLocked}{\MutGroup}{\StronglyLocked}{\Dec{\Type{\mu}{\C}}{\x}{\e}}$ by:
\begin{itemize}
  \item $\AuxTypeCheck{\Gamma}{\LentLocked}{\MutGroup}{\StronglyLocked}{\e}{\Gamma(\x)}$ if $\mu\not=\lent$
  \item $\AuxTypeCheck{\Gamma}{\LentLocked'}{\MutGroup'}{\StronglyLocked}{\e}{\Gamma(\x)}$ if $\mu=\lent$ and $\LentLocked\ \xs=\LentLocked'\ \xs'$ where $\x\in\xs'$.
\end{itemize}
$\AuxDecsOK{\Gamma}{\LentLocked}{\MutGroup}{\StronglyLocked}{\decs}$ if for all $\dec\in\decs$, $\AuxDecsOK{\Gamma}{\LentLocked}{\MutGroup}{\StronglyLocked}{\dec}$.
\end{definition}

\begin{lemma} [Inversion for blocks]\label{lemma:inversionBlock}
If $\AuxTypeCheck{{\Gamma}}{\LentLocked}{\MutGroup}{\StronglyLocked}{\Block{\decs}{\val}}{\Type{\mu}{\C}}$, then for some $\LentLocked'$ and $\xs'$
\begin{enumerate}
\item $\AuxDecsOK{\SubstFun{\Gamma}{\TypeEnv{\decs}}}{\LentLocked'}{\MutGroup'}{\StronglyLocked'}{\decs}$
\item  $\AuxTypeCheck{{\SubstFun{\Gamma}{\TypeEnv{\decs}}}}{{\LentLocked'}}{\MutGroup'}{\StronglyLocked'}{\val}{\Type{\mu'}{\C}}$ 
\item $\LessEq{(\LentLocked\ \xs){\setminus}\dom{\TypeEnv{\decs}}}{\LentLocked'\ \xs'}$ and
\item $\mu\not=\imm$ and $\mu\not=\capsule$ implies $\mu'\leq\mu$.
\end{enumerate}
\end{lemma}
\begin{proof}
The proof is in~\ref{sect:proof-cf}.\qed
\end{proof}
The two following lemmas characterize the shape of $\mutable$ right-values. 
For constructor values we also have that if the mutable variables are restricted then all fields
must have immutable types.
\begin{lemma}[Constructor value]\label{lemma:constrMut}
Let $\fields{\C}{=}\Field{\Type{\mu_1}{\C_1}}{\f_1}\ldots\Field{\Type{\mu_n}{\C_n}}{\f_n}$.
\begin{enumerate}
  \item If $\AuxTypeCheck{\Gamma}{\LentLocked}{\MutGroup}{\StronglyLocked}{\ConstrCall{\C}{\z_1,\ldots,\z_n}}{\Type{\mutable}{\C}}$, then: for all $i\in1..n$, if $\mu_i=\mutable$, then $\z_i\in\xs$, otherwise $\AuxTypeCheck{\Gamma}{\LentLocked}{\MutGroup}{\StronglyLocked}{\z_i}{\Type{\mu}{\C_i}}$ with $\mu\leq\imm$.
\item If $\AuxTypeCheck{\Gamma}{\LentLocked}{\emptyset}{\domGeqMut(\Gamma) }{\ConstrCall{\C}{\z_1,\ldots,\z_n}}{\Type{\readable}{\C}}$, then: for all $i\in1..n$ we have that $\AuxTypeCheck{\Gamma}{\LentLocked}{\MutGroup}{\StronglyLocked}{\z_i}{\Type{\mu}{\C_i}}$ with $\mu\leq\imm$.
\end{enumerate}
\end{lemma}
\begin{proof}
The proof is in~\ref{sect:proof-cf}.\qed
\end{proof}
In case a right-value is a  block, the fact that all the declared variables are connected to the
body of the block implies that the modifiers of all the variables must be either $\mutable$ or $\leq\imm$.
\begin{lemma}[Block value]\label{lemma:blockMut}
If $\AuxTypeCheck{\Gamma}{\LentLocked}{\MutGroup}{\StronglyLocked}{\Block{\dvs}{{\cOrx}}}{\Type{\mutable}{\C}}$, where
$\dvs=\Dec{\Type{\mu_1}{\C_1}}{\z_1}{\stVal_1}\cdots\Dec{\Type{\mu_n}{\C_n}}{\z_n}{\stVal_n}$, then
  \begin{enumerate}
  \item $\AuxTypeCheck{\Gamma[\TypeEnv{\dvs}]}{{\LentLocked'}}{\xs'}{\StronglyLocked}{{\cOrx}}{\Type{\mutable}{\C}}$,
  \item $\AuxTypeCheck{\Gamma[\TypeEnv{\dvs}]}{{\LentLocked'}}{\xs'}{\StronglyLocked}{\stVal_i}{\Type{\mutable}{\C_i}}$ 
  $\Range{i}{1}{n}$ 
\item for all $\z\in\FV{\Block{\dvs}{\cOrx}}$, if $\Gamma(\z)=\Type{\mutable}{\D}$, then $\z\in\xs$, otherwise 
$\AuxTypeCheck{\Gamma}{\LentLocked}{\MutGroup}{\StronglyLocked}{z}{\Type{\mu}{\C}}$ with $\mu\leq\imm$.
\end{enumerate}
\end{lemma}
\begin{proof}
The proof is in~\ref{sect:proof-cf}.\qed
\end{proof}
As usual typing depends only on the free variable of expressions, as the following weakening lemma states.
\begin{lemma}[Weakening]\label{lemma:weakening}
Let $\dom{\Gamma'}\cap\FV{\e}=\emptyset$ and $\WellFormedTypeCtx{\SubstFun{\Gamma}{\Gamma'}};{\LentLocked};\StronglyLocked$. \\
$\TypeCheck{\SubstFun{\Gamma}{\Gamma'}}{\LentLocked}{\StronglyLocked}{\e}{\T}$ if and
only if  $\TypeCheck{{\Gamma}}{{\LentLocked}{\setminus}{{\dom{\Gamma'}}}}{{\StronglyLocked}{\setminus}{{\dom{\Gamma'}}}}{\e}{\T}$.
\end{lemma}
\begin{proof}
By induction on derivations. \qed
\end{proof}
The Canonical Forms theorem describes constraints on the free variables and the extracted type of well-typed right-values. In particular, $\capsule$ and $\imm$ right-values can contain only $\capsule$ or $\imm$ references, and 
$\mutable$ right-values cannot contain $\lent$ or $\readable$ references. Moreover, the type extracted from $\capsule$ and $\mutable$ right-values is necessarily $\mutable$.  
\begin{theorem}[Canonical Forms]\label{theo:canonicalForm}
If $\AuxTypeCheck{\Gamma}{\LentLocked}{\MutGroup}{\StronglyLocked}{\stVal}{\Type{\mu}{\_}}$, and $\y\in\FV{\stVal}$, then:
 \begin{enumerate}
 \item if $\mu=\capsule$, then $\AuxTypeCheck{\Gamma}{\LentLocked}{\MutGroup}{\StronglyLocked}{\y}{\Type{\imm}{\_}}$, and $\typeOf{\stVal}=\Type{\mutable}{\_}$ 
  \item if $\mu=\mutable$, then $\AuxTypeCheck{\Gamma}{\LentLocked}{\MutGroup}{\StronglyLocked}{\y}{\Type{\mu'}{\_}}$, with $\mu'\not\geq\lent$, and $\typeOf{\stVal}=\Type{\mutable}{\_}$
  \item if $\mu=\imm$, then $\AuxTypeCheck{\Gamma}{\LentLocked}{\MutGroup}{\StronglyLocked}{\y}{\Type{\imm}{\_}}$
   \item if $\mu=\lent$, then $\AuxTypeCheck{\Gamma}{\LentLocked}{\MutGroup}{\StronglyLocked}{\y}{\Type{\readable}{\_}}$, and $\typeOf{\stVal}\leq\Type{\lent}{\_}$ 
   \item if $\mu=\readable$, then $\AuxTypeCheck{\Gamma}{\LentLocked}{\MutGroup}{\StronglyLocked}{\y}{\Type{\readable}{\_}}$
 \end{enumerate}
\end{theorem}
\begin{proof} 
By structural induction on $\stVal$ and by cases on $\mu$. 

\medskip\noindent
Let \underline{$\stVal=\ConstrCall{\C}{\x_1,\ldots,\x_n}$} and $\fields{\C}{=}\Field{\Type{\mu_1}{\C_1}}{\f_1}\ldots\Field{\Type{\mu_n}{\C_n}}{\f_n}$.  By definition
$\typeOf{\ConstrCall{\C}{\x_1,\ldots,\x_n}}=\Type{\mutable}{\C}$.\\

\medskip\noindent
If \underline{$\mu=\capsule$},  from \refToLemma{typeStruct}.2 we can assume that the last rule applied in the derivation is
\rn{t-capsule}:
\[
\prooftree
\begin{array}{c}
\AuxTypeCheck{\Gamma}{\LentLocked\,\xs}{\emptyset}{\StronglyLocked}{{\ConstrCall{\C}{\zs}}}{\Type{\mutable}{\C}}
\end{array}
\justifies
\AuxTypeCheck{\Gamma}{\LentLocked}{\MutGroup}{\StronglyLocked}{{{\ConstrCall{\C}{\zs}}}}{\Type{\capsule}{\C}}
\using
\rn{t-capsule}
\endprooftree
\]
%where $\{\zs\}=\domMut{\Gamma}{\setminus}\LentLocked$. 
From \refToLemma{constrMut}.1, since the current mutable group is empty, 
$\Range{i}{1}{n}$ we have that
$\AuxTypeCheck{\Gamma}{\LentLocked}{\MutGroup}{\StronglyLocked}{\x_i}{\Type{\imm}{\C_i}}$.

\medskip\noindent
If \underline{$\mu=\mutable$}, from rule \rn{t-new}, 
$\AuxTypeCheck{\Gamma}{\LentLocked}{\MutGroup}{\StronglyLocked}{\ConstrCall{\C}{\x_1,\ldots,\x_n}}{\Type{\mutable}{\C}}$.
Therefore, from \refToLemma{constrMut}.1 we get the result. 

\medskip\noindent
Let \underline{$\mu=\imm$.} From \refToLemma{typeStruct}.2 we can assume that the last rule applied in the derivation is
either \rn{t-sub}, or \rn{t-imm}. 
In the first case
$\AuxTypeCheck{\Gamma}{\LentLocked}{\MutGroup}{\StronglyLocked}{{{\ConstrCall{\C}{\zs}}}}{\Type{\capsule}{\C}}$, so the proof for $\mu=\capsule$ applies.
In the second, we have 
\[
\prooftree
\begin{array}{c}
\deriv:\AuxTypeCheck{\Gamma}{\LentLocked\ \xs}{\emptyset}{\domGeqMut(\Gamma) }{{\ConstrCall{\C}{\zs}}}{\Type{\readable}{\C}}
\end{array}
\justifies
\AuxTypeCheck{\Gamma}{\LentLocked}{\MutGroup}{\StronglyLocked}{{{\ConstrCall{\C}{\zs}}}}{\Type{\imm}{\C}}
\using
\rn{t-imm}
\endprooftree
\]
%where $\zs=\domMut{\Gamma}{\setminus}\StronglyLocked$. 
From \refToLemma{constrMut}.2  we get the result. 

\medskip\noindent
Let \underline{$\mu=\lent$ or $\mu=\readable$.} Then to derive a type for $\ConstrCall{\C}{\xs}$ we have
to apply \rn{t-new} followed by \rn{t-sub}, so the result is obvious.\\

\bigskip\noindent
Let \underline{$\stVal=\Block{\dvs}{\cOrx}$} where $\dvs$ is 
$\Dec{\T_1}{\y_1}{\stVal_1}$ $\ldots$ $\Dec{\T_n}{\y_n}{\stVal_n}$, and  $\cOrx=\ConstrCall{\C}{\xs}$ or
$\cOrx=x$.  

\medskip\noindent
If \underline{$\mu=\capsule$}, 
then, from \refToLemma{typeStruct}.2 we can assume that the last rule applied in the derivation is
\rn{t-capsule}:
\[
\prooftree
\begin{array}{c}
\AuxTypeCheck{\Gamma}{\LentLocked\,\xs}{\emptyset}{\StronglyLocked}{\Block{\dvs}{\_}}{\Type{\mutable}{\C}}
\end{array}
\justifies
\AuxTypeCheck{\Gamma}{\LentLocked}{\MutGroup}{\StronglyLocked}{{{\Block{\dvs}{\_}}}}{\Type{\capsule}{\C}}
\using
\rn{t-capsule}
\endprooftree
\]
%where $\zs=\domMut{\Gamma}{\setminus}\LentLocked$. 
From \refToLemma{blockMut}, since the current mutable group is empty, 
$\Range{i}{1}{n}$ we have that
$\AuxTypeCheck{\Gamma}{\LentLocked}{\MutGroup}{\StronglyLocked}{\x_i}{\Type{\imm}{\C_i}}$.


\medskip\noindent
If \underline{$\mu=\mutable$}, \refToLemma{typeStruct}.1 implies that the last rule
of the type derivation is either an application of \rn{t-sub}, or of \rn{t-block}.
In the first case  the proof for $\mu=\capsule$ applies. In the second case, we have
\[
\TypeCheck{\Gamma}{\LentLocked}{\StronglyLocked}{\Block{\dvs}{\_}}{\Type{\mutable}{\C}}
\]
From \refToLemma{blockMut} we get the result.

\medskip\noindent
If \underline{$\mu=\imm$}, from \refToLemma{typeStruct}.2 we can assume that the last rule
applied is either \rn{t-sub} or \rn{t-imm}. If the rule
applied is \rn{t-sub}, then 
$\TypeCheck{\Gamma}{\LentLocked}{\StronglyLocked}{\Block{\dvs}{\_}}{\Type{\capsule}{\C}}
$, and the proof for $\mu=\capsule$ applies. \\
Let the last rule applied in $\deriv'$ be \rn{t-imm}, i.e.,
\[
\prooftree
\begin{array}{c}
\deriv:\AuxTypeCheck{\Gamma}{\LentLocked\ \xs}{\emptyset}{\domGeqMut(\Gamma) }{\Block{\dvs}{\_}}{\Type{\readable}{\C}}
\end{array}
\justifies
\AuxTypeCheck{\Gamma}{\LentLocked}{\MutGroup}{\StronglyLocked}{\Block{\dvs}{\_}}{\Type{\imm}{\C}}
\using
\rn{t-imm}
\endprooftree
\]
%where $\xs=\domMut{\Gamma}{\setminus}\LentLocked$.  \\
From \refToLemma{inversionBlock}, since we may assume that in the derivation $\deriv$ the sub-derivation
that ends with rule \rn{t-block} is not the antecedent of rule \rn{t-unrst} (otherwise the block would already have type $\Type{\imm}{\C}$) we have that
\begin{enumerate}[(a)]
\item  $\AuxTypeCheck{{\SubstFun{\Gamma}{\TypeEnv{\dvs}}}}{{\LentLocked'}}{\MutGroup'}{\StronglyLocked'}{\_}{\Type{\mu'}{\C}}$ 
\item $\AuxDecsOK{\SubstFun{\Gamma}{\TypeEnv{\dvs}}}{\LentLocked'}{\MutGroup'}{\StronglyLocked'}{\Dec{\Type{\mu_i}{\C_i}}{\z_i}{\stVal_i}}$  $\Range{i}{1}{n}$
\item $\LessEq{(\LentLocked\ \xs){\setminus}\dom{\TypeEnv{\dvs}}}{\LentLocked'\ \xs'}$ and
\item $\domGeqMut(\Gamma){\setminus}\dom{\dvs}=\StronglyLocked'$ 
\end{enumerate}
If $\y\in\FV{\Block{\dvs}{\_}}$, then either $\y\in\FV{{\_}}{\setminus}\dom{\dvs}$ or $\y\in\FV{\stVal_i}{\setminus}\dom{\dvs}$ for some $i\in 1..n$.\\
If $\y\in\FV{{\_}}{\setminus}\dom{\dvs}$ and $\y\in\domGeqMut(\Gamma)$, from (d) $\y\in\StronglyLocked'$. In order to apply rule \rn{t-var} to derive a type for $\y$ we
would have to apply rule \rn{t-unrst}. Therefore $\Gamma[\TypeEnv{\dvs}](\y)=\Type{\mu}{\D}$ with $\mu\leq\imm$.
(Rule \rn{t-unrst} cannot follow \rn{t-new}.) So $\y\not\in\domGeqMut(\Gamma)$, and $\Gamma(\y)=\Type{\mu}{\D}$ with $\mu\leq\imm$.\\
If for some $i\in 1..n$, $\y\in\FV{\stVal_i}{\setminus}\dom{\dvs}$ and $\y\in\domGeqMut(\Gamma)$, let 
$\deriv:\AuxTypeCheck{{\SubstFun{\Gamma}{\TypeEnv{\dvs}}}}{{\LentLocked_i}}{\MutGroup_i}{\StronglyLocked'}{\stVal_i}{\Type{\mu_i}{\C_i}}$. Since $\y\in\domGeqMut(\Gamma)$ there is a sub-derivation $\deriv'$ of $\deriv$, such that
$\deriv':\AuxTypeCheck{{\SubstFun{\Gamma}{\TypeEnv{\dvs}}}}{{\LentLocked'_i}}{\MutGroup'_i}{\StronglyLocked'}{\val'}{\Type{\mu'}{\C'}}$, the last  rule applied is \rn{t-unrst}, and $\y\in\FV{\val'}$, i.e.,
\[
\prooftree
\AuxTypeCheck{{\SubstFun{\Gamma}{\TypeEnv{\dvs}}}}{{\LentLocked'_i}}{\MutGroup'_i}{\emptyset}{\val'}{\Type{\imm}{\C'}}\ \ 
\justifies
\AuxTypeCheck{{\SubstFun{\Gamma}{\TypeEnv{\dvs}}}}{{\LentLocked'_i}}{\MutGroup'_i}{\StronglyLocked'}{\val'}{\Type{\imm}{\C'}}
\using
\rn{t-unrst}
\endprooftree
\] 
If $\val'=\y$ then $\SubstFun{\Gamma}{\TypeEnv{\dvs}}(\y)=\Type{\mu}{\D}$ with $\mu\leq\imm$, therefore also $\Gamma(\y)=\Type{\mu}{\D}$ with $\mu\leq\imm$, which is impossible since $\y\in\domGeqMut(\Gamma)$. \\
If $\val'=\stVal'$ for some $\stVal'$, by induction hypothesis
on $\stVal'$, we have that $\AuxTypeCheck{{\SubstFun{\Gamma}{\TypeEnv{\dvs}}}}{{\LentLocked'_i}}{\MutGroup'_i}{\emptyset}{\y}{\Type{\mu}{\D}}$ where $\mu\leq\imm$. Since $\y\not\in\dom{\dvs}$ and $\mu\leq\imm$, by \refToLemma{weakening} and \refToLemma{typeVars}, also $\Gamma(\y)=\Type{\mu}{\D}$ with $\mu\leq\imm$, which is again impossible since $\y\in\domGeqMut(\Gamma)$. \\
Therefore, for all $\y\in\FV{\Block{\dvs}{\_}}$, we have that 
$\TypeCheck{\Gamma}{\LentLocked}{\StronglyLocked}{\y}{\Type{\imm}{\D}}$ with $\mu\leq \imm$. 

\medskip\noindent
If \underline{$\mu=\lent$} or \underline{$\mu=\readable$}, then $\y$ must be such that
$\TypeCheck{\Gamma}{\LentLocked}{\StronglyLocked}{\y}{\Type{\mu'}{\D}}$, so also
$\TypeCheck{\Gamma}{\LentLocked}{\StronglyLocked}{\y}{\Type{\readable}{\D}}$.\\
Moreover, if \underline{$\mu=\lent$} and $\TypeCheck{\Gamma}{\LentLocked}{\StronglyLocked}{\Block{\dvs}{\cOrx}}{\Type{\lent}{\C}}$, in case $\cOrx=\ConstrCall{\C}{\xs'}$, we have that 
$\typeOf{\Block{\dvs}{\cOrx}}=\Type{\mutable}{\C}$, and if $\cOrx=\x$, so $\x=\y_1$,must
be that $\T_1=\Type{\mu_1}{C}$ with $\mu_1\leq\lent$.
\qed
\end{proof}
Note that there are no constraints for $\readable$ right-values (by subtyping any well-typed right-value is also $\readable$), and no constraints on free variables for $\lent$ right-values.  
Moreover, note that a right-value with only $\leq\imm$ free variables is $\imm$ regardless of its extracted type, since we can apply  {immutability recovery}.


\subsection{Soundness of the type system for the operational semantics}

As usual soundness is proved by proving that typability is preserved by reduction, ``subject reduction'',  and  that well-typed expressions are either values of reduce, ``progress''. The subject reduction result, in our system, is particularly
relevant, since invariants on the store are expressed at the syntactic level by the modifiers assigned to the expression
and by the lent and mutable groups used in the typing judgement. 
Preserving typability of expressions means  not only enforcing the properties expressed by modifiers, but also
preserving the relationship between variables expressed by the lent and mutable groups. 

To identify  subexpressions of expressions we define {\em general contexts} $\genCtx$ by:
\begin{quote}
\begin{grammatica}
\produzione{\genCtx}{\emptyctx\mid\FieldAccess{\genCtx}{\f}\mid\MethCall{\genCtx}{\m}{\vs}\mid\MethCall{\val}{\m}{\vs\ \genCtx\ \vs'}\mid\FieldAssign{\genCtx}{\f}{\val}
\mid\FieldAssign{\val}{\f}{\genCtx}}{}\\*
\seguitoproduzione{\mid \ConstrCall{\C}{\vs\ \genCtx\ \vs'}\mid\Block{\decs\ \Dec{\T}{\x}{\genCtx}\ \decs'}{\val}\mid\Block{\decs}{\genCtx}}{}
\end{grammatica}
\end{quote}

Given a general context $\genCtx$, $\TypeEnv{\genCtx}$ is defined by: 
\begin{itemize}
 \item $\TypeEnv{\emptyctx}=\emptyset$
 \item $\TypeEnv{\FieldAccess{\genCtx}{\f}}=\TypeEnv{\MethCall{\genCtx}{\m}{\vs}}=\TypeEnv{\MethCall{\val}{\m}{\vs\ \genCtx\ \vs'}}=\TypeEnv{\FieldAssign{\genCtx}{\f}{\val}}=\TypeEnv{\FieldAssign{\val}{\f}{\genCtx}}=\TypeEnv{\ConstrCall{\C}{\vs\ \genCtx\ \vs'}}=\TypeEnv{\genCtx}$
  \item $\TypeEnv{\Block{\decs\ \Dec{\T}{\x}{\genCtx}\ \decs'}{\val}}=\TypeEnv{\decs},\TypeEnv{\decs'},\TypeDec{\T}{\x} [\TypeEnv{\genCtx}]$
  \item $\TypeEnv{\Block{\decs}{\genCtx}}=\TypeEnv{\decs}[\TypeEnv{\genCtx}]$
 \end{itemize}
$\TypeEnv{\ctx}$ is defined similarly.  We also use $\LentVars{\genCtx}$ and $\LentVars{\ctx}$ with the obvious meaning.

The following lemma expresses the property that  subexpressions of well-typed expressions are themselves
well-typed in a type context that may contain more variables, introduced by inner blocks.
The equivalence relation on the variable of the expressions
induced by the partition determined by the lent and mutable groups is preserved.
However, more  variables may be added to existing groups, there
could be new groups, and due to rule \rn{t-swap}, 
the mutable group of the  subexpression may be different from the
one on the whole expression.
\begin{lemma}[Context]\label{lemma:subCtx}
If $\deriv:\AuxTypeCheck{\Gamma}{\LentLocked}{\MutGroup}{\StronglyLocked}{{\e}}{\T}$, $\e=\GenCtx{\e'}$ and
$\deriv':\AuxTypeCheck{\SubstFun{\Gamma}{\TypeEnv{\genCtx}}}{\LentLocked'}{\MutGroup'}{\StronglyLocked'}{{\e'}}{\T'}$ is a sub-derivation of $\deriv$, 
then
\begin{enumerate}  
\item 
\begin{itemize}
\item $\LessEq{(\ \LentLocked\ \xs\ ){\setminus}\dom{\TypeEnv{\genCtx}}}{\LentLocked'\ \xs'}$, 
\item $\dom{\TypeEnv{\genCtx}}=\emptyset$ implies 
$\LentLocked\ \xs=\LentLocked'\ \xs'$, and
\item
we may assume that the last rule applied in $\deriv'$ is a structural rule;
\end{itemize}
  \item if $\e''$ is such that $\deriv'':\TypeCheck{\SubstFun{\Gamma}{\TypeEnv{\genCtx}}}{\LentLocked'}{\StronglyLocked'}{{\e''}}{\T'}$, then substituting $\deriv''$ with $\deriv'$ in $\deriv$ we get a derivation for $\AuxTypeCheck{\Gamma}{\LentLocked}{\MutGroup}{\StronglyLocked}{\Ctx{\e'}}{\T}$.
\end{enumerate}
\end{lemma}
\begin{proof}
The proof is in~\ref{sect:proof-sound}.\qed
\end{proof}
Note that, if $\genCtx$ is $\emptyctx$ or $\FieldAccess{\emptyctx}{\f}$ or $\MethCall{\emptyctx}{\m}{\vs}$ or 
$\MethCall{\val}{\m}{\vs\ \emptyctx\ \vs'}$ or $\FieldAssign{\emptyctx}{\f}{\val}$ or 
$\FieldAssign{\val}{\f}{\emptyctx}$ or $\ConstrCall{\C}{\vs\ \emptyctx\ \vs'}$, then  $ \dom{\genCtx}=\emptyset$ and 
\refToLemma{subCtx} implies \refToLemma{nonStructural}. 

\begin{lemma}[Field access]\label{lemma:fieldAccess}
Let  $\fields{\C}=\Field{\Type{\mu_i}{\C_i}}{\f_1}\ldots\Field{\Type{\mu_n}{\C_n}}{\f_n}$.
\begin{enumerate} 
  \item If $\AuxTypeCheck{\Gamma}{\LentLocked}{\MutGroup}{\StronglyLocked}{\FieldAccess{\ConstrCall{\C}{\xs}}{\f_i}}{\Type{\mu}{\C_i}}$ where the last rule applied is \rn{t-field-access}, then $\TypeCheck{\Gamma}{\LentLocked}{\StronglyLocked}{\x_i}{\Type{\mu}{\C_i}}$.
  \item If $Aux\TypeCheck{\Gamma}{\LentLocked}{\MutGroup}{\StronglyLocked}{\FieldAccess{\stVal}{\f_i}}{\T}$ where the last rule applied in $\deriv$ is \rn{t-field-access},
%and $\deriv':\TypeCheck{\Gamma}{\LentLocked}{\StronglyLocked}{{\stVal}}{\Type{\mu}{\C}}$ is a sub-derivation of $\deriv$, then
$\AuxTypeCheck{\Gamma}{\LentLocked}{\MutGroup}{\StronglyLocked}{\fieldOf{\stVal}{i}}{\T}$.
\item If $\AuxTypeCheck{\Gamma}{\LentLocked}{\MutGroup}{\StronglyLocked}{\FieldAccess{\stVal}{\f_i}}{\Type{\imm}{\C_i}}$, then for all $\LentLocked'$, $\StronglyLocked'$ such that $\WellFormedTypeCtx{\Gamma;\LentLocked';\StronglyLocked'}$ we have that
$\AuxTypeCheck{\Gamma}{\LentLocked'}{\MutGroup'}{\StronglyLocked'}{\fieldOf{\stVal}{i}}{\Type{\imm}{\C_i}}$.
\end{enumerate}
\end{lemma}
\begin{proof}
The proof is in~\ref{sect:proof-sound}.\qed
\end{proof}
    \begin{lemma}[Field assign]\label{lemma:fieldAssign}
Let $\deriv:\AuxTypeCheck{\Gamma}{\LentLocked}{\MutGroup}{\StronglyLocked}{\FieldAssign{\x}{\f}{\valPrime}}{\T}$, and the
last rule applied in $\deriv$ be \rn{t-field-assign}. 
Then, $\x\in\xs$ and for all 
$\y\in\FV{\valPrime}$ such that $\TypeCheck{{\Gamma}}{\LentLocked}{\StronglyLocked}{\y}{\Type{\mu}{\_}}$ with
$\mu\geq\mutable$ we have that $\y\in\xs$.
\end{lemma}
\begin{proof}
The proof is in~\ref{sect:proof-sound}.\qed
\end{proof}

The following theorem asserts that reduction preserves typability of expressions. The theorem is proved
by cases on the reduction rule used.
Here we give the proof for the two difficult cases, which are the application of \rn{Field-Access}, and \rn{Field-Assign-Move}.
In the appendix we present the proof for other interesting cases, that use similar techinques.
The difficulty with the proof of subject reduction, for \rn{Field-Access}, is the non standard semantics of  the 
construct, that replaces the ``field access expression'' with the ``value of the
field''. The ``value of the field'' must be given the same type it had in the context of the declaration of the variable.
For the rule \rn{Field-Assign-Move}, the difficult cases are when declarations are moved outside a block to which
the recovery rules are applied. As highlighted in the proof, in these cases, the derivation contained application 
of the rule \rn{t-swap} (if it was a $\capsule$) and \rn{t-unrst} (if it was $\imm$), which are still applicable after
moving the declarations.

\begin{theorem}[Subject Reduction]\label{theo:subjectReductionShort}
Let $\TypeCheckGround{\e}{\T}$, and
$\reduce{{\e}}{{\e'}}$. Then $\TypeCheckGround{\e'}{\T}$.
\end{theorem}
\begin{proof} 
Let $\TypeCheckGround{\e}{\T}$. If $\reduce{{\e}}{{\e'}}$, then one of the
rules of \refToFigure{reduction} was applied. 
Here we prove the result for the most interesting rules: \rn{field-access}, and \rn{field-assign-move}, whose proofs embodies
the techniques used. In the \refToSection{proof-sound} we present two other interesting cases  \rn{field-assign}, and \rn{mut-move}, which are easier, but still interesting.

\medskip\noindent
Consider \underline{rule \rn{field-access}}.
In this case  
\begin{enumerate} [(1)]
\item $\e=\Ctx{\FieldAccess{{\val}}{\f}}$, and 
\item $\e'={\Ctx{\extractField{\ctx}{\val}{i}}}$, 
\end{enumerate}
where 
$\fields{\C}=\Field{\Type{\mu_i}{\C_i}}{\f_1}\ldots\Field{\Type{\mu_n}{\C_n}}{\f_n}\ \mbox{and}\  \f=\f_i$.\\
From (1) and  \refToLemma{subCtx}.1 for some $\T'$, $\LentLocked$ and $\StronglyLocked$ we have that
\begin{enumerate}[(1)]\addtocounter{enumi}{2}
  \item $\AuxTypeCheck{{\TypeEnv{\ctx}}}{\LentLocked}{\MutGroup}{\StronglyLocked}{\FieldAccess{{\val}}{\f}}{\T'}$ and
  \item the last rule appled in the derivation is \rn{Field-Access}.
\end{enumerate} 
From Proposition \ref{lemma:congruenceValue} either $\val\cong\z$ for some $\z$ or  $\val\cong\stVal$ for some 
\storableVal\  $\stVal$ such that $\WFrv{\stVal}$. \\
If \underline{$\val\cong\stVal$}, from (3), (4), \refToLemma{fieldAccess}.2 and  $\extractField{\ctx}{\val}{i}=\fieldOf{\stVal}{i}$ we have that 
$\TypeCheck{{\TypeEnv{\ctx}}}{\LentLocked}{\StronglyLocked}{\fieldOf{\stVal}{i}}{\T'}$. Therefore,
from \refToLemma{subCtx}.2 we derive $\TypeCheckGround{\e'}{\T}$.\\
If \underline{$\val\cong\x$}, since $\extractField{\ctx}{\x}{i}$ is defined $\e=\CtxP{\Block{\decs'}{\val'}}$ where 
\begin{enumerate}[(a)]
  \item $\decs'=\dvs\ \Dec{\Type{\mu_x}{\C}}{\x}{\stVal_x}\ \Dec{\T_z}{\z}{\ctx_z[\FieldAccess{{\x}}{\f}]}\ \decs$
  \item $\WFdv{\Dec{\Type{\mu_x}{\C}}{\x}{\stVal_x}}$
  \item $\noCapture{\x}{\HB{\ctx_z}}$ and we may assume that $\noCapture{\stVal_x}{\HB{\ctx_z}}$.
\end{enumerate}
Let $\Gamma'=\TypeEnv{\ctxP}[\TypeEnv{\decs'}]$, from Lemmas \ref{lemma:subCtx} and \ref{lemma:inversionBlock} for some $\LentLocked'$ and $\StronglyLocked'$ 
\begin{enumerate} [(a)]\addtocounter{enumi}{3}
\item $\AuxDecsOK{\Gamma'}{\LentLocked'}{\MutGroup'}{\StronglyLocked'}{\Dec{\Type{\mu_x}{\C}}{\x}{\stVal_x}}$, i.e., 
\begin{enumerate}[i]
\item $\AuxTypeCheck{\Gamma'}{\LentLocked'}{\MutGroup'}{\StronglyLocked'}{{\stVal_x}}{{\Gamma'}({x})}$ if $\mu_x\not=\lent$
\item $\AuxTypeCheck{\Gamma'}{\LentLocked_x}{\MutGroup_x}{\StronglyLocked'}{\stVal_x}{{\Gamma'}({x})}$ if $\mu_x=\lent$ and $\LentLocked'\ \xs'=\LentLocked_x\ \xs_x$ where $\x\in\xs_x$
\end{enumerate}
\item $\LessEq{(\LentLocked'\ \xs'){\setminus}\dom{\TypeEnv{\ctx_z}}}{\LentLocked\ \xs}$.
\end{enumerate}
From (3), (4) and rule \rn{Field-Access} we derive that 
\begin{enumerate} [(a)]\addtocounter{enumi}{5}
 \item $\AuxTypeCheck{{\TypeEnv{\ctx}}}{\LentLocked}{\MutGroup}{\StronglyLocked}{\x}{\Type{\mu'}{\C}}$ and
  \item $\T'=\Type{\mu}{\C_i}$ where: if either $\mu_i=\imm$ or $\mu'=\imm$, then $\mu=\imm$, else $\mu=\mu'$.
\end{enumerate}
Consider the case \underline{$\T'=\Type{\imm}{\C_i}$}: either $\mu'=\imm$ or $\mu_i=\imm$.\\
If \underline{$\mu'=\imm$}, $\TypeCheck{{\TypeEnv{\ctx}}}{\LentLocked}{\StronglyLocked}{\x}{\Type{\imm}{\C}}$. From Lemmas \ref{lemma:typeVars} and \ref{lemma:weakening} we have that also $\TypeCheck{{\Gamma'}}{\LentLocked'}{\StronglyLocked'}{\x}{\Type{\imm}{\C}}$, therefore from (d).i we derive that
$\AuxTypeCheck{\Gamma'}{\LentLocked'}{\MutGroup'}{\StronglyLocked'}{{\stVal_x}}{\Type{\imm}{\C}}$. 
Applying rule \rn{Field-Access},
$\TypeCheck{\Gamma'}{\LentLocked'}{\StronglyLocked'}{\FieldAccess{\stVal_x}{\f_i}}{\Type{\imm}{\C_i}}$.
From (c), (e), and \refToLemma{weakening} we have that also 
$\TypeCheck{\TypeEnv{\ctx}}{\LentLocked}{\StronglyLocked}{\FieldAccess{\stVal_x}{\f_i}}{\Type{\imm}{\C_i}}$.
From \refToLemma{fieldAccess}.2, we get that
$\TypeCheck{\TypeEnv{\ctx}}{\LentLocked}{\StronglyLocked}{\fieldOf{\stVal_x}{i}}{\Type{\imm}{\C_i}}$. Since 
$\extractField{\ctx}{\x}{i}=\fieldOf{\stVal_x}{i}$, from \refToLemma{subCtx}.2 we derive $\TypeCheckGround{\e'}{\T}$.\\
If \underline{$\mu_i=\imm$}, from (d).i (or (d).ii) and rule \rn{Field-Access} we derive
$\AuxTypeCheck{\Gamma'}{\LentLocked'}{\MutGroup'}{\StronglyLocked'}{\FieldAccess{\stVal_x}{\f_i}}{\Type{\imm}{\C_i}}$
(or $\AuxTypeCheck{\Gamma'}{\LentLocked_x}{\MutGroup_x}{\StronglyLocked'}{\FieldAccess{\stVal_x}{\f_i}}{\Type{\imm}{\C_i}}$). From 
(e), \refToLemma{fieldAccess}.3 and \refToLemma{weakening} we get $\TypeCheck{{\TypeEnv{\ctx}}}{\LentLocked}{\StronglyLocked}{\fieldOf{\stVal_x}{i}}{\Type{\imm}{\C_i}}$, which implies, as for the previous case, $\TypeCheckGround{\e'}{\T}$.\\
Consider the case \underline{$\T'=\Type{\mu'}{\C_i}$} where $\mu'\neq\imm$ and $\mu_i\neq\imm$, and, since we
do not allow forward references to unevaluated declarations, also $\mu'\neq\capsule$. Therefore
$\mu'\geq\mutable$ and $\mu_i=\mutable$. From $\mu'\geq\mutable$ and (b) the declaration of $\x$ is of the shape
\begin{enumerate} [(a)]\addtocounter{enumi}{7}
  \item $\Dec{\Type{\mu_x}{C}}{\x}{\ConstrCall{\C}{\xs}}$ where $\mu_x\geq\mutable$ and $\extractField{\ctx}{\x}{i}=\x_i$.
\end{enumerate}
If $\mu_x=\mutable$, then $\Gamma'(\x)=\Type{\mutable}{\C}$. From (d).i we have
$\AuxTypeCheck{\Gamma'}{\LentLocked'}{\MutGroup'}{\StronglyLocked'}{\ConstrCall{\C}{\xs}}{\Type{\mutable}{\C}}$
So $\x\in\MutGroup'$ and from \refToLemma{constrMut} also $\x_i\in\MutGroup'$.\\
If $\mu_x=\lent$, then $\Gamma'(\x)=\Type{\mutable}{\C}$. From (d).ii we have that $\AuxTypeCheck{\Gamma'}{\LentLocked_x}{\MutGroup_x}{\StronglyLocked'}{\ConstrCall{\C}{\xs}}{\Type{\mutable}{\C}}$
where $\x\in\MutGroup_x$. Again from \refToLemma{constrMut} also $\x_i\in\MutGroup_x$.\\
From (e) and (c) we have that $\x$ and $\x_i$ are in the same group in $\LentLocked\ \xs$. Therefore,
$\AuxTypeCheck{{\TypeEnv{\ctx}}}{\LentLocked}{\MutGroup}{\StronglyLocked}{{\x_i}}{\T'}$, and 
from \refToLemma{subCtx}.2 we derive $\TypeCheckGround{\e'}{\T}$.\\
Finally, if $\mu_x=\readable$, then $\Gamma'(\x)=\Type{\readable}{\C}$, and from (d).i and rule \rn{Field-Access}
we derive
$\AuxTypeCheck{\Gamma'}{\LentLocked'}{\MutGroup'}{\StronglyLocked'}{\FieldAccess{\stVal_x}{\f_i}}{\Type{\readable}{\C_i}}$. From $\AuxTypeCheck{\Gamma'}{\LentLocked'}{\MutGroup'}{\StronglyLocked'}{\x_i}{\Type{\readable}{\C_i}}$. and \refToLemma{weakening} we get $\TypeCheck{{\TypeEnv{\ctx}}}{\LentLocked}{\StronglyLocked}{\x_i}{\Type{\readable}{\C_i}}$, which implies, as for the previous case, $\TypeCheckGround{\e'}{\T}$.

\medskip\noindent
Consider \underline{rule \rn{field-assign-move}}.
In this case  
\begin{enumerate}[(1)]
\item $\e=\Ctx{\Block {\dvs'\ \Dec{\Type{\mu}{\C}}{\z'} {\e_1}\ \decs' }{\val'}}$, and 
\item $\e'=\Ctx{\Block{\dvs'\ \dvs\ \Dec{\Type{\mu}{\C}}{\z'} {\e_2}\  \decs'}{\val'}}$, 
\end{enumerate}
where
\begin{itemize}
\item $\e_1={\Block{\dvs\ \Dec{\Type{\mu_z}{\C_z}}{\z}{\CtxP{\FieldAssign{\x}{\f}{\valPrime}}}\ \decs}{\val}}$, 
\item $\e_2={\Block{\Dec{\Type{\mu_z}{\C_z}}{\z}{\CtxP{\FieldAssign{\x}{\f}{\valPrime}}}\ \decs}{\val}}$,
\end{itemize}
$\fields{\C}=\Field{\T_1}{\f_1}\ldots\Field{\T_n}{\f_n}$ with $\f=\f_i$ and
\begin{enumerate}[(1)]\addtocounter{enumi}{2}
\item $\FV{\valPrime}\cap\dom{\dvs}=\zs\neq\emptyset$,
\item $\noCapture{\x}{\HB{\ctxP}\cup\dom{\dvs}}$, $\noCapture{\valPrime}{\HB{\ctxP}}$,
$\noCapture{\Block{\dvs'\ \dvs}{\val'}}{\dom{\dvs}}$,   
\item $\z'\not\in\dom{\dvs}$ and $\Reduct{(\dvs\ \decs)}{{\zs}}=\dvs$.
\end{enumerate}
Moreover, since forward definitions are only allowed to evaluated declarations, we have that 
\begin{enumerate}[(1)]\addtocounter{enumi}{5}
\item $\z\not\in\FV{\dvs}$.
\end{enumerate}
From $\TypeCheckGround{\e}{\T}$ and \refToLemma{subCtx}.1 we get that, for some $\T_b$, $\LentLocked''$, and $\StronglyLocked''$
\begin{itemize}
  \item [$(\ast)$]$
\AuxTypeCheck{{\TypeEnv{\ctx}}}{\LentLocked''}{\MutGroup''}{\StronglyLocked''}{\Block {\dvs'\ \Dec{\Type{\mu_z}{\C}}{\z'} {\e_1}\ \decs' }{\val'}}{\T_b}.
$
\end{itemize}
Let $\Gamma=\TypeEnv{\dvs'},\TypeEnv{\decs'},\TypeDec{\T'}{z'}$ 
From $(\ast)$ and  \refToLemma{inversionBlock} for some $\T'$, $\mu'$, $\LentLocked'$ and $\StronglyLocked'$
\begin{enumerate} [(A)]
\item $\AuxDecsOK{{\SubstFun{\TypeEnv{\ctx}}{\Gamma}}}{\LentLocked'}{\MutGroup'}{\StronglyLocked'}{\dvs'\ \decs'}$,
\item $\AuxDecsOK{{\SubstFun{\TypeEnv{\ctx}}{\Gamma}}}{\LentLocked'}{\MutGroup'}{\StronglyLocked'}{\Dec{\Type{\mu}{\C}}{\z'} {e_1}}$, i.e.
\begin{enumerate}[i]
\item $\AuxTypeCheck{{\SubstFun{\TypeEnv{\ctx}}{\Gamma}}}{\LentLocked'}{\MutGroup'}{\StronglyLocked'}{e_1}{{\SubstFun{\TypeEnv{\ctx}}{\Gamma}}(\z')}$ if $\mu\neq\lent$
\item $\AuxTypeCheck{{\SubstFun{\TypeEnv{\ctx}}{\Gamma}}}{\LentLocked_{\z'}}{\MutGroup_{\z'}}{\StronglyLocked'}{\e_1}{{\SubstFun{\TypeEnv{\ctx}}{\Gamma}}(\z')}$ if $\mu=\lent$ and $\LentLocked'\ \xs'=\LentLocked_{\z'}\ \xs_{\z'}$ where $\z'\in\xs_{\z'}$
\end{enumerate}
\item $\TypeCheck{\SubstFun{\TypeEnv{\ctx}}{\Gamma}}{\LentLocked'}{\StronglyLocked'}{\val'}{\T'}$ and
\item $\LessEq{(\LentLocked''\ \xs''){\setminus}\dom{\Gamma}}{\LentLocked'\ \xs'}$.
\end{enumerate}
If \underline{$\mu\geq\mutable$} , we can give a proof similar to the 
case of {rule \rn{mut-move}}. Therefore we can assume that $\mu=\capsule$
or $\mu=\imm$. \\
In both cases $\mu\neq\lent$, so from (B).i we have that
\begin{itemize}
\item [(B1)] $\AuxTypeCheck{{\SubstFun{\TypeEnv{\ctx}}{\Gamma}}}{\LentLocked'}{\MutGroup'}{\StronglyLocked'}{e_1}{\Type{\mu}{\C}}$
\end{itemize}
Consider first \underline{$\mu=\capsule$}. \\
In this case, from \refToLemma{typeStruct}.2 we can assume that the last rule applied in the derivation of $\e_1$ is
\rn{t-capsule}:
\[
\prooftree
\AuxTypeCheck{{\SubstFun{\TypeEnv{\ctx}}{\Gamma}}}{\LentLocked'\ \xs'}{\emptyset}{\StronglyLocked'}{e_1}{\Type{\mutable}{\C}}
\justifies
\AuxTypeCheck{{\SubstFun{\TypeEnv{\ctx}}{\Gamma}}}{\LentLocked'}{\MutGroup'}{\StronglyLocked'}{e_1}{\Type{\capsule}{\C}}
\using
\rn{t-capsule}
\endprooftree
\]
Let $\Gamma'=\TypeEnv{\dvs},\TypeEnv{\decs},\TypeDec{\T}{\z}$. From \refToLemma{inversionBlock}, for some $\LentLocked$ and $\StronglyLocked$
we have that
\begin{enumerate} [(a)]
\item $\AuxDecsOK{{\SubstFun{\TypeEnv{\ctx}}{\SubstFun{\Gamma}{\Gamma'}}}}{\LentLocked}{\MutGroup}{\StronglyLocked}{\dvs}$,
\item $\AuxDecsOK{{\SubstFun{\TypeEnv{\ctx}}{\SubstFun{\Gamma}{\Gamma'}}}}{\LentLocked}{\MutGroup}{\StronglyLocked}{\decs}$,
\item $\AuxDecsOK{{\SubstFun{\TypeEnv{\ctx}}{\SubstFun{\Gamma}{\Gamma'}}}}{\LentLocked}{\MutGroup}{\StronglyLocked}{\Dec{\T_z}{\z}{\CtxP{\FieldAssign{\x}{\f}{\valPrime}}}}$, i.e.
\begin{enumerate}[i.]
\item $\AuxTypeCheck{{\SubstFun{\TypeEnv{\ctx}}{\SubstFun{\Gamma}{\Gamma'}}}}{\LentLocked}{\MutGroup}{\StronglyLocked}{\CtxP{\FieldAssign{\x}{\f}{\valPrime}}}{{\SubstFun{\TypeEnv{\ctx}}{\Gamma}}(\z)}$ if $\mu_z\neq\lent$
\item $\AuxTypeCheck{{\SubstFun{\TypeEnv{\ctx}}{\SubstFun{\Gamma}{\Gamma'}}}}{\LentLocked_{\z}}{\MutGroup_{\z}}{\StronglyLocked'}{\CtxP{\FieldAssign{\x}{\f}{\valPrime}}}{{\SubstFun{\TypeEnv{\ctx}}{\Gamma}}(\z)}$ if $\mu_z=\lent$ and $\LentLocked\ \xs=\LentLocked_{\z}\ \xs_{\z}$ where $\z'\in\xs_{\z'}$
\end{enumerate}
\item $\AuxTypeCheck{{\SubstFun{\TypeEnv{\ctx}}{\SubstFun{\Gamma}{\Gamma'}}}}{\LentLocked}{\MutGroup}{\StronglyLocked}{\val}{\Type{\mutable}{\C}}$
\item $\LessEq{(\LentLocked'\ \xs'){\setminus}\dom{\Gamma'}}{\LentLocked\ \xs}$. 
%\item $\mu'\leq\mu$.
\end{enumerate}
From (c) and \refToLemma{subCtx}.1 we have that, for some $\LentLocked_x$ and $\StronglyLocked_x$
\begin{itemize}
\item $\AuxTypeCheck{{\SubstFun{\TypeEnv{\ctx}}{\SubstFun{\Gamma}{\SubstFun{\Gamma'}{\TypeEnv{\ctxP}}}}}}{\LentLocked_x}{\MutGroup_x}{\StronglyLocked_x}{\FieldAssign{\x}{\f}{\valPrime}}{\T_i}$
\end{itemize}
From \refToLemma{fieldAssign} we derive that $\x\in\xs_x$, and, 
if $Y=\{\y\ |\ \dvs(\y)=\Dec{\Type{\mu_y}{\_}}{\y}{\_}\ \mu_y\geq\mutable\}$, then
for all $\y\in Y$ we have that $\y\in\xs_x$.\\
From (3), (4),  $\noCapture{\x}{\dom{\decs}}$ (forward references are only allowed to
evaluated declarations), (e) and (D) we have that $\{\x\}\cup Y$ is a subset of one of the groups 
in $\LentLocked\ \xs$. 
(Note that, in the derivation of the judgement (c).i or (c).ii, there must be an application of rule \rn{t-swap} to
make $\xs_x$ the mutable group, since $\x$ is in one of the groups of $\LentLocked\ \xs$.)
Define $\LentLocked^{\ast}$ and $\StronglyLocked^{\ast}$ as follows. If there is $\zs'\in\LentLocked'$ such that
$\x\in\zs'$, then $\LentLocked^{\ast}=(\LentLocked'-\zs')\ (\zs'\cup Y)$, otherwise  
$\LentLocked^{\ast}=\LentLocked'$. If $\x\in\StronglyLocked'$ then $\StronglyLocked^{\ast}=\StronglyLocked'\cup Y$,
otherwise $\StronglyLocked^{\ast}=\StronglyLocked'$.\\
From (4), (A), (B), (C) and \refToLemma{weakening}, we derive
\begin{itemize}
\item [(A1)] $\AuxDecsOK{{\SubstFun{\TypeEnv{\ctx}}{\Gamma,\TypeEnv{\dvs}}}}{\LentLocked^{\ast}}{\MutGroup^{\ast}}{\StronglyLocked^{\ast}}{\dvs'\ \decs'}$,
\item [(C1)] $\AuxTypeCheck{\SubstFun{\TypeEnv{\ctx}}{\Gamma,\TypeEnv{\dvs}}}{\LentLocked^{\ast}}{\MutGroup^{\ast}}{\StronglyLocked^{\ast}}{\val'}{\T'}$
\item [(D1)]  $\LessEq{(\LentLocked'\ \xs'){\setminus}{\dom{\dvs}}}{\LentLocked^{\ast}\ \xs^{\ast}}$ and  $\StronglyLocked'{\setminus}{\dom{\dvs}}=\StronglyLocked^{\ast}{\setminus}{\dom{\dvs}}$.
\end{itemize}
From (a), (4), and \refToLemma{weakening}, we have
\begin{itemize}
\item [(a1)] $\AuxDecsOK{{\SubstFun{\TypeEnv{\ctx}}{{\Gamma,\TypeEnv{\dvs}}}}}{\LentLocked^{\ast}}{\MutGroup^{\ast}}{\StronglyLocked^{\ast}}{\dvs}$
\end{itemize}
From (4) and the fact that, for well-formedness of
declarations $x\not\in\dom{\dvs}$, we derive
that $\SubstFun{\Gamma}{\TypeEnv{\dvs},\TypeEnv{\decs},\TypeDec{\T'}{\z'}}=\SubstFun{\Gamma,\TypeEnv{\dvs}}{\TypeEnv{\decs},\TypeDec{\T'}{\z'}}$.\\
From (b), (c), (d), (e) and  rule \rn{t-block},  we have that 
\begin{itemize}
\item$\AuxTypeCheck{{\SubstFun{\TypeEnv{\ctx}}{\Gamma,\TypeEnv{\dvs}}}}{\LentLocked'\ \xs'}{\emptyset}{\StronglyLocked'}{\Block{\Dec{\Type{\mu_z}{\C_z}}{\z}{\CtxP{\FieldAssign{\x}{\f}{\valPrime}}}\ \decs}{\val}}
{\Type{\mutable}{\C}}
$
\end{itemize}
and therefore applying rule \rn{t-capsule} we derive $\AuxDecsOK{{\SubstFun{\TypeEnv{\ctx}}{\Gamma,\TypeEnv{\dvs}}}}{\LentLocked'}{\MutGroup'}{\StronglyLocked'}{\e_2}
$. 
From (D1) and \refToLemma{weakening} we also have 
\begin{itemize}
\item [(B2)] $\AuxDecsOK{{\SubstFun{\TypeEnv{\ctx}}{\Gamma,\TypeEnv{\dvs}}}}{\LentLocked^{\ast}}{\MutGroup^{\ast}}{\StronglyLocked^{\ast}}{\Dec{\Type{\mu_z}{\C}}{\z'} {\e_2}}
$. 
\end{itemize} 
From (A1), (a1), (B2), (C1),  (D), and (D1), applying rule \rn{t-block}, we derive
\begin{center}
$\AuxTypeCheck{{\TypeEnv{\ctx}}}{\LentLocked''}{\MutGroup''}{\StronglyLocked''}{\Block{\dvs'\ \dvs\ \Dec{\Type{\mu}{\C}}{\z'}{\e_2}\ \decs'}{\val'}}{\T_b}$.
\end{center}
From \refToLemma{subCtx}.2, we obtain the result.\\
Consider now \underline{$\mu=\imm$}.\\
If the typing is obtained from \rn{t-capsule} followed by \rn{t-sub} the result follows from the
previous proof. If instead the last rule applied was \rn{t-imm}, from \refToLemma{typeStruct}.2 
\[
\prooftree
\AuxTypeCheck{{\SubstFun{\TypeEnv{\ctx}}{\Gamma}}}{\LentLocked'\ \xs'}{\emptyset}{\domGeqMut{\SubstFun{\TypeEnv{\ctx}}{\Gamma}}}{e_1}{\Type{\readable}{\C}}
\justifies
\AuxTypeCheck{{\SubstFun{\TypeEnv{\ctx}}{\Gamma}}}{\LentLocked'}{\MutGroup'}{\StronglyLocked'}{e_1}{\Type{\imm}{\C}}
\using
\rn{t-imm}
\endprooftree
\]
Let
$\Gamma'=\TypeEnv{\dvs},\TypeEnv{\decs},\TypeDec{\T'}{\z'}$.  From \refToLemma{inversionBlock}, and the fact that
there is no application of \rn{t-unrst} in the derivation of $\e_1$ (if there was, then  the type derived for $\e_1$ should be $\Type{\imm}{\C}$), for some $\LentLocked$, and $\ys$ 
\begin{enumerate} [(a)]
\item $\AuxDecsOK{{\SubstFun{\TypeEnv{\ctx}}{\SubstFun{\Gamma}{\Gamma'}}}}{\LentLocked}{\MutGroup}{\StronglyLocked}{\dvs}$,
\item $\AuxDecsOK{{\SubstFun{\TypeEnv{\ctx}}{\SubstFun{\Gamma}{\Gamma'}}}}{\LentLocked}{\MutGroup}{\StronglyLocked}{\decs}$,
\item $\AuxDecsOK{{\SubstFun{\TypeEnv{\ctx}}{\SubstFun{\Gamma}{\Gamma'}}}}{\LentLocked}{\MutGroup}{\StronglyLocked}{\Dec{\T_z}{\z}{\CtxP{\FieldAssign{\x}{\f}{\valPrime}}}}$, i.e.
\begin{enumerate}[i.]
\item $\AuxTypeCheck{{\SubstFun{\TypeEnv{\ctx}}{\SubstFun{\Gamma}{\Gamma'}}}}{\LentLocked}{\MutGroup}{\StronglyLocked}{\CtxP{\FieldAssign{\x}{\f}{\valPrime}}}{{\SubstFun{\TypeEnv{\ctx}}{\Gamma}}(\z)}$ if $\mu_z\neq\lent$
\item $\AuxTypeCheck{{\SubstFun{\TypeEnv{\ctx}}{\SubstFun{\Gamma}{\Gamma'}}}}{\LentLocked_{\z}}{\MutGroup_{\z}}{\StronglyLocked}{\CtxP{\FieldAssign{\x}{\f}{\valPrime}}}{{\SubstFun{\TypeEnv{\ctx}}{\Gamma}}(\z)}$ if $\mu_z=\lent$ and $\LentLocked\ \xs=\LentLocked_{\z}\ \xs_{\z}$ where $\z'\in\xs_{\z'}$
\end{enumerate}
\item $\AuxTypeCheck{{\SubstFun{\TypeEnv{\ctx}}{\SubstFun{\Gamma}{\Gamma'}}}}{\LentLocked}{\MutGroup}{\StronglyLocked}{\val}{\Type{\readable}{\C}}$
\item $\LessEq{(\LentLocked'\ \xs'){\setminus}\dom{\Gamma'}}{\LentLocked\ \xs}$
\item $\StronglyLocked=\domGeqMut{\SubstFun{\TypeEnv{\ctx}}{\Gamma}}{\setminus}\dom{\Gamma'}$
%\item $\mu'\leq\mu$.
\end{enumerate}
As before, we can assume that  $\noCapture{\x}{\dom{\decs}}$. From (c), (4), \refToLemma{subCtx}.1, and rule \rn{t-field-assign}, we derive that $\x\in\domMut{\SubstFun{\TypeEnv{\ctx}}{\Gamma}}$. So $\x\in\domGeqMut{\SubstFun{\TypeEnv{\ctx}}{\Gamma}}$.
Therefore, in the derivation of the judgement (c).i or (c).ii, there must be an application of rule \rn{t-unrst} to make possible the
application of rule \rn{t-var} to $\x$.\\
The proof of this case now proceeds as for the case of $\mu=\capsule$ defining $\LentLocked^{\ast}$ and $\StronglyLocked^{\ast}$ and proving that $\e_2$ and then the resulting expression are typeable.
\qed
\end{proof}

To prove progress, given an expression we need to find a rule that may be applied, and prove that 
its side conditions are verified. To match an expression with the left-side of a rule we define
 {\em pre-redexes}, $\preRedex$, by:
\begin{center}
$
\begin{array}{ll}
\preRedex ::=& \FieldAccess{\val}{\f}\mid\MethCall{{\val}}{\m}{{\vals}}\mid\FieldAssign{\val}{\f}{\val'}
  \\
 &
 \mid \Block{\dvs\ \Dec{\T}{\x}{\val}\ \decs}{\val}\quad\WFdv{\dvs}\ \wedge\ \not\WFdv{\Dec{\T}{\x}{\val}}\\
\end{array}
$
\end{center}
Any expression can be uniquely decomposed in a in a context filled a pre-redex.
\begin{lemma}[Unique Decomposition]\label{lemma:decomposition}
Let $\e$ be an expression. Either $\congruence{\e}{\val}$ where $\val$ is well-formed,
and if $\val=\Block{\dvs}{\cOrx}$, then $\WFdv{\dvs}$,
or there are $\ctx$ and $\preRedex$ such that $\congruence{\e}{\Ctx{\preRedex}}$.
\end{lemma}
\begin{proof}
The proof is in~\ref{sect:proof-sound}.\qed
\end{proof}
The progress result is proved by structural induction on expressions. The interesting case
is field assignment, in which we have to prove that one of the three rules may be applied. 
\begin{theorem}[Progress]\label{theo:progress}
Let $\TypeCheckGround{\e}{\T}$. Then either $\congruence{\e}{\val}$ where $\val$ is well-formed,
and if $\val=\Block{\dvs}{\cOrx}$, then  $\WFdv{\dvs}$, or $\reduce{\e}{\e'}$ for some $\e'$.
\end{theorem}
\begin{proof}
Let $\e$ be such that $\TypeCheckGround{\e}{\T}$
for some $\T$, and for no $\val$ we have that
$\congruence{\e}{\val}$ where $\val$ is well-formed,
and if $\val=\Block{\dvs}{\cOrx}$, then $\WFdv{\dvs}$.
By \refToLemma{decomposition}, for some $\ctx$ and $\preRedex$
we have that $\congruence{\e}{\Ctx{\preRedex}}$, and so
$\TypeCheckGround{\Ctx{\preRedex}}{\T}$.
From \refToLemma{subCtx}, 
$\TypeCheck{{\TypeEnv{\ctx}}}{\LentLocked}{\StronglyLocked}{{\preRedex}}{\T'}$,
for some $\LentLocked$, $\StronglyLocked$,
$\T'$. \\
Let \underline{$\preRedex=\FieldAccess{\val}{\f}$}. 
From $\TypeCheck{\TypeEnv{\ctx}}{\LentLocked}{\StronglyLocked}{\FieldAccess{\val}{\f}}{\T'}$, we have that 
$\TypeCheck{\TypeEnv{\ctx}}{\LentLocked}{\StronglyLocked}{{{\val}}}{\Type{\_}{\C}}$ with 
$\fields{\C}=\Field{\T_1}{\f_1}\ldots\Field{\T_n}{\f_n}$ and $\f=\f_i$.
Therefore $\typeOf{\ctx,\val}=\Type{\_}{\C}$ and $\extractField{\ctx}{\val}{i}$ 
is defined. So rule \rn{Field-Access}
is applicable.\\
Let \underline{$\preRedex=\MethCall{{\val}}{\m}{{\vals}}$}, then rule \rn{Invk}
is applicable.\\
Let \underline{$\preRedex=\FieldAssign{\val}{\f}{\val'}$}, and $\notRef{\val}$. Then rule \rn{field-assign-prop} is applicable.\\
Let \underline{$\preRedex=\FieldAssign{\x}{\f}{\val}$}. From from $\TypeCheck{{\TypeEnv{\ctx}}}{\LentLocked}{\StronglyLocked}{{\FieldAssign{\x}{\f}{\val}}}{\T'}$, we have that
$\TypeCheck{{\TypeEnv{\ctx}}}{\LentLocked}{\StronglyLocked}{\x}{\Type{\mutable}{\C}}$ with
$\fields{\C}=\Field{\T_1}{\f_1}\ldots\Field{\T_n}{\f_n}$ and $\f=\f_i$. Since $\x\in\HB{\ctx}$, 
then $\ctx=\DecEvCtx{\ctx}{\x}[\Block{\dvs\ \Dec{\T}{\y}{\ctxP}\ \decs}{\valPrime}]$ for some $\ctxP$ and $\ctx_x$ such that
$\dvs(\x)=\dv$ and $\x\not\in\HB{\ctxP}$. From $\WFdv{\dv}$, we have that $\dv=\Dec{\Type{\mutable}{\C}}{\x}{\ConstrCall{\C}{{\xs}}}$.\\
There are two cases: either $\noCapture{\val}{\HB{\ctxP}}$, or for some $\z\in\FV{\val}$, $\z\in\HB{\ctxP}$.\\
In the first case rule \rn{field-assign} is applicable. \\
In the second, let $\ctxP={\ctx_z}[\Block{\dvs'\ \Dec{\T'}{\y'}{\ctx''}\ \decs'}{\valPrime'} ]$
such that $\noCapture{\val}{\HB{\ctx''}}$, and $\dvs'(\z)=\dv'$ and $\z\not\in\HB{\ctx''}$. \\
If $\ctx_z$ is $\emptyctx$, then \rn{field-assign-move} is applicable to
\begin{center}
$\DecEvCtx{\ctx}{\x}[\Block{\dvs\ \Dec{\T}{\y}{\Block{\dvs'\ \Dec{\T'}{\y'}{\ctx''[{\FieldAssign{\x}{\f}{\val}}]}\ \decs'}{\valPrime'}}\ \decs}{\valPrime}]$.
\end{center}
Otherwise,  $\ctx_z=\ctx'_z[\Block{\dvs''\ \Dec{\T''}{\y''}{\Block{\dvs'\ \Dec{\T'}{\y'}{\ctx''}\ \decs'}{\valPrime'}}\ \decs''}{\valPrime''}]$ and  
then \rn{field-assign-move} is applicable to
\begin{center}
$\DecEvCtx{\ctx}{\x}[\ctx'_z[\Block{\dvs''\ \Dec{\T''}{\y''}{\Block{\dvs'\ \Dec{\T'}{\y'}{\ctx''[{\FieldAssign{\x}{\f}{\val}}]}\ \decs'}{\valPrime'}}\ \decs''}{\valPrime''}]]$.
\end{center}
Therefore, there is always a rule applicable to $\ctx[\FieldAssign{\x}{\f}{\val}]$. \\
Let \underline{$\preRedex=\Block{\dvs\ \Dec{\T'}{\x}{\val'}\ \decs}{\val}$}, $\WFdv{\dvs}$ and $\not\WFdv{\Dec{\T'}{\x}{\val'}}$.
From Proposition \ref{lemma:congruenceValue}, either $\congruence{\val}{\x}$ or 
$\congruence{\val}{\stVal}$ for some $\WFrv{\stVal}$. In the first case rule \rn{Alias-Elim} is applicable. In the
second, let $\T'=\Type{\mu}{\D}$ for some, $\mu$ and $\D$. By cases on $\mu$.\\
If \underline{$\mu=\capsule$}, then \rn{Capsule-Elim} is applicable. \\
If \underline{$\mu\geq\mutable$}, then since $\not\WFdv{\Dec{\T'}{\x}{\stVal}}$ we have that $\stVal=\Block{\dvs'}{\val'}$.
By renaming bound variables in $\Block{\dvs'}{\val'}$ we can have that 
$\noCapture{\Block{\dvs\ \decs}{\val}}{\dom{\dvs'}}$. Therefore, rule \rn{Mut-Move} is applicable
moving $\dvs'$ outside.\\
If \underline{$\mu=\imm$},  
let $\dvs'=\dvs_{im}\ \dvs_{mt}$, where $\dv\in\dvs_{im}$ if $\extractMod{\dv}\leq\imm$,
and $\dv\in\dvs_{mt}$ if $\extractMod{\dv}\geq\mutable$. 
The side condition, 
$\noCapture{\Block{\dvs\ \decs}{\val}}{\dom{\dvs_{im}}}$, can be satisfied by renaming
of declared variables in $\Block{\dvs'}{\val'}$. We have to show that 
$\noCapture{\dvs_{im}}{\dom{\dvs_{mt}}}$. 
Let $\y\in\FV{\stVal'}$ for some $\Dec{\Type{\mu'}{\C'}}{\x'}{\stVal'}\in\dvs_{im}$ with $\mu'\leq\imm$. 
So $\TypeCheck{\SubstFun{\Gamma}{\TypeEnv{\dvs_{im}\ \dvs_{mt}}}}{\LentLocked''}{\StronglyLocked''}{\stVal'}{\Type{\mu'}{\C'}}$,
and from \refToTheorem{canonicalForm}.1 and 3, we have that
$\TypeCheck{\SubstFun{\Gamma}{\TypeEnv{\dvs_{im}\ \dvs_{mt}}}}{\LentLocked''}{\StronglyLocked''}{\y}{\Type{\imm}{\_}}$.
Therefore, $\y\not\in\dom{\dvs_{mt}}$, and rule \rn{Imm-Move} can be applied
since $\noCapture{\dvs_{im}}{\dom{\dvs_{mt}}}$ holds. \qed
\end{proof}

\subsection{Properties of expressions having immutable and capsule modifiers}

In addition to the standard soundness property, we prove two theorems stating that the $\capsule$ and $\imm$ qualifier, respectively, have the expected behaviour.  A nice consequence of our non standard operational model 
is that this can be formally expressed {and proved} in a simple way. 

In the two theorems, we need to trace the reduction of the right-hand side of a reference declaration.  
 To lighten the notation, we assume in the following that expressions contain at most one declaration for a variable (no shadowing, as can be always obtained by alpha-conversion).


We need some notations and lemmas. We define \emph{contexts} $\decctx{\mux}$:

\begin{quote}
\begin{grammatica}
\produzione{\decctx{\mux}}{\Block{\decs\ \Dec{\Type{\mu}{\C}}{\x}{\emptyctx}\ \decs'}{\val}\mid\FieldAccess{\decctx{\mux}}{\f}\mid\MethCall{\decctx{\mux}}{\m}{\vals}\mid\MethCall{\val}{\m}{\vals\ \decctx{\mux}\ \vals'}}{}\\*
\seguitoproduzione{\mid\FieldAssign{\decctx{\mux}}{\f}{\val}
\mid\FieldAssign{\val}{\f}{\decctx{\mux}}\mid \ConstrCall{\C}{\vals\ \decctx{\mux}\ \vals'}}{}\\*
\seguitoproduzione{\mid\Block{\decs\ \Dec{\T}{\y}{\decctx{\mux}}\ \decs'}{\val}\mid\Block{\decs}{\decctx{\mux}}}{}
\end{grammatica}
\end{quote}

That is, in $\Decctx{\mux}{\e}$ the expression $\e$ occurs as right-hand side of the (unique) declaration for reference $\x$, which has qualifier $\mu$. We will simply write $\Xctx$ when the $\mux$ suffix is not relevant.  


The type assignment extracted from a context $\Xctx$, denoted $\TypeEnv{\Xctx}$, is defined as for the general contexts.
% \begin{quote}
% $\TypeEnv{\Block{\decs\ \Dec{{\T}}{\x}{\emptyctx}\ \decs'}{\val}}=
%\SubstFun{\TypeEnv{\decs\ \decs'}}{\TypeDec{\T}{\x}}$\\
%$\TypeEnv{\FieldAccess{\Xctx}{\f}}=\ldots=\TypeEnv{\ConstrCall{\C}{\vals\ \Xctx\ \vals'}}=\TypeEnv{\Xctx}$\\
% $\TypeEnv{\Block{\decs\ \Dec{\T}{\y}{\Xctx}\ \decs'}{\val}}=
%\SubstFun{ \SubstFun{\TypeEnv{\decs\ \decs'}}{\TypeDec{\T}{\y}}}{\TypeEnv{\Xctx}}$\\
%$\TypeEnv{\Block{\decs}{\Xctx}}=\SubstFun{\TypeEnv{\decs}}{\TypeEnv{\Xctx}}$
% \end{quote}
 
 The declaration for a variable $\y$ in a context $\Xctx$, denoted $\extractDec{\Xctx}{\y}$, can be defined analogously.
 
The following lemma states that the type of a reference under the type assignment extracted from the surrounding context is a subtype of the declared type  {when the type modifier is not equal to $\lent$. If the type modifier of the declaration is $\lent$, the variable
could be accessed in a sub-context in which, due to the \rn{t-swap} rule, the variable belongs to the current mutable
group and so we derive a type with the $\mutable$ modifier.}
\begin{lemma}\label{lemma:dec}
If $\TypeCheck{\TypeEnv{\Xctx}}{\LentLocked}{\StronglyLocked}{{\y}}{\Type{\mu}{\_}}$, then $\extractDec{\Xctx}{\y}=\Dec{\Type{\mu'}{\_}}{\y}{\_}$  {is such that $\mu'\neq\lent$ implies $\mu'\leq\mu$}.
\end{lemma}
\begin{proof}
By induction on $\Xctx$. \qed
\end{proof}
%Note that this lemma would trivially hold in a type system with no  {recovery}  {and swapping rules}, that is, with no way to move the type of an expression against the subtype hierarchy. 

 The following lemma states that a subexpression which occurs as right-hand-side of a declaration in a well-typed expression is well-typed, under the type assignment extracted from the surrounding context, and has a subtype of the reference type.
\begin{lemma}\label{lemma:decctx}
If $\IsWellTyped{\Decctx{\mux}{\e}}$, then $\TypeCheck{\TypeEnv{\decctx{\mux}}}{\LentLocked}{\StronglyLocked}{{\e}}{\Type{\mu'}{\_}}$ for some
$\LentLocked$, $\StronglyLocked$, and $\mu'\leq\mu$.
\end{lemma}
\begin{proof}
By induction on $\decctx{\mux}$. \qed
\end{proof}

The expected behaviour of the $\capsule$ qualifier is, informally,  that the reachable object subgraph denoted by a $\capsule$ reference should not contain nodes reachable from the outside, unless they are immutable.
In our calculus, a reachable object subgraph is a right-value $\stVal$, nodes reachable from the outside are free variables,  hence the condition can be formally expressed by requiring that free variables in $\stVal$ are declared $\imm$ or $\capsule$ in the surrounding context:
\begin{quote}
$\ImmClosed{\Xctx}{\stVal}$ iff for all $\y\in\FV{\stVal}$, $\extractDec{\Xctx}{\y}=\Dec{\Type{\mu}{\_}}{\y}{\_}$ with ${\mu}\leq\imm$
\end{quote}
Moreover, the reachable object subgraph denoted by a $\capsule$ reference should be typable $\mutable$, since it can be assigned to a mutable reference.
Altogether, the fact that the $\capsule$ qualifier guarantees the expected behaviour can be formally stated as in the theorem below, where the qualifier $\capsule$ is abbreviated \lstinline{c}.
\begin{theorem}[Capsule]\label{theo:capsule}
If $\IsWellTyped{\Capsulectx{\e}}$ and $\Capsulectx{\e}\longrightarrow^\star\CapsulectxP{\stVal}$, then:
\begin{itemize}
\item {$\typeOf{\stVal}=\Type{\mutable}{\_}$}
\item $\ImmClosed{{\cal C}'_{\cx}}{\stVal}$
\end{itemize}
\end{theorem}
\begin{proof}
By subject reduction (\refToTheorem{subjectReductionShort}) we get $\IsWellTyped{\CapsulectxP{\stVal}}$. 
Then, from \refToLemma{decctx}, $\TypeCheck{\Gamma}{\LentLocked}{\StronglyLocked}{\stVal}{\Type{\capsule}{\_}}$ with $\Gamma=\TypeEnv{{\cal C}'_{\cx}}$. Hence, from \refToTheorem{canonicalForm}, for all $\y\in\FV{\stVal}$, $\TypeCheck{\Gamma}{\LentLocked}{\StronglyLocked}{\y}{\Type{\imm}{\_}}${, and $\typeOf{\stVal}=\Type{\mutable}{\_}$. Hence,} by \refToLemma{dec},  $\extractDec{{\cal C}'_{\cx}}{\y}=\Dec{\Type{\mu}{\_}}{\y}{\_}$ with $\mu\leq\imm$.\qed
\end{proof}

Note that the context can change since it is not an evaluation context and, moreover, reduction can modify the store. Consider for instance $\Capsulectx{\e}$ to be the following expression:
\begin{lstlisting}
mut C y= new C(0); mut C z= new C(y.f=1); capsule C x= $\e$; ...
\end{lstlisting}
Before reducing to a right value the initialization expression of \lstinline{x}, the initialization expression of \lstinline{z} should be reduced, and this has a side-effect on the right value of \lstinline{y}. Hence $\CapsulectxP{\stVal}$ is:
\begin{lstlisting}
mut C y= new C(1); mut C z= new C(1); capsule C x= $\stVal$; ...
\end{lstlisting}

The expected behaviour of the $\imm$ qualifier is, informally, that the reachable object subgraph denoted by an $\imm$ reference should not be modified through any alias. 
Hence, the right value of an immutable reference:
\begin{itemize}
\item should not be modified
\item should only refer to external references which are immutable
\end{itemize}
as formally stated in the theorem below, where the qualifier $\imm$ is abbreviated \lstinline{i}.
 \begin{theorem}[Immutable]\label{theo:imm}
If $\IsWellTyped{\mbox{$\Immctx{\e}$}}$ and $\Immctx{\e}\longrightarrow^\star\ImmctxP{\stVal}$, then:
\begin{itemize}
\item $\ImmctxP{\stVal}\longrightarrow^\star\ImmctxS{\stVal'}$
implies $\stVal=\stVal'$
\item $\ImmClosed{{\cal C}'_{\ix}}{\stVal}$
\end{itemize}
\end{theorem}
\begin{proof}
The first property is directly ensured by reduction rules, since rule \rn{field-assign} is only applicable on $\geq\mutable$ references (note that the progress theorem guarantees that reduction cannot be stuck for this reason). The second property can be proved analogously to \refToTheorem{capsule} above.\qed
\end{proof}

Again, the context can change during reduction. Consider for instance $\Immctx{\e}$ to be the following expression:
\begin{lstlisting}
mut C y= new C(0); imm C x= $\e$; mut C z= new C(y.f=1); ...
\end{lstlisting}
In this case, the initialization expression of $\x$ is firstly reduced to a right value, hence $\ImmctxP{\stVal}\equiv\Immctx{\stVal}$ is:
\begin{lstlisting}
mut C y= new C(0); imm C x= $\stVal$; mut C z= new C(y.f=1); ...
\end{lstlisting}
In the following reduction steps, the context can change, for instance $\ImmctxS{\stVal}$ can be:
\begin{lstlisting}
mut C y= new C(1); imm C x= $\stVal$; mut C z= new C(1); ...
\end{lstlisting}
However, the right-value $\stVal$ cannot be modified.

