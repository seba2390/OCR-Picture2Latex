\section{Introduction}
In languages with state and  mutations,  keeping control of aliasing relations is a key issue for correctness. This is  {exacerbated} by  concurrency mechanisms, since side-effects in one thread can affect the behaviour of another thread, hence unpredicted aliasing can induce unplanned/unsafe communication.

For these reasons, the last few decades have seen considerable interest in type systems for controlling sharing and interference, to make programs easier to maintain and understand, notably using \emph{type qualifiers} to restrict the usage of references \cite{ZibinEtAl10,GordonEtAl12,NadenEtAl12,ClebschEtAl15}. 

{In particular, it is very useful for a programmer to be able to rely on the \emph{capsule} and \emph{immutability} properties of a reference $\x$. To informally explain their meaning, let us consider the store as a graph, where nodes contain records of fields, which may be
references to other nodes, and let $\x$ be a reference\footnote{{Equivalently, a variable naming a reference: in our formalism the two notions are identified.}} to a node in the graph.
Depending on the qualifier of $\x$,  restrictions are imposed {and} assumptions can be made on the whole subgraph reachable from $\x$, as detailed below. }

{If $\x$ is $\capsule$, then the subgraph reachable from 
$\x$ is an isolated portion of store, that is, all its (non immutable) nodes
can be reached only through this reference. This allows programmers (and static analysis)
to identify state that can be safely handled by a thread.  In this paper we will use the name \emph{capsule} for this property, to avoid confusion with many variants in literature \cite{ClarkeWrigstad03,Almeida97,ServettoEtAl13a,Hogg91,DietlEtAl07,GordonEtAl12}.}

{If $\x$ is $\imm$ (immutable), then the subgraph reachable from 
$\x$ is an immutable portion of store. That is, we impose the restriction that fields of the node cannot be modified\footnote{This corresponds to the \emph{read-only} notion in literature.} ($\x.\f{=}\e$ is not legal), and we can assume that the subgraph reachable from $\x$ will 
not be modified through any other reference. As a consequence, an immutable reference can be safely shared
in a multithreaded environment.}

{Note that the above properties are \emph{deep/full}, that is, related to the whole object graph.} {For this reason,} unlike early
 discussions of read-only
  \cite{Boyland06} and the Rust language\footnote{\texttt{rust-lang.org}}, we do not offer any kind of back door to support internal mutability, since this would destroy the immutability assumption on the whole reachable object graph.
  
{In addition to $\capsule$ and $\imm$, we will consider three other type qualifiers.
\begin{itemize}
\item If $\x$ is $\mutable$ (mutable), then no restrictions are imposed and no assumptions can be made on $\x$.  
\item If $\x$ is $\lent$ \cite{ServettoZucca15,GianniniEtAl16}, also called \emph{borrowed} in literature \cite{Boyland01,NadenEtAl12}, then the subgraph reachable from $\x$ can be manipulated, but not shared, by a client. More precisely, the evaluation of an expression which uses a lent reference $\x$ can neither connect $\x$ to any other external reference, nor to the result of the expression, unless this expression is, in turn, lent.
\item Finally, if $\x$ is $\readable$ (readable), then  neither modification nor sharing are permitted. That is, we impose both the \emph{read-only} and the $\lent$ {(\emph{borrowed})} restriction; however, note that there is no immutability guarantee.
\end{itemize}}
  The
last two qualifiers
ensure intermediate properties used to derive the  capsule and immutable properties. 

%{First, the type system permits to recover uniqueness and immutability properties from mutable or readable references even in the presence of other mutable or readable references. This is achieved by rules which restrict the use of such other references in the portion of code for which the property should be recovered.}

Whereas (variants of) such qualifiers have appeared in previous literature (see \refToSection{related} 
for a detailed discussion on related work), there are two key novelties in our approach. 

{Firstly and more importantly, the type system is very expressive, as will be illustrated by many examples in \refToSection{examples}. Indeed, it adopts the \emph{recovery} technique \cite{GordonEtAl12,ClebschEtAl15}, that is, references (more in general, expressions) originally typed as mutable or readable can be recovered to be capsule or immutable, respectively, provided that no other mutable/readable references are used. 
However, expressivity of recovery is greatly enhanced, that is, the type system permits recovery of $\capsule$ and $\imm$ properties in much more situations. Indeed,  other mutable/readable references are not hidden once and for all. Instead, they can be used in a controlled way, so that it can be ensured that they will be not referenced from the result of the expression for which we want to recover the capsule or immutability property. Notably, they can be possibly used in some subexpression, if certain conditions are satisfied, thanks to two rules for \emph{swapping} and \emph{unrestricting} which are the key of the type system's expressive power.}

Secondly, we adopt an innovative execution model \cite{ServettoLindsay13,CapriccioliEtAl15,ServettoZucca15} for imperative languages which, 
differently from traditional  ones, {is a \emph{reduction relation on language terms}. That is, we do not add an auxiliary structure which mimics 
physical memory, but such structure is encoded in the language itself.} Whereas this makes no difference from a programmer's point of view, it is important on the foundational side, since, as will be {informally illustrated in \refToSection{informal} and formalized in \refToSection{results}, it makes possible to express %and prove 
uniqueness and immutability properties in a simple and direct way. }

This paper is an improved and largely extended version of \cite{GianniniEtAl16}. The novel contributions include reduction rules, more examples and proofs of results.

The rest of the paper is organized as follows: we provide syntax and an informal introduction in \refToSection{informal}, type system in \refToSection{typesystem},  {programming} examples in \refToSection{examples}, reduction rules in \refToSection{calculus},  results in \refToSection{results}, 
related work in \refToSection{related}, summary of paper contribution and outline of further work in \refToSection{conclu}. 
The appendix contains some of the proofs omitted from \refToSection{results}.
