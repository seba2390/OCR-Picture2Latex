% !TEX root =main.tex

\theoremstyle{plain}

\newtheorem*{theorem*}{Theorem}
\newtheorem{theorem}{Theorem}
\newtheorem{lemma}[theorem]{{\bf Lemma}}
\newtheorem{definition}[theorem]{{\bf Definition}}
\newtheorem{proposition}[theorem]{{\bf Proposition}}
\newtheorem{example}[theorem]{{\bf Example}}
\renewenvironment{proof}[1][Proof]{\begin{trivlist}
\item[\hskip \labelsep {\bfseries #1}]}{\end{trivlist}}

%formatting code in the text
\newcommand*{\ttfamilywithbold}{\fontfamily{pcr}\selectfont}
\definecolor{darkRed}{RGB}{100,0,10}
\definecolor{darkBlue}{RGB}{10,0,100}
\lstset{language=Java,
  basicstyle=\ttfamily,%withbold,%\ttfamily,%\scriptsize,
  keywordstyle=\ttfamilywithbold\bfseries\color{darkRed},
  showstringspaces=false,
  mathescape=true,
  xleftmargin=0pt,
  xrightmargin=0pt,
  breaklines=false,
  breakatwhitespace=false,
  breakautoindent=false,
  linewidth=4\textwidth,% should be enough
%  identifierstyle=\idstyle
morekeywords={%
  mut,imm,
  read,capsule,lent
  }%
 }

%marco
\newcommand{\Q}{\lstinline}

%tipografiche
\newcommand{\refToFigure}[1]{Figure~\ref{fig:#1}}
\newcommand{\refToSection}[1]{Section~\ref{sect:#1}}
\newcommand{\refToLemma}[1]{Lemma~\ref{lemma:#1}}
\newcommand{\refToTheorem}[1]{Theorem~\ref{theo:#1}}
\newenvironment{ProofOf}[2]{\noindent{\it Proof of \refToTheorem{#1} (#2)}.}{}
\newcommand{\Space}{\hskip 0.8em}
\newcommand{\HSep}{\hbox to \textwidth{\bf\hrulefill}}
\newcommand{\qued}{\ensuremath{\hfill \square}}

%matematiche generali
\newcommand{\Tuple}[1]    {\langle{#1}\rangle}
\newcommand{\Pair}[2]     {\Tuple{{#1},{#2}}}
\newcommand{\Triple}[3]     {\Tuple{{#1},{#2},{#3}}}
\newcommand{\FourTuple}[4]     {\Tuple{{#1},{#2},{#3},{#4}}}
\newcommand{\SubstFun}[2]{#1[#2]}

%metaregole
\newcommand{\Rule}[3]{\displaystyle                  
\frac{#1}{#2}        %  #1 = premesse (modo math) %  #2 = conseguenza (modo math)
{#3}     %  #3 = side conditions (modo math)
}

\newcommand{\NamedRule}[4]{\scriptstyle{\textsc{(#1)}}
\displaystyle                  %  #1 = nome regola
\frac{%
\begin{array}{l}#2\end{array}%
}{#3}        %  #2 = premesse (modo math) %  #3 = conseguenza (modo math)
\begin{array}{l}#4\end{array}     %  #4 = side conditions (modo math)
}
\newcommand{\SmallNamedRule}[4]{\scriptstyle{\textsc{(#1)}}
\displaystyle                  %  #1 = nome regola
\frac{%
\begin{array}{l}#2\end{array}%
}{#3}\Space           %  #2 = premesse (modo math) %  #3 = conseguenza (modo math)
\scriptstyle{\begin{array}{l}#3\end{array}}     %  #4 = side conditions (modo math)
}

%metaregole in un asola riga
\newcommand{\NamedRuleOL}[3]{\scriptstyle{\textsc{(#1)}}
\displaystyle                  %  #1 = nome regola
{%
\begin{array}{l}#2\end{array}}\           
\begin{array}{l}#3\end{array}     %  #3 = side conditions (modo math)
}
\newcommand{\SmallNamedRuleOL}[3]{\scriptstyle{\textsc{(#1)}}
\displaystyle                  %  #1 = nome regola
{#2}       % 
\scriptstyle{\begin{array}{l}#3\end{array}}     %  #3 = side conditions (modo math)
}

\newcommand{\rn}[1]{{\scriptsize (\textsc{#1})}}					% Rule name

%grammatiche
\newenvironment{grammatica}{$\begin{array}{lcll}}{\end{array}$}
\newcommand{\produzione}[3]{#1&::=&#2&\mbox{#3}}
\newcommand{\produzioneinline}[2]{#1::=#2}
\newcommand{\seguitoproduzione}[2]{&&#1&\mbox{#2}}
\newcommand{\terminale}[1]{\texttt{#1}}
\newcommand{\metavariable}[1]{\mathit{#1}}

%metavariabili
\newcommand{\n}{\metavariable{n}}
\newcommand{\cd}{\metavariable{cd}}
\newcommand{\cds}{\metavariable{cds}}
\newcommand{\fd}{\metavariable{fd}}
\newcommand{\fds}{\metavariable{fds}}
\newcommand{\md}{\metavariable{md}}
\newcommand{\mds}{\metavariable{mds}}
\newcommand{\x}{\metavariable{x}}
\newcommand{\y}{\metavariable{y}}
\newcommand{\z}{\metavariable{z}}
\newcommand{\xs}{\metavariable{xs}}
\newcommand{\ys}{\metavariable{ys}}
\newcommand{\zs}{\metavariable{zs}}
\newcommand{\m}{\metavariable{m}}
\newcommand{\e}{\metavariable{e}}
\newcommand{\es}{\metavariable{es}}
\newcommand{\C}{\metavariable{C}}
\newcommand{\D}{\metavariable{D}}
\newcommand{\f}{\metavariable{f}}
\newcommand{\dec}{\metavariable{d}}
\newcommand{\decs}{\metavariable{ds}}
\newcommand{\X}{\metavariable{X}} %insieme di variabili

%%terminali
\newcommand{\this}{\terminale{this}}
\newcommand{\mutable}{\terminale{mut}}
\newcommand{\imm}{\terminale{imm}}
\newcommand{\lent}{\terminale{lent}}
\newcommand{\readable}{\terminale{read}}
\newcommand{\capsule}{\terminale{capsule}}

%%costrutti sintattici
\newcommand{\Field}[2]{#1\ #2\terminale{;}}
\newcommand{\MethDec}[5]{{#1\ #2\ \terminale{(}#3,#4\terminale{)}\ \terminale{\{}#5\terminale{\}}}}
\newcommand{\Param}[2]{#1\ #2}
\newcommand{\FieldAccess}[2]{#1\terminale{.}#2}
\newcommand{\MethCall}[3]{#1\terminale{.}#2\terminale{(}#3\terminale{)}}
\newcommand{\FieldAssign}[3]{#1\terminale{.}#2\terminale{=}#3}
\newcommand{\ConstrCall}[2]{\terminale{new}\,#1\!\terminale{(}#2\terminale{)}}
\newcommand{\Block}[2]{\terminale{\{}#1\ #2\terminale{\}}}
\newcommand{\Dec}[3]{#1\,#2\terminale{=}#3\terminale{;}}
%\newcommand{\Field}[2]{#1\ #2}
%\newcommand{\PrExp}[2]{(#1)\,#2}
\newcommand{\Sequence}[2]{#1\terminale{;} #2}
\newcommand{\emptyDvs}{\epsilon}

%%type system
\newcommand{\WellFormedTypeCtx}[1]{\vdash#1}
\newcommand{\Encoded}[2]{#1\rightsquigarrow #2}
\newcommand{\MutGroup}{\metavariable{xs}}
\newcommand{\intType}{\terminale{int}}
\newcommand{\T}{\metavariable{T}}
\newcommand{\TPrime}{{\T}'}
\newcommand{\Type}[2]{#1\,#2}
\newcommand{\IsWellTyped}[1]{\vdash #1}
\newcommand{\TypeCheckGround}[2]{\vdash #1:#2}
\newcommand{\TypeCheck}[5]{#1;#2;#3\vdash #4:#5}
\newcommand{\AuxTypeCheck}[6]{#1;#2\,[#3];#4\vdash #5:#6}
\newcommand{\TypeCheckShort}[3]{#1\vdash #2:#3}
\newcommand{\TypeDec}[2]{#2{:}#1}
\newcommand{\LentLocked}{\metavariable{xss}}
\newcommand{\StronglyLocked}{\metavariable{ys}}
\newcommand{\Reduct}[2]{#1_{|#2}}
\newcommand{\Extends}[3]{#1\sqsubseteq_{#2}#3}
\newcommand{\LentVars}[1]{\aux{lentVars}(#1)}
\newcommand{\MutVars}[1]{\aux{mutVars}(#1)}
\newcommand{\LessEq}[2]{#1\sqsubseteq#2}

%%valori e contesti
\newcommand{\dv}{\metavariable{dv}}
\newcommand{\dvs}{\metavariable{dvs}}
\newcommand{\Ctx}[1]{\ctx[#1]}
\newcommand{\XCtx}[1]{\Xctx[#1]}
\newcommand{\CtxP}[1]{\ctxP[#1]}
\newcommand{\CtxS}[1]{\ctxS[#1]}
\newcommand{\XCtxP}[1]{\XctxP[#1]}
\newcommand{\ctx}{{\cal{E}}}
\newcommand{\genCtx}{{\cal{G}}}
\newcommand{\GenCtx}[1]{\genCtx[#1]}
\newcommand{\genCtxP}{{\cal{G'}}}
\newcommand{\GenCtxP}[1]{\genCtxP[#1]}
\newcommand{\Xctx}{{\cal{C}}}
\newcommand{\ctxP}{{\cal{E}'}}
\newcommand{\ctxS}{{\cal{E}''}}
\newcommand{\XctxP}{{\cal{C}'}}
\newcommand{\emptyctx}{[\ ]}
\newcommand{\val}{\metavariable{v}}
\newcommand{\stVal}{\metavariable{rv}}
\newcommand{\valPrime}{\metavariable{u}}
\newcommand{\vals}{\metavariable{vs}}
\newcommand{\vs}{\metavariable{vs}}

%%riduzione
\newcommand{\reduce}[2]{#1\longrightarrow#2}
%\newcommand{\reducestar}[2]{#1\longrightarrow^\star#2}
\newcommand{\Subst}[3]{#1[#2/#3]}
\newcommand{\Update}[4]{{#1[#2.#3{=}#4]}}

%%equivalenza
\newcommand{\congruence}[2]{{#1}\cong{#2}}

%%funzioni ausiliarie
\newcommand{\extractMod}[1]{\mu^{#1}}
\newcommand{\aux}[1]{\textsf{#1}}
\newcommand{\notRef}[1]{{\aux{notVar}}(#1)}
\newcommand{\extractDec}[2]{\aux{dec}(#1,#2)}
\newcommand{\extractType}[2]{\aux{type}(#1,#2)}
\newcommand{\typeOf}[1]{\aux{type}(#1)}
\newcommand{\extractField}[3]{\aux{get}(#1,#2,#3)}
\newcommand{\fieldOf}[2]{\aux{get}(#1,#2)}
\newcommand{\fields}[1]{\aux{fields}(#1)}
\newcommand{\method}[2]{{\aux{method}(#1,#2)}}
\newcommand{\FV}[1]{\aux{FV}(#1)}
\newcommand{\HB}[1]{\aux{HB}(#1)}
\newcommand{\dom}[1]{\aux{dom}(#1)}
\newcommand{\domMut}[1]{\aux{dom}^\mutable(#1)}
\newcommand{\domLent}[1]{\aux{dom}^\lent(#1)}
\newcommand{\domGeqMut}{{\aux{dom}^{\geq\terminale{mut}}}}
\newcommand{\ImmClosed}[2]{#1\models\aux{imm-closed}(#2)}
\newcommand{\allImm}[1]{\forall^\imm(#1)}
\newcommand{\allMut}[1]{\forall^{\geq\mutable}(#1)}
\newcommand{\noCapture}[2]{#1{\parallel}#2}

%----------scaled enviroment-------------
\newsavebox{\saveScaled}
\newcommand*{\varScalableEnvironmentWidth}{1}
\newcommand*{\varScalableEnvironmentHeight}{1}
\newenvironment{Scaled}[2]{%
  \renewcommand*{\varScalableEnvironmentWidth}{#1}
  \renewcommand*{\varScalableEnvironmentHeight}{#2}
  \begin{lrbox}{\saveScaled}
  }{
  \end{lrbox}
  \noindent\scalebox{\varScalableEnvironmentWidth}[\varScalableEnvironmentHeight]{\usebox{\saveScaled}}
  }
%
%
%%%% MACRO PAOLA
%
\newcommand{\storableVal}{right-value}
\newcommand{\storableValues}{right-values}
\newcommand{\storableVals}{right-values}
\newcommand{\preRedex}{{\rho}}
\newcommand{\deriv}{{\cal D}}
\newcommand{\WFrv}[1]{\models#1}
\newcommand{\WFdvs}[1]{\models#1}
\newcommand{\WFdv}[1]{\models_{st}#1}
\newcommand{\WFval}[1]{\PG{\models_{st}#1}}
%\newcommand{\WFdvMod}[2]{\models_{#1}#2}
\newcommand{\dependVar}[2]{\aux{dpd}(#1,#2)}
\newcommand{\mutDvs}[1]{\aux{mutDvs}(#1)}
%\newcommand{\TypeEnv}[1]{\aux{tenv}(#1)}
\newcommand{\TypeEnv}[1]{\Gamma_{\!#1}}
\newcommand{\TypeEnvDec}[1]{\aux{tenv}(#1)}
\newcommand{\DecsOK}[4]{#1;#2;#3\vdash #4\,{\tt OK}}
\newcommand{\AuxDecsOK}[5]{#1;#2[#3];#4\vdash #5\,{\tt OK}}
\newcommand{\leqLL}{\sqsubseteq}
\newcommand{\Depends}[2]{\aux{dpd}(#1,#2)}

\newcommand{\capsulectx}{\decctx{\cx}}
\newcommand{\immctx}{\decctx{\ix}}
\newcommand{\Capsulectx}[1]{\Decctx{\cx}{#1}}
\newcommand{\Immctx}[1]{\Decctx{\ix}{#1}}
\newcommand{\CapsulectxP}[1]{{\cal C}'_{\cx}[#1]}
\newcommand{\ImmctxP}[1]{{\cal C}'_{\ix}[#1]}
\newcommand{\ImmctxS}[1]{{\cal C}''_{\ix}[#1]}
\newcommand{\decctx}[1]{{\cal C}\!_{#1}}
\newcommand{\Decctx}[2]{{\cal C}\!_{#1}[#2]}
\newcommand{\DecEvCtx}[2]{#1_{#2}}

\newcommand{\mux}{\mu\hspace{.015cm}\x}
\newcommand{\cx}{{\tt c}\hspace{.015cm}\x}
\newcommand{\ix}{{\tt i}\hspace{.015cm}\x}

%\newcommand{\Extend}[2]{{\it Xtd}(#1,#2)}
\newcommand{\Range}[3]{\forall#1\in#2..#3}

%\newcommand{\subderivation}[2]{#1{\Longrightarrow}#2}

%\newenvironment{myitemize}
%               {\begin{itemize}\vspace{-2pt}\topsep0pt\parskip0pt\partopsep0pt\itemsep0pt\leftmargin-100pt\itemsep-1pt\labelwidth0pt\labelsep3pt}
%               {\vspace{-1pt}\end{itemize}}
               
%\newtheorem*{theorem*}{Theorem}

\newcommand{\cOrx}{w}

\newcommand{\connected}[3]{#2\stackrel{#1}{\longrightarrow}#3}
%\newcommand{\TypeOf}[3]{{#1}_{#2}(#3)}


