\section{Classification, completeness and decidability}

\noindent
We keep the notation conventions in the first paragraph of the preceding section and moreover assume in this section that $(F, K; \chi) \models \TN_p$. For a field $K$, we let $K^{\text{ac}}$ denote an algebraic closure of $K$.
We classify the models of $\TN_p$ up to isomorphism. From this we deduce that the theory $\TN_p$ is complete and decidable.


\begin{prop}\label{CharacterIsomorphism}
Suppose $(F, K; \chi_1)$ and $(F, K, \chi_2)$ are models of $\TN_p$ with \( \qq\big(\chi_1(F) \big)= \qq\big(\chi_2(F) \big)\). Then there is an automorphism  \(\sigma\) of \(K\) with \( \chi_2 = \sigma \circ \chi_1 \). 
\end{prop}

\begin{proof}
Suppose $F, K,\chi_1$ and $\chi_2$ are as stated. Let \(\alpha= (\alpha_i)_{i \in I}\) be a listing of the elements of \(F^\times\). As \( \chi_1, \chi_2 \) are group homomorphisms, \( \text{mtp}\big(\chi_1(\alpha)\big) = \text{mtp}\big(\chi_2(\alpha)\big)  \). By Proposition~\ref{GenericEquivalence}, \( \text{atp}\big(\chi_1(\alpha)\big) = \text{atp}\big(\chi_2(\alpha)\big) \),  and so there is a field automorphism 
$$\sigma: \qq\big(\chi_1(F) \big) \to \qq\big(\chi_2(F) \big) $$ 
such that \( \chi_2 = \sigma \circ \chi_1 \).
We can further extend $\sigma$ to a field automorphism of \(\qq\big(\chi_1(F) \big)^{ac}= \qq\big(\chi_2(F) \big)^{ac}  \) and then to an automorphism of \(K\).
\end{proof}

\begin{cor} \label{UniquenessOfChar}
If $p$ is prime, $F = \fpa$ and  $K = \qa$, then there is a unique injective character from $F$ to $K$ up to isomorphism.
\end{cor}

\begin{cor} \label{AutExt}
If $\chi: F \to K$ is generic and if $\sigma$ is an automorphism of $F$, then $\sigma$ can be extended to an automorphism of \((F, K;\chi)\).
\end{cor}

\noindent
We say  $(F,K;\chi) \models \TNp$ is {\bf $(\kappa,\lambda)$-transcendental} if $\trdeg(F \mid \ff_p)=\kappa $ and $\trdeg\big(K \mid \qq(G)\big)  = \lambda $ with $G=\chi(F^\times)$.

\begin{thm}\label{ThmIso}

For any $p$,$\kappa$ and $\lambda$, there is a unique $(\kappa,\lambda)$-transcendental model of $\TNp$ up to isomorphism .
\end{thm}


\begin{proof}

We first prove the uniqueness part of the lemma. Suppose $(F_1,K_1;\chi_1)$ and $(F_2,K_2;\chi_2)$ are $(\kappa,\lambda)$-transcendental models of $\TNp$. Let $G_1$ be $\chi (F_1^\times)$ and $G_2$ be $\chi (F_2^\times)$. As $F_1$ and $F_2$ are algebraically closed of same characteristic and  $\trdeg(F_1 \mid\ff_p) = \trdeg(F_2 \mid \ff_p) $, there is an isomorphism $$\sigma: F_1 \to F_2.$$ Using Proposition~\ref{GenericEquivalence} in a similar way as in the proof of Proposition~\ref{CharacterIsomorphism}, $\sigma$ induces an isomorphism between $ \qq(G_1)$ and $\qq(G_2)$; we will also call this $\sigma$. Finally, since $\trdeg\big(K_1\mid\qq(G_1)\big)$ is equal to $\trdeg\big(K_2\mid \qq(G_2)\big)$ we can extend $\sigma$ to a field isomorphism from $K_1$ to $K_2$. It is easy to check that this is an isomorphism of $L$-structures.

We next prove the existence part of the lemma. For $p>0$, $\TN_p$ clearly has a model. For $p=0$, $\TN_p$ has a model by compactness. We can arrange to have for each $p$ a model $(F, K; \chi)$ of $\TN_p$ such that $|F|, |K| > \max\{\kappa, \lambda, \aleph_0\}$. Choose an algebraically closed subfield $F'$  of $F$ with $\trdeg(F' \mid \ff_p) = \kappa$. Then we have $\trdeg\big(K \mid \qq( \chi(F'))\big) > \lambda$. Choose an algebraically closed subfield $K'$ of $K$ containing $\chi(F')$ with $\trdeg\big(K' \mid \qq(\chi(F'))\big) =\lambda$. We can check that $(F', K'; \chi\upharpoonright_{F'})$ is a $(\kappa,\lambda)$-transcendental model of $\TNp$.
\end{proof}

\begin{cor}
$\TN_p$ is superstable, shallow, without the dop, without the otop, without the fcp.
\end{cor}

\begin{proof}
The first four properties follow from Shelah's main gap theorem \cite[XII.6.1]{Shelah}. The last property follows from  \cite[VII.3.4]{Shelah}.
\end{proof}

\noindent
Next we prove an analog of upward L\"owenheim-Skolem theorem.
%In the next few lemmas we prove that any model $(F, K; \chi)$ of $\TNp$ has a $(\kappa, \kappa)$-transcendental elementary extension for any $\kappa \geq \max(|F|, |K|)$. This should be seen as an analog of Upward L\"owenheim-Skolem theorem. 
\begin{lem}
For $\chi: F \to K$ generic, $K$ is an infinite extension of $\qq\big(\chi(F)\big)$.
\end{lem}

\begin{proof}
Suppose $F, K$ and $\chi$ are as stated. Let $G= \chi( F)$. By Proposition~\ref{Regularity}, if $U$ consists of the roots of unity in $G$, then $\qq(G)$ is a regular extension of $\qq(U)$. Hence, 
$$\big[\qa:\qq(U)\big]\ \leq\ \big[\qa\qq(G): \qq(G) \big].$$ By Galois theory, $\big[\qa:\qq(U)\big] = \infty$. Therefore, $\big[\qa\qq(G): \qq(G)\big]= \infty$ and so $\big[K: \qq(G)\big]= \infty$.
\end{proof}












\begin{lem} \label{KappaExt}
Every model $(F, K; \chi)$ of $\TNp$  has a $(\kappa ,\kappa)$-transcendental elementary extension $(F', K'; \chi')$ for any cardinal $\kappa \geq \max(|F|, |K|)$.
\end{lem}

\begin{proof}
Let $(F, K; \chi)$ and $\kappa$ be as stated.
We construct an elementary extension $(F'', K''; \chi'')$ of $(F, K; \chi)$ with $\text{trdeg}(F''\mid \ff_p) \geq \kappa$ and $\text{trdeg}(K''\mid G'') \geq \kappa$ with $G'' = \chi({F''}^\times)$.
For the later two conditions to hold, it suffices to ensure there are $$\alpha\ \in\ (F'')^\kappa\ \text{ and }\ a\ \in\ (K'')^\kappa$$ such that components of $\alpha$ are all distinct and the components of $a$ are algebraically independent over $G''$. 
Using compactness, we can reduce the problem to verifying the following: for arbitrary $k,m,n$, $w$  of length $m$, $x$ of length $n$ and arbitrary polynomials $P_1, \ldots, P_l $ in $\qq[w,x]$, there are $\alpha$ in $F^k$ and $a$ in $K^n$ such that components of $\alpha$ are pairwise different, and
$$P_i\big(\chi(\beta), a\big)\ \neq\ 0\ \text{ for all } \beta \in F^m \text{ and } i \in \{1 \ldots l\}.$$
It is easy to find $\alpha$ with the desired property. By preceding lemma, $ \big[K: \qq(G)\big]$ is infinite, so we can choose  $a$ so that $\big[\qq(G, a_1, \ldots, a_j): \qq(G, a_1 \ldots,a_{j-1})\big]>N$ for $j \in \{ 1, \ldots, n\}$ where $N$ is the maximum degree of $P_i$ for $ i \in \{1 \ldots l\} $.
We see that this choice of $a$ works. We then get the desired $(F', K'; \chi')$ from $(F'', K''; \chi'')$ by taking the Skolem Hull of the suitable elements.
\end{proof}

\begin{thm}\label{CompAxiom}
For all $p$, $\TNp$ is complete and  decidable. When $p>0$, $\TNp$ axiomatizes $\mathrm{Th}(\ff, \cc; \chi)$ where $\chara(\ff) = p$.
\end{thm}


\begin{proof}
We first show that any two arbitrary models \( (F_1, K_1; \chi_1) \) and \( (F_2, K_2; \chi_2) \) of $\TNp$ are elementarily equivalent. By the preceding lemma, we can arrange that \( (F_1, K_1; \chi_1) \) and \( (F_2, K_2; \chi_2)\)  are both $(\kappa, \kappa)$-transcendental. It follows from  Theorem ~\ref{ThmIso} that for all $p$, $\TNp$ is complete. The remaining conclusions are immediate.
\end{proof}




\begin{cor}
Let $\tau$ be an $L$-statement. The following are equivalent:
\begin{enumerate}
\item $\tau$ is true in some model of $\TN_0$;
\item there are arbitrarily large primes $p$ such that $\tau$ is true in some model of $\TNp$;
\item there is a number $m$ such that for all primes $p>m$, $\tau$ is true in all models of $\TNp$.
\end{enumerate}
\end{cor}



