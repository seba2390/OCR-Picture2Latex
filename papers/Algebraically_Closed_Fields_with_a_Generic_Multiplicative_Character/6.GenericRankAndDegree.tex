\section{Definable sets II}

%------------------------------------------------------------------------------------------------------------------------------------------------------------------------------------------------------------------------------------------------------------------------------------

\noindent
We keep the notation conventions of the preceding section. Furthermore, We assume that, with possible decorations, $V, W,  C$ are $K$-algebraic subsets of their ambient spaces and $C$ is $K$-irreducible; also with possible decorations, $S$ is an algebraically presentable subset of its ambient space.

The goal of this section is to obtain an ultimate description of definable sets in $(F, K; \chi)$ which allows us to define good notions of dimension and multiplicity. 
This is done in several steps by introducing intermediate descriptions which gradually increase our geometric understanding of definable sets.
This analysis is complicated by the fact that not all definable sets are algebraically presentable; indeed, algebraically presentable sets are definable with only existential formulas while $\TN_p$ is not model complete. If we try to replace ``algebraic'' with ``constructible'', we will still run into the same problem. Therefore, we will need to take one step further.

We call $T \subseteq K^n$ a {\bf pseudo-constructible set} (or {\bf pc-set}) if there are $ V, S \subseteq K^n$   such that $T= V \backslash S$. A pc-set is clearly definable. If $V'$ is the closure of $T$ in the $K$-topology then $T= V' \backslash (S \cap V')$, and $S \cap V'$ is also algebraically presentable. Hence, if $T$ is a pc-set, there is a choice of $V, S$ such that $V$ is the closure of $T$ in the $K$-topology. Throughout the rest of this section, $T$ with possible decorations is a pc-subset of its ambient space. If $T= V \backslash S$, and $S$ has an algebraic presentation with only finitely many elements, then $T$ is a constructible set in the $K$-topology. This section is based on the observation that we can almost pretend pc-sets are constructible sets in the $K$-topology. The underlying reason is the following:

\begin{lem}\label{Smallness1}
Suppose $S \subseteq K^n$ has algebraic presentation $\{W_\beta\}_{\beta \in E}$ and $C$  is a subset of $S$. Then $C$ is a subset of $W_\beta$ for some $\beta \in E$.
\end{lem}

\begin{proof}
Suppose $S, C$ and $\{W_\beta\}_{ \beta \in E}$ are as stated. By Corollary~\ref{SaturatedExtension}, we can arrange that $K$ is $|F|^+$-saturated as a field. As $|E| \leq |F|$, $C \subseteq \bigcup_{\beta \in E} W_\beta $ implies $C$ is a subset of a union of finitely many elements in $\{W_\beta\}_{ \beta \in E}$. Since $C$ is irreducible, $C \subseteq W_\beta$ for some $\beta \in E$.
\end{proof}


\begin{cor} \label{Smallness2}
If $C, S, S' \subseteq K^n$ are such that $C \subseteq S\cup S'$, then either $C \subseteq S$ or $C \subseteq S'$.
\end{cor}

\noindent
The above lemma allows us to analyze pc-sets through their closures.

\begin{cor} \label{Smallness3}
Suppose $C, S$ are subsets of $K^n$. Then  $C \backslash S$ has closure $C$ if and only if $C \nsubseteq S$.
\end{cor}

\begin{proof}
 Suppose $C, S \in K^n$ are as stated. The forward direction is clear. Suppose $ \{W_\beta \}_{\beta \in E}$ is an algebraic presentation of $S$ and $C \nsubseteq S$. Let $V$ be the closure of $C \backslash S$. Then $C \backslash S \subseteq V \subseteq C \subseteq V \cup S $. By the preceding corollary, the last inclusion implies either $C \subseteq V$ or $C \subseteq S$. By assumption $C \nsubseteq S$, so $C \subseteq V$. Thus, $C = V$ as desired.
\end{proof}

\begin{cor} \label{Smallness4}
Suppose $V=C_1\cup\ldots\cup C_k, X = V_1 \backslash S_1 \cup \ldots V_l \backslash S_l$ are subsets of $K^n$, and $X$ is a subset of $ V$. Then $V$ is the  $K$-closure of $X$ if and only if for each $i \in \{1, \ldots,k\}$, there is $j \in  \{1, \ldots, l\} $ such that $C_i \subseteq V_j$ and $C_i \nsubseteq S_j$.
\end{cor}

\begin{proof}
We have $V$ is the $K$-closure of $X$ if and only if for each $i \in {1, \ldots, k}$, $C_i$ is the closure of $C_i \cap X$. For each $i$, as $C_i$ is $K$-irreducible, $C_i$ is the closure of $C_i \cap X$ if and only if there is  $j \in \{1, \ldots, l\} $ such that $C_i$ is the closure of $ C_i \cap ( V_j \backslash S_j)= (C_i \cap V_j) \backslash S_j$. By the preceding corollary, this happens if and only if $C_i \cap V_j =C_i$ and $C_i \nsubseteq S_j$.
\end{proof}

\begin{cor} \label{ClosurePreserve}
Suppose $T, V$ are subsets of $K^n$. If $V$ is the closure of $T$ in the $K$-topology, then this also holds in any elementary extension of the model.
\end{cor}

\begin{proof}
This follows immediately from the preceding corollary.
\end{proof}

\noindent
The collection of pc-sets is not closed under complement. The following definitions allow us to overcome this limitation.
Suppose $T,T'$ are subset of $K^n$ and $V'$ is the closure of $T'$ in the $K$-topology. We define $T \capdot T'$, to be $T \cap V'$. Note that this definition is not symmetric. Set $T \dotminus T'$ to be $ T \backslash V'$. Clearly, $T = (T \capdot T') \cup (T \dotminus T')$.
By a routine manipulation of formulas we get:
 
\begin{lem} 
If $T, T' \subseteq K^n$ then $T \cap T'$, $T \capdot T'$, $T \dotminus T'$ are pc-sets. Likewise, for $T \subseteq K^n$, $T'\subseteq K^{n'}$, we have that $ T \times T' \subseteq K^{n+n'}$ is also a pc-set.
\end{lem}

\noindent
Suppose $T$, $T'$, $V$, $V'$ are subsets of $K^n$, and $V$ is the $K$-closure of $T'$, $V'$ is the $K$-closure of $T'$. The {\bf$K$-Morley rank} of $T$ is defined by $\mr_K(T)= \mr_K(V)$; the {\bf$K$-Morley degree}  of $T$ is defined by $\md_K(T)=\md_K(V)$.



We say $T$ is {\bf almost a subset} of $T'$, if  $\mr_K(T) = \mr_K(T') =\mr_K(T \cap T')$ and $\md_K(T\cap T') =\md_K(T)$, and denote it by $T  \subsim T'$. We note that in our definition $T\subseteq T'$ does not imply $T \subsim T'$ as we might have $\mr_K(T) < \mr_K(T')$. The definition is given in this way to simplify the notation in the case of $\mr_K(T) =\mr_K(T')$, which is our focus.
We say $T$ and $T'$ are {\bf almost equal}, denoted by $T\sim T'$, if $T \subsim T'$ and $T' \subsim T$. We say $T$ and $T'$ are { \bf almost disjoint}, denoted by $T \simperp T'$ if $\mr_K(T)=\mr_K(T')$ and $\mr_K(T\cap T')< \mr_K(T)$.

The following facts are very natural analogues of what we expect to be true about constructible sets in $K$-topology. All are either straightforward from the definitions or easy consequences of Corollaries~\ref{Smallness2} and~\ref{Smallness4}.
\begin{prop} \label{lies}
Suppose $T$, $T_1$, $T_2$, $V_1$, $V_2$ are subsets of $ K^n$ and $T'$, $T_1'$, $T_2'$ are subsets of $K^{n'}$, and $V_1$,$V_2$ are respectively the closure of $T_1$,$T_2$ in the $K$-Zariski topology. Then we have the following:
\begin{enumerate}
\item if $T_1$ is a subset of $T_2$, then either $\mr_K(T_1) \leq \mr_K(T_2)$ or $\mr_K(T_1) = \mr_K(T_2)$ and $\md_K(T_1) \leq \md_K(T_2)$;
\item if $T = T_1 \cup T_2$, then $\mr_K(T) = \max\{ \mr_K(T_1), \mr_K(T_2)\} $;
\item if $T = T_1 \cup T_2$, $T_1 \simperp T_2$, then $\md_K(T) = \md_K(T_1)+\md_K(T_2)$;
\item the relation $\subsim$ is transitive; the relation $\sim$ is an equivalent relation;
\item if $\mr_K(T_1)=\mr_K(T_2)= \mr_K(T _1 \cap T_2)$, then $T_1 \cap T_2 \sim T_1 \capdot T_2 \sim V_1 \cap V_2$;
\item if $\mr_K(T_1) = \mr_K(T_2)$, then exactly one of the following can happen: $T_1 \simperp T_2$, $T_1 \subsim T_2$ or $\mr_K(T_1)=\mr_K(T_2)= \mr_K(T _1 \capdot T_2) = \mr_K( T_1 \dotminus T_2)$;
\item if $T_1 \simperp T_2$, then $\mr_K(T_1 \capdot T_2) < \mr(T_1) $ and $T_1 \dotminus T_2 \sim T_1$;
\item if $T_1 \subsim T_2$, then $ T_1 \cap T_2 \sim T_1 \capdot T_2 \sim T_1$;
\item if $\mr_K(T_1)=\mr_K(T_2)= \mr_K(T _1 \capdot T_2) = \mr_K( T_1 \dotminus T_2)$, then 
$$\md_K(T _1 \capdot T_2)+\md_K(T_1 \dotminus T_2) \ =\ \md_K(T_1);$$
\item if $\mr_K(T_1)= \mr_K(T_2) = \mr_K(T_1 \cap T_2)$ and $T'_1 \sim T_1, T'_2 \sim T_2$, then $T'_1 \cap T'_2 \sim T_1 \cap T_2$. The same conclusion holds if as we replace all appearances of $\cap$ in the previous statement with one of $\cup, \capdot, \dotminus$;
\item $\mr_K( T \times T') = \mr_K(T) + \mr_K(T')$, $ \md_K( T \times T') = \md_K(T)\md_K(T')$;
\item if $T_1  \sim T_2$ and $T_1' \sim T'_2$, then $T_1 \times T'_1 \sim T_2 \times T'_2$; if $T_1 \sim T_2$ and $T'_1 \simperp T'_2$, then $T_1 \times T'_1 \simperp T_2 \times T'_2$; if   $T_1 \simperp T_2$ and $T'_1 \simperp T'_2$, then $T_1 \times T'_1 \simperp T_2 \times T'_2$.
\end{enumerate}
\end{prop}











%%%%%%%%%%%%%%%%%%%%% End of part 1
\noindent
We now introduce our first intermediate description of definable sets in a model of $\TN_p$.
A {\bf pseudo constructible presentation}  (or {\bf pc-presentation}) of a set $X \subseteq K^n$ is a definable family $ \{ T_\alpha\}_{ \alpha \in D} $ where $D \subseteq F^k$ for some $k$, such that $X = \bigcup_{ \alpha \in D} T_\alpha $.  We will also talk of a pc-presentation without mentioning $X$; by that we mean a pc-presentation for some $X$, but $X$ plays no important role. If $X$ has a pc-presentation then $X$ is definable. We will show that the converse is also true. The following is immediate:

\begin{lem}
Suppose  $\{T_\alpha\}_{ \alpha \in D}$  and  $\{T'_{\alpha'}\}_{ \alpha' \in D'}$ are pc-presentations definable over $c \in K^m$. A
fiberwise intersection (respectively fiberwise union, disjoint combination or fiberwise product) of $\{T_\alpha\}_{ \alpha \in D}$ and $\{T'_{\alpha'}\}_{ \alpha' \in D'}$ can be chosen to be also a pc-presentation definable over $c$. 
\end{lem}

\begin{prop} \label{pc-pres}
Every definable subset of $K^n$ has a pc-presentation.
\end{prop}

\begin{proof}
We will first show the statement for $X \subseteq K^n$ of the form $ S \backslash S'$. By definition $ S $ has an algebraic presentation $ \{ V_\alpha\}_{ \alpha \in D} $. For $ \alpha \in D$, let $ T_\alpha = V_\alpha \backslash S' $. It can be easily checked that $ \{ T_\alpha\}_{ \alpha \in D} $ is a pc-presentation of $ S \backslash S'$. By Theorem~\ref{QuanRed}, every definable subsets of $K^n$ can be written as a finite union of sets of the form $ S \backslash S'$ where $S, S' \subseteq K^n$. By the preceding lemma and Lemma~\ref{fiber}, the collection of sets having a pc-presentation is closed under finite union. 
The conclusion follows.
\end{proof}



\noindent
The next proposition allows us to define a geometrical invariant of a definable set based on a choice of its pc-presentation and yet independent of such choice.

\begin{prop} \label{Kgeorank}
If $X \subseteq X' \subseteq K^n$ are definable, $X$ has a pc-presentation $\{ T_\alpha\}_{ \alpha \in D}$ and $X'$ has a pc-presentation $\{ T'_{\alpha'}\}_{ \alpha' \in D'}$, then $\max_{\alpha \in D} \mr_K(T_\alpha) \leq \max_{\alpha' \in D'} \mr_K(T'_{\alpha'}) $. 
\end{prop}

\begin{proof}
Suppose $X$, $X'$, $\{ T_\alpha\}_{ \alpha \in D}$, $\{ T'_{\alpha'}\}_{ \alpha' \in D'}$ are as given. Let $\alpha$ be such that $\mr_K(T_{\alpha}) =\max_{\beta \in D}\mr_K(T_\beta) $. We can arrange that $T_\alpha = V_\alpha \backslash S_\alpha$ with $V_\alpha$ the $K$-closure of $T_\alpha$; let $C_\alpha$ be one of the the components of $V_\alpha$ with $K$-dimension $\mr_K(T_\alpha)$. Then by Lemma~\ref{Smallness4}, if $\{W_\beta\}_{\beta \in E}$ is an algebraic presentation of $S_\alpha$, 
$$\mr_K(C_\alpha \cap W_\beta)\ <\ \mr_K(T_\alpha)\ \text{ for each }\ \beta \in E.$$
Now suppose for all $\alpha' \in D'$, $\mr_K( T'_{\alpha'})< \mr_K(T_\alpha)$. For each $\alpha' \in D'$, let $V'_{\alpha'}$ be the $K$-closure of $T'_{\alpha'}$. We note that the family $\{ V'_{\alpha'}\}_{ \alpha' \in D'}$ has cardinality at most $|F|$; also, by Corollary~\ref{SaturatedExtension}, we can arrange that $K$ is $|F|^+$-saturated as a field. By dimension comparison, $C_\alpha$ is not a subset of a union of finitely many elements of $\{ V'_{\alpha'}\}_{ \alpha' \in D'}$ and finitely many elements of $\{W_\beta \cap C_\alpha\}_{\beta \in E}$. Thus $C_\alpha$ is not a subset of the union of $\{ V'_{\alpha'}\}_{ \alpha' \in D'}$ and $\{C_\alpha \cap W_\beta\}_{\beta \in E}$. This implies $T_\alpha$ is not a subset of the union $\{ V'_{\alpha'}\}_{ \alpha' \in D'}$, a contradiction; the conclusion follows.
\begin{comment}
There is some $c \in K^l$ such that the family $\{ T'_{\alpha'}\}_{ \alpha' \in D'}$ is definable over $c$. Hence, for each $\alpha' \in D'$, $T'_{\alpha'}$ is definable over $c,\chi(F)$; by lemma~\ref{DefClosure1}, for each $\alpha' \in D'$, $V'_{\alpha'}$ is definable over $\qq(c, \chi(F))$ in the field sense; 
\end{comment}
\end{proof}

\begin{cor}
If $X \subseteq K^n$ has pc-presentations $\{ T_\alpha\}_{ \alpha \in D}$ and $\{ T'_{\alpha'}\}_{ \alpha' \in D'}$, then $\max_{\alpha \in D} \mr_K(T_\alpha) = \max_{\alpha' \in D'} \mr_K(T'_{\alpha'}) $.
\end{cor}

\noindent
Suppose $X\subseteq K^n$ has  a pc-presentation $\{ T_\alpha\}_{ \alpha \in D}$. Then the {\bf $K$-geometric rank} of $X$, denoted by $ \gr_K(X)$, is defined to be  $\max_{\alpha \in D} \mr_K(T_\alpha)$. The following is also immediate from the previous proposition:

\begin{cor}
Suppose $X, X' \subseteq K^n$ are definable. Then we have the following:
\begin{enumerate}
\item  if $X \subseteq X'$, then $\gr_K(X) \leq \gr_K(X')$;
\item $\gr_K(X \cup X') = \max\{ \gr_K(X), \gr_K(X') \}$.
\end{enumerate}
\end{cor}


%%%%%%%%%%%%%%%%%%%%%% End of part 2
\medskip \noindent
Next, we introduce the second intermediate description of definable sets in a model of $\TN_p$.
A family $\{T_b\}_{b \in Y}$ of pc-sets is {\bf essentially disjoint} if for any $b, b' \in Y$ such that $ \mr_K(T_b) =\mr_K(T_{b'}) = \max_{b \in Y} \mr_K(T_b)$, we have  either $T_b \sim T_{b'}$ or  $T_b \simperp T_{b'}$. An essentially disjoint pc-presentation is a pc-presentation which is essentially disjoint as a family of pc-sets.


\begin{lem}
Suppose  $\{T_b\}_{ b \in Y}$  and  $\{T'_{b'}\}_{ b' \in Y'}$ are families of pc-sets. Then an arbitrary fiberwise intersection (respectively fiberwise product) of $\{T_b\}_{ b \in Y}$  and  $\{T'_{b'}\}_{ b' \in Y'}$ is also essentially disjoint.
\end{lem}
\begin{proof}
This follows from (10), (11) and (12) of Lemma~\ref{lies}.
\end{proof}

\noindent
Towards replacing pc-presentation with essentially disjoint pc-presentation in Proposition~\ref{pc-pres} we need the following auxiliary result:
\begin{comment}
\begin{lem} \label{DefClosure1}
If $T \subseteq K^n$ is definable over $c \in K^m$, then there are $l$ and $\beta \in F^l$ such that the closure $V$ of $T$ in $K$-Zariski topology is definable in the field sense over $ \chi(\beta),c$.
\end{lem}

\begin{proof}
Suppose $T, c, V$ are as in the statement. Clearly $V$ is definable in $K$. Also any field automorphism  $\sigma$ of $K$ which fixes $c$ and $\chi(F)$ pointwise can be extended to an automorphism  $\sigma$ of $(F, K;\chi)$. Hence, $\sigma(T)=T$ and so $\sigma(V) =V$. Therefore, $V$ is definable over $\chi(F), c$ in the field sense. By property of definability, we can choose finite $\beta \in F^l$ as desired.
\end{proof}
\end{comment}

\begin{lem} \label{DefinabilityOfClosure}
Suppose $\{T_b\}_{b \in Y}$ is a definable family of pc-subsets of $K^n$ and for each $b \in Y$, $V_b$ is the closure of $T_b$ in the $K$-topology. Then the family  $\{V_b\}_{b \in Y}$ is definable.

\end{lem}

\begin{proof}

Suppose $\{T_b\}_{b \in Y}$ and $\{V_b\}_{b \in Y}$ are as in the assumption of the lemma. Let $\mathscr{C}$ be a choice of systems $Q^{(1)}, \ldots, Q^{(l)}$  in $\qq[w,x]$, a system $P'$ in $\qq[w',x]$, a system $P''$ in $\qq[w'',x, z'']$, a parameter free $L_r$-formula $\phi''(u'',v'')$ with  variables in the first sort  and $|v''|=|z''|$. We note that there are countably many such choices $\mathscr{C}$. We define a relation $R_\mathscr{C} \subseteq K^n \times Y \times F^{|v''|} \times K^{|w|}\times K^{|w'|} \times K^{|w''|}$ as follows. For $a \in K^n$, $b \in Y$, $\gamma'' \in F^{|v''|}$, $c \in K^{|w|}$, $c' \in K^{|w'|}$, and $c'' \in K^{|w''|}$, $R_\mathscr{C}(a, b, \gamma'', c, c', c'')$ holds if and only if the following conditions hold:
\begin{enumerate}[(a)]
\item $T_b = V'_{c'} \backslash S''_{ \gamma'', c''}$ where $V'_{c'}$ is the zero set of $P'(c',x)$ and $S''_{\gamma'', c''}$ is the set defined by $\exists v''\big( \phi''(\gamma'', v'') \wedge P''(c'', x, \chi(v'')) =0\big)$;
\item $Z\big( Q^{(1)}(c, x)\big), \ldots, Z\big( Q^{(l)}(c, x)\big) $ are irreducible;
\item $\bigcup_{i=1}^l Z\big( Q^{(i)}(c, x)\big) $ is the closure of $V'_{c'} \backslash S''_{\gamma'', c''}$ in $K$-topology, where $V'_{c'}, S''_{\gamma'', c''}$ are the same as in (a);
\item $a$ is in  $ \bigcup_{i=1}^l Z\Big( Q^{(i)}(c, x)\Big) $.
\end{enumerate}
We note that (a), (d) are clearly definable, (b) is definable as irreducibility is definable in families in models of $\text{ACF}$, and under the condition that (b) holds, (c) is definable by Corollary~\ref{Smallness4}. Hence, $R_\mathscr{C}$ is a definable relation. Let $R^2_\mathscr{C} \subseteq Y$ be the projection of $R_\mathscr{C}$ on $Y$, $R^{1,2}_\mathscr{C}$ be the projection of $R_\mathscr{C}$ on $K^n \times Y$. Then $R^2_\mathscr{C}, R^{1,2}_\mathscr{C} $ are also definable. We also note that  if  $b \in R^2_\mathscr{C}$ then $V_b$ is precisely the set $\big\{ a \in K^n : (a, b) \in R^{1,2}_\mathscr{C} \big\}$.

For each $b \in Y$, there is a choice $\mathscr{C}$ as above such that $b \in R^2_{\mathscr{C}}$. By a standard compactness argument, there are finitely many choices $\mathscr{C}_1, \ldots, \mathscr{C}_k$ as above such that for any $b \in Y$, there is $i \in \{ 1, \ldots, k\} $ such that $b \in R^2_{\mathscr{C}_i}$. Then the family $\{V_b\}_{b \in Y}$ as a subset of $K^n \times Y$ consists of $(a, b)$ such that for some $i \in \{1, \ldots k\}$, $(a, b) \in R^{1,2}_{\mathscr{C}_i}$.  Thus $\{V_b\}_{b \in Y}$ is definable.
\end{proof}

\begin{cor} \label{Definability}
Suppose $\{T_b\}_{b \in Y}, \{T'_{b'}\}_{b' \in Y'}$ are definable families of pc-subsets of $K^n$. The following sets are definable:
\begin{enumerate}
\item for $k \in \nn$, the set $\big\{ b \in Y : \mr_K(T_b) = k \big\}$;
\item for $k,l \in \nn$, the set $\big\{ b \in Y : \mr_K(T_b) = k,  \md_K(T_b) = l \big\}$;
\item the set $ \big\{ (b,b') \in Y \times Y' : \mr_K(T_b) \leq \mr_K( T'_{b'}) \big\}$;
\item the sets obtained by replacing $\leq$ in (3) with $<, =$ or replacing $\mr_K$ with $\md_K$;
\item the set $ \big\{ (b,b') \in Y \times Y' : T_b \subsim T'_{b'} \big\}$;
\item the sets obtained by replacing $ \subsim$ with $\sim$ and $\simperp$ in (5). 
\end{enumerate}
\end{cor}

\begin{proof}
All the above statements have similar proof ideas, so we will only provide the proof of the first statement as an example. Suppose $\{T_b\}_{b \in Y}$ is as in the statement and let $\{V_b\}_{b \in Y}$ be as in the preceding lemma. Then $\big\{ b \in Y \mid \mr_K(T_b) = k \big\}$ is by definition the same as $\big\{ b \in Y \mid \mr_K(V_b) = k \big\}$. The desired conclusion follows from the preceding lemma and the fact that the Morley rank is definable in family in a model of ACF (see Lemma~\ref{DOE1}).
\end{proof}

\noindent
Suppose $\{T_\alpha\}_{\alpha \in D}$ is a pc-presentation. The {\bf primary index set} $\hat{D}$ of $\{T_\alpha\}_{\alpha \in D}$ consists of   $\alpha \in D$ such that $\mr_K(T_\alpha) = \max_{ \alpha \in D} \mr_K(T_\alpha)$.

\begin{prop} \label{essdispre}
Every definable set has an essentially disjoint pc-presentation.
\end{prop}
 
\begin{proof}
Let $ \{ T_\alpha\}_{ \alpha \in D}$ be a pc-presentation of $X$, $r= \max_{ \alpha \in D} \mr_K(T_\alpha)$ and  $\hat{D}$ be the primary index set of $ \{ T_\alpha\}_{ \alpha \in D}$. Let  $ \hat{D}_1 $ be the set $$\{ \alpha \in \hat{D}: \text{ for all } \beta \in \hat{D}, \text{either } T_\alpha \sim T_\beta \text{ or } T_\alpha \simperp T_\beta\} $$ and let $ \hat{D}_2 = D\backslash \hat{D}_1$. Set $l=0$ if $\hat{D}_2$ is empty and $l = \max_{ \alpha \in \hat{D}_2} \md_K(T_\alpha)$ otherwise. We note that $\big\{ \md_K(T_\alpha):  \alpha \in \hat{D}_2 \big\}$ has an upper bound by the preceding lemma and a standard compactness argument.

We now make a number of reductions. If $r = 0$, the statement of the proposition is immediate. Towards using induction,  assume $r>0$ and we have proven the statement for all $X'$ with a pc-presentation $ \{ {T'}_{\alpha'}\}_{ \alpha' \in D'}$ such that for similarly defined $r'$, we have $r'<r$. Note that $\hat{D}, \hat{D}_1$ and $\hat{D}_2$ are definable by the preceding lemma. Let $X^-$ and $\hat{X}$ be the definable subsets of $K^n$ given by:
$$X^-\ =\ \bigcup_{ \alpha \in D\backslash \hat{D}}T_\alpha\ \text{ and }\ \hat{X}\ =\ \bigcup_{ \alpha \in \hat{D}}T_\alpha. $$
By the induction assumption, $X^-$ has an essentially disjoint pc-presentation. If $l=0$, then $\{ T_\alpha\}_{ \alpha \in \hat{D}}$ is an essentially disjoint pc-presentation of $\hat{X}$; we can take the disjoint combination of the former and a pc-presentation of $X^-$ to get an essentially disjoint pc-presentation of $X$. Towards using induction, assume that $l>0$ and we have proven the statement for all $X'$ with a pc-presentation $ \{ {T'}_{\alpha'}\}_{ \alpha' \in D'}$ such that for similarly defined $l'$, we have $l'<l$.

Let $ R $ be the set of $ (\alpha, \beta) \in \hat{D}_2 \times \hat{D}_2$ such that neither $  T_\alpha\sim T_\beta $ nor $T_\alpha\simperp T_\beta$. By preceding corollary $R$ is definable.
Let $ \{ T^{(1)}_{\alpha, \beta}\}_{ (\alpha, \beta) \in R}$ and  $ \{ T^{(2)}_{\alpha, \beta}\}_{ (\alpha, \beta) \in R}$ be  definable families given by 
$$T^{(1)}_{\alpha, \beta}\ =\ T_\alpha \capdot T_\beta\ \text{ and }\ T^{(2)}_{\alpha, \beta}\ =\ T_\alpha \dotminus T_\beta.$$
The above two families are definable families of pc-sets.
By the assumption of the preceding paragraph, for all $\alpha \in D_2$, there is $\beta \in D_2$, such that $( \alpha,\beta)$ is in $R$. For such $\beta$, 
$$T_\alpha\ =\ T^{(1)}_{\alpha, \beta} \cup T^{(2)}_{\alpha, \beta}.$$
Hence, taking the disjoint combination of $ \{ T_\alpha\}_{ \alpha \in \hat{D}_1}$, $ \{ T_\alpha\}_{ \alpha \in {D\backslash \hat{D}}}$, $ \{ T^{(1)}_{\alpha, \beta}\}_{ (\alpha, \beta) \in R}$, $ \{ T^{(2)}_{\alpha, \beta}\}_{ (\alpha, \beta) \in R}$, we get a new pc-presentation $ \{ T'_{\alpha'}\}_{ \alpha' \in D'}$ of $X$. We note that by (6), (7), (8), (9) of Lemma~\ref{lies}, for every $(\alpha, \beta) \in R$ and $i \in \{1,2\}$ 
$$\text{either }\ \mr_K(T^{(i)}_{\alpha, \beta})\ <\ \max_{ \beta \in \hat{D}_2} \mr_K(T_\beta) \ \text{ or } \ \md_K(T^{(i)}_{\alpha, \beta})\ <\ \max_{ \beta \in \hat{D}_2} \md_K(T_\beta).$$
Hence, for $l'$ defined for $\{ T'_{\alpha'}\}_{ \alpha' \in D'}$ in the same way as in the preceding paragraph, we have $l'<l$. The conclusion follows by the induction assumption from the preceding paragraph.
\end{proof}

\noindent
Let $\{T_\alpha\}_{\alpha \in D}$ be an essentially disjoint pc-presentation and $\hat{D}$ be its primary index set. A {\bf primary index quotient} of $\{T_\alpha\}_{\alpha \in D}$ consists of a definable subset $\widetilde{D}$ of $F^l$ for some $l$ and a map $\pi: \hat{D} \to \widetilde{D}, \alpha \mapsto \tilde{\alpha}$ such that $\tilde{\alpha}= \tilde{\beta}$ if and only if $T_\alpha \sim T_\beta$. However, we will systematically abuse notation calling $\widetilde{D}$ a primary index quotient of $T_\alpha$ and regarding the map $\pi$ as implicitly given. Moreover, we will write $\tilde{\alpha} \in \widetilde{D}$ as an abbreviation for $\alpha$ is an element of $\hat{D}$ and $\tilde{\alpha}$ is the image of $\alpha$ under $\pi$.
We note that we can always find a primary index quotient of an essentially disjoint pc-presentation by Lemma~\ref{Definability}, Lemma~\ref{QuanRed} and the fact that \text{ACF} has elimination of imaginaries. 
For a property (P) of pc-sets which is preserved under the equivalent relation $\sim$,
let $\widetilde{D}_{\text{(P)}}$ 
consist of $\tilde{\alpha} \in \widetilde{D}$ such that $T_\alpha$ satisfies (P).
With (P) as above, we say (P) holds {\bf for most} $\tilde{\alpha} \in \widetilde{D}$ if $\mr_F(\widetilde{D} \backslash \widetilde{D}_{\text{(P)}}) < \mr_F( \widetilde{D})$.


The next proposition allows us to define another geometrical invariant of a definable set based on a choice of its essentially disjoint pc-presentation and yet independent of such choice.

\begin{prop} \label{Fgeorank}
Suppose $X \subseteq X' \subseteq K^n$, $X$ has an essentially disjoint pc-presentation $\{ T_\alpha\}_{ \alpha \in D}$ with  a primary index quotient $\widetilde{D}$ and $X'$ has an essentially disjoint pc-presentation $\{ T'_{\alpha'}\}_{ \alpha' \in D'}$ with a primary index quotient $\widetilde{D}'$. Then either $\gr_K(X)< \gr_K(X')$ or $\gr_K(X)= \gr_K(X')$ and $\mr_F(\widetilde{D}) \leq \mr_F(\widetilde{D}')$.
\end{prop}

\begin{proof}
Suppose $X, X'$, $\{ T_\alpha\}_{ \alpha \in D}$, $\{ T'_{\alpha'}\}_{ \alpha' \in D'}$, $\widetilde{D}$ and $\widetilde{D}'$ are as given.
By Proposition~\ref{Kgeorank}, we have $\gr_K(X) \leq \gr_K(X')$, so suppose $\gr_K(X) = \gr_K(X')$.
Let  $\displaystyle R \subseteq \widetilde{D} \times \widetilde{D}'$ be the relation consisting of $(\tilde{\alpha}, \tilde{\alpha}') \in R$  such that
$$  \mr_K( T_{\alpha} \cap T'_{\alpha'})\ =\ \gr_K(X)\ =\ \gr_K(X')\ \text{ for } \alpha \in \pi^{-1}(\tilde{\alpha}) \text{ and } \alpha' \in \pi'{}^{-1}(\tilde{\alpha}'). $$ By (10) of Proposition~\ref{lies}, $R$ is well defined. By (1) of Corollary~\ref{Definability}  and Theorem~\ref{StablyEmbbed}, $R$ is definable in $F$.
Since $ X\subseteq X'$,  for every $ \tilde{\alpha}$ there is at least one $ \tilde{\alpha}' $ such that $  \tilde{\alpha} R \tilde{\alpha}' $.
Since the pc-presentation of $X$ is essentially disjoint for each $  \tilde{\alpha}'$, there is at most finitely many $ \tilde{\alpha}$ such that $ \tilde{\alpha} R \tilde{\alpha}'$.
Hence, $ \mr_F (\widetilde{D}) \leq  \mr_F (\widetilde{D}') $, and so the conclusion follows.
\end{proof}

\begin{cor}
If $X \subseteq K^n$ has essentially disjoint pc-presentations $\{ T_\alpha\}_{ \alpha \in D}$ and $\{ T'_{\alpha'}\}_{ \alpha' \in D'}$ with respective primary index quotients $(\widetilde{D}, \pi)$ and  $(\widetilde{D}', \pi')$, then $\mr_F(\widetilde{D})= \mr_F(\widetilde{D}')$.
\end{cor}


\noindent
Suppose $X \subseteq K^n$ has an essentially disjoint pc-presentation $\{ T_\alpha\}_{ \alpha \in D}$ with a primary index quotient $(\widetilde{D}, \pi)$. The { \bf $F$-geometric rank} of $X$, denoted by $\gr_F(X)$, is defined by $\mr_F(\widetilde{D})$. The {\bf geometric rank} of $X$, denoted by $\gr(X)$ is defined by $\omega\cdot \gr_K(X)+ \gr_F(X)$.





\begin{cor} \label{grbehavior2}
Suppose $X, X' \subseteq K^n $ are definable. Then we have the following:
\begin{enumerate}
\item if $X \subseteq X'$, then $\gr(X) \leq \gr(X')$;
\item $\gr(X \cup X') = \max\{\gr(X), \gr(X') \}$.
\end{enumerate}

\end{cor}

\begin{proof}
(1) follows from Propositions~\ref{Kgeorank} and~\ref{Fgeorank}. For (2), suppose $X, X'$ are as given. We can reduce to the case where $X, X'$ are disjoint, and $\gr(X) \geq \gr(X')$. Choose essentially disjoint pc-presentations of $X,X'$ and take the disjoint combination to get a pc-presentation of $X \cup X'$. As $X$ and $X'$ are disjoint, the obtained pc-presentation remains essentially disjoint. Calculating geometric rank yields the desired result.
\end{proof}

%%End of part 3
\noindent
We finally introduce the ultimate description of definable set in a model of $\TN_p$. Let $P$ be a system of polynomials in $ K[w, x]$ and $T$ is a subset of $K^n$. We say $P$ {\bf divides $T$} if there is some $\beta \in F^{|w|}$  such that for $W = Z\big(P( \chi(\beta),x)\big)$, we have $\mr_K(T \cap W) = \mr_K(T \backslash W)$. We note that if $T' \sim T$, and $P$ as above divides $T$ then $P$ also divides $T'$.
\begin{comment}
We say $T$ is $K$-indivisible if no polynomial $P \in K[w, x]$ divides $T$. It follows that if $V$ is the closure of $T$ in the $K$-topology, then $T$ is $K$-indivisible if and only if $V$ has one single irreducible component of dimension $\mr_K(T)$ in the $K$-topology
\end{comwith ment}








For a system $P$ of polynomials in $K[w, x]$ with $w \in \vartu$ and $x \in \vart{n}$, let $\widetilde{D}_P$ be the set of $\tilde{\alpha} \in \widetilde{D} $ such that for an arbitrary representative $\alpha$ of $\tilde{\alpha}$, $P$ divides $T_\alpha$. The pc-presentation $ \{ T_\alpha\}_{ \alpha \in D}$ is {\bf geometric} if for an arbitrary $P$ of polynomials as above, we have $\mr_F( \widetilde{D}_P)< \mr_F( \widetilde{D})$. It is easy to see that the above definition does not change if we add the assumption that $P$ consists of a single polynomial. 

Keep the notation as in the previous paragraph and suppose further that $D$ is finite, $X =\bigcup_{\alpha \in D} T_\alpha$ and for each $\alpha \in D$, $T_\alpha$ is algebraic. Then  $ \{ T_\alpha\}_{ \alpha \in D}$ is geometric if and only if $ \{ T_\alpha\}_{ \alpha \in D}$ is a decomposition of $X$ into algebraic sets such that for all $\alpha \in D$, $\mr_K(T_\alpha) =\mr_K(X)$ implies $\md_K(T_\alpha)=1$. Hence, the notion of geometric pc-presentation of a definable set is almost a generalization of the notion of decomposition of a constructible set into quasi-affine varieties. We could have refined the notion of geometric presentation so that we can delete the word ``almost'' in the preceding statement. However, we leave this for the interested reader as this will require much more effort and not needed in the subsequent developments.
\end{comment}
Let $ \{ T_\alpha\}_{ \alpha \in D}$ be an essentially disjoint pc-presentation. 
We say $P$ {\bf essentially divides} $ \{ T_\alpha\}_{ \alpha \in D}$ if  
$$\mr_F( \widetilde{D}_P)\ =\ \mr_F( \widetilde{D}) \ \text{ where }\ \widetilde{D}_P\ =\ \{ \tilde{\alpha} \in  \widetilde{D} : P \text{ divides } T_\alpha  \}  .$$ 
We note that if a system $P$ essentially divides $ \{ T_\alpha\}_{ \alpha \in D}$ then some polynomial in the system already essentially divides $ \{ T_\alpha\}_{ \alpha \in D}$. 


We say $ \{ T_\alpha\}_{ \alpha \in D}$ is  a {\bf geometric pc-presentation} if  $P$ does not essentially divide $ \{ T_\alpha\}_{ \alpha \in D}$ for any $P \in K[w, x]$.



\begin{prop} \label{geoired}
Suppose $X \subseteq K^n$ is definable. The following are equivalent:
\begin{enumerate}
\item if $X' \subseteq X$ with $\gr(X')= \gr(X)$, then $\gr(X \backslash X') < \gr(X)$;
\item $X$ has a geometric pc-presentation  $ \{ T_\alpha\}_{ \alpha \in D}$ with a primary index quotient $\widetilde{D}$ satisfying $\md_F(\widetilde{D}) =1$.
\end{enumerate}
\end{prop}
\begin{proof}
Towards the proof of (1) implying (2), we make the observation that for $X' \subseteq X$ with $\gr(X \backslash X') < \gr(X)$, if (2) holds for $X'$ then (2) holds for $X$. Suppose $\{ {T'}_{\alpha'}\}_{ \alpha' \in D'}$ is a pc-presentation of $X'$ as described in (2). Choose an arbitrary essentially disjoint pc-presentation of $X \backslash X'$ and take a disjoint combination of it and $\{ {T'}_{\alpha'}\}_{ \alpha' \in D'}$ to get an essentially disjoint pc-presentation $ \{ T_\alpha\}_{ \alpha \in D}$ of $X$. The assumption that $\gr(X \backslash X') < \gr(X)$ implies that $ \{ T_\alpha\}_{ \alpha \in D}$ is still a presentation as described in (2).

We also make a few more observations and reductions. Suppose $X \subseteq K^n$ has the property (1) and $\{ T_\alpha\}_{ \alpha \in D}$ is an essentially disjoint pc-presentation of $X$ with $\hat{D}$ the primary index set and $\widetilde{D}$ a primary index quotient. It follows from (1) that $\md_F(\widetilde{D}) =1$. 
Let $l$ be the natural number such that 
$$\md_K( T_\alpha)\ =\ l\ \text{ for most } \tilde{\alpha} \in \widetilde{D}\textbf{} .$$
Suppose $l=1$, then $\{ T_\alpha\}_{ \alpha \in D}$ is geometric with the desired properties and we are done. Towards using induction, assume we have proven (2) for all $X'$ satisfying (1) with an essentially disjoint presentation  $\{ T'_{\alpha'}\}_{ \alpha' \in D'}$ such that for similarly defined $l'$, we have $l'<l$. Suppose  $\{ T_\alpha\}_{ \alpha \in D}$ is not geometric as otherwise we are done.
Then there is a polynomial $P \in K[w,x]$ such that 
$$\mr_F(\widetilde{D}_P)\ =\ \mr_F(\widetilde{D})\ \text{ where }\ \widetilde{D}_P\ =\ \{ \tilde{\alpha} \in \widetilde{D} \mid P  \text{ divides } T_\alpha \} .$$
By the observation in the preceding paragraph, we can arrange to have 
$$D\ =\ \hat{D},\  \widetilde{D}\ =\ \widetilde{D}_P\ \text{ and  for some } l,\ \md_K( T_\alpha)\ =\ l\ \text{ for all }  \tilde{\alpha} \in \widetilde{D}. $$
We prove that (1) implies (2). For each $\beta \in F^k$, let $W_\beta$ be $Z\big( P(\chi(\beta),x)\big)$. Let $D' $ be the set consisting of $(\alpha, \beta) \in D \times F^{k}$ such that  
$\mr_K( T_\alpha \cap W_\beta) = \mr_K(T_\alpha \backslash W_\beta)$. Let $X'$ be given by the pc-presentation $ \{ T_\alpha \cap W_\beta \}_{ (\alpha, \beta) \in D'}$ and $X''$ be given by the pc-presentation $ \{ T_\alpha \backslash W_\beta \}_{ (\alpha, \beta) \in D'}$. 
Then $$X\ =\ X'\cup X'',\ \text{and so }\ \gr(X)\ =\ \max\big\{\gr(X'), \gr(X'')\big\}.$$ We assume $\gr(X) = \gr(X')$; the other case can be dealt with similarly.  We note that for all $(\alpha, \beta) \in D'$, $\md_K(T_\alpha \cap W_\beta )<l$. The pc-presentation $ \{ T_\alpha \cap W_\beta \}_{ (\alpha, \beta) \in D'}$ might not be essentially disjoint, but by applying the same procedure as in Proposition~\ref{essdispre} we can produce an essentially disjoint $\{ T'_{\alpha'}\}_{ \alpha' \in D'}$. By construction, for similarly defined $l'$, we have $l'<l$. Therefore, by the assumption in the preceding paragraph $X'$ satisfies (2). By $(1)$, we have $\gr(X \backslash X')<\gr$ and so by the observation in the first paragraph $X$ also satisfies (2).

Next, we make some preparation for the proof of $(2)$ implies $(1)$. 
Towards a contradiction, suppose $X \subseteq K^n$ has a pc-presentation $\{ T_\alpha\}_{ \alpha \in D}$ with a primary index quotient $\widetilde{D}$ as in $(2)$ but there is $X'\subseteq X$ such that $\gr(X')= \gr(X \backslash X') = \gr(X)$.
Replacing $X$ by $X_0 \subseteq X$ and $\gr(X \backslash X_0)< \gr(X)$ if needed, we can arrange that  $$\mr_K(T_\alpha)\ =\ \gr_K(X)\ \text{ for all } \alpha \in D.$$ 
Choose an essentially disjoint pc-presentation $\{ {T'}_{\alpha'}\}_{ \alpha' \in D'}$ of $X'$. As usual, let $\widetilde{D}'$ be a primary index quotient of  $\{ {T'}_{\alpha'}\}_{ \alpha' \in D'}$. 
Replacing this pc-presentation by its fiber-wise intersection with $\{ T_\alpha\}_{ \alpha \in D}$ if needed,  we can arrange that for each $\alpha' \in D'$, there is $\alpha \in D$ such that ${T'}_{\alpha'} \subseteq T_\alpha$.
For each $\alpha' \in D'$, let $V'_{\alpha'}$ be the closure of ${T'}_{\alpha'}$ in the $K$-topology.
By Lemma~\ref{DefinabilityOfClosure}, the family $\{ V'_{\alpha'} \}_{\alpha' \in D'}$ is definable. By Lemma~\ref{APsimple2}, there are finitely many systems of polynomials $P'_1, \ldots, P'_k$ in $K[w', x]$ such that for all $\alpha' \in D'$, 
$$ V'_{\alpha'}\ =\ Z\Big(P_i'\big(\chi(\beta'),x\big)\Big)\ \text{ for some } \beta' \in F^{|w'|} \text{ and } i \in \{1, \ldots, k\}.$$ 
By shrinking $X'$ if needed, we can assume that $\md_F(\widetilde{D}') =1$. 
By shrinking $X'$ further, we can assume there is  a system $P'$ in $K[w', x]$ such that for any $\alpha' \in D'$, there is $\beta'$ in $F^{|w'|}$ with $ V'_{\alpha'} =Z\big(P'(\chi(\beta'),x)\big)$. 
Let $\{ {T''}_{\alpha''}\}_{ \alpha'' \in D''}$ be an essentially disjoint pc-presentation of $X \backslash X'$. Replacing this pc-presentation by its fiber-wise intersection with $\{ T_\alpha\}_{ \alpha \in D}$ if needed,  we arrange that  for all $\alpha'' \in D''$, there is $\alpha \in D$ such that ${T''}_{\alpha''} \subseteq T_\alpha$. 


We continue the proof of $(2)$ implies $(1)$ with the notations as in the preceding paragraph. 
Set
$$\widetilde{D}_1\  =\ \{ \tilde{\alpha} \in \widetilde{D} : \text{ there is } \tilde{\alpha}' \in \widetilde{D}' \text{ such that } T'_{\alpha'}\subsim T_\alpha \}\ \text{ and }\ X_1\ =\ \bigcup_{\tilde{\alpha} \in \widetilde{D}_1} T_\alpha. $$
$$\widetilde{D}_2\  =\ \{ \tilde{\alpha} \in \widetilde{D} : \text{ there is } \tilde{\alpha}'' \in \widetilde{D}'' \text{ such that } T''_{\alpha''}\subsim T_\alpha \}\ \text{ and }\ X_2\ =\ \bigcup_{\tilde{\alpha} \in \widetilde{D}_2} T_\alpha. $$
By the arrangement in the preceding paragraph, for each $\tilde{\alpha}' \in \widetilde{D}'$, there is $\tilde{\alpha} \in \widetilde{D}$ such that ${T'}_{\alpha'} \subsim T_\alpha$, so $\gr( X' \backslash X_1) < \gr(X)$. 
 As $X_1 \subseteq X $, $\gr(X_1) \leq \gr(X)$. By assumption, $\gr(X) =\gr(X')$. By putting the last three statements together and using Corollary~\ref{grbehavior2}, we have 
$$\gr( X')\ =\ \gr(X_1)\ =\ \gr(X)\ \text{ and so }\ \mr_F(\widetilde{D}_1)\ =\ \mr_F(\widetilde{D}')\ =\ \mr_F(\widetilde{D}).$$
Similarly, we get $\mr_F(\widetilde{D}_2) =\mr_F(\widetilde{D}'')=\mr_F(\widetilde{D}) $.
As  $\md_F(\widetilde{D})=1$, $\mr_F(\widetilde{D}_1 \cap \widetilde{D}_2) = \mr_F(\widetilde{D}) $. 
We will now show that 
$$ P' \text{ divides }T_\alpha \text{ for all } \tilde{\alpha} \in \widetilde{D}_1 \cap \widetilde{D}_2. $$
This is the desired contradiction as $P'$ will then essentially divide $\{ T_\alpha\}_{ \alpha \in D}$. Suppose  $\tilde{\alpha} $ is in $\widetilde{D}_1 \cap \widetilde{D}_2$.  
Then there are $\tilde{\alpha}' \in \tilde{D}'$ and $\tilde{\alpha}'' \in \tilde{D}''$ such that ${T'}_{\alpha'} \subsim T_\alpha$ and ${T''}_{\alpha''} \subsim T_\alpha$.
 By preceding paragraph, there is $\beta' \in F^{|w'|}$, such that the closure $V'_{\alpha'}$  of $T'_{\alpha'}$ in $K$-topology is $Z\big(P'(\chi(\beta'),x)\big)$. 
Then $$\mr_K(T_\alpha \cap V'_{\alpha'})\ \geq\ \mr_K({T'}_{\alpha'})\ =\ \mr_K(T_{\alpha}),\ \text{ and so }\ \mr_K(T_\alpha \cap V'_{\alpha'})\ =\ \mr_K(T_{\alpha}).$$ 
On the other hand, by $(10)$ of Proposition~\ref{lies} and the fact that ${T'}_{\alpha'} \cap {T''}_{\alpha''}$ is empty, $\mr_K({T''}_{\alpha''} \cap V'_{\alpha'})< \mr_K(T_{\alpha})$. By $(7)$ of Proposition~\ref{lies}, $\mr_K({T''}_{\alpha''} \backslash V'_{\alpha'}) = \mr_K(T_{\alpha})$. 
Therefore, $$\mr_K(T_\alpha \backslash V'_{\alpha'})\ \geq\ \mr_K({T''}_{\alpha''} \backslash V'_{\alpha'})\ =\ \mr_K(T_{\alpha}),\ \text{ and so }\ \mr_K({T}_{\alpha} \backslash V'_{\alpha'})\ =\ \mr_K( T_\alpha).$$
Thus,  $\mr_K(T_\alpha \cap V'_{\alpha'}) = \mr_K({T}_{\alpha} \backslash V'_{\alpha'})$ and $P'$ divides $T_\alpha$ as desired.
\end{proof}

\noindent
If $X \subseteq K^n$ is definable and satisfies one of the two statements of Proposition~\ref{geoired}, we say $X$ is {\bf geometrically irreducible}. 

\begin{lem} \label{irredlemma}
Suppose $X, X', X'' \subseteq K^n$ are definable. Then
\begin{enumerate}
\item if $X$ is geometrically irreducible and $\gr(X')< \gr(X)$, then $X \cup X'$ is geometrically irreducible;
\item if $X$ is geometrically irreducible and $\gr(X')< \gr(X)$, then $X \backslash X'$ is geometrically irreducible;
\item if $X, X', X''$ are geometrically irreducible, $\gr(X \cap X')=\gr(X)=\gr(X')$ and $\gr(X'\cap X'') = \gr(X') = \gr(X'')$, then $\gr(X \cap X'') = \gr(X) =\gr(X'')$.
\end{enumerate}
\end{lem}
\begin{proof}
We have that $(1)$ follows from condition $(2)$ of Proposition~\ref{geoired} and $(2)$, $(3)$ follow from condition $(1)$ of Proposition~\ref{geoired}.
\end{proof}

\begin{comment}
\noindent
Suppose $\{T_\alpha \}_{\alpha \in D}$ and $\{T'_{\alpha'} \}_{\alpha' \in D'}$ are two essentially disjoint pc-presentations with respective primary index quotients $\widetilde{D}$ and $\widetilde{D}'$. We say $\{T_\alpha \}_{\alpha \in D}$ { \bf essentially contains} $\{T'_{\alpha'} \}_{\alpha' \in D'}$ if for most $\tilde{\alpha}' \in \widetilde{D}'$ there is $\tilde{\alpha} \in \widetilde{D}$ such that $T_\alpha \sim T'_{\alpha'}$. We say that $\{T_\alpha \}_{\alpha \in D}$ and $\{T'_{\alpha'} \}_{\alpha' \in D'}$ are { \bf essentially equivalent } if each of them essentially contains the other. 
\end{comment}





\begin{thm}
Every definable $X\subseteq K^n$ has a geometric pc-presentation. 
%Moreover, any two geometric pc-presentations of $X$ are essentially equivalent.
\end{thm}

\begin{proof}
We first prove an auxiliary result. Suppose $X$ has an essentially disjoint pc-presentation $\{ T_\alpha\}_{ \alpha \in D}$  with a primary index  quotient $\widetilde{D}$ satisfying $\md_F( \widetilde{D}) =1$ and there is $l>0$ such that $$\mr_K(T_\alpha)\ =\ \gr_K(X)\ \text{ and }\ \md_K(T_\alpha)\ =\ l\ \text{ for all } \tilde{\alpha} \in \widetilde{D}. $$
We will show that $X$ can be written as a disjoint union of at most $l$  definable sets, all with geometrical rank $\gr(X)$. Suppose towards a contradiction that $X$ can be written as a disjoint union of $ X_1, \ldots, X_{l+1}$, all with of geometrical rank $\gr(X)$. For each $i \in \{1, \ldots, l+1\}$, let $\{ T_{i,\beta}\}_{ \beta \in E_i}$ be an essentially disjoint pc-presentation of $X_i$, set 
$$D_i\ =\ \big\{ \alpha \in D : \text{ there is } \beta \in E_i \text{ with  }\mr_K( T_\alpha \cap T_{i,\beta}) = \gr_K(X) \big\}$$  and let $\widetilde{D}_i$ be a primary index quotient of $\{ T_\alpha\}_{ \alpha \in D_i}$. As $\gr(X_i) = \gr(X)$, we have $\mr_F(\widetilde{D}_i)=\mr_F(\widetilde{D})$ for $i \in \{1, \ldots, l+1\}$. Hence, we can find $\alpha $ in $\bigcap_{i =1}^{l+1} D_i$. Then $\md_K(T_\alpha)> l$, a contradiction.

Suppose $X \subseteq K^n$ is definable. We will next show there is a finite collection of geometrically irreducible definable sets $\{X_i\}_{i \in I}$ each with $\gr(X_i)=\gr(X)$ such that $X = \bigcup_{i \in I} X_i$.
If $X = X' \cup X''$ such that $\gr(X')=\gr(X'') =\gr(X)$, and we have proven the statement for $X', X''$, the statement for $X$ also follows.
Therefore, we can arrange that $X$ has an essentially disjoint pc-presentation $\{ T_\alpha\}_{ \alpha \in D}$ where $\md_F( \widetilde{D}) =1$. 
By possibly removing a set of geometrical rank smaller than $\gr(X)$ and using $(1)$ of the preceding lemma, we can also arrange that there is $l>0$ such that for all $\alpha \in D$,  $\mr_K(T_\alpha) =\gr_K(X)$ and $\md_K(T_\alpha)=l$. By the preceding paragraph, $X$ can be written as a disjoint union of at most $l$  definable sets, all of geometrical rank $\gr(X)$. Let $X$ be written as a disjoint union of the most number of definable sets, all of geometrical rank $\gr(X)$. Then it can be easily seen that each of these must be geometrically irreducible.


We now prove the proposition. Suppose $X = \bigcup_{i\in I} X_i$ is as in the previous paragraph. For $i, j \in I$, we write $X_i \approx X_j$ if $\gr(X_i \cap X_j) =\gr(X)$. By $(2)$ of the preceding lemma, $\approx$ is an equivalence relation on $\{X_i\}_{i \in I}$. We can choose $J \subseteq I$ such that $\{X_j\}_{j \in J}$ are representatives of the equivalent classes. Then using Corollary~\ref{grbehavior2} and characterization $(1)$ of geometric irreducibility, it is easy to see that $X =X' \cup \left( \bigcup_{j \in J} X_j\right)$ such that $\gr(X')<\gr(X)$. Using $(1)$ of the preceding lemma, we can reduce to the case where $X =\bigcup_{j \in J} X_j$. By $(2)$ of the preceding lemma, we can further reduce to the case where $X_i, X_j$ are disjoint for distinct $i, j \in J$. For all $i \in J$, $X_i$ has a geometric pc-presentation.  Then $X$ has a geometric  pc-presentation which is a combination of the geometric pc-presentations of $X_i$.
\end{proof}

\noindent With the next proposition, we get the final geometric invariant of a definable set which is based on a choice of its geometric presentation but independent of such choice.

\begin{prop}
There is a unique $d$ such that $X$ is a disjoint union of $d$ geometrically irreducible sets of geometric rank $\gr(X)$. Moreover, if $\{ T_\alpha\}_{ \alpha \in D}$ is a geometric pc-presentations of $X \subseteq K^n$ then $\md_F(\widetilde{D})=d$.
\end{prop}
\begin{proof}
Suppose $\{ T_\alpha\}_{ \alpha \in D}$ is a geometric pc-presentation of $X$ with $\hat{D}$ its primary index set, $\widetilde{D}$ its primary index quotient and $\md_F(\widetilde{D})=d$. We will show that $X$  is a disjoint union of $d$ geometrically irreducible sets of geometric rank $\gr(X)$.
Let $\widetilde{D}_1, \ldots, \widetilde{D}_d$ be the disjoint subsets of $\widetilde{D}$ such that $\widetilde{D} = \bigcup_{i=1}^d \widetilde{D}_i$ and  $\mr_F(\widetilde{D}_i)=1$  for each $i \in \{1, \ldots,d\}$. Let $\hat{D}_1, \ldots, \hat{D}_d \subseteq \hat{D}$ be the corresponding inverse images under the canonical map $\hat{D} \to \widetilde{D}$.
Set $$X_1\ =\ \bigcup_{\alpha \in \hat{D}_1 \cup D\backslash \hat{D}} T_\alpha\ \text{ and }\ X_i\ =\ \bigcup_{\alpha \in D_i} T_\alpha \  \text{ for }\ i \in \{2, \ldots, d \}.$$ Since $\{ T_\alpha\}_{ \alpha \in D}$ is geometric, each $X_i$ is geometrically irreducible of geometrical rank $\gr(X)$.

It remains to show the uniqueness part of the first statement of the proposition. Suppose the above $X$ is the disjoint union of geometrically irreducible sets $X'_1, \ldots, X'_{d'}$, all of geometrical rank $\gr(X)$. Then for every $i \in \{1, \ldots, d\}$, there is a unique $j \in \{1, \ldots, d'\}$ such that $\gr(X_i \cap X'_{j}) = \gr(X_i)$ . Conversely, for each $j \in \{1, \ldots, d'\} $, there is a unique $i \in \{1, \ldots, d\}$ such that $\gr(X_i \cap X'_{j}) = \gr(X_i) $. The conclusion follows.
\end{proof}

\noindent
For each $X \subseteq K^n$, we call the number $d$ as in the above proposition the {\bf geometric degree} of $X$ and denote this by $\gd(X)$. The following properties of this notion are immediate from the preceding proposition.

\begin{cor} \label{gdbehavior}
Suppose $X, X'$ are definable and $\gr(X)=\gr(X')$. We have the following:
\begin{enumerate}
\item if $X \subseteq X'$, then $\gd(X) \leq \gd(X')$;
\item if $\gr(X \cap X')< \gr(X)$, then $\gd(X \cup X') =\gd(X) +\gd(X')$;
\item if $\gr(X \cap X')= \gr(X)$, then $ \gd(X \cup X') =\gd(X) +\gd(X') - \gd(X \cap X')$.
\end{enumerate}
\end{cor}

