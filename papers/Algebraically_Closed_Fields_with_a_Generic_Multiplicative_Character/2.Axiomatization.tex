\section{Axiomatization}

\noindent Throughout, let $m, n$ ranges over the set of natural numbers (which includes zero), $p$ be either a prime number or zero, $t =(t_1, \ldots, t_n)$, $u =(u_1, \ldots, u_m)$ be tuples of variables of the first sort and $x= (x_1, \ldots, x_n)$, $y = (y_1, \ldots, y_m)$ be tuples of variables of the second sort. If $a$ is in $X^n$, then $a = (a_1, \ldots, a_n)$ with $a_i \in X$ for $i \in \{1, \ldots n\}$. If $A \subseteq K^\neq := K \setminus \{ 0 \}$, set $\la A \ra_K $ to be the set of elements in $K^\neq$ which are in the subgroup generated by $A$ in the fraction field of $K$; when the context is clear, we will write $\la A \ra $ instead of $\la A \ra_K $. We note that in the preceding definition, $\la A \ra $  is a submonoid of $K^\neq$ and is a subgroup of $K^\times$ if $K$ is a field. If $P_1, \ldots, P_m$ are systems of polynomial in $K[x]$, we let $Z(P_1, \ldots, P_m) \subseteq K^n$ be the set of their common zeros.

In this section, we also assume that $A, B \subseteq K^\neq$ and $C \subseteq K$. Let \( \acl_C(A) \) denote the elements of $K$ satisfying a nontrivial polynomial equation with coefficients in $\zz[A,C]$. We will give a definition of the notion of genericity which is slightly more general than what was given in the introduction.
This is necessary  for the purpose of axiomatization and will also play an important role in the next two sections.

%-------------- 


The \textbf{multiplicative closure} of $A$ \textbf{over} $B$, denoted by $\mcl_B(A)$, is the set 
$$ \big\{ a \in K^\neq : a^n \in \la A \cup B \ra \text{ for some } n \big\}. $$ 
%
%We say  $A$ is \textbf{multiplicatively closed over} $B$ if $\mcl_B(A) = A$. It can be checked that \((K^\neq , \mcl_B)  \)  is a pregeometry.
%
We note that if $K$ is a field, the notion of multiplicative closure over $B$ coincides with the notion of divisible closure over $B$, viewing \(K^\times\) as a \( \zz\)-module. 
%
We say \(A\) is {\bf multiplicatively independent over} \(B\) if 
$$ a \notin \mcl_B \big(A \backslash \{a\}\big) \text{ for all } a \in A. $$
%
A {\bf multiplicative basis} of \( A\) \textbf{over} \(B\) is an \(A'\subseteq A\) such that $A'$ is multiplicatively independent over \(B\) and \( A \subseteq \mcl_B(A') \).
%
General facts about pregeometry give us that
%
there is a multiplicative basis of \(A\) over \(B\); furthermore, any two such bases have the same cardinality.
%
When $B =\emptyset$, we omit the phrase {\it over $B$} in the definition and the subscript $B$ in the notation. We also note that $\mcl( \emptyset)= \{ 1\} $.
%
\newpage

\noindent We say \( A \) is {\bf generic} if for all multiplicatively independent $a \in \la A \ra^n$ we also have $a$ is algebraically independent.
%------------
We say $A$ is \(C\)-\textbf{generic over} \(B\) if  for all $B$-multiplicatively independent $a \in \la A \ra^n$ we also have $a$ is algebraically independent over $B \cup C$. The following follows easily from the exchange property of $\text{mcl}$:

%----------------------







\begin{lem} \label{GenericRobust}
Suppose $A$ is $C$-generic over $B$. Then the preceding statement continues to hold as we:
\begin{enumerate}
\item replace $K$ by an integral domain $K'$ such that $A, B \subseteq {K'}^\times$ and $C \subseteq K'$,
\item replace $B$ with $B' \subseteq K^\neq$ such that $\mcl(B') = \mcl(B)$,
\item replace $C$ with $C' \subseteq K$ such that $\acl_B(C') =\acl_B(C)$,
\item replace $A$ with $A'\subseteq K^\neq$ such that $\mcl_B(A') = \mcl_B(A)$.
\end{enumerate}
\end{lem}
\begin{comment}
\begin{proof}
Suppose $A$ is $C$-generic over $B$. Then (1), (2), (3) are clear from the definition. For (4), we can arrange that $K$ is a field, $A$ and $B$ are groups, and $A' =  \mcl_B(A)$. 
%
Suppose $a' \in (A')^n$ is multiplicatively independent over $B$. 
%
As $A' =  \mcl_B(A)$, for some $k$ we have $(a')^k$ is in $\la A \cup B \ra = AB$. 
%
Hence, $(a')^k = (a_1b_1, \ldots, a_n b_n)$. By exchange property of $\mcl_B$, $(a_1, \ldots, a_n)$ remains multiplicatively independent over $B$, and hence algebraically independent over $B\cup C$.
%
By exchange property of $\acl_{B\cup C}$, $(a')^k = (a_1b_1, \ldots, a_nb_n)$ remains algebraically independent over $B \cup C$.
%
Therefore, $a'$ is also algebraically independent over $B\cup C$, which is the desired conclusion.
\end{proof}
\end{comment}


\begin{cor} \label{Equiv1}
The following equivalence holds: \(A\) is \(C\)-generic over \(B\) if and only if there is a family \( A' \subseteq A\) such that $A'$ is algebraically independent over $B\cup C$ and  \( A \subseteq \mcl_B(A')  \).
\end{cor}

\begin{comment}


\begin{proof}
Suppose \(A\) is \(C\)-generic over \(B\). Choose $A'$ to be a multiplicative basis of \(A\) over \(B\).
%
So $A'$ is multiplicatively independent over \(B\), and therefore also algebraically independent over $B\cup C$ by \(C\)-genericity.
%
Conversely, suppose the later condition holds.
%
Clearly, $A'$ is $C$-generic over $B$, and \( \mcl_B(A) = \mcl_B(A')  \). We apply (4) of the preceding proposition to get  the desired conclusion.
\end{proof}
\end{comment}

%
\noindent The notions of {\it multiplicative closure} and {\it multiplicative independence} can be understood using polynomials. A {\bf monomial} in $x$ is an element of $\qq[x]$ of the form  $x^k$ with $k \in \nn^n$. Likewise, a {\bf $B$-monomial} in $x$ is an element of $K[x]$ the form $b^l x^k$ with $b\in B^m$, $l \in \nn^m$ and $k \in \nn^n$. In this section $M$ and $N$ are $B$-monomials. 
%
A {\bf $B$-binomial} is a polynomial of the form $M-N$. If, moreover, $M$ and $N$ are monomials, we call $M-N$ a {\bf binomial}.
%
We call a $B$-binomial $M-N$ {\bf nontrivial}  if 
$$M\ =\ b^{l_M}x^{k_M}\ \text{ and }\ N\ =\ b^{l_N}x^{k_N}\ \text{ for some }\ b \in B^m\ \text{ and distinct }\ k_M, k_N \in \nn^n.$$ 
%
It is easy to see that for $a \in K^\neq$, $a$ is in $\mcl_B(A)$ if and only if $a$ is a zero of a non-trivial $(A \cup B)$-binomial of one variable. 
%
Then $A$ is multiplicatively independent over $B$ if for all $n$, for all $a \in A^n$, $a$ is not in the zero-set of a nontrivial $B$-binomial of $n$ variables.

%--------------



Suppose $K$ is a field, $H \subseteq G \subseteq K^\times $ are groups, $C$ is a subfield of $K$, $g \in G^n$, and $a \in K^n$.
%
The  \textbf{multiplicative type} of $g$ \textbf{over} \(H\), denoted by \( \mtp_H(g)\), is the quantifier free type of \(g\) in the language of groups with parameters from \(H\).
%
We can easily see that \( \mtp_H(g)\) is completely characterized by the $H$-binomials vanishing on \(g\). 
%
If \(H=\{1\}\), we simply call this the  \textbf{multiplicative type} of $g$, and denote this as \( \mtp(g)\).
%
Likewise, the  \textbf{algebraic type} of $a$ \textbf{over} \(C\), denoted by \( \atp_C(a)\), is the quantifier free type of \(a\) in the language of rings with parameters from \(C\).
%
Then \( \atp_C(a)\) is completely characterized by the polynomials in $C[x]$ vanishing on \(a\).
%
If \(C=\qq\), we call this the  \textbf{algebraic type} of $a$, and denote this by \( \atp(a)\). Suppose \( c\) is an $n$-tuple of elements in \( K\) and $d$ is an element in $K$. 
%
A solution \(a\) of the equation \(c\cdot x =d \) is called {\bf non-degenerate} if we have \(c_{i_1}a_{i_1}+\cdots+c_{i_m}a_{i_m} \neq 0 \) for all \( \{i_1, \ldots, i_m  \} \subsetneq
 \{1, \ldots, n\} \).

\begin{prop}\label{Equiv2}
 Suppose $K$ is a field, $H \subseteq G \subseteq K^\times $ are groups, and $C$ is a subfield of $K$. Moreover, suppose $\mcl(H)\cap G = H$. The following are equivalent:
\begin{enumerate}

\item $G$ is $C$-generic over \(H\);

\item for all $g, g' \in G^n$, if \( \mtp_H(g)= \mtp_H(g')\) then \( \atp_{C(H)}(g)= \atp_{C(H)}(g') \);

\item for all $ g \in G^n$ and all $P\in C(H)[x]$, $P$ vanishes on $g$ if and only if $P$ is in the ideal $I_g$ of $C(H)[x]$ generated by $H$-binomials vanishing on $g$;



\item if $c\in C^n$, and \(g \in G^n\) is a non-degenerate solution of the equation \( c\cdot x =1 \), then \(g\) is in \(H^n\).

\end{enumerate}
Without the condition  $\mcl(H)\cap G = H$, we still have \((4) \Rightarrow (3) \Rightarrow (2) \Rightarrow (1)\).
\end{prop}



\begin{proof}

Throughout the proof, we suppose $K, C, G$ and $H$ are as given. We first show that $(4)$ implies $(3)$.
%
Suppose \((4)\) holds, and \(P\) is in \(C(H)[x] \) such that \(P(g)=0\). 
For our purpose, we can arrange that
$$P\ =\ \sum_{i=1}^{k} c_iM_i- M_{k+1}\ \text{ where } c_i \text{ is in } C^\times \text{ and } M_i \text{ are } H\text{-monomials for } i \in \{1, \ldots, k\}.$$ 
%
The cases where $k=0, 1$ are immediate. Using induction, suppose $k>1$ is the least case the statement has not been proven.
%
Then \( \big( M_1(g), \ldots, M_{k+1}(g)\big)\) is a non-degenerate solution of \( c_1y_1+ \cdots+ c_{k}y_{k} - y_{k+1} =0 \).
%
Hence, $M_{k+1}(g) \neq 0$ and 
$$ \left( M_1(g)M_{k+1}^{-1}(g), \ldots, M_{k}(g)M_{k+1}^{-1}(g) \right) $$
is a non-degenerate solution of $c_1y_1+ \cdots+c_{k}y_{k} = 1 $.
%
Hence, it follows from (4) that  $M_i(g) M_{k+1}^{-1}(g) = h_i \in H$ for $i \in \{ 1,  \ldots, k\}$.
%
As a consequence,  $$( c_1h_1+ \cdots+ c_{k}h_{k} - 1) M_{k+1} \ =\ P - \sum_{i=1}^{k}c_i(M_i-h_iM_{k+1}) $$ vanishes on $g$. 
%
As $M_{k+1}(g) \neq 0$, the above implies $c_1h_1+ \cdots+ c_{k-1}h_{k-1} - 1 =0$. Thus 
$P = \sum_{i=1}^{k}c_i(M_i-h_iM_{k+1})$ which is in $I_g$. The conclusion follows.

To show that $(3)$ implies $(2)$, let $g$ and $g'$ be as in $(2)$.
%
Then a $H$-binomial vanishes on \(g\) if and only if it vanishes on \( g'\), and so \( I_g =I_{g'}\). 
%
The desired conclusion then follows from \( (2) \).

%----------------

We now show that $(2)$ implies $(1)$.
%
Suppose we have \((2)\) and $g\in G^n$ is multiplicatively independent over $H$. 
%
We can arrange that \(K\) is algebraically closed by $(1)$ of Lemma~\ref{GenericRobust}. 
%
The case where $n=0$ is trivial.
%
Using induction, suppose $n>0$ is the least case the statement has not been proven. 
%
Then $g_1, \ldots, g_{n-1}$ are algebraically independent over $C(H)$.
%
Assume \(P \in C(H)[x]\) is non-trivial. 
%
As \( g_1, \ldots, g_{n-1} \) are algebraically independent over \(C(H)\), we get that 
$$P(g_1, \ldots, g_{n-1}, x_n)\ \neq\ 0 \ \text{ in } C(H, g_1, \ldots, g_{n-1})[x_n],$$ and so it has at most finitely many roots. 
%
As a consequence, \( P(g_1, \ldots, g_{n-1}, g_n^m) \ \neq\  0 \) for some $m > 0$. 
%
Because \(g = (g_1, \ldots, g_n)\) is multiplicatively independent over \(H\), for all $m$, \((g_1, \ldots, g_{n-1},g_n^m)\) has the same multiplicative type over \(H\) as \((g_1, \ldots, g_n)\). By (2), for all $m$, \((g_1, \ldots, g_{n-1}, g_n^m)\) has the same algebraic type over $C(H)$ as \((g_1, \ldots, g_n)\).
%
Therefore, \(P(g_1, \ldots, g_n) \neq 0\). Since \(P\) is chosen arbitrarily, \(g\) is algebraically independent over \( C(H)\), and so we have \((1) \).

%---------------------

We show that  $(1)$ implies $(4)$.
%
Suppose we have \( (1)\), \(\mcl(H) \cap G=H \), and $g \in G^n$ is a non-degenerate solution of  \( c\cdot x =1 \).
%
Let $G'$ be the subgroup of $G$ generated by $g$. 
%
As $\mcl(H) \cap G=H $, the group $G'\slash (H \cap G')$ is torsion-free of finite rank, and so we can choose $g'_1, \ldots, g'_k$ in $G'$ multiplicatively independent over $H$ such that 
$$ g_i\ =\ M'_i(g'_1, \ldots, g'_k)\ \text{ for some } H\text{-monomial  }M'_i \text{ for } i \in \{1, \ldots, n\}.$$
%
As $g'_1, \ldots, g'_k$ are multiplicatively independent over $H$, they are algebraically independent over $C(H)$ by (1).
%
As 
$$g\ =\ \big(M'_1(g'_1, \ldots, g'_k), \ldots, M'_n(g'_1, \ldots, g'_k)\big)$$ is a non-denegerate solution of the equation $c\cdot x =1$, $g'_j$ must appear with power $0$ in all $M'_i$ for all $i \in \{ 1, \ldots, n\}$ and $j \in \{1, \ldots, k\}$.
%
Hence $g$ is in $H^n$.  

Finally, we observe that the condition \(\mcl(H) \cap G=H \) is only used in showing $(1)$ implies $(4)$. Thus, the other implications still hold without this condition.
\end{proof}



\noindent
Here, we present another property of genericity as a corollary of the previous proposition.

\begin{cor} \label{GenTrans} 
We have the following:
\begin{enumerate}
\item for \(A \subseteq A' \subseteq A'' \subseteq K^ \times\), \( A'\) is \(C\)-generic over \(A\) and \( A''\) is \(C\)-generic over \(A'\) if and only if \(A''\) is \(C\)-generic over \(A\);
\item suppose $\{A_\alpha\}_{\alpha < \kappa} \) is a sequence of subsets of $K^\times$ such that   $A_\alpha \subseteq A_{\alpha+1}$ and \( A_{\alpha+1}\) is \(C\)-generic over \(A_\alpha\) for all $\alpha < \kappa$, and $A_\beta = \bigcup_{\alpha < \beta} A_\alpha$  for all limit ordinals $\beta$. If $A = \bigcup_{\alpha < \kappa} A_\alpha$, then $A$ is $C$-generic over $A_\alpha$ for all $\alpha < \kappa$.
\end{enumerate}

\end{cor}

\begin{proof}
By Lemma~\ref{GenericRobust}, we can arrange that $C$ and $K$ are fields and all the $A_\alpha$'s involved are multiplicatively closed in $K$. In particular, each $A_\alpha$ with the multiplication is a group. The conclusions follow easily from the equivalence of $(1)$ and $(4)$ of Proposition~\ref{Equiv2}.
\end{proof}


\noindent We call a polynomial in $\qq[x]$ {\bf special} if it has the form $\prod_\zeta (M- \zeta N)$ where $\zeta$ ranges over the set of $k$-th primitive roots of unity for some $k>0$ and some monomials $M$ and $N$.


\begin{prop}\label{GenericEquivalence}
Suppose $K$ is a field, $G \subseteq K^\times $ is a group, and $U$ is the set of all roots of unity in $K$. Moreover, suppose \(\chara(K)=0  \). Then the following are equivalent:
\begin{enumerate}
\item G is generic;
\item for all \(g, g' \in G^n\), if \(\mtp(g)= \mtp(g') \) then \(\atp(g)=\atp(g') \); 
\item for all $g \in G^n$ and $P \in \qq[x]$, $P$ vanishes on $g$ if and only if $P$ is in $\sqrt{J_g}$ where $J_g \subseteq  \qq[x]$ is the ideal generated by the special polynomials vanishing on $g$;

\item if $c$ is in $\qq^n$, and $g \in G^n$ is a non-degenerate solution of the equation $ c\cdot x =1 $ then $g$ is in $U^n$.


\end{enumerate}
\end{prop}

\begin{proof}
Throughout the proof, we suppose $K, G$ and $U$ are as stated.
We first prove that $ (1)$ implies $(3)$. As the statement is independent of the ambient field, we can arrange that \(K\) is algebraically closed. It is clear even without assuming (1) that the backward implication of (3) holds. 
Now we suppose $(1)$ and prove the forward implication of $(3)$. We reduce the problem to finding finitely many special polynomials \(S_1, \ldots, S_l \) such that 
$$ Z(S_1, \ldots, S_l)\ \subseteq\ Z(P) .$$ 
Indeed, suppose we managed to do so. Then, by the Nullstellensatz, this implies \( P^m \) is in the ideal generated by \( S_1, \ldots, S_l\) in \(K[x]\). Hence \(P^m\) is a \(K\)-linear combination of products \(M_i S_j\) for $i \in \{1, \ldots, k\} $, $j \in \{ 1,  \ldots, l\}$ and each $M_i$ a monomial in $x$. By taking a linear basis of \(K\) over \(\qq\) and taking into account the assumption that \( P\) is in \( \qq(x) \), we get \(P^m\) is a \(\qq\)-linear combination of products of \( M_iS_j\) as above. Therefore, \(P\) is in \(\sqrt{J_g}\).

By equivalence of $(1)$ and $(3)$ in Proposition~\ref{Equiv2}, we have that \( P\) lies in the ideal \(I_g\) of \(\qq(U)[x]\) generated by polynomials of the form \( M - \zeta N \) vanishing on $g$ with \(M, N\) monomials in \(x\) and \( \zeta\) a root of unity.
As \(\qq(U)[x]\) is Noetherian, there are binomials \( M_1 - \zeta_1 N_1, \ldots,  M_l - \zeta_l N_l\) generating \(I_g\).
Hence, 
$$ Z(M_1 - \zeta_1 N_1, \ldots,  M_l - \zeta_l N_l)\ \subseteq\ Z(P). $$ Let $\zeta$ be a generator of the subgroup of $U$ generated by $\zeta_1, \ldots, \zeta_l$.
Then there are  natural numbers $s_1, \ldots, s_l$ and $t_1, \ldots, t_l$ such that  $\zeta = \zeta_1^{s_1}\ldots \zeta_l^{s_l}$ and $\zeta_i = \zeta^{t_i}$ for all $i \in \{1, \ldots l \}$. 
Let $$M'\ =\ \prod_{i=1}^l(M_i)^{s_i}\ \text{ and }\ N'\ =\ \prod_{i=1}^l(N_i)^{s_i}.$$
We note that $Z(M_1 - \zeta_1 N_1, \ldots,  M_l - \zeta_l N_l)$ is equal to $$ Z \big(M' - \zeta N', (N')^{t_1}M_1- (M')^{t_1}N_1, \ldots, (N')^{t_l}M_l- (M')^{t_l}N_l\big).$$
Therefore, we might as well assume $P$ vanishes on the zero set of polynomials $M_1(x) - \zeta N_1(x)$, $M_2(x)-N_2(x), \ldots, M_l(x) - N_l(x)$.

With \(\zeta, M_i, N_i \) as in the preceding statement, let \( \zeta\) be a primitive \(k\)-th root of unity. Set 
$$ S_1\ =\ \prod_\varepsilon (M_1- \varepsilon N_1 )\ \text{ where } \varepsilon \text{ ranges over the primitive } k\text{-th roots of unity }$$
and \( S_2 = M_2(x)-N_2(x), \ldots, S_l = M_l(x)-N_l(x) \). 
Note that each \(S_i\) is special. 
Suppose,  \(a \in K^n \) is in the zero set of the ideal of $\qq[x]$ generated by \( S_1, \ldots, S_l\). 
Then there is a primitive \(k\)-th root of unity \( \varepsilon\) such that \( M_1(a)- \varepsilon N_1(a)=0 \). Since \(\chara(K)=0\), there is an automorphism \(\sigma\) of \(K\) such that \( \sigma( \varepsilon) = \zeta \). 
Hence, $$ M_1\big(\sigma(a)\big)- \zeta N_1\big(\sigma(a)\big) \ =\ S_2\big( \sigma(a)\big)\ =\ \ldots\ =\ S_l\big( \sigma(a) \big)\ =\ 0.$$
By the choice of \(\zeta, M_i, N_i \), we have \( P\big( \sigma(a) \big) =0\). As \(P \) is in \(\qq[x]\), \(P(a) =0\). Thus, we have proven the reduction and hence (3).

Next, we prove that $(3)$ implies $(2)$. Suppose \( (3) \), and \(g, g'\) have the same multiplicative type.
Let $ S $ be a special polynomial such that 
$S(g)=0$ and $S = \prod_\zeta (M- \zeta N)$ where $\zeta$ ranges over all the primitive $k$-th roots of unity for some $k>0$.
Then $M(g)N^{-1}(g)$ is a primitive $k$-th $k$-th root of unity, so $M^k-N^k$ vanishes on $g$ but $M^l-N^l$ does not vanish on $g$ for $0<l<k$.
As $g, g'$ have the same multiplicative type, $$M^k(g')-N^k(g')\ =\ 0\ \text{ but }\ M^l(g')-N^l(g')\ \neq\ 0\ \text{ for }0<l<k.$$
So $M(g')N^{-1}(g')$ is a primitive $k$-th root of unity and $S(g')=0$. 
Hence \( J_g = J_{g'} \), and so \(\atp(g)=\atp(g') \). Thus, we have \( (2)\). 

The argument for (2) implying (1)  is the same as the argument for (2) implying (1) in Proposition~\ref{Equiv2}. Finally, by $(2)$ of Lemma~\ref{GenericRobust}, \( G\) is generic if and only if \(G\) is generic over \( G \cap U\). We note that \(\mcl(G\cap U) \cap G = G \cap U \), so the equivalence between \( (1) \) and \( (4) \) follows immediately from the equivalence between \( (1) \) and \( (4) \) in Proposition~\ref{Equiv2}. 
\end{proof}

\begin{prop} \label{Regularity}
Suppose $K$ is a field, $G \subseteq K^\times $ is a group, $g$ is in $G^n$ and $H$ is a subgroup of $G$ such that $G$ is generic over $H$. Moreover, suppose \(\chara(K)=0  \), and $\mcl(H)\cap G = H$. Then $\qq(G)$ is a regular field extension of $\qq(H)$.
\end{prop}

\begin{proof}
As $\chara(K)=0$, $\qq(G)$ is a separable field extension of $\qq(H)$, so it suffices to check that $\qq(H)$ is algebraically closed in $\qq(G)$. 
Suppose $P, Q \in \qq[x]$, and $g \in G^n$ is such that $P(g)Q^{-1}(g)$ is algebraic over $\qq(H)$. Let $G'$ be the subgroup of $G$ generated by $g$. 
As $\mcl(H) \cap G=H $, $G'\slash (H \cap G')$ is torsion-free of finite rank, we can choose $g'_1, \ldots, g'_k$ in $G'$ multiplicatively independent over $H$ such that $$g_i\ =\ M'_i(g'_1, \ldots, g'_k)\ \text{ where }  M'_i \text{ is } H\text{-monomial for }i \in \{1, \ldots, n\}$$
Hence we can find  $P', Q'$ coprime in $\qq(H)[y_1, \ldots, y_k]$ such that $P'(g')Q'^{-1}(g')$ is equal to  $P(g)Q^{-1}(g)$.
As $g'_1, \ldots, g'_k$ are multiplicatively independent over $H$, they are algebraically independent over $\qq(H)$. 
Therefore, in order to have $P'(g')Q'^{-1}(g')$ algebraic over $\qq(H)$, the polynomials $P', Q'$ must have degree $0$ and  so $P'(g')Q'^{-1}(g')$ is in $\qq(H)$. The conclusion follows.
\end{proof}

\noindent We recall the following version of a theorem of Mann from \cite{Mann}:

\begin{thm*}[Mann] 
Let $U$ be the group of roots of unity in $\qa$. There is a recursive function \( d: \nn \to \nn \) such that if \(a_1, \ldots, a_n\) are in \( \qq\)  and \((y_1, \ldots, y_n)\) in \(U^n\) is a tuple of non-degenerate solution of the equation $a_1y_1+\cdots+a_ny_n=1$, then \( y_i^{d(n)}=1 \) for all \(i\).
\end{thm*}

\noindent
For an $L$-structure $(F, K; \chi)$, it is easy to see that $\chi$ in generic if and only if $\chi(F^\times)$ is generic  in the sense of this section. As a consequence we have:

\begin{prop} \label{GenericAxiom}
There is a recursive set of universal statements in $L$ whose models $(F, K, \chi)$ are precisely the $L$-structures with $\chi$ generic.
\end{prop}

\begin{proof}
Suppose $F'$ and $K'$ are respectively the fraction fields of $F$ and $K$.
Using only the conditions that $\chi$ is multiplication preserving, $\chi(0) =0$ and $\chi$ is injective, we can extend $\chi$ to an injective character $\chi': F' \to K'$; moreover, $\chi'$ maps multiplicatively independent elements to algebraically independent elements if and only if $\chi$ does so by Lemma~\ref{Equiv1}.
We also note that $(F', K'; \chi')$ is interpretable in $(F, K; \chi)$ in the obvious way.
Hence we can reduce the problem to the case where  $F$ and $K$ are fields.

 Combining  the equivalence between \((1)\) and \((4) \) of Proposition~\ref{GenericEquivalence} and Mann's theorem, \(\chi\) is generic if and only if for all $n$ and all non-degenerate solutions of $a_1x_1+\cdots+a_nx_n=1$ in \(\big(\chi(F)^\times\big)^n\) with $a$ in \(\qq^n\), we have \( x_i^{d(n)} =1 \) for \(i \in \{ 1, \ldots, n\}\).
It is clear that being a non-degenerate solution is definable by a quantifier-free formula.
So we have the desired universal axiom scheme.
\end{proof}

\begin{thm}
There is a recursive set $\TN$ of $\forall\exists$-axioms in $L$ such that:
\begin{enumerate}
\item for all $(F, K;\chi)$, $(F, K;\chi)\models \TN$ if and only if $F$ and $K$ are algebraically closed fields, $\chara(K)=0$ and $\chi:F \to K$ is generic;
\item for $p>0$, $(\fpa, \qa; \chi)\models \TN$.
\end{enumerate}
\end{thm}

\begin{proof}
It follows easily from proposition~\ref{GenericAxiom} that we have the desired axiomatization. 
When $F= \fpa, K = \qa$ and $\chi: F \to K$ is injective, we note that $\chi$ is automatically generic because there is no multiplicative independence between elements of $\chi(F^\times)$.
\end{proof}

\noindent Let $\mathscr{Q}$ be the set of prime powers. For each $q \in \mathscr{Q} $, let $\chi_q: \ff_q \to \qa$ be an injective map with $\chi_q(0)=0$ and $\chi_q(ab)=\chi_q(a)\chi_q(b)$ for all $a,b\in \ff_q$. With exactly the same method we get: 

\begin{prop}
There is a recursive set of $\forall\exists$-axioms $T$ in $L$ with the following properties:
\begin{enumerate}
\item for all $(F, K;\chi)$, $(F, K;\chi)\models T$ if and only if $K$ is an algebraically closed fields with $\chara(K)=0$, $F$ is a pseudo-finite field and $\chi:F \to K$ is generic;
\item  if $\mathscr{U}$ is a non-principal ultrafilter on $\mathscr{Q}$, then $\left( \prod_{q \in \mathscr{Q}}(\ff_q, \qa; \chi_q) \right) \slash \mathscr{U} \models T$.
\end{enumerate}
\end{prop}

\noindent
This also allows us to conjecture that for every $T$-model $(F,K;\chi)$, there is an ultrafilter $\mathscr{U}$ on $\mathscr{Q}$ such that  $ (F,K;\chi) \equiv \left( \prod_{q \in \mathscr{Q}}(\ff_q, \qa; \chi_q) \right) \slash \mathscr{U}$.