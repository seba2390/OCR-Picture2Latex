\section{Substructures and elementary substructures}
\noindent
From now on, let $k, l$ range over the set of natural numbers, $s=(s_1, \ldots, s_k)$, $v =(v_1, \ldots, v_l)$ be tuples of variables of the first sort and $w= (w_1, \ldots, w_k)$, $z = (z_1, \ldots, z_l)$ be tuples of variables of the second sort. We also implicitly assume similar conventions for these letters with additional decorations.

\noindent In addition to the notation conventions in the first paragraph of section 2, we assume in this section that $(F, K; \chi)$ has $\chara(K)=0$. We use $\subseteq$ and $\preccurlyeq$ to denote the $L$-substructure and elementary $L$-substructure relations respectively. 
We will characterize the substructures and elementary substructures of a model of $\TN_p$. 



\begin{prop} \label{Substructures}
We have $(F, K; \chi)$ is an $L$-substructure of an $\TN_p$-model if and only if $\chi$ is generic and $\chara(F)=p$.
\end{prop}

\begin{proof}
The forward implication is clear. For the other direction, suppose $\chi$ is generic and $\chara(F)=p$. We can embed $(F, K; \chi)$ into an $L$-structure $(F'', K'';\chi'')$ where $F'', K'' $ are respectively the fraction fields of $F, K$ and $\chi''$ is the natural extension of $\chi$ to $F''$. We note that $\chi''$ is still generic. Therefore, we can arrange that $F$ and $K$ are already fields.


Let $G$ be $ \chi(F^\times)$, $F'$ be the algebraic closure of $F$, and  $K'$ be an algebraically closed field containing $K$  such that $\text{trdeg}(K' \mid K)>|F'|$. Let $\{\alpha_i \}_{ i < \kappa}$ be a multiplicative basis of ${F'}^\times$ over $F^\times$. As $\text{trdeg}(K'\mid K)>|F'|$, we can define a map $$\chi': \{\alpha_i \}_{ i < \kappa} \to K'$$ such that the image is algebraically independent over $K$.
Since $\chara(K)=0$, we have $\mcl_G\big(\{ \chi'(\alpha_i) \}_{ i < \kappa}\big)$ in ${K'}^\times$ is divisible. Hence we can extend $\chi'$ to an injective map $\chi': {F'}^\times \to {K'}^\times$ extending $\chi$. Let $G' = \chi({F'}^\times)$.
Then $G'$ is $K$-generic over $G$ by Corollary~\ref{Equiv1}.
Since $G $ is generic, $G'$ is also generic by Corollary~\ref{GenTrans}. Thus the structure $(F', K'; \chi')$ is the desired model of $\TNp$.
\end{proof}

\noindent
Let $(F', K';\chi')$ be an $L$-structure. We say that $(F, K; \chi)$ is a {\bf regular substructure} of $(F', K';\chi')$, denoted as  $(F, K; \chi) \sqsubseteq (F', K';\chi')$,  if  $(F, K; \chi) \subseteq (F', K';\chi')$ and $\chi'( {F'}^\times)$ is $K$-generic over $\chi( F^\times)$. With the use of Proposition~\ref{Equiv2}, it can be seen that the above proof also gives us the following stronger statement:

\begin{cor}\label{Substructures2}
If $\chi$  is generic then there is a model $(F', K'; \chi')$ of $\TNp$ such that 
$(F, K; \chi) \sqsubseteq (F', K';\chi')$.
\end{cor}


\begin{comment}

\begin{prop}\label{Substructures}

Let $(R, A; \chi)$ be an $\Lprc$-structure where $R$ and $A$ are integral domains and $\chi$ is a reduction of a generic character between $Frac(R)$ and $Frac(A)$. Then $(R, A; \chi)\models \TNp(\forall)$.
\end{prop}

\begin{proof}


Denote $F=Frac(R)$ and $K=Frac(A)$. 
So we can assume that we have a character $\chi:F\to K$. $(F,K;\chi)$ still satisfies to the condition of the proposition. So it is enough to prove the statement for this structure. We will find a suitable model of $\TNp$ which contains $(F,K;\chi)$ as a substructure. We can choose $L\supseteq K$ a very big algebraically closed field, such that $trdeg(L|K(\chi(F)))>|F^{ac}|$.

Let $S$ be the set of $(F',\rho)$'s, where $F'$ is a multiplicative subgroup of $(F^{ac})^{\times}$ containing $F$ and $\rho:F'\to L$ is a generic character.
We also introduce a partial order on $S$ as $(F'_1,\rho_1)<(F'_2,\rho_2)$ if and only if $F'_1\subseteqeq F'_2$ and $\rho_2\res_{F'_1}=\rho_1$.
As $S$ is not empty because it contains $(F^{\times},\chi)$, by Zorn's lemma we can easily induce that $S$ has a maximal element. 
Let $(E,\rho)$ be a maximal element in $S$. We claim that $E=(F^{ac})^{\times}$. Assume $E\neq F^{ac}$. Take an element $f\in F^{ac}\setminus E$ and consider the group generated by $E$ and $f$. The following cases are possible:

Case 1: the element $f$ is multiplicatively independent from $E$.
In this case, we take an extension of $\rho$, by mapping $f$ to a transcendental element over $K(\rho(E))$.
Then, also define $\rho(f^{-1})=(\rho(f))^{-1}$.
This is possible by our assumptions on $L$. This will give us an element in $S$, thus, giving a contradiction.

Case 2: the element $f$ and $E$ are multiplicatively dependent.
Choose the least natural number $n$ so that $f^n$ is equal to a monomial in $E$. Extend $\rho$ by mapping $f$ to any $n$-th root of the image of the monomial under the $\rho$. It is easy to see, that this extension is well-defined. This will induce a generic character on the group generated by $E$ and $f$ and thus contradicting to the maximality of the pair $(E,\rho)$.

This maximal element $(E,\rho)$ of $S$ will give us a character $\rho: F^{ac}\to L$ such that $(F^{ac},L;\rho)$ is a model of $\TNp$.
\end{proof}

\end{comment}


\noindent
We now characterize the regular substructure relation for models of $\TN_p$.

\begin{prop}\label{GenericSubstructure}
Suppose $(F,K;\chi) \subseteq (F',K';\chi')$ are models of $\TN_p$. Let $G= \chi(F^\times)$ and $G' = \chi'({F'}^\times)$. Then the following are equivalent:

\begin{enumerate}
\item $(F, K; \chi) \sqsubseteq (F', K';\chi')$;

\item for all $n$, all $P_1, \ldots, P_n \in \qq[w]$ and all $a_1, \ldots, a_n \in K$, if there is a tuple $g' \in {G'}^{k}$ with $P_1(g'), \ldots, P_n(g')$ not all $0$ and  $a_1P_1(g') + \cdots+ a_nP_n(g') =0$, then we can find such a tuple in $G^{k}$;

\item $\qq(G')$ and $K$ are linearly disjoint over $\qq(G)$ in $K'$.
\end{enumerate}

\end{prop}


\begin{proof}

Towards showing that $(1)$ implies $(2)$, suppose (1). Fix $n$, polynomials $P_1, \ldots, P_n \in \qq[w]$, $K$-elements $a_1, \ldots, a_n$ and $g'$ as in $(2)$. 
We want to find $g \in G^{k}$ with $P_1(g), \ldots, P_n(g)$ not all $0$ and  
$$a_1P_1(g) + \cdots+ a_nP_n(g)\ =\ 0.$$
Replacing $(F,K;\chi)$ and $(F',K';\chi')$ concurrently with elementary extensions and noting that $G'$ remains $K$-generic over $G$ by the equivalence between $(1)$ and $(4)$ of Proposition~\ref{Equiv2}, we can arrange that $(F,K;\chi)$ is $\aleph_0$-saturated. By the equivalence between $(1)$ and $(2)$ of Proposition~\ref{GenericEquivalence}, 
$$ \text { if }\ \text{mtp}(g)\ =\ \text{mtp}(g')\ \text{ then }\ P_1(g), \ldots, P_n(g)\ \text{ are not all }0.$$ 
By the equivalence between (1) and (3) of Proposition~\ref{Equiv2}, there are $G$-binomials $M_1-N_1, \ldots, M_l - N_l$ vanishing on $g'$ such that $$M_1(g)-N_1(g)\ =\  \cdots \ =\  M_l(g) - N_l(g) \ =\ 0\ \text{ implies }\ a_1P_1(g) + \cdots+ a_nP_n(g) \ =\ 0.$$
Let $\alpha=\chi^{-1}(g)$, $\alpha'=\chi^{-1}(g')$ and $\chi^{-1}M_i, \chi^{-1}N_i$ be the pullbacks of $M_i$ and $N_i$ under $\chi$ for $i \in  \{1, \ldots,l\}$. 
It suffices to find $\alpha \in (F^\times)^{k}$ with $\mtp(\alpha)=\mtp(\alpha')$ and $$\chi^{-1}M_1(\alpha)-\chi^{-1}N_1(\alpha)\ =\  \cdots\ =\ \chi^{-1}M_l(\alpha)-\chi^{-1}N_l(\alpha)\ =\ 0.$$ Such $\alpha$ can be found as $F$ is an elementary substructure of $F'$ in the language of field and $F$ is $\aleph_0$-saturated. Thus we have (2).

It is immediate that  (2) implies (3). Towards showing that (3) implies (1), suppose (3) and $g' \in (G')^n$ is algebraically dependent over $K(G) =K$. We need to show that $g'$ is multiplicatively dependent over $G$. Pick a non-trivial $P \in K[x]$ with $P(g')=0$. Choose a linear basis $(b_i)_{i\in I}$ of $K$ over $\qq(G)$. Then 
$$P\ =\ \sum_{i \in I} P_ib_i\  \text{ with }\ P_i \in \qq(G)[x]\ \text{ for } i \in I $$ 
and $P_i = 0$ for all but finitely many $i \in I$. 
Hence $\sum_{i \in I} P_i(g')b_i =0 $. By (3), $(b_i)_{i\in I}$ remains linearly independent over $\qq(G')$. Therefore, $\sum_{i \in I} P_i(g')b_i =0 $ implies that $P_i(g')=0$ for all $i \in I$. 
Since $P$ is non-trivial, at least one $P_i$ is non-trivial, and hence $g'$ is algebraically dependent over $\qq(G)$. Now, $G'$ is generic so $G'$ is generic over $G$ by (1) of Corollary~\ref{GenTrans}. By the definition of genericity, $G'$ is $\qq(G)$-generic over $G$.  Hence, $g'$ is multiplicatively dependent over $G$ which is the desired conclusion.
\end{proof}



\begin{cor} \label{ElSub1}
For $(F, K; \chi), (F', K';\chi') \models \TN_p$, if  $(F,K;\chi) \preccurlyeq (F',K';\chi')$, then $(F,K;\chi)\sqsubseteq (F',K';\chi')$.
\end{cor}

\begin{proof}
This follows from the equivalence between (1) and (2) of Proposition~\ref{GenericSubstructure}.
\end{proof}


\begin{cor}
For all $p$, $\TN_p$ is not model complete, and has no model companion in $L$. The same conclusion applies to $\TN$.
\end{cor}

\begin{proof}
We show that $\TNp$ is not model complete. Let $(F, K; \chi)$ and $(F', K'; \chi)$  be models of $\TNp$ such that the former is an $L$-substructure of the latter and the latter is $\kappa$-saturated with $\kappa > |F| + |K|$. Set $G = \chi(F^\times)$ and $G'= \chi({F'}^\times) $. We get by saturation $a, b \in G'$ algebraically independent over $K$.
We will show that 
$$(F, K''; \chi)\ \not \sqsubseteq\ (F', K'; \chi)\ \text{ where }\ K''\ =\ K (a + b) ^{ac},$$ which yields the desired conclusion by Corollary~\ref{ElSub1}.
Fix $g'\in G' \cap K''$. Then $g'$ and $a + b$ are algebraically dependent over $K$ and therefore so are $g'$, $a$ and $b$. 
Suppose $G'$ is $K''$-generic over $G $. As a consequence, $g'$, $a$ and $b$ are also multiplicatively dependent over $G$.  By replacing $g'$ with some power of it if needed, we arrange $g' = M(a, b)$ for some $G$-monomial $M$. Then $M(a, b)$ is in $K'' = K(a, b)^{ac}$ and so $a$ and $b$  are algebraically dependent over $K$, a contradiction. As a consequence, $(F, K''; \chi) \not \sqsubseteq (F', K'; \chi)$.

\noindent Suppose $\TNp$ has a model companion $T$ in $L$.
Take any model $\mathscr{M}$ of $T$. 
Then $\mathscr{M}$ is an $L$-substructure of $(F,K;\chi) \models \TN_p$ which itself is an $L$-substructure of $\mathscr{N} \models T$.
Let $\varphi(x)$ be an existential formula in $L$ such that $\TN_p \models \forall x \varphi(x)$. Hence, for all $a$ with components in $\mathscr{M}$ of suitable sorts, $(F,K;\chi) \models \varphi(a)$ and so $\mathscr{N} \models \varphi(a)$. As $T$ is model complete, we also have  $\mathscr{M} \models \varphi(a)$. 
Since $\TN_p$ is  a set of $\forall \exists$-formulas, $\mathscr{M} \models \TN_p$.
On the other hand, $\TNp$ is complete. Hence, $T = \text{Th}(  \mathscr{M}) = \TNp$, a contradiction as $\TN_p$ is not model complete.

It is easy to see that if a theory $T$ has a model companion, then any of its extension also has a model  companion. The final conclusion thus follows.
\end{proof}

\noindent
There are clearly some obstructions for one model of $\TN_p$  to be an elementary submodel of another model of $\TN_p$ that contains it. We will show that these are the only obstructions. For the main theorem of this section we need the following two technical lemmas:






\begin{lem} The following statements hold:
\begin{enumerate}
\item If $(F, K;\chi) \sqsubseteq (F', K';\chi')$ and $ (F', K';\chi') \sqsubseteq (F'', K'';\chi'')$, then we have $(F, K;\chi) \sqsubseteq (F'', K'';\chi'')$.
\item Suppose $ (F_0, K_0;\chi_0) = (F, K; \chi)$ and $ (F_m, K_m;\chi_m) \sqsubseteq (F_{m+1}, K_{m+1};\chi_{m+1})$ for every $m$. If \(F' = \bigcup_m F_m, K' = \bigcup_m K_m \) and $\chi' = \bigcup_m \chi_m $, then we have $(F, K;\chi) \sqsubseteq (F', K';\chi')$.
\end{enumerate}
\end{lem}

\begin{proof}
This follows from Corollary~\ref{GenTrans} and the definition of genericity.
\end{proof}

\begin{lem}
Let $ (F_0, K_0;\chi_0) = (F, K; \chi)$, $ (F_m, K_m;\chi_m) \sqsubseteq (F_{m+1}, K_{m+1};\chi_{m+1})$ for each $m$, and \(F' = \bigcup_m F_m, K' = \bigcup_m K_m \), $\chi' = \bigcup_m \chi_m $. If $(F_m, K_m;\chi_m)$ is a model of $\TN_p$ with $|K_m| =\kappa$ for each $m$ and $(F, K; \chi)$
 is $(\kappa, \kappa)$-transcendental, then $ ( F', K'; \chi')$ is a $(\kappa, \kappa)$-transcendental model of $\TN_p$. 
\end{lem}

\begin{proof}
In addition to the above notations, let $G = \chi( F^\times)$ and $G' = \chi( {F'}^\times)$. As $\TNp$  is a set of $\forall\exists$-formulas,  $(F', K'; \chi')$ is a model of $\TNp$.  
Since $ (F, K; \chi)$ is $(\kappa,\kappa)$-transcendental, there is $a \in K^\kappa$ with all components algebraically independent over $\qq(G)$. 
By the preceding lemma, $$ (F, K; \chi)\ \sqsubseteq\ (F', K'; \chi').$$
By Proposition~\ref{GenericSubstructure}, $K $ and $ \qq(G')$ are linearly disjoint over $ \qq(G)$ in $K'$.
Hence the components of $a$ remain algebraically independent over $\qq(G')$. Therefore, $\trdeg\big(K' \mid  \qq(G')\big) \geq \kappa$.
Also, $\trdeg(F' \mid \ff_p) \geq  \kappa$.
Hence $|F'|=|K'|= \kappa$ by a cardinality argument. 
Thus, $ ( F', K'; \chi')$  must be $(\kappa, \kappa)$-transcendental.
\end{proof}

\begin{thm} \label{ElSub2}
\(\TNp \) is the regular model companion of \( \TNp(\forall) \). That is:
\begin{enumerate}
\item every model of \( \TNp(\forall) \) is a regular substructure of a model of \( \TNp \);
\item when $(F, K; \chi) \sqsubseteq (F',K';\chi')$ are models of $\TN_p$, then we have $(F, K; \chi) \preccurlyeq (F',K';\chi')$.
\end{enumerate}
\end{thm}


\begin{proof}

We have (1) follows from Corollary~\ref{Substructures2}. The proof of (2) requires some preparation. We let $L^{+}$ be the language obtained by adding to $L$ an $n$-ary relation $R_{P_1,\ldots, P_n}$ for each $n$, and each choice of polynomials $P_1, \ldots, P_n \in \qq[w]$. The theory $\TNp^+$ is obtained by adding to $\TN_p$ the following axioms for each choice of $n,P_1, \ldots, P_n$:
$$R_{P_1,\ldots, P_n}(x)\leftrightarrow \exists s \left( \Big( \bigvee_{i=1}^n P_i \big(\chi(s)\big) \neq 0\Big) \wedge \Big( x_1 P_1 \big(\chi(s) \big) + \cdots+ x_n P_n \big(\chi(s) \big) =0 \Big)  \right). $$ 
We note that $\TNp^+$ is still a complete $\forall\exists$-theory. If $(F, K ; \chi)$ is a model of $\TN_p$, we will let $(F, K ; \chi, R)$ be its natural expansion to a model of $\TNp^+$; here, $R$ represents all the possible $R_{P_1,\ldots, P_n}$ for simplicity of notation. Then, by equivalence of $(1)$ and $(2)$ of Proposition~\ref{GenericSubstructure}, (2) of this theorem is equivalent to saying the theory $\TNp^+$ is model complete in $L^+$. 

It suffices to show that all models of $\TNp^+$ are existentially closed. Suppose we have a counterexample $ (F, K; \chi, R)$. We first reduce to the case where $ (F, K; \chi)$ is moreover $(\kappa, \kappa)$-transcendental for some infinite $\kappa$. By assumption, there is an $\TN_p^+$-model $(F'', K''; \chi'', R'')$ extending $ (F, K; \chi, R)$ as a $L^+$-substructure such that the latter is not existentially closed in the former. Consider the structure $(F'', K''; \chi'', R'', F, K, \chi, R)$ in the language where $F, K, R, \chi$ are regarded as relations on $(F'', K''; \chi'', R'')$.
Note that if we replace this structure with an elementary extension we will still have $ (F, K; \chi, R)$ a non-existentially closed $\TN_p^+$-submodel of $ (F'', K''; \chi'', R'')$. Using a similar trick as in Lemma~\ref{KappaExt}, we can add the condition that $ (F, K; \chi)$ is $(\kappa, \kappa)$-transcendental.

Next, we will construct $ (F', K'; \chi', R')$ existentially closed such that $(F',K'; \chi')$ is $(\kappa,\kappa)$-transcendental. We start with $ (F_0, K_0; \chi_0, R_0) = (F, K; \chi, R) $, the structure obtained at the end of the previous paragraph, and for each $m>0$ construct the $\TNp^+$-model $ (F_m, K_m; \chi_m, R_m)$ as follows.
Choose $(F_{m+1}, K_{m+1}; \chi_{m+1}, R_{m+1})$ to be an $\TNp^+$-model extending $(F_m, K_m; \chi_m, R_m)$ realizing a maximal consistent set of existential formulas with parameters from $(F_m, K_m; \chi_m, R_m$); concurrently, we use downward L\"owenheim-Skolem theorem to arrange $|K_m| = \kappa$.  Let $$F'\ =\ \bigcup_m F_m,\ K'\ =\ \bigcup_m K_m,\ \chi'\ =\ \bigcup_m \chi_m,\ R'\ =\ \bigcup_m R_m.$$ By construction, $(F',K'; \chi')$ is an existentially closed model of $\TN_p^+$. By the equivalence between (1) and (2) of Proposition~\ref{GenericSubstructure},  $( F_m, K_m; \chi_m)$ is a regular substructure of $( F_{m+1}, K_{m+1}; \chi_{m+1})$.
It follows from the preceding lemma that $(F', K'; \chi')$ is $(\kappa, \kappa)$-transcendental.

Finally, by Theorem~\ref{ThmIso}, $ (F, K; \chi)$  and  $ (F', K'; \chi')$ are isomorphic. Hence, $(F, K; \chi, R)$ is also isomorphic to $ (F', K'; \chi', R')$, a contradiction to the fact that the former is not existentially closed but the latter is.
\end{proof}

\begin{cor} \label{ElementarilyEmbeddable}
Suppose $(F, K; \chi) \models \TN_p$ is $(\kappa, \lambda)$-transcendental and $(F', K'; \chi') \models \TN_p$ is $(\kappa', \lambda')$-transcendental. Then $(F, K; \chi) $ can be elementarily embedded into $(F', K'; \chi')$ if and only if $\kappa \leq \kappa'$ and    $\lambda \leq \lambda'$.
\end{cor}

\begin{proof}
We prove the forward direction. Suppose $(F, K; \chi)$ and $(F', K'; \chi')$ are as stated and $(F, K; \chi)$ is elementarily embeddable into $(F', K'; \chi')$. We can arrange that $(F, K; \chi) \preccurlyeq (F', K'; \chi') $. Clearly, $\kappa' \geq \kappa$.  Furthermore, by Corollary~\ref{ElSub1} and $1 \Leftrightarrow(3)$ of Proposition~\ref{GenericSubstructure}, $\qq\big(\chi({F'}^\times)\big)$ and $K$ are linearly disjoint over $\qq\big(\chi(F^\times)\big)$ in $K'$, and so  $\lambda' \geq \lambda$.

For the backward direction, using Theorem~\ref{ThmIso} it suffices to show that a fixed $(\kappa, \lambda)$-transcendental model $(F, K; \chi)$ of $ \TN_p$ has a $(\kappa', \lambda')$-transcendental elementary extension. Find $F'$ extending $F$ with $|F'| =\kappa'$, take $K''$ a sufficiently large algebraically closed field containing $K$ and construct $K' \subseteq K''$ in the same fashion as in the proof of Proposition~\ref{Substructures} to obtain $(F', K'; \chi')$ such that $(F,K;\chi) \sqsubseteq (F',K';\chi')$. This is the desired model by the preceding theorem.
\end{proof}
