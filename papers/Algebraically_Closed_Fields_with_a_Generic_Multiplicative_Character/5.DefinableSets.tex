\section{Definable sets I}
\noindent
We keep the notation conventions in the first paragraphs of sections 2 and 4. Moreover, we assume in this section that $ (F, K; \chi) \models \TN_p$. A set $X\subseteq K^n$ is {\it definable in  the field $K$} if it is definable in the underlying field $K$. 
In this case, we use $\mr_K(X), \md_K(X)$ to denote the corresponding Morley rank and degree. %
We equip $K^n$ with the Zariski topology on $K$, also referred to as the  {\it $K$-topology}. A  {\it $K$-algebraic set} is a closed set in this topology.
%
We define the corresponding notions for $F$ in a similar fashion. In this section, we show that definable sets in a model of $\TN_p$ has a geometrically and syntactically simple description. The following observation is immediate:
%
%----------------



\begin{comment}
With possible decorations, $u=(u_1, \ldots, u_k)$ and $v $  with suitable length are tuples of variables of the first-sort, $ w$ with suitable length and  $x=( x_1 \ldots, x_n)$ are tuples of variables of the second sort.  
\end{comment}


\begin{prop} \label{transferRemark}
Let $\chi: F^k \times K^n \to K^{k+n}: ( \alpha, a) \mapsto \big( \chi(\alpha), a\big) $.
%
If $X \subseteq F^k \times K^n$  is definable, then $\chi \res_X: X \to \chi(X)$ is a definable bijection. 
%
Moreover, $X \subseteq K^n$ is definable over $( \gamma, c) \in F^l \times K^m$ if and only if $X$ is definable over  $\big( \chi(\gamma), c\big) \in K^{l+m}$. 
\end{prop}

\noindent
Hence, we restrict our attention to definable subsets of $K^n$. For a similar reason, we only need to consider sets definable over $c \in K^m$.

Suppose $(H_b)_{ b \in Y}$ and $(H'_{b'})_{ b' \in Y'}$ are families of subsets of $K^{n}$. We say $(H_b)_{ b \in Y}$ {\bf contains}  $(H'_{b'})_{ b' \in Y'}$ if for each $b' \in Y'$, there is $b\in Y$ such that $H_b = H'_{b'}$; $(H_b)_{ b \in Y}$ and $(H'_{b'})_{ b' \in Y'}$ are {\bf equivalent} if each contains the other.
A {\bf combination} of $(H_b)_{ b \in Y}$ and $(H'_{b'})_{ b' \in Y'}$ is any family of subsets of $K^n$ containing both $(H_b)_{ b \in Y}$ and $(H'_{b'})_{ b' \in Y'}$, which is minimal with these properties in the obvious sense.
A {\bf fiberwise intersection} of $(H_b)_{ b \in Y}$  and $(H'_{b'})_{ b' \in Y'}$ is any family of subsets of $K^n$ equivalent to $( H_b \cap H'_{b'})_{(b,b') \in Y \times Y'}$. 
A {\bf fiberwise union} of $(H_b)_{ b \in Y}$ and $(H'_{b'})_{ b' \in Y'}$ is any family of subsets of $K^n$ equivalent to $( H_b \cup H'_{b'})_{(b,b') \in Y \times Y'}$. 
A {\bf fiberwise product} of $(H_b)_{ b \in Y}$ and $(H'_{b'})_{ b' \in Y'}$ is any family of subsets of $K^{2n}$ equivalent to $( H_b \times H'_{b'})_{(b,b') \in Y \times Y'}$; this definition can be generalized in an obvious way for two families of subsets of different ambient spaces. The following is immediate from the above definitions:

\begin{lem} \label{fiber}
Suppose $(H_b)_{ b \in Y}$  and $(H'_{b'})_{ b' \in Y'}$ are families of subsets of $K^n$. Let $X= \bigcup_{b \in Y} H_b$ and $X'= \bigcup_{b' \in Y'} H'_{b'}$. Then we have the following:
\begin{enumerate}
\item $X \cup X'$ is the union of any combination of $(H_b)_{ b \in Y}$ and $(H'_{b'})_{ b' \in Y'}$;
\item $X \cap X'$ is the union of any fiberwise intersection of  $(H_b)_{ b \in Y}$ and $(H'_{b'})_{ b' \in Y'}$;
\item $X \cup X'$ is the union of any fiberwise union of $(H_b)_{ b \in Y}$ and $(H'_{b'})_{ b' \in Y'}$;
\item $X \times X'$ is the union of any fiberwise product of $(H_b)_{ b \in Y}$ and $(H'_{b'})_{ b' \in Y'}$. This part of the lemma can be generalized in an obvious way for two families of subsets of different ambient spaces.
\end{enumerate}
\end{lem}

\noindent
A family $(X_b)_{ b \in Y}$ of subsets of $K^n$ is { \bf definable (over $c$)} if both $Y$ and the set  $$\big\{(a,b) \in K^n \times Y : (a,b) \in X_b  \big\} $$ are definable (over $c$).
We note that if $(X_b)_{ b \in Y}$ is definable over $c$, then for each $b \in Y$, $X_b$ is definable over $(b,c)$ but not necessarily over $c$. 

For two families of subsets of $K^n$ which are definable (over $c$), we can choose  a combination, a fiberwise intersection, a fiberwise union and a fiberwise product of these two families to be definable (over $c$); the statement about fiberwise product can be generalized in an obvious way for two families of subsets of different ambient spaces.
A {\bf presentation} of  $X \subseteq K^n$ is a definable family $( H_\alpha)_{\alpha \in D}$ such that $$X\ =\ \bigcup_{\alpha \in D} H_\alpha\ \text{ and }\ D\subseteq F^k\ \text{ for some } k.$$ 
An {\bf algebraic presentation} $( V_\alpha)_{\alpha \in D}$ of $S\subseteq K^n$ is a presentation of $S$ such that for each $\alpha \in D$, $V_\alpha$ is $K$-algebraic.
If $ S \subseteq K^n$ has an algebraic presentation (which is definable over $c$), we say $S$ is { \bf algebraically presentable (over $c$)}; if $S\subseteq K^n$ has an algebraic presentation which is $0$-definable, we say $S$ is {\bf $0$-algebraically presentable}. If $S$ plays no important role, we sometimes use the term {\it algebraic presentation} without mentioning $S$. For the rest of this section, $S$ is an algebraically presentable subset of its ambient space. It is easy to observe that:

\begin{lem}
Suppose  $(V_\alpha)_{ \alpha \in D}$  and  $(V'_{\alpha'})_{ \alpha' \in D'}$ are algebraic presentations definable over $c \in K^m$.
We can choose a combination (fiberwise intersection, fiberwise union, fiberwise product) of $(V_\alpha)_{ \alpha \in D}$ and $(V'_{\alpha'})_{ \alpha' \in D'}$  to also be an algebraic presentation definable over $c$. 
\end{lem}


\noindent
An algebraically presentable set can be considered geometrically simple, and next we show that $0$-algebraically presentable sets are also syntactically simple.

\begin{lem} \label{APsimple1}
Suppose $ S \subseteq K^n$ is algebraically presentable over $c$. Then we can find an algebraic presentation $(V_\alpha)_{ \alpha \in D}$ and a system of polynomials $P$ in $\qq(c)[w,x]$ such that $V_\alpha = Z\Big( P\big(\chi(\alpha),x\big)\Big)$ for all $\alpha \in D$.
\end{lem}

\begin{proof}
Suppose $S$ has an  algebraic presentation $( W_\beta)_{ \beta \in E}$ definable over $c$. For each choice $\mathscr{C}$ of $k$ and a system $P$ of polynomials in $\qq(c)[w,x]$, define $R_{\mathscr{C}} \subseteq F^k \times E$ by  
$$(\alpha, \beta) \in R_{\mathscr{C}}\ \text{ if and only if }\ W_\beta \ =\  Z\Big( P\big(\chi(\alpha),x\big)\Big).$$
Then the relation $R_{\mathscr{C}}$ is definable and so are its projections $R^1_{\mathscr{C}}$ on $F^k$  and $R^2_{\mathscr{C}}$ on $E$. For each $\beta \in E $, any automorphism of $K$ fixing $\chi(F)$ and $c$ will also fix $W_\beta$. 
Therefore, for each $\beta \in E$, $W_\beta$ is definable  in the field sense over $\qq\big(c, \chi( \alpha)\big)$ for some $\alpha \in F$. Hence, there is a choice $\mathscr{C}$ as above such that $\beta \in R^2_{\mathscr{C}}$. There are countably many such choices $\mathscr{C}$.
By replacing $(F,K;\chi)$ by an elementary extension, if necessary, we can without loss of generality assume that the structure $(F,K;\chi)$ is $\aleph_0$-saturated.
Hence, there are choices $\mathscr{C}_1, \ldots, \mathscr{C}_l$ such that  $E$ is covered by $R^2_{\mathscr{C}_i}$ as $i$ ranges over $\{1, \ldots, l\}$. 

For $i \in \{1, \ldots, l\}$, obtain $k_i$ and $P_i$ from  the choice $\mathscr{C}_i$ and let $D_i= R^1_{\mathscr{C}_i} \subseteq F^{k_i}$. 
Set $D = D_1 \times \cdots \times D_l$, $P= P_1\cdots P_l$ and $V_\alpha = Z\Big( P\big(\chi(\alpha),x\big)\Big)$. It is easy to check that the family $(V_\alpha)_{ \alpha \in D}$ satisfies the desired requirements.
\end{proof}

\noindent
We have a slightly different version of the above lemma which will be used later.

\begin{lem}  \label{APsimple2}
Suppose $ S \subseteq K^n$ has an algebraic presentation $( W_\beta)_{ \beta \in E}$ definable over $c$. We can find an algebraic presentation $(V_\alpha)_{ \alpha \in D}$ and systems $P_1, \ldots, P_l$ of polynomials in $\qq(c)[w,x]$, such that  $(V_\alpha)_{ \alpha \in D}$ is equivalent to $( W_\beta)_{ \beta \in E}$, $D\subseteq F^k$ is the disjoint union of $D_1, \ldots, D_l$, each definable over $c$, and $ V_\alpha = Z\big( P_i(\chi(\alpha),x)\big)$ for $i\in \{ 1,\ldots, l\}$ and $\alpha \in D_i$. 
\end{lem}


\begin{proof}
We get the choices $\mathscr{C}_1, \ldots, \mathscr{C}_l$ in exactly the same way as in the first paragraph of the proof of the preceding lemma.
By adding extra variables, if needed, we can arrange that $k_1 = \cdots = k_l = k$ where $k_i$ is taken from the choice $\mathscr{C}_i$. We define $D_i$ inductively. For each $i \in { 1,\ldots, l} $, set
$$D_i \ =\  \bigg\{ \alpha \in F^k \backslash ( \bigcup_{j<i} D_j): \text{ there is } \beta \in E \text{ with } W_\beta = Z\Big( P_i\big(\chi(\alpha),x\big)\Big) \bigg \}.$$
Let $D = \bigcup_{i=1}^l D_i$, and $(V_\alpha)_{ \alpha \in D} $ be given by $V_\alpha = Z \big( P_i(\chi(\alpha),x)\big)$. It is easy to check that $(V_\alpha)_{ \alpha \in D}$ is the desired algebraic presentation.
\end{proof}

\noindent
Next, we prove that $F$ is $0$-stably embedded into $(F, K; \chi)$.
\begin{lem}\label{StablyEmbedednessForParameterFreeSet}
If $ D \subseteq F^k$ is $0$-definable, then it is $0$-definable in the field $F$.
\end{lem}

\begin{proof}
By changing the model if needed, we can arrange that $ (F, K; \chi) $ realizes all the $0$-types. By Stone's representation theorem, it suffices to show that if $ \alpha $ and $\alpha'$ are arbitrary elements in $ F^k$ with the same $0$-type in the field $F$, then they have the same $0$-type. Fix such $\alpha$ and $\alpha'$. As  $F$ is a model of $ \text{ACF} $, there is an automorphism of $F$ sending $ \alpha $ to $ \alpha'$. This automorphism can be extended to an automorphism of $(F, K;\chi) $ by Corollary~\ref{AutExt}, so $ \alpha $ and $ \alpha'$ have the same  $0$-type. 
\end{proof}

\begin{prop}
If $S\subseteq K^n$ is $0$-algebraically presentable, then we can find a formula $\varphi(s)$ in the language of rings and a system of polynomials $P \in \qq[w,x]$ such that $S$ is defined by 
$$ \exists s \Big(\varphi(s)\wedge P\big(\chi(s),x\big)=0 \Big).$$
\end{prop}

\begin{proof}
This follows from Lemma~\ref{APsimple1} and Lemma~\ref{StablyEmbedednessForParameterFreeSet}.
\end{proof}

\noindent
We next show that $0$-definable sets are just boolean combinations of $0$-algebraically presentable sets. Towards this, we need a number of lemmas.

\begin{lem} \label{SaturatedExtension}
The model $(F, K; \chi)$ has an elementary extension $(F', K'; \chi')$ such that $F'= F$ and $K'$ is $|F'|^+$-saturated as a model of $\ACF$.
\end{lem}

\begin{proof}
This follows from Corollary~\ref{ElementarilyEmbeddable}.
\end{proof}

\noindent
The following lemma is well known about $\ACF$. The proof is a consequence, for example, of the results in \cite{LouSmith}. 

\begin{lem}\label{DOE1}
Let $( X_b)_{b\in Y}$ be a family of subsets of $K^n$ definable ($0$-definable) in the field $K$. We have the following:

\begin{enumerate}
\item (Definability of dimension in families)

\noindent
the set $Y_k= \big\{b \in Y : \mr_K(X_b) =k\big\}$ is definable ($0$-definable) in the field $K$;
\item (Definability of multiplicity in families)

\noindent
the set $Y_{k, l}= \big\{b \in Y : \mr_K(X_b) =k,\  \md_K(X_b) = l \big\}$ is definable ($0$-definable) in the field $K$;
\item (Definability of irreducibility algebraic families)

\noindent
if $X_b$ is $K$-algebraic for all $b \in Y$, then $Y_{\mathrm{ired}}= \{b \in Y : X_b \text{ is irreducible} \}$ is definable ($0$-definable) in the field $K$.
\end{enumerate}
\end{lem}

%\noindent
%In our case the preceding lemma has the following consequence:

\begin{cor}\label{DOE2}
Let $( X_b)_{b\in Y}$ be a definable (0-definable) family of subsets of $K^n$ with $X_b$ definable in the field $K$ for all $b \in Y$. Then we have the following:
\begin{enumerate}
\item (Definability of dimension in families)

\noindent
the set $Y_k= \big\{b \in Y : \mr_K(X_b) =k \big\}$ is definable ($0$-definable);
\item (Definability of multiplicity in families)

\noindent
the set $Y_{k, l}= \big\{b \in Y : \mr_K(X_b) =k,\  \md_K(X_b) = l \big\}$ is definable ($0$-definable);
\item (Definability of irreducibility in algebraic families)

\noindent
if $X_b$ is $K$-algebraic for all $b \in Y$, then $Y_{\mathrm{ired}}= \{b \in Y : X_b \text{ is irreducible} \}$ is definable ($0$-definable).
\end{enumerate}
\end{cor}

\begin{proof}
We first prove (1) for the definable case. Let  $( X_b)_{b\in Y}$ be a definable family as stated. For each $b \in Y$, there is a parameter free formula $\varphi(w, x)$ in the language of rings such that there is $c \in K^{k}$ with $X_b$ defined by $\varphi(c, x)$.
We note that there are only countably many parameter free formulas $\varphi(w,x)$ in the language of rings. 
By a standard compactness argument and a simple reduction we arrange that there is  a formula $\varphi(w,x)$ such that for any $b$ in $Y$, there is $c \in K^{k}$ such that $X_b$ coincides with $X'_c$  where $X'_d \subseteq K^n$ is defined by $\varphi(d, x)$ for $d \in K^k$.  With $Y_k$ as in the statement of the lemma, we have
$$Y_k\ =\ \big\{ b \in Y : \text{ there is } c\in K^{k} \text{ such that } X_b = X'_c \text{ and } \mr_K(X'_c) = k  \big\}. $$
The definability of $Y_k$ then follows from (1) of the preceding lemma. 


For the 0-definable case, we can arrange that $(F, K; \chi)$ is $\aleph_0$-saturated and check that any automorphism of the structure fixing $( X_b)_{b\in Y}$ also fixes $Y_{k}$ for all $k$.
The statements (2) and  (3) can be proven similarly.
\end{proof}

\noindent
Towards obtaining the main theorem, we need the following auxiliary lemma.

\begin{lem}\label{QuanRedLem}
For $a \in K^n$, choose $V \subseteq K^n$ containing $a$ and definable in the field $K$ over $ \qq\big( \chi(F^\times)\big) $  such that $\big(\mr_K(V), \md_K(V)\big)$ is lexicographically minimized with respect to these conditions. Likewise, choose $V'\subseteq K^n$ for $a' \in K^n$.
If there are $ \alpha, \alpha' \in F^k$ of the same $0$-type in $F$ and a system $P$ of polynomials in $\qq[w,x]$  with $ V = Z\big(P( \chi( \alpha), x) \big)  $ and $V' =Z\big(P( \chi( \alpha'), x) \big)$, then  $ a$ and $a'$ have the same $0$-type.
\end{lem}

\begin{proof}
Using Lemma~\ref{SaturatedExtension}, we can arrange that $ K$ is $ |F|^{+}$-saturated as a model of $\ACF$. Suppose $a,a', V, V', \alpha, \alpha'$ and $P$ are as stated. Then we get an automorphism $\sigma_F$ of $F$ mapping $\alpha$ to $\alpha'$.
By Corollary~\ref{AutExt}, this can be extended to an automorphism  $ (\sigma_F, \sigma_K)$ of $ (F, K; \chi)$. In particular, 
$$\sigma_K: \chi(\alpha) \mapsto \chi(\alpha')\ \text{ and }\ \sigma_K(V) = V'.$$
Then $V'$ contains $\sigma(a)$, is defined over $ \qq\big( \chi(F^\times)\big) $ and $\big(\mr_K(V'), \md_K(V')\big)$ achieves the minimum value under these conditions.
Hence, for an algebraic  set $W \subseteq K^n$ definable in the field $K$ over $ \qq\big( \chi(F^\times)\big) $, 
$$\sigma(a) \in W\ \text{ if and only if }\ \big(\mr_K(V'\cap W), \md_K(V' \cap W)\big)\ =\ \big(\mr_K(V'), \md_K(V')\big). $$
By the choice of $V'$, exactly the same statement holds when $\sigma(a)$ replaced with $a'$. 
By the quantifier elimination of $\ACF$, $ \sigma(a) $ and $ a'$ have the same type over $ \qq\big( \chi(F^\times)\big) $ in the field $K$.
As $ K$ is $ |F|^{+}$-saturated, there is an automorphism $\tau_K$ of $K$ fixing $\qq\big(\chi(F^\times)\big)$ pointwise and mapping $\sigma(a)$ to $a'$.
It is easy to check that $(\sigma_F, \tau_K \circ \sigma_K)$ is an automorphism of $(F, K; \chi)$ mapping $a$ to $a'$. Therefore, $a$ and $a'$ have the same $0$-type.
\end{proof}



\begin{thm} \label{QuanRed}
If $X \subseteq K^n$ is $0$-definable, then $X$ is a boolean combination of $0$-algebraically presentable subsets of $K^n$.  
\end{thm}

\begin{proof}
We say $a, a' \in K^n$ have the same 0-ap-type if they belong to the same $0$-algebraically presentable sets. 
By changing the model, if needed, we can arrange that $ (F, K; \chi) $ realizes all the $0$-types. 
By Stone's representation theorem, it suffices to show that if $ a$ and $ a' $ are arbitrary elements in $  K^n $ with the same $0$-ap-type then they have the same $0$-type. 

\noindent
Fix $a$ and $a'$ in $K^n$ with the same  $0$-ap-type. 
Choose $V \subseteq K^n$ containing $a$ and definable in the field $K$ over $ \qq\big( \chi(F^\times)\big) $  such that $$\big(\mr_K(V), \md_K(V)\big) \text{ is lexicographically minimized with respect to these conditions}. $$ 
Moreover, pick $k$, $D \subseteq F^k$ $0$-definable in the field $F$, $\alpha \in D$ and a system $P$ of polynomials in $\qq[w,x]$ such that $V = Z\big( P( \chi( \alpha), x)\big)$ and $$\big(\mr_F(D), \md_F(D) \big) \text{ is minimized under these conditions}.$$
We will find $\alpha'$ and $ V'$ in order to use Lemma~\ref{QuanRedLem}. 
Set $$E\  =\ \bigg\{ \beta \in D : \text{ if } V_\beta = Z\Big( P\big( \chi( \beta), x\big)\Big) \text{, then } \big(\mr_K(V_\beta), \md_K(V_\beta)\big)=\big(\mr_K(V), \md_K(V)\big)\bigg\} .$$ 
We note that $E $ is $0$-definable by Corollary~\ref{DOE2}, and so by Lemma~\ref{StablyEmbedednessForParameterFreeSet}, $ E$ is also $0$-definable in the field $F$. 
As $\alpha$ is in $E$, 
$$ \big(\mr_F(E), \md_F(E) \big) \ = \ \big(\mr_F(D), \md_F(D) \big) $$ by the choice of $D$. 
With $S = \big( Z\big(P(\chi(\beta),x)\big)\big)_{\beta \in E}$, we have $a\in S$, and so we also have $a' \in S$ since $a$ and $ a'$ have the same 0-ap-type. Hence, there is $\alpha'\in E$ such that $a'$ is an element of $V' = Z\big(P(\chi(\alpha'),x)\big)$.

We next verify that $ \alpha'$ and $ V'$ satisfy the conditions of Lemma~\ref{QuanRedLem}. It will then follow that $a$ and $a'$ have the same $0$-type. We first check that
$$\big(\mr_K(V'), \md_K(V')\big)\ =\  \min\Big\{ \big(\mr_K(W'), \md_K(W')\big) : W' \subseteq K^n \text{ is } K\text{-algebraic, } a' \in W' \Big\}.$$
As $ \alpha'$ is in $E$, $$\big(\mr_K(V'), \md_K(V')\big)\ =\ \big(\mr_K(V), \md_K(V)\big).$$ 
Suppose towards a contradiction that there is an irreducible algebraic set $ W' \subseteq K^n$ containing $a'$ with $$ \big(\mr_K(W'), \md_K(W')\big) \ <_{\text{lex}}\ \big(\mr_K(V'), \md_K(V')\big).$$ 
We can do the same construction as above in the reverse direction to get $ W''$ with 
$$ \big(\mr_K(W''), \md_K(W'')\big)\ <_{\text{lex}} \  \big(\mr_K(V), \md_K(V)\big)$$ containing $a$, a contradiction to the choice of $V$. 
We next check that $\alpha$ and $\alpha'$ have the same $0$-type in the field $F$. Suppose otherwise. Let $ D'$ be the  smallest 0-definable $F$-algebraic set containing $\alpha'$.
Then $$ \big(\mr_F(D'), \md_F(D') \big) \ <_{\text{lex}}\ \big(\mr_F(D), \md_F(D) \big).$$
Do the same construction in the reverse direction again to get $ \alpha'' \in D'$ such that $a $ satisfies $ P\big( \chi( \alpha''), x\big) =0$.
If $ D''$ is the smallest 0-definable $F$-algebraic set containing $\alpha''$, then $$ \big(\mr_F(D''), \md_F(D'') \big)\ \leq_{\text{lex}}\  \big(\mr_F(D'), \md_F(D') \big)\ <_{\text{lex}} \ \big(\mr_F(D), \md_F(D) \big), $$ a contradiction to our choice of $D, \alpha$ and $P$.
\end{proof}


\noindent
Suppose $D \subseteq F^k$ is definable. By Proposition~\ref{transferRemark}, $D$ can be identified with $\chi(D) $, which has a simple description by the preceding theorem. In the rest of the section, we give an improvement of the above result for this special case. 
For a system  $P$ in $K[w]$, we abuse the notation and let $Z\big(P(\chi(s))\big) \subseteq F^k$ be the set defined by  $P\big(\chi(s)\big)=0$.

\begin{lem}
For each $k$ there is a system $Q$ of polynomials in $ F[s] $ such that the set defined by $ \chi(s_1) + \cdots+ \chi(s_k) =0 $ is $Z(Q)$.
\end{lem}
\begin{proof}

For $ I \subseteq \{1, \ldots, k\} $, let $ \sum_{i\in I} \chi(s_i) \ndeq 0 $ denote the system which consists of $\sum_{i\in I} \chi(s_i) = 0 \text{ and }\sum_{i\in I'} \chi(s_i) \neq 0$ for each non-empty proper subset $I'$ of $I$. By Mann's theorem, there are $\alpha^{(1)}, \ldots, \alpha^{(l)}$ in $F^I$, such that the set defined by $ \sum_{i\in I} \chi(s_i) \ndeq 0  $ precisely  consists of  $ \beta \alpha^{(j)}$  with $\beta \in F^\times$ and $j \in \{1, \ldots, l \}$. Hence, if $ I \subseteq \{1, \ldots, k\} $, then there is a system $ Q_{I} $ of polynomials in $F[s]$  such that the set defined by $ \sum_{i\in I} \chi(s_i) \ndeq 0 $ is $ Z(Q_I)$.


Consider all the partitions $ \mathscr{P} $ of the set $\{1,\ldots, k\}$ into non-empty subsets. Then we have the set defined by $\chi(s_1) + \cdots + \chi(s_k) = 0 $ is $ \bigcup_{\mathscr{P}} \bigcap_{I \in \mathscr{P}} Z(Q_I)$. Note that finite unions and finite intersections of $F$-algebraic sets are again $F$-algebraic. Thus, we can find a system $Q$ of polynomials in $ F[s] $ as desired.
\end{proof}

\begin{lem} \label{Continuity}
The map $ \chi: F^k \to K^k $ is continuous. 
\end{lem}

\begin{proof}
For the statement of the lemma, we need to show that if $ V\subseteq K^k$ is $K$-closed then $ \chi^{-1}(V) $ is $F$-closed. It suffices to show that if $P$ is in $ \qq[w, x] $ and $ a$ is a tuple of elements in $K$, then  $ Z\big(P(\chi(s), a)\big) $ is $F$-algebraic. 
Choose a linear basis $ B$ of $\qq\big(\chi(F^\times), a\big)$ over $\qq\big(\chi(F^\times)\big)$. 
Then 
$$P\big(\chi(s), a\big) \ = \ P_1\big(\chi(s)\big)b_1+\cdots+ P_m\big(\chi(s)\big)b_m $$ where $P_i $ has coefficients in $ \qq(\chi(F^\times))$, $b_i \in B$ for $i \in \{1, \ldots, m\}$, and $b_i \neq b_j$ for distinct $i,j \in \{ 1 , \ldots, m\}$. 
Therefore,  $P\big(\chi(s), a\big) = 0$ is equivalent to $ P_i\big(\chi(s)\big)=0$ for all $i \in \{1, \ldots, m \}$. 
Furthermore, for each $i \in \{ 1, \ldots,m \} $, $ P_i\big(\chi(s)\big) =0 $ is equivalent to an equation of the form 
$$ \chi\big(M_1(s, \alpha)\big)+\cdots+\chi\big(M_{l_i}(s, \alpha)\big)\ =\ 0 $$ 
where $ \alpha$ is a tuple of elements in $F$, and $ M_j$ is a monomial for $j \in \{1, \ldots, l_i\}$. By the result of the preceding lemma, for each $i \in \{1, \ldots, m\}$, the polynomial equation $P_i\big(\chi(s)\big)=0$ is equivalent to a system $ Q_i\big( M_1(s, \alpha), \ldots, M_{l_i}(s, \alpha)\big) =0 $. Thus, $Z\big(P(\chi(s), a)\big) = \bigcap_{i=1}^k Z\big(P_i(\chi(s))\big)$ is $F$-algebraic.
\end{proof}

\begin{thm} \label{StablyEmbbed}
Let $D$ be a subset of $F^k$. If $D$ is definable, then $D$ is definable in the field $F$. Moreover, when $D$ is $0$-definable, $D$ is $0$-definable in the field $F$. If $D = \chi^{-1}(V)$ with $K$-algebraic $V \subseteq K^n$, then $D$ is an $F$-algebraic set. Moreover, when $V = Z(P)$ with $P$ a system in $\zz[w]$, $D = Z(Q)$ with $Q$ a system in $\zz[s]$. 
\end{thm}

\begin{proof}

We prove the first assertion. It suffices to show that if $ X \subseteq K^k$ is definable, then $ \chi^{-1}(X) $ is definable in the field $F$.
By Theorem~\ref{QuanRed}, we only need to show that if $S \subseteq K^{k+m}$ is $0$-algebraically presentable and $X = \big\{a : (a,b) \in S \big\}$ with $b \in K^m$ then  $ \chi^{-1}(X) $ is definable in the field $F$. 
It is easy to see that $ X $ is defined by a formula of the form 
$$ \exists t \Big( \varphi(t) \wedge P\big(w, \chi(t)\big) = 0 \Big)\ \text{ where } P \text{ is a system of polynomials in } K[w, x].$$
Let $V$ be  $ Z(P)$. Then by the preceding lemma, $ \chi^{-1}(V) $ is $Z(Q)$ for some system $Q$ in $F[s, t]$. Hence, $ \chi^{-1}(X)$, which is defined by  $ \exists t \big( \varphi(t) \wedge P(\chi(s),\chi(t)) =0 \big) $, is also defined by $ \exists t \big( \varphi(t) \wedge Q(s,t) =0 \big)  $. Thus, $ \chi^{-1}(X)$ is definable in the field $F$ as desired.  The second assertion is just Lemma~\ref{StablyEmbedednessForParameterFreeSet}. The third assertion is Lemma~\ref{Continuity}. The forth assertion follows from the second and third assertions. 
\end{proof}