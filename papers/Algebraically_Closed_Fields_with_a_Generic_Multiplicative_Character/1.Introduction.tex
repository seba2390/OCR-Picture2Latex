\section{Introduction}

\noindent 
Fields with characters occur in many places; see for example 
Kowalski~\cite{Kowalski} for a case where also definability plays a role.
This suggested to us to
look for model-theoretically tame pairs of fields with character maps 
between them.  

Throughout, $(F, K; \chi)$ is a structure where $F$ and $K$ are integral 
domains (usually fields), and $\chi: F \to K$ satisfies 
$\chi(ab)=\chi(a)\chi(b)$ for all $a,b\in F $, $\chi(0)=0 \text{ and } \chi(1)=1.$
Then $(F, K; \chi)$ is naturally a structure in  the $2$-sorted language $L$ which consists of two disjoint copies of the language of rings, augmented by a unary function symbol $\chi$. We call $\chi$ with the above properties a \textbf{character}.


We are particularly interested in the cases where $F$ is an algebraic closure $\ff$ of a finite field, $K$ is the field $\cc$ of complex numbers and $\chi: \ff \to \cc$ is {\em injective}. 
From now on, we let $(\ff, \cc; \chi)$ range over structures with these properties.
Corollary~\ref{UniquenessOfChar} below says 
that for each prime $p$, there is up to $L$-isomorphism exactly one  $(\ff, \cc; \chi)$
such that $\chara(\ff) = p$. 
In this paper we show that the 
$L$-theory $\text{Th}(\ff, \cc; \chi)$ is tame in various ways. For precise statements we
need some more terminology. 

Let $(F,K;\chi)$ be given. For a tuple $\alpha=(\alpha_1,\dots, \alpha_n)\in F^n$, $n \in \nn^{\geq 1}$, and $k=(k_1,\dots,k_n)\in \nn^n$ we set $\alpha^k:=\alpha_1^{k_1}\cdots\alpha_n^{k_n}$. We call $\alpha$ {\bf multiplicatively dependent}
if $\alpha^k=\alpha^l$ for some distinct $k,l\in \nn^n$, and {\bf multiplicatively independent} otherwise.
We say that \( \chi: F \to K\) is { \bf generic} if it is injective and 
for all multiplicatively independent $\alpha \in F^n$, $n \in \nn^{\geq 1}$,
the tuple $\chi(\alpha):=\big(\chi(\alpha_1),\dots, \chi(\alpha_n)\big) \in K^n$ is algebraically independent in the fraction field of $K$ over its prime field.

\begin{thm}
There is a recursive set $\TN$ of $\forall\exists$-axioms in $L$ such that:
\begin{enumerate}
\item for all $(F, K;\chi)$, $(F, K;\chi)\models \TN$ if and only if $F$ and $K$ are algebraically closed fields, $\chara(K)=0$ and $\chi:F \to K$ is generic;
\item for all $p$ prime, if $\chara(\ff)=p$, then $(\ff, \cc; \chi)\models \TN$.
\end{enumerate}
\end{thm}

\noindent If $p$ is either prime or zero, let $\TN_p$ be the set of $\forall\exists$-axioms in $L$ obtained from $\TN$ by adding the statements expressing  $\chara(F)=p$ where $(F, K;\chi)$ is an $L$-structure. 


\noindent Let $\kappa, \lambda$ be (possibly finite) cardinals. In Section 3 we prove the following classification result. (If $p=0$, set $\ff_p:=\qq$.) 

\begin{thm} \label{1.2}
For any $p,\kappa$ and $\lambda$, there is up to isomorphism a unique model $(F, K; \chi)$ of $\TN_p$ such that $\trdeg(F \mid \ff_p) = \kappa, \trdeg\big(K \mid \qq( \chi(F))\big)= \lambda$. 
\end{thm}

\noindent
By the wealth of results by Shelah in \cite{Shelah}, we can get the following:

\begin{cor}
$\TN_p$ is superstable, shallow, without the dop, without the otop, without the fcp.
\end{cor}

\noindent
By an analogue of Vaught Test, we have:

\begin{thm}
$\TN_p$ is complete.
\end{thm}

\noindent
In section 4 we characterize the substructures of models of $\TN_p$:  
\begin{prop}\label{sub} Given $(F, K; \chi)$, the following are equivalent:
\begin{enumerate}
\item $(F, K; \chi)$ is a substructure of a model of $\TNp$;
\item$\chi$ is generic and $\chara(F)=p$.
%\item The structure $(F, K; \chi)$ is a model of \( \TNp(\forall) \).
\end{enumerate}
\end{prop}

\noindent
When is a substructure of a model of  $\TNp$ an 
elementary submodel?  It is not enough that the substructure is 
a model of $\TNp$:

\begin{prop}
$\TN_p$ is not model complete.
\end{prop}

\noindent
To deal with the above question we define
a {\bf regular submodel} of a model $(F', K';\chi')$ of $\TN_p$ to be
a substructure $(F, K;\chi)\models \TN_p$ of $(F', K';\chi')$ such that
 $\qq\big(\chi(F')\big) $ is linearly disjoint with $ K$ over $\qq\big(\chi(F)\big)$ in $K'$. 
The more complicated notion of {\bf regular $L$-substructure} will be defined in Section 4. Below we fix a set  $\TNp(\forall)$ of universal 
$L$-sentences whose models are the substructures of models of $\TNp$, as in
Proposition~\ref{sub}. 

\begin{thm}
\(\TNp \) is the {\rm regular} model companion of \( \TNp(\forall) \). That is:
\begin{enumerate}
\item for models of $\TNp$, the notions of {\rm regular submodel} and 
{\rm elementary submodel} are equivalent;
\item every model of \( \TNp(\forall) \) is a regular substructure of a model of \( \TNp \).
\end{enumerate}
\end{thm}

\noindent In Section $5$, we show that every definable set in a fixed model $(F, K; \chi)$ of $\TN_p$ has a simple description. This is comparable to the fact that every definable set in a model of $\ACF$ is a boolean combination of algebraic sets. A set $S \subseteq K^n$ is {\bf algebraically presentable} if 
$$S\  =\ \bigcup_{\alpha \in D}V_\alpha$$
for some definable $D\subseteq F^m$ and
definable family $\{ V_\alpha\}_{\alpha \in D}$ of $K$-algebraic subsets of $K^n$. 
 Algebraically presentable sets should be thought of as geometrically simple. They are also
existentially definable of a particular form. We also define in Section~$5$ the related notion of {\bf $0$-algebraically presentable} sets. The main result is:

\begin{thm}
If $X \subseteq K^n$ is definable, then $X$ is a boolean combination of algebraically presentable subsets of $K^n$. Furthermore, if $X$ is $0$-definable, $X$ is a boolean combination of 0-algebraically presentable subsets of $K^n$.
\end{thm}

\noindent
The general case where a definable set is not a subset of $K^n$ can be easily reduced to the above special case.

\newpage
\noindent We have better results for definable subsets of $F^m$:

\begin{thm}
Let $D\subseteq F^m$. If $D$ is definable, then $D$ is definable in the field $F$. If $D$ is $0$-definable, then $D$ is $0$-definable in the field $F$. 
Suppose $D = \chi^{-1}(V)$ with  $V\subseteq K^m$ a $K$-algebraic set. Then $D$ is $F$-algebraic. If $V$ is defined over $\qq$ in the field sense, then $D$ is defined over $\ff_p$ in the field sense. 
\end{thm}

\noindent
Still working in a fixed model $(F, K; \chi)$ of $\TN_p$, we show in Section $6$ that every definable set has another description which is comparable to the fact that every definable set in a model of $\ACF$ is a finite union of quasi-affine varieties. A special case of the above description is when $X = \bigcup_{\alpha \in D} V_\alpha$ where $\{ V_\alpha\}_{\alpha \in D}$ is a definable family of varieties over $K$ such that 
\begin{enumerate}
\item for some $k\in \nn$, $D$ is a subset of $F^k$ definable in the language of rings,
\item $V_\alpha$  and $V_\beta$ are disjoint for distinct $ \alpha, \beta \in D $.
\end{enumerate}

\noindent We can show that not every definable set $X \subseteq K^n$ has a description as in the above special case.
However, a description approximating the above picture can be obtained. This in particular allows us to define a {\bf geometric rank} $\gr$  and a {\bf geometric degree} $\gd$ on the definable sets. 
\begin{comment}
In the above special case, let $\mr_K$ be the Morley rank  of $V_\alpha$ in the $\ACF$-model $K$ for some $\alpha \in D$, let $\mr_F, \md_F$ be the Morley rank and Morley degree of $D$ in the $\ACF$-model $F$. (Note that we can replace Morley rank and Morley degree here by any other reasonable notions of dimension and multiplicity.) Then the dimension and multiplicity of the definable set $X$ is $\omega\cdot \mr_K + \mr_F$ and $\md_F$ respectively.
\end{comment}


In Section $7$, we show that the geometric rank and the geometric degree defined in the previous section coincide with the model-theoretic notions of Morley rank and Morley degree.
Using this, it is easy to deduce that the theory is $\omega$-stable. We then study some behaviors of these notions of ranks.

\begin{prop}
If $X \subseteq K^n$ and $ X'\subseteq K^{n'}$ are definable, $\gr(X) = \omega\cdot \rho_{{}_K} + \rho_{{}_F}$ and $  \gr(X') = \omega\cdot \rho'_{{}_K} + \rho'_{{}_F}$ with $\rho_{{}_K}, \rho_{{}_F}, \rho'_{{}_K}, \rho'_{{}_F} \in \nn$,  then  $\gr(X\times X') = \omega\cdot (\rho_{{}_K}+\rho'_{{}_K}) +\rho_{{}_F}+\rho'_{{}_F}$.
\end{prop}

\begin{thm}
Suppose $ \{ X_b\}_{b \in Y}$ is a definable family of subsets of $K^n$. Then for each ordinal \(\rho\), the set \( \big\{ b \in Y :  \gr(X_b)= \rho \big\} \) is definable.
\end{thm}

\noindent
The technique we developed might also provide a step towards proving that $\TN_p$ has the
{\em definable  multiplicity} property. 
We use the geometric understanding of Morley rank and Morley degree to classify strongly minimal sets up to non-orthogonality:

\begin{prop}
For any strongly minimal set $X \subseteq K^n$ there is a finite-to-one definable map from $X$ to $F$.
\end{prop}

\noindent
The structure $(\ff, \cc; \chi)$ is similar to various known structures, 
for example $(\cc, \qa)$ where $\qa$ is the set of algebraic numbers regarded as an additional unary relation on $\cc$.
The study of the latter stretches back to Robinson (see ~\cite{Robinson}). Analogues of some of our results for $(\cc, \qa)$ seem to be known as folklore; see
for example \cite{Louominpair}. However, our structure is mathematically even more closely related to $( \cc, \uu)$ where $\uu \subseteq \cc$ is the group of all roots of unity regarded as an additional unary relation.
In fact, we can almost view $(\ff, \cc; \chi)$ as $( \cc, \uu)$ with some 
extra relations on $\uu$.
In consequence, several results of this paper are either directly implied or easy adaptations of results in \cite{Zilber} and \cite{DriesGun}.
These include axiomatization, $\omega$-stability, quantifier reduction; the corresponding result of $(\kappa, \lambda)$-transcendental categoricity is known, according to Pillay,  but not written down anywhere.
There are also several results that hold in the above mentioned two structures and ought to have suitable analogues in our structure but we have not proven them yet.
These include the study of imaginaries and definable groups;  see \cite{Pillay} and \cite{Haydar}.


\noindent On the other hand, some of our results are new, which also yield more information on the structures $(\cc, \qa)$ and $( \cc, \uu)$ as well.
Through the notion of genericity, we obtain a more conceptual characterization of the class of models of $\TN_p$ other than using the axioms.  From \cite{Zilber},
we can already see that every model of $\TN_p$  satisfies the properties of this characterization. In this paper, we show the reverse direction.
It is clear that one can obtain a characterization of the class of models of $\text{Th}( \cc, \uu)$  in the same way.
Even though both use Mann's theorem in an essential way, our axiomatization strategy is slightly different from the strategy used in \cite{Zilber} and \cite{DriesGun}. 
This modification, in particular, allows us to also axiomatize the class of substructures of the models and achieve the regular model companion result mentioned above. 
The regular model companion result should have analogues for $( \cc, \qa)$ and $(\cc, \uu)$ as well. To our knowledge, the method in Sections $6$ and $7$ and the result about definability of Morley Rank in families is not known.
We expect that this method  can be applied to $( \cc, \qa)$ and $(\cc, \uu)$ and generalized further to study the appropriate notions of dimension and multiplicity in other types of pairs.

A natural continuation of our project is to study the expansion $(\ff, \cc; \chi, \rr)$ of $(\ff, \cc; \chi)$ where $\rr\subseteq \cc$ is the set of real numbers.  At the definability level, this amounts to also including the metric structure on $\cc$ into the picture. Towards this end, the second author has considered a reduct of $(\ff, \cc; \chi, \rr)$; see \cite{Minh} for details.