\section{Rank, degree and their behaviors}

\begin{comment}

\begin{lem}
Suppose \((F, K; \chi) \elsub (E, L; \chi) \) are models of \(\TNp \), \( (F, K; \chi)\) countable, \( (E, L; \chi) \)  \(\aleph_1 \)-saturated, and \(a,b \) are elements of \(L\). We have the following:
\begin{enumerate}
\item If \(a,b \) are both transcendental over \(K(\chi(E))  \) then they have the same type over \((F, K; \chi)\).
\item If \(a\) is algebraic over \(K(\chi(E))\), then there is a tuple \(e_a\) of elements of \(E\) and a polynomial \(Q_a(x,y)\) with coefficient in \(K\)  such that \(Q_a(x,e_a) \) is the minimal polynomial of \(a\) over \(K(\chi(E))\). Furthermore, if \(b\) is algebraic over \(K(\chi(E))\), \(e_b, Q_b\) are defined in similar way, \(e_a, e_b \) have the same type over \(F\), \(Q_a=Q_b \), then \( a,b\) have the same type over \((F, K; \chi)\).
\end{enumerate}
\end{lem}

\end{comment}

\noindent
We continue working in a fixed model \( (F, K; \chi) \) of \( \TN_p\) and keeping the notations and conventions of the preceding section. We show in this section that the geometric rank and geometric degree defined in the preceding section agrees with Morley rank and Morley degree. 


It is also convenient to define the so-called Cantor rank and Cantor  degree for a Boolean algebra. Suppose $B$ is a Boolean algebra and $b \in B$, we set $\crk(b) = -\infty$ for $b = 0$ and $\crk(b) \geq 0$ if $b \neq 0$. For an ordinal $\alpha$, we set $\crk(b) \geq \alpha $ if for all $\beta< \alpha$ there is an infinite disjoint family $\{ b_n \}_{n \in \nn}$ with $\crk(b_i) \geq \beta$ such that $b\wedge b_i = b_i$. Set 
$$\crk(b)\ =\ \max \big\{ \alpha : \crk(b) \geq \alpha \big\}\ \text{ if  this exists and }\ \crk(b)\ =\ \infty\ \text{ otherwise}.$$
If $b = b_1 \vee \cdots \vee b_m$ with $ b_1, \ldots, b_m$ disjoint and $ \crk(b_i) = \crk(b) < \infty $ for $i \in \{ 1, \ldots, m \}$, we say $\cdg(b) \geq m$. Set 
$$\cdg(b)\ =\ \max \big\{ m : \cdg(b) \geq m \big\}.$$
We apply the above definition in the case where $B$ is the Boolean algebra of definable subsets of $F^m$ or of $K^n$. Note that if $(F, K; \chi) $ is $\aleph_0$-saturated, then $\crk(X)= \mmr(X)$ and $\cdg(X)= \mmd(X)$ for all definable $X \subseteq K^n$.



\begin{comment}
\begin{lem}
MR$(F) = 1$ and MR$(K) \geq \omega $.
\end{lem}

\begin{proof}
As $F$ is not finite, then MR$(F)\geq 1$.
On the other hand, $F$ is stably embedded into $(K, F; \chi)$, and so if MR$(F) \geq 2$, then we would get $X_1, X_2, \ldots$ infinite definable subsets of $F$, disjoint from each other. Using stably embeddedness, all $X_i$ are definable in the field structure of $F$ too, therefore they are either finite or co-finite. Thus a contradiction. Therefore MR$(F) = 1$.

Now, as there is a one-to-one and onto correspondence between $F$ and $\chi(F)$, it follows, that MR$(\chi(F)) = 1$ as well. Without loss of generality, we can assume $K$ is enough saturated, as Morley rank stays the same when we go to an elementary extension. Take infinitely many elements $a_1, a_2, \ldots$ that are algebraically independent over $\qq(\chi(F))$. Then there is a bijection between the set $a_1 \chi(F) + a_2 \chi(F) + \ldots +a_n \chi(F) \subseteq K$ and the cartesian $n$-th power of ${\chi(F)}$, i.e. ${\chi(F)}^n$, which has Morley rank $n$ by, again, stably embeddedness. Thus, MR$(K)\geq \omega$.
\end{proof}
\end{comment}

\begin{comment}

\begin{cor}
MR$(K^n)\geq \omega \cdot n$.
\end{cor}
\begin{proof}
Obvious.
\end{proof}

To prove, that MR$(K)=\omega$, it suffices to prove that for any partition of $K$ into union of disjoint definable sets, one of them has finite Morley rank. As we know from the previous section, $K$ is irreducible of dimension $(1, 0)$, so if we take any partition of $K$ into disjoint definable sets, at least one of them will have dimension strictly smaller than $(1, 0)$. Thus, its dimension is $(0, d)$ for some $d$, and by induction hypothesis, it has Morley rank $d < \omega$. Therefore MR$(K)=\omega$.
\end{comment}

\begin{prop}
If $D \subseteq F^m$ is definable, then $\mmr(D)= \crk(D)=\mr_F(D)$ and $\mmd(D)=\cdg(D)=\md_F(D)$.
\end{prop}
\begin{proof}
By Theorem~\ref{StablyEmbbed} if $D \subseteq F^m$ is definable, then $D$ is already definable in $F$ as a field. The conclusion follows.
\end{proof}

\noindent Let $P$ be an $m$-ary second-order property about definable sets in models of $\TN_p$ and $X_1, \ldots, X_m$ be definable in $(F, K;\chi)$.  We say $P(X_1, \ldots, X_m)$ is {\bf preserved under elementary extensions} if for every elementary extension $(F', K';\chi')$ of $(F, K;\chi)$, we have that $P(X_1, \ldots, X_m)$ is equivalent to $P(X_1', \ldots, X'_m)$ where $X'_i$ is defined by the $L$-formula with parameters in $K$ defining $X_i$ for each $i \in \{1, \ldots, m\}$. We define being {\bf preserved under elementary extensions} likewise for $f(X_1, \ldots, X_m)$ where $f$ is an $m$-ary function on definable sets in models of $\TN_p$. 

\begin{lem} \label{Preservative}
If $X\subseteq K^n$ is definable, then $ \gr(X)$ and $\gd(X)$ are preserved under elementary extensions.
\end{lem}

\begin{proof}
We first note that if $T, V \subseteq K^n$, then the property that $V$ is the closure of $T$ in the $K$-topology is preserved under elementary extensions. Suppose $X \subseteq K^n$ has a geometric pc-presentation $\{T_\alpha \}_{\alpha \in D}$ with primary index quotient $\widetilde{D}$. This fact is preserved under elementary extensions as the notion of geometric pc-presentation is defined using notions of closure in $K$-topology, $K$-Morley rank and degree, $F$-Morley rank and degree. Finally, $\gr(X)$ and $\gd(X)$ are calculated using $\{T_\alpha \}_{\alpha \in D}$, $\widetilde{D}$, $K$-Morley rank, $F$-Morley rank and $F$-Morley degree which are invariant under elementary extensions. Hence, the conclusion follows.
\end{proof}


\begin{lem}
If $X \subseteq K^n$ is definable,
then $\gr(X) = \crk(X)$ and $\gd(X)=\cdg(X)$. 
\end{lem}

\begin{proof}
Suppose $X \subseteq K^n$ is definable. Let $\{ T_\alpha \}_{\alpha \in D }$ be a geometric pc-presentation of $X$ with primary index quotient $\widetilde{D}$. 

First, consider the case when $\gr(X)=0$. Using Corollary~\ref{grbehavior2}, we can reduce to the case when $\gd(X)=1$.  
Then  $|\widetilde{D}| = 1$ and for each $\tilde{\alpha} \in \widetilde{D}$, $T_\alpha$ is finite. 
It follows from the latter that  if $T_\alpha \sim T_\beta$ then $T_\alpha =T_\beta$ for $\tilde{\alpha}, \tilde{\beta} \in \widetilde{D}$. 
Therefore, for each $\tilde{\alpha} \in \widetilde{D}$, no $P \in K[w, x]$ divides $T_\alpha$. Hence, for each $\tilde{\alpha} \in \widetilde{D}$, $T_\alpha$ has only one element. Thus $|X| = 1$, $\crk(X)= \gr(X) =0$ and $\cdg(X)=\gd(X)=1$.

Towards a proof by induction, suppose we have shown the statement for all $Y$ with $\gr(Y)< \gr(X)$. We will next show that $\crk(X)\geq \gr(X)$.
Again by using Corollary~\ref{grbehavior2}, we can reduce to the case when $\gd(X)=1$.
First, consider the case when $\mr_F(\widetilde{D})=\gr_F(X)>0$.
We can choose disjoint family $\{\widetilde{D}_i \}_{i \in \nn}$ of $F$-definable subsets of $\widetilde{D}$ such that 
$$\mr_F(\widetilde{D}_i)\ =\ \mr_F(\widetilde{D}) - 1\ \text{ for }\ i\in \nn.$$
Let $X_i = \bigcup_{\alpha \in D_i} T_\alpha$ for $i\in \nn$.
Then $\gr(X_i)= \gr(X)-1 \text{ for all } i\in \nn$. Moreover, $$\gr_K(X_i \cap X_j)\ <\ \gr_K(X)\ \text{ and so }\ \gr(X_i \cap X_j)\ <\ \gr(X)-1\ \text{ for distinct }\ i,j \in \nn .$$
By induction hypothesis, $\crk(X_i) = \gr(X)-1 $ for all $i \in \nn$ and $\crk(X_i \cap X_j)<\gr(X)-1$ for distinct $i, j \in \nn$.
Thus, $\crk(X) \geq \gr(X)$.

We continue showing that $\crk(X)\geq \gr(X)$ for the remaining case when $\gd(X)=1$, $\mr_F(\widetilde{D})=\gr_F(X)=0$. 
%Using the induction hypothesis, we can reduce to the case that $D= \dot{D}$. 
Then $\widetilde{D} = \{ \tilde{\alpha} \}$ and no $P \in K[w, x]$ divides $T_\alpha$. Using the induction hypothesis, we can reduce to the case when $X = T_\alpha$ is an irreducible variety of dimension $\gr_K(X)$.
By Noether normalization lemma, we can further reduce to the case when $$X\ =\ K^m\ \text{ with }\ m=\gr_K(X).$$
We have previously covered the case when $m = 0$. If $m>0$, let $a_1, \ldots, a_k \in K$ be algebraically independent elements over $\qq\big(\chi(F)\big)$. Set $$Y\ =\ a_{1}\chi(F) + \cdots  + a_{k} \chi(F)\ \text{ and }\ Z\ =\ Y \times K^{m-1}.$$
Then $Z\subseteq X$ is in a one-to-one correspondence with $\chi(F)^k \times K^{m-1}$.
Note that $\chi(F)^k \times K^{m-1}$ has a geometric pc-presentation $\{ T'_{\beta} \}_{ \beta \in F^k}$ where for each $\beta \in F^k$, $T'_\beta = \big\{ \chi(\beta) \big\} \times K^{m-1}$. Hence, by induction hypothesis, $\crk(Z)= \omega \cdot (m-1) +k$. Thus, $\crk(X) \geq \omega\cdot m = \gr(X)$.

Now we will prove that  $\crk(X) \leq \gr(X)$. By Corollary~\ref{grbehavior2}, we can reduce to the case when $\gd(X)=1$. Using the induction hypothesis, it suffices to prove that for any partition of $X$ into a union of two disjoint definable sets, one of them has Cantor rank less than $\gr(X)$. This follows from Corollary~\ref{gdbehavior}. 

Finally, we will verify that $\cdg(X)=\gd(X)$. Using Corollary~\ref{gdbehavior}, we can reduce to the case when $\gd(X)=1$. Suppose $\cdg(X)>1$, then $X$ is the disjoint union of $X_1, X_2$, where $\gr(X_1) =\crk(X_1)=\crk(X_2) =\gr(X_2)$. But by Corollary~\ref{gdbehavior} again, we get $\gd(X)>1$, a contradiction.
\end{proof}

\begin{thm} \label{rankAgreement}
Suppose $X \subseteq K^n$ is definable. Then $\gr(X) = \crk(X) =\mmr(X)$ and $\gd(X)=\cdg(X)=\mmd(X)$. 
\end{thm}

\begin{proof}
This follows from the preceding two lemmas and the fact that in an $\aleph_0$-saturated model of $\TN_p$, Cantor rank agrees with Morley rank and Cantor degree agrees with Morley degree.
\end{proof}


\begin{cor}
The theory $\TN$ is $\omega$-stable. 
\end{cor}

\begin{proof}
For any definable $X \subseteq K^n$, $\mmr(X) = \gr(X) < \omega^2$. The conclusion follows.
\end{proof}
\newpage
\noindent
Another consequence of Theorem~\ref{rankAgreement} is the following:

\begin{prop}
For any strongly minimal set $X \subseteq K^n$, there is a finite-to-one definable map from $X$ to $F$.
\end{prop}
\begin{proof}

We first show that if $Y \subseteq F^k$ is a strongly minimal set, then there is a finite-to-one definable map from $Y$ to $F$.
Indeed, by Theorem~\ref{StablyEmbbed}, the algebraically closed field $F$ is stably embedded into $(F, K; \chi)$ and hence $Y$ is strongly minimal in the field $F$.
By quantifier elimination of ACF, $Y$ differs from an irreducible variety $Y'$ over the field $F$ by finitely many elements.
By Noether normalization lemma, there is a finite map from $Y'$ to $F$. This can be easily modified to give a definable finite-to-one map from $Y$ to $F$.

Suppose $X$ is as in the statement of the proposition.
Then $\gr(X)=\gd(X)=1$.
Let $\{ T_\alpha\}_{\alpha \in D}$ be a geometric pc-presentation of $X$ with a primary index quotient $\widetilde{D}$.
Then for each $\tilde{\alpha} \in \widetilde{D}$, $T_\alpha$ has finitely many elements. Moreover, $T_\alpha$ and $T_\beta$  either consist of the same elements or are disjoint for all $\tilde{\alpha},\tilde{\beta} \in \widetilde{D}$.
We have a finite-to-one map from $X$ to $\widetilde{D}$ given by $a \mapsto \tilde{\alpha}$ if $a$ is an element of $T_\alpha$. By the previous paragraph, there is a finite-to-one definable map from $\widetilde{D}$ to $F$. The desired map is the composition of these two maps. 
\end{proof}


\noindent
We will next prove that Morley rank is definable in families in a model of $\TN_p$.
An {\bf fpc-formula} is an $L$-formula of the form 
$$P(x, z) \wedge \neg \exists s \Big( \phi(s, z) \wedge Q\big(\chi(s),x, z\big)\Big)$$
where $P$ is a system of polynomials in $\qq[x, z]$, $\phi(s, z)$ is a parameter free $L$-formula and $Q$ a system of polynomials in $\qq[w,x,z]$. If  $\psi(x,z)$  is an fpc-formula and $T_c \subseteq K^n$ is the set defined by $\psi(x, c)$ for $c \in K^l$, then $\{ T_c\}_{c \in K^l}$ is a definable family of pc-sets. On the other hand, as a consequence of Lemma~\ref{APsimple1}, an arbitrary pc-set is defined by $\psi(x,c)$ for some fpc-formula $\psi(x,z)$ and some $c$ in $K^{l}$.

\begin{lem}  \label{APsimple3}
Suppose \( X \subseteq K^n\) has an algebraic presentation $\{T_\alpha\}_{ \alpha \in D}$. Then, there are fpc-formulas $\psi_1(x,z), \ldots, \psi_k(x,z)$ such that for each $\alpha \in D$, there are $c \in K$ and $i \in \{1, \ldots, k\}$ such that $T_\alpha$ is defined by $\psi_i( x,c)$.
\end{lem}
\begin{proof}
For each $\alpha$ in $D$, there is a formula $\psi(x,z)$ such that for some $c \in K$, $T_\alpha$ is defined by $\psi(x,c)$. There are only countably many fpc-formulas. The conclusion follows by a standard compactness argument.
\end{proof}

\begin{thm}
Suppose $ \{ X_b\}_{b \in Y}$ is a definable family of subsets of $K^n$. Then for each ordinal \(\rho\), the set \( \big\{ b \in Y :  \gr(X_b)= \rho\big\} \) is definable.
\end{thm}

\begin{proof}
Suppose $ \{ X_b\}_{b \in Y}$ is as stated. Let $L_{\mathrm{r}}$ be the language of rings. For each choice $\mathscr{C}$ of an $L$-formula $\eta(u,x,z)$, an $L$-formula $\delta(u,z)$, an $L$-formula $\hat{\delta}(u,z)$,   an $L_{\mathrm{r}}$-formula $\tilde{\delta}(u',s)$,  an $L$-formula $\phi(u,u',z)$, fpc-formulas $\psi_1(x,z'), \ldots, \psi_e(x,z')$ (with $e \in \nn^{\geq 1}$), we will define the relation 
$$R_{\mathscr{C}}\ \subseteq\ F^{m} \times  F^{m}\times F^{k} \times K^n \times Y \times K^{l} .$$
The relation $R_{\mathscr{C}}(\alpha, \beta, \gamma, a, b, c)$ holds if and only if for  $H_\alpha$ defined by $\eta(\alpha,x,c)$, $D$ defined by ${\delta}(u,c)$, $\hat{D}$ defined by $\hat{\delta}(u,c)$,  $\widetilde{D}$ defined by $\tilde{\delta}(u',\gamma)$, $\pi$ defined by $\phi(u,u',c)$, and  $T_{i,c'}$ defined by $\psi_i( x,c')$, we have the following:
\begin{enumerate}[(a)]
\item $X_b =\bigcup_{\alpha \in D} H_\alpha$;
\item for each  $\alpha \in D$,  there are $c' \in K^{l'}$ and $i \in \{1, \ldots, e \} $ such that $H_\alpha = T_{i,c'}$;
\item $\alpha$ is in $\hat{D}$ if and only if for all $\beta \in D$,  $c' \in K^{l'}$, $i \in \{1, \ldots, e \} $, $d' \in K^{l'}$, $j \in \{1, \ldots, e\} $ with  $H_\alpha = T_{i,c'}$, $H_\beta = T_{j,d'}$, we have $\mr_K(T_{i,c'}) \geq \mr_K(T_{j,d'})$;
\item if $\alpha, \beta$ are in $\hat{D}$, for all $\beta \in D$,  $c' \in K^{l'}$, $i \in \{1, \ldots, e \} $, $d' \in K^{l'}$, $j \in \{1, \ldots, e \} $ with  $H_\alpha = T_{i,c'}$, $H_\beta = T_{j,d'}$, then we have either $T_{i,c'} \sim T_{j,d'}$ or $T_{i,c'} \simperp T_{j,d'}$;
\item $\pi$ is a surjective function from $\hat{D}$ to $\widetilde{D}$; $\pi(\alpha) =\pi(\beta)$ if and only if for all $c' \in K^{l'}$, $i \in \{1, \ldots, e \} $, $d' \in K^{l'}$, $j \in \{1, \ldots, e \} $ with  $H_\alpha = T_{i,c'}$, $H_\beta = T_{j,d'}$, we have $T_{i,c'} \sim T_{j,d'}$.
\end{enumerate}
%
We note that (a) and (b) can clearly be translated into statements involving $\eta(u,x,z)$, $\delta(u,z)$, $\hat{\delta}(u,z)$  $\tilde{\delta}(u',s)$, $\phi(u,u',z)$ and $\psi_1(x,z'), \ldots, \psi_e(x,z')$. For (c), (d) and (e), we can do so with the further use of Corollary~\ref{Definability}. Hence, $R_{\mathscr{C}}$ is definable. Let $R^5_{\mathscr{C}}$ be the projection of $R_{\mathscr{C}}$ on $Y$. Then $R^5_{\mathscr{C}}$ is also definable. Intuitively, $R^5_{\mathscr{C}}(b)$ means $X_b$ has an essentially disjoint pc-presentation given by $\mathscr{C}$ with some parameters.

For each $b \in Y$, there is one choice  $\mathscr{C}$ as above such that $R^5_{\mathscr{C}}(b)$.  Also, there are only countably many such choices $\mathscr{C}$. By a standard compactness argument,  $Y$ is covered by the union of $R^5_{\mathscr{C}}$ for finitely many such choices $\mathscr{C}$. We can reduce the problem to the case where $Y$ is covered by $R^5_{\mathscr{C}}$ for one choice $\mathscr{C}$ as above. 
Suppose an ordinal $\rho = \omega\cdot \rho_{{}_K} + \rho_{{}_F}$ is given where $\rho_{{}_K}, \rho_{{}_F}$ are natural number. With the notation as in the definition of $R_{\mathscr{C}}$, we have $\gr(X_b) = \rho$ if and only if $r_F(\widetilde{D}) = \rho_{{}_F} $ and $r_K(T_{i,c'}) = \rho_{{}_K}$ under the condition $T_{i,c'} = H_\alpha $ for some $\alpha \in \hat{D}$. The former is definable by the fact that Morley rank is definable in families in a model of ACF and the latter is definable as a consequence of Corollary~\ref{Definability}.
\end{proof}


\begin{cor}
Suppose $ \{ X_b\}_{b \in Y}$ is a definable family of subsets of $K^n$. Then there are ordinals $\rho_1, \ldots, \rho_k$ such that for all \(b\), for some $i \in \{1, \ldots, k\}$, $\gr(X_b) =\rho_i$.
\end{cor}

\begin{proof}
This is a consequence of the preceding theorem Lemma~\ref{Preservative} and a standard application of compactness.
\end{proof}





\begin{prop}
If $X \subseteq K^n$ and $ X'\subseteq K^{n'}$ are definable, $\gr(X) = \omega\cdot \rho_{{}_K} + \rho_{{}_F}$ and $  \gr(X') = \omega\cdot \rho'_{{}_K} + \rho'_{{}_F}$ with $\rho_{{}_K}, \rho_{{}_F}, \rho'_{{}_K}, \rho'_{{}_F} \in \nn$,  then  $\gr(X\times X') = \omega\cdot (\rho_{{}_K}+\rho'_{{}_K}) +\rho_{{}_F}+\rho'_{{}_F}$.
\end{prop}

\begin{proof}

We give a proof by induction on geometric rank of $X$ and $X'$. The case where $\gr(X)=\gr(X')=0$ is clear. Suppose we have proven the proposition for all $Y, Y'$ such that $\gr(Y)\leq \gr(X)$ and $\gr(Y') \leq \gr(X')$ and at least one of the equalities does not hold.  Let $\{T_\alpha\}_{ \alpha \in D}$ be an essentially disjoint pc-presentation of $X$ with primary index quotient $\widetilde{D}$. Likewise, we define $\{T'_{\alpha'}\}_{ \alpha' \in D'}$ and $\widetilde{D}'$. Then $\{T_\alpha \times T'_{\alpha'}\}_{ (\alpha, \alpha') \in D\times D'}$ is a fiberwise product of $\{T_\alpha\}_{ \alpha \in D}$ and  $\{T'_{\alpha'}\}_{ \alpha' \in D'}$, and hence an essentially disjoint pc-presentation of $X \times X'$. We note that, by $(11)$ of Lemma~\ref{lies}, we have 
$$\max_{(\alpha, \alpha') \in D\times D'}\gr_K(T_\alpha \times T'_{\alpha'})\ =\ \max_{\alpha\in D} \gr_K(T_\alpha) +  \max_{\alpha' \in D'} \gr_K(T'_{\alpha'}).$$
Also, by part (12) of Lemma~\ref{lies}, a primary index quotient of $\{T_\alpha \times T'_{\alpha'}\}_{ (\alpha, \alpha') \in D\times D'}$ can be chosen to be $\widetilde{D} \times \widetilde{D}'$. The desired conclusion follows.
\end{proof}

\section*{Acknowledgments}
\noindent
The authors thank Lou van den Dries for numerous discussions and comments on this paper.