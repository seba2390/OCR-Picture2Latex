\documentclass[10pt,twocolumn,letterpaper]{article}

\usepackage{iccv}
\usepackage{times}
\usepackage{epsfig}
\usepackage{graphicx}
\usepackage{amsmath}
\usepackage{amssymb}
\usepackage{multirow}
\usepackage{makecell}
\usepackage{bbm}
\usepackage{commath}
\usepackage{lipsum}
\usepackage{stfloats}
\usepackage{comment}
\usepackage{caption}
\usepackage[dvipsnames]{xcolor}
\usepackage[pagebackref=true,breaklinks=true,letterpaper=true,colorlinks,bookmarks=false]{hyperref}
% \usepackage[accsupp]{axessibility}  % Improves PDF readability for those with disabilities.

\iccvfinalcopy % *** Uncomment this line for the final submission

\def\httilde{\mbox{\tt\raisebox{-.5ex}{\symbol{126}}}}


\ificcvfinal\pagestyle{empty}\fi

\begin{document}

\setlength{\abovedisplayskip}{3pt}
\setlength{\belowdisplayskip}{3pt}

\makeatletter
\newcommand{\thickhline}{%
    \noalign {\ifnum 0=`}\fi \hrule height 1pt
    \futurelet \reserved@a \@xhline
}

\makeatother
\newcommand{\MATTHIAS}[1]{{\emph{\textcolor{red}{\textbf{Matthias:~#1}}}}}
\newcommand{\ANGIE}[1]{{\emph{\textcolor{blue}{Angie: #1}}}}
\newcommand{\JUSTUS}[1]{{\emph{\textcolor{magenta}{Justus: #1}}}}
\newcommand{\YAWAR}[1]{{\emph{\textcolor{ForestGreen}{Yawar:~#1}}}}
\newcommand{\FANGCHANG}[1]{{\emph{\textcolor{brown}{Fangchang:~#1}}}}
\newcommand{\TODO}[1]{{\emph{\textcolor{BrickRed}{TODO: #1}}}}


%%%%%%%%% TITLE
\title{RetrievalFuse: Neural 3D Scene Reconstruction with a Database}


\author{
Yawar Siddiqui$^1$~~~
Justus Thies$^{1,2}$~~~
Fangchang Ma$^3$~~~
Qi Shan$^3$~~~
Matthias Nie{\ss}ner$^1$
Angela Dai$^1$~~~
\vspace{0.2cm} \\ 
$^1$Technical University of Munich~~~
$^2$Max Planck Institute for Intelligent Systems, Tübingen~~~
$^3$Apple
\vspace{0.2cm} \\ 
}

\twocolumn[{%
	\renewcommand\twocolumn[1][]{#1}%
	\maketitle
	\begin{center}
	    \vspace{-0.65cm}
		\includegraphics[width=\linewidth]{figures/method_overview.jpg}
	    \vspace{-0.515cm}
		\captionof{figure}{
		We present a new approach for 3D reconstruction conditioned on sparse point clouds or low-resolution geometry. 
		Rather than encoding the full generative process in the neural network, which can struggle to represent local detail, we leverage an additional database of volumetric chunks from train scene data. 
		For a given input, multiple approximate reconstructions are first created with retrieved database chunks, which are then fused together with an attention-based blending~-- facilitating transfer of coherent structures and local detail from the retrieved train chunks to the output reconstruction.
		}
		\label{fig:teaser}
	\end{center}    
	\vspace{0.15cm}
}]

\maketitle
% Remove page # from the first page of camera-ready.
\ificcvfinal\thispagestyle{empty}\fi

%%%%%%%%% ABSTRACT

Quantum encryption is a well studied problem for both classical and quantum information. However, little is known about quantum encryption schemes which enable the user, under different keys, to learn different functions of the plaintext, given the ciphertext. In this paper, we give a novel one-bit secret-key quantum encryption scheme, a classical extension of which allows different key holders to learn different length subsequences of the plaintext from the ciphertext. We prove our quantum-classical scheme secure under the notions of quantum semantic security, quantum entropic indistinguishability, and recent security definitions from the field of functional encryption.

%%%%%%%%% BODY TEXT
\section{Introduction}
3D human pose estimation has ubiquitous applications in sport analysis, human-computer interaction, and fitness and dance teaching. While there has been remarkable progress in 3D pose estimation from a monocular image or video~\cite{hmrKanazawa17, Moon_2020_ECCV_I2L-MeshNet, kolotouros2019spin, kocabas2019vibe, xiang2019monocular}, inevitable challenges such as the depth ambiguity and the self-occlusion are still unsolved. 



\begin{figure}
     \centering
     \begin{subfigure}[h]{0.23\textwidth}
         \centering
         \includegraphics[width=\textwidth]{figures/cover/image-comp.jpg}
         \caption*{Input image}
     \end{subfigure}
     \begin{subfigure}[h]{0.23\textwidth}
         \centering
         \includegraphics[width=\textwidth]{figures/cover/smplify-comp.jpg}
         \caption*{SMPLify-X~\cite{SMPL-X:2019}}
     \end{subfigure}
     \vspace*{0.2cm}
     \begin{subfigure}[h]{0.45\textwidth}
         \centering
         \includegraphics[width=0.98\linewidth, trim=25 50 25 50]{figures/cover/scene_cover_green-comp.png}
         \caption*{3D visualization of our (left) and SMPLify-X (right) results}
     \end{subfigure}
     \vspace*{-0.2cm}
     \caption{While the state-of-the-art single-view 3D pose estimator~\cite{SMPL-X:2019} yields a small reprojection error, the recovered 3D poses may be erroneous due to the depth ambiguity. We make use of the mirror in the image to resolve the ambiguity and reconstruct more accurate human pose as well as the mirror geometry.}
     \vspace*{-0.5cm}
    \label{fig:demo1}
\end{figure}



In many scenes like dancing rooms and gyms, people are often in front of a mirror. In this case, we are able to see the person and his/her mirror image simultaneously. The mirror image actually provides an additional virtual view of the person, which can resolve the single-view depth ambiguity if the mirror is properly placed. Moreover, unseen part of the person can also be observed from the mirror image, so that the occlusion problem can be alleviated. 


In this paper, we investigate the feasibility of leveraging such mirror images to improve the accuracy of 3D human pose estimation. We develop an optimization-based framework with mirror symmetry constraints that are applicable without knowing the mirror geometry and camera parameters. We also provide a method to utilize the properties of vanishing points to recover the mirror normal along with the camera parameters, so that an additional mirror normal constraint can be imposed to further improve the human pose estimation accuracy. The effectiveness of our framework is validated on a new dataset for this new task with 3D pose ground-truth provided by a multi-view camera system. 


An important application of the proposed approach is to generate pseudo ground-truth annotations to train existing 3D pose estimators. To this end, we collect a large-scale set of Internet images that contain people and mirrors and generate 3D pose annotations with the proposed optimization method. The dataset is named Mirrored-Human.  
Compared with existing 3D human pose datasets~\cite{h36m_pami,mono-3dhp2017,vonMarcard2018} that are captured with very few subjects and background scenes, Mirrored-Human has a significantly larger diversity in human poses, appearances and backgrounds, as shown in Fig.~\ref{fig:dataset}. The experiments show that, by combining Mirrored-Human with existing datasets as training data, both accuracy and generalizability of existing 3D pose estimation methods can be significantly improved for both single-person and multi-person cases.   

In summary, we make the following contributions:
\begin{itemize}
    \item We introduce a new task of reconstructing  human pose from a single image in which we can see the person and the person's mirror image. 
    \item We develop a novel optimization-based framework with mirror symmetry constraints to solve this new task, as well as a method to recover mirror geometry from a single image.
    \item We collect a large-scale dataset named Mirrored-Human from the Internet, provide our reconstructed 3D poses as pseudo ground-truth, and show that training on this new dataset can improve the performance of existing 3D human pose estimators. 
\end{itemize}








\section{Related Works}

\paragraph{Learned 3D Shape Reconstruction.} 
3D shape reconstruction is a long-standing problem in computer vision. 
We refer readers to Szeliski~\cite{szeliski2010computer} for a more comprehensive review of the classic techniques. 
Recently, inspired by the progress of deep learning for images, many developments have been made in deep generative models for reconstructing 3D shapes, largely focusing on leveraging different geometric representations.

Early generative neural networks focused on voxel grids as a natural extension of pixels, with a regular structure well-suited for convolutions, but can struggle with cubic growth in dimension \cite{maturana2015voxnet,wu20153d,choy20163d,dai2017shape}.
Multi-resolution representations were proposed~\cite{hane2017hierarchical, tatarchenko2017octree} to address the cubic complexity with hierarchical data structures.
%
Rather than operating on a regular grid, point cloud based approaches propose to generate points only on the geometric surface \cite{fan2017point,yang2019pointflow}, but do not encode structural connectivity. 
%
Mesh-based approaches have also been proposed to efficiently capture surface geometry while encoding connectivity, but tend to rely on strong topological assumptions such as a template mesh that is then deformed~\cite{wang2018pixel2mesh}, or a small number of vertices for free-form generation~\cite{dai2019scan2mesh}.
%
Implicit representations encoded directly by the neural network enable modeling of a continuous surface, typically as binary occupancies or signed distance fields \cite{mescheder2019occupancy,deepsdf,chibane2020implicit}; such representations have seen notable success in modeling single objects but can struggle to directly scale to scenes.

\paragraph{Learned 3D Scene Reconstruction.} 
Compared to shape reconstruction, scene-level reconstruction is significantly more challenging due to the scale, variance, and complexity of geometry. 
Several approaches have been proposed to combine local implicit functions with a coarse volumetric basis \cite{jiang2020local,deep_local_shapes,peng2020convolutional} to capture complex, large-scale scene reconstructions.
SG-NN~\cite{dai2020sg} leverages a single, sparse volumetric network for large-scale scene completion in a self-supervised fashion. These approaches rely on encoding the full generative process into network parameters, whereas we leverage a basis of existing scene geometry, that does not need to be fully encoded but rather refined to transfer desired geometric characteristics from the valid scene geometry (e.g., clean structures, local details). 

\paragraph{2D/3D Retrieval.} Our approach is related to 2D image retrieval and completion applications~\cite{datta2008image}, where recent work~\cite{radenovic2018fine, xu2020texture} focuses on developing a CNN to automatically retrieve relevant patches from a large collection of unordered images. 
Note that memorization is also an active area of research in language models~\cite{khandelwal2019generalization}.

For 3D retrieval, the pioneering work of Chen \etal~\cite{chen2003visual} proposed a 3D shape retrieval system based on visual similarity.
More recently, several works have been proposed to leverage 3D CAD model retrieval to represent objects in input images or 3D scans \cite{li2015database,izadinia2017im2cad,avetisyan2019scan2cad,kuo2020mask2cad,izadinia2020licp}, but are limited to the objects in the CAD dataset, while we use our retrieval as a basis for enabling more accurate reconstruction from learned selection and blending of retrieved scene geometry. 


\section{Method}


Fig.~\ref{fig:method} presents the pipeline of our framework. Given an image that contains a person and a mirror, our goal is to recover the human mesh considering the mirror geometry. The key insight is that the person and his/her mirror image can be treated as two people, and we reconstruct them together with the mirror symmetry constraints. This section will be organized as follows. First, the formulation of single-person mesh recovery is introduced (Sec.~\ref{sec:spmr}). Then the mirror symmetry constraints that relate the two people will be elaborated (Sec.~\ref{sec:mi_geo}). Finally, the objective functions and the whole optimization are described (Sec.~\ref{sec:opt}).

\subsection{Human mesh recovery with SMPL model}
\label{sec:spmr}

We adopt the SMPL model~\cite{SMPL:2015} as our human representation. The SMPL model is a differentiable function $\bm M(\bm \theta, \bm \beta) \in \mathbb R^{3\times N_v}$ mapping the pose parameters $\bm \theta \in \mathbb R^{72}$ and the shape parameters $\bm{\beta} \in \mathbb R^{10}$ to a triangulated mesh with $N_v = 6890$ vertices. The 3D body joints $\bm J(\bm\theta, \bm\beta)$ of the model can be defined as a linear combination of the mesh vertices. Hence for $N_j$ joints, we defined the body joints $\bm J(\bm\theta, \bm\beta) \in \mathbb{R}^{3\times N_j} = \mathcal{J}(\bm M(\bm\theta, \bm\beta))$, where $\mathcal{J}$ is a pre-trained linear regressor. Let $\bm R \in SO(3)$ and $\bm T \in \mathbb R^3$ denote the global rotation and translation, respectively.

Given an image and the detected 2D bounding boxes, the 2D human keypoints $\bm W$ can be estimated with the cropped regions. The objective function for human mesh recovery generally consists of a reprojection term $L_{2d}$ and a prior term $L_p$ with respect to variables $\bm \theta$, $\bm \beta$, $\bm R$ and $\bm T$.

The reprojection term penalizes the weighted 2D distance between the estimated 2D keypoints $\bm{W}$ with the confidence $c$, and the corresponding projected SMPL joints:
\begin{equation}
    L_{2d} = \sum_i c_i\rho(\bm{W}_i - \Pi_K(\bm{R}\bm{J}(\bm\theta, \bm\beta)_i + \bm{T})),
\end{equation}
where $\Pi_K$ is the projection from 3D to 2D through the intrinsic parameter $K$. $\rho$ denotes the Geman-McClure robust error function for suppressing noisy detections. 

The human body priors are used to encourage realistic 3D human mesh results. Since the pose and shape parameters ($ \bm{\Tilde{\theta}}, \bm{\Tilde{\beta}}$) estimated by a neural network can be viewed as learned prior, the final results are supposed to be close to them:
\begin{equation}
    L_{p} = ||\bm{\theta} - \bm{\Tilde{\theta}}||_2^2  + \lambda_{\beta}|| \bm{\beta} - \bm{\Tilde{\beta}}||_2^2,
\end{equation}
where $\lambda_{\beta}$ is a weight.

\subsection{Mirror-induced constraints}
\label{sec:mi_geo}
If there is a mirror in the image, the relation between the person and the mirrored person can be used to enhance the reconstruction performance. This relation is a simple reflection transformation if the mirror geometry is known, which however is impracticable for an arbitrary image from the Internet. To tackle this problem and take advantage of the characteristic of the mirror, the following mirror-induced constraints are introduced, as illustrated in Fig.~\ref{fig:mirrorsym}. Note that all symbols with the superscript prime refer to variables related to the mirrored person unless specifically mentioned.

\paragraph{Mirror symmetry constraints:}
Since the adopted human representation disentangles the orientation $\bm R$, pose parameters $\bm \theta$ and shape parameters $\bm \beta$, $\bm \beta$ can be shared by the person and the mirrored person, and $\bm \theta$ is related to $\bm \theta'$ by a simple reflection operation as follows:
\begin{equation}
\label{eq:param}
    \bm{\beta}' = \bm{\beta}, ~\bm{\theta}' = \mathcal{S}(\bm \theta),
\end{equation}
where $\mathcal{S}(\cdot)$ denotes the reflection operation on axis angles. 
\begin{figure}[t]
\centering
\includegraphics[trim=3cm 18.5cm 10cm 3.5cm, width=0.8\linewidth,clip]{figures/mirrorloss.pdf}
\caption{\textbf{An illustration of mirror-induced constraints.} The line segment connecting the joint $\bm{J}_i$ and its mirrored joint $\bm{J}_i'$ has the direction $\bm n_i$ and the middle point $\bm p_i$. Theoretically, $\bm{n}_i // \bm{n}_j$, and $\bm{n}_i \perp \overline{\bm{p}_i\bm{p}_j}$. If the mirror normal $\bm{n}$ (red arrow) is known, $\bm{n} // \bm{n}_i$ and $\bm{n} // \bm{n}_j$ should be satisfied as well.}
\label{fig:mirrorsym}
\end{figure}

As Eq.~\ref{eq:param} does not take $\bm R$ and $\bm{T}$ into consideration, the constraint on 3D keypoints can be imposed to estimate the human orientation and position better. We abbreviate the global coordinates of the $i$-th joint $\bm{R}\bm{J}(\bm\theta, \bm\beta)_i + \bm{T}$ as $\bm{J}_i$. Given a pair of body joints $i, j$, we denote the direction of the line segment $\overline{\bm{J}_i\bm{J}_i'}$, $\overline{\bm{J}_j\bm{J}_j'}$ as $\bm{n}_i$, $\bm{n}_j$ and the middle point of them as $\bm p_i$, $\bm p_j$, respectively. Ideally, $\bm n_i$ should be parallel to $\bm n_j$ and $\bm p_i, \bm{p}_j$ are supposed to be on the mirror plane. Despite the fact that the mirror geometry is unknown, it needs to be satisfied that $\bm{n}_i$ is perpendicular to the line  $\overline{\bm{p}_i\bm{p}_j}$. So for 
any pair of joints, we minimize the sum of the L2 norm of the cross product between $\bm{n}_i$ and $\bm{n}_j$, and the inner product between $\bm{n}_i$ and $\bm{p}_j - \bm{p}_i$: 
\begin{equation}\label{eq:mirrorsym}
    L_{s} = \sum_{(i, j)}(||\bm{n}_i \times \bm{n}_j||_2 + || \bm{n}_i\cdot (\bm{p}_j - \bm{p}_i) ||_2).
\end{equation}

\paragraph{Mirror normal constraint:}
A mirror can be represented as a plane, parameterized as its normal and position. If its normal $\bm{n}$ is known, the geometric properties of the mirror can thus be utilized explicitly by constraining $\bm n_i$ and $\bm n$ to be parallel with the following loss function:
\begin{equation}\label{eq:mirrorgt}
    L_{n} =  \sum_i||\bm{n} \times \bm{n}_i||_2. 
\end{equation}
\vspace{-0.5cm}
\begin{figure}[t]
	\centering
	\includegraphics[width=1\linewidth,trim={8cm 8cm 8cm 7.5cm},clip]{figures/vp_small.pdf}
	\includegraphics[width=\linewidth, trim={2.5cm 2cm 1cm 3cm},clip]{figures/vpdemo.pdf}
	\vspace{-0.8cm}
	\caption{\textbf{Vanishing points in an image containing a person and a mirror.} In most cases at least two vanishing points can be found, where $\bm v_0$ comes from 2D human keypoints, and $\bm v_1$ comes from the annotated mirror edges. $O_c$ denotes the camera center. Note that $\overline{O_c v_0} // \bm n$ and $\overline{O_c v_1} \perp \bm n$, where $\bm n$ is the mirror normal.
	}
	\label{fig:vp}
\end{figure}
\paragraph{Mirror normal estimation:}
Though the mirror normal is not directly available, the vanishing points can be used to estimate it. The vanishing point of lines with direction $\bm n$ in 3D space is the intersection $\bm v$ of the image plane with a ray through the camera center with direction $\bm n$~\cite{hartley2003multiple}:
\begin{equation}
\bm v=K \bm n,
\label{eq:vanish}
\end{equation}
where the vanishing point $\bm v\in \mathbb{R}^3$ is in the form of homogeneous coordinates and $K$ is the camera intrinsic matrix. 

Eq.~\ref{eq:vanish} reveals that obtaining the mirror normal $\bm n$ requires both $K$ and $\bm v$. As the parallel lines connecting points on the real object and corresponding points on the mirrored object are perpendicular to the mirror, the vanishing point $\bm v$ with this direction can be estimated through their 2D positions.
To get such correspondences, some previous works require additional inputs such as masks \cite{Hu2005MultipleView3R}, which is infeasible for images from the Internet.
Fortunately, since 2D human keypoints provide robust semantic correspondences, \eg the left ankle of the real person and the right ankle of the mirrored person, this vanishing point can be acquired naturally and automatically in our setting ($\bm v_0$ in Fig.~\ref{fig:vp}).

Note that if the intrinsic matrix $K$ is provided, the mirror normal can thus be solved easily through $\bm n=K^{-1} \bm v_0$, otherwise $K$ should be calibrated from a single image if possible. From the projective geometry~\cite{hartley2003multiple}, we know that it is possible to calibrate the camera intrinsic parameters from a single image. Suppose the camera has zero skew and square pixels. The intrinsic matrix $K$ can be computed via three orthogonal vanishing points. Additionally, if the principal point is assumed to be in the image center (only the focal length is unknown), $K$ can be computed via only two orthogonal vanishing points. Please refer to the supplementary material for more details. 

As we have stated, one vanishing point $\bm v_0$ has been acquired based on reliable 2D human keypoints. Different from the general scene where finding orthogonal relations may be difficult, our setting contains richer information. Fig.~\ref{fig:vp} shows that if we annotate the mirror edges, at least one vanishing point $\bm v_1$ orthogonal to $\bm v_0$ can be obtained. With these vanishing points, the calibration can be performed. Note that images from the same video share the same intrinsic matrix $K$, thus the annotation process is not laborious.

The mirror normal constraint is optional, which depends on how easy it is to find mirror edges. In the experiment, we will show that our method can still achieve satisfactory performance without the mirror normal constraint.

\subsection{Objective function and optimization}
\label{sec:opt}
Combining all discussed above, the final objective function to optimize can be written as:
\begin{equation}
\label{eq:loss0}
\begin{split}
    \min_{\substack{\Theta, \Theta'}}~L_{2d}+L_{2d}' + \lambda_p (L_{p}&+L_{p}') + \lambda_s L_s + \lambda_n L_n \\
     s.t.~~ \bm{\beta}' = \bm{\beta}&, ~\bm{\theta}' = \mathcal{S}(\bm \theta),
\end{split}
\end{equation}
where $\Theta=\{\bm\theta, \bm\beta, \bm R, \bm T\}$ and $\Theta'=\{\bm\theta', \bm\beta', \bm R', \bm T'\}$. $L_{2d}'$ and $L_p'$ refer to the reprojection term and the prior term of the mirrored person, respectively. $\lambda_p$, $\lambda_s$ and $\lambda_n$ are weights. $\lambda_n$ is set to zero whenever the mirror normal is unavailable. If there are two or more people, the optimization can be done for each subject separately.

We optimize Eq.~\ref{eq:loss0} with respect to all parameters using L-BFGS and PyTorch. An off-the-shelf model~\cite{kolotouros2019spin} is adopted to generate the initial estimation. Given the 2D keypoints~\cite{sun2019deep, cao2017realtime}, $\bm R$ and $\bm T$ are further optimized by aligning the initial SMPL model to the 2D keypoints. To improve the robustness of the initialization, we select the person with smaller reprojection error and apply the selected pose parameter to the other person after a reflection operation. 


























































\section{Results}
\section{Experiments}
\label{sec:exp}


We evaluate the effect of \tokdetok{} when included in various \llm{}s. 
We choose \textsc{BERT-base-cased}~\cite{devlin-etal-2019-bert}, \textsc{GPT2-small}~\cite{radford2019language}, and \textsc{RoBERTa-base}~\cite{liu2019roberta} as the base \llm{}s to be manipulated.\footnote{Models loaded from the Huggingface repository~\cite{wolf-etal-2020-transformers}.}
The former contains roughly 108M parameters, and the latter two roughly 125M, a difference accounted for by their larger subword vocabulary (50k vs. 29k) and the resulting larger embedding table.
All models are case sensitive, but they differ in their strategy for preserving the original space-delimited word sequence: BERT's tokenizer marks word-non-initial tokens with a \say{\#\#} string, while GPT-2's tokenizer marks spaces with a special underline character and appends them to the following word.
The difference manifests itself in sequence-initial words, whose initial token in the GPT-2 representation shares the form of sequence-medial word-medial tokens, rather than that of sequence-medial word-initial tokens; and in symbols pre-tokenized without a preceding space, such as punctuation and apostrophes.
RoBERTa's tokenizer adopts the GPT-2 marking strategy but avoids the first pitfall by internally prepending all input sequences with a space character.
We perform the \ppt{} phase for each model on a collection of English tweets from 2016 obtained from the Firehose and preprocessed to replace all `@'-mentions with \texttt{@user}.
We later ablate this domain change effect by training models with \ppt{} text from the English Wikipedia March 2019 dump (see \S\ref{ssec:abl}).
We sample both resources to create pre-training corpora of roughly 725MB (unzipped), several orders of magnitude smaller than what contemporary models use for the first pre-training phase, and train for a single epoch.

In preliminary experiments on several character-level \tok{} architectures, we found that a convolutional net outperforms bidirectional LSTMs and small transformer stacks.
We pass the input characters through three separate convolution layers of width 2, 3, and 4 (characters), then pass the outputs through max-pooling layers and a \texttt{ReLU} activation, and finally project the concatenation of the results onto the base models' embedding dimension.
% \todo{Add figure if feeling masochistic.}
This \tok{} component contains $\sim$1M parameters, negligible compared to the transformers' parameter count.\footnote{The Wikipedia-trained models contain $\sim$2.8M parameters, the difference owing to language-ID filtering performed on the Twitter data, leading to a much smaller character set.} % \detok{}, which uses a 2-layer MLP with hidden layer same-dim as output dim, has 81M params for Wiki and 540K params for Twitter (but we don't use it in our tasks).
% exact numbers: 1071760, 2819360, 81470895, 539539
We implement \detok{} as a 2-layer unidirectional LSTM whose hidden layer is initialized by projecting the context vector $\vh_i$ output from \mmod{} into the hidden dimension.
Characters are generated by projecting the LSTM's output through two linear layers with a \texttt{tanh} activation.

During \ppt{}, we insert a cycle dependency batch every 5,000 LM steps.
For the replacement policies we choose $\pi^t_{(l)}$ to sample uniformly random sets of tokens representing 15\% of space-delimited words to pass as a target loss
for \tok{}, and $\pi^t_{(u)}$ to replace the embedding input to \mmod{} with \tok{}'s output for all multi-token words.
$\pi^g$ selects all tokens to pass as target losses for \detok{}, calculated as a sum of the cross-entropy loss for each character in the target sequence.
Cycle batches consist of sampling $k$ words out of the $K$ most frequent words from the training corpus, with replacement, and $k$ vectors from a Gaussian distribution blown to concentrate around the surface of the unit sphere in hidden-dimension space:
\[ \tilde{\ve} \sim \frac{1}{\sqrt{d}}\cdot\mathcal{N}(0,\vI^{(d)}). \]
\tdloop{} loops are optimized for a
character-level cross-entropy
loss, whereas \dtloop{} loops target a
euclidean distance loss.
When a \tok{} embedding correspond to multiple $\tau$ tokens, the learning target is created by max-pooling their embeddings.\footnote{This $\agg$ function outperformed average-pooling and first-token selection in preliminary experiments.}
We set $K=$25,000 and $k=$1,000.


\subsection{Intrinsic Assessment}

Across all base models and both \ppt{} corpora selected for our experiments, we observed a steady decrease in the \llm{} models' built-in loss metrics (masked prediction / autoregressive prediction) until stabilizing at roughly half the initial value before the end of the \ppt{} epoch.
This indicated that the transformer layers are able to process inputs from both the embedding table and \tok{} and reconcile them.
\autoref{fig:updates} depicts the parameter updates in RoBERTa by parameter type, across layers, comparing parameter values before and after the \ppt{} phase on Twitter data.
It shows that the change along the model layers is fairly stable, with mildly more extensive updates in the bottom and top layers.
The former is to be expected given the introduction of inputs from \tok{}; the latter can also be influenced by encountering Twitter data, which is substantially different than what RoBERTa is \say{used to}.

\begin{figure}
    \centering
    \includegraphics[width=8.25cm]{figures/layer_updates.png}
    \caption{Euclidean distance between RoBERTa weight parameter values before and after a \ppt{} training phase on the Twitter corpus.}
    \label{fig:updates}
\end{figure}

\begin{table*}
    \centering
    \footnotesize
    \begin{tabular}{lHcccccc}
        \toprule
        Model & Uncased BERT & BERT & \multicolumn{2}{c}{GPT-2} & \multicolumn{2}{c}{RoBERTa} \\ % 202102052207 & 202103171733 & 202104091857 & 202102260408 & 202103192237 & 202103192237 \\
        Corpus & Wikipedia & Wikipedia & Wikipedia & Twitter & Twitter & Twitter \\
        Steps & 76,000 & 9,000 & 13,500 & 21,000 & 7,500 & 57,500 \\
        Sequences ($10^3$) & 3,648 & 5,184 & 7,776 & 10,080 & 2,160 & 16,560 \\
         % 48x & 576x & 576x & 480x & 288x & 288x \\
         \midrule
        & ppercented & proming & crordman & d & orereren & everyone \\
        & or & dy & sssion & . & ant & kerned \\
        & ter & deded & gental & the & re & levernger \\
        & peprepored & terse & 2 & @ & cerent & and \\
        & essed & h & ther & ==666!!!!!!!!!!!!!!! & ennte & ed \\
         \bottomrule
    \end{tabular}
    \caption{Example generated words from random locations near the surface of the unit sphere in $\mathds{R}^{768}$.}
    \label{tab:gen}
\end{table*}


% More, from RoBERTa: Wiki-5\% after step 6,300 (576x):
% .<e> reenter<e> rencement<e> o<e> e<e> ĵ<e> ,<e> reciention<e> underesting<e> and<e>

% RoBERTa 13,000 Twitter-7.5\% (576x): reverally<e>	the<e>	revering<e>	.<e>	prot<e>	handers<e>	a<e>	there<e>	.<e>	@<e>





Another artifact of \ppt{} is the outputs of the \detok{} module.
While monitoring the training procedure, we periodically sample words from random locations centered around the surface of the unit sphere of the embedding space, to see what \say{priors} the generative net is learning from the vectors encountered during training.
\autoref{tab:gen} presents some of these samples for different models and different corpora at different points in the training phase.
Masked models appear to be learning well-formed fractions of English or pseudo-English words at early stages of the training phase from both Wikipedia and Twitter data.
As training continues, fewer sequences containing repetitions are observed, fewer generations occur across  samples, and more in-vocabulary words appear, suggesting convergence of the vector space towards representing well-formed and diverse English vocabulary.
The autoregressive GPT-2, on the other hand, struggles to produce meaningful sequences beyond short words and punctuation symbols when trained on the informal Twitter input, suggesting a difficulty in learning a mapping of language from vector space without availability of a two-sided context.



\subsection{Downstream Evaluation}
\label{ssec:finetune}

During task fine-tuning and inference,
since we do not evaluate on generative tasks, we do not use \detok{}.
We perform minimal hyperparameter search for each base model + task combination, and fine-tune model parameters during downstream training in all tasks but NYTWIT.
For these tasks, we also experimented with setups where \mmod{} and \tok{} are used as feature extractors only, and where \tok{} training is supplemented by an additional embedding loss (as described in \S\ref{ssec:pretr} for the \ppt{} phase) computed against embeddings of the task input.
Neither setup provided improvement on any task during our tuning experiments (see \S\ref{ssec:abl}).

Downstream models are implemented as follows: for sequence classification and ranking, a two-layer perceptron with \texttt{ReLU} activation is trained to make the prediction from the top-layer representation of the initial \texttt{[CLS]} token (in BERT and RoBERTa models) or of the final token in the sequence (in GPT-2).
For NER, an LSTM is run over the sequence of each word's top-layer representation, followed by a single linear layer which makes the prediction.
For NYTWIT, \mmod{}'s contextualized vector for the target word is used as input for a single logistic layer.
In cases of multi-token words, the prediction from the first token is selected.
We set \tok{}'s character embedding dimension to 200 and the convolutional layers to 256 channels.
Following \newcite{sun2019fine}, we set the maximum learning rate to $10^{-3}$ for the task models and $2\times 10^{-5}$ for fine-tuning, and perform warm-up for 10\% of the total expected training steps before linearly decaying the rates to zero.
All parameters are optimized using Adam~\cite{adam} with default settings.
We run all NER models for twenty epochs and sequence-level task models for three, evaluating on the validation set after each epoch using the metrics reported below, and stopping early if performance has not improved for four epochs.
In order to avoid unfairly favorable conditions for \tokdetok{} models, task hyperparameters are all tuned on the base models, with \tokdetok{} models using only values on which their base equivalents have also been evaluated.

We evaluate the effectiveness of \tokdetok{}'s concepts and components by comparing the following setups:\footnote{A setup where all words are represented by \tok{} obtained noncompetitive results on all tasks.}
\begin{itemize}
    \item \textsc{\textbf{None}} uses only the base model;
    \item \textsc{\textbf{None+\ppt{}}} uses a version of the base model that was further pre-trained on the same Twitter corpus on which \tokdetok{} is trained (adaptive fine-tuning~\cite{ruder2021lmfinetuning}), in order to control for the increase in total unlabeled text seen by the model;
    \item \textsc{\textbf{Scaffolding}} is a model trained with \tokdetok{} in a \ppt{} phase, but only using base model embeddings during task fine-tuning;
    \item \textsc{\textbf{Stochastic}} samples 10\% of the words in the downstream datasets, calling \tok{} on their character sequences while using the base model's embedding(s) for the remaining 90\%;
    \item \textsc{\textbf{All no-suff}} calls \tok{} on all multi-token words which are not of the form \texttt{[token suff]}, where \texttt{suff} is a member of the \textsc{Suffixes} set described in \S\ref{sec:model}, and uses the base model's embedding on the rest.
\end{itemize}

\begin{table*}
    \centering
    \small
    \begin{tabular}{llccccccccc}
        \toprule
        Base & \tokdetok{} & \multicolumn{2}{c}{Emoji} & \multicolumn{2}{c}{Twitter NER} & \multicolumn{2}{c}{Emerging NER} & QA & NYT- & Avg. \\
        & & Dev & Test & Dev & Test & Dev & Test &  & WIT & Test \\
        \midrule
        BERT & None & 24.30 & 37.29 & 29.06 & 28.66 & 38.14 & 29.01 & 46.78 & 45.21 & 37.39 \\
        & None+\ppt{} & \underline{\textbf{29.25}} & \textbf{39.40} & 32.05 & 29.50 & 39.79 & \textbf{29.25} & \underline{\textbf{50.02}} & 44.87 & 38.61 \\
        & Scaffolding & 28.17 & 38.60 & 32.45 & \textbf{31.38} & \textbf{41.10} & \textbf{29.25} & 49.50 & \textbf{53.62} & \textbf{40.47} \\
        & Stochastic & 27.54 & 37.41 & 30.77 & 27.51 & 37.99 & 27.98 & 48.12 & 49.80 & 38.16 \\
        & All no-suff & 27.12 & 38.63 & \textbf{34.09} & 29.59 & 39.47 & 28.43 & 48.64 & 34.76 & 36.01 \\
        \midrule
        GPT2 & None & 24.89 & 38.72 & 30.25 & 26.70 & 40.06 & 28.40 & 44.90 & 47.25 & 37.19 \\
        & None+\ppt{} & 25.47 & \textbf{40.95} & 32.00 & 28.51 & 40.68 & 29.90 & \textbf{47.18} & 47.44 & 38.80 \\
        & Scaffolding & 25.29 & 40.70 & 30.46 & 28.97 & 41.16 & 28.23 & 46.99 & \textbf{50.99} & \textbf{39.18} \\
        & Stochastic & 25.23 & 38.55 & 32.20 & 28.10 & 39.58 & 28.35 & 46.11 & 46.37 & 37.50 \\
        & All no-suff & \textbf{26.30} & 34.27 & \textbf{36.27} & \textbf{32.48} & \textbf{49.20} & \textbf{34.87} & 44.98 & 33.21 & 35.96 \\
        \midrule
        RoBERTa & None & 25.07 & 39.50 & 48.57 & 44.86 & 56.43 & \underline{\textbf{46.22}} & 45.12 & 48.31 & 44.80 \\
        & None+\ppt{} & \textbf{27.04} & \underline{\textbf{42.82}} & 47.39 & 43.84 & 57.13 & 45.12 & \textbf{48.98} & 48.43 & 45.84 \\
        & Scaffolding & 25.87 & 41.12 & \underline{\textbf{49.71}} & \underline{\textbf{45.51}} & 56.28 & 44.60 & 48.26 & \underline{\textbf{53.98}} & \underline{\textbf{46.69}} \\
        & Stochastic & 26.38 & 40.09 & 49.57 & 44.82 & \underline{\textbf{58.21}} & 45.26 & 48.54 & 51.80 & 46.10 \\
        & All no-suff & 26.82 & 33.07 & 46.55 & 42.72 & 55.07 & 43.91 & 47.45 & 35.65 & 40.56 \\
        \midrule
        \multicolumn{2}{l}{SOTA (reported)} & & 47.46 & & 52.4\phantom{0} & & 49.6\phantom{0} & & 48.4\phantom{0} & \\
        \bottomrule
    \end{tabular}
    \caption{Results on all models (Emoji, NER, NYTWIT: Micro-F1 $\times$ 100; QA: MRR $\times$ 100), all results except for \textsc{None} are averaged over three models initialized on different random seeds.
    Best result for each base model in bold, best across all underlined.
    % \todo{Report intervals somehow, maybe in appendix.}
    % \todo{Or maybe report median?}
    }
    \label{tab:main_results}
\end{table*}






We present the results of the downstream prediction tasks in \autoref{tab:main_results}.
All transformer models except for \textsc{None} were 2nd-phase pre-trained three separate times using different random seeds, and the mean results are reported.

The first observation we make is the dominance of RoBERTa, a thoroughly optimized masked language model, over the other models on the NER datasets in all its variants.
This suggests RoBERTa has captured fine-grained information about individual words that it was able to retain in its representations for them; the struggle of the \tokdetok{} model to provide improvement over the \textsc{None} versions of the model strengthens this hypothesis.
GPT-2, despite having access to only left-side context of each word, still outperforms the basic BERT model on most setups in the NER tasks, perhaps due to its larger subword vocabulary size.
At the same time, its gains from the \tok{} representations are much more considerable, suggesting that its left-context-only inference may in fact be detrimental to the its \mmod{}'s performance as a whole.
In general, models perform better on Emerging NER than on Twitter NER, which we attribute to several possible causes or their combination: first, the source shift in the Emerging NER test set from Twitter to Reddit is meant to encumber the models, but given the extensive pre-training they undergo they may actually benefit from the fact that test sequences are longer on average than those in the Twitter NER dataset;
second, more prosaically, the Emerging NER dataset contains fewer entity types (6 vs. 10), making the task itself somewhat easier.

The other word-level task, NYTWIT, demonstrates substantial gains made by the \tokdetok{} training regime: the \textsc{Scaffolding} setup preforms best in all three base models, and in both masked langugage models the \textsc{Stochastic} setup outperforms both \textsc{None} variants.
Together with the NER results, this suggests that \tokdetok{} succeeds in providing transformer models with \textbf{word-level} representations that allow coarse-grained classifications (such as named entity type or novel word origin), better than default subword segmentations, for both edited and user-generated text.

We find that \tokdetok{} is less successful in improving sequence classification and ranking performance than on word-level tasks.
The \textsc{None+\ppt} setup obtains the best results in most models on the Emoji and QA datasets, suggesting the improvements seen in \tokdetok{}-based models is mostly attributable to the domain shift introduced by the Twitter pre-training corpus.\footnote{We note the complete inconsistency in both model performance and comparative model ranking present in the Emoji dataset, which calls back the systemic issue we identified in~\S\ref{sec:tasks}:
The time span over which the test set was collected contained a fundamental shift in Twitter's properties --- doubling the tweet length limit from 140 to 280 characters --- and so exhibits an unpredictable corpus incompatible with the train and dev partitions.
As a result, model performance over the dev set does not predict the test set results (a large difference on performance in this task between dev and test sets was also observed in macro-F1 scores by~\newcite{barbieri-etal-2020-tweeteval}).
In fact, under such specific data shift circumstances, it could be the case that the simpler base model is better equipped for facing longer test data, which scales generalizations made over the training and dev sets, as opposed to \tok{}-augmented models which have more levels of generalization to acquire during training and cannot anticipate the scale change.
Post-hoc inspection of specific model outputs resulted in some more concrete hypotheses for the cause of discrepancy in the major categories of confusion, for example the distributions of \say{@} presence in heart emoji
%{\NotoEmoji\symbol{"2764}}
tweets and heart-eyes emoji
%{\NotoEmoji\symbol{"1F60D}}
tweets shifted to a degree which could explain the models' growing confusion between the two, but no signals accounted for the entire difference in performance and we conclude that the main reason remains the change in sequence length distributions.}
This suggests that \tok{} may be a good learner for word-internal phenomena, picking up structural cues as to their roles within or without context, making it more useful \textbf{locally} than subword tokens' uninformed embeddings, while not being strong enough to provide a better semantic prior for \mmod{} to aggregate together with surrounding well-formed words, making its \textbf{global} utility limited.


\begin{table}
    \centering
    \small
    \begin{tabular}{lrrr}
        \toprule
        Ablation & BERT & GPT2 & RoBERTa \\
        \midrule
        Full & 37.17 & 38.27 & 41.97 \\
        No FT & $-$0.31 & $-$0.25 & $-$0.80 \\
        Wiki \ppt{} & $-$3.11 & $-$2.67 & $-$2.17 \\
        % Wiki+Emb loss &  & $-$4.83 &  \\
        No loops & $+$0.82 & $+$0.26 & $-$1.12 \\
        All multi & $-$1.39 & $+$0.44 & $-$1.81 \\	
        \bottomrule
    \end{tabular}
    \caption{Dev set average effect of model variants compared with the \textit{All no-suff} condition: % on a single random seed:
    \say{No-FT} --- pre-trained model used only for feature extraction;
    \say{Wiki} --- \tokdetok{} trained on Wikipedia data instead of Twitter;
    % \say{Emb loss} --- single-token words in downstream data fed for training \tok{};
    \say{No loops} --- trained without the cycle dependency loops;
    \say{All multi} --- common-suffix words also inferred using \tok{}.}
    \label{tab:ablations}
\end{table}

% \begin{table}
%     \centering
%     \small
%     \begin{tabular}{lrr}
%         \toprule
%         Model / Noise & Random case & Repeat \\
%         \midrule
%         Base & 42.73 & 43.57 \\
%         Base + \ppt{} & 43.54 & 45.07 \\
%         Scaffolding & 43.15 & 44.31 \\
%         Stochastic & 42.54 & 43.92 \\
%         All-no-suff & 41.72 & 42.33 \\
%         \bottomrule
%     \end{tabular}
%     \caption{Noising experiments on RoBERTa models on the QA dataset.}
%     \label{tab:ablations}
% \end{table}




\subsection{Ablations}
\label{ssec:abl}
We compare the dev set results of the \textsc{All no-suff} condition on several modified versions of the model, presented in Table~\ref{tab:ablations}.
First, we find that fine-tuning \mmod{} and \tokdetok{} parameters during downstream task application is beneficial for results across base models, indicating susceptibility of \tok{}'s network to tune itself on task data and not solely on LM and the vectorization signal.
Next, and most substantially, we note the vast improvement of Twitter-trained models compared with \ppt{} performed over a Wikipedia corpus with comparable size;
even though some tasks are on edited text, the overall effect of domain change during second pre-training is apparent (and, indeed, least impactful on the NYTWIT task which features the best-edited text).
Other decisions made during the \ppt{} phase appear to be less decisive: removing the dependency loops helps performance on BERT and GPT2, but makes a large dent in the best-performing RoBERTa, indicating potential gains to be made by applying \detok{} in generative tasks not pursued within our scope;
the suffix-based dialing down of inference in pre-training helps the masked models but hurts GPT2 performance, possibly because its autoregressive application prohibits it from looking at a simply-inflected word's suffix when processing its stem, a problem not incurred in masked modeling.




\section{Conclusion}
\section{Conclusion}

This paper proposes an optimization scheme for variational quantum circuits by combining the gradient estimate obtained from simultaneous perturbation stochastic approximation with gradient-based optimizers like SGD, Adam, AMSGrad, or RMSProp. We demonstrate with noiseless simulations on simple regression tasks that using the SPSA-approximated gradient improves both the convergence rate by a factor of three and the final absolute error by a factor of more than two compared to the parameter-shift rule. We further observe that the combined SPSA-AMSGrad optimizer consistently outperforms all other methods, including standard SPSA.

Adam, AMSGrad, and RMSProp clearly outperform SGD for SPSA-inferred gradients when considering shot- and hardware noise. The gap in performance grows to a factor of 1.5 for the full noise model. The performance boost between methods remains the same even when error mitigation addresses the hardware noise.

We conclude that combining the computationally cheap to obtain gradient estimate of SPSA with modern gradient-based optimizers drastically speeds up the training process of VQCs and leads to improved convergence.

% funding is already mentioned on the first page
%The research presented in this paper is supported by the Bavarian Ministry of Economic Affairs, Regional Development and Energy with funds from the Hightech Agenda Bayern.

\paragraph{Acknowledgements.}
{
\small
This work was supported by the Bavarian State Ministry of Science and the Arts  coordinated by the Bavarian Research Institute for Digital Transformation (bidt), a TUM-IAS Rudolf M{\"o}{\ss}bauer Fellowship, an NVidia Professorship Award, the ERC Starting Grant Scan2CAD (804724), and the German Research Foundation (DFG) Grant Making Machine Learning on Static and Dynamic 3D Data Practical. Apple was not involved in the evaluations and implementation of the code. 
}

\newpage
{\small
\bibliographystyle{ieee_fullname}
\bibliography{bib}
}

\clearpage
\newpage
\begin{appendix}
%
In this appendix, we discuss additional experiments that we conducted with our neural 3D scene reconstruction method \textit{RetrievalFuse} (Sec.~\ref{sec:appendix_evaluation}).
%
Specifically, we show additional ablation studies and results for both the 3D super-resolution and surface reconstruction.
%
We also provide implementation details of our method and the used baselines (Section~\ref{sec:appendix_impl}), as well as our data generation (Sec.~\ref{sec:appendix_datagen}).
%
We conclude with a discussion about limitations.

\section{Implementation Details}
\label{sec:appendix_impl}

%%%%%%%%%%%%%%%%%%
\paragraph{Levels of Operation for Scene Reconstruction.}
%%%%%%%%%%%%%%%%%%
Fig.~\ref{fig:level_of_operation} shows the different levels of operation at which our method operates on to reconstruct a 3D scene. 
Larger scenes are split into fixed size windows, chunk retrievals are made on smaller sized chunks for more expressability, and attention-based blending works on yet smaller sized patches to allow the method to choose among different retrievals at a finer detail. 

%%%%%%%%%%%%%%%%%%
\paragraph{Network Architecture.}
%%%%%%%%%%%%%%%%%%
Fig.~\ref{fig:architecture_refine} details the architecture of our networks for 3D super-resolution task. All networks are implemented in PyTorch~\cite{NEURIPS2019_9015}.


\begin{figure}[b!]
	\centering
	\includegraphics[width=\linewidth]{figures/appendix_level_of_operation.jpg}
	\caption{In our experiments, we use $64^3$ target chunks for target geometry, and larger scenes work in a sliding window fashion (left). The retrieval candidates are $16^3$ chunks (middle), and attention-based blending works on $4^3$ patches (right).}
	\label{fig:level_of_operation}
\end{figure}


%%%%%%%%%%%%%%%%%%
\paragraph{Inference Time and Number of Parameters.}
%%%%%%%%%%%%%%%%%%
We report the number of trainable parameters and the inference time for our method (both retrieval and refinement stage) along with that of the baselines in Tab.~\ref{tab:inference_params} for the 3D super-resolution task. All runtimes are reported on a machine with Intel(R) Xeon(R) Gold 6240 CPU @ 2.60GHz processor with an NVIDIA 2080Ti GPU. We use FLANN~\cite{muja2009fast} to speed up nearest neighbor lookups from the database. Our retrieval inference time is significantly higher than refinement due to multiple disk reads to retrieve chunks ($=$ number of chunks $\times$ number of retrievals). To avoid this overhead during training, once the retrieval networks have been trained, we preprocess the entire training set to extract retrievals before starting refinement stage training.
{
\begin{table}
    \centering
    \small
    \begin{tabular}{|l|l|l|} 
    \hline
    Method & Inference Time (s) & \# Parameters ($\times10^6$) \\
    \hline
    SGNN~\cite{dai2020sg} & 2.297 & 0.64\\
    ConvOcc~\cite{peng2020convolutional} & 1.707 & 1.04\\
    IFNet~\cite{chibane2020implicit} & 0.708 & 2.95\\
    Ours (Retrieval) & 0.784 & 0.77\\
    Ours (Refinement) & 0.012 & 1.49\\
    \hline
    \end{tabular}
    \caption{Comparison of inference time and number of trainable parameters on the 3D super-resolution task.}
    \vspace{-0.25cm}
    \label{tab:inference_params}
\end{table}
}

{
\setlength{\tabcolsep}{5pt}
\begin{table}
    \centering
    \small
    \resizebox{\linewidth}{!}{
    \begin{tabular}{|l|l|l|l|l|l|l|l|} 
    \hline
    \multirow{2}{*}{ \makecell{Chunk\\side (m)}} & \multicolumn{4}{c|}{Retrieval} & \multicolumn{3}{c|}{Refinement}  \\ 
    \cline{2-8}
                                & IoU$\uparrow$ & CD$\downarrow$ & NC$\uparrow$ & Entries & IoU$\uparrow$ & CD$\downarrow$ & NC$\uparrow$    \\ 
    \hline
    3.467                            & 0.53 & 0.074 &  0.72 & 43092  & 0.71 & 0.029 & 0.91   \\
    1.733                            & 0.60 & 0.041 & 0.85  &  344249 & 0.72 & 0.028 & 0.91 \\
    0.867                            & \textbf{0.67} & \textbf{0.033} & \textbf{0.87} & 2093592 & \textbf{0.75} & \textbf{0.026} & \textbf{0.92}  \\
    \hline
    \end{tabular}
    }
    \caption{Smaller sized chunk retrievals improve the performance of both retrieval and refinement, although at cost of a larger database. Evaluation performed on 3D super-resolution task on 3DFront dataset.}
    \label{tab:patchsize_ablation}
\end{table}
}

\begin{table}
    \centering
    \small
    \begin{tabular}{|l|l|l|l|l|l|} 
        \hline
        Variant & IoU$\uparrow$ & CD$\downarrow$ & F1$\uparrow$ & NC$\uparrow$ \\
        \thickhline
        Retrieval & 0.364 & 0.781 & 0.525 & 0.708 \\
        Backbone & 0.463 & 0.647 & 0.602 & 0.813 \\
        Naive & 0.432 & 0.684 & 0.576 & 0.798 \\
        Ours  & 0.478 & 0.635 & 0.601 & 0.811 \\
        \hline
    \end{tabular}
    \caption{In case of suboptimal retrievals, our method does not provide significant improvement over the backbone reconstruction quality. However, it is more robust to bad retrievals compared to a naive blending of retrieval features with input features. Networks trained on a ShapeNet subset with 8 classes and evaluated on a disjoined subset with 5 classes.}
    \label{tab:appendix_unseen_classes}
\end{table}

\begin{table}
    \centering
    \small
    \begin{tabular}{|l|l|l|l|} 
        \hline
        \# Train Scenes & IoU$\uparrow$ & CD$\downarrow$ & F1$\uparrow$ \\ 
        \hline
        3750 ~~~(25\%) & 0.711 & 0.0283 & 0.784 \\ 
        7500 ~~~(50\%) & 0.728 & 0.0275 & 0.791 \\ 
        11250 ~(75\%) & 0.741 & 0.0269 & 0.796 \\ 
        15000 ~(100\%) & 0.751 &  0.0265 & 0.801 \\
        \hline
    \end{tabular}
    \vspace{0.15cm}
    \caption{Ablation study w.r.t. the number of train scenes, evaluated on the 3D super-resolution task using the 3DFront dataset.}
    \label{tab:num_scene}
\end{table}

\begin{figure*}
	\centering
	\includegraphics[width=\linewidth]{figures/appendix_architecture.jpg}
	\caption{Network architecture used in our 3D super-resolution experiments. Convolution parameters are given as (input features, output feature, kernel size, stride), with default stride of 1 if not specified. Array of circles represent fully connected (FC) layers. 
	For the task of point cloud to surface reconstruction, the input chunk embedding network is a convolutional layer instead of MLP with a fully connected layer at the end on account of larger input chunk size (since input is a $128^3$ grid for surface reconstruction in comparison to $8^3$ grid for super-resolution, we use a chunk size of $32^3$ for inputs there). Additionally, the input feature extractor is deeper for point cloud to surface reconstruction on account on bigger input grid.}
	\label{fig:architecture_refine}
\end{figure*}

%%%%%%%%%%%%%%%%%%
\subsection{IFNet-based RetrievalFuse}
%%%%%%%%%%%%%%%%%%

\begin{figure}
	\centering
	\includegraphics[width=\linewidth]{figures/appendix_implicit.png}
	\caption{Integration of our RetrievalFuse approach to the implicit network of IFNet~\cite{chibane2020implicit}. 
	We use IFNet's encoder as the input feature encoder and their decoder as the implicit decoder. Additionally, we use a retrieval encoder similar to the IFNet encoder for obtaining features for the retrieval approximations. Further, a patch attention layer computes a blend coefficient grid and attention weight grid. For a given query point in space, features are sampled from input feature grids, retrieval feature grids. A blend coefficient value and attention weights are sampled from the blend coefficient grid and attention weight grid at the queried point. The sampled input features and retrieval features are blended based on these valued and finally decoded to an occupancy value by the IFNet decoder.}
	\label{fig:architecture implicit}
\end{figure}

%
To demonstrate the wide applicability of our method, we also demonstrate our approach integrated into the implicit-based reconstruction of IFNet~\cite{chibane2020implicit} to leverage our retrieved approximations.
%
We keep the IFNet encoder and decoder unmodified, and add an additional retrieval encoder for processing the retrieved reconstruction approximations.
%
This retrieval encoder is based on the original IFNet encoder, and works with chunks from the retrievals.
%
For a given point in space, features sampled at the point from feature volumes at different levels of the input encoder make up the input features.
%
Features sampled from the retrieval features volumes at this point for each of the $k$ retrievals make up the retrieval features.
%
Next, based on the feature volume at last layer of input and retrieval encoder, a blending coefficient grid and an attention weight grid is obtained.
%
To obtain these, the $8\times8\times8$ input feature volume and the $32\times32\times32$ retrieval feature volume are interpreted as 512 patch volumes of shape $1\times1\times1$ and $4\times4\times4$ respectively.
%
These input and corresponding retrieval patch volumes are mapped to a shared embedding space, from which we can get the blending coefficient (Eq.~6, main paper) and attention weights (Eq.~4, main paper).
%
Once we have the blending coefficient grid and attention weight grid, we can sample their values at the queried point.
%
Finally we blend the sampled input features and the sampled $k$ retrieved features (Eq.~5, main paper) to give the blended feature that is decoded by the IFNet decoder.
%


%%%%%%%%%%%%%%%%%%
\subsection{Baselines}
%%%%%%%%%%%%%%%%%%
We use the official implementations provided by the authors of IFNet~\cite{chibane2020implicit}, Convolutional Occupancy Networks~\cite{mescheder2019occupancy}, SGNN~\cite{dai2020sg}, Local Implicit Grids~\cite{jiang2020local} and Screened Poisson Reconstruction~\cite{kazhdan2013screened} in our experiments.
%
For 3D super-resolution experiments, the methods are provided with low-resolution distance field grids as inputs instead of voxel grid inputs.
%
In particular, for IFNet we use the \textit{ShapeNet32Vox} model for 3D super-resolution.
%
For surface reconstruction from point clouds for IFNet, the $128^3$ discretized point cloud is used with the \textit{ShapeNetPoints} model.
%
For Convolutional Occupancy Networks we use the $32^3$ \textit{voxel simple encoder} for 3D super-resolution, and a $64^3$ \textit{point net local pool} encoder for point cloud surface reconstruction.
%
For SGNN, we use a $64^3$ resolution with nearest-neighbor upsampling to a $64^3$ grid for the input.
%
For Local Implicit Grids we found that the part sizes $0.25\times$ shape size for ShapeNet and $0.35\times$ window size for 3DFront and Matterport3D worked best at the sparsity of the input point cloud.

%%%%%%%%%%%%%%%%%%%%%%%%%%%%%%%%%%%%%%%%%%%%%%%%%%%%%%%%%%%%%%%%%%%%%%%%
%%%%%%%%%%%%%%%%%%%%%%%%%%%%%%%%%%%%%%%%%%%%%%%%%%%%%%%%%%%%%%%%%%%%%%%%
%%%%%%%%%%%%%%%%%%%%%%%%%%%%%%%%%%%%%%%%%%%%%%%%%%%%%%%%%%%%%%%%%%%%%%%%

\section{Data Generation and Evaluation Metrics}
\label{sec:appendix_datagen}

%%%%%%%%%%%%%%%%%%
\paragraph{Data generation.}
%%%%%%%%%%%%%%%%%%
%
As specified in the main paper, the targets for both 3D super-resolution and surface reconstruction from point cloud tasks are $64^3$ distance field grids.
%
Training and inference on larger scenes is done in a sliding window manner with a window stride of 64.
%
We use SDFGen\footnote{\href{https://github.com/christopherbatty/SDFGen}{https://github.com/christopherbatty/SDFGen}} to generate these distance field targets.
%
Low-resolution distance field inputs are generated in a similar manner at a coarser resolution.
%
Point cloud samples for surface reconstruction task are generated as random samples on the surface of meshes generated from target distance fields. 
%

For IFNet~\cite{chibane2020implicit}, Convolutional Occupancy Networks~\cite{mescheder2019occupancy}, and our implicit variant, all of which need supervision in the form of points along with their occupancies, we first extract meshes from the target distance fields using the marching cubes algorithm~\cite{lorensen1987marching}.
%
These meshes are then made watertight using \textit{implicit waterproofing}~\cite{chibane2020implicit} from which points and their occupancies are finally sampled.
%
SGNN is provided the same inputs and targets as ours for training, with the respective inputs upsampled to match the target $64^3$ resolution grid.
%
Local Implicit Grids~\cite{jiang2020local} is trained on ShapeNet, and Screened Poisson Reconstruction~\cite{kazhdan2013screened} does not require training; however, both methods are provided high-resolution normals to obtain oriented point clouds as inputs.


%%%%%%%%%%%%%%%%%%
\paragraph{Evaluation Metrics.}
%%%%%%%%%%%%%%%%%%
%
We follow the definition and implementations of Chamfer $\ell_1$ Distance, Normal Consistency, and F-Score from \cite{peng2020convolutional}.
%
Specifically, Chamfer $\ell_1$ Distance (CD) is defined as:
\begin{equation*}
    \small
    \begin{split}
        \mathrm{CD}(\mathcal{M}_{pred}, \mathcal{M}_{gt}) = \frac{1}{2}(\mathrm{Acc}(\mathcal{M}_{pred}, \mathcal{M}_{gt}) \\ +  \mathrm{Comp}(\mathcal{M}_{pred}, \mathcal{M}_{gt}))
    \end{split}
\end{equation*}
where $\mathcal{M}_{pred}$ and $\mathcal{M}_{gt}$ are the predicted and target meshes (obtained by running marching cubes on predicted and target distance fields).
%
$Acc(.)$ and $Comp(.)$ are accuracy and completeness given as:
\small
\begin{equation*}
        \mathrm{Acc}(\mathcal{M}_{pred}, \mathcal{M}_{gt}) = \frac{1}{\left|\partial\mathcal{M}_{pred}\right|}\int_{\partial\mathcal{M}_{pred}}\min_{\mathbf{q}\in \partial\mathcal{M}_{gt}}\norm{\mathbf{p} - \mathbf{q}}\mathrm{d}\mathbf{p},
\end{equation*}
\normalsize
and
\small
\begin{equation*}
    \mathrm{Comp}(\mathcal{M}_{pred}, \mathcal{M}_{gt}) = \frac{1}{\left|\partial\mathcal{M}_{gt}\right|}\int_{\partial\mathcal{M}_{gt}}\min_{\mathbf{p}\in \partial\mathcal{M}_{pred}}\norm{\mathbf{p} - \mathbf{q}}\mathrm{d}\mathbf{q}
\end{equation*}
\normalsize
with $\partial\mathcal{M}_{pred}$ and $\partial\mathcal{M}_{gt}$ denoting the surfaces of the meshes.
%
Normal Consistency (NC) is defined as:
\begin{align*}
    \footnotesize
    \begin{split}
    \mathrm{NC}(\mathcal{M}_{pred}, \mathcal{M}_{gt}) &= \frac{1}{2\left|\partial\mathcal{M}_{pred}\right|}\int_{\partial\mathcal{M}_{pred}}\abs{n(\mathbf{p}) \cdot n(\mathrm{proj}_2(\mathbf{p}))}\mathrm{d}\mathbf{p} \\ &+
    \frac{1}{2\left|\partial\mathcal{M}_{gt}\right|}\int_{\partial\mathcal{M}_{gt}}\abs{n(\mathbf{q}) \cdot n(\mathrm{proj}_1(\mathbf{q}))}\mathrm{d}\mathbf{q}
    \end{split}
\end{align*}
where $(.)$ indicates inner product, $n(\mathbf{p})$ and $n(\mathbf{q})$ are the unit normal vectors on the mesh surface, and $\mathrm{proj}_2(\mathbf{p})$ and $\mathrm{proj}_1(\mathbf{q})$ are projections of $\mathbf{p}$ and $\mathbf{q}$ onto mesh surfaces $\partial\mathcal{M}_{pred}$ and $\partial\mathcal{M}_{gt}$ respectively.
%
F-Score \cite{tatarchenko2019single} is defined as the harmonic mean of precision and recall, where recall is fraction of points on $\mathcal{M}_{gt}$ that lie within a certain distance to $\mathcal{M}_{pred}$, and precision is the fraction of points on $\mathcal{M}_{pred}$ that lie within a certain distance to $\mathcal{M}_{gt}$.
%
For calculating the volumetric IoU, we first voxelize the meshes $\mathcal{M}_{gt}$ and $\mathcal{M}_{pred}$ with voxel sizes of $0.054$m for 3DFront, $0.0375$m for Matterport3D, and resolutions $64^3$ for ShapeNet.
%
The IoU is then given as:
\begin{equation*}
    \small
    \mathrm{IoU} = \frac{\mathrm{Voxels}(\mathcal{M}_{pred}) \cap \mathrm{Voxels}(\mathcal{M}_{gt})}{\mathrm{Voxels}(\mathcal{M}_{pred}) \cup \mathrm{Voxels}(\mathcal{M}_{gt})}
\end{equation*}

\begin{figure*}
	\centering
	\includegraphics[width=\linewidth]{figures/appendix_ablation_components.JPG}
	\caption{Additional qualitative evaluation of our method (\textit{Ours}) in comparison to $1^\mathrm{st}$ nearest neighbor retrieval (\textit{1-NN Retrieval}), our refinement network without retrievals (\textit{Backbone}) and naive fusion of retrieved approximations during refinement (\textit{Naive}).}
	\label{fig:appendix_components}
\end{figure*}

%%%%%%%%%%%%%%%%%%%%%%%%%%%%%%%%%%%%%%%%%%%%%%%%%%%%%%%%%%%%%%%%%%%%%%%%
%%%%%%%%%%%%%%%%%%%%%%%%%%%%%%%%%%%%%%%%%%%%%%%%%%%%%%%%%%%%%%%%%%%%%%%%
%%%%%%%%%%%%%%%%%%%%%%%%%%%%%%%%%%%%%%%%%%%%%%%%%%%%%%%%%%%%%%%%%%%%%%%%

\section{Additional Evaluation}
\label{sec:appendix_evaluation}

%%%%%%%%%%%%%%%%%%
\subsection{Ablation Studies}
%%%%%%%%%%%%%%%%%%

\paragraph{Chunk Embedding Space Visualization.}
%
Fig.~\ref{fig:latent_space} visualizes the embedding space used for retrieving chunks from our database.
%
Chunks with similar geometry end up lying closer in this space.

\begin{figure*}
	\centering
	\includegraphics[width=\linewidth]{figures/appendix_latent_space.jpg}
	\caption{(a) Chunk embedding space visualized for 5000 chunks from 3DFront test set. This embedding space used for retrievals from the database by projecting an input chunk into this space (visualized as green dots) and retrieving k-nearest database chunks (visualized by yellow dots) from it. (b) Input queries and their corresponding 4 nearest neighbors from the embedding space. For the sake of visual clarity, input queries are visualized as their corresponding ground truth reconstruction.}
	\label{fig:latent_space}
\end{figure*}


\paragraph{Effect of retrieved chunk size on the performance of our method.}
%
Tab.~\ref{tab:patchsize_ablation} evaluates our method with retrieval approximations of different chunk sizes for retrieval.
%
A chunk size that is too large cannot effectively capture the diversity of various scene arrangements, while smaller sizes can represent a wider variety of geometry, at the cost of an increased database size.
%

\paragraph{Effect of number of training scenes used for creating the database}
%
Tab.~\ref{tab:num_scene} shows the effect of number of chunks in the database on our method's performance. Availability of a wider variety of chunks helps reconstruction.

%%%%%%%%%%%%%%%%%%
\subsection{Additional Qualitative Results}
%%%%%%%%%%%%%%%%%%
\begin{figure*}
	\centering
	\includegraphics[width=\linewidth]{figures/appendix_bad_retrieval_3dfront.JPG}
	\caption{Suboptimal retrievals do not improve results significantly over our Backbone network. However, reconstruction produced are also not degraded due to subobtimal retrievals. Qualitative results from 3DFront super-resolution task.}
	\label{fig:bad_retrieval_3dfront}
\end{figure*}

\begin{figure*}
	\centering
	\includegraphics[width=\linewidth]{figures/appendix_robust.JPG}
	\caption{(Left) Suboptimal retrievals (NN1) when the our method is trained on a ShapeNet subset of 8 classes and evaluated on another 5 classes. The database contains chunks only from the original 8 classes. In this case, the suboptimal retrievals don't help the reconstruction, and the quality of reconstruction does not significantly improve over our backbone network. However, in contrast to naive fusion of retrieval features, our reconstruction quality does not degrade over the backbone. (Right) If the database if augmented with new chunks from train set of the new 5 classes, the reconstruction quality visibly improves without retraining.}
	\label{fig:shapenet_transfer_robust}
\end{figure*}

\begin{figure*}
	\centering
	\includegraphics[width=\linewidth]{figures/appendix_superresolution.JPG}
	\caption{Additional qualitative results on 3DFront (left three) and Matterport3D (right three) on 3D super-resolution task.}
	\label{fig:appendix_superresolution}
\end{figure*}

\begin{figure*}
	\centering
	\includegraphics[width=\linewidth]{figures/appendix_surface_reconstruction.JPG}
	\caption{Additional qualitative results on 3DFront (left three) and Matterport3D (right three) on point cloud to surface reconstruction task.}
	\label{fig:appendix_surface_reconstruction}
\end{figure*}

\begin{figure*}
	\centering
	\includegraphics[width=\linewidth]{figures/appendix_shapenet.JPG}
	\caption{Qualitative results on ShapeNet dataset on 3D super-resolution (left three) and point cloud to surface reconstruction (right three) tasks.}
	\label{fig:appendix_shapenet}
\end{figure*}


We provide additional qualitative evaluation of our method on 3DFront and Matterport3D super-resolution and point cloud to surface reconstruction tasks in Fig.~\ref{fig:appendix_superresolution} and Fig.~\ref{fig:appendix_surface_reconstruction} respectively. Qualitative evaluation on ShapeNet for both of the tasks is provided in Fig.~\ref{fig:appendix_shapenet}. Further, additional qualitative visualization for \textit{Effect of retrieval and attention-based refinement} (main paper section~4.3) is provided in Fig.~\ref{fig:appendix_components}. 

%%%%%%%%%%%%%%%%%%%%%%%%%%%%%%%%%%%%%%%%%%%%%%%%%%%%%%%%%%%%%%%%%%%%%%%%
%%%%%%%%%%%%%%%%%%%%%%%%%%%%%%%%%%%%%%%%%%%%%%%%%%%%%%%%%%%%%%%%%%%%%%%%
%%%%%%%%%%%%%%%%%%%%%%%%%%%%%%%%%%%%%%%%%%%%%%%%%%%%%%%%%%%%%%%%%%%%%%%%

\section{Additional Discussion}

The result in the main paper as well is the additional experiments in this document show the broad applicability of our method, achieving state-of-the-art reconstruction and super-resolution outputs.
%
Nevertheless, our approach still has limitations as discussed in the main paper.
%
In particular, if the retrieval approximations are suboptimal, they will not help in the refinement process.
%
Fig.~\ref{fig:bad_retrieval_3dfront} visualizes some samples where the retrieval approximations don't help the reconstruction.
%
However, in these cases, even though the retrievals don't help the reconstruction, they also don't worsen the reconstruction.
%
This is achieved by the blending network effectively ignoring the retrievals in such cases.
%
The dependence on good retrievals can be observed more clearly in the following experiment.
%
We train our retrieval and refinement networks on a ShapeNet subset of $8$ classes.
%
The dictionary is created using chunks from the same $8$ classes.
%
The trained networks are evaluated on a subset of new 5 classes.
%
As shown in Tab.~\ref{tab:appendix_unseen_classes} and Fig.~\ref{fig:shapenet_transfer_robust}, our method doesn't improve significantly over the backbone network due to low quality retrievals.
%
Compared to a naive fusion of features from retrievals however, which learns to rely on retrievals during training, our method is more robust.
%

%
A limitation of our method is cubic growth in number of chunks in the database with the decrease in patch size.
%
As observed in Tab. \ref{tab:patchsize_ablation}, smaller chunk retrievals help both retrieval and refinement.
%
This however comes at the cost of more patches in the database, making the database indexing and retrieval slower.
%

\end{appendix}

\end{document}
