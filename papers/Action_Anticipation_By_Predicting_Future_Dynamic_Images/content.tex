\documentclass[10pt,twocolumn,letterpaper]{article}

\usepackage{cvpr}
\usepackage{times}
\usepackage{epsfig}
\usepackage{graphicx}
\usepackage{amsmath}
\usepackage{amssymb}

% Include other packages here, before hyperref.

% If you comment hyperref and then uncomment it, you should delete
% egpaper.aux before re-running latex.  (Or just hit 'q' on the first latex
% run, let it finish, and you should be clear).
\usepackage[breaklinks=true,bookmarks=false]{hyperref}

\cvprfinalcopy % *** Uncomment this line for the final submission

\def\cvprPaperID{****} % *** Enter the CVPR Paper ID here
\def\httilde{\mbox{\tt\raisebox{-.5ex}{\symbol{126}}}}

% Pages are numbered in submission mode, and unnumbered in camera-ready
%\ifcvprfinal\pagestyle{empty}\fi
\setcounter{page}{1}
\begin{document}

%%%%%%%%% TITLE
\title{Action Anticipation Through Generating The Future Video Representantion.}

\author{Cristian Rodriguez \qquad Basura Fernando \qquad Hongdong Li \\
% For a paper whose authors are all at the same institution,
% omit the following lines up until the closing ``}''.
% Additional authors and addresses can be added with ``\and'',
% just like the second author.
% To save space, use either the email address or home page, not both
ARC Centre if Excellence for Robotic Vision (ACRV)\\
Research School of Engineering, Australian National University\\
{\tt\small firstname.lastname@anu.edu.au}
}

\maketitle
%\thispagestyle{empty}


\begingroup
\let\clearpage\relax
% In this section, we aim to empirically evaluate the statistical and computational behaviours of our proposed methods. To this end, we consider three sets of examples.
\begin{itemize}
    \item The first model performs joint parameter and state estimation for a discretely observed stochastic differential equation. This model was used in \citet{mider2021continuous} to assess the performance of their forward-guiding backward-filtering method. We demonstrate here how to use auxiliary samplers for the same purpose and show the competitiveness of our approach.
    \item The second one is a multivariate stochastic volatility model known to be challenging for Gaussian approximations and used as a benchmark in, for example, \citet{guarniero2017iterated,finke2021csmc}. This model has latent Gaussian dynamics, and an observation model which happen to both be differentiable with respect to the latent state, so that all the methods of Section~\ref{subsec:auxiliary-lgssm} and Section~\ref{sec:pgibbs_samplers} apply. We consider the same parametrisation as in \citet{finke2021csmc}, which makes the system lack ergodicity and the standard particle Gibbs samplers not converge.
    \item The last one is a spatio-temporal model with independent latent Gaussian dynamics and is used in \citet{cruscino2022highdim} as a benchmark for high dimensional filtering. This model is akin to a type of dynamic random effect model in the sense that the latent states only interact at the level of the observations. This model is used to illustrate how latent structure can be used to design computationally efficient Kalman samplers that beat cSMC ones when runtime is taken into account.
\end{itemize}


Throughout this section, when using an auxiliary cSMC sampler, be it the sequential or the parallel-in-time formulation, we use $N=25$ particles, and a target acceptance rate of $25\%$ across all time steps. This is more conservative than the recommendation of \citet{finke2021csmc}, corresponding to $1 - (1 + N)^{-1/3}\approx 66\%$. The difference stems from the fact that it may happen that the methods do not reach the relatively high acceptance rate implied by the more optimistic target for all time steps, even with very small $\delta$ values. As a consequence, the sampler is ``stuck'' by only proposing very correlated trajectories in some places. This is mostly due to the largely longer time series considered here. Softening this resulted in empirically better mixing. Furthermore, for all the experiments, and following \citet[][]{titsias2018,finke2021csmc}, we consider $\delta \Sigma_t = \delta_t I$, with $\delta_t$ being constant across time steps for the Kalman samplers. Choosing a specific form for $\Sigma_t$ is akin to pre-conditioning and is left for future works. We then calibrate $\delta_t$ to achieve the desired acceptance rate (globally for Kalman samplers or per time step for the cSMC samplers) and the actual acceptance rate is reported in the relevant sections.

The implementation details for all the experiments are as follows: whenever we say that a method was run on a CPU, we have used an AMD\textsuperscript{\textregistered} Ryzen Threadripper 3960X with 24 cores, and whenever the method has been run on a GPU, we used a Nvidia\textsuperscript{\textregistered} GeForce RTX 3090 GPU with 24GB memory. All the experiments were implemented in Python~\citep{Rossum2009Python} using the JAX library~\citep{jax2018github} which natively supports CPU and GPU backends as well as automatic differentiation that we use to compute all the gradients we required. The code to reproduce the experiments listed below can be found at the following address:~\url{https://github.com/AdrienCorenflos/aux-ssm-samplers}.



\subsection{Prefix-sum sampling for LGSSMs}
\label{subsec:prefix-sum-sampling}
We now describe how to sample from $q(\cdot \mid y_{0:T})$ using similar methods as in \citet{Sarkka2021temporal,yaghoobi2021parallel,Yaghoobi2022sqrt}. Given that both the target of the log-likelihood (as a sum of $T$ independent terms) and the marginal log-likelihood of the LGSSM approximation~\citep{Sarkka2021temporal} can be computed in $\bigO(\log(T))$ on parallel hardware, this is the only missing piece for implementing a parallel-in-time version of our auxiliary Kalman samplers. While several different formulations~\citep[see, e.g.,][]{Doucet:2010,Fruhwirth1994data} may be employed to do so, we here focus on the forward filtering backward sampling (FFBS)~\citep{Fruhwirth1994data} approach.

We can compute the filtering distributions $q(x_t \mid y_{0:t}) = \mathcal{N}(x_t; m_t, P_t)$ for \eqref{eq:general-LGSSM} in parallel using the methods of \citet{Sarkka2021temporal}. Furthermore, we know~\citep[by adding a bias term to][Proposition 1]{Fruhwirth1994data} that
\begin{equation}
    \label{eq:backward}
    \begin{split}
        q(x_T \mid y_{0:T}) &= \mathcal{N}(x_T; m_T, P_T) \\
        q(x_t \mid x_{t+1}, y_{0:t}) &= \mathcal{N}\left(x_t; m_t + G_{t} x_{t+1} - F_t m_t - b_t, \Sigma_t\right), \quad t < T,
    \end{split}
\end{equation}
where $G_t = P_t F_t^{\top}\left(F_t P_t F_t^{\top} + Q_t\right)^{-1}$ and $\Sigma_t = P_t - G_t (F_t P_t F_t^{\top} + Q_t) G_t^{\top}$ for all $t < T$.

We can furthermore rearrange the terms to express $\hat{X}_t \sim q(x_t \mid \hat{X}_{t+1}, y_{0:t})$ recursively as
\begin{equation}
    \label{eq:recursive-sampling}
    \hat{X}_t = G_{t} \hat{X}_{t+1} + \nu_t,
\end{equation}
where all the $\nu_t$'s are independent, and $\nu_t \sim \mathcal{N}(m_t - G_t (F_t m_t + b_t), \Sigma_t)$ for all $t < T$. We also let $G_T = 0$, so that we can then define $\nu_T \sim \mathcal{N}(m_T, P_T)$. Because the means and covariances of the $\nu_t$'s only depend on the LGSSM coefficients and the filtering means and covariances at time $t$, they can be sampled fully in parallel. To sample from $q(x_{0:T} \mid y_{0:T})$ we then need to apply \eqref{eq:recursive-sampling} to the pre-sampled sequence $U_t \sim \nu_t$, $t=0, \ldots, T$. However, the recursive dependency in \eqref{eq:recursive-sampling} is not directly parallelisable, and we instead need to rephrase it in terms of an associative operator, which will allow us to use prefix-sum primitives~\citep{blelloch1989scans}. Thankfully, this is readily done by considering the $\circ$ operator defined as follows
\begin{equation}
    \label{eq:sampling-op}
    \begin{split}
    (G_{ij}, U_{ij})
        &= (G_i, U_i) \circ (G_j, U_j), \quad \text{where} \\
        G_{ij} &= G_i G_j, \quad U_{ij} = G_i U_j + U_i.
    \end{split}
\end{equation}
\begin{proposition}
    \label{prop:prefix-sum-sampling}
    The backward prefix-sum of operator $\circ$ applied to the sequence $(G_t, U_t)$, $t=0, \ldots, T$, recovers the pathwise smoothing distribution $q(x_{0:T} \mid y_{0:T})$, that is, if $(\tilde{G}_t, \tilde{U}_t) = (G_t, U_t) \circ \ldots \circ (G_T, U_T)$, then $(\tilde{U}_0, \ldots, \tilde{U}_T)$ is distributed according to $q(x_{0:T} \mid y_{0:T})$.
\end{proposition}
\begin{proof}
    The operator $\circ$ defined in \eqref{eq:sampling-op} is clearly associative. We prove that its result corresponds to sampling from the pathwise smoothing distribution by reversed induction: suppose that $(\tilde{U}_t, \ldots, \tilde{U}_T)$ is distributed according to $q(x_{t:T} \mid y_{0:T})$, then $\tilde{U}_{t-1} = G_{t-1} \tilde{U}_t + U_{t-1}$, which is distributed according to $q(x_{t-1} \mid \tilde{U}_{t}, y_{0:t-1})$ as discussed before, so that $(\tilde{U}_{t-1}, \ldots, \tilde{U}_T)$ is distributed according to $q(x_{t-1:T} \mid y_{0:T})$. The initial case follows from the definition of $U_T$.
\end{proof}

To summarise, in order to perform prefix-sum sampling of LGSSMs, it suffices to use the parallel-in-time Kalman filtering method of \citet{Sarkka2021temporal} to compute the filtering means and covariances $m_t$, $P_t$, $t=0, \ldots, T$, then form all the elements $G_t$ and sample $U_t$ fully in parallel, and finally, apply the prefix-sum primitive~\citep{blelloch1989scans} to $(G_t, U_t)_{t=0}^T$ with the associative operator $\circ$.

\subsection{Divide-and-conquer sampling for LGSSMs}
\label{subsec:dnc-sampling}
We now present a divide-and-conquer alternative to Section~\ref{subsec:prefix-sum-sampling} for PIT sampling from the pathwise smoothing distribution of LGSSMs. The method is based on recursively finding tractable Gaussian expressions for the ``bridging'' $q(x_l \mid y_{0:T}, x_k, x_m)$, $0 \leq k < l < m \leq T$ of the smoothing distribution. This will allow us to derive a tree-based divide-and-conquer sampling mechanism for the pathwise smoothing distribution $q(x_{0:T} \mid y_{0:T})$.

Suppose we are given the LGSSM \eqref{eq:general-LGSSM}, then given three indices $0 \leq k < l < m \leq T$. We have
\begin{equation}
    \begin{split}
        q(x_l \mid y_{0:T}, x_k, x_m)
        = \frac{q(x_k, x_l \mid y_{0:T}, x_m)}
        {q(x_k \mid y_{0:T}, x_m)}
    \end{split}
\end{equation}
with, furthermore,
\begin{equation}
    q(x_k, x_l \mid y_{0:T}, x_m) = q(x_k \mid y_{0:T}, x_l) q(x_l \mid y_{0:T}, x_m)
\end{equation}
thanks the to Markovian structure of the model.
Now let $q(x_k \mid y_{0:T}, x_l)$ and $q(x_l \mid y_{0:T}, x_m)$ be given by
\begin{equation}
    \begin{split}
        q(x_k \mid y_{0:T}, x_l) &= \mathcal{N}(x_k; E_{k:l} x_l + g_{k:l}, L_{k:l})\\
        q(x_l \mid y_{0:T}, x_m) &= \mathcal{N}(x_k; E_{l:m} x_m + g_{l:m}, L_{l:m})
    \end{split}
\end{equation}
for some parameters $E_{k:l}$, $g_{k:l}$, $L_{k:l}$, $E_{l:m}$, $g_{l:m}$, and $L_{l:m}$ that we will define below. Then we can write
\begin{equation}
    \begin{split}
        &q(x_k, x_l \mid y_{0:T}, x_m)\\
        &= \mathcal{N}\left( \begin{pmatrix}
                                 x_l \\ x_k
        \end{pmatrix} ;
        \begin{pmatrix}
            E_{l:m} x_{m} + g_{l:m} \\
            E_{k:l} E_{l:m} x_{m} + E_{k:l} g_{l:m} + g_{k:l}
        \end{pmatrix},
        \begin{pmatrix}
            L_{l:m}         & L_{l:m} E_{k:l}^\top                   \\
            E_{k:l} L_{l:m} & E_{k:l} L_{l:m} E_{k:l}^\top + L_{k:l}
        \end{pmatrix} \right)
    \end{split}
\end{equation}
giving both the marginal distribution of $x_k$
\begin{equation}
    \begin{split}
        q(x_k \mid y_{0:T}, x_m)
        &= \mathcal{N}(x_k;
        E_{k:l} E_{l:m} x_{m} + E_{k:l} g_{l:m} + g_{k:l},
        E_{k:l} L_{l:m} E_{k:l}^\top + L_{k:l} ) \\
        &= \mathcal{N}( x_k; E_{k:m} x_{m} + g_{k:m}, L_{k:m} ),
    \end{split}
\end{equation}
where
\begin{equation}
    \begin{split}
        E_{k:m} = E_{k:l} E_{l:m}, \quad g_{k:m} = E_{k:l} g_{l:m} + g_{k:l}, \quad
        L_{k:m} = E_{k:l} L_{l:m} E_{k:l}^\top + L_{k:l},
    \end{split}
    \label{eq:EgL_comb}
\end{equation}
and (after simplification for \eqref{eq:EgL_comb}) the conditional distribution of $x_l$
\begin{equation}
    \begin{split}
        q(x_l \mid y_{0:T}, x_k, x_m) = \mathcal{N}(x_l; G_{k:l:m} x_k + \Gamma_{k:l:m} x_m + w_{k:l:m}, V_{k:l:m} ),
    \end{split}
    \label{eq:gauss_l_km}
\end{equation}
for
\begin{equation}
    \label{eq:recursive_params}
    \begin{split}
        G_{k:l:m} &= L_{l:m} E_{k:l}^\top L_{k:m}^{-1}, \\
        \Gamma_{k:l:m} &= E_{l:m} - G_{k:l:m} E_{k:m},
    \end{split} \qquad
    \begin{split}
        w_{k:l:m} &= g_{l:m} - G_{k:l:m} g_{k:m}, \\
        V_{k:l:m} &= L_{l:m} - G_{k:l:m} L_{k:m} G_{k:l:m}^\top.
    \end{split}
\end{equation}

This construction provides a recursive tree structure for sampling from $q(x_{0:T} \mid y_{0:T})$ which can be initialised by
\begin{equation}
    \begin{split}
        q(x_t \mid y_{0:T}, x_{t+1})
        &= \mathcal{N}(x_t;
        E_{t:t+1} x_{t+1} + g_{t:t+1}, L_{t:t+1} ),
    \end{split}
\end{equation}
with
\begin{equation}
    \label{eq:init-dnc}
    \begin{split}
        E_{t:t+1} = P_t F_t^\top (F_t P_t F_t^\top + Q_t)^{-1}, \quad
        g_{t:t+1} = m_t - E_{t:t+1} (F_t m_t + b_t), \quad
        L_{t:t+1} = P_t - E_{t:t+1} F_t P_t,
    \end{split}
\end{equation}
and $q(x_T \mid y_{0:T}) = \mathcal{N}(x_T; m_T, P_T)$. Finally, noting that
\begin{equation}
    q(x_0 \mid y_{0:T}, x_T) = \mathcal{N}(x_0; E_{0:T} m_T + g_{0:T}, L_{0:T}),
\end{equation}
we can combine these identities to form a divide-and-conquer algorithm.

To summarise, in order to perform divide-and-conquer sampling of LGSSMs, it suffices, as in Section~\ref{subsec:prefix-sum-sampling}, to use the parallel-in-time Kalman filtering method of \citet{Sarkka2021temporal} to compute the filtering means and covariances $m_t$, $P_t$, $t=0, \ldots, T$. After this, we can recursively compute the tree of elements $E_{k:m}, g_{k:m}, L_{k:m}$, together with the auxiliary variables $G_{k:l:m}, w_{k:l:m}, \Gamma_{k:l:m}, V_{k:l:m}$, starting from $E_{t:t+1}, g_{t:t+1}, L_{t:t+1}$, for $t=0, 1, \ldots, T-1$, then $E_{t-1:t+1}, g_{t-1:t+1}, L_{t-1:t+1}$, for $t=1, 3, 5, \ldots, 2\lfloor (T - 1)/2\rfloor + 1$, etc. Once this has been done, we can then sample from $q(x_T \mid y_{0:T})$, then from $q(x_0 \mid y_{0:T}, x_T)$, then $x_{\lfloor T/2 \rfloor}$ conditionally on $x_0$ and $x_T$, then, in parallel $x_{\lfloor T/4 \rfloor}$ and $x_{\lfloor 3 T/4 \rfloor}$, conditionally on the rest, and continue until all have been sampled.

\begin{abstract}
In this work we propose a method to generate image video representation that will be used for early activity recognition and video forecasting.
\end{abstract}

\section{Introduction}
% present the problem and explain why action anticipation is important 1 paragraph?
Almost all human interaction is based in certain way of predicting what will be the next action of others humans. For examples, when we are driving cars we tend to predict if some pedestrian is going to cross in a share street to avoid any accident. When we we are playing sports, like tennis, we tend to predict where the ball will go depending on the human pose. When we greet another people, we try to predict what is the way that the other person want to salute and we perform depending on that, i.e. handshake, high five, hug or kiss. This particular ability, that is very important for humans, has to be transfer to computers if we want to create safer robots with better sport and social abilities.

In the intention to understand actions from a video, computer vision community had defined two different task. Action recognition, which aims to recognize what is the action that has been perform in a complete video sequence, with great progress CITES. Action anticipation, which become popular recently and tries to recognize the action of a video sequence as early as it is possible, thus, using only a few frames of the video sequence \cite{aliakbarian2017encouraging, ma2016learning, ryoo2011human, lan2014hierarchical, soomro2016online, soomro2016predicting, yu2012predicting}. 

In this paper, we introduce a method to generate future video representation that can be used for action anticipation and hallucinate what is the next frame in the sequence. In specific we are using an autoencoder network that receive as an input image video representation (dynamic image) in time $t$ and  the output is the dynamic image in time $t+1$.

Moreover, we present different loss functions that has been used in a multitask learning scheme to generate better dynamic images.

In summary, we make the following contributions:
\begin{itemize}
 \item We hallucinate future image video representations. 
 \item We propose several loss to train our generative module using multitask learning scheme.
 \item We generate a future appearance image using the generated future image video representation .
 \item Activity recognition using the generated future appearance image and video representation.
 \item We archieve state-of-the-art performance for early activity recognition.
\end{itemize}

The rest of the paper is organized as follows: Section 2 provides an overview of related work in image video representation, early action recognition and video forecasting. Section 3 describe our method to generate future video representations. Section 4 presents our experiments over JHMDB-21 and UCF101 . Section 5 summarizes our findings and discusses future directions.

\section{Related Work}
\section{Method}
\subsection{Dynamic Image}

Dynamic image has been proposed by Bilen et al. in 2016 \cite{bilen2016dynamic} are still RGB image that summarize the appearance and dynamics of a whole video sequence, as can be seen in the Figure \ref{fig:dis}. It has been used the rank pooling proposed by Fernando et al \cite{FernandoGMGT15} to create such still image. Rank pooling consist on represent a video as a ranking function for its frames $I_1, ..., I_n$. In more details let ψ(It ) $\in$ Rd be a representation or feature vector extracted from each individual frame It in the video. Let Vt = τ =1 ψ(Iτ ) be
ttime average of these features up to time t. The ranking
function associates to each time t a score S(t|d) = hd, Vt i,
where d ∈ Rd is a vector of parameters. The function pa-
rameters d are learned so that the scores reflect the rank of
the frames in the video. Therefore, later times are associ-
ated with larger scores, i.e. q > t =⇒ S(q|d) > S(t|d).
Learning d is posed as a convex optimization problem using
the RankSVM [26] formulation:

d∗ = ρ(I1, . . . , IT ; ψ) = argmin E(d),
d
λ
E(d) = kdk2 +
 (1)
2
2
 X
×
 max{0, 1 − S(q|d) + S(t|d)}.
T (T − 1)q>t
The first term in this objective function is the usual
quadratic regularizer used in SVMs. The second term is
a hinge-loss soft-counting how many pairs q > t are incor-
rectly ranked by the scoring function. Note in particular that
a pair is considered correctly ranked only if scores are sep-
arated by at least a unit margin, i.e. S(q|d) > S(t|d) + 1.
Optimizing eq. (1) defines a function ρ(I1 , . . . , IT ; ψ)
that maps a sequence of T video frames to a single vector
d∗ . Since this vector contains enough information to rank
all the frames in the video, it aggregates information from
all of them and can be used as a video descriptor. In the rest
of the paper we refer to the process of constructing d∗ from
a sequence of video frames as rank pooling.

Computing a dynamic image entails solving the opti-
mization problem of eq. (1). While this is not particularly
slow with modern solvers, in this section we propose an ap-
proximation to rank pooling which is much faster and works
as well in practice. Later, this technique, which we call
approximate rank pooling, will be critical in incorporating
rank pooling in intermediate layers of a deep CNN and to
allow back-prop training through it.


% Although compute dynamic image is not difficult for moderns solvers a fast dynamic image computation is proposed in \cite{bilen2016dynamic}, which is This approximation of rank pooling will be used to generate the last RGB image of a dynamic image.


\begin{figure}[ht!]
 \centering
 \includegraphics[width=0.20\textwidth]{imgs/DI1.jpg}
 \hspace{-4pt}
 \includegraphics[width=0.20\textwidth]{imgs/DI2.jpg}\\
 \vspace{0pt}
 \includegraphics[width=0.20\textwidth]{imgs/DI3.jpg}
 \hspace{-4pt}
 \includegraphics[width=0.20\textwidth]{imgs/DI4.jpg}
 \caption{Samples of dynamic images that summarize videos from UCF101 dataset.}
 \label{fig:dis}
\end{figure}

\begin{equation}
 DI = \displaystyle \sum_{t+1}^T \alpha_t I_t
\end{equation}

\begin{equation}
 \alpha_t = 2t - T -1
\end{equation}
\subsection{Future Dynamic Image Generation}
\begin{equation}
 \mathcal{L}_{DL} = ||\hat{DI}_{t+1} - DI_{t+1}||_2
\end{equation}

\subsection{Future Frame Generation}
\begin{align}
 DI_{10} &=  \displaystyle \sum_{i=1}^{10} \alpha_i I_t \\
 DI_{10} &=  \alpha_{10}        I_{10} + \displaystyle \sum_{i=1}^{9} \alpha_i I_t \\
 I_{10} &= \frac{DI_{10} - \displaystyle \sum_{i=1}^{9} \alpha_i I_t}{\alpha_{10}}
\end{align}

\subsection{Multitask learning}
\begin{equation}
 \mathcal{L}_{SL} = ||\hat{I}_{t+1} - I_{t+1}||_2
\end{equation}

\begin{equation}
 \mathcal{L}_{CL} = - \sum_i y_i \log \hat{y}_i
\end{equation}

\begin{equation}
 \mathcal{L}_{M} = \max\{0, 1 - (d_{t}^g - d_{t+1}^g)\}
\end{equation}
Where, $d_{t}^g = || \hat{DI}_{t+1} - DI_{t} ||_2$ and $d_{t+1}^g = || \hat{DI}_{t+1} - DI_{t+1} ||_2$

\section{Experiments and Results}
\subsection{Datasets}

We perform our experiments using two popular datasets for action recognition UCF101 \cite{soomro2012ucf101} and JHMDB21 \cite{jhuang2013towards} which are briefly describe below.

\textbf{UCF-101} dataset consists of 13,320 videos (each contains a single action) of 101 action classes including a broad set of activities such as sports, playing musical instruments and human-object interaction, with an average length of 7.2 seconds. UCF-101 is one of the most challenging datasets due to its large diversity in terms of actions and to the presence of large variations in camera motion, cluttered background and illumination conditions. There are three standard training/test splits for this dataset. In our comparisons to the state-of-the-art for both action anticipation and recognition, we report the average accuracy over the three splits.

\textbf{JHMDB-21} is a subset of the HMDB51 dataset containing 928 videos and 21 action classes. Similar to UCF-101 dataset, each video contains one action starting from the beginning of the video.
\subsection{}
\begin{table}[ht!]
    \centering
  \begin{tabular}{lccc}
  \hline
  & \textbf{JHMDB21} & \textbf{UCF101} \\ \hline
Single DI & 45\% & \\ 
Multiple DI & 48\% &\\
Single RGB  & 39\%  & \\
Multiple RGB & 41\% & \\ 
MDI + MRGB  & 53\% &\\ 
\hline
  \end{tabular}
\end{table}

\subsection{Comparison}

The prediction of the action is evaluated using the accuracy over the percentage of interaction observed \cite{soomro2016online, aliakbarian2017encouraging}. Another standard practice to report the action anticipation is using the so-called earliest and latest prediction accuracies. However, there is no real agreement on the proportion of frames that the earliest setting corresponds to. Therefore we have to use for each dataset of the proportion that has been employed by the baselines (i.e., either 20\% or 50\%).

\begin{table}[ht!]
    \centering
  \begin{tabular}{lcc}
    \multicolumn{3}{c}{\textbf{JHMDB-21}}\\ \hline
    Method & Earliest & All video \\ \hline
    DP-SVM & 5\% & 46\% \\
S-SVM & 5\% & 43\% \\
Where/What & 10\% & 43\% \\
Ranking Loss & 29\% & 43\% \\
Context-Aware+Loss of & 28\% & 43\% \\
Context-Aware+Loss of & 33\% & 39\% \\
E-LSTM \cite{aliakbarian2017encouraging} & 55\% & 58\% \\ \hline
    Ours (1) & 42.9\% & 53.3\% \\ 
    Ours (2) & \% & 53\% \\ \hline
  \end{tabular}
\end{table}

\begin{table}[ht!]
    \centering
  \begin{tabular}{lcc}
    \multicolumn{3}{c}{\textbf{UCF-101}}\\ \hline
    Method & Earliest & All video \\ \hline
    Context-Aware+Loss of & 30.6\% & 71.1\% \\
    Context-Aware+Loss of & 22.6\% & 73.1\% \\
    E-LSTM \cite{aliakbarian2017encouraging} & 80.5\% & 83.4\% \\ \hline
    Ours (1) & - & - \% \\ 
    Ours (2) & - & - \% \\ \hline
  \end{tabular}
\end{table}

\section{Conclusions}
\endgroup

{\small
\bibliographystyle{ieee}
\bibliography{egbib}
}

\end{document}
