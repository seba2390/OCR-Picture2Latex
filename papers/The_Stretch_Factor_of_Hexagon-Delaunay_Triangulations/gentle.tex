%This section is devoted to the proof of the Gentle Path
%Lemma~\ref{le:boundedGentelSections}. 


\begingroup
\def\thetheorem{\ref{le:boundedGentelSections}}
\begin{lemma}[The Gentle Path Lemma] 
Let $T_{1n}$ be a linear sequence of triangles with respect to a line $st$ with
slope $m_{st}$ such that $0 < m_{st} < \frac{1}{\sqrt{3}}$. If $T_{1n}$ is
standard and contains a gentle path then the path can be extended
to a canonical gentle path in $T_{1n}$.
\end{lemma}
\addtocounter{theorem}{-1}
\endgroup

The main idea behind the proof of this lemma is that a gentle path between
$u_r \in U$ and $l_s \in L$ (where, say, $r \leq s$ and 
${\tt x}(u_r) < {\tt x}(u_s)$) in $T_{1n}$ can 
be extended using edge $(u_{r-1},u_r)$, unless $r=0$ or $u_r$ is a right
induction vertex of $T_{r}$, or using edge $(l_{s},l_{s+1})$, unless $s=n$ or
$l_s$ is a left induction vertex of $T_{s+1}$. In other words, a gentle path
can be extended unless it is canonical.

\begin{proof}%[Proof of Lemma~\ref{le:boundedGentelSections}]
%\label{lem:canonical}
We break the proof into several cases.

{\bf Case A.}
We first prove the claim in the case when $T_{1n}$ contains no irregular gentle
edge. In that case, because $T_{1n}$ is standard and by Lemma~\ref{lem:props},
the left vertex of $T_1$, whether it is $l_{0}$ or $u_{0}$ (or $l_0=u_0$),
lies on side $w$ of $H_1$ and the right vertex of $T_n$, whether it is $l_n$
or $u_n$ (or $u_n=l_n$), lies on side $e$ of $H_n$.

Let $T_{1n}$ contain a gentle path between $u_r \in U$ and
$l_s \in L$. W.l.o.g., we assume that $u_r$ occurs before $l_s$ 
(i.e., $r \leq s$ and ${\tt x}(u_r) < {\tt x}(l_r)$), 
as the case when $l_s$ occurs before $u_r$ can be argued
using a symmetric argument. 
Consider the sequence of points $u_{0}, u_1, \dots, u_r$ and
let $p=u_{r'}$ be the last point that is a right induction vertex (of $H_{r'}$)
or $p=u_{r'}=u_{0}$ if there are no right induction vertices in the sequence.
Note that in the second case, $p$ cannot be a base vertex of $T_1$ 
(since $T_{1n}$ is standard) and so $p$ must be the left vertex of
$T_1$.
%, if such a point
%exists; if such a point does not exist, we set $r'=i-1$ and $p$ to be the
%left vertex of $T_i$, which lies on the W side of $T_i$ and could be $u_{i-1}$
%or $l_{i-1}$. 
The edges of path $u_{r'}, \dots u_r$ are edges $(u_{i-1},u_i)$ for every $i$
such that $r' < i \leq r$ and $u_i$ is a right vertex of $H_i$. Because $T_{1n}$
contains no irregular edges and the fact that $u_i$ cannot lie on the $e$ side
of $H_i$ (because if it did $u_i$ would be a right induction vertex) or
the $s_e$ side of $H_i$ (by Lemma~\ref{lem:props}), the endpoints $u_{i-1}$ 
and $u_i$ lie on the $w$ or $n_w$
and $n_w$ or $n_e$ sides, respectively, of $H_i$. This implies that
${\tt x}(u_{i-1}) < {\tt x}(u_i)$ and that the length of $(u_{i-1},u_i)$ is
bounded by
$\sqrt{3}d_x(u_{i-1},u_i) - ({\tt y}(u_{i-1})-{\tt y}(u_i))$, as illustrated in 
Fig.~\ref{fig:caseA}. %In the case when $r'=i-1$,
%the length of the edge $(p,u_i)$
%can be bounded by $\sqrt{3}d_x(p,u_i)) - ({\tt y}(p)-{\tt y}(u_i))$. We use these
%inequalities to obtain that 
Therefore, ${\tt x}(p=u_{r'}) < {\tt x}(u_r)$ and the length of the path 
$p=u_{r'}, \dots, u_r$ (path $\mathcal{P}$)
in $T_{1n}$ is at most $\sqrt{3}d_x(p,u_{r})-({\tt y}(p)-{\tt y}(u_r))$. 

\begin{figure}[!b]
\center{\caseA}
\caption{Illustration of case A in the proof of Lemma~\ref{le:boundedGentelSections}.
Shown is the case when the gentle path from $u_r$ to $l_s$ is just a gentle
edge. For every $i$ such that $r' < i \leq r$ and $u_i$ is
the right vertex of $H_i$, hexagon $H_i$ and the edge $(u_{i-1},u_i)$ are shown
in red; the length of $(u_{i-1},u_i)$ is bounded by
$\sqrt{3}d_x(u_{i-1},u_i) - ({\tt y}(u_{i-1})-{\tt y}(u_i))$, a value equal
to the total length of the two intersecting, red, dashed segments going north
from $u_{i-1}$ and north-west from $u_i$. The total length of the two dashed
blue segments is an upper bound on the length of the edge $(u_r,l_s)$ and
the total length of the dotted red line segments represent the upper bound
$\sqrt{3}d_x(p,q) - ({\tt y}(p)-{\tt y}(q))$ on the length of the path 
$p=u_{r'}, \dots, u_r, l_{s}, l_{s+1}, \dots, l_{s'}=q$.}
\label{fig:caseA}
\end{figure}

Similarly, we consider the sequence of points $l_s, l_{s+1}, \dots, l_n$ and
set $q=l_{s'}$ to be the first point that is a left induction vertex (of $H_{s'+1}$)
or $q = l_{s'} = l_n$ if there are no left induction vertices in the sequence.
For every $i$ such that $s \leq i < s'$, if $l_i$ is a left vertex of $H_{i+1}$
then edge $(l_{i},l_{i+1})$ has endpoints $l_{i}$ and $l_{i+1}$ lying on the 
$s_w$ or $s_e$ and $s_e$ and $e$ sides, respectively, of $H_{i+1}$. This implies
that ${\tt x}(l_{i}) < {\tt x}(l_{i+1})$ and that the length of
$(l_{i},l_{i+1})$ is bounded by 
$\sqrt{3}d_x(l_{i},l_{i+1}) - ({\tt y}(l_{i})-{\tt y}(l_{i+1}))$. Therefore, 
${\tt x}(l_s) < {\tt x}(l_{s'}=q)$ and the length of the path 
$l_{s}, l_{s+1}, \dots, l_{s'}=q$ (path $\mathcal{Q}$) is at most
$\sqrt{3}d_x(l_{s},q)-({\tt y}(l_s)-{\tt y}(q))$, as shown in 
Fig.~\ref{fig:caseA}. 
We combine the paths $\mathcal{P}$ and $\mathcal{Q}$ with the gentle path from
$u_r$ to $l_s$ and this combined
path is gentle, and therefore canonical, since $r' < s'$,  
${\tt x}(p=u_{r'}) < {\tt x}(l_{s'}=q)$, and:
\begin{align*}
d_T (p, q) & \leq d_T(p, u_r) + d_T(u_r, l_s) + d_T(l_s, q) \\
& \leq \sqrt{3}d_x(p,u_r)-({\tt y}(p)-{\tt y}(u_r))+ \sqrt{3}d_x(u_r,l_s))-({\tt y}(u_r)-{\tt y}(l_s)) \\
 & \quad + \sqrt{3}d_x(l_s, q)) - ({\tt y}(l_s) - {\tt y}(q)) \\
& = \sqrt{3}d_x(p,q)-({\tt y}(p)-{\tt y}(q)).
\end{align*}
% since its length is
%\begin{align*} 
%d_T(p,q) & \leq  d_T(p,u_{r}) + d_T(u_{r},l_{s}) +  d_T(l_{s},q) \\
%& \leq \sqrt{3}d_x(p,u_{r})-({\tt y}(p)-{\tt y}(u_{r})) + \sqrt{3}d_x(u_{r},l_{s})-({\tt y}(u_{r})-{\tt y}(l_{s})) + \\
%& \quad \sqrt{3}d_x(l_{s},q)-({\tt y}(l_{s})-{\tt y}(q)) \\
%& =  \sqrt{3}d_x(p,q)-({\tt y}(p)-{\tt y}(q)).
%\end{align*}

{\bf Case B.} 
We now consider the case when $T_{1n}$ contains an irregular gentle edge. 
We assume
w.l.o.g. that the gentle edge has an endpoint $u_r$ that lies on side $s_w$ of
$H_{r+1}$ and also that the gentle edge is the first such edge (in the sense
that for all $i$ such that $0 \leq i < r$, $u_i$ does not lie on side $s_w$
of $H_{i+1}$). (For the case when $u_r$ lies on side $s_e$ of $H_r$ we would
consider the last such edge and use an argument that is symmetric to the one
we make below; the cases when the irregular gentle edge has an endpoint $l_r$
on side $n_w$ of $H_{r+1}$ or on side
$n_e$ of $H_r$ are symmetric to the cases when $u_r$ is on side $s_w$
of $H_{r+1}$ and side $s_e$ of $H_r$, respectively.)

By Lemma~\ref{lem:props}, the irregular gentle edge under consideration is 
$(u_r,l_{r+1})$, with $l_{r+1}$ lying on side $s_e$ or $e$ of $H_{r+1}$. Let $p$
and $q$ be as defined in Case A. Because of our assumption that $(u_r,l_{r+1})$
is first, the case A arguments can be applied to obtain 
${\tt x}(p) < {\tt x}(u_r)$ and bound the distance from $p$ to
$u_r$ by $\sqrt{3}d_x(p,u_r)-({\tt y}(p)-{\tt y}(u_r))$. We cannot do the same
to bound the distance from $l_{r+1}$ to $q = l_{s'}$ because it is possible that for
some $i$ such that $r+1 < i \leq s'$, $l_i$ lies on side $n_e$ of $H_i$ and
${\tt x}(l_{i-1}) > {\tt x}(l_i)$. So we proceed instead with induction and
prove that
\begin{equation*}
\label{eq:induction}
{\tt x}(u_r) < {\tt x}(l_i) \mbox{ and } d_T(u_r, l_i) \leq \sqrt{3}d_x(u_r, l_i) - ({\tt y}(u_r) - {\tt y}(l_i)) \mbox{ for every $i$ such that } r < i \leq s'
\end{equation*}
which will complete the proof.

The base case $i=r+1$ holds because $(u_r,l_{r+1})$ is gentle and 
${\tt x}(u_r) < {\tt x}(l_{r+1})$. For the induction step, we assume that 
${\tt x}(u_r) < {\tt x}(l_i)$ and 
$d_T(u_r, l_i) \leq \sqrt{3}d_x(u_r, l_i)-({\tt y}(u_r)-{\tt y}(l_i))$ 
for all $i$ such that $r < i < s \leq s'$ and show that the
inequality holds for $i = s$ as well. If $l_s = l_{s-1}$, that is trivially
true. Otherwise, $l_s$ is a right vertex of $H_s$. If
$l_s$ lies on the $e$ or $s_e$ side of $H_s$ then we use the arguments from
Case A to get that ${\tt x}(l_{s-1}) < {\tt x}(l_s)$, that the length of edge
$(l_{s-1}, l_s)$ is less than
$\sqrt{3}d_x(l_{s-1}, l_s) - ({\tt y}(l_{s-1}) - {\tt y}(l_s))$, 
and that therefore the inductive step again easily holds.

\begin{figure}
\center{\caseB}
\vspace{-1.2cm}
\center{(a) \hspace{6.35cm} (b)}

\caption{Illustration of {\em Case B.1} in the proof of 
Lemma~\ref{le:boundedGentelSections}. (a) $T_{(r+1)s}$. Point $u_r$ lies on side 
$s_w$ of $H_{r+1}$, point $l_s$ lies on side $n_e$ of $H_s$, and no
$(l_t, u_t)$, for $r < t < s$, has positive slope. (b) $\mathcal{R}(T_{(r+1)s})$
satisfies the conditions of Lemma~\ref{le:mainlemmaB} with $p = u_r$ and
$q = l_s$; because the slope of the line passing through $u_r$ and $l_s$ is less
than $-\frac{1}{\sqrt{3}}$, the length of the red, dashed segments (shown also in (a))
 is an upper bound on the length of a path in $T_{(r+1)s}$ from $u_r$ to $l_s$.}
\label{fig:caseB}
\end{figure}

If, however, $l_s$ lies on side $n_e$ of $H_s$ then we cannot use the Case A
argument because ${\tt x}(l_{s-1}) < {\tt x}(l_s)$ is not necessarily true.
In that case, by Lemma~\ref{lem:props} $(u_{s-1},l_s)$ must be an
(irregular) gentle edge with $u_{s-1}$ lying on the $w$ or $n_w$ side of $H_s$
(as illustrated in 
Fig.~\ref{fig:caseB}-(a)). First we note that ${\tt x}(u_r) < {\tt x}(l_s)$
holds because otherwise $l_s$ would appear below and to the right of $u_r$ and
then $(u_{s-1},l_s)$ would appear before $(u_r,l_{r+1})$
when traveling along segment $[st]$ from $s$ to $t$, a contradiction.
In order to bound $d_T(u_r,l_s)$, 
we consider indices $r+1,...,s-1$ and set $t$ to be
the last one such that edge $(l_t,u_t)$ has positive slope 
(i.e., ${\tt x}(l_t) < {\tt x}(u_t)$ and ${\tt y}(l_t) < {\tt y}(u_t))$, 
if one exists. We consider two subcases:



{\em Case B.1.} If no $(l_t, u_t)$, for $r < t < s$,
has positive slope  (as is the case in Fig.~\ref{fig:caseB}-(a)) we consider
the transformation of the linear sequence 
%$T'_{(r+1)s}$ of 
$T_{(r+1)s}$ obtained by rotating the plane clockwise by
an angle of $\pi/3$, % and then reflecting the plane with respect to the $x$-axis,
as illustrated in Fig.~\ref{fig:caseB}-(b). 

%Let $T'_{r+1}, \dots, T'_s$ be the
%corresponding transformations of triangles $T_{r+1}, \dots, T_s$
%and let $d'_x(u_r,l_s)$ be the difference between the abscissas of $l_s$ and
%$u_r$ in the transformed plane.

In the rotated $T_{(r+1)s}$, which we denote $\mathcal{R}(T_{(r+1)s})$, 
we have $-\sqrt{3} < m_{st} < -\frac{1}{\sqrt{3}}$, $u_r$ is a left induction 
vertex of $T_{r+1}$, and $l_s$ is a right induction vertex of $T_s$. We show
next that $\mathcal{R}(T_{(r+1)s})$ is regular. If,
in $\mathcal{R}(T_{(r+1)s})$, $u_t$ were to lie on a $s$ side of $H_t$ for some
$r < t < s$ then the slope of $(l_t,u_t)$ would have to be greater than
$m_{st}$ and less than the slope of the $s_e$ side of hexagon $H_t$. So,
the slope of $(l_t,u_t)$ in $\mathcal{R}(T_{(r+1)s})$ would have to be
between $-\sqrt{3}$ and $\frac{1}{\sqrt{3}}$ which would imply that
$(l_t,u_t)$ has positive slope in $T_{(r+1)s}$, a contradiction. Note
that if $u_{t-1}$ were to lie on a $s$ side of $H_t$ then so would $u_t$ so
we need not consider that case. We similarly show that no $l_{t-1}$ or $l_t$
lies on a $n$ side of $H_t$. Finally, we note that a gentle edge
$(l_t, u_t)$ in $\mathcal{R}(T_{(r+1)s})$ would have to have positive slope in
$T_{(r+1)s}$, a contradiction. So $\mathcal{R}(T_{(r+1)s})$ contains no gentle
edge and it is thus regular.

All the conditions of (Technical) Lemma~\ref{le:mainlemmaA} therefore apply to
$\mathcal{R}(T_{(r+1)s})$. By the lemma, the distance from $u_r$ to $l_s$
in $\mathcal{R}(T_{(r+1)s})$ (and thus in the original $T_{(r+1)s}$ as well) is
bounded by 
$\frac{4}{\sqrt{3}} d^{\mathcal{R}}_x(u_r,l_s)$, where $d^{\mathcal{R}}_x(u_r,l_s)$
is the difference
between the abscissas of $u_r$ and $l_s$ in $\mathcal{R}(T_{(r+1)s})$). Note that
$\frac{4}{\sqrt{3}} d^{\mathcal{R}}_x(u_r,l_s)$ is the sum of the lengths of
two sides of an equilateral triangle of height $d^{\mathcal{R}}_x(u_r,l_s)$.
Therefore, because
the slope of the line through $u_r$ and $l_s$ in $\mathcal{R}(T_{(r+1)s})$ is
less than $-\frac{1}{\sqrt{3}}$, 
$\frac{4}{\sqrt{3}} d^{\mathcal{R}}_x(u_r,l_s)$ is less than the
length of the piecewise linear curve, shown in Fig.~\ref{fig:caseB}-(b),
consisting of a (longer) vertical segment down from $u_r$ followed by a
(shorter) segment with slope $\frac{1}{\sqrt{3}}$ to $l_s$. The length of that
curve is exactly $\sqrt{3}d_x(u_r,l_s)-({\tt y}(u_r)-{\tt y}(l_s))$ in
$T_{(r+1)s}$ as illustrated in Fig.~\ref{fig:caseB}-(a), which completes the
proof in this case.


{\em Case B.2.} If $(l_t,u_t)$ exists, as illustrated in
Fig.~\ref{fig:othercase}, then ${\tt x}(l_t) < {\tt x}(u_t)$ and 
$d_T(l_t,u_t) \leq \sqrt{3}d_x(l_t,u_t)-({\tt y}(l_t)-{\tt y}(u_t))$. Also,
by induction, ${\tt x}(u_r) \leq {\tt x}(l_t)$ and 
$d_T(u_r,l_t) \leq \sqrt{3}d_x(u_r,l_t)-({\tt y}(u_r)-{\tt y}(l_t))$.
What remains to be done is to show that ${\tt x}(u_t) < {\tt x}(l_s)$ and 
that $d_T(u_t,l_s) \leq \sqrt{3}d_x(u_t,l_s)-({\tt y}(u_t)-{\tt y}(l_s))$. 
We now have two cases to consider. Note first that, because $t$ is last, for 
every $j$ such that $t < j < s$, $u_j$ cannot lie on side $e$ or $s_e$ of $H_j$
(because otherwise $(l_j,u_j)$ would have to have positive slope); this means
that $u_j$ cannot be a right induction point of $H_j$.



\begin{figure}
\center{\othercase}

\caption{Illustration of {\em Case B.2} in the proof of 
Lemma~\ref{le:boundedGentelSections}. If $(l_t,u_t)$ has positive slope then
${\tt x}(l_t) < {\tt x}(u_t)$ and 
$d_T(l_t,u_t) \leq \sqrt{3}d_x(l_t,u_t)-({\tt y}(l_t)-{\tt y}(u_t))$.
By induction, ${\tt x}(u_r) \leq {\tt x}(l_t)$ and 
$d_T(u_r,l_t) \leq \sqrt{3}d_x(u_r,l_t)-({\tt y}(u_r)-{\tt y}(l_t))$. In
{\em Case B.2.i}, no $u_{j-1}$ lies on side $s_w$ of $H_j$, where $t<j<s$, and so
${\tt x}(u_t) \leq \dots \leq {\tt x}(u_{s-1}) \leq {\tt x}(l_s)$ and 
$d_T(u_u,l_s) \leq \sqrt{3}d_x(u_t,l_s)-({\tt y}(u_t)-{\tt y}(l_s))$.}
\label{fig:othercase}
\end{figure}





{\em Case B.2.i.} If we also have that no $u_{j-1}$ lies on side $s_w$ of $H_j$,
where $t<j<s$ and as illustrated in Fig.~\ref{fig:othercase}, 
then, using the arguments from Case A, we can show that
${\tt x}(u_t) \leq \dots \leq {\tt x}(u_{s-1}) \leq {\tt x}(l_s)$ and 
can bound the length of the path $u_t, \dots, u_{s-1}, l_s$ with 
$\sqrt{3}d_x(u_t,l_s)-({\tt y}(u_t)-{\tt y}(l_s))$. 

%Putting
%everything together, we get ${\tt x}(u_r) \leq {\tt x}(l_s)$ and:
%\begin{align*}
%d_T (u_r, l_s) & \leq d_T(u_r, l_t) + d_T(l_t, u_t) + d_T(u_t, l_s) \\
%& \leq \sqrt{3}d_x(u_r,l_t)-({\tt y}(u_r)-{\tt y}(l_t))+ \sqrt{3}d_x(l_t,u_t))-({\tt y}(l_t)-{\tt y}(u_t)) \\
% & \quad + \sqrt{3}d_x(u_t, l_s)) - ({\tt y}(u_t) - {\tt y}(l_s)) \\
%& \leq \sqrt{3}d_x(u_r,l_s)-({\tt y}(u_r)-{\tt y}(l_s)).
%\end{align*}

{\em Case B.2.ii.} Finally, suppose that $u_{t'-1}$, for some $t'$ such that
$t < t' < s$, lies on side $s_w$ of $H_{t'}$ and let us assume that $t'$ is
leftmost (in the sense that no $u_{j-1}$ lies on side $s_w$ of $H_j$ for $j$
such that $t < j < t'$). Using arguments from Case A we obtain that 
${\tt x}(u_t) < {\tt x}(u_{t'})$ and 
$d_T(u_t, u_{t'}) \leq \sqrt{3}d_x(u_t, u_{t'})) - ({\tt y}(u_t) -
{\tt y}(u_{t'} ))$. Using the approach from {\em Case B.1} we get
${\tt x}(u_{t'}) < {\tt x}(l_s)$ and 
$d_T(u_{t'}, l_s) \leq \sqrt{3}d_x(u_{t'}, l_s) - ({\tt y}(u_{t'})-{\tt y}(l_s))$.
We complete the proof of the lemma by combining all these inequalities.
\end{proof}

