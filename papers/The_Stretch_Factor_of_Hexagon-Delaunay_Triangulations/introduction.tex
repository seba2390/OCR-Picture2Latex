In this paper we consider the problem of computing a tight bound
on the worst case stretch factor of a Delaunay triangulation. 

Given a set $P$ of points on the plane, the Delaunay triangulation $T$ on $P$
is a plane graph such that for every pair $u,v \in P$, $(u,v)$ is an edge of
$T$ if and only if there is a {\em circle} passing through $u$ and $v$ with no
point of $P$ in its interior\footnote{This definition assumes that points in
$P$ are in general position which we discuss in Section~\ref{sec:prelim}.}. 
In this paper, we refer to Delaunay triangulations defined using the
{\em circle} as
$\ocircle$-Delaunay triangulations. The $\ocircle$-Delaunay triangulation 
$T$ of $P$ is a plane subgraph of the complete, weighted Euclidean graph
$\cE^P$ on $P$ in which the weight of an edge is the Euclidean distance between
its endpoints. Graph $T$ is also a {\em spanner}, defined as a subgraph of
$\cE^P$ with
the property that the distance in the subgraph between any pair of points is
no more than a constant multiplicative ratio of the distance in $\cE^P$ between
the points. The constant ratio is referred to as the {\em stretch factor} (or
{\em spanning ratio}) of the spanner.

The problem of computing a tight bound on the worst case stretch factor of the
$\ocircle$-Delaunay triangulation has been open for more than three decades. 
In the 1980s, when $\ocircle$-Delaunay triangulations were not known to be
spanners, Chew considered related, ``easier'' structures. In
1986~\cite{Chew86}, Chew proved that a $\square$-Delaunay
triangulation---defined using a {\em fixed-orientation square}
%\footnote{\label{note1}Defined more
%precisely in Section~\ref{sec:prelim}}
instead of a circle---is a 
spanner with stretch factor at most $\sqrt{10}$. Following this, Chew
proved that the $\triangle$-Delaunay
triangulation---defined using a
{\em fixed-orientation equilateral triangle}
--- has a stretch factor of~$2$~\cite{Chew89}. Significantly, this bound is
tight: one can construct $\triangle$-Delaunay
triangulations with stretch factor arbitrarily close to 2.
Finally, Dobkin et al.~\cite{DFS90} showed that the $\ocircle$-Delaunay
triangulation is a spanner as well. The bound on the stretch factor they obtained was
subsequently improved by Keil and Gutwin~\cite{KG92} as shown in
Table~\ref{ta:related}.
The bound by Keil and Gutwin stood unchallenged
for many years until Xia recently improved the bound to below
2~\cite{Xia13}.

\begin{table}[!b]
\caption{Key stretch factor upper bounds (tight bounds are bold).}
\centering
\begin{tabular}{llr}
{\bf Paper} &  {\bf Graph} & {\bf Upper Bound} \\ \hline
\cite{DFS90} & $\ocircle$-Delaunay & $\pi(1+\sqrt{5})/2 \approx 5.08$ \\ \hline
\cite{KG92} & $\ocircle$-Delaunay & $4\pi/(3\sqrt{3}) \approx 2.41$ \\ \hline
\cite{Xia13} & $\ocircle$-Delaunay & $1.998$ \\ \hline \hline
\cite{Chew89} & $\triangle$-Delaunay & $\mathbf{2}$ \\ \hline \hline
\cite{Chew86} & $\square$-Delaunay & $\sqrt{10} \approx 3.16$ \\ \hline
\cite{BGHP15} & ${\square}$-Delaunay \hspace{3ex} &
$\mathbf{\sqrt{4+2\sqrt{2}}\approx 2.61}$ \\ \hline \hline
{\bf [This paper]} & {\Large\varhexagon}-Delaunay \hspace{3ex} &
$\mathbf{2}$\\ \hline
\end{tabular}
\label{ta:related}
\end{table}

On the lower bound side, some progress has been made on bounding the worst
case stretch factor of a $\ocircle$-Delaunay triangulation. The trivial lower
bound of $\pi/2 \approx 1.5707$ has been improved to
$1.5846$~\cite{BDLSV11} and then to $1.5932$~\cite{XZ11}. 

After three decades of research, we know that the worst case
stretch factor of $\ocircle$-Delaunay triangulations is somewhere between
$1.5932$ and $1.998$. Unfortunately, the techniques that have been
developed so far seem inadequate for proving a tight stretch factor bound.

Rather than attempting to improve further the bounds on
the stretch factor of $\ocircle$-Delaunay triangulations, we follow an
alternative approach. Just like Chew turned to $\triangle$- and
$\square$-Delaunay triangulations to develop insights useful for showing
that $\ocircle$-Delaunay triangulations are spanners, we make use of Delaunay
triangulations defined using regular polygons to develop techniques for
computing tight stretch factor bounds. Delaunay triangulations based on
regular polygons are known to be spanners (Bose et al.~\cite{BCCS08}).
Tight bounds are known for $\triangle$-Delaunay triangulations~\cite{Chew89}
and also for $\square$-Delaunay triangulations (Bonichon et al.~\cite{BGHP15})
as shown in Table~\ref{ta:related}. %This paper is part of a project that builds
%on these results to extend them to Delaunay triangulations defined using other
%regular polygons. The motivation is that the work results in the development
%of techniques that may be useful for proving a tight bound on the stretch factor
%of $\ocircle$-Delaunay triangulations.

In this paper, we show that the worst case stretch factor of
{\Large\varhexagon}-Delaunay triangulations is 2. We present an overview of
our proof in Section~\ref{sec:main}. 
\iftoggle{abstract} 
{The overview makes use of three lemmas 
whose detailed proofs are ommitted; the proofs (briefly discussed in 
Sections~\ref{sec:gentle}, \ref{sec:proofA}, and  \ref{sec:proofB}) are
in the full version of the paper (Appendix).}
{The overview makes use of three
lemmas whose detailed proofs are in Sections~\ref{sec:gentle}, \ref{sec:proofA}, and  \ref{sec:proofB}.} 
We think that our main
contribution consists of two techniques that we use to compute tight upper 
bounds  on the stretch factor of particular types of
{\Large\varhexagon}-Delaunay triangulations. In Section~\ref{sec:conclusion}
we review the role of the techniques in the paper and explore their potential
to be applied to other kinds of Delaunay triangulations.

