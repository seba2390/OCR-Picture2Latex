In this section we state our main result and provide an overview of our proof.
\iftoggle{abstract}
{We start with a technical lemma that is used to prove the two
key lemmas needed for the main result.}
{We start by introducing a technical lemma that will be used to prove the two
key lemmas needed for the main result.}

\subsection{Technical lemma}
\label{sub:keylemma}

Let $T$ be the {\Large\varhexagon}-Delaunay triangulation on a set of
points $P$ in the plane. 
\begin{definition}
Let $T_1, T_2, \dots, T_n$ be a sequence of triangles of $T$ that a line $st$
of finite slope intersects. This sequence of triangles is said to be
{\em linear} w.r.t. line $st$ if for every $i=1,\dots,n-1$:
\begin{itemize}
\item triangles $T_i$ and $T_{i+1}$ share an edge, and %which we refer to as a
%{\em cross edge}, and
\item line $st$ intersects the interior of that shared edge (not an endpoint).
\end{itemize}
\end{definition}

Our goal is to prove an upper bound on the length of the shortest path from the 
``leftmost'' point of $T_1$ to the ``rightmost'' point of $T_n$, when certain conditions hold. We 
introduce some notation and definitions, illustrated in 
Fig.~\ref{fig:regular}, to make this more precise.

%Thanks to the hexagon's rotational and reflective symmetries, we can rotate
%the plane around the origin and/or reflect the plane with respect to
%the $x$-axis to ensure that line $st$ has slope in the range 
%$[-\sqrt{3}, \sqrt{3})$
%and we assume that in order to define our notation unambiguously. 
%Thanks to the hexagon's rotational symmetries, we can rotate
%the plane around the origin to ensure that line $st$ has finite slope
%and we assume that in order to define our notation unambiguously. 
We consider the $n-1$ shared triangle edges intersected by line $st$ from left
to right (where left and right are defined with respect to $x$-coordinates)
and label the endpoints of the $i$-th edge $u_i$ and $l_i$, with
$u_i$ being above line $st$ and $l_i$ below. We note that points typically get
multiple labels and identify a point with its label(s). 
%Recall that each endpoint is a point in $P$ as well as
%the vertex of two or more triangles in the linear sequence, and that it may
%receive multiple labels (e.g., Fig ???). 
If line $st$ goes through the vertex of $T_1$ other than $u_1$ and $l_1$,
we assign that vertex both labels $u_0$ and $l_0$ (as shown in
Fig.~\ref{fig:definitions}); otherwise, we assign labels
$u_0$ and $l_0$ to the endpoints of the edge of $T_1$ intersected by line $st$
other than $(u_1,l_1)$, with $u_0$ being above line $st$ (as shown in
Fig.~\ref{fig:regular}). Similarly, if line 
$st$ goes through the vertex of $T_n$ other than $u_{n-1}$ and $l_{n-1}$, 
we assign it both labels $u_n$ and $l_n$ (as shown in
Fig.~\ref{fig:definitions}); otherwise, we assign labels 
$u_n$ and $l_n$ to the endpoints of the other edge of $T_n$ intersected by
line $st$, with $u_n$ being above line $st$ (as shown in
Fig.~\ref{fig:regular}). Note that for $1 \leq i \leq n$:
\begin{itemize}
\item either $T_i = \triangle(u_i, l_i, l_{i-1})$, in which case we call
$l_{i-1}$ and $l_i$ the {\em left} and {\em right} vertices of $T_i$ 
%(e.g., in
%Fig.~\ref{fig:definitions} $u_6$ is the base and
%$l_5$ and $l_6$ are the left and right vertices of $T_6$), 
\item or $T_i = \triangle(u_{i-1}, u_i, l_i)$, in which case we call $u_{i-1}$
and $u_i$ the {\em left} and {\em right} vertices of $T_i$.
\end{itemize}
Note that only one of the above holds, except for $T_1$ if $u_0=l_0$
(in which case both hold) or for $T_n$ if $u_n=l_n$ (in which case again both
hold). For every $i = 1, \dots, n$,
when $u_i = u_{i-1}$ or $l_i = l_{i-1}$ we call the corresponding vertex of $T_i$
the {\em base} vertex of $T_i$. Note that $T_1$ has no base vertex if $u_0=l_0$
and $T_n$ has no base vertex if $u_n=l_n$ (as is the case in 
Fig.~\ref{fig:definitions}). Let $U$ and $L$ be the sets of all point labels
$u_i$ and $l_i$, respectively, and let $T_{1n}$ be the union of
$T_1,T_2,\dotsc,T_{n}$ which we will refer to as a linear sequence of triangles
as well.

\iftoggle{abstract}
{\begin{figure}[!b]}
{\begin{figure}[!b]}
\center{\regular}
\caption{The dotted line ($st$) intersects the linear sequence of triangles
$T_1, T_2, \dots, T_5$. The vertices of each triangle $T_i$
($u_{i-1}$, $u_i$, $l_{i-1}$, $l_i$, two of which are equal) lie on the boundary
of the hexagon $H_i$. Note that $l_2$ is the left, $l_3$ is the right,
and $u_2=u_3$ is the base vertex of $T_3$, for example. The linear sequence is
regular since $T_1$ has a left induction vertex, $T_5$ has a right induction
vertex, and, for $i = 1, \dots, 4$, no $u_i$ is a $s$ vertex of $T_i$ or
$T_{i+1}$, no $l_i$ is a $n$ vertex of $T_i$ or $T_{i+1}$, and no $(u_i,l_i)$
is gentle.%  Edge $(u_4,l_5)$ is gentle since its slope is within 
%$[-\frac{1}{\sqrt{3}}, \frac{1}{\sqrt{3}}]$.
}
\label{fig:regular}
\end{figure}


Let $H_i$, for $1 \leq i \leq n$, be the (empty) hexagon passing through the
vertices of $T_i$; note that $H_i$ is unique due to the general position
assumption. A vertex of $T_i$ is said to be a $w$, $e$, $n$,
or $s$ vertex of $T_i$ if it lies on the $w$ side, $e$ side, one of
the $n$ sides, or one of the $s$ sides, respectively, of $H_i$  
(see Fig.~\ref{fig:regular}).
A left vertex of $T_i$ that is a $w$ vertex of
$T_i$ is referred to as a {\em left} {\em induction vertex}
of $T_i$; similarly, a right vertex that is a $e$ vertex is referred to as a {\em right} {\em induction vertex}
of $T_i$. 

Note that a base vertex cannot be an induction vertex.

%A cross edge $(u_i,l_i)$ is
%{\em forbidden} if $u_i$ lies on the south ($s_w$ and $s_e$) sides of $H_i$ or
%$l_i$ lies and
%$l_{i-1}$ and $l_i$ do not lie on the north ($n_w$ and $n_e$) sides of $H_i$.

%{\bf ?????????}



\begin{definition}
We call an edge $(u_i,l_j)$ {\em gentle} if its slope is between
$-\frac{1}{\sqrt{3}}$ and $\frac{1}{\sqrt{3}}$.
\end{definition}
In Fig.~\ref{fig:regular} no edge $(u_i,l_j)$ is gentle while in 
Fig.~\ref{fig:definitions} $(u_0,l_1)$ and $(u_8,l_8)$ are gentle.

\begin{definition}
%The left (resp., right) vertex of $T_i$ is a {\em startpoint}
%(resp. {\em endpoint}) of $T_i$ if it lies on the $w$ (resp., $e$) side of
%$H_i$. 
%{\bf A linear sequence $T_1, \dots, T_n$ is {\em standard} if $T_1$ has a 
%startpoint and $T_n$ has an endpoint.
The linear sequence of triangles $T_{1n}$ is {\em regular} if $T_1$ has a left
induction vertex, $T_n$ has a right induction vertex, and if, for
every $i = 1, \dots, n-1$:
\begin{itemize}
\item $u_i$ is {\em not} a $s$ vertex of $T_i$ and $T_{i+1}$,
\item $l_i$ is {\em not} a $n$ vertex of $T_i$ and $T_{i+1}$, and 
\item $(u_i,l_i)$ is not gentle.
\end{itemize} 
\end{definition}
The linear sequence in Fig.~\ref{fig:regular} is regular while the one in
Fig.~\ref{fig:definitions} is not (because $u_8$ lies on the $s_w$ side of
$H_9$---the red hexagon passing through the vertices of
$T_9 = \triangle(u_8, u_9, l_9)$---and also because edge $(u_8,l_8)$ is gentle). 

\iftoggle{abstract}
{The proof of the following technical lemma is discussed in Section~\ref{sec:proofA}.}
{The following technical lemma is proven in Section~\ref{sec:proofA}.}
\begin{lemma}[The Technical Lemma]
\label{le:mainlemmaA}
If $T_{1n}$ is a regular linear sequence of triangles then there is a path in
$T_{1n}$ from
the left induction vertex $p$ of $T_1$ to the right induction vertex $q$ of
$T_n$ of length at most  $\frac{4}{\sqrt{3}} d_x(p,q)$.
%Let $0 \leq m \leq \sqrt{3}$ and $0 \leq i \leq j \leq n$. 
%If $T_{ij}$ contains no gentle edge, $T_i$  has a vertex $p$ on side $w$
%of $H_i$, and $T_j$ has vertex $q$ on side $e$ of $H_j$ then there
%exists a path in $T_{ij}$ from $p$ to $q$ of length at most
%$\frac{4}{\sqrt{3}} d_x(p,q)$.
\end{lemma}

Actually, what we show in Section~\ref{sec:proofA} implies something stronger: 
If $T_{1n}$ is regular then the lengths of the {\em upper} path 
$p,u_0,\dots,u_n,q$ and of the {\em lower} path $p,l_0,\dots,l_n,q$ 
add up to at most $\frac{8}{\sqrt{3}} d_x(p,q)$. 
It is useful to informally describe now the techniques we use to do this. For
that purpose we introduce, for a point $o$ on a side of $H_i$, functions
$p_N(o,i)$ and $p_S(o,i)$ as the {\em signed} shortest distances around the
perimeter of $H_i$ from $o$ to the $N$ vertex and $S$ vertex, respectively;
the sign is positive if $o$ lies on sides
$n_w$, $w$, or $s_w$ of $H_i$ and negative otherwise 
(see Fig.~\ref{fig:discretedNdS}-(a) and Fig.~\ref{fig:discretedNdS}-(b)).

\iftoggle{abstract}
{\begin{figure}[!b]}
{\begin{figure}[!b]}
\discretedNdS
\begin{center}
\vspace{-8pt}\hspace{-1.3cm} (a) \hspace{4.35cm} (b) \hspace{4.55cm} (c)
\end{center}
\caption{(a) The values of $p_N(o,i)$ are illustrated, for various points $o$
lying on the boundary of $H_i$, as signed hexagon arc lengths. (b) The 
values of $p_S(o,i)$ are illustrated similarly. (c) The length of edge
$(u_{i-1},u_i)$ is bounded by $p_N(u_{i-1}, i) - p_N(u_i, i)$.}
\label{fig:discretedNdS}
\end{figure}
%We also introduce {\em discrete} functions $U(i)$ and $L(i)$ that are defined
%as follows for
%$i = 1, \dots, n+1$:
%\[U(i) = d_{T_{1i}}(p,u_i) + p_N(i,u_i) \hspace{2cm} L(i) = d_{T_{1i}}(p,l_i) + p_S(i,l_i)\]
%(where $T_{n+1}$ is a triangle of radius $0$ centered at point
%$u_{n+1}=l_{n+1}=q$).

%Finally, we introduce the {\em discrete} function $P(i)$ to be $U(i) + L(i)$
%and note that $P(n+1)$ is exactly twice the distance in $T_{1n}$ from $p$ to
%$q$. 

Note that the length of each edge 
$(u_{i-1},u_i)$ (assuming $u_{i-1} \not= u_i$) can be
bounded by the distance
from $u_{i-1}$ to $u_i$ when traveling clockwise along the sides of $H_i$.
This distance is exactly $p_N(u_{i-1}, i) - p_N(u_i, i)$ as
illustrated in Fig.~\ref{fig:discretedNdS}-(c).
This motivates the following {\em discrete} function, defined for
$i=0,1,\dots,n$ and, for convenience's sake, 1) assuming that $p=u_0$ and
$q=u_n$ and 2) using an additional hexagon $H_{n+1}$ of radius $0$ centered
at point $q$:
\[\bar{U}(i) = \sum_{j=1}^{i} (p_N(u_{j-1}, j) - p_N(u_{j}, j)) + p_N(u_{i},i+1).\]

Function $\bar{U}(i)$ can be used to bound the length of upper path fragments;
in particular, $\bar{U}(n)$ bounds the length of the upper path from 
$p$ to $q$. A function $\bar{L}(i)$ bounding the length of the lower
path can be defined similarly. In Section~\ref{sec:proofA}, we will compute
an upper bound for $\bar{U}+\bar{L}$ by
1) switching the analysis from a discrete one to a continuous one, with
functions $p_N$ and $p_S$ defined not in terms of index $i$ but
in terms of coordinate $x$ for every $x$ between ${\tt x}(p)$ and ${\tt x}(q)$
and 2) analyzing the growth rates, with respect to $x$, of the continuous
functions $p_N$, $p_S$, and $\bar{U}+\bar{L}$. We
will show that (the continuous versions of) $p_N$ and $p_S$ are piecewise linear functions with growth rates
$\frac{2}{\sqrt{3}}$, $\frac{4}{\sqrt{3}}$, or $\frac{6}{\sqrt{3}}$, and that
$\bar{U}+\bar{L}$ is also piecewise linear with growth rate equal to the growth
rate of $p_N+p_S$ which can be $\frac{4}{\sqrt{3}}$, $\frac{6}{\sqrt{3}}$, or
$\frac{8}{\sqrt{3}}$. 
Lemma~\ref{le:mainlemmaA} will follow from the last (largest) growth rate.

%With the technical lemma in hand, we can now state the main
%result of this paper and provide a high level overview of its proof. 

With the technical lemma in hand, we can now state the first of the two
key lemmas that we need to prove our main result.





\subsection{The Amortization Lemma}

The first of our two key lemmas is a strengthening of the (Technical) Lemma~\ref{le:mainlemmaA}
under two restrictions. The first restriction is that $T_{1n}$ is %a linear
%sequence 
defined with respect to a line $st$ whose slope $m_{st}$ is restricted to
$0 < m_{st} < \frac{1}{\sqrt{3}}$. With that restriction we get the following
properties:
%we refer to the edge as a {\em gentle edge}. %Note that
%an edge is gentle if and only if the slope of the line going through its
%endpoints is between $-\frac{1}{\sqrt{3}}$ and  $\frac{1}{\sqrt{3}}$. 

%The following lemma is proven in Section~\ref{sec:proofA}.
%\begin{lemma}
%\label{le:mainlemmaA}
%If $T_{1n}$ is regular and contains no
%gentle edge then there is a path in $T_{1n}$ from
%the left induction vertex $p$ of $T_1$ to the right induction vertex $q$ of
%$T_n$ of length at most  $\frac{4}{\sqrt{3}} d_x(p,q)$.
%Let $0 \leq m \leq \sqrt{3}$ and $0 \leq i \leq j \leq n$. 
%If $T_{ij}$ contains no gentle edge, $T_i$  has a vertex $p$ on side $w$
%of $H_i$, and $T_j$ has vertex $q$ on side $e$ of $H_j$ then there
%exists a path in $T_{ij}$ from $p$ to $q$ of length at most
%$\frac{4}{\sqrt{3}} d_x(p,q)$.
%\end{lemma}

\begin{lemma}
\label{lem:props}
Let $T_{1n}$ be a linear sequence with respect to line $st$ with slope $m_{st}$
such that $0 < m_{st} < \frac{1}{\sqrt{3}}$. For every $i$ s.t. 
$1 \leq i \leq n$:
\begin{itemize}
\item If $u_{i-1}$ lies on side $s_w$ of $H_{i}$ or $l_{i}$ lies on 
side $n_e$ of $H_{i}$ then $(u_{i-1},l_{i})$ is gentle.
\item If $l_{i-1}$ lies on side $n_w$ of $H_{i}$ or $u_{i}$ lies on the $s_e$
side of $H_{i}$ then $(l_{i-1},u_{i})$ is gentle.
\item None of the following can occur: $u_{i-1}$ lies on side $s_e$ of $H_{i}$,
$l_{i}$ lies on side $n_w$ of $H_{i}$, $l_{i-1}$ lies on side $n_e$ of $H_{i}$,
and $u_{i}$ lies on the $s_w$ side of $H_{i}$. 
\end{itemize}
\end{lemma}
\noindent Note, for example, that $u_8$ lies on side $s_w$ of hexagon
$H_9$ in Fig.~\ref{fig:definitions} and that edge $(u_8,l_9)$ is gentle. 
A gentle edge that satisfies one of the four cases in the first two bullet points of
Lemma~\ref{lem:props} will
be called {\em irregular}. 


%Note that, since the line with slope $m$ must
%intesect the
%interior of $H_1$, $s$ can only lie on the $n_w$, $w$, or $s_w$ sides of $H_1$;
%by Lemma~\ref{lem:props}, if $s$ lies on the $n_w$ side of $H_1$ then
%$(s,u_1) = (l_0,u_1)$ is gentle, and
%if $s$ lies on the $s_w$ side of $H_1$ then $(s,l_1) = (u_0,l_1)$ is gentle. 
%Similarly, $t$ can only lie on the $n_e$, $e$, or $s_e$ sides of $H_n$; if
%$t$ lies on the $s_w$ side of $H_n$ then $(t,l_{n-1}) = (u_n,l_{n-1})$ is 
%gentle, and if $t$ lies on the $n_w$ side of $H_n$ then 
%$(t,u_{n-1}) = (l_n,u_{n-1})$ is gentle. 
\begin{proof}
If $u_{i-1}$ lies on side $s_w$ of some hexagon $H_{i}$ then, since
$0 < m_{st} < \frac{1}{\sqrt{3}}$ and by general position assumptions, 
either $u_{i-1}=u_{i}$ and $l_{i-1}$ and $l_{i}$
must lie on sides $s_e$ and $e$ of $H_{i}$, respectively, or $l_{i-1}=l_{i}$ must
lie on side $s_e$ or $e$ of $H_{i}$. Either way, the slope of
the line going through $u_{i-1}$ and $l_{i}$ must be between 
$-\frac{1}{\sqrt{3}}$ and  $\frac{1}{\sqrt{3}}$. Similar arguments can be used
to handle the remaining three cases in the first two bullet points.

Let the left and right intersection points of line $st$ with hexagon $H_{i}$
be  $h_{i-1}$ and $h_{i}$. Note that when traveling clockwise along the sides of
$H_{i}$ the points will be visited in this order: 
$h_{i-1},u_{i-1},u_{i},h_{i}, l_{i}, l_{i-1}$.
If $u_{i-1}$ lies on side $s_e$ of $H_{i}$ then $i>1$ and, because
$0 < m_{st} < \frac{1}{\sqrt{3}}$, either $u_{i}$ (if $u_{i-1} \not= u_{i}$)
or $l_{i}$ (if $l_{i-1} \not= l_{i}$) would have to lie on side $s_e$
of $H_{i}$ as well, which violates our general position assumption for the
set of  points P. The remaining three cases are handled similarly.
\end{proof}



%A left (resp., right) vertex of $T_i$ lying on side $w$ (resp. $e$) of $H_i$
%will be referred to as the {\em left (resp., right) induction point} of $T_i$
%or $H_i$. In Fig.~\ref{fig:definitions}, for example, $u_0=l_0$, $l_2$, and
%$u_9$ are left induction points of $T_1$, $T_3$, and $T_{10}$, respectively.
%We will also say that $s$ is a left induction point of $T_1$
%and $t$ is a right induction point of $T_n$ regardless of the side of $H_1$
%and $H_n$ they lie on. 
%Note that, by Lemma~\ref{lem:props}, if $s$ lies on side $n_w$ or $s_w$ of
%$H_1$ then $(s=l_0,u_1)$ or $(s=u_0,l_1)$, respectively, is a short gentle edge;
%similarly,
%if $t$ lies on side $n_e$ or $s_e$ of $H_n$ then $(t=l_n,u_{n-1})$ or
%$(t=u_n,l_{n-1})$, respectively, is a short gentle edge.

By the above lemma, under the restriction $0 < m_{st} < \frac{1}{\sqrt{3}}$,
if $T_{1n}$ has no gentle edge then it is regular and (Technical) 
Lemma~\ref{le:mainlemmaA} applies. A narrower but much stronger
version of (Technical) Lemma~\ref{le:mainlemmaA} applies as well if another
restriction is made. To state the second restriction we need
some additional terminology. 

Let $l_i \in L$ and $u_j \in U$. If $i \leq j$ and ${\tt x}(l_i) < {\tt x}(u_j)$
then we say that $l_i$ {\em occurs before} $u_j$, and if $j \leq i$ and
${\tt x}(u_j) < {\tt x}(l_i)$ then we say that $u_j$ {\em occurs before}
$l_i$.
\begin{definition}
Given points $l_i \in L$ and $u_j \in U$ such that one occurs before the other,
a path between them is \emph{gentle} if the length of the path is not greater
than $\sqrt{3}d_x(u_j,l_i)-({\tt y}(u_j)-{\tt y}(l_i))$.
\end{definition}
%A gentle path is a low cost path that, under the right conditions, will
%be used as a subpath of a path from $s$ to $t$ that satisfies
%Equation~(\ref{eq:main}).
See Fig.~\ref{fig:definitions} for an illustration of a gentle path. Note that
a gentle edge is a gentle path (e.g., $(u_0,l_1)$ and 
$(u_8,l_8=l_9)$ in Fig.~\ref{fig:definitions}).

The following is the key to our proof of the main result of this paper:
\begin{lemma}[The Amortization Lemma]
\label{le:mainlemmaB}
Let $T_{1n}$ be a regular linear sequence with respect to line $st$ with slope
$m_{st}$. If $0 < m_{st} < \frac{1}{\sqrt{3}}$ and if $T_{1n}$ contains no
gentle path then there is a path in $T_{1n}$ from 
the left induction vertex $p$ of $T_1$ to the right induction vertex $q$ of
$T_n$ of length at most  $(\cT) d_x(p,q)$.
\end{lemma}
\iftoggle{abstract}{We will discuss the proof of the Amortization Lemma in 
Section~\ref{sec:proofB}; the proof builds 
on the analysis done in Section~\ref{sec:proofA} to prove (Technical) 
Lemma~\ref{le:mainlemmaA}.}{We will prove the Amortization Lemma in Section~\ref{sec:proofB} by building 
on the analysis done in Section~\ref{sec:proofA} to prove (Technical) 
Lemma~\ref{le:mainlemmaA}.} 
Instead of using function $\bar{U}$, however, we consider the discrete 
function
\begin{equation*}
U(i) = d_{T_{1i}}(p,u_i) + p_N(u_{i},i+1)
\end{equation*}
defined for $i=0,1,\dots,n$ and, for convenience's sake, 1) assuming that 
$p=u_0$ and $q=u_n$ and 2) using additional hexagon $H_{n+1}$ of radius $0$
centered at point $q$. An equivalent discrete function $L(i)$ using points 
$l_i$ instead of $u_i$ can be defined. Note that $U(n) + L(n)$ is exactly twice
the distance in
$T_{1n}$ from $p$ to $q$. To bound $U + L$, we will switch the analysis to
a continuous one just as we did for ${\bar U} + {\bar L}$. We will see that,
except for a finite number of discontinuities, the continuous version of $U+L$
has the same growth rate as ${\bar U} + {\bar L}$, which is the growth
rate of $p_N+p_S$. 
We will consider the intervals when the growth rate of (the continuous version of)
$U+L$ is higher than $2(\cT)$ (i.e., when its growth rate is $\frac{8}{\sqrt{3}}$) 
and we will amortize the extra $2 - \frac{2}{\sqrt{3}}$ growth over 
intervals when its growth rate is smaller than $2(\cT)$ (i.e., when its
growth rate is $\frac{4}{\sqrt{3}}$ or $\frac{6}{\sqrt{3}}$).
The amortization can usually be done because the intervals when the growth rate
of $U+L$ is large must be relatively short compared to intervals when
its growth is smaller, otherwise a gentle path can be shown to exist. To get our
tight bound however, we will need to do more and show that at certain points
(which are points of discontinuity)  we need to use ``cross-edges'' ($l_i,u_i)$.
This is because when the amortization is not
possible there is a long enough interval, say from hexagon
$H_i$ to hexagon $H_j$, when the growth rate of $U+L$ is mostly
$\frac{8}{\sqrt{3}}$.
It turns out that in that case one of $U$ or $L$ has growth rate
bounded by $\frac{2}{\sqrt{3}}$ (say, $U$) and the other ($L$) by 
$\frac{6}{\sqrt{3}}$. This means that path $l_i, l_{i+1}, \dots, l_j$ has
relatively large length with respect to $\Delta(x)$ 
and that $u_i, u_{i+1}, \dots, u_{j}$ is a relatively
short path that can be used to replace the long subpath $l_i, l_{i+1}, \dots, l_j$
with the shorter subpath $l_i, u_i, u_{i+1}, \dots, u_{j}, l_j$ in a path from $p$
to $q$. The $\cT$
stretch factor bound is the result of a min-max optimization between the two
subpaths from $l_i$ to $l_j$, and it is tight as we show in
Section~\ref{sec:conclusion}.

\iftoggle{abstract}
{\begin{figure}}
{\begin{figure}}
\center{\definitions}
\caption{%Triangles $T_1$, $T_2$, \dots, $T_{12}$ are visited in that order when
%moving from $s$ to $t$ along the dotted segment $[st]$. The vertices of each
%triangle $T_i$ ($u_{i-1}$, $u_i$, $l_{i-1}$, $l_i$, two of which are equal)
%lie on the boundary of the hexagon $H_i$.
A gentle path from $u_2$
to $l_{11}$ is one whose length is at most 
$\sqrt{3}d_x(u_2,l_{11})-({\tt y}(u_2)-{\tt y}(l_{11}))$, i.e. the length of the
red dashed piecewise linear curve from $u_2$ to $l_{11}$ (consisting of two 
vertical segments and a third with slope $-\frac{1}{\sqrt{3}}$). The path 
$u_2=u_3,u_4,u_5=u_6,u_7=u_8,l_8=l_9=l_{10},l_{11}$, easily seen to be
bounded--in length--by the red dotted piecewise linear curve, is gentle.
This path can be extended with edge $(l_{11},t)$ to a canonical gentle path
from $u_2$ to $t$; the proof of (Main) Lemma~\ref{le:divide}, in this particular
case, combines the bound on the
length of this path together with the bound on the length of a path from $s$
to $u_2$ obtained via induction.}
\label{fig:definitions}
\end{figure}

Next we turn to the case when the sequence of triangles $T_{1n}$ contains
a gentle path.


\subsection{The Gentle Path Lemma}

Just as in the previous subsection, we consider a linear sequence of triangles
$T_{1n}$ defined with respect to a line $st$ with slope $m_{st}$ satisfying
$0 < m_{st} < \frac{1}{\sqrt{3}}$. We now consider the case when $T_{1n}$
contains a gentle path and state the other of our two key lemmas. We start 
with two definitions:
%We say that $T_{ij}$ is {\em standard} if the left
%vertex of $T_i$ is a left induction point of $T_i$ and the right vertex of
%$T_j$ is a right induction point of $T_j$. Note that $T_{1n}$ is standard.
%A standard $T_{ij}$ is said to be {\em super-standard} if neither the base
%vertex of $T_i$ nor the base vertex of $T_j$ is the endpoint of a gentle path
%in $T_{ij}$. Note that $T_{1n}$ is
%trivially super-standard because $T_1$ and $T_n$ have no base vertices.
\begin{definition}
We say that linear sequence $T_{1n}$ is {\em standard}
if $T_1$ has a left induction vertex or $u_0=l_0$, $T_n$ has a right induction
vertex or $u_n=l_n$, and neither the base vertex of $T_1$ (if any) nor the base
vertex of $T_n$ (if any) is the endpoint of a gentle path in $T_{1n}$. 
\end{definition}
Note that if $u_0=l_0$ and $u_n=l_n$ both hold (i.e., line $st$ goes through
those points) then $T_{1n}$ is trivially standard because $T_1$ and $T_n$
cannot  have base vertices. % Note also that $T_1$ and $T_n$ are the only triangles in $T_{1n}$
%without base vertices.

\begin{definition}
Let $T_{1n}$ be a standard linear sequence.
A gentle path in $T_{1n}$ from $p$ to $q$, where $p$ occurs before $q$, is
{\em canonical in $T_{1n}$} (or simply {\em canonical} if $T_{1n}$ is clear
from the context) if $p$ is a right induction vertex of $T_{i}$ for some
$i \geq 1$  or $p$ is the left vertex of $T_1$ and if $q$ is a left induction
vertex of $T_{j}$ for some $j \leq n$ or $q$ is the right vertex of $T_n$.
\end{definition}
For example, the gentle path 
$u_2=u_3,u_4,u_5=u_6,u_7=u_8,l_8=l_9=l_{10},l_{11},l_{12}$
in Fig.~\ref{fig:definitions} is canonical.



%\begin{lemma}
%\label{le:standard}
%If $T_{ij}$ contains no gentle path then $p_{i-1}$ lies on the W side of $T_i$
%and $p_{j}$ lies on the E side of $T_j$.
%\end{lemma}
%\begin{proof}
%Note that $s$ or $t$ being on the upper side of the $H_1$ or $H_n$ implies the existence of a gentle edges since the high point would be forced above them at an angle of less than $30$ degrees.  Similarly,  $s$ or $t$ being on the lower side of the $H_1$ or $H_n$ implies the existence of a gentle edges since the low point would be forced below them at an angle of less than $30$ degrees.  Since there are no gentle edges in $T_{ij}$, $p_{i-1}$ must be on the west side of $H_i$ and $p_j$ must be on the east side of $H_j$.  
%\end{proof}
%\begin{lemma}
%If $T_{ij}$, for some $i,j$ such that $1 \leq i \leq j \leq n$, contains no
%gentle edge then $T_{ij}$ is standard.
%\end{lemma}
%\begin{proof}
%TO DO.
%\end{proof}
The second key lemma, which we will use alongside (Amortization) 
Lemma~\ref{le:mainlemmaB} to prove our main result, is stated
\iftoggle{abstract}{next; its proof is discussed in Section~\ref{sec:gentle}.}{next and proven in Section~\ref{sec:gentle}.}
%\begin{lemma}[The Key Lemma]
%\label{le:mainlemmaB}
%If $T_{ij}$ contains no gentle path, $T_i$  has a vertex $p$ on side $w$
%of $H_i$, and $T_j$ has vertex $q$ on side $e$ of $H_j$ then there
%a path in $T_{ij}$ from $p$ to $q$ of length at most $(\cT) d_x(p,q)$.
%\end{lemma}
%A more general but weaker version of the Key Lemma is our second technical
%lemma:
%\begin{lemma}
%\label{le:mainlemmaA}
%If $T_{ij}$ contains no gentle edge, $T_i$  has a vertex $p$ on side $w$
%of $H_i$, and $T_j$ has vertex $q$ on side $e$ of $H_j$ then there
%exists a path in $T_{ij}$ from $p$ to $q$ of length at most
%$\frac{4}{\sqrt{3}} d_x(p,q)$.
%\end{lemma}
%We prove this lemma in the next section. %Another technical lemma that we
%require and that we prove in the next section is:
%\begin{lemma}
%\label{le:mainlemma}
%\end{lemma}
%The remainder of this section is devoted to proving Lemma~\ref{le:divide} using
%the Lemma~\ref{le:mainlemmaB} and the following insight (whose proof makes use
%of Lemma~\ref{le:mainlemmaA}):
\begin{lemma}[The Gentle Path Lemma] 
\label{le:boundedGentelSections}
Let $T_{1n}$ be a linear sequence of triangles with respect to a line $st$ with
slope $m_{st}$ such that $0 < m_{st} < \frac{1}{\sqrt{3}}$. If $T_{1n}$ is
standard and contains a gentle path then the path can be extended
to a canonical gentle path in $T_{1n}$.
\end{lemma}
The main idea behind the proof of this lemma is that a gentle path between
$u_r \in U$ and $l_s \in L$ (where, say, $r \leq s$ and 
${\tt x}(u_r) < {\tt x}(u_s)$) in $T_{1n}$ can 
be extended using edge $(u_{r-1},u_r)$, unless $r=0$ or $u_r$ is a right
induction vertex of $T_{r}$, or using edge $(l_{s},l_{s+1})$, unless $s=n$ or
$l_s$ is a left induction vertex of $T_{s+1}$. In other words, a gentle path
can be extended unless it is canonical.

We are now ready to state our main result and provide a proof that uses
the two key lemmas.


\subsection{The main result and the Main Lemma}

\begin{theorem}
\label{th:main}
The stretch factor of a {\Large\varhexagon}-Delaunay triangulation is at most $2$.
\end{theorem}

To prove Theorem~\ref{th:main} we need to show that between any two points
$s$ and
$t$ of a set of points $P$ there is, in the {\Large\varhexagon}-Delaunay
triangulation $T$ on $P$, a path from $s$ to $t$ of length
at most $2d_2(s,t)$. Let $m_{st}$ be the slope of the line $st$ 
passing through $s$ and $t$.
%W.l.o.g., we assume that $s$ has coordinates $(0, 0)$.
Thanks to the hexagon's rotational and reflective symmetries as well as our general 
position assumptions, we can rotate
the plane around $s$ and possibly reflect the plane with respect to the $x$-axis
to ensure that %$t$ is in the positive quadrant of the coordinate system and
$0 < m_{st} < \frac{1}{\sqrt{3}}$. Given this assumption, our main theorem
will follow from:
\begin{lemma}[The Main Lemma] \label{le:divide}
For every pair of points $s,t \in P$ with 
$0 < m_{st} < \frac{1}{\sqrt{3}}$:
\begin{equation}
\label{eq:main}
d_T(s,t) \leq \max\Bigl\{\cT, \sqrt{3}+m_{st}\Bigr\}d_x(s,t).
\end{equation}
\end{lemma}
Before we prove this lemma, we show that it implies the main theorem.
\begin{proof}[Proof of Theorem~\ref{th:main}]
W.l.o.g., we assume that $s$ has coordinates $(0,0)$, $t$ lies in the
positive quadrant, $m_{st} < \frac{1}{\sqrt{3}}$, and $d_2(s,t)=1$.
With these assumptions it follows that 
$\frac{\sqrt{3}}{2} < {\tt x}(t) = d_x(s,t) < 1$
and we need to show that $d_T(s,t) \leq 2$.


By Lemma \ref{le:divide}, either 
$d_T(s,t)\leq (\cT) d_x(s,t) \leq (\cT) < 2$ or
% \[\begin{split}d_T(s,t) &\leq (\sqrt{3}+m) d_x(s,t) \\
%                         &= \sqrt{3}d_x(s,t) + d_y(s,t) \\
%                         &=\sqrt{3}d_x(s,t)+ \sqrt{1-d_x(s,t)^2} \end{split}\]
 \[d_T(s,t) \leq (\sqrt{3}+m_{st}) d_x(s,t) = \sqrt{3}d_x(s,t) + d_y(s,t) = \sqrt{3}d_x(s,t)+ \sqrt{1-d_x(s,t)^2}\]
 which attains its maximum, over the interval $[\frac{\sqrt{3}}{2},1]$, at 
$d_x(s,t)=\frac{\sqrt{3}}{2}$ giving $d_T(s,t) \leq 2$.
\end{proof}

We now turn to the proof of (Main) Lemma~\ref{le:divide}. We start by noting that if
there is a point $p$ of $P$ on the segment $[st]$
then~(\ref{eq:main}) would follow if~(\ref{eq:main}) holds for the pairs
of points $s,p$ and $p,t$; we can therefore assume that no point of $P$ other
than $s$ and $t$ lies on the segment $[st]$. We can also assume, as argued in
Section~\ref{sec:prelim}, that segment
$[st]$ does not intersect the outer face of the triangulation $T$.
We assume w.l.o.g. that
$s$ has coordinates $(0,0)$ and thus $t$ lies in the positive quadrant.%, and
%the slope $m_{st}$ of the line through $s$ and $t$ is at most
%$\frac{1}{\sqrt{3}}$. %The line containing segment $[st]$ divides
%the plane into two half-planes; a point in the same half-plane as point
%$(0, 1)$ is said to be \emph{above} segment $[st]$, otherwise it is
%\emph{below}. 

Let $T_1, T_2, T_3, \dots, T_n$ be the sequence of
triangles of the %modified 
triangulation $T$ that line segment $[st]$ intersects
when moving from $s$ to $t$ (refer to Fig.~\ref{fig:definitions}). (Recall
that we assume that segment $[st]$ does not intersect the outer face of
$T$.)
Clearly, $T_{1n}$ is a linear
sequence of triangles and we assign labels $u_i$ and $l_i$ to the points
and define sets $U$ and $L$ as described in Subsection~\ref{sub:keylemma}.
We note that all arguments in the rest of this paper use only points
and edges in $T_{1n}$. 


%\subsection{The Key Lemma}

%???????????????

%$T_{1n}$ as defined in setion 3.1!!!!!!!!!!!!!!


%Let $l_i \in L$ and $u_j \in U$. If $i \leq j$ and ${\tt x}(l_i) < {\tt x}(u_j)$
%then we will say that $l_i$ {\em occurs before} $u_j$, and if $j \leq i$ and
%${\tt x}(u_j) < {\tt x}(l_i)$ then we will say that $u_j$ {\em occurs before}
%$l_i$.
%\begin{definition}
%Given points $l_i \in L$ and $u_j \in U$ such that one occurs before the other,
%a path between them is \emph{gentle} if the length of the path is not greater
%than $\sqrt{3}d_x(u_j,l_i)-({\tt y}(u_j)-{\tt y}(l_i))$.
%\end{definition}

%See Fig.~\ref{fig:definitions} for an illustration of a gentle path. Note that
%a gentle edge is a gentle path (e.g., $(u_0,l_1)$ and 
%$(u_8,l_8=l_9)$ in Fig.~\ref{fig:definitions}).

%\begin{lemma}
%\label{lem:props}
%Let $s,t \in P$ with $0 < m_{st} < \frac{1}{\sqrt{3}}$. For every $i$ s.t.
%$0 \leq i < n$:
%\begin{itemize}
%\item If $u_i$ lies on side $s_w$ of $H_{i+1}$ or $l_{i+1}$ lies on 
%side $n_e$ of $H_{i+1}$ then $(u_i,l_{i+1})$ is gentle.
%\item If $l_i$ lies on side $n_w$ of $H_{i+1}$ or $u_{i+1}$ lies on the $s_e$
%side of $H_{i+1}$ then $(l_i,u_{i+1})$ is gentle.
%\item None of the following can occur: $u_i$ lies on side $s_e$ of $H_{i+1}$,
%$l_{i+1}$ lies on side $n_w$ of $H_{i+1}$, $l_i$ lies on side $n_e$ of $H_{i+1}$,
%and $u_{i+1}$ lies on the $s_w$ side of $H_{i+1}$. 
%\end{itemize}
%\end{lemma}
%\noindent Note, for example, that $u_8$ lies on side $s_w$ of hexagon
%$H_9$ in Fig.~\ref{fig:definitions} and that edge $(u_8,l_9)$ is gentle. 
%A gentle edge that satisfies one of the four cases in the first two bullet points of
%Lemma~\ref{lem:props} will
%be called {\em irregular}. 

Note that, since $st$ must intersect the
interior of $H_1$, $s$ can only lie on the $n_w$, $w$, or $s_w$ sides of $H_1$;
by Lemma~\ref{lem:props}, if $s$ lies on the $n_w$ side of $H_1$ then
$(s,u_1) = (l_0,u_1)$ is gentle, and
if $s$ lies on the $s_w$ side of $H_1$ then $(s,l_1) = (u_0,l_1)$ is gentle. 
Similarly, $t$ can only lie on the $n_e$, $e$, or $s_e$ sides of $H_n$; if
$t$ lies on the $s_w$ side of $H_n$ then $(t,l_{n-1}) = (u_n,l_{n-1})$ is 
gentle, and if $t$ lies on the $n_w$ side of $H_n$ then 
$(t,u_{n-1}) = (l_n,u_{n-1})$ is gentle. Note that this means that if $T_{1n}$ has
no gentle edge then it is regular.
%(Technical) Lemma~\ref{le:mainlemmaA} applies.


%\begin{proof}
%If $u_i$ lies on side $s_w$ of some hexagon $H_{i+1}$ then, since
%$0 < m_{st} < \frac{1}{\sqrt{3}}$, either $u_i=u_{i+1}$ and $l_i$ and $l_{i+1}$
%must lie on sides $s_e$ and $e$ of $H_{i+1}$, respectively, or $l_i=l_{i+1}$ must
%lie on side $s_e$ or $e$ of $H_{i+1}$. Either way, the slope of
%the line going through $u_i$ and $l_{i+1}$ must be between 
%$-\frac{1}{\sqrt{3}}$ and  $\frac{1}{\sqrt{3}}$. Similar arguments can be used
%to handle the remaining three cases in the first two bullet points.

%Let the left and right intersection points of line $st$ with hexagon $H_{i+1}$
%be  $h_i$ and $h_{i+1}$. Note that when traveling clockwise along the sides of
%$H_{i+1}$ the points will be visited in this order: 
%$h_i,u_i,u_{i+1},h_{i+1}, l_{i+1}, l_i$.
%If $u_i$ lies on side $s_e$ of $H_{i+1}$ then $i>0$ and, because
%$0 < m_{st} < \frac{1}{\sqrt{3}}$, either $u_{i+1}$ (if $u_i \not= u_{i+1}$)
%or $l_{i+1}$ (if $l_i \not= l_{i+1}$) would have to lie on side $s_e$
%of $H_{i+1}$ as well, which violates our general position assumption for the
%set of  points P. The remaining three cases are handled similarly.
%\end{proof}


%By the above lemma, it $T_{1n}$ has no gentle edge then it is regular and 
%(Technical) Lemma~\ref{le:mainlemmaA} applies. A narrower but much stronger
%version of (Technical) Lemma~\ref{le:mainlemmaA} applies as well and it is
%the key to our proof of the main result of this paper:
%\begin{lemma}[The Key Lemma]
%\label{le:mainlemmaB}
%Let $0 < m_{st} < \frac{1}{\sqrt{3}}$. If $T_{1n}$ is regular and contains no
%gentle path then there is a path in $T_{1n}$ from 
%the left induction vertex $p$ of $T_1$ to the right induction vertex $q$ of
%$T_n$ of length at most  $(\cT) d_x(p,q)$.
%\end{lemma}
%We will prove the Key Lemma in Section~\ref{sec:proofB} by building on the
%analysis done in Section~\ref{sec:proofA} to prove (Technical) 
%Lemma~\ref{le:mainlemmaA}. 
%We will consider function $\bar{U}+\bar{L}$ that we informally described in 
%Subsection~\ref{sub:keylemma} and that bounds the sum of the lengths of the
%upper path $p,u_0,\dots,u_n,q$ and lower path $p,l_0,\dots,l_n,q$ in $T_{1n}$.
%We will consider the intervals when its growth rate
%is higher than $2(\cT)$ (i.e., when its growth rate is $\frac{8}{\sqrt{3}}$)
%and we will amortize the extra $2 - \frac{2}{\sqrt{3}}$ growth over 
%intervals when
%its growth rate is smaller than $2(\cT)$ (i.e., when its growth rate is
%$\frac{4}{\sqrt{3}}$ or $\frac{6}{\sqrt{3}}$). When the amortization is not
%possible, it will mean that there is a long interval, say from hexagon
%$H_i$ to hexagon $H_j$, when the growth rate of $\bar{U}+\bar{L}$ is mostly
%$\frac{8}{\sqrt{3}}$.
%It turns out that in that case one of $\bar{U}$ or $\bar{L}$ has growth rate
%bounded by $\frac{2}{\sqrt{3}}$ (say $\bar{U}$) and the other ($\bar{L}$) by 
%$\frac{6}{\sqrt{3}}$. This means that path $l_i, l_{i+1}, \dots, l_j$ has
%relatively large length and that $u_i, u_{i+1}, \dots, u_{j}$ is a relatively
%short path that can be used to replace subpath $l_i, l_{i+1}, \dots, l_j$ with
%$l_i, u_i, u_{i+1}, \dots, u_{j}, l_j$ in a path from $p$ to $q$. The $\cT$
%stretch factor bound is the result of a min-max optimization between the two
%subpaths from $l_i$ to $l_j$, and it is tight as we show in
%Section~\ref{sec:conclusion}.

%\subsection{The Gentle Path Lemma}

%Let T1n be a regular linear sequence with respect to line st with slope m 
%Now we turn our attention to the case when the sequence of triangle 

%$T_{1n}$ contains a gentle path. We
%start with two definitions:
%\begin{definition}
%We say that linear sequence $T_{ij}$ is {\em standard}, for 
%$1 \leq i \leq j \leq n$, 
%if $T_i$ has a left induction vertex or $i=1$, $T_j$ has a right induction
%vertex or $j=n$, and neither the base vertex of $T_i$ (if any) nor the base
%vertex of $T_j$ (if any) is the endpoint of a gentle path in $T_{ij}$. 
%\end{definition}
%Note that $T_{1n}$ is trivially standard because $T_1$ and $T_n$ have no base
%vertices. Note also that $T_1$ and $T_n$ are the only triangles in $T_{1n}$
%without base vertices.

%\begin{definition}
%Let $T_{ij}$ be standard for some $i,j$ such that $1 \leq i \leq j \leq n$.
%A gentle path in $T_{ij}$ from $p$ to $q$, where $p$ occurs before $q$, is
%{\em canonical in $T_{ij}$} (or simply {\em canonical} if $T_{ij}$ is clear
%from the context) if $p$ is a right induction vertex of $T_{i'}$ for some
%$i' \geq i$  or $p$ is the left vertex of $T_i$ and if $q$ is a left induction
%vertex of $T_{j'}$ for some $j' \leq j$ or $q$ is the right vertex of $T_j$.
%\end{definition}
%For example, the gentle path 
%$u_2=u_3,u_4,u_5=u_6,u_7=u_8,l_8=l_9=l_{10},l_{11},l_{12}$
%in Fig.~\ref{fig:definitions} is canonical.

%The second important lemma, which we will use alongside (Key) 
%Lemma~\ref{le:mainlemmaB} to prove (Main) Lemma~\ref{le:divide}, is stated next
%and proven in Section~\ref{sec:gentle}.

%\begin{lemma}[The Gentle Path Lemma] 
%\label{le:boundedGentelSections}
%Let $0 < m_{st} < \frac{1}{\sqrt{3}}$ and $0 \leq i \leq j \leq n$. 
%If $T_{ij}$ is standard and contains a gentle path then the path can be extended
%to a canonical gentle path in $T_{ij}$.
%\end{lemma}
%The main idea behind the proof of this lemma is that a gentle path between
%$u_r \in U$ and $l_s \in L$ (where, say, $r \leq s$ and 
%${\tt x}(u_r) < {\tt x}(u_s)$) in $T_{ij}$ can 
%be extended using edge $(u_{r-1},u_r)$, unless $r=i-1$ or $u_r$ is a right
%induction vertex of $T_{r}$, or using edge $(l_{s},l_{s+1})$, unless $s=j$ or
%$l_s$ is a left induction vertex of $T_{s+1}$. In other words, a gentle path
%can be extended unless it is canonical.

We now informally describe the approach we use to prove (Main) 
Lemma~\ref{le:divide}. We first note that (Amortization) Lemma~\ref{le:mainlemmaB}
and (Gentle Path) Lemma~\ref{le:boundedGentelSections}
rely on (Technical) Lemma~\ref{le:mainlemmaA}. We will prove (Main) 
Lemma~\ref{le:divide} that
bounds the length of the shortest path in $T_{1n}$ from $s$ to $t$ as follows. 
If $T_{1n}$ does not contain a gentle path then it is regular and the proof follows
from (Amortization) Lemma~\ref{le:mainlemmaB}. If $T_{1n}$ contains a gentle path
then by (Gentle Path) Lemma~\ref{le:boundedGentelSections} it must contain a
canonical gentle path $\mathcal{G}$ from, in general, a right induction vertex of $T_i$ to
a left induction vertex of $T_j$, where $0 \leq i < j \leq n$. We can assume, 
using (Gentle Path) Lemma~\ref{le:boundedGentelSections}, that $\mathcal{G}$ is maximal in
the sense that it is not a subpath of any other gentle path in $T_{1n}$.  
The maximality of $\mathcal{G}$ will guarantee that neither $T_{1i}$ nor $T_{jn}$ contains
a gentle path whose endpoint is the base vertex of $T_i$ or the base
vertex of $T_j$, respectively. Therefore $T_{1i}$ and $T_{jn}$ are standard and
we then proceed by induction to prove a ``more general'' version of (Main)
Lemma~\ref{le:divide} for $T_{1i}$ and $T_{jn}$. The obtained bounds on the
lengths of shortest paths from $s$ to the right induction vertex of $T_i$ and 
from the left induction vertex of $T_j$ to $t$ are combined
with the bound on the length of gentle path $\mathcal{G}$ to complete the proof of (Main)
Lemma~\ref{le:divide}. Our reliance on induction means that we need to restate
the Main Lemma so it is amenable to an inductive proof:
\begin{lemma}[The Generalized Main Lemma]
Let $s,t \in P$ such that $0 < m_{st} < \frac{1}{\sqrt{3}}$ and let $T_{1n}$
be the linear sequence of triangles that segment $[st]$ intersects. If
$T_{ij}$, for some $i,j$
such that $1 \leq i \leq j \leq n$, is standard, $p$ is the left vertex
of $T_i$, and $q$ is the right vertex of $T_j$ then
\[d_{T_{ij}}(p,q) \leq  \max\{\cT,\sqrt{3}+m_{st}\}d_x(p,q).\]
\end{lemma}
Note that (Main) Lemma \ref{le:divide} is a special case of this
statement when $i=1$ and $j=n$ since $T_{1n}$ is (trivially) standard, 
$s$ is the left vertex of $T_1$, and $t$
is the right vertex of $T_n$.

\begin{proof}
%As indicated in the above paragraph, we use induction to prove a more general
%statement: 
%\begin{quote}
%if $T_{ij}$, for some $i,j$
%such that $1 \leq i \leq j \leq n$, is standard, $p$ is the left vertex
%of $T_i$, and $q$ is the right vertex of $T_j$ then
%$$d_{T_{ij}}(p,q) \leq  \max\{\cT,\sqrt{3}+m_{st}\}d_x(p,q)$$
%where $m_{st}$ is the slope of $[st]$. 
%\end{quote}
%The claim in Lemma \ref{le:divide} is a
%special case of this statement when $i=1$ and $j=n$.

We proceed by induction on $j-i$. If $T_{ij}$ is standard and there
is no gentle path in $T_{ij}$ (the base case) then, by Lemma~\ref{lem:props},
the linear sequence of triangles in $T_{ij}$ is regular and thus, by (Amortization)
Lemma~\ref{le:mainlemmaB}, we have $d_T(p,q) \leq (\cT) d_x(p,q)$.

If $T_{ij}$ is standard and there is a gentle path in $T_{ij}$,
then, by Lemma~\ref{le:boundedGentelSections}, there
exist points $u_{i'}$ and $l_{j'}$ in $T_{ij}$ such that there is a canonical
gentle path between $u_{i'}$ and $l_{j'}$ in $T_{ij}$. We also
assume that the canonical path between $u_{i'}$ and $l_{j'}$ is maximal in the
sense that it is not a proper subpath of a gentle path in $T_{ij}$. W.l.o.g.,
we assume that $u_{i'}$ occurs before $l_{j'}$, and so 
$i-1 \leq i' \leq j' \leq j$, ${\tt x}(u_{i'}) < {\tt x}(l_{j'})$, and
$d_T(u_{i'},l_{j'}) \leq \sqrt{3}d_x(u_{i'},l_{j'})-({\tt y}(u_{i'})-{\tt y}(l_{j'}))$.
Since $u_{i'}$ is either $s$
or above $st$ and $l_{j'}$ is either $t$ or below $st$, it follows that
$-({\tt y}(u_{i'})-{\tt y}(l_{j'}))\leq m_{st}d_x(u_{i'},l_{j'})$. Therefore,
$d_T(u_{i'},l_{j'}) \leq (\sqrt{3}+m_{st})d_x(u_{i'},l_{j'})$.

Since the gentle path from $u_{i'}$ to $l_{j'}$ is canonical, either $u_{i'}$ 
is a right induction vertex of $T_{i'}$ and $i' \geq i$ or $u_{i'}=u_{i-1}$.
In the first case, because $u_{i'}$ is on side $e$ of $H_{i'}$ the base
vertex $l_{i'-1}=l_{i'}$ of $T_{i'}$ must satisfy 
${\tt x}(l_{i'}) < {\tt x}(u_{i'})$. Suppose that
$l_{i'}$ is the endpoint of a gentle path in $T_{ii'}$ from, say, point
$u_{i''}$ then we would have
\begin{align*}
d_{T_{ij}} (u_{i''},l_{j'}) & \leq d_{T_{ij}}(u_{i''},l_{i'}) + d_2(l_{i'},u_{i'}) + d_{T_{ij}}(u_{i'},l_{j'}) \\
             & \leq \sqrt{3}d_x(u_{i''},l_{i'})-({\tt y}(u_{i''})-{\tt y}(l_{i'})) + \sqrt{3}d_x(l_{i'},u_{i'}) - ({\tt y}(l_{i'})-{\tt y}(u_{i'})) \\
             & \quad + \sqrt{3}d_x(u_{i'},l_{j'})) - ({\tt y}(u_{i'})-{\tt y}(l_{j'})) \\
             & \leq \sqrt{3}d_x(u_{i''},l_{j'})-({\tt y}(u_{i''})-{\tt y}(l_{j'}).
\end{align*}
This contradicts the maximality of the canonical gentle path from $u_{i'}$ to
$l_{j'}$. This means that $l_{i'}$ is not the endpoint of a gentle path in
$T_{ii'}$. Therefore $T_{ii'}$ is standard, the inductive hypothesis
applies, and $d_T(p,u_{i'}) \leq \max\{\cT, \sqrt{3}+m_{st}\}d_x(p,u_{i'})$. 
In the second case, because $T_{ij}$ is standard, $u_{i'}$
cannot be the base vertex of $T_i$ and so $u_{i'}=p$. The 
inequality $d_T(p,u_{i'}) \leq \max\{\cT, \sqrt{3}+m_{st}\}d_x(p,u_{i'})$ then
holds trivially. 

Similarly, we can show that
$d_T(l_{j'},q) \leq \max\{\cT, \sqrt{3}+m_{st}\}d_x(l_{j'},q)$. Thus: 
\begin{align*} d_T(p,q) & \leq d_T(p,u_{i'})+d_T(u_{i'},l_{j'})+d_T(l_{j'},q) \\
                        & \leq \max\{\cT, \sqrt{3}+m_{st}\}(d_x(p,u_{i'}) + d_x(l_{j'},q)) + (\sqrt{3}+m_{st})d_x(u_{i'},l_{j'}) \\
                        & \leq \max\{\cT , \sqrt{3}+m_{st}\}d_x(p,q)\end{align*}\end{proof}



