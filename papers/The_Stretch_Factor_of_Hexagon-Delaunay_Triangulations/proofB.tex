\begingroup
\def\thetheorem{\ref{le:mainlemmaB}}
\begin{lemma}[The Amortization Lemma]
%\label{le:mainlemmaB}
Let $T_{1n}$ be a regular linear sequence with respect to line $st$ with slope
$m_{st}$. If $0 < m_{st} < \frac{1}{\sqrt{3}}$ and if $T_{1n}$ contains no
gentle path then there is a path in $T_{1n}$ from 
the left induction vertex $p$ of $T_1$ to the right induction vertex $q$ of
$T_n$ of length at most  $\left(\cT\right) d_x(p,q)$.
\end{lemma}
\addtocounter{theorem}{-1}
\endgroup

The proof of the lemma builds on the framework developed in the previous
section and on a careful analysis of the growth rates shown in
Lemma~\ref{lem:growthrates} when $T_{1n}$ contains no gentle path. 
We show that in that case
the average growth rate of $P(x)=U(x)+L(x)$ is at most $2\left(\cT\right)$.

As Lemma~\ref{lem:growthrates} demonstrates, the growth rate of $P(x)$ at a
particular time $x$ (we find it useful to use the time intuition and think of
$x$ as time going from time ${\tt x}(p)$ to time ${\tt x}(q)$) 
is one of $\frac{4}{\sqrt{3}}$, $\frac{6}{\sqrt{3}}$, or
$\frac{8}{\sqrt{3}}$. We will refer to transitions 
$t_{wn_e},t_{s_we},t_{s_ew},t_{en_w}$---transitions for which
the growth rate $\frac{\Delta P(x)}{\Delta x}$ is $\frac{8}{\sqrt{3}}$---as
{\em bad}. Note that when $t(x)$ is not bad, the growth rate of $P(x)$ is at
most $\frac{6}{\sqrt{3}}$, well under the desired growth rate of $2\left(\cT\right)$. 
Our goal is to, when possible, spread (i.e., amortize) the 
``extra'' $\frac{2}{\sqrt{3}}$ of the growth rate
of $P(x)$ when $t(x)$ is bad over wider intervals of time. % to show that the
%average growth rate of $P(x)$ is at  most $2\left(\cT\right)$. 
To do this we define intervals of time
during which a particular bad transition takes place %.
%monitor what intervals
%of time we are amortizing a particular extra growth over. 
%To help us with this
%we introduce the following definitions, 
(see also Fig.~\ref{fig:badintervals}):
% To aid discussion we will refer to $t_8$ and $t_9$ upper bad transitions and $t_{wn_w}$ and $t_{ww}$ as lower bad transitions.  Similarly, we will refer to $t_8$ and $t_{ww}$ as left bad transitions and $t_9$ and $t_{wn_w}$ as right bad transitions.

%From now on we will no longer refer to the original centers of Hexagons, only positions between $0$ and $p_k$.  We will borrow standard notation and take $[x_i,x_j]$ to be the interval from $x_i$ to $x_j$.  Let $\Delta P(x_i,x_j)= P(x_i)-P(x_j)$.  We must control the bad transitions by amortizing their cost over better transitions.  To this end we define bad sections to group bad transitions to be more easily handled.  To keep things simple we will think of the transition at $x=0$ to be of type $t_7$ and the transition at $x=x(p_k)$ to be $t_{wn_w}$ each being instantaneous and causing no bad growth. 

\begin{definition}
Given $x_l, x_r$ such that ${\tt x}(p) \leq x_l < x_r \leq {\tt x}(q)$, the interval
$[x_l,x_r]$ is a 
\begin{itemize}
\item $t_{wn_e}$-interval if $t(x_l)= t_{wn_e}$ and $t(x) \not= t_{\ast w}$ for all
$x_l < x < x_r$ and a strict $t_{wn_e}$-interval if, in addition,
$t(x) \not= t_{en_w}, t_{\ast e}$ when $x_l < x < x_r$.
\item $t_{s_ew}$-interval if $t(x_l)= t_{s_ew}$ and $t(x) \not= t_{w \ast}$ for all
$x_l < x < x_r$ and a strict $t_{s_ew}$-interval if, in addition,
$t(x) \not= t_{s_we}, t_{e \ast}$ when $x_l < x < x_r$.
\item $t_{s_we}$-interval if $t(x_r)= t_{s_we}$ and $t(x) \not= t_{e \ast}$ for all
$x_l < x < x_r$ and a strict $t_{s_we}$-interval if, in addition,
$t(x) \not= t_{s_ew}, t_{w \ast}$ when $x_l < x < x_r$
\item $t_{en_w}$-interval if $t(x_r)= t_{en_w}$ and $t(x) \not= t_{\ast e}$ for all
$x_l < x < x_r$ and a strict $t_{en_w}$-interval if, in addition,
$t(x) \not= t_{wn_e}, t_{\ast w}$ when $x_l < x < x_r$
\end{itemize}
\end{definition}

\begin{figure}[!b]
\begin{center}
\badintervals
\end{center}

\caption{Illustrations of (left) $t_{wn_e}$- and $t_{s_ew}$-intervals and (right)
$t_{s_we}$- and $t_{en_w}$-intervals. In a $t_{wn_e}$-interval $[x_l,x_r]$, for
example, $t(x) \not= t_{\ast w}$; if the interval is strict then 
$t(x) \not= t_{en_w}, t_{\ast e}$ as well.}
\label{fig:badintervals}
\end{figure}



Note that if $t(x) = t_{ij}$ is bad for some $x$ in $[{\tt x}(p), {\tt x}(q)]$
then a strict $t_{ij}$-interval contains $x$.
We refer to $t_{wn_e}$- and $t_{s_ew}$-intervals as {\em left intervals} and to
$t_{s_we}$- and $t_{en_w}$-intervals as {\em right intervals}. Note that left
intervals do not intersect each other and neither do the right intervals.
A maximal strict left interval can intersect at most one maximal strict right
interval and vice versa and when that is the case the left interval is to the
left of the right interval.
We will take advantage of the limited interaction between maximal
strict intervals 
to amortize the bad growth taking place within a strict interval over
intervals of time
that do not overlap or, if they do, overlap in a restricted way.


We introduce notation that will help us keep track of the relative horizontal
positions of $u(x)$ and $\ell(x)$: the {\em forward} abcsissa 
$f(x) = \max\{{\tt x}(\ell(x)), {\tt x}(u(x))\}$ and the {\em back} abcsissa
$b(x) = \min\{{\tt x}(\ell(x)), {\tt x}(u(x))\}$. The following facts will
be used heavily without reference throughout this section:
\begin{proposition}
\label{lem:bandf}
Given the assumptions of Lemma~\ref{le:mainlemmaB}, for $i = 1,\dots,n$:
\renewcommand{\labelenumi}{(\alph{enumi})}
\begin{enumerate}
%\item If $l_{i-1}=l_i$ is the base
%vertex of $T_i$ then $u_{i-1}, u_i$ lie on sides $w, n_w$ or $w, n_e$ or
%$n_w, n_e$ or $n_w,e$ or $n_e, e$ of $H_i$; if $u_{i-1}=u_i$ is the base
%vertex of $T_i$ then $l_{i-1}, l_i$ lie on sides $w, s_w$ or $w, s_e$ or
%$s_w, s_e$ or $s_w,e$ or $s_e, e$ of $H_i$. 
\item ${\tt x}(l_{i-1}) \leq {\tt x}(l_i)$ and ${\tt x}(u_{i-1}) \leq {\tt x}(u_i)$.
\end{enumerate}
and for ${\tt x}(p) \leq x_l < x_r \leq {\tt x}(q)$:
\begin{enumerate}
\addtocounter{enumi}{1}
\item $u(x_l) \leq u(x_r)$ and $\ell(x_l) \leq \ell(x_r)$.
\item $w(x_l) \leq w(x_r)$ and $e(x_l) \leq e(x_r)$.
\item $b(x_l) \leq b(x_r)$ and $f(x_l) \leq f(x_r)$.
\end{enumerate}
\end{proposition}

\begin{proof}
Note that if $l_{i-1}=l_i$ is the base vertex of $T_i$ then $u_{i-1}, u_i$ lie 
on sides $w, n_w$ or $w, n_e$ or $n_w, n_e$ or $n_w,e$ or $n_e, e$ of $H_i$; 
if $u_{i-1}=u_i$ is the base vertex of $T_i$ then $l_{i-1}, l_i$ lie on sides
$w, s_w$ or $w, s_e$ or $s_w, s_e$ or $s_w,e$ or $s_e, e$ of $H_i$.
% Part {\it (a)} follows from the definition of a linear sequence of triangles
%and Lemma~\ref{lem:props} and {\it (b)} follows from {\it (a)}. 
Part {\it (b)} follows from this and parts {\it (c)}
and {\it (e)} follow from  {\it (b)}. Part {\it (d)} follows from 
Lemma~\ref{lem:growthrates}.
\end{proof}

%
% Finally, we define $\theta_f(x) = f(x) - x$ and 
% $\theta_b(x) = x - b(x)$. 

Finally, we define key functions $\delta_f(x) = e(x) - f(x)$ and 
$\delta_b(x) = b(x) - w(x)$ that will be used when $t(x)$ is bad.
Note that when $t(x)= t_{wn_e},t_{s_ew}$ we have 
$\delta_f(x) \leq {\tt r}(x)$ and when $t(x)=t_{s_we},t_{en_w}$ 
then $\delta_b(x) \leq {\tt r}(x)$. 

We now prove several technical lemmas that we need. Unlike our bounds for
${\bar U}(x)$ and ${\bar L}(x)$ that only made use of edges of the upper and
lower path, we will sometimes make use of {\em cross-edges} $(u_i,l_i)$ in 
computing tighter bounds for $U(x)$ or $L(x)$, and therefore $P(x)$. We will,
in particular, use edges  $(\ell(x),u(x))$ for values of $x$ when $t(x)$ is bad.
To motivate the next lemma which bounds the cost of using such edges, consider,
for example, a  $t_{wn_e}$-interval starting at $x_l$ and the edge 
$(\ell(x_l), u(x_l)) = (l_{i-1}=l_i, u_i)$ between the $w$ and $n_e$ sides of 
$H(x)$. Then
\begin{align*} 
U(x_l) & = d_{T_{1i}}(p,u_i) + p_N(u_i, x_l) \nonumber \\
       & \leq \min \begin{cases}
                 d_{T_{1(i-1)}}(p,u_{i-1}) + d_2(u_{i-1},u_i) + p_N(u_i, x_l) \\ 
                 d_{T_{1(i-1)}}(p,l_{i-1}) + d_2(l_{i-1},u_i) + p_N(u_i, x_l) \\
                   \end{cases} \\ \nonumber
       & \leq L(x_l) - p_S(l_{i-1},x_l) + d_2(l_{i-1},u_i) + p_N(u_i, x_l)
\end{align*}

The term $- p_S(l_{i-1},x_l) + + d_2(l_{i-1},u_i) + p_N(u_i, x_l)$ can be seen
as the cost of using edge $(l_{i-1},u_i)$. The following lemma provides a bound
on this cost in terms of $\delta_f(x)$ as well as the cost of using other 
{\em bad cross-edges} (see also Fig.~\ref{fig:gentlepath}-(a)):
\begin{lemma}
\label{lem:switchpath}
Given the assumptions of Lemma~\ref{le:mainlemmaB}, let $t(x)$ be bad for some
$x \in [{\tt x}(p), {\tt x}(q)]$.% and let $l_i = \ell(x)$ and $u_i = u(x)$. 
%, l_j = \ell(x_r),$ and $u_j = u(x_r)$ for some $i \leq j$.
%If $[x_l,x_r]$ is
%
\begin{itemize}
\item If $t(x) = t_{wn_e}$ then
$-p_S(x) + d_2(\ell(x),u(x)) + p_N(x) \leq \left(2 - \frac{2}{\sqrt{3}}\right)\delta_f(x)$
\item If $t(x) = t_{s_ew}$ then
$p_S(x) + d_2(\ell(x),u(x)) - p_N(x) \leq \left(2 - \frac{2}{\sqrt{3}}\right)\delta_f(x)$
\item If $t(x) = t_{s_we}$ then
$p_S(x) + d_2(\ell(x),u(x)) - p_N(x) \leq \left(2 - \frac{2}{\sqrt{3}}\right)\delta_b(x)$
\item If $t(x) = t_{en_w}$ then
$-p_S(x) + d_2(\ell(x),u(x)) + p_N(x) \leq \left(2 - \frac{2}{\sqrt{3}}\right)\delta_b(x)$
\end{itemize}
\end{lemma}
%
\begin{proof}
Assuming that $t(x) = t_{wn_e}$ (the other cases follow
by symmetry),
%
% Also, we note that $-p_S(l_i,x_l) + d_2(l_i, u_i)$ is maximized
% when $l_i$ is the SW corner of $H(x_l)$, as illustrated in Fig.~\ref{fig:gentlepath}-(b).
% %	
% Assuming that is the case, we let $\theta = \angle u_i l_i N(x_l)$, and
%
we let $w$ to be the SW corner of $H(x)$ and let $\theta = \angle u_i w N(x)$
as illustrated in Fig.~\ref{fig:gentlepath}-(a). We then
%
obtain
$p_S(x)=p_S(\ell(x),x) \geq \frac{2}{\sqrt{3}}{\tt r}(x)$,
$d_2(\ell(x), u(x)) \leq \frac{2}{\cos(\theta)}{\tt r}(x)$, and
$p_N(x)=p_N(u(x),x) = -2\tan(\theta){\tt r}(x)$.
%	
Noting that
$0 \leq \theta \leq \frac{\pi}{6}$ and
	using
$\frac{1}{\cos(\theta)} \leq 1 + (2-\sqrt{3})\tan(\theta)$ when
$\theta \in [0,\frac{\pi}{6}]$, 
we have
\begin{align*}
-p_S(x) + d_2(\ell(x),u(x)) + p_N(x) & \leq \left(-\frac{2}{\sqrt{3}} + \frac{2}{\cos(\theta)} -2\tan(\theta)\right){\tt r}(x)\\
& \leq \left(-\frac{2}{\sqrt{3}} + 2 + 2(2-\sqrt{3})\tan(\theta) -2\tan(\theta)\right){\tt r}(x) \\
& = \left(2-\frac{2}{\sqrt{3}}\right)\left(1-\sqrt{3}\tan(\theta)\right){\tt r}(x)
\end{align*}
Since ${\tt r}(x) - \delta_f(x) = \sqrt{3}\tan(\theta){\tt r}(x)$, the proof
follows from the last inequality.
\end{proof}

\iftoggle{abstract}
{\begin{figure}[!b]}
{\begin{figure}}
\gentlepath

\hspace{1.9cm}(a) \hspace{2.6in} (b)

\caption{(a) Illustration and proof of Lemma~\ref{lem:switchpath} in the case
of $t(x) = t_{wn_e}$. (b) 
Lemma~\ref{lem:gentlepath} in the case of a $t_{wn_e}$-interval 
$[x_l,x_r]$. The dotted curve from $N(x_l)$ to $N(x_r)$ consists of a
sequence of piecewise linear curves each representing the growth rate
$\Delta p_N(x)$ over an interval $\Delta x$ during which $t(x)$ is a fixed
transition. A few examples of such piecewise linear curves are shown in bold in
Fig.~\ref{fig:growth}. The dotted curve has total length 
$\bar{U}(x_r) - \bar{U}(x_l)$ and its slope is $-\frac{1}{\sqrt{3}}$ exactly
when $t(x)$ is either $t_{\ast n_e}$ (which includes bad transition $t_{wn_e}$) 
or $t_{\ast e}$. Lemma~\ref{lem:gentlepath} states that the time (shown with
thick blue segments) spent in those transitions within interval $[x_l, x_r]$
is bounded by ${\tt r}(x_l) + (\sqrt{3} - 1)\delta_f(x_l)$, if
$T_{1n}$ contains no gentle path.}
\label{fig:gentlepath}
\end{figure}


The following important lemma bounds the amount of time
certain transitions, and especially bad transition $t_{ij}$, can take place 
within a
$t_{ij}$-interval $[x_l,x_r]$ if $T_{1n}$ contains no gentle path. By {\it
``amount of time a transition $t_{ij}$ takes place in an interval $[x_l,x_r]$''}
we mean the Lebesgue measure of the set 
$\{x \in [x_l,x_r]: t(x) = t_{ij}\}$
which is the union of a finite number of disjoint open intervals inside
$[x_l,x_r]$. The lemma we state below captures 
the following insight. When $t(x)$ is a bad transition
and the growth rate of $P(x)$ is $\frac{8}{\sqrt{3}}$ then
by Lemma~\ref{lem:growthrates} one of $U(x)$ or $L(x)$ has growth rate 
$\frac{2}{\sqrt{3}}$. More specifically, if, say, $t(x) = t_{wn_e}$ then 
$\frac{\Delta U(x)}{\Delta x} = \frac{2}{\sqrt{3}}$, meaning that the
upper path fragment $u(x_l)=u_i, u_{i+1}, \dots, u(x_r)$ has a
``very gentle'' average slope. Therefore if $t(x) = t_{wn_e}$ for a sufficiently
long enough time within $t_{wn_e}$-interval $[x_l,x_r]$ then the path 
$\ell(x_l), u(x_l), \dots, u(x_r)$ would have to be a gentle path.
We actually need the contrapositive of this insight (see also Fig.~\ref{fig:gentlepath}-(b)):

\begin{lemma}
\label{lem:gentlepath}
Given the assumptions of Lemma~\ref{le:mainlemmaB},
let ${\tt x}(p) \leq x_l \leq x_r \leq {\tt x}(q)$,
and let $z_{ij}$ be the amount of time within interval $[x_l,x_r]$ spent
in transition $t_{ij}$. If $[x_l,x_r]$ is
%
% \begin{itemize}
% \item a $t_{wn_e}$-interval then
% $z_{\ast n_e} + z_{\ast e} \leq \theta_f(x_l) + \sqrt{3}({\tt r}(x_l) - \theta_f(x_l))$
% \item a $t_{s_ew}$-interval then
% $z_{s_e \ast} + z_{e \ast} \leq \theta_f(x_l) + \sqrt{3}({\tt r}(x_l) - \theta_f(x_l))$
% \item a $t_{s_we}$-interval then
% $z_{w \ast} + z_{s_w \ast} \leq \theta_b(x_r) + \sqrt{3}({\tt r}(x_r) - \theta_b(x_r))$
% \item a $t_{en_w}$-interval then
% $z_{\ast n_w} + z_{\ast w} \leq \theta_b(x_r) + \sqrt{3}({\tt r}(x_r) - \theta_b(x_r))$
% \end{itemize}
%
\begin{itemize}
\item a $t_{wn_e}$-interval then
$z_{\ast n_e} + z_{\ast e} \leq {\tt r}(x_l) + (\sqrt{3} - 1)\delta_f(x_l)$
\item a $t_{s_ew}$-interval then
$z_{s_e \ast} + z_{e \ast} \leq {\tt r}(x_l) + (\sqrt{3} - 1)\delta_f(x_l)$
\item a $t_{s_we}$-interval then
$z_{w \ast} + z_{s_w \ast} \leq {\tt r}(x_r) + (\sqrt{3} - 1)\delta_b(x_r)$
\item a $t_{en_w}$-interval then
$z_{\ast n_w} + z_{\ast w} \leq {\tt r}(x_r) + (\sqrt{3} - 1)\delta_b(x_r)$
\end{itemize}
%
where wildcard notations $z_{\ast j}$ and $z_{i \ast}$ refer to the amount
of time spent in transitions $t_{\ast j}$ and $t_{i \ast}$, respectively.
\end{lemma}

\begin{proof}
We assume that $[x_l,x_r]$ is a $t_{wn_e}$-interval (the other cases follow
by symmetry). We also assume that $t(x_r) \not= t_{\ast n_w}$ implying that
$u(x_r)$ lies either on the N vertex,
side $n_e$, the NE vertex, or side $e$ of $H(x_r)$ (or, with some abuse of
the definition of $t(x)$, $t(x_r) = t_{\ast n_e},t_{\ast e})$.  If that is not the
case, we simply consider the
shorter interval $[x_l,x_q]$ where $x_q$ is such that $u(x_{q})$ is the N
vertex of $H(x_q)$ and $t(x) = t_{\ast n_w}$ for $x_q < x < x_r$.




Consider the path in $T_{1n}$ from $\ell(x_l)$ to $u(x_r)$ that starts with
the edge $(\ell(x_l), u(x_l)) = (l_i,u_i)$ and then visits the vertices
$u_i, u_{i+1}, \dots, u_{i+s} = u(x_r)$ in order
(refer to Fig.~\ref{fig:gentlepath}-(b)). 
Since $T_{1n}$ contains no gentle paths, the length of this path is at least
$\sqrt{3}d_x(l_i,u_{i+s}) - ({\tt y}(u_{i+s})-{\tt y}(l_i))$ and so:
\begin{align}
\label{eq:nogentle}
d_2(l_i,u_i)+ \sum_{t=i}^{i+s-1} d_2(u_t,u_{t+1}) + ({\tt y}(u_{i+s})-{\tt y}(l_i)) & \geq \sqrt{3}d_x(l_i,u_{i+s}) 
\end{align}
%
%
% The length of edge $(l_i, u_i)$ is at most the length of
% $[wu_i]$ where $w$ is the SW vertex of $H(x_l)$, and the length of
% $[wu_i]$ is at most $\frac{2{\tt r}(x_l)}{\cos(\Theta)}$, where $\Theta$ is the
% angle $\angle N(x_l) w u_i$. So:
% \begin{equation}
% \label{eq:crossedge}
% d_2(l_i,u_i) \leq \frac{2{\tt r}(x_l)}{\cos(\Theta)}  
% \end{equation}
%  Note that $0 \leq \Theta \leq \frac{\pi}{6}$. Note also that $d_y(l_i, N(x_l)) \leq \sqrt{3} {\tt r}(x_l)$ and so, by
% Lemma~\ref{lem:growthrates},
% \begin{align}
% \label{eq:crossvertical}
% {\tt y}(u_{i+s})-{\tt y}(l_i) & = ({\tt y}(N(x_l)) - {\tt y}(l_i)) + ({\tt y}(N(x_r)) - {\tt y}(N(x_l))) %\notag \\
%                    %& \quad 
% - ({\tt y}(N(x_r) - {\tt y}(u_{i+s}))  \notag \\
%                    & \leq \sqrt{3} {\tt r}(x_l) + 
%                           \left(\frac{1}{\sqrt{3}}z_{\ast n_w} - \frac{1}{\sqrt{3}}z_{\ast n_e} -
%                         \frac{3}{\sqrt{3}}z_{s_ee} - \frac{5}{\sqrt{3}}z_{s_we}\right) \notag \\ 
%                    & \quad     - \left(\frac{1}{\sqrt{3}}({\tt x}(u_{i+s})-x_r) + \max\{0,{\tt y}(v) - {\tt y}(u_{i+s})\}\right)
% \end{align}
%%%%%%
Note that ${\tt y}(N(x_l)) - {\tt y}(l_i) + p_S(l_i,x_l) = \frac{5}{\sqrt{3}} {\tt r}(x_l)$ and so, by
Lemma~\ref{lem:growthrates},
\begin{align}
\label{eq:crossvertical}
{\tt y}(u_{i+s})-{\tt y}(l_i) & = ({\tt y}(N(x_l)) - {\tt y}(l_i)) + ({\tt y}(N(x_r)) - {\tt y}(N(x_l))) %\notag \\
                   %& \quad 
- ({\tt y}(N(x_r) - {\tt y}(u_{i+s}))  \notag \\
		   & \leq \frac{5}{\sqrt{3}} {\tt r}(x_l) - p_S(l_i,x_l) + 
                          \left(\frac{1}{\sqrt{3}}z_{\ast n_w} - \frac{1}{\sqrt{3}}z_{\ast n_e} -
                        \frac{3}{\sqrt{3}}z_{s_ee} - \frac{5}{\sqrt{3}}z_{s_we}\right) \notag \\ 
                   & \quad     - \left(\frac{1}{\sqrt{3}}({\tt x}(u_{i+s})-x_r) + \max\{0,{\tt y}(v) - {\tt y}(u_{i+s})\}\right)
\end{align}
%%%%%%
where $v$ is the NE vertex of $H(x_r)$. The $\max$ term in (\ref{eq:crossvertical}) is 0 or
positive depending on whether $u_{i+s}$ is on
side $n_e$ or $e$, respectively, of $H(x_r)$. 
Each edge $(u_t,u_{t+1})$ on the path $u_i, u_{i+1}, \dots, u_{i+s}$
can be bounded by $p_N(u_t, {\tt x}(c_{t+1})) - p_N(u_{t+1},{\tt x}(c_{t+1}))$ and so:
\begin{align}
\label{eq:upperpath}
\sum_{t=i}^{i+s-1} d_2(u_t,u_{t+1}) & \leq \sum_{t=i}^{i+s-1} (p_N(u_t, {\tt x}(c_{t+1})) - p_N(u_{t+1},{\tt x}(c_{t+1}))) \notag \\
	& = (\bar{U}(x_r) - p_N(u_{i+s},x_r)) - (\bar{U}(x_l) - p_N(u_i,x_l)) \notag \\
	& = p_N(u_i,x_l) + (\bar{U}(x_r)- \bar{U}(x_l)) - p_N(u_{i+s},x_r)  \notag \\
	& = p_N(u_i,x_l) + \left(\frac{2}{\sqrt{3}}(z_{\ast n_w} + z_{\ast n_e}) + \frac{4}{\sqrt{3}}z_{s_ee} + \frac{6}{\sqrt{3}}z_{s_we}\right) \notag \\
   & \quad  + \left(\frac{2}{\sqrt{3}}({\tt x}(u_{i+s})-x_r) + \max\{0,{\tt y}(v) - {\tt y}(u_{i+s})\}\right)
\end{align}
Substituting the left-hand side of
(\ref{eq:nogentle}) with  (\ref{eq:crossvertical}) and 
(\ref{eq:upperpath}) gives us 
\begin{multline*}
d_2(l_i,u_i) + \frac{5}{\sqrt{3}} {\tt r}(x_l) - p_S(l_i,x_l) + p_N(u_i,x_l) +
\frac{3}{\sqrt{3}}z_{\ast n_w}+\frac{1}{\sqrt{3}}(z_{\ast n_e}+z_{\ast e}+{\tt x}(u_{i+s})-x_r)\\
\geq \frac{3}{\sqrt{3}}({\tt x}(u_{i+s})-{\tt x}(l_i)) 
= \frac{3}{\sqrt{3}}({\tt x}(u_{i+s})- x_l) + \frac{3}{\sqrt{3}}{\tt r}(x_l) 
\end{multline*}
where we use ${\tt x}(l_i) = x_l - {\tt r}(x_l)$. Using
$z_{\ast n_w} + z_{\ast e} + z_{\ast n_e} = x_r - x_l$ and ${\tt x}(u_{i+s}) \geq x_r$, and
applying Lemma~\ref{lem:switchpath},
we get 
\[
\frac{2}{\sqrt{3}} {\tt r}(x_l) + \left(2 - \frac{2}{\sqrt{3}}\right)\delta_f(x_l) \geq
\frac{2}{\sqrt{3}}(z_{\ast n_e}+z_{\ast e})
\]
which is equivalent to the lemma statement.
%We complete our proof by multiplying both sides by $\frac{\sqrt{3}}{2}$.
%
%To prove our
%claim, it therefore suffices to show the following inequality:
%\begin{align*}
%	0 & \leq \left(2 - \frac{\sqrt{3}}{\cos(\Theta)}\right){\tt r}(x_l) - (2 - \sqrt{3}) \delta_f(x_l) \\
%	& = \frac{\cos(\Theta) - 1}{\cos(\Theta)}\sqrt{3}{\tt r}(x_l) + (2 - \sqrt{3}) ({\tt r}(x_l) - \delta_f(x_l) ) \\
%	& = \frac{\cos(\Theta) - 1}{\cos(\Theta)}\frac{\sqrt{3}}{2}\cos(\Theta)|wu_i| + (2 - \sqrt{3}) \frac{\sqrt{3}}{2}\sin(\Theta)|wu_i|
%\end{align*}
%where the last equality follows from the right triangle $w N(x_l) u_i$ from
%Figure~\ref{fig:gentlepath}-(b).
%%
%Thus, it suffices to show $1 - \cos(\Theta) \leq (2 - \sqrt{3})\sin(\Theta)$
%for all $\Theta$ for $0 \leq \Theta \leq \frac{\pi}{6}$.  Since both sides are
%nonnegative, squaring both sides and using $\sin^2(\Theta) = 1 -\cos^2(\Theta) = (1 - \cos(\Theta)) (1 + \cos(\Theta))$
%this inequality boils down to $1 - \cos(\Theta) \leq (2 - \sqrt{3})^2(1 + \cos(\Theta))$.
%We know that the last inequality and thus our claim holds since $\cos(\Theta) \geq \frac{\sqrt{3}}{2}$
%for all $\Theta$ for $0 \leq \Theta \leq \frac{\pi}{6}$.
\end{proof}

The following two lemmas build on Lemmma~\ref{lem:gentlepath}. They show
different ways that the ``extra'' $\frac{2}{\sqrt{3}}$ growth (i.e., the
growth above $\frac{6}{\sqrt{3}}$) of a bad transition $t_{ij}$ within a
$t_{ij}$-interval can be spread (amortized) over a wider time interval (refer
also to Fig.~\ref{fig:linearworstcase}).

\begin{lemma}  
\label{lem:linearworstcase}
Given the assumptions of Lemma~\ref{lem:gentlepath}, if $[x_l,x_r]$ is a 
$t_{wn_e}$-, $t_{s_ew}$-, $t_{en_w}$-, or $t_{s_we}$-interval, if
$t(x) \not= t_{en_w}$, $t_{s_we}$, $t_{wn_e}$, or $t_{s_ew}$, respectively, when
$x_l \leq x \leq x_r$, and if 
$t(x_r)=t_{\ast e}$, $t_{e \ast}$, $t_{\ast w}$, or $t_{w \ast}$, respectively, then
%
% \begin{align}
% \label{eq:LWC}
% \frac{2}{\sqrt{3}}z_{wn_e}, \frac{2}{\sqrt{3}}z_{s_ew} & \leq (\frac{2}{\sqrt{3}}-1)(e(x_r)-w(x_l)-2\theta_f(x_l)) \\ \label{eq:LWC2}
% & \leq (\frac{4}{\sqrt{3}}-2)(x_r-w(x_l)-\theta_f(x_l))                      
% \end{align}
% %\begin{align*}
% \[\frac{2}{\sqrt{3}}z_{s_we}, \frac{2}{\sqrt{3}}z_{en_w} %& 
% \leq (\frac{2}{\sqrt{3}}-1)(e(x_r)-w(x_l)-2\theta_b(x_r)) 
% %\\ & 
% \leq (\frac{4}{\sqrt{3}}-2)(e(x_r)-x_l-\theta_b(x_r))                      
% %\end{align*}
% \]
%
\begin{align}
\label{eq:LWC}
\frac{2}{\sqrt{3}}z_{wn_e}, \frac{2}{\sqrt{3}}z_{s_ew} & \leq \left(\frac{2}{\sqrt{3}}-1\right)(e(x_r)-e(x_l)+2\delta_f(x_l)) \\
\label{eq:LWC2}
& \leq \left(\frac{4}{\sqrt{3}}-2\right)(x_r-x_l+\delta_f(x_l)) \\
%\end{align}
%\begin{align*}
%\[
\frac{2}{\sqrt{3}}z_{s_we}, \frac{2}{\sqrt{3}}z_{en_w} & 
\leq \left(\frac{2}{\sqrt{3}}-1\right)(w(x_r)-w(x_l)+2\delta_b(x_r)) \nonumber \\ 
&  \leq \left(\frac{4}{\sqrt{3}}-2\right)(x_r-x_l+\delta_b(x_r)), \nonumber 
\end{align}
respectively.
\end{lemma}

\begin{figure}[!b]
\linearworstcase

\caption{Illustration of Lemmas~\ref{lem:linearworstcase}
and~\ref{lem:softrecovery} for the case of a
$t_{wn_e}$-interval $[x_l,x_r]$. (a) Lemma~\ref{lem:linearworstcase}: the
$\frac{2}{\sqrt{3}}$ extra cost of the bad transition $t_{wn_e}$ taking place
within the interval $[x_l,x_r]$ can be amortized in one of two ways:
either as cost $\frac{1}{2}(\frac{4}{\sqrt{3}}-2)$ spread over the interval
$[e(x_l)-2\delta_f(x_l),e(x_r)]$ (shown in blue) or as cost
$\frac{4}{\sqrt{3}}-2$ spread over the interval $[x_l-\delta_f(x_l),x_r]$
(shown in red).
(b) 
Lemma~\ref{lem:softrecovery}: the $\frac{2}{\sqrt{3}}$ extra cost of the
bad transition $t_{wn_e}$ within the interval $[x_l,x_r]$ can be amortized as
cost $\frac{1}{2}(\frac{4}{\sqrt{3}}-2)$ spread over the interval
$[e(x_l)-2\delta_f(x_l),w(x_r)]$ (shown in blue) {\em plus} cost
$\frac{1}{\sqrt{3}}$ spread over intervals of time $z_{s_wn_w}$ and $z_{s_en_w}$
(contained within the dashed red interval).}
\label{fig:linearworstcase}
\end{figure}



\begin{proof}
We prove the lemma for the case of a $t_{wn_e}$-interval; the other 3 cases
follow by symmetry. By Lemma~\ref{lem:growthrates}, if
$[x_l,x_r]$ is a $t_{wn_e}$-interval satisfying the conditions of the lemma then
${\tt r}(x_l)+z_{wn_e}+z_{wn_w}+\frac{1}{2}z_{s_wn_w} = {\tt r}(x_r) + \frac{1}{2}z_{s_en_e}+z_{en_e}+z_{s_we} + z_{s_ee}$
which implies
\begin{equation}
\label{eq:radii}
\sqrt{3} z_{wn_e} - \frac{\sqrt{3}}{2}z_{s_en_e} \leq \sqrt{3}({\tt r}(x_r) - {\tt r}(x_l) + z_{en_e}+z_{s_we} + z_{s_ee})
\end{equation}
Since $t(x_r) = t_{\ast e}$, we have 
$t(x) \not= t_{\ast w}, t_{\ast n_w}$ for $x_r \leq x \leq e(x_r)$, and thus, the interval
$[x_l,e(x_r)]$ is a $t_{wn_e}$-interval such that $t(x) = t_{\ast n_e}$ or
$t(x) = t_{\ast e}$ when $x_r \leq x \leq e(x_r)$.  Therefore, 
by Lemma~\ref{lem:gentlepath} we have 
%
% ${\tt r}(x_r) + z_{\ast n_e} + z_{\ast e}+(\sqrt{3}-1)\theta_f(x_l) \leq \sqrt{3}{\tt r}(x_l)$
%
${\tt r}(x_r) + z_{\ast n_e} + z_{\ast e} \leq {\tt r}(x_l) + (\sqrt{3}-1)\delta_f(x_l)$
%
which implies
%
\[
z_{wn_e} + z_{s_en_e} \leq (\sqrt{3}-1)\delta_f(x_l) - ({\tt r}(x_r) - {\tt r}(x_l) + z_{en_e}+z_{s_we} + z_{s_ee})
	\]
%
together with inequality~(\ref{eq:radii}), we get
\[(\sqrt{3}+1)z_{wn_e} \leq (\sqrt{3}-1)({\tt r}(x_r) - {\tt r}(x_l) +z_{en_e}+z_{s_we}+z_{s_ee}+\delta_f(x_l))\]
and multiplying both sides by $(\sqrt{3} - 1)/\sqrt{3}$
\begin{equation}
\label{eq:radii2}
\frac{2}{\sqrt{3}}z_{wn_e} \leq \left(\frac{4}{\sqrt{3}}-2\right)({\tt r}(x_r) - {\tt r}(x_l) +z_{en_e}+z_{s_we}+z_{s_ee}+\delta_f(x_l))
\end{equation}
%
% The fact that $e(x_r) - w(x_l) = {\tt r}(x_l) + z_{\ast n_w}+z_{\ast e}+z_{\ast n_e} + {\tt r}(x_r)$ (refer to Fig.~\ref{fig:linearworstcase}-(a))
% together with~(\ref{eq:radii}) gives us
% ${\tt r}(x_r)+z_{en_e}+z_{s_we}+z_{s_ee} \leq \frac{1}{2}(e(x_r) - w(x_l))$ and
% so~(\ref{eq:radii2}) implies inequality~(\ref{eq:LWC}). Because
% $e(x_r)-w(x_l) \leq 2(x_r-w(x_l))$, inequality~(\ref{eq:LWC2}) follows. 
%
By Lemma~\ref{lem:growthrates}, we have
%
$z_{en_e}+z_{s_we}+z_{s_ee} \leq \frac{1}{2}(w(x_r) - w(x_l))$ and 
so~(\ref{eq:radii2}) implies inequality~(\ref{eq:LWC}).
%
Because $0 \leq w(x_r) - w(x_l)$ inequality~(\ref{eq:LWC}) implies
inequality~(\ref{eq:LWC2}). 
%
\end{proof}

\begin{lemma}
\label{lem:softrecovery}
(Refer to Fig.~\ref{fig:linearworstcase}-(b)) 
Given the assumptions of Lemma~\ref{lem:gentlepath}, 
if $[x_l,x_r]$ is a $t_{wn_e}$-, $t_{s_ew}$-, $t_{en_w}$-, or $t_{s_we}$-interval,
if $t(x) \not= t_{en_w}$, $t_{s_we}$, $t_{wn_e}$, or $t_{s_ew}$, respectively, when
$x_l \leq x \leq x_r$, and if $t(x_r) = t_{\ast w}$, $t_{w \ast}$, $t_{\ast e}$,
or $t_{e \ast}$, respectively, then
\begin{align}
& \quad \frac{2}{\sqrt{3}}z_{wn_e} - \frac{1}{\sqrt{3}}(z_{s_wn_w}+z_{s_en_w}), 
\frac{2}{\sqrt{3}}z_{s_ew} - \frac{1}{\sqrt{3}}(z_{s_wn_w}+z_{s_wn_e})  \nonumber \\
\leq & \quad \left(\frac{2}{\sqrt{3}}-1\right)(w(x_r) - e(x_l) + 2\delta_f(x_l)) \label{eq:softrecovery} \\
& \quad \frac{2}{\sqrt{3}}z_{en_w} - \frac{1}{\sqrt{3}}(z_{s_wn_e}+z_{s_en_e}),
\frac{2}{\sqrt{3}}z_{s_we} - \frac{1}{\sqrt{3}}(z_{s_en_w}+z_{s_en_e}) \nonumber \\
\leq & \quad \left(\frac{2}{\sqrt{3}}-1\right)(w(x_r) - e(x_l) + 2\delta_b(x_r)), \nonumber
\end{align}
respectively.
\end{lemma}

\begin{proof}
We prove the lemma for the case of a $t_{wn_e}$-interval only; the other 3 cases
can be seen to follow by symmetry. Then, 
since $[x_l,x_r]$ is a $t_{wn_e}$-interval and $t(x_r)=t_{\ast w}$, point $u(x_r)$
must be the NW point of $H(x_r)$ as shown in
Fig.~\ref{fig:linearworstcase}-(b). The point $u(x_r)$ must lie on side $n_w$ of 
$H(x)$ for $w(x_r) \leq x \leq x_r$ and so  $t(x) = t_{\ast n_w}$ when
$w(x_r) \leq x \leq x_r$. This, together with Lemma~\ref{lem:growthrates} on the growth of $w(\cdot)$
and the fact that $t(x) \not= t_{en_w}$ when $w(x_r) \leq x \leq x_r$,
implies that $w(x_r) - w(w(x_r)) = {\tt r}(w(x_r)) \leq \frac{1}{2}z_{s_wn_w}+z_{s_en_w} \leq z_{s_wn_w}+z_{s_en_w}$.
It also implies that all $t_{\ast n_e}$ and $t_{\ast e}$ transitions
in interval $[x_l,x_r]$ take place within interval $[x_l,w(x_r)]$ and so by
Lemma~\ref{lem:growthrates} on the growth of $e(\cdot)$,  $2z_{wn_e} \leq e(w(x_r)) - e(x_l)$.
Therefore, $2z_{wn_e}-(z_{s_wn_w}+z_{s_en_w}) \leq e(w(x_r))-{\tt r}(w(x_r))-e(x_l) = w(x_r)-e(x_l)$ and:
\begin{equation}
\label{eq:softrecovery2}
\left(\frac{4}{\sqrt{3}} - 2\right)z_{wn_e}-\left(\frac{2}{\sqrt{3}} - 1\right)(z_{s_wn_w}+z_{s_en_w}) \leq\
\left(\frac{2}{\sqrt{3}}-1\right)(w(x_r)-e(x_l))
\end{equation}


By Lemma~\ref{lem:gentlepath} we have
$z_{\ast e}+z_{\ast n_e} \leq {\tt r}(x_l)+(\sqrt{3}-1)\delta_f(x_l)$ and by
Lemma~\ref{lem:growthrates} on the growth rate of ${\tt r}(\cdot)$, we have 
${\tt r}(x_l)+z_{wn_e}-(z_{s_wn_e}+z_{s_en_e}+z_{en_e}+z_{s_we}+z_{s_ee}) \leq {\tt r}(w(x_r))$.
Using ${\tt r}(w(x_r)) \leq z_{s_wn_w}+z_{s_en_w}$ from above, we get
$2z_{wn_e} - (z_{s_wn_w}+z_{s_en_w}) \leq (\sqrt{3}-1)\delta_f(x_l)$
and multiplying both sides by $1 - \frac{1}{\sqrt{3}}$
\[\left(2-\frac{2}{\sqrt{3}}\right)z_{wn_e} - \left(1-\frac{1}{\sqrt{3}}\right)(z_{s_wn_w}+z_{s_en_w}) \leq
	\left(\frac{4}{\sqrt{3}}-2\right)\delta_f(x_l)\]
Summing~(\ref{eq:softrecovery2}) with this inequality,
we obtain~(\ref{eq:softrecovery}).
\end{proof}

We define a time $x$ in the interval $[{\tt x}(p), {\tt x}(q)]$ to be
{\em left critical} 
or {\em right critical} if $x$ is the left boundary of a maximal strict 
left interval or the right boundary of a maximal strict 
right interval, respectively. We also define the start and end
coordinates ${\tt x}(p)$ and ${\tt x}(q)$ to be both left and right critical. A
time $x$ is said  to be critical if it is left or right critical.
This lemma, illustrated in Fig.~\ref{fig:induction}, 
 bounds $P(x)$ by making use of amortization
Lemmas~\ref{lem:linearworstcase} and~\ref{lem:softrecovery}:
%inductively
%prove that the growth rate of $P(x)$ is at most $\frac{1n_w}{\sqrt{3}}-2$ on
%average:
%Let $P'(x)=P(x)-\frac{6}{\sqrt{3}}x$ and $\Delta P'(x_i,x_j)= P(x_i,x_j)-\frac{6}{\sqrt{3}}(x_j-x_i)$.  Then:

\begin{figure}
\induction

\caption{Illustration of Lemma~\ref{lem:criticalpoints}. (a) If $x$ is a left
critical point then $P(x)$ is no greater than
$(\frac{10}{\sqrt{3}}-2)x - (\frac{4}{\sqrt{3}} - 2)\delta_f(x)$.
In the case of a $t_{wn_e}$-interval starting at $x$, this allows some of the
extra $\frac{2}{\sqrt{3}}$ growth of bad transition $t_{wn_e}$ within the interval
to be charged to the ``past'' as cost $\frac{4}{\sqrt{3}}-2$ over the interval
$[x-\delta_f(x),x]$. 
(b) If $x$ is a right critical point then $P(x)$ is no greater than
$(\frac{10}{\sqrt{3}}-2)x + (\frac{4}{\sqrt{3}} - 2)\delta_b(x)$.
In the case of a $t_{en_w}$-interval ending at $x$, this means that some
of the extra $\frac{2}{\sqrt{3}}$ growth of bad transition $t_{en_w}$ within the
interval has been charged to the ``future'' as cost $\frac{4}{\sqrt{3}}-2$
over the interval $[x, x+\delta_b(x)]$.}
\label{fig:induction}
\end{figure}


\begin{lemma}
\label{lem:criticalpoints}
Given the assumptions of Lemma~\ref{le:mainlemmaB}:
\begin{itemize}
\item If $x$ is a left critical point then $P(x) \leq \left(\frac{10}{\sqrt{3}}-2\right)x - \left(\frac{4}{\sqrt{3}}-2\right)\delta_f(x)$
\item If $x$ is a right critical point then $P(x) \leq \left(\frac{10}{\sqrt{3}}-2\right)x + \left(\frac{4}{\sqrt{3}}-2\right)\delta_b(x)$
\end{itemize}
\end{lemma}


\begin{proof}
We proceed by induction, using  the left to right ordering of critical points.
The first critical point (i.e., the base case) is ${\tt x}(p)$. Since
$P(x) = 0 = {\tt x}(p) = \delta_f({\tt x}(p)) = \delta_b({\tt x}(p))$,
the claim holds. We assume now that the claim holds
for critical point $x_l$ and prove that it holds for the next critical point
$x_r$ (and so $x_l < x_r$ and no critical point exists between $x_l$ and $x_r$).

\noindent{\bf Case LL:} We first consider the case when $x_l$ and $x_r$ are both left critical points. We will show that in that case:
%	
\begin{align}
\label{eq:leftleft}
\Delta P(x_l,x_r) \leq \left(\frac{10}{\sqrt{3}}-2\right)(x_r - x_l) - \left(\frac{4}{\sqrt{3}}-2\right)
({\tt r}(x_r) - \delta_f(x_l))
\end{align}
Combining the above inequality with the inductive hypothesis for $P(x_l)$,
we obtain $P(x_r) = P(x_l) + \Delta P(x_l,x_r) \leq (\frac{10}{\sqrt{3}}-2)x_r
- (\frac{4}{\sqrt{3}}-2){\tt r}(x_r) \leq (\frac{10}{\sqrt{3}}-2)x_r
- (\frac{4}{\sqrt{3}}-2)\delta_f(x_r)$, as illustrated in Fig.~\ref{fig:proofA},
and complete the proof for this case.

We now show that~(\ref{eq:leftleft}) holds. 	
Both $x_l$ and $x_r$ being left critical implies that transitions $t_{s_we}$ and $t_{en_w}$ do not take place
within interval $[x_l, x_r]$. W.l.o.g. we assume that $x_l$ is the left
boundary of a maximal strict $t_{wn_e}$-interval. If $x_q$ is the right boundary
of this interval then $x_q \leq x_r$, $t(x_q) = t_{\ast w}$ or $t(x_q) = t_{s_ee}$,
and $t(x)$ is not bad when $x_q < x < x_r$. Let $z_{ij}$ be the time within
interval $[x_l,x_q]$ spent in transition $t_{ij}$, for every transition $t_{ij}$,
and let $z = \sum z_{ij} = x_q-x_l$.

If $t(x_q) = t_{\ast w}$ then by Lemma~\ref{lem:growthrates} we have:
\begin{align*}
\Delta P(x_l,x_r) & \leq \frac{8}{\sqrt{3}}z_{wn_e} + \frac{4}{\sqrt{3}}(z_{s_wn_w}+z_{s_en_w}) + \frac{6}{\sqrt{3}}(z - z_{wn_e} - z_{s_wn_w} - z_{s_en_w} + x_r - x_q) \\
                  & \leq \frac{6}{\sqrt{3}}(x_r - x_l) + \frac{2}{\sqrt{3}}z_{wn_e} - \frac{2}{\sqrt{3}}(z_{s_wn_w}+z_{s_en_w}) \\
                  & \leq \frac{6}{\sqrt{3}}(x_r - x_l) + \left(\frac{4}{\sqrt{3}}-2\right)(w(x_q)-x_l+\delta_f(x_l))
\end{align*}
The last inequality follows from Lemma~\ref{lem:softrecovery} and from 
$x_l < e(x_l)$ and $w(x_q)-x_l > {\tt x}(u(x_q)) - {\tt x}(u(x_l)) \geq 0$. 
Using $w(x_q) \leq w(x_r) = x_r - {\tt r}(x_r)$, we prove (\ref{eq:leftleft}).
\begin{figure}
\proofA
\caption{The proof of Lemma~\ref{lem:criticalpoints} in the case when $x_l$ and $x_r$ are both left critical and
$x_l$ is the left boundary of a $t_{wn_e}$-interval. By induction,
$P(x_l)$ is amortized as illustrated in Fig.~\ref{fig:induction}-(a) and shown
above in blue. We show that
$\Delta P(x_l,x_r)$ can be amortized as cost $\frac{10}{\sqrt{3}}-2$ over
interval $[x_l,w(x_r)]$ {\em plus} cost
$\frac{6}{\sqrt{3}}$ over interval $[w(x_r), x_r]$, shown in red.
This implies that $P(x_r)$ can be amortized as shown in Fig.~\ref{fig:induction}-(a).}
\label{fig:proofA}
\end{figure}

If $t(x_q) = t_{s_ee}$ then by Lemma~\ref{lem:growthrates} we have:

\begin{align*}
\Delta P(x_l,x_r) & \leq \frac{8}{\sqrt{3}}z_{wn_e} + \frac{6}{\sqrt{3}}(z - z_{wn_e} + x_r - x_q) \leq \frac{6}{\sqrt{3}}(x_r - x_l) + \frac{2}{\sqrt{3}}z_{wn_e} \\
                  & \leq \frac{6}{\sqrt{3}}(x_r - x_l) + \left(\frac{4}{\sqrt{3}}-2\right)(x_q-x_l+\delta_f(x_l))
\end{align*}
The last inequality follows from Lemma~\ref{lem:linearworstcase}.
Now, since $t(x_q) = t_{s_ee}$, then $x_q \leq b(x_q) \leq b(x_r) = w(x_r) = x_r - {\tt r}(x_r)$, we prove (\ref{eq:leftleft}) and complete the proof in this case as well.

\noindent{\bf Case RR:} The case when $x_l$ and $x_r$ are both right critical
points can be handled using an argument that is symmetric to the one we used for
case LL.
 
\noindent{\bf Case RL:} If $x_l$ is right critical and $x_r$ is left critical then no bad transitions
take place within the interval $[x_l,x_r]$ and
$\Delta P(x_l,x_r) \leq \frac{6}{\sqrt{3}}(x_r-x_l)$. If $x_l+\delta_b(x_l) \leq x_r-\delta_f(x_r)$ then by induction:
\begin{align*}
	P(x_r) & = P(x_l) + \Delta P(x_l,x_r) \\
               & \leq \left(\frac{10}{\sqrt{3}}-2\right)x_l+\left(\frac{4}{\sqrt{3}}-2\right)\delta_b(x_l)
	       + \frac{6}{\sqrt{3}}(x_r-x_l) \\
               & \leq \left(\frac{10}{\sqrt{3}}-2\right)x_r-\left(\frac{4}{\sqrt{3}}-2\right)\delta_f(x_r)
\end{align*}

If $x_l+\delta_b(x_l) > x_r-\delta_f(x_r)$, consider the interval
$I = [x_r-\delta_f(x_r), x_l+\delta_b(x_l)]$.
We argue that this interval
is contained within interval $[x_l, x_r]$ and that $\Delta P$ has a growth
rate of just $\frac{4}{\sqrt{3}}$ in interval $I$. To do
this, we assume w.l.o.g. that $t(x_l) = t_{en_w}$ with $u(x_l) = u_s$ and
$\ell(x_l) = l_s$, as illustrated in Fig.~\ref{fig:proofB}-(a). %Suppose that
%$u(x_l)$ is not the point on the W side of $H(x_r)$ (i.e., 
%$b(x_l) \not= b(x_r) = w(x_r)$). 
%Then either the point on the W side of $H(x_r)$
Since $x_r$ is left critical, one of $u(x_r)$ or $\ell(x_r)$ must lie on the
$w$ side of $H(x_r)$. However, the point on the $w$ side of $H(x_r)$ and 
having abscissa $w(x_r)$ cannot be a point $\ell(x_r) = l_t$ in $L$ 
(with $t \geq s$) because %, using Lemma~\ref{lem:bandf}-(b), 
we would have
$w(x_r) = {\tt x}(l_t) \geq {\tt x}(l_s) = e(x_l)$ which contradicts 
$x_l+\delta_b(x_l) > x_r-\delta_f(x_r)$.
%
Therefore, the point on the $w$ side of $H(x_r)$ must be 
a point $u_t \in U$ for some $t \geq s$ and $t(x) = t_{\ast n_w}$ for $x \in
[{\tt x}(u_t),x_r]$.  Furthermore, letting $x' = \max\{x_l, {\tt x}(u_t)\}$,
$t(x)$ must be $t_{wn_w}$, $t_{s_wn_w}$, or $t_{s_en_w}$ for $x$ in interval
$[x',x_r]$, and by
Lemma~\ref{lem:growthrates}, $\frac{\Delta {\tt r}(x)}{\Delta x} \geq 0$ for
$x$ in that interval. Thus, $x_r = w(x_r) + {\tt r}(x_r) \geq {\tt x}(u_t) + {\tt r }(x')$.
%
If $x' = x_l$ then $u_t = u_s$ as illustrated in Fig.~\ref{fig:proofB}-(a), and
then $x_r \geq {\tt x}(u_s) + {\tt r}(x_l) = x_l + \delta_b(x_l)$. 
%
If, on the other hand, $x' = {\tt x}(u_t)$ then $x_r \geq {\tt x}(u_t) + {\tt r}({\tt x}(u_t)) = e({\tt x}(u_t)) \geq e(x_l)$ %by Lemma~\ref{lem:bandf}-(d), 
and we still get $x_r \geq e(x_l) = x_l + {\tt r}(x_l) \geq x_l + \delta_b(x_l)$. 
%
%If $b(x_l) = b(x_r)$ (see Fig.~\ref{fig:proofB}) then $t(x) = t_{*n_w}$ for all
%$x$ in $[x_l,x_r]$ which, using 
%Lemma~\ref{lem:growthrates}, implies that $x_r - x_l > {\tt r}(x_l) - \theta_b(x_l)$,
%or $x_r > e(x_l) - \theta_b(x_l)$. 
A symmetric argument can be used to show
that $x_l \leq x_r-\delta_f(x_r)$ and therefore interval I is contained within
$[x_l,x_r]$.

\begin{figure}[!b]
\proofB

\caption{The proof of Lemma~\ref{lem:criticalpoints}. (a) The case when
$x_l$ is right critical, $x_r$ is left critical,
$x_l+\delta_b(x_l) > x_r-\delta_f(x_r)$, $t(x_l) = t_{en_w}$, and
$x' = \max\{x_l, {\tt x}(u_t)\} = x_l$. By induction,
$P(x_l)$ is amortized as illustrated in Fig.~\ref{fig:induction}-(b) and shown
above in blue. $\Delta P(x_l,x_r)$ can be amortized as cost 
$\frac{4}{\sqrt{3}}$ over interval 
$I=[x_r-\delta_f(x_r),x_l+\delta_b(x_l)]$ (shown in red) {\em plus} cost 
$\frac{6}{\sqrt{3}}$
over the interval $[x_l,x_r]$ not including $I$ (empty in the example). 
Thus $P(x_r)$ can be amortized as shown in Fig.~\ref{fig:induction}-(a).
(b) The case when $x_l$ is left critical and $x_r$ is right critical. By
induction, $P(x_l)$ is amortized as illustrated in Fig.~\ref{fig:induction}-(a)
and shown above in blue. $\Delta P(x_l,x_r)$ can be amortized as cost 
$\frac{10}{\sqrt{3}}-2$ over interval $[x_l,x_r]$ {\em plus} cost 
$\frac{4}{\sqrt{3}}-2$ over intervals $[x_l-\delta_f(x_l),x_l]$ and 
$[x_r,x_r+\delta_b(x_r)]$ (shown in red). Thus $P(x_r)$ can be amortized
as shown in Fig.~\ref{fig:induction}-(b).}
\label{fig:proofB}
\end{figure}


For every $x$ in interval $I$, %using Lemma~\ref{lem:bandf}-(e) 
we have
$x \geq w(x_r) = b(x_r) \geq b(x)$ and $x \leq e(x_l) \leq f(x_l) \leq f(x)$.
Given that no bad transitions take place
within $[x_l,x_r]$, $t(x)$ must be $t_{s_en_w}$ for $x \in I$ and therefore
$\Delta P$ has a growth rate bounded by $\frac{4}{\sqrt{3}}$ in $I$. Then,
as illustrated in Fig.~\ref{fig:proofB}-(a),
\begin{align*}
	P(x_r) & = P(x_l) + \Delta P(x_l,x_r) \\
	       & \leq \left(\frac{10}{\sqrt{3}}-2\right)x_l + \left(\frac{4}{\sqrt{3}}-2\right)\delta_b(x_l)
	       + \frac{4}{\sqrt{3}}|I| + \frac{6}{\sqrt{3}}(x_r-x_l-|I|) \\
	       & = \frac{6}{\sqrt{3}}x_r + \left(\frac{4}{\sqrt{3}}-2\right)(x_l + \delta_b(x_l))
	       - \frac{2}{\sqrt{3}}(x_l + \delta_b(x_l) - (x_r -\delta_f(x_r))) \\
	       & \leq \frac{6}{\sqrt{3}}x_r + \left(\frac{4}{\sqrt{3}}-2\right)(x_l + \delta_b(x_l))
	       - \left(\frac{4}{\sqrt{3}}-2\right)(x_l + \delta_b(x_l) - (x_r -\delta_f(x_r))) \\
               & = \left(\frac{10}{\sqrt{3}}-2\right)x_r - \left(\frac{4}{\sqrt{3}}-2\right)\delta_f(x_r)
\end{align*}

%Otherwise $f(x_i)> b(x_j)$.  Here it is still the case that there are no bad transitions in $[x_i,x_j]$ but now there are also good transitions.  Let $m=\max\{b(x_j),x_i\}$ and $M=\min\{f(x_i),x_j\}$.  Then $[m,M]$ is the largest region contained in both $[x_i,x_j]$ and in $[b(x_j),f(x_i)]$. For $x \in [b(x_j),f(x_i)]$,  $b(x)\leq b(f_j)<x$ and, $f(x) \geq f(x_i)>x$.  Since $[m,M]$ is contained in $[b(x_j),f(x_i)]$ and $[x_i,x_j]$, and since $[x_i,x_j]$ contains no bad transitions, every transition in $[m,M]$ is either $t_{s_en_w}$ or $t_{s_wn_e}$.  Since both of these transitions change the perimeter at a rate of $\frac{4}{\sqrt{3}}$  we have $\Delta P'(x_i,x_j) \leq -\frac{2}{\sqrt{3}} (M-m)$.  Using this we have:
%\[\begin{split}
% P'(x_j) \leq P'(x_i)+\Delta P'(x_i,x_j) &\leq (\frac{4}{\sqrt{3}}-2)(x_i+({\tt r}(x_i)-\theta_b(x_i))) -\frac{2}{\sqrt{3}} (M-m) \\
% &\leq (\frac{4}{\sqrt{3}}-2)(x_i+({\tt r}(x_i)-\theta_b(x_i))-(M-m)) \\
%							&= (\frac{4}{\sqrt{3}}-2)((m-\theta_b(x_i))+(f(x_i)-M))
%\end{split}\]

%Note that $\max\{b(x_j),x_i\}-\theta_b(x_i) \leq b(x_j)$ and that $f(x_i)-\min\{f(x_i),x_j\} \leq \theta_f(x_j)$.  Thus we have:
%\[\begin{split}
% P'(x_i)+\Delta P'(x_i,x_j) &\leq (\frac{4}{\sqrt{3}}-2)(b(x_j)+\theta_f(x_j))\\
% 							&= (\frac{4}{\sqrt{3}}-2)(x_j-({\tt r}(x_j)-\theta_f(x_j)))
%\end{split}\]

% % % % % % % % % % % % % % % % % % % % %ABOVE LATER % % % % % % % % %

\noindent{\bf Case LR:} 
Finally, we consider the case when $x_l$ is left critical and $x_r$ is right
critical. We will show that in this case:
\begin{equation}
\label{eq:keyineq0}
\Delta P(x_l,x_r) \leq \left(\frac{10}{\sqrt{3}}-2\right)(x_r - x_l) + \left(\frac{4}{\sqrt{3}}-2\right)(\delta_f(x_l) + \delta_b(x_r))
\end{equation}
Note that if this inequality holds then (refer to Fig.~\ref{fig:proofB}-(b)):
\begin{align*}
P(x_r) & = P(x_l) + \Delta P(x_l,x_r) \leq \left(\frac{10}{\sqrt{3}}-2\right)x_r + \left(\frac{4}{\sqrt{3}}-2\right)\delta_b(x_r)
\end{align*}

% % % % % % % % % % % % % % % % Start of Full section Lemma % % % % % % % %
Let $z_{ij}$ be the time within
interval $[x_l,x_r]$ spent in transition $t_{ij}$, for every transition $t_{ij}$,
and let $z = \sum z_{ij} = x_r-x_l$.
We assume w.l.o.g. that $t(x_l) = t_{wn_e}$. Note that either
$t(x_r) = t_{s_we}$ or $t(x_r) = t_{en_w}$.

\noindent{\bf Subcase LR.1:} We consider the case when
$t(x_r) = t_{s_we}$ first; this means that $t(x) \not= t_{s_ew}, t_{en_w}$
when $x \in [x_l, x_r]$. 
Either $[x_l,x_r]$ is a $t_{wn_e}$-interval with $t(x_r) = t_{s_we}$, and so
Lemma~\ref{lem:linearworstcase} applies to interval $[x_l,x_r]$, or
$[x_l,x_q]$ is a $t_{wn_e}$-interval for some $x_q < x_r$ with 
$t(x_q) = t_{s_ww}$,
and thus Lemma~\ref{lem:softrecovery} applies to interval $[x_l,x_q]$. In the
second case note that $t(x) \not= t_{wn_e}$ when $x \in [x_q, x_r]$ because, 
otherwise, there would be a left critical point between $x_l$ and $x_r$, 
a contradiction. This, together with $w(x_q) \leq w(x_r) < e(x_r)$,
%(by Lemma~\ref{lem:bandf}-(d)), 
gives:
\begin{equation}
\label{eq:keyineq}
\frac{2}{\sqrt{3}}z_{wn_e}-\frac{1}{\sqrt{3}}(z_{s_wn_w}+z_{s_en_w}) \leq \left(\frac{2}{\sqrt{3}}-1\right)(e(x_r)-e(x_l)+2\delta_f(x_l))
\end{equation}
Note that this inequality holds for the first case as well
(using Lemma~\ref{lem:linearworstcase}). Via symmetric arguments either $[x_l,x_r]$ is a
$t_{s_we}$-interval with $t(x_l) = t_{wn_e}$, and so Lemma~\ref{lem:linearworstcase}
applies to interval $[x_l,x_r]$, 
or $[x_{q'},x_r]$ is a $t_{s_we}$-interval for some $x_{q'} > x_l$ with
$t(x_{q'}) = t_{en_e}$, and thus Lemma~\ref{lem:softrecovery} applies to interval
$[x_{q'},x_r]$. Either way, the inequality
\[\frac{2}{\sqrt{3}}z_{s_we}-\frac{1}{\sqrt{3}}(z_{s_en_w}+z_{s_en_e})\leq \left(\frac{2}{\sqrt{3}}-1\right)(w(x_r)-w(x_l)+2\delta_b(x_r))\]
holds and it, together with inequality~(\ref{eq:keyineq}), gives:
\[\frac{2}{\sqrt{3}}(z_{wn_e} +z_{s_we})-\frac{2}{\sqrt{3}}(z_{s_wn_w}+z_{s_en_w}+z_{s_wn_e}+z_{s_en_e})\leq \left(\frac{4}{\sqrt{3}}-2\right)(x_r-x_l+\delta_b(x_r)+\delta_f(x_l))\]
We can now show that (\ref{eq:keyineq0}) holds:
\begin{align*} \Delta P(x_l, x_r) & \leq \frac{6}{\sqrt{3}}z + \frac{2}{\sqrt{3}}(z_{wn_e} +z_{s_we}) - \frac{2}{\sqrt{3}}(z_{s_wn_w}+z_{s_en_w}+z_{s_wn_e}+z_{s_en_e}) \\ 
	& \leq \frac{6}{\sqrt{3}}(x_r - x_l) + \left(\frac{4}{\sqrt{3}}-2\right)(x_r-x_l+\delta_b(x_r)+\delta_f(x_l)) \\
        & = \left(\frac{10}{\sqrt{3}}-2\right)(x_r - x_l) + \left(\frac{4}{\sqrt{3}}-2\right)(\delta_b(x_r) + \delta_f(x_l))
\end{align*}

\noindent{\bf Subcase LR.2:} We now consider the case when $t(x_l) = t_{wn_e}$ and $t(x_r) = t_{en_w}$. 
Note that this means that $t(x) \not= t_{s_ew}, t_{s_we}$ when $x \in [x_l, x_r]$. 
If 
\begin{equation}
\label{eq:keyineq2}
\frac{2}{\sqrt{3}}(z_{wn_e} +z_{en_w})-\frac{2}{\sqrt{3}}(z_{s_wn_w}+z_{s_en_w}+z_{s_wn_e}+z_{s_en_e})\leq
	\left(\frac{4}{\sqrt{3}}-2\right)(x_r-x_l+\delta_b(x_r)+\delta_f(x_l))
\end{equation}
then the above argument can be applied to obtain~(\ref{eq:keyineq0}). We will show next that if $t(x) = t_{\ast w}$ or $t(x) = t_{\ast e}$ for 
some $x \in [x_l, x_r]$ then inequality~(\ref{eq:keyineq2}) holds.

Consider the maximal strict $t_{wn_e}$-interval $[x_l,x_q]$ and the maximal
strict $t_{en_w}$-interval $[x_{q'},x_r]$. Recall that $t(x) \not= t_{en_w},t_{\ast w},t_{\ast e}$ for $x \in [x_l,x_q]$ and that $t(x) \not= t_{wn_e},t_{\ast w},t_{\ast e}$ for $x \in [x_{q'},x_r]$ and thus $x_l < x_q,x_{q'} < x_r$. 
If $t(x) = t_{\ast w}$ or $t(x) = t_{\ast e}$ 
for  some $x$ such that $x_l < x < x_r$, and using the assumption that there
are no critical points between $x_l$ and $x_r$, we must have that 
$x_l < x_q \leq x_{q'} < x_r$ and that $t(x) \not= t_{wn_e},t_{en_w}$ when
$x \in [x_q,x_{q'}]$. This also means that $t(x_q)$ is either $t_{s_ww}$ or 
$t_{s_ee}$ and that $t(x_{q'})$ is either $t_{s_ww}$ or $t_{s_ee}$. 

If $t(x_q)=t_{s_ww}$ then Lemma~\ref{lem:softrecovery} applies to interval
$[x_l,x_q]$. If $t(x_q) = t_{s_ee}$ then Lemma~\ref{lem:linearworstcase} applies
to interval $[x_l,x_q]$. Since $t(x) \not= t_{wn_e}$ when $x \in [x_q, x_r]$ 
and $w(x_q) \leq w(x_r) \leq e(x_r)$ and just as in case LR.1, it follows in 
both cases that
\begin{equation}
\label{eq:16}
\frac{2}{\sqrt{3}} z_{wn_e} - \frac{1}{\sqrt{3}}(z_{s_wn_w} + z_{s_en_w}) \leq \left(\frac{2}{\sqrt{3}} - 1\right) (e(x_r) - e(x_l) + 2 \delta_f(x_l))
\end{equation}
If $t(x_{q'})=t_{s_ee}$ then Lemma~\ref{lem:softrecovery}
applies to interval $[x_{q'},x_r]$. If $t(x_{q'}) = t_{s_ww}$ then Lemma~\ref{lem:linearworstcase} applies to interval $[x_{q'},x_r]$. Since $t(x) \not= t_{en_w}$ when $x \in [x_l, x_{q'}]$ and $w(x_l) \leq w(x_{q'}) \leq e(x_{q'})$ it follows 
in both cases that
\begin{equation}
\label{eq:17} 
\frac{2}{\sqrt{3}} z_{en_w} - \frac{1}{\sqrt{3}}(z_{s_wn_e} + z_{s_en_e}) \leq \left(\frac{2}{\sqrt{3}} - 1\right) (w(x_r) - w(x_l) + 2 \delta_b(x_r))
\end{equation}
By combining inequalities (\ref{eq:16}) and (\ref{eq:17}) we obtain inequality 
(\ref{eq:keyineq2}).

%%%%%%%%%%%%%
%%%%%%%%%%%%%
%Consider the maximal $t_{wn_e}$-interval $[x_l,x_q]$ and note that
%$x_q \leq x_r$. If $t(x_q) = t_{s_ww}$ then Lemma~\ref{lem:softrecovery}
%applies to interval $[x_l,x_q]$. Since $t(x) \not=t_{wn_e}$ 
%when $x_q < x < x_r$ (by definition of critical points) 
%and since $w(x_q) \leq w(x_r) \leq e(x_r)$ it follows that:
%\begin{equation}
%\frac{2}{\sqrt{3}}z_{wn_e} - \frac{1}{\sqrt{3}}(z_{s_wn_w}+z_{s_en_w}) \leq (\frac{2}{\sqrt{3}}-1)(e(x_r) - w(x_l) - 2\theta_f(x_l)).
%\end{equation}
%
%
%
%If $t(x_q) = t_{en_w}$ then $[x_l,x_q]$ is a $t_{wn_e}$-interval that
%satisfies the conditions of Lemma~\ref{lem:linearworstcase} and 
%
%Consider now the maximal $t_{en_w}$-interval $[x_{q'},x_r]$ and note that
%$x_l \leq x_{q'}$. If $t(x_{q'}) \not= t_{wn_e}$ then $t(x_{q'}) = t_{s_ee}$ 
%and Lemma~\ref{lem:softrecovery} applies to interval $[x_{q'},x_r]$. 
%Since $t(x) \not=t_{en_w}$ when $x_l < x < x_{q'}$ and $w(x_l) \leq w(x_{q'})$ and
%$w(x_r) \leq e(x_r)$ it follows that:
%\begin{equation}
%\frac{2}{\sqrt{3}}z_{en_w} - \frac{1}{\sqrt{3}}(z_{s_wn_e}+z_{s_en_e}) \leq (\frac{2}{\sqrt{3}}-1)(e(x_r) - w(x_l) - 2\theta_f(x_l))
%\end{equation}
%If $t(x_{q'}) = t_{wn_e}$ then $[x_{q'},x_r]$ is a $t_{en_w}$-interval that
%satisfies the conditions of Lemma~\ref{lem:linearworstcase} and 
%
%
%\begin{equation}
%\frac{2}{\sqrt{3}}z_{en_w} - \frac{1}{\sqrt{3}}(z_{s_wn_e}+z_{s_en_e}) \leq (\frac{2}{\sqrt{3}}-1)(e(x_r) - w(x_l) - 2\theta_f(x_l)).
%\end{equation}
%
%
%
%In summary, if $t(x) = t_{\ast w}$ for some $x$ such that $x_l < x < x_r$ then
%inequality~(\ref{eq:keyineq2}) holds. A symmetric argument can be used
%to show that the inequality also holds if $t(x) = t_{\ast e}$ for some $x$ such
%that $x_l < x < x_r$.
%%%%%%%%%%%%%%%%%%
%%%%%%%%%%%%%%%%%%

We assume now that $t(x) \not= t_{*w}, t_{*e}$ for all $x \in [x_l,x_r]$.
We can also assume that
inequality~(\ref{eq:keyineq2}) does not hold which implies
\[\frac{2}{\sqrt{3}}(z_{wn_e} +z_{en_w}) > \left(\frac{4}{\sqrt{3}}-2\right)(x_r - x_l + \delta_b(x_r)+\delta_f(x_l))
\]
which, using $z_{wn_e} +z_{en_w} \leq x_r - x_l$, implies:
\begin{equation}
\label{eq:keyineq3}
\left(4 - \frac{6}{\sqrt{3}}\right)(\delta_b(x_r)+\delta_f(x_l)) < \left(\frac{6}{\sqrt{3}}-2\right)(x_r - x_l)
\end{equation}

Now, $\Delta P(x_l,x_r) = U(x_r) - U(x_l) + L(x_r) - L(x_l)$. For all
$x \in [x_l,x_r]$, $t(x) = t_{*n_w}, t_{*n_e}$  and so, by
Lemma~\ref{lem:growthrates},
$\frac{\Delta p_N(x)}{\Delta x} = \frac{2}{\sqrt{3}}$. It follows that
$U(x_r) - U(x_l) \leq \frac{2}{\sqrt{3}}(x_r-x_l)$. In order to bound
$L(x_r) - L(x_l)$, we consider the following path $\mathcal{P}$ from point $\ell(x_l)$,
say $l_i$, that lies on side $w$ of $H(x_l)$ to point $\ell(x_r)$, say $l_j$
(where $i < j$), that lies on side $e$ of $H(x_r)$: 
$\ell(x_l)=l_i, u_i,u_{i+1},\dots,u_j,l_j=\ell(x_r)$ (refer to Fig.~\ref{fig:pathcost}).
Then, if $|\mathcal{P}|$ is the length of $\mathcal{P}$:
\begin{figure}
\center{\pathcost}

\caption{
The proof of Lemma~\ref{lem:criticalpoints}. Shown is the case when $x_l$ is
left critical, $x_r$ is right critical, $t(x_l) = t_{wn_e}$, $t(x_r)=t_{en_w}$,
and $t(x) \not= t_{\ast w}, t_{\ast e}$ for $x \in [x_l,x_r]$. The path
$\mathcal{P}$, defined as $l_i,u_i,u_{i+1},\dots,u_j,l_j$, is effectively
a shortcut for $l_i, l_{i+1}, \dots,l_j$.
% (a) Illustration of path $P$: $l_i,u_i,u_{i+1},\dots,u_j,l_j$.
% (b) $-p_S(l_i,x_l) + |l_iu_i|$ is maximized when $l_i$ is the SW corner
% of $H(x_l)$.
}
\label{fig:pathcost}
\end{figure}
\begin{align*}
L(x_r) - L(x_l) & = d_{T_{1j}}(p,l_j) + p_S(l_j,x_r) - d_{T_{1i}}(p,l_i) - p_S(l_i,x_l) \\
& \leq d_{T_{1i}}(p,l_i) + |\mathcal{P}| + p_S(l_j,x_r) - d_{T_{1i}}(p,l_i) -p_S(l_i,x_l) \\
& = -p_S(l_i,x_l) + |\mathcal{P}| + p_S(l_j,x_r) \\
& \leq -p_S(l_i,x_l) + d_2(l_i,u_i) + p_N(u_i,x_l) + U(x_r) - U(x_l) \\
& \quad - p_N(u_j,x_r) + d_2(l_j,u_j) + p_S(l_j,x_r)
\end{align*}
using the fact that the length of the path $u_i,u_{i+1},\dots,u_j$ is at most
$p_N(u_i,x_l) + U(x_r) - U(x_l) - p_N(u_j,x_r)$. 
%
% We note that $-p_S(l_i,x_l) + |l_iu_i|$ is maximized when $l_i$ is the SW corner
% of $H(x_l)$, as illustrated in Fig.~\ref{fig:pathcost}-(b). Assuming that is
% the case, let $\theta = \angle u_i l_i N(x_l)$ and note that
% $0 \leq \theta \leq \frac{\pi}{6}$. Then
% $p_S(l_i,x_l) = \frac{2}{\sqrt{3}}r(x)$,
% $|l_iu_i| = \frac{2}{\cos(\theta)}r(x)$, and
% $p_N(u_i,x_l) = -2\tan(\theta)r(x)$. Using
% $\frac{1}{\cos(\theta)} \leq 1 + (2-\sqrt{3})\tan(\theta)$ when
% $\theta \in [0,\frac{\pi}{6}]$, 
% we have
% % we can bound $-p_S(l_i,x_l) + |l_iu_i| + p_N(u_i,x_l)$ with
% \begin{align*}
% -p_S(l_i,x_l) + |l_iu_i| + p_N(u_i,x_l) & = \left(-\frac{2}{\sqrt{3}} + \frac{2}{\cos(\theta)} -2\tan(\theta)\right)r(x)\\
% & \leq \left(-\frac{2}{\sqrt{3}} + 2 + 2(2-\sqrt{3})\tan(\theta) -2\tan(\theta)\right)r(x) \\
% & = \left(2-\frac{2}{\sqrt{3}}\right)\left({\tt r}(x_l)-\sqrt{3}\tan(\theta){\tt r}(x_l)\right) \\
% & = \left(2-\frac{2}{\sqrt{3}}\right)\delta_f(x_l)
% \end{align*}
% Similarly, $-p_N(u_j,x_r) + |u_jl_j| + p_S(l_j,x_r) \leq \left(2-\frac{2}{\sqrt{3}}\right)\delta_b(x_r)$. Combining the bounds on
%
By Lemma~\ref{lem:switchpath},
$-p_S(l_i,x_l) + d_2(l_i,u_i) + p_N(u_i,x_l) \leq
 \left(2-\frac{2}{\sqrt{3}}\right)\delta_f(x_l)$ and
$-p_N(u_j,x_r) + d_2(l_j,u_j) + p_S(l_j,x_r) \leq \left(2-\frac{2}{\sqrt{3}}\right)\delta_b(x_r)$. Combining the bounds on
$U(x_r) - U(x_l)$ and $L(x_r) - L(x_l)$, we get
\[
\Delta P(x_l,x_r) \leq \left(2-\frac{2}{\sqrt{3}}\right)(\delta_b(x_r)+\delta_f(x_l))+\frac{4}{\sqrt{3}}(x_r-x_l)
\]
Summing the above inequality with inequality~(\ref{eq:keyineq3}) yields 
(\ref{eq:keyineq0}) and completes the inductive proof in this case as well.
\end{proof}

We can now provide a proof of (Amortization) Lemma~\ref{le:mainlemmaB}.
\begin{proof}[Proof of Lemma~\ref{le:mainlemmaB}]
Note that ${\tt x}(q)$ is a left critical point and that
$\delta_f({\tt x}(q)) = 0$.
	Therefore, by Lemma \ref{lem:criticalpoints},
$P({\tt x}(q)) \leq (\frac{10}{\sqrt{3}}-2){\tt x}(q)$. 
Since $2d_{T_{1n}}(p, q) = P({\tt x}(q))$, the lemma follows.
\end{proof}




