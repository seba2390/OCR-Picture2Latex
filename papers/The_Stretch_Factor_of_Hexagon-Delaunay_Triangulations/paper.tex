\documentclass[11pt,USenglish]{article}
%\documentclass[letterpaper,USenglish,numberwithinsect,cleveref, autoref]{socg-lipics-v2019}
%This is a template for producing LIPIcs articles. 
%See lipics-manual.pdf for further information.
%for A4 paper format use option "a4paper", for US-letter use option "letterpaper"
%for british hyphenation rules use option "UKenglish", for american hyphenation rules use option "USenglish"
%for section-numbered lemmas etc., use "numberwithinsect"
%for enabling cleveref support, use "cleveref"
%for enabling cleveref support, use "autoref"


%\graphicspath{{./graphics/}}%helpful if your graphic files are in another directory


\bibliographystyle{plainurl}% the mandatory bibstyle

%\usepackage{microtype}%if unwanted, comment out or use option "draft"

%%%%%%%%%%%%%%%%%%%%%%%
%%%%%%%%%%%%%%%%%%%%%%%
\usepackage{geometry}
\newgeometry{vmargin={1in}, hmargin={1in,1in}}  

\usepackage{tikz}
\usetikzlibrary{shapes,arrows,arrows.meta,shadows,positioning,matrix,patterns,decorations.markings,decorations.pathreplacing,svg.path,calc,angles,quotes} 
\usepackage{color}
\usepackage{amssymb}
\usepackage[integrals]{wasysym}
\usepackage{bm}
\usepackage{wrapfig}
\usepackage{amsthm}
\usepackage{amsmath}
\usepackage{amsfonts}
%\usepackage{xparse}
%\usepackage{comment}
\usepackage{tabu}
%\usepackage{wrapfig}
\usepackage{enumerate}
\usepackage[mathscr]{eucal}
\usepackage{caption}
\usepackage{authblk}
\usepackage{sectsty}

\usepackage{etoolbox}   % need for command \newtoggle
\newtoggle{abstract}    % flag to control whether output is abstract or full version
%\toggletrue{abstract}   % uncomment for abstract version
\togglefalse{abstract} % uncomment for full version


%\allsectionsfont{\normalfont\sffamily\bfseries}

\theoremstyle{plain}% default
\newtheorem{theorem}{Theorem}
\newtheorem{lemma}[theorem]{Lemma}
\newtheorem{proposition}[theorem]{Proposition}
\newtheorem{corollary}{Corollary}

\theoremstyle{definition}
\newtheorem{definition}[theorem]{Definition}

\def\cE{\mathscr{E}}
\newcommand{\sgn}{\operatorname{sgn}}

\def\eps{\varepsilon}
\def\cT{\frac{5}{\sqrt{3}}-1}
\def\cP{\frac{6}{\sqrt{3}}}
%%%%%%%%%%%%%%%%%%%%%%%%%%%%
%%%%%%%%%%%%%%%%%%%%%%%%%%%


\title{The Stretch Factor of Hexagon-Delaunay Triangulations}


\author[1]{Michael Dennis\thanks{michael\_dennis@cs.berkeley.edu}}

\author[2]{Ljubomir Perkovi\'{c}\thanks{lperkovic@cs.depaul.edu (corresponding author)}}

\author[2]{Duru T\"{u}rko\u{g}lu\thanks{duru@cs.uchicago.edu}}


\affil[1]{Computer Science Division, University of California at Berkeley, Berkeley, CA, USA}
\affil[2]{School of Computing, DePaul University, Chicago, IL, USA}

\date{}

% U triangle.
\def\Utriangle{
\m{\begin{qcircuit}[scale=0.39]
    \grid{2.8}{0,3.2,4.8}
    \utriangle{1.4}{0}{4.8}{1.6,4.8}
\end{qcircuit}}
=
\m{\begin{qcircuit}[scale=0.39]
    \Period{7.6}{0}
    \grid{7.2}{0,3.2,4.8}
    \Ugate{n_1}{1.4}{4.8}{3.2}
    \colgate{white, white}{$\bcdots$}{3.68,4.8}
    \colgate{white, white}{$\bcdots$}{3.68,3.2}
    \colgate{white, white}{$\bddots$}{3.68,1.6}
    \colgate{white, white}{$\bcdots$}{3.68,0}
    \Ugate{n_k}{5.8}{4.8}{0}
\end{qcircuit}}
}

% V trapezoid.
\def\Vtrapezoid{
\m{\begin{qcircuit}[scale=0.39]
    \grid{2.8}{0,3.2,4.8,6.4}
    \vtrapezoid{1.4}{6.4}{4.8}{0}{2,1}
\end{qcircuit}}
=
\m{\begin{qcircuit}[scale=0.39]
    \grid{7.2}{0,3.2,4.8,6.4}
    \Vgate{n_1}{1.4}{3.2}{6.4}{4.8}
    \colgate{white, white}{$\bcdots$}{3.68,6.4}
    \colgate{white, white}{$\bcdots$}{3.68,4.8}
    \colgate{white, white}{$\bcdots$}{3.68,3.2}
    \colgate{white, white}{$\bddots$}{3.68,1.6}
    \colgate{white, white}{$\bcdots$}{3.68,0}
    \Vgate{n_k}{5.8}{4.8}{6.4}{0}
\end{qcircuit}}
}

% V triangle.
\def\Vtriangle{
\m{\begin{qcircuit}[scale=0.39]
    \grid{2.8}{0,1.6,4.8,6.4}
    \vtriangle{1.4}{0}{6.4}{1.6,6.4}
\end{qcircuit}}
=
\m{\begin{qcircuit}[scale=0.39]
    \Period{7.6}{0}
    \grid{7.2}{0,1.6,4.8,6.4}
    \vtrapezoid{1.4}{6.4}{4.8}{0}{2,1}
    \colgate{white, white}{$\bcdots$}{3.68,6.4}
    \colgate{white, white}{$\bcdots$}{3.68,4.8}
    \colgate{white, white}{$\bddots$}{3.68,3.2}
    \colgate{white, white}{$\bcdots$}{3.68,1.6}
    \colgate{white, white}{$\bcdots$}{3.68,0}
    \vtrapezoid{5.8}{6.4}{1.6}{0}{2,1}
\end{qcircuit}}
}

% Stair.
\def\stair{
\m{\begin{qcircuit}[scale=0.39]
    \grid{5}{0,1.6,3.2,6.4,8}
    \colgate{white, white}{$\bvdots$}{0.8,4.8}
    \colgate{white, white}{$\bvdots$}{4.2,4.8}
    \staircasealt{1.4}{0}{3.6}{8}{2.5,4}
\end{qcircuit}}
=
\m{\begin{qcircuit}[scale=0.39]
    \Period{16.4}{0}
    \grid{16}{0,1.6,3.2,6.4,8}
    \colgate{white, white}{$\bvdots$}{1.4,4.8}
    \colgate{white, white}{$\bcdots$}{10.2,8}
    \colgate{white, white}{$\bcdots$}{10.2,6.4}
    \colgate{white, white}{$\bcdots$}{10.2,4.8}
    \colgate{white, white}{$\biddots$}{10.2,4.8}
    \colgate{white, white}{$\bcdots$}{10.2,3.2}
    \colgate{white, white}{$\bcdots$}{10.2,1.6}
    \colgate{white, white}{$\bcdots$}{10.2,0}
    \Swap{1.4}{0}{1.6}
    \CX{n_1}{3.6}{1.6}{0}
    \Swap{5.8}{1.6}{3.2}
    \CX{n_2}{8}{3.2}{1.6}
    \Swap{12.4}{8}{6.4}
    \CX{n_k}{14.6}{8}{6.4}
    \colgate{white, white}{$\bvdots$}{14.6,4.8}
\end{qcircuit}} 
}

% Ladder.
\def\ladder{
\m{\begin{qcircuit}[scale=0.39]
    \grid{2.8}{0,3.2,6.4}
    \colgate{white, white}{$\bvdots$}{0.4,1.6}
    \colgate{white, white}{$\bvdots$}{0.4,4.8}
    \colgate{white, white}{$\bvdots$}{2.4,1.6}
    \colgate{white, white}{$\bvdots$}{2.4,4.8}
    \updownalt{1.4}{0}{1.4}{6.4}{1.4}{0}{1.4,0.9}
\end{qcircuit}}
=
\m{\begin{qcircuit}[scale=0.39]
    \Period{8.7}{0}
    \grid{8.3}{0,3.2,6.4}
    \colgate{white, white}{$\bvdots$}{0.4,1.6}
    \colgate{white, white}{$\bvdots$}{0.4,4.8}
    \staircasealt{1.4}{0}{3.6}{6.4}{2.5,3.2}
    \colgate{white, white}{$\bvdots$}{7.9,1.6}
    \colgate{white, white}{$\bvdots$}{7.9,4.8}
    \staircasealt{6.9}{3.2}{5.8}{6.4}{6.3,4.8}
\end{qcircuit}}
}

% Affine normal form.
\def\nfaff{
\m{\begin{qcircuit}[scale=0.39]
    \grid{11.6}{0,3.2,4.8}
    \updownalt{1.4}{4.8}{1.4}{4.8}{1.4}{4.8}{1.37,4.65}
    \updownalt{3.6}{3.2}{3.6}{4.8}{3.6}{3.2}{3.55,3.2}
    \colgate{white, white}{$\bcdots$}{5.8,4.8}
    \colgate{white, white}{$\bcdots$}{5.8,3.2}
    \colgate{white, white}{$\bddots$}{5.8,1.6}
    \colgate{white, white}{$\bcdots$}{5.8,0}
    \updownalt{8}{0}{8}{4.8}{8}{0}{8,0.8}
    \Xblock{}{10.2}{4.8}
    \Xblock{}{10.2}{3.2}
    \Xblock{}{10.2}{0}
    \colgate{white, white}{$\bvdots$}{10.2,1.6}
  \end{qcircuit}}
}

% Affine example.
\def\exampleAff{
\m{\begin{qcircuit}[scale=0.39]
    \Period{21}{0}
    \grid{20.6}{0,1.6,3.2}
    \Swap{1.4}{3.2}{1.6}
    \Swap{3.6}{0}{1.6}
    \Swap{5.8}{1.6}{3.2}
    \CX{}{8}{3.2}{1.6}
    \Swap{10.2}{1.6}{3.2}
    \CX{}{12.4}{3.2}{1.6}
    \Swap{14.6}{0}{1.6}
    \CX{}{16.8}{1.6}{0}
    \Xgate{}{19}{3.2}
\end{qcircuit}} 
}

% Diagonal normal form.
\def\nfdiag{
\m{\begin{qcircuit}[scale=0.39]
    \Omeg{k}{-1}{2.4} 
    \grid{20.4}{0,3.2,4.8}
    \Tblock{}{1.4}{4.8}
    \colgate{white, white}{$\bvdots$}{1.4,1.6}
    \Tblock{}{1.4}{3.2}
    \Tblock{}{1.4}{0}
    \utriangle{3.6}{0}{4.8}{3.75,4.8}
    \utriangle{5.8}{0}{3.2}{5.95,3.2}
    \colgate{white, white}{$\bcdots$}{8.08,4.8}
    \colgate{white, white}{$\bddots$}{8.08,1.6}
    \colgate{white, white}{$\bcdots$}{8.08,3.2}
    \colgate{white, white}{$\bcdots$}{8.08,0}
    \utriangle{10.2}{0}{0}{10.65,0.2}
    \vtriangle{12.4}{0}{4.8}{12.55,4.8}
    \vtriangle{14.6}{0}{3.2}{14.75,3.2}
    \colgate{white, white}{$\bcdots$}{16.88,4.8}
    \colgate{white, white}{$\bddots$}{16.88,1.6}
    \colgate{white, white}{$\bcdots$}{16.88,3.2}
    \colgate{white, white}{$\bcdots$}{16.88,0}
    \vtriangle{19}{0}{0}{19.45,0.2}
  \end{qcircuit}}
}

% Diagonal example.
\def\nfCCZ{
\m{\begin{qcircuit}[scale=0.39]
    \grid{2.8}{0,1.6,3.2}
    \Ccz{1.4}{0}{1.6}{3.2}
\end{qcircuit}}
=
\m{\begin{qcircuit}[scale=0.39]
    \Period{12}{0}
    \grid{11.6}{0,1.6,3.2}
    \Tgate{}{1.4}{0}
    \Tgate{}{1.4}{1.6}
    \Tgate{}{1.4}{3.2}
    \Ugate{3}{3.6}{1.6}{3.2}
    \Ugate{3}{5.8}{0}{3.2}
    \Ugate{3}{8}{0}{1.6}
    \Vgate{}{10.2}{0}{1.6}{3.2}
\end{qcircuit}}
}

% Bifunctoriality.
\def\bifunctoriality{
\m{\begin{qcircuit}[scale=0.39]
    \grid{5}{0,1.6}
    \colgate{white!20}{$f$}{1.4,1.6}
    \colgate{white!20}{$g$}{3.6,0}
\end{qcircuit}}
=
\m{\begin{qcircuit}[scale=0.39]
    \Period{5.3}{0}
    \grid{4.9}{0,1.6}
    \colgate{white!20}{$g$}{1.4,0}
    \colgate{white!20}{$f$}{3.6,1.6}
\end{qcircuit}}
}

% Symmetries.
\def\symmetries{
\m{\begin{qcircuit}[scale=0.39]
    \grid{2.8}{0,1.6}
    \Swap{1.4}{0}{1.6}
\end{qcircuit}}
\qquad\mbox{ and }\qquad
\m{\begin{qcircuit}[scale=0.39] 
    \grid{2.8}{0,1.6,3.2}
    \cleargate{$T$}{1.4,0}
    \cleargate{$T$}{1.4,1.6}
    \cleargate{$T$}{1.4,3.2} 
    \diagwire{1.4}{0}{3.2} 
    \diagwire{1.4}{3.2}{1.6} 
    \diagwire{1.4}{1.6}{0}
\end{qcircuit}}
}

% Symmetries (slides).
\def\symmetriesslides{
\m{\begin{qcircuit}[scale=0.39]
    \grid{2.8}{0,1.6}
    \Swap{1.4}{0}{1.6}
\end{qcircuit}}
\qquad\mbox{ and }\qquad
\m{\begin{qcircuit}[scale=0.39] 
    \Period{3.2}{0}
    \grid{2.8}{0,1.6,3.2}
    \cleargate{$T$}{1.4,0}
    \cleargate{$T$}{1.4,1.6}
    \cleargate{$T$}{1.4,3.2} 
    \diagwire{1.4}{0}{3.2} 
    \diagwire{1.4}{3.2}{1.6} 
    \diagwire{1.4}{1.6}{0}
\end{qcircuit}}
}

% Naturality.
\def\Naturality{
\m{\begin{qcircuit}[scale=0.39]
    \grid{5}{0,1.6}
    \colgate{white!20}{$f$}{1.4,1.6}
    \colgate{white!20}{$g$}{1.4,0}
    \Swap{3.6}{0}{1.6}
\end{qcircuit}}
=
\m{\begin{qcircuit}[scale=0.39]
    \grid{5}{0,1.6}
    \colgate{white!20}{$g$}{3.6,1.6}
    \colgate{white!20}{$f$}{3.6,0}
    \Swap{1.4}{0}{1.6}
\end{qcircuit}}
\qquad\mbox{ and }\qquad
\m{\begin{qcircuit}[scale=0.39] 
    \grid{5}{0,1.6,3.2}
    \widebigcolgate{$h$}{1.4,1.6}{1.4,3.2}{.8}{white}
    \cleargate{$T$}{3.6,0}
    \cleargate{$T$}{3.6,1.6}
    \cleargate{$T$}{3.6,3.2} 
    \diagwire{3.6}{0}{3.2} 
    \diagwire{3.6}{3.2}{1.6} 
    \diagwire{3.6}{1.6}{0}
    \colgate{white!20}{$f$}{1.4,0}
\end{qcircuit}}
=
\m{\begin{qcircuit}[scale=0.39] 
    \Period{5.4}{0}
    \grid{5}{0,1.6,3.2}
    \colgate{white!20}{$f$}{3.6,3.2}
    \widebigcolgate{$h$}{3.6,0}{3.6,1.6}{.8}{white}
    \cleargate{$T$}{1.4,0}
    \cleargate{$T$}{1.4,1.6}
    \cleargate{$T$}{1.4,3.2} 
    \diagwire{1.4}{0}{3.2} 
    \diagwire{1.4}{3.2}{1.6} 
    \diagwire{1.4}{1.6}{0}
\end{qcircuit}}
}

% Spatiality.
\def\spatiality{ 
\m{\begin{qcircuit}[scale=0.39]
    \grid{2.8}{0}
    \colgate{white!20}{$\lambda$}{1.4,1.6}
\end{qcircuit}}
=
\m{\begin{qcircuit}[scale=0.39]
    \grid{2.8}{1.6}
    \colgate{white!20}{$\lambda$}{1.4,0}
\end{qcircuit}}
}

% Coherence.
\def\coherence{
\m{\begin{qcircuit}[scale=0.39]
    \grid{5}{0,1.6,3.2}
    \Swap{1.4}{0}{1.6}
    \Swap{3.6}{1.6}{3.2}
\end{qcircuit}}
=
\m{\begin{qcircuit}[scale=0.39] 
    \Period{3.2}{0}
    \grid{2.8}{0,1.6,3.2}
    \cleargate{$T$}{1.4,0}
    \cleargate{$T$}{1.4,1.6}
    \cleargate{$T$}{1.4,3.2} 
    \diagwire{1.4}{0}{3.2} 
    \diagwire{1.4}{3.2}{1.6} 
    \diagwire{1.4}{1.6}{0}
\end{qcircuit}}
}

% Arbitrary block.
\def\block{
\m{\begin{qcircuit}[scale=0.39] \grid{2.8}{0}
    \colgate{white!20}{$f^n$}{1.4,0}
\end{qcircuit}} 
=
\mp{0.75}{\begin{qcircuit}[scale=0.39] 
    \colgate{white, white}{$\bdot$}{7.5,0} 
    \colgate{white, white}{$\bcdots$}{3.67,0} 
    \grid{2.8}{0} 
    \gridx{4.4}{7.2}{0} 
    \colgate{white!20}{$f$}{1.4,0} 
    \colgate{white!20}{$f$}{5.8,0} 
    \brace{$n$}{0.6}{6.6}{-1}
\end{qcircuit}}
}

% Non-adjacent.
\def\NonAdjacent{
\m{\begin{qcircuit}[scale=0.39] 
    \grid{2.8}{0,1.6,3.2}
    \controlwires{1.4,0}{3.2}
    \colgate{white}{$f$}{1.4,0} 
    \colgate{white}{$f$}{1.4,3.2}
\end{qcircuit}} 
= 
\m{\begin{qcircuit}[scale=0.39]
    \Period{7.6}{0}
    \grid{7.2}{0,1.6,3.2} 
    \Swap{1.4}{1.6}{0}
    \controlwires{3.6,1.6}{3.2}
    \colgate{white}{$f$}{3.6,1.6} 
    \colgate{white}{$f$}{3.6,3.2}
    \Swap{5.8}{1.6}{0}
\end{qcircuit}}
}

% Derivation commutation.
\def\DerivationOne{
\begin{align*}
    \m{\begin{qcircuit}[scale=0.39]
        \grid{5}{0,1.6}
        \Swap{1.4}{0}{1.6}
        \Ugate{}{3.6}{0}{1.6}
      \end{qcircuit}}
    &= \m{\begin{qcircuit}[scale=0.39]
        \grid{18.2}{0,1.6}
        \CX{}{1.4}{1.6}{0}
        \Swap{3.6}{1.6}{0}
        \CX{}{5.8}{1.6}{0}
        \Swap{8}{1.6}{0}
        \CX{}{10.2}{1.6}{0}
        \CX{}{12.4}{1.6}{0}
        \Tgate{}{14.6}{0}
        \CX{}{16.8}{1.6}{0}
      \end{qcircuit}} \\
    &= \m{\begin{qcircuit}[scale=0.39]
        \grid{13.8}{0,1.6}
        \CX{}{1.4}{1.6}{0}
        \Swap{3.6}{1.6}{0}
        \CX{}{5.8}{1.6}{0}
        \Swap{8}{1.6}{0}
        \Tgate{}{10.2}{0}
        \CX{}{12.4}{1.6}{0}
      \end{qcircuit}} \\
    &= \m{\begin{qcircuit}[scale=0.39]
        \grid{13.8}{0,1.6}
        \CX{}{1.4}{1.6}{0}
        \Swap{3.6}{1.6}{0}
        \CX{}{5.8}{1.6}{0}
        \Tgate{}{8}{1.6}
        \Swap{10.2}{1.6}{0}
        \CX{}{12.4}{1.6}{0}
      \end{qcircuit}} \\
    &= \m{\begin{qcircuit}[scale=0.39]
        \grid{13.8}{0,1.6}
        \CX{}{1.4}{1.6}{0}
        \Swap{3.6}{1.6}{0}
        \Tgate{}{5.8}{1.6}
        \CX{}{8}{1.6}{0}
        \Swap{10.2}{1.6}{0}
        \CX{}{12.4}{1.6}{0}
      \end{qcircuit}} \\
    &= \m{\begin{qcircuit}[scale=0.39]
        \grid{13.8}{0,1.6}
        \CX{}{1.4}{1.6}{0}
        \Tgate{}{3.6}{0}
        \Swap{5.8}{1.6}{0}
        \CX{}{8}{1.6}{0}
        \Swap{10.2}{1.6}{0}
        \CX{}{12.4}{1.6}{0}
      \end{qcircuit}} \\
    &= \m{\begin{qcircuit}[scale=0.39]
        \grid{18.2}{0,1.6}
        \CX{}{1.4}{1.6}{0}
        \Tgate{}{3.6}{0}
        \CX{}{5.8}{1.6}{0}
        \CX{}{8}{1.6}{0}
        \Swap{10.2}{1.6}{0}
        \CX{}{12.4}{1.6}{0}
        \Swap{14.6}{1.6}{0}
        \CX{}{16.8}{1.6}{0}
      \end{qcircuit}}
     = \m{\begin{qcircuit}[scale=0.39]
        \Period{5.4}{0}
        \grid{5}{0,1.6}
        \Ugate{}{1.4}{0}{1.6}
        \Swap{3.6}{0}{1.6}
      \end{qcircuit}}
\end{align*}
}

% Diagonal commutations.
\def\FourCases{
\m{\begin{qcircuit}[scale=0.39]
    \grid{5}{0,1.6}
    \Tgate{}{1.4}{1.6}
    \Ugate{}{3.6}{0}{1.6}
  \end{qcircuit}}
= \m{\begin{qcircuit}[scale=0.39]
    \grid{5}{0,1.6}
    \Ugate{}{1.4}{0}{1.6}
    \Tgate{}{3.6}{1.6}
  \end{qcircuit}} \quad
  \m{\begin{qcircuit}[scale=0.39]
    \grid{5}{0,1.6,3.2}
    \Tgate{}{1.4}{3.2}
    \Vgate{}{3.6}{0}{1.6}{3.2}
  \end{qcircuit}}
= \m{\begin{qcircuit}[scale=0.39]
    \grid{5}{0,1.6,3.2}
    \Vgate{}{1.4}{0}{1.6}{3.2}
    \Tgate{}{3.6}{3.2}
  \end{qcircuit}} \quad
  \m{\begin{qcircuit}[scale=0.39]
    \grid{5}{0,1.6,3.2}
    \Ugate{}{1.4}{1.6}{3.2}
    \Vgate{}{3.6}{0}{1.6}{3.2}
  \end{qcircuit}}
= \m{\begin{qcircuit}[scale=0.39]
    \grid{5}{0,1.6,3.2}
    \Vgate{}{1.4}{0}{1.6}{3.2}
    \Ugate{}{3.6}{1.6}{3.2}
  \end{qcircuit}} \quad
  \m{\begin{qcircuit}[scale=0.39]
    \grid{5}{0,1.6,3.2,4.8}
    \Ugate{}{1.4}{3.2}{4.8}
    \Vgate{}{3.6}{0}{1.6}{3.2}
  \end{qcircuit}}
= \m{\begin{qcircuit}[scale=0.39]
    \grid{5}{0,1.6,3.2,4.8}
    \Vgate{}{1.4}{0}{1.6}{3.2}
    \Ugate{}{3.6}{3.2}{4.8}
  \end{qcircuit}}
}

% Derivation diagonal commutations.
\def\DerivationTwo{
\begin{align*}
\m{\begin{qcircuit}[scale=0.39]
    \grid{13.8}{0,1.6,3.2,4.8}
    \Ugate{}{1.4}{3.2}{4.8}
    \CX{}{3.6}{3.2}{1.6}
    \CX{}{5.8}{1.6}{0}
    \Tgate{}{8.0}{0}
    \CX{}{10.2}{1.6}{0}
    \CX{}{12.4}{3.2}{1.6}
\end{qcircuit}}
=
\m{\begin{qcircuit}[scale=0.39]
    \grid{13.8}{0,1.6,3.2,4.8}
    \CX{}{1.4}{3.2}{1.6}
    \CX{}{3.6}{1.6}{0}
    \Tgate{}{5.8}{0}
    \CX{}{8.0}{1.6}{0}
    \CX{}{10.2}{3.2}{1.6}
    \Ugate{}{5.8}{3.2}{4.8}
\end{qcircuit}}
=
\m{\begin{qcircuit}[scale=0.39]
    \Period{14.2}{0}
    \grid{13.8}{0,1.6,3.2,4.8}
    \CX{}{1.4}{3.2}{1.6}
    \CX{}{3.6}{1.6}{0}
    \Tgate{}{5.8}{0}
    \CX{}{8.0}{1.6}{0}
    \CX{}{10.2}{3.2}{1.6}
    \Ugate{}{12.4}{3.2}{4.8}
\end{qcircuit}}
\end{align*}
}


\begin{document}

%\definecolor{qqwuqq}{rgb}{0.,0.39215686274509803,0.}
%\definecolor{wwzzff}{rgb}{0.4,0.6,1.}

\maketitle

\begin{abstract}
The problem of computing the exact stretch factor (i.e., the tight bound on the
worst case stretch factor) of a Delaunay triangulation is one of the
longstanding open problems in computational geometry. Over the
years, a series of upper and lower bounds on the exact stretch factor have
been obtained but the gap between them is still large. An alternative approach
to solving the problem is to develop techniques for computing the exact stretch
factor of ``easier'' types of Delaunay triangulations, in particular those
defined using regular-polygons instead
of a circle. Tight bounds exist for Delaunay triangulations defined using
an equilateral triangle~\cite{Chew89} and a square~\cite{BGHP15}. In this paper,
we determine the exact stretch factor of Delaunay triangulations defined using
a regular hexagon: It is $2$.

We think that the main contribution of this paper are the two techniques we
have developed to compute tight upper bounds for the stretch factor of
Hexagon-Delaunay triangulations.% The two techniques correspond to the two
%different lower bound construtions which are equivalent to the two tightest
%known lower bound for traditional Delaunay triangulations.  
 %We think that the techniques
%we
%have developed in this paper will prove useful in future work on
%computing the exact stretch factor of classical Delaunay triangulations and
%other plane spanners.
\end{abstract}




%\clearpage
%\setcounter{page}{1}


\section{Introduction}
\label{sec:intro}
% \leavevmode
% \\
% \\
% \\
% \\
% \\
\section{Introduction}
\label{introduction}

AutoML is the process by which machine learning models are built automatically for a new dataset. Given a dataset, AutoML systems perform a search over valid data transformations and learners, along with hyper-parameter optimization for each learner~\cite{VolcanoML}. Choosing the transformations and learners over which to search is our focus.
A significant number of systems mine from prior runs of pipelines over a set of datasets to choose transformers and learners that are effective with different types of datasets (e.g. \cite{NEURIPS2018_b59a51a3}, \cite{10.14778/3415478.3415542}, \cite{autosklearn}). Thus, they build a database by actually running different pipelines with a diverse set of datasets to estimate the accuracy of potential pipelines. Hence, they can be used to effectively reduce the search space. A new dataset, based on a set of features (meta-features) is then matched to this database to find the most plausible candidates for both learner selection and hyper-parameter tuning. This process of choosing starting points in the search space is called meta-learning for the cold start problem.  

Other meta-learning approaches include mining existing data science code and their associated datasets to learn from human expertise. The AL~\cite{al} system mined existing Kaggle notebooks using dynamic analysis, i.e., actually running the scripts, and showed that such a system has promise.  However, this meta-learning approach does not scale because it is onerous to execute a large number of pipeline scripts on datasets, preprocessing datasets is never trivial, and older scripts cease to run at all as software evolves. It is not surprising that AL therefore performed dynamic analysis on just nine datasets.

Our system, {\sysname}, provides a scalable meta-learning approach to leverage human expertise, using static analysis to mine pipelines from large repositories of scripts. Static analysis has the advantage of scaling to thousands or millions of scripts \cite{graph4code} easily, but lacks the performance data gathered by dynamic analysis. The {\sysname} meta-learning approach guides the learning process by a scalable dataset similarity search, based on dataset embeddings, to find the most similar datasets and the semantics of ML pipelines applied on them.  Many existing systems, such as Auto-Sklearn \cite{autosklearn} and AL \cite{al}, compute a set of meta-features for each dataset. We developed a deep neural network model to generate embeddings at the granularity of a dataset, e.g., a table or CSV file, to capture similarity at the level of an entire dataset rather than relying on a set of meta-features.
 
Because we use static analysis to capture the semantics of the meta-learning process, we have no mechanism to choose the \textbf{best} pipeline from many seen pipelines, unlike the dynamic execution case where one can rely on runtime to choose the best performing pipeline.  Observing that pipelines are basically workflow graphs, we use graph generator neural models to succinctly capture the statically-observed pipelines for a single dataset. In {\sysname}, we formulate learner selection as a graph generation problem to predict optimized pipelines based on pipelines seen in actual notebooks.

%. This formulation enables {\sysname} for effective pruning of the AutoML search space to predict optimized pipelines based on pipelines seen in actual notebooks.}
%We note that increasingly, state-of-the-art performance in AutoML systems is being generated by more complex pipelines such as Directed Acyclic Graphs (DAGs) \cite{piper} rather than the linear pipelines used in earlier systems.  
 
{\sysname} does learner and transformation selection, and hence is a component of an AutoML systems. To evaluate this component, we integrated it into two existing AutoML systems, FLAML \cite{flaml} and Auto-Sklearn \cite{autosklearn}.  
% We evaluate each system with and without {\sysname}.  
We chose FLAML because it does not yet have any meta-learning component for the cold start problem and instead allows user selection of learners and transformers. The authors of FLAML explicitly pointed to the fact that FLAML might benefit from a meta-learning component and pointed to it as a possibility for future work. For FLAML, if mining historical pipelines provides an advantage, we should improve its performance. We also picked Auto-Sklearn as it does have a learner selection component based on meta-features, as described earlier~\cite{autosklearn2}. For Auto-Sklearn, we should at least match performance if our static mining of pipelines can match their extensive database. For context, we also compared {\sysname} with the recent VolcanoML~\cite{VolcanoML}, which provides an efficient decomposition and execution strategy for the AutoML search space. In contrast, {\sysname} prunes the search space using our meta-learning model to perform hyperparameter optimization only for the most promising candidates. 

The contributions of this paper are the following:
\begin{itemize}
    \item Section ~\ref{sec:mining} defines a scalable meta-learning approach based on representation learning of mined ML pipeline semantics and datasets for over 100 datasets and ~11K Python scripts.  
    \newline
    \item Sections~\ref{sec:kgpipGen} formulates AutoML pipeline generation as a graph generation problem. {\sysname} predicts efficiently an optimized ML pipeline for an unseen dataset based on our meta-learning model.  To the best of our knowledge, {\sysname} is the first approach to formulate  AutoML pipeline generation in such a way.
    \newline
    \item Section~\ref{sec:eval} presents a comprehensive evaluation using a large collection of 121 datasets from major AutoML benchmarks and Kaggle. Our experimental results show that {\sysname} outperforms all existing AutoML systems and achieves state-of-the-art results on the majority of these datasets. {\sysname} significantly improves the performance of both FLAML and Auto-Sklearn in classification and regression tasks. We also outperformed AL in 75 out of 77 datasets and VolcanoML in 75  out of 121 datasets, including 44 datasets used only by VolcanoML~\cite{VolcanoML}.  On average, {\sysname} achieves scores that are statistically better than the means of all other systems. 
\end{itemize}


%This approach does not need to apply cleaning or transformation methods to handle different variances among datasets. Moreover, we do not need to deal with complex analysis, such as dynamic code analysis. Thus, our approach proved to be scalable, as discussed in Sections~\ref{sec:mining}.

\section{Preliminaries}
\label{sec:prelim}
%!TEX root = hopfwright.tex
%

In this section we systematically recast the Hopf bifurcation problem in Fourier space. 
We introduce appropriate scalings, sequence spaces of Fourier coefficients and convenient operators on these spaces. 
To study Equation~\eqref{eq:FourierSequenceEquation} we consider Fourier sequences $ \{a_k\}$ and fix a Banach space in which these sequences reside. It is indispensable for our analysis that this space have an algebraic structure. 
The Wiener algebra of absolutely summable Fourier series is a natural candidate, which we use with minor modifications. 
In numerical applications, weighted sequence spaces with algebraic and geometric decay have been used to great effect to study periodic solutions which are $C^k$ and analytic, respectively~\cite{lessard2010recent,hungria2016rigorous}. 
Although it follows from Lemma~\ref{l:analytic} that the Fourier coefficients of any solution decay exponentially, we choose to work in a space of less regularity. 
The reason is that by working in a space with less regularity, we are better able to connect our results with the global estimates in \cite{neumaier2014global}, see Theorem~\ref{thm:UniqunessNbd2}.


%
%
%\begin{remark}
%	Although it follows from Lemma~\ref{l:analytic} that the Fourier coefficients of any solution decay exponentially, we choose to work in a space of less regularity, namely summable Fourier coefficients. This allows us to draw SOME MORE INTERESTING CONCLUSION LATER.
%	EXPLAIN WHY WE CHOOSE A NORM WITH ALMOST NO DECAY!
%	% of s Periodic solutions to Wright's equation are known to be real analytic and so their  Fourier coefficients must decay geometrically [Nussbaum].
%	% We do not use such a strong result;  any periodic solution must be continuously differentiable, by which it follows that $ \sum | c_k| < \infty$.
%\end{remark}


\begin{remark}\label{r:a0}
There is considerable redundancy in Equation~\eqref{eq:FourierSequenceEquation}. First, since we are considering real-valued solutions $y$, we assume $\c_{-k}$ is the complex conjugate of $\c_k$. This symmetry implies it suffices to consider Equation~\eqref{eq:FourierSequenceEquation} for $k \geq 0$.
Second, we may effectively ignore the zeroth Fourier coefficient of any periodic solution \cite{jones1962existence}, since it is necessarily equal to $0$. 
%In \cite{jones1962existence}, it is shown that if $y \not\equiv -1$ is a periodic solution of~\eqref{eq:Wright} with frequency $\omega$, then $ \int_0^{2\pi/\omega} y(t) dt =0$. 
		The self contained argument is as follows. 
		As mentioned in the introduction, any periodic solution to Wright's equation must satisfy $ y(t) > -1$ for all $t$. 
	By dividing Equation~\eqref{eq:Wright} by $(1+y(t))$, which never vanishes, we obtain
	\[
	\frac{d}{dt} \log (1 + y(t)) = - \alpha y(t-1).
	\]  
	Integrating over one period $L$ we derive the condition 
	$0=\int_0^L y(t) dt $.
	Hence $a_0=0$ for any periodic solution. 
	It will be shown in Theorem~\ref{thm:FourierEquivalence1} that a related argument implies that we do not need to consider Equation~\eqref{eq:FourierSequenceEquation} for $k=0$.
\end{remark}

%%%
%%%
%%%\begin{remark}\label{r:c0} 
%%%In \cite{jones1962existence}, it is shown that if $y \not\equiv -1$ is a periodic solution of~\eqref{eq:Wright} with frequency $\omega$, then $ \int_0^{2\pi/\omega} y(t) dt =0$. 
%%%PERHAPS TOO MUCH DETAIL HERE. The self contained argument is as follows.
%%%If $y \not\equiv -1$ then $y(t) \neq -1$ for all $t$, since if $y(t_0)=-1$ for some $t_0 \in \R$ then $y'(t_0)=0$ by~\eqref{eq:Wright} and in fact by differentiating~\eqref{eq:Wright} repeatedly one obtains that all derivatives of $y$ vanish at $t_0$. Hence $y \equiv -1$ by Lemma~\ref{l:analytic}, a contradiction. Now divide~\eqref{eq:Wright} by $(1+y(t))$, which never vanishes, to obtain
%%%\[
%%%  \frac{d}{dt} \log |1 + y(t)| = - \alpha y(t-1).
%%%\]  
%%%Integrating over one period we obtain $\int_0^L y(t) dt =0$.
%%%\end{remark}



%Furthermore, the condition that $y(t)$ is real forces $\c_{-k} = \overline{\c}_{k}$.  
%
We define the spaces of absolutely summable Fourier series
\begin{alignat*}{1}
	\ell^1 &:= \left\{ \{ \c_k \}_{k \geq 1} : 
    \sum_{k \geq 1} | \c_k| < \infty  \right\} , \\
	\ell^1_\bi &:= \left\{ \{ \c_k \}_{k \in \Z} : 
    \sum_{k \in \Z} | \c_k| < \infty  \right\} .
\end{alignat*} 
We identify any semi-infinite sequence $ \{ \c_k \}_{k \geq 1} \in \ell^1$ with the bi-infinite sequence $ \{ \c_k \}_{k \in \Z} \in \ell^1_\bi$ via the conventions (see Remark~\ref{r:a0})
\begin{equation}
  \c_0=0 \qquad\text{ and }\qquad \c_{-k} = \c_{k}^*. 
\end{equation}
In other word, we identify $\ell^1$ with the set
\begin{equation*}
   \ell^1_\sym := \left\{ \c \in \ell^1_\bi : 
	\c_0=0,~\c_{-k}=\c_k^* \right\} .
\end{equation*}
On $\ell^1$ we introduce the norm
\begin{equation}\label{e:lnorm}
  \| \c \| = \| \c \|_{\ell^1} := 2 \sum_{k = 1}^\infty |\c_k|.
\end{equation}
The factor $2$ in this norm is chosen to have a Banach algebra estimate.
Indeed, for $\c, \tilde{\c} \in \ell^1 \cong \ell^1_\sym$ we define
the discrete convolution 
\[
\left[ \c * \tilde{\c} \right]_k = \sum_{\substack{k_1,k_2\in\Z\\ k_1 + k_2 = k}} \c_{k_1} \tilde{\c}_{k_2} .
\]
Although $[\c*\tilde{\c}]_0$ does not necessarily vanish, we have $\{\c*\tilde{\c}\}_{k \geq 1} \in \ell^1 $ and 
\begin{equation*}
	\| \c*\tilde{\c} \| \leq \| \c \| \cdot  \| \tilde{\c} \| 
	\qquad\text{for all } \c , \tilde{\c} \in \ell^1, 
\end{equation*}
hence $\ell^1$ with norm~\eqref{e:lnorm} is a Banach algebra.

By Lemma~\ref{l:analytic} it is clear that any periodic solution of~\eqref{eq:Wright} has a well-defined Fourier series $\c \in \ell^1_\bi$. 
The next theorem shows that in order to study periodic orbits to Wright's equation we only need to study Equation~\eqref{eq:FourierSequenceEquation} 
for $k \geq 1$. For convenience we introduce the notation 
\[
G(\alpha,\omega,\c)_k=
( i \omega k + \alpha e^{ - i \omega k}) \c_k + \alpha \sum_{k_1 + k_2 = k} e^{- i \omega k_1} \c_{k_1} \c_{k_2} \qquad \text{for } k \in \N.
\]
We note that we may interpret the trivial solution $y(t)\equiv 0$ as a periodic solution of arbitrary period.
\begin{theorem}
\label{thm:FourierEquivalence1}
Let $\alpha>0$ and $\omega>0$.
If $\c \in \ell^1 \cong \ell^1_{\sym}$ solves
$G(\alpha,\omega,\c)_k =0$  for all $k \geq 1$,
then $y(t)$ given by~\eqref{eq:FourierEquation} is a periodic solution of~\eqref{eq:Wright} with period~$2\pi/\omega$.
Vice versa, if $y(t)$ is a periodic solution of~\eqref{eq:Wright} with period~$2\pi/\omega$ then its Fourier coefficients $\c \in \ell^1_\bi$ lie in $\ell^1_\sym \cong \ell^1$ and solve $G(\alpha,\omega,\c)_k =0$ for all $k \geq 1$.
\end{theorem}

\begin{proof}	
	If $y(t)$ is a periodic solution of~\eqref{eq:Wright} then it is real analytic by Lemma~\ref{l:analytic}, hence its Fourier series $\c$ is well-defined and $\c \in \ell^1_{\sym}$ by Remark~\ref{r:a0}.
	Plugging the Fourier series~\eqref{eq:FourierEquation} into~\eqref{eq:Wright} one easily derives that $\c$ solves~\eqref{eq:FourierSequenceEquation} for all $k \geq 1$.

To prove the reverse implication, assume that $\c \in \ell^1_\sym$ solves
Equation~\eqref{eq:FourierSequenceEquation} for all $k \geq 1$. Since $\c_{-k}
= \c_k^*$, Equation \eqref{eq:FourierSequenceEquation} is also satisfied for
all $k \leq -1$. It follows from the Banach algebra property and
\eqref{eq:FourierSequenceEquation} that $\{k \c_k\}_{k \in \Z} \in \ell^1_\bi$,
hence $y$, given by~\eqref{eq:FourierEquation}, is continuously differentiable.
% (and by bootstrapping one infers that $\{k^m c_k \} \in \ell^1_\bi$, 
% hence $y \in C^m$ for any $m \geq 1$).
	Since~\eqref{eq:FourierSequenceEquation} is satisfied for all $k \in \Z \setminus \{0\}$ (but not necessarily for $k=0$) one may perform the inverse Fourier transform on~\eqref{eq:FourierSequenceEquation} to conclude that
	$y$ satisfies the delay equation 
\begin{equation}\label{eq:delaywithK}
   	y'(t) = - \alpha y(t-1) [ 1 + y(t)] + C
\end{equation}
	for some constant $C \in \R$. 
   Finally, to prove that $C=0$ we argue by contradiction.
   Suppose $C \neq 0$. Then $y(t) \neq -1$ for all $t$.
   Namely, at any point where $y(t_0) =-1$ one would have $y'(t_0) = C$
   which has fixed sign,   hence it would follow that $y$ is not periodic
   ($y$ would not be able to cross $-1$ in the opposite direction, 
   preventing $y$  from being periodic).  
  We may thus divide~\eqref{eq:delaywithK} through by $1 + y(t)$ and obtain 
\begin{equation*}
	\frac{d}{dt} \log | 1 + y(t) | = - \alpha y(t-1) + \frac{C}{1+y(t)} .
\end{equation*}
	By integrating both sides of the equation over one period $L$ and by using that $\c_0=0$, we 
	obtain
	\[
	 C \int_0^L \frac{1}{1+y(t)} dt =0.
	\]
	Since the integrand is either strictly negative or strictly positive, this implies that $C=0$. Hence~\eqref{eq:delaywithK} reduces to~\eqref{eq:Wright},
	and $y$ satisfies Wright's equation. 
\end{proof}






To efficiently study Equation~\eqref{eq:FourierSequenceEquation}, we introduce the following linear operators on $ \ell^1$:
\begin{alignat*}{1}
   [K \c ]_k &:= k^{-1} \c_k  , \\ 
   [ U_\omega \c ]_k &:= e^{-i k \omega} \c_k  .
\end{alignat*}
The map $K$ is a compact operator, and it has a densely defined inverse $K^{-1}$. The domain of $K^{-1}$ is denoted by
\[
  \ell^K := \{ \c \in \ell^1 : K^{-1} \c \in \ell^1 \}.  
\]
The map $U_{\omega}$ is a unitary operator on $\ell^1$, but
it is discontinuous in $\omega$. 
With this notation, Theorem~\ref{thm:FourierEquivalence1} implies that our problem of finding a SOPS to~\eqref{eq:Wright} is equivalent to finding an $\c \in \ell^1$ such that
\begin{equation}
\label{e:defG}
  G(\alpha,\omega,\c) :=
  \left( i \omega K^{-1} + \alpha U_\omega \right) \c + \alpha \left[U_\omega \, \c \right] * \c  = 0.
\end{equation}


%In order for the solutions of Equation \ref{eq:FHat} to be isolated we need to impose a phase condition. 
%If there is a sequence $ \{ c_k \} $ which satisfies  Equation \ref{eq:FHat}, then $ y( t + \tau) = \sum_{k \in \Z} c_k e^{ i k \omega (t + \tau)}$ satisfies Wright's equation at parameter $\alpha$. 
%Fix $ \tau = - Arg[c_1] / \omega$ so that $ c_1  e^{ i \omega \tau} $ is a nonnegative real number. 
%By Proposition \ref{thm:FourierEquivalence1} it follows that $\{ c'_k \} =  \{c_k e^{ i \omega k \tau }   \}$ is a solution to Equation \ref{eq:FHat}, and furthermore that $ c'_1 = \epsilon$ for some $ \epsilon \geq 0$. 


Periodic solutions are invariant under time translation: if $y(t)$ solves Wright's equation, then so does $ y(t+\tau)$ for any $\tau \in \R$. 
We remove this degeneracy by adding a phase condition. 
Without loss of generality, if $\c \in \ell^1$ solves Equation~\eqref{e:defG}, we may assume that $\c_1 = \epsilon$ for some 
\emph{real non-negative}~$\epsilon$:
\[
  \ell^1_{\epsilon} := \{\c \in \ell^1 : \c_1 = \epsilon \} 
  \qquad \text{where } \epsilon \in \R,  \epsilon \geq 0.
\]
In the rest of our analysis, we will split elements $\c \in \ell^1$ into two parts: $\c_1$ and $\{\c_{k}\}_{k \geq 2}$.  
We define the basis elements $\e_j \in \ell^1$ for $j=1,2,\dots$ as
\[
  [\e_j]_k = \begin{cases}
  1 & \text{if } k=j, \\
  0 & \text{if } k \neq j.
  \end{cases}
\]
We note that $\| \e_j \|=2$. 
Then we can decompose
% We define
% \[
%   \tilde{\epsilon} := (\epsilon,0,0,0,\dots) \in \ell^1
% \]
% and
% For clarity when referring to sequences $\{c_{k}\}_{k \geq 2}$, we make the following definition:
% \[
% \ell^1_0  := \{ \tc \in \ell^1 : \tc_1 = 0 \}.
% \]
% With the
any $\c \in \ell^1_\epsilon$ uniquely as
\begin{equation}\label{e:aepsc}
  \c= \epsilon \e_1 + \tc \qquad \text{with}\quad 
  \tc \in \ell^1_0 := \{ \tc \in \ell^1 : \tc_1 = 0 \}.
\end{equation}
We follow the classical approach in studying Hopf bifurcations and consider 
$\c_1 = \epsilon$ to be a parameter, and then find periodic solutions with Fourier modes in $\ell^1_{\epsilon}$.
This approach rewrites the function $G: \R^2 \times \ell^K \to \ell^1$ as a function $\tilde{F}_\epsilon : \R^2 \times \ell^K_0 \to \ell^1$, where 
we denote 
\[
\ell^K_0 := \ell^1_0 \cap \ell^K.
\]
% I AM ACTUALLY NOT SURE IF YOU WANT TO DEFINE THIS WITH RANGE IN $\ell^1$
% OR WITH DOMAIN IN $\ell^1_0$ ?? IT SEEMS TO DEPEND ON WHICH GLOBAL STATEMENT YOU WANT/NEED TO MAKE!?
\begin{definition}
We define the $\epsilon$-parameterized family of  functions $\tilde{F}_\epsilon: \R^2 \times \ell^K_0  \to \ell^1$ 
by 
\begin{equation}
\label{eq:fourieroperators}
\tilde{F}_{\epsilon}(\alpha,\omega, \tc) := 
\epsilon [i \omega + \alpha e^{-i \omega}] \e_1 + 
( i \omega K^{-1} + \alpha U_{\omega}) \tc + 
\epsilon^2 \alpha e^{-i \omega}  \e_2  +
\alpha \epsilon L_\omega \tc + 
\alpha  [ U_{\omega} \tc] * \tc ,
\end{equation}
where
$L_\omega : \ell^1_0 \to \ell^1$ is given by
\[
   L_{\omega} := \sigma^+( e^{- i \omega} I + U_{\omega}) + \sigma^-(e^{i \omega} I + U_{\omega}),
\]
with $I$ the identity and  $\sigma^\pm$ the shift operators on $\ell^1$:
\begin{alignat*}{2}
\left[ \sigma^- a \right]_k &:=  a_{k+1}  , \\
\left[ \sigma^+ a \right]_k &:=  a_{k-1}  &\qquad&\text{with the convention } \c_0=0.
\end{alignat*}
The operator $ L_\omega$ is discontinuous in $\omega$ and $ \| L_\omega \| \leq 4$. 
\end{definition} 

%The maps $ \sigma^{+}$ and $ \sigma^-$ are shift up and shift down operators respectively. 
We reformulate Theorem~\ref{thm:FourierEquivalence1}  in terms of the map  $\tilde{F}$. 
We note that it follows from Lemma~\ref{l:analytic} and 
%\marginpar{Reformulate}
%one's choice of  
Equation~\eqref{eq:FourierSequenceEquation}  
%or Equation ~\eqref{eq:fourieroperators},
that the Fourier coefficients of any periodic solution of~\eqref{eq:Wright} lie in $\ell^K$.
These observations are summarized in the following theorem.
\begin{theorem}
\label{thm:FourierEquivalence2}
	Let $ \epsilon \geq 0$,  $\tc \in \ell^K_0$, $\alpha>0$ and $ \omega >0$. 
	Define $y: \R\to \R$ as 
\begin{equation}\label{e:ytc}
	y(t) = 
	\epsilon \left( e^{i \omega t }  + e^{- i \omega t }\right) 
	+  \sum_{k = 2}^\infty   \tc_k e^{i \omega k t }  + \tc_k^* e^{- i \omega k t } .
\end{equation}
%	and suppose that $ y(t) > -1$. 
	Then $y(t)$ solves~\eqref{eq:Wright} if and only if $\tilde{F}_{\epsilon}( \alpha , \omega , \tc) = 0$. 
	Furthermore, up to time translation, any periodic solution of~\eqref{eq:Wright} with period $2\pi/\omega$ is described by a Fourier series of the form~\eqref{e:ytc} with $\epsilon \geq 0$ and $\tc \in \ell^K_0$.
\end{theorem}


%We note that for $\epsilon>0$ such solutions are truly periodic, while for $\epsilon=0$ a zero of $\tilde{F}_\epsilon$ may either correspond to a periodic solution or to the trivial solution $y(t) \equiv 0$. 



% \begin{proof}
%  By Proposition \ref{thm:FourierEquivalence1}, it suffices to show that $\tilde{F}(\alpha,\omega,c) =0$ is equivalent to Equation \ref{eq:FourierSequenceEquation} being satisfied for $k \geq 1$.
%  Since Equation \ref{eq:FourierSequenceEquation} is equivalent to Equation \ref{eq:FHat}, we expand  Equation \ref{eq:FHat} by writing $ \hat{c} = \hat{\epsilon } + c$  where $ \hat{\epsilon} := (\epsilon,0,0,\dots) \in \ell^1$ as below:
%  \begin{equation}
%  0=  \left( i \omega K^{-1} + \alpha U_\omega \right) (\hat{\epsilon}+ c) + \alpha \left[U_\omega \, (\hat{\epsilon}+ c) \right] * (\hat{\epsilon}+ c) \label{eq:Intial}
%  \end{equation}
%  The RHS of Equation \ref{eq:Intial} is $ \tilde{F}(\alpha,\omega,c)$, so the theorem is proved.
% \end{proof}



Since we want to analyze a Hopf bifurcation, we will want to solve $\tilde{F}_\epsilon = 0$ for small values of~$\epsilon$. 
However, at the bifurcation point, $ D \tilde{F}_0(\pp  ,\pp , 0)$ is not invertible.
In order for our asymptotic analysis to be non-degenerate,
we work with a rescaled version of the problem. To this end, for any $\epsilon >0$, we rescale both $\tc$ and $\tilde{F}$ as follows. Let $\tc = \epsilon c$ and 
\begin{equation}\label{e:changeofvariables}
  \tilde{F}_\epsilon (\alpha,\omega,\epsilon c) = \epsilon F_\epsilon (\alpha,\omega,c).
\end{equation}
For $\epsilon>0$ the problem then reduces to finding zeros of 
\begin{equation}
\label{eq:FDefinition}
	F_\epsilon(\alpha,\omega, c) := 
	[i \omega + \alpha e^{-i \omega}] \e_1 + 
	( i \omega K^{-1} + \alpha U_{\omega}) c + 
	\epsilon \alpha e^{-i \omega} \e_2  +
	\alpha \epsilon L_\omega c + 
	\alpha \epsilon [ U_{\omega} c] * c.
\end{equation}
We denote the triple $(\alpha,\omega,c) \in \R^2 \times \ell^1_0$ by $x$.
To pinpoint the components of $x$ we use the projection operators
\[
   \pi_\alpha x = \alpha, \quad \pi_\omega x = \omega, \quad 
  \pi_c x = c \qquad\text{for any } x=(\alpha,\omega,c).
\]

After the change of variables~\eqref{e:changeofvariables} we now have an invertible Jacobian $D F_0(\pp  ,\pp , 0)$ at the bifurcation point.
On the other hand, for $\epsilon=0$ the zero finding problems for $\tilde{F}_\epsilon$ and $F_\epsilon$ are not equivalent. 
However, it follows from the following lemma that any nontrivial periodic solution having $ \epsilon=0$ must have a relatively large size when $ \alpha $ and $ \omega $ are close to the bifurcation point. 

\begin{lemma}\label{lem:Cone}
	Fix $ \epsilon \geq 0$ and $\alpha,\omega >0$. 
	Let
	\[
	b_* :=  \frac{\omega}{\alpha} - \frac{1}{2} - \epsilon  \left(\frac{2}{3}+ \frac{1}{2}\sqrt{2 + 2 |\omega-\pp| } \right).
	\]
Assume that $b_*> \sqrt{2} \epsilon$. 
Define
% \begin{equation*}%\label{e:zstar}
% 	z^{\pm}_* :=b_* \pm \sqrt{(b_*)^2- \epsilon^2 } .
% \end{equation*}
% \note[J]{Proposed change to match Lemma E.4}
\begin{equation}\label{e:zstar}
z^{\pm}_* :=b_* \pm \sqrt{(b_*)^2- 2 \epsilon^2 } .
\end{equation}
If there exists a $\tc \in \ell^1_0$ such that $\tilde{F}_\epsilon(\alpha, \omega,\tc) = 0$, then \\
\mbox{}\quad\textup{(a)} either $ \|\tc\| \leq  z_*^-$ or $ \|\tc\| \geq z_*^+  $.\\
\mbox{}\quad\textup{(b)} 
$ \| K^{-1} \tc \| \leq (2\epsilon^2+ \|\tc\|^2) / b_*$. 
\end{lemma}
\begin{proof}
	The proof follows from Lemmas~\ref{lem:gamma} and~\ref{lem:thecone} in Appendix~\ref{appendix:aprioribounds}, combined with the observation that
$\frac{\omega}{\alpha} - \gamma \geq b_*$,
% \[
%   \frac{\omega}{\alpha} - \gamma \geq b_*
%  \qquad\text{for all }
% | \alpha - \pp| \leq r_\alpha \text{ and } 
%   | \omega - \pp| \leq r_\omega.
% \]
with $\gamma$ as defined in Lemma~\ref{lem:gamma}.
\end{proof}

\begin{remark}\label{r:smalleps}
We note that for $\alpha < 2\omega$
\begin{alignat*}{1}
z^+_* &\geq   \frac{2 \omega - \alpha}{\alpha} 
- \epsilon \left(4/3+\sqrt{2 + 2 |\omega-\pp| } \, \right) + \cO(\epsilon^2)
\\[1mm]
z^-_* & \leq   \cO(\epsilon^2)
\end{alignat*}
for small $\epsilon$. 
Hence Lemma~\ref{lem:Cone} implies that for values of $(\alpha,\omega)$ near $(\pp,\pp)$ any solution has either $\|\tc\|$ of order 1 or $\|\tc\| =  \cO(\epsilon^2)$. 
The asymptotically small term bounding $z_*^-$ is explicitly calculated in Lemma~\ref{lem:ZminusBound}. 
A related consequence is that for $\epsilon=0$ there are no nontrivial solutions 
of $\tilde{F}_0(\alpha,\omega,\tc)=0$ with 
$\| \tc \| < \frac{2 \omega - \alpha}{\alpha} $. 
\end{remark}

\begin{remark}\label{r:rhobound}
In Section~\ref{s:contraction} we will work on subsets of $\ell^K_0$ of the form
\[
  \ell_\rho := \{ c \in \ell^K_0 : \|K^{-1} c\| \leq \rho \} .
\]
Part (b) of Lemma~\ref{lem:Cone} will be used in Section~\ref{s:global} to guarantee that we are not missing any solutions by considering $\ell_\rho$ (for some specific choice of $\rho$) rather than the full space $\ell^K_0$.
In particular, we infer from Remark~\ref{r:smalleps} that  small solutions (meaning roughly that $\|\tc\| \to 0$ as $\epsilon \to 0$)
satisfy $\| K^{-1} \tc \| = \cO(\epsilon^2)$.
\end{remark}

The following theorem guarantees that near the bifurcation point the problem of finding all periodic solutions is equivalent to considering the rescaled problem $F_\epsilon(\alpha,\omega,c)=0$.
\begin{theorem}
\label{thm:FourierEquivalence3}
\textup{(a)} Let $ \epsilon > 0$,  $c \in \ell^K_0$, $\alpha>0$ and $ \omega >0$. 
	Define $y: \R\to \R$ as 
\begin{equation}\label{e:yc}
	y(t) = 
	\epsilon \left( e^{i \omega t }  + e^{- i \omega t }\right) 
	+ \epsilon  \sum_{k = 2}^\infty   c_k e^{i \omega k t }  + c_k^* e^{- i \omega k t } .
\end{equation}
%	and suppose that $ y(t) > -1$. 
	Then $y(t)$ solves~\eqref{eq:Wright} if and only if $F_{\epsilon}( \alpha , \omega , c) = 0$.\\
\textup{(b)}
Let $y(t) \not\equiv 0$ be a periodic solution of~\eqref{eq:Wright} of period $2\pi/\omega$
 with Fourier coefficients $\c$.
Suppose $\alpha < 2\omega$ and $\| \c \| < \frac{2 \omega - \alpha}{\alpha} $.
Then, up to time translation, $y(t)$ is described by a Fourier series of the form~\eqref{e:yc} with $\epsilon > 0$ and $c \in \ell^K_0$.
\end{theorem}

\begin{proof}
Part (a) follows directly from Theorem~\ref{thm:FourierEquivalence2} and the  change of variables~\eqref{e:changeofvariables}.
To prove part (b) we need to exclude the possibility that there is a nontrivial solution with $\epsilon=0$. The asserted bound on the ratio of $\alpha$ and $\omega$ guarantees, by Lemma~\ref{lem:Cone} (see also Remark~\ref{r:smalleps}), that indeed $\epsilon>0$ for any nontrivial solution. 
\end{proof}

We note that in practice (see Section~\ref{s:global}) a bound on $\| \c \|$ is derived from a bound on $y$ or $y'$ using Parseval's identity.

\begin{remark}\label{r:cone}
It follows from Theorem~\ref{thm:FourierEquivalence3} and Remark~\ref{r:smalleps} that for values of $(\alpha,\omega)$ near $(\pp,\pp)$ any reasonably bounded solution satisfies $\| c\| =  O(\epsilon)$ as well as $\|K^{-1} c \| = O(\epsilon)$ asymptotically (as $\epsilon \to 0$).
These bounds will be made explicit (and non-asymptotic) for specific choices of the parameters in Section~\ref{s:global}.
\end{remark}

% We are able to rule out such large amplitude solutions using global estimates such as those in \cite{neumaier2014global}.
% Hence, near the bifurcation point, the problem of describing periodic solutions of~\eqref{eq:Wright} reduces to studying the family of zeros finding problems $F_\epsilon=0$.





%Specifically, if a solution having $ \epsilon = 0$ does in fact correspond to a nontrivial periodic solution and $\alpha  < 2\omega $, then $ \| \tilde{c} \| > 2 \omega \alpha^{-1} -1$. 
%%PERHAPS THIS NEEDS A FORMULATION AS A THEOREM AS WELL?
%%IN OTHER WORDS: ARE WE SURE WE HAVE FOUND ALL ZEROS OF $\tilde{F}_0$, I.E. ALL SOLUTIONS WITH $\epsilon=0$ NEAR THE BIFURCATION POINT? AFTER RESCALING THESE ARE INVISIBLE?
%%THERE SHOULD BE A STATEMENT ABOUT THIS SOMEWHERE! EITHER HERE OR SOME





We finish this section by defining a curve of approximate zeros $\bx_\epsilon$ of $F_\epsilon$ 
(see \cite{chow1977integral,hassard1981theory}). 
%(see \cite{chow1977integral,morris1976perturbative,hassard1981theory}). 


\begin{definition}\label{def:xepsilon}
Let
\begin{alignat*}{1}
	\balpha_\epsilon &:= \pp + \tfrac{\epsilon^2}{5} ( \tfrac{3\pi}{2} -1)  \\
	\bomega_\epsilon &:= \pp -  \tfrac{\epsilon^2}{5} \\
	\bc_\epsilon 	 &:= \left(\tfrac{2 - i}{5}\right) \epsilon \,  \e_2 \,.
\end{alignat*}
We define the approximate solution 
$ \bx_\epsilon := \left( \balpha_\epsilon , \bomega_\epsilon  , \bc_\epsilon \right)$
for all $\epsilon \geq 0$.
\end{definition}

We leave it to the reader to verify that both 
 $F_\epsilon(\pp,\pp,\bc_{\epsilon})=\cO(\epsilon^2)$ and $F_\epsilon(\bx_\epsilon)=\cO(\epsilon^2)$.
%%%	
%%%	
%%%	}{Better like this?}
%%%\annote[J]{ $F_\epsilon(\bx_0)=\cO(\epsilon^2)$ and $F_\epsilon(\bx_\epsilon)=\cO(\epsilon^2)$.}{I think we'd still need the $ \bar{c}_\epsilon$ term in $\bar{x}_0$ to be of order $ \epsilon$.}
%%%\remove[JB]{We show in Proposition A.1
%%%%\ref{prop:ApproximateSolutionWorks} 
%%% that any $ x \in \R^2 \times \ell^1_0$ which is $ \cO(\epsilon^2)$ close to $ \bar{x}_\epsilon $ will yield the estimate $F_\epsilon(x) = \cO(\epsilon^2)$.
%%%Hence choosing $\{ \pp , \pp, \bar{c}_\epsilon\}$ as our approximate solution would also have been a natural choice for performing an $\cO(\epsilon^2)$ analysis and would have simplified several of our calculations.
%%%However,} 
%%%
We choose to use the more accurate approximation 
for the $ \alpha$ and $ \omega $ components to improve our final quantitative results. 














%
% Values for $ (\alpha, \omega,c)$ which approximately solve $\tilde{F}(\alpha,\omega,c) = 0$  are computed in  \cite{chow1977integral,morris1976perturbative,hassard1981theory} and are as follows:
%  \begin{eqnarray}
%  \tilde{\alpha}( \epsilon) &:=& \pi /2 + \tfrac{\epsilon^2}{5} ( \tfrac{3\pi}{2} -1) \nonumber \\
%  \tilde{\omega}( \epsilon) &:=& \pi /2 -  \tfrac{\epsilon^2}{5} \label{eq:ScaleApprox} \\
%  \tc(\epsilon) 	  &:=& \{ \left(\tfrac{2 - i}{5}\right)  \epsilon^2 , 0,0, \dots \} \nonumber
%  \end{eqnarray}
% In Appendix \ref{sec:OperatorNorms} we illustrate an alternative method for deriving this approximation.
%
%
%
%
% We want to solve $ \tilde{F}(\alpha , \omega, \hat{c}) =0$ for small values of $ \epsilon$.
% However $ D \tilde{F}(\alpha , \omega , c)$ is not invertible at $ ( \pp , \pp , 0)$ when $ \epsilon = 0$.
% In order for our asymptotic analysis to be non-degenerate, we need to make the change of variables $ c \mapsto \epsilon c$.
% Under this change of variables, we define the function $ F$ below so that $ \tilde{F}(\alpha , \omega , \epsilon c) =\epsilon  F( \alpha , \omega , c)$.
%
%
%
% \begin{definition}
% Construct an $\epsilon$-parameterized family of densely defined functions  $F : \R^2 \oplus \ell^1 / \C \to \ell^1$ by:
% \begin{equation}
% \label{eq:FDefinition}
% 	F(\alpha,\omega, c) :=
% 	[i \omega + \alpha e^{-i \omega}]_1 +
% 	( i \omega K^{-1} + \alpha U_{\omega}) c +
% 	[\epsilon \alpha e^{-i \omega}]_2  +
% 	\alpha \epsilon L_\omega c +
% 	\alpha \epsilon [ U_{\omega} c] * c.
% \end{equation}
% \end{definition}

%%
%%
%%\begin{corollary}
%%	\label{thm:FourierEquivalence3}
%%	Fix $ \epsilon > 0$, and $ c \in \ell^1 / \C $, and $ \omega >0$. Define $y: \R\to \R$ as 
%%	\[
%%	y(t) = 
%%	\epsilon \left( e^{i \omega t }  + e^{- i \omega t }\right) 
%%	+  \epsilon  \left( \sum_{k = 2}^\infty   c_k e^{i \omega k t }  + \overline{c}_k e^{- i \omega k t } \right) 
%%	\]
%%	and suppose that $ y(t) > -1$. 
%%	Then $y(t)$ solves Wright's equation at parameter $ \alpha > 0 $ if and only if $ F( \alpha , \omega , c) = 0$ at parameter $ \epsilon$. 
%%	
%%	
%%	
%%\end{corollary}
%%
%%
%%\begin{proof}
%%	Since $ \tilde{F}(\alpha,\omega, \epsilon c) = \epsilon F( \alpha , \omega , c)$, the result follows from Theorem \ref{thm:FourierEquivalence2}.
%%\end{proof}

% If we can find $(\alpha , \omega, c)$ for which $ F( \alpha , \omega,c)=0$ at parameter $\epsilon$, then $ \tilde{F}(\alpha ,\omega, c)=0$.
% By Theorem \ref{thm:FourierEquivalence2} this amounts to finding a periodic solution to Wright's equation.
% Lastly, because we have performed the change of variables $ c \mapsto \epsilon c$, we need to  apply this change of variables to our approximate solution as well.
%
% \begin{definition}
% 	Define the approximate solution $ x( \epsilon) = \left\{ \alpha(\epsilon ) , \omega ( \epsilon ) , c(\epsilon) \right\}$ as below,  where $c(\epsilon) = \{ c_2( \epsilon) , 0 ,0 , \dots\} $.
% 	We may also write $ x_\epsilon = x(\epsilon) $.
% 	\begin{eqnarray}
% 	\alpha( \epsilon) &:=& \pi /2 + \tfrac{\epsilon^2}{5} ( \tfrac{3\pi}{2} -1) \nonumber \\
% 	\omega( \epsilon) &:=& \pi /2 -  \tfrac{\epsilon^2}{5} \label{eq:Approx} \\
% 	c_2(\epsilon) 	  &:=& \left(\tfrac{2 - i}{5}\right) \epsilon \nonumber
% 	\end{eqnarray}
%
% \end{definition}


\section{Main result}
\label{sec:main}
\pdfoutput=1
\documentclass{article}
\usepackage[final]{pdfpages}
\begin{document}
\includepdf[pages=1-9]{CVPR18VOlearner.pdf}
\includepdf[pages=1-last]{supp.pdf}
\end{document}

\section{Proof of (Gentle Path) Lemma~\ref{le:boundedGentelSections}}
\label{sec:gentle}
\iftoggle{abstract}
{The main idea behind the proof of the Gentle Path lemma is that a gentle path
between $u_r \in U$ and $l_s \in L$ (where, say, $r \leq s$ and 
${\tt x}(u_r) < {\tt x}(u_s)$) in $T_{1n}$ can 
be extended using edge $(u_{r-1},u_r)$, unless $r=0$ or $u_r$ is a right
induction vertex of $T_{r}$, or using edge $(l_{s},l_{s+1})$, unless $s=n$ or
$l_s$ is a left induction vertex of $T_{s+1}$. 
In other words, a gentle path from $r$ to $s$ is either canonical
or can be extended to a canonical path from $u_{r'}$ to $l_{s'}$ as illustrated
in Fig.~\ref{fig:caseA}.

\begin{figure}[h]
\center{\caseA}
\caption{Illustration of the proof of 
Lemma~\ref{le:boundedGentelSections} in the case when the gentle path from $u_r$ to $l_s$ is just a gentle
edge. For every $i$ such that $r' < i \leq r$ and $u_i$ is
the right vertex of $H_i$, hexagon $H_i$ and the edge $(u_{i-1},u_i)$ are shown
in red. Each edge $(u_{i-1},u_i)$ has slope greater than $-\frac{1}{\sqrt{3}}$
and therefore has length bounded by
$\sqrt{3}d_x(u_{i-1},u_i) - ({\tt y}(u_{i-1})-{\tt y}(u_i))$, a value equal
to the total length of the two intersecting, red, dashed segments going north
from $u_{i-1}$ and north-west from $u_i$. The total length of the two dashed
blue segments is an upper bound on the length of the edge $(u_r,l_s)$ and
the total length of the dotted red line segments represent the upper bound
$\sqrt{3}d_x(p,q) - ({\tt y}(p)-{\tt y}(q))$ on the length of the path 
$p=u_{r'}, \dots, u_r, l_{s}, l_{s+1}, \dots, l_{s'}=q$.}
\label{fig:caseA}
\end{figure}

}
{%This section is devoted to the proof of the Gentle Path
%Lemma~\ref{le:boundedGentelSections}. 


\begingroup
\def\thetheorem{\ref{le:boundedGentelSections}}
\begin{lemma}[The Gentle Path Lemma] 
Let $T_{1n}$ be a linear sequence of triangles with respect to a line $st$ with
slope $m_{st}$ such that $0 < m_{st} < \frac{1}{\sqrt{3}}$. If $T_{1n}$ is
standard and contains a gentle path then the path can be extended
to a canonical gentle path in $T_{1n}$.
\end{lemma}
\addtocounter{theorem}{-1}
\endgroup

The main idea behind the proof of this lemma is that a gentle path between
$u_r \in U$ and $l_s \in L$ (where, say, $r \leq s$ and 
${\tt x}(u_r) < {\tt x}(u_s)$) in $T_{1n}$ can 
be extended using edge $(u_{r-1},u_r)$, unless $r=0$ or $u_r$ is a right
induction vertex of $T_{r}$, or using edge $(l_{s},l_{s+1})$, unless $s=n$ or
$l_s$ is a left induction vertex of $T_{s+1}$. In other words, a gentle path
can be extended unless it is canonical.

\begin{proof}%[Proof of Lemma~\ref{le:boundedGentelSections}]
%\label{lem:canonical}
We break the proof into several cases.

{\bf Case A.}
We first prove the claim in the case when $T_{1n}$ contains no irregular gentle
edge. In that case, because $T_{1n}$ is standard and by Lemma~\ref{lem:props},
the left vertex of $T_1$, whether it is $l_{0}$ or $u_{0}$ (or $l_0=u_0$),
lies on side $w$ of $H_1$ and the right vertex of $T_n$, whether it is $l_n$
or $u_n$ (or $u_n=l_n$), lies on side $e$ of $H_n$.

Let $T_{1n}$ contain a gentle path between $u_r \in U$ and
$l_s \in L$. W.l.o.g., we assume that $u_r$ occurs before $l_s$ 
(i.e., $r \leq s$ and ${\tt x}(u_r) < {\tt x}(l_r)$), 
as the case when $l_s$ occurs before $u_r$ can be argued
using a symmetric argument. 
Consider the sequence of points $u_{0}, u_1, \dots, u_r$ and
let $p=u_{r'}$ be the last point that is a right induction vertex (of $H_{r'}$)
or $p=u_{r'}=u_{0}$ if there are no right induction vertices in the sequence.
Note that in the second case, $p$ cannot be a base vertex of $T_1$ 
(since $T_{1n}$ is standard) and so $p$ must be the left vertex of
$T_1$.
%, if such a point
%exists; if such a point does not exist, we set $r'=i-1$ and $p$ to be the
%left vertex of $T_i$, which lies on the W side of $T_i$ and could be $u_{i-1}$
%or $l_{i-1}$. 
The edges of path $u_{r'}, \dots u_r$ are edges $(u_{i-1},u_i)$ for every $i$
such that $r' < i \leq r$ and $u_i$ is a right vertex of $H_i$. Because $T_{1n}$
contains no irregular edges and the fact that $u_i$ cannot lie on the $e$ side
of $H_i$ (because if it did $u_i$ would be a right induction vertex) or
the $s_e$ side of $H_i$ (by Lemma~\ref{lem:props}), the endpoints $u_{i-1}$ 
and $u_i$ lie on the $w$ or $n_w$
and $n_w$ or $n_e$ sides, respectively, of $H_i$. This implies that
${\tt x}(u_{i-1}) < {\tt x}(u_i)$ and that the length of $(u_{i-1},u_i)$ is
bounded by
$\sqrt{3}d_x(u_{i-1},u_i) - ({\tt y}(u_{i-1})-{\tt y}(u_i))$, as illustrated in 
Fig.~\ref{fig:caseA}. %In the case when $r'=i-1$,
%the length of the edge $(p,u_i)$
%can be bounded by $\sqrt{3}d_x(p,u_i)) - ({\tt y}(p)-{\tt y}(u_i))$. We use these
%inequalities to obtain that 
Therefore, ${\tt x}(p=u_{r'}) < {\tt x}(u_r)$ and the length of the path 
$p=u_{r'}, \dots, u_r$ (path $\mathcal{P}$)
in $T_{1n}$ is at most $\sqrt{3}d_x(p,u_{r})-({\tt y}(p)-{\tt y}(u_r))$. 

\begin{figure}[!b]
\center{\caseA}
\caption{Illustration of case A in the proof of Lemma~\ref{le:boundedGentelSections}.
Shown is the case when the gentle path from $u_r$ to $l_s$ is just a gentle
edge. For every $i$ such that $r' < i \leq r$ and $u_i$ is
the right vertex of $H_i$, hexagon $H_i$ and the edge $(u_{i-1},u_i)$ are shown
in red; the length of $(u_{i-1},u_i)$ is bounded by
$\sqrt{3}d_x(u_{i-1},u_i) - ({\tt y}(u_{i-1})-{\tt y}(u_i))$, a value equal
to the total length of the two intersecting, red, dashed segments going north
from $u_{i-1}$ and north-west from $u_i$. The total length of the two dashed
blue segments is an upper bound on the length of the edge $(u_r,l_s)$ and
the total length of the dotted red line segments represent the upper bound
$\sqrt{3}d_x(p,q) - ({\tt y}(p)-{\tt y}(q))$ on the length of the path 
$p=u_{r'}, \dots, u_r, l_{s}, l_{s+1}, \dots, l_{s'}=q$.}
\label{fig:caseA}
\end{figure}

Similarly, we consider the sequence of points $l_s, l_{s+1}, \dots, l_n$ and
set $q=l_{s'}$ to be the first point that is a left induction vertex (of $H_{s'+1}$)
or $q = l_{s'} = l_n$ if there are no left induction vertices in the sequence.
For every $i$ such that $s \leq i < s'$, if $l_i$ is a left vertex of $H_{i+1}$
then edge $(l_{i},l_{i+1})$ has endpoints $l_{i}$ and $l_{i+1}$ lying on the 
$s_w$ or $s_e$ and $s_e$ and $e$ sides, respectively, of $H_{i+1}$. This implies
that ${\tt x}(l_{i}) < {\tt x}(l_{i+1})$ and that the length of
$(l_{i},l_{i+1})$ is bounded by 
$\sqrt{3}d_x(l_{i},l_{i+1}) - ({\tt y}(l_{i})-{\tt y}(l_{i+1}))$. Therefore, 
${\tt x}(l_s) < {\tt x}(l_{s'}=q)$ and the length of the path 
$l_{s}, l_{s+1}, \dots, l_{s'}=q$ (path $\mathcal{Q}$) is at most
$\sqrt{3}d_x(l_{s},q)-({\tt y}(l_s)-{\tt y}(q))$, as shown in 
Fig.~\ref{fig:caseA}. 
We combine the paths $\mathcal{P}$ and $\mathcal{Q}$ with the gentle path from
$u_r$ to $l_s$ and this combined
path is gentle, and therefore canonical, since $r' < s'$,  
${\tt x}(p=u_{r'}) < {\tt x}(l_{s'}=q)$, and:
\begin{align*}
d_T (p, q) & \leq d_T(p, u_r) + d_T(u_r, l_s) + d_T(l_s, q) \\
& \leq \sqrt{3}d_x(p,u_r)-({\tt y}(p)-{\tt y}(u_r))+ \sqrt{3}d_x(u_r,l_s))-({\tt y}(u_r)-{\tt y}(l_s)) \\
 & \quad + \sqrt{3}d_x(l_s, q)) - ({\tt y}(l_s) - {\tt y}(q)) \\
& = \sqrt{3}d_x(p,q)-({\tt y}(p)-{\tt y}(q)).
\end{align*}
% since its length is
%\begin{align*} 
%d_T(p,q) & \leq  d_T(p,u_{r}) + d_T(u_{r},l_{s}) +  d_T(l_{s},q) \\
%& \leq \sqrt{3}d_x(p,u_{r})-({\tt y}(p)-{\tt y}(u_{r})) + \sqrt{3}d_x(u_{r},l_{s})-({\tt y}(u_{r})-{\tt y}(l_{s})) + \\
%& \quad \sqrt{3}d_x(l_{s},q)-({\tt y}(l_{s})-{\tt y}(q)) \\
%& =  \sqrt{3}d_x(p,q)-({\tt y}(p)-{\tt y}(q)).
%\end{align*}

{\bf Case B.} 
We now consider the case when $T_{1n}$ contains an irregular gentle edge. 
We assume
w.l.o.g. that the gentle edge has an endpoint $u_r$ that lies on side $s_w$ of
$H_{r+1}$ and also that the gentle edge is the first such edge (in the sense
that for all $i$ such that $0 \leq i < r$, $u_i$ does not lie on side $s_w$
of $H_{i+1}$). (For the case when $u_r$ lies on side $s_e$ of $H_r$ we would
consider the last such edge and use an argument that is symmetric to the one
we make below; the cases when the irregular gentle edge has an endpoint $l_r$
on side $n_w$ of $H_{r+1}$ or on side
$n_e$ of $H_r$ are symmetric to the cases when $u_r$ is on side $s_w$
of $H_{r+1}$ and side $s_e$ of $H_r$, respectively.)

By Lemma~\ref{lem:props}, the irregular gentle edge under consideration is 
$(u_r,l_{r+1})$, with $l_{r+1}$ lying on side $s_e$ or $e$ of $H_{r+1}$. Let $p$
and $q$ be as defined in Case A. Because of our assumption that $(u_r,l_{r+1})$
is first, the case A arguments can be applied to obtain 
${\tt x}(p) < {\tt x}(u_r)$ and bound the distance from $p$ to
$u_r$ by $\sqrt{3}d_x(p,u_r)-({\tt y}(p)-{\tt y}(u_r))$. We cannot do the same
to bound the distance from $l_{r+1}$ to $q = l_{s'}$ because it is possible that for
some $i$ such that $r+1 < i \leq s'$, $l_i$ lies on side $n_e$ of $H_i$ and
${\tt x}(l_{i-1}) > {\tt x}(l_i)$. So we proceed instead with induction and
prove that
\begin{equation*}
\label{eq:induction}
{\tt x}(u_r) < {\tt x}(l_i) \mbox{ and } d_T(u_r, l_i) \leq \sqrt{3}d_x(u_r, l_i) - ({\tt y}(u_r) - {\tt y}(l_i)) \mbox{ for every $i$ such that } r < i \leq s'
\end{equation*}
which will complete the proof.

The base case $i=r+1$ holds because $(u_r,l_{r+1})$ is gentle and 
${\tt x}(u_r) < {\tt x}(l_{r+1})$. For the induction step, we assume that 
${\tt x}(u_r) < {\tt x}(l_i)$ and 
$d_T(u_r, l_i) \leq \sqrt{3}d_x(u_r, l_i)-({\tt y}(u_r)-{\tt y}(l_i))$ 
for all $i$ such that $r < i < s \leq s'$ and show that the
inequality holds for $i = s$ as well. If $l_s = l_{s-1}$, that is trivially
true. Otherwise, $l_s$ is a right vertex of $H_s$. If
$l_s$ lies on the $e$ or $s_e$ side of $H_s$ then we use the arguments from
Case A to get that ${\tt x}(l_{s-1}) < {\tt x}(l_s)$, that the length of edge
$(l_{s-1}, l_s)$ is less than
$\sqrt{3}d_x(l_{s-1}, l_s) - ({\tt y}(l_{s-1}) - {\tt y}(l_s))$, 
and that therefore the inductive step again easily holds.

\begin{figure}
\center{\caseB}
\vspace{-1.2cm}
\center{(a) \hspace{6.35cm} (b)}

\caption{Illustration of {\em Case B.1} in the proof of 
Lemma~\ref{le:boundedGentelSections}. (a) $T_{(r+1)s}$. Point $u_r$ lies on side 
$s_w$ of $H_{r+1}$, point $l_s$ lies on side $n_e$ of $H_s$, and no
$(l_t, u_t)$, for $r < t < s$, has positive slope. (b) $\mathcal{R}(T_{(r+1)s})$
satisfies the conditions of Lemma~\ref{le:mainlemmaB} with $p = u_r$ and
$q = l_s$; because the slope of the line passing through $u_r$ and $l_s$ is less
than $-\frac{1}{\sqrt{3}}$, the length of the red, dashed segments (shown also in (a))
 is an upper bound on the length of a path in $T_{(r+1)s}$ from $u_r$ to $l_s$.}
\label{fig:caseB}
\end{figure}

If, however, $l_s$ lies on side $n_e$ of $H_s$ then we cannot use the Case A
argument because ${\tt x}(l_{s-1}) < {\tt x}(l_s)$ is not necessarily true.
In that case, by Lemma~\ref{lem:props} $(u_{s-1},l_s)$ must be an
(irregular) gentle edge with $u_{s-1}$ lying on the $w$ or $n_w$ side of $H_s$
(as illustrated in 
Fig.~\ref{fig:caseB}-(a)). First we note that ${\tt x}(u_r) < {\tt x}(l_s)$
holds because otherwise $l_s$ would appear below and to the right of $u_r$ and
then $(u_{s-1},l_s)$ would appear before $(u_r,l_{r+1})$
when traveling along segment $[st]$ from $s$ to $t$, a contradiction.
In order to bound $d_T(u_r,l_s)$, 
we consider indices $r+1,...,s-1$ and set $t$ to be
the last one such that edge $(l_t,u_t)$ has positive slope 
(i.e., ${\tt x}(l_t) < {\tt x}(u_t)$ and ${\tt y}(l_t) < {\tt y}(u_t))$, 
if one exists. We consider two subcases:



{\em Case B.1.} If no $(l_t, u_t)$, for $r < t < s$,
has positive slope  (as is the case in Fig.~\ref{fig:caseB}-(a)) we consider
the transformation of the linear sequence 
%$T'_{(r+1)s}$ of 
$T_{(r+1)s}$ obtained by rotating the plane clockwise by
an angle of $\pi/3$, % and then reflecting the plane with respect to the $x$-axis,
as illustrated in Fig.~\ref{fig:caseB}-(b). 

%Let $T'_{r+1}, \dots, T'_s$ be the
%corresponding transformations of triangles $T_{r+1}, \dots, T_s$
%and let $d'_x(u_r,l_s)$ be the difference between the abscissas of $l_s$ and
%$u_r$ in the transformed plane.

In the rotated $T_{(r+1)s}$, which we denote $\mathcal{R}(T_{(r+1)s})$, 
we have $-\sqrt{3} < m_{st} < -\frac{1}{\sqrt{3}}$, $u_r$ is a left induction 
vertex of $T_{r+1}$, and $l_s$ is a right induction vertex of $T_s$. We show
next that $\mathcal{R}(T_{(r+1)s})$ is regular. If,
in $\mathcal{R}(T_{(r+1)s})$, $u_t$ were to lie on a $s$ side of $H_t$ for some
$r < t < s$ then the slope of $(l_t,u_t)$ would have to be greater than
$m_{st}$ and less than the slope of the $s_e$ side of hexagon $H_t$. So,
the slope of $(l_t,u_t)$ in $\mathcal{R}(T_{(r+1)s})$ would have to be
between $-\sqrt{3}$ and $\frac{1}{\sqrt{3}}$ which would imply that
$(l_t,u_t)$ has positive slope in $T_{(r+1)s}$, a contradiction. Note
that if $u_{t-1}$ were to lie on a $s$ side of $H_t$ then so would $u_t$ so
we need not consider that case. We similarly show that no $l_{t-1}$ or $l_t$
lies on a $n$ side of $H_t$. Finally, we note that a gentle edge
$(l_t, u_t)$ in $\mathcal{R}(T_{(r+1)s})$ would have to have positive slope in
$T_{(r+1)s}$, a contradiction. So $\mathcal{R}(T_{(r+1)s})$ contains no gentle
edge and it is thus regular.

All the conditions of (Technical) Lemma~\ref{le:mainlemmaA} therefore apply to
$\mathcal{R}(T_{(r+1)s})$. By the lemma, the distance from $u_r$ to $l_s$
in $\mathcal{R}(T_{(r+1)s})$ (and thus in the original $T_{(r+1)s}$ as well) is
bounded by 
$\frac{4}{\sqrt{3}} d^{\mathcal{R}}_x(u_r,l_s)$, where $d^{\mathcal{R}}_x(u_r,l_s)$
is the difference
between the abscissas of $u_r$ and $l_s$ in $\mathcal{R}(T_{(r+1)s})$). Note that
$\frac{4}{\sqrt{3}} d^{\mathcal{R}}_x(u_r,l_s)$ is the sum of the lengths of
two sides of an equilateral triangle of height $d^{\mathcal{R}}_x(u_r,l_s)$.
Therefore, because
the slope of the line through $u_r$ and $l_s$ in $\mathcal{R}(T_{(r+1)s})$ is
less than $-\frac{1}{\sqrt{3}}$, 
$\frac{4}{\sqrt{3}} d^{\mathcal{R}}_x(u_r,l_s)$ is less than the
length of the piecewise linear curve, shown in Fig.~\ref{fig:caseB}-(b),
consisting of a (longer) vertical segment down from $u_r$ followed by a
(shorter) segment with slope $\frac{1}{\sqrt{3}}$ to $l_s$. The length of that
curve is exactly $\sqrt{3}d_x(u_r,l_s)-({\tt y}(u_r)-{\tt y}(l_s))$ in
$T_{(r+1)s}$ as illustrated in Fig.~\ref{fig:caseB}-(a), which completes the
proof in this case.


{\em Case B.2.} If $(l_t,u_t)$ exists, as illustrated in
Fig.~\ref{fig:othercase}, then ${\tt x}(l_t) < {\tt x}(u_t)$ and 
$d_T(l_t,u_t) \leq \sqrt{3}d_x(l_t,u_t)-({\tt y}(l_t)-{\tt y}(u_t))$. Also,
by induction, ${\tt x}(u_r) \leq {\tt x}(l_t)$ and 
$d_T(u_r,l_t) \leq \sqrt{3}d_x(u_r,l_t)-({\tt y}(u_r)-{\tt y}(l_t))$.
What remains to be done is to show that ${\tt x}(u_t) < {\tt x}(l_s)$ and 
that $d_T(u_t,l_s) \leq \sqrt{3}d_x(u_t,l_s)-({\tt y}(u_t)-{\tt y}(l_s))$. 
We now have two cases to consider. Note first that, because $t$ is last, for 
every $j$ such that $t < j < s$, $u_j$ cannot lie on side $e$ or $s_e$ of $H_j$
(because otherwise $(l_j,u_j)$ would have to have positive slope); this means
that $u_j$ cannot be a right induction point of $H_j$.



\begin{figure}
\center{\othercase}

\caption{Illustration of {\em Case B.2} in the proof of 
Lemma~\ref{le:boundedGentelSections}. If $(l_t,u_t)$ has positive slope then
${\tt x}(l_t) < {\tt x}(u_t)$ and 
$d_T(l_t,u_t) \leq \sqrt{3}d_x(l_t,u_t)-({\tt y}(l_t)-{\tt y}(u_t))$.
By induction, ${\tt x}(u_r) \leq {\tt x}(l_t)$ and 
$d_T(u_r,l_t) \leq \sqrt{3}d_x(u_r,l_t)-({\tt y}(u_r)-{\tt y}(l_t))$. In
{\em Case B.2.i}, no $u_{j-1}$ lies on side $s_w$ of $H_j$, where $t<j<s$, and so
${\tt x}(u_t) \leq \dots \leq {\tt x}(u_{s-1}) \leq {\tt x}(l_s)$ and 
$d_T(u_u,l_s) \leq \sqrt{3}d_x(u_t,l_s)-({\tt y}(u_t)-{\tt y}(l_s))$.}
\label{fig:othercase}
\end{figure}





{\em Case B.2.i.} If we also have that no $u_{j-1}$ lies on side $s_w$ of $H_j$,
where $t<j<s$ and as illustrated in Fig.~\ref{fig:othercase}, 
then, using the arguments from Case A, we can show that
${\tt x}(u_t) \leq \dots \leq {\tt x}(u_{s-1}) \leq {\tt x}(l_s)$ and 
can bound the length of the path $u_t, \dots, u_{s-1}, l_s$ with 
$\sqrt{3}d_x(u_t,l_s)-({\tt y}(u_t)-{\tt y}(l_s))$. 

%Putting
%everything together, we get ${\tt x}(u_r) \leq {\tt x}(l_s)$ and:
%\begin{align*}
%d_T (u_r, l_s) & \leq d_T(u_r, l_t) + d_T(l_t, u_t) + d_T(u_t, l_s) \\
%& \leq \sqrt{3}d_x(u_r,l_t)-({\tt y}(u_r)-{\tt y}(l_t))+ \sqrt{3}d_x(l_t,u_t))-({\tt y}(l_t)-{\tt y}(u_t)) \\
% & \quad + \sqrt{3}d_x(u_t, l_s)) - ({\tt y}(u_t) - {\tt y}(l_s)) \\
%& \leq \sqrt{3}d_x(u_r,l_s)-({\tt y}(u_r)-{\tt y}(l_s)).
%\end{align*}

{\em Case B.2.ii.} Finally, suppose that $u_{t'-1}$, for some $t'$ such that
$t < t' < s$, lies on side $s_w$ of $H_{t'}$ and let us assume that $t'$ is
leftmost (in the sense that no $u_{j-1}$ lies on side $s_w$ of $H_j$ for $j$
such that $t < j < t'$). Using arguments from Case A we obtain that 
${\tt x}(u_t) < {\tt x}(u_{t'})$ and 
$d_T(u_t, u_{t'}) \leq \sqrt{3}d_x(u_t, u_{t'})) - ({\tt y}(u_t) -
{\tt y}(u_{t'} ))$. Using the approach from {\em Case B.1} we get
${\tt x}(u_{t'}) < {\tt x}(l_s)$ and 
$d_T(u_{t'}, l_s) \leq \sqrt{3}d_x(u_{t'}, l_s) - ({\tt y}(u_{t'})-{\tt y}(l_s))$.
We complete the proof of the lemma by combining all these inequalities.
\end{proof}

}

\section{Proof of (Technical) Lemma \ref{le:mainlemmaA}}
\label{sec:proofA}
\iftoggle{abstract}
{We prove this lemma via a framework that uses continuous versions
of the discrete functions ($p_N$, $p_S$, etc.) informally introduced in
Subsection~\ref{sub:keylemma}. We start by defining functions $H(x)$, $T(x)$, $u(x)$,
$\ell(x)$, ${\tt r}(x)$, $w(x)$, and $e(x)$ for
${\tt x}(p) \leq x \leq {\tt x}(q)$ as illustrated in
Fig.~\ref{fig:notation}.

\begin{figure}[h]
\center{\notation}
\caption{Let point $c_i$ be the center of hexagon $H_i$, for $i=1,\dots,n$. For $x$ such that
${\tt x}(c_i) \leq x < {\tt x}(c_{i+1})$, $H(x)$ is the hexagon whose center has abscissa
$x$ and that has points $u_i=u(x)$ and $l_i=\ell(x)$ on its boundary.
Intuitively, function $H(x)$ from $x= {\tt x}(c_i)$ to $x= {\tt x}(c_{i+1})$
models the ``pushing''
of hexagon $H_i$ through $u_i$ and $l_i$ up until it becomes $H_{i+1}$. Function
${\tt r}(x)$ is the minimum radius of $H(x)$ and $w(x) = x-{\tt r}(x)$ and
$e(x) = x + {\tt r}(x)$ are the abscissas of the $w$ and $e$ sides, respectively,
of $H(x)$. Finally, we define $T(x) = T_{1i}$ when ${\tt x}(c_i) \leq x < {\tt x}(c_{i+1})$.} % $N(x)$ and $S(x)$ are the vertices $N$ and $S$ of $H(x)$.}
\label{fig:notation}
\end{figure}

For a point $o$ on a side of $H(x)$, we define functions $p_N(o, x)$ and 
$p_S(o, x)$ as the {\em signed} shortest distances around the perimeter of
$H(x)$ to the $N$ vertex and $S$ vertex, respectively, with sign 
$\sgn(x - x(o))$. As Fig.~\ref{fig:dNdS}-(a) and Fig.~\ref{fig:dNdS}-(b)
illustrate, these signs are positive for $o$ on the $n_w$, $w$, or $s_w$
sides of $H(x)$ and negative for $o$ on $n_e$, $e$, or $s_e$ sides. 
We omit $o$ and use the shorthand notation $p_N(x)$ if $o = u(x)$ and
$p_S(x)$ if $o = \ell(x)$. 

\begin{figure}
\dNdS

\hspace{1.45cm} (a) \hspace{4.35cm} (b) \hspace{4.55cm} (c)
\caption{(a) The values of $p_N(o, x)$ are shown, for various points $o$
lying on the boundary of $H(x)$, as signed hexagon arc lengths. (b) The 
values of $p_S(o, x)$ are shown similarly. (c) The length of edge
$(u_{i-1},u_i)$, with $u_{i-1},u_i$ lying on the boundary of $H(x)=H_i$, 
is bounded by $p_N(u_{i-1}, x) - p_N(u_i,x)$.}
\label{fig:dNdS}
\end{figure}



Functions $U(x)$ and $L(x)$, used to bound the length of the shortest path from $p$ to $q$ and illustrated in Fig.~\ref{fig:UandL}-(a), are defined as follows for ${\tt x}(p) \leq x \leq {\tt x}(q)$:
\begin{align*}
& U(x) = d_{T(x)}(p,u(x)) + p_N(x) \mbox{\hspace{2cm}} L(x) = d_{T(x)}(p,\ell(x)) + p_S(x)
\end{align*}
We note that $U({\tt x}(q))+L({\tt x}(q))$ is exactly twice the distance in 
$T_{1n}$ from $p$ to $q$. We will compute an upper bound for function $U+L$ 
by bounding its growth rate.

Functions ${\bar U}(x)$ and ${\bar L}(x)$, used to bound the lengths of the upper and lower paths in $T_{1n}$ and illustrated in Fig.~\ref{fig:UandL}-(b), are defined as follows for ${\tt x}(c_i) \leq x \leq {\tt x}(c_{i+1})$:

\begin{figure}[!b]
\begin{center}\UandL

(a) \hspace{6cm} (b)
\end{center}
\caption{(a) Definition of $U(x)$ and $L(x)$. For example, $U(x)$ for 
${\tt x}(c_i) \leq x < {\tt x}(c_{i+1})$ is the sum
of the length of the shortest path from $p$ to $u_i$ in $T_{1i}$ 
(illustrated as the red dashed path) and $p_N(x)$ (of negative value and 
represented as a red arrow). (b) Definition of $\bar{U}(x)$ and $\bar{L}(x)$. 
When ${\tt x}(c_i) \leq x < {\tt x}(c_{i+1})$ for example,
$\bar{U}(x) - p_N(x)$ is an upper bound (equal to the length of the 
sequence of red dashed
hexagon arcs going from $p$ to $u_i$) on the length
of the upper path $p, u_0, u_1, \dots, u_{i-1}, u_i$.}
\label{fig:UandL}
\end{figure}


\begin{align*}
\bar{U}(x) & = \sum_{j=1}^{i} (p_N(u({\tt x}(c_{j-1})), {\tt x}(c_{j})) - p_N(u({\tt x}(c_{j})),{\tt x}(c_{j}))) + p_N(x) \\
\bar{L}(x) & = \sum_{j=1}^{i} (p_S(\ell({\tt x}(c_{j-1})), {\tt x}(c_{j})) - p_S(\ell({\tt x}(c_{j})),{\tt x}(c_{j}))) + p_S(x)
\end{align*}



Functions ${\bar U}(x)$ and ${\bar L}(x)$, 
as well as $U(x)$ and $L(x)$, have rates of growth when
${\tt x}(c_i) < x < {\tt x}(c_{i+1})$ that depend solely on the
last term ($p_N(x)$ or $p_S(x)$).
%For ${\tt x}(c_i) < x < {\tt x}(c_{i+1})$, the terms 
%$d_{T_{1i}}(p,u_i)$ and $d_{T_{1i}}(p,l_i)$ in the definitions of 
%$U(x)$ and $L(x)$, respectively, are constant. 
%This means that for such $x$ the rate of growth of
%functions $U(x)$ and $L(x)$ is determined solely by the terms 
%$p_N(x)$ and $p_S(x)$, respectively. 
We show that functions $p_N$ and 
$p_S$ are monotonically increasing piecewise linear 
 and bound the rate of growth of $p_N$ and $p_S$ using elementary geometric 
arguments illustrated in Fig.~\ref{fig:growth}. Figure~\ref{fig:growth}-(c)
illustrates a case when the growth rate of $p_N+p_S$, and therefore also of 
${\bar U}+{\bar L}$ and of $U+L$, is $\frac{8}{\sqrt{3}}$.
%with respect to horizontal
%distance $\Delta x$
%, which is also the growth
%rate of ${\bar U} + {\bar L}$ and also $U+L$ when 
%${\tt x}(c_i) < x < {\tt x}(c_{i+1})$, is at most  
%Figure~\ref{fig:growth} illustrates the the proof.

\begin{figure}
\growth

\hspace{2cm} (a) \hspace{3.8cm} (b) \hspace{4.2cm} (c)
\caption{Constructions demonstrating growth rates, with respect to 
$\Delta x = 1$, of $p_N$, $p_S$ and other
functions for three different placements of $u(x)=u$ and $\ell(x)=l$ on the
boundary of $H(x)$. 
%Note that the growth
%$\Delta p_N(x)$ when $t(x)$ is a single transition can be represented using
%a piecewise linear curve of length $\Delta p_N(x)$.
}
\label{fig:growth}
\end{figure}



}
{\begingroup
\def\thetheorem{\ref{le:mainlemmaA}}
\begin{lemma}[The Technical Lemma]
%If $T_{1n}$ is regular, $p$ is the left induction vertex of $T_1$, and
%$q$ is the right induction vertex of $T_n$ then
%\begin{itemize}
%\item the slope of the line through $p$ and $q$ is between $-\sqrt{3}$ and 
%$\sqrt{3}$
%\item there is a path in $T_{1n}$ from $p$ to $q$ of length at most $\frac{4}{\%sqrt{3}} d_x(p,q)$.
%\end{itemize}
If $T_{1n}$ is a regular linear sequence of triangles then there is a path in
$T_{1n}$ from
the left induction vertex $p$ of $T_1$ to the right induction vertex $q$ of
$T_n$ of length at most  $\frac{4}{\sqrt{3}} d_x(p,q)$.
\end{lemma}
\addtocounter{theorem}{-1}
\endgroup

To prove this lemma we develop a framework that uses continuous versions
of the discrete functions ($p_N$, $p_S$, etc.) informally introduced in
Subsection~\ref{sub:keylemma}. We start by defining functions $H(x)$, $u(x)$,
and $\ell(x)$ for
${\tt x}(p) \leq x \leq {\tt x}(q)$:
\begin{itemize}
\item If $c_i$ is the center of hexagon $H_i$,
for $i = 1, \dots, n$, we define $H({\tt x}(c_i)) = H_i$. We also define
$u({\tt x}(c_i))$ and $\ell({\tt x}(c_i))$ to be $u_i$ and $l_i$,
respectively. 
\item Then, for every $i = 1, \dots, n-1$ and $x$ such that
${\tt x}(c_i) < x < {\tt x}(c_{i+1})$,
we define $H(x)$ to be the hexagon whose center has abscissa $x$ and that
has points $u_i$ and $l_i$ on its boundary; we also define $u(x)$ to be $u_i$
and $\ell(x)$ to be $l_i$ (see Fig.~\ref{fig:notation}).
\iftoggle{abstract}
{\begin{figure}}
{\begin{figure}}
\center{\notation}
\caption{For $x$ such that
${\tt x}(c_i) < x < {\tt x}(c_{i+1})$, $H(x)$ is the hexagon whose center has abscissa
$x$ and that has points $u_i=u(x)$ and $l_i=\ell(x)$ on its boundary.
Intuitively, function $H(x)$ from $x= {\tt x}(c_i)$ to $x= {\tt x}(c_{i+1})$
models the ``pushing''
of hexagon $H_i$ through $u_i$ and $l_i$ up until it becomes $H_{i+1}$. Function
${\tt r}(x)$ is the minimum radius of $H(x)$ and $w(x) = x-{\tt r}(x)$ and
$e(x) = x + {\tt r}(x)$ are the abscissas of the $w$ and $e$ sides, respectively,
of $H(x)$. $N(x)$ and $S(x)$ are the vertices $N$ and $S$ of $H(x)$.}
\label{fig:notation}
\end{figure}
We note that $H(x)$ is
uniquely defined because $T_{1n}$ contains no gentle edges and so $u(x)$ and
$\ell(x)$ cannot lie on sides $e$ and $w$, respectively, or on sides $w$ and
$e$, respectively, of $H(x)$.
\item
As we will soon see, function $H(x)$ has a specific growth pattern that depends
on what sides of $H(x)$ points $u(x)$ and $\ell(x)$ lie on. In order to
simplify our presentation, we define $H(x)$ when 
${\tt x}(p) \leq x < {\tt x}(c_1)$ and ${\tt x}(c_n) < x \leq {\tt x}(q)$ 
in a way that fits that pattern. Let $W_S^*$ be vertex
$W_S$ of $H_1=H({\tt x}(c_1))$ and let $H^*$ be the hexagon with $p$ and $W_S^*$
as its $W_N$ and $W_S$ vertices, respectively. Let $c^*$ be the center of $H^*$.
When ${\tt x}(p) \leq x \leq {\tt x}(c^*)$ we define $H(x)$ to be the hexagon 
whose center has abscissa $x$ and that has point $p$ as its $W_N$ vertex; 
we also define $u(x) = \ell(x) = p$. When 
${\tt x}(c^*) \leq x < {\tt x}(c_1)$ we define $H(x)$ to be the hexagon whose
center has abscissa $x$ and that has point $W_S^*$ as its $W_S$ vertex; we also
define $u(x) = \ell(x) = p$. We define
$H(x)$ when ${\tt x}(c_n) < x \leq {\tt x}(q)$ in a symmetric fashion
with $u(x) = \ell(x) = q$ in that case.
\end{itemize}

Next, we define ${\tt r}(x)$ to be the minimum radius of $H(x)$ (i.e., the
distance between the center of $H(x)$ to its $w$ side). Note that hexagons 
$H({\tt x}(p))$ and
$H({\tt x}(q))$ both have radius $0$ and define their centers to be $c_0 = p$
and $c_{n+1}=q$. We also extend the notation $T_{ij}$ to include $T_{10} = c_0$
and define $T(x) = T_{1i}$ when ${\tt x}(c_i) \leq x < {\tt x}(c_{i+1})$ and
$T({\tt x}(c_{n+1}))=T_{1n}$.
We define $N(x)$ and $S(x)$ to be the $N$ and $S$ vertex, respectively, of
$H(x)$. Finally, we define functions $w(x) = x-{\tt r}(x)$ and
$e(x) = x + {\tt r}(x)$ that keep track of the abscissa of the $w$ and $e$
sides, respectively, of $H(x)$ (refer to Fig.~\ref{fig:notation}). Note that
functions ${\tt r}(x), w(x), e(x)$ as well as functions ${\tt y}(N(x))$ and
${\tt y}(S(x))$ (the ordinates of $N(x)$ and $N(y)$, resp.) are continuous. 


\begin{figure}[!b]
\dNdS

\vspace{0.3cm}
\hspace{1.45cm} (a) \hspace{4.35cm} (b) \hspace{4.55cm} (c)
\caption{(a) The values of $p_N(o, x)$ are shown, for various points $o$
lying on the boundary of $H(x)$, as signed hexagon arc lengths. (b) The 
values of $p_S(o, x)$ are shown similarly. (c) The length of edge
$(u_{i-1},u_i)$, with $u_{i-1},u_i$ lying on the boundary of $H(x)=H_i$, 
is bounded by $p_N(u_{i-1}, x) - p_N(u_i,x)$.}
\label{fig:dNdS}
\end{figure}


For a point $o$ on a side of $H(x)$, we define functions $p_N(o, x)$ and 
$p_S(o, x)$ as the {\em signed} shortest distances around the perimeter of
$H(x)$ to the $N$ vertex and $S$ vertex, respectively, with sign 
$\sgn(x - x(o))$. As Fig.~\ref{fig:dNdS}-(a) and Fig.~\ref{fig:dNdS}-(b)
illustrate, these signs are positive for $o$ on the $n_w$, $w$, or $s_w$
sides of $H(x)$ and negative for $o$ on $n_e$, $e$, or $s_e$ sides. 
We omit $o$ and use the shorthand notation $p_N(x)$ if $o = u(x)$ and
$p_S(x)$ if $o = \ell(x)$. 

\begin{figure}
\begin{center}\UandL

\vspace{0.6cm}
(a) \hspace{6cm} (b)
\end{center}
\caption{(a) Definition of $U(x)$ and $L(x)$. For example, $U(x)$ for 
${\tt x}(c_i) \leq x < {\tt x}(c_{i+1})$ is the sum
of the length of the shortest path from $p$ to $u_i$ in $T_{1i}$ 
(illustrated as the red dashed path) and $p_N(x)$ (of negative value and 
represented as a red arrow). (b) Definition of $\bar{U}(x)$ and $\bar{L}(x)$. $\bar{U}(x) - p_N(x)$,
for example, is an upper bound (equal to the length of the 
sequence of red dashed
hexagon arcs going from $p$ to $u_i$) on the length
of the upper path $p, u_0, u_1, \dots, u_{i-1}, u_i$.}
\label{fig:UandL}
\end{figure}


The key functions $U(x)$ and $L(x)$, illustrated in Fig.~\ref{fig:UandL}-(a), 
are defined as follows for
${\tt x}(p) = {\tt x}(c_0) \leq x \leq {\tt x}(c_{n+1}) = {\tt x}(q)$:
%\begin{align*}
%& U(x) = d_{T_{1i}(p,u_i) + p_N(x) \mbox{ when } {\tt x}(c_i) \leq x < {\tt x}(c_{i+1}) & U({\tt x}(q)) = d_{T_{1n}}(p,q)) \\
%& L(x) = d_{T_{1i}}(p,l_i) + p_S(x) \mbox{ when } {\tt x}(c_i) \leq x < {\tt x}(c_{i+1}) & L({\tt x}(q)) = d_{T_{1n}}(p,q))
%\end{align*}
\begin{equation*}
U(x) = d_{T(x)}(p,u(x)) + p_N(x) \mbox{\hspace{2cm}} L(x) = d_{T(x)}(p,\ell(x)) + p_S(x)
\end{equation*}
Finally, we define the {\em potential function} $P(x)$ to be $U(x) + L(x)$. 
We note that
$P({\tt x}(q))$ is exactly twice the distance in $T_{1n}$ from $p$ to $q$. The 
main goal of this paper is to compute an upper bound for function $P(x)$.
We will do this by bounding its growth rate.

For ${\tt x}(c_i) < x < {\tt x}(c_{i+1})$, the terms 
$d_{T(x)}(p,u(x))$ and $d_{T(x)}(p,\ell(x))$ in the definitions of 
$U(x)$ and $L(x)$, respectively, are constant. 
This means that for such $x$ the rate of growth of
functions $U(x)$ and $L(x)$ is determined solely by the terms 
$p_N(x)$ and $p_S(x)$, respectively. 




We note that the length of each edge $(u_{i-1},u_i)$ 
(assuming $u_{i-1} \not= u_i$) can be bounded by the distance
from $u_{i-1}$ to $u_i$ when traveling clockwise along the sides of $H_i$.
This distance is exactly $p_N(u_{i-1}, {\tt x}(c_i)) - p_N(u_i,{\tt x}(c_i))$ as
illustrated in Fig.~\ref{fig:dNdS}-(c).
This, and a similar observation about each edge $(l_{i-1},l_i)$, motivates
the following definitions of functions $\bar{U}(x)$ and $\bar{L}(x)$ that we
will use to bound lengths of subpaths of the upper path $p,u_0, \dots,u_n,q$ 
and the lower path $p, l_0, \dots, l_n,q$ (see also Fig.~\ref{fig:UandL}-(b)). 
When ${\tt x}(c_i) \leq x < {\tt x}(c_{i+1})$ or $x = {\tt x}(c_{n+1}) = {\tt x}(q)$, we define
\begin{align*} 
\bar{U}(x) & = \sum_{j=1}^{i} (p_N(u({\tt x}(c_{j-1})), {\tt x}(c_{j})) - p_N(u({\tt x}(c_{j})),{\tt x}(c_{j}))) + p_N(x) \\
\bar{L}(x) & = \sum_{j=1}^{i} (p_S(\ell({\tt x}(c_{j-1})), {\tt x}(c_{j})) - p_S(\ell({\tt x}(c_{j})),{\tt x}(c_{j}))) + p_S(x)
\end{align*}
%\begin{align*} 
%\bar{U}(x) & = \begin{cases}
%p_N(x) & \mbox{ if } i=0 \\
%\sum_{j=0}^{i-1} (p_N(u_j, {\tt x}(c_{j+1})) - p_N(u_{j+1},{\tt x}(c_{j+1})))
%+ p_N(x) & \mbox{ if } i > 0
%\end{cases} \\
%\bar{L}(x) & = \begin{cases}
%p_S(x) & \mbox{ if } i=0 \\
%\sum_{j=0}^{i-1} (p_S(l_j, {\tt x}(c_{j+1})) - p_S(l_{j+1},{\tt x}(c_{j+1})))
%+ p_S(x) & \mbox{ if } i > 0
%\end{cases}
%\end{align*}
and let ${\bar P}(x)={\bar U}(x) + {\bar L}(x)$. We note that
${\bar P}(x)$ is an upper bound for $P(x)$ and that ${\bar P}({\tt x}(q))$
bounds the sum of the lengths of paths $p,u_0, \dots,u_n,q$ 
and $p, l_0, \dots, l_n,q$. Functions ${\bar U}(x)$ and ${\bar L}(x)$, 
just like $U(x)$ and $L(x)$, have rates of growth when
${\tt x}(c_i) < x < {\tt x}(c_{i+1})$ that are determined solely by the
last term ($p_N(x)$ or $p_S(x)$). We will show that functions 
$p_N(x)=p_N(u(x),x)$ and $p_S(x) = p_S(\ell(x),x)$ are monotonically
increasing piecewise linear functions whose rates of growth depend solely
on the sides of $H(x)$ that $u(x)$ and $\ell(x)$ lie on. In order to capture
precisely this rate of growth, we define the {\em transition
function} $t(x)$ to be {\em transition} $t_{ij}$ if $\ell(x)$ lies in the interior
of side $i$ and $u(x)$ lies in the interior of side $j$. We use the wildcard
notations $t_{\ast j}$ and $t_{i \ast}$ to refer to any
transition with $u(x)$ on side $j$ and $\ell(x)$ on side $i$, respectively,
of $H(x)$. 
%We can now define the growth rates of the various functions we will need:


\newpage
\begin{lemma}
\label{lem:growthrates}
%\label{lem:transitions}
Given the assumptions of Lemma~\ref{le:mainlemmaA}, for every 
$x \in [{\tt x}(p), {\tt x}(q)]$:
\begin{itemize}
\item $t(x)$, when defined, is one of
\begin{equation*}
t_{wn_w},t_{wn_e},t_{s_ww},  t_{s_wn_w}, t_{s_wn_e},  t_{s_we},  t_{s_ew},  t_{s_en_w},  t_{s_en_e},  t_{s_ee},  t_{en_w},  t_{en_e}.
\end{equation*}
\item Functions ${\tt y}(N(x)), {\tt y}(S(x)), {\tt r}(x), w(x), e(x)$
and, for $x \in [{\tt x}(c_i), {\tt x}(c_{i+1}))$ and \newline $i=0,\dots,n-1$,
functions $p_N(x)$, $p_S(x)$, and ${\bar P}(x)$ are all
piecewise linear functions with the following growth rates where defined:
\end{itemize}\vspace{-0.5cm}\begin{center}
{\tabulinesep=1.2mm
\begin{tabu} to \textwidth{X[2l]X[r]X[r]X[r]X[r]X[r]X[r]X[r]X[r]X[r]X[r]X[r]X[r]}
& \transitiononezero & \transitiononefive & \transitiontwoone & \transitiontwozero & \transitiontwofive & \transitiontwofour & \transitionthreeone & \transitionthreezero & \transitionthreefive & \transitionthreefour & \transitionfourzero & \transitionfourfive \\
{$\bm{t(x)}$} &  {$\bm{t_{wn_w}}$} & {$\bm{t_{wn_e}}$} &  {$\bm{t_{s_ww}}$} & {$\bm{t_{s_wn_w}}$} &  {$\bm{t_{s_wn_e}}$} & {$\bm{t_{s_we}}$} &  {$\bm{t_{s_ew}}$} & {$\bm{t_{s_en_w}}$} &  {$\bm{t_{s_en_e}}$} & {$\bm{t_{s_ee}}$} &  {$\bm{t_{en_w}}$} & {$\bm{t_{en_e}}$} \\ \hline
{$\bm{\frac{\Delta {\bar P}(x)}{\Delta x}}$ } & $\frac{6}{\sqrt{3}}$ & $\frac{8}{\sqrt{3}}$ & $\frac{6}{\sqrt{3}}$ & $\frac{4}{\sqrt{3}}$ & $\frac{4}{\sqrt{3}}$ & $\frac{8}{\sqrt{3}}$ & $\frac{8}{\sqrt{3}}$ & $\frac{4}{\sqrt{3}}$ & $\frac{4}{\sqrt{3}}$ & $\frac{6}{\sqrt{3}}$ & $\frac{8}{\sqrt{3}}$ & $\frac{6}{\sqrt{3}}$ \\ \hline
{$\bm{\frac{\Delta p_N(x)}{\Delta x}}$ } & $\frac{2}{\sqrt{3}}$ & $\frac{2}{\sqrt{3}}$ & $\frac{4}{\sqrt{3}}$ & $\frac{2}{\sqrt{3}}$ & $\frac{2}{\sqrt{3}}$ & $\frac{6}{\sqrt{3}}$ & $\frac{6}{\sqrt{3}}$ & $\frac{2}{\sqrt{3}}$ & $\frac{2}{\sqrt{3}}$ & $\frac{4}{\sqrt{3}}$ & $\frac{2}{\sqrt{3}}$ & $\frac{2}{\sqrt{3}}$ \\ \hline
{$\bm{\frac{\Delta {\tt y}(N(x))}{\Delta x}}$ } & $\frac{1}{\sqrt{3}}$ & $-\frac{1}{\sqrt{3}}$ & $\frac{3}{\sqrt{3}}$ & $\frac{1}{\sqrt{3}}$ & $-\frac{1}{\sqrt{3}}$ & $-\frac{5}{\sqrt{3}}$ & $\frac{5}{\sqrt{3}}$ & $\frac{1}{\sqrt{3}}$ & $-\frac{1}{\sqrt{3}}$ & $-\frac{3}{\sqrt{3}}$ & $\frac{1}{\sqrt{3}}$ & $-\frac{1}{\sqrt{3}}$ \\ \hline
{$\bm{\frac{\Delta p_S(x)}{\Delta x}}$ } & $\frac{4}{\sqrt{3}}$ & $\frac{6}{\sqrt{3}}$ & $\frac{2}{\sqrt{3}}$ & $\frac{2}{\sqrt{3}}$ & $\frac{2}{\sqrt{3}}$ & $\frac{2}{\sqrt{3}}$ & $\frac{2}{\sqrt{3}}$ & $\frac{2}{\sqrt{3}}$ & $\frac{2}{\sqrt{3}}$ & $\frac{2}{\sqrt{3}}$ & $\frac{6}{\sqrt{3}}$ & $\frac{4}{\sqrt{3}}$ \\ \hline
{$\bm{\frac{\Delta {\tt y}(S(x))}{\Delta x}}$ } & $-\frac{3}{\sqrt{3}}$ & $-\frac{5}{\sqrt{3}}$ & $-\frac{1}{\sqrt{3}}$ & $-\frac{1}{\sqrt{3}}$ & $-\frac{1}{\sqrt{3}}$ & $-\frac{1}{\sqrt{3}}$ & $\frac{1}{\sqrt{3}}$ & $\frac{1}{\sqrt{3}}$ & $\frac{1}{\sqrt{3}}$ & $\frac{1}{\sqrt{3}}$ & $\frac{5}{\sqrt{3}}$ & $\frac{3}{\sqrt{3}}$ \\ \hline
{$\bm{\frac{\Delta {\tt r}(x)}{\Delta x}}$ } & $1$ & $1$ & $1$ & $\frac{1}{2}$ & $0$ & $-1$ & $1$ & $0$ & $-\frac{1}{2}$ & $-1$ & $-1$ & $-1$ \\ \hline
{$\bm{\frac{\Delta w(x)}{\Delta x}}$ } & $0$ & $0$ & $0$ & $\frac{1}{2}$ & $1$ & $2$ & $0$ & $1$ & $\frac{3}{2}$ & $2$ & $2$ & $2$ \\ \hline
{$\bm{\frac{\Delta e(x)}{\Delta x}}$ } & $2$ & $2$ & $2$ & $\frac{3}{2}$ & $1$ & $0$ & $2$ & $1$ & $\frac{1}{2}$ & $0$ & $0$ & $0$ \\ \hline
\end{tabu}}
\end{center}
\end{lemma}

\begin{figure}[!b]
\growth

\hspace{2cm} (a) \hspace{3.8cm} (b) \hspace{4.2cm} (c)
\caption{Constructions demonstrating Lemma~\ref{lem:growthrates} for
transitions (a) $t_{s_wn_e}$, (b) $t_{wn_w}$, and (c) $t_{s_we}$. In all three
cases the growths shown are with respect to $\Delta x = 1$. $\Delta {\tt r}(x)$
can be obtained from $\frac{1}{2}(\Delta e(x) - \Delta w(x))$ and
$\Delta P(x)$ from $\Delta p_N(x) + \Delta p_S(x)$. Note that the growth
$\Delta p_N(x)$ when $t(x)$ is a single transition can be represented using
a piecewise linear curve of length $\Delta p_N(x)$.}
\label{fig:growth}
\end{figure}

\begin{proof}
The first part follows from the definition of a regular linear sequence of
triangles. The growth rates for transitions
$t_{s_wn_e}$, $t_{wn_w}$, and $t_{s_we}$ follow from elementary
geometric constructions illustrated in Fig.~\ref{fig:growth}. The constructions
for the remaining transitions are similar.
\end{proof}



%We can now prove Lemma~\ref{lem:rotation}:
%\begin{proof}[Proof of Lemma~\ref{lem:rotation}]
%The growth rate of $P(x)$ is the growth rate of $p_N(x)+p_S(x)$ which
%is at most $\frac{8}{\sqrt{3}}$ by Lemma~\ref{lem:growthrates}. This
%means that there is a path from $p$ to $q$ in $T_{1n}$ whose length is at
%most $\frac{4}{\sqrt{3}}d_{\tt x}(p,q) \leq (\sqrt{3}+\mbox{slope}(p,q))d_x(p,q)$.
%\end{proof}

We now consider the behavior of $U(x)$, $L(x)$, ${\bar U}(x)$, and ${\bar L}(x)$
at $x = {\tt x}(c_i)$ for $i=1,\dots,n+1$. Note that 
\begin{align*}
\bar{U}({\tt x}(c_i)) & = \sum_{j=1}^{i} (p_N(u({\tt x}(c_{j-1})), {\tt x}(c_{j})) - p_N(u({\tt x}(c_{j})),{\tt x}(c_{j}))) + p_N(u_i, {\tt x}(c_i)) \\
& = \sum_{j=1}^{i-1} (p_N(u({\tt x}(c_{j-1})), {\tt x}(c_{j})) - p_N(u({\tt x}(c_{j})),{\tt x}(c_{j}))) + p_N(u({\tt x}(c_{i-1})), {\tt x}(c_i))
%\bar{U}({\tt x}(c_i)) & = \sum_{j=0}^{i-1} (p_N(u_j, {\tt x}(c_{j+1})) - p_N(u_{j+1},{\tt x}(c_{j+1}))) + p_N(u_i, {\tt x}(c_i)) \\
%& = \sum_{j=0}^{i-2} (p_N(u_j, {\tt x}(c_{j+1})) - p_N(u_{j+1},{\tt x}(c_{j+1}))) +
%(p_N(u_{i-1}, {\tt x}(c_{i}))
\end{align*}
which is the limit for ${\bar U}(x)$ when $x \rightarrow {\tt x}(c_i)$ from 
the left.
So ${\bar U}(x)$ is continuous from ${\tt x}(c_0)$ to ${\tt x}(c_{n+1})$ and,
similarly, so is ${\bar L}(x)$ and therefore ${\bar P}(x)$ as well.
Since, by Lemma~\ref{lem:growthrates}, ${\bar P}({\tt x}(q))$ is bounded by
$\frac{8}{\sqrt{3}}d_x(p,q)$, it follows that the sum of the lengths of the upper path
$p,u_0, \dots,u_n,q$ and the lower path $p, l_0, \dots, l_n,q$ is at most 
$\frac{8}{\sqrt{3}}d_x(p,q)$. Therefore, one of the two paths has length
bounded by $\frac{4}{\sqrt{3}}d_x(p,q)$ which proves the Technical Lemma:

\begin{proof}[Proof of (Technical) Lemma~\ref{le:mainlemmaA}]
$d_{T_{1n}}(p,q) \leq \frac{1}{2}{\bar P}({\tt x}(q)) \leq \frac{4}{\sqrt{3}}d_x(p,q)$
by Lemma~\ref{lem:growthrates}.% and Lemma~\ref{lem:discontinuity}.
\end{proof}

In order to prove the stronger bound of Amortization Lemma~\ref{le:mainlemmaB},
in addition to edges of the upper and lower path we will need to include edges
$(l_i,u_i)$ in the analysis and obtain a tighter bound on $P(x)$.
For this reason, we complete the growth rate analysis of
functions $U(x)$, $L(x)$, and $P(x)$. We show that they are not necessarily 
continuous at $x = {\tt x}(c_i)$ but, when discontinuous, they do not increase:
\begin{lemma} 
\label{lem:discontinuity}
Functions $U(x)$, $L(x)$, and $P(x)$, when discontuous at $x={\tt x}(c_i)$
for some $i = 1, \dots, n$, do not increase at $x$. 
\end{lemma}
\begin{proof}
%Consider ${\tt x}(c_i)$ for some $i=1,\dots,n$. 
First note that either 
$u_i \not= u_{i-1}$ and $u({\tt x}(c_{i-1})) = u_{i-1}$, or
$u_i = u_{i-1}$ and $i > 1$ and $u({\tt x}(c_{i-1})) = u_{i-1}$, or 
$u_i = u_{i-1}$ and $i = 1$ and $u({\tt x}(c_{i-1})) = p$.  

In the second case, we note that $l_{i-1} \not= l_i$ and that point $l_i$
cannot be on the shortest path from $p$ to $u_i$ in $T_{1i}$. Therefore 
$U({\tt x}(c_i))=d_{T_{1i}}(p,u_i) + p_N(u_i, {\tt x}(c_i)) = d_{T_{1{i-1}}}(p,u_{i-1}) + p_N(u_{i-1}, {\tt x}(c_i))$. The last term is the limit for $U(x)$ when $x \rightarrow {\tt x}(c_i)$ from
the left so $U(x)$ is continuous around $x = {\tt x}(c_i)$ which completes the 
proof for this case.

For the first and third cases we set $u^* =  u({\tt x}(c_{i-1}))$. Then
\begin{align*} 
U({\tt x}(c_i)) & = d_{T_{1i}}(p,u_i) + p_N(u_i, {\tt x}(c_i)) \\ 
       & \leq d_{T_{1(i-1)}}(p,u^*) + d_2(u^*,u_i) + p_N(u_i, {\tt x}(c_i)) \\ 
       & \leq d_{T_{1(i-1)}}(p,u^*) + p_N(u^*,{\tt x}(c_i)) - p_N(u_i,{\tt x}(c_i)) + p_N(u_i, {\tt x}(c_i)) \\ \nonumber
       & = d_{T_{1(i-1)}}(p,u^*) + p_N(u^*, {\tt x}(c_i)) \nonumber
\end{align*}
%\begin{align*} 
%U({\tt x}(c_i)) & = d_{T_{1i}}(p,u_i) + p_N(u_i, {\tt x}(c_i)) \nonumber \\
%       & \leq \min \begin{cases} \nonumber 
%                 d_{T_{1(i-1)}}(p,u_{i-1}) + d_2(u_{i-1},u_i) + p_N(u_i, {\tt x}(c_i)) \\ 
%                 d_{T_{1(i-1)}}(p,l_{i-1}) + d_2(l_{i-1},u_i) + p_N(u_i, {\tt x}(c_i)) \\
%                   \end{cases} \\ %\label{eq:frombelow}
%       & \leq d_{T_{1(i-1)}}(p,u_{i-1}) + p_N(u_{i-1},{\tt x}(c_i)) - p_N(u_i,{\tt x}(c_i)) + p_N(u_i, {\tt x}(c_i)) \\ \nonumber
%       & = d_{T_{1(i-1)}}(p,u_{i-1}) + p_N(u_{i-1}, {\tt x}(c_i)) \nonumber
%\end{align*}
The last term is the limit for $U(x)$ when $x \rightarrow {\tt x}(c_i)$
from the left and so the claim holds for $U(x)$.
The claim for $L(x)$ holds using equivalent arguments, and the claim for 
$P(x)$ follows from $P(x) = L(x) + U(x)$.
\end{proof}

Before we end this section, we note that we have not defined $t(x)$ at values
of $x$ when $\ell(x)$ or $u(x)$
is a vertex of $H(x)$, which is when $p_S(x)$ or $p_N(x)$, and thus
$\bar{L}(x)$ or $\bar{U}(x)$, respectively, $L(x)$ or $U(x)$, respectively, 
and $P(x)$ are not smooth
and differentiable. In what follows, for clarity of presentation we will
sometimes abuse our definition of $t(x)$ to include such points. Since there
are only a finite number of such points, they do not affect our analysis.




}

\section{Proof of (Amortization) Lemma~\ref{le:mainlemmaB}}
\label{sec:proofB}
\iftoggle{abstract}
{The proof of the lemma builds on the framework discussed in the previous
section and on a careful analysis of the growth rates of $p_N$ and $p_S$
when $T_{1n}$ contains no gentle path.  We show that in that case
the average growth rate of $U+L$ is at most $2\left(\cT\right)$. 

Our main approach is to spread (i.e., amortize) the 
``extra'' $\frac{2}{\sqrt{3}}$ of the $\frac{8}{\sqrt{3}}$ growth rate
over wider intervals of time that, as we show, include time intervals during
which the growth rate is smaller. To achieve our tight bound of 
$2\left(\cT\right)$, however, we need to do more and also include 
``cross-edges'' $(l_i,u_i)$ as illustrated in Fig.~\ref{fig:pathcost}.


\begin{figure}[h]
\center{\anotherpathcost}

\caption{Illustrated is a situation in which the growth rate of $p_S(x)$ is 
$\frac{6}{\sqrt{3}}$ between $x=x_l$ and $x=x_r$. In that case 
the growth rate of $p_N(x)$ is $\frac{2}{\sqrt{3}}$. For large enough such
intervals $[x_l,x_r]$,  
the path $l_i,u_i,u_{i+1},\dots,u_j,l_j$ is a shortcut for 
$l_i, l_{i+1}, \dots,l_j$ and therefore $L(x_r)$ is smaller than what the growth
rate of $p_S(x)$ would indicate. The stretch factor bound
we obtain is the result of a min-max optimization between the two subpaths 
from $l_i$ to $l_j$, and it is tight as we show in Section~\ref{sec:conclusion}.
% (a) Illustration of path $P$: $l_i,u_i,u_{i+1},\dots,u_j,l_j$.
% (b) $-p_S(l_i,x_l) + |l_iu_i|$ is maximized when $l_i$ is the SW corner
% of $H(x_l)$.
}
\label{fig:pathcost}
\end{figure}

%The growth rate of $U(x)+L(x)$ at a
%particular time $x$ (we will find it useful to use the time intuition and think of
%$x$ as time going from time ${\tt x}(p)$ to time ${\tt x}(q)$) 
%is one of $\frac{4}{\sqrt{3}}$, $\frac{6}{\sqrt{3}}$, or
%$\frac{8}{\sqrt{3}}$. 
%Our goal is to spread (i.e., amortize) the 
%``extra'' $\frac{2}{\sqrt{3}}$ of the $\frac{8}{\sqrt{3}}$ growth rate
%over wider intervals of time that, as we show, include time intervals during
%which the growth rate is smaller. To achieve our tight bound of
%$2\left(\cT\right)$ 


% to show that the
%average growth rate of $P(x)$ is at  most $2\left(\cT\right)$. 
}
{\begingroup
\def\thetheorem{\ref{le:mainlemmaB}}
\begin{lemma}[The Amortization Lemma]
%\label{le:mainlemmaB}
Let $T_{1n}$ be a regular linear sequence with respect to line $st$ with slope
$m_{st}$. If $0 < m_{st} < \frac{1}{\sqrt{3}}$ and if $T_{1n}$ contains no
gentle path then there is a path in $T_{1n}$ from 
the left induction vertex $p$ of $T_1$ to the right induction vertex $q$ of
$T_n$ of length at most  $\left(\cT\right) d_x(p,q)$.
\end{lemma}
\addtocounter{theorem}{-1}
\endgroup

The proof of the lemma builds on the framework developed in the previous
section and on a careful analysis of the growth rates shown in
Lemma~\ref{lem:growthrates} when $T_{1n}$ contains no gentle path. 
We show that in that case
the average growth rate of $P(x)=U(x)+L(x)$ is at most $2\left(\cT\right)$.

As Lemma~\ref{lem:growthrates} demonstrates, the growth rate of $P(x)$ at a
particular time $x$ (we find it useful to use the time intuition and think of
$x$ as time going from time ${\tt x}(p)$ to time ${\tt x}(q)$) 
is one of $\frac{4}{\sqrt{3}}$, $\frac{6}{\sqrt{3}}$, or
$\frac{8}{\sqrt{3}}$. We will refer to transitions 
$t_{wn_e},t_{s_we},t_{s_ew},t_{en_w}$---transitions for which
the growth rate $\frac{\Delta P(x)}{\Delta x}$ is $\frac{8}{\sqrt{3}}$---as
{\em bad}. Note that when $t(x)$ is not bad, the growth rate of $P(x)$ is at
most $\frac{6}{\sqrt{3}}$, well under the desired growth rate of $2\left(\cT\right)$. 
Our goal is to, when possible, spread (i.e., amortize) the 
``extra'' $\frac{2}{\sqrt{3}}$ of the growth rate
of $P(x)$ when $t(x)$ is bad over wider intervals of time. % to show that the
%average growth rate of $P(x)$ is at  most $2\left(\cT\right)$. 
To do this we define intervals of time
during which a particular bad transition takes place %.
%monitor what intervals
%of time we are amortizing a particular extra growth over. 
%To help us with this
%we introduce the following definitions, 
(see also Fig.~\ref{fig:badintervals}):
% To aid discussion we will refer to $t_8$ and $t_9$ upper bad transitions and $t_{wn_w}$ and $t_{ww}$ as lower bad transitions.  Similarly, we will refer to $t_8$ and $t_{ww}$ as left bad transitions and $t_9$ and $t_{wn_w}$ as right bad transitions.

%From now on we will no longer refer to the original centers of Hexagons, only positions between $0$ and $p_k$.  We will borrow standard notation and take $[x_i,x_j]$ to be the interval from $x_i$ to $x_j$.  Let $\Delta P(x_i,x_j)= P(x_i)-P(x_j)$.  We must control the bad transitions by amortizing their cost over better transitions.  To this end we define bad sections to group bad transitions to be more easily handled.  To keep things simple we will think of the transition at $x=0$ to be of type $t_7$ and the transition at $x=x(p_k)$ to be $t_{wn_w}$ each being instantaneous and causing no bad growth. 

\begin{definition}
Given $x_l, x_r$ such that ${\tt x}(p) \leq x_l < x_r \leq {\tt x}(q)$, the interval
$[x_l,x_r]$ is a 
\begin{itemize}
\item $t_{wn_e}$-interval if $t(x_l)= t_{wn_e}$ and $t(x) \not= t_{\ast w}$ for all
$x_l < x < x_r$ and a strict $t_{wn_e}$-interval if, in addition,
$t(x) \not= t_{en_w}, t_{\ast e}$ when $x_l < x < x_r$.
\item $t_{s_ew}$-interval if $t(x_l)= t_{s_ew}$ and $t(x) \not= t_{w \ast}$ for all
$x_l < x < x_r$ and a strict $t_{s_ew}$-interval if, in addition,
$t(x) \not= t_{s_we}, t_{e \ast}$ when $x_l < x < x_r$.
\item $t_{s_we}$-interval if $t(x_r)= t_{s_we}$ and $t(x) \not= t_{e \ast}$ for all
$x_l < x < x_r$ and a strict $t_{s_we}$-interval if, in addition,
$t(x) \not= t_{s_ew}, t_{w \ast}$ when $x_l < x < x_r$
\item $t_{en_w}$-interval if $t(x_r)= t_{en_w}$ and $t(x) \not= t_{\ast e}$ for all
$x_l < x < x_r$ and a strict $t_{en_w}$-interval if, in addition,
$t(x) \not= t_{wn_e}, t_{\ast w}$ when $x_l < x < x_r$
\end{itemize}
\end{definition}

\begin{figure}[!b]
\begin{center}
\badintervals
\end{center}

\caption{Illustrations of (left) $t_{wn_e}$- and $t_{s_ew}$-intervals and (right)
$t_{s_we}$- and $t_{en_w}$-intervals. In a $t_{wn_e}$-interval $[x_l,x_r]$, for
example, $t(x) \not= t_{\ast w}$; if the interval is strict then 
$t(x) \not= t_{en_w}, t_{\ast e}$ as well.}
\label{fig:badintervals}
\end{figure}



Note that if $t(x) = t_{ij}$ is bad for some $x$ in $[{\tt x}(p), {\tt x}(q)]$
then a strict $t_{ij}$-interval contains $x$.
We refer to $t_{wn_e}$- and $t_{s_ew}$-intervals as {\em left intervals} and to
$t_{s_we}$- and $t_{en_w}$-intervals as {\em right intervals}. Note that left
intervals do not intersect each other and neither do the right intervals.
A maximal strict left interval can intersect at most one maximal strict right
interval and vice versa and when that is the case the left interval is to the
left of the right interval.
We will take advantage of the limited interaction between maximal
strict intervals 
to amortize the bad growth taking place within a strict interval over
intervals of time
that do not overlap or, if they do, overlap in a restricted way.


We introduce notation that will help us keep track of the relative horizontal
positions of $u(x)$ and $\ell(x)$: the {\em forward} abcsissa 
$f(x) = \max\{{\tt x}(\ell(x)), {\tt x}(u(x))\}$ and the {\em back} abcsissa
$b(x) = \min\{{\tt x}(\ell(x)), {\tt x}(u(x))\}$. The following facts will
be used heavily without reference throughout this section:
\begin{proposition}
\label{lem:bandf}
Given the assumptions of Lemma~\ref{le:mainlemmaB}, for $i = 1,\dots,n$:
\renewcommand{\labelenumi}{(\alph{enumi})}
\begin{enumerate}
%\item If $l_{i-1}=l_i$ is the base
%vertex of $T_i$ then $u_{i-1}, u_i$ lie on sides $w, n_w$ or $w, n_e$ or
%$n_w, n_e$ or $n_w,e$ or $n_e, e$ of $H_i$; if $u_{i-1}=u_i$ is the base
%vertex of $T_i$ then $l_{i-1}, l_i$ lie on sides $w, s_w$ or $w, s_e$ or
%$s_w, s_e$ or $s_w,e$ or $s_e, e$ of $H_i$. 
\item ${\tt x}(l_{i-1}) \leq {\tt x}(l_i)$ and ${\tt x}(u_{i-1}) \leq {\tt x}(u_i)$.
\end{enumerate}
and for ${\tt x}(p) \leq x_l < x_r \leq {\tt x}(q)$:
\begin{enumerate}
\addtocounter{enumi}{1}
\item $u(x_l) \leq u(x_r)$ and $\ell(x_l) \leq \ell(x_r)$.
\item $w(x_l) \leq w(x_r)$ and $e(x_l) \leq e(x_r)$.
\item $b(x_l) \leq b(x_r)$ and $f(x_l) \leq f(x_r)$.
\end{enumerate}
\end{proposition}

\begin{proof}
Note that if $l_{i-1}=l_i$ is the base vertex of $T_i$ then $u_{i-1}, u_i$ lie 
on sides $w, n_w$ or $w, n_e$ or $n_w, n_e$ or $n_w,e$ or $n_e, e$ of $H_i$; 
if $u_{i-1}=u_i$ is the base vertex of $T_i$ then $l_{i-1}, l_i$ lie on sides
$w, s_w$ or $w, s_e$ or $s_w, s_e$ or $s_w,e$ or $s_e, e$ of $H_i$.
% Part {\it (a)} follows from the definition of a linear sequence of triangles
%and Lemma~\ref{lem:props} and {\it (b)} follows from {\it (a)}. 
Part {\it (b)} follows from this and parts {\it (c)}
and {\it (e)} follow from  {\it (b)}. Part {\it (d)} follows from 
Lemma~\ref{lem:growthrates}.
\end{proof}

%
% Finally, we define $\theta_f(x) = f(x) - x$ and 
% $\theta_b(x) = x - b(x)$. 

Finally, we define key functions $\delta_f(x) = e(x) - f(x)$ and 
$\delta_b(x) = b(x) - w(x)$ that will be used when $t(x)$ is bad.
Note that when $t(x)= t_{wn_e},t_{s_ew}$ we have 
$\delta_f(x) \leq {\tt r}(x)$ and when $t(x)=t_{s_we},t_{en_w}$ 
then $\delta_b(x) \leq {\tt r}(x)$. 

We now prove several technical lemmas that we need. Unlike our bounds for
${\bar U}(x)$ and ${\bar L}(x)$ that only made use of edges of the upper and
lower path, we will sometimes make use of {\em cross-edges} $(u_i,l_i)$ in 
computing tighter bounds for $U(x)$ or $L(x)$, and therefore $P(x)$. We will,
in particular, use edges  $(\ell(x),u(x))$ for values of $x$ when $t(x)$ is bad.
To motivate the next lemma which bounds the cost of using such edges, consider,
for example, a  $t_{wn_e}$-interval starting at $x_l$ and the edge 
$(\ell(x_l), u(x_l)) = (l_{i-1}=l_i, u_i)$ between the $w$ and $n_e$ sides of 
$H(x)$. Then
\begin{align*} 
U(x_l) & = d_{T_{1i}}(p,u_i) + p_N(u_i, x_l) \nonumber \\
       & \leq \min \begin{cases}
                 d_{T_{1(i-1)}}(p,u_{i-1}) + d_2(u_{i-1},u_i) + p_N(u_i, x_l) \\ 
                 d_{T_{1(i-1)}}(p,l_{i-1}) + d_2(l_{i-1},u_i) + p_N(u_i, x_l) \\
                   \end{cases} \\ \nonumber
       & \leq L(x_l) - p_S(l_{i-1},x_l) + d_2(l_{i-1},u_i) + p_N(u_i, x_l)
\end{align*}

The term $- p_S(l_{i-1},x_l) + + d_2(l_{i-1},u_i) + p_N(u_i, x_l)$ can be seen
as the cost of using edge $(l_{i-1},u_i)$. The following lemma provides a bound
on this cost in terms of $\delta_f(x)$ as well as the cost of using other 
{\em bad cross-edges} (see also Fig.~\ref{fig:gentlepath}-(a)):
\begin{lemma}
\label{lem:switchpath}
Given the assumptions of Lemma~\ref{le:mainlemmaB}, let $t(x)$ be bad for some
$x \in [{\tt x}(p), {\tt x}(q)]$.% and let $l_i = \ell(x)$ and $u_i = u(x)$. 
%, l_j = \ell(x_r),$ and $u_j = u(x_r)$ for some $i \leq j$.
%If $[x_l,x_r]$ is
%
\begin{itemize}
\item If $t(x) = t_{wn_e}$ then
$-p_S(x) + d_2(\ell(x),u(x)) + p_N(x) \leq \left(2 - \frac{2}{\sqrt{3}}\right)\delta_f(x)$
\item If $t(x) = t_{s_ew}$ then
$p_S(x) + d_2(\ell(x),u(x)) - p_N(x) \leq \left(2 - \frac{2}{\sqrt{3}}\right)\delta_f(x)$
\item If $t(x) = t_{s_we}$ then
$p_S(x) + d_2(\ell(x),u(x)) - p_N(x) \leq \left(2 - \frac{2}{\sqrt{3}}\right)\delta_b(x)$
\item If $t(x) = t_{en_w}$ then
$-p_S(x) + d_2(\ell(x),u(x)) + p_N(x) \leq \left(2 - \frac{2}{\sqrt{3}}\right)\delta_b(x)$
\end{itemize}
\end{lemma}
%
\begin{proof}
Assuming that $t(x) = t_{wn_e}$ (the other cases follow
by symmetry),
%
% Also, we note that $-p_S(l_i,x_l) + d_2(l_i, u_i)$ is maximized
% when $l_i$ is the SW corner of $H(x_l)$, as illustrated in Fig.~\ref{fig:gentlepath}-(b).
% %	
% Assuming that is the case, we let $\theta = \angle u_i l_i N(x_l)$, and
%
we let $w$ to be the SW corner of $H(x)$ and let $\theta = \angle u_i w N(x)$
as illustrated in Fig.~\ref{fig:gentlepath}-(a). We then
%
obtain
$p_S(x)=p_S(\ell(x),x) \geq \frac{2}{\sqrt{3}}{\tt r}(x)$,
$d_2(\ell(x), u(x)) \leq \frac{2}{\cos(\theta)}{\tt r}(x)$, and
$p_N(x)=p_N(u(x),x) = -2\tan(\theta){\tt r}(x)$.
%	
Noting that
$0 \leq \theta \leq \frac{\pi}{6}$ and
	using
$\frac{1}{\cos(\theta)} \leq 1 + (2-\sqrt{3})\tan(\theta)$ when
$\theta \in [0,\frac{\pi}{6}]$, 
we have
\begin{align*}
-p_S(x) + d_2(\ell(x),u(x)) + p_N(x) & \leq \left(-\frac{2}{\sqrt{3}} + \frac{2}{\cos(\theta)} -2\tan(\theta)\right){\tt r}(x)\\
& \leq \left(-\frac{2}{\sqrt{3}} + 2 + 2(2-\sqrt{3})\tan(\theta) -2\tan(\theta)\right){\tt r}(x) \\
& = \left(2-\frac{2}{\sqrt{3}}\right)\left(1-\sqrt{3}\tan(\theta)\right){\tt r}(x)
\end{align*}
Since ${\tt r}(x) - \delta_f(x) = \sqrt{3}\tan(\theta){\tt r}(x)$, the proof
follows from the last inequality.
\end{proof}

\iftoggle{abstract}
{\begin{figure}[!b]}
{\begin{figure}}
\gentlepath

\hspace{1.9cm}(a) \hspace{2.6in} (b)

\caption{(a) Illustration and proof of Lemma~\ref{lem:switchpath} in the case
of $t(x) = t_{wn_e}$. (b) 
Lemma~\ref{lem:gentlepath} in the case of a $t_{wn_e}$-interval 
$[x_l,x_r]$. The dotted curve from $N(x_l)$ to $N(x_r)$ consists of a
sequence of piecewise linear curves each representing the growth rate
$\Delta p_N(x)$ over an interval $\Delta x$ during which $t(x)$ is a fixed
transition. A few examples of such piecewise linear curves are shown in bold in
Fig.~\ref{fig:growth}. The dotted curve has total length 
$\bar{U}(x_r) - \bar{U}(x_l)$ and its slope is $-\frac{1}{\sqrt{3}}$ exactly
when $t(x)$ is either $t_{\ast n_e}$ (which includes bad transition $t_{wn_e}$) 
or $t_{\ast e}$. Lemma~\ref{lem:gentlepath} states that the time (shown with
thick blue segments) spent in those transitions within interval $[x_l, x_r]$
is bounded by ${\tt r}(x_l) + (\sqrt{3} - 1)\delta_f(x_l)$, if
$T_{1n}$ contains no gentle path.}
\label{fig:gentlepath}
\end{figure}


The following important lemma bounds the amount of time
certain transitions, and especially bad transition $t_{ij}$, can take place 
within a
$t_{ij}$-interval $[x_l,x_r]$ if $T_{1n}$ contains no gentle path. By {\it
``amount of time a transition $t_{ij}$ takes place in an interval $[x_l,x_r]$''}
we mean the Lebesgue measure of the set 
$\{x \in [x_l,x_r]: t(x) = t_{ij}\}$
which is the union of a finite number of disjoint open intervals inside
$[x_l,x_r]$. The lemma we state below captures 
the following insight. When $t(x)$ is a bad transition
and the growth rate of $P(x)$ is $\frac{8}{\sqrt{3}}$ then
by Lemma~\ref{lem:growthrates} one of $U(x)$ or $L(x)$ has growth rate 
$\frac{2}{\sqrt{3}}$. More specifically, if, say, $t(x) = t_{wn_e}$ then 
$\frac{\Delta U(x)}{\Delta x} = \frac{2}{\sqrt{3}}$, meaning that the
upper path fragment $u(x_l)=u_i, u_{i+1}, \dots, u(x_r)$ has a
``very gentle'' average slope. Therefore if $t(x) = t_{wn_e}$ for a sufficiently
long enough time within $t_{wn_e}$-interval $[x_l,x_r]$ then the path 
$\ell(x_l), u(x_l), \dots, u(x_r)$ would have to be a gentle path.
We actually need the contrapositive of this insight (see also Fig.~\ref{fig:gentlepath}-(b)):

\begin{lemma}
\label{lem:gentlepath}
Given the assumptions of Lemma~\ref{le:mainlemmaB},
let ${\tt x}(p) \leq x_l \leq x_r \leq {\tt x}(q)$,
and let $z_{ij}$ be the amount of time within interval $[x_l,x_r]$ spent
in transition $t_{ij}$. If $[x_l,x_r]$ is
%
% \begin{itemize}
% \item a $t_{wn_e}$-interval then
% $z_{\ast n_e} + z_{\ast e} \leq \theta_f(x_l) + \sqrt{3}({\tt r}(x_l) - \theta_f(x_l))$
% \item a $t_{s_ew}$-interval then
% $z_{s_e \ast} + z_{e \ast} \leq \theta_f(x_l) + \sqrt{3}({\tt r}(x_l) - \theta_f(x_l))$
% \item a $t_{s_we}$-interval then
% $z_{w \ast} + z_{s_w \ast} \leq \theta_b(x_r) + \sqrt{3}({\tt r}(x_r) - \theta_b(x_r))$
% \item a $t_{en_w}$-interval then
% $z_{\ast n_w} + z_{\ast w} \leq \theta_b(x_r) + \sqrt{3}({\tt r}(x_r) - \theta_b(x_r))$
% \end{itemize}
%
\begin{itemize}
\item a $t_{wn_e}$-interval then
$z_{\ast n_e} + z_{\ast e} \leq {\tt r}(x_l) + (\sqrt{3} - 1)\delta_f(x_l)$
\item a $t_{s_ew}$-interval then
$z_{s_e \ast} + z_{e \ast} \leq {\tt r}(x_l) + (\sqrt{3} - 1)\delta_f(x_l)$
\item a $t_{s_we}$-interval then
$z_{w \ast} + z_{s_w \ast} \leq {\tt r}(x_r) + (\sqrt{3} - 1)\delta_b(x_r)$
\item a $t_{en_w}$-interval then
$z_{\ast n_w} + z_{\ast w} \leq {\tt r}(x_r) + (\sqrt{3} - 1)\delta_b(x_r)$
\end{itemize}
%
where wildcard notations $z_{\ast j}$ and $z_{i \ast}$ refer to the amount
of time spent in transitions $t_{\ast j}$ and $t_{i \ast}$, respectively.
\end{lemma}

\begin{proof}
We assume that $[x_l,x_r]$ is a $t_{wn_e}$-interval (the other cases follow
by symmetry). We also assume that $t(x_r) \not= t_{\ast n_w}$ implying that
$u(x_r)$ lies either on the N vertex,
side $n_e$, the NE vertex, or side $e$ of $H(x_r)$ (or, with some abuse of
the definition of $t(x)$, $t(x_r) = t_{\ast n_e},t_{\ast e})$.  If that is not the
case, we simply consider the
shorter interval $[x_l,x_q]$ where $x_q$ is such that $u(x_{q})$ is the N
vertex of $H(x_q)$ and $t(x) = t_{\ast n_w}$ for $x_q < x < x_r$.




Consider the path in $T_{1n}$ from $\ell(x_l)$ to $u(x_r)$ that starts with
the edge $(\ell(x_l), u(x_l)) = (l_i,u_i)$ and then visits the vertices
$u_i, u_{i+1}, \dots, u_{i+s} = u(x_r)$ in order
(refer to Fig.~\ref{fig:gentlepath}-(b)). 
Since $T_{1n}$ contains no gentle paths, the length of this path is at least
$\sqrt{3}d_x(l_i,u_{i+s}) - ({\tt y}(u_{i+s})-{\tt y}(l_i))$ and so:
\begin{align}
\label{eq:nogentle}
d_2(l_i,u_i)+ \sum_{t=i}^{i+s-1} d_2(u_t,u_{t+1}) + ({\tt y}(u_{i+s})-{\tt y}(l_i)) & \geq \sqrt{3}d_x(l_i,u_{i+s}) 
\end{align}
%
%
% The length of edge $(l_i, u_i)$ is at most the length of
% $[wu_i]$ where $w$ is the SW vertex of $H(x_l)$, and the length of
% $[wu_i]$ is at most $\frac{2{\tt r}(x_l)}{\cos(\Theta)}$, where $\Theta$ is the
% angle $\angle N(x_l) w u_i$. So:
% \begin{equation}
% \label{eq:crossedge}
% d_2(l_i,u_i) \leq \frac{2{\tt r}(x_l)}{\cos(\Theta)}  
% \end{equation}
%  Note that $0 \leq \Theta \leq \frac{\pi}{6}$. Note also that $d_y(l_i, N(x_l)) \leq \sqrt{3} {\tt r}(x_l)$ and so, by
% Lemma~\ref{lem:growthrates},
% \begin{align}
% \label{eq:crossvertical}
% {\tt y}(u_{i+s})-{\tt y}(l_i) & = ({\tt y}(N(x_l)) - {\tt y}(l_i)) + ({\tt y}(N(x_r)) - {\tt y}(N(x_l))) %\notag \\
%                    %& \quad 
% - ({\tt y}(N(x_r) - {\tt y}(u_{i+s}))  \notag \\
%                    & \leq \sqrt{3} {\tt r}(x_l) + 
%                           \left(\frac{1}{\sqrt{3}}z_{\ast n_w} - \frac{1}{\sqrt{3}}z_{\ast n_e} -
%                         \frac{3}{\sqrt{3}}z_{s_ee} - \frac{5}{\sqrt{3}}z_{s_we}\right) \notag \\ 
%                    & \quad     - \left(\frac{1}{\sqrt{3}}({\tt x}(u_{i+s})-x_r) + \max\{0,{\tt y}(v) - {\tt y}(u_{i+s})\}\right)
% \end{align}
%%%%%%
Note that ${\tt y}(N(x_l)) - {\tt y}(l_i) + p_S(l_i,x_l) = \frac{5}{\sqrt{3}} {\tt r}(x_l)$ and so, by
Lemma~\ref{lem:growthrates},
\begin{align}
\label{eq:crossvertical}
{\tt y}(u_{i+s})-{\tt y}(l_i) & = ({\tt y}(N(x_l)) - {\tt y}(l_i)) + ({\tt y}(N(x_r)) - {\tt y}(N(x_l))) %\notag \\
                   %& \quad 
- ({\tt y}(N(x_r) - {\tt y}(u_{i+s}))  \notag \\
		   & \leq \frac{5}{\sqrt{3}} {\tt r}(x_l) - p_S(l_i,x_l) + 
                          \left(\frac{1}{\sqrt{3}}z_{\ast n_w} - \frac{1}{\sqrt{3}}z_{\ast n_e} -
                        \frac{3}{\sqrt{3}}z_{s_ee} - \frac{5}{\sqrt{3}}z_{s_we}\right) \notag \\ 
                   & \quad     - \left(\frac{1}{\sqrt{3}}({\tt x}(u_{i+s})-x_r) + \max\{0,{\tt y}(v) - {\tt y}(u_{i+s})\}\right)
\end{align}
%%%%%%
where $v$ is the NE vertex of $H(x_r)$. The $\max$ term in (\ref{eq:crossvertical}) is 0 or
positive depending on whether $u_{i+s}$ is on
side $n_e$ or $e$, respectively, of $H(x_r)$. 
Each edge $(u_t,u_{t+1})$ on the path $u_i, u_{i+1}, \dots, u_{i+s}$
can be bounded by $p_N(u_t, {\tt x}(c_{t+1})) - p_N(u_{t+1},{\tt x}(c_{t+1}))$ and so:
\begin{align}
\label{eq:upperpath}
\sum_{t=i}^{i+s-1} d_2(u_t,u_{t+1}) & \leq \sum_{t=i}^{i+s-1} (p_N(u_t, {\tt x}(c_{t+1})) - p_N(u_{t+1},{\tt x}(c_{t+1}))) \notag \\
	& = (\bar{U}(x_r) - p_N(u_{i+s},x_r)) - (\bar{U}(x_l) - p_N(u_i,x_l)) \notag \\
	& = p_N(u_i,x_l) + (\bar{U}(x_r)- \bar{U}(x_l)) - p_N(u_{i+s},x_r)  \notag \\
	& = p_N(u_i,x_l) + \left(\frac{2}{\sqrt{3}}(z_{\ast n_w} + z_{\ast n_e}) + \frac{4}{\sqrt{3}}z_{s_ee} + \frac{6}{\sqrt{3}}z_{s_we}\right) \notag \\
   & \quad  + \left(\frac{2}{\sqrt{3}}({\tt x}(u_{i+s})-x_r) + \max\{0,{\tt y}(v) - {\tt y}(u_{i+s})\}\right)
\end{align}
Substituting the left-hand side of
(\ref{eq:nogentle}) with  (\ref{eq:crossvertical}) and 
(\ref{eq:upperpath}) gives us 
\begin{multline*}
d_2(l_i,u_i) + \frac{5}{\sqrt{3}} {\tt r}(x_l) - p_S(l_i,x_l) + p_N(u_i,x_l) +
\frac{3}{\sqrt{3}}z_{\ast n_w}+\frac{1}{\sqrt{3}}(z_{\ast n_e}+z_{\ast e}+{\tt x}(u_{i+s})-x_r)\\
\geq \frac{3}{\sqrt{3}}({\tt x}(u_{i+s})-{\tt x}(l_i)) 
= \frac{3}{\sqrt{3}}({\tt x}(u_{i+s})- x_l) + \frac{3}{\sqrt{3}}{\tt r}(x_l) 
\end{multline*}
where we use ${\tt x}(l_i) = x_l - {\tt r}(x_l)$. Using
$z_{\ast n_w} + z_{\ast e} + z_{\ast n_e} = x_r - x_l$ and ${\tt x}(u_{i+s}) \geq x_r$, and
applying Lemma~\ref{lem:switchpath},
we get 
\[
\frac{2}{\sqrt{3}} {\tt r}(x_l) + \left(2 - \frac{2}{\sqrt{3}}\right)\delta_f(x_l) \geq
\frac{2}{\sqrt{3}}(z_{\ast n_e}+z_{\ast e})
\]
which is equivalent to the lemma statement.
%We complete our proof by multiplying both sides by $\frac{\sqrt{3}}{2}$.
%
%To prove our
%claim, it therefore suffices to show the following inequality:
%\begin{align*}
%	0 & \leq \left(2 - \frac{\sqrt{3}}{\cos(\Theta)}\right){\tt r}(x_l) - (2 - \sqrt{3}) \delta_f(x_l) \\
%	& = \frac{\cos(\Theta) - 1}{\cos(\Theta)}\sqrt{3}{\tt r}(x_l) + (2 - \sqrt{3}) ({\tt r}(x_l) - \delta_f(x_l) ) \\
%	& = \frac{\cos(\Theta) - 1}{\cos(\Theta)}\frac{\sqrt{3}}{2}\cos(\Theta)|wu_i| + (2 - \sqrt{3}) \frac{\sqrt{3}}{2}\sin(\Theta)|wu_i|
%\end{align*}
%where the last equality follows from the right triangle $w N(x_l) u_i$ from
%Figure~\ref{fig:gentlepath}-(b).
%%
%Thus, it suffices to show $1 - \cos(\Theta) \leq (2 - \sqrt{3})\sin(\Theta)$
%for all $\Theta$ for $0 \leq \Theta \leq \frac{\pi}{6}$.  Since both sides are
%nonnegative, squaring both sides and using $\sin^2(\Theta) = 1 -\cos^2(\Theta) = (1 - \cos(\Theta)) (1 + \cos(\Theta))$
%this inequality boils down to $1 - \cos(\Theta) \leq (2 - \sqrt{3})^2(1 + \cos(\Theta))$.
%We know that the last inequality and thus our claim holds since $\cos(\Theta) \geq \frac{\sqrt{3}}{2}$
%for all $\Theta$ for $0 \leq \Theta \leq \frac{\pi}{6}$.
\end{proof}

The following two lemmas build on Lemmma~\ref{lem:gentlepath}. They show
different ways that the ``extra'' $\frac{2}{\sqrt{3}}$ growth (i.e., the
growth above $\frac{6}{\sqrt{3}}$) of a bad transition $t_{ij}$ within a
$t_{ij}$-interval can be spread (amortized) over a wider time interval (refer
also to Fig.~\ref{fig:linearworstcase}).

\begin{lemma}  
\label{lem:linearworstcase}
Given the assumptions of Lemma~\ref{lem:gentlepath}, if $[x_l,x_r]$ is a 
$t_{wn_e}$-, $t_{s_ew}$-, $t_{en_w}$-, or $t_{s_we}$-interval, if
$t(x) \not= t_{en_w}$, $t_{s_we}$, $t_{wn_e}$, or $t_{s_ew}$, respectively, when
$x_l \leq x \leq x_r$, and if 
$t(x_r)=t_{\ast e}$, $t_{e \ast}$, $t_{\ast w}$, or $t_{w \ast}$, respectively, then
%
% \begin{align}
% \label{eq:LWC}
% \frac{2}{\sqrt{3}}z_{wn_e}, \frac{2}{\sqrt{3}}z_{s_ew} & \leq (\frac{2}{\sqrt{3}}-1)(e(x_r)-w(x_l)-2\theta_f(x_l)) \\ \label{eq:LWC2}
% & \leq (\frac{4}{\sqrt{3}}-2)(x_r-w(x_l)-\theta_f(x_l))                      
% \end{align}
% %\begin{align*}
% \[\frac{2}{\sqrt{3}}z_{s_we}, \frac{2}{\sqrt{3}}z_{en_w} %& 
% \leq (\frac{2}{\sqrt{3}}-1)(e(x_r)-w(x_l)-2\theta_b(x_r)) 
% %\\ & 
% \leq (\frac{4}{\sqrt{3}}-2)(e(x_r)-x_l-\theta_b(x_r))                      
% %\end{align*}
% \]
%
\begin{align}
\label{eq:LWC}
\frac{2}{\sqrt{3}}z_{wn_e}, \frac{2}{\sqrt{3}}z_{s_ew} & \leq \left(\frac{2}{\sqrt{3}}-1\right)(e(x_r)-e(x_l)+2\delta_f(x_l)) \\
\label{eq:LWC2}
& \leq \left(\frac{4}{\sqrt{3}}-2\right)(x_r-x_l+\delta_f(x_l)) \\
%\end{align}
%\begin{align*}
%\[
\frac{2}{\sqrt{3}}z_{s_we}, \frac{2}{\sqrt{3}}z_{en_w} & 
\leq \left(\frac{2}{\sqrt{3}}-1\right)(w(x_r)-w(x_l)+2\delta_b(x_r)) \nonumber \\ 
&  \leq \left(\frac{4}{\sqrt{3}}-2\right)(x_r-x_l+\delta_b(x_r)), \nonumber 
\end{align}
respectively.
\end{lemma}

\begin{figure}[!b]
\linearworstcase

\caption{Illustration of Lemmas~\ref{lem:linearworstcase}
and~\ref{lem:softrecovery} for the case of a
$t_{wn_e}$-interval $[x_l,x_r]$. (a) Lemma~\ref{lem:linearworstcase}: the
$\frac{2}{\sqrt{3}}$ extra cost of the bad transition $t_{wn_e}$ taking place
within the interval $[x_l,x_r]$ can be amortized in one of two ways:
either as cost $\frac{1}{2}(\frac{4}{\sqrt{3}}-2)$ spread over the interval
$[e(x_l)-2\delta_f(x_l),e(x_r)]$ (shown in blue) or as cost
$\frac{4}{\sqrt{3}}-2$ spread over the interval $[x_l-\delta_f(x_l),x_r]$
(shown in red).
(b) 
Lemma~\ref{lem:softrecovery}: the $\frac{2}{\sqrt{3}}$ extra cost of the
bad transition $t_{wn_e}$ within the interval $[x_l,x_r]$ can be amortized as
cost $\frac{1}{2}(\frac{4}{\sqrt{3}}-2)$ spread over the interval
$[e(x_l)-2\delta_f(x_l),w(x_r)]$ (shown in blue) {\em plus} cost
$\frac{1}{\sqrt{3}}$ spread over intervals of time $z_{s_wn_w}$ and $z_{s_en_w}$
(contained within the dashed red interval).}
\label{fig:linearworstcase}
\end{figure}



\begin{proof}
We prove the lemma for the case of a $t_{wn_e}$-interval; the other 3 cases
follow by symmetry. By Lemma~\ref{lem:growthrates}, if
$[x_l,x_r]$ is a $t_{wn_e}$-interval satisfying the conditions of the lemma then
${\tt r}(x_l)+z_{wn_e}+z_{wn_w}+\frac{1}{2}z_{s_wn_w} = {\tt r}(x_r) + \frac{1}{2}z_{s_en_e}+z_{en_e}+z_{s_we} + z_{s_ee}$
which implies
\begin{equation}
\label{eq:radii}
\sqrt{3} z_{wn_e} - \frac{\sqrt{3}}{2}z_{s_en_e} \leq \sqrt{3}({\tt r}(x_r) - {\tt r}(x_l) + z_{en_e}+z_{s_we} + z_{s_ee})
\end{equation}
Since $t(x_r) = t_{\ast e}$, we have 
$t(x) \not= t_{\ast w}, t_{\ast n_w}$ for $x_r \leq x \leq e(x_r)$, and thus, the interval
$[x_l,e(x_r)]$ is a $t_{wn_e}$-interval such that $t(x) = t_{\ast n_e}$ or
$t(x) = t_{\ast e}$ when $x_r \leq x \leq e(x_r)$.  Therefore, 
by Lemma~\ref{lem:gentlepath} we have 
%
% ${\tt r}(x_r) + z_{\ast n_e} + z_{\ast e}+(\sqrt{3}-1)\theta_f(x_l) \leq \sqrt{3}{\tt r}(x_l)$
%
${\tt r}(x_r) + z_{\ast n_e} + z_{\ast e} \leq {\tt r}(x_l) + (\sqrt{3}-1)\delta_f(x_l)$
%
which implies
%
\[
z_{wn_e} + z_{s_en_e} \leq (\sqrt{3}-1)\delta_f(x_l) - ({\tt r}(x_r) - {\tt r}(x_l) + z_{en_e}+z_{s_we} + z_{s_ee})
	\]
%
together with inequality~(\ref{eq:radii}), we get
\[(\sqrt{3}+1)z_{wn_e} \leq (\sqrt{3}-1)({\tt r}(x_r) - {\tt r}(x_l) +z_{en_e}+z_{s_we}+z_{s_ee}+\delta_f(x_l))\]
and multiplying both sides by $(\sqrt{3} - 1)/\sqrt{3}$
\begin{equation}
\label{eq:radii2}
\frac{2}{\sqrt{3}}z_{wn_e} \leq \left(\frac{4}{\sqrt{3}}-2\right)({\tt r}(x_r) - {\tt r}(x_l) +z_{en_e}+z_{s_we}+z_{s_ee}+\delta_f(x_l))
\end{equation}
%
% The fact that $e(x_r) - w(x_l) = {\tt r}(x_l) + z_{\ast n_w}+z_{\ast e}+z_{\ast n_e} + {\tt r}(x_r)$ (refer to Fig.~\ref{fig:linearworstcase}-(a))
% together with~(\ref{eq:radii}) gives us
% ${\tt r}(x_r)+z_{en_e}+z_{s_we}+z_{s_ee} \leq \frac{1}{2}(e(x_r) - w(x_l))$ and
% so~(\ref{eq:radii2}) implies inequality~(\ref{eq:LWC}). Because
% $e(x_r)-w(x_l) \leq 2(x_r-w(x_l))$, inequality~(\ref{eq:LWC2}) follows. 
%
By Lemma~\ref{lem:growthrates}, we have
%
$z_{en_e}+z_{s_we}+z_{s_ee} \leq \frac{1}{2}(w(x_r) - w(x_l))$ and 
so~(\ref{eq:radii2}) implies inequality~(\ref{eq:LWC}).
%
Because $0 \leq w(x_r) - w(x_l)$ inequality~(\ref{eq:LWC}) implies
inequality~(\ref{eq:LWC2}). 
%
\end{proof}

\begin{lemma}
\label{lem:softrecovery}
(Refer to Fig.~\ref{fig:linearworstcase}-(b)) 
Given the assumptions of Lemma~\ref{lem:gentlepath}, 
if $[x_l,x_r]$ is a $t_{wn_e}$-, $t_{s_ew}$-, $t_{en_w}$-, or $t_{s_we}$-interval,
if $t(x) \not= t_{en_w}$, $t_{s_we}$, $t_{wn_e}$, or $t_{s_ew}$, respectively, when
$x_l \leq x \leq x_r$, and if $t(x_r) = t_{\ast w}$, $t_{w \ast}$, $t_{\ast e}$,
or $t_{e \ast}$, respectively, then
\begin{align}
& \quad \frac{2}{\sqrt{3}}z_{wn_e} - \frac{1}{\sqrt{3}}(z_{s_wn_w}+z_{s_en_w}), 
\frac{2}{\sqrt{3}}z_{s_ew} - \frac{1}{\sqrt{3}}(z_{s_wn_w}+z_{s_wn_e})  \nonumber \\
\leq & \quad \left(\frac{2}{\sqrt{3}}-1\right)(w(x_r) - e(x_l) + 2\delta_f(x_l)) \label{eq:softrecovery} \\
& \quad \frac{2}{\sqrt{3}}z_{en_w} - \frac{1}{\sqrt{3}}(z_{s_wn_e}+z_{s_en_e}),
\frac{2}{\sqrt{3}}z_{s_we} - \frac{1}{\sqrt{3}}(z_{s_en_w}+z_{s_en_e}) \nonumber \\
\leq & \quad \left(\frac{2}{\sqrt{3}}-1\right)(w(x_r) - e(x_l) + 2\delta_b(x_r)), \nonumber
\end{align}
respectively.
\end{lemma}

\begin{proof}
We prove the lemma for the case of a $t_{wn_e}$-interval only; the other 3 cases
can be seen to follow by symmetry. Then, 
since $[x_l,x_r]$ is a $t_{wn_e}$-interval and $t(x_r)=t_{\ast w}$, point $u(x_r)$
must be the NW point of $H(x_r)$ as shown in
Fig.~\ref{fig:linearworstcase}-(b). The point $u(x_r)$ must lie on side $n_w$ of 
$H(x)$ for $w(x_r) \leq x \leq x_r$ and so  $t(x) = t_{\ast n_w}$ when
$w(x_r) \leq x \leq x_r$. This, together with Lemma~\ref{lem:growthrates} on the growth of $w(\cdot)$
and the fact that $t(x) \not= t_{en_w}$ when $w(x_r) \leq x \leq x_r$,
implies that $w(x_r) - w(w(x_r)) = {\tt r}(w(x_r)) \leq \frac{1}{2}z_{s_wn_w}+z_{s_en_w} \leq z_{s_wn_w}+z_{s_en_w}$.
It also implies that all $t_{\ast n_e}$ and $t_{\ast e}$ transitions
in interval $[x_l,x_r]$ take place within interval $[x_l,w(x_r)]$ and so by
Lemma~\ref{lem:growthrates} on the growth of $e(\cdot)$,  $2z_{wn_e} \leq e(w(x_r)) - e(x_l)$.
Therefore, $2z_{wn_e}-(z_{s_wn_w}+z_{s_en_w}) \leq e(w(x_r))-{\tt r}(w(x_r))-e(x_l) = w(x_r)-e(x_l)$ and:
\begin{equation}
\label{eq:softrecovery2}
\left(\frac{4}{\sqrt{3}} - 2\right)z_{wn_e}-\left(\frac{2}{\sqrt{3}} - 1\right)(z_{s_wn_w}+z_{s_en_w}) \leq\
\left(\frac{2}{\sqrt{3}}-1\right)(w(x_r)-e(x_l))
\end{equation}


By Lemma~\ref{lem:gentlepath} we have
$z_{\ast e}+z_{\ast n_e} \leq {\tt r}(x_l)+(\sqrt{3}-1)\delta_f(x_l)$ and by
Lemma~\ref{lem:growthrates} on the growth rate of ${\tt r}(\cdot)$, we have 
${\tt r}(x_l)+z_{wn_e}-(z_{s_wn_e}+z_{s_en_e}+z_{en_e}+z_{s_we}+z_{s_ee}) \leq {\tt r}(w(x_r))$.
Using ${\tt r}(w(x_r)) \leq z_{s_wn_w}+z_{s_en_w}$ from above, we get
$2z_{wn_e} - (z_{s_wn_w}+z_{s_en_w}) \leq (\sqrt{3}-1)\delta_f(x_l)$
and multiplying both sides by $1 - \frac{1}{\sqrt{3}}$
\[\left(2-\frac{2}{\sqrt{3}}\right)z_{wn_e} - \left(1-\frac{1}{\sqrt{3}}\right)(z_{s_wn_w}+z_{s_en_w}) \leq
	\left(\frac{4}{\sqrt{3}}-2\right)\delta_f(x_l)\]
Summing~(\ref{eq:softrecovery2}) with this inequality,
we obtain~(\ref{eq:softrecovery}).
\end{proof}

We define a time $x$ in the interval $[{\tt x}(p), {\tt x}(q)]$ to be
{\em left critical} 
or {\em right critical} if $x$ is the left boundary of a maximal strict 
left interval or the right boundary of a maximal strict 
right interval, respectively. We also define the start and end
coordinates ${\tt x}(p)$ and ${\tt x}(q)$ to be both left and right critical. A
time $x$ is said  to be critical if it is left or right critical.
This lemma, illustrated in Fig.~\ref{fig:induction}, 
 bounds $P(x)$ by making use of amortization
Lemmas~\ref{lem:linearworstcase} and~\ref{lem:softrecovery}:
%inductively
%prove that the growth rate of $P(x)$ is at most $\frac{1n_w}{\sqrt{3}}-2$ on
%average:
%Let $P'(x)=P(x)-\frac{6}{\sqrt{3}}x$ and $\Delta P'(x_i,x_j)= P(x_i,x_j)-\frac{6}{\sqrt{3}}(x_j-x_i)$.  Then:

\begin{figure}
\induction

\caption{Illustration of Lemma~\ref{lem:criticalpoints}. (a) If $x$ is a left
critical point then $P(x)$ is no greater than
$(\frac{10}{\sqrt{3}}-2)x - (\frac{4}{\sqrt{3}} - 2)\delta_f(x)$.
In the case of a $t_{wn_e}$-interval starting at $x$, this allows some of the
extra $\frac{2}{\sqrt{3}}$ growth of bad transition $t_{wn_e}$ within the interval
to be charged to the ``past'' as cost $\frac{4}{\sqrt{3}}-2$ over the interval
$[x-\delta_f(x),x]$. 
(b) If $x$ is a right critical point then $P(x)$ is no greater than
$(\frac{10}{\sqrt{3}}-2)x + (\frac{4}{\sqrt{3}} - 2)\delta_b(x)$.
In the case of a $t_{en_w}$-interval ending at $x$, this means that some
of the extra $\frac{2}{\sqrt{3}}$ growth of bad transition $t_{en_w}$ within the
interval has been charged to the ``future'' as cost $\frac{4}{\sqrt{3}}-2$
over the interval $[x, x+\delta_b(x)]$.}
\label{fig:induction}
\end{figure}


\begin{lemma}
\label{lem:criticalpoints}
Given the assumptions of Lemma~\ref{le:mainlemmaB}:
\begin{itemize}
\item If $x$ is a left critical point then $P(x) \leq \left(\frac{10}{\sqrt{3}}-2\right)x - \left(\frac{4}{\sqrt{3}}-2\right)\delta_f(x)$
\item If $x$ is a right critical point then $P(x) \leq \left(\frac{10}{\sqrt{3}}-2\right)x + \left(\frac{4}{\sqrt{3}}-2\right)\delta_b(x)$
\end{itemize}
\end{lemma}


\begin{proof}
We proceed by induction, using  the left to right ordering of critical points.
The first critical point (i.e., the base case) is ${\tt x}(p)$. Since
$P(x) = 0 = {\tt x}(p) = \delta_f({\tt x}(p)) = \delta_b({\tt x}(p))$,
the claim holds. We assume now that the claim holds
for critical point $x_l$ and prove that it holds for the next critical point
$x_r$ (and so $x_l < x_r$ and no critical point exists between $x_l$ and $x_r$).

\noindent{\bf Case LL:} We first consider the case when $x_l$ and $x_r$ are both left critical points. We will show that in that case:
%	
\begin{align}
\label{eq:leftleft}
\Delta P(x_l,x_r) \leq \left(\frac{10}{\sqrt{3}}-2\right)(x_r - x_l) - \left(\frac{4}{\sqrt{3}}-2\right)
({\tt r}(x_r) - \delta_f(x_l))
\end{align}
Combining the above inequality with the inductive hypothesis for $P(x_l)$,
we obtain $P(x_r) = P(x_l) + \Delta P(x_l,x_r) \leq (\frac{10}{\sqrt{3}}-2)x_r
- (\frac{4}{\sqrt{3}}-2){\tt r}(x_r) \leq (\frac{10}{\sqrt{3}}-2)x_r
- (\frac{4}{\sqrt{3}}-2)\delta_f(x_r)$, as illustrated in Fig.~\ref{fig:proofA},
and complete the proof for this case.

We now show that~(\ref{eq:leftleft}) holds. 	
Both $x_l$ and $x_r$ being left critical implies that transitions $t_{s_we}$ and $t_{en_w}$ do not take place
within interval $[x_l, x_r]$. W.l.o.g. we assume that $x_l$ is the left
boundary of a maximal strict $t_{wn_e}$-interval. If $x_q$ is the right boundary
of this interval then $x_q \leq x_r$, $t(x_q) = t_{\ast w}$ or $t(x_q) = t_{s_ee}$,
and $t(x)$ is not bad when $x_q < x < x_r$. Let $z_{ij}$ be the time within
interval $[x_l,x_q]$ spent in transition $t_{ij}$, for every transition $t_{ij}$,
and let $z = \sum z_{ij} = x_q-x_l$.

If $t(x_q) = t_{\ast w}$ then by Lemma~\ref{lem:growthrates} we have:
\begin{align*}
\Delta P(x_l,x_r) & \leq \frac{8}{\sqrt{3}}z_{wn_e} + \frac{4}{\sqrt{3}}(z_{s_wn_w}+z_{s_en_w}) + \frac{6}{\sqrt{3}}(z - z_{wn_e} - z_{s_wn_w} - z_{s_en_w} + x_r - x_q) \\
                  & \leq \frac{6}{\sqrt{3}}(x_r - x_l) + \frac{2}{\sqrt{3}}z_{wn_e} - \frac{2}{\sqrt{3}}(z_{s_wn_w}+z_{s_en_w}) \\
                  & \leq \frac{6}{\sqrt{3}}(x_r - x_l) + \left(\frac{4}{\sqrt{3}}-2\right)(w(x_q)-x_l+\delta_f(x_l))
\end{align*}
The last inequality follows from Lemma~\ref{lem:softrecovery} and from 
$x_l < e(x_l)$ and $w(x_q)-x_l > {\tt x}(u(x_q)) - {\tt x}(u(x_l)) \geq 0$. 
Using $w(x_q) \leq w(x_r) = x_r - {\tt r}(x_r)$, we prove (\ref{eq:leftleft}).
\begin{figure}
\proofA
\caption{The proof of Lemma~\ref{lem:criticalpoints} in the case when $x_l$ and $x_r$ are both left critical and
$x_l$ is the left boundary of a $t_{wn_e}$-interval. By induction,
$P(x_l)$ is amortized as illustrated in Fig.~\ref{fig:induction}-(a) and shown
above in blue. We show that
$\Delta P(x_l,x_r)$ can be amortized as cost $\frac{10}{\sqrt{3}}-2$ over
interval $[x_l,w(x_r)]$ {\em plus} cost
$\frac{6}{\sqrt{3}}$ over interval $[w(x_r), x_r]$, shown in red.
This implies that $P(x_r)$ can be amortized as shown in Fig.~\ref{fig:induction}-(a).}
\label{fig:proofA}
\end{figure}

If $t(x_q) = t_{s_ee}$ then by Lemma~\ref{lem:growthrates} we have:

\begin{align*}
\Delta P(x_l,x_r) & \leq \frac{8}{\sqrt{3}}z_{wn_e} + \frac{6}{\sqrt{3}}(z - z_{wn_e} + x_r - x_q) \leq \frac{6}{\sqrt{3}}(x_r - x_l) + \frac{2}{\sqrt{3}}z_{wn_e} \\
                  & \leq \frac{6}{\sqrt{3}}(x_r - x_l) + \left(\frac{4}{\sqrt{3}}-2\right)(x_q-x_l+\delta_f(x_l))
\end{align*}
The last inequality follows from Lemma~\ref{lem:linearworstcase}.
Now, since $t(x_q) = t_{s_ee}$, then $x_q \leq b(x_q) \leq b(x_r) = w(x_r) = x_r - {\tt r}(x_r)$, we prove (\ref{eq:leftleft}) and complete the proof in this case as well.

\noindent{\bf Case RR:} The case when $x_l$ and $x_r$ are both right critical
points can be handled using an argument that is symmetric to the one we used for
case LL.
 
\noindent{\bf Case RL:} If $x_l$ is right critical and $x_r$ is left critical then no bad transitions
take place within the interval $[x_l,x_r]$ and
$\Delta P(x_l,x_r) \leq \frac{6}{\sqrt{3}}(x_r-x_l)$. If $x_l+\delta_b(x_l) \leq x_r-\delta_f(x_r)$ then by induction:
\begin{align*}
	P(x_r) & = P(x_l) + \Delta P(x_l,x_r) \\
               & \leq \left(\frac{10}{\sqrt{3}}-2\right)x_l+\left(\frac{4}{\sqrt{3}}-2\right)\delta_b(x_l)
	       + \frac{6}{\sqrt{3}}(x_r-x_l) \\
               & \leq \left(\frac{10}{\sqrt{3}}-2\right)x_r-\left(\frac{4}{\sqrt{3}}-2\right)\delta_f(x_r)
\end{align*}

If $x_l+\delta_b(x_l) > x_r-\delta_f(x_r)$, consider the interval
$I = [x_r-\delta_f(x_r), x_l+\delta_b(x_l)]$.
We argue that this interval
is contained within interval $[x_l, x_r]$ and that $\Delta P$ has a growth
rate of just $\frac{4}{\sqrt{3}}$ in interval $I$. To do
this, we assume w.l.o.g. that $t(x_l) = t_{en_w}$ with $u(x_l) = u_s$ and
$\ell(x_l) = l_s$, as illustrated in Fig.~\ref{fig:proofB}-(a). %Suppose that
%$u(x_l)$ is not the point on the W side of $H(x_r)$ (i.e., 
%$b(x_l) \not= b(x_r) = w(x_r)$). 
%Then either the point on the W side of $H(x_r)$
Since $x_r$ is left critical, one of $u(x_r)$ or $\ell(x_r)$ must lie on the
$w$ side of $H(x_r)$. However, the point on the $w$ side of $H(x_r)$ and 
having abscissa $w(x_r)$ cannot be a point $\ell(x_r) = l_t$ in $L$ 
(with $t \geq s$) because %, using Lemma~\ref{lem:bandf}-(b), 
we would have
$w(x_r) = {\tt x}(l_t) \geq {\tt x}(l_s) = e(x_l)$ which contradicts 
$x_l+\delta_b(x_l) > x_r-\delta_f(x_r)$.
%
Therefore, the point on the $w$ side of $H(x_r)$ must be 
a point $u_t \in U$ for some $t \geq s$ and $t(x) = t_{\ast n_w}$ for $x \in
[{\tt x}(u_t),x_r]$.  Furthermore, letting $x' = \max\{x_l, {\tt x}(u_t)\}$,
$t(x)$ must be $t_{wn_w}$, $t_{s_wn_w}$, or $t_{s_en_w}$ for $x$ in interval
$[x',x_r]$, and by
Lemma~\ref{lem:growthrates}, $\frac{\Delta {\tt r}(x)}{\Delta x} \geq 0$ for
$x$ in that interval. Thus, $x_r = w(x_r) + {\tt r}(x_r) \geq {\tt x}(u_t) + {\tt r }(x')$.
%
If $x' = x_l$ then $u_t = u_s$ as illustrated in Fig.~\ref{fig:proofB}-(a), and
then $x_r \geq {\tt x}(u_s) + {\tt r}(x_l) = x_l + \delta_b(x_l)$. 
%
If, on the other hand, $x' = {\tt x}(u_t)$ then $x_r \geq {\tt x}(u_t) + {\tt r}({\tt x}(u_t)) = e({\tt x}(u_t)) \geq e(x_l)$ %by Lemma~\ref{lem:bandf}-(d), 
and we still get $x_r \geq e(x_l) = x_l + {\tt r}(x_l) \geq x_l + \delta_b(x_l)$. 
%
%If $b(x_l) = b(x_r)$ (see Fig.~\ref{fig:proofB}) then $t(x) = t_{*n_w}$ for all
%$x$ in $[x_l,x_r]$ which, using 
%Lemma~\ref{lem:growthrates}, implies that $x_r - x_l > {\tt r}(x_l) - \theta_b(x_l)$,
%or $x_r > e(x_l) - \theta_b(x_l)$. 
A symmetric argument can be used to show
that $x_l \leq x_r-\delta_f(x_r)$ and therefore interval I is contained within
$[x_l,x_r]$.

\begin{figure}[!b]
\proofB

\caption{The proof of Lemma~\ref{lem:criticalpoints}. (a) The case when
$x_l$ is right critical, $x_r$ is left critical,
$x_l+\delta_b(x_l) > x_r-\delta_f(x_r)$, $t(x_l) = t_{en_w}$, and
$x' = \max\{x_l, {\tt x}(u_t)\} = x_l$. By induction,
$P(x_l)$ is amortized as illustrated in Fig.~\ref{fig:induction}-(b) and shown
above in blue. $\Delta P(x_l,x_r)$ can be amortized as cost 
$\frac{4}{\sqrt{3}}$ over interval 
$I=[x_r-\delta_f(x_r),x_l+\delta_b(x_l)]$ (shown in red) {\em plus} cost 
$\frac{6}{\sqrt{3}}$
over the interval $[x_l,x_r]$ not including $I$ (empty in the example). 
Thus $P(x_r)$ can be amortized as shown in Fig.~\ref{fig:induction}-(a).
(b) The case when $x_l$ is left critical and $x_r$ is right critical. By
induction, $P(x_l)$ is amortized as illustrated in Fig.~\ref{fig:induction}-(a)
and shown above in blue. $\Delta P(x_l,x_r)$ can be amortized as cost 
$\frac{10}{\sqrt{3}}-2$ over interval $[x_l,x_r]$ {\em plus} cost 
$\frac{4}{\sqrt{3}}-2$ over intervals $[x_l-\delta_f(x_l),x_l]$ and 
$[x_r,x_r+\delta_b(x_r)]$ (shown in red). Thus $P(x_r)$ can be amortized
as shown in Fig.~\ref{fig:induction}-(b).}
\label{fig:proofB}
\end{figure}


For every $x$ in interval $I$, %using Lemma~\ref{lem:bandf}-(e) 
we have
$x \geq w(x_r) = b(x_r) \geq b(x)$ and $x \leq e(x_l) \leq f(x_l) \leq f(x)$.
Given that no bad transitions take place
within $[x_l,x_r]$, $t(x)$ must be $t_{s_en_w}$ for $x \in I$ and therefore
$\Delta P$ has a growth rate bounded by $\frac{4}{\sqrt{3}}$ in $I$. Then,
as illustrated in Fig.~\ref{fig:proofB}-(a),
\begin{align*}
	P(x_r) & = P(x_l) + \Delta P(x_l,x_r) \\
	       & \leq \left(\frac{10}{\sqrt{3}}-2\right)x_l + \left(\frac{4}{\sqrt{3}}-2\right)\delta_b(x_l)
	       + \frac{4}{\sqrt{3}}|I| + \frac{6}{\sqrt{3}}(x_r-x_l-|I|) \\
	       & = \frac{6}{\sqrt{3}}x_r + \left(\frac{4}{\sqrt{3}}-2\right)(x_l + \delta_b(x_l))
	       - \frac{2}{\sqrt{3}}(x_l + \delta_b(x_l) - (x_r -\delta_f(x_r))) \\
	       & \leq \frac{6}{\sqrt{3}}x_r + \left(\frac{4}{\sqrt{3}}-2\right)(x_l + \delta_b(x_l))
	       - \left(\frac{4}{\sqrt{3}}-2\right)(x_l + \delta_b(x_l) - (x_r -\delta_f(x_r))) \\
               & = \left(\frac{10}{\sqrt{3}}-2\right)x_r - \left(\frac{4}{\sqrt{3}}-2\right)\delta_f(x_r)
\end{align*}

%Otherwise $f(x_i)> b(x_j)$.  Here it is still the case that there are no bad transitions in $[x_i,x_j]$ but now there are also good transitions.  Let $m=\max\{b(x_j),x_i\}$ and $M=\min\{f(x_i),x_j\}$.  Then $[m,M]$ is the largest region contained in both $[x_i,x_j]$ and in $[b(x_j),f(x_i)]$. For $x \in [b(x_j),f(x_i)]$,  $b(x)\leq b(f_j)<x$ and, $f(x) \geq f(x_i)>x$.  Since $[m,M]$ is contained in $[b(x_j),f(x_i)]$ and $[x_i,x_j]$, and since $[x_i,x_j]$ contains no bad transitions, every transition in $[m,M]$ is either $t_{s_en_w}$ or $t_{s_wn_e}$.  Since both of these transitions change the perimeter at a rate of $\frac{4}{\sqrt{3}}$  we have $\Delta P'(x_i,x_j) \leq -\frac{2}{\sqrt{3}} (M-m)$.  Using this we have:
%\[\begin{split}
% P'(x_j) \leq P'(x_i)+\Delta P'(x_i,x_j) &\leq (\frac{4}{\sqrt{3}}-2)(x_i+({\tt r}(x_i)-\theta_b(x_i))) -\frac{2}{\sqrt{3}} (M-m) \\
% &\leq (\frac{4}{\sqrt{3}}-2)(x_i+({\tt r}(x_i)-\theta_b(x_i))-(M-m)) \\
%							&= (\frac{4}{\sqrt{3}}-2)((m-\theta_b(x_i))+(f(x_i)-M))
%\end{split}\]

%Note that $\max\{b(x_j),x_i\}-\theta_b(x_i) \leq b(x_j)$ and that $f(x_i)-\min\{f(x_i),x_j\} \leq \theta_f(x_j)$.  Thus we have:
%\[\begin{split}
% P'(x_i)+\Delta P'(x_i,x_j) &\leq (\frac{4}{\sqrt{3}}-2)(b(x_j)+\theta_f(x_j))\\
% 							&= (\frac{4}{\sqrt{3}}-2)(x_j-({\tt r}(x_j)-\theta_f(x_j)))
%\end{split}\]

% % % % % % % % % % % % % % % % % % % % %ABOVE LATER % % % % % % % % %

\noindent{\bf Case LR:} 
Finally, we consider the case when $x_l$ is left critical and $x_r$ is right
critical. We will show that in this case:
\begin{equation}
\label{eq:keyineq0}
\Delta P(x_l,x_r) \leq \left(\frac{10}{\sqrt{3}}-2\right)(x_r - x_l) + \left(\frac{4}{\sqrt{3}}-2\right)(\delta_f(x_l) + \delta_b(x_r))
\end{equation}
Note that if this inequality holds then (refer to Fig.~\ref{fig:proofB}-(b)):
\begin{align*}
P(x_r) & = P(x_l) + \Delta P(x_l,x_r) \leq \left(\frac{10}{\sqrt{3}}-2\right)x_r + \left(\frac{4}{\sqrt{3}}-2\right)\delta_b(x_r)
\end{align*}

% % % % % % % % % % % % % % % % Start of Full section Lemma % % % % % % % %
Let $z_{ij}$ be the time within
interval $[x_l,x_r]$ spent in transition $t_{ij}$, for every transition $t_{ij}$,
and let $z = \sum z_{ij} = x_r-x_l$.
We assume w.l.o.g. that $t(x_l) = t_{wn_e}$. Note that either
$t(x_r) = t_{s_we}$ or $t(x_r) = t_{en_w}$.

\noindent{\bf Subcase LR.1:} We consider the case when
$t(x_r) = t_{s_we}$ first; this means that $t(x) \not= t_{s_ew}, t_{en_w}$
when $x \in [x_l, x_r]$. 
Either $[x_l,x_r]$ is a $t_{wn_e}$-interval with $t(x_r) = t_{s_we}$, and so
Lemma~\ref{lem:linearworstcase} applies to interval $[x_l,x_r]$, or
$[x_l,x_q]$ is a $t_{wn_e}$-interval for some $x_q < x_r$ with 
$t(x_q) = t_{s_ww}$,
and thus Lemma~\ref{lem:softrecovery} applies to interval $[x_l,x_q]$. In the
second case note that $t(x) \not= t_{wn_e}$ when $x \in [x_q, x_r]$ because, 
otherwise, there would be a left critical point between $x_l$ and $x_r$, 
a contradiction. This, together with $w(x_q) \leq w(x_r) < e(x_r)$,
%(by Lemma~\ref{lem:bandf}-(d)), 
gives:
\begin{equation}
\label{eq:keyineq}
\frac{2}{\sqrt{3}}z_{wn_e}-\frac{1}{\sqrt{3}}(z_{s_wn_w}+z_{s_en_w}) \leq \left(\frac{2}{\sqrt{3}}-1\right)(e(x_r)-e(x_l)+2\delta_f(x_l))
\end{equation}
Note that this inequality holds for the first case as well
(using Lemma~\ref{lem:linearworstcase}). Via symmetric arguments either $[x_l,x_r]$ is a
$t_{s_we}$-interval with $t(x_l) = t_{wn_e}$, and so Lemma~\ref{lem:linearworstcase}
applies to interval $[x_l,x_r]$, 
or $[x_{q'},x_r]$ is a $t_{s_we}$-interval for some $x_{q'} > x_l$ with
$t(x_{q'}) = t_{en_e}$, and thus Lemma~\ref{lem:softrecovery} applies to interval
$[x_{q'},x_r]$. Either way, the inequality
\[\frac{2}{\sqrt{3}}z_{s_we}-\frac{1}{\sqrt{3}}(z_{s_en_w}+z_{s_en_e})\leq \left(\frac{2}{\sqrt{3}}-1\right)(w(x_r)-w(x_l)+2\delta_b(x_r))\]
holds and it, together with inequality~(\ref{eq:keyineq}), gives:
\[\frac{2}{\sqrt{3}}(z_{wn_e} +z_{s_we})-\frac{2}{\sqrt{3}}(z_{s_wn_w}+z_{s_en_w}+z_{s_wn_e}+z_{s_en_e})\leq \left(\frac{4}{\sqrt{3}}-2\right)(x_r-x_l+\delta_b(x_r)+\delta_f(x_l))\]
We can now show that (\ref{eq:keyineq0}) holds:
\begin{align*} \Delta P(x_l, x_r) & \leq \frac{6}{\sqrt{3}}z + \frac{2}{\sqrt{3}}(z_{wn_e} +z_{s_we}) - \frac{2}{\sqrt{3}}(z_{s_wn_w}+z_{s_en_w}+z_{s_wn_e}+z_{s_en_e}) \\ 
	& \leq \frac{6}{\sqrt{3}}(x_r - x_l) + \left(\frac{4}{\sqrt{3}}-2\right)(x_r-x_l+\delta_b(x_r)+\delta_f(x_l)) \\
        & = \left(\frac{10}{\sqrt{3}}-2\right)(x_r - x_l) + \left(\frac{4}{\sqrt{3}}-2\right)(\delta_b(x_r) + \delta_f(x_l))
\end{align*}

\noindent{\bf Subcase LR.2:} We now consider the case when $t(x_l) = t_{wn_e}$ and $t(x_r) = t_{en_w}$. 
Note that this means that $t(x) \not= t_{s_ew}, t_{s_we}$ when $x \in [x_l, x_r]$. 
If 
\begin{equation}
\label{eq:keyineq2}
\frac{2}{\sqrt{3}}(z_{wn_e} +z_{en_w})-\frac{2}{\sqrt{3}}(z_{s_wn_w}+z_{s_en_w}+z_{s_wn_e}+z_{s_en_e})\leq
	\left(\frac{4}{\sqrt{3}}-2\right)(x_r-x_l+\delta_b(x_r)+\delta_f(x_l))
\end{equation}
then the above argument can be applied to obtain~(\ref{eq:keyineq0}). We will show next that if $t(x) = t_{\ast w}$ or $t(x) = t_{\ast e}$ for 
some $x \in [x_l, x_r]$ then inequality~(\ref{eq:keyineq2}) holds.

Consider the maximal strict $t_{wn_e}$-interval $[x_l,x_q]$ and the maximal
strict $t_{en_w}$-interval $[x_{q'},x_r]$. Recall that $t(x) \not= t_{en_w},t_{\ast w},t_{\ast e}$ for $x \in [x_l,x_q]$ and that $t(x) \not= t_{wn_e},t_{\ast w},t_{\ast e}$ for $x \in [x_{q'},x_r]$ and thus $x_l < x_q,x_{q'} < x_r$. 
If $t(x) = t_{\ast w}$ or $t(x) = t_{\ast e}$ 
for  some $x$ such that $x_l < x < x_r$, and using the assumption that there
are no critical points between $x_l$ and $x_r$, we must have that 
$x_l < x_q \leq x_{q'} < x_r$ and that $t(x) \not= t_{wn_e},t_{en_w}$ when
$x \in [x_q,x_{q'}]$. This also means that $t(x_q)$ is either $t_{s_ww}$ or 
$t_{s_ee}$ and that $t(x_{q'})$ is either $t_{s_ww}$ or $t_{s_ee}$. 

If $t(x_q)=t_{s_ww}$ then Lemma~\ref{lem:softrecovery} applies to interval
$[x_l,x_q]$. If $t(x_q) = t_{s_ee}$ then Lemma~\ref{lem:linearworstcase} applies
to interval $[x_l,x_q]$. Since $t(x) \not= t_{wn_e}$ when $x \in [x_q, x_r]$ 
and $w(x_q) \leq w(x_r) \leq e(x_r)$ and just as in case LR.1, it follows in 
both cases that
\begin{equation}
\label{eq:16}
\frac{2}{\sqrt{3}} z_{wn_e} - \frac{1}{\sqrt{3}}(z_{s_wn_w} + z_{s_en_w}) \leq \left(\frac{2}{\sqrt{3}} - 1\right) (e(x_r) - e(x_l) + 2 \delta_f(x_l))
\end{equation}
If $t(x_{q'})=t_{s_ee}$ then Lemma~\ref{lem:softrecovery}
applies to interval $[x_{q'},x_r]$. If $t(x_{q'}) = t_{s_ww}$ then Lemma~\ref{lem:linearworstcase} applies to interval $[x_{q'},x_r]$. Since $t(x) \not= t_{en_w}$ when $x \in [x_l, x_{q'}]$ and $w(x_l) \leq w(x_{q'}) \leq e(x_{q'})$ it follows 
in both cases that
\begin{equation}
\label{eq:17} 
\frac{2}{\sqrt{3}} z_{en_w} - \frac{1}{\sqrt{3}}(z_{s_wn_e} + z_{s_en_e}) \leq \left(\frac{2}{\sqrt{3}} - 1\right) (w(x_r) - w(x_l) + 2 \delta_b(x_r))
\end{equation}
By combining inequalities (\ref{eq:16}) and (\ref{eq:17}) we obtain inequality 
(\ref{eq:keyineq2}).

%%%%%%%%%%%%%
%%%%%%%%%%%%%
%Consider the maximal $t_{wn_e}$-interval $[x_l,x_q]$ and note that
%$x_q \leq x_r$. If $t(x_q) = t_{s_ww}$ then Lemma~\ref{lem:softrecovery}
%applies to interval $[x_l,x_q]$. Since $t(x) \not=t_{wn_e}$ 
%when $x_q < x < x_r$ (by definition of critical points) 
%and since $w(x_q) \leq w(x_r) \leq e(x_r)$ it follows that:
%\begin{equation}
%\frac{2}{\sqrt{3}}z_{wn_e} - \frac{1}{\sqrt{3}}(z_{s_wn_w}+z_{s_en_w}) \leq (\frac{2}{\sqrt{3}}-1)(e(x_r) - w(x_l) - 2\theta_f(x_l)).
%\end{equation}
%
%
%
%If $t(x_q) = t_{en_w}$ then $[x_l,x_q]$ is a $t_{wn_e}$-interval that
%satisfies the conditions of Lemma~\ref{lem:linearworstcase} and 
%
%Consider now the maximal $t_{en_w}$-interval $[x_{q'},x_r]$ and note that
%$x_l \leq x_{q'}$. If $t(x_{q'}) \not= t_{wn_e}$ then $t(x_{q'}) = t_{s_ee}$ 
%and Lemma~\ref{lem:softrecovery} applies to interval $[x_{q'},x_r]$. 
%Since $t(x) \not=t_{en_w}$ when $x_l < x < x_{q'}$ and $w(x_l) \leq w(x_{q'})$ and
%$w(x_r) \leq e(x_r)$ it follows that:
%\begin{equation}
%\frac{2}{\sqrt{3}}z_{en_w} - \frac{1}{\sqrt{3}}(z_{s_wn_e}+z_{s_en_e}) \leq (\frac{2}{\sqrt{3}}-1)(e(x_r) - w(x_l) - 2\theta_f(x_l))
%\end{equation}
%If $t(x_{q'}) = t_{wn_e}$ then $[x_{q'},x_r]$ is a $t_{en_w}$-interval that
%satisfies the conditions of Lemma~\ref{lem:linearworstcase} and 
%
%
%\begin{equation}
%\frac{2}{\sqrt{3}}z_{en_w} - \frac{1}{\sqrt{3}}(z_{s_wn_e}+z_{s_en_e}) \leq (\frac{2}{\sqrt{3}}-1)(e(x_r) - w(x_l) - 2\theta_f(x_l)).
%\end{equation}
%
%
%
%In summary, if $t(x) = t_{\ast w}$ for some $x$ such that $x_l < x < x_r$ then
%inequality~(\ref{eq:keyineq2}) holds. A symmetric argument can be used
%to show that the inequality also holds if $t(x) = t_{\ast e}$ for some $x$ such
%that $x_l < x < x_r$.
%%%%%%%%%%%%%%%%%%
%%%%%%%%%%%%%%%%%%

We assume now that $t(x) \not= t_{*w}, t_{*e}$ for all $x \in [x_l,x_r]$.
We can also assume that
inequality~(\ref{eq:keyineq2}) does not hold which implies
\[\frac{2}{\sqrt{3}}(z_{wn_e} +z_{en_w}) > \left(\frac{4}{\sqrt{3}}-2\right)(x_r - x_l + \delta_b(x_r)+\delta_f(x_l))
\]
which, using $z_{wn_e} +z_{en_w} \leq x_r - x_l$, implies:
\begin{equation}
\label{eq:keyineq3}
\left(4 - \frac{6}{\sqrt{3}}\right)(\delta_b(x_r)+\delta_f(x_l)) < \left(\frac{6}{\sqrt{3}}-2\right)(x_r - x_l)
\end{equation}

Now, $\Delta P(x_l,x_r) = U(x_r) - U(x_l) + L(x_r) - L(x_l)$. For all
$x \in [x_l,x_r]$, $t(x) = t_{*n_w}, t_{*n_e}$  and so, by
Lemma~\ref{lem:growthrates},
$\frac{\Delta p_N(x)}{\Delta x} = \frac{2}{\sqrt{3}}$. It follows that
$U(x_r) - U(x_l) \leq \frac{2}{\sqrt{3}}(x_r-x_l)$. In order to bound
$L(x_r) - L(x_l)$, we consider the following path $\mathcal{P}$ from point $\ell(x_l)$,
say $l_i$, that lies on side $w$ of $H(x_l)$ to point $\ell(x_r)$, say $l_j$
(where $i < j$), that lies on side $e$ of $H(x_r)$: 
$\ell(x_l)=l_i, u_i,u_{i+1},\dots,u_j,l_j=\ell(x_r)$ (refer to Fig.~\ref{fig:pathcost}).
Then, if $|\mathcal{P}|$ is the length of $\mathcal{P}$:
\begin{figure}
\center{\pathcost}

\caption{
The proof of Lemma~\ref{lem:criticalpoints}. Shown is the case when $x_l$ is
left critical, $x_r$ is right critical, $t(x_l) = t_{wn_e}$, $t(x_r)=t_{en_w}$,
and $t(x) \not= t_{\ast w}, t_{\ast e}$ for $x \in [x_l,x_r]$. The path
$\mathcal{P}$, defined as $l_i,u_i,u_{i+1},\dots,u_j,l_j$, is effectively
a shortcut for $l_i, l_{i+1}, \dots,l_j$.
% (a) Illustration of path $P$: $l_i,u_i,u_{i+1},\dots,u_j,l_j$.
% (b) $-p_S(l_i,x_l) + |l_iu_i|$ is maximized when $l_i$ is the SW corner
% of $H(x_l)$.
}
\label{fig:pathcost}
\end{figure}
\begin{align*}
L(x_r) - L(x_l) & = d_{T_{1j}}(p,l_j) + p_S(l_j,x_r) - d_{T_{1i}}(p,l_i) - p_S(l_i,x_l) \\
& \leq d_{T_{1i}}(p,l_i) + |\mathcal{P}| + p_S(l_j,x_r) - d_{T_{1i}}(p,l_i) -p_S(l_i,x_l) \\
& = -p_S(l_i,x_l) + |\mathcal{P}| + p_S(l_j,x_r) \\
& \leq -p_S(l_i,x_l) + d_2(l_i,u_i) + p_N(u_i,x_l) + U(x_r) - U(x_l) \\
& \quad - p_N(u_j,x_r) + d_2(l_j,u_j) + p_S(l_j,x_r)
\end{align*}
using the fact that the length of the path $u_i,u_{i+1},\dots,u_j$ is at most
$p_N(u_i,x_l) + U(x_r) - U(x_l) - p_N(u_j,x_r)$. 
%
% We note that $-p_S(l_i,x_l) + |l_iu_i|$ is maximized when $l_i$ is the SW corner
% of $H(x_l)$, as illustrated in Fig.~\ref{fig:pathcost}-(b). Assuming that is
% the case, let $\theta = \angle u_i l_i N(x_l)$ and note that
% $0 \leq \theta \leq \frac{\pi}{6}$. Then
% $p_S(l_i,x_l) = \frac{2}{\sqrt{3}}r(x)$,
% $|l_iu_i| = \frac{2}{\cos(\theta)}r(x)$, and
% $p_N(u_i,x_l) = -2\tan(\theta)r(x)$. Using
% $\frac{1}{\cos(\theta)} \leq 1 + (2-\sqrt{3})\tan(\theta)$ when
% $\theta \in [0,\frac{\pi}{6}]$, 
% we have
% % we can bound $-p_S(l_i,x_l) + |l_iu_i| + p_N(u_i,x_l)$ with
% \begin{align*}
% -p_S(l_i,x_l) + |l_iu_i| + p_N(u_i,x_l) & = \left(-\frac{2}{\sqrt{3}} + \frac{2}{\cos(\theta)} -2\tan(\theta)\right)r(x)\\
% & \leq \left(-\frac{2}{\sqrt{3}} + 2 + 2(2-\sqrt{3})\tan(\theta) -2\tan(\theta)\right)r(x) \\
% & = \left(2-\frac{2}{\sqrt{3}}\right)\left({\tt r}(x_l)-\sqrt{3}\tan(\theta){\tt r}(x_l)\right) \\
% & = \left(2-\frac{2}{\sqrt{3}}\right)\delta_f(x_l)
% \end{align*}
% Similarly, $-p_N(u_j,x_r) + |u_jl_j| + p_S(l_j,x_r) \leq \left(2-\frac{2}{\sqrt{3}}\right)\delta_b(x_r)$. Combining the bounds on
%
By Lemma~\ref{lem:switchpath},
$-p_S(l_i,x_l) + d_2(l_i,u_i) + p_N(u_i,x_l) \leq
 \left(2-\frac{2}{\sqrt{3}}\right)\delta_f(x_l)$ and
$-p_N(u_j,x_r) + d_2(l_j,u_j) + p_S(l_j,x_r) \leq \left(2-\frac{2}{\sqrt{3}}\right)\delta_b(x_r)$. Combining the bounds on
$U(x_r) - U(x_l)$ and $L(x_r) - L(x_l)$, we get
\[
\Delta P(x_l,x_r) \leq \left(2-\frac{2}{\sqrt{3}}\right)(\delta_b(x_r)+\delta_f(x_l))+\frac{4}{\sqrt{3}}(x_r-x_l)
\]
Summing the above inequality with inequality~(\ref{eq:keyineq3}) yields 
(\ref{eq:keyineq0}) and completes the inductive proof in this case as well.
\end{proof}

We can now provide a proof of (Amortization) Lemma~\ref{le:mainlemmaB}.
\begin{proof}[Proof of Lemma~\ref{le:mainlemmaB}]
Note that ${\tt x}(q)$ is a left critical point and that
$\delta_f({\tt x}(q)) = 0$.
	Therefore, by Lemma \ref{lem:criticalpoints},
$P({\tt x}(q)) \leq (\frac{10}{\sqrt{3}}-2){\tt x}(q)$. 
Since $2d_{T_{1n}}(p, q) = P({\tt x}(q))$, the lemma follows.
\end{proof}




}

\section{Conclusion}
\label{sec:conclusion}
% \vspace{-0.5em}
\section{Conclusion}
% \vspace{-0.5em}
Recent advances in multimodal single-cell technology have enabled the simultaneous profiling of the transcriptome alongside other cellular modalities, leading to an increase in the availability of multimodal single-cell data. In this paper, we present \method{}, a multimodal transformer model for single-cell surface protein abundance from gene expression measurements. We combined the data with prior biological interaction knowledge from the STRING database into a richly connected heterogeneous graph and leveraged the transformer architectures to learn an accurate mapping between gene expression and surface protein abundance. Remarkably, \method{} achieves superior and more stable performance than other baselines on both 2021 and 2022 NeurIPS single-cell datasets.

\noindent\textbf{Future Work.}
% Our work is an extension of the model we implemented in the NeurIPS 2022 competition. 
Our framework of multimodal transformers with the cross-modality heterogeneous graph goes far beyond the specific downstream task of modality prediction, and there are lots of potentials to be further explored. Our graph contains three types of nodes. While the cell embeddings are used for predictions, the remaining protein embeddings and gene embeddings may be further interpreted for other tasks. The similarities between proteins may show data-specific protein-protein relationships, while the attention matrix of the gene transformer may help to identify marker genes of each cell type. Additionally, we may achieve gene interaction prediction using the attention mechanism.
% under adequate regulations. 
% We expect \method{} to be capable of much more than just modality prediction. Note that currently, we fuse information from different transformers with message-passing GNNs. 
To extend more on transformers, a potential next step is implementing cross-attention cross-modalities. Ideally, all three types of nodes, namely genes, proteins, and cells, would be jointly modeled using a large transformer that includes specific regulations for each modality. 

% insight of protein and gene embedding (diff task)

% all in one transformer

% \noindent\textbf{Limitations and future work}
% Despite the noticeable performance improvement by utilizing transformers with the cross-modality heterogeneous graph, there are still bottlenecks in the current settings. To begin with, we noticed that the performance variations of all methods are consistently higher in the ``CITE'' dataset compared to the ``GEX2ADT'' dataset. We hypothesized that the increased variability in ``CITE'' was due to both less number of training samples (43k vs. 66k cells) and a significantly more number of testing samples used (28k vs. 1k cells). One straightforward solution to alleviate the high variation issue is to include more training samples, which is not always possible given the training data availability. Nevertheless, publicly available single-cell datasets have been accumulated over the past decades and are still being collected on an ever-increasing scale. Taking advantage of these large-scale atlases is the key to a more stable and well-performing model, as some of the intra-cell variations could be common across different datasets. For example, reference-based methods are commonly used to identify the cell identity of a single cell, or cell-type compositions of a mixture of cells. (other examples for pretrained, e.g., scbert)


%\noindent\textbf{Future work.}
% Our work is an extension of the model we implemented in the NeurIPS 2022 competition. Now our framework of multimodal transformers with the cross-modality heterogeneous graph goes far beyond the specific downstream task of modality prediction, and there are lots of potentials to be further explored. Our graph contains three types of nodes. while the cell embeddings are used for predictions, the remaining protein embeddings and gene embeddings may be further interpreted for other tasks. The similarities between proteins may show data-specific protein-protein relationships, while the attention matrix of the gene transformer may help to identify marker genes of each cell type. Additionally, we may achieve gene interaction prediction using the attention mechanism under adequate regulations. We expect \method{} to be capable of much more than just modality prediction. Note that currently, we fuse information from different transformers with message-passing GNNs. To extend more on transformers, a potential next step is implementing cross-attention cross-modalities. Ideally, all three types of nodes, namely genes, proteins, and cells, would be jointly modeled using a large transformer that includes specific regulations for each modality. The self-attention within each modality would reconstruct the prior interaction network, while the cross-attention between modalities would be supervised by the data observations. Then, The attention matrix will provide insights into all the internal interactions and cross-relationships. With the linearized transformer, this idea would be both practical and versatile.

% \begin{acks}
% This research is supported by the National Science Foundation (NSF) and Johnson \& Johnson.
% \end{acks}

%%%%%%%%%%%%%%%%%%%%%%%%%%%%%%%%%%%%%%%%%%%%%%%%%%%%%%%%%%%%%%%%%%%%
%\section*{References}
%\bibliographystyle{alpha}
\bibliography{paper}

\end{document}
