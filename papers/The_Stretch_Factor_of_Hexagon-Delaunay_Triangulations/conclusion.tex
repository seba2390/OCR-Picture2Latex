


The approach we use to bound the length of the shortest path in a Delaunay
triangulation $T$ between points $s$ and $t$ is to consider the linear
sequence $T_{1n}$ of triangles of $T$ that segment $[st]$ intersects.
We show that, in general, 
$T_{1n}$ can be split into 1) disjoint linear sequences of triangles
$T_{i_1j_1}, T_{i_2j_2}, \dots, T_{i_k,j_k}$ that contain no gentle path
and 2) $k-1$ gentle paths with a gentle path connecting the right vertex of
$T_{j_l}$ with the left vertex of $T_{i_{l+1}}$ for $l = 1, \dots, k-1$. 
The worst case stretch factor for the Delaunay triangulation is then the
maximum between the worst case stretch factors for 1) a path connecting the
leftmost and rightmost points in a linear sequence $T_{ij}$ that contains no
gentle path and 2) a gentle path.

(Main) Lemma~\ref{le:divide} and Lemma~\ref{le:lower_bound} show
that the worst case stretch factor for {\Large\varhexagon}-Delaunay
triangulations comes from gentle path constructions. It turns out that
similar conclusions can also be made regarding $\triangle$- and
$\square$-Delaunay triangulations.

For $\ocircle$-Delaunay triangulations, the situation seems to be different.
The lower bound construction by Bose et al.~\cite{BDLSV11} corresponds to
a gentle path construction and has stretch factor $1.5846$. The lower
bound construction by Xia and Zhang~\cite{XZ11} corresponds to a linear
sequence that contains no gentle path and has stretch factor $1.5932$.
We think that the worst case stretch factor for $\ocircle$-Delaunay 
triangulations will come from a construction similar to the one by Xia and 
Zhang~\cite{XZ11}. Therefore, to get a tight bound on the stretch factor of a
$\ocircle$-Delaunay triangulation one needs to develop techniques that
give tight bounds on the stretch factor of a linear sequence that contains
no gentle path.

%\begin{wrapfigure}{r}{6cm}
%\begin{center}
\iftoggle{abstract}
{\begin{figure}[!b]}
{\begin{figure}}
\begin{center}
%\vspace{-0.545cm}
\mickey
\caption{The Mickey Mouse {\Large\varhexagon}-Delaunay triangulation. The
inradii of $H_1$ and $H_n$ are both set to $1$. Edges that belong to a
shortest path from $s$ to $t$ are in bold.}
\label{fig:mickey}
\end{center}\end{figure}
%\end{center}
%\end{wrapfigure}

We have done so for {\Large\varhexagon}-Delaunay triangulations. 
Our (Amortization) Lemma~\ref{le:mainlemmaB} implies that for 
{\Large\varhexagon}-Delaunay triangulations the worst case
stretch factor for a linear sequence $T_{ij}$ with no gentle paths is $(\cT)$. 
It turns out that our analysis is tight: Figure~\ref{fig:mickey}
shows a construction--which we name the Mickey Mouse 
{\Large\varhexagon}-Delaunay triangulation--that, for any $\epsilon > 0$, 
can be extended to a
{\Large\varhexagon}-Delaunay triangulation whose shortest path between $s$ and
$t$ is at least $(\cT) d_x(s,t)-\epsilon$. Unsurprisingly, the construction
corresponds to the lower bound construction by Xia and Xhang~\cite{XZ11} for 
$\ocircle$-Delaunay triangulations.

Based on this we think that the techniques we developed for obtaining the 
tight bound in Lemma~\ref{le:mainlemmaB} will be useful in obtaining better
upper bounds for the stretch factor of other kinds of Delaunay triangulations.


%Proving a tight bound on the stretchj factor for the case when $T_{1n}$ contains no 
%gentle path was not necessary to prove the main result of this paper: a bound
%of $2$ would have been sufficient as can be seen from the proof of
%Theorem~\ref{th:main}. The siginificance of a tight bound for the case
%when $T_{1n}$ contains no  gentle path transcends {\Large\varhexagon}-Delaunay
%triangulations. 


%is because of the significance
%decided to go beyond what was necesary
%and proved a tight bound is because of its significance for the computing 

%$\ocircle$-Delaunay triangulations.
%pushed is because of whatit means for the 



%While this construction does not
%yield a 



%We believe that the approach used by
%Xia and Zhang is the right approach for tighter lower bound constructions and,
%for this reason, that the techniques we developed for
%proving Lemma~\ref{le:mainlemmaB} will be useful in obtaining better upper bounds
%for the stretch factor of $\ocircle$-Delaunay triangulations.

%The approach used in this paper builds on the techniques developed
%by Bonichon et. al.~\cite{BGHP15}. In \cite{BGHP15}, gentle paths of
%average degree up to $\pi/4$ are used to eliminate gentle edges. In
%this paper, gentle paths of average degree up to $\pi/6$ are used instead.
%In both cases, gentle edges need to be removed because the growth rate of
%a potential function that bounds the shortest distance between two
%points ($P(x)$ in this paper) is too high when gentle edges are present.
%Thus the gentle path and the potential
%function techniques complement each other and the optimum gentle path
%average degree leads to a tight stretch factor bound.
%We believe that this overall approach generalizes to other regular polygons and,
%most importantly, the circle.






