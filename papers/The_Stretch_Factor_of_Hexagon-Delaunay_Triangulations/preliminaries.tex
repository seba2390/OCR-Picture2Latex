We consider a finite set $P$ of points in the two-dimensional plane with an
orthogonal coordinate system. The $x$- and $y$-coordinates of a point $p$ will
be denoted by ${\tt x}(p)$ and ${\tt y}(p)$, respectively. The
Euclidean graph $\cE^P$ of $P$ is the complete weighted graph embedded in the
plane whose nodes are identified with the points of $P$. For
every pair of nodes $p$ and $q$, the edge $(p,q)$ represents the
segment $[pq]$ and the weight of $(p,q)$ is the Euclidean
distance between $p$ and $q$ which is 
$d_2(p,q) = \sqrt{({\tt x}(p)-{\tt x}(q))^2+({\tt y}(p)-{\tt y}(q))^2}$.
Our arguments also use the $x$-coordinate distance between 
$p$ and $q$ which we denote as $d_x(p,q) = |{\tt x}(p)-{\tt x}(q)|$.

Let $T$ be a subgraph of $\cE^P$. The length of a path in $T$ is the sum of the
weights of the edges of the path and the distance $d_T(p,q)$ in $T$
between two points $p$ and $q$ is the length of the shortest
path in $T$ between them. $T$ is a $t$-spanner for some constant $t>0$ 
if for every pair of points $p,q$ of $P$, $d_T(p,q) \leq t \cdot d_2(p,q)$. The
constant $t$ is referred to as the {\em stretch factor} of $T$.

We define a {\em family of spanners} to be a set of graphs $T^P,$
one for every finite set $P$ of points in the plane, such that for some
constant $t>0$, every $T^P$ is a $t$-spanner of $\cE^P$. 
We say that the stretch factor $t$ is exact (tight) for the family
(or that the worst case stretch factor is $t$) if for every $\epsilon > 0$
there exists a set of points $P$ such that $T^P$ is {\em not} a
$(t-\epsilon)$-spanner of $\cE^P$. 

The families of spanners we consider are various types of Delaunay
triangulations on a set $P$ of points in the plane. Given a set $P$ of
points on the plane, we say that a convex, closed, simple curve in the plane is
empty if it contains no point of $P$ in its interior. The $\ocircle$-Delaunay
triangulation $T$ on $P$ is defined as follows: For every pair $u,v \in P$,
$(u,v)$ is an edge of $T$ if and only if there is an empty {\em circle} passing through
$u$ and $v$. (This definition assumes that
points are in general position which in the case of $\ocircle$-Delaunay
triangulations means that no four points of $P$ are co-circular.) If,
in the definition, 
{\em circle} is replaced by {\em fixed-orientation square} (e.g., a square
whose sides are axis-parallel) or by {\em fixed-orientation equilateral
triangle} then different triangulations are obtained: the $\square$-
and the $\triangle$-Delaunay triangulations.

%\begin{wrapfigure}{r}{3.8cm}
%\begin{center}
%\begin{tikzpicture}[scale=1]
%\draw [dashed] (0,0) rectangle (3, -3);
%\node (h) [draw, color=gray, shape border rotate=30, minimum size=1.5in, regular polygon, regular polygon sides=6] at (0,0) {};



%\foreach \anchor/\placement/\label in
%    {corner 1/above/N, corner 2/above left/NW, corner 3/below left/SW, corner 4/below/S, corner 5/below right/SE, corner 6/above right/NE}
%\draw[shift=(h.\anchor)] plot[mark=*,mark size=1pt] coordinates{(0,0)} 
%node [\placement] {{\small\label}};

%\foreach \anchor/\placement/\label in
%    {side 1/above/NW, side 2/left/W, side 3/below/SW, side 4/below/SE, side 5/right/E, side 6/above/NE}
%\draw[shift=(h.\anchor)] plot coordinates{(0,0)} 
%node [\placement] {{\small\label}};

%\end{tikzpicture}
%\end{center}
%\caption{The hexagon orientation and the side and vertex labels that we use}
%\label{fig:lower}
%\end{wrapfigure}


If, in the definition of the $\ocircle$-Delaunay triangulation, we change
{\em circle} to {\em fixed-orientation regular hexagon}, then a
{\Large\varhexagon}-Delaunay triangulation is obtained. In this paper
we focus on such triangulations. While any fixed orientation of the
hexagon is possible, we choose w.l.o.g. the orientation that has two sides of
the hexagon parallel to the $y$-axis as shown in Fig.~\ref{fig:lower}-(a).
In the remainder of the paper, {\em hexagon} will always refer to a regular
hexagon with such an orientation. We find it useful to label the vertices of
the hexagon $N$, $E_N$, $E_S$, $S$, $W_S$, and $W_N$, in clockwise order and
starting with the
top one. We also label the sides $n_e$, $e$, $s_e$, $s_w$, $w$, and $n_w$ as
shown in  Fig.~\ref{fig:lower}-(a); we will sometimes refer to the $s_e$
and $s_w$ sides as the $s$ sides and to the $n_e$ and $n_w$ sides as the $n$
sides.

\iftoggle{abstract}
{\begin{figure}}
{\begin{figure}[!b]}
\center{\lowerbound

\hspace{-0.4cm} (a) \hspace{4.6cm} (b) \hspace{4.9cm} (c)
}
\caption{(a) The hexagon orientation and the side and vertex labels that we
use (b) A {\Large\varhexagon}-Delaunay triangulation with points
$p$, $q$, $p_k$, and $q_0$ having coordinates $(0,0)$,
$(1, \frac{1}{\sqrt{3}})$, $(\delta, \frac{2}{\sqrt{3}}-\sqrt{3}\delta)$,
and $(1-\delta, -\frac{1}{\sqrt{3}}+\sqrt{3}\delta)$, respectively.
\iftoggle{abstract}{For $\delta$ small enough, $d_T(p,q) \geq (2-\epsilon)d_2(p,q)$.}{} (c) A closer look at the bottom faces of this triangulation.}

\label{fig:lower}
\end{figure}

The definition of the {\Large\varhexagon}-Delaunay triangulation 
assumes that no four points lie on the boundary of an empty
hexagon. Our arguments also assume that no two points lie on a line whose
slope matches the slope of a side of the hexagon (i.e. slopes 
$\infty, \frac{1}{\sqrt{3}}, -\frac{1}{\sqrt{3}}$). 
%; note that this implies that no two points
%lie on a side of an empty hexagon.
The {\em general position} assumption we therefore make in this paper consists
of the above two restrictions. This assumption is made solely for the purpose of
simplifying the presentation; the arguments in the paper could be extended so
the results apply to all {\Large\varhexagon}-Delaunay triangulations.
Finally, we need to be aware that unlike the $\ocircle$-Delaunay
triangulation on $P$, the {\Large\varhexagon}-Delaunay (and also the
$\square$- and  $\triangle$-Delaunay) triangulation on $P$ may not contain
all edges on the convex hull of $P$. To handle this and simplify our
arguments, we add to $P$ six additional points, very close to but not exactly 
(in order to satisfy the above assumptions) at coordinates $(0,\pm M)$ and 
$(\pm M \cos(\pi/6), \pm M \sin(\pi/6))$ where $M > 50\max_{s,t \in P}d_2(s,t)$.
The {\Large\varhexagon}-Delaunay triangulation on this modified set of points
$P$, consisting of the original triangulation plus additional
edges between the new points and original points and also between the new
points themselves,  
includes the edges on the convex hull of $P$. Also, any path in
this triangulation between two points $s$ and $t$ from the original set $P$
with length bounded by $2d_2(s,t)$ cannot possibly use the added points. Thus
a proof of our main result for the modified triangulation will also be a proof
for the original one and so we assume that the {\Large\varhexagon}-Delaunay
triangulation on $P$ includes the edges on the convex hull of $P$.

We end this section with a lower bound, by Bonichon~\cite{Bonichon}, on the
worst case stretch factor of {\Large\varhexagon}-Delaunay triangulations.
\iftoggle{abstract}
{The
lower bound construction is illustrated in Fig.~\ref{fig:lower}-(b) and
Fig.~\ref{fig:lower}-(c). The proof is ommited but appears in the Appendix.
}{}
\begin{lemma}
\label{le:lower_bound}
For every $\eps > 0 $, there exists a set $P$ of points in the
plane such that the {\Large\varhexagon}-Delaunay triangulation on $P$
has stretch factor at least  $2 - \eps$.
\end{lemma}


\iftoggle{abstract}
{}{\begin{proof}
Let $k$ be some positive integer and let points $p=p_0$, $q=q_k$, $p_k$, and
$q_0$ have coordinates $(0,0)$, $(1, \frac{1}{\sqrt{3}})$,
$(\delta, \frac{2}{\sqrt{3}}-\sqrt{3}\delta)$,
and $(1-\delta, -\frac{1}{\sqrt{3}}+\sqrt{3}\delta)$, respectively, where 
$\delta = \frac{1}{k+2}$ (see Fig.~\ref{fig:lower}-(b)). Additional $k-1$ 
points $p_1,\dots,p_{k-1}$ are placed on line segment $[p_0p_k]$ and another
$k-1$ points $q_1,\dots,q_{k-1}$ on line segment $[q_0q_k]$ so that all segments
$[p_{i-1}p_i]$ and $[q_{i-1}q_i]$, for $i=1,\dots,k$,  have equal length. 

For every $i = 1, 2, \dots, k$, if $H_i$ is the hexagon of minimum width
$1-\delta$ with $p_{i-1}$ and $p_{i}$ on its $w$ and $n_w$ sides,
then $q_{i-1}$ is exactly the $E_S$ vertex of $H_i$
(e.g., refer to $p_0$, $p_1$, $q_0$, and $H_1$ in Fig.~\ref{fig:lower}-(c)).
This means that all points $q_j$ with
$j \not= i-1$ as well as all points $p_j$ with $j \not= i-1, i$ lie outside
of $H_i$. Therefore, for every $i = 1, 2, \dots, k$, points $p_{i-1}$, $p_i$,
and $q_{i-1}$ define a triangle in the {\Large\varhexagon}-Delaunay
triangulation $T$ on $P$. A similar argument shows that for every
$i = 1, 2, \dots, k$, points $q_{i-1}$, $q_i$, and $p_i$ define a triangle
in $T$ and so the triangulation is as shown in Fig.~\ref{fig:lower}-(b).

A shortest path from $p$ to $q$ in $T$ is, for example,
$p=p_{0}, p_1, \dots, p_k, q$ and so $d_T(p,q) = d_2(p_0,p_k) + d_2(p_k,q_k)$
which tends to $\frac{4}{\sqrt{3}}$ from below as $\delta \rightarrow 0$ or
$k \rightarrow \infty$.
The distance between $p$ and $q$, on the other hand is $\frac{2}{\sqrt{3}}$.
Therefore, for any $\epsilon > 0$, it is possible to choose $k$, $\delta$,
and a set $P$ of points such that the {\Large\varhexagon}-Delaunay
triangulation on $P$ has stretch factor at least $2 - \eps$.
\end{proof}
}



%% As discussed in section~\ref{sec:prelim}, segment $[st]$ could intersect
%% the outer face of the triangulation $T$. The following technical lemma
%% exposes the local structure of the triangulation in that case. We define
%% an {\em infinite hexagon} to be the limit hexagon as its radius tends to
%% $\infty$.
%% \begin{lemma}
%% \label{lem:infinity} Let 
%% $s,t \in P$ with $0 < m_{st} < \frac{1}{\sqrt{3}}$, and
%% let $(v_0,v_1), (v_1, v_2),\dots,$ $(v_{k-1}, v_k)$, with $k > 1$, be edges of
%% $T$ such that:
%% \begin{itemize}
%% \item each edge borders the outer face of $T$ on its right side
%% when travelling from $v_0$ to $v_k$,
%% \item segment $[st]$ intersects edges $(v_0,v_1)$ and $(v_{k-1},v_k)$ and no
%% other edge $(v_{i-1}, v_i)$. 
%% \end{itemize}
%% Then one of the following two cases holds:
%% \begin{itemize}
%% \item $v_1, \dots, v_{k-1}$ lie above line line $st$, $v_0$ and $v_k$ are
%% below or on line $st$, and for every $i = 1, \dots, k$, there is an empty infinite hexagon
%% with $v_{i-1}$ and $v_i$ lying on its $n_w$ and $n_e$ sides, respectively
%% (see Fig.~\ref{fig:infinity}-(a)), or 
%% \item $v_1, \dots, v_{k-1}$ lie below line line $st$, $v_0$ and $v_k$ are
%% above or on line $st$ , and for every $i = 1, \dots, k$, there is an empty infinite hexagon
%% with $v_{i-1}$ and $v_i$ lying on its $s_w$ and $s_e$ sides, respectively.
%% \end{itemize}
%% \end{lemma}

%% \begin{figure}
%% \center{\infinity}
%% \vspace{-0.1cm}
%% \center{(a) \hspace{6.35cm} (b)}

%% \caption{%Triangles $T_1$, $T_2$, \dots, $T_{12}$ are visited in that order when
%% %moving from $s$ to $t$ along the dotted segment $[st]$. The vertices of each
%% %triangle $T_i$ ($u_{i-1}$, $u_i$, $l_{i-1}$, $l_i$, two of which are equal)
%% %lie on the boundary of the hexagon $H_i$. 
%% (a) Illustration of the first case of Lemma~\ref{lem:infinity}, showing
%% a fragment of the Delaunay triangulation and of the segment $[st]$ (as
%% a dotted line); also shown are fragments of the three empty infinite hexagons
%% whose $n$ sides pass through $v_0$ and $v_1$, $v_1$ and $v_2$, and $v_2$ and
%% $v_3$. (b) With the addition to $P$ of $p^{-M}=(0,-M)$, for some large enough $M>0$, 
%% the resulting Delaunay triangulation includes triangles 
%% $\triangle(v_0, v_1, p^{-M})$, $\triangle(v_1, v_2, p^{-M})$, 
%% and $\triangle(v_2, v_3, p^{-M})$.}
%% \label{fig:infinity}
%% \end{figure}

%% \begin{proof}
%% Note first that points $v_1, \dots, v_{k-1}$ must lie on one side of line $st$
%% and points $v_0$ and $v_k$ must lie on the other or on the line itself
%% (which may occur if $v_0$ or $v_k$ correspond to points $s$ or $t$.)
%% We consider the case when $v_1, \dots, v_{k-1}$ lie above line $st$ and show
%% that the first case of the lemma holds
%% (when $v_1, \dots, v_{k-1}$ are below, a symmetric argument shows that the
%% second case holds.) The lemma assumptions imply that edge $(v_0,v_1)$ must be to the
%% left of $(v_{k-1},v_k)$, in the sense that it must be crossed earlier when
%% travelling along $[st]$ from $s$ to $t$. 

%% We will say that the side to our right when travelling along the path
%% $v_0, v_1, v_2, \dots, v_k$ is {\em below} each edge; the side to our left is
%% then {\em above}. Note that the lemma's and our assumptions imply that
%% the outer face lies below each edge $(v_{i-1},v_i)$. This means that for
%% every edge $(v_{i-1},v_i)$, there must exist an {\em empty} infinite
%% hexagon $H^\infty_i$ with $v_{i-1}$ and $v_i$ lying on adjacent sides $s_1$ and
%% $s_2$, respectively, of the hexagon such that $s_2$ immediately follows $s_1$
%% when considering the sides of $H^\infty_i$ in clockwise order.

%% Now, if $v_0$ and $v_1$ lie, respectively, on the $w$ and $n_w$ or the $s_w$ and
%% $w$ sides of $H^\infty_1$ then point $t$ would have to lie in its interior, a
%% contradiction. Similarly, if $v_0$ and $v_1$ lie, respectively, on the
%% $n_e$ and $e$ or the $e$ and $s_e$ sides of $H^\infty_1$, then point $s$ would
%% have to lie in the interior of $H^\infty_1$, again a contradiction. Finally, 
%% if $v_0$ and $v_1$ lie on
%% the $s_e$ and $s_w$ sides of $H_1^\infty$ then $H_1^\infty$ would intersect the
%% interior of some edge $(v_{i-1},v_i)$ for $i > 1$ which is not possible
%% because any hexagon with $v_{i-1}$ and $v_i$ on its boundary, including
%% $H^\infty_i$, would have to
%% have $v_0$ or $v_1$ in its interior. Therefore $v_0$ and
%% $v_1$ must lie on $n_w$ and $n_e$ sides, respectively, of $H_1^\infty$.

%% Having proven the lemma for $i=1$, we now use induction to complete the proof.
%% We assume that $v_{i-1}$ and $v_i$ lie on $n_w$ and $n_e$ sides, respectively,
%% of $H_i^\infty$. Consider the infinite empty hexagon $H^{\infty}$ having $v_i$
%% on its $n_w$ side that is rightmost as defined by the abscissa of the center of
%% the hexagon. This hexagon exists because 1) there is an infinite empty hexagon
%% having $v_i$ as its north vertex and 2) there is an infinite, but not
%% necessarily empty, hexagon having
%% $v_i$ on its $n_w$ side and $v_k$ on its $n_e$ side. Because there must be a
%% point of $P$ on the $n_e$ side of $H^\infty$ incident to an edge to $v_i$ that
%% has the outer face below, that point must be $v_{i+1}$.
%% \end{proof}

%% So, if $[st]$ intersects the outer face of $T$ and in order to avoid special
%% cases and streamline our arguments, we add to point set $P$ two additional
%% points, $p^{-M}$ at $(0,-M)$ and $p^M$ at $(0,M)$ for some positive value
%% $M > 2d_x(s,t)+2d_y(s,t)$, and consider the {\Large\varhexagon}-Delaunay
%% triangulation on this modified set of points (see Fig.~\ref{fig:infinity}-(b)).
%% Note that for every path $(v_0,v_1), (v_1, v_2),\dots,(v_{k-1}, v_k)$ satisfying
%% the first case of Lemma~\ref{lem:infinity} in the original triangulation,
%% $p^{-M}$
%% would have to be inside the infinite hexagon with $v_{i-1}$ and $v_i$ lying on
%% its $n_w$ and $n_e$ sides, for every $i = 1, \dots, k$. Therefore, the
%% modified triangulation includes triangles $\triangle(v_0,v_1,p^{-M}), \triangle(v_1, v_2, p^{-M}),\dots,\triangle(v_{k-1}, v_k, p^{-M})$. 
%% Similarly, for every path $(v_0,v_1), (v_1, v_2),\dots,(v_{k-1}, v_k)$ satisfying
%% the second case of Lemma~\ref{lem:infinity} in the original triangulation, the
%% modified triangulation includes triangles $\triangle(v_0,v_1,p^{M}), \triangle(v_1, v_2, p^{M}),\dots,$ $\triangle(v_{k-1}, v_k, p^{M})$. 
%% So, the modified triangulation is exactly
%% the original one if points $p^{-M}$ and $p^M$ and the edges incident to them are
%% removed. Furthermore, our choice of $M$ ensures that $p^{-M}$ and $p^M$ cannot
%% be part of any path from $s$ to $t$ whose length is within the bound in
%% (\ref{eq:main}). Thefore we will prove Lemma~\ref{le:divide} on the original
%% point set if we show that it holds for the modified point set. Finally, we 
%% note that the triangles of the {\Large\varhexagon}-Delaunay
%% triangulation on the modified set of points completely cover the segment $[st]$
%% .

%% %to obtain the set of points $P^M$ and the {\Large\varhexagon}-Delaunay
%% %triangulation $T^M$ on $P^M$ (see Fig.~\ref{fig:infinity}-(b)). 
%% %Note that, by Lemma~\ref{lem:infinity},
%% %$T^M$ is exactly $T$ if points $p^{-M}$
%% %and $p^M$ and the edges incident to them are removed. Our choice of $M$
%% %ensures that
%% %$p^{-M}$ and $p^M$ cannot be part of any path from $s$
%% %to $t$ whose length is within the bound in (\ref{eq:main}). Finally,
%% %note that the triangles of $T^\infty$ completely cover the segment $[st]$.

