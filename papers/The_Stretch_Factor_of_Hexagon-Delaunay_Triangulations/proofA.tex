\begingroup
\def\thetheorem{\ref{le:mainlemmaA}}
\begin{lemma}[The Technical Lemma]
%If $T_{1n}$ is regular, $p$ is the left induction vertex of $T_1$, and
%$q$ is the right induction vertex of $T_n$ then
%\begin{itemize}
%\item the slope of the line through $p$ and $q$ is between $-\sqrt{3}$ and 
%$\sqrt{3}$
%\item there is a path in $T_{1n}$ from $p$ to $q$ of length at most $\frac{4}{\%sqrt{3}} d_x(p,q)$.
%\end{itemize}
If $T_{1n}$ is a regular linear sequence of triangles then there is a path in
$T_{1n}$ from
the left induction vertex $p$ of $T_1$ to the right induction vertex $q$ of
$T_n$ of length at most  $\frac{4}{\sqrt{3}} d_x(p,q)$.
\end{lemma}
\addtocounter{theorem}{-1}
\endgroup

To prove this lemma we develop a framework that uses continuous versions
of the discrete functions ($p_N$, $p_S$, etc.) informally introduced in
Subsection~\ref{sub:keylemma}. We start by defining functions $H(x)$, $u(x)$,
and $\ell(x)$ for
${\tt x}(p) \leq x \leq {\tt x}(q)$:
\begin{itemize}
\item If $c_i$ is the center of hexagon $H_i$,
for $i = 1, \dots, n$, we define $H({\tt x}(c_i)) = H_i$. We also define
$u({\tt x}(c_i))$ and $\ell({\tt x}(c_i))$ to be $u_i$ and $l_i$,
respectively. 
\item Then, for every $i = 1, \dots, n-1$ and $x$ such that
${\tt x}(c_i) < x < {\tt x}(c_{i+1})$,
we define $H(x)$ to be the hexagon whose center has abscissa $x$ and that
has points $u_i$ and $l_i$ on its boundary; we also define $u(x)$ to be $u_i$
and $\ell(x)$ to be $l_i$ (see Fig.~\ref{fig:notation}).
\iftoggle{abstract}
{\begin{figure}}
{\begin{figure}}
\center{\notation}
\caption{For $x$ such that
${\tt x}(c_i) < x < {\tt x}(c_{i+1})$, $H(x)$ is the hexagon whose center has abscissa
$x$ and that has points $u_i=u(x)$ and $l_i=\ell(x)$ on its boundary.
Intuitively, function $H(x)$ from $x= {\tt x}(c_i)$ to $x= {\tt x}(c_{i+1})$
models the ``pushing''
of hexagon $H_i$ through $u_i$ and $l_i$ up until it becomes $H_{i+1}$. Function
${\tt r}(x)$ is the minimum radius of $H(x)$ and $w(x) = x-{\tt r}(x)$ and
$e(x) = x + {\tt r}(x)$ are the abscissas of the $w$ and $e$ sides, respectively,
of $H(x)$. $N(x)$ and $S(x)$ are the vertices $N$ and $S$ of $H(x)$.}
\label{fig:notation}
\end{figure}
We note that $H(x)$ is
uniquely defined because $T_{1n}$ contains no gentle edges and so $u(x)$ and
$\ell(x)$ cannot lie on sides $e$ and $w$, respectively, or on sides $w$ and
$e$, respectively, of $H(x)$.
\item
As we will soon see, function $H(x)$ has a specific growth pattern that depends
on what sides of $H(x)$ points $u(x)$ and $\ell(x)$ lie on. In order to
simplify our presentation, we define $H(x)$ when 
${\tt x}(p) \leq x < {\tt x}(c_1)$ and ${\tt x}(c_n) < x \leq {\tt x}(q)$ 
in a way that fits that pattern. Let $W_S^*$ be vertex
$W_S$ of $H_1=H({\tt x}(c_1))$ and let $H^*$ be the hexagon with $p$ and $W_S^*$
as its $W_N$ and $W_S$ vertices, respectively. Let $c^*$ be the center of $H^*$.
When ${\tt x}(p) \leq x \leq {\tt x}(c^*)$ we define $H(x)$ to be the hexagon 
whose center has abscissa $x$ and that has point $p$ as its $W_N$ vertex; 
we also define $u(x) = \ell(x) = p$. When 
${\tt x}(c^*) \leq x < {\tt x}(c_1)$ we define $H(x)$ to be the hexagon whose
center has abscissa $x$ and that has point $W_S^*$ as its $W_S$ vertex; we also
define $u(x) = \ell(x) = p$. We define
$H(x)$ when ${\tt x}(c_n) < x \leq {\tt x}(q)$ in a symmetric fashion
with $u(x) = \ell(x) = q$ in that case.
\end{itemize}

Next, we define ${\tt r}(x)$ to be the minimum radius of $H(x)$ (i.e., the
distance between the center of $H(x)$ to its $w$ side). Note that hexagons 
$H({\tt x}(p))$ and
$H({\tt x}(q))$ both have radius $0$ and define their centers to be $c_0 = p$
and $c_{n+1}=q$. We also extend the notation $T_{ij}$ to include $T_{10} = c_0$
and define $T(x) = T_{1i}$ when ${\tt x}(c_i) \leq x < {\tt x}(c_{i+1})$ and
$T({\tt x}(c_{n+1}))=T_{1n}$.
We define $N(x)$ and $S(x)$ to be the $N$ and $S$ vertex, respectively, of
$H(x)$. Finally, we define functions $w(x) = x-{\tt r}(x)$ and
$e(x) = x + {\tt r}(x)$ that keep track of the abscissa of the $w$ and $e$
sides, respectively, of $H(x)$ (refer to Fig.~\ref{fig:notation}). Note that
functions ${\tt r}(x), w(x), e(x)$ as well as functions ${\tt y}(N(x))$ and
${\tt y}(S(x))$ (the ordinates of $N(x)$ and $N(y)$, resp.) are continuous. 


\begin{figure}[!b]
\dNdS

\vspace{0.3cm}
\hspace{1.45cm} (a) \hspace{4.35cm} (b) \hspace{4.55cm} (c)
\caption{(a) The values of $p_N(o, x)$ are shown, for various points $o$
lying on the boundary of $H(x)$, as signed hexagon arc lengths. (b) The 
values of $p_S(o, x)$ are shown similarly. (c) The length of edge
$(u_{i-1},u_i)$, with $u_{i-1},u_i$ lying on the boundary of $H(x)=H_i$, 
is bounded by $p_N(u_{i-1}, x) - p_N(u_i,x)$.}
\label{fig:dNdS}
\end{figure}


For a point $o$ on a side of $H(x)$, we define functions $p_N(o, x)$ and 
$p_S(o, x)$ as the {\em signed} shortest distances around the perimeter of
$H(x)$ to the $N$ vertex and $S$ vertex, respectively, with sign 
$\sgn(x - x(o))$. As Fig.~\ref{fig:dNdS}-(a) and Fig.~\ref{fig:dNdS}-(b)
illustrate, these signs are positive for $o$ on the $n_w$, $w$, or $s_w$
sides of $H(x)$ and negative for $o$ on $n_e$, $e$, or $s_e$ sides. 
We omit $o$ and use the shorthand notation $p_N(x)$ if $o = u(x)$ and
$p_S(x)$ if $o = \ell(x)$. 

\begin{figure}
\begin{center}\UandL

\vspace{0.6cm}
(a) \hspace{6cm} (b)
\end{center}
\caption{(a) Definition of $U(x)$ and $L(x)$. For example, $U(x)$ for 
${\tt x}(c_i) \leq x < {\tt x}(c_{i+1})$ is the sum
of the length of the shortest path from $p$ to $u_i$ in $T_{1i}$ 
(illustrated as the red dashed path) and $p_N(x)$ (of negative value and 
represented as a red arrow). (b) Definition of $\bar{U}(x)$ and $\bar{L}(x)$. $\bar{U}(x) - p_N(x)$,
for example, is an upper bound (equal to the length of the 
sequence of red dashed
hexagon arcs going from $p$ to $u_i$) on the length
of the upper path $p, u_0, u_1, \dots, u_{i-1}, u_i$.}
\label{fig:UandL}
\end{figure}


The key functions $U(x)$ and $L(x)$, illustrated in Fig.~\ref{fig:UandL}-(a), 
are defined as follows for
${\tt x}(p) = {\tt x}(c_0) \leq x \leq {\tt x}(c_{n+1}) = {\tt x}(q)$:
%\begin{align*}
%& U(x) = d_{T_{1i}(p,u_i) + p_N(x) \mbox{ when } {\tt x}(c_i) \leq x < {\tt x}(c_{i+1}) & U({\tt x}(q)) = d_{T_{1n}}(p,q)) \\
%& L(x) = d_{T_{1i}}(p,l_i) + p_S(x) \mbox{ when } {\tt x}(c_i) \leq x < {\tt x}(c_{i+1}) & L({\tt x}(q)) = d_{T_{1n}}(p,q))
%\end{align*}
\begin{equation*}
U(x) = d_{T(x)}(p,u(x)) + p_N(x) \mbox{\hspace{2cm}} L(x) = d_{T(x)}(p,\ell(x)) + p_S(x)
\end{equation*}
Finally, we define the {\em potential function} $P(x)$ to be $U(x) + L(x)$. 
We note that
$P({\tt x}(q))$ is exactly twice the distance in $T_{1n}$ from $p$ to $q$. The 
main goal of this paper is to compute an upper bound for function $P(x)$.
We will do this by bounding its growth rate.

For ${\tt x}(c_i) < x < {\tt x}(c_{i+1})$, the terms 
$d_{T(x)}(p,u(x))$ and $d_{T(x)}(p,\ell(x))$ in the definitions of 
$U(x)$ and $L(x)$, respectively, are constant. 
This means that for such $x$ the rate of growth of
functions $U(x)$ and $L(x)$ is determined solely by the terms 
$p_N(x)$ and $p_S(x)$, respectively. 




We note that the length of each edge $(u_{i-1},u_i)$ 
(assuming $u_{i-1} \not= u_i$) can be bounded by the distance
from $u_{i-1}$ to $u_i$ when traveling clockwise along the sides of $H_i$.
This distance is exactly $p_N(u_{i-1}, {\tt x}(c_i)) - p_N(u_i,{\tt x}(c_i))$ as
illustrated in Fig.~\ref{fig:dNdS}-(c).
This, and a similar observation about each edge $(l_{i-1},l_i)$, motivates
the following definitions of functions $\bar{U}(x)$ and $\bar{L}(x)$ that we
will use to bound lengths of subpaths of the upper path $p,u_0, \dots,u_n,q$ 
and the lower path $p, l_0, \dots, l_n,q$ (see also Fig.~\ref{fig:UandL}-(b)). 
When ${\tt x}(c_i) \leq x < {\tt x}(c_{i+1})$ or $x = {\tt x}(c_{n+1}) = {\tt x}(q)$, we define
\begin{align*} 
\bar{U}(x) & = \sum_{j=1}^{i} (p_N(u({\tt x}(c_{j-1})), {\tt x}(c_{j})) - p_N(u({\tt x}(c_{j})),{\tt x}(c_{j}))) + p_N(x) \\
\bar{L}(x) & = \sum_{j=1}^{i} (p_S(\ell({\tt x}(c_{j-1})), {\tt x}(c_{j})) - p_S(\ell({\tt x}(c_{j})),{\tt x}(c_{j}))) + p_S(x)
\end{align*}
%\begin{align*} 
%\bar{U}(x) & = \begin{cases}
%p_N(x) & \mbox{ if } i=0 \\
%\sum_{j=0}^{i-1} (p_N(u_j, {\tt x}(c_{j+1})) - p_N(u_{j+1},{\tt x}(c_{j+1})))
%+ p_N(x) & \mbox{ if } i > 0
%\end{cases} \\
%\bar{L}(x) & = \begin{cases}
%p_S(x) & \mbox{ if } i=0 \\
%\sum_{j=0}^{i-1} (p_S(l_j, {\tt x}(c_{j+1})) - p_S(l_{j+1},{\tt x}(c_{j+1})))
%+ p_S(x) & \mbox{ if } i > 0
%\end{cases}
%\end{align*}
and let ${\bar P}(x)={\bar U}(x) + {\bar L}(x)$. We note that
${\bar P}(x)$ is an upper bound for $P(x)$ and that ${\bar P}({\tt x}(q))$
bounds the sum of the lengths of paths $p,u_0, \dots,u_n,q$ 
and $p, l_0, \dots, l_n,q$. Functions ${\bar U}(x)$ and ${\bar L}(x)$, 
just like $U(x)$ and $L(x)$, have rates of growth when
${\tt x}(c_i) < x < {\tt x}(c_{i+1})$ that are determined solely by the
last term ($p_N(x)$ or $p_S(x)$). We will show that functions 
$p_N(x)=p_N(u(x),x)$ and $p_S(x) = p_S(\ell(x),x)$ are monotonically
increasing piecewise linear functions whose rates of growth depend solely
on the sides of $H(x)$ that $u(x)$ and $\ell(x)$ lie on. In order to capture
precisely this rate of growth, we define the {\em transition
function} $t(x)$ to be {\em transition} $t_{ij}$ if $\ell(x)$ lies in the interior
of side $i$ and $u(x)$ lies in the interior of side $j$. We use the wildcard
notations $t_{\ast j}$ and $t_{i \ast}$ to refer to any
transition with $u(x)$ on side $j$ and $\ell(x)$ on side $i$, respectively,
of $H(x)$. 
%We can now define the growth rates of the various functions we will need:


\newpage
\begin{lemma}
\label{lem:growthrates}
%\label{lem:transitions}
Given the assumptions of Lemma~\ref{le:mainlemmaA}, for every 
$x \in [{\tt x}(p), {\tt x}(q)]$:
\begin{itemize}
\item $t(x)$, when defined, is one of
\begin{equation*}
t_{wn_w},t_{wn_e},t_{s_ww},  t_{s_wn_w}, t_{s_wn_e},  t_{s_we},  t_{s_ew},  t_{s_en_w},  t_{s_en_e},  t_{s_ee},  t_{en_w},  t_{en_e}.
\end{equation*}
\item Functions ${\tt y}(N(x)), {\tt y}(S(x)), {\tt r}(x), w(x), e(x)$
and, for $x \in [{\tt x}(c_i), {\tt x}(c_{i+1}))$ and \newline $i=0,\dots,n-1$,
functions $p_N(x)$, $p_S(x)$, and ${\bar P}(x)$ are all
piecewise linear functions with the following growth rates where defined:
\end{itemize}\vspace{-0.5cm}\begin{center}
{\tabulinesep=1.2mm
\begin{tabu} to \textwidth{X[2l]X[r]X[r]X[r]X[r]X[r]X[r]X[r]X[r]X[r]X[r]X[r]X[r]}
& \transitiononezero & \transitiononefive & \transitiontwoone & \transitiontwozero & \transitiontwofive & \transitiontwofour & \transitionthreeone & \transitionthreezero & \transitionthreefive & \transitionthreefour & \transitionfourzero & \transitionfourfive \\
{$\bm{t(x)}$} &  {$\bm{t_{wn_w}}$} & {$\bm{t_{wn_e}}$} &  {$\bm{t_{s_ww}}$} & {$\bm{t_{s_wn_w}}$} &  {$\bm{t_{s_wn_e}}$} & {$\bm{t_{s_we}}$} &  {$\bm{t_{s_ew}}$} & {$\bm{t_{s_en_w}}$} &  {$\bm{t_{s_en_e}}$} & {$\bm{t_{s_ee}}$} &  {$\bm{t_{en_w}}$} & {$\bm{t_{en_e}}$} \\ \hline
{$\bm{\frac{\Delta {\bar P}(x)}{\Delta x}}$ } & $\frac{6}{\sqrt{3}}$ & $\frac{8}{\sqrt{3}}$ & $\frac{6}{\sqrt{3}}$ & $\frac{4}{\sqrt{3}}$ & $\frac{4}{\sqrt{3}}$ & $\frac{8}{\sqrt{3}}$ & $\frac{8}{\sqrt{3}}$ & $\frac{4}{\sqrt{3}}$ & $\frac{4}{\sqrt{3}}$ & $\frac{6}{\sqrt{3}}$ & $\frac{8}{\sqrt{3}}$ & $\frac{6}{\sqrt{3}}$ \\ \hline
{$\bm{\frac{\Delta p_N(x)}{\Delta x}}$ } & $\frac{2}{\sqrt{3}}$ & $\frac{2}{\sqrt{3}}$ & $\frac{4}{\sqrt{3}}$ & $\frac{2}{\sqrt{3}}$ & $\frac{2}{\sqrt{3}}$ & $\frac{6}{\sqrt{3}}$ & $\frac{6}{\sqrt{3}}$ & $\frac{2}{\sqrt{3}}$ & $\frac{2}{\sqrt{3}}$ & $\frac{4}{\sqrt{3}}$ & $\frac{2}{\sqrt{3}}$ & $\frac{2}{\sqrt{3}}$ \\ \hline
{$\bm{\frac{\Delta {\tt y}(N(x))}{\Delta x}}$ } & $\frac{1}{\sqrt{3}}$ & $-\frac{1}{\sqrt{3}}$ & $\frac{3}{\sqrt{3}}$ & $\frac{1}{\sqrt{3}}$ & $-\frac{1}{\sqrt{3}}$ & $-\frac{5}{\sqrt{3}}$ & $\frac{5}{\sqrt{3}}$ & $\frac{1}{\sqrt{3}}$ & $-\frac{1}{\sqrt{3}}$ & $-\frac{3}{\sqrt{3}}$ & $\frac{1}{\sqrt{3}}$ & $-\frac{1}{\sqrt{3}}$ \\ \hline
{$\bm{\frac{\Delta p_S(x)}{\Delta x}}$ } & $\frac{4}{\sqrt{3}}$ & $\frac{6}{\sqrt{3}}$ & $\frac{2}{\sqrt{3}}$ & $\frac{2}{\sqrt{3}}$ & $\frac{2}{\sqrt{3}}$ & $\frac{2}{\sqrt{3}}$ & $\frac{2}{\sqrt{3}}$ & $\frac{2}{\sqrt{3}}$ & $\frac{2}{\sqrt{3}}$ & $\frac{2}{\sqrt{3}}$ & $\frac{6}{\sqrt{3}}$ & $\frac{4}{\sqrt{3}}$ \\ \hline
{$\bm{\frac{\Delta {\tt y}(S(x))}{\Delta x}}$ } & $-\frac{3}{\sqrt{3}}$ & $-\frac{5}{\sqrt{3}}$ & $-\frac{1}{\sqrt{3}}$ & $-\frac{1}{\sqrt{3}}$ & $-\frac{1}{\sqrt{3}}$ & $-\frac{1}{\sqrt{3}}$ & $\frac{1}{\sqrt{3}}$ & $\frac{1}{\sqrt{3}}$ & $\frac{1}{\sqrt{3}}$ & $\frac{1}{\sqrt{3}}$ & $\frac{5}{\sqrt{3}}$ & $\frac{3}{\sqrt{3}}$ \\ \hline
{$\bm{\frac{\Delta {\tt r}(x)}{\Delta x}}$ } & $1$ & $1$ & $1$ & $\frac{1}{2}$ & $0$ & $-1$ & $1$ & $0$ & $-\frac{1}{2}$ & $-1$ & $-1$ & $-1$ \\ \hline
{$\bm{\frac{\Delta w(x)}{\Delta x}}$ } & $0$ & $0$ & $0$ & $\frac{1}{2}$ & $1$ & $2$ & $0$ & $1$ & $\frac{3}{2}$ & $2$ & $2$ & $2$ \\ \hline
{$\bm{\frac{\Delta e(x)}{\Delta x}}$ } & $2$ & $2$ & $2$ & $\frac{3}{2}$ & $1$ & $0$ & $2$ & $1$ & $\frac{1}{2}$ & $0$ & $0$ & $0$ \\ \hline
\end{tabu}}
\end{center}
\end{lemma}

\begin{figure}[!b]
\growth

\hspace{2cm} (a) \hspace{3.8cm} (b) \hspace{4.2cm} (c)
\caption{Constructions demonstrating Lemma~\ref{lem:growthrates} for
transitions (a) $t_{s_wn_e}$, (b) $t_{wn_w}$, and (c) $t_{s_we}$. In all three
cases the growths shown are with respect to $\Delta x = 1$. $\Delta {\tt r}(x)$
can be obtained from $\frac{1}{2}(\Delta e(x) - \Delta w(x))$ and
$\Delta P(x)$ from $\Delta p_N(x) + \Delta p_S(x)$. Note that the growth
$\Delta p_N(x)$ when $t(x)$ is a single transition can be represented using
a piecewise linear curve of length $\Delta p_N(x)$.}
\label{fig:growth}
\end{figure}

\begin{proof}
The first part follows from the definition of a regular linear sequence of
triangles. The growth rates for transitions
$t_{s_wn_e}$, $t_{wn_w}$, and $t_{s_we}$ follow from elementary
geometric constructions illustrated in Fig.~\ref{fig:growth}. The constructions
for the remaining transitions are similar.
\end{proof}



%We can now prove Lemma~\ref{lem:rotation}:
%\begin{proof}[Proof of Lemma~\ref{lem:rotation}]
%The growth rate of $P(x)$ is the growth rate of $p_N(x)+p_S(x)$ which
%is at most $\frac{8}{\sqrt{3}}$ by Lemma~\ref{lem:growthrates}. This
%means that there is a path from $p$ to $q$ in $T_{1n}$ whose length is at
%most $\frac{4}{\sqrt{3}}d_{\tt x}(p,q) \leq (\sqrt{3}+\mbox{slope}(p,q))d_x(p,q)$.
%\end{proof}

We now consider the behavior of $U(x)$, $L(x)$, ${\bar U}(x)$, and ${\bar L}(x)$
at $x = {\tt x}(c_i)$ for $i=1,\dots,n+1$. Note that 
\begin{align*}
\bar{U}({\tt x}(c_i)) & = \sum_{j=1}^{i} (p_N(u({\tt x}(c_{j-1})), {\tt x}(c_{j})) - p_N(u({\tt x}(c_{j})),{\tt x}(c_{j}))) + p_N(u_i, {\tt x}(c_i)) \\
& = \sum_{j=1}^{i-1} (p_N(u({\tt x}(c_{j-1})), {\tt x}(c_{j})) - p_N(u({\tt x}(c_{j})),{\tt x}(c_{j}))) + p_N(u({\tt x}(c_{i-1})), {\tt x}(c_i))
%\bar{U}({\tt x}(c_i)) & = \sum_{j=0}^{i-1} (p_N(u_j, {\tt x}(c_{j+1})) - p_N(u_{j+1},{\tt x}(c_{j+1}))) + p_N(u_i, {\tt x}(c_i)) \\
%& = \sum_{j=0}^{i-2} (p_N(u_j, {\tt x}(c_{j+1})) - p_N(u_{j+1},{\tt x}(c_{j+1}))) +
%(p_N(u_{i-1}, {\tt x}(c_{i}))
\end{align*}
which is the limit for ${\bar U}(x)$ when $x \rightarrow {\tt x}(c_i)$ from 
the left.
So ${\bar U}(x)$ is continuous from ${\tt x}(c_0)$ to ${\tt x}(c_{n+1})$ and,
similarly, so is ${\bar L}(x)$ and therefore ${\bar P}(x)$ as well.
Since, by Lemma~\ref{lem:growthrates}, ${\bar P}({\tt x}(q))$ is bounded by
$\frac{8}{\sqrt{3}}d_x(p,q)$, it follows that the sum of the lengths of the upper path
$p,u_0, \dots,u_n,q$ and the lower path $p, l_0, \dots, l_n,q$ is at most 
$\frac{8}{\sqrt{3}}d_x(p,q)$. Therefore, one of the two paths has length
bounded by $\frac{4}{\sqrt{3}}d_x(p,q)$ which proves the Technical Lemma:

\begin{proof}[Proof of (Technical) Lemma~\ref{le:mainlemmaA}]
$d_{T_{1n}}(p,q) \leq \frac{1}{2}{\bar P}({\tt x}(q)) \leq \frac{4}{\sqrt{3}}d_x(p,q)$
by Lemma~\ref{lem:growthrates}.% and Lemma~\ref{lem:discontinuity}.
\end{proof}

In order to prove the stronger bound of Amortization Lemma~\ref{le:mainlemmaB},
in addition to edges of the upper and lower path we will need to include edges
$(l_i,u_i)$ in the analysis and obtain a tighter bound on $P(x)$.
For this reason, we complete the growth rate analysis of
functions $U(x)$, $L(x)$, and $P(x)$. We show that they are not necessarily 
continuous at $x = {\tt x}(c_i)$ but, when discontinuous, they do not increase:
\begin{lemma} 
\label{lem:discontinuity}
Functions $U(x)$, $L(x)$, and $P(x)$, when discontuous at $x={\tt x}(c_i)$
for some $i = 1, \dots, n$, do not increase at $x$. 
\end{lemma}
\begin{proof}
%Consider ${\tt x}(c_i)$ for some $i=1,\dots,n$. 
First note that either 
$u_i \not= u_{i-1}$ and $u({\tt x}(c_{i-1})) = u_{i-1}$, or
$u_i = u_{i-1}$ and $i > 1$ and $u({\tt x}(c_{i-1})) = u_{i-1}$, or 
$u_i = u_{i-1}$ and $i = 1$ and $u({\tt x}(c_{i-1})) = p$.  

In the second case, we note that $l_{i-1} \not= l_i$ and that point $l_i$
cannot be on the shortest path from $p$ to $u_i$ in $T_{1i}$. Therefore 
$U({\tt x}(c_i))=d_{T_{1i}}(p,u_i) + p_N(u_i, {\tt x}(c_i)) = d_{T_{1{i-1}}}(p,u_{i-1}) + p_N(u_{i-1}, {\tt x}(c_i))$. The last term is the limit for $U(x)$ when $x \rightarrow {\tt x}(c_i)$ from
the left so $U(x)$ is continuous around $x = {\tt x}(c_i)$ which completes the 
proof for this case.

For the first and third cases we set $u^* =  u({\tt x}(c_{i-1}))$. Then
\begin{align*} 
U({\tt x}(c_i)) & = d_{T_{1i}}(p,u_i) + p_N(u_i, {\tt x}(c_i)) \\ 
       & \leq d_{T_{1(i-1)}}(p,u^*) + d_2(u^*,u_i) + p_N(u_i, {\tt x}(c_i)) \\ 
       & \leq d_{T_{1(i-1)}}(p,u^*) + p_N(u^*,{\tt x}(c_i)) - p_N(u_i,{\tt x}(c_i)) + p_N(u_i, {\tt x}(c_i)) \\ \nonumber
       & = d_{T_{1(i-1)}}(p,u^*) + p_N(u^*, {\tt x}(c_i)) \nonumber
\end{align*}
%\begin{align*} 
%U({\tt x}(c_i)) & = d_{T_{1i}}(p,u_i) + p_N(u_i, {\tt x}(c_i)) \nonumber \\
%       & \leq \min \begin{cases} \nonumber 
%                 d_{T_{1(i-1)}}(p,u_{i-1}) + d_2(u_{i-1},u_i) + p_N(u_i, {\tt x}(c_i)) \\ 
%                 d_{T_{1(i-1)}}(p,l_{i-1}) + d_2(l_{i-1},u_i) + p_N(u_i, {\tt x}(c_i)) \\
%                   \end{cases} \\ %\label{eq:frombelow}
%       & \leq d_{T_{1(i-1)}}(p,u_{i-1}) + p_N(u_{i-1},{\tt x}(c_i)) - p_N(u_i,{\tt x}(c_i)) + p_N(u_i, {\tt x}(c_i)) \\ \nonumber
%       & = d_{T_{1(i-1)}}(p,u_{i-1}) + p_N(u_{i-1}, {\tt x}(c_i)) \nonumber
%\end{align*}
The last term is the limit for $U(x)$ when $x \rightarrow {\tt x}(c_i)$
from the left and so the claim holds for $U(x)$.
The claim for $L(x)$ holds using equivalent arguments, and the claim for 
$P(x)$ follows from $P(x) = L(x) + U(x)$.
\end{proof}

Before we end this section, we note that we have not defined $t(x)$ at values
of $x$ when $\ell(x)$ or $u(x)$
is a vertex of $H(x)$, which is when $p_S(x)$ or $p_N(x)$, and thus
$\bar{L}(x)$ or $\bar{U}(x)$, respectively, $L(x)$ or $U(x)$, respectively, 
and $P(x)$ are not smooth
and differentiable. In what follows, for clarity of presentation we will
sometimes abuse our definition of $t(x)$ to include such points. Since there
are only a finite number of such points, they do not affect our analysis.




