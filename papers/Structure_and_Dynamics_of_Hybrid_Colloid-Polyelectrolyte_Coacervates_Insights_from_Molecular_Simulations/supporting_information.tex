\documentclass[journal=mamobx, manuscript=article]{achemso}

\usepackage{eurosym}
\usepackage{enumitem}
\usepackage{tabularx}
\usepackage{graphics}
\usepackage{epsfig}
\usepackage{multirow}
\usepackage{amssymb}
\usepackage{amsmath}
\usepackage{leftidx}
\usepackage{epsfig}
\usepackage{color,soul}
\usepackage[explicit]{titlesec}
\usepackage{natbib}

\DeclareMathOperator{\Tr}{Tr}

\title{\LARGE Supporting Information \\
\bigskip
\Large Structure and Dynamics of Hybrid Colloid-Polyelectrolyte Coacervates: Insights from Molecular Simulations}


\author{Boyuan Yu}
\affiliation{Pritzker School of Molecular Engineering, University of Chicago, Chicago, Illinois 60637, United States}

\author{Heyi Liang}
\affiliation{Pritzker School of Molecular Engineering, University of Chicago, Chicago, Illinois 60637, United States}

\author{Paul F. Nealey}
\affiliation{Pritzker School of Molecular Engineering, University of Chicago, Chicago, Illinois 60637, United States}
\alsoaffiliation{Center for Molecular Engineering, Argonne National Laboratory, Lemont, Illinois 60439, United States}

\author{Matthew Tirrell}
\affiliation{Pritzker School of Molecular Engineering, University of Chicago, Chicago, Illinois 60637, United States}
\alsoaffiliation{Center for Molecular Engineering, Argonne National Laboratory, Lemont, Illinois 60439, United States}


\author{Artem~M.~Rumyantsev}
\affiliation{Pritzker School of Molecular Engineering, University of Chicago, Chicago, Illinois 60637, United States}
\alsoaffiliation{Department of Chemical and Biomolecular Engineering, North Carolina State University, Raleigh, North Carolina 27695-7905, United States}


\author{Juan~J.~de~Pablo}
\affiliation{Pritzker School of Molecular Engineering, University of Chicago, Chicago, Illinois 60637, United States}
\alsoaffiliation{Center for Molecular Engineering, Argonne National Laboratory, Lemont, Illinois 60439, United States}
\email{depablo@uchicago.edu}

\begin{document}
\renewcommand{\thefigure}{S\arabic{figure}}
\renewcommand{\thetable}{S\arabic{table}}

\newpage

\section{Effect of the Colloid Charge Assignment on the Coacervate Properties}
\label{subsection:charge-assignment}

To test the effects of charge assignment for each particle on the coacervate properties, we use the same model as described in the main text except that the charges are not assigned as a point-like charge at the center of the particle. Instead, for each particle, we discretize the spherical surface into $n$ virtual interaction sites by Fibonacci sphere algorithm,~\cite{Gonzalez2010-bg} and each site carries the charge $z_{i} = Q/n$. Then we group the spherical particle (with no charge now) and its $n$ virtual charged sites into one rigid body, i.e. the new charged particle. To maintain the same excluded volume interactions, each virtual site has no LJ interactions with other beads, particles, or sites; it only interacts with other charged beads or sites via Coulomb forces. In addition, for the new charged particle, we assign $m = 0.5$ to the spherical particle and $m = 0.5/n$ to each virtual charged site so that the new particle has the same mass as the original particle with the net charge in the center.

\begin{table}[h!]
\centering
\begin{tabular}{||c c c c||} 
 \hline
 charge distribution & $\phi$ & $H [\sigma]$ & $D_{p} [\sigma^{2}\tau_{LJ}^{-1}] \times [10^{3}\tau_{LJ}]$ \\ [0.5ex]  
 \hline\hline
 $n = 1$ & $0.267 \pm 0.001$ & $2.18 \pm 0.01$ & $3.4 \pm 0.2$ \\ 
 $n = 64$ & $0.267 \pm 0.001$ & $2.18 \pm 0.01$ & $3.3 \pm 0.2$ \\
 $n = 128$ & $0.267 \pm 0.001$ & $2.17 \pm 0.01$ & $3.5 \pm 0.2$ \\
 $n = 256$ & $0.268 \pm 0.001$ & $2.17 \pm 0.01$ & $3.5 \pm 0.2$ \\[1ex] 
 \hline
\end{tabular}
\caption{Structural and dynamic properties of the hybrid coacervate phase for the different distributions of the particle charge: $n = 1$ denotes the case of the single charge at the center of the particle; $n = 64$, $128$, and $256$ represent different degrees of the sufficiently uniform charge smearing throughout the particle surface. $\phi$ is the polymer layer density, $H$ is the polymer layer thickness, and $D_{p}$ is the diffusion coefficient of particle center of mass.
The simulation parameters are $N = 120$, $f = 0.2$, $R = 2 \sigma$, $Q = 24e$, and $l_{B} = \sigma$. 
}
\label{table:2}
\end{table}

\begin{table}[h!]
\centering
\begin{tabular}{||c c c c||} 
 \hline
 charge distribution & $\phi$ & $H [\sigma]$ & $D_{p} [\sigma^{2}\tau_{LJ}^{-1}] \times [10^{5}\tau_{LJ}]$ \\ [0.5ex]  
 \hline\hline
 $n = 1$ & $0.411 \pm 0.001$ & $2.45 \pm 0.01$ & $2.0 \pm 0.1$ \\ 
 $n = 128$ & $0.411 \pm 0.001$ & $2.46 \pm 0.01$ & $1.9 \pm 0.1$ \\
 $n = 256$ & $0.410 \pm 0.001$ & $2.45 \pm 0.01$ & $2.0 \pm 0.1$ \\[1ex] 
 \hline
\end{tabular}
\caption{Structural and dynamic properties of hybrid coacervate phase for the different particle charge distributions: $n = 1$ corresponds to the charges at the center of the particle; $n = 128$ and $256$ represent different degrees of the charge smearing throughout the particle surface. $\phi$ is the polymer layer density, $H$ is the polymer layer thickness, and $D_{p}$ is the diffusion coefficient of the particle center of mass (i.e. the center of the partcile).
The simulation parameters are $N = 200$, $f = 0.2$, $R = 2\sigma$, $Q = 40e$, and $l_{B} = \sigma$.
}
\label{table:3}
\end{table}


We perform the same NPT simulations to study the coacervate phase formed from the PE chains and the charged particles with different charge distributions, i.e. different $n$ values. The structural and dynamic properties of the hybrid coacervate phases for different charge distributions (i.e., different $n$ values) are listed in Table~\ref{table:2} and~\ref{table:3}. One can see that the coacervate properties for the particles with the charges assigned at their center ($n = 1$) are the same as for the particles carrying the surface charges with various degrees of surface charge smearing ($n = 64$, $128$, and $256$).


\begin{figure}%[ht]
\centering
\includegraphics[width=0.65\linewidth]{5-s1.pdf}
\caption{Mean squared displacement (MSD) of the center of mass of the charged particles, $MSD_{p}$, as the function of time (log-log plot). $n = 1$ is the particle with the charge at its center; $n = 64$, $128$, and $256$ represent different degrees of the charge smearing. The simulation parameters are $N = 120$, $f = 0.2$, $R = 1/5\sigma$, $Q = 24e$, and $l_{B} = \sigma$.}
\label{fig:MSD-SI}
\end{figure}

We can further compare their internal structures by calculating the partial radial distribution functions (RDF) between the different species in the coacervate phase. Figure~\ref{fig:rdf} in Section~\ref{subsection:rdf} shows that the curves for the different degrees of the surface charge smearing, $n = 64$, $128$, and $256$, and particle with the point-like charge, $n=1$, perfectly coincide. 


Finally, in Figure~\ref{fig:MSD-SI}, we construct the mean squared displacement (MSD) of the center of mass of the charged particles as the function of time to reveal both the short- and the long-time diffusion behavior of the original and the new type of particles. It is seen that the distribution of charges changes the particle diffusion at neither short nor long time scales.





\newpage

\section{Radial Density Profile of Adsorbed PE Layers}
\label{subsection:distribution}

\begin{figure}%[ht]
\centering
\includegraphics[width=0.6\linewidth]{5-s2.pdf}
\caption{The layer density $\phi$ as a function of the distance from the particle center. The simulation parameters are $N = 200$, $f = 0.2$, $R = 1.5\sigma$, $ Q = 40$, and $l_{B} = \sigma$.}
\label{fig:distri}
\end{figure}




\newpage

\section{Radial Distribution Function of Colloids and Ionic Monomers}
\label{subsection:rdf}

\begin{figure}[ht]
\centering
\includegraphics[width=\linewidth]{5-s3.pdf}
\caption{
a) Radial distribution function (RDF) of the particle, $g_{p, p}(r)$, as the function of the distance between their center of mass, $r$,  The black dash line shows the position of the first peak. 
b) RDF for the particle center of mass and the negatively charged monomer $g_{p, neg}(r)$.  
c) RDF for the positively charged sites on the particle surface and the negatively charged polyelectrolyte monomers, $g_{pos, neg}(r)$.
For all the plots, the simulation parameters are $N = 120$, $f = 0.2$, $R = 2\sigma$, $Q = 24e$, and $l_{B} = \sigma$. Different curves correspond to the different charge distributions on the colloid particle: $n1$ corresponds to the particle with a single point-like charge in the center; $n64$, $n128$, and $n256$ represent particles with the charges evenly distributed at the particle's surface (see Section~\ref{subsection:charge-assignment} for details).
}
\label{fig:rdf}
\end{figure}

\newpage

\section{Bulk Modulus Calculation}
\label{subsection:bulk}

The bulk (or the osmotic) modulus $K$ of the hybrid coacervate phase is calculated in the following way.~\cite{wu2019bulk} First, an external pressure $P$ is applied to compress the coacervate phase uniformly along three orthogonal directions, and the corresponding volume $V$ is measured. Then this process is repeated for different $P$ values to obtain the relationship between $P$ and the specific volume $V_{sp} = V/N_{total}$ per bead of any type. Here $N_{total}$ is the total number of the polymer monomers and particles in the system; recall that it is assumed that all beads have the same unit mass. To maintain the linear relationship assumption, the volume change is limited to the maximal of $5\%$.~\cite{wu2019bulk} Finally, the bulk modulus is obtained from the slope of the line fitting the dependence of the pressure on the natural logarithm of the specific volume: $K = -{dP} / {d(\ln V_{sp})}$. The representative example of this fitting procedure is shown in Figure~\ref{fig:bulk-fit}.

\begin{figure}[ht]
\centering
\includegraphics[width=0.65\linewidth]{5-s5.pdf}
\caption{The relationship between the applied external pressure $P$ and the natural logarithm of the specific volume $\ln V_{sp}$ per interacting entity (monomer bead or the particle). Simulation data are shown in blue points. A linear function fits data, and the legend shows the slope. The simulation parameters are $N = 120$, $f = 0.2$, $R = 1.5\sigma$, $Q = 24e$, and $l_{B} = \sigma$.}
\label{fig:bulk-fit}
\end{figure}

\newpage
\section{Structure Factors of the Colloids in the Hybrid Coacervates}
\label{subsection:sq-SI}

\begin{figure}[ht]
\centering
\includegraphics[width=0.8\linewidth]{sq.pdf}
\caption{The structure factors of the colloids for a) the disordered state ($Q = 24e$) and b) the crystalline state of the hybrid coacervate ($Q=150e$). The simulation parameters are $N = 120$ and $N=750$ for $Q = 24e$ and $Q=150e$, respectively; the values of $f = 0.2$ and $l_{B} = \sigma$ are the same for both systems. The structure factors here are calculated on the systems eight times larger than the systems shown in Figure 4 in the main text.}
\label{fig:d-SI}
\end{figure}

\bibliography{bibliography.bib}

\end{document}