% $Id: overview.tex 4010 2013-11-27 07:06:01Z wotte $

%\section{Overview}
%\label{sec:overview}


\section{Architecture}
\label{sec:arch}
Four goals guided our design:

\begin{enumerate}
    \item \emph{Discoverability}. Users should be able to use their pre-existing 
        knowledge of Kubernetes to discover what their application is doing, how 
        to modify it, and how Streams works in general. Streams applications 
        should also be discoverable by other workloads on the same Kubernetes 
        cluster, through standard Kubernetes mechanisms.
    \item \emph{Composability}. By using Kubernetes first-class service
        endpoints, cloud native Streams should interoperate with other applications
        and middleware without further configuration. 
    \item \emph{Application state preservation}. Stateless services are easy to
        manage in Kubernetes, as simply restarting them is always an option.
        But most Streams applications have state. We have an implicit contract
        with users that once they deploy an application, they will not lose any
        accumulated state---barring application failure and users taking
        explicit action to restart it.
    \item \emph{Backwards compatibility}. The cloud native version of Streams 
        should run legacy Streams applications unchanged. This goal means that 
        we cannot change any public APIs, and our task is to find the most 
        Kubernetes-like way to express functionality originally designed 
        for an on-premises cluster.
\end{enumerate}

We also have one anti-goal: we do not maintain API compatibility with the legacy
platform. Trying to do so would force cloud native Streams to take on
responsibilities that should belong to Kubernetes, which would end up
conflicting with our stated goals. We want user's \emph{applications} to remain
unchanged, but we are assuming that they are adopting cloud native Streams as
part of an overall effort to consolidate and simplify management and
administration.

\subsection{Overview}

All aspects of a Streams application exposed to users are represented as CRDs or
existing Kubernetes resources. We apply the patterns described in
\S~\ref{sec:patterns}: each resource is managed by a controller; when we
need to monitor the status of multiple kinds of resources in order to take an
action we use a conductor; and when multiple actors need to change the state for
a particular resource, we use a coordinator.

The CRD is the foundational unit in our design. CRDs are exposed to users in the
same way as any other Kubernetes resource, which means that representing Streams
concepts as CRDs gains not just native integration into the Kubernetes system,
but also the user interfaces. Any state that we must maintain goes into a CRD;
all state not in CRDs is ephemeral and can be lost without consequence.
Kubernetes delivers reliable event notifications when CRDs and other resources
are created, deleted and modified. Reacting to these notifications in
controllers and conductors is the primary communication mechanism between all of
the actors in our system.

The CRDs in Figure~\ref{fig:actors} define the following resources:

\begin{itemize}
    \item \code{Job}: A single Streams job. The job 
        controller initiates the job submission and tracks unique job identifiers. 
        The job conductor manages the job submission process and update 
        its status when completed.
    \item \code{ProcessingElement}: A PE in a job. The 
        PE controller tracks launch count and restores voluntarily deleted PEs.
    \item \code{ParallelRegion}: A single parallel region in a job. It 
        exposes a \code{width} attribute that can be directly altered by users using 
        \code{kubectl edit} or the Kubernetes client API. The parallel region
        controller handles width changes applied to parallel regions.
   \item \code{HostPool}: A host pool in a job.
   \item \code{Import}: An imported stream in a job. The import
       controller monitors the addition and modification of these resources and 
       matches them with export resources.
   \item \code{Export}: An exported stream in a job. The export
       controller monitors the addition and modification of these resources and 
       matches them with import resources.
    \item \code{ConsistentRegion}: A consistent region in a job. The 
        consistent region controller coordinates application checkpoints and 
        restarts for a single region.
    \item \code{ConsistentRegionOperator}: tracks all consistent regions in a 
        job. Created on-demand during a job submission with a consistent region.
        Its controllers monitor the deployments used to create the operators.
\end{itemize}


We also leverage the following Kubernetes resources:

\begin{itemize}
    \item \code{ConfigMap}: Shares job specific configuration between controllers and pods, 
        such as the graph metadata used by PEs to inform them of the operators they contain 
        and how to connect to other PEs. 
    \item \code{Deployment}: Manages the instance operator and the consistent region operator.
    \item \code{Pod}: Executes PEs. We use a pod controller to monitor and 
        manage the life cycle of pods within jobs. The pod conductor waits until all required
        resources are available before starting a pod for a PE.
    \item \code{Service}: Exports PE entrypoints as well as user-defined 
        services within PEs.
\end{itemize}

Figure~\ref{fig:actors} shows how our actors interact with each other. There
are four kinds of actions they can take:

\begin{enumerate}
    \item \emph{observes}: the actor either receives events from Kubernetes about that 
        resource, or passively views its store.
    \item \emph{creates}: the actor creates new instances of the resource through 
        commands to Kubernetes.
    \item \emph{deletes}: the actor deletes particular instances of a resource 
        through commands to Kubernetes.
    \item \emph{modifies}: the actor makes changes to an already created resource 
        through that resource's coordinator.
\end{enumerate}

None of our actors communicate directly with each other; all communication happens 
by creating, modifying or deleting Kubernetes resources.

Figure~\ref{fig:streams_v4_v6} depicts the deployed artifacts of an instance of cloud
native Streams.  The Streams \emph{instance operator} contains all the
controllers, conductors and coordinators for a Streams instance. Each Kubernetes
namespace can have one Streams instance operator. The instance operator maps to
the legacy concept of an instance. The legacy concept of a Streams
domain---management of Streams instances---is no longer needed as the Kubernetes
cluster serves that role.

In Streams, the PE is the vehicle for executing user code. PEs contain an
arbitrary number of user operators and the application runtime.  In our design,
we always assign one PE to a pod. This design decision is fundamental. It
allows us to: tie a runtime PE's life cycle to that of the pod that contains
it; fully offload PE scheduling to Kubernetes; rely on the Kubernetes DNS
service for establishing direct TCP connections between our PEs.  Any other
design would have required bespoke implementations of life cycle management,
scheduling and network name resolution.

\begin{figure}[t!]
  \centering
  \includegraphics[width=1\linewidth]{figures/streams_v4_v6}
  \caption{Legacy and cloud native Streams deployments}
  \label{fig:streams_v4_v6}
\end{figure}

While the PE runtime is unchanged, we had to implement a new translation layer
between the PE and Kubernetes. It implements the platform abstraction for the PE
runtime, as well as instantiates and initializes the PE. At runtime, it
also: collects metrics from the PE using a pre-existing interface and exposes
them to Prometheus~\cite{prometheus}; monitors and reports the status of all
PE-to-PE connections; and monitors liveness and reports it to Kubernetes.

\subsection{Loose Coupling}
\label{sec:coupling}

The legacy Streams platform was tightly coupled, which lead to operational
difficulty and implementation complexity. Cloud native Streams applies the
concept of loose coupling.

\paragraph{Name resolution:} PEs communicate with each other over TCP
connections. It is the platform's responsibility to define PE ports, give them
names, and allow PEs to find each other's ports by those names.

In legacy Streams, each PE port is assigned a \code{(peId, portId)} tuple called
a \emph{port label} that uniquely identifies that port in the instance.
At initialization, PEs must establish their remote connections to other PEs
using their port labels. To that end, each PE first creates the socket receiver
for each of its receiver ports, determines its local TCP port, and publishes to 
ZooKeeper its mapping of port label to \code{(hostname, tcpPort)}. PEs already
know the port label that each sending port needs to connect to through graph
metadata that the platform provides at PE startup. After publishing its own
receiver port labels, each PE then looks up the translation of the remote
receiver port label in ZooKeeper for each of its own sender ports and
establishes those connections.  Even with some caching (used to reestablish lost
connections), the thundering herd aspect of this initialization process and the
strain it applies to the ZooKeeper ensemble delays initial deployment times.

Cloud native Streams relies on the Kubernetes name resolution system to resolve
inter-PE connections. There are similarities with the legacy system name
resolution system, as it relies on etcd to store its Service configurations, and
it also has some currently unresolved latency issues~\cite{codacydns,machudns}.
But from an application perspective, it is easier to use as name resolution is
done using standard BSD functions such as \code{gethostbyname()}. Lastly, from
an administration perspective, it is simpler to manage as it binds directly to
the container's \code{/etc/resolv.conf} subsystem and can be easily superseded.

\paragraph{Message bus:} The message bus in legacy Streams between the platform
and PEs uses full-duplex, synchronous communication channels implemented with
JMX~\cite{jmx}. All initiated communications must succeed. Failed
communications are retried with increasing backoff delays before being
escalated as more general system failures where it may restart a PE. As job
count and PE size increases, communications tend to time out more frequently,
leading to failure escalations reducing responsiveness. We have witnessed tens
of minutes to list all the PEs in an overloaded instance.

Cloud native Streams decouples the instance operator from the PEs by relying on
the states stored in Kubernetes to achieve operational availability. The
controller pattern (\S~\ref{sec:controller}) is used by all agents
interested in keeping track of those states. Agents that need to notify the
instance operator of internal state changes do so through Kubernetes events.  In
turn, the instance operator synchronously applies those changes to the custom
resources, preventing potential race conditions (\S~\ref{sec:coordinator}).

The SPL runtime, including the PEs, are implemented in C++. As of this writing,
no library capable of implementing our controller pattern is available in C/C++.
As an alternative, we temporarily resort to a set of REST services hosted by the
instance operator. State changes within agents other than the instance operator
periodically send REST operations to those services to notify the operator of
internal changes. In turn, the operator applies these changes to the related
custom resources. Implementing the proper library and removing the REST layer is
part of our future work.

\subsection{Fault Tolerance and Rolling Upgrades}

Fault tolerance and general high-availability is a primary goal in the design of
Streams since streaming applications are expected to run for months without
interruption. To that end, the legacy Streams platform was designed such that:

\begin{enumerate}
    \item All platform related state is persisted in ZooKeeper. Upon failure, 
        platform services restart and retrieve their state from ZooKeeper.
    \item Streaming applications continue to run during platform service
        failures or upgrades.
    \item Applications seamlessly resume operations after the loss of a
        PE or a host.
\end{enumerate}

In cloud native Streams, we use Kubernetes to preserve
these attributes.

\paragraph{Persistent states:} Kubernetes exposes state persistence to users
through CRDs. Cloud native Streams makes heavy use of CRDs 
to maintain states critical for recovery. However, where the legacy
platform implementation favored storing the state of the system as-is,
cloud native Streams stores only what is necessary and sufficient
to reach the current state of the system through recomputation. The reasons
behind that radical shift in the computation versus space trade-off of
our system are:

\begin{enumerate}
    \item We discovered through empirical measurements that the
        amount of time required to perform state recomputation is negligible
        compared to other operations in the system and appear instantaneous to
        human users.
    \item Minimizing the amount of data persisted drastically reduces the
        pressure on the persistent ensemble.
    \item Re-computing intermediate state simplifies the design of our system.
\end{enumerate}

\paragraph{Instance operator:} The Streams instance operator is designed to be
resilient to its pod restarting.  All of the actors in the instance operator
will receive the full history of Kubernetes events that they are subscribed to,
allowing them to catch-up to the current state of the system.  The applications
themselves do not need the instance operator for normal operation, so they can
continue unharmed. Because of this resiliency, the instance operator can easily
recover from failure. Upgrades are also trivial: change the image for the
instance operator and restart the pod. The combination of how we defined our
CRDs, the patterns we use to manage them and Kubernetes' reliable event delivery
enable these capabilities.

\paragraph{Applications:} We consider two types of application failures:
voluntary failures, when a user deletes a resource; and involuntary failures,
when a PE crashes or a node becomes unavailable. The voluntary deletion of job
resources are caught by the \code{onDeletion()} callback in that resource's
controller. In this situation, the deleted resource is recreated
by the controller if the owning job exists and is in the \code{Submitted} state.

Special care needs to be taken in the event of pod failure or PE deletion. To
maintain Streams' application consistency logic
(\S~\ref{sec:consistentregion}), restarting a pod needs to be coordinated with
both the PE and pod controllers through a causal chain (\S~\ref{sec:causal}). 



\subsection{Distributed Application Description}
\label{subsec:prob_desc}

A distributed software application is a graph of software components
that are partitioned into processes\footnote{Components hosted within
a process are located within the same memory space} and hosted on
several interconnected computing nodes. Interaction relationships
between the components i.e. the edges can be generally grouped into
kinds: (a) Peer to peer dependencies and (b) Group interactions.

Peer to peer dependencies, are realized using \textbf{Facets} that are
collections of operations (interfaces) provided by a component and
\textbf{Receptacles} that are collections of operation required.
These two ports can be used to implement synchronous and asynchronous
point to point interactions. Generally, a peer to peer dependency
implies an order of deployment between components, i.e. the server
providing the facet should be deployed before the client with the
receptacle.

% Subhav, I'm not sure we really depend on DDS being the driver for the underlying
% pub/sub interaction.  I would make this a bit more general.
The group interactions are always asynchronous in nature and do not
imply ordered deployment. Groups interactions are realized using
\textbf{Publisher} and \textbf{Subscriber} ports, which provide a way
for components to interact in a global data space defined over
\textbf{Topic}s.
This feature is provided using the underlying OMG Data Distribution
Services (DDS) middleware.


\subsection{Deployment Resources}
% Subhav, I don't understand this subsection. Is the computation
% resource needed to do some deployment decision? But the description
% here sounds more like what an application will need.
The deployment and configuration problem is defined over a set of
\emph{deployment resources}. In general, deployment resources are
divided into \emph{computation} and \emph{communication}
resources. Computation resources correspond to hardware facilities
required to execute computation tasks at a given computation
node. These include processing speed (number of instructions per
second), memory size (amount of random access memory required), and
specific hardware required for certain tasks such as sensors, DSP,
etc.  Communication resources on the other hand correspond to
facilities required for interaction between tasks executing on
different computation nodes. This includes communication bandwidth,
available security measures such as encryption etc. Overall, these
resources create a dynamic graph that depends upon the current
connectivity in the cluster of distributed computing nodes.

\subsection{Application Deployment Configuration State}
At any time (t), the status of the distributed application deployed on
the set of distributed resources is called the application deployment
configuration. Compatibility implies that application and platform is
defined with respect to a common resource domain.

% Subhav, it is not clear why this mathematical model is needed. I
% would understand if we were solving an optimization problem. It
% might be worthwhile to mention why this is needed.
Assume that the application is represented as a tuple $A=(C, D,
\sigma, \alpha)$, where,
\begin{enumerate}
\item $C$ is the set of all component instances in the
    application, $type(c)$, is the component instance type. Note that
    there can be more than one instance of the same component in the
    application.
\item $D \subseteq C \times C$ is a peer-to-peer dependency relation
between component, where $(c_i,c_j)\in D$ iff $c_j$ services one or
more requests from $c_i$.
\item $\sigma: C \rightarrow N$ is the virtual node assignment for
each component. $N$ is the set of all computing nodes nominally
available in the cluster. Note that the nominal node assignment only
identifies a virtual node of a particular kind. During deployment this
virtual node can be mapped to any one of the available physical nodes
with same characteristics.
\item $\alpha \subseteq C \times C$ is the collocation relation, where
$(c_i,c_j)\in \alpha$ implies that $c_i$ and $c_j$ are collocated
within the same process.
\end{enumerate}

Thus, the deployment and configuration problem is to initially set up
the system such that all virtual node assignments for an application are
mapped to a concrete node. Nominally, the deployment infrastructure
always maps one virtual node to the same physical node. Then, the
deployment procedure requires instantiating all components, setting up
their connections and then activating them. Failures can occur at any
time, during the setup or after the application has been activated. A
resilient deployment and configuration infrastructure must be able to
detect and isolate the failure and then trigger the redeployment and
reconfiguration of the affected portions of the application.

In this paper we consider the failure modes which result in one of the
nodes in the group to go offline. We do not assume any time constraint
on when this failure mode can appear.


%
%Typically, the deployment procedure always maps
%
%configure the system such that
%
%
%can be divided into two related problems:
%
%\begin{itemize}
%	\item Initial:  Deploy an application $A$ on a given set of  computation Nodes $N$ in the architecture. This procedure can fail if any node $n_i$ becomes unavailable during the deployment. If this happens, the framework must detect this failure and then deploy the application on an equivalent reserve node $n_j$.
%	\item Reconfiguration: Any failures in the application components or the computing nodes after the application has been redeployed should trigger the redeployment and reconfiguration of the affected portions of the application.
%	%summarize the Apres Work
%\end{itemize}
