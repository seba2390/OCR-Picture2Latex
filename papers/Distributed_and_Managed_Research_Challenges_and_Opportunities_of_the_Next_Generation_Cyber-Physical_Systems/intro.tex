%\vspace{-0.1in}
\section{Introduction}
\label{sec:intro}

Distributed real-time embedded systems that interact with the physical
world are ubiquitous and pervasive. We are relying on an increasing
number of such systems that provide services to a large number of
users. Fractionated spacecraft (i.e., cluster of satellites)
that performs wide-area sensing of the Earth, swarms of UAVs that
survey storm damage, and the intelligent power devices that are
essential for a `smart' (power) grid are just a few illustrative
examples for this new generation of systems. While distributed and
real-time systems have been built for many decades, there are some
novel properties and requirements for the engineering of such systems
that we need to recognize and address.

First, we have to note that these systems are `cyber-physical', that is,
they interact with the physical world. Hence all software design,
implementation, and verification decisions should be guided by the fact
that physics imposes timing constraints on the computational and
communication activities, and the implementation must obey these
constraints. Furthermore, as the software system may effect changes in
its physical environment these changes must verifiably satisfy safety
requirements for the overall system.

Second, we have to understand that these systems are 
\textit{platforms}. That is, they are increasingly built not as a
single use, single function network, but as networked platforms that
can be used by many, possibly concurrent users. The platform is
relatively stable and provide common core services to all applications. 
However,  the applications those run on the platform
change fairly regularly due to software updates or because new applications
have been developed. 
%Note that the platform is expected to
%provide some common core services to all applications. 
Figure~\ref{fig:cloud_cps} shows a typical node of this
distributed platform on the left, along with a cloud of nodes that are
communicating via a network where at least one of the nodes has a
communication link to a control node. Nodes can join and leave the cloud 
during operation.

\begin{figure}[ht]
\centering
\includegraphics[width=0.86\columnwidth]{pfigs-cps}
%\includegraphics[width=0.45\textwidth]{layered_architecture2}
\caption{Typical node and cloud of nodes.}
\label{fig:cloud_cps}
\end{figure}

Third, these systems are used for \textit{distributed}
applications. Applications typically span multiple nodes, for reasons
related to the availability of resources: some nodes may have sensors,
some may have actuators, some may have the computing or storage
resources, some applications need more than the processing power
available on one node. Therefore, applications that use these resources
have to be architected such that they rely on loosely connected,
interacting components, running on different processors. Applications
can be organically assembled from components that provide specific
services, and components may be used (or re-used) by many active
applications. Obviously, the cluster of computing nodes runs
many applications concurrently.

Fourth, the platform is often a critical resource - possibly a
societal system, whose use must be carefully monitored and controlled
by a responsible owner. Therefore, these systems are \textit{managed} by some authority. Furthermore, as the
platform can be used and shared by many applications, possibly originating from
different organizations, the  platform and thus the system needs to be actively managed 
to avoid `tragedy of the commons'
type failures. Additionally, because of the embedded nature of the system,  deployment and control of applications need to ensure that the 
systems' (often scarce) resources are provisioned. 

% to be
%managed by some management entity -- to avoid `tragedy of the commons'
%type failures. The systems' (often scarce) resources need to be
%provisioned, and because of the embedded nature of the system, the
%deployment and control of applications needs to be done by a single
%entity. 

%%The platform is often a critical resource - possibly a
%societal system, whose use must be carefully monitored and controlled
%by a responsible owner.

Fifth, \textit{security} cannot be an afterthought. Information flows in
general and access to shared resources in particular should be
controlled under some overarching security policy. For instance, high
quality, sensitive customer data (from the electric grid) cannot be
made available to untrusted applications that are supplied by parties
needing access to derived data containing daily averages only -- and
those applications should not have any means to access that high-grade,
sensitive data. Furthermore, applications supplied by users cannot be
trusted, and the platform must protect itself from abuse by such
applications. If multiple applications run on the platform
concurrently, and there is a need for some degree of data sharing
among the applications, the platform must permit that while enforcing
the security policies defined for the system.

Sixth, \textit{resilience} is essential. Anything can go wrong at any
time: faults in the computing and communication hardware, in the
platform, in the application software. Moreover unanticipated changes
in the system (erroneous updates) or in the environment must be
survivable and the system should recover. The system here includes
both the platform, as well as the distributed applications.

Reading the above list one might argue that existing cloud computing
platforms based on virtualization technologies already provide a
solution for all these requirements. However, this is not the case for
the following reasons. Existing cloud computing platforms were not
designed with the requirements of real-time embedded systems, where
operating under resource constraints and timing requirements are
essential. The distributed applications here need to not only scale,
but to also  satisfy timing and security requirements. Interactions
with physical devices (sensor, actuators, special purpose hardware) is
rarely an issue in conventional cloud computing platforms --
everything is virtualized, without consideration for the management of
resources that are part of the system but not the computing
platform. It appears that current cloud computing platforms are not
prepared for mission critical real-time embedded systems, in general.

Arguably, the challenges listed above define a new a category of
systems that is emerging today. In this paper we present some initial
ideas and relevant research questions that will hopefully be addressed
by the research community. The next section discusses the issues of an
overall architecture for such systems. The section following discusses
the needs for a development toolsuite, which is followed by a section
on some initial results. A review of relevant related work is followed
by a summary and conclusions.



