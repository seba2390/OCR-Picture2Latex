% \vspace{-0.1in}
\section{Related work}
\label{sec:rel_work}

%This section compares our work presented in this paper with related work in the area of distributed real-time embedded systems that is relevant for our vision. 

There are several architecture description languages for embedded systems, including the Architecture Analysis and Design Language (AADL) \cite{AADL_Intro:06}, SysML ~\cite{SysML_v1.3:12} and the Modeling and Analysis of Realtime and Embedded (MARTE) ~\cite{MARTE_v1.1:11} systems profile for UML. These are general purpose modeling languages that can be used across a wide variety of systems. Because of specific features that are tightly integrated into our system, such as security labels and partition scheduling, we designed a dedicated domain-specific modeling language to describe DREMS systems and applications. However, automated transformation from our modeling language to these general purpose languages is possible and may be used to leverage some of their analysis capabilities.

A similar toolsuite that also uses a domain-specific approach for component-based systems is described in ~\cite{Metamodel_ECFMA:10}. That work focuses primarily on support for highly dynamic environments that require adaptation, and hence their environment supports dynamic updating and reconfiguration of models based on feedback from the running system. The biggest differences between that work and DREMS are that DREMS supports multiple messaging semantics and has built-in support for security in both the kernel and middleware layers.

Our previous work in modeling component-based systems includes the CoSMIC ~\cite{gokhale2008model, Schmidt:05e} tool suite, which assists with the model-based development, configuration and deployment of CORBA Component Model-based applications. While DREMS is more extensive than CoSMIC and provides the ability to model elements like hardware and task schedules, experience from the CoSMIC project helped guide certain design aspects of component modeling inside DREMS.

The ARINC-653 Component Model (ACM)~\cite{ACM_SPE:10}, which implements a component model for the ARINC-653 standard~\cite{ARINC-653} for avionics computing, forms the basis for the DREMS component model. DREMS extends the temporal partitioning scheduling method used by ACM by allowing multiple actors (processes) per temporal partition, a valuable feature for components that interact through synchronous messages. Further, the DREMS component model is designed to promote deadlock/race condition-free behavior in components.

The secure transport feature of DREMS is based on multi-level security (MLS) \cite{BellLaPadula}. All messages have a security label and must obey a set of mandatory access control (MAC) policies. The main novelty in DREMS with respect to MLS is the concept of multi-domain labels \cite{OlinSibertLabels} to support secure communication among actors from different organizations.

A more detailed description of the system requirements for DREMS and design principles used to meet those requirements is available in \cite{DREMS13Software}.