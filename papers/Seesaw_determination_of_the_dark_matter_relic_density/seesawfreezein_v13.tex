%%%%%%%%%%%%%%%%%%%%%%%%%%%%%%%%%%%%%%%%%
\documentclass[prd,a4paper,twocolumn,preprintnumbers,nofootinbib,superscriptaddress]{revtex4}
%\pdfoutput=1
\usepackage[english]{babel}
\usepackage{amsmath,amssymb,amsfonts, bm,bbm,slashed, subdepth}
\usepackage{graphicx}
\usepackage{enumerate}
\usepackage{setspace}
\usepackage{booktabs, tabularx}
\usepackage{units}
\usepackage{color}
%\usepackage{caption}
%\usepackage{subcaption}
\usepackage{multirow}
%\usepackage{ulem}
\usepackage[dvipsnames]{xcolor}
\usepackage[pscoord]{eso-pic}
\usepackage[normalem]{ulem}


\usepackage{scalerel}
\usepackage{soul}


\usepackage{hyperref}
\hypersetup{ 
	setpagesize=false,
	bookmarksnumbered=true,
	colorlinks=true,
	linkcolor=blue,
	citecolor=red,
	hypertexnames=true
}


\newcommand{\be}{\begin{equation}}
	\newcommand{\ee}{\end{equation}}
\newcommand{\ba}{\begin{array}}
	\newcommand{\ea}{\end{array}}
\newcommand{\bea}{\begin{eqnarray}}
	\newcommand{\eea}{\end{eqnarray}}
\newcommand{\balg}{\begin{align}}
	\newcommand{\ealg}{\end{align}}
\newcommand{\bit}{\begin{itemize}}
	\newcommand{\eit}{\end{itemize}}
\newcommand{\trm}[1]{\textrm{#1}}
\newcommand{\mbf}[1]{\mathbf{#1}}
\newcommand{\tbf}[1]{\textbf{#1}}
\newcommand{\mcl}[1]{\mathcal{#1}}
\newcommand{\mbb}[1]{\mathbb{#1}}
\newcommand{\msc}[1]{\mathscr{#1}}
\newcommand{\lrpar}{\stackrel{\leftrightarrow}{\partial}}
\newcommand{\lpar}{\stackrel{\leftarrow}{\partial}}
\newcommand{\rpar}{\stackrel{\rightarrow}{\partial}}
\newcommand{\rw}{\rightarrow}

\newcommand{\nn}{\nonumber}
\newcommand{\nt}{\noindent}
\newcommand{\ra}{\rightarrow}
\newcommand{\sv}{\langle \sigma v\rangle}
\newcommand{\sva}{\langle \sigma_\text{anni} v\rangle}
\newcommand{\ssi}{\sigma_\text{SI}}

% New commands to add, comment and remove text
\newcommand{\thomas}[1]{\textcolor{red}{#1}}
\newcommand{\thomasc}[1]{\textcolor{red}{\textbf{[TH: #1]}}}
\newcommand{\thomasr}[2]{\textcolor{red}{\sout{#1} {#2}}}

\newcommand{\rupert}[1]{\textcolor{blue}{#1}}
\newcommand{\rupertc}[1]{\textcolor{blue}{\textbf{[RC: #1]}}}
\newcommand{\rupertr}[2]{\textcolor{blue}{\sout{#1} {#2}}}

\definecolor{bostonuniversityred}{rgb}{0.8, 0.0, 0.0}
\newcommand{\aritra}[1]{\textcolor{bostonuniversityred}{#1}}
\newcommand{\aritrac}[1]{\textcolor{bostonuniversityred}{\textbf{[RC: #1]}}}
\newcommand{\aritrar}[2]{\textcolor{bostonuniversityred}{\sout{#1} {#2}}}

\newcommand*\diff{\mathop{}\!\mathrm{d}}

\makeindex

\begin{document}
	\preprint{ULB-TH/21-04}
	
	% =============================================================================
	\title{
		%Dark matter from Higgs boson decay through seesaw freeze-in
		Seesaw determination of the dark matter relic density
		% \\((dark matter from seesaw induced $H$, $W$ and $Z$ boson decay.))
	}
	% =============================================================================
	
	\author{Rupert Coy}
	%\email{...@ulb.ac.be}
	\affiliation{Service de Physique Th\'eorique, Universit\'e Libre de Bruxelles, Boulevard du Triomphe, CP225, 1050 Brussels, Belgium}
	\author{Aritra Gupta}
	%\email{....@ulb.ac.be}
	\affiliation{Service de Physique Th\'eorique, Universit\'e Libre de Bruxelles, Boulevard du Triomphe, CP225, 1050 Brussels, Belgium}
	\author{Thomas Hambye}
	%\email{thambye@ulb.ac.be}
	\affiliation{Service de Physique Th\'eorique, Universit\'e Libre de Bruxelles, Boulevard du Triomphe, CP225, 1050 Brussels, Belgium}
	
	
\begin{abstract}
	In this article we show that in the usual type-I seesaw framework, augmented solely by a neutrino portal interaction, the dark matter (DM) relic density can be created through freeze-in in a manner fully determined by the seesaw interactions  and the DM particle mass. This simple freeze-in scenario, where dark matter is not a seesaw state, proceeds through slow, seesaw-induced decays of Higgs, $W$ and $Z$ bosons. We identify two scenarios, one of which predicts the existence of an observable neutrino line. 
		%In the seesaw mechanism one expects that the Yukawa couplings of the right-handed neutrinos are small if the seesaw scale is low. 
		%If at least one right-handed neutrino is lighter than the Higgs boson, this implies a very slow decay of the Higgs boson into a right-handed 
		%neutrino and a SM lepton. Provided the SM neutrino mass spectrum is hierarchical, this allows a slow freezein creation 
		%of right-handed neutrinos which can subsequently entirely decay into DM particles with the observed relic density. This only requires that, 
		%on top of the usual seesaw model, DM couples to the right-handed neutrino via a neutrino portal interaction. In this way the abundance of DM is entirely determined by its mass and the seesaw parameters, in particular the mass and Yukawa couplings of the lightest right-handed neutrino, related to the lightest neutrino mass value. This simple mechanism is testable from the observation of a neutrino line emitted from
		%the slow DM decay that this scenario induces. 
	\end{abstract}
	
	\maketitle
	%==============================================================================
	
The nature of DM as a particle and the origin of neutrino masses constitute two of the main conundrums of particle physics today.
Whether these two enigmas could be closely related is a fascinating question.
In the type-I seesaw scenario, the right-handed Majorana neutrinos and their Yukawa interactions 
%coupling them to SM lepton and scalar doublets, 
allow for a particularly simple and motivated explanation of neutrino masses \cite{Minkowski:1977sc,Yanagida:1979as,GellMann:1980vs,Mohapatra:1979ia}.
In this article, we are interested in the possibility that these interactions could play an important role in the existence of DM today.
% is the question one will consider here.
If the right-handed neutrinos lie around the electroweak scale or below, these seesaw interactions are expected to be small, so that they induce sufficiently suppressed neutrino masses. Thus, in this case, if these interactions are to play an important role in the production of DM, one would expect that it is  through out-of-equilibrium freeze-in production \cite{McDonald:2001vt,Hall:2009bx} rather than freeze-out production (i.e.~exit from thermal equilibrium). This possibility, where DM is slowly produced out of equilibrium in the early Universe thermal bath through processes where the small seesaw Yukawa couplings are involved, has been considered in several recent works \cite{Aoki:2015nza,Becker:2018rve,Bandyopadhyay:2020qpn,Ma:2021bzl,Chianese:2021toe}.
Putting aside the possibility of sterile neutrino DM (see e.g.~\cite{Dodelson:1993je,Shi:1998km,Boyarsky:2018tvu,Lucente:2021har} for studies of the constraints on this scenario), this requires an interaction between the right-handed neutrino(s) and DM. The most minimal possibility is to assume a neutrino portal Yukawa interaction
where the right-handed neutrinos couple to new scalar and fermion particles, one or both constituting the DM. 
In this article, we point out that based on this simple seesaw/neutrino portal structure, the DM relic density could be produced from Higgs, $W$ and $Z$ boson decays via freeze-in in a manner that depends only on the seesaw parameters and the mass of the DM particle(s). 
	
	
	
	
\section{General setup}
	
We begin by displaying the Lagrangian assumed. On top of the usual type-I seesaw interactions, 
\begin{eqnarray}
	\hspace{-0.2cm}
	{\cal L}_\text{seesaw}&=&i \overline{N_R}\partial \hspace{-1.5mm} \slash N_R-\frac{1}{2} m_N (\overline{N_R} N_R^c+\overline{N_R^c} N_R)\nonumber\\
	&& -(Y_\nu \overline{N_R} \tilde{H}^\dagger L+h.c.)\, ,
	\label{seesaw}
	\end{eqnarray}
one assumes only a neutrino portal interaction, 
\begin{equation}
	\delta{\cal L} =  -Y_\chi \overline{N} \phi \chi  +h.c.\,.
	\label{LagrchiN}
\end{equation}
A sum over the various right-handed neutrinos is implicit.
%where $\chi$ and $\phi$ are new particles which can be each or both DM particles.
We assume that $\chi$ is a two-component Majorana spinor. The generalisation to a four-component Dirac spinor is straightforward. We do not assume any symmetry at this stage, i.e.~$\chi$ and $\phi$ are singlets of all existing symmetries. 
The possibility that they are charged under a discrete, global or local symmetry will be discussed later.
	
As is well known, in the seesaw mechanism the neutrino masses follow from the diagonalisation of the induced neutrino mass matrix, ${\cal M}_\nu= - Y_\nu^T m_N^{-1}Y_\nu v^2/2$, where $v=246$ GeV is the vev of the SM scalar boson.
Given the value of the atmospheric and solar neutrino mass splittings, this implies that two right-handed neutrinos, which we will call $N_{2,3}$, necessarily have Yukawa couplings much larger than the typical $10^{-10}$-$10^{-13}$ values one needs to produce the observed relic density through freeze-in. 
However, one of the three right-handed neutrinos, which we will call $N_1$, or simply ``$N$'', could nevertheless have smaller couplings as the absolute neutrino mass scale is not known, i.e. the value of the lightest neutrino mass could be very tiny or even vanishing. 
If this right-handed neutrino is lighter than the Higgs boson, the Higgs boson can decay to $N+\nu_i$ (i$=$e,$\mu$,$\tau$) with a decay width\footnote{We will assume the two other right-handed neutrinos are heavier, $m_{N_{2,3}}> m_h$, and have negligible neutrino portal interactions. These could, for instance, be responsible for successful baryogenesis through leptogenesis (and without much washout of the $L$ asymmetry produced by $N_1$ interactions, given the smallness of the $N_1$ interactions and possible flavour effects). Alternatively, they could also never have been produced if the inflation reheating temperature is smaller than their masses.}
\begin{equation}
	\Gamma_{h\rightarrow \overline{N}\,\nu_i+N\bar{\nu}_i}=\frac{1}{16\pi} m_h |Y_{\nu i}|^2 \Big(  1-\frac{m_{N}^2}{m_h^2}\Big)^2 \,. 
\end{equation}
Similarly, in the electroweak broken phase, the decays of the $W^\pm$ to $N+l^\pm$ and $Z$ to $N \nu$ occur through $N$-$\nu$ mixing, if kinematically allowed, 
%\begin{equation}
%\Gamma_{W^\pm \rightarrow N\,l_i^\pm} = \frac{1}{48\pi} m_W |Y_{\nu i}|^2\, f(m_{N}^2/m_W^2) \, ,  
%\end{equation}
%\begin{equation}
%\Gamma_{Z \rightarrow \overline{N}\,\nu_i+N \bar{\nu}_i}= \frac{1}{48\pi} m_Z |Y_{\nu i}|^2\, f(m_{N}^2/m_Z^2) \,, 
%\end{equation}
\begin{eqnarray}
	\Gamma_{W^\pm \rightarrow N\,l_i^\pm} &=& \frac{1}{48\pi} m_W |Y_{\nu i}|^2\, f(m_{N}^2/m_W^2) \, ,  \\
	\Gamma_{Z \rightarrow \overline{N}\,\nu_i+N \bar{\nu}_i}&=& \frac{1}{48\pi} m_Z |Y_{\nu i}|^2\, f(m_{N}^2/m_Z^2) \,, 
\end{eqnarray}
where $f(x) = (1-x)^2(1+2/x)$. 
Note the $m^2_{W,Z}/m_N^2$ enhancement (due to $N$-$\nu_i$ mixing) of the gauge boson decay widths with respect to the $h$ one.
Given the neutrino mass constraints, it is perfectly possible that these decays have never been in thermal equilibrium. 
%in the early Universe thermal bath. 
For the $Z$ decay, which is the fastest, this requires 
$\Gamma_{Z\rightarrow \overline{N}\,\nu+N\bar{\nu}_i}/H|_{T\simeq m_Z}\lesssim 1$, i.e.
\begin{equation}
	\sum_i|Y_{\nu i}|^2 \lesssim 1 \cdot 10^{-16}\cdot \Big(\frac{m_N}{10\,\hbox{GeV}}\Big)^2\,,
	%5 \cdot 10^{-16}\cdot \Big(\frac{m_N}{10\,\hbox{GeV}}\Big)^2\,,
	\label{Ynunonthermal}
\end{equation}
or
\begin{equation}
	m_{\nu_1}\leq \tilde{m}_1< 3\cdot 10^{-4}\,\hbox{eV} \, 
	(m_N/10\,\hbox{GeV}) \, ,
	%1.5\cdot 10^{-4}\,\hbox{eV} \,(m_N/10\,\hbox{GeV})^2
\end{equation}
where we have neglected $m^2_{N}/m_Z^2$ corrections and have used the well-known seesaw inequality 
$m_{\nu_1}  \leq  \tilde{m}_1\equiv \sum_i|Y_{\nu i}|^2v^2/(2 m_{N})$, with $m_{\nu_1}$  the mass of the lightest SM neutrino. 
%Thus Eq.~(\ref{Ynunonthermal}) means $m_{\nu_1}\leq 1.5\cdot 10^{-4}\,\hbox{eV} \,(m_N/10\,\hbox{GeV})^2$.
%This value of the Yukawa couplings have to be compared with the relevant neutrino mass seesaw constraint, which requires
%$\tilde{m}_1\equiv \sum_i|Y_{\nu i}|^2v^2/(2 m_{N})>m_{\nu_1}$, or equivalently $\sum_i|Y_{\nu i}|^2\gtrsim ??? m_{N}/100 \hbox{GeV}$ where  
As is also well known, the value of $\tilde{m}_1$ is experimentally allowed to be anywhere between $m_{\nu_1}$ and values much larger than the neutrino masses but, barring cancellations between the Yukawa couplings in the neutrino mass formula, it is expected below the upper bound on neutrino masses $\sim 1.1$~eV \cite{Schluter:2020gdr}, typically of order $m_{\nu_1}$. 
%\tilde{m}_1\equiv \frac{\sum_i|Y_{1i}|^2 v^2}{2m_{N}}\lesssim 10^{-3}~eV\cdot {m_{N}}{m_h}
%\end{equation}
	
If $N$ has never been in thermal equilibrium (and not created at the end of cosmic inflation), then it can only be created through freeze-in. For $m_{N}\lesssim m_{h,W,Z}$, the dominant freeze-in production mechanism is from the above decay channel(s).\footnote{Scattering processes involving two powers of $Y_\nu$ in the amplitude clearly have a very suppressed contribution. Scattering processes from SM fermions involving only one power, e.g. $f\bar{f} \rightarrow N L$,  bring a smaller contribution ($\sim 20\%$) than the decays and for clarity we will limit ourselves to the contribution of the decays. Three-body SM fermion decays, $f\rightarrow {f}' \bar{L} N$, also give a subleading contribution.}
The resulting number of right-handed neutrinos from $Z$ boson decays is given by
\begin{equation}
	Y_{N}\equiv\frac{n_{N}}{s}= c_Z \cdot \frac{\gamma_{Z\rightarrow \overline{N} \nu+N\bar{\nu}}}{sH} \Big|_{T=m_Z} \, .
	\label{N1yield}
\end{equation}
Here we have simply multiplied the 
number of decays per unit time and volume, $\gamma_{Z\rightarrow \overline{N} \nu+N\bar{\nu}}$, by the age of the universe $\sim 1/H$, everything taken at about the freeze-in production peak temperature, $T\sim m_Z$, with
\begin{align}
	\gamma_{Z\rightarrow \overline{N} \nu+N\bar{\nu}}&=\langle n_{Z}^{eq}\Gamma_{Z\rightarrow \overline{N} \nu+N\bar{\nu}}\frac{E}{m_{Z}}\rangle \notag \\
	&=  \frac{m_Z^3 T}{32\pi^3} K_1(m_Z/T) f(m_N^2/m_Z^2) \sum_i |Y_{\nu i}|^2 \, , 
\end{align}
which depends on the Bessel function $K_1$, and where the brackets refer to the thermal average. 
%Multiplied by the age of the universe $\sim 1/H$ this gives the yield up to a multiplicative constant
This is valid up to a constant, $c_Z$, of order unity, cf. Eq.~\eqref{N1yield}. 
Integrating the corresponding Boltzmann equation gives $c_Z=3\pi/[2K_1(1)] = 7.8$.
The same formula holds for $W\rightarrow N l_i$ and $h\rightarrow N \nu_i$, and we find $c_W=c_h=c_Z$.
	
Once produced this way, the right-handed neutrinos can decay dominantly to $\chi+\phi$ through neutrino portal interactions, provided $m_{N} > m_\chi+m_\phi$ (see below for possible effects of three-body decays). This simply gives
\begin{equation}
	Y_\chi=Y_\phi=Y_{N} \, .
\end{equation}
Summing the contributions from $Z$, $W$ and $h$ decays, one finally obtains
\begin{equation}
	\Omega_{DM} h^2 \simeq
	%Y_{N}s_0\frac{m_\chi+m_\psi}{\rho_c}=
	10^{23} \, \sum_i |Y_{\nu i}|^2\, \Big(\frac{m_\chi+m_\phi}{1\,\hbox{GeV}}\Big)\,\Big(\frac{10\,\hbox{GeV}}{m_N} \Big)^2\,. 
	\label{Omegafreezein}
\end{equation}
%where $\rho_c$ is the critical density.
Note that we have assumed here that both $\chi$ and $\phi$ are stable, so that both are DM components. If one of these particles is unstable, one has to drop the corresponding mass from this equation. Thus, one gets the observed value $\Omega_{DM}h^2=0.12$ if
\begin{equation}
	\sum_i |Y_{\nu i}|^2\simeq 10^{-24} \cdot \Big( \frac{m_N}{10\,\hbox{GeV}}\Big)^2\Big(\frac{1\,\hbox{GeV}}{m_\chi+m_\phi}\Big) \, , \label{Ynufreezein}
\end{equation}
which implies
\begin{equation} 
	m_{\nu_1}< \tilde{m}_1= 4 \cdot 10^{-12}\,\hbox{eV} \cdot \frac{10 \hbox{ GeV}}{m_{N}}\cdot \left( \frac{1\,\hbox{GeV}}{m_\chi+m_\phi} \right) \,. 
	%m_{\nu_1}< \tilde{m}_1= 3\cdot 10^{-13}\,\hbox{eV} \cdot \frac{10 \hbox{ GeV}}{m_{N}}\cdot \frac{1\hbox{GeV}}{m_\chi+m_\psi} 
	\label{numass}
\end{equation}
%Given the bound $\tilde{m}_1\leq m_{\nu_1}$,  this implies 
% \begin{equation} 
%m_{\nu_1}< 10^{-??}\,\hbox{eV} \cdot \frac{50 \hbox{ GeV}}{m_{N}}\cdot \frac{10\hbox{ GeV}}{m_\chi+m_\psi}\,,
% \end{equation}
Again, $m_{\nu_1}\simeq \tilde{m}_1$ holds approximately, unless there are cancellations between various Yukawa couplings in the neutrino mass formula.
Thus, the DM relic density is determined only by its mass and the seesaw parameters. In particular, one finds an interesting relation between the value of the lightest neutrino mass and the DM relic density. 
Such a tiny neutrino mass value would be very difficult to probe, but is falsifiable from absolute neutrino mass scale and neutrinoless double beta decay experiments such as KATRIN and GERDA \cite{Ackermann:2012xja}, as well as from cosmology.
%As another numerical example for $m_{N}=10$~GeV, $m_\chi+m_\phi=1$~GeV, one gets ....
	
	
\section{DM stability}
This very simple mechanism above has to be confronted with several constraints, which we will now discuss. 
Firstly, one has to check that the interactions assumed above 
do not lead to too fast a DM decay. There are different possibilities depending on the existence of various possible symmetries. Before discussing these, we will first limit ourselves to the decay which must occur simply as a result of the interaction assumed above, irrespective of the symmetry assumed.
	
\begin{figure}[t]
	\centering
	\includegraphics[width=0.85\columnwidth]{plot1.png}
	\vspace{-0.3cm}
	\caption{Constraints on neutrino line production~\cite{Collaboration:2011jza,Zhang:2013tua,Malek:2002ns,Bellini:2010gn,FrankiewiczonbehalfoftheSuper-KamiokandeCollaboration:2016pkv}. For various values of $m_N$, experimental lower bounds (see \cite{Garcia-Cely:2017oco,Coy:2020wxp}) are contrasted with the theoretical upper bound on the lifetime of $\chi\rightarrow \phi \nu$ decay, assuming that $m_\chi \gg m_\phi$, so that the neutrino line occurs at an energy equal to $m_\chi/2$. 
	%The latter assumption is appropriate because we intend to search for a neutrino line at a specific energy}. 
	The black lines denote the value of the lifetime below which the two-body decay channel is dominant, so that the seesaw/DM relic density correspondence holds. The red lines give the upper bound from structure formation constraints, Eq. \eqref{structureformation}.}
		\label{plot1}
\end{figure}
	
	
If $m_\chi>m_\phi$, the $\chi$ particle can decay in several ways. At energies of order or below the electroweak scale, as is the case here, the dominant decay is into $\nu+\phi$ via the $Y_\chi$ interaction and $\nu-N$ seesaw-induced mixing,\footnote{The $m_\phi<m_\chi$ case is very similar to the $m_\chi<m_\phi$ case, just inverting $\chi$ and $\phi$.  The relevant decay width is $\Gamma_{\phi\rightarrow \chi\nu}=\frac{1}{16 \pi} m_\phi \sum_i |Y_\chi|^2 |Y_{\nu i}|^2 v^2/m^2_{N}$,
%\frac{1}{8 \pi} m_\phi \sum_i |Y_\chi|^2 |Y_{\nu i}|^2 v^2/m^2_{N}$,
leading to the same possibility of indirect detection
% the same visible final states 
and about the same constraint on the $Y_\chi$ coupling.}
\begin{equation}
	\Gamma_{\chi\rightarrow \phi\nu}=\frac{1}{32\pi} |Y_\chi|^2 \frac{ \sum_i|Y_{\nu i}|^2 v^2}{m^2_{N}}m_{\chi}\Big(1-\frac{m_\phi^2}{m_\chi^2}\Big)^2 \, .
		\label{GammaChi}
\end{equation}
%via the decay chain, $\chi\rightarrow \phi+N^*\rightarrow \phi+\nu+H^*\rightarrow \phi+\nu+f+\bar{f}$, where the asterix refers to a virtual state, and $f$ to a SM fermion (for instance dominantly a $b$ quark if $m_\chi>m_\chi+2m_b$).
%\begin{equation}
%\Gamma_{\chi\rightarrow \phi\nu b \bar{b}=\fgrac{1}{5912???\pi^5}\, \sum_i |Y_{1i}|^2 |Y_\chi|^2 |Y_b|^2 \frac{m_\chi^{8?}}{m_h^4m_{N}^4 
%\end{equation}
Here there are two options, depending on whether the %corresponding 
$\chi$ lifetime is larger or smaller than the age of the universe.
We will mainly consider the former option (A), and also briefly discuss the latter option (B).
Option A has the nice feature of producing a monochromatic flux of neutrinos, i.e.~a DM smoking gun neutrino line. 
Note that since we consider $m_\chi$ up to the EW scale, this neutrino line can be much more energetic than, for instance, the case of keV sterile neutrino DM \cite{Dodelson:1993je,Shi:1998km,Boyarsky:2018tvu,Lucente:2021har}. 
The current lower bound on the lifetime of DM decays producing a neutrino line is given in Fig.~\ref{plot1}. For $m_{DM} \sim$ GeV, it is of order $10^{24}$~sec. Such a long lifetime requires, on top of the large $|Y_\nu|^2$ suppression of the decay width (see Eq.~\eqref{Ynufreezein}), an approximately equal suppression from $|Y_\chi|^2$,
%Decays to charged leptons are subdominant. 
%Given the fact that the decay width is largely suppressed by the small value of $\sum_i |Y_{\nu i}|^2$ which is required by the freeze-in, Eq.~(\ref{Ynufreezein}), it is not difficult to observe that the lifetime can be obtained larger than the lower bound from indirect detection of decaying DM producing a neutrino line , which for instance is of order $10^{-24}$~sec for a DM mass in the GeV to 100 GeV range \cite{}. Inserting Eq.~(\ref{Ynufreezein}) in Eq.~(\ref{GammaChi}) this requires a small value of the $Y_\chi$ couplings
\begin{equation}
	|Y_\chi|^2\lesssim 10^{-25}\,\Big(\frac{10^{24}\,\hbox{sec}}{\tau_{\chi}^{\rm{obsv}}}\Big) \, .
		\label{Ychiupperbound}
\end{equation} 
%\rupert{to $\mathcal{O}(m_\phi/m_\chi)$.}
Approximately saturating this bound leads to an observable neutrino line.
This small value of $Y_\chi$ implies that the decay of $N$ is quite slow,
\begin{eqnarray}
	\label{GammaN2body}
	\Gamma_{N\rightarrow  \chi\phi}&\simeq &\frac{1}{16\pi}m_{N} |Y_\chi|^2\left(1+\frac{2\,m_\chi}{m_N}\right)\\
	&&\lesssim 3\cdot 10^{-2}\,\hbox{sec}^{-1} \cdot \Big(\frac{10^{24}\,\hbox{sec}}{\tau_{\chi}^{{\rm obsv}}}\Big)\Big(\frac{m_N}{10\,\hbox{GeV}}\Big)\nonumber \, ,
\end{eqnarray}
where in the inequality we neglected corrections of order $(m_{\chi,\phi}/m_N)$.
Note, importantly, that even if suppressed in this way, the $N\rightarrow \chi \phi$ decay width driven by $Y_\chi$ can still easily dominate over the various three-body decays that are induced by the $Y_\nu$ couplings, $N\rightarrow \nu f \bar{f}$ and $N \to \ell f \bar{f'}$. Thus, there is indeed a whole range of parameter space for which the one-to-one seesaw/DM relic density relation, Eq. \eqref{Omegafreezein}, holds. To see this, one has to compare the two-body decay width with the three-body one, which for the neutrino channel is
\begin{equation}
	\Gamma_{N\rightarrow\nu f \bar{f}}=\frac{N_c}{1536\,\pi^3}\,|Y_{\nu i}|^2\frac{g_2^2}{\cos \theta_W^2} (g_L^2+g_R^2)\frac{m_{N}^3}{m_Z^2} \, ,
\end{equation}
and similarly for $N \to \ell f \bar{f'}$.
As usual, $g_{L,R}=T_3-Q\,\sin^2 \theta_W$, with $g_2$ and $\theta_W$ the $SU(2)_L$ gauge coupling and Weinberg mixing angle, and $T_3$, $Q$, $N_c$  the weak isospin, electric charge, 
%(\aritra{in units of $e\simeq0.3$}
and number of colours of the SM fermion considered. 
%\aritra{Also, $s_W \equiv \sin\,\theta_W$ and $c_W \equiv \cos\,\theta_W$, $\theta_W$ being the Weinberg angle.} 
Summing the three-body decays over all quarks and leptons allowed in the final state, 
%the three-body to two-body decay width ratio is $(a_Z/96\,\pi^2)(\sum_i|Y_{\nu i}|^2/|Y_\chi|^2) (m_{N}^2/m_Z^2)$, {\bf with $a_Z \equiv (e/(s_W\,c_W))^2\,\sum_f N_c \, (g_L^2+g_R^2)$.}
%%\rupert{where for} $m_N > 2m_b$, we have $a_Z \equiv (e/(s_W\,c_W))^2\,\sum_f N_c \, (g_L^2+g_R^2)=2.06$.
%Hence, 
the two-body decay width dominates when $|Y_\chi|^2\gtrsim 10^{-4} \,\sum_i|Y_{\nu i}|^2 \,(m_{N}/10\,\hbox{GeV})^2 $. This lower bound must be compared with the upper bound on $Y_\chi$ such that it doesn't induce too intense a neutrino line, Eq.~\eqref{Ychiupperbound}. 
%, which gives $Y_\chi^2< 3\cdot 10^{-26}(10^{24}\,\hbox{sec}/\tau_\chi)(m_N/10\,\hbox{GeV})^2(1\,\hbox{GeV}/m_\chi)$. 
%\sout{Both conditions leave a \sout{all} range where the one-to-one relation holds. }
	
	
	
The black lines in Fig.~\ref{plot1} display for various values of $m_N$ the value of $\Gamma_{\rm min,\chi\rightarrow \phi \nu}^{-1} \propto m_\chi/m_N^4$ below which the two-body decay dominates and hence for which there is the one-to-one correspondence.
%This is shown in Fig.~\ref{plot1} where the dominance of the two-body decay condition is translated into an upper bound on the $\chi$ lifetime (\sout{i.e.~}more precisely, on $\Gamma_{\chi\rightarrow \phi \nu}^{-1}$). Everything between these lines and the experimental bounds leads to the one-to-one correspondence.
To a large extent, the region where the correspondence holds 
%For $m_{\chi}$ above a few GeV, this scenario 
predicts a neutrino line that we may hope to detect soon (except for $m_\chi$ well below GeV, where the experimental lower bound on the lifetime is less stringent).
%, the range becomes larger as the experimental lower bound on the lifetime is less stringent. 
Note, importantly, that if $m_{N}> m_{W,Z,h}$, one can show that unless $m_\chi$ is below $\sim \mathcal{O}(10)$\,MeV, the one-to-one correspondence between the seesaw parameters and the DM relic density is lost because in this case $N$ decays much faster through two-body decays into a SM lepton and a SM boson. In this case, freeze-in works through scattering processes \cite{Becker:2018rve,Bandyopadhyay:2020qpn,Chianese:2021toe,Cosme:2020mck,Chianese:2018dsz}.
%, proportionally to the seesaw couplings and the branching ratio but the correspondence is lost.  
%For smaller values of $m_{DM}$, an one-to-one relation nevertheless holds even if $m_N>m_{W,Z,h}$, in the case \aritra{where \st{that the}} $N$\aritra{can also be produced from \st{production is}} from scatterings (a possibility we will not further consider here).
%The range above also translates into a upper bound on the lifetime  of $\chi$ (if $m_\chi>m_\phi$) or of $\phi$ (if $m_\chi<m_\phi$). These bounds are also shown in Fig.~?. They basically mean that the seesaw determined DM relic density scenario we consider predicts an observable neutrino line for DM mass around GeV. 
%This can be compared on the same plot with 
%the current bound on the lifetime (following from the compilation of various experimental results in Coy2020). The corresponding lifetime of $N$ one obtains is given in Fig.~?. 
%((\aritra{skip??[I agree]}: Fig.~\ref{plot1} also shows the lifetime of $\chi$ one gets as a function of the neutrino energy (assuming $m_\phi<<m_\chi$ so that $E_\nu=m_\chi/2$), for various values of $m_{N}$ and $Y_\chi$, assuming that the scenario reproduce the observed DM relic density. ))
	
	
	
	
	
	
The lower bound on the lifetime of $N$, Eq.~(\ref{GammaN2body}), may be a few orders of magnitude larger than the age of the universe at the BBN epoch, $\tau_{BBN}\sim 1-100$~sec. %\footnote{The BBN time is in fact half way between characteristic electroweak scale  particle lifetimes induced by coupling of order unity and the typical lower bound on the DM lifetime required by DM indirect detection, see also \cite{} for a scenario which takes advantage of this coincidence.} 
One could therefore wonder if BBN is a matter of concern. However, it is not the case because the number of $N$ particles decaying is very limited, and they negligibly contribute to the total energy density at this time (hence to the Hubble expansion rate),  even if $N$ decays into two particles which are relativistic.
%   ((((check better)))) \rupert{[I agree. I find $\rho_N(T = \text{MeV}) \sim 10^{-6} [m_N/(m_\chi + m_\phi)] \text{MeV}^4 \ll \rho_\nu \sim \text{MeV}^4$]}.
Moreover, the decay is into $\chi$ and $\phi$, which do not cause any photo-disintegration of nuclei since they do not produce any electromagnetic or hadronic material, (unless the $\phi$ scalar has a vev and decays through a Higgs portal).
The late $N$ decay, producing relativistic DM, is nevertheless a matter of concern for structure formation. 
Imposing that DM, which has kinetic energy $\sim m_N/2$ when produced from $N$ decays, Eq.~\eqref{GammaN2body}, redshifts enough so that it is non-relativistic when $T\sim $~keV (so that it doesn't affect structure formation too much \cite{Irsic:2017ixq}) gives an upper bound on the $\chi$ lifetime,
\begin{equation}
	\tau_\chi \lesssim 10^{28}\,\hbox{sec}\,\Big(\frac{m_{DM}}{m_N}\Big)^2\Big(\frac{m_N}{10\,\hbox{GeV}}\Big) \, .
	\label{structureformation}
\end{equation}
We show the corresponding constraint in red in  Fig.~\ref{plot1}. 
%This excludes the case where $m_{DM}\ll m_N$. 
For most of the parameter space (i.e.~wherever the red lines are below the black lines in Fig.~\ref{plot1}) this constraint implies that the one-to-one correspondence holds.
% implies that only the region where there is a one-to-one correspondence is allowed \rupert{[don't understand this sentence]}\aritra{[me too]}. 
	
	
	
%Next we analyse what are the conditions that must hold to have the one-to-one relationship between the values of the seesaw parameters
%and the DM relic density (modulo the value of the mass of the DM particles). The main condition is that $N$ must dominantly decay to DM through the neutrino portal. This requires that the decays induced by the seesaw Yukawa couplings are subdominant. These are the $N\rightarrow \nu f \bar{f}$ 3-body decays. 
%\begin{equation}
%\Gamma_{N\rightarrow\nu f \bar{f}}=\frac{1}{3072\pi^3}\,|Y_{\nu i}|^2 (g_L^2+g_R^2)\frac{m_{N}^3}{m_Z^2}\,N_c
%\end{equation}
%where as usual $g_{R,L}=T_3-Q\sin^2 \theta_W$ with $T_3$, $Q$ and $N_c$ the weak isospin, electric charge and number of colors of the fermion considered.
%This has to be compared with the 2 body decay, Eq.~(\ref{GammaN2body}).
%\begin{equation}
%\Gamma_{N\rightarrow \chi \phi}=\frac{1}{16\pi}|Y_\chi|^2 m_{N}
%\end{equation}
%Summing the 3-body decays over all quarks and leptons, (which for instance for the case $m_N>2m_b$ gives $c_Z\equiv \sum_f N_c (g_L^2+g_R^2)=4.79$) the 3-body to 2 body decay width ratio is $(c_Z/192\pi^2)(\sum_i|Y_{\nu i}|^2/|Y_\chi|^2) (m_{N}^2/m_Z^2)$ (so $10^{-5}$ for  $m_{N}=10$~GeV for example).
%Given the fact that $\sum_i|Y_{\nu i}|^2$ is of order $10^{-24}$, the 2-body decay width dominates if  $Y^2_\chi\gtrsim 6\cdot 10^{-30} (m_{N}/10\,\hbox{GeV})^2 $. This has to be compared with the upper bound on $Y_\chi$ not to induce a too intense neutrino line, Eq.~(\ref{Ychiupperbound}), which gives $Y_\chi^2< 3\cdot 10^{-26}(10^{24}\,\hbox{sec}/\tau_\chi)(m_N/10\,\hbox{GeV})^2(1\,\hbox{GeV}/m_\chi)$. Both conditions leaves a range of possible value for any value of $m_\chi<m_N<m_W$. Thus for $m_{\chi}$ above a few GeV this scenario predicts a neutrino line around the corner. For smaller $m_\chi$ quickly the range gets larger as the experimental lower bound on the lifetime is less getting less stringent, so the smaller the DM mass the larger the range.
	
	
Another related constraint that must be fulfilled in order that the one-to-one relationship above holds is that the $\phi$ particle, if still stable today and without a vev, is negligibly produced by a possible Higgs portal $H^\dagger H \phi^2$ interaction. Note that if it has a sizable vev, even a very tiny Higgs portal interaction would largely destabilise it. 
%Beside this interaction which is not forbidden by any symmetry, other interactions may also destabilize or break the one-to-one correspondance but their presence depends on the symmetry assumed.
	
	
At this point, let us revisit the bound on the lightest neutrino mass, Eq. \eqref{numass}. 
If we insist on the presence of a neutrino line, then $m_N > m_\chi + m_\phi > 3.6$ MeV, since Borexino detects $\bar{\nu}_e$ via inverse beta decay, for which the kinematic threshold is $E_{\bar{\nu}} > 1.8$ MeV (see e.g. \cite{Garcia-Cely:2017oco}). 
Combining this with the bound on the DM lifetime from Eq. \eqref{structureformation} and enforcing $\tau_\chi > \tau_U$ gives $m_{\nu_1} \lesssim 3 \times 10^{-6}$ eV, still much lighter than the neutrino mass scale being probed by present experiments.  
Moreover, since Fig. \ref{plot1} typically bounds $\tau_\chi$ to be several orders of magnitude above $\tau_U$, the upper bound on $m_{\nu_1}$ would be correspondingly strengthened. 
Thus, a neutrino line in this model would imply an extremely light neutrino. 
If, on the other hand, we do not insist on the possibility of a neutrino line signal, then a very small neutrino mass could be avoided by taking sufficiently small $m_\chi$, $m_\phi$ and $m_N$. 
	
	
As already mentioned above, an alternative ``option B" is to consider that the heaviest  particle among $\chi$ and $\phi$ has a lifetime shorter than the age of the universe, i.e.~to consider much larger values of $Y_\chi$. In this case, DM is made of only the lightest species and no neutrino line can be observed. %Note however that the one-to-one connection between the seesaw parameter and the DM relic density doesn't necessarily allow arbitrarily large values of $Y_\chi$. 
A large $Y_\chi$ coupling can change the scenario greatly because it can lead to thermalisation of $N$, $\chi$ and $\phi$. Then the thermalised hidden sector (HS) is characterized by a temperature, $T'$, smaller than the visible SM sector temperature, $T$. Thus one could believe that the one-to-one connection between DM and neutrino mass is lost. This would be the case if later on DM undergoes a non-relativistic, secluded freeze-out in the hidden sector, see  \cite{Feng:2009mn,Chu:2011be}, because in this case the relic density would depend on the annihilation cross section and thus on $Y_\chi$. However, since DM is lighter than the two other particles in the hidden sector, the neutrino portal annihilation processes (for instance $\phi \phi \leftrightarrow \chi\chi$, $N N \leftrightarrow \chi\chi$ or $N N \leftrightarrow \phi\phi$) will in general not decouple when DM is non-relativistic but rather when DM is still relativistic. 
In this case, the relic density doesn't depend on the annihilation cross section but only on $T'/T$ (along the $T'/T$ relativistic floor scenario, see details in \cite{Hambye:2020lvy}). Thus, since $T'/T$ is set by the $SM\rightarrow N$ freeze-in induced by the $Y_\nu$ coupling, here one also finds a one-to-one relation between seesaw parameters and DM relic density. 
	
The value of $T'/T$ can be estimated by considering that at the peak of $N$ freeze-in production, when $T\simeq m_Z$, each $N$ has an energy $\simeq m_Z$, so that the HS energy density is
\begin{equation}
	\rho_{HS}|_{T\simeq m_Z} \simeq n_{N}|_{T\simeq m_Z}\, m_Z=(\pi^2/30)g^\star_{HS}T'^4 \, ,
	\label{rhoscenarioB}
\end{equation}
with $n_N$ given by Eq.~(\ref{N1yield}), and $g^\star_{HS} =9/2$ the number of HS degrees of freedom (from $N$, $\chi$ and $\phi$). Plugging $T'/T$ in the relativistic floor equation for the relic density (Eq.~2 of \cite{Hambye:2020lvy}, see also Eq.~9 of \cite{Hambye:2019tjt}) gives
\begin{eqnarray}
	\Omega_{DM}h^2 &\simeq&c \cdot 10^{18} \,\Big( \sum_i |Y_{\nu i}|^2 \Big)^{3/4} \nonumber \\
	&&  \cdot \,g_{DM}\,\Big( \frac{1\,{\rm GeV}}{m_N}\Big)^{3/2} \Big(\frac{m_{DM}}{100\hbox{ MeV}}\Big)  \, ,
\end{eqnarray}
where $c$ is equal to $2.5$ when determined from Eq.~\eqref{rhoscenarioB} and $9.5$ when properly determined from the energy transfer Boltzmann equation setting $\rho_{HS}$, see Eq.~(49) of \cite{Chu:2011be}. 
This requires slightly smaller values of $Y_\nu$ couplings than in Eq.~(\ref{Omegafreezein}), because the HS thermalisation process increases the number of DM particles. This also implies $m_{\nu_1} \lesssim 8.8 \times 10^{-14}\,\hbox{eV} \,(m_N/1\,\hbox{GeV})\, (100\, \hbox{MeV}/m_{DM})^{4/3}\,(1/g_{DM}^{4/3})$. 
Further details, including the fact that relativistic decoupling in this case requires $m_N\lesssim m_Z/100$ and $m_{DM}\lesssim m_N/10$, are left for another publication. % on ``scenario B"
%(unless the DM decoupling is relativistic as the annihilation cross section can be cutted off by the mass of the exchanged heavier $N$ mediator, see \cite{}, a possibility we will not consider here, but which can in principle also lead to the one-to-one relation). There is also a large window of values of $Y_\chi$ for which $Y_\chi$ is small enough to avoid thermalization, but large enough to have a lifetime shorter  than the age of the Universe. This case is not necessarily in contradiction with BBN or CMB constraints for the same reasons as above, but could modify the formation of large scale structure, in particular if the decay occur after radiation-matter equality epoch, a possibility we will not consider further here too.  
	
	
%leadin of the hidden sector also that one could consider an option where the neutrino portal is way larger so that the lifetime of the heaviest particle among $\chi$ and $\phi$ would be shorter than the age of the Universe, so that it is not a DM component.  In this case, given the small value of neutrino mixing, this particles necessarily decays when BBN has already started, which is typically harmless because it decays to a DM particle and a $\nu$, a possibility we will nevertheless not investigate further here. 
	
	
\section{Symmetries}
	
The setup assumed above is compatible with various types of symmetries. All symmetry patterns we will consider below assume new symmetries under which both $\chi$ and $\phi$ have non trivial charges while all SM particles, as well as the right-handed neutrinos, are singlets. Thus, the general structure is one of two well-defined sectors, the SM visible sector and a dark sector, each containing particles that are singlets of the symmetries of the other. Both sectors communicate through the right-handed neutrinos, which are natural particles to couple to both sectors, being singlets of each. The dark sector may of course contain more than just $\chi$ and $\phi$. The assumption that DM is created from SM particles through freeze-in can be easily justified on the basis of such a general pattern. One need just assume that the inflaton ``belongs'' to the visible sector, so that reheating proceeds into this sector.
	
	
\underline{$\mathbb{Z}_2$ discrete symmetry}: the simplest possibility to justify the existence of a neutrino portal interaction, without  other interactions that could induce DM decays that are too fast, is to assume a discrete $\mathbb{Z}_2$ symmetry under which $\chi$ and $\phi$ are odd, with all other particles being even. In this case, if $\phi$ doesn't develop any vev (so that a possible Higgs portal interaction doesn't destabilise it), the heavier particle of $\chi$ and $\phi$ slowly decays to the lighter one through seesaw interactions, as discussed above in both scenarios A and B.
%The lightest particle among $\chi$ and $\phi$ must be stable, whereas the heavier one necessarily decays very slowly to the lighter one from the decay above, leading to the possibility of neutrino line indirect detection outlined in scenario A. f
	
\underline{Global symmetry}: assuming a global $U(1)'$ symmetry under which $\chi$ and $\phi$  have opposite charge doesn't result in any important difference with respect to the discrete symmetry case, as long as $\phi$, which is for this case a complex scalar, doesn't develop any vev. %((((check better)))). \rupert{[$\phi$ is now a complex scalar, has a (pseudo) Nambu-Goldstone boson\ldots]}
	
\underline{Local symmetry}: considering a local $U(1)'$ symmetry under which $\chi$ and $\phi$ have opposite charge potentially induces a number of new phenomena. This will be the case in particular if the scalar $\phi$ breaks this $U(1)^\prime$ symmetry by acquiring a vev, so that the associated gauge boson is massive (as it generally must be). In this case, new extra decay channels for both $\chi$ and $\phi$ arise, which could in particular easily destabilise the $\phi$ state
%$\phi$ is not expected to be a DM component anymore because it could be easily destabilized, 
(for instance by a Higgs portal interaction $H^\dagger H \phi^\dagger\phi$, even if it is extremely tiny).
The $\chi$ DM particle can also be destabilised by new decay channels, in particular from the fact that it mixes with the SM neutrinos proportionally to both the $Y_\nu$ and $Y_\chi$ couplings and to both vevs, $v$ and $v_\phi$, with $\sin \theta_{\nu-\chi} \sim Y_\nu Y_\chi v v_\phi/(m_N m_\chi)$. 
%This implies that ....((((check)))).
The existence of a dark photon, $\gamma'$, implies a possible $\chi\rightarrow \nu \gamma'$ decay with width proportional to $\sin^2 \theta_{\nu-\chi}$ and the $U(1)'$ fine structure constant, $\alpha'$. It can make the $\chi$ too unstable unless the couplings are small enough or the dark photon is heavy enough.
% mixing is suppressed enough, the dark photon is heavy enough, or the $U(1)$ gauge coupling is tiny ((((check)))) \rupert{[Agreed]}. 
A $\chi\rightarrow \nu e^+ e^-$ decay is also possible if there is kinetic mixing between the new $U(1)'$ and the SM hypercharge $U(1)_Y$ group. Thus, the seesaw/DM relic density correspondence is viable but requires that quite a number of interactions are tiny.
%, {\bf unless the seesaw scale is sufficiently high.
%{\bf In a different vein for a seesaw scale of order the GUT scale and spin-1 DM this is viable without the need for assuming any tiny parameter, as recently studied in \cite{Coy:2020wxp}, generically leading to a neutrino line too.}
%, where it was shown that the dark photon is a sufficiently stable DM candidate\ldots]}
%Anomaly cancellations issue... (but trivial).
	
\underline{No symmetries}: If ``just so'' there are no symmetries beyond the SM ones, a number of \textit{a priori} allowed couplings must necessarily be extremely tiny (so as not to destabilise the $\chi$ or $\phi$ DM component(s)), such as $\chi LH$ or $\phi H^\dagger H$ interactions. While possible, this appears more \textit{ad hoc} than with a symmetry.
%not as attractive as when justified by a symmetry.
	
	
In summary, seesaw-induced $W$, $Z$ and $h$ decays could be at the origin of the DM relic density, even though DM is not a seesaw sterile neutrino.
Given the current neutrino mass and mixing constraints, the usual type-I seesaw model turns out to have sufficient flexibility to allow freeze-in  production of DM from these decays in a way which is determined only by the seesaw parameters and the mass(es) of the DM particle(s). Two scenarios have been proposed above. As always for freeze-in, these scenarios are not easily testable because they are based upon the existence of tiny interactions, here the seesaw Yukawa couplings of at least one of the right-handed neutrinos, and possibly also the neutrino portal interactions. However, for a whole range of DM masses, scenario A predicts a neutrino-line within reach of existing or near-future neutrino telescopes. Moreover, both scenarios A and B are falsifiable as they predict a small mass for the lightest neutrino. 
	
	
	
\section*{Acknowledgments}
This work is supported by the ``Probing dark matter with neutrinos" ULB-ARC convention, by the F.R.S./FNRS under the Excellence of Science (EoS) project No. 30820817 - be.h ``The H boson gateway to physics beyond the Standard Model'', and by the IISN convention 4.4503.15. 
R.C. thanks the UNSW School of Physics, where he is a Visiting Fellow, for their hospitality during this project.
	
	
	
\bibliographystyle{apsrev4-1}
\bibliography{ref.bib}
	
\end{document}


