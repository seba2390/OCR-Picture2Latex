\documentclass[a4,10pt]{article}
\usepackage{graphicx}
\usepackage{amsmath,amssymb}
%\usepackage[notref, notcite]{showkeys}
\usepackage{ulem}

\usepackage{color}
\usepackage{here}

\begin{document}
\setlength{\baselineskip}{15pt} %20pt
\def\pot{\mathaccent"7017} %={}^o 
\makeatletter
 \renewcommand{\theequation}{%
  \thesection.\arabic{equation}}
 \@addtoreset{equation}{section}
\makeatother


\newtheorem{remark}{Remark}[section]
\newtheorem{theo}{Theorem}[section]
\newtheorem{lemma}{Lemma}[section]
\newtheorem{prop}{Proposition}[section]
\newtheorem{assume}{Assumption}[section]
\newtheorem{cor}{Corollary}[section]
\newtheorem{example}{Example}[section]

\def\ep{\varepsilon}
\def\ov{\overline}
\def\un{\underline}
\def\del{\partial}
\def\norm{\parallel}
\def\no{\noindent}
\def\lam{\lambda}
\def\dis{\displaystyle}
\def\bhat{\widehat}
\def\lap{\bigtriangleup}
\def\R{\mbox{\bf R}}
\def\C{\mbox{\bf C}}
\def\lg{\langle}
\def\rg{\rangle}

\newcommand{\Qed}{ $\square$ }



\title{
Single Transition Layer in Mass-Conserving Reaction-Diffusion 
Systems with Bistable Nonlinearity
}



\author{
Masataka Kuwamura\thanks{
Graduate School of Human Development and Environment,
Kobe University, Kobe 
657-8501, Japan,
email: kuwamura@main.h.kobe-u.ac.jp }\;
,
Takashi Teramoto\thanks{
Department of Data Science, Kyoto Women's University, Kyoto 605-8501, Japan, 
email: teramotot@kyoto-wu.ac.jp}\;
and
Hideo Ikeda\thanks{
Department of Mathematics, University of Toyama, Toyama,
930-8555, Japan,
email: hideoikeda5@gmail.com}
}


\date{}

%\pagebreak

\maketitle
\begin{abstract}
Mass-conserving reaction-diffusion 
systems with bistable nonlinearity
are useful models for studying cell polarity formation,
which is a key process in cell division and differentiation.
We show the existence and stability of stationary solutions with a
single internal transition layer in such reaction-diffusion systems
under general assumptions by the singular perturbation theory.
\end{abstract}

\vspace{1ex}

Abbreviated title:  Transition Layer in Reaction-Diffusion Systems

\vspace{1ex}

Key words: reaction-diffusion system, mass conservation, stability, \\ 
\hspace{2.3cm} transition layer

\vspace{1ex}

AMS subject classifications: 35B25, 35K57 



\section{Introduction } \label{intoro}

Reaction-diffusion systems are useful models for studying the mechanism of appearance of non-uniform patterns in various fields of science and technology.
In this paper, we consider the existence and stability of stationary solutions with a
single internal transition layer in the following reaction-diffusion system:
%
\begin{equation}\label{a1}
\left \{
\begin{array}{rcl}
u_t & = & \ep^2 u_{xx} +  f(u,v)  \\[1ex]
v_t & = & Dv_{xx}  -  f(u,v)
\end{array} 
\right.
\end{equation}
%
on an interval $0 < x < 1$ under the Neumann boundary condition, 
where $\ep$ and $D$ are positive constants satisfying $0 < \ep \ll D$.
The nonlinear term $f$ is a smooth cubic function such that 
the ODE $u_t = f(u, v)$ is bistable in $u$ for each fixed $v$.
This type of reaction-diffusion system is proposed by \cite{MJE1} 
for studying the wave-pinning phenomena in cell division and differentiation:
A transient and localized stimulus to 
an unpolarized cell is spatially amplified to result in a robust subdivision of the cell into
two clearly defined regions, front and back. 
In simple terms, this biological phenomena can be mathematically interpreted as 
the dynamics of \eqref{a1} where a traveling front solution starting from the edge 
converges to a stationary solution with a single internal transition layer. 
In fact, \cite{MJE1,MJE2} concluded that 
%
\begin{equation}\label{a1x}
\left \{
\begin{array}{rcl}
u_{\tau} & = & \ep^2 u_{xx} +  f(u,v)  \\[1ex]
v_{\tau} & = & Dv_{xx}  -  f(u,v)
\end{array} 
\right.
\end{equation}
%
has stable stationary solutions with a single internal transition layer 
under certain conditions by using a formal analysis and a perturbative argument
against the background of cell biology.
Here, $\tau = t/\ep$ represents the fast time scale as compared to the time scale $t$, 
and the dynamics of \eqref{a1x} is equivalent to that of \eqref{a1}.
They confirmed their theoretical results by numerical simulations for \eqref{a1x}
with specific nonlinear terms. 


We note that \eqref{a1} is a typical example of reaction-diffusion systems
%
\begin{equation}\label{a2xx}
\left \{
\begin{array}{rcl}
u_t & = & d_1 u_{xx} +  f(u,v)  \\[1ex]
v_t & = &d_2 v_{xx}  -  f(u,v),
\end{array} 
\right.
\end{equation}
%
where the nonlinear term $f$ is an appropriate smooth function. 
They were proposed by \cite{Is, Ot} as a conceptual model for studying cell polarity formation which plays a key role in cell division and differentiation.
We immediately find that
any (smooth) solution of \eqref{a2xx} satisfies
%
\[%begin{equation}\label{a2zz}
\int_0^1 \left( u(x, 0) + v(x, 0) \right) dx \equiv 
\int_0^1 \left( u(x, t) + v(x, t) \right) dx
\]%end{equation}
%
under the Neumann or periodic boundary conditions.
Therefore, \eqref{a2xx} are called as the mass-conserving reaction-diffusion systems
\cite{BJF}.
Similarly, we immediately find that 
any (smooth) stationary solution of \eqref{a2xx} 
satisfies 
%
\[%begin{equation}\label{a2zzz}
d_1 u(x) + d_2 v(x) \equiv Const.
\]%end{equation}
%
under the Neumann or periodic boundary conditions. This fact suggests that
the asymptotic behavior of \eqref{a2xx} near stationary solutions can be similar to that of
scalar reaction-diffusion equations.


We should consider that  \eqref{a1} or \eqref{a1x} can include the dynamics which is 
similar to the dynamics of scalar 
reaction-diffusion equations with cubic nonlinearity such as the Allen-Cahn equation.
Recalling that the Allen-Cahn equation exhibits metastable patterns with
a single internal transition layer moving at the speed of $O(e^{-C/\ep})$ for some positive constant $C$
as shown in \cite{CP, HF, KO},
we wonder whether or not numerical solutions regarded as
stable stationary solutions in \cite{MJE1, MJE2} are metastable patterns.
In fact, metastable patterns are often misidentified as stable stationary solutions 
in numerical simulations; this would be a careful problem
in the study of the bifurcation structure of 
mass conserving reaction-diffusion systems with bistable nonlinearity
referring to numerical simulations as seen in \cite{MJE2}.
Moreover, the formal analysis and perturbative argument by \cite{MJE1,MJE2}
cannot distinguish stable stationary solutions from
metastable patterns.
Therefore, it is necessary and important to give a rigorous proof of the assertion that
\eqref{a1} or \eqref{a1x} has stable stationary solutions with a single internal transition layer.

The purpose of this paper is to show the existence and stability 
of stationary solutions with a
single internal transition layer in reaction-diffusion 
systems \eqref{a1} under general assumptions by the singular perturbation method \cite{HS, I, MTH, NF}.
Our approach provides a new point of view for studying mass-conserving 
reaction-diffusion systems.
In fact, to the best of 
our knowledge, existing mathematical analyses for mass-conserving 
reaction-diffusion systems
are concerned with localized unimodal patterns (spike solutions) under the 
influence of the ideas of the variational method \cite{CMS,ET,JM,EKS,LS,MO}.
First, we mention assumptions to obtain our main result.

\begin{assume}\label{ass1}
{\rm (A1)} 
The ODE $u_t = f(u, v)$
is bistable in $u$ for each fixed $v \in I = (\un{v}, \ov{v})$.
That is, $f(u, v)=0$ has exactly three roots 
$h^-(v) < h^0(v) < h^+(v) $ for each $v \in I$ satisfying
%
\[%begin{equation}\label{a3}
f_u ( h^{\pm}(v), v ) < 0
\ \ \
\text{and}
\ \ \ 
f_u ( h^{0}(v), v ) > 0. 
\]%end{equation}
%
{\rm (A2)} The function 
%
\begin{equation}\label{a4}
J(v) = \int_{h^-(v)}^{h^+(v)} f(u, v)du \ \ \ ( v \in I)
\end{equation}
%
has an isolated zero at $v = v^* \in I$ such that 
%
\begin{equation}\label{a4x}
J'(v^*) = \int_{h^-(v^*)}^{h^+(v^*)} f_v(u, v^*)du \neq 0.
\end{equation}

\no
{\rm (A3)} 
\[%begin{equation}\label{a5x}
f_u ( h^{\pm}(v), v ) < f_v ( h^{\pm}(v), v )  \ \ \ ( v \in I).
\]%end{equation}
\end{assume}

The assumptions (A1), (A2) and (A3) were used in the formal analysis and
perturbative argument by \cite{MJE2}. 
We emphasize that these assumptions can be easily verified in 
many practical problems including those of \cite{MJE1, MJE2}. 
Let
%
\[%begin{equation}\label{a2yy}
\xi := \int_0^1 \left( u(x, 0) + v(x, 0) \right) dx.
\]%end{equation}
%
Then, the conservation law
%
\begin{equation}\label{a2}
\int_0^1 \left( u(x, t) + v(x, t) \right) dx \equiv \xi
\end{equation}
%
holds. We choose $\xi$ in such a way that the following inequality holds:
%
\begin{equation}\label{a6}
h^-(v^*) + v^* < \xi < h^+(v^*) + v^*.
\end{equation}

Let us define 
%
\begin{equation}\label{a8}
U^*(x) = 
\left \{
\begin{array}{l}
h^-(v^*)  \ \ \ (0 \leq x \leq x^*) \\[1ex]
h^+(v^*)  \ \ \ (x^* < x \leq 1) 
\end{array} 
\right.
\end{equation}
%
and
%
\begin{equation}\label{a9}
V^*(x) = v^* \ \ \ ( 0 \leq x \leq 1),
\end{equation}
%
where 
%
\begin{equation}\label{a7}
x^* := \dis\frac{ h^+(v^*) + v^* - \xi}{h^+(v^*) - h^-(v^*)}
\end{equation}
%
satisfies $0 < x^* < 1$ by \eqref{a6}. We note that $x^*$
is determined by
%
$$%\begin{equation}\label{a2}
\int_0^1 \left( U^*(x) + V^*(x) \right) dx = \xi,
$$%\end{equation}
%
and that 
$(U^*(x), V^*(x))$ is 
the leading term of the outer approximation of stationary solutions 
constructed in Section~2.
In simple terms, our main result is summarized as follows:

\begin{theo}\label{th1}
Assume (A1) and (A2).
For any given $\xi$ satisfying \eqref{a6}, 
the reaction-diffusion system \eqref{a1} has a 
family of stationary solutions
$(u^{\ep}(x), v^{\ep}(x))$
satisfying 
%
\begin{equation}\label{a2xyz}
\int_0^1 \left( u^\ep(x) + v^\ep(x) \right) dx = \xi 
\end{equation}
%
for small $\ep$. They satisfy 
$$
 \lim_{\ep \to 0} u^\ep(x)  =  U^*(x) \ \mbox{uniformly on} \ [0, x^* - \sigma]
\cup [x^* + \sigma, 1]
$$
for any $\sigma$ with $0 < \sigma < \min(x^*, 1-x^*)$, and 
$$
 \lim_{\ep \to 0} v^\ep(x)  =  V^*(x) \ \, \mbox{uniformly on} \ [0,1]. 
$$
Moreover, under the assumptions (A1), (A2) and (A3), the stationary solutions
$(u^{\ep}(x), v^{\ep}(x))$ are stable if $J'(v^*) > 0 $.
\end{theo}




\begin{remark}\label{rem0x}
The stationary solutions given by Theorem~\ref{th1} are jump-up solutions.
If we take 
%
$$
U^*(x) = 
\left \{
\begin{array}{l}
h^+(v^*)  \ \ \ (0 \leq x \leq x^*) \\[1ex]
h^-(v^*)  \ \ \ (x^* < x \leq 1),
\end{array} 
\right.
$$
%
we can obtain jump-down stationary solutions, which are stable if $J'(v^*) > 0 $.
\end{remark}


The mathematically precise version of this theorem is presented by Theorem~\ref{th2},
Corollary~\ref{cor1} and Theorem~\ref{th4}.
They give a mathematical justification for the results in 
\cite{MJE1, MJE2}, which are obtained by a formal analysis and 
a perturbative argument with the aid of numerical simulations.

The organization of this paper is as follows.
In the next section, we construct stationary solutions with a single internal transition layer
by the matched asymptotic expansion method \cite{I} which was used for
scalar reaction-diffusion equations with bistable nonlinearity dependent on
the spatial variable $x$. 
The first key of our construction is to find a rule for recursively performing
the matched asymptotic expansion of any order under the constraint condition \eqref{a2xyz}.
Moreover, the second key is to apply the implicit function
theorem to guarantee that the stationary solutions exactly satisfy the constraint condition \eqref{a2xyz}. 
In Section \ref{stability}, we show the stability of the stationary solutions with a single internal
transition layer on the basis of the singular perturbation method by \cite{HS, NF} which were used for
reaction-diffusion systems with bistable nonlinearity of FitzHugh-Nagumo type.
Here, we cannot apply 
the Lax-Milgram theorem
to the singular limit eigenvalue problem (SLEP) for investigating the behavior of
a critical eigenvalue which essentially determines
the stability of the stationary solutions. 
In spite of this complication, we can give the precise characterization of the critical eigenvalue due to a natural constrained condition 
derived from the conservation law.
Consequently, we can show the stability of the 
stationary solutions with a single internal transition layer under 
the assumptions (A1), (A2) and (A3).
Section \ref{conclude} is devoted to concluding remarks.







\section{Existence of single transition layer solutions } \label{existence} % Section 2 


\noindent
In this section, we consider single transition layer solutions of 
%
\begin{equation}\label{b1}
\left \{
\begin{array}{l}
\begin{array}{l}
\ep^2 u_{xx} +  f(u,v) = 0, \\[1ex]
Dv_{xx}  - f(u, v) = 0, 
\end{array} \quad  x \in (0,1) \\[0.5cm]
(u_x, v_x)(0) = (u_x, v_x)(1) = (0, 0)
\end{array} 
\right.
\end{equation}
%
satisfying 
%
\begin{equation}\label{b1_1}
\xi = \int_0^1 \left( u(x) + v(x) \right) dx 
\end{equation}
%
for a given constant $\xi$ with \eqref{a6} 
under the assumptions (A1) and (A2).

Since $\ep^2 u_{xx} + Dv_{xx} = 0$ by \eqref{b1}, noting the Neumann
boundary condition, we have
%
\[%begin{equation}\label{b2}
\ep^2 u + Dv = C(\ep),
\]%end{equation}
%
where $C(\ep)$ is a constant dependent on $\ep$. 
Substituting $v = ( C(\ep) - \ep^2 u )/D$ into the first equation of \eqref{b1}, 
and \eqref{b1_1}, we have a single equation for $u$ 
%
\begin{equation}\label{b3}
\left \{
\begin{array}{l}
\ep^2 u_{xx} + f(u,  (C(\ep) - \ep^2 u )/D ) = 0, \ \ \ x \in (0, 1)  \\[1ex]
u_x(0) = u_x(1) = 0, 
\end{array} 
\right.
\end{equation}
%
and 
\begin{equation}\label{b3_1}
\xi = \dis\frac{C(\ep)}{D} + \left( 1 - \frac{\ep^2}{D} \right) \int_0^1 u(x ; \ep)dx,
\end{equation}
respectively. Though the problem \eqref{b3} is quite similar to the one that was discussed in \cite{I}, there are two differences between them. First, the nonlinear term $f$ in \eqref{b3} does not depend on $x$ explicitly, and second the extra condition  \eqref{b3_1} is added. We assume that a solution $u$ of \eqref{b3} exhibits a sharp jump-up transition layer at $x= x^*(\ep) \in (0,1)$.\par 
To solve this problem, in Section~\ref{S2.1}, we assume that $C(\ep)$ and $ x^*(\ep)$ 
are represented by 
\begin{equation}\label{b8}
C(\ep) = C_0 + \ep C_1, \ (C_0/D \in I) 
\end{equation}
%
and
%
\begin{equation}\label{b9}
x^*(\ep) = x_0 +  \ep x_1,
\end{equation}
respectively
for small $\ep > 0$, where 
$I = (\un{v}, \ov{v})$ is given in the assumption (A1).
Since we consider an approximate solution of \eqref{b3} 
up to $O(\ep)$, the error terms $O(\ep^2)$ do not appear in the
right hand side of \eqref{b8} and \eqref{b9}, i.e., 
$C_1$ and $x_1$ are to be determined as functions of $\ep$, respectively, in Section~\ref{S2.4}.
We divide the interval $[0, 1]$ into two subintervals $[0, x^*(\ep)]$ and $[x^*(\ep), 1]$, and consider the following two boundary value problems: 
%
\begin{equation}\label{b5}
\left \{
\begin{array}{l}
\ep^2 u_{xx} + f( u,  (C(\ep) - \ep^2 u )/D ) = 0, \ \ \ x \in  (0, x^*(\ep))   \\[1ex]
u_x(0) = 0, \ u(x^*(\ep)) = \alpha
\end{array} 
\right.
\end{equation}
%
and
%
\begin{equation}\label{b6}
\left \{
\begin{array}{l}
\ep^2 u_{xx} + f( u,  (C(\ep) - \ep^2 u )/D ) = 0, \ \ \ x \in  (x^*(\ep), 1) \\[1ex]
u(x^*(\ep)) = \alpha, \ u_x(1) = 0, 
\end{array} 
\right.
\end{equation}
where $\alpha$ is an arbitrary fixed constant satisfying $h^-(C_0/D) < \alpha < h^+(C_0/D)$ and $C_0/D \in I$.
By using the singular perturbation method used in \cite{I}, we show the existence of solutions satisfying \eqref{b5} and \eqref{b6}. In Section \ref{S2.2}, we match these solutions in $C^1$-sense at $x = x^*(\ep)$, from which we find an approximate solution of \eqref{b3} up to $O(\ep)$ by determining the relations between $C_j$ and $x_j$ for $j=0,1$. Similarly, we use the equation \eqref{b3_1}, and obtain the other relations  between $C_j$ and $x_j$ for $j=0,1$ in Section \ref{S2.3}. 
Finally, in Section \ref{S2.4}, using the result in Sections \ref{S2.2} and \ref{S2.3}, we determine the unknown constants $C(\ep)$ and 
$x^*(\ep)$ uniquely, and obtain the desired result about the existence of single transition layer solutions. 

Here, we again note the difference between our problem and that in \cite{I}. In \cite{I}, since the nonlinear term $f$ does depend on $x$ explicitly, $x^*(\ep)$ is determined by the $C^1$-matching at $x = x^*(\ep)$ and then the condition \eqref{b3_1} is not necessary. While in our problem \eqref{b3}, since $f$ does not depend on $x$ explicitly and contains an unknown constant $C(\ep)$, then to determine $C(\ep)$ and $x^*(\ep)$ uniquely we need two relations;  the $C^1$-matching at $x = x^*(\ep)$ and the condition \eqref{b3_1}. This difference comes from the property of mass conservation. 

We use the following function spaces with positive $\ep$ in this section: 
\begin{equation*}
\begin{array}{lcl}
C^2_\ep[0,1] & = & \left\{ u \in C^2[0,1] \ | \ \dis \sum_{k=0}^2 \max_{0 \leq x \leq 1} \left| \left( \ep \frac {d}{dx}\right)^{\! k} \! u(x) \right| < \infty \right\}, \\[3.5ex]
\pot{C}^2_\ep[0,1] & = & \left\{ u \in C^2_\ep[0,1] \ | \ u_x(0) = 0, \ u_x(1) = 0 \right\},  \\[2ex]
C^2_{\ep,0}[0,1] & = & \left\{ u \in C^2_\ep[0,1] \ | \ u_x(0) = 0, \ u(1) = 0 \right\},  \\[2ex]
C^2_{\ep,1}[0,1] & = & \left\{ u \in C^2_\ep[0,1] \ | \ u(0) = 0, \ u_x(1) = 0 \right\}.
\end{array} 
\end{equation*}

\subsection{Solutions of \eqref{b5} and \eqref{b6}}\label{S2.1} % Subsection 2.1 

First, we consider the approximation of the solution of \eqref{b5} up to $O(\ep)$. Applying the change of variables $x = x^*(\ep)y$, we have
%
\begin{equation}\label{b7}
\left \{
\begin{array}{l}
\ep^2 u_{yy} + (x^*(\ep))^2 f( u,  (C(\ep) - \ep^2 u )/D ) = 0, \ \ \ y \in (0, 1)   \\[1ex]
u_y(0) = 0, \ u(1) = \alpha.
\end{array} 
\right.
\end{equation}
%
In order to construct the outer approximation of the solution of \eqref{b7},
substituting 
%
\[%begin{equation}\label{b10}
u(y) = U^-_0(y) + \ep U^-_1(y) + O(\ep^2)
\]%end{equation}
%
into \eqref{b7}, and comparing each coefficients of powers of $\ep$, we have 
%
\begin{equation}\label{b11}
f( U^-_0, C_0/D) = 0 
\end{equation}
%
and
%
\begin{equation}\label{b12}
f_u^- U^-_1 + f_v^- C_1/D = 0,
\end{equation}
%
%
where $f_u^- = f_u( U^-_0, C_0/D)$, $f_v^- = f_v( U^-_0, C_0/D)$. 
Since we consider a jump-up solution at $y=1$, it follows from \eqref{b11} and
\eqref{b12} that
%
\begin{equation}\label{b14}
U^-_0(y) = h^{-}(C_0/D)
\end{equation}
%
and
%
\begin{equation}\label{b15}
U^-_1(y;C_1) = - (f_v^- / f_u^-) \cdot (C_1/D) = h^-_v(C_0/D) \cdot (C_1/D),
\end{equation}
%
where we used the relation $f_u(h^-(v), v) h^-_v(v) + f_v(h^-(v), v) = 0$
obtained by the differentiation of $f( h^-(v), v) = 0$ in $v$.
%
It should be noted that $U^-_j$ $(j=0, 1)$ are constants independent of $y$. 
%
Since these outer approximations do not satisfy the boundary condition at $y = 1$, 
we must consider the correction of the above approximation to the solution of \eqref{b7} in a neighborhood of $y = 1$ 
with the aid of the inner approximation given by
%
%
\begin{equation}\label{b17}
\begin{array}{l}
u(y)  =  U^-_0(y) + \ep U^-_1(y;C_1) + \phi^-_0( \frac{y-1}{\ep}) 
 + \ep \phi^-_1(  \frac{y-1}{\ep} ) + O(\ep^2).
\end{array}
\end{equation}
%
%
Introducing the stretched coordinate $z = (y-1)/\ep$, and substituting 
\eqref{b17} into \eqref{b7}, and 
comparing each coefficients of powers of $\ep$, we have 
%
\begin{equation}\label{b18}
\left \{
\begin{array}{l}
\ddot{\phi}_0^- + x_0^2 \tilde{f}^- = 0, \ \ \ z \in (-\infty, 0)  
 \\[1ex]
\phi^-_0(-\infty) = 0, \ \phi^-_0(0) = \alpha - U^-_0(1) 
\end{array} 
\right.
\end{equation}
%
and
%
\begin{equation}\label{b19}
\left \{
\begin{array}{l}
\ddot{\phi}_1^- + x_0^2 \tilde{f}^-_u \phi^-_1 = F_1^-(z;C_1,x_1), \ \ \ z \in (-\infty, 0)  
 \\[1ex]
\phi^-_1(-\infty) = 0, \ \phi^-_1(0) = -U^-_1(1;C_1),
\end{array} 
\right.
\end{equation}
%
%
where the dot notation denotes $d/dz$, and 
%
$$
F_1^-(z;C_1,x_1) =  -2x_0 x_1 \tilde{f}^- - x_0^2 \tilde{f}^-_u U^-_1(1;C_1) -x_0^2 \tilde{f}^-_v C_1/D ,
$$
%
%
$$ \tilde{f}^- = f( h^-(C_0/D) + \phi^-_0, C_0/D ), \ 
\tilde{f}^-_u = f_u( h^-(C_0/D) + \phi^-_0, C_0/D ),
$$
and $\tilde{f}^-_v$ is similarly defined. 
%
From the assumptions (A1) and (A2), 
we find that \eqref{b18} has a unique monotone
increasing solution $\phi^-_0(z)$ since $\alpha > h^-(C_0/D)$ and $C_0/D \in I$. 
Moreover, we see that the solution of \eqref{b19} 
is explicitly given by
%
\begin{equation}\label{b21}
\begin{array}{l}
\phi^-_1(z;C_1,x_1) = -U^-_1(1;C_1) \dis\frac{ \dot{\phi}_0^-(z) }{ \dot{\phi}_0^-(0) } \\[1ex]
\hspace*{2cm} - \ \dot{\phi}_0^-(z)  \dis\int_z^0 \frac{1}{ (\dot{\phi}_0^-(\eta))^2} 
\dis\int_{-\infty}^{\eta}  \dot{\phi}_0^-(\zeta) 
F_1(\zeta;C_1,x_1) d\zeta d\eta. 
\end{array} 
\end{equation}
%
\par
Now, we put 
\begin{equation}\label{b22zz}
\begin{array}{lcl}
U^-(y;\ep;C_1,x_1) & = & U^-_0(y) + \ep U^-_1(y;C_1) \\[0.2cm]
 &  & + \ \theta(y)\phi^-_0( \frac{y-1}{\ep}) 
+  \ep \theta(y)\phi^-_1(  \frac{y-1}{\ep};C_1,x_1) ,
\end{array}
\end{equation}
%
where $\theta(y) \in C^{\infty}[0,1]$ is a cut-off function satisfying 
\begin{equation*}
\begin{array}{c}
\theta(y) = 0, \quad y \in [0, 1/2]; \quad \theta(y) = 1, \quad y \in [3/4, 1]; \\[1ex]
 0 \leqq \theta(y) \leqq 1, \quad y \in (1/2, 3/4). 
\end{array}
\end{equation*}
Since $U^-(y;\ep;C_1,x_1)$ is an $O(\ep)$ approximation to a solution of \eqref{b7}, we set
%
\begin{equation}\label{b22_0}
\tilde{u}^-(y;\ep;C_1,x_1) = U^-(y;\ep;C_1,x_1) + \ep r^-(y;\ep;C_1,x_1) 
\end{equation}
%
and rewrite \eqref{b7} as the following form with respect to the remainder term $r^-$: 
%
\begin{equation}\label{b22_1}
\left \{
\begin{array}{c}
\ep^2 r^-_{yy} + (x^*(\ep))^2 f(U^- +\ep r^- ,  (C(\ep) - \ep^2 U^- -\ep^3 r^- )/D ) /\ep \\[1ex]
+ \ \ep U^-_{yy}(y;\ep;C_1,x_1) \ = \ 0, \ \ \ y \in (0, 1)   \\[1ex]
r^-_y(0) = 0, \ r^-(1) = 0.
\end{array} 
\right.
\end{equation}
%
When we simply write \eqref{b22_1} as 
\begin{equation}\label{b22_2}
\begin{array}{c}
T(r^-;\ep;C_1,x_1) \ = \ 0, 
\end{array} 
\end{equation}
$T$ is a smooth mapping from $C^2_{\ep,0}[0,1]$ to $C[0,1]$, and then we have the following lemma: 
%
\begin{lemma}\label{lem1b} 
For any given constants $C_1^*$ and $x_1^*$, there exist $\ep_0 > 0$, $\rho_0 > 0$, and $K >0$ such that for any 
$\ep \in (0, \ep_0)$ and $(C_1,x_1) \in \Delta_{\rho_0} \equiv \{(C_1,x_1) \in {\bf R}^2 \ | \ |(C_1,x_1) - (C^*_1,x^*_1)| \leq \rho_0 \}$, 
\begin{description}
\item[(i)] \  $||T(0;\ep;C_1,x_1) ||_{C[0,1]} = o(1)$ uniformly in $(C_1,x_1) \in \Delta_{\rho_0}$ as $\ep \to 0$; 
\item[(ii)] \ for any $r_1, r_2 \in C^2_{\ep,0}[0,1]$, \\[1ex]
$\displaystyle{ \left|\left| \frac {\partial T}{\partial r^-}(r_1;\ep;C_1,x_1) -  \frac {\partial T}{\partial r^-}(r_2;\ep;C_1,x_1) \right|\right|_{C^2_{\ep,0}[0,1] \to C[0,1]}
 \ \leq K || r_1 - r_2||_{C^2_{\ep,0}[0,1]} }$; 
\item[(iii)] \ \hspace*{0.5cm}$ \displaystyle{ \left|\left|  \left( \frac {\partial T}{\partial r^-} \right)^{ \! \! -1}  \! \! (0;\ep;C_1,x_1) \right|\right|_{C[0,1] \to C^2_{\ep,0}[0,1]}  \leq K.  }$ 
\end{description}
Moreover, the results (i)-(iii) hold also for $\partial T/\partial C_1$ and $\partial T/\partial x_1$ in place of $T$.
\end{lemma}

Since this lemma is proved by the argument similar to that of \cite[Lemma 4.3]{MTH}, we omit the proof. 

Owing to Lemma \ref{lem1b}, we can apply the implicit function theorem to \eqref{b22_2}, and thus obtain the following: 
\begin{prop}\label{prop1b} 
There exist $\ep_1 > 0$ and $\rho_1 > 0$ such that for any $\ep \in (0,\ep_1)$ and $\rho \in \Delta_{\rho_1}$, there exists $r^-(y;\ep;C_1,x_1) \in C^2_{\ep,0}[0,1]$ satisfying 
%
$$   
T(r^-(y;\ep;C_1,x_1);\ep;C_1,x_1) \ = \ 0. 
$$
%
Moreover, $r^-(y;\ep;C_1,x_1)$, $\partial r^- \! / \partial C_1(y;\ep;C_1,x_1)$ and $\partial r^- \! / \partial x_1(y;\ep;C_1,x_1)$ are uniformly continuous with respect to $(\ep,C_1,x_1) \in (0, \ep_1) \times \Delta_{\rho_1}$ in $C^2_{\ep,0}[0,1]$-topology and satisfy 
\begin{equation*}
\left. 
\begin{array}{l}
 ||r^-(y;\ep;C_1,x_1) ||_{C^2_{\ep,0}[0,1]} \\[2ex]
\displaystyle{ \left|\left| \frac {\partial r^-}{\partial C_1}(y;\ep;C_1,x_1) \right|\right|_{C^2_{\ep,0}[0,1] }} \\[3ex]
\displaystyle{ \left|\left| \frac {\partial r^-}{\partial x_1}(y;\ep;C_1,x_1) \right|\right|_{C^2_{\ep,0}[0,1] }}
\end{array} 
\right\}
= o(1) \  \mbox{uniformly in} \ (C_1,x_1) \in \Delta_{\rho_1} \ \mbox{as} \ \ep \to 0. 
\end{equation*}
\end{prop}

Therefore, we obtain the solution of \eqref{b5} on $[0, x^*(\ep)]$, which takes the form
%
\begin{equation}\label{b34}
\begin{array}{l}
u^-(x ; \ep;C_1,x_1)  \ :=  \tilde{u}^-(\frac{x}{x^*(\ep)}; \ep; C_1,x_1) \\[1ex]
\hspace*{1cm}  = U^-_0( \frac{x}{x^*(\ep)}) + \ep U^-_1(\frac{x}{x^*(\ep)};C_1) + \theta(\frac{x}{x^*(\ep)}) \phi^-_0( \frac{x-x^*(\ep)}{\ep x^*(\ep)}) \\[1ex]
\hspace{1.2cm} 
+ \  \ep \theta(\frac{x}{x^*(\ep)}) \phi^-_1(   \frac{x-x^*(\ep)}{\ep x^*(\ep)};C_1,x_1) + \ep r^-(\frac{x}{x^*(\ep)};\ep;C_1,x_1 ), \\[1ex]
\hspace*{6cm} \ x \in [0, x^*(\ep)]. 
\end{array}
\end{equation}

Next, we consider the solution of \eqref{b6}.
Applying the change of variables $x = x^*(\ep) + (1-x^*(\ep))y$, we have
%
\begin{equation}\label{b23}
\left \{
\begin{array}{l}
\ep^2 u_{yy} + (1- x^*(\ep))^2 f( u,  (C(\ep) - \ep^2 u )/D ) = 0, \ \ \ y \in (0, 1)   \\[1ex]
u(0) = \alpha, \ u_y(1) = 0, \ 
\end{array} 
\right.
\end{equation}
%
where $C(\ep)$ and $x^*(\ep)$ are given by \eqref{b8} and \eqref{b9}, respectively.
Applying the same lines of argument as applied to \eqref{b7}, 
we can obtain the outer approximation of \eqref{b23} 
%
\[%begin{equation}\label{b24}
u(y)  =  U^+_0(y) + \ep U^+_1(y;C_1) + O(\ep^2),
\]%end{equation}
%
where
%
\begin{equation}\label{b25}
U^+_0(y) = h^{+}(C_0/D) 
\end{equation}
and
%
\begin{equation}\label{b26}
U^+_1(y;C_1) = - (f_v^+ / f_u^+) \cdot (C_1/D) = h^+_v(C_0/D) \cdot (C_1/D),
\end{equation}
%
%
and $f_u^+ = f_u( U^+_0, C_0/D)$, $f_v^+ = f_v( U^+_0, C_0/D)$.
We note that $U^+_j$ $(j=0, 1)$ are constants independent of $y$.


Similarly to the problem \eqref{b7}, since these outer approximations do not satisfy the boundary condition at $y = 0$, 
we must consider the correction of the above approximation in a neighborhood of $y = 0$ 
with the aid of the inner approximation given by 
%
\begin{equation}\label{b28}
\begin{array}{l}
u(y)  =  U^+_0(y) + \ep U^+_1(y;C_1) + \phi^+_0( \frac{y}{\ep}) 
+  \ep \phi^+_1(  \frac{y}{\ep} ) + O(\ep^2).
\end{array}
\end{equation}
%
%
Introducing the stretched coordinate $z = y/\ep$, and substituting 
\eqref{b28} into \eqref{b23}, and 
comparing each coefficients of powers of $\ep$, we have 
%
\begin{equation}\label{b29}
\left \{
\begin{array}{l}
\ddot{\phi}_0^+ + (1-x_0)^2 \tilde{f}^+ = 0, \ \ \ z \in (0, \infty)  
 \\[1ex]
\phi^+_0(0) = \alpha - U^+_0(0), \  \phi^+_0(\infty) = 0 \ 
\end{array} 
\right.
\end{equation}
%
and
%
\begin{equation}\label{b30}
\left \{
\begin{array}{l}
\ddot{\phi}_1^+ + (1-x_0)^2 \tilde{f}^+_u \phi^+_1 = F_1^+(z;C_1,x_1), \ \ \ z \in (0, \infty)  
 \\[1ex]
 \phi^+_1(0) = -U^+_1(0;C_1), \ \phi^+_1(\infty) = 0, \
\end{array} 
\right.
\end{equation}
%
where the dot notation denotes $d/dz$, and 
%
$$
F_1^+(z;C_1,x_1) =  2(1-x_0) x_1 \tilde{f}^+ - (1- x_0)^2 \tilde{f}^+_u U^+_1(0;C_1) 
-(1-x_0)^2 \tilde{f}^+_v C_1/D ,
$$
%
%
$$ \tilde{f}^+ = f( h^+(C_0/D) + \phi^+_0, C_0/D ), \ 
\tilde{f}^+_u = f_u( h^+(C_0/D) + \phi^+_0, C_0/D ),
$$
%
and $\tilde{f}^+_v$ is similarly defined.
From the assumptions (A1) and (A2), 
we find that \eqref{b29} has a unique monotone
increasing solution $\phi^+_0(z)$ since $\alpha < h^+(C_0/D)$ and $C_0/D \in I$. 
Moreover, we see that the solution of \eqref{b30} is explicitly given by
%
\begin{equation}\label{b32}
\begin{array}{l}
\phi^+_1(z;C_1,x_1) = - \ U^+_1(0;C_1) \dis\frac{ \dot{\phi}_0^+(z) }{ \dot{\phi}_0^+(0) } \\[2ex]
\hspace*{2cm} - \ \dot{\phi}_0^+(z)  \dis\int_0^z \frac{1}{ (\dot{\phi}_0^+(\eta))^2} 
\dis\int_{\eta}^{\infty}  \dot{\phi}_0^+(\zeta) 
F_1^+(\zeta;C_1,x_1) d\zeta d\eta. 
\end{array} 
\end{equation}
%

Applying an argument similar to the above, we can find the solution 
$\tilde{u}^+(y;\ep;C_1,x_1)$ of \eqref{b23} as follows:
$$  
\tilde{u}^+(y;\ep;C_1,x_1)  =  U^+(y;\ep;C_1,x_1) + \ep r^+(y;\ep;C_1,x_1), 
$$
%
\begin{equation}\label{any1}
\begin{array}{l}
U^+(y;\ep;C_1,x_1) =  U^+_0(y) + \ep U^+_1(y;C_1)  \hspace*{1cm} \\[0.2cm]
\hspace*{1cm}+ \ \theta(1-y)\phi^+_0( \frac{y}{\ep}) 
+  \ep \theta(1-y)\phi^+_1(  \frac{y}{\ep};C_1,x_1 ). 
\end{array}
\end{equation}
%
Here, $r^+(y;\ep;C_1,x_1) \in C^2_{\ep,1}[0,1]$ satisfies 
\begin{equation*}
\left. 
\begin{array}{l}
 ||r^+(y;\ep;C_1,x_1) ||_{C^2_{\ep,1}[0,1]} \\[2ex]
\displaystyle{ \left|\left| \frac {\partial r^+}{\partial C_1}(y;\ep;C_1,x_1) \right|\right|_{C^2_{\ep,1}[0,1] }} \\[3ex]
\displaystyle{ \left|\left| \frac {\partial r^+}{\partial x_1}(y;\ep;C_1,x_1) \right|\right|_{C^2_{\ep,1}[0,1] }}
\end{array} 
\right\}
= o(1) \  \mbox{uniformly in} \ (C_1,x_1) \in \Delta_{\rho_2} \ \mbox{as} \ \ep \to 0. 
\end{equation*}
%
Moreover, 
$r^+(y;\ep;C_1,x_1)$, $\partial r^+ \! /\partial C_1(y;\ep;C_1,x_1)$ and $\partial r^+ \! /\partial x_1(y;\ep;C_1,x_1)$ are uniformly continuous with respect to $(\ep,C_1,x_1) \in (0, \ep_2) \times \Delta_{\rho_2}$
in $C^2_{\ep,1}[0,1]$-topology, 
where $\ep_2$ and $\rho_2$ are positive constants. 
Thus, we obtain the solutions of \eqref{b6} on $[x^*(\ep), 1]$ which takes the form 
%
%
%
%
\begin{equation}\label{b35}
\begin{array}{l}
\! \! \! u^+(x ; \ep;C_1,x_1)  \ :=  \tilde{u}^+(\frac{x-x^*(\ep)}{1- x^*(\ep)}; \ep;C_1,x_1) \\[1ex]
\hspace*{1cm}  = 
U^+_0( \frac{x-x^*(\ep)}{1- x^*(\ep)}) + \ep U^+_1(\frac{x-x^*(\ep)}{1- x^*(\ep)};C_1) + \theta(\frac{1-x}{1- x^*(\ep)}) \phi^+_0( \frac{x-x^*(\ep)}{\ep(1- x^*(\ep))}) 
 \\[1ex]
\hspace{1.2cm} 
+ \ \ep \theta(\frac{1-x}{1- x^*(\ep)}) \phi^+_1(  \frac{x-x^*(\ep)}{\ep(1- x^*(\ep))};C_1,x_1) + \ep r^+(y;\ep,C_1,x_1), \\[1ex]
\hspace*{6cm} x \in [x^*(\ep), 1]. 
\end{array}
\end{equation}
%
%

\subsection{$C^1$-matching at $x = x^*(\ep)$}\label{S2.2} %Subsection 2.2 

We now consider the $C^1$-matching of $u^-(x ; \ep;C_1,x_1)$ and $u^+(x ; \ep;C_1,x_1)$ at $x = x^*(\ep)$ to obtain the approximation of the solution of \eqref{b3} up to $O(\ep)$. Since these two solutions are already continuous at $x = x^*(\ep)$, we then determine the values of $C_j$ and $x_j$ $(j=0, 1)$ in such a way that
%
\[
\Phi(\ep) := \ep x^*(\ep) ( 1 - x^*(\ep)) \{
\frac{d}{dx}u^-(x^*(\ep) ; \ep;C_1,x_1) - \frac{d}{dx}u^+(x^*(\ep) ; \ep;C_1,x_1) \} = o(\ep) 
\]
%
holds for small $\ep >0$. Noting that $U^-_j$ and $U^+_j$ are constants,  
it follows from \eqref{b34} and \eqref{b35} that
%
\[
\begin{array}{lcl}
\Phi(\ep) & = & \ep  ( 1 - x^*(\ep)) \{
\dot{\phi}^-_0(0)/\ep + \dot{\phi}^-_1(0;C_1,x_1) + o(1) \} \\[1ex]  
 & & - \ \ep  x^*(\ep) \{
\dot{\phi}^+_0(0)/\ep + \dot{\phi}^+_1(0;C_1,x_1)  + o(1) \} \\[1ex] 
& = & \Phi_0 + \ep \Phi_1(C_1,x_1) + o(\ep),
\end{array}
\]
%
where
%
\begin{equation}\label{b35zz}
\Phi_0 = (1-x_0) \dot{\phi}^-_0(0) - x_0  \dot{\phi}^+_0(0), 
\end{equation}
%
$$
\Phi_1(C_1,x_1) =  (1-x_0) \dot{\phi}^-_1(0;C_1,x_1) - x_1  \dot{\phi}^-_0(0) - x_0  \dot{\phi}^+_1(0;C_1,x_1)
- x_1  \dot{\phi}^+_0(0).
$$
%

First, we consider $\Phi_0 =  (1-x_0) \dot{\phi}^-_0(0) - x_0  \dot{\phi}^+_0(0) = 0 $.
It follows from \eqref{b18} that 
%
\[
\begin{array}{rcl}
0 & = & \dis\int_{-\infty}^0 \{ \ddot{\phi}^-_0 \dot{\phi}^-_0
+ x_0^2 f( h^-(C_0/D) + \phi^-_0, C_0/D ) \dot{\phi}^-_0 \} dz 
\\[3ex]
& = & 
\dis\frac{(\dot{\phi}^-_0(0))^2}{2} + x_0^2
\dis\int_{h^-(C_0/D)}^\alpha f(u, C_0/D ) du,
\end{array}
\]
%
which implies 
%
\[
\dot{\phi}^-_0(0) = x_0 \sqrt{-2 \dis\int_{h^-(C_0/D)}^\alpha f(u, C_0/D ) du}. 
\]
%
Similarly, it follows from \eqref{b29} that 
%
\[
\dot{\phi}^+_0(0) = (1-x_0) \sqrt{ 2 \dis\int_\alpha^{h^+(C_0/D)} f(u, C_0/D ) du}. 
\]
%
Therefore, we have
%
\[
\Phi_0 = 
-\dis\frac{ 2 x_0(1-x_0) J(C_0/D)}
{ \sqrt{- 2 \dis\int_{h^-(C_0/D)}^\alpha f(u, C_0/D ) du}
+ \sqrt{  2 \dis\int_\alpha^{h^+(C_0/D)} f(u, C_0/D ) du} },
\]
%
where $J=J(v)$ is given by \eqref{a4}. Hence, noting the assumptions (A.1) and (A.2),
it follows from \eqref{b14}, \eqref{b25} and $\Phi_0 = 0$ that
%
\begin{equation}\label{b35x}
C_0 = Dv^* \ \ \ \text{and} \ \ \ U^{\pm}_0(y) = h^{\pm}(v^*).
\end{equation}
%
We note that though $\Phi_0$ depends on both $C_0$ and $x_0$, the solution satisfying $\Phi_0 = 0$ is determined by only $C_0 = D v^*$ for any $x_0 \in (0,1)$.
Moreover, we have
%
\begin{equation}\label{b36}
\dot{\phi}^-_0(0) = x_0 \sqrt{-2 \dis\int_{h^-(v^*)}^\alpha f(u, v^* ) du}
\end{equation}
%
and
%
\begin{equation}\label{b37}
\dot{\phi}^+_0(0) = (1-x_0) \sqrt{ 2 \dis\int_\alpha^{h^+(v^*)} f(u, v^* ) du}. 
\end{equation}

Next, we consider
$\Phi_1(C_1,x_1) = (1-x_0) \dot{\phi}^-_1(0;C_1,x_1) - x_1  \dot{\phi}^-_0(0) - x_0  \dot{\phi}^+_1(0;C_1,x_1) - x_1  \dot{\phi}^+_0(0) = 0$.
Note the following relation: 
%
\[
\dis\int_{-\infty}^0 x_0^2 \tilde{f}^-_u  \dot{\phi}^-_0 dz
=
 - \dis\int_{-\infty}^0 \dddot{\phi}^-_0  dz
= 
-  \ddot{\phi}^-_0 (0), 
\]
%
where we used the relation $\dddot{\phi}^-_0 + x_0^2 \tilde{f}_u^- \dot{\phi}_0^- = 0$ obtained by
the differentiation of the first equation of \eqref{b18} in $z$. 
Then, it follows from \eqref{b21} that
%
\begin{equation}\label{b38}
\begin{array}{l}
 \dot{\phi}^-_1(0;C_1,x_1) 
= \dis  -U^-_1(1;C_1)\frac {\ddot{\phi}^-_0 (0)}{\dot{\phi}_0^-(0)} + 
 \frac{1}{\dot{\phi}_0^-(0)}\dis\int_{-\infty}^0 F_1^-(z;C_1,x_1) \dot{\phi}^-_0 dz
\\[3ex]
\ \ =
\dis\frac{1}{\dot{\phi}_0^-(0)} \left(   -U^-_1(1;C_1)\ddot{\phi}^-_0 (0) 
-2x_0 x_1 \dis\int_{-\infty}^0  \tilde{f}^- \dot{\phi}^-_0 dz \right. \\[3ex]
\left. \hspace*{2.5cm} - x_0^2 U^-_1(1;C_1) \dis\int_{-\infty}^0  \tilde{f}^-_u \dot{\phi}^-_0 dz
 - \frac{ x_0^2 C_1 }{D} \dis\int_{-\infty}^0 \tilde{f}^-_v \dot{\phi}^-_0 dz \right)
\\[3ex]
\ \ = 
\dis\frac{1}{\dot{\phi}_0^-(0)} \left(
-2x_0 x_1 \dis\int_{h^-(v^*)}^\alpha  f(u, v^*) du
- \dis\frac{ x_0^2 C_1 }{D} \dis\int_{h^-(v^*)}^\alpha  f_v(u, v^*) du \right). 
\end{array}
\end{equation}
%
Similarly, it follows from \eqref{b29} and \eqref{b32} that
%
\begin{equation}\label{b39}
\begin{array}{l}
\dot{\phi}^+_1(0;C_1,x_1)  
= 
\dis\frac{1}{\dot{\phi}_0^+(0)} \left(
-2(1-x_0 ) x_1 \dis\int^{h^+(v^*)}_\alpha  f(u, v^*) du \right.
\\[3ex]
\ \ \ \ \ \ \ \ \ \ \ \ \ \ \ \ \ \ \ \ \ \ \ \ \ \ \ \ \ \ \ \ \ \ \ 
\left. + \dis\frac{ (1-x_0)^2 C_1 }{D} \dis\int^{h^+(v^*)}_\alpha  f_v(u, v^*) du \right).
\end{array}
\end{equation}
%
Hence, by \eqref{b38} and \eqref{b39}, we have
%
\[
\begin{array}{rcl}
\Phi_1(C_1,x_1) & = & (1-x_0) \dot{\phi}^-_1(0;C_1,x_1) - x_1  \dot{\phi}^-_0(0) - x_0  \dot{\phi}^+_1(0;C_1,x_1) - x_1  \dot{\phi}^+_0(0)
\\[2ex]
& =: & K(x_0)x_1 + M(x_0)C_1 + R(x_0),
\end{array}
\]
%
where
%
\begin{equation}\label{b40}
\begin{array}{l}
K(x_0) = \dis\frac{x_0(1-x_0)}{\dot{\phi}^-_0(0)}
\left( -2 \dis\int_{h^-(v^*)}^\alpha f(u, v^* ) du \right) - \dot{\phi}^-_0(0)
\\[3ex]
\ \ \ \ \ \ \ \ \ \ \ 
+ \dis\frac{x_0(1-x_0)}{\dot{\phi}^+_0(0)}
\left( 2 \dis\int_\alpha^{h^+(v^*)} f(u, v^* ) du \right) - \dot{\phi}^+_0(0),
\end{array}
\end{equation}
%
\begin{equation}\label{b41}
\begin{array}{l}
M(x_0) = - \dis\frac{x_0^2(1-x_0)}{D\dot{\phi}^-_0(0)}
\left(  \dis\int_{h^-(v^*)}^\alpha f_v(u, v^* ) du \right) 
\\[3ex]
\ \ \ \  \ \ \ \ \ \ \ \ \ \ 
- \dis\frac{x_0(1-x_0)^2}{D\dot{\phi}^+_0(0)}
\left( \dis\int_\alpha^{h^+(v^*)} f_v(u, v^* ) du \right),
\end{array}
\end{equation}
%
and $R(x_0) = 0$. Moreover, it follows from \eqref{b36}, \eqref{b37} and 
\eqref{b40} that
%
\[%\begin{equation}\label{b42}
\begin{array}{rcl}
K(x_0) & = & (1-2x_0) \left( \sqrt{ -2 \dis\int_{h^-(v^*)}^\alpha f(u, v^* ) du } 
- \sqrt{ 2 \dis\int_\alpha^{h^+(v^*)} f(u, v^* ) du } \right)
\\[4ex]
& = & \dis\frac{ -2(1-2x_0) J(v^*)}{ \sqrt{ -2 \dis\int_{h^-(v^*)}^\alpha f(u, v^* ) du }
+ \sqrt{ 2 \dis\int_\alpha^{h^+(v^*)} f(u, v^* ) du } } = 0,
\end{array}
\]%\end{equation}
%
where $J(v)$ is given by \eqref{a4}. 
This implies that $\Phi_1$ does not depend on $x_1$.
Noting
%
\[
\dis\int_\alpha^{h^+(v^*)} f(u, v^* ) du = - \dis\int_{h^-(v^*)}^\alpha f(u, v^* ) du
\]
%
by $J(v^*) = 0$, 
it follows from \eqref{b36}, \eqref{b37} and 
\eqref{b41} that
%
\[%\begin{equation}\label{b43}
M(x_0)  = 
- \dis\frac{x_0(1-x_0)}{D  \sqrt{ -2 \dis\int_{h^-(v^*)}^\alpha f(u, v^* ) du } }
\dis\int_{h^-(v^*)}^{h^+(v^*)} f_v(u, v^* ) du \neq 0 
\]%\end{equation}
%
by \eqref{a4x}. Therefore, it follows from \eqref{b15}, \eqref{b26} and $\Phi_1 = 0$
that 
%
\begin{equation}\label{b43x}
C_1 = -R(x_0)/M(x_0) = 0 \ \ \ \ \text{and} \ \ \ \  U^{\pm}_1(y;C_1) = h_v^{\pm}(v^*)C_1/D = 0.
\end{equation}
%

\subsection{Computation of \eqref{b3_1}}\label{S2.3} %Subsection 2.3 

To complete the construction of the approximate solution of \eqref{b3} satisfying \eqref{b3_1}, 
we determine the values of $C_j$ and $x_j$ $(j=0,1)$ by the conservation law \eqref{b3_1}. 
Here, we put 
%
\begin{equation}\label{b44}
\Psi(\ep) := \dis\frac{C(\ep)}{D} + \left( 1 - \frac{\ep^2}{D} \right) \int_0^1 u(x;\ep;C_1,x_1)dx - \xi. 
\end{equation}
%
\eqref{b3_1} is equivalent to $\Psi(\ep) = 0$. 
Using \eqref{b34}, \eqref{b35}, \eqref{b35x} and \eqref{b43x}, we 
have
%
\begin{equation}\label{b45}
\int_0^1 u(x;\ep;C_1,x_1)dx  = \int_0^{x^*(\ep)} u^-(x;\ep;C_1,x_1)dx  
+ \int_{x^*(\ep)}^1 u^+(x;\ep;C_1,x_1)dx,
\end{equation}
%
where
\[
\begin{array}{l}
\dis\int_0^{x^*(\ep)} u^-(x;\ep;C_1,x_1)dx  = x^*(\ep) \{
\dis\int_0^1 ( U^-_0 +\ep  U^-_1 + o(\ep) ) dy 
\\[3ex]
\ \ \ \ \ \ \ \ \ \ \ \ \ \ \ 
+ \, \ep \dis\int_{-\infty}^0  ( {\phi}_0^-(z) +  \ep {\phi}_1^-(z;C_1,x_1) + o(\ep) )dz
\, \} 
\\[3ex]
\ \ \ \ \ 
= ( x_0 + \ep x_1 + o(\ep) )\{ h^-(v^*) + \ep h_v^-(v^*)C_1/D 
+ \ep \dis\int_{-\infty}^0   {\phi}_0^-(z) dz + o(\ep) \}
\end{array}
\]
%
and
%
\[
\begin{array}{l}
\dis\int_{x^*(\ep)}^1 u^+(x;\ep;C_1,x_1)dx  = (1-x^*(\ep)) \{
\dis\int_0^1 ( U^+_0 +\ep  U^+_1 + o(\ep) ) dy 
\\[3ex]
\ \ \ \ \ \ \ \ \ \ \ \ \ \ \ 
+ \, \ep \dis\int^{\infty}_0  ( {\phi}_0^+(z) +  \ep {\phi}_1^+(z;C_1,x_1) + o(\ep))dz
\, \} 
\\[3ex]
\ \ \ \ \ 
= (1- x_0 - \ep x_1 + o(\ep) )\{ h^+(v^*) + \ep h_v^+(v^*)C_1/D 
+  \ep \dis\int^{\infty}_0   {\phi}_0^+(z) dz + o(\ep) \}.
\end{array}
\]
%
Substituting \eqref{b8} and \eqref{b45} into \eqref{b44}, we have  
\begin{equation}\label{b45_1}
\begin{array}{l}
\Psi(\ep) = \dis \{v^* + x_0 h^-(v^*) + (1- x_0) h^+(v^*) - \xi \} \\[1ex]
+ \ \ep \left\{ x_0 h_v^-(v^*)C_1/D + 
x_0 \dis\int_{-\infty}^0   {\phi}_0^-(z) dz + x_1 h^-(v^*)\right. \\[3ex]
\left. + \ (1-x_0) h_v^+(v^*)C_1/D 
+ (1-x_0) \dis\int^{\infty}_0   {\phi}_0^+(z) dz   - x_1 h^+(v^*) \right\} 
+ \ o(\ep) \\[3ex]
 =: \Psi_0 + \ep \Psi_1(C_1,x_1) + o(\ep). 
\end{array}
\end{equation}
%
Comparing each coefficients of powers of $\ep$ in \eqref{b45_1}, we have 
$ \Psi_i = 0 \ (i=0,1)$. 
%
%
Noting \eqref{a6} and $h^-(v^*) < h^+(v^*)$ by the assumption (A1), 
it follows from $\Psi_0 = 0$ that
%
\begin{equation}\label{b48_0}
\begin{array}{l}
x_0 = x^* = \dis\frac{ h^+(v^*) + v^* - \xi}{h^+(v^*) - h^-(v^*) }
\ \ \ \text{and} \ \ \ 
0 < x_0 < 1.
\end{array}
\end{equation}
%
Moreover, it follows from $\Psi_1(C_1,x_1) = 0$ that
%
\begin{equation}\label{b48_1}
\begin{array}{l}
\{x_0 h_v^-(v^*)+(1-x_0)h_v^+(v^*)\}C_1/D \hspace{1cm} \\[1ex]
\hspace{2cm} -  \ \{h^+(v^*)-h^-(v^*)\}x_1 + I_1(x_0) = 0,
\end{array}
\end{equation}
%
where 
\[
I_1(x_0) =  x_0 \dis\int_{-\infty}^0   {\phi}_0^-(z) dz 
+ (1-x_0) \dis\int^{\infty}_0   {\phi}_0^+(z) dz
\]
%
is a function of $x_0$. Since $C_1 = 0$ by \eqref{b43x}, 
it follows from \eqref{b48_1} that 
$x_1$ is uniquely determined by 
\begin{equation}\label{b48_2}
\begin{array}{l}
 x_1 = \dis \frac {I_1(x_0)}{h^+(v^*)-h^-(v^*)}.
\end{array}
\end{equation}
Thus, we see that \eqref{b34} and \eqref{b35} give 
the approximation of the solution of \eqref{b3} with \eqref{b3_1}. 

\subsection{Determination of $C(\ep)$ and $x^*(\ep)$}\label{S2.4} % Subsection 2.4 
Finally, we determine  $C(\ep)$ and $x^*(\ep)$ uniquely such that \eqref{b3} with \eqref{b3_1} have a 
single transition layer solution $u(x;\ep)$ at the layer position $ x = x^*(\ep)$. 



First, the coefficients $C_i$ and $x_i$ $(i=0,1)$ are determined step by step as follows:  $C_0$ (by \eqref{b35x}) $\longrightarrow$ 
$x_0$ (by \eqref{b48_0})  $\longrightarrow$ $C_1$ (by \eqref{b43x})  $\longrightarrow$ $x_1$ (by \eqref{b48_2}). We note that $\Phi(\ep) = o(\ep)$ and $\Psi(\ep) = o(\ep)$; $\Phi$ and $\Psi$ are not identically zero for these $C_1$ and $x_1$. 

Next, we set $(C_1^*, x_1^*) = (C_1, x_1)$ in Lemma \ref{lem1b}, and 
consider $C(\ep) = C_0 + \ep \bar{C}_1$ and 
$x(\ep) = x_0 + \ep \bar{x}_1$. 
We can take $(\bar{C}_1, \bar{x}_1)$ around $(C_1^*, x_1^*)$ 
so as to satisfy $\Phi(\ep) = 0$ and $\Psi(\ep) = 0$ as follows:
Let us define $\Phi^*(\bar{C}_1, \bar{x}_1;\ep)$ and  
$\Psi^*(\bar{C}_1, \bar{x}_1;\ep)$ by $\Phi(\ep) = \ep \Phi^*(\bar{C}_1, \bar{x}_1;\ep)$ and $\Psi(\ep) = \ep \Psi^*(\bar{C}_1, \bar{x}_1;\ep)$, respectively. We easily find that there exist two positive constants $\delta$ and $\ep_3 (< \min \{\ep_1, \ep_2\})$ 
such that $\Phi^*(\bar{C}_1, \bar{x}_1;\ep)$ and $\Psi^*(\bar{C}_1, \bar{x}_1;\ep)$ are continuous in $\bar{C}_1 \in (C_1^*-\delta, C_1^*+\delta), \bar{x}_1 \in (x_1^*-\delta, x_1^*+\delta)$ and $\ep \in [0,\ep_3)$, and are $C^1$-functions of $\bar{C}_1$ and $\bar{x}_1$. Moreover, we can easily find that 
%
\[ \left\{ 
\begin{array}{l}
\Phi^*(C_1^*, x_1^*;0) = 0, \
\dis\frac {\partial \Phi^*}{\partial \bar{C}_1}(C_1^*, x_1^*;0) = M(x_0), \
\dis\frac {\partial \Phi^*}{\partial \bar{x}_1}(C_1^*, x_1^*;0) = 0, 
\\[3ex]
\dis\Psi^*(C_1^*, x_1^*;0) = 0, \ 
\dis\frac {\partial \Psi^*}{\partial \bar{C}_1}(C_1^*, x_1^*;0) 
= ( x_0 h_v^-(v^*) +  (1 - x_0) h_v^+(v^*))/D,  
\\[3ex]
%
\dis\frac {\partial \Psi^*}{\partial \bar{x}_1}(C_1^*, x_1^*;0) = h^+(v^*) - h^-(v^*), 
\end{array} \right.
\]
%
which implies that 
%
\[
\begin{array}{l}
\dis\frac {\partial (\Phi^*, \Psi^*)}{\partial (\bar{C}_1, \bar{x}_1)}(C_1^*, x_1^*;0) 
 = M(x_0) (h^+(v^*) - h^-(v^*)) \  \ne \ 0. 
\end{array}
\]
%
Then, we can apply the implicit function theorem to $\Phi^*(\bar{C}_1, \bar{x}_1;\ep) = 0$ and  
$\Psi^*(\bar{C}_1, \bar{x}_1;\ep) = 0$, and find that there exist $\bar{C}_1 = \bar{C}_1(\ep)$
and $\bar{x}_1 = 
\bar{x}_1(\ep)$ for $\ep \in [0,\ep_3)$ satisfying $\bar{C}_1(0) = C_1^*, \ \bar{x}_1(0) = 
 x_1^*$,  
$\Phi^*(\bar{C}_1, \bar{x}_1;\ep) = 0$, and $\Psi^*(\bar{C}_1, \bar{x}_1;\ep) = 0$.

Substituting $C_1 = \bar{C}_1(\ep)$ and $x_1 = \bar{x}_1(\ep)$ into 
\eqref{b8}, \eqref{b9}, \eqref{b34} and \eqref{b35}, we obtain the following existence result: 

\begin{theo}\label{th2}
Suppose the assumptions (A1) and (A2).
For any $\ep \in (0, \ep_3)$, there exists a family of solutions $u^\ep(x) \in \pot{C}^2_\ep[0,1]$ of \eqref{b3} and \eqref{b3_1}. Furthermore, the following estimate holds:
\begin{equation*}
\begin{array}{l}
\dis \left|\left| u^\ep(x) - U^-(\frac x{x^*(\ep)};\ep;  \bar{C}_1(\ep), \bar{x}_1(\ep)) \right|\right|_
{C^2_{\ep,0}[0,x^*(\ep)]} \\[4ex]
+ \ \dis \left|\left| u^\ep(x) - U^+(\frac {x-x^*(\ep)}{1-x^*(\ep)};\ep; \bar{C}_1(\ep), \bar{x}_1(\ep)) \right|\right|_{C^2_{\ep,1}[x^*(\ep),1]} = o(\ep)
\end{array}
\end{equation*}
as $\ep \to 0$. 
\end{theo}


\begin{remark}
In Section \ref{S2.1}, we took $U_0^-(y) = h^-(C_0/D)$ in \eqref{b14} and $U_0^+(y) = h^+(C_0/D)$ in \eqref{b25} as a jump-up solution. If we take $U_0^-(y) = h^+(C_0/D)$ in \eqref{b14} and $U_0^+(y) = h^-(C_0/D)$ in \eqref{b25}, we can obtain a jump-down solution $u^\ep(x)$ at $x = x^*(\ep)$. 
\end{remark}


\begin{remark}
We can perform the matched asymptotic expansion of any order $k$, i.e., we can obtain 
a family of solutions $u^\ep(x) \in \pot{C}^2_\ep[0,1]$ of \eqref{b3} and \eqref{b3_1}
with the estimate
\begin{equation*}
\begin{array}{l}
\dis \left|\left| u^\ep(x) - U^-_k(\frac x{x^*(\ep)};\ep;  \bar{C}_k(\ep), \bar{x}_k(\ep)) \right|\right|_
{C^2_{\ep,0}[0,x^*(\ep)]} \\[4ex]
+ \ \dis \left|\left| u^\ep(x) - U^+_k(\frac {x-x^*(\ep)}{1-x^*(\ep)};\ep; \bar{C}_k(\ep), \bar{x}_k(\ep)) \right|\right|_{C^2_{\ep,1}[x^*(\ep),1]} = o(\ep^k)
\end{array}
\end{equation*}
as $\ep \to 0$, where $U^{-}_k$ and $U^+_k$ are the $k$th order approximate solutions 
given by equations similar
to \eqref{b22zz} and \eqref{any1}, respectively.
\end{remark}


By using the relation $v^\ep(x) = (C(\ep) - \ep^2 u^\ep(x))/D$, we obtain 

\begin{cor}\label{cor1} \rm
For sufficiently small $\ep >0$, \eqref{a1} has a family of  stationary solutions $(u^\ep(x), v^\ep(x))$ 
satisfying \eqref{a2xyz} with the following properties:

\no (i)
\begin{equation}\label{b101}
\dis\lim_{\ep \to 0} v^\ep(x) = V^*(x) 
\end{equation}
uniformly on $x \in [0, 1]$, where $V^*(x)$ is given by \eqref{a9}.

\no (ii)
For each $\sigma > 0$ with $ 0 < \sigma < \min(x^*, 1-x^*)$, 
\begin{equation}\label{b102}
\dis\lim_{\ep \to 0} u^\ep(x) = U^*(x) 
\end{equation}
uniformly on $x \in [0, x^* - \sigma] \cup [x^* + \sigma, 1]$,
where $U^*(x)$ is given by \eqref{a8}.

\no (iii) Let 
$\tilde{u}^\ep(\zeta) = u^\ep(x^*(\ep)+\ep \zeta)$ and 
$\tilde{v}^\ep(\zeta) = v^\ep(x^*(\ep)+\ep \zeta)$, where
$\zeta = (x - x^*(\ep))/\ep$ is the stretched coordinate around the $C^1$-matching point 
$x = x^*(\ep)$. Then, for each $\sigma > 0$, 
%
\[%begin{equation}\label{b102y}
\dis\lim_{\ep \to 0} \tilde{v}^\ep(\zeta) = v^*
\]%end{equation}
%
and
%
\[%begin{equation}\label{b102x}
\dis\lim_{\ep \to 0} \tilde{u}^\ep(\zeta) = Q(\zeta)
\]%end{equation}
%
uniformly on $\zeta \in [- \sigma, \sigma]$, where $Q(\zeta)$ satisfies
%
%
\begin{equation}\label{b103y}
\left \{
\begin{array}{l}
Q_{\zeta \zeta} + f( Q, v^*) = 0, \\[1ex]
Q(\pm \infty) = h^{\pm}(v^*), \ \ Q(0) = h^0(v^*). 
\end{array} 
\right.
\end{equation}
\end{cor}

{\bf Proof}. (i) Since $u^\ep(x)$ is bounded, it follows from \eqref{b8} and \eqref{b35x} that
$\lim_{\ep \to 0} v^\ep(x) = \lim_{\ep \to 0} (C(\ep) - \ep^2 u^\ep(x))/D = v^*$
uniformly on $x \in [0, 1]$. 

\no
(ii) It follows from \eqref{b9}, \eqref{b22zz}, \eqref{b35x} and \eqref{b48_0} that
$$
\lim_{\ep \to 0} U^-(\frac x{x^*(\ep)};\ep;  \bar{C}_1(\ep), \bar{x}_1(\ep)) = h^-(v^*)
$$
uniformly on $x \in [0, x^* - \sigma]$. Similarly, we see that
$$
\lim_{\ep \to 0} U^+(\frac {x-x^*(\ep)}{1-x^*(\ep)};\ep; \bar{C}_1(\ep), \bar{x}_1(\ep)) 
= h^+(v^*)
$$
uniformly on $x \in [x^* + \sigma, 1]$. Therefore, we have
\eqref{b102} by Theorem~\ref{th2}.

\no
(iii) 
It is clear that 
$\lim_{\ep \to 0} \tilde{v}^\ep(\zeta) = v^*$
uniformly on $\zeta \in [- \sigma, \sigma]$ because of the assertion (i).
When $\zeta \leq 0$, noting $x = x^*(\ep) + \ep \zeta$, $x= x^*(\ep)y$,
$z = (y-1)/\ep$ and \eqref{b35x}, we have
%
$$
\begin{array}{l}
\tilde{u}^\ep(\zeta)
 =  u^-( x^*(\ep) + \ep \zeta ; \ep, \bar{C}_1(\ep), \bar{x}_1(\ep))
= \tilde{u}^-(1 + \ep z ; \ep; \bar{C}_1(\ep), \bar{x}_1(\ep)) 
\\[1ex]
\ \ \ \ \ \ \ \, =  U^-( 1 + \ep z ; \ep, \bar{C}_1(\ep), \bar{x}_1(\ep)) + O(\ep)
= U_0^-(1) + \phi_0^-(z) + O(\ep) 
\\[1ex]
\ \ \ \ \ \ \ \, =  h^-(v^*) + \phi_0^-(\frac{\zeta}{x_0}) + O(\ep),
\end{array} 
$$ 
%
where $u^-$, $\tilde{u}^-$, $U^-$ and $\phi_0^-$ are given by \eqref{b34}, \eqref{b22_0},
\eqref{b22zz} and \eqref{b18}, respectively. On the other hand, applying the change of 
variable $z = \eta/x_0$ to \eqref{b18}, we find that 
$Q^-(\eta) := h^-(v^*) + \phi_0^-(\eta/x_0)$ satisfies 
%
\[%begin{equation}\label{b103p}
\left \{
\begin{array}{l}
Q^-_{\eta \eta} + f( Q^-, v^*) = 0, \\[1ex]
Q^-(- \infty) = h^{-}(v^*), \ \ Q^-(0) = h^0(v^*), \ \ 
Q^-_{\eta}(0) = \dot{\phi}^-_0(0)/x_0, 
\end{array} 
\right.
\]%end{equation}
%
where we choose $\alpha = h^0(v^*) \in (h^-(v^*), h^+(v^*))$. 
Similarly, when $\zeta \geq 0$, we have 
$\tilde{u}^\ep(\zeta) = h^+(v^*) + \phi_0^+(\zeta/(1-x_0)) + O(\ep)$, and 
find that $Q^+(\eta) := h^+(v^*) + \phi_0^+(\eta/(1-x_0))$
satisfies 
%
\[%begin{equation}\label{b103q}
\left \{
\begin{array}{l}
Q^+_{\eta \eta} + f( Q^+, v^*) = 0, \\[1ex]
Q^+(+ \infty) = h^{+}(v^*), \ \ Q^+(0) = h^0(v^*), \ \ 
Q^+_{\eta}(0) = \dot{\phi}^+_0(0)/(1-x_0).
\end{array} 
\right.
\]%end{equation}
%
We can obtain 
$$
Q(\eta) = 
\left \{
\begin{array}{l}
Q^-(\eta)  \ \ \ (\eta \leq 0) \\[1ex]
Q^+(\eta) \ \ \ (\eta > 0) 
\end{array} 
\right.
$$ 
satisfying the same equation as
\eqref{b103y} by the $C^1$-matching of 
$Q^-(\eta)$ and $Q^+(\eta)$ at $\eta = 0$ because 
$\dot{\phi}^-_0(0)/x_0  = \dot{\phi}^+_0(0) /(1-x_0)$ by $\Phi_0 = 0$, where
$\Phi_0$ is given by \eqref{b35zz}. 
Since we can identify $\zeta$ with $\eta$
in the limit of $\ep \to 0$, by using a similar argument in the proof of \cite[Lemma 1.1]{NF},
we see that 
$\lim_{\ep \to 0} \tilde{u}^\ep(\zeta) = Q(\zeta)$ 
uniformly on $\zeta \in [- \sigma, \sigma]$, where $Q(\zeta)$ satisfies \eqref{b103y}.
\Qed 



\section{Stability analysis of transition layer solutions} \label{stability} 
% Section 3

In this section, we perform the stability analysis of $(u^\ep(x), v^\ep(x))$ given by
Corollary~\ref{cor1} under a constraint derived from 
the conservation law \eqref{a2}.

We consider the linearized eigenvalue problem
%
\begin{equation}\label{c1}
\mathcal{L}^\ep \Phi^\ep = \lambda^\ep \Phi^\ep, \ \ \
%
\Phi^\ep = 
\left(
\begin{array}{c}
\varphi^\ep \\
\psi^\ep
\end{array}
\right) \in X, 
%
\ \ \
\mathcal{L}^\ep =
\left(
\begin{array}{cc}
L^\ep & f_v^\ep \\
- f_u^\ep & M^\ep
\end{array}
\right),
\end{equation}
%
under the Neumann boundary condition, where 
$f_u^\ep = f_u ( u^\ep(x), v^\ep(x) )$, 
$f_v^\ep = f_v ( u^\ep(x), v^\ep(x) )$, 
%
\[
L^\ep := \ep^2 \dis\frac{d^2}{dx^2} + f_u ( u^\ep(x), v^\ep(x) ), \ \ 
M^\ep := D \dis\frac{d^2}{dx^2} - f_v ( u^\ep(x), v^\ep(x) ),
\]
%
%
\begin{equation}\label{c2}
X = \{ \ (\varphi, \psi) \in L^2(0, 1) \times L^2(0, 1) \ | \ \int_0^1 (\varphi + \psi) dx = 0 \ \},
\end{equation}
%
and the domain $D(\mathcal{L}^\ep)$ is naturally defined. 
We note that the constrained condition in the definition $X$ is naturally 
derived from the conservation law \eqref{a2}.

In this section, we use the notations
%
\begin{equation}\label{c2xx}
f_u^*(x) = f_u( U^*(x), V^*(x)) \ \ 
\text{and} \ \ 
f_v^*(x) = f_v( U^*(x), V^*(x)),
\end{equation}
%
where $(U^*(x), V^*(x))$ is given by \eqref{a8} and \eqref{a9}. 
From the assumptions (A1) and (A3), we have
%
\begin{equation}\label{fa1}
f_u^*(x) < 0 \ \  \ \ ( 0 \leq x \leq 1)
\end{equation}
%
and
%
\begin{equation}\label{fa2}
f_v^*(x) - f_u^*(x) > 0 \ \  \ \ ( 0 \leq x \leq 1),
\end{equation}
%
respectively. Moreover, we denote the inner product in 
$L^2(0,1)$ by $\lg \, \cdot \, , \, \cdot \, \rg$.

\subsection{Preliminaries}   % Section 3.1 

We denote by $\mu_j^\ep$ $(j =0,1,2 \cdots )$ the eigenvalues of 
$L^\ep : L^2(0, 1) \to L^2(0, 1)$ under the Neumann boundary condition;
%
\[
\sigma(L^\ep) = \{ \mu_j^\ep \}_{j=0}^\infty,  \ \ \ 
\mu_0^\ep > \mu_1^\ep > \cdots > \mu_j^\ep \to - \infty \quad (j \to \infty).
\]

We note that 
$V^*(x) \equiv v^*$ satisfies $V^*_x(x^*) = 0$,
where $V^*(x)$ is given by \eqref{a9} and $V^*(x) = \lim_{\ep \to 0} v^\ep(x) $
by \eqref{b101}. 
Moreover, we note that $f(h^\pm(v^*), v^*) = 0$ by the assumption (A1).
Noting Corollary~\ref{cor1}, 
we see that the following propositions are true by
\cite[Proposition 5.1]{HS} and
\cite[Corollary 1.3, Lemma 1.4 and Lemma 2.3]{NF}. 

\begin{prop}\label{prop1c}
Under the assumptions (A1) and (A2), for sufficiently small $\ep > 0$, the following properties hold:

\no
{\rm (i)}  Let $\phi_0^\ep(x)$ be the $L^2$-normalized eigenfunction of $L^\ep$
corresponding to $\mu_0^\ep$. Then, 
%
\begin{equation}\label{c3}
\mu_0^\ep = O(e^{-C/\ep}) \ \  \text{as} \ \ \ep \to 0
\end{equation}
%
for some $C>0$, and
%
\begin{equation}\label{c4}
\lim_{\ep \to 0} \frac{1}{\sqrt{\ep}} \int_0^1 \phi_0^\ep(x) dx 
= \kappa^*(h^+(v^*) - h^-(v^*)),
\end{equation}
%
where 
%
\begin{equation}\label{c4x}
\kappa^* = \left( \int_{-\infty}^\infty Q_\zeta(\zeta)^2 d\zeta  \right)^{-1/2}
\end{equation}
%
and $Q=Q(\zeta)$ is the solution of \eqref{b103y}. 

%
\no
{\rm (ii)} 
for each $p \in H^1(0, 1)$, we have
%
\begin{equation}\label{c6}
\dis\lim_{\ep \to 0} \left\lg p, \left(\frac{\phi_0^\ep}{\sqrt{\ep}} \right) \! f_u^\ep \right\rg = 0
\end{equation}
%
and
%
\begin{equation}\label{c7}
\dis\lim_{\ep \to 0} \left\lg p, \left(\frac{\phi_0^\ep}{\sqrt{\ep}} \right) \! f_v^\ep \right\rg = 
 \kappa^* p(x^*)J'(v^*),
\end{equation}
%
where $\kappa^*$ and $J'(v^*)$ are given by \eqref{c4x}
and \eqref{a4x}, respectively.
\end{prop}


\begin{prop}\label{prop2c} 
Under the assumptions (A1) and (A2), for sufficiently small $\ep > 0$, the following properties hold:

\no
{\rm (i)}  There exist $\mu_* > 0$ independent of $\ep$
such that $ \mu_1^\ep <  -\mu_* < \mu_0^\ep$. Moreover,
$(L^\ep - \lambda )^{-1}P^\ep : L^2(0, 1) \to L^2(0, 1)$ is well-defined 
for any $\lambda \in \Lambda = 
\{ \, \lambda \in \C \  | \ \mbox{\rm Re} \lambda > - \mu_* \, \}$, and
%
\begin{equation}\label{c5z}
|| (L^\ep - \lambda )^{-1}P^\ep q ||_{L^2} \leq \dis\frac{1}{|\lambda - \underline{\mu}|} ||q||_{L^2} \ \  \text{for} \ q \in L^2(0,1)
\end{equation}
%
holds for any given $\underline{\mu} \in ( \mu_1^\ep, -\mu_*)$,
where $P^\ep$ is the orthonormal projection onto the orthogonal complement of $\phi_0^\ep$.

\no
{\rm (ii)} For $q \in L^\infty(0, 1)$ and $\lambda \in \Lambda$, 
%
\begin{equation}\label{c5}
\dis\lim_{\ep \to 0} (L^\ep - \lambda)^{-1} P^\ep q = \frac{ q(x)}{f_u^*(x) - \lambda}
\ \ \text{strongly in} \ L^2(0, 1).
\end{equation}
%
Moreover, the convergence in \eqref{c5}
is uniform with respect to $\lambda \in \Lambda$ and $q$ on a $H^1$-bounded set.
\end{prop}



\subsection{Properties of eigenvalues} \rm

Noting \eqref{fa1}, we define
%
\begin{equation}\label{c8}
\nu = \min\{ \mu_*  , \ \inf_{0 \leq x \leq 1} (- f_u^*(x) ) \} /2  > 0,
\end{equation}
%
where $\mu_*$ is given by Proposition~\ref{prop2c}(i), and 
$f_u^*(x)$ and $f_v^*(x)$ are given by \eqref{c2xx}. 
We consider the eigenvalues of \eqref{c1} in
$\Lambda_* = 
\{ \, \lambda \in \C \  | \ \mbox{\rm Re} \lambda > - \nu \, \}$.

Decomposing the first component of the eigenfuction of
$\mathcal{L}^\ep$ as $\varphi^\ep = a^\ep \phi_0^\ep + w^\ep$,
the eigenvalue problem \eqref{c1} is rewritten as

\begin{equation}\label{c9}
\left \{
\begin{array}{l}
a^\ep(\mu_0^\ep - \lambda^\ep) = - \lg \psi^\ep, f_v^\ep \phi^\ep_0 \rg  \\[2ex]
( L^\ep - \lambda^\ep ) w^\ep = - P^\ep(f_v^\ep \psi^\ep) \\[2ex]
( M^\ep - \lambda^\ep) \psi^\ep - f_u^\ep w^\ep = a^\ep f_u^\ep \phi^\ep_0,
\end{array} 
\right. 
\end{equation}

\no
where $(\mu_0^\ep, \phi_0^\ep(x))$ is the principal eigenpair of
$L^\ep$, $a^\ep \in \mbox{\bf C}$, $w^\ep$ satisfies $\lg w^\ep, 
\phi_0^\ep \rg =0$, and $P^\ep$ is the orthogonal projection
onto the orthogonal complement of $\phi_0^\ep$. 
We note that $(\varphi^\ep, \psi^\ep) = ( a^\ep \phi_0^\ep + w^\ep, \psi^\ep) \in X$, which implies 
that the constrained condition
%
\begin{equation}\label{c9x}
a^\ep \int_0^1 \phi_0^\ep dx + \int_0^1 w^\ep dx  +   \int_0^1 \psi^\ep dx = 0
\end{equation}
%
holds. This equation plays a key role to characterize a critical eigenvalue
which essentially determines the stability of the stationary solutions.
Noting $\lambda^\ep \in \Lambda_* \subset \Lambda$, it follows from Proposition~\ref{prop2c}(i) that

\begin{equation}\label{c10}
w^\ep = - ( L^\ep - \lambda^\ep )^{-1} P^\ep(f_v^\ep \psi^\ep).
\end{equation}

\no
Substituting \eqref{c10} into the third equation in \eqref{c9}, we have

\begin{equation}\label{c11}
( M^\ep - \lambda^\ep) \psi^\ep + f_u^\ep ( L^\ep - \lambda^\ep )^{-1} P^\ep(f_v^\ep \psi^\ep) = a^\ep f_u^\ep \phi^\ep_0.
\end{equation}

In what follows, we suppose $\lg \psi^\ep, \psi^\ep \rg = 1$. 
In fact, 
if $\psi^\ep \equiv 0$, then
$w^\ep \equiv 0$ and $a^\ep =0$ by \eqref{c4}, \eqref{c9x} and \eqref{c10}.


\begin{lemma}\label{lem1c}
Under the assumptions (A1) and (A2),
for each $p \in H^1(0, 1)$, we have
%
\begin{equation}\label{c12}
- \lg D \psi^\ep_{x}, p_x \rg + \lg - f_v^\ep \psi^\ep - \lambda^\ep \psi^\ep
 + \frac{f_u^\ep f_v^\ep }{ f_u^* - \lambda^\ep }\psi^\ep, p \rg = o(1)
\ \ \text{as} \ \ \ep \to 0.
\end{equation}
%
\end{lemma}

{\bf Proof}. 
Noting $|| \psi^\ep ||_{L^2} = 1$, we have
%
$$
\left| \int_0^1 \psi^\ep dx \right| \leq || \psi^\ep ||_{L^2} = 1.
$$
%
Moreover, noting $\lambda^\ep \in \Lambda_* \subset \Lambda$, it follows from \eqref{c5z} and
\eqref{c10} that
%
$$
 || w^\ep ||_{L^2}  
= || ( L^\ep - \lambda^\ep )^{-1} P^\ep(f_v^\ep \psi^\ep)||_{L^2} 
\leq \dis\frac{1}{|\lambda^\ep - \underline{\mu}|} ||f_v^\ep \psi^\ep||_{L^2}
$$
%
holds for any given $\underline{\mu} \in ( \mu_1^\ep, -\mu_*)$.
Since $\text{Re}\lambda^\ep > - \nu \geq -\mu_*/2$ by 
$\lambda^\ep \in \Lambda_*$, we have
%
$$
|| w^\ep ||_{L^2} \leq  \frac{2}{\mu_*} || f^\ep_v \psi^\ep ||_{L^2},
$$
%
which implies
%
$$
\left| \int_0^1 w^\ep dx \right| \leq  || w^\ep ||_{L^2} \leq \frac{2}{\mu_*} || f^\ep_v \psi^\ep ||_{L^2} 
\leq \frac{2}{\mu_*} || f^\ep_v ||_{L^\infty}.
$$
%
Since $||f^\ep_v||_{L^\infty}$ is uniformly bounded with respect to $\ep$, it follows from \eqref{c4} and \eqref{c9x} that
%
$$
\left| a^\ep \! \! \int_0^1 \phi_0^\ep dx \right| = \left| \int_0^1 w^\ep dx  +   \int_0^1 \psi^\ep dx \right| \leq 
1 + \frac{2}{\mu_*} || f^\ep_v ||_{L^\infty} 
$$
%
and $| a^\ep | \leq C/\sqrt{\ep}$ for sufficiently small $\ep$, where $C$ is a positive constant independent of $\ep$.
Thus, it follows from \eqref{c6} that 
%
$$
| \lg a^\ep f_u^\ep \phi^\ep_0, p \rg | = 
| a^\ep |  \sqrt{\ep} \, \Big{|} \Big{\lg} p, \left(\frac{\phi_0^\ep}{\sqrt{\ep}} \right) \! f_u^\ep \Big{\rg} \Big{|} = o(1) 
\ \ \text{as} \ \ \ep \to 0.
$$
%
Thus, by using integration by parts, we obtain \eqref{c12} due to \eqref{c5} and \eqref{c11}. \Qed


\begin{lemma}\label{lem2c}
Under the assumptions (A1) and (A2), there exists a positive constant
$R$ independent of $\ep$ such that
$|\lambda^\ep| \leq R$ holds.
\end{lemma}


{\bf Proof}. It follows from \eqref{c12} that
$$
- \lg D \psi^\ep_{x}, \psi^\ep_x \rg + \lg - f_v^\ep \psi^\ep - \lambda^\ep \psi^\ep
 + \frac{f_u^\ep f_v^\ep }{f_u^* - \lambda^\ep}\psi^\ep, \psi^\ep \rg = o(1)
\ \ \text{as} \ \ \ep \to 0.
$$
By using $\lg \psi^\ep, \psi^\ep \rg =1$, we have
%
$$
\lambda^\ep = - D \lg  \psi^\ep_x, \psi^\ep_x \rg
- \lg f_v^\ep \psi^\ep, \psi^\ep \rg 
+ \lg \dis\frac{f_u^\ep f_v^\ep }{f_u^* - \lambda^\ep}\psi^\ep, \psi^\ep \rg + o(1) \ \ \text{as} \ \ \ep \to 0.
$$
%
Hence, noting $\lambda^\ep \in \Lambda_*$,
$| \lg f_v^\ep \psi^\ep, \psi^\ep \rg | \leq  || f_v^\ep ||_{L^\infty} ||\psi^\ep ||_{L^2}^2$ and  $|| \psi^\ep ||_{L^2} = 1$, we have
$$
\begin{array}{l}
-\nu \ < \ \text{Re}\lambda^\ep = - D \lg  \psi^\ep_x, \psi^\ep_x \rg
- \lg f_v^\ep \psi^\ep, \psi^\ep \rg 
+ \text{Re} \lg \dis\frac{f_u^\ep f_v^\ep }{f_u^* - \lambda^\ep}\psi^\ep, \psi^\ep \rg + o(1) 
\\[3ex]
\ \ \ \ \ \ \ \ \ \ \ \ \ \ \ \ \ \,  \leq   \ || f_v^\ep ||_{L^\infty}  
+ \text{Re}\lg \dis\frac{f_u^\ep f_v^\ep }{f_u^* - \lambda^\ep}\psi^\ep, \psi^\ep \rg + o(1) \ \ \text{as} \ \ \ep \to 0
\end{array}
$$
%
and
%
$$
\text{Im}\lambda^\ep  =  \text{Im}\lg \dis\frac{f_u^\ep f_v^\ep }{f_u^* - \lambda^\ep}\psi^\ep, \psi^\ep \rg + o(1) \ \ \text{as} \ \ \ep \to 0.
$$
%
Here, we assume that $\lim_{\ep \to 0} |\lambda^\ep| = +\infty$. 
Since
$$
| f_u^* - \lambda^\ep| \geq |\lambda^\ep| - | f_u^*| \geq |\lambda^\ep| -
|| f_u^* ||_{L^\infty} > 0
$$
holds for sufficiently small $\ep$, noting $|| \psi^\ep ||_{L^2} = 1$, we have
$$
| \lg \dis\frac{f_u^\ep f_v^\ep }{f_u^* - \lambda^\ep}\psi^\ep, \psi^\ep \rg |
\leq \int_0^1 \dis\frac{|f_u^\ep | |f_v^\ep| }{|f_u^* - \lambda^\ep|} |\psi^\ep|^2 dx
\leq \dis\frac{ ||f_u^\ep ||_{L^\infty}  ||f_v^\ep||_{L^\infty}  }{ |\lambda^\ep| -
|| f_u^* ||_{L^\infty} } \to 0
$$
as $\ep \to 0$. This implies that $|\lambda^\ep|$ is bounded for sufficiently small 
$\ep$, which leads to a contradiction. Thus,
we see that $\lim_{\ep \to 0} |\lambda^\ep| < +\infty$ holds. \Qed \\


Under the assumptions (A1) and (A2), 
it follows from Lemma~\ref{lem2c} that there are no eigenvalues in $
\{ \, \lambda \in \C \ | \ \text{Re} \lambda > -\nu, \  |\lambda| > R \, \}$,
where $\nu$ is given by \eqref{c8}.
In what follows, we consider eigenvalues in 
$\Lambda_\delta = \{ \, \lambda \in \C \ | \ \text{Re} \lambda \geq -\delta, \ |\lambda| \leq R \, \}$, where $\delta$ 
is an arbitrarily given constant satisfying $0 < \delta \leq \nu$.
Noting that $\lambda^\ep$ is continuous in $\ep$ on a compact set 
$\Lambda_\delta$, we suppose that $\lim_{\ep \to 0} \lambda^\ep$ exists in $ \Lambda_\delta$ because we are interested in the behavior of eigenvalues which 
determine the stability of $(u^\ep(x), v^\ep(x))$.


\begin{theo}\label{th3} 
Under the assumptions (A1), (A2) and (A3), any eigenvalue 
$\lambda^\ep \in \Lambda_{\delta_1}$ must satisfy
$$ \dis\lim_{\ep \to 0} \text{\rm Re} \lambda^\ep \leq 0 $$
and 
$$ \dis\lim_{\ep \to 0} \text{\rm Re} \lambda^\ep < 0 \ \ 
\text{if} \ \ \dis\lim_{\ep \to 0} \text{\rm Im} \lambda^\ep \neq 0, $$ 
where $\delta_1 = \min\{ \nu, \dis\inf_{0 \leq x \leq 1 } \dis\frac{f_v^*(x) - f_u^*(x)}{2} \} > 0$.
\end{theo}

{\bf Proof}. 
We note that $\delta_1$ is well-defined by \eqref{fa2} and \eqref{c8}.
It follows from \eqref{c12} that
$$
- \lg D \psi^\ep_{x},  \psi^\ep_{x} \rg + \lg - f_v^\ep \psi^\ep - \lambda^\ep \psi^\ep
 + \frac{f_u^\ep f_v^\ep }{f_u^* - \lambda^\ep}\psi^\ep, \psi^\ep \rg = o(1)
\ \ \text{as} \ \ \ep \to 0,
$$
which implies
%
\[%begin{equation}\label{c13}
D \lg  \psi^\ep_x, \psi^\ep_x \rg  + \lambda^\ep \Big{\lg} 
\frac{\lambda^\ep + (f_v^\ep - f_u^\ep )}{\lambda^\ep - f_u^*}\psi^\ep, \psi^\ep \Big{\rg} = o(1)
\ \ \text{as} \ \ \ep \to 0.
\]%end{equation}

Let $\alpha^\ep = \text{Re} \lambda^\ep$ and $\beta^\ep = \text{Im} \lambda^\ep$. Then, we have
%
$$
\begin{array}{l}
\Big{\lg} \dis\frac{\lambda^\ep + (f_v^\ep - f_u^\ep )}{\lambda^\ep - f_u^*}\psi^\ep, \psi^\ep \Big{\rg}
=
\Big{\lg} \dis\frac{ (f_v^\ep - f_u^\ep ) + \alpha^\ep + i \beta^\ep}{(\alpha^\ep - f_u^*) + i \beta^\ep}\psi^\ep, \psi^\ep \Big{\rg} \\[3ex]
\ \ \ \ 
=
\Big{\lg} \dis\frac{ (\alpha^\ep - f_u^* )^2 +  f^\ep_v (\alpha^\ep - f_u^* ) + (\beta^\ep)^2
- i  f^\ep_v \beta^\ep}{(\alpha^\ep - f_u^*)^2 + (\beta^\ep)^2}\psi^\ep, \psi^\ep \Big{\rg} + o(1)
\end{array}
$$
%
as $\ep \to 0$ because $\lim_{\ep \to 0} f_u^\ep(x) = f_u^*(x)$ uniformly on 
$x \in [0, x^* - \sigma] \cup [x^* + \sigma, 1]$ for any 
$\sigma > 0$ with $ 0 < \sigma < \min(x^*, 1-x^*)$ due to Corollary~\ref{cor1}.
Let 
$$
K^\ep_1 :=  (\alpha^\ep - f_u^* )^2 +  f^\ep_v (\alpha^\ep - f_u^* ) + (\beta^\ep)^2 \ \
\text{and} \ \  K^\ep_2 := (\alpha^\ep - f_u^*)^2 + (\beta^\ep)^2. 
$$
%
Since $\alpha^\ep > -\delta_1$ and \eqref{c8} by $\lambda^\ep \in \Lambda_{\delta_1}$, when $\ep$ is sufficiently small, it follows from \eqref{fa1} that 
%
$$
\alpha^\ep - f_u^* > -  f_u^*/3 \geq \delta_1/3
$$
%
holds, which implies 
$K^\ep_2 \geq \delta_1^2/9 > 0$.
Similarly, when $\ep$ is sufficiently small, it follows from \eqref{fa2} that 
$$
\begin{array}{l}
 \alpha^\ep + (f_v^\ep - f_u^* ) =  \alpha^\ep - \{ - (f_v^\ep - f_u^*) \} 
\\[2ex]
\ \ \ \ \ = 
 \alpha^\ep - \{ - (f_v^* - f_u^*) \} + (f_v^\ep - f_v^*)
\\[2ex]
\ \ \ \ \  > (f_v^* - f_u^* )/3 \geq \delta_1/3
\end{array}
$$
on $x \in [0, x^* - \sigma] \cup [x^* + \sigma, 1]$ 
for any $\sigma > 0$ with $ 0 < \sigma < \min(x^*, 1-x^*)$
because 
$\lim_{\ep \to 0} f_v^\ep(x) = f_v^*(x)$ uniformly on 
$x \in [0, x^* - \sigma] \cup [x^* + \sigma, 1]$  due to Corollary~\ref{cor1}.
Therefore, we have 
$$
K^\ep_1 = (\alpha^\ep - f_u^* ) \{ \alpha^\ep + (f_v^\ep - f_u^* ) \} + (\beta^\ep)^2 \geq \delta_1^2/9 > 0
$$ 
on $x \in [0, x^* - \sigma] \cup [x^* + \sigma, 1]$ for sufficiently small $\ep$.
In addition, by Lemma~\ref{lem2c} and $|| f_v^\ep ||_{L^\infty} = O(1)$ as
$\ep \to 0$, we have
%
\begin{equation}\label{c13x}
0 < C_1 < \Big{\lg} \frac{K_1^\ep }{K_2^\ep}\psi^\ep, \psi^\ep \Big{\rg} < C_2
\end{equation}
%
for sufficiently small $\ep$, where $C_1$ and $C_2$ are independent of $\ep$.
Then, we have
%
$$
\begin{array}{l}
\lambda^\ep \Big{\lg} 
\dis\frac{\lambda^\ep + (f_v^\ep - f_u^\ep )}{\lambda^\ep - f_u^*}\psi^\ep, \psi^\ep \Big{\rg} 
= 
(\alpha^\ep + i \beta^\ep ) \Big{\lg} 
\frac{K_1^\ep - i f_v^\ep \beta^\ep}{K_2^\ep}\psi^\ep, \psi^\ep \Big{\rg} + o(1)
\\[3ex]
\ \ \ \ 
= \alpha^\ep \Big{\lg} \dis\frac{K_1^\ep }{K_2^\ep}\psi^\ep, \psi^\ep \Big{\rg} 
+  (\beta^\ep)^2 \Big{\lg}  \frac{f_v^\ep }{K_2^\ep}\psi^\ep, \psi^\ep \Big{\rg} 
\\[3ex]
\ \ \ \ \ \ \ \ \ \ \ 
+ i \beta^\ep \Big{\{} - \alpha^\ep \Big{\lg}  \dis\frac{f_v^\ep }{K_2^\ep}\psi^\ep, \psi^\ep \Big{\rg} 
+ \Big{\lg} \frac{K_1^\ep }{K_2^\ep}\psi^\ep, \psi^\ep \Big{\rg} \Big{\}} + o(1) 
\ \ \text{as} \ \ \ep \to 0,
\end{array}
$$
%
which implies
%
\[%begin{equation}\label{c14x}
D \lg  \psi^\ep_x, \psi^\ep_x \rg + \alpha^\ep \Big{\lg} \dis\frac{K_1^\ep }{K_2^\ep}\psi^\ep, \psi^\ep \Big{\rg} 
+  (\beta^\ep)^2 \Big{\lg}  \frac{f_v^\ep }{K_2^\ep}\psi^\ep, \psi^\ep \Big{\rg}  = o(1)
\]%end{equation}
%
and
%
\[%begin{equation}\label{c14y}
\beta^\ep \Big{\{} - \alpha^\ep \Big{\lg}  \dis\frac{f_v^\ep }{K_2^\ep}\psi^\ep, \psi^\ep \Big{\rg} 
+ \Big{\lg} \frac{K_1^\ep }{K_2^\ep}\psi^\ep, \psi^\ep \Big{\rg} \Big{\}} = o(1) 
\]%end{equation}
%
as $\ep \to 0$. Therefore, we have
%
\begin{equation}\label{c14}
\alpha^\ep D \lg  \psi^\ep_x, \psi^\ep_x \rg 
+ \{ (\alpha^\ep)^2 +  (\beta^\ep)^2 \} \Big{\lg} \dis\frac{K_1^\ep }{K_2^\ep}\psi^\ep, \psi^\ep \Big{\rg} 
= o(1)
\end{equation}
%
%
as $\ep \to 0$. 
Noting \eqref{c13x}, it follows from \eqref{c14} that
%
$$%\begin{equation}\label{c16}
\left( \alpha^\ep + \frac{D \lg  \psi^\ep_x, \psi^\ep_x \rg }{2 \big{\lg} (K_1^\ep/K_2^\ep) \psi^\ep, \psi^\ep \big{\rg} } 
\right)^2 + ( \beta^\ep )^{2} -
\left( \frac{D \lg  \psi^\ep_x, \psi^\ep_x \rg }{2  \big{\lg} (K_1^\ep/K_2^\ep) \psi^\ep, \psi^\ep\big{\rg} } \right)^2 = o(1)
$$%\end{equation}
%
as $\ep \to 0$. In the limit of $\ep \to 0$, this equation seemingly expresses
a circle included in the 
left half plain in $\mbox{\bf C}$, and the circle is tangential to the imaginary axis at the origin.
Thus, we obtain Theorem~\ref{th3}.  \Qed \\

\begin{theo}\label{th4}
Under the assumptions (A1), (A2) and (A3), the stationary solutions
$(u^\ep(x), v^\ep(x))$ are stable if $J'(v^*) > 0$. i.e., any eigenvalue $\lambda^\ep$
of the linearized operator of \eqref{a1} at $(u^\ep(x), v^\ep(x))$ in $X$ given by \eqref{c2}
must satisfy $\text{Re}\lambda^\ep < 0$ for sufficiently small $\ep  > 0$ if $J'(v^*) > 0$.
\end{theo}

{\bf Proof}.
By Theorem~\ref{th3}, it suffices to prove that 
any eigenvalue
$\lambda^\ep \in \Lambda_{\delta_1}$ satisfying
$\lim_{\ep \to 0} \lambda^\ep = 0$ must satisfy 
$\text{Re}\lambda^\ep < 0$ for sufficiently small $\ep  > 0$ if $J'(v^*) > 0$.
It follows from \eqref{c12} that
%
$$
\lg D \psi^*_{x}, p_x \rg = 0 \ \ \text{for} \  \forall p \in H^1(0,1),
$$
%
where $\psi^* = \lim_{\ep \to 0} \psi^\ep$.
Hence, noting $\lg \psi^\ep, \psi^\ep \rg = 1$, we have $\psi^* \equiv 1$. 
Moreover, by \eqref{c5} and \eqref{c10}, we have $w^* = - f^*_v/f^*_u$, where
$w^* = \lim_{\ep \to 0} w^\ep$. Therefore, it follows from 
\eqref{c4} and \eqref{c9x} that
%
$$
\begin{array}{rcl}
\dis\lim_{\ep \to 0} a^\ep \sqrt{\ep} & = & - \dis\lim_{\ep \to 0} 
\dis\frac{  \dis\int_0^1 w^\ep dx  +   \int_0^1 \psi^\ep dx }
{ \dis\frac{1}{\sqrt{\ep}} \int_0^1 \phi_0^\ep dx }
\\[5ex]
& = &
 \dis\frac{1 }
{ \kappa^* (h^+(v^*) - h^-(v^*))}  \int_0^1 \dis\frac{f_v^* - f_u^*}{f_u^*} dx.
\end{array}
$$
%
Thus, by \eqref{c3}, \eqref{c7} and the first equation of \eqref{c9}, we obtain
%
\begin{equation}\label{cri}
\lambda^\ep = \frac{1}{a^\ep} \lg \psi^\ep, f_v^\ep \phi^\ep_0 \rg + \mu_0^\ep = 
 \ep (\kappa^*)^2 \cdot \dis\frac{ h^+(v^*) - h^-(v^*) }{  \dis\int_0^1 \dis\frac{f_v^* - f_u^*}{f_u^*} dx } 
\cdot J'(v^*) + o(\ep)
\end{equation}
%
as $\ep \to 0$. 
Thus, by \eqref{fa1} and \eqref{fa2}, we see that
$\lambda^\ep$ satisfying
$\lim_{\ep \to 0} \lambda^\ep = 0$
must satisfy $\mbox{\rm Re} \lambda^\ep < 0$ for 
sufficiently small $\ep$ if $J'(v^*) > 0$.
\Qed


\begin{remark}\label{rem3y}
Notice that \eqref{cri} gives the characterization of an eigenvalue satisfying 
$\lim_{\ep \to 0} \lambda^\ep = 0$. This eigenvalue is called the {\it critical eigenvalue}
which essentially determines the stability of $(u^\ep(x), v^\ep(x))$.
Although our result suggests that the stationary solutions $(u^\ep(x), v^\ep(x))$ are unstable if $J'(v^*) < 0$, we cannot conclude that it is true
because we have not yet proved the existence of the critical eigenvalue in $\Lambda_{\delta_1}$.
In order to prove this existence, 
we must consider the solvability of \eqref{c11} 
with respect to $\psi^\ep$ 
when $\lambda^\ep = O(\ep)$ as $\ep \to 0$.
In the case of reaction-diffusion systems of FitzHugh-Nagumo type studied 
by \cite{HS,NF}, such solvability problems can be solved by 
the Lax-Milgram theorem \cite{NS} which is the most powerful and
standard tool for 
solving the linear elliptic PDEs.  
In contrast to their cases, to solve
our problem by the Lax-Milgram theorem, we need 
additional assumptions concerning 
the convergence orders of 
\eqref{c6} and \eqref{c5} in $\ep$, respectively.
However, such assumptions cannot be verified in many practical problems
including those of \cite{MJE1, MJE2}.
\end{remark}

\subsection{Examples}

In this subsection, we present two helpful examples for understanding our results;
we consider \eqref{a1} on an interval $0 < x < 1$ under the Neumann boundary condition,
where $0 < \ep \ll D$ and $f(u,v)$ is a smooth function with bistable nonlinearity.  

\begin{example} \rm

The first example is given by a simple cubic function
%
\begin{equation}\label{cubic}
f(u, v) = \alpha v - u(u-\beta)(u-1),
\end{equation}
%
where $\alpha$ and $\beta$ are constants satisfying $\alpha >0$ and $0 < \beta < 1$. 
\end{example}

When $f(u, v) = 0$, we have
%
\[
v = \frac{1}{\alpha}u(u-1)(u-\beta) =:g(u)
\ \ \
\text{and}
\ \ \ 
\frac{dv}{du} = - \frac{f_u}{\alpha},
\]
%
where $f_u = -3u^2 + 2(\beta+1)u - \beta$. It is easy to see that the assumptions (A1) and (A3) hold by  $f_v = \alpha$. The ODE $\dot{u} = f(u,v)$ is bistable in $u$ for each 
$v \in (\un{v}, {\ov{v}})$ with $\un{v} = g(h^+(\un{v}))$ and $\ov{v} = g(h^-(\ov{v}))$
as seen in Figure~\ref{example1}(a). Here, $h^+(\un{v}) = (1 + \beta + \sqrt{\rho(\beta)})/3$
and $h^-(\ov{v}) = (1 + \beta - \sqrt{\rho(\beta)})/3$
are the solutions of $f_u=0$, where $\rho(\beta) = 1 - \beta + \beta^2$.
%

We immediately find that 
$$%\begin{equation}
\begin{array}{l}
J'(v) =  \dis\int_{h^-(v)}^{h^+(v)} f_v(u, v)du =  \alpha (h^+(v) - h^-(v)) > 0
\end{array} 
$$%\end{equation}
%
for all $v \in I$. Moreover, we have
%
$$%\begin{equation}
J(0) = \int_{h^-(0)}^{h^+(0)} f(u, 0)du  = - \int_0^1 u(u-1)(u-\beta) du = \frac{1}{12} - \frac{\beta}{6} 
$$%\end{equation}
%
by $h^-(0) = 0$ and $h^+(0) = 1$. 
When $\beta = 1/2$, it is easy to see that the assumption (A2) 
holds with $J(0)= 0$ and $J'(0) > 0 $.
%
In addition, up to the leading order, we can explicitly give the expression of solutions with
a single internal transition layer
$$
u^\ep(x) = \dis \frac{1}{2} \left( h^+(v^*) + h^-(v^*) + (h^+(v^*) - h^-(v^*)) \tanh \left( \frac{x - x^*}{2 \sqrt{2} \ep} \right) \right) + O(\ep^2)
$$
and 
$$ 
v^\ep(x) = \dis v^* + O(\ep^2), 
$$
%
where the layer position $x^* = 1 - \xi$ is given by \eqref{a7}.
Figure~\ref{example1}(a) shows 
a stable stationary solution with a single internal transition layer when $\beta = 1/2$. 
%
\begin{figure}[ht!]
\centering
\includegraphics[width=14cm, height=12cm, bb=0 0 642 544]{./example1.png}
  \caption{Stable stationary solutions with a single internal transition layer when
the bistable nonlinearity is given by the cubic function (\ref{cubic}). 
  The parameter values are given by $\alpha = 0.2$ and $\beta = 0.5$ for (a) and 
  $\beta = 0.3$ for (b), respectively. Upper and lower panels show the graphs of $f(u,v) =0$ and profiles of single transition layer solutions, respectively. In the lower panels, the red and blue curves indicate the $u$- and $v$-components, respectively.
 The value of the conserved mass $\xi$ is given by $\xi = 0.35$ for both (a) and (b). 
 The position of a layer and the value of $v$-component are obtained 
as $(x^*, v^*) = (0.650, 0)$ for (a) and $(x^*, v^*) \approx (0.422, -0.164)$ for (b), as indicated by broken lines, respectively. The values of the diffusion coefficients $\ep$ and $D$ are given by $\ep = 0.01$ and $D=1.0$, respectively. 
  }
  \label{example1}
\end{figure}
%

When $\beta \neq 1/2$, we can factorize 
$f(u, \un{v})$ and $f(u, \ov{v})$ into 
$
f(u, \un{v})= (u - h^+(\un{v}))^2 (h^-(\un{v}) - u)
$
and 
$
f(u, \ov{v})= (u - h^-(\ov{v}))^2 (h^+(\ov{v}) - u),
$
%
respectively, where
%
$$ 
\displaystyle{ h^-(\un{v}) = \frac{1 + \beta - 2 \sqrt{\rho(\beta)} }{3} }
\ \ \text{and} \ \ 
\displaystyle{ h^+(\ov{v}) =  \frac{1 + \beta + 2 \sqrt{\rho(\beta)} }{3} }. 
$$
%
Then, we have
$$
J (\un{v})  =  \int_{h^-(\un{v})}^{h^+(\un{v})} f(u, \un{v})du  = - \frac{\rho(\beta)^2}{12} < 0
$$
and 
$$ 
J (\ov{v})  =  \int_{h^-(\ov{v})}^{h^+(\ov{v})} f(u, \ov{v})du =  \frac{\rho(\beta)^2}{12} > 0. 
$$
Therefore, we find that $J(v^*) = 0$ and $J'(v^*) > 0$ hold for some $v^* \in I$, which implies that the assumption (A2) holds for $\beta \neq 1/2$.
Figure \ref{example1}(b) shows a stable stationary solution with a single internal 
transition layer, in which the value of $v^*$ appears in $(\un{v}, 0)$ due to 
$\beta \neq 1/2$.


\begin{example} \rm
%
The second example is given by a Hill type function
%
\begin{equation}\label{mori}
f(u, v) = 
\displaystyle{
\left( \kappa + \frac{u^2}{1+u^2} \right) v - u,
}
\end{equation}
%
where $\kappa$ is a positive constant satisfying $0 < \kappa < 1/8$.
%
\end{example}

This model proposed by \cite{MJE2} has a bistable nonlinearity. In fact, 
we can numerically obtain the curve of $f(u,v) = 0$ for $\kappa = 0.067$, 
as shown in Figure~\ref{example2}(a). 
%
The ODE $\dot{u} = f(u, v)$  is bistable in $u$ for $v \in I = (\un{v}, \ov{v})$ with  
$$
\displaystyle{ \un{v} = \frac{1 + \kappa - \phi}{2 \kappa (1+\kappa) }  \sqrt{ \frac{\phi}{1+ \kappa} } }
\ \ \text{and} \ \  
\displaystyle{ \ov{v} = \frac{1 + \kappa - \psi}{2 \kappa (1+\kappa) }
 \sqrt{ \frac{\psi}{1+ \kappa} } },
$$
where 
$$
\displaystyle{  \phi(\kappa) = \frac{1 - 2 \kappa + \sqrt{1-8 \kappa} }{2}  }
\ \ \text{and} \ \  
\displaystyle{  \psi(\kappa) = \frac{1 - 2 \kappa - \sqrt{1-8 \kappa} }{2}  } .
$$
Moreover, with the aid of the software MATHEMATICA, 
we can obtain
$$
\displaystyle{ h^+(\un{v}) =  \sqrt{ \frac{\phi}{1+ \kappa} }}
\ \ \text{and} \ \ 
\displaystyle{ h^-(\un{v}) = \frac{1 + \kappa - \phi }{2 \sqrt{(1 + \kappa) \phi}} }
$$
and
$$
\displaystyle{ h^+(\ov{v}) = \frac{1 + \kappa - \psi }{2 \sqrt{(1 + \kappa) \psi}} }
\ \ \text{and} \ \ 
\displaystyle{ h^-(\ov{v}) =  \sqrt{ \frac{\psi}{1+ \kappa} }}.
$$
%
Furthermore, we have
\begin{eqnarray*}
\displaystyle{ \frac{d h^-(\un{v})}{d \kappa} } & = & 
\displaystyle{\frac{(1+\kappa + \phi)(\phi - (1+\kappa) \phi'(\kappa))}{4 (\sqrt{(1+\kappa) \phi})^3}} > 0
\end{eqnarray*}
%
for $0 < \kappa < 1/ 8$, which implies that $h^-(\un{v})$ is 
a monotone increasing function in $\kappa \in (0, 1/8)$.
Since $\lim_{\kappa \to 0}  h^-(\un{v}) = 0$ 
and $\lim_{\kappa \to 1/8}  h^-(\un{v}) = 1/ \sqrt{3}$, 
we have $0 < h^-(\un{v}) < 1/\sqrt{3}$. Similarly, we can show that $h^+(\ov{v})$ is a monotone decreasing function in $\kappa \in (0, 1/8)$ with $1/\sqrt{3} < h^+(\ov{v}) < +\infty $. 
Therefore, noting
$$
\displaystyle{f_u(u, v) = \frac{2 u v}{(1+u^2)^2}-1}
\ \ \text{and} \ \ 
\displaystyle{f_v(u, v) = \kappa + \frac{u^2}{1+u^2}},
$$
we see that the assumptions (A1) and (A3) hold.

Next, we check the assumption (A2). A direct calculation shows
%
$$
\begin{array}{rcl}
J'(v) & = &
\displaystyle{ \int_{h^-(v)}^{h^+(v)} f_v(u, v)du  } 
\\[2ex]
& = &
\dis{(h^+(v) - h^-(v)) 
\left( \kappa + 1 - \frac{\tan^{-1} (h^+(v)) - \tan^{-1} (h^-(v)) } {h^+(v) - h^-(v)} \right) }  
\\[3ex]
 & > &  \kappa (h^+(v) - h^-(v)),
\end{array}
$$
%
which implies that $J'(v) > 0$ holds for all $v \in I$. Moreover, we have 
%
$$
J (\un{v})  =  
\frac{(\kappa - 2 \phi)(1 + \kappa - \phi)^2 + 4 \phi^2 (1 - \phi) }{8 \phi \kappa (1 + \kappa) }
- \frac{1 + \kappa - \phi}{2 \kappa (1 + \kappa)} \sqrt{ \frac{\phi}{1 + \kappa}  }
\cdot K_1
$$
and 
$$ 
J (\ov{v})  =  
- \frac{(\kappa - 2 \psi)(1 + \kappa - \psi)^2 + 4 \psi^2 (1 - \psi) }{8 \psi \kappa (1 + \kappa) } 
  +   \frac{1 + \kappa - \psi}{2 \kappa (1 + \kappa)} \sqrt{ \frac{\psi}{1 + \kappa}  }
\cdot K_2,
$$
%
where
%
$$
K_1 = 
\tan^{-1} \left(  \sqrt{ \frac{\phi}{1 + \kappa} } \right) - 
\tan^{-1} \left(  \frac{1 + \kappa - \phi}{2 (1 + \kappa) }  \sqrt{ \frac{1 + \kappa}{\phi} } \right)   
$$
%
and
%
$$ 
K_2 = 
\tan^{-1} \left(  \sqrt{ \frac{\psi}{1 + \kappa} }  \right) -
\tan^{-1} \left(  \frac{1 + \kappa - \psi}{2 (1 + \kappa) }  \sqrt{ \frac{1 + \kappa}{\psi} } \right). 
$$
%
As shown in Figure~\ref{example2}(c), we see that 
$J (\un{v})$ and $J (\ov{v})$ are continuous functions 
in $\kappa$, and that
$J (\un{v}) < 0$ and $J (\ov{v}) > 0$ for any $\kappa \in (0, 1/8)$.
In addition, we have
$\lim_{\kappa \to 1/8} J (\un{v}) =  \lim_{\kappa \to 1/8} J (\ov{v}) = 0$, 
$\lim_{\kappa \to 0} J (\un{v}) = (3 - \pi)/2 < 0 $ and $\lim_{\kappa \to 0} J (\ov{v}) = + \infty$. 
Therefore, we find that $J(v^*) = 0$ and $J'(v^*) > 0$ hold for some $v^* \in I$, which implies
that the assumption (A2) holds.

Figure~\ref{example2}(b) shows a stable stationary solution with a single internal 
transition layer, in which 
the layer position $x^*$ given by \eqref{a7} is numerically computed.

%
\begin{figure}[ht!]
 \centering
\includegraphics[width=12cm, bb=0 0 541 336]{./example2.png}
  \caption {Stable stationary solutions with a single internal transition layer when
the bistable nonlinearity is given by the Hill type function \eqref{mori}. 
    Panels (a) and (b) show the graph of $f(u,v) =0$ and typical profiles 
 of a single transition layer solution, respectively. The parameter values are given by $\kappa = 0.067$. In (b), the red and blue curves indicate the $u$- and $v$-components, respectively. The value of the conserved mass $\xi$ is given by $\xi = 2.3$.
 The position of a layer and the value of $v$-component are obtained 
as $(x^*, v^*) \approx (0.660, 1.802)$ as indicated by broken lines. The values of the diffusion coefficients $\ep$ and $D$ are given by $\ep = 0.01$ and $D=1.0$, respectively. 
 (c) The graphs of $J (\un{v})$ and $J (\ov{v})$ which are continuous
functions in $\kappa$.  
  }
  \label{example2}
\end{figure}
 
\section{Concluding remarks} \label{conclude}

In this paper, we show that mass-conserving reaction-diffusion systems with 
bistable nonlinearity can have stable stationary solutions 
with a single internal transition layer under general assumptions.
Our approach is based on the singular perturbation method by \cite{HS, I, MTH, NF}.
In spite of the complication that we cannot apply 
the Lax-Milgram theorem to the singular limit eigenvalue problem (SLEP) concerning
the stability of the stationary solutions, 
we can give the precise characterization of a critical eigenvalue 
which essentially determines the stability 
due to a natural constrained condition derived from the conservation law.
Consequently, we can 
give a rigorous proof of the stability of the stationary solutions under general 
assumptions. 
Our result provides a basic information for studying the bifurcation structure of
mass-conserving reaction-diffusion systems with bistable nonlinearity
\cite{MKNTY, MJE2}.
Moreover, we emphasize that our approach can be applicable to singular perturbation problems with higher spatial dimension as seen in 
reaction-diffusion systems with bistable nonlinearity of FitzHugh-Nagumo 
type \cite{SS1, SS2}.

Our characterization of the critical eigenvalue suggests that the stationary solutions 
with a single internal transition layer are unstable if $J'(v^*) < 0$.
That is, the sign of $J'(v^*)$ determines
the stability/instability of the stationary solutions; 
it is a conjecture to be investigated for further studies.
Since the spatial dimension of our problem is one, 
we can apply a totally different method by \cite{I}
for studying the stability of the stationary solutions, which 
is substantially based on the ideas of the Evans function theory. In this case, 
we also encounter a problem related to that in Remark~\ref{rem3y}, which will require a delicate mathematical analysis. 
At present, we can prove that this conjecture is true if we replace the assumption (A3) 
in Assumption~\ref{ass1} by a somewhat stronger one.


We consider mass-conserving reaction-diffusion systems with bistable nonlinearity,
in which the ODE $u_t = f(u, v)$ 
obtained by dropping the diffusion term is bistable in $u$ for each fixed $v$. 
They are related to the wave-pinning phenomena in cell division and 
differentiation \cite{MJE1,MJE2}. 
On the other hand, 
we can consider another mass-conserving reaction-diffusion systems with bistable nonlinearity,
in which the ODE $v_t = - f(u, v)$ is bistable in $v$ for each fixed $u$. 
It was shown in \cite{KIE, OkS} that these systems are related to 
cell polarity oscillations which cause the reversal of cell polarity. 
It is expected that mathematical analyses for mass-conserving reaction-diffusion systems with 
bistable nonlinearity can provide a useful viewpoint for 
understanding cell polarity formation which plays a key role in cell division and differentiation.



\vspace{1cm}

{\bf Acknowledgments.} 
%The authors would like to express their appreciation to 
%the referees for their useful suggestions and comments, which have improved the 
%original manuscript.
The second and third authors were supported 
in part by the JSPS KAKENHI Grant Numbers JP23K03209 and JP19K03618, respectively.

\vspace{1cm}


\begin{thebibliography}{99}


\bibitem{BJF} F. Brauns, J. Jalatek, E. Frey, 
Phase-Space Geometry of Mass-Conserving Reaction-Diffusion Dynamics, 
Phys. Rev. X {\bf 10}  (2020) 041036.

\bibitem{CP} J. Carr, R. L. Pego,
Invariant manifolds for metastable patterns in $u_t = \ep^2 u_{xx} - f(u)$,
Proc. Roy. Soc. Edinburgh {\bf 116A} (1990), 133-160.

\bibitem{CMS} J. L. Chern, Y. Morita, and T. T. Shieh,
Asymptotic behavior of equilibrium states of reaction-diffusion systems 
with mass conservation,
J. Diff. Eqns. {\bf 264} (2018),  550--574.

\bibitem{ET}
S.-I. Ei, S.-Y. Tzeng,
Spike solutions for a mass conservation reaction-diffusion system,
DCDS {\bf 40} (2020), 3357-3374.

\bibitem{HF} J. K. Hale, G. Fusco,
Slow-motion manifolds, dormant instability, and singular perturbations,
J. Dyns. Diff. Eqns {\bf 1} (1989) 75-94.

\bibitem{HS}
J. K. Hale, K. Sakamoto,
A Lyapunov-Schmidt method for transition layers in reaction-diffusion 
systems,
Hiroshima Math. J., {\bf 35} (2005), 205-249.

\bibitem{I} H. Ikeda, 
Stability characteristics of transition layer solutions,
J. Dyns. Diff. Eqns {\bf 5} (1993) 625-671.

\bibitem{Is} S. Ishihara, M. Otsuji, A. Mochizuki, 
Transient and steady state of mass-conserved reaction-diffusion systems,
Phys. Rev. E {\bf 75} (2007), 015203.  


\bibitem{JM} S. Jimbo, Y. Morita,  
Lyapunov function and spectrum comparison for a reaction-diffusion system with mass conservation, 
J. Diff. Eqns {\bf 255} (2013) 1657-1683. 

\bibitem{KO} K. Kawasaki, T. Ohta,
Kink dynamics in one-dimensional nonlinear systems,
Physica A {\bf 116} (1982) 573-593.

\bibitem{KIE} 
M. Kuwamura, H. Izuhara, S.-I. Ei, 
Oscillations and Bifurcation Structure of Reaction-Diffusion Model for Cell Polarity Formation,
J. Math. Biol. {\bf 84} (2022), article no.22. 

\bibitem{EKS} 
M. Kuwamura, S.-S. Lee, S.-I. Ei, 
Dynamics of localized unimodal patterns in reaction-diffusion systems for cell polarization by extracellular signaling,
SIAM J. Appl. Math. {\bf 78} (2018), 3238-3257.

\bibitem{LS} 
E. Latos, T. Suzuki,
Global dynamics of a reaction-diffusion system with mass conservation,
J. Math. Anal. Appl. {\bf 411} (2014), 107--118.


\bibitem{MTH} M. Mimura, M. Tabata, Y. Hosono,
Multiple solutions of two point boundary value problems of Neumann type with a small parameter,
SIAM J. Math. Anal. {\bf 11} (1981), 613-631.

\bibitem{MKNTY}
T. Mori, K. Kuto, M. Nagayama, T. Tsujikawa, S. Yotsutani,
Global bifurcation sheet and diagrams of wave-pinning in a reaction-diffusion model 
for cell polarization,
Dynamical Systems and Differential Equations, AIMS Proceedings 2015,
Proceedings of the 10th AIMS International Conference (Madrid, Spain), 861-877.

\bibitem{MJE1} Y. Mori, A. Jilkine, L. Edelstein-Keshet,
Wave-pinning and cell polarity from a bistable reaction-diffusion system,
Biophys. J.  {\bf 94} (2008), 3684-3697. 

\bibitem{MJE2} Y. Mori, A. Jilkine, L. Edelstein-Keshet,
Asymptotic and bifurcation analysis of wave-pinning in a reaction-diffusion model 
for cell polarization,
SIAM J. Appl. Math {\bf 71} (2011), 1401-1427. 

\bibitem{MO} Y. Morita, T. Ogawa, 
Stability and bifurcation of nonconstant solutions to a reaction-diffusion system with conservation of mass, Nonlinearity {\bf 23} (2010) 1387-1411.  

\bibitem{NS} A. W. Naylor and G. R. Sell, 
Linear Operator Theory in Engineering and Science,
Springer-Verlag, New York, 1989.

\bibitem{NF} Y. Nishiura, H. Fujii, 
Stability of singularly perturbed solutions to systems of reaction-diffusion equations,
SIAM J. Math. Anal. {\bf 18} (1987), 1726-1770.

%\bibitem{NS}
%Y. Nishiura and H. Suzuki,
%Higher dimensional SLEP equation and applications to morphological stability in 
%polymer problems, 
%SIAM J. Math. Anal. {\bf 36} (2004), pp. 916-966.

\bibitem{OkS} T. Okuda Sakamoto,
Hopf bifurcation in a reaction-diffusion system with conservation of mass,
Nonlinearity {\bf 26} (2013), 2027-2049.

\bibitem{Ot} M. Otsuji, S. Ishihara, C. Co, K. Kaibuchi, A. Mochizuki, S. Kuroda,  
{A mass conserved reaction-diffusion system captures properties of cell polarity}, 
PLoS Comp. Biol. {\bf 3} (2007) e108. 

\bibitem{SS1}
K. Sakamoto, H. Suzuki, Spherically symmetric internal layers for activator-inhibitor systems. I. Existence by a Lyapunov-Schmidt reduction,
J. Diff. Eqns {\bf 204} (2004), 56-92. 

\bibitem{SS2}
K. Sakamoto, H. Suzuki,
Spherically symmetric internal layers for activator-inhibitor systems. II. Stability and symmetry breaking bifurcations,
J. Diff. Eqns {\bf 204} (2004), 93-122.


\end{thebibliography}

\end{document}


