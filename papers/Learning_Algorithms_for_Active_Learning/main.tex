%%%%%%%%%%%%%%%%%%%%%%%%%%%%%%%%%%%%%%%%%%%%%%%%%%%%%%%%%%%%%%%%%%
%%%%%%%% ICML 2016 EXAMPLE LATEX SUBMISSION FILE %%%%%%%%%%%%%%%%%
%%%%%%%%%%%%%%%%%%%%%%%%%%%%%%%%%%%%%%%%%%%%%%%%%%%%%%%%%%%%%%%%%%

% Use the following line _only_ if you're still using LaTeX 2.09.
%\documentstyle[icml2016,epsf,natbib]{article}
% If you rely on Latex2e packages, like most moden people use this:
\documentclass{article}

% use Times
\usepackage{times}
% For figures
\usepackage{graphicx} % more modern
%\usepackage{epsfig} % less modern
\usepackage{subfigure} 

% For citations
\usepackage{natbib}

% For algorithms
\usepackage{multirow}
\usepackage{algorithm}
\usepackage{algorithmic}
% \usepackage[ruled,vlined,linesnumbered]{algorithm2e}


% As of 2011, we use the hyperref package to produce hyperlinks in the
% resulting PDF.  If this breaks your system, please commend out the
% following usepackage line and replace \usepackage{icml2016} with
% \usepackage[nohyperref]{icml2016} above.
\usepackage{hyperref}

% Packages hyperref and algorithmic misbehave sometimes.  We can fix
% this with the following command.
\newcommand{\theHalgorithm}{\arabic{algorithm}}

% Employ the following version of the ``usepackage'' statement for
% submitting the draft version of the paper for review.  This will set
% the note in the first column to ``Under review.  Do not distribute.''
%\usepackage{icml2017}

% Employ this version of the ``usepackage'' statement after the paper has
% been accepted, when creating the final version.  This will set the
% note in the first column to ``Proceedings of the...''
\usepackage[accepted]{icml2017}


\usepackage{algorithm}

\usepackage{latexsym}
\usepackage[utf8]{inputenc} % allow utf-8 input
\usepackage[T1]{fontenc}   % use 8-bit T1 fonts
\usepackage{hyperref}       % hyperlinks
\usepackage{url}            % simple URL typesetting
\usepackage{booktabs}       % professional-quality tables
\usepackage{amsfonts}       % blackboard math symbols
\usepackage{nicefrac}       % compact symbols for 1/2, etc.
\usepackage{microtype}      % microtypography
\usepackage[centertags]{amsmath}
\usepackage{amssymb}
\usepackage{graphicx}
\usepackage{tabularx}
\usepackage{subfigure}
\usepackage{paralist}
\usepackage[inline,shortlabels]{enumitem}

\def\x{\times}
\def\S{\mathbf{S}}
\def\H{\mathbf{H}}
\renewcommand{\b}[1]{\mathbf{#1}}

% mathcal shortcuts
\newcommand{\eye}{\mathbf{I}}
\newcommand{\CC}{\mathcal{C}}
\newcommand{\DD}{\mathcal{D}}
\newcommand{\FF}{\mathcal{F}}
\newcommand{\GG}{\mathcal{G}}
\newcommand{\HH}{\mathcal{H}}
\newcommand{\II}{\mathcal{I}}
\newcommand{\KK}{\mathcal{K}}
\newcommand{\LL}{\mathcal{L}}
\newcommand{\MM}{\mathcal{M}}
\newcommand{\NN}{\mathcal{N}}
\newcommand{\OO}{\mathcal{O}}
\newcommand{\PP}{\mathcal{P}}
\newcommand{\QQ}{\mathcal{Q}}
\newcommand{\RR}{\mathcal{R}}
\newcommand{\SSS}{\mathcal{S}}
\newcommand{\TT}{\mathcal{T}}
\newcommand{\VV}{\mathcal{V}}
\newcommand{\WW}{\mathcal{W}}
\newcommand{\XX}{\mathcal{X}}
\newcommand{\YY}{\mathcal{Y}}
\newcommand{\ZZ}{\mathcal{Z}}
\newcommand{\cev}[1]{\reflectbox{\ensuremath{\vec{\reflectbox{\ensuremath{#1}}}}}}

% nicely formatting math operators
\DeclareMathOperator*{\argmin}{arg\,min}
\DeclareMathOperator*{\argmax}{arg\,max}
\DeclareMathOperator*{\expect}{\mathbb{E}}
\DeclareMathOperator*{\minimize}{minimize}
\DeclareMathOperator*{\maximize}{maximize}
\DeclareMathOperator*{\xent}{xent}
\DeclareMathOperator*{\ent}{ent}
\DeclareMathOperator*{\softmax}{softmax}
\DeclareMathOperator*{\KL}{KL}
\DeclareMathOperator*{\lrelu}{lrelu}
\DeclareMathOperator*{\relu}{relu}
\DeclareMathOperator*{\conv}{conv}


% The \icmltitle you define below is probably too long as a header.
% Therefore, a short form for the running title is supplied here:
\icmltitlerunning{Learning Algorithms for Active Learning}

\begin{document} 

\twocolumn[
\icmltitle{Learning Algorithms for Active Learning}

% It is OKAY to include author information, even for blind
% submissions: the style file will automatically remove it for you
% unless you've provided the [accepted] option to the icml2017
% package.

% list of affiliations. the first argument should be a (short)
% identifier you will use later to specify author affiliations
% Academic affiliations should list Department, University, City, Region, Country
% Industry affiliations should list Company, City, Region, Country

% you can specify symbols, otherwise they are numbered in order
% ideally, you should not use this facility. affiliations will be numbered
% in order of appearance and this is the preferred way.
\icmlsetsymbol{equal}{*}

\begin{icmlauthorlist}
\icmlauthor{Philip Bachman}{equal,maluuba}
\icmlauthor{Alessandro Sordoni}{equal,maluuba}
\icmlauthor{Adam Trischler}{maluuba}
\end{icmlauthorlist}

\icmlaffiliation{maluuba}{Microsoft Maluuba, Montreal, Canada}

\icmlcorrespondingauthor{P.~ Bachman}{phbachma@microsoft.com}
\icmlcorrespondingauthor{A.~ Sordoni}{alsordon@microsoft.com}

% You may provide any keywords that you 
% find helpful for describing your paper; these are used to populate 
% the "keywords" metadata in the PDF but will not be shown in the document
\icmlkeywords{boring formatting information, machine learning, ICML}

\vskip 0.3in
]

% this must go after the closing bracket ] following \twocolumn[ ...

% This command actually creates the footnote in the first column
% listing the affiliations and the copyright notice.
% The command takes one argument, which is text to display at the start of the footnote.
% The \icmlEqualContribution command is standard text for equal contribution.
% Remove it (just {}) if you do not need this facility.

%\printAffiliationsAndNotice{}  % leave blank if no need to mention equal contribution
\printAffiliationsAndNotice{\icmlEqualContribution} % otherwise use the standard text.

\begin{abstract}
We introduce a model that learns active learning algorithms via metalearning. For a distribution of related tasks, our model jointly learns: a data representation, an item selection heuristic, and a method for constructing prediction functions from labeled training sets. Our model uses the item selection heuristic to gather labeled training sets from which to construct prediction functions. Using the Omniglot and MovieLens datasets, we test our model in synthetic and practical settings.
\end{abstract}


\section{Introduction}
\label{sec:intro}
\section{Introduction}  \label{sec:introduction}

\newcommand\inexpIntro[3]{#1?(#2,#3).}
\newcommand\rinexpIntro[3]{*#1?(#2,#3).}
\newcommand\outexpIntro[3]{#1!(#2,#3).}
\newcommand\outatomIntro[3]{#1!(#2,#3)}

We propose a fully automated method for proving termination of \(\pi\)-calculus processes.
Although there have been a lot of studies on termination analysis for the \(\pi\)-calculus
and related calculi~\cite{Deng06IC,Demangeon07,SangiorgiTermination,KobayashiHybrid,Yoshida04IC,DBLP:journals/jlp/DemangeonHS10,Venet98SAS}, most of them have been rather theoretical,
and there have been surprisingly little efforts in developing  fully automated termination
verification methods and tools based on them. To our knowledge,
Kobayashi's \typical{}~\cite{TyPiCal,KobayashiHybrid} is the only exception that
can prove termination of \(\pi\)-calculus processes (extended with natural numbers)
fully automatically, but its termination analysis is quite limited (see Section~\ref{sec:relatedwork}).

Our method is based on a reduction to termination analysis for sequential programs:
we translate a \(\pi\)-calculus process \(P\) to a sequential program \(S_P\), so that
if \(S_P\) is terminating, so is \(P\). The reduction allows us to use
powerful, mature methods and tools
for termination analysis of sequential programs~\cite{heizmann2016ultimate,freqterm,DBLP:conf/lics/PodelskiR04,Kuwahara2014Termination,DBLP:journals/cacm/CookPR11}.

The idea of the translation is to convert a chain of communications on replicated input
channels to a chain of recursive function calls of the target sequential program.
Let us consider the following Fibonacci process:
\begin{align*}
    & \rinexpIntro{\fib}{n}{r}
        \ifexp{n<2}{ \soutatom{r}{1} \\ &\quad}
                   { \nuexp{s_1} \nuexp{s_2} (\outatomIntro{\fib}{n-1}{s_1} \PAR \outatomIntro{\fib}{n-2}{s_2} \PAR \sinexp{s_1}{x}\sinexp{s_2}{y}\soutatom{r}{x+y}) \\}
    & \PAR \outatomIntro{\fib}{m}{r}
\end{align*}
Here, the process
$\rinexpIntro{\fib}{n}{r} \ldots$ is a function server that computes the \(n\)-th Fibonacci number
in parallel and returns the result to \(r\),
and $\outatom{\fib}{m}{r}$ sends a request for computing the \(m\)-th Fibonacci number;
those who are not familiar with the syntax of the \(\pi\)-calculus may wish to consult
Section~\ref{sec:targetlanguage} first.
To prove that the process above is terminating for any integer \(m\),
it suffices to show that there is no infinite chain of communications on $\fib$:
\[
    \fib(m,r) \to \fib(m_1,r_1) \to \fib(m_2,r_2) \to \cdots.
\]
We convert the process above to the following program:\footnote{The actual translation
  given later is a little more complex.}
\begin{verbatim}
 let rec fib(n) = if n<2 then () else (fib(n-1) [] fib(n-2)) in
 fib(m)
\end{verbatim}
Here, \texttt{[]} represents the non-deterministic choice.
Note that, although the calculation of Fibonacci numbers is not preserved,
for each chain of communications on \texttt{fib}, there is a corresponding
sequence of recursive calls:
\[
\mathtt{fib}(m) \to \mathtt{fib}(m_1) \to \mathtt{fib}(m_2) \to \cdots.
\]
Thus, the termination of the sequential program above implies the termination of
the original process.
As shown in the example above, (i) each communication on a replicated input channel
is converted to a function call, (ii) each communication on a non-replicated input
channel is just removed (or, in the actual translation, replaced by a call of
a trivial function defined by \(f(\seq{x})=(\,)\)), and (iii) parallel composition
is replaced by a non-deterministic choice.
We formalize the translation outlined above and prove its correctness.

The basic translation sketched above sometimes loses too much information.
For example, consider the following process:
\begin{align*}
    & \rinexpIntro{\pre}{n}{r} \soutatom{r}{n-1} \\
    & \PAR \rinexpIntro{f}{n}{r} \ifexp{n<0}{ \soutatom{r}{1} }
                                       { \nuexp{s} (\outatomIntro{\pre}{n}{s} \PAR \sinexp{s}{x}\outatomIntro{f}{x}{r}) } \\
    & \PAR \outatomIntro{f}{m}{r}
\end{align*}
The translation sketched above would yield:
\begin{verbatim}
  let pred(n) = n-1 in
  let rec f(n) = if n<0 then () else (pred(n) [] f(*)) in
  f(m)
\end{verbatim}
Here, \texttt{*} represents a non-deterministic integer: since we have removed
the input $\sinatom{s}{x}$, we do not have information about the value of \( x \).
As a result, the sequential program above is non-terminating, although the original
process is terminating.
To remedy this problem, we also refine the basic translation above by using a refinement
type system for the \(\pi\)-calculus. Using the refinement type system,
we can infer that the value of \(x\) in the original process is less than \(n\),
so that we can refine the definition of \texttt{f} to:
\begin{verbatim}
 let rec f(n) = ... else (pred(n) [] let x=* in assume(x<n);f(x))
\end{verbatim}
The target program is now terminating, from which
we can deduce that the original process is also terminating.
We have implemented an automated tool based on the refined translation above.

The contributions of this paper are summarized as follows.
\begin{itemize}
\item The formalization of the basic translation from the \(\pi\)-calculus
  (extended with integers) to sequential programs, and a proof of its correctness.
\item The formalization of a refined translation based on a refinement type system.
\item An implementation of the refined translation, including automated refinement type
  inference based on CHC solving, and experiments to evaluate the effectiveness of
  our method.
\end{itemize}

The rest of this paper is structured as follows.
Section~\ref{sec:targetlanguage} introduces the source and target languages
of our translation.
Section~\ref{sec:approach} 
formalizes the basic translation, and proves its correctness.
Section~\ref{sec:refinement} refines the basic translation by using a refinement type system.
Section~\ref{sec:implementation} reports an implementation and experiments.
Section~\ref{sec:relatedwork} discusses related work,
and Section~\ref{sec:conclusion} concludes the paper.


\section{Related Work}
\label{sec:related}
%!TEX root = main.tex

% Active learning is motivated by the idea that a learning algorithm can perform better while training on fewer labeled data if it can choose the data on which it trains~\citep{cohn1996active}.
Various heuristics have been proposed to guide the selection of which examples to label during active learning~\cite{settles2010active}. For instance, \citet{lewis1994sequential} and \citet{tong2001support} developed policies based on the confidence of the classifier, while \citet{gilad2005query} used the disagreement of a \emph{committee} of classifiers. \citet{houlsby2011bayesian} presented an approach based on Bayesian information theory,
in which examples are selected in order to maximally reduce the entropy of the posterior distribution over classifier parameters. 
%Heuristic design is typically (and reasonably) based on balancing computational tractability against closeness to an optimal heuristic.

% An ideal selection heuristic would closely approximate the oracle policy, which selects the items that provide maximal information about the unlabelled set.

% There are two common AL settings: \emph{stream-based} and \emph{pool-based}~\citep{settles2010active}.
% In stream-based active learning the model decides, while observing a stream of items in random order, either to predict an item's label or to pay a cost to observe its label.
% In pool-based active learning the model has access to a static collection of unlabeled data and decides for which items to observe labels and in what order.

The idea of learning an active learning algorithm end-to-end, via \emph{meta} active learning, was recently investigated  by~\citet{activeoneshot}. Building on the memory-augmented neural network (MANN)~\citep{santoro2016one}, the authors developed a \emph{stream-based} active learner. In stream-based active learning the model decides, while observing items presented in an exogenously-determined order, whether to predict each item's label or to pay a cost to observe its label. Our proposed model instead falls into the class of \emph{pool-based} active learners, i.e.~it has access to a static collection of unlabeled data and selects both the items for which to observe labels, and the order in which to observe them.

%
% This needs fixing.
%
% Active learning is useful when the cost balance between prediction error and labeling
% can vary. I.e., if prediction error is always more costly, we should always ask for labels,
% and if labeling is always more costly, we should always predict (even if wildly wrong).
%
% We expect active learning to be useful when sometimes it is more costly to make a
% prediction error (e.g. because the model may be _very_ wrong, or correctness is critical),
% and sometimes more costly to request a label.
% 
% If the relative costs of error and labeling never swapped, we would always prefer to do
% the one with lower cost, which would make active learning irrelevant.
%
Active learning can be useful when the cost incurred for labeling an item may be traded for lower prediction error, and where the model must be data efficient (e.g.~in medical imaging~\citep{medical}). We explicitly train our model to balance between task performance and labeling cost. In this sense, we build an \emph{anytime} active learner~\cite{zilberstein1996using}, with the model trained at each step to output the best possible prediction on the evaluation set.

Our model builds on the matching-networks (MN) architecture presented by~\citet{vinyals2016matching}, which enables ``one-shot'' learning, i.e.~learning the appearance of a class from just a single example of that class~\cite{santoro2016one,koch2015siamese}.~\citet{vinyals2016matching} assume that at least one example per class exists in the labeled support set available to the model. Confronted with the harder task of composing a labeled support set from a larger pool of unlabeled examples, we show that the active learning policy learnt by our model obtains, in some cases, an equally effective support set. As in the recent one-shot learning work of \citet{santoro2016one} and \citet{vinyals2016matching}, and the active learning work of \citet{activeoneshot}, we evaluate our model on the \emph{Omniglot} dataset. This dataset was developed for the foundational one-shot learning work of \citet{lake2015human}, which focused on probabilistic program induction.

%% Speak about personalization here.
The cold-start problem is ubiquitous in recommendation systems~\citep{aggarwal2016recommender,rs1,harpale2008personalized,sun2013cold,elahi2016survey}. Instead of bootstrapping from a cold-start by randomly selecting items for a user to rate, an active learner asks for particular items to help learn a strong user model more quickly. In model-free strategies~\cite{rashid2008learning}, items are selected according to general empirical statistics such as popularity or informativeness. These approaches are computationally cheap, but lack the benefits of adaptation and personalization.
Proposals for learning an adaptive selection strategy have been made in the form of Bayesian methods that learn the parameters of a user model~\cite{houlsby2014cold,harpale2008personalized}, and in the form of decision-trees learned from existing ratings~\cite{sun2013cold}.
An extensive review can be found in~\citet{elahi2016survey}.
Intuitively, our model learns a compact, parametric representation of a decision tree end-to-end, by directly maximizing task performance. We evaluate our active learner on MovieLens-20M, a standard dataset for recommendation tasks.

We provide hints to our model during training using samples from an oracle policy that knows all the labels. Related approaches have been explored in previous work on imitation learning and learning to search \cite{ross2014, chang2015}. These methods, which focus the cost of sampling from the oracle policy on states visited by the model policy, have recently been adopted by researchers working with deep networks for representation learning \cite{zhang2017, sun2017}.


\section{Model Description}
\label{sec:model}
\section{The \MakeLowercase{i}W\MakeLowercase{inr}NFL model}
\label{sec:model}

In this section we are going to present the data we used to develop our in-game probability model as well as the design details of {\method}. 

{\bf Data: }In order to perform our analysis we utilize a dataset collected from NFL's Game Center for all the regular season games between the seasons 2009 and 2016. 
We access the data using the Python {\tt nflgame} API \cite{nflgame}. 
The dataset includes detailed play-by-play information for every game that took place during these seasons. 
This information is used to obtain the state of the game that will drive the design of {\method}. 
In total, we collected information for 2,048 regular season games and a total of 338,294 snaps/plays. 

{\bf Model: }
{\method} is based on a logistic regression model that calculates the probability of the home team winning given the current status of the game as: 

\begin{equation}
\Pr(H=1| \mathbf{x})= \frac{\exp(\mathbf{\weight}^T\cdot\mathbf{x})}{1+\exp(\mathbf{\weight}^T\cdot\mathbf{x})}
\label{eq:reg}
\end{equation}
where $H$ is the dependent random variable of our model representing whether the home team wins or not, $\mathbf{x}$ is the vector with the independent variables, while the coefficient vector $\mathbf{\weight}$ includes the weights for each independent variable and is estimated using the corresponding data.  
For a game of infinite duration a linear model could be a very good approximation.  
However, the boundary effects from the finite duration of a game create several non-linearities \cite{winston2012mathletics}.  
For this reason, we enhance our model - using the same set of features - with a Support Vector Machine classifier with radial kernel for the last three minutes of regulation.  
In order to obtain a probability output from the SVM classifier, we further use Platt's scaling \cite{platt1999probabilistic}: 

\begin{equation}
\Pr(H=1| \mathbf{x})= \frac{1}{1+\exp{(Af(x)+B)}}
\label{eq:platt}
\end{equation}
where $f(x)$ is the uncalibrated value produced by the SVM classifier: 

\begin{equation}
f(x) = \sum_{i} (\alpha_i y_i k(\mathbf{x}_i\cdot\mathbf{x}))+ b
\label{eq:svm}
\end{equation}
where $k(\mathbf{x},\mathbf{x}')$ is the kernel used for the SVM.   
Figure \ref{fig:iwinrNFL} depicts the simple flow chart of {\method}. 


\begin{figure}[t]
\begin{center}
\includegraphics[scale=0.35]{plots/iwinrNFL.pdf}%\vspacecap
 \caption{{\method} includes a linear and a non-linear component.}
 \label{fig:iwinrNFL}
\end{center}
\end{figure}

In order to describe the status of the game we use the following variables:

\begin{enumerate}
\item {\bf Ball Possession Team:} This binary feature captures whether the home or the visiting team has the ball possession
\item {\bf Score Differential:} This feature captures the current score differential (home - visiting)
\item {\bf Timeouts Remaining:} This feature is represented by two independent variables - one for the home and one for the away team - and they capture the number of timeouts remaining for each of the teams
%\item {\bf Quarter:} This feature captures the current quarter of the game
%\item {\bf Time Remaining:} This feature captures the time (in seconds) remaining for the current quarter to end
\item {\bf Time Elapsed: } This feature captures the time elapsed since the beginning of the game
\item {\bf Down:} This feature represents the down of the team in possession
\item {\bf Field Position:} This feature captures the distance covered by the team in possession from their own yard line
\item {\bf Yards-to-go:} This variables represents the number of yards needed for a first down
\item {\bf Ball Possession Time: } This variable captures the time that the offensive unit of the home team is on the field 
\item {\bf Ranking Differential: } This variable represents the difference of the win percentage for the two team (home - visiting)
\end{enumerate}

The last independent variable is representative of the power ranking difference between the two teams. 
Most of the existing models that include such a variable are using the Vegas line spread for each game.  
We choose not to do so for the following reason.  
The objective of the Vegas line is not to predict game outcomes but rather distribute money across the different bets.  
Exactly because of this objective the line is changing during the week before the game.  
While this line can change due to new information for the competing teams (e.g., injury updates), the line is mainly changing when a particular team has accumulated the majority of the bets. 
In this case it will also be hard to choose which line to use (e.g., the opening, the closing or some average of them).  
Therefore, we choose to use the win percentage differential of the two teams as an indicator of their strength (even though this has its own issues given the uneven schedule in NFL).  
However, note that if one would like to use the point spread as a variable this can be easily incorporated in the model. 
Table \ref{tab:iwinrnfl} presents the coefficients of the logistic regression model of {\method} with standardized independent variables for better comparisons. 


\begin{table}[ht]
\begin{center}
\def\sym#1{\ifmmode^{#1}\else\(^{#1}\)\fi}
\begin{tabular}{l*{1}{c}}
\toprule
                    &\multicolumn{1}{c}{(1)}\\
                    &\multicolumn{1}{c}{Winner}\\
\midrule
Possession Team (H)         &      0.41\sym{***}\\
                    &     (49.19)         \\
\addlinespace
Score Differential           &      3.59\sym{***}\\
                    &    (247.34)         \\
\addlinespace
Home Timeouts           &     0.12\sym{***}\\
                    &      (8.74)         \\
\addlinespace
Away Timeouts           &     -0.11\sym{***}\\
                    &    (-12.47)         \\
\addlinespace
Ball Possession Time  &     -0.05.\\
                    &    (-1.66)         \\
\addlinespace
Time Lapsed       &   -0.05.\\
                    &      (-1.66)         \\
\addlinespace
Down                &   -0.01         \\
                    &      (0.04)         \\
\addlinespace
Field Position            &   0.02\sym{**} \\
                    &      (2.71)         \\
\addlinespace
Yards-to-go                &  -0.01         \\
                    &      (0.23)         \\
\addlinespace
Rating differential         &       0.75\sym{***}\\
                    &     (80.47)         \\
\addlinespace
Intercept            &       0.57\sym{*}\\
                    &    (2.09)         \\
\midrule
Observations        &      338,294         \\
\bottomrule
\multicolumn{2}{l}{\footnotesize \textit{t} statistics in parentheses}\\
\multicolumn{2}{l}{\footnotesize \sym{$_.$} \(p<0.1\), \sym{*} \(p<0.05\), \sym{**} \(p<0.01\), \sym{***} \(p<0.001\)}\\
\end{tabular}
\end{center}
\caption{Standardized logisitic regression coefficients for {\method}.}
\label{tab:iwinrnfl}
\end{table}


As we can see, as one might have expected the current scoring differential exhibits the strongest correlation with the in-game win probability.  
The only factors that do not appear to be statistically significant predictors of the dependent variable are the down and the yards-to-go. 
Even though the corresponding coefficients are negative as one might have expected (e.g., being at an earlier down gives you more chances to advance the ball), they are not significant in estimating the win probability. 
On the contrary, all else being equal timeouts appear to be quiet important since they can help a team stop the clock, while teams with better win percentage appear to have an advantage as well, since this can be a sign of a better team. 
In the following section we provide a detailed evaluation of {\method}.

\section{Experiments}
\label{sec:exp}
\subsection{Omniglot}
\label{sec:omniglot}
%!TEX root = main.tex

%\begin{table*}[]
%\centering
%\caption{Results for our active learner and baselines for the $N$-way, $K$-shot classification settings.}
%\label{tab:res_kway_kshot}
%\begin{tabular}{@{}lllllll@{}}
%\toprule
%\multicolumn{1}{c}{\multirow{2}{*}{Model}} & \multicolumn{3}{c}{\textbf{5-way}} & \multicolumn{3}{c}{\textbf{10-way}} \\ \cmidrule(l){2-7} 
%\multicolumn{1}{c}{}                       & 1-shot     & 2-shot    & 3-shot    & 1-shot     & 2-shot     & 3-shot    \\ \midrule
%\textbf{Matching Net (random)}             & 70.1\%     & 93.2\%    & 98.5\%    & 67.4\%     & 91.1\%     & 97.5\%    \\
%\textbf{Matching Net (balanced)}           & 98.0\%     & 99.0\%    & 99.1\%    & 96.6\%     & 98.5\%     & 98.6\%    \\
%\textbf{Active MN}                         & 97.8\%     & 98.9\%    & 99.2\%    & 94.4\%     & 98.2\%     & 98.5\%    \\
%\midrule
%\textbf{Min-Max-Cos}                       & 97.7\%     & 99.3\%    & 99.4\%    & 94.0\%     & 98.4\%     & 98.8\%    \\ \bottomrule
%\end{tabular}
%\vspace{-0.25cm}
%\end{table*}

\begin{table*}[]
\centering
\caption{Results for our active learner and baselines for the $N$-way, $K$-shot classification settings.}
\label{tab:res_kway_kshot}
\begin{tabular}{@{}lllllll@{}}
\toprule
\multicolumn{1}{c}{\multirow{2}{*}{Model}} & \multicolumn{3}{c}{\textbf{5-way}} & \multicolumn{3}{c}{\textbf{10-way}} \\ \cmidrule(l){2-7} 
\multicolumn{1}{c}{}                       & 1-shot     & 2-shot    & 3-shot    & 1-shot     & 2-shot     & 3-shot    \\ \midrule
\textbf{Matching Net (random)} & 69.8\%$_{\pm 0.10}$     & 93.1\%$_{\pm 0.07}$    & 98.5\%$_{\pm 0.04}$ & 67.3\%$_{\pm 0.10}$  & 91.2\%$_{\pm 0.06}$ & 97.6\%$_{\pm 0.06}$    \\
\textbf{Matching Net (balanced)} & 97.9\%$_{\pm 0.07}$     & 98.9\%$_{\pm 0.07}$    & 99.2\%$_{\pm 0.06}$    & 96.5\%$_{\pm 0.04}$     & 98.3\%$_{\pm 0.03}$    & 98.7\%$_{\pm 0.05}$    \\
\textbf{Active MN} & 97.4\%$_{\pm 0.11}$     & 99.0\%$_{\pm 0.08}$    & 99.3\%$_{\pm 0.03}$    & 94.3\%$_{\pm 0.24}$    & 98.0\%$_{\pm 0.07}$     & 98.5\%$_{\pm 0.06}$    \\
\midrule
\textbf{Min-Max-Cos} & 97.4\%$_{\pm 0.11}$     & 99.3\%$_{\pm 0.02}$    & 99.4\%$_{\pm 0.04}$    & 93.5\%$_{\pm 0.11}$    & 98.4\%$_{\pm 0.02}$     & 98.8\%$_{\pm 0.03}$    \\ \bottomrule
\end{tabular}
\vspace{-0.25cm}
\end{table*}

\begin{figure}[t]
\begin{center}
\vspace{-0.1cm}
\hspace{-0.5cm}
\includegraphics[width=0.9\linewidth]{./og_annot.pdf}
\vspace{-0.25cm}
\caption{A rollout of our active learning policy for Omniglot, using a support set of 20 items from 10 different classes with 2 items per class. Each row represents the support set at different active iterations. For visualization purposes, each column represents a class. Unlabeled items have white background while selected items have black background. Here, the model behaves intelligently, by selecting at each step an item with a yet-unseen label.}
\label{fig:og_rollout}
\end{center}
\vspace{-0.5cm}
\end{figure}

We run our first experiments on the Omniglot dataset~\citep{lake2015human} consisting of 1623 characters from 50 different alphabets, each hand-written by 20 different people.
Following~\citet{vinyals2016matching}, we divide the dataset into 1200 characters for training and keep the rest for testing. When measuring test performance, our model interacts with characters it did not encounter during training.
%
% here, specifications about the omniglot encoder
%

For the context-free embedding function we use a three-layer convolutional network. The first two layers use $5 \times 5$ convolutions with 64 filters and downsample with a double stride. The third layer uses a $3 \times 3$ convolution with 64 filters and no downsampling. These layers produce a $7 \times 7 \times 64$ feature map that we flatten and pass through a fully connected layer. All convolutional layers use the leaky ReLU nonlinearity \cite{Maas2013}.

We setup $N$-way, $K$-shot Omniglot classification as follows. We randomly pick $N$ character classes from the available train/test classes. Then, we build a support set by randomly sampling 5 items for each character class,~e.g. in the 5-way setting, there are 25 items in the support set. The held-out set is always obtained by randomly sampling 1 item per class.
In our active learning setting, $K$-shot is proportional to how many labels the model can acquire. In the $N$-way, $K$-shot setting, the model asks for $NK$ labels before performing held-out prediction. For example, in 5-way, 1-shot classification, the model asks for 5 labels. Following each label query, we also measure anytime performance of the fast prediction module on the items remaining in $S^u_t$. Note that the 1-shot setting is particularly challenging for our model, as it needs to ask for different classes at each step, and the ability to identify missing classes is limited by the accuracy of the underlying one-shot classifier.

We compare our active learner to four baselines. To compute a pessimistic estimate of potential performance, we use a matching network where we label $NK$ items chosen at random from the full support set (Matching Net (random)). As the labels are randomly sampled, it is possible that a given class is never represented among the labeled items and the model cannot classify perfectly, even in principle. To compute an optimistic estimate of potential performance, we measure the ``ideal'' matching network accuracy by labeling a class-balanced subset of items from the full support set (Matching Net (balanced)). This baseline represents a highly-performant policy that the active learner can, in principle, learn. For the last baseline (Min-Max-Cos), we formulate a heuristic policy. At each active learning step, we select the item which has minimum maximum cosine similarity to unlabeled items in the support set. This heuristic selects item that are different from each other, a strategy well-suited to the Omniglot classification task where items are drawn from a consistent set of underlying classes.

We report the results in Table~\ref{tab:res_kway_kshot}. Matching Networks operating on a randomly sampled set of labels suffer the most in 1-shot scenarios, where the probability of all classes being represented is particularly low (especially in the 10-way case). Overall, the active policy nearly matches the performance of the optimistic balanced Matching Network baseline. Degradation of performance by 2.2\% is observed for the 1-shot, 10-way case. This is not surprising since augmenting the number of classes in the support set, while keeping the number of shots fixed, considerably increases the difficulty of the problem for the active learner.
Figure~\ref{fig:og_rollout} shows a roll-out of the model policy in the 10-way setting.

Figure~\ref{fig:mega_plot} provides results for the more challenging setting of 20-way classification. We tested two properties of our model: its anytime performance beyond the 1-shot setting, and its ability to generalize to problems with more classes than were seen during training. The model performed well on 20-way classification, and quickly approached the optimistic performance estimate after acquiring more labels. We also found that policies trained for as little as 10-way classification could generalize to the 20-way setting.

Our model relies on a number of moving parts. When designing the architecture, we followed the simple approach of minimizing changes to the original Matching Network from \citet{vinyals2016matching}. We now provide ablation test results for several parts of our model. In the 10-way, 1-shot setting accuracy dropped from 94.5 to 86.0 when we removed attention temperature from the fast prediction module. Reducing the number of matching steps from 3 to 2 or 1 had no significant effect in this setting. Removing the context-sensitive encoder also had no significant effect. Streamlining our architecture is clearly a useful topic for future work aimed at scaling our approach to more realistic settings.

\begin{figure*}
\begin{center}
\includegraphics[width=0.38\linewidth]{./mega_plot_type1.pdf}
\includegraphics[width=0.38\linewidth]{./og_gen_test_20_type1.pdf}
% \vspace{-0.2cm}
\caption{Experiment results for our model and baselines on Omniglot. The left plot shows how prediction accuracy improves with the number of labels requested in a challenging 20-way setting. After 20 label requests~(corresponding to a 20-way, 1-shot problem), the active policy outperforms random policy and random MN baselines, but is inferior to the balanced MN. After 30 labels, the active policy nearly matches the performance of the balanced MN using 40 labels (20-way, 2-shot). The right plot shows the number of unique labels with respect to the number of requested labels for models trained on problems with 5-20 classes, and tested on 20-way classification. This gives an idea of how models search for labels from unseen groups and generalize to problems with different numbers of classes.}
\label{fig:mega_plot}
\end{center}
\vspace{-0.5cm}
\end{figure*}

\subsection{MovieLens}
\label{sec:movielens}
%!TEX root = main.tex

\begin{figure}[t]
%\vspace{-0.5cm}
\begin{center}
\includegraphics[width=0.8\linewidth]{./rmse_per_step_type1.pdf}
\caption{Performance of the model and baselines measured with RMSE on the Movielens dataset.}
\label{fig:mlens_plot}
\end{center}
\vspace{-0.5cm}
\end{figure}

\subsubsection{Setup}
We test our model in the ``cold-start'' collaborative filtering scenario using the publicly available MovieLens-20M dataset.\footnote{Available at \url{http://grouplens.org/datasets/movielens/}}
The dataset contains approximately 20M ratings on 27K movies by 138K users. The ratings are on an ordinal 10-point scale, from 0.5 to 5 with intervals of 0.5. We subsample the dataset by selecting 4000 movies and 6000 users with the most ratings. After filtering, the dataset contains approximately 1M ratings. We partition the data randomly into 5000 training users and 1000 test users. The training set represents the users already in the system who are used to fit the model parameters. We use the test users to evaluate our active learning approach. For each user, we randomly pick 50 ratings to include in the support set (movies that the user can be queried about) and 10 movies and ratings for the held-out set. We ensure that movies in the held-out set and in the support set do not overlap. We train our active learner to minimize the mean-squared error (MSE) with respect to the true rating. We adapt the prediction modules of our model to output the rating for a held-out item as follows: we compute a convex combination of the ratings of ``visible'' movies in the support set (the movies the active learner has already queried about), where the weights are given by the final attention step of the slow predictor. Although more complex strategies are possible, we empirically found this simple strategy to work well in our experiments. 
For evaluation, we sample 25000 episodes~(each comprising 50 support ratings and 10 held-out ratings from some user) from the test set and we compute the average per-user root mean-squared error (RMSE). We report the average performance obtained by 3 runs with different random seeds.

\subsubsection{Movie Embeddings}
For each movie, we pretrain an embedding vector by decomposing the full user/movie rating matrix using a latent factor model~\citep{koren2010factor}. This process only uses the training set. For each user $u$ and movie $m$, we estimate the true rating $r_{u, m}$ with a linear model $\hat{r}_{u, m} = x_u^\top x_m + b_u + b_m + \beta$, where $x_u, x_m$ are the user and movie embedding respectively and $b_u, b_m, \beta$ are the user, movie, and global bias, respectively. We train the latent factor model by minimizing the mean squared error between true rating $r$ and predicted rating $\hat{r}$. We use the trained $x_m$ as input representations for the movies throughout our experiments.

\subsubsection{Results}
In Figure~\ref{fig:mlens_plot} we report the results of our active model against various baselines.
The Regression baseline performs regularized linear regression on movies from the support set whose ratings have been observed incrementally in random order.
Because of the small amount of training data, for each additional label we tune the regularization parameter by monitoring performance on a separate set of validation episodes.
The Gaussian Process baseline selects the next movie to label in proportion to the variance of the predictive posterior distribution over its rating.
This gives an idea of the impact of using MN one-shot capabilities rather than standard regression techniques.
The Popular-Entropy, Min-Max-Cos, and Entropy Sampling baselines train our model end-to-end, but using fixed selection policies.
Specifically, we train our architecture end-to-end, but instead of training an active learning policy through the select module we choose items from the support set incrementally according to a heuristic policy.
This gives an idea of the importance of learning the selection policy.
The Popular-Entropy policy, adapted from the cold-start work of~\citet{rashid2002getting}, scores each item in the support set a priori, according to the logarithm of its popularity multiplied by the entropy of the item's ratings measured across users.
% It then iterates through the support set selecting items with probability inversely proportional to their score.
This strategy aims to first collect the ratings for those movies that are both popular and have been rated differently by different users.
Although it is simplistic, the policy achieves competitive performance for bootstrapping a system from a cold-start setting~\cite{elahi2016survey}.
The Min-Max-Cos policy is identical to the synonymous baseline used for Omniglot,~i.e. it selects the unrated movie which has minimum maximum cosine similarity to any of the rated movies.
Roughly, this selects the unrated movie which differs most from the rated movies. Entropy Sampling selects movies in proportion to rating prediction entropy.

The active policy learned end-to-end outperforms the baselines in terms of RMSE, particularly after requesting only the first few labels.
After 10 ratings, our model achieves an improvement of 2.5\% in RMSE against the best baseline.
Unsurprisingly, the gap diminishes with a higher number of labels requested.
After observing 5 labels, the Popular-Entropy baseline and our architecture equipped with the Min-Max-Cos heuristic converge toward the active policy but never quite match it.
For MovieLens, where labels are user-dependent and not tied to an underlying class, a data-driven selection policy may adapt better to the task.
This contrasts with the Omniglot setting, where there is no aspect of personalization and Active MN and Min-Max-Cos perform similarly.
The Min-Max-Cos heuristic is designed to not select items similar to those it has already seen, but selecting similar items can be beneficial in personalized settings~\cite{elahi2016survey}.

% From Figure~\ref{fig:mlens_plot}(b) it is clear that the active policy also outperforms the baselines in terms of mean absolute error (MAE), defined as the average of $|\hat{r} - r|$ over ratings. This measure represents the mean shift of each model with respect to the true rating.

\section{Conclusion}
\label{sec:conc}

\begin{comment}
\begin{figure}
\includegraphics[width=\linewidth]{figs/beyond_tss_lesion.pdf}
\caption[]{End-to-End runtime lesion study of the entire MNIST dataset and the FMA featurized music dataset. Each of DROP's contributions provides a runtime improvement.}
\label{fig:beyond_lesion}
\end{figure}
\end{comment}



\section{Conclusion}
\label{sec:conclusion}

Advanced data analytics techniques must scale to rising data volumes. 
DR techniques offer a powerful toolkit when processing these datasets, with PCA frequently outperforming popular techniques in exchange for high computational cost. 
In response, we propose DROP, a new dimensionality reduction optimizer. 
DROP combines progressive sampling, progress estimation, and online aggregation to identify high quality low dimensional bases via PCA without processing the entire dataset by balancing the runtime of downstream tasks and achieved dimensionality. 
Thus, DROP provides a first step in bridging the gap between quality and efficiency in end-to-end DR for downstream \red{analytics}. 

%We revisit canonical operators for time series dimensionality reduction and the measurement study of~\cite{keogh-study}, and show that PCA is more effective than popular alternatives in the data mining literature often by a margin of over $2\times$ on average on gold-standard time series benchmark data sets with respect to output data dimension. More surprisingly, we empirically demonstrate that a small number of samples are sufficient to accurately characterize directions of maximum variance and obtain a high-quality low-dimensional transformation.




{\footnotesize
\bibliography{ggrefs}
\bibliographystyle{icml2017}
}

\end{document} 

