%%%%%%%%%%%%%%%%%%%%%%%%%%%%%%%%%%%%%%%%%%%%%%%%%%%%%%%%%%%%%%%%%%
%%%%%%%% ICML 2016 EXAMPLE LATEX SUBMISSION FILE %%%%%%%%%%%%%%%%%
%%%%%%%%%%%%%%%%%%%%%%%%%%%%%%%%%%%%%%%%%%%%%%%%%%%%%%%%%%%%%%%%%%

% Use the following line _only_ if you're still using LaTeX 2.09.
%\documentstyle[icml2016,epsf,natbib]{article}
% If you rely on Latex2e packages, like most moden people use this:
\documentclass{article}

% use Times
\usepackage{times}
% For figures
\usepackage{graphicx} % more modern
%\usepackage{epsfig} % less modern
\usepackage{subfigure} 

% For citations
\usepackage{natbib}

% For algorithms
\usepackage{multirow}
\usepackage{algorithm}
\usepackage{algorithmic}
% \usepackage[ruled,vlined,linesnumbered]{algorithm2e}


% As of 2011, we use the hyperref package to produce hyperlinks in the
% resulting PDF.  If this breaks your system, please commend out the
% following usepackage line and replace \usepackage{icml2016} with
% \usepackage[nohyperref]{icml2016} above.
\usepackage{hyperref}

% Packages hyperref and algorithmic misbehave sometimes.  We can fix
% this with the following command.
\newcommand{\theHalgorithm}{\arabic{algorithm}}

% Employ the following version of the ``usepackage'' statement for
% submitting the draft version of the paper for review.  This will set
% the note in the first column to ``Under review.  Do not distribute.''
%\usepackage{icml2017}

% Employ this version of the ``usepackage'' statement after the paper has
% been accepted, when creating the final version.  This will set the
% note in the first column to ``Proceedings of the...''
\usepackage[accepted]{icml2017}


\usepackage{algorithm}

\usepackage{latexsym}
\usepackage[utf8]{inputenc} % allow utf-8 input
\usepackage[T1]{fontenc}   % use 8-bit T1 fonts
\usepackage{hyperref}       % hyperlinks
\usepackage{url}            % simple URL typesetting
\usepackage{booktabs}       % professional-quality tables
\usepackage{amsfonts}       % blackboard math symbols
\usepackage{nicefrac}       % compact symbols for 1/2, etc.
\usepackage{microtype}      % microtypography
\usepackage[centertags]{amsmath}
\usepackage{amssymb}
\usepackage{graphicx}
\usepackage{tabularx}
\usepackage{subfigure}
\usepackage{paralist}
\usepackage[inline,shortlabels]{enumitem}

\def\x{\times}
\def\S{\mathbf{S}}
\def\H{\mathbf{H}}
\renewcommand{\b}[1]{\mathbf{#1}}

% mathcal shortcuts
\newcommand{\eye}{\mathbf{I}}
\newcommand{\CC}{\mathcal{C}}
\newcommand{\DD}{\mathcal{D}}
\newcommand{\FF}{\mathcal{F}}
\newcommand{\GG}{\mathcal{G}}
\newcommand{\HH}{\mathcal{H}}
\newcommand{\II}{\mathcal{I}}
\newcommand{\KK}{\mathcal{K}}
\newcommand{\LL}{\mathcal{L}}
\newcommand{\MM}{\mathcal{M}}
\newcommand{\NN}{\mathcal{N}}
\newcommand{\OO}{\mathcal{O}}
\newcommand{\PP}{\mathcal{P}}
\newcommand{\QQ}{\mathcal{Q}}
\newcommand{\RR}{\mathcal{R}}
\newcommand{\SSS}{\mathcal{S}}
\newcommand{\TT}{\mathcal{T}}
\newcommand{\VV}{\mathcal{V}}
\newcommand{\WW}{\mathcal{W}}
\newcommand{\XX}{\mathcal{X}}
\newcommand{\YY}{\mathcal{Y}}
\newcommand{\ZZ}{\mathcal{Z}}
\newcommand{\cev}[1]{\reflectbox{\ensuremath{\vec{\reflectbox{\ensuremath{#1}}}}}}

% nicely formatting math operators
\DeclareMathOperator*{\argmin}{arg\,min}
\DeclareMathOperator*{\argmax}{arg\,max}
\DeclareMathOperator*{\expect}{\mathbb{E}}
\DeclareMathOperator*{\minimize}{minimize}
\DeclareMathOperator*{\maximize}{maximize}
\DeclareMathOperator*{\xent}{xent}
\DeclareMathOperator*{\ent}{ent}
\DeclareMathOperator*{\softmax}{softmax}
\DeclareMathOperator*{\KL}{KL}
\DeclareMathOperator*{\lrelu}{lrelu}
\DeclareMathOperator*{\relu}{relu}
\DeclareMathOperator*{\conv}{conv}


% The \icmltitle you define below is probably too long as a header.
% Therefore, a short form for the running title is supplied here:
\icmltitlerunning{Learning Algorithms for Active Learning}

\begin{document} 

\twocolumn[
\icmltitle{Learning Algorithms for Active Learning}

% It is OKAY to include author information, even for blind
% submissions: the style file will automatically remove it for you
% unless you've provided the [accepted] option to the icml2017
% package.

% list of affiliations. the first argument should be a (short)
% identifier you will use later to specify author affiliations
% Academic affiliations should list Department, University, City, Region, Country
% Industry affiliations should list Company, City, Region, Country

% you can specify symbols, otherwise they are numbered in order
% ideally, you should not use this facility. affiliations will be numbered
% in order of appearance and this is the preferred way.
\icmlsetsymbol{equal}{*}

\begin{icmlauthorlist}
\icmlauthor{Philip Bachman}{equal,maluuba}
\icmlauthor{Alessandro Sordoni}{equal,maluuba}
\icmlauthor{Adam Trischler}{maluuba}
\end{icmlauthorlist}

\icmlaffiliation{maluuba}{Microsoft Maluuba, Montreal, Canada}

\icmlcorrespondingauthor{P.~ Bachman}{phbachma@microsoft.com}
\icmlcorrespondingauthor{A.~ Sordoni}{alsordon@microsoft.com}

% You may provide any keywords that you 
% find helpful for describing your paper; these are used to populate 
% the "keywords" metadata in the PDF but will not be shown in the document
\icmlkeywords{boring formatting information, machine learning, ICML}

\vskip 0.3in
]

% this must go after the closing bracket ] following \twocolumn[ ...

% This command actually creates the footnote in the first column
% listing the affiliations and the copyright notice.
% The command takes one argument, which is text to display at the start of the footnote.
% The \icmlEqualContribution command is standard text for equal contribution.
% Remove it (just {}) if you do not need this facility.

%\printAffiliationsAndNotice{}  % leave blank if no need to mention equal contribution
\printAffiliationsAndNotice{\icmlEqualContribution} % otherwise use the standard text.

\begin{abstract}
We introduce a model that learns active learning algorithms via metalearning. For a distribution of related tasks, our model jointly learns: a data representation, an item selection heuristic, and a method for constructing prediction functions from labeled training sets. Our model uses the item selection heuristic to gather labeled training sets from which to construct prediction functions. Using the Omniglot and MovieLens datasets, we test our model in synthetic and practical settings.
\end{abstract}


\section{Introduction}
\label{sec:intro}
% \leavevmode
% \\
% \\
% \\
% \\
% \\
\section{Introduction}
\label{introduction}

AutoML is the process by which machine learning models are built automatically for a new dataset. Given a dataset, AutoML systems perform a search over valid data transformations and learners, along with hyper-parameter optimization for each learner~\cite{VolcanoML}. Choosing the transformations and learners over which to search is our focus.
A significant number of systems mine from prior runs of pipelines over a set of datasets to choose transformers and learners that are effective with different types of datasets (e.g. \cite{NEURIPS2018_b59a51a3}, \cite{10.14778/3415478.3415542}, \cite{autosklearn}). Thus, they build a database by actually running different pipelines with a diverse set of datasets to estimate the accuracy of potential pipelines. Hence, they can be used to effectively reduce the search space. A new dataset, based on a set of features (meta-features) is then matched to this database to find the most plausible candidates for both learner selection and hyper-parameter tuning. This process of choosing starting points in the search space is called meta-learning for the cold start problem.  

Other meta-learning approaches include mining existing data science code and their associated datasets to learn from human expertise. The AL~\cite{al} system mined existing Kaggle notebooks using dynamic analysis, i.e., actually running the scripts, and showed that such a system has promise.  However, this meta-learning approach does not scale because it is onerous to execute a large number of pipeline scripts on datasets, preprocessing datasets is never trivial, and older scripts cease to run at all as software evolves. It is not surprising that AL therefore performed dynamic analysis on just nine datasets.

Our system, {\sysname}, provides a scalable meta-learning approach to leverage human expertise, using static analysis to mine pipelines from large repositories of scripts. Static analysis has the advantage of scaling to thousands or millions of scripts \cite{graph4code} easily, but lacks the performance data gathered by dynamic analysis. The {\sysname} meta-learning approach guides the learning process by a scalable dataset similarity search, based on dataset embeddings, to find the most similar datasets and the semantics of ML pipelines applied on them.  Many existing systems, such as Auto-Sklearn \cite{autosklearn} and AL \cite{al}, compute a set of meta-features for each dataset. We developed a deep neural network model to generate embeddings at the granularity of a dataset, e.g., a table or CSV file, to capture similarity at the level of an entire dataset rather than relying on a set of meta-features.
 
Because we use static analysis to capture the semantics of the meta-learning process, we have no mechanism to choose the \textbf{best} pipeline from many seen pipelines, unlike the dynamic execution case where one can rely on runtime to choose the best performing pipeline.  Observing that pipelines are basically workflow graphs, we use graph generator neural models to succinctly capture the statically-observed pipelines for a single dataset. In {\sysname}, we formulate learner selection as a graph generation problem to predict optimized pipelines based on pipelines seen in actual notebooks.

%. This formulation enables {\sysname} for effective pruning of the AutoML search space to predict optimized pipelines based on pipelines seen in actual notebooks.}
%We note that increasingly, state-of-the-art performance in AutoML systems is being generated by more complex pipelines such as Directed Acyclic Graphs (DAGs) \cite{piper} rather than the linear pipelines used in earlier systems.  
 
{\sysname} does learner and transformation selection, and hence is a component of an AutoML systems. To evaluate this component, we integrated it into two existing AutoML systems, FLAML \cite{flaml} and Auto-Sklearn \cite{autosklearn}.  
% We evaluate each system with and without {\sysname}.  
We chose FLAML because it does not yet have any meta-learning component for the cold start problem and instead allows user selection of learners and transformers. The authors of FLAML explicitly pointed to the fact that FLAML might benefit from a meta-learning component and pointed to it as a possibility for future work. For FLAML, if mining historical pipelines provides an advantage, we should improve its performance. We also picked Auto-Sklearn as it does have a learner selection component based on meta-features, as described earlier~\cite{autosklearn2}. For Auto-Sklearn, we should at least match performance if our static mining of pipelines can match their extensive database. For context, we also compared {\sysname} with the recent VolcanoML~\cite{VolcanoML}, which provides an efficient decomposition and execution strategy for the AutoML search space. In contrast, {\sysname} prunes the search space using our meta-learning model to perform hyperparameter optimization only for the most promising candidates. 

The contributions of this paper are the following:
\begin{itemize}
    \item Section ~\ref{sec:mining} defines a scalable meta-learning approach based on representation learning of mined ML pipeline semantics and datasets for over 100 datasets and ~11K Python scripts.  
    \newline
    \item Sections~\ref{sec:kgpipGen} formulates AutoML pipeline generation as a graph generation problem. {\sysname} predicts efficiently an optimized ML pipeline for an unseen dataset based on our meta-learning model.  To the best of our knowledge, {\sysname} is the first approach to formulate  AutoML pipeline generation in such a way.
    \newline
    \item Section~\ref{sec:eval} presents a comprehensive evaluation using a large collection of 121 datasets from major AutoML benchmarks and Kaggle. Our experimental results show that {\sysname} outperforms all existing AutoML systems and achieves state-of-the-art results on the majority of these datasets. {\sysname} significantly improves the performance of both FLAML and Auto-Sklearn in classification and regression tasks. We also outperformed AL in 75 out of 77 datasets and VolcanoML in 75  out of 121 datasets, including 44 datasets used only by VolcanoML~\cite{VolcanoML}.  On average, {\sysname} achieves scores that are statistically better than the means of all other systems. 
\end{itemize}


%This approach does not need to apply cleaning or transformation methods to handle different variances among datasets. Moreover, we do not need to deal with complex analysis, such as dynamic code analysis. Thus, our approach proved to be scalable, as discussed in Sections~\ref{sec:mining}.

\section{Related Work}
\label{sec:related}
%!TEX root = main.tex

% Active learning is motivated by the idea that a learning algorithm can perform better while training on fewer labeled data if it can choose the data on which it trains~\citep{cohn1996active}.
Various heuristics have been proposed to guide the selection of which examples to label during active learning~\cite{settles2010active}. For instance, \citet{lewis1994sequential} and \citet{tong2001support} developed policies based on the confidence of the classifier, while \citet{gilad2005query} used the disagreement of a \emph{committee} of classifiers. \citet{houlsby2011bayesian} presented an approach based on Bayesian information theory,
in which examples are selected in order to maximally reduce the entropy of the posterior distribution over classifier parameters. 
%Heuristic design is typically (and reasonably) based on balancing computational tractability against closeness to an optimal heuristic.

% An ideal selection heuristic would closely approximate the oracle policy, which selects the items that provide maximal information about the unlabelled set.

% There are two common AL settings: \emph{stream-based} and \emph{pool-based}~\citep{settles2010active}.
% In stream-based active learning the model decides, while observing a stream of items in random order, either to predict an item's label or to pay a cost to observe its label.
% In pool-based active learning the model has access to a static collection of unlabeled data and decides for which items to observe labels and in what order.

The idea of learning an active learning algorithm end-to-end, via \emph{meta} active learning, was recently investigated  by~\citet{activeoneshot}. Building on the memory-augmented neural network (MANN)~\citep{santoro2016one}, the authors developed a \emph{stream-based} active learner. In stream-based active learning the model decides, while observing items presented in an exogenously-determined order, whether to predict each item's label or to pay a cost to observe its label. Our proposed model instead falls into the class of \emph{pool-based} active learners, i.e.~it has access to a static collection of unlabeled data and selects both the items for which to observe labels, and the order in which to observe them.

%
% This needs fixing.
%
% Active learning is useful when the cost balance between prediction error and labeling
% can vary. I.e., if prediction error is always more costly, we should always ask for labels,
% and if labeling is always more costly, we should always predict (even if wildly wrong).
%
% We expect active learning to be useful when sometimes it is more costly to make a
% prediction error (e.g. because the model may be _very_ wrong, or correctness is critical),
% and sometimes more costly to request a label.
% 
% If the relative costs of error and labeling never swapped, we would always prefer to do
% the one with lower cost, which would make active learning irrelevant.
%
Active learning can be useful when the cost incurred for labeling an item may be traded for lower prediction error, and where the model must be data efficient (e.g.~in medical imaging~\citep{medical}). We explicitly train our model to balance between task performance and labeling cost. In this sense, we build an \emph{anytime} active learner~\cite{zilberstein1996using}, with the model trained at each step to output the best possible prediction on the evaluation set.

Our model builds on the matching-networks (MN) architecture presented by~\citet{vinyals2016matching}, which enables ``one-shot'' learning, i.e.~learning the appearance of a class from just a single example of that class~\cite{santoro2016one,koch2015siamese}.~\citet{vinyals2016matching} assume that at least one example per class exists in the labeled support set available to the model. Confronted with the harder task of composing a labeled support set from a larger pool of unlabeled examples, we show that the active learning policy learnt by our model obtains, in some cases, an equally effective support set. As in the recent one-shot learning work of \citet{santoro2016one} and \citet{vinyals2016matching}, and the active learning work of \citet{activeoneshot}, we evaluate our model on the \emph{Omniglot} dataset. This dataset was developed for the foundational one-shot learning work of \citet{lake2015human}, which focused on probabilistic program induction.

%% Speak about personalization here.
The cold-start problem is ubiquitous in recommendation systems~\citep{aggarwal2016recommender,rs1,harpale2008personalized,sun2013cold,elahi2016survey}. Instead of bootstrapping from a cold-start by randomly selecting items for a user to rate, an active learner asks for particular items to help learn a strong user model more quickly. In model-free strategies~\cite{rashid2008learning}, items are selected according to general empirical statistics such as popularity or informativeness. These approaches are computationally cheap, but lack the benefits of adaptation and personalization.
Proposals for learning an adaptive selection strategy have been made in the form of Bayesian methods that learn the parameters of a user model~\cite{houlsby2014cold,harpale2008personalized}, and in the form of decision-trees learned from existing ratings~\cite{sun2013cold}.
An extensive review can be found in~\citet{elahi2016survey}.
Intuitively, our model learns a compact, parametric representation of a decision tree end-to-end, by directly maximizing task performance. We evaluate our active learner on MovieLens-20M, a standard dataset for recommendation tasks.

We provide hints to our model during training using samples from an oracle policy that knows all the labels. Related approaches have been explored in previous work on imitation learning and learning to search \cite{ross2014, chang2015}. These methods, which focus the cost of sampling from the oracle policy on states visited by the model policy, have recently been adopted by researchers working with deep networks for representation learning \cite{zhang2017, sun2017}.


\section{Model Description}
\label{sec:model}
Online convex optimization with memory has emerged as an important and challenging area with a wide array of applications, see \citep{lin2012online,anava2015online,chen2018smoothed,goel2019beyond,agarwal2019online,bubeck2019competitively} and the references therein.  Many results in this area have focused on the case of online optimization with switching costs (movement costs), a form of one-step memory, e.g., \citep{chen2018smoothed,goel2019beyond,bubeck2019competitively}, though some papers have focused on more general forms of memory, e.g., \citep{anava2015online,agarwal2019online}. In this paper we, for the first time, study the impact of feedback delay and nonlinear switching cost in online optimization with switching costs. 

An instance consists of a convex action set $\mathcal{K}\subset\mathbb{R}^d$, an initial point $y_0\in\mathcal{K}$, a sequence of non-negative convex cost functions $f_1,\cdots,f_T:\mathbb{R}^d\to\mathbb{R}_{\ge0}$, and a switching cost $c:\mathbb{R}^{d\times(p+1)}\to\mathbb{R}_{\ge0}$. To incorporate feedback delay, we consider a situation where the online learner only knows the geometry of the hitting cost function at each round, i.e., $f_t$, but that the minimizer of $f_t$ is revealed only after a delay of $k$ steps, i.e., at time $t+k$.  This captures practical scenarios where the form of the loss function or tracking function is known by the online learner, but the target moves over time and measurement lag means that the position of the target is not known until some time after an action must be taken. 
To incorporate nonlinear (and potentially nonconvex) switching costs, we consider the addition of a known nonlinear function $\delta$ from $\mathbb{R}^{d\times p}$ to $\mathbb{R}^d$ to the structured memory model introduced previously.  Specifically, we have
\begin{align}
c(y_{t:t-p}) = \frac{1}{2}\|y_t-\delta(y_{t-1:t-p})\|^2,    \label{e.newswitching}
\end{align}
where we use $y_{i:j}$ to denote either $\{y_i, y_{i+1}, \cdots, y_j\}$ if $i\leq j$, or  $\{y_i, y_{i-1}, \cdots, y_j\}$ if $i > j$ throughout the paper. Additionally, we use $\|\cdot\|$ to denote the 2-norm of a vector or the spectral norm of a matrix.

In summary, we consider an online agent that interacts with the environment as follows:
% \begin{inparaenum}[(i)] 
\begin{enumerate}%[leftmargin=*]
    \item The adversary reveals a function $h_t$, which is the geometry of the $t^\mathrm{th}$ hitting cost, and a point $v_{t-k}$, which is the minimizer of the $(t-k)^\mathrm{th}$ hitting cost. Assume that $h_t$ is $m$-strongly convex and $l$-strongly smooth, and that $\arg\min_y h_t(y)=0$.
    \item The online learner picks $y_t$ as its decision point at time step $t$ after observing $h_t,$  $v_{t-k}$.
    \item The adversary picks the minimizer of the hitting cost at time step $t$: $v_t$. 
    \item The learner pays hitting cost $f_t(y_t)=h_t(y_t-v_t)$ and switching cost $c(y_{t:t-p})$ of the form \eqref{e.newswitching}.
\end{enumerate}

The goal of the online learner is to minimize the total cost incurred over $T$ time steps, $cost(ALG)=\sum_{t=1}^Tf_t(y_t)+c(y_{t:t-p})$, with the goal of (nearly) matching the performance of the offline optimal algorithm with the optimal cost $cost(OPT)$. The performance metric used to evaluate an algorithm is typically the \textit{competitive ratio} because the goal is to learn in an environment that is changing dynamically and is potentially adversarial. Formally, the competitive ratio (CR) of the online algorithm is defined as the worst-case ratio between the total cost incurred by the online learner and the offline optimal cost: $CR(ALG)=\sup_{f_{1:T}}\frac{cost(ALG)}{cost(OPT)}$.

It is important to emphasize that the online learner decides $y_t$ based on the knowledge of the previous decisions $y_1\cdots y_{t-1}$, the geometry of cost functions $h_1\cdots h_t$, and the delayed feedback on the minimizer $v_1\cdots v_{t-k}$. Thus, the learner has perfect knowledge of cost functions $f_1\cdots f_{t-k}$, but incomplete knowledge of $f_{t-k+1}\cdots f_t$ (recall that $f_t(y)=h_t(y-v_t)$).

Both feedback delay and nonlinear switching cost add considerable difficulty for the online learner compared to versions of online optimization studied previously. Delay hides crucial information from the online learner and so makes adaptation to changes in the environment more challenging. As the learner makes decisions it is unaware of the true cost it is experiencing, and thus it is difficult to track the optimal solution. This is magnified by the fact that nonlinear switching costs increase the dependency of the variables on each other. It further stresses the influence of the delay, because an inaccurate estimation on the unknown data, potentially magnifying the mistakes of the learner. 

The impact of feedback delay has been studied previously in online learning settings without switching costs, with a focus on regret, e.g., \citep{joulani2013online,shamir2017online}.  However, in settings with switching costs the impact of delay is magnified since delay may lead to not only more hitting cost in individual rounds, but significantly larger switching costs since the arrival of delayed information may trigger a very large chance in action.  To the best of our knowledge, we give the first competitive ratio for delayed feedback in online optimization with switching costs. 

We illustrate a concrete example application of our setting in the following.

\begin{example}[Drone tracking problem]
\label{example:drone} \emph{
Consider a drone with vertical speed $y_t\in\mathbb{R}$. The goal of the drone is to track a sequence of desired speeds $y^d_1,\cdots,y^d_T$ with the following tracking cost:}
\begin{equation}
    \sum_{t=1}^T \frac{1}{2}(y_t-y^d_t)^2 + \frac{1}{2}(y_t-y_{t-1}+g(y_{t-1}))^2,
\end{equation}
\emph{where $g(y_{t-1})$ accounts for the gravity and the aerodynamic drag. One example is $g(y)=C_1+C_2\cdot|y|\cdot y$ where $C_1,C_2>0$ are two constants~\cite{shi2019neural}. Note that the desired speed $y_t^d$ is typically sent from a remote computer/server. Due to the communication delay, at time step $t$ the drone only knows $y_1^d,\cdots,y_{t-k}^d$.}

\emph{This example is beyond the scope of existing results in online optimization, e.g.,~\cite{shi2020online,goel2019beyond,goel2019online}, because of (i) the $k$-step delay in the hitting cost $\frac{1}{2}(y_t-y_t^d)$ and (ii) the nonlinearity in the switching cost $\frac{1}{2}(y_t-y_{t-1}+g(y_{t-1}))^2$ with respective to $y_{t-1}$. However, in this paper, because we directly incorporate the effect of delay and nonlinearity in the algorithm design, our algorithms immediately provide constant-competitive policies for this setting.}
\end{example}


\subsection{Related Work}
This paper contributes to the growing literature on online convex optimization with memory.  
Initial results in this area focused on developing constant-competitive algorithms for the special case of 1-step memory, a.k.a., the Smoothed Online Convex Optimization (SOCO) problem, e.g., \citep{chen2018smoothed,goel2019beyond}. In that setting, \citep{chen2018smoothed} was the first to develop a constant, dimension-free competitive algorithm for high-dimensional problems.  The proposed algorithm, Online Balanced Descent (OBD), achieves a competitive ratio of $3+O(1/\beta)$ when cost functions are $\beta$-locally polyhedral.  This result was improved by \citep{goel2019beyond}, which proposed two new algorithms, Greedy OBD and Regularized OBD (ROBD), that both achieve $1+O(m^{-1/2})$ competitive ratios for $m$-strongly convex cost functions.  Recently, \citep{shi2020online} gave the first competitive analysis that holds beyond one step of memory.  It holds for a form of structured memory where the switching cost is linear:
$
    c(y_{t:t-p})=\frac{1}{2}\|y_t-\sum_{i=1}^pC_iy_{t-i}\|^2,
$
with known $C_i\in\mathbb{R}^{d\times d}$, $i=1,\cdots,p$. If the memory length $p = 1$ and $C_1$ is an identity matrix, this is equivalent to SOCO. In this setting, \citep{shi2020online} shows that ROBD has a competitive ratio of 
\begin{align}
    \frac{1}{2}\left( 1 + \frac{\alpha^2 - 1}{m} + \sqrt{\Big( 1 + \frac{\alpha^2 - 1}{m}\Big)^2 + \frac{4}{m}} \right),
\end{align}
when hitting costs are $m$-strongly convex and $\alpha=\sum_{i=1}^p\|C_i\|$. 


Prior to this paper, competitive algorithms for online optimization have nearly always assumed that the online learner acts \emph{after} observing the cost function in the current round, i.e., have zero delay.  The only exception is \citep{shi2020online}, which considered the case where the learner must act before observing the cost function, i.e., a one-step delay.  Even that small addition of delay requires a significant modification to the algorithm (from ROBD to Optimistic ROBD) and analysis compared to previous work. 

As the above highlights, there is no previous work that addresses either the setting of nonlinear switching costs nor the setting of multi-step delay. However, the prior work highlights that ROBD is a promising algorithmic framework and our work in this paper extends the ROBD framework in order to address the challenges of delay and non-linear switching costs. Given its importance to our work, we describe the workings of ROBD in detail in Algorithm~\ref{robd}. 

\begin{algorithm}[t!]
  \caption{ROBD \citep{goel2019beyond}}
  \label{robd}
\begin{algorithmic}[1]
  \STATE {\bfseries Parameter:} $\lambda_1\ge0,\lambda_2\ge0$
  \FOR{$t=1$ {\bfseries to} $T$}
  \STATE {\bfseries Input:} Hitting cost function $f_t$, previous decision points $y_{t-p:t-1}$
  \STATE $v_t\leftarrow\arg\min_yf_t(y)$
  \STATE $y_t\leftarrow\arg\min_yf_t(y)+\lambda_1c(y,y_{t-1:t-p})+\frac{\lambda_2}{2}\|y-v_t\|^2_2$
  \STATE {\bfseries Output:} $y_t$
  \ENDFOR
   
\end{algorithmic}
\end{algorithm}

Another line of literature that this paper contributes to is the growing understanding of the connection between online optimization and adaptive control. The reduction from adaptive control to online optimization with memory was first studied in \citep{agarwal2019online} to obtain a sublinear static regret guarantee against the best linear state-feedback controller, where the approach is to consider a disturbance-action policy class with some fixed horizon.  Many follow-up works adopt similar reduction techniques \citep{agarwal2019logarithmic, brukhim2020online, gradu2020adaptive}. A different reduction approach using control canonical form is proposed by \citep{li2019online} and further exploited by \citep{shi2020online}. Our work falls into this category.  The most general results so far focus on Input-Disturbed Squared Regulators, which can be reduced to online convex optimization with structured memory (without delay or nonlinear switching costs).  As we show in \Cref{Control}, the addition of delay and nonlinear switching costs leads to a significant extension of the generality of control models that can be reduced to online optimization. 

\section{Experiments}
\label{sec:exp}
\subsection{Omniglot}
\label{sec:omniglot}
%!TEX root = main.tex

%\begin{table*}[]
%\centering
%\caption{Results for our active learner and baselines for the $N$-way, $K$-shot classification settings.}
%\label{tab:res_kway_kshot}
%\begin{tabular}{@{}lllllll@{}}
%\toprule
%\multicolumn{1}{c}{\multirow{2}{*}{Model}} & \multicolumn{3}{c}{\textbf{5-way}} & \multicolumn{3}{c}{\textbf{10-way}} \\ \cmidrule(l){2-7} 
%\multicolumn{1}{c}{}                       & 1-shot     & 2-shot    & 3-shot    & 1-shot     & 2-shot     & 3-shot    \\ \midrule
%\textbf{Matching Net (random)}             & 70.1\%     & 93.2\%    & 98.5\%    & 67.4\%     & 91.1\%     & 97.5\%    \\
%\textbf{Matching Net (balanced)}           & 98.0\%     & 99.0\%    & 99.1\%    & 96.6\%     & 98.5\%     & 98.6\%    \\
%\textbf{Active MN}                         & 97.8\%     & 98.9\%    & 99.2\%    & 94.4\%     & 98.2\%     & 98.5\%    \\
%\midrule
%\textbf{Min-Max-Cos}                       & 97.7\%     & 99.3\%    & 99.4\%    & 94.0\%     & 98.4\%     & 98.8\%    \\ \bottomrule
%\end{tabular}
%\vspace{-0.25cm}
%\end{table*}

\begin{table*}[]
\centering
\caption{Results for our active learner and baselines for the $N$-way, $K$-shot classification settings.}
\label{tab:res_kway_kshot}
\begin{tabular}{@{}lllllll@{}}
\toprule
\multicolumn{1}{c}{\multirow{2}{*}{Model}} & \multicolumn{3}{c}{\textbf{5-way}} & \multicolumn{3}{c}{\textbf{10-way}} \\ \cmidrule(l){2-7} 
\multicolumn{1}{c}{}                       & 1-shot     & 2-shot    & 3-shot    & 1-shot     & 2-shot     & 3-shot    \\ \midrule
\textbf{Matching Net (random)} & 69.8\%$_{\pm 0.10}$     & 93.1\%$_{\pm 0.07}$    & 98.5\%$_{\pm 0.04}$ & 67.3\%$_{\pm 0.10}$  & 91.2\%$_{\pm 0.06}$ & 97.6\%$_{\pm 0.06}$    \\
\textbf{Matching Net (balanced)} & 97.9\%$_{\pm 0.07}$     & 98.9\%$_{\pm 0.07}$    & 99.2\%$_{\pm 0.06}$    & 96.5\%$_{\pm 0.04}$     & 98.3\%$_{\pm 0.03}$    & 98.7\%$_{\pm 0.05}$    \\
\textbf{Active MN} & 97.4\%$_{\pm 0.11}$     & 99.0\%$_{\pm 0.08}$    & 99.3\%$_{\pm 0.03}$    & 94.3\%$_{\pm 0.24}$    & 98.0\%$_{\pm 0.07}$     & 98.5\%$_{\pm 0.06}$    \\
\midrule
\textbf{Min-Max-Cos} & 97.4\%$_{\pm 0.11}$     & 99.3\%$_{\pm 0.02}$    & 99.4\%$_{\pm 0.04}$    & 93.5\%$_{\pm 0.11}$    & 98.4\%$_{\pm 0.02}$     & 98.8\%$_{\pm 0.03}$    \\ \bottomrule
\end{tabular}
\vspace{-0.25cm}
\end{table*}

\begin{figure}[t]
\begin{center}
\vspace{-0.1cm}
\hspace{-0.5cm}
\includegraphics[width=0.9\linewidth]{./og_annot.pdf}
\vspace{-0.25cm}
\caption{A rollout of our active learning policy for Omniglot, using a support set of 20 items from 10 different classes with 2 items per class. Each row represents the support set at different active iterations. For visualization purposes, each column represents a class. Unlabeled items have white background while selected items have black background. Here, the model behaves intelligently, by selecting at each step an item with a yet-unseen label.}
\label{fig:og_rollout}
\end{center}
\vspace{-0.5cm}
\end{figure}

We run our first experiments on the Omniglot dataset~\citep{lake2015human} consisting of 1623 characters from 50 different alphabets, each hand-written by 20 different people.
Following~\citet{vinyals2016matching}, we divide the dataset into 1200 characters for training and keep the rest for testing. When measuring test performance, our model interacts with characters it did not encounter during training.
%
% here, specifications about the omniglot encoder
%

For the context-free embedding function we use a three-layer convolutional network. The first two layers use $5 \times 5$ convolutions with 64 filters and downsample with a double stride. The third layer uses a $3 \times 3$ convolution with 64 filters and no downsampling. These layers produce a $7 \times 7 \times 64$ feature map that we flatten and pass through a fully connected layer. All convolutional layers use the leaky ReLU nonlinearity \cite{Maas2013}.

We setup $N$-way, $K$-shot Omniglot classification as follows. We randomly pick $N$ character classes from the available train/test classes. Then, we build a support set by randomly sampling 5 items for each character class,~e.g. in the 5-way setting, there are 25 items in the support set. The held-out set is always obtained by randomly sampling 1 item per class.
In our active learning setting, $K$-shot is proportional to how many labels the model can acquire. In the $N$-way, $K$-shot setting, the model asks for $NK$ labels before performing held-out prediction. For example, in 5-way, 1-shot classification, the model asks for 5 labels. Following each label query, we also measure anytime performance of the fast prediction module on the items remaining in $S^u_t$. Note that the 1-shot setting is particularly challenging for our model, as it needs to ask for different classes at each step, and the ability to identify missing classes is limited by the accuracy of the underlying one-shot classifier.

We compare our active learner to four baselines. To compute a pessimistic estimate of potential performance, we use a matching network where we label $NK$ items chosen at random from the full support set (Matching Net (random)). As the labels are randomly sampled, it is possible that a given class is never represented among the labeled items and the model cannot classify perfectly, even in principle. To compute an optimistic estimate of potential performance, we measure the ``ideal'' matching network accuracy by labeling a class-balanced subset of items from the full support set (Matching Net (balanced)). This baseline represents a highly-performant policy that the active learner can, in principle, learn. For the last baseline (Min-Max-Cos), we formulate a heuristic policy. At each active learning step, we select the item which has minimum maximum cosine similarity to unlabeled items in the support set. This heuristic selects item that are different from each other, a strategy well-suited to the Omniglot classification task where items are drawn from a consistent set of underlying classes.

We report the results in Table~\ref{tab:res_kway_kshot}. Matching Networks operating on a randomly sampled set of labels suffer the most in 1-shot scenarios, where the probability of all classes being represented is particularly low (especially in the 10-way case). Overall, the active policy nearly matches the performance of the optimistic balanced Matching Network baseline. Degradation of performance by 2.2\% is observed for the 1-shot, 10-way case. This is not surprising since augmenting the number of classes in the support set, while keeping the number of shots fixed, considerably increases the difficulty of the problem for the active learner.
Figure~\ref{fig:og_rollout} shows a roll-out of the model policy in the 10-way setting.

Figure~\ref{fig:mega_plot} provides results for the more challenging setting of 20-way classification. We tested two properties of our model: its anytime performance beyond the 1-shot setting, and its ability to generalize to problems with more classes than were seen during training. The model performed well on 20-way classification, and quickly approached the optimistic performance estimate after acquiring more labels. We also found that policies trained for as little as 10-way classification could generalize to the 20-way setting.

Our model relies on a number of moving parts. When designing the architecture, we followed the simple approach of minimizing changes to the original Matching Network from \citet{vinyals2016matching}. We now provide ablation test results for several parts of our model. In the 10-way, 1-shot setting accuracy dropped from 94.5 to 86.0 when we removed attention temperature from the fast prediction module. Reducing the number of matching steps from 3 to 2 or 1 had no significant effect in this setting. Removing the context-sensitive encoder also had no significant effect. Streamlining our architecture is clearly a useful topic for future work aimed at scaling our approach to more realistic settings.

\begin{figure*}
\begin{center}
\includegraphics[width=0.38\linewidth]{./mega_plot_type1.pdf}
\includegraphics[width=0.38\linewidth]{./og_gen_test_20_type1.pdf}
% \vspace{-0.2cm}
\caption{Experiment results for our model and baselines on Omniglot. The left plot shows how prediction accuracy improves with the number of labels requested in a challenging 20-way setting. After 20 label requests~(corresponding to a 20-way, 1-shot problem), the active policy outperforms random policy and random MN baselines, but is inferior to the balanced MN. After 30 labels, the active policy nearly matches the performance of the balanced MN using 40 labels (20-way, 2-shot). The right plot shows the number of unique labels with respect to the number of requested labels for models trained on problems with 5-20 classes, and tested on 20-way classification. This gives an idea of how models search for labels from unseen groups and generalize to problems with different numbers of classes.}
\label{fig:mega_plot}
\end{center}
\vspace{-0.5cm}
\end{figure*}

\subsection{MovieLens}
\label{sec:movielens}
%!TEX root = main.tex

\begin{figure}[t]
%\vspace{-0.5cm}
\begin{center}
\includegraphics[width=0.8\linewidth]{./rmse_per_step_type1.pdf}
\caption{Performance of the model and baselines measured with RMSE on the Movielens dataset.}
\label{fig:mlens_plot}
\end{center}
\vspace{-0.5cm}
\end{figure}

\subsubsection{Setup}
We test our model in the ``cold-start'' collaborative filtering scenario using the publicly available MovieLens-20M dataset.\footnote{Available at \url{http://grouplens.org/datasets/movielens/}}
The dataset contains approximately 20M ratings on 27K movies by 138K users. The ratings are on an ordinal 10-point scale, from 0.5 to 5 with intervals of 0.5. We subsample the dataset by selecting 4000 movies and 6000 users with the most ratings. After filtering, the dataset contains approximately 1M ratings. We partition the data randomly into 5000 training users and 1000 test users. The training set represents the users already in the system who are used to fit the model parameters. We use the test users to evaluate our active learning approach. For each user, we randomly pick 50 ratings to include in the support set (movies that the user can be queried about) and 10 movies and ratings for the held-out set. We ensure that movies in the held-out set and in the support set do not overlap. We train our active learner to minimize the mean-squared error (MSE) with respect to the true rating. We adapt the prediction modules of our model to output the rating for a held-out item as follows: we compute a convex combination of the ratings of ``visible'' movies in the support set (the movies the active learner has already queried about), where the weights are given by the final attention step of the slow predictor. Although more complex strategies are possible, we empirically found this simple strategy to work well in our experiments. 
For evaluation, we sample 25000 episodes~(each comprising 50 support ratings and 10 held-out ratings from some user) from the test set and we compute the average per-user root mean-squared error (RMSE). We report the average performance obtained by 3 runs with different random seeds.

\subsubsection{Movie Embeddings}
For each movie, we pretrain an embedding vector by decomposing the full user/movie rating matrix using a latent factor model~\citep{koren2010factor}. This process only uses the training set. For each user $u$ and movie $m$, we estimate the true rating $r_{u, m}$ with a linear model $\hat{r}_{u, m} = x_u^\top x_m + b_u + b_m + \beta$, where $x_u, x_m$ are the user and movie embedding respectively and $b_u, b_m, \beta$ are the user, movie, and global bias, respectively. We train the latent factor model by minimizing the mean squared error between true rating $r$ and predicted rating $\hat{r}$. We use the trained $x_m$ as input representations for the movies throughout our experiments.

\subsubsection{Results}
In Figure~\ref{fig:mlens_plot} we report the results of our active model against various baselines.
The Regression baseline performs regularized linear regression on movies from the support set whose ratings have been observed incrementally in random order.
Because of the small amount of training data, for each additional label we tune the regularization parameter by monitoring performance on a separate set of validation episodes.
The Gaussian Process baseline selects the next movie to label in proportion to the variance of the predictive posterior distribution over its rating.
This gives an idea of the impact of using MN one-shot capabilities rather than standard regression techniques.
The Popular-Entropy, Min-Max-Cos, and Entropy Sampling baselines train our model end-to-end, but using fixed selection policies.
Specifically, we train our architecture end-to-end, but instead of training an active learning policy through the select module we choose items from the support set incrementally according to a heuristic policy.
This gives an idea of the importance of learning the selection policy.
The Popular-Entropy policy, adapted from the cold-start work of~\citet{rashid2002getting}, scores each item in the support set a priori, according to the logarithm of its popularity multiplied by the entropy of the item's ratings measured across users.
% It then iterates through the support set selecting items with probability inversely proportional to their score.
This strategy aims to first collect the ratings for those movies that are both popular and have been rated differently by different users.
Although it is simplistic, the policy achieves competitive performance for bootstrapping a system from a cold-start setting~\cite{elahi2016survey}.
The Min-Max-Cos policy is identical to the synonymous baseline used for Omniglot,~i.e. it selects the unrated movie which has minimum maximum cosine similarity to any of the rated movies.
Roughly, this selects the unrated movie which differs most from the rated movies. Entropy Sampling selects movies in proportion to rating prediction entropy.

The active policy learned end-to-end outperforms the baselines in terms of RMSE, particularly after requesting only the first few labels.
After 10 ratings, our model achieves an improvement of 2.5\% in RMSE against the best baseline.
Unsurprisingly, the gap diminishes with a higher number of labels requested.
After observing 5 labels, the Popular-Entropy baseline and our architecture equipped with the Min-Max-Cos heuristic converge toward the active policy but never quite match it.
For MovieLens, where labels are user-dependent and not tied to an underlying class, a data-driven selection policy may adapt better to the task.
This contrasts with the Omniglot setting, where there is no aspect of personalization and Active MN and Min-Max-Cos perform similarly.
The Min-Max-Cos heuristic is designed to not select items similar to those it has already seen, but selecting similar items can be beneficial in personalized settings~\cite{elahi2016survey}.

% From Figure~\ref{fig:mlens_plot}(b) it is clear that the active policy also outperforms the baselines in terms of mean absolute error (MAE), defined as the average of $|\hat{r} - r|$ over ratings. This measure represents the mean shift of each model with respect to the true rating.

\section{Conclusion}
\label{sec:conc}
% \vspace{-0.5em}
\section{Conclusion}
% \vspace{-0.5em}
Recent advances in multimodal single-cell technology have enabled the simultaneous profiling of the transcriptome alongside other cellular modalities, leading to an increase in the availability of multimodal single-cell data. In this paper, we present \method{}, a multimodal transformer model for single-cell surface protein abundance from gene expression measurements. We combined the data with prior biological interaction knowledge from the STRING database into a richly connected heterogeneous graph and leveraged the transformer architectures to learn an accurate mapping between gene expression and surface protein abundance. Remarkably, \method{} achieves superior and more stable performance than other baselines on both 2021 and 2022 NeurIPS single-cell datasets.

\noindent\textbf{Future Work.}
% Our work is an extension of the model we implemented in the NeurIPS 2022 competition. 
Our framework of multimodal transformers with the cross-modality heterogeneous graph goes far beyond the specific downstream task of modality prediction, and there are lots of potentials to be further explored. Our graph contains three types of nodes. While the cell embeddings are used for predictions, the remaining protein embeddings and gene embeddings may be further interpreted for other tasks. The similarities between proteins may show data-specific protein-protein relationships, while the attention matrix of the gene transformer may help to identify marker genes of each cell type. Additionally, we may achieve gene interaction prediction using the attention mechanism.
% under adequate regulations. 
% We expect \method{} to be capable of much more than just modality prediction. Note that currently, we fuse information from different transformers with message-passing GNNs. 
To extend more on transformers, a potential next step is implementing cross-attention cross-modalities. Ideally, all three types of nodes, namely genes, proteins, and cells, would be jointly modeled using a large transformer that includes specific regulations for each modality. 

% insight of protein and gene embedding (diff task)

% all in one transformer

% \noindent\textbf{Limitations and future work}
% Despite the noticeable performance improvement by utilizing transformers with the cross-modality heterogeneous graph, there are still bottlenecks in the current settings. To begin with, we noticed that the performance variations of all methods are consistently higher in the ``CITE'' dataset compared to the ``GEX2ADT'' dataset. We hypothesized that the increased variability in ``CITE'' was due to both less number of training samples (43k vs. 66k cells) and a significantly more number of testing samples used (28k vs. 1k cells). One straightforward solution to alleviate the high variation issue is to include more training samples, which is not always possible given the training data availability. Nevertheless, publicly available single-cell datasets have been accumulated over the past decades and are still being collected on an ever-increasing scale. Taking advantage of these large-scale atlases is the key to a more stable and well-performing model, as some of the intra-cell variations could be common across different datasets. For example, reference-based methods are commonly used to identify the cell identity of a single cell, or cell-type compositions of a mixture of cells. (other examples for pretrained, e.g., scbert)


%\noindent\textbf{Future work.}
% Our work is an extension of the model we implemented in the NeurIPS 2022 competition. Now our framework of multimodal transformers with the cross-modality heterogeneous graph goes far beyond the specific downstream task of modality prediction, and there are lots of potentials to be further explored. Our graph contains three types of nodes. while the cell embeddings are used for predictions, the remaining protein embeddings and gene embeddings may be further interpreted for other tasks. The similarities between proteins may show data-specific protein-protein relationships, while the attention matrix of the gene transformer may help to identify marker genes of each cell type. Additionally, we may achieve gene interaction prediction using the attention mechanism under adequate regulations. We expect \method{} to be capable of much more than just modality prediction. Note that currently, we fuse information from different transformers with message-passing GNNs. To extend more on transformers, a potential next step is implementing cross-attention cross-modalities. Ideally, all three types of nodes, namely genes, proteins, and cells, would be jointly modeled using a large transformer that includes specific regulations for each modality. The self-attention within each modality would reconstruct the prior interaction network, while the cross-attention between modalities would be supervised by the data observations. Then, The attention matrix will provide insights into all the internal interactions and cross-relationships. With the linearized transformer, this idea would be both practical and versatile.

% \begin{acks}
% This research is supported by the National Science Foundation (NSF) and Johnson \& Johnson.
% \end{acks}

{\footnotesize
\bibliography{ggrefs}
\bibliographystyle{icml2017}
}

\end{document} 

