\section{Conclusion}\label{sec:conclusion}
In this paper, we described a new approach for solving the problem of grouping stack traces that uses not only the information about the values of similarity between stack traces, but also the information about their time of occurrence.
We evaluated our approach on the open-source NetBeans dataset, but also collected a new dataset from the proprietary data of JetBrains, a large software engineering company.
The implementation of our approach is available online on GitHub: \url{https://github.com/nkarasovd/AggregationModel}.
Upon acceptance, we plan to publish our dataset to facilitate further research and its industrial application.

Our experiments have demonstrated the superiority of our approach over the state-of-the-art methods by $15$ and $8$ percentage points in Recall Rate Top-$1$ metric on both the open-source NetBeans data and our dataset, respectively.
Also, using simpler $k$-NN-based approaches did not allow us to obtain the same increase in performance. 

Our work can be continued in different directions.
Firstly, we plan to improve the \ag itself, experimenting with different features.
Secondly, we want to experiment with a neural network architecture that would create embeddings of stack traces. This will allow us to create an aggregated representation of a group using the embeddings of stack traces in it and the incoming stack trace.
Finally, it is of great interest to improve the results of different models for real-world applications.
In this regard, the provided JetBrains dataset will be of particular use, since both types of aggregation demonstrated smaller improvements on such data.