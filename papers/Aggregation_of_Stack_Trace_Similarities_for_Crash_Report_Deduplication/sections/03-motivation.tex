\section{Motivating example}\label{sec:motivation}

Consider the following example, presented in \Cref{fig:hists}.
This example was taken from the NetBeans data, a labeled dataset collected in our previous work~\cite{s3m}. 
Using the approach proposed by Lerch and Mezini~\cite{lerch}, we calculated the values of the similarity between a potential incoming stack trace and all stack traces in all groups, two of which are shown in the figure. 
It can be seen that the values of similarity differ among themselves very strongly inside each group.
For $Group_1$, the stack trace with the highest similarity to the incoming one (the right-most one) is very different from the rest of the group, indicating that it might not be suitable to make a judgement about the entire group.
The values of the maximum similarity for the $Group_1$ and $Group_2$ are equal to $4211$ and $4044$, respectively.
Thus, if selecting the group based on the most similar stack trace, the stack trace should be assigned to $Group_1$.
However, in this case, $Group_2$ is the \textit{ground truth} group, into which the stack trace should actually be defined based on the historical data.
This example clearly demonstrates the drawbacks of the most common way of assigning groups.

\begin{figure}[t]
\centering
    \includegraphics[width=\columnwidth]{figures/histograms.pdf}
    \centering
    \vspace{-0.4cm}
    \caption{An example distributions of similarity values of a certain incoming stack trace to stack traces of two groups.}
    \label{fig:hists}
\end{figure}

We believe that if we use the information about the values of all the similarities and the information about the structure of the group, as well as the time of occurrence of the stack traces, then this will avoid an error and correctly assign the incoming stack trace to $Group_2$.
