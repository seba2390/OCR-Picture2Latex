%% JWS example using Elsevier elsart classes, adapted by Tim 
%% Finin, finin@cs.umbc.edu

%% comment out one of these to get preprint vs. journal format

%\documentclass{elsart}       %% one column for review, preprint
\documentclass{elsart3p}    %% two columns for publication

\usepackage{graphicx,amssymb}

\renewcommand\floatpagefraction{.2}
\makeatletter
\def\elsartstyle{%
    \def\normalsize{\@setfontsize\normalsize\@xiipt{14.5}}
    \def\small{\@setfontsize\small\@xipt{13.6}}
    \let\footnotesize=\small
    \def\large{\@setfontsize\large\@xivpt{18}}
    \def\Large{\@setfontsize\Large\@xviipt{22}}
    \skip\@mpfootins = 18\p@ \@plus 2\p@
    \normalsize
}
\@ifundefined{square}{}{\let\Box\square}
\makeatother

%% tell Latex where to look for files menioned in \includegraphics
\graphicspath{{jwsGraphics/}}




\begin{document}

\begin{frontmatter}

%% This TITLE has a note reference (titleNote) whose 
%% contants is specified about 10 lines later.

\title{LaTex submissions for \\ the Journal of Web Semantics\thanksref{titleNote}}

%% Here are the three AUTHORS.  Note the separating commas
%% with no and.

\author[umbc]{Tim Finin\thanksref{fininGrants}\corauthref{C}},
\ead{finin@cs.umbc.edu}
\author[man]{Carole Goble},
\ead{carole.goble@manchester.ac.uk}
\author[osaka]{Riichiro Mizoguchi}
\ead{miz@ei.sanken.osaka-u.ac.jp}

%% these specify any 'notes' for the title and authors,
%% including 'thanks' that acknowledge support, etc. and an
%% indication of who the corresponding author is if there's
%% more than one author.

\thanks[titleNote]{Adapted from a document written by Simon Pepping}
\thanks[fininGrants]{Partisl support provided by NSF award NSF-ITR-IIS-0326460} 
\corauth[C]{Corresponding author. Tel: +1 410 455 3522}

\medskip

%% Here are the addresses refereenced for the authors.  If
%% all of the authors are from the same institution you need
%% not have the lables for the author and address commands.

\address[umbc]{University of Maryland, Baltimore County, Baltimore MD 21250, USA}
\address[man]{University of Manchester, Manchester, M13 9PL, UK}
\address[osaka]{Osaka University, 8-1 Mihogaoka, Ibaraki, Osaka, 567-0047 Japan}

\begin{abstract} 

We discuss the preparation of articles for the Journal of
Web Semantics using Elsevier's {\em elsart} and the
{\em elsart3p} document styles with numbered style
bibliographic references.  The first is a single column
format suitable for preprints and the second closely
approximates the format used to prepare the journal.  We
recommend that you use {\em elsart3p} to check that your
submission is of appropriate length and that figures,
tables, and other forms of non-running text fit into the
format used by JWS.  This document was adapted from an
example written by Simon Pepping of Elsevier.

\end{abstract}

%% you choose your own keywords

\begin{keyword}
 Journal of Web Semantics, Latex, template, {\em elsart}, {\em elsart3p}
\end{keyword}

\end{frontmatter}


\section{Introduction}
\label{intro}

This article describes how to prepare articles with
Elsevier's {\em elsart} and {\em elsart3p} document
classes.  {\em elsart3p} produces output that is very close
what is actually used by Elsevier to prepare your paper.
Using this allows you to have a very accurate idea of how
your paper will appear in the JWS. {\em elsart} is a single
column format and is useful for preparing a preprint.  For
more general information about \LaTeX{}, see Leslie
Lamport's manual \cite{lamport1994}.

\section{The Journal of Web Semantics}
\label{JWS}

The Journal of Web Semantics is an interdisciplinary journal based on
research and applications of various subject areas that contribute to
the development of a knowledge-intensive and intelligent service
Web. These areas include: knowledge technologies, ontology, agents,
databases and the semantic grid, obviously disciplines like
information retrieval, language technology, human-computer interaction
and knowledge discovery are of major relevance as well. All aspects of
the Semantic Web development are covered. The publication of
large-scale experiments and their analysis is also encouraged to
clearly illustrate scenarios and methods that introduce semantics into
existing Web interfaces, contents and services. The journal emphasizes
the publication of papers that combine theories, methods and
experiments from different subject areas in order to deliver
innovative semantic methods and applications.

\begin{figure}[t]
\centering \includegraphics*[width=0.6\columnwidth]{jwscover}
\caption{This figure of the JWS cover will appear in a
  column and its width scaled to be 60\% of the column's
  width.}
\label{fig:jws}
\end{figure}

The Journal of Web Semantics addresses various prominent
application areas including: e-business, e-community, knowledge
management, e-learning, digital libraries and e-sciences.  It
publishes high quality papers that cover theoretical issues,
methods, tools, system descriptions and applications. Papers
should be normal journal-length or short, descriptive papers
e.g. systems or applications. Special sections will cover
conference and workshop reports as well as book reviews.

\subsection{JWS Web site}

The Journal of Web Semantics features a multi-purpose web
site\footnote{Currently http://www.semanticwebjournal.org/ but
expected to change in summer 2007}. Accepted papers are
immediately available online (associated with appropriate
metadata) and readers are able to annotate documents with
additional comments and links to related material. The unique
character of the journal makes it possible to extend the scope
beyond traditional journals. Open-source software and tools that
help to advance the field more rapidly, will be published and
demonstrated based on a rigorous review and selection
process. Submissions are especially encouraged that demonstrate
the amalgamation of papers and code-content that goes beyond the
range of traditional print journals.  The web site itself is also
available to interested researchers as a test site for developing
and refining Semantic Web technologies and for conduction of user
studies (please contact the Online Editor-in-Chief for further
information).

\subsection{Topic areas}

The Journal of Web Semantics includes, but is not limited to, the
following major technology areas: the Semantic Web; knowledge
technologies; ontology; agents; databases; semantic grid and
peer-to-peer technology; information retrieval; language
technology; human-computer interaction; knowledge discovery and
Web standards.  Major application areas that are covered by the
Journal of Web Semantics are: eBusiness; eCommunity; knowledge
management; eLearning; digital libraries and e-Science.

\section{Latex files}

Elsevier has prepared the following \LaTeX{} support files
for authors, as shown in table~\ref{table:files}.  The files
are freely available from Elsevier's Author
Gateway\footnote{http://authors.elsevier.com/locate/latex}.

\begin{table*}[tb]

\begin{center}


\begin{tabular*}{0.85\textwidth}{ @{\extracolsep{\fill}} | c | l |} \hline
  {\em elsart.cls} &  single column preprint format that is good for a review copy \\ \hline
  {\em elsart3p.cls} & two column journal format approximates how Elsevier will typeset your article \\ \hline
  {\em elsart-num.bst} & JWS's bibliography style -- numbered references listed alphabetically by author  \\ \hline
\end{tabular*}

\caption{The key latex files you will need to put in your
  directory if they are not already installed on your
  system. This table will run across both columns and use
  85\% of the width and appear at the at top or bottom of a page.}

\label{table:files}

\end{center}
\end{table*}


Elsevier's \LaTeX{} files can also be obtained from one of
the servers of the Comprehensive TeX Archive Network (CTAN),
a mirrored network of FTP servers, with sever web
interfaces\footnote{See http://www.tex.ac.uk and
http://www.ctan.org.}. The network is widely
mirrored\footnote{http://www.tug.org/tex-archive/CTAN.sites}
and holds up-to-date copies of all the public-domain
versions of \TeX, \LaTeX, Metafont and ancillary programs.
CTAN is not related to Elsevier, and that Elsevier's author
support cannot accept complaints or answer questions about
the availability of any CTAN server.  The non-Elsevier macro
packages recommended later in this document and many other
useful macro packages can also be obtained from CTAN.  In
the following sections we show how you may use the {\em
elsart} document class.

\section{Options}

The {\em elsart} document class enables the following options:

\begin{description}
  
\item[doublespacing, reviewcopy] This is a single option with two
  names to obtain double line spacing.
  
\item[seceqn, secthm] The option \texttt{seceqn} numbers the
  equation environments per section. The option \texttt{secthm}
  does the same for the \texttt{thm} environment. In elsart all
  predefined theorem environments except Algorithm, Note, Summary
  and Case use the same counter as the \texttt{thm} environment.

\item[draft, final] As in many other document classes, these are
  options to produce draft and final layout. In the draft layout
  you will see warnings for overfull boxes. You also need draft
  layout to test your formulas on a narrower display width, see
  option \texttt{narrowdisplay}.
  
\item[narrowdisplay] The Journal of Web Semantics uses a two
  column format. Because the preprint layout uses a larger line
  width than such columns, the formulas are too wide for the line
  width in print. In draft mode (see the \texttt{draft} option)
  you can use the \texttt{narrowdisplay} option to force a
  narrower width for displayed formulas. The width is roughly
  equal to the column width of the printed journals, compensated
  for the larger font size of the preprint layout. The
  \texttt{narrowdisplay} option is ineffective with packages
  which redefine the equation environments, such as
  \texttt{amsmath}.
  
  The \texttt{narrowdisplay} option is especially useful for
  journals for which the articles are printed from the author's
  \LaTeX{} file.  When you break your formulas such that they fit
  in the narrow column width, the typesetter will be able to
  retain most of your breaks. Article for other journals are
  printed after transformation to an XML file.  For such journals
  the formula layout in the \LaTeX{} file is always lost in the
  transformation. The narrow display width is obtained by giving the formulas a
  larger indent. Too wide formulas in the one-line display
  environments \texttt{equation} and \texttt{displaymath} will
  show an overfull rule.

\end{description}

\section{Print layout}
\label{printlayout}

Elsevier also makes available document classes that roughly
reproduce the layout of the printed journals. The JWS format
is best approximated by the class {\em elsart3p} which uses
a text width 39 picas (164 mm), text height 51 lines, and
twocolumns.  This class is used in the same way as
elsart. If you prepared an article for elsart, you can run
it with one of these print layout classes without changes to
the markup. In fact, they use elsart and you must have
elsart on your system as well.  Note that the layout is only
roughly the same as that of the printed journal. One major
source of differences is the font. The printer uses a
different font with different character widths, which will
cause deviations in layout. There are various other sources
of small differences.

\section{Frontmatter}
\label{frontmatter}

The \texttt{elsart} document class has a separate
\texttt{frontmatter} environment for the title, authors,
addresses, abstract and keywords.  Various commands are used
to specify the consitutent parts.

\begin{itemize}

\item \verb|\title|: The title command works as in standard
\LaTeX, e.g.  \verb|\title{Semantic Pudding}|.

\item \verb|\author|: The author command differs from the
standard in that it contains only one author and no address.
Multiple authors go into multiple \verb|\author| commands,
separated from each other by commas.  The address goes into
a separate \verb|\address| command. 

\item \verb|\address|: Here goes the address,
e.g. \verb|\address{Baltimore, MD USA}|.

\item \verb|\thanks| and \verb|\thanksref|: 
These provide footnotes to the title, authors and addresses. 
The \verb|\thanksref| command takes a label: \verb|\thanksref{label}|
to relate it to the \verb|\thanks| command with 
the same label: \verb|\thanks[label]|. There can be several
references to a single \verb|\thanks| command. Example:\\
\verb|\title{Model\thanksref{titlefn}}| and\\
\verb|\thanks[titlefn]{Supported by grants.}|

\item \verb|\corauth| and \verb|\corauthref|: 
These provide footnotes to mark the corresponding author and the
correspondence address. They are used in the same manner as
\verb|\thanks| and \verb|\thanksref|. Example:\\
\verb|\author{A. Name\corauthref{cor}}| and\\
\verb|\corauth[cor]{Corresponding author. Address: ... .}|

\item \verb|\ead|:
This command should be used for the email address or the URL of the
author. It refers to the `current author', i.e., the author last
mentioned before the command.
When it holds a URL, this should be indicated by setting the
optional argument to `url'. Example:
\verb|\ead{s.pepping@elsevier.com}|,
\verb|\ead[url]{authors.elsevier.com/locate/latex}|.

\end{itemize}

It is not necessary to give a \verb|\maketitle| command. The title,
authors and addresses are printed as soon as \TeX{} sees them.

The authors and addresses can be combined in one of two ways:
\begin{itemize}
\item The simplest way lists the authors of one address or one group
  of addresses, followed by the address or addresses, and so on for
  all addresses or groups of addresses.
\item The other way first lists all authors, and then all addresses.
The authors and addresses are related to each other by labels:
\verb|\author[label1]{Name1}| corresponds to
\verb|\address[label1]{Address1}|. Example:
\begin{verbatim}
\author[South]{T.R. Marsh},
\author[Oxford]{S.R. Duck}
\address[South]{University of Southampton, UK}
\address[Oxford]{University of Oxford, UK}
\end{verbatim}
\end{itemize}

If you put the frontmatter in an included file, that file should
contain the whole frontmatter, including its \texttt{begin} and
\texttt{end} commands. Otherwise, the labels of the frontmatter will
remain undefined.


\section{Abstract}

The abstract should be self-contained. Abstracts usually do
not cite references.  If you feel a need to do so, do not
refer to the list of references but rather quote the
reference in full, e.g., Wettig \& Brown (1996, NewA, 1,
17).

\section{Keywords}
\label{keywd}
\enlargethispage*{2.5pc}

In electronic publications a proper classification is more
important than ever. The journal of Web Semantics uses {\em
uncontrolled keywords} provided by authors.  To maximize the
consistency with which such keywords are assigned by different
authors, the following guidelines have been drawn up.

\begin{itemize}
  \item Each keyword (which can be a phrase of more than one word)
   should describe one single concept. Often words like "and"
   or "of" should be avoided.
  \item Avoid very general keywords which become meaningless once
   in a keyword list. Examples to avoid are "action",
   "computer", "Web". Check whether the keywords as a
   whole describe the outlines of the article.
  \item Use natural language: for instance "automatic error
   recovery" rather than "error recovery, automatic".
  \item Use simple nouns and adjectives as much as possible
   (i.e. use "automatic error recovery" rather than
   "recovering errors automatically"). Do not use nouns in the
    plural form.
  \item Avoid the use of abbreviations as much as possible, unless
   an abbreviation is so well-established that the full term
   is rarely used (e.g. use "laser" instead of "Light
   Amplification by Stimulated Emission of Radiation", but use
   "computer aided design" instead of "CAD").
\end{itemize}

Keywords are entered below the abstract in the following way:
\begin{verbatim}
\begin{keyword}
Keyword \sep Keyword
\end{keyword}
\end{verbatim}

\section{Cross-references}
\label{xrefs}

In electronic publications articles may be internally hyperlinked.
Hyperlinks are generated from proper cross-references in the article.
For example, the words Fig. 1 will never be more than simple text,
whereas the proper cross-reference \verb|\ref{mapfigure}| may be
turned into a hyperlink to the figure itself.  In the same way, the
words Ref. [1] will fail to turn into a hyperlink; the proper
cross-reference is \verb|\cite{goble2001}| as in \cite{goble2001}.
Cross-referencing is possible in \LaTeX{} for sections, subsections,
formulae, figures, tables, and literature references.

\section{PostScript figures}
\label{psfigs}

\LaTeX{} and PostScript have had a long and successful
relationship. There are three packages for including PostScript
figures:

\begin{itemize}

\item \texttt{graphics}.  This simple package provides the
command\\ \verb|\includegraphics*[<llx,lly>][<urx,ury>]{file}|.
The \texttt{*} is optional; it enables the PostScript feature of
clipping.  In its simplest form,\\ \verb|\includegraphics{file}|,
it includes the figure in the PostScript file \texttt{file}
without resizing.

\item \texttt{graphicx}.  This package provides the command\\
\verb|\includegraphics*[key--value list]{file}|.  The \texttt{*}
is optional; it enables the PostScript feature of clipping.
Often used keys are: \def\labelitemii{--} 

\begin{itemize}
\item \texttt{scale=.40} to scale the size of the figure with
  40\%,
\item \texttt{width=25pc}, \texttt{height=15pc} to set the width
  or height of the figure,
\item \texttt{angle=90} to rotate the figure over $90^\circ$.
\end{itemize}

\item \texttt{epsfig}.  This package is really the
\texttt{graphicx} package, but it allows one to include
PostScript figures using the familiar commands from the earlier
packages \texttt{epsfig} and \texttt{psfig}.  

\end{itemize}

For detailed information, see the documentation of the
\texttt{graphics} packages, in particular the file
\texttt{grfguide.tex}.

\begin{figure}

\centering \includegraphics*[width=0.95\columnwidth]{swoogle}

\caption{This figure will be scaled to be 95\% of the width of a
single column and also appear in a single column when using either
elsart or elsart3p.}

\label{fig:exmp}
\end{figure}


\begin{figure*}


\centering \includegraphics*[width=0.9\textwidth]{swoogle}


\caption{This figure will be saled to be 90\% of the width of the
text on the page and run across all columns in either elsart or
elsart3p.}

\label{fig:exmp2}
\end{figure*}


\section{Mathematical symbols}
\label{symbols}

Many authors require more mathematical symbols than the few that
are provided in standard \LaTeX. A useful package for additional
symbols is the \texttt{amssymb} package, developed by the
American Mathematical Society.  This package includes such oft
used symbols as \verb|\lesssim| for $\lesssim$, \verb|\gtrsim|
for $\gtrsim$ or \verb|\hbar| for $\hbar$.  Note that your \TeX{}
system should have the \texttt{msam} and \texttt{msbm} fonts
installed.  If you need only a few symbols, such as \verb|\Box|
for $\Box$, you might try the package \texttt{latexsym}.  In the
\texttt{elsart} document class vectors are preferably coded as
\verb|\vec{a}| instead of \verb|\bf{a}| or \verb|\pol{a}|.

\section{The Bibliography}
\label{thebib}

In \LaTeX{} literature references are listed in the
\verb|thebibliography| environment. Each reference is a
\verb|\bibitem|; each \verb|\bibitem| is identified by a label,
by which it can be cited in the text: \verb|\bibitem{ESG96}| is
cited as \verb|\cite{finin94cikm}|.

\section{Final Comments} 
\label{comments}

There is a template article {\em templat-num.tex}, which you can use as a
skeleton for your own article. It is available from Elsevier's Author
Gateway\footnote{http://authors.elsevier.com/locate/latex}.

\nocite{nardi2003,finin94cikm}

\bibliographystyle{elsart-num-sort}

\bibliography{jwsbib}

\end{document}
