\documentclass[a4paper,11pt]{article}
\pdfoutput=1 % if your are submitting a pdflatex (i.e. if you have
             % images in pdf, png or jpg format)

\usepackage{jcappub} % for details on the use of the package, please
                     % see the JCAP-author-manual

\usepackage[T1]{fontenc} % if needed
\usepackage{subcaption}
\usepackage{physics}



\title{\boldmath Effects of Neutrino Masses and Asymmetries on Dark Matter Halo Assembly}


%% %simple case: 2 authors, same institution
%% \author{A. Uthor}
%% \author{and A. Nother Author}
%% \affiliation{Institution,\\Address, Country}

% more complex case: 4 authors, 3 institutions, 2 footnotes
\author[a]{Hiu Wing Wong}
\author[a]{and Ming-chung Chu}

% The "\note" macro will give a warning: "Ignoring empty anchor..."
% you can safely ignore it.

\affiliation[a]{Department of Physics, The Chinese University of Hong Kong, Shatin NT, Hong Kong}
%\affiliation[b]{Another University,\\different-address, Country}
%\affiliation[c]{A School for Advanced Studies,\\some-location, Country}

% e-mail addresses: one for each author, in the same order as the authors
\emailAdd{hwwong@phy.cuhk.edu.hk}
\emailAdd{mcchu@phy.cuhk.edu.hk}





\abstract{Massive cosmological neutrinos suppress the Large-Scale Structure (LSS) in the Universe by smoothing the cosmic over-densities, and hence structure formation is delayed relative to that in the standard Lambda-Cold Dark Matter ($\Lambda$CDM) model.
We characterize the merger and mass accretion history of dark matter halos with the halo formation time $a_{1/2}$, tree entropy $s$ and halo leaf function $\ell(X)$ and measure them using neutrino-involved N-body simulations. 
We show that a non-zero sum of neutrino masses $M_\nu$ delays the $a_{1/2}$ for halos with virial mass between $10^{13} M_\odot$ and $3\times 10^{13} M_\odot$, whereas a non-zero neutrino asymmetry parameter $\eta^2$ has the opposite effect.
While the mean tree entropy $\bar s$ does not depend significantly on either $M_\nu$ or $\eta^2$, the halo leaf function does.  Furthermore, the dependencies of $\ell$ on $M_\nu$ and $\eta^2$ have significant evolution in redshift $z$, with the relative contributions of $M_\nu$ and $\eta^2$ showing a sigmoid-like transition as a function of $z$ around $z \approx 0.6$.
Together with the matter power spectrum, these halo parameters allow us to break the parameter degeneracy between $M_\nu$ and $\eta^2$ so that they can both be constrained in principle.
}
\keywords{cosmological simulations, cosmological neutrinos, neutrino masses from cosmology, neutrino properties}
%\arxivnumber{XXXX.XXXXXX}



\begin{document}
\maketitle
\flushbottom

\section{Introduction}
\label{sec:Introduction}


The goal in top-$\size$ recommendation is to recommend to each
consumer a small set of $\size$ items from a large collection of
items~\cite{cremonesi2010performance}.  For example, Netflix may want
to recommend $\size$ appealing movies to each consumer.  Collaborative
Filtering (CF)~\cite{herlocker2002empirical,lee2012comparative} is a
common top-$\size$ recommendation method.  CF infers user interests by
analyzing partially observed user-item interaction data, such as user
ratings on movies or historical purchase
logs~\cite{kanagal2012supercharging}. The main assumption in CF is that
users with similar interaction patterns have similar interests.


Standard CF methods for top-$\size$ recommendation focus on making  suggestions  that accurately reflect the user's preference history. However, as  observed in previous work,  CF recommendations are generally biased toward  popular items, leading to a rich get richer effect~\cite{vargas2014improving,steck2011item}.  The major reasons for this are \textit{popularity bias} and \textit{sparsity} of CF interaction data (detailed in Section~\ref{sec:related-work}). In a nutshell, to maintain  accuracy, recommendations are generated from the dense regions of the data,  where the popular items lie.  

However,  accurately suggesting popular items, may not be satisfactory for the consumers. For example, in Netflix, an accuracy-focused movie recommender may recommend ``Star Wars: The Force Awakens'' to users who have seen ``Star Wars: Rogue One''.  But, those users are probably already aware of ``The Force Awakens''. Considering additional factors, such as novelty of recommendations,  can lead to more effective suggestions~\cite{cremonesi2010performance,Castells2015,zhang2008avoiding,ziegler2005improving,zhang2012auralist}. 
%Second, accuracy-focused models typically achieve a   overall item-space coverage across their recommendations,  whereas high item-space coverage helps providers of the items increase revenue
%, users satisfaction since they are  likely already aware of or can find these items on their own.  

Focusing on popular items also adversely affects the satisfaction of  the providers of the items. This is because  accuracy-focused models typically achieve a  low overall item space coverage across their recommendations, whereas   high item space coverage helps providers of the items increase their revenue~\cite{vargas2014improving,Castells2015,adomavicius2011maximizing,anderson2006thelongtail, yin2012challenging,adomavicius2012improving}.
%accuracy-focused models typically achieve a

In contrast to the relatively small number of popular items, there are copious  {\it long-tail\/} items that have fewer observations (e.g., ratings) available. More precisely,  using the Pareto  principle (i.e.,~the $80/20$ rule),  long-tail items can be defined as items that generate the lower $20\%$ of observations~\cite{yin2012challenging}. Experimentally we found that these items correspond to almost $85\%$ of the items in several datasets (Sections~\ref{sec:Notation} and \ref{sec:Experiments}). %Table~\ref{tab:DatasetStatsticsSmall})


As previously shown, one way to improve the novelty of top-$\size$ sets is to recommend interesting long-tail items~\cite{cremonesi2010performance,ge2010beyond}.  The intuition  is that since they have fewer observations available,  they are more likely to be unseen~\cite{Kaminskas:2016:DSN:3028254.2926720}.  
 %For example, in online commerce,  newly added items are long-tail items that are yet to be discovered.  
Moreover, long-tail item promotion also results in higher overall coverage of the item space%, which increases profits for providers of the items
~\cite{vargas2014improving,Castells2015,zhang2008avoiding,zhang2012auralist,adomavicius2011maximizing,anderson2006thelongtail,yin2012challenging,jambor2010optimizing}. Because long-tail promotion reduces accuracy~\cite{steck2011item}, there are trade-offs to be explored.


%original submitted to ICDE
%This work studies three aspects of top-$\size$ recommendation: accuracy, novelty, and item-space coverage, and examines their trade-offs. In most previous work, predictions of a base recommendation system are re-ranked to handle their trade-offs~\cite{adomavicius2012improving,jambor2010optimizing,zhang2013personalize,wang2009portfolio}. Due to performance considerations, however, these techniques are not customized per user. For example,  parameters that balance the trade-off between novelty and accuracy are cross-validated at a global level.  This can be detrimental since users have varying preferences for  objectives such as long-tail novelty. We explore how to  automatically infer  user  preference for long-tail novelty, and how to leverage  it to correct  the popularity bias in standard recommender models. Our work does not rely on any additional contextual data, although such data, if available, can help promote newly-added long-tail items~\cite{agarwal2009regression,Saveski:2014:ICR:2645710.2645751}.

This work studies three aspects of top-$\size$ recommendation: accuracy, novelty, and item space coverage, and examines their trade-offs. In most previous work, predictions of a base recommendation algorithm are \textit{re-ranked} to handle these trade-offs~\cite{adomavicius2012improving,jambor2010optimizing,zhang2013personalize,wang2009portfolio}. The re-ranking models are computationally efficient but suffer from two drawbacks. First, due to performance considerations,  parameters that balance the trade-off between novelty and accuracy  are not customized per user. Instead they are cross-validated at a global level.  This can be detrimental since users have varying preferences for  objectives such as long-tail novelty. Second,  the re-ranking methods are often limited to a specific base recommender  that may be sensitive to dataset density. 
As a result, the datasets are pruned and the problem is studied in dense settings~\cite{adomavicius2012improving,ho2014likes}; but real world  scenarios are often sparse~\cite{kanagal2012supercharging,liu2017experimental}.   
% Because  dataset density can impact the performance of most base recommenders (like R-SVD), which in turn affects the performance of the re-ranking model, 

\iffalse
We address these limitations by directly inferring  user  preference for long-tail novelty  from interaction data.  This  allows us to customize the re-ranking  per user, and design a \textit{generic} framework, which resolves the second problem. In particular, since the long-tail novelty preferences are estimated independently of any base  recommender model, we can  plug-in an appropriate base recommender w.r.t. the dataset sparsity.% including ones that are more suitable for sparse settings.  

Modelling  user  preference for  long-tail novelty using only item popularity statistics, e.g., the average popularity of rated items as in~\cite{jugovac2017efficient}, disregards additional information like whether the user found the item interesting and the long-tail preferences of other users  of the items. \iffalse To incorporate them, we introduce the notion of  \emph{item long-tail importance}. Both  user long-tail preferences and item long-tail importance are dependent:  a user has high preference for discovering long-tail items if she is interested in important long-tail items, and an item that is associated with many of these kinds of users is likely to be more important.  We propose a joint optimization framework to directly learn,  from interaction data, both the users' long-tail preferences and the  items' long-tail importance. \fi
We propose an optimization approach that  incorporates  this information and  directly learns,  from interaction data, the users' long-tail novelty preferences.

Next, we use these learned preferences  to design a  top-$\size$ recommendation framework thats is generic, and provides customized balance between accuracy, novelty, and coverage. We refer to it as framework as GANC.  Using GANC, we design a novel algorithm, {\it Ordered Sampling-based Locally Greedy (OSLG)\/}, that relies on the learned long-tail novelty preferences  to scalably correct for popularity bias. Our work does not rely on any additional contextual data, although such data, if available, can help promote newly-added long-tail items~\cite{agarwal2009regression,Saveski:2014:ICR:2645710.2645751}. In summary:
\fi

We address the first limitation by directly inferring  user  preference for long-tail novelty  from interaction data.   Estimating these  preferences  using only item popularity statistics, e.g., the average popularity of rated items as in~\cite{jugovac2017efficient}, disregards additional information, like whether the user found the item interesting or the long-tail preferences of other users  of the items. We propose an approach that  incorporates  this information and  learns the users' long-tail novelty preferences from interaction data.

This approach allows us to customize the re-ranking  per user, and  design a \textit{generic} re-ranking framework, which resolves the second limitation of prior work. In particular, since the long-tail novelty preferences are estimated independently of any base recommender, we can  plug-in an appropriate one w.r.t. different factors, such as the dataset sparsity.

Our top-$\size$ recommendation framework, \textbf{GANC}, is \textbf{G}eneric, and provides customized balance between \textbf{A}ccuracy, \textbf{N}ovelty, and \textbf{C}overage. % Moreover, based on the learned long-tail novelty preferences, we also design a novel algorithm, {\it Ordered Sampling-based Locally Greedy (OSLG)\/}, that relies on the learned long-tail novelty preferences  to scalably correct for popularity bias. 
Our work does not rely on any additional contextual data, although such data, if available, can help promote newly-added long-tail items~\cite{agarwal2009regression,Saveski:2014:ICR:2645710.2645751}. In summary:

%Consider  the following toy example:
\vspace{-0.2cm}
\begin{table}[htb]
\centering
\scriptsize
%\small
\begin{tabular}{ccccccc} 
%\toprule
%&\multirow{2}{*}{}&\multicolumn{7}{c}{Ratings}\\
& & \cellcolor{blue!35}$w_1$ &\cellcolor{blue!18} $w_2$ & $\dots$ &\cellcolor{blue!8} $w_{89}$  &\cellcolor{blue!8} $w_{99}$   
\\
&   &$i_1$&$i_2$&$\dots$&$i_{89}$&$i_{90}$\\ 
\cmidrule(r){3-7} 	 
%\midrule
\cellcolor{red!35}$\theta_1$  &$u_1 $   &5 &   & $\dots$ &  &   \\
\cellcolor{red!28}$\theta_2$  &$u_2$     &5 &    & $\dots$ &  &  \\
 $\theta_3=?$  &$\bf u_3$  &5 &  &   $\dots$ &  &  \\
\cellcolor{red!10}$\theta_4$ & $u_4$  &  &5   & $\dots$ & &\\ 
\cellcolor{red!10}$\theta_5$ & $u_5$  &  & 5  & $\dots$ & &\\ 
$\theta_6=?$  & $\bf u_6$ & &5  &      $\dots$& &  \\ 
 & & $\hdots$  &$\hdots$   &$\hdots$   &$\hdots$   &$\hdots$  \\
%\midrule 
\cmidrule(r){3-7} 	 
\multicolumn{2}{c}{item pop.}  & 3  & 3  & $\dots$ &50&60\\  
%\bottomrule
%$ f_i$    &3  &3  &1  &3  &1  &2  \\  \hline
\end{tabular}
%#.
\caption{Simplified user-item interaction data. The user long-tail novelty preference ($\theta_u$), item long-tail importance weight ($w_i$) are highlighted. Darker colors indicate larger values. } \label{tab:example}
\end{table} 
\vspace{-0.2cm}
\begin{example}  
In Table~\ref{tab:example}, we are interested in estimating $\theta_3$ and $\theta_6$,  the long-tail preference of users $u_3$ and $u_6$ who have each rated a single movie. Additional ratings for other users  are not included here.  Considering only rating information, we observe $i_1$ and $i_2$ are  equally popular $|\mathcal{U}_{i_1}^{\trainset}| = |\mathcal{U}_{i_2}^{\trainset}|=3$, and $r_{31}=5$ and $r_{62}=5$. Using Eq.~\ref{eq:tfidf-risk}  we have $\theta_3 = \theta_6$. However, if we were given the long-tail preferences of the each item's user set, specifically that $u_1$ and $u_2$ have high long-tail preference (darker red), while $u_4$ and $u_5$ have lower long-tail preference (lighter red), we could conclude $i_1$ is a more important long-tail item compared to $i_2$ (indicated by a darker blue shade for $w_1$), and we expect  $\theta_3 \geq \theta_6$.

% On the other hand, if we knew that $u_4$ and $u_5$ have lower long-tail preference, we could conclude $i_2$ is a  less significant long-tail item. Therefore, However, if we  consider the long-tail preferences of other users, we may reason differently.    We need another variable $w_i$ which captures this information. 
%we would conclude that $u_3$ has higher long-tail preference compared to $u_6$, since the users $i_1$ is a more prominent long-tail item. 

% Relying only  on item popularity information, we would  conclude   $u_3$ and $u_6$ have equal long-tail preference, since $i_1$ and $i_2$ are  equally popular. However, considering  the second column,  long-tail preference of users,  long-tail importance for each item,  which captures the long-tail preference of its users. Since  that  both users of $i_1$ have high long-tail preference while  the users of $i_2$ have lower preference,  we may conclude $i_1$ is a more important long-tail item compared to $i_2$. Therefore, $u_3$'s long-tail preference should be at least as large as $u_6$'s preference. Specifically, consider two  items $i_1$ and $i_2$, with the following rating data: $i_1=\{u_1:5, u_2:5, u_3:5 \}$, $i_2=\{u_4:5, u_5:5, u_6:5\}$.  

%Table~\ref{tab:example} shows  simplified rating data. We want an estimate of the long-tail preference of $u_3$ and $u_6$, who have each  rated a single movie.  Relying only  on movie popularity information, we would  conclude   $u_3$ and $u_6$ have similar long-tail preference, since $m_1$ and $m_2$ are  equally popular. However, considering the long-tail preferences of other users of those movies, we may reason differently: since $u_1$ and $u_2$ have high long-tail preference, and $u_4$ and $u_5$ have low long-tail preference, $m_1$ is a more prominent long-tail item compared to $m_2$. Therefore, it is likely that $u_3$ has higher long-tail preference compared to $u_6$.considering the long-tail preferences of other users of those movies, we may reason differently.  For example, 
\label{ex:running}
\end{example}



%------------------------------

\iffalse
\begin{example}
Table~\ref{tab:example} shows rating data for a simplified system. %Note the user-item interaction matrix is sparse.
For this example, we define popular movies as those that have received  three or more ratings; $\{m_1, m_2, m_4\}$ are popular and  $\{m_3, m_5, m_6\}$ are niche movies. We observe $u_1$ and $u_3$  have rated relatively popular movies (risk-averse) while $u_2$ and $u_4$ have rated niche movies (risk-loving). 
\label{ex:running}
\end{example}

\begin{table}[htb]
\centering
\scriptsize
\begin{tabular}{ccccccc} 
\toprule
			&$m_1$ &$m_2$   &$m_3$    &$m_4$   &$m_5$ &$m_6$  \\ \hline 
$u_1 $ &5  &4  & - &-  &-  &-   \\
$u_2$  &-  &-  &-  &-  &5  &5   \\
$u_3$  &-  &4  &-  &5  &-  &-   \\
$u_4$  &-  &-  &3  &-  &-  &4   \\ 
$u_5$  &5  &-  &-  &3  &-  &-   \\ 
$u_6$  &4  &2  &-  &4  &-  &-   \\ 
\bottomrule
%$ f_i$    &3  &3  &1  &3  &1  &2  \\  \hline
\end{tabular}
\caption{User-Movie rating data} \label{tab:example}
\end{table}

It is essential to consider consumer characteristics in designing recommender systems so that they promote long-tail items to the right group of users and spread demand evenly between hit and niche items.  

\fi





%------------------------------
\iffalse
\begin{table}[htb]
\centering
\scriptsize
\begin{tabular}{ccccccc} 
\toprule
			&$m_1$ &$m_2$   &$m_3$    &$m_4$   &$m_5$ &$m_6$  \\ \hline 
$u_1 $ &\textbf{5}  & \textbf{4}  &\textcolor{gray}{ 1.2} &-  &-  &-   \\
$u_2$  &-  &-  &-  &-  & \textbf{5}  &\textbf{5}   \\
$u_3$  &-  &\textbf{4}  &-  &\textbf{5}  &-  &-   \\
$u_4$  &-  &-  &\textbf{3}  &-  &-  &\textbf{4}   \\ 
$u_5$  &\textbf{5}  &-  &-  &\textbf{3}  &-  &-   \\ 
$u_6$  &\textbf{4}  &\textbf{2}  &-  &\textbf{4}  &-  &-   \\ 
\bottomrule
%$ f_i$    &3  &3  &1  &3  &1  &2  \\  \hline
\end{tabular}
\caption{User-Movie rating data} \label{tab:example}
\end{table}
% $\mathcal{P}^1= \{ \mathcal{P}_1^1 \{i_1,i_2,i_3\}, \mathcal{P}_2^1:\{i_2,i_3,i_5\}  \}$
 %$\mathcal{P}^2= \{ \mathcal{P}_1^2: \{i_1,i_2,i_3\}, \mathcal{P}_2^2:\{i_2,i_5,i_6\}  \}$
 %$\mathcal{P}^3= \{ \mathcal{P}_1^3: \{i_7,i_8,i_9\}, \mathcal{P}_2^3:\{i_{10},i_{11},i_{12}\}  \}$
\begin{table}[htb]
\centering
\tiny
\begin{tabular}{ccc} 
\toprule
		&$u_1$&$u_2$  \\ \hline 
$\mathcal{P}^1 $ & $\{i_1,i_2,i_3\}$ & $\{i_2,i_3,i_5\} $ \\
$\mathcal{P}^2$ & $\{i_1,i_2,i_3\}$ & $\{i_2,i_5,i_6\} $ \\
$\mathcal{P}^3$ & $\{i_7,i_8,i_9\}$ & $\{i_{10},i_{11},i_{12} \}$ \\
\bottomrule
%$ f_i$    &3  &3  &1  &3  &1  &2  \\  \hline
\end{tabular}
\caption{Top-$\size$ allocations to users.} \label{tab:paretoExamples}
\end{table}
\fi


\iffalse
When considering long-tail items, it is important to consider consumers' willingness  to explore niche or unpopular items and their propensity towards similar items. In particular, they can be characterized by their  {\it risk degree\/} and {\it focusing degree\/}, respectively.  We compute these estimates  based on historical rating information. The following example further describes these notions in the context of movie rating data. 

\begin{example}  
Table~\ref{tab:example} shows rating data for a simplified system with $6$ users, $6$ movies, and $3$ genres. $m_i^{j}$ implies that movie $m_i$ belongs to genre $j$. Note the user-item interaction matrix is sparse. 
  For this setting, we define popular movies as those that have received  three or more ratings; $\{m_1, m_2, m_4\}$ are popular and  $\{m_3, m_5, m_6\}$ are niche movies. We now profile the users according to their risk and focusing degree. E.g., $u_1$ has rated relatively popular movies belonging to the same genre (risk-averse, high focusing degree); $u_2$ has rated niches movies in the same genre (risk-loving, high focusing degree); $u_3$ has rated popular movies in two different genres (risk-averse, low focusing degree), and $u_4$ has rated niches movies in two different genres (risk-loving, low focusing degree). 
\label{ex:running}
\end{example}
\begin{table}[htb]
\centering
\tiny
\begin{tabular}{ccccccc} 
\toprule
			&$m_1^{1}$ &$m_2^{1}$   &$m_3^{2}$    &$m_4^{3}$   &$m_5^{3}$ &$m_6^{3}$  \\ \hline 
$u_1 $ &5  &4  &-  &-  &-  &-   \\
$u_2$  &-  &-  &-  &-  &5  &5   \\
$u_3$  &-  &4  &-  &5  &-  &-   \\
$u_4$  &-  &-  &3  &-  &-  &4   \\ 
$u_5$  &5  &-  &-  &3  &-  &-   \\ 
$u_6$  &4  &2  &-  &4  &-  &-   \\ 
\bottomrule
%$ f_i$    &3  &3  &1  &3  &1  &2  \\  \hline
\end{tabular}
\caption{User-Movie rating data} \label{tab:example}
\end{table}
It is essential to consider these consumer characteristics in designing recommender systems so that they promote long-tail items to the right group of users and spread demand evenly between the hit and niche items.  
\fi
\iffalse
\begin{center}
\begin{figure*}[tp]
%\scalebox{0.5}{%
\resizebox{1\textwidth}{!}{%
%\small%\addtolength{\tabcolsep}{5pt}% below sums to 8
\begin{tabularx}{1.5\textwidth}{>{\hsize=2.5\hsize}X>{\hsize=2.5\hsize}X>{\hsize=0.5\hsize}X>{\hsize=0.5\hsize}X>{\hsize=0.5\hsize}X>{\hsize=0.5\hsize}X>{\hsize=0.5\hsize}X>{\hsize=0.5\hsize}X}
    \multirow{12}{*}{\includegraphics[scale=0.3]{codeForExample/popularity-movie.png}} & \multirow{12}{*}{\includegraphics[scale=0.3]{codeForExample/scatterplot.png}} & & & & & & \\
%   & &               &       &       &       &       &       \\
    & &\multicolumn{1}{l|}{}               &$m_1^{g1}$   	&$m_2^{g1}$    	&$m_3^{g2}$    &$m_4^{g2}$      &$m_5^{g3}$    \\ \cline{3-8}%\hline
    & &\multicolumn{1}{l|}{u1}          &5  &5  &-  &-   &-  \\
    & &\multicolumn{1}{l|}{u2}    		&-  &-  &4  &4  &5  \\
    & &\multicolumn{1}{l|}{u3}   			&1  &2  &1  &-  &-   \\
    & &\multicolumn{1}{l|}{u4}     		&1  &-  &-  &-  &-  \\
    & &               &       &       &       &       &       \\
    & &               &       &       &       &       &       \\
    & &               &       &       &       &       &       \\
    & &               &       &       &       &       &	\\
    \\
\end{tabularx}}
\caption{User-Movie interaction data a) Popularity-Movie histogram b)Movie genres/clusters c) User-Movie rating data} \label{fig:example}
\end{figure*}
\end{center}
\fi



%We propose a novel approach that allows us to  promote long-tail items in a targeted manner, thereby improving the novelty of top-$\size$ sets, the overall item-space coverage across recommendations, while maintaining reasonable levels of accuracy.

%Next, we integrate these learned preferences  in a generic  top-$\size$ recommendation framework to provide customized balance between accuracy and coverage.

%sequentially make recommendations, while adjusting its parameters with regard to the set of top-$\size$ recommendations made so far. However, since  sequential parameter updates  cause  scalability issues, we propose a sampling based algorithm. This variant of our framework, called {\it Ordered Sampling-based Locally Greedy (OSLG)\/},  allows us to  correct for the popularity bias in recommendations with regard to individual user long-tail preferences. 

%ICDE submission
%Our framework differs with  prior work in the following aspects:  unlike~\cite{adomavicius2011maximizing,adomavicius2012improving,zhang2013personalize,ho2014likes},  the long-tail preference personalization in our framework is learned rather than optimized using cross-validation or parameter tuning. In other words, our personalization method is independent of the underlying base  recommendation models.  Moreover, our framework is  generic. This enables us to  plug-in several base recommenders, and evaluate their  effectiveness without requiring  extensive tuning for the accuracy and coverage trade-off. 


%\vspace{-2.8pt}
\begin{itemize}

\item  We examine various measures for estimating user long-tail novelty preference in Section~\ref{sec:lt-pref} and formulate an optimization problem  to directly learn users' preferences for long-tail  items from interaction data in Section~\ref{sec:learning-lt-pref}. %In addition, we introduce several heuristics for measuring the user preference for less common items from historical rating data.% 

\item  We integrate the user preference estimates into GANC %, a generic re-ranking framework that provides customized balance between accuracy, novelty, and coverage 
(Section~\ref{sec:RiskbasedReranking}), and  introduce {\it Ordered Sampling-based Locally Greedy (OSLG)\/}, a scalable algorithm that relies  on user long-tail preferences to correct the popularity bias (Section~\ref{sec:optimizationAlgorithm}).
%We introduce OSLG, a scalable algorithm that relies  on user long-tail preferences to  maximize item space coverage \textcolor{red}{while maintaining acceptable levels of accuracy} (Section~\ref{sec:optimizationAlgorithm}).

\item   We conduct an extensive empirical study and evaluate performance from  accuracy, novelty, and coverage perspectives (Section~\ref{sec:Experiments}).  We use five  datasets with varying density and difficulty levels. %:  Netflix, MovieTweetings, and MovieLens (100K, 1M, 10M). 
  In contrast to most related work,  our evaluation considers realistic settings that include a large number of infrequent  items and users. %This enables us to study the impact of  data density on the performance trade-offs of several  state of the art top-$\size$ recommendation algorithms. %   %,  and use the all-items ranking protocol~\cite{steck2013evaluation,vargas2014improving}, where performance is measured using all items with train data. to evaluate the performance of several  state of the art top-$\size$ recommendation algorithms 
 
\item Our empirical results confirm that the performance of re-ranking models is impacted by the underlying   base recommender and the dataset density. Our generic approach enables us to easily incorporate a suitable base recommender to devise an effective solution for both dense and sparse settings. In dense settings, we use the same base recommender as existing re-ranking approaches, and we outperform them in accuracy and coverage metrics. For sparse settings, we plug-in a more suitable base recommender, and devise an effective solution that is competitive with existing top-$\size$ recommendation methods in accuracy and novelty. 

%Directly estimating the long-tail novelty preferences allows us to customize re-ranking per user, and  devise a generic framework.   
 
\end{itemize}

Section~\ref{sec:related-work} describes related work. Section~\ref{sec:conclusion} concludes.

\section{Grid-based neutrino simulation}
\label{sec:grid}

To study the neutrino free-streaming effect on LSS, we can include neutrinos as another type of particles in N-body simulations. This particle-based method is accurate in principle but computationally expensive.
Not only will it bring extra particles and interactions, but it also requires more integration steps compared to the Cold Dark Matter (CDM)-only simulation with the same number of particles due to the larger velocity dispersion of the neutrinos. 
On the other hand, the grid-based neutrino-involved simulation includes only CDM particles in the simulation box. The neutrino information is carried by the neutrino over-density field $\delta_\nu$ contained in the Particle-Mesh (PM) grid, which is responsible for the long-range interaction in a Tree-PM code like \texttt{Gadget2} \cite{Gadget2}. 

Another advantage of the grid-based method is that we can also investigate the effects of the chemical potentials of neutrinos, which can be incorporated easily in the simulation through the Fermi-Dirac distribution of the cosmological neutrinos.

\subsection{Linear evolution for neutrino over-density}
The linear equation that governs the evolution of $\delta_\nu$ is \cite{LBE}: 
\begin{equation}
\label{eq:LBE}
    \Tilde{\delta}_\nu (\chi,\mathbf{k}) = \Phi(\mathbf{k}\chi) \Tilde{\delta}_\nu(0,\mathbf{k}) +
    4\pi G \int^\chi_0 a^4(\chi')(\chi-\chi') \Phi[\mathbf{k}(\chi-\chi')]
    [\Bar{\rho}_{cdm}(\chi') \Tilde{\delta}_{cdm} (\chi',\mathbf{k}) + \Bar{\rho}_\nu(\chi') \Tilde{\delta}_\nu (\chi',\mathbf{k})] d\chi'.
\end{equation}


Here, $\Tilde{\delta}_\nu$ ($\Tilde{\delta}_{cdm}$) and $\bar \rho_\nu$  $(\bar\rho_{cdm})$ are the over-density in Fourier space and mean density of neutrinos (CDM), respectively. $\bf k$ is the wave vector, $\chi$ is the co-moving coordinate where $d\chi=dt/a^2(t)$, and $\Phi$ is a special function that will be discussed in Appendix \ref{app:phi}.

Neutrinos cannot cluster below their free-streaming scale, which is larger than their non-linear scales; therefore, the evolution of neutrinos is well described by the linear equation. Although we use a linear equation to describe the evolution of $\delta_\nu$, the non-linear $\delta_{cdm}$ (from N-body simulation) is involved in the evolution of neutrinos. Hence, the non-linearities in structure formation are still fully preserved. 
Previous studies have also shown that both particle-based and grid-based simulations produce consistent results for the matter power spectrum \cite{Carton}. 


\subsection{Total over-density}
Once we obtain the neutrino over-density $\delta_\nu$, the total over-density field $\delta_t$ is then given by:
\begin{equation}
\label{eq:avg}
    \delta_t = (1-f_\nu)\delta_{cdm} + f_\nu \delta_\nu,
\end{equation}
where $f_\nu=\Omega_\nu / \Omega_m$, the ratio of the cosmological neutrino and CDM densities. Although $f_\nu$ is small, the non-linearity in structure formation will mix up different modes, and the final density $\delta_t $ may change by a factor much greater than $(1- f_\nu)$.

\subsection{Implementation}
We implemented the grid-based neutrino method in our own modified version of \texttt{Gadget2}. The detailed procedure is discussed in \cite{Carton}. Here we briefly summarize the steps:

\begin{enumerate}
    \item The initial snapshot generated by \texttt{MUSIC} \cite{music} and initial neutrino power spectrum generated by \texttt{CAMB} are fed into \texttt{Gadget2} as the initial conditions, and thus we have both $\Tilde{\delta}_\nu(0,k)$ and $\Tilde{\delta}_{cdm}(0,k)$.
    
    \item To evolve the system, the CDM particles are drifted first. With a new CDM power spectrum after the drift $\Tilde{\delta}_{cdm}(\chi, k)$, we solve Eq.(\ref{eq:LBE}) iteratively. We use linear interpolation to approximate $\Tilde{\delta}_{cdm}(\chi', k)$ for $\chi' \in [0,\chi]$ as the time difference is usually small between two PM steps.
    
    \item The over-densities are assumed to carry the same phase:
    \begin{equation}
        \Tilde{\delta}_\nu (\chi, {\bf k}) = \frac{\Tilde{\delta}_\nu(\chi, k)}{\Tilde{\delta}_{cdm}(\chi, k)} \Tilde{\delta}_{cdm}(\chi, {\bf k}).
    \end{equation}
    The total over-density field $\Tilde{\delta}_t(\chi, {\bf k})$ is then obtained using Eq.(\ref{eq:avg}). Finally the original $\Tilde{\delta}_{cdm}(\chi, {\bf k})$ is replaced by $\Tilde{\delta}_t(\chi, {\bf k})$ to give the correct PM potential to evolve the CDM particles.
    
    \item $\Tilde{\delta}_\nu(\chi, k)$ and $\Tilde{\delta}_{cdm}(\chi, k)$ are stored as the initial conditions for the next PM calculation, and we iterate back to step 1 until the final time (usually today).
    
\end{enumerate}

\subsection{Simulation parameters}


To specify the fiducial cosmology for our N-body simulation, 6 cosmological parameters are needed. 
They are the physical CDM density $\Omega_ch^2$, the physical baryon density $\Omega_bh^2$, the observed angular size of the sound horizon at recombination $\theta$, the reionization optical depth $\tau$, 
the initial super-horizon amplitude of curvature perturbations $A_s$ at $k$ = 0.05 Mpc$^{-1}$ and the primordial spectral index $n_s$. 
All of them are consistently refitted from the Planck CMB data using the Planck 2018 \texttt{plikHM\_TTTEEE} likelihood for each set of $M_\nu$ and $\eta^2$ values. The parameters relevant to the N-body simulations are listed in Table \ref{tab:simparam}, and their posterior distributions for selected sets of $M_\nu$ and $\eta^2$ are plotted in Figure \ref{fig:post}. 


\begin{table}[!hbt]
		\begin{center}
		% Table itself: here we have two columns which are centered and have lines to the left, right and in the middle: |c|c|
		\begin{tabular}{|c|c|c|c|c|c|c|c|c|c|c|}
			% To create a horizontal line, type \hline
			\hline
			% To end a column type &
			% For a linebreak type \\
			No. &$M_\nu$ & $\eta^2 $ & $H_0$ & $\Omega_c+\Omega_b$ & $\Omega_\nu$ &$\Omega_\Lambda$ & $\sigma_8$ & $n_s$ & $A_s \,(10^{-9})$\\
			\hline
			\hline
%			A0 & 0 & 0 & 67.74 & 0.3097 & 0 & 0.6903 \\
			A1 & 0.06 & 0 & 67.37 & 0.3141 & 0.00140 & 0.6845 & 0.814 & 0.965 & 2.10 \\
			A2 & 0.06 & 0.253 & 68.13 & 0.3107 & 0.00142 & 0.68788 & 0.818 & 0.969 & 2.11\\
			A3 & 0.06 & 1.012 & 70.49  & 0.3009 & 0.00145 & 0.69765 & 0.833 & 0.982 & 2.15 \\
			\hline
			B1 & 0.15 & 0 & 66.43 & 0.3236 & 0.0036 & 0.6728 & 0.794 & 0.964 & 2.10 \\
			B2 & 0.15 & 0.253 & 67.12 & 0.3208 & 0.00365 & 0.67555 & 0.799 & 0.968 & 2.11\\
			B3 & 0.15 & 1.012 & 69.44 & 0.3106 & 0.00375 & 0.68565 & 0.811 & 0.981 & 2.15\\
			\hline
            C1 & 0.24 & 0 & 65.46 & 0.3337 & 0.00595 & 0.66035 & 0.775 & 0.963 & 2.11\\
			C2 & 0.24 & 0.253 & 66.17 & 0.3308 & 0.00600 & 0.6632 & 0.778 & 0.967 & 2.12\\
			C3 & 0.24 & 1.012 & 68.39 & 0.3192 & 0.00618 & 0.67462 & 0.790 & 0.981 & 2.15\\
			\hline
		\end{tabular}
		\caption{\label{tab:simparam} N-body simulation parameters.}
		\end{center}
\end{table}


Simulation snapshots are generated using our modified version of $\texttt{Gadget2}$ to incorporate the neutrino effects. We made 9 runs, each with $1024^3$ particles and a volume of over $(1000 h^{-1}\mathrm{Mpc})^3$ with a mass resolution of $8.15\times10^{10}h^{-1} M_\odot$. 
The initial conditions for the N-body simulations are generated using \texttt{MUSIC} with second-order Lagrangian corrections,
while the initial conditions of CDM and neutrino power spectra are obtained from the transfer function generated by \texttt{CAMB}, at the initial redshift $z=49$. 

To capture the halo assembly history, we stored 128 snapshots between $z=3$ to $z=0$ so that we can track potentially small changes of $a_{1/2}$ due to the neutrinos. The halo catalog is constructed using \texttt{Rockstar} \cite{rockstar}. The halo radius is defined to be the radius where the over-density equals $\Delta=200\rho_c$, where $\rho_c$ is the critical density of the universe. \texttt{Rockstar} is a 6D phase space friends-of-friends halo-finding algorithm, which specializes in identifying subhalos and tracking merger events. 
The halo merger tree is constructed by linking halos across different time steps using \texttt{Consistent-trees} \cite{ct} together with \texttt{Rockstar}. Finally we implement the calculation of $a_{1/2}$, $s$ and $\ell(X)$ with the built-in tool \texttt{read\_tree} inside \texttt{Consistent-trees}.
\section{Halo assembly statistics}
\label{sec:halo}

\subsection{Halo merger tree}
Dark matter halos can grow in two different ways: by accreting nearby matter or annexing other nearby self-bounded halos. A simple illustration of a halo merger tree is shown in Figure \ref{fig:MT}. For every halo existing at scale factor $a=1$ as the root, the halo merger tree branches out for each of its progenitors, reaching to the past and repeating until no progenitor is found. Once we construct the halo merger tree, the assembly history of a halo is specified, and the evolution of every halo property, such as halo mass, spin and concentration is captured.


\begin{figure}[!h]
\begin{center}
\begin{subfigure}[h]{0.48\columnwidth}
        \includegraphics[width=\columnwidth]{PNG/accretion_final.png}
        \caption{Halo assemble by smooth accretion}
\end{subfigure}
\begin{subfigure}[h]{0.48\columnwidth}
        \includegraphics[width=\columnwidth]{PNG/merger_final.png}
        \caption{Halo assemble by major merger}
\end{subfigure}
	    \caption{\label{fig:MT} Illustration of two extreme types of halo merger trees \cite{tree entropy}.} 
\end{center}
\end{figure}



Although the halo merger tree is a powerful tool to visualize how the halos assemble, it is not easy to compare two halo merger trees directly. Therefore, we use three parameters to quantify the characteristics of a halo merger tree:
the rate of growth of the halo mass, the fraction of the mass of a halo coming from mergers, and the number of protohalos merging into a single halo we observe today.
These characteristics of a halo assembly can be captured in the halo formation time $a_{1/2}$, tree entropy $s$, and halo leaf function $\ell(X)$ as we will see.


\subsection{Halo formation time $a_{1/2}$}
To characterize the mass growth rate of a halo, we can record its mass, trace one step back to the merger tree, pick the most massive progenitor (main progenitor) and repeat. It is a reduced representation of the halo assembly history called the mass accretion history (MAH) of the main branch.

We then define the halo formation time $a_{1/2}$ as the latest scale factor at which the main-branch halo reaches half of its current mass, i.e.,
\begin{equation}
    a_{1/2} = \mathrm{Max}(a') \mid M(a') = M(a=1)/2,
\end{equation}
where $M(a')$ is the main-branch halo mass at the scale factor $a'$.
We follow this traditional definition of halo formation time to quantify the rate of mass accretion, since $a_{1/2}$ was shown to be most correlated with the present-day Navarro–Frenk–White (NFW) concentration $c$ independent of the halo mass \cite{Kuan}.

Although we cannot observe the MAH of a particular halo in sky surveys, there is an observational proxy for $a_{1/2}$. 
It is known that the mass fraction of the main substructure $f_{main}$ is tightly correlated with $a_{1/2}$ in high-resolution N-body simulations \cite{Wang_2011_2}; 
the relationship is robust for different masses of the host halos. 
Using halo abundance matching (HAM), we can relate $f_{main}$ with $f_*$, which is the stellar mass fraction of the central galaxy.
The Sloan Digital Sky Survey (SDSS) data \cite{Wang_2011} shows a strong correlation between $f_*$ and galaxy properties such as color and star formation rate \cite{Lim_2015}. 
As a result, the neutrino effects on MAH, quantified by $a_{1/2}$, can be measured by direct observables.

Intuitively, as the neutrino masses suppress structure formation, halos should grow slower compared to those in the $\Lambda$CDM universe with zero neutrino mass, and we expect to see a delay in $a_{1/2}$ that depends on $M_\nu$. We do a simple quadratic fit on the middle panel of Figure 1 in \cite{Lim_2015} to get a relation between $a_{1/2}$ and $f_*$.
\begin{align}
\label{eq:emp_law}
     z_{1/2} = \frac{1}{a_{1/2}} -1 = (0.446\pm0.06) (\log_{10}(f_*))^2 + (2.4\pm0.3) \log_{10}(f_*) + (3.9\pm0.3)
\end{align}
Assuming a typical $a_{1/2}$ value of 0.5, Eq.(\ref{eq:emp_law}) implies a $1\%$ delay in $a_{1/2}$ compared to that in the $\Lambda$CDM universe with zero neutrino mass would result in a $1.4\%$ decrease in $\log_{10}(f_*)$.

\subsection{Tree entropy $s$}
Besides MAH, the merger history is another way to describe the halo assembly. Two halos may have similar MAH but entirely different merger histories (see Figure \ref{fig:ass_hist}). 

To quantify the merger history, the concept of tree entropy $s(a)$ is adopted \cite{tree entropy}. This dimensionless parameter captures the mass ratios of the mergers and the complexity of the merger tree geometry. Zero tree entropy $(s=0)$ corresponds to a tree with a single branch, i.e., with no merger. The maximum tree entropy $(s=1)$ corresponds to a fractal history of equal-mass binary mergers. The evolution of $s$ is calculated as follows: 

\begin{figure}[!ht]
	\begin{center}% Center the figure.
	\includegraphics[width=0.8\columnwidth]{PDF/ass_hist.pdf}
	\caption{\label{fig:ass_hist}Assembly histories of two halos: the main-branch halo mass $M$ (black solid line) and tree entropy $s$ (blue bashed line) are plotted against the scale factor $a$. Both halos have similar formation time and final masses, but they have different merger histories. In the top panel, there is a tree entropy spike from $0$ to $\sim 0.75$ at $a=0.2$, indicating a binary merger with almost equal mass. The bottom panel shows a halo growing by smooth accretion instead. Two minor mergers occur, but they do not make up a significant mass fraction, and therefore the tree entropy remains small.}
	\end{center}
\end{figure}
\paragraph{Halos without progenitor} 
We set the initial tree entropy $s_{init} = 0$ for any halo without progenitor; these protohalos are formed by smooth accretion.

\paragraph{Mass growth by merger} For $n$ halos that merge together, each with mass $m_i$ and tree entropy $s_i$, the new tree entropy for the merged halo $s_{new}$ is calculated by:
\begin{subequations}
\label{S_equ}
\begin{align}
    x_i &= m_i/ \sum_{i=1}^n m_i ,\\
    H &= -f \sum_{i=1}^n x^\alpha_i \ln{x_i} , \\
    s_{new} &= H + (1+bH+cH^2) \sum_{i=1}^n x^2_i(s_i - H),
\end{align}
\end{subequations}
where $f=e(\alpha-1),\, b=(2-\gamma)/f,\, c= (1-\beta)e^{1/(\alpha-1)}-1-b$ are normalization constants. The 3 parameters $\alpha, \beta$ and $\gamma$ in turn govern the behavior of the tree entropy. For instance, $\alpha$ controls the impact of the merger order.
For a lower value of $\alpha$, a high-order merger ($n$ large) produces more tree entropy relative to a low-order one ($n$ small).
We would like binary mergers $(n=2)$ to have greater impact compared to triple mergers $(n=3)$, as the latter are more "accretion-like" than the former. This condition alone will fix $\alpha=1+ 1/ \ln(2)$.
$\beta$ on the other hand, controls the impact of the most destructive merger (equal-mass binary mergers) on $s$. $\gamma$ is related to the tree entropy loss when the halo is accreting mass smoothly.


\paragraph{Mass growth by accretion} 
The evolution of $s$ for smooth accretion of mass $\Delta m = M(t_2) - M(t_1)$, assuming no merger occurs between time $t_1$ and $t_2$, needs to be consistent with Eq.(\ref{S_equ}). We break down the smooth accretion as a series of consecutive $n^{th}$-order "mergers" $p$ times, with each "merging halo" having a mass $\delta m = \Delta m/[p(n-1)]$. In the limit of $p \to \infty$, regardless of $n$, Eq.(\ref{S_equ}) becomes:
\begin{equation}
    s_{new} = \left( \frac{m}{m+\Delta m} \right) ^\gamma.
\end{equation}


Here we follow the default choices for $(\alpha,\beta,\gamma)=(1+1/ \ln(2),\, 3/4,\, 1/3)$ specified in \cite{tree entropy}.
Using the tree entropy, we can identify the merger-rich halos and select them for specific studies.
As massive neutrinos suppress the small-scale correlation due to their free-streaming effect, one might expect that the merger histories of halos would be altered significantly, as they are very sensitive to small-scale correlation. 

\subsection{Halo leaf function $\ell(X)$}
To quantify the merger histories of halos, we can also count the number of leaves (protohalos) in the halo merger trees. 
The halo leaf count $X$ is independent of $a_{1/2}$ and $s$. 
Imagine two halos with similar mass accretion histories, which give them similar $a_{1/2}$. 
One of them gains its mass from major mergers (binary mergers with comparable masses), and one of them gains its mass from many minor mergers. 
The latter must have more leaves than the former since the mass of each merging halo is smaller. 
The same argument can be applied to $s$: two halos might have similar $s$, but their $X$ can be drastically different (See Figure \ref{fig:leaf_hist}).

We define the halo leaf function $\ell(X)$ to be the number of halos with more than $X$ leaves.
The halo leaf function $\ell(X)$ is conceptually similar to the halo mass function, for which the halos are binned into different mass bins. 

\begin{figure}[htp]
	\begin{center}% Center the figure.
	\includegraphics[width=\columnwidth]{PDF/s_leaf2x2_hist.pdf}
	\caption{\label{fig:leaf_hist}Assembly histories of two halos: 
    The left panel is similar to Figure \ref{fig:ass_hist}.
	In the right panel,
	the tree entropy $s$ (black solid line) and halo leaf count $X$ (blue bashed line) are plotted against the scale factor $a$. 	
	Both halos have similar final masses and tree entropies $s\approx0.46$, but they have different $X$. 
	In the top panel, the halo gains its mass through three major mergers, shown by the spikes of $s$: each merger brings a huge impact to $s$ but only a small number of halo leaf counts to the main branch.
	The bottom panel shows a halo growing by many minor mergers instead. A huge spike of $X$ at $a=0.7$ is accompanied by only a small change in $s$, implying that the merger is minor, but the merged halo inherits the tree entropy from its entropy-rich progenitor.}

	\end{center}

\end{figure}

\section{Experiments}

In this section, we demonstrate the effectiveness of our approach at exploiting dyadic interactions. To this end, we first introduce our \lindyhop{} dataset depicting couples that perform lindy hop dance movements.

\subsection{LindyHop600K}
Lindy hop is a type of swing dance with fast-paced steps synchronized with the music. It constitutes a good example of motions with strong mutual dependencies between the subjects, who are engaged in close interactions. To build this dataset, we filmed three men and four women dancers paired up in different combinations. Overall, \lindyhop{} contains nine dance sequences, each two to three minutes long, with a maximum of eight cameras at 60 fps. We use the shortest two sequences as validation and test sets. Table~\ref{table:seq_lhop} shows the details of the dataset organization. Our dataset displays standard lindy hop dancer positions and steps, such as the so-called open, closed, side and behind positions. In the open and closed positions, the dancers are facing each other with a varying distance between them. In the side position, both are facing the same direction, and in the behind position, the leader stands directly behind the follower, both facing the same direction. In each position, the dancers communicate through hand and shoulder grips. To the best of our knowledge, \lindyhop{} is the first large dance dataset involving the videos and 3D ground-truth poses of dancers.

\begin{table}[t] 
	\centering 
	\scalebox{0.9}{
		
		\begin{tabular}{ c|c|c|c|c } 
			\hline
			Sequence & Couple & Frames & Cameras & Split \\
			\hline
			{1} & A1 & 10152 & 5 & Train  \\ 
			{2} & B2 & 8819 & 8 & Train  \\ 
			{3} & C3 & 6519 & 8 & Validation  \\ 
			{4} & A4 & 7687 & 8 & Test \\ 
			{5} & B1 & 9977 & 8 & Train \\ 
			{6} & C2 & 9636 & 8 & Train\\ 
			{7} & A3 & 8930 & 7 & Train \\ 
			{8} & B4 & 9027 & 8 & Train \\ 
			{9} & C1 & 9635 & 8 & Train \\ 
			\hline
		\end{tabular}
	}
	\caption[\lindyhop{} dataset structure]{ \textbf{\lindyhop{} dataset structure.}}
	\label{table:seq_lhop}
	\vspace{-4mm}
\end{table}

To obtain the 3D poses of the dancers, we first extract 2D pixel locations of the visible joints using OpenPose~\cite{Cao17}. Because our dataset was captured with multiple cameras, this lets us obtain the  3D joint coordinates by performing a bundle adjustment using the 2D joint locations in all the views. However, this process comes with several problems because it requires annotating the poses of both subjects together. The major issues encompass body part confusions, missing 2D annotations and tracking errors in the OpenPose predictions, which occur when two people are very close to each other or wear similar garments. An example of this is shown in Fig.~\ref{fig:optimizing_3dposes}. To remedy this, we adopt a solution based on temporal smoothness. Specifically, we assign manually the 2D joint locations to each dancer in the first frame of each sequence. For the subsequent frames, the low confidence joint detections are replaced with ones interpolated using the high confidence joints from the neighboring frames. Despite these 2D joint corrections, the 3D locations extracted from the bundle adjustment procedure can still be very noisy. Thus, we employ a third degree spline interpolation across 30 frames coupled with an optimization scheme to generate the final 3D poses. Since the spline interpolation is done separately for each dimension of each joint, the length of each limb varies from one frame to another. To tackle this problem, we implement an optimization scheme which minimizes the squared difference between the length of a limb $c$ in the current frame and the average length of  limb $c$. We combine this loss function with additional regularizers penalizing feet from sliding on the floor, constraining the shape of the hips and shoulders, and preventing the optimization to the initial 3D pose estimates. For more detail, we refer the reader to the supplementary material.


\begin{figure}
	\centering
	\begin{tabular}{c}
		
		\includegraphics[width=0.67\linewidth]{figures/lindyhop_failure.pdf} \\
		(a) \footnotesize OpenPose 2D detection failure and the optimized 3D poses \\ \\
		\includegraphics[width=0.67\linewidth]{figures/lindyhop_success.pdf} \\
		(b) \footnotesize Correct OpenPose detections and the optimized 3D poses\\
	\end{tabular}
	\caption[Optimizing 3D poses in the \lindyhop{} dataset]{\textbf{Optimizing 3D poses in the \lindyhop{} dataset.} (a) Example of OpenPose 2D detection failure. The left leg of the woman is mapped to the left leg of the man. Our multi-view footage and refinement strategy allow us to obtain accurate 3D poses of the dancers despite the mismatch in the 2D detections. (b) Example of correct OpenPose detections and the optimized 3D ground truth poses.}
	\label{fig:optimizing_3dposes}
	\vspace{-4mm}
\end{figure}

\subsection{Data Pre-processing}
Each video sequence is first downsampled to 30 fps. The human body skeleton in the \lindyhop{} dataset originally comprises of $25$ body joints. We remove some of the facial, hand and foot joints and train our models with a skeleton of $19$ joints. The 3D joint locations are represented in the world coordinates. Since the position and orientation of the dancers change from one frame to another, we apply a rigid transformation to the poses.  We first subtract the global position of the hip center joint from every joint coordinate in every frame. Then, for each sequence, we take the first pose as  reference and rotate it such that the unit vector from the left to right shoulder is aligned with the $x$-axis and the unit vector from the center hip joint to the neck is aligned with the $z$-axis. We apply the same rotation to all the other poses in the sequence. 

\subsection{Results}

In this section, we evaluate our approach depicted by Fig.~\ref{fig:overview_3dmotion_forecasting} on our new \lindyhop{} dataset. We compare our method with the state-of-the-art single person approaches. They include HRI~\cite{Mao20}, which relies on an attention mechanism and a GCN decoder~\cite{Mao19} to predict the future poses of the individuals in isolation; HRI-Itr, which uses the output of the predictor as input and predicts the future motion recursively; TIM~\cite{Lebailly20}, which extends~\cite{Mao19} by combining it with a temporal inception layer to process the input at different subsequence lengths; and MSR-GCN~\cite{Lingwei21}, the most recent method, which extracts features from the human body at different scales by grouping the joints in close proximity. All the baselines rely on a GCN architecture that is trained and tested according to the data split shown in Table~\ref{table:seq_lhop}. They take as input a sequence of $60$ poses as  past motion. Except for HRI-Itr that recursively predicts $10$ poses at a time, all the baselines predict $30$ poses in the future. 

In Table~\ref{table:sota_lhop}, we report the MPJPE for short-term ($<$ 500ms) and long-term ($>$ 500ms) motion prediction in mm. Our method outperforms the baselines by a large margin. Fig.~\ref{fig:qualitative_lhop_sota} depicts qualitative results of our approach and the best performing three baselines for the \lindyhop{} test subjects with the corresponding follower and leader roles in the top two and bottom two portions, respectively. In contrast to the baselines, our method accurately predicts moves that are hard to anticipate in the long term, such as fast changing feet movements and less frequent arm openings. Although the observed motion of the primary subject does not include sufficient clues for such moves, the second person provides a useful prior so that our model can learn to predict the motion complementary or symmetric to that of the auxiliary subject. Therefore, we attribute this performance to our modeling of the motion dependencies via our pairwise attention mechanism. We provide additional qualitative results and further analysis on the learned pairwise attention scores in the supplementary material.

\begin{figure*}
	\vspace{-4mm}
	\centering
	\begin{tabular}{c}
		\includegraphics[width=0.93\linewidth]{figures/sota_qual_lhop_two_people.pdf} \\
	\end{tabular}
	\vspace{-4mm}
	\caption[Qualitative 3D motion prediction results on the \lindyhop{} test subjects]{\textbf{Qualitative evaluation of our results on the LindyHop600K test subjects compared to the state-of-the-art methods.} Black: Ground truth, green: TIM~\cite{Lebailly20}, blue: MSR-GCN~\cite{Lingwei21}, violet: HRI~\cite{Mao20}, red: Ours-Dyadic. Top two portions show the predictions for dancer with the follower role. Bottom two portions show the predictions for the dancer with the leader role. The left side of the vertical bar in the black row depicts the sampled input to our model and the right side shows the ground truth future poses. The colored rows correspond to the predictions of the state-of-the-art single person approaches. The red row depicts the output of our model shown in Fig.~\ref{fig:overview_3dmotion_forecasting}. The numbers at the top indicate the timestamp in milliseconds and the green region highlights the long-term predictions.}
	\label{fig:qualitative_lhop_sota}
\end{figure*}




\begin{table*}[t]
	%\vspace{0.2cm}
	\centering
	\scalebox{1.0}{
		\begin{tabular}{lccccccccccc}
			\toprule
			milliseconds											&100	&200	&300	&400	&500	&600 &700 &800 &900 &1000 &Average  \\ 
			\midrule
			
			
			{TIM~\cite{Lebailly20}}				   &6.06 &12.39 &19.83 &29.35 &41.80 &56.91 &73.17 &89.23 &104.31 &118.20    &51.13 \\
			{MSR-GCN~\cite{Lingwei21}}		&9.02 &17.02 &24.79 &33.26 &43.69 &56.34 &70.49 &85.00  &98.37   &109.73 &51.11  \\	
			{HRI-Itr~\cite{Mao20}}				   &2.21 &4.94 &9.51 &17.71 &30.93 &49.66 &72.95 &98.39 &122.93 &144.24  &50.41\\
			{HRI~\cite{Mao20}}						&5.34 &9.95 &15.08 &22.19 &32.45 &45.82 &61.29 &77.40 &92.47 &105.15    &43.17 \\
			{Ours}		&\textbf{1.31} &\textbf{4.31} &\textbf{9.49} &\textbf{17.33} &\textbf{27.42} &\textbf{39.85} &\textbf{54.22} &\textbf{70.20} &\textbf{86.23} &\textbf{100.09} &\textbf{37.57}\\	
			\bottomrule 
		\end{tabular}
		
	}  \\
	\caption[Comparison of our dyadic motion prediction approach with the state-of-the-art methods on the \lindyhop{} dataset]{\textbf{Comparison of our dyadic motion prediction approach with the state-of-the-art single person methods on the \lindyhop{} dataset.} We present the MPJPE for short-term ($<$ 500ms) and long-term ($>$ 500ms) motion prediction in mm. Despite the fast-paced and nonrepetitive nature of the dance moves, our method outperforms all the baselines for both short-term and long-term prediction. The best results in each column are shown in bold.}
	\label{table:sota_lhop}
\end{table*}


\subsection{Ablation Study}

\begin{table*}
	%\vspace{0.2cm}
	\centering
	\scalebox{1.0}{
		\renewcommand{\tabcolsep}{1.5mm}
		\begin{tabular}{lccccccccccc}
			\toprule
			milliseconds											&100	&200	&300	&400	&500	&600 &700 &800 &900 &1000 &Average  \\ 
			\midrule
			
			
			{HRI-Concat}	   &17.13 &33.99 &51.32 &69.89 &90.67 &113.41 &136.00 &156.10 &172.06 &183.40 &96.34\\	
			{Ours-SumPooling} &5.77&10.78&16.07&22.86&32.41&45.17&60.63&77.40&93.45&106.94&43.54\\
			{Ours-AvgPooling} &5.66&10.47&15.90&23.53&34.46&48.68&65.13&82.19&97.99&111.02&45.77\\
			{Ours-MaxPooling} &5.07&9.50&14.57&21.65&31.79&44.89&60.13&76.26&91.61&104.72&42.48\\
			{Ours-w/oPairwiseAtt} &3.60 &11.48 &25.08 &43.00 &62.22  &81.41 &100.25 &118.70 &135.48 &149.39 &68.04 \\	
			{Ours-w/o$\Delta$Pose}		&3.28 &8.36 &16.84 &23.87 &36.77 &52.22 &68.67 &85.02 &100.02 &112.07 &46.33\\	
			{Ours-EarlyMerge}		 &4.25 &8.11 &12.78 &19.25 &28.45 &40.84 &56.05 &73.11 &90.27 &105.40 &40.27\\		
			{Ours-w/SelfAttAux} &1.30 &5.04 &10.47 &18.12 &28.95 &42.41 &57.89 &74.52 &90.47 &104.09 &39.76\\
			{Ours-PairwiseAtt$\textbf{U}^{12}$ } &\textbf{1.17}&4.48&9.74&17.82&28.35&41.27&56.25&72.32&88.09&101.77&38.66\\
			{Ours}	&1.31 &\textbf{4.31} &\textbf{9.49} &\textbf{17.33} &\textbf{27.42} &\textbf{39.85} &\textbf{54.22} &\textbf{70.20} &\textbf{86.23} &\textbf{100.09} &\textbf{37.57}\\	\\	
			\bottomrule 
		\end{tabular}
		
	}  \\
	\caption[Ablation study for incorporating interactions]{\textbf{Ablation study for incorporating interactions.} We present the MPJPE for short-term ($<$ 500ms) and long-term ($>$ 500ms) motion prediction in mm. Here, we analyze different ways of incorporating interactions. HRI-Concat concatenates the motion history of the primary and auxiliary subject to treat them as one person. Ours-SumPooling, Ours-AvgPooling and Ours-MaxPooling use the social pooling layers from~\cite{Adeli20}. The remaining baselines show the benefits of the different components in our approach. Ours, depicted in Fig.~\ref{fig:overview_3dmotion_forecasting}, outperforms all other baselines and poses an effective way of handling coupled motion. The best results in each column are shown in bold.}
	\label{table:ablation_study_lhop}
	\vspace{-3mm}
\end{table*}

We evaluate the effect of modeling interactions via different strategies: \\
\textit{HRI-Concat} concatenates the motion history of the primary and auxiliary subject to treat them as one person. \\
\textit{Ours-SumPooling}, \textit{Ours-AvgPooling} and \textit{Ours-MaxPooling} discard the pairwise attention module, apply self-attention on the sequences of both subjects independently and combines the individual embeddings using the different pooling strategies proposed by~\cite{Adeli20}. The resulting vector is fed to the GCN decoder to predict the future poses of the primary subject. \\
\textit{Ours-w/oPairwiseAtt} excludes the pairwise attention module, applies self-attention and the GCN decoder on the sequences of both subjects independently and merges the GCN outputs from the two people to predict the future poses of the primary subject. \\
 \textit{Ours-w/o$\Delta$Pose} is our model which takes as input the past motion of the auxiliary subject directly instead of their relative motion to the primary subject.\\
 \textit{Ours-EarlyMerge} merges the pairwise embeddings $\textbf{U}^{12}$ and $\textbf{U}^{21}$ with the self-attention embedding of the primary subject $\textbf{U}^{1}$ before feeding them to the GCN module. \\
\textit{Ours-w/SelfAttAux} applies self-attention also on the sequence of the auxiliary subject and merges the result with the pairwise embeddings $\textbf{U}^{12}$ and $\textbf{U}^{21}$. \\
\textit{Ours-PairwiseAtt$\textbf{U}^{12}$ } excludes the pairwise attention that takes the keys and values from the auxiliary and the query from the primary subject. 
 

As can be seen in Table~\ref{table:ablation_study_lhop}, our method achieves the highest MPJPE in all timestamps. The comparison with \textit{HRI-Concat} shows that the naive way of combining the motion of the subjects is not an effective strategy to model their dependencies. The results of \textit{Ours-SumPooling}, \textit{Ours-AvgPooling} and \textit{Ours-MaxPooling} show that the social pooling layers proposed by~\cite{Adeli20} are suboptimal in the presence of strong interactions. The comparison to the remaining baselines evidence the benefits of the different components in our approach, which all contribute to the final results. 

\subsection{Limitations}
In Fig.~\ref{fig:qualitative_lhop_sota} and in the additional qualitative results, we observe that the lower arms and feet joints are usually difficult to predict and deviate the most from the ground-truth positions. Although Lindy Hop is a structured dance with highly correlated coupled motion, the dancers have their own styles. Therefore, predicting a single future is likely not to accurately match the body extremities which undergo the largest motion. This, however, can be overcome performing multiple diverse motion prediction, following a similar strategy to that used in~\cite{Yuan20,Aliakbarian21,Mao21b} for single-person motion prediction.

Another limitation of our model and many other motion prediction works in general is its use of complete sequences of ground-truth 3D poses as input. This may make our model sensitive to missing or faulty observations. To remedy this, as future work, we aim to incorporate the 3D poses obtained from the input images into our forecasting network and handle incomplete or noisy sequences to predict realistic future 3D poses for the interacting people.



\begin{comment}
\begin{figure}
\includegraphics[width=\linewidth]{figs/beyond_tss_lesion.pdf}
\caption[]{End-to-End runtime lesion study of the entire MNIST dataset and the FMA featurized music dataset. Each of DROP's contributions provides a runtime improvement.}
\label{fig:beyond_lesion}
\end{figure}
\end{comment}



\section{Conclusion}
\label{sec:conclusion}

Advanced data analytics techniques must scale to rising data volumes. 
DR techniques offer a powerful toolkit when processing these datasets, with PCA frequently outperforming popular techniques in exchange for high computational cost. 
In response, we propose DROP, a new dimensionality reduction optimizer. 
DROP combines progressive sampling, progress estimation, and online aggregation to identify high quality low dimensional bases via PCA without processing the entire dataset by balancing the runtime of downstream tasks and achieved dimensionality. 
Thus, DROP provides a first step in bridging the gap between quality and efficiency in end-to-end DR for downstream \red{analytics}. 

%We revisit canonical operators for time series dimensionality reduction and the measurement study of~\cite{keogh-study}, and show that PCA is more effective than popular alternatives in the data mining literature often by a margin of over $2\times$ on average on gold-standard time series benchmark data sets with respect to output data dimension. More surprisingly, we empirically demonstrate that a small number of samples are sufficient to accurately characterize directions of maximum variance and obtain a high-quality low-dimensional transformation.





\acknowledgments
We acknowledge Shek Yeung for modifying \texttt{CosmoMC} to fit the Planck 2018 data with different neutrino cosmologies and performing all the MCMC refittings. 
HWW thanks Jianxiong Chen for his mentorship on N-body simulations and Zhichao Zeng for discussion on neutrino-involved simulations.
All simulations were performed using the Central Research Computing Cluster at CUHK.
This work is supported partially by grants from the Research Grants Council of the Hong Kong Special Administrative Region, China (Project Nos. AoE/P-404/18 and C7015-19G).


\appendix
\section{Neutrino over-density linear evolution }
\label{app:phi}
We follow the derivation in \cite{Carton, LBE},  starting with the Vlasov equation:
\begin{equation}
\label{eq:a1}
    \frac{dF_\nu}{dt} = \pdv{F_\nu}{t} + \pdv{F_\nu}{r_i} \frac{dr_i}{dt} + \pdv{F_\nu}{p_i} \frac{dp_i}{dt} = 0,
\end{equation}
where $F_\nu(r_i,p_i,t)$ is the neutrino distribution function. Since we are considering a linear evolution equation, $F_\nu$ can be separated into the unperturbed Fermi-Dirac term $f_\nu^0(p)$ and a first-order perturbation $f'_\nu(\bold r, \bold p)$, i.e., $F_\nu = f_\nu^0 + f'_\nu$. In the following derivation, we will keep $f'_\nu$ up to the first order. In \texttt{Gadget2}, the Newtonian potential used is non-relativistic. Therefore,
\begin{equation}
\label{eq:a2}
    \bold{\dot p} = m \nabla \phi = -Gm \int \rho_t \frac{\bold{r - r'}}{|\bold{r-r'}|^3 } d^3r',
\end{equation}
where $\rho_t$ is the total matter energy density, including neutrinos and CDM. Substituting Eq.(\ref{eq:a2}) into Eq.(\ref{eq:a1}) and transforming to the co-moving coordinates with the follow rules:
\begin{align}
    d\chi &\equiv \frac{dt}{a^2(t)} \nonumber, \\
    \bold x &\equiv \frac{\bold r}{a(t)}, \\
    \bold u &\equiv \frac{d \bold x}{d\chi} = a(t) \bold v - \dot a(t) \bold r \nonumber,
\end{align}
we arrive at:
\begin{equation}
\label{eq:a4}
    \frac{1}{a^2}\pdv{f'_\nu}{\chi} + \frac{\bold u}{a^2} \cdot \pdv{f'_\nu}{\bold x} - \ddot a a \bold x \cdot \pdv{f_\nu^0}{\bold u} - Ga^2 \pdv{f_\nu^0}{\bold u} \cdot \int \rho_t \frac{\bold{x - x'}}{|\bold{x-x'}|^3 } d^3x' = 0.
\end{equation}
Recognizing the Dirac delta function in the integrand in Eq.(\ref{eq:a4}), we can combine it with the third term as it contains $\bold x$, and we can eliminate $\ddot a$ using the Friedmann equation:
\begin{align}
\label{eq:a5}
    \frac{\ddot a}{a} &= - \frac{4\pi G}{3} \bar \rho_t \nonumber,\\
    \frac{4\pi}{3} \bold x &= \int \frac{\bold{x - x'}}{|\bold{x-x'}|^3 } d^3x' \nonumber ,\\
    \ddot a \bold x &= -Ga \bar \rho_t \int \frac{\bold{x - x'}}{|\bold{x-x'}|^3 } d^3x',
\end{align}
where $\bar \rho_t$ is the mean total matter density. We then put Eq.(\ref{eq:a4}) to Eq.(\ref{eq:a5}) and multiply both sides by $a^2$ to obtain
\begin{equation}
\label{eq:a6}
    \pdv{f'_\nu}{\chi} + \bold u \cdot \pdv{f'_\nu}{\bold x} - Ga^4 \pdv{f_\nu^0}{\bold u} \cdot \int \bar \rho_t \delta_t( \chi, \bold x') \frac{\bold{x - x'}}{|\bold{x-x'}|^3 } d^3x' = 0,
\end{equation}
since by definition $\bar \rho_t \delta_t = \rho_t - \bar \rho_t$. Next we apply Fourier transform to Eq.(\ref{eq:a6}),  denoting $\Tilde{f}(\chi, \bold k ,\bold u) = \mathcal{F}[f(\chi, \bold x, \bold u)]$,
\begin{equation}
\label{eq:a7}
    \pdv{\Tilde{f'_\nu}}{\chi} + i\bold k \cdot \bold u \Tilde{f'_\nu} - Ga^4 \pdv{f^0_\nu}{\bold u} \int d^3x' \, \rho_t \delta_t(\chi, \bold x') \int d^3x\, e^{-i\bold k \cdot \bold x} \frac{\bold{x - x'}}{|\bold{x-x'}|^3 }  = 0.
\end{equation}
The last integral in Eq.(\ref{eq:a7}) can be evaluated,
\begin{equation}
    \int e^{-i\bold k \cdot \bold x} \frac{\bold{x - x'}}{|\bold{x-x'}|^3 } d^3x = -4\pi i \frac{\bold k}{k^2} e^{-i\bold k \cdot \bold x'}.
\end{equation}
Therefore, we have
\begin{equation}
    \pdv{\Tilde{f'_\nu}}{\chi} + i\bold k \cdot \bold u \Tilde{f'_\nu} + 4 \pi i Ga^4 \frac{\bold k}{k^2} \cdot \pdv{f^0_\nu}{\bold u} \bar \rho_t \Tilde{\delta_t}(\chi, \bold k) = 0.
\end{equation}
We then multiply both sides by $ e^{i \bold k \cdot \bold u \chi} $ and group the first two terms as a total derivative before integrating over co-moving time $\chi$,
\begin{align}
\label{eq:a10}
    \pdv{}{\chi}[\Tilde{f'_\nu} e^{i\bold k \cdot \bold u \chi}] + 4 \pi i Ga^4 e^{i \bold k \cdot \bold u \chi} \frac{\bold k}{k^2} \cdot \pdv{f^0_\nu}{\bold u} \bar \rho_t \Tilde{\delta_t}(\chi, \bold k) &= 0  \nonumber, \\
    \Tilde{f'_\nu}(\chi, \bold k, \bold u) + \int^\chi_0 4\pi iGa^4 e^{-i \bold k \cdot \bold u (\chi-\chi')} \frac{\bold k}{k^2} \cdot \pdv{f^0_\nu}{\bold u} \bar \rho_t \Tilde{\delta_t}(\chi, \bold k) d\chi' &= \Tilde{f'_\nu}(0, \bold k, \bold u) e^{-i \bold k \cdot \bold u \chi}.
\end{align}
Now Eq.(\ref{eq:a10}) is recognizable as Eq.(\ref{eq:LBE}). With an initial perturbation and the total over-density $\bar \rho_t \delta_t \equiv \bar \rho_{cdm} \delta_{cdm} + \bar \rho_\nu \delta_\nu$, we can evolve the neutrino perturbation function. Now we convert the distribution function $f'_\nu$ to over-density $\delta_\nu$ by integrating over the momentum space:
\begin{equation}
\label{eq:a11}
    \Tilde{\rho}_\nu(\chi, \bold k) + \int e^{-i \bold k \cdot \bold u (\chi-\chi')} \pdv{f^0_\nu}{\bold u} d^3u \cdot \int^\chi_0 4\pi iGa^4  \frac{\bold k}{k^2}  \bar \rho_t \Tilde{\delta_t}(\chi, \bold k) d\chi' = \int \Tilde{f'_\nu}(0, \bold k, \bold u) e^{-i \bold k \cdot \bold u \chi} d^3u.
\end{equation}
Using integration by parts and treating the perturbation as first order:
\begin{align}
\label{eq:a12}
    \int e^{-i \bold k \cdot \bold u (\chi-\chi')} \pdv{f^0_\nu}{\bold u} d^3u &= i\bold k(\chi-\chi') \int e^{-i \bold k \cdot \bold u (\chi-\chi')} {f_\nu^0 d^3u}, \\
\label{eq:a13}
    \Tilde{f'_\nu}(0, \bold k, \bold u) &\approx f_\nu^0(0, \bold u) \Tilde{\delta}_\nu(0, \bold k).
\end{align}
Putting Eq.(\ref{eq:a12}) and Eq.(\ref{eq:a13}) into Eq.(\ref{eq:a11}) and defining:
\begin{equation}
    \Phi(\bold q) \equiv \frac{\int f^0_\nu e^{-i \bold q \cdot \bold u d^3u}}{\int f^0_\nu d^3u},
\end{equation}
we have the equation governing the neutrino linear growth:
\begin{equation}
    \Tilde{\delta}_\nu (\chi,\mathbf{k}) = \Phi(\mathbf{k}\chi) \Tilde{\delta}_\nu(0,\mathbf{k}) + 
    4\pi G \int^\chi_0 a^4(\chi')(\chi-\chi') \Phi[\mathbf{k}(\chi-\chi')]
    [\Bar{\rho}_{cdm}(\chi') \Tilde{\delta}_{cdm} (\chi',\mathbf{k}) + \Bar{\rho}_\nu(\chi') \Tilde{\delta}_\nu (\chi',\mathbf{k})] d\chi'.
\end{equation}

We now turn our focus to $\Phi(\bold q)$. The denominator of $\Phi(\bold q)$ can be evaluated numerically. However, the numerator is highly oscillatory:
\begin{equation}
\label{eq:a16}
    I =\int f^0_\nu e^{-i \bold q \cdot \bold u d^3u} = 2\pi \int^\infty_0\int^\pi_0 \frac{u^2 [\cos(qu \cos\theta)-i \sin(qu\cos \theta)]\sin\theta}{e^{mu/T-\xi} +1} du d\theta + \mathrm{anti.},
\end{equation}
where anti. is the contribution from anti-neutrinos, which has $e^{mu/T+\xi} +1$ as the denominator. The imaginary part of the integrand in the right hand side of Eq.(\ref{eq:a16}) vanishes as we integrate it over $\theta$, and the integral $I$ becomes:
\begin{align}
    I &= 2\pi \int^\infty_0 \frac{2u^2 \sin(qu)}{qu(e^{mu/T-\xi} +1)} du + \mathrm{anti.} \nonumber ,\\
\label{eq:a17}
    &=4\pi \frac{T^2}{qm^2} \int^\infty_0 \left[\frac{x \sin(Ax)}{e^{x-\xi} +1 } + \frac{x \sin(Ax)}{e^{x+\xi} +1 }\right] dx,
\end{align}
where $x=mu/T$ and $A=qT/m$. We can expand the anti-neutrino term as a geometric series:
\begin{equation}
    \frac{1}{e^{x+\xi}+1}= \frac{e^{-x-\xi}}{1-(-e^{-x-\xi})} = e^{-(x+\xi)} \sum_{n=0}^\infty (-1)^n e^{-n(x+\xi)} = \sum_{n=1}^\infty (-1)^{n+1} e^{-n(x+\xi)}.
\end{equation}
The integral for anti-neutrino in Eq.(\ref{eq:a17}) becomes:
\begin{equation}
    \int^\infty_0 \frac{x \sin(Ax)}{e^{x+\xi} +1 } dx = \sum_{n=1}^\infty (-1)^{n+1} e^{-n\xi} \frac{2nA}{(A^2+n^2)^2},
\end{equation}
For the neutrino part, we separate the integral into two parts:
\begin{equation}
    \int^\infty_0 \frac{x \sin(Ax)}{e^{x-\xi} +1 } dx = \int^\xi_0 \frac{x \sin(Ax)}{e^{x-\xi} +1 } dx + \int^\infty_0 \frac{y+\xi \sin[A(y+\xi)]}{e^{y} +1 } dy.
\end{equation}
The first term can be evaluated directly, and we expand the second term again. We define:
\begin{align}
    B_1(n) &\equiv \int^\infty_0 e^{-ny} \cos(Ay) dy = \frac{n}{A^2+n^2}\nonumber, \\
    B_2(n) &\equiv \int^\infty_0 e^{-ny} \sin(Ay) dy = \frac{A}{A^2 + n^2}\nonumber, \\
    B_3(n) &\equiv \int^\infty_0 ye^{-ny} \cos(Ay) dy = \frac{n^2-A^2}{(A^2+n^2)^2} , \\
    B_4(n) &\equiv \int^\infty_0 ye^{-ny} \sin(Ay) dy = \frac{2nA}{(A^2+n^2)^2}\nonumber.
\end{align}
Therefore the numerator $I$ is,
\begin{equation}
\begin{split}
    &I = 4\pi \frac{T^2}{qm^2}\int^\xi_0 \frac{x \sin(Ax)}{e^{x-\xi} +1 } dx + 4\pi \frac{T^2}{qm^2} \times \\ 
    \sum_{n=1}^\infty (-1)^{n+1} \{ \xi B_1(n)\sin(A\xi)+&\xi B_2(n)\cos(A\xi)+B_3(n)\sin(A\xi)+B_4(n)\cos(A\xi)[\cos(A\xi)+e^{-n\xi}\},
\end{split}
\end{equation}
and this is how we evaluate $\Phi(\bold q)$ numerically.

\section{Neutrinos' effects on cosmological parameters}
\label{app:posterior}
We did not separate neutrinos' free-streaming effect from that of the CMB refitting in the N-body simulation due to the computational cost. However, we can extract the neutrinos' effect on the cosmological parameters alone, and we determine the rotation angle between the $(M_\nu, \eta^2)$ and $(x_1, x_2)$ bases. Our regression models are:
\begin{align}
\label{eq:b1}
    \Delta X(M_\nu, \eta^2)[\%] &\equiv \frac{X(M_\nu, \eta^2)}{X(0.06\,\mathrm{eV}, 0)} -1,\nonumber \\
    \Delta X(M_\nu, \eta^2)[\%] &= E_m  M' + E_\eta\eta^2, \\
    \Delta X(x_1, x_2)[\%]&= E_1 x_1 + E_2 x_2\nonumber,
\end{align}
where $X\in\{H_0, \Omega_b, \Omega_c, A_s, n_s\}$, with eigenbasis $x_1=M' \cos \rho\, -\, \eta^2 \sin \rho$ and $x_2 = M' \sin \rho \,+\, \eta^2 \cos \rho$ following similar definitions as in Eq.(\ref{eq:rot}).

\begin{table}[!h]
		\begin{center}
		\begin{tabular}{|c|c c|c|c c|}
			\hline
			Parameter & $E_m$ & $E_\eta$ & $\rho (\pi)$ & $E_1$ & $E_2$\\
			\hline \hline
			$\Delta H_0$ & $-1.62\pm0.02$ & $4.47\pm0.05$ & 0.40 & $-4.76\pm0.05$ & $-0.16 \pm 0.02$ \\
			$\Delta \Omega_b$ & $3.05\pm0.01$ & $-7.06\pm0.03$ & 0.39 & $7.68\pm0.03$ & $0.48\pm0.01$ \\
			$\Delta \Omega_c$ & $3.56\pm0.02$ & $-3.58\pm 0.04$ & 0.39 & $4.58\pm0.04$ & $2.14\pm0.02$ \\
			$\Delta A_s$ & $0.10\pm0.01$ & $2.26\pm0.02$ & 0.40 & $-2.12\pm0.02$ & $0.80\pm0.01$ \\
			$\Delta n_s$ & $-0.10\pm0.00$ & $1.77\pm0.01$ & 0.40 & $-1.71\pm0.01$ & $0.45\pm0.00$ \\
			\hline
		\end{tabular}
		\caption{\label{tab:CP_reg} Regression results for changes of cosmological parameters  as functions of ($M'$, $\eta^2$) and ($x_1, x_2)$ (see Eq.(\ref{eq:b1})).}
		\end{center}
\end{table}

From Figure \ref{fig:post} and Table \ref{tab:CP_reg} we can see that the effects of $M_\nu$ and $\eta$ on the cosmological parameters are again opposite to each other. Furthermore, all parameters share the same rotation angle $\rho$ of $0.4\pi$.

\begin{figure}[!h]
		\begin{center}% Center the figure.
		\includegraphics[width=\columnwidth]{PDF/post.pdf}
		\caption{\label{fig:post}1D posterior probability densities and 2D contours ($68\%$ and $95\%$ C.L.) for cosmological parameters extracted from the Planck 2018 CMB data, comparing $M_\nu = 0.06$ eV,  $\eta^2 = 0$ (black lines); $M_\nu = 0.24$ eV,  $\eta^2 = 0$ (red lines); and $M_\nu = 0.24$ eV,  $\eta^2 = 1$ (blue lines)} 
		\end{center}
\end{figure}








% The bibliography will probably be heavily edited during typesetting.
% We'll parse it and, using the arxiv number or the journal data, will
% query inspire, trying to verify the data (this will probalby spot
% eventual typos) and retrive the document DOI and eventual errata.
% We however suggest to always provide author, title and journal data:
% in short all the informations that clearly identify a document.

\begin{thebibliography}{99}

\bibitem{PP} Particle Data Group collaboration, C. Patrignani et al, \emph{Review of Particle Physics}, \emph{Chin. Phys.} {\bf C40} (2016) 100001. 

\bibitem{FARZAN200159} Y. Farzan, O. L. G. Peres, A. Yu. Smirnov,
\emph{Neutrino Mass Spectrum and Future Beta Decay Experiments}, \emph{Nuclear Physics B} {\bf 612} (2001) 59-97
\href{https://arxiv.org/abs/hep-ph/0105105}{ [hepo-ph/0105105]}.

\bibitem{Planck2018} Planck Collaboration, N.Aghnanim et la.,  \emph{Planck 2018 results. VI. Cosmological parameters}, \emph{A\&A} {\bf A6} (2020) 641
\href{https://arxiv.org/abs/1807.06209}{ [arxiv:1807.06209]}.

\bibitem{M nu} S. R. Choudhury and S. Hannestad, \emph{Updated results on neutrino mass and mass hierarchy from cosmology with
Planck 2018 likelihoods}, \emph{JCAP} {\bf 07} (2020) 037
\href{https://arxiv.org/abs/1907.12598}{ [arxiv:1907.12598]}.


\bibitem{camb} A. Lewis, A. Challinor, and A. Lasenby, \emph{Efficient computation of CMB anisotropies in closed FRW models}, \emph{ApJ} {\bf 538} (2000) 473-476
\href{https://arxiv.org/abs/astro-ph/9911177}{ [astro-ph/9911177]}.

\bibitem{class} D. Blas, J. Lesgourgues, and T. Tram, \emph{The Cosmic Linear Anisotropy Solving System (CLASS) II: Approximation schemes} , \emph{JCAP} {\bf 07} (2011) 034
\href{https://arxiv.org/abs/1104.2933}{ [arxiv:1104.2933]}.

\bibitem{part}J. Brandbyge and S. Hannestad \emph{Resolving Cosmic Neutrino Structure: A Hybrid Neutrino N-body Scheme}, \emph{JCAP} {\bf01} (2010) 021 \href{https://arxiv.org/abs/0908.1969}{[arxiv:0908.1969]}.

\bibitem{grid} J. Brandbyge and S. Hannestad, \emph{Grid Based Linear Neutrino Perturbations in Cosmological N-body Simulations}, \emph{JCAP} {\bf05} (2009) 002, \href{https://arxiv.org/abs/0812.3149}{[arxiv:0812.3149]}.

\bibitem{grid2} Y. Ali-Haimoud and S. Bird, \emph{An effcient implementation of massive neutrinos in non-linear structure formation simulations}, \emph{MNRAS} {\bf428} (2013) 3375 \href{https://arxiv.org/abs/1209.0461}{[arxiv:1209.0461]}.

\bibitem{Carton} Z. Zheng, S. Yeung and M. C. Chu, \emph{Effects of neutrino mass and asymmetry on cosmological structure formation}, \emph{JCAP} {\bf03} (2019) 015
\href{https://arxiv.org/abs/1808.00357}{ [arxiv:1808.00357]}.

\bibitem{Yvvone}J. Z. Chen, A. Upadhye and Yvonne Y. Y. Wong, \emph{One line to run them all: SuperEasy massive neutrino linear response in
N-body simulations}, \href{https://arxiv.org/abs/2011.12504}{[arxiv:2011.12504]}.

\bibitem{Yvvone fluid}J. Z. Chen, A. Upadhye and Yvonne Y. Y. Wong, \emph{The cosmic neutrino background as a collection of fluids in large-scale structure simulations}, 
\href{https://arxiv.org/abs/2011.12503}{[arxiv:2011.12503]}.

\bibitem{nuconcept} J. Dakin, J. Brandbyge, S. Hannestad, T. Haugbølle and T. Tram,  \emph{$\nu$CONCEPT: Cosmological neutrino simulations from the non-linear Boltzmann hierarchy}, \emph{JCAP} {\bf02} (2019) 052
\href{https://arxiv.org/abs/1712.03944}{[arxiv:1712.03944]}.

\bibitem{mu} G. Barenboim, W. H. Kinney and W. Park, \emph{ Flavor versus mass eigenstates in neutrino asymmetries: implications for cosmology}, \emph{Eur. Phys. J. C.} {\bf 77} (2017) 590
\href{https://arxiv.org/abs/1609.03200}{ [arxiv:1609.03200]}.


\bibitem{nu CMB} K.N. Abazajian et al, \emph{Neutrino Physics from the Cosmic Microwave Background and Large Scale Structure}, \emph{Astropart. Phys.} {\bf63} (2015) 66-80 \href{https://arxiv.org/abs/1309.5383}{ [arxiv:1309.5383]}.

\bibitem{bbn1} G. Mangano, G. Miele, S. Pastor, O. Pisanti, S. Sarikas, \emph{Constraining the cosmic radiation density due to lepton number with Big Bang Nucleosynthesis}, \emph{JCAP} {\bf03} (2011) 035 \href{https://arxiv.org/abs/1011.0916v3}{[arxiv:1011.0916]}.

\bibitem{bbn2} A.D. Dolgov, S.H. Hansen, S. Pastor, S.T. Petcov, G.G. Raffelt, D.V. Semikoz, \emph{A.D. Dolgov, S.H. Hansen, S. Pastor, S.T. Petcov, G.G. Raffelt, D.V. Semikoz}, \emph{Nucl. Phys. B} {\bf632} (2002) 363-382 \href{https://arxiv.org/abs/hep-ph/0201287}{[hep-ph/021287]}.


\bibitem{cosmomc} A. Lewis and S. Bridle, \emph{Cosmological parameters from CMB and other data: a Monte-Carlo
approach}, \emph{Phys. Rev. D} {\bf66} (2002) 103511 \href{https://arxiv.org/abs/astro-ph/0205436}{[astro-ph/0205436]}.


\bibitem{formtime} D. H. Zhao, Y. P. Jing, H. J. Mo, and G. Börner, \emph{Accurate universal models for the mass accretion histories and concentrations of dark matter halos}, \emph{ApJ} {\bf707} (2009) 354 \href{https://arxiv.org/abs/0811.0828}{[arxiv:0811.0828]}.

\bibitem{eta_effect} S. Yeung, King Lau and M.-C. Chu, \emph{Relic neutrino degeneracies and their impact on cosmological parameters}, \emph{JCAP} {\bf04} (2021) 024 \href{https://arxiv.org/abs/2010.01696}{[arxiv:2010.01696]}

\bibitem{tree entropy} D. Obreschkow, P. J. Elahi, C. P. Lagos, R. J. J. Poulton and A. D. Ludlow, \emph{Characterising the Structure of Halo Merger Trees Using a Single Parameter: The Tree Entropy}, \emph{MNRAS} {\bf 493} (2020) 4551-4569 \href{https://arxiv.org/abs/1911.11959}{ [arxiv:1911.11959]}.

\bibitem{Gadget2} V. Springel, \emph{The cosmological simulation code GADGET-2}, \emph{MNRAS} {\bf 364} (2005) 1105-1134
\href{https://arxiv.org/abs/astro-ph/0505010}{ [astro-ph/0505010]}.

\bibitem{LBE}S. Xiang and L. Feng, \emph{The formation of the cosmological structure}
, \emph{2ed, Astronomical Series
of NAOC}, Chinese Science and Technology Press (2012), 257-261.


\bibitem{music} O. Hahn and T. Abel, \emph{Multi-scale initial conditions for cosmological simulations}, \emph{NMRAS} {\bf415} (2011) 2101-2121 \href{https://arxiv.org/abs/1103.6031}{ [arxiv:1103.6031]}.


\bibitem{rockstar} P. S. Behroozi, R. H. Wechsler and H. Y. Wu, \emph{The Rockstar Phase-Space Temporal Halo Finder and the Velocity Offsets of Cluster Cores}, \emph{ApJ} {\bf762} (2013) 109 \href{https://arxiv.org/abs/1110.4372}{ [arxiv:1110.4372]}.

\bibitem{ct} P. S. Behroozi et al, \emph{Gravitationally Consistent Halo Catalogs and Merger Trees for Precision Cosmology}, \emph{ApJ} {\bf 763} (2013) 18 \href{https://arxiv.org/abs/1110.4370}{[arxiv:1110.4370]}.

\bibitem{Kuan} K. Wang et al, \emph{Concentrations of Dark Haloes Emerge from Their Merger Histories}, \emph{MNRAS} {\bf 498} (2020) 4450-4464 \href{https://arxiv.org/abs/2004.13732}{ [arxiv:2004.13732]}.




\bibitem{Wang_2011_2}Huiyuan Wang, H. J. Mo, Y.P. Jing, Xiaohu Yang and Yu Wang, \emph{Internal properties and environments of dark matter halos}, \emph{MNRAS} {\bf 413} (2011) 1974-1990, \href{https://arxiv.org/abs/1007.0612}{[arxiv/1007.0612]}.


\bibitem{Wang_2011} Huiyuan Wang, H. J. Mo, Xiaohu Yang and Frank C. van den Bosch, \emph{Reconstructing the cosmic velocity and tidal fields with galaxy groups selected from the Sloan Digital Sky Survey}, \emph{MNRAS} {\bf 420} 2012 1809–1824, \href{https://arxiv.org/abs/1108.1008}{[arxiv:1108.1008]}.

\bibitem{Lim_2015} Seunghwan Lim, Houjun Mo, Huiyuan Wang and Xiaohu Yang, \emph{An observational proxy of halo assembly time and its correlation with galaxy properties}, \emph{MNRAS}, {\bf 455} (2016) 499-510, \href{https://arxiv.org/abs/1502.01256}{[arxiv:1502.01256]}.


\bibitem{halo bias}T. Lazeyras, F. Villaescusa-Navarro and M. Viel, \emph{The impact of massive neutrinos on halo assembly bias}, \emph{JCAP} {\bf 03} (2021) 022,  \href{https://arxiv.org/abs/2008.12265}{[arxiv:2008.12265]} .

% Please avoid comments such as "For a review'', "For some examples",
% "and references therein" or move them in the text. In general,
% please leave only references in the bibliography and move all
% accessory text in footnotes.

% Also, please have only one work for each \bibitem.

%\bibitem{a} Author, \emph{Title}, \emph{J. Abbrev.} {\bf vol} (year) pg.

%\bibitem{b} Author, \emph{Title}, arxiv:1234.5678.

%\bibitem{c} Author, \emph{Title}, Publisher (year).



\end{thebibliography}
\end{document}
