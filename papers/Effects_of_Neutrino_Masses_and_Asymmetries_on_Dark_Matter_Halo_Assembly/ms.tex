\documentclass[a4paper,11pt]{article}
\pdfoutput=1 % if your are submitting a pdflatex (i.e. if you have
             % images in pdf, png or jpg format)

\usepackage{jcappub} % for details on the use of the package, please
                     % see the JCAP-author-manual

\usepackage[T1]{fontenc} % if needed
\usepackage{subcaption}
\usepackage{physics}



\title{\boldmath Effects of Neutrino Masses and Asymmetries on Dark Matter Halo Assembly}


%% %simple case: 2 authors, same institution
%% \author{A. Uthor}
%% \author{and A. Nother Author}
%% \affiliation{Institution,\\Address, Country}

% more complex case: 4 authors, 3 institutions, 2 footnotes
\author[a]{Hiu Wing Wong}
\author[a]{and Ming-chung Chu}

% The "\note" macro will give a warning: "Ignoring empty anchor..."
% you can safely ignore it.

\affiliation[a]{Department of Physics, The Chinese University of Hong Kong, Shatin NT, Hong Kong}
%\affiliation[b]{Another University,\\different-address, Country}
%\affiliation[c]{A School for Advanced Studies,\\some-location, Country}

% e-mail addresses: one for each author, in the same order as the authors
\emailAdd{hwwong@phy.cuhk.edu.hk}
\emailAdd{mcchu@phy.cuhk.edu.hk}





\abstract{Massive cosmological neutrinos suppress the Large-Scale Structure (LSS) in the Universe by smoothing the cosmic over-densities, and hence structure formation is delayed relative to that in the standard Lambda-Cold Dark Matter ($\Lambda$CDM) model.
We characterize the merger and mass accretion history of dark matter halos with the halo formation time $a_{1/2}$, tree entropy $s$ and halo leaf function $\ell(X)$ and measure them using neutrino-involved N-body simulations. 
We show that a non-zero sum of neutrino masses $M_\nu$ delays the $a_{1/2}$ for halos with virial mass between $10^{13} M_\odot$ and $3\times 10^{13} M_\odot$, whereas a non-zero neutrino asymmetry parameter $\eta^2$ has the opposite effect.
While the mean tree entropy $\bar s$ does not depend significantly on either $M_\nu$ or $\eta^2$, the halo leaf function does.  Furthermore, the dependencies of $\ell$ on $M_\nu$ and $\eta^2$ have significant evolution in redshift $z$, with the relative contributions of $M_\nu$ and $\eta^2$ showing a sigmoid-like transition as a function of $z$ around $z \approx 0.6$.
Together with the matter power spectrum, these halo parameters allow us to break the parameter degeneracy between $M_\nu$ and $\eta^2$ so that they can both be constrained in principle.
}
\keywords{cosmological simulations, cosmological neutrinos, neutrino masses from cosmology, neutrino properties}
%\arxivnumber{XXXX.XXXXXX}



\begin{document}
\maketitle
\flushbottom

\IEEEraisesectionheading{\section{Introduction}}

\IEEEPARstart{V}{ision} system is studied in orthogonal disciplines spanning from neurophysiology and psychophysics to computer science all with uniform objective: understand the vision system and develop it into an integrated theory of vision. In general, vision or visual perception is the ability of information acquisition from environment, and it's interpretation. According to Gestalt theory, visual elements are perceived as patterns of wholes rather than the sum of constituent parts~\cite{koffka2013principles}. The Gestalt theory through \textit{emergence}, \textit{invariance}, \textit{multistability}, and \textit{reification} properties (aka Gestalt principles), describes how vision recognizes an object as a \textit{whole} from constituent parts. There is an increasing interested to model the cognitive aptitude of visual perception; however, the process is challenging. In the following, a challenge (as an example) per object and motion perception is discussed. 



\subsection{Why do things look as they do?}
In addition to Gestalt principles, an object is characterized with its spatial parameters and material properties. Despite of the novel approaches proposed for material recognition (e.g.,~\cite{sharan2013recognizing}), objects tend to get the attention. Leveraging on an object's spatial properties, material, illumination, and background; the mapping from real world 3D patterns (distal stimulus) to 2D patterns onto retina (proximal stimulus) is many-to-one non-uniquely-invertible mapping~\cite{dicarlo2007untangling,horn1986robot}. There have been novel biology-driven studies for constructing computational models to emulate anatomy and physiology of the brain for real world object recognition (e.g.,~\cite{lowe2004distinctive,serre2007robust,zhang2006svm}), and some studies lead to impressive accuracy. For instance, testing such computational models on gold standard controlled shape sets such as Caltech101 and Caltech256, some methods resulted $<$60\% true-positives~\cite{zhang2006svm,lazebnik2006beyond,mutch2006multiclass,wang2006using}. However, Pinto et al.~\cite{pinto2008real} raised a caution against the pervasiveness of such shape sets by highlighting the unsystematic variations in objects features such as spatial aspects, both between and within object categories. For instance, using a V1-like model (a neuroscientist's null model) with two categories of systematically variant objects, a rapid derogate of performance to 50\% (chance level) is observed~\cite{zhang2006svm}. This observation accentuates the challenges that the infinite number of 2D shapes casted on retina from 3D objects introduces to object recognition. 

Material recognition of an object requires in-depth features to be determined. A mineralogist may describe the luster (i.e., optical quality of the surface) with a vocabulary like greasy, pearly, vitreous, resinous or submetallic; he may describe rocks and minerals with their typical forms such as acicular, dendritic, porous, nodular, or oolitic. We perceive materials from early age even though many of us lack such a rich visual vocabulary as formalized as the mineralogists~\cite{adelson2001seeing}. However, methodizing material perception can be far from trivial. For instance, consider a chrome sphere with every pixel having a correspondence in the environment; hence, the material of the sphere is hidden and shall be inferred implicitly~\cite{shafer2000color,adelson2001seeing}. Therefore, considering object material, object recognition requires surface reflectance, various light sources, and observer's point-of-view to be taken into consideration.


\subsection{What went where?}
Motion is an important aspect in interpreting the interaction with subjects, making the visual perception of movement a critical cognitive ability that helps us with complex tasks such as discriminating moving objects from background, or depth perception by motion parallax. Cognitive susceptibility enables the inference of 2D/3D motion from a sequence of 2D shapes (e.g., movies~\cite{niyogi1994analyzing,little1998recognizing,hayfron2003automatic}), or from a single image frame (e.g., the pose of an athlete runner~\cite{wang2013learning,ramanan2006learning}). However, its challenging to model the susceptibility because of many-to-one relation between distal and proximal stimulus, which makes the local measurements of proximal stimulus inadequate to reason the proper global interpretation. One of the various challenges is called \textit{motion correspondence problem}~\cite{attneave1974apparent,ullman1979interpretation,ramachandran1986perception,dawson1991and}, which refers to recognition of any individual component of proximal stimulus in frame-1 and another component in frame-2 as constituting different glimpses of the same moving component. If one-to-one mapping is intended, $n!$ correspondence matches between $n$ components of two frames exist, which is increased to $2^n$  for one-to-any mappings. To address the challenge, Ullman~\cite{ullman1979interpretation} proposed a method based on nearest neighbor principle, and Dawson~\cite{dawson1991and} introduced an auto associative network model. Dawson's network model~\cite{dawson1991and} iteratively modifies the activation pattern of local measurements to achieve a stable global interpretation. In general, his model applies three constraints as it follows
\begin{inlinelist}
	\item \textit{nearest neighbor principle} (shorter motion correspondence matches are assigned lower costs)
	\item \textit{relative velocity principle} (differences between two motion correspondence matches)
	\item \textit{element integrity principle} (physical coherence of surfaces)
\end{inlinelist}.
According to experimental evaluations (e.g.,~\cite{ullman1979interpretation,ramachandran1986perception,cutting1982minimum}), these three constraints are the aspects of how human visual system solves the motion correspondence problem. Eom et al.~\cite{eom2012heuristic} tackled the motion correspondence problem by considering the relative velocity and the element integrity principles. They studied one-to-any mapping between elements of corresponding fuzzy clusters of two consecutive frames. They have obtained a ranked list of all possible mappings by performing a state-space search. 



\subsection{How a stimuli is recognized in the environment?}

Human subjects are often able to recognize a 3D object from its 2D projections in different orientations~\cite{bartoshuk1960mental}. A common hypothesis for this \textit{spatial ability} is that, an object is represented in memory in its canonical orientation, and a \textit{mental rotation} transformation is applied on the input image, and the transformed image is compared with the object in its canonical orientation~\cite{bartoshuk1960mental}. The time to determine whether two projections portray the same 3D object
\begin{inlinelist}
	\item increase linearly with respect to the angular disparity~\cite{bartoshuk1960mental,cooperau1973time,cooper1976demonstration}
	\item is independent from the complexity of the 3D object~\cite{cooper1973chronometric}
\end{inlinelist}.
Shepard and Metzler~\cite{shepard1971mental} interpreted this finding as it follows: \textit{human subjects mentally rotate one portray at a constant speed until it is aligned with the other portray.}



\subsection{State of the Art}

The linear mapping transformation determination between two objects is generalized as determining optimal linear transformation matrix for a set of observed vectors, which is first proposed by Grace Wahba in 1965~\cite{wahba1965least} as it follows. 
\textit{Given two sets of $n$ points $\{v_1, v_2, \dots v_n\}$, and $\{v_1^*, v_2^* \dots v_n^*\}$, where $n \geq 2$, find the rotation matrix $M$ (i.e., the orthogonal matrix with determinant +1) which brings the first set into the best least squares coincidence with the second. That is, find $M$ matrix which minimizes}
\begin{equation}
	\sum_{j=1}^{n} \vert v_j^* - Mv_j \vert^2
\end{equation}

Multiple solutions for the \textit{Wahba's problem} have been published, such as Paul Davenport's q-method. Some notable algorithms after Davenport's q-method were published; of that QUaternion ESTimator (QU\-EST)~\cite{shuster2012three}, Fast Optimal Attitude Matrix \-(FOAM)~\cite{markley1993attitude} and Slower Optimal Matrix Algorithm (SOMA)~\cite{markley1993attitude}, and singular value decomposition (SVD) based algorithms, such as Markley’s SVD-based method~\cite{markley1988attitude}. 

In statistical shape analysis, the linear mapping transformation determination challenge is studied as Procrustes problem. Procrustes analysis finds a transformation matrix that maps two input shapes closest possible on each other. Solutions for Procrustes problem are reviewed in~\cite{gower2004procrustes,viklands2006algorithms}. For orthogonal Procrustes problem, Wolfgang Kabsch proposed a SVD-based method~\cite{kabsch1976solution} by minimizing the root mean squared deviation of two input sets when the determinant of rotation matrix is $1$. In addition to Kabsch’s partial Procrustes superimposition (covers translation and rotation), other full Procrustes superimpositions (covers translation, uniform scaling, rotation/reflection) have been proposed~\cite{gower2004procrustes,viklands2006algorithms}. The determination of optimal linear mapping transformation matrix using different approaches of Procrustes analysis has wide range of applications, spanning from forging human hand mimics in anthropomorphic robotic hand~\cite{xu2012design}, to the assessment of two-dimensional perimeter spread models such as fire~\cite{duff2012procrustes}, and the analysis of MRI scans in brain morphology studies~\cite{martin2013correlation}.

\subsection{Our Contribution}

The present study methodizes the aforementioned mentioned cognitive susceptibilities into a cognitive-driven linear mapping transformation determination algorithm. The method leverages on mental rotation cognitive stages~\cite{johnson1990speed} which are defined as it follows
\begin{inlinelist}
	\item a mental image of the object is created
	\item object is mentally rotated until a comparison is made
	\item objects are assessed whether they are the same
	\item the decision is reported
\end{inlinelist}.
Accordingly, the proposed method creates hierarchical abstractions of shapes~\cite{greene2009briefest} with increasing level of details~\cite{konkle2010scene}. The abstractions are presented in a vector space. A graph of linear transformations is created by circular-shift permutations (i.e., rotation superimposition) of vectors. The graph is then hierarchically traversed for closest mapping linear transformation determination. 

Despite of numerous novel algorithms to calculate linear mapping transformation, such as those proposed for Procrustes analysis, the novelty of the presented method is being a cognitive-driven approach. This method augments promising discoveries on motion/object perception into a linear mapping transformation determination algorithm.



\section{Grid-based neutrino simulation}
\label{sec:grid}

To study the neutrino free-streaming effect on LSS, we can include neutrinos as another type of particles in N-body simulations. This particle-based method is accurate in principle but computationally expensive.
Not only will it bring extra particles and interactions, but it also requires more integration steps compared to the Cold Dark Matter (CDM)-only simulation with the same number of particles due to the larger velocity dispersion of the neutrinos. 
On the other hand, the grid-based neutrino-involved simulation includes only CDM particles in the simulation box. The neutrino information is carried by the neutrino over-density field $\delta_\nu$ contained in the Particle-Mesh (PM) grid, which is responsible for the long-range interaction in a Tree-PM code like \texttt{Gadget2} \cite{Gadget2}. 

Another advantage of the grid-based method is that we can also investigate the effects of the chemical potentials of neutrinos, which can be incorporated easily in the simulation through the Fermi-Dirac distribution of the cosmological neutrinos.

\subsection{Linear evolution for neutrino over-density}
The linear equation that governs the evolution of $\delta_\nu$ is \cite{LBE}: 
\begin{equation}
\label{eq:LBE}
    \Tilde{\delta}_\nu (\chi,\mathbf{k}) = \Phi(\mathbf{k}\chi) \Tilde{\delta}_\nu(0,\mathbf{k}) +
    4\pi G \int^\chi_0 a^4(\chi')(\chi-\chi') \Phi[\mathbf{k}(\chi-\chi')]
    [\Bar{\rho}_{cdm}(\chi') \Tilde{\delta}_{cdm} (\chi',\mathbf{k}) + \Bar{\rho}_\nu(\chi') \Tilde{\delta}_\nu (\chi',\mathbf{k})] d\chi'.
\end{equation}


Here, $\Tilde{\delta}_\nu$ ($\Tilde{\delta}_{cdm}$) and $\bar \rho_\nu$  $(\bar\rho_{cdm})$ are the over-density in Fourier space and mean density of neutrinos (CDM), respectively. $\bf k$ is the wave vector, $\chi$ is the co-moving coordinate where $d\chi=dt/a^2(t)$, and $\Phi$ is a special function that will be discussed in Appendix \ref{app:phi}.

Neutrinos cannot cluster below their free-streaming scale, which is larger than their non-linear scales; therefore, the evolution of neutrinos is well described by the linear equation. Although we use a linear equation to describe the evolution of $\delta_\nu$, the non-linear $\delta_{cdm}$ (from N-body simulation) is involved in the evolution of neutrinos. Hence, the non-linearities in structure formation are still fully preserved. 
Previous studies have also shown that both particle-based and grid-based simulations produce consistent results for the matter power spectrum \cite{Carton}. 


\subsection{Total over-density}
Once we obtain the neutrino over-density $\delta_\nu$, the total over-density field $\delta_t$ is then given by:
\begin{equation}
\label{eq:avg}
    \delta_t = (1-f_\nu)\delta_{cdm} + f_\nu \delta_\nu,
\end{equation}
where $f_\nu=\Omega_\nu / \Omega_m$, the ratio of the cosmological neutrino and CDM densities. Although $f_\nu$ is small, the non-linearity in structure formation will mix up different modes, and the final density $\delta_t $ may change by a factor much greater than $(1- f_\nu)$.

\subsection{Implementation}
We implemented the grid-based neutrino method in our own modified version of \texttt{Gadget2}. The detailed procedure is discussed in \cite{Carton}. Here we briefly summarize the steps:

\begin{enumerate}
    \item The initial snapshot generated by \texttt{MUSIC} \cite{music} and initial neutrino power spectrum generated by \texttt{CAMB} are fed into \texttt{Gadget2} as the initial conditions, and thus we have both $\Tilde{\delta}_\nu(0,k)$ and $\Tilde{\delta}_{cdm}(0,k)$.
    
    \item To evolve the system, the CDM particles are drifted first. With a new CDM power spectrum after the drift $\Tilde{\delta}_{cdm}(\chi, k)$, we solve Eq.(\ref{eq:LBE}) iteratively. We use linear interpolation to approximate $\Tilde{\delta}_{cdm}(\chi', k)$ for $\chi' \in [0,\chi]$ as the time difference is usually small between two PM steps.
    
    \item The over-densities are assumed to carry the same phase:
    \begin{equation}
        \Tilde{\delta}_\nu (\chi, {\bf k}) = \frac{\Tilde{\delta}_\nu(\chi, k)}{\Tilde{\delta}_{cdm}(\chi, k)} \Tilde{\delta}_{cdm}(\chi, {\bf k}).
    \end{equation}
    The total over-density field $\Tilde{\delta}_t(\chi, {\bf k})$ is then obtained using Eq.(\ref{eq:avg}). Finally the original $\Tilde{\delta}_{cdm}(\chi, {\bf k})$ is replaced by $\Tilde{\delta}_t(\chi, {\bf k})$ to give the correct PM potential to evolve the CDM particles.
    
    \item $\Tilde{\delta}_\nu(\chi, k)$ and $\Tilde{\delta}_{cdm}(\chi, k)$ are stored as the initial conditions for the next PM calculation, and we iterate back to step 1 until the final time (usually today).
    
\end{enumerate}

\subsection{Simulation parameters}


To specify the fiducial cosmology for our N-body simulation, 6 cosmological parameters are needed. 
They are the physical CDM density $\Omega_ch^2$, the physical baryon density $\Omega_bh^2$, the observed angular size of the sound horizon at recombination $\theta$, the reionization optical depth $\tau$, 
the initial super-horizon amplitude of curvature perturbations $A_s$ at $k$ = 0.05 Mpc$^{-1}$ and the primordial spectral index $n_s$. 
All of them are consistently refitted from the Planck CMB data using the Planck 2018 \texttt{plikHM\_TTTEEE} likelihood for each set of $M_\nu$ and $\eta^2$ values. The parameters relevant to the N-body simulations are listed in Table \ref{tab:simparam}, and their posterior distributions for selected sets of $M_\nu$ and $\eta^2$ are plotted in Figure \ref{fig:post}. 


\begin{table}[!hbt]
		\begin{center}
		% Table itself: here we have two columns which are centered and have lines to the left, right and in the middle: |c|c|
		\begin{tabular}{|c|c|c|c|c|c|c|c|c|c|c|}
			% To create a horizontal line, type \hline
			\hline
			% To end a column type &
			% For a linebreak type \\
			No. &$M_\nu$ & $\eta^2 $ & $H_0$ & $\Omega_c+\Omega_b$ & $\Omega_\nu$ &$\Omega_\Lambda$ & $\sigma_8$ & $n_s$ & $A_s \,(10^{-9})$\\
			\hline
			\hline
%			A0 & 0 & 0 & 67.74 & 0.3097 & 0 & 0.6903 \\
			A1 & 0.06 & 0 & 67.37 & 0.3141 & 0.00140 & 0.6845 & 0.814 & 0.965 & 2.10 \\
			A2 & 0.06 & 0.253 & 68.13 & 0.3107 & 0.00142 & 0.68788 & 0.818 & 0.969 & 2.11\\
			A3 & 0.06 & 1.012 & 70.49  & 0.3009 & 0.00145 & 0.69765 & 0.833 & 0.982 & 2.15 \\
			\hline
			B1 & 0.15 & 0 & 66.43 & 0.3236 & 0.0036 & 0.6728 & 0.794 & 0.964 & 2.10 \\
			B2 & 0.15 & 0.253 & 67.12 & 0.3208 & 0.00365 & 0.67555 & 0.799 & 0.968 & 2.11\\
			B3 & 0.15 & 1.012 & 69.44 & 0.3106 & 0.00375 & 0.68565 & 0.811 & 0.981 & 2.15\\
			\hline
            C1 & 0.24 & 0 & 65.46 & 0.3337 & 0.00595 & 0.66035 & 0.775 & 0.963 & 2.11\\
			C2 & 0.24 & 0.253 & 66.17 & 0.3308 & 0.00600 & 0.6632 & 0.778 & 0.967 & 2.12\\
			C3 & 0.24 & 1.012 & 68.39 & 0.3192 & 0.00618 & 0.67462 & 0.790 & 0.981 & 2.15\\
			\hline
		\end{tabular}
		\caption{\label{tab:simparam} N-body simulation parameters.}
		\end{center}
\end{table}


Simulation snapshots are generated using our modified version of $\texttt{Gadget2}$ to incorporate the neutrino effects. We made 9 runs, each with $1024^3$ particles and a volume of over $(1000 h^{-1}\mathrm{Mpc})^3$ with a mass resolution of $8.15\times10^{10}h^{-1} M_\odot$. 
The initial conditions for the N-body simulations are generated using \texttt{MUSIC} with second-order Lagrangian corrections,
while the initial conditions of CDM and neutrino power spectra are obtained from the transfer function generated by \texttt{CAMB}, at the initial redshift $z=49$. 

To capture the halo assembly history, we stored 128 snapshots between $z=3$ to $z=0$ so that we can track potentially small changes of $a_{1/2}$ due to the neutrinos. The halo catalog is constructed using \texttt{Rockstar} \cite{rockstar}. The halo radius is defined to be the radius where the over-density equals $\Delta=200\rho_c$, where $\rho_c$ is the critical density of the universe. \texttt{Rockstar} is a 6D phase space friends-of-friends halo-finding algorithm, which specializes in identifying subhalos and tracking merger events. 
The halo merger tree is constructed by linking halos across different time steps using \texttt{Consistent-trees} \cite{ct} together with \texttt{Rockstar}. Finally we implement the calculation of $a_{1/2}$, $s$ and $\ell(X)$ with the built-in tool \texttt{read\_tree} inside \texttt{Consistent-trees}.
\section{Halo assembly statistics}
\label{sec:halo}

\subsection{Halo merger tree}
Dark matter halos can grow in two different ways: by accreting nearby matter or annexing other nearby self-bounded halos. A simple illustration of a halo merger tree is shown in Figure \ref{fig:MT}. For every halo existing at scale factor $a=1$ as the root, the halo merger tree branches out for each of its progenitors, reaching to the past and repeating until no progenitor is found. Once we construct the halo merger tree, the assembly history of a halo is specified, and the evolution of every halo property, such as halo mass, spin and concentration is captured.


\begin{figure}[!h]
\begin{center}
\begin{subfigure}[h]{0.48\columnwidth}
        \includegraphics[width=\columnwidth]{PNG/accretion_final.png}
        \caption{Halo assemble by smooth accretion}
\end{subfigure}
\begin{subfigure}[h]{0.48\columnwidth}
        \includegraphics[width=\columnwidth]{PNG/merger_final.png}
        \caption{Halo assemble by major merger}
\end{subfigure}
	    \caption{\label{fig:MT} Illustration of two extreme types of halo merger trees \cite{tree entropy}.} 
\end{center}
\end{figure}



Although the halo merger tree is a powerful tool to visualize how the halos assemble, it is not easy to compare two halo merger trees directly. Therefore, we use three parameters to quantify the characteristics of a halo merger tree:
the rate of growth of the halo mass, the fraction of the mass of a halo coming from mergers, and the number of protohalos merging into a single halo we observe today.
These characteristics of a halo assembly can be captured in the halo formation time $a_{1/2}$, tree entropy $s$, and halo leaf function $\ell(X)$ as we will see.


\subsection{Halo formation time $a_{1/2}$}
To characterize the mass growth rate of a halo, we can record its mass, trace one step back to the merger tree, pick the most massive progenitor (main progenitor) and repeat. It is a reduced representation of the halo assembly history called the mass accretion history (MAH) of the main branch.

We then define the halo formation time $a_{1/2}$ as the latest scale factor at which the main-branch halo reaches half of its current mass, i.e.,
\begin{equation}
    a_{1/2} = \mathrm{Max}(a') \mid M(a') = M(a=1)/2,
\end{equation}
where $M(a')$ is the main-branch halo mass at the scale factor $a'$.
We follow this traditional definition of halo formation time to quantify the rate of mass accretion, since $a_{1/2}$ was shown to be most correlated with the present-day Navarro–Frenk–White (NFW) concentration $c$ independent of the halo mass \cite{Kuan}.

Although we cannot observe the MAH of a particular halo in sky surveys, there is an observational proxy for $a_{1/2}$. 
It is known that the mass fraction of the main substructure $f_{main}$ is tightly correlated with $a_{1/2}$ in high-resolution N-body simulations \cite{Wang_2011_2}; 
the relationship is robust for different masses of the host halos. 
Using halo abundance matching (HAM), we can relate $f_{main}$ with $f_*$, which is the stellar mass fraction of the central galaxy.
The Sloan Digital Sky Survey (SDSS) data \cite{Wang_2011} shows a strong correlation between $f_*$ and galaxy properties such as color and star formation rate \cite{Lim_2015}. 
As a result, the neutrino effects on MAH, quantified by $a_{1/2}$, can be measured by direct observables.

Intuitively, as the neutrino masses suppress structure formation, halos should grow slower compared to those in the $\Lambda$CDM universe with zero neutrino mass, and we expect to see a delay in $a_{1/2}$ that depends on $M_\nu$. We do a simple quadratic fit on the middle panel of Figure 1 in \cite{Lim_2015} to get a relation between $a_{1/2}$ and $f_*$.
\begin{align}
\label{eq:emp_law}
     z_{1/2} = \frac{1}{a_{1/2}} -1 = (0.446\pm0.06) (\log_{10}(f_*))^2 + (2.4\pm0.3) \log_{10}(f_*) + (3.9\pm0.3)
\end{align}
Assuming a typical $a_{1/2}$ value of 0.5, Eq.(\ref{eq:emp_law}) implies a $1\%$ delay in $a_{1/2}$ compared to that in the $\Lambda$CDM universe with zero neutrino mass would result in a $1.4\%$ decrease in $\log_{10}(f_*)$.

\subsection{Tree entropy $s$}
Besides MAH, the merger history is another way to describe the halo assembly. Two halos may have similar MAH but entirely different merger histories (see Figure \ref{fig:ass_hist}). 

To quantify the merger history, the concept of tree entropy $s(a)$ is adopted \cite{tree entropy}. This dimensionless parameter captures the mass ratios of the mergers and the complexity of the merger tree geometry. Zero tree entropy $(s=0)$ corresponds to a tree with a single branch, i.e., with no merger. The maximum tree entropy $(s=1)$ corresponds to a fractal history of equal-mass binary mergers. The evolution of $s$ is calculated as follows: 

\begin{figure}[!ht]
	\begin{center}% Center the figure.
	\includegraphics[width=0.8\columnwidth]{PDF/ass_hist.pdf}
	\caption{\label{fig:ass_hist}Assembly histories of two halos: the main-branch halo mass $M$ (black solid line) and tree entropy $s$ (blue bashed line) are plotted against the scale factor $a$. Both halos have similar formation time and final masses, but they have different merger histories. In the top panel, there is a tree entropy spike from $0$ to $\sim 0.75$ at $a=0.2$, indicating a binary merger with almost equal mass. The bottom panel shows a halo growing by smooth accretion instead. Two minor mergers occur, but they do not make up a significant mass fraction, and therefore the tree entropy remains small.}
	\end{center}
\end{figure}
\paragraph{Halos without progenitor} 
We set the initial tree entropy $s_{init} = 0$ for any halo without progenitor; these protohalos are formed by smooth accretion.

\paragraph{Mass growth by merger} For $n$ halos that merge together, each with mass $m_i$ and tree entropy $s_i$, the new tree entropy for the merged halo $s_{new}$ is calculated by:
\begin{subequations}
\label{S_equ}
\begin{align}
    x_i &= m_i/ \sum_{i=1}^n m_i ,\\
    H &= -f \sum_{i=1}^n x^\alpha_i \ln{x_i} , \\
    s_{new} &= H + (1+bH+cH^2) \sum_{i=1}^n x^2_i(s_i - H),
\end{align}
\end{subequations}
where $f=e(\alpha-1),\, b=(2-\gamma)/f,\, c= (1-\beta)e^{1/(\alpha-1)}-1-b$ are normalization constants. The 3 parameters $\alpha, \beta$ and $\gamma$ in turn govern the behavior of the tree entropy. For instance, $\alpha$ controls the impact of the merger order.
For a lower value of $\alpha$, a high-order merger ($n$ large) produces more tree entropy relative to a low-order one ($n$ small).
We would like binary mergers $(n=2)$ to have greater impact compared to triple mergers $(n=3)$, as the latter are more "accretion-like" than the former. This condition alone will fix $\alpha=1+ 1/ \ln(2)$.
$\beta$ on the other hand, controls the impact of the most destructive merger (equal-mass binary mergers) on $s$. $\gamma$ is related to the tree entropy loss when the halo is accreting mass smoothly.


\paragraph{Mass growth by accretion} 
The evolution of $s$ for smooth accretion of mass $\Delta m = M(t_2) - M(t_1)$, assuming no merger occurs between time $t_1$ and $t_2$, needs to be consistent with Eq.(\ref{S_equ}). We break down the smooth accretion as a series of consecutive $n^{th}$-order "mergers" $p$ times, with each "merging halo" having a mass $\delta m = \Delta m/[p(n-1)]$. In the limit of $p \to \infty$, regardless of $n$, Eq.(\ref{S_equ}) becomes:
\begin{equation}
    s_{new} = \left( \frac{m}{m+\Delta m} \right) ^\gamma.
\end{equation}


Here we follow the default choices for $(\alpha,\beta,\gamma)=(1+1/ \ln(2),\, 3/4,\, 1/3)$ specified in \cite{tree entropy}.
Using the tree entropy, we can identify the merger-rich halos and select them for specific studies.
As massive neutrinos suppress the small-scale correlation due to their free-streaming effect, one might expect that the merger histories of halos would be altered significantly, as they are very sensitive to small-scale correlation. 

\subsection{Halo leaf function $\ell(X)$}
To quantify the merger histories of halos, we can also count the number of leaves (protohalos) in the halo merger trees. 
The halo leaf count $X$ is independent of $a_{1/2}$ and $s$. 
Imagine two halos with similar mass accretion histories, which give them similar $a_{1/2}$. 
One of them gains its mass from major mergers (binary mergers with comparable masses), and one of them gains its mass from many minor mergers. 
The latter must have more leaves than the former since the mass of each merging halo is smaller. 
The same argument can be applied to $s$: two halos might have similar $s$, but their $X$ can be drastically different (See Figure \ref{fig:leaf_hist}).

We define the halo leaf function $\ell(X)$ to be the number of halos with more than $X$ leaves.
The halo leaf function $\ell(X)$ is conceptually similar to the halo mass function, for which the halos are binned into different mass bins. 

\begin{figure}[htp]
	\begin{center}% Center the figure.
	\includegraphics[width=\columnwidth]{PDF/s_leaf2x2_hist.pdf}
	\caption{\label{fig:leaf_hist}Assembly histories of two halos: 
    The left panel is similar to Figure \ref{fig:ass_hist}.
	In the right panel,
	the tree entropy $s$ (black solid line) and halo leaf count $X$ (blue bashed line) are plotted against the scale factor $a$. 	
	Both halos have similar final masses and tree entropies $s\approx0.46$, but they have different $X$. 
	In the top panel, the halo gains its mass through three major mergers, shown by the spikes of $s$: each merger brings a huge impact to $s$ but only a small number of halo leaf counts to the main branch.
	The bottom panel shows a halo growing by many minor mergers instead. A huge spike of $X$ at $a=0.7$ is accompanied by only a small change in $s$, implying that the merger is minor, but the merged halo inherits the tree entropy from its entropy-rich progenitor.}

	\end{center}

\end{figure}

% obs-noise = 0.05, derivative-obs-noise = 0.2
\begin{tabular}{llll}
\toprule
            & HIP-GP & SVGP   & Exact GP \\
\midrule
RMSE        & 0.0192 & 0.0192 & 0.0192 \\
Uncertainty & 0.0198 & 0.0206 & 0.0198   \\
\bottomrule
\end{tabular}


\iffalse
% obs-noise = 0.05, derivative-obs-noise = 0.03
\begin{tabular}{llll}
\toprule
            & HIP-GP & SVGP   & Exact GP \\
\midrule
RMSE        & 0.0165 & 0.0165 & 0.0165   \\
Uncertainty & 0.0167 & 0.0175 & 0.0167   \\
\bottomrule
\end{tabular}


% obs-noise = 0.05, derivative-obs-noise = 0.1
\begin{tabular}{llll}
\toprule
            & HIP-GP & SVGP   & Exact GP \\
\midrule
RMSE        & 0.0173 & 0.0172 & 0.0173  \\
Uncertainty & 0.0181 & 0.0189 & 0.0181   \\
\bottomrule
\end{tabular}
\fi

% \vspace{-0.5em}
\section{Conclusion}
% \vspace{-0.5em}
Recent advances in multimodal single-cell technology have enabled the simultaneous profiling of the transcriptome alongside other cellular modalities, leading to an increase in the availability of multimodal single-cell data. In this paper, we present \method{}, a multimodal transformer model for single-cell surface protein abundance from gene expression measurements. We combined the data with prior biological interaction knowledge from the STRING database into a richly connected heterogeneous graph and leveraged the transformer architectures to learn an accurate mapping between gene expression and surface protein abundance. Remarkably, \method{} achieves superior and more stable performance than other baselines on both 2021 and 2022 NeurIPS single-cell datasets.

\noindent\textbf{Future Work.}
% Our work is an extension of the model we implemented in the NeurIPS 2022 competition. 
Our framework of multimodal transformers with the cross-modality heterogeneous graph goes far beyond the specific downstream task of modality prediction, and there are lots of potentials to be further explored. Our graph contains three types of nodes. While the cell embeddings are used for predictions, the remaining protein embeddings and gene embeddings may be further interpreted for other tasks. The similarities between proteins may show data-specific protein-protein relationships, while the attention matrix of the gene transformer may help to identify marker genes of each cell type. Additionally, we may achieve gene interaction prediction using the attention mechanism.
% under adequate regulations. 
% We expect \method{} to be capable of much more than just modality prediction. Note that currently, we fuse information from different transformers with message-passing GNNs. 
To extend more on transformers, a potential next step is implementing cross-attention cross-modalities. Ideally, all three types of nodes, namely genes, proteins, and cells, would be jointly modeled using a large transformer that includes specific regulations for each modality. 

% insight of protein and gene embedding (diff task)

% all in one transformer

% \noindent\textbf{Limitations and future work}
% Despite the noticeable performance improvement by utilizing transformers with the cross-modality heterogeneous graph, there are still bottlenecks in the current settings. To begin with, we noticed that the performance variations of all methods are consistently higher in the ``CITE'' dataset compared to the ``GEX2ADT'' dataset. We hypothesized that the increased variability in ``CITE'' was due to both less number of training samples (43k vs. 66k cells) and a significantly more number of testing samples used (28k vs. 1k cells). One straightforward solution to alleviate the high variation issue is to include more training samples, which is not always possible given the training data availability. Nevertheless, publicly available single-cell datasets have been accumulated over the past decades and are still being collected on an ever-increasing scale. Taking advantage of these large-scale atlases is the key to a more stable and well-performing model, as some of the intra-cell variations could be common across different datasets. For example, reference-based methods are commonly used to identify the cell identity of a single cell, or cell-type compositions of a mixture of cells. (other examples for pretrained, e.g., scbert)


%\noindent\textbf{Future work.}
% Our work is an extension of the model we implemented in the NeurIPS 2022 competition. Now our framework of multimodal transformers with the cross-modality heterogeneous graph goes far beyond the specific downstream task of modality prediction, and there are lots of potentials to be further explored. Our graph contains three types of nodes. while the cell embeddings are used for predictions, the remaining protein embeddings and gene embeddings may be further interpreted for other tasks. The similarities between proteins may show data-specific protein-protein relationships, while the attention matrix of the gene transformer may help to identify marker genes of each cell type. Additionally, we may achieve gene interaction prediction using the attention mechanism under adequate regulations. We expect \method{} to be capable of much more than just modality prediction. Note that currently, we fuse information from different transformers with message-passing GNNs. To extend more on transformers, a potential next step is implementing cross-attention cross-modalities. Ideally, all three types of nodes, namely genes, proteins, and cells, would be jointly modeled using a large transformer that includes specific regulations for each modality. The self-attention within each modality would reconstruct the prior interaction network, while the cross-attention between modalities would be supervised by the data observations. Then, The attention matrix will provide insights into all the internal interactions and cross-relationships. With the linearized transformer, this idea would be both practical and versatile.

% \begin{acks}
% This research is supported by the National Science Foundation (NSF) and Johnson \& Johnson.
% \end{acks}


\acknowledgments
We acknowledge Shek Yeung for modifying \texttt{CosmoMC} to fit the Planck 2018 data with different neutrino cosmologies and performing all the MCMC refittings. 
HWW thanks Jianxiong Chen for his mentorship on N-body simulations and Zhichao Zeng for discussion on neutrino-involved simulations.
All simulations were performed using the Central Research Computing Cluster at CUHK.
This work is supported partially by grants from the Research Grants Council of the Hong Kong Special Administrative Region, China (Project Nos. AoE/P-404/18 and C7015-19G).


\appendix
\section{Neutrino over-density linear evolution }
\label{app:phi}
We follow the derivation in \cite{Carton, LBE},  starting with the Vlasov equation:
\begin{equation}
\label{eq:a1}
    \frac{dF_\nu}{dt} = \pdv{F_\nu}{t} + \pdv{F_\nu}{r_i} \frac{dr_i}{dt} + \pdv{F_\nu}{p_i} \frac{dp_i}{dt} = 0,
\end{equation}
where $F_\nu(r_i,p_i,t)$ is the neutrino distribution function. Since we are considering a linear evolution equation, $F_\nu$ can be separated into the unperturbed Fermi-Dirac term $f_\nu^0(p)$ and a first-order perturbation $f'_\nu(\bold r, \bold p)$, i.e., $F_\nu = f_\nu^0 + f'_\nu$. In the following derivation, we will keep $f'_\nu$ up to the first order. In \texttt{Gadget2}, the Newtonian potential used is non-relativistic. Therefore,
\begin{equation}
\label{eq:a2}
    \bold{\dot p} = m \nabla \phi = -Gm \int \rho_t \frac{\bold{r - r'}}{|\bold{r-r'}|^3 } d^3r',
\end{equation}
where $\rho_t$ is the total matter energy density, including neutrinos and CDM. Substituting Eq.(\ref{eq:a2}) into Eq.(\ref{eq:a1}) and transforming to the co-moving coordinates with the follow rules:
\begin{align}
    d\chi &\equiv \frac{dt}{a^2(t)} \nonumber, \\
    \bold x &\equiv \frac{\bold r}{a(t)}, \\
    \bold u &\equiv \frac{d \bold x}{d\chi} = a(t) \bold v - \dot a(t) \bold r \nonumber,
\end{align}
we arrive at:
\begin{equation}
\label{eq:a4}
    \frac{1}{a^2}\pdv{f'_\nu}{\chi} + \frac{\bold u}{a^2} \cdot \pdv{f'_\nu}{\bold x} - \ddot a a \bold x \cdot \pdv{f_\nu^0}{\bold u} - Ga^2 \pdv{f_\nu^0}{\bold u} \cdot \int \rho_t \frac{\bold{x - x'}}{|\bold{x-x'}|^3 } d^3x' = 0.
\end{equation}
Recognizing the Dirac delta function in the integrand in Eq.(\ref{eq:a4}), we can combine it with the third term as it contains $\bold x$, and we can eliminate $\ddot a$ using the Friedmann equation:
\begin{align}
\label{eq:a5}
    \frac{\ddot a}{a} &= - \frac{4\pi G}{3} \bar \rho_t \nonumber,\\
    \frac{4\pi}{3} \bold x &= \int \frac{\bold{x - x'}}{|\bold{x-x'}|^3 } d^3x' \nonumber ,\\
    \ddot a \bold x &= -Ga \bar \rho_t \int \frac{\bold{x - x'}}{|\bold{x-x'}|^3 } d^3x',
\end{align}
where $\bar \rho_t$ is the mean total matter density. We then put Eq.(\ref{eq:a4}) to Eq.(\ref{eq:a5}) and multiply both sides by $a^2$ to obtain
\begin{equation}
\label{eq:a6}
    \pdv{f'_\nu}{\chi} + \bold u \cdot \pdv{f'_\nu}{\bold x} - Ga^4 \pdv{f_\nu^0}{\bold u} \cdot \int \bar \rho_t \delta_t( \chi, \bold x') \frac{\bold{x - x'}}{|\bold{x-x'}|^3 } d^3x' = 0,
\end{equation}
since by definition $\bar \rho_t \delta_t = \rho_t - \bar \rho_t$. Next we apply Fourier transform to Eq.(\ref{eq:a6}),  denoting $\Tilde{f}(\chi, \bold k ,\bold u) = \mathcal{F}[f(\chi, \bold x, \bold u)]$,
\begin{equation}
\label{eq:a7}
    \pdv{\Tilde{f'_\nu}}{\chi} + i\bold k \cdot \bold u \Tilde{f'_\nu} - Ga^4 \pdv{f^0_\nu}{\bold u} \int d^3x' \, \rho_t \delta_t(\chi, \bold x') \int d^3x\, e^{-i\bold k \cdot \bold x} \frac{\bold{x - x'}}{|\bold{x-x'}|^3 }  = 0.
\end{equation}
The last integral in Eq.(\ref{eq:a7}) can be evaluated,
\begin{equation}
    \int e^{-i\bold k \cdot \bold x} \frac{\bold{x - x'}}{|\bold{x-x'}|^3 } d^3x = -4\pi i \frac{\bold k}{k^2} e^{-i\bold k \cdot \bold x'}.
\end{equation}
Therefore, we have
\begin{equation}
    \pdv{\Tilde{f'_\nu}}{\chi} + i\bold k \cdot \bold u \Tilde{f'_\nu} + 4 \pi i Ga^4 \frac{\bold k}{k^2} \cdot \pdv{f^0_\nu}{\bold u} \bar \rho_t \Tilde{\delta_t}(\chi, \bold k) = 0.
\end{equation}
We then multiply both sides by $ e^{i \bold k \cdot \bold u \chi} $ and group the first two terms as a total derivative before integrating over co-moving time $\chi$,
\begin{align}
\label{eq:a10}
    \pdv{}{\chi}[\Tilde{f'_\nu} e^{i\bold k \cdot \bold u \chi}] + 4 \pi i Ga^4 e^{i \bold k \cdot \bold u \chi} \frac{\bold k}{k^2} \cdot \pdv{f^0_\nu}{\bold u} \bar \rho_t \Tilde{\delta_t}(\chi, \bold k) &= 0  \nonumber, \\
    \Tilde{f'_\nu}(\chi, \bold k, \bold u) + \int^\chi_0 4\pi iGa^4 e^{-i \bold k \cdot \bold u (\chi-\chi')} \frac{\bold k}{k^2} \cdot \pdv{f^0_\nu}{\bold u} \bar \rho_t \Tilde{\delta_t}(\chi, \bold k) d\chi' &= \Tilde{f'_\nu}(0, \bold k, \bold u) e^{-i \bold k \cdot \bold u \chi}.
\end{align}
Now Eq.(\ref{eq:a10}) is recognizable as Eq.(\ref{eq:LBE}). With an initial perturbation and the total over-density $\bar \rho_t \delta_t \equiv \bar \rho_{cdm} \delta_{cdm} + \bar \rho_\nu \delta_\nu$, we can evolve the neutrino perturbation function. Now we convert the distribution function $f'_\nu$ to over-density $\delta_\nu$ by integrating over the momentum space:
\begin{equation}
\label{eq:a11}
    \Tilde{\rho}_\nu(\chi, \bold k) + \int e^{-i \bold k \cdot \bold u (\chi-\chi')} \pdv{f^0_\nu}{\bold u} d^3u \cdot \int^\chi_0 4\pi iGa^4  \frac{\bold k}{k^2}  \bar \rho_t \Tilde{\delta_t}(\chi, \bold k) d\chi' = \int \Tilde{f'_\nu}(0, \bold k, \bold u) e^{-i \bold k \cdot \bold u \chi} d^3u.
\end{equation}
Using integration by parts and treating the perturbation as first order:
\begin{align}
\label{eq:a12}
    \int e^{-i \bold k \cdot \bold u (\chi-\chi')} \pdv{f^0_\nu}{\bold u} d^3u &= i\bold k(\chi-\chi') \int e^{-i \bold k \cdot \bold u (\chi-\chi')} {f_\nu^0 d^3u}, \\
\label{eq:a13}
    \Tilde{f'_\nu}(0, \bold k, \bold u) &\approx f_\nu^0(0, \bold u) \Tilde{\delta}_\nu(0, \bold k).
\end{align}
Putting Eq.(\ref{eq:a12}) and Eq.(\ref{eq:a13}) into Eq.(\ref{eq:a11}) and defining:
\begin{equation}
    \Phi(\bold q) \equiv \frac{\int f^0_\nu e^{-i \bold q \cdot \bold u d^3u}}{\int f^0_\nu d^3u},
\end{equation}
we have the equation governing the neutrino linear growth:
\begin{equation}
    \Tilde{\delta}_\nu (\chi,\mathbf{k}) = \Phi(\mathbf{k}\chi) \Tilde{\delta}_\nu(0,\mathbf{k}) + 
    4\pi G \int^\chi_0 a^4(\chi')(\chi-\chi') \Phi[\mathbf{k}(\chi-\chi')]
    [\Bar{\rho}_{cdm}(\chi') \Tilde{\delta}_{cdm} (\chi',\mathbf{k}) + \Bar{\rho}_\nu(\chi') \Tilde{\delta}_\nu (\chi',\mathbf{k})] d\chi'.
\end{equation}

We now turn our focus to $\Phi(\bold q)$. The denominator of $\Phi(\bold q)$ can be evaluated numerically. However, the numerator is highly oscillatory:
\begin{equation}
\label{eq:a16}
    I =\int f^0_\nu e^{-i \bold q \cdot \bold u d^3u} = 2\pi \int^\infty_0\int^\pi_0 \frac{u^2 [\cos(qu \cos\theta)-i \sin(qu\cos \theta)]\sin\theta}{e^{mu/T-\xi} +1} du d\theta + \mathrm{anti.},
\end{equation}
where anti. is the contribution from anti-neutrinos, which has $e^{mu/T+\xi} +1$ as the denominator. The imaginary part of the integrand in the right hand side of Eq.(\ref{eq:a16}) vanishes as we integrate it over $\theta$, and the integral $I$ becomes:
\begin{align}
    I &= 2\pi \int^\infty_0 \frac{2u^2 \sin(qu)}{qu(e^{mu/T-\xi} +1)} du + \mathrm{anti.} \nonumber ,\\
\label{eq:a17}
    &=4\pi \frac{T^2}{qm^2} \int^\infty_0 \left[\frac{x \sin(Ax)}{e^{x-\xi} +1 } + \frac{x \sin(Ax)}{e^{x+\xi} +1 }\right] dx,
\end{align}
where $x=mu/T$ and $A=qT/m$. We can expand the anti-neutrino term as a geometric series:
\begin{equation}
    \frac{1}{e^{x+\xi}+1}= \frac{e^{-x-\xi}}{1-(-e^{-x-\xi})} = e^{-(x+\xi)} \sum_{n=0}^\infty (-1)^n e^{-n(x+\xi)} = \sum_{n=1}^\infty (-1)^{n+1} e^{-n(x+\xi)}.
\end{equation}
The integral for anti-neutrino in Eq.(\ref{eq:a17}) becomes:
\begin{equation}
    \int^\infty_0 \frac{x \sin(Ax)}{e^{x+\xi} +1 } dx = \sum_{n=1}^\infty (-1)^{n+1} e^{-n\xi} \frac{2nA}{(A^2+n^2)^2},
\end{equation}
For the neutrino part, we separate the integral into two parts:
\begin{equation}
    \int^\infty_0 \frac{x \sin(Ax)}{e^{x-\xi} +1 } dx = \int^\xi_0 \frac{x \sin(Ax)}{e^{x-\xi} +1 } dx + \int^\infty_0 \frac{y+\xi \sin[A(y+\xi)]}{e^{y} +1 } dy.
\end{equation}
The first term can be evaluated directly, and we expand the second term again. We define:
\begin{align}
    B_1(n) &\equiv \int^\infty_0 e^{-ny} \cos(Ay) dy = \frac{n}{A^2+n^2}\nonumber, \\
    B_2(n) &\equiv \int^\infty_0 e^{-ny} \sin(Ay) dy = \frac{A}{A^2 + n^2}\nonumber, \\
    B_3(n) &\equiv \int^\infty_0 ye^{-ny} \cos(Ay) dy = \frac{n^2-A^2}{(A^2+n^2)^2} , \\
    B_4(n) &\equiv \int^\infty_0 ye^{-ny} \sin(Ay) dy = \frac{2nA}{(A^2+n^2)^2}\nonumber.
\end{align}
Therefore the numerator $I$ is,
\begin{equation}
\begin{split}
    &I = 4\pi \frac{T^2}{qm^2}\int^\xi_0 \frac{x \sin(Ax)}{e^{x-\xi} +1 } dx + 4\pi \frac{T^2}{qm^2} \times \\ 
    \sum_{n=1}^\infty (-1)^{n+1} \{ \xi B_1(n)\sin(A\xi)+&\xi B_2(n)\cos(A\xi)+B_3(n)\sin(A\xi)+B_4(n)\cos(A\xi)[\cos(A\xi)+e^{-n\xi}\},
\end{split}
\end{equation}
and this is how we evaluate $\Phi(\bold q)$ numerically.

\section{Neutrinos' effects on cosmological parameters}
\label{app:posterior}
We did not separate neutrinos' free-streaming effect from that of the CMB refitting in the N-body simulation due to the computational cost. However, we can extract the neutrinos' effect on the cosmological parameters alone, and we determine the rotation angle between the $(M_\nu, \eta^2)$ and $(x_1, x_2)$ bases. Our regression models are:
\begin{align}
\label{eq:b1}
    \Delta X(M_\nu, \eta^2)[\%] &\equiv \frac{X(M_\nu, \eta^2)}{X(0.06\,\mathrm{eV}, 0)} -1,\nonumber \\
    \Delta X(M_\nu, \eta^2)[\%] &= E_m  M' + E_\eta\eta^2, \\
    \Delta X(x_1, x_2)[\%]&= E_1 x_1 + E_2 x_2\nonumber,
\end{align}
where $X\in\{H_0, \Omega_b, \Omega_c, A_s, n_s\}$, with eigenbasis $x_1=M' \cos \rho\, -\, \eta^2 \sin \rho$ and $x_2 = M' \sin \rho \,+\, \eta^2 \cos \rho$ following similar definitions as in Eq.(\ref{eq:rot}).

\begin{table}[!h]
		\begin{center}
		\begin{tabular}{|c|c c|c|c c|}
			\hline
			Parameter & $E_m$ & $E_\eta$ & $\rho (\pi)$ & $E_1$ & $E_2$\\
			\hline \hline
			$\Delta H_0$ & $-1.62\pm0.02$ & $4.47\pm0.05$ & 0.40 & $-4.76\pm0.05$ & $-0.16 \pm 0.02$ \\
			$\Delta \Omega_b$ & $3.05\pm0.01$ & $-7.06\pm0.03$ & 0.39 & $7.68\pm0.03$ & $0.48\pm0.01$ \\
			$\Delta \Omega_c$ & $3.56\pm0.02$ & $-3.58\pm 0.04$ & 0.39 & $4.58\pm0.04$ & $2.14\pm0.02$ \\
			$\Delta A_s$ & $0.10\pm0.01$ & $2.26\pm0.02$ & 0.40 & $-2.12\pm0.02$ & $0.80\pm0.01$ \\
			$\Delta n_s$ & $-0.10\pm0.00$ & $1.77\pm0.01$ & 0.40 & $-1.71\pm0.01$ & $0.45\pm0.00$ \\
			\hline
		\end{tabular}
		\caption{\label{tab:CP_reg} Regression results for changes of cosmological parameters  as functions of ($M'$, $\eta^2$) and ($x_1, x_2)$ (see Eq.(\ref{eq:b1})).}
		\end{center}
\end{table}

From Figure \ref{fig:post} and Table \ref{tab:CP_reg} we can see that the effects of $M_\nu$ and $\eta$ on the cosmological parameters are again opposite to each other. Furthermore, all parameters share the same rotation angle $\rho$ of $0.4\pi$.

\begin{figure}[!h]
		\begin{center}% Center the figure.
		\includegraphics[width=\columnwidth]{PDF/post.pdf}
		\caption{\label{fig:post}1D posterior probability densities and 2D contours ($68\%$ and $95\%$ C.L.) for cosmological parameters extracted from the Planck 2018 CMB data, comparing $M_\nu = 0.06$ eV,  $\eta^2 = 0$ (black lines); $M_\nu = 0.24$ eV,  $\eta^2 = 0$ (red lines); and $M_\nu = 0.24$ eV,  $\eta^2 = 1$ (blue lines)} 
		\end{center}
\end{figure}








% The bibliography will probably be heavily edited during typesetting.
% We'll parse it and, using the arxiv number or the journal data, will
% query inspire, trying to verify the data (this will probalby spot
% eventual typos) and retrive the document DOI and eventual errata.
% We however suggest to always provide author, title and journal data:
% in short all the informations that clearly identify a document.

\begin{thebibliography}{99}

\bibitem{PP} Particle Data Group collaboration, C. Patrignani et al, \emph{Review of Particle Physics}, \emph{Chin. Phys.} {\bf C40} (2016) 100001. 

\bibitem{FARZAN200159} Y. Farzan, O. L. G. Peres, A. Yu. Smirnov,
\emph{Neutrino Mass Spectrum and Future Beta Decay Experiments}, \emph{Nuclear Physics B} {\bf 612} (2001) 59-97
\href{https://arxiv.org/abs/hep-ph/0105105}{ [hepo-ph/0105105]}.

\bibitem{Planck2018} Planck Collaboration, N.Aghnanim et la.,  \emph{Planck 2018 results. VI. Cosmological parameters}, \emph{A\&A} {\bf A6} (2020) 641
\href{https://arxiv.org/abs/1807.06209}{ [arxiv:1807.06209]}.

\bibitem{M nu} S. R. Choudhury and S. Hannestad, \emph{Updated results on neutrino mass and mass hierarchy from cosmology with
Planck 2018 likelihoods}, \emph{JCAP} {\bf 07} (2020) 037
\href{https://arxiv.org/abs/1907.12598}{ [arxiv:1907.12598]}.


\bibitem{camb} A. Lewis, A. Challinor, and A. Lasenby, \emph{Efficient computation of CMB anisotropies in closed FRW models}, \emph{ApJ} {\bf 538} (2000) 473-476
\href{https://arxiv.org/abs/astro-ph/9911177}{ [astro-ph/9911177]}.

\bibitem{class} D. Blas, J. Lesgourgues, and T. Tram, \emph{The Cosmic Linear Anisotropy Solving System (CLASS) II: Approximation schemes} , \emph{JCAP} {\bf 07} (2011) 034
\href{https://arxiv.org/abs/1104.2933}{ [arxiv:1104.2933]}.

\bibitem{part}J. Brandbyge and S. Hannestad \emph{Resolving Cosmic Neutrino Structure: A Hybrid Neutrino N-body Scheme}, \emph{JCAP} {\bf01} (2010) 021 \href{https://arxiv.org/abs/0908.1969}{[arxiv:0908.1969]}.

\bibitem{grid} J. Brandbyge and S. Hannestad, \emph{Grid Based Linear Neutrino Perturbations in Cosmological N-body Simulations}, \emph{JCAP} {\bf05} (2009) 002, \href{https://arxiv.org/abs/0812.3149}{[arxiv:0812.3149]}.

\bibitem{grid2} Y. Ali-Haimoud and S. Bird, \emph{An effcient implementation of massive neutrinos in non-linear structure formation simulations}, \emph{MNRAS} {\bf428} (2013) 3375 \href{https://arxiv.org/abs/1209.0461}{[arxiv:1209.0461]}.

\bibitem{Carton} Z. Zheng, S. Yeung and M. C. Chu, \emph{Effects of neutrino mass and asymmetry on cosmological structure formation}, \emph{JCAP} {\bf03} (2019) 015
\href{https://arxiv.org/abs/1808.00357}{ [arxiv:1808.00357]}.

\bibitem{Yvvone}J. Z. Chen, A. Upadhye and Yvonne Y. Y. Wong, \emph{One line to run them all: SuperEasy massive neutrino linear response in
N-body simulations}, \href{https://arxiv.org/abs/2011.12504}{[arxiv:2011.12504]}.

\bibitem{Yvvone fluid}J. Z. Chen, A. Upadhye and Yvonne Y. Y. Wong, \emph{The cosmic neutrino background as a collection of fluids in large-scale structure simulations}, 
\href{https://arxiv.org/abs/2011.12503}{[arxiv:2011.12503]}.

\bibitem{nuconcept} J. Dakin, J. Brandbyge, S. Hannestad, T. Haugbølle and T. Tram,  \emph{$\nu$CONCEPT: Cosmological neutrino simulations from the non-linear Boltzmann hierarchy}, \emph{JCAP} {\bf02} (2019) 052
\href{https://arxiv.org/abs/1712.03944}{[arxiv:1712.03944]}.

\bibitem{mu} G. Barenboim, W. H. Kinney and W. Park, \emph{ Flavor versus mass eigenstates in neutrino asymmetries: implications for cosmology}, \emph{Eur. Phys. J. C.} {\bf 77} (2017) 590
\href{https://arxiv.org/abs/1609.03200}{ [arxiv:1609.03200]}.


\bibitem{nu CMB} K.N. Abazajian et al, \emph{Neutrino Physics from the Cosmic Microwave Background and Large Scale Structure}, \emph{Astropart. Phys.} {\bf63} (2015) 66-80 \href{https://arxiv.org/abs/1309.5383}{ [arxiv:1309.5383]}.

\bibitem{bbn1} G. Mangano, G. Miele, S. Pastor, O. Pisanti, S. Sarikas, \emph{Constraining the cosmic radiation density due to lepton number with Big Bang Nucleosynthesis}, \emph{JCAP} {\bf03} (2011) 035 \href{https://arxiv.org/abs/1011.0916v3}{[arxiv:1011.0916]}.

\bibitem{bbn2} A.D. Dolgov, S.H. Hansen, S. Pastor, S.T. Petcov, G.G. Raffelt, D.V. Semikoz, \emph{A.D. Dolgov, S.H. Hansen, S. Pastor, S.T. Petcov, G.G. Raffelt, D.V. Semikoz}, \emph{Nucl. Phys. B} {\bf632} (2002) 363-382 \href{https://arxiv.org/abs/hep-ph/0201287}{[hep-ph/021287]}.


\bibitem{cosmomc} A. Lewis and S. Bridle, \emph{Cosmological parameters from CMB and other data: a Monte-Carlo
approach}, \emph{Phys. Rev. D} {\bf66} (2002) 103511 \href{https://arxiv.org/abs/astro-ph/0205436}{[astro-ph/0205436]}.


\bibitem{formtime} D. H. Zhao, Y. P. Jing, H. J. Mo, and G. Börner, \emph{Accurate universal models for the mass accretion histories and concentrations of dark matter halos}, \emph{ApJ} {\bf707} (2009) 354 \href{https://arxiv.org/abs/0811.0828}{[arxiv:0811.0828]}.

\bibitem{eta_effect} S. Yeung, King Lau and M.-C. Chu, \emph{Relic neutrino degeneracies and their impact on cosmological parameters}, \emph{JCAP} {\bf04} (2021) 024 \href{https://arxiv.org/abs/2010.01696}{[arxiv:2010.01696]}

\bibitem{tree entropy} D. Obreschkow, P. J. Elahi, C. P. Lagos, R. J. J. Poulton and A. D. Ludlow, \emph{Characterising the Structure of Halo Merger Trees Using a Single Parameter: The Tree Entropy}, \emph{MNRAS} {\bf 493} (2020) 4551-4569 \href{https://arxiv.org/abs/1911.11959}{ [arxiv:1911.11959]}.

\bibitem{Gadget2} V. Springel, \emph{The cosmological simulation code GADGET-2}, \emph{MNRAS} {\bf 364} (2005) 1105-1134
\href{https://arxiv.org/abs/astro-ph/0505010}{ [astro-ph/0505010]}.

\bibitem{LBE}S. Xiang and L. Feng, \emph{The formation of the cosmological structure}
, \emph{2ed, Astronomical Series
of NAOC}, Chinese Science and Technology Press (2012), 257-261.


\bibitem{music} O. Hahn and T. Abel, \emph{Multi-scale initial conditions for cosmological simulations}, \emph{NMRAS} {\bf415} (2011) 2101-2121 \href{https://arxiv.org/abs/1103.6031}{ [arxiv:1103.6031]}.


\bibitem{rockstar} P. S. Behroozi, R. H. Wechsler and H. Y. Wu, \emph{The Rockstar Phase-Space Temporal Halo Finder and the Velocity Offsets of Cluster Cores}, \emph{ApJ} {\bf762} (2013) 109 \href{https://arxiv.org/abs/1110.4372}{ [arxiv:1110.4372]}.

\bibitem{ct} P. S. Behroozi et al, \emph{Gravitationally Consistent Halo Catalogs and Merger Trees for Precision Cosmology}, \emph{ApJ} {\bf 763} (2013) 18 \href{https://arxiv.org/abs/1110.4370}{[arxiv:1110.4370]}.

\bibitem{Kuan} K. Wang et al, \emph{Concentrations of Dark Haloes Emerge from Their Merger Histories}, \emph{MNRAS} {\bf 498} (2020) 4450-4464 \href{https://arxiv.org/abs/2004.13732}{ [arxiv:2004.13732]}.




\bibitem{Wang_2011_2}Huiyuan Wang, H. J. Mo, Y.P. Jing, Xiaohu Yang and Yu Wang, \emph{Internal properties and environments of dark matter halos}, \emph{MNRAS} {\bf 413} (2011) 1974-1990, \href{https://arxiv.org/abs/1007.0612}{[arxiv/1007.0612]}.


\bibitem{Wang_2011} Huiyuan Wang, H. J. Mo, Xiaohu Yang and Frank C. van den Bosch, \emph{Reconstructing the cosmic velocity and tidal fields with galaxy groups selected from the Sloan Digital Sky Survey}, \emph{MNRAS} {\bf 420} 2012 1809–1824, \href{https://arxiv.org/abs/1108.1008}{[arxiv:1108.1008]}.

\bibitem{Lim_2015} Seunghwan Lim, Houjun Mo, Huiyuan Wang and Xiaohu Yang, \emph{An observational proxy of halo assembly time and its correlation with galaxy properties}, \emph{MNRAS}, {\bf 455} (2016) 499-510, \href{https://arxiv.org/abs/1502.01256}{[arxiv:1502.01256]}.


\bibitem{halo bias}T. Lazeyras, F. Villaescusa-Navarro and M. Viel, \emph{The impact of massive neutrinos on halo assembly bias}, \emph{JCAP} {\bf 03} (2021) 022,  \href{https://arxiv.org/abs/2008.12265}{[arxiv:2008.12265]} .

% Please avoid comments such as "For a review'', "For some examples",
% "and references therein" or move them in the text. In general,
% please leave only references in the bibliography and move all
% accessory text in footnotes.

% Also, please have only one work for each \bibitem.

%\bibitem{a} Author, \emph{Title}, \emph{J. Abbrev.} {\bf vol} (year) pg.

%\bibitem{b} Author, \emph{Title}, arxiv:1234.5678.

%\bibitem{c} Author, \emph{Title}, Publisher (year).



\end{thebibliography}
\end{document}
