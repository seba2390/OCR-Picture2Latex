
\section{Conclusion}
\label{sec:conclusion}
In this paper, we study the effects of neutrino masses and asymmetries on the halo assembly, including mass accretion and merger histories. Our simulations include effects of not only the neutrino free-streaming but also the refitted cosmological parameters with finite $M_\nu$ and $\eta^2$ from the Planck 2018 data.

Our simulations show that the neutrino asymmetry parameter $\eta^2$ and the sum of neutrino masses $M_\nu$ both have noticeable effects on the mean halo formation time $\bar a$ and halo leaf function $\ell(X)$.
While a larger $M_\nu$ would delay $\bar a$ and suppress $\ell(2)$, a non-zero $\eta^2$ has the opposite effect.
The mean tree entropy $\bar s$, on the other hand, is not sensitive to $M_\nu$ and has only a weak dependence on $\eta^2$. Further investigation is needed to separate the neutrino effects on the merger order and merger mass ratio.



We also present a linear regression of deviations of $\bar a$ and  $\ell(2)$ from their baseline on $M_\nu$ and $\eta^2$. Rotating to the uncorrelated bases ($\{v_1, v_2\}$ and $\{w_1, w_2\}$), we find that both $\Delta \bar a$ and $\Delta \ell_{>2}$ only depend on one combination of $M_\nu$ and $\eta^2$, $v_1$ and $w_1$, respectively. Therefore, we cannot constrain both $M_\nu$ and $\eta^2$ using only $\bar {a}$ or $\ell(2)$.


However, if we follow $\Delta \ell_{>2}$ as a function of $z$, we find that $w_1$ depends on $z$, 
and the rotation angle $\phi$ specifying the relative contributions of $M_\nu$ and $\eta^2$ experiences a smooth transition from $0.24\pi$ to $0.34\pi$. 
Similar but delayed $\phi$ transitions occur for $\Delta \ell_{>4}$ and $\Delta \ell_{>8}$ as well.
Further investigation is needed to determine the cause of such a transition.

There are some proxies to measure the $a_{1/2}$ of a halo in sky surveys. A previous study using N-body simulations with a semi-analytical model has established an empirical relation of the stellar mass ratio in the central galaxy $f_*$ and $a_{1/2}$ of its host halo \cite{Lim_2015}. 
However, the law assumes a fixed cosmology and some baryonic physics. We still need to study how the relation changes in different cosmologies, and how those baryonic physics interact with the neutrino physics even though they govern the LSS in different length scales.

The halo leaf count $X$ is a new construct to measure the number of merger events to form a single halo, and the halo leaf function $\ell(X)$ quantifies how many such merger-rich halos exist. Conceptually, the halo leaf count $X$ could be inferred by other halo observables, such as halo spin. One would expect a halo with a very large spin to have a large $X$ as it is likely formed by merging with other halos. We can then correlate $\ell$ with $X$ in the halo catalog. This is an interesting follow-up project.

When studying neutrinos using cosmological probes, the matter power spectrum is often used \cite{Carton, Yvvone, Yvvone fluid, nuconcept} as it is easy to compute and the effects are significant even if we only consider the neutrino free-streaming effect. 
However, as shown in \cite{halo bias}, the halo observables, such as the halo spin and NFW concentration often show unnoticeable differences for a wide range of $M_\nu$. 
The halo assembly history may be a good starting point to explore the effects of neutrinos on halo properties.
Here, we have shown that $a_{1/2}$ and $\ell_{>2}$ can be used to break the parameter degeneracy between $M_\nu$ and $\eta^2$ so that they can both be measured in principle.