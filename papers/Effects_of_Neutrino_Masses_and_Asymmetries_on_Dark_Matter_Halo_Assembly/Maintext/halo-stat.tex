\section{Halo assembly statistics}
\label{sec:halo}

\subsection{Halo merger tree}
Dark matter halos can grow in two different ways: by accreting nearby matter or annexing other nearby self-bounded halos. A simple illustration of a halo merger tree is shown in Figure \ref{fig:MT}. For every halo existing at scale factor $a=1$ as the root, the halo merger tree branches out for each of its progenitors, reaching to the past and repeating until no progenitor is found. Once we construct the halo merger tree, the assembly history of a halo is specified, and the evolution of every halo property, such as halo mass, spin and concentration is captured.


\begin{figure}[!h]
\begin{center}
\begin{subfigure}[h]{0.48\columnwidth}
        \includegraphics[width=\columnwidth]{PNG/accretion_final.png}
        \caption{Halo assemble by smooth accretion}
\end{subfigure}
\begin{subfigure}[h]{0.48\columnwidth}
        \includegraphics[width=\columnwidth]{PNG/merger_final.png}
        \caption{Halo assemble by major merger}
\end{subfigure}
	    \caption{\label{fig:MT} Illustration of two extreme types of halo merger trees \cite{tree entropy}.} 
\end{center}
\end{figure}



Although the halo merger tree is a powerful tool to visualize how the halos assemble, it is not easy to compare two halo merger trees directly. Therefore, we use three parameters to quantify the characteristics of a halo merger tree:
the rate of growth of the halo mass, the fraction of the mass of a halo coming from mergers, and the number of protohalos merging into a single halo we observe today.
These characteristics of a halo assembly can be captured in the halo formation time $a_{1/2}$, tree entropy $s$, and halo leaf function $\ell(X)$ as we will see.


\subsection{Halo formation time $a_{1/2}$}
To characterize the mass growth rate of a halo, we can record its mass, trace one step back to the merger tree, pick the most massive progenitor (main progenitor) and repeat. It is a reduced representation of the halo assembly history called the mass accretion history (MAH) of the main branch.

We then define the halo formation time $a_{1/2}$ as the latest scale factor at which the main-branch halo reaches half of its current mass, i.e.,
\begin{equation}
    a_{1/2} = \mathrm{Max}(a') \mid M(a') = M(a=1)/2,
\end{equation}
where $M(a')$ is the main-branch halo mass at the scale factor $a'$.
We follow this traditional definition of halo formation time to quantify the rate of mass accretion, since $a_{1/2}$ was shown to be most correlated with the present-day Navarro–Frenk–White (NFW) concentration $c$ independent of the halo mass \cite{Kuan}.

Although we cannot observe the MAH of a particular halo in sky surveys, there is an observational proxy for $a_{1/2}$. 
It is known that the mass fraction of the main substructure $f_{main}$ is tightly correlated with $a_{1/2}$ in high-resolution N-body simulations \cite{Wang_2011_2}; 
the relationship is robust for different masses of the host halos. 
Using halo abundance matching (HAM), we can relate $f_{main}$ with $f_*$, which is the stellar mass fraction of the central galaxy.
The Sloan Digital Sky Survey (SDSS) data \cite{Wang_2011} shows a strong correlation between $f_*$ and galaxy properties such as color and star formation rate \cite{Lim_2015}. 
As a result, the neutrino effects on MAH, quantified by $a_{1/2}$, can be measured by direct observables.

Intuitively, as the neutrino masses suppress structure formation, halos should grow slower compared to those in the $\Lambda$CDM universe with zero neutrino mass, and we expect to see a delay in $a_{1/2}$ that depends on $M_\nu$. We do a simple quadratic fit on the middle panel of Figure 1 in \cite{Lim_2015} to get a relation between $a_{1/2}$ and $f_*$.
\begin{align}
\label{eq:emp_law}
     z_{1/2} = \frac{1}{a_{1/2}} -1 = (0.446\pm0.06) (\log_{10}(f_*))^2 + (2.4\pm0.3) \log_{10}(f_*) + (3.9\pm0.3)
\end{align}
Assuming a typical $a_{1/2}$ value of 0.5, Eq.(\ref{eq:emp_law}) implies a $1\%$ delay in $a_{1/2}$ compared to that in the $\Lambda$CDM universe with zero neutrino mass would result in a $1.4\%$ decrease in $\log_{10}(f_*)$.

\subsection{Tree entropy $s$}
Besides MAH, the merger history is another way to describe the halo assembly. Two halos may have similar MAH but entirely different merger histories (see Figure \ref{fig:ass_hist}). 

To quantify the merger history, the concept of tree entropy $s(a)$ is adopted \cite{tree entropy}. This dimensionless parameter captures the mass ratios of the mergers and the complexity of the merger tree geometry. Zero tree entropy $(s=0)$ corresponds to a tree with a single branch, i.e., with no merger. The maximum tree entropy $(s=1)$ corresponds to a fractal history of equal-mass binary mergers. The evolution of $s$ is calculated as follows: 

\begin{figure}[!ht]
	\begin{center}% Center the figure.
	\includegraphics[width=0.8\columnwidth]{PDF/ass_hist.pdf}
	\caption{\label{fig:ass_hist}Assembly histories of two halos: the main-branch halo mass $M$ (black solid line) and tree entropy $s$ (blue bashed line) are plotted against the scale factor $a$. Both halos have similar formation time and final masses, but they have different merger histories. In the top panel, there is a tree entropy spike from $0$ to $\sim 0.75$ at $a=0.2$, indicating a binary merger with almost equal mass. The bottom panel shows a halo growing by smooth accretion instead. Two minor mergers occur, but they do not make up a significant mass fraction, and therefore the tree entropy remains small.}
	\end{center}
\end{figure}
\paragraph{Halos without progenitor} 
We set the initial tree entropy $s_{init} = 0$ for any halo without progenitor; these protohalos are formed by smooth accretion.

\paragraph{Mass growth by merger} For $n$ halos that merge together, each with mass $m_i$ and tree entropy $s_i$, the new tree entropy for the merged halo $s_{new}$ is calculated by:
\begin{subequations}
\label{S_equ}
\begin{align}
    x_i &= m_i/ \sum_{i=1}^n m_i ,\\
    H &= -f \sum_{i=1}^n x^\alpha_i \ln{x_i} , \\
    s_{new} &= H + (1+bH+cH^2) \sum_{i=1}^n x^2_i(s_i - H),
\end{align}
\end{subequations}
where $f=e(\alpha-1),\, b=(2-\gamma)/f,\, c= (1-\beta)e^{1/(\alpha-1)}-1-b$ are normalization constants. The 3 parameters $\alpha, \beta$ and $\gamma$ in turn govern the behavior of the tree entropy. For instance, $\alpha$ controls the impact of the merger order.
For a lower value of $\alpha$, a high-order merger ($n$ large) produces more tree entropy relative to a low-order one ($n$ small).
We would like binary mergers $(n=2)$ to have greater impact compared to triple mergers $(n=3)$, as the latter are more "accretion-like" than the former. This condition alone will fix $\alpha=1+ 1/ \ln(2)$.
$\beta$ on the other hand, controls the impact of the most destructive merger (equal-mass binary mergers) on $s$. $\gamma$ is related to the tree entropy loss when the halo is accreting mass smoothly.


\paragraph{Mass growth by accretion} 
The evolution of $s$ for smooth accretion of mass $\Delta m = M(t_2) - M(t_1)$, assuming no merger occurs between time $t_1$ and $t_2$, needs to be consistent with Eq.(\ref{S_equ}). We break down the smooth accretion as a series of consecutive $n^{th}$-order "mergers" $p$ times, with each "merging halo" having a mass $\delta m = \Delta m/[p(n-1)]$. In the limit of $p \to \infty$, regardless of $n$, Eq.(\ref{S_equ}) becomes:
\begin{equation}
    s_{new} = \left( \frac{m}{m+\Delta m} \right) ^\gamma.
\end{equation}


Here we follow the default choices for $(\alpha,\beta,\gamma)=(1+1/ \ln(2),\, 3/4,\, 1/3)$ specified in \cite{tree entropy}.
Using the tree entropy, we can identify the merger-rich halos and select them for specific studies.
As massive neutrinos suppress the small-scale correlation due to their free-streaming effect, one might expect that the merger histories of halos would be altered significantly, as they are very sensitive to small-scale correlation. 

\subsection{Halo leaf function $\ell(X)$}
To quantify the merger histories of halos, we can also count the number of leaves (protohalos) in the halo merger trees. 
The halo leaf count $X$ is independent of $a_{1/2}$ and $s$. 
Imagine two halos with similar mass accretion histories, which give them similar $a_{1/2}$. 
One of them gains its mass from major mergers (binary mergers with comparable masses), and one of them gains its mass from many minor mergers. 
The latter must have more leaves than the former since the mass of each merging halo is smaller. 
The same argument can be applied to $s$: two halos might have similar $s$, but their $X$ can be drastically different (See Figure \ref{fig:leaf_hist}).

We define the halo leaf function $\ell(X)$ to be the number of halos with more than $X$ leaves.
The halo leaf function $\ell(X)$ is conceptually similar to the halo mass function, for which the halos are binned into different mass bins. 

\begin{figure}[htp]
	\begin{center}% Center the figure.
	\includegraphics[width=\columnwidth]{PDF/s_leaf2x2_hist.pdf}
	\caption{\label{fig:leaf_hist}Assembly histories of two halos: 
    The left panel is similar to Figure \ref{fig:ass_hist}.
	In the right panel,
	the tree entropy $s$ (black solid line) and halo leaf count $X$ (blue bashed line) are plotted against the scale factor $a$. 	
	Both halos have similar final masses and tree entropies $s\approx0.46$, but they have different $X$. 
	In the top panel, the halo gains its mass through three major mergers, shown by the spikes of $s$: each merger brings a huge impact to $s$ but only a small number of halo leaf counts to the main branch.
	The bottom panel shows a halo growing by many minor mergers instead. A huge spike of $X$ at $a=0.7$ is accompanied by only a small change in $s$, implying that the merger is minor, but the merged halo inherits the tree entropy from its entropy-rich progenitor.}

	\end{center}

\end{figure}
