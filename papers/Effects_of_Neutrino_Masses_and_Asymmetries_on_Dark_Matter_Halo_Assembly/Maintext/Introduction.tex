
\section{Introduction}
\label{sec:intro}
There are still many open questions about neutrinos, such as what their masses are, and whether they are Dirac or Majorana particles. 
Neutrino flavor oscillation experiments show that at least two types of neutrinos must be massive. Assuming normal hierarchy and the smallest neutrino mass eigenvalue to be zero, (i.e. $m_{\nu_3} >> m_{\nu_2} > m_{\nu_1}=0$, $m_{\nu_i}$, $i=1,2,3$ denoting mass eigenstates), we can put a lower bound on $M_\nu \equiv \sum_i m_{\nu_i} > 0.06$ eV \cite{PP}.


Particle physics experiments such as neutrinoless double beta decay measurements and neutrino mass spectrum experiments also try to answer these open questions. If neutrinos are Majorana particles, neutrinoless double beta decays are possible.
However, there is no solid evidence for such decay events yet. Since the mass of neutrinos is related to the decay rate, the non-observation of neutrinoless double beta decay suggests an upper bound on the effective Majorana mass of electron-type neutrinos $m_{ee}=\abs{\sum_i U_{ei}^2 m_{\nu_i}}<0.34$ eV (90\% C.L.) \cite{FARZAN200159}.

Neutrino mass spectrum experiments, on the other hand, measure the endpoint of the electron energy spectrum in beta decays. The upper bound on the mass of anti-electron neutrinos $m_{\Bar{\nu}_e} < 2.5$ eV (95\% C.L.) was obtained from Troitzk's results \cite{PP}. 
This bound from direct detection is quite conservative, and it is only valid for electron-type neutrinos. 
We know nothing about muon and tau neutrinos from these experiments.

The Cosmic Neutrino Background (C$\nu$B) has long been studied, but direct detection of the C$\nu$B is very difficult since neutrinos only participate in weak and gravitational interactions. Currently, the tightest upper bound on $M_\nu$ is provided by Planck from Cosmic Microwave Background (CMB) Anisotropies data. 
A non-zero $M_\nu$ will delay the radiation-matter equality epoch and modify the Hubble expansion rate which in turn impacts the CMB power spectrum. Planck 2018 gives a constraint on $M_\nu<0.12$ eV (95\% C.L.) \cite{Planck2018,M nu}.

Besides CMB data, the Large-Scale Structure (LSS) in the Universe is another powerful tool to study neutrino cosmology, as the LSS is more sensitive to the value of $M_\nu$ than CMB.
The structure growth is governed by three competing factors: the expansion of the universe, the kinetic energy and self-gravitation of matter, and massive neutrinos play a role in all of them. 
In the standard $\Lambda$CDM model, neutrinos are treated as radiation, whereas massive neutrinos transform from being ultra-relativistic (radiation-like) to non-relativistic (matter-like) as the universe expands and cools, resulting in a small but non-negligible change in the expansion history compared to that of $\Lambda$CDM. 
Cosmological neutrinos are considered as Hot Dark Matter (HDM) with large thermal velocities. They do not cluster significantly on small scale and tend to erase structures below the free-stream scale $k_{fs}$. This neutrino free-streaming effect is well studied in the linear regime using Boltzmann codes such as \texttt{CAMB} \cite{camb} and \texttt{CLASS} \cite {class}. 
However, the linear method breaks down when the density contrasts exceed unity, such as in a dark matter halo. In the non-linear regime, the N-body method should be used. Different methods have been proposed to incorporate neutrino effects into N-body simulations, such as the particle-based \cite{part}, grid-based \cite{grid, grid2, Carton}, "SuperEasy" \cite{Yvvone} and fluid-based \cite{Yvvone fluid, nuconcept} methods. 
They give consistent results, showing that massive neutrinos suppress the matter power spectrum below their free-streaming scale ($k>>k_{fs}$) by up to $\mathcal{O}(5\%)$ for $M_\nu = 0.06$ eV compared to that without the free-streaming neutrinos, and such a difference would be measurable with modern observation programs.


Another property of neutrinos, which governs the asymmetries of neutrinos and anti-neutrinos, is the neutrino chemical potentials $\{\mu_i\}$. The chemical potentials of anti-neutrinos would be $\{-\mu_i\}$. If neutrinos are Majorana particles, $\mu_i=0$.
As the neutrino distribution functions are frozen after decoupling, the dimensionless quantities $\{\xi_i=\mu_i/k_BT\}$ are constant throughout the expansion of the universe, where $k_B$ and $T$ are the Boltzmann constant and neutrino temperature, respectively.
We also know from Big Bang Nucleosynthesis (BBN) that the chemical potential for electron-type neutrinos is very small.
We follow \cite{Carton} and \cite{mu} to set $\xi_e=0$ and $\xi_\mu=\xi_\tau$, so that we have only one independent parameter, $\eta^2 = \sum_i \xi_i^2$, denoted the neutrino asymmetry parameter. 
The fact that the muon and tau neutrinos have strong mixing, as shown in neutrino oscillation experiments, makes $\xi_\mu=\xi_\tau$ a good approximation \cite{bbn1, bbn2}. Currently, the total neutrino asymmetry $\sum_i \xi_i$ is tightly constrained by the BBN Helium-4 mass fraction $Y_p$. However, $Y_p$ itself cannot constrain $\eta^2$, particularly since $\xi_1$ and $\xi_2$ tend to have opposite signs and cancel each other quite well \cite{eta_effect}. $\eta^2$ is only mildly constrained by other observational data such as the CMB power spectrum as we will discuss below.

When neutrino masses and asymmetries are considered, the cosmological parameters obtained from fitting of CMB data would be different from those of $\Lambda$CDM \cite{nu CMB}, and any change in the cosmology will then affect the LSS formation. To ensure self-consistency, refitting of cosmological parameters is needed for each $M_\nu$ and $\eta^2$, using a Markov-Chain Monte-Carlo (MCMC) code such as \texttt{CosmoMC} \cite{cosmomc}. In Planck 2018, the standard cosmological parameters are obtained by assuming three neutrino species, two massless states $(m_{\nu_1} = m_{\nu_2} = 0)$ plus a single massive neutrino of mass $m_{\nu_3} = 0.06$ eV, without any neutrino asymmetries \cite{Planck2018}. Therefore, we choose a baseline of $M_\nu = 0.06$ eV and $\eta^2 = 0$ to compare against when analyzing the simulation results.

The authors in \cite{Carton} explicitly tested the effect of free-streaming neutrinos with chemical potential on the matter power spectrum, and the results in Figure 3a in \cite{Carton} show that the free-streaming effect of neutrinos is not affected by $\eta^2$.
Rather, the effect of $\eta^2$ on the structure growth is mainly due to the changes in the refitted cosmological parameters and therefore the expansion history of the universe.
In \cite{eta_effect}, $M_\nu$ and $\eta^2$ are treated as free parameters and varied together with other $\Lambda$CDM parameters to fit the Planck CMB data. The finding is that $M_\nu$ and $\eta^2$ work against each other in virtually every cosmological parameter.
For instance, while $M_\nu$ has a negative correlation with both $H_0$ and $\sigma_8$, $\eta^2$ has a positive correlation with both (see Figure 10 in \cite{eta_effect}). 
We would then expect $M_\nu$ and $\eta^2$ to have the opposite effects on the structure growth via their effects on the cosmological parameters.


Indeed, a recent study showed that there is a parameter degeneracy between $M_\nu$ and $\eta^2$ in their effects on the matter power spectrum, as a non-zero $\eta^2$ would enhance the matter power spectrum and compensate the suppression from a finite $M_\nu$ \cite{Carton}. To break such a degeneracy, other cosmological observables should be considered. Since massive neutrinos suppress large-scale structures, it's natural to ask how neutrinos alter the merging and assembling of dark matter halos, as these processes are highly non-linear and very sensitive to the initial conditions of halo formation. 

There are two different ways to look at the halo assembly history: the mass accretion history and the halo merger history. The latter, characterized by the halo merger tree, keeps track of how smaller halos merge to become a bigger halo, which is different from the concept of the mass growth rate. The mass accretion history is easily quantified by the time needed for a particular halo to double its mass, which is the traditional definition of the halo formation time $a_{1/2}$ \cite{formtime}. 

Recently, the idea of the tree entropy $s$ was proposed \cite{tree entropy}, which is based on Shannon's information entropy. The tree entropy $s$ captures the complexity and geometry of a halo merger tree. The tree entropy $s$ is shown to be information-rich and especially useful for linking the galaxies to their host halos. For example, the morphology of a galaxy is closely related to the merger history of its host halo; many semi-analytical models use the merger tree of the galaxy's host halo to predict its morphology. 
A clear positive correlation between the tree entropy $s$ and the galaxy's bulge-to-total mass ratio was found in the mock galaxy catalog generated by a semi-analytical model \cite{tree entropy}. If neutrinos indeed bring a significant impact to $s$, we may be able to constrain $M_\nu$ and $\eta^2$ by measuring the morphologies of galaxies.

Finally, the halo leaf function $\ell(X)$ is defined to be the number of halos with more than $X$ leaves in their merger tree. Such a measure is conceptually similar to the halo mass function, but we bin the halos according to their merger histories instead of their masses. 

In this work, we investigate the effects of the sum of neutrino masses $M_\nu$ and neutrino asymmetry parameter $\eta^2$ on the halo assembly history, by studying the halo formation time $a_{1/2}$, tree entropy $s$ and halo leaf function $\ell(X)$. These parameters together with the matter power spectrum will allow us to break the parameter degeneracy between $M_\nu$ and $\eta^2$.

This paper is organised as follows. In Section \ref{sec:grid} we briefly introduce the grid-based neutrino method in N-body simulations along with the simulation parameters. The halo assembly statistics are elaborated in Section \ref{sec:halo}. The simulation results are examined in Section \ref{sec:result}, where we present an empirical formula for the neutrino effects on the halo assembly statistics. The discussion and conclusion are in Section \ref{sec:conclusion}. 

