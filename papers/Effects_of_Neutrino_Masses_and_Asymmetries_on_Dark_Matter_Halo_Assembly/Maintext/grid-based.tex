\section{Grid-based neutrino simulation}
\label{sec:grid}

To study the neutrino free-streaming effect on LSS, we can include neutrinos as another type of particles in N-body simulations. This particle-based method is accurate in principle but computationally expensive.
Not only will it bring extra particles and interactions, but it also requires more integration steps compared to the Cold Dark Matter (CDM)-only simulation with the same number of particles due to the larger velocity dispersion of the neutrinos. 
On the other hand, the grid-based neutrino-involved simulation includes only CDM particles in the simulation box. The neutrino information is carried by the neutrino over-density field $\delta_\nu$ contained in the Particle-Mesh (PM) grid, which is responsible for the long-range interaction in a Tree-PM code like \texttt{Gadget2} \cite{Gadget2}. 

Another advantage of the grid-based method is that we can also investigate the effects of the chemical potentials of neutrinos, which can be incorporated easily in the simulation through the Fermi-Dirac distribution of the cosmological neutrinos.

\subsection{Linear evolution for neutrino over-density}
The linear equation that governs the evolution of $\delta_\nu$ is \cite{LBE}: 
\begin{equation}
\label{eq:LBE}
    \Tilde{\delta}_\nu (\chi,\mathbf{k}) = \Phi(\mathbf{k}\chi) \Tilde{\delta}_\nu(0,\mathbf{k}) +
    4\pi G \int^\chi_0 a^4(\chi')(\chi-\chi') \Phi[\mathbf{k}(\chi-\chi')]
    [\Bar{\rho}_{cdm}(\chi') \Tilde{\delta}_{cdm} (\chi',\mathbf{k}) + \Bar{\rho}_\nu(\chi') \Tilde{\delta}_\nu (\chi',\mathbf{k})] d\chi'.
\end{equation}


Here, $\Tilde{\delta}_\nu$ ($\Tilde{\delta}_{cdm}$) and $\bar \rho_\nu$  $(\bar\rho_{cdm})$ are the over-density in Fourier space and mean density of neutrinos (CDM), respectively. $\bf k$ is the wave vector, $\chi$ is the co-moving coordinate where $d\chi=dt/a^2(t)$, and $\Phi$ is a special function that will be discussed in Appendix \ref{app:phi}.

Neutrinos cannot cluster below their free-streaming scale, which is larger than their non-linear scales; therefore, the evolution of neutrinos is well described by the linear equation. Although we use a linear equation to describe the evolution of $\delta_\nu$, the non-linear $\delta_{cdm}$ (from N-body simulation) is involved in the evolution of neutrinos. Hence, the non-linearities in structure formation are still fully preserved. 
Previous studies have also shown that both particle-based and grid-based simulations produce consistent results for the matter power spectrum \cite{Carton}. 


\subsection{Total over-density}
Once we obtain the neutrino over-density $\delta_\nu$, the total over-density field $\delta_t$ is then given by:
\begin{equation}
\label{eq:avg}
    \delta_t = (1-f_\nu)\delta_{cdm} + f_\nu \delta_\nu,
\end{equation}
where $f_\nu=\Omega_\nu / \Omega_m$, the ratio of the cosmological neutrino and CDM densities. Although $f_\nu$ is small, the non-linearity in structure formation will mix up different modes, and the final density $\delta_t $ may change by a factor much greater than $(1- f_\nu)$.

\subsection{Implementation}
We implemented the grid-based neutrino method in our own modified version of \texttt{Gadget2}. The detailed procedure is discussed in \cite{Carton}. Here we briefly summarize the steps:

\begin{enumerate}
    \item The initial snapshot generated by \texttt{MUSIC} \cite{music} and initial neutrino power spectrum generated by \texttt{CAMB} are fed into \texttt{Gadget2} as the initial conditions, and thus we have both $\Tilde{\delta}_\nu(0,k)$ and $\Tilde{\delta}_{cdm}(0,k)$.
    
    \item To evolve the system, the CDM particles are drifted first. With a new CDM power spectrum after the drift $\Tilde{\delta}_{cdm}(\chi, k)$, we solve Eq.(\ref{eq:LBE}) iteratively. We use linear interpolation to approximate $\Tilde{\delta}_{cdm}(\chi', k)$ for $\chi' \in [0,\chi]$ as the time difference is usually small between two PM steps.
    
    \item The over-densities are assumed to carry the same phase:
    \begin{equation}
        \Tilde{\delta}_\nu (\chi, {\bf k}) = \frac{\Tilde{\delta}_\nu(\chi, k)}{\Tilde{\delta}_{cdm}(\chi, k)} \Tilde{\delta}_{cdm}(\chi, {\bf k}).
    \end{equation}
    The total over-density field $\Tilde{\delta}_t(\chi, {\bf k})$ is then obtained using Eq.(\ref{eq:avg}). Finally the original $\Tilde{\delta}_{cdm}(\chi, {\bf k})$ is replaced by $\Tilde{\delta}_t(\chi, {\bf k})$ to give the correct PM potential to evolve the CDM particles.
    
    \item $\Tilde{\delta}_\nu(\chi, k)$ and $\Tilde{\delta}_{cdm}(\chi, k)$ are stored as the initial conditions for the next PM calculation, and we iterate back to step 1 until the final time (usually today).
    
\end{enumerate}

\subsection{Simulation parameters}


To specify the fiducial cosmology for our N-body simulation, 6 cosmological parameters are needed. 
They are the physical CDM density $\Omega_ch^2$, the physical baryon density $\Omega_bh^2$, the observed angular size of the sound horizon at recombination $\theta$, the reionization optical depth $\tau$, 
the initial super-horizon amplitude of curvature perturbations $A_s$ at $k$ = 0.05 Mpc$^{-1}$ and the primordial spectral index $n_s$. 
All of them are consistently refitted from the Planck CMB data using the Planck 2018 \texttt{plikHM\_TTTEEE} likelihood for each set of $M_\nu$ and $\eta^2$ values. The parameters relevant to the N-body simulations are listed in Table \ref{tab:simparam}, and their posterior distributions for selected sets of $M_\nu$ and $\eta^2$ are plotted in Figure \ref{fig:post}. 


\begin{table}[!hbt]
		\begin{center}
		% Table itself: here we have two columns which are centered and have lines to the left, right and in the middle: |c|c|
		\begin{tabular}{|c|c|c|c|c|c|c|c|c|c|c|}
			% To create a horizontal line, type \hline
			\hline
			% To end a column type &
			% For a linebreak type \\
			No. &$M_\nu$ & $\eta^2 $ & $H_0$ & $\Omega_c+\Omega_b$ & $\Omega_\nu$ &$\Omega_\Lambda$ & $\sigma_8$ & $n_s$ & $A_s \,(10^{-9})$\\
			\hline
			\hline
%			A0 & 0 & 0 & 67.74 & 0.3097 & 0 & 0.6903 \\
			A1 & 0.06 & 0 & 67.37 & 0.3141 & 0.00140 & 0.6845 & 0.814 & 0.965 & 2.10 \\
			A2 & 0.06 & 0.253 & 68.13 & 0.3107 & 0.00142 & 0.68788 & 0.818 & 0.969 & 2.11\\
			A3 & 0.06 & 1.012 & 70.49  & 0.3009 & 0.00145 & 0.69765 & 0.833 & 0.982 & 2.15 \\
			\hline
			B1 & 0.15 & 0 & 66.43 & 0.3236 & 0.0036 & 0.6728 & 0.794 & 0.964 & 2.10 \\
			B2 & 0.15 & 0.253 & 67.12 & 0.3208 & 0.00365 & 0.67555 & 0.799 & 0.968 & 2.11\\
			B3 & 0.15 & 1.012 & 69.44 & 0.3106 & 0.00375 & 0.68565 & 0.811 & 0.981 & 2.15\\
			\hline
            C1 & 0.24 & 0 & 65.46 & 0.3337 & 0.00595 & 0.66035 & 0.775 & 0.963 & 2.11\\
			C2 & 0.24 & 0.253 & 66.17 & 0.3308 & 0.00600 & 0.6632 & 0.778 & 0.967 & 2.12\\
			C3 & 0.24 & 1.012 & 68.39 & 0.3192 & 0.00618 & 0.67462 & 0.790 & 0.981 & 2.15\\
			\hline
		\end{tabular}
		\caption{\label{tab:simparam} N-body simulation parameters.}
		\end{center}
\end{table}


Simulation snapshots are generated using our modified version of $\texttt{Gadget2}$ to incorporate the neutrino effects. We made 9 runs, each with $1024^3$ particles and a volume of over $(1000 h^{-1}\mathrm{Mpc})^3$ with a mass resolution of $8.15\times10^{10}h^{-1} M_\odot$. 
The initial conditions for the N-body simulations are generated using \texttt{MUSIC} with second-order Lagrangian corrections,
while the initial conditions of CDM and neutrino power spectra are obtained from the transfer function generated by \texttt{CAMB}, at the initial redshift $z=49$. 

To capture the halo assembly history, we stored 128 snapshots between $z=3$ to $z=0$ so that we can track potentially small changes of $a_{1/2}$ due to the neutrinos. The halo catalog is constructed using \texttt{Rockstar} \cite{rockstar}. The halo radius is defined to be the radius where the over-density equals $\Delta=200\rho_c$, where $\rho_c$ is the critical density of the universe. \texttt{Rockstar} is a 6D phase space friends-of-friends halo-finding algorithm, which specializes in identifying subhalos and tracking merger events. 
The halo merger tree is constructed by linking halos across different time steps using \texttt{Consistent-trees} \cite{ct} together with \texttt{Rockstar}. Finally we implement the calculation of $a_{1/2}$, $s$ and $\ell(X)$ with the built-in tool \texttt{read\_tree} inside \texttt{Consistent-trees}.