\section{Neutrino over-density linear evolution }
\label{app:phi}
We follow the derivation in \cite{Carton, LBE},  starting with the Vlasov equation:
\begin{equation}
\label{eq:a1}
    \frac{dF_\nu}{dt} = \pdv{F_\nu}{t} + \pdv{F_\nu}{r_i} \frac{dr_i}{dt} + \pdv{F_\nu}{p_i} \frac{dp_i}{dt} = 0,
\end{equation}
where $F_\nu(r_i,p_i,t)$ is the neutrino distribution function. Since we are considering a linear evolution equation, $F_\nu$ can be separated into the unperturbed Fermi-Dirac term $f_\nu^0(p)$ and a first-order perturbation $f'_\nu(\bold r, \bold p)$, i.e., $F_\nu = f_\nu^0 + f'_\nu$. In the following derivation, we will keep $f'_\nu$ up to the first order. In \texttt{Gadget2}, the Newtonian potential used is non-relativistic. Therefore,
\begin{equation}
\label{eq:a2}
    \bold{\dot p} = m \nabla \phi = -Gm \int \rho_t \frac{\bold{r - r'}}{|\bold{r-r'}|^3 } d^3r',
\end{equation}
where $\rho_t$ is the total matter energy density, including neutrinos and CDM. Substituting Eq.(\ref{eq:a2}) into Eq.(\ref{eq:a1}) and transforming to the co-moving coordinates with the follow rules:
\begin{align}
    d\chi &\equiv \frac{dt}{a^2(t)} \nonumber, \\
    \bold x &\equiv \frac{\bold r}{a(t)}, \\
    \bold u &\equiv \frac{d \bold x}{d\chi} = a(t) \bold v - \dot a(t) \bold r \nonumber,
\end{align}
we arrive at:
\begin{equation}
\label{eq:a4}
    \frac{1}{a^2}\pdv{f'_\nu}{\chi} + \frac{\bold u}{a^2} \cdot \pdv{f'_\nu}{\bold x} - \ddot a a \bold x \cdot \pdv{f_\nu^0}{\bold u} - Ga^2 \pdv{f_\nu^0}{\bold u} \cdot \int \rho_t \frac{\bold{x - x'}}{|\bold{x-x'}|^3 } d^3x' = 0.
\end{equation}
Recognizing the Dirac delta function in the integrand in Eq.(\ref{eq:a4}), we can combine it with the third term as it contains $\bold x$, and we can eliminate $\ddot a$ using the Friedmann equation:
\begin{align}
\label{eq:a5}
    \frac{\ddot a}{a} &= - \frac{4\pi G}{3} \bar \rho_t \nonumber,\\
    \frac{4\pi}{3} \bold x &= \int \frac{\bold{x - x'}}{|\bold{x-x'}|^3 } d^3x' \nonumber ,\\
    \ddot a \bold x &= -Ga \bar \rho_t \int \frac{\bold{x - x'}}{|\bold{x-x'}|^3 } d^3x',
\end{align}
where $\bar \rho_t$ is the mean total matter density. We then put Eq.(\ref{eq:a4}) to Eq.(\ref{eq:a5}) and multiply both sides by $a^2$ to obtain
\begin{equation}
\label{eq:a6}
    \pdv{f'_\nu}{\chi} + \bold u \cdot \pdv{f'_\nu}{\bold x} - Ga^4 \pdv{f_\nu^0}{\bold u} \cdot \int \bar \rho_t \delta_t( \chi, \bold x') \frac{\bold{x - x'}}{|\bold{x-x'}|^3 } d^3x' = 0,
\end{equation}
since by definition $\bar \rho_t \delta_t = \rho_t - \bar \rho_t$. Next we apply Fourier transform to Eq.(\ref{eq:a6}),  denoting $\Tilde{f}(\chi, \bold k ,\bold u) = \mathcal{F}[f(\chi, \bold x, \bold u)]$,
\begin{equation}
\label{eq:a7}
    \pdv{\Tilde{f'_\nu}}{\chi} + i\bold k \cdot \bold u \Tilde{f'_\nu} - Ga^4 \pdv{f^0_\nu}{\bold u} \int d^3x' \, \rho_t \delta_t(\chi, \bold x') \int d^3x\, e^{-i\bold k \cdot \bold x} \frac{\bold{x - x'}}{|\bold{x-x'}|^3 }  = 0.
\end{equation}
The last integral in Eq.(\ref{eq:a7}) can be evaluated,
\begin{equation}
    \int e^{-i\bold k \cdot \bold x} \frac{\bold{x - x'}}{|\bold{x-x'}|^3 } d^3x = -4\pi i \frac{\bold k}{k^2} e^{-i\bold k \cdot \bold x'}.
\end{equation}
Therefore, we have
\begin{equation}
    \pdv{\Tilde{f'_\nu}}{\chi} + i\bold k \cdot \bold u \Tilde{f'_\nu} + 4 \pi i Ga^4 \frac{\bold k}{k^2} \cdot \pdv{f^0_\nu}{\bold u} \bar \rho_t \Tilde{\delta_t}(\chi, \bold k) = 0.
\end{equation}
We then multiply both sides by $ e^{i \bold k \cdot \bold u \chi} $ and group the first two terms as a total derivative before integrating over co-moving time $\chi$,
\begin{align}
\label{eq:a10}
    \pdv{}{\chi}[\Tilde{f'_\nu} e^{i\bold k \cdot \bold u \chi}] + 4 \pi i Ga^4 e^{i \bold k \cdot \bold u \chi} \frac{\bold k}{k^2} \cdot \pdv{f^0_\nu}{\bold u} \bar \rho_t \Tilde{\delta_t}(\chi, \bold k) &= 0  \nonumber, \\
    \Tilde{f'_\nu}(\chi, \bold k, \bold u) + \int^\chi_0 4\pi iGa^4 e^{-i \bold k \cdot \bold u (\chi-\chi')} \frac{\bold k}{k^2} \cdot \pdv{f^0_\nu}{\bold u} \bar \rho_t \Tilde{\delta_t}(\chi, \bold k) d\chi' &= \Tilde{f'_\nu}(0, \bold k, \bold u) e^{-i \bold k \cdot \bold u \chi}.
\end{align}
Now Eq.(\ref{eq:a10}) is recognizable as Eq.(\ref{eq:LBE}). With an initial perturbation and the total over-density $\bar \rho_t \delta_t \equiv \bar \rho_{cdm} \delta_{cdm} + \bar \rho_\nu \delta_\nu$, we can evolve the neutrino perturbation function. Now we convert the distribution function $f'_\nu$ to over-density $\delta_\nu$ by integrating over the momentum space:
\begin{equation}
\label{eq:a11}
    \Tilde{\rho}_\nu(\chi, \bold k) + \int e^{-i \bold k \cdot \bold u (\chi-\chi')} \pdv{f^0_\nu}{\bold u} d^3u \cdot \int^\chi_0 4\pi iGa^4  \frac{\bold k}{k^2}  \bar \rho_t \Tilde{\delta_t}(\chi, \bold k) d\chi' = \int \Tilde{f'_\nu}(0, \bold k, \bold u) e^{-i \bold k \cdot \bold u \chi} d^3u.
\end{equation}
Using integration by parts and treating the perturbation as first order:
\begin{align}
\label{eq:a12}
    \int e^{-i \bold k \cdot \bold u (\chi-\chi')} \pdv{f^0_\nu}{\bold u} d^3u &= i\bold k(\chi-\chi') \int e^{-i \bold k \cdot \bold u (\chi-\chi')} {f_\nu^0 d^3u}, \\
\label{eq:a13}
    \Tilde{f'_\nu}(0, \bold k, \bold u) &\approx f_\nu^0(0, \bold u) \Tilde{\delta}_\nu(0, \bold k).
\end{align}
Putting Eq.(\ref{eq:a12}) and Eq.(\ref{eq:a13}) into Eq.(\ref{eq:a11}) and defining:
\begin{equation}
    \Phi(\bold q) \equiv \frac{\int f^0_\nu e^{-i \bold q \cdot \bold u d^3u}}{\int f^0_\nu d^3u},
\end{equation}
we have the equation governing the neutrino linear growth:
\begin{equation}
    \Tilde{\delta}_\nu (\chi,\mathbf{k}) = \Phi(\mathbf{k}\chi) \Tilde{\delta}_\nu(0,\mathbf{k}) + 
    4\pi G \int^\chi_0 a^4(\chi')(\chi-\chi') \Phi[\mathbf{k}(\chi-\chi')]
    [\Bar{\rho}_{cdm}(\chi') \Tilde{\delta}_{cdm} (\chi',\mathbf{k}) + \Bar{\rho}_\nu(\chi') \Tilde{\delta}_\nu (\chi',\mathbf{k})] d\chi'.
\end{equation}

We now turn our focus to $\Phi(\bold q)$. The denominator of $\Phi(\bold q)$ can be evaluated numerically. However, the numerator is highly oscillatory:
\begin{equation}
\label{eq:a16}
    I =\int f^0_\nu e^{-i \bold q \cdot \bold u d^3u} = 2\pi \int^\infty_0\int^\pi_0 \frac{u^2 [\cos(qu \cos\theta)-i \sin(qu\cos \theta)]\sin\theta}{e^{mu/T-\xi} +1} du d\theta + \mathrm{anti.},
\end{equation}
where anti. is the contribution from anti-neutrinos, which has $e^{mu/T+\xi} +1$ as the denominator. The imaginary part of the integrand in the right hand side of Eq.(\ref{eq:a16}) vanishes as we integrate it over $\theta$, and the integral $I$ becomes:
\begin{align}
    I &= 2\pi \int^\infty_0 \frac{2u^2 \sin(qu)}{qu(e^{mu/T-\xi} +1)} du + \mathrm{anti.} \nonumber ,\\
\label{eq:a17}
    &=4\pi \frac{T^2}{qm^2} \int^\infty_0 \left[\frac{x \sin(Ax)}{e^{x-\xi} +1 } + \frac{x \sin(Ax)}{e^{x+\xi} +1 }\right] dx,
\end{align}
where $x=mu/T$ and $A=qT/m$. We can expand the anti-neutrino term as a geometric series:
\begin{equation}
    \frac{1}{e^{x+\xi}+1}= \frac{e^{-x-\xi}}{1-(-e^{-x-\xi})} = e^{-(x+\xi)} \sum_{n=0}^\infty (-1)^n e^{-n(x+\xi)} = \sum_{n=1}^\infty (-1)^{n+1} e^{-n(x+\xi)}.
\end{equation}
The integral for anti-neutrino in Eq.(\ref{eq:a17}) becomes:
\begin{equation}
    \int^\infty_0 \frac{x \sin(Ax)}{e^{x+\xi} +1 } dx = \sum_{n=1}^\infty (-1)^{n+1} e^{-n\xi} \frac{2nA}{(A^2+n^2)^2},
\end{equation}
For the neutrino part, we separate the integral into two parts:
\begin{equation}
    \int^\infty_0 \frac{x \sin(Ax)}{e^{x-\xi} +1 } dx = \int^\xi_0 \frac{x \sin(Ax)}{e^{x-\xi} +1 } dx + \int^\infty_0 \frac{y+\xi \sin[A(y+\xi)]}{e^{y} +1 } dy.
\end{equation}
The first term can be evaluated directly, and we expand the second term again. We define:
\begin{align}
    B_1(n) &\equiv \int^\infty_0 e^{-ny} \cos(Ay) dy = \frac{n}{A^2+n^2}\nonumber, \\
    B_2(n) &\equiv \int^\infty_0 e^{-ny} \sin(Ay) dy = \frac{A}{A^2 + n^2}\nonumber, \\
    B_3(n) &\equiv \int^\infty_0 ye^{-ny} \cos(Ay) dy = \frac{n^2-A^2}{(A^2+n^2)^2} , \\
    B_4(n) &\equiv \int^\infty_0 ye^{-ny} \sin(Ay) dy = \frac{2nA}{(A^2+n^2)^2}\nonumber.
\end{align}
Therefore the numerator $I$ is,
\begin{equation}
\begin{split}
    &I = 4\pi \frac{T^2}{qm^2}\int^\xi_0 \frac{x \sin(Ax)}{e^{x-\xi} +1 } dx + 4\pi \frac{T^2}{qm^2} \times \\ 
    \sum_{n=1}^\infty (-1)^{n+1} \{ \xi B_1(n)\sin(A\xi)+&\xi B_2(n)\cos(A\xi)+B_3(n)\sin(A\xi)+B_4(n)\cos(A\xi)[\cos(A\xi)+e^{-n\xi}\},
\end{split}
\end{equation}
and this is how we evaluate $\Phi(\bold q)$ numerically.
