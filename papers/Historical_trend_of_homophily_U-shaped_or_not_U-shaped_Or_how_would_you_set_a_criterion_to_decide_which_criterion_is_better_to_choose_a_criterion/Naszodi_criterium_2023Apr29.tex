

\documentclass[12pt,letterpaper]{article}


%\usepackage[showframe,paper=a4paper,margin=1in]{geometry}


\usepackage{subcaption}

\usepackage{setspace} % load setspace before footmisc
\usepackage{footmisc}
\setlength{\footnotesep}{\baselineskip} %use 1.67\baselineskip for a double space

\usepackage{amsfonts}
\usepackage{amsmath}
\usepackage{theorem}

\usepackage{multirow}

% set margins 
\usepackage[top=2.5cm, bottom=2.5cm, left=2.5cm, right=2.5cm]{geometry}


\usepackage{amsmath}
\usepackage{xcolor}

\newcommand\overmat[2]{%
	\makebox[0pt][l]{$\smash{\color{white}\overbrace{\phantom{%
					\begin{matrix}#2\end{matrix}}}^{\text{\color{black}#1}}}$}#2}
\newcommand\bovermat[2]{%
	\makebox[0pt][l]{$\smash{\overbrace{\phantom{%
					\begin{matrix}#2\end{matrix}}}^{\text{#1}}}$}#2}
\newcommand\partialphantom{\vphantom{\frac{\partial e_{P,M}}{\partial w_{1,1}}}}

%special characters heart
\usepackage{pifont}
%\usepackage{arev}

%natbib.sty 
\usepackage{url}
\usepackage[T1]{fontenc}

% make entire document double spaced
\usepackage{setspace}
\doublespacing

% ensure footnotes are full sized and double spaced
\usepackage{footmisc}
%\renewcommand{\footnotelayout}{\doublespacing\normalsize}
\renewcommand{\footnotelayout}{\doublespacing\small}
\setlength{\skip\footins}{8mm}

% remove page numbers
%\usepackage{nopageno}

% left justify text
%\makeatletter
%\newcommand\iraggedright{%
%	\let\\\@centercr\@rightskip\@flushglue \rightskip\@rightskip
%	\leftskip\z@skip}
%\makeatother
%\iraggedright

% set bibtex bibliography to chicago style
%\bibliographystyle{chicago}
\usepackage[natbibapa]{apacite}
\bibliographystyle{apacite}



%with these I optain & in the reference
%\usepackage{inputenc}
\usepackage{amssymb}
%\usepackage{subfigure}
%\usepackage{natbib}
\setcitestyle{aysep={}} 

%\usepackage{natbib}
%\bibliographystyle{jpe}
\bibliographystyle{elsarticle-harv}

\usepackage{url}
%\usepackage{hyperref}


\usepackage{mathtools}
%\usepackage{amsfonts}
\usepackage{amsmath}
\usepackage[latin2]{inputenc}
\usepackage{graphicx}
\usepackage{subcaption}
\usepackage{multirow}
\usepackage{esvect}
\usepackage{afterpage}

\usepackage{lipsum}
\usepackage{booktabs,array,dcolumn}

%to rotate table
\usepackage{adjustbox}
\usepackage{graphicx}
\usepackage{pdflscape}

\usepackage{caption}
\usepackage[labelfont=bf]{caption}
%\usepackage[font=bf]{caption}
\usepackage[figurename=FIGURE]{caption}
\usepackage[tablename=TABLE]{caption}

% save ampersand for other uses:
\let\ampersand\&
% let \& to give ``und''
\renewcommand*\&{and}

%\usepackage[natbib]{biblatex}

%\usepackage[backend=bibtex,giveninits=false]{biblatex}

% Method proposed in "The LaTeX Companion", 2nd ed.:
\makeatletter
\def\@seccntformat#1{\@ifundefined{#1@cntformat}%
	{\csname the#1\endcsname\space}%    default
	{\csname #1@cntformat\endcsname}}%  enable individual control
\newcommand\section@cntformat{\thesection.\space}       % section-level
\newcommand\subsection@cntformat{\thesubsection.\space} % subsection-level
\makeatother

\usepackage{atbegshi}% http://ctan.org/pkg/atbegshi
\AtBeginDocument{\AtBeginShipoutNext{\AtBeginShipoutDiscard}}

\begin{document}
	\setcounter{page}{1}	
	%	\addtocounter{page}{-1}
	\begin{titlepage}
		%\thispagestyle{empty}	
		\date{}
		
		\begin{flushleft}
			
			\title{\Large  \textbf{Historical trend of homophily: U-shaped or not U-shaped?    
					Or, how would you set a criterion to decide which criterion is better to choose a criterion?}}
			
			
			
		\end{flushleft}
		
		\begin{flushleft}
						\singlespacing{\author{Anna NASZODI*}}
			
\thanks{*Email: anna.naszodi@gmail.com, Anna Naszodi is an honorary member of the KRTK-KTI.}


		\end{flushleft}
		

		
		\maketitle
		\setcounter{page}{1}
		
		%\newpage
		\noindent 
		
		\singlespacing{\textbf{Abstract:}  
			
Measuring the extent to which educational marital homophily differs in two consecutive generations is challenging when the educational distributions of marriageable men and women are also generation-specific. 
We propose a set of criteria that indicators may have to satisfy to be considered as suitable measures of homophily.  
One of our analytical criteria is on the robustness to the number of educational categories. 
Another analytical criterion defined by us  is on the association between intergenerational mobility and homophily. 
A third criterion is empirical and concerns the identified historical trend of homophily, a comprehensive aspect of inequality, in the US between 1960 and 2015.    
While the ordinal Liu--Lu-indicator and the cardinal indicator constructed with the Naszodi--Mendonca method satisfy all three criteria, most indices commonly applied in the literature do not. 
Our analysis sheds light on the link between the violation of certain criteria and the sensitivity of the historical trend of homophily 
obtained in the empirical assortative mating literature.}
	
		
		\begin{flushleft}
			\small{\textbf{Keywords:}
				Assortative mating;  Counterfactual decomposition; Educational homophily;  Iterative proportional fitting algorithm;  NM-method.}\\
			\small{\textbf{JEL classification:}  C02, C18, D63, J12.}
		\end{flushleft}
		
		
		
	\end{titlepage}

%%%%%%%%%%%%%%%%%%%%%%%%%%%%%%%%%%%%%%%%%%%%%%%%55

\newpage


\section{Introduction}\label{sec:intro}

	Inequality and social cohesion has became a prominent part of the political discourse since the publication of the seminal book by \cite{Piketty2014}.  
	Studying marital homophily by economists and sociologists, while dating back several decades (see \citealp{Becker1981} and \citealp{Mare1991}),  
	has also attracted renewed attention in the recent years.   
	The reason is that the strength of homophily is informative about the width of the social gap between different groups:    
	if homophily is strengthening then it signals an increase in inequality.    
	More precisely, a rise in inclination of people in a given generation relative to their peers in an older generation to choose a partner from one's own social group signals an increasing inequality between the groups.\footnote{Conversely, a change in the opposite direction in aggregate marital/ mating preferences over inter-group versus intra-group romantic relationships is a sign of decreasing inequality and strengthening social cohesion between the groups.}  
	
	Most of the papers in assortative mating written by economists study homogamy along the monetary dimensions, such as income or education, where the latter is used as a proxy of one's ability to generate income. Thereby, the representatives of the economic discipline working on the field of marital homophily  analyze inequality between people in different income brackets or different education groups typically.  
	
	This paper contributes to the educational assortative mating literature. 
	In particular, \textit{we shed light on the link  between the  historical trend of homophily, as a specific aspect of inequality, and some choices about the data used and the models/methods applied.}  
	
	
	In the educational assortative mating literature, changes in the directly unobservable homophily is quantified either (i) directly with changes in some indicators of homophily  (e.g. the correlation between husbands and wives years of schooling, or the odds ratio of their contingency table) calculated from the joint educational distributions of couples in the generations compared, or, (ii) indirectly through its effect on the observable inter-generational change in the prevalence of homogamy by controlling for changes in some other determinants of the share of homogamous couples. Typically, papers applying the indirect approach control for changes in the structural determinant of the prevalence of homogamy, i.e., the educational distribution of marriageable men and women to identify the non-structural determinant. {Since the directly unobservable non-structural drivers, -- such as marital preferences, marital norms, social barriers to marrying out of ones' own group, residential and school segregation limiting who meets whom, -- shape and signal homophily, the broad category of the non-structural factors are simply referred to and identified as homophily.}         
	
	As a third alternative, we mention, -- however, we counter-recommend to apply--, the practice of (iii) quantifying changes in homophily directly with changes in an ordinal indicator while overlooking the fact that the indicator is not cardinal.\footnote{This practice is applied by a recent paper,   
			where assortative mating is found to have began to ``sharply increase'' after 2000 in the US with an ordinal measure.}  
      
		
    There is a growing agreement in the literature over the historical trend of income inequality.
	In particular, it is a commonly held view that this dimension of inequality did not change monotonically over time in the US, having first decreased during and after the Great Depression, while it begun to increase around 1980   (see \citealp{PikettySaez2003},    \citealp{SaezZucman2016}). 
	Even though there is still an ongoing debate about the exact shape of the trend, the stylized  U-curve pattern itself has not been challenged (see \citealp{Bricker2016}, \citealp{AutenSplinter2022},  \citealp{Geloso2022}). Thus, it is found to be robust to how  income inequality is quantified (e.g. by the top 1 percent,  or the top 10 percent income share). 
		
By contrast, the assortative mating literature is divided about the historical trend of educational marital homophily. {As it is noted by  \cite{NaszodiMendonca2023_RACEDU}, there is no consensus despite the fact that  ``... unlike the studies on income and wealth inequality, the papers in the assortative mating literature do not perform any perilous exercise of patching together data from different sources. Their main input,  the joint educational distribution of couples,  is provided by the statistical offices  `packed up and parceled' ready for analysis. Also, while under-reporting of income and wealth is a general concern of the researchers, under-reporting  of education level  is not.''}  

	In this paper, {we make the following points: the lack of consensus is not surprising because there is no agreement about   
	  what indicators to use for characterizing homophily.} 
  {Also, among the papers that identify homophily indirectly from the prevalence of homogamy, a disagreement concerns how to control for changes in the structural determinant of the share of homogamous couples.}
    	
	
As a third point, we stress that there is no consensus over the set of analytical criteria any indicator has to satisfy to provide a suitable measure of homophily. 
	Relatedly, we formulate our view based on the regression argument of Sextus Empiricus 
	 that  
	there is probably no consensus over the criteria fit for deciding which set of analytical criteria to use.\footnote{See the comics entitled ``How the exchange of ideas moves science forward?'' \url{https://www.sites.google.com/site/annanaszodi/funnyphilosophical} }  
	
Finally, inspired by the forming consensus in the income inequality literature, we propose and apply an empirical criterion against the direct and the indirect measures of educational homophily: 
suitable measures should exhibit a U-shaped inter-generational trend in the US for the generations born during and after the Great Depression irrespective of where one looks at the educational distribution.  
 
	 

The structure of this paper is the following.  
First,  we review a comprehensive set of analytical criteria and an equally comprehensive set of homophily measures used in the literature.   
Then, we highlight the link between the criteria chosen and the identified historical trend of homophily.  
Finally, we conclude that there is only one indicator and there is only one indirect identification method among those reviewed   
 that do not violate our empirical criterion.   
 
 

\section{Measuring  homophily}\label{sec:measur}

In this section, we review a set of cardinal and ordinal \textit{homophily indicators}   
that are candidates for characterizing  the strength of homophily.   
In addition, we visit a set of \textit{methods} put forward in the literature for identifying changes in the strength of homophily indirectly. 


\subsection{Measuring  homophily with indicators}\label{sec:measur_ind}

We study 10 indicators that we define by their closed form formulas in the simplest case of dichotomous assorted trait, i.e., 
where the education level can be either low (L), or  high (H).  

The joint educational distribution of couples together with the educational distributions of single women and single men  are represented by    
the following  contingency table:\\ 
%The joint educational distribution of the couples in the population of this generation in year $t$ is given by the 2-by-2 contingency table $K$:\\
\begin{table}[ht]
	\begin{center}
		\caption{Contingency table with two education levels and singles}
		\begin{tabular}{ll c c c c}  \hline \hline
			$Q$	&    & \multicolumn{4}{c}{Women}         \\ 
			&    & \multicolumn{3}{c}{in couple} &         \\  \cline{3-5}
			&    &                          L  & H & sum  & single \\ \cline{3-6} 
			\parbox[t]{2mm}{\multirow{3}{*}{\rotatebox[origin=c]{90}{Men}}} &L  & $a$     & $b$     & $a+b$  &  $e$ \\ 
			&                             H & $c$     & $d$     & $c+d$ &  $f$   \\  \cline{3-5}
			&sum                                  & $a+c$   & $b+d$   & $a+b+c+d$ & \\ 
			&single                               &  $g$    & $h$ &         &  \\   \hline \hline
		\end{tabular}
		\label{tab:CT}
		\vspace{2mm}\\
	\end{center}
\end{table}


Next, we present the 10 indicators and their formulas.\\
(I1) Odds-ratio (it is the most widely used indicator according to \citealp{Chiapporietal2021}): 
\begin{equation}\label{eq:OR} 
\text{OR}(Q)= ad/bc  \;. 
\end{equation} 


(I2) Matrix determinant (suggested by \citealp{Permanyer2013} and applied by \citealp{Permanyer2019}):
\begin{equation}\label{eq:MD} 
\text{det}(Q)= ad-bc  \;. 
\end{equation} 


(I3) Covariance coefficient (applied by \citealp{Class2017}): 
\begin{equation}\label{eq:cov} 
\text{cov}(Q)= \frac{\text{det}(Q)}{(a+b+c+d)^2}  \;. 
\end{equation} 


(I4) Correlation coefficient (applied by \citealp{Kremer1997} and \citealp{FernandezEtAl2005}):
\begin{equation}\label{eq:cor} 
\rho(Q)=\frac{\text{det}(Q)}{\sqrt{(a+b)(c+d)(a+c)(b+d)}}\;. 
\end{equation} 


(I5) Regression coefficient (obtained by regressing either the male partners' education on the female partners' education, or the other way around. 
The latter indicator was applied by \citealp{Greenwood2014}): 
\begin{equation}\label{eq:reg} 
\beta_{wm}(Q)=\text{cov}(Q) \left[\frac{b+d}{a+b+c+d}-\left(\frac{b+d}{a+b+c+d}\right)^2 \right] \; \text{or}  
\end{equation} 
\begin{equation*}
\beta_{mw}(Q)=\text{cov}(Q) \left[\frac{c+d}{a+b+c+d}-\left(\frac{c+d}{a+b+c+d}\right)^2 \right]   \;, 
\end{equation*} 
depending on whether wives' dichotomous education variable (taking the value of 0 or 1) is explained by husbands' dichotomous education variable (taking the value of 0 or 1), or the other way around.




(I6) Aggregate marital sorting parameter (proposed and applied by  \citealp{Eika2019}): 
\begin{equation*}
\mbox{MSP}^{\mbox{agg}}(Q)= \frac{\mbox{MSP}_L(Q) a +\mbox{MSP}_H(Q) d }{a+d}  
\end{equation*}
  is the
weighted average of the marital sorting parameters (i.e., $\mbox{MSP}_L(Q)$ and $\mbox{MSP}_H(Q)$) along the diagonal of the contingency table.
Unlike $\mbox{MSP}^{\mbox{agg}}(Q)$, the marital sorting parameters  are local measures of sorting:  
 for L,L couples it is  
\begin{equation*}
\mbox{MSP}_L(Q)= \frac{{a}/(a+b+c+d)}{a^{\text{counterf.}}/(a^{\text{counterf.}}+b^{\text{counterf.}}+c^{\text{counterf.}}+d^{\text{counterf.}})}  \;,
\end{equation*}
while for  
H,H couples, it is  
\begin{equation*}
\mbox{MSP}_H(Q)= \frac{{d}/(a+b+c+d)}{d^{\text{counterf.}}/(a^{\text{counterf.}}+b^{\text{counterf.}}+c^{\text{counterf.}}+d^{\text{counterf.}})}  \;,
	\end{equation*}
where $\mbox{MSP}_L(Q)$ captures the probability that  an L-type man  marries  an L-type woman, relative to
the probability under a counterfactual. Whereas  $\mbox{MSP}_H(Q)$ captures the same likelihood ratio, but for the H,H-type couples.


If the counterfactual is the random matching, as it is in the paper by \cite{Eika2019},  then the denominator of $\mbox{MSP}_L(Q)$ is $\frac{ (a+b)(a+c) }{(a+b+c+d)^2} $ and 
the denominator of $\mbox{MSP}_H(Q)$ is $\frac{ (c+d)(b+d) }{(a+b+c+d)^2}$ making 
$\mbox{MSP}_L(Q)= \frac{a (a+b+c+d)} {(a+b)(a+c)}$ and 
$\mbox{MSP}_H(Q)= \frac{d (a+b+c+d)} {(c+d)(b+c)}$.
Finally, under the random counterfactual, 
\begin{equation}\label{eq:amsp}
\mbox{MSP}^{\mbox{agg}}(Q)=  \frac{a+b+c+d}{a+d}  \left(\frac{a^2 } {(a+b)(a+c)} + \frac{d^2 } {(c+d)(b+c)}  \right)  \;. 
\end{equation}


(I7) V-value (proposed by \citealp{FernandezRogerson2001} and applied also by \citealp{Abbott2019}):
\begin{equation}\label{eq:v} 
\text{V}(Q)=\frac{\text{det}(Q)}{A}\;,
\end{equation} 
where $A=(c+d)(a+c)$ if $b \geq c$ and $A=(b+d)(a+b)$ if $c > b$.\footnote{\cite{Chiapporietal2021} write that the V-indicator coincides  with the LL-indicator in the context of their analysis. This is not the case in the context of our analysis.} 


(I8) Marital surplus matrix (proposed by \citealp{ChooSiow2006}):
\begin{equation}\label{eq:CS}
\text{MSM}(Q)= \begin{bmatrix}
a/\sqrt{eg}    & b/\sqrt{eh} \\
c/\sqrt{fg}   & d/\sqrt{fh}
\end{bmatrix}   \;. 
\end{equation}
   

(I9) LL-indicator (proposed by \citealp{LiuLu2006}). 
The formula of the Simplified LL-indicator:  
\begin{equation}\label{eq:LL} 
\text{LL}^s(Q)=\frac{d - \text{int}(R)  }{\text{min}(b+d, c+d )-\text{int}(R) }\;, 
\end{equation} 
where $R=(c+d)(b+d)/(a+b+c+d)$.
The original LL-indicator is identical to its  \textit{simplified  version} in case of positively assorted trait such as education.


(I10) Matrix-valued  generalized LL-indicator (proposed by \citealp{NaszodiMendonca2021}):  
\begin{equation}\label{LiuLugengen}
\text{LL}^{\text{gen}}_{j,k} (Q)= 
\text{LL}( V_j  Q  W^T_k )    \;,
\end{equation}
where $\text{LL}^{\text{gen}}_{j,k} (Q)$ is the  $(j,k)$-th  element of the $\text{LL}^{\text{gen}}$ matrix in case   
of  $Q$ is an $n \times  m$ matrix  with $n \geq 2$, or $m \geq 2$ , or both.   
Further,    
$V_j$ is the $2 \times  n$ matrix \vspace{6mm} \\ 
$V_j = \scriptsize{ \begin{bmatrix}
	\bovermat{\textit{j}}{1    & \cdots &  1} & \bovermat{\textit{n-j}}{ 0  & \cdots & 0}  \\
	0    & \cdots  & 0 & 1  & \cdots  & 1  	
	\end{bmatrix} }$   and  
$W^T_k$ is the $m \times 2$ matrix given by the transpose of \vspace{6mm} \\
$W_k = \scriptsize{ \begin{bmatrix}
	\bovermat{\textit{k}}{1    & \cdots & 1} & \bovermat{\textit{m-k}}{ 0  & \cdots  & 0}  \\
	0    & \cdots  & 0 & 1  & \cdots  & 1  	
	\end{bmatrix} }$ with   $ j \in \{1, \ldots, n-1 \} $, and  $k \in \{1, \ldots, m-1 \}$.  



\subsection{Measuring  homophily with counterfactual decompositions}\label{sec:measur_meth}

   
We study a set of \textit{methods} $\{M_1,...M_7\}$ for quantifying changes in homophily indirectly from an earlier generation to a later generation.    
It is a common feature of the methods to be reviewed that they \textit{construct a counterfactual contingency table} $Q_{M_i}^{\text{counterf.}}(Q,P)$ from the observed contingency tables $Q$ and $P$ characterizing the two generations, respectively. The counterfactual population of couples resembles one generation in terms of the structural factor, while it resembles the other generation in terms of the non-structural factor. 
Once $Q_{M_i}^{\text{counterf.}}(Q,P)$ is constructed, one can calculate how the prevalence of homogamy would have been changed if only the non-structural factor had changed from the earlier generation to the later generation. 

Thereby, one can use the methods to quantify the contribution of the non-structural factor to the change in 
the share of homogamous couples. We use the contribution calculated as our \emph{indirect measure of the change in homophily}. 
Since $Q_{M_i}^{\text{counterf.}}(Q,P)$ depends on the method $M_i$ itself, the identified change in homophily  
is also method-specific.          

The assortative mating literature offers the following methods for constructing counterfactuals:\\
(M1) Iterative Proportional Fitting (henceforth IPF) algorithm, where the non-structural factor (typically referred to as the degree of marital sorting) is kept fixed with the unchanged odds-ratios.  
The \cite{BS2005} paper is an example for an early application of the IPF in the context of analyzing assortative mating with counterfactual decomposition.\\ 
(M2) Matrix Determinant-based Approach (henceforth MDbA), where the non-structural factor is controlled for with the matrix determinant. It was suggested  by \cite{Permanyer2013} and applied by \cite{Permanyer2019}.\\  
(M3) Minimum Euclidean Distance Approach (henceforth MEDA), where the non-structural factor is kept fixed with the unchanged scalar-valued V-indicator. The V-indicator interprets as the weight minimizing the Euclidean distance between the matrix to be transformed and the matrix obtained as the convex combination of the two extreme cases of random and perfectly assortative matching.  MEDA was  applied by \cite{FernandezRogerson2001} and \cite{Abbott2019}.\\
(M4) \cite{ChooSiow2006} model-based approach (henceforth CSA), where the non-structural factor is kept fixed with the unchanged marital surplus matrix.\\ 
(M5) NM-approach, where the non-structural factor (referred to as the aggregate marital preferences over a single dimensional spousal trait) are controlled for by the unchanged (generalized) LL-indicator. 
It was proposed by \cite{NaszodiMendonca2021} and first applied by \cite{NaszodiPB2019}.\footnote{About the difference between the NM-method and the IPF as well as about the difference between the NM-method and the MEDA, see \url{https://en.wikipedia.org/wiki/NM-method}.}\\  
(M6) GNM-approach, where the aggregate marital preferences over multiple traits (e.g.,  spousal education level and race) is kept fixed with the unchanged (generalized) trait-specific LL-indicators. It was developed and applied by \cite{NaszodiMendonca2023_RACEDU}.\\ 
(M7) GS-approach, where matching is made by the Gale--Shapley matching algorithm, while the nonstructural factor, i.e., the aggregate marital preferences over a single dimensional trait is kept fixed with the unchanged gender-specific and education level-specific distributions of the reservation points. It was proposed and applied by \cite{NaszodiMendonca2019}, who used the search criteria on a dating site as a proxy for the reservation points.\\



\section{The set of analytical criteria}\label{sec:Acrit}


We introduce the notation $\text{SMH}(Q)$ for the \emph{suitable measure of homophily} that is assumed to meet the following set of \textit{analytical criteria}:\\
(AC1) being a \textit{cardinal} measure.\\
(AC2) \textit{Scale invariance}, i.e., invariance to the change in the total number of couples, while the couples' joint educational distribution is unchanged: $\text{SMH}(Q)=\text{SMH}(rQ)$ for any $r \in \mathbb{R^+}$.\\
(AC3) \textit{Gender symmetry}, i.e, invariance to interchanging wives' data and husbands' data: $\text{SMH}(Q)=\text{SMH}(Q^T)$, where $Q^T$ denotes the transpose of $Q$.\\
(AC4) \textit{Category symmetry}, defined for the special case of distinguishing only two educational categories of wives and husbands. It is a criterion on the invariance to interchanging the low and the high categories both for wives and husbands:
$\text{SMH} \left(\begin{bmatrix}
a    & b \\
c   & d
\end{bmatrix}  \right) =\text{SMH} \left(\begin{bmatrix}
d    & c \\
b   & a
\end{bmatrix}  \right)$.\\
(AC5)  \textit{Immunity to a certain class of changes in the marginal distributions}.
E.g.  immunity to the type-1 changes in the marginals\\
$\;\;$(AC5.1):
\begin{equation}\label{eq:AC5.1}
\text{SMH} \left(\begin{bmatrix}
a    & b \\
c   & d
\end{bmatrix}  \right) =\text{SMH} \left(\begin{bmatrix}
\alpha a    & \alpha b \\
c   & d
\end{bmatrix}  \right)=
\text{SMH} \left(\begin{bmatrix}
\alpha a    &  b \\
\alpha c   & d
\end{bmatrix}  \right)
\end{equation}
where $\alpha \in \mathbb{R^+}$. \\

Another example:  immunity to the type-2 changes in the marginals \\
$\;\;$(AC5.2):
\begin{equation}\label{eq:AC5.2}
\text{SMH} \left(\begin{bmatrix}
a    & b \\
c   & d
\end{bmatrix}  \right) =\text{SMH} \left(\begin{bmatrix}
(1-\alpha) a    & (1-\alpha) b \\
c +  \alpha a  & d + \alpha b
\end{bmatrix}  \right) =
\text{SMH} \left(\begin{bmatrix}
(1-\alpha) a    &  b+\alpha a \\
(1-\alpha) c  & d + \alpha c
\end{bmatrix}  \right)
\end{equation}. \\
Under the type-2 changes in the marginals, $\alpha$ share of the L-type (wo)men are reclassified to H-type.\\
(AC6) \textit{Weak criterion on the indicator}: in case of Maximally Positive Assortative Matching (MPAM) and with no intermarriages, i.e., if everybody ``marries his or her own type'', the indicator takes its maximum value. MPAM is defined as the matching, where people in a certain education group can marry someone from a lower ranked education group then their own only if no one in the opposite sex remained unmatched in the  educational groups ranked higher. The weak criterion can be formulated as
$\text{SMH}(Q|Q \text{ is diagonal}) = max_{P \in \{P_1,... P_n \}}(\text{SMH}(P))$, where $\{P_1,... P_n \}$ is the set of all possible matches under given marginals.
This criterion implicitly assumes that the educational distribution of marriageable men is identical to the educational distribution of marriageable women, which is a very special case.\\
(AC7) \textit{Strong criterion on the indicator}:  in case of MPAM,  the indicator takes its maximum value irrespective of the number of intermarriages.
It can be formulated as\\
$\text{SMH}(Q|Q \text{ represents a MPAM}) = max_{P \in \{P_1,... P_n \}}(\text{SMH}(P))$, where $\{P_1,... P_n \}$ is the set of all possible matches with given marginals.\\
(AC8) \textit{Monotonicity in the diagonal cells}: $\text{SMH}(Q)\leq \text{SMH}(Q+A)$ for any  positive diagonal matrix $A$ of the same size as $Q$.\\
(AC9) \textit{Monotonicity in Inter-Generational Mobility} (henceforth IGM): for any $P$ and $Q$ marital contingency tables representing the joint  distributions of ascribed and attained traits of couples, if the attained traits are more dependent on the ascribed traits for both men and women in the population characterized by $P$ relative to the population characterized by $Q$, 
then  $\text{SMH}(Q)\leq \text{SMH}(P)$.\\
(AC10) \textit{Monotonicity in the number of voluntary singles}, where voluntary singles refer to single individuals to whom not even the most highly educated individuals with  opposite sex are acceptable as partners (see \citealp{NaszodiMendonca2019}). Using the notation in Table \ref{tab:CT}, we can formalize the related criterion as $\text{SMH}(Q)\leq \text{SMH}(Q+S)$ for any non-negative matrix $S$ of the form
$\begin{bmatrix}
0    & 0 & e_\text{VS} \\
0   & 0  & f_\text{VS} \\
g_\text{VS}   & h_\text{VS}     \\
\end{bmatrix} $
where  the non-zero elements denote the number of additional voluntary single people, who are either low educated men ($e_\text{VS}$), or   high educated men ($f_\text{VS}$), or low educated women ($g_\text{VS}$), or  high educated women ($h_\text{VS}$) ($e_\text{VS},f_\text{VS}, g_\text{VS}, h_\text{VS} \in \mathbb{N}$) .\\
(AC11) \textit{Immunity to the additional number of involuntary singles}, where additional involuntary singles  are those single individuals in a given generation who have the same marital preferences as those who managed to form couples in an earlier generation. The involuntary singles would also like to be matched and could have been in a couple provided the structural availability of potential partners and competitors had not changed (see \citealp{NaszodiMendonca2019}). Using the notation in Table \ref{tab:CT}, we can formalize the related criterion as
$\text{SMH}(Q) = \text{SMH}(Q+Z)$ for any  matrix $Z$ of the form
$\begin{bmatrix}
0    & 0 & e_\text{IS} \\
0   & 0  & f_\text{IS} \\
g_\text{IS}   & h_\text{IS}     \\
\end{bmatrix} $
where  $e_\text{IS}, f_\text{IS}, g_\text{IS}, h_\text{IS} \in \mathbb{N}$ denote the number of additional involuntary singles.\\
(AC12) \textit{Weak Robustness to the number of Educational Categories} (henceforth  weak REC). This criterion can be imposed against methods, such as M1-7 that construct a   counterfactual contingency table $Q_{M_i}^{\text{counterf.}}(Q,P)$. The criterion can be formalized as\\
$Q_{M_i}^{\text{counterf.}}\left(\text{Merge}(Q),\text{Merge}(P) \right)=\text{Merge}\left( Q_{M_i}^{\text{counterf.}}(Q,P)\right)$.
Under this criterion, the operation constructing the counterfactual table commutes with the operation $\text{Merge()}$ of merging neighboring educational categories.\\
(AC13) \textit{Strong Robustness to the number of Educational Categories} (henceforth strong REC). This criterion may be imposed against the indicators constructed by the methods meeting weak REC.
Under this criterion, the indirectly defined indicator is immune to the chosen number of educational categories, where the indicator
$\Delta\text{SMH}(Q,P,Q_{M_i}^{\text{counterf.}})$ captures the ceteris paribus contribution of the changing non-structural factor to the change in the share of homogamous couples. The criterion can be formalized as\\
$\Delta\text{SMH}(Q,P,Q_{M_i}^{\text{counterf.}})=\Delta\text{SMH} \left(\text{Merge}(Q),\text{Merge}(P),\text{Merge}(Q_{M_i}^{\text{counterf.}}) \right)$.


The criteria reviewed (AC1-13) are proposed by either of the following papers: 
\cite{Eika2019}, \cite{NaszodiMendonca2021}, \cite{Chiapporietal2021}, \cite{NaszodiMendonca2019}, \cite{Naszodi2022_2m}, \cite{Naszodi2023Grad},  \cite{Naszodi2023WP} and \cite{NaszodiMendonca2023_RACEDU}. 
As it is shown by Tables \ref{tab:crit_ind} and \ref{tab:crit_meth}, none of our 10 direct and 7 indirect measures of homophily satisfies all the 13 criteria.  

For instance, the odds-ratio (I1) satisfies AC5.1, but violates AC5.2.
By contrast, the  aggregate marital sorting parameter (I6) meets AC5.2, while violates AC5.1.  
This point illustrates that for testing ``immunity to changes in the marginals'', one has to be specific about the kind of change in the marginals to be controlled for. 
{Although it can be self-evident for some researchers that under ``immunity to changes in the marginals'' one should mean AC5.1, other researchers, such as \cite{Eika2019} define the concept differently. Otherwise, they would not have applied the aggregate marital sorting parameter (I6).}
	
Also, Tables \ref{tab:crit_ind} and \ref{tab:crit_meth}  show that researchers should lower their standards and instead of requiring the suitable measures of homophily to fulfill all the 13 criteria, they should impose only a restricted set of criteria.  
What criteria are selected to the restricted set of criteria determines what indicators are considered to be suitable measures.    
Since some criteria are mutually exclusive (e.g. AC5.1 and AC5.2), so are some indicators (e.g. I1 and I6). 

	

\newcommand{\STAB}[1]{\begin{tabular}{@{}c@{}}#1\end{tabular}}

%E:\structured2016\work\papers\Criternion_commute\paper
%Table2Latex

%\newcommand{\STAB}[1]{\begin{tabular}{@{}c@{}}#1\end{tabular}}																													

\begin{landscape}																																			
	\begin{table}[!ht]																																			
		\caption{Analytical criteria and homophily indicators}																																	
		\small 																																		
		\centering																																		
		\begin{tabular}{ l l l l l c  c c c c c c c c c  }																																		
			\hline \hline																																	
			~ 	&	 ~ 	&	 ~ 	&	 ~ 	&	 ~ 	&																	 \multicolumn{10}{c}{Statistical indicators} 		 \\ \cline{6-15} 						
			~ 	&	 ~ 	&	 ~ 	&	 ~ 	&	 ~ 			&			  \rotatebox{90}{Odds-ratio} 	&	 \rotatebox{90}{Matrix determinant} 	&	 \rotatebox{90}{Covariance coef.} 	&	 \rotatebox{90}{Correlation coef.} 	&	 \rotatebox{90}{Regression coef.}  	&	 \rotatebox{90}{Aggregate MSP} 	&	 \rotatebox{90}{V-value} 	&	 \rotatebox{90}{Marital surplus indicator} 	&	 \rotatebox{90}{LL-indicator} 	&	 \rotatebox{90}{Matrix-valued GLL-indicator} 	\\	
			\multicolumn{4}{c}{}   			&	 ~  	&			 (I1) 	&	 (I2) 	&	 (I3) 	&	 (I4) 	&	 (I5) 	&	 (I6) 	&	 (I7) 	&	 (I8) 	&	 (I9) 	&	 (I10) 	 \\ \cline{3-15} 	
			\multirow{12}{*}{\STAB{\rotatebox[origin=c]{90}{\underline{$\;\;\;\;\;\;$Analytical criteria$\;\;\;\;\;\;$}}}}  	&	 \multirow{4}{*}{\STAB{\rotatebox[origin=c]{90}{\underline{$\;\;$Basic$\;\;$}}}} 	&	 (AC1) 	&	 Cardinal 	&	 ~ 	&	Y	&	Y	&	Y	&	Y	&	Y	&	Y	&	N	&	Y	&	N	&	N	  \\	
			&	  	&	 (AC2) 	&	 \multicolumn{2}{l}{Scale invariance} 			&	Y	&	Y	&	Y	&	Y	&	Y	&	Y	&	Y	&	Y	&	Y	&	Y	  \\	
			&	  	&	 (AC3) 	&	 \multicolumn{2}{l}{Gender symmetry} 			&	Y	&	Y	&	Y	&	Y	&	N	&	Y	&	Y	&	Y	&	Y	&	Y	  \\	
			&	  	&	 (AC4) 	&	 \multicolumn{2}{l}{Category symmetry} 			&	Y	&	Y	&	Y	&	Y	&	Y	&	N	&	N	&	NA	&	Y	&	N***	  \\	
			&	 \multirow{5}{*}{\STAB{\rotatebox[origin=c]{90}{\underline{Standard}}}} 	&	 (AC5) 	&	 \multicolumn{2}{l}{Changes in marginal distr. is controlled for}   			&	 ~ 	&	 ~ 	&	 ~ 	&	 ~ 	&	 ~ 	&	 ~ 	&	 ~ 	&	 ~ 	&	 ~ 	&	 ~ 	  \\	
			&		&	$\;\;$ (AC5.1) 	&	\multicolumn{2}{l}{$\;\;$in a given way*} 			&	Y	&	N	&	N	&	N	&	N	&	N	&	N	&	N	&	N	&	N	  \\	
			&		&	$\;\;$ (AC5.2) 	&	\multicolumn{2}{l}{$\;\;$in another given way**}			&	N	&	N	&	N	&	N	&	N	&	Y	&	N	&	N	&	N	&	N	  \\	
			&	  	&	 (AC6) 	&	 Weak criterion  	&	 \multirow{2}{*}{MPAM}  	&	Y	&	Y	&	Y	&	Y	&	?	&	?	&	Y	&	?	&	Y	&	Y	  \\	
			&	  	&	 (AC7) 	&	 Strong criterion  	&	 ~ 	&	Y	&	N	&	N	&	N	&	?	&	?	&	N	&	?	&	Y	&	Y	  \\	
			&	 \multirow{4}{*}{\STAB{\rotatebox[origin=c]{90}{\underline{Advanced}}}} 	&	 (AC8) 	&	 \multirow{3}{*}{Monotonicity in}  	&	 diagonal cells 	&	Y	&	Y	&	Y	&	Y	&	Y	&	Y	&	Y	&	Y	&	Y	&	Y	 \\	
			&	  	&	 (AC9) 	&	 ~ 	&	 IGM 	&	N	&	?	&	?	&	?	&	?	&	?	&	?	&	?	&	Y	&	Y	  \\	
			&	  	&	 (AC10) 	&	 ~ 	&	number of VSs 	&	NA	&	NA	&	NA	&	NA	&	NA	&	NA	&	NA	&	N	&	NA	&	NA	 \\	
			&		&	 (AC11) 	&	\multicolumn{2}{l}{Immunity to the additional ISs}			&	NA	&	NA	&	NA	&	NA	&	NA	&	NA	&	NA	&	N	&	NA	&	NA	  \\   \hline  \hline	
					\end{tabular} \\																																		
		\textit{Notes}: Y abbreviates yes, N abbreviates no, NA stands for not applicable. *  see Eq.\ref{eq:AC5.1}, **  see Eq.\ref{eq:AC5.2}. ***: if more than 2 categories are considered, then the LL is applicable only for ordinal assorted trait with ordered categories.    																																		
		\label{tab:crit_ind}																																		
	\end{table}																																			
\end{landscape}																																			

Similarly to the indicators, none of the methods fulfill all the 13 criteria (see Table \ref{tab:crit_meth}).
This is not surprising, because each of the first 6 methods rely on one the 10 indicators.


However, the homophily measures obtained by the indirect identification approaches (M1-7)  perform somewhat  better than the  direct indicators in general according to the analytical criteria (see Table \ref{tab:crit_meth}).   
It is worth to remark that some indirect indicators  under-perform, especially relative to their popularity.  
For instance, one may have the prior view that the commonly applied IPF meets all of the relevant analytical criteria,  
		otherwise it would not be used by many scholars. 


		
What can update this prior view?\\ 
First, a close look at Table \ref{tab:crit_meth} showing that  the IPF  violates the criterion   
AC12 of weak REC. 
This point is illustrated with a numerical example by \cite{Naszodi2023WP}. 
Moreover, \cite{Naszodi2022_2m} offers a number of additional theoretical reasons for why the IPF is not suitable for constructing counterfactuals. 


Second, what may also question the prior view is the recognition that researchers do not chose their methods independently of each other.   
For instance, the fact that the IPF is the only readily available method in SPSS harmonizes the choices of SPSS users.  

Third, watching the TEDx Talk of Paul Rulkens, where he explains "Why the majority is always wrong".\footnote{See \url{https://www.youtube.com/watch?v=VNGFep6rncY}.} 

	
		
What may potentially confirm the prior view?\\  	
First, if AC12 of  weak REC would turn out not to be an important criterion, one could consider the IPF as a suitable method.  
However, the importance of this criterion is commonly recognized: researchers would like to avoid 
their measure of homophily to be sensitive to the number of educational categories chosen due to the poor analytical property of the indicator applied.  
We emphasize that this kind of sensitivity is not an empirical phenomenon: irrespective of the data analyzed, the IPF may fail to quantify  homophily robust to the categories chosen.   
 
Second, the authority of the inventors of the IPF may also confirm the prior view. 
However, \cite{StephanDeming1940}  warned that  their  algorithm is ``not by itself useful for prediction'' (see  p.444).  
So, the authority of Stephan Deming rather challenges the prior view, rather than conforms it.   
Unfortunately, their words were disregarded for several decades.  
	
			
Now, let us turn to criterion AC13 of strong REC. 		
We emphasize that unlike AC12 of weak REC, AC13 is empirical. 
This criterion may be violated by each of the methods considered (see Table \ref{tab:crit_meth}).  
\cite{Naszodi2023WP} argues that it would be overambitious to impose this criterion against any indicator or method:   
it cannot be met even by the methods satisfying the criterion AC12 of weak REC.  
			
In particular, she illustrates with an example that the quantified change in homophily will never be independent of the number of educational categories chosen: ''We warn that the number of categories should be chosen carefully even if the method used for constructing the counterfactuals commutes with the operation of merging neighboring categories. Imagine that we analyze matching along age (or any other trait described by a continuous variable). A couple is considered as being homogamous if their age difference is below a certain threshold. Provided the number of age categories is chosen to be extremely high with a threshold as low as one minute, the share of homogamous couples is close to zero. In addition, contrary to common sense, the share of homogamous couples is practically unchanged across any pair of consecutive generations under such an extremely granular set of age categories.'' 
		

 
\begin{landscape}																																			
	\begin{table}[!ht]																																			
		\caption{Analytical criteria and  measuring  changes in homophily indirectly with counterfactuals constructed by various methods}																																	
		\small 																																		
		\centering																																		
		\begin{tabular}{ l l l l l c  c c c c c c  }																																		
			\hline \hline																																	
			~ 	&	 ~ 	&	 ~ 	&	 ~ 	&	 ~ 	&																	 \multicolumn{7}{c}{Fixed point}		 \\ 						
			~ 	&	 ~ 	&	 ~ 	&	 ~ 	&	 ~ 	&																	 \multicolumn{7}{c}{transformation-based methods}  		 \\ \cline{6-12} 						
			~ 	&	 ~ 	&	 ~ 	&	 ~ 	&	 ~ 			&			 \rotatebox{90}{Odds-ratio based IPF} 	&	 \rotatebox{90}{MDbM} 	&	 \rotatebox{90}{MEDA} 	&	 \rotatebox{90}{CS-model based approach} 	&	 \rotatebox{90}{GLL-indicator based NM} 	&	 \rotatebox{90}{GLL-indicator based GNM} 	&	 \rotatebox{90}{GS-matching}\\ 								
			\multicolumn{4}{c}{}   			&	 ~  	&			 (M1) 	&	 (M2) 	&	 (M3) 	&	 (M4)   	&	 (M5)   	&	 (M6) 	&	 (M7) 	 \\ \cline{3-12} 							
			\multirow{12}{*}{\STAB{\rotatebox[origin=c]{90}{\underline{$\;\;\;\;\;\;$Analytical criteria$\;\;\;\;\;\;$}}}}  	&	 \multirow{4}{*}{\STAB{\rotatebox[origin=c]{90}{\underline{$\;\;$Basic$\;\;$}}}} 	&	 (AC1) 	&	 Cardinal 	&	 ~ 	&	Y	&	Y	&	Y	&	Y	&	Y	&	Y	&	Y	 \\							
			&	  	&	 (AC2) 	&	 \multicolumn{2}{l}{Scale invariance} 			&	Y	&	Y	&	Y	&	Y	&	Y	&	Y	&	Y	 \\							
			&	  	&	 (AC3) 	&	 \multicolumn{2}{l}{Gender symmetry} 			&	Y	&	Y	&	Y	&	Y	&	Y	&	Y	&	Y	 \\							
			&	  	&	 (AC4) 	&	 \multicolumn{2}{l}{Category symmetry} 			&	Y	&	Y	&	Y	&	Y	&	N***	&	N***	&	N***	 \\							
			&	 \multirow{3}{*}{\STAB{\rotatebox[origin=c]{90}{\underline{Standard}}}} 	&	 (AC5) 	&	 \multicolumn{2}{l}{Changes in marginal distr. is controlled for*}   			&	Y	&	Y	&	Y	&	Y	&	Y	&	Y	&	Y	 \\							
			&	  	&	 (AC6) 	&	 Weak criterion  	&	 \multirow{2}{*}{MPAM}  	&	?	&	?	&	?	&	?	&	Y	&	Y	&	Y	 \\							
			&	  	&	 (AC7) 	&	 Strong criterion  	&	 ~ 	&	?	&	?	&	?	&	?	&	Y	&	Y	&	Y	 \\							
			&	 \multirow{6}{*}{\STAB{\rotatebox[origin=c]{90}{\underline{Advanced}}}} 	&	 (AC8) 	&	 \multirow{3}{*}{Monotonicity in}  	&	 diagonal cells 	&	Y	&	Y	&	Y	&	Y	&	Y	&	Y	&	Y	 \\							
			&	  	&	 (AC9) 	&	 ~ 	&	 IGM 	&	N	&	NA	&	NA	&	NA	&	Y	&	Y	&	Y	 \\							
			&	  	&	 (AC10) 	&	 ~ 	&	number of VSs 	&	NA	&	NA	&	NA	&	N	&	NA	&	NA	&	Y	 \\							
			&		&	 (AC11) 	&	\multicolumn{2}{l}{Immunity to the additional ISs}			&	NA	&	NA	&	NA	&	N	&	NA	&	NA	&	Y	 \\							
			&	  	&	 (AC12) 	&	 Weak criterion  	&	 \multirow{2}{*}{REC}  	&	N	&	N	&	N	&	N	&	Y	&	Y	&	?	 \\							
			&	  	&	 (AC13) 	&	 Strong criterion 	&	 ~ 	&	N	&	N	&	N	&	N	&	N	&	N	&	N	 \\  \hline   \hline							
		\end{tabular} \\																																		
		\textit{Notes}: Y abbreviates yes, N abbreviates no, NA stands for not applicable. *  each method is designed to controll for changes in the marginals, but the methods do it differently. *** these methods are applicable for ordered categories of the trait variable.  																																		
		\label{tab:crit_meth}																																		
	\end{table}																																			
\end{landscape}																																			

In principle, we could elaborate on what criterion to impose against the set of restricted analytical criteria that selects the suitable measures of homophily. 
Instead, we present an empirical criterion in the next section.  



\subsection{The empirical criterion}\label{sec:Ecrit}


%E:\structured2016\work\papers\US_50States\cross_states_Comparison_Health_marriage_2023\data\marriage
%Description_US_51_trends_College_noC_VB
%Description_US_51_trends_HighS_NoH

The  \textit{empirical criterion}  we apply against the suitable indicators/methods is that the indicators should identify the marital homophily to have had a U-shaped trend  in the US after 1960. 
More precisely, the first four generations born during and after the Great Depression should be found to have had preferences over spousal education that were less and less homophilic relative to the previous generations, while the sixth generation should be found to have had more homophilic preferences than the fifth generation had.

The first four generations are the early Silent generation (whose members where most active on the marriage market around 1960), the late Silent generation (whose members gradually replaced the early Silent generation on the marriage market by 1970), the  early Boomers (whose members arrived the market around 1980),  the late Boomers (whose members where most active on the market around 1990). 
Whereas the fifth generation is the early GenX, whose members were gradually replaced by the sixth generation, the late Genxers on the market by 2010. 

We test whether the indicators (I1-10) and methods (M1-7) pass our set of empirical criteria  on the direction of change in homophily from one generation to the consecutive generation by using American young couples' educational data. 
We use linked census data on couples education 
from IPUMS both for the US and for each States. 

In our benchmark analysis, we  considered a couple to be young if the male partner is  between 30 and 34 years old.  
Thereby, our data from the census years 1960, 1970, 1980, 1990, 2000, 2010 (, and 2015) cover husbands and heterosexual male partners from the early Silent generation, the late Silent generation, the early Boomer generation, the late Boomer generation, the early GenX, the late Genx (,and the early Millennial generation), respectively. The wives/ female partners may not necessarily be from the same generation as the observed men, however, they typically are. 


\subsubsection{Motivation for the empirical criterion}\label{sec:Mot_Ecrit}


We emphasize that it is not only the U-shaped pattern of income inequality that makes us impose our empirical criterion, but also 
two additional patterns. First, the survey evidence analyzed  by  \cite{NaszodiMendonca2021} and \cite{Naszodi2023WP} also corroborates 
the U-shaped pattern of homophily. 

Second, the evolution of the Social Security System in the US is also consistent with the U-pattern. 
 The original Social Security Act of 1935 contained no provisions for the payment of any type of dependents' benefits. 
 However, in 1939, at the start of the program, it  was transformed into a family-benefits social insurance system.   
Originally, the family-benefits social insurance has provided benefits only to the children under 18 of
the deceased parents. 
In 1956 it was extended to disability benefits. 
So, unlike the  members of the early Silent generation, the  members of the late Silent generation could already benefit from the extended scheme as children. 

Between 1965 and 1982, payments were extended to age 22 if the child remained enrolled full time in school (see \citealp{Dynarski2003}).
So, some members of the early Boomers in our data could  benefit from the second most significant extension of the programme, while the members of the earlier generations could not. 

Between 1965 and 1977, more and more college student with deceased, disabled, or retired parents were supported by the program (see: \url{https://www.ssa.gov/history/studentbenefit.html}).  
The winner generation of the run up  was the generation of late Boomers: the program extended to college students peaked at 700,000 students in 1977.  
Especially that generation benefited the most, who we observe as [30,34] years old in the 1990 census: they were senior high school students between 1974 and 1978.


Finally, those who were [30,34] in 2000, i.e., the early GenXers in our data, were already effected by the gradual elimination of the program in 1981, 
since they turned  18 between 1984 and 1988. 
By the 1984--1985 academic year, program spending had dropped by 3 billion dollars and college enrollment has also reduced (see \citealp{Dynarski2003}). 

 
\subsubsection{Empirical analysis}\label{sec:EMP_FIND}

TO BE WRITTEN






\subsection{An additional criterion}\label{sec:EAcrit}

TO BE REWRITTEN


Finally, we introduce an additional criterion against the methods  (M1-7)  that is partly empirical and partly analytical in nature:\\   
(EAC1) the method should signal if it is not able  to control for the changes in the educational distributions without constructing an impossible counterfactual.  

\cite{NaszodiMendonca2021} present two  examples for impossible counterfactuals. First,  ``the one under which Europeans already know how to make popcorn without having discovered the North American continent. Researchers might need to construct such a counterfactual if they aim at disentangling the effects of inventions and discoveries on growth in Europe over the history of mankind while being restricted to use observations only from the endpoints of the time period.'' 

Their second example is  about the counterfactual population of young couples, whose marital preferences are the same as that of the Portuguese young couples were in 2011, while the educational distributions of men and women are the same as that of the Portuguese young men and women were in 1981. The general level of education of Portuguese young people has been risen so remarkably between 1981 and 2011 that preferences over spousal education could not remain unchanged. According to  the analysis of \cite{NaszodiMendonca2021}, the NM-method constructs a counterfactual contingency table that contains a
negative element, while its alternatives, e.g., the counterfactual tables constructed by the IPF algorithm and the CS-model based method are always non-negative. 
So, only the NM-method could signal in case of Portugal that the counterfactual to be constructed can not obtain. 

      	
\section{The homophily measures applied in the literature and the identified historical trends}\label{sec:ind_trend}

TO BE WRITTEN

\subsection{A remark on majority vote}\label{sec:majority}

Sections \ref{sec:ind_trend} and \ref{sec:Acrit} have shown that  papers disagree on the trend of homophily since they can agree neither on the suitable indicators, nor on the 
set of criteria imposed against the suitable indicators.  

In case of such a disagreement, one may propose to check if researchers agree on how to choose the criterion against the criterion to choose the set of criteria. 
However, one does not have to be a philosopher familiar with epistemology to recognize that 
any criterion can be endlessly questioned.

Alternatively, one may propose a ``majority vote''. 
Here, we argue that the ``majority vote''  does not meet the standards of modern scientific debates.
Even if a large set of indicators suggests that the historical trend of inequality is of a given shape, 
the size of this set cannot be used as a final argument in favor of  the given shape.\footnote{In other words, typical robustness analyzes to the choice of the indicator might  not qualify as a criterion on the set of criteria.} 
The reason is that  the larger set of indicators can be from the same family of indicators (see also \citealp{NaszodiMendonca2021} on this point). 
If the entire family is ill suited  for the analysis then each of the indicators in the family is ill suited irrespective of the size of the family.   

Consider an example about the distribution of extreme values of financial returns. 
The most impressive theorem in extreme value theory, the Fisher--Tippett--Gnedenko theorem states that 
statistical distribution of the largest value drawn from a sample of a given size has only three possible shapes: 
it is either a Weibull, a Fr�chet or a Gumbel extreme value distributions depending on the distribution of the population where the sample is taken. 
In this sense, any distribution belongs either to the Weibull-family of distributions, or the Fr�chet-family or the Gumbel-family. 

Many well-known distributions  (e.g. the most popular Gaussian distribution) commonly assumed to characterize returns predict a low value at risk of a financial portfolio 
 as they belong to the Gumbel-family.  
Whereas  distributions in the Fr�chet-family  (e.g.  the Cauchy distribution) predict large value at risk.   
Even if we can name much more distributions in the Gumbel-family  than in the Fr�chet-family,  
it would be misleading to give more credit to the moderate value at risk. 




\section{Conclusion}\label{sec:concl} 


Following the non-argumentative approach of the animals in Orwell's  Animal farm, we could say that 
U-shape is good, non-U-shape is bad. 
Similarly, we could say that those criteria fit for the purpose of selecting suitable indicators  
 that are violated by the indicators resulting in non-U-shaped trend,   
while those criteria are not fit for the same purpose  that are violated by the indicators resulting in U-shaped trend.  

In this paper, we applied a more sophisticated approach:  
our argument in favor of the LL-indicator and the NM-method relied on (i) an analytical approach of selecting indicators, (ii)  a  literature review, and (iii) the brief introduction of the historical evolution of the American Social Security System. 
 
 

 

	\bibliography{Naszodi_criterium_2023Apr29}
\end{document}

