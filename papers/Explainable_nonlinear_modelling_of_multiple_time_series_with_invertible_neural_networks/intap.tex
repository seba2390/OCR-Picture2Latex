\documentclass{llncs}

\usepackage{graphicx} % figures

\usepackage{amsmath} % math
\usepackage{amsfonts}
\usepackage{amssymb}
\usepackage{accents}
\newcommand{\ubar}[1]{\underaccent{\bar}{#1}}

\usepackage[pdftex,dvipsnames]{xcolor}  % Coloured text etc.

\usepackage{listings} % code listings
\lstset{frame=tb,
  language=Python,
  aboveskip=3mm,
  belowskip=3mm,
  showstringspaces=false,
  columns=flexible,
  basicstyle={\small\ttfamily},
  numbers=none,
  numberstyle=\tiny\color{gray},
  keywordstyle=\color{blue},
  commentstyle=\color{dkgreen},
  stringstyle=\color{mauve},
  breaklines=true,
  breakatwhitespace=true,
  tabsize=3
}

% Marking equal contribution.
\makeatletter
\newcommand{\printfnsymbol}[1]{%
  \textsuperscript{\@fnsymbol{#1}}%
}
\makeatother
%%%%%%%%%%%%%%%%%%%%%%%%%

\begin{document}

\title{Explainable nonlinear modelling of multiple time series with invertible neural networks\thanks{
The work in this paper was supported by the SFI Offshore Mechatronics
grant 237896/O30.
}
}
\titlerunning{Explainable VAR modeling with invertible NNs}

\author{Luis Miguel Lopez-Ramos\thanks{Equal contribution in terms of working hours.}%\inst{1,2}
\orcidID{0000-0001-8072-3994} \and
\\ 
Kevin Roy\printfnsymbol{2}%\inst{1,2}
%\orcidID{1111-2222-3333-4444} 
\and
Baltasar Beferull-Lozano%\inst{1,2}
\orcidID{0000-0002-0902-6245}
}

\authorrunning{Lopez-Ramos & Roy et al.}

\institute{SFI Offshore Mechatronics Center, University of Agder
\and
Intelligent Signal Processing and Wireless Networks (WISENET) Center
\and
Department of ICT, University of Agder, Grimstad, Norway
}
\maketitle

\begin{abstract}
    A method for nonlinear topology identification is proposed, based on the assumption that a collection of time series are generated in two steps: i) a vector autoregressive process in a latent space, and ii) a nonlinear, component-wise, monotonically increasing observation mapping. The latter mappings are assumed invertible, and are modeled as shallow neural networks, so that their inverse can be numerically evaluated, and their parameters can be learned using a technique inspired in deep learning. Due to the function inversion, the backpropagation step is not straightforward, and this paper explains the steps needed to calculate the gradients applying implicit differentiation. Whereas the model explainability is the same as that for linear VAR processes, preliminary numerical tests show that the prediction error becomes smaller.
\end{abstract}

\keywords{Vector autoregressive model
\and nonlinear
\and network topology inference
\and invertible neural network
}

\IEEEraisesectionheading{\section{Introduction}}

\IEEEPARstart{V}{ision} system is studied in orthogonal disciplines spanning from neurophysiology and psychophysics to computer science all with uniform objective: understand the vision system and develop it into an integrated theory of vision. In general, vision or visual perception is the ability of information acquisition from environment, and it's interpretation. According to Gestalt theory, visual elements are perceived as patterns of wholes rather than the sum of constituent parts~\cite{koffka2013principles}. The Gestalt theory through \textit{emergence}, \textit{invariance}, \textit{multistability}, and \textit{reification} properties (aka Gestalt principles), describes how vision recognizes an object as a \textit{whole} from constituent parts. There is an increasing interested to model the cognitive aptitude of visual perception; however, the process is challenging. In the following, a challenge (as an example) per object and motion perception is discussed. 



\subsection{Why do things look as they do?}
In addition to Gestalt principles, an object is characterized with its spatial parameters and material properties. Despite of the novel approaches proposed for material recognition (e.g.,~\cite{sharan2013recognizing}), objects tend to get the attention. Leveraging on an object's spatial properties, material, illumination, and background; the mapping from real world 3D patterns (distal stimulus) to 2D patterns onto retina (proximal stimulus) is many-to-one non-uniquely-invertible mapping~\cite{dicarlo2007untangling,horn1986robot}. There have been novel biology-driven studies for constructing computational models to emulate anatomy and physiology of the brain for real world object recognition (e.g.,~\cite{lowe2004distinctive,serre2007robust,zhang2006svm}), and some studies lead to impressive accuracy. For instance, testing such computational models on gold standard controlled shape sets such as Caltech101 and Caltech256, some methods resulted $<$60\% true-positives~\cite{zhang2006svm,lazebnik2006beyond,mutch2006multiclass,wang2006using}. However, Pinto et al.~\cite{pinto2008real} raised a caution against the pervasiveness of such shape sets by highlighting the unsystematic variations in objects features such as spatial aspects, both between and within object categories. For instance, using a V1-like model (a neuroscientist's null model) with two categories of systematically variant objects, a rapid derogate of performance to 50\% (chance level) is observed~\cite{zhang2006svm}. This observation accentuates the challenges that the infinite number of 2D shapes casted on retina from 3D objects introduces to object recognition. 

Material recognition of an object requires in-depth features to be determined. A mineralogist may describe the luster (i.e., optical quality of the surface) with a vocabulary like greasy, pearly, vitreous, resinous or submetallic; he may describe rocks and minerals with their typical forms such as acicular, dendritic, porous, nodular, or oolitic. We perceive materials from early age even though many of us lack such a rich visual vocabulary as formalized as the mineralogists~\cite{adelson2001seeing}. However, methodizing material perception can be far from trivial. For instance, consider a chrome sphere with every pixel having a correspondence in the environment; hence, the material of the sphere is hidden and shall be inferred implicitly~\cite{shafer2000color,adelson2001seeing}. Therefore, considering object material, object recognition requires surface reflectance, various light sources, and observer's point-of-view to be taken into consideration.


\subsection{What went where?}
Motion is an important aspect in interpreting the interaction with subjects, making the visual perception of movement a critical cognitive ability that helps us with complex tasks such as discriminating moving objects from background, or depth perception by motion parallax. Cognitive susceptibility enables the inference of 2D/3D motion from a sequence of 2D shapes (e.g., movies~\cite{niyogi1994analyzing,little1998recognizing,hayfron2003automatic}), or from a single image frame (e.g., the pose of an athlete runner~\cite{wang2013learning,ramanan2006learning}). However, its challenging to model the susceptibility because of many-to-one relation between distal and proximal stimulus, which makes the local measurements of proximal stimulus inadequate to reason the proper global interpretation. One of the various challenges is called \textit{motion correspondence problem}~\cite{attneave1974apparent,ullman1979interpretation,ramachandran1986perception,dawson1991and}, which refers to recognition of any individual component of proximal stimulus in frame-1 and another component in frame-2 as constituting different glimpses of the same moving component. If one-to-one mapping is intended, $n!$ correspondence matches between $n$ components of two frames exist, which is increased to $2^n$  for one-to-any mappings. To address the challenge, Ullman~\cite{ullman1979interpretation} proposed a method based on nearest neighbor principle, and Dawson~\cite{dawson1991and} introduced an auto associative network model. Dawson's network model~\cite{dawson1991and} iteratively modifies the activation pattern of local measurements to achieve a stable global interpretation. In general, his model applies three constraints as it follows
\begin{inlinelist}
	\item \textit{nearest neighbor principle} (shorter motion correspondence matches are assigned lower costs)
	\item \textit{relative velocity principle} (differences between two motion correspondence matches)
	\item \textit{element integrity principle} (physical coherence of surfaces)
\end{inlinelist}.
According to experimental evaluations (e.g.,~\cite{ullman1979interpretation,ramachandran1986perception,cutting1982minimum}), these three constraints are the aspects of how human visual system solves the motion correspondence problem. Eom et al.~\cite{eom2012heuristic} tackled the motion correspondence problem by considering the relative velocity and the element integrity principles. They studied one-to-any mapping between elements of corresponding fuzzy clusters of two consecutive frames. They have obtained a ranked list of all possible mappings by performing a state-space search. 



\subsection{How a stimuli is recognized in the environment?}

Human subjects are often able to recognize a 3D object from its 2D projections in different orientations~\cite{bartoshuk1960mental}. A common hypothesis for this \textit{spatial ability} is that, an object is represented in memory in its canonical orientation, and a \textit{mental rotation} transformation is applied on the input image, and the transformed image is compared with the object in its canonical orientation~\cite{bartoshuk1960mental}. The time to determine whether two projections portray the same 3D object
\begin{inlinelist}
	\item increase linearly with respect to the angular disparity~\cite{bartoshuk1960mental,cooperau1973time,cooper1976demonstration}
	\item is independent from the complexity of the 3D object~\cite{cooper1973chronometric}
\end{inlinelist}.
Shepard and Metzler~\cite{shepard1971mental} interpreted this finding as it follows: \textit{human subjects mentally rotate one portray at a constant speed until it is aligned with the other portray.}



\subsection{State of the Art}

The linear mapping transformation determination between two objects is generalized as determining optimal linear transformation matrix for a set of observed vectors, which is first proposed by Grace Wahba in 1965~\cite{wahba1965least} as it follows. 
\textit{Given two sets of $n$ points $\{v_1, v_2, \dots v_n\}$, and $\{v_1^*, v_2^* \dots v_n^*\}$, where $n \geq 2$, find the rotation matrix $M$ (i.e., the orthogonal matrix with determinant +1) which brings the first set into the best least squares coincidence with the second. That is, find $M$ matrix which minimizes}
\begin{equation}
	\sum_{j=1}^{n} \vert v_j^* - Mv_j \vert^2
\end{equation}

Multiple solutions for the \textit{Wahba's problem} have been published, such as Paul Davenport's q-method. Some notable algorithms after Davenport's q-method were published; of that QUaternion ESTimator (QU\-EST)~\cite{shuster2012three}, Fast Optimal Attitude Matrix \-(FOAM)~\cite{markley1993attitude} and Slower Optimal Matrix Algorithm (SOMA)~\cite{markley1993attitude}, and singular value decomposition (SVD) based algorithms, such as Markley’s SVD-based method~\cite{markley1988attitude}. 

In statistical shape analysis, the linear mapping transformation determination challenge is studied as Procrustes problem. Procrustes analysis finds a transformation matrix that maps two input shapes closest possible on each other. Solutions for Procrustes problem are reviewed in~\cite{gower2004procrustes,viklands2006algorithms}. For orthogonal Procrustes problem, Wolfgang Kabsch proposed a SVD-based method~\cite{kabsch1976solution} by minimizing the root mean squared deviation of two input sets when the determinant of rotation matrix is $1$. In addition to Kabsch’s partial Procrustes superimposition (covers translation and rotation), other full Procrustes superimpositions (covers translation, uniform scaling, rotation/reflection) have been proposed~\cite{gower2004procrustes,viklands2006algorithms}. The determination of optimal linear mapping transformation matrix using different approaches of Procrustes analysis has wide range of applications, spanning from forging human hand mimics in anthropomorphic robotic hand~\cite{xu2012design}, to the assessment of two-dimensional perimeter spread models such as fire~\cite{duff2012procrustes}, and the analysis of MRI scans in brain morphology studies~\cite{martin2013correlation}.

\subsection{Our Contribution}

The present study methodizes the aforementioned mentioned cognitive susceptibilities into a cognitive-driven linear mapping transformation determination algorithm. The method leverages on mental rotation cognitive stages~\cite{johnson1990speed} which are defined as it follows
\begin{inlinelist}
	\item a mental image of the object is created
	\item object is mentally rotated until a comparison is made
	\item objects are assessed whether they are the same
	\item the decision is reported
\end{inlinelist}.
Accordingly, the proposed method creates hierarchical abstractions of shapes~\cite{greene2009briefest} with increasing level of details~\cite{konkle2010scene}. The abstractions are presented in a vector space. A graph of linear transformations is created by circular-shift permutations (i.e., rotation superimposition) of vectors. The graph is then hierarchically traversed for closest mapping linear transformation determination. 

Despite of numerous novel algorithms to calculate linear mapping transformation, such as those proposed for Procrustes analysis, the novelty of the presented method is being a cognitive-driven approach. This method augments promising discoveries on motion/object perception into a linear mapping transformation determination algorithm.



\section{Background and Motivation}

\subsection{IBM Streams}

IBM Streams is a general-purpose, distributed stream processing system. It
allows users to develop, deploy and manage long-running streaming applications
which require high-throughput and low-latency online processing.

The IBM Streams platform grew out of the research work on the Stream Processing
Core~\cite{spc-2006}.  While the platform has changed significantly since then,
that work established the general architecture that Streams still follows today:
job, resource and graph topology management in centralized services; processing
elements (PEs) which contain user code, distributed across all hosts,
communicating over typed input and output ports; brokers publish-subscribe
communication between jobs; and host controllers on each host which
launch PEs on behalf of the platform.

The modern Streams platform approaches general-purpose cluster management, as
shown in Figure~\ref{fig:streams_v4_v6}. The responsibilities of the platform
services include all job and PE life cycle management; domain name resolution
between the PEs; all metrics collection and reporting; host and resource
management; authentication and authorization; and all log collection. The
platform relies on ZooKeeper~\cite{zookeeper} for consistent, durable metadata
storage which it uses for fault tolerance.

Developers write Streams applications in SPL~\cite{spl-2017} which is a
programming language that presents streams, operators and tuples as
abstractions. Operators continuously consume and produce tuples over streams.
SPL allows programmers to write custom logic in their operators, and to invoke
operators from existing toolkits. Compiled SPL applications become archives that
contain: shared libraries for the operators; graph topology metadata which tells
both the platform and the SPL runtime how to connect those operators; and
external dependencies. At runtime, PEs contain one or more operators. Operators
inside of the same PE communicate through function calls or queues. Operators
that run in different PEs communicate over TCP connections that the PEs
establish at startup. PEs learn what operators they contain, and how to connect
to operators in other PEs, at startup from the graph topology metadata provided
by the platform.

We use ``legacy Streams'' to refer to the IBM Streams version 4 family. The
version 5 family is for Kubernetes, but is not cloud native. It uses the
lift-and-shift approach and creates a platform-within-a-platform: it deploys a
containerized version of the legacy Streams platform within Kubernetes.

\subsection{Kubernetes}

Borg~\cite{borg-2015} is a cluster management platform used internally at Google
to schedule, maintain and monitor the applications their internal infrastructure
and external applications depend on. Kubernetes~\cite{kube} is the open-source
successor to Borg that is an industry standard cloud orchestration platform.

From a user's perspective, Kubernetes abstracts running a distributed
application on a cluster of machines. Users package their applications into
containers and deploy those containers to Kubernetes, which runs those
containers in \emph{pods}. Kubernetes handles all life cycle management of pods,
including scheduling, restarting and migration in case of failures.

Internally, Kubernetes tracks all entities as \emph{objects}~\cite{kubeobjects}.
All objects have a name and a specification that describes its desired state.
Kubernetes stores objects in etcd~\cite{etcd}, making them persistent,
highly-available and reliably accessible across the cluster. Objects are exposed
to users through \emph{resources}. All resources can have
\emph{controllers}~\cite{kubecontrollers}, which react to changes in resources.
For example, when a user changes the number of replicas in a
\code{ReplicaSet}, it is the \code{ReplicaSet} controller which makes sure the
desired number of pods are running. Users can extend Kubernetes through
\emph{custom resource definitions} (CRDs)~\cite{kubecrd}. CRDs can contain
arbitrary content, and controllers for a CRD can take any kind of action.

Architecturally, a Kubernetes cluster consists of nodes. Each node runs a
\emph{kubelet} which receives pod creation requests and makes sure that the
requisite containers are running on that node. Nodes also run a
\emph{kube-proxy} which maintains the network rules for that node on behalf of
the pods. The \emph{kube-api-server} is the central point of contact: it
receives API requests, stores objects in etcd, asks the scheduler to schedule
pods, and talks to the kubelets and kube-proxies on each node. Finally,
\emph{namespaces} logically partition the cluster. Objects which should not know
about each other live in separate namespaces, which allows them to share the
same physical infrastructure without interference.

\subsection{Motivation}
\label{sec:motivation}

Systems like Kubernetes are commonly called ``container orchestration''
platforms. We find that characterization reductive to the point of being
misleading; no one would describe operating systems as ``binary executable
orchestration.'' We adopt the idea from Verma et al.~\cite{borg-2015} that
systems like Kubernetes are ``the kernel of a distributed system.'' Through CRDs
and their controllers, Kubernetes provides state-as-a-service in a distributed
system. Architectures like the one we propose are the result of taking that view 
seriously.

The Streams legacy platform has obvious parallels to the Kubernetes
architecture, and that is not a coincidence: they solve similar problems.
Both are designed to abstract running arbitrary user-code across a distributed
system.  We suspect that Streams is not unique, and that there are many
non-trivial platforms which have to provide similar levels of cluster
management.  The benefits to being cloud native and offloading the platform
to an existing cloud management system are: 
\begin{itemize}
    \item Significantly less platform code.
    \item Better scheduling and resource management, as all services on the cluster are 
        scheduled by one platform.
    \item Easier service integration.
    \item Standardized management, logging and metrics.
\end{itemize}
The rest of this paper presents the design of replacing the legacy Streams 
platform with Kubernetes itself.


\section{Application-Level Resilience \\ Modeling}
\label{sec:modeling}
This section describes our modeling methodology. %in details. 
We start with a classification of the application-level fault masking,
and then introduce a metric and investigate how to use it
to quantify the application resilience. %based on the classification. 

\subsection{General Description}
\label{sec:general_bg}
%Application-level fault masking can be manifested in different representations. 
Application-level fault masking has various representations.  
Listing~\ref{fig:general_desc} gives an example to illustrate the application-level fault masking. 
In this example, we focus on a data object, $par\_A$, which is a sparse matrix with 1$K$ non-zero data elements. We study \textit{fault masking happened in this data object}. $par\_A$ is involved in 4 statements (Lines 6, 7, 9 and 13). 
%To make this example easy to describe, we assume that the fault propagated to the application 
%is a single bit-flip in the least significant bit of . %of mantissa,
%but the application-level fault masking can happen to various faults. 
%the other faults can be tolerated by the application-level fault masking as well. 


\begin{comment}
\begin{figure}
	\begin{center}
		%\includegraphics[height=0.4\textheight,keepaspectratio]{general_desc.PNG} 
		\includegraphics[width=0.35\textheight,keepaspectratio]{general_desc.pdf} 
		\vspace{-8pt}
		\caption{An example code to show application-level fault masking}
		\label{fig:general_desc}
		\vspace{-20pt}
	\end{center}
\end{figure}
\end{comment}


In this example, the statement at Line 6 has a fault masking event:
% a data -> the data by anzheng
if a fault happened at a data element $par\_A[0].data$ of the target data object ($par\_A$), the fault can be overwritten by an assignment operation.
%The second statement has one fault masking event: the value of $c$ is determined by $b$ (not $2*a[2]$), because $b$ is significant bigger than $a[2]$.
The statement at Line 7 has no explicit fault masking event happened in the target data
%a->the by anzheng
object, but if a fault at a data element $par\_A[2].data$ occurs, the fault is propagated to $c$ by multiplication and assignment operations.
%and then indirectly masked at the statement of Line 8 by an addition operation.
At Line 9, assuming that the value of $c$ is much smaller than the value of a variable $GIANT$, 
%the operation result is determined by a variable $GIANT$ whose value is much bigger than $c$, 
the impact of the corrupted $c$ on the application outcome is ignorable.
%tolerating the fault in the second operation. 
Hence, the fault propagated from Line 7 to Line 9 can be indirectly masked.

\begin{lstlisting}[label={fig:general_desc}, caption={An example code to show application-level fault masking}]
void func (Matrix *par_A, Vector *par_b, Vector *par_x) {
	// the data object par_A has 1K data elements;
    float c=0.0;
    
    // pre-processing par_A
    par_A[0].data=sqrt(initInfo);
    c=par_A[2].data*2;
    if (c>THR) {
    	par_A[4].data=c+GIANT; // GIANT >> c
    }
    
    // using the algebraic multi-grid solve
    AMG_Sover(par_A, par_b, par_x);
}
\end{lstlisting}


At Line 9, there is also an explicit fault masking event (i.e., fault overwritten by an assignment operation) for $par\_A[4].data$ if a fault happens in $par\_A[4].data$. This fault masking is similar to the one at Line 6.
At Line 13, there is an invocation of an algebraic multi-grid solver (AMG)
%whose fault masking is quantified based on the algorithm-level analysis (see below).
that can tolerate faults in the matrix because of the algorithm-level semantics of AMG (particularly, AMG's iterative, multilevel structure~\cite{mg_ics12}).

This example reveals many interesting facts.
In essence, a program can be regarded as a combination of data objects and
operations performed on the data objects.
An operation refers to the arithmetic computation, assignment, logical and comparison operations,  
%occurred in a basic block ~\footnote{A basic block is a single entrance, single exit sequence of instructions.}
or an invocation of an algorithm implementation (e.g.,  a multigrid solver, a conjugate gradient method, or a Monte Carlo simulation).  %a conjugate gradient method,
%The operation can be a statement within a basic block or a routine implementing an algorithm.
%The operation can cause a data object to interact with other data objects by reading/writing the data objects, and at last impact the application outcome.
%(\textbf{TODO: Dong: add a sentence here to correlate the operation with LLVM IR}).
An operation may inherently come with fault masking effects, exemplified at Line 6 (fault overwritten);
An operation may propagate faults, exemplified at Line 7. 
%which affects the interaction between the target data objects and other data objects.
Different operations have different fault masking effects, and hence
impact the application outcome differently.
Different applications can have different operations because of
algorithm implementation and compiler optimization, hence the
applications can have different application-level resilience.
%application-level resilience, because 
%the applications have different program constructs, algorithm implementations, and 
%compiler optimizations, which impact operations.


Based on the above discussion, we classify application-level fault masking 
%commonly found in applications 
into three classes.

(1) \textbf{Operation-level fault masking.} At individual operations, a fault happened in a data object is masked because of the semantics of the operations. Line 6 in
%Figure->Listing by anzheng
Listing~\ref{fig:general_desc} is an example.

(2) \textbf{Fault masking due to fault propagation.} 
Some fault masking events are implicit and have to be identified beyond a single operation. %within a larger application context.
In particular, a corrupted bit in a data object is not masked in the current operation (e.g., Line 7 in 
%figure->listing by anzheng
Listing~\ref{fig:general_desc}),
but the fault is propagated to another data object and masked in another operation (e.g., Line 9).
Note that simply relying on the operation-level analysis without the fault propagation analysis is not sufficient to recognize these fault masking events.

(3) \textbf{Algorithm-level fault masking.}
Identification of some fault masking events happened in a data object must include algorithm-level information.
The identification of those events is beyond the first two classes.
Examples of such events include %some fault tolerant algorithms, such as 
the multigrid solver~\cite{mg_ics12}, some iterative methods~\cite{2-shantharam2011characterizing}, and certain sorting algorithm~\cite{prdc13:sharma}.  
Furthermore, some application domains, such as image processing and machine learning~\cite{isca07:li}, can also tolerate faults because of less 
strict requirements on the correctness of data values. 

In general, the first two classes are caused by program constructs, and the third class is caused by algorithm semantics. Due to the random nature, 
the traditional random fault injection may omit some fault masking events, or capture them multiple times.
%Hence, the traditional fault injection can be inaccurate.
%and have to rely on massive number of tests to generate sufficient coverage.
Relying on analytical modeling, we can avoid or control the randomness of the fault injection, hence greatly improve resilience evaluation. %accuracy and repeatability. 

Our resilience modeling is analytical, and %on the application-level resilience 
relies on the quantification of the above application-level fault masking events happened on data objects.
We create a new metric to quantify the application-level resilience at \textit{data objects}, and introduce methods to measure the metric based on the above classification of fault masking events.
%Why do we need a metric? How to identify fault masking? Give a general introduction here.
%(\textbf{Dong: add an example to show why random fault masking does not work??. Add a figure here?})

\vspace{-10pt}
\subsection{aDVF: An Application-Level Resilience \\ Metric}
\label{sec:metric}
To quantify the resilience of a data object due to fault masking events, we could simply count the number of fault masking events
that happen to the target data object. 
%However, the number of fault masking events is related to the code size and
%the access intensity of the target data objects. 
However, a direct resilience comparison between data objects in terms of the number of fault masking events cannot provide meaningful quantification of the resilience of data objects. 
%\textcolor{green}{Because the fault masking ratio varies across fault masking events on different data objects.} 
For example, a data object %referenced in most of basic blocks 
may be involved in a lot of fault masking events, 
but this does not necessarily mean this data object is more resilient to faults
than other data objects with fewer fault masking events, because the fault masking events of this data object 
%can spread throughout the application execution and only happen sporadically given a time frame.
%that happens sporadically can be accumulated throughout the program.
can come from a few repeated operations, and the number of fault masking events is accumulated throughout application execution;
This data object could be not resilient, if most of other operations for this data object do not have fault masking. 
%quantify->quantifying by anzheng
Hence, the key to quantifying the resilience of a data object is
to quantify \textit{how often} fault masking happens to the data object.
%the operations that happen to the data object have fault masking.
We introduce a new metric, \textit{aDVF} (i.e., the application-level Data Vulnerability Factor), 
to quantify application-inherent resilience due to fault masking in data objects. aDVF is defined as follows. %For an operation, 
%if an fault following a specific pattern (represented with $ep$) 
%happens at the target data object before this operation, 
%Before an operation happens, 

For an operation performed on a data element of a data object, we reason that if a fault happens at the data element in this operation, 
the application outcome could or could not remain correct in terms of the outcome value and application semantics. %after the operation.
%Here, the operation is defined at the application statement level.
%It can be a memory reference (load or store), an arithmetic operation, 
If the fault does not cause an incorrect application outcome,
then a fault masking event happens to the target data object.
A single operation can operate on one or more data elements of the target data object. 
For a specific operation, aDVF of the target data object is defined as the total number of fault masking events divided by the total number of data elements of the target data object operated on by the operation.

For example, an assignment operation $a[1] = w$ 
%has three operations happened to a data object, the array $a$. 
%The three operations are two reads and one addition.
happens to a data object, the array $a$.
This operation involves one data element ($a[1]$) of the data object $a$.
%An fault can happen at any operand involved in the operation. 
%Use the example $a[1]+C$ again. 
We calculate aDVF for the target data object $a$ in this operation as follows.
%Assuming that a fault happens to $a[1]$ and $C$ is always significantly larger than the erroneous $a[1]$, we reason that 
If a fault happens to $a[1]$, we deduce that 
the erroneous $a[1]$ does not impact application correctness and the fault in $a[1]$ is always masked. Hence, the number of fault masking events for
the target data object $a$ in this operation is 1. Also, the total number of data elements involved in the operation is 1.
Hence, the aDVF value for the target data object in this addition operation is $1/1=1$.

%The normalization binds the aDVF value to [0, 1], 
%such that we can establish analysis semantics consistent with that of the algorithm-level analysis (see below).
%for the convenience of integration with the algorithm-level analysis (see below). %also the effect the data object size
Based on the above discussion, the definition of aDVF for a data object $X$ in an operation ($aDVF^{X}_{op}$)
is formulated in Equation~\ref{eq:dvf}, where 
$x_i$ is a data element of the target data object $X$, and $m$ is the number of data elements operated on by the operation;
%changed a lot by anzheng
$f$ is a function to count fault masking events happened on a data element. %the i-$th$ data element $x_i$.
\begin{comment}
$f(i)$ %$f(i,ep)$ 
is a function to count fault masking events
happened to a data element $i$ of the target data object operated on by the operation. There are $m$ data elements of the target data object operated on by the operation.
\end{comment}
\vspace{-1pt}
\begin{equation} 
\label{eq:dvf}
%\scriptsize
\footnotesize
	aDVF^{X}_{op} = \sum_{i=0}^{m-1}f(x_i)/m
\end{equation}
\vspace{-5pt}
%We calculate $aDVF_{op}$ for each operation performed on the target data object. %or other interacting data objects,
%aDVF of the target data object for a code region is the arithmetic mean of aDVF of all related operations in the region.
%We calculate $aDVF$ for a code region as follows

The calculation of aDVF for a code segment is similar to the above for an operation, except that
$m$ is the total number of $x$
%add dynamic here by anzheng
involved in all %\textsl{dynamic} 
operations of the code segment. 
To further explain it, we use as an example
a code segment from LU benchmark in SNU\_NPB benchmark suite 1.0.3 (a C-based implementation of the Fortran-based NPB) shown in 
Listing~\ref{fig:advf_example}.
%all Figure -> Listing by anzheng

%\begin{minipage}{\linewidth}
\begin{lstlisting}[label={fig:advf_example}, caption={A code segment from LU.}]
void l2norm(int ldx, int ldy, int ldz, int nx0, \
	int ny0, int nz0, int ist, int iend, int jst, \
    int jend, double v[][ldy/2*2+1][ldx/2*2+1][5], \
    double sum[5])
{
	int i, j, k, m;
    for (m=0;m<5;m++) //the first loop
    	sum[m]=0.0;  //Statement A
    
    for (k=1;k<nz0-1;k++){  //the second loop
    	for (j=jst;j<jend;j++){
        	for (i=ist;i<iend;i++){
            	for (m=0;m<5,m++){
            		sum[m]=sum[m]+v[k][j][i][m]  \
                    	*v[k][j][i][m]; //Statement B
                }
            }
        }
    }
    
    for (m=0;m<5;m++){  //the third loop
    	sum[m]=sqrt(sum[m]/((nx0-2)*  \
        	(ny0-2)*(nz0-2))); //Statement C
    }
} 
\end{lstlisting}
%\end{minipage}


\textbf{An example from LU.} We calculate aDVF for the array $sum[]$. 
%the statement->Statement by Anzheng
Statement $A$ has an assignment operation involving one data element ($sum[m]$) and one fault masking event (i.e., if a fault happens to $sum[m]$, the fault is overwritten by the assignment). Considering that there are five iterations in the first loop ($iter_{num1} = 5$), there are 5 fault masking events happened in 5 data elements of $sum[]$

Statement B has two operations related to $sum[]$ (i.e., an assignment and an addition). The assignment operation involves one data element ($sum[m]$) and one fault masking; the addition operation involves one data element ($sum[m]$) and one potential fault masking (i.e., certain corruptions in $sum[m]$ can be ignored, if ($v[k][j][i][m]*v[k][j][i][m]$) is significantly larger than $sum[m]$). This potential fault masking is counted as $r^\prime$ ($0 \leq r^\prime \leq 1$), depending on where a corruption happens in $sum[m]$ and fault propagation analysis result (see Sections~\ref{sec:statement_analysis} and~\ref{sec:impl} for further discussion). 
Considering the loop structure, there are ($(1+r^\prime) * iter_{num2}$) fault masking events happened in ($2 * iter_{num2}$) elements of $sum[]$, where ``1'' and ``$r^\prime$'' come from the assignment and addition operations respectively. $iter_{num2}$ is the number of iterations in the second loop, which is equal to ($(nz0-2)*(jend-jst)*(iend-jst)*5$).

Statement C has two operations 
%with->to by anzheng
related to $sum[]$ (i.e., an assignment and a division), but only the assignment operation has fault masking.
Considering that there are 5 iterations in the third loop ($iter_{num3} = 5$), there are 5 fault masking events happened on 5 data elements of the target data object in the third loop. In 
%add the  by anzheng
summary, the aDVF calculation for $sum[]$ is shown in Figure~\ref{fig:advf_cal}.  

\begin{comment}
\begin{figure}[t]
	\centering
	\vspace{-10pt}
	\includegraphics[height=0.45\textheight, width=0.48\textwidth]{advf_example.pdf}
	\vspace{-15pt}
	\caption{A code segment from LU. }
	\label{fig:advf_example}
	\vspace{-10pt}
\end{figure}
\end{comment}

\begin{figure}
	\centering
	\includegraphics[height=0.15\textheight, width=0.48\textwidth]{advf_lu_calculation.pdf}
	\vspace{-8pt}
	\caption{Calculating aDVF for a target data object, the array \textit{sum}[] in a code segment from LU. In the figure, $iter_{num1}=5, iter_{num3}=5$ and $iter_{num2} = (nz0-2)*(jend-jst)*(iend-ist)*5.$}
	\label{fig:advf_cal}
	\vspace{-15pt}
\end{figure}

To calculate aDVF for a data object, we must rely on effective identification and counting of fault masking events (i.e., the function $f$).
In Sections~\ref{sec:statement_analysis},~\ref{sec:fault_propagation_analysis} and ~\ref{sec:algo_analysis}, 
we introduce a series of counting methods based on the classification of fault masking events. %(see Section~\ref{sec:general_bg}). 

\subsection{Operation-Level Analysis}
\label{sec:statement_analysis}
To identify fault masking events at the operation level, we analyze 
all possible operations. %performed on any data object.
In particular, we analyze 
architecture-independent, LLVM instructions %code representation
%(see Section~\ref{sec:impl} for implementation details),
and characterize them based on the instruction result sensitivity to corrupted operands. We classify the operation-level fault masking as follows. 

%\begin{itemize}
(1) \textbf{Value overwriting}.  
An operation writes a new value into the target data object, 
and the fault in the target data object is masked. 
For example, the store operation overwrites the fault in the store destination. 
%the add operation overwrites the fault %pre-existing fault in the result variable. 
We also include \textit{trunc} and bit-shifting operations into this category, because the fault can be truncated or shifted away in those operations.

(2) \textbf{Logical and comparison operations}.
If a fault in the target data object does not
change the correctness of logical and comparison operations, the fault is masked.  
Examples of such operations 
include logical \textit{AND} and the predicate expression in a \textit{switch} statement.
%We also attribute bit-shifting operations to this category, because
%these operations are often involved in the logical and comparison operations.    

(3) \textbf{Value shadowing}.
If the corrupted data value in an operand of an operation 
is shadowed by other correct operands involved in the operation,
then the corrupted data has an 
%add an by anzheng
ignorable impact on the correctness of the operation.
%and the incorrect data value is shadowed by other correct operands involved in the operation.
The addition operation at Line 9 in Figure~\ref{fig:general_desc} is such an example. 
We can find many other examples, such as arithmetic multiplication. %and square root.
%, and trunc operation for type casting.
The effectiveness of value shadowing is coupled with the application semantics.  
An operation of $1000+0.0012$ can be treated as equal to $1000+0.0011$ without impacting the execution correctness of application,
while such tiny difference in the two data values may be intolerable in a different application. We will discuss how to identify value shadowing in details in Section~\ref{sec:impl}.
%\end{itemize}
%(\textbf{Dong: add more to describe how we choose a threshold to determine value shadowing.})

Since we focus on the \textit{application}-level resilience modeling,
we do not consider those LLVM instructions that do not have
directly corresponding operations at the application statement level for fault masking analysis. 
Examples of those instructions 
include \textit{getelementptr} (getting the address of a sub-element of an aggregate data structure)
and \textit{phi} (implementing the $\phi$ node in the SSA graph~\cite{llvm_lrm}).

The effectiveness of the operation-level fault masking heavily relies on the fault pattern.
The fault pattern is defined by how fault bits are distributed within
a faulty data element (e.g., single-bit vs. spatial multiple-bit, least significant bit vs. most significant bit, mantissa vs. exponent).
To account for the effects of various fault patterns, an ideal method to count fault masking events %on a particular platform
would be to collect fault patterns in a production environment during a sufficiently long time period, and then use the realistic fault patterns to guide fault masking analysis. 
However, this method is not always practical. 
In the practice of our resilience modeling, we enumerate possible fault patterns for a given operation, %and a data element of the target data object, 
and derive the existence of fault masking for each fault pattern.
Suppose there are $n$ fault patterns, and $m$ ($0 \leq m \leq n$) of which have fault masking happened.
Then, the number of fault masking events is calculated as $m$/$n$,
which is a statistical quantification of possible fault masking.
Using this statistical quantification means that the number of fault masking events can be non-integer. 
%from the probability perspective.
We employ the above enumeration analysis to model fault masking for single-bit faults in our evaluation section, but the method of the enumeration analysis can be applied to analyze all fault patterns.
\vspace{-10pt}

\subsection{Fault Propagation Analysis}
\label{sec:fault_propagation_analysis}
At an operation performed on the target data object, 
if a fault happened in the target data object cannot be masked at the current operation, 
then we use the fault propagation analysis to track whether the corrupted data
is propagated to other data object(s) and the faults (including the original one and the new ones propagated to other data object(s)) 
are masked in the successor operations.
If all of the faults are masked, then we claim that the original fault happened in the target data object is masked.   

For the fault propagation analysis, a big challenge is to 
track all contaminated data which can quickly increase as the fault propagates. 
\begin{comment}
%handle fault explosion.
We use an example shown in Figure~\ref{fig:fault_propagation_code}
as a running example for the fault propagation analysis. %in this section.
This example is from the CG benchmark in SNU\_NPB.
%Figure~\ref{fig:fault_propagation} shows an example of the fault propagation analysis for NPB CG benchmark.
The target data object in this example is the array $r$, and we calculate aDVF for
the assignment operation in the statement $A$. 
If $r[j]$ has a fault at the statement $A$, the fault cannot be masked.
Instead, within the successor four statements (B-E), the fault quickly propagates to four data objects ($rho, d, alpha$ and $z[]$).
%The number of contaminated data objects grows at least linearly with the number of statements. 
\end{comment}
Tracking a large number of contaminated data objects largely increases
analysis time and memory usage. %code complexity. 
%The symbols in the blocks of the figure are a notation for fault masking analysis.
%In particular, the symbol $st_{x}:op_{y}:d_{z}$ refers to an fault occurred in an operation $y$ 
%of an statement $x$, and the fault happens in the data $z$ before the operation.
%For the example of the figure,  we analyze the fault masking of xxx data.
%Within xxx statements, we have xxx different data tainted by the corrupted data at the statement xxx.
To handle the above fault propagation problem, we avoid tracking fault propagation along a long chain of operations
to accelerate the analysis.
%This can address the fault explosion problem, and accelerate the analysis. 
%We introduce two optimization techniques to avoid long tracking.
We introduce an optimization technique to avoid long tracking.

\begin{comment}
\textbf{Optimization 1: leveraging intermediate states.}
%Avoiding tracking fault propagation along a long chain of operations is an effective way to address the fault explosion problem. 
During the fault propagation analysis, if we know the valid data values (i.e., the intermediate state) of some data objects, 
then we can compare the valid intermediate states with the data values in the fault propagation analysis.
A mismatch between the two indicates that the faults occurred in previous operations
will not be masked in the future.
%the pending data states do not have fault masking; otherwise, there is fault masking.
Hence, the valid intermediate state works as ``analysis shortcut'' that allows us to deduce fault masking without tracking fault propagation to the end of the application execution. 

\begin{figure}[h]
	\begin{center}
		\includegraphics[height=0.3\textheight,keepaspectratio]{error_propagation_code.PNG}
		\caption{A code excerpt from the SNU\_NPB CG benchmark to show the fault propagation from the statement $A$ at an iteration $j$.  The target data object is the array $r[]$. A fault in $r[j]$ is propagated to four data objects ($rho$, $d$, $alpha$ and $z$[]) in the statements B-E.}
		\label{fig:fault_propagation_code}
		\vspace{-20pt}
	\end{center}
\end{figure}
\begin{figure}[h]
	\begin{center}
		\includegraphics[height=0.3\textheight,keepaspectratio]{error_propagation_ddg.pdf}
		\caption{The data dependency graph to show the fault propagation for Figure~\ref{fig:fault_propagation_code}.}
		\label{fig:fault_propagation_ddg}
	\end{center}
	\vspace{-20pt}
\end{figure}

To further explain the idea, we use the example in Figure~\ref{fig:fault_propagation_code} again.
We use a dynamic dependency graph (Figure~\ref{fig:fault_propagation_ddg}) of the example to explain the idea.
%shows an example for using the intermediate state. %helps us immediately identify fault masking at the statement C.
%without reaching the statement xxx.
The dynamic dependency graph (DDG) captures the dynamic dependencies among data objects in the course of program execution~\cite{prdc05:Pattabiraman, tdsc11:pattabiraman}. 
%the values produced  in the course of program execution. 
A node in DDG represents a value of a data object produced in the program, and the node is associated with a dynamic operation that produced the value.
An edge in DDG represents an operation.
The source node of the outgoing edge corresponds to an operation operand,
and the destination node corresponds to the value produced by the operation.
The same memory location can be mapped onto
multiple nodes in DDG (e.g., the data object $rho$ in Figure~\ref{fig:fault_propagation_ddg}), just as a memory
location can have multiple value instances during the execution.
%DDG can be generated based on LLVM instrumentation.
%Function calls and returns are represented in the DDG. 
%The reason why DDG  --> (1) fanout; (2) fault propagation;

Tracking the edges in Figure~\ref{fig:fault_propagation_ddg},  we can know how a fault is propagated when $r[j]$ has the fault.
Assume that we know the valid value range\footnote{A \textit{valid} value always results in acceptable application outcomes. The \textit{valid} used in here and in the rest of the paper is defined in terms of application outcomes.} of the data object $alpha$ at the statement $D$.
If a fault that happens in $r[j]$ at the statement $A$ is propagated to $alpha$ and we find that the faulty $alpha$ is not in the valid value range,
then we can deduce that the original fault will not be masked, and 
we avoid the fault tracking after the statement $D$. 

To collect valid intermediate states to accelerate the fault propagation analysis,  
%we instrument the memory references to the target data object, and 
we record the values of some variables (e.g., the values of $alpha$ at the statement $D$) with fault-free execution. %after some operations. 
%In particular, the values of critical data objects are recorded immediately after an operation,
%if the operation has more than x\% of successor operations involving critical data objects 
%or the target data object in the next $y$ operations ($x=40, y=50$ in our tests). 
%involving critical data objects or the target data object in the next $y$ operations ($x=40, y=50$ in our tests). 
We record valid values of a variable if a value of the variable in DDG has at least 10 predecessor nodes.
%In Figure~\ref{fig:fault_propagation_ddg}, the node $alpha$ has 10 predecessor nodes. 
%We use such method to record variable values, because having a large number of predecessor nodes indicates that the recorded value is potentially helpful to address many fault prorogation analysis.
Recording those values is useful, because it is potentially helpful to resolve many pending fault propagation analysis.
%indicates how many nodes are directly impacted by an fault in that no
\end{comment}

\begin{comment}
We choose those operations to output the values, because 
those operations maximize the fanout of data corruption, 
and hence have big potential to result in the above fault propagation problem.
%Those operations heavily involve the target data object, and hence maximum the fanout of the orginia data corruption. 
%Those operations have big potential to lead the fault explosion problem. 
(\textbf{Dong: explain more what is fanout}).

\textbf{Fanout: the fanout of a node is the set of all immediate successors of the node in DDG. The fanout of a node indicates how many nodes are directly impacted by an fault in that node.}
\end{comment}


\textbf{Optimization: bounding propagation path.} 
%Another method to avoid long tracking of fault propagation is to bound the fault propagation path.
We take a sample of the whole fault propagation path.
In particular, we only track the first $k$ operations. 
%sample the whole fault propagation path with the first $k$ operations.
%If the original fault happened in the target data object 
If the original fault and the new faults propagated to other data object(s)
cannot be masked within the first $k$ operations, then we conclude that 
%the original fault 
all of the faults will not be masked after the $k$ operations.
%in the following operations. 

This method, as an analysis approximation, could introduce analysis inaccuracy because of the sampling nature of the method. 
%The effectiveness of this optimization varies from one operation to another.
However, for a fault that propagates to a large amount of data objects, 
%through successor operations, 
bounding the fault propagation path does not cause inaccurate analysis, because given a large amount of corrupted data, it is highly unlikely that all faults are masked, and
%In Bounding the propagation path and immediately 
making a conclusion of no fault masking is correct in most cases.
In the evaluation section, we explore the sensitivity of analysis correctness to the length of the fault propagation path (i.e., $k$). We find that setting the propagation path to 10 is
good to achieve accurate resilience modeling in most cases (87.5\% of all cases). Setting it to 50 is good for all cases.

\vspace{-10pt}
\subsection{Algorithm-Level Analysis}
\label{sec:algo_analysis}
Identifying the algorithm-level fault masking demands domain and algorithm knowledge.  
In our resilience modeling, we want to minimize the usage of domain and algorithm knowledge, such that
the modeling methodology can be general across different domains.

%Recognizing algorithm-level fault masking is challenging, because the domain knowledge is often demanded 
%to determine if an application outcome with the data corruption occurred in an operation is valid.
%The requirement of the domain knowledge imposes a challenge to make the tool portable and generalizable across different domains.
%From the perspective of a programmer who improves program reliability and fault tolerance mechanisms, we want to minimize
%the introduction of domain knowledge.

We use the following strategy to identify the algorithm-level fault masking (see the next paragraph). 
Furthermore, the user can optionally provide a threshold to indicate a satisfiable solution quality.
For example, for an iterative solver such as conjugate gradient and successive over relaxation,
this threshold can be the threshold that governs the convergence of the algorithms.
For the support vector machine algorithm (an artificial intelligence algorithm), this threshold can be a percentage (e.g., 5\%)
of result difference after the fault corruption.
Working hand-in-hand, the strategy (see the next paragraph) and user-defined threshold treat the algorithm as a black box without
%on->of by anzheng
requiring detailed knowledge of the algorithm internal mechanisms and semantics. 
We explain the strategy as follows.

\textbf{A practical strategy for algorithm-level analysis: deterministic fault injection.}
The traditional random fault injection treats the program as a black-box. 
Hence, using the traditional random fault injection could be an effective tool to identify the algorithm-level fault masking.
However, to avoid the limitation of the traditional random fault injection (i.e., randomness), %and high cost),
we use the operation-level analysis and fault propagation analysis to guide fault injection, 
without blindly enforcing fault injection as the traditional method.
%In particular, when determining fault masking for an operation $x$ by the fault propagation analysis and reaching the boundary of the fault propagation analysis,
In particular, when we cannot determine whether a fault masking can happen in the target data object for an operation $op$ because of fault propagation,  we track fault propagation until 
reaching the boundary of the fault propagation analysis.
%(see Optimization 2 in Section~\ref{sec:fault_propagation_analysis}),
If we still cannot determine fault masking at the boundary, then  
we inject a fault into the target data object in $op$, %right after the boundary, 
and then run the application to completion. 
%If the algorithm result is the same as the one without fault injection or does not go beyond the user-provided threshold, 
If the application result is different from the fault-free result, %without fault injection,
but does not go beyond the user-defined threshold, we claim that the algorithm-level fault masking takes effect. %for the operation $x$.

\begin{comment}
As described above, our fault injection has a deterministic plan on when and where to inject faults. Also, the operation-level and fault propagation analysis is complementary to our fault injection. Hence we avoid fault injection if possible.c
\end{comment}

\textbf{Discussion: coupling between fault propagation and algorithm level analysis.}
The fault propagation analysis and algorithm-level analysis are tightly coupled.
If we reach the boundary of the fault propagation analysis and cannot determine fault masking, we use the algorithm-level analysis. %(i.e., the guided fault injection).
%the fault masking attributed to the algorithm-level fault masking may actually come from the fault propagation-based fault masking. 
However, by doing this, 
some of the fault masking events due to the fault propagation and operation-level fault masking after the boundary may be accounted as algorithm-level fault masking.
Although this mis-counting will not impact the correctness of aDVF value, it would overestimate the algorithm-level fault masking.
%we are at a risk of losing accuracy for counting those fault masking events at the level of fault propagation.
%A correct application execution after the deterministic fault injection 
%may be because of fault masking during fault propagation, not because of the algorithm-level fault masking.
\begin{comment}
However, the application execution deemed to be correct at the end of the execution 
may be a result of both the algorithm-level fault masking and the fault propagation-based fault masking.
We cannot count those fault propagation-based fault masking, because we set an upper bound on the number of operations 
for the fault propagation analysis.
Simply speaking, we may lose accuracy trading for simplification of the fault propagation analysis.
\end{comment}

The fundamental reason for the above overestimation is that we bound the boundary of the fault propagation analysis.
However, our study (Section~\ref{sec:eval_sen}) reveals that we can have very good modeling
%in->on by anzheng
accuracy on our count of the algorithm level fault masking,
even if we set the boundary of the fault propagation analysis.
The reason is as follows. %because of the fault explosion:
After the boundary of the fault propagation analysis, the fault is widely propagated, and
the chance to mask all propagated faults by the operation-level fault masking is extremely low.
In fact, in our tests, we found that even if we use a longer fault propagation path to identify fault masking, we are not able to find more fault masking based on the fault propagation analysis.
Hence, as long as the threshold is sufficiently large (e.g., 10), 
we do not overestimate the algorithm-level fault masking. %for identifying fault masking. 
\vspace{-10pt}
In this section we conduct comprehensive experiments to emphasise the effectiveness of DIAL, including evaluations under white-box and black-box settings, robustness to unforeseen adversaries, robustness to unforeseen corruptions, transfer learning, and ablation studies. Finally, we present a new measurement to test the balance between robustness and natural accuracy, which we named $F_1$-robust score. 


\subsection{A case study on SVHN and CIFAR-100}
In the first part of our analysis, we conduct a case study experiment on two benchmark datasets: SVHN \citep{netzer2011reading} and CIFAR-100 \cite{krizhevsky2009learning}. We follow common experiment settings as in \cite{rice2020overfitting, wu2020adversarial}. We used the PreAct ResNet-18 \citep{he2016identity} architecture on which we integrate a domain classification layer. The adversarial training is done using 10-step PGD adversary with perturbation size of 0.031 and a step size of 0.003 for SVHN and 0.007 for CIFAR-100. The batch size is 128, weight decay is $7e^{-4}$ and the model is trained for 100 epochs. For SVHN, the initial learinnig rate is set to 0.01 and decays by a factor of 10 after 55, 75 and 90 iteration. For CIFAR-100, the initial learning rate is set to 0.1 and decays by a factor of 10 after 75 and 90 iterations. 
%We compared DIAL to \cite{madry2017towards} and TRADES \citep{zhang2019theoretically}. 
%The evaluation is done using Auto-Attack~\citep{croce2020reliable}, which is an ensemble of three white-box and one black-box parameter-free attacks, and various $\ell_{\infty}$ adversaries: PGD$^{20}$, PGD$^{100}$, PGD$^{1000}$ and CW$_{\infty}$ with step size of 0.003. 
Results are averaged over 3 restarts while omitting one standard deviation (which is smaller than 0.2\% in all experiments). As can be seen by the results in Tables~\ref{black-and_white-svhn} and \ref{black-and_white-cifar100}, DIAL presents consistent improvement in robustness (e.g., 5.75\% improved robustness on SVHN against AA) compared to the standard AT 
%under variety of attacks 
while also improving the natural accuracy. More results are presented in Appendix \ref{cifar100-svhn-appendix}.


\begin{table}[!ht]
  \caption{Robustness against white-box, black-box attacks and Auto-Attack (AA) on SVHN. Black-box attacks are generated using naturally trained surrogate model. Natural represents the naturally trained (non-adversarial) model.
  %and applied to the best performing robust models.
  }
  \vskip 0.1in
  \label{black-and_white-svhn}
  \centering
  \small
  \begin{tabular}{l@{\hspace{1\tabcolsep}}c@{\hspace{1\tabcolsep}}c@{\hspace{1\tabcolsep}}c@{\hspace{1\tabcolsep}}c@{\hspace{1\tabcolsep}}c@{\hspace{1\tabcolsep}}c@{\hspace{1\tabcolsep}}c@{\hspace{1\tabcolsep}}c@{\hspace{1\tabcolsep}}c@{\hspace{1\tabcolsep}}c}
    \toprule
    & & \multicolumn{4}{c}{White-box} & \multicolumn{4}{c}{Black-Box}  \\
    \cmidrule(r){3-6} 
    \cmidrule(r){7-10}
    Defense Model & Natural & PGD$^{20}$ & PGD$^{100}$  & PGD$^{1000}$  & CW$^{\infty}$ & PGD$^{20}$ & PGD$^{100}$ & PGD$^{1000}$  & CW$^{\infty}$ & AA \\
    \midrule
    NATURAL & 96.85 & 0 & 0 & 0 & 0 & 0 & 0 & 0 & 0 & 0 \\
    \midrule
    AT & 89.90 & 53.23 & 49.45 & 49.23 & 48.25 & 86.44 & 86.28 & 86.18 & 86.42 & 45.25 \\
    % TRADES & 90.35 & 57.10 & 54.13 & 54.08 & 52.19 & 86.89 & 86.73 & 86.57 & 86.70 &  49.50 \\
    $\DIAL_{\kl}$ (Ours) & 90.66 & \textbf{58.91} & \textbf{55.30} & \textbf{55.11} & \textbf{53.67} & 87.62 & 87.52 & 87.41 & 87.63 & \textbf{51.00} \\
    $\DIAL_{\ce}$ (Ours) & \textbf{92.88} & 55.26  & 50.82 & 50.54 & 49.66 & \textbf{89.12} & \textbf{89.01} & \textbf{88.74} & \textbf{89.10} &  46.52  \\
    \bottomrule
  \end{tabular}
\end{table}


\begin{table}[!ht]
  \caption{Robustness against white-box, black-box attacks and Auto-Attack (AA) on CIFAR100. Black-box attacks are generated using naturally trained surrogate model. Natural represents the naturally trained (non-adversarial) model.
  %and applied to the best performing robust models.
  }
  \vskip 0.1in
  \label{black-and_white-cifar100}
  \centering
  \small
  \begin{tabular}{l@{\hspace{1\tabcolsep}}c@{\hspace{1\tabcolsep}}c@{\hspace{1\tabcolsep}}c@{\hspace{1\tabcolsep}}c@{\hspace{1\tabcolsep}}c@{\hspace{1\tabcolsep}}c@{\hspace{1\tabcolsep}}c@{\hspace{1\tabcolsep}}c@{\hspace{1\tabcolsep}}c@{\hspace{1\tabcolsep}}c}
    \toprule
    & & \multicolumn{4}{c}{White-box} & \multicolumn{4}{c}{Black-Box}  \\
    \cmidrule(r){3-6} 
    \cmidrule(r){7-10}
    Defense Model & Natural & PGD$^{20}$ & PGD$^{100}$  & PGD$^{1000}$  & CW$^{\infty}$ & PGD$^{20}$ & PGD$^{100}$ & PGD$^{1000}$  & CW$^{\infty}$ & AA \\
    \midrule
    NATURAL & 79.30 & 0 & 0 & 0 & 0 & 0 & 0 & 0 & 0 & 0 \\
    \midrule
    AT & 56.73 & 29.57 & 28.45 & 28.39 & 26.6 & 55.52 & 55.29 & 55.26 & 55.40 & 24.12 \\
    % TRADES & 58.24 & 30.10 & 29.66 & 29.64 & 25.97 & 57.05 & 56.71 & 56.67 & 56.77 & 24.92 \\
    $\DIAL_{\kl}$ (Ours) & 58.47 & \textbf{31.19} & \textbf{30.50} & \textbf{30.42} & \textbf{26.91} & 57.16 & 56.81 & 56.80 & 57.00 & \textbf{25.87} \\
    $\DIAL_{\ce}$ (Ours) & \textbf{60.77} & 27.87 & 26.66 & 26.61 & 25.98 & \textbf{59.48} & \textbf{59.06} & \textbf{58.96} & \textbf{59.20} & 23.51  \\
    \bottomrule
  \end{tabular}
\end{table}


% \begin{table}[!ht]
%   \caption{Robustness comparison of DIAL to Madry et al. and TRADES defense models on the SVHN dataset under different PGD white-box attacks and the ensemble Auto-Attack (AA).}
%   \label{svhn}
%   \centering
%   \begin{tabular}{llllll|l}
%     \toprule
%     \cmidrule(r){1-5}
%     Defense Model & Natural & PGD$^{20}$ & PGD$^{100}$ & PGD$^{1000}$ & CW$_{\infty}$ & AA\\
%     \midrule
%     $\DIAL_{\kl}$ (Ours) & $\mathbf{90.66}$ & $\mathbf{58.91}$ & $\mathbf{55.30}$ & $\mathbf{55.12}$ & $\mathbf{53.67}$  & $\mathbf{51.00}$  \\
%     Madry et al. & 89.90 & 53.23 & 49.45 & 49.23 & 48.25 & 45.25  \\
%     TRADES & 90.35 & 57.10 & 54.13 & 54.08 & 52.19 & 49.50 \\
%     \bottomrule
%   \end{tabular}
% \end{table}


\subsection{Performance comparison on CIFAR-10} \label{defence-settings}
In this part, we evaluate the performance of DIAL compared to other well-known methods on CIFAR-10. 
%To be consistent with other methods, 
We follow the same experiment setups as in~\cite{madry2017towards, wang2019improving, zhang2019theoretically}. When experiment settings are not identical between tested methods, we choose the most commonly used settings, and apply it to all experiments. This way, we keep the comparison as fair as possible and avoid reporting changes in results which are caused by inconsistent experiment settings \citep{pang2020bag}. To show that our results are not caused because of what is referred to as \textit{obfuscated gradients}~\citep{athalye2018obfuscated}, we evaluate our method with same setup as in our defense model, under strong attacks (e.g., PGD$^{1000}$) in both white-box, black-box settings, Auto-Attack ~\citep{croce2020reliable}, unforeseen "natural" corruptions~\citep{hendrycks2018benchmarking}, and unforeseen adversaries. To make sure that the reported improvements are not caused by \textit{adversarial overfitting}~\citep{rice2020overfitting}, we report best robust results for each method on average of 3 restarts, while omitting one standard deviation (which is smaller than 0.2\% in all experiments). Additional results for CIFAR-10 as well as comprehensive evaluation on MNIST can be found in Appendix \ref{mnist-results} and \ref{additional_res}.
%To further keep the comparison consistent, we followed the same attack settings for all methods.


\begin{table}[ht]
  \caption{Robustness against white-box, black-box attacks and Auto-Attack (AA) on CIFAR-10. Black-box attacks are generated using naturally trained surrogate model. Natural represents the naturally trained (non-adversarial) model.
  %and applied to the best performing robust models.
  }
  \vskip 0.1in
  \label{black-and_white-cifar}
  \centering
  \small
  \begin{tabular}{cccccccc@{\hspace{1\tabcolsep}}c}
    \toprule
    & & \multicolumn{3}{c}{White-box} & \multicolumn{3}{c}{Black-Box} \\
    \cmidrule(r){3-5} 
    \cmidrule(r){6-8}
    Defense Model & Natural & PGD$^{20}$ & PGD$^{100}$ & CW$^{\infty}$ & PGD$^{20}$ & PGD$^{100}$ & CW$^{\infty}$ & AA \\
    \midrule
    NATURAL & 95.43 & 0 & 0 & 0 & 0 & 0 & 0 &  0 \\
    \midrule
    TRADES & 84.92 & 56.60 & 55.56 & 54.20 & 84.08 & 83.89 & 83.91 &  53.08 \\
    MART & 83.62 & 58.12 & 56.48 & 53.09 & 82.82 & 82.52 & 82.80 & 51.10 \\
    AT & 85.10 & 56.28 & 54.46 & 53.99 & 84.22 & 84.14 & 83.92 & 51.52 \\
    ATDA & 76.91 & 43.27 & 41.13 & 41.01 & 75.59 & 75.37 & 75.35 & 40.08\\
    $\DIAL_{\kl}$ (Ours) & 85.25 & $\mathbf{58.43}$ & $\mathbf{56.80}$ & $\mathbf{55.00}$ & 84.30 & 84.18 & 84.05 & \textbf{53.75} \\
    $\DIAL_{\ce}$ (Ours)  & $\mathbf{89.59}$ & 54.31 & 51.67 & 52.04 &$ \mathbf{88.60}$ & $\mathbf{88.39}$ & $\mathbf{88.44}$ & 49.85 \\
    \midrule
    $\DIAL_{\awp}$ (Ours) & $\mathbf{85.91}$ & $\mathbf{61.10}$ & $\mathbf{59.86}$ & $\mathbf{57.67}$ & $\mathbf{85.13}$ & $\mathbf{84.93}$ & $\mathbf{85.03}$  & \textbf{56.78} \\
    $\TRADES_{\awp}$ & 85.36 & 59.27 & 59.12 & 57.07 & 84.58 & 84.58 & 84.59 & 56.17 \\
    \bottomrule
  \end{tabular}
\end{table}



\paragraph{CIFAR-10 setup.} We use the wide residual network (WRN-34-10)~\citep{zagoruyko2016wide} architecture. %used in the experiments of~\cite{madry2017towards, wang2019improving, zhang2019theoretically}. 
Sidelong this architecture, we integrate a domain classification layer. To generate the adversarial domain dataset, we use a perturbation size of $\epsilon=0.031$. We apply 10 of inner maximization iterations with perturbation step size of 0.007. Batch size is set to 128, weight decay is set to $7e^{-4}$, and the model is trained for 100 epochs. Similar to the other methods, the initial learning rate was set to 0.1, and decays by a factor of 10 at iterations 75 and 90. 
%For being consistent with other methods, the natural images are padded with 4-pixel padding with 32-random crop and random horizontal flip. Furthermore, all methods are trained using SGD with momentum 0.9. For $\DIAL_{\kl}$, we balance the robust loss with $\lambda=6$ and the domains loss with $r=4$. For $\DIAL_{\ce}$, we balance the robust loss with $\lambda=1$ and the domains loss with $r=2$. 
%We also introduce a version of our method that incorporates the AWP double-perturbation mechanism, named DIAL-AWP.
%which is trained using the same learning rate schedule used in ~\cite{wu2020adversarial}, where the initial 0.1 learning rate decays by a factor of 10 after 100 and 150 iterations. 
See Appendix \ref{cifar10-additional-setup} for additional details.

\begin{table}[ht]
  \caption{Black-box attack using the adversarially trained surrogate models on CIFAR-10.}
  \vskip 0.1in
  \label{black-box-cifar-adv}
  \centering
  \small
  \begin{tabular}{ll|c}
    \toprule
    \cmidrule(r){1-2}
    Surrogate (source) model & Target model & robustness \% \\
    % \midrule
    \midrule
    TRADES & $\DIAL_{\ce}$ & $\mathbf{67.77}$ \\
    $\DIAL_{\ce}$ & TRADES & 65.75 \\
    \midrule
    MART & $\DIAL_{\ce}$ & $\mathbf{70.30}$ \\
    $\DIAL_{\ce}$ & MART & 64.91 \\
    \midrule
    AT & $\DIAL_{\ce}$ & $\mathbf{65.32}$ \\
    $\DIAL_{\ce}$ & AT  & 63.54 \\
    \midrule
    ATDA & $\DIAL_{\ce}$ & $\mathbf{66.77}$ \\
    $\DIAL_{\ce}$ & ATDA & 52.56 \\
    \bottomrule
  \end{tabular}
\end{table}

\paragraph{White-box/Black-box robustness.} 
%We evaluate all defense models using Auto-Attack, PGD$^{20}$, PGD$^{100}$, PGD$^{1000}$ and CW$_{\infty}$ with step size 0.003. We constrain all attacks by the same perturbation $\epsilon=0.031$. 
As reported in Table~\ref{black-and_white-cifar} and Appendix~\ref{additional_res}, our method achieves better robustness compared to the other methods. Specifically, in the white-box settings, our method improves robustness over~\citet{madry2017towards} and TRADES by 2\% 
%using the common PGD$^{20}$ attack 
while keeping higher natural accuracy. We also observe better natural accuracy of 1.65\% over MART while also achieving better robustness over all attacks. Moreover, our method presents significant improvement of up to 15\% compared to the the domain invariant method suggested by~\citet{song2018improving} (ATDA).
%in both natural and robust accuracy. 
When incorporating 
%the double-perturbation mechanism of 
AWP, our method improves the results of $\TRADES_{\awp}$ by almost 2\%.
%and reaches state-of-the-art results for robust models with no additional data. 
% Additional results are available in Appendix~\ref{additional_res}.
When tested on black-box settings, $\DIAL_{\ce}$ presents a significant improvement of more than 4.4\% over the second-best performing method, and up to 13\%. In Table~\ref{black-box-cifar-adv}, we also present the black-box results when the source model is taken from one of the adversarially trained models. %Then, we compare our model to each one of them both as the source model and target model. 
In addition to the improvement in black-box robustness, $\DIAL_{\ce}$ also manages to achieve better clean accuracy of more than 4.5\% over the second-best performing method.
% Moreover, based on the auto-attack leader-board \footnote{\url{https://github.com/fra31/auto-attack}}, our method achieves the 1st place among models without additional data using the WRN-34-10 architecture.

% \begin{table}
%   \caption{White-box robustness on CIFAR-10 using WRN-34-10}
%   \label{white-box-cifar-10}
%   \centering
%   \begin{tabular}{lllll}
%     \toprule
%     \cmidrule(r){1-2}
%     Defense Model & Natural & PGD$^{20}$ & PGD$^{100}$ & PGD$^{1000}$ \\
%     \midrule
%     TRADES ~\cite{zhang2019theoretically} & 84.92  & 56.6 & 55.56 & 56.43  \\
%     MART ~\cite{wang2019improving} & 83.62  & 58.12 & 56.48 & 56.55  \\
%     Madry et al. ~\cite{madry2017towards} & 85.1  & 56.28 & 54.46 & 54.4  \\
%     Song et al. ~\cite{song2018improving} & 76.91 & 43.27 & 41.13 & 41.02  \\
%     $\DIAL_{\ce}$ (Ours) & $ \mathbf{90}$  & 52.12 & 48.88 & 48.78  \\
%     $\DIAL_{\kl}$ (Ours) & 85.25 & $\mathbf{58.43}$ & $\mathbf{56.8}$ & $\mathbf{56.73}$ \\
%     \midrule
%     $\DIAL_{\kl}$+AWP (Ours) & $\mathbf{85.91}$ & $\mathbf{61.1}$ & - & -  \\
%     TRADES+AWP \cite{wu2020adversarial} & 85.36 & 59.27 & 59.12 & -  \\
%     % MART + AWP & 84.43 & 60.68 & 59.32 & -  \\
%     \bottomrule
%   \end{tabular}
% \end{table}


% \begin{table}
%   \caption{White-box robustness on MNIST}
%   \label{white-box-mnist}
%   \centering
%   \begin{tabular}{llllll}
%     \toprule
%     \cmidrule(r){1-2}
%     Defense Model & Natural & PGD$^{40}$ & PGD$^{100}$ & PGD$^{1000}$ \\
%     \midrule
%     TRADES ~\cite{zhang2019theoretically} & 99.48 & 96.07 & 95.52 & 95.22 \\
%     MART ~\cite{wang2019improving} & 99.38  & 96.99 & 96.11 & 95.74  \\
%     Madry et al. ~\cite{madry2017towards} & 99.41  & 96.01 & 95.49 & 95.36 \\
%     Song et al. ~\cite{song2018improving}  & 98.72 & 96.82 & 96.26 & 96.2  \\
%     $\DIAL_{\kl}$ (Ours) & 99.46 & 97.05 & 96.06 & 95.99  \\
%     $\DIAL_{\ce}$ (Ours) & $\mathbf{99.49}$  & $\mathbf{97.38}$ & $\mathbf{96.45}$ & $\mathbf{96.33}$ \\
%     \bottomrule
%   \end{tabular}
% \end{table}


% \paragraph{Attacking MNIST.} For consistency, we use the same perturbation and step sizes. For MNIST, we use $\epsilon=0.3$ and step size of $0.01$. The natural accuracy of our surrogate (source) model is 99.51\%. Attacks results are reported in Table~\ref{black-and_white-mnist}. It is worth noting that the improvement margin is not conclusive on MNIST as it is on CIFAR-10, which is a more complex task.

% \begin{table}
%   \caption{Black-box robustness on MNIST and CIFAR-10 using naturally trained surrogate model and best performing robust models}
%   \label{black-box-mnist-cifar}
%   \centering
%   \begin{tabular}{lllllll}
%     \toprule
%     & \multicolumn{3}{c}{MNIST} & \multicolumn{3}{c}{CIFAR-10} \\
%     \cmidrule(r){2-4} 
%     \cmidrule(r){5-7}  
%     Defense Model & PGD$^{40}$ & PGD$^{100}$ & PGD$^{1000}$ & PGD$^{20}$ & PGD$^{100}$ & PGD$^{1000}$ \\
%     \midrule
%     TRADES ~\cite{zhang2019theoretically} & 98.12 & 97.86 & 97.81 & 84.08 & 83.89 & 83.8 \\
%     MART ~\cite{wang2019improving} & 98.16 & 97.96 & 97.89  & 82.82 & 82.52 & 82.47 \\
%     Madry et al. ~\cite{madry2017towards}  & 98.05 & 97.73 & 97.78 & 84.22 & 84.14 & 83.96 \\
%     Song et al. ~\cite{song2018improving} & 97.74 & 97.28 & 97.34 & 75.59 & 75.37 & 75.11 \\
%     $\DIAL_{\kl}$ (Ours) & 98.14 & 97.83 & 97.87  & 84.3 & 84.18 & 84.0 \\
%     $\DIAL_{\ce}$ (Ours)  & $\mathbf{98.37}$ & $\mathbf{98.12}$ & $\mathbf{98.05}$  & $\mathbf{89.13}$ & $\mathbf{88.89}$ & $\mathbf{88.78}$ \\
%     \bottomrule
%   \end{tabular}
% \end{table}



% \subsubsection{Ensemble attack} In addition to the white-box and black-box settings, we evaluate our method on the Auto-Attack ~\citep{croce2020reliable} using $\ell_{\infty}$ threat model with perturbation $\epsilon=0.031$. Auto-Attack is an ensemble of parameter-free attacks. It consists of three white-box attacks: APGD-CE which is a step size-free version of PGD on the cross-entropy ~\citep{croce2020reliable}. APGD-DLR which is a step size-free version of PGD on the DLR loss ~\citep{croce2020reliable} and FAB which  minimizes the norm of the adversarial perturbations, and one black-box attack: square attack which is a query-efficient black-box attack ~\citep{andriushchenko2020square}. Results are presented in Table~\ref{auto-attack}. Based on the auto-attack leader-board \footnote{\url{https://github.com/fra31/auto-attack}}, our method achieves the 1st place among models without additional data using the WRN-34-10 architecture.

%Additional results can be found in Appendix ~\ref{additional_res}.

% \begin{table}
%   \caption{Auto-Attack (AA) on CIFAR-10 with perturbation size $\epsilon=0.031$ with $\ell_{\infty}$ threat model}
%   \label{auto-attack}
%   \centering
%   \begin{tabular}{lll}
%     \toprule
%     \cmidrule(r){1-2}
%     Defense Model & AA \\
%     \midrule
%     TRADES ~\cite{zhang2019theoretically} & 53.08  \\
%     MART ~\cite{wang2019improving} & 51.1  \\
%     Madry et al. ~\cite{madry2017towards} & 51.52    \\
%     Song et al.   ~\cite{song2018improving} & 40.18 \\
%     $\DIAL_{\ce}$ (Ours) & 47.33  \\
%     $\DIAL_{\kl}$ (Ours) & $\mathbf{53.75}$ \\
%     \midrule
%     DIAL-AWP (Ours) & $\mathbf{56.78}$ \\
%     TRADES-AWP \cite{wu2020adversarial} & 56.17 \\
%     \bottomrule
%   \end{tabular}
% \end{table}


% \begin{table}[!ht]
%   \caption{Auto-Attack (AA) Robustness (\%) on CIFAR-10 with $\epsilon=0.031$ using an $\ell_{\infty}$ threat model}
%   \label{auto-attack}
%   \centering
%   \begin{tabular}{cccccc|cc}
%     \toprule
%     % \multicolumn{8}{c}{Defence Model}  \\
%     % \cmidrule(r){1-8} 
%     TRADES & MART & Madry & Song & $\DIAL_{\ce}$ & $\DIAL_{\kl}$ & DIAL-AWP  & TRADES-AWP\\
%     \midrule
%     53.08 & 51.10 & 51.52 &  40.08 & 47.33  & $\mathbf{53.75}$ & $\mathbf{56.78}$ & 56.17 \\

%     \bottomrule
%   \end{tabular}
% \end{table}

% \begin{table}[!ht]
% \caption{$F_1$-robust measurement using PGD$^{20}$ attack in white-box and black-box settings on CIFAR-10}
%   \label{f1-robust}
%   \centering
%   \begin{tabular}{ccccccc|cc}
%     \toprule
%     % \multicolumn{8}{c}{Defence Model}  \\
%     % \cmidrule(r){1-8} 
%     Defense Model & TRADES & MART & Madry & Song & $\DIAL_{\kl}$ & $\DIAL_{\ce}$ & DIAL-AWP  & TRADES-AWP\\
%     \midrule
%     White-box & 0.659 & 0.666 & 0.657 & 0.518 & $\mathbf{0.675}$  & 0.643 & $\mathbf{0.698}$ & 0.682 \\
%     Black-box & 0.844 & 0.831 & 0.846 & 0.761 & 0.847 & $\mathbf{0.895}$ & $\mathbf{0.854}$ &  0.849 \\
%     \bottomrule
%   \end{tabular}
% \end{table}

\subsubsection{Robustness to Unforeseen Attacks and Corruptions}
\paragraph{Unforeseen Adversaries.} To further demonstrate the effectiveness of our approach, we test our method against various adversaries that were not used during the training process. We attack the model under the white-box settings with $\ell_{2}$-PGD, $\ell_{1}$-PGD, $\ell_{\infty}$-DeepFool and $\ell_{2}$-DeepFool \citep{moosavi2016deepfool} adversaries using Foolbox \citep{rauber2017foolbox}. We applied commonly used attack budget 
%(perturbation for PGD adversaries and overshot for DeepFool adversaries) 
with 20 and 50 iterations for PGD and DeepFool, respectively.
Results are presented in Table \ref{unseen-attacks}. As can be seen, our approach  gains an improvement of up to 4.73\% over the second best method under the various attack types and an average improvement of 3.7\% over all threat models.


\begin{table}[ht]
  \caption{Robustness on CIFAR-10 against unseen adversaries under white-box settings.}
  \vskip 0.1in
  \label{unseen-attacks}
  \centering
%   \small
  \begin{tabular}{c@{\hspace{1.5\tabcolsep}}c@{\hspace{1.5\tabcolsep}}c@{\hspace{1.5\tabcolsep}}c@{\hspace{1.5\tabcolsep}}c@{\hspace{1.5\tabcolsep}}c@{\hspace{1.5\tabcolsep}}c@{\hspace{1.5\tabcolsep}}c}
    \toprule
    Threat Model & Attack Constraints & $\DIAL_{\kl}$ & $\DIAL_{\ce}$ & AT & TRADES & MART & ATDA \\
    \midrule
    \multirow{2}{*}{$\ell_{2}$-PGD} & $\epsilon=0.5$ & 76.05 & \textbf{80.51} & 76.82 & 76.57 & 75.07 & 66.25 \\
    & $\epsilon=0.25$ & 80.98 & \textbf{85.38} & 81.41 & 81.10 & 80.04 & 71.87 \\\midrule
    \multirow{2}{*}{$\ell_{1}$-PGD} & $\epsilon=12$ & 74.84 & \textbf{80.00} & 76.17 & 75.52 & 75.95 & 65.76 \\
    & $\epsilon=7.84$ & 78.69 & \textbf{83.62} & 79.86 & 79.16 & 78.55 & 69.97 \\
    \midrule
    $\ell_{2}$-DeepFool & overshoot=0.02 & 84.53 & \textbf{88.88} & 84.15 & 84.23 & 82.96 & 76.08 \\\midrule
    $\ell_{\infty}$-DeepFool & overshoot=0.02 & 68.43 & \textbf{69.50} & 67.29 & 67.60 & 66.40 & 57.35 \\
    \bottomrule
  \end{tabular}
\end{table}


%%%%%%%%%%%%%%%%%%%%%%%%% conference version %%%%%%%%%%%%%%%%%%%%%%%%%%%%%%%%%%%%%
\paragraph{Unforeseen Corruptions.}
We further demonstrate that our method consistently holds against unforeseen ``natural'' corruptions, consists of 18 unforeseen diverse corruption types proposed by \citet{hendrycks2018benchmarking} on CIFAR-10, which we refer to as CIFAR10-C. The CIFAR10-C benchmark covers noise, blur, weather, and digital categories. As can be shown in Figure \ref{corruption}, our method gains a significant and consistent improvement over all the other methods. Our method leads to an average improvement of 4.7\% with minimum improvement of 3.5\% and maximum improvement of 5.9\% compared to the second best method over all unforeseen attacks. See Appendix \ref{corruptions-apendix} for the full experiment results.


\begin{figure}[h]
 \centering
  \includegraphics[width=0.4\textwidth]{figures/spider_full.png}
%   \caption{Summary of accuracy over all unforeseen corruptions compared to the second and third best performing methods.}
  \caption{Accuracy comparison over all unforeseen corruptions.}
  \label{corruption}
\end{figure}


%%%%%%%%%%%%%%%%%%%%%%%%% conference version %%%%%%%%%%%%%%%%%%%%%%%%%%%%%%%%%%%%%

%%%%%%%%%%%%%%%%%%%%%%%%% Arxiv version %%%%%%%%%%%%%%%%%%%%%%%%%%%%%%%%%%%%%
% \newpage
% \paragraph{Unforeseen Corruptions.}
% We further demonstrate that our method consistently holds against unforeseen "natural" corruptions, consists of 18 unforeseen diverse corruption types proposed by \cite{hendrycks2018benchmarking} on CIFAR-10, which we refer to as CIFAR10-C. The CIFAR10-C benchmark covers noise, blur, weather, and digital categories. As can be shown in Figure  \ref{spider-full-graph}, our method gains a significant and consistent improvement over all the other methods. Our approach leads to an average improvement of 4.7\% with minimum improvement of 3.5\% and maximum improvement of 5.9\% compared to the second best method over all unforeseen attacks. Full accuracy results against unforeseen corruptions are presented in Tables \ref{corruption-table1} and \ref{corruption-table2}. 

% \begin{table}[!ht]
%   \caption{Accuracy (\%) against unforeseen corruptions.}
%   \label{corruption-table1}
%   \centering
%   \tiny
%   \begin{tabular}{lcccccccccccccccccc}
%     \toprule
%     Defense Model & brightness & defocus blur & fog & glass blur & jpeg compression & motion blur & saturate & snow & speckle noise  \\
%     \midrule
%     TRADES & 82.63 & 80.04 & 60.19 & 78.00 & 82.81 & 76.49 & 81.53 & 80.68 & 80.14 \\
%     MART & 80.76 & 78.62 & 56.78 & 76.60 & 81.26 & 74.58 & 80.74 & 78.22 & 79.42 \\
%     AT &  83.30 & 80.42 & 60.22 & 77.90 & 82.73 & 76.64 & 82.31 & 80.37 & 80.74 \\
%     ATDA & 72.67 & 69.36 & 45.52 & 64.88 & 73.22 & 63.47 & 72.07 & 68.76 & 72.27 \\
%     DIAL (Ours)  & \textbf{87.14} & \textbf{84.84} & \textbf{66.08} & \textbf{81.82} & \textbf{87.07} & \textbf{81.20} & \textbf{86.45} & \textbf{84.18} & \textbf{84.94} \\
%     \bottomrule
%   \end{tabular}
% \end{table}


% \begin{table}[!ht]
%   \caption{Accuracy (\%) against unforeseen corruptions.}
%   \label{corruption-table2}
%   \centering
%   \tiny
%   \begin{tabular}{lcccccccccccccccccc}
%     \toprule
%     Defense Model & contrast & elastic transform & frost & gaussian noise & impulse noise & pixelate & shot noise & spatter & zoom blur \\
%     \midrule
%     TRADES & 43.11 & 79.11 & 76.45 & 79.21 & 73.72 & 82.73 & 80.42 & 80.72 & 78.97 \\
%     MART & 41.22 & 77.77 & 73.07 & 78.30 & 74.97 & 81.31 & 79.53 & 79.28 & 77.8 \\
%     AT & 43.30 & 79.58 & 77.53 & 79.47 & 73.76 & 82.78 & 80.86 & 80.49 & 79.58 \\
%     ATDA & 36.06 & 67.06 & 62.56 & 70.33 & 64.63 & 73.46 & 72.28 & 70.50 & 67.31 \\
%     DIAL (Ours) & \textbf{48.84} & \textbf{84.13} & \textbf{81.76} & \textbf{83.76} & \textbf{78.26} & \textbf{87.24} & \textbf{85.13} & \textbf{84.84} & \textbf{83.93}  \\
%     \bottomrule
%   \end{tabular}
% \end{table}


% \begin{figure}[!ht]
%   \centering
%   \includegraphics[width=9cm]{figures/spider_full.png}
%   \caption{Accuracy comparison with all tested methods over unforeseen corruptions.}
%   \label{spider-full-graph}
% \end{figure}
% %%%%%%%%%%%%%%%%%%%%%%%%% Arxiv version %%%%%%%%%%%%%%%%%%%%%%%%%%%%%%%%%%%%%
%%%%%%%%%%%%%%%%%%%%%%%%% Arxiv version %%%%%%%%%%%%%%%%%%%%%%%%%%%%%%%%%%%%%

\subsubsection{Transfer Learning}
Recent works \citep{salman2020adversarially,utrera2020adversarially} suggested that robust models transfer better on standard downstream classification tasks. In Table \ref{transfer-res} we demonstrate the advantage of our method when applied for transfer learning across CIFAR10 and CIFAR100 using the common linear evaluation protocol. see Appendix \ref{transfer-learning-settings} for detailed settings.

\begin{table}[H]
  \caption{Transfer learning results comparison.}
  \vskip 0.1in
  \label{transfer-res}
  \centering
  \small
\begin{tabular}{c|c|c|c}
\toprule

\multicolumn{2}{l}{} & \multicolumn{2}{c}{Target} \\
\cmidrule(r){3-4}
Source & Defence Model & CIFAR10 & CIFAR100 \\
\midrule
\multirow{3}{*}{CIFAR10} & DIAL & \multirow{3}{*}{-} & \textbf{28.57} \\
 & AT &  & 26.95  \\
 & TRADES &  & 25.40  \\
 \midrule
\multirow{3}{*}{CIFAR100} & DIAL & \textbf{73.68} & \multirow{3}{*}{-} \\
 & AT & 71.41 & \\
 & TRADES & 71.42 &  \\
%  \midrule
% \multirow{3}{}{SVHN} & DIAL &  &  & \multirow{3}{}{-} \\
%  & Madry et al. &  &  &  \\
%  & TRADES &  &  &  \\ 
\bottomrule
\end{tabular}
\end{table}


\subsubsection{Modularity and Ablation Studies}

We note that the domain classifier is a modular component that can be integrated into existing models for further improvements. Removing the domain head and related loss components from the different DIAL formulations results in some common adversarial training techniques. For $\DIAL_{\kl}$, removing the domain and related loss components results in the formulation of TRADES. For $\DIAL_{\ce}$, removing the domain and related loss components results in the original formulation of the standard adversarial training, and for $\DIAL_{\awp}$ the removal results in $\TRADES_{\awp}$. Therefore, the ablation studies will demonstrate the effectiveness of combining DIAL on top of different adversarial training methods. 

We investigate the contribution of the additional domain head component introduced in our method. Experiment configuration are as in \ref{defence-settings}, and robust accuracy is based on white-box PGD$^{20}$ on CIFAR-10 dataset. We remove the domain head from both $\DIAL_{\kl}$, $\DIAL_{\awp}$, and $\DIAL_{\ce}$ (equivalent to $r=0$) and report the natural and robust accuracy. We perform 3 random restarts and omit one standard deviation from the results. Results are presented in Figure \ref{ablation}. All DIAL variants exhibits stable improvements on both natural accuracy and robust accuracy. $\DIAL_{\ce}$, $\DIAL_{\kl}$, and $\DIAL_{\awp}$ present an improvement of 1.82\%, 0.33\%, and 0.55\% on natural accuracy and an improvement of 2.5\%, 1.87\%, and 0.83\% on robust accuracy, respectively. This evaluation empirically demonstrates the benefits of incorporating DIAL on top of different adversarial training techniques.
% \paragraph{semi-supervised extensions.} Since the domain classifier does not require the class labels, we argue that additional unlabeled data can be leveraged in future work.
%for improved results. 

\begin{figure}[ht]
  \centering
  \includegraphics[width=0.35\textwidth]{figures/ablation_graphs3.png}
  \caption{Ablation studies for $\DIAL_{\kl}$, $\DIAL_{\ce}$, and $\DIAL_{\awp}$ on CIFAR-10. Circle represent the robust-natural accuracy without using DIAL, and square represent the robust-natural accuracy when incorporating DIAL.
  %to further investigate the impact of the domain head and loss on natural and robust accuracy.
  }
  \label{ablation}
\end{figure}

\subsubsection{Visualizing DIAL}
To further illustrate the superiority of our method, we visualize the model outputs from the different methods on both natural and adversarial test data.
% adversarial test data generated using PGD$^{20}$ white-box attack with step size 0.003 and $\epsilon=0.031$ on CIFAR-10. 
Figure~\ref{tsne1} shows the embedding received after applying t-SNE ~\citep{van2008visualizing} with two components on the model output for our method and for TRADES. DIAL seems to preserve strong separation between classes on both natural test data and adversarial test data. Additional illustrations for the other methods are attached in Appendix~\ref{additional_viz}. 

\begin{figure}[h]
\centering
  \subfigure[\textbf{DIAL} on natural logits]{\includegraphics[width=0.21\textwidth]{figures/domain_ce_test.png}}
  \hspace{0.03\textwidth}
  \subfigure[\textbf{DIAL} on adversarial logits]{\includegraphics[width=0.21\textwidth]{figures/domain_ce_adversarial.png}}
  \hspace{0.03\textwidth}
    \subfigure[\textbf{TRADES} on natural logits]{\includegraphics[width=0.21\textwidth]{figures/trades_test.png}}
    \hspace{0.03\textwidth}
    \subfigure[\textbf{TRADES} on adversarial logits]{\includegraphics[width=0.21\textwidth]{figures/trades_adversarial.png}}
  \caption{t-SNE embedding of model output (logits) into two-dimensional space for DIAL and TRADES using the CIFAR-10 natural test data and the corresponding PGD$^{20}$ generated adversarial examples.}
  \label{tsne1}
\end{figure}


% \begin{figure}[ht]
% \centering
%   \begin{subfigure}{4cm}
%     \centering\includegraphics[width=3.3cm]{figures/domain_ce_test.png}
%     \caption{\textbf{DIAL} on nat. examples}
%   \end{subfigure}
%   \begin{subfigure}{4cm}
%     \centering\includegraphics[width=3.3cm]{figures/domain_ce_adversarial.png}
%     \caption{\textbf{DIAL} on adv. examples}
%   \end{subfigure}
  
%   \begin{subfigure}{4cm}
%     \centering\includegraphics[width=3.3cm]{figures/trades_test.png}
%     \caption{\textbf{TRADES} on nat. examples}
%   \end{subfigure}
%   \begin{subfigure}{4cm}
%     \centering\includegraphics[width=3.3cm]{figures/trades_adversarial.png}
%     \caption{\textbf{TRADES} on adv. examples}
%   \end{subfigure}
%   \caption{t-SNE embedding of model output (logits) into two-dimensional space for DIAL and TRADES using the CIFAR-10 natural test data and the corresponding adversarial examples.}
%   \label{tsne1}
% \end{figure}



% \begin{figure}[ht]
% \centering
%   \begin{subfigure}{6cm}
%     \centering\includegraphics[width=5cm]{figures/domain_ce_test.png}
%     \caption{\textbf{DIAL} on nat. examples}
%   \end{subfigure}
%   \begin{subfigure}{6cm}
%     \centering\includegraphics[width=5cm]{figures/domain_ce_adversarial.png}
%     \caption{\textbf{DIAL} on adv. examples}
%   \end{subfigure}
  
%   \begin{subfigure}{6cm}
%     \centering\includegraphics[width=5cm]{figures/trades_test.png}
%     \caption{\textbf{TRADES} on nat. examples}
%   \end{subfigure}
%   \begin{subfigure}{6cm}
%     \centering\includegraphics[width=5cm]{figures/trades_adversarial.png}
%     \caption{\textbf{TRADES} on adv. examples}
%   \end{subfigure}
%   \caption{t-SNE embedding of model output (logits) into two-dimensional space for DIAL and TRADES using the CIFAR-10 natural test data and the corresponding adversarial examples.}
%   \label{tsne1}
% \end{figure}



\subsection{Balanced measurement for robust-natural accuracy}
One of the goals of our method is to better balance between robust and natural accuracy under a given model. For a balanced metric, we adopt the idea of $F_1$-score, which is the harmonic mean between the precision and recall. However, rather than using precision and recall, we measure the $F_1$-score between robustness and natural accuracy,
using a measure we call
%We named it
the
\textbf{$\mathbf{F_1}$-robust} score.
\begin{equation}
F_1\text{-robust} = \dfrac{\text{true\_robust}}
{\text{true\_robust}+\frac{1}{2}
%\cdot
(\text{false\_{robust}}+
\text{false\_natural})},
\end{equation}
where $\text{true\_robust}$ are the adversarial examples that were correctly classified, $\text{false\_{robust}}$ are the adversarial examples that were miss-classified, and $\text{false\_natural}$ are the natural examples that were miss-classified.
%We tested the proposed $F_1$-robust score using PGD$^{20}$ on CIFAR-10 dataset in white-box and black-box settings. 
Results are presented in Table~\ref{f1-robust} and demonstrate that our method achieves the best $F_1$-robust score in both settings, which supports our findings from previous sections.

% \begin{table}[!ht]
%   \caption{$F_1$-robust measurement using PGD$^{20}$ attack in white and black box settings on CIFAR-10}
%   \label{f1-robust}
%   \centering
%   \begin{tabular}{lll}
%     \toprule
%     \cmidrule(r){1-2}
%     Defense Model & White-box & Black-box \\
%     \midrule
%     TRADES & 0.65937  & 0.84435 \\
%     MART & 0.66613  & 0.83153  \\
%     Madry et al. & 0.65755 & 0.84574   \\
%     Song et al. & 0.51823 & 0.76092  \\
%     $\DIAL_{\ce}$ (Ours) & 0.65318   & $\mathbf{0.88806}$  \\
%     $\DIAL_{\kl}$ (Ours) & $\mathbf{0.67479}$ & 0.84702 \\
%     \midrule
%     \midrule
%     DIAL-AWP (Ours) & $\mathbf{0.69753}$  & $\mathbf{0.85406}$  \\
%     TRADES-AWP & 0.68162 & 0.84917 \\
%     \bottomrule
%   \end{tabular}
% \end{table}

\begin{table}[ht]
\small
  \caption{$F_1$-robust measurement using PGD$^{20}$ attack in white and black box settings on CIFAR-10.}
  \vskip 0.1in
  \label{f1-robust}
  \centering
%   \small
  \begin{tabular}{c
  @{\hspace{1.5\tabcolsep}}c @{\hspace{1.5\tabcolsep}}c @{\hspace{1.5\tabcolsep}}c @{\hspace{1.5\tabcolsep}}c
  @{\hspace{1.5\tabcolsep}}c @{\hspace{1.5\tabcolsep}}c @{\hspace{1.5\tabcolsep}}|
  @{\hspace{1.5\tabcolsep}}c
  @{\hspace{1.5\tabcolsep}}c}
    \toprule
    % \cmidrule(r){8-9}
     & TRADES & MART & AT & ATDA & $\DIAL_{\ce}$ & $\DIAL_{\kl}$ & $\DIAL_{\awp}$ & $\TRADES_{\awp}$ \\
    \midrule
    White-box & 0.659 & 0.666 & 0.657 & 0.518 & 0.660 & \textbf{0.675} & \textbf{0.698} & 0.682 \\
    Black-box & 0.844 & 0.831 & 0.845 & 0.761 & \textbf{0.890} & 0.847 & \textbf{0.854} & 0.849 \\ 
    \bottomrule
  \end{tabular}
\end{table}


\section{Conclusion}

A method for inferring nonlinear VAR models has been proposed and validated. The modeling assumption that the observed data are the outputs of nodal nonlinearities applied to the individual time series of a linear VAR process lying in an unknown latent vector space. Since the number of parameters that determine the topology does not increase, the model interpretability remains the same as that with linear VAR modeling, making the proposed model amenable for Granger causality testing and network topology identification. The optimization method, similar to that of DNN training, can be extended with state-of-the-art tools to accelerate training and avoid undesired effects such as convergence to unstable points and overfitting.

\subsubsection{Acknowledgement:}

The authors would like to thank Emilio Ruiz Moreno for helping us manage a more elegant derivation of the gradient of $g_i(\cdot)$.

\bibliographystyle{IEEEtran}
\bibliography{intap}

\end{document}