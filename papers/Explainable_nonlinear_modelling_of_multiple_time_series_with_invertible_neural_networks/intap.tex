\documentclass{llncs}

\usepackage{graphicx} % figures

\usepackage{amsmath} % math
\usepackage{amsfonts}
\usepackage{amssymb}
\usepackage{accents}
\newcommand{\ubar}[1]{\underaccent{\bar}{#1}}

\usepackage[pdftex,dvipsnames]{xcolor}  % Coloured text etc.

\usepackage{listings} % code listings
\lstset{frame=tb,
  language=Python,
  aboveskip=3mm,
  belowskip=3mm,
  showstringspaces=false,
  columns=flexible,
  basicstyle={\small\ttfamily},
  numbers=none,
  numberstyle=\tiny\color{gray},
  keywordstyle=\color{blue},
  commentstyle=\color{dkgreen},
  stringstyle=\color{mauve},
  breaklines=true,
  breakatwhitespace=true,
  tabsize=3
}

% Marking equal contribution.
\makeatletter
\newcommand{\printfnsymbol}[1]{%
  \textsuperscript{\@fnsymbol{#1}}%
}
\makeatother
%%%%%%%%%%%%%%%%%%%%%%%%%

\begin{document}

\title{Explainable nonlinear modelling of multiple time series with invertible neural networks\thanks{
The work in this paper was supported by the SFI Offshore Mechatronics
grant 237896/O30.
}
}
\titlerunning{Explainable VAR modeling with invertible NNs}

\author{Luis Miguel Lopez-Ramos\thanks{Equal contribution in terms of working hours.}%\inst{1,2}
\orcidID{0000-0001-8072-3994} \and
\\ 
Kevin Roy\printfnsymbol{2}%\inst{1,2}
%\orcidID{1111-2222-3333-4444} 
\and
Baltasar Beferull-Lozano%\inst{1,2}
\orcidID{0000-0002-0902-6245}
}

\authorrunning{Lopez-Ramos & Roy et al.}

\institute{SFI Offshore Mechatronics Center, University of Agder
\and
Intelligent Signal Processing and Wireless Networks (WISENET) Center
\and
Department of ICT, University of Agder, Grimstad, Norway
}
\maketitle

\begin{abstract}
    A method for nonlinear topology identification is proposed, based on the assumption that a collection of time series are generated in two steps: i) a vector autoregressive process in a latent space, and ii) a nonlinear, component-wise, monotonically increasing observation mapping. The latter mappings are assumed invertible, and are modeled as shallow neural networks, so that their inverse can be numerically evaluated, and their parameters can be learned using a technique inspired in deep learning. Due to the function inversion, the backpropagation step is not straightforward, and this paper explains the steps needed to calculate the gradients applying implicit differentiation. Whereas the model explainability is the same as that for linear VAR processes, preliminary numerical tests show that the prediction error becomes smaller.
\end{abstract}

\keywords{Vector autoregressive model
\and nonlinear
\and network topology inference
\and invertible neural network
}

\section{Introduction}
\label{sec:Introduction}


The goal in top-$\size$ recommendation is to recommend to each
consumer a small set of $\size$ items from a large collection of
items~\cite{cremonesi2010performance}.  For example, Netflix may want
to recommend $\size$ appealing movies to each consumer.  Collaborative
Filtering (CF)~\cite{herlocker2002empirical,lee2012comparative} is a
common top-$\size$ recommendation method.  CF infers user interests by
analyzing partially observed user-item interaction data, such as user
ratings on movies or historical purchase
logs~\cite{kanagal2012supercharging}. The main assumption in CF is that
users with similar interaction patterns have similar interests.


Standard CF methods for top-$\size$ recommendation focus on making  suggestions  that accurately reflect the user's preference history. However, as  observed in previous work,  CF recommendations are generally biased toward  popular items, leading to a rich get richer effect~\cite{vargas2014improving,steck2011item}.  The major reasons for this are \textit{popularity bias} and \textit{sparsity} of CF interaction data (detailed in Section~\ref{sec:related-work}). In a nutshell, to maintain  accuracy, recommendations are generated from the dense regions of the data,  where the popular items lie.  

However,  accurately suggesting popular items, may not be satisfactory for the consumers. For example, in Netflix, an accuracy-focused movie recommender may recommend ``Star Wars: The Force Awakens'' to users who have seen ``Star Wars: Rogue One''.  But, those users are probably already aware of ``The Force Awakens''. Considering additional factors, such as novelty of recommendations,  can lead to more effective suggestions~\cite{cremonesi2010performance,Castells2015,zhang2008avoiding,ziegler2005improving,zhang2012auralist}. 
%Second, accuracy-focused models typically achieve a   overall item-space coverage across their recommendations,  whereas high item-space coverage helps providers of the items increase revenue
%, users satisfaction since they are  likely already aware of or can find these items on their own.  

Focusing on popular items also adversely affects the satisfaction of  the providers of the items. This is because  accuracy-focused models typically achieve a  low overall item space coverage across their recommendations, whereas   high item space coverage helps providers of the items increase their revenue~\cite{vargas2014improving,Castells2015,adomavicius2011maximizing,anderson2006thelongtail, yin2012challenging,adomavicius2012improving}.
%accuracy-focused models typically achieve a

In contrast to the relatively small number of popular items, there are copious  {\it long-tail\/} items that have fewer observations (e.g., ratings) available. More precisely,  using the Pareto  principle (i.e.,~the $80/20$ rule),  long-tail items can be defined as items that generate the lower $20\%$ of observations~\cite{yin2012challenging}. Experimentally we found that these items correspond to almost $85\%$ of the items in several datasets (Sections~\ref{sec:Notation} and \ref{sec:Experiments}). %Table~\ref{tab:DatasetStatsticsSmall})


As previously shown, one way to improve the novelty of top-$\size$ sets is to recommend interesting long-tail items~\cite{cremonesi2010performance,ge2010beyond}.  The intuition  is that since they have fewer observations available,  they are more likely to be unseen~\cite{Kaminskas:2016:DSN:3028254.2926720}.  
 %For example, in online commerce,  newly added items are long-tail items that are yet to be discovered.  
Moreover, long-tail item promotion also results in higher overall coverage of the item space%, which increases profits for providers of the items
~\cite{vargas2014improving,Castells2015,zhang2008avoiding,zhang2012auralist,adomavicius2011maximizing,anderson2006thelongtail,yin2012challenging,jambor2010optimizing}. Because long-tail promotion reduces accuracy~\cite{steck2011item}, there are trade-offs to be explored.


%original submitted to ICDE
%This work studies three aspects of top-$\size$ recommendation: accuracy, novelty, and item-space coverage, and examines their trade-offs. In most previous work, predictions of a base recommendation system are re-ranked to handle their trade-offs~\cite{adomavicius2012improving,jambor2010optimizing,zhang2013personalize,wang2009portfolio}. Due to performance considerations, however, these techniques are not customized per user. For example,  parameters that balance the trade-off between novelty and accuracy are cross-validated at a global level.  This can be detrimental since users have varying preferences for  objectives such as long-tail novelty. We explore how to  automatically infer  user  preference for long-tail novelty, and how to leverage  it to correct  the popularity bias in standard recommender models. Our work does not rely on any additional contextual data, although such data, if available, can help promote newly-added long-tail items~\cite{agarwal2009regression,Saveski:2014:ICR:2645710.2645751}.

This work studies three aspects of top-$\size$ recommendation: accuracy, novelty, and item space coverage, and examines their trade-offs. In most previous work, predictions of a base recommendation algorithm are \textit{re-ranked} to handle these trade-offs~\cite{adomavicius2012improving,jambor2010optimizing,zhang2013personalize,wang2009portfolio}. The re-ranking models are computationally efficient but suffer from two drawbacks. First, due to performance considerations,  parameters that balance the trade-off between novelty and accuracy  are not customized per user. Instead they are cross-validated at a global level.  This can be detrimental since users have varying preferences for  objectives such as long-tail novelty. Second,  the re-ranking methods are often limited to a specific base recommender  that may be sensitive to dataset density. 
As a result, the datasets are pruned and the problem is studied in dense settings~\cite{adomavicius2012improving,ho2014likes}; but real world  scenarios are often sparse~\cite{kanagal2012supercharging,liu2017experimental}.   
% Because  dataset density can impact the performance of most base recommenders (like R-SVD), which in turn affects the performance of the re-ranking model, 

\iffalse
We address these limitations by directly inferring  user  preference for long-tail novelty  from interaction data.  This  allows us to customize the re-ranking  per user, and design a \textit{generic} framework, which resolves the second problem. In particular, since the long-tail novelty preferences are estimated independently of any base  recommender model, we can  plug-in an appropriate base recommender w.r.t. the dataset sparsity.% including ones that are more suitable for sparse settings.  

Modelling  user  preference for  long-tail novelty using only item popularity statistics, e.g., the average popularity of rated items as in~\cite{jugovac2017efficient}, disregards additional information like whether the user found the item interesting and the long-tail preferences of other users  of the items. \iffalse To incorporate them, we introduce the notion of  \emph{item long-tail importance}. Both  user long-tail preferences and item long-tail importance are dependent:  a user has high preference for discovering long-tail items if she is interested in important long-tail items, and an item that is associated with many of these kinds of users is likely to be more important.  We propose a joint optimization framework to directly learn,  from interaction data, both the users' long-tail preferences and the  items' long-tail importance. \fi
We propose an optimization approach that  incorporates  this information and  directly learns,  from interaction data, the users' long-tail novelty preferences.

Next, we use these learned preferences  to design a  top-$\size$ recommendation framework thats is generic, and provides customized balance between accuracy, novelty, and coverage. We refer to it as framework as GANC.  Using GANC, we design a novel algorithm, {\it Ordered Sampling-based Locally Greedy (OSLG)\/}, that relies on the learned long-tail novelty preferences  to scalably correct for popularity bias. Our work does not rely on any additional contextual data, although such data, if available, can help promote newly-added long-tail items~\cite{agarwal2009regression,Saveski:2014:ICR:2645710.2645751}. In summary:
\fi

We address the first limitation by directly inferring  user  preference for long-tail novelty  from interaction data.   Estimating these  preferences  using only item popularity statistics, e.g., the average popularity of rated items as in~\cite{jugovac2017efficient}, disregards additional information, like whether the user found the item interesting or the long-tail preferences of other users  of the items. We propose an approach that  incorporates  this information and  learns the users' long-tail novelty preferences from interaction data.

This approach allows us to customize the re-ranking  per user, and  design a \textit{generic} re-ranking framework, which resolves the second limitation of prior work. In particular, since the long-tail novelty preferences are estimated independently of any base recommender, we can  plug-in an appropriate one w.r.t. different factors, such as the dataset sparsity.

Our top-$\size$ recommendation framework, \textbf{GANC}, is \textbf{G}eneric, and provides customized balance between \textbf{A}ccuracy, \textbf{N}ovelty, and \textbf{C}overage. % Moreover, based on the learned long-tail novelty preferences, we also design a novel algorithm, {\it Ordered Sampling-based Locally Greedy (OSLG)\/}, that relies on the learned long-tail novelty preferences  to scalably correct for popularity bias. 
Our work does not rely on any additional contextual data, although such data, if available, can help promote newly-added long-tail items~\cite{agarwal2009regression,Saveski:2014:ICR:2645710.2645751}. In summary:

%Consider  the following toy example:
\vspace{-0.2cm}
\begin{table}[htb]
\centering
\scriptsize
%\small
\begin{tabular}{ccccccc} 
%\toprule
%&\multirow{2}{*}{}&\multicolumn{7}{c}{Ratings}\\
& & \cellcolor{blue!35}$w_1$ &\cellcolor{blue!18} $w_2$ & $\dots$ &\cellcolor{blue!8} $w_{89}$  &\cellcolor{blue!8} $w_{99}$   
\\
&   &$i_1$&$i_2$&$\dots$&$i_{89}$&$i_{90}$\\ 
\cmidrule(r){3-7} 	 
%\midrule
\cellcolor{red!35}$\theta_1$  &$u_1 $   &5 &   & $\dots$ &  &   \\
\cellcolor{red!28}$\theta_2$  &$u_2$     &5 &    & $\dots$ &  &  \\
 $\theta_3=?$  &$\bf u_3$  &5 &  &   $\dots$ &  &  \\
\cellcolor{red!10}$\theta_4$ & $u_4$  &  &5   & $\dots$ & &\\ 
\cellcolor{red!10}$\theta_5$ & $u_5$  &  & 5  & $\dots$ & &\\ 
$\theta_6=?$  & $\bf u_6$ & &5  &      $\dots$& &  \\ 
 & & $\hdots$  &$\hdots$   &$\hdots$   &$\hdots$   &$\hdots$  \\
%\midrule 
\cmidrule(r){3-7} 	 
\multicolumn{2}{c}{item pop.}  & 3  & 3  & $\dots$ &50&60\\  
%\bottomrule
%$ f_i$    &3  &3  &1  &3  &1  &2  \\  \hline
\end{tabular}
%#.
\caption{Simplified user-item interaction data. The user long-tail novelty preference ($\theta_u$), item long-tail importance weight ($w_i$) are highlighted. Darker colors indicate larger values. } \label{tab:example}
\end{table} 
\vspace{-0.2cm}
\begin{example}  
In Table~\ref{tab:example}, we are interested in estimating $\theta_3$ and $\theta_6$,  the long-tail preference of users $u_3$ and $u_6$ who have each rated a single movie. Additional ratings for other users  are not included here.  Considering only rating information, we observe $i_1$ and $i_2$ are  equally popular $|\mathcal{U}_{i_1}^{\trainset}| = |\mathcal{U}_{i_2}^{\trainset}|=3$, and $r_{31}=5$ and $r_{62}=5$. Using Eq.~\ref{eq:tfidf-risk}  we have $\theta_3 = \theta_6$. However, if we were given the long-tail preferences of the each item's user set, specifically that $u_1$ and $u_2$ have high long-tail preference (darker red), while $u_4$ and $u_5$ have lower long-tail preference (lighter red), we could conclude $i_1$ is a more important long-tail item compared to $i_2$ (indicated by a darker blue shade for $w_1$), and we expect  $\theta_3 \geq \theta_6$.

% On the other hand, if we knew that $u_4$ and $u_5$ have lower long-tail preference, we could conclude $i_2$ is a  less significant long-tail item. Therefore, However, if we  consider the long-tail preferences of other users, we may reason differently.    We need another variable $w_i$ which captures this information. 
%we would conclude that $u_3$ has higher long-tail preference compared to $u_6$, since the users $i_1$ is a more prominent long-tail item. 

% Relying only  on item popularity information, we would  conclude   $u_3$ and $u_6$ have equal long-tail preference, since $i_1$ and $i_2$ are  equally popular. However, considering  the second column,  long-tail preference of users,  long-tail importance for each item,  which captures the long-tail preference of its users. Since  that  both users of $i_1$ have high long-tail preference while  the users of $i_2$ have lower preference,  we may conclude $i_1$ is a more important long-tail item compared to $i_2$. Therefore, $u_3$'s long-tail preference should be at least as large as $u_6$'s preference. Specifically, consider two  items $i_1$ and $i_2$, with the following rating data: $i_1=\{u_1:5, u_2:5, u_3:5 \}$, $i_2=\{u_4:5, u_5:5, u_6:5\}$.  

%Table~\ref{tab:example} shows  simplified rating data. We want an estimate of the long-tail preference of $u_3$ and $u_6$, who have each  rated a single movie.  Relying only  on movie popularity information, we would  conclude   $u_3$ and $u_6$ have similar long-tail preference, since $m_1$ and $m_2$ are  equally popular. However, considering the long-tail preferences of other users of those movies, we may reason differently: since $u_1$ and $u_2$ have high long-tail preference, and $u_4$ and $u_5$ have low long-tail preference, $m_1$ is a more prominent long-tail item compared to $m_2$. Therefore, it is likely that $u_3$ has higher long-tail preference compared to $u_6$.considering the long-tail preferences of other users of those movies, we may reason differently.  For example, 
\label{ex:running}
\end{example}



%------------------------------

\iffalse
\begin{example}
Table~\ref{tab:example} shows rating data for a simplified system. %Note the user-item interaction matrix is sparse.
For this example, we define popular movies as those that have received  three or more ratings; $\{m_1, m_2, m_4\}$ are popular and  $\{m_3, m_5, m_6\}$ are niche movies. We observe $u_1$ and $u_3$  have rated relatively popular movies (risk-averse) while $u_2$ and $u_4$ have rated niche movies (risk-loving). 
\label{ex:running}
\end{example}

\begin{table}[htb]
\centering
\scriptsize
\begin{tabular}{ccccccc} 
\toprule
			&$m_1$ &$m_2$   &$m_3$    &$m_4$   &$m_5$ &$m_6$  \\ \hline 
$u_1 $ &5  &4  & - &-  &-  &-   \\
$u_2$  &-  &-  &-  &-  &5  &5   \\
$u_3$  &-  &4  &-  &5  &-  &-   \\
$u_4$  &-  &-  &3  &-  &-  &4   \\ 
$u_5$  &5  &-  &-  &3  &-  &-   \\ 
$u_6$  &4  &2  &-  &4  &-  &-   \\ 
\bottomrule
%$ f_i$    &3  &3  &1  &3  &1  &2  \\  \hline
\end{tabular}
\caption{User-Movie rating data} \label{tab:example}
\end{table}

It is essential to consider consumer characteristics in designing recommender systems so that they promote long-tail items to the right group of users and spread demand evenly between hit and niche items.  

\fi





%------------------------------
\iffalse
\begin{table}[htb]
\centering
\scriptsize
\begin{tabular}{ccccccc} 
\toprule
			&$m_1$ &$m_2$   &$m_3$    &$m_4$   &$m_5$ &$m_6$  \\ \hline 
$u_1 $ &\textbf{5}  & \textbf{4}  &\textcolor{gray}{ 1.2} &-  &-  &-   \\
$u_2$  &-  &-  &-  &-  & \textbf{5}  &\textbf{5}   \\
$u_3$  &-  &\textbf{4}  &-  &\textbf{5}  &-  &-   \\
$u_4$  &-  &-  &\textbf{3}  &-  &-  &\textbf{4}   \\ 
$u_5$  &\textbf{5}  &-  &-  &\textbf{3}  &-  &-   \\ 
$u_6$  &\textbf{4}  &\textbf{2}  &-  &\textbf{4}  &-  &-   \\ 
\bottomrule
%$ f_i$    &3  &3  &1  &3  &1  &2  \\  \hline
\end{tabular}
\caption{User-Movie rating data} \label{tab:example}
\end{table}
% $\mathcal{P}^1= \{ \mathcal{P}_1^1 \{i_1,i_2,i_3\}, \mathcal{P}_2^1:\{i_2,i_3,i_5\}  \}$
 %$\mathcal{P}^2= \{ \mathcal{P}_1^2: \{i_1,i_2,i_3\}, \mathcal{P}_2^2:\{i_2,i_5,i_6\}  \}$
 %$\mathcal{P}^3= \{ \mathcal{P}_1^3: \{i_7,i_8,i_9\}, \mathcal{P}_2^3:\{i_{10},i_{11},i_{12}\}  \}$
\begin{table}[htb]
\centering
\tiny
\begin{tabular}{ccc} 
\toprule
		&$u_1$&$u_2$  \\ \hline 
$\mathcal{P}^1 $ & $\{i_1,i_2,i_3\}$ & $\{i_2,i_3,i_5\} $ \\
$\mathcal{P}^2$ & $\{i_1,i_2,i_3\}$ & $\{i_2,i_5,i_6\} $ \\
$\mathcal{P}^3$ & $\{i_7,i_8,i_9\}$ & $\{i_{10},i_{11},i_{12} \}$ \\
\bottomrule
%$ f_i$    &3  &3  &1  &3  &1  &2  \\  \hline
\end{tabular}
\caption{Top-$\size$ allocations to users.} \label{tab:paretoExamples}
\end{table}
\fi


\iffalse
When considering long-tail items, it is important to consider consumers' willingness  to explore niche or unpopular items and their propensity towards similar items. In particular, they can be characterized by their  {\it risk degree\/} and {\it focusing degree\/}, respectively.  We compute these estimates  based on historical rating information. The following example further describes these notions in the context of movie rating data. 

\begin{example}  
Table~\ref{tab:example} shows rating data for a simplified system with $6$ users, $6$ movies, and $3$ genres. $m_i^{j}$ implies that movie $m_i$ belongs to genre $j$. Note the user-item interaction matrix is sparse. 
  For this setting, we define popular movies as those that have received  three or more ratings; $\{m_1, m_2, m_4\}$ are popular and  $\{m_3, m_5, m_6\}$ are niche movies. We now profile the users according to their risk and focusing degree. E.g., $u_1$ has rated relatively popular movies belonging to the same genre (risk-averse, high focusing degree); $u_2$ has rated niches movies in the same genre (risk-loving, high focusing degree); $u_3$ has rated popular movies in two different genres (risk-averse, low focusing degree), and $u_4$ has rated niches movies in two different genres (risk-loving, low focusing degree). 
\label{ex:running}
\end{example}
\begin{table}[htb]
\centering
\tiny
\begin{tabular}{ccccccc} 
\toprule
			&$m_1^{1}$ &$m_2^{1}$   &$m_3^{2}$    &$m_4^{3}$   &$m_5^{3}$ &$m_6^{3}$  \\ \hline 
$u_1 $ &5  &4  &-  &-  &-  &-   \\
$u_2$  &-  &-  &-  &-  &5  &5   \\
$u_3$  &-  &4  &-  &5  &-  &-   \\
$u_4$  &-  &-  &3  &-  &-  &4   \\ 
$u_5$  &5  &-  &-  &3  &-  &-   \\ 
$u_6$  &4  &2  &-  &4  &-  &-   \\ 
\bottomrule
%$ f_i$    &3  &3  &1  &3  &1  &2  \\  \hline
\end{tabular}
\caption{User-Movie rating data} \label{tab:example}
\end{table}
It is essential to consider these consumer characteristics in designing recommender systems so that they promote long-tail items to the right group of users and spread demand evenly between the hit and niche items.  
\fi
\iffalse
\begin{center}
\begin{figure*}[tp]
%\scalebox{0.5}{%
\resizebox{1\textwidth}{!}{%
%\small%\addtolength{\tabcolsep}{5pt}% below sums to 8
\begin{tabularx}{1.5\textwidth}{>{\hsize=2.5\hsize}X>{\hsize=2.5\hsize}X>{\hsize=0.5\hsize}X>{\hsize=0.5\hsize}X>{\hsize=0.5\hsize}X>{\hsize=0.5\hsize}X>{\hsize=0.5\hsize}X>{\hsize=0.5\hsize}X}
    \multirow{12}{*}{\includegraphics[scale=0.3]{codeForExample/popularity-movie.png}} & \multirow{12}{*}{\includegraphics[scale=0.3]{codeForExample/scatterplot.png}} & & & & & & \\
%   & &               &       &       &       &       &       \\
    & &\multicolumn{1}{l|}{}               &$m_1^{g1}$   	&$m_2^{g1}$    	&$m_3^{g2}$    &$m_4^{g2}$      &$m_5^{g3}$    \\ \cline{3-8}%\hline
    & &\multicolumn{1}{l|}{u1}          &5  &5  &-  &-   &-  \\
    & &\multicolumn{1}{l|}{u2}    		&-  &-  &4  &4  &5  \\
    & &\multicolumn{1}{l|}{u3}   			&1  &2  &1  &-  &-   \\
    & &\multicolumn{1}{l|}{u4}     		&1  &-  &-  &-  &-  \\
    & &               &       &       &       &       &       \\
    & &               &       &       &       &       &       \\
    & &               &       &       &       &       &       \\
    & &               &       &       &       &       &	\\
    \\
\end{tabularx}}
\caption{User-Movie interaction data a) Popularity-Movie histogram b)Movie genres/clusters c) User-Movie rating data} \label{fig:example}
\end{figure*}
\end{center}
\fi



%We propose a novel approach that allows us to  promote long-tail items in a targeted manner, thereby improving the novelty of top-$\size$ sets, the overall item-space coverage across recommendations, while maintaining reasonable levels of accuracy.

%Next, we integrate these learned preferences  in a generic  top-$\size$ recommendation framework to provide customized balance between accuracy and coverage.

%sequentially make recommendations, while adjusting its parameters with regard to the set of top-$\size$ recommendations made so far. However, since  sequential parameter updates  cause  scalability issues, we propose a sampling based algorithm. This variant of our framework, called {\it Ordered Sampling-based Locally Greedy (OSLG)\/},  allows us to  correct for the popularity bias in recommendations with regard to individual user long-tail preferences. 

%ICDE submission
%Our framework differs with  prior work in the following aspects:  unlike~\cite{adomavicius2011maximizing,adomavicius2012improving,zhang2013personalize,ho2014likes},  the long-tail preference personalization in our framework is learned rather than optimized using cross-validation or parameter tuning. In other words, our personalization method is independent of the underlying base  recommendation models.  Moreover, our framework is  generic. This enables us to  plug-in several base recommenders, and evaluate their  effectiveness without requiring  extensive tuning for the accuracy and coverage trade-off. 


%\vspace{-2.8pt}
\begin{itemize}

\item  We examine various measures for estimating user long-tail novelty preference in Section~\ref{sec:lt-pref} and formulate an optimization problem  to directly learn users' preferences for long-tail  items from interaction data in Section~\ref{sec:learning-lt-pref}. %In addition, we introduce several heuristics for measuring the user preference for less common items from historical rating data.% 

\item  We integrate the user preference estimates into GANC %, a generic re-ranking framework that provides customized balance between accuracy, novelty, and coverage 
(Section~\ref{sec:RiskbasedReranking}), and  introduce {\it Ordered Sampling-based Locally Greedy (OSLG)\/}, a scalable algorithm that relies  on user long-tail preferences to correct the popularity bias (Section~\ref{sec:optimizationAlgorithm}).
%We introduce OSLG, a scalable algorithm that relies  on user long-tail preferences to  maximize item space coverage \textcolor{red}{while maintaining acceptable levels of accuracy} (Section~\ref{sec:optimizationAlgorithm}).

\item   We conduct an extensive empirical study and evaluate performance from  accuracy, novelty, and coverage perspectives (Section~\ref{sec:Experiments}).  We use five  datasets with varying density and difficulty levels. %:  Netflix, MovieTweetings, and MovieLens (100K, 1M, 10M). 
  In contrast to most related work,  our evaluation considers realistic settings that include a large number of infrequent  items and users. %This enables us to study the impact of  data density on the performance trade-offs of several  state of the art top-$\size$ recommendation algorithms. %   %,  and use the all-items ranking protocol~\cite{steck2013evaluation,vargas2014improving}, where performance is measured using all items with train data. to evaluate the performance of several  state of the art top-$\size$ recommendation algorithms 
 
\item Our empirical results confirm that the performance of re-ranking models is impacted by the underlying   base recommender and the dataset density. Our generic approach enables us to easily incorporate a suitable base recommender to devise an effective solution for both dense and sparse settings. In dense settings, we use the same base recommender as existing re-ranking approaches, and we outperform them in accuracy and coverage metrics. For sparse settings, we plug-in a more suitable base recommender, and devise an effective solution that is competitive with existing top-$\size$ recommendation methods in accuracy and novelty. 

%Directly estimating the long-tail novelty preferences allows us to customize re-ranking per user, and  devise a generic framework.   
 
\end{itemize}

Section~\ref{sec:related-work} describes related work. Section~\ref{sec:conclusion} concludes.


% Panoptic segmentation

% 3D segmentation

% Multi-object tracking

% Online 3D panoptic:

% PanopticFusion: (IROS 2019)
% https://arxiv.org/pdf/1903.01177.pdf
%
% - most similar to ours
% - PSPNet + M-RCNN + 2D fusion
% - volumetric mapping, 
% - greedy matching with IoU -> optimal only with 0.5 threshold
% - voxel & class weighting
% - CRF regularisation
%
% - good:
%
% - bad:
%  - CRF post-processing step
%  - greedy data-association
%    - can't be tuned for lower overlap ratios -> has to have high framerate, large changes in viewpoint could break this
%    - IoU: sensitive to 2D labels projecting over object borders (CRF and voxel weighting seem to alleviate this)

% Voxblox++: (Robotics & automation letters 2019)
% https://arxiv.org/pdf/1903.00268.pdf
% https://github.com/ethz-asl/voxblox-plusplus
%
% - M-RCNN + geometric segmentation + fusion 
% - data association of geometric segments with 3D overlap (no. points inside volume), fixed threshold for min number of points
% - instance label is assigned to a segment based on highest overlap
% - only one detected segment per reference label, as in PanopticFusion and Ours
% - TSDF Integration 
%
% good: 
% - because of geometric segmentation objects with no associated semantic class can also be segmented
% bad:
% - two different object segment types -> confusing, overly complicated ?
% - quite inaccurate (fixed below)

% Reconstructing Interactive 3D Scenes by Panoptic Mapping and CAD Model Alignments (ICRA 2021)
% https://arxiv.org/pdf/2103.16095.pdf
% https://github.com/hmz-15/Interactive-Scene-Reconstruction
%
% - based heavily on Voxblox++, much more accurate
% - Scene-graph ("contact graph") for mapping object relations
% - Search & replace voxels with CAD models, with geometrical and physical constraints
% - Object 6D pose
% - Format for robot interaction
%
% - Segmentation: bilateral fusion of geomatric and semantic segments -> reduce segmentation noise compared to Voxblox++
% - Fusion: triplet count improves consistency over Voxblox++ pairwise count strategy (take semantic label into account in addition to instance and geometry)
% - Fusion: instance labels are also combined if there is enough overlap with common geometric label for long enough time
%   - this means multiple detections can match the same reference unlike ours, voxblox++ and PanopticFusion ?
%

% Panoptic-MOPE: (ROBOTICS AND AUTOMATION LETTERS 2020)
% https://ieeexplore.ieee.org/stamp/stamp.jsp?tp=&arnumber=8977356
% https://github.com/hoangcuongbk80/Object-RPE/tree/panoptic-mope
%
% - novel RGB-D semantic segmentation model + M-RCNN
% - camera tracking based on "addaptively weighted optimization of geometric, appearance, and semantic cues"
% - surfel map: 
%   - how does it scale ? authors satate they tested on room-sized environments, but could be applied in larger scale as well ...
%     - could maybe be applied as VO in a SLAM algorithm ...
%   - demo only on a small pallet + surroundings, might not be applicable in large-scale SLAM

% US VS THEM:
%
% - based heavily on PanopticFusion, with modifications:
%   - instead of greedy data-association (which seems to be the case in others as well), we solve LAP (JPDA?)
%     - overlap threshold can be tuned, which renders the algorithm more flexible
%     - could be extended to dynamic tracking ?
%   - multiple options for association likelihood
%   - outlier rejection (either clustering or probabilistic)
%   - test different options for decreasing processing time
%   - no post-processing
%
% - model-agnostic:
%   - completely separated from segmentation
%   - does not care how point clouds are obtained -> applicable for LIDAR segmentation (e.g. EfficientLPS) as well
%
% - also agnostic to localisation method
%   - could, however, be utilised to find landmark locations / poses

% More compact version of this paragraph to introduction to save space?
%Panoptic segmentation -- proposed in \cite{panoptic_segmentation} -- aims to solve the unified task of semantic- and instance segmentation. Semantic classes are separated to \textit{stuff} -- amorphous, unquantifiable regions like sky, road or floor -- and \textit{things} -- quantifiable objects. The distinction between the two can vary depending on the application, but a semantic class can only belong to one or another. The article also proposes a unified panoptic evaluation metric, coined \textbf{Panoptic Quality} (PQ). Many 2D approaches to panoptic segmentation -- \textit{e.g.} \cite{panopticfpn,seamless,panoptic_deeplab,efficientps} -- have since been proposed. Deep neural networks for performing semantic- or instance segmentation directly on the 3D reconstruction -- \textit{e.g.} on \cite{scannet,s3dis,paris_lille_3d} -- have also been proposed, but since they require the reconstructed 3D scene, they are mostly offline approaches and therefore out of scope for this work. Some recent works also apply panoptic segmentation to point clouds -- \textit{e.g.} methods in the SemanticKITTI panoptic segmentation competition \cite{semantic_kitti} -- mostly aimed at segmenting LiDAR output. They are suitable for online processing, but similar to RGB-D images require a method for tracking object instances persistent in both time and space. In fact, our proposed method, as well as some others mentioned in this work, could use segmented LiDAR point clouds as an input similarly to RGB-D images.

PanopticFusion \cite{panopticfusion} is the first work to propose online integration of panoptic image segmentations to a 3D reconstruction. They integrate point clouds generated from segmented images to a TSDF voxel volume \cite{tsdf,voxblox} by greedily matching detected segments with the reconstruction and regulating each voxel's corresponding instance with a weighting function. Semantic labels are inferred in a bayesian manner based on confidence scores provided by the segmentation model. They also apply a Conditional Random Field (CRF) to regularise the reconstruction, improving results significantly. Voxblox++ \cite{voxblox++} -- introduced later the same year -- is a similar approach that also integrates segmented RGB-D images into a TSDF volume. It leverages geometric segmentation of depth images to improve instance segmentation accuracy. Both geometric and semantic segments are used to compute a pair-wise weight, which is used to greedily match them with segments in the reconstruction. Because of the geometric segmentation, the method allows segmentation of objects with no known semantic class in addition to objects recognised by the instance segmentation model. 

Recently, \cite{interactive_3d_scenes} built upon the idea of Voxblox++. They apply Voxblox++ for 3D instance integration, with two small but effective modifications: the pair-wise weight is replaced by a triplet weight that also takes semantic labels into account in the fusion, and -- in addition to geometric segments -- instance segments are fused if they overlap by a significant amount. The article introduces a method for searching and aligning CAD models to reconstructed objects based on geometry and semantic class, as well as geometrical and physical rules. With the CAD models, a contact graph and interactive virtual scene are reconstructed to allow a robot to simulate its interaction with the environment. SceneGraphFusion \cite{scenegraphfusion} is another approach that forms a scene graph online from a stream of RGB-D images, but unlike the above-mentioned approach, it generates the graph with a deep neural network, after which the panoptic labels for geometrically segmented portions of the 3D reconstruction are produced a side product.

Panoptic-MOPE \cite{panoptic_mope} is another recent approach, which integrates sequences of RGB-D images into a surfel reconstruction. Unlike other mentioned approaches -- which assume the camera pose either known or estimated elsewhere -- it also tracks camera movements based on geometric-, appearance- and semantic cues. The method also applies a novel RGB-D panoptic segmentation model. Although it is only tested on room-sized environments, the authors claim it could be scaled to larger environments as well.
\section{Modelling}
\label{sec:modelling}

The linear coefficients in (1) are tailored to assessing only linear mediating dependencies. To overcome this
limitation, this work considers a non-linear model by introducing a set of node dependent nonlinear functions $\{f_i\}_{i=1}^{N}$. 
Previous works on nonlinear topology identification \cite{shen2019nonlinear,tank2017interpretable,money2021online} estimate nonlinear multivariate models without necessarily assuming linear dependencies in an underlying space; rather, they directly estimate non-linear functions from and into the real measurement space without assuming an underlying structure. In our work, we assume that the multivariate data can be explained as the nonlinear output of a set of observation functions $\{f_i\}_{i=1}^{N}$ with a VAR process as an input. Each function $f_i$ represents a different non-linear distortion at the $i$-th node.

Given data time series, the task is to jointly learn the non-linearities together with a VAR topology in a feature space which is linear in nature, where the outputs of the functions $\{f_i\}_{i=1}^{N}$ belong to. Such functions are required to be invertible, so that sensor measurements can be mapped into the latent feature space, where the linear topology (coefficients) can be used to generate predictions, which can be taken back to the real space through $\{f_i\}_{i=1}^{N}$. In our model, prediction involves the composition of several functions, which can be modeled as neural networks. The nonlinear observation function at each node can be parameterized by a NN that is in turn a universal function approximator \cite{cybenko1989approximation}. 
%\textcolor{red}{ADD REFERENCE TO UNIVERSAL REPRESENTATION THEOREM}. 
Consequently, the topology and non-linear per-node transformations can be seen in aggregation as a DNN, and its parameters can be estimated using appropriate deep learning techniques.

\begin{figure}[h]
\vspace{-0.6cm}
\centering
\includegraphics[width=0.8\textwidth]{figures/lowres/circle_only.jpg}
\vspace{-0.5cm}
\caption{{Causal dependencies among a set of time series are linear in the latent space represented by the green circle. However, the variables in the latent space are not available, only nonlinear observations (output of the functions $f_i$) are available.}\vspace{-0.3cm}} 
\label{fig:swn3}
\end{figure}


The idea is illustrated in Figure \ref{fig:swn3}. The green circle represents the underlying latent vector space. The exterior of the circle is the space where the sensor measurements lie, which need not be a vector space. The blue lines show the linear dependency between the time series inside the latent space. The red line from each time series shows the transformation to the measurement space. Each sensor is associated with a different nonlinear function. Specifically, if $y_i[t]$ denotes the $i$-th time series in the latent space, the measurement (observation) is modeled as $z_{i}[t]=f_{i}\left(y_{i}[t]\right)$. The function $f_i$  is parameterized as a neural network layer with $M$ units, expressed as follows:
\begin{align} 
\label{eq:f_model}
    f_{i}\left(y_{i}\right)=\sum_{j=1}^{M} \alpha_{ij} \sigma\left(w_{ij}y_{i}-k_{ij}\right)+b_{i}
\end{align}

For the function $f_i$ to be monotonically increasing (which guarantees invertibility), it suffices to ensure that $\alpha_{ij}$ and  $w_{ji}$ are positive $\forall j$. The pre-image of $f_i$ is the whole set of real numbers, but the image is an interval $(\ubar{z}_i, \bar{z}_i)$, which is in accordance ro the fact that sensor data are usually restricted to a dynamic range. If the range is not available a priori but sufficient data is available, bounds for the operation interval can be easily inferred.

Let us remark three important advantages in the proposed model: 
\begin{itemize}
    \item It is substantially more expressive than the linear model, while capturing non-linear dependencies with lower complexity than other non-linear models. 
    \item It allows to predict with longer time horizons ahead within the linear latent space. Under a generic non-linear model, the variance of a long-term prediction explodes with the time horizon.
    \item Each non-linear nodal mapping can also adapt and capture any possible drift or irregularity in the sensor measurement, thus, it can directly incorporate imperfections in the sensor measurement itself due to, e.g. lack of calibration.    
\end{itemize}

\subsection{Prediction}

\label{sec:prediction}

Given accurate estimates of the nonlinear functions $\{f_i\}_{i=1}^N$, their inverses, and the parameters of the VAR model, future measurements can be easily predicted. Numerical evaluation of the inverse of $f_i$ as defined in \eqref{eq:f_model} can easily be done with a bisection algorithm. 

Let us define $g_i = f_i^{-1}$. Then, the prediction consists of three steps, the first one being mapping the previous samples back into the latent vector space:
\begin{subequations}
    \begin{equation} \label{eq:forward_g}
        \tilde{y}_{i}[t-p] =g_{i}\left(z_{i}[t-p]\right)
    \end{equation}
Then, the VAR model parameters are used to predict the signal value at time t (also in the latent space):
    \begin{equation} \label{eq:forward_A}
        \hat{y}_{i}[t] =\sum_{p=1}^{p}\sum_{j=1}^{n}  a_{i j}^{(p)} \tilde{y}_{j} [t-p] 
    \end{equation}
Finally, the predicted measurement at each node is obtained by applying $f_i$ to the latent prediction:
    \begin{equation} \label{eq:forward_f}
        \hat{z}_{i}[t] =f_{i}\left(\hat{y}_{i}[t]\right) 
    \end{equation}
\end{subequations}

\begin{figure}[t]
\vspace{-1.7cm}
\hspace{-1.2cm}
\vspace{-0cm}
\includegraphics[width=1.2\textwidth]{figures/network.png}
\vspace{-2.8cm}
\caption{{ Schematic for modeling Granger causality for a toy example with 2 sensors. }} 
\label{fig:example_2sensors}
\end{figure}

These prediction steps can be intuitively visualized as a neural network. The next section formulates an optimization problem intended to learn the parameters of such a neural network. For a simple example with 2 sensors, the network structure is shown in Figure \ref{fig:example_2sensors}.

\section{Problem formulation}

The functional optimization problem consists in minimizing $\|{z}[t]-\hat z\|_{2}^{2}$ (where $z[t]$ is a vector collecting the measurements for all N sensors at time $t$), subject to the constraint of $f_i$ being invertible $\forall i$, and the image of $f_i$ being $(\ubar{z}_i, \bar{z}_i)$. The saturating values can be obtained from the nominal range of the corresponding sensors, or can be inferred from data. 
 
 Incorporating equation (1),  the optimization problem can be written as:
\begin{subequations}
\label{eq:optimization_problem}
\begin{align}
    \min _{f, A} \;\;& \| {z}[t]-f\Big(\sum_{p=1}^{p} A^{(p)}\big[g(z[t-p])\big]\Big) \|_{2}^{2} 
    \\
    \textrm{s. to:}\;\  
    & \sum_{j=1}^{M}\alpha_{j i} 
      = \bar{z}_i %max(z_{i}[t-p])
        - \ubar{z}_i %min(z_{i}[t-p]) 
        \; \forall i
        \label{eq:constraint_alpharange}
    \\
    & b_{i} = \bar{z}_i%min(z_{i}[t-p]) 
        \; \forall i
        \label{eq:constraint_barz}
    \\
    & \alpha_{j i} \geq0 
        \; \forall i, j
        \label{eq:constraint_alphapos}
        \\
    & w_{j i} \geq0 
        ; \forall i, j
        \label{eq:constraint_wpos}
\end{align}
 \end{subequations}
% \textcolor{red}{comment about the max and min saying that the function f saturates at z bar and  below if you know a priori you impose  directly or else infer from the data}     
         
The functional optimization over $f_i$ is tantamount to optimizing over $\alpha_{j i}$,$w_{j i}$,  $k_{j i}$ and $b_{i}$. The main challenge to solve this problem is that there is no closed form for the inverse function $g_i$. This is addressed in the ensuing section.

\iffalse
\begin{lstlisting}
// Algorithm for function g implementation
def g(x,i):
  niter = 10000
  vy = 0
  for j in range(niter):
    vy = vy - (ginverse(vy,i)-x)/gprime(vy,i)  
  return vy
def gprime(x,i): 
  a=0.01
  for j in range(m):
            a = a + alpha[j][i] * sigmoid(x-k[j][i]) * (1-sigmoid(x-k[j][i])) 
  return a
def ginverse(x,i): 
  a = 0   
  for j in range(m):
      a = a + alpha[j][i] * sigmoid(x-k[j][i]) + b[i][0]
  return a
\end{lstlisting}
\fi



%\subsection{Implementation as a neural network}



\section{Learning algorithm}

Without a closed form for $g$, we cannot directly obtaining gradients with automatic differentiation such as Pytorch,%\footnote{One could try to differentiate through all iterations of the Newton method that numerically inverts $f_i$, but there is no guarantee of obtaining a stable gradient estimate, and the computational cost is probably high.}
as is typically done in deep learning with a stochastic gradient-based optimization algorithm. Fortunately, once $\{g_i(\cdot)\}$ is numerically evaluated, the gradient at that point can be calculated with a relatively simple algorithm, derived via implicit differentiation in Sec. \ref{ss:backprop}. Once that gradient is available, the rest of the steps of the backpropagation algorithm are rather standard.

\subsection{Forward equations}
The forward propagation equations are given by the same steps that are used to predict next values of the time series $z$:

\begin{subequations}
    \begin{equation} \label{eq:forward_g}
        \tilde{y}_{i}[t-p] =g_{i}\left(z_{i}[t-p], \theta_{i}\right)
    \end{equation}
    \begin{equation} \label{eq:forward_A}
        \hat{y}_{i}[t] =\sum_{p=1}^{p}\sum_{j=1}^{n}  a_{i j}^{(p)} \tilde{y}_{j} [t-p] 
    \end{equation}
    \begin{equation} \label{eq:forward_f}
        \hat{z}_{i}[t] =f_{i}\left(\hat{y}_{i}[t], \theta_{i}\right) 
    \end{equation}
    \begin{equation} \label{eq:forward_cost}
        C[t] =\sum_{n=1}^{N}\left(z_{n}[t]-\hat{z}_{n}[t]\right)^{2}_\cdot
    \end{equation}
\end{subequations}
Here, the dependency of the nonlinear functions with the neural network parameters is made explicit, where $$\theta_{i}=\left[\begin{array}{l}\alpha_{ i} \\  w_i\\ k_{ i} \\ b_{i}\end{array}\right] \text{ and } 
\alpha_{i}=\left[\begin{array}{c}
\alpha_{i 1} \\
\alpha_{i 2} \\
\vdots \\
\alpha_{i M}
\end{array}\right],w_{i}=\left[\begin{array}{c}
k_{i 1} \\
k_{i 2} \\
\vdots \\
k_{i M}
\end{array}\right], k_{i}=\left[\begin{array}{c}
k_{i 1} \\
k_{i 2} \\
\vdots \\
k_{i M}
\end{array}\right]_. $$

\subsection{Backpropagation equations}
\label{ss:backprop}
The goal of backpropagation is to  calculate the gradient of the cost function with respect to the VAR parameters and the node dependent function parameters $\theta_i$.

%The dimension of the parameter vector $\theta_{i}$ is for the i th sensor  will be (2M+1). 
The gradient of the cost is obtained by applying the chain rule as following:
%\setcounter{equation}{9}
\begin{equation} \label{chainrule1}
\begin{array}{c}

\frac{d C[t]}{d \theta_{i}}=\sum_{n=1}^{N} \frac{\partial C}{\partial \hat{z}_{n}[t]}  \frac{\hat{z}_{n}[t]}{\partial \theta_{i}} \\
\text { where } \frac{\partial C}{\partial \hat{z}_{n}[t]}=2(\hat{z}_n[t]-z_n[t]) = S_n
\end{array}
\end{equation}


\begin{equation} \label{chainrule2}
\frac{\partial \hat{z}_{n}[t]}{\partial \theta_{i}}=\frac{\partial f_{n}}{\partial \hat{y}_{n}}  \frac{\partial \hat{y}_{n}}{\partial \theta_{i}}+\frac{\partial f_{n}}{\partial \theta_{n}}  \frac{\partial \theta_{n}}{\partial \theta_{i}} \\
\quad 
\end{equation}
$$
\text { where } \frac{\partial \theta_{n}}{\partial \theta_{i}}=\left\{
\begin{array}{l}
I, n = i \\
0, n  \neq i
\end{array}\right.
$$
Substituting equation \eqref{chainrule1} into \eqref{chainrule2} yields
\begin{equation} \label{derivativecost1}
\frac{d C[t]}{d \theta_{i}}=\sum_{n=1}^{N} S_{n}\left(\frac{\partial f_{n}}{\partial \hat{y}_{n}}  \frac{\partial \hat{y}_{n}}{\partial \theta_{i}}+\frac{\partial f_{n}}{\partial \theta_{n}}  \frac{\partial \theta_{n}}{\partial \theta_{i}}\right)_\cdot
\end{equation}
Equation\eqref{derivativecost1} can be simplified as:

\begin{equation} \label{derivativecost2}
    \frac{d C[t]}{d \theta_{i}}
    =
    S_{i}
    \frac{\partial f_{i}}{\partial \theta_{i}}
    +\sum_{n=1}^{N} S_{n}\frac{\partial f_{n}}{\partial \hat{y}_{n}}  \frac{\partial \hat{y_n}}{\partial \theta_{i}}.
\end{equation}


$\text { The next step is to derive } \frac{\partial \hat{y}_{n}}{\partial \theta_{i}} \text { and } \frac{\partial f_{i}}{\partial \theta_{i}}$ of  equation \eqref{derivativecost2}:


\begin{equation} \label{gradientyhat}
    \frac{\partial \hat{y}_{n}[t]}{\partial \theta_i}
    =
    \sum_{p=1}^{P}\sum_{j=1}^{N}  
        a_{n j}^{(p)} 
        \frac{\partial}{\partial \theta_{j}} \tilde{y}_{j}[t-p]
        \frac{\partial \theta_{j}}{\partial \theta_{i}} .
\end{equation}

With $f_{i}^{\prime}\left(z\right)= 
\frac{\partial f_i\left(z, \theta_{i}\right)}{\partial\left(z\right)}, $  
expanding $\tilde{y}_j[t-p]$ in equation \eqref{gradientyhat} changes \eqref{derivativecost2} to:

%\begin{equation} \label{derivativecost3}
%\text { So } \frac{d C[t]}{d \theta_{i}}=S_{i}\left(\frac{\partial f_{i}}{\partial \theta_{i}}\right)+\sum_{p=1}^{P} \sum_{n=1}^{N} S_{n} f_{n}^{\prime}(\hat{y}_n[t])  a_{n i}^{(p)} \frac{\partial}{\partial \theta_{i}} g_{i}\left(z_{i}[t-p],\theta_{i}\right)_\cdot
%\end{equation}

%equation \eqref{derivativecost3} becomes:

\begin{equation} \label{derivativecost4}
 \frac{d C[t]}{d \theta_{i}}=S_{i}\left(\frac{\partial f_{i}}{\partial \theta_{i}}\right)+\sum_{n=1}^{N} S_{n}\left(f_{n}^{\prime}(\hat{y}_{n}[t]) \sum_{p=1}^{P} a_{n i}^{(p)} \frac{\partial}{\partial \theta_{i}} g_{i}\left(z_{i}[t-p],\theta_{i}\right)\right)_\cdot
\end{equation}

Here, the vector 
$$\frac{\partial f_i\left(z, \theta_{i}\right)}{\partial \theta_{i}} %\text{ in equation } \eqref{derivativecost4} 
= \left[
\frac{\partial f_i\left(z, \theta_{i}\right)}{\partial \alpha_{i}} 
\frac{\partial f_i\left(z, \theta_{i}\right)}{\partial w_{i}} 
\frac{\partial f_i\left(z, \theta_{i}\right)}{\partial k_{i}} 
\frac{\partial f_i\left(z, \theta_{i}\right)}{\partial b_{i}} \right]$$ can be obtained by standard or automated differentiation via, e.g., Pytorch \cite{NEURIPS2019_9015}.


However, \eqref{derivativecost4} involves the calculation of $\frac{\partial g_i(z, \theta_{i})}{\partial \theta_{i}}$, which is not straightforward to obtain. Since $g_i(z)$ can be computed numerically, the derivative can be obtained by implicit differentiation, realizing that the composition of $f_i$ and $g_i$ remains invariant, so that its total derivative is zero:

\begin{equation} \label{dfwrttheta1}
\frac{d}{d \theta_{i}}\left[f_i\left(g_i\left(z, \theta_{i}\right), \theta_{i}\right)\right]=0
\end{equation}

\begin{equation} \label{dfwrttheta2}
\Rightarrow \frac{\partial f_i\left(g_i\left(z, \theta_{i}\right), \theta_{i}\right)}{\partial g\left(z, \theta_{i}\right)} \frac{\partial g\left(z, \theta_{i}\right)}{\partial \theta_{i}}+\frac{\partial f_i\left(z, \theta_{i}\right)}{\partial \theta_{i}}=0
\end{equation}

\begin{equation} \label{dfwrttheta3}
\Rightarrow {f^{\prime}_i(g_i(z,\theta_{i}))} \frac{\partial g\left(z, \theta_{i}\right)}{\partial \theta_{i}}+\frac{\partial f_i\left(z, \theta_{i}\right)}{\partial \theta_{i}}=0
\end{equation}


\begin{equation} \label{dgwrttheta1}
\text { Hence } \frac{\partial g_i\left(z, \theta_{i}\right)}{\partial \theta_{i}}=
-\big\{f^{\prime}_i(g_i(z,\theta_{i}))\big\}^{-1}{\left(\frac{\partial f_i\left(z, \theta_{i}\right)}{\partial \theta_{i}}\right)}_\cdot 
\end{equation}
%
%$$\text{where }\frac{\partial g_i\left(z, \theta_{i}\right)}{\partial \theta_{i}} =  \left[\begin{array}{l}
%\frac{\partial g_i\left(z, \theta_{i}\right)}{\partial \alpha_{i}} \\
%\frac{\left.\partial g_i\left(z\right, \theta_{i}\right)}{\partial k_{i}} \\
%\frac{\partial g_i(z,\theta_{i})}{\partial b_{i}}
%\end{array}\right]_\cdot$$
%

The gradient of $C_T$ w.r.t. the VAR coefficient $a^{(p)}_{ij}$ is calculated as follows: 

\begin{equation} \label{dCwrtthetaa}
\frac{d C[t]}{d a^{(p)}_{i j}}=\sum_{n=1}^{N} S_{n}  \frac{\partial f_{n}}{\partial \hat y_{n}}  \frac{\partial \hat{y}_{n}}{\partial a_{i j}^{(p)}}
\end{equation}

\begin{equation} \label{dCwrtthetaa1} \nonumber
\frac{\partial \hat{y}_{n}[t]}{\partial a_{i j}^{(p)}}=\frac{\partial}{\partial a_{i j}^{(p)}} \sum_{p^\prime=1}^{P} \sum_{q=1}^{N} a_{n q}^{(p^\prime)} \tilde{y}_{q}[t-p]
\end{equation}

\begin{equation} \label{dCwrtthetaa2}
\begin{array}{c}
\text { where } 
\frac{\partial a_{n q}^{(p^\prime)}}{\partial a_{i j}^{(p)}}=\left\{\begin{array}{l}
1, n=i, p = p^\prime, \text { and } q=j \\
0, \text {otherwise}
\end{array}\right.
\end{array}
\end{equation}

%\begin{equation} \label{dCwrtthetaa3}
%\frac{\partial \hat{y}_{n}[t]}{\partial a_{i j}^{(p)}}=\left\{\begin{array}{c}
%\sum_{p=1}^{P} \tilde{y}_{j}[t-p], i=n \\
%0, \quad i \neq n
%\end{array}\right.
%\end{equation}

\begin{equation} \label{dCwrtthetaa4}
\frac{d C[t]}{d a_{i j}^{(p)}}=S_i f_{i}^{\prime}\left(\hat{y}_{i}[t]\right) \tilde{y}_{j}[t-p]_\cdot 
\end{equation}

Even though the backpropagation cannot be done in a fully automated way, it can be realized by implementing equations \eqref{dgwrttheta1} and \eqref{derivativecost4} after automatically obtaining the necessary expressions.

\subsection{Parameter optimization}

The elements in $\{A^{(p)}\}_{p=1}^P,$ and $\{\theta_i\}_{i=1}^{N}$ can be seen as the parameters of a NN. Recall from Fig. \ref{fig:example_2sensors} that the prediction procedure resembles a typical feedforward NN as it interleaves component-wise nonlinearities with multidimensional linear mappings. The only difference is that one of the layers computes the inverse of a given function, and its backward step has been derived. Moreover, the cost function in \eqref{eq:optimization_problem} is the mean squared error (MSE).

The aforementioned facts support the strategy of learning the parameters using state-of-the-art NN training techniques. A first implementation has been developed using stochastic gradient descent (SGD) and its adaptive-moment variant Adam \cite{kingma2014adam}. Constraints \eqref{eq:constraint_alpharange}-\eqref{eq:constraint_wpos} are imposed by projecting the output of the optimizer into the feasible set at each iteration.

The approach is flexible enough to be extended with neural training regularization techniques such as dropout %\cite{dropout} 
or adding a penalty based on the L1 or L2 norm of the coefficients, to address the issue of over-fitting and/or promote sparsity. The batch normalization technique can be proposed to improve the training speed and stability. %Last but not least, future developments include improving the interpretability by imposing sparsity. 

\iffalse
\begin{lstlisting}
// Algorithm for back propagation implementation
def backward_propagation(H,alpha,b,A,k,X_train, z_pred):
for i in range(n):
          for p in range(n):
              a1=0
              for j in range(len(X_train)):
                  a1  = a1-2*(X_train[j][p] - z_pred[j][p])*(g(X_train[j][i],i))*(f(H[j][p],p))
              dA[i][p] =a1
    for i in range(n): 
        for p in range(n):                        
            for j in range(len(X_train)):
                dalpha[j][i] = dalpha[j][i] -2*(X_train[j][i] - z_pred[j][i])  *(f(H[j][p],p)*A[i][p]*dalphag(X_train[j][i],i)) -2*(X_train[j][i] - z_pred[j][i])*sigmoid(H[j][i]-k[j][i])
    for i in range(n): 
        for p in range(n):                       
            for j in range(len(X_train)):
                dk[j][i] = dk[j][i] -2*(X_train[j][i] - z_pred[j][i]) * (f(H[j][p],p)*A[i][p]*dkg(X_train[j][i],i)) -2*(X_train[j][i] - z_pred[j][i])*alpha[j][i]*sigmoid(H[j][i]-k[j][i])
                *(1-sigmoid(H[j][i]-k[j][i]))
    for i in range(n): 
        a1 = 0
        for p in range(n):                        
            for j in range(len(X_train)):
                a1  = a1 -2*(X_train[j][i] - z_pred[j][i])  *(f(H[j][p],p)*A[i][p]*dbg(X_train[j][i],i)) -2*(X_train[j][i] - z_pred[j][i])*alpha[j][i]*sigmoid(H[j][i]-k[j][i])
        a1 = db[i][0]            
    return 0

\end{lstlisting}
\fi

\section{Experiments}\label{sec:experiments}
We validate our approach using multiple datasets containing real-life data from the fields of criminal risk assessment, credit, lending, and college admissions. In each of the datasets we select a binary feature and treat it as the protected attribute (e.g., race or gender), which is the feature we require our trained classifier to behave fairly upon. Our proposed method performs well on all of these datasets, succeeding in removing unfairness almost entirely, at a very modest price in terms of accuracy.


\begin{table*}[h]
\centering
\resizebox{\textwidth}{!}{
\def\arraystretch{1.2}

\begin{tabular}{c c c | c | c | c || c | c | c || c | c | c |}

\cline{4-12}
&&&
\multicolumn{9}{ c| }{\textbf{COMPAS Dataset}}
\\ \cline{4-12}
&&&
\multicolumn{3}{ c|| }{\textbf{FPR Considerations}}&
\multicolumn{3}{ c|| }{\textbf{FNR Considerations}}&
\multicolumn{3}{ c| }{\textbf{Both Considerations}}
\\ \cline{4-12}
&&&
 $\mathbf{Acc.}$ &  $\mathbf{D_{FPR}}$ &  $\mathbf{D_{FNR}}$ &  $\mathbf{Acc.}$ &  $\mathbf{D_{FPR}}$ &  $\mathbf{D_{FNR}}$ &  $\mathbf{Acc.}$ &  $\mathbf{D_{FPR}}$ &  $\mathbf{D_{FNR}}$
\\  \cline{4-12}
\vspace*{-0.5ex}
\\ \cline{1-2} \cline{4-12}
\multicolumn{1}{ |c  }{} &
\multicolumn{1}{ c|  }{  \textbf{Our Method (AVD Penalizers)}}  &&
$\mathbf{0.660}$    &  $\mathbf{0.01}$  &  $0.04$ &
$\mathbf{0.653}$    &  $0.02$   &  $\mathbf{0.04}$ &
$\mathbf{0.654}$    &  $\mathbf{0.02}$  &  $\mathbf{0.04}$
\\ \cline{1-2} \cline{4-12}
\multicolumn{1}{ |c  }{} &
\multicolumn{1}{ c|  }{  \textbf{Our Method (SD Penalizers)}}  &&
$\mathbf{0.664}$    &  $\mathbf{0.02}$  &  $0.09$ &
$\mathbf{0.661}$    &  $0.05$   &  $\mathbf{0.03}$ &
$\mathbf{0.661}$    &  $\mathbf{0.02}$  &  $\mathbf{0.03}$
\\ \cline{1-2} \cline{4-12}
\multicolumn{1}{ |c  }{} &
\multicolumn{1}{ c|  }{  Zafar et al.~(\citeyear{disparatemistreatment})}  &&
$0.660$    &   $0.06$    &   $0.14$  &
$0.662$    &   $0.03$    &   $0.10$  &
$0.661$    &   $0.03$    &   $0.11$
\\ \cline{1-2} \cline{4-12}
\multicolumn{1}{ |c  }{} &
\multicolumn{1}{ c|  }{  Zafar et al. Baseline~(\citeyear{disparatemistreatment})}  &&
$0.643$    &   $0.03$    &   $0.11$  &
$0.660$    &   $0.00$    &   $0.07$  &
$0.660$    &   $0.01$    &   $0.09$
\\ \cline{1-2} \cline{4-12}
\multicolumn{1}{ |c  }{} &
\multicolumn{1}{ c|  }{  Hardt et al.~(\citeyear{hardt})}  &&
$0.659$    &  $0.02$    &   $0.08$  &
$0.653$    &  $0.06$   &    $0.01$  &
$0.645$    &  $0.01$   &    $0.01$
\\ \cline{1-2} \cline{4-12}
\multicolumn{1}{ |c  }{} &
\multicolumn{1}{ c|  }{  \textbf{Vanilla Regularized Logistic Regression}}  &&
$\mathbf{0.672}$    &   $\mathbf{0.20}$    &   $\mathbf{0.30}$  &
$\mathbf{0.672}$    &   $\mathbf{0.20}$    &   $\mathbf{0.30}$  &
$\mathbf{0.672}$    &   $\mathbf{0.20}$    &   $\mathbf{0.30}$
\\ \cline{1-2} \cline{4-12}
\end{tabular}
}
\vspace{3mm}
\caption{Performance comparison on the COMPAS dataset. For the approaches in bold -- Accuracy, FPR difference and FNR difference are evaluated on the test set, averaging over five runs and using a 70-30 training/test split. The performance of the remaining three approaches is stated as reported in Zafar et al.~(\citeyear{disparatemistreatment}).} \label{table:comparison_results}
\end{table*}



\begin{figure*}[b]
  \includegraphics[scale=0.6]{compas0-400.png}
  \caption{COMPAS Dataset. Accuracy, FPR difference ($\mathbf{D_{FPR}}$), and FNR difference ($\mathbf{D_{FNR}}$) (all evaluated on the test set) of the learned classifier, as a function of the weight $c=c_1 = c_2 \geq 0$ placed on the fairness penalizer terms. On the left we use the Absolute Value Difference (AVD) penalizer, and the Squared Difference (SD) penalizer on the right, both as presented in Section~\ref{regularization}. ``Relaxed FPR/FNR Diff.'' plots the value of the relevant penalization term.} %In this particular run, parameters chosen for the absolute value relaxation were: $c=80, q_c=60$, and for the squared relaxation: $c=220, q_c=30$.}
  \label{fig:compas}
\end{figure*}


\subsection{Implementation}
\textbf{Our method} 
%We instantiate our method in the following way: Given dataset $Q$, we split it randomly into a training set $S$ (which we will use for learning) and a test set $T$ (which we will only use for reporting performance). 
For the purpose of comparison with  Zafar et al.~(\citeyear{disparatemistreatment}) and Hardt et al.~\cite{hardt} on the COMPAS data, we use a parameter $c$ to induce three possible combinations of weights on the FPR and FNR penalization terms: $c = c_1$ and $c_2 = 0$; $c_1 = 0$ and $c = c_2$; and $c = c_1 = c_2$. For the other three datasets, we consider only $c = c_1 = c_2$.\footnote{The reason for varying the values of $c$ in the training phase is since we shifted to a proxy problem, in which we rely on the distance from the decision boundary rather the actual classifications. 
%Our hope is that there is no need for a worst-case cross validation between all of the combinations of $c_1, c_2, c_3$, and that the training scheme we propose is sufficient. 
It is possible, of course, that even better results are attainable using our scheme with other combinations of $c_1, c_2$, and $q$.} To explore the accuracy/fairness trade-off curve for the relaxed optimization problem~(\ref{eq:2}), we train for different values of $c$, starting at $c=0$ (which is just standard logistic regression), and growing gradually.



Given a dataset $Q$ and fixing a $d_1, d_2 \in \{0, 1\}$ of interest, we use the following training scheme:
\begin{enumerate}
\item Split $Q$ at random into training set $S$ and test set $T$.
\item For each $c$, perform cross-validation on $S$ to select the corresponding best value $q_c$ for the regularization parameter.
\item For each $(c,q_c)$, let $\theta_c = \argmin\limits_{\theta} \text{Proxy}(\theta;S,c,c,q_c)$.
\item Select $\theta^* \in \argmin\limits_{\theta_c} \text{Objective}(\theta_c;S,d_1,d_2)$.
\item Evaluate performance using $\theta^*$ on test set $T$.
\end{enumerate}
We report the average of five such runs, each with a fresh training-test split.




%We instantiate our method by solving the relaxed optimization problem~(\ref{eq:2}), in place of the original, non-convex problem~(\ref{eq:1}).  
%We test our approach with three different combinations of weights on the penalization terms:
%\katrina{What are the $d$, and how are they related to the $c$s?}
%\begin{enumerate}
%\item FPR considerations only: $d_1 = 1, d_2 = 0$.
%\item FNR considerations only: $d_1 = 0, d_2 = 1$.
%\item Both FPR, FNR considerations, assigned similar significance: $d_1 = 1, d_2 = 1$.
%\end{enumerate}
%One could, of course, pick any other combination of the FPR and FNR penalty weights.

%\katrina{I don't understand how the below is distinct from the list above}
%Learning is done by training the parameters of a logistic regressor to solve~\ref{eq:2}, while picking the value of $c_1, %c_2$ as the following:
%\begin{enumerate}
%\item FPR considerations only: $c_1 = c \geq 0$, $c_2 = 0$.
%\item FNR considerations only: $c_1 = 0$, $c_2 = c \geq 0$.
%\item Both FPR, FNR considerations, assigned similar significance: $c_1 = c_2 = c \geq 0$
%\end{enumerate}



% We then cross-validate to pick the best $c_3$ (the weight on the standard $\ell_2$-regularization term) given $c$.\footnote{The reason for varying the values of $c$ in the training phase is since we shifted to a proxy problem, in which we rely on the distance from the decision boundary rather the actual classifications. 
%Our hope is that there is no need for a worst-case cross validation between all of the combinations of $c_1, c_2, c_3$, and that the training scheme we propose is sufficient. 
%It is possible, of course, that even better results are attainable using our scheme with other combinations of $c_1, c_2, c_3$.} For each such combination, we report results as the averages of multiple \katrina{how many?} different runs, each time splitting data randomly into training and test sets.
%\yahav{We need to shorten this description.}

We solve the relaxed convex optimization problem using the CVXPY solver. Due to stability issues with large training sets, we use a train/test split of 30-70 on the larger datasets, rather than 70-30 as on the COMPAS dataset\footnote{The code implementing our method can be found at https://github.com/jjgold012/lab-project-fairness}.

%
%
%We then report the results (as evaluated on the test set) attained by a regressor $\theta \in \mathbb{R}^d$ that minimizes (on the training set $S$) a weighted combination of the $0$-$1$ loss and the differences in FPR and FNR across populations:
%\begin{equation*}
%\begin{aligned}
%&\underset{\theta}{\text{argmin}}
%& & L_{S}^{0\text{-}1}(\theta) \\
%&&& + d_1|FPR_{A=0}(\theta;S)-FPR_{A=1}(\theta;S)| \\
%&&& + d_2|FNR_{A=0}(\theta;S)-FNR_{A=1}(\theta;S)|
%\end{aligned}
%\end{equation*}
%
%\katrina{What is $d_1$ vs. $c_1$ etc.?}



%For classification, we decided use a standard cut-off threshold of $c=0.5$. There are of course, further possible interactions between the FPR, FNR considerations, and picking a certain cut-off level. These are not straightforward, since  these interactions are data-specific. 



%allows for flexibility in picking the values of $c_1, c_2$, which reflect the significance we wish to place on the objectives of achieving accuracy, equal FPR, and equal FNR. As for $c_3$, we will want to find the value of it that achieves the best results, for any combined objective of accuracy and fairness defined by a specific selection of $c_1,c_2$. Therefore, given a specific selection of $c_1, c_2$, we apply cross-validation to select the value of $c_3$. 




We briefly describe the other algorithmic approaches to which we compare:\\
\textbf{Zafar et al.}~(\citeyear{disparatemistreatment}) performs optimization by considering a proxy for the bias: the covariance between the samples' sensitive attributes and the signed distance between the feature vectors of misclassified users and the classifier decision boundary.\\
\textbf{Zafar et al. Baseline}~(\citeyear{disparatemistreatment}) tries to enforce equal FP/FN rates on the different groups by introducing different penalties for misclassified data points with different sensitive attribute values during the training phase.\\
\textbf{Hardt et al.}~(\citeyear{hardt}) performs post-processing on a standard trained (unfair) logistic regressor, picking different decision thresholds for different groups, and possibly adding randomization.


\subsection{Experimental Results}

In what follows, we use the following notation, given a trained classifier $\hat{Y}$:
\begin{align*}
\mathbf{D_{FPR}}&=\left|FPR_{A=0}(\hat{Y})-FPR_{A=1}(\hat{Y})\right| \\ 
\mathbf{D_{FNR}}&=\left|FNR_{A=0}(\hat{Y})-FNR_{A=1}(\hat{Y})\right|
\end{align*}
The values $FPR_{A=0}(\hat{Y})$, $FPR_{A=1}(\hat{Y})$, $FNR_{A=0}(\hat{Y})$, $FNR_{A=1}(\hat{Y})$ are reported as evaluated on the test set.

\paragraph{The COMPAS Dataset\footnote{https://github.com/propublica/compas-analysis}} The Correctional Offender Management Profiling for Alternative Sanctions (COMPAS) records from Broward County, Florida 2013-2014, made available online by ProPublica, are perhaps the best-studied data in the context of fairness.  The goal in this scenario is to successfully predict recidivism within two years, based on features such as age, gender, race, number of prior offenses, and charge degree. The dataset contains 5,278 samples. The protected attribute in this scenario is race, where $A$ indicates black or white. We filtered the dataset using the same features as Zafar et al.~(\citeyear{disparatemistreatment}), to allow for comparison.

%\begin{table}[h]
%\centering
%\begin{tabularx}{\columnwidth}{c|c|c|c}
%\hline
%  &  Recid. ($y = 1$)        & No Recid.  ($y = 0$)       & Total \\ \hline
%Black &  $ 1661   $ & $ 1514 $ &  $ 3175 $ \\ \hline
%White &  $ 822   $  & $1281  $ &  $ 2103 $ \\ \hline
%Total &  $ 2483  $  & $2795 $ &  $ 5278 $ \\\hline
%\end{tabularx}
%\caption{Statistics of the ProPublica COMPAS data.} \label{table:compas-stats}
%\label{tab:stats}
%\end{table}
%\vspace{-1em}

%\begin{table}[h]
%\centering
%\begin{tabularx}{\columnwidth}{c|c}
%\hline
%Feature  &  Description \\ \hline
%Age Category &  $<25$, between $25$ and $45$, $>45$ \\
%Gender &  Male or Female \\
%Race &  White or Black \\
%Priors Count &  0--37 \\
%Charge Degree &  Misconduct or Felony \\
%\hline
%2-year-recid. & Whether or not the  \\
%(target feature)  & defendant recidivated within two years
%\end{tabularx}
%\caption{Description of features used from ProPublica COMPAS data.} \label{table:compas-features}
%\label{tab:features}
%\end{table}




\begin{table*}[t]
\centering
\caption{A description of the datasets used, along with parameters of the training procedure used for each.}
\label{table:datasets_description}
\begin{adjustbox}{max width=\textwidth}
\begin{tabular}{|l|l|l|l|l|l|l|l|}
\hline
\textbf{Dataset} & \textbf{No. Samples} & \textbf{No. Features} & \textbf{Train/Test Split} & \textbf{No. Repetitions} & \textbf{No. Folds in CV} & \textbf{Protected Feature} & \textbf{Target Variable} \\ \hline
COMPAS           & 5,278                     & 5                          & 70-30                     & 5                        & 5                                 & Race                       & 2-Year-Recidivism        \\ \hline
Adult            & 30,162                    & 10                         & 30-70                     & 5                        & 5                                 & Gender                     & Income Over/Under 50K    \\ \hline
Default          & 30,000                    & 23                         & 30-70                     & 5                        & 3                                 & Gender                     & Defaulting On Payments   \\ \hline
Admissions       & 20,839                    & 17                         & 30-70                     & 5                        & 3                                 & Race                       & Passing Bar Exam         \\ \hline
\end{tabular}
\end{adjustbox}
\end{table*}


\begin{table*}[t]
\centering
\resizebox{\textwidth}{!}{
\def\arraystretch{1.2}

\begin{tabular}{c c c | c | c | c || c | c | c || c | c | c |}

\cline{4-12}
&&&
\multicolumn{3}{ c|| }{\textbf{Adult Dataset}}&
\multicolumn{3}{ c|| }{\textbf{Default Dataset}}&
\multicolumn{3}{ c| }{\textbf{Admissions Dataset}}
\\ \cline{4-12}
%&&&
%\multicolumn{3}{ c|| }{\textbf{Both Considerations}}&
%\multicolumn{3}{ c|| }{\textbf{Both Considerations}}&
%\multicolumn{3}{ c| }{\textbf{Both Considerations}}
%\\ \cline{4-12}
&&&
 $\mathbf{Acc.}$ &  $\mathbf{D_{FPR}}$ &  $\mathbf{D_{FNR}}$ &  $\mathbf{Acc.}$ &  $\mathbf{D_{FPR}}$ &  $\mathbf{D_{FNR}}$ &  $\mathbf{Acc.}$ &  $\mathbf{D_{FPR}}$ &  $\mathbf{D_{FNR}}$
\\  \cline{4-12}
\vspace*{-0.5ex}
\\ \cline{1-2} \cline{4-12}
\multicolumn{1}{ |c  }{} &
\multicolumn{1}{ c|  }{  \textbf{Our Method (AVD Penalizers)}}  &&
$\mathbf{0.776}$    &  $\mathbf{0.00}$  &  $\mathbf{0.04}$ &
$\mathbf{0.807}$    &  $\mathbf{0.00}$   &  $\mathbf{0.01}$ &
$\mathbf{0.950}$    &  $\mathbf{0.01}$  &  $\mathbf{0.00}$
\\ \cline{1-2} \cline{4-12}
\multicolumn{1}{ |c  }{} &
\multicolumn{1}{ c|  }{  \textbf{Our Method (SD Penalizers)}}  &&
$\mathbf{0.783}$    &  $\mathbf{0.00}$  &  $\mathbf{0.09}$ &
$\mathbf{0.806}$    &  $\mathbf{0.01}$   &  $\mathbf{0.02}$ &
$\mathbf{0.950}$    &  $\mathbf{0.00}$  &  $\mathbf{0.00}$
\\ \cline{1-2} \cline{4-12}
\multicolumn{1}{ |c  }{} &
\multicolumn{1}{ c|  }{  \textbf{Vanilla Regularized Logistic Regression}}  &&
$\mathbf{0.800}$    &   $\mathbf{0.08}$    &   $\mathbf{0.39}$  &
$\mathbf{0.807}$    &   $\mathbf{0.01}$    &   $\mathbf{0.05}$  &
$\mathbf{0.951}$    &   $\mathbf{0.16}$    &   $\mathbf{0.02}$
\\ \cline{1-2} \cline{4-12}
\end{tabular}
}
\vspace{3mm}
\caption{Performance on the Adult, Loan Default, and Admissions datasets, penalizing for both FPR and FNR difference. Accuracy, FPR difference and FNR difference are evaluated on the test set, averaging over five runs and using a 30-70 training/test split.} \label{table:comparison_results_rest}
\end{table*}


In Table~\ref{table:comparison_results}, we compare the performance of our approach with that of three other techniques from the literature. Each method was trained based on logistic regression.  As a basis for comparison, we also present the performance of vanilla logistic regression, absent fairness considerations, with the regularization parameter selected via cross-validation.\footnote{Zafar et al.~(\citeyear{disparatemistreatment}) do not incorporate regularization in any of the approaches they report.}
%Results are reported as the averages of 5 different runs \katrina{Is that still correct?}, each time splitting data evenly and randomly into training and test sets. 
Results for Zafar et al., Zafar et al. baseline, and Hardt et al. appear here as reported in Zafar et al.~(\citeyear{disparatemistreatment}).\footnote{Our method selects the classifier based on the training set only and reports its performance over the test set. Results for the three other approaches, reported by Zafar et al.~(\citeyear{disparatemistreatment}), are based on tuning parameters after seeing the trade-off curve over the test set, and reporting according to the best selection of these parameters.}
%\katrina{Perhaps here is the right place for a footnote about the discrepancy with the Zafar baseline}

We find that the vanilla logistic regressor (absent fairness considerations) results in significant unfairness, as $\mathbf{D_{FPR}}=0.20$, and $\mathbf{D_{FNR}}=0.30$. The overall accuracy of this classifier measured on the test set was $0.672$.\footnote{Zafar et al.~(\citeyear{disparatemistreatment}) report a slightly different baseline of: Accuracy = 0.668, $\mathbf{D_{FPR}}=0.18$, $\mathbf{D_{FNR}}=0.30$.} Our SD penalization approach empirically achieves approximately the same accuracy as the Zafar et al.~(\citeyear{disparatemistreatment}) approach, with significantly better fairness. It is difficult to compare fairness-accuracy tradeoffs with the Hardt et al.~(\citeyear{hardt}) approach, since their accuracy is significantly lower than ours. A more direct comparison is possible by noting that our learned classifier can be post-processed to improve its fairness at a direct cost to accuracy. Hence, we can achieve accuracy of $0.659$ with $\mathbf{D_{FPR}} = \mathbf{D_{FNR}} = 0.01$, which compares very favorably with the Hardt et al. accuracy rate of 0.645 given the same FPR and FNR rates.\footnote{For completeness, we note that using a 50-50 training-test split (again not using the test set for parameter selection), our method (SD, both considerations) produces a classifier that provides: Accuracy = 0.659, $\mathbf{D_{FPR}} = 0.01, \mathbf{D_{FNR}} = 0.05$. This classifier can be post-processed to achieve rates of: Accuracy = 0.655, $\mathbf{D_{FPR}} = \mathbf{D_{FNR}} = 0.01$.}

Figure \ref{fig:compas} illustrates the accuracy/fairness trade-offs achievable using our scheme. Increasing the weight $c$ on the proxy fairness penalizers results in reducing their magnitude. The figure also illustrates how our relaxed penalizers succeed in tracking the real FPR and FNR differences. 
%
%
%\katrina{Must rewrite the following paragraph}
%We observe that our method succeeds in eliminating unfairness almost completely on the COMPAS dataset, while retaining most of the accuracy, when compared to the vanilla logistic regression. We achieve very low difference rates when penalizing for achieving each of the FPR and FNR criteria individually, and also for both. We achieve preferable results comparing to Zafar et al. and Zafar et al. baseline in all 3 scenarios, and also comparing to Hardt et al. in the settings of false positive/false negative considerations only. In the setting of both considerations - The Hardt et al. method removes a larger portion of the unfairness, however it results in major accuracy loss as it achieves accuracy rate of 0.645 in comparison to our method which results in accuracy of 0.665, retaining most of the original accuracy rate while removing most of the unfairness.




%The Hardt et al.~\cite{hardt} approach as reported removes a smaller portion of the bias in the different scenarios, however for FP/FN constraints alone, it provides higher accuracy rates. The Zafar et al.~(\citeyear{disparatemistreatment}) approach as reported retains significant bias (in most cases), but in some cases  achieves slightly superior accuracy rates to the methods above. 

%These performance comparisons are incomplete in the sense that each of the compared techniques has the potential to trade off between accuracy and fairness, using some degree of parameter tuning; what we report here is only one point on the achievable trade-off frontier for each algorithm. The ``correct'' trade-off, and, in particular, the best manner in which to weigh unfairness in the FPR against unfairness in the FNR, are matters of opinion. We have chosen to report our method's performance under parameters designed to very aggressively mitigate unfairness, at some cost to the accuracy.

%It would certainly be desirable to evaluate these and other approaches to fair learning on other datasets and on different tasks, particularly on larger datasets, which might afford both greater accuracy and better bias-reduction. The present empirical evaluations, however, suggest that our regularization-based approach provides a new tool worthy of consideration---we succeed in almost entirely eliminating bias on the hold-out set, at a modest price in terms of accuracy.

%Due to the fact that our true objective includes the original non-convex penalization terms, our approach does not carry any formal guarantees. However, the ease of implementation, generality, and empirical results are encouraging. Figure~\ref{fig:test1} illustrates the rate of convergence to a fair, accurate classifier on this dataset.
%In terms of computation costs, given that at each iteration we must calculate the gradient according to the FPR and FNR regularizers, we are required to predict the labels for the entire training set at each step. 
%However, this does not pose a computational burden, as it is already required by the (classic) gradient descent algorithm in our logistic regressor fitting scheme. Furthermore, when given a sufficiently large dataset (one or two orders of magnitude larger than the one currently available for the COMPAS scores data), this could be relaxed to sampling only a mini-batch of samples from the training data set at each iteration (much as is done in stochastic gradient descent).






\subsection{Additional Datasets}


Table~\ref{table:datasets_description} provides summary statistics on each of the datasets on which we tested our approach. We also briefly describe the datasets below. 


{\bf The Adult Dataset}\footnote{http://archive.ics.uci.edu/ml/datasets/Adult} is based on 1994 US Census data. The task we consider is to predict whether the income of each individual is over or under 50K dollars per year, based on features such as occupation, marital status, and education. The protected attribute selected in this task is gender. 

{\bf The Loan Default Dataset}\footnote{{\scriptsize https://archive.ics.uci.edu/ml/datasets/default+of+credit+card+clients}}
contains data regrading Taiwanese credit card users. The task we consider is to predict whether an individual will default on payments, based on features such as history of past payments, age, and the amount of given credit. The protected attribute is gender.

{\bf The Admissions Dataset}\footnote{http://www2.law.ucla.edu/sander/Systemic/Data.htm}
contains records of law school students who went on to take the bar exam. The task we consider is to predict whether a student will pass the exam based on features such as LSAT score, undergraduate GPA, and family income. The protected attribute is set to race.

Table~\ref{table:comparison_results_rest} describes the performance of our approach on these datasets, and Figures~\ref{fig:adult},~\ref{fig:default}, and~\ref{fig:lawschool} illustrate the fairness-accuracy trade-offs we achieve in each context. Overall, we see that unfairness is nearly eliminated while accuracy remains quite high. The dataset on which accuracy suffers most under our approach is the Adult dataset, which is also the dataset on which the vanilla regression is the most unfair.


\begin{figure*}[]
  \includegraphics[scale=0.6]{adult0-800.png}
  \caption{Adult Dataset. Fairness-Accuracy tradeoffs, as in Figure~\ref{fig:compas}.}
  \label{fig:adult}  
\end{figure*}



\begin{figure*}[]
  \includegraphics[scale=0.6]{default0-50.png}
  \caption{Loan Default Dataset. Fairness-Accuracy tradeoffs, as in Figure~\ref{fig:compas}.}
  \label{fig:default}
\end{figure*}



\begin{figure*}[]
  \includegraphics[scale=0.6]{admissions0-400.png}
  \caption{Admissions Dataset. Fairness-Accuracy tradeoffs, as in Figure~\ref{fig:compas}.}
  \label{fig:lawschool}
\end{figure*}




\section{Conclusion}

A method for inferring nonlinear VAR models has been proposed and validated. The modeling assumption that the observed data are the outputs of nodal nonlinearities applied to the individual time series of a linear VAR process lying in an unknown latent vector space. Since the number of parameters that determine the topology does not increase, the model interpretability remains the same as that with linear VAR modeling, making the proposed model amenable for Granger causality testing and network topology identification. The optimization method, similar to that of DNN training, can be extended with state-of-the-art tools to accelerate training and avoid undesired effects such as convergence to unstable points and overfitting.

\subsubsection{Acknowledgement:}

The authors would like to thank Emilio Ruiz Moreno for helping us manage a more elegant derivation of the gradient of $g_i(\cdot)$.

\bibliographystyle{IEEEtran}
\bibliography{intap}

\end{document}