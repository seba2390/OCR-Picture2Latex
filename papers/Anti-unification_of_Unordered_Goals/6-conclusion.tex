\section{Conclusions and Future Work}\label{section-conclusion}
In this work, we have systematically studied different key notions and results concerning anti-unification of unordered goals, i.e. sets of atoms. We have defined different anti-unification operators and we have studied several desirable characteristics for a common generalization, namely optimal cardinality (lcg), highest $\tau$-value (msg) and variable dataflow optimizations. For each case we have provided detailed worst-case time complexity results and proofs. An interesting case arises when one wants to minimize the number of generalization variables or constrain the generalization relations so as they are built on injective substitutions. In both cases, computing a relevant generalization becomes an NP-complete problem, results that we have formally established.
In addition, we have proven that an interesting abstraction -- namely $k$-swap stability which was introduced in earlier work -- can be computed in polynomially bounded time, a result that was only conjectured in  earlier work. 

Our discussion of dataflow optimization in Section~\ref{section-relation-2} essentially corresponds to a reframing of what authors of related work sometimes call the \textit{merging} operation in rule-based anti-unification approaches as in~\cite{Baumgartner2017}. Indeed, if the "store" manipulated by these approaches contains two anti-unification problems with variables generalizing the same terms, then one can "merge" the two variables to produce their most specific generalization. If the merging is exhaustive, this technique results in a generalization with as few different variables as possible. In this work we isolated dataflow optimization from that specific use case and discussed it as an anti-unification problem in its own right.

While anti-unification of goals in logic programming is not in itself a new subject, to the best of our knowledge our work is the first systematic treatment of the problem in the case where the goals are not sequences but unordered sets. Our work is motivated by the need for a practical (i.e. tractable) generalization algorithm in this context. The current work provides the theoretical basis behind these abstractions, and our concept of $k$-swap stability is a first attempt that is worth exploring in work on clone detection such as~\cite{clones}. 

Other topics for further work include adapting the $k$-swap stable abstraction from the $\preceq^\iota$ relation to dealing with the $\sqsubseteq^\iota$ relation. 
A different yet related topic in need of further research is the question about what anti-unification relation is best suited for what applications. For example, in our own work centered around clone detection in Constraint Logic Programming, anti-unification is seen as a way to measure the distance amongst predicates in order to guide successive syntactic transformations. Which generalization relation is best suited to be applied at a given moment and whether this depends on the underlying constraint context remain open questions that we plan to investigate in the future. 

%The main results of this paper are the polynomial algorithms solving specific anti-unification problems, along with several worst-case time complexity results and proofs. 

% have made efforts to extend the classical anti-unification concepts to the case where the artefacts to generalize are unordered goals. We have done this by considering different levels of atomic abstraction through different generalization relations. W





%Throughout the paper, we have introduced four generalization relations. Figure~\ref{fig-interconnexion} shows how the four relations are linked on a conceptual level. $\sqsubseteq$ is the most general relation as generalization is defined with any substitution. Restricting the definition to injective substitutions or to renamings yields more specific relations, the intersection of which is relation $\preceq^\iota$ where variables are generalized through injective renamings. 

%\begin{figure}[htbp]
%	\begin{center}
%		\begin{tikzpicture}[x=0.75pt,y=0.75pt,yscale=-1,xscale=1]
%		%uncomment if require: \path (0,300); %set diagram left start at 0, and has height of 300
%		
%		%Shape: Ellipse [id:dp330479544492589] 
%		\draw   (150.05,144.64) .. controls (150.05,72.29) and (186.09,13.64) .. (230.55,13.64) .. controls (275,13.64) and (311.05,72.29) .. (311.05,144.64) .. controls (311.05,216.99) and (275,275.64) .. (230.55,275.64) .. controls (186.09,275.64) and (150.05,216.99) .. (150.05,144.64) -- cycle ;
%		%Shape: Ellipse [id:dp3140532351606715] 
%		\draw   (165,87.62) .. controls (165,64.99) and (193.75,46.64) .. (229.22,46.64) .. controls (264.69,46.64) and (293.45,64.99) .. (293.45,87.62) .. controls (293.45,110.26) and (264.69,128.61) .. (229.22,128.61) .. controls (193.75,128.61) and (165,110.26) .. (165,87.62) -- cycle ;
%		%Shape: Ellipse [id:dp9580468020324391] 
%		\draw   (261.25,53.46) .. controls (285.77,54.29) and (304.38,90.37) .. (302.81,134.06) .. controls (301.24,177.74) and (280.08,212.49) .. (255.56,211.67) .. controls (231.03,210.85) and (212.43,174.76) .. (214,131.08) .. controls (215.57,87.39) and (236.72,52.64) .. (261.25,53.46) -- cycle ;
%		
%		% Text Node
%		\draw (172,266) node   {$\sqsubseteq $};
%		% Text Node
%		\draw (235,216) node   {$\preceq $};
%		% Text Node
%		\draw (170,125) node   {$\sqsubseteq^\iota $};
%		% Text Node
%		\draw (254,92) node   {$\preceq^\iota $};
%		\end{tikzpicture}
%	\end{center}
%	\caption{The interconnexions of four generalization relations}
%	\label{fig-interconnexion}
%\end{figure}

%Figure~\ref{fig-interconnexion} shows how the four relations are linked on a conceptual level. When needed in concrete applications, the right generalization operator (or an abstraction) should be used; this of course depends on whether or not the atomic structure should be generalized and the variable dataflow preserved. 

%Future work will focus on the use of such generalization operators in the purpose of applying synctatic transformations on predicates in such a way that the structural distance between them decreases; such a synctatic distance can be evaluated over the most specific generalization of the predicates under scrutiny.