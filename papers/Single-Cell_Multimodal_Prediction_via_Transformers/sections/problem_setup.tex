\section{Problem Statement}
% \vspace{-0.5em}
% In this section, we formalize the problem which we try to solve.

Before we state the problem, we first introduce the notations used in the following sections. For clarity and simplicity, we use the subscripts "g", "p" and "c" for gene, protein, and cell, respectively. For instance, we use $\mathbf{h}_g$, $\mathbf{h}_p$, $\mathbf{h}_c$ to denote the embeddings of genes, proteins and cells, respectively.

In this work, we follow the problem setting in the NeurIPS 2022 competition, i.e., Multimodal Single-Cell Integration Across Time, Individuals, and Batches\footnote{https://www.kaggle.com/competitions/open-problems-multimodal/}. The goal of this competition is to predict a paired modality with a given modality and to infer how DNA, RNA, and protein measurements co-vary in single cells. Specifically, we focus on using gene expression (RNA) to predict surface protein level. We denote $\mathbf{X}_{\text{g}}$ and $\mathbf{X}_{\text{p}}$ as the measurement counts of gene and protein, respectively. With $\mathbf{X}_{\text{g}}$, we try to learn a mapping function that can best describe the relationship between two modalities. We denote $\mathcal{L} \left(\cdot,\cdot\right)$ as the objective function that measures the dissimilarity between the predicted and the true protein level. Formally, we describe our target as an optimization problem:
% \vskip 0.5em

\textit{Given $\mathbf{X}_{\text{g}}$ and the objective function $\mathcal{L}\left(\cdot,\cdot\right)$, we aim to find a mapping function $f_{\theta}^*$ (parameterized by $\theta$) that minimize the objective loss:
\begin{equation}
    f_{\theta}^* = \arg\min_{f_{\theta}} \mathcal{L} \left( f_{\theta} \left(\mathbf{X}_{\text{g}}\right), \mathbf{X}_{\text{p}}\right).
\end{equation}}

In the subsequent sections, we formulate the mapping $f_{\theta}^*$ using transformers and GNNs and employ Root Mean Square Error, Mean Absolute Error, and Pearson correlation coefficient as evaluation metrics for our predictions.