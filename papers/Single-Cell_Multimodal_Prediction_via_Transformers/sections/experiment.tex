\section{Experiment}
In this section, we present the experimental results of \method{} against baselines on benchmark datasets. In particular, we aim to answer the following questions:
\begin{itemize}[leftmargin=*]
    \item \textbf{RQ1:} How does \method{} perform compare against baselines based on various evaluation metrics?
    \item \textbf{RQ2:} Given various choices of the PEs, how do they affect the performance of \method{}?
    \item \textbf{RQ3:} What is the best way to fuse information across modalities?
    \item \textbf{RQ4:} Can \method{} handle other modality prediction tasks?
    \item \textbf{RQ5:} Can \method{} process large datasets?
    \item \textbf{RQ6:} How does each of the model component impact the performance of \method{}?
\end{itemize}
Before presenting our experimental results and observations, we first introduce the experimental settings.
% We evaluate the ablation of remove transformers.

\subsection{Experimental Settings}
\subsubsection{Datasets}
We follow the setting of the NeurIPS multimodal single-cell integration competition of the year 2021~\cite{luecken2021sandbox} and 2022
% \footnote{https://nips.cc/virtual/2022/competition/50092} 
and collect the joint measurements of gene expression and surface protein levels datasets from the competitions. Both datasets contain the raw counts, which represent the number of reads per gene per cell, as well as the normalized counts. For the NeurIPS 2021 competition, we pick the data corresponding to the task of protein abundance prediction via gene expression and refer to it as ``GEX2ADT''. The processed data is centered and log-transformed for denoising purposes. For the competition in 2022, which we refer to as ``CITE'', the objective is to utilize CITE-seq~\cite{stoeckius2017simultaneous} data measured from days 2, 3, and 4 to predict the protein level on day 7 from different individuals. It is worth mentioning that the protein level testing data is not available during the completion of this work. Therefore, we simulate the competition scenario by treating the training data from day 4 as our testing set. The processed RNA data is centered and log-transformed, while the normalized protein levels are denoised and scaled by background~\cite{kotliarov2020broad}. We summarize the dataset details in Table~\ref{table:dataset} in Appendix~\ref{app:exp}. 
% The experimental results on CBMC are shown in Appendix~\ref{app:cbmc}.
% \section{The Semantic Urban Mesh Dataset}\label{sec:framework}
\subsection{Dataset Specification}

We have used Helsinki's 3D texture meshes as input and annotated them as a benchmark dataset of semantic urban meshes. 
The Helsinki's raw dataset covers about 12 $ km^2 $, and it was generated in 2017 from oblique aerial images that have about a 7.5 $cm$  ground sampling distance (GSD) using an off-the-shelf commercial software namely ContextCapture~\citep{contextcap}.
The source images have three colour channels (i.e., red, green, and blue) and are collected from an airplane with five cameras that have $80\%$ length coverage and $60\%$ side coverage.
To recover the 3D water bodies that do not fulfil the Lambertian hypothesis, 2D vector maps and ortho-photos are used when performing the surface reconstruction.
Furthermore, processing like aerial triangulation, dense image matching, and mesh surface reconstruction were all performed with ContextCapture.
It should be noticed that the entire region of Helsinki is split into tiles, and each of them covers about 250 $ m^2 $~\citep{kalasatamaReport}.
As shown in Figure \ref{fig:overview},  we have selected the central region of Helsinki as the study area, which includes 64 tiles and covers about 4 $km^2$ map area (8 $km^2$ surface area) in total.   

\subsection{Object Classes}
We define the semantic categories for urban meshes by the most common objects in the urban environment with unambiguous geometry and texture appearance.
Moreover, each triangle face is assigned to a label of one of the six semantic classes. 
Ambiguous regions (which account for about 2.6\% of the total mesh surface area), such as shadowed regions or distorted surfaces, are labelled as unclassified (see Figure \ref{fig:ambigious}).
The object classes we consider in the benchmark dataset are: 
\begin{itemize}
	\item \textbf{terrain}: roads, bridges, grass fields, and impervious surfaces;
	\item \textbf{building}: houses,high-rises, monuments, and security booths;
	\item \textbf{high vegetation}: trees, shrubs, and bushes;
	\item \textbf{water}: rivers, sea, and pools;
	\item \textbf{vehicle}: cars, buses, and lorries;  
	\item \textbf{boat}: boats, ships, freighters, and sailboats;
	\item \textbf{unclassified}: incomplete objects like buses and trains, distorted surfaces like tables, tents and facades, construction sites, underground walls.
\end{itemize}

\begin{figure}[!tb]
	\includegraphics[height=0.48\textwidth]{figures/overview_grids/yaxis.png}
	\begin{subfigure}[t]{0.48\textwidth}
		\includegraphics[width=\linewidth]{figures/overview_grids/texture_global_birdsview00.png}
		\includegraphics[width=\linewidth]{figures/overview_grids/xaxis.png}
		\label{fig:textop}
	\end{subfigure}
	\hspace*{\fill}
	\begin{subfigure}[t]{0.48\textwidth}		
		\includegraphics[width=\linewidth]{figures/overview_grids/semantic_global_birdsview00.png}
		\vspace*{-0.78cm}
		\begin{center}
		\includegraphics[width=0.8\linewidth]{figures/semantic_results/semantic_legend2.png}
		\end{center}
		\label{fig:semtop}
	\end{subfigure}
	\vspace*{-0.7cm}
	\caption{Overview of the semantic urban mesh benchmark.
	Left: the texture meshes covering about 4 $km^2$ map area. Right: the ground truth meshes.
	More views of the same scene (with different visualization styles) are shown in Figures \ref{fig:texside} and \ref{fig:semside}.}
	\label{fig:overview}
\end{figure}

\begin{figure}[!tb]
	\centering
	\begin{subfigure}[t]{0.48\textwidth}
		\includegraphics[width=\linewidth]{figures/ambigious/shadow_tex_zoom.png}
		\caption{}
	\end{subfigure}
	\hspace*{\fill}
	\begin{subfigure}[t]{0.48\textwidth}
		\includegraphics[width=\linewidth]{figures/ambigious/shadow_fc_zoom.png}
		\caption{}
	\end{subfigure}
	\begin{subfigure}[t]{0.48\textwidth}
		\includegraphics[width=\linewidth]{figures/ambigious/distort_tex_zoom.png}
		\caption{}
	\end{subfigure}
	\hspace*{\fill}
	\begin{subfigure}[t]{0.48\textwidth}
		\includegraphics[width=\linewidth]{figures/ambigious/distort_fc_zoom.png}
		\caption{}
	\end{subfigure}
	\caption{Ambiguous regions are labelled as unclassified (in black). 
		(a) Shadow region with texture.
		(b) Shadow region with semantic colour.
		(c) Distorted region with texture.
		(d) Distorted region with semantic colour.} 
	\label{fig:ambigious}
\end{figure}


\subsection{Semi-automatic Mesh Annotation}  \label{sec:mesh_annota}
Rather than manually labelling each triangle face of the raw meshes, we design a semi-automatic mesh labelling framework to accelerate the labelling process. Figure~\ref{fig:pipeline} shows the overall pipeline of our labelling workflow.

Given the fact that urban environments consist of a large number of planar regions in the data, we opt to label the data at the segment level instead of individual triangle faces. 
Specifically, we over-segment the input meshes into a set of planar segments. 
These segments can enrich local contextual information for feature extraction and serve as the basic annotation unit to improve annotation efficiency.

\begin{figure}[!tb]
	\centering
	\includegraphics[width=\textwidth]{figures/pipeline/pipeline_L1.png}
	\caption{The pipeline of the labelling workflow.}
	\label{fig:pipeline}
\end{figure}

Instead of randomly choosing a mesh tile as input for annotation and refinement, which is insufficient for manual annotation progress, we favour picking a mesh tile that is more difficult to classify.
Similar to active learning, we first compute the feature diversity (see Equation \ref{eq:fea_div}) to optimally select a mesh tile containing a variety of classes and objects at different scales and complexity.
The feature diversity $F_{m}$ of tile $m$ is computed as
\begin{equation}\label{eq:fea_div}
	F_{m}=\frac{\sum_{i=1}^{N_{f}}\left ( f_i - \bar{f} \right )^{2}}{N_{f}}
\end{equation}
where $f_i$ represents each handcrafted feature which describe in Section \ref{sec:initial_seg}, and $\bar{f}$ is mean value of a $N_{f}$ dimensional feature vector.
To acquire the first ground truth data, we manually annotate the mesh (with segments) that is selected with the highest feature diversity.
Then, we add the first labelled mesh into the training dataset for the supervised classification.
Specifically, we use the segment-based features as input for the classifier, and the output is a pre-labelled mesh dataset.
Next, we use the mesh annotation tool to manually refine the pre-labelled mesh according to the feature diversity.
Finally, the new refined mesh will be added to the training dataset to improve the automatic classification accuracy incrementally.


\subsubsection{Initial Segmentation}\label{sec:initial_seg}

To avoid redundant computations of numerous triangles, we first apply mesh over-segmentation (i.e., linear least-squares fitting of planes) based on region growing on the input data to group triangle faces into homogeneous regions~\citep{lafarge2012creating}.
Such grouped regions are beneficial for computing local contextual features.
We then extract both geometric and radiometric features from those mesh segments as follows: 
\begin{itemize}
	\item[$\bullet$] \textit{Eigen-based features} are computed from the covariance matrix of the triangle vertices with respect to the average centre within each segment, which is beneficial for identifying urban objects with various surface distributions.
	The linearity $= (\lambda_{1} - \lambda_{2}) / \lambda_{1}$, sphericity $= \lambda_{3}/ \lambda_{1}$ and change of curvature $= \lambda_{3} / (\lambda_{1} + \lambda_{2} + \lambda_{3})$ are computed based on the three eigenvalues $\lambda_{1} \geq \lambda_{2} \geq \lambda_{3}\geq 0$.
	The local eigenvectors $\mathbf{n}_{i} $ and the unit normal vector $\mathbf{n}_{z} $ along Z-axis are used to compute the verticality $=1-\left | \mathbf{n}_{i}\cdot \mathbf{n}_{z} \right | $~\citep{hackel2016fast}.
	Note that many eigen-based features have been studied in literature~\citep{hackel2016fast,west2004context,weinmann2013feature}, and some of them were designed for and tested on LiDAR point clouds. 
	\textcolor{ao}{
	These eigen-based features are mostly computed per point based on its spherical neighbourhood, which often contains noise and does not form a surface. 
	Our chosen eigen-based features are defined on a segment representing the surface of a mesh, and thus they can capture non-local geometric properties of an object.
	}
	Additionally, in this work, we have tested all eigen-based features from the literature~\citep{hackel2016fast}, and we only present the ones that are effective for texture meshes.
	\item[$\bullet$] \textit{Elevation} is divided into absolute elevation $z_{a}$, relative elevation $z_{r}$ and multiscale elevations $z_{m}$.
	Where $z_{a}$ is the average elevation of the segment;
	the relative elevation is computed as $z_{r} = z_{a}-z_{r_{min}}$;
	the multiscale elevation~\citep{Verdie2015,Rouhani2017} $z_{m} = \sqrt{\frac{z_{a} - z_{min}}{z_{max} - z_{min}}}$.
	And $z_{r_{min}}$ denotes the lowest elevation of the local largest ground segment computed within a cylindrical neighbourhood with 30 meters radius around the segment centre.
	$z_{min}$ and $z_{max}$ represent the local minimum and maximum elevation values of a cylindrical neighbourhood within the scale of 10 meters, 20 meters, and 40 meters.
	Such large cylindrical neighbourhoods allow to find the local ground considering the resilience to hilly environments, \textcolor{ao}{and the square root ensures that small relative height values (i.e., values smaller than 1 $ m $) get a larger elevation attribute to enlarge elevation differences between small objects and the local ground (e.g., cars against the ground, boats against the water surfaces).}
	More importantly, due to the influence of terrain fluctuations and various scales of urban objects, the elevation of these three categories can complement each other.
	\item[$\bullet$] \textit{Segment area} is computed as $area(S_k) = \sum_{i = 1}^{N} area(f_i) $, where $f_i$ denotes a triangle of the segment $S_k$, and $N$ denotes the total number of triangles in $S_k$.
	\item[$\bullet$] \textit{Triangle density} is defined as $density(S_k) = \frac{N}{area(S_k)} $,  which reveals the object complexity, especially for adaptive urban meshes.
	\item[$\bullet$] \textit{Interior radius of 3D medial axis transform (InMAT)}~\citep{ma20123d,peters2016robust} of a segment $S_k$ is formulated as $r_k = \frac{\sum_{i=1}^{M} r_i}{M}$, where $M$ denotes the total number of triangle vertices of $S_k$, and $r_i$ denotes the interior radius of the shrinking ball that touches the vertex $v_i$ within the segment $S_k$. 
	It is designed to distinguish objects with different scales. 
	\item[$\bullet$] \textit{HSV colour-based features} are derived from the RGB channel of the entire texture map.
	We use the HSV colour space since it can better differentiate different objects than RGB.
	We compute the average colour, the variance of the colour distribution of all pixels within each segment, and we further discretize it into a histogram that consists of 15 bins of the hue channel, five bins of the saturation channel, and five bins of the value channel.
	\item[$\bullet$] \textit{Greenness} $a_{g}$ is used to classify objects that are similar to green vegetation.
	Specifically, it is computed according to the averaged RGB colour of each segment via $a_{g}=G-0.39\cdot R-0.61\cdot B$~\citep{mckinnon2017comparing}. 
\end{itemize}
	All the above features are concatenated into a 44-dimensional feature vector used by our random forest (RF) classifier in the initial segmentation. 

\subsubsection{Annotation Tool for Refinement}

Because of the under-segmentation errors and the imperfect results of the semantic mesh segmentation process, we design a mesh annotation tool (see Figure \ref{fig:annotator}) to manually correct the labelling errors.
Our mesh annotation tool is developed based on the labelling tool of CGAL~\citep{cgal:eb-20b}.

\begin{figure}[!tb]
	\centering
	\includegraphics[width=\textwidth]{figures/annotator/annotator.png}
	\caption{The interface of our annotation tool for 3D texture meshes. }
	\label{fig:annotator}
\end{figure}

As shown in Table \ref{tab:annotation_operation}, it consists of three operation categories: view, selection, and annotation.
The	view operations provide essential functions for the user to manipulate the scene camera, such as translate, rotate, zoom, or set the new pivot for the scene.
In addition, to use textures as a reference for labelling, we map texture and face colour with a certain degree of transparency, and we visualize the segment border to differentiate each segment. 

\begin{table}[!tb]
	\centering
	\noindent\adjustbox{max width=0.8\textwidth}
	{
		\begin{threeparttable}
			\centering
			\begin{tabular}{ccc}
				\toprule
				Categories & Operations & Objects \\
				\midrule
				\multirow{4}[2]{*}{View} & Translate & Camera \\
				& Rotate & Camera \\
				& Zoom in / out & Camera \\
				& Set pivot & Camera \\
				\midrule
				\multirow{6}[2]{*}{Selection} & Multi-selection / Lasso & Triangles / Segments \\
				& Expand / Reduce & Triangles / Segments \\
				& Semantic selection & Segments \\
				& Split region & Segments \\
				& Planar region extraction & Triangles \\
				& Split mesh & Triangles \\
				\midrule
				\multirow{3}[2]{*}{Annotation} & Probability slider & Segments \\
				& Segment area slider & Segments \\
				& Progress bar & Triangles \\
				& Switch semantic view & Triangles \\ 
				& Labelling & Triangles / Segments \\
				\bottomrule
			\end{tabular}%
		\end{threeparttable}
	}
	\caption{Basic operations in our annotation tool.} 
	\label{tab:annotation_operation}%
\end{table}%


The	selection operations allow the user to select or deselect either triangle faces (see Figure \ref{fig:tri_sel}) or segments (see Figure \ref{fig:seg_sel}) freely via a brush or a lasso.
Specifically, the face selection operation is used to fix the under-segmentation errors and generate new segments, and the segment selection operation is to fix incorrect segment labels.

\begin{figure}[!tb]
	\centering
	\begin{subfigure}[t]{0.32\textwidth}
		\includegraphics[width=\linewidth]{figures/pipeline/tri_select_a.png}
		\caption{}
	\end{subfigure}
	\hspace*{\fill}
	\begin{subfigure}[t]{0.32\textwidth}
		\includegraphics[width=\linewidth]{figures/pipeline/tri_select_b.png}
		\caption{}
	\end{subfigure}
	\hspace*{\fill}
	\begin{subfigure}[t]{0.32\textwidth}
		\includegraphics[width=\linewidth]{figures/pipeline/tri_select_c.png}
		\caption{}
	\end{subfigure}
	\caption{An example of labelling by selecting triangles using the lasso tool (blue edges: segment boundaries). 
		(a) Before selection.
		(b) Lasso selection result (in red).
		(c) The correct label has been assigned to the selected region. 
		In this example, the label of the selected region has been changed from `ground' to `vehicle'.
	} 
	\label{fig:tri_sel}
\end{figure}


\begin{figure}[!tb]
	\centering
	\begin{subfigure}[t]{0.32\textwidth}
		\includegraphics[width=\linewidth]{figures/pipeline/seg_select_a.png}
		\caption{}
	\end{subfigure}
	\hspace*{\fill}
	\begin{subfigure}[t]{0.32\textwidth}
		\includegraphics[width=\linewidth]{figures/pipeline/seg_select_b.png}
		\caption{}
	\end{subfigure}
	\hspace*{\fill}
	\begin{subfigure}[t]{0.32\textwidth}
		\includegraphics[width=\linewidth]{figures/pipeline/seg_select_c.png}
		\caption{}
	\end{subfigure}
	\caption{An example of segment labelling. 
		(a) Part of a wall of the building was previously labelled as `high vegetation' (in green).
		(b) Segment selection result (in red).
		(c) The label of the selected segment has been corrected with the new label `building'.
	}
	\label{fig:seg_sel}
\end{figure}

We also allow the user to edit the selection of each individual segment with splitting functions (see Figure \ref{fig:pnp_func}) and automatic extraction of the most planar region (see Figure \ref{fig:seg_func}). 
As for splitting, we first detect the potential planar and non-planar segments marked by user strokes, and then the non-planar one is split according to the vertex-to-plane distance.
It allows generating candidate non-planar regions (with respect to the detected planar segment) for the user to edit, and
it is useful to split a segment that covers large non-planar regions or contains more than one dominant planar area.
To extract the most planar region, we apply the region growing algorithm~\citep{lafarge2012creating} within the selected segment to automatically generate the candidate triangle faces with user-defined thresholds (i.e., the maximum distance to the plane, the maximum accepted angle, and the minimum region size).
Such an operation allows the user to filter out some small bumpy regions of the selected segment.

\begin{figure}[!tb]
	\centering
	\begin{subfigure}[t]{0.48\textwidth}
		\includegraphics[width=\linewidth]{figures/annotator/pnp_pipeline1.png}
		\caption{}
	\end{subfigure}
	\hspace*{\fill}
	\begin{subfigure}[t]{0.48\textwidth}
		\includegraphics[width=\linewidth]{figures/annotator/pnp_pipeline2.png}
		\caption{}
	\end{subfigure}
	\caption{An example splitting planar and non-planar regions. 
		(a) The user draws a stroke (in red) across the border of the non-planar segment and the planar segment. 
		(b) The detected non-planar segment has been split into two parts (i.e., a non-planar region shown in red and a planar segment shown in green).
	} 
	\label{fig:pnp_func}
\end{figure}

\begin{figure}[!tb]
	\centering
	\begin{subfigure}[t]{0.48\textwidth}
		\includegraphics[width=\linewidth]{figures/annotator/planar_split_pipeline1.png}
		\caption{}
	\end{subfigure}
	\hspace*{\fill}
	\begin{subfigure}[t]{0.48\textwidth}
		\includegraphics[width=\linewidth]{figures/annotator/planar_split_pipeline3.png}
		\caption{}
	\end{subfigure}
	\caption{Editing an individual segment. 
		(a) A segment is selected (highlighted in green) for splitting. 
		(b) Automatic extraction of the most planar region (shown in red) within the selected segment according to user-defined thresholds.} 
	\label{fig:seg_func}
\end{figure}

Besides, probability and area-based sliders and a progress bar are provided in the annotation panel to improve annotation efficiency and experience, respectively. 
Specifically, the probability slider is introduced for the user to visually inspect the segments that are most likely misclassified.
Moreover, the user can further use it to inspect a specific class by switching the view to highlight a specific semantic class.
The segment area slider is used to identify isolated tiny segments, which commonly appear as errors.
The progress bar is used to indicate the estimated labelling progress during the annotation.
After performing the selection, the user can easily assign the corresponding label to the selected area.


\subsubsection{Baselines}
We evaluate the performance of \method{} against state-of-art multimodal prediction models among the task of using gene expression to predict surface protein levels. The selected baselines are as follows:
\begin{itemize}[leftmargin=*]
    \item \textbf{Cross-modal Autoencoders}~\cite{yang2021multi}, short for CMAE, incorporated multiple autoencoders to integrate multimodal data and utilized domain knowledge by adding discriminative loss to the training process to align shared markers or clusters.
    \item \textbf{BABEL}~\cite{wu2021babel} proposed a general framework for multimodal translation with modality-specific encoders and decoders. Note that initially, BABEL focused on RNA and ATAC-seq~\cite{buenrostro2013transposition} data. In this evaluation, we repurpose BABEL to the RNA to protein setting.
    \item \textbf{scMM}~\cite{minoura2021mixture} modeled the multimodal data with generative setting. We note that the input of scMM is restricted to raw counts, and the predictions are scaled as centered log-transformed data.
    \item \textbf{ScMoGNN}~\cite{wen2022graph} involved domain knowledge like biological pathways to enhance the GNNs. The original ScMoGNN followed a transductive setting. In this work, we implement an inductive setting of ScMoGNN for a fair comparison with the baselines.
\end{itemize}

\subsubsection{Parameter Setting}
To benchmark the performance of baselines and \method{}, we uniformly employ inductive settings among both datasets. On the CITE, we use the data measured on day 4 for testing and randomly split $80/20\%$ of the data prior to day 4 for training and validation. On the GEX2ADT, we randomly pick $15\%$ of the training data for validation and evaluate the predictions on the testing set. For BABEL, the hidden dimension is tuned from $\{16, 32, 64, 128\}$. For CMAE, the weights of adversarial loss and reconstruction loss are chosen from $\{0.1, 1, 2.5, 5, 10\}$. For scMM, the hidden dimensions are tuned from $\{16, 32, 64, 128\}$. For ScMoGNN, the weight decay parameter of the optimizer is tuned from $\{5\times 10^{-6},1 \times 10^{-5}, 5 \times 10^{-5}, 1 \times 10^{-4}\}$.

\subsection{Evaluation of Predictions}
\begin{table*}[!ht]
\centering
\tabcolsep 7pt
\begin{tabular}{lcclllll}
 \multicolumn{3}{c}{Evaluation Protocol$\rightarrow$} & \multicolumn{2}{c}{ImageNet} & \multicolumn{2}{c}{Low-shot} & ADE20K \\
\cmidrule(lr){4-5}\cmidrule(lr){6-7}\cmidrule(lr){8-8}  
Method&Target&Epoch& ft(\%) & lin(\%) & 1\% & 10\% & mIOU \\
\midrule
\multicolumn{8}{@{\;}l}{\bf Supervised learning} \\
\quad DeiT III\cite{DEiT-v3} &- & 800 & 83.8 &- & - & - & 49.3 \\
\midrule
\multicolumn{7}{@{\;}l}{\bf Masked Image Modeling w/ pre-trained target generator} \\
\quad BEiT\cite{BEiT} &DALLE&800&83.2 & 56.7 & - & - & 45.6 \\
\quad CAE\cite{CAE} & DALLE&800&83.8&68.6& - & - & 49.7 \\
\quad MILAN$^{*}$\cite{MILAN} & CLIP-B&400&85.4&78.9&67.5 & 79.7 & 52.7\\
\quad BEiT-v2\cite{BEiTv2}& VQ-KD&1600&85.5& 80.1 &-&-&53.1\\
\quad MaskDistill\cite{MASKdistill} &CLIP-B&800&85.5&-&-&-&54.3\\
\midrule
\multicolumn{7}{@{\;}l}{\bf Masked Image Modeling w/o pre-trained target generator} \\
\quad MaskFeat$^{*}$\cite{MaskFeat}&HOG&1600&84.0&62.3&52.9&73.5 & 48.3\\
\quad SemMAE\cite{SemMAE} & RGB &800&83.4&65.0&- & - & 46.3\\
\quad SimMIM\cite{SimMIM}& RGB &800&83.8&56.7&-&- & -\\
\hdashline
\quad MAE$^{*}$\cite{MAE} & RGB & 300 & 82.8 & 61.5 & 41.4 & 70.5 & 43.9 \\
\quad \textbf{MFF}$_\text{\tt MAE}$ &RGB&300&{{83.3} \more{(+0.5)}}&{63.3} \more{(+1.8)} & 43.7 \more{(+2.3)} & 71.4 \more{(+0.9)} &{47.7} \more{(+3.6)}\\
\quad MAE$^{*}$\cite{MAE} & RGB & 800 & 83.3 & 65.6 & 45.4 & 71.2 & 46.1 \\
\quad \textbf{MFF}$_\text{\tt MAE}$ &RGB&800& 83.6 \more{(+0.3)} & 67.0 \more{(+1.4)} & 48.0 \more{(+2.6)}& 72.0 \more{(+0.8)}  & 47.9 \more{(+1.8)}\\
% \quad MAE$^{*}$\cite{MAE} & RGB & 1600 & 83.5 & 67.8 & 47.8 & 72.4 & 48.1 \\
% \quad \textbf{MFF}$_\text{\tt MAE}$ &RGB&1600& 83.7 \more{(+0.2)} & 69.6 \more{(+1.8)} & 51.9 \more{(+3.1)} & 73.4 \more{(+1.0)}  & 48.3 \more{(+0.2)}\\
\hdashline

\quad PixMIM\cite{pixmim} & RGB&800 & 83.5 & 67.2 & 47.9 & 72.2 & 47.3\\
\quad \textbf{MFF}$_\text{\tt PixMIM}$ &RGB&800&83.6 \more{(+0.1)}&68.2 \more{(+1.0)} & 49.0 \more{(+1.1)} & 73.0 \more{(+0.8)}  &48.6 \more{(+1.3)}\\
% \quad PixMIM\cite{pixmim} & RGB & 1600 & 83.6 & 69.3 & 50.9 & 72.9 & 48.7 \\
% \quad \textbf{MFF}$_\text{\tt PixMIM}$ &RGB&1600&83.9 \more{(+0.3)}&71.1 \more{(+1.8)}& 53.9 \more{(+3.0)} & 74.1 \more{(+1.2)}  &49.1 \more{(+0.4)}\\

\end{tabular}
\caption{\textbf{Performance comparison of MIM methods on various downstream tasks.} We report the results with fine-tuning (ft) and linear probing (lin) experiments on ImageNet-1K, objection detection on COCO, and semantic segmentation on ADE20K. The backbone of all experiments is ViT-B\cite{ViT}. $*$: numbers are reported by running the official code release. Low-shot: end-to-end fine-tuning with 1\% and 10\% of the training set.}
\label{tab:comparison}
\end{table*}

We evaluate the final protein-level prediction performance using Root Mean Square Error (RMSE) and Mean Absolute Error (MAE). Meanwhile, because multimodal data usually suffers from the influence of batch effects and unbalanced measuring depth, the count's scale of each cell may vary significantly, which will substantially affect the RMSE and MAE metrics. Therefore, we also include the Pearson correlation coefficient (Corr), which is cell-wise normalized by the mean and variance of the input, as a robust and scale-free metric to evaluate the predictions. A lower RMSE or MAE score indicates a geometrically closer estimation of the protein levels, while a higher Corr score suggests a statistically more similar match to the actual value. We report the mean and the standard deviation of each metric across five different runs, and the results are illustrated in Table~\ref{table:main}. The best performance is highlighted in bold. 

To answer the first question, we note that our \method{} consistently outperforms all other baselines according to all three metrics on both datasets, indicating that \method{} successfully captures the quantitative characteristics of target protein levels given the input gene expression measurements. Particularly, for the CITE, \method{} achieved significantly lower RMSE compared to the second-best model ScMoGNN, by $0.04$. More importantly, \method{} achieved a significant improvement in terms of the Pearson correlation metrics over all other baselines, with a noticeably lower performance variation across runs.
% indicating the stability of our model.

We further analyze the performance of different models on proteins that are least well captured by any models. Specifically, for each model, we compute the RMSE for each protein separately and identify ten proteins that resulted in the highest average RMSE across all models. As shown in Figure~\ref{fig:bot10comp}, \method{} and ScMoGNN achieved relatively stable results and are consistently better compared to BABEl and CMAE.

\begin{figure}[htb]
    \centering
    \vspace{-1.2em}
    \caption{Least well-predicted protein comparison.}
    % \vspace{-1em}
    \includegraphics[width=0.47\textwidth]{images/bot10_perf_comp_boxplot.png}
    \vspace{-1.2em}
    \label{fig:bot10comp}
\end{figure}

\vspace{-0.5em}
\subsection{Positional Encoding}
\label{sec:pe}
As mentioned in Section~\ref{sec:graph_trans}, we implement Laplacian PE~\cite{dwivedi2020generalization} and random walk PE~\cite{dwivedi2022graph} to capture positional information of prior knowledge graph. For ease of notation, we use the abbreviation PE to refer to the PE. To benchmark the impact of the two types of PE among two datasets and answer the second question, we show the performance of \method{} with each PE and compare them with the scenario without any PE. The mean and standard deviation of RMSE scores of five runs are shown in Table~\ref{table:pe}. 

According to the results, the influence of PE varies among datasets. \method{} reaches the best RMSE score on CITE without PE, while two types of PE both improve the performance on GEX2ADT. Notice that in Table~\ref{table:dataset}, the RNA zero rate of CITE is significantly lower compared to the GEX2ADT, providing the model with greater access to data-specific information. If the data contains sufficient information, then the neighborhood information from the GNNs alone is adequate and there is no need for the extra prior knowledge from the PEs. This is further supported by the observation that random walk PE performs better than Laplacian PE in both datasets. The Laplacian PE models global information by using the spectral information of the graph Laplacian, while the random walk PE encodes local information by accessing the landing probability of a $k$-step random walk. In cases where the prior knowledge may be noisy for downstream tasks, the local information alone is enough for predictions and the global structure becomes redundant.

\begin{table}[tb]
    \centering
    \vspace{-1em}
    \caption{Prediction RMSE results of different positional encoding (score $\pm$ std).}
    \vspace{-1.2em}
    \scalebox{1.}{
    \begin{tabular}{c|c|c}
        \toprule
        & \textbf{CITE} & \textbf{GEX2ADT} \\ \midrule
        Laplacian PE & 1.63161 $\pm$ 0.01082 &  0.42025 $\pm$ 0.00243 \\
        Random Walk PE & 1.63014 $\pm$ 0.01129 & \textbf{0.41987 $\pm$ 0.00234} \\
        w/o PE & \textbf{1.62720 $\pm$ 0.00731} & 0.42202 $\pm$ 0.00399 \\ 
        \bottomrule
    \end{tabular}
    }
    \vspace{-1.5em}
    \label{table:pe}
\end{table}

\subsection{Comparasion of Different Fusion Strategies}

Our framework is based on hybrid fusion. Layer-wise speaking, we refer to our fusion strategy as \textit{concurrent} fusion since node representations simultaneously go through GNNs and transformers. In the case of protein nodes, information from protein nodes and gene nodes is fused after being processed by protein transformers and gene-protein (gene to protein) GNNs, respectively. In the following experiment, we compared the performance of layer-wise \textit{concurrent} fusion, \textit{GNN-first} fusion, and \textit{mixed} fusion. Still, in the case of protein nodes, we implemented \textit{GNN-first} fusion by first summing protein embeddings and the gene-protein GNNs outputs, and then passing through protein transformers. For \textit{mixed} fusion, we utilized \textit{concurrent} fusion on protein and gene nodes while applying \textit{GNN-first} fusion on cell nodes. We summarize the prediction RMSEs and the standard deviations across five runs in Table~\ref{tab:fusion}.

% Please add the following required packages to your document preamble:
% \usepackage{booktabs}
\begin{table}[h]
\centering
\vspace{-0.5em}
\caption{The prediction RMSEs for different Fusion strategies.}\label{tab:fusion}
\vspace{-1.2em}
\resizebox{0.48\textwidth}{!}{
\begin{tabular}{@{}lccc@{}}
\toprule
        & Concurrent          & GNN-first           & Mixed               \\ \midrule
CITE    & \textbf{1.62720}$\pm$\textbf{0.00731} & 1.63054$\pm$0.01018 & 1.63092$\pm$0.00631 \\
GEX2ADT & \textbf{0.41987}$\pm$\textbf{0.00234} & 0.42861$\pm$0.00094 & 0.42948$\pm$0.00158 \\ \bottomrule
\end{tabular}
}
\vspace{-1.2em}
\end{table}

As presented in the table, the original \textit{concurrent} fusion strategy exhibited superior performance compared to the other two fusion strategies. The \textit{concurrent} setting allows each modality to utilize intra-modal information prior to combining with other modalities, resulting in better performance since each modality holds varying importance for downstream tasks. However, the improvement in performance was not significant enough for the CITE. This observation is consistent with the findings of other experiments. Since the RNA zero rate in the CITE is considerably lower than that of the GEX2ADT, the significance of data-specific information (cell nodes) in the CITE outweighs the importance of other modalities, resulting in a relatively smaller performance boost.

\subsection{Handling Other Single-Cell Multimodal Prediction Tasks}

In this work, we focused on the specific task GEX2ADT (gene expression to protein levels) as a showcase. However, our framework is versatile and can be applied to other modality prediction tasks, i.e., the other three tasks mentioned in the NeurIPS 2021 competition~\cite{luecken2021sandbox} including ADT2GEX (protein levels to gene expression), GEX2ATAC (gene expression to chromatin accessibility) and ATAC2GEX (chromatin accessibility to gene expression). 
% Specifically, \method{} can be extended to:
% \begin{itemize}

\noindent \textbf{ADT2GEX:} In our framework, we constructed a multimodal heterogeneous graph consisting of gene, protein and cell nodes. For the GEX2ADT task, we removed the cell-protein edges to eliminate information leakage. Similarly, for the ADT2GEX task, we incorporate protein measurements into the cell-protein edges while removing the cell-gene edges. Cell embeddings were initialized by the reduced protein levels, and protein embeddings were initialized by the weighted sum of cell embeddings, where the weights are with the normalized protein levels. %With the same multimodal transformer module and a prediction layer, \method{} can predict gene expression from protein levels.

\noindent \textbf{GEX2ATAC and ATAC2GEX:} In the context of the GEX2ATAC and ATAC2GEX tasks, we removed the protein nodes from the multimodal graph. To initialize the cell embeddings for the GEX2ATAC task, we used the reduced gene expression values, while for gene embeddings, we computed a weighted sum of cell embeddings using normalized gene measurements as weights. For the ATAC2GEX task, we masked the cell-gene edges in the testing set. We initialized the cell embeddings using the reduced ATAC measurements, and the gene embeddings were randomly initialized.
% \end{itemize}

We have compared the performance of \method{} with scMoGNN on the four tasks in Table~\ref{tab:others}. Note that \method{} outperformed scMoGNN in tasks that involve protein modality and vice versa. These results suggest that incorporating prior information of protein nodes can enhance the performance of \method{} when protein modality is present. The RMSE scores across five runs are summarized as in Table~\ref{tab:others}.

% Please add the following required packages to your document preamble:
% \usepackage{booktabs}
\begin{table}[h]
\centering
\vspace{-0.5em}
\caption{Results on other Single-Cell Mutimodal Tasks.}\label{tab:others}
\vspace{-1.2em}
\resizebox{0.48\textwidth}{!}{
\begin{tabular}{l|cccc}
\toprule
           & GEX2ADT             & ADT2GEX             & GEX2ATAC            & ATAC2GEX            \\ \midrule
scMoFormer & \textbf{0.4198}7$\pm$\textbf{0.00234} & \textbf{0.31547}$\pm$\textbf{0.00184} & 0.17885$\pm$0.00008 & 0.23988$\pm$0.00031 \\
scMoGNN    & 0.42576$\pm$0.01180 & 0.32250$\pm$0.00136 & \textbf{0.17823}$\pm$\textbf{0.00011} & \textbf{0.23021}$\pm$\textbf{0.00219} \\ \bottomrule
\end{tabular}
}
\vspace{-1.8em}
\end{table}

\subsection{Training Efficiency of \method{}} \label{sec:efficiency}
Since scMoGNN is the best-performing baseline, we compared the running time and total GPU memory of \method{} and scMoGNN across five runs on one Quadro RTX 8000 GPU. The results are summarized in Table~\ref{tab:efficiency}. We observed that with higher GPU consumption, \method{} required a significantly shorter running time compared to scMoGNN. It is worth noting that \method{} was configured with a relatively large batch size of 8000 cells per batch and a hidden dimension of 512, in order to achieve better performance and better utilization of computing resources. One can surely reduce the required GPU memory of \method{} by limiting the setting accordingly. 

% Please add the following required packages to your document preamble:
% \usepackage{booktabs}
\begin{table}[h]
\centering
\vspace{-0.5em}
\caption{Efficiency Comparison}\label{tab:efficiency}
\vspace{-1.2em}
\resizebox{0.48\textwidth}{!}{
\begin{tabular}{@{}l|cc|cc@{}}
\toprule
\multicolumn{1}{l|}{} & \multicolumn{2}{l|}{\textbf{Running time (min)}} & \multicolumn{2}{l}{\textbf{GPU memory (GB)}}         \\ \midrule
                      & CITE                   & GEX2ADT                 & CITE                   & \multicolumn{1}{l}{GEX2ADT} \\
\method{}            & 24.82$\pm$4.16         & 17.95$\pm$3.56          & \multicolumn{1}{r}{38} & 21                          \\
scMoGNN               & 58.89$\pm$7.77         & 108.54$\pm$21.22        & \multicolumn{1}{r}{26} & 12                          \\ \bottomrule
\end{tabular}
}
\vspace{-0.8em}
\end{table}

\noindent \textbf{Can \method{} process large datasets?} In practical applications, it exhibits the capability to process large datasets. We address this issue from the perspectives of both the model and the data:
\begin{itemize}[leftmargin=*]
\item \textbf{Data aspect:} Due to technological or biological reasons, the number of RNA nodes and the number of protein nodes will be similar across datasets. Therefore, large datasets mean more cells, which can be handled by mini-batching cells.
\item \textbf{Model aspect:} In our multimodal transformer module, we employed linearized transformers~\cite{choromanski2021rethinking} with linear space and time complexity, which can be conveniently adapted to large datasets. Concerning GNNs, we incorporated GraphSAGE~\cite{hamilton2017inductive}, whose space and time complexity are related to the number of edges and hidden layer dimensions. Indeed, we can control the number of edges by mini-batching cells and make appropriate adjustments based on available computational resources.
\end{itemize}

\subsection{Ablation Study}
% To have a comprehensive understanding of our propose framework, we perform ablation study to analyze the influence of modules within \method{}. Specifically, we first examine the individual impact of multimodal transformer modules and later demonstrate the performance boost of transformers, equipped with prior knowledge, over GNNs. 
Table~\ref{table:main} demonstrates that models that incorporate domain knowledge perform better in modality prediction compared to those that do not. BABEL, ScMoGNN, and \method{} are the three models that make use of domain knowledge, and they show improved performance compared to the other two models. Among these three models, ScMoGNN, which is based on GNNs, performs better than BABEL, while \method{} outperforms all other models with its combination of transformers and GNNs framework. Given that \method{} includes a multimodal heterogeneous graph and three transformers, this raises the questions: \textit{Why no cell-cell graph and cell-protein graph? Which transformer has the biggest impact on performance? How much do the transformers contribute to the improvement in performance?}

\subsubsection{Exclusion of Cell-Cell Graph and Cell-Protein Graph.}\label{app:remark}
In Remark of Section~\ref{sec:subgraph}, we provided an explanation for our decision to exclude the cell-cell and cell-protein graphs. Additionally, through our empirical analysis, we observed a decrease in performance on both datasets when these links were incorporated into our heterogeneous graph. In comparison to the original w/o neither graph setting, the performance drop was evident in both the w/ cell-protein graph and w/ cell-cell graph settings, with the former demonstrating a more significant decline. The prediction RMSEs and standard deviations of five runs are summarized in Table~\ref{tab:graph_variants}.

% Please add the following required packages to your document preamble:
% \usepackage{booktabs}
\begin{table}[h]
\vspace{-0.5em}
\caption{The prediction RMSEs on different graph variants.}\label{tab:graph_variants}
\vspace{-1.2em}
\resizebox{0.48\textwidth}{!}{
\begin{tabular}{@{}lcccc@{}}
\toprule
        & w/o neither graph   & w/ cell-cell graph  & w/ cell-protein graph & w/ both graphs      \\ \midrule
CITE    & \textbf{1.62720}$\pm$\textbf{0.00731} & 1.68932$\pm$0.00751 & 1.71742$\pm$0.01692   & 1.70094$\pm$0.02207 \\
GEX2ADT & \textbf{0.41987}$\pm$\textbf{0.00234} & 0.42441$\pm$0.00094 & 0.43911$\pm$0.00414   & 0.42983$\pm$0.00231 \\ \bottomrule
\end{tabular}
}
\vspace{-1.2em}
\end{table}

\noindent \textbf{Why cell-cell graph did not help:} In the above experiment, we constructed a k-NN cell-cell graph by measuring the similarity in gene expression between cells. Nevertheless, since gene expression measurements often suffer from noise and sparsity, this static cell-cell graph may have introduced biased information into the downstream task. To address this issue, we utilized a cell transformer module to learn the dynamic cell-cell interactions via multi-head attention mechanism, which led to improved performance.

\noindent \textbf{Why cell-protein graph did not help:} For the cell-protein links, we incorporated target surface protein levels into edge weights, similar to how we built the cell-gene graph. However, these links conveys information about the prediction targets of the training set, causing the model to overfit easily. Hence, eliminating the cell-protein links served to eradicate information leakage.

\subsubsection{Influence of Every Transformer} The propose multimodal transformers consist of three different transformers, namely the cell transformer, gene transformer and protein transformers. As our predictions are based on the cell readout, it is expected that each of the three transformers will have different levels of impact on the performance. To quantify the specific impact of a single transformer, we conduct an experiment by removing the other two transformers and measuring the prediction RMSE scores. The results of the evaluation, including the scores of three partial models and the model with no transformers, are summarized in 
% Table~\ref{table:one_trans}.
Figure~\ref{fig:one_trans}.
% \begin{table}[tb]
    \centering
    \caption{Prediction RMSE results of keeping only one Transformer (score $\pm$ std).}
    \scalebox{1.}{
    \begin{tabular}{c|c|c}
        \toprule
        & \textbf{CITE} & \textbf{GEX2ADT} \\ \midrule
        Cell Transformer & 1.62994 $\pm$ 0.00967 & \textbf{0.41996 $\pm$ 0.00355} \\
        Gene Transformer & 1.62878 $\pm$ 0.00810 & 0.42670 $\pm$ 0.00191 \\
        Protein Transformer & \textbf{1.62742 $\pm$ 0.00781} & 0.42714 $\pm$ 0.00129 \\ 
        No Transformer & 1.63020 $\pm$ 0.00672 & 0.42731 $\pm$ 0.00138 \\ 
        \bottomrule
    \end{tabular}
    }
    \label{table:one_trans}
\end{table}
\begin{figure}[htb]%
    \vspace{-2.em}
    \centering
    \caption{RMSE$\downarrow$ results of keeping only one Transformer.}%
    \vspace{-1.2em}
    \subfloat[CITE]{{\includegraphics[width=0.5\linewidth]{images/cite_onetrans.png} }}%
    \subfloat[GEX2ADT]{{\includegraphics[width=0.5\linewidth]{images/gex_onetrans.png} }}%
    \label{fig:one_trans}%
    \vspace{-1.8em}
\end{figure}

% \vspace{-1.2em}
The performance of the gene transformer and protein transformer is better than that of the cell transformer in the CITE, while it is the opposite in the GEX2ADT. This can be explained by the difference in RNA zero rate between the two datasets, as shown in Table~\ref{table:dataset}. For the GEX2ADT, the high RNA zero rate means less information, making the cell transformer crucial in increasing performance by drawing more information from the data. On the other hand, the CITE has a lower zero rate, meaning it provides more information, allowing the gene transformer and protein transformer to enhance the model by adding external biological knowledge.

\subsubsection{How to Utilize Prior Knowledge} To answer the third question, we compare \method{} with two GNN-based models in Table~\ref{table:ablation}. The model "GNN-prior" refers to the GNNs that built on the same graph in Section~\ref{sec:graph_const}, while the "GNN" model is constructed using only the cell-gene graph without incorporating any prior information. The results show that the incorporation of prior knowledge into the graph results in a slight performance boost in both datasets. However, when multimodal transformers are included, the performance improvement is much more pronounced. This highlights the usefulness of prior knowledge and the importance of transformers to effectively incorporate this information into the model.
\begin{table}
  \centering 
  \setlength{\tabcolsep}{5mm}
  \caption{The ablation of different modules in Omni-Relational Network reported with MSE on MNIST given 10$\%$ context points by ablating (1) graph structure (2) attentive pooling (3) positional embedding (P.E.) (4) information bottleneck constrain (I.B.).}
  \begin{tabular}{lcccc}
    \toprule
      Graph       & Attention       & P.E.            & I.B.             & MSE \\
    \midrule
                  &                 &                &                   & $0.073$\\
    $\checkmark$   &              &                 &                    & $0.066$\\
    $\checkmark$  &               &               &    $\checkmark$    & $0.058$\\
    $\checkmark$  &                 &   $\checkmark$     &  $\checkmark$   & $0.053$\\
    $\checkmark$  &  $\checkmark$  &               &     $\checkmark$   & $0.049$\\
    $\checkmark$  &  $\checkmark$  &   $\checkmark$    &  $\checkmark$   & $0.045$\\
    \bottomrule
  \end{tabular}
  \label{tab:3-ablation}
\end{table}








