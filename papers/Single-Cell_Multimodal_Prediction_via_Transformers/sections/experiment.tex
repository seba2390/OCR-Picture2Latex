\section{Experiment}
In this section, we present the experimental results of \method{} against baselines on benchmark datasets. In particular, we aim to answer the following questions:
\begin{itemize}[leftmargin=*]
    \item \textbf{RQ1:} How does \method{} perform compare against baselines based on various evaluation metrics?
    \item \textbf{RQ2:} Given various choices of the PEs, how do they affect the performance of \method{}?
    \item \textbf{RQ3:} What is the best way to fuse information across modalities?
    \item \textbf{RQ4:} Can \method{} handle other modality prediction tasks?
    \item \textbf{RQ5:} Can \method{} process large datasets?
    \item \textbf{RQ6:} How does each of the model component impact the performance of \method{}?
\end{itemize}
Before presenting our experimental results and observations, we first introduce the experimental settings.
% We evaluate the ablation of remove transformers.

\subsection{Experimental Settings}
\subsubsection{Datasets}
We follow the setting of the NeurIPS multimodal single-cell integration competition of the year 2021~\cite{luecken2021sandbox} and 2022
% \footnote{https://nips.cc/virtual/2022/competition/50092} 
and collect the joint measurements of gene expression and surface protein levels datasets from the competitions. Both datasets contain the raw counts, which represent the number of reads per gene per cell, as well as the normalized counts. For the NeurIPS 2021 competition, we pick the data corresponding to the task of protein abundance prediction via gene expression and refer to it as ``GEX2ADT''. The processed data is centered and log-transformed for denoising purposes. For the competition in 2022, which we refer to as ``CITE'', the objective is to utilize CITE-seq~\cite{stoeckius2017simultaneous} data measured from days 2, 3, and 4 to predict the protein level on day 7 from different individuals. It is worth mentioning that the protein level testing data is not available during the completion of this work. Therefore, we simulate the competition scenario by treating the training data from day 4 as our testing set. The processed RNA data is centered and log-transformed, while the normalized protein levels are denoised and scaled by background~\cite{kotliarov2020broad}. We summarize the dataset details in Table~\ref{table:dataset} in Appendix~\ref{app:exp}. 
% The experimental results on CBMC are shown in Appendix~\ref{app:cbmc}.
% \section{Dataset}
\label{sec:dataset}
%\sarah{add statistics about distribution of merge patterns}
%\alexey{I added some numbers in the section 4 (around line 270). Detailed numbers are in Appendix. We can move it up here if needed...}
%To create a dataset for self-supervised pretraining, we clone all non-fork repositories with more than 20 stars in GitHub that have C, C++, C\#, Python, Java, JavaScript, TypeScript, PHP, Go, and Ruby as their top language. The resulting dataset comprises over 64 million source code files. 
%\chris{why do we list languages here that we don't ever evaluate on?  A reviewer will find this confusing and ask about it.  We found that language specific models work better than multi-lingual models, right?}

The finetuning dataset is mined from over 100,000 open source software repositories in multiple programming languages with merge conflicts. It contains commits from git histories with exactly two parents, which resulted in a merge conflict.  We replay \texttt{git merge} on the two parents to see if it generates any conflicts. Otherwise, we ignore the merge from our dataset. We use the approach introduced by~\citet{Dinella2021} to extract resolution regions---however, we do not restrict ourselves to conflicts with less than 30 lines only.  Lastly, we extract token-level conflicts and conflict resolution classification labels (introduced in Section \ref{formulation}) from line-level conflicts and resolutions. Tab.~\ref{tab:fintuning_dataset} provides a summary of the finetuning dataset.

\begin{table}[htb]
\centering
\caption{Number of merge conflicts in the dataset.}
\begin{tabular}{llllllllllll} \toprule
\textbf{Programming language} & \textbf{Development set}  & \textbf{Test set} \\ \midrule
C\# & 27874 & 6969 \\ 
JavaScript & 66573 & 16644\\ 
TypeScript & 22422 & 5606\\ 
Java & 103065 & 25767 \\ 
\bottomrule
\end{tabular}
\label{tab:fintuning_dataset}
\end{table}
The finetuning dataset is split into development and test sets in the proportion 80/20 at random at the file-level. The development set is further split into training and validation sets in 80/20 proportion at the merge conflict level.    


\subsubsection{Baselines}
We evaluate the performance of \method{} against state-of-art multimodal prediction models among the task of using gene expression to predict surface protein levels. The selected baselines are as follows:
\begin{itemize}[leftmargin=*]
    \item \textbf{Cross-modal Autoencoders}~\cite{yang2021multi}, short for CMAE, incorporated multiple autoencoders to integrate multimodal data and utilized domain knowledge by adding discriminative loss to the training process to align shared markers or clusters.
    \item \textbf{BABEL}~\cite{wu2021babel} proposed a general framework for multimodal translation with modality-specific encoders and decoders. Note that initially, BABEL focused on RNA and ATAC-seq~\cite{buenrostro2013transposition} data. In this evaluation, we repurpose BABEL to the RNA to protein setting.
    \item \textbf{scMM}~\cite{minoura2021mixture} modeled the multimodal data with generative setting. We note that the input of scMM is restricted to raw counts, and the predictions are scaled as centered log-transformed data.
    \item \textbf{ScMoGNN}~\cite{wen2022graph} involved domain knowledge like biological pathways to enhance the GNNs. The original ScMoGNN followed a transductive setting. In this work, we implement an inductive setting of ScMoGNN for a fair comparison with the baselines.
\end{itemize}

\subsubsection{Parameter Setting}
To benchmark the performance of baselines and \method{}, we uniformly employ inductive settings among both datasets. On the CITE, we use the data measured on day 4 for testing and randomly split $80/20\%$ of the data prior to day 4 for training and validation. On the GEX2ADT, we randomly pick $15\%$ of the training data for validation and evaluate the predictions on the testing set. For BABEL, the hidden dimension is tuned from $\{16, 32, 64, 128\}$. For CMAE, the weights of adversarial loss and reconstruction loss are chosen from $\{0.1, 1, 2.5, 5, 10\}$. For scMM, the hidden dimensions are tuned from $\{16, 32, 64, 128\}$. For ScMoGNN, the weight decay parameter of the optimizer is tuned from $\{5\times 10^{-6},1 \times 10^{-5}, 5 \times 10^{-5}, 1 \times 10^{-4}\}$.

\subsection{Evaluation of Predictions}
\section{Main Results}\label{main_results}
In this section, we first introduce a mild ``informative'' condition for unsupervised pretraining. We show this ``informative'' condition is necessary for pretraining to benefit downstream tasks. We then provide our main results---statistical guarantees for unsupervised pretraining and downstream tasks for Algorithm \ref{mle+erm}. Finally, in Section \ref{weak_informative_model}, we generalize our results to a more technical but weaker version of the ``informative'' condition, which turns out to be useful in capturing our third example of contrastive learning (Section \ref{contrastive_learning}).


% the notion of $\kappa^{-1}$-informative model, which guarantees the utility of unsupervised pretraining for downstream tasks learning. We then present a statistical guarantee for the MLE step in line 2 of Algorithm \ref{mle+erm} and upper bound excess risk of Algorithm \ref{mle+erm}. Our theoretical result rigorously shows the advantage of unsupervised pretraining in the sense of reducing the sample complexity. In Section \ref{weak_informative_model}, we show that the same theoretical guarantee can be achieved under a weaker version of $\kappa^{-1}$-informative model assumption.

% In Section \ref{informative_model}, we first introduce the notion of $\kappa^{-1}$-informative model, which guarantees the utility of unsupervised pretraining for downstream tasks learning. We then present a statistical guarantee for the MLE step in line 2 of Algorithm \ref{mle+erm} and upper bound excess risk of Algorithm \ref{mle+erm}. Our theoretical result rigorously shows the advantage of unsupervised pretraining in the sense of reducing the sample complexity. In Section \ref{weak_informative_model}, we show that the same theoretical guarantee can be achieved under a weaker version of $\kappa^{-1}$-informative model assumption.


% \subsection{Theoretical Guarantee for \texorpdfstring{$\kappa^{-1}$}{Lg}-Informative Model}\label{informative_model}

\paragraph{Informative pretraining tasks.}
We first note that under our generic setup, unsupervised pretraining may not benefit downstream tasks at all in the worst case if no further conditions are assumed.
%\paragraph{$\kappa^{-1}$-Informative}
% Notice that, further suitable assumption is necessary for revealing the nature of unsupervised pretraining.
\begin{proposition} \label{prop:informative_necessary}
There exist classes $(\Phi, \Psi)$ as in Section \ref{sec:prob_setup} such that, regardless of unsupervised pretraining algorithms used, pretraining using unlabeled data provides no additional information towards learning predictor $g_{\phi^* , \psi^* }$.
\end{proposition}
Consider the latent variable model $z=Ax$, where $x \sim \mathcal{N}(0,I_{d})$, $A\in \Phi$ is the parameter of the model. Then, no matter how many unlabeled $\{x_{i}\}$ we have, we can gain no information of $A$ from the data! In this case, unsupervised pretraining is not beneficial for any downstream task. 

Therefore, it's crucial to give an assumption that guarantees our unsupervised pretraining is informative. As a thought experiment, suppose that in the pretraining step, we find an exact density estimator $\hat{\phi}$ for the marginal distribution of $x,s$ , i.e., $p_{\hat{\phi}}(x,s)=p_{\phi^{*}}(x,s)$ holds for every $x,s$. We should expect that this estimator also fully reveals the relationship between $x$ and $z$, i.e., $p_{\hat{\phi}}(x,z)=p_{\phi^{*}}(x,z)$ holds for every $x,z$. Unfortunately, this condition does not hold in most practical setups and is often too strong. As an example, consider Gaussian mixture models, where $z \in [K]$ is the cluster that data point $x \in \mathbb{R}^{d}$ belongs to. Then in this case, it is impossible for us to ensure $p_{\hat{\phi}}(x,z)=p_{\phi^{*}}(x,z)$, since a permutation of $z$ makes no difference in the marginal distribution of $x$. However, notice that in many circumstances, a permutation of the class label will not affect the downstream task learning. In these cases, a permutation of the clusters is allowed. Motivated by this observation, we introduce the following informative assumption which allows certain ``transformation'' induced by the downstream task:

%As a thought experiment, suppose that in the pretraining step, we find an exact density estimatior $\hat{\phi}$ for marginal distribution of $x$ , i.e., $p_{\hat{\phi}}(x)=p_{\phi^{*}}(x)$ holds for every $x$. We should expect that this estimator $\hat{\phi}$ can serve as the ground truth $\phi^{*}$ in our second step. Notice that the notion "serve" is related to the structure of the downstream task: if for every $\psi \in \Psi$, there exists $\tilde{\psi}$ such that $p_{\hat{\phi}, \tilde{\psi}}(x,y)=p_{\phi^{*},\psi}(x,y)$ holds for every $x,y$, then $\hat{\phi}$ and $\phi^{*}$ can be treated as the same in the downstream task learning. In fact, here we are assuming that our model allows some kinds of transferability. For example, in Gaussian mixture models, we often want to find the clusters for unlabeled data. Permutation of the clusters are allowed, since a permutation of class label will not affect the downstream task learning in many circumstances.  We formalize this idea, by assuming our model satisfies the following transferability assumption.
%\jiawei{Add $z=Ax$ does not work.}



\begin{assumption}[$\kappa^{-1}$-informative condition]\label{invariance}
We assume that the model class $\Phi$ is $\kappa^{-1}$-informative with respect to a transformation group $\mathcal{T}_{\Phi}$. That is, for any $\phi\in\Phi$, there exists $T_1\in\mT_{\Phi}$ such that
\%\label{informative}
\TV\big(\P_{T_1\circ\phi}(x,z),\P_{\phi^* }(x,z)\big)\leq  \kappa \cdot\TV\big(\P_{\phi}(x,s),\P_{\phi^* }(x,s)\big).
\%
Here $\phi^* $ is the ground truth parameter. Furthermore, we assume that $\mathcal{T}_{\Phi}$ is induced by transformation group $\mathcal{T}_{\Psi}$ on $\Psi$, i.e., for any $T_1\in\mT_{\Phi}$, there exists $T_2\in\mT_{\Psi}$ such that for any $(\phi,\psi)\in\Phi\times\Psi$,
\%\label{101701}
\P_{\phi,\psi}(x,y)=\P_{T_1\circ\phi,T_2\circ\psi}(x,y).
\%
\end{assumption}


% \chijin{think about alternative name.}

% \jiawei{$\kappa^{-1}$-informative, after the assumption add: sometimes $x$ is not informative but adding $s$ makes $(x,s)$ informative}

% \begin{assumption}[$\kappa^{-1}$-Informative]\label{invariance}
% We say the model is $\kappa^{-1}$-informative iff for any $\phi\in\Phi$, there exists $\psi\in\Psi$ such that
% \%
% \TV\big(\P_{\phi,\psi}(x,y),\P_{\phi^* ,\psi^* }(x,y)\big)\leq \kappa \cdot\TV\big(\P_{\phi}(x),\P_{\phi^* }(x)\big).
% \%
% Here we denote by $\phi^* ,\psi^* $ the ground truth parameters.
% \end{assumption}
% \chijin{think about alternative name.}

% \jiawei{$\kappa^{-1}$-informative, after the assumption add: sometimes $x$ is not informative but adding $s$ makes $(x,s)$ informative}


% We make the following remarks on the above transferability assumption:
% \begin{itemize}
%     \item Suppose that $\mT_{\Phi}$ and $\mT_{\Psi}$ are two transformation groups defined on $\Phi$ and $\Psi$, respectively. And for any $T_1\in\mT_{\Phi}$, there exists $T_2\in\mT_{\Psi}$ such that for any $(\phi,\psi)\in\Phi\times\Psi$,
%     \%\label{101701}
%     \P_{\phi,\psi}(x,y)=\P_{T_1\circ\phi,T_2\circ\psi}(x,y).
%     \%
%     Then the transferability in $z$ implies the transferability in the model. To be specific, if we further have for any $\phi\in\Phi$, there exits a transformation $T_1\in\mT_{\Phi}$ such that 
%     \%\label{101702}
%     \TV\big(\P_{T_1\circ\phi}(x,z),\P_{\phi^* }(x,z)\big)\leq  \kappa \cdot\TV\big(\P_{\phi}(x),\P_{\phi^* }(x)\big),
%     \%
%     then the model has $\kappa$-transferability.
%     \jiawei{Assumption 4.1: (8), the transformation group we consider is defined as (7)}. To see this, for any $\phi\in\Phi$, we choose $T_1$ that satisfies \eqref{101702} and $T_2$ that satisfies \eqref{101701}. Let $\psi=T^{-1}_2\circ\psi^* $. It then holds that
%     \$
%     \TV\big(\P_{\phi,\psi}(x,y),\P_{\phi^* ,\psi^* }(x,y)\big)&=\TV\big(\P_{T_1\circ\phi,\psi^* }(x,y),\P_{\phi^* ,\psi^* }(x,y)\big)\\
%     &\leq\TV\big(\P_{T_1\circ\phi}(x,z),\P_{\phi^* }(x,z)\big)\leq \kappa \cdot\TV\big(\P_{\phi}(x),\P_{\phi^* }(x)\big).
%     \$
%     \item Note that the MLE step actually gives us a theoretical guarantee on the hellinger distance between the estimated distribution and the ground truth distribution, which is stronger than the TV distance guarantee. Thus, we can weaken our $\kappa$-transferability assumption to a hellinger distance version, i.e., for any $\phi\in\Phi$, there exists $\psi\in\Psi$ such that
%     \%\label{weak_ti}
%     \TV\big(\P_{\phi,\psi}(x,y),\P_{\phi^* ,\psi^* }(x,y)\big)\leq \kappa \cdot H\big(\P_{\phi}(x),\P_{\phi^* }(x)\big).
%     \%
% %     We refer to \eqref{weak_ti} as weak $\kappa$-transferability. Similarly, weak $\kappa$-transferability in $z$ implies weak $\kappa$-transferability in the model. See Appendix \ref{proof_contrastive_learning} for the details.
%     \item For settings with side information, we say the model has weak $\kappa$-transferability with side information iff for any $\phi\in\Phi$ there exists $\psi\in\Psi$ such that
%     \%\label{weak_ti_side}
%     \TV\big(\P_{\phi,\psi}(x,y),\P_{\phi^* ,\psi^* }(x,y)\big)\leq \kappa \cdot H\big(\P_{\phi}(x,s),\P_{\phi^* }(x,s)\big).
%     \%
%     See section \ref{contrastive_learning} for further details.
% \end{itemize}


% \begin{assumption}[Translation Invariance]\label{invariance}
% There exists transformation groups $\mT_{\Phi}$ and $\mT_{\Psi}$ defined on $\Phi$ and $\Psi$ such that the following properties hold:
% \begin{itemize}
%      \item For any $(\phi,\psi,T_1)\in\Phi\times\Psi\times\mT_{\Phi}$, there exists $T_2\in\mT_{\Psi}$ 
%     \$
%     \P_{\phi,\psi}(x,y)=\P_{T_1\circ\phi,T_2\circ\psi}(x,y).
%     \$
%     \item For any $\phi\in\Phi$, there exits a transformation $T_1\in\mT_{\Phi}$ such that 
%     \$
%     \TV\big(\P_{T_1\circ\phi}(x,z),\P_{\phi^* }(x,z)\big)\leq  \kappa \cdot\TV\big(\P_{\phi}(x),\P_{\phi^* }(x)\big)
%     \$ 
%     for some model-related parameters $c$.
% \end{itemize}
% \end{assumption}
% Assumption \ref{invariance} implies, for any $\phi\in\Phi$, there exists $\psi \in \Psi$ such that
% \$
% \TV\big(\P_{\phi,\psi}(x,y),\P_{\phi^* ,\psi^* }(x,y)\big)\leq \kappa \cdot\TV\big(\P_{\phi}(x),\P_{\phi^* }(x)\big).    
% \$
% \chijin{make this equation as our main assumption. Remark 1: often it suffices to check transferrability in $z$; Remark 2: Hellinger}
Under Assumption \ref{invariance}, if the pretrained $\hat{\phi}$ accurately estimates the marginal distribution of $x,s$ up to high accuracy, then it also reveals the correct relation between $x$ and representation $z$ up to some transformation $\mathcal{T}_{\Phi}$ which is allowed by the downstream task, which makes it possible to learn the downstream task using less labeled data. 

Proposition \ref{prop:informative_necessary} shows that the informative condition is necessary for pretraining to bring advantage since the counter example in the proposition is precisely $0$-informative. We will also show this informative condition is rich enough to capture a wide range of unsupervised pretraining methods in Section \ref{factor_model}, \ref{gmm}, \ref{contrastive_learning}, including factor models, Gaussian mixture models, and contrastive learning models.

% Therefore it is possible to learn a good $\hat{\psi}$ that captures the relationship between $x$ and $y$ based on the pretrained $\hat{\phi}$. We will show that this informative assumption holds for widely used models such as factor models, Gaussian mixture models and contrastive learning models.



%%%%%%%%%%%%%%%%%%%%%%%%%%%%%%%%%%%%%%%%%%%%%%%%%%%%%%%%%%%%%%%%%%%%%%%%%%%%%%%%%%%%%%%%%%%%%%%%%%%%%%%%%%%%%%%%%%%%%%%%%%%%%%%%%%%%%%%%%%%%%%%%%%%%%%%%%%%%%%%%%%%%%%%%%%%%%%%%%%%%%%%


% \subsection{Unsupervised Pretraining}
\paragraph{Guarantees for unsupervised pretraining.}
% Given unlabeled data $\{x_i\}^m_{i=1}$, we estimate $\phi^* $ by $\hat\phi$ via MLE, i.e.,
% \%\label{mle}
% \hat\phi\leftarrow\argmax_{\phi\in\Phi}\sum^m_{i=1}\log p_{\phi}(x_i).
% \%
Recall that $\mP_{\mathcal{X}\times\mathcal{S}}(\Phi):=\{p_{\phi}(x,s)\,|\,\phi\in\Phi\}$. We have the following guarantee for the MLE step (line 2) of Algorithm \ref{mle+erm}.

\begin{theorem}\label{tv_mle}
Let $\hat\phi$ be the maximizer defined in \eqref{mle}. Then, with probability at least $1-\delta$, we have
\$
\TV\big(\P_{\hat\phi}(x,s),\P_{\phi^* }(x,s)\big)\leq 3\sqrt{\frac{1}{m}\log\frac{N_{\b}(\mP_{\mathcal{X}\times\mathcal{S}}(\Phi),\frac{1}{m})}{\delta}},
\$
where $N_{\b}$ is the bracketing number as in Definition \ref{def:bracketing}.
\end{theorem}

Theorem \ref{tv_mle} claims that the TV error in estimating the joint distribution of $(x, s)$ decreases as $\mathcal{O}(\mathcal{C}_\Phi/m)$ where $m$ is the number of unlabeled data, and $\mathcal{C}_\Phi = \log N_{\b}(\mP_{\mathcal{X}\times\mathcal{S}}(\Phi), 1/m)$ measures the complexity of learning the latent variable models $\Phi$. 
This result mostly follows from standard analysis of MLE \citep{van2000empirical}. We include the proof in Appendix \ref{main1} for completeness.  If the model is $\kappa^{-1}$-informative, Theorem \ref{tv_mle} further implies that with probability at least $1-\delta$,
\$
&\min_{\psi}\E_{\P_{\phi^* ,\psi^* }}\big[\ell\big(g_{\hat\phi,\psi}(x),y\big)\big]-\E_{\P_{\phi^* ,\psi^* }}\big[\ell\big(g_{\phi^* ,\psi^* }(x),y\big)\big]\leq 12\kappa L\sqrt{\frac{1}{m}\log\frac{N_{\b}(\mP_{\mathcal{X}\times\mathcal{S}}(\Phi),1/m)}{\delta}}.
\$
See Lemma \ref{error_dtv} for the details. This inequality claims that if we learn a \emph{perfect} downstream predictor using the estimated representation $\hat{\phi}$, excess risk is small.





%%%%%%%%%%%%%%%%%%%%%%%%%%%%%%%%%%%%%%%%%%%%%%%%%%%%%%%%%%%%%%%%%%%%%%%%%%%%%%%%%%%%%%%%%%%%%%%%%%%%%%%%%%%%%%%%%%%%%%%%%%%%%%%%%%%%%%%%%%%%%%%%%%%%%%%%%%%%%%%%%%%%%%%%%%%%%%%%%%%%%%%
\paragraph{Guarantees for downstream task learning.}
% Let $\hat\phi$ be the pretrained estimator. In the learning of downstream tasks, we fix $\hat\phi$ and estimate $\psi^* $ using the labeled data $\{x_j,y_j\}^n_{j=1}$ via ERM, i.e.,
% \$
% \hat\psi\leftarrow\argmin_{\psi\in\Psi}\sum^n_{j=1}\ell\big(g_{\hat\phi,\psi}(x_j),y_j\big).
% \$
In practice, we can only learn an approximate downstream predictor using a small amount of labeled data. We upper bound the excess risk of Algorithm \ref{mle+erm} as follows.
\begin{theorem}\label{error_bound}
Let $\hat\phi$ and $\hat\psi$ be the outputs of Algorithm \ref{mle+erm}. Suppose that the loss function $\ell:\mathcal{Y}\times\mathcal{Y}\rightarrow \R$ is $L$-bounded and our model is $\kappa^{-1}$-informative. Then, with probability at least $1-\delta$, the excess risk of Algorithm \ref{mle+erm} is bounded as:
\$
{\rm Error}_{\ell}(\hat\phi,\hat\psi)&\leq 2\max_{\phi\in\Phi} R_n(\ell\circ \mathcal{G}_{\phi,\Psi})+12\kappa L\cdot\sqrt{\frac{1}{m}\log\frac{2N_{\b}(\mP_{\mathcal{X}\times\mathcal{S}}(\Phi),1/m)}{\delta}}+2L\cdot\sqrt{\frac{2}{n}\log\frac{4}{\delta}}.
\$
Here $R_n(\cdot)$ denotes the Rademacher complexity, and
\$
\ell\circ \mathcal{G}_{\phi,\Psi}:=\big\{\ell\big( g_{\phi,\psi}(x),y\big):\mathcal{X}\times\mathcal{Y}\rightarrow [-L,L]\,\big|\,\psi\in\Psi\big\}.
\$
\end{theorem}

Note that the Rademacher complexity of a function class can be bounded by its metric entropy. We then have the following corollary.
\begin{corollary}\label{rc_covering}
Under the same preconditions as Theorem \ref{error_bound}, we have:
% Let $\hat\phi$ and $\hat\psi$ be the outputs of Algorithm \ref{mle+erm}. Suppose that the loss function $\ell:\mathcal{Y}\times\mathcal{Y}\rightarrow \R$ is $L$-bounded and our model is $\kappa^{-1}$-informative. Then, with probability at least $1-\delta$, the excess risk of Algorithm \ref{mle+erm} can be bounded as follows,
\$
{\rm Error}_{\ell}(\hat\phi,\hat\psi)&\leq \tilde{c} \max_{\phi\in\Phi} L \sqrt{\frac{\log N (\ell\circ \mathcal{G}_{\phi,\Psi},L/\sqrt{n},\|\cdot\|_{\infty})}{n}}+2L\sqrt{\frac{2}{n}\log\frac{4}{\delta}}\notag\\
 &~+12\kappa L\sqrt{\frac{1}{m}\log\frac{2N_{\b}(\mP_{\mathcal{X}\times\mathcal{S}}(\Phi),1/m)}{\delta}}\notag,
\$
where $\tilde{c}$ is an absolute constant, $N(\mathcal{F}, \delta, \| \cdot \|_{\infty})$ is the $\delta-$covering number of function class $\mathcal{F}$ with respect to the metric $\| \cdot \|_{\infty}$.
\end{corollary} 
By Corollary \ref{rc_covering}, 
%we can see that our excess risk bound can be decomposed into two parts. The first term is $\tilde{\mathcal{O}}(\sqrt{\log N (\ell\circ \mathcal{G}_{\phi,\Psi})/n}) $, and the second term is $\tilde{\mathcal{O}}(\sqrt{\log N_{[]} (\mP(\Phi))/n}) $. Notice that $\ell\circ \mathcal{G}_{\phi,\Psi}$ is a function class that only the index $\psi$ varies, $\log N (\ell\circ \mathcal{G}_{\phi,\Psi})$ can roughly be view as the complexity measure of class $\Psi$. Similarly $\log N_{[]} (\mP(\Phi))$ can be view as the complexity of $\Phi$. Therefore, our results shows that
the excess risk of our Algorithm \ref{mle+erm} is approximately $\tilde{\mathcal{O}}(\sqrt{\mathcal{C}_\Phi/m} + \sqrt{\mathcal{C}_\Psi/n})$, where $\mathcal{C}_\Phi$ and $\mathcal{C}_\Psi$ are roughly the log bracketing number of class $\Phi$ and the log covering number of $\Psi$. Note that excess risk for the baseline algorithm that learns downstream task using only labeled data is $\tilde{\mathcal{O}}( \sqrt{\mathcal{C}_{\Phi \circ \Psi}/n})$, where $\mathcal{C}_{\Phi \circ \Psi}$ is the log covering number of composite function class $\Phi \circ \Psi$.  In many practical scenarios such as training a linear predictor on top of a pretrained deep neural networks, the complexity $\mathcal{C}_{\Phi \circ \Psi}$ is much larger than $\mathcal{C}_{\Psi}$. We also often have significantly more unlabeled data than labeled data ($m \gg n$). In these scenarios, our result rigorously shows the significant advantage of unsupervised pretraining compared to the baseline algorithm which directly performs supervised learning without using unlabeled data.
% the performance of downstream task learning with unsupervised pretraining is significantly better than the baseline of supervised learning with no pretraining.

%\jiawei{Add subsection: weak $\kappa$-informative (10), argue why this is weaker than Assumption 4.1 (Hellinger weaker than TV, $(x,y)$ better than $(x,z)$ guarantee)}

% Given unlabeled data $\{x_i\}^m_{i=1}$ and labeled data $\{x_j,y_j\}^n_{j=1}$, another natural scheme is to use a two-phase MLE. To be specific, in the first phase, we use MLE to estimate $\phi^* $ based on the unlabeled data $\{x_i\}^m_{i=1}$. In the second phase, we use MLE again to estimate $\psi^* $ based on pretrained $\hat\phi$ and the labeled data $\{x_j,y_j\}^{n}_{j=1}$. However, we can show that this two-phase MLE scheme fails in some cases. See Appendix \ref{counter_example} for the details.


% To see what's the value of this risk bound, let's compare it to the usual bound for learning downstream task using only labeled data. The standard risk bound for supervised learning is given by $\tilde{\mathcal{O}}( \sqrt{\mathcal{C}_{\Phi \circ \Psi}/n})$. In some circumstances $\mathcal{C}_{\Phi \circ \Psi}$ is much larger than $\mathcal{C}_{\Psi}$. Therefore, when $m$ is sufficiently large and $n$ is relatively small (i.e., unlabeled data is abundant, while labeled data is scarce), the performance of downstream task learning with unsupervised pretraining is significantly better than the baseline of supervised learning with no pretraining.



%%%%%%%%%%%%%%%%%%%%%%%%%%%%%%%%%%%%%%%%%%%%%%%%%%%%%%%%%%%%%%%%%%%%%%%%%%%%%%%%%%%%%%%%%%%%%%%%%%%%%%%%%%%%%%%%%%%%%%%%%%%%%%%%%%%%%%%%%%%%%%%%%%%%%%%%%%%%%%%%%%%%%%%%%%%%%%%%%%%%%%%
\subsection{Guarantees for weakly informative models}\label{weak_informative_model}

%\chijin{whole paper: weak informative $\rightarrow$ weakly informative}
%\subsection{Weak $\kappa^{-1}$-Informative}

We introduce a relaxed version of Assumption \ref{invariance}, which allows us to capture a richer class of examples.
% such as contrastive learning.
\begin{assumption}[$\kappa^{-1}$-weakly-informative condition]\label{weak_invariance}
We assume model $(\Phi, \Psi)$ is $\kappa^{-1}$-weakly-informative, that is, for any $\phi\in\Phi$, there exists $\psi\in\Psi$ such that
\%\label{weak_informative}
\TV\big(\P_{\phi,\psi}(x,y),\P_{\phi^* ,\psi^* }(x,y)\big)\leq \kappa \cdot H\big(\P_{\phi}(x,s),\P_{\phi^* }(x,s)\big).
\%
Here we denote by $\phi^{*}, \psi^{*}$ the ground truth parameters.
\end{assumption}

Assumption \ref{weak_invariance} relaxes Assumption \ref{invariance} by making two modifications: (i) replace the LHS of \eqref{informative} by the TV distance between the joint distribution of $(x, y)$; (ii) replace the TV distance on the RHS by the Hellinger distance. See more on the relation of two assumptions in Appendix \ref{relation}.

% \chijin{Move these justifications to appendix}
% {\color{blue}
% Assumption \ref{weak_invariance} is actually a relaxation of Assumption \ref{invariance}. To see this, by Assumption \ref{invariance}, for any $\phi\in\Phi$, we choose $T_1$ that satisfies \eqref{informative} and $T_2$ that satisfies \eqref{101701}. Let $\psi=T^{-1}_2\circ\psi^* $. It then holds that
% \$
% &\TV\big(\P_{\phi,\psi}(x,y),\P_{\phi^* ,\psi^* }(x,y)\big)\notag\\
% &=\TV\big(\P_{T_1\circ\phi,\psi^* }(x,y),\P_{\phi^* ,\psi^* }(x,y)\big)\\
% &\leq\TV\big(\P_{T_1\circ\phi}(x,z),\P_{\phi^* }(x,z)\big)\notag\\
% &\leq \kappa \cdot\TV\big(\P_{\phi}(x,s),\P_{\phi^* }(x,s)\big).
% \$
% Note that the TV distance can be upper bounded by the Hellinger distance. Thus, Assumption \ref{invariance} directly implies Assumption \ref{weak_invariance}.}

% It can be seen that the MLE step in line 2 of Algorithm \ref{mle+erm} actually gives us a theoretical guarantee on the Hellinger distance between the estimated distribution and the ground truth distribution. Thus, 
In fact, Assumption \ref{weak_invariance} is sufficient for us to achieve the same theoretical guarantee as that in Theorem \ref{error_bound}.

\begin{theorem}\label{weak_error_bound}
Theorem \ref{error_bound} still holds under the $\kappa^{-1}$-weakly-informative assumptions.
\end{theorem}

The proof of Theorem \ref{weak_error_bound} requires a stronger version of MLE guarantee than Theorem \ref{tv_mle}, which guarantees the closeness in terms of Hellinger distance. We leave the details in Appendix \ref{main4}.

%\chijin{Appendix ???, cite the place for the proof of this theorem}.
We evaluate the final protein-level prediction performance using Root Mean Square Error (RMSE) and Mean Absolute Error (MAE). Meanwhile, because multimodal data usually suffers from the influence of batch effects and unbalanced measuring depth, the count's scale of each cell may vary significantly, which will substantially affect the RMSE and MAE metrics. Therefore, we also include the Pearson correlation coefficient (Corr), which is cell-wise normalized by the mean and variance of the input, as a robust and scale-free metric to evaluate the predictions. A lower RMSE or MAE score indicates a geometrically closer estimation of the protein levels, while a higher Corr score suggests a statistically more similar match to the actual value. We report the mean and the standard deviation of each metric across five different runs, and the results are illustrated in Table~\ref{table:main}. The best performance is highlighted in bold. 

To answer the first question, we note that our \method{} consistently outperforms all other baselines according to all three metrics on both datasets, indicating that \method{} successfully captures the quantitative characteristics of target protein levels given the input gene expression measurements. Particularly, for the CITE, \method{} achieved significantly lower RMSE compared to the second-best model ScMoGNN, by $0.04$. More importantly, \method{} achieved a significant improvement in terms of the Pearson correlation metrics over all other baselines, with a noticeably lower performance variation across runs.
% indicating the stability of our model.

We further analyze the performance of different models on proteins that are least well captured by any models. Specifically, for each model, we compute the RMSE for each protein separately and identify ten proteins that resulted in the highest average RMSE across all models. As shown in Figure~\ref{fig:bot10comp}, \method{} and ScMoGNN achieved relatively stable results and are consistently better compared to BABEl and CMAE.

\begin{figure}[htb]
    \centering
    \vspace{-1.2em}
    \caption{Least well-predicted protein comparison.}
    % \vspace{-1em}
    \includegraphics[width=0.47\textwidth]{images/bot10_perf_comp_boxplot.png}
    \vspace{-1.2em}
    \label{fig:bot10comp}
\end{figure}

\vspace{-0.5em}
\subsection{Positional Encoding}
\label{sec:pe}
As mentioned in Section~\ref{sec:graph_trans}, we implement Laplacian PE~\cite{dwivedi2020generalization} and random walk PE~\cite{dwivedi2022graph} to capture positional information of prior knowledge graph. For ease of notation, we use the abbreviation PE to refer to the PE. To benchmark the impact of the two types of PE among two datasets and answer the second question, we show the performance of \method{} with each PE and compare them with the scenario without any PE. The mean and standard deviation of RMSE scores of five runs are shown in Table~\ref{table:pe}. 

According to the results, the influence of PE varies among datasets. \method{} reaches the best RMSE score on CITE without PE, while two types of PE both improve the performance on GEX2ADT. Notice that in Table~\ref{table:dataset}, the RNA zero rate of CITE is significantly lower compared to the GEX2ADT, providing the model with greater access to data-specific information. If the data contains sufficient information, then the neighborhood information from the GNNs alone is adequate and there is no need for the extra prior knowledge from the PEs. This is further supported by the observation that random walk PE performs better than Laplacian PE in both datasets. The Laplacian PE models global information by using the spectral information of the graph Laplacian, while the random walk PE encodes local information by accessing the landing probability of a $k$-step random walk. In cases where the prior knowledge may be noisy for downstream tasks, the local information alone is enough for predictions and the global structure becomes redundant.

\begin{table}[tb]
    \centering
    \vspace{-1em}
    \caption{Prediction RMSE results of different positional encoding (score $\pm$ std).}
    \vspace{-1.2em}
    \scalebox{1.}{
    \begin{tabular}{c|c|c}
        \toprule
        & \textbf{CITE} & \textbf{GEX2ADT} \\ \midrule
        Laplacian PE & 1.63161 $\pm$ 0.01082 &  0.42025 $\pm$ 0.00243 \\
        Random Walk PE & 1.63014 $\pm$ 0.01129 & \textbf{0.41987 $\pm$ 0.00234} \\
        w/o PE & \textbf{1.62720 $\pm$ 0.00731} & 0.42202 $\pm$ 0.00399 \\ 
        \bottomrule
    \end{tabular}
    }
    \vspace{-1.5em}
    \label{table:pe}
\end{table}

\subsection{Comparasion of Different Fusion Strategies}

Our framework is based on hybrid fusion. Layer-wise speaking, we refer to our fusion strategy as \textit{concurrent} fusion since node representations simultaneously go through GNNs and transformers. In the case of protein nodes, information from protein nodes and gene nodes is fused after being processed by protein transformers and gene-protein (gene to protein) GNNs, respectively. In the following experiment, we compared the performance of layer-wise \textit{concurrent} fusion, \textit{GNN-first} fusion, and \textit{mixed} fusion. Still, in the case of protein nodes, we implemented \textit{GNN-first} fusion by first summing protein embeddings and the gene-protein GNNs outputs, and then passing through protein transformers. For \textit{mixed} fusion, we utilized \textit{concurrent} fusion on protein and gene nodes while applying \textit{GNN-first} fusion on cell nodes. We summarize the prediction RMSEs and the standard deviations across five runs in Table~\ref{tab:fusion}.

% Please add the following required packages to your document preamble:
% \usepackage{booktabs}
\begin{table}[h]
\centering
\vspace{-0.5em}
\caption{The prediction RMSEs for different Fusion strategies.}\label{tab:fusion}
\vspace{-1.2em}
\resizebox{0.48\textwidth}{!}{
\begin{tabular}{@{}lccc@{}}
\toprule
        & Concurrent          & GNN-first           & Mixed               \\ \midrule
CITE    & \textbf{1.62720}$\pm$\textbf{0.00731} & 1.63054$\pm$0.01018 & 1.63092$\pm$0.00631 \\
GEX2ADT & \textbf{0.41987}$\pm$\textbf{0.00234} & 0.42861$\pm$0.00094 & 0.42948$\pm$0.00158 \\ \bottomrule
\end{tabular}
}
\vspace{-1.2em}
\end{table}

As presented in the table, the original \textit{concurrent} fusion strategy exhibited superior performance compared to the other two fusion strategies. The \textit{concurrent} setting allows each modality to utilize intra-modal information prior to combining with other modalities, resulting in better performance since each modality holds varying importance for downstream tasks. However, the improvement in performance was not significant enough for the CITE. This observation is consistent with the findings of other experiments. Since the RNA zero rate in the CITE is considerably lower than that of the GEX2ADT, the significance of data-specific information (cell nodes) in the CITE outweighs the importance of other modalities, resulting in a relatively smaller performance boost.

\subsection{Handling Other Single-Cell Multimodal Prediction Tasks}

In this work, we focused on the specific task GEX2ADT (gene expression to protein levels) as a showcase. However, our framework is versatile and can be applied to other modality prediction tasks, i.e., the other three tasks mentioned in the NeurIPS 2021 competition~\cite{luecken2021sandbox} including ADT2GEX (protein levels to gene expression), GEX2ATAC (gene expression to chromatin accessibility) and ATAC2GEX (chromatin accessibility to gene expression). 
% Specifically, \method{} can be extended to:
% \begin{itemize}

\noindent \textbf{ADT2GEX:} In our framework, we constructed a multimodal heterogeneous graph consisting of gene, protein and cell nodes. For the GEX2ADT task, we removed the cell-protein edges to eliminate information leakage. Similarly, for the ADT2GEX task, we incorporate protein measurements into the cell-protein edges while removing the cell-gene edges. Cell embeddings were initialized by the reduced protein levels, and protein embeddings were initialized by the weighted sum of cell embeddings, where the weights are with the normalized protein levels. %With the same multimodal transformer module and a prediction layer, \method{} can predict gene expression from protein levels.

\noindent \textbf{GEX2ATAC and ATAC2GEX:} In the context of the GEX2ATAC and ATAC2GEX tasks, we removed the protein nodes from the multimodal graph. To initialize the cell embeddings for the GEX2ATAC task, we used the reduced gene expression values, while for gene embeddings, we computed a weighted sum of cell embeddings using normalized gene measurements as weights. For the ATAC2GEX task, we masked the cell-gene edges in the testing set. We initialized the cell embeddings using the reduced ATAC measurements, and the gene embeddings were randomly initialized.
% \end{itemize}

We have compared the performance of \method{} with scMoGNN on the four tasks in Table~\ref{tab:others}. Note that \method{} outperformed scMoGNN in tasks that involve protein modality and vice versa. These results suggest that incorporating prior information of protein nodes can enhance the performance of \method{} when protein modality is present. The RMSE scores across five runs are summarized as in Table~\ref{tab:others}.

% Please add the following required packages to your document preamble:
% \usepackage{booktabs}
\begin{table}[h]
\centering
\vspace{-0.5em}
\caption{Results on other Single-Cell Mutimodal Tasks.}\label{tab:others}
\vspace{-1.2em}
\resizebox{0.48\textwidth}{!}{
\begin{tabular}{l|cccc}
\toprule
           & GEX2ADT             & ADT2GEX             & GEX2ATAC            & ATAC2GEX            \\ \midrule
scMoFormer & \textbf{0.4198}7$\pm$\textbf{0.00234} & \textbf{0.31547}$\pm$\textbf{0.00184} & 0.17885$\pm$0.00008 & 0.23988$\pm$0.00031 \\
scMoGNN    & 0.42576$\pm$0.01180 & 0.32250$\pm$0.00136 & \textbf{0.17823}$\pm$\textbf{0.00011} & \textbf{0.23021}$\pm$\textbf{0.00219} \\ \bottomrule
\end{tabular}
}
\vspace{-1.8em}
\end{table}

\subsection{Training Efficiency of \method{}} \label{sec:efficiency}
Since scMoGNN is the best-performing baseline, we compared the running time and total GPU memory of \method{} and scMoGNN across five runs on one Quadro RTX 8000 GPU. The results are summarized in Table~\ref{tab:efficiency}. We observed that with higher GPU consumption, \method{} required a significantly shorter running time compared to scMoGNN. It is worth noting that \method{} was configured with a relatively large batch size of 8000 cells per batch and a hidden dimension of 512, in order to achieve better performance and better utilization of computing resources. One can surely reduce the required GPU memory of \method{} by limiting the setting accordingly. 

% Please add the following required packages to your document preamble:
% \usepackage{booktabs}
\begin{table}[h]
\centering
\vspace{-0.5em}
\caption{Efficiency Comparison}\label{tab:efficiency}
\vspace{-1.2em}
\resizebox{0.48\textwidth}{!}{
\begin{tabular}{@{}l|cc|cc@{}}
\toprule
\multicolumn{1}{l|}{} & \multicolumn{2}{l|}{\textbf{Running time (min)}} & \multicolumn{2}{l}{\textbf{GPU memory (GB)}}         \\ \midrule
                      & CITE                   & GEX2ADT                 & CITE                   & \multicolumn{1}{l}{GEX2ADT} \\
\method{}            & 24.82$\pm$4.16         & 17.95$\pm$3.56          & \multicolumn{1}{r}{38} & 21                          \\
scMoGNN               & 58.89$\pm$7.77         & 108.54$\pm$21.22        & \multicolumn{1}{r}{26} & 12                          \\ \bottomrule
\end{tabular}
}
\vspace{-0.8em}
\end{table}

\noindent \textbf{Can \method{} process large datasets?} In practical applications, it exhibits the capability to process large datasets. We address this issue from the perspectives of both the model and the data:
\begin{itemize}[leftmargin=*]
\item \textbf{Data aspect:} Due to technological or biological reasons, the number of RNA nodes and the number of protein nodes will be similar across datasets. Therefore, large datasets mean more cells, which can be handled by mini-batching cells.
\item \textbf{Model aspect:} In our multimodal transformer module, we employed linearized transformers~\cite{choromanski2021rethinking} with linear space and time complexity, which can be conveniently adapted to large datasets. Concerning GNNs, we incorporated GraphSAGE~\cite{hamilton2017inductive}, whose space and time complexity are related to the number of edges and hidden layer dimensions. Indeed, we can control the number of edges by mini-batching cells and make appropriate adjustments based on available computational resources.
\end{itemize}

\subsection{Ablation Study}
% To have a comprehensive understanding of our propose framework, we perform ablation study to analyze the influence of modules within \method{}. Specifically, we first examine the individual impact of multimodal transformer modules and later demonstrate the performance boost of transformers, equipped with prior knowledge, over GNNs. 
Table~\ref{table:main} demonstrates that models that incorporate domain knowledge perform better in modality prediction compared to those that do not. BABEL, ScMoGNN, and \method{} are the three models that make use of domain knowledge, and they show improved performance compared to the other two models. Among these three models, ScMoGNN, which is based on GNNs, performs better than BABEL, while \method{} outperforms all other models with its combination of transformers and GNNs framework. Given that \method{} includes a multimodal heterogeneous graph and three transformers, this raises the questions: \textit{Why no cell-cell graph and cell-protein graph? Which transformer has the biggest impact on performance? How much do the transformers contribute to the improvement in performance?}

\subsubsection{Exclusion of Cell-Cell Graph and Cell-Protein Graph.}\label{app:remark}
In Remark of Section~\ref{sec:subgraph}, we provided an explanation for our decision to exclude the cell-cell and cell-protein graphs. Additionally, through our empirical analysis, we observed a decrease in performance on both datasets when these links were incorporated into our heterogeneous graph. In comparison to the original w/o neither graph setting, the performance drop was evident in both the w/ cell-protein graph and w/ cell-cell graph settings, with the former demonstrating a more significant decline. The prediction RMSEs and standard deviations of five runs are summarized in Table~\ref{tab:graph_variants}.

% Please add the following required packages to your document preamble:
% \usepackage{booktabs}
\begin{table}[h]
\vspace{-0.5em}
\caption{The prediction RMSEs on different graph variants.}\label{tab:graph_variants}
\vspace{-1.2em}
\resizebox{0.48\textwidth}{!}{
\begin{tabular}{@{}lcccc@{}}
\toprule
        & w/o neither graph   & w/ cell-cell graph  & w/ cell-protein graph & w/ both graphs      \\ \midrule
CITE    & \textbf{1.62720}$\pm$\textbf{0.00731} & 1.68932$\pm$0.00751 & 1.71742$\pm$0.01692   & 1.70094$\pm$0.02207 \\
GEX2ADT & \textbf{0.41987}$\pm$\textbf{0.00234} & 0.42441$\pm$0.00094 & 0.43911$\pm$0.00414   & 0.42983$\pm$0.00231 \\ \bottomrule
\end{tabular}
}
\vspace{-1.2em}
\end{table}

\noindent \textbf{Why cell-cell graph did not help:} In the above experiment, we constructed a k-NN cell-cell graph by measuring the similarity in gene expression between cells. Nevertheless, since gene expression measurements often suffer from noise and sparsity, this static cell-cell graph may have introduced biased information into the downstream task. To address this issue, we utilized a cell transformer module to learn the dynamic cell-cell interactions via multi-head attention mechanism, which led to improved performance.

\noindent \textbf{Why cell-protein graph did not help:} For the cell-protein links, we incorporated target surface protein levels into edge weights, similar to how we built the cell-gene graph. However, these links conveys information about the prediction targets of the training set, causing the model to overfit easily. Hence, eliminating the cell-protein links served to eradicate information leakage.

\subsubsection{Influence of Every Transformer} The propose multimodal transformers consist of three different transformers, namely the cell transformer, gene transformer and protein transformers. As our predictions are based on the cell readout, it is expected that each of the three transformers will have different levels of impact on the performance. To quantify the specific impact of a single transformer, we conduct an experiment by removing the other two transformers and measuring the prediction RMSE scores. The results of the evaluation, including the scores of three partial models and the model with no transformers, are summarized in 
% Table~\ref{table:one_trans}.
Figure~\ref{fig:one_trans}.
% \begin{table}[tb]
    \centering
    \caption{Prediction RMSE results of keeping only one Transformer (score $\pm$ std).}
    \scalebox{1.}{
    \begin{tabular}{c|c|c}
        \toprule
        & \textbf{CITE} & \textbf{GEX2ADT} \\ \midrule
        Cell Transformer & 1.62994 $\pm$ 0.00967 & \textbf{0.41996 $\pm$ 0.00355} \\
        Gene Transformer & 1.62878 $\pm$ 0.00810 & 0.42670 $\pm$ 0.00191 \\
        Protein Transformer & \textbf{1.62742 $\pm$ 0.00781} & 0.42714 $\pm$ 0.00129 \\ 
        No Transformer & 1.63020 $\pm$ 0.00672 & 0.42731 $\pm$ 0.00138 \\ 
        \bottomrule
    \end{tabular}
    }
    \label{table:one_trans}
\end{table}
\begin{figure}[htb]%
    \vspace{-2.em}
    \centering
    \caption{RMSE$\downarrow$ results of keeping only one Transformer.}%
    \vspace{-1.2em}
    \subfloat[CITE]{{\includegraphics[width=0.5\linewidth]{images/cite_onetrans.png} }}%
    \subfloat[GEX2ADT]{{\includegraphics[width=0.5\linewidth]{images/gex_onetrans.png} }}%
    \label{fig:one_trans}%
    \vspace{-1.8em}
\end{figure}

% \vspace{-1.2em}
The performance of the gene transformer and protein transformer is better than that of the cell transformer in the CITE, while it is the opposite in the GEX2ADT. This can be explained by the difference in RNA zero rate between the two datasets, as shown in Table~\ref{table:dataset}. For the GEX2ADT, the high RNA zero rate means less information, making the cell transformer crucial in increasing performance by drawing more information from the data. On the other hand, the CITE has a lower zero rate, meaning it provides more information, allowing the gene transformer and protein transformer to enhance the model by adding external biological knowledge.

\subsubsection{How to Utilize Prior Knowledge} To answer the third question, we compare \method{} with two GNN-based models in Table~\ref{table:ablation}. The model "GNN-prior" refers to the GNNs that built on the same graph in Section~\ref{sec:graph_const}, while the "GNN" model is constructed using only the cell-gene graph without incorporating any prior information. The results show that the incorporation of prior knowledge into the graph results in a slight performance boost in both datasets. However, when multimodal transformers are included, the performance improvement is much more pronounced. This highlights the usefulness of prior knowledge and the importance of transformers to effectively incorporate this information into the model.
\begin{table}[t]
\vspace{-10pt}
\caption{Ablation results, evaluated using F1-score (\%). '\textit{w/o Sep}' excludes the separator from the backbone architecture; '\textit{w/o Decomp}' replaces \(L_{\text{dec}}\) with \(L_{\text{rec}}\); '\textit{w/o Augment}' omits pretraining on a synthetic dataset; and '\textit{Iterative}' involves iterative training between the decomposition and anomaly detection tasks.}\label{tab:ablation}
\centering
\resizebox{0.9\linewidth}{!}{
\renewcommand{\multirowsetup}{\centering}
% \setlength{\tabcolsep}{3pt}
\begin{tabular}{c|ccccc}
\toprule
Ablation & UCR & SMD & SWaT & PSM & WADI \\
\midrule
TADNet              & 98.74          & \textbf{93.35} & \textbf{90.21} & \textbf{98.66} & 88.15          \\
\midrule
\textit{w/o Sep}    & 32.68          & 66.24          & 76.89          & 83.28          & 47.66          \\
% r/ DPRNN w/ Conv  & 96.32          & 91.58          & 89.22          & 97.16          & 86.23          \\
\textit{w/o Decomp} & 48.69          & 84.12          & 88.41          & 95.57          & 65.72          \\
\textit{w/o Augment}& 40.12          & 74.17          & 83.26          & 98.01          & 62.15          \\
\textit{Iterative}  & \textbf{99.12} & 92.14          & 86.55          & 96.58          & \textbf{92.06} \\
\bottomrule
\end{tabular}
}
\vspace{-10pt}
\end{table}










