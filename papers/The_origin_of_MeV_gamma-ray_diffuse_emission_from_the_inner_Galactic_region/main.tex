% Please make sure you insert your
% data according to the instructions in PoSauthmanual.pdf
\documentclass[a4paper,11pt]{article}
\usepackage{pos}

%%%%%%%%%%%%%%%%%%%%%%%%%%%%%%%%%
%%%%%%%%%%%%%%%%%%%%%%%%%%%%%%%%%

\title{
The origin of MeV gamma-ray diffuse emission from the inner Galactic region
%\comment{and all-sky?}
}
%\ShortTitle{The inner Galactic diffuse emission in MeV}
\ShortTitle{MeV gamma-ray diffuse emission and all sky}

%%%%%%%%%%%%%%%%%%%%%%%%%%%%%%%%%
%%%%%%%%%%%%%%%%%%%%%%%%%%%%%%%%%
%%%%%%%%%%%%%%%%%%%%%%%%%%%%%
%%%%%  New Command      		   		     %%%%%
%%%%%%%%%%%%%%%%%%%%%%%%%%%%%

\def\figref#1{Figure~\ref{#1}} %\def\figref#1{\hbox{Figure~\ref{#1}}} 
\def\tabref#1{Table~\ref{#1}} %\def\tabref#1{\hbox{Table~\ref{#1}}} 
%\def\eqref#1{Eq.~(\ref{#1})} %\def\eqref#1{\hbox{Eq.~(\ref{#1})}} 
\def\eqref#1{Equation~\ref{#1}} %\def\eqref#1{\hbox{Eq.~(\ref{#1})}} 
\def\secref#1{Section~\ref{#1}} %\def\secref#1{\hbox{Section~\ref{#1}}} 

%\def\check{\textcolor{red}{?????}}
\def\check#1{\textcolor{red}{#1???}} %
%\def\comment#1{\textcolor{red}{[NT] #1}} %
\def\comment#1{\textcolor{cyan}{\bf [NT] #1}} %
\def\revise#1{\textcolor{red}{\bf #1}}


% Object names
%%%%%%%%%%%%%%%%%%%%%%%%%%%%%
\def\rxj{RX J1713.7$-$3946}
\def\rxjG{G347.3$-$0.5}
\def\cco{1WGA J1713.4$-$3949}

\def\slbu{4U 1700$-$377}
\def\slbsax{SAX J1712.6$-$3739}

%\def\go{G1.9$+$0.3}
%\def\gt{G330.2$+$1.0}
\def\vela{Vela Jr.}
\def\casA{Cassiopeia A}
\def\tycho{Tycho's SNR}
\def\hessj1731{HESS J1731$-$347}

% Satellites, ground-based telescopes
%%%%%%%%%%%%%%%%%%%%%%%%%%%%%
\def\chandra{{\it Chandra}}
\def\nustar{{\it NuSTAR}}
\def\xmm{{\it XMM-Newton}}
\def\suzaku{{\it Suzaku}}
\def\asca{{\it ASCA}}
\def\athena{{\it Athena}}
\def\bat{{\it Swift}-BAT}
\def\swift{{\it Swift}}
\def\integral{{\it INTEGRAL}}
\def\ibis{{\it INTEGRAL}-IBIS}
\def\spi{{\it INTEGRAL}-SPI}

%\def\comptel{{\it COMPTEL}}
\def\comptel{COMPTEL}

\def\hess{H.~E.~S.~S.~}
\def\fermi{{\it Fermi}}
\def\lat{{\it Fermi}-LAT}

\def\ams{AMS-02}
\def\voyager{\textit{Voyager}}


\def\galprop{{\sc GalProp}}


%%%%%%%%%%%%%%%%%%%%%%%%%%%%%
\def\dag{\hbox{$^\dagger$}}
\def\ddag{\hbox{$^\ddagger$}}
\def\half{\hbox{$\frac{1}{2}$}}


%% Units
%%%%%%%%%%%%%%%%%%%%%%%%%%%%%
\def\uG{\hbox{$\mu \mathrm{G}$}}
\def\flux{\hbox{${\rm erg}~{\rm cm}^{-2}~{\rm s}^{-1}$}}
\def\eflux{\hbox{${\rm erg}~{\rm cm}^{-2}~{\rm s}^{-1}$}} % energy flux
\def\Mflux{\hbox{MeV~cm$^{-2}$~s$^{-1}$~sr$^{-1}$}}

\def\kms{\hbox{${\rm km}~ {\rm s}^{-1}$}}
\def\cc{\hbox{${\rm cm}^{-3}$}}
%\def\angstrom{\hbox{${\rm A}^{\circ}$}}     --> \AA
\def\diffcoeff{\hbox{${\rm cm}^2 ~ {\rm s}^{-1}$}}    % (cm2/s) diffusion coefficient
\def\columnd{\hbox{${\rm cm}^{-2}$}}  % (/cm2) column density

%% Models
%%%%%%%%%%%%%%%%%%%%%%%%%%%%%
\def\lED{\ell_{\rm ED}}
\def\tcore{t_{\rm core}}
\def\tST{t_{\rm ST}}

\def\Mej{M_{\rm ej}}
\def\Eej{E_{\rm ej}}
\def\Eejsub{E_{\rm ej, 51}}
\def\ns{n_s}
\def\nz{n_0}
\def\nt{n_2}
\def\SNerg{10^{51}\ {\rm erg}}
\def\Sol{\hbox{$M_{\odot}$}}

\def\dens{\hbox{$\mu_{\rm H}/{\rm cm}^3$}}
\def\mass{\mu_{\rm H}}

\def\nd{{\rm cm}^{-3}}
\def\md{\mu_{\rmH} {\rm cm}^{-3}}

\def\NH{\hbox{$N_{\rm H}$}}
\def\nH{\hbox{$n_{\rm H}$}}
\def\pizero{\hbox{$\pi^0$}}
\def\pindex{\hbox{$\Gamma$}}

\def\EM{\hbox{$\int n_{\rm e} n_{\rm H} dl$}}
\def\EMeq{\hbox{\int n_{\rm e} n_{\rm H} dl}}

% hydrodynamics



% ACRO
%%%%%%%%%%%%%%%%%%%%%%%%%%%%%
%\usepackage[=true]{acro}
\usepackage[]{acro}

\DeclareAcronym{gde}{
  short = GDE ,
  long  = Galactic Diffuse Emission,
}


\DeclareAcronym{cgb}{
  short = CGB ,
  long  = cosmic gamma-ray background,
}

\DeclareAcronym{grb}{
  short = GRB ,
  long  = gamma-ray burst,
}
 
\DeclareAcronym{agn}{
  short = AGN ,
  long  = active galactic nucleus ,
  long-plural-form = active galactic nuclei
}
\DeclareAcronym{fsrq}{
  short = FSRQ ,
  long  = flat spectrum radio quasar ,
}
  
\DeclareAcronym{lmxb}{
  short = LMXB ,
  long  = low mass X-ray binary ,
  long-plural-form = low mass X-ray binaries ,
}
\DeclareAcronym{hmxb}{
  short = HMXB ,
  long  = high mass X-ray binary ,
  long-plural-form = high mass X-ray binaries ,
}
  
\DeclareAcronym{ic}{
  short = IC ,
  long  = inverse Compton ,
}


\DeclareAcronym{nustar}{
  short = {\it NuSTAR} ,
  long  = Nuclear Spectroscopic Telescope Array ,
}

\DeclareAcronym{bi}{
  short = BI ,
  long  = backside illumination ,
}
\DeclareAcronym{fi}{
  short = FI ,
  long  = frontside illumination ,
}
\DeclareAcronym{fov}{
  short = FoV ,
  long  = field of view ,
}
  
\DeclareAcronym{sim}{
  short = SIM ,
  long  = Science Instrument Module ,
}
\DeclareAcronym{hetg}{
  short = HETG ,
  long  = High Energy Transmission Grating ,
}
\DeclareAcronym{letg}{
  short = LETG ,
  long  = Low Energy Transmission Grating ,
}
\DeclareAcronym{hrc}{
  short = HRC ,
  long  = High Resolution Camera ,
}
\DeclareAcronym{acis}{
  short = ACIS ,
  long  = Advanced CCD Imaging Spectrometer ,
}
\DeclareAcronym{hrma}{
  short = HRMA ,
  long  = High Resolution Mirror Assembly,
}
  
  
\DeclareAcronym{compton}{
  short = Compton ,
  long  = Compton Gamma Ray Observatory,
}
\DeclareAcronym{hst}{
  short = HST ,
  long  = Hubble Space Telescope ,
}
\DeclareAcronym{iact}{
  short = IACT ,
  long  = Imaging Atmospheric Cherenkov Telescope ,
}
\DeclareAcronym{wcd}{
  short = WCD ,
  long  = Water Cherenkov Detector ,
}

\DeclareAcronym{hawc}{
  short = HAWC ,
  long  = High-Altitude Water Cherenkov ,
}
\DeclareAcronym{cta}{
  short = CTA ,
  long  = Cherenkov Telescope Array ,
}


\DeclareAcronym{em}{
  short = EM ,
  long  = electromagnetic ,
}
\DeclareAcronym{ism}{
  short = ISM ,
  long  = interstellar medium ,
}
\DeclareAcronym{csm}{
  short = CSM ,
  long  = circumstellar medium ,
}
\DeclareAcronym{sne}{
  short = SNe ,
  long  = supernovae , 
}

\DeclareAcronym{iss}{
  short = ISS ,
  long  = International Space Station ,
}

\DeclareAcronym{uhecr}{
  short = UHECRs ,
  long  = ultra high energy cosmic rays , 
}
\DeclareAcronym{ta}{
  short = TA ,
  long  = Telescope Array , 
}
\DeclareAcronym{auger}{
  short = Auger ,
  long  = Pierre Auger Observatory , 
}
\DeclareAcronym{ams}{
  short = AMS ,
  long  = Alpha Magnetic Spectrometer , 
}
\DeclareAcronym{pamela}{
  short = PAMELA ,
  long  = Payload for Antimatter Matter Exploration and Light-nuclei Astrophysics , 
}
   
\DeclareAcronym{cmb}{
  short = CMB ,
  long  = Cosmic Microwave Background , 
}
\DeclareAcronym{sed}{
  short = SED ,
  long  = spectral energy distribution , 
}
\DeclareAcronym{mhd}{
  short = MHD ,
  long  = magnetohydrodynamical ,
}
\DeclareAcronym{dof}{
  short = dof ,
  long  = degree of freedom ,
}
\DeclareAcronym{cco}{
  short = CCO ,
  long  = central compact object ,
  first-style = default
}
\DeclareAcronym{lmc}{
  short = LMC ,
  long  = Large Magellanic Cloud ,
}
\DeclareAcronym{smc}{
  short = SMC ,
  long  = Small Magellanic Cloud ,
}


\DeclareAcronym{hess}{
  short = H.E.S.S. ,
  long  = High Energy Spectroscopic System ,
  first-style = default
}
\DeclareAcronym{snr}{
  short = SNR ,
  long  = supernova remnant ,
}
\DeclareAcronym{pwn}{
  short = PWN ,
  short-plural = e ,
  long  = pulsar wind nebula ,
%  long-plural-form = pulsar wind nebulae,
  long-plural  = e ,
}

\DeclareAcronym{sn}{
  short = SN ,
  short-plural = e ,
  long  = supernova ,
  long-plural  = e ,
  first-style = default
}

\DeclareAcronym{nw}{
  short = NW ,
  long  = northwest ,
  first-style = default
}
\DeclareAcronym{hxc}{
  short = HXC ,
  long  = hard X-ray component ,
  first-style = default
}

\DeclareAcronym{cr}{
  short = CR ,
  long  = cosmic ray ,
}
\DeclareAcronym{psf}{
  short = PSF ,
  long  = point spread function ,
}
\DeclareAcronym{hpd}{
  short = HPD ,
  long  = half power diameter ,
}
\DeclareAcronym{fwhm}{
  short = FWHM ,
  long  = full width of half maximum ,
}

\DeclareAcronym{pic}{
  short = PIC ,
  long  = particle-in-cell ,
  tag = numerical ,
}
\DeclareAcronym{cxb}{
  short = CXB ,
  long  = Cosmic X-ray Background ,
}
\DeclareAcronym{grxe}{
  short = GRXE ,
  long  = Galactic Ridge X-ray Emission ,
}
\DeclareAcronym{pa}{
  short = PA ,
  long  = Positional Angle ,
}
\DeclareAcronym{dsa}{
  short = DSA ,
  long  = diffusive shock acceleration ,
}




%% Models
%%%%%%%%%%%%%%%%%%%%%%%%%%%%%
\def\lED{\ell_{\rm ED}}
\def\tcore{t_{\rm core}}
\def\tST{t_{\rm ST}}

\def\Mej{M_{\rm ej}}
\def\Eej{E_{\rm ej}}
\def\Eejsub{E_{\rm ej, 51}}
\def\ns{n_s}
\def\nz{n_0}
\def\nt{n_2}
\def\SNerg{10^{51}\ {\rm erg}}

\def\dens{\hbox{$\mu_{\rm H}/{\rm cm}^3$}}
\def\mass{\mu_{\rm H}}

\def\nd{{\rm cm}^{-3}}
\def\md{\mu_{\rmH} {\rm cm}^{-3}}

\def\NH{\hbox{$N_{\rm H}$}}
\def\nh{\hbox{$N_{\rm H}$}}
\def\pizero{\hbox{$\pi^0$}}
\def\pindex{\hbox{$\Gamma$}}

\def\EM{\hbox{$\int n_{\rm e} n_{\rm H} dl$}}
\def\EMeq{\hbox{\int n_{\rm e} n_{\rm H} dl}}

%% DSA
%%%%%%%%%%%%%%%%%%%%%%%%%%%%
\def\alfven{\hbox{Alfv$\acute{\rm e}$n wave}}

%% Diffusion
\def\Bo{{\bf B}_0}
\def\dB{\delta{\bf B}}
\def\dfdt{\frac{\partial f}{\partial t}}
\def\dfdx{\frac{\partial f}{\partial {\bf x}}}
\def\dfdp{\frac{\partial f}{\partial {\bf p}}}
\def\dfdz{\frac{\partial f}{\partial z}}
\def\dfdmu{\frac{\partial f}{\partial \mu}}
\def\avedpdt{\left< \frac{\Delta {\bf p}}{\Delta t} \right>}
\def\avedptwodt{\left< \frac{\Delta {\bf p}^2}{\Delta t} \right>}
\def\avedmutwodt{\left< \frac{\Delta \mu^2}{\Delta t} \right>}
\def\ddp{\frac{\partial}{\partial {\bf p}} }
\def\const{{\rm const.}}




%%%%%%%%%%%%%%%%%%%%%%%%%%%%
% The Sun
%\def\sun{\hbox{$\odot$}}
% less sim
\def\la{\mathrel{\mathchoice {\vcenter{\offinterlineskip\halign{\hfil
$\displaystyle##$\hfil\cr<\cr\sim\cr}}}
{\vcenter{\offinterlineskip\halign{\hfil$\textstyle##$\hfil\cr
<\cr\sim\cr}}}
{\vcenter{\offinterlineskip\halign{\hfil$\scriptstyle##$\hfil\cr
<\cr\sim\cr}}}
{\vcenter{\offinterlineskip\halign{\hfil$\scriptscriptstyle##$\hfil\cr
<\cr\sim\cr}}}}}
% greater sim
\def\ga{\mathrel{\mathchoice {\vcenter{\offinterlineskip\halign{\hfil
$\displaystyle##$\hfil\cr>\cr\sim\cr}}}
{\vcenter{\offinterlineskip\halign{\hfil$\textstyle##$\hfil\cr
>\cr\sim\cr}}}
{\vcenter{\offinterlineskip\halign{\hfil$\scriptstyle##$\hfil\cr
>\cr\sim\cr}}}
{\vcenter{\offinterlineskip\halign{\hfil$\scriptscriptstyle##$\hfil\cr
>\cr\sim\cr}}}}}


%% Cooridinate
%\newcommand{\gtrsim}{\ga}
%\newcommand{\lesssim}{\la}
% degree, arcmin, arcsec
\def\degr{\hbox{$^\circ$}}
\def\arcmin{\hbox{$^\prime$}}
\def\arcsec{\hbox{$^{\prime\prime}$}}
%
\def\utw{\smash{\rlap{\lower5pt\hbox{$\sim$}}}}
\def\udtw{\smash{\rlap{\lower6pt\hbox{$\approx$}}}}
%
\def\fd{\hbox{$.\!\!^{\rm d}$}}
\def\fh{\hbox{$.\!\!^{\rm h}$}}
\def\fm{\hbox{$.\!\!^{\rm m}$}}
\def\fs{\hbox{$.\!\!^{\rm s}$}}
\def\hour{\hbox{$^{\rm h}$}}
\def\min{\hbox{$^{\rm m}$}}
\def\fdg{\hbox{$.\!\!^\circ$}}
\def\dg{\hbox{$^{\rm d}$}}
\def\farcm{\hbox{$.\mkern-4mu^\prime$}}
\def\farcs{\hbox{$.\!\!^{\prime\prime}$}}




% Bibliography and bibfile
%%%%%%%%%%%%%%%%%%%%%%%%%%%%
\def\aj{AJ}%
          % Astronomical Journal
\def\araa{ARA\&A}%
          % Annual Review of Astron and Astrophys
\def\apj{ApJ}%
          % Astrophysical Journal
\def\apjl{ApJ}%
          % Astrophysical Journal, Letters
\def\apjs{ApJS}%
          % Astrophysical Journal, Supplement
\def\ao{Appl.~Opt.}%
          % Applied Optics
\def\apss{Ap\&SS}%
          % Astrophysics and Space Science
\def\aap{A\&A}%
          % Astronomy and Astrophysics
\def\aapr{A\&A~Rev.}%
          % Astronomy and Astrophysics Reviews
\def\aaps{A\&AS}%
          % Astronomy and Astrophysics, Supplement
\def\azh{AZh}%
          % Astronomicheskii Zhurnal
\def\baas{BAAS}%
          % Bulletin of the AAS
\def\jrasc{JRASC}%
          % Journal of the RAS of Canada
\def\memras{MmRAS}%
          % Memoirs of the RAS
\def\mnras{MNRAS}%
          % Monthly Notices of the RAS
\def\pra{Phys.~Rev.~A}%
          % Physical Review A: General Physics
\def\prb{Phys.~Rev.~B}%
          % Physical Review B: Solid State
\def\prc{Phys.~Rev.~C}%
          % Physical Review C
\def\prd{Phys.~Rev.~D}%
          % Physical Review D
\def\pre{Phys.~Rev.~E}%
          % Physical Review E
\def\prl{Phys.~Rev.~Lett.}%
          % Physical Review Letters
\def\pasp{PASP}%
          % Publications of the ASP
\def\pasj{PASJ}%
          % Publications of the ASJ
\def\qjras{QJRAS}%
          % Quarterly Journal of the RAS
\def\skytel{S\&T}%
          % Sky and Telescope
\def\solphys{Sol.~Phys.}%
          % Solar Physics
\def\sovast{Soviet~Ast.}%
          % Soviet Astronomy
\def\ssr{Space~Sci.~Rev.}%
          % Space Science Reviews
\def\zap{ZAp}%
          % Zeitschrift fuer Astrophysik
\def\nat{Nature}%
          % Nature
\def\iaucirc{IAU~Circ.}%
          % IAU Cirulars
\def\aplett{Astrophys.~Lett.}%
          % Astrophysics Letters
\def\apspr{Astrophys.~Space~Phys.~Res.}%
          % Astrophysics Space Physics Research
\def\bain{Bull.~Astron.~Inst.~Netherlands}%
          % Bulletin Astronomical Institute of the Netherlands
\def\fcp{Fund.~Cosmic~Phys.}%
          % Fundamental Cosmic Physics
\def\gca{Geochim.~Cosmochim.~Acta}%
          % Geochimica Cosmochimica Acta
\def\grl{Geophys.~Res.~Lett.}%
          % Geophysics Research Letters
\def\jcp{J.~Chem.~Phys.}%
          % Journal of Chemical Physics
\def\jgr{J.~Geophys.~Res.}%
          % Journal of Geophysics Research
\def\jqsrt{J.~Quant.~Spec.~Radiat.~Transf.}%
          % Journal of Quantitiative Spectroscopy and Radiative Trasfer
\def\memsai{Mem.~Soc.~Astron.~Italiana}%
          % Mem. Societa Astronomica Italiana
\def\nphysa{Nucl.~Phys.~A}%
          % Nuclear Physics A
\def\physrep{Phys.~Rep.}%
          % Physics Reports
\def\physscr{Phys.~Scr}%
          % Physica Scripta
\def\planss{Planet.~Space~Sci.}%
          % Planetary Space Science
\def\procspie{Proc.~SPIE}%
          % Proceedings of the SPIE

%%%%%%%%%%%%%%%%%%%%%%%%%%%
%%%%%  Others
%%%%%%%%%%%%%%%%%%%%%%%%%%%




%\usepackage[hang,small,bf]{caption}
\usepackage[subrefformat=parens]{subcaption}
\captionsetup{compatibility=false}

%%%%%%%%%%%%%%%%%%%%%%%%%%%%%%%%%
%%%%%%%%%%%%%%%%%%%%%%%%%%%%%%%%%

\author*[a,b]{Naomi Tsuji}
\author[c,b,d]{Yoshiyuki Inoue}
\author[e]{Hiroki Yoneda}
\author[f]{Reshmi Mukherjee}
\author[g]{Hirokazu Odaka}

\affiliation[a]{Faculty of Science, Kanagawa University, \\
    2946 Tsuchiya, Hiratsuka-shi, Kanagawa 259-1293, Japan}
\affiliation[b]{Interdisciplinary Theoretical \& Mathematical Science Program (iTHEMS), RIKEN \\
    2-1 Hirosawa, Wako, Saitama 351-0198, Japan}
\affiliation[c]{Department of Earth and Space Science, Graduate School of Science, Osaka University, \\
    Toyonaka, Osaka 560-0043, Japan}
\affiliation[d]{Kavli Institute for the Physics and Mathematics of the Universe (WPI), UTIAS, The University of Tokyo,\\
    5-1-5 Kashiwanoha,Kashiwa, Chiba 277-8583, Japan}
\affiliation[e]{Nishina Center, RIKEN, \\
    2-1 Hirosawa, Wako, Saitama 351-0198, Japan}
\affiliation[f]{Department of Physics and Astronomy, Barnard College, Columbia University, \\
    New York, NY, 10027, USA}
\affiliation[g]{Department of Physics, The University of Tokyo, \\
    7-3-1 Hongo, Bunkyo, Tokyo 113-0033, Japan}

\emailAdd{ntsuji@kanagawa-u.ac.jp}
%\emailAdd{s.author@univ.country}


\if0
%\email{ntsuji@kanagawa-u.ac.jp}
%\author[0000-0002-0786-7307]{Naomi Tsuji}
\affiliation{Faculty of Science, Kanagawa University, 2946 Tsuchiya, Hiratsuka-shi, Kanagawa 259-1293, Japan}
\affiliation{Interdisciplinary Theoretical \& Mathematical Science Program (iTHEMS), RIKEN, 2-1 Hirosawa, Wako, Saitama 351-0198, Japan}
\affiliation{Department of Physics, Rikkyo University, 3-34-1 Nishi Ikebukuro, Toshima-ku, Tokyo 171-8501, Japan}

\author[0000-0002-7272-1136]{Yoshiyuki Inoue}
\affiliation{Department of Earth and Space Science, Graduate School of Science, Osaka University, Toyonaka, Osaka 560-0043, Japan}
\affiliation{Interdisciplinary Theoretical \& Mathematical Science Program (iTHEMS), RIKEN, 2-1 Hirosawa, Wako, Saitama 351-0198, Japan}
\affiliation{Kavli Institute for the Physics and Mathematics of the Universe (WPI), The University of Tokyo, Kashiwa 277-8583, Japan}

\author{Hiroki Yoneda}
\affiliation{Nishina Center, RIKEN, 2-1 Hirosawa, Wako, Saitama 351-0198, Japan}

\author{Reshmi Mukherjee}
\affiliation{Department of Physics and Astronomy, Barnard College, Columbia University, New York, NY, 10027, USA}

\author{Hirokazu Odaka}
\affiliation{Department of Physics, The University of Tokyo, 7-3-1 Hongo, Bunkyo, Tokyo 113-0033, Japan}
\affiliation{Kavli Institute for the Physics and Mathematics of the Universe (WPI), The University of Tokyo, Kashiwa 277-8583, Japan}
\fi



%%%%%%%%%%%%%%%%%%%%%%%%%%%%%%%%%
%%%%%%%%%%%%%%%%%%%%%%%%%%%%%%%%%
\abstract{
The origin of the inner Galactic emission, measured by \comptel\ with a flux of $\sim ~ 10^{-2}$~\Mflux\ in the 1--30 MeV range, %from a region of |l|<60 degree and |b|<10 degree, 
has remained unsettled since its discovery in 1994. 
We investigate the origin of this emission by taking into account individual sources which are not resolved by \comptel\ and the Galactic diffuse emission. 
The source contribution is estimated for sources crossmatched between the \bat\ and \lat\ catalogs by interpolating the energy spectra in the hard X-ray and GeV gamma-ray ranges, as well as unmatched sources. This results in a flux of $\sim$20\% of the \comptel\ excess. 
The Galactic diffuse emission is calculated by \galprop\ to reconcile the cosmic-ray and gamma-ray spectra with observations by AMS-02, \voyager, and \lat, resulting in a flux of $\sim$30--80\% of the \comptel\ emission. 
Thus, we show that the COMPTEL emission could be roughly reproduced by a combination of the sources and the Galactic diffuse emission. 
%We will give the details of the analysis and show that the COMPTEL emission could be reproduced by a combination of the Galactic diffuse emission, resolved sources, and likely the gamma-ray cosmic background. 
Furthermore, combined with the extragalactic emission,
we construct all-sky images in the MeV gamma-ray range
%\revise{
to pinpoint some potential interesting targets for future missions, 
%\sout{and report on the importance of future missions}
which would be critical for bridging the ``MeV gap'' in the spectra of gamma-ray sources.
%\\  \comment{
%6 pages!!! (4 pages for the main text, one for the cover page, and one for references) \\ Cite Tsuji+ 2022 (\comptel\ excess paper)? \\ Submit: login and workarea \url{https://pos.sissa.it/cgi-bin/reader/info.cgi?p=authors} }
}


%%%%%%%%%%%%%%%%%%%%%%%%%%%%%%%%%
%%%%%%%%%%%%%%%%%%%%%%%%%%%%%%%%%

\FullConference{%
  7th Heidelberg International Symposium on High-Energy Gamma-Ray Astronomy (Gamma2022)\\
  4-8 July 2022\\
  Barcelona, Spain
}

%% \tableofcontents


\begin{document}
\maketitle
%%%%%%%%%%%%%%%%%%%%%%%%%%%%%%%%%
%%%%%%%%%%%%%%%%%%%%%%%%%%%%%%%%%


\section{Introduction} 
\label{sec:intro}
%%%%%%%%%%%%%%%%%%%%%%%%%
%%%%%%%%%%%%%%%%%%%%%%%%%

%%% Intro
%MeV gap? Feedback on Galaxy evolution? approved COSI?
The MeV gamma-ray domain is the only unexplored window among recent multiwavelength observations in astrophysics, often referred to as the ``MeV gap''.
One of the open issues in MeV gamma-ray astrophysics is the origin of diffuse emission from the inner Galactic region.
%Investigation of this diffuse emission covers several important facets, including the Galactic diffuse emission from low-energy (sub-GeV) \acp{cr}, individual MeV gamma-ray objects, and/or new populations of MeV gamma-ray radiation originated from dark matter or neutrinos.
%Thus, the study of the MeV gamma-ray diffuse emission would have a lot of influence on these broad topics, especially in the next decade when some missions will give us a new insight into the MeV gap.
%
%%% COMPTEL excess
The Imaging Compton Telescope \comptel\ onboard the Compton Gamma-Ray Observatory (CGRO) reported the detection of diffuse emission of 10$^{-2}$ \Mflux\ in 1--30 MeV from the inner Galactic region with $|\ell| \leq 30\degr$ and $|b| \leq 15\degr$ \citep{bouchet_diffuse_2011,strong_diffuse_1996}.
%This measurement revealed that the emission was actually diffuse, also confirmed by OSSE \citep{strong_diffuse_1996}, and consistent with the result of COS-B \citep{strong_radial_1988}.
%The \comptel\ diffuse emission was calculated as follows. The gamma-ray intensity model consisted of \ac{gde} (i.e., radiation from interactions of \acp{cr} with gas and photon fields), source (only Crab), and an isotropic term. The instrumental background, which dominated the detected gamma-ray events, was implemented with a fixed spectral shape \citep{strong_diffuse_1994}. The overall fit was performed by combining the gamma-ray intensity model weighted by the response function and the instrumental background. Based on the best-fit parameters of \ac{gde}, the aforementioned diffuse emission was estimated.
This emission was derived by considering the instrumental background and \ac{cgb}, thus it would contain \ac{gde} and unresolved sources.
%There are several results of the \comptel\ diffuse emission obtained with different models, data (observation phase), and size of the inner Galactic region \citep{strong_diffuse_1994,strong_diffuse_1996,strong_diffuse_2004,bouchet_diffuse_2011}, and the latest one with $|\ell| \leq 30\degr$ and $|b| \leq 15\degr$ \citep{bouchet_diffuse_2011} is presented in this proceeding.
Recently, the inner Galactic diffuse emission has been confirmed by other observations such as \spi\ \cite{siegert_diffuse_2022} and the electron-tracking Compton camera (ETCC) aboard the balloon mission of SMILE-2$+$ \cite{takada_first_2022}.

The origin of the inner Galactic diffuse emission has been in active debate. 
If the \ac{gde} model to account for the diffuse emission by \lat\ \cite{ackermann_fermi_2012} is extrapolated to the MeV energy range, 
there is an apparent excess component (e.g., \cite{strong_diffuse_2004}), which is commonly referred to as the \comptel\ excess.
There are several scenarios for reproducing the \comptel\ excess:
(1) Individual MeV gamma-ray sources should be taken into consideration. 
%Only the Crab was considered when calculating the \comptel\ emission, although \comptel\ detected 25 steady sources \citep{Schonfelder2000}. Furthermore, sources which were not resolved by \comptel\ would also have a fraction of the contribution. 
(2) There are non-negligible uncertainties on the model of \ac{gde}, since it has a lot of unconstrained parameters (e.g., photon field densities, \ac{cr} source distribution, \ac{cr} injection spectra, and propagation mechanism).
Enhancement of one or more of these parameters can make \ac{gde} higher to reach the \comptel\ excess \citep{bouchet_diffuse_2011}.
(3) New populations, such as annihilation or decay of dark matter \citep{boddy_indirect_2015} and/or cascaded gamma rays accompanying cosmic neutrinos \citep{fang_tev_2022}, might be present.

%%% origin of COMPTEL excess
\if0
The origin of the inner Galactic diffuse emission by \comptel\ has been in active debate. 
%%% Fermi diffuse
%Almost two decades since its discovery, an all-sky gamma-ray survey by \lat\ unveiled the detailed spectroscopy of the diffuse emission in the GeV energy band. It revealed that the diffuse emission observed by \lat\ was fairly explained by a combination of the \ac{gde}, resolved GeV gamma-ray sources, and extragalactic gamma-ray background (i.e., \ac{cgb}) \citep{ackermann_fermi_2012}. Note that some locally characteristic radiation, such as the \fermi\ bubble and the Galactic center excess, still remain elusive \citep{su_giant_2010,ackermann_fermi_2017,murgia_fermilat_2020}.
If the \ac{gde} model of \lat\ is extrapolated to the MeV energy range, 
there is an apparent excess component to account for the \comptel\ emission (e.g., \citealt{strong_diffuse_2004}), which is commonly referred to as the \comptel\ excess.
There are several scenarios for reproducing the \comptel\ excess:
(1) Individual MeV gamma-ray sources should be taken into consideration. 
Only the Crab was considered when calculating the \comptel\ emission, although \comptel\ detected 25 steady sources \citep{Schonfelder2000}.
Furthermore, sources which were not resolved by \comptel\ would also have a fraction of the contribution. % to the \comptel\ excess.
(2) There are non-negligible % (more than a factor of few)
uncertainties on the model of \ac{gde}, since it has a lot of unconstrained parameters (e.g., photon field densities, \ac{cr} source distribution, \ac{cr} injection spectra, and propagation mechanism).
Enhancement of one or more of these parameters can make \ac{gde} higher to reach the \comptel\ excess \citep{bouchet_diffuse_2011}.
%(3) \ac{cgb} might not be fully subtracted.
(3) New populations, such as annihilation or decay of dark matter \citep{boddy_indirect_2015,christy_indirect_2022,Binder_2022} and/or cascaded gamma rays accompanying cosmic neutrinos \citep{fang_tev_2022}, might be present.

%%% updates since COMPTEL
We address a few updates on observations of the MeV gamma-ray diffuse emission since \comptel.
\integral-SPI measured a spectrum of the diffuse emission in 0.02–2.4 MeV, which comprised a continuum component and four gamma-ray lines (i.e., positron annihilation, $^{26}$Al and $^{60}$Fe lines) \citep{bouchet_diffuse_2011}.
The continuum component is consistent with the emission by \comptel\ in the overlapping energy range of 1–2.4 MeV, as shown in \figref{fig:size}.
\cite{bouchet_diffuse_2011} argued that the diffuse emission by \integral-SPI can be roughly reproduced by the standard model of \ac{gde}.
The fit, however, became much improved if they increased the normalization of the primary \ac{cr} electron spectrum or the interstellar radiation field in the Galactic bulge or a large Galactic CR halo.
The latest result of the 0.5--8 MeV observation by \integral-SPI was presented in \cite{siegert_diffuse_2022}.
Using the new analysis with the lower level of signal-to-noise ratio, they confirmed that the obtained diffuse emission showed a mismatch of a factor of 2–3 in normalization with respect to the baseline model of \ac{gde}.
The \spi\ spectra are also illustrated in \figref{fig:size}.
%This may arise from enhanced target photon densities and/or electron source spectra, slightly modified diffusion properties, or an unresolved population of MeV gamma-ray sources.
Besides \spi, the electron-tracking Compton camera (ETCC) aboard the balloon mission of SMILE-2$+$ retrieved a gamma-ray lightcurve %\textcolor{blue}{YI: gamma-ray map?}
in 0.15–2.1 MeV during flight, showing enhanced gamma rays when it was pointing at the vicinity of the Galactic center \citep{takada_first_2022}.
\fi


%%% ToC
%\comment{can cut this paragraph}
In this proceeding, we investigate the \comptel\ excess by taking into account MeV gamma-ray sources unresolved by \comptel\ and \ac{gde} in \secref{sec:analysis}. We also explore all-sky images in the MeV gamma-ray range in \secref{sec:allsky}. 
%The summary is given in \secref{sec:conclusion}.




\section{%Analysis --- 
The inner Galactic diffuse emission in MeV}
\label{sec:analysis}
%%%%%%%%%%%%%%%%%%%%%%%%%
%%%%%%%%%%%%%%%%%%%%%%%%%

\subsection{MeV gamma-ray sources}
\label{sec:sources}
%%%%%%%%%%%%%%%%%%%%%%%%%
%%%%%%%%%%%%%%%%%%%%%%%%%

%%% Summary of \bat\ and \lat\ cross-match. (Tsuji+ 2021)
Although the previous studies (e.g., \cite{strong_diffuse_1996,orlando_imprints_2018,siegert_diffuse_2022}) proposed that the \comptel\ excess would be attributed by radiation from individual unresolved sources,
the quantitative estimation has not been done yet.
We estimate this source contribution from a MeV gamma-ray source catalog in \cite{tsuji_cross-match_2021}, which presented a crossmatching between the 105-month \bat\ \citep{Bird2016} and 10-yr \lat\ (4FGL-DR2) \cite{4fgldr2} catalogs, resulting in 156 point-like and 31 extended crossmatched sources.
%Note that among them, 136 point sources and 15 extended sources are firmly matched (i.e., the hard X-ray and GeV gamma-ray emission are originated from the same source), and 16 sources were actually detected by \comptel\ \citep{tsuji_cross-match_2021}.
These crossmatched sources, which are both hard X-ray and GeV gamma-ray emitters, are prominent sources in the MeV gamma-ray sky.
%
%%% source breakdown 
Among them,
19 point sources and 14 extended sources are located in the inner Galactic region with $|\ell| \leq 30\degr$ and $|b| \leq 15\degr$.
%The point sources compose three blazars, three pulsars, one X-ray binary (RX J1826.2$-$1450), Galactic center (Sgr A$^\star$), one globular cluster (ESO 520$-$27), four unidentified sources, and six false matches\footnote{Falsely matched sources indicate spatially crossmatched sources, but the hard X-ray (\bat) and gamma-ray (\lat) sources are different origins \citep{tsuji_cross-match_2021}.}. 
%The extended sources are six PNWe, two SNRs, four spp\footnote{Sources that are candidates of SNR or PNW.}, an unidentified source, and a false match.


%%% Fitting Model
We jointly fit the \acp{sed} of \bat\ in 14--195 keV and \lat\ in 50 MeV--300 GeV.
%We adopt a log-parabola model first.  The log-parabola model can be applied to most of the sources, in particular blazars like \acp{fsrq} or pulsars, of which the MeV gamma-ray radiation is originated from a single electron population.
%We find that nine sources (one pulsar, three false-matched sources, one globular cluster, one unidentified, Galactic center, one PWN, and one spp) cannot be well fitted by the log-parabola model, inferred from a large reduced chi-squared of $>$10. For these sources, we adopt a two-component model, which is a superposition of the models in the \bat\ and \lat\ catalogs.
%Additionally, we also use the same two-component model for \rxj, because it is a \ac{snr}, of which the X-ray and gamma-ray emission do not originate from the same process. Likewise, a choice of the fitting model may cause uncertainty on the source flux.
For the fitting model, we adopt log-parabola or two-component models, depending on the sources. The two-component model is a superposition of the models in the \bat\ and \lat\ catalogs.
%We checked that the choice of the fitting model (i.e., log-parabola or broken power-law models) does not have a large effect on the undermentioned result.
%
%%% Contribution to COMPTEL excess
Based on the best-fit model %\revise{
determined by the minimum $\chi^2$, we estimate spectra in the MeV gamma-ray energy range, sum up all the spectra of the 33 sources in the inner Galactic region, and divide it by the region size.
The result of the accumulated source spectrum is shown in \figref{fig:model}.
The contribution of all crossmatched sources to the \comptel\ excess is about 10\%. 
%The result is shown in \figref{fig:model}.
%The contribution of all crossmatched sources to the \comptel\ excess is about 10\%. 
%and the ratio of the point sources to the extended sources is roughly 1–1.5.
%It should be noted that if we select only the firmly matched sources in the inner Galactic region (nine point-like and 12 extended sources), the total source contribution is reduced to 60--80\% of the spectrum shown in \figref{fig:model}.
See \cite{tsuji_2022} for details.


%%% unmatched sources
Besides the crossmatched sources in \cite{tsuji_cross-match_2021}, there exist many unmatched sources that would have a significant contribution accumulatively.
In the region with $|\ell| \leq 30\degr$ and $|b| \leq 15\degr$, there are 152 \bat\ and 708 \lat\ sources, where the crossmatched sources are excluded.
%We assume that 
These unmatched sources (860 in total) would be fainter than $10^{-12}$ \flux\ in the MeV energy band since they are not detected by \bat\ or \lat, with the sensitivity being approximately 10$^{-12}$~\flux.
%Then, 
%\revise{
If we assume each unmatched source has a flux of  $10^{-12}$ \flux,
the accumulative source flux is $\sim 10^{-3}$ \Mflux, 
which should be considered as an upper limit.
This upper limit flux of the unmatched sources is roughly comparable to that of the crossmatched sources.
Combined with the crossmatched sources, the contribution of the sources is $\sim$20\% of the \comptel\ excess (\figref{fig:model}).
%Compared with the GeV gamma-ray source component by \lat\ \citep{ackermann_fermi_2012} in the region with $|\ell| \leq 80\degr$ and $|b| \leq 10\degr$, our source spectrum is roughly in agreement within a factor of $\sim$2 at 200 MeV.
%Detailed modeling of the unmatched sources will be presented in a future publication.




\subsection{Galactic diffuse emission}
\label{sec:gde}
%%%%%%%%%%%%%%%%%%%%%%%%%
%%%%%%%%%%%%%%%%%%%%%%%%%

%%% GALPROP
To evaluate \acf{gde}, we make use of \galprop\ (version 54 of WebRun), which is designed to calculate astrophysics of \acp{cr} (i.e., propagation and energy loss) and photon emissions in the radio to gamma-ray energy bands \citep{porter_high-energy_2017}.
%%% Models in the literature.
%In this proceeding, we consider three models of \ac{gde} in the literature: one model from \cite{ackermann_fermi_2012} and two models from \cite{orlando_imprints_2018} 
%%% Ackermann+ 2012
%These models are constructed to be reconciled with the gamma-ray observations by \lat\ and the observed \ac{cr} spectra by several \ac{cr} experiments.
%%% Orlando
%Using the results of the latest \ac{cr} measurements with \voyager, \cite{orlando_imprints_2018} modified the \ac{gde} models in the literature, especially the injection parameters of electrons and propagation parameters.
%%%
In this proceeding, we use the \ac{gde} models in the literature \cite{ackermann_fermi_2012,orlando_imprints_2018}, which are developed to be consistent with the \ac{cr} observations by AMS-02 (and \voyager) and the gamma-ray observations by \lat.
A baseline model of 
$^{\rm S}S ^{\rm Z}4 ^{\rm R}20 ^{\rm T}150 ^{\rm C}5$\footnote{This model assumes that the source distribution of \acp{cr} is SNRs, the Galactic disk is characterized by the height of $z=$4 kpc and the galactocentric radius of $R=$20 kpc, and $T_s$=150 K and $E(B-V) = 5 $ mag cut is adopted for determining the gas-to-dust ratio (see \cite{ackermann_fermi_2012} for details).} 
is selected as a representative of the models in \cite{ackermann_fermi_2012} and referred to as Model~1. 
From \cite{orlando_imprints_2018}, we adopt the DRE (i.e., diffusion and re-acceleration) and DRELowV (modified DRE\footnote{
%\revise{
DRE model with some modifications on parameters of diffusion and particle injection in order to reproduce the \ac{cr} measurements \cite{orlando_imprints_2018}.}) models, hereafter referred to as Model 2 and Model~3, respectively.

%%% Compare GDE Model 1-3
Since GDE below $\sim$100 MeV is dominated by the \ac{ic} component, the difference in Models 1--3 arises from \ac{cr} electrons in 0.1--1 GeV.
Models 2 and 3 are respectively the highest and lowest with the difference of a factor of a few, and Model 1 is in the middle of them.
%Model~2 almost can reach the \comptel\ emission, while it is slightly lower than the flux at the lower energy bins. 
Although there is such uncertainty on the \ac{gde} models, $\gtrsim$30\% of the \comptel\ excess is contributed by GDE.

\if0
%%% Example of DGE SED
%\figref{fig:fig1} shows \ac{gde} spectra of Model 1--3.
The \ac{gde} spectra of Model~1, shown in \figref{fig:fig1}, consist of three components of radiation: \ac{ic} scattering, Bremsstrahlung, and pion-decay radiation.
In the energy channel of \comptel, \ac{ic} is dominant, while Bremsstrahlung is subdominant because of ionization loss of electrons at the lower energy, and the hadronic component is less effective due to the pion bump.
The \ac{ic} scattering in the MeV gamma-ray range is attributed to sub-GeV electrons that up-scatter seed photon fields of optical, infrared, and \ac{cmb}.


%%% Compare GDE Model 1-3
\figref{fig:fig1} also compares the aforementioned GDE models, Models 1--3.
Since GDE below $\sim$100 MeV is dominated by the \ac{ic} component, the difference in Models 1--3 arises from \ac{cr} electrons in 0.1–1 GeV (\figref{fig:cr} in Appendix \ref{sec:gde_appendix}).
Models 2 and 3 are respectively the highest and lowest with the difference of a factor of few, and Model 1 is in the middle of them.
Model~2 almost can reach to the \comptel\ emission, while it is slightly lower than the flux at the lower energy bins. This trend is the same for the diffuse emission measured in the different sizes in \figref{fig:size} (Appendix \ref{sec:region_appendix}).
Although there is such uncertainty on the \ac{gde} models, $\gtrsim$30\% of the \comptel\ excess is contributed by GDE.


\begin{figure}[ht!]
\centering
\includegraphics[width=12cm]{figures/SED_detailed_mix_1h_noSource_v2_l30.0_b15.0.pdf}
\caption{
The spectra of the inner Galactic diffuse emission taken by \comptel\ in light green, shown with the continuum emission and gamma-ray lines by \spi\ in black and magenta, respectively \citep{bouchet_diffuse_2011}.
The components of \ac{gde} of Model~1 \citep{ackermann_fermi_2012} are shown in solid lines: total in black, Bremsstrahlung in blue, \pizero -decay in red, and \ac{ic} in green. 
The flux points observed by \lat, including the diffuse emission and the gamma-ray sources, are indicated by black circles with the grey shadow being the error, although it was obtained from the region with $|\ell| \leq 80\degr$ and $|b| \leq 8\degr$ \citep{ackermann_fermi_2012}.
The total \ac{gde} of Model~2 and Model~3 are illustrated with black dashed and dash-dotted lines, respectively.
\label{fig:fig1}
}
\end{figure}

\fi



\subsection{Results and discussion}
\label{sec:discussion}
%%%%%%%%%%%%%%%%%%%%%%%%%
%%%%%%%%%%%%%%%%%%%%%%%%%

%%%%%%%%%%%%%%%%%%%%%%%%%%%%
\begin{figure}[ht!]
\centering
\includegraphics[width=0.9\hsize]{figures/SED_mix_grid_v1.2_Gal_30.0x15.0_binFalse_CGBfactor0_addUNmatchTrue.pdf}
%\includegraphics[width=0.95\hsize]{figures/SED_mix_grid_v1.1_Gal_30.0x15.0_binTrue_CGBfactor0_addUNmatchTrue.pdf}
\caption{
%\comment{Male the labels larger!}
The \acp{sed} of \ac{gde} (dashed black line) and
the sources (solid red line for all the sources, dashed red for the crossmatched sources, and dotted red for the unmatched sources).
The combined spectrum of these two components is illustrated with a solid black line.
The results with \ac{gde} Models 1, 2, and 3 are respectively shown in the upper left, upper right, and lower left.
The \comptel\ emission \citep{bouchet_diffuse_2011} is indicated by light green points, 
%\revise{
and light green squares are its systematic error \citep{strong_diffuse_1994}.
\label{fig:model}}
\end{figure}
%%%%%%%%%%%%%%%%%%%%%%%%%%%%

\figref{fig:model} compares the \comptel\ data points and our models with the three different models of \ac{gde}.
We show the spectrum of MeV gamma-ray sources, \ac{gde} Models~1--3, and the combined spectra of these two components.
It should be noted that this direct comparison did not take into account energy dispersion, which may have a significant effect on the result, as described in the data analysis of \spi\ \citep{strong_gamma-ray_2005}.
%The issue might be large in the \comptel\ data due to its property of functioning as a Compton telescope. Applying the energy response function, however, is beyond the scope of this study, and it is currently impossible because the \comptel\ response is not available. Although a small ($\sim 1\sigma$) difference between the data points of \comptel\ and the models might not be counted as an excess unless the energy dispersion is adequately considered, 
%In the following, we report the results of the direct comparison between the data and the models.
%
%%% Comments on each result.
We find that the combination of \ac{gde} and the sources can roughly reproduce the \comptel\ excess:
the entire spectrum can be sufficiently explained with Model~2 (\figref{fig:model} upper right), and
the lowest and highest energy bins of the \comptel\ data can be reproduced with Model~1 (\figref{fig:model} upper left).
Model~3 (\figref{fig:model} lower left) is slightly lower than the \comptel\ excess.


%%% Region dependence
\if0
The inner Galactic diffuse emission is roughly $10^{-2}$ \Mflux\ for all the previous studies,
however the extracted region is different;
$|\ell| \leq 30\degr$ and $|b| \leq 15\degr$ in \cite{bouchet_diffuse_2011}, $|\ell| \leq 60\degr$ and $|b| \leq 10\degr$ in \cite{strong_diffuse_2004}, $|\ell| \leq 30\degr$ and $|b| \leq 5\degr$ in \cite{strong_comptel_1999}, $|\ell| \leq 45.7\degr$ and $|b| \leq 45.7\degr$ in \cite{siegert_diffuse_2022}, as illustrated in \figref{fig:size}.
This region difference is crucial for calculating the model.
The trend in \figref{fig:model} is similar if we adopt for the region with  $|\ell| \leq 60\degr$ and $|b| \leq 10\degr$ or  $|\ell| \leq 30\degr$ and $|b| \leq 5\degr$ (see Appendix \ref{sec:region_appendix} for the results with the different regions).
In the region with $|\ell| \leq 45.7\degr$ and $|b| \leq 45.7\degr$, the model fails to reproduce the emission since it is extended to the high-latitude region, where \ac{gde} becomes faint and the number of sources decreases.
\fi


%%% Source spectrum
The spectral shape would provide us with a new constraint.
The power-law spectral index is $s \sim -1.9$ ($dN/dE \propto E^s$) for the \comptel\ excess.
%, while it is harder with  $s=-1.39$ for the latest measurement of the \spi\ emission in the region extended to the higher latitude \citep{siegert_diffuse_2022}.
\figref{fig:model} shows that the spectrum of the accumulated sources is almost flat in the SED with a spectral index of $s \sim -2$.
Since the \ac{gde} models have $s \sim -1.5$, the MeV gamma-ray sources should play an important role in reproducing the observed spectrum by \comptel\ with $s \sim -1.9$.
More precise measurements of the spectrum of each source will enable constraining the accumulative source spectrum, which in turn will be useful to determine the \ac{gde} spectrum, especially the index of the primary electrons responsible for the \ac{ic} radiation.


%%% Uncertainty of GDE
Since the dominant component in the energy range of the \comptel\ excess is \ac{gde}, the uncertainty of \ac{gde} prevents us from reaching a robust conclusion.
%The uncertainty of \ac{gde} arises from (1) the amount of \ac{cr} electrons and (2) the photon fields being up-scattered by the electrons. 
%%% CR electron
The uncertainty arises from the amount of \ac{cr} electrons.
The CR electrons in 100--1000 MeV, which produce 1--30 MeV photons via \ac{ic} scattering, are different by a factor of $\sim$4, depending on the models. 
To distinguish these models is important in the perspective of \ac{cr} feedback on galaxy evolution:
\acp{cr} can produce a non-thermal pressure gradient and enhance the degree of ionization in molecular clouds, significantly affecting a star-forming activity
(e.g., \cite{hopkins_first_2021}). 
%%% photon fields
The density of the photon fields across the Galaxy is not well constrained, which also makes the \ac{gde} model somewhat uncertain in the MeV gamma-ray band.
If we assume that the interstellar radiation field is locally enhanced, such as in the Galactic bulge or a large Galactic CR halo, \ac{gde} in the inner Galactic region is increased enough to reach the flux level of the \comptel\ excess \citep{bouchet_diffuse_2011}.



\if0
\subsection{Cosmic gamma-ray background}
\label{sec:cgb}
%%%%%%%%%%%%%%%%%%%%%%%%%
%%%%%%%%%%%%%%%%%%%%%%%%%
%\comment{move this subsection to Discussion section?}

%%% COMPTEL excess includes CGB or not?
%In this section, 
%In addition to the sources and \ac{gde} (\secref{sec:sources} and \secref{sec:gde}) as primary components that must be accounted for the \comptel\ excess, we discuss how much it would be affected by uncertainty of subtraction of \ac{cgb} as a secondary component.
Here, we discuss the uncertainty on the subtraction of \ac{cgb}.
When calculating the \comptel\ emission, there is an isotropic term, $I_B$, in Equation (1) in \cite{strong_diffuse_1994,strong_diffuse_1996}, which likely corresponds to \ac{cgb}.
Later, \cite{strong_comptel_1999,strong_diffuse_2004,bouchet_diffuse_2011} presented the \comptel\ diffuse emission with \ac{cgb} being removed, as the base level (i.e., the zero-flux level) was set to the high-latitude sky.
We need to be cautious, however, of the treatment of the isotropic term: 
First, since the isotropic term is a term with only the normalization being free, the spectral shape of \ac{cgb} was not taken into account. 
Second, the uncertainty of the background subtraction might be included in the isotropic term, although the uncertainty of the overall fit was dominated by systematic errors, estimated to be of order of $\sim$25\%  \citep{strong_diffuse_1994}. 
Therefore, there might be a possibility that the isotropic term could not completely represent \ac{cgb}.


%%% CGB
\ac{cgb} in the MeV gamma-ray band derived from \comptel\ \citep{weidenspointner_cosmic_2000,kappadath_total_1997,
kappadath_measurement_1998} is reproduced by a broken power-law model:
\begin{equation}
    I(E) = 2.2 \times 10^{-4} \left(\frac{E}{3~\mathrm{MeV}} \right)^{-\Gamma} ~  (\mathrm{MeV ~ cm}^2 ~\mathrm{s ~ sr})^{-1} , 
    %% case
    %    \left\{
    %\begin{array}{ll}
    % 2.2 \times 10^{-4} \left(\frac{E}{3~\mathrm{MeV}} \right)^{-3.3}  (E \leq 3~\mathrm{keV}) \\
    %E^{-*}  (E> 3~\mathrm{keV}) \\
    %\end{array} \right.
    \label{eq:CGB}
\end{equation}
where the spectral slope $\Gamma$ is 3.3 for $E \leq 3$ MeV and 2 for $E > 3$ MeV.
This is roughly comparable with the observation of the Solar Maximum Mission (SMM) Gamma-Ray Spectrometer (GRS) \citep{watanabe_mev_2000}. 
%%% uncertainty of CGB subtraction
A fraction of CGB (\eqref{eq:CGB}) to the \comptel\ excess is approximately 20\%, except for the lowest energy bin of which the fraction is 60\%.
Since this fraction at the higher energy bins is smaller than the systematic uncertainty of 25\%, the \comptel\ data points include the uncertainty of the subtraction of CGB.
\fi


%%% other scenarios
%Although more investigation is necessary to reveal whether there indeed is an excess in the inner Galactic diffuse emission in the MeV gamma-ray energy band, we address the other scenarios besides \ac{gde}, the sources, and \ac{cgb}.
As indicated by Model~3, we would need additional component(s) to reconcile with the \comptel\ emission.
Possible explanations are low-mass ($\lesssim$280 MeV) annihilating dark matter coupling to first-generation quarks \citep{boddy_indirect_2015} and/or 
cascaded gamma rays accompanying cosmic neutrinos \citep{fang_tev_2022},
which would open up a new window for these studies.


%   All sky
%%%%%%%%%%%%%%%%%%%%%%%%%
%%%%%%%%%%%%%%%%%%%%%%%%%
\section{MeV gamma-ray all-sky map} \label{sec:allsky}


\begin{figure}[t!]
\centering
  %%%%%%%%%%%%%%%%%%%%
  \begin{minipage}[b]{0.45\linewidth}
    \centering
    \includegraphics[keepaspectratio, width=\hsize]{figures/figure_allsky_total.pdf}
    \subcaption{All components}
  \end{minipage}
  %%%%%%%%%%%%%%%%%%%%
  \begin{minipage}[b]{0.45\linewidth}
    \centering
    \includegraphics[keepaspectratio, width=\hsize]{figures/figure_allsky_source_2.pdf}
    \subcaption{Source map}
  \end{minipage}
  %\\ \vspace{1mm} \\
  \vspace{1mm}
  %%%%%%%%%%%%%%%%%%%%
  \begin{minipage}[b]{0.45\linewidth}
    \centering
    \includegraphics[keepaspectratio, width=\hsize]{figures/figure_allsky_galactic.pdf}
    \subcaption{\acf{gde} Model~1}
  \end{minipage}
  %%%%%%%%%%%%%%%%%%%%
  \begin{minipage}[b]{0.45\linewidth}
    \centering
    \includegraphics[keepaspectratio, width=\hsize]{figures/figure_allsky_cgb.pdf}
    \subcaption{Cosmic gamma-ray background (CGB)}
  \end{minipage}
\caption{
The all-sky maps of the 1--10 MeV flux, shown in the Galactic coordinate and hammer projection. 
%The flux is derived in 1--10 MeV.
%\comment{All components, source, GDE, CGB on one figure}
\label{fig:allsky}
}
\end{figure}


This section presents predicted MeV gamma-ray all-sky maps, which would be useful for observation strategies of future missions.
The all-sky map comprises three components: sources, \acf{gde}, and \acf{cgb}.
%%% source
The source map (\figref{fig:allsky}b) consists of 187 crossmatched sources in \cite{tsuji_cross-match_2021}. 
The flux is estimated by fitting the \bat\ and \lat\ \acp{sed} by log-parabola or tow-component models, as mentioned in \secref{sec:sources}.
The extended sources are illustrated as point sources here, but this will be improved in the future work.
%%% Galactic diffuse
The \ac{gde} map (\figref{fig:allsky}c) makes use of Model~1 \cite{ackermann_fermi_2012} described in \secref{sec:gde}.
%%% CGB
For the \ac{cgb} map (\figref{fig:allsky}d), we assume the spectrum in the literature \cite{weidenspointner_cosmic_2000} and the isotropic distribution.


The 1--10 MeV all-sky maps of each component and the total are illustrated in \figref{fig:allsky}.
\ac{gde} is dominant in the low-latitude sky, while it is dominated by \ac{cgb} at the higher latitude.
The flux of \ac{gde} and \ac{cgb} are respectively 0.18--15 and 3.0 in units of $10^{-12}$~\flux,
and 147 sources have flux larger than these values at each location.
Our model predicts that there are 123 and 174 sources with the flux in 1--10 MeV being larger than $10^{-11}$ and $10^{-12}$ \flux, respectively.
These sources would be good targets for future missions \cite{COSI_SMEX,fleischhack_amego-x_2021,Aramaki2020,takada_first_2022}.
%These maps would be helpful for developing a strategy to maximize scientific outcomes of future missions \cite{COSI_SMEX,fleischhack_amego-x_2021,Aramaki2020,takada_first_2022}.

%\comment{describe how many sources} In the inner Galactic region ($|\ell| \leq 30\degr$ and $|b| \leq 15$\degr), our source model predicts that there are 5, 26, and 28 sources with the flux in 1--10 MeV being larger than $10^{-10}$, $10^{-11}$, and $10^{-12}$ \flux, respectively.







%\subsection{Data release}
\paragraph{Data release}
We provide all of the resources of this study (MeV gamma-ray source catalog, Galdef files of the \ac{gde} models, and all-sky FITS files) in \url{https://tsuji703.github.io/MeV-All-Sky}, which will be updated sometimes.



%%%%%%%%%%%%%%%%%%%%%%%%%
%%%%%%%%%%%%%%%%%%%%%%%%%


%\section{Prospects for future missions and conclusions} \label{sec:conclusion}
%\section{Summary} \label{sec:conclusion}
%%%%%%%%%%%%%%%%%%%%%%%%%
%%%%%%%%%%%%%%%%%%%%%%%%%

%%% Future mission. 
\if0
In order to have an advanced understanding of the inner Galactic diffuse emission, observations with much better performance (i.e., greater angular resolution and larger effective area) are desired. 
Such observations have not been achieved in the two decades since \comptel.
There are several ongoing or planned projects of the MeV gamma-ray observation,
such as COSI-SMEX \citep{COSI_SMEX} to be launched as a satellite in 2026,
e-ASTROGAM \citep{eASTROGAM2018}, AMEGO-X \citep{fleischhack_amego-x_2021}, GRAMS \citep{Aramaki2020}, SMILE-3 \citep{takada_first_2022}, GECCO \citep{moiseev_new_2021}, and a CubeSat for MeV observations (MeVCube) \citep{lucchetta_introducing_2022}.
With these future missions, we need to resolve the individual MeV gamma-ray sources first.
In the inner Galactic region ($|\ell| \leq 30\degr$ and $|b| \leq 15$\degr), our source model predicts that there are 5, 26, and 28 sources with the flux in 1--10 MeV being larger than
$10^{-10}$, $10^{-11}$, and $10^{-12}$ \flux, respectively.
%In the inner Galactic region ($|\ell| \leq 60\degr$ and $|b| \leq 10$\degr), our source model predicts that there are 8, 32, and 36 sources with the flux in 1–10 MeV being larger than $10^{-10}$, $10^{-11}$, and $10^{-12}$ \flux, respectively.
These sources can be detectable by the observatory whose sensitivity is improved by 1--2 orders of magnitude from \comptel.
After subtracting the source contribution and \ac{cgb}, we can constrain \ac{gde} with higher accuracy, %, combined with the cosmic-ray, GeV gamma-ray, and synchrotron radio data.
then we can clarify the presence of the \comptel\ excess.
\fi


%\section{Conclusion} \label{sec:conclusion}
%%%%%%%%%%%%%%%%%%%%%%%%%
%%%%%%%%%%%%%%%%%%%%%%%%%
\if0
We elaborated on models consisting of MeV gamma-ray objects, resulting in $\sim$20\% contribution to the inner Galactic diffuse emission. Combined with \ac{gde}, the \comptel\ emission can be roughly reproduced. 
We also produced and provided preliminary MeV gamma-ray all-sky maps, consisting of the sources, \ac{gde}, and \ac{cgb}.
These maps would be helpful for developing a strategy to maximize scientific outcomes of future missions \cite{COSI_SMEX,fleischhack_amego-x_2021,Aramaki2020,takada_first_2022}.
\fi


\if0
To clarify the origin of the \comptel\ excess, we elaborated on models consisting of MeV gamma-ray objects and \ac{gde}.
The crossmatched sources (both the hard X-ray and GeV gamma-ray emitters) have contributions of $\sim$10\% to the \comptel\ diffuse emission, and the contribution of the unmatched sources  (either of the hard X-ray or GeV gamma-ray emitters) is also at the same level. 
Although the most uncertain component of \ac{gde} prevents us from a robust conclusion, we found that the combination of all the components can roughly reproduce the \comptel\ excess, except for the \ac{gde} model with the smallest flux. 
With future missions \cite{COSI_SMEX,eASTROGAM2018,fleischhack_amego-x_2021,Aramaki2020,takada_first_2022,moiseev_new_2021,lucchetta_introducing_2022}, 
we would be able to discriminate between the \ac{gde} models, enabling us to determine the amount of low-energy \ac{cr} electrons and characterize their role in the galaxy evolution, and confirm the existence of the \comptel\ excess, opening up a new window for dark matter or neutrinos if it exists.
\fi



%%%%%%%%%%%%%%%%%%%%%%%%%
%%%%%%%%%%%%%%%%%%%%%%%%%
\acknowledgments
%{\small
We thank the GRAMS collaboration \cite{Aramaki2020} and the MeV gamma-ray community.
N.T. acknowledges support from the Japan Society for the Promotion of Science KAKENHI grant No. 22K14064.
%}


%%%%%%%%%%%%%%%%%%%%%%%%%%%%%%%%%
%   reference
%%%%%%%%%%%%%%%%%%%%%%%%%%%%%%%%%
%\section*{Recerences}
%\footnotesize

%\comment{Cite Tsuji+ 2022 (\comptel\ excess paper)}

%\bibliographystyle{JHEP}
%\bibliography{references.bib}

\begin{thebibliography}{99}


\bibitem[{Ackermann {et~al.}(2012)Ackermann, Ajello, Atwood, Baldini, Ballet,
  Barbiellini, Bastieri, Bechtol, Bellazzini, Berenji, Blandford, Bloom,
  Bonamente, Borgland, Brandt, Bregeon, Brigida, Bruel, Buehler, Buson,
  Caliandro, Cameron, Caraveo, Cavazzuti, Cecchi, Charles, Chekhtman, Chiang,
  Ciprini, Claus, Cohen-Tanugi, Conrad, Cutini, de~Angelis, de~Palma, Dermer,
  Digel, do~Couto~e Silva, Drell, Drlica-Wagner, Falletti, Favuzzi, Fegan,
  Ferrara, Focke, Fortin, Fukazawa, Funk, Fusco, Gaggero, Gargano, Germani,
  Giglietto, Giordano, Giroletti, Glanzman, Godfrey, Grove, Guiriec,
  Gustafsson, Hadasch, Hanabata, Harding, Hayashida, Hays, Horan, Hou, Hughes,
  Jóhannesson, Johnson, Johnson, Kamae, Katagiri, Kataoka, Knödlseder, Kuss,
  Lande, Latronico, Lee, Lemoine-Goumard, Longo, Loparco, Lott, Lovellette,
  Lubrano, Mazziotta, McEnery, Michelson, Mitthumsiri, Mizuno, Monte, Monzani,
  Morselli, Moskalenko, Murgia, Naumann-Godo, Norris, Nuss, Ohsugi, Okumura,
  Omodei, Orlando, Ormes, Paneque, Panetta, Parent, Pesce-Rollins,
  Pierbattista, Piron, Pivato, Porter, Rainò, Rando, Razzano, Razzaque,
  Reimer, Reimer, Sadrozinski, Sgrò, Siskind, Spandre, Spinelli, Strong,
  Suson, Takahashi, Tanaka, Thayer, Thayer, Thompson, Tibaldo, Tinivella,
  Torres, Tosti, Troja, Usher, Vandenbroucke, Vasileiou, Vianello, Vitale,
  Waite, Wang, Winer, Wood, Wood, Yang, Ziegler, \&
  Zimmer}]{ackermann_fermi_2012}
Ackermann, M., Ajello, M., Atwood, W.~B., {et~al.} 2012, ApJ, 750, 3 %\dodoi{10.1088/0004-637X/750/1/3}

\if0
\bibitem[{Ackermann {et~al.}(2017)Ackermann, Ajello, Albert, Atwood, Baldini,
  Ballet, Barbiellini, Bastieri, Bellazzini, Bissaldi, Blandford, Bloom,
  Bonino, Bottacini, Brandt, Bregeon, Bruel, Buehler, Burnett, Cameron, Caputo,
  Caragiulo, Caraveo, Cavazzuti, Cecchi, Charles, Chekhtman, Chiang, Chiappo,
  Chiaro, Ciprini, Conrad, Costanza, Cuoco, Cutini, D’Ammando, Palma,
  Desiante, Digel, Lalla, Mauro, Venere, Drell, Favuzzi, Fegan, Ferrara, Focke,
  Franckowiak, Fukazawa, Funk, Fusco, Gargano, Gasparrini, Giglietto, Giordano,
  Giroletti, Glanzman, Gomez-Vargas, Green, Grenier, Grove, Guillemot, Guiriec,
  Gustafsson, Harding, Hays, Hewitt, Horan, Jogler, Johnson, Kamae, Kocevski,
  Kuss, Mura, Larsson, Latronico, Li, Longo, Loparco, Lovellette, Lubrano,
  Magill, Maldera, Malyshev, Manfreda, Martin, Mazziotta, Michelson, Mirabal,
  Mitthumsiri, Mizuno, Moiseev, Monzani, Morselli, Negro, Nuss, Ohsugi,
  Orienti, Orlando, Ormes, Paneque, Perkins, Persic, Pesce-Rollins, Piron,
  Principe, Rainò, Rando, Razzano, Razzaque, Reimer, Reimer, Sánchez-Conde,
  Sgrò, Simone, Siskind, Spada, Spandre, Spinelli, Suson, Tajima, Tanaka,
  Thayer, Tibaldo, Torres, Troja, Uchiyama, Vianello, Wood, Wood, Zaharijas,
  Zimmer, \& {(The Fermi LAT Collaboration)}}]{ackermann_fermi_2017}
Ackermann, M., Ajello, M., Albert, A., {et~al.} 2017, The Astrophysical
  Journal, 840, 43, \dodoi{10.3847/1538-4357/aa6cab}
\fi

\bibitem[{Aramaki {et~al.}(2020)Aramaki, Adrian, Karagiorgi, \&
  Odaka}]{Aramaki2020}
Aramaki, T., Adrian, P. O.~H., Karagiorgi, G., \& Odaka, H. 2020, Astroparticle Physics, 114, 107%, \dodoi{10.1016/j.astropartphys.2019.07.002}

\bibitem[{Ballet {et~al.}(2020)Ballet, Burnett, Digel, \& Lott}]{4fgldr2}
Ballet, J., Burnett, T.~H., Digel, S.~W., \& Lott, B. 2020, arXiv:2005.11208
%  .\newblock \url{http://arxiv.org/abs/2005.11208}

\bibitem[{Binder {et~al.}(2022)Binder, Chakraborti, Matsumoto, \&
  Watanabe}]{Binder_2022}
Binder, T., Chakraborti, S., Matsumoto, S., \& Watanabe, Y. 2022, 
%A {Global}  {Analysis} of {Resonance}-enhanced {Light} {Scalar} {Dark} {Matter}, Tech.  Rep.,
arXiv:2205.10149
%\newblock \url{http://arxiv.org/abs/2205.10149}

\bibitem[{Bird {et~al.}(2016)Bird, Bazzano, Malizia, Fiocchi, Sguera, Bassani,
  Hill, Ubertini, \& Winkler}]{Bird2016}
Bird, A.~J., Bazzano, A., Malizia, A., {et~al.} 2016, ApJ
  Supplement Series, 223, 15%, \dodoi{10.3847/0067-0049/223/1/15}

\bibitem[{Boddy \& Kumar(2015)}]{boddy_indirect_2015}
Boddy, K.~K., \& Kumar, J. 2015, Physical Review D, 92, 023533%,  \dodoi{10.1103/PhysRevD.92.023533}

\bibitem[{Bouchet {et~al.}(2011)Bouchet, Strong, Porter, Moskalenko, Jourdain,
  \& Roques}]{bouchet_diffuse_2011}
Bouchet, L., Strong, A.~W., Porter, T.~A., {et~al.} 2011, ApJ, 739, 29%, \dodoi{10.1088/0004-637X/739/1/29}

\if0
\bibitem[{Christy {et~al.}(2022)Christy, Kumar, \&
  Rajaraman}]{christy_indirect_2022}
Christy, J.~G., Kumar, J., \& Rajaraman, A. 2022, Indirect {Detection} of
  {Low}-mass {Dark} {Matter} {Through} the
  \${\textbackslash}pi{\textasciicircum}0\$ and \${\textbackslash}eta\$
  {Windows}, Tech. Rep. arXiv:2205.09356, arXiv.
%\newblock \url{http://arxiv.org/abs/2205.09356}


\bibitem[{Cummings {et~al.}(2016)Cummings, Stone, Heikkila, Lal, Webber,
  Jóhannesson, Moskalenko, Orlando, \& Porter}]{cummings_galactic_2016}
Cummings, A.~C., Stone, E.~C., Heikkila, B.~C., {et~al.} 2016, The
  Astrophysical Journal, 831, 18, \dodoi{10.3847/0004-637X/831/1/18}

\bibitem[{De~Angelis {et~al.}(2018)De~Angelis, Tatischeff, Grenier, McEnery,
  Mallamaci, Tavani, Oberlack, Hanlon, Walter, Argan, Von~Ballmoos, Bulgarelli,
  Bykov, Hernanz, Kanbach, Kuvvetli, Pearce, Zdziarski, Conrad, Ghisellini,
  Harding, Isern, Leising, Longo, Madejski, Martinez, Mazziotta, Paredes, Pohl,
  Rando, Razzano, Aboudan, Ackermann, Addazi, Ajello, Albertus, Álvarez,
  Ambrosi, Antón, Antonelli, Babic, Baibussinov, Balbo, Baldini, Balman,
  Bambi, Barres~de Almeida, Barrio, Bartels, Bastieri, Bednarek, Bernard,
  Bernardini, Bernasconi, Bertucci, Biland, Bissaldi, Boettcher, Bonvicini,
  Bosch-Ramon, Bottacini, Bozhilov, Bretz, Branchesi, Brdar, Bringmann, Brogna,
  Budtz~Jørgensen, Busetto, Buson, Busso, Caccianiga, Camera, Campana,
  Caraveo, Cardillo, Carlson, Celestin, Cermeño, Chen, Cheung, Churazov,
  Ciprini, Coc, Colafrancesco, Coleiro, Collmar, Coppi, Curado~da Silva,
  Cutini, D'Ammando, De~Lotto, de~Martino, De~Rosa, Del~Santo, Delgado, Diehl,
  Dietrich, Dolgov, Domínguez, Dominis~Prester, Donnarumma, Dorner, Doro,
  Dutra, Elsaesser, Fabrizio, Fernández-Barral, Fioretti, Foffano, Formato,
  Fornengo, Foschini, Franceschini, Franckowiak, Funk, Fuschino, Gaggero,
  Galanti, Gargano, Gasparrini, Gehrz, Giammaria, Giglietto, Giommi, Giordano,
  Giroletti, Ghirlanda, Godinovic, Gouiffés, Grove, Hamadache, Hartmann,
  Hayashida, Hryczuk, Jean, Johnson, José, Kaufmann, Khelifi, Kiener,
  Knödlseder, Kole, Kopp, Kozhuharov, Labanti, Lalkovski, Laurent, Limousin,
  Linares, Lindfors, Lindner, Liu, Lombardi, Loparco, López-Coto, López~Moya,
  Lott, Lubrano, Malyshev, Mankuzhiyil, Mannheim, Marchã, Marcianò, Marcote,
  Mariotti, Marisaldi, McBreen, Mereghetti, Merle, Mignani, Minervini, Moiseev,
  Morselli, Moura, Nakazawa, Nava, Nieto, Orienti, Orio, Orlando, Orleanski,
  Paiano, Paoletti, Papitto, Pasquato, Patricelli, Pérez-García, Persic,
  Piano, Pichel, Pimenta, Pittori, Porter, Poutanen, Prandini, Prantzos,
  Produit, Profumo, Queiroz, Rainó, Raklev, Regis, Reichardt, Rephaeli, Rico,
  Rodejohann, Rodriguez~Fernandez, Roncadelli, Roso, Rovero, Ruffini, Sala,
  Sánchez-Conde, Santangelo, Saz~Parkinson, Sbarrato, Shearer, Shellard,
  Short, Siegert, Siqueira, Spinelli, Stamerra, Starrfield, Strong, Strümke,
  Tavecchio, Taverna, Terzić, Thompson, Tibolla, Torres, Turolla, Ulyanov,
  Ursi, Vacchi, Van~den Abeele, Vankova-Kirilovai, Venter, Verrecchia, Vincent,
  Wang, Weniger, Wu, Zaharijaš, Zampieri, Zane, Zimmer, \&
  Zoglauer}]{eASTROGAM2018}
De~Angelis, A., Tatischeff, V., Grenier, I.~A., {et~al.} 2018, Journal of High
  Energy Astrophysics, 19, 1, \dodoi{10.1016/j.jheap.2018.07.001}
\fi

\bibitem[{Fang {et~al.}(2022)Fang, Gallagher, \& Halzen}]{fang_tev_2022}
Fang, K., Gallagher, J.~S., \& Halzen, F. 2022, arXiv:2205.03740
%\newblock \url{http://arxiv.org/abs/2205.03740}

\bibitem[{Fleischhack (2021)}]{fleischhack_amego-x_2021}
Fleischhack, H., 2021, 
%{AMEGO}-{X}: {MeV} gamma-ray {Astronomy} in the{Multimessenger} {Era},  
arXiv.2108.02860


%\bibitem[{Hayashida {et~al.}(2013)Hayashida, Stawarz, Cheung, Bechtol,
%  Madejski, Ajello, Massaro, Moskalenko, Strong, \& Tibaldo}]{hayashida_discovery_2013} Hayashida, M., Stawarz, L., Cheung, C.~C., {et~al.} 2013, The Astrophysical Journal, 779, 131, \dodoi{10.1088/0004-637X/779/2/131}

\bibitem[{Hopkins {et~al.}(2021)Hopkins, Butsky, Panopoulou, Ji, Quataert,
  Faucher-Giguere, \& Keres}]{hopkins_first_2021}
Hopkins, P.~F., Butsky, I.~S., Panopoulou, G.~V., {et~al.} 2021,
  arXiv:2109.09762
%\newblock \url{http://arxiv.org/abs/2109.09762}

\if0
\bibitem[{Jubelgas {et~al.}(2008)Jubelgas, Springel, Enßlin, \&
  Pfrommer}]{jubelgas_cosmic_2008}
Jubelgas, M., Springel, V., Enßlin, T., \& Pfrommer, C. 2008, A\& A, 481, 33, \dodoi{10.1051/0004-6361:20065295}

\bibitem[{Kappadath(1998)}]{kappadath_measurement_1998}
Kappadath, S.~C. 1998, PhD thesis, UNIVERSITY OF NEW HAMPSHIRE.
\newblock \url{https://ui.adsabs.harvard.edu/abs/1998PhDT.........3K}

\bibitem[{Kappadath {et~al.}(1997)Kappadath, Ryan, Bennett, Bloemen, Diehl,
  Hermsen, McConnell, Schönfelder, Varendorff, Weidenspointner, \&
  Winkler}]{kappadath_total_1997}
Kappadath, S.~C., Ryan, J., Bennett, K., {et~al.} 1997, in {AIP} {Conference}
  {Proceedings} (Williamsburg, Virginia (USA): AIP), 1218--1222,
  \dodoi{10.1063/1.54105}

\bibitem[{Lucchetta {et~al.}(2022)Lucchetta, Ackermann, Berge, \&
  Bühler}]{lucchetta_introducing_2022}
Lucchetta, G., Ackermann, M., Berge, D., \& Bühler, R. 2022, arXiv:2204.01325
  .
\newblock \url{http://arxiv.org/abs/2204.01325}

\bibitem[{Moiseev \& {GECCO collaboration}(2021)}]{moiseev_new_2021}
Moiseev, A., \& {GECCO collaboration}. 2021, in Proceedings of 37th
  {International} {Cosmic} {Ray} {Conference} — {PoS}({ICRC2021}) (Berlin,
  Germany - Online: Sissa Medialab), 648, \dodoi{10.22323/1.395.0648}

\bibitem[{Murgia(2020)}]{murgia_fermilat_2020}
Murgia, S. 2020, Annual Review of Nuclear and Particle Science, 70, 455,
  \dodoi{10.1146/annurev-nucl-101916-123029}
\fi 

\bibitem[{Orlando(2018)}]{orlando_imprints_2018}
Orlando, E. 2018, MNRAS, 475, 2724%, \dodoi{10.1093/mnras/stx3280}

%\bibitem[{Owen {et~al.}(2021)Owen, On, Lai, \& Wu}]{owen_observational_2021}
%Owen, E.~R., On, A. Y.~L., Lai, S.-P., \& Wu, K. 2021, The Astrophysical Journal, 913, 52, \dodoi{10.3847/1538-4357/abee1a}

\bibitem[{Porter {et~al.}(2017)Porter, Jóhannesson, \&
  Moskalenko}]{porter_high-energy_2017}
Porter, T.~A., Jóhannesson, G., \& Moskalenko, I.~V. 2017, ApJ, 846, 67%, \dodoi{10.3847/1538-4357/aa844d}

%\bibitem[{Porter {et~al.}(2022)Porter, Jóhannesson, \& Moskalenko}]{galprop_v57}
%---. 2022, The Astrophysical Journal Supplement Series, 262, 30,
% \dodoi{10.3847/1538-4365/ac80f6}

%\bibitem[{Schoenfelder {et~al.}(1993)Schoenfelder, Aarts, Bennett, de~Boer,  Clear, Collmar, Connors, Deerenberg, Diehl, von Dordrecht, den Herder,  Hermsen, Kippen, Kuiper, Lichti, Lockwood, Macri, McConnell, Morris, Much,  Ryan, Simpson, Snelling, Stacy, Steinle, Strong, Swanenburg, Taylor,  de~Vries, \& Winkler}]{Comptel} Schoenfelder, V., Aarts, H., Bennett, K., {et~al.} 1993, ApJ Supplement Series, 86, 657 %, \dodoi{10.1086/191794}

\if0
\bibitem[{Schönfelder {et~al.}(2000)Schönfelder, Bennett, Blom, Bloemen,
  Collmar, Connors, Diehl, Hermsen, Iyudin, Kippen, Knödlseder, Kuiper,
  Lichti, McConnell, Morris, Much, Oberlack, Ryan, Stacy, Steinle, Strong,
  Suleiman, Dijk, Varendorff, Winkler, \& Williams}]{Schonfelder2000}
Schönfelder, V., Bennett, K., Blom, J.~J., {et~al.} 2000, Astronomy and
  Astrophysics Supplement Series, 143, 145, \dodoi{10.1051/aas:2000101}
\fi

\bibitem[{Siegert {et~al.}(2022)Siegert, Berteaud, Calore, Serpico, \&
  Weinberger}]{siegert_diffuse_2022}
Siegert, T., Berteaud, J., Calore, F., Serpico, P.~D., \& Weinberger, C. 2022,
  A\& A 600, A130 %, \dodoi{10.1051/0004-6361/202142639}

\if0
\bibitem[{Strong {et~al.}(1999)Strong, Bloemen, Diehl, Hermsen, \&
  Schönfelder}]{strong_comptel_1999}
Strong, A.~W., Bloemen, H., Diehl, R., Hermsen, W., \& Schönfelder, V. 1999,
  Astrophysical Letters and Communications, 39, 209.
\newblock \url{https://ui.adsabs.harvard.edu/abs/1999ApL&C..39..209S/abstract}
\fi

\bibitem[{Strong {et~al.}(2005)Strong, Diehl, Halloin, Schönfelder, Bouchet,
  Mandrou, Lebrun, \& Terrier}]{strong_gamma-ray_2005}
Strong, A.~W., Diehl, R., Halloin, H., {et~al.} 2005, A\&A, 444, 495 %, \dodoi{10.1051/0004-6361:20053798}

\bibitem[{Strong {et~al.}(2004)Strong, Moskalenko, \&
  Reimer}]{strong_diffuse_2004}
Strong, A.~W., Moskalenko, I.~V., \& Reimer, O. 2004, ApJ, 613, 962 %, \dodoi{10.1086/423193}

%\bibitem[{Strong {et~al.}(1988)Strong, Bloemen, Dame, Grenier, Hermsen, Lebrun, Nyman, Pollock, \& Thaddeus}]{strong_radial_1988}
%Strong, A.~W., Bloemen, J. B. G.~M., Dame, T.~M., {et~al.} 1988, Astronomy and Astrophysics, Vol. 207, p. 1-15 (1988), 207, 1.
%\newblock \url{https://ui.adsabs.harvard.edu/abs/1988A%26A...207....1S/abstract}

\bibitem[{Strong {et~al.}(1994)Strong, Bennett, Bloemen, Diehl, Hermsen,
  Morris, Schoenfelder, Stacy, de~Vries, Varendorff, Winkler, \&
  Youssefi}]{strong_diffuse_1994}
Strong, A.~W., Bennett, K., Bloemen, H., {et~al.} 1994, A\&A, Vol. 292, p. 82-91 %(1994), 292, 82
%\newblock  \url{https://ui.adsabs.harvard.edu/abs/1994A%26A...292...82S/abstract}

\bibitem[{Strong {et~al.}(1996)Strong, Bennett, Bloemen, Diehl, Hermsen,
  Purcell, Schoenfelder, Stacy, Winkler, \& Youssefi}]{strong_diffuse_1996}
Strong, A.~W., Bennett, K., Bloemen, H., {et~al.} 1996, A\&A Supplement, v.120, p.381-387 %, 120, 381
%\newblock \url{https://ui.adsabs.harvard.edu/abs/1996A%26AS..120C.381S/abstract}

%\bibitem[{Su {et~al.}(2010)Su, Slatyer, \& Finkbeiner}]{su_giant_2010}
%Su, M., Slatyer, T.~R., \& Finkbeiner, D.~P. 2010, The Astrophysical Journal, 724, 1044, \dodoi{10.1088/0004-637X/724/2/1044}

\bibitem[{Takada {et~al.}(2022)Takada, Takemura, Yoshikawa, Mizumura, Ikeda,
  Nakamura, Onozaka, Abe, Hamaguchi, Kubo, Kurosawa, Miuchi, Saito, Sawano, \&
  Tanimori}]{takada_first_2022}
Takada, A., Takemura, T., Yoshikawa, K., {et~al.} 2022, arXiv:2107.00180
%\newblock \url{http://arxiv.org/abs/2107.00180}

\if0
\bibitem[{{The AMS Collaboration} {et~al.}(2014){The AMS Collaboration},
  {Aguilar}, {M.}, Aisa, Alvino, Ambrosi, Andeen, Arruda, Attig, Azzarello,
  Bachlechner, \& {et al.}}]{AMS2014_ep}
{The AMS Collaboration}, {Aguilar}, {M.}, {et~al.} 2014, Physical Review
  Letters, 113, 121102, \dodoi{10.1103/PhysRevLett.113.121102}

\bibitem[{{The AMS Collaboration} {et~al.}(2015){The AMS Collaboration},
  {Aguilar}, {M.}, Aisa, Alpat, Alvino, Ambrosi, Andeen, Arruda, Attig,
  Azzarello, Bachlechner, \& {et al.}}]{AMS2015_proton}
---. 2015, Physical Review Letters, 114, 171103,
  \dodoi{10.1103/PhysRevLett.114.171103}
\fi

\bibitem[{Tomsick {et~al.}(2019)Tomsick, Zoglauer, Sleator, Lazar, Beechert,
  Boggs, Roberts, Siegert, Lowell, Wulf, Grove, Phlips, Brandt, Smale, Kierans,
  Burns, Hartmann, Leising, Ajello, Fryer, Amman, Chang, Jean, \& von
  Ballmoos}]{COSI_SMEX}
Tomsick, J.~A., Zoglauer, A., Sleator, C., {et~al.} 2019, arXiv:1908.04334
%\newblock \url{http://arxiv.org/abs/1908.04334}

\bibitem[{Tsuji {et~al.}(2021)Tsuji, Yoneda, Inoue, Aramaki, Karagiorgi,
  Mukherjee, \& Odaka}]{tsuji_cross-match_2021}
Tsuji, N., Yoneda, H., Inoue, Y., {et~al.} 2021, ApJ,
  916, 28%, \dodoi{10.3847/1538-4357/ac0341}

\bibitem[{Tsuji {et~al.}(2022)Tsuji, Inoue, Yoneda, Mukherjee, \& Odaka}]{tsuji_2022}
Tsuji, N., Inoue, Y., Yoneda, H., {et~al.} 2023,  
ApJ, 943, 48
%arXiv:2212.05713 

\if0
\bibitem[{Vladimirov {et~al.}(2011)Vladimirov, Digel, Jóhannesson, Michelson,
  Moskalenko, Nolan, Orlando, Porter, \& Strong}]{vladimirov_galprop_2011}
Vladimirov, A.~E., Digel, S.~W., Jóhannesson, G., {et~al.} 2011, Computer
  Physics Communications, 182, 1156, \dodoi{10.1016/j.cpc.2011.01.017}

\bibitem[{Watanabe(2000)}]{watanabe_mev_2000}
Watanabe, K. 2000, in {AIP} {Conference} {Proceedings}, Vol. 510 (Portsmouth,
  New Hampshire (USA): AIP), 471--475, \dodoi{10.1063/1.1303252}
\fi

\bibitem[{Weidenspointner(2000)}]{weidenspointner_cosmic_2000}
Weidenspointner, G. 2000, in {AIP} {Conference} {Proceedings}, Vol. 510, 467--470%, \dodoi{10.1063/1.1307028}

\end{thebibliography}
%\documentclass[10pt,twocolumn]{article} 
\usepackage{simpleConference}
\usepackage{times}
\usepackage{graphicx}
\usepackage{amssymb}
\usepackage{url,hyperref}
\usepackage{caption}
\usepackage{subcaption}

\begin{document}

\title{EFFICIENTWORD-NET: AN OPEN SOURCE HOTWORD DETECTION ENGINE BASED ON ONE-SHOT LEARNING}

\author{Chidhambararajan R, Aman Rangapur, Dr. Sibi Chakkravarthy \\
\\
Vellore Institute of Technology-AP, India \\
%  \\
% \today
\\
%  \\
}


% }

\maketitle
\thispagestyle{empty}

\begin{abstract}
Voice assistants like Siri, Google Assistant, Alexa etc. are used widely across the globe for home automation, these require the use of special phrases also known as hotwords to wake it up and perform an action like "Hey Alexa!", "Ok, Google!", "Hey Siri!" etc. These hotwords are detected with lightweight real-time engines whose purpose is to detect the hotwords uttered by the user. This paper presents the design and implementation of a lightweight, easy-to-implement hotword detection engine based on one-shot learning which detects the hotword uttered by the user in real-time with just one or few training samples of the hotword. This approach is efficient compared to existing implementations because the process of adding a new hotword in the existing systems requires enormous amounts of positive and negative training samples and the model needs to retrain for every hotword. This makes the existing implementations inefficient in terms of computation and cost. The architecture proposed in this paper has achieved an accuracy of 94.51\%.
\end{abstract}


\Section{INTRODUCTION}
With the advent of the Internet of Things (IoT) and home automation, there is a growing need for voice automation in edge devices, but running a heavy Text To Speech (TTS) Engine is too computationally expensive in these edge devices \cite{amodei2015deep}. Instead, we can run engines that need to listen for specific activation phrases called "Hotwords" to perform certain actions since the detection of hotwords is computationally less expensive than full-blown TTS engines \cite{Yang_Jee_Leblanc_Weaver_Armand_2020}.

Lightweight models \cite{Yang_Jee_Leblanc_Weaver_Armand_2020} are trained for detecting these hotwords from audio streams. This is used to save resources from heavy models such as speech recognition from running all day. The Core application for hotword detection is shown in Fig. \ref{FIG:Illustration_of_the_proposed_architecture}.


\begin{figure}
\centering
\includegraphics[width=0.45\textwidth]{images/Fig1_Illustration_proposed_architecture}
\caption{Overview of the proposed model.}
\label{FIG:Illustration_of_the_proposed_architecture}
\end{figure}

Convolutional Neural Networks (CNNs) have proven to be the best in analysing image data \cite{Hershey_2017}. Audio files are converted into Log Mel spectrograms where various frequencies are distributed on the Mel scale and plotted as an image \cite{Atsavasirilert_2019,9413528}. This image data is further analysed by the CNNs to get maximum optimacy \cite{Hershey_2017}. The base network is further attached to a Siamese network which learns to output embedding vectors with less distance for similar hotwords and huge distance for dissimilar hotwords. This way a state of the art accuracy is achieved for hotword detection with fewer audio examples of the hotwords. To our knowledge, this is the first attempt to solve the problem of retraining hotwords with one shot / few-shot learning. This approach is highly inspired by Face-Net \cite{Schroff_2015}, one-shot learning deployed for face recognition, which allows us to add a new face to the system without retraining the model \cite{Schroff_2015}. Except for the well-known hotword detection engines with low accuracy, other engines require huge datasets with positive and negative samples for training new hotword and all of these are closed source \cite{Yang_Jee_Leblanc_Weaver_Armand_2020,kalith_2012}. 

EfficientNet \cite{tan2020efficientnet} is one of the most efficient CNN architecture to the date, and the first four blocks of EfficientNet (B0 variant) is chosen for the base model in the Siamese network. To train the Siamese network, positive and negative pairs of audios are given to the network and trained to output 1, 0 for positive pairs and negative pairs respectively to know how similar the pronunciations of words. In the edge device, raw audio is continuously read(with 1 sec time window), converted to Log Mel spectrogram, from which real-time vector embeddings are calculated, these embeddings are compared against a pre-calculated vector embedding of the desired hotword for similarity \cite{Vargas_2020}.


\Section{EXISTING RELATED WORKS}
\label{Sec:Related_Works}
The problem of detecting hotwords from audio streams started ever since the advent of voice-enabled IoT devices \cite{Ooi_2019,Michaely_2017, inproceedings2_Reis_2018,inproceedings3_Todisco_2019,Tom_2018,He_Kaiming_2016,He_2016,Lin_2018,becker2019interpreting,Uitdenbogerd_2004,Tang_2018}. Porcupine \cite{picovoice_alireza} is a closed source hotword detection framework which detects hotwords but requires a commercial licence. It has an accuracy of 94\%. A customizable hotword detection engine called Snowboy can be used to create your own hotwords \cite{Yang_Jee_Leblanc_Weaver_Armand_2020}. It is a closed source project and requires a huge amount of data samples. PocketSphinx \cite{kalith_2012} is a lightweight variation of the CMU-Sphinx \cite{kalith_2012}, an offline Speech-to-Text (STT) engine with low accuracy. 

There are other frameworks such as howl \cite{tang-etal-2020-howl}, these frameworks require large amounts of positive and negative samples of the hotword to train and recognize the new words. Snowboy \cite{chen_yao} and howl’s \cite{tang-etal-2020-howl} accuracy depends on the size of training dataset for each hotword. STT engines can also be used to detect hotwords \cite{amodei2015deep}, existing engines STT \cite{Kubota_patent_2014} have very high accuracy but require a constant internet connection and an expensive subscription to their cloud service to run 24/7. Offline STT open source engines like DeepSpeech (from Mozilla) \cite{amodei2015deep} and Silero \cite{Silero_Models} have good accuracy, yet require lots of on-device resources, hence cannot be run 24/7. Rhino \cite{kenarsari_2018}, an on-device STT engine achieved the best among existing engines, by giving better accuracy and low resource requirement but closed source and requires a commercial license.

Most existing audio processing neural networks employ the usage of Log Mel Spectrograms \cite{Atsavasirilert_2019} and Mel Frequency Cepstral Coefficient \cite{Atsavasirilert_2019} since it conveys a better picture of the audio than conventional audio stream bits, this is due to the representation of changes for various frequencies across the audio streams. The initial development of hotword detection was achieved without noise \cite{Huang_2019} and then deployed with active noise cancellation in real-time \cite{Huang_Yiteng_Wan_Li_2019}. Active noise cancellation is often achieved with hardware first or software first or hybrid approaches. The software first approaches require audio samples recorded from the ambient space. Out of these recorded sample metrics, the maximum amplitude for noise is calculated and used as a threshold for voice activity detection. Acoustic noise reduction is applied over real-time audio with the help of obtained noise only audio samples. This approach is not practical since noise only audio samples recorded from ambient space are required. To circumvent this issue, edge voice assistant’s like Google, Amazon’s Alexa often deploy multi-microphone array systems \cite{Huang_Yiteng_Wan_Li_2019} to gather audio from all directions and separate speech audio with ease. This is a relatively simpler task as devices are surrounded with uniform noise in all directions but speech audio is not uniform in all directions. This allows speech audio to be separated from noise audio. The separated speech audio is of high quality thereby helping in achieving low False Acceptance Rates (FAR).

In a hybrid approach, hardware-based noise cancellation is utilized and the audio processing neural networks are often trained with noise, this allows the system to achieve exceptionally low FAR \cite{Huang_Yiteng_Wan_Li_2019}.

Existing audio processing neural networks are designed as audio classification neural networks \cite{Hershey_2017}. The network needs a lot of initial layers to understand the audio fragment. Later, half a dozen of layers are required for the network to understand the logic of classification. These extra layers dedicated for classification requires additional computational time thereby crippling the model while running it on real-time edge devices. In this paper, this problem was resolved using EfficientNetB0 as a base network with one-shot learning. In one-shot learning, the network requires a lot of layers to understand audio samples, but the need for additional layers to understand classification is waived off, therefore, resulting in faster inference in the edge devices.

As mentioned, the existing lightweight models \cite{Yang_Jee_Leblanc_Weaver_Armand_2020,kalith_2012} are binary classifiers that need to be retrained with a huge number of negative and positive samples for a new hotword, this results in expensive and inefficient gathering of datasets \cite{kalith_2012}. These networks needs to be retrained for newer hotwords again. Also, these engines are closed source where there is no scope of development in the future and users need to spend lots of money for the useage. Hardware first and a hybrid approach are applicable in the scenarios where the edge device’s hardware specifications are under the control of developers like edge voice assistants like Alexa \cite{Huang_2019}. But this approach doesn’t work well when the edge device is not designed by developers \cite{Huang_Yiteng_Wan_Li_2019}. Hence, there is a need for audio processing neural networks to be trained with very high amounts of noise to work robustly without need of hardware interventions.

EfficientWord-Net is an open source engine that solves the process of retraining the model for new hotwords by eliminating the requirement of huge datasets. It works efficiently with the audio samples with decent noise added in the background with a great inference time on small devices like Raspberry Pi. Moreover, our system outperformed previous existing approaches in terms of accuracy and inference.

\Section{ONE-SHOT LEARNING/SIMILARITY LEARNING}
\label{Sec:oneshotlearning}
One of the demanding situations of face recognition/hotword recognition is to achieve performance with fewer samples of the target, which means, for maximum face recognition programs, model should recognize a person given with the aid of using one photo of the man or woman's face. Traditionally deep learning algorithms do not work well with the simplest one training example or one data point for a class. In one-shot learning, a model learns from one sample to apprehend or recognize the person, and the industry needs most face recognition models to use this due to the fact a company has one or few images of every of their personnel in the database. 

% One method is to enter the photo of that person, feed it to a CNN and feature an output label using a SoftMax unit with 5 outputs corresponding to 4 persons or none of them. But this method won’t be efficient as you have got only one image and having such a small training dataset is inadequate to train a robust neural network for this particular task. This causes more trouble when a new person joins your team, as an extra person to apprehend with six outputs. And the model needs to be retrained to achieve the confidence every time, which is not a great approach \cite{Yang_Jee_Leblanc_Weaver_Armand_2020,kalith_2012, inproceedings4_Platen_2020}.

Similarity learning is a type of supervised learning where a network is trained to identify the similarity between 2 data points of the same class instead of classification or regression. The network is also trained to learn the dissimilarity between 2 data points of different sources. This similarity is used to determine whether an unlabelled data point belongs to the same class or not.

When 2 images are fed to a neural network to learn the similarity between them (inputs two images and outputs the degree of difference between the two images), the output would be a small number if the two images are of the same person. And if they are of two different people, the output would be a large number. However, the use of different types of functions keeps the value between 0 and 1. A hyper parameter($\tau$) is used as a threshold if the degree of difference is less than the threshold value, these pictures belong to the same person and vice versa. Similarly, in this paper, a threshold value of 0.2 is defined, if the degree of difference between the two hotwords is less than 0.2, those two hotwords are the same, else they are different.
\\

d\emph{(hotword1, hotword2)} $=$ degree of difference between hotwords.

d\emph{(hotword1, hotword2)}  $ < \tau$, both hotwords are same.

d\emph{(hotword1, hotword2)}  $ > \tau$, hotwords are different. 
\\


This approach of feeding 2 different images to the same convolutional network and comparing the encodings of them is called a Siamese Network \cite{Vargas_2020}.

\begin{figure} 
\centering
	\includegraphics[width=0.5\textwidth]{images/Fig2_One-shot_learning_architecture-hotword_detection}
\caption{One-shot learning architecture for hotword detection}
\label{FIG:OneshotLearningArchitecture}
\end{figure}

In Figure. \ref{FIG:OneshotLearningArchitecture}, two raw audio segments are fed to the same neural network, output encodings of input audios are captured, Euclidean distance between these two vectors are calculated, if the Euclidean distance is less than the threshold value, the hotwords in the audio are same and vice versa.

\Section{PROPOSED METHOD}
\label{Sec:efficientword-net}
\subsection{Preparation Of Dataset}
The dataset used for training the network is homegrown artificially synthesized data that was made with naturally sounding neural voices from Azure Cloud Platform and Siri. Furthermore, the voices were selected to include all available voice accents with the respective countries to ensure better performance across accents and gender. Hotword detection does not rely on a word’s meaning but only on its pronunciation. So, to generate the audio, a pool of words is selected in which each word sounds unique compared to the other words in the pool. Google’s 10000-word list \cite{kaufman_bathman_myers_hingston} is used while training the model, and these words were converted to respective phoneme sequences. The sequences were checked for similarity, the words which shared $\ge$80\% of the same phoneme sequence were removed to ensure low similarity in pronunciation among the word pool. The word pool was converted to the audio pool with the above mentioned text-to-speech services. The generated audio is combined with noises such as traffic sounds, market place sounds, office and room ambient sounds to simulate a naturally collected dataset. For each word, 5 such audio samples were generated in which each sample had a randomly chosen voice and randomly chosen background noise to ensure variety. The randomly chosen noises are then combined with the original audio with a noise factor randomly chosen between 0.05 to 0.2 (noise factor is a fraction of noise’s volume in the resulting audio). A sampling rate of 16000 Hz was chosen since audio quality below 16000 Hz became very poor. Finally, the audios are converted to Log Mel Spectrograms.

Two audios generated from the same word are chosen to make the true pair and two audios generated from two different words are chosen to make the false pair. The total number of true and false pairs generated in this method were 2694. 80\% of the data was split for training, and remaining 20\% was used as testing data to eliminate over-fitting. The network is fed with the true pair and false pairs to output a higher similarity score for true pairs and lower similarity score for false pairs.

\subsection{Network Architecture}
The input for each base network is 98x64x1 (Refer Fig. \ref {FIG:Base_network_Block}. The base network of the model is made up of the first four blocks of EfficientNetB0 architecture \cite{tan2020efficientnet}. The output from the EfficientNet layers is processed further with Conv2D layer with 32 filters and 3 stride values, processed by batch normalization and max pool layer, this is fed to a similar stack of layers. This is done to reduce the number of feature points efficiently. Finally, the output from the convolutions is flattened and attached to the dense layer with 256 units followed by L2 regularization to give the reduced vector approximation of the input.


\begin{figure}
\centering
\includegraphics[width=0.5\textwidth]{images/Fig3-Base_network_arch}
\caption{Base network Block.}
\label{FIG:Base_network_Block}
\end{figure}


A true pair or a false pair in the dataset is fed to two parallel blocks of the base network, where these parallel blocks share the weights. Euclidean distance between corresponding output vector pairs is calculated. Table. \ref{TBL:Euclidean_Distance_and_Similarity_score} shows the Euclidean distance mapped to similarity score to the scale 0-1.0. % as shown in Fig. \label{FIG:graph_similarity_score}. 

\begin{table}
\centering
\caption{Euclidean Distance and Similarity score}
\label{TBL:Euclidean_Distance_and_Similarity_score}
\begin{tabular}{ p{2cm} p{2cm}   p{2cm}}
 \hline \hline
Euclidean Distance  &Similarity Score \\
 \hline
0   &1.0 \\
 $<\tau$  &1.0 - 0.5 \\
$\tau$ &0.5 \\
 $>\tau$	&0.5 - 0 \\
\hline
\end{tabular}
\end{table}%


\begin{figure}
\centering
\includegraphics[width=0.5\textwidth]{images/Fig5-Model_Scaling_in_EfficientNet}
\caption{Model Scaling in EfficientNet.}
\label{FIG:Model_Scaling_in_EfficientNet}
\end{figure}

\subsection{Training Parameters and Loss Function}
The problem of analysing raw audio to examine images is resolved by converting the audio into Log Mel spectrograms images, this helps the neural network by allowing it to directly analyse frequency distribution over time, learning first to identify different frequencies and analyse them.

These generated images were fed to a convolution neural network that follows the EfficientNet Architecture (Refer Fig. \ref{FIG:Model_Scaling_in_EfficientNet}). It is a convolutional neural network architecture and a scaling method that uniformly scales all dimensions of depth/width/resolution using a compound coefficient. 

This network architecture was chosen since the accuracy was similar to that of ResNet which held the previous state of the art top5 accuracy of 94.51\%. Moreover, the number of parameters in ResNet was 4.9x higher than EfficientNetB0’s parameter count, thereby making it computationally more efficient than ResNet. Only 4 blocks of EfficientNetB0 were taken for the base network, the output was further attached to a Conv2D block, which was later flattened and l2 normalized to give the output vector.

Conventional Siamese neural networks use triplet loss where a baseline (anchor) input is compared to a positive(true) input and a negative(false) input with vector distance calculation metrics such as Euclidean distance, cosine distance, etc.
\\


Triplet Loss Function =
\(\max\left( {\| f\left( x^{a} \right) - f\left( x^{p} \right) \|}^{2} - {\| f\left( x^{a} \right) - f\left( x^{n} \right) \|}^{2},0 \right)\)
\\

\(x^{a}\) is an~\emph{anchor}~example.

\(x^{p}\) is a~\emph{positive}~example that has the same identity as the anchor.

\(x^{n}\) is a~\emph{negative}~example that represents a different
entity.
\\


The triplet loss function makes the neural network minimize the distance from the baseline(anchor) input to the positive (true pair), and maximize the distance from the baseline(anchor) input to the negative (false pair). It is also equipped with a threshold, which forces the network to reduce the distance between true pairs below the threshold and false pairs beyond the threshold.

The calculated distance between the pair is sent through a function F(x) which outputs close to 1 when distance is low and 0 when distance is high, thereby making the function give a score close to 1 for similar pairs and close to 0 for dissimilar pairs. This was done so that the network will be able to tell similarities between a pair of samples in terms of percentage. 
\[F\left( x \right) = 1 - \frac{x^{4}}{({\tau}^{4}+x^{4})}\]


Here x is the calculated distance between the vectors, this function gives 1 when x is 0, gradually reduces to 0.5 when x = $\tau$ (threshold,t[in the above equation]) and eventually to zero. The function is symmetric making f(x) go to zero when x\textless0, with Euclidean distances $\ge$ 0.

For each true pair, the ground truth was set 1 and for each false pair the ground truth was set to 0, this allowed us to treat the problem as a binary classification problem and easily apply binary cross-entropy loss function while training the engine.
\\


\textbf{Binary cross-entropy loss is defined by}

\[loss = - \frac{1}{N}\sum_{i = 1}^{N}y_{i} \bullet log\left( p\left( y_{i} \right) \right) + \left( 1 - y_{i} \right) \bullet log\left( 1 - p\left( y_{i} \right) \right)\]


\subsection{Optimization Strategies}

\subsubsection{Log Mel Spectrogram:~}

Many audio processing neural networks directly process the audio with several Conv1D layers stacked on top of each other to make the network understand the audio and it has a drawback. The network would first have to understand the concept of frequency and check for the distribution of various frequencies across the audio. This forces the network to allocate its initial Conv1D layers to understand the concept of frequency and learn to check for the distribution of various frequencies across the audio, which would be further analysed by the next layers to make sense of the audio. These additional preprocessing layers can be skipped by going for a Log Mel Spectrogram (a heatmap distributing various frequencies across the audio).

Since the network is being directly fed with the distribution of various frequencies across the audio, the network can allocate all of its resources to make sense of the audio directly, thereby enhancing the accuracy of the system. This method is largely inspired by Google's TensorFlow magenta (a set of audio processing tools for TensorFlow).

\begin{figure}
\centering
\includegraphics[width=0.5\textwidth]{images/Fig7-Sample_audio_converted_to_Log_Mel_Spectrogram}
\caption{Sample audio converted to Log Mel Spectrogram.}
\label{FIG:SampleaudioconvertedtoLogMelSpectrogram}
\end{figure}

\subsubsection{L2 Regularization to an output vector of base network:}

Initially, the network was trained with no L2 regularization, the accuracy was never able to cross 89.9\%, we have tried various techniques out of which adding L2 regularization to the output vector of the base network gave the best performance. With that added, our network was able to reach 95\% accuracy with noise. This performance increase was due to the fact that L2 regularization confined the output vector to the surface of a 256-dimensional unit sphere at the origin; without this confinement, it was very easy for the network to give huge distances for false pairs. Hence, the network eventually became biased, worked well only for false pairs and didn't give small distances for true pairs resulting in the performance drop.

\Section{RESULTS}
\label{Sec:results}
The network was trained with noise using a batch size of 64, 75 training steps per epoch for 42 epochs beyond which the training loss started to oscillate. Adam optimization algorithm was used with an initial learning rate of 1e-3, this learning rate was reduced by a learning rate scheduler which checks for lack of decrease in training loss for 3 epochs and reduces the learning rate by a factor of 0.1. The reduction of the learning rate was stopped when the learning rate reached an absolute minimum of 1e-5. Early stopping was scheduled with patience of 6 which stops the training process when the training loss starts to oscillate in more than 6 epochs resulting in the maximum validation accuracy of 94.51\%. Fig. \ref{FIG:EpochvsAcc1} shows the training graph accuracy/loss vs epochs with noise.

\begin{figure*} [htbp]
\begin{subfigure}{0.5\textwidth}
  \centering
  % include first image
  \includegraphics[width=.7\linewidth]{images/Fig8-Model_training_Epochs_vs_Accuracy_with_noise}  
  \caption{Model training(Epochs vs Accuracy) with noise}
  \label{FIG:EpochvsAcc1}
\end{subfigure}
\begin{subfigure}{.5\textwidth}
  \centering
  % include second image
  \includegraphics[width=.7\linewidth]{images/Fig9-Model_training_Epochs_vs_Loss_with_noise}  
  \caption{Model training(Epochs vs Loss) with noise}
  \label{FIG:EpochvsLoss1}
\end{subfigure}\\
\begin{subfigure}{.5\textwidth}
  \centering
  % include third image
  \includegraphics[width=.7\linewidth]{images/Fig10-Model_training_Epochs_vs_Accuracy_without_noise}  
\caption{Model training(Epochs vs Accuracy) without noise}
 \label{FIG:EpochvsAcc2}
  \end{subfigure}
\begin{subfigure}{.5\textwidth}
  \centering
  % include four image
  \includegraphics[width=.7\linewidth]{images/Fig11-Model_training-Epochs_vs-Loss_without_noise}  
\caption{Model training(Epochs vs Loss) without noise}
  \label{FIG:EpochvsLoss2}
  \end{subfigure}
\caption{Model Accuracy and Loss}
\label{fig:model_accuracy_loss}
\end{figure*}

The maximum validation accuracy reached without noise is 96.8\%, this accuracy was achieved by retraining the previous model with the same hyperparameters and noise factor = 0 for 14 epochs. Fig. \ref{FIG:EpochvsAcc1} - Fig. \ref{FIG:EpochvsLoss2} shows the training graph for epoch/accuracy vs loss with and without noise.  Since the real-time audio is chunked into 1-sec windows with 0.25-sec hop length for inference, the base network was found to perform inference in 0.08 seconds in Raspberry Pi 4. Hence, the model will be able to perform inference from real-time audio streams in edge devices with no latency issues. Fig. \ref{FIG:EpochvsLoss1} shows the noise training graph for loss vs epochs.


\begin{figure*}[htbp]
\begin{subfigure}{.5\textwidth}
  \centering
  \includegraphics[width=.6\linewidth]{images/Fig12-FRP_vs_FAR_for_hotword_Alexa}  
  \caption{FRP vs FAR for hotword Alexa}
  \label{FIG:FRPvsFARalexa}
\end{subfigure}
\begin{subfigure}{.5\textwidth}
  \centering
  \includegraphics[width=.6\linewidth]{images/Fig13-FRP_vs_FAR_for_hotword_computer}  
  \caption{FRP vs FAR for hotword computer}
  \label{FIG:FRPvsFARcomputer}
\end{subfigure}\\
\begin{subfigure}{.5\textwidth}
  \centering
  \includegraphics[width=.6\linewidth]{images/Fig14-FRP_vs_FAR_for_hotword_people}  
\caption{FRP vs FAR for hotword people}
 \label{FIG:FRPvsFARpeople}
  \end{subfigure}
\begin{subfigure}{.5\textwidth}
  \centering
 \includegraphics[width=.6\linewidth]{images/Fig15-FRP_vs_FAR_for_hotword_restaurant}  
\caption{FRP vs FAR for hotword restaurant}
  \label{FIG:FRPvsFARrestaurant}
  \end{subfigure}
\caption{FRP vs FAR for different Hotwords}
\label{fig:frp_vs_far}
\end{figure*}


\begin{table}
\centering
\caption{Model benchmarks}
\label{TBL:Model_benchmarks}
\begin{tabular}{ p{2cm} p{2cm}   p{2cm}}
 \hline \hline
Model   &Accuracy &Inference \\
 \hline
Efficientword-Net (Current paper) &94.51\%   &0.071ms \\
Porcupine \cite{picovoice_alireza} &94.78\% &0.02ms \\
PocketSphinx \cite{kalith_2012} &54.23\% &0.076ms \\
Snowboy \cite{Yang_Jee_Leblanc_Weaver_Armand_2020} &88.43\% &0.091ms \\
\hline
\end{tabular}
\end{table}%

After performing significance test, the resulting trained model was benchmarked with other hotword detection systems on Raspberry Pi 3 clocked at 1.2GHz (4 core) and displayed in the Table. \ref{TBL:Model_benchmarks} and was found to outperform existing closed source models in terms of accuracy by a small level.

\begin{figure*}[htbp]
\begin{subfigure}{.5\textwidth}
  \centering
  \includegraphics[width=.5\linewidth]{images/Fig16-Bar_graph_for_FRP_for_hotword_alexa}  
  \caption{Bar graph for FRP for hotword Alexa}
  \label{FIG:BarGraphAlexa}
\end{subfigure}
\begin{subfigure}{.5\textwidth}
  \centering
  \includegraphics[width=.5\linewidth]{images/Fig17-Bar_graph_for_FRP_for_hotword_computer}
  \caption{Bar graph for FRP for hotword computer}
  \label{FIG:BarGraphComputer}
\end{subfigure}
\\
\begin{subfigure}{.5\textwidth}
  \centering
  \includegraphics[width=.5\linewidth]{images/Fig18-Bar_graph_for_FRP_for_hotword_people}  
\caption{Bar graph for FRP for hotword people}
\label{FIG:BarGraphPeople}
  \end{subfigure}
\begin{subfigure}{.5\textwidth}
  \centering
 \includegraphics[width=.5\linewidth]{images/Fig19-Bar_graph_for_FRP_for_hotword_restaurant}  
\caption{Bar graph for FRP for hotword restaurant}
 \label{FIG:BarGraphRestaurant}
  \end{subfigure}
\caption{Bar graphs for FRP of different Hotwords}
\label{fig:bar_graphs}
\end{figure*}

For a given sensitivity value, False Rejection Rate (FRP) – (True Negatives) is measured by playing a set of sample audio files which include the utterance of the hotword, and then calculate the ratio of rejections to the total number of samples. False Acceptance Rate (FAR-False Positives) is  measured by playing a background audio file which must not include any utterance of the hotword calculated by dividing the number of false acceptances by the length of the background audio in hours. Figures Fig. \ref{FIG:FRPvsFARalexa} - \ref{FIG:FRPvsFARrestaurant} and \ref{FIG:BarGraphAlexa} - \ref{FIG:BarGraphRestaurant} illustrate FRP vs FAR and model performance against existing implementations for various hotwords. All these hotwords are not included in the training dataset.


\Section{CONCLUSION}
\label{Sec:conclusion}

In this paper, we proposed a one-shot learning-based hotword detection engine to solve the problem of retraining and huge dataset requirements for each new hotword with good inference time on light-weight devices. To achieve the same we implemented Siamese neural network architecture with an image processing base network made with EfficientNet, which processes the Log Mel spectrograms of the respective input audio samples. Moreover, this network could also be repurposed for phrase detection's where a program needs to check for the occurrence of a specific sentence removing the requirement of heavy speech-to-text engines in edge devices. Such an engine can allow the end-users to set custom hotwords in their systems with minimal effort.

\Section{ACKNOWLEDGMENTS} 

This research is carried out at Artificial Intelligence and Robotics (AIR) Research Centre, VIT-AP University. We also thank the management for motivating and supporting AIR Research Centre, VIT-AP University in building this project.





\bibliographystyle{abbrv}
\bibliography{output}
% \graphicspath{{images/}}
\end{document}




%%%%%%%%%%%%%%%%%%%%%%%%%%%%%%%%%
%%%%%%%%%%%%%%%%%%%%%%%%%%%%%%%%%


\end{document}
