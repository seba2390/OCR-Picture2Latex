\begin{abstract}

Users are demanding increased data security.
As a result, security is rapidly becoming a first-order design constraint in next generation computing systems.
Researchers and practitioners are exploring various security technologies to meet user demand such as trusted execution environments (e.g., Intel SGX, ARM TrustZone), homomorphic encryption, and differential privacy.
Each technique provides some degree of security, but differs with respect to threat coverage, performance overheads, as well as implementation and deployment challenges.
In this paper, we present a systemization of knowledge (SoK) on these design considerations and trade-offs using several prominent security technologies.
Our study exposes the need for \textit{software-hardware-security} codesign to realize efficient and effective solutions of securing user data.
In particular, we explore how design considerations across applications, hardware, and security mechanisms must be combined to overcome fundamental limitations in current technologies so that we can minimize performance overhead while achieving sufficient threat model coverage.
Finally, we propose a set of guidelines to facilitate putting these secure computing technologies into practice.

\end{abstract}
