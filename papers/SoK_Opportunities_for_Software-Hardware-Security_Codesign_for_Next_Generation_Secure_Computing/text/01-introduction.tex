\section{Introduction}
\label{sec:introduction}

Over the past decade, data security has emerged as a first-order design constraint.
Users have begun to demand increased accountability from data aggregators; they want to know how their data is being managed and protected against misuse or abuse.
As a result, the development of and interest in secure computing technologies has progressed rapidly over the last few years.
For example, technologies such as trusted execution environments (TEEs), homomorphic encryption (HE), and differential privacy (DP) are all rapidly maturing areas of research.

Despite recent advances, each technology presents a range of trade-offs and challenges; these challenges include:
(1) implementing the secure computing technology,
(2) verifying that the security guards are correctly implemented and provide intended threat model support,
(3) porting the technology to support a variety of applications,
and (4) maintaining practical performance targets.


Secure computing solutions present trade-offs that impact security and performance which must be carefully balanced to satisfy performance and security specification targets.
We argue that, if carefully engineered, secure computing technologies can enable the best of both worlds (good performance and security) instead of mandating one or the other (performance or security).
For example, TEEs offer confidentiality and integrity for computations executing within an isolated memory location, i.e., an enclave.
TEE support for an application like machine learning (ML) inference on sensitive data would be critical.
To deploy the application, the designer needs to port the ML inference to work with the chosen TEE's API and consider the cost of each call into the TEE.
For example, SGX enclaves have limited memory capacity.
Therefore large ML models need to be encrypted and stored on a larger, enclave-external memory and the resulting computation partitioned properly.
Model partitions must be moved into the SGX enclave, which causes page swapping overheads.
Combined, these design considerations require expertise across the software/application, hardware, and security in tandem to achieve an efficient and secure implementation.
Overcoming these challenges mandates a careful re-examination of complex cross-stack design choices to find effective and efficient solutions.

More broadly, software-hardware-security codesign has emerged as a design philosophy for realizing performance- and power-efficient secure computing technologies.
This notion has historical precedence; software-hardware codesign has a proven track record in improving the overall efficiency of computing solutions that we build to support modern applications.
Examples include fixed-point approximation for neural networks, domain-specific accelerators like digital signal processor (DSP) units, and hardware-friendly mathematical approximations.
These examples of software-hardware codesign have had tremendous practical impact; aggressive fixed-point quantization has enabled power-efficient implementations of machine learning and DSPs have enabled high performance telecommunication applications that form the backbone of mobile networks.
However, these innovations required feedback between software and hardware design considerations to arrive at the highly optimized solutions and architectures we have today.

A valuable codesign process by adding security to the mix can analogously enhance the efficacy of secure computing technologies.
To address the added complexities introduced by security considerations, experts from across hardware, software, and security need to be able to communicate through systematic and well-defined abstraction boundaries.
Unfortunately, the process of identifying these opportunities is not well systemized; many seemingly disparate computing techniques fall into the category of codesign.
It is, therefore, valuable to concretely systemize when there are codesign opportunities to allow this design iteration to occur for emerging secure computing technologies such as TEEs, HE, and DP.

{
\setlength{\belowcaptionskip}{-2mm}
\begin{figure}[t]
    \centering
    \includegraphics[width=\linewidth]{figures/codesign_flows.pdf}
    \caption{Hardware acceleration versus software-hardware codesign and software-hardware-security codesign. (a) Hardware acceleration does not integrate hardware design feedback. (b) Software/hardware codesign integrates feedback. (c) Software-hardware-security codesign additionally integrates security considerations.
    }
    \label{fig:sw_hw_codesign}
\end{figure}
}

We propose a systemization of knowledge to analyze the hardware, software, and security design considerations required to realize practical and efficient solutions for emerging secure computing technologies.
We start by defining the notion of software-hardware-security codesign (\autoref{sec:background}).
We then show there is historical precedence in the selection process of the advanced encryption standard and apply our systemization to emerging secure computing technologies (\autoref{sec:systemization}): differential privacy, trusted execution environments, and homomorphic encryption.
We find that a common theme across these secure computing technologies is the prevalence of \textit{feedback loops} to inform and drive the implementations towards better designs.
Finally, we synthesize these insights (\autoref{sec:themes}) to motivate the need to put software-hardware-security codesign into practice and enable security as a first-order design constraint.
