\section{Looking Forward}

\label{sec:future-work}

Our systemization illustrates that the codesign process for secure computing technologies is an interdisciplinary process which currently requires deep specialized knowledge across applications, hardware, and security to conduct.
In other words, only an expert or group of experts who understand all three disciplines can successfully perform the codesign exercise.
Going forward, to facilitate wider adoption and deployment, experts who understand the software-hardware-security codesign considerations will need to build automation support and infrastructure to abstract away the expertise requirements so that non-experts can focus on building on top of these technologies.
To do this, we need to build accessible and robust tooling infrastructure similar to how LLVM and GCC compiler stacks abstracted away the complexities of compilation passes and optimization to allow programmers to focus on building applications.

Ongoing efforts towards these types of tools include compilation support for homomorphic encryption~\cite{chet, eva}, automatic partitioning for SGX~\cite{lind2017glamdring}, and programming abstractions for differential privacy~\cite{barthe2016programming, mcsherry2009privacy, roy2010airavat}.
In each case, the tools abstract away the details of how the underlying secure computing technology is implemented and automates the specialized expertise so that it can be reused by non-experts.
This allows a designer who may have no knowledge of these secure computing technologies to still leverage the technology.
While many efforts in this space are still nascent, these are the sorts of tools that will enable wider adoption by designers and put secure computing technologies into practice.
