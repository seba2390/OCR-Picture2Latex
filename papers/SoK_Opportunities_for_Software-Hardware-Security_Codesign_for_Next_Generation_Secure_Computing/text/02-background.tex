\section{Codesign Methodologies}
\label{sec:background}

This section defines the notion of software-hardware-security codesign and shows it is a natural extension to the existing software-hardware codesign process.

\subsection{Software-Hardware Codesign}

Software-hardware codesign is a popular concept that originated in the 1990s as a way to co-optimize embedded systems software with the underlying hardware implementation~\cite{kalavade1993hardware}.
Software-hardware codesign \textit{simultaneously considers both hardware and software constraints to make better design decisions and improve the overall quality of results (i.e., performance, power, accuracy, etc.) of the system} (\autoref{fig:sw_hw_codesign}b).
The key distinction between standard \textit{hardware acceleration} (\autoref{fig:sw_hw_codesign}a) and \textit{software-hardware codesign} is an iterative feedback loop between software and hardware considerations.
Hardware acceleration is a one-way process where the application is not affected by hardware constraints; software-hardware codesign allows design consideration feedback in either direction between software and hardware.
This distinction is important because this feedback loop is what enables co-optimization of the software and hardware to ultimately yield a more efficient overall solution.

A canonical example of software-hardware codesign when building a hardware accelerator is fixed-point approximation (i.e., fixed-point optimization).
In fixed-point approximation, a designer may convert over-provisioned floating-point units to fixed-point arithmetic since hardware implementations of floating points require more power, area, and latency.
However, replacing floating-point with fixed-point units introduces some error to the original functionality.
For this substitution to be acceptable, the designer must revisit the application to ensure that the approximation does not adversely impact accuracy metrics.
In this case, the application drives the initial hardware accelerator design but power-efficiency considerations feeds information back to the application design.
The impact on accuracy and performance further informs whether the quantization approximation is acceptable; if not, the hardware is adjusted to provide better fidelity and the process iterates.

\subsection{Software-Hardware-Security Codesign}

It is natural to extend this idea of having a tightly coupled, synergistic codesign feedback loop to include security constraints.
We define the notion of software-hardware-security codesign as \emph{a design process which simultaneously considers software, hardware, and security design parameters when optimizing for power- and performance-efficient, high-fidelity, and threat-model-optimal execution solutions.} (\autoref{fig:sw_hw_codesign}c).
More precisely, we define \textit{software} to encompass both the software implementation (i.e., source code) as well as application design choices.
\textit{Hardware} includes architectural, microarchitectural, and silicon implementation decisions such as compute units, memory sizes, and communication overheads.
\textit{Security} refers to parameters associated with, and constraints imposed by the secure computing technology; for instance, these can manifest as encryption scheme parameters and instruction set restrictions.
We also include threat model considerations as part of the security specifications that each technology tries to mitigate against.

The concept of software-hardware-security codesign has been proposed in the past under different variations such as software-hardware codesign for instances of secure computing technologies.
For example, software-hardware codesign to enhance security has been proposed for resource restricted embedded systems such as ~\cite{vrased} for remote attestation and for implementing minimal roots of trust~\cite{eldefrawy2012smart}.
Power, energy, performance, and area constrained embedded systems must now also consider and satisfy security targets.
Our work seeks to highlight this repeating theme and shed light on the fact that these codesign opportunities should be seriously considered for maturing secure computing technologies.

We organize design consideration insights into a dependency graph to understand the codesign interactions between hardware, software, and security (\autoref{fig:sw_hw_codesign}c).
This serves as our basic blueprint to systemize the codesign considerations between hardware, software, and security.
Each node in the dependency graph represents a collection of design parameters associated with hardware, software, and security.
An edge from \textit{A} to \textit{B}, indicates that changing a design parameter in \textit{A} impacts a design consideration in \textit{B}.

Our goal is to establish a set of design consideration dependencies for each secure computing technology, and expose the feedback loops between the nodes and the opportunities for enabling iterative codesign.
A complete systemization for a secure computing technology will require many edges between nodes to capture every possible interaction.
Similar to how software-hardware codesign delicately balances power, performance, accuracy, area, and energy  costs to meet respective specification targets, security considerations must also be carefully balanced with other implementation costs.
We are \textit{not} proposing sacrificing security for the sake of prioritizing other quality metrics.
Rather we are noting that like power, performance, energy, and area that there is a similar balancing act that can trade-off the strength of the secure computing technology \textit{while still achieving a sufficient security guarantee} that defends against the target threat model.
For instance, protecting against \textit{all} possible threats is typically not feasible; security targets needs to be adjusted to a sufficient level of security to address practical attack scenarios.

Next, we look at instances of software-hardware-security codesign and explore historical precedence before systemizing modern technologies.
By exposing the role of software-hardware-codesign, we expect that the insight will provide guidance into the research directions required to drive future codesign opportunities.
