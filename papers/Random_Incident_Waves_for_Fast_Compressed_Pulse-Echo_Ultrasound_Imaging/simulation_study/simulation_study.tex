% Remember to use the PAST TENSE throughout - the work being reported is done, and was performed in the past, not the future.
%---------------------------------------------------------------------------------------------------------------
% 4.) results of the simulation study
%---------------------------------------------------------------------------------------------------------------
% a) numerical simulation of a pulse-echo scan configuration in the two-dimensional Euclidean space validates the emissions of single random incident waves
%The numerical simulation of ... validates
%The paper numerically simulates
%the pulse-echo measurement process in
%the two-dimensional Euclidean space, i.e. $d = 2$, emitting
%single realizations of
%the random waves by
%a linear transducer array to validate
%the proposed method for
%% 1.) wire phantom
%a wire phantom and
%% 2.) tissue-mimicking phantom
%a tissue-mimicking phantom.
%% b) former phantom permits a sparse representation of its spatial compressibility fluctuations in the canonical basis, whereas the latter phantom requires the Fourier basis
%The former phantom permits
%a sparse representation of
%its spatial compressibility fluctuations in
%% 1.) canonical basis
%the canonical basis, whereas
%the latter phantom requires
%% 2.) Fourier basis
%the \name{Fourier} basis.
%% c) random incident waves feature erratic spatial fluctuations in their associated incident acoustic pressure fields
%Unlike
%the \ac{QPW},
%the random incident waves feature
%erratic spatial fluctuations in
%their associated incident acoustic pressure fields.
%% d) erratic spatial fluctuations decorrelated the pulse echoes of the objects' admissible structural building blocks
%These decorrelate
%the pulse echoes of
%the objects' admissible structural building blocks and, consequently, facilitate
%their discrimination in
%the image recovery.

% d) significant effect of the four introduced types of incident waves on the observation process will be investigated
% TODO: seltsamer Satz, warum Verweis?
%The significant effect of
% 1.) four introduced types of incident waves
%the four introduced types of
%incident waves
%(cf. \cref{subsec:syn_p_in_types}) on
% 2.) observation process (all pulse-echo measurements, multifrequent, all transducer elements)
%the observation process
%\eqref{eqn:recovery_sys_lin_eq_v_rx_born_all_f_all_in_mat} and
%% 3.) pulse echoes of the admissible structural building blocks
%\TODO{building blocks of what?}
%the pulse echoes of
%the object's admissible structural building blocks will be investigated in
%\cref{sec:simulation_study}.

%---------------------------------------------------------------------------------------------------------------
% 1.) validation of the proposed method using synthetic RF voltage signals
%---------------------------------------------------------------------------------------------------------------
% a) proposed method was validated using synthetic RF voltage signals
The proposed method was validated using
synthetic \ac{RF} voltage signals.
% b) voltage signals were generated by numerical simulations of the pulse-echo measurement process probing two lossy heterogeneous objects
These were generated by
numerical simulations of
a typical pulse-echo measurement process probing
% 1.) two lossy heterogeneous objects
two lossy heterogeneous objects by
% 2.) each type of incident wave
each type of
incident wave.
% c) first object mimicked a typical wire phantom
The first object mimicked
a wire phantom, whereas %, i.e.
%an ensemble of
%isolated thin wires immersed in
%a lossy homogeneous fluid, whereas
% d) second object approximated the structure and the properties of human soft tissues
the second object approximated
the structure and
the properties of
human soft tissues.
% e) discretized compressibility fluctuations permitted sparse representations in the canonical and the Fourier bases
%Their compressibility fluctuations permitted
%sparse representations in
%the canonical and
%the \name{Fourier} bases. %, which defined
% f) bases defined the admissible structural building blocks as
%the admissible structural building blocks as
% 1.) individual samples
%individual samples and
% 2.) complex exponential functions of distinct spatial frequencies
%complex exponential functions of
%distinct spatial frequencies.
% g) additive errors of five distinct energy levels corrupted these synthetic RF voltage signals
Additive errors of
five distinct energy levels corrupted
these synthetic \ac{RF} voltage signals.

%---------------------------------------------------------------------------------------------------------------
% 2.) recovery experiments, coherence, incident acoustic pressure fields, column norms, adjoint normalized sensing matrices
%---------------------------------------------------------------------------------------------------------------
% a) two reference observation processes facilitated the assessment of the four distinct sensing matrices for each heterogeneous object
%Two reference observation processes, which
%did not originate from
%the proposed linear physical model for
%the pulse-echo measurement process, facilitated
%the assessment of
%the four distinct sensing matrices
%\eqref{eqn:recovery_reg_sensing_matrix} induced by
%% 1.) four types of incident waves
%the four types of
%incident waves for
%each heterogeneous object.
% e) reference sensing matrices served as benchmarks
% TODO: comparison ?
%The coherence of
%two fictitious reference sensing matrices, one of which satisfied
%the \ac{RIP} with
%very high probability
%(cf. \cref{sec:compressed_sensing}), served as
%benchmark and revealed
%the limitations imposed by
% 1.) proposed linear physical model for the pulse-echo measurement process
%the proposed linear physical model for
%the pulse-echo measurement process on
%the image recovery.
% f) inspections of the incident acoustic pressure fields and the l2-norms of the sensing matrices' column vectors
%Inspections of
% 1.) incident acoustic pressure fields associated with the four introduced types of incident waves
%(i)
%the incident acoustic pressure fields associated with
%the four introduced types of
%incident waves,
% 2.) Euclidean norms of the sensing matrices' column vectors
%(ii)
%the Euclidean norms of
%the sensing matrices' column vectors, and
% 3.) adjoint
%(iii)
%the left multiplications of
% 1.) unit vectors aggregating the relevant Fourier coefficients of the received RF voltage signals (all pulse-echo measurements, multifrequent, all transducer elements)
%the unit vectors aggregating
%the relevant \name{Fourier} coefficients of
%the received \ac{RF} voltage signals
%\eqref{eqn:recovery_reg_norm_obs_trans_coef_error} by
% 2.) adjoint normalized sensing matrices
%the adjoint normalized sensing matrices
%\eqref{eqn:recon_reg_norm_sensing_matrix} further characterized
%the benefits of
%the random incident waves.

%---------------------------------------------------------------------------------------------------------------
% 2.) structure of the section
%---------------------------------------------------------------------------------------------------------------
% a) section first details all simulation parameters
%This section first details
%all simulation parameters.
% b) section subsequently explains the performed calculations and analyses
%It subsequently explains
%the performed calculations and
%analyses.
% c) calculations and analyses were carried out for both objects unless indicated otherwise
%These were carried out for
%both objects unless
%indicated otherwise.

%%%%%%%%%%%%%%%%%%%%%%%%%%%%%%%%%%%%%%%%%%%%%%%%%%%%%%%%%%%%%%%%%%%%%%%%%%%%%%%%%%%%%%%%%%%%%%%%%%%%%%%%%%%%%%%%
% 1.) parameters
%%%%%%%%%%%%%%%%%%%%%%%%%%%%%%%%%%%%%%%%%%%%%%%%%%%%%%%%%%%%%%%%%%%%%%%%%%%%%%%%%%%%%%%%%%%%%%%%%%%%%%%%%%%%%%%%
\subsection{Parameters}
\label{subsec:sim_study_parameters}
%%%%%%%%%%%%%%%%%%%%%%%%%%%%%%%%%%%%%%%%%%%%%%%%%%%%%%%%%%%%%%%%%%%%%%%%%%%%%%%%%%%%%%%%%%%%%%%%%%%%%%%%%%%%%%%%
% table: values of all simulation parameters (two-dimensional Euclidean space)
%%%%%%%%%%%%%%%%%%%%%%%%%%%%%%%%%%%%%%%%%%%%%%%%%%%%%%%%%%%%%%%%%%%%%%%%%%%%%%%%%%%%%%%%%%%%%%%%%%%%%%%%%%%%%%%%
\begin{table*}[tb]
 \centering
 \caption{%
  Values of
  all simulation parameters for
  the two-dimensional Euclidean space, i.e.
  $d = 2$,
  $\bar{b} = b / ( 2 \pi )^{\zeta}$.
 }
 \renewcommand{\arraystretch}{1}
 \label{tab:sim_study_parameters}
 \small
 \begin{tabular}{%
  @{}%
  l%    01.) Geometric and electromechanical parameters of the instrumentation / Wire phantom
  @{\hspace{0.9em}}%
  l%    02.) Geometric and electromechanical parameters of the instrumentation / Wire phantom
  @{\hspace{0.9em}}%
  l%    03.) Pulse-echo parameters / Tissue-mimicking phantom
  @{\hspace{0.9em}}%
  l%    04.) Synthesis parameters / Tissue-mimicking phantom
  @{\hspace{0.9em}}%
  l%    05.) Geometric parameters of the discretizations / Regularization
  @{}%
 }
 \toprule
  \multicolumn{2}{@{}H}{a) Instrumentation (cf. \cref{tab:lin_mod_scan_config_instrum_params})} &
  \multicolumn{1}{H}{b) Pulse-echo measurements} &
  \multicolumn{1}{H}{c) Wave syntheses} &
  \multicolumn{1}{H@{}}{d) Discretization (cf. \cref{tab:recon_disc_params})}\\
  \cmidrule(r){1-2}\cmidrule(lr){3-3}\cmidrule(lr){4-4}\cmidrule(l){5-5}
 \addlinespace
 %--------------------------------------------------------------------------------------------------------------
  % 1.a.1) total number of identical array elements (geometric parameter of the linear transducer array)
  % 1.a.2) number of vibrating faces along the r_{1}-axis (geometric parameter of the linear transducer array)
  $N_{\text{el}} = N_{\text{el}, 1} = 128$ &
  % 1.b) width of the identical kerfs separating the vibrating faces along the r_{1}-axis (geometric parameter of the linear transducer array)
  $k_{\text{el}, 1} = 0$ & % real value: $k_{\text{el}, 1} = \SI{25}{\micro\meter}$
  % 1.c) number of sequential pulse-echo measurements / sampling rate (pulse-echo parameter)
  $N_{\text{in}} = 1$,
  $f_{\text{s}}^{(0)} = \SI{20}{\mega\hertz}$ & % T_{\text{s}}^{(0)} = \SI{50}{\nano\second}
  % 1.d) reference voltage signal identically exciting all array elements (synthesis parameter)
  $\tilde{u}^{(\text{tx}, 0)}( t ) = \hat{u} \sin( \omega_{\text{c}} t )$ & % $t \in [ 0; T_{\text{c}} ]$
  % 1.e.1) total number of grid points per vibrating face (spatial discretization)
  % 1.e.2) number of grid points per vibrating face along the r_{1}-axis (spatial discretization)
  $N_{\text{mat}} = N_{\text{mat}, 1} = 4$\\
 %--------------------------------------------------------------------------------------------------------------
  % 2.a.1) constant spacing between the centers of the adjacent faces along the r_{1}-axis (element pitch, geometric parameter of the linear transducer array)
  % 2.a.2) identical width of the vibrating faces along the r_{1}-axis (geometric parameter of the linear transducer array)
  $\Delta r_{\text{el}, 1} = w_{\text{el}, 1} = \SI{304.8}{\micro\meter}$ & % real value: $w_{\text{el}, 1} = \SI{279.8}{\micro\meter}$
  % 2.b) transmitter and receiver apodization functions (geometric parameter of the linear transducer array)
  $\chi_{m, l}^{(\text{tx})}( r_{1} ) = \chi_{m, l}^{(\text{rx})}( r_{1} ) = 1$ & % $r_{1} \in L_{m}$
  % 2.c) lower frequency bound (pulse-echo parameter)
  $f_{\text{lb}}^{(0)} = \SI{2.6}{\mega\hertz}$ &
  % 2.d) frequency of the clock signal in the quantization operator providing the admissible time delays (synthesis parameter)
  $f_{\text{clk}} = \SI{80}{\mega\hertz}$ & % \SI{12.5}{\nano\second}
  % 2.e) constant spacing between the adjacent grid points on each vibrating face along the r_{1}-axis (spatial discretization)
  $\Delta r_{\text{mat}, 1} = \SI{76.2}{\micro\meter}$\\
 %--------------------------------------------------------------------------------------------------------------
  % 3.a) center frequency (electromechanical parameter of the instrumentation)
  $f_{\text{c}} = \omega_{\text{c}} / ( 2 \pi ) = \SI{4}{\mega\hertz}$ &
  % 3.b) fractional bandwidth (electromechanical parameter of the instrumentation)
  $B_{h, \text{frac}} = 0.7$ &
  % 3.c) upper frequency bound (pulse-echo parameter)
  $f_{\text{ub}}^{(0)} = \SI{5.4}{\mega\hertz}$ &
  % 3.d) preferred direction of propagation (synthesis parameter)
  \acs{QPW}: $\uvect{\vartheta}^{(0)} = \uvect{2}$ &
  % 3.e) numbers of grid points in the FOV along both coordinate axes (spatial discretization)
  $N_{\text{lat}, 1} = N_{\text{lat}, 2} = \num{512}$\\
 %--------------------------------------------------------------------------------------------------------------
  % 4.a) receiver electromechanical transfer functions (electromechanical parameter of the instrumentation)
  $h_{m, l}^{(\text{rx})} = \SI{1}{ \volt \meter \per \newton }$ &
  % 4.b)
  &
  % 4.c) effective bandwidth of the recorded RF voltage signals (pulse-echo parameter)
  $B_{ u }^{(0)} = \SI{2.8}{\mega\hertz}$ &
  % 4.d) preferred direction of propagation (synthesis parameter)
  $\uvectcomp{ \vartheta }{ 1 }^{(0)} \approx \cos( \SI{77.6}{\degree} )$&
  % 4.e) constant spacings between the adjacent grid points in the FOV along both coordinate axes (spatial discretization)
  $\Delta r_{\text{lat}, 1} = \Delta r_{\text{lat}, 2} = \Delta r_{\text{mat}, 1}$\\
 %--------------------------------------------------------------------------------------------------------------
  % 5.a) cutoff time (electromechanical parameter of the instrumentation)
  $t_{\text{cut}} = 12 \ln( 10 ) / ( \omega_{\text{c}} B_{h, \text{frac}} )$ &
  % 5.b) parameter a in modulated Gaussian pulse (electromechanical parameter of the instrumentation)
  $a = 3 \ln( 10 ) / {t_{\text{cut}}}^{2}$ &
  % 5.c)  
  &
  % 5.d)
  &
  % 5.e) arbitrary offset of the grid points in the FOV (spatial discretization)
  $\vect{r}_{\text{lat}, 0} = \trans{ ( - 511, 1 ) } \Delta r_{\text{lat}, 1} / 2$\\
 %--------------------------------------------------------------------------------------------------------------
  % 6.a) combined transmitter electromechanical transfer functions (electromechanical parameter of the instrumentation)
  \multicolumn{2}{@{}l}{%
    $\tilde{h}_{m}^{(\text{c})}( t ) = e^{ -a ( t - t_{\text{cut}} )^{2} } \cos[ \omega_{\text{c}} ( t - t_{\text{cut}} ) ]$,
    $t \in [ 0; 2 t_{\text{cut}} ]$
  } &
  % 6.c)
  &
  % 6.d)
  &
  % 6.e) total number of grid points in the FOV (spatial discretization)
  $N_{\text{lat}} = \num{262144}$\\
 \addlinespace
  \multicolumn{2}{@{}H}{e) Wire phantom} &
  \multicolumn{2}{H}{f) Tissue-mimicking phantom} &
  \multicolumn{1}{H@{}}{g) Regularization}\\
  \cmidrule(r){1-2}\cmidrule(lr){3-4}\cmidrule(l){5-5}
 \addlinespace
 %--------------------------------------------------------------------------------------------------------------
  % 1.f) number of nonzero components (wire phantom)
  $\tnorm{ \vectsym{\gamma}^{(\kappa)} }{0} = 21$ &
  % 1.g) quantized bounds in the specified common observation time interval for the recorded RF voltage signals (wire phantom)
  $q_{\text{lb}}^{(0)} = \num{1}$, % \SI{50}{\nano\second}
  $q_{\text{ub}}^{(0)} = \num{1648}$ & % \SI{82.4}{\micro\second}
  % 1.h) number of nonzero components (tissue-mimicking phantom)
  $\tnorm{ \vectsym{\theta}^{(\kappa)} }{0} = 10$ &
  % 1.i) quantized bounds in the specified common observation time interval for the recorded RF voltage signals (tissue-mimicking phantom)
  $q_{\text{lb}}^{(0)} = \num{0}$,
  $q_{\text{ub}}^{(0)} = \num{1607}$ & % \SI{80.35}{\micro\second}
  % 1.j) admissible maximum number of iterations in SPGL1 (regularization)
  $N_{\text{iter}} = \num{1000}$\\
 %--------------------------------------------------------------------------------------------------------------
  % 2.f) canonical basis defined the admissible structural building blocks (wire phantom)
  $\mat{\Psi} = \mat{I}$ &
  % 2.g) quantized common observation time / number of quantized real-valued samples per recorded RF voltage signal (wire phantom)
  $T_{ \text{rec} }^{(0)} = 1647 T_{\text{s}}^{(0)}$ &
  % 2.h) discrete Fourier basis defined the admissible structural building blocks (tissue-mimicking phantom)
  $\mat{\Psi} = \mat{\Psi}_{\text{\acs{DFT}}}$ &
  % 2.i) quantized common observation time / number of quantized real-valued samples per recorded RF voltage signal (tissue-mimicking phantom)
  $T_{ \text{rec} }^{(0)} = 1607 T_{\text{s}}^{(0)}$ &
  % 2.j) norm parameters $q \in \{ 0.5; 1 \} were investigated (regularization)
  $q \in \{ 0.5; 1 \}$\\
 %--------------------------------------------------------------------------------------------------------------
  % 3.f) absorption parameters in the complex-valued wavenumber (wire phantom)
  $b = \SI{0.217}{ \deci\bel \mega\hertz\tothe{-2} \per \meter }$ & % $b = \SI{2.17e-3}{ \deci\bel \mega\hertz\tothe{-2} \per \centi\meter }$
  % 3.g) lower and upper bounds defining the set of admissible frequency indices (wire phantom)
  $l_{\text{lb}}^{(0)} = \num{215}$,
  $l_{\text{ub}}^{(0)} = \num{444}$ &
  % 3.h) absorption parameters in the complex-valued wavenumber (tissue-mimicking phantom)
  $b = \SI{0.5}{ \deci\bel \mega\hertz\tothe{-1} \per \centi\meter }$ &
  % 3.i) lower and upper bounds defining the set of admissible frequency indices (tissue-mimicking phantom)
  $l_{\text{lb}}^{(0)} = \num{209}$,
  $l_{\text{ub}}^{(0)} = \num{433}$ &
  % 3.j) number of realizations of the additive errors per admissible reference SNR (regularization)
  $N_{\text{rcn}} = 10$\\  
 %--------------------------------------------------------------------------------------------------------------
  % 4.f) reference angular frequency (wire phantom)
  $\omega_{\text{ref}} = \omega_{\text{c}}$ &
  % 4.g) number of relevant discrete frequencies (wire phantom)
  $N_{f, \text{BP}}^{(0)} = \num{230}$ &
  % 4.h) reference angular frequency (tissue-mimicking phantom)
  $\omega_{\text{ref}} = \omega_{\text{c}}$ &
  % 4.i) number of relevant discrete frequencies (tissue-mimicking phantom)
  $N_{f, \text{BP}}^{(0)} = \num{225}$ &
  % 4.j) normalization parameters generating a sequence of renormalized CS problems in Foucart's algorithm (regularization)
  $\epsilon_{n} = 1 / ( 2 + n )$, $n \in \setconsnonneg{ 4 }$\\
 %--------------------------------------------------------------------------------------------------------------
  % 5.f) reference phase velocity [exponent \zeta = \num{2} prevented dispersion and the phase velocity was constant] (wire phantom)
  $c_{\text{ref}} = \SI{1500}{\meter\per\second}$ &
  % 5.g) total number of observations in the observation process (wire phantom)
  $N_{\text{obs}} = \num{29440}$ &
  % 5.h) reference phase velocity (tissue-mimicking phantom)
  $c_{\text{ref}} = \SI{1540}{\meter\per\second}$ &  
  % 5.i) total number of observations in the observation process (tissue-mimicking phantom)
  $N_{\text{obs}} = \num{28800}$ &
  % 5.j) (regularization)
  \\
 \addlinespace
 \bottomrule
 \end{tabular}
\end{table*}

%%%%%%%%%%%%%%%%%%%%%%%%%%%%%%%%%%%%%%%%%%%%%%%%%%%%%%%%%%%%%%%%%%%%%%%%%%%%%%%%%%%%%%%%%%%%%%%%%%%%%%%%%%%%%%%%
% 1.) instrumentation
%%%%%%%%%%%%%%%%%%%%%%%%%%%%%%%%%%%%%%%%%%%%%%%%%%%%%%%%%%%%%%%%%%%%%%%%%%%%%%%%%%%%%%%%%%%%%%%%%%%%%%%%%%%%%%%%
\subsubsection{Instrumentation}
%\label{subsubsec:sim_study_params_scan_config}
%---------------------------------------------------------------------------------------------------------------
% 1.) geometric and electromechanical parameters of the instrumentation
%---------------------------------------------------------------------------------------------------------------
% a) commercial linear transducer array was emulated in the two-dimensional Euclidean space to reduce the computational costs
A commercial linear transducer array was emulated in
% 1.) two-dimensional Euclidean space
the two-dimensional Euclidean space
(cf. \cref{tab:sim_study_parameters}(a)).
% b) kerfs of width zero simplified the implementation
The kerfs of
width zero simplified
the implementation.
% c) products -j \omega_{l} \rho_{0} h_{m, l}^{(\text{tx})} corresponded to a modulated Gaussian pulse in the time domain
% thesis:Schiffner2018
% pulse echoes: e_{m, n}^{(\text{B})} =  h_{m}^{(\text{rx})} [ - j \omega \rho_{0} h_{n}^{(\text{tx})} ] u_{n}^{(\text{tx})}
% (physical unit: $\tunit{ e_{m, n}^{(\text{B})} } = \si{ \volt \meter\tothe{ - \mathit{d} } }$)
%
% article:Schiffner2018, Sect. V. Image Recovery Based on Compressed Sensing / Sect. V-B. Computations of the Incident Acoustic Pressure Fields
% - The insertions of
%   the apodized spatial transmit functions \eqref{eqn:syn_sup_qsw_p_in_qsw_spat_trans} and
%   the discretized transmitter apodization functions \eqref{eqn:recovery_disc_space_trans_array_spat_trans_tx} into
%   the incident acoustic pressure fields \eqref{eqn:syn_sup_qsw_p_in} yield the discretizations
%   [ p_{l}^{(\text{in}, n)}( \vect{r}_{\text{lat}, i} ) = - j 2 \omega_{l} \rho_{0} \Delta A \sum_{ m = 0 }^{ N_{\text{el}} - 1 } h_{m, l}^{(\text{tx})} u_{m, l}^{(\text{tx}, n)} \sum_{ \nu = 0 }^{ N_{\text{mat}} - 1 } \chi_{m, \nu, l}^{(\text{tx})} g_{l}\bigl[ \vect{r}_{\text{lat}, i} - \vect{r}_{\text{mat}, \nu}^{(m)} \bigr] ]
%   for all $( n, l, i ) \in \setconsnonneg{ N_{\text{in}} - 1 } \times \setsymbol{L}_{ \text{BP} }^{(n)} \times \setconsnonneg{ N_{\text{lat}} - 1 }$, where
%   [...].
%
% MATLAB:
% f_tx = 4 * 1e6;		% transmit center frequency (Hz)
% f_s = 20 * 1e6;		% sampling rate (Hz)
% frac_bw = 0.7;		% fractional bandwidth of incident pulse
% frac_bw_ref = -60;		% dB value that determines frac_bw
%
% tc = gauspuls( 'cutoff', f_tx, frac_bw, frac_bw_ref, -60 );	% calculate cutoff time
% t = (-tc:1/f_s:tc);						% time axis
% impulse_response = gauspuls( t, f_tx, frac_bw, frac_bw_ref );
%
% excitation = sin(2*pi*f_tx*(0:1/f_s:1/f_tx)) * 0.5e5 * f_s;
% N_samples_A_in_td = numel(excitation) + numel(impulse_response) - 1;
% impulse_response_dft = fft( impulse_response, N_samples_A_in_td );
% A_in_td_dft = impulse_response_dft .* fft( excitation, N_samples_A_in_td );
% A_in_td = ifft( A_in_td_dft ) / f_s;
%
% cs_2d_mlfma_inverse_scattering_v16.m:
% A_in = fft( A_in_td, N_samples_t ) / ( sqrt( N_samples_t ) * f_s );
% A_in_analy_cropped = 2 * A_in( indicator_omega );
%
% p_incident_ref( :, :, index_f ) = 0.25j * A_in_analy_cropped(index_f) * besselh( 0, 2, axis_k_tilde(index_f) * D_ref_tx );
%
% compare with: -j 2 \omega_{l} \rho_{0} \Delta A h_{m, l}^{(\text{tx})} u_{m, l}^{(\text{tx}, n)} \chi_{m, \nu, l}^{(\text{tx})} g_{l}\bigl[ \vect{r}_{\text{lat}, i} - \vect{r}_{\text{mat}, \nu}^{(m)} \bigr]
% => A_in_analy_cropped(index_f) = -j 2 \omega_{l} \rho_{0} \Delta A h_{m, l}^{(\text{tx})} u_{m, l}^{(\text{tx}, n)} \chi_{m, \nu, l}^{(\text{tx})}
%
% set: \chi_{m, \nu, l}^{(\text{tx})} = 1
% neglect: 2 \Delta A
% => A_in_analy_cropped(index_f) = -j \omega_{l} \rho_{0} h_{m, l}^{(\text{tx})} u_{m, l}^{(\text{tx}, n)}
\TODO{unit in table}
The products
% 1.) h_{m, l}^{(\text{c})} = -j \omega_{l} \rho_{0} h_{m, l}^{(\text{tx})}
$h_{m, l}^{(\text{c})} = -j \omega_{l} \rho_{0} h_{m, l}^{(\text{tx})}$ in
% 2.) discretized incident acoustic pressure fields [superpositions of quasi-(d-1)-spherical waves]
the incident acoustic pressure fields
\eqref{eqn:recovery_p_in} corresponded to
% 3.) modulated Gaussian pulse in the time domain
a modulated Gaussian pulse in
the time domain.


%%%%%%%%%%%%%%%%%%%%%%%%%%%%%%%%%%%%%%%%%%%%%%%%%%%%%%%%%%%%%%%%%%%%%%%%%%%%%%%%%%%%%%%%%%%%%%%%%%%%%%%%%%%%%%%%
% 2.) pulse-echo measurement process
%%%%%%%%%%%%%%%%%%%%%%%%%%%%%%%%%%%%%%%%%%%%%%%%%%%%%%%%%%%%%%%%%%%%%%%%%%%%%%%%%%%%%%%%%%%%%%%%%%%%%%%%%%%%%%%%
\subsubsection{Pulse-Echo Measurement Process}
%\label{subsubsec:sim_study_params_measurement_process}
%---------------------------------------------------------------------------------------------------------------
% 1.) pulse-echo measurement process
%---------------------------------------------------------------------------------------------------------------
% a) single pulse-echo measurement was simulated for each type of incident wave
% article:Schiffner2018, Sect. III. Linear Physical Model for the Pulse-Echo Measurement Process / Sect. III.A. Pulse-Echo Measurement Process
% - The \ac{UI} system SEQUENTIALLY PERFORMS $N_{\text{in}} \in \N$ INDEPENDENT PULSE-ECHO MEASUREMENTS using a planar transducer array
%   (cf. \cref{fig:lin_mod_scan_configuration,tab:lin_mod_scan_config_instrum_params}).
A single pulse-echo measurement was simulated for
% 1.) each type of incident wave
each type of
incident wave
(cf. \cref{tab:sim_study_parameters}(b)).
% b) distinct recording time intervals were specified for the RF voltage signals recorded from each object
% article:Schiffner2018, Sect. III. Linear Physical Model for the Pulse-Echo Measurement Process / Sect. III.A. Pulse-Echo Measurement Process
% - Each measurement begins at the time instant $t = 0$ and triggers
%   the CONCURRENT RECORDING OF THE \ac{RF} VOLTAGE SIGNALS $\tilde{u}_{m}^{(\text{rx}, n)}: \setsymbol{T}_{ \text{rec} }^{(n)} \mapsto \R$ GENERATED BY
%   ALL ARRAY ELEMENTS $m \in \setconsnonneg{ N_{\text{el}} - 1 }$ in the SPECIFIED TIME INTERVAL
%   [ \setsymbol{T}_{ \text{rec} }^{(n)} = \bigl[ t_{\text{lb}}^{(n)}; t_{\text{ub}}^{(n)} \bigr], ] (eqn:lin_mod_scan_config_volt_rx_obs_interval) where
%   $t_{\text{lb}}^{(n)} \in \Rnonneg$ and $t_{\text{ub}}^{(n)} > t_{\text{lb}}^{(n)}$ denote
%   its lower and upper bounds, respectively.
% 1.) specified recording time interval for the RF voltage signals generated by all array elements
The object-specific recording time interval
\eqref{eqn:lin_mod_scan_config_volt_rx_obs_interval} was identical for
% 1.) all types of incident waves
all types of
incident waves.
% c) lower and upper frequency bounds were derived from the modulated Gaussian pulse
The lower and
upper frequency bounds were derived from
the modulated Gaussian pulse.
% d) relevant Fourier coefficients were determined at the approximate efficiency of \text{Efficiency}^{(0)} \approx \SI{28}{\percent}
The relevant \name{Fourier} coefficients were determined at
% 1.) approximate efficiency of the regular sampling in combination with the subsequent computation of the DFTs
the approximate efficiency
\eqref{eqn:imp_fourier_coef_efficiency} of
$\text{Efficiency}^{(0)} \approx \SI{28}{\percent}$.


%%%%%%%%%%%%%%%%%%%%%%%%%%%%%%%%%%%%%%%%%%%%%%%%%%%%%%%%%%%%%%%%%%%%%%%%%%%%%%%%%%%%%%%%%%%%%%%%%%%%%%%%%%%%%%%%
% 3.) syntheses of the incident waves
%%%%%%%%%%%%%%%%%%%%%%%%%%%%%%%%%%%%%%%%%%%%%%%%%%%%%%%%%%%%%%%%%%%%%%%%%%%%%%%%%%%%%%%%%%%%%%%%%%%%%%%%%%%%%%%%
\subsubsection{Syntheses of the Incident Waves}
\label{subsubsec:sim_study_params_inc_waves}
%---------------------------------------------------------------------------------------------------------------
% 1.) syntheses of the incident waves
%---------------------------------------------------------------------------------------------------------------
% a) reference voltage signal in the excitation voltages corresponded to a single period of a sinusoid
% article:Schiffner2018, Sect. IV. Syntheses of the Incident Waves / Sect. IV-B Types of Incident Waves (subsec:syn_p_in_types)
% - The generation of the excitation voltages typically applies quantized apodization weights $a_{m}^{(n)} \in \R$ and time delays $\Delta t_{m}^{(n)} \in \Rnonneg$ to
%   the REFERENCE VOLTAGE SIGNALS $u_{l}^{(\text{tx}, n)} \in \C$, whose
%   electric energies are constant for all $n \in \setconsnonneg{ N_{\text{in}} - 1 }$.
% - [...] the generated excitation voltages are
%   [ u_{m, l}^{(\text{tx}, n)} = u_{l}^{(\text{tx}, n)} a_{m}^{(n)} e^{ -j \omega_{l} \mathcal{Q} \left[ \Delta t_{m}^{(n)} \right] } ] (eqn:syn_p_in_types_v_tx_expression)
%   for all $( n, l, m ) \in \setconsnonneg{ N_{\text{in}} - 1 } \times \setsymbol{L}_{ \text{BP} }^{(n)} \times \setconsnonneg{ N_{\text{el}} - 1 }$ [...].
%
% MATLAB:
% excitation = sin( 2 * pi * f_tx * (0:1/f_s:1/f_tx) ) * 0.5e5 * f_s;
% N_samples_A_in_td = numel(excitation) + numel(impulse_response) - 1;
% impulse_response_dft = fft(impulse_response, N_samples_A_in_td);
% A_in_td_dft = impulse_response_dft .* fft(excitation, N_samples_A_in_td);
% A_in_td = ifft(A_in_td_dft) / f_s;
The reference voltage signal in
% 1.) excitation voltages (single pulse-echo measurement, monofrequent, single array element)
the excitation voltages
\eqref{eqn:syn_p_in_types_v_tx} corresponded to
% 2.) single period of a sinusoid
a single period of
a sinusoid, whose amplitude was
% 3.) irrelevant
irrelevant in
% 4.) LTI measurement process
the \ac{LTI} measurement process
(cf. \cref{tab:sim_study_parameters}(c)).
% b) frequency of the clock signal in the quantization operator matched the specifications of a commercial UI system
% article:ChengITUFFC2006: Extended high-frame rate imaging method with limited-diffraction beams
% V. In Vitro and In Vivo Experiments
% - For steered plane wave or conventional delay-and-sum methods,
%   linear time delays are applied to the transducers to steer the beams. (p. 890)
% - The precision of the time delay of the system is 6.25 ns, which is determined by a 160 MHz clock. (p. 890)
The frequency of
% 1.) clock signal
the clock signal in
% 2.) quantization operator providing the admissible time delays
the quantization operator
\eqref{eqn:syn_p_in_types_v_tx_quantization} matched
% 3.) specifications of commercial UI systems
the specifications of
commercial \ac{UI} systems.
% c) steered QPW propagated preferentially along the r_{2}-axis
The steered \ac{QPW} propagated preferentially along
the $r_{2}$-axis.
% d) fixed time period permuting the time delays for a steered QPW
% article:Schiffner2018, Sect. IV-B.3) Superpositions of Randomly-Delayed Quasi-(d-1)-Spherical Waves (subsubsec:syn_p_in_types_rnd_del)
% - In the two-dimensional Euclidean space, i.e. $d = 2$, they simplify to
%   $T_{\text{inc}}^{(n)} = \Delta r_{\text{el}, 1} \tabs{ \uvectcomp{ \vartheta }{ 1 }^{(n)} } / c_{\text{ref}}$ and induce
%   RANDOM PERMUTATIONS OF THE TIME DELAYS SPECIFIED FOR THE STEERED \acp{QPW} in \eqref{eqn:syn_p_in_types_v_tx_qpw}.
%
% => \tabs{ \uvectcomp{ \vartheta }{ 1 }^{(0)} } = T_{\text{inc}}^{(0)} c_{\text{ref}} / \Delta r_{\text{el}, 1}
% => \cos( \vartheta_{0} ) = T_{\text{inc}}^{(0)} c_{\text{ref}} / \Delta r_{\text{el}, 1} [ \cos( \vartheta_{0} ) >= 0 ]
% => \vartheta_{0} = \acos( T_{\text{inc}}^{(0)} c_{\text{ref}} / \Delta r_{\text{el}, 1} )
%
% wire phantom:
% c_{\text{ref}} = \SI{1500}{\meter\per\second}, T_{\text{inc}}^{(0)} = \SI{43.6369}{\nano\second}
% => \vartheta_{0} \approx 77.5992°
% tissue-mimicking phantom:
% c_{\text{ref}} = \SI{1540}{\meter\per\second}, T_{\text{inc}}^{(0)} = \SI{42.5035}{\nano\second}
% => \vartheta_{0} \approx 77.5992°
The fixed time period
\eqref{eqn:syn_p_in_types_v_tx_rnd_del_interval} emerged from
% 1.) preferred direction of propagation
the direction
$\uvect{\vartheta}^{(0)} = \trans{ ( \cos( \vartheta ), \sin( \vartheta ) ) }$.


%%%%%%%%%%%%%%%%%%%%%%%%%%%%%%%%%%%%%%%%%%%%%%%%%%%%%%%%%%%%%%%%%%%%%%%%%%%%%%%%%%%%%%%%%%%%%%%%%%%%%%%%%%%%%%%%
% 4.) spatial discretizations
%%%%%%%%%%%%%%%%%%%%%%%%%%%%%%%%%%%%%%%%%%%%%%%%%%%%%%%%%%%%%%%%%%%%%%%%%%%%%%%%%%%%%%%%%%%%%%%%%%%%%%%%%%%%%%%%
\subsubsection{Spatial Discretizations}
\label{subsubsec:sim_study_params_disc_space}
%---------------------------------------------------------------------------------------------------------------
% 1.) spatial discretizations
%---------------------------------------------------------------------------------------------------------------
% a) constant spacing between the adjacent grid points on each vibrating face along the r_{1}-axis ensured approximately \num{3.7} points per smallest wavelength
% independent parameters required for the coordinates of the grid points on each vibrating face ( d = 2, \delta = 1 ):
% 1.) N_{\text{mat}, 1} \in \N
%     [number of grid points per vibrating face along the r_{1}-axis]
% => \Delta r_{\text{mat}, 1} = w_{\text{el}, 1} / N_{\text{mat}, 1}
%    [constant spacing between the adjacent grid points on each vibrating face along the r_{1}-axis]
% => \mathcal{V}_{m} = \{ \vect{r}_{\text{mat}, \nu}^{(m)} \in \R^{2}: \vect{r}_{\text{mat}, \nu}^{(m)} = \vect{r}_{\text{el}, m} + ( \nu_{1} - M_{\text{mat}, 1} ) \Delta r_{\text{mat}, 1} \uvect{1}, \nu_{1} \in \setconsnonneg{ N_{\text{mat}, 1} - 1 }, \nu = \mathcal{I}( \vectsym{\nu}, \vect{N}_{\text{mat}} ) \}
%    [coordinates of the grid points on each vibrating face]
% => N_{\text{mat}} = \tabs{ \mathcal{V}_{m} } = N_{\text{mat}, 1}
%    [total number of grid points per vibrating face]
% => \Delta A = \Delta r_{\text{mat}, 1}
%    [(d-1)-dimensional surface element]
%
% number of grid points per vibrating face along the r_{1}-axis and its relationship to the wavelength:
% linear transducer array:
%   w_{\text{el}, 1} = 304.8 um | N_{\text{mat}, 1} = 4
% => \Delta r_{\text{mat}, 1} = w_{\text{el}, 1} / N_{\text{mat}, 1} = 304.8 um / 4 = 76.2 um
% wire phantom:
%   c_{l} = c_{\text{ref}} = 1500 m / s; f_{l} = 5.3916211293 MHz for l = l_{\text{ub}}^{(0)} = 444
%   => \lambda_{l} = c_{l} / f_{l} = 278.2095 um
%   => \lambda_{l} / 4 = 69.5524 um
% => \lambda_{l} / \Delta r_{\text{mat}, 1} = 3.6510
% tissue-mimicking phantom:
%   c_{l} = 1540.4124 m / s; f_{l} = 5.3889234599 MHz for l = l_{\text{ub}}^{(0)} = 433
%   => \lambda_{l} = c_{l} / f_{l} = 285.8479 um
%   => \lambda_{l} / 4 = 71.4620 um
% => \lambda_{l} / \Delta r_{\text{mat}, 1} = 3.7513
The constant spacing between
% 1.) adjacent grid points
the adjacent grid points on
% 2.) each vibrating face
each vibrating face along
% 3.) r_{1}-axis
the $r_{1}$-axis ensured
% 4.) approximately four grid points
approximately \num{3.7} points per
% 5.) smallest wavelength
% article:Schiffner2018, Sect. III. Linear Physical Model for the Pulse-Echo Measurement Process / Sect. B. Acoustic Model for Human Soft Tissues
% - Given reference values of
%   the angular frequency $\omega_{\text{ref}} \in \Rplus$ and
%   the associated phase velocity $c_{\text{ref}} \in \Rplus$, the complex-valued wavenumber \cite{article:WatersITUFFC2005,article:SzaboJASA1995}
%   [ \munderbar{k}_{l} = \frac{ \omega_{l} }{ c_{\text{ref}} } + \beta_{\text{E,ref}, l} - j \bar{b} \abs{ \omega_{l} }^{ \zeta } ]
%   [                     \beta_{l} = \omega_{l} / c_{l} ] where
%   the phase term $\beta_{l} \in \R$ sums the real-valued wavenumber $k_{\text{ref}, l} = \omega_{l} / c_{\text{ref}}$ and
%   the excess dispersion term [...] combines power-law absorption with dispersion.
smallest wavelength
(cf. \cref{tab:sim_study_parameters}(d)).
% b) FOV was square shaped and laterally centered in front of the linear transducer array
% independent parameters required for the coordinates of the grid points in the FOV ( d = 2, \delta \in \{ 1, 2 \} ):
% 1.) N_{\text{lat}, 1}, N_{\text{lat}, 2} \in \N
%     [number of grid points in the FOV along the r_{1}- and r_{2}-axes]
% 2.) \Delta r_{\text{lat}, 1}, \Delta r_{\text{lat}, 2} \in \Rplus
%     [constant spacing between the adjacent grid points in the FOV along the r_{1}- and r_{2}-axes]
% 3.) \vect{r}_{\text{lat}, 0} = \trans{ ( r_{\text{lat}, 0, 1}, r_{\text{lat}, 0, 2} ) } \in \R \times \Rplus
%     [arbitrary offset of the grid points in the FOV]
% => \mathcal{L} = \{ \vect{r}_{\text{lat}, i} \in \R^{2}: \vect{r}_{\text{lat}, i} = \vect{r}_{\text{lat}, 0} + \sum_{ \delta = 1 }^{ 2 } i_{\delta} \Delta r_{\text{lat}, \delta} \uvect{\delta}, i_{\delta} \in \setconsnonneg{ N_{\text{lat}, \delta} - 1 }, i = \mathcal{I}\left( \vect{i}, \vect{N}_{\text{lat}} \right) \}
%    [coordinates of the grid points in the FOV]
% => N_{\text{lat}} = \abs{ \mathcal{L} } = \prod_{ \delta = 1 }^{ 2 } N_{\text{lat}, \delta}
%    [total number of grid points in the FOV]
% => \Delta V = \prod_{ \delta = 1 }^{ 2 } \Delta r_{\text{lat}, \delta}
%    [d-dimensional volume element]
The \ac{FOV} was
% 1.) square shaped
square shaped and
% 2.) laterally centered
laterally centered in front of
% 3.) linear transducer array
the linear transducer array.
% c) symmetry enables simple computation of the incident field
The identical spacings between
the adjacent grid points on
each vibrating face and
in the \ac{FOV}, i.e.
$\Delta r_{\text{lat}, 1} = \Delta r_{\text{lat}, 2} = \Delta r_{\text{mat}, 1}$, simplified
the computations of
% 1.) discretized incident acoustic pressure fields [superpositions of quasi-(d-1)-spherical waves]
the incident acoustic pressure fields
\eqref{eqn:recovery_p_in} and
% 2.) implementation of the FMM
the implementation of
the \ac{FMM}.


%%%%%%%%%%%%%%%%%%%%%%%%%%%%%%%%%%%%%%%%%%%%%%%%%%%%%%%%%%%%%%%%%%%%%%%%%%%%%%%%%%%%%%%%%%%%%%%%%%%%%%%%%%%%%%%%
% 5.) wire phantom
%%%%%%%%%%%%%%%%%%%%%%%%%%%%%%%%%%%%%%%%%%%%%%%%%%%%%%%%%%%%%%%%%%%%%%%%%%%%%%%%%%%%%%%%%%%%%%%%%%%%%%%%%%%%%%%%
\subsubsection{Wire Phantom}
\label{subsubsec:sim_study_params_obj_A}
%---------------------------------------------------------------------------------------------------------------
% 1.) wire phantom
%---------------------------------------------------------------------------------------------------------------
% a) wires were represented by 21 identical nonzero components in the vector aggregating the regular samples in the discretized relative spatial fluctuations in the unperturbed compressibility
The wires were represented by
identical nonzero components in
% 1.) vector stacking the regular samples in the discretized relative spatial fluctuations in the unperturbed compressibility
the compressibility fluctuations
\eqref{eqn:recovery_sys_lin_eq_gamma_kappa_bp_vector}
(cf. \cref{tab:sim_study_parameters}(e)).
\TODO{Why does the simulation recover real vectors?!?}
% b) axial distances from the linear transducer array / axial and lateral spacings
Their axial distances from
% 1.) linear transducer array
the linear transducer array ranged from
% 2.) 5 - 37 mm
\SIrange{5}{37}{\milli\meter}, and
% 3.) axial and lateral spacings
their axial and
lateral spacings amounted to
% 4.) 5 mm and 10 mm
approximately \SI{5}{\milli\meter} and
\SI{10}{\milli\meter},
respectively.
% c) canonical basis defined the admissible structural building blocks as individual samples and induced a 21-sparse representation
The canonical basis defined
% 1.) structural building block
the structural building block with
% 2.) index n \in \setcons{ N_{\text{lat}} }
the index
$n \in \setcons{ N_{\text{lat}} }$ as
% 3.) individual sample located at the position \vect{r}_{ \text{lat}, n - 1 }
% index_1 = \ceil{ n / N_{\text{lat}, 2} }
% index_2 = n - ( index_1 - 1 ) * N_{\text{lat}, 2}
% r_{ \text{lat}, n, 1 } = ( index_1 - 513 / 2 ) * \Delta r_{\text{lat}, 1}
% r_{ \text{lat}, n, 2 } = ( index_2 - 0.5 ) * \Delta r_{\text{lat}, 1}
the individual sample located at
the position
$\vect{r}_{ \text{lat}, n - 1 } \in \mathcal{L}$ and induced
% 4.) nearly-sparse representation
a sparse representation
\eqref{eqn:recovery_reg_sparse_representation}.
% d) absorption parameters equaled those of pure water at a temperature of 20 °C
% book:Duck1990, Chapter 4: Acoustic Properties of Tissue at Ultrasonic Frequencies / Sect. 4.1.11: Acoustic velocity through some materials other than tissue
% Sect. 4.1.11.1: Water
% - The acoustic velocity in water is given in Table 4.8, including its temperature dependence. (p. 94)
% - Pure water non-dispersive. (p. 95)
% - Table 4.8: Acoustic velocity and attenuation, and non-linearity parameter B/A for pure water at atmospheric pressure (p. 95)
%   20°C | 1482.3 m/s | 25 * 1e-3 Np / ( m MHz^{2} ) | 4.96 B/A
The absorption parameters in
% 1.) complex-valued wavenumber with respect to k_{\text{ref}}
the wavenumber
\eqref{eqn:lin_mod_mech_model_tis_abs_time_causal_wavenumber_complex_kref} equaled
those of
% 2.) pure water at a temperature of 20 °C
pure water at
a temperature of
$\SI{20}{\celsius}$
\cite[Table 4.8]{book:Duck1990}, where
% 3.) quadratic frequency dependence
the quadratic frequency dependence prevented
% 4.) dispersion
dispersion.
% e) quantized recording time interval and the associated number of relevant discrete frequencies resulted in the number of observations of N_{\text{obs}} / N_{\text{lat}} \approx \SI{11.23}{\percent}
% T_{ \text{rec} }^{(0)} = 1647 T_{\text{s}}^{(0)}
% N_{f, \text{BP}}^{(0)} = 230
% N_{\text{lat}} = 262144
% => N_{\text{obs}} = 29440
% => N_{\text{obs}} / N_{\text{lat}} = 29440 / 262144 \approx 11.2305 %
The quantized recording time interval
\eqref{eqn:lin_mod_scan_config_volt_rx_obs_interval} and
% 1.) number of relevant discrete frequencies (effective time-bandwidth products)
the associated number of
relevant discrete frequencies
\eqref{eqn:recon_disc_axis_f_discrete_BP_TB_product} resulted in
% 2.) number of observations (all pulse-echo measurements, multifrequent, all array elements)
%the number of
%observations
%\eqref{eqn:recovery_sys_lin_eq_num_obs} of
the ratio
% indeterminancy
$N_{\text{obs}} / N_{\text{lat}} \approx \SI{11.23}{\percent}$.


%%%%%%%%%%%%%%%%%%%%%%%%%%%%%%%%%%%%%%%%%%%%%%%%%%%%%%%%%%%%%%%%%%%%%%%%%%%%%%%%%%%%%%%%%%%%%%%%%%%%%%%%%%%%%%%%
% 6.) tissue-mimicking phantom
%%%%%%%%%%%%%%%%%%%%%%%%%%%%%%%%%%%%%%%%%%%%%%%%%%%%%%%%%%%%%%%%%%%%%%%%%%%%%%%%%%%%%%%%%%%%%%%%%%%%%%%%%%%%%%%%
\subsubsection{Tissue-Mimicking Phantom}
\label{subsubsec:sim_study_params_obj_B}
%---------------------------------------------------------------------------------------------------------------
% 1.) tissue-mimicking phantom
%---------------------------------------------------------------------------------------------------------------
% a) discrete Fourier basis defined the structural building blocks as complex exponential functions
The discrete \name{Fourier} basis defined
% 1.) structural building block
the structural building block with
% 2.) index n \in \setcons{ N_{\text{lat}} }
the index
$n \in \setcons{ N_{\text{lat}} }$ as
% 3.) complex exponential function with the normalized discrete lateral and axial frequencies
% index_1 = \ceil{ n / N_{\text{lat}, 2} }
% index_2 = n - ( index_1 - 1 ) * N_{\text{lat}, 2}
% \hat{K}_{ n, 1 } = ( \ceil{ N_{\text{lat}, 1} / 2 } - N_{\text{lat}, 1} + index_1 - 1 ) / N_{\text{lat}, 1} = ( \tceil{ n / 512 } - 257 ) / 512
% \hat{K}_{ n, 2 } = ( index_2 - 1 ) / N_{\text{lat}, 2} = ( n - ( index_1 - 1 ) * N_{\text{lat}, 2} - 1 ) / N_{\text{lat}, 2} = ( n + 511 ) / 512 - \tceil{ n / 512 }
the complex exponential function with
the normalized discrete lateral and
axial frequencies
$\hat{K}_{ n, 1 } = ( \tceil{ n / 512 } - 257 ) / 512$ and
$\hat{K}_{ n, 2 } = ( n + 511 ) / 512 - \tceil{ n / 512 }$,
respectively
(cf. \cref{tab:sim_study_parameters}(f)).
% b) random specification of 10 coefficients with respect to the discrete Fourier basis generated the relative spatial fluctuations in the unperturbed compressibility
% MATLAB:
% theta_kappa_abs_mu = 1e-1;
% theta_kappa = zeros( N_lattice_axis(2), N_lattice_axis(1) );
% theta_kappa( indices_k_kappa ) = theta_kappa_abs_mu .* exp( 2j * pi * rand( 1, N_coefficients_kappa ) );
% gamma_kappa = reshape( psi_fourier( N_lattice_axis, theta_kappa, 2, [] ), [N_lattice_axis(2), N_lattice_axis(1)] );
% gamma_kappa_abs_max = max( abs( gamma_kappa(:) ) );
% theta_kappa = theta_kappa / gamma_kappa_abs_max * 1e-1;
% gamma_kappa = gamma_kappa / gamma_kappa_abs_max * 1e-1;
Nonzero components of
% 1.) identical absolute value
identical absolute value and
% 2.) uniformly distributed phase
uniformly distributed phase in
% 3.) nearly-sparse representation
the sparse representation
\eqref{eqn:recovery_reg_sparse_representation} spawned
% 4.) vector stacking the regular samples in the discretized relative spatial fluctuations in the unperturbed compressibility
dense compressibility fluctuations
\eqref{eqn:recovery_sys_lin_eq_gamma_kappa_bp_vector}.
% c) typical absorption parameters for soft tissues governed the wavenumber
% article:JensenProgBMB2007: Medical ultrasound imaging
% 2. Basic ultrasound
% - Typically, an ATTENUATION OF 0.5 dB/(MHz cm) IS EXPERIENCED IN THE SOFT TISSUES. (p. 154)
% book:Duck1990, Chapter 4: Acoustic Properties of Tissue at Ultrasonic Frequencies / Sect. 4.3: Ultrasonic attenuation: absorption and scatter
% Sect. 4.3.8: Values of acoustic absorption coefficients in tissue
% - Measured values of ABSORPTION COEFFICIENTS FOR ULTRASOUND IN SOFT TISSUE are given in Tables 4.19 and 4.20. (p. 115)
% - Values at particular frequencies are included in Table 4.19, and
%   the POWER-LAW EXPRESSION Equation 4.30 used as the basis for the values given in Table 4.20. (p. 115)
% - Table 4.20: Ultrasound absorption coefficient (ii); \alpha = a f^{b} (p. 117)
Typical absorption parameters for
% 1.) soft tissues
soft tissues
\cite[Table 4.20]{book:Duck1990} governed
% 2.) complex-valued wavenumber with respect to k_{\text{ref}}
the wavenumber
\eqref{eqn:lin_mod_mech_model_tis_abs_time_causal_wavenumber_complex_kref}, where
% 3.) linear frequency dependence
the linear frequency dependence implied
% 4.) anomalous dispersion
anomalous dispersion.
% d) quantized recording time interval and the associated number of relevant discrete frequencies resulted in the ratio N_{\text{obs}} / N_{\text{lat}} \approx \SI{10.99}{\percent}
% T_{ \text{rec} }^{(0)} = 1607 T_{\text{s}}^{(0)}
% N_{f, \text{BP}}^{(0)} = 225
% N_{\text{lat}} = 262144
% => N_{\text{obs}} = 28800
% => N_{\text{obs}} / N_{\text{lat}} = 28800 / 262144 \approx 10.9863 %
The quantized recording time interval
\eqref{eqn:lin_mod_scan_config_volt_rx_obs_interval} and
% 1.) number of relevant discrete frequencies (effective time-bandwidth product)
the associated number of
relevant discrete frequencies
\eqref{eqn:recon_disc_axis_f_discrete_BP_TB_product} resulted in
% 2.) number of observations (all pulse-echo measurements, multifrequent, all array elements)
%the number of
%observations
%\eqref{eqn:recovery_sys_lin_eq_num_obs} of
the ratio
$N_{\text{obs}} / N_{\text{lat}} \approx \SI{10.99}{\percent}$.


%%%%%%%%%%%%%%%%%%%%%%%%%%%%%%%%%%%%%%%%%%%%%%%%%%%%%%%%%%%%%%%%%%%%%%%%%%%%%%%%%%%%%%%%%%%%%%%%%%%%%%%%%%%%%%%%
% 7.) additive errors
%%%%%%%%%%%%%%%%%%%%%%%%%%%%%%%%%%%%%%%%%%%%%%%%%%%%%%%%%%%%%%%%%%%%%%%%%%%%%%%%%%%%%%%%%%%%%%%%%%%%%%%%%%%%%%%%
\subsubsection{Additive Errors}
\label{subsubsec:sim_study_params_obs_errors}
%---------------------------------------------------------------------------------------------------------------
% 1.) additive errors
%---------------------------------------------------------------------------------------------------------------
% a) five variances specified additive errors of distinct energy levels
% article:Schiffner2018, Sect. VI. Implementation / Sect. VI-B. Additive Errors (subsec:imp_obs_errors)
% - ADDITIVE ERRORS, which were statistically modeled as
%   \ac{GWN} (cf. e.g. \cite[110]{book:Manolakis2005}) WITH ZERO MEAN AND THE VARIANCE $\sigma_{\eta}^{2}$, CORRUPTED
%   THE RECORDED SAMPLES OF ALL \ac{RF} VOLTAGE SIGNALS.
The five variances
\begin{equation}
 %--------------------------------------------------------------------------------------------------------------
 % variances of the zero-mean GWN
 %--------------------------------------------------------------------------------------------------------------
  \sigma_{\eta}^{2}
  =
  \frac{
    2
    \norm{ \vect{u}^{(\text{B}, \text{\acs{QPW}})} }{2}^{2}
  }{
    N_{\text{el}}
  }
  10^{ -\frac{ \text{SNR}_{\text{dB}} }{ 10 \si{\deci\bel} } },
 \label{eqn:sim_study_params_obs_errors_variance}
\end{equation}
where
% 1.) energy of the Born approximation induced by the QPW
% time-domain energy: 2 N_{t}^{(0)} \tnorm{ \vect{u}^{(\text{B}, \text{\acs{QPW}})} }{2}^{2}
% time-domain power: 2 \tnorm{ \vect{u}^{(\text{B}, \text{\acs{QPW}})} }{2}^{2} / N_{\text{el}}
$\tnorm{ \vect{u}^{(\text{B}, \text{\acs{QPW}})} }{2}^{2}$ equals
the energy of
% 2.) approximate vector stacking the relevant Fourier coefficients of the recorded RF voltage signals (all pulse-echo measurements, multifrequent, all array elements)
the \name{Born} approximation of
the recorded \ac{RF} voltage signals
\eqref{eqn:recovery_sys_lin_eq_v_rx_born_all_f_all_in_v_rx_born} induced by
% 3.) quasi-plane wave (QPW)
the \ac{QPW}, and
% 4.) reference signal-to-noise ratio (SNR) in \deci\bel
$\text{SNR}_{\text{dB}} \in \{ \SI{3}{\deci\bel}, \SI{6}{\deci\bel}, \SI{10}{\deci\bel}, \SI{20}{\deci\bel}, \SI{30}{\deci\bel} \}$ is
the reference \ac{SNR}, specified
% 5.) additive errors
additive errors of
% 6.) five distinct energy levels
distinct energy levels.


%%%%%%%%%%%%%%%%%%%%%%%%%%%%%%%%%%%%%%%%%%%%%%%%%%%%%%%%%%%%%%%%%%%%%%%%%%%%%%%%%%%%%%%%%%%%%%%%%%%%%%%%%%%%%%%%
% 8.) regularization
%%%%%%%%%%%%%%%%%%%%%%%%%%%%%%%%%%%%%%%%%%%%%%%%%%%%%%%%%%%%%%%%%%%%%%%%%%%%%%%%%%%%%%%%%%%%%%%%%%%%%%%%%%%%%%%%
\subsubsection{Regularization}
\label{subsubsec:sim_study_params_regularization}
%---------------------------------------------------------------------------------------------------------------
% 1.) regularization
%---------------------------------------------------------------------------------------------------------------
% a) approximation of the estimated l2-norm of the normalized additive errors in the normalized CS problem
% 1.) variances of the zero-mean GWN
The variances
\eqref{eqn:sim_study_params_obs_errors_variance},
% 2.) number of relevant discrete frequencies (effective time-bandwidth product)
the effective time-bandwidth product
\eqref{eqn:recon_disc_axis_f_discrete_BP_TB_product}, and
% 3.) quantized recording time
the quantized recording time permitted
the approximation of
% 4.) estimated l2-norm of the normalized additive errors in the normalized CS problem
the estimated $\ell_{2}$-norm of
the normalized additive errors
\eqref{eqn:imp_data_acq_rel_obs_error_est} as
\begin{equation*}
 %--------------------------------------------------------------------------------------------------------------
 % approximation of the estimated l2-norm of the normalized additive errors in the normalized CS problem
 %--------------------------------------------------------------------------------------------------------------
  \hat{ \bar{\eta} }
  \approx
  \left[
    1
    +
    \frac{
      \norm{ \vect{u}^{(\text{B})} }{2}^{2}
      f_{\text{s}}^{(0)}
    }{
      \norm{ \vect{u}^{(\text{B}, \text{\acs{QPW}})} }{2}^{2}
      2 B_{ u }^{(0)}
    }
    10^{ \frac{ \text{SNR}_{\text{dB}} }{ 10 \si{\deci\bel} } }
  \right]^{ - \frac{1}{2} }.
\end{equation*}
% b) empirical threshold factors for the normalization of the sensing matrices
% 1.) QPW
% SNR_cs = [3, 6, 10, 20, 30, inf];		% specify SNR in dB
% norms_cols_thresh = 10.^(-SNR_cs / 20);	% thresholds for normalization according to actual SNR
% 2.) random incident waves
% SNR_cs_act = 10 * log10( data_RF_tgc_cs_power_mean ./ noise_RF_tgc_cs_variance );
% => SNR_cs_act = 10 * log10( norm( data_RF_tgc_cs(:) )^2 ./ norm( data_RF_tgc_qpw(:) )^2 ) + SNR_cs;
% norms_cols_thresh = 10.^(-SNR_cs_act / 20);	% thresholds for normalization according to actual SNR
% => norms_cols_thresh = norm( data_RF_tgc_qpw(:) ) ./ norm( data_RF_tgc_cs(:) ) * 10.^( - SNR_cs / 20 );
\TODO{exception: equal threshold for 3 and 6 dB}
For
each reference \ac{SNR},
% 1.) empirical threshold factors for the normalization of the sensing matrices
the empirical factors
\begin{equation}
 %--------------------------------------------------------------------------------------------------------------
 % empirical threshold factors for the normalization of the sensing matrices
 %--------------------------------------------------------------------------------------------------------------
  \xi
  =
  \frac{
    \norm{ \vect{u}^{(\text{B}, \text{\acs{QPW}})} }{2}
  }{
    \norm{ \vect{u}^{(\text{B})} }{2}
  }
  10^{ - \frac{ \text{SNR}_{\text{dB}} }{ 20 \si{\deci\bel} } }
 \label{eqn:sim_study_params_reg_factor_threshold}
\end{equation}
specified
% 2.) lower bounds on the l2-norms of the sensing matrices' column vectors
the lower bounds on
the $\ell_{2}$-norms of
the column vectors
\eqref{eqn:recovery_reg_norm_l2_norms_lb}.
% c) maximum number of iterations in SPGL1 was N_{\text{iter}}
% article:Schiffner2018, Sect. VI. Implementation / Sect. VI-C. Sparsity-Promoting lq-Minimization Method (subsec:imp_lq_minimization)
% - \acs{SPGL1} is ITERATIVE and left multiplied a sequence of recursively-generated vectors by
%   the potentially densely-populated normalized sensing matrix \eqref{eqn:recon_reg_norm_sensing_matrix} or its adjoint.
The maximum number of
iterations in
\ac{SPGL1} was
$N_{\text{iter}}$
(cf. \cref{tab:sim_study_parameters}(g)).
% d) normalization parameters \epsilon_{n} induced a sequence of five renormalized CS problems in Foucart's algorithm
% article:Schiffner2018, Sect. VI. Implementation / Sect. VI-C. Sparsity-Promoting lq-Minimization Method (subsec:imp_lq_minimization)
% - \name{Foucart}'s algorithm \cite[Sect. 4]{article:FoucartACHA2009} iteratively applied this method based on \ac{SPGL1} to
%   a sequence of RENORMALIZED \ac{CS} PROBLEMS to approximate the nonconvex $\ell_{q}$-minimization method \eqref{eqn:recovery_reg_norm_lq_minimization} for
%   the half-open parameter interval $q \in [ 0; 1 )$.
% - cs_2d_mlfma_options.q = 0.5;
% - cs_2d_mlfma_options.epsilon_n = 1 ./ (1 + (1:5)); [cs_2d_mlfma_options.epsilon_n = 1 ./ (2 + (0:4))]
The normalization parameters
$\epsilon_{n}$ induced
a sequence of
five renormalized \ac{CS} problems in
\name{Foucart}'s algorithm
(cf. \cref{subsec:imp_lq_minimization}).
% e) Foucart's algorithm entailed six l1 minimizations
Since
\ac{SPGL1} provided
the initial guess,
\name{Foucart}'s algorithm entailed
six $\ell_{1}$ minimizations.


%%%%%%%%%%%%%%%%%%%%%%%%%%%%%%%%%%%%%%%%%%%%%%%%%%%%%%%%%%%%%%%%%%%%%%%%%%%%%%%%%%%%%%%%%%%%%%%%%%%%%%%%%%%%%%%%
% 9.) reference sensing matrices
%%%%%%%%%%%%%%%%%%%%%%%%%%%%%%%%%%%%%%%%%%%%%%%%%%%%%%%%%%%%%%%%%%%%%%%%%%%%%%%%%%%%%%%%%%%%%%%%%%%%%%%%%%%%%%%%
\subsubsection{Reference Sensing Matrices}
\label{subsubsec:sim_study_params_ref_sens_mat}
%---------------------------------------------------------------------------------------------------------------
% 1.) reference sensing matrices
%---------------------------------------------------------------------------------------------------------------
% a) two types of sensing matrices served as benchmarks
Two types of
sensing matrices, which emerged from
\ac{GWN}, served as
benchmarks.
% b) first reference observation process and the associated sensing matrix met the RIP with very high probability
% article:Schiffner2018, Sect. II. Compressed Sensing in a Nutshell (sec:compressed_sensing)
% - Fortunately, CERTAIN TYPES OF RANDOM SENSING MATRICES \eqref{eqn:cs_math_prob_general_sensing_matrix} also obey
%   THE \ac{RIP} WITH VERY HIGH PROBABILITY, if
%   THE NUMBER OF OBSERVATIONS IS SUFFICIENTLY LARGE \cite[6]{book:Foucart2013}, \cite{article:TroppPIEEE2010}.
% - REALIZATIONS OF \ac{IID} RANDOM VARIABLES governed by certain distributions, e.g.
%   GAUSSIAN or \name{Bernoulli}, as entries \cite[Thm. 5.2]{article:BaraniukCA2008} and
%   randomly and uniformly chosen scaled rows of a \name{Fourier} basis \cite[Thm. 3.3]{article:RudelsonCPAM2008}, for example, require
%   $M \in \bigomega{ s \ln( N / s ) }$ and
%   $M \in \bigomega{ s \ln^{4}( N ) }$ observations, respectively.
% article:BaraniukCA2008: A Simple Proof of the Restricted Isometry Property for Random Matrices (real-valued CS problem / canonical basis)
% 6 Discussion
% - Furthermore, we prove above that the RIP HOLDS for \Phi(ω) WITH HIGH PROBABILITY when
%   the matrix is drawn according to one of the distributions
%   [ \phi_{ i, j } \sim \gaussian{ 0 }{ 1 / n } ] (4.4),
%   [ \phi_{ i, j } := + 1 / \sqrt{n} with probability 0.5; - 1 / \sqrt{n} with probability 0.5 ] (4.5), or
%   [ \phi_{ i, j } := + \sqrt{ 3 / n } with probability 1/6; 0 with probability 2/3; - \sqrt{ 3 / n } with probability 1/6 ] (4.6) []. (p. 261)
For
a sufficiently large
% 1.) number of observations (all pulse-echo measurements, multifrequent, all array elements)
number of
observations
\eqref{eqn:recovery_sys_lin_eq_num_obs}, both
% 2.) first reference observation process (RIP)
the real-valued random
$N_{\text{obs}} \times N_{\text{lat}}$ observation process
\begin{subequations}
\begin{align}
 %--------------------------------------------------------------------------------------------------------------
 % first reference observation process and its entries (RIP)
 %--------------------------------------------------------------------------------------------------------------
  \mat{\Phi}^{(\text{\acs{RIP}})}
  =
  \vertcat_{ m = 1 }^{ N_{\text{obs}} }
    \horzcat_{ i = 1 }^{ N_{\text{lat}} }
      \phi_{ m, i }^{(\text{\acs{RIP}})},
  & &
  \phi_{ m, i }^{(\text{\acs{RIP}})}
  \underset{ \text{\acs{IID}} }{ \sim }
  \dgaussian{ 0 }{ \frac{ 1 }{ N_{\text{obs}} } }{1}
 \label{eqn:sim_study_params_ref_obs_proc_rip}
\end{align}
and
% 3.) first reference sensing matrix (RIP)
the associated complex-valued
$N_{\text{obs}} \times N_{\text{lat}}$ sensing matrix
\begin{equation}
 %--------------------------------------------------------------------------------------------------------------
 % first reference sensing matrix (RIP)
 %--------------------------------------------------------------------------------------------------------------
  \mat{A}^{(\text{\acs{RIP}})}
  =
  \mat{\Phi}^{(\text{\acs{RIP}})}
  \mat{\Psi}
 \label{eqn:sim_study_params_ref_sens_mat_rip}
\end{equation}
\end{subequations}
met
% 4.) restricted isometry property (RIP)
the \ac{RIP} with
very high probability
(cf. \cref{sec:compressed_sensing}).
% c) specified variance ensured recorded electric energies of unity expectation
The specified variance ensured
% 1.) recorded electric energies in the pulse echoes (all pulse-echo measurements, multifrequent, all array elements)
recorded electric energies
\eqref{eqn:recovery_reg_v_rx_born_trans_coef_energy} of
% 2.) unity expectation
unity expectation.
% d) replacement of the incident acoustic pressure field in the observation process by complex-valued GWN additionally formed the observation process
The replacement of
% 1.) discretized incident acoustic pressure fields [superpositions of quasi-(d-1)-spherical waves]
the incident acoustic pressure field
\eqref{eqn:recovery_p_in} in
% 2.) observation process (all pulse-echo measurements, multifrequent, all array elements)
the observation process
\eqref{eqn:recovery_sys_lin_eq_v_rx_born_all_f_all_in_mat} by
% 3.) complex-valued GWN (realizations of i.i.d. complex-valued Gaussian random variables)
% MATLAB:
% p_incident_theta_act{index_f} = (randn( N_lattice_axis(2), N_lattice_axis(1) ) + 1j * randn( N_lattice_axis(2), N_lattice_axis(1)) ) * 1e-5;
%$\treal{ p_{l}^{(\text{in}, 0)}( \vect{r}_{\text{lat}, i} ) } \sim \gaussian{ 0 }{ 1 }$
%$\timag{ p_{l}^{(\text{in}, 0)}( \vect{r}_{\text{lat}, i} ) } \sim \gaussian{ 0 }{ 1 }$
complex-valued \ac{GWN} additionally formed
% 4.) second reference observation process (GWN)
the complex-valued structured
$N_{\text{obs}} \times N_{\text{lat}}$ observation process
\TODO{complex GWN}
\begin{subequations}
\begin{align}
 %--------------------------------------------------------------------------------------------------------------
 % second reference observation process and its entries (GWN)
 %--------------------------------------------------------------------------------------------------------------
  \mat{\Phi}^{(\text{\acs{GWN}})}
  =
  \mat{\Phi}\bigl[ p^{(\text{in})} \bigr],
  & &
  p_{l}^{(\text{in}, 0)}( \vect{r}_{\text{lat}, i} )
  \underset{ \text{\acs{IID}} }{ \sim }
  \gaussian{ 0 }{ 1 }
 \label{eqn:sim_study_params_ref_obs_proc_gwn}
\end{align}
%for
% 1.) all admissible frequency indices and all grid points
%$( l, i ) \in \setsymbol{L}_{ \text{BP} }^{(0)} \times \setconsnonneg{ N_{\text{lat}} - 1 }$.
and
% 5.) second reference sensing matrix (GWN)
the associated complex-valued
$N_{\text{obs}} \times N_{\text{lat}}$ sensing matrix
\begin{equation}
 %--------------------------------------------------------------------------------------------------------------
 % second reference sensing matrix (GWN)
 %--------------------------------------------------------------------------------------------------------------
  \mat{A}^{(\text{\acs{GWN}})}
  =
  \mat{\Phi}^{(\text{\acs{GWN}})}
  \mat{\Psi}.
 \label{eqn:sim_study_params_ref_sens_mat_gwn}
\end{equation}
\end{subequations}
% e) complex-valued GWN violated the Helmholtz equations
Although
% 1.) complex-valued GWN
the complex-valued \ac{GWN} violated
% 2.) Helmholtz equations for the incident acoustic pressure fields
the \name{Helmholtz} equations
\eqref{eqn:lin_mod_sol_wave_eq_pde_p_in},
% f) replacement correctly respected the monopole scattering and the reception by the instrumentation
this replacement correctly respected
% 2.) monopole scattering
the monopole scattering and
% 3.) reception by the instrumentation
the reception by
the instrumentation.



%%%%%%%%%%%%%%%%%%%%%%%%%%%%%%%%%%%%%%%%%%%%%%%%%%%%%%%%%%%%%%%%%%%%%%%%%%%%%%%%%%%%%%%%%%%%%%%%%%%%%%%%%%%%%%%%
% 2.) methods
%%%%%%%%%%%%%%%%%%%%%%%%%%%%%%%%%%%%%%%%%%%%%%%%%%%%%%%%%%%%%%%%%%%%%%%%%%%%%%%%%%%%%%%%%%%%%%%%%%%%%%%%%%%%%%%%
\subsection{Methods}
%\label{subsec:sim_study_methods}
%%%%%%%%%%%%%%%%%%%%%%%%%%%%%%%%%%%%%%%%%%%%%%%%%%%%%%%%%%%%%%%%%%%%%%%%%%%%%%%%%%%%%%%%%%%%%%%%%%%%%%%%%%%%%%%%
% 1.) incident acoustic pressure fields
%%%%%%%%%%%%%%%%%%%%%%%%%%%%%%%%%%%%%%%%%%%%%%%%%%%%%%%%%%%%%%%%%%%%%%%%%%%%%%%%%%%%%%%%%%%%%%%%%%%%%%%%%%%%%%%%
\subsubsection{Incident Acoustic Pressure Fields}
%\label{subsubsec:sim_study_methods_p_in}
%---------------------------------------------------------------------------------------------------------------
% 1.) incident acoustic pressure fields
%---------------------------------------------------------------------------------------------------------------
% a) acoustic pressure fields were computed for all types of incident waves
% 1.) discretized incident acoustic pressure fields [superpositions of quasi-(d-1)-spherical waves]
The acoustic pressure fields
\eqref{eqn:recovery_p_in} were computed for
% 2.) all types of incident waves
all types of
incident waves.
% b) spatial and spectral dependencies were analyzed for the discrete frequency closest to the center frequency and three closely spaced positions next to the r_{2}-axis
For
% 1.) wire phantom
the wire phantom,
% 2.) spatial and spectral dependencies
their spatial and
spectral dependencies were analyzed for
% 3.) discrete frequency closest to the center frequency
% f_{ l_{\text{c}} } = \SI{3.9951}{\mega\hertz} \approx f_{\text{c}}, with the index $l_{\text{c}} = 329$
the discrete frequency closest to
the center frequency and
% 4.) three closely spaced positions next to the r_{2}-axis
% \Delta r_{\text{lat}, 1} = \Delta r_{\text{lat}, 2} = \Delta r_{\text{mat}, 1} = \SI{76.2}{\micro\meter}
% indices_pos_ref_x = [256, 256, 256];
% indices_pos_ref_z = [301, 316, 331];
% pos_ref_x = -38.1 um / -38.1 um / -38.1 um
% pos_ref_z = 22.8981 mm / 24.0411 mm / 25.1841 mm
%$\vect{r}_{\text{ref}, i} = \trans{ ( - \Delta r_{\text{lat}, 1} / 2, r_{\text{ref}, i, 2} ) }$,
%$i \in \setcons{ 3 }$, with
%$r_{\text{ref}, 1, 2} = 300.5 \Delta r_{1} \approx \SI{22.9}{\milli\meter}$,
%$r_{\text{ref}, 2, 2} = 315.5 \Delta r_{1} \approx \SI{24}{\milli\meter}$, and
%$r_{\text{ref}, 3, 2} = 330.5 \Delta r_{1} \approx \SI{25.2}{\milli\meter}$.
three closely spaced positions next to
the $r_{2}$-axis,
respectively.
% c) least-squares fit of an affine linear model was subtracted from all unwrapped phases to emphasize the differences
The least-squares fit of
% 1.) affine linear model
an affine linear model to
% 2.) unwrapped phase
the unwrapped phase of
% 3.) incident acoustic pressure field associated with the QPW
the acoustic pressure field
\eqref{eqn:recovery_p_in} associated with
the \ac{QPW} at
% 4.) first position $\vect{r}_{\text{ref}, 1}$
the first position was subtracted from
% 5.) all unwrapped phases
all unwrapped phases to emphasize
% 6.) differences
the differences.


%%%%%%%%%%%%%%%%%%%%%%%%%%%%%%%%%%%%%%%%%%%%%%%%%%%%%%%%%%%%%%%%%%%%%%%%%%%%%%%%%%%%%%%%%%%%%%%%%%%%%%%%%%%%%%%%
% 2.) recorded radio frequency voltage signals
%%%%%%%%%%%%%%%%%%%%%%%%%%%%%%%%%%%%%%%%%%%%%%%%%%%%%%%%%%%%%%%%%%%%%%%%%%%%%%%%%%%%%%%%%%%%%%%%%%%%%%%%%%%%%%%%
\subsubsection{Recorded Radio Frequency Voltage Signals}
%\label{subsubsec:sim_study_methods_v_rx}
%---------------------------------------------------------------------------------------------------------------
% 1.) recorded radio frequency voltage signals
%---------------------------------------------------------------------------------------------------------------
% a) ten realizations of the recorded RF voltage signals were derived from their Born approximation for each type of incident wave and each SNR
% d) number of recovery experiments
 %The number of
 %recovery experiments conducted for
 %each reference \ac{SNR} and
 %each type of
 %incident wave amounted to
 %$N_{\text{rcn}} = 10$.
Ten realizations of
%($N_{\text{rcn}} = 10$) of
% 1.) vectors stacking the relevant Fourier coefficients of the recorded RF voltage signals (all pulse-echo measurements, multifrequent, all array elements)
the recorded \ac{RF} voltage signals
\eqref{eqn:recovery_sys_lin_eq_v_rx_born_all_f_all_in_v_rx} were derived from
% 2.) approximate vectors stacking the relevant Fourier coefficients of the recorded RF voltage signals (all pulse-echo measurements, multifrequent, all array elements)
their \name{Born} approximation
\eqref{eqn:recovery_sys_lin_eq_v_rx_born_all_f_all_in_v_rx_born} for
% 3.) each type of incident wave
each type of
incident wave and
% 4.) each SNR
each \ac{SNR} by inserting
% 5.) nearly-sparse representation / vector of transform coefficients
the sparse representation
\eqref{eqn:recovery_reg_sparse_representation} and
% 6.) additive errors of the specified energy levels
the additive errors into
% 7.) linear algebraic system (all pulse-echo measurements, multifrequent, all array elements, additive errors)
the linear algebraic system
\eqref{eqn:recovery_reg_prob_general_obs_trans_coef_error}.


%%%%%%%%%%%%%%%%%%%%%%%%%%%%%%%%%%%%%%%%%%%%%%%%%%%%%%%%%%%%%%%%%%%%%%%%%%%%%%%%%%%%%%%%%%%%%%%%%%%%%%%%%%%%%%%%
% 3.) recorded electric energies
%%%%%%%%%%%%%%%%%%%%%%%%%%%%%%%%%%%%%%%%%%%%%%%%%%%%%%%%%%%%%%%%%%%%%%%%%%%%%%%%%%%%%%%%%%%%%%%%%%%%%%%%%%%%%%%%
\subsubsection{Recorded Electric Energies}
%\label{subsubsec:sim_study_methods_E_rx}
%---------------------------------------------------------------------------------------------------------------
% 1.) recorded electric energies
%---------------------------------------------------------------------------------------------------------------
% a) recorded electric energies were computed for all sensing matrices except for the first reference
% 1.) recorded electric energies in the pulse echoes (all pulse-echo measurements, multifrequent, all array elements)
The recorded electric energies
\eqref{eqn:recovery_reg_v_rx_born_trans_coef_energy} were computed for
% 2.) all sensing matrices (all pulse-echo measurements, multifrequent, all array elements)
all sensing matrices
\eqref{eqn:recovery_reg_sensing_matrix} except for
% 3.) first reference sensing matrix (RIP)
the first reference
\eqref{eqn:sim_study_params_ref_sens_mat_rip}, which approximately induced
% 4.) expected energies of unity
the expected energies of
unity.
% b) visual inspections revealed the transfer behaviors of the sensing matrices for the tissue-mimicking phantom
% TODO: high dynamic ranges
Their visual inspections revealed
% 1.) transfer behaviors of the associated sensing matrices
the transfer behaviors of
% 2.) sensing matrices (all pulse-echo measurements, multifrequent, all transducer elements)
the sensing matrices
\eqref{eqn:recovery_reg_sensing_matrix} for
% 3.) tissue-mimicking phantom
the tissue-mimicking phantom.


%%%%%%%%%%%%%%%%%%%%%%%%%%%%%%%%%%%%%%%%%%%%%%%%%%%%%%%%%%%%%%%%%%%%%%%%%%%%%%%%%%%%%%%%%%%%%%%%%%%%%%%%%%%%%%%%
% 4.) transform point spread functions
%%%%%%%%%%%%%%%%%%%%%%%%%%%%%%%%%%%%%%%%%%%%%%%%%%%%%%%%%%%%%%%%%%%%%%%%%%%%%%%%%%%%%%%%%%%%%%%%%%%%%%%%%%%%%%%%
\subsubsection{Transform Point Spread Functions}
%\label{subsubsec:sim_study_methods_tpsf}
%---------------------------------------------------------------------------------------------------------------
% 1.) transform point spread functions
%---------------------------------------------------------------------------------------------------------------
% a) TPSFs associated with all sensing matrices were evaluated for all ( n_{1}, n_{2} ) \in \setcons{ N_{\text{lat}} } \times \setsymbol{I}
% article:Schiffner2018, Sect. II. Compressed Sensing in a Nutshell (sec:compressed_sensing)
% - It [TPSF] equals the mutual correlation coefficient of the column vectors given by \cite[(23)]{article:ProvostITMI2009}, \cite[(2)]{article:LustigMRM2007}
%   [ \tpsf{ \mat{A} }{ n_{1} }{ n_{2} } = \frac{ \inprod{ \vect{a}_{ n_{1} } }{ \vect{a}_{ n_{2} } } }{ \norm{ \vect{a}_{ n_{1} } }{2} \norm{ \vect{a}_{ n_{2} } }{2} } ] (eqn:cs_math_tpsf) for
%   all $( n_{1}, n_{2} ) \in \setcons{ N }^{2}$.
% - Owing to the HIGH DIMENSIONALITY OF THE SENSING MATRICES \eqref{eqn:cs_math_prob_general_sensing_matrix},
%   PRACTICAL EVALUATIONS OF THE \ac{TPSF} \eqref{eqn:cs_math_tpsf} USUALLY FIX
%   THE SECOND INDEX according to
%   the expected support of the nearly-sparse representation \eqref{eqn:def_transform_coefficients}, i.e. $n_{2} \in \supp( \vectsym{\theta} )$
%   (cf. e.g. \cite[Fig. 1]{article:ProvostITMI2009}, \cite[Figs. 4 and 5]{article:LustigMRM2007}).
The \acp{TPSF}
\eqref{eqn:cs_math_tpsf} associated with
% 1.) all sensing matrices [sensing matrices \eqref{eqn:recovery_reg_sensing_matrix} induced by all incident waves / both reference sensing matrices]
all sensing matrices were evaluated for
% 2.) all pairs of indices
all $( n_{1}, n_{2} ) \in \setcons{ N_{\text{lat}} } \times \setsymbol{I}$, where
% 3.) nine indices n_{2} \in \setcons{ N_{\text{lat}} }
$\setsymbol{I} \subset \setcons{ N_{\text{lat}} }$ fixed
nine indices.
% b) positions \vect{r}_{ \text{lat}, n_{2} - 1 } were approximately uniformly distributed along the diagonal from ( -17.5, 2 ) mm to ( 17.5, 37 ) mm
% MATLAB:
% N_coordinates = 9;
% direction = N_lattice_axis_cs' - ones(2,1);
% tpsf_coordinates = round( ones(N_coordinates, 2) + linspace(0.05, 0.95, N_coordinates)' * direction' );
% cs_2d_mlfma_options.tpsf_indices = [tpsf_coordinates(:,1) - 1, tpsf_coordinates(:,2)] * [N_lattice_axis_cs(2); 1];
%
% tpsf_coordinates_1 = \round{ 1 + ( 0.05 + ( s - 1 ) * 0.9 / 8 ) * 511 }
% tpsf_coordinates_2 = \round{ 1 + ( 0.05 + ( s - 1 ) * 0.9 / 8 ) * 511 }
% tpsf_indices = ( tpsf_coordinates_1 - 1 ) * 512 + tpsf_coordinates_2
%	       = \round{ ( 0.05 + ( s - 1 ) * 0.9 / 8 ) * 511 } * 513 + 1
%	       = 513 \tround{ 25.55 + ( s - 1 ) 57.4875 } + 1
% index_1 = \ceil{ n / N_{\text{lat}, 2} }
% index_2 = n - ( index_1 - 1 ) * N_{\text{lat}, 2}
% r_{ \text{lat}, n - 1, 1 } = ( index_1 - 513 / 2 ) \Delta r_{\text{lat}, 1} = ( \ceil{ n / 512 } - 513 / 2 ) \Delta r_{\text{lat}, 1}
% r_{ \text{lat}, n - 1, 2 } = ( index_2 - 1 / 2 ) \Delta r_{\text{lat}, 2} = ( n - 512 \ceil{ n / 512 } + 511.5 ) \Delta r_{\text{lat}, 2}
% \Delta r_{\text{lat}, 1} = \Delta r_{\text{lat}, 2} = \SI{76.2}{\micro\meter}
For
% 1.) wire phantom
the wire phantom,
% 2.) positions of the individual compressibility fluctuations
the positions
$\vect{r}_{ \text{lat}, n_{2} - 1 }$ were
% 3.) uniformly distributed
approximately uniformly distributed along
% 4.) diagonal from (-17.4879 mm, 2.0193 mm ) to ( 17.4879 mm, 36.9951 mm )
the diagonal from
$\trans{ ( \SI{-17.5}{\milli\meter}, \SI{2}{\milli\meter} ) }$ to
$\trans{ ( \SI{17.5}{\milli\meter}, \SI{37}{\milli\meter} ) }$ and numbered from
\numrange{1}{9} with
increasing axial coordinate. %, i.e.
%$\setsymbol{I} = \{ n_{2} \in \setcons{ N_{\text{lat}} }: n_{2} = 513 \tround{ 25.55 + ( s - 1 ) 57.4875 } + 1, s \in \setcons{ 9 } \}$.
% c) normalized spatial frequencies \hat{\vect{K}}_{ n_{2} } were approximately uniformly distributed along the semicircle with the center \hat{\vect{K}}_{ \text{c} } and the radius \hat{K}_{ \text{r} }
% MATLAB:
% N_tpsf = 9;
% indices_center = [257, 101];
% indices_radius = 101;
% M_indices = (N_tpsf - 1) / 2;
% indices_phi = (-M_indices:M_indices) * pi / N_tpsf + pi / 2;
% tpsf_indices = round( repmat( indices_center, [N_tpsf, 1] ) + indices_radius * [ cos( indices_phi(:) ), sin( indices_phi(:) ) ] );
%
% \hat{K}_{1} = ( \ceil{ N_{\text{lat}, 1} / 2 } - N_{\text{lat}, 1} + index_1 - 1 ) / N_{\text{lat}, 1} = ( index_1 - 257 ) / 512
% \hat{K}_{2} = ( index_2 - 1 ) / N_{\text{lat}, 2} = ( index_2 - 1 ) / 512
%
% indices_phi = ( s - 0.5 ) \pi / 9 (CHECKED!)
% tpsf_indices_1 = 257 + \tround{ 101 \cos[ ( s - 0.5 ) \pi / 9 ] } (CHECKED!)
% tpsf_indices_2 = 101 + \tround{ 101 \sin[ ( s - 0.5 ) \pi / 9 ] } (CHECKED!)
% tpsf_indices = ( tpsf_indices_1 - 1 ) * 512 + tpsf_indices_2
%	       = 131173 + 512 \tround{ 101 \cos[ ( s - 0.5 ) \pi / 9 ] } + \tround{ 101 \sin[ ( s - 0.5 ) \pi / 9 ] } (CHECKED!)
For
% 1.) tissue-mimicking phantom
the tissue-mimicking phantom,
% 2.) normalized spatial frequencies of the complex exponential functions
the normalized spatial frequencies
$\hat{\vect{K}}_{ n_{2} } = \trans{ ( \hat{K}_{ n_{2}, 1 }, \hat{K}_{ n_{2}, 2 } ) }$ were
% 3.) uniformly distributed
approximately uniformly distributed along
% 4.) semicircle
the semicircle with
% 5.) center \hat{\vect{K}}_{ \text{c} } = \trans{ ( 0, 25 ) } / 128
the center
$\hat{\vect{K}}_{ \text{c} } = \trans{ ( 0, 25 ) } / 128$ and
% 6.) radius \hat{K}_{ \text{r} } = 101 / 512
the radius
$\hat{K}_{ \text{r} } = 101 / 512$ and numbered from
\numrange{1}{9} with
increasing polar angle. %, i.e.
%$\setsymbol{I} = \{ n_{2} \in \setcons{ N_{\text{lat}} }: n_{2} = 131173 + 512 \tround{ 101 \cos[ ( s - 0.5 ) \pi / 9 ] } + \tround{ 101 \sin[ ( s - 0.5 ) \pi / 9 ] }, s \in \setcons{ 9 } \}$.
% d) thresholded l2-norms of the column vectors substituted the original l2-norms in the denominators of the TPSFs to avoid numerical inaccuracies
% TODO: does the replacement affect the reference sensing matrices? -> probably not
The thresholded $\ell_{2}$-norms of
the column vectors
\eqref{eqn:recovery_reg_norm_l2_norms_thresholded}, however, substituted
% 1.) original l2-norms
the original $\ell_{2}$-norms in
the denominators of
% 2.) transform point spread functions (TPSFs)
the \acp{TPSF}
\eqref{eqn:cs_math_tpsf} for
% 3.) tissue-mimicking phantom
this phantom to avoid
% 4.) numerical inaccuracies
the numerical inaccuracies caused by
% 5.) high dynamic ranges
their high dynamic ranges.
% e) empirical threshold factors for the reference SNR of \text{SNR}_{\text{dB}} = \SI{10}{\deci\bel} specified their lower bounds
The empirical factors
\eqref{eqn:sim_study_params_reg_factor_threshold} with
% 1.) reference SNR
$\text{SNR}_{\text{dB}} = \SI{10}{\deci\bel}$ specified
% 2.) lower bounds on the l2-norms of the sensing matrices' column vectors
their lower bounds
\eqref{eqn:recovery_reg_norm_l2_norms_lb}.
% f) each computed TPSF was characterized by its FEHM for each index and its empirical CDF
In addition to
a visual inspection,
% 1.) transform point spread function (TPSF)
each computed \ac{TPSF}
\eqref{eqn:cs_math_tpsf} was characterized by
% 2.) full extent at half maximum (FEHM)
its \ac{FEHM} for
% 3.) each index n_{2} \in \setsymbol{I}
each index
$n_{2} \in \setsymbol{I}$ and
% 4.) empirical cumulative distribution function (CDF)
its empirical \ac{CDF}.
% g) empirical CDF excluded all n_{1} = n_{2} and stated the percentages of diverse pulse echoes whose correlation coefficient did not exceed a specified threshold
The latter excluded
% 1.) all identical pairs of indices
all $n_{1} = n_{2}$ and, thus, stated
% 2.) percentages of diverse pulse echoes
the percentages of
diverse pulse echoes whose
% 3.) correlation coefficient
correlation coefficient did not exceed
% 4.) specified threshold
a specified threshold.


%%%%%%%%%%%%%%%%%%%%%%%%%%%%%%%%%%%%%%%%%%%%%%%%%%%%%%%%%%%%%%%%%%%%%%%%%%%%%%%%%%%%%%%%%%%%%%%%%%%%%%%%%%%%%%%%
% 5.) adjoint normalized sensing matrices
%%%%%%%%%%%%%%%%%%%%%%%%%%%%%%%%%%%%%%%%%%%%%%%%%%%%%%%%%%%%%%%%%%%%%%%%%%%%%%%%%%%%%%%%%%%%%%%%%%%%%%%%%%%%%%%%
\subsubsection{Adjoint Normalized Sensing Matrices}
%\label{subsubsec:sim_study_methods_adjoint}
%---------------------------------------------------------------------------------------------------------------
% 1.) adjoint normalized sensing matrices
%---------------------------------------------------------------------------------------------------------------
% a) normalized recorded RF voltage signals generated by all types of incident waves were left multiplied by the associated adjoint normalized sensing matrices
% 1.) normalized linear algebraic system (all pulse-echo measurements, multifrequent, all array elements, additive errors)
The normalized recorded \ac{RF} voltage signals
\eqref{eqn:recovery_reg_norm_obs_trans_coef_error} generated by
% 2.) all types of incident waves
all types of
incident waves were
left multiplied by
% 3.) adjoint normalized sensing matrices (all pulse-echo measurements, multifrequent, all array elements)
the adjoint normalized sensing matrices
\eqref{eqn:recon_reg_norm_sensing_matrix}.
% b) visual inspections of these products revealed the interference effects affecting the recovery
% article:Schiffner2018, Sect. VI. Implementation / Sect. VI-C Sparsity-Promoting lq-Minimization Method
% - \acs{SPGL1} is iterative and left multiplied
%   a sequence of recursively-generated vectors by
%   the potentially densely-populated normalized sensing matrix \eqref{eqn:recon_reg_norm_sensing_matrix} or its adjoint.
The visual inspections of
these products, which underlay
the implementation of
% 1.) sparsity-promoting lq-minimization method
the sparsity-promoting $\ell_{q}$-minimization method
\eqref{eqn:recovery_reg_norm_lq_minimization}, revealed
% 2.) interference effects
important interference effects affecting
the recovery.


%%%%%%%%%%%%%%%%%%%%%%%%%%%%%%%%%%%%%%%%%%%%%%%%%%%%%%%%%%%%%%%%%%%%%%%%%%%%%%%%%%%%%%%%%%%%%%%%%%%%%%%%%%%%%%%%
% 6.) recovery by lq-minimization
%%%%%%%%%%%%%%%%%%%%%%%%%%%%%%%%%%%%%%%%%%%%%%%%%%%%%%%%%%%%%%%%%%%%%%%%%%%%%%%%%%%%%%%%%%%%%%%%%%%%%%%%%%%%%%%%
\subsubsection{Recovery by $\ell_{q}$-Minimization}
%\label{subsubsec:sim_study_methods_lq_minimization}
%---------------------------------------------------------------------------------------------------------------
% 1.) recovery by lq-minimization
%---------------------------------------------------------------------------------------------------------------
% a) forty instances of the normalized CS problem were solved by the sparsity-promoting lq-minimization method
% 10 realizations per SNR per type of wave => 10 * 5 * 4
The $\num{200}$ instances of
% 1.) CS problem associated with the normalized linear algebraic system
the normalized \ac{CS} problem
\eqref{eqn:recovery_reg_norm_prob_general} generated by
% 2.) all types of incident waves
all types of
incident waves and
% 3.) realizations of the vectors stacking the relevant Fourier coefficients of the recorded RF voltage signals (all pulse-echo measurements, multifrequent, all array elements)
all realizations of
the recorded \ac{RF} voltage signals
\eqref{eqn:recovery_sys_lin_eq_v_rx_born_all_f_all_in_v_rx} were solved by
% 4.) sparsity-promoting lq-minimization method
the sparsity-promoting $\ell_{q}$-minimization method
\eqref{eqn:recovery_reg_norm_lq_minimization}.
% b) structural differences were quantified by the mean SSIM indices
% article:WangISPM2009: Mean squared error: Love it or leave it? A new look at Signal Fidelity Measures
% - These local similarities are expressed using simple, easily computed statistics, and combined together to form local SSIM [7]
%   (2) [local SSIM index], where
%   µ_{x} and µ_{y} are (respectively) the local sample means of x and y,
%   σ_{x} and σ_{y} are (respectively) the local sample standard deviations of x and y, and
%   σ_{xy} is the sample cross correlation of x and y after removing their means. (pp. 105, 106)
% article:WangITIP2004: Image quality assessment: From error visibility to structural similarity
% - This results in a specific form of the SSIM index (13). (p. 605)
\TODO{visual inspection}
In addition to
a visual inspection,
structural differences between
% 1.) estimated vectors stacking the regular samples in the discretized relative spatial fluctuations in the unperturbed compressibility
the recovered compressibility fluctuations
\eqref{eqn:recovery_reg_norm_lq_minimization_sol_mat_params} and
% 2.) specified vectors stacking the regular samples in the discretized relative spatial fluctuations in the unperturbed compressibility
their specified version
\eqref{eqn:recovery_sys_lin_eq_gamma_kappa_bp_vector} were quantified by
% 3.) mean SSIM indices
the mean \ac{SSIM} indices
\cite[(2)]{article:WangISPM2009}, whereas
% c) quantitative differences were measured by the relative RMSEs
quantitative differences were measured by
% 1.) relative RMSEs
the relative \acp{RMSE}.
% d) sparsities and speed of convergence were gauged by the numbers of components within the illustrated dynamic range and the numbers of iterations in SPGL1
The sparsity and
% 1.) speed of convergence
speed of
convergence were gauged by
% 2.) numbers of components within the illustrated dynamic range
the numbers of
components within
the illustrated dynamic range and
% 3.) numbers of iterations in SPGL1
the numbers of
iterations in
\ac{SPGL1},
respectively.
% e) incident acoustic energies and the recorded electric energies were related to the sample means of the relative RMSEs caused by the nonconvex l0.5-minimization method
For
each wire,
% 1.) incident acoustic energies at a specified grid point (all pulse-echo measurements, multifrequent)
the incident acoustic energies
\eqref{eqn:recovery_p_in_energy} and
% 2.) recorded electric energies in the pulse echoes (all pulse-echo measurements, multifrequent, all array elements)
the recorded electric energies
\eqref{eqn:recovery_reg_v_rx_born_trans_coef_energy} were related to
% 3.) sample means of the relative RMSEs caused by the nonconvex l0.5-minimization method
the sample means of
the relative \acp{RMSE} caused by
% 4.) nonconvex l0.5-minimization method
the nonconvex $\ell_{0.5}$-minimization method
\eqreflqmin{eqn:recovery_reg_norm_lq_minimization}{ 0.5 }.


