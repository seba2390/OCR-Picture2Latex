%---------------------------------------------------------------------------------------------------------------
% 1.) recovery by lq-minimization
%---------------------------------------------------------------------------------------------------------------
% a) forty instances of the normalized CS problem were solved by the sparsity-promoting lq-minimization method
% 10 realizations per SNR per type of wave => 10 * 5 * 4
The $\num{200}$ instances of
% 1.) CS problem associated with the normalized linear algebraic system
the normalized \ac{CS} problem
\eqref{eqn:recovery_reg_norm_prob_general} generated by
% 2.) all types of incident waves
all types of
incident waves and
% 3.) realizations of the vectors stacking the relevant Fourier coefficients of the recorded RF voltage signals (all pulse-echo measurements, multifrequent, all array elements)
all realizations of
the recorded \ac{RF} voltage signals
\eqref{eqn:recovery_sys_lin_eq_v_rx_born_all_f_all_in_v_rx} were solved by
% 4.) sparsity-promoting lq-minimization method
the sparsity-promoting $\ell_{q}$-minimization method
\eqref{eqn:recovery_reg_norm_lq_minimization}.
% b) structural differences were quantified by the mean SSIM indices
% article:WangISPM2009: Mean squared error: Love it or leave it? A new look at Signal Fidelity Measures
% - These local similarities are expressed using simple, easily computed statistics, and combined together to form local SSIM [7]
%   (2) [local SSIM index], where
%   µ_{x} and µ_{y} are (respectively) the local sample means of x and y,
%   σ_{x} and σ_{y} are (respectively) the local sample standard deviations of x and y, and
%   σ_{xy} is the sample cross correlation of x and y after removing their means. (pp. 105, 106)
% article:WangITIP2004: Image quality assessment: From error visibility to structural similarity
% - This results in a specific form of the SSIM index (13). (p. 605)
\TODO{visual inspection}
In addition to
a visual inspection,
structural differences between
% 1.) estimated vectors stacking the regular samples in the discretized relative spatial fluctuations in the unperturbed compressibility
the recovered compressibility fluctuations
\eqref{eqn:recovery_reg_norm_lq_minimization_sol_mat_params} and
% 2.) specified vectors stacking the regular samples in the discretized relative spatial fluctuations in the unperturbed compressibility
their specified version
\eqref{eqn:recovery_sys_lin_eq_gamma_kappa_bp_vector} were quantified by
% 3.) mean SSIM indices
the mean \ac{SSIM} indices
\cite[(2)]{article:WangISPM2009}, whereas
% c) quantitative differences were measured by the relative RMSEs
quantitative differences were measured by
% 1.) relative RMSEs
the relative \acp{RMSE}.
% d) sparsities and speed of convergence were gauged by the numbers of components within the illustrated dynamic range and the numbers of iterations in SPGL1
The sparsity and
% 1.) speed of convergence
speed of
convergence were gauged by
% 2.) numbers of components within the illustrated dynamic range
the numbers of
components within
the illustrated dynamic range and
% 3.) numbers of iterations in SPGL1
the numbers of
iterations in
\ac{SPGL1},
respectively.
% e) incident acoustic energies and the recorded electric energies were related to the sample means of the relative RMSEs caused by the nonconvex l0.5-minimization method
For
each wire,
% 1.) incident acoustic energies at a specified grid point (all pulse-echo measurements, multifrequent)
the incident acoustic energies
\eqref{eqn:recovery_p_in_energy} and
% 2.) recorded electric energies in the pulse echoes (all pulse-echo measurements, multifrequent, all array elements)
the recorded electric energies
\eqref{eqn:recovery_reg_v_rx_born_trans_coef_energy} were related to
% 3.) sample means of the relative RMSEs caused by the nonconvex l0.5-minimization method
the sample means of
the relative \acp{RMSE} caused by
% 4.) nonconvex l0.5-minimization method
the nonconvex $\ell_{0.5}$-minimization method
\eqreflqmin{eqn:recovery_reg_norm_lq_minimization}{ 0.5 }.
