%---------------------------------------------------------------------------------------------------------------
% 1.) transform point spread functions
%---------------------------------------------------------------------------------------------------------------
% a) TPSFs associated with all sensing matrices were evaluated for all ( n_{1}, n_{2} ) \in \setcons{ N_{\text{lat}} } \times \setsymbol{I}
% article:Schiffner2018, Sect. II. Compressed Sensing in a Nutshell (sec:compressed_sensing)
% - It [TPSF] equals the mutual correlation coefficient of the column vectors given by \cite[(23)]{article:ProvostITMI2009}, \cite[(2)]{article:LustigMRM2007}
%   [ \tpsf{ \mat{A} }{ n_{1} }{ n_{2} } = \frac{ \inprod{ \vect{a}_{ n_{1} } }{ \vect{a}_{ n_{2} } } }{ \norm{ \vect{a}_{ n_{1} } }{2} \norm{ \vect{a}_{ n_{2} } }{2} } ] (eqn:cs_math_tpsf) for
%   all $( n_{1}, n_{2} ) \in \setcons{ N }^{2}$.
% - Owing to the HIGH DIMENSIONALITY OF THE SENSING MATRICES \eqref{eqn:cs_math_prob_general_sensing_matrix},
%   PRACTICAL EVALUATIONS OF THE \ac{TPSF} \eqref{eqn:cs_math_tpsf} USUALLY FIX
%   THE SECOND INDEX according to
%   the expected support of the nearly-sparse representation \eqref{eqn:def_transform_coefficients}, i.e. $n_{2} \in \supp( \vectsym{\theta} )$
%   (cf. e.g. \cite[Fig. 1]{article:ProvostITMI2009}, \cite[Figs. 4 and 5]{article:LustigMRM2007}).
The \acp{TPSF}
\eqref{eqn:cs_math_tpsf} associated with
% 1.) all sensing matrices [sensing matrices \eqref{eqn:recovery_reg_sensing_matrix} induced by all incident waves / both reference sensing matrices]
all sensing matrices were evaluated for
% 2.) all pairs of indices
all $( n_{1}, n_{2} ) \in \setcons{ N_{\text{lat}} } \times \setsymbol{I}$, where
% 3.) nine indices n_{2} \in \setcons{ N_{\text{lat}} }
$\setsymbol{I} \subset \setcons{ N_{\text{lat}} }$ fixed
nine indices.
% b) positions \vect{r}_{ \text{lat}, n_{2} - 1 } were approximately uniformly distributed along the diagonal from ( -17.5, 2 ) mm to ( 17.5, 37 ) mm
% MATLAB:
% N_coordinates = 9;
% direction = N_lattice_axis_cs' - ones(2,1);
% tpsf_coordinates = round( ones(N_coordinates, 2) + linspace(0.05, 0.95, N_coordinates)' * direction' );
% cs_2d_mlfma_options.tpsf_indices = [tpsf_coordinates(:,1) - 1, tpsf_coordinates(:,2)] * [N_lattice_axis_cs(2); 1];
%
% tpsf_coordinates_1 = \round{ 1 + ( 0.05 + ( s - 1 ) * 0.9 / 8 ) * 511 }
% tpsf_coordinates_2 = \round{ 1 + ( 0.05 + ( s - 1 ) * 0.9 / 8 ) * 511 }
% tpsf_indices = ( tpsf_coordinates_1 - 1 ) * 512 + tpsf_coordinates_2
%	       = \round{ ( 0.05 + ( s - 1 ) * 0.9 / 8 ) * 511 } * 513 + 1
%	       = 513 \tround{ 25.55 + ( s - 1 ) 57.4875 } + 1
% index_1 = \ceil{ n / N_{\text{lat}, 2} }
% index_2 = n - ( index_1 - 1 ) * N_{\text{lat}, 2}
% r_{ \text{lat}, n - 1, 1 } = ( index_1 - 513 / 2 ) \Delta r_{\text{lat}, 1} = ( \ceil{ n / 512 } - 513 / 2 ) \Delta r_{\text{lat}, 1}
% r_{ \text{lat}, n - 1, 2 } = ( index_2 - 1 / 2 ) \Delta r_{\text{lat}, 2} = ( n - 512 \ceil{ n / 512 } + 511.5 ) \Delta r_{\text{lat}, 2}
% \Delta r_{\text{lat}, 1} = \Delta r_{\text{lat}, 2} = \SI{76.2}{\micro\meter}
For
% 1.) wire phantom
the wire phantom,
% 2.) positions of the individual compressibility fluctuations
the positions
$\vect{r}_{ \text{lat}, n_{2} - 1 }$ were
% 3.) uniformly distributed
approximately uniformly distributed along
% 4.) diagonal from (-17.4879 mm, 2.0193 mm ) to ( 17.4879 mm, 36.9951 mm )
the diagonal from
$\trans{ ( \SI{-17.5}{\milli\meter}, \SI{2}{\milli\meter} ) }$ to
$\trans{ ( \SI{17.5}{\milli\meter}, \SI{37}{\milli\meter} ) }$ and numbered from
\numrange{1}{9} with
increasing axial coordinate. %, i.e.
%$\setsymbol{I} = \{ n_{2} \in \setcons{ N_{\text{lat}} }: n_{2} = 513 \tround{ 25.55 + ( s - 1 ) 57.4875 } + 1, s \in \setcons{ 9 } \}$.
% c) normalized spatial frequencies \hat{\vect{K}}_{ n_{2} } were approximately uniformly distributed along the semicircle with the center \hat{\vect{K}}_{ \text{c} } and the radius \hat{K}_{ \text{r} }
% MATLAB:
% N_tpsf = 9;
% indices_center = [257, 101];
% indices_radius = 101;
% M_indices = (N_tpsf - 1) / 2;
% indices_phi = (-M_indices:M_indices) * pi / N_tpsf + pi / 2;
% tpsf_indices = round( repmat( indices_center, [N_tpsf, 1] ) + indices_radius * [ cos( indices_phi(:) ), sin( indices_phi(:) ) ] );
%
% \hat{K}_{1} = ( \ceil{ N_{\text{lat}, 1} / 2 } - N_{\text{lat}, 1} + index_1 - 1 ) / N_{\text{lat}, 1} = ( index_1 - 257 ) / 512
% \hat{K}_{2} = ( index_2 - 1 ) / N_{\text{lat}, 2} = ( index_2 - 1 ) / 512
%
% indices_phi = ( s - 0.5 ) \pi / 9 (CHECKED!)
% tpsf_indices_1 = 257 + \tround{ 101 \cos[ ( s - 0.5 ) \pi / 9 ] } (CHECKED!)
% tpsf_indices_2 = 101 + \tround{ 101 \sin[ ( s - 0.5 ) \pi / 9 ] } (CHECKED!)
% tpsf_indices = ( tpsf_indices_1 - 1 ) * 512 + tpsf_indices_2
%	       = 131173 + 512 \tround{ 101 \cos[ ( s - 0.5 ) \pi / 9 ] } + \tround{ 101 \sin[ ( s - 0.5 ) \pi / 9 ] } (CHECKED!)
For
% 1.) tissue-mimicking phantom
the tissue-mimicking phantom,
% 2.) normalized spatial frequencies of the complex exponential functions
the normalized spatial frequencies
$\hat{\vect{K}}_{ n_{2} } = \trans{ ( \hat{K}_{ n_{2}, 1 }, \hat{K}_{ n_{2}, 2 } ) }$ were
% 3.) uniformly distributed
approximately uniformly distributed along
% 4.) semicircle
the semicircle with
% 5.) center \hat{\vect{K}}_{ \text{c} } = \trans{ ( 0, 25 ) } / 128
the center
$\hat{\vect{K}}_{ \text{c} } = \trans{ ( 0, 25 ) } / 128$ and
% 6.) radius \hat{K}_{ \text{r} } = 101 / 512
the radius
$\hat{K}_{ \text{r} } = 101 / 512$ and numbered from
\numrange{1}{9} with
increasing polar angle. %, i.e.
%$\setsymbol{I} = \{ n_{2} \in \setcons{ N_{\text{lat}} }: n_{2} = 131173 + 512 \tround{ 101 \cos[ ( s - 0.5 ) \pi / 9 ] } + \tround{ 101 \sin[ ( s - 0.5 ) \pi / 9 ] }, s \in \setcons{ 9 } \}$.
% d) thresholded l2-norms of the column vectors substituted the original l2-norms in the denominators of the TPSFs to avoid numerical inaccuracies
% TODO: does the replacement affect the reference sensing matrices? -> probably not
The thresholded $\ell_{2}$-norms of
the column vectors
\eqref{eqn:recovery_reg_norm_l2_norms_thresholded}, however, substituted
% 1.) original l2-norms
the original $\ell_{2}$-norms in
the denominators of
% 2.) transform point spread functions (TPSFs)
the \acp{TPSF}
\eqref{eqn:cs_math_tpsf} for
% 3.) tissue-mimicking phantom
this phantom to avoid
% 4.) numerical inaccuracies
the numerical inaccuracies caused by
% 5.) high dynamic ranges
their high dynamic ranges.
% e) empirical threshold factors for the reference SNR of \text{SNR}_{\text{dB}} = \SI{10}{\deci\bel} specified their lower bounds
The empirical factors
\eqref{eqn:sim_study_params_reg_factor_threshold} with
% 1.) reference SNR
$\text{SNR}_{\text{dB}} = \SI{10}{\deci\bel}$ specified
% 2.) lower bounds on the l2-norms of the sensing matrices' column vectors
their lower bounds
\eqref{eqn:recovery_reg_norm_l2_norms_lb}.
% f) each computed TPSF was characterized by its FEHM for each index and its empirical CDF
In addition to
a visual inspection,
% 1.) transform point spread function (TPSF)
each computed \ac{TPSF}
\eqref{eqn:cs_math_tpsf} was characterized by
% 2.) full extent at half maximum (FEHM)
its \ac{FEHM} for
% 3.) each index n_{2} \in \setsymbol{I}
each index
$n_{2} \in \setsymbol{I}$ and
% 4.) empirical cumulative distribution function (CDF)
its empirical \ac{CDF}.
% g) empirical CDF excluded all n_{1} = n_{2} and stated the percentages of diverse pulse echoes whose correlation coefficient did not exceed a specified threshold
The latter excluded
% 1.) all identical pairs of indices
all $n_{1} = n_{2}$ and, thus, stated
% 2.) percentages of diverse pulse echoes
the percentages of
diverse pulse echoes whose
% 3.) correlation coefficient
correlation coefficient did not exceed
% 4.) specified threshold
a specified threshold.
