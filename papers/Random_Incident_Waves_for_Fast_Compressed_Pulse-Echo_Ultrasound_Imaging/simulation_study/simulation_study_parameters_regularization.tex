%---------------------------------------------------------------------------------------------------------------
% 1.) regularization
%---------------------------------------------------------------------------------------------------------------
% a) approximation of the estimated l2-norm of the normalized additive errors in the normalized CS problem
% 1.) variances of the zero-mean GWN
The variances
\eqref{eqn:sim_study_params_obs_errors_variance},
% 2.) number of relevant discrete frequencies (effective time-bandwidth product)
the effective time-bandwidth product
\eqref{eqn:recon_disc_axis_f_discrete_BP_TB_product}, and
% 3.) quantized recording time
the quantized recording time permitted
the approximation of
% 4.) estimated l2-norm of the normalized additive errors in the normalized CS problem
the estimated $\ell_{2}$-norm of
the normalized additive errors
\eqref{eqn:imp_data_acq_rel_obs_error_est} as
\begin{equation*}
 %--------------------------------------------------------------------------------------------------------------
 % approximation of the estimated l2-norm of the normalized additive errors in the normalized CS problem
 %--------------------------------------------------------------------------------------------------------------
  \hat{ \bar{\eta} }
  \approx
  \left[
    1
    +
    \frac{
      \norm{ \vect{u}^{(\text{B})} }{2}^{2}
      f_{\text{s}}^{(0)}
    }{
      \norm{ \vect{u}^{(\text{B}, \text{\acs{QPW}})} }{2}^{2}
      2 B_{ u }^{(0)}
    }
    10^{ \frac{ \text{SNR}_{\text{dB}} }{ 10 \si{\deci\bel} } }
  \right]^{ - \frac{1}{2} }.
\end{equation*}
% b) empirical threshold factors for the normalization of the sensing matrices
% 1.) QPW
% SNR_cs = [3, 6, 10, 20, 30, inf];		% specify SNR in dB
% norms_cols_thresh = 10.^(-SNR_cs / 20);	% thresholds for normalization according to actual SNR
% 2.) random incident waves
% SNR_cs_act = 10 * log10( data_RF_tgc_cs_power_mean ./ noise_RF_tgc_cs_variance );
% => SNR_cs_act = 10 * log10( norm( data_RF_tgc_cs(:) )^2 ./ norm( data_RF_tgc_qpw(:) )^2 ) + SNR_cs;
% norms_cols_thresh = 10.^(-SNR_cs_act / 20);	% thresholds for normalization according to actual SNR
% => norms_cols_thresh = norm( data_RF_tgc_qpw(:) ) ./ norm( data_RF_tgc_cs(:) ) * 10.^( - SNR_cs / 20 );
\TODO{exception: equal threshold for 3 and 6 dB}
For
each reference \ac{SNR},
% 1.) empirical threshold factors for the normalization of the sensing matrices
the empirical factors
\begin{equation}
 %--------------------------------------------------------------------------------------------------------------
 % empirical threshold factors for the normalization of the sensing matrices
 %--------------------------------------------------------------------------------------------------------------
  \xi
  =
  \frac{
    \norm{ \vect{u}^{(\text{B}, \text{\acs{QPW}})} }{2}
  }{
    \norm{ \vect{u}^{(\text{B})} }{2}
  }
  10^{ - \frac{ \text{SNR}_{\text{dB}} }{ 20 \si{\deci\bel} } }
 \label{eqn:sim_study_params_reg_factor_threshold}
\end{equation}
specified
% 2.) lower bounds on the l2-norms of the sensing matrices' column vectors
the lower bounds on
the $\ell_{2}$-norms of
the column vectors
\eqref{eqn:recovery_reg_norm_l2_norms_lb}.
% c) maximum number of iterations in SPGL1 was N_{\text{iter}}
% article:Schiffner2018, Sect. VI. Implementation / Sect. VI-C. Sparsity-Promoting lq-Minimization Method (subsec:imp_lq_minimization)
% - \acs{SPGL1} is ITERATIVE and left multiplied a sequence of recursively-generated vectors by
%   the potentially densely-populated normalized sensing matrix \eqref{eqn:recon_reg_norm_sensing_matrix} or its adjoint.
The maximum number of
iterations in
\ac{SPGL1} was
$N_{\text{iter}}$
(cf. \cref{tab:sim_study_parameters}(g)).
% d) normalization parameters \epsilon_{n} induced a sequence of five renormalized CS problems in Foucart's algorithm
% article:Schiffner2018, Sect. VI. Implementation / Sect. VI-C. Sparsity-Promoting lq-Minimization Method (subsec:imp_lq_minimization)
% - \name{Foucart}'s algorithm \cite[Sect. 4]{article:FoucartACHA2009} iteratively applied this method based on \ac{SPGL1} to
%   a sequence of RENORMALIZED \ac{CS} PROBLEMS to approximate the nonconvex $\ell_{q}$-minimization method \eqref{eqn:recovery_reg_norm_lq_minimization} for
%   the half-open parameter interval $q \in [ 0; 1 )$.
% - cs_2d_mlfma_options.q = 0.5;
% - cs_2d_mlfma_options.epsilon_n = 1 ./ (1 + (1:5)); [cs_2d_mlfma_options.epsilon_n = 1 ./ (2 + (0:4))]
The normalization parameters
$\epsilon_{n}$ induced
a sequence of
five renormalized \ac{CS} problems in
\name{Foucart}'s algorithm
(cf. \cref{subsec:imp_lq_minimization}).
% e) Foucart's algorithm entailed six l1 minimizations
Since
\ac{SPGL1} provided
the initial guess,
\name{Foucart}'s algorithm entailed
six $\ell_{1}$ minimizations.
