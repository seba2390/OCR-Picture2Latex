%---------------------------------------------------------------------------------------------------------------
% 1.) geometric and electromechanical parameters of the instrumentation
%---------------------------------------------------------------------------------------------------------------
% a) commercial linear transducer array was emulated in the two-dimensional Euclidean space to reduce the computational costs
A commercial linear transducer array was emulated in
% 1.) two-dimensional Euclidean space
the two-dimensional Euclidean space
(cf. \cref{tab:sim_study_parameters}(a)).
% b) kerfs of width zero simplified the implementation
The kerfs of
width zero simplified
the implementation.
% c) products -j \omega_{l} \rho_{0} h_{m, l}^{(\text{tx})} corresponded to a modulated Gaussian pulse in the time domain
% thesis:Schiffner2018
% pulse echoes: e_{m, n}^{(\text{B})} =  h_{m}^{(\text{rx})} [ - j \omega \rho_{0} h_{n}^{(\text{tx})} ] u_{n}^{(\text{tx})}
% (physical unit: $\tunit{ e_{m, n}^{(\text{B})} } = \si{ \volt \meter\tothe{ - \mathit{d} } }$)
%
% article:Schiffner2018, Sect. V. Image Recovery Based on Compressed Sensing / Sect. V-B. Computations of the Incident Acoustic Pressure Fields
% - The insertions of
%   the apodized spatial transmit functions \eqref{eqn:syn_sup_qsw_p_in_qsw_spat_trans} and
%   the discretized transmitter apodization functions \eqref{eqn:recovery_disc_space_trans_array_spat_trans_tx} into
%   the incident acoustic pressure fields \eqref{eqn:syn_sup_qsw_p_in} yield the discretizations
%   [ p_{l}^{(\text{in}, n)}( \vect{r}_{\text{lat}, i} ) = - j 2 \omega_{l} \rho_{0} \Delta A \sum_{ m = 0 }^{ N_{\text{el}} - 1 } h_{m, l}^{(\text{tx})} u_{m, l}^{(\text{tx}, n)} \sum_{ \nu = 0 }^{ N_{\text{mat}} - 1 } \chi_{m, \nu, l}^{(\text{tx})} g_{l}\bigl[ \vect{r}_{\text{lat}, i} - \vect{r}_{\text{mat}, \nu}^{(m)} \bigr] ]
%   for all $( n, l, i ) \in \setconsnonneg{ N_{\text{in}} - 1 } \times \setsymbol{L}_{ \text{BP} }^{(n)} \times \setconsnonneg{ N_{\text{lat}} - 1 }$, where
%   [...].
%
% MATLAB:
% f_tx = 4 * 1e6;		% transmit center frequency (Hz)
% f_s = 20 * 1e6;		% sampling rate (Hz)
% frac_bw = 0.7;		% fractional bandwidth of incident pulse
% frac_bw_ref = -60;		% dB value that determines frac_bw
%
% tc = gauspuls( 'cutoff', f_tx, frac_bw, frac_bw_ref, -60 );	% calculate cutoff time
% t = (-tc:1/f_s:tc);						% time axis
% impulse_response = gauspuls( t, f_tx, frac_bw, frac_bw_ref );
%
% excitation = sin(2*pi*f_tx*(0:1/f_s:1/f_tx)) * 0.5e5 * f_s;
% N_samples_A_in_td = numel(excitation) + numel(impulse_response) - 1;
% impulse_response_dft = fft( impulse_response, N_samples_A_in_td );
% A_in_td_dft = impulse_response_dft .* fft( excitation, N_samples_A_in_td );
% A_in_td = ifft( A_in_td_dft ) / f_s;
%
% cs_2d_mlfma_inverse_scattering_v16.m:
% A_in = fft( A_in_td, N_samples_t ) / ( sqrt( N_samples_t ) * f_s );
% A_in_analy_cropped = 2 * A_in( indicator_omega );
%
% p_incident_ref( :, :, index_f ) = 0.25j * A_in_analy_cropped(index_f) * besselh( 0, 2, axis_k_tilde(index_f) * D_ref_tx );
%
% compare with: -j 2 \omega_{l} \rho_{0} \Delta A h_{m, l}^{(\text{tx})} u_{m, l}^{(\text{tx}, n)} \chi_{m, \nu, l}^{(\text{tx})} g_{l}\bigl[ \vect{r}_{\text{lat}, i} - \vect{r}_{\text{mat}, \nu}^{(m)} \bigr]
% => A_in_analy_cropped(index_f) = -j 2 \omega_{l} \rho_{0} \Delta A h_{m, l}^{(\text{tx})} u_{m, l}^{(\text{tx}, n)} \chi_{m, \nu, l}^{(\text{tx})}
%
% set: \chi_{m, \nu, l}^{(\text{tx})} = 1
% neglect: 2 \Delta A
% => A_in_analy_cropped(index_f) = -j \omega_{l} \rho_{0} h_{m, l}^{(\text{tx})} u_{m, l}^{(\text{tx}, n)}
\TODO{unit in table}
The products
% 1.) h_{m, l}^{(\text{c})} = -j \omega_{l} \rho_{0} h_{m, l}^{(\text{tx})}
$h_{m, l}^{(\text{c})} = -j \omega_{l} \rho_{0} h_{m, l}^{(\text{tx})}$ in
% 2.) discretized incident acoustic pressure fields [superpositions of quasi-(d-1)-spherical waves]
the incident acoustic pressure fields
\eqref{eqn:recovery_p_in} corresponded to
% 3.) modulated Gaussian pulse in the time domain
a modulated Gaussian pulse in
the time domain.
