%---------------------------------------------------------------------------------------------------------------
% 1.) tissue-mimicking phantom
%---------------------------------------------------------------------------------------------------------------
% a) discrete Fourier basis defined the structural building blocks as complex exponential functions
The discrete \name{Fourier} basis defined
% 1.) structural building block
the structural building block with
% 2.) index n \in \setcons{ N_{\text{lat}} }
the index
$n \in \setcons{ N_{\text{lat}} }$ as
% 3.) complex exponential function with the normalized discrete lateral and axial frequencies
% index_1 = \ceil{ n / N_{\text{lat}, 2} }
% index_2 = n - ( index_1 - 1 ) * N_{\text{lat}, 2}
% \hat{K}_{ n, 1 } = ( \ceil{ N_{\text{lat}, 1} / 2 } - N_{\text{lat}, 1} + index_1 - 1 ) / N_{\text{lat}, 1} = ( \tceil{ n / 512 } - 257 ) / 512
% \hat{K}_{ n, 2 } = ( index_2 - 1 ) / N_{\text{lat}, 2} = ( n - ( index_1 - 1 ) * N_{\text{lat}, 2} - 1 ) / N_{\text{lat}, 2} = ( n + 511 ) / 512 - \tceil{ n / 512 }
the complex exponential function with
the normalized discrete lateral and
axial frequencies
$\hat{K}_{ n, 1 } = ( \tceil{ n / 512 } - 257 ) / 512$ and
$\hat{K}_{ n, 2 } = ( n + 511 ) / 512 - \tceil{ n / 512 }$,
respectively
(cf. \cref{tab:sim_study_parameters}(f)).
% b) random specification of 10 coefficients with respect to the discrete Fourier basis generated the relative spatial fluctuations in the unperturbed compressibility
% MATLAB:
% theta_kappa_abs_mu = 1e-1;
% theta_kappa = zeros( N_lattice_axis(2), N_lattice_axis(1) );
% theta_kappa( indices_k_kappa ) = theta_kappa_abs_mu .* exp( 2j * pi * rand( 1, N_coefficients_kappa ) );
% gamma_kappa = reshape( psi_fourier( N_lattice_axis, theta_kappa, 2, [] ), [N_lattice_axis(2), N_lattice_axis(1)] );
% gamma_kappa_abs_max = max( abs( gamma_kappa(:) ) );
% theta_kappa = theta_kappa / gamma_kappa_abs_max * 1e-1;
% gamma_kappa = gamma_kappa / gamma_kappa_abs_max * 1e-1;
Nonzero components of
% 1.) identical absolute value
identical absolute value and
% 2.) uniformly distributed phase
uniformly distributed phase in
% 3.) nearly-sparse representation
the sparse representation
\eqref{eqn:recovery_reg_sparse_representation} spawned
% 4.) vector stacking the regular samples in the discretized relative spatial fluctuations in the unperturbed compressibility
dense compressibility fluctuations
\eqref{eqn:recovery_sys_lin_eq_gamma_kappa_bp_vector}.
% c) typical absorption parameters for soft tissues governed the wavenumber
% article:JensenProgBMB2007: Medical ultrasound imaging
% 2. Basic ultrasound
% - Typically, an ATTENUATION OF 0.5 dB/(MHz cm) IS EXPERIENCED IN THE SOFT TISSUES. (p. 154)
% book:Duck1990, Chapter 4: Acoustic Properties of Tissue at Ultrasonic Frequencies / Sect. 4.3: Ultrasonic attenuation: absorption and scatter
% Sect. 4.3.8: Values of acoustic absorption coefficients in tissue
% - Measured values of ABSORPTION COEFFICIENTS FOR ULTRASOUND IN SOFT TISSUE are given in Tables 4.19 and 4.20. (p. 115)
% - Values at particular frequencies are included in Table 4.19, and
%   the POWER-LAW EXPRESSION Equation 4.30 used as the basis for the values given in Table 4.20. (p. 115)
% - Table 4.20: Ultrasound absorption coefficient (ii); \alpha = a f^{b} (p. 117)
Typical absorption parameters for
% 1.) soft tissues
soft tissues
\cite[Table 4.20]{book:Duck1990} governed
% 2.) complex-valued wavenumber with respect to k_{\text{ref}}
the wavenumber
\eqref{eqn:lin_mod_mech_model_tis_abs_time_causal_wavenumber_complex_kref}, where
% 3.) linear frequency dependence
the linear frequency dependence implied
% 4.) anomalous dispersion
anomalous dispersion.
% d) quantized recording time interval and the associated number of relevant discrete frequencies resulted in the ratio N_{\text{obs}} / N_{\text{lat}} \approx \SI{10.99}{\percent}
% T_{ \text{rec} }^{(0)} = 1607 T_{\text{s}}^{(0)}
% N_{f, \text{BP}}^{(0)} = 225
% N_{\text{lat}} = 262144
% => N_{\text{obs}} = 28800
% => N_{\text{obs}} / N_{\text{lat}} = 28800 / 262144 \approx 10.9863 %
The quantized recording time interval
\eqref{eqn:lin_mod_scan_config_volt_rx_obs_interval} and
% 1.) number of relevant discrete frequencies (effective time-bandwidth product)
the associated number of
relevant discrete frequencies
\eqref{eqn:recon_disc_axis_f_discrete_BP_TB_product} resulted in
% 2.) number of observations (all pulse-echo measurements, multifrequent, all array elements)
%the number of
%observations
%\eqref{eqn:recovery_sys_lin_eq_num_obs} of
the ratio
$N_{\text{obs}} / N_{\text{lat}} \approx \SI{10.99}{\percent}$.
