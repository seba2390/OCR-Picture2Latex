%---------------------------------------------------------------------------------------------------------------
% 1.) wire phantom
%---------------------------------------------------------------------------------------------------------------
% a) wires were represented by 21 identical nonzero components in the vector aggregating the regular samples in the discretized relative spatial fluctuations in the unperturbed compressibility
The wires were represented by
identical nonzero components in
% 1.) vector stacking the regular samples in the discretized relative spatial fluctuations in the unperturbed compressibility
the compressibility fluctuations
\eqref{eqn:recovery_sys_lin_eq_gamma_kappa_bp_vector}
(cf. \cref{tab:sim_study_parameters}(e)).
\TODO{Why does the simulation recover real vectors?!?}
% b) axial distances from the linear transducer array / axial and lateral spacings
Their axial distances from
% 1.) linear transducer array
the linear transducer array ranged from
% 2.) 5 - 37 mm
\SIrange{5}{37}{\milli\meter}, and
% 3.) axial and lateral spacings
their axial and
lateral spacings amounted to
% 4.) 5 mm and 10 mm
approximately \SI{5}{\milli\meter} and
\SI{10}{\milli\meter},
respectively.
% c) canonical basis defined the admissible structural building blocks as individual samples and induced a 21-sparse representation
The canonical basis defined
% 1.) structural building block
the structural building block with
% 2.) index n \in \setcons{ N_{\text{lat}} }
the index
$n \in \setcons{ N_{\text{lat}} }$ as
% 3.) individual sample located at the position \vect{r}_{ \text{lat}, n - 1 }
% index_1 = \ceil{ n / N_{\text{lat}, 2} }
% index_2 = n - ( index_1 - 1 ) * N_{\text{lat}, 2}
% r_{ \text{lat}, n, 1 } = ( index_1 - 513 / 2 ) * \Delta r_{\text{lat}, 1}
% r_{ \text{lat}, n, 2 } = ( index_2 - 0.5 ) * \Delta r_{\text{lat}, 1}
the individual sample located at
the position
$\vect{r}_{ \text{lat}, n - 1 } \in \mathcal{L}$ and induced
% 4.) nearly-sparse representation
a sparse representation
\eqref{eqn:recovery_reg_sparse_representation}.
% d) absorption parameters equaled those of pure water at a temperature of 20 °C
% book:Duck1990, Chapter 4: Acoustic Properties of Tissue at Ultrasonic Frequencies / Sect. 4.1.11: Acoustic velocity through some materials other than tissue
% Sect. 4.1.11.1: Water
% - The acoustic velocity in water is given in Table 4.8, including its temperature dependence. (p. 94)
% - Pure water non-dispersive. (p. 95)
% - Table 4.8: Acoustic velocity and attenuation, and non-linearity parameter B/A for pure water at atmospheric pressure (p. 95)
%   20°C | 1482.3 m/s | 25 * 1e-3 Np / ( m MHz^{2} ) | 4.96 B/A
The absorption parameters in
% 1.) complex-valued wavenumber with respect to k_{\text{ref}}
the wavenumber
\eqref{eqn:lin_mod_mech_model_tis_abs_time_causal_wavenumber_complex_kref} equaled
those of
% 2.) pure water at a temperature of 20 °C
pure water at
a temperature of
$\SI{20}{\celsius}$
\cite[Table 4.8]{book:Duck1990}, where
% 3.) quadratic frequency dependence
the quadratic frequency dependence prevented
% 4.) dispersion
dispersion.
% e) quantized recording time interval and the associated number of relevant discrete frequencies resulted in the number of observations of N_{\text{obs}} / N_{\text{lat}} \approx \SI{11.23}{\percent}
% T_{ \text{rec} }^{(0)} = 1647 T_{\text{s}}^{(0)}
% N_{f, \text{BP}}^{(0)} = 230
% N_{\text{lat}} = 262144
% => N_{\text{obs}} = 29440
% => N_{\text{obs}} / N_{\text{lat}} = 29440 / 262144 \approx 11.2305 %
The quantized recording time interval
\eqref{eqn:lin_mod_scan_config_volt_rx_obs_interval} and
% 1.) number of relevant discrete frequencies (effective time-bandwidth products)
the associated number of
relevant discrete frequencies
\eqref{eqn:recon_disc_axis_f_discrete_BP_TB_product} resulted in
% 2.) number of observations (all pulse-echo measurements, multifrequent, all array elements)
%the number of
%observations
%\eqref{eqn:recovery_sys_lin_eq_num_obs} of
the ratio
% indeterminancy
$N_{\text{obs}} / N_{\text{lat}} \approx \SI{11.23}{\percent}$.
