%---------------------------------------------------------------------------------------------------------------
% 1.) spatial discretizations
%---------------------------------------------------------------------------------------------------------------
% a) constant spacing between the adjacent grid points on each vibrating face along the r_{1}-axis ensured approximately \num{3.7} points per smallest wavelength
% independent parameters required for the coordinates of the grid points on each vibrating face ( d = 2, \delta = 1 ):
% 1.) N_{\text{mat}, 1} \in \N
%     [number of grid points per vibrating face along the r_{1}-axis]
% => \Delta r_{\text{mat}, 1} = w_{\text{el}, 1} / N_{\text{mat}, 1}
%    [constant spacing between the adjacent grid points on each vibrating face along the r_{1}-axis]
% => \mathcal{V}_{m} = \{ \vect{r}_{\text{mat}, \nu}^{(m)} \in \R^{2}: \vect{r}_{\text{mat}, \nu}^{(m)} = \vect{r}_{\text{el}, m} + ( \nu_{1} - M_{\text{mat}, 1} ) \Delta r_{\text{mat}, 1} \uvect{1}, \nu_{1} \in \setconsnonneg{ N_{\text{mat}, 1} - 1 }, \nu = \mathcal{I}( \vectsym{\nu}, \vect{N}_{\text{mat}} ) \}
%    [coordinates of the grid points on each vibrating face]
% => N_{\text{mat}} = \tabs{ \mathcal{V}_{m} } = N_{\text{mat}, 1}
%    [total number of grid points per vibrating face]
% => \Delta A = \Delta r_{\text{mat}, 1}
%    [(d-1)-dimensional surface element]
%
% number of grid points per vibrating face along the r_{1}-axis and its relationship to the wavelength:
% linear transducer array:
%   w_{\text{el}, 1} = 304.8 um | N_{\text{mat}, 1} = 4
% => \Delta r_{\text{mat}, 1} = w_{\text{el}, 1} / N_{\text{mat}, 1} = 304.8 um / 4 = 76.2 um
% wire phantom:
%   c_{l} = c_{\text{ref}} = 1500 m / s; f_{l} = 5.3916211293 MHz for l = l_{\text{ub}}^{(0)} = 444
%   => \lambda_{l} = c_{l} / f_{l} = 278.2095 um
%   => \lambda_{l} / 4 = 69.5524 um
% => \lambda_{l} / \Delta r_{\text{mat}, 1} = 3.6510
% tissue-mimicking phantom:
%   c_{l} = 1540.4124 m / s; f_{l} = 5.3889234599 MHz for l = l_{\text{ub}}^{(0)} = 433
%   => \lambda_{l} = c_{l} / f_{l} = 285.8479 um
%   => \lambda_{l} / 4 = 71.4620 um
% => \lambda_{l} / \Delta r_{\text{mat}, 1} = 3.7513
The constant spacing between
% 1.) adjacent grid points
the adjacent grid points on
% 2.) each vibrating face
each vibrating face along
% 3.) r_{1}-axis
the $r_{1}$-axis ensured
% 4.) approximately four grid points
approximately \num{3.7} points per
% 5.) smallest wavelength
% article:Schiffner2018, Sect. III. Linear Physical Model for the Pulse-Echo Measurement Process / Sect. B. Acoustic Model for Human Soft Tissues
% - Given reference values of
%   the angular frequency $\omega_{\text{ref}} \in \Rplus$ and
%   the associated phase velocity $c_{\text{ref}} \in \Rplus$, the complex-valued wavenumber \cite{article:WatersITUFFC2005,article:SzaboJASA1995}
%   [ \munderbar{k}_{l} = \frac{ \omega_{l} }{ c_{\text{ref}} } + \beta_{\text{E,ref}, l} - j \bar{b} \abs{ \omega_{l} }^{ \zeta } ]
%   [                     \beta_{l} = \omega_{l} / c_{l} ] where
%   the phase term $\beta_{l} \in \R$ sums the real-valued wavenumber $k_{\text{ref}, l} = \omega_{l} / c_{\text{ref}}$ and
%   the excess dispersion term [...] combines power-law absorption with dispersion.
smallest wavelength
(cf. \cref{tab:sim_study_parameters}(d)).
% b) FOV was square shaped and laterally centered in front of the linear transducer array
% independent parameters required for the coordinates of the grid points in the FOV ( d = 2, \delta \in \{ 1, 2 \} ):
% 1.) N_{\text{lat}, 1}, N_{\text{lat}, 2} \in \N
%     [number of grid points in the FOV along the r_{1}- and r_{2}-axes]
% 2.) \Delta r_{\text{lat}, 1}, \Delta r_{\text{lat}, 2} \in \Rplus
%     [constant spacing between the adjacent grid points in the FOV along the r_{1}- and r_{2}-axes]
% 3.) \vect{r}_{\text{lat}, 0} = \trans{ ( r_{\text{lat}, 0, 1}, r_{\text{lat}, 0, 2} ) } \in \R \times \Rplus
%     [arbitrary offset of the grid points in the FOV]
% => \mathcal{L} = \{ \vect{r}_{\text{lat}, i} \in \R^{2}: \vect{r}_{\text{lat}, i} = \vect{r}_{\text{lat}, 0} + \sum_{ \delta = 1 }^{ 2 } i_{\delta} \Delta r_{\text{lat}, \delta} \uvect{\delta}, i_{\delta} \in \setconsnonneg{ N_{\text{lat}, \delta} - 1 }, i = \mathcal{I}\left( \vect{i}, \vect{N}_{\text{lat}} \right) \}
%    [coordinates of the grid points in the FOV]
% => N_{\text{lat}} = \abs{ \mathcal{L} } = \prod_{ \delta = 1 }^{ 2 } N_{\text{lat}, \delta}
%    [total number of grid points in the FOV]
% => \Delta V = \prod_{ \delta = 1 }^{ 2 } \Delta r_{\text{lat}, \delta}
%    [d-dimensional volume element]
The \ac{FOV} was
% 1.) square shaped
square shaped and
% 2.) laterally centered
laterally centered in front of
% 3.) linear transducer array
the linear transducer array.
% c) symmetry enables simple computation of the incident field
The identical spacings between
the adjacent grid points on
each vibrating face and
in the \ac{FOV}, i.e.
$\Delta r_{\text{lat}, 1} = \Delta r_{\text{lat}, 2} = \Delta r_{\text{mat}, 1}$, simplified
the computations of
% 1.) discretized incident acoustic pressure fields [superpositions of quasi-(d-1)-spherical waves]
the incident acoustic pressure fields
\eqref{eqn:recovery_p_in} and
% 2.) implementation of the FMM
the implementation of
the \ac{FMM}.
