%---------------------------------------------------------------------------------------------------------------
% 1.) pulse-echo measurement process
%---------------------------------------------------------------------------------------------------------------
% a) single pulse-echo measurement was simulated for each type of incident wave
% article:Schiffner2018, Sect. III. Linear Physical Model for the Pulse-Echo Measurement Process / Sect. III.A. Pulse-Echo Measurement Process
% - The \ac{UI} system SEQUENTIALLY PERFORMS $N_{\text{in}} \in \N$ INDEPENDENT PULSE-ECHO MEASUREMENTS using a planar transducer array
%   (cf. \cref{fig:lin_mod_scan_configuration,tab:lin_mod_scan_config_instrum_params}).
A single pulse-echo measurement was simulated for
% 1.) each type of incident wave
each type of
incident wave
(cf. \cref{tab:sim_study_parameters}(b)).
% b) distinct recording time intervals were specified for the RF voltage signals recorded from each object
% article:Schiffner2018, Sect. III. Linear Physical Model for the Pulse-Echo Measurement Process / Sect. III.A. Pulse-Echo Measurement Process
% - Each measurement begins at the time instant $t = 0$ and triggers
%   the CONCURRENT RECORDING OF THE \ac{RF} VOLTAGE SIGNALS $\tilde{u}_{m}^{(\text{rx}, n)}: \setsymbol{T}_{ \text{rec} }^{(n)} \mapsto \R$ GENERATED BY
%   ALL ARRAY ELEMENTS $m \in \setconsnonneg{ N_{\text{el}} - 1 }$ in the SPECIFIED TIME INTERVAL
%   [ \setsymbol{T}_{ \text{rec} }^{(n)} = \bigl[ t_{\text{lb}}^{(n)}; t_{\text{ub}}^{(n)} \bigr], ] (eqn:lin_mod_scan_config_volt_rx_obs_interval) where
%   $t_{\text{lb}}^{(n)} \in \Rnonneg$ and $t_{\text{ub}}^{(n)} > t_{\text{lb}}^{(n)}$ denote
%   its lower and upper bounds, respectively.
% 1.) specified recording time interval for the RF voltage signals generated by all array elements
The object-specific recording time interval
\eqref{eqn:lin_mod_scan_config_volt_rx_obs_interval} was identical for
% 1.) all types of incident waves
all types of
incident waves.
% c) lower and upper frequency bounds were derived from the modulated Gaussian pulse
The lower and
upper frequency bounds were derived from
the modulated Gaussian pulse.
% d) relevant Fourier coefficients were determined at the approximate efficiency of \text{Efficiency}^{(0)} \approx \SI{28}{\percent}
The relevant \name{Fourier} coefficients were determined at
% 1.) approximate efficiency of the regular sampling in combination with the subsequent computation of the DFTs
the approximate efficiency
\eqref{eqn:imp_fourier_coef_efficiency} of
$\text{Efficiency}^{(0)} \approx \SI{28}{\percent}$.
