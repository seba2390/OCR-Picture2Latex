%---------------------------------------------------------------------------------------------------------------
% 1.) syntheses of the incident waves
%---------------------------------------------------------------------------------------------------------------
% a) reference voltage signal in the excitation voltages corresponded to a single period of a sinusoid
% article:Schiffner2018, Sect. IV. Syntheses of the Incident Waves / Sect. IV-B Types of Incident Waves (subsec:syn_p_in_types)
% - The generation of the excitation voltages typically applies quantized apodization weights $a_{m}^{(n)} \in \R$ and time delays $\Delta t_{m}^{(n)} \in \Rnonneg$ to
%   the REFERENCE VOLTAGE SIGNALS $u_{l}^{(\text{tx}, n)} \in \C$, whose
%   electric energies are constant for all $n \in \setconsnonneg{ N_{\text{in}} - 1 }$.
% - [...] the generated excitation voltages are
%   [ u_{m, l}^{(\text{tx}, n)} = u_{l}^{(\text{tx}, n)} a_{m}^{(n)} e^{ -j \omega_{l} \mathcal{Q} \left[ \Delta t_{m}^{(n)} \right] } ] (eqn:syn_p_in_types_v_tx_expression)
%   for all $( n, l, m ) \in \setconsnonneg{ N_{\text{in}} - 1 } \times \setsymbol{L}_{ \text{BP} }^{(n)} \times \setconsnonneg{ N_{\text{el}} - 1 }$ [...].
%
% MATLAB:
% excitation = sin( 2 * pi * f_tx * (0:1/f_s:1/f_tx) ) * 0.5e5 * f_s;
% N_samples_A_in_td = numel(excitation) + numel(impulse_response) - 1;
% impulse_response_dft = fft(impulse_response, N_samples_A_in_td);
% A_in_td_dft = impulse_response_dft .* fft(excitation, N_samples_A_in_td);
% A_in_td = ifft(A_in_td_dft) / f_s;
The reference voltage signal in
% 1.) excitation voltages (single pulse-echo measurement, monofrequent, single array element)
the excitation voltages
\eqref{eqn:syn_p_in_types_v_tx} corresponded to
% 2.) single period of a sinusoid
a single period of
a sinusoid, whose amplitude was
% 3.) irrelevant
irrelevant in
% 4.) LTI measurement process
the \ac{LTI} measurement process
(cf. \cref{tab:sim_study_parameters}(c)).
% b) frequency of the clock signal in the quantization operator matched the specifications of a commercial UI system
% article:ChengITUFFC2006: Extended high-frame rate imaging method with limited-diffraction beams
% V. In Vitro and In Vivo Experiments
% - For steered plane wave or conventional delay-and-sum methods,
%   linear time delays are applied to the transducers to steer the beams. (p. 890)
% - The precision of the time delay of the system is 6.25 ns, which is determined by a 160 MHz clock. (p. 890)
The frequency of
% 1.) clock signal
the clock signal in
% 2.) quantization operator providing the admissible time delays
the quantization operator
\eqref{eqn:syn_p_in_types_v_tx_quantization} matched
% 3.) specifications of commercial UI systems
the specifications of
commercial \ac{UI} systems.
% c) steered QPW propagated preferentially along the r_{2}-axis
The steered \ac{QPW} propagated preferentially along
the $r_{2}$-axis.
% d) fixed time period permuting the time delays for a steered QPW
% article:Schiffner2018, Sect. IV-B.3) Superpositions of Randomly-Delayed Quasi-(d-1)-Spherical Waves (subsubsec:syn_p_in_types_rnd_del)
% - In the two-dimensional Euclidean space, i.e. $d = 2$, they simplify to
%   $T_{\text{inc}}^{(n)} = \Delta r_{\text{el}, 1} \tabs{ \uvectcomp{ \vartheta }{ 1 }^{(n)} } / c_{\text{ref}}$ and induce
%   RANDOM PERMUTATIONS OF THE TIME DELAYS SPECIFIED FOR THE STEERED \acp{QPW} in \eqref{eqn:syn_p_in_types_v_tx_qpw}.
%
% => \tabs{ \uvectcomp{ \vartheta }{ 1 }^{(0)} } = T_{\text{inc}}^{(0)} c_{\text{ref}} / \Delta r_{\text{el}, 1}
% => \cos( \vartheta_{0} ) = T_{\text{inc}}^{(0)} c_{\text{ref}} / \Delta r_{\text{el}, 1} [ \cos( \vartheta_{0} ) >= 0 ]
% => \vartheta_{0} = \acos( T_{\text{inc}}^{(0)} c_{\text{ref}} / \Delta r_{\text{el}, 1} )
%
% wire phantom:
% c_{\text{ref}} = \SI{1500}{\meter\per\second}, T_{\text{inc}}^{(0)} = \SI{43.6369}{\nano\second}
% => \vartheta_{0} \approx 77.5992°
% tissue-mimicking phantom:
% c_{\text{ref}} = \SI{1540}{\meter\per\second}, T_{\text{inc}}^{(0)} = \SI{42.5035}{\nano\second}
% => \vartheta_{0} \approx 77.5992°
The fixed time period
\eqref{eqn:syn_p_in_types_v_tx_rnd_del_interval} emerged from
% 1.) preferred direction of propagation
the direction
$\uvect{\vartheta}^{(0)} = \trans{ ( \cos( \vartheta ), \sin( \vartheta ) ) }$.
