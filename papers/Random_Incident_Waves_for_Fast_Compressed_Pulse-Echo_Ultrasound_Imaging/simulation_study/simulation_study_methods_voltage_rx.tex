%---------------------------------------------------------------------------------------------------------------
% 1.) recorded radio frequency voltage signals
%---------------------------------------------------------------------------------------------------------------
% a) ten realizations of the recorded RF voltage signals were derived from their Born approximation for each type of incident wave and each SNR
% d) number of recovery experiments
 %The number of
 %recovery experiments conducted for
 %each reference \ac{SNR} and
 %each type of
 %incident wave amounted to
 %$N_{\text{rcn}} = 10$.
Ten realizations of
%($N_{\text{rcn}} = 10$) of
% 1.) vectors stacking the relevant Fourier coefficients of the recorded RF voltage signals (all pulse-echo measurements, multifrequent, all array elements)
the recorded \ac{RF} voltage signals
\eqref{eqn:recovery_sys_lin_eq_v_rx_born_all_f_all_in_v_rx} were derived from
% 2.) approximate vectors stacking the relevant Fourier coefficients of the recorded RF voltage signals (all pulse-echo measurements, multifrequent, all array elements)
their \name{Born} approximation
\eqref{eqn:recovery_sys_lin_eq_v_rx_born_all_f_all_in_v_rx_born} for
% 3.) each type of incident wave
each type of
incident wave and
% 4.) each SNR
each \ac{SNR} by inserting
% 5.) nearly-sparse representation / vector of transform coefficients
the sparse representation
\eqref{eqn:recovery_reg_sparse_representation} and
% 6.) additive errors of the specified energy levels
the additive errors into
% 7.) linear algebraic system (all pulse-echo measurements, multifrequent, all array elements, additive errors)
the linear algebraic system
\eqref{eqn:recovery_reg_prob_general_obs_trans_coef_error}.
