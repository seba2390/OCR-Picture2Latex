%---------------------------------------------------------------------------------------------------------------
% 1.) reference sensing matrices
%---------------------------------------------------------------------------------------------------------------
% a) two types of sensing matrices served as benchmarks
Two types of
sensing matrices, which emerged from
\ac{GWN}, served as
benchmarks.
% b) first reference observation process and the associated sensing matrix met the RIP with very high probability
% article:Schiffner2018, Sect. II. Compressed Sensing in a Nutshell (sec:compressed_sensing)
% - Fortunately, CERTAIN TYPES OF RANDOM SENSING MATRICES \eqref{eqn:cs_math_prob_general_sensing_matrix} also obey
%   THE \ac{RIP} WITH VERY HIGH PROBABILITY, if
%   THE NUMBER OF OBSERVATIONS IS SUFFICIENTLY LARGE \cite[6]{book:Foucart2013}, \cite{article:TroppPIEEE2010}.
% - REALIZATIONS OF \ac{IID} RANDOM VARIABLES governed by certain distributions, e.g.
%   GAUSSIAN or \name{Bernoulli}, as entries \cite[Thm. 5.2]{article:BaraniukCA2008} and
%   randomly and uniformly chosen scaled rows of a \name{Fourier} basis \cite[Thm. 3.3]{article:RudelsonCPAM2008}, for example, require
%   $M \in \bigomega{ s \ln( N / s ) }$ and
%   $M \in \bigomega{ s \ln^{4}( N ) }$ observations, respectively.
% article:BaraniukCA2008: A Simple Proof of the Restricted Isometry Property for Random Matrices (real-valued CS problem / canonical basis)
% 6 Discussion
% - Furthermore, we prove above that the RIP HOLDS for \Phi(ω) WITH HIGH PROBABILITY when
%   the matrix is drawn according to one of the distributions
%   [ \phi_{ i, j } \sim \gaussian{ 0 }{ 1 / n } ] (4.4),
%   [ \phi_{ i, j } := + 1 / \sqrt{n} with probability 0.5; - 1 / \sqrt{n} with probability 0.5 ] (4.5), or
%   [ \phi_{ i, j } := + \sqrt{ 3 / n } with probability 1/6; 0 with probability 2/3; - \sqrt{ 3 / n } with probability 1/6 ] (4.6) []. (p. 261)
For
a sufficiently large
% 1.) number of observations (all pulse-echo measurements, multifrequent, all array elements)
number of
observations
\eqref{eqn:recovery_sys_lin_eq_num_obs}, both
% 2.) first reference observation process (RIP)
the real-valued random
$N_{\text{obs}} \times N_{\text{lat}}$ observation process
\begin{subequations}
\begin{align}
 %--------------------------------------------------------------------------------------------------------------
 % first reference observation process and its entries (RIP)
 %--------------------------------------------------------------------------------------------------------------
  \mat{\Phi}^{(\text{\acs{RIP}})}
  =
  \vertcat_{ m = 1 }^{ N_{\text{obs}} }
    \horzcat_{ i = 1 }^{ N_{\text{lat}} }
      \phi_{ m, i }^{(\text{\acs{RIP}})},
  & &
  \phi_{ m, i }^{(\text{\acs{RIP}})}
  \underset{ \text{\acs{IID}} }{ \sim }
  \dgaussian{ 0 }{ \frac{ 1 }{ N_{\text{obs}} } }{1}
 \label{eqn:sim_study_params_ref_obs_proc_rip}
\end{align}
and
% 3.) first reference sensing matrix (RIP)
the associated complex-valued
$N_{\text{obs}} \times N_{\text{lat}}$ sensing matrix
\begin{equation}
 %--------------------------------------------------------------------------------------------------------------
 % first reference sensing matrix (RIP)
 %--------------------------------------------------------------------------------------------------------------
  \mat{A}^{(\text{\acs{RIP}})}
  =
  \mat{\Phi}^{(\text{\acs{RIP}})}
  \mat{\Psi}
 \label{eqn:sim_study_params_ref_sens_mat_rip}
\end{equation}
\end{subequations}
met
% 4.) restricted isometry property (RIP)
the \ac{RIP} with
very high probability
(cf. \cref{sec:compressed_sensing}).
% c) specified variance ensured recorded electric energies of unity expectation
The specified variance ensured
% 1.) recorded electric energies in the pulse echoes (all pulse-echo measurements, multifrequent, all array elements)
recorded electric energies
\eqref{eqn:recovery_reg_v_rx_born_trans_coef_energy} of
% 2.) unity expectation
unity expectation.
% d) replacement of the incident acoustic pressure field in the observation process by complex-valued GWN additionally formed the observation process
The replacement of
% 1.) discretized incident acoustic pressure fields [superpositions of quasi-(d-1)-spherical waves]
the incident acoustic pressure field
\eqref{eqn:recovery_p_in} in
% 2.) observation process (all pulse-echo measurements, multifrequent, all array elements)
the observation process
\eqref{eqn:recovery_sys_lin_eq_v_rx_born_all_f_all_in_mat} by
% 3.) complex-valued GWN (realizations of i.i.d. complex-valued Gaussian random variables)
% MATLAB:
% p_incident_theta_act{index_f} = (randn( N_lattice_axis(2), N_lattice_axis(1) ) + 1j * randn( N_lattice_axis(2), N_lattice_axis(1)) ) * 1e-5;
%$\treal{ p_{l}^{(\text{in}, 0)}( \vect{r}_{\text{lat}, i} ) } \sim \gaussian{ 0 }{ 1 }$
%$\timag{ p_{l}^{(\text{in}, 0)}( \vect{r}_{\text{lat}, i} ) } \sim \gaussian{ 0 }{ 1 }$
complex-valued \ac{GWN} additionally formed
% 4.) second reference observation process (GWN)
the complex-valued structured
$N_{\text{obs}} \times N_{\text{lat}}$ observation process
\TODO{complex GWN}
\begin{subequations}
\begin{align}
 %--------------------------------------------------------------------------------------------------------------
 % second reference observation process and its entries (GWN)
 %--------------------------------------------------------------------------------------------------------------
  \mat{\Phi}^{(\text{\acs{GWN}})}
  =
  \mat{\Phi}\bigl[ p^{(\text{in})} \bigr],
  & &
  p_{l}^{(\text{in}, 0)}( \vect{r}_{\text{lat}, i} )
  \underset{ \text{\acs{IID}} }{ \sim }
  \gaussian{ 0 }{ 1 }
 \label{eqn:sim_study_params_ref_obs_proc_gwn}
\end{align}
%for
% 1.) all admissible frequency indices and all grid points
%$( l, i ) \in \setsymbol{L}_{ \text{BP} }^{(0)} \times \setconsnonneg{ N_{\text{lat}} - 1 }$.
and
% 5.) second reference sensing matrix (GWN)
the associated complex-valued
$N_{\text{obs}} \times N_{\text{lat}}$ sensing matrix
\begin{equation}
 %--------------------------------------------------------------------------------------------------------------
 % second reference sensing matrix (GWN)
 %--------------------------------------------------------------------------------------------------------------
  \mat{A}^{(\text{\acs{GWN}})}
  =
  \mat{\Phi}^{(\text{\acs{GWN}})}
  \mat{\Psi}.
 \label{eqn:sim_study_params_ref_sens_mat_gwn}
\end{equation}
\end{subequations}
% e) complex-valued GWN violated the Helmholtz equations
Although
% 1.) complex-valued GWN
the complex-valued \ac{GWN} violated
% 2.) Helmholtz equations for the incident acoustic pressure fields
the \name{Helmholtz} equations
\eqref{eqn:lin_mod_sol_wave_eq_pde_p_in},
% f) replacement correctly respected the monopole scattering and the reception by the instrumentation
this replacement correctly respected
% 2.) monopole scattering
the monopole scattering and
% 3.) reception by the instrumentation
the reception by
the instrumentation.
