%---------------------------------------------------------------------------------------------------------------
% change format for equation numbers
%---------------------------------------------------------------------------------------------------------------
\renewcommand{\theequation}{A.\arabic{equation}}

%---------------------------------------------------------------------------------------------------------------
% 1.) outgoing free-space Green's functions (two- and three-dimensional Euclidean spaces)
%---------------------------------------------------------------------------------------------------------------
% a) outgoing free-space Green's functions uniquely solve the fundamental inhomogeneous Helmholtz equations subject to the SRCs
% book:Devaney2012, Chapter 2: Radiation and boundary-value problems in the frequency domain / Sect. 2.2: Green functions / Sect. 2.2.1: Green functions in two space dimensions
% - The actual computation of the 2D Green functions is left as a problem at the end of the chapter. One finds that
%   [ G_{+}( \vect{R}, \omega ) = - \frac{ j }{ 4 } H_{0}^{+}( k R ), ] (2.19a)
%   where H_{0}^{+}( . ) is the zeroth-order Hankel function of the first kind. (p. 53)
% book:Devaney2012, Chapter 2: Radiation and boundary-value problems in the frequency domain / Sect. 2.2: Green functions
% - The SRC is equivalent to the requirement of causality in the time domain and the resulting Green function is the outgoing-wave Green function
%   [ G_{+}( \vect{R}, \omega ) = - \frac{ 1 }{ 4 \pi } \frac{ e^{ j k R } }{ R }, ] (2.14)
%   derived by Fourier transformation of the causal time-domain Green function^{2} (retarded Green function)
%   g_{+}(  \vect{r} -  \vect{r}', t - t' ) of the wave equation in Section 1.2.2 of Chapter 1. (p. 51)
% book:Devaney2012, Chapter 1: Radiation and initial-value problems for the wave equation / Sect. 1.2: Green functions / Sect. 1.2.2: Frequency-domain Green functions
% - Using the latter procedure [temporal Fourier transform] we obtain
%   [ G_{+}( \vect{R}, \omega ) = - \frac{ 1 }{ 4 \pi } \frac{ e^{ j k R } }{ R } ] (1.23a)
%   with k = ω/c. (pp. 9, 10)
% book:Natterer2001, Chapter 3: Tomography / Sect. 3.3: Diffraction Tomography
% - For n = 2, we have
%   [ G_{2}( x ) = \frac{ j }{ 4 } H_{0}( k \abs{x} ) ] (3.14)
%   with H_{0} the zero order Hankel function of the first kind, and
%   [ G_{3}( x ) = \frac{ e^{ j k \abs{x} } }{ 4 \pi \abs{x} } ] (3.15)
%   for n = 3. (p. 47)
The outgoing free-space \name{Green}'s functions
%(unit: $\si{\meter}^{2-d}$)
(cf. e.g.
\cite[(2.14) and (2.19)]{book:Devaney2012},
\cite[(3.14) and (3.15)]{book:Natterer2001}%
)
\begin{equation}
 %--------------------------------------------------------------------------------------------------------------
 % outgoing free-space Green's functions (two- and three-dimensional Euclidean spaces)
 %--------------------------------------------------------------------------------------------------------------
  g_{l}( \vect{r} )
  =
  \begin{cases}
   %------------------------------------------------------------------------------------------------------------
   % a) two-dimensional Euclidean space
   %------------------------------------------------------------------------------------------------------------
    j \hankel{0}{2}{ \munderbar{k}_{l} \norm{ \vect{r} }{2} } / 4
    & \text{for } d = 2,\\
   %------------------------------------------------------------------------------------------------------------
   % b) three-dimensional Euclidean space
   %------------------------------------------------------------------------------------------------------------
    -
    e^{ - j \munderbar{k}_{l} \norm{ \vect{r} }{2} } / ( 4 \pi \norm{ \vect{r} }{2} )
    & \text{for } d = 3,
  \end{cases}
 \label{eqn:app_helmholtz_green_free_space_2_3_dim}
\end{equation}
uniquely solve
% 1.) fundamental inhomogeneous Helmholtz equations (d-dimensional Euclidean space)
the fundamental inhomogeneous \name{Helmholtz} equations
\begin{equation*}
 %--------------------------------------------------------------------------------------------------------------
 % fundamental inhomogeneous Helmholtz equations (d-dimensional Euclidean space)
 %--------------------------------------------------------------------------------------------------------------
  \left( \Delta + {\munderbar{k}_{l}}^{2} \right)
  g_{l}( \vect{r} )
  =
  \delta( \vect{r} )
 \label{eqn:app_helmholtz_fund_n_dim}
\end{equation*}
subject to
% 2.) Sommerfeld radiation conditions (d-dimensional Euclidean space)
% book:Devaney2012, Chapter 2: Radiation and boundary-value problems in the frequency domain / Sect. 2.1: Frequency-domain formulation of the radiation problem / Sect. 2.1.4: The Sommerfeld radiation condition in dispersive media
% - The SRC can be stated in either of the two forms (cf. Eqs. (1.45a) and (1.48))
%   [ \lim_{ r \rightarrow \infty } r \left[ \partial{ U_{+}( \vect{r}, \omega ) }{ r } - j k U_{+}( \vect{r}, \omega ) \right] \rightarrow 0, ] (2.8a)
%   [ U_{+}( \vect{r}, \omega ) \sim f( \vect{s}, \omega ) \frac{ e^{ j k r } }{ r }, ] (2.8b)
%   where \vect{s} = \vect{r} / r is the unit vector along the \vect{r} direction and, as usual,
%   we have used the subscript + to denote the field that satisfies the SRC. (p. 48)
% book:Devaney2012, Chapter 1: Radiation and initial-value problems for the wave equation / Sect. 1.5: Frequency-domain solution of the radiation problem / Sect. 1.5.1: The radiation pattern and the Sommerfeld radiation condition
% - The asymptotic expression Eq. (1.45a) is one form of the famed SOMMERFELD RADIATION CONDITION (SRC). (p. 24)
% - An alternative, and the most often quoted, form of the SRC is given by
%   [ \lim_{ r \rightarrow \infty } r \left[ \partial{ U_{+}( \vect{r}, \omega ) }{ r } - j k U_{+}( \vect{r}, \omega ) \right] \rightarrow 0, ] (1.48)
%   with a similar expression holding for G_{+}. (p. 24)
% - The equivalence of the two forms of the SRC is easily established. (p. 24)
% book:Natterer2001
the \acp{SRC}
(cf. e.g.
\cite[(1.48) or (2.8)]{book:Devaney2012},		% complex-valued k, 3-dimensional, neg. sign convention
\cite[(7.61)]{book:Natterer2001}%			% real-valued k, n-dimensional, neg. sign convention, existence of unique solution is mentioned explicitly
)
\begin{equation}
 %--------------------------------------------------------------------------------------------------------------
 % Sommerfeld radiation conditions (d-dimensional Euclidean space)
 %--------------------------------------------------------------------------------------------------------------
  \underset{ r \rightarrow \infty }{ \lim }
  \underset{ \norm{ \vect{r} }{2} = r }{ \max }
    \norm{ \vect{r} }{2}^{ \frac{ d - 1 }{ 2 } }
    \bigl[
      \inprod{ \nabla g_{l}( \vect{r} ) }{ \uvect{r}( \vect{r} ) }
      +
      j \munderbar{k}_{l} g_{l}( \vect{r} )
    \bigr]
  = 0
 \label{eqn:app_helmholtz_src}
\end{equation}
for
% 3.) all relevant discrete frequencies
all $l \in \setsymbol{L}_{ \text{BP} }^{(n)}$, where
% 4.) zero-order Hankel function of the second kind
% book:OlverNHMF2010, §10.4 Connection Formulas
% - [ \hankel{\nu}{2}{ z } = \bessel{\nu}{ z } - j \neumann{\nu}{ z } ] (10.4.3)
% book:OlverNHMF2010, §10.2(ii) Standard Solutions / Bessel Functions of the Third Kind (Hankel Functions)
% - These solutions of (10.2.1) [Bessel’s Equation] are denoted by Hν(1)⁡(z) and Hν(2)⁡(z), and their defining properties are given by 10.2.5 [...] and 10.2.6 [...].
$\hankelsymbol{0}{2}$ denotes
the zero-order \name{Hankel} function of
the second kind
\cite[§10.2(ii) and 10.4.3]{book:OlverNHMF2010},
% 5.) Dirac delta distribution
$\delta$ indicates
the \name{Dirac} delta distribution, and
% 6.) radial unit vector
$\uvect{r}( \vect{r} ) = \vect{r} / \tnorm{ \vect{r} }{2}$ for
all $\vect{r} \in \R^{d} \setminus \{ \vect{0} \}$ is
the radial unit vector.
% b) SRCs account for a lossy homogeneous fluid of infinite extent and ensure the causality in the time domain
% book:Devaney2012, Chapter 2: Radiation and boundary-value problems in the frequency domain / Sect. 2.1: Frequency-domain formulation of the radiation problem / Sect. 2.1.4: The Sommerfeld radiation condition in dispersive media
% - The choice of boundary conditions is dictated by the physics of the problem at hand, and for the case of
%   the RADIATION PROBLEM IN AN INFINITE HOMOGENEOUS BACKGROUND MEDIUM the appropriate boundary condition is
%   the SOMMERFELD RADIATION CONDITION (SRC) (Sommerfeld, 1967). (p. 47)
% - The SRC was discussed at some length in Chapter 1 and, as shown in that chapter, is equivalent to
%   the requirement that the associated TIME-DOMAIN FIELD AND GREEN FUNCTION BE CAUSAL. (pp. 47, 48)
% - In the traditional form of the SRC the wavenumber k = ω/c is a real-valued quantity and the SRC defines an outgoing-wave field. (p. 48)
% - In the case of DISPERSIVE MEDIA where k is complex with a positive imaginary part
%   the SRC is still equivalent to causality in the time domain and still requires that
%   the radiated field behave as an outgoing spherical wave as the field point r tends to infinity, but
%   this condition also requires that its amplitude decay exponentially with increasing distance r from the origin. (p. 48)
% - This, of course, is a requirement that is imposed by the fact that \imag{ k } > 0, corresponding to
%   an ABSORBING BACKGROUND MEDIUM that attenuates any propagating wave. (p. 48)
% book:Devaney2012, Chapter 1: Radiation and initial-value problems for the wave equation / Sect. 1.2: Green functions / Sect. 1.2.2: Frequency-domain Green functions
% - Like the inhomogeneous wave equation satisfied by the time-domain Green function Eq. (1.16),
%   the inhomogeneous Helmholtz equation Eq. (1.22) does not possess a unique solution until
%   an APPROPRIATE BOUNDARY CONDITION IS APPENDED. (p. 9)
% - The REQUIREMENT OF CAUSALITY IN THE TIME DOMAIN yields
%   [1.)] a boundary condition in the frequency domain known as the SOMMERFELD RADIATION CONDITION (SRC) (Sommerfeld, 1967) and
%   [2.)] a Green function denoted by G_{+}( \vect{R}, \omega ) that is generally referred to as the “outgoing-wave” Green function for reasons to be discussed below. (p. 9)
The \acp{SRC}
\eqref{eqn:app_helmholtz_src} account for
% 1.) lossy homogeneous fluid of infinite extent
a lossy homogeneous fluid of
infinite extent and ensure
% 2.) causality in the time domain
the causality in
the time domain
\cite[Sect. 2.1.4]{book:Devaney2012}.
