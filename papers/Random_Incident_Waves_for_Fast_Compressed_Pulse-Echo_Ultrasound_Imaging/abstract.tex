%---------------------------------------------------------------------------------------------------------------
% 1.) state of the art in fast pulse-echo ultrasound imaging
%---------------------------------------------------------------------------------------------------------------
% a) established image recovery methods in fast UI trade the image quality for the high frame rate
Established image recovery methods in
fast \acl{UI}, e.g.
% 1.) delay-and-sum
\acl{DAS}, trade
% 2.) image quality
the image quality for
% 3.) high frame rate
the high frame rate.
\TODO{model inaccuracies}
% b) cutting-edge inverse scattering methods based on CS disrupt this tradeoff via a priori information
Cutting-edge inverse scattering methods based on
% 1.) compressed sensing
\ac{CS} disrupt
% 2.) tradeoff
this tradeoff via
% 3.) a priori information
\emph{a priori} information.
% c) cutting-edge inverse scattering methods based on CS iteratively recover a high-quality image from only a few sequential pulse-echo measurements or less echo signals
They iteratively recover
% 1.) high-quality image
a high-quality image from
% 2.) only a few sequential pulse-echo measurements
only a few sequential pulse-echo measurements or
% 3.) less echo signals
less echo signals, if
% 4.) known dictionary of structural building blocks represents the image almost sparsely
(i) a known dictionary of
structural building blocks represents
the image almost sparsely, and
% 5.) individual pulse echoes are sufficiently uncorrelated
(ii) their individual pulse echoes, which are predicted by
a linear model, are
sufficiently uncorrelated.
% d) exclusive modeling of the incident waves as steered PWs or cylindrical waves has so far limited the convergence speed and the image quality by violating condition (ii)
The exclusive modeling of
% 1.) incident waves
the incident waves as
% 2.) steered PWs
steered \aclp{PW} or
% 3.) outgoing 1-spherical waves (cylindrical waves)
cylindrical waves, however, has so far limited
% 3.) convergence speed
the convergence speed,
% 4.) image quality
the image quality, and
%by violating
% 5.) potential
the potential to meet
% 6.) condition (ii) [ individual pulse echoes are sufficiently uncorrelated ]
condition (ii).

%---------------------------------------------------------------------------------------------------------------
% 2.) contributions of the paper
%---------------------------------------------------------------------------------------------------------------
% a) novel method for the fast compressed acquisition and the subsequent recovery of images is proposed to overcome these limitations
Motivated by
% 1.) benefits
the benefits of
% 2.) randomness
randomness in
% 3.) compressed sensing (CS)
\ac{CS},
% 4.) novel method
a novel method for
% 5.) fast compressed acquisition
the fast compressed acquisition and
% 6.) subsequent recovery
the subsequent recovery of
images is proposed to overcome
% 6.) limitations [convergence speed and image quality]
these limitations.
% b) novel method recovers the spatial compressibility fluctuations in weakly-scattering soft tissue structures by a sparsity-promoting lq-minimization method
It recovers
% 1.) spatial compressibility fluctuations
the spatial compressibility fluctuations in
% 2.) weakly-scattering soft tissue structures
weakly-scattering soft tissue structures, where
% 3.) orthonormal basis represents the image almost sparsely
an orthonormal basis meets
% 4.) condition (i) [known dictionary of structural building blocks represents the image almost sparsely]
condition (i), by
% 5.) sparsity-promoting lq-minimization method
a sparsity-promoting $\ell_{q}$-minimization method,
% 5.) parameter q \in [ 0; 1 ]
$q \in [ 0; 1 ]$.
% c) realistic d-dimensional model predicts the pulse echoes of the individual basis functions
A realistic $d$-dimensional model, $d \in \{ 2, 3 \}$, accounting for
% 1.) diffraction
diffraction,
% 2.) single monopole scattering within the first Born approximation
single monopole scattering,
% 3.) combination of power-law absorption and dispersion
the combination of
power-law absorption and
dispersion, and
% 4.) specifications of a planar transducer array
the specifications of
a planar transducer array, predicts
% 5.) pulse echoes
the pulse echoes of
% 6.) individual basis functions
the individual basis functions.
% d) three innovative types of incident waves aid in meeting condition (ii)
Three innovative types of
% 1.) incident waves
incident waves, whose
% 2.) syntheses
syntheses leverage
% 3.) random apodization weights
random apodization weights,
% 4.) random time delays
time delays, or
% 5.) combinations thereof
combinations thereof, aid in meeting
% 9.) condition (ii) [ individual pulse echoes are sufficiently uncorrelated ]
condition (ii).

%---------------------------------------------------------------------------------------------------------------
% 3.) results of the numerical simulations
%---------------------------------------------------------------------------------------------------------------
% a) single realizations of the random waves outperform the prevalent QPW for both the canonical and the Fourier bases
In
% 1.) two-dimensional numerical simulations
two-dimensional numerical simulations,
% 2.) single realizations
single realizations of
% 3.) random waves
these waves outperform
% 4.) quasi-plane wave (QPW)
the prevalent \acl{QPW} for both
% 5.) canonical basis
the canonical and
% 6.) Fourier basis
the \name{Fourier} bases.
% b) single realizations of the random waves significantly decorrelate the pulse echoes
% article:Schiffner2018, Sect. VIII. Results / Sect. VIII-B. Tissue-Mimicking Phantom / Sect. VIII-B.2) Transform Point Spread Functions (subsubsec:results_phantom_tissue_tpsf)
% - In fact, the MAXIMUM NORMALIZED DIFFERENCES ranged from
%   \SI{41.2}{\percent} at the fixed spatial frequencies $s \in \{ 1, 9 \}$ to
%   \SI{62.5}{\percent} at the fixed spatial frequencies $s \in \{ 4, 6 \}$.
% article:Schiffner2018, Sect. VIII. Results / Sect. VIII-A. Wire Phantom / Sect. VIII-A.2) Point Spread Functions (subsubsec:results_phantom_wire_psf)
% - The MAXIMUM NORMALIZED DIFFERENCES ranged from
%   \SI{23.5}{\percent} for the superposition of both randomly-apodized and randomly-delayed \acp{QCW} at the ninth fixed position, i.e. $s = 9$, to
%   \SI{73.7}{\percent} for the superposition of randomly-delayed \acp{QCW} at the first fixed position, i.e. $s = 1$.
They significantly decorrelate
% 1.) pulse echoes
the pulse echoes, e.g.
% 2.) single realizations
they reduce
% 3.) full extents at half maximum (FEHMs)
the \aclp{FEHM} of
% 4.) point spread functions (PSFs)
the \aclp{PSF} by
% 5.) up to 73.7 %
up to \SI{73.7}{\percent}.
% c) single realizations of the random waves improve the convergence speed and the image quality in terms of the mean SSIM indices and the relative RMSEs
For
% 1.) tissue-mimicking phantom
a tissue-mimicking phantom and
% 2.) parameter q \in \{ 1, 0.5 \}
$q \in \{ 1, 0.5 \}$,
% 3.) single realizations of the random waves
they improve
% 4.) convergence speed
% article:Schiffner2018, Sect. VIII. Results / Sect. VIII-B. Tissue-Mimicking Phantom / Sect. VIII-B.4) Recovery by lq-Minimization (subsubsec:results_phantom_tissue_lq_minimization)
% - For all reference \acp{SNR} except $\text{SNR}_{\text{dB}} = \SI{30}{\deci\bel}$,
%   the sample means of the normalized numbers of iterations drastically exceeded those for
%   the random waves by up to \SI{22.9}{\percent}.
% MATLAB:
% N_iter_max = 1127.6
% N_iter_spgl1_l1_qpw(1,:)*1e2/N_iter_max	  = 6.4473    7.8131    9.3650   11.3427   13.1607   20.9294
% N_iter_spgl1_lq_qpw(1,:)*1e2/N_iter_max	  = 36.4580   45.4949   54.5938   62.7527   66.1848   87.7971
%
% N_iter_spgl1_l1_rnd_apo(1,:)*1e2/N_iter_max	  = 5.3210    5.9507    7.3430   11.3249   27.1639   44.3420
% N_iter_spgl1_lq_rnd_apo(1,:)*1e2/N_iter_max	  = 29.4342   29.1327   33.0259   45.8762  100.0000  111.9191
%
% N_iter_spgl1_l1_rnd_del(1,:)*1e2/N_iter_max	  = 5.4363    5.6669    6.6690    9.9858   22.8716   88.6839
% N_iter_spgl1_lq_rnd_del(1,:)*1e2/N_iter_max	  = 29.5495   28.6360   31.7311   42.7279   88.5154  167.0805
%
% N_iter_spgl1_l1_rnd_apo_del(1,:)*1e2/N_iter_max = 5.2412    5.7201    6.9262    9.6932   25.2128   88.6839
% N_iter_spgl1_lq_rnd_apo_del(1,:)*1e2/N_iter_max = 30.0550   29.5938   33.0702   41.0518   94.5371  157.5913
%
% differences to QPW (q = 1):
% rnd. apo.:		1.1263    1.8624    2.0220    0.0177  -14.0032  -23.4126
% rnd. del.:		1.0110    2.1462    2.6960    1.3569   -9.7109  -67.7545
% rnd. apo. del.:	1.2061    2.0929    2.4388    1.6495  -12.0521  -67.7545
% minimum: -14.0032 (rnd. apo. @ 30 dB)
% maximum:   2.6960 (rnd. del. @ 10 dB)
%
% differences to QPW (q = 0.5):
% rnd. apo.:		7.0238   16.3622   21.5679   16.8766  -33.8152  -24.1220
% rnd. del.:		6.9085   16.8588   22.8627   20.0248  -22.3306  -79.2834
% rnd. apo. del.:	6.4030   15.9010   21.5236   21.7010  -28.3523  -69.7943
% minimum: -33.8152 (rnd. apo. @ 30 dB)
% maximum:  22.8627 (rnd. del. @ 10 dB)
the convergence speed and
% 5.) image quality
% article:Schiffner2018, Sect. VIII. Results / Sect. VIII-B. Tissue-Mimicking Phantom / Sect. VIII-B.4) Recovery by lq-Minimization (subsubsec:results_phantom_tissue_lq_minimization)
% - Using the NONCONVEX $\ell_{0.5}$-minimization method \eqreflqmin{eqn:recovery_reg_norm_lq_minimization}{ 0.5 },
%   the RANDOM WAVES consistently achieved
%   MEAN \ac{SSIM} INDICES CLOSE TO UNITY and
%   RELATIVE \acp{RMSE} BELOW \SI{6.9}{\percent} for
%   all reference \acp{SNR}.
the image quality in terms of
% 6.) mean SSIM indices
% MATLAB:
% SSIM_index_spgl1_l1_qpw(1,:)*1e2	   = 36.6000   41.2488   46.4364   90.4708   99.9325   99.9368
% SSIM_index_spgl1_lq_qpw(1,:)*1e2	   = 43.6973   49.3404   56.1582   93.0138   99.8934   99.8323
%
% SSIM_index_spgl1_l1_rnd_apo(1,:)*1e2	   = 80.6549   81.5283   83.8071   86.1310   86.2028   86.1152
% SSIM_index_spgl1_lq_rnd_apo(1,:)*1e2	   = 99.1838   99.7243   99.8458   99.9954   99.9958   99.9934
%
% SSIM_index_spgl1_l1_rnd_del(1,:)*1e2	   = 80.2673   82.3483   84.2717   86.3884   90.6166   78.3656
% SSIM_index_spgl1_lq_rnd_del(1,:)*1e2	   = 98.7991   99.7009   99.8962   99.9935   99.9957   99.9928
%
% SSIM_index_spgl1_l1_rnd_apo_del(1,:)*1e2 = 80.2278   81.4986   85.3900   85.2345   86.2158   80.0985
% SSIM_index_spgl1_lq_rnd_apo_del(1,:)*1e2 = 98.6458   99.6973   99.8945   99.9930   99.9962   99.9940
%
% differences to QPW (q = 1):
% rnd. apo.:		44.0549   40.2794   37.3707   -4.3397  -13.7297  -13.8217
% rnd. del.:		43.6673   41.0994   37.8353   -4.0823   -9.3159  -21.5713
% rnd. apo. del.:	43.6279   40.2498   38.9537   -5.2362  -13.7167  -19.8383
% minimum: -13.7297 (rnd. apo. @ 30 dB)
% maximum:  44.0549 (rnd. apo. @ 3 dB)
%
% differences to QPW (q = 0.5):
% rnd. apo.:		55.4865   50.3839   43.6877    6.9816    0.1024    0.1611
% rnd. del.:		55.1018   50.3605   43.7381    6.9797    0.1023    0.1605
% rnd. apo. del.:	54.9485   50.3569   43.7363    6.9792    0.1029    0.1618
% minimum:  0.1023 (rnd. del. @ 30 dB)
% maximum: 55.4865 (rnd. apo. @ 3 dB)
the mean \acl{SSIM} indices and
% 7.) relative RMSEs
% MATLAB:
% rel_RMSE_spgl1_l1_qpw(1,:)*1e2	 = 74.9216   69.2613   63.4289   20.9912    1.3762    2.0671
% rel_RMSE_spgl1_lq_qpw(1,:)*1e2         = 65.8391   60.6768   54.3489   16.3019    1.4052    2.8359
%
% rel_RMSE_spgl1_l1_rnd_apo(1,:)*1e2	 = 32.8928   31.8203   29.1923   26.2924   26.1477   26.3357
% rel_RMSE_spgl1_lq_rnd_apo(1,:)*1e2	 =  5.3555    3.0675    2.0639    0.4661    0.3106    0.6444
%
% rel_RMSE_spgl1_l1_rnd_del(1,:)*1e2	 = 33.2994   31.0622   28.7238   26.0569   20.5849   36.2633
% rel_RMSE_spgl1_lq_rnd_del(1,:)*1e2	 =  6.5540    3.2644    1.7489    0.5210    0.3171    0.6548
%
% rel_RMSE_spgl1_l1_rnd_apo_del(1,:)*1e2 = 33.2734   31.8067   26.8410   27.4907   26.2436   32.9772
% rel_RMSE_spgl1_lq_rnd_apo_del(1,:)*1e2 =  6.8596    3.2637    1.7899    0.5512    0.2990    0.6206
%
% differences to QPW (q = 1):
% rnd. apo.:		42.0289   37.4409   34.2366   -5.3012  -24.7715  -24.2686
% rnd. del.:		41.6222   38.1990   34.7052   -5.0657  -19.2087  -34.1962
% rnd. apo. del.:	41.6482   37.4546   36.5880   -6.4995  -24.8674  -30.9101
% minimum: -24.8674 (rnd. apo. del. @ 30 dB)
% maximum:  42.0289 (rnd. apo. @ 3 dB)
%
% differences to QPW (q = 0.5):
% rnd. apo.:		60.4836   57.6093   52.2851   15.8357    1.0946    2.1916
% rnd. del.:		59.2850   57.4124   52.6000   15.7809    1.0882    2.1811
% rnd. apo. del.:	58.9795   57.4131   52.5590   15.7507    1.1062    2.2153
% minimum:  1.0882 (rnd. del. @ 30 dB)
% maximum: 60.4836 (rnd. apo. @ 3 dB)
the relative \aclp{RMSE} by
% 8.) up to 2.7 and 22.9 %
up to
$\{ \SI{2.7}{\percent}, \SI{22.9}{\percent} \}$,
% 9.) up to 44.1 and 55.5 %
$\{ \SI{44.1}{\percent}, \SI{55.5}{\percent} \}$, and
% 10.) up to 42 and 60.5 %
$\{ \SI{42}{\percent}, \SI{60.5}{\percent} \}$,
respectively.
%---------------------------------------------------------------------------------------------------------------
% 4.) results of the experiments
%---------------------------------------------------------------------------------------------------------------
% a) experiment with a wire phantom validates the feasibility of the proposed method
%An experiment with
%a wire phantom validates
%the feasibility of
%the proposed method.
