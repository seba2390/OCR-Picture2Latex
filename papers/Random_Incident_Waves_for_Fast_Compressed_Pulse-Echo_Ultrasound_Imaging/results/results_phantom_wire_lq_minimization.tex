%%%%%%%%%%%%%%%%%%%%%%%%%%%%%%%%%%%%%%%%%%%%%%%%%%%%%%%%%%%%%%%%%%%%%%%%%%%%%%%%%%%%%%%%%%%%%%%%%%%%%%%%%%%%%%%%
% images: wire phantom (lq-minimization, single pulse-echo measurement, ref. SNR: 30 dB)
%%%%%%%%%%%%%%%%%%%%%%%%%%%%%%%%%%%%%%%%%%%%%%%%%%%%%%%%%%%%%%%%%%%%%%%%%%%%%%%%%%%%%%%%%%%%%%%%%%%%%%%%%%%%%%%%
\graphictwocols{results/object_A/kappa_only/sr_1/normalize_true/figures/latex/sim_study_obj_A_sr_1_images.tex}%
{% a) figure illustrates the absolute values of the recovered relative spatial fluctuations in the unperturbed compressibility
 Absolute values of
 % 1.) recovered relative spatial fluctuations in the unperturbed compressibility
 the recovered compressibility fluctuations
 \eqref{eqn:recovery_reg_norm_lq_minimization_sol_mat_params} for
 % 2.) quasi-plane wave (QPW)
 the \acl{QPW}\acused{QPW}
 (cf.
 \subref{fig:sim_study_obj_A_sr_1_images_spgl1_l1_qpw} and
 \subref{fig:sim_study_obj_A_sr_1_images_spgl1_lq_qpw}%
 ) and
 the superpositions of
 % 3.) superposition of randomly-apodized QCWs
 randomly-apodized \acfp{QCW}\acused{QCW}
 (cf.
 \subref{fig:sim_study_obj_A_sr_1_images_spgl1_l1_rnd_apo} and
 \subref{fig:sim_study_obj_A_sr_1_images_spgl1_lq_rnd_apo}%
 ),
 % 4.) superposition of randomly-delayed QCWs
 randomly-delayed \acp{QCW}
 (cf.
 \subref{fig:sim_study_obj_A_sr_1_images_spgl1_l1_rnd_del} and
 \subref{fig:sim_study_obj_A_sr_1_images_spgl1_lq_rnd_del}%
 ), and both
 % 5.) superposition of both randomly-apodized and randomly-delayed QCWs
 randomly-apodized and
 randomly-delayed \acp{QCW}
 (cf.
 \subref{fig:sim_study_obj_A_sr_1_images_spgl1_l1_rnd_apo_del} and
 \subref{fig:sim_study_obj_A_sr_1_images_spgl1_lq_rnd_apo_del}%
 ).
 % b) top row shows the results of the convex l1-minimization method
 The top row
 (cf.
 \subref{fig:sim_study_obj_A_sr_1_images_spgl1_l1_qpw} to
 \subref{fig:sim_study_obj_A_sr_1_images_spgl1_l1_rnd_apo_del}%
 ) shows
 the results of
 % 1.) convex l1-minimization method
 the convex $\ell_{1}$-minimization method
 \eqreflqmin{eqn:recovery_reg_norm_lq_minimization}{ 1 }, whereas
 % c) bottom row shows the results of the nonconvex l0.5-minimization method
 the bottom row
 (cf.
 \subref{fig:sim_study_obj_A_sr_1_images_spgl1_lq_qpw} to
 \subref{fig:sim_study_obj_A_sr_1_images_spgl1_lq_rnd_apo_del}%
 ) shows
 those of
 % 1.) nonconvex l0.5-minimization method
 the nonconvex $\ell_{0.5}$-minimization method
 \eqreflqmin{eqn:recovery_reg_norm_lq_minimization}{ 0.5 }.
 % d) large images represent the nonzero absolute values by crosshairs of proportional gray values and sizes
 The large images represent
 the nonzero absolute values by
 crosshairs of
 proportional gray values and
 sizes, whereas
 % e) inset images exclusively use gray values to magnify the regions indicated by the white squares
 the inset images exclusively use
 gray values to magnify
 the regions indicated by
 the white squares.
 % f) reference SNR amounted to \text{SNR}_{\text{dB}} = \SI{30}{\deci\bel}
 The reference \ac{SNR} amounted to
 $\text{SNR}_{\text{dB}} = \SI{30}{\deci\bel}$.
}%
{sim_study_obj_A_sr_1_images_kap}

%---------------------------------------------------------------------------------------------------------------
% 1.) visual inspection of the recovered images (lq-minimization, single pulse-echo measurement, ref. SNR: 30 dB)
%---------------------------------------------------------------------------------------------------------------
% a) all incident waves enabled both the accurate detection and the precise localization of the wires
All incident waves enabled both
% 1.) accurate detection
the accurate detection and
% 2.) precise localization
the precise localization of
% 3.) wires
the wires
(cf. \cref{fig:sim_study_obj_A_sr_1_images_kap}).
% b) spatial extents recovered by the convex l1-minimization method were smaller for the random waves than for the QPW
% article:Schiffner2018, Sect. VIII. Results / Sect. VIII-A. Wire Phantom / Sect. VIII-A.2) Point Spread Functions (subsubsec:results_phantom_wire_psf)
% - The random waves achieved \acp{FEHM} that were smaller than or equal to
%   those of the \ac{QPW} for all fixed positions, except those numbered $s \in \{ 6, 7 \}$ (cf. \cref{tab:sim_study_obj_A_sr_1_tpsf_fehm}).
The spatial extents recovered by
% 1.) convex l1-minimization method
the convex $\ell_{1}$-minimization method
\eqreflqmin{eqn:recovery_reg_norm_lq_minimization}{ 1 } were
% 2.) smaller
smaller for
% 3.) random waves
the random waves than for
% 4.) quasi-plane wave (QPW)
the \ac{QPW}.
% c) reductions were more pronounced for smaller axial coordinates and approximately followed the trend of the normalized differences in the FEHMs of the PSFs
% article:Schiffner2018, Sect. VIII. Results / Sect. VIII-A. Wire Phantom / Sect. VIII-A.2) Point Spread Functions (subsubsec:results_phantom_wire_psf)
% - The \acp{FEHM} generally increased with the axial coordinate of these positions.
% - The maximum normalized differences ranged from
%   \SI{23.5}{\percent} for the superposition of both randomly-apodized and randomly-delayed \acp{QCW} at the NINTH FIXED POSITION, i.e. $s = 9$, to
%   \SI{73.7}{\percent} for the superposition of randomly-delayed \acp{QCW} at the FIRST FIXED POSITION, i.e. $s = 1$.
These reductions were more pronounced for
% 1.) smaller axial coordinates
smaller axial coordinates and, thus, approximately followed
% 2.) trend
the trend of
% 3.) normalized differences in the FEHMs
% 1: QPW (max)              2: QPW (max)             3: QPW (max)         4: QPW (max)              5: QPW (max)              6: rnd. del. (max)   7: rnd. apo. (max)   8: QPW (max)              9: QPW = rnd. apo. (max)
%    rnd. del.:      73.68%    rnd. apo.:     50%       rnd. del.: 57.14%    rnd. del.:      54.17%    rnd. del.:      50%       Rnd. apo.: 45.45%    rnd. del.: 54.55%    rnd. apo. del.: 33.33%    rnd. apo. del.: 23.53%
%    rnd. apo. del.: 42.11%    rnd. apo. del: 16.67%    rnd. apo.: 33.33%    rnd. apo. del.: 20.83%    rnd. apo. del.: 17.86%    QPW:        3.03%    QPW:       13.63%    rnd. apo.:      14.29%    rnd. del.:      9.8%
the normalized differences in
the \acp{FEHM} of
% 4.) point spread functions (PSFs)
the \acp{PSF}
\eqref{eqn:cs_math_tpsf}
(cf. \cref{tab:sim_study_obj_A_sr_1_tpsf_fehm}).
% d) random waves caused less artifacts near the wires than the QPW
Moreover,
% 1.) random waves
the random waves caused less
% 2.) artifacts
artifacts near
% 3.) wires
the wires than
% 3.) quasi-plane wave (QPW)
the \ac{QPW}
(cf. inset images).
% e) both advantages agreed with the reduced numbers of components within the illustrated dynamic range relative to the QPW
Both advantages agreed with
% 1.) reduced numbers
the reduced numbers of
% 2.) components within the illustrated dynamic range
components within
the illustrated dynamic range relative to
% 2.) quasi-plane wave (QPW)
the \ac{QPW}.
% f) normalized differences ranged from 58.4 to 66.4 %
% MATLAB:
% indicator_qpw = sim_obj_A_data_spgl1_l1_qpw{5}.gamma_kappa_recon_dB >= -70;
% indicator_rnd_apo = sim_obj_A_data_spgl1_l1_rnd_apo{5}.gamma_kappa_recon_dB >= -70;
% indicator_rnd_del = sim_obj_A_data_spgl1_l1_rnd_del{5}.gamma_kappa_recon_dB >= -70;
% indicator_rnd_apo_del = sim_obj_A_data_spgl1_l1_rnd_apo_del{5}.gamma_kappa_recon_dB >= -70;
% sum( indicator_qpw(:) ) = 1070
% sum( indicator_rnd_apo(:) ) = 445 	=> 58.4112 %
% sum( indicator_rnd_del(:) ) = 360 	=> 66.3551 %
% sum( indicator_rnd_apo_del(:) ) = 417 => 61.0280 %
The normalized differences ranged from
% 1.) 58.4 %
\SI{58.4}{\percent} for
% 2.) superposition of randomly-apodized QCWs
the superposition of
randomly-apodized \acp{QCW} to
% 3.) 66.4 %
\SI{66.4}{\percent} for
% 4.) superposition of randomly-delayed QCWs
the superposition of
randomly-delayed \acp{QCW}.
% g) nonconvex l0.5-minimization method consistently recovered isolated significant components that strongly resembled
In contrast,
% 1.) nonconvex l0.5-minimization method
the nonconvex $\ell_{0.5}$-minimization method
\eqreflqmin{eqn:recovery_reg_norm_lq_minimization}{ 0.5 } consistently recovered
% 2.) isolated components
isolated components that matched
% 3.) specified vector stacking the regular samples in the discretized relative spatial fluctuations in the unperturbed compressibility
the specified compressibility fluctuations
\eqref{eqn:recovery_sys_lin_eq_gamma_kappa_bp_vector}.
% h) numbers of components within the illustrated dynamic range equaled the number of wires
% MATLAB:
% indicator_qpw = sim_obj_A_data_spgl1_lq_qpw{5}.gamma_kappa_recon_dB >= -70;
% indicator_rnd_apo = sim_obj_A_data_spgl1_lq_rnd_apo{5}.gamma_kappa_recon_dB >= -70;
% indicator_rnd_del = sim_obj_A_data_spgl1_lq_rnd_del{5}.gamma_kappa_recon_dB >= -70;
% indicator_rnd_apo_del = sim_obj_A_data_spgl1_lq_rnd_apo_del{5}.gamma_kappa_recon_dB >= -70;
% sum( indicator_qpw(:) ) = 21
% sum( indicator_rnd_apo(:) ) = 36	=> -71.4286 %
% sum( indicator_rnd_del(:) ) = 21	=> 0 %
% sum( indicator_rnd_apo_del(:) ) = 22	=> -4.7619 %
The numbers of
% 1.) components within the illustrated dynamic range
components within
the illustrated dynamic range equaled
% 2.) number of wires
the number of
% 3.) wires
wires.


%%%%%%%%%%%%%%%%%%%%%%%%%%%%%%%%%%%%%%%%%%%%%%%%%%%%%%%%%%%%%%%%%%%%%%%%%%%%%%%%%%%%%%%%%%%%%%%%%%%%%%%%%%%%%%%%
% graphic: statistics of the lq-minimization (mean SSIM indices, rel. RMSEs, and number of iterations vs. SNR)
%%%%%%%%%%%%%%%%%%%%%%%%%%%%%%%%%%%%%%%%%%%%%%%%%%%%%%%%%%%%%%%%%%%%%%%%%%%%%%%%%%%%%%%%%%%%%%%%%%%%%%%%%%%%%%%%
\graphictwocols{results/object_A/kappa_only/sr_1/normalize_true/figures/latex/sim_study_obj_A_sr_1_mean_ssim_index_rel_rmse_N_iter_vs_snr.tex}%
{% a) figure illustrates the sample means and the sample standard deviations of the mean SSIM indices, the relative RMSEs, and the normalized numbers of iterations
 Sample means and
 sample standard deviations of
 % 1.) mean SSIM indices
 the mean \acf{SSIM}\acused{SSIM} indices and
 % 2.) relative RMSEs
 the relative \acfp{RMSE}\acused{RMSE} achieved by
 % 3.) recovered relative spatial fluctuations in the unperturbed compressibility
 the recovered compressibility fluctuations
 \eqref{eqn:recovery_reg_norm_lq_minimization_sol_mat_params} and
 % 4.) normalized numbers of iterations
 the normalized numbers of
 iterations in
 \ac{SPGL1}.
 % b) assignment of all incident waves and both parameters q to the columns and rows
 The assignment of
 all incident waves and
 both parameters
 $q \in \{ 0.5; 1 \}$ governing
 % 1.) sparsity-promoting lq-minimization method
 the sparsity-promoting $\ell_{q}$-minimization method
 \eqref{eqn:recovery_reg_norm_lq_minimization} to
 the columns and rows in
 this figure equals
 that in
 \cref{fig:sim_study_obj_A_sr_1_images_kap}.
 % c) dashed red lines indicate the reference SNR
 The dashed red lines indicate
 the reference \ac{SNR} selected for
 \cref{fig:sim_study_obj_A_sr_1_images_kap}.
 % d) reference value for normalization
 The maximum sample mean of
 $\num{1348.6}$ normalized
 the numbers of
 iterations.
}%
{sim_study_obj_A_sr_1_ssim_index_rel_rmse_N_iter_vs_snr_kap}

%---------------------------------------------------------------------------------------------------------------
% 2.) statistics of the image recovery (lq-minimization, single pulse-echo measurement, all ref. SNRs)
%---------------------------------------------------------------------------------------------------------------
% a) mean SSIM indices confirmed the excellent structural recovery
% article:Schiffner2018, Sect. VIII. Results / Sect. VIII-A. Wire Phantom / Sect. VIII-A.4) Recovery by lq-Minimization (subsubsec:results_phantom_wire_lq_minimization)
% - All incident waves enabled both the ACCURATE DETECTION and the PRECISE LOCALIZATION of the wires
%   (cf. \cref{fig:sim_study_obj_A_sr_1_images_kap}).
The mean \ac{SSIM} indices confirmed
% 1.) excellent structural recovery
the excellent structural recovery, whereas
% b) relative RMSEs revealed an increased sensitivity of the quantitative recovery using the random waves to the energy of the additive errors
the relative \acp{RMSE} revealed
% 1.) increased sensitivity
an increased sensitivity of
% 2.) quantitative recovery
the quantitative recovery using
% 3.) random waves
the random waves to
% 4.) energy
the energy of
% 5.) additive errors
the additive errors
(cf. \cref{fig:sim_study_obj_A_sr_1_ssim_index_rel_rmse_N_iter_vs_snr_kap}).
% c) all incident waves achieved mean SSIM indices close to unity and comparable trends in both the relative RMSEs and the numbers of iterations [l1-minimization]
All incident waves achieved
% 1.) mean SSIM indices close to unity
% MATLAB:
% SSIM_index_spgl1_l1_qpw(1,:)*1e2	   = 95.6015   94.9718   97.2248   97.7469   98.3861   99.0972
% SSIM_index_spgl1_l1_rnd_apo(1,:)*1e2	   = 88.4886   94.0421   93.8590   98.2373   98.9030   99.2924
% SSIM_index_spgl1_l1_rnd_del(1,:)*1e2	   = 95.2499   95.6733   96.3136   97.2284   98.8742   99.1607
% SSIM_index_spgl1_l1_rnd_apo_del(1,:)*1e2 = 96.1631   96.4825   96.0857   96.6423   98.6328   99.5759
% minimum: 88.4886 (rnd. apo. @ 3 dB)
% maximum: 98.9030 (rnd. apo. @ 30 dB)
mean \ac{SSIM} indices close to
unity and
% 2.) comparable trends
comparable trends in both
% 3.) relative RMSEs
the relative \acp{RMSE} and
% 4.) numbers of iterations
the numbers of
iterations for
% 5.) all reference SNRs
all reference \acp{SNR} and
% 6.) convex l1-minimization method
the convex $\ell_{1}$-minimization method
\eqreflqmin{eqn:recovery_reg_norm_lq_minimization}{ 1 }.
% d) sample means of the relative RMSEs decreased from at most 87.6 % to at least 34.1 % [l1-minimization]
% MATLAB:
% rel_RMSE_spgl1_l1_qpw(1,:)*1e2         = 85.3469   82.1552   58.9715   55.1754   47.6686   39.1278
% rel_RMSE_spgl1_l1_rnd_apo(1,:)*1e2	 = 87.5926   74.9386   73.5188   43.6958   36.4024   36.3388
% rel_RMSE_spgl1_l1_rnd_del(1,:)*1e2	 = 80.0539   69.9126   61.3011   43.5096   34.0714   40.0679
% rel_RMSE_spgl1_l1_rnd_apo_del(1,:)*1e2 = 74.5985   68.7890   61.5668   52.7007   40.0753   31.3463
% minimum: 34.0714 (rnd. del. @ 30 dB)
% maximum: 87.5926 (rnd. apo. @ 3 dB)
The sample means of
% 1.) relative RMSEs
the relative \acp{RMSE} decreased from
% 2.) at most 87.6 %
at most \SI{87.6}{\percent} for
% 3.) superposition of randomly-apodized QCWs
the superposition of
randomly-apodized \acp{QCW} at
% 4.) lowest reference SNR
the lowest reference \ac{SNR} to
% 5.) at least 34.1 %
at least \SI{34.1}{\percent} for
% 6.) superposition of randomly-delayed QCWs
the superposition of
randomly-delayed \acp{QCW} at
% 7.) highest reference SNR
the highest reference \ac{SNR}.
% e) sample means of the normalized numbers of iterations increased from at least 3.4 % to at most 28.3 % [l1-minimization]
% MATLAB:
% N_iter_max = 1348.6
% N_iter_spgl1_l1_qpw(1,:)*1e2/N_iter_max	  = 3.4406    3.8188    8.3494   15.4605   24.1436   39.7449
% N_iter_spgl1_l1_rnd_apo(1,:)*1e2/N_iter_max	  = 7.2149    7.0888    8.4680   12.8355   23.0535   38.1878
% N_iter_spgl1_l1_rnd_del(1,:)*1e2/N_iter_max	  = 4.7827    6.8071    7.7339   14.7560   25.8194   40.1898
% N_iter_spgl1_l1_rnd_apo_del(1,:)*1e2/N_iter_max = 6.5401    7.2297    9.5284   14.3111   28.3034   62.8059
% minimum:  3.4406 (QPW @ 3 dB)
% maximum: 28.3034 (rnd. apo. del. @ 30 dB)
Concurrently,
% 1.) sample means
the sample means of
% 2.) normalized numbers of iterations
the normalized numbers of
iterations increased from
% 3.) at least 3.4 %
at least \SI{3.4}{\percent} for
% 4.) quasi-plane wave (QPW)
the \ac{QPW} to
% 5.) at most 28.3 %
at most \SI{28.3}{\percent} for
% 6.) superposition of both randomly-apodized and randomly-delayed QCWs
the superposition of both
randomly-apodized and
randomly-delayed \acp{QCW}.
% f) sample standard deviations of the relative RMSEs exceeded those of the mean SSIM indices and the normalized numbers of iterations [l1-minimization]
% MATLAB:
% rel_RMSE_spgl1_l1_qpw(2,:)*1e2         = 2.4453    0.4627   14.5006   11.6007   15.9304         0
% rel_RMSE_spgl1_l1_rnd_apo(2,:)*1e2	 = 2.1477    2.0824    3.4309   10.2053    4.9393         0
% rel_RMSE_spgl1_l1_rnd_del(2,:)*1e2	 = 2.1170    4.3834    7.2373   17.8836    8.9579         0
% rel_RMSE_spgl1_l1_rnd_apo_del(2,:)*1e2 = 1.9915    1.1288    5.9226    8.8835    8.3409         0
% minimum: 0.4627 (QPW @ 6 dB)
% maximum: 17.8836 (rnd. del. @ 20 dB)
The sample standard deviations of
% 1.) relative RMSEs
the relative \acp{RMSE} exceeded
% 2.) sample standard deviations
those of
% 3.) mean SSIM indices
% MATLAB:
% SSIM_index_spgl1_l1_qpw(2,:)*1e2         = 2.0646    1.0759    1.5117    1.0335    0.7981         0
% SSIM_index_spgl1_l1_rnd_apo(2,:)*1e2	   = 3.5880    1.2707    1.8291    1.1469    0.2225         0
% SSIM_index_spgl1_l1_rnd_del(2,:)*1e2	   = 1.1922    1.1272    0.8742    2.6744    0.4605         0
% SSIM_index_spgl1_l1_rnd_apo_del(2,:)*1e2 = 1.2267    0.5903    1.7911    1.6325    0.7683         0
% minimum: 0.2225 (rnd. apo. @ 30 dB)
% maximum: 3.5880 (rnd. apo. @ 3 dB)
the mean \ac{SSIM} indices and
% 4.) normalized numbers of iterations
% MATLAB:
% N_iter_max = 1348.6
% N_iter_spgl1_l1_qpw(2,:)*1e2/N_iter_max	  = 0.8340    0.2645    1.8444    1.8935    2.2622         0
% N_iter_spgl1_l1_rnd_apo(2,:)*1e2/N_iter_max	  = 2.6984    0.6147    1.1228    1.2296    2.4624         0
% N_iter_spgl1_l1_rnd_del(2,:)*1e2/N_iter_max	  = 1.1652    1.2819    1.1766    1.8879    3.9669         0
% N_iter_spgl1_l1_rnd_apo_del(2,:)*1e2/N_iter_max = 0.6866    0.9701    1.2873    1.3484    3.1884         0
% minimum: 0.2645 (QPW @ 6 dB)
% maximum: 3.9669 (rnd. del. @ 30 dB)
the normalized numbers of
iterations for
% 5.) reference SNRs of 10, 20, and 30 dB
$\text{SNR}_{\text{dB}} \geq \SI{10}{\deci\bel}$.
% g) nonconvex l0.5-minimization method consistently improved both the mean SSIM indices and the relative RMSEs for all reference SNRs
The nonconvex $\ell_{0.5}$-minimization method
\eqreflqmin{eqn:recovery_reg_norm_lq_minimization}{ 0.5 } consistently improved both
% true for each single recovery!
% 1.) mean SSIM indices
% MATLAB:
% SSIM_index_spgl1_lq_qpw(1,:)*1e2         = 99.7325   99.8129   99.9100   99.9999  100.0000  100.0000
% SSIM_index_spgl1_lq_rnd_apo(1,:)*1e2	   = 98.7629   99.0758   99.3342   99.9890   99.9997  100.0000
% SSIM_index_spgl1_lq_rnd_del(1,:)*1e2	   = 99.4363   99.5682   99.7416   99.9975   99.9999  100.0000
% SSIM_index_spgl1_lq_rnd_apo_del(1,:)*1e2 = 99.3561   99.3499   99.6033   99.9839   99.9967   99.9985
the mean \ac{SSIM} indices and
% 2.) relative RMSEs
% MATLAB:
% rel_RMSE_spgl1_lq_qpw(1,:)*1e2         =  3.3658    1.9743    1.2019    0.5398    0.2027    0.5261
% rel_RMSE_spgl1_lq_rnd_apo(1,:)*1e2	 = 44.8397   34.3090   37.5912    2.2907    0.2474    0.6639
% rel_RMSE_spgl1_lq_rnd_del(1,:)*1e2	 = 31.0277   25.9121   14.8017    0.8747    0.2224    0.6272
% rel_RMSE_spgl1_lq_rnd_apo_del(1,:)*1e2 = 34.9563   31.3846   27.5957    2.6183    0.5178    0.7432
the relative \acp{RMSE} for
% 3.) all reference SNRs
all reference \acp{SNR}.
% h) random waves caused significantly larger relative RMSEs than the QPW at the low reference SNRs [l0.5-minimization]
The random waves, however, caused
% 1.) significantly larger relative RMSEs
significantly larger relative \acp{RMSE} than
% 2.) quasi-plane wave (QPW)
the \ac{QPW} at
% 3.) low reference SNRs of 3, 6, and 10 dB
the low reference \acp{SNR}, i.e.
$\text{SNR}_{\text{dB}} \in \setsymbol{Q} = \{ \SI{3}{\deci\bel}, \SI{6}{\deci\bel}, \SI{10}{\deci\bel} \}$, and
% 4.) superposition of randomly-apodized QCWs
the superposition of
randomly-apodized \acp{QCW} performed
% 5.) worst
worst.
% i) sample means of the normalized numbers of iterations increased [l0.5-minimization]
% MATLAB:
% N_iter_max = 1348.6
%
% N_iter_spgl1_lq_qpw(1,:)*1e2/N_iter_max	  = 21.4445   23.5207   36.3933   52.2913   75.7897   90.6125
% N_iter_spgl1_l1_qpw(1,:)*1e2/N_iter_max	  =  3.4406    3.8188    8.3494   15.4605   24.1436   39.7449
%
% N_iter_spgl1_lq_rnd_apo(1,:)*1e2/N_iter_max	  = 29.1932   33.8203   43.8974   57.6598   88.6846  116.1946
% N_iter_spgl1_l1_rnd_apo(1,:)*1e2/N_iter_max	  =  7.2149    7.0888    8.4680   12.8355   23.0535   38.1878
%
% N_iter_spgl1_lq_rnd_del(1,:)*1e2/N_iter_max	  = 25.6785   33.2938   37.1867   57.5337   89.4335  103.2923
% N_iter_spgl1_l1_rnd_del(1,:)*1e2/N_iter_max	  =  4.7827    6.8071    7.7339   14.7560   25.8194   40.1898
%
% N_iter_spgl1_lq_rnd_apo_del(1,:)*1e2/N_iter_max = 31.2695   34.9844   41.4578   64.7338  100.0000  133.1751
% N_iter_spgl1_l1_rnd_apo_del(1,:)*1e2/N_iter_max =  6.5401    7.2297    9.5284   14.3111   28.3034   62.8059
% 
% differences:
% QPW:			18.0039   19.7019   28.0439   36.8308   51.6462   50.8676
% rnd. apo.:		21.9783   26.7314   35.4293   44.8243   65.6310   78.0068
% rnd. del.:		20.8957   26.4867   29.4528   42.7777   63.6141   63.1025
% rnd. apo. del.:	24.7293   27.7547   31.9294   50.4227   71.6966   70.3693
% minimum: 18.0039 (QPW @ 3 dB)
% maximum: 71.6966 (rnd. apo. del. @ 30 dB)
The sample means of
% 1.) normalized numbers of iterations
the normalized numbers of
iterations increased significantly by
% 2.) at least 18 %
at least \SI{18}{\percent} for
% 3.) quasi-plane wave (QPW)
the \ac{QPW} at
% 4.) lowest reference SNR
the lowest reference \ac{SNR} to
% 5.) at most 71.7 %
at most \SI{71.7}{\percent} for
% 6.) superposition of both randomly-apodized and randomly-delayed QCWs
the superposition of both
randomly-apodized and
randomly-delayed \acp{QCW} at
% 7.) highest reference SNR
the highest reference \ac{SNR}.
% k) sample standard deviations of the relative RMSEs were reduced [l0.5-minimization]
% MATLAB:
% rel_RMSE_spgl1_lq_qpw(2,:)*1e2         = 0.9370    0.5312    0.4390    0.1265    0.0415         0
% rel_RMSE_spgl1_l1_qpw(2,:)*1e2         = 2.4453    0.4627   14.5006   11.6007   15.9304         0
%
% rel_RMSE_spgl1_lq_rnd_apo(2,:)*1e2	 = 3.2399    5.4849    2.5371    1.8768    0.0453         0
% rel_RMSE_spgl1_l1_rnd_apo(2,:)*1e2	 = 2.1477    2.0824    3.4309   10.2053    4.9393         0
%
% rel_RMSE_spgl1_lq_rnd_del(2,:)*1e2	 = 6.5740    4.0448    2.9058    0.4946    0.0586         0
% rel_RMSE_spgl1_l1_rnd_del(2,:)*1e2	 = 2.1170    4.3834    7.2373   17.8836    8.9579         0
%
% rel_RMSE_spgl1_lq_rnd_apo_del(2,:)*1e2 = 6.6002    4.5136    3.5523    2.8522    0.9964         0
% rel_RMSE_spgl1_l1_rnd_apo_del(2,:)*1e2 = 1.9915    1.1288    5.9226    8.8835    8.3409         0
In addition,
% 1.) sample standard deviations
the sample standard deviations of
% 2.) relative RMSEs
the relative \acp{RMSE} were reduced for
% 3.) reference SNRs of 10, 20, and 30 dB
$\text{SNR}_{\text{dB}} \geq \SI{10}{\deci\bel}$.


%%%%%%%%%%%%%%%%%%%%%%%%%%%%%%%%%%%%%%%%%%%%%%%%%%%%%%%%%%%%%%%%%%%%%%%%%%%%%%%%%%%%%%%%%%%%%%%%%%%%%%%%%%%%%%%%
% graphic: incident acoustic energies, recorded electric energies, and sample means of the relative RMSEs
%%%%%%%%%%%%%%%%%%%%%%%%%%%%%%%%%%%%%%%%%%%%%%%%%%%%%%%%%%%%%%%%%%%%%%%%%%%%%%%%%%%%%%%%%%%%%%%%%%%%%%%%%%%%%%%%
\graphic{results/object_A/kappa_only/sr_1/normalize_true/figures/latex/sim_study_obj_A_sr_1_p_in_Phi_energy_rel_rec_error_SNR_3.tex}%
{% a) figure illustrates the normalized incident acoustic energies, the normalized first Born approximations of the recorded electric energies, and the mean relative recovery errors
 % 1.) normalized incident acoustic energies (multiple pulse-echo measurements, multifrequent)
 Normalized incident acoustic energies
 \eqref{eqn:recovery_p_in_energy}
 (cf. \subref{fig:sim_study_obj_A_p_in_energy}),
 % 2.) normalized recorded electric energies in the pulse echoes (all pulse-echo measurements, multifrequent, all array elements)
 normalized recorded electric energies
 \eqref{eqn:recovery_reg_v_rx_born_trans_coef_energy}
 (cf. \subref{fig:sim_study_obj_A_v_rx_born_energy}), and
 % 3.) sample means of the relative RMSEs caused by the nonconvex l0.5-minimization method
 sample means of
 the relative \acfp{RMSE}\acused{RMSE} caused by
 % 4.) nonconvex l0.5-minimization method
 the nonconvex $\ell_{0.5}$-minimization method
 \eqreflqmin{eqn:recovery_reg_norm_lq_minimization}{ 0.5 }
 (cf. \subref{fig:sim_study_obj_A_sr_1_mean_rel_errors}) for
 all individual wires.
 % b) maximum energies normalized both types of energy
 The maximum energies normalized %, which were attained by
 % 1.) superposition of both randomly-apodized and randomly-delayed QCWs
 %the superposition of both
 %randomly-apodized and
 %randomly-delayed \aclp{QCW}\acused{QCW} for
 % 2.) wire with the index 3
 %the wire with
 %the index $3$, normalized
 both types of
 energy.
 % c) r_{2}-coordinate of each wire increased monotonically with its index
 The $r_{2}$-coordinate of
 each wire increased monotonically with
 its index.
 % d) reference SNR amounted to \text{SNR}_{\text{dB}} = \SI{3}{\deci\bel}
 The reference \ac{SNR} amounted to
 $\text{SNR}_{\text{dB}} = \SI{3}{\deci\bel}$.
}%
{sim_study_obj_A_sr_1_p_in_Phi_energy_rel_rec_error}

%---------------------------------------------------------------------------------------------------------------
% 3.) variations in the incident acoustic energies, the SNRs, and the mean relative RMSEs
%---------------------------------------------------------------------------------------------------------------
% a) variations in the incident acoustic energies were negligible for the QPW but strong and erratic with dynamic ranges between 7.64 and 8.69 dB for the random waves
% article:Schiffner2018, Sect. VIII. Results / Sect. VIII-A. Wire Phantom / Sect. VIII-A.4) Recovery by lq-Minimization (subsubsec:results_phantom_wire_lq_minimization)
% - The nonconvex $\ell_{0.5}$-minimization method \eqreflqmin{eqn:recovery_reg_norm_lq_minimization}{ 0.5 } CONSISTENTLY IMPROVED both
%   [1.)] the MEAN \ac{SSIM} INDICES and
%   [2.)] the RELATIVE \acp{RMSE} for
%   all reference \acp{SNR}.
% - The RANDOM WAVES, however, caused
%   SIGNIFICANTLY LARGER RELATIVE \acp{RMSE} THAN THE \ac{QPW} AT THE LOW REFERENCE \acp{SNR}, i.e.
%   $\text{SNR}_{\text{dB}} \in \setsymbol{Q} = \{ \SI{3}{\deci\bel}, \SI{6}{\deci\bel}, \SI{10}{\deci\bel} \}$, and
%   the superposition of randomly-apodized \acp{QCW} performed worst.
% article:Schiffner2018, Sect. VII. Simulation Study / Sect. VII-B. Methods / Sect. VII-B.6) Recovery by lq-Minimization (subsubsec:sim_study_methods_lq_minimization)
% - FOR EACH WIRE,
%   [1.)] the incident acoustic energies \eqref{eqn:recovery_p_in_energy} and
%   [2.)] the recorded electric energies \eqref{eqn:recovery_reg_v_rx_born_trans_coef_energy} were related to
%   the sample means of the relative \acp{RMSE} caused by
%   the nonconvex $\ell_{0.5}$-minimization method \eqreflqmin{eqn:recovery_reg_norm_lq_minimization}{ 0.5 }.
The variations in
% 1.) incident acoustic energies at a specified grid point (all pulse-echo measurements, multifrequent)
the incident acoustic energies
\eqref{eqn:recovery_p_in_energy} across
% 2.) isolated positions
the isolated positions of
% 3.) wires
the wires were negligible for
% 4.) quasi-plane wave (QPW)
the \ac{QPW} but
% 5.) strong and erratic
strong and
erratic for
% 6.) random waves
the random waves
(cf. \cref{fig:sim_study_obj_A_sr_1_p_in_Phi_energy_rel_rec_error}).
% b) dynamic ranges amounted to at least 7.6 dB for the superposition of randomly-delayed QCWs and at most 8.7 dB for the superposition of randomly-apodized QCWs
% 10*log10( max( p_incident_qpw_1.p_incident_pos_kappa_energy ) / min( p_incident_qpw_1.p_incident_pos_kappa_energy ) ) = 0.3104 dB
% 10*log10( max(p_incident_rnd_apo_1.p_incident_pos_kappa_energy) / min(p_incident_rnd_apo_1.p_incident_pos_kappa_energy) ) = 8.6924 dB
% 10*log10( max(p_incident_rnd_del_1.p_incident_pos_kappa_energy) / min(p_incident_rnd_del_1.p_incident_pos_kappa_energy) ) = 7.6395 dB
% 10*log10( max(p_incident_rnd_apo_del_1.p_incident_pos_kappa_energy) / min(p_incident_rnd_apo_del_1.p_incident_pos_kappa_energy) ) = 8.3245 dB
The dynamic ranges amounted to
% 1.) at least 7.6 dB
at least \SI{7.6}{\deci\bel} for
% 2.) superposition of randomly-delayed QCWs
the superposition of
randomly-delayed \acp{QCW} and
% 3.) at most 8.7 dB
at most \SI{8.7}{\deci\bel} for
% 4.) superposition of randomly-apodized QCWs
the superposition of
randomly-apodized \acp{QCW}.
% c) recorded electric energies strongly reflected these variations and generally decreased with increasing axial coordinates of the identical wires
The recorded electric energies
\eqref{eqn:recovery_reg_v_rx_born_trans_coef_energy} strongly reflected
% 1.) variations in the incident acoustic energies
these variations and generally decreased with
% 2.) increasing axial coordinates
increasing axial coordinates of
% 3.) identical wires
the identical wires.
% d) recorded electric energies reflected the SNRs of the corrupted RF voltage signals induced by the individual wires
Fixing
% 1.) energy
the energy of
% 2.) additive errors
the additive errors in
% 3.) linear algebraic system (all pulse-echo measurements, multifrequent, all array elements, additive errors)
the linear algebraic system
\eqref{eqn:recovery_reg_prob_general_obs_trans_coef_error},
% 4.) recorded electric energies in the pulse echoes (all pulse-echo measurements, multifrequent, all array elements)
they additionally reflected
% 5.) signal-to-noise ratios (SNRs)
the \acp{SNR} of
% 6.) vectors stacking the relevant Fourier coefficients of the recorded RF voltage signals (all pulse-echo measurements, multifrequent, all array elements)
the recorded \ac{RF} voltage signals
\eqref{eqn:recovery_sys_lin_eq_v_rx_born_all_f_all_in_v_rx} induced by
% 7.) individual wires
the individual wires.
% e) individual wires insonified by relatively low incident acoustic energies induced corrupted RF voltage signals of worse SNR than those insonified by relatively high incident acoustic energy
Those insonified by
% 1.) incident acoustic energies at a specified grid point (all pulse-echo measurements, multifrequent)
relatively low incident acoustic energies
\eqref{eqn:recovery_p_in_energy}, e.g.
% 2.) wires with the indices 6, 8, and 16
the wires with
the indices $6$, $8$, and $16$ for
% 3.) superposition of randomly-apodized QCWs
the superposition of
randomly-apodized \acp{QCW}
(cf. blue bars in \subref{fig:sim_study_obj_A_p_in_energy}), induced
% 4.) vectors stacking the relevant Fourier coefficients of the recorded RF voltage signals (all pulse-echo measurements, multifrequent, all array elements)
recorded \ac{RF} voltage signals
\eqref{eqn:recovery_sys_lin_eq_v_rx_born_all_f_all_in_v_rx} of
% 5.) worse SNR
worse \ac{SNR}
(cf. blue bars in \subref{fig:sim_study_obj_A_v_rx_born_energy}) than
% 6.) individual wires
those insonified by
% 7.) incident acoustic energies at a specified grid point (all pulse-echo measurements, multifrequent)
relatively high incident acoustic energies
\eqref{eqn:recovery_p_in_energy}, e.g.
% 8.) wires with the indices 3, 11, and 16
the wires with
the indices $3$, $11$, and $16$ for
% 9.) superposition of both randomly-apodized and randomly-delayed QCWs
the superposition of both
randomly-apodized and
randomly-delayed \acp{QCW}
(cf. gray bars in \subref{fig:sim_study_obj_A_p_in_energy} and \subref{fig:sim_study_obj_A_v_rx_born_energy}).
% f) variations in the SNRs induced variations in the mean relative RMSEs caused by the nonconvex l0.5-minimization method
These variations in
% 1.) signal-to-noise ratios (SNRs)
the \acp{SNR} induced
% 2.) variations
variations in
% 3.) mean relative RMSEs
the mean relative \acp{RMSE} caused by
% 4.) nonconvex l0.5-minimization method
the nonconvex $\ell_{0.5}$-minimization method
\eqreflqmin{eqn:recovery_reg_norm_lq_minimization}{ 0.5 }
(cf. \subref{fig:sim_study_obj_A_sr_1_mean_rel_errors}).
