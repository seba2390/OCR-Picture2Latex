%%%%%%%%%%%%%%%%%%%%%%%%%%%%%%%%%%%%%%%%%%%%%%%%%%%%%%%%%%%%%%%%%%%%%%%%%%%%%%%%%%%%%%%%%%%%%%%%%%%%%%%%%%%%%%%%
% spectra: tissue-mimicking phantom (adjoint, single wave emission)
%%%%%%%%%%%%%%%%%%%%%%%%%%%%%%%%%%%%%%%%%%%%%%%%%%%%%%%%%%%%%%%%%%%%%%%%%%%%%%%%%%%%%%%%%%%%%%%%%%%%%%%%%%%%%%%%
\graphic{results/object_B/kappa_only/adj_1/normalize_true/figures/latex/sim_study_obj_B_adj_1_dft.tex}%
{% a) figure displays the absolute values of the matrix-vector products between the adjoint normalized sensing matrix and the normalized recorded RF voltage signals
 Absolute values of
 the matrix-vector products between
 % 1.) adjoint normalized sensing matrix (all pulse-echo measurements, multifrequent, all array elements)
 the adjoint normalized sensing matrix
 \eqref{eqn:recon_reg_norm_sensing_matrix} and
 % 2.) normalized linear algebraic system (all pulse-echo measurements, multifrequent, all array elements, additive errors)
 the normalized recorded \ac{RF} voltage signals
 \eqref{eqn:recovery_reg_norm_obs_trans_coef_error} for
 % 3.) quasi-plane wave (QPW)
 the \acl{QPW}\acused{QPW}
 (cf. \subref{fig:sim_study_obj_B_adj_1_dft_qpw}) and
 the superpositions of
 % 4.) superposition of randomly-apodized QCWs
 randomly-apodized \acfp{QCW}\acused{QCW}
 (cf. \subref{fig:sim_study_obj_B_adj_1_dft_rnd_apo}),
 % 5.) superposition of randomly-delayed QCWs
 randomly-delayed \acp{QCW}
 (cf. \subref{fig:sim_study_obj_B_adj_1_dft_rnd_del}), and both
 % 6.) superposition of both randomly-apodized and randomly-delayed QCWs
 randomly-apodized and
 randomly-delayed \acp{QCW}
 (cf. \subref{fig:sim_study_obj_B_adj_1_dft_rnd_apo_del}).
 % b) green crosshairs indicate the specified spatial frequencies and mostly coincide with local maxima
 % mostly: QPW at \hat{\vect{K}} = \trans{ ( -0.2324, 0.2539 ) } -> no local maximum
 % rnd. apo. at \hat{\vect{K}} = \trans{ ( -0.1445, 0.2148 ) }, \trans{ ( 0.1465, 0.207 ) } -> no local maxima
 % rnd. del. at \hat{\vect{K}} = \trans{ ( -0.1445, 0.2148 ) } -> no local maximum
 % rnd. apo. del. at \hat{\vect{K}} = \trans{ ( 0.1465, 0.207 ) } -> no local maximum
 The green crosshairs indicate
 the specified spatial frequencies and mostly coincide with
 local maxima.
 % c) inset images magnify the regions indicated by the white squares
 The inset images magnify
 the regions indicated by
 the white squares.
 % d) reference SNR amounted to \text{SNR}_{\text{dB}} = \SI{10}{\deci\bel}
 The reference \ac{SNR} amounted to
 $\text{SNR}_{\text{dB}} = \SI{10}{\deci\bel}$.
}%
{sim_study_obj_B_adj_1_dft_kap}

%---------------------------------------------------------------------------------------------------------------
% 1.) significance of the adjoint normalized sensing matrices (adjoint, single wave emission)
%---------------------------------------------------------------------------------------------------------------
% a) all incident waves accurately detected at least 80 % of the specified spatial frequencies
All incident waves accurately detected
% 1.) at least 80 %
at least
\SI{80}{\percent} of
% 2.) specified spatial frequencies
the specified spatial frequencies
(cf. \cref{fig:sim_study_obj_B_adj_1_dft_kap}).
% b) QPW produced smooth coherent sidelobes and erroneously indicated the presence of multiple unspecified spatial frequencies by isolated local maxima
% article:Schiffner2018, Sect. VIII. Results / Sect. VIII-B. Tissue-Mimicking Phantom / Sect. VIII-B.2) Transform Point Spread Functions (subsubsec:results_phantom_tissue_tpsf)
% - In contrast, the sensing matrix \eqref{eqn:recovery_reg_sensing_matrix} induced by the \ac{QPW} formed
%   SMOOTH COHERENT SIDELOBES, whose absolute values lacked noise-like features, and
%   a SECONDARY ISOLATED MAXIMUM with
%   an absolute value of approximately \SI{-2.4}{\deci\bel} at
%   the normalized spatial frequency \hat{\vect{K}} \approx \trans{ ( 0.12, 0.35 ) }.
% - The latter ERRONEOUSLY INDICATED THE PRESENCE OF UNSPECIFIED LATERAL FREQUENCIES and MISGUIDED
%   THE SPARSITY-PROMOTING $\ell_{q}$-MINIMIZATION METHOD \eqref{eqn:recovery_reg_norm_lq_minimization} for SUFFICIENTLY LARGE ADDITIVE ERRORS.
The \ac{QPW}, which missed
% 1.) normalized spatial frequency \hat{\vect{K}} \approx \trans{ ( -0.23, 0.25 ) }
the normalized spatial frequency
$\hat{\vect{K}} \approx \trans{ ( \num{-0.23}, \num{0.25} ) }$, produced
% 2.) smooth coherent sidelobes (lack noise-like features)
smooth coherent sidelobes and erroneously indicated
% 3.) presence of multiple unspecified spatial frequencies
the presence of
multiple unspecified spatial frequencies by
% 4.) isolated local maxima
isolated local maxima.
% c) examples for local maxima
A pronounced local maximum was located at
% 1.) K_{1} = \trans{ ( -96, 126 ) } / 512 = \trans{ ( -0.1875, 0.2461 ) }
$\hat{\vect{K}}_{1} \approx \trans{ ( -0.19, 0.25 ) }$ and
% 2.) multiple smaller local maxima
% K_{2} = ( -0.1133, 0.4375 ), K_{3} = ( -0.0293, 0.4648 ), K_{4} = ( 0.01758, 0.4648 ), K_{5} = ( 0.07031, 0.4551 ), K_{6} = ( 0.1055, 0.2715 )
multiple smaller local maxima were located at
$\hat{\vect{K}}_{2} \approx \trans{ ( -0.11, 0.44 ) }$,
$\hat{\vect{K}}_{3} \approx \trans{ ( -0.03, 0.46 ) }$,
$\hat{\vect{K}}_{4} \approx \trans{ ( 0.02, 0.46 ) }$,
$\hat{\vect{K}}_{5} \approx \trans{ ( 0.07, 0.46 ) }$, and
$\hat{\vect{K}}_{6} \approx \trans{ ( 0.11, 0.27 ) }$.
% d) deviating local maxima erroneously indicated the presence of additional spatial frequencies in the object and misguided the sparsity-promoting lq-minimization method
These misguided
% 1.) sparsity-promoting lq-minimization method
the sparsity-promoting $\ell_{q}$-minimization method
\eqref{eqn:recovery_reg_norm_lq_minimization} for
% 2.) sufficiently large additive errors
sufficiently large additive errors.
% e) random waves substituted these coherent sidelobes and the isolated local maxima by a uniform distribution of noise-like features in the passbands
In contrast,
% 1.)
the random waves substituted both
% 2.) smooth coherent sidelobes (lack noise-like features)
these sidelobes and
% 3.) undesired local maxima
the undesired local maxima by
% 4.) similar noise-like artifacts
similar noise-like artifacts inside
% 4.) passbands
their passbands.
% f) similar noise-like artifacts facilitated the identification of the specified spatial frequencies
These facilitated
% 1.) identification
the identification of
% 2.) specified spatial frequencies
the specified spatial frequencies.
