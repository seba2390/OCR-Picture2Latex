%%%%%%%%%%%%%%%%%%%%%%%%%%%%%%%%%%%%%%%%%%%%%%%%%%%%%%%%%%%%%%%%%%%%%%%%%%%%%%%%%%%%%%%%%%%%%%%%%%%%%%%%%%%%%%%%
% images: point spread functions (PSFs, third fixed position)
%%%%%%%%%%%%%%%%%%%%%%%%%%%%%%%%%%%%%%%%%%%%%%%%%%%%%%%%%%%%%%%%%%%%%%%%%%%%%%%%%%%%%%%%%%%%%%%%%%%%%%%%%%%%%%%%
\graphic{results/object_A/kappa_only/sr_tpsf/figures/latex/sim_study_obj_A_sr_1_tpsf_images.tex}%
{% a) figure illustrates the absolute values of the PSFs associated with the random observation process and the observation processes induced by the GWN and all incident waves
 Absolute values of
 % 1.) point spread functions (PSFs)
 the \aclp{PSF}\acused{PSF}
 \eqref{eqn:cs_math_tpsf} associated with
 % 2.) random observation process (RIP)
 the random observation process
 (cf. \subref{fig:sim_study_obj_A_sr_1_tpsf_images_rip_1}) and
 % 3.) observation processes (all pulse-echo measurements, multifrequent, all array elements)
 the observation processes
 \eqref{eqn:recovery_sys_lin_eq_v_rx_born_all_f_all_in_mat} induced by
 % 3.a) GWN
 the \acl{GWN}\acused{GWN}
 (cf. \subref{fig:sim_study_obj_A_sr_1_tpsf_images_blgwn_1}),
 % 3.b) quasi-plane wave (QPW)
 the \acl{QPW}\acused{QPW}
 (cf. \subref{fig:sim_study_obj_A_sr_1_tpsf_images_qpw_1}), and
 the superpositions of
 % 3.c) superposition of randomly-apodized QCWs 
 randomly-apodized \acfp{QCW}\acused{QCW}
 (cf. \subref{fig:sim_study_obj_A_sr_1_tpsf_images_rnd_apo_1}),
 % 3.d) superposition of randomly-delayed QCWs
 randomly-delayed \acp{QCW}
 (cf. \subref{fig:sim_study_obj_A_sr_1_tpsf_images_rnd_del_1}), and both
 % 3.e) superposition of both randomly-apodized and randomly-delayed QCWs
 randomly-apodized and
 randomly-delayed \acp{QCW}
 (cf. \subref{fig:sim_study_obj_A_sr_1_tpsf_images_rnd_apo_del_1}).
 % b) green crosshairs indicate the third fixed position
 % index_tpsf = 3;
 % n_{2} = 72334
 % index_1 = 142
 % index_2 = 142
 % \vect{r}_{ n_{2} - 1 } \approx \trans{ ( \SI{-8.7249}{\milli\meter}, \SI{10.7823}{\milli\meter} ) }
 The green crosshairs indicate
 the third fixed position.
 %$\vect{r}_{ \text{lat}, n_{2} - 1 } \approx \trans{ ( \SI{-8.72}{\milli\meter}, \SI{10.78}{\milli\meter} ) }$.
 % c) inset images magnify the regions indicated by the white squares
 The inset images magnify
 the regions indicated by
 the white squares.
}%
{sim_study_obj_A_sr_1_tpsf_images}

%---------------------------------------------------------------------------------------------------------------
% 1.) overview of the computed PSFs for the reference observation processes and all incident waves
%---------------------------------------------------------------------------------------------------------------
% a) all computed PSFs correctly attained their maximum absolute values of unity at the fixed positions
Although
% 1.) point spread functions (PSFs)
all computed \acp{PSF}
\eqref{eqn:cs_math_tpsf} correctly attained
% 2.) maximum absolute values of unity
their maximum absolute values of
unity at
% 3.) fixed positions
the fixed positions,
% b) all computed PSFs differed in their behavior for the remaining positions
they differed in
% 1.) behavior
their behavior for
% 2.) remaining positions
the remaining positions
(cf. \cref{fig:sim_study_obj_A_sr_1_tpsf_images}).
% c) both reference observation processes produced random values close to zero that rendered the maxima sharp and isolated
Both reference observation processes produced
% 1.) random values close to zero
random values close to
zero that rendered
% 2.) isolated sharp maximum at the discrete position \vect{r}_{ n_{2} - 1 }
the maxima sharp and
isolated.
% d) random observation process uniformly distributed these values over the FOV
The random observation process
\eqref{eqn:sim_study_params_ref_obs_proc_rip} uniformly distributed
% 1.) random values close to zero
these values over
% 2.) field of view (FOV)
the \ac{FOV}, whereas
% e) structured version formed noticeable gaps that were laterally adjacent to the maxima and shaped hourglasses of larger absolute values
its structured version
\eqref{eqn:sim_study_params_ref_obs_proc_gwn} formed
% 1.) noticeable gaps
noticeable gaps that were
% 2.) laterally adjacent
laterally adjacent to
% 3.) maximum absolute values of unity
the maxima and shaped
% 4.) hourglasses of larger absolute values
hourglasses of
larger absolute values
(cf. inset image).
% f) observation processes induced by all incident waves concentrated relatively large absolute values close to unity in elliptical-shaped regions around the maxima
The observation processes
\eqref{eqn:recovery_sys_lin_eq_v_rx_born_all_f_all_in_mat} induced by
% 1.) all incident waves
all incident waves concentrated
% 2.) relatively large absolute values close to unity
relatively large absolute values close to
unity in
% 3.) elliptical-shaped regions
elliptical-shaped regions around
% 3.) maximum absolute values of unity
the maxima.
% g) lengths of the minor and major axes
% QPW: ( 0.30; 0.76) mm / rnd. apo.: ( 0.152; 0.61 ) mm / rnd. del.: ( 0.152; 0.46 ) mm / rnd. apo. del.: ( 0.152; 0.53 ) mm
The lengths of
the minor and major axes ranged from
\SIrange{0.15}{0.3}{\milli\meter} and from
\SIrange{0.46}{0.76}{\milli\meter},
respectively.
% h) observation processes induced by all incident waves distributed the nonzero values less uniformly and formed sidelobes of various characters
They distributed
% 1.) nonzero absolute values
the nonzero values less uniformly and formed
% 2.) sidelobes of various characters
sidelobes of
various characters.

%---------------------------------------------------------------------------------------------------------------
% 2.) detailed description of the PSFs for all incident waves
%---------------------------------------------------------------------------------------------------------------
% a) observation process induced by the QPW deviated most significantly from both reference observation processes
The observation process
\eqref{eqn:recovery_sys_lin_eq_v_rx_born_all_f_all_in_mat} induced by
% 1.) quasi-plane wave (QPW)
the \ac{QPW} deviated
most significantly from
% 2.) reference observation processes
both references.
% b) observation process induced by the QPW formed the largest elliptical-shaped region and coherent sidelobes of approximately constant absolute values
It formed
% 1.) largest elliptical-shaped region
the largest elliptical-shaped region and
% 2.) coherent sidelobes
coherent sidelobes of
approximately constant absolute values.
% c) observation processes induced by the random waves resembled that induced by the GWN
In contrast,
% 1.) observation processes (all pulse-echo measurements, multifrequent, all array elements)
the observation processes
\eqref{eqn:recovery_sys_lin_eq_v_rx_born_all_f_all_in_mat} induced by
% 2.) random waves
the random waves resembled
% 3.) observation process induced by the GWN
that induced by
the \ac{GWN}
\eqref{eqn:sim_study_params_ref_obs_proc_gwn}.
% d) sizes of the elliptical-shaped regions decreased relative to the QPW
The sizes of
% 1.) elliptical-shaped regions
the elliptical-shaped regions decreased relative to
% 2.) quasi-plane wave (QPW)
the \ac{QPW}.
% e) sidelobes diffused and fluctuated in their values resulting in more uniform distributions
The sidelobes diffused and
fluctuated in
their values, similar to
a speckle pattern, resulting in
more uniform distributions.
% f) both superpositions of QCWs using random time delays distributed the nonzero values slightly more uniformly than the superposition of randomly-apodized QCWs
Both superpositions of
\acp{QCW} using
random time delays distributed
the nonzero values slightly more uniformly than
% 1.) superposition of randomly-apodized QCWs
the superposition of
randomly-apodized \acp{QCW}.
% g) distributed values appeared more random
In addition,
the distributed values appeared more random.


%%%%%%%%%%%%%%%%%%%%%%%%%%%%%%%%%%%%%%%%%%%%%%%%%%%%%%%%%%%%%%%%%%%%%%%%%%%%%%%%%%%%%%%%%%%%%%%%%%%%%%%%%%%%%%%%
% table: full extents at half maximum (all fixed positions)
%%%%%%%%%%%%%%%%%%%%%%%%%%%%%%%%%%%%%%%%%%%%%%%%%%%%%%%%%%%%%%%%%%%%%%%%%%%%%%%%%%%%%%%%%%%%%%%%%%%%%%%%%%%%%%%%
\begin{table*}[tb]
 \centering
 \caption{%
  % a) table summarizes the FEHMs of the PSFs associated with the observation processes induced by all incident waves
  Full extents at
  half maximum of
  % 1.) point spread functions (PSFs)
  the \aclp{PSF}
  \eqref{eqn:cs_math_tpsf} associated with
  % 2.) observation processes (all pulse-echo measurements, multifrequent, all array elements)
  the observation processes
  \eqref{eqn:recovery_sys_lin_eq_v_rx_born_all_f_all_in_mat} induced by
  % 3.) all incident waves
  all incident waves.
  % b) FEHMs were evaluated for nine uniformly distributed positions along the diagonal from (-17.5, 2 ) mm to ( 17.5, 37 ) mm
  They were evaluated for
  % 1.) nine uniformly distributed positions
  nine uniformly distributed positions along
  % 2.) diagonal from (-17.4879 mm, 2.0193 mm ) to ( 17.4879 mm, 36.9951 mm )
  the diagonal from
  $\trans{ ( \SI{-17.5}{\milli\meter}, \SI{2}{\milli\meter} ) }$ to
  $\trans{ ( \SI{17.5}{\milli\meter}, \SI{37}{\milli\meter} ) }$ and numbered from
  \numrange{1}{9} with
  increasing axial coordinate.
 }
 \label{tab:sim_study_obj_A_sr_1_tpsf_fehm}
 \begin{tabular}{%
  @{}%
  l%																01.) type of incident wave
  S[table-format=1.2,table-number-alignment = right,table-auto-round]%								02.) 1st fixed position
  S[table-format=1.2,table-number-alignment = right,table-auto-round]%								03.) 2nd fixed position
  S[table-format=1.2,table-number-alignment = right,table-auto-round]%								04.) 3rd fixed position
  S[table-format=1.2,table-number-alignment = right,table-auto-round]%								05.) 4th fixed position
  S[table-format=1.2,table-number-alignment = right,table-auto-round]%								06.) 5th fixed position
  S[table-format=1.2,table-number-alignment = right,table-auto-round]%								07.) 6th fixed position
  S[table-format=1.2,table-number-alignment = right,table-auto-round]%								08.) 7th fixed position
  S[table-format=1.2,table-number-alignment = right,table-auto-round]%								09.) 8th fixed position
  S[table-format=1.2,table-number-alignment = right,table-auto-round]%								10.) 9th fixed position
  S[table-format=1.2(3),separate-uncertainty,table-align-uncertainty = true,table-number-alignment = right,table-auto-round]%	11.) sample mean & std. dev
  @{}%
 }
 \toprule
  \multicolumn{1}{@{}H}{\multirow{2}{*}{Incident wave}} &
  \multicolumn{10}{H@{}}{Full extent at half maximum (\si{\milli\meter\squared})}\\
  \cmidrule(l){2-11}
  &
  \multicolumn{1}{H}{1} &
  \multicolumn{1}{H}{2} &
  \multicolumn{1}{H}{3} &
  \multicolumn{1}{H}{4} &
  \multicolumn{1}{H}{5} &
  \multicolumn{1}{H}{6} &
  \multicolumn{1}{H}{7} &
  \multicolumn{1}{H}{8} &
  \multicolumn{1}{H}{9} &
  \multicolumn{1}{H@{}}{$\text{Sample mean} \pm \text{std. dev.}$}\\
  \cmidrule(r){1-1}\cmidrule(lr){2-2}\cmidrule(lr){3-3}\cmidrule(lr){4-4}\cmidrule(lr){5-5}
  \cmidrule(lr){6-6}\cmidrule(lr){7-7}\cmidrule(lr){8-8}\cmidrule(lr){9-9}\cmidrule(lr){10-10}\cmidrule(l){11-11}
 \addlinespace
  \ExpandableInput{results/object_A/kappa_only/sr_tpsf/tables/sim_study_obj_A_v2_sr_1_tpsf_areas_6dB.tex}
 \addlinespace
 \bottomrule
 \end{tabular}
\end{table*}

%---------------------------------------------------------------------------------------------------------------
% 3.) full extents at half maximum (all fixed positions)
%---------------------------------------------------------------------------------------------------------------
% a) random waves achieved FEHMs that were smaller than or equal to those of the QPW for all fixed positions, except those numbered s \in \{ 6, 7 \}
% 1: QPW (max)              2: QPW (max)             3: QPW (max)         4: QPW (max)              5: QPW (max)              6: rnd. del. (max)   7: rnd. apo. (max)   8: QPW (max)              9: QPW = rnd. apo. (max)
%    rnd. del.:      73.68%    rnd. apo.:     50%       rnd. del.: 57.14%    rnd. del.:      54.17%    rnd. del.:      50%       Rnd. apo.: 45.45%    rnd. del.: 54.55%    rnd. apo. del.: 33.33%    rnd. apo. del.: 23.53%
%    rnd. apo. del.: 42.11%    rnd. apo. del: 16.67%    rnd. apo.: 33.33%    rnd. apo. del.: 20.83%    rnd. apo. del.: 17.86%    QPW:        3.03%    QPW:       13.63%    rnd. apo.:      14.29%    rnd. del.:      9.8%
The random waves achieved
\acp{FEHM} that were
smaller than or
equal to
those of
the \ac{QPW} for
all fixed positions, except
those numbered
$s \in \{ 6, 7 \}$
(cf. \cref{tab:sim_study_obj_A_sr_1_tpsf_fehm}).
% b) FEHMs generally increased with the axial coordinate of these positions
The \acp{FEHM} generally increased with
the axial coordinate of
these positions.
% c) maximum normalized differences ranged from 23.5 % (rnd. apo. del., s = 9) to 73.7 % (rnd. del., s = 1)
The maximum normalized differences ranged from
% 1.) superposition of both randomly-apodized and randomly-delayed QCWs at the ninth fixed position (s = 9)
\SI{23.5}{\percent} for
the superposition of both
randomly-apodized and
randomly-delayed \acp{QCW} at
the ninth fixed position, i.e.
$s = 9$, to
% 2.) superposition of randomly-delayed QCWs at the first fixed position (s = 1)
\SI{73.7}{\percent} for
the superposition of
randomly-delayed \acp{QCW} at
the first fixed position, i.e.
$s = 1$.
% d) mean FEHMs reflected these reductions relative to the QPW
The mean \acp{FEHM} reflected
these reductions relative to
the \ac{QPW}.
% e) superposition of randomly-apodized QCWs produced the largest sample mean and sample standard deviation among the random waves
The superposition of
randomly-apodized \acp{QCW} produced
the largest sample mean and
sample standard deviation among
the random waves.
% f) both reference observation processes consistently achieved the minimum FEHM of a two-dimensional volume element for all fixed positions
Both reference observation processes consistently achieved
% 1.) minimum FEHM
the minimum \ac{FEHM} of
% 2.) two-dimensional volume element
a two-dimensional volume element
$\Delta V \approx \SI{5.81e-3}{\milli\meter\squared}$ for
% 3.) all fixed positions
all fixed positions.


%%%%%%%%%%%%%%%%%%%%%%%%%%%%%%%%%%%%%%%%%%%%%%%%%%%%%%%%%%%%%%%%%%%%%%%%%%%%%%%%%%%%%%%%%%%%%%%%%%%%%%%%%%%%%%%%
% empirical CDFs: point spread functions (PSFs)
%%%%%%%%%%%%%%%%%%%%%%%%%%%%%%%%%%%%%%%%%%%%%%%%%%%%%%%%%%%%%%%%%%%%%%%%%%%%%%%%%%%%%%%%%%%%%%%%%%%%%%%%%%%%%%%%
\graphic{results/object_A/kappa_only/sr_tpsf/figures/latex/sim_study_obj_A_sr_1_tpsf_ecdfs.tex}%
{% a) figure illustrates the empirical CDFs of the PSFs associated with both reference observation processes and the observation processes induced by all incident waves
 Empirical \acfp{CDF}\acused{CDF} of
 % 1.) point spread functions (PSFs)
 the \aclp{PSF}
 \eqref{eqn:cs_math_tpsf} associated with
 % 2.) reference observation processes
 both reference observation processes and
 % 3.) observation processes (all pulse-echo measurements, multifrequent, all array elements)
 the observation processes
 \eqref{eqn:recovery_sys_lin_eq_v_rx_born_all_f_all_in_mat} induced by
 % 4.) all incident waves
 all incident waves.
 % b) inset graphic magnifies the region indicated by the red rectangle
 The inset graphic magnifies
 the region indicated by
 the red rectangle.
}%
{sim_study_obj_A_sr_1_tpsf_ecdfs}

%---------------------------------------------------------------------------------------------------------------
% 4.) empirical CDFs (all fixed positions)
%---------------------------------------------------------------------------------------------------------------
% a) empirical CDFs confirmed the beneficial properties of the random waves
The empirical \acp{CDF} confirmed
% 1.) beneficial properties
the beneficial properties of
% 2.) random waves
the random waves
(cf. \cref{fig:sim_study_obj_A_sr_1_tpsf_ecdfs}).
% b) random observation process primarily attained absolute values ranging from -70 to -30.93 dB
% RIP: -30.9349 dB
% RIP: 4.3245 % @ -70 dB
The random observation process
\eqref{eqn:sim_study_params_ref_obs_proc_rip} primarily attained
% 1.) absolute values ranging from -70 to -30.93 dB
absolute values ranging from
\SIrange{-70}{-30.93}{\deci\bel}.
% c) only approximately 4.3 % of the FOV were attributed to smaller absolute values
Only approximately \SI{4.3}{\percent} of
% 1.) field of view (FOV)
the \ac{FOV} were attributed to
% 2.) smaller absolute values
smaller absolute values.
% d) structured random observation process deviated modestly from this behavior
The structured random observation process
\eqref{eqn:sim_study_params_ref_obs_proc_gwn} deviated
modestly from
this behavior.
% e) absolute values ranged from -70 to -16.79 dB
% BLGWN: -16.7930 dB
% BLGWN: 12.0914 % @ -70 dB
The absolute values ranged from
\SIrange{-70}{-16.79}{\deci\bel}.
% f) increased dynamic range reflected the two gaps that were laterally adjacent to the FEHMs
% article:Schiffner2018, Sect. VIII. Results / Sect. VIII-A. Wire Phantom / Sect. VIII-A.2) Point Spread Functions (subsubsec:results_phantom_wire_psf)
% - The random observation process \eqref{eqn:sim_study_params_ref_obs_proc_rip} uniformly distributed these values over the \ac{FOV}, whereas
%   ITS STRUCTURED VERSION \eqref{eqn:sim_study_params_ref_obs_proc_gwn} FORMED NOTICEABLE GAPS that were
%   laterally adjacent to the maxima and shaped hourglasses of larger absolute values (cf. inset image).
This increased dynamic range reflected
% 1.) noticeable gaps
the gaps that were
% 2.) laterally adjacent to the maxima
laterally adjacent to
the maxima
(cf. \cref{fig:sim_study_obj_A_sr_1_tpsf_images_blgwn_1}).
% g) observation processes induced by the random waves deviated in a stronger but similar fashion from both references
The observation processes
\eqref{eqn:recovery_sys_lin_eq_v_rx_born_all_f_all_in_mat} induced by
% 1.) random waves
the random waves deviated in
% 2.) stronger but similar fashion
a stronger but similar fashion from
% 3.) reference observation processes
both references.
% h) absolute values below -70 dB constituted approximately 49.5 to 56.2 % of the FOV and those above this threshold formed the remaining 43.8 to 50.5 %
% rnd. apo.: -0.1124 dB / rnd. del: -0.1164 dB / rnd. apo. del: -0.1349 dB
% rnd. apo.: 56.1824 % @ -70 dB / rnd. del: 49.4858 % @ -70 dB / rnd. apo. del: 49.7709 % @ -70 dB
The absolute values below
\SI{-70}{\deci\bel} constituted
% 1.) approximately 49.5 to 56.2 %
approximately \SIrange{49.5}{56.2}{\percent} of
% 2.) field of view
the \ac{FOV} and
% 3.) absolute values above -70 dB
those above this threshold, which reached up to
\SI{-0.11}{\deci\bel}, formed
% 4.) remaining 43.8 to 50.5 %
the remaining \SIrange{43.8}{50.5}{\percent}.
% i) superposition of randomly-apodized QCWs distributed the latter values over the smallest percentage of the FOV
The superposition of
randomly-apodized \acp{QCW} distributed
% 1.) absolute values above -70 dB
the latter values over
% 2.) smallest percentage
the smallest percentage of
% 3.) field of view
the \ac{FOV}.
% j) observation process induced by the QPW deviated strongest from both references
Clearly,
% 1.) observation process (all pulse-echo measurements, multifrequent, all array elements)
the observation process
\eqref{eqn:recovery_sys_lin_eq_v_rx_born_all_f_all_in_mat} induced by
% 2.) quasi-plane wave (QPW)
the \ac{QPW} deviated
% 3.) strongest
strongest from
% 4.) reference observation processes
both references.
% k) absolute values ranging from -70 to -0.15 dB strongly concentrated on only 20.1 % of the FOV and indicated the distinctive sidelobes
% article:Schiffner2018, Sect. VIII. Results / Sect. VIII-A. Wire Phantom / Sect. VIII-A.2) Point Spread Functions (subsubsec:results_phantom_wire_psf)
% - The observation process \eqref{eqn:recovery_sys_lin_eq_v_rx_born_all_f_all_in_mat} induced by the \ac{QPW} deviated most significantly from
%   both references.
% - It formed the largest elliptical-shaped region and coherent sidelobes of approximately constant absolute values.
% QPW: -0.1508 dB
% QPW: 79.9494 % @ -70 dB
The absolute values ranging from
\SIrange{-70}{-0.15}{\deci\bel} strongly concentrated on
% 1.) only 20.1 %
only \SI{20.1}{\percent} of
% 2.) field of view
the \ac{FOV} and indicated
% 3.) distinctive sidelobes
the distinctive sidelobes
(cf. \cref{fig:sim_study_obj_A_sr_1_tpsf_images_qpw_1}).
