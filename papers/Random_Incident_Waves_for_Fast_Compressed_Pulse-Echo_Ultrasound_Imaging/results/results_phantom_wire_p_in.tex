%%%%%%%%%%%%%%%%%%%%%%%%%%%%%%%%%%%%%%%%%%%%%%%%%%%%%%%%%%%%%%%%%%%%%%%%%%%%%%%%%%%%%%%%%%%%%%%%%%%%%%%%%%%%%%%%
% images: discretized incident acoustic pressure fields for the wire phantom (QPW, rnd. apo., rnd. del., rnd. apo. del.)
%%%%%%%%%%%%%%%%%%%%%%%%%%%%%%%%%%%%%%%%%%%%%%%%%%%%%%%%%%%%%%%%%%%%%%%%%%%%%%%%%%%%%%%%%%%%%%%%%%%%%%%%%%%%%%%%
\graphictwocols{results/object_A/p_in/figures/latex/sim_study_obj_A_setup_p_in_qpw_rnd_apo_rnd_del_rnd_apo_del_images.tex}%
{% a) figure illustrates the discretized incident acoustic pressure fields associated with all incident waves
 Incident acoustic pressure fields
 \eqref{eqn:recovery_p_in} associated with
 % 1.) quasi-plane wave (QPW)
 the \acl{QPW}
 (cf.
 \subref{fig:sim_study_obj_A_p_in_images_qpw_ctr} and
 \subref{fig:sim_study_obj_A_p_in_ref_qpw_f}%
 ) and
 the superpositions of
 % 2.) superposition of randomly-apodized QCWs
 randomly-apodized \acfp{QCW}\acused{QCW}
 (cf.
 \subref{fig:sim_study_obj_A_p_in_images_rnd_apo_ctr} and
 \subref{fig:sim_study_obj_A_p_in_ref_rnd_apo_f}%
 ),
 % 3.) superposition of randomly-delayed QCWs
 randomly-delayed \acp{QCW}
 (cf.
 \subref{fig:sim_study_obj_A_p_in_images_rnd_del_ctr} and
 \subref{fig:sim_study_obj_A_p_in_ref_rnd_del_f}%
 ), and both
 % 4.) superposition of both randomly-apodized and randomly-delayed QCWs
 randomly-apodized and
 randomly-delayed \acp{QCW}
 (cf.
 \subref{fig:sim_study_obj_A_p_in_images_rnd_apo_del_ctr} and
 \subref{fig:sim_study_obj_A_p_in_ref_rnd_apo_del_f}%
 ).
 %--------------------------------------------------------------------------------------------------------------
 % top row
 %--------------------------------------------------------------------------------------------------------------
 % a) top row displays these fields at the discrete frequency closest to the center frequency as functions of the position
 The top row
 (cf.
 \subref{fig:sim_study_obj_A_p_in_images_qpw_ctr} to
 \subref{fig:sim_study_obj_A_p_in_images_rnd_apo_del_ctr}%
 ) displays
 these fields at
 % 1.) discrete frequency closest to the center frequency
 % f_{ l_{\text{c}} } = \SI{3.9951}{\mega\hertz} \approx f_{\text{c}}
 the discrete frequency closest to
 the center frequency %, i.e. $f_{ l_{\text{c}} } \approx f_{\text{c}}$ with the index $l_{\text{c}} = 329$,
 as functions of
 % 2.) grid position
 the position.
 % b) large image shows the normalized absolute value in decibel, whereas the inset image shows the phase in radian
 For each type of
 incident wave,
 the large image shows
 % 1.) normalized absolute value in decibel
 the normalized absolute value
 (right colorbar), whereas
 the inset image shows
 % 2.) phase in radian
 the phase inside
 the region indicated by
 the red square 
 (top colorbar).
 % c) reference value for normalization is the maximum absolute value inside the specified FOV
 % \vect{r} = \trans{ ( \SI{-6.1341}{\milli\meter}; \SI{6.8199}{\milli\meter} ) }
 %The maximum absolute value, which
 %was attained by
 %the superposition of both
 %randomly-apodized and
 %randomly-delayed \acp{QCW} at
 %the position
 %$\vect{r} \approx \trans{ ( \SI{-6.13}{\milli\meter}; \SI{6.82}{\milli\meter} ) }$, normalized
 %the absolute values.
 %--------------------------------------------------------------------------------------------------------------
 % bottom row
 %--------------------------------------------------------------------------------------------------------------
 % a) bottom row displays the normalized absolute values and the normalized phase differences at the three positions as functions of the frequency
 The bottom row
 (cf.
 \subref{fig:sim_study_obj_A_p_in_ref_qpw_f} to
 \subref{fig:sim_study_obj_A_p_in_ref_rnd_apo_del_f}%
 ) displays
 % 1.) normalized absolute values in decibel
 the normalized absolute values and
 % 2.) normalized phase differences
 the normalized phase differences at
 the three positions indicated by
 the markers in
 the top row as
 functions of
 the frequency.
 % b) reference values for the normalizations are the maximum absolute value and the minimum phase difference at the three positions
 %The maximum absolute value and
 %the minimum phase difference at
 %these positions, which
 %were attained by
 %the superposition of
 %randomly-delayed \acp{QCW}, served as
 %reference values.
}%
{sim_study_obj_A_p_in_qpw_rnd_apo_rnd_del_rnd_apo_del_images}

%---------------------------------------------------------------------------------------------------------------
% 1.) spatial dependencies closest to the center frequency (QPW, rnd. apo., rnd. del., rnd. apo. del.)
%---------------------------------------------------------------------------------------------------------------
% a) random waves differed significantly from the QPW
The random waves differed significantly from
% 1.) quasi-plane wave (QPW)
the \ac{QPW}
(cf. \cref{fig:sim_study_obj_A_p_in_qpw_rnd_apo_rnd_del_rnd_apo_del_images}).
% b) interference of the QCWs resulted in beamlike fluctuations
The interference of
the \acp{QCW} introduced
beamlike fluctuations into
the absolute values.
% b)
These were
% 1.) relatively subtle and regular for the QPW
relatively subtle and
regular for
the \ac{QPW} but
% 2.) pronounced and erratic for the random waves
pronounced and
irregular for
the random waves.
% b) their [beamlike fluctuations] dynamic ranges increased from only 8.15 dB for the QPW to 67.54 dB for the superposition of randomly-delayed QCWs
% QPW: 8.1532 dB / rnd. apo.: 65.8998 dB / rnd. del.: 67.5402 dB / rnd. apo. del.: 61.9938 dB
Their dynamic ranges increased from
\SI{8.15}{\deci\bel} for
% 1.) quasi-plane wave (QPW)
the \ac{QPW} to
\SI{67.54}{\deci\bel} for
% 2.) superposition of randomly-delayed QCWs
the superposition of
randomly-delayed \acp{QCW}.
% e) approximately constant phase on lines parallel to the r_{1}-axis indicated plane wavefronts
The paths of
constant phase, which were
approximately parallel to
the $r_{1}$-axis for
the \ac{QPW}, turned
irregular with
heterogeneous normal vectors exhibiting
nonzero $r_{2}$-components for
the random waves.
% f) paths of constant phase indicated a transition from plane wavefronts to irregular wavefronts
\TODO{match adjective with Fig. 4}
They indicated
a transition from
plane to
irregular wavefronts.

%---------------------------------------------------------------------------------------------------------------
% 2.) spectral dependencies for three closely spaced positions next to the r_{2}-axis (QPW, rnd. apo., rnd. del., rnd. apo. del.)
%---------------------------------------------------------------------------------------------------------------
% a) behavior at the three indicated positions
At
the three indicated positions,
% 1.) normalized absolute values reflected the approximate Gaussian shape of the electromechanical pulse echo
the absolute values reflected
the approximate Gaussian shape of
the electromechanical pulse echo, and
% 2.) normalized phase differences depended approximately affine-linearly on the frequency
the phase differences depended approximately affine-linearly on
the frequency.
% b) distinct slopes indicated the times-of-flight of the wavefronts from the linear transducer array to each reference position
Their slopes indicated
% 1.) times-of-flights
the diverse \acp{TOF} of
% 2.) wavefronts
the wavefronts from
% 3.) linear transducer array
the linear transducer array to
% 4.) position
each position.
% c) QPW achieved very similar absolute values for all three reference positions, whereas the random waves induced notches and peaks
The \ac{QPW} achieved
very similar absolute values at
all three positions, whereas
% d) random waves induced notches and peaks at various frequencies that erratically modified the approximate Gaussian shape
the random waves induced
notches and peaks at
various frequencies that
erratically modified
the approximate Gaussian shape for
each position.
% d) QPW additionally achieved linear phases
The former additionally achieved
linear phases, whereas
% e.) normalized unwrapped phases approximately maintain the linear dependence on the normalized frequency
the latter introduced
erratic deviations from
this linear frequency dependence.
% e) modification is relatively modest for the superposition of randomly-apodized QCWs
The superposition of
randomly-apodized \acp{QCW} induced
relatively modest modifications, whereas
% f) both superpositions of QCWs using random time delays induced more pronounced modifications
both superpositions of
\acp{QCW} using
random time delays induced
more pronounced modifications.
