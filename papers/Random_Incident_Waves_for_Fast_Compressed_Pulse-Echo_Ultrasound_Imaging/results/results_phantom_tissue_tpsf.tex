%%%%%%%%%%%%%%%%%%%%%%%%%%%%%%%%%%%%%%%%%%%%%%%%%%%%%%%%%%%%%%%%%%%%%%%%%%%%%%%%%%%%%%%%%%%%%%%%%%%%%%%%%%%%%%%%
% images: transform point spread functions (TPSFs, seventh reference spatial frequency)
%%%%%%%%%%%%%%%%%%%%%%%%%%%%%%%%%%%%%%%%%%%%%%%%%%%%%%%%%%%%%%%%%%%%%%%%%%%%%%%%%%%%%%%%%%%%%%%%%%%%%%%%%%%%%%%%
\graphic{results/object_B/kappa_only/sr_tpsf/figures/latex/sim_study_obj_B_sr_1_tpsf_images.tex}%
{% a) figure illustrates the absolute values of the TPSFs associated with the random sensing matrix and the sensing matrices induced by the GWN and all incident waves
 Absolute values of
 % 1.) transform point spread functions (TPSFs)
 the \aclp{TPSF}\acused{TPSF}
 \eqref{eqn:cs_math_tpsf} associated with
 % 2.) random sensing matrix (RIP)
 the random sensing matrix
 (cf. \subref{fig:sim_study_obj_B_sr_1_tpsf_images_rip_1}) and
 % 3.) sensing matrices (all pulse-echo measurements, multifrequent, all array elements)
 the sensing matrices
 \eqref{eqn:recovery_reg_sensing_matrix} induced by
 % 3.a) GWN
 the \acl{GWN}\acused{GWN}
 (cf. \subref{fig:sim_study_obj_B_sr_1_tpsf_images_blgwn_1}),
 % 3.b) quasi-plane wave (QPW)
 the \acl{QPW}\acused{QPW}
 (cf. \subref{fig:sim_study_obj_B_sr_1_tpsf_images_qpw_1}), and
 the superpositions of
 % 3.c) superposition of randomly-apodized QCWs
 randomly-apodized \acfp{QCW}\acused{QCW}
 (cf. \subref{fig:sim_study_obj_B_sr_1_tpsf_images_rnd_apo_1}),
 % 3.d) superposition of randomly-delayed QCWs
 randomly-delayed \acp{QCW}
 (cf. \subref{fig:sim_study_obj_B_sr_1_tpsf_images_rnd_del_1}), and both
 % 3.e) superposition of both randomly-apodized and randomly-delayed QCWs
 randomly-apodized and
 randomly-delayed \acp{QCW}
 (cf. \subref{fig:sim_study_obj_B_sr_1_tpsf_images_rnd_apo_del_1}).
 % b) green crosshairs indicate the seventh fixed normalized spatial frequency
 % index_tpsf = 7;
 % n_{2} = 97970
 % \hat{\vect{K}}_{ n_{2} } \approx \trans{ ( \num{-0.1270}, \num{0.3457} ) }
 The green crosshairs indicate
 the seventh fixed normalized spatial frequency.
 %$\hat{\vect{K}}_{ n_{2} } \approx \trans{ ( \num{-0.13}, \num{0.35} ) }$.
 % c) inset images magnify the regions indicated by the white squares
 The inset images magnify
 the regions indicated by
 the white squares.
}%
{sim_study_obj_B_sr_1_tpsf_images}

%---------------------------------------------------------------------------------------------------------------
% 1.) visual inspection of all computed TPSFs
%---------------------------------------------------------------------------------------------------------------
% a) all computed TPSFs correctly attained their maximum absolute values of unity at the fixed spatial frequencies
Although
% 1.) transform point spread functions (TPSFs)
all computed \acp{TPSF}
\eqref{eqn:cs_math_tpsf} correctly attained
% 2.) maximum absolute values of unity
their maximum absolute values of
unity at
% 3.) fixed spatial frequencies
the fixed spatial frequencies,
% b) all computed TPSFs differed in their behavior for the remaining spatial frequencies
they differed in
% 1.) behavior
their behavior for
% 2.) remaining spatial frequencies
the remaining spatial frequencies
(cf. \cref{fig:sim_study_obj_B_sr_1_tpsf_images}).
% c) both reference sensing matrices produced similar results as the random observation process
% article:Schiffner2018, Sect. VIII. Results / Sect. VIII-A. Wire Phantom / Sect. VIII-A.2) Point Spread Functions (subsubsec:results_phantom_wire_psf)
% - Both reference observation processes produced
%   RANDOM VALUES CLOSE TO ZERO THAT RENDERED THE MAXIMA SHARP AND ISOLATED.
% - The RANDOM OBSERVATION PROCESS \eqref{eqn:sim_study_params_ref_obs_proc_rip} UNIFORMLY DISTRIBUTED THESE VALUES over
%   the \ac{FOV}, whereas [...].
Both reference sensing matrices produced
% 1.) similar results
similar results as
% 2.) first reference observation process and its entries (RIP)
the random observation process
\eqref{eqn:sim_study_params_ref_obs_proc_rip}, i.e.
% 3.) uniformly distributed random values close to zero
uniformly distributed random values close to
zero that rendered
% 4.) sharp isolated maximum at the discrete position \vect{r}_{ n_{2} - 1 }
the maxima sharp and
isolated
(cf. \cref{fig:sim_study_obj_A_sr_1_tpsf_images_rip_1}).
% d) structured random sensing matrix significantly elongated these maxima along the \hat{K}_{2}-axis in addition to modest lateral extensions
The structured random sensing matrix
\eqref{eqn:sim_study_params_ref_sens_mat_gwn}, however, significantly elongated
% 1.) sharp isolated maxima
these maxima along
% 2.) \hat{K}_{2}-axis
the $\hat{K}_{2}$-axis in addition to
% 3.) modest lateral extensions
modest lateral extensions
(cf. inset image).
% e) sensing matrices induced by the random waves produced similar noise-like artifacts inside their passbands
The sensing matrices
\eqref{eqn:recovery_reg_sensing_matrix} induced by
% 1.) random waves
the random waves approximately maintained these
% 2.) extended maxima along the \hat{K}_{2}-axis with modest lateral extensions
maxima but confined
% 3.) similar noise-like artifacts
similar noise-like artifacts to
% 4.) passbands
their passbands.
% f) sensing matrix induced by the QPW formed smooth coherent sidelobes and indicated the presence of unspecified spatial frequencies by secondary isolated maxima
In contrast,
% 1.) sensing matrix (all pulse-echo measurements, multifrequent, all array elements)
the sensing matrix
\eqref{eqn:recovery_reg_sensing_matrix} induced by
% 2.) quasi-plane wave (QPW)
the \ac{QPW} formed
% 3.) smooth coherent sidelobes (lack noise-like features)
smooth coherent sidelobes, whose
% 4.) absolute values
absolute values lacked
% 5.) noise-like features
noise-like features, and indicated
% 6.) presence
the presence of
% 7.) unspecified spatial frequencies
unspecified spatial frequencies by
% 8.) secondary isolated maxima
secondary isolated maxima, e.g.
% 9.) absolute value of approximately -2.4 dB
% K_{max} = \trans{ ( \num{-0.127}, \num{0.3457} ) }, K_{secmax} = \trans{ ( \num{0.123}, \num{0.3496} ) } @ -2.394 dB
an absolute value of
approximately \SI{-2.4}{\deci\bel} at
% 10.) normalized spatial frequency \hat{\vect{K}} \approx \trans{ ( 0.12, 0.35 ) }
the normalized spatial frequency
$\hat{\vect{K}} \approx \trans{ ( \num{0.12}, \num{0.35} ) }$.


%%%%%%%%%%%%%%%%%%%%%%%%%%%%%%%%%%%%%%%%%%%%%%%%%%%%%%%%%%%%%%%%%%%%%%%%%%%%%%%%%%%%%%%%%%%%%%%%%%%%%%%%%%%%%%%%
% table: full extents at half maximum (all fixed spatial frequencies)
%%%%%%%%%%%%%%%%%%%%%%%%%%%%%%%%%%%%%%%%%%%%%%%%%%%%%%%%%%%%%%%%%%%%%%%%%%%%%%%%%%%%%%%%%%%%%%%%%%%%%%%%%%%%%%%%
\begin{table*}[tb]
 \centering
 \caption{%
  % a) table summarizes the FEHMs of the TPSFs associated with the sensing matrices induced by all incident waves
  Full extents at
  half maximum of
  % 1.) transform point spread functions (TPSFs)
  the \aclp{TPSF}
  \eqref{eqn:cs_math_tpsf} associated with
  % 2.) sensing matrices (all pulse-echo measurements, multifrequent, all array elements)
  the sensing matrices
  \eqref{eqn:recovery_reg_sensing_matrix} induced by
  % 3.) all incident waves
  all incident waves.
  % b) FEHMs were evaluated for nine uniformly distributed normalized spatial frequencies along the semicircle with the center \hat{\vect{K}}_{ \text{c} } = \trans{ ( 0, 25 ) } / 128 and the radius \hat{K}_{ \text{r} } = 101 / 512
  They were evaluated for
  % 1.) nine uniformly distributed normalized spatial frequencies
  nine uniformly distributed normalized spatial frequencies along
  % 2.) semicircle
  the semicircle with
  % 3.) center \hat{\vect{K}}_{ \text{c} } = \trans{ ( 0, 25 ) } / 128
  the center
  $\hat{\vect{K}}_{ \text{c} } = \trans{ ( 0, 25 ) } / 128$ and
  % 4.) radius \hat{K}_{ \text{r} } = 101 / 512
  the radius
  $\hat{K}_{ \text{r} } = 101 / 512$ and numbered from
  \numrange{1}{9} with
  increasing polar angle.
 }
 \label{tab:sim_study_obj_B_sr_1_tpsf_fehm}
 \begin{tabular}{%
  @{}%
  l%																01.) type of incident wave
  S[table-format=2.2,table-number-alignment = right,table-auto-round]%								02.) 1st fixed spatial frequency
  S[table-format=2.2,table-number-alignment = right,table-auto-round]%								03.) 2nd fixed spatial frequency
  S[table-format=2.2,table-number-alignment = right,table-auto-round]%								04.) 3rd fixed spatial frequency
  S[table-format=2.2,table-number-alignment = right,table-auto-round]%								05.) 4th fixed spatial frequency
  S[table-format=2.2,table-number-alignment = right,table-auto-round]%								06.) 5th fixed spatial frequency
  S[table-format=2.2,table-number-alignment = right,table-auto-round]%								07.) 6th fixed spatial frequency
  S[table-format=2.2,table-number-alignment = right,table-auto-round]%								08.) 7th fixed spatial frequency
  S[table-format=2.2,table-number-alignment = right,table-auto-round]%								09.) 8th fixed spatial frequency
  S[table-format=2.2,table-number-alignment = right,table-auto-round]%								10.) 9th fixed spatial frequency
  S[table-format=2.2(4),separate-uncertainty,table-align-uncertainty = true,table-number-alignment = right,table-auto-round]%	11.) sample mean & std. dev
  @{}%
 }
 \toprule
  \multicolumn{1}{@{}H}{\multirow{2}{*}{Incident wave}} &
  \multicolumn{10}{H@{}}{Full extent at half maximum ($10^{-6}$)}\\
  \cmidrule(l){2-11}
  &
  \multicolumn{1}{H}{1} &
  \multicolumn{1}{H}{2} &
  \multicolumn{1}{H}{3} &
  \multicolumn{1}{H}{4} &
  \multicolumn{1}{H}{5} &
  \multicolumn{1}{H}{6} &
  \multicolumn{1}{H}{7} &
  \multicolumn{1}{H}{8} &
  \multicolumn{1}{H}{9} &
  \multicolumn{1}{H@{}}{$\text{Sample mean} \pm \text{std. dev.}$}\\
  \cmidrule(r){1-1}\cmidrule(lr){2-2}\cmidrule(lr){3-3}\cmidrule(lr){4-4}\cmidrule(lr){5-5}
  \cmidrule(lr){6-6}\cmidrule(lr){7-7}\cmidrule(lr){8-8}\cmidrule(lr){9-9}\cmidrule(lr){10-10}\cmidrule(l){11-11}
 \addlinespace
  \ExpandableInput{results/object_B/kappa_only/sr_tpsf/tables/sim_study_obj_B_v2_sr_1_tpsf_areas_6dB.tex}
 \addlinespace
 \bottomrule
 \end{tabular}
\end{table*}

%---------------------------------------------------------------------------------------------------------------
% 2.) full extents at half maximum (all fixed spatial frequencies)
%---------------------------------------------------------------------------------------------------------------
% a) random waves achieved smaller FEHMs than the QPW for all fixed spatial frequencies, except that numbered s = 5
% 1: QPW, 2: QPW, 3: QPW, 4: QPW, 5: rnd. ~, 6: QPW, 7: QPW, 8: QPW, 9: QPW
% 41.17%  57.15%  50.02%  62.52%  −200.26%   62.52%  50.02%  57.15%  41.17%
The random waves achieved
% 1.) smaller FEHMs
smaller \acp{FEHM} than
% 2.) quasi-plane wave (QPW)
the \ac{QPW} for
% 3.) all fixed spatial frequencies
all fixed spatial frequencies, except
% 4.) fixed spatial frequency number 5
that numbered
$s = 5$
(cf. \cref{tab:sim_study_obj_B_sr_1_tpsf_fehm}).
% b) random waves achieved identical FEHMs ranging from 11.44e-6 to 38.15e-6
These were identical for
% 1.) each fixed spatial frequency
each fixed spatial frequency, except
that numbered
$s = 8$, where
% c) superposition of randomly-delayed QCWs produced a slightly larger FEHM than the two remaining random waves
the superposition of
randomly-delayed \acp{QCW} produced
% 1.) slightly larger FEHM
a slightly larger \ac{FEHM} than
% 2.) other two random waves
the other two random waves.
% d) secondary maxima approximately doubled these FEHMs for all fixed spatial frequencies, except that numbered s = 5
% article:Schiffner2018, Sect. VIII. Results / Sect. VIII-B. Tissue-Mimicking Phantom / Sect. VIII-B.2) Transform Point Spread Functions (subsubsec:results_phantom_tissue_tpsf)
% - In contrast, the sensing matrix \eqref{eqn:recovery_reg_sensing_matrix} induced by the \ac{QPW} formed
%   smooth coherent sidelobes, whose absolute values lacked noise-like features, and
%   a SECONDARY ISOLATED MAXIMUM with
%   an absolute value of approximately \SI{-2.4}{\deci\bel} at
%   the normalized spatial frequency \hat{\vect{K}} \approx \trans{ ( 0.12, 0.35 ) }.
% - The latter erroneously indicated the presence of unspecified lateral frequencies and misguided
%   the sparsity-promoting $\ell_{q}$-minimization method \eqref{eqn:recovery_reg_norm_lq_minimization} for sufficiently large additive errors.
The secondary maxima, which were
erroneously formed by
% 1.) quasi-plane wave (QPW)
the \ac{QPW}
(cf. \cref{fig:sim_study_obj_B_sr_1_tpsf_images_qpw_1}), approximately doubled
% 2.) full extents at half maximum (FEHMs)
these \acp{FEHM} for
% 3.) all fixed spatial frequencies
all fixed spatial frequencies, except
that numbered
$s = 5$.
% e) maximum normalized differences ranged from 41.2 % (s \in \{ 1, 9 \}) to 62.5 % (s \in \{ 4, 6 \})
In fact,
% 1.) maximum normalized differences
the maximum normalized differences ranged from
% 2.) fixed spatial frequencies s \in \{ 1, 9 \}
\SI{41.2}{\percent} at
the fixed spatial frequencies
$s \in \{ 1, 9 \}$ to
% 2.) fixed spatial frequencies s \in \{ 4, 6 \}
\SI{62.5}{\percent} at
the fixed spatial frequencies
$s \in \{ 4, 6 \}$.
% f) FEHMs generally increase with the deviation of the polar angle from the axis of axial frequencies for all incident waves
% TODO: wirklich?
%They generally increase with
%the deviation of
%the polar angle from
%the axis of
%axial frequencies for
%all incident waves.
% g) mean FEHMs reflected these increases relative to the random waves
The mean \acp{FEHM} reflected
these increases relative to
the random waves.
% h) QPW outperformed the random waves by a normalized difference of 200 %
For $s = 5$,
% 1.) fixed normalized spatial frequency \hat{\vect{K}}_{ n_{2} } = \trans{ ( 0, 201 ) } / 512
the fixed normalized spatial frequency
$\hat{\vect{K}}_{ n_{2} } = \trans{ ( 0, 201 ) } / 512$ matched
% 2.) preferred direction of propagation
the preferred direction of
propagation of
% 3.) quasi-plane wave (QPW)
the \ac{QPW} that outperformed
% 4.) random waves
the random waves by
% 5.) normalized difference exceeding 200 %
a normalized difference exceeding
\SI{200}{\percent}.
% i) both reference sensing matrices consistently achieved the minimum FEHM of a two-dimensional normalized spatial frequency element for all fixed spatial frequencies
Both reference sensing matrices consistently achieved
% 1.) minimum FEHM
the minimum \ac{FEHM} of
% 2.) two-dimensional normalized spatial frequency element
a two-dimensional normalized spatial frequency element
$\Delta \hat{K} \approx \num{3.81e-6}$ for
% 3.) all fixed spatial frequencies
all fixed spatial frequencies.


%%%%%%%%%%%%%%%%%%%%%%%%%%%%%%%%%%%%%%%%%%%%%%%%%%%%%%%%%%%%%%%%%%%%%%%%%%%%%%%%%%%%%%%%%%%%%%%%%%%%%%%%%%%%%%%%
% empirical CDFs: transform point spread functions (TPSFs)
%%%%%%%%%%%%%%%%%%%%%%%%%%%%%%%%%%%%%%%%%%%%%%%%%%%%%%%%%%%%%%%%%%%%%%%%%%%%%%%%%%%%%%%%%%%%%%%%%%%%%%%%%%%%%%%%
\graphic{results/object_B/kappa_only/sr_tpsf/figures/latex/sim_study_obj_B_sr_1_tpsf_ecdfs.tex}%
{% a) figure illustrates the empirical CDFs of the TPSFs associated with both reference sensing matrices and the sensing matrices induced by all incident waves
 Empirical \acfp{CDF}\acused{CDF} of
 % 1.) transform point spread functions (TPSFs)
 the \aclp{TPSF}
 \eqref{eqn:cs_math_tpsf} associated with
 % 2.) reference sensing matrices
 both reference sensing matrices and
 % 3.) sensing matrices (all pulse-echo measurements, multifrequent, all array elements)
 the sensing matrices
 \eqref{eqn:recovery_reg_sensing_matrix} induced by
 % 4.) all incident waves
 all incident waves.
 % b) inset graphic magnifies the region indicated by the red rectangle
 The inset graphic magnifies
 the region indicated by
 the red rectangle.
}%
{sim_study_obj_B_sr_1_tpsf_ecdfs}

%---------------------------------------------------------------------------------------------------------------
% 3.) empirical CDFs (all fixed spatial frequencies)
%---------------------------------------------------------------------------------------------------------------
% a) empirical CDFs confirmed the beneficial properties of the random waves
The empirical \acp{CDF} confirmed
% 1.) beneficial properties
the beneficial properties of
% 2.) random waves
the random waves
(cf. \cref{fig:sim_study_obj_B_sr_1_tpsf_ecdfs}).
% b) random sensing matrix almost exclusively attained absolute values ranging from -70 to -33.2 dB
% RIP: -33.2029 dB
% RIP: 0.2835 % @ -70 dB
The random sensing matrix
\eqref{eqn:sim_study_params_ref_sens_mat_rip} almost exclusively attained
% 1.) absolute values ranging from -70 to -33.2 dB
absolute values ranging from
\SIrange{-70}{-33.2}{\deci\bel}.
% c) only approximately 0.3 % of the admissible spatial frequencies were attributed to smaller absolute values
Only approximately \SI{0.3}{\percent} of
% 1.) admissible spatial frequencies
the admissible spatial frequencies were attributed to
% 2.) smaller absolute values
smaller absolute values.
% d) structured random sensing matrix deviated marginally from this behavior
The structured random sensing matrix
\eqref{eqn:sim_study_params_ref_sens_mat_gwn} deviated
marginally from
this behavior.
% e) absolute values ranged from -70 dB to -7.78 dB
% BLGWN: -7.7835 dB
% BLGWN: 0.2585 % @ -70 dB
The absolute values, however, ranged from
\SIrange{-70}{-7.78}{\deci\bel}.
% f) increased dynamic range reflected the extended maxima
% article:Schiffner2018, Sect. VIII. Results / Sect. VIII-B. Tissue-Mimicking Phantom / Sect. VIII-B.2) Transform Point Spread Functions (subsubsec:results_phantom_tissue_tpsf)
% - The structured random sensing matrix \eqref{eqn:sim_study_params_ref_sens_mat_gwn}, however, significantly elongated
%   these maxima along the $\hat{K}_{2}$-axis in addition to modest lateral extensions (cf. inset image).
This increased dynamic range reflected
% 1.) extended maxima
the extended maxima
(cf. \cref{fig:sim_study_obj_B_sr_1_tpsf_images_blgwn_1}).
% g) sensing matrices induced by the random waves deviated in a stronger but almost identical fashion from both references
The sensing matrices
\eqref{eqn:recovery_reg_sensing_matrix} induced by
% 1.) random waves
the random waves deviated in
% 2.) stronger but almost identical fashion
a stronger but almost identical fashion from
% 3.) reference sensing matrices
both references.
% h) absolute values below -70 dB constituted approximately 20 % of the admissible spatial frequencies and those above this threshold form the remaining 80 %
% rnd. apo.: -0.4746 dB / rnd. del: -0.5178 dB / rnd. apo. del: -0.4665 dB
% rnd. apo.: 19.9779 % @ -70 dB / rnd. del: 21.1052 % @ -70 dB / rnd. apo. del: 21.5781 % @ -70 dB
The absolute values below
\SI{-70}{\deci\bel} constituted
% 1.) approximately 20 %
approximately \SI{20}{\percent} of
% 2.) admissible spatial frequencies
the admissible spatial frequencies and
% 3.) absolute values above -70 dB
those above this threshold, which reached up to
\SI{-0.47}{\deci\bel}, formed
% 4.) remaining 80 %
the remaining \SI{80}{\percent}.
% i) sensing matrix induced by the QPW deviated strongest from both references
Clearly,
% 1.) sensing matrix (all pulse-echo measurements, multifrequent, all array elements)
the sensing matrix
\eqref{eqn:recovery_reg_sensing_matrix} induced by
% 2.) quasi-plane wave (QPW)
the \ac{QPW} deviated
% 3.) strongest
strongest from
% 4.) reference sensing matrices
both references.
% j) absolute values ranging from -70 to -0.19 dB strongly concentrated on only 11 % of the admissible spatial frequencies and reflected both the distinctive sidelobes and the secondary maxima
% article:Schiffner2018, Sect. VIII. Results / Sect. VIII-B. Tissue-Mimicking Phantom / Sect. VIII-B.2) Transform Point Spread Functions (subsubsec:results_phantom_tissue_tpsf)
% - In contrast, the sensing matrix \eqref{eqn:recovery_reg_sensing_matrix} induced by the \ac{QPW} formed
%   smooth coherent sidelobes, whose absolute values lacked noise-like features, and
%   a SECONDARY ISOLATED MAXIMUM with
%   an absolute value of approximately \SI{-2.4}{\deci\bel} at
%   the normalized spatial frequency \hat{\vect{K}} \approx \trans{ ( 0.12, 0.35 ) }.
% QPW: -0.1869 dB
% QPW: 89.0438 % @ -70 dB
The absolute values ranging from
\SIrange{-70}{-0.19}{\deci\bel} strongly concentrated on
% 1.) only 11 %
only \SI{11}{\percent} of
% 2.) admissible spatial frequencies
the admissible spatial frequencies and reflected both
% 3.) distinctive sidelobes
the distinctive sidelobes and
% 4.) secondary maxima
the secondary maxima
(cf. \cref{fig:sim_study_obj_B_sr_1_tpsf_images_qpw_1}).
