%%%%%%%%%%%%%%%%%%%%%%%%%%%%%%%%%%%%%%%%%%%%%%%%%%%%%%%%%%%%%%%%%%%%%%%%%%%%%%%%%%%%%%%%%%%%%%%%%%%%%%%%%%%%%%%%
% images: recorded electric energies in the pulse echoes (tissue-mimicking phantom, single wave emission)
%%%%%%%%%%%%%%%%%%%%%%%%%%%%%%%%%%%%%%%%%%%%%%%%%%%%%%%%%%%%%%%%%%%%%%%%%%%%%%%%%%%%%%%%%%%%%%%%%%%%%%%%%%%%%%%%
\graphic{results/object_B/column_norms/figures/latex/sim_study_obj_B_norms_kappa.tex}%
{% a) figure illustrates the recorded electric energies induced by all incident waves
 Recorded electric energies
 \eqref{eqn:recovery_reg_v_rx_born_trans_coef_energy} induced by
 % 1.) quasi-plane wave (QPW) (dynamic range: -77.7554 dB)
 the \acl{QPW}\acused{QPW}
 (cf. \subref{fig:sim_study_obj_B_norms_kappa_qpw}) and
 the superpositions of
 % 2.) superposition of randomly-apodized QCWs (dynamic range: -72.0162 dB)
 randomly-apodized \acfp{QCW}\acused{QCW}
 (cf. \subref{fig:sim_study_obj_B_norms_kappa_rnd_apo}),
 % 3.) superposition of randomly-delayed QCWs (dynamic range: -70.6122 dB)
 randomly-delayed \acp{QCW}
 (cf. \subref{fig:sim_study_obj_B_norms_kappa_rnd_del}), and both
 % 4.) superposition of both randomly-apodized and randomly-delayed QCWs (dynamic range: -73.9546 dB)
 randomly-apodized and
 randomly-delayed \acp{QCW}
 (cf. \subref{fig:sim_study_obj_B_norms_kappa_rnd_apo_del}).
 % b) green contours indicate the values -6 dB, -10 dB, and -20 dB
 The green contours indicate
 the values
 \SI{-6}{\deci\bel},
 \SI{-10}{\deci\bel}, and
 \SI{-20}{\deci\bel}.
}%
{sim_study_obj_B_norms_kappa}

%---------------------------------------------------------------------------------------------------------------
% 1.) recorded electric energies in the pulse echoes
%---------------------------------------------------------------------------------------------------------------
% a) transfer behaviors of the sensing matrices resembled those of bandpass filters suppressing relatively low and high spatial frequencies
% article:Schiffner2018, Sect. V. Image Recovery Based on Compressed Sensing / Sect. V-D. Regularization of the Inverse Scattering Problem (subsec:recovery_regularization)
% - In fact,
%   the density of population and
%   the DYNAMIC RANGE OF THE RECORDED ELECTRIC ENERGIES
%   [ E_{i}^{(\text{B})} = \dnorm{ \vect{a}_{i}\bigl[ p^{(\text{in})} \bigr] }{2}{1}^{2} ] (eqn:recovery_reg_v_rx_born_trans_coef_energy) for
%   all $i \in \setcons{ N_{\text{lat}} }$, which
%   CHARACTERIZE THE TRANSFER BEHAVIOR, vary significantly with the orthonormal basis.
The transfer behaviors of
% 1.) sensing matrices (all pulse-echo measurements, multifrequent, all array elements)
the sensing matrices
\eqref{eqn:recovery_reg_sensing_matrix} induced by
% 2.) all incident waves
all incident waves resembled those of
% 3.) bandpass filters suppressing relatively low and high spatial frequencies
bandpass filters suppressing
relatively low and high
spatial frequencies
(cf. \cref{fig:sim_study_obj_B_norms_kappa}).
% b) high dynamic ranges indicated the existence of structural building blocks whose pulse echoes contained relatively low electric energies
The high dynamic ranges exceeding
\SI{70}{\deci\bel} indicated
% 1.) existence of structural building blocks
the existence of
structural building blocks whose
% 2.) pulse echoes
pulse echoes contained
% 3.) recorded electric energies in the pulse echoes (all pulse-echo measurements, multifrequent, all array elements)
relatively low electric energies
\eqref{eqn:recovery_reg_v_rx_born_trans_coef_energy}.
% c) QPW induced relatively large electric energies exceeding -20 dB in a sickle-shaped passband inside the interval of normalized spatial frequencies [ -0.24; 0.24 ] x [ 0.15; 0.49 ]
The \ac{QPW} induced
% 1.) relatively large recorded electric energies in the pulse echoes (all pulse-echo measurements, multifrequent, all array elements)
relatively large electric energies
\eqref{eqn:recovery_reg_v_rx_born_trans_coef_energy} exceeding
% 2.) -20 dB
\SI{-20}{\deci\bel} in
% 3.) sickle-shaped passband inside the interval of normalized spatial frequencies [ -0.24; 0.24 ] x [ 0.15; 0.49 ]
a sickle-shaped passband inside
the interval of
normalized spatial frequencies
$\hat{\vect{K}} \in [ \num{-0.24}; \num{0.24} ] \times [ \num{0.15}; \num{0.49} ]$, whereas
% d) random waves induced those in arbelos-shaped passbands inside the intervals of normalized spatial frequencies [ -0.43; 0.43 ] x [ 0; 0.49 ]
the random waves induced those in
% 1.) arbelos-shaped passbands
arbelos-shaped passbands inside
% 2.) intervals of normalized spatial frequencies [ -0.43; 0.43 ] x [ 0; 0.49 ]
the intervals of
normalized spatial frequencies
$\hat{\vect{K}} \in [ \num{-0.43}; \num{0.43} ] \times [ \num{0}; \num{0.49} ]$.
% e) random waves significantly enlarged the passbands
%The latter waves thus significantly enlarged
%the passbands.
% e) all formed passbands strongly agreed with the predictions of the FDT
% TODO: move to discussion spectral supports
All formed passbands, which were significantly enlarged by
the latter waves, strongly agreed with
the predictions of
the \ac{FDT}
(cf. footnote $\num{1}$ in \cref{sec:introduction}).
% f) bounded FOV expanded the predicted passbands by beamlike artifacts
The bounded \ac{FOV}, however, expanded
the predicted passbands by
beamlike artifacts.
% g) beamlike artifacts re-entered the illustrations at the maximum normalized axial frequency of unity
Owing to
% 1.) periodicity of the DFT
the periodicity of
the \ac{DFT},
% 2.) beamlike artifacts
these artifacts re-entered
the illustrations at
% 3.) maximum normalized axial frequency of unity
the maximum normalized axial frequency of
unity, i.e.
$\hat{K}_{2} = 1$.
% h) absence of aliasing confirmed the specifications of sufficiently small constant spacings between the adjacent grid points in the FOV
The absence of
% 1.) aliasing
aliasing confirmed
% 2.) specifications
the specifications of
% 3.) sufficiently small constant spacings between the adjacent grid points in the FOV along both coordinate axes
sufficiently small constant spacings between
the adjacent grid points in
the \ac{FOV}.
