%%%%%%%%%%%%%%%%%%%%%%%%%%%%%%%%%%%%%%%%%%%%%%%%%%%%%%%%%%%%%%%%%%%%%%%%%%%%%%%%%%%%%%%%%%%%%%%%%%%%%%%%%%%%%%%%
% graphic: block diagram of the pulse-echo measurement process
%%%%%%%%%%%%%%%%%%%%%%%%%%%%%%%%%%%%%%%%%%%%%%%%%%%%%%%%%%%%%%%%%%%%%%%%%%%%%%%%%%%%%%%%%%%%%%%%%%%%%%%%%%%%%%%%
\graphictwocols{linear_model/figures/latex/lin_mod_v_rx_sgn_proc_chain.tex}%
{% a) block diagram of the pulse-echo measurement process
 Block diagram of
 % 1.) pulse-echo measurement process (single pulse-echo measurement, monofrequent, single array element)
 the pulse-echo measurement process
 \eqref{eqn:lin_mod_v_rx_born_obs_proc}.
 % b) observation operators map the relative spatial fluctuations in compressibility to the Born approximation of of the recorded RF voltage signals
 Given
 % 1.) incident acoustic pressure fields
 the incident acoustic pressure fields
 $p_{l}^{(\text{in}, n)}: \R^{d} \mapsto \C$, which satisfy
 % 2.) Helmholtz equations for the incident acoustic pressure fields
 the \name{Helmholtz} equations
 \eqref{eqn:lin_mod_sol_wave_eq_pde_p_in},
 % 3.) observation operators
 the \name{Born} approximation linearly maps
 % 4.) relative spatial fluctuations in the unperturbed compressibility
 the compressibility fluctuations
 \eqref{eqn:lin_mod_mech_model_tis_simple_rel_fluctuations} to
 % 5.) Born approximation of the recorded RF voltage signals
 the recorded \ac{RF} voltage signals
 \eqref{eqn:lin_mod_v_rx_born}.
 %--------------------------------------------------------------------------------------------------------------
 % decomposition of the observation operators
 %--------------------------------------------------------------------------------------------------------------
 % a) first two multiplications yield the Born approximations of the contrast sources and the force densities
 %The first two multiplications yield
 %the \name{Born} approximations of
 % 1.) Born approximation of the contrast sources
 %the contrast sources
 %$\phi_{l}^{(\text{B}, n)}: \R^{d} \mapsto \C$ and
 % 2.) Born approximation of the force densities
 %the force densities
 %$f_{m, l}^{(\text{B}, n)}: \R^{d} \mapsto \C$.
 % b) receiver electromechanical transfer function
 %The multiplications of
 %their spatial integrals, i.e.
 % 1.) Born approximation of the compressive blocked forces
 %the \name{Born} approximation of
 %the compressive blocked forces
 %\eqref{eqn:lin_mod_exc_sup_qsw_blocked_force_born}, by
 % 2.) receiver electromechanical transfer functions
 %the receiver electromechanical transfer functions
 %$h_{m}^{(\text{rx})}$ according to
 %the transfer relations
 %\eqref{eqn:lin_mod_scan_config_trans_array_transfer_v_rx} yield
 % 3.) Born approximation of the recorded RF voltage signals
 %the desired signals.
}%
{lin_mod_v_rx_sgn_proc_chain}
%TODO: does contrast source include $\munderbar{k}^{2}$?

%---------------------------------------------------------------------------------------------------------------
% 1.) RF voltage signals generated by all array elements
%---------------------------------------------------------------------------------------------------------------
% a) array elements transduce the compressive blocked forces exerted by the free-space scattered acoustic pressure fields on the planar faces into the RF voltage signals
% article:Schiffner2018, Sect. III. Linear Physical Model for the Pulse-Echo Measurement Process / Sect. A. Pulse-Echo Measurement Process
% - The \ac{UI} system sequentially performs $N_{\text{in}} \in \N$ independent pulse-echo measurements using a planar transducer array
%   (cf. \cref{fig:lin_mod_scan_configuration,tab:lin_mod_scan_config_instrum_params}).
% - Each measurement begins at the time instant $t = 0$ and triggers
%   the concurrent recording of the \ac{RF} VOLTAGE SIGNALS $\tilde{u}_{m}^{(\text{rx}, n)}: \setsymbol{T}_{ \text{rec} }^{(n)} \mapsto \R$ GENERATED BY
%   ALL ARRAY ELEMENTS $m \in \setconsnonneg{ N_{\text{el}} - 1 }$ in the specified time interval [...].
% book:Schmerr2015, Sect. 1.3: Modeling Ultrasonic Phased Array Systems
% - When the acoustic pulses generated by the nth element interact with
%   scattering objects (such as surfaces of a component being inspected or flaws) and travel back to the array
%   they are RECEIVED BY THE MTH ELEMENT OF THE ARRAY AND CONVERTED TO A RECEIVED VOLTAGE PULSE. (p. 8)
% - This RECEIVING TRANSFER FUNCTION is a function of
%   [1.)] the electrical impedance and gain present in the RECEIVING CIRCUITS,
%   [2.)] the WIRING/CABLING present, and
%   [3.)] the electrical impedance and sensitivity of the mth PIEZOELECTRIC ELEMENT [Schmerr-Song]. (p. 9)
The array elements transduce
% 1.) compressive blocked forces
% book:Schmerr2015, Sect. 9.2 Linear System Modeling and Sound Reception
% - In fact, if we assume that we can represent the interaction of
%   the incident waves with
%   the element surfaces in these cases as PLANE WAVE INTERACTIONS, then
%   [1.)] the BLOCKED FORCE (the FORCE AT A PLANE RIGID, IMMOBILE SURFACE) is just
%   TWICE THE FORCE, F_{inc}( ω ), generated on the face of the array element by the INCIDENT WAVES only (i.e. when the element receiving surface is absent) and
%   [2.)] the FREE-SURFACE VELOCITY (the velocity at a plane stress-free surface) is
%   TWICE THE VELOCITY generated by only the incident waves, v_{inc}( ω ), so that in both the immersion and the contact cases
%   the acoustic equivalent sources shown in Figs. 9.8 and 9.11 become (see [Schmerr-Song] for an explicit proof in the immersion case;
%   the contact case follows in a similar manner):
%   [  F_{B}( ω ) = 2 F_{inc}( ω )
%     v_{fs}( ω ) = 2 v_{inc}( ω ). ] (9.13) (p. 187)
% book:Schmerr2015, Sect. 1.3: Modeling Ultrasonic Phased Array Systems
% - The BLOCKED FORCE appearing in Eq. (1.2) [ V_{mn}^{r}( f ) = t_{m}^{r}( f ) F_{mn}^{B}( f ) ] is defined as
%   the force exerted on the face of the receiving element when
%   the FACE OF THAT ELEMENT IS HELD RIGIDLY FIXED. (p. 9)
% - In [Schmerr-Song] and in Chap. 9 it is shown how this BLOCKED FORCE arises naturally in
%   describing the sound reception process for an ultrasonic system. (p. 9)
% book:Schmerr2007, Sect. 5.2: The Blocked Force
% - The force, F_{B}( ω ), appearing in both Eqs. (5.1) and (5.2) is a particular force acting on the receiving transducer called the BLOCKED FORCE.
%   This BLOCKED FORCE is defined as
%   the force that would be exerted on the receiving transducer if its face was held RIGIDLY FIXED (IMMOBILE). (p. 69)
% - To summarize: If we assume PLANE WAVE INTERACTIONS AT THE RECEIVING TRANSDUCER,
%   the BLOCKED FORCE, F_{B}( ω ), is just TWICE THE FORCE, F_{inc}( ω ), exerted by the waves INCIDENT on the AREA OF THE RECEIVER.
%   The force, F_{inc}( ω ), acting on S is computed from the INCIDENT WAVES as if the transducer were ABSENT. (p. 70)
% - We will also find it useful to use Eq. (5.7) [ F_{B}( ω ) = 2 F_{inc}( ω ) ] when obtaining the acoustic/elastic transfer function since then
%   we can model the pressure wave field of only the INCIDENT waves at the receiving transducer and
%   use Eq. (5.7) to obtain the BLOCKED FORCE, without having to consider explicitly any
%   more complex interactions of the incident waves with the receiving transducer. (p. 70)
% book:Cobbold2006, Chapter 5: Scattering of Ultrasound / Sect. 5.7: Pulse-Echo Response / Sect. 5.7.1: Continuum Model
% - Thus, the total force is given by [...], and the constant K will be assumed to be 2,
%   corresponding to an IDEAL REFLECTING PLANE. (p. 299)
% article:NgITUFFC2006: Modeling ultrasound imaging as a linear, shift-variant system
% III. The Wave Equation / D. The Force on the Receive Subaperture
% - In this subsection, we compute
%   THE SUMMATION OF THE SCATTERED PRESSURE FIELD OVER THE RECEIVE SUBAPERTURE. (p. 552)
% - Strictly speaking, this quantity, which will be denoted by F( \vect{x}_{0}, ω ), is
%   the FORCE EXERTED ON THE RECEIVE SUBAPERTURE [3]. (pp. 552, 553)
%   [3] article:ZempITUFFC2003: Linear system models for ultrasonic imaging: Application to signal statistics
% - If we represent the APODIZATION and FOCUSING on reception collectively in a single complex-valued term W( \vect{x}_{a}, ω ), then:
%   [ F( \vect{x}_{0}, ω ) = \int_{ \mathcal{A} } W( \vect{x}_{a}, ω ) P_{s}( \vect{x}_{0} + \vect{x}_{a}, \vect{x}_{0}, ω ) \text{d}^{2} \vect{x}_{a} ]. (12) (p. 553)
% article:StepanishenJASA1981: Pulsed transmit/receive response of ultrasonic piezoelectric transducers
% - The HARMONIC FORCE which ACTS on the TRANSDUCER as a result of the ACOUSTIC BACKSCATTERING PROCESS can
%   be expressed as: (5a), thus, (5b), where a DIFFRACTION CONSTANT of 2 has been used to ACCOUNT FOR
%   the RIGID BAFFLE EFFECT. (p. 1818)
% (physical unit: $[ F_{m} ] = \si{\newton \meter\tothe{\textit{d}-3}}$)
the compressive blocked forces exerted by
% 2.) free-space scattered acoustic pressure fields
the free-space scattered acoustic pressure fields
$p_{l}^{(\text{sc}, n)}: \R^{d} \mapsto \C$ on
% 3.) planar faces L_{m}
their planar faces
$L_{m} \subset \R^{d-1}$ into
% 4.) RF voltage signals
% book:Schmerr2015, Sect. 9.2: Linear System Modeling and Sound Reception
% - It then follows that we can REPLACE
%   [1.)] THE DETAILED MODELS of Fig. 9.13 with
%   [2.)] the SINGLE INPUT–SINGLE OUTPUT LTI SYSTEMS SHOWN IN FIG. 9.14 for the IMMERSION and CONTACT CASES. (p. 188)
% - The TRANSFER FUNCTION, t_{R}( ω ), of the IMMERSION CASE relates
%   [1.)] the BLOCKED FORCE INPUT, F_{B}( ω ), to
%   [2.)] the OUTPUT VOLTAGE, V_{e}( ω ). (p. 188)
% book:Schmerr2015, Sect. 1.3: Modeling Ultrasonic Phased Array Systems
% - We can relate the frequency components of
%   [1.)] this RECEIVED VOLTAGE, V_{mn}^{r}( f ), to
%   [2.)] the BLOCKED FORCE, F_{mn}^{B}( f ) (Fig. 1.10b), generated by the incident waves for
%   each pair of sending and receiving elements through
%   [3.)] a SOUND RECEPTION TRANSFER FUNCTION, t_{m}^{r}( f ), i.e. (1.2). (pp. 8, 9)
% article:LabyedITUFFC2014: TR-MUSIC inversion of the density and compressibility contrasts of point scatterers
% II. The Interelement Response Matrix for Point Scatterers With Density and Compressibility Contrasts
% - The spectrum of the ELECTRICAL SIGNAL MEASURED BY THE RECEIVING TRANSDUCER ELEMENT is given by [15], [16]
%   [ p_{m}( ω ) = W_{\text{r}}( ω ) \int_{ S_{\text{r}} } p_{\text{s}}( \vect{r}, ω ) \text{d} s ] (7), where
%   the integral is evaluated over the receiving-element area S_{\text{r}}, and
%   W_{\text{r}}( ω ) is the ELECTROMECHANICAL TRANSFER FUNCTION OF THE RECEIVER. (p. 17)
% article:NgITUFFC2006: Modeling ultrasound imaging as a linear, shift-variant system
% III. The Wave Equation / D. The Force on the Receive Subaperture
% - We recall from the physical description in Section II that
%   the RECEIVED RF VOLTAGE TRACE is obtained by
%   [1.)] summing the scattered pressure field over the receive subaperture and
%   [2.)] FILTERING THIS SUM BY THE ELECTROMECHANICAL RESPONSE OF THE PIEZOELECTRIC ELEMENTS. (p. 552)
% III. The Wave Equation / E. The RF Voltage Trace
% - We now MODEL THE ELECTROMECHANICAL CONVERSION of
%   [1.)] THE FORCE ON THE RECEIVE SUBAPERTURE INTO
%   [2.)] A VOLTAGE TRACE. (p. 553)
% - If we define the ELECTROMECHANICAL TRANSFER FUNCTION that models this conversion to be E_{m}( ω ) and
%   the VOLTAGE TRACE to be R( \vect{x}_{0}, ω ), we have
%   R( \vect{x}_{0}, ω ) = E_{m}( ω ) F( \vect{x}_{0}, ω ) [2], [3];
%   substituting in the expression for F( x_{0}, ω ) from (20) we get: (21). (p. 553)
% article:JensenJASA1991: A model for the propagation and scattering of ultrasound in tissue
% IV. CALCULATION OF THE RECEIVED SIGNAL
% - The received signal is
%   the SCATTERED PRESSURE FIELD INTEGRATED OVER THE TRANSDUCER SURFACE, convolved with
%   the ELECTROMECHANICAL IMPULSE RESPONSE E_{m}( t ) of the transducer. (p. 185)
% - The received signal is
%   [ p_{r}( \vect{r}_{5}, t ) = E_{m}( t ) * \int_{S} p_{s}( \vect{r}_{6} + \vect{r}_{5}, t ) d^{2} \vect{r}_{6} ]. (35) (p. 185)
% - Symbolically, this is written as
%   [ p_{r}( \vect{r}_{5}, t ) = v_{pe}( t ) *_{t} f_{m}( \vect{r}_{1} ) *_{r} h_{pe}( \vect{r}_{1}, \vect{r}_{5}, t ) ]. (45)
% - The PULSE-ECHO WAVELET is v_{pe}, which includes
%   the TRANSDUCER EXCITATION AND THE ELECTROMECHANICAL IMPULSE RESPONSE DURING EMISSION AND RECEPTION OF THE PULSE. (p. 186)
% - The term f_{m} accounts for
%   the INHOMOGENEITIES IN THE TISSUE DUE TO DENSITY AND PROPAGATION VELOCITY PERTURBATIONS that give rise to the scattered signal. (p. 186)
% - The term h_{pe} is the MODIFIED PULSE-ECHO SPATIAL IMPULSE RESPONSE that relates
%   the TRANSDUCER GEOMETRY TO THE SPATIAL EXTENT OF THE SCATTERED FIELD. (p. 186)
the \ac{RF} voltage signals
(cf. e.g.
\cite[Sect. 9.2]{book:Schmerr2015},
\cite{article:LabyedITUFFC2014,article:NgITUFFC2006,article:JensenJASA1991}%
)
\begin{equation}
 %--------------------------------------------------------------------------------------------------------------
 % recorded RF voltage signals provided by the receiving amplification networks
 %--------------------------------------------------------------------------------------------------------------
  u_{m, l}^{(\text{rx}, n)}
  =
  2 h_{m, l}^{(\text{rx})}
  \int_{ L_{m} }
    \chi_{m, l}^{(\text{rx})}( \vect{r}_{\rho} )
    p_{l}^{(\text{sc}, n)}( \vect{r}_{\rho}, 0 )
  \text{d} \vect{r}_{\rho}
 \label{eqn:lin_mod_scan_config_trans_array_transfer_v_rx}
\end{equation}
for
% 4.) all sequential pulse-echo measurements, all relevant discrete frequencies, and all array elements
all $( n, l, m ) \in \setconsnonneg{ N_{\text{in}} - 1 } \times \setsymbol{L}_{ \text{BP} }^{(n)} \times \setconsnonneg{ N_{\text{el}} - 1 }$, where
% 5.) receiver electromechanical transfer functions
$h_{m, l}^{(\text{rx})} \in \C$ denote
the electromechanical transfer functions and
% 6.) receiver apodization functions
$\chi_{m, l}^{(\text{rx})}: L_{m} \mapsto \C$ are
the apodization functions
(cf. \cref{tab:lin_mod_scan_config_instrum_params}).

%---------------------------------------------------------------------------------------------------------------
% 2.) Born approximation of the free-space scattered acoustic pressure fields
%---------------------------------------------------------------------------------------------------------------
% a) Born approximation uses the incident acoustic pressure fields to estimate the free-space scattered acoustic pressure fields
% book:Devaney2012, Chapter 6: Scattering theory / Sect. 6.7: The Born series / Subsect. 6.7.1: The Born approximation
% - The BORN APPROXIMATION Eq. (6.52) is seen to be
%   a LINEAR MAPPING FROM THE SCATTERING POTENTIAL V TO THE SCATTERED FIELD:
%   [ ] (6.53). (p. 256)
% - The inverse scattering problem within the BORN APPROXIMATION consists of inverting
%   the set of equations Eqs. (6.53) for the scattering potential from measurements of the scattered field obtained in
%   a suite of scattering experiments using a set of incident waves U(in)(r, ν). (p. 256)
% book:Natterer2001, Chapter 3: Tomography, Sect. 3.3: Diffraction Tomography
% - In order to derive the BORN APPROXIMATION,
%   we put u = u_{I} + \nu in (3.11) [reduced wave equation], where
%   \nu satisfies the Sommerfeld radiation condition and the differential equation (3.12). (p. 47)
% - Now we assume that
%   THE SCATTERED FIELD \nu IS SMALL IN COMPARISON WITH THE INCIDENT FIELD u_{I}.
%   Then we can neglect \nu on the right-hand side of (3.12), obtaining (3.13). (p. 47)
% - Using G_{n}, n = 2, 3, WE CAN REWRITE (3.13) AS (3.16). (p. 47)
% book:Kak2001, Chapter 6: Tomographic Imaging with Diffracting Sources / Sect. 6.2: Approximations to the Wave Equation / Sect. 6.2.1: The First Born Approximation
% - The integral of (37) [LS integral equation] is now written as (39) but
%   if the scattered field, u_{s}(r), is small compared to u_{0}(r) the effects of the second integral can be ignored to
%   arrive at the approximation (40). (p. 212)
% book:Born1999, Sect. 13.1.4: Multiple Scattering
% - As we noted earlier, if the scattering is weak (|U^{(s)}| << |U^{(i)}|),
%   one might expect to obtain a good approximation to the total field if U is replaced by U^{(i)} in
%   the integrand on the right-hand side of (54).
%   This gives the FIRST-ORDER BORN APPROXIMATION (57). (p. 708)
% book:Born1999, Sect. 13.1.2: The first-order Born approximation
% - This APPROXIMATE SOLUTION is generally referred to as
%   the BORN APPROXIMATION or, more precisely,
%   the FIRST-ORDER BORN APPROXIMATION (or just the FIRST BORN APPROXIMATION). (p. 700)
% article:JensenJASA1991: A model for the propagation and scattering of ultrasound in tissue
% II. CALCULATION OF THE SCATTERED FIELD
% - If G_{i} symbolizes the INTEGRAL OPERATOR REPRESENTING GREEN'S FUNCTION AND THE INTEGRATION, and
%   F_{op} the scattering operator, then
%   the FIRST-ORDER BORN APPROXIMATION can be written as
%   [ p_{s}( \vect{r}_{2}, t ) = G_{i} F_{op} p_{i}( \vect{r}_{1}, t_{1} ) ]. (21) (p. 184)
% article:GorePMB1977a: Ultrasonic backscattering from human tissue: A realistic model
% 2. The wave equation for ultrasound propagation through tissue
% - To FIRST APPROXIMATION (THE BORN APPROXIMATION) the scattered field
%   (measured at some location and time for which there is no contribution from the incident field) is given by
%   [ ... ] (4)
%   where the vector differentiation operator V is understood to be with respect to r’. (pp. 319, 320)
The \name{Born} approximation, which drives
% 1.) established image recovery methods in ultrafast UI
% article:ChernyakovaITUFFC2018: Fourier-Domain Beamforming and Structure-Based Reconstruction for Plane-Wave Imaging
% article:MoghimiradITUFFC2016: Synthetic Aperture Ultrasound Fourier Beamformation Using Virtual Sources
% proc:SchiffnerIUS2016a: A low-rate parallel Fourier domain beamforming method for ultrafast pulse-echo imaging
% article:LabyedITUFFC2014: TR-MUSIC inversion of the density and compressibility contrasts of point scatterers
% article:MontaldoITUFFC2009: Coherent plane-wave compounding for very high frame rate ultrasonography and transient elastography
% article:JensenUlt2006: Synthetic aperture ultrasound imaging [Dec.]
% article:ChengITUFFC2006: Extended high-frame rate imaging method with limited-diffraction beams [May]
% article:WalkerITUFFC2001: C- and D-weighted ultrasonic imaging using the translating apertures algorithm
% - The LACK OF MULTIPLE SCATTERING IN SOFT TISSUES IS COMMONLY ASSUMED WITH GOOD RESULTS [13].
% - [13] M. F. Insana and D. G. Brown, “Acoustic scattering theory applied to soft biological tissues,”
%   in Ultrasonic Scattering in Biological Tissues. K. K. Shung, Ed. Ann Arbor, MI: CRC Press, 1993, pp. 75–124.
% article:LuITUFFC1997: 2D and 3D High Frame Rate Imaging with Limited Diffraction Beams
the established image recovery methods in
ultrafast \ac{UI}
\cite{article:MoghimiradITUFFC2016,article:LabyedITUFFC2014,article:MontaldoITUFFC2009,article:JensenUlt2006,article:ChengITUFFC2006,article:LuITUFFC1997}, uses
% 2.) incident acoustic pressure fields
% article:JensenJASA1991: A model for the propagation and scattering of ultrasound in tissue
% III. CALCULATION OF THE INCIDENT FIELD
% - The INCIDENT FIELD is generated by the ultrasound transducer, assuming NO OTHER SOURCES EXIST IN THE TISSUE. (p. 184)
% - By this method [spatial impulse response]
%   the INCIDENT FIELD IS FOUND BY SOLVING THE WAVE EQUATION FOR THE HOMOGENEOUS CASE:
%   [ \nabla^{2} p_{1} - \frac{ 1 }{ c_{0}^{2} } \frac{ \partial^{2} p_{1} }{ \partial^{2} t } = 0 ]. (25) (p. 184)
the incident acoustic pressure fields
$p_{l}^{(\text{in}, n)}: \R^{d} \mapsto \C$ induced by
% 3.) transducer array
the transducer array in
the homogeneous fluid and governed by
% 4.) Helmholtz equations for the incident acoustic pressure fields
% book:Devaney2012, Chapter 6: Scattering theory / Sect. 6.1: Potential scattering theory
% - The BOUNDARY CONDITION SATISFIED BY THE FIELD U IS THAT IT REDUCES TO THE SUM OF
%   [1.)] AN INCIDENT WAVEFIELD U^{(in)} plus
%   [2.)] A SCATTERED FIELD U^{(s)} that is required to SATISFY THE SOMMERFELD RADIATION CONDITION (SRC); i.e.,
%   [ U( \vect{r}, \nu ) = U^{(in)}( \vect{r}, \nu ) + U_{+}^{(s)}( \vect{r}, \nu ) ~ U^{(in)}( \vect{r}, \nu ) + f( \vect{s}, \nu ) e^{ j k_{0} r } / r ] (6.4), where
%   [1.)] the INCIDENT WAVEFIELD U^{(in)} PROPAGATES IN THE UNIFORM BACKGROUND MEDIUM AND HENCE SATISFIES
%   THE HOMOGENEOUS HELMHOLTZ EQUATION
%   [ [ \Delta + k_{0}^{2} ] U^{(in)}( \vect{r}, \nu ) = 0 ], and
%   f( \vect{s}, \nu ) is the INDUCED SOURCE RADIATION PATTERN in the direction of the unit vector s = \hat{r} = r/r for the νth scattering experiment. (pp. 231, 232)
% article:NgITUFFC2006: Modeling ultrasound imaging as a linear, shift-variant system
% III. The Wave Equation / A. The Total Pressure Field
% - Because (2) is linear, we can write its GENERAL SOLUTION as the sum of
%   [1.)] the SOLUTION TO THE CORRESPONDING HOMOGENEOUS EQUATION (i.e., with the RHS set to zero) and
%   [2.)] any PARTICULAR SOLUTION [10]. (p. 550)
%   [10] G. F. Carrier and C. E. Pearson, Partial Differential Equations: Theory and Technique. New York: Academic, 1976.
% - Denoting
%   the solution to the HOMOGENEOUS EQUATION as P_{i}( \vect{x}, ω ) and
%   the PARTICULAR SOLUTION as P_{s}( \vect{x}, ω ), we, therefore, can write the TOTAL FIELD as:
%   [ P'( \vect{x}, ω ) = P_{i}( \vect{x}, ω ) + P_{s}( \vect{x}, ω ) ]. (5) (p. 550)
% - We see then that P_{i}( \vect{x}, ω ) IS THE PRESSURE FIELD THAT DEVELOPS IN THE ABSENCE OF ANY SCATTERERS which, by definition,
%   is the INCIDENT PRESSURE FIELD. (p. 550)
% book:Natterer2001, Chapter 3: Tomography, Sect. 3.3: Diffraction Tomography
% - In order to derive the BORN APPROXIMATION,
%   we put u = u_{I} + \nu in (3.11) [reduced wave equation], where
%   \nu satisfies the Sommerfeld radiation condition and the differential equation (3.12). (p. 47)
% book:Kak2001, Chapter 6: Tomographic Imaging with Diffracting Sources / Sect. 6.1: Diffracted Projections / Sect. 6.1.2: Inhomogeneous Wave Equation
% - We will CONSIDER THE FIELD, u(r), TO BE THE SUM OF TWO COMPONENTS, u_{0}(r) and u_{s}(r). (p. 210)
% - The COMPONENT u_{0}(F), known as the INCIDENT FIELD, is the field present without any inhomogeneities, or, equivalently, a solution to the equation
%   [ (\Delta + k_{0}^{2}) u_{0}(r) = 0 ] (30). (p. 210)
the \name{Helmholtz} equations
(cf. e.g.
\cite[(6.4)]{book:Devaney2012},    %
%\cite{article:NgITUFFC2006},		%
\cite[47]{book:Natterer2001},           %
\cite[(30)]{book:Kak2001}%
)
\begin{equation}
 %--------------------------------------------------------------------------------------------------------------
 % Helmholtz equations for the incident acoustic pressure fields
 %--------------------------------------------------------------------------------------------------------------
  \left( \Delta + {\munderbar{k}_{l}}^{2} \right)
  p_{l}^{(\text{in}, n)}( \vect{r} )
  = 0
 \label{eqn:lin_mod_sol_wave_eq_pde_p_in}
\end{equation}
to estimate
% 5.) free-space scattered acoustic pressure fields
% article:NgITUFFC2006: Modeling ultrasound imaging as a linear, shift-variant system
% III. The Wave Equation / A. The Total Pressure Field
% - We also know that the SCATTERED PRESSURE FIELD must obey (2), and so we can assign our particular solution P_{s}( \vect{x}, ω ) to
%   be the SCATTERED PRESSURE FIELD. (p. 550)
% book:Kak2001, Chapter 6: Tomographic Imaging with Diffracting Sources / Sect. 6.1: Diffracted Projections / Sect. 6.1.2: Inhomogeneous Wave Equation
% - The component u_{s}(r), known as the SCATTERED FIELD, will be that part of the total field that can be ATTRIBUTED SOLELY TO THE INHOMOGENEITIES. (p. 210)
the free-space scattered acoustic pressure fields as
(cf. e.g.
\cite[(6.53)]{book:Devaney2012},		% term: "Born approximation" (checked!)
%\cite[268, 287]{book:Cobbold2006},		% term: "Born approximation" (checked!)
\cite[(3.16)]{book:Natterer2001},		% term: "Born approximation", plane-wave insonification
\cite[(40)]{book:Kak2001},			% term: "first Born approximation"
\cite[(57)]{book:Born1999}%			% terms: first Born approximation, first-order Born approximation, Born approximation (checked!)
)
\begin{equation}
 %--------------------------------------------------------------------------------------------------------------
 % Born approximations of the free-space scattered acoustic pressure fields
 %--------------------------------------------------------------------------------------------------------------
  p_{l}^{(\text{sc}, n)}( \vect{r} )
  \approx
  {\munderbar{k}_{l}}^{2}
  \int_{ \Omega }
    \gamma^{(\kappa)}( \vect{r}' )
    p_{l}^{(\text{in}, n)}( \vect{r}' )
    g_{l}( \vect{r} - \vect{r}' )
  \text{d} \vect{r}'
 \label{eqn:lin_mod_v_rx_p_sc_born}
\end{equation}
for
% 6.) all sequential pulse-echo measurements and all relevant discrete frequencies
all $( n, l ) \in \setconsnonneg{ N_{\text{in}} - 1 } \times \setsymbol{L}_{ \text{BP} }^{(n)}$, where
% 7.) outgoing free-space Green's functions (two- and three-dimensional Euclidean spaces)
the outgoing free-space \name{Green}'s functions
\eqref{eqn:app_helmholtz_green_free_space_2_3_dim} account for
% 8.) diffraction
diffraction and
% 9.) monopole scattering
monopole scattering
(cf. Appendix \ref{app:helmholtz_green}).
% b) resulting single scattering is valid for weakly-scattering heterogeneous objects
% book:Cobbold2006, Chapter 5: Scattering of Ultrasound / Sect. 5.4: Integral Equation Methods / Sect. 5.4.3: Scattering Approximations
% - If the SCATTERING IS SUFFICIENTLY WEAK so that
%   THE SCATTERED PRESSURE IS MUCH LESS THAN THE INCIDENT PRESSURE,
%   THE INCIDENT WAVE WILL REMAIN VIRTUALLY UNCHANGED AS IT PROGRESSES THROUGH THE SCATTERING VOLUME, i.e.
%   p \approx p_{i}. (p. 287)
% - THIS IS KNOWN AS THE BORN APPROXIMATION and
%   it ENABLES THE SCATTERED PRESSURE TO BE EVALUATED without having to use, for example,
%   the method of successive approximations. (p. 287)
% book:Cobbold2006, Chapter 5: Scattering of Ultrasound
% - Since the integrand involves the sum of the incident and scattered fields, it is generally appropriate to make
%   the BORN APPROXIMATION in which the SCATTERED FIELD IS ASSUMED TO BE SMALL COMPARED TO THAT INCIDENT. (p. 268)
% article:NgITUFFC2006: Modeling ultrasound imaging as a linear, shift-variant system
% III. The Wave Equation
% - WE RESTRICT OURSELVES TO THE CASE OF WEAK SCATTERING in which
%   THE ENERGY OF THE SCATTERED WAVES IS MUCH LESS THAN THE ENERGY OF THE INCIDENT WAVES. (p. 550)
% III. The Wave Equation / C. The Scattered Pressure Field
% - Because we are dealing only with the case of weak scattering, we assume that
%   |P_{s}( \vect{x}, \vect{x}_{0}, ω )| \ll |P_{i}( \vect{x}, \vect{x}_{0}, ω )|. (p. 552)
% - P_{s}( \vect{x}, \vect{x}_{0}, ω ) in [ P'( \vect{x}, ω ) = P_{i}( \vect{x}, ω ) + P_{s}( \vect{x}, ω ) ] (5) then becomes negligible and
%   P'( \vect{x}, \vect{x}_{0}, ω ) \approx P_{i}( \vect{x}, \vect{x}_{0}, ω ). (p. 552)
% book:Born1999, Sect. 13.1.2: The first-order Born approximation
% - From expression (6) for the scattering potential it is clear that a medium will scatter weakly if its refractive index differs only slightly from unity.
%   Under these circumstances it is plausible to assume that one will obtain a good approximation to the total field U if the term U = U^{(i)} + U^{(s)} under
%   the integral in (16) is replaced by U^{(i)}. (pp. 699, 700)
% article:JensenJASA1991: A model for the propagation and scattering of ultrasound in tissue
% II. CALCULATION OF THE SCATTERED FIELD
% - Here [Born approximation],
%   p_{s} HAS BEEN SET TO ZERO IN (20) [sum of fields]. (p. 184)
% - Usually the SCATTERING FROM SMALL OBSTACLES IS CONSIDERED WEAK, so higher-order terms can be neglected. (p. 184)
% article:GorePMB1977a: Ultrasonic backscattering from human tissue: A realistic model
% 2. The wave equation for ultrasound propagation through tissue
% - For WEAK SCATTERING the sound field within V may be written as the sum of
%   the incident field p_{i}, and a weak scattered field p_{s}
%   [ p = p_{i} + p_{s} with \frac{ \abs{ p_{s} } }{ \abs{ p_{i} } } \ll 1 ]. (p. 319)
% 2.) single scattering
% article:NgITUFFC2006: Modeling ultrasound imaging as a linear, shift-variant system
% - Thus, in making the Born approximation, we have assumed implicitly that
%   MULTIPLY SCATTERED WAVES (i.e., waves scattered off a particle that are then scattered off other particles) ARE NEGLIGIBLE, and that
%   MULTIPLE SCATTERING CAN BE IGNORED [1], [2], [12]. (p. 552)
% book:Born1999, Sect. 13.1.4: Multiple Scattering
% - The physical significance of the successive terms is as follows:
%   The product U^{(i)}(r') F(r') d^{3}r', in [...] may be regarded as representing
%   the RESPONSE TO THE INCIDENT FIELD of the volume d^{3}r' around the point r' of the scatterer.
%   It acts as an EFFECTIVE SOURCE which makes a contribution U^{(i)}(r') F(r') G(r - r') d^{3}r' to
%   the field at another point, r, that may be situated either inside or outside V. (p. 709)
% - Evidently the Green's function G(r - r') acts as a propagator transferring the contribution from the point r' to the point r. (p. 709)
% - The integral over the volume V thus represents
%   the TOTAL CONTRIBUTION FROM ALL THE VOLUME ELEMENTS OF THE SCATTERER. (p. 709)
% -> THIS PROCESS IS KNOWN AS SINGLE SCATTERING and is illustrated in Fig. 13.7(a) (p. 709)
% article:GorePMB1977a: Ultrasonic backscattering from human tissue: A realistic model
% 2. The wave equation for ultrasound propagation through tissue
% - MULTIPLE SCATTERING EFFECTS HAVE BEEN NEGLECTED, which is CONSISTENT WITH THE ASSUMPTION OF WEAK SCATTERING, and the
%   possibility of a strongly reflecting interface within the region of interest V is clearly excluded. (p. 320)
The resulting single scattering is valid for
% 1.) weakly-scattering lossy heterogeneous objects
weakly-scattering heterogeneous objects, i.e.
% 2.) validity condition
% article:DevaneyJASA1985: Variable density acoustic tomography
% - The validity of the FIRST BORN APPROXIMATION clearly requires that
%   the STRENGTH OF THE SCATTERED FIELD COMPONENT OF THE PRESSURE FIELD (6) REMAINS SMALL THROUGHOUT THE VOLUME OF THE OBJECT (scatterer).
$\tabs{ p_{l}^{(\text{sc}, n)}( \vect{r} ) } \ll \tabs{ p_{l}^{(\text{in}, n)}( \vect{r} ) }$ for
% 3.) all object points
all $\vect{r} \in \Omega$.
%neglects the interactions of % 1.) scattered fields the scattered fields with the heterogeneous object and
% c) weakly-scattering heterogeneous objects feature both small absolute values of the compressibility fluctuations and small acoustic sizes
% article:WangJOSAA2011:
% - conditions for validity (6), (7), and (10)
% article:LiPIER2010: On the Validity of Born Approximation
% article:NgITUFFC2006: Modeling ultrasound imaging as a linear, shift-variant system
% - We restrict ourselves to the case of WEAK SCATTERING in which
%   THE ENERGY OF THE SCATTERED WAVES IS MUCH LESS THAN
%   THE ENERGY OF THE INCIDENT WAVES. (p. ?)
% book:Born1999, Sect. 13.1.2: The first-order Born approximation
% - From expression (6) for the SCATTERING POTENTIAL it is clear that
%   a MEDIUM WILL SCATTER WEAKLY if its REFRACTIVE INDEX differs only slightly from unity.  (p. 699)
% - Under these circumstances it is plausible to assume that one will obtain a good approximation to the total field U if
%   the term U = U^{(i)} + U^{(s)} under the integral in (16) [LS equation] is replaced by U^{(i)}. (pp. 699, 700)
% article:DevaneyJASA1985: Variable density acoustic tomography
% - This in turn requires that both
%   THE MAGNITUDE OF THE MATERIAL PARAMETERS $\gamma_{\kappa}$ and $\gamma_{\rho}$ be small and that
%   THE TOTAL VOLUME OF THE OBJECT BE SMALL.
% - This second condition is usually violated in tomographic applications where the objects can oftentimes be on the order of 100 wavelengths or more in extent.
% - Thus, the Rytov approximation is ideally suited to diffraction tomography of weakly inhomogeneous objects of arbitrary size.
These feature both
% 1.) small absolute value of the relative spatial fluctuations in compressibility
small absolute values of
% 1.) relative spatial fluctuations in the unperturbed compressibility
the compressibility fluctuations
\eqref{eqn:lin_mod_mech_model_tis_simple_rel_fluctuations} and
% 2.) small spatial extent of the lossy heterogeneous object
small acoustic sizes
\cite{article:LiPIER2010},
\cite[708]{book:Born1999}.

%---------------------------------------------------------------------------------------------------------------
% 3.) Born approximation of the recorded RF voltage signals
%---------------------------------------------------------------------------------------------------------------
% a) insertion into the receiver electromechanical transfer relations yield the recorded RF voltage signals
The \name{Born} approximation of
the scattered acoustic pressure fields
\eqref{eqn:lin_mod_v_rx_p_sc_born} estimates
% 2.) recorded RF voltage signals
the recorded \ac{RF} voltage signals
\eqref{eqn:lin_mod_scan_config_trans_array_transfer_v_rx} as
% 3.) Fredholm integral equations of the first kind
% article:AltürkJIASF2017: On Multidimensional Fredholm Integral Equations of the First Kind
% - MULTIDIMENSIONAL FREDHOLM INTEGRAL EQUATION OF THE FIRST KIND are of the form
%   [ . ] (1.1), where
%   x = ( x_{1}, x_{2}, ..., x_{n} ), t = ( t_{1}, t_{2}, ..., t_{n} ), Ω = [ a_{1}, b_{1} ] × ... × [ a_{n}, b_{n} ] ⊂ R^{n},
%   f(x) is the data function, K(x, t) is the kernel, F(φ(t)) is a linear or a nonlinear function of φ(t), and
%   φ is the only unknown in (1.1) that we wish to determine. (p. 85)
% - FREDHOLM INTEGRAL EQUATIONS OF THE FIRST KIND ARE USUALLY ILL-POSED IN THE HADAMARD SENSE [12], that is,
%   the [1.)] existence and [2.)] uniqueness of the solution and
%   [3.)] continuous dependency of the data function to the solution ARE NOT GUARANTEED for
%   ill-posed problems. (p. 85)
% book:Hansen2010, Chapter 2: Meet the Fredholm Integral Equation of the First Kind / Sect. 2.2: Properties of the Integral Equation
% - The FREDHOLM INTEGRAL EQUATION OF THE FIRST KIND TAKES THE GENERIC FORM
%   [ . ] (2.2). (p. 7)
% - Here, both the kernel K and the right-hand side g are KNOWN FUNCTIONS, while f is the unknown function. (p. 7)
% - This equation establishes a linear relationship between the two functions f and g, and
%   the kernel K describes the precise relationship between the two quantities. (p. 7)
% - Thus, the function K describes the underlying model.
% - In Chapter 7 we will encounter FORMULATIONS OF (2.2) IN MULTIPLE DIMENSIONS, but
%   our discussion until then will focus on the ONE-DIMENSIONAL CASE.
% book:Hansen2010, Chapter 7: Regularization Methods at Work: Solving Real Problems / Sect. 7.5: Deconvolution in 2D-Image Deblurring
% - In the continuous setting of Chapter 2, image deblurring is
%   a FIRST-KIND FREDHOLM INTEGRAL EQUATION of the generic form
%   [...], in which
%   the two functions f(t) and g(s) that represent the sharp and blurred images are both functions of
%   TWO SPATIAL VARIABLES s = (s1, s2) and t = (t1, t2). (p. 144)
% article:GroetschJOPCS2007: Integral equations of the first kind, inverse problems and regularization: a crash course
the \name{Fredholm} integral equations of
the first kind
%(cf. e.g.
%\cite[(1.1)]{article:AltuerkJIASF2017},
%\cite[(2.2)]{book:Hansen2010}%
%)
\begin{subequations}
\label{eqn:lin_mod_v_rx_born}
\begin{equation}
 %--------------------------------------------------------------------------------------------------------------
 % a) Born approximation of the recorded RF voltage signals
 %--------------------------------------------------------------------------------------------------------------
  u_{m, l}^{(\text{rx}, n)}
  \approx
  u_{m, l}^{(\text{B}, n)}
  =
  \dopobservation{ m, l }{ p_{l}^{(\text{in}, n)} }{ \gamma^{(\kappa)} }{1}
 \label{eqn:lin_mod_v_rx_born_expression}
\end{equation}
for
% 5.) all sequential pulse-echo measurements, all relevant discrete frequencies, and all array elements
all $( n, l, m ) \in \setconsnonneg{ N_{\text{in}} - 1 } \times \setsymbol{L}_{ \text{BP} }^{(n)} \times \setconsnonneg{ N_{\text{el}} - 1 }$, where
% 6.)
the properties of
% 1.) pulse-echo measurement process (single pulse-echo measurement, monofrequent, single transducer element)
the pulse-echo measurement process
\begin{equation}
 %--------------------------------------------------------------------------------------------------------------
 % b) pulse-echo measurement process (single pulse-echo measurement, monofrequent, single transducer element)
 %--------------------------------------------------------------------------------------------------------------
  \dopobservation{ m, l }{ p_{l}^{(\text{in}, n)} }{ \gamma^{(\kappa)} }{1}
  =
  - {\munderbar{k}_{l}}^{2} h_{m, l}^{(\text{rx})}
  \int_{ \Omega }
    \gamma^{(\kappa)}( \vect{r} )
    p_{l}^{(\text{in}, n)}( \vect{r} )
    \varUpsilon_{m, l}^{(\text{rx})}( \vect{r} )
  \text{d} \vect{r}
 \label{eqn:lin_mod_v_rx_born_obs_proc}
\end{equation}
significantly depend on
% 2.) incident waves
the incident waves
(cf. \cref{fig:lin_mod_v_rx_sgn_proc_chain}), and
% 4.) apodized spatial receive functions
the apodized spatial receive functions
\begin{equation}
 %--------------------------------------------------------------------------------------------------------------
 % c) apodized spatial receive functions
 %--------------------------------------------------------------------------------------------------------------
  \varUpsilon_{m, l}^{(\text{rx})}( \vect{r} )
  =
  - 2
  \int_{ L_{m} }
    \chi_{m, l}^{(\text{rx})}( \vect{r}_{\rho}' )
    g_{l}( \vect{r}_{\rho}' - \vect{r}_{\rho}, - r_{d} )
  \text{d} \vect{r}_{\rho}',
 \label{eqn:lin_mod_exc_sup_qsw_volt_rx_spat_trans}
\end{equation}
\end{subequations}
which correspond to
% 7.) spatial impulse responses in the time domain
% article:ChengUlt2011: A new algorithm for spatial impulse response of rectangular planar transducers
% 1. Introduction
% - Over the years,
%   ULTRASOUND SOURCES OF DIFFERENT GEOMETRICAL SHAPES, such as
%   CIRCLES [3], RECTANGLES [5–7], TRIANGLES [8], POLYGONS [9] AND CURVED STRIPS [10] have been studied. (p. 229)
% - The solution for spatial impulse response is HIGHLY DEPENDENT ON THE SHAPE OF THE SOURCES. (p. 229)
% - All the described approaches require either approximations or complex geometrical considerations.
%   Therefore, a simplified and exact solution is needed. (p. 230)
% coll:Jensen2002, Sect. 5: Spatial Impulse Responses / Sect. 5.2: Basic Theory
% - The INTEGRAL in this equation, (33) is called
%   the SPATIAL IMPULSE RESPONSE and CHARACTERIZES the 3-DIMENSIONAL EXTENT OF THE FIELD for
%   a PARTICULAR TRANSDUCER GEOMETRY. (p. 152)
% coll:Jensen2002, Sect. 5: Spatial Impulse Responses / Sect. 5.1: Fields in Linear Acoustic Systems
% - A VOLTAGE EXCITATION of the TRANSDUCER with a DELTA FUNCTION will give rise to a PRESSURE FIELD that
%   is measured by the hydrophone. The MEASURED RESPONSE is the ACOUSTIC IMPULSE RESPONSE for
%   this particular system with the given setup. (p. 149)
% - The SPATIAL IMPULSE RESPONSE is then found by observing the pressure waves at
%   a fixed position in space over time by having all the spherical waves pass the point of observation and
%   summing them. (p. 150)
% coll:Jensen2002, Sect. 5: Spatial Impulse Responses
% - Thus, a more accurate and general solution
%   [relation between the oscillation of the transducer surface and the ultrasound field] is needed,
%   and this is developed here. (p. 149)
% - The approach is based on the CONCEPT OF SPATIAL IMPULSE RESPONSES developed by
%   Tupholme [7] and Stepanishen [8,9]. (p. 149)
% article:JensenJASA1999: A new calculation procedure for spatial impulse responses in ultrasound
% INTRODUCTION
% - The impulse response has been found for a number of geometries
%   (round flat piston [2], round concave [4,5], flat rectangle [6,7], and flat triangle [8]). (p. 3266)
% article:JensenJASA1991: A model for the propagation and scattering of ultrasound in tissue
% III. CALCULATION OF THE INCIDENT FIELD
% - The function (31) is called the SPATIAL IMPULSE RESPONSE and
%   it RELATES the TRANSDUCER GEOMETRY TO THE ACOUSTICAL FIELD. (p. 185)
the spatial impulse responses in
the time domain
\cite{coll:Jensen2002,article:JensenJASA1991}, characterize
% 8.) anisotropic directivities of the planar faces
the anisotropic directivities of
the planar faces.

% book:Devaney2012, Chapter 6: Scattering theory / Sect. 6.1: Potential scattering theory
% - We should note that the INCIDENT FIELD WILL, in fact, BE PRODUCED BY SOME SOURCE RADIATING IN THE BACKGROUND MEDIUM and,
%   hence, will actually SATISFY THE INHOMOGENEOUS HELMHOLTZ EQUATION. (p. 232)
% - However, WE ASSUME THAT THIS SOURCE IS WELL SEPARATED FROM THE SCATTERER so that
%   THE FIELD U^{(in)} WILL SATISFY THE HOMOGENEOUS HELMHOLTZ EQUATION AT LEAST WITHIN THE SCATTERER VOLUME τ0. (p. 232)
% - Moreover, as we discussed in our treatment of the angular-spectrum expansion in Section 4.2 of Chapter 4,
%   the FIELD RADIATED BY A COMPACTLY SUPPORTED SOURCE CAN BE ACCURATELY APPROXIMATED AS
%   A FREE FIELD at distances that are MORE THAN A FEW WAVELENGTHS FROM THE SOURCE SUPPORT VOLUME. (p. 232)
% - Thus, insofar as the POTENTIAL SCATTERING PROBLEM is concerned
%   the INCIDENT FIELD CAN BE MODELED AS A FREE FIELD that
%   SATISFIES THE HOMOGENEOUS HELMHOLTZ EQUATION OVER ALL OF SPACE. (p. 232)
%the lossy heterogeneous object exceeds
%a few wavelengths
%\cite[232]{book:Devaney2012}.
