%---------------------------------------------------------------------------------------------------------------
% 1.) overview of the physical model
%---------------------------------------------------------------------------------------------------------------
% a) proposed d-dimensional physical model predicts the RF voltage signals provided by the instrumentation from the interactions of arbitrary incident waves with the human body
% article:NgITUFFC2006: Modeling ultrasound imaging as a linear, shift-variant system
% I. Introduction
% - The paper by Gore and Leeman [1] is one of the first publications to have developed
%   a REALISTIC MODEL FOR ULTRASONIC BACKSCATTERING IN HUMAN TISSUE by assuming
%   WEAK SCATTERING and a windowed monochromatic separable incident pulse. (p. 549)
%   [1] article:GorePMB1977a: Ultrasonic backscattering from human tissue: A realistic model
% - A more thorough analysis was carried out by Jensen [2], who derived
%   the wave equation from first principles and solved it to obtain
%   an ANALYTIC EXPRESSION FOR THE BACKSCATTERED RADIO-FREQUENCY (RF) TRACE IN THE TIME DOMAIN. (p. 549)
%   [2] article:JensenJASA1991: A model for the propagation and scattering of ultrasound in tissue
% - OUR ANALYSIS HERE IS SIMILAR to the treatments by Gore and Leeman [1] and by Jensen [2]. (p. 549)
% - Like Jensen, OUR AIM IS TO EXPRESS
%   THE BACKSCATTERED RF TRACE as the result of
%   LINEARLY FILTERING A MAP OF THE ACOUSTIC INHOMOGENEITIES IN THE IMAGED REGION by
%   a TRANSFER FUNCTION DETERMINED BY THE GEOMETRY AND MECHANICS OF THE ULTRASONIC TRANSCEIVING PROBE. (p. 549)
% - We restrict ourselves to consider LINEAR WAVE PROPAGATION ONLY. (p. 549)
% article:JensenJASA1991: A model for the propagation and scattering of ultrasound in tissue
% Abstract
% - The integral solution to the wave equation is combined with
%   A GENERAL DESCRIPTION OF THE FIELD FROM TYPICAL TRANSDUCERS USED IN CLINICAL ULTRASOUND to yield
%   A MODEL FOR THE RECEIVED PULSE-ECHO PRESSURE FIELD. (p. 182)
% INTRODUCTION
% - Developing algorithms that take advantage of this sampling [transducer signa] necessitates
%   quantitative knowledge of the received pulse-echo pressure field. (p. 182)
% - The object of this paper is to DEVELOP SUCH A MODEL FOR THE RECEIVED PRESSURE FIELD. (p. 182)
% article:GorePMB1977a: Ultrasonic backscattering from human tissue: A realistic model
% 1. Introduction
% - For a true understanding of the ULTRASOUND-TISSUE INTERACTION,
%   [1.)] a PRECISE SPECIFICATION OF THE SCATTERING ELEMENTS IN TISSUE (A TISSUE MODEL) is important and
%   [2.)] a THEORETICAL TREATMENT such as that presented here can tell whether
%         a postulated tissue model may be CHARACTERIZED BY, or even reconstructed from, THE SCATTERING DATA. (p. 318)
% - It also indicates what EXPERIMENTS are, in fact, necessary, for unique tissue characterization but
%   the RELEVANCE OF THE MODEL TO HUMAN TISSUE (or rather, to the true scattering elements in that medium) CAN BE DEDUCED ONLY FROM
%   DIRECT OBSERVATIONS. (p. 318)
The proposed $d$-dimensional, $d \in \{ 2, 3 \}$, physical model uses
% 1.) interactions of arbitrary incident waves with the human body
% article:JensenJASA1991: A model for the propagation and scattering of ultrasound in tissue
% Abstract
% - Analytic expressions are found in the literature for a number of transducers, and
%   ANY TRANSDUCER EXCITATION CAN BE INCORPORATED INTO THE MODEL. (p. 182)
% INTRODUCTION
% - NO RESTRICTIONS are enforced on the transducer geometry or its excitation, and
%   ANALYTIC EXPRESSIONS FOR A NUMBER OF GEOMETRIES CAN BE INCORPORATED INTO THE MODEL. (p. 182)
the interactions of
arbitrary incident waves with
the human body to predict
% 2.) RF voltage signals provided by the instrumentation
the \ac{RF} voltage signals provided by
the instrumentation.
% c) instrumentation consists of a planar transducer array, connecting cables, and driving and receiving electric circuits
%The instrumentation consists of
% 1.) planar transducer array
%a planar transducer array,
% 2.) connecting cables
%connecting cables, and
% 3.) driving and receiving electric circuits
%driving and
%receiving electric circuits.
% d) interactions include diffraction, single monopole scattering, and the combined effects of power-law absorption and dispersion
The interactions include
% 1.) diffraction
diffraction,
% 2.) single monopole scattering within the Born approximation
single monopole scattering, and
% 3.) combined effects of power-law absorption and dispersion
the combined effects of
power-law absorption and
dispersion.
%% d) proposed method linearizes the ISP based on the first Born approximation
%The paper linearizes
%the physical model based on
%the \name{Born} approximation
% e) complexity of the boundary conditions and the lack of accurate data justify their simplifying approximation by the half-space r_{d} > 0 bounded by a rigid baffle of
% book:Cobbold2006, Sect. 2.1.3: Bounded Region With No Internal Sources / Specifying the Green's Function
% - As noted earlier,
%   a SUITABLE GREEN'S FUNCTION MUST SATISFY THE HELMHOLTZ EQUATION AS WELL AS THE BOUNDARY CONDITIONS. (p. 105)
% - These requirements make it difficult, if not impossible, to obtain
%   an analytic form of the field distribution without making any approximations. (p. 105)
% - However, FOR SOME SIMPLE GEOMETRIC SHAPES, EXPRESSIONS CAN BE OBTAINED. (p. 105)
% - From (2.23) it can be seen that if either G or \partial G / \partial n is zero over the bounding surface, then
%   either \Phi or \partial \Phi / \partial n needs to be specified; in other words, it avoids the need for specifying both quantities, which
%   can lead to overspecification difficulties. (p. 105)
% - As will be seen, these two types of boundary conditions, as summarized in Table 2.1, are
%   GOOD APPROXIMATIONS FOR THE CONDITIONS THAT OCCUR IN PRACTICE. (p. 105)
% - We shall FIRST CONSIDER THE HOMOGENEOUS DIRICHLET CONDITION in which
%   G = 0 and \partial G / \partial n \neq 0 on the surface S_{0}. (p. 105)
% - This means that THE VELOCITY POTENTIAL \Phi MUST BE SPECIFIED THROUGHOUT S_{0}.
%   But because the pressure phasor is proportional to \Phi, this corresponds to
%   PRESCRIBING THE PRESSURE DISTRIBUTION. (pp. 105, 106)
% - In the SECOND TYPE OF BOUNDARY CONDITION \Phi = 0 and \partial \Phi / \partial n \neq 0, which is
%   THE HOMOGENEOUS NEUMANN CONDITION. (p. 106)
% - But \partial \Phi / \partial n is equal to the particle velocity phasor normal to the surface pointing inward, i.e.,
%   in the opposite direction to \tilde{\vect{n}}. Consequently,
%   THE NEUMANN CONDITION CORRESPONDS TO SPECIFYING
%   THE NORMAL COMPONENT OF THE PARTICLE VELOCITY ON THE SURFACE. (p. 106)
% - BOUNDARY CONDITIONS INTERMEDIATE BETWEEN THESE TWO EXTREMES MAY ALSO BE PRESENT. (p. 106)
% book:Cobbold2006, Chapter 2: Theoretical Basis for Field Calculations
% - In the CLASSICAL APPROACH FOR LINEAR PROPAGATION,
%   a HOMOGENEOUS, ISOTROPIC, INVISCID FLUID is assumed and SIMPLIFYING ASSUMPTIONS are made in regard to
%   the BOUNDARY CONDITIONS. (p. 96)
% - TWO DIFFERENT CONDITIONS are considered, leading to TWO EQUATIONS that
%   DESCRIBE THE FIELD DISTRIBUTION in terms of SURFACE INTEGRALS OVER THE PRESCRIBED BOUNDARY. (p. 96)
% - They are commonly referred to as the RAYLEIGH-SOMMERFELD DIFFRACTION EQUATIONS. (p. 96)
% - Finally, although not described in this chapter, it should be noted that if the field region is divided into
%   a GRID OF SUFFICIENTLY FINE RESOLUTION, NUMERICAL METHODS provide a FLEXIBLE MEANS FOR DETERMINING THE FIELD. (pp. 96, 97)
% 2.) rigid baffle of infinite extent coinciding with the hyperplane r_{d} = 0
A rigid baffle on
the hyperplane $r_{d} = 0$, which embeds
% 1.) vibrating faces of the transducer elements
the vibrating faces of
the transducer elements and bounds
% 2.) half-space r_{d} > 0
the half-space $r_{d} > 0$, approximates
the complex boundary conditions.
%The extraordinary complexity of
%the boundary conditions and %, which include
% 1.) vibrating faces of the transducer elements
% TODO: real measurements: boundary not known, impedance boundary
% thesis:Schiffner2017, Sect. 1.3.: Mechanical Model for Human Soft Tissues
% - Furthermore, the boundary conditions have to account for
%   the CONTOUR OF THE INVESTIGATED BODY,
%   any OBJECTS IMPENETRABLE FOR THE ULTRASONIC WAVES, and
%   the PRESENCE OF THE INSTRUMENTATION.
%the vibrating faces of
%the transducer elements,
% 2.) embedding materials
%their embedding materials, and
% 3.) contour of the investigated body
%the contour of
%the investigated body, and
%the lack of
%accurate data justify
%their simplifying approximation by
%
% d) simple geometry enables an explicit analytical representation of the incident acoustic pressure field by surface integrals over the specified boundary values
%These enable
%explicit analytical representations of
%the incident and scattered waves by
%integral equations.
% TODO: common model in the literature!
% a) temporal Fourier domain simplifies the mathematical formulation of the model and the inclusions of absorption and the bandpass character
% article:NgITUFFC2006: Modeling ultrasound imaging as a linear, shift-variant system
% III. The Wave Equation
% - To SIMPLIFY THE MATHEMATICS,
%   the WAVE EQUATION AND ITS SOLUTION WILL BE EXPRESSED IN TERMS OF ANGULAR FREQUENCY ω instead of time t;
%   although LESS INTUITIVE,
%   this representation has the advantage of IMPROVING NOTATIONAL CLARITY by reducing
%   convolutions in the time domain to multiplications in the frequency domain. (p. 550)
The temporal \name{Fourier} domain simplifies
% 1.) mathematical formulation of the model
the mathematical formulation of
the model and enables
% 2.) inclusions of absorption and the bandpass transfer behavior
%the inclusions of
% 3.) combined effects of power-law absorption and dispersion
%absorption and
% 4.) bandpass transfer behavior
%the bandpass transfer behavior.
% b) temporal Fourier domain additionally enables the parallel processing of distinct discrete frequencies in numerical evaluations
%It additionally enables
% 1.) parallel processing of distinct discrete frequencies in numerical evaluations
the parallel processing of
distinct discrete frequencies in
numerical evaluations.
% TODO: compute Fourier coefficients of the received voltage signals

% formulation as integral equation
% article:ChewITAP1997: Fast solution methods in electromagnetics
% III. INTEGRAL EQUATION SOLVERS
% - Integral equation solvers usually involve a SMALLER NUMBER OF UNKNOWNS than
%   differential equation solvers because ONLY THE INDUCED SOURCES ARE UNKNOWNS, whereas
%   in a differential equation, the field is the unknown. (p. 535)
% - The high-computational complexity of the aforementioned solution schemes precludes their application to
%   the analysis of scattering from large structures. (p. 535)

% book:Devaney2012, Chapter 6: Scattering theory / Sect. 6.7.1: The Born approximation
% - The PHILOSOPHY EMPLOYED IN THE BORN APPROXIMATION TO THE INVERSE SCATTERING PROBLEM is that
%   WE SEEK AN EXACT INVERSION to
%   an APPROXIMATE formulation of the (forward) scattering problem. (p. 256)
%	-> The Born approximation is considered as an exact formulation of the forward scattering problem
%	-> exact inversion of the Born approximation is sought
%	-> ASSUMPTION: Born approximation reflects the underlying physics exactly
% - alternative: seek approximate solutions to the exact LS integral equation (p. 256)
% 	-> The problem with the latter approach is that
%	   it can be addressed only by ad-hoc and, generally, numerically based inversion schemes and
%   	   does not lend itself to any SYSTEMATIC TREATMENT OF THE INVERSE PROBLEM. (p. 256)
% book:Devaney2012, Chapter 6: Scattering theory / Sect. 6.7.1: The Born approximation
% - The ADVANTAGE OF THE LINEARIZED BORN MODEL is that
%   we can employ a SYSTEMATIC PROCEDURE TO DEVELOP ANALYTIC INVERSION SCHEMES that [...]. (p. 256)
% book:Devaney2012, Chapter 6: Scattering theory
% - Fortunately,
%   IN MANY APPLICATIONS the LIPPMANN-SCHWINGER equation can be APPROXIMATELY LINEARIZED, thus
%   YIELDING AN APPROXIMATE LINEAR FORMULATION OF THE SCATTERING PROBLEM that effectively reduces it to
%   a SIMPLE RADIATION PROBLEM of the type treated in Chapter 2. (p. 230)
% book:Kak2001, Sect. 6.1: Diffracted Projections
% - There are NO DIRECT METHODS FOR SOLVING the problem of WAVE PROPAGATION in an INHOMOGENEOUS MEDIUM;
%   in practice, APPROXIMATE FORMALISMS ARE USED that
%   allow the THEORY OF HOMOGENEOUS MEDIUM WAVE PROPAGATION to be used for
%   GENERATING SOLUTIONS IN THE PRESENCE OF WEAK INHOMOGENEITIES. (p. 204)
% - The better known among these approximate methods go under the names of BORN and RYTOV APPROXIMATIONS. (p. 204)
% article:DevaneyJASA1985: Variable density acoustic tomography
% - One way out of this difficulty [nonlinear inverse scattering problem] is to LINEARIZE THE PROBLEM.

%%%%%%%%%%%%%%%%%%%%%%%%%%%%%%%%%%%%%%%%%%%%%%%%%%%%%%%%%%%%%%%%%%%%%%%%%%%%%%%%%%%%%%%%%%%%%%%%%%%%%%%%%%%%%%%%
% 1.) pulse-echo measurement process
%%%%%%%%%%%%%%%%%%%%%%%%%%%%%%%%%%%%%%%%%%%%%%%%%%%%%%%%%%%%%%%%%%%%%%%%%%%%%%%%%%%%%%%%%%%%%%%%%%%%%%%%%%%%%%%%
\subsection{Pulse-Echo Measurement Process}
\label{subsec:lin_mod_measurement_process}
%%%%%%%%%%%%%%%%%%%%%%%%%%%%%%%%%%%%%%%%%%%%%%%%%%%%%%%%%%%%%%%%%%%%%%%%%%%%%%%%%%%%%%%%%%%%%%%%%%%%%%%%%%%%%%%%
% graphic: pulse-echo measurement process (two-dimensional Euclidean space)
%%%%%%%%%%%%%%%%%%%%%%%%%%%%%%%%%%%%%%%%%%%%%%%%%%%%%%%%%%%%%%%%%%%%%%%%%%%%%%%%%%%%%%%%%%%%%%%%%%%%%%%%%%%%%%%%
\graphic{linear_model/figures/latex/lin_mod_scan_configuration_fov.tex}%
{% a) illustration of the pulse-echo measurement process in the two-dimensional Euclidean space
 Pulse-echo measurement process in
 the two-dimensional Euclidean space, i.e. $d = 2$.
 % b) linear transducer array emits a broadband incident wave into a lossy homogeneous fluid
 A linear transducer array emits
 % 1.) broadband incident wave
 a broadband incident wave into
 % 2.) lossy homogeneous fluid with the unperturbed values of the compressibility \kappa_{0} and the mass density \rho_{0}
 a lossy homogeneous fluid with
 the unperturbed values of
 % 3.) unperturbed compressibility
 the compressibility
 $\kappa_{0} \in \Rplus$ and
 % 4.) mass density
 the mass density
 $\rho_{0} \in \Rplus$.
 % c) broadband incident wave penetrates an embedded lossy heterogeneous object, and its interactions with the unperturbed compressibility induce a scattered wave
 This wave penetrates
 % 1.) embedded lossy heterogeneous object
 an embedded lossy heterogeneous object represented by
 the bounded set
 $\Omega \subset \{ \vect{r} \in \R^{d}: r_{d} > 0 \}$, and
 its interactions with
 % 2.) unperturbed compressibility
 the unperturbed compressibility
 $\kappa_{1}: \Omega \mapsto \Rplus$ induce
 % 3.) scattered wave
 a scattered wave.
 % d) a portion of the scattered wave mechanically excites the faces of the array elements
 % TODO: transduce
 A portion of
 the latter mechanically excites
 % 1.) faces of the array elements
 the faces of
 the array elements that generate
 % 2.) RF voltage signals
 \ac{RF} voltage signals.
 % e) RF voltage signals enable the imaging of the specified FOV represented by the bounded set \Omega_{\text{FOV}}
 These enable
 the imaging of
 the specified \ac{FOV} represented by
 the bounded set
 $\Omega_{\text{FOV}} \subset \{ \vect{r} \in \R^{d}: r_{d} > 0 \}$.
}%
{lin_mod_scan_configuration}

%%%%%%%%%%%%%%%%%%%%%%%%%%%%%%%%%%%%%%%%%%%%%%%%%%%%%%%%%%%%%%%%%%%%%%%%%%%%%%%%%%%%%%%%%%%%%%%%%%%%%%%%%%%%%%%%
% table: geometric and electromechanical parameters of the instrumentation
%%%%%%%%%%%%%%%%%%%%%%%%%%%%%%%%%%%%%%%%%%%%%%%%%%%%%%%%%%%%%%%%%%%%%%%%%%%%%%%%%%%%%%%%%%%%%%%%%%%%%%%%%%%%%%%%
\begin{table*}[t!]
 \centering
 \caption{%
  Geometric and
  electromechanical parameters of
  the instrumentation for
  all $\delta \in \setcons{ d - 1 }$,
  $m \in \setconsnonneg{ N_{\text{el}} - 1 }$,
  $l \in \setsymbol{L}_{ \text{BP} }^{(n)}$.
 }
 \label{tab:lin_mod_scan_config_instrum_params}
 \small
 \begin{tabular}{%
  @{}%
  >{$}l<{$}%		01.) symbol
  p{0.925\textwidth}%	02.) meaning
  @{}%
 }
 \toprule
  \multicolumn{1}{@{}H}{Symbol} &
  \multicolumn{1}{H@{}}{Meaning}\\
  \cmidrule(r){1-1}\cmidrule(l){2-2}
 \addlinespace
 %--------------------------------------------------------------------------------------------------------------
 % a) geometric parameters of the planar transducer array
 %--------------------------------------------------------------------------------------------------------------
  % 1.) number of elements along the r_{\delta}-axis
  N_{\text{el}, \delta} &
  Number of
  elements along
  the $r_{\delta}$-axis,
  $N_{\text{el}, \delta} \in \N$\\
  % 2.) width of the vibrating faces along the r_{\delta}-axis
  w_{\text{el}, \delta} &
  Width of
  the vibrating faces along
  the $r_{\delta}$-axis,
  $w_{\text{el}, \delta} \in \Rplus$\\
  % 3.) width of the kerfs separating the elements along the r_{\delta}-axis
  k_{\text{el}, \delta} &
  Width of
  the kerfs separating
  the elements along
  the $r_{\delta}$-axis,
  $k_{\text{el}, \delta} \in \Rnonneg$\\
  % 4.) constant spacing between the centers of the adjacent vibrating faces along the r_{\delta}-axis (element pitch)
  \Delta r_{\text{el}, \delta} &
  Element pitch, i.e.
  constant spacing between
  the centers of
  the adjacent vibrating faces along
  the $r_{\delta}$-axis,
  $\Delta r_{\text{el}, \delta} = w_{\text{el}, \delta} + k_{\text{el}, \delta}$\\
  % 5.) center coordinates of the vibrating faces
  \vect{r}_{\text{el}, m} &
  Center coordinates of
  the vibrating faces,\par
  $\mathcal{M} = \{ \vect{r}_{\text{el}, m} \in \R^{d}: \vect{r}_{\text{el}, m} = \sum_{ \delta = 1 }^{ d - 1 } ( m_{\delta} - M_{\text{el}, \delta} ) \Delta r_{\text{el}, \delta} \uvect{\delta}, m_{\delta} \in \setconsnonneg{ N_{\text{el}, \delta} - 1 }, m = \mathcal{I}( \vect{m}, \vect{N}_{\text{el}} ) \}$, where\par
  % 5.a) shift of index along the r_{\delta}-axis
  $M_{\text{el}, \delta} = ( N_{\text{el}, \delta} - 1 ) / 2$ and
  % 5.b) forward index transform
  $\mathcal{I}( \vect{m}, \vect{N}_{\text{el}} ) = \sum_{ \delta = 1 }^{ d - 2 } m_{\delta} \prod_{ k = \delta + 1 }^{ d - 1 } N_{\text{el}, k} + m_{d-1}$\\
  % 6.) total number of elements
  N_{\text{el}} &
  Total number of
  elements,
  $N_{\text{el}} = \tabs{ \mathcal{M} } = \prod_{ \delta = 1 }^{ d - 1 } N_{\text{el}, \delta}$\\
  % 7.) coplanar compact sets specifying the (d-1)-dimensional vibrating faces on the hyperplane r_{d} = 0
  L_{m} &
  Coplanar compact sets specifying
  the $(d-1)$-dimensional vibrating faces on
  the hyperplane $r_{d} = 0$,\par
  $L_{m} = \prod_{ \delta = 1 }^{ d - 1 } [ r_{\text{el}, m, \delta} - 0.5 w_{\text{el}, \delta}; r_{\text{el}, m, \delta} + 0.5 w_{\text{el}, \delta} ] \subset \R^{d-1}$\\
  % 8.) transmitter apodization functions
  \chi_{m, l}^{(\text{tx})} &
  Transmitter apodization functions accounting for
  the heterogeneous normal velocities and
  the acoustic lens,
  $\chi_{m, l}^{(\text{tx})}: L_{m} \mapsto \C$\\
  % 9.) receiver apodization functions
  \chi_{m, l}^{(\text{rx})} &
  Receiver apodization functions accounting for
  the heterogeneous sensitivities and
  the acoustic lens,
  $\chi_{m, l}^{(\text{rx})}: L_{m} \mapsto \C$\\
 %--------------------------------------------------------------------------------------------------------------
 % b) electromechanical parameters of the instrumentation
 %--------------------------------------------------------------------------------------------------------------
  % 1.) transmitter electromechanical transfer functions
  h_{m, l}^{(\text{tx})} &
  Transmitter electromechanical transfer functions accounting for
  the driving circuits,
  the cables, and
  the radiating elements,
  $h_{m, l}^{(\text{tx})} \in \C$\\
  % 2.) receiver electromechanical transfer functions
  h_{m, l}^{(\text{rx})} &
  Receiver electromechanical transfer functions accounting for
  the receiving elements,
  the cables, and
  the amplifiers,
  $h_{m, l}^{(\text{rx})} \in \C$\\
 \addlinespace
 \bottomrule
 \end{tabular}
\end{table*}

%---------------------------------------------------------------------------------------------------------------
% 1.) pulse-echo measurement process and Fourier series representation of the recorded RF voltage signals
%---------------------------------------------------------------------------------------------------------------
% a) UI system sequentially performs N_{\text{in}} independent pulse-echo measurements using a planar transducer array
% article:JensenProgBMB2007: Medical ultrasound imaging
% 1 Fundamental Ultrasound Imaging
% [description]
% article:NgITUFFC2006: Modeling ultrasound imaging as a linear, shift-variant system
% II. Background
% - Conventional ultrasound imaging interrogates a medium with high-frequency, band-limited acoustic waves and detects
%   echoes scattered by inhomogeneities (also referred to as scatterers) within the medium. (p. 549)
% - A single probe placed in contact with the subject is used for both
%   [1.)] the generation of these waves and
%   [2.)] the reception of their echoes. (p. 549)
% - On the contact surface of a typical probe is found
%   an array of piezoelectric crystals or elements (referred to as the aperture), each of which behaves as
%   an electromechanical transducer. (p. 549)
% - A focused beam is produced by coherently exciting
%   a set of adjacent elements that we refer to as the transmit subaperture. (p. 549)
% - In a similar way, backscattered echoes are detected by adjacent elements in the receive subaperture;
%   these echoes are then coherently summed, and the result is filtered to produce
%   a single RF voltage trace [2], [3]1. (pp. 549, 550)
% - At each transmission,
%   the emitted wave propagating through the medium gives rise to an incident pressure field, and
%   the scattered waves give rise to a scattered pressure field. (p. 550)
% - It can be shown that, at any moment in time, the total pressure field is the sum of these two fields (see Section III-A). (p. 550)
The \ac{UI} system sequentially performs
% 1.) N_{\text{in}} independent pulse-echo measurements
$N_{\text{in}} \in \N$ independent pulse-echo measurements using
% 2.) planar transducer array
a planar transducer array
(cf. \cref{fig:lin_mod_scan_configuration,tab:lin_mod_scan_config_instrum_params}).
% b) each measurement begins at the time instant t = 0 and triggers the concurrent recording of the RF voltage signals in the specified time interval
Each measurement begins at
% 1.) time instant t = 0
the time instant
$t = 0$ and triggers
% 2.) concurrent recording
the concurrent recording of
% 3.) RF voltage signals
the \ac{RF} voltage signals
$\tilde{u}_{m}^{(\text{rx}, n)}: \setsymbol{T}_{ \text{rec} }^{(n)} \mapsto \R$ generated by
% 4.) all array elements
all array elements
$m \in \setconsnonneg{ N_{\text{el}} - 1 }$ in
% 5.) specified time interval
% book:Briggs1995, Chapter 2: The Discrete Fourier Transform / Sect. 2.4.: DFT Approximations to Fourier Series Coefficients [NONPERIODIC FUNCTIONS]
% - There seems to be NO AGREEMENT IN THE LITERATURE about whether the INTERVAL FOR DEFINING FOURIER SERIES should be
%   [1.)] the CLOSED INTERVAL [-A/2, A/2],
%   [2.)] a HALF-OPEN INTERVAL (-A/2, A/2], or
%   [3.)] the OPEN INTERVAL (-A/2, A/2). (p. 38)
% - Arguments can be made for or against any of these choices. (p. 38)
% - We will use the CLOSED INTERVAL [-A/2, A/2] throughout the book to emphasize the point (the subject of sermons to come!) that
%   IN DEFINING THE INPUT TO THE DFT, VALUES OF THE SAMPLED FUNCTION AT BOTH ENDPOINTS CONTRIBUTE TO THE INPUT. (p. 38)
the specified time interval
\begin{equation}
 %--------------------------------------------------------------------------------------------------------------
 % specified recording time intervals for the RF voltage signals generated by all array elements
 %--------------------------------------------------------------------------------------------------------------
  \setsymbol{T}_{ \text{rec} }^{(n)}
  =
  \bigl[ t_{\text{lb}}^{(n)}; t_{\text{ub}}^{(n)} \bigr],
 \label{eqn:lin_mod_scan_config_volt_rx_obs_interval}
\end{equation}
where
% 6.) lower bounds in the specified recording time intervals
$t_{\text{lb}}^{(n)} \in \Rnonneg$ and
% 7.) upper bounds in the specified recording time intervals
$t_{\text{ub}}^{(n)} > t_{\text{lb}}^{(n)}$ denote
its lower and
upper bounds,
respectively.
% c) finite recording times enable the representation of these signals by the Fourier series
% book:Mallat2009, Chapter 3: Discrete Revolution / Sect. 3.2: Discrete Time-Invariant Filters / Sect. 3.2.2: Fourier Series
% - Theorem 3.6 proves that if f \in \ell^{2}( \Z ), the FOURIER SERIES
%   [ \hat{f}( \omega ) = \sum_{ n = -\infty }^{ \infty } f[n] e^{ -i \omega n } ] (3.39)
%   can be interpreted as the decomposition of \hat{f} in an orthonormal basis of L^{2}[ -\pi, \pi ]. (p. 73)
% - The FOURIER SERIES COEFFICIENTS can thus be written as inner products in L^{2}[ -\pi, \pi ]:
%   [ f[n] = \inprod{ \hat{f}( \omega ) }{ e^{ -i \omega n } } = \frac{ 1 }{ 2 \pi } \int_{ -\pi }^{ \pi } \hat{f}( \omega ) e^{ i \omega n } d \omega. ] (3.40) (p. 73)
% - The energy conservation of orthonormal bases (A.10) yields a Plancherel identity:
%   [ \sum_{ n = -\infty }^{ \infty } \abs{ f[n] }^{2} = \norm{ \hat{f} }{2}^{2} = \frac{ 1 }{ 2 \pi } \int_{ -\pi }^{ \pi } \abs{ \hat{f}( \omega ) }^{2} d \omega. ] (3.41) (p. 73)
% - It was only in 1966 that Carleson [149] was able to prove that
%   if \hat{f} \in L^{2}[ -\pi, \pi ] THEN ITS FOURIER SERIES CONVERGES ALMOST EVERYWHERE.
%   The proof is very technical. (p. 74)
% book:Manolakis2005, Sect. 2.2.1: Fourier Transforms and Fourier Series
% Fourier series for continuous-time periodic signals
% - If a CONTINUOUS-TIME SIGNAL x_{c}(t) IS PERIODIC WITH FUNDAMENTAL PERIOD T_{p},
%   IT CAN BE EXPRESSED AS A LINEAR COMBINATION OF HARMONICALLY RELATED COMPLEX EXPONENTIALS
%   [ . ] (2.2.1) where
%   F_{0} = 1 / T_{p} is the FUNDAMENTAL FREQUENCY, and (2.2.2) which
%   are termed the FOURIER COEFFICIENTS, or the SPECTRUM of x_{c}(t). (p. 37)
% book:Briggs1995, Chapter 2: The Discrete Fourier Transform / Sect. 2.4.: DFT Approximations to Fourier Series Coefficients [DEFINITION}
% > Fourier Series <
% - Let f be a function that is PERIODIC WITH PERIOD A (also called A-PERIODIC). (p. 33)
% - Then the FOURIER SERIES ASSOCIATED WITH f is the trigonometric series
%   [ f( x ) ~ \sum_{ k = -\infty }^{ \infty } c_{k} e^{ j 2 \pi k x / A }, ] (2.12) where
%   the coefficients c_{k} are given by
%   [ c_{k} = \frac{ 1 }{ A } \int_{ - A / 2 }^{ A / 2 } f( x ) e^{ -j 2 \pi k x / A } dx. ] (2.13) (p. 33)
% - The symbol ~ means that the Fourier series is ASSOCIATED WITH THE FUNCTION f. (p. 33)
% - We would prefer to make the stronger statement that the series equals the function at every point, but
%   without imposing additional conditions on f, this cannot be said. (p. 33)
% book:Briggs1995, Chapter 2: The Discrete Fourier Transform / Sect. 2.4.: DFT Approximations to Fourier Series Coefficients [CONVERGENCE]
% - THEOREM 2.4. CONVERGENCE OF FOURIER SERIES.
%   Let f be a piecewise smooth A-periodic function. Then the Fourier series for f
%   [ \sum_{ k = -\infty }^{ \infty } c_{k} e^{ j 2 \pi k x / A} where c_{k} = \frac{ 1 }{ A } \int_{ - A / 2 }^{ A / 2 } f( x ) e^{ -j 2 \pi k x / A } dx ]
%   converges (pointwise) for every x to the value
%   [ \frac{ f( x+ ) + f( x- ) }{ 2 }. ] (p. 37)
% - Since at a point of continuity the right- and left-hand limits of a function must be equal, and equal to the function value, it follows that
%   AT ANY POINT OF CONTINUITY, the Fourier series CONVERGES TO f(x). (p. 38)
% - AT ANY POINT OF DISCONTINUITY, the series CONVERGES TO THE AVERAGE VALUE OF THE RIGHT- AND LEFT-HAND LIMITS. (p. 38)
% book:Briggs1995, Chapter 2: The Discrete Fourier Transform / Sect. 2.4.: DFT Approximations to Fourier Series Coefficients [NONPERIODIC FUNCTIONS]
% - So far, the Fourier series has been defined only for periodic functions. (p. 38)
% - However, an important case that arises often is that in which
%   f IS DEFINED AND PIECEWISE SMOOTH ONLY ON THE INTERVAL [-A/2, A/2];
%   perhaps f is not defined outside of that interval, or perhaps it is not a periodic function at all. (p. 38)
% - In order to handle this situation we need to know about
%   the PERIODIC EXTENSION of f, the function h defined by
%   [ h( x + sA ) = f( x ), x \in ( -A/2, A/2 ), s = 0, \pm 1, \pm 2, ... . ] (p. 38)
% - The PERIODIC EXTENSION of f is simply
%   the REPETITION OF f EVERY A UNITS ON BOTH SIDES OF THE INTERVAL [-A/2, A/2]. (p. 38)
% - Here is the IMPORTANT ROLE OF THE PERIODIC EXTENSION h:
%   if the FOURIER SERIES for f CONVERGES ON [-A/2, A/2], then it CONVERGES
%   [1.)] to the VALUE of f at POINTS OF CONTINUITY on (-A/2, A/2),
%   [2.)] to the AVERAGE VALUE of f at POINTS OF DISCONTINUITY on (-A/2, A/2),
%   [3.)] to the VALUE OF THE PERIODIC EXTENSION of f at POINTS OF CONTINUITY outside of (-A/2, A/2), and
%   [4.)] to the AVERAGE VALUE OF THE PERIODIC EXTENSION at POINTS OF DISCONTINUITY outside of (-A/2, A/2). (p. 38)
% - The PERIODIC EXTENSION is
%   the FUNCTION TO WHICH THE FOURIER SERIES OF f CONVERGES FOR ALL x provided
%   we use AVERAGE VALUES at POINTS OF DISCONTINUITY. (p. 38)
% - In particular, if f( -A/2 ) \neq f( A/2 ) then
%   the Fourier series converges to the average of the function values at the right and left endpoints
%   [ \frac{ 1 }{ 2 } \left[ f( -A/2+ ) + f( A/2- ) \right]. ] (p. 38)
% - These facts must be observed scrupulously when a function is sampled for input to the DFT. (p. 38)
The finite recording times
$T_{ \text{rec} }^{(n)} = \tabs{ \setsymbol{T}_{ \text{rec} }^{(n)} } = t_{\text{ub}}^{(n)} - t_{\text{lb}}^{(n)}$ enable
the representation of
% 1.) RF voltage signals
these signals by
% 2.) Fourier series
the \name{Fourier} series
%\footnote{
  % a) adjective "stable" indicates that neither inaccurate observations nor a sparsity defect result in huge recovery errors
%  validity
%}
(cf. e.g.
%\cite[(3.39/40)]{book:Mallat2009},
\cite[(2.2.1/2)]{book:Manolakis2005},
\cite[(2.12/13)]{book:Briggs1995}%
)
\begin{subequations}
\label{eqn:recovery_disc_freq_v_rx_Fourier_series}
\begin{equation}
 %--------------------------------------------------------------------------------------------------------------
 % Fourier series representation of the recorded RF voltage signals (time domain)
 %--------------------------------------------------------------------------------------------------------------
  \tilde{u}_{m}^{(\text{rx}, n)}( t )
  =
  u_{m, 0}^{(\text{rx}, n)}
  +
  2
  \dreal{
    \sum_{ l = 1 }^{ \infty }
      u_{m, l}^{(\text{rx}, n)}
      e^{ j \omega_{l} t }
  }{2}
 \label{eqn:recovery_disc_freq_v_rx_Fourier_series_sum}
\end{equation}
for
% 3.) all sequential pulse-echo measurements and all array elements
all $( n, m ) \in \setconsnonneg{ N_{\text{in}} - 1 } \times \setconsnonneg{ N_{\text{el}} - 1 }$, where
% 4.) discrete angular frequencies
$\omega_{l} = 2 \pi f_{l} = 2 \pi l / T_{ \text{rec} }^{(n)}$ denote
the discrete angular frequencies, and
% 5.) Fourier coefficients of the recorded RF voltage signals
\begin{equation}
 %--------------------------------------------------------------------------------------------------------------
 % Fourier coefficients of the recorded RF voltage signals
 %--------------------------------------------------------------------------------------------------------------
  u_{m, l}^{(\text{rx}, n)}
  =
  \frac{ 1 }{ T_{ \text{rec} }^{(n)} }
  \int_{ \setsymbol{T}_{ \text{rec} }^{(n)} }
    \tilde{u}_{m}^{(\text{rx}, n)}( t )
    e^{ -j \omega_{l} t }
  \text{d} t
 \label{eqn:recovery_disc_freq_v_rx_Fourier_series_coef}
\end{equation}
\end{subequations}
are
the complex-valued coefficients, whose
% 6.) conjugate even symmetry
conjugate even symmetry renders
% 7.) negative frequency indices
the negative frequency indices redundant.

%---------------------------------------------------------------------------------------------------------------
% 2.) bandpass characters of the recorded RF voltage signals / truncation of the Fourier series
%---------------------------------------------------------------------------------------------------------------
% a) bandpass characters of the recorded RF voltage signals define the sets of relevant discrete frequencies
The bandpass characters of
% 1.) recorded RF voltage signals
the recorded \ac{RF} voltage signals, which are described by
the lower and
upper frequency bounds
% 2.) lower frequency bounds
$f_{\text{lb}}^{(n)} \in \Rplus$ and
% 3.) upper frequency bounds
$f_{\text{ub}}^{(n)} \geq f_{\text{lb}}^{(n)} + 1 / T_{ \text{rec} }^{(n)}$,
respectively, define
% 4.) sets of relevant discrete frequencies
the sets of
relevant discrete frequencies
\begin{subequations}
\label{eqn:recon_disc_axis_f_discrete_BP}
\begin{equation}
 %--------------------------------------------------------------------------------------------------------------
 % sets of relevant discrete frequencies
 %--------------------------------------------------------------------------------------------------------------
  \setsymbol{F}_{ \text{BP} }^{(n)}
  =
  \Bigl\{
    f_{l} \in \Rplus:
    f_{l} = \frac{ l }{ T_{ \text{rec} }^{(n)} },
    l \in \setsymbol{L}_{ \text{BP} }^{(n)}
  \Bigr\}
 \label{eqn:recon_disc_axis_f_discrete_BP_set}
\end{equation}
for
% 5.) all sequential pulse-echo measurements
all $n \in \setconsnonneg{ N_{\text{in}} - 1 }$, where
% 6.) sets of admissible frequency indices
the admissible index sets are
\begin{equation}
 %--------------------------------------------------------------------------------------------------------------
 % sets of admissible frequency indices
 %--------------------------------------------------------------------------------------------------------------
  \setsymbol{L}_{ \text{BP} }^{(n)}
  =
  \left\{
    l \in \N:
    l_{\text{lb}}^{(n)} \leq l \leq l_{\text{ub}}^{(n)}
  \right\}
 \label{eqn:recon_disc_axis_f_discrete_BP_indices}
\end{equation}
with
the lower and
upper bounds
% a) lower bounds on the admissible frequency indices
% 1.) t_{\text{lb}}^{(n)} \in \Rnonneg and t_{\text{ub}}^{(n)} > t_{\text{lb}}^{(n)}
% => T_{ \text{rec} }^{(n)} = t_{\text{ub}}^{(n)} - t_{\text{lb}}^{(n)} > 0
% 2.) f_{\text{lb}}^{(n)} \in \Rplus and f_{\text{ub}}^{(n)} \geq f_{\text{lb}}^{(n)} + 1 / T_{ \text{rec} }^{(n)} > f_{\text{lb}}^{(n)} > 0
% => T_{ \text{rec} }^{(n)} f_{\text{lb}}^{(n)} > 0
% => l_{\text{lb}}^{(n)} = \dceil{ T_{ \text{rec} }^{(n)} f_{\text{lb}}^{(n)} }{1} \in \N
% b) upper bounds on the admissible frequency indices
% => T_{ \text{rec} }^{(n)} f_{\text{ub}}^{(n)} \geq T_{ \text{rec} }^{(n)} f_{\text{lb}}^{(n)} + 1
% => l_{\text{ub}}^{(n)} = \dfloor{ T_{ \text{rec} }^{(n)} f_{\text{ub}}^{(n)} }{1} \geq \dfloor{ T_{ \text{rec} }^{(n)} f_{\text{lb}}^{(n)} }{1} + 1 \geq \dceil{ T_{ \text{rec} }^{(n)} f_{\text{lb}}^{(n)} }{1} = l_{\text{lb}}^{(n)}
\begin{align}
 %--------------------------------------------------------------------------------------------------------------
 % a) lower bounds on the admissible frequency indices
 %--------------------------------------------------------------------------------------------------------------
  l_{\text{lb}}^{(n)}
  &=
  \dceil{ T_{ \text{rec} }^{(n)} f_{\text{lb}}^{(n)} }{1}
  & \text{and} & &
 %--------------------------------------------------------------------------------------------------------------
 % b) upper bounds on the admissible frequency indices
 %--------------------------------------------------------------------------------------------------------------
  l_{\text{ub}}^{(n)}
  &=
  \dfloor{ T_{ \text{rec} }^{(n)} f_{\text{ub}}^{(n)} }{1},
 \label{eqn:recon_disc_axis_f_discrete_BP_indices_lb_ub}
\end{align}
\end{subequations}
respectively.
% b) sets of relevant discrete frequencies enable the truncation of each Fourier series and the representation of each pulse-echo measurement by N_{\text{el}} N_{f, \text{BP}}^{(n)} complex-valued coefficients
These enable
the truncation of
% 1.) Fourier series representation of the recorded RF voltage signals (time domain)
each \name{Fourier} series
\eqref{eqn:recovery_disc_freq_v_rx_Fourier_series_sum}, where, defining
% 2.) effective bandwidths
the effective bandwidths
$B_{ u }^{(n)} = f_{\text{ub}}^{(n)} - f_{\text{lb}}^{(n)}$,
% 3.) number of relevant discrete frequencies
the number of
relevant discrete frequencies approximates
% 4.) effective time-bandwidth products
the effective time-bandwidth products
\begin{equation}
 %--------------------------------------------------------------------------------------------------------------
 % numbers of relevant discrete frequencies (effective time-bandwidth products)
 %--------------------------------------------------------------------------------------------------------------
  N_{f, \text{BP}}^{(n)}
  =
  \dabs{ \setsymbol{L}_{ \text{BP} }^{(n)} }{1}
  =
  l_{\text{ub}}^{(n)} - l_{\text{lb}}^{(n)} + 1
  \approx
  T_{ \text{rec} }^{(n)} B_{ u }^{(n)}
 \label{eqn:recon_disc_axis_f_discrete_BP_TB_product}
\end{equation}
for
% 5.) all sequential pulse-echo measurements
all $n \in \setconsnonneg{ N_{\text{in}} - 1 }$, and
the representation of
each pulse-echo measurement by
% 6.) Fourier coefficients of the recorded RF voltage signals
$N_{\text{el}} N_{f, \text{BP}}^{(n)}$ coefficients
\eqref{eqn:recovery_disc_freq_v_rx_Fourier_series_coef}.
% c) let the subscript l indicate an admissible frequency index in the sets of relevant discrete frequencies in the following
%In the following,
%the subscript
%$l \in \setsymbol{L}_{ \text{BP} }^{(n)}$ indicates
%an admissible frequency index.
% d) dependence on the superscript n is implicitly understood
%For the sake of
%notational lucidity,
%its dependence on
%the superscript $n$, which identifies
%the sequential pulse-echo measurement, is
%implicitly understood.


%%%%%%%%%%%%%%%%%%%%%%%%%%%%%%%%%%%%%%%%%%%%%%%%%%%%%%%%%%%%%%%%%%%%%%%%%%%%%%%%%%%%%%%%%%%%%%%%%%%%%%%%%%%%%%%%
% 2.) acoustic model for human soft tissues
%%%%%%%%%%%%%%%%%%%%%%%%%%%%%%%%%%%%%%%%%%%%%%%%%%%%%%%%%%%%%%%%%%%%%%%%%%%%%%%%%%%%%%%%%%%%%%%%%%%%%%%%%%%%%%%%
\subsection{Acoustic Model for Human Soft Tissues}
%\label{subsec:lin_mod_scan_config_mat_params}
%---------------------------------------------------------------------------------------------------------------
% 1.) physical model for soft tissue structures and its relevant acoustic material parameters
%---------------------------------------------------------------------------------------------------------------
% a) medical UI usually models soft tissue structures as quiescent, lossless, and heterogeneous fluids that linearly propagate small-amplitude disturbances of the stationary state as longitudinal waves
% article:JensenProgBMB2007: Medical ultrasound imaging
% 3. Anatomic ultrasound imaging
% - MODERN MEDICAL ULTRASOUND SCANNERS are used for imaging
%   NEARLY ALL SOFT TISSUE STRUCTURES IN THE BODY. (p. 155)
% coll:Jensen2002: Ultrasound Imaging and Its Modeling
% ABSTRACT
% - MODERN MEDICAL ULTRASOUND SCANNERS are used to image
%   NEARLY ALL SOFT TISSUE STRUCTURES IN THE BODY. (p. 135)
% article:MastJASA1997:
% - Ultrasonic pulse propagation through the HUMAN ABDOMINAL WALL was modelled using
%   the equations of motion for a LOSSLESS FLUID WITH VARIABLE SOUND SPEED AND DENSITY.
% article:JensenJASA1991: A model for the propagation and scattering of ultrasound in tissue
% INTRODUCTION
% - ULTRASOUND is used with great success in
%   the DIAGNOSIS OF ABNORMALITIES in SOFT TISSUE STRUCTURES IN THE HUMAN BODY. (p. 182)
% article:GorePMB1977a: Ultrasonic backscattering from human tissue: A realistic model
% 1. Introduction
% - In this paper,
%   A TISSUE MODEL IS CHOSEN ON THE BASIS OF SIMPLE BUT REALISTIC ASSUMPTIONS and
%   the scattering of a typical diagnostic pulse is calculated. (p. 318)
Medical \ac{UI} usually models
soft tissue structures%
\footnote{
  % a) strict definition of the term "soft tissue" (anatomy)
  The anatomic term \term{soft tissue} refers to
  tendons, ligaments, skin, nerves,
  % website:NCIDictionary2017: NCI Dictionary of Cancer Terms
  % - 'soft tissue': Refers to MUSCLE, FAT, FIBROUS TISSUE, blood vessels, or other SUPPORTING TISSUE of the BODY.
  % fibrous tissue =  the common connective tissue of the body, composed of yellow or white parallel elastic and collagen fibers.
  muscle, fat, fibrous tissue, blood vessels, or
  other supporting tissue of
  the body
  \cite[\term{soft tissue}]{website:NCIDictionary2017}.
  % b) loose definition of the term "soft tissue" (UI)
  In the context of \ac{UI}, however,
  the term additionally includes
  organs like
  liver, kidney, thyroid, brain, and
  the heart.
} as
% 1.) quiescent
% article:MastJASA1997:
% - The TISSUE WAS ASSUMED TO BE MOTIONLESS except for small acoustic perturbations.
quiescent,
% 2.) lossless
% article:JensenJASA1991: A model for the propagation and scattering of ultrasound in tissue
% INTRODUCTION
% - The model includes attenuation due to propagation and scattering, but not
%   THE DISPERSIVE ATTENUATION OBSERVED FOR PROPAGATION IN TISSUE. (p. 182)
% - This can, however, be incorporated into the model as indicated in Sec. VI. (p. 182)
% I. DERIVATION OF THE WAVE EQUATION
% - Our second assumption is that
%   NO HEAT CONDUCTION OR CONVERSION OF ULTRASOUND TO THERMAL ENERGY TAKE PLACE. (p. 182)
% article:GorePMB1977a: Ultrasonic backscattering from human tissue: A realistic model
% 1. Introduction
% - For simplicity, however, such ABSORPTION EFFECTS ARE NOT CONSIDERED HERE,
%   but the introduction of simple exponential absorption leads to only minor changes in the theory. (p. 318)
lossless, and
% 3.) heterogeneous
% article:NgITUFFC2006: Modeling ultrasound imaging as a linear, shift-variant system
% III. The Wave Equation
% - Our analysis necessarily begins by considering
%   the PARTIAL DIFFERENTIAL EQUATION (PDE) that describes
%   the PROPAGATION OF ACOUSTIC WAVES IN A NONUNIFORM MEDIUM. (p. 550)
% article:JensenJASA1991: A model for the propagation and scattering of ultrasound in tissue
% Abstract
% - An INHOMOGENEOUS WAVE EQUATION is derived describing PROPAGATION AND SCATTERING OF ULTRASOUND IN AN INHOMOGENEOUS MEDIUM. (p. 182)
heterogeneous fluids that linearly propagate
% 4.) small-amplitude approximation
% article:NgITUFFC2006: Modeling ultrasound imaging as a linear, shift-variant system
% I. Introduction
% - We restrict ourselves to consider LINEAR WAVE PROPAGATION ONLY. (p. 549)
% article:JensenJASA1991: A model for the propagation and scattering of ultrasound in tissue
% I. DERIVATION OF THE WAVE EQUATION
% - To obtain a solvable wave equation, some ASSUMPTIONS AND APPROXIMATIONS MUST BE MADE. (p. 182)
% - The first one states that the instantaneous acoustic pressure and density can be written as
%   [ P_{\text{ins}}( \vect{r}, t ) = P + p_{1}( \vect{r}, t ) ], (1)
%   [ \rho_{\text{ins}}( \vect{r}, t ) = \rho( \vect{r} ) + \rho_{1}( \vect{r}, t ) ], (2)
%   in which P is the mean pressure of the medium and \rho is the density of the undisturbed medium. (p. 182)
% - The PRESSURE VARIATION p_{1} IS CAUSED BY THE ULTRASOUND WAVE AND IS CONSIDERED SMALL compared to P. (p. 182)
% - The density change caused by the wave is \rho_{1}. (p. 182)
% - Both p_{1} and \rho_{1} are SMALL QUANTITIES OF FIRST ORDER. (p. 182)
small-amplitude disturbances of
the stationary state as
% 5.) longitudinal waves
% article:GorePMB1977a: Ultrasonic backscattering from human tissue: A realistic model
% 2. The wave equation for ultrasound propagation through tissue
% - ULTRASOUND PROPAGATION BY MODES OTHER THAN PURELY LONGITUDINAL IS NEGLECTED,
%   not only for reasons of simplicity, but also because
%   their significance is not well documented or understood for scattering from tissue. (p. 320)
longitudinal waves
\cite{article:NgITUFFC2006,coll:Jensen2002,article:JensenJASA1991,article:GorePMB1977a}.
% b) relevant acoustic material parameters are the unperturbed values of both the compressibility and the mass density (normalized by spatial averages)
% article:NgITUFFC2006: Modeling ultrasound imaging as a linear, shift-variant system
% III. The Wave Equation
% - We shall use the WAVE EQUATION that is found in [1] and [9], in which
%   the ACOUSTIC PROPERTIES OF THE MEDIUM ARE SPECIFIED IN TERMS OF
%   its DENSITY AND ADIABATIC COMPRESSIBILITY. (p. 550)
%   [1] article:GorePMB1977a: Ultrasonic backscattering from human tissue: A realistic model
%   [9] book:Morse1986: Theoretical Acoustics
% III. The Wave Equation / A. The Total Pressure Field
% - In the ABSENCE OF ANY SCATTERERS,
%   WE CONSIDER OUR MEDIUM TO BE UNIFORM with DENSITY \rho_{o} AND ADIABATIC COMPRESSIBILITY \kappa_{0}. (p. 550)
% - The PRESENCE OF SCATTERERS in the medium may be modeled by adding
%   SPATIALLY-DEPENDENT TERMS ∆\rho( \vect{x} ) and ∆\kappa( \vect{x} ) to
%   THE DENSITY AND THE COMPRESSIBILITY, respectively. (p. 550)
% article:JensenJASA1991: A model for the propagation and scattering of ultrasound in tissue
% Abstract
% - The SCATTERING TERM IS A FUNCTION OF
%   [1.)] DENSITY AND
%   [2.)] PROPAGATION VELOCITY PERTURBATIONS. (p. 182)
% INTRODUCTION
% - In MEDICAL ULTRASOUND, a pulse is emitted into the body and
%   is SCATTERED AND REFLECTED BY
%   [1.)] DENSITY AND
%   [2.)] PROPAGATION VELOCITY PERTURBATIONS. (p. 182)
% - The RECEIVED FIELD CAN BE FOUND BY SOLVING AN APPROPRIATE WAVE EQUATION. (p. 182)
% - This has been done in a number of papers. [1,2] (p. 182)
% - Gore and Leeman [1] considered a wave equation where
%   THE SCATTERING TERM WAS A FUNCTION OF THE
%   [1.)] ADIABATIC COMPRESSIBILITY AND
%   [2.)] THE DENSITY.
%   The transducer was modeled by an axial and lateral pulse that were separable. (p. 182)
%   [1] article:GorePMB1977a: Ultrasonic backscattering from human tissue: A realistic model
% - Fatemi and Kak [2] used a wave equation where
%   THE SCATTERING ONLY ORIGINATED FROM VELOCITY FLUCTUATIONS, and
%   the transducer was restricted to be circularly symmetric and unfocused (flat). (p. 182)
%   [2] M. Fatemi and A. C. Kak, "Ultrasonic B-scan imaging: Theory of image formation and a technique for restoration," Ultrason. Imag. 2, 1-47 (1980).
% - The scattering term for the wave equation used in this paper is
%   A FUNCTION OF DENSITY AND PROPAGATION VELOCITY PERTURBATIONS, and
%   the wave equation is EQUIVALENT TO THE ONE USED BY GORE AND LEEMAN.[1] (p. 182)
% I. DERIVATION OF THE WAVE EQUATION
% - The wave equation [(16)] was derived in Chernov. [3] (p. 183)
%   [3] L. A. Chernov, [leave Propagation in a Random Medium (McGraw. Hill, New York, 1960).
% - It has also been considered in Gore and Leeman [1] and Morse and Ingard [4] in a slightly different form, where
%   the SCATTERING TERMS WERE A FUNCTION OF THE ADIABATIC COMPRESSIBILITY \kappa AND THE DENSITY. (p. 183)
% article:GorePMB1977a: Ultrasonic backscattering from human tissue: A realistic model
% Abstract
% - The propagation of ultrasound pulses in INHOMOGENEOUS MEDIA is described, and it is shown that they are scattered by
%   FLUCTUATIONS IN DENSITY AND COMPRESSIBILITY. (p. 317)
% 2. The wave equation for ultrasound propagation through tissue
% - The DENSITY AND COMPRESSIBILITY OF SMALL TISSUE SAMPLES FLUCTUATE FROM PLACE TO PLACE ABOUT THEIR MEAN VALUES so that in any region
%   the LOCAL ACOUSTIC PROPERTIES DIFFER FROM THE AVERAGE. (p. 318)
% - SOUND WAVES WILL BE SCATTERED IN SUCH AN INHOMOGENEOUS MEDIUM and
%   the ECHOES FROM INSIDE TISSUE CAN BE ATTRIBUTED TO DENSITY-COMPRESSIBILITY FLUCTUATIONS, which
%   may be DISTRIBUTED RANDOMLY OR REGULARLY throughout the tissue. (pp. 318, 319)
% - No assumptions are necessary as to the random nature, or otherwise, of the variables of interest, but
%   THE WEAKNESS OF THE SCATTERING OBSERVED IN PRACTICE IMPLIES THAT
%   THE MAGNITUDE OF FLUCTUATIONS MAY BE CONSIDERED TO BE SMALL. (p. 319)
The relevant acoustic material parameters are
the unperturbed values of both
% 1.) unperturbed compressibility
the compressibility and
% 2.) unperturbed mass density
the mass density, which are typically normalized by
% 3.) spatial averages
% book:Kak2001, Sect. 6.1.2: Inhomogeneous Wave Equation
% - \kappa_{0} and \rho_{0} are either
%   the compressibility and the density of the medium in which the object is immersed, or
%   THE AVERAGE COMPRESSIBILITY AND THE DENSITY OF THE OBJECT, depending upon how the process of imaging is modeled. (p. 210)
% article:JensenJASA1991: A model for the propagation and scattering of ultrasound in tissue
% I. DERIVATION OF THE WAVE EQUATION
% - We now assume that
%   [1.)] the PROPAGATION VELOCITY AND
%   [2.)] THE DENSITY ONLY VARY SLIGHTLY FROM
%   THEIR MEAN VALUES, so that
%   [ \rho( \vect{r} ) = \rho_{0} + \Delta \rho( \vect{r} ) ],
%   [ c( \vect{r} ) = c_{0}( \vect{r} ) + \Delta c( \vect{r} ) ], (12)
%   where \rho_{0} \gg \Delta \rho and c_{0} \gg \Delta c. (p. 183)
% article:GorePMB1977a: Ultrasonic backscattering from human tissue: A realistic model
% 2. The wave equation for ultrasound propagation through tissue
% - It is convenient to consider formally
%   the scattering region V to be embedded in SOME NON-DISPERSIVE MEDIUM WITH
%   CONSTANT DENSITY, \rho_{0}, and
%   COMPRESSIBILITY \kappa_{0} = ( \rho_{0} {c_{0}}^{2} )^{-1}, with
%   c_{0} the acoustic velocity in the embedding medium. (p. 319)
% - THE VALUES OF THESE PARAMETERS ARE CHOSEN TO BE THE MEAN VALUES THEY ASSUME INSIDE V. (p. 319)
their spatial averages
% 4.) reference value for the unperturbed compressibility
$\kappa_{0} \in \Rplus$ and
% 5.) reference value for the unperturbed mass density
$\rho_{0} \in \Rplus$,
respectively
\cite{article:NgITUFFC2006,article:JensenJASA1991,article:GorePMB1977a}.
% c) proposed model considers a homogeneous unperturbed mass density
For
the sake of
simplicity,
the proposed model considers
a homogeneous unperturbed mass density.
% d) global unperturbed compressibility / associated relative spatial fluctuations
The relative spatial fluctuations in
the unperturbed compressibility read
(cf. \cref{fig:lin_mod_scan_configuration})
\begin{equation}
 %--------------------------------------------------------------------------------------------------------------
 % relative spatial fluctuations in the unperturbed compressibility
 %--------------------------------------------------------------------------------------------------------------
  \gamma^{(\kappa)}( \vect{r} )
  =
  \begin{cases}
    1 - \kappa_{1}( \vect{r} ) / \kappa_{0} & \text{for } \vect{r} \in \Omega,\\
    0 & \text{for } \vect{r} \notin \Omega.\\
  \end{cases}
 \label{eqn:lin_mod_mech_model_tis_simple_rel_fluctuations}
\end{equation}

%---------------------------------------------------------------------------------------------------------------
% 2.) power-law dependence of the spatial amplitude absorption coefficient on the frequency
%---------------------------------------------------------------------------------------------------------------
% a) spatial amplitude absorption coefficient obeys the power law in the entire Euclidean space
% book:Szabo2013, Chapter 4: Attenuation / Sect. 4.1: Losses in Tissues
% - REAL TISSUE DATA INDICATE that ATTENUATION HAS A POWER-LAW DEPENDENCE on FREQUENCY. (p. 82)
% book:Szabo2013, Chapter 4: Attenuation / Sect. 4.1: Losses in Tissues / Sect. 4.1.2: Tissue Data
% - These simple loss and delay factors are NOT OBSERVED IN REAL MATERIALS AND TISSUES. (p. 84)
% - Data indicate that the absorption is a function of frequency. (p. 84)
% - MANY OF THESE LOSSES OBEY A FREQUENCY POWER LAW, defined as
%   (4.6A) [\alpha( f ) = \alpha_{0} + \alpha_{1} \abs{ f }^{y}], in which
%   \alpha_{0} is OFTEN ZERO and y is a POWER LAW EXPONENT. (p. 84)
% - A graph for the MEASURED ABSORPTION OF COMMON TISSUES AS A FUNCTION OF FREQUENCY is given in
%   Figure 4.2A. (p. 84)
% article:KellyJASA2008b: Analytical time-domain Green's functions for power-law media
% I. INTRODUCTION
% - The ATTENUATION COEFFICIENT FOR BIOLOGICAL TISSUE may be approximated by
%   A POWER LAW [1] OVER A WIDE RANGE OF FREQUENCIES. (p. 2861)
%   [1] book:Duck1990, pp. 99-124
% II. FPDE FORMULATIONS OF THE SZABO AND POWER-LAW WAVE EQUATIONS / A. Szabo wave equation
% - The SZABO WAVE EQUATION [4] APPROXIMATES POWER-LAW MEDIA with
%   AN ATTENUATION COEFFICIENT GIVEN BY (1) [\alpha( \omega ) = \alpha_{0} \abs{ \omega }^{ y }]. (p. 2862)
% II. FPDE FORMULATIONS OF THE SZABO AND POWER-LAW WAVE EQUATIONS / C. Power-law wave equation
% - In addition, Eq. (8) [reciprocal phase velocity] is in CLOSE AGREEMENT with
%   the EXPERIMENTAL DISPERSION DATA presented in Refs. 12–14 and 17.
%   [12] article:ODonnellJASA1981, [13] article:SzaboJASA1995, [14] article:WatersJASA2000a, [17] article:HeITUFFC1998
% - Thus, Eq. (2) [Szabo's approximate wave equation] supports
%   the POWER-LAW ATTENUATION AND DISPERSION that is
%   PREDICTED BY THE KRAMERS-KRONIG RELATIONSHIPS and SUPPORTED BY EXPERIMENTAL MEASUREMENTS. (p. 2863)
% book:Cobbold2006, Sect. 3.10: Effects of Attenuation / Sect. 3.10.1 Kramers-Kronig Relationships
% - As noted in 1.8.1 (see footnote 35),
%   LONGITUDINAL WAVE PROPAGATION IN MOST SOFT TISSUE IS FOUND TO HAVE
%   A FREQUENCY-DEPENDENT ATTENUATION that
%   IS GENERALLY WELL APPROXIMATED BY
%   [ \alpha = \alpha_{0}' \abs{ \omega }^{n} ] (3.97), where
%   \alpha_{0}' is the ANGULAR FREQUENCY ATTENUATION FACTOR and
%   n is a real positive number that TYPICALLY LIES IN THE RANGE 1 \leq n \leq 2. (p. 207)
% book:Cobbold2006, Sect. 1.8.1: Absorption and Scattering Attenuation Coefficients / Attenuation of Biological Tissues
% - It can be seen that
%   a GOOD APPROXIMATION for the FREQUENCY DEPENDENCE for most SOFT TISSUE is given by
%   [ \alpha = \alpha_{0} f^{n} ] (1.125), where
%   n lies in the RANGE FROM 1 to 2. (p. 74)
% article:WatersITUFFC2005: Causality-imposed (Kramers-Kronig) relationships between attenuation and dispersion
% IV. Applications of the Acoustic Kramers-Kronig Dispersion Relations
% - We investigate two cases often considered in ultrasonic research:
%   MEDIA WITH AN ATTENUATION COEFFICIENT OBEYING A FREQUENCY POWER LAW, and
%   suspensions with resonant scattering properties. (p. 825)
% - The first case of POWER-LAW ATTENUATION is often considered for
%   PROPAGATION IN MANY SOFT TISSUES [29], [48], [49] AND LIQUIDS [26], [50]. (p. 825)
% IV. Applications of the Acoustic Kramers-Kronig Dispersion Relations / A. Power-Law Attenuation
% - However, IT OFTEN IS OBSERVED THAT THE ATTENUATION IN MANY LIQUIDS (AND OTHER MEDIA) DOES NOT EXHIBIT an ω2-dependence. (p. 826)
% - In such cases, the ATTENUATION COEFFICIENT often is found to be well described by a POWER LAW: (8) [\alpha( \omega ) = \alpha_{0} \omega^{y}],
%   where \omega is angular frequency, and \alpha_{0} and y are material-dependent parameters with y typically between 1 and 2, inclusive. (p. 826)
% article:WatersJASA2000a: On the applicability of Kramers–Krönig relations for ultrasonic attenuation obeying a frequency power law
% I. THEORY
% - In a VARIETY OF MEDIA (e.g., LIQUIDS AND TISSUE) OVER A FINITE BANDWIDTH,
%   the ATTENUATION OF ULTRASONIC WAVES APPEARS TO BE ADEQUATELY MODELED BY
%   a POWER-LAW DEPENDENCE ON FREQUENCY, [2,3,11] (1) [ \alpha( \omega ) = \alpha_{0} \abs{ \omega }^{y} ], where
%   we assume \alpha_{0} and y are REAL CONSTANTS, with 0 < y <= 2 typically. (p. 556)
% book:Duck1990, Sect. 4.3.8: Values of acoustic absorption coefficients in tissue
% - Measured values of ABSORPTION COEFFICIENTS FOR ULTRASOUND IN SOFT TISSUE are given in
%   Tables 4.19 and 4.20. (p. 115)
% - VALUES AT PARTICULAR FREQUENCIES ARE INCLUDED IN Table 4.19, and
%   the POWER-LAW EXPRESSION Equation 4.30 [ \alpha = a f^{b} ] used as the basis for the values given in Table 4.20. (p. 115)
% - Many of the factors discussed for attenuation in Section 4.3.5 are EQUALLY RELEVANT TO ABSORPTION,
%   and reference should be made to this section. (p. 115)
% - Tab. 4.19: Ultrasound ABSORPTION COEFFICIENT for SOFT TISSUES (i) (specific frequencies)
% - Tab. 4.20: Ultrasound ABSORPTION COEFFICIENT (ii); \alpha = a f^{b} (power law parameters)
% book:Duck1990, Sect. 4.3.5.2: Frequency (Sect. 4.3.5: Factors affecting attenuation)
% - The FREQUENCY DEPENDENCY of ULTRASONIC ATTENUATION AND ABSORPTION can be represented by
%   the expression [ \alpha = a f^{b} ] (4.30) where
%   a [coefficient], b [exponent] are constants and f is frequency. (p. 112)
% - Several authors have fitted their data to this expression over
%   NARROW RANGES OF FREQUENCY. (p. 112)
% - Others have assumed a LINEAR FREQUENCY DEPENDENCE (b = 1), particularly when estimating
%   ATTENUATION COEFFICIENTS FROM IN-VIVO MEASUREMENTS USING THE FREQUENCY CONTENT OF BACKSCATTERED SOUND. (p. 112)
% - Data in both forms are included in Tables 4.16 and 4.17 [ATTENUATION COEFFICIENTS!]. (p. 112)
% - Tab. 4.16 Ultrasound amplitude ATTENUATION COEFFICIENT for NORMAL TISSUE: \alpha = a f^{b}
% - Tab. 4.17 Amplitude ATTENUATION COEFFICIENT, \alpha = a f^{b}; HUMAN PATHOLOGICAL TISSUES
% - There is some evidence that Equation 4.30 [\alpha = a f^{b}] may
%   NOT BE APPROPRIATE OVER A WIDER FREQUENCY RANGE. (p. 112)
% article:WellsUMB1975: Absorption and dispersion of ultrasound in biological tissue
% ABSORPTION AND VELOCITY DATA FOR BIOLOGICAL MATERIALS
% - The DATA FOR ABSORPTION are presented in Fig. 1. (p. 370)
% - It is IMMEDIATELY APPARENT that,
%   for BIOLOGICAL SOFT TISSUES, (4) [\alpha = a f^{b}], where
%   a [coefficient] and b [exponent] depend upon
%   THE CHARACTERISTICS OF THE PARTICULAR TISSUE and
%   THE CONDITIONS OF MEASUREMENT (such as temperature), and
%   have FAIRLY CONSTANT VALUES OVER LIMITED RANGES OF FREQUENCY. (p. 370)
The spatial amplitude absorption coefficient
$\alpha_{l} \in \Rnonneg$, which is
% 1.) commonly neglected
% article:GorePMB1977a: Ultrasonic backscattering from human tissue: A realistic model
% 1. Introduction
% - Ideally, the scattering from a particular region should be specified in a way which allows
%   the EFFECT OF THE OVERLYING TISSUE TO BE EASILY QUANTIFIED; in particular,
%   the EFFECT OF FREQUENCY DEPENDENT ATTENUATION SHOULD BE INCLUDED. (p. 318)
% 4. Discussion and implications of results
% - The TISSUE MODEL DISCUSSED HERE MAY BE EXTENDED TO INCLUDE EFFECTS SUCH AS ABSORPTION, and
%   further study of this is under way. (p. 325)
commonly neglected, obeys
% 2.) power law
the power law
(cf. e.g.
\cite[Sect. 4.3.8]{book:Duck1990},
\cite[(4)]{article:WellsUMB1975}%
)
\begin{equation}
 %--------------------------------------------------------------------------------------------------------------
 % spatial amplitude absorption coefficient
 %--------------------------------------------------------------------------------------------------------------
  \alpha_{l}
  =
  \bar{b} \abs{ \omega_{l} }^{ \zeta }
 \label{eqn:lin_mod_mech_model_tis_abs_power_law}
\end{equation}
in
% 3.) entire Euclidean space
the entire Euclidean space for
% 4.) all relevant discrete frequencies
all relevant discrete frequencies
$l \in \setsymbol{L}_{ \text{BP} }^{(n)}$, where
% 5.) pair of absorption parameters
the parameter pair
$( \bar{b}, \zeta ) \in \Rnonneg \times \Rnonneg$ depends on both
% 6.) specific type of tissue
the specific type of
tissue and
% 7.) ambient conditions
the ambient conditions.
% b.) exponent \zeta usually ranges between 1 and 1.5
% article:JensenProgBMB2007: Medical ultrasound imaging
% - Typically, an attenuation of 0.5 dB/(MHz cm) is experienced in the SOFT TISSUES.
% article:KellyJASA2008b: Analytical time-domain Green's functions for power-law media
% I. INTRODUCTION
% - MEASURED ATTENUATION COEFFICIENTS OF SOFT TISSUE TYPICALLY HAVE
%   LINEAR OR GREATER THAN LINEAR DEPENDENCE ON FREQUENCY. (p. 2861)
% - For example, breast fat has a power-law exponent of y = 1.5, while
%   breast tissue ranges [1] from y = 1 to y = 1.5. (p. 2861)
%   [1] book:Duck1990, pp. 99-124
% article:SzaboJASA2000: A model for longitudinal and shear wave propagation in viscoelastic media
% - [...] y is a POSITIVE NUMBER USUALLY LESS THAN 2 [...]. (p. 2437)
% - In contrast, for ACOUSTIC WAVES in VISCOELASTIC MEDIA, the FREQUENCY EXPONENT is most often not fractional,
%   but has been found to VARY FROM 0 TO 2. (p. 2442)
% - LONGITUDINAL MODE ABSORPTION obeys a POWER LAW with an EXPONENT most frequently in the range of
%   1 < y < 2 for a LARGE NUMBER OF MATERIALS, both FLUID and SOLID
%   [Duck (1990); Zeqiri (1988); Bamber (1986); Szabo (1993, 1994, 1995); O’Donnell (1981),He (1998a, 1998b, 1999)]. (pp. 2442, 2443)
% article:SzaboJASA1995: Causal theories and data for acoustic attenuation obeying a frequency power law
% I. TIME DOMAIN CAUSAL RELATIONSHIPS / B. Anomalous dispersion
% - MOST MATERIALS FALL IN THE RANGE 0 < y < 2. (p. 16)
% - FOR THE MAJORITY OF CASES, y >= 1, [...] (p. 16)
% II. CAUSAL RELATIONS IN THE FREQUENCY DOMAIN / A. Horton's dispersion relationships
% - Note that our PRIMARY INTEREST in this study is for
%   the EXPONENT RANGE 0 <= y <= 2 in which MOST MATERIALS FALL. (p. 16)
% book:Duck1990, Sect. 4.3.5 Factors affecting attenuation / Sect. 4.3.5.2 Frequency
% - Values for b for MOST SOFT TISSUE and BIOLOGICAL FLUIDS LIE IN THE RANGE 1.0 to 1.5. (p. 112)
% article:WellsUMB1975: Absorption and dispersion of ultrasound in biological tissue
% - The value of b [exponent] is GENERALLY ONLY A LITTLE GREATER THAN UNITY. (p. 370)
The exponent $\zeta$ usually ranges between
$1.0$ and $1.5$
\cite{article:KellyJASA2008b},
\cite[112]{book:Duck1990},
\cite{article:WellsUMB1975}.
% c) complex-valued wavenumber with respect to k_{\text{ref}} combines power-law absorption with dispersion
% article:KellyJASA2008b: Analytical time-domain Green's functions for power-law media
% II. FPDE FORMULATIONS OF THE SZABO AND POWER-LAW WAVE EQUATIONS / C. Power-law wave equation
% - In order to derive the analytical time-domain Green’s function solution,
%   A POWER-LAW DISPERSION RELATIONSHIP RELATING WAVENUMBER k AND ANGULAR FREQUENCY \omega is required. (p. 2863)
% - This DISPERSION RELATIONSHIP should yield
%   (1) a POWER-LAW ATTENUATION COEFFICIENT and
%   (2) a FREQUENCY-DEPENDENT PHASE SPEED. (p. 2863)
% - The power-law dispersion relationship that satisfies these requirements is (7) [complex-valued wavenumber] for
%   \omega \geq 0 and k( -\omega ) = \conj{k}( \omega ) to ensure real solutions. (p. 2863)
% - The imaginary part of Eq. (7) [complex-valued wavenumber] yields
%   the POWER-LAW ATTENUATION COEFFICIENT given by Eq. (1) [ \alpha( \omega ) = \alpha_{0} \abs{ \omega }^{y} ]. (p. 2863)
% - The real part produces (8) [reciprocal phase velocity]. (p. 2863)
% - Equation (9) [reciprocal phase velocity, ref. frequency] corresponds to
%   the PHASE VELOCITIES COMPUTED VIA
%   [1.)] THE KRAMERS-KRONIG RELATIONS [14, 27] AND
%   [2.)] THE TIME-CAUSAL THEORY. [13] (p. 2863)
%   [13] article:SzaboJASA1995, [14] article:WatersJASA2000a, [27] article:CobboldJASA2004
% article:WatersITUFFC2005: Causality-imposed (Kramers-Kronig) relationships between attenuation and dispersion
% IV. Applications of the Acoustic Kramers-Kronig Dispersion Relations / A. Power-Law Attenuation
% - The first approach extrapolates the measured ultrasonic properties beyond the experimental bandwidth, and
%   it applies the exact integral or differential forms of the K-K dispersion relations shown in Table I. (p. 826)
% - The corresponding dispersions are shown in Table II. (p. 826)
\TODO{check exponent}
Given
reference values of
% 1.) reference angular frequency
the angular frequency
$\omega_{\text{ref}} \in \Rplus$ and
% 2.) reference phase velocity
the associated phase velocity
$c_{\text{ref}} \in \Rplus$,
% 3.) complex-valued wavenumber with respect to k_{\text{ref}}
the complex-valued wavenumber
% TODO: Kelly?
\cite{article:WatersITUFFC2005,article:SzaboJASA1995}
\begin{subequations}
\label{eqn:lin_mod_mech_model_tis_abs_time_causal_wavenumber_complex_kref}
\begin{equation}
 %--------------------------------------------------------------------------------------------------------------
 % complex-valued wavenumber with respect to k_{\text{ref}}
 %--------------------------------------------------------------------------------------------------------------
  \munderbar{k}_{l}
  =
  \underbrace{
    \frac{ \omega_{l} }{ c_{\text{ref}} }
    +
    \beta_{\text{E,ref}, l}
  }_{ = \beta_{l} = \omega_{l} / c_{l} }
  - j
  \underbrace{
    \bar{b} \abs{ \omega_{l} }^{ \zeta }
  }_{ = \alpha_{l} },
 \label{eqn:lin_mod_mech_model_tis_abs_time_causal_wavenumber_complex_kref_sum}
\end{equation}
where
% 4.) phase term
the phase term
$\beta_{l} \in \R$ sums
% 5.) real-valued wavenumber with respect to c_{\text{ref}}
the real-valued wavenumber
$k_{\text{ref}, l} = \omega_{l} / c_{\text{ref}}$ and
% 6.) excess dispersion term with respect to k_{\text{ref}}
the excess dispersion term
\begin{equation}
 %--------------------------------------------------------------------------------------------------------------
 % excess dispersion term with respect to k_{\text{ref}}
 %--------------------------------------------------------------------------------------------------------------
  \beta_{\text{E,ref}, l}
  =
  \begin{cases}
   %------------------------------------------------------------------------------------------------------------
   % a) exponent of unity
   %------------------------------------------------------------------------------------------------------------
    -
    2 \bar{b} \omega_{l}
    \ln\bigl( \abs{ \omega_{l} / \omega_{\text{ref}} } \bigr) / \pi
    &
    \text{for } \zeta = 1,\\
   %------------------------------------------------------------------------------------------------------------
   % b) even integer or noninteger
   %------------------------------------------------------------------------------------------------------------
    \bar{b}
    \tan\bigl( \zeta \pi / 2 \bigr)
    \omega_{l}
    \bigl( \abs{ \omega_{l} }^{ \zeta - 1 } - \abs{ \omega_{\text{ref}} }^{ \zeta - 1 } \bigr)
    &
    \text{else},
  \end{cases}
 \label{eqn:lin_mod_mech_model_tis_abs_time_causal_wavenumber_complex_kref_excess_dispersion}
\end{equation}
\end{subequations}
combines
% 1.) power-law absorption
power-law absorption with
% 2.) anomalous dispersion [ phase velocity increases w/ frequency ]
% book:Cobbold2006, Sect. 3.10: Effects of Attenuation / Sect. 3.10.1: Kramers-Kronig Relationships
% - It will be noted that FOR n > 2 the SPEED DECREASES WITH INCREASING FREQUENCY and NORMAL DISPERSION is said to be present. (p. 207)
% - FOR n < 2 the OPPOSITE IS TRUE, and the DISPERSION is said to be ANOMALOUS. (p. 207)
% article:SzaboJASA1995: Causal theories and data for acoustic attenuation obeying a frequency power law
% I. TIME DOMAIN CAUSAL RELATIONSHIPS / B. Anomalous dispersion
% - Before proceeding, it is necessary to be MORE PRECISE ABOUT THE DEFINITION OF c_{0} and whether
%   VELOCITY DISPERSION INCREASES OR DECREASES WITH FREQUENCY. (p. 15)
% - According to conventions established in electromagnetic theory,
%   "NORMAL" DISPERSION is that in which PHASE VELOCITY DECREASES WITH AN INCREASE IN FREQUENCY. (p. 15)
% - "ANOMALOUS" DISPERSION is defined as the condition in which PHASE VELOCITY INCREASES WITH FREQUENCY
%   (Stratton, 1941; Jackson, 1975; Gurumurthy and Arthur, 1982). (pp. 15, 16)
% - For THESE CASES [power-law absorption], when phase velocity dispersion has been observed,
%   it was found to INCREASE WITH FREQUENCY in accordance with the ANOMALOUS CATEGORY. (p. 16)
% - The fact that DISPERSION is MOST OFTEN ANOMALOUS DETERMINES THE VALUE OF c_{0} used in the propagation factor \beta_{0} = \omega / c_{0}. (p. 16)
anomalous dispersion.


%%%%%%%%%%%%%%%%%%%%%%%%%%%%%%%%%%%%%%%%%%%%%%%%%%%%%%%%%%%%%%%%%%%%%%%%%%%%%%%%%%%%%%%%%%%%%%%%%%%%%%%%%%%%%%%%
% 3.) recorded radio frequency voltage signals
%%%%%%%%%%%%%%%%%%%%%%%%%%%%%%%%%%%%%%%%%%%%%%%%%%%%%%%%%%%%%%%%%%%%%%%%%%%%%%%%%%%%%%%%%%%%%%%%%%%%%%%%%%%%%%%%
\subsection{Recorded Radio Frequency Voltage Signals}
%\label{subsec:lin_mod_exc_sup_qsw_p_sc_born}
%%%%%%%%%%%%%%%%%%%%%%%%%%%%%%%%%%%%%%%%%%%%%%%%%%%%%%%%%%%%%%%%%%%%%%%%%%%%%%%%%%%%%%%%%%%%%%%%%%%%%%%%%%%%%%%%
% graphic: block diagram of the pulse-echo measurement process
%%%%%%%%%%%%%%%%%%%%%%%%%%%%%%%%%%%%%%%%%%%%%%%%%%%%%%%%%%%%%%%%%%%%%%%%%%%%%%%%%%%%%%%%%%%%%%%%%%%%%%%%%%%%%%%%
\graphictwocols{linear_model/figures/latex/lin_mod_v_rx_sgn_proc_chain.tex}%
{% a) block diagram of the pulse-echo measurement process
 Block diagram of
 % 1.) pulse-echo measurement process (single pulse-echo measurement, monofrequent, single array element)
 the pulse-echo measurement process
 \eqref{eqn:lin_mod_v_rx_born_obs_proc}.
 % b) observation operators map the relative spatial fluctuations in compressibility to the Born approximation of of the recorded RF voltage signals
 Given
 % 1.) incident acoustic pressure fields
 the incident acoustic pressure fields
 $p_{l}^{(\text{in}, n)}: \R^{d} \mapsto \C$, which satisfy
 % 2.) Helmholtz equations for the incident acoustic pressure fields
 the \name{Helmholtz} equations
 \eqref{eqn:lin_mod_sol_wave_eq_pde_p_in},
 % 3.) observation operators
 the \name{Born} approximation linearly maps
 % 4.) relative spatial fluctuations in the unperturbed compressibility
 the compressibility fluctuations
 \eqref{eqn:lin_mod_mech_model_tis_simple_rel_fluctuations} to
 % 5.) Born approximation of the recorded RF voltage signals
 the recorded \ac{RF} voltage signals
 \eqref{eqn:lin_mod_v_rx_born}.
 %--------------------------------------------------------------------------------------------------------------
 % decomposition of the observation operators
 %--------------------------------------------------------------------------------------------------------------
 % a) first two multiplications yield the Born approximations of the contrast sources and the force densities
 %The first two multiplications yield
 %the \name{Born} approximations of
 % 1.) Born approximation of the contrast sources
 %the contrast sources
 %$\phi_{l}^{(\text{B}, n)}: \R^{d} \mapsto \C$ and
 % 2.) Born approximation of the force densities
 %the force densities
 %$f_{m, l}^{(\text{B}, n)}: \R^{d} \mapsto \C$.
 % b) receiver electromechanical transfer function
 %The multiplications of
 %their spatial integrals, i.e.
 % 1.) Born approximation of the compressive blocked forces
 %the \name{Born} approximation of
 %the compressive blocked forces
 %\eqref{eqn:lin_mod_exc_sup_qsw_blocked_force_born}, by
 % 2.) receiver electromechanical transfer functions
 %the receiver electromechanical transfer functions
 %$h_{m}^{(\text{rx})}$ according to
 %the transfer relations
 %\eqref{eqn:lin_mod_scan_config_trans_array_transfer_v_rx} yield
 % 3.) Born approximation of the recorded RF voltage signals
 %the desired signals.
}%
{lin_mod_v_rx_sgn_proc_chain}
%TODO: does contrast source include $\munderbar{k}^{2}$?

%---------------------------------------------------------------------------------------------------------------
% 1.) RF voltage signals generated by all array elements
%---------------------------------------------------------------------------------------------------------------
% a) array elements transduce the compressive blocked forces exerted by the free-space scattered acoustic pressure fields on the planar faces into the RF voltage signals
% article:Schiffner2018, Sect. III. Linear Physical Model for the Pulse-Echo Measurement Process / Sect. A. Pulse-Echo Measurement Process
% - The \ac{UI} system sequentially performs $N_{\text{in}} \in \N$ independent pulse-echo measurements using a planar transducer array
%   (cf. \cref{fig:lin_mod_scan_configuration,tab:lin_mod_scan_config_instrum_params}).
% - Each measurement begins at the time instant $t = 0$ and triggers
%   the concurrent recording of the \ac{RF} VOLTAGE SIGNALS $\tilde{u}_{m}^{(\text{rx}, n)}: \setsymbol{T}_{ \text{rec} }^{(n)} \mapsto \R$ GENERATED BY
%   ALL ARRAY ELEMENTS $m \in \setconsnonneg{ N_{\text{el}} - 1 }$ in the specified time interval [...].
% book:Schmerr2015, Sect. 1.3: Modeling Ultrasonic Phased Array Systems
% - When the acoustic pulses generated by the nth element interact with
%   scattering objects (such as surfaces of a component being inspected or flaws) and travel back to the array
%   they are RECEIVED BY THE MTH ELEMENT OF THE ARRAY AND CONVERTED TO A RECEIVED VOLTAGE PULSE. (p. 8)
% - This RECEIVING TRANSFER FUNCTION is a function of
%   [1.)] the electrical impedance and gain present in the RECEIVING CIRCUITS,
%   [2.)] the WIRING/CABLING present, and
%   [3.)] the electrical impedance and sensitivity of the mth PIEZOELECTRIC ELEMENT [Schmerr-Song]. (p. 9)
The array elements transduce
% 1.) compressive blocked forces
% book:Schmerr2015, Sect. 9.2 Linear System Modeling and Sound Reception
% - In fact, if we assume that we can represent the interaction of
%   the incident waves with
%   the element surfaces in these cases as PLANE WAVE INTERACTIONS, then
%   [1.)] the BLOCKED FORCE (the FORCE AT A PLANE RIGID, IMMOBILE SURFACE) is just
%   TWICE THE FORCE, F_{inc}( ω ), generated on the face of the array element by the INCIDENT WAVES only (i.e. when the element receiving surface is absent) and
%   [2.)] the FREE-SURFACE VELOCITY (the velocity at a plane stress-free surface) is
%   TWICE THE VELOCITY generated by only the incident waves, v_{inc}( ω ), so that in both the immersion and the contact cases
%   the acoustic equivalent sources shown in Figs. 9.8 and 9.11 become (see [Schmerr-Song] for an explicit proof in the immersion case;
%   the contact case follows in a similar manner):
%   [  F_{B}( ω ) = 2 F_{inc}( ω )
%     v_{fs}( ω ) = 2 v_{inc}( ω ). ] (9.13) (p. 187)
% book:Schmerr2015, Sect. 1.3: Modeling Ultrasonic Phased Array Systems
% - The BLOCKED FORCE appearing in Eq. (1.2) [ V_{mn}^{r}( f ) = t_{m}^{r}( f ) F_{mn}^{B}( f ) ] is defined as
%   the force exerted on the face of the receiving element when
%   the FACE OF THAT ELEMENT IS HELD RIGIDLY FIXED. (p. 9)
% - In [Schmerr-Song] and in Chap. 9 it is shown how this BLOCKED FORCE arises naturally in
%   describing the sound reception process for an ultrasonic system. (p. 9)
% book:Schmerr2007, Sect. 5.2: The Blocked Force
% - The force, F_{B}( ω ), appearing in both Eqs. (5.1) and (5.2) is a particular force acting on the receiving transducer called the BLOCKED FORCE.
%   This BLOCKED FORCE is defined as
%   the force that would be exerted on the receiving transducer if its face was held RIGIDLY FIXED (IMMOBILE). (p. 69)
% - To summarize: If we assume PLANE WAVE INTERACTIONS AT THE RECEIVING TRANSDUCER,
%   the BLOCKED FORCE, F_{B}( ω ), is just TWICE THE FORCE, F_{inc}( ω ), exerted by the waves INCIDENT on the AREA OF THE RECEIVER.
%   The force, F_{inc}( ω ), acting on S is computed from the INCIDENT WAVES as if the transducer were ABSENT. (p. 70)
% - We will also find it useful to use Eq. (5.7) [ F_{B}( ω ) = 2 F_{inc}( ω ) ] when obtaining the acoustic/elastic transfer function since then
%   we can model the pressure wave field of only the INCIDENT waves at the receiving transducer and
%   use Eq. (5.7) to obtain the BLOCKED FORCE, without having to consider explicitly any
%   more complex interactions of the incident waves with the receiving transducer. (p. 70)
% book:Cobbold2006, Chapter 5: Scattering of Ultrasound / Sect. 5.7: Pulse-Echo Response / Sect. 5.7.1: Continuum Model
% - Thus, the total force is given by [...], and the constant K will be assumed to be 2,
%   corresponding to an IDEAL REFLECTING PLANE. (p. 299)
% article:NgITUFFC2006: Modeling ultrasound imaging as a linear, shift-variant system
% III. The Wave Equation / D. The Force on the Receive Subaperture
% - In this subsection, we compute
%   THE SUMMATION OF THE SCATTERED PRESSURE FIELD OVER THE RECEIVE SUBAPERTURE. (p. 552)
% - Strictly speaking, this quantity, which will be denoted by F( \vect{x}_{0}, ω ), is
%   the FORCE EXERTED ON THE RECEIVE SUBAPERTURE [3]. (pp. 552, 553)
%   [3] article:ZempITUFFC2003: Linear system models for ultrasonic imaging: Application to signal statistics
% - If we represent the APODIZATION and FOCUSING on reception collectively in a single complex-valued term W( \vect{x}_{a}, ω ), then:
%   [ F( \vect{x}_{0}, ω ) = \int_{ \mathcal{A} } W( \vect{x}_{a}, ω ) P_{s}( \vect{x}_{0} + \vect{x}_{a}, \vect{x}_{0}, ω ) \text{d}^{2} \vect{x}_{a} ]. (12) (p. 553)
% article:StepanishenJASA1981: Pulsed transmit/receive response of ultrasonic piezoelectric transducers
% - The HARMONIC FORCE which ACTS on the TRANSDUCER as a result of the ACOUSTIC BACKSCATTERING PROCESS can
%   be expressed as: (5a), thus, (5b), where a DIFFRACTION CONSTANT of 2 has been used to ACCOUNT FOR
%   the RIGID BAFFLE EFFECT. (p. 1818)
% (physical unit: $[ F_{m} ] = \si{\newton \meter\tothe{\textit{d}-3}}$)
the compressive blocked forces exerted by
% 2.) free-space scattered acoustic pressure fields
the free-space scattered acoustic pressure fields
$p_{l}^{(\text{sc}, n)}: \R^{d} \mapsto \C$ on
% 3.) planar faces L_{m}
their planar faces
$L_{m} \subset \R^{d-1}$ into
% 4.) RF voltage signals
% book:Schmerr2015, Sect. 9.2: Linear System Modeling and Sound Reception
% - It then follows that we can REPLACE
%   [1.)] THE DETAILED MODELS of Fig. 9.13 with
%   [2.)] the SINGLE INPUT–SINGLE OUTPUT LTI SYSTEMS SHOWN IN FIG. 9.14 for the IMMERSION and CONTACT CASES. (p. 188)
% - The TRANSFER FUNCTION, t_{R}( ω ), of the IMMERSION CASE relates
%   [1.)] the BLOCKED FORCE INPUT, F_{B}( ω ), to
%   [2.)] the OUTPUT VOLTAGE, V_{e}( ω ). (p. 188)
% book:Schmerr2015, Sect. 1.3: Modeling Ultrasonic Phased Array Systems
% - We can relate the frequency components of
%   [1.)] this RECEIVED VOLTAGE, V_{mn}^{r}( f ), to
%   [2.)] the BLOCKED FORCE, F_{mn}^{B}( f ) (Fig. 1.10b), generated by the incident waves for
%   each pair of sending and receiving elements through
%   [3.)] a SOUND RECEPTION TRANSFER FUNCTION, t_{m}^{r}( f ), i.e. (1.2). (pp. 8, 9)
% article:LabyedITUFFC2014: TR-MUSIC inversion of the density and compressibility contrasts of point scatterers
% II. The Interelement Response Matrix for Point Scatterers With Density and Compressibility Contrasts
% - The spectrum of the ELECTRICAL SIGNAL MEASURED BY THE RECEIVING TRANSDUCER ELEMENT is given by [15], [16]
%   [ p_{m}( ω ) = W_{\text{r}}( ω ) \int_{ S_{\text{r}} } p_{\text{s}}( \vect{r}, ω ) \text{d} s ] (7), where
%   the integral is evaluated over the receiving-element area S_{\text{r}}, and
%   W_{\text{r}}( ω ) is the ELECTROMECHANICAL TRANSFER FUNCTION OF THE RECEIVER. (p. 17)
% article:NgITUFFC2006: Modeling ultrasound imaging as a linear, shift-variant system
% III. The Wave Equation / D. The Force on the Receive Subaperture
% - We recall from the physical description in Section II that
%   the RECEIVED RF VOLTAGE TRACE is obtained by
%   [1.)] summing the scattered pressure field over the receive subaperture and
%   [2.)] FILTERING THIS SUM BY THE ELECTROMECHANICAL RESPONSE OF THE PIEZOELECTRIC ELEMENTS. (p. 552)
% III. The Wave Equation / E. The RF Voltage Trace
% - We now MODEL THE ELECTROMECHANICAL CONVERSION of
%   [1.)] THE FORCE ON THE RECEIVE SUBAPERTURE INTO
%   [2.)] A VOLTAGE TRACE. (p. 553)
% - If we define the ELECTROMECHANICAL TRANSFER FUNCTION that models this conversion to be E_{m}( ω ) and
%   the VOLTAGE TRACE to be R( \vect{x}_{0}, ω ), we have
%   R( \vect{x}_{0}, ω ) = E_{m}( ω ) F( \vect{x}_{0}, ω ) [2], [3];
%   substituting in the expression for F( x_{0}, ω ) from (20) we get: (21). (p. 553)
% article:JensenJASA1991: A model for the propagation and scattering of ultrasound in tissue
% IV. CALCULATION OF THE RECEIVED SIGNAL
% - The received signal is
%   the SCATTERED PRESSURE FIELD INTEGRATED OVER THE TRANSDUCER SURFACE, convolved with
%   the ELECTROMECHANICAL IMPULSE RESPONSE E_{m}( t ) of the transducer. (p. 185)
% - The received signal is
%   [ p_{r}( \vect{r}_{5}, t ) = E_{m}( t ) * \int_{S} p_{s}( \vect{r}_{6} + \vect{r}_{5}, t ) d^{2} \vect{r}_{6} ]. (35) (p. 185)
% - Symbolically, this is written as
%   [ p_{r}( \vect{r}_{5}, t ) = v_{pe}( t ) *_{t} f_{m}( \vect{r}_{1} ) *_{r} h_{pe}( \vect{r}_{1}, \vect{r}_{5}, t ) ]. (45)
% - The PULSE-ECHO WAVELET is v_{pe}, which includes
%   the TRANSDUCER EXCITATION AND THE ELECTROMECHANICAL IMPULSE RESPONSE DURING EMISSION AND RECEPTION OF THE PULSE. (p. 186)
% - The term f_{m} accounts for
%   the INHOMOGENEITIES IN THE TISSUE DUE TO DENSITY AND PROPAGATION VELOCITY PERTURBATIONS that give rise to the scattered signal. (p. 186)
% - The term h_{pe} is the MODIFIED PULSE-ECHO SPATIAL IMPULSE RESPONSE that relates
%   the TRANSDUCER GEOMETRY TO THE SPATIAL EXTENT OF THE SCATTERED FIELD. (p. 186)
the \ac{RF} voltage signals
(cf. e.g.
\cite[Sect. 9.2]{book:Schmerr2015},
\cite{article:LabyedITUFFC2014,article:NgITUFFC2006,article:JensenJASA1991}%
)
\begin{equation}
 %--------------------------------------------------------------------------------------------------------------
 % recorded RF voltage signals provided by the receiving amplification networks
 %--------------------------------------------------------------------------------------------------------------
  u_{m, l}^{(\text{rx}, n)}
  =
  2 h_{m, l}^{(\text{rx})}
  \int_{ L_{m} }
    \chi_{m, l}^{(\text{rx})}( \vect{r}_{\rho} )
    p_{l}^{(\text{sc}, n)}( \vect{r}_{\rho}, 0 )
  \text{d} \vect{r}_{\rho}
 \label{eqn:lin_mod_scan_config_trans_array_transfer_v_rx}
\end{equation}
for
% 4.) all sequential pulse-echo measurements, all relevant discrete frequencies, and all array elements
all $( n, l, m ) \in \setconsnonneg{ N_{\text{in}} - 1 } \times \setsymbol{L}_{ \text{BP} }^{(n)} \times \setconsnonneg{ N_{\text{el}} - 1 }$, where
% 5.) receiver electromechanical transfer functions
$h_{m, l}^{(\text{rx})} \in \C$ denote
the electromechanical transfer functions and
% 6.) receiver apodization functions
$\chi_{m, l}^{(\text{rx})}: L_{m} \mapsto \C$ are
the apodization functions
(cf. \cref{tab:lin_mod_scan_config_instrum_params}).

%---------------------------------------------------------------------------------------------------------------
% 2.) Born approximation of the free-space scattered acoustic pressure fields
%---------------------------------------------------------------------------------------------------------------
% a) Born approximation uses the incident acoustic pressure fields to estimate the free-space scattered acoustic pressure fields
% book:Devaney2012, Chapter 6: Scattering theory / Sect. 6.7: The Born series / Subsect. 6.7.1: The Born approximation
% - The BORN APPROXIMATION Eq. (6.52) is seen to be
%   a LINEAR MAPPING FROM THE SCATTERING POTENTIAL V TO THE SCATTERED FIELD:
%   [ ] (6.53). (p. 256)
% - The inverse scattering problem within the BORN APPROXIMATION consists of inverting
%   the set of equations Eqs. (6.53) for the scattering potential from measurements of the scattered field obtained in
%   a suite of scattering experiments using a set of incident waves U(in)(r, ν). (p. 256)
% book:Natterer2001, Chapter 3: Tomography, Sect. 3.3: Diffraction Tomography
% - In order to derive the BORN APPROXIMATION,
%   we put u = u_{I} + \nu in (3.11) [reduced wave equation], where
%   \nu satisfies the Sommerfeld radiation condition and the differential equation (3.12). (p. 47)
% - Now we assume that
%   THE SCATTERED FIELD \nu IS SMALL IN COMPARISON WITH THE INCIDENT FIELD u_{I}.
%   Then we can neglect \nu on the right-hand side of (3.12), obtaining (3.13). (p. 47)
% - Using G_{n}, n = 2, 3, WE CAN REWRITE (3.13) AS (3.16). (p. 47)
% book:Kak2001, Chapter 6: Tomographic Imaging with Diffracting Sources / Sect. 6.2: Approximations to the Wave Equation / Sect. 6.2.1: The First Born Approximation
% - The integral of (37) [LS integral equation] is now written as (39) but
%   if the scattered field, u_{s}(r), is small compared to u_{0}(r) the effects of the second integral can be ignored to
%   arrive at the approximation (40). (p. 212)
% book:Born1999, Sect. 13.1.4: Multiple Scattering
% - As we noted earlier, if the scattering is weak (|U^{(s)}| << |U^{(i)}|),
%   one might expect to obtain a good approximation to the total field if U is replaced by U^{(i)} in
%   the integrand on the right-hand side of (54).
%   This gives the FIRST-ORDER BORN APPROXIMATION (57). (p. 708)
% book:Born1999, Sect. 13.1.2: The first-order Born approximation
% - This APPROXIMATE SOLUTION is generally referred to as
%   the BORN APPROXIMATION or, more precisely,
%   the FIRST-ORDER BORN APPROXIMATION (or just the FIRST BORN APPROXIMATION). (p. 700)
% article:JensenJASA1991: A model for the propagation and scattering of ultrasound in tissue
% II. CALCULATION OF THE SCATTERED FIELD
% - If G_{i} symbolizes the INTEGRAL OPERATOR REPRESENTING GREEN'S FUNCTION AND THE INTEGRATION, and
%   F_{op} the scattering operator, then
%   the FIRST-ORDER BORN APPROXIMATION can be written as
%   [ p_{s}( \vect{r}_{2}, t ) = G_{i} F_{op} p_{i}( \vect{r}_{1}, t_{1} ) ]. (21) (p. 184)
% article:GorePMB1977a: Ultrasonic backscattering from human tissue: A realistic model
% 2. The wave equation for ultrasound propagation through tissue
% - To FIRST APPROXIMATION (THE BORN APPROXIMATION) the scattered field
%   (measured at some location and time for which there is no contribution from the incident field) is given by
%   [ ... ] (4)
%   where the vector differentiation operator V is understood to be with respect to r’. (pp. 319, 320)
The \name{Born} approximation, which drives
% 1.) established image recovery methods in ultrafast UI
% article:ChernyakovaITUFFC2018: Fourier-Domain Beamforming and Structure-Based Reconstruction for Plane-Wave Imaging
% article:MoghimiradITUFFC2016: Synthetic Aperture Ultrasound Fourier Beamformation Using Virtual Sources
% proc:SchiffnerIUS2016a: A low-rate parallel Fourier domain beamforming method for ultrafast pulse-echo imaging
% article:LabyedITUFFC2014: TR-MUSIC inversion of the density and compressibility contrasts of point scatterers
% article:MontaldoITUFFC2009: Coherent plane-wave compounding for very high frame rate ultrasonography and transient elastography
% article:JensenUlt2006: Synthetic aperture ultrasound imaging [Dec.]
% article:ChengITUFFC2006: Extended high-frame rate imaging method with limited-diffraction beams [May]
% article:WalkerITUFFC2001: C- and D-weighted ultrasonic imaging using the translating apertures algorithm
% - The LACK OF MULTIPLE SCATTERING IN SOFT TISSUES IS COMMONLY ASSUMED WITH GOOD RESULTS [13].
% - [13] M. F. Insana and D. G. Brown, “Acoustic scattering theory applied to soft biological tissues,”
%   in Ultrasonic Scattering in Biological Tissues. K. K. Shung, Ed. Ann Arbor, MI: CRC Press, 1993, pp. 75–124.
% article:LuITUFFC1997: 2D and 3D High Frame Rate Imaging with Limited Diffraction Beams
the established image recovery methods in
ultrafast \ac{UI}
\cite{article:MoghimiradITUFFC2016,article:LabyedITUFFC2014,article:MontaldoITUFFC2009,article:JensenUlt2006,article:ChengITUFFC2006,article:LuITUFFC1997}, uses
% 2.) incident acoustic pressure fields
% article:JensenJASA1991: A model for the propagation and scattering of ultrasound in tissue
% III. CALCULATION OF THE INCIDENT FIELD
% - The INCIDENT FIELD is generated by the ultrasound transducer, assuming NO OTHER SOURCES EXIST IN THE TISSUE. (p. 184)
% - By this method [spatial impulse response]
%   the INCIDENT FIELD IS FOUND BY SOLVING THE WAVE EQUATION FOR THE HOMOGENEOUS CASE:
%   [ \nabla^{2} p_{1} - \frac{ 1 }{ c_{0}^{2} } \frac{ \partial^{2} p_{1} }{ \partial^{2} t } = 0 ]. (25) (p. 184)
the incident acoustic pressure fields
$p_{l}^{(\text{in}, n)}: \R^{d} \mapsto \C$ induced by
% 3.) transducer array
the transducer array in
the homogeneous fluid and governed by
% 4.) Helmholtz equations for the incident acoustic pressure fields
% book:Devaney2012, Chapter 6: Scattering theory / Sect. 6.1: Potential scattering theory
% - The BOUNDARY CONDITION SATISFIED BY THE FIELD U IS THAT IT REDUCES TO THE SUM OF
%   [1.)] AN INCIDENT WAVEFIELD U^{(in)} plus
%   [2.)] A SCATTERED FIELD U^{(s)} that is required to SATISFY THE SOMMERFELD RADIATION CONDITION (SRC); i.e.,
%   [ U( \vect{r}, \nu ) = U^{(in)}( \vect{r}, \nu ) + U_{+}^{(s)}( \vect{r}, \nu ) ~ U^{(in)}( \vect{r}, \nu ) + f( \vect{s}, \nu ) e^{ j k_{0} r } / r ] (6.4), where
%   [1.)] the INCIDENT WAVEFIELD U^{(in)} PROPAGATES IN THE UNIFORM BACKGROUND MEDIUM AND HENCE SATISFIES
%   THE HOMOGENEOUS HELMHOLTZ EQUATION
%   [ [ \Delta + k_{0}^{2} ] U^{(in)}( \vect{r}, \nu ) = 0 ], and
%   f( \vect{s}, \nu ) is the INDUCED SOURCE RADIATION PATTERN in the direction of the unit vector s = \hat{r} = r/r for the νth scattering experiment. (pp. 231, 232)
% article:NgITUFFC2006: Modeling ultrasound imaging as a linear, shift-variant system
% III. The Wave Equation / A. The Total Pressure Field
% - Because (2) is linear, we can write its GENERAL SOLUTION as the sum of
%   [1.)] the SOLUTION TO THE CORRESPONDING HOMOGENEOUS EQUATION (i.e., with the RHS set to zero) and
%   [2.)] any PARTICULAR SOLUTION [10]. (p. 550)
%   [10] G. F. Carrier and C. E. Pearson, Partial Differential Equations: Theory and Technique. New York: Academic, 1976.
% - Denoting
%   the solution to the HOMOGENEOUS EQUATION as P_{i}( \vect{x}, ω ) and
%   the PARTICULAR SOLUTION as P_{s}( \vect{x}, ω ), we, therefore, can write the TOTAL FIELD as:
%   [ P'( \vect{x}, ω ) = P_{i}( \vect{x}, ω ) + P_{s}( \vect{x}, ω ) ]. (5) (p. 550)
% - We see then that P_{i}( \vect{x}, ω ) IS THE PRESSURE FIELD THAT DEVELOPS IN THE ABSENCE OF ANY SCATTERERS which, by definition,
%   is the INCIDENT PRESSURE FIELD. (p. 550)
% book:Natterer2001, Chapter 3: Tomography, Sect. 3.3: Diffraction Tomography
% - In order to derive the BORN APPROXIMATION,
%   we put u = u_{I} + \nu in (3.11) [reduced wave equation], where
%   \nu satisfies the Sommerfeld radiation condition and the differential equation (3.12). (p. 47)
% book:Kak2001, Chapter 6: Tomographic Imaging with Diffracting Sources / Sect. 6.1: Diffracted Projections / Sect. 6.1.2: Inhomogeneous Wave Equation
% - We will CONSIDER THE FIELD, u(r), TO BE THE SUM OF TWO COMPONENTS, u_{0}(r) and u_{s}(r). (p. 210)
% - The COMPONENT u_{0}(F), known as the INCIDENT FIELD, is the field present without any inhomogeneities, or, equivalently, a solution to the equation
%   [ (\Delta + k_{0}^{2}) u_{0}(r) = 0 ] (30). (p. 210)
the \name{Helmholtz} equations
(cf. e.g.
\cite[(6.4)]{book:Devaney2012},    %
%\cite{article:NgITUFFC2006},		%
\cite[47]{book:Natterer2001},           %
\cite[(30)]{book:Kak2001}%
)
\begin{equation}
 %--------------------------------------------------------------------------------------------------------------
 % Helmholtz equations for the incident acoustic pressure fields
 %--------------------------------------------------------------------------------------------------------------
  \left( \Delta + {\munderbar{k}_{l}}^{2} \right)
  p_{l}^{(\text{in}, n)}( \vect{r} )
  = 0
 \label{eqn:lin_mod_sol_wave_eq_pde_p_in}
\end{equation}
to estimate
% 5.) free-space scattered acoustic pressure fields
% article:NgITUFFC2006: Modeling ultrasound imaging as a linear, shift-variant system
% III. The Wave Equation / A. The Total Pressure Field
% - We also know that the SCATTERED PRESSURE FIELD must obey (2), and so we can assign our particular solution P_{s}( \vect{x}, ω ) to
%   be the SCATTERED PRESSURE FIELD. (p. 550)
% book:Kak2001, Chapter 6: Tomographic Imaging with Diffracting Sources / Sect. 6.1: Diffracted Projections / Sect. 6.1.2: Inhomogeneous Wave Equation
% - The component u_{s}(r), known as the SCATTERED FIELD, will be that part of the total field that can be ATTRIBUTED SOLELY TO THE INHOMOGENEITIES. (p. 210)
the free-space scattered acoustic pressure fields as
(cf. e.g.
\cite[(6.53)]{book:Devaney2012},		% term: "Born approximation" (checked!)
%\cite[268, 287]{book:Cobbold2006},		% term: "Born approximation" (checked!)
\cite[(3.16)]{book:Natterer2001},		% term: "Born approximation", plane-wave insonification
\cite[(40)]{book:Kak2001},			% term: "first Born approximation"
\cite[(57)]{book:Born1999}%			% terms: first Born approximation, first-order Born approximation, Born approximation (checked!)
)
\begin{equation}
 %--------------------------------------------------------------------------------------------------------------
 % Born approximations of the free-space scattered acoustic pressure fields
 %--------------------------------------------------------------------------------------------------------------
  p_{l}^{(\text{sc}, n)}( \vect{r} )
  \approx
  {\munderbar{k}_{l}}^{2}
  \int_{ \Omega }
    \gamma^{(\kappa)}( \vect{r}' )
    p_{l}^{(\text{in}, n)}( \vect{r}' )
    g_{l}( \vect{r} - \vect{r}' )
  \text{d} \vect{r}'
 \label{eqn:lin_mod_v_rx_p_sc_born}
\end{equation}
for
% 6.) all sequential pulse-echo measurements and all relevant discrete frequencies
all $( n, l ) \in \setconsnonneg{ N_{\text{in}} - 1 } \times \setsymbol{L}_{ \text{BP} }^{(n)}$, where
% 7.) outgoing free-space Green's functions (two- and three-dimensional Euclidean spaces)
the outgoing free-space \name{Green}'s functions
\eqref{eqn:app_helmholtz_green_free_space_2_3_dim} account for
% 8.) diffraction
diffraction and
% 9.) monopole scattering
monopole scattering
(cf. Appendix \ref{app:helmholtz_green}).
% b) resulting single scattering is valid for weakly-scattering heterogeneous objects
% book:Cobbold2006, Chapter 5: Scattering of Ultrasound / Sect. 5.4: Integral Equation Methods / Sect. 5.4.3: Scattering Approximations
% - If the SCATTERING IS SUFFICIENTLY WEAK so that
%   THE SCATTERED PRESSURE IS MUCH LESS THAN THE INCIDENT PRESSURE,
%   THE INCIDENT WAVE WILL REMAIN VIRTUALLY UNCHANGED AS IT PROGRESSES THROUGH THE SCATTERING VOLUME, i.e.
%   p \approx p_{i}. (p. 287)
% - THIS IS KNOWN AS THE BORN APPROXIMATION and
%   it ENABLES THE SCATTERED PRESSURE TO BE EVALUATED without having to use, for example,
%   the method of successive approximations. (p. 287)
% book:Cobbold2006, Chapter 5: Scattering of Ultrasound
% - Since the integrand involves the sum of the incident and scattered fields, it is generally appropriate to make
%   the BORN APPROXIMATION in which the SCATTERED FIELD IS ASSUMED TO BE SMALL COMPARED TO THAT INCIDENT. (p. 268)
% article:NgITUFFC2006: Modeling ultrasound imaging as a linear, shift-variant system
% III. The Wave Equation
% - WE RESTRICT OURSELVES TO THE CASE OF WEAK SCATTERING in which
%   THE ENERGY OF THE SCATTERED WAVES IS MUCH LESS THAN THE ENERGY OF THE INCIDENT WAVES. (p. 550)
% III. The Wave Equation / C. The Scattered Pressure Field
% - Because we are dealing only with the case of weak scattering, we assume that
%   |P_{s}( \vect{x}, \vect{x}_{0}, ω )| \ll |P_{i}( \vect{x}, \vect{x}_{0}, ω )|. (p. 552)
% - P_{s}( \vect{x}, \vect{x}_{0}, ω ) in [ P'( \vect{x}, ω ) = P_{i}( \vect{x}, ω ) + P_{s}( \vect{x}, ω ) ] (5) then becomes negligible and
%   P'( \vect{x}, \vect{x}_{0}, ω ) \approx P_{i}( \vect{x}, \vect{x}_{0}, ω ). (p. 552)
% book:Born1999, Sect. 13.1.2: The first-order Born approximation
% - From expression (6) for the scattering potential it is clear that a medium will scatter weakly if its refractive index differs only slightly from unity.
%   Under these circumstances it is plausible to assume that one will obtain a good approximation to the total field U if the term U = U^{(i)} + U^{(s)} under
%   the integral in (16) is replaced by U^{(i)}. (pp. 699, 700)
% article:JensenJASA1991: A model for the propagation and scattering of ultrasound in tissue
% II. CALCULATION OF THE SCATTERED FIELD
% - Here [Born approximation],
%   p_{s} HAS BEEN SET TO ZERO IN (20) [sum of fields]. (p. 184)
% - Usually the SCATTERING FROM SMALL OBSTACLES IS CONSIDERED WEAK, so higher-order terms can be neglected. (p. 184)
% article:GorePMB1977a: Ultrasonic backscattering from human tissue: A realistic model
% 2. The wave equation for ultrasound propagation through tissue
% - For WEAK SCATTERING the sound field within V may be written as the sum of
%   the incident field p_{i}, and a weak scattered field p_{s}
%   [ p = p_{i} + p_{s} with \frac{ \abs{ p_{s} } }{ \abs{ p_{i} } } \ll 1 ]. (p. 319)
% 2.) single scattering
% article:NgITUFFC2006: Modeling ultrasound imaging as a linear, shift-variant system
% - Thus, in making the Born approximation, we have assumed implicitly that
%   MULTIPLY SCATTERED WAVES (i.e., waves scattered off a particle that are then scattered off other particles) ARE NEGLIGIBLE, and that
%   MULTIPLE SCATTERING CAN BE IGNORED [1], [2], [12]. (p. 552)
% book:Born1999, Sect. 13.1.4: Multiple Scattering
% - The physical significance of the successive terms is as follows:
%   The product U^{(i)}(r') F(r') d^{3}r', in [...] may be regarded as representing
%   the RESPONSE TO THE INCIDENT FIELD of the volume d^{3}r' around the point r' of the scatterer.
%   It acts as an EFFECTIVE SOURCE which makes a contribution U^{(i)}(r') F(r') G(r - r') d^{3}r' to
%   the field at another point, r, that may be situated either inside or outside V. (p. 709)
% - Evidently the Green's function G(r - r') acts as a propagator transferring the contribution from the point r' to the point r. (p. 709)
% - The integral over the volume V thus represents
%   the TOTAL CONTRIBUTION FROM ALL THE VOLUME ELEMENTS OF THE SCATTERER. (p. 709)
% -> THIS PROCESS IS KNOWN AS SINGLE SCATTERING and is illustrated in Fig. 13.7(a) (p. 709)
% article:GorePMB1977a: Ultrasonic backscattering from human tissue: A realistic model
% 2. The wave equation for ultrasound propagation through tissue
% - MULTIPLE SCATTERING EFFECTS HAVE BEEN NEGLECTED, which is CONSISTENT WITH THE ASSUMPTION OF WEAK SCATTERING, and the
%   possibility of a strongly reflecting interface within the region of interest V is clearly excluded. (p. 320)
The resulting single scattering is valid for
% 1.) weakly-scattering lossy heterogeneous objects
weakly-scattering heterogeneous objects, i.e.
% 2.) validity condition
% article:DevaneyJASA1985: Variable density acoustic tomography
% - The validity of the FIRST BORN APPROXIMATION clearly requires that
%   the STRENGTH OF THE SCATTERED FIELD COMPONENT OF THE PRESSURE FIELD (6) REMAINS SMALL THROUGHOUT THE VOLUME OF THE OBJECT (scatterer).
$\tabs{ p_{l}^{(\text{sc}, n)}( \vect{r} ) } \ll \tabs{ p_{l}^{(\text{in}, n)}( \vect{r} ) }$ for
% 3.) all object points
all $\vect{r} \in \Omega$.
%neglects the interactions of % 1.) scattered fields the scattered fields with the heterogeneous object and
% c) weakly-scattering heterogeneous objects feature both small absolute values of the compressibility fluctuations and small acoustic sizes
% article:WangJOSAA2011:
% - conditions for validity (6), (7), and (10)
% article:LiPIER2010: On the Validity of Born Approximation
% article:NgITUFFC2006: Modeling ultrasound imaging as a linear, shift-variant system
% - We restrict ourselves to the case of WEAK SCATTERING in which
%   THE ENERGY OF THE SCATTERED WAVES IS MUCH LESS THAN
%   THE ENERGY OF THE INCIDENT WAVES. (p. ?)
% book:Born1999, Sect. 13.1.2: The first-order Born approximation
% - From expression (6) for the SCATTERING POTENTIAL it is clear that
%   a MEDIUM WILL SCATTER WEAKLY if its REFRACTIVE INDEX differs only slightly from unity.  (p. 699)
% - Under these circumstances it is plausible to assume that one will obtain a good approximation to the total field U if
%   the term U = U^{(i)} + U^{(s)} under the integral in (16) [LS equation] is replaced by U^{(i)}. (pp. 699, 700)
% article:DevaneyJASA1985: Variable density acoustic tomography
% - This in turn requires that both
%   THE MAGNITUDE OF THE MATERIAL PARAMETERS $\gamma_{\kappa}$ and $\gamma_{\rho}$ be small and that
%   THE TOTAL VOLUME OF THE OBJECT BE SMALL.
% - This second condition is usually violated in tomographic applications where the objects can oftentimes be on the order of 100 wavelengths or more in extent.
% - Thus, the Rytov approximation is ideally suited to diffraction tomography of weakly inhomogeneous objects of arbitrary size.
These feature both
% 1.) small absolute value of the relative spatial fluctuations in compressibility
small absolute values of
% 1.) relative spatial fluctuations in the unperturbed compressibility
the compressibility fluctuations
\eqref{eqn:lin_mod_mech_model_tis_simple_rel_fluctuations} and
% 2.) small spatial extent of the lossy heterogeneous object
small acoustic sizes
\cite{article:LiPIER2010},
\cite[708]{book:Born1999}.

%---------------------------------------------------------------------------------------------------------------
% 3.) Born approximation of the recorded RF voltage signals
%---------------------------------------------------------------------------------------------------------------
% a) insertion into the receiver electromechanical transfer relations yield the recorded RF voltage signals
The \name{Born} approximation of
the scattered acoustic pressure fields
\eqref{eqn:lin_mod_v_rx_p_sc_born} estimates
% 2.) recorded RF voltage signals
the recorded \ac{RF} voltage signals
\eqref{eqn:lin_mod_scan_config_trans_array_transfer_v_rx} as
% 3.) Fredholm integral equations of the first kind
% article:AltürkJIASF2017: On Multidimensional Fredholm Integral Equations of the First Kind
% - MULTIDIMENSIONAL FREDHOLM INTEGRAL EQUATION OF THE FIRST KIND are of the form
%   [ . ] (1.1), where
%   x = ( x_{1}, x_{2}, ..., x_{n} ), t = ( t_{1}, t_{2}, ..., t_{n} ), Ω = [ a_{1}, b_{1} ] × ... × [ a_{n}, b_{n} ] ⊂ R^{n},
%   f(x) is the data function, K(x, t) is the kernel, F(φ(t)) is a linear or a nonlinear function of φ(t), and
%   φ is the only unknown in (1.1) that we wish to determine. (p. 85)
% - FREDHOLM INTEGRAL EQUATIONS OF THE FIRST KIND ARE USUALLY ILL-POSED IN THE HADAMARD SENSE [12], that is,
%   the [1.)] existence and [2.)] uniqueness of the solution and
%   [3.)] continuous dependency of the data function to the solution ARE NOT GUARANTEED for
%   ill-posed problems. (p. 85)
% book:Hansen2010, Chapter 2: Meet the Fredholm Integral Equation of the First Kind / Sect. 2.2: Properties of the Integral Equation
% - The FREDHOLM INTEGRAL EQUATION OF THE FIRST KIND TAKES THE GENERIC FORM
%   [ . ] (2.2). (p. 7)
% - Here, both the kernel K and the right-hand side g are KNOWN FUNCTIONS, while f is the unknown function. (p. 7)
% - This equation establishes a linear relationship between the two functions f and g, and
%   the kernel K describes the precise relationship between the two quantities. (p. 7)
% - Thus, the function K describes the underlying model.
% - In Chapter 7 we will encounter FORMULATIONS OF (2.2) IN MULTIPLE DIMENSIONS, but
%   our discussion until then will focus on the ONE-DIMENSIONAL CASE.
% book:Hansen2010, Chapter 7: Regularization Methods at Work: Solving Real Problems / Sect. 7.5: Deconvolution in 2D-Image Deblurring
% - In the continuous setting of Chapter 2, image deblurring is
%   a FIRST-KIND FREDHOLM INTEGRAL EQUATION of the generic form
%   [...], in which
%   the two functions f(t) and g(s) that represent the sharp and blurred images are both functions of
%   TWO SPATIAL VARIABLES s = (s1, s2) and t = (t1, t2). (p. 144)
% article:GroetschJOPCS2007: Integral equations of the first kind, inverse problems and regularization: a crash course
the \name{Fredholm} integral equations of
the first kind
%(cf. e.g.
%\cite[(1.1)]{article:AltuerkJIASF2017},
%\cite[(2.2)]{book:Hansen2010}%
%)
\begin{subequations}
\label{eqn:lin_mod_v_rx_born}
\begin{equation}
 %--------------------------------------------------------------------------------------------------------------
 % a) Born approximation of the recorded RF voltage signals
 %--------------------------------------------------------------------------------------------------------------
  u_{m, l}^{(\text{rx}, n)}
  \approx
  u_{m, l}^{(\text{B}, n)}
  =
  \dopobservation{ m, l }{ p_{l}^{(\text{in}, n)} }{ \gamma^{(\kappa)} }{1}
 \label{eqn:lin_mod_v_rx_born_expression}
\end{equation}
for
% 5.) all sequential pulse-echo measurements, all relevant discrete frequencies, and all array elements
all $( n, l, m ) \in \setconsnonneg{ N_{\text{in}} - 1 } \times \setsymbol{L}_{ \text{BP} }^{(n)} \times \setconsnonneg{ N_{\text{el}} - 1 }$, where
% 6.)
the properties of
% 1.) pulse-echo measurement process (single pulse-echo measurement, monofrequent, single transducer element)
the pulse-echo measurement process
\begin{equation}
 %--------------------------------------------------------------------------------------------------------------
 % b) pulse-echo measurement process (single pulse-echo measurement, monofrequent, single transducer element)
 %--------------------------------------------------------------------------------------------------------------
  \dopobservation{ m, l }{ p_{l}^{(\text{in}, n)} }{ \gamma^{(\kappa)} }{1}
  =
  - {\munderbar{k}_{l}}^{2} h_{m, l}^{(\text{rx})}
  \int_{ \Omega }
    \gamma^{(\kappa)}( \vect{r} )
    p_{l}^{(\text{in}, n)}( \vect{r} )
    \varUpsilon_{m, l}^{(\text{rx})}( \vect{r} )
  \text{d} \vect{r}
 \label{eqn:lin_mod_v_rx_born_obs_proc}
\end{equation}
significantly depend on
% 2.) incident waves
the incident waves
(cf. \cref{fig:lin_mod_v_rx_sgn_proc_chain}), and
% 4.) apodized spatial receive functions
the apodized spatial receive functions
\begin{equation}
 %--------------------------------------------------------------------------------------------------------------
 % c) apodized spatial receive functions
 %--------------------------------------------------------------------------------------------------------------
  \varUpsilon_{m, l}^{(\text{rx})}( \vect{r} )
  =
  - 2
  \int_{ L_{m} }
    \chi_{m, l}^{(\text{rx})}( \vect{r}_{\rho}' )
    g_{l}( \vect{r}_{\rho}' - \vect{r}_{\rho}, - r_{d} )
  \text{d} \vect{r}_{\rho}',
 \label{eqn:lin_mod_exc_sup_qsw_volt_rx_spat_trans}
\end{equation}
\end{subequations}
which correspond to
% 7.) spatial impulse responses in the time domain
% article:ChengUlt2011: A new algorithm for spatial impulse response of rectangular planar transducers
% 1. Introduction
% - Over the years,
%   ULTRASOUND SOURCES OF DIFFERENT GEOMETRICAL SHAPES, such as
%   CIRCLES [3], RECTANGLES [5–7], TRIANGLES [8], POLYGONS [9] AND CURVED STRIPS [10] have been studied. (p. 229)
% - The solution for spatial impulse response is HIGHLY DEPENDENT ON THE SHAPE OF THE SOURCES. (p. 229)
% - All the described approaches require either approximations or complex geometrical considerations.
%   Therefore, a simplified and exact solution is needed. (p. 230)
% coll:Jensen2002, Sect. 5: Spatial Impulse Responses / Sect. 5.2: Basic Theory
% - The INTEGRAL in this equation, (33) is called
%   the SPATIAL IMPULSE RESPONSE and CHARACTERIZES the 3-DIMENSIONAL EXTENT OF THE FIELD for
%   a PARTICULAR TRANSDUCER GEOMETRY. (p. 152)
% coll:Jensen2002, Sect. 5: Spatial Impulse Responses / Sect. 5.1: Fields in Linear Acoustic Systems
% - A VOLTAGE EXCITATION of the TRANSDUCER with a DELTA FUNCTION will give rise to a PRESSURE FIELD that
%   is measured by the hydrophone. The MEASURED RESPONSE is the ACOUSTIC IMPULSE RESPONSE for
%   this particular system with the given setup. (p. 149)
% - The SPATIAL IMPULSE RESPONSE is then found by observing the pressure waves at
%   a fixed position in space over time by having all the spherical waves pass the point of observation and
%   summing them. (p. 150)
% coll:Jensen2002, Sect. 5: Spatial Impulse Responses
% - Thus, a more accurate and general solution
%   [relation between the oscillation of the transducer surface and the ultrasound field] is needed,
%   and this is developed here. (p. 149)
% - The approach is based on the CONCEPT OF SPATIAL IMPULSE RESPONSES developed by
%   Tupholme [7] and Stepanishen [8,9]. (p. 149)
% article:JensenJASA1999: A new calculation procedure for spatial impulse responses in ultrasound
% INTRODUCTION
% - The impulse response has been found for a number of geometries
%   (round flat piston [2], round concave [4,5], flat rectangle [6,7], and flat triangle [8]). (p. 3266)
% article:JensenJASA1991: A model for the propagation and scattering of ultrasound in tissue
% III. CALCULATION OF THE INCIDENT FIELD
% - The function (31) is called the SPATIAL IMPULSE RESPONSE and
%   it RELATES the TRANSDUCER GEOMETRY TO THE ACOUSTICAL FIELD. (p. 185)
the spatial impulse responses in
the time domain
\cite{coll:Jensen2002,article:JensenJASA1991}, characterize
% 8.) anisotropic directivities of the planar faces
the anisotropic directivities of
the planar faces.

% book:Devaney2012, Chapter 6: Scattering theory / Sect. 6.1: Potential scattering theory
% - We should note that the INCIDENT FIELD WILL, in fact, BE PRODUCED BY SOME SOURCE RADIATING IN THE BACKGROUND MEDIUM and,
%   hence, will actually SATISFY THE INHOMOGENEOUS HELMHOLTZ EQUATION. (p. 232)
% - However, WE ASSUME THAT THIS SOURCE IS WELL SEPARATED FROM THE SCATTERER so that
%   THE FIELD U^{(in)} WILL SATISFY THE HOMOGENEOUS HELMHOLTZ EQUATION AT LEAST WITHIN THE SCATTERER VOLUME τ0. (p. 232)
% - Moreover, as we discussed in our treatment of the angular-spectrum expansion in Section 4.2 of Chapter 4,
%   the FIELD RADIATED BY A COMPACTLY SUPPORTED SOURCE CAN BE ACCURATELY APPROXIMATED AS
%   A FREE FIELD at distances that are MORE THAN A FEW WAVELENGTHS FROM THE SOURCE SUPPORT VOLUME. (p. 232)
% - Thus, insofar as the POTENTIAL SCATTERING PROBLEM is concerned
%   the INCIDENT FIELD CAN BE MODELED AS A FREE FIELD that
%   SATISFIES THE HOMOGENEOUS HELMHOLTZ EQUATION OVER ALL OF SPACE. (p. 232)
%the lossy heterogeneous object exceeds
%a few wavelengths
%\cite[232]{book:Devaney2012}.

