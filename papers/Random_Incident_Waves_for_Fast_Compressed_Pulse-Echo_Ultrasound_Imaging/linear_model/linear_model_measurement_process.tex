%%%%%%%%%%%%%%%%%%%%%%%%%%%%%%%%%%%%%%%%%%%%%%%%%%%%%%%%%%%%%%%%%%%%%%%%%%%%%%%%%%%%%%%%%%%%%%%%%%%%%%%%%%%%%%%%
% graphic: pulse-echo measurement process (two-dimensional Euclidean space)
%%%%%%%%%%%%%%%%%%%%%%%%%%%%%%%%%%%%%%%%%%%%%%%%%%%%%%%%%%%%%%%%%%%%%%%%%%%%%%%%%%%%%%%%%%%%%%%%%%%%%%%%%%%%%%%%
\graphic{linear_model/figures/latex/lin_mod_scan_configuration_fov.tex}%
{% a) illustration of the pulse-echo measurement process in the two-dimensional Euclidean space
 Pulse-echo measurement process in
 the two-dimensional Euclidean space, i.e. $d = 2$.
 % b) linear transducer array emits a broadband incident wave into a lossy homogeneous fluid
 A linear transducer array emits
 % 1.) broadband incident wave
 a broadband incident wave into
 % 2.) lossy homogeneous fluid with the unperturbed values of the compressibility \kappa_{0} and the mass density \rho_{0}
 a lossy homogeneous fluid with
 the unperturbed values of
 % 3.) unperturbed compressibility
 the compressibility
 $\kappa_{0} \in \Rplus$ and
 % 4.) mass density
 the mass density
 $\rho_{0} \in \Rplus$.
 % c) broadband incident wave penetrates an embedded lossy heterogeneous object, and its interactions with the unperturbed compressibility induce a scattered wave
 This wave penetrates
 % 1.) embedded lossy heterogeneous object
 an embedded lossy heterogeneous object represented by
 the bounded set
 $\Omega \subset \{ \vect{r} \in \R^{d}: r_{d} > 0 \}$, and
 its interactions with
 % 2.) unperturbed compressibility
 the unperturbed compressibility
 $\kappa_{1}: \Omega \mapsto \Rplus$ induce
 % 3.) scattered wave
 a scattered wave.
 % d) a portion of the scattered wave mechanically excites the faces of the array elements
 % TODO: transduce
 A portion of
 the latter mechanically excites
 % 1.) faces of the array elements
 the faces of
 the array elements that generate
 % 2.) RF voltage signals
 \ac{RF} voltage signals.
 % e) RF voltage signals enable the imaging of the specified FOV represented by the bounded set \Omega_{\text{FOV}}
 These enable
 the imaging of
 the specified \ac{FOV} represented by
 the bounded set
 $\Omega_{\text{FOV}} \subset \{ \vect{r} \in \R^{d}: r_{d} > 0 \}$.
}%
{lin_mod_scan_configuration}

%%%%%%%%%%%%%%%%%%%%%%%%%%%%%%%%%%%%%%%%%%%%%%%%%%%%%%%%%%%%%%%%%%%%%%%%%%%%%%%%%%%%%%%%%%%%%%%%%%%%%%%%%%%%%%%%
% table: geometric and electromechanical parameters of the instrumentation
%%%%%%%%%%%%%%%%%%%%%%%%%%%%%%%%%%%%%%%%%%%%%%%%%%%%%%%%%%%%%%%%%%%%%%%%%%%%%%%%%%%%%%%%%%%%%%%%%%%%%%%%%%%%%%%%
\begin{table*}[t!]
 \centering
 \caption{%
  Geometric and
  electromechanical parameters of
  the instrumentation for
  all $\delta \in \setcons{ d - 1 }$,
  $m \in \setconsnonneg{ N_{\text{el}} - 1 }$,
  $l \in \setsymbol{L}_{ \text{BP} }^{(n)}$.
 }
 \label{tab:lin_mod_scan_config_instrum_params}
 \small
 \begin{tabular}{%
  @{}%
  >{$}l<{$}%		01.) symbol
  p{0.925\textwidth}%	02.) meaning
  @{}%
 }
 \toprule
  \multicolumn{1}{@{}H}{Symbol} &
  \multicolumn{1}{H@{}}{Meaning}\\
  \cmidrule(r){1-1}\cmidrule(l){2-2}
 \addlinespace
 %--------------------------------------------------------------------------------------------------------------
 % a) geometric parameters of the planar transducer array
 %--------------------------------------------------------------------------------------------------------------
  % 1.) number of elements along the r_{\delta}-axis
  N_{\text{el}, \delta} &
  Number of
  elements along
  the $r_{\delta}$-axis,
  $N_{\text{el}, \delta} \in \N$\\
  % 2.) width of the vibrating faces along the r_{\delta}-axis
  w_{\text{el}, \delta} &
  Width of
  the vibrating faces along
  the $r_{\delta}$-axis,
  $w_{\text{el}, \delta} \in \Rplus$\\
  % 3.) width of the kerfs separating the elements along the r_{\delta}-axis
  k_{\text{el}, \delta} &
  Width of
  the kerfs separating
  the elements along
  the $r_{\delta}$-axis,
  $k_{\text{el}, \delta} \in \Rnonneg$\\
  % 4.) constant spacing between the centers of the adjacent vibrating faces along the r_{\delta}-axis (element pitch)
  \Delta r_{\text{el}, \delta} &
  Element pitch, i.e.
  constant spacing between
  the centers of
  the adjacent vibrating faces along
  the $r_{\delta}$-axis,
  $\Delta r_{\text{el}, \delta} = w_{\text{el}, \delta} + k_{\text{el}, \delta}$\\
  % 5.) center coordinates of the vibrating faces
  \vect{r}_{\text{el}, m} &
  Center coordinates of
  the vibrating faces,\par
  $\mathcal{M} = \{ \vect{r}_{\text{el}, m} \in \R^{d}: \vect{r}_{\text{el}, m} = \sum_{ \delta = 1 }^{ d - 1 } ( m_{\delta} - M_{\text{el}, \delta} ) \Delta r_{\text{el}, \delta} \uvect{\delta}, m_{\delta} \in \setconsnonneg{ N_{\text{el}, \delta} - 1 }, m = \mathcal{I}( \vect{m}, \vect{N}_{\text{el}} ) \}$, where\par
  % 5.a) shift of index along the r_{\delta}-axis
  $M_{\text{el}, \delta} = ( N_{\text{el}, \delta} - 1 ) / 2$ and
  % 5.b) forward index transform
  $\mathcal{I}( \vect{m}, \vect{N}_{\text{el}} ) = \sum_{ \delta = 1 }^{ d - 2 } m_{\delta} \prod_{ k = \delta + 1 }^{ d - 1 } N_{\text{el}, k} + m_{d-1}$\\
  % 6.) total number of elements
  N_{\text{el}} &
  Total number of
  elements,
  $N_{\text{el}} = \tabs{ \mathcal{M} } = \prod_{ \delta = 1 }^{ d - 1 } N_{\text{el}, \delta}$\\
  % 7.) coplanar compact sets specifying the (d-1)-dimensional vibrating faces on the hyperplane r_{d} = 0
  L_{m} &
  Coplanar compact sets specifying
  the $(d-1)$-dimensional vibrating faces on
  the hyperplane $r_{d} = 0$,\par
  $L_{m} = \prod_{ \delta = 1 }^{ d - 1 } [ r_{\text{el}, m, \delta} - 0.5 w_{\text{el}, \delta}; r_{\text{el}, m, \delta} + 0.5 w_{\text{el}, \delta} ] \subset \R^{d-1}$\\
  % 8.) transmitter apodization functions
  \chi_{m, l}^{(\text{tx})} &
  Transmitter apodization functions accounting for
  the heterogeneous normal velocities and
  the acoustic lens,
  $\chi_{m, l}^{(\text{tx})}: L_{m} \mapsto \C$\\
  % 9.) receiver apodization functions
  \chi_{m, l}^{(\text{rx})} &
  Receiver apodization functions accounting for
  the heterogeneous sensitivities and
  the acoustic lens,
  $\chi_{m, l}^{(\text{rx})}: L_{m} \mapsto \C$\\
 %--------------------------------------------------------------------------------------------------------------
 % b) electromechanical parameters of the instrumentation
 %--------------------------------------------------------------------------------------------------------------
  % 1.) transmitter electromechanical transfer functions
  h_{m, l}^{(\text{tx})} &
  Transmitter electromechanical transfer functions accounting for
  the driving circuits,
  the cables, and
  the radiating elements,
  $h_{m, l}^{(\text{tx})} \in \C$\\
  % 2.) receiver electromechanical transfer functions
  h_{m, l}^{(\text{rx})} &
  Receiver electromechanical transfer functions accounting for
  the receiving elements,
  the cables, and
  the amplifiers,
  $h_{m, l}^{(\text{rx})} \in \C$\\
 \addlinespace
 \bottomrule
 \end{tabular}
\end{table*}

%---------------------------------------------------------------------------------------------------------------
% 1.) pulse-echo measurement process and Fourier series representation of the recorded RF voltage signals
%---------------------------------------------------------------------------------------------------------------
% a) UI system sequentially performs N_{\text{in}} independent pulse-echo measurements using a planar transducer array
% article:JensenProgBMB2007: Medical ultrasound imaging
% 1 Fundamental Ultrasound Imaging
% [description]
% article:NgITUFFC2006: Modeling ultrasound imaging as a linear, shift-variant system
% II. Background
% - Conventional ultrasound imaging interrogates a medium with high-frequency, band-limited acoustic waves and detects
%   echoes scattered by inhomogeneities (also referred to as scatterers) within the medium. (p. 549)
% - A single probe placed in contact with the subject is used for both
%   [1.)] the generation of these waves and
%   [2.)] the reception of their echoes. (p. 549)
% - On the contact surface of a typical probe is found
%   an array of piezoelectric crystals or elements (referred to as the aperture), each of which behaves as
%   an electromechanical transducer. (p. 549)
% - A focused beam is produced by coherently exciting
%   a set of adjacent elements that we refer to as the transmit subaperture. (p. 549)
% - In a similar way, backscattered echoes are detected by adjacent elements in the receive subaperture;
%   these echoes are then coherently summed, and the result is filtered to produce
%   a single RF voltage trace [2], [3]1. (pp. 549, 550)
% - At each transmission,
%   the emitted wave propagating through the medium gives rise to an incident pressure field, and
%   the scattered waves give rise to a scattered pressure field. (p. 550)
% - It can be shown that, at any moment in time, the total pressure field is the sum of these two fields (see Section III-A). (p. 550)
The \ac{UI} system sequentially performs
% 1.) N_{\text{in}} independent pulse-echo measurements
$N_{\text{in}} \in \N$ independent pulse-echo measurements using
% 2.) planar transducer array
a planar transducer array
(cf. \cref{fig:lin_mod_scan_configuration,tab:lin_mod_scan_config_instrum_params}).
% b) each measurement begins at the time instant t = 0 and triggers the concurrent recording of the RF voltage signals in the specified time interval
Each measurement begins at
% 1.) time instant t = 0
the time instant
$t = 0$ and triggers
% 2.) concurrent recording
the concurrent recording of
% 3.) RF voltage signals
the \ac{RF} voltage signals
$\tilde{u}_{m}^{(\text{rx}, n)}: \setsymbol{T}_{ \text{rec} }^{(n)} \mapsto \R$ generated by
% 4.) all array elements
all array elements
$m \in \setconsnonneg{ N_{\text{el}} - 1 }$ in
% 5.) specified time interval
% book:Briggs1995, Chapter 2: The Discrete Fourier Transform / Sect. 2.4.: DFT Approximations to Fourier Series Coefficients [NONPERIODIC FUNCTIONS]
% - There seems to be NO AGREEMENT IN THE LITERATURE about whether the INTERVAL FOR DEFINING FOURIER SERIES should be
%   [1.)] the CLOSED INTERVAL [-A/2, A/2],
%   [2.)] a HALF-OPEN INTERVAL (-A/2, A/2], or
%   [3.)] the OPEN INTERVAL (-A/2, A/2). (p. 38)
% - Arguments can be made for or against any of these choices. (p. 38)
% - We will use the CLOSED INTERVAL [-A/2, A/2] throughout the book to emphasize the point (the subject of sermons to come!) that
%   IN DEFINING THE INPUT TO THE DFT, VALUES OF THE SAMPLED FUNCTION AT BOTH ENDPOINTS CONTRIBUTE TO THE INPUT. (p. 38)
the specified time interval
\begin{equation}
 %--------------------------------------------------------------------------------------------------------------
 % specified recording time intervals for the RF voltage signals generated by all array elements
 %--------------------------------------------------------------------------------------------------------------
  \setsymbol{T}_{ \text{rec} }^{(n)}
  =
  \bigl[ t_{\text{lb}}^{(n)}; t_{\text{ub}}^{(n)} \bigr],
 \label{eqn:lin_mod_scan_config_volt_rx_obs_interval}
\end{equation}
where
% 6.) lower bounds in the specified recording time intervals
$t_{\text{lb}}^{(n)} \in \Rnonneg$ and
% 7.) upper bounds in the specified recording time intervals
$t_{\text{ub}}^{(n)} > t_{\text{lb}}^{(n)}$ denote
its lower and
upper bounds,
respectively.
% c) finite recording times enable the representation of these signals by the Fourier series
% book:Mallat2009, Chapter 3: Discrete Revolution / Sect. 3.2: Discrete Time-Invariant Filters / Sect. 3.2.2: Fourier Series
% - Theorem 3.6 proves that if f \in \ell^{2}( \Z ), the FOURIER SERIES
%   [ \hat{f}( \omega ) = \sum_{ n = -\infty }^{ \infty } f[n] e^{ -i \omega n } ] (3.39)
%   can be interpreted as the decomposition of \hat{f} in an orthonormal basis of L^{2}[ -\pi, \pi ]. (p. 73)
% - The FOURIER SERIES COEFFICIENTS can thus be written as inner products in L^{2}[ -\pi, \pi ]:
%   [ f[n] = \inprod{ \hat{f}( \omega ) }{ e^{ -i \omega n } } = \frac{ 1 }{ 2 \pi } \int_{ -\pi }^{ \pi } \hat{f}( \omega ) e^{ i \omega n } d \omega. ] (3.40) (p. 73)
% - The energy conservation of orthonormal bases (A.10) yields a Plancherel identity:
%   [ \sum_{ n = -\infty }^{ \infty } \abs{ f[n] }^{2} = \norm{ \hat{f} }{2}^{2} = \frac{ 1 }{ 2 \pi } \int_{ -\pi }^{ \pi } \abs{ \hat{f}( \omega ) }^{2} d \omega. ] (3.41) (p. 73)
% - It was only in 1966 that Carleson [149] was able to prove that
%   if \hat{f} \in L^{2}[ -\pi, \pi ] THEN ITS FOURIER SERIES CONVERGES ALMOST EVERYWHERE.
%   The proof is very technical. (p. 74)
% book:Manolakis2005, Sect. 2.2.1: Fourier Transforms and Fourier Series
% Fourier series for continuous-time periodic signals
% - If a CONTINUOUS-TIME SIGNAL x_{c}(t) IS PERIODIC WITH FUNDAMENTAL PERIOD T_{p},
%   IT CAN BE EXPRESSED AS A LINEAR COMBINATION OF HARMONICALLY RELATED COMPLEX EXPONENTIALS
%   [ . ] (2.2.1) where
%   F_{0} = 1 / T_{p} is the FUNDAMENTAL FREQUENCY, and (2.2.2) which
%   are termed the FOURIER COEFFICIENTS, or the SPECTRUM of x_{c}(t). (p. 37)
% book:Briggs1995, Chapter 2: The Discrete Fourier Transform / Sect. 2.4.: DFT Approximations to Fourier Series Coefficients [DEFINITION}
% > Fourier Series <
% - Let f be a function that is PERIODIC WITH PERIOD A (also called A-PERIODIC). (p. 33)
% - Then the FOURIER SERIES ASSOCIATED WITH f is the trigonometric series
%   [ f( x ) ~ \sum_{ k = -\infty }^{ \infty } c_{k} e^{ j 2 \pi k x / A }, ] (2.12) where
%   the coefficients c_{k} are given by
%   [ c_{k} = \frac{ 1 }{ A } \int_{ - A / 2 }^{ A / 2 } f( x ) e^{ -j 2 \pi k x / A } dx. ] (2.13) (p. 33)
% - The symbol ~ means that the Fourier series is ASSOCIATED WITH THE FUNCTION f. (p. 33)
% - We would prefer to make the stronger statement that the series equals the function at every point, but
%   without imposing additional conditions on f, this cannot be said. (p. 33)
% book:Briggs1995, Chapter 2: The Discrete Fourier Transform / Sect. 2.4.: DFT Approximations to Fourier Series Coefficients [CONVERGENCE]
% - THEOREM 2.4. CONVERGENCE OF FOURIER SERIES.
%   Let f be a piecewise smooth A-periodic function. Then the Fourier series for f
%   [ \sum_{ k = -\infty }^{ \infty } c_{k} e^{ j 2 \pi k x / A} where c_{k} = \frac{ 1 }{ A } \int_{ - A / 2 }^{ A / 2 } f( x ) e^{ -j 2 \pi k x / A } dx ]
%   converges (pointwise) for every x to the value
%   [ \frac{ f( x+ ) + f( x- ) }{ 2 }. ] (p. 37)
% - Since at a point of continuity the right- and left-hand limits of a function must be equal, and equal to the function value, it follows that
%   AT ANY POINT OF CONTINUITY, the Fourier series CONVERGES TO f(x). (p. 38)
% - AT ANY POINT OF DISCONTINUITY, the series CONVERGES TO THE AVERAGE VALUE OF THE RIGHT- AND LEFT-HAND LIMITS. (p. 38)
% book:Briggs1995, Chapter 2: The Discrete Fourier Transform / Sect. 2.4.: DFT Approximations to Fourier Series Coefficients [NONPERIODIC FUNCTIONS]
% - So far, the Fourier series has been defined only for periodic functions. (p. 38)
% - However, an important case that arises often is that in which
%   f IS DEFINED AND PIECEWISE SMOOTH ONLY ON THE INTERVAL [-A/2, A/2];
%   perhaps f is not defined outside of that interval, or perhaps it is not a periodic function at all. (p. 38)
% - In order to handle this situation we need to know about
%   the PERIODIC EXTENSION of f, the function h defined by
%   [ h( x + sA ) = f( x ), x \in ( -A/2, A/2 ), s = 0, \pm 1, \pm 2, ... . ] (p. 38)
% - The PERIODIC EXTENSION of f is simply
%   the REPETITION OF f EVERY A UNITS ON BOTH SIDES OF THE INTERVAL [-A/2, A/2]. (p. 38)
% - Here is the IMPORTANT ROLE OF THE PERIODIC EXTENSION h:
%   if the FOURIER SERIES for f CONVERGES ON [-A/2, A/2], then it CONVERGES
%   [1.)] to the VALUE of f at POINTS OF CONTINUITY on (-A/2, A/2),
%   [2.)] to the AVERAGE VALUE of f at POINTS OF DISCONTINUITY on (-A/2, A/2),
%   [3.)] to the VALUE OF THE PERIODIC EXTENSION of f at POINTS OF CONTINUITY outside of (-A/2, A/2), and
%   [4.)] to the AVERAGE VALUE OF THE PERIODIC EXTENSION at POINTS OF DISCONTINUITY outside of (-A/2, A/2). (p. 38)
% - The PERIODIC EXTENSION is
%   the FUNCTION TO WHICH THE FOURIER SERIES OF f CONVERGES FOR ALL x provided
%   we use AVERAGE VALUES at POINTS OF DISCONTINUITY. (p. 38)
% - In particular, if f( -A/2 ) \neq f( A/2 ) then
%   the Fourier series converges to the average of the function values at the right and left endpoints
%   [ \frac{ 1 }{ 2 } \left[ f( -A/2+ ) + f( A/2- ) \right]. ] (p. 38)
% - These facts must be observed scrupulously when a function is sampled for input to the DFT. (p. 38)
The finite recording times
$T_{ \text{rec} }^{(n)} = \tabs{ \setsymbol{T}_{ \text{rec} }^{(n)} } = t_{\text{ub}}^{(n)} - t_{\text{lb}}^{(n)}$ enable
the representation of
% 1.) RF voltage signals
these signals by
% 2.) Fourier series
the \name{Fourier} series
%\footnote{
  % a) adjective "stable" indicates that neither inaccurate observations nor a sparsity defect result in huge recovery errors
%  validity
%}
(cf. e.g.
%\cite[(3.39/40)]{book:Mallat2009},
\cite[(2.2.1/2)]{book:Manolakis2005},
\cite[(2.12/13)]{book:Briggs1995}%
)
\begin{subequations}
\label{eqn:recovery_disc_freq_v_rx_Fourier_series}
\begin{equation}
 %--------------------------------------------------------------------------------------------------------------
 % Fourier series representation of the recorded RF voltage signals (time domain)
 %--------------------------------------------------------------------------------------------------------------
  \tilde{u}_{m}^{(\text{rx}, n)}( t )
  =
  u_{m, 0}^{(\text{rx}, n)}
  +
  2
  \dreal{
    \sum_{ l = 1 }^{ \infty }
      u_{m, l}^{(\text{rx}, n)}
      e^{ j \omega_{l} t }
  }{2}
 \label{eqn:recovery_disc_freq_v_rx_Fourier_series_sum}
\end{equation}
for
% 3.) all sequential pulse-echo measurements and all array elements
all $( n, m ) \in \setconsnonneg{ N_{\text{in}} - 1 } \times \setconsnonneg{ N_{\text{el}} - 1 }$, where
% 4.) discrete angular frequencies
$\omega_{l} = 2 \pi f_{l} = 2 \pi l / T_{ \text{rec} }^{(n)}$ denote
the discrete angular frequencies, and
% 5.) Fourier coefficients of the recorded RF voltage signals
\begin{equation}
 %--------------------------------------------------------------------------------------------------------------
 % Fourier coefficients of the recorded RF voltage signals
 %--------------------------------------------------------------------------------------------------------------
  u_{m, l}^{(\text{rx}, n)}
  =
  \frac{ 1 }{ T_{ \text{rec} }^{(n)} }
  \int_{ \setsymbol{T}_{ \text{rec} }^{(n)} }
    \tilde{u}_{m}^{(\text{rx}, n)}( t )
    e^{ -j \omega_{l} t }
  \text{d} t
 \label{eqn:recovery_disc_freq_v_rx_Fourier_series_coef}
\end{equation}
\end{subequations}
are
the complex-valued coefficients, whose
% 6.) conjugate even symmetry
conjugate even symmetry renders
% 7.) negative frequency indices
the negative frequency indices redundant.

%---------------------------------------------------------------------------------------------------------------
% 2.) bandpass characters of the recorded RF voltage signals / truncation of the Fourier series
%---------------------------------------------------------------------------------------------------------------
% a) bandpass characters of the recorded RF voltage signals define the sets of relevant discrete frequencies
The bandpass characters of
% 1.) recorded RF voltage signals
the recorded \ac{RF} voltage signals, which are described by
the lower and
upper frequency bounds
% 2.) lower frequency bounds
$f_{\text{lb}}^{(n)} \in \Rplus$ and
% 3.) upper frequency bounds
$f_{\text{ub}}^{(n)} \geq f_{\text{lb}}^{(n)} + 1 / T_{ \text{rec} }^{(n)}$,
respectively, define
% 4.) sets of relevant discrete frequencies
the sets of
relevant discrete frequencies
\begin{subequations}
\label{eqn:recon_disc_axis_f_discrete_BP}
\begin{equation}
 %--------------------------------------------------------------------------------------------------------------
 % sets of relevant discrete frequencies
 %--------------------------------------------------------------------------------------------------------------
  \setsymbol{F}_{ \text{BP} }^{(n)}
  =
  \Bigl\{
    f_{l} \in \Rplus:
    f_{l} = \frac{ l }{ T_{ \text{rec} }^{(n)} },
    l \in \setsymbol{L}_{ \text{BP} }^{(n)}
  \Bigr\}
 \label{eqn:recon_disc_axis_f_discrete_BP_set}
\end{equation}
for
% 5.) all sequential pulse-echo measurements
all $n \in \setconsnonneg{ N_{\text{in}} - 1 }$, where
% 6.) sets of admissible frequency indices
the admissible index sets are
\begin{equation}
 %--------------------------------------------------------------------------------------------------------------
 % sets of admissible frequency indices
 %--------------------------------------------------------------------------------------------------------------
  \setsymbol{L}_{ \text{BP} }^{(n)}
  =
  \left\{
    l \in \N:
    l_{\text{lb}}^{(n)} \leq l \leq l_{\text{ub}}^{(n)}
  \right\}
 \label{eqn:recon_disc_axis_f_discrete_BP_indices}
\end{equation}
with
the lower and
upper bounds
% a) lower bounds on the admissible frequency indices
% 1.) t_{\text{lb}}^{(n)} \in \Rnonneg and t_{\text{ub}}^{(n)} > t_{\text{lb}}^{(n)}
% => T_{ \text{rec} }^{(n)} = t_{\text{ub}}^{(n)} - t_{\text{lb}}^{(n)} > 0
% 2.) f_{\text{lb}}^{(n)} \in \Rplus and f_{\text{ub}}^{(n)} \geq f_{\text{lb}}^{(n)} + 1 / T_{ \text{rec} }^{(n)} > f_{\text{lb}}^{(n)} > 0
% => T_{ \text{rec} }^{(n)} f_{\text{lb}}^{(n)} > 0
% => l_{\text{lb}}^{(n)} = \dceil{ T_{ \text{rec} }^{(n)} f_{\text{lb}}^{(n)} }{1} \in \N
% b) upper bounds on the admissible frequency indices
% => T_{ \text{rec} }^{(n)} f_{\text{ub}}^{(n)} \geq T_{ \text{rec} }^{(n)} f_{\text{lb}}^{(n)} + 1
% => l_{\text{ub}}^{(n)} = \dfloor{ T_{ \text{rec} }^{(n)} f_{\text{ub}}^{(n)} }{1} \geq \dfloor{ T_{ \text{rec} }^{(n)} f_{\text{lb}}^{(n)} }{1} + 1 \geq \dceil{ T_{ \text{rec} }^{(n)} f_{\text{lb}}^{(n)} }{1} = l_{\text{lb}}^{(n)}
\begin{align}
 %--------------------------------------------------------------------------------------------------------------
 % a) lower bounds on the admissible frequency indices
 %--------------------------------------------------------------------------------------------------------------
  l_{\text{lb}}^{(n)}
  &=
  \dceil{ T_{ \text{rec} }^{(n)} f_{\text{lb}}^{(n)} }{1}
  & \text{and} & &
 %--------------------------------------------------------------------------------------------------------------
 % b) upper bounds on the admissible frequency indices
 %--------------------------------------------------------------------------------------------------------------
  l_{\text{ub}}^{(n)}
  &=
  \dfloor{ T_{ \text{rec} }^{(n)} f_{\text{ub}}^{(n)} }{1},
 \label{eqn:recon_disc_axis_f_discrete_BP_indices_lb_ub}
\end{align}
\end{subequations}
respectively.
% b) sets of relevant discrete frequencies enable the truncation of each Fourier series and the representation of each pulse-echo measurement by N_{\text{el}} N_{f, \text{BP}}^{(n)} complex-valued coefficients
These enable
the truncation of
% 1.) Fourier series representation of the recorded RF voltage signals (time domain)
each \name{Fourier} series
\eqref{eqn:recovery_disc_freq_v_rx_Fourier_series_sum}, where, defining
% 2.) effective bandwidths
the effective bandwidths
$B_{ u }^{(n)} = f_{\text{ub}}^{(n)} - f_{\text{lb}}^{(n)}$,
% 3.) number of relevant discrete frequencies
the number of
relevant discrete frequencies approximates
% 4.) effective time-bandwidth products
the effective time-bandwidth products
\begin{equation}
 %--------------------------------------------------------------------------------------------------------------
 % numbers of relevant discrete frequencies (effective time-bandwidth products)
 %--------------------------------------------------------------------------------------------------------------
  N_{f, \text{BP}}^{(n)}
  =
  \dabs{ \setsymbol{L}_{ \text{BP} }^{(n)} }{1}
  =
  l_{\text{ub}}^{(n)} - l_{\text{lb}}^{(n)} + 1
  \approx
  T_{ \text{rec} }^{(n)} B_{ u }^{(n)}
 \label{eqn:recon_disc_axis_f_discrete_BP_TB_product}
\end{equation}
for
% 5.) all sequential pulse-echo measurements
all $n \in \setconsnonneg{ N_{\text{in}} - 1 }$, and
the representation of
each pulse-echo measurement by
% 6.) Fourier coefficients of the recorded RF voltage signals
$N_{\text{el}} N_{f, \text{BP}}^{(n)}$ coefficients
\eqref{eqn:recovery_disc_freq_v_rx_Fourier_series_coef}.
% c) let the subscript l indicate an admissible frequency index in the sets of relevant discrete frequencies in the following
%In the following,
%the subscript
%$l \in \setsymbol{L}_{ \text{BP} }^{(n)}$ indicates
%an admissible frequency index.
% d) dependence on the superscript n is implicitly understood
%For the sake of
%notational lucidity,
%its dependence on
%the superscript $n$, which identifies
%the sequential pulse-echo measurement, is
%implicitly understood.
