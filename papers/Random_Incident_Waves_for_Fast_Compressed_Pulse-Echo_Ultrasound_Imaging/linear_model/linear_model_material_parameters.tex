%---------------------------------------------------------------------------------------------------------------
% 1.) physical model for soft tissue structures and its relevant acoustic material parameters
%---------------------------------------------------------------------------------------------------------------
% a) medical UI usually models soft tissue structures as quiescent, lossless, and heterogeneous fluids that linearly propagate small-amplitude disturbances of the stationary state as longitudinal waves
% article:JensenProgBMB2007: Medical ultrasound imaging
% 3. Anatomic ultrasound imaging
% - MODERN MEDICAL ULTRASOUND SCANNERS are used for imaging
%   NEARLY ALL SOFT TISSUE STRUCTURES IN THE BODY. (p. 155)
% coll:Jensen2002: Ultrasound Imaging and Its Modeling
% ABSTRACT
% - MODERN MEDICAL ULTRASOUND SCANNERS are used to image
%   NEARLY ALL SOFT TISSUE STRUCTURES IN THE BODY. (p. 135)
% article:MastJASA1997:
% - Ultrasonic pulse propagation through the HUMAN ABDOMINAL WALL was modelled using
%   the equations of motion for a LOSSLESS FLUID WITH VARIABLE SOUND SPEED AND DENSITY.
% article:JensenJASA1991: A model for the propagation and scattering of ultrasound in tissue
% INTRODUCTION
% - ULTRASOUND is used with great success in
%   the DIAGNOSIS OF ABNORMALITIES in SOFT TISSUE STRUCTURES IN THE HUMAN BODY. (p. 182)
% article:GorePMB1977a: Ultrasonic backscattering from human tissue: A realistic model
% 1. Introduction
% - In this paper,
%   A TISSUE MODEL IS CHOSEN ON THE BASIS OF SIMPLE BUT REALISTIC ASSUMPTIONS and
%   the scattering of a typical diagnostic pulse is calculated. (p. 318)
Medical \ac{UI} usually models
soft tissue structures%
\footnote{
  % a) strict definition of the term "soft tissue" (anatomy)
  The anatomic term \term{soft tissue} refers to
  tendons, ligaments, skin, nerves,
  % website:NCIDictionary2017: NCI Dictionary of Cancer Terms
  % - 'soft tissue': Refers to MUSCLE, FAT, FIBROUS TISSUE, blood vessels, or other SUPPORTING TISSUE of the BODY.
  % fibrous tissue =  the common connective tissue of the body, composed of yellow or white parallel elastic and collagen fibers.
  muscle, fat, fibrous tissue, blood vessels, or
  other supporting tissue of
  the body
  \cite[\term{soft tissue}]{website:NCIDictionary2017}.
  % b) loose definition of the term "soft tissue" (UI)
  In the context of \ac{UI}, however,
  the term additionally includes
  organs like
  liver, kidney, thyroid, brain, and
  the heart.
} as
% 1.) quiescent
% article:MastJASA1997:
% - The TISSUE WAS ASSUMED TO BE MOTIONLESS except for small acoustic perturbations.
quiescent,
% 2.) lossless
% article:JensenJASA1991: A model for the propagation and scattering of ultrasound in tissue
% INTRODUCTION
% - The model includes attenuation due to propagation and scattering, but not
%   THE DISPERSIVE ATTENUATION OBSERVED FOR PROPAGATION IN TISSUE. (p. 182)
% - This can, however, be incorporated into the model as indicated in Sec. VI. (p. 182)
% I. DERIVATION OF THE WAVE EQUATION
% - Our second assumption is that
%   NO HEAT CONDUCTION OR CONVERSION OF ULTRASOUND TO THERMAL ENERGY TAKE PLACE. (p. 182)
% article:GorePMB1977a: Ultrasonic backscattering from human tissue: A realistic model
% 1. Introduction
% - For simplicity, however, such ABSORPTION EFFECTS ARE NOT CONSIDERED HERE,
%   but the introduction of simple exponential absorption leads to only minor changes in the theory. (p. 318)
lossless, and
% 3.) heterogeneous
% article:NgITUFFC2006: Modeling ultrasound imaging as a linear, shift-variant system
% III. The Wave Equation
% - Our analysis necessarily begins by considering
%   the PARTIAL DIFFERENTIAL EQUATION (PDE) that describes
%   the PROPAGATION OF ACOUSTIC WAVES IN A NONUNIFORM MEDIUM. (p. 550)
% article:JensenJASA1991: A model for the propagation and scattering of ultrasound in tissue
% Abstract
% - An INHOMOGENEOUS WAVE EQUATION is derived describing PROPAGATION AND SCATTERING OF ULTRASOUND IN AN INHOMOGENEOUS MEDIUM. (p. 182)
heterogeneous fluids that linearly propagate
% 4.) small-amplitude approximation
% article:NgITUFFC2006: Modeling ultrasound imaging as a linear, shift-variant system
% I. Introduction
% - We restrict ourselves to consider LINEAR WAVE PROPAGATION ONLY. (p. 549)
% article:JensenJASA1991: A model for the propagation and scattering of ultrasound in tissue
% I. DERIVATION OF THE WAVE EQUATION
% - To obtain a solvable wave equation, some ASSUMPTIONS AND APPROXIMATIONS MUST BE MADE. (p. 182)
% - The first one states that the instantaneous acoustic pressure and density can be written as
%   [ P_{\text{ins}}( \vect{r}, t ) = P + p_{1}( \vect{r}, t ) ], (1)
%   [ \rho_{\text{ins}}( \vect{r}, t ) = \rho( \vect{r} ) + \rho_{1}( \vect{r}, t ) ], (2)
%   in which P is the mean pressure of the medium and \rho is the density of the undisturbed medium. (p. 182)
% - The PRESSURE VARIATION p_{1} IS CAUSED BY THE ULTRASOUND WAVE AND IS CONSIDERED SMALL compared to P. (p. 182)
% - The density change caused by the wave is \rho_{1}. (p. 182)
% - Both p_{1} and \rho_{1} are SMALL QUANTITIES OF FIRST ORDER. (p. 182)
small-amplitude disturbances of
the stationary state as
% 5.) longitudinal waves
% article:GorePMB1977a: Ultrasonic backscattering from human tissue: A realistic model
% 2. The wave equation for ultrasound propagation through tissue
% - ULTRASOUND PROPAGATION BY MODES OTHER THAN PURELY LONGITUDINAL IS NEGLECTED,
%   not only for reasons of simplicity, but also because
%   their significance is not well documented or understood for scattering from tissue. (p. 320)
longitudinal waves
\cite{article:NgITUFFC2006,coll:Jensen2002,article:JensenJASA1991,article:GorePMB1977a}.
% b) relevant acoustic material parameters are the unperturbed values of both the compressibility and the mass density (normalized by spatial averages)
% article:NgITUFFC2006: Modeling ultrasound imaging as a linear, shift-variant system
% III. The Wave Equation
% - We shall use the WAVE EQUATION that is found in [1] and [9], in which
%   the ACOUSTIC PROPERTIES OF THE MEDIUM ARE SPECIFIED IN TERMS OF
%   its DENSITY AND ADIABATIC COMPRESSIBILITY. (p. 550)
%   [1] article:GorePMB1977a: Ultrasonic backscattering from human tissue: A realistic model
%   [9] book:Morse1986: Theoretical Acoustics
% III. The Wave Equation / A. The Total Pressure Field
% - In the ABSENCE OF ANY SCATTERERS,
%   WE CONSIDER OUR MEDIUM TO BE UNIFORM with DENSITY \rho_{o} AND ADIABATIC COMPRESSIBILITY \kappa_{0}. (p. 550)
% - The PRESENCE OF SCATTERERS in the medium may be modeled by adding
%   SPATIALLY-DEPENDENT TERMS ∆\rho( \vect{x} ) and ∆\kappa( \vect{x} ) to
%   THE DENSITY AND THE COMPRESSIBILITY, respectively. (p. 550)
% article:JensenJASA1991: A model for the propagation and scattering of ultrasound in tissue
% Abstract
% - The SCATTERING TERM IS A FUNCTION OF
%   [1.)] DENSITY AND
%   [2.)] PROPAGATION VELOCITY PERTURBATIONS. (p. 182)
% INTRODUCTION
% - In MEDICAL ULTRASOUND, a pulse is emitted into the body and
%   is SCATTERED AND REFLECTED BY
%   [1.)] DENSITY AND
%   [2.)] PROPAGATION VELOCITY PERTURBATIONS. (p. 182)
% - The RECEIVED FIELD CAN BE FOUND BY SOLVING AN APPROPRIATE WAVE EQUATION. (p. 182)
% - This has been done in a number of papers. [1,2] (p. 182)
% - Gore and Leeman [1] considered a wave equation where
%   THE SCATTERING TERM WAS A FUNCTION OF THE
%   [1.)] ADIABATIC COMPRESSIBILITY AND
%   [2.)] THE DENSITY.
%   The transducer was modeled by an axial and lateral pulse that were separable. (p. 182)
%   [1] article:GorePMB1977a: Ultrasonic backscattering from human tissue: A realistic model
% - Fatemi and Kak [2] used a wave equation where
%   THE SCATTERING ONLY ORIGINATED FROM VELOCITY FLUCTUATIONS, and
%   the transducer was restricted to be circularly symmetric and unfocused (flat). (p. 182)
%   [2] M. Fatemi and A. C. Kak, "Ultrasonic B-scan imaging: Theory of image formation and a technique for restoration," Ultrason. Imag. 2, 1-47 (1980).
% - The scattering term for the wave equation used in this paper is
%   A FUNCTION OF DENSITY AND PROPAGATION VELOCITY PERTURBATIONS, and
%   the wave equation is EQUIVALENT TO THE ONE USED BY GORE AND LEEMAN.[1] (p. 182)
% I. DERIVATION OF THE WAVE EQUATION
% - The wave equation [(16)] was derived in Chernov. [3] (p. 183)
%   [3] L. A. Chernov, [leave Propagation in a Random Medium (McGraw. Hill, New York, 1960).
% - It has also been considered in Gore and Leeman [1] and Morse and Ingard [4] in a slightly different form, where
%   the SCATTERING TERMS WERE A FUNCTION OF THE ADIABATIC COMPRESSIBILITY \kappa AND THE DENSITY. (p. 183)
% article:GorePMB1977a: Ultrasonic backscattering from human tissue: A realistic model
% Abstract
% - The propagation of ultrasound pulses in INHOMOGENEOUS MEDIA is described, and it is shown that they are scattered by
%   FLUCTUATIONS IN DENSITY AND COMPRESSIBILITY. (p. 317)
% 2. The wave equation for ultrasound propagation through tissue
% - The DENSITY AND COMPRESSIBILITY OF SMALL TISSUE SAMPLES FLUCTUATE FROM PLACE TO PLACE ABOUT THEIR MEAN VALUES so that in any region
%   the LOCAL ACOUSTIC PROPERTIES DIFFER FROM THE AVERAGE. (p. 318)
% - SOUND WAVES WILL BE SCATTERED IN SUCH AN INHOMOGENEOUS MEDIUM and
%   the ECHOES FROM INSIDE TISSUE CAN BE ATTRIBUTED TO DENSITY-COMPRESSIBILITY FLUCTUATIONS, which
%   may be DISTRIBUTED RANDOMLY OR REGULARLY throughout the tissue. (pp. 318, 319)
% - No assumptions are necessary as to the random nature, or otherwise, of the variables of interest, but
%   THE WEAKNESS OF THE SCATTERING OBSERVED IN PRACTICE IMPLIES THAT
%   THE MAGNITUDE OF FLUCTUATIONS MAY BE CONSIDERED TO BE SMALL. (p. 319)
The relevant acoustic material parameters are
the unperturbed values of both
% 1.) unperturbed compressibility
the compressibility and
% 2.) unperturbed mass density
the mass density, which are typically normalized by
% 3.) spatial averages
% book:Kak2001, Sect. 6.1.2: Inhomogeneous Wave Equation
% - \kappa_{0} and \rho_{0} are either
%   the compressibility and the density of the medium in which the object is immersed, or
%   THE AVERAGE COMPRESSIBILITY AND THE DENSITY OF THE OBJECT, depending upon how the process of imaging is modeled. (p. 210)
% article:JensenJASA1991: A model for the propagation and scattering of ultrasound in tissue
% I. DERIVATION OF THE WAVE EQUATION
% - We now assume that
%   [1.)] the PROPAGATION VELOCITY AND
%   [2.)] THE DENSITY ONLY VARY SLIGHTLY FROM
%   THEIR MEAN VALUES, so that
%   [ \rho( \vect{r} ) = \rho_{0} + \Delta \rho( \vect{r} ) ],
%   [ c( \vect{r} ) = c_{0}( \vect{r} ) + \Delta c( \vect{r} ) ], (12)
%   where \rho_{0} \gg \Delta \rho and c_{0} \gg \Delta c. (p. 183)
% article:GorePMB1977a: Ultrasonic backscattering from human tissue: A realistic model
% 2. The wave equation for ultrasound propagation through tissue
% - It is convenient to consider formally
%   the scattering region V to be embedded in SOME NON-DISPERSIVE MEDIUM WITH
%   CONSTANT DENSITY, \rho_{0}, and
%   COMPRESSIBILITY \kappa_{0} = ( \rho_{0} {c_{0}}^{2} )^{-1}, with
%   c_{0} the acoustic velocity in the embedding medium. (p. 319)
% - THE VALUES OF THESE PARAMETERS ARE CHOSEN TO BE THE MEAN VALUES THEY ASSUME INSIDE V. (p. 319)
their spatial averages
% 4.) reference value for the unperturbed compressibility
$\kappa_{0} \in \Rplus$ and
% 5.) reference value for the unperturbed mass density
$\rho_{0} \in \Rplus$,
respectively
\cite{article:NgITUFFC2006,article:JensenJASA1991,article:GorePMB1977a}.
% c) proposed model considers a homogeneous unperturbed mass density
For
the sake of
simplicity,
the proposed model considers
a homogeneous unperturbed mass density.
% d) global unperturbed compressibility / associated relative spatial fluctuations
The relative spatial fluctuations in
the unperturbed compressibility read
(cf. \cref{fig:lin_mod_scan_configuration})
\begin{equation}
 %--------------------------------------------------------------------------------------------------------------
 % relative spatial fluctuations in the unperturbed compressibility
 %--------------------------------------------------------------------------------------------------------------
  \gamma^{(\kappa)}( \vect{r} )
  =
  \begin{cases}
    1 - \kappa_{1}( \vect{r} ) / \kappa_{0} & \text{for } \vect{r} \in \Omega,\\
    0 & \text{for } \vect{r} \notin \Omega.\\
  \end{cases}
 \label{eqn:lin_mod_mech_model_tis_simple_rel_fluctuations}
\end{equation}

%---------------------------------------------------------------------------------------------------------------
% 2.) power-law dependence of the spatial amplitude absorption coefficient on the frequency
%---------------------------------------------------------------------------------------------------------------
% a) spatial amplitude absorption coefficient obeys the power law in the entire Euclidean space
% book:Szabo2013, Chapter 4: Attenuation / Sect. 4.1: Losses in Tissues
% - REAL TISSUE DATA INDICATE that ATTENUATION HAS A POWER-LAW DEPENDENCE on FREQUENCY. (p. 82)
% book:Szabo2013, Chapter 4: Attenuation / Sect. 4.1: Losses in Tissues / Sect. 4.1.2: Tissue Data
% - These simple loss and delay factors are NOT OBSERVED IN REAL MATERIALS AND TISSUES. (p. 84)
% - Data indicate that the absorption is a function of frequency. (p. 84)
% - MANY OF THESE LOSSES OBEY A FREQUENCY POWER LAW, defined as
%   (4.6A) [\alpha( f ) = \alpha_{0} + \alpha_{1} \abs{ f }^{y}], in which
%   \alpha_{0} is OFTEN ZERO and y is a POWER LAW EXPONENT. (p. 84)
% - A graph for the MEASURED ABSORPTION OF COMMON TISSUES AS A FUNCTION OF FREQUENCY is given in
%   Figure 4.2A. (p. 84)
% article:KellyJASA2008b: Analytical time-domain Green's functions for power-law media
% I. INTRODUCTION
% - The ATTENUATION COEFFICIENT FOR BIOLOGICAL TISSUE may be approximated by
%   A POWER LAW [1] OVER A WIDE RANGE OF FREQUENCIES. (p. 2861)
%   [1] book:Duck1990, pp. 99-124
% II. FPDE FORMULATIONS OF THE SZABO AND POWER-LAW WAVE EQUATIONS / A. Szabo wave equation
% - The SZABO WAVE EQUATION [4] APPROXIMATES POWER-LAW MEDIA with
%   AN ATTENUATION COEFFICIENT GIVEN BY (1) [\alpha( \omega ) = \alpha_{0} \abs{ \omega }^{ y }]. (p. 2862)
% II. FPDE FORMULATIONS OF THE SZABO AND POWER-LAW WAVE EQUATIONS / C. Power-law wave equation
% - In addition, Eq. (8) [reciprocal phase velocity] is in CLOSE AGREEMENT with
%   the EXPERIMENTAL DISPERSION DATA presented in Refs. 12–14 and 17.
%   [12] article:ODonnellJASA1981, [13] article:SzaboJASA1995, [14] article:WatersJASA2000a, [17] article:HeITUFFC1998
% - Thus, Eq. (2) [Szabo's approximate wave equation] supports
%   the POWER-LAW ATTENUATION AND DISPERSION that is
%   PREDICTED BY THE KRAMERS-KRONIG RELATIONSHIPS and SUPPORTED BY EXPERIMENTAL MEASUREMENTS. (p. 2863)
% book:Cobbold2006, Sect. 3.10: Effects of Attenuation / Sect. 3.10.1 Kramers-Kronig Relationships
% - As noted in 1.8.1 (see footnote 35),
%   LONGITUDINAL WAVE PROPAGATION IN MOST SOFT TISSUE IS FOUND TO HAVE
%   A FREQUENCY-DEPENDENT ATTENUATION that
%   IS GENERALLY WELL APPROXIMATED BY
%   [ \alpha = \alpha_{0}' \abs{ \omega }^{n} ] (3.97), where
%   \alpha_{0}' is the ANGULAR FREQUENCY ATTENUATION FACTOR and
%   n is a real positive number that TYPICALLY LIES IN THE RANGE 1 \leq n \leq 2. (p. 207)
% book:Cobbold2006, Sect. 1.8.1: Absorption and Scattering Attenuation Coefficients / Attenuation of Biological Tissues
% - It can be seen that
%   a GOOD APPROXIMATION for the FREQUENCY DEPENDENCE for most SOFT TISSUE is given by
%   [ \alpha = \alpha_{0} f^{n} ] (1.125), where
%   n lies in the RANGE FROM 1 to 2. (p. 74)
% article:WatersITUFFC2005: Causality-imposed (Kramers-Kronig) relationships between attenuation and dispersion
% IV. Applications of the Acoustic Kramers-Kronig Dispersion Relations
% - We investigate two cases often considered in ultrasonic research:
%   MEDIA WITH AN ATTENUATION COEFFICIENT OBEYING A FREQUENCY POWER LAW, and
%   suspensions with resonant scattering properties. (p. 825)
% - The first case of POWER-LAW ATTENUATION is often considered for
%   PROPAGATION IN MANY SOFT TISSUES [29], [48], [49] AND LIQUIDS [26], [50]. (p. 825)
% IV. Applications of the Acoustic Kramers-Kronig Dispersion Relations / A. Power-Law Attenuation
% - However, IT OFTEN IS OBSERVED THAT THE ATTENUATION IN MANY LIQUIDS (AND OTHER MEDIA) DOES NOT EXHIBIT an ω2-dependence. (p. 826)
% - In such cases, the ATTENUATION COEFFICIENT often is found to be well described by a POWER LAW: (8) [\alpha( \omega ) = \alpha_{0} \omega^{y}],
%   where \omega is angular frequency, and \alpha_{0} and y are material-dependent parameters with y typically between 1 and 2, inclusive. (p. 826)
% article:WatersJASA2000a: On the applicability of Kramers–Krönig relations for ultrasonic attenuation obeying a frequency power law
% I. THEORY
% - In a VARIETY OF MEDIA (e.g., LIQUIDS AND TISSUE) OVER A FINITE BANDWIDTH,
%   the ATTENUATION OF ULTRASONIC WAVES APPEARS TO BE ADEQUATELY MODELED BY
%   a POWER-LAW DEPENDENCE ON FREQUENCY, [2,3,11] (1) [ \alpha( \omega ) = \alpha_{0} \abs{ \omega }^{y} ], where
%   we assume \alpha_{0} and y are REAL CONSTANTS, with 0 < y <= 2 typically. (p. 556)
% book:Duck1990, Sect. 4.3.8: Values of acoustic absorption coefficients in tissue
% - Measured values of ABSORPTION COEFFICIENTS FOR ULTRASOUND IN SOFT TISSUE are given in
%   Tables 4.19 and 4.20. (p. 115)
% - VALUES AT PARTICULAR FREQUENCIES ARE INCLUDED IN Table 4.19, and
%   the POWER-LAW EXPRESSION Equation 4.30 [ \alpha = a f^{b} ] used as the basis for the values given in Table 4.20. (p. 115)
% - Many of the factors discussed for attenuation in Section 4.3.5 are EQUALLY RELEVANT TO ABSORPTION,
%   and reference should be made to this section. (p. 115)
% - Tab. 4.19: Ultrasound ABSORPTION COEFFICIENT for SOFT TISSUES (i) (specific frequencies)
% - Tab. 4.20: Ultrasound ABSORPTION COEFFICIENT (ii); \alpha = a f^{b} (power law parameters)
% book:Duck1990, Sect. 4.3.5.2: Frequency (Sect. 4.3.5: Factors affecting attenuation)
% - The FREQUENCY DEPENDENCY of ULTRASONIC ATTENUATION AND ABSORPTION can be represented by
%   the expression [ \alpha = a f^{b} ] (4.30) where
%   a [coefficient], b [exponent] are constants and f is frequency. (p. 112)
% - Several authors have fitted their data to this expression over
%   NARROW RANGES OF FREQUENCY. (p. 112)
% - Others have assumed a LINEAR FREQUENCY DEPENDENCE (b = 1), particularly when estimating
%   ATTENUATION COEFFICIENTS FROM IN-VIVO MEASUREMENTS USING THE FREQUENCY CONTENT OF BACKSCATTERED SOUND. (p. 112)
% - Data in both forms are included in Tables 4.16 and 4.17 [ATTENUATION COEFFICIENTS!]. (p. 112)
% - Tab. 4.16 Ultrasound amplitude ATTENUATION COEFFICIENT for NORMAL TISSUE: \alpha = a f^{b}
% - Tab. 4.17 Amplitude ATTENUATION COEFFICIENT, \alpha = a f^{b}; HUMAN PATHOLOGICAL TISSUES
% - There is some evidence that Equation 4.30 [\alpha = a f^{b}] may
%   NOT BE APPROPRIATE OVER A WIDER FREQUENCY RANGE. (p. 112)
% article:WellsUMB1975: Absorption and dispersion of ultrasound in biological tissue
% ABSORPTION AND VELOCITY DATA FOR BIOLOGICAL MATERIALS
% - The DATA FOR ABSORPTION are presented in Fig. 1. (p. 370)
% - It is IMMEDIATELY APPARENT that,
%   for BIOLOGICAL SOFT TISSUES, (4) [\alpha = a f^{b}], where
%   a [coefficient] and b [exponent] depend upon
%   THE CHARACTERISTICS OF THE PARTICULAR TISSUE and
%   THE CONDITIONS OF MEASUREMENT (such as temperature), and
%   have FAIRLY CONSTANT VALUES OVER LIMITED RANGES OF FREQUENCY. (p. 370)
The spatial amplitude absorption coefficient
$\alpha_{l} \in \Rnonneg$, which is
% 1.) commonly neglected
% article:GorePMB1977a: Ultrasonic backscattering from human tissue: A realistic model
% 1. Introduction
% - Ideally, the scattering from a particular region should be specified in a way which allows
%   the EFFECT OF THE OVERLYING TISSUE TO BE EASILY QUANTIFIED; in particular,
%   the EFFECT OF FREQUENCY DEPENDENT ATTENUATION SHOULD BE INCLUDED. (p. 318)
% 4. Discussion and implications of results
% - The TISSUE MODEL DISCUSSED HERE MAY BE EXTENDED TO INCLUDE EFFECTS SUCH AS ABSORPTION, and
%   further study of this is under way. (p. 325)
commonly neglected, obeys
% 2.) power law
the power law
(cf. e.g.
\cite[Sect. 4.3.8]{book:Duck1990},
\cite[(4)]{article:WellsUMB1975}%
)
\begin{equation}
 %--------------------------------------------------------------------------------------------------------------
 % spatial amplitude absorption coefficient
 %--------------------------------------------------------------------------------------------------------------
  \alpha_{l}
  =
  \bar{b} \abs{ \omega_{l} }^{ \zeta }
 \label{eqn:lin_mod_mech_model_tis_abs_power_law}
\end{equation}
in
% 3.) entire Euclidean space
the entire Euclidean space for
% 4.) all relevant discrete frequencies
all relevant discrete frequencies
$l \in \setsymbol{L}_{ \text{BP} }^{(n)}$, where
% 5.) pair of absorption parameters
the parameter pair
$( \bar{b}, \zeta ) \in \Rnonneg \times \Rnonneg$ depends on both
% 6.) specific type of tissue
the specific type of
tissue and
% 7.) ambient conditions
the ambient conditions.
% b.) exponent \zeta usually ranges between 1 and 1.5
% article:JensenProgBMB2007: Medical ultrasound imaging
% - Typically, an attenuation of 0.5 dB/(MHz cm) is experienced in the SOFT TISSUES.
% article:KellyJASA2008b: Analytical time-domain Green's functions for power-law media
% I. INTRODUCTION
% - MEASURED ATTENUATION COEFFICIENTS OF SOFT TISSUE TYPICALLY HAVE
%   LINEAR OR GREATER THAN LINEAR DEPENDENCE ON FREQUENCY. (p. 2861)
% - For example, breast fat has a power-law exponent of y = 1.5, while
%   breast tissue ranges [1] from y = 1 to y = 1.5. (p. 2861)
%   [1] book:Duck1990, pp. 99-124
% article:SzaboJASA2000: A model for longitudinal and shear wave propagation in viscoelastic media
% - [...] y is a POSITIVE NUMBER USUALLY LESS THAN 2 [...]. (p. 2437)
% - In contrast, for ACOUSTIC WAVES in VISCOELASTIC MEDIA, the FREQUENCY EXPONENT is most often not fractional,
%   but has been found to VARY FROM 0 TO 2. (p. 2442)
% - LONGITUDINAL MODE ABSORPTION obeys a POWER LAW with an EXPONENT most frequently in the range of
%   1 < y < 2 for a LARGE NUMBER OF MATERIALS, both FLUID and SOLID
%   [Duck (1990); Zeqiri (1988); Bamber (1986); Szabo (1993, 1994, 1995); O’Donnell (1981),He (1998a, 1998b, 1999)]. (pp. 2442, 2443)
% article:SzaboJASA1995: Causal theories and data for acoustic attenuation obeying a frequency power law
% I. TIME DOMAIN CAUSAL RELATIONSHIPS / B. Anomalous dispersion
% - MOST MATERIALS FALL IN THE RANGE 0 < y < 2. (p. 16)
% - FOR THE MAJORITY OF CASES, y >= 1, [...] (p. 16)
% II. CAUSAL RELATIONS IN THE FREQUENCY DOMAIN / A. Horton's dispersion relationships
% - Note that our PRIMARY INTEREST in this study is for
%   the EXPONENT RANGE 0 <= y <= 2 in which MOST MATERIALS FALL. (p. 16)
% book:Duck1990, Sect. 4.3.5 Factors affecting attenuation / Sect. 4.3.5.2 Frequency
% - Values for b for MOST SOFT TISSUE and BIOLOGICAL FLUIDS LIE IN THE RANGE 1.0 to 1.5. (p. 112)
% article:WellsUMB1975: Absorption and dispersion of ultrasound in biological tissue
% - The value of b [exponent] is GENERALLY ONLY A LITTLE GREATER THAN UNITY. (p. 370)
The exponent $\zeta$ usually ranges between
$1.0$ and $1.5$
\cite{article:KellyJASA2008b},
\cite[112]{book:Duck1990},
\cite{article:WellsUMB1975}.
% c) complex-valued wavenumber with respect to k_{\text{ref}} combines power-law absorption with dispersion
% article:KellyJASA2008b: Analytical time-domain Green's functions for power-law media
% II. FPDE FORMULATIONS OF THE SZABO AND POWER-LAW WAVE EQUATIONS / C. Power-law wave equation
% - In order to derive the analytical time-domain Green’s function solution,
%   A POWER-LAW DISPERSION RELATIONSHIP RELATING WAVENUMBER k AND ANGULAR FREQUENCY \omega is required. (p. 2863)
% - This DISPERSION RELATIONSHIP should yield
%   (1) a POWER-LAW ATTENUATION COEFFICIENT and
%   (2) a FREQUENCY-DEPENDENT PHASE SPEED. (p. 2863)
% - The power-law dispersion relationship that satisfies these requirements is (7) [complex-valued wavenumber] for
%   \omega \geq 0 and k( -\omega ) = \conj{k}( \omega ) to ensure real solutions. (p. 2863)
% - The imaginary part of Eq. (7) [complex-valued wavenumber] yields
%   the POWER-LAW ATTENUATION COEFFICIENT given by Eq. (1) [ \alpha( \omega ) = \alpha_{0} \abs{ \omega }^{y} ]. (p. 2863)
% - The real part produces (8) [reciprocal phase velocity]. (p. 2863)
% - Equation (9) [reciprocal phase velocity, ref. frequency] corresponds to
%   the PHASE VELOCITIES COMPUTED VIA
%   [1.)] THE KRAMERS-KRONIG RELATIONS [14, 27] AND
%   [2.)] THE TIME-CAUSAL THEORY. [13] (p. 2863)
%   [13] article:SzaboJASA1995, [14] article:WatersJASA2000a, [27] article:CobboldJASA2004
% article:WatersITUFFC2005: Causality-imposed (Kramers-Kronig) relationships between attenuation and dispersion
% IV. Applications of the Acoustic Kramers-Kronig Dispersion Relations / A. Power-Law Attenuation
% - The first approach extrapolates the measured ultrasonic properties beyond the experimental bandwidth, and
%   it applies the exact integral or differential forms of the K-K dispersion relations shown in Table I. (p. 826)
% - The corresponding dispersions are shown in Table II. (p. 826)
\TODO{check exponent}
Given
reference values of
% 1.) reference angular frequency
the angular frequency
$\omega_{\text{ref}} \in \Rplus$ and
% 2.) reference phase velocity
the associated phase velocity
$c_{\text{ref}} \in \Rplus$,
% 3.) complex-valued wavenumber with respect to k_{\text{ref}}
the complex-valued wavenumber
% TODO: Kelly?
\cite{article:WatersITUFFC2005,article:SzaboJASA1995}
\begin{subequations}
\label{eqn:lin_mod_mech_model_tis_abs_time_causal_wavenumber_complex_kref}
\begin{equation}
 %--------------------------------------------------------------------------------------------------------------
 % complex-valued wavenumber with respect to k_{\text{ref}}
 %--------------------------------------------------------------------------------------------------------------
  \munderbar{k}_{l}
  =
  \underbrace{
    \frac{ \omega_{l} }{ c_{\text{ref}} }
    +
    \beta_{\text{E,ref}, l}
  }_{ = \beta_{l} = \omega_{l} / c_{l} }
  - j
  \underbrace{
    \bar{b} \abs{ \omega_{l} }^{ \zeta }
  }_{ = \alpha_{l} },
 \label{eqn:lin_mod_mech_model_tis_abs_time_causal_wavenumber_complex_kref_sum}
\end{equation}
where
% 4.) phase term
the phase term
$\beta_{l} \in \R$ sums
% 5.) real-valued wavenumber with respect to c_{\text{ref}}
the real-valued wavenumber
$k_{\text{ref}, l} = \omega_{l} / c_{\text{ref}}$ and
% 6.) excess dispersion term with respect to k_{\text{ref}}
the excess dispersion term
\begin{equation}
 %--------------------------------------------------------------------------------------------------------------
 % excess dispersion term with respect to k_{\text{ref}}
 %--------------------------------------------------------------------------------------------------------------
  \beta_{\text{E,ref}, l}
  =
  \begin{cases}
   %------------------------------------------------------------------------------------------------------------
   % a) exponent of unity
   %------------------------------------------------------------------------------------------------------------
    -
    2 \bar{b} \omega_{l}
    \ln\bigl( \abs{ \omega_{l} / \omega_{\text{ref}} } \bigr) / \pi
    &
    \text{for } \zeta = 1,\\
   %------------------------------------------------------------------------------------------------------------
   % b) even integer or noninteger
   %------------------------------------------------------------------------------------------------------------
    \bar{b}
    \tan\bigl( \zeta \pi / 2 \bigr)
    \omega_{l}
    \bigl( \abs{ \omega_{l} }^{ \zeta - 1 } - \abs{ \omega_{\text{ref}} }^{ \zeta - 1 } \bigr)
    &
    \text{else},
  \end{cases}
 \label{eqn:lin_mod_mech_model_tis_abs_time_causal_wavenumber_complex_kref_excess_dispersion}
\end{equation}
\end{subequations}
combines
% 1.) power-law absorption
power-law absorption with
% 2.) anomalous dispersion [ phase velocity increases w/ frequency ]
% book:Cobbold2006, Sect. 3.10: Effects of Attenuation / Sect. 3.10.1: Kramers-Kronig Relationships
% - It will be noted that FOR n > 2 the SPEED DECREASES WITH INCREASING FREQUENCY and NORMAL DISPERSION is said to be present. (p. 207)
% - FOR n < 2 the OPPOSITE IS TRUE, and the DISPERSION is said to be ANOMALOUS. (p. 207)
% article:SzaboJASA1995: Causal theories and data for acoustic attenuation obeying a frequency power law
% I. TIME DOMAIN CAUSAL RELATIONSHIPS / B. Anomalous dispersion
% - Before proceeding, it is necessary to be MORE PRECISE ABOUT THE DEFINITION OF c_{0} and whether
%   VELOCITY DISPERSION INCREASES OR DECREASES WITH FREQUENCY. (p. 15)
% - According to conventions established in electromagnetic theory,
%   "NORMAL" DISPERSION is that in which PHASE VELOCITY DECREASES WITH AN INCREASE IN FREQUENCY. (p. 15)
% - "ANOMALOUS" DISPERSION is defined as the condition in which PHASE VELOCITY INCREASES WITH FREQUENCY
%   (Stratton, 1941; Jackson, 1975; Gurumurthy and Arthur, 1982). (pp. 15, 16)
% - For THESE CASES [power-law absorption], when phase velocity dispersion has been observed,
%   it was found to INCREASE WITH FREQUENCY in accordance with the ANOMALOUS CATEGORY. (p. 16)
% - The fact that DISPERSION is MOST OFTEN ANOMALOUS DETERMINES THE VALUE OF c_{0} used in the propagation factor \beta_{0} = \omega / c_{0}. (p. 16)
anomalous dispersion.
