%---------------------------------------------------------------------------------------------------------------
% 1.) discretized Born approximation of the recorded RF voltage signals (single pulse-echo measurement, monofrequent, single array element)
%---------------------------------------------------------------------------------------------------------------
% a)
With
the discretized versions of
% 1.) discretized receiver apodization functions
the receiver apodization functions
\eqref{eqn:recovery_disc_space_trans_array_spat_trans_rx} and
% 2.) discretized relative spatial fluctuations in the unperturbed compressibility
the compressibility fluctuations
\eqref{eqn:recovery_disc_space_fov_rel_fluctuations_bp_sampled_kappa},
% 3.) Born approximation of the recorded RF voltage signals
the \name{Born} approximation of
the recorded \ac{RF} voltage signals
\eqref{eqn:lin_mod_v_rx_born} yields
% 4.) discretized Born approximation of the recorded RF voltage signals
the linear algebraic equations
\begin{subequations}
\label{eqn:recovery_sys_lin_eq_v_rx_born}
\begin{equation}
\begin{split}
 %--------------------------------------------------------------------------------------------------------------
 % a) discretized Born approximation of the recorded RF voltage signals (single pulse-echo measurement, monofrequent, single array element)
 %--------------------------------------------------------------------------------------------------------------
  u_{m, l}^{(\text{rx}, n)}
  \approx
  u_{m, l}^{(\text{B}, n)}
  =
  \sum_{ i = 0 }^{ N_{\text{lat}} - 1 }
    \phi_{m, l, i}\bigl[ p_{l}^{(\text{in}, n)} \bigr]
    \gamma_{i}^{(\kappa)}
\end{split}
\label{eqn:recovery_sys_lin_eq_v_rx_born_sum}
\end{equation}
for
% 5.) all sequential pulse-echo measurements, all relevant discrete frequencies, and all array elements
all $( n, l, m ) \in \setconsnonneg{ N_{\text{in}} - 1 } \times \setsymbol{L}_{ \text{BP} }^{(n)} \times \setconsnonneg{ N_{\text{el}} - 1 }$, where
% 6.) entries of the observation process (single pulse-echo measurement, monofrequent, single array element)
the coefficients satisfy
% geometric requirement: $\vect{r}_{\text{mat}, \nu, \rho}^{(m)} \in L_{m}$, $\vect{r}_{\text{lat}, i} \in \Omega$ for delta functions to integrate properly
\begin{equation}
\begin{split}
 %--------------------------------------------------------------------------------------------------------------
 % b) entries of the observation process (single pulse-echo measurement, monofrequent, single array element)
 %--------------------------------------------------------------------------------------------------------------
  \phi_{m, l, i}\bigl[ p_{l}^{(\text{in}, n)} \bigr]
  =
  & 2 {\munderbar{k}_{l}}^{2} \Delta A \Delta V
    h_{m, l}^{(\text{rx})}
    p_{l}^{(\text{in}, n)}( \vect{r}_{\text{lat}, i} )\\
  & \times
    \sum_{ \nu = 0 }^{ N_{\text{mat}} - 1 }
      \chi_{m, \nu, l}^{(\text{rx})}
      g_{l}\bigl[ \vect{r}_{\text{mat}, \nu}^{(m)} - \vect{r}_{\text{lat}, i} \bigr],
\end{split}
\label{eqn:recovery_sys_lin_eq_v_rx_born_coef}
\end{equation}
\end{subequations}
and
% 7.) number of observations (all pulse-echo measurements, multifrequent, all array elements)
the number of
observations equals
\begin{equation}
 %--------------------------------------------------------------------------------------------------------------
 % number of observations (all pulse-echo measurements, multifrequent, all array elements)
 %--------------------------------------------------------------------------------------------------------------
  N_{\text{obs}}
  =
  N_{\text{el}}
  \sum_{ n = 0 }^{ N_{\text{in}} - 1 }
    N_{f, \text{BP}}^{(n)}.
 \label{eqn:recovery_sys_lin_eq_num_obs}
\end{equation}

%---------------------------------------------------------------------------------------------------------------
% 2.) linear algebraic system (all pulse-echo measurements, multifrequent, all array elements)
%---------------------------------------------------------------------------------------------------------------
% a) vertical stacking of all partial linear algebraic systems yields the linear algebraic system
Stacking
the regular samples in
% 1.) discretized relative spatial fluctuations in the unperturbed compressibility
the discretized compressibility fluctuations
\eqref{eqn:recovery_disc_space_fov_rel_fluctuations_bp_sampled_kappa} for
% 2.) all grid points
all grid points into
% 3.) vector of bandpass-filtered relative spatial fluctuations in compressibility
the complex-valued%
\footnote{
  % a) positivity of the relevant discrete frequencies forming the sets results in the recovery of complex-valued compressibility fluctuations that contain only positive spatial frequencies along the r_{d}-axis
  The positivity of
  % 1.) relevant discrete frequencies
  the relevant discrete frequencies forming
  % 2.) sets of relevant discrete frequencies
  the sets
  \eqref{eqn:recon_disc_axis_f_discrete_BP} results in
  the recovery of
  % 3.) vector stacking the regular samples in the discretized relative spatial fluctuations in the unperturbed compressibility
  complex-valued compressibility fluctuations
  \eqref{eqn:recovery_sys_lin_eq_gamma_kappa_bp_vector} that contain only
  % 4.) positive spatial frequencies along the r_{d}-axis
  positive spatial frequencies along
  the $r_{d}$-axis.
}
$N_{\text{lat}} \times 1$ vector
\begin{equation}
 %--------------------------------------------------------------------------------------------------------------
 % vector of bandpass-filtered relative spatial fluctuations in compressibility
 %--------------------------------------------------------------------------------------------------------------
  \vectsym{\gamma}^{(\kappa)}
  =
  \trans{
    \begin{bmatrix}
      \gamma_{ 0 }^{(\kappa)} & \hdots & \gamma_{ N_{\text{lat}} - 1 }^{(\kappa)}
    \end{bmatrix}
  }
 \label{eqn:recovery_sys_lin_eq_gamma_kappa_bp_vector}
\end{equation}
and
% 4.) recorded RF voltage signals and their Born approximation (single pulse-echo measurement, monofrequent, all array elements)
the recorded \ac{RF} voltage signals
\eqref{eqn:recovery_disc_freq_v_rx_Fourier_series_coef} and
% 5.) Born approximation
their \name{Born} approximation
\eqref{eqn:recovery_sys_lin_eq_v_rx_born} into
% 6.) recorded RF voltage signals and their Born approximation (all pulse-echo measurements, multifrequent, all array elements)
the complex-valued
$N_{\text{obs}} \times 1$ vectors
\begin{subequations}
\label{eqn:recovery_sys_lin_eq_v_rx_born_all_f_all_in}
\begin{align}
 %--------------------------------------------------------------------------------------------------------------
 % a) relevant Fourier coefficients of the recorded RF voltage signals (all pulse-echo measurements, multifrequent, all array elements)
 %--------------------------------------------------------------------------------------------------------------
  \vect{u}^{(\text{rx})}
  &=
  \vertcat_{ n = 0 }^{ N_{\text{in}} - 1 }
    \vertcat_{ l \in \setsymbol{L}_{ \text{BP} }^{(n)} }
      \vertcat_{ m = 0 }^{ N_{\text{el}} - 1 }
        u_{ m, l }^{(\text{rx}, n)},
 \label{eqn:recovery_sys_lin_eq_v_rx_born_all_f_all_in_v_rx}\\
 %--------------------------------------------------------------------------------------------------------------
 % b) approximate relevant Fourier coefficients of the recorded RF voltage signals (all pulse-echo measurements, multifrequent, all array elements)
 %--------------------------------------------------------------------------------------------------------------
  \vect{u}^{(\text{B})}
  &=
  \vertcat_{ n = 0 }^{ N_{\text{in}} - 1 }
    \vertcat_{ l \in \setsymbol{L}_{ \text{BP} }^{(n)} }
      \vertcat_{ m = 0 }^{ N_{\text{el}} - 1 }
        u_{ m, l }^{(\text{B}, n)},
 \label{eqn:recovery_sys_lin_eq_v_rx_born_all_f_all_in_v_rx_born}
\end{align}
respectively,
% 7.) observation process (all pulse-echo measurements, multifrequent, all array elements)
the complex-valued
$N_{\text{obs}} \times N_{\text{lat}}$ matrix
\begin{equation}
 %--------------------------------------------------------------------------------------------------------------
 % c) observation process (all pulse-echo measurements, multifrequent, all array elements)
 %--------------------------------------------------------------------------------------------------------------
  \mat{\Phi}\bigl[ p^{(\text{in})} \bigr]
  =
  \vertcat_{ n = 0 }^{ N_{\text{in}} - 1 }
    \vertcat_{ l \in \setsymbol{L}_{ \text{BP} }^{(n)} }
      \vertcat_{ m = 0 }^{ N_{\text{el}} - 1 }
        \horzcat_{ i = 0 }^{ N_{\text{lat}} - 1 }
          \phi_{m, l, i}\bigl[ p_{l}^{(\text{in}, n)} \bigr],
 \label{eqn:recovery_sys_lin_eq_v_rx_born_all_f_all_in_mat}
\end{equation}
represents
the pulse-echo measurement process and defines
% 8.) linear algebraic system (all pulse-echo measurements, multifrequent, all array elements)
the ill-conditioned, and, for
only a few sequential measurements,
% 9.) typically underdetermined dense linear algebraic system
typically underdetermined dense linear algebraic system
\begin{equation}
 %--------------------------------------------------------------------------------------------------------------
 % d) linear algebraic system (all pulse-echo measurements, multifrequent, all array elements)
 %--------------------------------------------------------------------------------------------------------------
  \vect{u}^{(\text{rx})}
  \approx
  \vect{u}^{(\text{B})}
  =
  \mat{\Phi}\bigl[ p^{(\text{in})} \bigr]
  \vectsym{\gamma}^{(\kappa)}.
 \label{eqn:recovery_sys_lin_eq_v_rx_born_all_f_all_in_mat_vec}
\end{equation}
\end{subequations}

%---------------------------------------------------------------------------------------------------------------
% 3.) unwanted properties of the discretized linear ISP and regularization by CS
%---------------------------------------------------------------------------------------------------------------
% a) unwanted properties prevent the direct recovery of the compressibility fluctuations from the recorded RF voltage signals
These unwanted properties, which result from
% 1.) discretization of the Fredholm integral equations of the first kind
% book:Hansen2010, Chapter 1:
%
% book:Hansen1998, Chapter 1: Setting the Stage / Sect. 1.1. Problems with Ill-Conditioned Matrices
% - The numerical treatment of SYSTEMS OF EQUATIONS WITH AN ILL-CONDITIONED COEFFICIENT MATRIX depends on
%   the type of ill-conditioning of A. (p. 2)
% - DISCRETE ILL-POSED PROBLEMS ARISE FROM
%   THE DISCRETIZATION OF ILL-POSED PROBLEMS such as
%   FREDHOLM INTEGRAL EQUATIONS OF THE FIRST KIND. (p. 2)
% - Here all the singular values of A, as well as the SVD components of the solution, on the average,
%   decay gradually to zero, and we say that
%   a DISCRETE PICARD CONDITION (see §4.5) IS SATISFIED. (p. 2)
% - Since there is no gap in the singular value spectrum, there is
%   NO NOTION OF A NUMERICAL RANK for these matrices. (p. 2)
% - For discrete ill-posed problems, the goal is to find a balance between
%   [1.)] the RESIDUAL NORM and
%   [2.)] the SIZE OF THE SOLUTION that matches
%   the errors in the data as well as one's expectations to the computed solution. (pp. 2, 3)
% - Here, "size" should be interpreted in a rather broad sense; e.g., size can be measured by a norm, a seminorm, or a Sobolev norm. (p. 3)
the discretization of
the \name{Fredholm} integral equations of
the first kind
\eqref{eqn:lin_mod_v_rx_born}
\cite[]{book:Hansen2010}
\cite[2, 3]{book:Hansen1998},
% 2.) relatively small number of sequential pulse-echo measurements per image
the relatively small number of
sequential pulse-echo measurements per
image, and
% 3.) slow spatial decays of the outgoing free-space Green's functions
% article:ChewITAP1997: Fast solution methods in electromagnetics
% I. Introduction
% - These numerical techniques involve either solving
%   [1.)] PARTIAL-DIFFERENTIAL EQUATIONS with THE FINITE-DIFFERENCE METHOD (FDM) [6]–[9] or THE FINITE-ELEMENT METHOD (FEM) [10]–[12] WHICH RESULT IN
%     SPARSE MATRICES, or
%   [2.)] INTEGRAL EQUATIONS which are CONVERTED INTO
%     DENSE MATRIX EQUATIONS USING THE METHOD OF MOMENTS (MoM) [1]–[5]. (p. 533)
% III. INTEGRAL EQUATION SOLVERS
% - However, INTEGRAL EQUATION SOLVERS RESULT IN DENSE MATRICES. (p. 535)
% - THE LONG RANGE INTERACTION IN ELECTRODYNAMICS FALLS OFF AS 1/r;
%   THIS DECAY CANNOT BE IGNORED EVEN OVER LARGE DISTANCES.
% article:RokhlinJCP1990: Rapid solution of integral equations of scattering theory in two dimensions
% - ONE OF PRINCIPAL DIFFICULTIES arising in
%   the solution of large-scale scattering problems of integral equations is the fact that
%   THE GREEN'S FUNCTION FOR THE HELMHOLTZ EQUATION DECAYS SLOWLY.
% - drawback: discretization yields DENSE large-scale systems of linear algebraic equations
% - As a result,
%   THE KERNELS OF THE OBTAINED INTEGRAL EQUATIONS ARE NOT SPARSE, AND THEIR DISCRETIZATION LEADS TO
%   DENSE LARGE-SCALE SYSTEMS OF LINEAR ALGEBRAIC EQUATIONS.
the slow spatial decays of
% 4.) outgoing free-space Green's functions (two- and three-dimensional Euclidean spaces)
the outgoing free-space \name{Green}'s functions
\eqref{eqn:app_helmholtz_green_free_space_2_3_dim}
\cite{article:ChewITAP1997,article:RokhlinJCP1990}, prevent
the direct recovery of
% 5.) vector stacking the regular samples in the discretized relative spatial fluctuations in the unperturbed compressibility
the compressibility fluctuations
\eqref{eqn:recovery_sys_lin_eq_gamma_kappa_bp_vector} from
% 6.) vector stacking the relevant Fourier coefficients of the recorded RF voltage signals (all pulse-echo measurements, multifrequent, all array elements)
the recorded \ac{RF} voltage signals
\eqref{eqn:recovery_sys_lin_eq_v_rx_born_all_f_all_in_v_rx}.
% b) reformulation of this discretized linear ISP as an instance of the CS problem circumvents this difficulty
The reformulation of
% 1.) discretized linear ISP
this discretized linear \ac{ISP} as
an instance of
% 2.) CS problem associated with the corrupted observations
the \ac{CS} problem
\eqref{eqn:cs_math_prob_general}, however, circumvents
this difficulty.
