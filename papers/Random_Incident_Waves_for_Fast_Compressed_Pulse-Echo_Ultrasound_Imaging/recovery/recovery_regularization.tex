%---------------------------------------------------------------------------------------------------------------
% 1.) linear algebraic system (all pulse-echo measurements, multifrequent, all array elements, additive errors)
%---------------------------------------------------------------------------------------------------------------
% a) complex-valued vector of transform coefficients constitutes a nearly-sparse representation of the compressibility fluctuations
% article:Schiffner2018, Sect. II. Compressed Sensing in a Nutshell (sec:compressed_sensing)
% - \ac{CS} replaces the identity by an inequality using the known upper bound on the $\ell_{2}$-norm of the additive errors and postulates that
%   a known dictionary of structural building blocks, e.g. an orthonormal basis or a frame, represents
%   the high-dimensional vector almost sparsely \cite{article:TroppPIEEE2010}.
% - Let the column vectors $\vectsym{\psi}_{n} \in \C^{ N }$, $n \in \setcons{ N }$, of
%   the unitary matrix $\mat{\Psi} \in \C^{ N \times N }$, which represents
%   a suitable orthonormal basis of $\C^{ N }$, e.g. the \name{Fourier}, a wavelet, or the canonical basis, define
%   the admissible structural building blocks.
% - The vector of transform coefficients constitutes a nearly-sparse representation of the high-dimensional vector, if
%   the sorted absolute values of its components decay rapidly.
Postulating
the knowledge of
% 1.) suitable orthonormal basis
a suitable orthonormal basis of
$\C^{ N_{\text{lat}} }$, which is represented by
% 2.) complex-valued unitary matrix of structural building blocks
the complex-valued unitary
$N_{\text{lat}} \times N_{\text{lat}}$ matrix
$\mat{\Psi}$ of
structural building blocks
$\vectsym{\psi}_{i}$,
% 3.) complex-valued vector of transform coefficients
the complex-valued
$N_{\text{lat}} \times 1$ vector of
transform coefficients
\begin{equation}
 %--------------------------------------------------------------------------------------------------------------
 % nearly-sparse representation / vector of transform coefficients
 %--------------------------------------------------------------------------------------------------------------
  \vectsym{\theta}^{(\kappa)}
  =
  \herm{ \mat{\Psi} }
  \vectsym{\gamma}^{(\kappa)}
 \label{eqn:recovery_reg_sparse_representation}
\end{equation}
constitutes
% 4.) nearly-sparse representation
a nearly-sparse representation of
% 5.) vector stacking the regular samples in the discretized relative spatial fluctuations in the unperturbed compressibility
the compressibility fluctuations
\eqref{eqn:recovery_sys_lin_eq_gamma_kappa_bp_vector}.
% b) linear algebraic system for the nearly-sparse representation and additive errors
% article:Schiffner2018, Sect. II. Compressed Sensing in a Nutshell (sec:compressed_sensing)
% - Both vectors satisfy the UNDERDETERMINED LINEAR ALGEBRAIC SYSTEM, where
%   the known matrix $\mat{\Phi} \in \C^{ M \times N }$ represents the nonadaptive observation process, and
%   the UNKNOWN VECTOR $\vectsym{\eta} \in \C^{ M }$ denotes ADDITIVE ERRORS OF BOUNDED $\ell_{2}$-NORM $\tnorm{ \vectsym{\eta} }{2} \leq \eta$.
Inserting
% 1.) nearly-sparse representation
this representation, defining
% 2.) sensing matrix (all pulse-echo measurements, multifrequent, all array elements)
the complex-valued
$N_{\text{obs}} \times N_{\text{lat}}$ sensing matrix
\begin{equation}
 %--------------------------------------------------------------------------------------------------------------
 % sensing matrix (all pulse-echo measurements, multifrequent, all array elements)
 %--------------------------------------------------------------------------------------------------------------
  \mat{A}\bigl[ p^{(\text{in})} \bigr]
  =
  \mat{\Phi}\bigl[ p^{(\text{in})} \bigr]
  \mat{\Psi},
 \label{eqn:recovery_reg_sensing_matrix}
\end{equation}
and accounting for
% 3.) additive errors of bounded l2-norm
an unknown complex-valued
$N_{\text{obs}} \times 1$ vector of
additive errors of
bounded $\ell_{2}$-norm
$\tnorm{ \vectsym{\eta} }{2} \leq \eta$,
% 4.) linear algebraic system (all pulse-echo measurements, multifrequent, all array elements)
the linear algebraic system
\eqref{eqn:recovery_sys_lin_eq_v_rx_born_all_f_all_in} becomes
\begin{equation}
 %--------------------------------------------------------------------------------------------------------------
 % linear algebraic system (all pulse-echo measurements, multifrequent, all array elements, additive errors)
 %--------------------------------------------------------------------------------------------------------------
  \vect{u}^{(\text{rx})}
  =
  \underbrace{
    \mat{\Phi}\bigl[ p^{(\text{in})} \bigr]
    \mat{\Psi}
  }_{ = \mat{A}[ p^{(\text{in})} ] }
  \vectsym{\theta}^{(\kappa)}
  +
  \vectsym{\eta}
  =
  \underbrace{
    \mat{A}\bigl[ p^{(\text{in})} \bigr]
    \vectsym{\theta}^{(\kappa)}
  }_{ = \vect{u}^{(\text{B})} }
  +\:
  \vectsym{\eta}.
 \label{eqn:recovery_reg_prob_general_obs_trans_coef_error}
\end{equation}
% c) additive errors reflect the inaccuracies of the discretized physical models and the voltage measurements
The additive errors reflect
the inaccuracies of
% 1.) discretized physical models
the discretized physical models and
% 2.) voltage measurements
the voltage measurements.

%---------------------------------------------------------------------------------------------------------------
% 2.) l2-normalization of the sensing matrix's column vectors
%---------------------------------------------------------------------------------------------------------------
% a) column vectors of the sensing matrix define the pulse echoes of the admissible structural building blocks
The column vectors of
% 1.) sensing matrix (all pulse-echo measurements, multifrequent, all array elements)
the sensing matrix
\eqref{eqn:recovery_reg_sensing_matrix}, i.e.
% 2.) approximate vectors stacking the relevant Fourier coefficients of the recorded RF voltage signals (all pulse-echo measurements, multifrequent, all array elements)
the \name{Born} approximations of
the recorded \ac{RF} voltage signals
\eqref{eqn:recovery_sys_lin_eq_v_rx_born_all_f_all_in_v_rx_born} induced by
the individual components of
% 3.) nearly-sparse representation
the nearly-sparse representation
\eqref{eqn:recovery_reg_sparse_representation}, define
% 4.) pulse echoes
the pulse echoes of
% 5.) admissible structural building blocks
the admissible structural building blocks.
% b) l2-normalization by diagonal weighting matrices minimizes both the restricted isometry ratio and the restricted isometry constant for nearly-sparse representations with s = 1 nonzero component
% article:Schiffner2018, Sect. II. Compressed Sensing in a Nutshell (sec:compressed_sensing)
% - In addition, the INTRODUCTION OF DIAGONAL WEIGHTING MATRICES, whose entries equal
%   the $\ell_{2}$-norms of the sensing matrix's column vectors $\tnorm{ \vect{a}_{n} }{2}$ or their reciprocals, into
%   the UNDERDETERMINED LINEAR ALGEBRAIC SYSTEM \eqref{eqn:cs_math_prob_general_obs_trans_coef_error} always enables
%   the $\ell_{2}$-NORMALIZATION OF THE SENSING MATRIX'S COLUMN VECTORS without violating the mathematical equivalence.
% - The resulting NORMALIZED SENSING MATRIX MINIMIZES both
%   [1.)] the RESTRICTED ISOMETRY RATIO and
%   [2.)] the RESTRICTED ISOMETRY CONSTANT for
%   $1$-SPARSE REPRESENTATIONS \eqref{eqn:def_transform_coefficients} and better conforms with
%   the associated SUFFICIENT CONDITIONS \cite[Prop. 6.2]{book:Foucart2013}.
% article:Schiffner2018, Sect. II. Compressed Sensing in a Nutshell (sec:compressed_sensing)
% - Multiple sufficient conditions on the sensing matrix \eqref{eqn:cs_math_prob_general_sensing_matrix} ensure
%   the stable recovery of the nearly-sparse representation \eqref{eqn:def_transform_coefficients} in
%   the \ac{CS} problem \eqref{eqn:cs_math_prob_general} by
%   the sparsity-promoting $\ell_{q}$-minimization method \eqref{eqn:cs_lq_minimization}.
% - These conditions impose specific upper bounds on various characteristic measures quantifying the suitability of the sensing matrix \eqref{eqn:cs_math_prob_general_sensing_matrix}, e.g.
%   [1.)] the null space constants \cite[Def. 4.21]{book:Foucart2013}, \cite[Def. 1.2]{book:Eldar2012},
%   [2.)] the restricted isometry ratio \cite{article:FoucartACHA2009}, and
%   [3.)] the restricted isometry constant \cite{article:FoucartACHA2010,article:CandesCRAS2008,article:CandesSPM2008}.
Although
their $\ell_{2}$-normalization by
diagonal weighting matrices minimizes both
% 1.) restricted isometry ratio
the restricted isometry ratio and
% 2.) restricted isometry constant
the restricted isometry constant for
% 3.) 1-sparse representations
$1$-sparse representations
\eqref{eqn:recovery_reg_sparse_representation}
(cf. \cref{sec:compressed_sensing}),
% c) l2-normalization by diagonal weighting matrices potentially amplifies the additive errors
it potentially amplifies
the additive errors.
% d) recorded electric energies characterize the transfer behavior of the sensing matrix
In fact,
% 1.) density of population
the density of
population and
% 2.) dynamic range of the recorded electric energies
the dynamic range of
the recorded electric energies
\begin{equation}
 %--------------------------------------------------------------------------------------------------------------
 % recorded electric energies in the pulse echoes (all pulse-echo measurements, multifrequent, all array elements)
 %--------------------------------------------------------------------------------------------------------------
  E_{i}^{(\text{B})}
  =
  \dnorm{ \vect{a}_{i}\bigl[ p^{(\text{in})} \bigr] }{2}{1}^{2}
 \label{eqn:recovery_reg_v_rx_born_trans_coef_energy}
\end{equation}
for
% 1.) all coefficients
all $i \in \setcons{ N_{\text{lat}} }$, which characterize
% 2.) transfer behavior
the transfer behavior, vary significantly with
the orthonormal basis.
% e) imposition of a hard threshold on the l2-norms mitigates this amplification (pseudo-inverse filter)
The imposition of
% 1.) hard threshold on the l2-norms
a hard threshold on
the $\ell_{2}$-norms, whose value is dictated by
% 2.) signal-to-noise ratio (SNR)
the \ac{SNR} of
% 3.) vector stacking the relevant Fourier coefficients of the recorded RF voltage signals (all pulse-echo measurements, multifrequent, all array elements)
the recorded \ac{RF} voltage signals
\eqref{eqn:recovery_sys_lin_eq_v_rx_born_all_f_all_in_v_rx}, mitigates
this amplification.

%---------------------------------------------------------------------------------------------------------------
% 3.) l2-normalization of the sensing matrix by diagonal weighting matrices
%---------------------------------------------------------------------------------------------------------------
% a) factor \xi \in ( 0; 1 ] specifies the lower bound on the l2-norms of the sensing matrix's column vectors
Let
% 1.) factor \xi \in ( 0; 1 ]
the factor
$\xi \in ( 0; 1 ]$ specify
% 2.) lower bound on the l2-norms of the sensing matrix's column vectors
the lower bound on
the $\ell_{2}$-norms of
the sensing matrix's column vectors
\begin{equation}
 %--------------------------------------------------------------------------------------------------------------
 % lower bound on the l2-norms of the sensing matrix's column vectors
 %--------------------------------------------------------------------------------------------------------------
  a_{\xi, \text{lb}}\bigl[ p^{(\text{in})} \bigr]
  =
  \xi
  \underset{ i \in \setcons{ N_{\text{lat}} } }{ \max }
  \left\{
    \dnorm{ \vect{a}_{i}\bigl[ p^{(\text{in})} \bigr] }{2}{1}
  \right\}.
 \label{eqn:recovery_reg_norm_l2_norms_lb}
\end{equation}
% b) thresholded l2-norms constitute the individual entries of the weighting matrix and its inverse matrix
The thresholded $\ell_{2}$-norms of
these column vectors
\begin{equation}
 %--------------------------------------------------------------------------------------------------------------
 % thresholded l2-norms of the sensing matrix's column vectors
 %--------------------------------------------------------------------------------------------------------------
  a_{ \xi, i }\bigl[ p^{(\text{in})} \bigr]
  =
  \max
  \left\{
    \dnorm{ \vect{a}_{i}\bigl[ p^{(\text{in})} \bigr] }{2}{1},
    a_{\xi, \text{lb}}\bigl[ p^{(\text{in})} \bigr]
  \right\}
 \label{eqn:recovery_reg_norm_l2_norms_thresholded}
\end{equation}
populate
% 1.) diagonal weighting matrix
the real-valued
$N_{\text{lat}} \times N_{\text{lat}}$ weighting matrix
\begin{subequations}
\label{eqn:recovery_reg_norm_weighting_matrices}
\begin{align}
 %--------------------------------------------------------------------------------------------------------------
 % a) diagonal weighting matrix
 %--------------------------------------------------------------------------------------------------------------
  \mat{W}_{\xi}
  &=
  \ddiag{
    \begin{matrix}
      a_{ \xi, 1 }\bigl[ p^{(\text{in})} \bigr] & \hdots & a_{ \xi, N_{\text{lat}} }\bigl[ p^{(\text{in})} \bigr]
    \end{matrix}
  }{2}
 \label{eqn:recovery_reg_norm_weighting_matrix}\\
\intertext{%
and
% 2.) diagonal inverse weighting matrix
its inverse matrix%
}
 %--------------------------------------------------------------------------------------------------------------
 % b) diagonal inverse weighting matrix
 %--------------------------------------------------------------------------------------------------------------
  \mat{W}_{\xi}^{-1}
  &=
  \diag{
    \begin{matrix}
      \frac{ 1 }{ a_{ \xi, 1 }\left[ p^{(\text{in})} \right] } & \hdots & \frac{ 1 }{ a_{ \xi, N_{\text{lat}} }\left[ p^{(\text{in})} \right] }
    \end{matrix}
  },
 \label{eqn:recovery_reg_norm_weighting_matrix_inv}
\end{align}
\end{subequations}
whose
dependences on
% 3.) discretized incident acoustic pressure fields [superpositions of quasi-(d-1)-spherical waves]
the incident acoustic pressure fields
\eqref{eqn:recovery_p_in} are omitted for
the sake of
notational lucidity.
% c) right multiplication of the sensing matrix by the diagonal inverse weighting matrix yields the normalized sensing matrix
The right multiplication of
% 1.) sensing matrix (all pulse-echo measurements, multifrequent, all array elements)
the sensing matrix
\eqref{eqn:recovery_reg_sensing_matrix} by
% 2.) diagonal inverse weighting matrix
the diagonal inverse weighting matrix
\eqref{eqn:recovery_reg_norm_weighting_matrix_inv} yields
% 3.) normalized sensing matrix (all pulse-echo measurements, multifrequent, all array elements)
the complex-valued normalized
$N_{\text{obs}} \times N_{\text{lat}}$ sensing matrix
\begin{equation}
 %--------------------------------------------------------------------------------------------------------------
 % normalized sensing matrix (all pulse-echo measurements, multifrequent, all array elements)
 %--------------------------------------------------------------------------------------------------------------
  \bar{\mat{A}}_{\xi}\bigl[ p^{(\text{in})} \bigr]
  =
  \mat{A}\bigl[ p^{(\text{in})} \bigr]
  \mat{W}_{\xi}^{-1}
  =
  \mat{\Phi}\bigl[ p^{(\text{in})} \bigr]
  \mat{\Psi}
  \mat{W}_{\xi}^{-1},
 \label{eqn:recon_reg_norm_sensing_matrix}
\end{equation}
whose
% 4.) column vectors
column vectors exhibit
unity $\ell_{2}$-norms, if
the $\ell_{2}$-norm of
the corresponding column vector in
% 5.) original sensing matrix (all pulse-echo measurements, multifrequent, all array elements)
the original sensing matrix
\eqref{eqn:recovery_reg_sensing_matrix} is
not smaller than
% 6.) lower bound on the l2-norms of the sensing matrix's column vectors
the specified lower bound
\eqref{eqn:recovery_reg_norm_l2_norms_lb}.
% d) normalized sensing matrix maintains the potential dense population of the sensing matrix
This matrix maintains
% 1.) potential dense population
the potential dense population of
% 2.) sensing matrix (all pulse-echo measurements, multifrequent, all array elements)
the sensing matrix
\eqref{eqn:recovery_reg_sensing_matrix}.

%---------------------------------------------------------------------------------------------------------------
% 4.) normalized CS problem and sparsity-promoting methods for its solution
%---------------------------------------------------------------------------------------------------------------
% a) insertions of the weighting matrices and the normalized sensing matrix into the linear algebraic system yield the equivalent normalized linear algebraic system
With
the normalized versions of
% 1.) unit vector stacking the relevant Fourier coefficients of the recorded RF voltage signals (all pulse-echo measurements, multifrequent, all array elements)
the recorded \ac{RF} voltage signals
$\bar{\vect{u}}^{(\text{rx})} = \vect{u}^{(\text{rx})} / \tnorm{ \vect{u}^{(\text{rx})} }{2}$,
% 2.) normalized additive errors of bounded l2-norm
the additive errors
$\bar{\vectsym{\eta}} = \vectsym{\eta} / \tnorm{ \vect{u}^{(\text{rx})} }{2}$ of
bounded $\ell_{2}$-norm
$\tnorm{ \bar{\vectsym{\eta}} }{2} \leq \bar{\eta} = \eta / \tnorm{ \vect{u}^{(\text{rx})} }{2}$, and
% 3.) normalized nearly-sparse representation
the nearly-sparse representation
\begin{equation}
 %--------------------------------------------------------------------------------------------------------------
 % normalized nearly-sparse representation / nearly-sparse normalized vector of transform coefficients
 %--------------------------------------------------------------------------------------------------------------
  \bar{\vectsym{\theta}}_{\xi}^{(\kappa)}
  =
  \frac{ 1 }{ \dnorm{ \vect{u}^{(\text{rx})} }{2}{1} }
  \mat{W}_{\xi}
  \vectsym{\theta}^{(\kappa)},
 \label{eqn:recon_reg_norm_trans_coef}
\end{equation}
the insertions of
% 4.) weighting matrices
the weighting matrices
\eqref{eqn:recovery_reg_norm_weighting_matrices} and
% 5.) normalized sensing matrix
the normalized sensing matrix
\eqref{eqn:recon_reg_norm_sensing_matrix} into
% 6.) linear algebraic system (all pulse-echo measurements, multifrequent, all array elements, additive errors)
the linear algebraic system
\eqref{eqn:recovery_reg_prob_general_obs_trans_coef_error} yield
% 7.) normalized linear algebraic system (all pulse-echo measurements, multifrequent, all array elements, additive errors)
the equivalent system
\begin{equation}
\begin{split}
 %--------------------------------------------------------------------------------------------------------------
 % normalized linear algebraic system (all pulse-echo measurements, multifrequent, all array elements, additive errors)
 %--------------------------------------------------------------------------------------------------------------
  \bar{\vect{u}}^{(\text{rx})}
  &=
  \frac{ 1 }{ \dnorm{ \vect{u}^{(\text{rx})} }{2}{1} }
  \Bigl[
    \underbrace{
      \mat{A}\bigl[ p^{(\text{in})} \bigr]
      \mat{W}_{\xi}^{-1}
    }_{ = \bar{\mat{A}}_{\xi}[ p^{(\text{in})} ] }
    \mat{W}_{\xi}
    \vectsym{\theta}^{(\kappa)}
    +
    \vectsym{\eta}
  \Bigr]\\
  &=
  \bar{\mat{A}}_{\xi}\bigl[ p^{(\text{in})} \bigr]
  \bar{\vectsym{\theta}}_{\xi}^{(\kappa)}
  +
  \bar{\vectsym{\eta}}.
\end{split}
\label{eqn:recovery_reg_norm_obs_trans_coef_error}
\end{equation}
% b) CS problem associated with the normalized linear algebraic system
The associated normalized \ac{CS} problem
\eqref{eqn:cs_math_prob_general} reads
\begin{equation}
\begin{alignedat}{2}
 %--------------------------------------------------------------------------------------------------------------
 % CS problem associated with the normalized linear algebraic system
 %--------------------------------------------------------------------------------------------------------------
  &
  \text{Recover}
  &
  \text{nearly-sparse }
  \bar{\vectsym{\theta}}_{\xi}^{(\kappa)}
  \in
  \C^{ N_{\text{lat}} }\\
  &
  \text{subject to}
  \quad
  &
  \dnorm{ \bar{\vect{u}}^{(\text{rx})} - \bar{\mat{A}}_{\xi}\bigl[ p^{(\text{in})} \bigr] \bar{\vectsym{\theta}}_{\xi}^{(\kappa)} }{2}{1}
  &\leq
  \bar{\eta}
\end{alignedat}
\label{eqn:recovery_reg_norm_prob_general}
\end{equation}
and
% c) summary of sparsity-promoting methods for the solution of the CS problem
the sparsity-promoting $\ell_{q}$-minimization method for
its stable solution
\eqref{eqn:cs_lq_minimization},
$q \in [ 0; 1 ]$, recovers
% 1.) complex-valued normalized vector of recovered transform coefficients
the complex-valued normalized
$N_{\text{lat}} \times 1$ vector of
transform coefficients
\begin{equation}
\begin{alignedat}{2}
 %--------------------------------------------------------------------------------------------------------------
 % sparsity-promoting lq-minimization method
 %--------------------------------------------------------------------------------------------------------------
  \hspace{-0.75em}
  \hat{\bar{\vectsym{\theta}}}_{\xi}^{(\kappa, q, \eta)}
  &\in
  \underset{ \tilde{\vectsym{\theta}} \in \C^{ N_{\text{lat}} } }{ \arg\min }
  \dnorm{ \tilde{\vectsym{\theta}} }{q}{1}\\
  &
  \mspace{24.5mu}
  \text{subject to}
  &
  \dnorm{ \bar{\vect{u}}^{(\text{rx})} - \bar{\mat{A}}_{\xi}\bigl[ p^{(\text{in})} \bigr] \tilde{\vectsym{\theta}} }{2}{1}
  &\leq
  \bar{\eta}.
\end{alignedat}
\tag{$\recmethodnorm{ q }{ \xi }{ \eta }$}
\label{eqn:recovery_reg_norm_lq_minimization}
\end{equation}
% d) inversions of both the normalization and the basis transform estimate the compressibility fluctuations
The inversions of both
% 1.) inversion of the normalization
the normalization in
\eqref{eqn:recon_reg_norm_trans_coef} and
% 2.) inversion of the basis transform
the basis transform in
\eqref{eqn:recovery_reg_sparse_representation} estimate
% 3.) vector stacking the regular samples in the discretized relative spatial fluctuations in the unperturbed compressibility
the compressibility fluctuations
\eqref{eqn:recovery_sys_lin_eq_gamma_kappa_bp_vector} as
\begin{equation}
 %--------------------------------------------------------------------------------------------------------------
 % estimated vector stacking the regular samples in the discretized relative spatial fluctuations in the unperturbed compressibility
 %--------------------------------------------------------------------------------------------------------------
  \hat{\vectsym{\gamma}}_{\xi}^{(\kappa, q, \eta)}
  =
  \dnorm{ \vect{u}^{(\text{rx})} }{2}{1}
  \mat{\Psi}
  \mat{W}_{\xi}^{-1}
  \hat{\bar{\vectsym{\theta}}}_{\xi}^{(\kappa, q, \eta)}.
 \label{eqn:recovery_reg_norm_lq_minimization_sol_mat_params}
\end{equation}
% e) doubled real parts estimate the physically meaningful real-valued regular samples in the discretized compressibility fluctuations for all grid points
\TODO{shorten! estimate two times...}
Their doubled real parts estimate
% 1.) physically meaningful real-valued regular samples
the physically meaningful real-valued regular samples in
% 2.) discretized relative spatial fluctuations in the unperturbed compressibility
the discretized compressibility fluctuations
\eqref{eqn:recovery_disc_space_fov_rel_fluctuations_bp_sampled_kappa} for
% 3.) all grid points
all grid points.
