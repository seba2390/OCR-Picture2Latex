%---------------------------------------------------------------------------------------------------------------
% 1.) additive errors
%---------------------------------------------------------------------------------------------------------------
% a) additive errors corrupted the recorded samples of all RF voltage signals
% article:Schiffner2018, Sect. VI. Implementation / Sect. VI-A Determination of the Relevant Fourier Coefficients (subsec:imp_fourier_coefficients)
% - It [concurrent ADC] RECORDED
%   $N_{t}^{(n)} = q_{\text{ub}}^{(n)} - q_{\text{lb}}^{(n)}$ REAL-VALUED SAMPLES PER SIGNAL and, thus,
%   $N_{\text{el}} N_{t}^{(n)}$ SAMPLES PER PULSE-ECHO MEASUREMENT for
%   all $n \in \setconsnonneg{ N_{\text{in}} - 1 }$.
Additive errors, which
were statistically modeled as
% 1.) Gaussian white noise
% book:Manolakis2005, Chapter 3: Random Variables, Vectors, and Sequences / Sect. 3.3: DISCRETE-TIME STOCHASTIC PROCESSES / Sect. 3.3.6: Frequency-Domain Description of Stationary Processes
% White noise.
% - A RANDOM SEQUENCE w( n ) is called
%   a (SECOND-ORDER) WHITE NOISE PROCESS with mean µ_{w} and variance σ_{w}^{2}, denoted by
%   [ w( n ) \sim WN( µ_{w}, σ_{w}^{2} ) ] (3.3.46)
%   if and only if
%   [1.)] E{ w( n ) } = µ_{w} and
%   [2.)] [ r_{w}( l ) = E{ w( n ) \conj{w}( n - l ) } = σ_{w}^{2} δ( l ) ] (3.3.47)
%   which implies that
%   [ R_{w}( e^{jω} ) = σ_{w}^{2}, -\pi \leq ω \leq \pi. ] (3.3.48) (p. 110)
% => wide-sense stationary, uncorrelated samples
% - The term white noise is used to emphasize that ALL FREQUENCIES CONTRIBUTE THE SAME AMOUNT OF POWER, as
%   in the case of white light, which is obtained by mixing all possible colors by the same amount. (p. 110)
% - If, in addition, the PDF OF x( n ) IS GAUSSIAN, then the process is called
%   a (SECOND-ORDER) WHITE GAUSSIAN NOISE PROCESS, and it will be denoted by
%   WGN( µ_{w}, σ_{w}^{2} ). (p. 110)
% => wide-sense stationary, uncorrelated samples, Gaussian distribution
% - If the random variables w( n ) are
%   INDEPENDENTLY AND IDENTICALLY DISTRIBUTED with mean µ_{w} and variance σ_{w}^{2}, then we shall write
%   [ w( n ) \sim IID( µ_{w}, σ_{w}^{2} ). ] (3.3.49) (p. 110)
% - This is sometimes referred to as a STRICT WHITE NOISE. (p. 110)
% - We emphasize that the conditions of uncorrelatedness or independence do not put any restriction on
%   the form of the probability density function of w( n ). (p. 110)
% - Thus we can have an IID process with any type of probability distribution. (p. 110)
% - Clearly, white noise is the simplest random process because it does not have any structure. (p. 110)
\ac{GWN}
(cf. e.g. \cite[110]{book:Manolakis2005}) with
% 2.) zero mean
zero mean and
% 3.) variance \sigma_{\eta}^{2}
the variance $\sigma_{\eta}^{2}$, corrupted
% 4.) recorded samples of all RF voltage signals
the recorded samples of
all \ac{RF} voltage signals.
% b) expected energy of the recorded RF voltage signals
% article:Schiffner2018, Sect. V. Image Recovery Based on Compressed Sensing / Sect. V-D Regularization of the Inverse Scattering Problem (subsec:recovery_regularization)
% - Inserting this representation [nearly-sparse],
%   defining the complex-valued $N_{\text{obs}} \times N_{\text{lat}}$ sensing matrix
%   [ \mat{A}\bigl[ p^{(\text{in})} \bigr] = \mat{\Phi}\bigl[ p^{(\text{in})} \bigr] \mat{\Psi}, ] (eqn:recovery_reg_sensing_matrix) and accounting for
%   an UNKNOWN COMPLEX-VALUED $N_{\text{obs}} \times 1$ VECTOR OF ADDITIVE ERRORS OF BOUNDED $\ell_{2}$-NORM
%   $\tnorm{ \vectsym{\eta} }{2} \leq \eta$,
%   the linear algebraic system \eqref{eqn:recovery_sys_lin_eq_v_rx_born_all_f_all_in} becomes
%   [ \vect{u}^{(\text{rx})} = \underbrace{ \mat{\Phi}\bigl[ p^{(\text{in})} \bigr] \mat{\Psi} }_{ = \mat{A}[ p^{(\text{in})} ] } \vectsym{\theta}^{(\kappa)} + \vectsym{\eta} = \underbrace{ \mat{A}\bigl[ p^{(\text{in})} \bigr] \vectsym{\theta}^{(\kappa)} }_{ = \vect{u}^{(\text{B})} } +\: \vectsym{\eta}. ]
%   (eqn:recovery_reg_prob_general_obs_trans_coef_error)
% - The ADDITIVE ERRORS reflect the inaccuracies of the discretized physical models and the voltage measurements.
The expected energy of
% 1.) vector stacking the relevant Fourier coefficients of the recorded RF voltage signals (all pulse-echo measurements, multifrequent, all array elements)
the recorded \ac{RF} voltage signals
\eqref{eqn:recovery_sys_lin_eq_v_rx_born_all_f_all_in_v_rx} in
% 2.) linear algebraic system (all pulse-echo measurements, multifrequent, all array elements, additive errors)
the linear algebraic system
\eqref{eqn:recovery_reg_prob_general_obs_trans_coef_error} was
\begin{equation*}
 %--------------------------------------------------------------------------------------------------------------
 % expected energy of the vector stacking the relevant Fourier coefficients of the recorded RF voltage signals
 %--------------------------------------------------------------------------------------------------------------
  \expect{
    \dnorm{ \vect{u}^{(\text{rx})} }{2}{1}^{2}
  }
  =
  \dnorm{ \vect{u}^{(\text{B})} }{2}{1}^{2}
  +
  \sigma_{\eta}^{2}
  N_{\text{el}}
  \sum_{ n = 0 }^{ N_{\text{in}} - 1 }
    \frac{
      N_{f, \text{BP}}^{(n)}
    }{
      N_{t}^{(n)}
    }.
\end{equation*}
% c) expected energy permitted the l2-norm of the normalized additive errors in the normalized CS problem and the sparsity-promoting lq-minimization method to be estimated
% article:Schiffner2018, Sect. V. Image Recovery Based on Compressed Sensing / Sect. V-D Regularization of the Inverse Scattering Problem (subsec:recovery_regularization)
% - With the NORMALIZED VERSIONS of
%   the recorded \ac{RF} voltage signals $\bar{\vect{u}}^{(\text{rx})} = \vect{u}^{(\text{rx})} / \tnorm{ \vect{u}^{(\text{rx})} }{2}$,
%   the ADDITIVE ERRORS $\bar{\vectsym{\eta}} = \vectsym{\eta} / \tnorm{ \vect{u}^{(\text{rx})} }{2}$ of bounded $\ell_{2}$-norm
%   $\tnorm{ \bar{\vectsym{\eta}} }{2} \leq \bar{\eta} = \eta / \tnorm{ \vect{u}^{(\text{rx})} }{2}$, and
%   the nearly-sparse representation [...], the insertions of
%   the weighting matrices \eqref{eqn:recovery_reg_norm_weighting_matrices} and
%   the normalized sensing matrix \eqref{eqn:recon_reg_norm_sensing_matrix} into
%   the linear algebraic system \eqref{eqn:recovery_reg_prob_general_obs_trans_coef_error} yield
%   the equivalent system [...].
% => \tnorm{ \bar{\vectsym{\eta}} }{2} = \tnorm{ \vectsym{\eta} }{2} / \tnorm{ \vect{u}^{(\text{rx})} }{2} \leq \bar{\eta}
% \bar{\eta} = \tnorm{ \vectsym{\eta} }{2} / \tnorm{ \vect{u}^{(\text{rx})} }{2}
It permitted
% 1.) l2-norm of the normalized additive errors
the $\ell_{2}$-norm of
the normalized additive errors in
% 2.) normalized CS problem
the normalized \ac{CS} problem
\eqref{eqn:recovery_reg_norm_prob_general} and
% 3.) sparsity-promoting lq-minimization method
the sparsity-promoting $\ell_{q}$-minimization method
\eqref{eqn:recovery_reg_norm_lq_minimization} to be estimated as
\begin{equation}
 %--------------------------------------------------------------------------------------------------------------
 % estimated l2-norm of the normalized additive errors in the normalized CS problem
 %--------------------------------------------------------------------------------------------------------------
  \hat{ \bar{\eta} }
  =
  \left[
    1
    +
    \frac{
      \norm{ \vect{u}^{(\text{B})} }{2}^{2}
    }{
      \sigma_{\eta}^{2}
      N_{\text{el}}
      \sum_{ n = 0 }^{ N_{\text{in}} - 1 }
        N_{f, \text{BP}}^{(n)} / N_{t}^{(n)}
    }
  \right]^{ - \frac{1}{2} }.
 \label{eqn:imp_data_acq_rel_obs_error_est}
\end{equation}
