%---------------------------------------------------------------------------------------------------------------
% 1.) determination of the relevant Fourier coefficients
%---------------------------------------------------------------------------------------------------------------
% a) concurrent ADC of the RF voltage signals generated by all array elements at the rates f_{\text{s}}^{(n)} quantized the endpoints of each recording time interval
% article:Schiffner2018, Sect. III. Linear Physical Model for the Pulse-Echo Measurement Process / Sect. III.A. Pulse-Echo Measurement Process
% - The \ac{UI} system sequentially performs $N_{\text{in}} \in \N$ independent pulse-echo measurements using a planar transducer array
%   (cf. \cref{fig:lin_mod_scan_configuration,tab:lin_mod_scan_config_instrum_params}).
% - Each measurement begins at the time instant $t = 0$ and triggers
%   the CONCURRENT RECORDING OF
%   THE \ac{RF} VOLTAGE SIGNALS $\tilde{u}_{m}^{(\text{rx}, n)}: \setsymbol{T}_{ \text{rec} }^{(n)} \mapsto \R$ GENERATED BY
%   ALL ARRAY ELEMENTS $m \in \setconsnonneg{ N_{\text{el}} - 1 }$ in the specified time interval
%   [ \setsymbol{T}_{ \text{rec} }^{(n)} = \bigl[ t_{\text{lb}}^{(n)}; t_{\text{ub}}^{(n)} \bigr], ] (eqn:lin_mod_scan_config_volt_rx_obs_interval) where
%   $t_{\text{lb}}^{(n)} \in \Rnonneg$ and $t_{\text{ub}}^{(n)} > t_{\text{lb}}^{(n)}$ denote
%   its lower and upper bounds, respectively.
The concurrent \ac{ADC} of
% 1.) RF voltage signals generated by all array elements
the \ac{RF} voltage signals generated by
all array elements at
% 2.) rates f_{\text{s}}^{(n)}
the rates
$f_{\text{s}}^{(n)} = 1 / T_{\text{s}}^{(n)} > 2 f_{\text{ub}}^{(n)}$ quantized
% 3.) endpoints
the endpoints of
% 4.) specified recording time intervals for the RF voltage signals generated by all array elements
each recording time interval
\eqref{eqn:lin_mod_scan_config_volt_rx_obs_interval} as
% 5.) quantized lower bounds in the specified recording time intervals (t_{\text{lb}}^{(n)} \in \Rnonneg)
$t_{\text{lb}}^{(n)} = q_{\text{lb}}^{(n)} T_{\text{s}}^{(n)}$ and
% 6.) quantized upper bounds in the specified recording time intervals (t_{\text{ub}}^{(n)} > t_{\text{lb}}^{(n)})
$t_{\text{ub}}^{(n)} = q_{\text{ub}}^{(n)} T_{\text{s}}^{(n)}$, where
% 7.) nonnegative quantized bounds
$q_{\text{lb}}^{(n)}, q_{\text{ub}}^{(n)} \in \Nnonneg$ and
% 8.) quantized upper bounds are larger than quantized lower bounds
$q_{\text{ub}}^{(n)} > q_{\text{lb}}^{(n)}$.
% b) concurrent ADC recorded N_{t}^{(n)} real-valued samples per signal and N_{\text{el}} N_{t}^{(n)} samples per pulse-echo measurement
% book:Briggs1995, Chapter 2: The Discrete Fourier Transform / Sect. 2.4.: DFT Approximations to Fourier Series Coefficients [DERIVATION OF THE DFT]
% - As closely as the DFT is related to the Fourier transform, it may be argued that
%   IT HOLDS EVEN MORE KINSHIP TO THE COEFFICIENTS OF THE FOURIER SERIES. (p. 33)
% - It is a simple matter to use the Fourier series to derive the DFT formula, and we will do so shortly. (p. 33)
% - With this prelude to Fourier series, we are now in a position to derive
%   the DFT AS AN APPROXIMATION to the integral that gives
%   the FOURIER SERIES COEFFICIENTS c_{k}. (p. 38)
% - Now we consider APPROXIMATIONS TO THE INTEGRAL
%   [ c_{k} = \frac{ 1 }{ A } \int_{ - A / 2 }^{ A / 2 } f( x ) e^{ -j 2 \pi k x / A } dx. ] (2.17) (p. 38)
% - As before,
%   the INTERVAL OF INTEGRATION is divided into N SUBINTERVALS OF EQUAL LENGTH, and
%   let the GRID SPACING BE \Delta x = A / N. (p. 39)
% - A grid with N + 1 equally spaced points over the interval [ -A/2, A/2 ] is defined by
%   the points x_{n} = n \Delta x for n = -N/2:N/2. (p. 39)
% - Furthermore, we let [ g( x ) = f( x ) e^{ -j 2 \pi k x / A } ] be the integrand in this expression. (p. 39)
% - Applying the TRAPEZOID RULE gives the approximations
%   [ c_{k} = \frac{ 1 }{ A } \int_{ - A / 2 }^{ A / 2 } g( x ) dx \approx \frac{ \Delta x }{ 2 A } [ g( -A/2 ) + 2 \sum_{ n = - N/2 + 1 }^{ N/2 - 1 } g( x_{n} ) + g( A/2 ) ]. ] (p. 39)
It recorded
% 1.) number of recorded real-valued samples per RF voltage signal
$N_{t}^{(n)} = q_{\text{ub}}^{(n)} - q_{\text{lb}}^{(n)}$ real-valued samples per
signal%
\footnote{%
 % a) assuming identical samples at the endpoints of the recording time intervals
 % book:Briggs1995, Chapter 2: The Discrete Fourier Transform / Sect. 2.4.: DFT Approximations to Fourier Series Coefficients [DERIVATION OF THE DFT]
 % - NOW THE QUESTION OF ENDPOINT VALUES ENTERS IN A CRITICAL WAY. (p. 39)
 % - We have already seen that if
 %   the periodic extension of f is DISCONTINUOUS AT THE ENDPOINTS x = \pm A/2, then, when its Fourier series converges,
 %   it converges to the AVERAGE VALUE
 %   [ \frac{ f( -A/2+ ) + f( A/2- ) }{ 2 }. ] (p. 39)
 % - Therefore, it is the AVERAGE VALUE OF f AT THE ENDPOINTS THAT MUST BE USED IN THE TRAPEZOID RULE. (p. 39)
 % - Noting that the kernel e^{ -j 2 \pi k x / A } has the value (-1)^{k} at x = \pm A/2, we see that
 %   function g that must be used for the trapezoid rule is
 %   [ g( x ) = f( x ) e^{ -j 2 \pi k x / A } for x \neq \pm A/2,
 %            = \frac{ (-1)^{k} }{ 2 } [ f( -A/2+ ) + f( A/2- ) ] for x = \pm A/2. ] (pp. 39, 40)
 % - It should be verified that this choice of g, dictated by the convergence properties of the Fourier series, guarantees that
 %   g( -A/2 ) = g( A/2 ). (p. 40)
 Assuming identical samples at
 % 1.) endpoints
 the endpoints of
 % 2.) specified recording time intervals for the RF voltage signals generated by all array elements
 the recording time intervals
 \eqref{eqn:lin_mod_scan_config_volt_rx_obs_interval}, i.e.
 $\tilde{u}_{m}^{(\text{rx}, n)}[ t_{\text{lb}}^{(n)} ] = \tilde{u}_{m}^{(\text{rx}, n)}[ t_{\text{ub}}^{(n)} ]$ for
 % 3.) all sequential pulse-echo measurements and all array elements
 all $( n, m ) \in \setconsnonneg{ N_{\text{in}} - 1 } \times \setconsnonneg{ N_{\text{el}} - 1 }$.
 % b) averages of the left and right limits have to be used at any point of discontinuity
 % book:Briggs1995, Chapter 2: The Discrete Fourier Transform / Sect. 2.4.: DFT Approximations to Fourier Series Coefficients [DERIVATION OF THE DFT]
 % - In a similar way,
 %   an AVERAGE VALUE must be used at ANY GRID POINTS AT WHICH f HAS DISCONTINUITIES. (p. 40)
 % - There are SUBTLETIES CONCERNING THE USE OF AVERAGE VALUES AT THE ENDPOINTS AND DISCONTINUITES, but
 %   the importance of this issue will be emphasized many times in hopes of removing the subtlety! (p. 40)
 In general,
 the averages of
 the left and right limits replace
 any discontinuities
 \cite[40]{book:Briggs1995}.
} and, thus,
% 2.) number of recorded real-valued samples per pulse-echo measurement
$N_{\text{el}} N_{t}^{(n)}$ samples per
pulse-echo measurement for
% 3.) all sequential pulse-echo measurements
all $n \in \setconsnonneg{ N_{\text{in}} - 1 }$.
% c) normalized N_{t}^{(n)}-point DFTs provided the relevant Fourier coefficients forming the vector for the quantized recording times
% book:Mallat2009, Chapter 3: Discrete Revolution / Sect. 3.3: Finite Signals / Sect. 3.3.2: Discrete Fourier Transform
% - BY DEFINITION, THE DISCRETE FOURIER TRANSFORM (DFT) OF f IS
%   [ \hat{f}[k] = \inprod{ f }{ e_{k} } = \sum_{ n = 0 }^{ N - 1 } f[n] exp( -i 2 \pi k n / N ). ] (3.49) (p. 77)
% - Since \norm{ e_{k} }{2}^{2} = N, (3.48) gives an INVERSE DISCRETE FOURIER FORMULA:
%   [ f[n] = \frac{ 1 }{ N } \sum_{ k = 0 }^{ N - 1 } \hat{f}[k] exp( i 2 \pi k n / N ). ] (3.50) (p. 77)
% - The orthogonality of the basis also implies a PLANCHEREL FORMULA:
%   [ \norm{ f }{2}^{2} = \sum_{ n = 0 }^{ N - 1 } \abs{ f[n] }^{2} = \frac{ 1 }{ N } \sum_{ k = 0 }^{ N - 1 } \abs{ \hat{f}[k] }^{2}. ] (3.51) (p. 77)
% - The DISCRETE FOURIER TRANSFORM (DFT) of a signal f of period N is computed from its values for 0 \leq n < N. (p. 77)
% - Then why is it important to consider it a periodic signal with period N rather than a finite signal of N samples? (p. 77)
% - The answer lies in the interpretation of the Fourier coefficients. (p. 77)
% - The DISCRETE FOURIER SUM (3.50) DEFINES A SIGNAL OF PERIOD N for which the samples f[0] and f[N-1] are side by side. (p. 77)
% - If f[0] and f[N-1] are very different, this produces a BRUTAL TRANSITION IN THE PERIODIC SIGNAL, creating
%   RELATIVELY HIGH-AMPLITUDE FOURIER COEFFICIENTS AT HIGH FREQUENCIES. (p. 77)
% book:Manolakis2005, Chapter 2: Fundamentals of Discrete-Time Signal Processing / Sect. 2.2: Transform-Domain Representation of Deterministic Signals / Sect. 2.2.3: The Discrete Fourier Transform
% - The N-POINT DISCRETE FOURIER TRANSFORM (DFT) OF AN N-POINT SEQUENCE {x(n), n = 0, 1, ..., N-1} is defined by
%   (2.2.25). (p. 42)
% - The N-point sequence {x(n), n = 0, 1, ..., N-1} can be recovered from its DFT coefficients {\tilde{X}(k), k = 0, 1, ..., N-1} by
%   the following INVERSE DFT FORMULA: (2.2.26) (p. 42)
% DFT of finite-duration signals
% - This implies that the N-POINT DFT OF A FINITE-DURATION SIGNAL WITH LENGTH N IS EQUAL TO
%   THE FOURIER TRANSFORM OF THE SIGNAL AT FREQUENCIES ωk = (2π/N)k, 0 ≤ k ≤ N − 1. (p. 42)
% - Hence, in this case, the N-POINT DFT CORRESPONDS TO THE UNIFORM SAMPLING of the Fourier transform of
%   a discrete-time signal at N equidistant points, that is, sampling in the frequency domain. (p. 42)
% DFT of periodic signals
% - Suppose now that x(n) is a periodic sequence with fundamental period N. (p. 43)
% - This sequence can be decomposed into frequency components by using the Fourier series in (2.2.8) and (2.2.9). (p. 43)
% - Comparison of (2.2.26) with (2.2.8) shows that (2.2.28) that is,
%   the DFT of one period of a periodic signal is given by the Fourier series coefficients of the signal scaled by the fundamental period. (p. 43)
% book:Briggs1995, Chapter 6: Errors in the DFT / Sect. 6.2: Periodic, Band-Limited Input
% - We are now in a position to investigate our first question:
%   HOW WELL DOES THE DFT APPROXIMATE THE FOURIER COEFFICIENTS of f? (p. 181)
% - Assume that the given function f has PERIOD A;
%   this includes the situation in which
%   f may be DEFINED ONLY ON THE INTERVAL [ -A/2; A/2 ] and is then EXTENDED PERIODICALLY. (p. 181)
% - Something rather remarkable can be discovered right away with one simple calculation. (p. 181)
% - > Discrete Poisson Summation Formula <
%   [ F_{k} = c_{k} + \sum_{ j = 1 }^{ \infty } ( c_{ k + j N } + c_{ k - j N } ) ] for k = - N/2+1:N/2. (6.4) (p. 182)
% - Expression (6.4) tells us that if f is
%   [1.)] periodic and
%   [2.)] band-limited with M < N/2, then
%   [3.)] c_{k} = 0 for \abs{ k } > N/2, and
%   [4.)] THE N-POINT DFT EXACTLY REPRODUCES THE N FOURIER COEFFICIENTS of f. (p. 182)
% - By this we mean that F_{ k } = c_{ k } for k = - N/2+1:N/2. (p. 182)
% - If f is not band-limited or is band-limited with M > N/2,
%   we can expect to see errors in the DFT. (p. 182)
% book:Briggs1995, Chapter 6: Errors in the DFT / Sect. 6.1: Introduction
% - As we will see, the form of the input sequence dictates how the DFT is used and how its output should be interpreted.
% - In some cases, the DFT will provide approximations to the Fourier coefficients of the input;
%   in other cases, the DFT will provide approximations to (samples of) the Fourier transform of the input. (p. 180)
% - The goal of this chapter is to understand exactly what the DFT approximates in each case, and then to estimate
%   the size of the errors in those approximations. (p. 180)
% book:Briggs1995, Chapter 2: The Discrete Fourier Transform / Sect. 2.4.: DFT Approximations to Fourier Series Coefficients [DERIVATION OF THE DFT]
% - Letting f_{n} = f( x_{n} ),
%   we see that an APPROXIMATION TO THE FOURIER SERIES COEFFICIENT c_{k} is given by
%   [ c_{k} \approx \sum_{ n = -N/2 + 1 }^{ N/2 } f_{n} e^{ -j 2 \pi k n / N } = D{ f_{n} }_{k} ] [sic!] which, for k = -N/2 + 1 : N/2, is
%   PRECISELY THE DEFINITION OF THE DFT. (p. 40)
% - Thus we see that
%   the DFT GIVES APPROXIMATIONS TO THE FIRST N FOURIER COEFFICIENTS OF A FUNCTION f on a given interval [ -A/2, A/2 ] in
%   a very natural way. (p. 40)
Normalized $N_{t}^{(n)}$-point \acp{DFT}
%\footnote{%
%average of first and last sample%
%\begin{equation*}
  %u_{m, l}^{(\text{rx}, n)}
  %=
  %\frac{ 1 }{ N_{t}^{(n)} }
  %e^{ -j 2 \pi l q_{\text{lb}}^{(n)} / N_{t}^{(n)} }
  %\sum_{ q = 0 }^{ N_{t}^{(n)} - 1 }
    %\tilde{u}_{m}^{(\text{rx}, n)}[ ( q_{\text{lb}}^{(n)}  + q ) T_{\text{s}}^{(n)} ]
    %e^{ -j 2 \pi l q / N_{t}^{(n)} }
%\end{equation*}
%$\tilde{u}_{m}^{(\text{rx}, n)}( q_{\text{lb}}^{(n)} T_{\text{s}}^{(n)} )$ is the average value
%}
(cf. e.g.
\cite[Sect. 3.3.2]{book:Mallat2009},
\cite[Sect. 2.2.3]{book:Manolakis2005},
\cite[Sect. 6.2]{book:Briggs1995}%
) provided
% 1.) Fourier coefficients of the recorded RF voltage signals
the \name{Fourier} coefficients
\eqref{eqn:recovery_disc_freq_v_rx_Fourier_series_coef} forming
% 2.) vector stacking the relevant Fourier coefficients of the recorded RF voltage signals (all pulse-echo measurements, multifrequent, all array elements)
the vector
\eqref{eqn:recovery_sys_lin_eq_v_rx_born_all_f_all_in_v_rx} for
% 3.) quantized recording times
the quantized recording times
$T_{ \text{rec} }^{(n)} = N_{t}^{(n)} T_{\text{s}}^{(n)}$.
% d) approximate efficiencies of the regular sampling in combination with the subsequent computation of the DFTs
% article:SchiffnerITUFFC2018, Sect. III.D Pulse-Echo Measurement Process (subsec:lin_mod_measurement_process)
% - Multiple time-domain methods for the \ac{ADC} of the received \ac{RF} voltage signals permit the determination of
%   the relevant \name{Fourier} coefficients.
% - These methods differ in their \emph{efficiency}, i.e. the quotient of
%   the data volume occupied by the quantized relevant \name{Fourier} coefficients and
%   the data volume digitized during the pulse-echo measurement.
% 1.) numbers of relevant discrete frequencies (effective time-bandwidth products)
The effective time-bandwidth products
\eqref{eqn:recon_disc_axis_f_discrete_BP_TB_product},
% 2.) quantized recording times
the quantized recording times,
% 3.) lower bounds on the sampling rates
the lower bounds on
the sampling rates, and
% 4.) effective bandwidths
the effective bandwidths approximate
% 5.) efficiencies
the efficiencies of
these procedures as
\begin{equation}
 %--------------------------------------------------------------------------------------------------------------
 % approximate efficiencies of the regular sampling in combination with the subsequent computation of the DFTs
 %--------------------------------------------------------------------------------------------------------------
  \text{Efficiency}^{(n)}
  =
  \frac{
    2 N_{f, \text{BP}}^{(n)}
  }{
    N_{t}^{(n)}
  }
  \approx
  2 T_{\text{s}}^{(n)} B_{ u }^{(n)}
  <
  1 - \frac{ f_{\text{lb}}^{(n)} }{ f_{\text{ub}}^{(n)} }
 \label{eqn:imp_fourier_coef_efficiency}
\end{equation}
for
% 6.) all sequential pulse-echo measurements
all $n \in \setconsnonneg{ N_{\text{in}} - 1 }$, where
% 7.) quantized complex-valued Fourier coefficient
a \name{Fourier} coefficient occupies
twice the data volume of
% 8.) quantized real-valued sample
a signal sample.
% e) upper bounds show that the digitized data volumes exceeded those occupied by the relevant Fourier coefficients
The upper bounds show that
the digitized data volumes exceeded
those occupied by
the relevant \name{Fourier} coefficients.
