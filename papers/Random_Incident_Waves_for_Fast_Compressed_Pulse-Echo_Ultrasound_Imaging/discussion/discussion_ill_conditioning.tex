%---------------------------------------------------------------------------------------------------------------
% 1.) multiple sequential pulse-echo measurements per image
%---------------------------------------------------------------------------------------------------------------
% a) number of sequential pulse-echo measurements per image simultaneously controls both the acquisition time and the number of observations
The number of
sequential pulse-echo measurements per
image simultaneously controls both
% 1.) acquisition time
the acquisition time and
% 2.) number of observations (all pulse-echo measurements, multifrequent, all transducer elements)
the number of
observations
\eqref{eqn:recovery_sys_lin_eq_num_obs}.
% b) its sufficient increase formally resolves the typical underdeterminedness of the linear algebraic system in ultrafast UI
% article:Schiffner2018, Sect. V.D. Regularization of the Discretized Linear Inverse Scattering Problem (subsec:recovery_regularization)
% - The linear algebraic system \eqref{eqn:recovery_sys_lin_eq_v_rx_born_all_f_all_in} is
%   [1.)] ill-conditioned and, for only a few sequential pulse-echo measurements,
%   [2.)] TYPICALLY UNDERDETERMINED.
% - The UNDERDETERMINEDNESS RESULTS FROM THE RELATIVELY SMALL NUMBER OF SEQUENTIAL PULSE-ECHO MEASUREMENTS PER IMAGE in
%   ultrafast \ac{UI}.
% - It necessitates the identification of the true vector stacking the regular samples in the discretized compressibility fluctuations
%   \eqref{eqn:recovery_sys_lin_eq_gamma_kappa_bp_vector} among infinitely many admissible vectors.
Its sufficient increase formally resolves
% 1.) typical underdeterminedness
the typical underdeterminedness of
% 2.) linear algebraic system (all pulse-echo measurements, multifrequent, all transducer elements)
the linear algebraic system
\eqref{eqn:recovery_sys_lin_eq_v_rx_born_all_f_all_in} in
% 3.) ultrafast UI
ultrafast \ac{UI}.
% c) wire and the tissue-mimicking phantoms required the minimal numbers 9 and 10 for this purpose
% total number of lattice points: N_{\text{lat}} = \num{262144}
% wire phantom:             N_{\text{lat}} / N_{\text{obs}} = 8.904347826 ( N_{\text{obs}} = \num{29440} )
% tissue-mimicking phantom: N_{\text{lat}} / N_{\text{obs}} = 9.102222222 ( N_{\text{obs}} = \num{28800} )
% 1.) wire phantom
The wire and
% 2.) tissue-mimicking phantom
the tissue-mimicking phantoms, for instance, required
the minimal numbers
$N_{\text{in}} \geq 9$ and
$N_{\text{in}} \geq 10$,
respectively, for
this purpose.
% d) even an observation process of full mathematical rank remains ill-conditioned and exhibits a nontrivial quasi-nullspace
% article:Schiffner2018, Sect. V.D. Regularization of the Discretized Linear Inverse Scattering Problem (subsec:recovery_regularization)
% - The ILL-CONDITIONING RESULTS FROM THE DISCRETIZATION OF THE \name{FREDHOLM} INTEGRAL EQUATIONS OF THE FIRST KIND representing
%   the \name{Born} approximation of the received \ac{RF} voltage signals \eqref{eqn:lin_mod_v_rx_born} \cite{book:Hansen2010,book:Hansen1998}.
% - It inflates numerical inaccuracies and potential deviations of the vector stacking the relevant \name{Fourier} coefficients of
%   the received \ac{RF} voltage signals \eqref{eqn:recovery_sys_lin_eq_v_rx_born_all_f_all_in_v_rx} from
%   its \name{Born} approximation \eqref{eqn:recovery_sys_lin_eq_v_rx_born_all_f_all_in_v_rx_born} to
%   prohibitively large recovery errors.
% book:Hansen2010, Chapter 1: Introduction and Motivation
% - ILL-CONDITIONED PROBLEMS ARE EFFECTIVELY UNDERDETERMINED. (p. 4)
% book:Hansen1998, Chapter 1: Setting the Stage
% - A VERY LARGE CONDITION NUMBER of the coefficient matrix A in a linear system of equations Ax = b implies that
%   SOME (OR ALL) OF THE EQUATIONS ARE NUMERICALLY LINEARLY DEPENDENT. (p. 1)
% book:Hansen1998, Chapter 1: Setting the Stage / Sect. 1.1. Problems with Ill-Conditioned Matrices
% - In particular, the CONDITION NUMBER OF A is defined as
%   THE RATIO BETWEEN THE LARGEST AND THE SMALLEST SINGULAR VALUES OF A. (p. 2)
% - Discrete ill-posed problems arise from
%   the DISCRETIZATION OF ILL-POSED PROBLEMS such as
%   FREDHOLM INTEGRAL EQUATIONS OF THE FIRST KIND. (p. 2)
% - Here all the singular values of A, as well as the SVD components of the solution, on the average, decay gradually to zero, and we say that
%   a discrete Picard condition (see §4.5) is satisfied. (p. 2)
% - Since there is no gap in the singular value spectrum, there is
%   NO NOTION OF A NUMERICAL RANK FOR THESE MATRICES. (p. 2)
% - For discrete ill-posed problems, the goal is to find a balance between
%   the RESIDUAL NORM and
%   the SIZE OF THE SOLUTION that matches the errors in the data as well as one's expectations to the computed solution. (pp. 2, 3)
% - Due to the LARGE CONDITION NUMBER OF A, BOTH CLASSES OF PROBLEMS [rank-deficient (gap), discrete ill-posed (overall decay)] ARE EFFECTIVELY UNDERDETERMINED, and therefore
%   many of the regularization methods described in this book can be used for both classes of problems. (p. 3)
% article:HansenBIT1990: The discrete picard condition for discrete ill-posed problems
% - Often, due to rounding errors as well as errors in the data,
%   SUCH ILL-CONDITIONED MATRICES HAVE FULL RANK IN A STRICT MATHEMATICAL SENSE and
%   the finite-dimensional least squares problem (1.1) is therefore not ill-posed in the original sense due to Hadamard (see e.g. I12, Section 1,1]). (p. 658)
Even
% 1.) observation process (all pulse-echo measurements, multifrequent, all transducer elements)
an observation process
\eqref{eqn:recovery_sys_lin_eq_v_rx_born_all_f_all_in_mat} of
% 2.) full mathematical rank
full mathematical rank, however, remains
% 3.) ill-conditioned
ill-conditioned and, thus, exhibits
% 4.) nontrivial quasi-nullspace
a nontrivial quasi-nullspace, i.e.
a set of
% 5.) nonzero vectors stacking the regular samples in the discretized relative spatial fluctuations in the unperturbed compressibility
nonzero compressibility fluctuations
\eqref{eqn:recovery_sys_lin_eq_gamma_kappa_bp_vector} whose pulse echoes contain
% 6.) relatively low electric energies
relatively low electric energies
\cite[4]{book:Hansen2010},
\cite[1, 3]{book:Hansen1998}.
% e) vectors represent lossy heterogeneous objects that are almost invisible to the pulse-echo measurement process
These vectors represent
lossy heterogeneous objects that are
% book:Devaney2012, Chapter 6: Scattering theory / Sect. 6.9 Non-scattering potentials / Sect. 6.9.4: Almost-invisible weak scatterers
% - A truly invisible object would produce no scattering for any arbitrary incident wavefield. (p. 269)
% - However, as we noted earlier, the scattering potential for such an object within the Born scattering model would have to have
%   a SPATIAL FOURIER TRANSFORM THAT VANISHED EVERYWHERE WITHIN THE EWALD LIMITING SPHERE and
%   such a potential would, of necessity, be INFINITE IN EXTENT. (p. 269)
% - Although no truly invisible compactly supported weakly scattering potential exists, it is simple to construct such potentials that
%   will be invisible in any finite set of scattering experiments employing incident plane waves or incident waves that can be expressed as
%   a superposition of a finite set of plane waves. (p. 269)
% - We will refer to such potentials as “ALMOST-INVISIBLE SCATTERING POTENTIALS.” (p. 269)
% book:Devaney2012, Chapter 6: Scattering theory / Sect. 6.9 Non-scattering potentials / Sect. 6.9.1: Non-scattering potentials within the Born approximation
% - Another class of scattering potentials within the Born scattering model that generate a zero scattered field are
%   those potentials whose SPATIAL FOURIER TRANSFORMS VANISH EVERYWHERE WITHIN THE EWALD LIMITING SPHERE. (p. 267)
% - Such scattering potentials would then have \tilde{V}[ k_{0} ( \vect{s} − \vect{s}_{0} ) ] = 0 for
%   any set of incident and scattered wave vectors and, hence, would be
%   TRULY "INVISIBLE" IN ANY SCATTERING EXPERIMENT. (p. 267)
% - However, such scattering potentials are NOT COMPACTLY SUPPORTED WITHIN A FINITE SCATTERING VOLUME τ_{0} and, hence,
%   are not “NON-SCATTERING POTENTIALS” within the definition used here. (p. 267)
almost invisible to
the pulse-echo measurement process
\cite[Sects. 6.9.1 and 6.9.4]{book:Devaney2012}.
% TODO: enable discretization -> invisible differences to continuous object; rows vs cols
% f) their existence renders the linear algebraic system effectively underdetermined
% book:Hansen2010, Chapter 1: Introduction and Motivation
% - Hence
%   WE CAN ADD A LARGE AMOUNT OF THIS VECTOR TO THE SOLUTION VECTOR WITHOUT CHANGING THE RESIDUAL VERY MUCH;
%   THE SYSTEM BEHAVES ALMOST LIKE AN UNDERDETERMINED SYSTEM. (p. 4)
% - By supplying the correct additional information we can compute a good approximate solution. (p. 4)
% - The main difficulty is how to choose the parameter δ when we have little knowledge about the exact solution. (p. 4)
Their existence renders
% 1.) linear algebraic system (all pulse-echo measurements, multifrequent, all transducer elements)
the linear algebraic system
\eqref{eqn:recovery_sys_lin_eq_v_rx_born_all_f_all_in} effectively underdetermined because
the additions of
% 2.) linear combinations
their linear combinations to
% 3.) exact solution
any exact solution only induce
% 4.) negligible residuals
negligible residuals
\cite[4]{book:Hansen2010},
\cite[1, 3]{book:Hansen1998}.
% g) complex exponential functions of relatively low and high spatial frequencies exemplified such vectors
% article:Schiffner2018, Sect. VIII. Results / Sect. VIII-B. Tissue-Mimicking Phantom / Sect. VIII-B.1) Recorded Electric Energies (subsubsec:results_phantom_tissue_energy_rx)
% - The TRANSFER BEHAVIORS OF THE SENSING MATRICES \eqref{eqn:recovery_reg_sensing_matrix} induced by all incident waves resembled those of
%   BANDPASS FILTERS SUPPRESSING RELATIVELY LOW AND HIGH SPATIAL FREQUENCIES (cf. \cref{fig:sim_study_obj_B_norms_kappa}).
% - The HIGH DYNAMIC RANGES exceeding \SI{70}{\deci\bel} indicated the existence of structural building blocks whose
%   pulse echoes contained RELATIVELY LOW ELECTRIC ENERGIES \eqref{eqn:recovery_reg_v_rx_born_trans_coef_energy}.
The complex exponential functions of
relatively low and
high spatial frequencies exemplified
such vectors for
% 1.) observation processes (all pulse-echo measurements, multifrequent, all array elements)
the observation processes
\eqref{eqn:recovery_sys_lin_eq_v_rx_born_all_f_all_in_mat} induced by
% 2.) all incident waves
all waves incident on
% 3.) tissue-mimicking phantom
the tissue-mimicking phantom
(cf. \cref{fig:sim_study_obj_B_norms_kappa}).
% TODO: complex exponential functions potentially falsify any recovered vector
% h) maximal passbands are invariant to the number of sequential pulse-echo measurements per image
The maximal passbands, which are limited by
% 1.) electromechanical transfer behavior of the instrumentation
the electromechanical transfer behavior of
the instrumentation and approximately achieved by
% 2.) random waves
the random waves, are invariant to
% 3.) number of sequential pulse-echo measurements per image
the number of
sequential pulse-echo measurements per image.
% i) increase of the latter [number of sequential pulse-echo measurements per image] does not eliminate the need for regularization
Although
the increase of
% 1.) number of sequential pulse-echo measurements per image
the latter does not eliminate
% 2.) need for regularization
the need for
regularization,
% j) increase of the latter enlarges the suboptimal passbands achieved by the steered QPWs, improves the robustness of the recovered compressibility fluctuations against the additive errors
% article:Schiffner2018, Sect. IX. Discussion / Sect. IX-C. Spatially Extended Structural Building Blocks Increase the Robustness Against the Additive Errors
% - The variations in the incident acoustic energies \eqref{eqn:recovery_p_in_energy} across the isolated positions of the wires and
%   the resulting variations in the mean relative \acp{RMSE} (cf. \cref{fig:sim_study_obj_A_p_in_energy,fig:sim_study_obj_A_sr_1_mean_rel_errors}) suggest that
%   (i) the energy-guided relocation of the wires,
%   (ii) different realizations of the random waves, or
%   (iii) MULTIPLE SEQUENTIAL PULSE-ECHO MEASUREMENTS OVERCOME
%   THIS SENSITIVITY.
% - A fixed experimental setup, however, excludes option (i), and only option (iii) ensures consistent results for various configurations.
it (i) enlarges
% 1.) suboptimal passbands achieved by the steered QPWs
the suboptimal passbands achieved by
the steered \acp{QPW} for
% 2.) preferred directions of propagation
various preferred directions of
propagation and
%
(ii) improves
the robustness of
% 3.) recovered relative spatial fluctuations in the unperturbed compressibility
the recovered compressibility fluctuations
\eqref{eqn:recovery_reg_norm_lq_minimization_sol_mat_params} against
% 4.) additive errors
the additive errors
(cf. \cref{subsec:discussion_robustness_additive_errors}).
