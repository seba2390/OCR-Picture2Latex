%---------------------------------------------------------------------------------------------------------------
% 1.) important investigations left for future research
%---------------------------------------------------------------------------------------------------------------
% a) multiple important investigations had to be left for future research
Multiple important investigations had to be left for
future research.
% b) important investigations include [...]
% proc:ChartrandICASSP2008: Iteratively reweighted algorithms for compressive sensing
% - Fig. 1. Plots of recovery frequency as a function of K.
%   Regularized IRLS has a much higher recovery rate than unregularized IRLS, except when p = 1 when they are almost identical.
%   Regularized IRLS recovers the greatest range of signals when p is small, while unregularized IRLS performs less well for small p than when p = 1. (p. 3871)
% - Fig. 2. Same data as in Figure 1, but with every p shown.
%   Unregularized IRLS is best at p = 0.9, then decays quickly for smaller p,
%   Regularized IRLS improves as p gets smaller, and recovers many more signals than unregularized IRLS. (p. 3871)
% article:ChartrandISPL2007: Exact Reconstruction of Sparse Signals via Nonconvex Minimization
% - Fig. 3. Observed probabilities of exact reconstruction for
%   different numbers of measurements (M) and
%   sparsity-to-measurements ratio (K/M), for four values of p: p = 1 (left), p = 0.95 (second from left), p = 0.75 (second from right), and p = 0.5 (right).
%   Compared with p = 1, exact reconstruction is obtained with larger values of K/M, even for p = 0.95, and almost double the value of K/M for p = 0.5. (p. 709)
% - Fig. 4. Observed probabilities of exact reconstruction for
%   a signal of sparsity K = 16, for different numbers of measurements M and four values of p.
%   Compared with p = 1, substantially fewer measurements are required for exact reconstruction when p = 0.95.
%   Decreasing p further decreases the required value of M but by less and less as p gets smaller. (p. 709)
These mainly include
statistical analyses of
% 1.) admissible numbers of significant components
the admissible numbers of
significant components in
% 2.) nearly-sparse representation
the nearly-sparse representation
\eqref{eqn:recovery_reg_sparse_representation} for various
% 3.) numbers of sequential pulse-echo measurements per image
numbers of
sequential pulse-echo measurements per image,
% 4.) parameters q governing the sparsity-promoting lq-minimization method
parameters $q$ governing
the sparsity-promoting $\ell_{q}$-minimization method
\eqref{eqn:recovery_reg_norm_lq_minimization},
% 5.) energy levels of the additive errors
energy levels of
the additive errors,
% 6.) realizations of the random waves
realizations of
the random waves, and
% 7.) orthonormal bases
orthonormal bases
(cf. e.g.
  \cite[Figs. 1 and 2]{proc:ChartrandICASSP2008},
  \cite[Figs. 3 and 4]{article:ChartrandISPL2007}%
).
% c) large numbers of parameter combinations and the high computational costs currently prevent such combinatorial analyses for normalized CS problems of the investigated sizes
The large numbers of
parameter combinations and
% 1.) high computational costs induced by the proposed implementation of the sparsity-promoting lq-minimization method
the high computational costs induced by
the proposed implementation of
% 2.) sparsity-promoting lq-minimization method
the sparsity-promoting $\ell_{q}$-minimization method
\eqref{eqn:recovery_reg_norm_lq_minimization} currently prevent
such combinatorial analyses for
% 3.) normalized CS problems
normalized \ac{CS} problems
\eqref{eqn:recovery_reg_norm_prob_general} of
the investigated sizes.
% d) presented study focused on simple types of random waves that can readily be synthesized by programmable UI systems
Furthermore,
the presented study focused on
% 1.) simple types of random waves
simple types of
random waves that
can readily be synthesized by
% 2.) programmable UI systems
programmable \ac{UI} systems.
% e) more advanced syntheses increase the complexity of the transmit hardware
Although
more advanced syntheses, e.g.
\TODO{use correct description}
% 1.) element specific coded excitation voltages
element specific coded excitation voltages, increase
the complexity of
the transmit hardware,
% f) more advanced syntheses potentially further decorrelate the pulse echoes and deserve additional studies
they potentially further decorrelate
the pulse echoes and, thus, deserve
additional studies.
\TODO{article:IlovitshNatComBio2018, structured illumination}
% g) numerical simulations of the pulse-echo measurement process in the two-dimensional Euclidean space exclude variations along the elevational direction
Moreover,
the numerical simulations of
% 1.) pulse-echo measurement process in the two-dimensional Euclidean space
the pulse-echo measurement process in
the two-dimensional Euclidean space, i.e. $d = 2$, exclude
% 2.) variations along the elevational direction
variations along
the elevational direction.
% b) two-dimensional Euclidean space prevented the simulation of the vibrating faces' height and the elevational focus induced by the acoustic lens
%Although
%this space prevented
%the simulation of
% 1.) width of the vibrating faces along the r_{2}-axis
%the vibrating faces' height and
% 2.) elevational focus induced by the acoustic lens
%the elevational focus induced by
%the acoustic lens,
% h) numerical simulations [two-dimensional] underestimate the diffraction-induced decay of the ultrasonic waves and predict less realistic results
They underestimate
% 1.) diffraction-induced decay of the ultrasonic waves
the diffraction-induced decay of
the ultrasonic waves, which is asymptotically proportional to
$\norm{ \vect{r} }{2}^{ - ( d - 1 ) / 2 }$ for
% 2.) outgoing free-space Green's functions (two- and three-dimensional Euclidean spaces)
the outgoing free-space \name{Green}'s functions
\eqref{eqn:app_helmholtz_green_free_space_2_3_dim}, and predict
% 3.) less realistic results
less realistic results.
% i) rectilinear boundary conditions prohibit the usage of curved transducer arrays
% TODO: curved arrays incompatible with Rayleigh-Sommerfeld diffraction integrals, 
The rectilinear boundary conditions prohibit
the usage of
% 1.) curved transducer arrays
curved transducer arrays, and
% j) nonlinear wave propagation
% article:NgITUFFC2006: Modeling ultrasound imaging as a linear, shift-variant system
% I. Introduction
% - Modern clinical practice sometimes exploits higher-order harmonics generated by nonlinearities during transmission. (p. 549)
% - We note a comment in [3] that, although the forward propagation of waves in such a case is nonlinear,
%   A LINEAR MODEL WOULD STILL HOLD FOR THE BACK PROPAGATION provided that the scattering is weak
%   (which is usually the case in soft tissue). (p. 549)
the linear wave propagation neglects
finite amplitude effects.
% j) first Born approximation excludes multiple scattering
% article:Schiffner2018,
% - The \name{Born} approximation \eqref{eqn:lin_mod_pert_born_p_B_1} represents single scattering \cite[709]{book:Born1999}, i.e.
%   the approximated scattered acoustic pressure field is the exclusive response to
%   the incident acoustic pressure field and does not interact with
%   the compressibility fluctuations \eqref{eqn:lin_mod_mech_model_tis_simple_rel_fluctuations}.
% - This map dominates practical image recovery methods in \ac{UI} and allows the systematic derivation of analytical inversion schemes.
Eventually,
the \name{Born} approximation, which dominates
% 1.) practical image recovery methods in UI
practical image recovery methods in
\ac{UI}, excludes
% 2.) multiple scattering
multiple scattering and, thus, any
% 3.) phase aberrations
phase aberrations.
% k) presented results outline general potential pitfalls and benefits of random incident waves
Despite
these limitations and
open investigations, however,
the presented results reliably outline
the potential drawbacks and
the benefits of
random incident waves in
fast compressed pulse-echo \ac{UI}.
