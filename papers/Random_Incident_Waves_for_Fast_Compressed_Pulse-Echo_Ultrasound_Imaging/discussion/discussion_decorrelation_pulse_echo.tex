%---------------------------------------------------------------------------------------------------------------
% 1.) isolated sharp maxima and resolutions (reference sensing matrices)
%---------------------------------------------------------------------------------------------------------------
% a) isolated maxima of minimum FEHMs indicate optimal spatial and spectral resolutions
% article:Schiffner2018, Sect. VIII. Results / Sect. VIII-B. Tissue-Mimicking Phantom / Sect. VIII-B.2) Transform Point Spread Functions (subsubsec:results_phantom_tissue_tpsf)
% - Both REFERENCE SENSING MATRICES produced similar results as
%   the RANDOM OBSERVATION PROCESS \eqref{eqn:sim_study_params_ref_obs_proc_rip}, i.e.
%   UNIFORMLY DISTRIBUTED RANDOM VALUES CLOSE TO ZERO THAT RENDERED THE MAXIMA SHARP AND ISOLATED
%   (cf. \cref{fig:sim_study_obj_A_sr_1_tpsf_images_rip_1}).
% article:Schiffner2018, Sect. VIII. Results / Sect. VIII-A. Wire Phantom / Sect. VIII-A.2) Point Spread Functions (subsubsec:results_phantom_wire_psf)
% - Both REFERENCE OBSERVATION PROCESSES produced
%   RANDOM VALUES CLOSE TO ZERO THAT RENDERED THE MAXIMA SHARP AND ISOLATED.
% - The random observation process \eqref{eqn:sim_study_params_ref_obs_proc_rip} UNIFORMLY DISTRIBUTED these values over
%   the \ac{FOV}, whereas its structured version \eqref{eqn:sim_study_params_ref_obs_proc_gwn} formed
%   NOTICEABLE GAPS that were laterally adjacent to the maxima and shaped hourglasses of larger absolute values
%   (cf. inset image).
The isolated maxima of
% 1.) minimum FEHMs
% article:Schiffner2018, Sect. VIII. Results / Sect. VIII-B. Tissue-Mimicking Phantom / Sect. VIII-B.2) Transform Point Spread Functions (subsubsec:results_phantom_tissue_tpsf)
% - Both REFERENCE SENSING MATRICES CONSISTENTLY ACHIEVED
%   THE MINIMUM \ac{FEHM} of a TWO-DIMENSIONAL NORMALIZED SPATIAL FREQUENCY ELEMENT $\Delta \hat{K} \approx \num{3.81e-6}$ for all fixed spatial frequencies.
% article:Schiffner2018, Sect. VIII. Results / Sect. VIII-A. Wire Phantom / Sect. VIII-A.2) Point Spread Functions (subsubsec:results_phantom_wire_psf)
% - Both REFERENCE OBSERVATION PROCESSES CONSISTENTLY ACHIEVED
%   THE MINIMUM \ac{FEHM} of a TWO-DIMENSIONAL VOLUME ELEMENT $\Delta V \approx \SI{5.81e-3}{\milli\meter\squared}$ for all fixed positions.
minimum \acp{FEHM}, which were embedded in
% 2.) random values close to zero
random values close to zero and characterized
% 3.) transform point spread function (TPSF)
the \acp{TPSF}
\eqref{eqn:cs_math_tpsf} associated with
% 4.) reference sensing matrices
all reference sensing matrices
(cf. \cref{%
  fig:sim_study_obj_A_sr_1_tpsf_images_rip_1,%
  fig:sim_study_obj_A_sr_1_tpsf_images_blgwn_1,%
  fig:sim_study_obj_B_sr_1_tpsf_images_rip_1,%
  fig:sim_study_obj_B_sr_1_tpsf_images_blgwn_1%
}), indicate
% 5.) optimal spatial and spectral resolutions
optimal spatial and
spectral resolutions.
% b) isolated maxima of minimum FEHMs resolved the adjacent grid points in the FOV for the wire phantom and the adjacent discrete spatial frequencies for the tissue-mimicking phantom
They resolved
% 1.) adjacent grid points in the FOV
the adjacent grid points in
the \ac{FOV} for
% 2.) wire phantom
the wire phantom and
% 3.) adjacent discrete spatial frequencies
the adjacent discrete spatial frequencies for
% 4.) tissue-mimicking phantom
the tissue-mimicking phantom.
% c) RIP for 4-sparse representations ensured the stable recovery of all 2-sparse representations
% article:Schiffner2018, Sect. VII. Simulation Study / Sect. VII-A. Parameters / Sect. VII-A.9) Reference Sensing Matrices
% - For a sufficiently large number of observations \eqref{eqn:recovery_sys_lin_eq_num_obs}, both
%   the real-valued random $N_{\text{obs}} \times N_{\text{lat}}$ observation process
%   [ ... ] (eqn:sim_study_params_ref_obs_proc_rip) and
%   the ASSOCIATED COMPLEX-VALUED $N_{\text{obs}} \times N_{\text{lat}}$ SENSING MATRIX [ ... ] (eqn:sim_study_params_ref_sens_mat_rip) MET
%   the \ac{RIP} WITH VERY HIGH PROBABILITY (cf. \cref{sec:compressed_sensing}).
In fact,
% 1.) restricted isometry property (RIP)
the \ac{RIP} for
% 2.) 4-sparse representations
$4$-sparse representations
\eqref{eqn:recovery_reg_sparse_representation}, which was met by
% 3.) first reference sensing matrices (RIP)
the random sensing matrices
\eqref{eqn:sim_study_params_ref_sens_mat_rip} with
% 4.) very high probability
very high probability, ensured
% 5.) stable recovery
the stable recovery of
% 6.) all 2-sparse representations
all $2$-sparse representations
\eqref{eqn:recovery_reg_sparse_representation}, including those whose
% 7.) nonzero components
nonzero components populate
% 8.) adjacent grid points or discrete spatial frequencies
adjacent grid points or
discrete spatial frequencies.

%---------------------------------------------------------------------------------------------------------------
% 2.) distribution of the random values close to zero (reference sensing matrices)
%---------------------------------------------------------------------------------------------------------------
% a) incoherent aliasing prevented their sparse approximations and enabled their removal by the sparsity-promoting lq-minimization method
% article:Schiffner2018, Sect. II. Compressed Sensing in a Nutshell (sec:compressed_sensing)
% - For $n_{1} \neq n_{2}$, however, both column vectors typically differ, and the absolute value of the \ac{TPSF} \eqref{eqn:cs_math_tpsf} is desired to be
%   AS CLOSE TO ZERO AS POSSIBLE WITH NOISE-LIKE FEATURES, e.g. a UNIFORM DISTRIBUTION OF THE FLUCTUATING ENERGY OVER THE INDICES
%   \cite{article:ProvostITMI2009,article:LustigMRM2007}.
%   => (1) CLOSE TO ZERO AS POSSIBLE
%   => (2) NOISE-LIKE FEATURES: UNIFORM DISTRIBUTION, FLUCTUATING ENERGY
% - The SMALL ABSOLUTE VALUES indicate that
%   the observation process reliably DISTINGUISHES THE ADMISSIBLE STRUCTURAL BUILDING BLOCKS, and
%   the NOISE-LIKE FEATURES, which are referred to as INCOHERENT ALIASING, do not misguide
%   the sparsity-promoting $\ell_{q}$-minimization method \eqref{eqn:cs_lq_minimization}.
The noise-like properties of
% 1.) random values close to zero
random values close to zero enabled
% 2.) removal
their removal by
% 3.) sparsity-promoting lq-minimization method
the sparsity-promoting $\ell_{q}$-minimization method
\eqref{eqn:recovery_reg_norm_lq_minimization}.
% b) gaps resulted from the single reception angle characterizing the pulse-echo setup
% article:Schiffner2018, Sect. VIII. Results / Sect. VIII-A. Wire Phantom / Sect. VIII-A.2) Point Spread Functions (subsubsec:results_phantom_wire_psf)
% - The random observation process \eqref{eqn:sim_study_params_ref_obs_proc_rip} UNIFORMLY DISTRIBUTED these values over
%   the \ac{FOV}, whereas its structured version \eqref{eqn:sim_study_params_ref_obs_proc_gwn} formed
%   NOTICEABLE GAPS THAT WERE LATERALLY ADJACENT TO THE MAXIMA and shaped
%   HOURGLASSES OF LARGER ABSOLUTE VALUES (cf. inset image).
The gaps formed by
% 1.) second reference observation process and its entries (GWN)
the structured random observation process
\eqref{eqn:sim_study_params_ref_obs_proc_gwn} for
% 2.) wire phantom
the wire phantom
(cf. \cref{fig:sim_study_obj_A_sr_1_tpsf_images_blgwn_1}) arose from
% 3.) single reception angle
the single reception angle in
% 4.) pulse-echo setup
the pulse-echo setup, i.e.
% 5.) fixed position
the fixed position of
% 6.) linear transducer array
the linear transducer array on
% 6.) single edge of the FOV
a single edge of
the \ac{FOV}.
% c) difference indicates potential benefits of the random waves for spatially extended basis functions
% article:Schiffner2018, Sect. VIII. Results / Sect. VIII-B. Tissue-Mimicking Phantom / Sect. VIII-B.2) Transform Point Spread Functions (subsubsec:results_phantom_tissue_tpsf)
% - The STRUCTURED RANDOM SENSING MATRIX \eqref{eqn:sim_study_params_ref_sens_mat_gwn}, however, significantly elongated
%   THESE MAXIMA ALONG THE $\hat{K}_{2}$-axis IN ADDITION TO MODEST LATERAL EXTENSIONS (cf. inset image).
The elimination of
these gaps by
% 1.) second reference sensing matrix (GWN)
the structured random sensing matrix
\eqref{eqn:sim_study_params_ref_sens_mat_gwn} for
% 2.) tissue-mimicking phantom
the tissue-mimicking phantom
(cf. \cref{fig:sim_study_obj_B_sr_1_tpsf_images_blgwn_1}) hints at
% 3.) potential benefits
potential benefits provided by
% 4.) random waves
the random waves for
% 5.) specified structural building blocks
the specified structural building blocks, i.e.
% 6.) complex exponential functions of distinct spatial frequencies
the complex exponential functions.

%---------------------------------------------------------------------------------------------------------------
% 3.) all incident waves degrade the optimal spatial and spectral resolutions
%---------------------------------------------------------------------------------------------------------------
% a) increased FEHMs of the TPSFs associated with the sensing matrices induced by all incident waves indicate degraded spatial and spectral resolutions
The increased \acp{FEHM} of
% 1.) transform point spread functions (TPSFs)
the \acp{TPSF}
\eqref{eqn:cs_math_tpsf} associated with
% 2.) sensing matrices (all pulse-echo measurements, multifrequent, all array elements)
the sensing matrices
\eqref{eqn:recovery_reg_sensing_matrix} induced by
% 3.) all incident waves
all incident waves
(cf. \cref{%
  tab:sim_study_obj_A_sr_1_tpsf_fehm,%
  tab:sim_study_obj_B_sr_1_tpsf_fehm%
}), which resulted from
% 4.) elliptical-shaped regions of absolute values close to unity around the maxima
% article:Schiffner2018, Sect. VIII. Results / Sect. VIII-A. Wire Phantom / Sect. VIII-A.2) Point Spread Functions (subsubsec:results_phantom_wire_psf)
% - The observation processes \eqref{eqn:recovery_sys_lin_eq_v_rx_born_all_f_all_in_mat} induced by all incident waves concentrated
%   RELATIVELY LARGE ABSOLUTE VALUES CLOSE TO UNITY IN ELLIPTICAL-SHAPED REGIONS AROUND THE MAXIMA.
% - The lengths of the minor and major axes ranged from \SIrange{0.15}{0.3}{\milli\meter} and from \SIrange{0.46}{0.76}{\milli\meter}, respectively.
% - They DISTRIBUTED THE NONZERO VALUES LESS UNIFORMLY and FORMED SIDELOBES OF VARIOUS CHARACTERS.
the elliptical-shaped regions of
absolute values close to
unity around
the maxima for
% 5.) wire phantom
the wire phantom
(cf. \cref{%
  fig:sim_study_obj_A_sr_1_tpsf_images_qpw_1,%
  fig:sim_study_obj_A_sr_1_tpsf_images_rnd_apo_1,%
  fig:sim_study_obj_A_sr_1_tpsf_images_rnd_del_1,%
  fig:sim_study_obj_A_sr_1_tpsf_images_rnd_apo_del_1%
}) and
% 6.) extended maxima
% article:Schiffner2018, Sect. VIII. Results / Sect. VIII-B. Tissue-Mimicking Phantom / Sect. VIII-B.2) Transform Point Spread Functions (subsubsec:results_phantom_tissue_tpsf)
% - The sensing matrices \eqref{eqn:recovery_reg_sensing_matrix} induced by the random waves APPROXIMATELY MAINTAINED
%   THESE MAXIMA but CONFINED SIMILAR NOISE-LIKE ARTIFACTS TO THEIR PASSBANDS.
% - In contrast, the sensing matrix \eqref{eqn:recovery_reg_sensing_matrix} induced by the \ac{QPW} formed
%   [1.)] SMOOTH COHERENT SIDELOBES, whose absolute values lacked noise-like features, and indicated
%   [2.)] the PRESENCE OF UNSPECIFIED SPATIAL FREQUENCIES BY SECONDARY ISOLATED MAXIMA, e.g.
%   an absolute value of approximately \SI{-2.4}{\deci\bel} at
%   the normalized spatial frequency $\hat{\vect{K}} \approx \trans{ ( \num{0.12}, \num{0.35} ) }$.
the extended maxima for
% 7.) tissue-mimicking phantom
the tissue-mimicking phantom
(cf. \cref{%
  fig:sim_study_obj_B_sr_1_tpsf_images_qpw_1,%
  fig:sim_study_obj_B_sr_1_tpsf_images_rnd_apo_1,%
  fig:sim_study_obj_B_sr_1_tpsf_images_rnd_del_1,%
  fig:sim_study_obj_B_sr_1_tpsf_images_rnd_apo_del_1%
}), indicate
% 8.) spatial and spectral resolutions
degraded spatial and
spectral resolutions.
% b) incident acoustic pressure fields strongly correlated the pulse echoes of the adjacent grid points and the adjacent discrete spatial frequencies
% article:Schiffner2018, Sect. VII. Simulation Study / Sect. VII-A. Parameters / Sect. VII-A.9) Reference Sensing Matrices (subsubsec:sim_study_params_ref_sens_mat)
% - The replacement of the incident acoustic pressure field \eqref{eqn:recovery_p_in} in the observation process \eqref{eqn:recovery_sys_lin_eq_v_rx_born_all_f_all_in_mat} by
%   COMPLEX-VALUED \ac{GWN} additionally formed the complex-valued structured $N_{\text{obs}} \times N_{\text{lat}}$ observation process
%   [ \mat{\Phi}^{(\text{\acs{GWN}})} = \mat{\Phi}\bigl[ p^{(\text{in})} \bigr], p_{l}^{(\text{in}, 0)}( \vect{r}_{\text{lat}, i} ) \underset{ \text{\acs{IID}} }{ \sim } \gaussian{ 0 }{ 1 } ]
%   (eqn:sim_study_params_ref_obs_proc_gwn) and the associated complex-valued $N_{\text{obs}} \times N_{\text{lat}}$ sensing matrix
%   [ \mat{A}^{(\text{\acs{GWN}})} = \mat{\Phi}^{(\text{\acs{GWN}})} \mat{\Psi}. ] (eqn:sim_study_params_ref_sens_mat_gwn)
In contrast to
% 1.) complex-valued GWN
the \ac{GWN} forming
% 2.) second reference sensing matrix (GWN)
the structured random sensing matrix
\eqref{eqn:sim_study_params_ref_sens_mat_gwn},
% 3.) incident acoustic pressure fields
the incident acoustic pressure fields met
% 4.) Helmholtz equations for the incident acoustic pressure fields
the \name{Helmholtz} equations
\eqref{eqn:lin_mod_sol_wave_eq_pde_p_in} and, thus, strongly correlated
% 5.) strongly correlated pulse echoes
the pulse echoes of
% 6.) adjacent grid points in the FOV
the adjacent grid points and
% 7.) adjacent discrete spatial frequencies
the adjacent discrete spatial frequencies.
% c) FEHMs confirm smaller regions of highly-correlated observations, and the empirical CDFs attest statistical properties more similar to both reference observation processes
In
their attempt to replicate
the desirable properties of
the \ac{GWN}, however,
% 1.) random waves
the random waves significantly outperformed
% 2.) quasi-plane wave (QPW)
the \ac{QPW}.

%---------------------------------------------------------------------------------------------------------------
% 4.) advantages of the random waves over the QPW (wire and tissue-mimicking phantoms)
%---------------------------------------------------------------------------------------------------------------
% a) noise-like artifacts with more uniform distributions of the nonzero values resembled those induced by the structured random sensing matrices
% article:Schiffner2018, Sect. VIII. Results / Sect. VIII-B. Tissue-Mimicking Phantom / Sect. VIII-B.2) Transform Point Spread Functions (subsubsec:results_phantom_tissue_tpsf)
% - The sensing matrices \eqref{eqn:recovery_reg_sensing_matrix} induced by the random waves APPROXIMATELY MAINTAINED
%   THESE MAXIMA but CONFINED SIMILAR NOISE-LIKE ARTIFACTS TO THEIR PASSBANDS.
% - In contrast, the sensing matrix \eqref{eqn:recovery_reg_sensing_matrix} induced by the \ac{QPW} formed
%   [1.)] SMOOTH COHERENT SIDELOBES, whose absolute values lacked noise-like features, and indicated
%   [2.)] the PRESENCE OF UNSPECIFIED SPATIAL FREQUENCIES BY SECONDARY ISOLATED MAXIMA, e.g.
%   an absolute value of approximately \SI{-2.4}{\deci\bel} at
%   the normalized spatial frequency $\hat{\vect{K}} \approx \trans{ ( \num{0.12}, \num{0.35} ) }$.
% article:Schiffner2018, Sect. VIII. Results / Sect. VIII-A. Wire Phantom / Sect. VIII-A.2) Point Spread Functions (subsubsec:results_phantom_wire_psf)
% - The observation process \eqref{eqn:recovery_sys_lin_eq_v_rx_born_all_f_all_in_mat} induced by
%   the \ac{QPW} DEVIATED MOST SIGNIFICANTLY FROM BOTH REFERENCES.
% - It formed
%   [1.)] the LARGEST ELLIPTICAL-SHAPED REGION and
%   [2.)] COHERENT SIDELOBES OF APPROXIMATELY CONSTANT ABSOLUTE VALUES.
% - In contrast, the observation processes \eqref{eqn:recovery_sys_lin_eq_v_rx_born_all_f_all_in_mat} induced by
%   the RANDOM WAVES RESEMBLED THAT INDUCED BY THE \ac{GWN} \eqref{eqn:sim_study_params_ref_obs_proc_gwn}.
% - The SIZES OF THE ELLIPTICAL-SHAPED REGIONS DECREASED RELATIVE TO THE \ac{QPW}.
% - The SIDELOBES DIFFUSED AND FLUCTUATED IN THEIR VALUES, similar to a speckle pattern, RESULTING IN MORE UNIFORM DISTRIBUTIONS.
In fact,
% 1.) noise-like artifacts
the noise-like artifacts
(cf. \cref{%
  fig:sim_study_obj_A_sr_1_tpsf_images_rnd_apo_1,%
  fig:sim_study_obj_A_sr_1_tpsf_images_rnd_del_1,%
  fig:sim_study_obj_A_sr_1_tpsf_images_rnd_apo_del_1,%
  fig:sim_study_obj_B_sr_1_tpsf_images_rnd_apo_1,%
  fig:sim_study_obj_B_sr_1_tpsf_images_rnd_del_1,%
  fig:sim_study_obj_B_sr_1_tpsf_images_rnd_apo_del_1%
}) with
\TODO{exceeding -70 dB}
% 2.) nonzero values
more uniform distributions of
the nonzero values, which populated
% 3.) 43.8 to 50.5 % of the FOV
% article:Schiffner2018, Sect. VIII. Results / Sect. VIII-A. Wire Phantom / Sect. VIII-A.2) Point Spread Functions (subsubsec:results_phantom_wire_psf)
% - The absolute values below \SI{-70}{\deci\bel} constituted approximately \SIrange{49.5}{56.2}{\percent} of
%   the \ac{FOV} and those above this threshold, which reached up to \SI{-0.11}{\deci\bel}, formed the remaining \SIrange{43.8}{50.5}{\percent}.
\SIrange{43.8}{50.5}{\percent} of
the \ac{FOV} for
% 4.) wire phantom
the wire phantom
(cf. \cref{fig:sim_study_obj_A_sr_1_tpsf_ecdfs}) and
% 5.) 80 % of the admissible spatial frequencies
% article:Schiffner2018, Sect. VIII. Results / Sect. VIII-B. Tissue-Mimicking Phantom / Sect. VIII-B.2) Transform Point Spread Functions (subsubsec:results_phantom_tissue_tpsf)
% - The absolute values below \SI{-70}{\deci\bel} constituted approximately \SI{20}{\percent} of
%   the admissible spatial frequencies and those above this threshold, which reached up to \SI{-0.47}{\deci\bel}, formed the remaining \SI{80}{\percent}.
\SI{80}{\percent} of
the admissible spatial frequencies for
% 6.) tissue-mimicking phantom
the tissue-mimicking phantom
(cf. \cref{fig:sim_study_obj_B_sr_1_tpsf_ecdfs}), resembled
% 7.) noise-like artifacts
those induced by
% 8.) second reference sensing matrix (GWN)
the structured random sensing matrices
\eqref{eqn:sim_study_params_ref_sens_mat_gwn}
(cf. \cref{%
  fig:sim_study_obj_A_sr_1_tpsf_images_blgwn_1,%
  fig:sim_study_obj_B_sr_1_tpsf_images_blgwn_1%
}).
% b) noise-like artifacts prevented sparse approximations and enabled their removal by the sparsity-promoting lq-minimization method
Unlike
% 1.) coherent sidelobes
the coherent sidelobes and
% 2.) secondary maxima
the secondary maxima induced by
% 3.) quasi-plane wave (QPW)
the \ac{QPW}
(cf. \cref{%
  fig:sim_study_obj_A_sr_1_tpsf_images_qpw_1,%
  fig:sim_study_obj_B_sr_1_tpsf_images_qpw_1%
}), they prevented
% 1.) sparse approximations
sparse approximations and, thus, enabled
% 2.) removal
their removal by
% 3.) sparsity-promoting lq-minimization method
the sparsity-promoting $\ell_{q}$-minimization method
\eqref{eqn:recovery_reg_norm_lq_minimization}.
% c) reductions in the FEHMs indicate improved spatial and spectral resolutions
% article:Schiffner2018, Sect. VIII. Results / Sect. VIII-B. Tissue-Mimicking Phantom / Sect. VIII-B.2) Transform Point Spread Functions (subsubsec:results_phantom_tissue_tpsf)
% - In fact, the maximum normalized differences ranged from
%   \SI{41.17}{\percent} at the fixed spatial frequencies $s \in \{ 1, 9 \}$ to
%   \SI{62.52}{\percent} at the fixed spatial frequencies $s \in \{ 4, 6 \}$.
% article:Schiffner2018, Sect. VIII. Results / Sect. VIII-A. Wire Phantom / Sect. VIII-A.2) Point Spread Functions (subsubsec:results_phantom_wire_psf)
% - The MAXIMUM NORMALIZED DIFFERENCES ranged from
%   \SI{23.53}{\percent} for the superposition of both randomly-apodized and randomly-delayed \acp{QCW} at
%   the ninth fixed position, i.e. $s = 9$, to
%   \SI{73.68}{\percent} for the superposition of randomly-delayed \acp{QCW} at
%   the first fixed position, i.e. $s = 1$.
The reductions in
% 1.) full extents at half maximum (FEHMs)
the \acp{FEHM}, which ranged from
% 1.) 23.53 to 73.68 %
\SIrange{23.53}{73.68}{\percent} for
% 2.) wire phantom
the wire phantom
(cf. \cref{tab:sim_study_obj_A_sr_1_tpsf_fehm}) and from
% 3.) 41.17 to 62.52 %
\SIrange{41.17}{62.52}{\percent} for
% 4.) tissue-mimicking phantom
the tissue-mimicking phantom
(cf. \cref{tab:sim_study_obj_B_sr_1_tpsf_fehm}) with
% 5.) few exceptions
few exceptions, indicate
% 6.) spatial and spectral resolutions
improved spatial and
spectral resolutions.
% d) slightly increased mean FEHM and the deviation in the empirical CDF produced by the superposition of randomly-apodized QCWs indicate potential benefits for the inclusion of random time delays for the wire phantom
% article:Schiffner2018, Sect. VIII. Results / Sect. VIII-A. Wire Phantom / Sect. VIII-A.2) Point Spread Functions (subsubsec:results_phantom_wire_psf)
% - The SUPERPOSITION OF RANDOMLY-APODIZED \acp{QCW} produced
%   the LARGEST SAMPLE MEAN AND SAMPLE STANDARD DEVIATION among the random waves.
The slightly increased mean \ac{FEHM} and
% 1.) deviation
the deviation in
% 2.) empirical CDF
% article:Schiffner2018, Sect. VIII. Results / Sect. VIII-A. Wire Phantom / Sect. VIII-A.2) Point Spread Functions (subsubsec:results_phantom_wire_psf)
% - The SUPERPOSITION OF RANDOMLY-APODIZED \acp{QCW} distributed
%   the latter values [absolute values above -70 dB] over the smallest percentage of the \ac{FOV}.
the empirical \ac{CDF} produced by
% 3.) superposition of randomly-apodized QCWs
the superposition of
randomly-apodized \acp{QCW} for
% 4.) wire phantom
the wire phantom, however, indicate
% 5.) potential benefits
potential benefits provided by
% 6.) random time delays
the random time delays.
