%---------------------------------------------------------------------------------------------------------------
% 1.) significance of the FMM for both auxiliary functions in the proposed matrix-free implementation
%---------------------------------------------------------------------------------------------------------------
% a) numerical evaluations of tens to hundreds of matrix-vector products emphasize the significance of both auxiliary functions in the proposed matrix-free implementation
% article:Schiffner2018, Sect. VI: Implementation / Sect. VI-C: Sparsity-Promoting lq-Minimization Method (subsec:imp_lq_minimization)
% - \ac{SPGL1} is ITERATIVE and LEFT MULTIPLIED A SEQUENCE OF RECURSIVELY-GENERATED VECTORS by
%   the potentially densely-populated normalized sensing matrix \eqref{eqn:recon_reg_norm_sensing_matrix} or
%   its adjoint.
% - Its MATRIX-FREE IMPLEMENTATION interpreted each type of matrix-vector product as a linear map and dedicated
%   a CUSTOMIZED AUXILIARY FUNCTION to its numerical evaluation.
\TODO{relationship between number of iterations and matrix-vector products?}
The numerical evaluations of
% 1.) tens
tens to
% 2.) hundreds
hundreds of
% 3.) matrix-vector products
matrix-vector products involving
% 4.) normalized sensing matrix (all pulse-echo measurements, multifrequent, all array elements)
the normalized sensing matrix
\eqref{eqn:recon_reg_norm_sensing_matrix} or
% 5.) adjoint of the normalized sensing matrix (all pulse-echo measurements, multifrequent, all array elements)
its adjoint by
% 6.) sparsity-promoting lq-minimization method
the sparsity-promoting $\ell_{q}$-minimization method
\eqref{eqn:recovery_reg_norm_lq_minimization}
(cf. \cref{fig:sim_study_obj_A_sr_1_ssim_index_rel_rmse_N_iter_vs_snr_kap,fig:sim_study_obj_B_sr_1_ssim_index_rel_rmse_N_iter_vs_snr_kap}) emphasize
% 7.) significance
the significance of
% 8.) customized auxiliary function for the numerical evaluation of each type of matrix-vector product
both auxiliary functions in
% 9.) proposed matrix-free implementation
the proposed matrix-free implementation.
% b) normalized reductions in the memory consumption and the number of multiplications provided by the FMM confirm its effectiveness in achieving both aims of the auxiliary functions
% article:Schiffner2018, Sect. VIII: Results / Sect. VIII-C: Memory Consumption and Computational Costs (subsec:results_memory_flops)
% - The approximate decomposition of the observation process \eqref{eqn:recovery_sys_lin_eq_v_rx_born_all_f_all_in_mat} by
%   the \ac{FMM} REDUCED THE MEMORY CONSUMPTION, which theoretically amounted to
%   $M_{\text{conv}} = \SI{115}{\gibi\byte}$ for the wire phantom and
%   $M_{\text{conv}} = \SI{112.5}{\gibi\byte}$ for the tissue-mimicking phantom, to
%   $M_{\text{\acs{FMM}}} \approx \SI{2.24}{\percent} M_{\text{conv}} \approx \SI{2.58}{\gibi\byte}$ and
%   $M_{\text{\acs{FMM}}} \approx \SI{2.21}{\percent} M_{\text{conv}} \approx \SI{2.49}{\gibi\byte}$, respectively.
% - It concurrently REDUCED THE NUMBER OF MULTIPLICATIONS, which theoretically amounted to
%   $N_{\text{mul},\text{conv}} \approx \num{7.72e9}$ for the wire phantom and
%   $N_{\text{mul},\text{conv}} \approx \num{7.55e9}$ for the tissue-mimicking phantom, to
%   $N_{\text{mul},\text{\acs{FMM}}} \approx \SI{93.91}{\percent} N_{\text{mul},\text{conv}} \approx \num{7.25e9}$ and
%   $N_{\text{mul},\text{\acs{FMM}}} \approx \SI{92.32}{\percent} N_{\text{mul},\text{conv}} \approx \num{6.97e9}$, respectively.
The normalized reductions in
% 1.) memory consumption
the memory consumption and
% 2.) number of multiplications executed by the associated matrix-vector product
the number of
multiplications achieved by
% 3.) fast multipole method (FMM)
the \ac{FMM}, which amounted up to
% 4.) 97.8 %
\SI{97.8}{\percent} and
% 5.) 7.7 %
\SI{7.7}{\percent},
respectively, confirm
% 6.) effectiveness
its effectiveness in achieving
% 7.) aims
% article:Schiffner2018, Sect. VI: Implementation / Sect. VI-C: Sparsity-Promoting lq-Minimization Method (subsec:imp_lq_minimization)
% - Both functions aimed at
%   [1.)] circumventing the explicit storage of the associated matrix in the fast but limited \ac{RAM} and
%   [2.)] accelerating the numerical computations.
both aims of
% 8.) auxiliary functions
these functions.
% c) modern UI systems typically lack the amounts of fast RAM required for the explicit storage of the observation process
% article:Schiffner2018, Sect. VI: Implementation / Sect. VI-C: Sparsity-Promoting lq-Minimization Method (subsec:imp_lq_minimization)
% - In fact,
%   the memory consumption of the normalized sensing matrix \eqref{eqn:recon_reg_norm_sensing_matrix} and
%   the number of multiplications executed by the associated matrix-vector product pose
%   CHALLENGES FOR MODERN \ac{UI} SYSTEMS.
Although
% 1.) modern UI systems
modern \ac{UI} systems typically lack
% 2.) amounts of fast RAM
the amounts of
fast \ac{RAM} required for
% 3.) explicit storage
the explicit storage of
% 4.) observation process (all pulse-echo measurements, multifrequent, all array elements)
the observation process
\eqref{eqn:recovery_sys_lin_eq_v_rx_born_all_f_all_in_mat},
% d) modern UI systems can readily store its [observation process] approximate decomposition provided by the FMM
they can readily store
% 1.) approximate decomposition of the observation process
its approximate decomposition provided by
% 2.) fast multipole method (FMM)
the \ac{FMM}.
% e) detailed analysis of the FMM for the excitation by steered PWs reported even higher normalized reductions in the memory consumption and the number of multiplications of 99.75 % and 76 %, respectively
% proc:SchiffnerIUS2014: Pulse-Echo Ultrasound Imaging Combining Compressed Sensing and the Fast Multipole Method
% Abstract
% - For an EXAMPLE OF TYPICAL SIZE and in comparison to the conventional approach, we showed that
%   the FMM REQUIRES
%   [1.)] LESS THAN 0.25 % OF THE MEMORY AND
%   [2.)] LESS THAN 24 % OF THE NUMBER OF COMPLEX-VALUED MULTIPLICATIONS. (p. 2205)
A detailed analysis of
% 1.) fast multipole method (FMM)
the \ac{FMM} for
% 2.) excitation
the excitation by
% 3.) steered PWs
steered \acp{PW} reported
% 4.) higher normalized reductions
even higher normalized reductions in
% 5.) memory consumption
the memory consumption and
% 6.) number of multiplications executed by the associated matrix-vector product
the number of
multiplications of
$\SI{99.75}{\percent}$ and
$\SI{76}{\percent}$,
respectively
\cite{proc:SchiffnerIUS2014}.
% f) author hypothesizes that further optimizations enable even higher reductions
% article:Schiffner2018, Sect. VI: Implementation / Sect. VI-D: Fast Multipole Method for the Observation Process (subsec:imp_fmm_obs_process)
% - It [FMM] substituted the OUTGOING FREE-SPACE \name{Green}'s FUNCTIONS \eqref{eqn:app_helmholtz_green_free_space_2_3_dim} in
%   the entries of the observation process \eqref{eqn:recovery_sys_lin_eq_v_rx_born_coef} by
%   ERROR-REGULATED TRUNCATED MULTIPOLE EXPANSIONS if the grid points
%   $\vect{r}_{\text{lat}, i} \in \mathcal{L}$ and $\vect{r}_{\text{mat}, \nu}^{(m)} \in \mathcal{V}_{m}$, satisfied
%   a specific geometric relationship \cite[Chapt. 9]{book:Gibson2014}, \cite{article:CoifmanIAPM1993}.
The author hypothesizes that
% 1.) additional optimizations
additional optimizations in
% 2.) discretization
the discretization of
% 3.) truncated multipole expansions
the truncated multipole expansions enable
% 4.) further improvements
further improvements.

%---------------------------------------------------------------------------------------------------------------
% 2.) advantage of the FMM over algebraic methods for the sparse approximation of the observation process
%---------------------------------------------------------------------------------------------------------------
% a) availability of the FMM results from the usage of the outgoing free-space Green's functions in the proposed physical models
% article:Ghanbarzadeh-DagheyanSensors2018: Holey-Cavity-Based Compressive Sensing for Ultrasound Imaging [Apr. 2018]
% article:BerthonPMB2018: Spatiotemporal matrix image formation for programmable ultrasound scanners [Feb. 2018]
% Abstract:
% - In this work, we argue that as the computational power keeps increasing, it is becoming practical to
%   DIRECTLY IMPLEMENT AN APPROXIMATION TO THE MATRIX OPERATOR LINKING
%   [1.)] REFLECTOR POINT TARGETS to
%   [2.)] THE CORRESPONDING RADIOFREQUENCY SIGNALS via
%   [3.)] THOROUGHLY VALIDATED AND WIDELY AVAILABLE SIMULATIONS SOFTWARE. (p. 1)
% 1. Introduction
% - Specifically, WE PROPOSE AN IMAGE FORMATION FRAMEWORK based on
%   the EXPLICIT CONSTRUCTION OF THE FORWARD PROBLEM SPATIOTEMPORAL MATRIX, which
%   consists of a DICTIONARY of pixel-specific transmit-receive signals in the form of reference emitted and received signals defined for each pixel. (p. 2)
% - These can be INVERSED to form images via simple, highly optimized and widely available matrix-vector product algorithms. (p. 2)
% - This DICTIONARY CAN INCLUDE ANY AMOUNT OF A PRIORI INFORMATION obtained by leveraging
%   EXISTING, OPEN-SOURCE OR FREEWARE SIMULATION SOFTWARE (for example Matlab-based software such as K-wave or Field II) or
%   EXPERIMENTAL MEASUREMENTS, and thus streamlines the inclusion of all the available a priori knowledge in the image formation process. (p. 2)
The availability of
% 1.) fast multipole method (FMM)
the \ac{FMM}, which results from
% 2.) usage
the usage of
% 3.) outgoing free-space Green's functions (two- and three-dimensional Euclidean spaces)
the outgoing free-space \name{Green}'s functions
\eqref{eqn:app_helmholtz_green_free_space_2_3_dim}, is
% 4.) crucial advantage
a crucial advantage of
% 5.) analytical derivation
the analytical derivation of
% 6.) observation process (all pulse-echo measurements, multifrequent, all array elements)
the observation process
\eqref{eqn:recovery_sys_lin_eq_v_rx_born_all_f_all_in_mat} over
% 7.) numerical constructions
the numerical constructions using
% 8.) free or commercial simulation software
simulation software or
% 9.) experimental measurements
experimental measurements proposed in
\cite{article:Ghanbarzadeh-DagheyanSensors2018,article:BerthonPMB2018}.
% b) FMM provides explicit analytical expressions for the sparse approximation of the observation process
The \ac{FMM} provides
% 1.) explicit analytical expressions
explicit analytical expressions for
% 2.) sparse approximation
the sparse approximation of
% 3.) observation process (all pulse-echo measurements, multifrequent, all array elements)
the observation process
\eqref{eqn:recovery_sys_lin_eq_v_rx_born_all_f_all_in_mat}, whereas
% c) algebraic methods rely on the numerical values of its entries or its action on a suitable vector
algebraic methods, e.g.
% 1.) singular value decomposition
% article:ChaillatJCP2012: FaIMS: A fast algorithm for the inverse medium problem with multiple frequencies and multiple sources for the scalar Helmholtz equation
% - algorithm to compute an approximate singular value decomposition (SVD) of M or least-squares operators (approximate SVD of the Born operator)
% - can be used to accelerate matrix-vector multiplications and to precondition iterative solvers
% - we consider small perturbations of the background medium and, by invoking the Born approximation, we obtain a linear least-squares problem
% - Finally one could form M and use a dense factorization algorithm, say, use an classical SVD factorization [15].
%   A dense SVD is prohibitively expensive because its work complexity is OðminðNsNxNd;NÞ? ?maxðNsNxNd;NÞÞ.
% - compute approximate SVDs of small submatrices by applying the randomized SVD
%	-> each submatrix is approximated by a low-rank matrix
% - recursive SVD to combine the approximate SVDs of the submatrices
% - complexity is orders of magnitude smaller than the standard SVD factorization
% - matrix-free
%,SchotlandJOSA2001
% \cite[Chapt. 9]{book:Gibson2014}: LU factorization (exponential in the number of unknowns) -> non-iterative
the \acl{SVD}
\cite{book:Hansen2010,book:Hansen1998} and
its enhancements
\cite{article:ChaillatJCP2012},
% 2.) QR-factorization: -> see Chaillat
% article:ChewITAP1997: Fast solution methods in electromagnetics
% III. INTEGRAL EQUATION SOLVERS
% - If the matrix equation is then solved by
%   LU decomposition (Gaussian elimination) or
%   alternatively by an iterative technique such as the CG or related methods [19], [20],
%   the computational labor may be excessive. (p. 535)
%the QR decomposition,
% 2.) adaptive cross approximation (ACA)
% article:Hesford2010: The fast multipole method and Fourier convolution for the solution of acoustic scattering on regular volumetric grids
% - Like methods such as the ADAPTIVE CROSS APPROXIMATION [15–17], efficient solutions are obtained by eliminating approximately redundant information from
%   interactions between sufficiently separated groups.
% - The ADAPTIVE CROSS APPROXIMATION uses an ALGEBRAIC METHOD to construct
%   PRODUCTS OF LOW-RANK MATRICES that cast all pairwise interactions in terms of fewer, dominant interactions.
% article:ZhaoITEC2005: The adaptive cross approximation algorithm for accelerated method of moments computations of EMC problems
the \acl{ACA}
\cite{article:ZhaoITEC2005}, or
% 3.) various transforms
% article:DemanetFCompMath2010: Scattering in Flatland: Efficient Representations via Wave Atoms
% - lossy numerical compression strategy for the boundary integral equation of acoustic scattering in two dimensions
% - system of equations has oscillatory kernels that are represented in a basis of wave atoms, and compressed by thresholding the small coefficients to zero
% - Numerical experiments support the estimate and show that this wave atom representation may be of interest for applications where the same scattering problem needs to
%   be solved for many boundary conditions, for example, the computation of radar cross sections.
% - efficient representation of the operator as a sparse matrix in a system of wave atoms
% - Most of the approaches on sparsifying (1) in well-chosen bases require the construction of the full integral operator
%	-> computational difficulty for large k values
% article:DemanetACHA2007: Wave atoms and sparsity of oscillatory patterns
% article:ChewITAP1997: Fast solution methods in electromagnetics
% III. INTEGRAL EQUATION SOLVERS
% - WAVELET TRANSFORMS [56]–[61] have also been used to yield SPARSE MATRICES that can be solved rapidly. (p. 535)
various transforms
\cite{article:DemanetFCompMath2010,article:DemanetACHA2007,article:ChewITAP1997}, rely on
% 4.) numerical values
the numerical values of
% 5.) entries of the observation process (single pulse-echo measurement, monofrequent, single array element)
its entries
\eqref{eqn:recovery_sys_lin_eq_v_rx_born_coef} or
% 6.) action
its action on
% 7.) suitable vector
a suitable vector.
% d) constancy of the observation process in multiple instances of the normalized CS problem reduces the relevance of the computational overhead
The constancy of
% 1.) observation process (all pulse-echo measurements, multifrequent, all array elements)
the observation process
\eqref{eqn:recovery_sys_lin_eq_v_rx_born_all_f_all_in_mat} in
% 2.) multiple instances
multiple instances of
% 3.) normalized CS problem
the normalized \ac{CS} problem
\eqref{eqn:recovery_reg_norm_prob_general}, which frequently occurs in
% 4.) practice
practice, reduces
% 5.) relevance
the relevance of
% 6.) computational overhead
the increased complexity.

%---------------------------------------------------------------------------------------------------------------
% 3.) simple alternative to the FMM
%---------------------------------------------------------------------------------------------------------------
% a) simple alternative to the FMM exploits the observation that the numerical evaluations of the matrix-vector products do not require the simultaneous storage of all matrix entries
%A simple alternative to
%the \ac{FMM} exploits
%the observation that
%the numerical evaluations of
%the matrix-vector products involving
% 1.) observation process (all pulse-echo measurements, multifrequent, all array elements)
%the observation process
%\eqref{eqn:recovery_sys_lin_eq_v_rx_born_all_f_all_in_mat} or
% 2.) adjoint of the observation process (all pulse-echo measurements, multifrequent, all array elements)
%its adjoint do not require
%the simultaneous storage of
%all matrix entries.
% b) specified groups of entries may be computed on demand during each numerical evaluation of a matrix-vector product and discarded afterwards
%Instead,
%specified groups of
%entries may be computed on demand during
%each numerical evaluation of
%a matrix-vector product and
%discarded afterwards.
% c) redundant computations of the matrix entries increase the computational costs
%Although
%the redundant computations of
%the matrix entries increase
%the computational costs,
% d) parallel execution by specialized hardware
%their parallel execution by
%specialized hardware, e.g.
%\acp{GPU}
%\cite{proc:SchiffnerIUS2013a,proc:SchiffnerIUS2013b,proc:SchiffnerIUS2012,article:SchiffnerBMT2012,proc:SchiffnerIUS2011}, is
%sufficiently efficient.
