%---------------------------------------------------------------------------------------------------------------
% 1.) explanation of the substantial relative RMSEs for the low reference SNRs
%---------------------------------------------------------------------------------------------------------------
% a) substantial relative RMSEs at the low reference SNRs revealed an undesired sensitivity to the energy of the additive errors
% article:Schiffner2018, Sect. VIII. Results / Sect. VIII-A. Wire Phantom / Sect. VIII-A.4) Recovery by lq-Minimization (subsubsec:results_phantom_wire_lq_minimization)
% - The mean SSIM indices confirmed the excellent structural recovery, whereas
%   the RELATIVE \acp{RMSE} REVEALED AN INCREASED SENSITIVITY OF
%   THE QUANTITATIVE RECOVERY BASED ON THE RANDOM WAVES TO THE ENERGY OF THE ADDITIVE ERRORS
%   (cf. \cref{fig:sim_study_obj_A_sr_1_ssim_index_rel_rmse_N_iter_vs_snr_kap}).
% - The nonconvex $\ell_{0.5}$-minimization method \eqreflqmin{eqn:recovery_reg_norm_lq_minimization}{ 0.5 } consistently improved both
%   the mean \ac{SSIM} indices and the relative \acp{RMSE} for all investigated reference \acp{SNR}.
% - The RANDOM WAVES, however, CAUSED SIGNIFICANTLY HIGHER RELATIVE \acp{RMSE} THAN THE \ac{QPW} FOR
%   THE LOW REFERENCE \acp{SNR} of $\text{SNR}_{\text{dB}} \in \setsymbol{Q} = \{ \SI{3}{\deci\bel}, \SI{6}{\deci\bel}, \SI{10}{\deci\bel} \}$.
% explained
% 4.) significant relative RMSEs
%the significant relative \acp{RMSE} for
% 5.) random waves
%the random waves and
% 6.) nonconvex l0.5-minimization method
%the nonconvex $\ell_{0.5}$-minimization method
%\eqreflqmin{eqn:recovery_reg_norm_lq_minimization}{ 0.5 } at
% 7.) low reference SNRs
%the low reference \acp{SNR}.
% TODO: sensitivity emphasized by std. dev. in rel. RMSEs
For
% 1.) wire phantom
the wire phantom,
% 2.) substantial relative RMSEs
the substantial relative \acp{RMSE} at
% 3.) low reference SNRs of 3, 6, and 10 dB
the low reference \acp{SNR}
(cf. \cref{%
  fig:sim_study_obj_A_sr_1_quality_vs_snr_spgl1_lq_rnd_apo,%
  fig:sim_study_obj_A_sr_1_quality_vs_snr_spgl1_lq_rnd_del,%
  fig:sim_study_obj_A_sr_1_quality_vs_snr_spgl1_lq_rnd_apo_del%
}) revealed
% 4.) undesired sensitivity
an undesired sensitivity of
% 5.) estimated vector stacking the regular samples in the discretized relative spatial fluctuations in the unperturbed compressibility
the compressibility fluctuations
\eqref{eqn:recovery_reg_norm_lq_minimization_sol_mat_params} recovered by
% 6.) random waves
the random waves to
% 7.) energy
the energy of
% 8.) additive errors
the additive errors.
% b) strong variations in the incident acoustic energies at these positions and the resulting variations in the sample means of the relative RMSEs suggest that [...]
The variations in
% 1.) incident acoustic energy at a specified grid point (all pulse-echo measurements, multifrequent)
the incident acoustic energies
\eqref{eqn:recovery_p_in_energy} across
% 2.) isolated positions
the isolated positions of
% 3.) wires
the wires and
% 4.) variations
\TODO{chance of low energy and bad SNR}
% TODO: variations in the SNR of the Recorded electric energies
the resulting variations in
% 5.) mean relative RMSEs
the mean relative \acp{RMSE}
(cf. \cref{%
  fig:sim_study_obj_A_p_in_energy,%
  fig:sim_study_obj_A_sr_1_mean_rel_errors%
}) suggest that
% 6.) relocation of the wires to positions of high incident acoustic energies
(i) the energy-guided relocation of
the wires,
% 7.) different realizations of the random waves
(ii) different realizations of
the random waves, or
% 8.) multiple sequential pulse-echo measurements
(iii) multiple sequential pulse-echo measurements overcome
% 9.) sensitivity
this sensitivity.
% c) fixed experimental setup excludes option (i), and only option (iii) ensures consistent results for various configurations
A fixed experimental setup, however, excludes
% 1.) option (i)
option (i), and
% 2.) option (iii)
only option (iii) ensures
% 3.) consistent results
consistent results for
% 4.) various configurations
various configurations.
% d) variations further suggest that spatially extended structural building blocks reduce this sensitivity
The variations further suggest that
% 1.) spatially extended structural building blocks
(iv) spatially extended structural building blocks reduce
% 2.) sensitivity
this sensitivity.

%---------------------------------------------------------------------------------------------------------------
% 2.) enlarged passbands and increased robustness against the unknown additive errors
%---------------------------------------------------------------------------------------------------------------
% a) negligible relative RMSEs at the low reference SNRs indicate the robustness of the compressibility fluctuations recovered by the random waves against the additive errors
% article:Schiffner2018, Sect. VIII. Results / Sect. VIII-B. Tissue-Mimicking Phantom / Sect. VIII-B.4) Recovery by lq-Minimization (subsubsec:results_phantom_tissue_lq_minimization)
% - Using the nonconvex $\ell_{0.5}$-minimization method \eqreflqmin{eqn:recovery_reg_norm_lq_minimization}{ 0.5 },
%   the RANDOM WAVES consistently achieved
%   [1.)] MEAN \ac{SSIM} INDICES CLOSE TO UNITY and
%   [2.)] RELATIVE \acp{RMSE} BELOW \SI{6.86}{\percent} for all investigated reference \acp{SNR}.
% - The sample means of the normalized numbers of iterations increased from
%   at least \SI{27.19}{\percent} at the lowest reference \ac{SNR} to
%   at most \SI{82.13}{\percent} at the highest reference \ac{SNR} for
%   the superposition of randomly-apodized \acp{QCW}.
% - In contrast, the \ac{QPW} PRODUCED ONLY SLIGHTLY BETTER MEAN \ac{SSIM} INDICES and RELATIVE \acp{RMSE} than for
%   the convex $\ell_{1}$-minimization method \eqreflqmin{eqn:recovery_reg_norm_lq_minimization}{ 1 }.
Indeed, for
% 1.) tissue-mimicking phantom
the tissue-mimicking phantom,
% 2.) negligible relative RMSEs
the negligible relative \acp{RMSE} at
% 3.) low reference SNRs of 3, 6, and 10 dB
the low reference \acp{SNR}
(cf. \cref{%
  fig:sim_study_obj_B_sr_1_quality_vs_snr_spgl1_lq_rnd_apo,%
  fig:sim_study_obj_B_sr_1_quality_vs_snr_spgl1_lq_rnd_del,%
  fig:sim_study_obj_B_sr_1_quality_vs_snr_spgl1_lq_rnd_apo_del%
}) indicate
% 4.) robustness
the robustness of
% 5.) recovered relative spatial fluctuations in the unperturbed compressibility
the compressibility fluctuations
\eqref{eqn:recovery_reg_norm_lq_minimization_sol_mat_params} recovered by
% 6.) random waves
the random waves against
% 7.) additive errors
the additive errors.
% b) complex exponential functions scattered the incident acoustic pressure fields at all grid points in the FOV and completely reradiated their spatial variations
In contrast to
% 1.) punctiform wires
the wires,
% 2.) spatially extended complex exponential functions
the spatially extended complex exponential functions scattered
% 3.) discretized incident acoustic pressure fields [superpositions of quasi-(d-1)-spherical waves]
the incident acoustic pressure fields
\eqref{eqn:recovery_p_in} at
% 4.) all grid points in the FOV
all grid points in
the \ac{FOV} and, thus, completely reradiated
% 5.) spatial variations
their spatial variations.
% c) significantly enlarged passbands of the sensing matrices induced by the random waves prove that their scattered waves interfere less destructively on the faces of the array elements than those caused by the QPW
% article:Schiffner2018, Sect. VIII. Results / Sect. VIII-B. Tissue-Mimicking Phantom / Sect. VIII-B.1) Recorded Electric Energies (subsubsec:results_phantom_tissue_energy_rx)
% - The TRANSFER BEHAVIORS OF THE SENSING MATRICES \eqref{eqn:recovery_reg_sensing_matrix} induced by all incident waves resembled those of
%   BANDPASS FILTERS SUPPRESSING RELATIVELY LOW AND HIGH SPATIAL FREQUENCIES (cf. \cref{fig:sim_study_obj_B_norms_kappa}).
% - The \ac{QPW} induced relatively large electric energies \eqref{eqn:recovery_reg_v_rx_born_trans_coef_energy} exceeding \SI{-20}{\deci\bel} in
%   a SICKLE-SHAPED PASSBAND inside the interval of normalized spatial frequencies
%   $\hat{\vect{K}} \in [ \num{-0.24}; \num{0.24} ] \times [ \num{0.15}; \num{0.49} ]$, whereas
%   the RANDOM WAVES induced those in ARBELOS-SHAPED PASSBANDS inside the intervals of normalized spatial frequencies
%   $\hat{\vect{K}} \in [ \num{-0.43}; \num{0.43} ] \times [ \num{0}; \num{0.49} ]$.
% - All formed passbands, which were SIGNIFICANTLY ENLARGED BY THE LATTER WAVES, strongly agreed with
%   the predictions of the \ac{FDT} (cf. footnote $\num{1}$ in \cref{sec:introduction}).
The significantly enlarged passbands of
% 1.) sensing matrices (all pulse-echo measurements, multifrequent, all array elements)
the sensing matrices
\eqref{eqn:recovery_reg_sensing_matrix} induced by
% 2.) random waves
the random waves
(cf. \cref{%
  fig:sim_study_obj_B_norms_kappa_rnd_apo,%
  fig:sim_study_obj_B_norms_kappa_rnd_del,%
  fig:sim_study_obj_B_norms_kappa_rnd_apo_del%
}) prove that
% 3.) scattered waves
their scattered waves interfere
% 4.) less destructively
less destructively on
% 5.) faces of the array elements
the faces of
the array elements than
% 5.) scattered waves
those caused by
% 6.) quasi-plane wave (QPW)
the \ac{QPW}.
% d) significantly enlarged passbands explain both the reductions of the mean FEHMs relative to the QPW for the wire phantom and the robustness against the additive errors
In addition to
% 1.) robustness against the additive errors
the robustness, they explain
% 2.) reductions of the mean FEHMs relative to the QPW
the reductions of
% 3.) mean FEHMs
the mean \acp{FEHM} relative to
% 4.) quasi-plane wave (QPW)
the \ac{QPW} for
% 5.) wire phantom
the wire phantom
(cf. \cref{tab:sim_study_obj_A_sr_1_tpsf_fehm}).
% e) strong agreement with the predictions of the FDT further indicates the correctness of the numerical simulations
% article:Schiffner2018, Sect. VIII. Results / Sect. VIII-B. Tissue-Mimicking Phantom / Sect. VIII-B.1) Recorded Electric Energies (subsubsec:results_phantom_tissue_energy_rx)
% - All formed passbands, [...], STRONGLY AGREED WITH THE PREDICTIONS OF THE \ac{FDT} (cf. footnote $\num{1}$ in \cref{sec:introduction}).
Their strong agreement with
% 1.) predictions
the predictions of
% 2.) FDT
the \ac{FDT} further indicates
% 3.) correctness
the correctness of
% 4.) numerical simulations
the numerical simulations.
