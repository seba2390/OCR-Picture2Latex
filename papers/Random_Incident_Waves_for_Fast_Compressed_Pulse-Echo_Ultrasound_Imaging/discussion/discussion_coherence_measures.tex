%%%%%%%%%%%%%%%%%%%%%%%%%%%%%%%%%%%%%%%%%%%%%%%%%%%%%%%%%%%%%%%%%%%%%%%%%%%%%%%%%%%%%%%%%%%%%%%%%%%%%%%%%%%%%%%%
% table: upper bounds on the numbers of nonzero components and underlying worst-case coherences
%%%%%%%%%%%%%%%%%%%%%%%%%%%%%%%%%%%%%%%%%%%%%%%%%%%%%%%%%%%%%%%%%%%%%%%%%%%%%%%%%%%%%%%%%%%%%%%%%%%%%%%%%%%%%%%%
\begin{table*}[tb]
 \centering
 \caption{%
  % a) table summarizes the upper bounds on the numbers of nonzero components
  Upper bounds on
  % 1.) numbers of nonzero components in the sparse representations
  the numbers of
  nonzero components
  \eqref{eqn:disc_coherence_sparsity_ub} for
  % 2.) reference sensing matrices
  both reference sensing matrices and
  % 3.) sensing matrices (all pulse-echo measurements, multifrequent, all array elements)
  the sensing matrices
  \eqref{eqn:recovery_reg_sensing_matrix} induced by
  % 4.) all incident waves
  all incident waves.
  % b) best possible values arose from the Welch lower bounds on the worst-case coherences
  Their best possible values arose from
  % 1.) Welch lower bounds on the worst-case coherences of the sensing matrices
  the \name{Welch} lower bounds on
  the worst-case coherences
  \eqref{eqn:disc_coherence_lb_welch}, whereas
  % c) remaining values arose from the lower bounds on the worst-case coherences of the sensing matrices
  their remaining values arose from
  % 1.) lower bounds on the worst-case coherences of the sensing matrices
  the lower bounds on
  the worst-case coherences
  \eqref{eqn:disc_coherence_lb} provided by
  the arguments of
  the maxima in
  the empirical \acp{CDF} of
  the \acp{TPSF}
  \eqref{eqn:cs_math_tpsf}.
 }
 \label{tab:results_sr_1_tpsf_coherence}
 \begin{tabular}{%
  @{}%
  l%										01.) object
  S[table-format = 2,table-number-alignment = right,table-auto-round]%		02.) Welch lower bound
  @{ }>{[}l<{]}%                                                            	03.) Welch lower bound
  S[table-format = 1,table-number-alignment = right,table-auto-round]%		04.) RIP
  @{ }>{[}l<{]}%                                                            	05.) RIP
  S[table-format = 1,table-number-alignment = right,table-auto-round]%		06.) GWN
  @{ }>{[}l<{]}%                                                            	07.) GWN
  S[table-format = 1,table-number-alignment = right,table-auto-round]%		08.) QPW
  @{ }>{[}l<{]}%                                                            	09.) QPW
  S[table-format = 1,table-number-alignment = right,table-auto-round]%		10.) rnd. apo.
  @{ }>{[}l<{]}%                                                            	11.) rnd. apo.
  S[table-format = 1,table-number-alignment = right,table-auto-round]%		12.) rnd. del.
  @{ }>{[}l<{]}%                                                            	13.) rnd. del.
  S[table-format = 1,table-number-alignment = right,table-auto-round]%		14.) rnd. apo. del.
  @{ }>{[}l<{]}%                                                            	15.) rnd. apo. del.
  @{}%
 }
 \toprule
  \multicolumn{1}{@{}H}{\multirow{2}{*}{Object}} &
  \multicolumn{14}{H@{}}{Upper bound on the number of nonzero components (1) [lower bound on the worst-case coherence (\si{\percent})]}\\
  \cmidrule(lr){2-15}
  &
  \multicolumn{2}{H}{\name{Welch} lower bound} &
  \multicolumn{2}{H}{\acs{RIP}} &
  \multicolumn{2}{H}{\acs{GWN}} &
  \multicolumn{2}{H}{\acs{QPW}} &
  \multicolumn{2}{H}{Rnd. apo.} &
  \multicolumn{2}{H}{Rnd. del.} &
  \multicolumn{2}{H@{}}{Rnd. apo. del.}\\
  \cmidrule(r){1-1}\cmidrule(lr){2-3}\cmidrule(lr){4-5}\cmidrule(lr){6-7}\cmidrule(lr){8-9}
  \cmidrule(lr){10-11}\cmidrule(lr){12-13}\cmidrule(l){14-15}
 \addlinespace
  \ExpandableInput{results/object_A/kappa_only/sr_tpsf/tables/sim_study_obj_A_v2_sr_1_tpsf_sparsity_ub_coherence_lb.tex}
  \ExpandableInput{results/object_B/kappa_only/sr_tpsf/tables/sim_study_obj_B_v2_sr_1_tpsf_sparsity_ub_coherence_lb.tex}
 \addlinespace
 \bottomrule
 \end{tabular}
\end{table*}

%---------------------------------------------------------------------------------------------------------------
% 1.) upper bounds on the numbers of nonzero components and underlying worst-case coherences
%---------------------------------------------------------------------------------------------------------------
% a) arguments of the maxima in the empirical CDFs of the TPSFs bounded from below the worst-case coherences of the sensing matrices
The arguments of
the maxima in
the empirical \acp{CDF} of
the \acp{TPSF}
\eqref{eqn:cs_math_tpsf}
(cf. \cref{%
  fig:sim_study_obj_A_sr_1_tpsf_ecdfs,%
  fig:sim_study_obj_B_sr_1_tpsf_ecdfs%
}) bounded from
below
% 1.) worst-case coherences of the sensing matrices (all pulse-echo measurements, multifrequent, all array elements)
% book:Foucart2013, Chapter 5: Coherence / Sect. 5.1: Definitions and Basic Properties
% - Definition 5.1.
%   Let \mat{A} \in \C^{ m \times N } be a matrix with l2-NORMALIZED COLUMNS \vect{a}_{1}, ..., \vect{a}_{N}, i.e.,
%   \norm{ \vect{a}_{i} }{2} = 1 for all i \in \setcons{N}.
%   The coherence \mu = \mu( \mat{A} ) of the matrix \mat{A} is defined as
%   [ \mu := \underset{ 1 \leq i \neq j \leq N }{ \max } \abs{ \inprod{ \vect{a}_{i} }{ \vect{a}_{j} } } ]. (5.1) (p. 111)
the worst-case coherences of
the sensing matrices
\eqref{eqn:recovery_reg_sensing_matrix}, which are defined as
\cite[Def. 5.1]{book:Foucart2013}
\begin{equation}
 %--------------------------------------------------------------------------------------------------------------
 % worst-case coherences of the sensing matrices (all pulse-echo measurements, multifrequent, all array elements)
 %--------------------------------------------------------------------------------------------------------------
  \coher\bigl( \mat{A}\bigl[ p^{(\text{in})} \bigr] \bigr)
  =
  \underset{ n_{1} \neq n_{2} }{ \max }
  \Bigl\{
    \dabs{ \dtpsf{ \mat{A}\bigl[ p^{(\text{in})} \bigr] }{1}( n_{1}, n_{2} ) }{1}
  \Bigr\}
 \label{eqn:disc_coherence}
\end{equation}
and obey
% 2.) Welch lower bounds on the worst-case coherences of the sensing matrices
% book:Foucart2013, Chapter 5: Coherence / Sect. 5.
% - Unsurprisingly, a system of l2-normalized vectors is called
%   an EQUIANGULAR TIGHT FRAME if it is both an EQUIANGULAR SYSTEM AND A TIGHT FRAME. (p. 114)
% - Such systems are the ones achieving the lower bound given below and known as the WELCH BOUND. (p. 114)
% - Theorem 5.7.
%   The coherence of a matrix \mat{A} \in \K^{ m \times N } with l2-NORMALIZED COLUMNS satisfies
%   [ mu \geq \sqrt{ \frac{ N - m }{ m ( N - 1 ) } } ]. (5.4)
%   Equality holds if and only if the columns \vect{a}_{1}, ..., \vect{a}_{N} of the matrix \mat{A} form
%   an EQUIANGULAR TIGHT FRAME. (p. 114)
the \name{Welch} lower bounds
\cite[Thm. 5.7]{book:Foucart2013}
\begin{equation}
 %--------------------------------------------------------------------------------------------------------------
 % Welch lower bounds on the worst-case coherences of the sensing matrices
 %--------------------------------------------------------------------------------------------------------------
  \coherlbwelch\bigl( \mat{A}\bigl[ p^{(\text{in})} \bigr] \bigr)
  =
  \sqrt{ \frac{ N_{\text{lat}} - N_{\text{obs}} }{ N_{\text{obs}} ( N_{\text{lat}} - 1 ) } },
 \label{eqn:disc_coherence_lb_welch}
\end{equation}
according to
the inequality
% 3.) lower bounds on the worst-case coherences of the sensing matrices
\begin{equation}
 %--------------------------------------------------------------------------------------------------------------
 % lower bounds on the worst-case coherences of the sensing matrices
 %--------------------------------------------------------------------------------------------------------------
  \coherlbwelch\bigl( \mat{A}\bigl[ p^{(\text{in})} \bigr] \bigr)
  \leq
  \coherlb\bigl( \mat{A}\bigl[ p^{(\text{in})} \bigr] \bigr)
  \leq
  \coher\bigl( \mat{A}\bigl[ p^{(\text{in})} \bigr] \bigr).
 \label{eqn:disc_coherence_lb}
\end{equation}
% b) inequality for the loose upper bound on the RIC provided by the worst-case coherence
% article:Schiffner2018, Sect. II: Compressed Sensing (sec:compressed_sensing)
% - In contrast, the evaluation of a coarser characteristic measure named the worst-case coherence \cite[Def. 5.1]{book:Foucart2013} is relatively simple.
% - It provides a LOOSE UPPER BOUND ON THE RIC \cite[Prop. 6.2]{book:Foucart2013} and, by
%   its \name{Welch} lower bound \cite[Thm. 5.7]{book:Foucart2013}, \cite[Lem. 3.7]{article:KutyniokGAMM2013}, ensures
%   the \ac{RIP} for a given number of significant components $s$ in the nearly-sparse representation \eqref{eqn:def_transform_coefficients}, if
%   the number of observations meets $M \in \bigomega{ s^{2} }$.
% book:Foucart2013, Chapter 6: Restricted Isometry Property / Sect. 6.1 Definitions and Basic Properties
% - As with the coherence, small restricted isometry constants are desired. (p. 133)
% - Proposition 6.2.:
%	If the matrix A HAS l2-NORMALIZED COLUMNS a_{1}, ..., a_{N}, i.e., \norm{ a_{j} }{2} = 1 for all j ∈ [N], then
%	\delta_{1} = 0, \delta_{2} = \mu, \delta_{s} \leq \mu_{1} (s - 1) \leq \mu (s - 1), s \geq 2. (p. 134)
% book:Foucart2013, Chapter 6: Restricted Isometry Property
% - THE COHERENCE IS A SIMPLE AND USEFUL MEASURE OF THE QUALITY OF A MEASUREMENT MATRIX.
%   However, THE LOWER BOUND ON THE COHERENCE in Theorem 5.7 limits
%   the performance analysis of recovery algorithms to RATHER SMALL SPARSITY LEVELS. (p. 133)
% - A finer measure of the quality of a measurement matrix is needed to overcome this limitation. (p. 133)
% - This is provided by the concept of restricted isometry property, also known as uniform uncertainty principle. (p. 133)
These measures, in turn, loosely bounded from above
% 1.) restricted isometry constants (RICs)
the \acp{RIC} of
% 2.) normalized sensing matrices (all pulse-echo measurements, multifrequent, all array elements)
the normalized sensing matrices
\eqref{eqn:recon_reg_norm_sensing_matrix} for
% 3.) 2s-sparse representations
$2s$ nonzero components
\cite[Prop. 6.2]{book:Foucart2013}, i.e.
\begin{equation}
 %--------------------------------------------------------------------------------------------------------------
 % loose upper bounds on the RICs provided by the worst-case coherences
 %--------------------------------------------------------------------------------------------------------------
  \resisoconst{ 2s }\bigl( \bar{\mat{A}}_{\xi}\bigl[ p^{(\text{in})} \bigr] \bigr)
  \leq
  ( 2s - 1 )
  \coher\bigl( \mat{A}\bigl[ p^{(\text{in})} \bigr] \bigr)
  <
  \resisoconstub{ 2s }
 \label{eqn:disc_coherence_resisoconst_ub}
\end{equation}
for
% 1.) all factors for the normalization of the sensing matrices
all factors
$\xi \leq \underset{ i \in \setcons{ N_{\text{lat}} } }{ \min }\{ \tnorm{ \vect{a}_{i}[ p^{(\text{in})} ] }{2} \} / \underset{ i \in \setcons{ N_{\text{lat}} } }{ \max }\{ \tnorm{ \vect{a}_{i}[ p^{(\text{in})} ] }{2} \}$, where
% 2.) specified upper bound on the RIC
% article:Schiffner2018, Sect. II: Compressed Sensing (sec:compressed_sensing)
% - Multiple sufficient conditions on the sensing matrix \eqref{eqn:cs_math_prob_general_sensing_matrix} ensure
%   the stable recovery of the nearly-sparse representation \eqref{eqn:def_transform_coefficients} in
%   the \ac{CS} problem \eqref{eqn:cs_math_prob_general} by the sparsity-promoting $\ell_{q}$-minimization method \eqref{eqn:cs_lq_minimization}.
% - These conditions impose specific upper bounds on various characteristic measures quantifying the suitability of
%   the sensing matrix \eqref{eqn:cs_math_prob_general_sensing_matrix}, e.g.
%   [1.)] the null space constants \cite[Def. 4.21]{book:Foucart2013}, \cite[Def. 1.2]{book:Eldar2012},
%   [2.)] the restricted isometry ratio \cite{article:FoucartACHA2009}, and
%   [3.)] the \ac{RIC} \cite{article:FoucartACHA2010,article:CandesCRAS2008,article:CandesSPM2008}.
$\resisoconstub{ 2s }$ denotes
the specific upper bound imposed by
a suitable sufficient condition ensuring
the stable recovery.
% c) complete relaxation of this upper bound and the insertions of the lower bounds on the worst-case coherences yield the upper bounds on the numbers of nonzero components in the sparse representations
% book:Foucart2013, Chapter 6: Restricted Isometry Property / Sect. 6.1: Definitions and Basic Properties
% - We make a FEW REMARKS before establishing the equivalence of these two definitions. (p. 134)
% - The second one is that, although \resisoconst{ s } ≥ 1 is not forbidden,
%   THE RELEVANT SITUATION OCCURS FOR \resisoconst{ s } < 1. (p. 134)
% - Indeed, (6.2) says that EACH COLUMN SUBMATRIX \mat{A}_{S}, S \subset \setcons{N} with \card{S} \leq s,
%   HAS ALL ITS SINGULAR VALUES IN THE INTERVAL [ 1 − \resisoconst{ s }, 1 + \resisoconst{ s } ] and
%   IS THEREFORE INJECTIVE WHEN \resisoconst{ s } < 1. (p. 134)
% - In fact, \resisoconst{ 2s } < 1 IS MORE RELEVANT, since the inequality (6.1) yields
%   \norm{ \mat{A} ( \vect{x} − \vect{x}' ) }{2}^{2} > 0 for
%   all distinct s-sparse vectors \vect{x}, \vect{x}' \in \C^{N}; hence,
%   DISTINCT s-SPARSE VECTORS HAVE DISTINCT MEASUREMENT VECTORS. (p. 134)
The complete relaxation of
% 1.) upper bound on the RIC
this upper bound, i.e.
$\resisoconstub{ 2s } = 1$
\cite[134]{book:Foucart2013}, and
the successive insertions of
% 2.) lower bounds on the worst-case coherences of the sensing matrices
both lower bounds on
the worst-case coherences
\eqref{eqn:disc_coherence_lb} yield
the upper bounds on
% 3.) numbers of nonzero components in the sparse representations
the numbers of
nonzero components
\begin{equation}
 %--------------------------------------------------------------------------------------------------------------
 % upper bounds on the numbers of nonzero components in the sparse representations
 %--------------------------------------------------------------------------------------------------------------
  s
  <
  \frac{ 1 }{ 2 }
  \biggl[
    \frac{ 1 }{ \coherlb\bigl( \mat{A}\bigl[ p^{(\text{in})} \bigr] \bigr) }
    +
    1
  \biggr]
  \leq
  \frac{ 1 }{ 2 }
  \biggl[
    \frac{ 1 }{ \coherlbwelch\bigl( \mat{A}\bigl[ p^{(\text{in})} \bigr] \bigr) }
    +
    1
  \biggr].
 \label{eqn:disc_coherence_sparsity_ub}
\end{equation}

%---------------------------------------------------------------------------------------------------------------
% 2.) upper bounds on the numbers of nonzero components did not provide any practical guarantees
%---------------------------------------------------------------------------------------------------------------
% a) upper bounds [number of nonzero components] did not provide any practical guarantees and seemed to contradict the numerical simulations
These upper bounds, however, did not provide
any practical guarantees for
% 1.) normalized sensing matrices (all pulse-echo measurements, multifrequent, all array elements)
the normalized sensing matrices
\eqref{eqn:recon_reg_norm_sensing_matrix} induced by
% 2.) all incident waves
all incident waves and, thus, seemed to contradict
% 3.) numerical simulations
the numerical simulations
(cf. \cref{tab:results_sr_1_tpsf_coherence}).
% b) both reference sensing matrices only ensured the recovery of at most 22 nonzero components
% TODO: really? bound on RIC is 1! => injectivity only
Moreover,
both reference sensing matrices only ensured
the recovery of
% s = 22 because s < 23
at most $22$ nonzero components, and
% c) neither of the investigated sensing matrices achieved the Welch lower bounds on the worst-case coherences
% article:KutyniokGAMM2013: Theory and applications of compressed sensing
% Sect. 3: Conditions for Sparse Recovery / Sect. 3.2: Sufficient Conditions / Sect. 3.2.1: Mutual Coherence
% - The LOWER BOUND presented in the next result, also KNOWN AS THE WELCH BOUND, is more interesting. (p. 88)
% - Lemma 3.7:
%	- Let A be an m×n matrix. Then we have [\mu(\mat{A}) \in [ \sqrt{ \frac{ n - m }{ m (n - 1) } } ; 1 ]]. (p. 88)
neither of
% 1.) sensing matrices (all pulse-echo measurements, multifrequent, all array elements)
the investigated sensing matrices
\eqref{eqn:recovery_reg_sensing_matrix} achieved
% 2.) Welch lower bounds on the worst-case coherences of the sensing matrices
the \name{Welch} lower bounds
\eqref{eqn:disc_coherence_lb_welch}.
% d) multiple independent studies confirmed the worst-case coherences close to unity in pulse-echo UI
\TODO{literature}
% article:BessonITUFFC2018: Ultrafast ultrasound imaging as an inverse problem: Matrix-free sparse image reconstruction
% VI. RESULTS: COMPRESSED BEAMFORMING / A. A deep dive into coherence
% - Fig. 6: Mutual coherence µ(H_{d}Ψ) against the number of measurements for
%   (a) Dirac basis and (b) Haar basis for
%   the uniform selection of transducer elements,
%   the random selection of transducer elements, CMIX and CTMIX (Dt = 10).
%   Additionally, the coherence is evaluated
%   (c) for CMIX and CTMIX with mixing coefficients drawn using normal and Rademacher distributions and
%   (d) for CTMIX at different depths.
% article:BessonITUFFC2016: A Sparse Reconstruction Framework for Fourier-Based Plane-Wave Imaging
% -
% proc:BessonICIP2016: Compressed delay-and-sum beamforming for ultrafast ultrasound imaging
% 4. EXPERIMENTS
% - In this section, the two undersampling schemes described in section 3.2 are studied
%   [1.)] FIRSLTY IN TERMS OF COHERENCE and
%   [2.)] then in terms of QUALITY OF THE RECONSTRUCTION. (p. )
% - In order to have a baseline for comparisons, we also include the results for a third undersampling scheme (Scheme 3), which consists in
%   a 2D point-wise random subsampling of the raw data in a similar way to [12]. (p. )
%   [12] article:DavidJASA2015
% - In this scheme, the raw data are undersampled randomly at each time instant. (p. )
% 4. EXPERIMENTS / 4.1. Coherence study of the undersampling schemes
% - The undersampling schemes are firstly compared in terms of coherence. (p. )
% - In order to do so, the probe used in the sections 4.2 and 4.3 is simulated.
%   Then, the corresponding matrix H is built, for an imaging depth of 5 cm, and its coherence is computed using equation (5). (p. )
% - The results are reported in table 1 for different values of J [# of active RX transducers]. (p. )
% - For schemes 2 and 3, the coherence values displayed in table 1 correspond to an average over 200 runs. (p. )
% - Table 1: Coherence values for the three proposed schemes. [smallest value: 0.98]
% - From table 1, it can be concluded that
%   THE DIFFERENT SCHEMES ARE EQUIVALENT IN TERMS OF COHERENCE. (p. )
% article:DavidJASA2015: Time domain compressive beam forming of ultrasound signals
% IV. TIME DOMAIN 2D COMPRESSED BEAM FORMING / A. 2D beam forming matrix / 2. Relationship with DAS
% - As we will see further in the development the MUTUAL COHERENCE of that matrix, that accounts for
%   the similarity of its column vectors, is of tremendous importance in
%   the RESOLUTION ACHIEVABLE IN BOTH DAS AND t-CBF FRAMEWORKS. (p. 2779)
% IV. TIME DOMAIN 2D COMPRESSED BEAM FORMING / A. 2D beam forming matrix / 3. Practical implementation
% - Intuitively, one can expect this dictionary to be suitable for CS as long as
%   the number of scatterers in the medium is small enough to ensure sparsity, and
%   the GRID SPACING IN DEPTH AND AZIMUTH IS CHOSEN WISELY, to ensure that
%   the DICTIONARY HAS A LOW COHERENCE. (p. 2779)
% V. RESULTS AND DISCUSSION / B. t-CBF using a plane wave excitation and 16 transducers in reception
% - Reducing the number of transducers in acquisition can take several forms. (p. 2781)
% - In fact, using less elements in the probe raises a simple question:
%   HOW TO SELECT THE ELEMENTS IN A WAY THAT SATISFIES THE PRINCIPLES OF CS. (p. 2781)
% - The SELECTION PROCESS IS NOT TRIVIAL, as it DIRECTLY IMPACTS
%   THE MUTUAL COHERENCE OF THE MEASUREMENT MATRIX. (p. 2781)
% - Figure 9 shows the EVOLUTION OF THE MUTUAL COHERENCE OF THE MATRIX G. (p. 2781)
% - It is generated by
%   [1.)] COMPUTING THE SCALAR PRODUCTS OF ALL THE COLUMNS OF G and
%   [2.)] SORTING THEM BY DESCENDING ORDER OF MAGNITUDE. (p. 2781)
% - The mutual coherence \mu of a matrix G is commonly defined as
%   the maximum absolute value of the cross-correlations of the columns of G,
%   [ \mu_{G} = \underset{ 1 \leq i,j \leq N }{ \max } \abs{ \trans{G_{i}} G_{j} } ] (33) (p. 2781)
% V. RESULTS AND DISCUSSION / C. t-CBF and super-resolution / 1. Principle
% - If the dictionary includes wavefronts originating from scatterers closer than \lambda / 2,
%   we can hope to separate them. (p. 2782)
% - Of course, one could object that with such a fine grid,
%   the COHERENCE OF G WILL INCREASE DRASTICALLY. (p. 2782)
Multiple authors confirmed
% 1.) impractical upper bounds
% article:TillmannITIT2014: The Computational Complexity of the Restricted Isometry Property, the Nullspace Property, and Related Concepts in Compressed Sensing
% V. CONCLUDING REMARKS
% - Instead of the INTRACTABLE RIP, NSP, or spark,
%   the weaker but efficiently computable MUTUAL COHERENCE [24] is sometimes used. (p. 1257)
% - It can be shown that
%   THE MUTUAL COHERENCE YIELDS BOUNDS ON THE RIC, NSC, and the SPARK; see, for instance, [27], [58]. (p. 1257)
% - Thus,
%   IMPOSING CERTAIN CONDITIONS INVOLVING THE MUTUAL COHERENCE OF A MATRIX CAN YIELD UNIQUENESS AND RECOVERABILITY
%   (by basis pursuit or other heuristics), see, e.g., [58]. (p. 1257)
% - However, the SPARSITY LEVELS for which the MUTUAL COHERENCE can guarantee recoverability are
%   quite often TOO SMALL TO BE OF PRACTICAL USE. (p. 1257)
this impracticality
\cite{article:TillmannITIT2014} and
% 2.) worst-case coherences close to unity underlying these upper bounds in pulse-echo UI
the worst-case coherences close to unity in
pulse-echo \ac{UI}
\cite[Fig. 6]{article:BessonITUFFC2018},
%\cite{article:BessonITUFFC2016},
\cite[Table 1]{proc:BessonICIP2016},
\cite[Fig. 9]{article:DavidJASA2015}.

%---------------------------------------------------------------------------------------------------------------
% 3.) reasons for the impracticality of the upper bounds on the numbers of nonzero components
%---------------------------------------------------------------------------------------------------------------
% a) impracticality of the upper bounds arose from both the looseness of the upper bounds and the limited spatial and spectral resolutions
The impracticality of
% 1.) upper bounds on the numbers of nonzero components in the sparse representations
the upper bounds
\eqref{eqn:disc_coherence_sparsity_ub} arose from both
% 2.) looseness of the upper bounds on the RICs provided by the worst-case coherences
the looseness of
the upper bounds on
the \acp{RIC} provided by
the worst-case coherences
\eqref{eqn:disc_coherence_resisoconst_ub} and
% 3.) limited spatial and spectral resolutions
the limited spatial and
spectral resolutions.
% b) limited spatial and spectral resolutions prevented the recovery of sparse representations whose nonzero components populated adjacent grid points or discrete spatial frequencies
The latter prevented
the recovery of
% 1.) sparse representations
sparse representations
\eqref{eqn:recovery_reg_sparse_representation} whose
% 2.) nonzero components
nonzero components populated
% 3.) adjacent grid points or discrete spatial frequencies
adjacent grid points or
discrete spatial frequencies.
% c) upper bounds s < 2 prohibited such configurations and reflected this limitation
The upper bounds $s < 2$ prohibited
such configurations and reflected
this limitation.
% d) upper bounds s < 2 did not contradict the numerical simulations because specific configurations meeting the resolution requirements were recoverable
They did not contradict
the numerical simulations because
specific configurations meeting
the resolution requirements were
recoverable.
% e) increase of the constant spacing between the adjacent grid points along each coordinate axis effectively reduces the worst-case coherences and increases the upper bounds
The increase of
% 1.) constant spacings between the adjacent grid points along each coordinate axis
the constant spacings between
the adjacent grid points in
the \ac{FOV} trivially decorrelates
% 2.) pulse echoes
their pulse echoes and, thus, effectively reduces
% 3.) worst-case coherences of the sensing matrices (all pulse-echo measurements, multifrequent, all array elements)
the worst-case coherences
\eqref{eqn:disc_coherence} and increases
% 4.) upper bounds on the numbers of nonzero components in the sparse representations
the upper bounds
\eqref{eqn:disc_coherence_sparsity_ub}.
% e) faithful discrete representations of the continuous physical models impose upper bounds on these spacings and enforce a trade-off
Faithful discrete representations of
the continuous physical models, however, impose
upper bounds on
these spacings and enforce
a trade-off.
% f) findings emphasize the importance of alternative concepts to assess the suitability of the sensing matrices
% article:TillmannITIT2014: The Computational Complexity of the Restricted Isometry Property, the Nullspace Property, and Related Concepts in Compressed Sensing
% V. CONCLUDING REMARKS
% - This [too small sparsity levels to be of practical use] emphasizes the IMPORTANCE OF OTHER CONCEPTS. (p. 1257)
The impracticality of
% 1.) upper bounds on the numbers of nonzero components in the sparse representations
the upper bounds
\eqref{eqn:disc_coherence_sparsity_ub} emphasizes
the importance of
alternative concepts to assess
the suitability of
% 2.) normalized sensing matrices (all pulse-echo measurements, multifrequent, all array elements)
the normalized sensing matrices
\eqref{eqn:recon_reg_norm_sensing_matrix}, e.g.
% 2.) investigation of the TPSFs
% article:Schiffner2018, Sect. II: Compressed Sensing (sec:compressed_sensing)
% - The \ac{TPSF} frequently quantifies the coherence of the sensing matrices \eqref{eqn:cs_math_prob_general_sensing_matrix} in
%   medical imaging technologies (cf. e.g. \cite{article:ProvostITMI2009,article:LustigMRM2007}).
the presented and
frequently-used investigation of
the \acp{TPSF}
(cf. e.g.
\cite{article:ProvostITMI2009,article:LustigMRM2007}%
) and
% 3.) quantitative recovery experiments
quantitative recovery experiments.
% f) investigation of the TPSFs and recovery experiments empirically revealed the advantages of the random waves over the QPW
These empirically revealed
the advantages of
% 1.) random waves
the random waves over
% 2.) quasi-plane wave (QPW)
the \ac{QPW}.
