%---------------------------------------------------------------------------------------------------------------
% 1.) reduction of the number of observations and its consequences
%---------------------------------------------------------------------------------------------------------------
% a) number of observations defines the minimal data volume required for the determination of the relevant Fourier coefficients
The number of
observations
\eqref{eqn:recovery_sys_lin_eq_num_obs} defines
% 1.) minimal data volume
the minimal data volume required for
% 2.) determination
the determination of
% 3.) relevant Fourier coefficients
the relevant \name{Fourier} coefficients
\eqref{eqn:recovery_disc_freq_v_rx_Fourier_series_coef}. % forming
% 2.) vector stacking the relevant Fourier coefficients of the recorded RF voltage signals (all pulse-echo measurements, multifrequent, all array elements)
%the vector
%\eqref{eqn:recovery_sys_lin_eq_v_rx_born_all_f_all_in_v_rx}.
% b) it [total number of observations] simultaneously controls both the memory consumption and the computational costs raised by the numerical evaluations of the matrix-vector products
It simultaneously controls both
% 1.) memory consumption
the memory consumption and
% 2.) computational costs
the computational costs raised by
the numerical evaluations of
the matrix-vector products involving
% 3.) normalized sensing matrix (all pulse-echo measurements, multifrequent, all array elements)
the normalized sensing matrix
\eqref{eqn:recon_reg_norm_sensing_matrix} or
% 4.) adjoint of the normalized sensing matrix (all pulse-echo measurements, multifrequent, all array elements)
its adjoint.
% c) methods for the reduction of the number of observations
% article:Schiffner2018, Sect. V-C. Systems of Linear Algebraic Equations Obtained from the First Born Approximation (subsec:recovery_systems_linear_equations)
% - The NUMBER OF OBSERVATIONS \eqref{eqn:recovery_sys_lin_eq_num_obs} exclusively depends on
%   [1.)] the total number of array elements $N_{\text{el}}$,
%   [2.)] the numbers of relevant discrete frequencies \eqref{eqn:recon_disc_axis_f_discrete_BP_TB_product}, and
%   [3.)] the number of sequential pulse-echo measurements $N_{\text{in}}$.
% article:Schiffner2018, Sect. III-D. Pulse-Echo Measurement Process
% - These methods differ in their \emph{efficiency}, i.e. the quotient of
%   the data volume occupied by the quantized relevant \name{Fourier} coefficients and
%   the DATA VOLUME DIGITIZED DURING THE PULSE-ECHO MEASUREMENT.
Besides minimizing
% 1.) number of sequential pulse-echo measurements per image
the number of
sequential pulse-echo measurements per
image,
% 2.) deactivation of selected receiving array elements
% article:BessonITUFFC2018: Ultrafast Ultrasound Imaging as an Inverse Problem: Matrix-Free Sparse Image Reconstruction
% III. SPARSITY-DRIVEN IMAGE RECONSTRUCTION METHODS / C. Compressed Beamforming / 1) Selection of Transducer Elements
% - In our previous work [12], we have proposed several designs for the UNDERSAMPLING OPERATOR, which selects
%   a SUBSET OF TRANSDUCER ELEMENTS, either uniformly or randomly spaced. (p. 343)
%   [12] proc:BessonICIP2016: Compressed delay-and-sum beamforming for ultrafast ultrasound imaging
% - It has been shown that HIGH-QUALITY RECONSTRUCTIONS may be achieved with
%   APPROXIMATELY 40%–50% OF THE INITIAL DATA, for PW and DW imaging [32].
%   [32] proc:BessonIUS2016
% proc:BessonICIP2016: Compressed delay-and-sum beamforming for ultrafast ultrasound imaging
% 3. COMPRESSED DAS BEAMFORMING / 3.2. The undersampling scheme
% - In US imaging, the easiest undersampling scheme can be achieved by
%   SHUTTING DOWN SEVERAL TRANSDUCERS AT RECEPTION. (p. )
% - Instead of solving it [minimize mutual coherence],
%   we will consider the TWO FOLLOWING ACQUISITION SCHEMEs that intuitively seems quite interesting to study:
%   - Scheme 1 - UNIFORMLY SPACED TRANSDUCERS:
%     The transducers are uniformly selected across the aperture.
%     This seems logical in terms of coherence since the closer the transducers,
%     the more similar the information they are sensing, the more coherent their contribution in the sensing matrix.
%   - Scheme 2 - RANDOMLY CHOSEN TRANSDUCERS:
%     The transducers are selected randomly in the aperture.
%     This non-uniform spacing has proven to be suited for CS in radar imaging as described in [20].
% 4. EXPERIMENTS
% - In this section, the two undersampling schemes described in section 3.2 are studied
%   [1.)] FIRSLTY IN TERMS OF COHERENCE and
%   [2.)] then in terms of QUALITY OF THE RECONSTRUCTION. (p. )
% - In order to have a baseline for comparisons, we also include the results for a THIRD UNDERSAMPLING SCHEME (Scheme 3), which consists in
%   A 2D POINT-WISE RANDOM SUBSAMPLING OF THE RAW DATA in a similar way to [12]. (p. )
%   [12] article:DavidJASA2015: Time domain compressive beam forming of ultrasound signals
% - In this scheme, the RAW DATA ARE UNDERSAMPLED RANDOMLY AT EACH TIME INSTANT. (p. )
% article:DavidJASA2015: Time domain compressive beam forming of ultrasound signals
% V. RESULTS AND DISCUSSION / B. t-CBF using a plane wave excitation and 16 transducers in reception
% - One of the great benefits of CS is the ability to DECREASE THE NUMBER OF MEASUREMENTS necessary to perform an accurate reconstruction. (p. 2781)
% - In our case, that could mean
%   ACQUIRING LESS SAMPLES IN TIME, or LESS SAMPLES IN SPACE, or BOTH. (p. 2781)
% - It could also mean IMAGING A MEDIUM USING LESS TRANSMITS. (p. 2781)
% - In this paper, we focus on
%   [1.)] ACQUIRING LESS SPATIAL SAMPLES, and
%   [2.)] LESS TRANSMITS, as it seems to be the most beneficial way to use CS for ultrasound beam forming. (p. 2781)
% - The use of one, or a few, non-focalized transmits would allow for HIGHER FRAME RATES. (p. 2781)
% - Reducing the number of transducers in acquisition can take several forms. (p. 2781)
% - In fact, using less elements in the probe raises a simple question:
%   HOW TO SELECT THE ELEMENTS IN A WAY THAT SATISFIES THE PRINCIPLES OF CS. (p. 2781)
% - The SELECTION PROCESS IS NOT TRIVIAL, as it DIRECTLY IMPACTS
%   THE MUTUAL COHERENCE OF THE MEASUREMENT MATRIX. (p. 2781)
% - The acquisition using the entire aperture of 128 elements is here taken as a reference and compared against
%   DIFFERENT ELEMENT SELECTION STRATEGIES: the signal is acquired using
%   (a) the N_{\text{acq}} CENTRAL TRANSDUCERS; or
%   (b) N_{\text{acq}} TRANSDUCERS EQUALLY SPACED, and spanning the entire aperture; or
%   (c) N_{\text{acq}} TRANSDUCERS SELECTED AT RANDOM, and spanning the entire aperture, used throughout the acquisition; or finally
%   (d) N_{\text{acq}} TRANSDUCERS SELECTED AT RANDOM AT EACH TIME SAMPLE. (p. 2781)
(i) the deactivation of
selected receiving array elements
\cite[Sect. III-C.1)]{article:BessonITUFFC2018},
\cite[Sects. 3.2 and 4]{proc:BessonICIP2016},
\cite[Sect. V-B]{article:DavidJASA2015},
% 3.) random temporal mixing of the recorded RF voltage signals
% article:BessonITUFFC2018: Ultrafast Ultrasound Imaging as an Inverse Problem: Matrix-Free Sparse Image Reconstruction
% III. SPARSITY-DRIVEN IMAGE RECONSTRUCTION METHODS / C. Compressed Beamforming / 2) Mixing of the Raw Data
% - In order to overcome this drawback [increase of the coherence induced by simple undersampling schemes], TWO STRATEGIES, denoted as
%   [1.)] “CHANNEL MIXING” (CMIX) and
%   [2.)] “CHANNEL AND TIME MIXING” (CTMIX), are proposed. (p. 343)
% - CMIX consists of a RANDOM SUMMATION OF THE SIGNALS coming from the different transducers elements at a given time instant in order to create a mixed output. (p. 343)
% - CTMIX extends the principle of CMIX by considering mixing across both transducer elements and D_{t} TIME SAMPLES to limit the complexity. (p. 343)
(ii) the random temporal mixing of
the recorded \ac{RF} voltage signals
\cite[Sect. III-C.2)]{article:BessonITUFFC2018},
% 4.) random Gaussian projections
% article:ZhangUlt2013: A measurement-domain adaptive beamforming approach for ultrasound instrument based on distributed compressed sensing: Initial development
(iii) random temporal projections
\cite{article:ZhangUlt2013}, and
% 5.) customized alternative discretizations of the frequency axis
% article:ProvostITMI2009: The application of compressed sensing for photo-acoustic tomography
% VI. COMPRESSED SENSING / B. PA Forward Operator as a CS-Matrix / Proof of Concept on Small Phantoms
% - However, one needs to model the transducer response g_{n} appearing in K_{(m,n)(i,j)}. (p. 590)
% - As a first approximation, we used the SIMPLEST BAND-PASS FILTER:
%   frequencies at each angles have been sampled in a rectangular window between 0.2 and 1.2 MHz. (p. 590)
% - To completely define K_{(m,n)(i,j)}, we considered
%   a fixed number of tomographic angles indexed by m with
%   40 RANDOMLY CHOSEN k_{n} / 2 \pi c’s inside the [0.2, 1.2] MHz window for each and
%   the numerical derivative as a sparse basis. (p. 590)
% VII. RESULTS / A. Simulations
% - As before, we generated measurements y_{(m,n)} from a phantom x_{(i,j)} using K_{(m,n)(i,j)} and changed four parameters:
%   [1.)] the number of tomographic angles,
%   [2.)] the frequency window width,
%   [3.)] the SAMPLING OF THIS FREQUENCY WINDOW, and
%   [4.)] the basis used for reconstruction. (p. 591)
% VII. RESULTS / B. Experiments
% - Fig. 6(a) shows reconstruction using 56 angles taken outside a dead angle of 45° and
%   128 frequency samples RANDOMLY CHOSEN inside the [0.5, 6] MHz window. (p. 593)
% - Fig. 6(b) shows reconstruction using 50 angles taken outside a dead angle of 35° and
%   128 frequency samples RANDOMLY CHOSEN inside the [0.5, 6] MHz window. (p. 593)
(iv) customized alternative discretizations of
the frequency axis
\cite[Sects. VI-B and VII]{article:ProvostITMI2009} enable
% 6.) significant reduction
its significant reduction.
% d) customized alternative discretizations specifically reduce the cardinalities of the sets of relevant discrete frequencies below the effective time-bandwidth products
The latter specifically reduce
% 1.) cardinalities
the cardinalities of
% 2.) sets of relevant discrete frequencies
the sets of
relevant discrete frequencies
\eqref{eqn:recon_disc_axis_f_discrete_BP} below
% 3.) numbers of relevant discrete frequencies (effective time-bandwidth products)
the effective time-bandwidth products
\eqref{eqn:recon_disc_axis_f_discrete_BP_TB_product}.
% e) approaches potentially discard essential observations discriminating the pulse echoes of the structural building blocks and typically increase their correlations
% article:BessonITUFFC2018: Ultrafast Ultrasound Imaging as an Inverse Problem: Matrix-Free Sparse Image Reconstruction
% III. SPARSITY-DRIVEN IMAGE RECONSTRUCTION METHODS / C. Compressed Beamforming / 1) Selection of Transducer Elements
% - However, the corresponding measurement model H_{d} [subset of transducer elements] suffers from a HIGH COHERENCE [10], [12].
%   [10] article:DavidJASA2015: Time domain compressive beam forming of ultrasound signals
%   [12] proc:BessonICIP2016: Compressed delay-and-sum beamforming for ultrafast ultrasound imaging
% III. SPARSITY-DRIVEN IMAGE RECONSTRUCTION METHODS / C. Compressed Beamforming / 2) Mixing of the Raw Data
% - Taking into account the INCREASE OF THE COHERENCE induced by these simple undersampling schemes [element deactivation],
%   one may think about DESIGNING A STRATEGY such that
%   the MUTUAL COHERENCE OF THE MEASUREMENT OPERATOR IS OPTIMIZED. (p. 343)
These approaches, however, potentially discard
% 1.) essential observations
essential observations discriminating
% 2.) pulse echoes
the pulse echoes of
% 3.) structural building blocks
the structural building blocks and typically increase
% 4.) correlations
their correlations.

%---------------------------------------------------------------------------------------------------------------
% 2.) optimal selections of the active array elements / treated discrete frequencies
%---------------------------------------------------------------------------------------------------------------
% a) optimal methods minimize these correlations and capture Fourier coefficients of relatively high SNRs
% article:ChernyakovaITUFFC2014: Fourier-Domain Beamforming: The Path to Compressed Ultrasound Imaging
% IV. Rate Reduction by Beamforming in Frequency / B. Reduced Rate Sampling
% - We now address the following question:
%   HOW DO WE OBTAIN THE REQUIRED SET β [set of Fourier coefficients] CORRESPONDING TO THE EFFECTIVE BAND-PASS BANDWIDTH, USING
%   B [cardinality of β] LOW-RATE SAMPLES OF EACH ONE OF THE RECEIVED SIGNALS? (p. 1259)
% 3.) FURTHER REDUCTION IN RATE can be achieved if we want to obtain only a PARTIAL FREQUENCY BEAM'S DATA. (p. 1259)
%	- Explicitly, assume that now we are interested in μ_{BF} ⊂ β_{BF} of size M_{BF} of Fourier coefficients of the beam. (p. 1259)
% 	- THE CHALLENGE NOW IS TO RECOVER THE BEAM FROM SUCH PARTIAL FREQUENCY DATA, because
%	  a simple inverse Fourier transform is insufficient in this case. (p. 1259)
% 	- The choice of μ [set of selected FCs of RF signals], and consequently the analog kernel, is dictated by
%	  the TRANSMITTED PULSE SHAPE. (p. 1260)
Optimal methods therefore (i) minimize
% 1.) correlations [ pulse echoes of the structural building blocks ]
these correlations and (ii) capture
% 2.) Fourier coefficients of the recorded RF voltage signals
\name{Fourier} coefficients
\eqref{eqn:recovery_disc_freq_v_rx_Fourier_series_coef} of
% 3.) relatively high SNR
relatively high \acp{SNR}.
% b) suppression of the admissible noise-like incoherent aliasing of sufficiently small energy enables sub-Nyquist spatiotemporal sampling rates
% article:Schiffner2018, Sect. II: Compressed Sensing in a Nutshell (sec:compressed_sensing)
% - For $n_{1} \neq n_{2}$, however, both column vectors typically differ, and
%   the ABSOLUTE VALUE OF THE \ac{TPSF} \eqref{eqn:cs_math_tpsf} IDEALLY APPROACHES ZERO WITH NOISE-LIKE STATISTICS
%   \cite{article:ProvostITMI2009,article:LustigMRM2007}.
% - These properties, which are referred to as INCOHERENT ALIASING, indicate
%   the reliable discrimination of the admissible structural building blocks by the observation process and guide
%   the sparsity-promoting $\ell_{q}$-minimization method \eqref{eqn:cs_lq_minimization}.
The suppression of
% 1.) incoherent aliasing
the incoherent aliasing by
% 2.) sparsity-promoting lq-minimization method
the sparsity-promoting $\ell_{q}$-minimization method
\eqref{eqn:recovery_reg_norm_lq_minimization} then enables
% 3.) sub-Nyquist spatiotemporal sampling rates
sub-\name{Nyquist} spatiotemporal sampling rates.
% c) approximate Gaussian distribution permits the selection of consecutive discrete frequencies around the center frequency
% article:BurshteinITUFFC2016: Sub-Nyquist Sampling and Fourier Domain Beamforming in Volumetric Ultrasound Imaging
% III. BEAMFORMING IN FREQUENCY / B. Rate Reduction by Beamforming in Frequency
% - Using XAMPLING, we can obtain
%   an ARBITRARY AND POSSIBLY NONCONSECUTIVE SET κ, COMPOSED OF K FREQUENCY COMPONENTS OF THE INDIVIDUAL DETECTED SIGNALS, from
%   K pointwise samples of the signal filtered with an analog kernel s^{*}(t), designed according to κ. (p. 707)
% [1.)] MODULATED GAUSSIAN PULSE:
% - In ULTRASOUND IMAGING WITH MODULATED GAUSSIAN PULSES, THE TRANSMITTED SIGNAL HAS ONE MAIN BAND OF ENERGY. (p. 707)
% - As a result, the ANALOG FILTER TAKES ON THE FORM OF A BAND-PASS FILTER, leading to a simple low-rate sampling scheme [32]. (p. 707)
%   [32] article:ChernyakovaITUFFC2014
% - The choice of κ dictates the BANDWIDTH OF THE FILTER AND THE RESULTING SAMPLING RATE. (p. 707)
% - Further rate reduction is possible by acquiring A PART OF THE BANDWIDTH OF THE BEAMFORMED SIGNAL, μ ⊂ β, |μ| = M. (p. 708)
% - We may calculate it from M + L1 + L2 ≈ M samples of the individual signals, which are sampled at a rate that is N / M lower than
%   the standard beamforming rate fs. (p. 708)
% article:ChernyakovaITUFFC2014: Fourier-Domain Beamforming: The Path to Compressed Ultrasound Imaging
% IV. Rate Reduction by Beamforming in Frequency / B. Reduced Rate Sampling
%   [1.)] MODULATED GAUSSIAN PULSE:
%   - When imaging is performed with a MODULATED GAUSSIAN PULSE,
%     the OPTIMAL CHOICE OF μ is to take M CONSECUTIVE ELEMENTS AROUND THE CENTRAL FREQUENCY. (p. 1260)
%   - THIS CHOICE CAPTURES THE MAXIMAL AMOUNT OF THE SIGNAL ENERGY and
%     can be implemented with a BAND-PASS FILTER defined by the frequency band corresponding to μ. (p. 1260)
In
the simulation study, for instance,
the approximate Gaussian distribution of
the energy in
the \name{Fourier} coefficients
\eqref{eqn:recovery_disc_freq_v_rx_Fourier_series_coef} over
the frequency axis
(cf. \cref{tab:sim_study_parameters}) permits
the selection of
only a few consecutive discrete frequencies around
the center frequency
\cite[Sect. III-B]{article:BurshteinITUFFC2016},
\cite[Sect. IV-B]{article:ChernyakovaITUFFC2014}.
% d) uniform distribution would permit randomly and uniformly distributed discrete frequencies
% article:ChernyakovaITUFFC2014: Fourier-Domain Beamforming: The Path to Compressed Ultrasound Imaging
% IV. Rate Reduction by Beamforming in Frequency / B. Reduced Rate Sampling
%   [2.)] FLAT SPECTRUM:
%   - On the other hand, if the SPECTRUM OF THE TRANSMITTED PULSE IS FLAT, which is the case for LINEAR FREQUENCY-MODULATED CHIRPS [27], [28], then
%     the performance of CS recovery improves when μ is COMPRISED OF ELEMENTS OF β CHOSEN UNIFORMLY AT RANDOM. (p. 1260)
%   - The RESULTING SAMPLING OPERATION CAN BE IMPLEMENTED USING THE TECHNIQUES PROPOSED IN [3] and [11]. (p. 1260)
%     [3] article:TurITSP2011, [11] article:Baransky2012
A uniform distribution of
this energy, in contrast, would permit
the selection of
randomly and uniformly distributed discrete frequencies
\cite[Sect. IV-B]{article:ChernyakovaITUFFC2014}.

%---------------------------------------------------------------------------------------------------------------
% 3.) comparison to compressed beamforming
%---------------------------------------------------------------------------------------------------------------
% a) proposed method resembles the compressed beamforming methods
% article:ChernyakovaITUFFC2018: Fourier-Domain Beamforming and Structure-Based Reconstruction for Plane-Wave Imaging
% Abstract
% - To further reduce the rate [sampling and processing] we exploit
%   the STRUCTURE OF THE BEAMFORMED SIGNAL and use
%   COMPRESSED SENSING METHODS to RECOVER THE BEAMFORMED SIGNAL from its PARTIAL FREQUENCY DATA obtained at a SUB-NYQUIST RATE. (p. )
% proc:SchiffnerIUS2016a: A low-rate parallel Fourier domain beamforming method for ultrafast pulse-echo imaging
% article:BurshteinITUFFC2016: Sub-Nyquist Sampling and Fourier Domain Beamforming in Volumetric Ultrasound Imaging
% article:ChernyakovaITUFFC2014: Fourier-Domain Beamforming: The Path to Compressed Ultrasound Imaging
% V. Further Reduction Through Compressed Sensing
% - We now address RECONSTRUCTION OF THE BEAMFORMED SIGNAL FROM PARTIAL FREQUENCY DATA. (p. 1260)
% - Explicitly, we aim to reconstruct the beamformed signal from its M_{BF} Fourier coefficients, denoted by μ_{BF}. (p. 1260)
% - To this end, we use CS techniques while EXPLOITING THE FRI STRUCTURE OF THE BEAMFORMED SIGNAL. (p. 1260)
% article:WagnerITSP2012: Compressed Beamforming in Ultrasound Imaging
Using
sub-\name{Nyquist} temporal sampling rates,
% 1.) proposed method
the proposed method resembles
% 2.) compressed beamforming methods
the compressed beamforming methods
\cite{article:ChernyakovaITUFFC2018,proc:SchiffnerIUS2016a,article:BurshteinITUFFC2016,article:ChernyakovaITUFFC2014,article:WagnerITSP2012}.
% b) both types of methods recover specific signals from only a few Fourier coefficients of the recorded RF voltage signals
% article:ChernyakovaITUFFC2014: Fourier-Domain Beamforming: The Path to Compressed Ultrasound Imaging
% V. Further Reduction Through Compressed Sensing / B. Prior Work
% - In the presence of MODERATE TO HIGH NOISE LEVELS,
%   the UNKNOWN PARAMETERS CAN BE EXTRACTED MORE EFFICIENTLY USING A CS APPROACH, as was shown in [5]. (p. 1261)
%   [5] article:WagnerITSP2012
% - The solution set can be narrowed down to a single value by exploiting
%   THE STRUCTURE OF THE UNKNOWN VECTOR \vect{b}. (p. 1261)
% - In the CS framework, it is assumed that the vector of interest is reasonably sparse, whether
%   in the standard coordinate basis or with respect to some other basis. (p. 1261)
% - The regularization introduced in [5] relies on the assumption that the coefficient vector \vect{b} is L-SPARSE. (p. 1261)
% - The formulation in (19) then has a form of a CLASSIC CS PROBLEM, where the goal is to
%   reconstruct an N-dimensional L-sparse vector \vect{b} from
%   its projection onto K orthogonal rows captured by the measurement matrix A. (p. 1261)
% - This problem can be solved using numerous CS techniques when A satisfies well-known properties such as
%   restricted isometry (RIP) or coherence [6]. (p. 1261)
% - A typical beamformed ultrasound signal is comprised of
%   a relatively small number of strong reflections, corresponding to strong perturbations in the tissue, and
%   many weaker scattered echoes, originating from microscopic changes in acoustic impedance of the tissue. (p. 1261)
% - The framework proposed in [5] [article:WagnerITSP2012] attempts to recover
%   only strong reflectors in the tissue and treat weak echoes as noise. (p. 1261)
% - Hence, the vector of interest \vect{b} is indeed L-sparse with L << N. (p. 1261)
% - To recover \vect{b}, Wagner et al. consider the following optimization problem:
%   [ \underset{ \vect{b} }{ \min } \norm{ \vect{b} }{0} subject to \norm{ \mat{A} \vect{b} - \vect{c} }{2} \leq \epsilon ], (20)
%   where \epsilon is an appropriate noise level, and approximate its solution using
%   ORTHOGONAL MATCHING PURSUIT (OMP) [32]. (p. 1261)
% - A SIGNIFICANT DRAWBACK OF THIS METHOD IS ITS INABILITY TO RESTORE WEAK REFLECTORS. (p. 1261)
% - In the context of this approach, they are treated as noise and are disregarded by the signal model. (p. 1261)
% - As a result, the SPECKLE — the granular pattern that can be seen in Fig. 3 — IS LOST. (p. 1261)
% article:WagnerITSP2012: Compressed Beamforming in Ultrasound Imaging
In fact,
both types of
methods leverage
% 1.) sparsity-promoting lq-minimization method
a sparsity-promoting $\ell_{q}$-minimization method
\eqref{eqn:cs_lq_minimization} to recover
% 2.) specific signals
specific signals from
% 3.) Fourier coefficients of the recorded RF voltage signals
only a few \name{Fourier} coefficients
\eqref{eqn:recovery_disc_freq_v_rx_Fourier_series_coef} associated with
the selected discrete frequencies.
% c) proposed method recovers d-dimensional acoustic material parameters based on realistic physical models
The former, however, recovers
$d$-dimensional acoustic material parameters based on
realistic physical models.
% d) proposed method minimizes the number of sequential pulse-echo measurements per image and maximizes the frame rate
It minimizes
the number of
sequential pulse-echo measurements per
image and, thus, maximizes
the frame rate.
% e) compressed beamforming methods use simple DAS procedures based on geometric times-of-flight to electronically focus the recorded RF voltage signals
% article:BurshteinITUFFC2016: Sub-Nyquist Sampling and Fourier Domain Beamforming in Volumetric Ultrasound Imaging
% IV. RECOVERY FROM SUB-NYQUIST SAMPLES
% - When only a SUBSET OF THE [Fourier] COEFFICIENTS is obtained by sub-Nyquist sampling and processing,
%   we EXPLOIT THE STRUCTURE OF THE BEAM TO RECONSTRUCT IT FROM ITS PARTIAL FREQUENCY DATA. (p. 708)
% - According to [24],
%   we may MODEL THE DETECTED SIGNALS AT THE INDIVIDUAL TRANSDUCER ELEMENTS, { ϕ_{m,n}( t; θ_{x}, θ_{y} ) }_{ (m,n) ∈ M }, AS
%   FRI SIGNALS. (p. 708)
%   [24] article:TurITSP2011
% - That is, we assume that the individual signals can be regarded as
%   A SUM OF PULSES, ALL REPLICAS OF A KNOWN TRANSMITTED PULSE SHAPE
%   [ ϕ_{m,n}( t; θ_{x}, θ_{y} ) = \sum_{ l = 1 }^{ L } \tilde{a}_{l,m,n} h( t - t_{l,m,n} ) ]. (16) (p. 708)
% - It is shown in Appendix B that
%   the BEAMFORMED SIGNAL IN 3-D IMAGING APPROXIMATELY SATISFIES THE FRI MODEL, just as it does in 2-D imaging [31]. (p. 708)
%   [31] article:WagnerITSP2012
% - Namely, it can be written as
%   [ ϕ( t; θ_{x}, θ_{y} ) \approx \sum_{ l = 1 }^{ L } \tilde{b}_{l} h( t - t_{l} ) ] (17)
%   where [...]. (p. 708)
% article:ChernyakovaITUFFC2014: Fourier-Domain Beamforming: The Path to Compressed Ultrasound Imaging
% V. Further Reduction Through Compressed Sensing
% - To formulate the recovery as a CS problem, we begin with a PARAMETRIC REPRESENTATION OF THE BEAM. (p. 1260)
% A. Parametric Representation
% - According to [5], a BEAMFORMED SIGNAL OBEYS AN FRI MODEL; that is,
%   it can be modeled as a SUM OF REPLICAS OF THE KNOWN TRANSMITTED PULSE, h(t), WITH UNKNOWN AMPLITUDES AND DELAYS:
%   [ \Phi( t; \theta ) \approx \sum_{ l = 1 }^{ L } \tilde{b}_{ l } h( t - t_{ l } ) ]. (15) (p. 1260)
%   [5] article:WagnerITSP2012
% - Because the TRANSMITTED PULSE IS KNOWN, such a signal is completely defined by
%   2L UNKNOWN PARAMETERS: the AMPLITUDES and the DELAYS. (p. 1260)
The latter methods, in contrast, use
% 1.) popular DAS method
% article:Schiffner2018, Sect. I: Introduction (sec:introduction)
% - The popular \ac{DAS} method, for example, focuses
%   the echo signals on specified points in the \ac{FOV} to quantify their echogeneity.
% - Emitting steered PWs \cite{article:ProvostPMB2014,article:MaceITUFFC2013,article:MontaldoITUFFC2009}, whose
%   spatial extent and energy content are unlimited, or
%   outgoing $\{ 1, 2 \}$-spherical waves \cite{article:ProvostPMB2014,article:PapadacciITUFFC2014,article:JensenUlt2006}, whose
%   isotropic sources are points, it adds
%   the signal samples at the round-trip \acp{TOF} \cite[(2), (6)]{article:MontaldoITUFFC2009}, \cite[(4), (5)]{article:JensenUlt2006}.
the popular \ac{DAS} method to focus
% 2.) recorded RF voltage signals
the recorded \ac{RF} voltage signals.
% f) compressed beamforming methods model the one-dimensional focused signals composing the image as quantized finite streams of known pulses
They model
% 1.) one-dimensional focused signals
the one-dimensional focused signals composing
% 2.) image
the image as
% 3.) finite streams of known pulses
quantized finite streams of
known pulses%
\footnote{
  % definition of FRI signals
  % article:BluISPM2008: Sparse Sampling of Signal Innovations
  % - less obvious: sampling schemes taking advantage of some sort of sparsity in the signal
  % - classes of nonbandlimited parametric signals: sample and perfectly reconstruct signals
  %	using sparse sampling; sampling rate is characterized by how sparse the signals are
  %	per unit of time (rate of innovation)
  % - stream of Dirac: not a subspace; estimation of parameters is a nonlinear problem
  % - comparison to CS:
  % 	FRI signals may be seen as sparse in the time domain
  %	however: domain is not discrete
  %	assumptions: innovations fall on some discrete grid known a priori
  %	-method is not as direct as annhihilating filter
  %	-CS does not reach critical sampling rate:
  %		FRI:2K +1 samples for 2K innovations / CS: O(K log(N'))
  %	+CS can account for arbitrary sampling kernels -> flexibility
  %	- in the presence of noise CS does not provide an exactly sparse solution!
  %	- FRI framework is able to reach Cramer-Rao lower bounds, no evidence for CS !
  %	- CS requires random measurements / C-R lower bounds say sensing matrix should be optimized!
  % article:VetterliITSP2002: Sampling signals with finite rate of innovation
  % - Consider CLASSES OF SIGNALS that have a FINITE NUMBER of DEGREES OF FREEDOM per UNIT OF TIME and
  %   call this number the RATE OF INNOVATION.
  % - Thus, we PROVE SAMPLING THEOREMS for CLASSES OF SIGNALS AND KERNELS that
  %   GENERALIZE the classic “bandlimited and sinc kernel” case.
  % I. INTRODUCTION
  % - In the present paper, this NUMBER OF DEGREES OF FREEDOM PER UNIT OF TIME is called the RATE OF INNOVATION of a signal and is denoted by \rho.
  % - In the sequel, we are interested in SIGNALS that have a FINITE RATE OF INNOVATION, either on INTERVALS or on AVERAGE.
  % II. SIGNALS WITH FINITE RATE OF INNOVATION
  % - Let us introduce more precisely SETS OF SIGNALS having a FINITE RATE OF INNOVATION.
  % - DEFINITION 1:
  %   A signal with a FINITE RATE OF INNOVATION is a signal whose PARAMETRIC REPRESENTATION is given in (5) and (6) and with a FINITE \rho, as defined in (7).
  % - The REASON for introducing the RATE OF INNOVATION \rho is that one HOPES to be able to “MEASURE” a signal by TAKING \rho SAMPLES PER UNIT OF TIME and
  %   be able to RECONSTRUCT it. We know this to be true in the uniform case given in (4).
  % - The MAIN CONTRIBUTION of this paper is to SHOW that it is also possible for MANY CASES OF INTEREST in the MORE GENERAL CASES given by (5) and (6).
  % VI. CONCLUSION
  % - We considered signals with FINITE RATE OF INNOVATION that allow
  %   UNIFORM SAMPLING after APPROPRIATE SMOOTHING and PERFECT RECONSTRUCTION from the samples.
  % - To PROVE the SAMPLING THEOREMS, we assumed DETERMINISTIC, NOISELESS SIGNALS.
  Finite streams of
  known pulses are
  special instances of
  \acl{FRI} signals, i.e.
  analog signals defined by
  a finite number of
  parameters in
  an underlying signal model per
  unit of
  time
  (cf. e.g.
  \cite{article:BluISPM2008} and
  \cite{article:VetterliITSP2002}%
  ).
} and individually recover
% 4.) hundreds
hundreds of
% 5.) nearly-sparse parameter vectors
nearly-sparse parameter vectors defining
% 6.) streams
these streams.
% g) compressed beamforming methods primarily reduce the sampling rates and only the most recent evolution permits comparably high frame rates
% article:ChernyakovaITUFFC2018: Fourier-Domain Beamforming and Structure-Based Reconstruction for Plane-Wave Imaging
% proc:SchiffnerIUS2016a: A low-rate parallel Fourier domain beamforming method for ultrafast pulse-echo imaging
They primarily reduce
the temporal sampling rates, and only
their most recent versions additionally support
high frame rates
\cite{article:ChernyakovaITUFFC2018,proc:SchiffnerIUS2016a}.
% h) superior physical model driving the proposed method further increases the frame rate and significantly improves the image quality
% article:ChernyakovaITUFFC2014: Fourier-Domain Beamforming: The Path to Compressed Ultrasound Imaging
% VII. Discussion and Conclusions
% - The FRI MODEL IMPLIES AN ASSUMPTION THAT THE TRANSMITTED PULSE SHAPE REMAINS UNCHANGED during its propagation through the tissue. (p. 1266)
% - This is, of course, a SIMPLIFIED MODEL OF ULTRASOUND PROPAGATION, because frequency-dependent attenuation [46, ch. 5] is not taken into account. (p. 1266)
% - The RESULTS REPORTED IN THIS WORK CAN BE POTENTIALLY IMPROVED WITH AN APPROPRIATE GENERALIZATION OF THE FRI MODEL. (p. 1266)
The superior physical model driving
the proposed method, however, further increases
% 1.) frame rate
the frame rate and significantly improves
% 2.) image quality
the image quality.
