%---------------------------------------------------------------------------------------------------------------
% 1.) summary of the major findings (take-home message)
%---------------------------------------------------------------------------------------------------------------
% a) proposed syntheses of random waves applied random apodization weights, time delays, or combinations thereof to reference excitation voltages
% article:Schiffner2018, Sect. IV.B. Types of Incident Waves (subsec:syn_p_in_types)
% - Modern \ac{UI} systems typically WEIGHT AND DELAY A COMMON REFERENCE VOLTAGE SIGNAL TO PRODUCE THE INDIVIDUAL EXCITATION VOLTAGES.
% - This paper thus specifies excitation voltages of the general form [...] for all transducer elements, where
%   $u^{(\text{tx})} \in \C$ is the amplitude of the COMMON REFERENCE VOLTAGE SIGNAL at
%   the angular frequency $\omega$ IDENTICALLY EXCITING THESE ELEMENTS,
%   $a_{m} \in \R$ are frequency-independent apodization weights, and
%   $\mathcal{Q}$ denotes an operator implementing the quantization of the nonnegative time delays $\Delta t_{m} \in \Rnonneg$.
% article:Schiffner2018, Sect. IV: Syntheses of the Incident Waves (sec:syn_p_in)
% - The \ac{UI} system excites
%   the individual physical elements of the planar transducer array by
%   specified voltage signals to synthesize
%   various types of incident waves.
% - Modern \ac{UI} systems typically generate the individual excitation voltages by applying
%   specified apodization weights, time delays, or combinations thereof to
%   a common reference voltage signal.
The proposed syntheses of
random waves applied
% 1.) random apodization weights
random apodization weights,
% 2.) random time delays
time delays, or
% 3.) combinations thereof
combinations thereof to
% 4.) reference excitation voltages
reference excitation voltages and aided in meeting
% 5.) central requirement of the CS framework in fast compressed pulse-echo UI
a central requirement of
the \ac{CS} framework in
% 6.) fast compressed pulse-echo UI
fast compressed pulse-echo \ac{UI}.
% b) proposed syntheses decorrelated the pulse echoes of the admissible structural building blocks composing the lossy heterogeneous object to be imaged relative to the steered QPWs
% article:Schiffner2018, Sect. II: Compressed Sensing
% - This latter constraint effectively reduces the total number of unknown components to
%   a relatively small number of unknown coefficients associated with
%   the INDIVIDUAL BASIS FUNCTIONS REPRESENTING THE RELEVANT STRUCTURAL BUILDING BLOCKS.
% - Let the unitary matrix $\mat{\Psi} \in \C^{ N \times N }$ represent a suitable orthonormal basis of $\C^{ N }$, i.e.
%   $\mat{\Psi} \herm{ \mat{\Psi} } = \herm{ \mat{\Psi} } \mat{\Psi} = \mat{I}$, e.g. the \name{Fourier}, a wavelet, or the canonical basis.
% - Its column vectors $\vectsym{\psi}_{n} \in \C^{ N }$, $n \in \setcons{ N }$, DEFINE THE ADMISSIBLE STRUCTURAL BUILDING BLOCKS.
They decorrelated
the pulse echoes of
% 1.) admissible structural building blocks
the admissible structural building blocks composing
% 2.) lossy heterogeneous object to be imaged
the lossy heterogeneous object to be imaged relative to
% 3.) steered QPWs
the prevalent steered \acp{QPW}.
% c) structural building blocks equaled the individual basis functions in a nearly-sparse representation of the spatial compressibility fluctuations
These blocks equaled
the individual basis functions in
a nearly-sparse representation of
the spatial compressibility fluctuations.
% d) proposed method aims at recovering the compressibility fluctuations inside the specified FOV from only a few sequential pulse-echo measurements of the received RF voltage signals
% 1.) sparsity-promoting lq-minimization method
A sparsity-promoting $\ell_{q}$-minimization method enabled both
% 2.) structural recovery
the structural and
% 3.) quantitative recovery
the quantitative recovery of
% 4.) two phantoms
two phantoms from
synthetic \ac{RF} voltage signals.
% e)
These were generated by
single realizations of
the random waves in
numerical simulations of
the pulse-echo measurement process in
the two-dimensional Euclidean space.
% e) spatial variations in the incident acoustic energies caused residual errors for the sparse wire phantom
Although
the spatial variations in
% 1.) incident acoustic energies (multiple pulse-echo measurements, multifrequent)
the incident acoustic energies caused
residual errors at
low \acp{SNR} for
the sparse wire phantom,
% f) discrete Fourier basis converts the erratic spatial variations in the incident acoustic energy into beneficial enlarged passbands
they improved
the identification of
the spatially extended structural building blocks defined by
the discrete \name{Fourier} basis and significantly enlarged
the passbands of
the sensing matrices for
the tissue-mimicking phantom.
% f)
%spatially extended structural building blocks of
%the tissue-mimicking phantom thus increase
%the robustness against
%the additive errors.
% g) focus on the efficient implementation using the FMM
The \ac{FMM} enabled
an efficient \ac{GPU}-based implementation of
both types of
matrix-vector products required by
the iterative algorithms.

%---------------------------------------------------------------------------------------------------------------
% 2.) outlook and current research
%---------------------------------------------------------------------------------------------------------------
% article:KruizingaSciAdv2017: Compressive 3D ultrasound imaging using a single sensor
% DISCUSSION
% - Furthermore, we think that a coded aperture mask could potentially be used in conjunction with conventional 1D ultrasound arrays to better estimate
%   the out-of-plane signals and possibly extend the normal 2D imaging capabilities to 3D. (p. 9)
% a) proposed physical models provide the flexibility to investigate alternative types of incident waves
The proposed physical models for
% 1.) linear physical model for the pulse-echo measurement process
the pulse-echo measurement process and
% 2.) syntheses of the incident waves
the syntheses of
the incident waves provide
the flexibility to
investigate alternative types of
incident waves, e.g.
% 1.) superpositions of multiple individually coded quasi-(d-1)-spherical waves
% article:ZhangITMI2018: Ultrafast Ultrasound Imaging With Cascaded Dual-Polarity Waves
% Abstract
% - The newly designed CDW ultrafast ultrasound imaging technique achieved
%   HIGHER QUALITY B-MODE IMAGES than
%   coherent plane-wave compounding (CPWC) and multiplane wave (MW) imaging in
%   a calibration phantom, ex vivo pork belly, and in vivo human back muscle. (p. 906)
% - CDW imaging shows a SIGNIFICANT IMPROVEMENT in the
%   SNR (10.71 dB versus CPWC and 7.62 dB versus MW),
%   PENETRATION DEPTH (36.94% versus CPWC and 35.14% versus MW), and
%   CONTRAST RATIO in deep regions (5.97 dB versus CPWC and 5.05 dB versus MW) without compromising
%   other image quality metrics, such as spatial resolution and frame rate. (p. 906)
% article:ZhaoPMB2017: Coded excitation for diverging wave cardiac imaging: a feasibility study
% article:TiranPMB2015: Multi-plane wave imaging increases signal-to-noise ratio in ultrafast ultrasound imaging
% [12] Elodie Tiran, Thomas Deffieux, and Mafalda Correia et al., “Multi-plane wave imaging increases signal-to-noise ratio in ultrafast ultrasound imaging,” Physics in medicine and biology, vol. 60, no. 21, pp. 8549, 2015.
% proc:BujoreanuACSSC2017: Inverse problem approaches for coded high frame rate ultrasound imaging
% article:GranITUFFC2008: Spatial encoding using a code division technique for fast ultrasound imaging
% [10] Fredrik Gran and Jorgen Arendt Jensen, “Spatial encoding using a code division technique for fast ultrasound imaging,” Ultrasonics, Ferroelectrics, and Frequency Control, IEEE Transactions on, vol. 55, no. 1, pp. 12–23, 2008.
% article:GammelmarkITMI2003: Multielement synthetic transmit aperture imaging using temporal encoding
% [30] K. L. Gammelmark and J. A. Jensen, “Multielement synthetic transmit aperture imaging using temporal encoding,” IEEE Trans. Med. Imag., vol. 22, no. 4, pp. 552–563, Apr. 2003.
%"Diverging wave compounding with spatio-temporal encoding using orthogonal Golay pairs for high frame rate imaging"
% [14] Jian Shen and Emad S. Ebbini, “A new coded-excitation ultrasound imaging system. i. basic principles,” Ultrasonics, Ferroelectrics, and Frequency Control, IEEE Transactions on, vol. 43, no. 1, pp. 131–140, 1996.
% article:MisaridisITUFFC2005: Use of Modulated Excitation Signals in Medical Ultrasound. Part III: High Frame Rate Imaging
% Abstract
% - The use of synthetic transmit aperture imaging is also considered, and it is here shown that
%   Hadamard spatial encoding in transmit with FM emission signals can be used to increase the frame rate by 12 to 25 times with either
%   a SLIGHT OR NO REDUCTION IN SIGNAL-TO-NOISE RATIO AND IMAGE QUALITY. (p. 208)
% XII. Conclusion
% - In synthetic aperture imaging, utilization of FM signals can yield images with resolution and SNR comparable to
%   those of phased array imaging with only four emissions. (p. 217)
superpositions of
multiple individually coded quasi-$(d-1)$-spherical waves
\cite{article:MisaridisITUFFC2005},
% 2.) random focused beams
random focused beams,
% 3.) structured sequences
% article:IlovitshNatComBio2018: Acoustical structured illumination for super-resolution ultrasound imaging
structured waves
\cite{article:IlovitshNatComBio2018}, and even
% 4.) arbitrary experimentally measured waves
arbitrary experimentally measured waves.
% sparse arrays
% article:RouxSciRep2018: Experimental 3-D Ultrasound Imaging with 2-D Sparse Arrays using Focused and Diverging Waves
% Abstract
% - However, the experimental test of new 3-D US approaches is contrasted by
%   the NEED OF CONTROLLING VERY LARGE NUMBERS OF PROBE ELEMENTS.
% - Although this problem [need of controlling very large numbers of probe elements] may be overcome by
%   the use of 2-D SPARSE ARRAYS, just a few experimental results have so far corroborated the validity of this approach. (p. 1)
% - In this paper, we experimentally compare the performance of
%   [1.)] a FULLY WIRED 1024-ELEMENT (32 × 32) ARRAY, assumed as reference, to that of
%   [2.)] a 256-ELEMENT RANDOM and of
%   [3.)] an “OPTIMIZED” 2-D SPARSE ARRAY, in both
%   FOCUSED AND COMPOUNDED DIVERGING WAVE (DW) TRANSMISSION MODES. (p. 1)
% - Furthermore,
%   the experimental results in 3-D DW mode and 3-D focused mode are also compared for the first time and they show that both
%   the contrast and the resolution performance are higher when using the 3-D DW at volume rates up to 90/second which represent
%   a 36x speed up factor compared to the focused mode. (p. 1)
\TODO{mixing schemes}
% b) author is currently investigating the superpositions of multiple individually coded quasi-(d-1)-spherical waves to further decorrelate the pulse echoes
The author is currently investigating
% 1.) superpositions of multiple individually coded quasi-(d-1)-spherical waves
the superpositions of
multiple individually coded quasi-$(d-1)$-spherical waves to further decorrelate
the pulse echoes.
% c) author is simultaneously extending the implementation to the three-dimensional Euclidean space
% book:Devaney2012, Chapter 6: Scattering theory / Sect. 6.7.1: The Born approximation
% - The ADVANTAGE OF THE LINEARIZED BORN MODEL is that
%   we can employ a SYSTEMATIC PROCEDURE TO DEVELOP ANALYTIC INVERSION SCHEMES that
%   CAN LATER BE GENERALIZED TO INCLUDE NON-LINEAR EFFECTS in the actual scattering experiments. (p. 256)
% - For example, we can EXTEND ALL OF THE ANALYSIS contained in this and the following chapter TO
%   NON-CONSTANT BACKGROUNDS CHARACTERIZED BY A (KNOWN) WAVENUMBER k0 = k0(r) THAT VARIES WITH POSITION. (p. 256)
% - Such a generalization is based on the so-called distorted-wave Born approximation (DWBA), which
%   will be developed in Chapter 9. (p. 256)
%Its solution can potentially be generalized to include
%the nonlinear effects
%\cite[256]{book:Devaney2012}.
He is simultaneously extending
% 1.) three-dimensional Euclidean space
the implementation to
the three-dimensional Euclidean space and exploring
% 2.) nonlinear CS to regularize the nonlinear ISP
nonlinear \ac{CS} to regularize
the nonlinear \ac{ISP} generalizing
the \name{Born} approximation.
% d) alternative orthonormal bases and customized redundant dictionaries deserve additional studies
Alternative orthonormal bases, e.g.
wavelet bases, and
% 2.) customized redundant dictionaries
customized redundant dictionaries
\cite{article:CandesACHA2011}, which can be learned from
the ultrasound images of
interest
\cite{article:LorintiuITMI2015}, potentially further improve
% TODO: improves the quasi-continuous tradeoff between
fast compressed pulse-echo \ac{UI} and deserve
additional studies.
