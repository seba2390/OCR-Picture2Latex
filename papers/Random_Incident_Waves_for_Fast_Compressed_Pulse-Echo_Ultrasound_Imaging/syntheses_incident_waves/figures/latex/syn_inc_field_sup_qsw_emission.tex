%---------------------------------------------------------------------------------------------------------------
% TikZ: schematic representation of the proposed physical model by a signal processing chain
%---------------------------------------------------------------------------------------------------------------
% author: Martin Schiffner
% date: 2018-02-05
%
\begin{tikzpicture}

%---------------------------------------------------------------------------------------------------------------
% independent parameters
%---------------------------------------------------------------------------------------------------------------
% total number of transducer elements
\pgfmathsetmacro{\NElements}{20}

% geometric parameters of the linear array
\pgfmathsetmacro{\FactorScale}{0.9}
\pgfmathsetmacro{\ElementWidth}{0.2*\FactorScale}
\pgfmathsetmacro{\ElementKerf}{0.05*\FactorScale}
\pgfmathsetmacro{\ElementHeight}{0.3*\FactorScale}

% length of the connecting cables
\pgfmathsetmacro{\VoltageAmplitude}{0.4*\ElementWidth}
\pgfmathsetmacro{\VoltageDuration}{\ElementHeight / 5}

% height of axis ticks
\pgfmathsetmacro{\ElementTickHeight}{0.1}
\pgfmathsetmacro{\TimeTickWidth}{0.1}

% preferred direction of propagation
\pgfmathsetmacro{\Theta}{65}

% permutation matrix for the time delays / apodization weights
\def\ListIndicesPermuted{1,18,12,14,8,15,20,10,11,3,13,17,16,4,6,2,7,9,5,19}
\def\ListApodizationWeights{1,-1,1,-1,1,1,1,-1,1,-1,-1,1,-1,1,1,1,1,1,-1,-1}

% lateral distance between plots
\pgfmathsetmacro{\DistanceX}{0.5}

% parallel foreach loop template
\pgfset{
  foreach/parallel foreach/.style args={#1in#2via#3}{evaluate=#3 as #1 using {{#2}[#3-1]}},
}

% dependent parameters
\pgfmathsetmacro{\MElements}{(\NElements - 1) / 2}
\pgfmathsetmacro{\ElementPitch}{\ElementWidth + \ElementKerf}
\pgfmathsetmacro{\DirectionX}{cos(\Theta)}
\pgfmathsetmacro{\DirectionY}{sin(\Theta)}
\pgfmathsetmacro{\WavefrontX}{sin(\Theta)}
\pgfmathsetmacro{\WavefrontY}{-cos(\Theta)}
\pgfmathsetmacro{\SlopeWavefront}{- \DirectionX / \DirectionY}
\pgfmathsetmacro{\MaxLengthX}{\MElements * \ElementPitch * ( 1 + 2 * abs(\DirectionX) ) + \ElementPitch}
\pgfmathsetmacro{\MaxLengthY}{3.25}
\pgfmathsetmacro{\PosXRnd}{2*\MaxLengthX + \DistanceX}

% temporal axis
\pgfmathsetmacro{\tAxisPosX}{(-\MElements - 1) * \ElementPitch}
\pgfmathsetmacro{\tAxisPosYStart}{\ElementHeight/2}

\pgfmathsetmacro{\tAxisPosYZero}{2*\ElementHeight}
\pgfmathsetmacro{\tAxisPosYMax}{\tAxisPosYZero + (\NElements - 1) * \ElementPitch * abs(\DirectionX)}
\pgfmathsetmacro{\tAxisPosYCenter}{(\tAxisPosYZero + \tAxisPosYMax) / 2}

% reference position depends on the preferred direction of propagation
\ifthenelse{0 < 1}{%
  \pgfmathsetmacro{\PosXReference}{-\MElements * \ElementPitch}%
}{%
  \pgfmathsetmacro{\PosXReference}{\MElements * \ElementPitch}%
}

%---------------------------------------------------------------------------------------------------------------
% draw linear transducer array, connecting cables, and coordinate axes
%---------------------------------------------------------------------------------------------------------------
\newcommand{\drawarray}{%

  % coordinates of the coordinate axes
  \coordinate (r1AxisMin) at (-\MaxLengthX, 0);
  \coordinate (r1AxisMax) at (\MaxLengthX, 0);
  \coordinate (r2AxisMin) at (0, 3 * \ElementHeight / 2);
  \coordinate (r2AxisMax) at (0, -\MaxLengthY);
  \coordinate (tAxisMin) at (\tAxisPosX, \tAxisPosYStart);
  \coordinate (tAxisMax) at (\tAxisPosX, \tAxisPosYMax + \tAxisPosYZero - \tAxisPosYStart);
  \coordinate (tAxisLineZeroStart) at ({-(\MElements + 1) * \ElementPitch},\tAxisPosYZero);
  \coordinate (tAxisLineZeroStop) at ({(\MElements + 1) * \ElementPitch},\tAxisPosYZero);
  \coordinate (tAxisLineMaxStart) at ({-(\MElements + 1) * \ElementPitch},\tAxisPosYMax);
  \coordinate (tAxisLineMaxStop) at ({(\MElements + 1) * \ElementPitch},\tAxisPosYMax);

  % draw spatial coordinate axes
  \draw [->, thin] (r1AxisMin) -- (r1AxisMax);	% r1-axis
  \draw [->, thin] (r2AxisMin) -- (r2AxisMax);	% r2-axis

  % draw temporal coordinate axis
  \draw [->, thin] (tAxisMin) -- (tAxisMax);	% t-axis

  % draw ticks on temporal axis
  \draw [draw = gray20, draw opacity = 1, thin, dash pattern = on 2pt off 1.5pt] (tAxisLineZeroStart) -- (tAxisLineZeroStop);
  \draw [draw = gray20, draw opacity = 1, thin, dash pattern = on 2pt off 1.5pt] (tAxisLineMaxStart) -- (tAxisLineMaxStop);
  \draw [thin] (\tAxisPosX - \TimeTickWidth / 2, \tAxisPosYZero) -- (\tAxisPosX + \TimeTickWidth / 2, \tAxisPosYZero);
  \draw [thin] (\tAxisPosX - \TimeTickWidth / 2, \tAxisPosYMax) -- (\tAxisPosX + \TimeTickWidth / 2, \tAxisPosYMax);
  
  \foreach \x in {1,2,...,\NElements}{%

    % current center coordinate
    \pgfmathsetmacro{\PosXCenter}{(\x - 1 - \MElements) * \ElementPitch}

    % coordinates of the transducer elements
    \coordinate (ELB) at ({\PosXCenter - \ElementWidth / 2}, 0);
    \coordinate (ERU) at ({\PosXCenter + \ElementWidth / 2}, \ElementHeight);

    % draw connecting cables
    \draw [draw = gray20, draw opacity = 1, thin] (\PosXCenter, \ElementHeight) -- (\PosXCenter, \ElementHeight + \tAxisPosYMax);

    % draw transducer elements
    \draw [fill = gray20, fill opacity = 1, draw = black, draw opacity = 1, thin, text opacity = 1] (ELB) rectangle (ERU);

    % draw ticks
    \draw [thin] (\PosXCenter, -\ElementTickHeight/2) -- (\PosXCenter, \ElementTickHeight/2);
  }

  %--------------------------------------------------------------------------------------------------------------
  % labels
  %--------------------------------------------------------------------------------------------------------------
  % a) labels for spatial coordinate axes
  \node [below] at (r1AxisMax) {\footnotesize $\displaystyle r_{1}$};
  \node [right] at (r2AxisMax) {\footnotesize $\displaystyle r_{2}$};
  \node [below] at (-\MElements * \ElementPitch, 0) {\footnotesize $\displaystyle r_{\text{el}, 0, 1}$};
  \node [below] at (\MElements * \ElementPitch, 0) {\footnotesize $\displaystyle r_{\text{el}, N_{\text{el}} - 1, 1}$};

  % b) labels for temporal coordinate axis
  \node [left] at (tAxisMax) {\footnotesize $\displaystyle t$};
  \node [left] at (\tAxisPosX,\tAxisPosYZero) {\scriptsize $\displaystyle 0$};
  \node [left] at (\tAxisPosX,\tAxisPosYMax) {\scriptsize $\displaystyle \mathcal{Q} \bigl[ \Delta t_{\text{max}}^{(0)} \bigr]$};

  % c) labels for lossy homogeneous fluid
  \node [above left] at (\MaxLengthX, -\MaxLengthY) (LossyHomFluidLabel) {\footnotesize \parbox[t]{2.5cm}{Lossy homogeneous\\fluid $\displaystyle ( \kappa_{0}, \rho_{0}, \bar{b}, \zeta )$}};
  \draw [draw = black, draw opacity = 1, thin] (LossyHomFluidLabel.west) -- ($(LossyHomFluidLabel.west) - (0.75, 0.2)$);
}

%---------------------------------------------------------------------------------------------------------------
% steered QPW
%---------------------------------------------------------------------------------------------------------------
\begin{scope}

  %-------------------------------------------------------------------------------------------------------------
  % linear transducer array, connecting cables, and coordinate axes
  %-------------------------------------------------------------------------------------------------------------
  % draw linear transducer array, connecting cables, and coordinate axes
  \drawarray

  %-------------------------------------------------------------------------------------------------------------
  %
  %-------------------------------------------------------------------------------------------------------------
  % iterate transducer elements
  \foreach \x in {1,2,...,\NElements}{%

    % current center coordinate
    \pgfmathsetmacro{\PosXCenter}{(\x - 1 - \MElements) * \ElementPitch}

    % propagated distance
    \pgfmathsetmacro{\DeltaDistance}{(\PosXCenter - \PosXReference) * \DirectionX}

    % draw voltage pulse
    \pgfmathsetmacro{\PosYVoltage}{\tAxisPosYZero + \DeltaDistance}
    \draw [draw = black, draw opacity = 1, thin] (\PosXCenter, \PosYVoltage) --
						 (\PosXCenter - \VoltageAmplitude, \PosYVoltage) --
						 (\PosXCenter - \VoltageAmplitude, \PosYVoltage + \VoltageDuration) --
						 (\PosXCenter + \VoltageAmplitude, \PosYVoltage + \VoltageDuration) --
						 (\PosXCenter + \VoltageAmplitude, \PosYVoltage + 2*\VoltageDuration) --
						 (\PosXCenter, \PosYVoltage + 2*\VoltageDuration);

    % draw semicircle representing the emitted QCW
    \pgfmathsetmacro{\Radius}{ 2 * \MElements * \ElementPitch * abs(\DirectionX) - \DeltaDistance }
    \coordinate (EC) at (\PosXCenter - \Radius, 0);
    \draw [draw = gray40, draw opacity = 1, thin] (EC) arc (-180: 0: \Radius);
  }

  %--------------------------------------------------------------------------------------------------------------
  % approximated planar wavefront
  %--------------------------------------------------------------------------------------------------------------
  % a) draw approximated planar wavefront
  \pgfmathsetmacro{\PosXWavefrontStart}{(-\MElements - 2) * \ElementPitch}
  \pgfmathsetmacro{\PosXWavefrontStop}{(\MElements + 2) * \ElementPitch}
  \draw [draw = black, draw opacity = 1, thin, dash pattern = on 2pt off 1.5pt] (\PosXWavefrontStart, {-\SlopeWavefront * (\PosXWavefrontStart + \PosXReference)}) -- (\PosXWavefrontStop, {-\SlopeWavefront * (\PosXWavefrontStop + \PosXReference)});

  % b) draw longest distance to approximated planar wavefront
  \coordinate (LotStart) at (\PosXReference, 0);
  \coordinate (LotCenter) at (\PosXReference - \DirectionX * \PosXReference * \DirectionX, \DirectionX * \PosXReference * \DirectionY);
  \coordinate (LotStop) at (\PosXReference - 2 * \DirectionX * \PosXReference * \DirectionX, 2 * \DirectionX * \PosXReference * \DirectionY);
  \draw [<->, draw = black, draw opacity = 1, thin, dash pattern = on 2pt off 1.5pt] (LotStart) -- (LotStop);
  \node [left] at (LotCenter) {\footnotesize $\displaystyle \mathcal{Q} \bigl[ \Delta t_{\text{max}}^{(0)} \bigr] c_{\text{ref}}$};
  \draw [draw = black, draw opacity = 1, thin, dash pattern = on 2pt off 1.5pt] (\PosXReference - 2 * \DirectionX * \PosXReference * \DirectionX + 0.7*\DirectionX, {-\SlopeWavefront * (\PosXReference - 2 * \DirectionX * \PosXReference * \DirectionX + 0.7*\DirectionX + \PosXReference)}) arc ({90-\Theta}:{180-\Theta}:{0.7*\DirectionX});
  \node at ({\PosXReference - 2 * \DirectionX * \PosXReference * \DirectionX + 0.35*\DirectionX*cos(135-\Theta)}, {2 * \DirectionX * \PosXReference * \DirectionY + 0.35*\DirectionX*sin(135-\Theta)}) {\scriptsize $\cdot$};

  % c) draw preferred direction of propagation
  \coordinate (DirectionStart) at (-\MElements * \ElementPitch, {-\SlopeWavefront * (-\MElements * \ElementPitch + \PosXReference)});
  \coordinate (DirectionStop) at (-\MElements * \ElementPitch + \DirectionX, {-\SlopeWavefront * (-\MElements * \ElementPitch + \PosXReference) - \DirectionY});
  \draw [->, draw = black, draw opacity = 1, thick] (DirectionStart) -- (DirectionStop);
  \node [right] at (DirectionStop) {\footnotesize $\uvect{\vartheta}^{(0)}$};

  % d) draw angle \vartheta
  \draw [draw = black, draw opacity = 1, thin, dash pattern = on 2pt off 1.5pt] (-\MElements * \ElementPitch - \DirectionX, {-\SlopeWavefront * (-\MElements * \ElementPitch + \PosXReference)}) -- (-\MElements * \ElementPitch + \DirectionX, {-\SlopeWavefront * (-\MElements * \ElementPitch + \PosXReference)});
  \draw [draw = black, draw opacity = 1, thin, dash pattern = on 2pt off 1.5pt, arrows={->[black]}] (-\MElements * \ElementPitch + 0.7*\DirectionX, {-\SlopeWavefront * (-\MElements * \ElementPitch + \PosXReference)}) arc (0:-\Theta:{0.7*\DirectionX}) node[right] {\scriptsize \textcolor{black}{$\vartheta$}};

  % e) label approximated planar wavefront
  \node [right] at (0, {-\SlopeWavefront * (-\MElements * \ElementPitch + \PosXReference)}) (WavefrontLabel) {\footnotesize \parbox[t]{3cm}{Approximated\\planar wavefront}};
  \draw [draw = black, draw opacity = 1, thin] (WavefrontLabel.north west) -- ($(LotStop)!0.3!(\MElements * \ElementPitch,0)$);

  %--------------------------------------------------------------------------------------------------------------
  % labels
  %--------------------------------------------------------------------------------------------------------------
  % labels for excitation voltages
  \node [right] at (-\MaxLengthX,\tAxisPosYCenter) (ExcVolt) {\footnotesize \parbox[t]{1.2cm}{Excitation\\voltages}};
  \draw (ExcVolt.east) -- (- \MElements * \ElementPitch, \tAxisPosYCenter);

  %--------------------------------------------------------------------------------------------------------------
  % title
  %--------------------------------------------------------------------------------------------------------------
  % title type of emission
  \node [above right] at (-\MaxLengthX,\tAxisPosYMax + 0.7) {\phantomsubcaption\label{fig:lin_mod_exc_sup_qsw_emission_qpw}\sffamily(a) Steered \acl{QPW}};

\end{scope}

%---------------------------------------------------------------------------------------------------------------
% superposition of both randomly-apodized and randomly-delayed QCWs
%---------------------------------------------------------------------------------------------------------------
\begin{scope}[xshift=\PosXRnd cm]

  % draw linear transducer array, connecting cables, and coordinate axes
  \drawarray

  % iterate transducer elements
  \foreach \x [count=\c,parallel foreach=\y in \ListIndicesPermuted via \c,parallel foreach=\a in \ListApodizationWeights via \c] in {1,2,...,\NElements}{%

    % current center coordinate
    \pgfmathsetmacro{\PosXCenter}{(\x - 1 - \MElements) * \ElementPitch}
    \pgfmathsetmacro{\PosXCenterPermuted}{ \y - 1 - \MElements) * \ElementPitch}

    % propagated distance
    \pgfmathsetmacro{\DeltaDistance}{(\PosXCenter - \PosXReference) * \DirectionX}

    % draw voltage pulse
    \pgfmathsetmacro{\PosYVoltage}{\tAxisPosYZero + \DeltaDistance}
    \draw [draw = black, draw opacity = 1, thin] (\PosXCenterPermuted, \PosYVoltage) --
						 (\PosXCenterPermuted - \a * \VoltageAmplitude, \PosYVoltage) --
						 (\PosXCenterPermuted - \a * \VoltageAmplitude, \PosYVoltage + \VoltageDuration) --
						 (\PosXCenterPermuted + \a * \VoltageAmplitude, \PosYVoltage + \VoltageDuration) --
						 (\PosXCenterPermuted + \a * \VoltageAmplitude, \PosYVoltage + 2*\VoltageDuration) --
						 (\PosXCenterPermuted, \PosYVoltage + 2*\VoltageDuration);

    %\node [above] at (\PosXCenterPermuted, \PosYVoltage + 2*\VoltageDuration) {\footnotesize $\a$};

    % draw semicircle representing the emitted QCW
    \pgfmathsetmacro{\Radius}{ 2 * \MElements * \ElementPitch * abs(\DirectionX) - \DeltaDistance }
    \coordinate (EC) at (\PosXCenterPermuted - \Radius, 0);
    \ifthenelse{0 < \a}{%
      \draw [draw = gray40, draw opacity = 1, thin] (EC) arc (-180: 0: \Radius);%
    }{%
      \draw [draw = gray40, draw opacity = 1, thin, dash pattern = on 2pt off 1.5pt] (EC) arc (-180: 0: \Radius);%
    }
  }

  %--------------------------------------------------------------------------------------------------------------
  % synthesized irregular wavefront
  %--------------------------------------------------------------------------------------------------------------
  % a) draw longest distance to synthesized irregular wavefront
  \coordinate (LotStart) at (\PosXReference, 0);
  \coordinate (LotCenter) at (\PosXReference - \DirectionX * \PosXReference * \DirectionX, \DirectionX * \PosXReference * \DirectionY);
  \coordinate (LotStop) at (\PosXReference - 2 * \DirectionX * \PosXReference * \DirectionX, 2 * \DirectionX * \PosXReference * \DirectionY);
  \draw [<->, draw = black, draw opacity = 1, thin, dash pattern = on 2pt off 1.5pt] (LotStart) -- (LotStop);
  \node [left] at (LotCenter) {\footnotesize $\displaystyle \mathcal{Q} \bigl[ \Delta t_{\text{max}}^{(0)} \bigr] c_{\text{ref}}$};

  % b) label synthesized irregular wavefront
  \draw [draw = black, draw opacity = 1, thin] (0, {-\SlopeWavefront * (-\MElements * \ElementPitch + \PosXReference)}) -- (LotStop);
  \node [right] at (0, {-\SlopeWavefront * (-\MElements * \ElementPitch + \PosXReference)}) {\footnotesize \parbox[t]{3cm}{Synthesized\\irregular wavefront}};

  %--------------------------------------------------------------------------------------------------------------
  % labels
  %--------------------------------------------------------------------------------------------------------------
  % labels for excitation voltages
  \node [right] at (-\MaxLengthX,\tAxisPosYCenter) (ExcVolt) {\footnotesize \parbox[t]{1.2cm}{Weighted,\\permuted\\excitation\\voltages}};
  \draw (ExcVolt.east) -- (- \MElements * \ElementPitch, \tAxisPosYCenter);

  %--------------------------------------------------------------------------------------------------------------
  % title
  %--------------------------------------------------------------------------------------------------------------
  % title type of emission
  \node [above right] at (-\MaxLengthX,\tAxisPosYMax + 0.7) {\phantomsubcaption\label{fig:lin_mod_exc_sup_qsw_emission_rnd_apo_del}\sffamily (b) Realization of a proposed type of random wave};

\end{scope}

\end{tikzpicture}
