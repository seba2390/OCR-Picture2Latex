%---------------------------------------------------------------------------------------------------------------
% TikZ: schematic representation of the proposed physical model by a signal processing chain
%---------------------------------------------------------------------------------------------------------------
\begin{tikzpicture}

%---------------------------------------------------------------------------------------------------------------
% define size of the signal processing chain
%---------------------------------------------------------------------------------------------------------------
\pgfmathsetmacro{\Width}{16.5}
\pgfmathsetmacro{\Height}{5.75}

%---------------------------------------------------------------------------------------------------------------
% specification of coordinates
%---------------------------------------------------------------------------------------------------------------
% set sizes of the basic units
\pgfmathsetmacro{\FractionWidthBase}{33}
\pgfmathsetmacro{\FractionHeightBase}{23}

% position of bifurcation (amplitude identically exciting all transducer elements)
\pgfmathsetmacro{\FractionsWidthPosBi}{4}
\pgfmathsetmacro{\FractionsHeightPosBi}{12.5}

% position of multiplication (apodization weights and time delays)
\pgfmathsetmacro{\FractionsWidthACenter}{7.5}
\pgfmathsetmacro{\FractionsWidthARadius}{0.5}
\pgfmathsetmacro{\FractionsHeightACenterU}{18}
\pgfmathsetmacro{\FractionsHeightACenterL}{7}
\pgfmathsetmacro{\FractionsHeightAArrow}{2}

% position of multiplication (transmitter electromechanical transfer function)
\pgfmathsetmacro{\FractionsWidthBCenter}{16.5}
\pgfmathsetmacro{\FractionsWidthBRadius}{0.5}
\pgfmathsetmacro{\FractionsHeightBCenterU}{18}
\pgfmathsetmacro{\FractionsHeightBCenterL}{7}
\pgfmathsetmacro{\FractionsHeightBArrow}{2}

% position of multiplication (spatial transmit function)
\pgfmathsetmacro{\FractionsWidthCCenter}{21.5}
\pgfmathsetmacro{\FractionsWidthCRadius}{0.5}
\pgfmathsetmacro{\FractionsHeightCCenterU}{18}
\pgfmathsetmacro{\FractionsHeightCCenterL}{7}
\pgfmathsetmacro{\FractionsHeightCArrow}{2}

% position of addition (superposition of the incident quasi-(d-1)-spherical waves)
\pgfmathsetmacro{\FractionsWidthDCenter}{28}
\pgfmathsetmacro{\FractionsWidthDRadius}{1}
\pgfmathsetmacro{\FractionsHeightDCenter}{12.5}
\pgfmathsetmacro{\FractionsHeightDArrow}{2}

% position of system E (specified excitation voltages)
\pgfmathsetmacro{\FractionsWidthEStart}{3}
\pgfmathsetmacro{\FractionsWidthEStop}{12}
\pgfmathsetmacro{\FractionsHeightEStart}{0}
\pgfmathsetmacro{\FractionsHeightEStop}{23}

% position of system F (synthesis operator)
\pgfmathsetmacro{\FractionsWidthFStart}{15}
\pgfmathsetmacro{\FractionsWidthFStop}{30}
\pgfmathsetmacro{\FractionsHeightFStart}{0}
\pgfmathsetmacro{\FractionsHeightFStop}{23}

\pgfmathsetmacro{\FactorShiftDots}{0.6}

% line break symbol
\tikzset{ext/.pic={
\draw [draw = gray05, draw opacity = 1, fill = gray05, fill opacity = 1, thick] (-0.15, -0.15) -- (0.15, 0.05) -- (0.15, 0.15) -- (-0.15, -0.05) -- cycle;
\draw [thick] (-0.15, -0.15) -- (0.15, 0.05);
\draw [thick] (-0.15, -0.05) -- (0.15, 0.15);
}}

%---------------------------------------------------------------------------------------------------------------
% calculation of coordinates
%---------------------------------------------------------------------------------------------------------------
% bifurcation (amplitude identically exciting all transducer elements)
\pgfmathsetmacro{\PosBiHoriz}{\FractionsWidthPosBi * \Width / \FractionWidthBase}
\pgfmathsetmacro{\PosBiVert}{\FractionsHeightPosBi * \Height / \FractionHeightBase}

% multiplication (apodization weights and time delays)
\pgfmathsetmacro{\FractionsWidthAStart}{\FractionsWidthACenter - \FractionsWidthARadius}
\pgfmathsetmacro{\FractionsWidthAStop}{\FractionsWidthACenter + \FractionsWidthARadius}
\pgfmathsetmacro{\AHorizStart}{\FractionsWidthAStart * \Width / \FractionWidthBase}
\pgfmathsetmacro{\AHorizStop}{\FractionsWidthAStop * \Width / \FractionWidthBase}

\pgfmathsetmacro{\FractionsHeightAStartU}{\FractionsHeightACenterU - \FractionsWidthARadius * \FractionHeightBase / \FractionWidthBase * \Width / \Height}
\pgfmathsetmacro{\FractionsHeightAStopU}{\FractionsHeightACenterU + \FractionsWidthARadius * \FractionHeightBase / \FractionWidthBase * \Width / \Height}
\pgfmathsetmacro{\AVertStartU}{\FractionsHeightAStartU * \Height / \FractionHeightBase}
\pgfmathsetmacro{\AVertStopU}{\FractionsHeightAStopU * \Height / \FractionHeightBase}

\pgfmathsetmacro{\FractionsHeightAStartL}{\FractionsHeightACenterL - \FractionsWidthARadius * \FractionHeightBase / \FractionWidthBase * \Width / \Height}
\pgfmathsetmacro{\FractionsHeightAStopL}{\FractionsHeightACenterL + \FractionsWidthARadius * \FractionHeightBase / \FractionWidthBase * \Width / \Height}
\pgfmathsetmacro{\AVertStartL}{\FractionsHeightAStartL * \Height / \FractionHeightBase}
\pgfmathsetmacro{\AVertStopL}{\FractionsHeightAStopL * \Height / \FractionHeightBase}

\pgfmathsetmacro{\ARadius}{\FractionsWidthARadius * \Width / \FractionWidthBase}

\pgfmathsetmacro{\AWidth}{\AHorizStop - \AHorizStart}
\pgfmathsetmacro{\AHeight}{\AVertStopU - \AVertStartU}

\pgfmathsetmacro{\AHorizCenter}{\AHorizStart + \AWidth / 2}
\pgfmathsetmacro{\AVertCenterU}{\AVertStartU + \AHeight / 2}
\pgfmathsetmacro{\AVertCenterL}{\AVertStartL + \AHeight / 2}
\pgfmathsetmacro{\AVertCenterC}{(\AVertCenterU + \AVertCenterL) / 2}

\pgfmathsetmacro{\AVertArrowStartU}{\AVertStartU - \FractionsHeightAArrow * \Height / \FractionHeightBase}
\pgfmathsetmacro{\AVertArrowStartL}{\AVertStartL - \FractionsHeightAArrow * \Height / \FractionHeightBase}
\pgfmathsetmacro{\AVertArrowStartC}{\AVertArrowStartL + \FactorShiftDots * (\AVertArrowStartU - \AVertArrowStartL)}

% multiplication (transmitter electromechanical transfer function)
\pgfmathsetmacro{\FractionsWidthBStart}{\FractionsWidthBCenter - \FractionsWidthBRadius}
\pgfmathsetmacro{\FractionsWidthBStop}{\FractionsWidthBCenter + \FractionsWidthBRadius}
\pgfmathsetmacro{\BHorizStart}{\FractionsWidthBStart * \Width / \FractionWidthBase}
\pgfmathsetmacro{\BHorizStop}{\FractionsWidthBStop * \Width / \FractionWidthBase}

\pgfmathsetmacro{\FractionsHeightBStartU}{\FractionsHeightBCenterU - \FractionsWidthBRadius * \FractionHeightBase / \FractionWidthBase * \Width / \Height}
\pgfmathsetmacro{\FractionsHeightBStopU}{\FractionsHeightBCenterU + \FractionsWidthBRadius * \FractionHeightBase / \FractionWidthBase * \Width / \Height}
\pgfmathsetmacro{\BVertStartU}{\FractionsHeightBStartU * \Height / \FractionHeightBase}
\pgfmathsetmacro{\BVertStopU}{\FractionsHeightBStopU * \Height / \FractionHeightBase}

\pgfmathsetmacro{\FractionsHeightBStartL}{\FractionsHeightBCenterL - \FractionsWidthBRadius * \FractionHeightBase / \FractionWidthBase * \Width / \Height}
\pgfmathsetmacro{\FractionsHeightBStopL}{\FractionsHeightBCenterL + \FractionsWidthBRadius * \FractionHeightBase / \FractionWidthBase * \Width / \Height}
\pgfmathsetmacro{\BVertStartL}{\FractionsHeightBStartL * \Height / \FractionHeightBase}
\pgfmathsetmacro{\BVertStopL}{\FractionsHeightBStopL * \Height / \FractionHeightBase}

\pgfmathsetmacro{\BRadius}{\FractionsWidthBRadius * \Width / \FractionWidthBase}

\pgfmathsetmacro{\BWidth}{\BHorizStop - \BHorizStart}
\pgfmathsetmacro{\BHeight}{\BVertStopU - \BVertStartU}

\pgfmathsetmacro{\BHorizCenter}{\BHorizStart + \BWidth / 2}
\pgfmathsetmacro{\BVertCenterU}{\BVertStartU + \BHeight / 2}
\pgfmathsetmacro{\BVertCenterL}{\BVertStartL + \BHeight / 2}
\pgfmathsetmacro{\BVertCenterC}{(\BVertCenterU + \BVertCenterL) / 2}

\pgfmathsetmacro{\BVertArrowStartU}{\BVertStartU - \FractionsHeightBArrow * \Height / \FractionHeightBase}
\pgfmathsetmacro{\BVertArrowStartL}{\BVertStartL - \FractionsHeightBArrow * \Height / \FractionHeightBase}
\pgfmathsetmacro{\BVertArrowStartC}{\BVertArrowStartL + \FactorShiftDots * (\BVertArrowStartU - \BVertArrowStartL)}

% multiplication (spatial transmit function)
\pgfmathsetmacro{\FractionsWidthCStart}{\FractionsWidthCCenter - \FractionsWidthCRadius}
\pgfmathsetmacro{\FractionsWidthCStop}{\FractionsWidthCCenter + \FractionsWidthCRadius}
\pgfmathsetmacro{\CHorizStart}{\FractionsWidthCStart * \Width / \FractionWidthBase}
\pgfmathsetmacro{\CHorizStop}{\FractionsWidthCStop * \Width / \FractionWidthBase}

\pgfmathsetmacro{\FractionsHeightCStartU}{\FractionsHeightCCenterU - \FractionsWidthCRadius * \FractionHeightBase / \FractionWidthBase * \Width / \Height}
\pgfmathsetmacro{\FractionsHeightCStopU}{\FractionsHeightCCenterU + \FractionsWidthCRadius * \FractionHeightBase / \FractionWidthBase * \Width / \Height}
\pgfmathsetmacro{\CVertStartU}{\FractionsHeightCStartU * \Height / \FractionHeightBase}
\pgfmathsetmacro{\CVertStopU}{\FractionsHeightCStopU * \Height / \FractionHeightBase}

\pgfmathsetmacro{\FractionsHeightCStartL}{\FractionsHeightCCenterL - \FractionsWidthCRadius * \FractionHeightBase / \FractionWidthBase * \Width / \Height}
\pgfmathsetmacro{\FractionsHeightCStopL}{\FractionsHeightCCenterL + \FractionsWidthCRadius * \FractionHeightBase / \FractionWidthBase * \Width / \Height}
\pgfmathsetmacro{\CVertStartL}{\FractionsHeightCStartL * \Height / \FractionHeightBase}
\pgfmathsetmacro{\CVertStopL}{\FractionsHeightCStopL * \Height / \FractionHeightBase}

\pgfmathsetmacro{\CRadius}{\FractionsWidthCRadius * \Width / \FractionWidthBase}

\pgfmathsetmacro{\CWidth}{\CHorizStop - \CHorizStart}
\pgfmathsetmacro{\CHeight}{\CVertStopU - \CVertStartU}

\pgfmathsetmacro{\CHorizCenter}{\CHorizStart + \CWidth / 2}
\pgfmathsetmacro{\CVertCenterU}{\CVertStartU + \CHeight / 2}
\pgfmathsetmacro{\CVertCenterL}{\CVertStartL + \CHeight / 2}
\pgfmathsetmacro{\CVertCenterC}{(\CVertCenterU + \CVertCenterL) / 2}

\pgfmathsetmacro{\CVertArrowStartU}{\CVertStartU - \FractionsHeightCArrow * \Height / \FractionHeightBase}
\pgfmathsetmacro{\CVertArrowStartL}{\CVertStartL - \FractionsHeightCArrow * \Height / \FractionHeightBase}
\pgfmathsetmacro{\CVertArrowStartC}{\CVertArrowStartL + \FactorShiftDots * (\CVertArrowStartU - \CVertArrowStartL)}

% addition (superposition of incident quasi-(d-1)-spherical waves)
\pgfmathsetmacro{\FractionsWidthDStart}{\FractionsWidthDCenter - \FractionsWidthDRadius}
\pgfmathsetmacro{\FractionsWidthDStop}{\FractionsWidthDCenter + \FractionsWidthDRadius}
\pgfmathsetmacro{\DHorizStart}{\FractionsWidthDStart * \Width / \FractionWidthBase}
\pgfmathsetmacro{\DHorizStop}{\FractionsWidthDStop * \Width / \FractionWidthBase}

\pgfmathsetmacro{\FractionsHeightDStart}{\FractionsHeightDCenter - \FractionsWidthDRadius * \FractionHeightBase / \FractionWidthBase * \Width / \Height}
\pgfmathsetmacro{\FractionsHeightDStop}{\FractionsHeightDCenter + \FractionsWidthDRadius * \FractionHeightBase / \FractionWidthBase * \Width / \Height}
\pgfmathsetmacro{\DVertStart}{\FractionsHeightDStart * \Height / \FractionHeightBase}
\pgfmathsetmacro{\DVertStop}{\FractionsHeightDStop * \Height / \FractionHeightBase}

\pgfmathsetmacro{\DRadius}{\FractionsWidthDRadius * \Width / \FractionWidthBase}

\pgfmathsetmacro{\DWidth}{\DHorizStop - \DHorizStart}
\pgfmathsetmacro{\DHeight}{\DVertStop - \DVertStart}

\pgfmathsetmacro{\DHorizCenter}{\DHorizStart + \DWidth / 2}
\pgfmathsetmacro{\DVertCenter}{\DVertStart + \DHeight / 2}

% system E (specified excitation voltages)
\pgfmathsetmacro{\EHorizStart}{\FractionsWidthEStart * \Width / \FractionWidthBase}
\pgfmathsetmacro{\EHorizStop}{\FractionsWidthEStop * \Width / \FractionWidthBase}

\pgfmathsetmacro{\EVertStart}{\FractionsHeightEStart * \Height / \FractionHeightBase}
\pgfmathsetmacro{\EVertStop}{\FractionsHeightEStop * \Height / \FractionHeightBase}

\pgfmathsetmacro{\EWidth}{\EHorizStop - \EHorizStart}
\pgfmathsetmacro{\EHeight}{\EVertStop - \EVertStart}

\pgfmathsetmacro{\EHorizCenter}{\EHorizStart + \EWidth / 2}

\pgfmathsetmacro{\EVertCenter}{\EVertStart + \EHeight / 2}

% system F (synthesis operator)
\pgfmathsetmacro{\FHorizStart}{\FractionsWidthFStart * \Width / \FractionWidthBase}
\pgfmathsetmacro{\FHorizStop}{\FractionsWidthFStop * \Width / \FractionWidthBase}

\pgfmathsetmacro{\FVertStart}{\FractionsHeightFStart * \Height / \FractionHeightBase}
\pgfmathsetmacro{\FVertStop}{\FractionsHeightFStop * \Height / \FractionHeightBase}

\pgfmathsetmacro{\FWidth}{\FHorizStop - \FHorizStart}
\pgfmathsetmacro{\FHeight}{\FVertStop - \FVertStart}

\pgfmathsetmacro{\FHorizCenter}{\FHorizStart + \FWidth / 2}

\pgfmathsetmacro{\FVertCenter}{\FVertStart + \FHeight / 2}

% system A <-> system B
\pgfmathsetmacro{\ABHorizCenter}{(\AHorizStop + \BHorizStart) / 2}
\pgfmathsetmacro{\ABVertCenterU}{(\AVertCenterU + \BVertCenterU) / 2}
\pgfmathsetmacro{\ABVertCenterL}{(\AVertCenterL + \BVertCenterL) / 2}
\pgfmathsetmacro{\ABVertCenterC}{(\AVertCenterC + \BVertCenterC) / 2}

% system B <-> system C
\pgfmathsetmacro{\BCHorizCenter}{(\BHorizStop + \CHorizStart) / 2}
\pgfmathsetmacro{\BCVertCenterC}{(\BVertCenterC + \CVertCenterC) / 2}

% system C <-> system D
\pgfmathsetmacro{\CDHorizCenter}{(\CHorizStop + \DHorizCenter) / 2}

% system E <-> system F
\pgfmathsetmacro{\EFHorizCenter}{(\EHorizStop + \FHorizStart) / 2}

%---------------------------------------------------------------------------------------------------------------
% define coordinates
%---------------------------------------------------------------------------------------------------------------
% bifurcation (amplitude identically exciting all transducer elements)
\coordinate (Bi) at (\PosBiHoriz, \PosBiVert);
\coordinate (BiU) at (\PosBiHoriz, \AVertCenterU);
\coordinate (BiL) at (\PosBiHoriz, \AVertCenterL);

% coordinates for the multiplication (apodization weights and time delays)
\coordinate (ACU) at (\AHorizCenter, \AVertCenterU);
\coordinate (ACL) at (\AHorizCenter, \AVertCenterL);

% coordinates for the multiplication (transmitter electromechanical transfer function)
\coordinate (BCU) at (\BHorizCenter, \BVertCenterU);
\coordinate (BCL) at (\BHorizCenter, \BVertCenterL);

% coordinates for the multiplication (spatial transmit function)
\coordinate (DCU) at (\CHorizCenter, \CVertCenterU);
\coordinate (DCL) at (\CHorizCenter, \CVertCenterL);

% coordinates for the addition (superposition of incident quasi-(d-1)-spherical waves)
\coordinate (DC) at (\DHorizCenter, \DVertCenter);

% coordinates for system E (specified excitation voltages)
\coordinate (ELB) at (\EHorizStart, \EVertStart);
\coordinate (ELU) at (\EHorizStart, \EVertStop);
\coordinate (ERU) at (\EHorizStop, \EVertStop);

% coordinates for system F (synthesis operator)
\coordinate (FLB) at (\FHorizStart, \FVertStart);
\coordinate (FLU) at (\FHorizStart, \FVertStop);
\coordinate (FRU) at (\FHorizStop, \FVertStop);

% coordinates for arrows
\coordinate (ALU) at (\AHorizStart, \AVertCenterU);
\coordinate (ALL) at (\AHorizStart, \AVertCenterL);

\coordinate (XC) at (0, \AVertCenterC);
\coordinate (YC) at (\EHorizStart, \AVertCenterC);

\coordinate (ABU) at (\AHorizCenter, \AVertStartU);
\coordinate (ASU) at (\AHorizCenter, \AVertArrowStartU);

\coordinate (ABL) at (\AHorizCenter, \AVertStartL);
\coordinate (ASL) at (\AHorizCenter, \AVertArrowStartL);

\coordinate (ASC) at (\AHorizCenter, \AVertArrowStartC);

\coordinate (ARU) at (\AHorizStop, \AVertCenterU);
\coordinate (ARUBLUC) at (\EFHorizCenter, \ABVertCenterU);
\coordinate (BLU) at (\BHorizStart, \BVertCenterU);

\coordinate (ARL) at (\AHorizStop, \AVertCenterL);
\coordinate (ARLBLLC) at (\EFHorizCenter, \ABVertCenterL);
\coordinate (BLL) at (\BHorizStart, \BVertCenterL);

\coordinate (ARCBLCC) at (\EFHorizCenter, \ABVertCenterC);

\coordinate (BBU) at (\BHorizCenter, \BVertStartU);
\coordinate (BSU) at (\BHorizCenter, \BVertArrowStartU);

\coordinate (BBL) at (\BHorizCenter, \BVertStartL);
\coordinate (BSL) at (\BHorizCenter, \BVertArrowStartL);

\coordinate (BSC) at (\BHorizCenter, \BVertArrowStartC);

\coordinate (BRU) at (\BHorizStop, \BVertCenterU);
\coordinate (BRUCLUC) at (\BCHorizCenter, \BVertCenterU);
\coordinate (CLU) at (\CHorizStart, \CVertCenterU);

\coordinate (BRL) at (\BHorizStop, \BVertCenterL);
\coordinate (BRLCLLC) at (\BCHorizCenter, \BVertCenterL);
\coordinate (CLL) at (\CHorizStart, \CVertCenterL);

\coordinate (BRCCLCC) at (\BCHorizCenter, \BCVertCenterC);

\coordinate (CBU) at (\CHorizCenter, \CVertStartU);
\coordinate (CSU) at (\CHorizCenter, \CVertArrowStartU);

\coordinate (CBL) at (\CHorizCenter, \CVertStartL);
\coordinate (CSL) at (\CHorizCenter, \CVertArrowStartL);

\coordinate (CSC) at (\CHorizCenter, \CVertArrowStartC);

\coordinate (CRU) at (\CHorizStop, \CVertCenterU);
\coordinate (CRUDLUC) at (\CDHorizCenter, \CVertCenterU);
\coordinate (CCU) at (\CHorizCenter, \CVertCenterU);

\coordinate (CRL) at (\CHorizStop, \CVertCenterL);
\coordinate (CRLDLLC) at (\CDHorizCenter, \CVertCenterL);
\coordinate (CCL) at (\CHorizCenter, \CVertCenterL);

\coordinate (DCU) at (\DHorizCenter, \CVertCenterU);
\coordinate (DCL) at (\DHorizCenter, \CVertCenterL);

\coordinate (CRCDLCC) at (\CDHorizCenter, \CVertCenterC);

\coordinate (DT) at (\DHorizCenter, \DVertStop);
\coordinate (DB) at (\DHorizCenter, \DVertStart);
\coordinate (DR) at (\DHorizStop, \DVertCenter);

\coordinate (Z) at (\Width, \DVertCenter);

% coordinates for labels for domains
\coordinate (LAU) at (\AHorizCenter, \AVertCenterU);
\coordinate (LAL) at (\AHorizCenter, \AVertCenterL);
\coordinate (LBU) at (\BHorizCenter, \BVertCenterU);
\coordinate (LBL) at (\BHorizCenter, \BVertCenterL);
\coordinate (LCU) at (\CHorizCenter, \CVertCenterU);
\coordinate (LCL) at (\CHorizCenter, \CVertCenterL);
\coordinate (LD) at (\DHorizCenter, \DVertCenter);
\coordinate (LE) at (\EHorizCenter, \EVertStart);

%---------------------------------------------------------------------------------------------------------------
% draw systems
%---------------------------------------------------------------------------------------------------------------
% system E (specified excitation voltages)
\draw [fill = gray05, fill opacity = 1, draw = black, draw opacity = 1, rounded corners = 5, thick, text opacity = 1, dashed] (ELB) rectangle (ERU);

% system F (synthesis operator)
\draw [fill = gray05, fill opacity = 1, draw = black, draw opacity = 1, rounded corners = 5, thick, text opacity = 1, dashed] (FLB) rectangle (FRU);

% multiplication (apodization weights and time delays)
\draw [draw = black, draw opacity = 1, thick, text opacity = 1] (ACU) circle (\ARadius);
\draw [draw = black, draw opacity = 1, thick, text opacity = 1] (ACL) circle (\ARadius);

% multiplication (transmitter electromechanical transfer function)
\draw [draw = black, draw opacity = 1, thick, text opacity = 1] (BCU) circle (\BRadius);
\draw [draw = black, draw opacity = 1, thick, text opacity = 1] (BCL) circle (\BRadius);

% multiplication (spatial transmit function)
\draw [draw = black, draw opacity = 1, thick, text opacity = 1] (CCU) circle (\CRadius);
\draw [draw = black, draw opacity = 1, thick, text opacity = 1] (CCL) circle (\CRadius);

% addition (superposition of incident quasi-(d-1)-spherical waves)
\draw [draw = black, draw opacity = 1, thick, text opacity = 1] (DC) circle (\DRadius);

%---------------------------------------------------------------------------------------------------------------
% draw signals
%---------------------------------------------------------------------------------------------------------------
% bifurcation (amplitude identically exciting all transducer elements)
\draw [thick] (XC) -- (Bi);
\draw [->, >=stealth, thick] (Bi) -- (BiU) -- (ALU);
%\draw [->, thick] (Bi) -- node [fill = gray05, rotate = 90, pos = 0.5] (Cont) {$\backslash\backslash$} (BiL) -- (ALL);
%\path [draw] (Bi) -- pic [fill = gray05] {ext} (BiL) -- (ALL);
\draw [->, >=stealth, thick] (Bi) -- pic [-] {ext} (BiL) -- (ALL);

% apodization weights and time delays
\draw [->, >=stealth, thick] (ASU) -- (ABU);
\draw [->, >=stealth, thick] (ASL) -- (ABL);

% excitation voltages
\draw [->, >=stealth, thick] (ARU) -- (BLU);
\draw [->, >=stealth, thick] (ARL) -- (BLL);

% transmitter electromechanical transfer functions
\draw [->, >=stealth, thick] (BSU) -- (BBU);
\draw [->, >=stealth, thick] (BSL) -- (BBL);

% homogeneous r_{d}-component of the particle velocities
\draw [->, >=stealth, thick] (BRU) -- (CLU);
\draw [->, >=stealth, thick] (BRL) -- (CLL);

% spatial transmit functions
\draw [->, >=stealth, thick] (CSU) -- (CBU);
\draw [->, >=stealth, thick] (CSL) -- (CBL);

% individual incident acoustic pressure fields (quasi-(d-1)-spherical waves)
\draw [->, >=stealth, thick] (CRU) -- (DCU) -- (DT);
\draw [->, >=stealth, thick] (CRL) -- (DCL) -- (DB);
\draw [->, >=stealth, thick] (CRL) -- (DCL) -- pic [-,fill = gray05] {ext} (DB);

% synthesized incident acoustic pressure field (synthesized incident wave)
\draw [->, >=stealth, thick] (DR) -- (Z);

%---------------------------------------------------------------------------------------------------------------
% format options for subcaptions
%---------------------------------------------------------------------------------------------------------------
%\captionsetup[sub]{justification=raggedright,singlelinecheck=0,labelfont={normalsize,sf},textfont={normalsize,sf}}

%---------------------------------------------------------------------------------------------------------------
% labels
%---------------------------------------------------------------------------------------------------------------
% systems
\node [align=center] at (LAU) {$\displaystyle \times$};
\node [align=center] at (LAL) {$\displaystyle \times$};
\node [align=center] at (LBU) {$\displaystyle \times$};
\node [align=center] at (LBL) {$\displaystyle \times$};
\node [align=center] at (LCU) {$\displaystyle \times$};
\node [align=center] at (LCL) {$\displaystyle \times$};
\node [align=center] at (LD) {$\displaystyle \sum$};
\node [below right] at (ELU) {\phantomsubcaption\label{fig:syn_sup_qsw_sgn_proc_chain_v_tx}\sffamily (a) Voltage generation};
\node [below right] at (FLU) {\phantomsubcaption\label{fig:syn_sup_qsw_sgn_proc_chain_synthesis}\sffamily (b) Synthesis process};

% signals
% amplitude identically exciting all transducer elements
\node [above] at (XC) {$u_{l}^{(\text{tx}, n)}$};

% apodization weights and time delays
\node [below] at (ASU) {$a_{0}^{(n)} e^{ -j \omega_{l} \mathcal{Q} [ \Delta t_{0}^{(n)} ] }$};
\node at (ASC) {$\vdots$};
\node [below] at (ASL) {$a_{N_{\text{el}} - 1}^{(n)} e^{ -j \omega_{l} \mathcal{Q} [ \Delta t_{N_{\text{el}} - 1}^{(n)} ] }$};

% excitation voltages
\node [above] at (ARUBLUC) {$u_{0, l}^{(\text{tx}, n)}$};
\node at (ARCBLCC) {$\vdots$};
\node [above] at (ARLBLLC) {$u_{N_{\text{el}} - 1, l}^{(\text{tx}, n)}$};

% transmitter electromechanical transfer functions
\node [below] at (BSU) {$h_{0, l}^{(\text{tx})}$};
\node at (BSC) {$\vdots$};
\node [below] at (BSL) {$h_{N_{\text{el}} - 1, l}^{(\text{tx})}$};

% homogeneous r_{d}-component of the particle velocities
\node [above] at (BRUCLUC) {$v_{0, l, d}^{(n)}$};
\node at (BRCCLCC) {$\vdots$};
\node [above] at (BRLCLLC) {$v_{N_{\text{el}} - 1, l, d}^{(n)}$};

% spatial transmit functions
\node [below] at (CSU) {$j \omega_{l} \rho_{0} \varUpsilon_{0, l}^{(\text{tx})}( \vect{r} )$};
\node at (CSC) {$\vdots$};
\node [below] at (CSL) {$j \omega_{l} \rho_{0} \varUpsilon_{N_{\text{el}} - 1, l}^{(\text{tx})}( \vect{r} )$};

% individual incident acoustic pressure fields (quasi-(d-1)-spherical waves)
\node [above] at (CRUDLUC) {$p_{l}^{(\text{in}, n)}( \vect{r}, L_{0} )$};
\node at (CRCDLCC) {$\vdots$};
\node [above] at (CRLDLLC) {$p_{l}^{(\text{in}, n)}( \vect{r}, L_{N_{\text{el}} - 1} )$};

% synthesized incident acoustic pressure field (synthesized incident wave)
\node [above] at (Z) {$p_{l}^{(\text{in}, n)}( \vect{r} )$};

\end{tikzpicture}
