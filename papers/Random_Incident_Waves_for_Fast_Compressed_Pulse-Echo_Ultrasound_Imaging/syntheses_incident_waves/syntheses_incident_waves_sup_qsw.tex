%%%%%%%%%%%%%%%%%%%%%%%%%%%%%%%%%%%%%%%%%%%%%%%%%%%%%%%%%%%%%%%%%%%%%%%%%%%%%%%%%%%%%%%%%%%%%%%%%%%%%%%%%%%%%%%%
% graphic: block diagram of the syntheses of the incident waves
%%%%%%%%%%%%%%%%%%%%%%%%%%%%%%%%%%%%%%%%%%%%%%%%%%%%%%%%%%%%%%%%%%%%%%%%%%%%%%%%%%%%%%%%%%%%%%%%%%%%%%%%%%%%%%%%
\graphictwocols{syntheses_incident_waves/figures/latex/syn_inc_field_sup_qsw_sgn_proc_chain.tex}%
{% a) block diagram of the syntheses of the incident waves
 Block diagram of
 % 1.) syntheses of the incident waves
 the syntheses of
 the incident waves.
 % b) voltage generation maps the reference voltage signals to the excitation voltages
 Given
 % 1.) apodization weights
 the apodization weights
 $a_{m}^{(n)} \in \R$ and
 % 2.) time delays
 the time delays
 $\Delta t_{m}^{(n)} \in \Rnonneg$,
 % 3.) voltage generation
 the voltage generation
 (cf. \subref{fig:syn_sup_qsw_sgn_proc_chain_v_tx}) maps
 % 4.) reference voltage signals exciting all array elements
 the reference voltage signals
 $u_{l}^{(\text{tx}, n)} \in \C$ to
 % 5.) excitation voltages
 the excitation voltages
 \eqref{eqn:syn_p_in_types_v_tx}.
 % c) excitation voltages control the synthesis process that superimposes the quasi-(d-1)-spherical waves emitted by the individual elements of the planar transducer array to form various types of incident waves
 These voltages control
 % 1.) synthesis process
 the synthesis process
 (cf. \subref{fig:syn_sup_qsw_sgn_proc_chain_synthesis}) that superimposes
 % 2.) quasi-(d-1)-spherical waves
 the quasi-$(d-1)$-spherical waves emitted by
 the individual elements of
 the planar transducer array
 \eqref{eqn:syn_sup_qsw_p_in_qsw} to form
 % 3.) incident waves generated by the entire planar transducer array
 various types of
 incident waves
 \eqref{eqn:syn_sup_qsw_p_in}.
}%
{syn_sup_qsw_sgn_proc_chain}

%---------------------------------------------------------------------------------------------------------------
% 1.) incident acoustic pressure fields associated with a superposition of quasi-(d-1)-spherical waves
%---------------------------------------------------------------------------------------------------------------
% a) array elements transduce the compressive blocked forces exerted by the free-space scattered acoustic pressure fields on the planar faces into the RF voltage signals
% book:Schmerr2015, Sect. 9.1: Linear System Modeling and Sound Generation
% - Since the OUTPUT FORCE and VELOCITY on the face of an element are RELATED THROUGH THE ACOUSTIC RADIATION IMPEDANCE,
%   WE CAN MODEL
%   [1.)] THE DRIVING CIRCUITS,
%   [2.)] CABLING, AND
%   [3.)] SENDING ELEMENT AS
%   A SERIES OF TWO-PORT SYSTEMS TERMINATED AT BOTH PORTS, as shown in Fig. 9.6a, and
%   WE CAN REPLACE THIS SERIES OF SYSTEMS BY
%   A SINGLE INPUT, SINGLE OUTPUT LINEAR TIME-SHIFT INVARIANT (LTI) SYSTEM whose input is
%   the THÉVENIN EQUIVALENT VOLTAGE, V_{\text{i}}(\omega), OF THE DRIVING CIRCUITS. (p. 183)
% - On the OUTPUT SIDE, we can take either
%   the VELOCITY, v_{t}( ω ), or
%   the COMPRESSIVE FORCE ON THE FACE OF THE ARRAY ELEMENT, F_{t}( ω ), as
%   the QUANTITY TO DESCRIBE THE ACOUSTIC FIELDS. (pp. 183, 184)
% book:Schmerr2015, Sect. 1.3: Modeling Ultrasonic Phased Array Systems
% - During sound generation
%   the driving circuits produce a voltage pulse which travels down the wiring and cabling to
%   the element (usually a piezoelectric material) where
%   the electrical signals (voltage, current) are transformed into
%   acoustic pulses (force, velocity). (p. 8)
% - This TRANSFER FUNCTION is a function of
%   the electrical impedance of the DRIVING CIRCUITS,
%   the electrical properties of the WIRING/CABLING connecting
%   the driving circuits to the piezoelectric element, and
%   the electrical impedance and sensitivity of
%   the ARRAY ELEMENT [Schmerr-Song]. (p. 8)
% article:LabyedITUFFC2014: TR-MUSIC inversion of the density and compressibility contrasts of point scatterers
% II. The Interelement Response Matrix for Point Scatterers With Density and Compressibility Contrasts
% - The ULTRASOUND INCIDENT FIELD IS GENERATED BY AN ULTRASOUND TRANSDUCER ELEMENT, assuming no other sources exist in the medium. (p. 17)
% - In the classical theory of sound in a fluid that exhibits viscous loss,
%   the pressure phasor is given by [14]
%   [ p_{\text{inc}}( \vect{r}, ω ) = \frac{ i \munderbar{k}^{2} }{ ω \kappa_{0} } W_{\text{t}}( ω ) E( ω ) \int_{ S_{\text{t}} } \frac{ ... }{ ... } \text{d} s_{0} ] (6), where
%   the integral is evaluated over the surface of the transmitting element S_{\text{t}},
%   W_{\text{t}}( ω ) is the ELECTROMECHANICAL TRANSFER FUNCTION OF THE TRANSMITTER, and
%   E( ω ) is the INPUT-VOLTAGE TRANSFER FUNCTION. (p. 17)
% article:StepanishenJASA1981: Pulsed transmit/receive response of ultrasonic piezoelectric transducers
% I. BASIC APPROACH
% - TWO IMPORTANT TRANSFER FUNCTIONS OF INTEREST are defined here as follows:
%   (2a), (2b). (p. 1818)
% - The TRANSFER FUNCTION, T_{1}(ka), RELATES
%   the HARMONIC VELOCITY OF THE RADIATING SURFACE to
%   the APPLIED EXCITATION VOLTAGE whereas, T_{2}(ka), RELATES
%   the RECEIVE VOLTAGE AT THE ELECTRICAL TERMINALS OF THE TRANSDUCER TO
%   THE EXCITATION FORCE. (p. 1818)
% - Both transfer functions can be readily expressed in terms of the CASCADE PARAMETERS in Eq. (1); however,
%   the CASCADE PARAMETERS FOR THE ELECTRICAL NETWORK may differ for
%   the TRANSMIT AND RECEIVE CASES if a T/R switch is employed. (p. 1818)
% II. SPECIAL CASES OF INTEREST
% - The VELOCITY TRANSFER FUNCTION T_{1}(ka) is shown in Fig. 15(a) and
%   the RECEIVE VOLTAGE TRANSFER FUNCTION T_{2}(ka) is shown in Fig. 15(b). (p. 1822)
The array elements transduce
% 1.) excitation voltages
their excitation voltages
$u_{m, l}^{(\text{tx}, n)} \in \C$ into
% 2.) homogeneous r_{d}-component of the particle velocity
the homogeneous normal velocities on
% 3.) planar faces L_{m}
their planar faces
$L_{m} \subset \R^{d-1}$
\cite{article:LabyedITUFFC2014,article:NgITUFFC2006}.
%  $v_{m, l, d}^{(n)} \in \C$,
% b) RS diffraction equations uniquely solve the Helmholtz equation for the rigid baffle and represent the individual quasi-(d-1)-spherical waves by the incident acoustic pressure fields
% book:Cobbold2006, Sect. 2.1: Rayleigh-Sommerfeld Diffraction Equations
% - To DETERMINE THE PRESSURE OR VELOCITY FIELD PRODUCED BY
%   A VIBRATING SOURCE EXCITED BY AN ARBITRARY WAVEFORM, we
%   must solve the APPROPRIATE WAVE EQUATION that characterizes the manner in which the wave is propagated and
%   must CONSTRAIN THE SOLUTION TO MEET THE BOUNDARY CONDITIONS defined by the problem. (p. 97)
The \acl{RS} diffraction equations uniquely solve
% 1.) Helmholtz equations for the incident acoustic pressure fields
the \name{Helmholtz} equations
\eqref{eqn:lin_mod_sol_wave_eq_pde_p_in} for
% 2.) rigid baffle
these boundary conditions
%the rigid baffle
(cf. e.g.
\cite[(2.48)]{book:Devaney2012},		% "wavefield", d-dimensional space, dispersive homogeneous medium, temporal frequency domain [CHECKED: CORRECT! (Dirichlet, d=2,3)] (3-34)]
\cite[(13) of §8.11]{book:Born1999}%		% "wavefield", three-dimensional space, lossless homogeneous medium, temporal frequency domain [CHECKED: CORRECT! (12)]
) and represent
% 3.) individual quasi-(d-1)-spherical waves
the individual quasi-$(d-1)$-spherical waves by
% 4.) incident acoustic pressure fields [individual quasi-(d-1)-spherical waves]
the incident acoustic pressure fields
\begin{subequations}
\label{eqn:syn_sup_qsw_p_in_qsw}
\begin{equation}
 %--------------------------------------------------------------------------------------------------------------
 % a) incident acoustic pressure fields [individual quasi-(d-1)-spherical waves]
 %--------------------------------------------------------------------------------------------------------------
  p_{l}^{(\text{in}, n)}( \vect{r}, L_{m} )
  =
  j \omega_{l} \rho_{0}
  h_{m, l}^{(\text{tx})}
  u_{m, l}^{(\text{tx}, n)}
  \varUpsilon_{m, l}^{(\text{tx})}( \vect{r} )
 \label{eqn:syn_sup_qsw_p_in_qsw_exp}
\end{equation}
for
% 6.) all sequential pulse-echo measurements, all relevant discrete frequencies, and all array elements
all $( n, l, m ) \in \setconsnonneg{ N_{\text{in}} - 1 } \times \setsymbol{L}_{ \text{BP} }^{(n)} \times \setconsnonneg{ N_{\text{el}} - 1 }$, where
% 7.) transmitter electromechanical transfer functions
$h_{m, l}^{(\text{tx})} \in \C$ denote
the electromechanical transfer functions, and
% 8.) apodized spatial transmit functions
the apodized spatial transmit functions
\begin{equation}
 %--------------------------------------------------------------------------------------------------------------
 % b) apodized spatial transmit functions
 %--------------------------------------------------------------------------------------------------------------
  \varUpsilon_{m, l}^{(\text{tx})}( \vect{r} )
  =
  - 2
  \int_{ L_{m} }
    \chi_{m, l}^{(\text{tx})}( \vect{r}_{\rho}' )
    g_{l}( \vect{r}_{\rho} - \vect{r}_{\rho}', r_{d} )
  \text{d} \vect{r}_{\rho}'
 \label{eqn:syn_sup_qsw_p_in_qsw_spat_trans}
\end{equation}
\end{subequations}
characterize
% 9.) anisotropic directivities of the vibrating faces
the anisotropic directivities of
the vibrating faces for
% 10.) positive axial coordinate
all $r_{d} > 0$, similar to
% 11.) apodized spatial receive functions
the apodized spatial receive functions
\eqref{eqn:lin_mod_exc_sup_qsw_volt_rx_spat_trans}
(cf. \cref{tab:lin_mod_scan_config_instrum_params}).
%\begin{equation}
 %--------------------------------------------------------------------------------------------------------------
 % homogeneous r_{d}-component of the particle velocity on the planar face L_{m}
 %--------------------------------------------------------------------------------------------------------------
  %v_{m, l, d}^{(n)}
  %=
  %h_{m, l}^{(\text{tx})}
  %u_{m, l}^{(\text{tx}, n)},
 %\label{eqn:lin_mod_scan_config_trans_array_transfer_v_d}
%\end{equation}
% 1.) transmitter apodization functions
%the transmitter apodization functions
%$\chi_{m, l}^{(\text{tx})}: L_{m} \mapsto \C$ into %, which account for
% b) apodized spatial transmit functions relate the geometry of the vibrating faces and the effect of the acoustic lens to the velocity potential induced by a temporally-impulsive normal velocity
%They relate
% 1.) geometry of the vibrating faces
%the geometry of
%the vibrating faces and
% 2.) effect of the acoustic lens
%the effect of
%the acoustic lens to
% 3.) velocity potential induced by a temporally-impulsive normal velocity
%the velocity potential induced by
%a temporally-impulsive normal velocity.
% c) superpositions represent the incident waves emitted by the entire planar transducer array by the incident acoustic pressure fields
Their superpositions represent
% 1.) incident waves emitted by the entire planar transducer array
the incident waves by
% 2.) incident acoustic pressure fields [superpositions of quasi-(d-1)-spherical waves]
the acoustic pressure fields
\begin{equation}
 %--------------------------------------------------------------------------------------------------------------
 % incident acoustic pressure fields [superpositions of quasi-(d-1)-spherical waves]
 %--------------------------------------------------------------------------------------------------------------
  p_{l}^{(\text{in}, n)}( \vect{r} )
  =
  j \omega_{l} \rho_{0}
  \sum_{ m = 0 }^{ N_{\text{el}} - 1 }
    h_{m, l}^{(\text{tx})}
    u_{m, l}^{(\text{tx}, n)}
    \varUpsilon_{m, l}^{(\text{tx})}( \vect{r} )
 \label{eqn:syn_sup_qsw_p_in}
\end{equation}
for
% 3.) all sequential pulse-echo measurements and all relevant discrete frequencies
all $( n, l ) \in \setconsnonneg{ N_{\text{in}} - 1 } \times \setsymbol{L}_{ \text{BP} }^{(n)}$, where
% 4.) excitation voltages
the excitation voltages determine
% 5.) synthesized types of incident waves
the synthesized types of
incident waves
(cf. \cref{fig:syn_sup_qsw_sgn_proc_chain_synthesis}).
