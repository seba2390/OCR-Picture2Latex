%---------------------------------------------------------------------------------------------------------------
% 1.) apodization weights and time delays for the superpositions of randomly-delayed quasi-(d-1)-spherical waves
%---------------------------------------------------------------------------------------------------------------
% a) second type of random wave assigns random permutations of uniformly spaced time instants to the time delays in the excitation voltages
The second type of
random wave assigns
% 1.) random permutations of uniformly spaced time instants
random permutations of
uniformly spaced time instants to
the time delays in
% 2.) excitation voltages
the excitation voltages
\eqref{eqn:syn_p_in_types_v_tx}, whereas
% b) all apodization weights equal unity
all apodization weights equal
unity.
% c) apodization weights and time delays in the excitation voltages
These specifications yield
\begin{subequations}
\label{eqn:syn_p_in_types_v_tx_rnd_del}
\begin{align}
 %--------------------------------------------------------------------------------------------------------------
 % a) apodization weights
 %--------------------------------------------------------------------------------------------------------------
  a_{m}^{(n)}
  &=
  1
  & \text{and} & &
 %--------------------------------------------------------------------------------------------------------------
 % b) time delays
 %--------------------------------------------------------------------------------------------------------------
  \Delta t_{m}^{(n)}
  &=
  \dpermutel{ \setconsnonneg{ N_{\text{el}} - 1 } }{ m }{ n }{1}
  T_{\text{inc}}^{(n)}
 \label{eqn:syn_p_in_types_v_tx_rnd_del_apo_del}
\end{align}
for
% 1.) all sequential pulse-echo measurements and all transducer elements
all $( n, m ) \in \setconsnonneg{ N_{\text{in}} - 1 } \times \setconsnonneg{ N_{\text{el}} - 1 }$, where
% 2.) elements of index m in random permutations
$\tpermutel{ \setconsnonneg{ N_{\text{el}} - 1 } }{ m }{ n }$ denote
the elements of
index $m$ in
random permutations of
the set
$\setconsnonneg{ N_{\text{el}} - 1 }$ and
% 3.) fixed time periods
$T_{\text{inc}}^{(n)} \in \Rplus$ are
fixed time periods.
% d) fixed time periods significantly influence the properties of the incident waves and those of the observation operators
The latter significantly influence
the properties of
% 1.) incident acoustic pressure fields generated by the entire planar transducer array
the superpositions and, consequently,
those of
% 2.) pulse-echo measurement process
the pulse-echo measurement process
\eqref{eqn:lin_mod_v_rx_born_obs_proc}.

%---------------------------------------------------------------------------------------------------------------
% 2.) specifications of the fixed time periods
%---------------------------------------------------------------------------------------------------------------
% a) superpositions converge to steered QPWs with the preferred directions of propagation \uvect{d}
In
the limiting cases
$T_{\text{inc}}^{(n)} \rightarrow 0+$,
the superpositions converge to
% 1.) steered QPWs
steered \acp{QPW} with
% 2.) preferred directions of propagation
the preferred directions of
propagation
$\uvect{\vartheta}^{(n)} = \uvect{d}$, because
% 3.) apodization weights and time delays in the excitation voltages (superpositions of randomly-delayed quasi-(d-1)-spherical waves)
the apodization weights and
the time delays in
\eqref{eqn:syn_p_in_types_v_tx_rnd_del_apo_del} converge to
% 4.) apodization weights and time delays in the excitation voltages (steered QPWs)
those in
\eqref{eqn:syn_p_in_types_v_tx_qpw}.
% b) fixed time periods smaller than the upper bounds in the specified observation time intervals induce range ambiguities that can be resolved by the proposed method
%\TODO{notwendig? ordentlich!}
% TODO: move to discussion!
%Fixed time periods smaller than
%the round-trip times-of-flight (TOFs)
%the upper bounds in
% 1.) specified observation time intervals for the received RF voltage signals
%the specified observation time intervals for
%the received \ac{RF} voltage signals
%\eqref{eqn:lin_mod_scan_config_volt_rx_obs_interval}, i.e.
%$T_{\text{inc}}^{(n)} < t_{\text{ub}}^{(n)}$, induce
%range ambiguities that
%are undesired in
%the established image recovery methods but
%can be resolved by
%the proposed method.
%Since the choice of
%a fixed time interval
%$T_{\text{inc}}$ equal to or exceeding
%the upper bound in
%% 1.) specified observation time interval for the received RF voltage signals
%the specified observation time interval for
%the received \ac{RF} voltage signals
%\eqref{eqn:lin_mod_scan_config_volt_rx_obs_interval}, i.e.
%$T_{\text{inc}} \geq t_{\text{ub}}^{(n)}$, results in
%% 1.) complete SA acquisition sequence
%the complete \ac{SA} acquisition sequence,
% c) specific fixed time periods result in the syntheses times of the steered QPWs
The specific fixed time periods
\begin{equation}
\begin{split}
 %--------------------------------------------------------------------------------------------------------------
 % c) fixed time periods permuting the time delays for steered QPWs
 %--------------------------------------------------------------------------------------------------------------
  T_{\text{inc}}^{(n)}
  &=
  \frac{ 1 }{ N_{\text{el}} - 1 }
  \underset{ m \in \setconsnonneg{ N_{\text{el}} - 1 } }{ \max }
  \left\{
    \frac{
      \dinprod{ \vect{r}_{\text{el}, m} - \vect{r}_{\text{ref}}^{(n)} }{ \uvect{\vartheta}^{(n)} }{1}
    }{
      c_{\text{ref}}
    }
  \right\}\\
  &=
  \frac{ 1 }{ N_{\text{el}} - 1 }
  \sum_{ \delta = 1 }^{ d - 1 }
    ( N_{\text{el}, \delta} - 1 )
    \frac{
      \Delta r_{\text{el}, \delta}
      \dabs{ \uvectcomp{ \vartheta }{ \delta }^{(n)} }{1}
    }{
      c_{\text{ref}}
    }
\end{split}
\label{eqn:syn_p_in_types_v_tx_rnd_del_interval}
\end{equation}
\end{subequations}
result in
% 1.) syntheses times
the syntheses times of
% 2.) steered QPWs
the steered \acp{QPW}.
% d) specific fixed time periods induce random permutations of the time delays specified for the steered QPWs (two-dimensional Euclidean space)
In
the two-dimensional Euclidean space, i.e. $d = 2$,
they simplify to
$T_{\text{inc}}^{(n)} = \Delta r_{\text{el}, 1} \tabs{ \uvectcomp{ \vartheta }{ 1 }^{(n)} } / c_{\text{ref}}$ and induce
random permutations of
the time delays specified for
% 2.) steered QPWs
the steered \acp{QPW} in
\eqref{eqn:syn_p_in_types_v_tx_qpw}.
