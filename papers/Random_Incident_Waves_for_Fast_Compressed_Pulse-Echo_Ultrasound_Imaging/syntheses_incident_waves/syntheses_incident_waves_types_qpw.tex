%%%%%%%%%%%%%%%%%%%%%%%%%%%%%%%%%%%%%%%%%%%%%%%%%%%%%%%%%%%%%%%%%%%%%%%%%%%%%%%%%%%%%%%%%%%%%%%%%%%%%%%%%%%%%%%%
% graphic: emission of a steered quasi-plane wave (QPW) by a linear transducer array (two-dimensional Euclidean space)
%%%%%%%%%%%%%%%%%%%%%%%%%%%%%%%%%%%%%%%%%%%%%%%%%%%%%%%%%%%%%%%%%%%%%%%%%%%%%%%%%%%%%%%%%%%%%%%%%%%%%%%%%%%%%%%%
\graphictwocols{syntheses_incident_waves/figures/latex/syn_inc_field_sup_qsw_emission.tex}%
{%--------------------------------------------------------------------------------------------------------------
 % 1.) general description
 %--------------------------------------------------------------------------------------------------------------
 % a) syntheses of a steered QPW and a superposition of both randomly-apodized and randomly-delayed QCWs in the two-dimensional Euclidean space
 Syntheses of
 % 1.) steered QPW
 a steered \acf{QPW}\acused{QPW}
 (cf. \subref{fig:lin_mod_exc_sup_qsw_emission_qpw}) and
 % 2.) superposition of both randomly-apodized and randomly-delayed QCWs
 a proposed type of
 random wave
 (cf. \subref{fig:lin_mod_exc_sup_qsw_emission_rnd_apo_del}) in
 % 3.) two-dimensional Euclidean space
 the two-dimensional Euclidean space, i.e. $d = 2$.
 % b) gray semicircles represent the individual QCWs emitted by each element of the linear transducer array
 The gray semicircles represent
 the individual \acfp{QCW}\acused{QCW} emitted by
 each element of
 the linear transducer array.
 % c) dashed semicircles indicate negatively-apodized QCWs
 Their dashed variants indicate
 negatively-apodized \acp{QCW}, i.e.
 $a_{m}^{(0)} = - 1$ in
 \eqref{eqn:syn_p_in_types_v_tx_rnd_apo_del}, whereas
 % d) solid semicircles indicate positively-apodized QCWs
 their solid variants indicate
 positively-apodized \acp{QCW}, i.e.
 $a_{m}^{(0)} = 1$ in
 \eqref{eqn:syn_p_in_types_v_tx_qpw} and
 \eqref{eqn:syn_p_in_types_v_tx_rnd_apo_del}.
 % f) distinct radii reflect the time delays
 Their distinct radii reflect
 the time delays in
 \eqref{eqn:syn_p_in_types_v_tx_qpw} and
 \eqref{eqn:syn_p_in_types_v_tx_rnd_apo_del}, whose
 % g) maximum time delays
 maxima equal
 $\Delta t_{\text{max}}^{(0)} = \max_{ m \in \setconsnonneg{ N_{\text{el}} - 1 } }\{ \Delta t_{m}^{(0)} \} = ( N_{\text{el}} - 1 ) \Delta r_{\text{el}, 1} \tabs{ \uvectcomp{ \vartheta }{ 1 }^{(0)} } / c_{\text{ref}}$. % = ( N_{\text{el}} - 1 ) T_{\text{inc}}^{(n)}$.
 %, at
 % 4.) time instant \Delta t_{\text{max}}
 %the time instant
 %$t = \mathcal{Q}[ \Delta t_{\text{max}} ]$.
 %--------------------------------------------------------------------------------------------------------------
 % 2.) steered QPW
 %--------------------------------------------------------------------------------------------------------------
 % a) dashed black line indicates the approximated planar wavefront
 The dashed black line indicates
 the approximated planar wavefront.
 % b) black arrow shows the preferred direction of propagation
 The black arrow shows
 the preferred direction of
 propagation.
 %$\uvect{\vartheta}^{(0)} = \trans{ ( \cos( \vartheta ), \sin( \vartheta ) ) }$ with
 %$\vartheta = 13 \pi / 36 = \SI{65}{\degree}$.
 % c) positive r_{1}-component requires the reference position to equal r_{\text{el}, 0, 1} \uvect{1}
 %Its positive $r_{1}$-component requires
 % 1.) components of the reference positions for the time delays
 %the reference position with
 %the components
 %\eqref{eqn:syn_p_in_types_v_tx_qpw_r_ref} to equal
 %$\vect{r}_{\text{ref}}^{(0)} = r_{\text{el}, 0, 1} \uvect{1}$.
}%
{lin_mod_exc_sup_qsw_emission}

%---------------------------------------------------------------------------------------------------------------
% 1.) apodization weights and time delays for steered QPWs
%---------------------------------------------------------------------------------------------------------------
% a) steered QPWs denote the approximations of steered PWs synthesized by the UI system
Steered \acp{QPW} denote
the approximations of
steered \acp{PW} synthesized by
the \ac{UI} system.
% b) time delays in the excitation voltages depend affine-linearly on the center coordinates of the vibrating faces on each coordinate axis
The time delays in
% 1.) excitation voltages
the excitation voltages
\eqref{eqn:syn_p_in_types_v_tx} depend affine-linearly on
% 2.) center coordinates of the vibrating faces
the center coordinates of
the vibrating faces on
each coordinate axis, whereas
% c) all apodization weights equal unity
all apodization weights equal
unity.
% d) apodization weights and time delays in the excitation voltages
% book:Schmerr2015, Sect. 8.1 Beam Steering in 3-D
% - Consider an element of a 2-D array as shown in Fig. 8.1 where we want to steer the ultrasonic beam of
%   the array in the direction of the unit vector, u. (p. 169)
% - Steering of the beam in this direction can be accomplished by applying a LINEARLY VARYING TIME SHIFT,
%   ∆t= u*x / c, over the face of the array and evaluating that phase at the centroids of the individual elements. (p. 169)
% - Since the delays in Eq. (8.1) contain both positive and negative values, we can simply add a constant delay equal to
%   the magnitude of largest negative value to obtain a proper time delay law, ∆t_{mn}^{d}, given by (8.4). (p. 169)
These specifications yield
\begin{subequations}
\label{eqn:syn_p_in_types_v_tx_qpw}
\begin{align}
 %--------------------------------------------------------------------------------------------------------------
 % a) apodization weights
 %--------------------------------------------------------------------------------------------------------------
  a_{m}^{(n)}
  &=
  1
  & \text{and} & &
 %--------------------------------------------------------------------------------------------------------------
 % b) time delays
 %--------------------------------------------------------------------------------------------------------------
  \Delta t_{m}^{(n)}
  &=
  \frac{
    \dinprod{ \vect{r}_{\text{el}, m} - \vect{r}_{\text{ref}}^{(n)} }{ \uvect{\vartheta}^{(n)} }{1}
  }{
    c_{\text{ref}}
  }
 \label{eqn:syn_p_in_types_v_tx_qpw_apo_del}
\end{align}
for
% 1.) all sequential pulse-echo measurements and all array elements
all $( n, m ) \in \setconsnonneg{ N_{\text{in}} - 1 } \times \setconsnonneg{ N_{\text{el}} - 1 }$, where
% 2.) center coordinates of the vibrating faces
$\vect{r}_{\text{el}, m} \in \mathcal{M}$ denote
the center coordinates of
the vibrating faces,
% 3.) preferred directions of propagation
$\uvect{\vartheta}^{(n)} = \trans{ ( \uvectcomp{ \vartheta }{ 1 }^{(n)}, \dotsc, \uvectcomp{ \vartheta }{ d }^{(n)} ) } \in \uhemisphere{d-1}$ indicate
the preferred directions of
propagation, and
% 4.) reference positions for the time delays and their components
$\vect{r}_{\text{ref}}^{(n)} = \trans{ ( r_{\text{ref}, 1}^{(n)}, \dotsc, r_{\text{ref}, d-1}^{(n)}, 0 ) }$ are
the reference positions with
the components
\begin{equation}
 %--------------------------------------------------------------------------------------------------------------
 % c) components of the reference positions for the time delays
 %--------------------------------------------------------------------------------------------------------------
  r_{\text{ref}, \delta}^{(n)}
  =
  \begin{cases}
    - M_{\text{el}, \delta} \Delta r_{\text{el}, \delta} & \text{for } \uvectcomp{ \vartheta }{ \delta }^{(n)} \geq 0,\\
      M_{\text{el}, \delta} \Delta r_{\text{el}, \delta} & \text{for } \uvectcomp{ \vartheta }{ \delta }^{(n)} < 0,
  \end{cases}
 \label{eqn:syn_p_in_types_v_tx_qpw_r_ref}
\end{equation}
\end{subequations}
for
% 5.) all coordinate axes
all $\delta \in \setcons{ d - 1 }$
(cf. \cref{tab:lin_mod_scan_config_instrum_params}).
% e) components of the reference positions ensure the nonnegativity of the time delays and the causality of the voltage generation
The latter ensure
% 1.) nonnegativity of the time delays
the nonnegativity of
the time delays and, thus,
% 2.) causality of the voltage generation
the causality of
the voltage generation.
% f) finite number of elements and their anisotropic directivities limit the accuracies of the approximations to a bounded area in front of the array
% article:Schiffner2018, Sect. I. Introduction (sec:introduction)
% - The FINITE NUMBER OF EVENLY SPACED ARRAY ELEMENTS and their ANISOTROPIC DIRECTIVITIES, for example, prevent
%   the exact syntheses of
%   [1.)] NON-DIFFRACTING BEAMS, whose spatial extent and amount of transferred energy are unlimited, and
%   [2.)] outgoing $(d-1)$-spherical waves, whose isotropic sources are points.
The finite number of
elements and
their anisotropic directivities limit
the accuracies of
the approximations to
bounded areas in front of
the array
(cf. \cref{fig:lin_mod_exc_sup_qsw_emission_qpw}).
