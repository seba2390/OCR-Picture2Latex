%---------------------------------------------------------------------------------------------------------------
% 1.) three major innovations of the proposed method
%---------------------------------------------------------------------------------------------------------------
% - Be sure to clearly state the purpose and /or hypothesis that you investigated.
% - When you are first learning to write in this format it is okay, and actually preferable, to use a pat statement like,
%   "The purpose of this study was to...." or
%   "We investigated three possible mechanisms to explain the ... (1) blah, blah..(2) etc.
% - It is most usual to place the statement of purpose near
%   THE END OF THE INTRODUCTION, often as the
%   TOPIC SENTENCE OF THE FINAL PARAGRAPH.
%
% - Provide a CLEAR STATEMENT OF THE RATIONALE FOR YOUR APPROACH to the problem studied.
% - State BRIEFLY HOW YOU APPROACHED THE PROBLEM (e.g., you studied oxidative respiration pathways in isolated mitochondria of cauliflower).
%   This will usually follow your statement of purpose in the last paragraph of the Introduction.
% - Do not discuss here the actual techniques or protocols used in your study (this will be done in the Materials and Methods);
%   your readers will be quite familiar with the usual techniques and approaches used in your field.
% - If you are using a novel (new, revolutionary, never used before) technique or methodology,
%   the merits of the new technique/method versus the previously used methods should be presented in the Introduction.
% a) method for the fast compressed acquisition and the subsequent recovery of images is proposed that features three major innovations
A method for
% 1.) fast compressed acquisition
the fast compressed acquisition and
% 2.) subsequent recovery
the subsequent recovery of
images is proposed that features
% 3.) three major innovations
three major innovations.
% b) proposed method significantly enhances current inverse scattering methods by realistic d-dimensional physical models
First,
% 1.) realistic d-dimensional physical models
realistic $d$-dimensional physical models for
% 2.) linear physical model for the pulse-echo measurement process
the pulse-echo measurement process and
% 3.) syntheses of the incident waves
the syntheses of
the incident waves minimize
% 4.) minimize model inaccuracies
inaccuracies and leverage
% 5.) abilities
the abilities of
% 6.) programmable UI systems
programmable \ac{UI} systems.
% c) proposed method recovers the compressibility fluctuations by a sparsity-promoting lq-minimization method
%   What are the SCIENTIFIC MERITS of this particular model system?
They linearly relate
% 1.) spatial compressibility fluctuations
the spatial compressibility fluctuations in
% 2.) weakly-scattering soft tissue structures
weakly-scattering soft tissue structures to
% 3.) recorded RF voltage signals
the \ac{RF} voltage signals provided by
% 4.) instrumentation
the instrumentation.
% d) proposed physical models are universal and permit additional applications beyond ultrafast UI
They readily support
% 1.) calibration procedures
calibration procedures,
% 5.) usage
the usage of
% 6.) measured incident fields
measured incident fields, and
% 7.) applications
applications beyond
% 8.) ultrafast UI
ultrafast \ac{UI}, e.g.
% 5.) conventional UI based on progressive scanning
progressive scanning,
% 6.) structured insonification
structured insonification, or
% 7.) simulation studies
simulation studies.
% a) three innovative types of random incident waves aid in meeting condition (ii)
Second,
% 1.) three innovative types of energy equivalent random waves
three innovative types of
energy equivalent random waves are synthesized using
% 2.) pseudo-random apodization weights
random apodization weights,
% 3.) random time delays
time delays, or
% 4.) combinations thereof
combinations thereof.
% b) associated spatial codes decorrelate the pulse echoes of the structural building blocks defined by an orthonormal basis and facilitate their discrimination in the image recovery
The associated
% 1.) spatial codes
spatial codes decorrelate
% 2.) pulse echoes
the pulse echoes of
% 3.) structural building blocks
the structural building blocks defined by
% 4.) orthonormal basis
an orthonormal basis meeting
% 5.) condition (i) [known dictionary of structural building blocks represents the image almost sparsely]
condition (i) and, thus, improve
% 6.) conformity
the conformity with
% 7.) condition (ii): [individual pulse echoes are sufficiently uncorrelated]
condition (ii).
% c) convex and nonconvex variants of a sparsity-promoting lq-minimization method enable the quantitative recovery of the compressibility fluctuations
Third, both
% 1.) convex
convex and
% 2.) nonconvex
nonconvex variants of
% 3.) sparsity-promoting lq-minimization method
a sparsity-promoting $\ell_{q}$-minimization method, $q \in [ 0; 1 ]$, enable
% 4.) quantitative recovery
the quantitative recovery of
% 5.) compressibility fluctuations
the compressibility fluctuations.

%---------------------------------------------------------------------------------------------------------------
% 2.) results of the simulation study
%---------------------------------------------------------------------------------------------------------------
% - Why did you choose this kind of experiment or experimental design?
% a) numerical simulation of a pulse-echo in the two-dimensional Euclidean space validates the emissions of single random incident waves
Two-dimensional numerical simulations validate
the proposed method using
single realizations of
the random waves for
% 1.) wire phantom
a wire phantom and
% 2.) tissue-mimicking phantom
a tissue-mimicking phantom.
% -> basis transform: Fourier
% d) Liu et al. require on the order of 30 sequential pulse-echo measurements per image, whereas the method proposed in this paper only requires the minimum number of a single wave emission
%In contrast to Liu and Kruizinga,
%the proposed method only requires
%the minimum number of
%a single pulse-echo measurement per
%image and uses additional types of
%random waves.
% b) former phantom permits a sparse representation of its spatial compressibility fluctuations in the canonical basis, whereas the latter phantom requires the Fourier basis
%The former phantom permits
%a sparse representation of
%its spatial compressibility fluctuations in
% 1.) canonical basis
%the canonical basis, whereas
%the latter phantom requires
% 2.) Fourier basis
%the \name{Fourier} basis.
% TODO: insight into the transfer behavior of the UI system
% TODO: amount of information collected by the array through different excited spatial frequencies is increased compared to conventional waves
% c) random incident waves decreased the robustness to
%   What ADVANTAGES DOES IT CONFER in answering the particular question(s) you are posing?
Although
the random waves decrease
% 1.) robustness against additive errors
the robustness against
additive errors for
the wire phantom,
they significantly increase both
% 2.) image quality
the image quality and
% 3.) speed of convergence
the convergence speed for
the tissue-mimicking phantom.
% a) author published two abstracts outlining the fundamental ideas of this paper in connection with oral presentations at two conferences
The study significantly expands
the initial results published in
two abstracts
\cite{proc:SchiffnerIUS2017,article:SchiffnerJASA2017}.
%% c) recovery experiments confirm better performance
%Numerical recovery experiments additionally demonstrate
%% 1.) increased SSIM indices
%increased \ac{SSIM} indices,
%% 2.) reduced relative RMSEs
%reduced relative \acp{RMSE}, and
%% 3.) faster convergence
%faster convergence.

%---------------------------------------------------------------------------------------------------------------
% 3.) results of the experimental validation
%---------------------------------------------------------------------------------------------------------------
% For the experimental validation of
% the random incident waves,
% we acquired pulse-echo measurement data from
% a real phantom consisting of nine wires.


%%%%%%%%%%%%%%%%%%%%%%%%%%%%%%%%%%%%%%%%%%%%%%%%%%%%%%%%%%%%%%%%%%%%%%%%%%%%%%%%%%%%%%%%%%%%%%%%%%%%%%%%%%%%%%%%
% table: summary of the mathematical symbols used throughout the paper
%%%%%%%%%%%%%%%%%%%%%%%%%%%%%%%%%%%%%%%%%%%%%%%%%%%%%%%%%%%%%%%%%%%%%%%%%%%%%%%%%%%%%%%%%%%%%%%%%%%%%%%%%%%%%%%%
\begin{table*}[tb]
 \centering
 \caption{%
  Summary of
  the mathematical symbols used
  throughout the paper.
 }
 \label{tab:list_symbols_math}
 \small
 \begin{tabular}{%
  @{}%
  >{$}l<{$}%		01.) symbol
  p{0.9\textwidth}%	02.) meaning
  @{}%
 }
 \toprule
  \multicolumn{1}{@{}H}{Symbol} &
  \multicolumn{1}{H@{}}{Meaning}\\
  \cmidrule(r){1-1}\cmidrule(l){2-2}
 \addlinespace
 %--------------------------------------------------------------------------------------------------------------
 % a) number sets
 %--------------------------------------------------------------------------------------------------------------
  % 1.) set of consecutive positive integers
  \setcons{ N } &
  Set of consecutive positive integers,
  $\setcons{ N } = \{ 1, 2, \dotsc, N \}$ for $N \in \N$\\
  % 2.) set of consecutive nonnegative integers
  \setconsnonneg{ N } &
  Set of consecutive nonnegative integers,
  $\setconsnonneg{ N } = \{ 0, 1, \dotsc, N \}$ for $N \in \Nnonneg$\\
 %--------------------------------------------------------------------------------------------------------------
 % b) inner product
 %--------------------------------------------------------------------------------------------------------------
  % 1.) inner product
  \inprod{ \vect{a} }{ \vect{b} } &
  Inner product of
  the vectors
  $\vect{a} = \trans{ ( a_{1}, \dotsc, a_{N} ) } \in \C^{ N }$ and
  $\vect{b} = \trans{ ( b_{1}, \dotsc, b_{N} ) } \in \C^{ N }$,
  $\inprod{ \vect{a} }{ \vect{b} } = \sum_{ n = 1 }^{ N } a_{n} \conj{ b }_{n}$\\
 %--------------------------------------------------------------------------------------------------------------
 % c) norms, quasinorms, and NNZC
 %--------------------------------------------------------------------------------------------------------------
  % 1.) lq-norm or lq-quasinorm
  \tnorm{ \vect{a} }{q} &
  $\ell_{q}$-norm, $q \in [ 1; \infty )$, or
  $\ell_{q}$-quasinorm, $q \in ( 0; 1 )$, of
  the vector
  $\vect{a} \in \C^{ N }$,
  $\tnorm{ \vect{a} }{q}^{q} = \sum_{ n = 1 }^{ N } \tabs{ a_{n} }^{q}$\\
  % 2.) number of nonzero components
  % book:Foucart2013: A Mathematical Introduction to Compressive Sensing / Chapter 2: Sparse Solutions of Underdetermined Systems / Sect. 2.1: Sparsity and Compressibility
  % - The customary notation \tnorm{ \vect{x} }{0} - the notation \tnorm{ \vect{x} }{0}^{0} would in fact be more appropriate - comes from the observation that
  %   \norm{ \vect{x} }{p}^{p} = \sum_{ j = 1 }^{ N } \abs{ x_{j} }^{p} \rightarrow \sum_{ j = 1 }^{ N } \indicator{ x_{j} \neq 0 }. (p. 41)
  % - In other words the quantity \tnorm{ \vect{x} }{0} is
  %   the limit as p decreases to zero of the pth power of the \ell_{p}-quasinorm of x. (p. 42)
  \tnorm{ \vect{a} }{0} &
  Number of nonzero components,
  $\tnorm{ \vect{a} }{0} := \tnorm{ \vect{a} }{0}^{0} = \lim_{q \rightarrow 0} \tnorm{ \vect{a} }{q}^{q} = \tabs{ \{ n \in \setcons{ N }: a_{n} \neq 0 \} }$\\
 %--------------------------------------------------------------------------------------------------------------
 % c) spatial coordinates
 %--------------------------------------------------------------------------------------------------------------
  % 1.) spatial position
  \vect{r} &
  Spatial position in
  the $d$-dimensional Euclidean space,
  $\vect{r} = \trans{ ( r_{1}, \dotsc, r_{d} ) } \in \R^{d}$\\
  % 2.) lateral coordinates and axial coordinate
  \vect{r}_{\rho}, r_{d} &
  % 2.1) lateral coordinates
  Lateral coordinates
  $\vect{r}_{\rho} = \trans{ ( r_{1}, \dotsc, r_{d-1} ) } \in \R^{d-1}$ and
  % 2.2) axial coordinate
  axial coordinate
  $r_{d} \in \R$ of
  the spatial position
  $\vect{r} = \trans{ ( \trans{ \vect{r}_{\rho} }, r_{d} ) }$\\
  % 3.) unit (d-1)-sphere
  \usphere{d-1} &
  Unit $(d-1)$-sphere,
  $\usphere{d-1} = \{ \vect{r} \in \R^{d}: \norm{ \vect{r} }{2} = 1 \}$\\
  % 4.) unit (d-1)-hemisphere
  \uhemisphere{d-1} &
  Unit $(d-1)$-hemisphere,
  $\uhemisphere{d-1} = \{ \vect{r} \in \R^{d}: \norm{ \vect{r} }{2} = 1, r_{d} \in \Rplus \}$\\
  % 5.) unit vector indicates the direction of the r_{\delta}-axis in a d-dimensional Cartesian coordinate system
  \uvect{\delta} &
  Unit vector indicating
  the direction of
  the $r_{\delta}$-axis, $\delta \in \setcons{ d }$, in
  a $d$-dimensional Cartesian coordinate system,
  $\uvect{\delta} \in \usphere{d-1}$\\
 %--------------------------------------------------------------------------------------------------------------
 % d) matrices
 %--------------------------------------------------------------------------------------------------------------
  % 1.) superscript H indicates an adjoint matrix
  ^{\hermsymbol} &
  Superscript indicating
  an adjoint (conjugate transpose) matrix\\
  % 2.) I denotes the identity matrix
  \mat{I} &
  Identity matrix\\
 %--------------------------------------------------------------------------------------------------------------
 % e) signal
 %--------------------------------------------------------------------------------------------------------------
  % 1.) tilde symbol identifies time-domain signals
  % TODO: recorded \ac{RF} voltage signals in the time domain.
  \tilde{ \cdot } &
  Tilde accent indicating
  a time-domain signal\\%, e.g. voltage signals
 \addlinespace
 \bottomrule
 \end{tabular}
\end{table*}

%---------------------------------------------------------------------------------------------------------------
% 3.) structure of the presentation and list of mathematical symbols
%---------------------------------------------------------------------------------------------------------------
% a) contributions are organized as follows
These contributions are organized as
follows.
% 2.) compressed sensing
\Cref{sec:compressed_sensing} briefly reviews
the \ac{CS} framework.
% 3.) linear physical model for the pulse-echo measurement process
% 4.) syntheses of the incident waves
\Cref{sec:linear_model,sec:syn_p_in} present
the physical models for
the pulse-echo measurement process and
the syntheses of
the incident waves.
% 5.) image recovery based on compressed sensing
\Cref{sec:recovery} details
the image recovery based on
\ac{CS}, and
% 6.) implementation
\cref{sec:implementation} adds
an efficient matrix-free implementation.
% 7.) simulation study
% 8.) experimental validation
%\cref{sec:experimental_validation}
% 9.) results
\Cref{sec:simulation_study} summarizes
the parameters of
the numerical simulations, and
\cref{sec:results} presents
the results.
% 10.) discussion
\Cref{sec:discussion} discusses
these results and
the proposed method.
% 11.) conclusion and outlook
Eventually,
\cref{sec:conclusion_outlook} concludes
the paper.
% b) table summarizes the mathematical symbols
\Cref{tab:list_symbols_math} summarizes
the mathematical symbols.
