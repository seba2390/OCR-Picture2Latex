%---------------------------------------------------------------------------------------------------------------
% 1.) advantages of medical ultrasound imaging, programmable UI systems, and ultrafast imaging modes
%---------------------------------------------------------------------------------------------------------------
% a) medical UI is a safe, cost-effective, portable, and frequently-used modality that achieves a high resolution in real-time
% book:Szabo2013, Chapter 1: Introduction / Sect. 1.11: Conclusion
% - With the exception of standard X-ray exams,
%   ULTRASOUND is the [4.)] LEADING IMAGING MODALITY WORLDWIDE and in the United States. (p. 34)
% - The ULTRASOUND GROWTH shown by Fig. 1.15, can be explained by
%   a NUMBER OF DYNAMIC FACTORS given by Table 1.2. (p. 35)
% - Ultrasound, on the other hand, has
%   [5.)] HIGH BUT VARIABLE RESOLUTION AND PENETRATION and
%   [x.)] LIMITED ACCESS TO CERTAIN PORTIONS OF THE BODY (intestines, lungs, and bones). (p. 34)
% - Ultrasound is used to image SOFT TISSUES ONLY, but it has the advantages of
%   [2.)] LOW COST, [3.)] PORTABILITY, and [6.)] REAL-TIME INTERACTIVE AND INTERVENTIONAL IMAGING; and
%   its ABILITY TO IMAGE LARGER 3D VOLUMES IS IMPROVING. (p. 34)
% - Other ADVANTAGES OF ULTRASOUND DIAGNOSTIC IMAGING include
%   [1.)] SAFETY, flexibility, adaptability, [3.)] PORTABILITY, mobility, and [2.)] REDUCED COST, which increase
%   the range of potential applications and users. (p. 35)
% book:Szabo2013, Chapter 1: Introduction / Sect. 1.8: Ultrasound and Other Diagnostic Imaging Modalities / Sect. 1.8.2: Ultrasound
% - Two other strengths of diagnostic ultrasound imaging are its
%   [2.)] RELATIVELY LOW COST and [3.)] PORTABILITY. (p. 27)
% book:Szabo2013, Chapter 1: Introduction / Sect. 1.8: Ultrasound and Other Diagnostic Imaging Modalities / Sect. 1.8.1: Imaging Modalities Compared
% - Ultrasound, because of its EFFICACY AND [2.)] LOW COST, is often the [4.)] PREFERRED IMAGING MODALITY. (p. 23)
% - EACH MAJOR DIAGNOSTIC IMAGING METHOD IS EXAMINED in the following sections, and
%   the OVERALL RESULTS ARE TALLIED IN TABLE 1.2 AND COMPARED IN FIGURE 1.15. (p. 24)
% - The first edition of this book had a GRAPH COMPARING DIFFERENT IMAGING MODALITIES FOR THE YEAR 2000;
%   in Figure 1.15, this data is compared with 2011 NUMBERS. (pp. 24, 26)
% - While other imaging modalities made modest gains,
%   DIAGNOSTIC ULTRASOUND GREW BY AN ORDER OF MAGNITUDE in the DECADE since the first edition of this book! (p. 26)
% - Reasons for this change are explored in Section 1.11. (p. 26)
% - Figure 1.15 Comparison of ESTIMATED NUMBER OF IMAGING EXAMS GIVEN WORLDWIDE FOR THE YEARS 2000 AND 2011 (data courtesy of Daniel Cote).
%   CT, computed tomography; DXR, digital X-ray; iXR, interventional X-ray; MRI, magnetic resonance imaging; Nuc Med, nuclear medicine;
%   PET, positron emission tomography; US, diagnostic ultrasound. (p. 25)
% - Table 1.2 Comparison of Imaging Modalities (p. 25)
% article:BierigJDMS2009: Accuracy and Cost Comparison of Ultrasound Versus Alternative Imaging Modalities, Including CT, MR, PET, and Angiography
% Background
% - In addition, imaging technologies using magnetic resonance (MR) imaging, computed tomography (CT), contrast angiography (CA), and
%   single-photon emission computed tomography (SPECT) are often
%   [2.)] CONSIDERABLY MORE EXPENSIVE, include [1.)] RADIATION EXPOSURE, are [3.)] LESS PORTABLE, or
%   have an increased risk of [1.)] COMPLICATIONS FROM CONTRAST MEDIA.[12] (p. 139)
% Conclusion
% - US provides the ability to RAPIDLY EVALUATE AND DIAGNOSE ABNORMALITIES throughout the spectrum of clinical medicine. (p. 142)
% - Its [ultrasound (US) imaging] ACCURACY and [2.)] COST-EFFECTIVENESS in a variety of applications have led to its [4.)] WIDE-SPREAD ADOPTION and USE. (p. 142)
% - The use of US compared to the use of alternative imaging methods leads to [2.)] INCREASED COST-EFFICIENCY in the DIAGNOSIS and MANAGEMENT of patients. (p. 142)
% article:JensenProgBMB2007: Medical ultrasound imaging
% 1. Introduction
% - Ultrasound has since then EVOLVED TO BE ONE OF THE [4.)] MAJOR MEDICAL IMAGING SYSTEMS and is used in
%   NEARLY ALL HOSPITALS AND CLINICS for diagnostic purposes due to its EASE OF USE AND [1.)] SAFETY. (p. 154)
% 3. Anatomic ultrasound imaging
% - The technique is [4.)] WIDELY USED, since it does not use ionizing radiation and is [1.)] SAFE AND PAINLESS for the patient. (p. 155)
% coll:Jensen2002: Ultrasound Imaging and Its Modeling
% ABSTRACT
% - The technique is [4.)] WIDELY USED, since it does not use ionizing radiation and is [1.)] SAFE AND PAINLESS for the patient. (p. 135)
\IEEEPARstart{M}{edical} \ac{UI} is
% 1.) safe
% book:Szabo2013, Chapter 1: Introduction / Sect. 1.8: Ultrasound and Other Diagnostic Imaging Modalities / Sect. 1.8.2: Ultrasound
% - Ultrasound is regarded as [1.)] SAFE and does not have any cumulative biological side effects (more on this topic can be found in Chapter 15). (p. 27)
a safe,
% 2.) cost-effective
cost-effective,
% 3.) portable
% book:Szabo2013, Chapter 1: Introduction / Sect. 1.8: Ultrasound and Other Diagnostic Imaging Modalities / Sect. 1.8.2: Ultrasound
% - With the widespread availability of
%   [3.)] MINIATURE PORTABLE and POCKET ultrasound systems for SCREENING and IMAGING,
%   these two factors will continue to improve. (p. 27)
portable, and
% 4.) frequently-used
frequently-used modality that provides
% 5.) sub-millimeter spatial resolutions
sub-millimeter spatial resolutions in
% 6.) real-time
% book:Szabo2013, Chapter 1: Introduction / Sect. 1.11: Conclusion
% - KEY STRENGTHS of ULTRASOUND are its abilities to reveal
%   ANATOMY, the DYNAMIC MOVEMENT OF ORGANS, and DETAILS OF BLOOD FLOW IN REAL TIME. (p. 34)
% book:Szabo2013, Chapter 1: Introduction / Sect. 1.8: Ultrasound and Other Diagnostic Imaging Modalities / Sect. 1.8.2: Ultrasound
% - The DYNAMIC MOTION OF ORGANS such as the heart can be revealed by ultrasound operating up to HUNDREDS of FRAMES PER SECOND. (p. 26)
% article:JensenProgBMB2007: Medical ultrasound imaging
% 3. Anatomic ultrasound imaging
% - The imaging is performed in REAL TIME with 20–100 images/s. (p. 155)
% - The available frame rate is closely linked to the image depth. (p. 155)
% - Dynamic imaging is, thus, possible, and the anatomy and its dynamics can be easily visualized in REAL TIME. (p. 155)
% coll:Jensen2002: Ultrasound Imaging and Its Modeling
% ABSTRACT
% - The imaging is performed in [6.)] REAL TIME with 20 to 100 images per second. (p. 135)
real-time
\cite[Fig. 1.15, Table 1.2]{book:Szabo2013},
\cite{article:BierigJDMS2009}.
% b) typical progressive scanning of a specified FOV by focused beams requires hundreds of sequential pulse-echo measurements per image and limits the frame rate
% article:GarciaITUFFC2013: Stolt's f-k migration for plane wave ultrasound imaging
% I. Introduction
% - Conventional medical ultrasound imaging consists in
%   SCANNING A MEDIUM USING A SERIES OF SUCCESSIVE FOCUSED OR MULTI-FOCUSED BEAMS
%   SWEEPING ALONG THE REGION OF INTEREST. (p. 1853)
% - The RESULTING SCANLINES are then stacked together to reconstruct a single image. (p. 1853)
% - The time required to build one frame is thus proportional to
%   [1.)] the NUMBER OF GATHERED LINES and
%   [2.)] the MAXIMAL IMAGING DEPTH. (p. 1853)
% book:Bushberg2011, Chapter 14: Ultrasound / Sect. 14.6: Two-Dimensional Image Display and Storage
% - A 2D ultrasound image is acquired by
%   [1.)] SWEEPING A PULSED ULTRASOUND BEAM OVER THE VOLUME OF INTEREST and
%   [2.)] displaying echo signals using B-mode conversion of the A-mode signals. (p. 536)
% - The 2D image is PROGRESSIVELY BUILT UP or CONTINUOUSLY UPDATED as the beam is swept through the object. (p. 536)
% book:Bushberg2011, Chapter 14: Ultrasound / Sect. 14.6: Two-Dimensional Image Display and Storage / Electronic Scanning and Real-Time Display
% - State-of-the-art ultrasound scanners employ
%   [1.)] ARRAY TRANSDUCERS WITH MULTIPLE PIEZOELECTRIC ELEMENTS to
%   [2.)] ELECTRONICALLY SWEEP AN ULTRASOUND BEAM ACROSS THE VOLUME OF INTEREST for dynamic ultrasound imaging. (p. 537)
% - FIGURE 14-31 Linear and curvilinear array transducers produce an image by activating
%   a SUBGROUP OF THE TRANSDUCER ELEMENTS that form one A-line of data in the scanned object and shifting
%   the active elements by one to acquire the next line of data. [...] (p. 538)
% - A SMALL GROUP OF ADJACENT ELEMENTS (~15 to 20) IS SIMULTANEOUSLY ACTIVATED TO CREATE AN ACTIVE TRANSDUCER AREA defined by
%   the width (sum of the individual element widths in the group) and the height of the elements. (p. 537)
% - A SHIFT OF ONE OR MORE TRANSDUCER ELEMENTS and
%   REPEATING THE SIMULTANEOUS EXCITATION OF THE GROUP PRODUCE THE NEXT A-LINE OF DATA. (p. 537)
% - Electronic delays within the subgroup of transducer elements allow transmit and dynamic receive focusing for improved lateral resolution with depth. (pp. 537, 538)
% - Phased-array transducers are typically comprised of a tightly grouped array of 64, 128, or 256 transducer elements in a 3- to 5-cm-wide enclosure. (p. 538)
% - ALL TRANSDUCER ELEMENTS ARE INVOLVED IN PRODUCING THE ULTRASOUND BEAM AND RECORDING THE RETURNING ECHOES. (p. 538)
% - The ultrasound beam is steered by adjusting the delays applied to the individual transducer elements by the beam former. (p. 538)
% - FIGURE 14-32 The phased-array transducer electronically steers the ultrasound beam by introducing phase delays during
%   the transmit and receive timing of the beam former.
%   Lateral focusing also occurs along the beam direction. A sector format composite image is produced (right), with
%   the number of A-lines dependent on several imaging factors discussed in the text. (p. 538)
% - A similar time-delay strategy (dynamic receive focusing) is used to spatially synchronize
%   the returning ultrasound echoes as they strike each transducer element (see Fig. 14-26). (p. 539)
% - In addition to the beam steering capabilities, lateral focusing, multiple transmit focal zones, and dynamic receive focusing are used in phased arrays. (p. 539)
% article:JensenUlt2006: Synthetic aperture ultrasound imaging
% Abstract
% - [...] today’s commercial systems, where the IMAGE IS ACQUIRED SEQUENTIALLY ONE IMAGE LINE AT A TIME. (p. e5)
% article:WellsPMB2006: Ultrasound imaging
% 2. Mainstream technologies / 2.2. Beam forming
% - The simplest arrangement is a linear array, within which
%   AN APERTURE IS FORMED FROM, say, 16 CONTIGUOUS ELEMENTS and which is
%   STEPPED ALONG THE ARRAY ELEMENT BY ELEMENT TO ACQUIRE AN IMAGE WITH, in this example, 241 LINES. (p. R86)
% - Also, the number of elements in the active aperture can be dynamically increased with increasing depth of penetration to
The typical progressive scanning of
% 1.) specified FOV
a specified \ac{FOV} by
% 2.) focused beams
focused beams, however, requires
% 3.) hundreds of sequential pulse-echo measurements per image
% book:Bushberg2011, Chapter 14: Ultrasound / Sect. 14.6: Two-Dimensional Image Display and Storage / Electronic Scanning and Real-Time Display
% - The ULTRASOUND BEAM SWEEPS ACROSS THE VOLUME OF INTEREST IN A SEQUENTIAL FASHION, with
%   the NUMBER OF A-LINES APPROXIMATELY EQUAL TO THE NUMBER OF TRANSDUCER ELEMENTS. (p. 537)
% article:JensenProgBMB2007: Medical ultrasound imaging
% 3. Anatomic ultrasound imaging
% - N_{l} [number of lines in the image] is typically 200 lines and D = 10cm results in f_{r} = 38.5 Hz. (p. 155)
hundreds of
sequential pulse-echo measurements per
image and, owing to
% 4.) finite sound speed
the finite sound speed, limits
% 5.) frame rate
% article:JensenUlt2006: Synthetic aperture ultrasound imaging
% Abstract
% - This [linewise acquisition] puts a strict limit on
%   [1.)] the FRAME RATE and
%   [2.)] the possibility of acquiring a sufficient amount of data for HIGH PRECISION FLOW ESTIMATION. (p. e5)
the frame rate
\cite[536--539]{book:Bushberg2011},
\cite{article:JensenProgBMB2007,article:WellsPMB2006}.
% c) advances in electronic miniaturization and processing power have recently led to freely programmable UI systems and software-based ultrafast imaging modes that overcome this limitation
% article:IlovitshNatComBio2018: Acoustical structured illumination for super-resolution ultrasound imaging
% - Advances in ultrasound technologies have now led to user-programmable systems, capable of
%   [1.)] a nearly infinite variety of transmitted pulse trains, and
%   [2.)] schemes for image reconstruction. (p. 2)
% article:TanterITUFFC2014: Ultrafast imaging in biomedical ultrasound
% I. Introduction
% - TECHNOLOGICAL ADVANCES within the last decade
%   (parallel computing, high-performance data transfer, high-speed processors, etc.) now allow
%   production of ultrasound scanners able to generate
%   A FULL IMAGE FROM A SINGLE TRANSMIT, thus offering the opportunity for ULTRAFAST IMAGING. (p. 1853)
% book:Szabo2013, Chapter 1: Introduction / Sect. 1.11: Conclusion
% - Unlike the relatively stable or incremental improvements of the other imaging modalities,
%   ULTRASOUND IS ACQUIRING SIGNIFICANTLY ENHANCED IMAGING CAPABILITIES in several areas due to
%   the SYNERGISTIC EFFECT OF DISRUPTIVE TECHNOLOGIES (see Table 1.1). (p. 34)
% - Diagnostic ultrasound continues to evolve by improving in
%   diagnostic capability, image quality, convenience, ease of use, image transfer and management, and portability. (p. 34)
% - The MOST DRAMATIC CHANGES have been through
%   [1.)] the CONTINUAL MINIATURIZATION OF ELECTRONICS in accordance with
%   [2.)] a MODIFIED MOORE'S LAW and NEW ARCHITECTURES. (p. 35)
% - SMALLER-SIZED COMPONENTS led to the first commercially available phased-array imaging systems as well as to
%   new, pocket imaging systems, which weigh less than 0.5 kg. (p. 35)
% - A REVISED MOORE'S LAW and recent VERTICAL CHIP INTEGRATION indicate that imminent manufacturing advances will bring
%   HIGHER CHIP COMPLEXITY IN SMALLER PACKAGES AND AT LOWER COSTS. (p. 35)
% - [1.)] NEW ARCHITECTURES such as those with GRAPHICAL PROCESSING UNITS and
%   [2.)] NEW IMAGING PARADIGMS are rapidly INCREASING SPATIAL AND TEMPORAL RESOLUTION and providing
%   more diagnostic information at greater speeds (see Chapters 10 and 11);
%   because of the time lag of technology implementation,
%   the latest developments have not had their full impact on ultrasound imaging. (p. 35)
% - The combination of
%   [1.)] continual improvements in electronics and
%   [2.)] a better understanding of the interaction of ultrasound with tissues will lead to
%   imaging systems of increased complexity. (p. 35)
Advances in
% 1.) electronic miniaturization
electronic miniaturization and
% 2.) processing power
processing power have recently led to
% 3.) freely programmable UI systems
freely programmable \ac{UI} systems and
% 4.) software-based ultrafast imaging modes
software-based \term{ultrafast} imaging modes, e.g.
% 5.) coherent plane-wave compounding
% article:ChernyakovaITUFFC2018: Fourier-Domain Beamforming and Structure-Based Reconstruction for Plane-Wave Imaging
% Abstract
% - Ultrafast imaging based on COHERENT PLANE-WAVE COMPOUNDING IS
%   ONE OF THE MOST IMPORTANT RECENT DEVELOPMENTS in medical ultrasound. (p. 1)
% - It significantly improves image quality and allows for MUCH FASTER IMAGE ACQUISITION. (p. 1)
% article:MontaldoITUFFC2009: Coherent plane-wave compounding for very high frame rate ultrasonography and transient elastography
coherent plane-wave compounding
\cite{article:MontaldoITUFFC2009},
% 6.) synthetic aperture imaging
% article:MoghimiradITUFFC2016: Synthetic Aperture Ultrasound Fourier Beamformation Using Virtual Sources
% article:JensenUlt2006: Synthetic aperture ultrasound imaging
% Abstract
% - SA imaging is a radical break with today’s commercial systems [...]. (p. e5)
% - These constrictions [limited frame rate, insufficient data for high precision flow estimation] can be lifted by employing SA imaging. (p. e5)
% - It is also possible to improve penetration depth by employing codes during ultrasound transmission. (p. e5)
% 3. Introduction to synthetic aperture imaging
% - SA imaging makes it possible to decouple frame rate and pulse repetition time, as only
%   a sparse set of emissions can be used for creating a full image. (p. e7)
% - Very fast imaging can, therefore, be made albeit with a lower resolution and higher side-lobes. (p. e7)
\ac{SA} imaging
\cite{article:MoghimiradITUFFC2016,article:JensenUlt2006}, or
% 7.) limited-diffraction beam imaging
% article:ChengITUFFC2006: Extended high-frame rate imaging method with limited-diffraction beams
% Abstract
% - Previously, a HIGH-FRAME RATE (HFR) IMAGING THEORY was developed in which
%   a pulsed plane wave was used in transmission, and
%   LIMITED-DIFFRACTION ARRAY BEAM WEIGHTINGS WERE APPLIED TO RECEIVED ECHO SIGNALS to produce
%   a spatial Fourier transform of object function for 3-D image reconstruction. (p. 880)
% - In this paper, THE THEORY IS EXTENDED TO INCLUDE EXPLICITLY VARIOUS TRANSMISSION SCHEMES [...]. (p. 880)
% article:LuITUFFC1997: 2D and 3D High Frame Rate Imaging with Limited Diffraction Beams
limited-diffraction beam imaging
\cite{article:ChengITUFFC2006,article:LuITUFFC1997}, that overcome
% 8.) limited frame rate
this limitation and capture
% 9.) large FOVs
% article:BerthonPMB2018: Spatiotemporal matrix image formation for programmable ultrasound scanners
% 1. Introduction
% - This enables frame rates approximately 100 times faster than standard ultrasound sequences, even in three dimensions (Provost et al 2014). (p. 1)
% - Images can then be formed in real-time, i.e. at frame rates higher than 30 images per second, or in post-processing using software-based approaches
%   (Lu 1997, Montaldo et al 2009, Garcia et al 2013). (p. 1)
% article:TanterITUFFC2014: Ultrafast imaging in biomedical ultrasound
% III. The concept of Plane-Wave Imaging and Plane-Wave compounding
% - It is important to keep in mind that
%   the ULTRAFAST TERMINOLOGY describes any acquisition sequence enabling
%   2-D OR 3-D IMAGING OVER A LARGE FIELD OF VIEW AT VERY HIGH FRAME RATES, typically in the KILOHERTZ RANGE
%   (i.e., the COMBINATION OF WIDE-FIELD TRANSMISSIONS AND PARALLEL RECEIVE BEAMFORMING). (p. 106)
large \acp{FOV} at
% 10.) rates in the kilohertz range
% article:GarciaITUFFC2013: Stolt's f-k Migration for Plane Wave Ultrasound Imaging
% I. Introduction
% - Theoretically, frame rates up to 15000 Hz can thus be attained for a 5-cm imaging depth [2]. (p. 1853)
rates in
the kilohertz range
\cite{article:TanterITUFFC2014},
\cite[Sect. 1.11]{book:Szabo2013}.
% d) ultrafast imaging modes sequentially insonify the entire FOV by only a few unfocused waves and process the RF voltage signals generated by all elements
% article:BerthonPMB2018: Spatiotemporal matrix image formation for programmable ultrasound scanners
% 1. Introduction
% - For instance, ULTRAFAST ULTRASOUND IMAGING (UUI) is based on
%   the TRANSMISSION OF A SMALL NUMBER OF DEFOCUSED WAVES such as
%   [1.)] plane (Montaldo et al 2009),
%   [2.)] circular (Lockwood et al 1998, Nikolov and Jensen 2003, Couade et al 2009, Provost et al 2012, Papadacci et al 2014), or
%   [3.)] spherical diverging waves (Provost et al 2014) to
%   INSONIFY THE ENTIRE FIELD OF VIEW (FOV) AT EACH EMISSION. (p. 1)
% article:GarciaITUFFC2013: Stolt's f-k Migration for Plane Wave Ultrasound Imaging
% IV. Discussion
% - ULTRAFAST ULTRASOUND PWI, in comparison with the conventional focusing approaches,
%   ALLOWS ONE TO OBTAIN A FULL IMAGE WITH A SINGLE TRANSMIT by
%   migration of the resulting RF signals. (p. 1861)
% I. Introduction
% - Contrarily to conventional ultrasound imaging,
%   the frame rate reached by plane wave ultrasonography is only LIMITED BY THE TIME REQUIRED FOR A WAVE TO MAKE A TWO-WAY TRIP. (p. 1853)
% - Whereas focusing approaches concentrate the acoustic energy at one or several locations,
%   PLANE WAVE GENERATION TENDS TO MINIMIZE THE DIFFERENCES in amplitude and phase over the cross-section delimited by the wavefront. (p. 1853)
% article:ChengITUFFC2006: Extended high-frame rate imaging method with limited-diffraction beams
% I. Introduction
% - Because ONE TRANSMISSION CAN BE USED TO RECONSTRUCT AN IMAGE, HIGH IMAGE FRAME RATE CAN BE ACHIEVED.  (p. 880)
% article:JensenUlt2006: Synthetic aperture ultrasound imaging
% 3. Introduction to synthetic aperture imaging
% - A SINGLE ELEMENT IN THE TRANSDUCER APERTURE is used for transmitting
%   A SPHERICAL WAVE COVERING THE FULL IMAGE REGION. (p. e6)
% article:JensenUlt2006: Synthetic aperture ultrasound imaging
% Abstract
% - Here DATA IS ACQUIRED SIMULTANEOUSLY FROM ALL DIRECTIONS OVER A NUMBER OF EMISSIONS, and
%   the full image can be reconstructed from this data. (p. e5)
% 3. Introduction to synthetic aperture imaging
% - The RECEIVED SIGNALS FOR ALL OR PART OF THE ELEMENTS in the aperture ARE SAMPLED FOR EACH TRANSMISSION. (p. e6)
Using
% 1.) fully-sampled transducer array
a fully-sampled transducer array,
% 2.) sequential insonification of the entire FOV by only a few unfocused waves
they sequentially insonify
% 3.) entire FOV
the entire \ac{FOV} by
% 4.) a few unfocused waves
only a few unfocused waves and process
% 5.) RF voltage signals
the echo signals generated by
%the \ac{RF} voltage signals generated by
% 6.) all elements
all elements.
% e) benefits and applications of ultrafast imaging modes
% article:ProvostPMB2014: 3D ultrafast ultrasound imaging in vivo
% Abstract
% - Its ability to track in 3D transient phenomena occurring in the millisecond range within a single ultrafast acquisition was demonstrated for
%   3D Shear-Wave Imaging, 3D Ultrafast Doppler Imaging, and, finally, 3D Ultrafast combined Tissue and Flow Doppler Imaging. (p. L1)
% - The propagation of shear waves was tracked in a phantom and used to characterize its stiffness. (p. L1)
% - 3D Ultrafast Doppler was used to obtain 3D maps of Pulsed Doppler, Color Doppler, and Power Doppler quantities in a single acquisition and revealed,
%   at thousands of volumes per second, the complex 3D flow patterns occurring in the ventricles of the human heart during an entire cardiac cycle, as well as
%   the 3D in vivo interaction of blood flow and wall motion during the pulse wave in the carotid at the bifurcation. (p. L1-L2)
% - This study demonstrates the potential of 3D Ultrafast Ultrasound Imaging for the 3D mapping of stiffness, tissue motion, and flow in humans in vivo and promises
%   new clinical applications of ultrasound with reduced intra— and inter-observer variability. (p. L2)
% book:Szabo2013, Chapter 1: Introduction / Sect. 1.11: Conclusion
% - Development of new imaging modalities such as
%   elastography and ARFI continue to expand opportunities for
%   improved diagnostic capabilities with effective benefit/cost ratios. (p. 35)
% article:GarciaITUFFC2013: Stolt's f-k Migration for Plane Wave Ultrasound Imaging
% I. Introduction
% - Recent studies indicate the growing emergence of ultrasound plane wave imaging (PWI). (p. 1853)
% - This technique has proven itself as a reliable method in several original and promising applications such as
%   transient elastography, ultrafast doppler imaging, ultrafast vector flow mapping,
%   electromechanical wave imaging, and functional brain imaging [3]–[7]. (p. 1853)
% article:ChengITUFFC2006: Extended high-frame rate imaging method with limited-diffraction beams
% I. Introduction
% - This [quasi-continuous tradeoff] is useful because in some applications, such as
%   imaging of liver and kidney in which high-frame rate imaging is not crucial,
%   high quality images can be obtained at the expense of image-frame rate. (p. 881)
Besides rendering \ac{UI}
% 1.) safety
% safety issues: as flat transmitted beams impose much less constraints than classical focused beams
% article:BercoffITUFFC2004: Supersonic Shear Imaging: A New Technique for Soft Tissue Elasticity Mapping
% less \emph{spatial peak-temporal average acoustic intensity} (cf. \cite[Sect. 14.11]{book:Bushberg2011}) than focused sound beams.
%[15] Food and Drug Administration, “Information for Manufacturers Seeking Marketing Clearance of Diagnostic Ultrasound Systems and Transducer,” U. S. Dept. Health and Human Services, Food and Drug Administration, Center for Devices and Radiological Health., 1997.
%\cite{guidelines:FDAUltrasound2008}. % standard:IEC60601-2-37-2007,
safer
\cite{article:BercoffITUFFC2004} and
% 2.) sensitivity
% article:MaceITUFFC2013: Functional Ultrasound Imaging of the Brain: Theory and Basic Principles
% - Here, we present a µDoppler ultrasound method able to detect and map the cerebral blood volume (CBV) over the entire brain with
%   an IMPORTANT INCREASE IN SENSITIVITY. (p. 492)
% article:BercoffITUFFC2011: Ultrafast Compound Doppler Imaging: Providing Full Blood Flow Characterization
more sensitive
\cite{article:BercoffITUFFC2011},
% 4.) ultrafast imaging modes
they have enabled
% 5.) observation
the observation of
% 6.) time-variant objects
moving objects and
% 7.) transient phenomena
% 1.) cardiac electrophysiology
%     article:PapadacciITUFFC2014: High-Contrast Ultrafast Imaging of the Heart
% 2.) functional brain imaging
%     article:MaceITUFFC2013: Functional Ultrasound Imaging of the Brain: Theory and Basic Principles
%     article:MaceNatMeth2011: Functional ultrasound imaging of the brain
%     article:BercoffITUFFC2011: Ultrafast Compound Doppler Imaging: Providing Full Blood Flow Characterization
% 3.) soft tissue elasticity mapping
%     article:BercoffITUFFC2004: Supersonic Shear Imaging: A New Technique for Soft Tissue Elasticity Mapping
%     article:BercoffApplPhysLett2004: Sonic boom in soft materials: The elastic Cerenkov effect
transient phenomena
\cite{article:PapadacciITUFFC2014,article:MaceITUFFC2013,article:BercoffITUFFC2004}, even in
% 8.) three-dimensional ultrafast UI
% article:ProvostPMB2014: 3D ultrafast ultrasound imaging in vivo
% article:JensenProgBMB2007: Medical ultrasound imaging
% 3. Anatomic ultrasound imaging / 3.2. Three-dimensional imaging
% - Often the 3D volume consists of 64x64 image lines and the time for one image to a depth of 15 cm is then
%   T_{i} = N_{l} 2 D / c = 0.53 s and
%   the frame rate is then less than 2 Hz. (p. 159)
% - This is insufficient for CARDIAC IMAGING, and parallel beam formation is then employed. (p. 159)
% article:ChengITUFFC2006: Extended high-frame rate imaging method with limited-diffraction beams
% I. Introduction
% - High-frame rate imaging is important for IMAGING OF FAST MOVING OBJECTS SUCH AS THE HEART, especially,
%   in 3-D imaging in which many 2-D image frames are needed to form a 3-D volume that
%   may reduce image frame rate dramatically with conventional imaging methods. (p. 880)
three dimensions
\cite{article:ProvostPMB2014}.

% multi-transmit beam
% TODO: [7] Ling Tong, Alessandro Ramalli, Ruta Jasaityte, Piero Tortoli, and Jan D’hooge, “Multi-transmit beam forming for fast cardiac imagingex- perimental validation and in vivo application,” IEEE transactions on medical imaging, vol. 33, no. 6, pp. 1205–1219, 2014.

% article:ChengITUFFC2006: Extended high-frame rate imaging method with limited-diffraction beams
% II. Theory / A. Extension of High-Frame Rate Imaging Theory
% - In the following, (15) of [27] will be generalized to include explicitly
%   VARIOUS TRANSMISSION SCHEMES such as
%   [1.)] MULTIPLE LIMITED-DIFFRACTION ARRAY BEAMS and
%   [2.)] STEERED PLANE WAVES. (p. 882)
% - Assuming that the TRANSMITTING TRANSFER FUNCTION OF THE TRANSDUCER is A(k) that includes both
%   [1.)] ELECTRICAL RESPONSE OF THE DRIVING CIRCUITS and
%   [2.)] ELECTRO-ACOUSTICAL COUPLING CHARACTERISTICS [79] of the transducer elements. (p. 882)
% - Then,
%   a broadband, limited-diffraction array beam [29], [32] or
%   pulsed steered plane wave (a plane wave is a special case of limited diffraction beams) incident on the object can be expressed as
%   [see (5) and (6) of [27] and their derivations from X waves [44], [45]]:
%   [27] article:LuITUFFC1997
%   [ \Phi_{Array}^{T}( \vect{r}_{0}, t ) = \frac{ 1 }{ 2 \pi } \int_{-\infty}^{\infty} A(k) H(k) e^{ j k_{x,T} x_{0} + j k_{y,T} y_{0} + j k_{z,T} z_{0} } e^{ - j \omega t } dk ], (1)
%   [...].
% - And f( \vect{r}_{0} ) is an object function that is related to the scattering strength of a scatterer at point \vect{r}_{0}. (p. 883)
% - It should be emphasized that (10) is also a 2-D Fourier transform of
%   the echo signals in terms of both x1 and y1 over
%   the transducer surface (aperture) [61], [62]. (p. 883)
% - Therefore, (13) indicates that (10) represents
%   a 2-D Fourier transform over the transducer surface for
%   echo signals produced from all point scatterers in the volume, V. (p. 884)
% - The phase and amplitude of each point source (scatterer) are modified by
%   the transmitted plane wave, (A(k)H(k)/c) e^{ j k_{x,T} x_{0} + j k_{y,T} y_{0} + j k_{z,T} z_{0} } [see (2)] as well as
%   the object function, f( \vect{r}_{0} ). (p. 884)
% - Thus, LIMITED-DIFFRACTION ARRAY BEAM WEIGHTING THEORY [27] IS EXACTLY THE SAME AS
%   A 3-D FOURIER TRANSFORMATION OF ECHO SIGNALS over both
%   the transducer aperture (2-D) and time (1-D), which
%   DECOMPOSES ECHO SIGNALS INTO PLANE WAVES or limited diffraction array beams [29], [32]. (p. 884)
% - The previous work [27], [28], [59], [60] on
%   the STEERED PLANE WAVES and
%   the LIMITED-DIFFRACTION ARRAY BEAM WEIGHTINGS IN TRANSMISSIONS is equivalent to
%   the PHASE AND AMPLITUDE MODIFICATIONS OF THE OBJECT FUNCTION shown in (13) or (10). (p. 884)
% - This proves that \tilde{R}_{kx+kxT ,ky+kyT ,kz+kzT}(ω) in (10) or (13) can be obtained directly by
%   3-D Fourier transform of
%   the received echo signals over a 2-D transducer aperture and 1-D time! (p. 884)
% - It can be shown from computer simulations and experiments in the later sections that
%   these approximations do not affect the quality of reconstructed images as compared to those obtained with
%   conventional, dynamically focused pulse-echo imaging systems. (p. 884)
% II. Theory / B. Special Cases of the High-Frame Rate Imaging Theory
% II. Theory / C. 2-D High-Frame Rate Imaging Theory
% - Eq. (35) and (36) [2-D equations} are the equations for reconstructions of images from data obtained with simulations and experiments. (p. 886)
% III. Relationships Between Fourier Domains of Echoes and Object Function / B. Image Reconstruction with Steered Plane Waves
% IV. Computer Simulation
% - Reconstructed images of the first object are shown in
%   Figs. 4 and 5 with limited-diffraction array beams and steered plane waves in transmissions, respectively. (p. 888)
% - In each figure, there are four panels for images reconstructed with
%   one transmission (up to 5500 frames/second with a speed of sound of 1540 m/s) [Figs. 4(a) and 5(a)],
%   11 transmissions (up to 500 frames/second) [Figs. 4(b) and 5(b)],
%   91 transmissions (up to 60 frames/second) [Figs. 4(c) and 5(c)], and
%   263 transmissions (up to 21 frames/second) [Figs. 4(d) and 5(d)], respectively, for a depth of 140 mm. (p. 888)
% - As a comparison,
%   panels Figs. 4(d) and 5(d) are the same and are obtained with
%   the CONVENTIONAL DELAY-AND-SUM METHOD with its transmission focus at 70 mm. (p. 888)
% - For 91 transmissions, it is seen from Figs. 4 and 5 that
%   image resolution is high and
%   sidelobe is low (images are log-compressed in 50 dB) for both
%   limited-diffraction array beam and
%   steered plane wave transmissions as compared to the conventional delay-and-sum method. (p. 888)
% - Even with one transmission,
%   the results are still comparable to that of delay-and-sum except near the transmission focal depth. (p. 888)
% - The results for limited-diffraction array beam and steered plane wave transmissions are similar, except
%   the former has a somewhat higher resolution. (p. 888)
% - Simulations with the second object are done to compare the image quality of both
%   limited-diffraction array beam and
%   steered plane wave transmissions with
%   the conventional delay-and-sum method of transmission focusing at all depths (dynamic transmission focusing) (Fig. 6). (p. 888)
% - It should be mentioned that both
%   limited-diffraction array beam and
%   steered plane wave methods have high computation efficiency due to the use of FFT
%   (the difference is on how to transmit — sine and cosine weighting are used to produce limited-diffraction array beams, and
%   linear-time delay is used to produce a steered plane wave). (p. 890)
% V. In Vitro and In Vivo Experiments
% - To test the extended HFR imaging theory in practical situations, both
%   in vitro and in vivo experiments are carried out with
%   a homemade HFR imaging system [62], [77], [78]. (p. 890)
% V. In Vitro and In Vivo Experiments / B. In Vitro Experiments
% - It is seen that, as the number of transmissions increases,
%   image contrast is increased significantly for both
%   limited-diffraction array beam and
%   steered plane wave transmissions. (p. 892)
% - Because the noise of the HFR imaging system is relatively high,
%   the contrast of cystic targets is lowered because the noise fills into the cystic areas. (p. 892)
% - This is more acute for limited-diffraction array beam transmissions in which
%   the sine or cosine weightings further reduce the transmission power by half [62]. (p. 892)
% V. In Vitro and In Vivo Experiments / C. In Vivo Experiments
% - From both Figs. 13 and 14, it is seen that
%   the steered plane wave with 91 transmissions produces images that
%   are better in both resolution and contrast than the conventional delay-and-sum method of a similar number of transmissions. (p. 892)
% - For the heart imaging, it is seen that the image quality can be traded off with frame rate. (p. 892)
% - Motion artifacts were a concern for
%   the extended HFR imaging theory when more than one transmission is used; however, from the in vivo heart images, it seems that
%   the extended theory is not very sensitive to the motion. (p. 892)
% - This is because the HFR imaging theory allows one transmission to reconstruct a complete image [27], i.e.,
%   the quality of each subimage is not affected by the motion, and
%   the heart does not move that fast to distort the image when subimages obtained from
%   different transmissions are superposed. (p. 892)
% VII. Conclusions
% - A HFR imaging theory was developed in 1997 [27], in which
%   a pulsed plane wave was used in transmission, and
%   limited-diffraction array beam weightings were applied to received echo signals to produce
%   a spatial Fourier transform of object function for 3-D image reconstruction. (p. 896)
% - In addition,
%   the use of steered plane waves in transmissions to increase image field of view and reduce speckle noises [27], [28], [59], [60] or
%   the use of limited-diffraction array beams [26], [29], [32] in transmission to increase field of view and spatial Fourier domain coverage to increase image resolution was suggested. (p. 896)
% - In this paper,
%   the HFR imaging theory was extended to include explicitly various transmission schemes such as
%   multiple limited-diffraction array beams and
%   steered plane waves [61], [62] (the first report was given in [61]). (p. 896)
% - Moreover, limited-diffraction array beam weightings of received echo signals over
%   a 2-D transducer aperture were proved to be the same as
%   a 2-D Fourier transform of these signals over the same aperture [61], [62]. (p. 896)
% - Because image frame rate is inversely proportional to
%   the number of transmissions used to obtain a single frame of image,
%   the extended theory provides a continuous compromise between image quality and frame rate. (p. 896)
% - This is desirable in applications in which HFR imaging is not crucial, such as
%   maging of livers and kidneys, high quality images can be reconstructed at the expense of image frame rate. (p. 896)
% - Both sim- ulations and experiments (in vitro and in vivo) show that the extended theory can be used to reconstruct high qual- ity images with little motion artifacts as compared to the conventional delay-and-sum method of a fixed transmis- sion focus as well as the delay-and-sum method that syn- thesizes its dynamic transmission focuses with a montage process. (p. 896)
% - Because the method can be implemented with the FFT that has very high computation efficiency,
%   imaging systems that use the method would be simplified greatly. (p. 896)
% - This is important for 3-D imaging with a fully populated 2-D array transducer in which
%   the computation efficiency and image frame rate are of paramount importance. (p. 896)

%---------------------------------------------------------------------------------------------------------------
% 2.) shortcomings of the established image recovery methods underlying ultrafast UI
%---------------------------------------------------------------------------------------------------------------
% a) established image recovery methods gradually trade the image quality for the frame rate
% article:GarciaITUFFC2013: Stolt's f-k Migration for Plane Wave Ultrasound Imaging
% article:MontaldoITUFFC2009: Coherent plane-wave compounding for very high frame rate ultrasonography and transient elastography
% II. Theory / A. The Single Plane Wave Imaging Mode
% - Because we do not have any focusing for the transmit beam,
%   the IMAGE RESOLUTION IS OBTAINED ONLY BY A PARALLEL PROCESSING DURING THE RECEPTION MODE by
%   ADDING COHERENTLY THE ECHOES COMING FROM THE SAME SCATTER. (p. 491)
% article:JensenUlt2006: Synthetic aperture ultrasound imaging
% Abstract
% - Due to the complete data set, it is possible to have both
%   DYNAMIC TRANSMIT AND RECEIVE FOCUSING to improve
%   CONTRAST AND RESOLUTION. (p. e5)
% 3. Introduction to synthetic aperture imaging
% - This data [received signals for all or part of the elements] can be used for making
%   A LOW RESOLUTION IMAGE, which is ONLY FOCUSED IN RECEIVE due to the un-focused transmission. (p. e6)
% - This [DAS in rx] is done for every point in the resulting image to yield
%   A LOW RESOLUTION IMAGE. (p. e7)
% - COMBINING THE LOW RESOLUTION IMAGES THEN RESULTS IN A HIGH RESOLUTION IMAGE, since
%   fully dynamic focusing has been performed for all points in the image. (p. e7)
% - The TRANSMIT FOCUSING is, thus, SYNTHESIZED BY COMBINING THE LOW RESOLUTION IMAGES, and
%   the focusing calculation makes the transmit focus DYNAMIC FOR ALL POINTS IN THE IMAGE. (p. e7)
% - The FOCUS is, therefore, BOTH DYNAMIC IN TRANSMIT AND RECEIVE and
%   the HIGHEST POSSIBLE RESOLUTION FOR DELAY-SUM BEAMFORMING IS OBTAINED EVERYWHERE IN THE IMAGE. (p. e7)
% article:ChengITUFFC2006: Extended high-frame rate imaging method with limited-diffraction beams
% Abstract
% - Results show that
%   IMAGE RESOLUTION AND CONTRAST ARE INCREASED OVER A LARGE FIELD OF VIEW AS
%   MORE AND MORE
%   [1.)] limited-diffraction array beams with different parameters or
%   [2.)] PLANE WAVES STEERED AT DIFFERENT ANGLES
%   ARE USED IN TRANSMISSIONS. (p. 880)
% - Thus, the method provides
%   A CONTINUOUS COMPROMISE BETWEEN IMAGE QUALITY AND IMAGE FRAME RATE that is
%   INVERSELY PROPORTIONAL TO THE NUMBER OF TRANSMISSIONS USED TO OBTAIN A SINGLE FRAME OF IMAGE. (p. 880)
% I. Introduction
% - Results show that
%   IMAGE RESOLUTION AND CONTRAST ARE INCREASED OVER A LARGE FIELD OF VIEW AS
%   MORE AND MORE
%   [1.)] limited-diffraction array beams with different parameters or
%   [2.)] PLANE WAVES STEERED AT DIFFERENT ANGLES
%   are transmitted. (p. 881)
% - Thus, the method provides
%   A CONTINUOUS COMPROMISE BETWEEN IMAGE QUALITY AND IMAGE FRAME RATE that is
%   INVERSELY PROPORTIONAL TO THE NUMBER OF TRANSMISSIONS USED IN A SINGLE FRAME OF IMAGE. (p. 881)
The established image recovery methods, which are
% 1.) explicit (closed-form expressions)
% article:BerthonPMB2018: Spatiotemporal matrix image formation for programmable ultrasound scanners
% 1. Introduction
% - Both these STANDARD APPROACHES [DAS, Fourier-based] are based on
%   the IMPLEMENTATION OF CLOSED-FORM SOLUTIONS TO THE INVERSE PROBLEM, i.e.
%   finding the object to image from a collection of measurements. (p. 1)
% - While such an approach is IDEAL TO GENERATE FAST ALGORITHMS [...]. (p. 1)
explicit and
% 2.) computationally efficient
computationally efficient, gradually trade
% 3.) image quality
the image quality for
% 4.) frame rate
the frame rate
\cite{article:GarciaITUFFC2013,article:MontaldoITUFFC2009,article:JensenUlt2006,article:ChengITUFFC2006}.
% b) physical models neglect various effects and basic abilities of programmable UI systems
% article:IlovitshNatComBio2018: Acoustical structured illumination for super-resolution ultrasound imaging
% - Despite these advances [user-programmable systems, schemes for image reconstruction],
%   ULTRASOUND STILL SUFFERS FROM LIMITATIONS in
%   RESOLUTION, CONTRAST and SIGNAL TO NOISE RATIO (SNR), and from ARTIFACTS [4].
%   [4] 4. Kremkau, F. W. & Taylor, K. J. Artifacts in ultrasound imaging. J Ultrasound Med. 5, 227–237 (1986).
Their physical models neglect
% 1.) various effects
various effects, e.g.
% 2.) finite number of array elements
% article:GarciaITUFFC2013: Stolt's f-k Migration for Plane Wave Ultrasound Imaging
% IV. Discussion / C. PWI Using f-k Migration: Limitations and Perspectives
% - Ultrasound PWI needs the WAVEFRONTS TO BE PLANAR AND TILTED with the desired incident angle. (p. 1863)
% - To get high-quality images by PWI, one must ensure that
%   a PLANAR WAVEFIELD IS SYNTHESIZED PROPERLY. (p. 1863)
% - An unlimited number of coplanar elementary sources can produce a perfect slant plane wave. (p. 1863)
% - In practice though,
%   the AMOUNT OF ELEMENTS IN A LINEAR-ARRAY TRANSDUCER IS LIMITED to 64, 128, or 192. (p. 1863)
% - This technical limitation may cause ADVERSE EFFECTS that may negatively affect the resulting images. (p. 1863)
% - More importantly, a more disturbing effect may rise from
%   the GRATING LOBES WHICH ARE INDUCED BY THE REGULAR SPACING of the individual transducer elements. (p. 1863)
% - It is known that the grating lobes are of larger magnitude as the steering angle increases [47]. (p. 1863)
the finite number of
array elements and
% 3.) anisotropic directivities
their anisotropic directivities, and
% 4.) basic abilities
basic abilities of
% 5.) freely programmable UI systems
programmable \ac{UI} systems, e.g.
% 6.) syntheses of complex incident waves
the syntheses of
complex incident waves.
% c) popular DAS method focuses the echo signals on specified points in the FOV to quantify their echogeneity
% article:BessonITUFFC2018: Ultrafast Ultrasound Imaging as an Inverse Problem: Matrix-Free Sparse Image Reconstruction
% Abstract
% - Conventional ultrasound (US) image reconstruction methods rely on
%   DELAY-AND-SUM (DAS) BEAMFORMING, which is
%   a RELATIVELY POOR SOLUTION to the image reconstruction problem. (p. 339)
% I. INTRODUCTION
% - Current real-time US imaging is GENERALLY BASED ON THE WELL-KNOWN DELAY-AND-SUM (DAS) BEAMFORMING ALGORITHM, which can be seen as
%   A BACKPROJECTION SOLUTION OF THE INVERSE PROBLEM under several assumptions [2]. (p. 339)
% - While being suitable for REAL-TIME IMAGING, DAS suffers from
%   [1.)] A RELATIVELY LOW QUALITY, in terms of SIGNAL-TO-NOISE RATIO, and requires
%   [2.)] SAMPLING THE ELEMENT RAW-DATA AT A RATE FEW TIMES HIGHER THAN THE NYQUIST RATE for
%         sufficient accuracy in the delay calculations [3], [4]. (p. 339)
% II. PARAMETRIC MATRIX-FREE FORMULATIONS OF THE MEASUREMENT MODEL AND ITS ADJOINT / B. Adjoint of the Proposed Measurement Model and Its Relationship With the Delay-and-Sum Algorithm
% - The DAS ALGORITHM IS THE STANDARD IMAGE RECONSTRUCTION METHOD employed in US imaging because of
%   its SIMPLICITY AND REAL-TIME CAPABILITY. (p. 341)
% - The DAS solution, while suited to real-time imaging, is therefore
%   A RELATIVELY POOR APPROXIMATION OF THE SOLUTION OF PROBLEM (2). (p. 341)
% - Indeed, DAS beamforming
%   [1.)] does not address the PROBLEM OF THE NOISE n,
%   [2.)] NEGLECTS PULSE SHAPE, and
%   [3.)] REQUIRES A SAMPLING RATE HIGHER THAN NYQUIST SAMPLING requirements because of the high-delay resolution required [3], [4]. (p. 341)
% - Indeed, to apply the delay defined in (13) digitally, received signals must be sampled on a sufficiently dense grid. (p. 341)
% article:BerthonPMB2018: Spatiotemporal matrix image formation for programmable ultrasound scanners
% 1. Introduction
% - The MOST COMMONLY USED APPROACH TO IMAGE FORMATION IN ULTRAFAST IMAGING is
%   the DELAY-AND-SUM (DAS) ALGORITHM, which is based on a number of GEOMETRICAL ASSUMPTIONS. (p. 1)
% article:MoghimiradITUFFC2016: Synthetic Aperture Ultrasound Fourier Beamformation Using Virtual Sources
% I. INTRODUCTION
% - Usually,
%   MEDICAL SYNTHETIC APERTURE IMPLEMENTATIONS ARE PERFORMED USING DELAY-AND-SUM (DAS) BEAMFORMING IN
%   THE TIME DOMAIN [11]–[16]. (p. 2018)
% - This STRAIGHTFORWARD ALGORITHM IS QUITE TIME CONSUMING on general-purpose computers, due to
%   the LARGE NUMBER OF OPERATIONS. (p. 2018)
% - Starting with a MEDIUM REFLECTIVITY FUNCTION assumes that
%   superposition applies; and inverting the problem,
%   ALL THE BEAMFORMATION ALGORITHMS TRY TO RECONSTRUCT AN IMAGE USING ECHO SIGNALS. (p. 2018)
% - The SIMPLEST ALGORITHM IS DAS, which is based on
%   THE GEOMETRIC DISTANCE BETWEEN THE POINT AND THE APERTURE. (p. 2018)
% - The algorithm [DAS] CAN BE USED WITH DIFFERENT ARRAY GEOMETRIES; however,
%   the NUMBER OF COMPUTATIONS IS HIGH because of point-by-point processing. (p. 2018)
% - In medical ultrasound imaging, however,
%   THE CLASSICAL TIME DOMAIN DAS METHOD IS STILL MORE POPULAR [14]–[16], [35], [36] than
%   the frequency domain algorithms. (p. 2019)
% VII. CONCLUSION AND DISCUSSION
% - As previously mentioned,
%   SYNTHETIC APERTURE IMAGING in its classical form uses
%   DAS BEAMFORMATION WITH A HUGE COMPUTATIONAL LOAD due to the point-by-point processing protocol. (p. 2027)
% article:GarciaITUFFC2013: Stolt's f-k Migration for Plane Wave Ultrasound Imaging
% I. Introduction
% - In its SIMPLEST FORM,
%   MIGRATION BY SUMMATION OF TRACE AMPLITUDES ALONG HYPERBOLIC TRAJECTORIES (known as DIFFRACTION SUMMATION) has been
%   a basic tool for geophysicists since the 1950s [9]. (p. 1853)
% - This method has been EXTENSIVELY USED IN ULTRASOUND IMAGING under the name
%   DELAY-AND-SUM (DAS). (pp. 1853, 1854)
% IV. Discussion / A. Differences Among the Three Migration Methods
% 1) DAS:
% - The DAS IS ACTUALLY EQUIVALENT TO THE SO-CALLED DIFFRACTION SUMMATION,
%   the SIMPLEST MIGRATION METHOD used by the geophysicists since the 1960s [9]. (p. 1862)
% - This is a GEOMETRIC STRATEGY which consists of
%   SUMMING THE BACKSCATTERED SIGNALS ALONG THE HYPERBOLIC TRACES OF THE DIFFRACTION RESPONSES. (p. 1862)
% - The DAS THUS PROVIDES A BASIC SOLUTION OF THE MIGRATION PROBLEM. (p. 1862)
% - Although this procedure makes good sense and provides good outputs in plane wave imaging,
%   IT IS THEORETICALLY INCORRECT [29]. (p. 1862)
% - A more exact process would be, for instance, given by the KIRCHHOFF'S INTEGRAL THEOREM, which adds
%   AMPLITUDE AND PHASE CORRECTIONS to the data before summation [29], [41], [42]. (p. 1862)
The popular \ac{DAS} method, for example, focuses
% 1.) recorded RF voltage signals
the echo signals on
% 2.) specified points
specified points in
% 3.) field-of-view
the \ac{FOV} to quantify
% 4.) echogeneity
their echogeneity.
% d) popular DAS method adds the signal samples at the round-trip TOFs
% article:BerthonPMB2018: Spatiotemporal matrix image formation for programmable ultrasound scanners
% 1. Introduction
% - For a GIVEN INCIDENT WAVE, it [DAS ALGORITHM] consists in
%   [1.)] CALCULATING THE TIME OF FLIGHT associated to each pixel/element pair and
%   [2.)] SUMMING THE CORRESPONDING SIGNALS measured for the different elements to populate the image (Montaldo et al 2009). (p. 1)
% article:GarciaITUFFC2013: Stolt's f-k Migration for Plane Wave Ultrasound Imaging
% I. Introduction
% - The DAS simply consists in
%   INTEGRATING THE ULTRASOUND RF SIGNALS OVER
%   ALL THE HYPERBOLAS PRESENT IN THE RF SIGNALS. (p. 1854)
% - The PIXELS OF THE RESULTING MIGRATED RF IMAGE ARE THUS ASSIGNED THE INTEGRAL VALUES. (p. 1854)
% article:MontaldoITUFFC2009: Coherent plane-wave compounding for very high frame rate ultrasonography and transient elastography
% II. Theory / A. The Single Plane Wave Imaging Mode
% - Because we do not have any focusing for the transmit beam,
%   the IMAGE RESOLUTION IS OBTAINED ONLY BY A PARALLEL PROCESSING DURING THE RECEPTION MODE by
%   ADDING COHERENTLY THE ECHOES COMING FROM THE SAME SCATTER. (p. 491)
% - For a PLANE-PULSED WAVE, Fig. 1(b) shows that
%   the TRAVELING TIME to the point (x,z) and back to a transducer placed in x1 is given by
%   [ \tau( x_{1}, x, z ) = ( z + \sqrt{ z^{2} + ( x - x_{1} )^{2} } ) / c ], (1)
%   where c is the SPEED OF SOUND THAT WE ASSUMED TO BE CONSTANT IN THE MEDIUM. (p. 491)
% - Each point (x,z) of the image is obtained by ADDING COHERENTLY THE CONTRIBUTION OF EACH SCATTER, i.e.,
%   [1.)] DELAYING THE RF(x1,t) SIGNALS BY τ(x1,x,z) and
%   [2.)] ADDING THEM IN THE ARRAY DIRECTION x1
%   [ s( x, z ) = \int_{ x - a }^{ x + a } RF( x_{1}, \tau( x_{1}, x, z ) ) d x_{1} ]. (2)
% - The aperture 2a must take into account only the elements that contribute to the signal. (p. 491)
% - This aperture 2a is always lower than the total length of the array L, and it can be expressed by the F-number defined as
%   [ F = z / 2a ]. (3) (p. 491, 492)
% II. Theory / B. The Coherent Plane Wave Compound
% - Then an image is obtained in the same way as shown in (2), but with
%   the NEW DELAYS FROM (6). (p. 493)
% article:JensenUlt2006: Synthetic aperture ultrasound imaging
% 3. Introduction to synthetic aperture imaging
% - Focusing is performed by
%   [1.)] FINDING THE GEOMETRIC DISTANCE from the transmitting element to the imaging point and back to the receiving element. (p. e7)
% - [2.)] DIVIDING THIS DISTANCE BY THE SPEED OF SOUND c gives the time instance t_{p}( i, j ) to
%   [3.)] TAKE OUT THE PROPER SIGNAL VALUE FOR SUMMATION. (p. e7)
% - For an image point \vect{r}_{p} THE TIME IS, thus:
%   [ t_{p}( i, j ) = \frac{ \norm{ \vect{r}_{p} - \vect{r}_{e}( i ) }{2} + \norm{ \vect{r}_{p} - \vect{r}_{r}( j ) }{2} }{ c } ] (4)
%   where \vect{r}_{e}( i ) denotes the position of the transmitting element i and \vect{r}_{r}( j ) the receiving element j’s position. (p. e7)
% - The FINAL FOCUSED SIGNAL y_{f}( \vect{r}_{p} ) is then:
%   [ y_{f}( \vect{r}_{p} ) = \sum_{ j = 1 }^{ N } \sum_{ i = 1 }^{ M } a( t_{p}( i, j ), i, j ) y_{r}( t_{p}( i, j ), i, j ) ] (5)
%   where y_{r}( t_{p}( i, j ), i, j ) is the received signal for emission i on element j,
%   a( t_{p}( i, j ), i, j ) is the weighting function (apodization) applied onto this signal,
%   N is the number of transducer elements, and M is the number of emissions. (p. e7)
Emitting
% 1.) steered PWs
% article:ProvostPMB2014: 3D ultrafast ultrasound imaging in vivo [Oct. 2014]
% Abstract
% - In this study,
%   we present the FIRST IMPLEMENTATION OF ULTRAFAST ULTRASOUND IMAGING IN 3D based on the use of EITHER
%   [1.)] DIVERGING or
%   [2.)] PLANE WAVES EMANATING FROM
%   A SPARSE VIRTUAL ARRAY LOCATED BEHIND THE PROBE. (p. L1)
% - It achieves high contrast and resolution while maintaining IMAGING RATES OF THOUSANDS OF VOLUMES PER SECOND. (p. L1)
% - A customized portable ultrasound system was developed to
%   sample 1024 independent channels and to
%   drive a 32 × 32 matrix-array probe. (p. L1)
% article:MaceITUFFC2013: Functional Ultrasound Imaging of the Brain: Theory and Basic Principles
% Abstract
% - This method is based on imaging the brain at an ultrafast frame rate (1 kHz) using
%   COMPOUNDED PLANE WAVE EMISSIONS. (p. 492)
% article:BercoffITUFFC2011: Ultrafast Compound Doppler Imaging: Providing Full Blood Flow Characterization
% Abstract
% - This technique is called ULTRAFAST COMPOUND DOPPLER IMAGING and is based on the following concept:
%   instead of successively insonifying the medium with focused beams,
%   SEVERAL TILTED PLANE WAVES ARE SENT INTO THE MEDIUM and the backscattered signals are coherently summed to produce
%   high-resolution ultrasound images. (p. 134)
% article:MontaldoITUFFC2009: Coherent plane-wave compounding for very high frame rate ultrasonography and transient elastography
% article:BercoffITUFFC2004: Supersonic Shear Imaging: A New Technique for Soft Tissue Elasticity Mapping
% II. Ultrafast Imaging of Remotely Induced Shear Waves / B. Experimental Protocol / 1. The Ultrafast Scanner
% - The ultrafast frame rate is achieved by reducing the emitting mode to a single, PLANE-WAVE INSONIFICATION. (p. 398)
steered \acp{PW}
\cite{article:ProvostPMB2014,article:MaceITUFFC2013,article:BercoffITUFFC2011,article:MontaldoITUFFC2009,article:BercoffITUFFC2004}, whose
% 2.) spatial extent and energy content are unlimited
% article:ChengITUFFC2006: Extended high-frame rate imaging method with limited-diffraction beams
% I. Introduction
% - The advantage of LIMITED DIFFRACTION BEAMS is that,
%   even if they are produced with FINITE APERTURE AND ENERGY, they have
%   A VERY LARGE DEPTH OF FIELD. (p. 880)
% II. Theory / A. Extension of High-Frame Rate Imaging Theory
% - As shown in Fig. 1, a 2-D array transducer located at z = 0 plane is excited to generate
%   a broadband, limited-diffraction array beam or a steered pulsed plane wave. (p. 882)
% - The same transducer also is used to receive echoes scattered from objects. (p. 882)
% - The APERTURE OF THE TRANSDUCER IS ASSUMED TO BE INFINITELY LARGE, and
%   the SIZE OF EACH TRANSDUCER ELEMENT IS INFINITELY SMALL. (p. 882)
spatial extent and
energy content are
unlimited, or
% 3.) outgoing (d-1)-spherical waves
% single array element:
%   article:JensenUlt2006: Synthetic aperture ultrasound imaging
%   4. Penetration problem
%   - A major problem in SA imaging is the limited penetration depth, since
%     AN UN-FOCUSED WAVE IS USED IN TRANSMIT and
%     ONLY A SINGLE ELEMENT EMITS ENERGY. (p. e8)
% virtual point sources:
%   article:PapadacciITUFFC2014: High-Contrast Ultrafast Imaging of the Heart [Feb. 2014]
%   Abstract
%   - In this paper, we propose ultrafast imaging of the heart with adapted sector size by coherently compounding
%     diverging waves emitted from a standard transthoracic cardiac phased-array probe. (p. 288)
%   - As in ultrafast imaging with plane wave coherent compounding,
%     diverging waves can be summed coherently to obtain high-quality images of the entire heart at high frame rate in a full field of view. (p. 288)
%   article:JensenUlt2006: Synthetic aperture ultrasound imaging
%   4. Penetration problem
%   - The problem [limited penetration depth] can be solved by
%     COMBINING SEVERAL ELEMENTS FOR TRANSMISSION and
%     USING LONGER WAVEFORMS EMITTING MORE ENERGY. (p. e8)
%   - Karman et al. [10] suggested COMBINING SEVERAL ELEMENTS N_{t} IN TRANSMIT, with
%     a DELAY CURVE TO DE-FOCUS THE EMISSION TO EMULATE A SPHERICAL WAVE.  (p. e8)
%   - This can increase the emitted amplitude be a factor of \sqrt{ N_{t} }. (p. e8)
outgoing $\{ 1, 2 \}$-spherical waves
\cite{article:ProvostPMB2014,article:PapadacciITUFFC2014,article:JensenUlt2006}, whose
% 4.) isotropic sources are points
% article:MoghimiradITUFFC2016: Synthetic Aperture Ultrasound Fourier Beamformation Using Virtual Sources
% II. MULTISTATIC WAVENUMBER ALGORITHM
% - This essentially assumes that
%   the ARRAY CONSISTS OF POINT SOURCES AND RECEIVERS. (p. 2020)
% - It has been proved that the fields can be assumed spherical for a distance from the source [44],
%   so the model is acceptable for medical ultrasound field. (p. 2020)
isotropic sources are
points,
% 5.) popular DAS method
it adds
% 6.) signal samples
the signal samples at
% 7.) round-trip times-of-flight
the round-trip \acp{TOF}
\cite[(2), (6)]{article:MontaldoITUFFC2009},
\cite[(4), (5)]{article:JensenUlt2006}.
% e) competing Fourier methods invert the wave equation and potentially improve the image quality at reduced computational costs
% article:BerthonPMB2018: Spatiotemporal matrix image formation for programmable ultrasound scanners
% 1. Introduction
% - FOURIER-BASED APPROACHES HAVE ALSO BEEN DEVELOPED AND PROVIDE AN INTERESTING ALTERNATIVE, which is useful to
%   [1.)] ACCELERATE COMPUTATION and to
%   [2.)] POTENTIALLY IMPROVE IMAGE QUALITY since such approaches are based on
%   a MORE ACCURATE SOLUTION TO THE WAVE EQUATION (Garcia et al 2013). (p. 1)
% article:ZhangITUFFC2016: Extension of Fourier-Based Techniques for Ultrafast Imaging in Ultrasound With Diverging Waves
% Abstract
% article:MoghimiradITUFFC2016: Synthetic Aperture Ultrasound Fourier Beamformation Using Virtual Sources
% Abstract
% - The concept is based on
%   the FREQUENCY DOMAIN WAVENUMBER ALGORITHM FROM RADAR AND SONAR and is extended to
%   a multielement transmit/receive configuration using virtual sources. (p. 2018)
% - Window functions are used to extract the azimuth processing bandwidths and weight the data to reduce side lobes in the final image. (p. 2018)
% - Evaluating the results over a wide variety of parameters and
%   having almost the same results for simulated and measured data demonstrates the ability of FBV in
%   PRESERVING THE QUALITY OF IMAGE AS DAS, while providing
%   A MORE EFFICIENT ALGORITHM WITH 20 TIMES LESS COMPUTATIONS. (p. 2018)
% I. INTRODUCTION
% - There is still a tradeoff between the cost of the hardware and the processing time, especially in real-time imaging, 3-D imaging, and for
%   portable scanners. (p. 2018)
% - An alternative is Fourier domain algorithms, which use block processing instead of point-by-point processing. (p. 2018)
% - Range/Doppler is the first Fourier domain algorithm introduced and is shown step-by-step in Fig. 1 [20], [21]. (p. 2018)
% - The basic concept is that the coordinate transformation can put the data into a domain that is both
%   time- and spatial-wavenumber-dependent. (p. 2019)
% - An improved version is the wavenumber algorithm, which uses less approximations. (p. 2019)
% - It contains a 2-D Fourier transform to convert the data into the (ω, ku) domain,
%   a Stolt mapping algorithm to reshape the samples, and a phase correction to shift the data to the center of the domain. (p. 2019)
% - It is also called Stolt mapping, \Omega-K, or the range migration algorithm [22]. (p. 2019)
% - A new set of equations, therefore, have to be adapted for
%   Fourier beamformation of multielement transmission SA ultrasound imaging. (p. 2019)
% - In this paper, the block-processing wavenumber algorithm is adapted to
%   the medical field using a multistatic configuration with focused emissions assuming
%   a virtual source behind or in front of the transducer. (p. 2019)
% - Evaluations show that Fourier Beamformation Using Virtual Sources (FBV) can
%   [1.)] REDUCE THE PROCESSING TIME BY A FACTOR OF 20 WHILE
%   [2.)] RETAINING THE IMAGE QUALITY. (p. 2019)
% article:GarciaITUFFC2013: Stolt's f-k Migration for Plane Wave Ultrasound Imaging
% Abstract
% - To perform beamforming of plane wave echo RFs and
%   RETURN HIGH-QUALITY IMAGES AT HIGH FRAME RATES,
%   WE PROPOSE A NEW MIGRATION METHOD CARRIED OUT IN THE FREQUENCY-WAVENUMBER (f-k) DOMAIN. (p. 1853)
% I. Introduction
% - MIGRATION IMPROVES FOCUSING by essentially achieving AMPLITUDE AND PHASE RECTIFICATIONS to correct for
%   the effects of the spreading of ray paths as waves propagate [8]. (p. 1853)
% II. Theoretical Background: Adapting the Fourier Domain Stolt's Migration for Plane Wave Imaging / B. Stolt's f-k Migration
% - We now propose a new migration method that
%   also OPERATES ENTIRELY IN THE FOURIER FREQUENCY-WAVENUMBER DOMAIN (f-k migration). (p. 1857)
% - The F-K MIGRATION WAS DEVELOPED FROM THE LINEAR WAVE EQUATION USING FOURIER TRANSFORMS. (p. 1857)
% IV. Discussion
% - The MAIN ADVANTAGE of the f-k migration over conventional DAS is that
%   IT WORKS COMPLETELY IN THE FOURIER SPACE. (p. 1861)
% article:ChengITUFFC2006: Extended high-frame rate imaging method with limited-diffraction beams
% Abstract
% - Previously, a HIGH-FRAME RATE (HFR) IMAGING THEORY was developed in which
%   a PULSED PLANE WAVE WAS USED IN TRANSMISSION, and
%   LIMITED-DIFFRACTION ARRAY BEAM WEIGHTINGS WERE APPLIED TO RECEIVED ECHO SIGNALS to produce
%   a SPATIAL FOURIER TRANSFORM OF OBJECT FUNCTION for 3-D image reconstruction. (p. 880)
% - A relationship between
%   [1.)] THE LIMITED-DIFFRACTION ARRAY BEAM WEIGHTING of received echo signals and
%   [2.)] A 2-D FOURIER TRANSFORM OF THE SAME SIGNALS OVER A TRANSDUCER APERTURE is established. (p. 880)
% article:LuITUFFC1997: 2D and 3D High Frame Rate Imaging with Limited Diffraction Beams
% - TODO: steered PW?
The competing \name{Fourier} methods, in contrast, invert
% 1.) wave equation
the wave equation and potentially improve
% 2.) image quality (similar or higher)
% article:MoghimiradITUFFC2016: Synthetic Aperture Ultrasound Fourier Beamformation Using Virtual Sources
% Abstract
% - FBV SHOWS
%   A BETTER LATERAL RESOLUTION AT ALL DEPTHS, and
%   THE AXIAL AND CYSTIC RESOLUTIONS of −6, −12, and −20 dB ARE ALMOST THE SAME FOR FBV AND DAS. (p. 2018)
% - Results show that
%   THE ALGORITHMS HAVE A DIFFERENT PERFORMANCE IN THE CYST CENTER AND NEAR THE BOUNDARY. (p. 2018)
% - FBV has A BETTER PERFORMANCE NEAR THE BOUNDARY; however,
%   DAS IS BETTER IN THE MORE CENTRAL AREA OF THE CYST. (p. 2018)
% VII. CONCLUSION AND DISCUSSION
% - The simulated point target results show the LATERAL RESOLUTIONS OF 0.53 AND 0.66 mm, respectively, for FBV and DAS. (p. 2028)
% - Corresponding axial resolutions were 0.21 mm for FBV and 0.20 mm for DAS. (p. 2028)
% - FBV shows a BETTER LATERAL RESOLUTION at depths higher than 30 mm; however, DAS has a better performance at the lower depths. (p. 2028)
% - Results show that the FBV has a better performance near the boundary; however, DAS is better in the more central area of the cyst. (p. 2028)
% article:GarciaITUFFC2013: Stolt's f-k Migration for Plane Wave Ultrasound Imaging
% Abstract
% - IN VITRO EXPERIMENTS were performed to outline
%   the ADVANTAGES OF PWI WITH STOLT'S f-k MIGRATION OVER
%   the CONVENTIONAL DELAY-AND-SUM (DAS) APPROACH. (p. 1853)
% - Our findings show that
%   MULTI-ANGLE COMPOUNDED f-k MIGRATED IMAGES ARE OF QUALITY SIMILAR to those obtained with
%   a state-of-the-art DYNAMIC FOCUSING MODE. (p. 1853)
% - This remained true even with a very small number of steering angles, thus ensuring a highly competitive frame rate. (p. 1853)
% - In addition, the new FFT-based f-k migration provides
%   COMPARABLE OR BETTER CONTRAST-TO-NOISE RATIO AND LATERAL RESOLUTION than
%   the Lu’s and DAS migration schemes. (p. 1853)
% I. Introduction
% - To
%   [1.)] POTENTIALLY IMPROVE THE IMAGE QUALITY and
%   [2.)] INCREASE THE COMPUTATIONAL EFFICIENCY OF PWI MIGRATION, WE PROPOSE
%   A NEW MIGRATION METHOD CARRIED OUT IN THE FREQUENCY-WAVENUMBER (f-k) DOMAIN, THUS
%   [3.)] BRINGING THE BENEFIT OF MUCH FASTER COMPUTATIONAL SPEED, resulting from the use of the fast Fourier transform (FFT) algorithm, while
%   [4.)] KEEPING HIGH CONTRAST-TO-NOISE RATIO (CNR) and LATERAL RESOLUTION. (p. 1854)
% - In vitro results are presented to outline
%   the BENEFITS OF THE STOLT'S f-k MIGRATION, IN TERMS OF IMAGE QUALITY, over dynamic focusing and DAS. (p. 1854)
% - The new f-k migration process is also compared with Lu’s method. (p. 1854)
% IV. Discussion
% - The resulting new f-k migration yielded
%   HIGH-QUALITY IMAGES IN TERMS OF CNR AND LATERAL RESOLUTION. (p. 1861)
% IV. Discussion / A. Differences Among the Three Migration Methods
% - In this study, we have shown that the new FFT-based f-k migration provided
%   COMPARABLE OR BETTER CONTRAST-TO-NOISE RATIO AND LATERAL RESOLUTION than
%   the Lu’s and DAS migration schemes. (p. 1862)
% V. Conclusion
% - More importantly, the f-k migration has the potential to return BETTER LATERAL RESOLUTION. (p. 1864)
% article:ChengITUFFC2006: Extended high-frame rate imaging method with limited-diffraction beams
% II. Theory / A. Extension of High-Frame Rate Imaging Theory
% - It is worth noting that, because both
%   the HIGH-FRAME RATE IMAGING METHOD [27] and CURRENT EXTENSION ARE BASED ON
%   THE RIGOROUS THEORY OF THE GREEN'S FUNCTION in (14) and (16),
%   IT COULD BE MORE ACCURATE to reflect the scatterer distributions in space THAN
%   THE SIMPLE DELAY-AND-SUM [85] METHOD used in almost all commercial ultrasound scanners. (p. 884)
the image quality at
% 3.) reduced computational costs
% article:MoghimiradITUFFC2016: Synthetic Aperture Ultrasound Fourier Beamformation Using Virtual Sources
% Abstract
% - The results confirm applicability of FBV in ultrasound, and
%   20 times less processing time is attained in comparison with DAS. (p. 2018)
% I. INTRODUCTION
% - Unlike medical applications,
%   most synthetic aperture implementations in the framework of SAR and SAS are processed using blockprocessing algorithms in
%   the FREQUENCY DOMAIN, WHICH IS FASTER THAN DAS because of processing over the whole matrix instead
%   of point-by-point. (p. 2018)
% IV. NUMBER OF COMPUTATIONS
% - In this section, the number of computations for the two methods, DAS and FBV, is calculated and compared. (p. 2023)
% - TABLE I NUMBER OF CALCULATIONS FOR DAS AND FBV (p. 2023)
% - Assume N \approx N_{r} and N_{e} N_{r} >> N, the order of calculation number for DAS to the FBV is approximately then
%   [ O_{DAS} / O_{FBV} = 23 N / ( 59 + 5 log_{2}( M N ) ) ] (31)
%   where O is used for “order of calculations.” (p. 2024)
% - For typical number of M × N = 2000 × 150, it shows that the FBV is about 23 times faster than DAS. (p. 2024)
% VI. RESULTS / C. Processing Time and Memory Usage
% - Assuming a simple implementation using 8-B double-precision numbers, and
%   having six data structures of size 1590 × 64 × 192 (corresponding to the phantom shown in Fig. 13) need
%   937 MB of memory to process the whole data set. (p. 2027)
% - The processing time ratio of DAS to FBV is of the order of 20, which is close to the theoretically estimated one of 23. (p. 2027)
% VII. CONCLUSION AND DISCUSSION
% - In addition, the number of computations for two reconstruction methods, DAS, and FBV has been calculated. (p. 2028)
% - The ORDER OF THE CALCULATIONS IS ABOUT 20 TIMES SMALLER for FBV than DAS. (p. 2028)
% - The cyst measured data evaluated using electronic SNR measurement confirm
%   the applicability of FBV in real medical imaging with about 20 TIMES LESS PROCESSING TIME in comparison with DAS. (p. 2028)
% article:GarciaITUFFC2013: Stolt's f-k Migration for Plane Wave Ultrasound Imaging
% II. Theoretical Background: Adapting the Fourier Domain Stolt's Migration for Plane Wave Imaging / B. Stolt's f-k Migration
% - The Stolt f-k migration is currently, by a wide margin,
%   the FASTEST MIGRATION TECHNIQUE but it is limited to a CONSTANT PROPAGATION WAVE VELOCITY [11]. (p. 1857)
% III. In Vitro Experiments: Stolt's f-k migration against DAS and Lu's method / E. Computational Complexity: Stolt’s f-k Migration Versus DAS
% - The NUMBER OF FLOATING OPERATIONS IS SIGNIFICANTLY REDUCED when
%   the migration of the RF data are carried out in the FOURIER DOMAIN. (p. 1861)
% IV. Discussion
% - This [operation in Fourier space] MAKES THE ALGORITHM MUCH FASTER [than conventional DAS], and
%   any filtering process in the frequency domain can be included without significant increase in computation time. (p. 1861)
% V. Conclusion
% - An advantage of the f-k migration over the classical DAS is
%   ITS FASTER COMPUTATIONAL SPEED, RESULTING FROM THE USE OF FFTs. (p. 1864)
% article:KruizingaITUFFC2012: Plane-wave ultrasound beamforming using a nonuniform fast fourier transform
% I. Introduction
% - The main advantage of the NUFFT with respect to the NDFT is its COMPUTATIONAL EFFICIENCY. (p. 2685)
% - The MAJOR ARGUMENT FOR USING THE FOURIER-DOMAIN APPROACH instead of conventional delay-and-sum (d&s) beamforming is
%   the COMPUTATIONAL GAIN. (p. 2685)
% - Fig. 3. Theoretical values of computation speed at 1 teraflops considering the operations needed for the various beamforming methods. (p. 2686)
% IV. Discussion
% - Fourier domain beamforming can be derived directly from the Green’s function for the imaging system, and hence produces
%   a MORE ACCURATE RECONSTRUCTION OF THE ACTUAL SCATTERER DISTRIBUTION, without additional apodization and fine interpolation [3]–[5]. (p. 2688)
% article:ChengITUFFC2006: Extended high-frame rate imaging method with limited-diffraction beams
% I. Introduction
% - In addition, because Fourier transform can be implemented with a fast Fourier transform (FFT) that is computationally efficient,
%   simpler imaging systems could be constructed to implement the method. (p. 880)
reduced computational costs
\cite{article:ZhangITUFFC2016,article:MoghimiradITUFFC2016,article:GarciaITUFFC2013,article:KruizingaITUFFC2012,proc:SchiffnerAI2012,article:ChengITUFFC2006,article:LuITUFFC1997}.
% f) competing Fourier methods derive the d-dimensional spatial Fourier transform of the image using either an exploding reflector model for steered PWs or variants of the FDT
% article:MoghimiradITUFFC2016: Synthetic Aperture Ultrasound Fourier Beamformation Using Virtual Sources
% III. FOURIER BEAMFORMATION WITH VIRTUAL SOURCES / B. Implementation
% - Basically, the FBV algorithm
%   [1.)] CONVERTS THE MEASURED DATA INTO THE WAVENUMBER DOMAIN and
%   [2.)] PERFORMS A COORDINATE TRANSFORM. (p. 2023)
% article:GarciaITUFFC2013: Stolt's f-k Migration for Plane Wave Ultrasound Imaging
% IV. Discussion / A. Differences Among the Three Migration Methods
% 2) Spectral Mapping of the Fourier-Based Methods:
% - Both Lu and f-k migrations are FOURIER-BASED METHODS. (p. 1862)
% - The ONLY DIFFERENCE LIES IN THEIR SPECTRAL REMAPPING. (p. 1862)
% article:ChengITUFFC2006: Extended high-frame rate imaging method with limited-diffraction beams
% I. Introduction
% - From limited diffraction beam studies,
%   a HIGH-FRAME RATE IMAGING THEORY was developed in 1997 [27], [28] in which
%   [1.)] A PULSED PLANE WAVE WAS USED IN TRANSMISSION, and
%   [2.)] LIMITED-DIFFRACTION ARRAY BEAM WEIGHTINGS were applied to received echo signals to produce
%   a SPATIAL FOURIER TRANSFORM OF OBJECT FUNCTION for 3-D image reconstruction. (p. 880)
% II. Theory
% - In this section,
%   the theory of high-frame rate imaging [27], [28] is extended to include explicitly
%   various transmission schemes such as
%   multiple limited-diffraction array beams and
%   steered plane waves (the derivations will be in parallel with those in [27]) [61], [62]. (p. 881)
% - The proof that
%   limited-diffraction array beam weightings of received echo signals over a 2-D transducer aperture are the same as
%   a 2-D Fourier transform of these signals over the same aperture is also given [61], [62]. (p. 881)
% VI. Discussion / E. Finite Aperture and Image Resolution
% - The theory of the HFR imaging method was developed with the assumption that
%   the TRANSDUCER APERTURE IS INFINITELY LARGE. (p. 896)
% - However, the aperture of a practical transducer is ALWAYS FINITE. (p. 896)
% - As is well-known from Goodman's book [84], a finite aperture will decrease image resolution. (p. 896)
Detecting
% 1.) scattered waves
the scattered waves on
% 2.) infinite planes
infinite planes,
% 3.) competing Fourier methods
they derive
% 4.) d-dimensional spatial Fourier transform
% article:GarciaITUFFC2013: Stolt's f-k Migration for Plane Wave Ultrasound Imaging
% IV. Discussion
% - Although the theory has been derived in TWO DIMENSIONS,
%   the f-k migration for PWI can be generalized in three dimensions. (p. 1861)
the $\{ 2, 3 \}$-dimensional spatial \name{Fourier} transform of
% 5.) image
the image using either
% 6.) exploding reflector model for steered PWs
% article:GarciaITUFFC2013: Stolt's f-k Migration for Plane Wave Ultrasound Imaging
% Abstract
% - As a PLANE WAVE reaches a given scatterer, the latter becomes a secondary source emitting upward spherical waves and
%   CREATING A DIFFRACTION HYPERBOLA IN THE RECEIVED RF SIGNALS. (p. 1853)
% - To produce an image of the scatterers,
%   ALL THE HYPERBOLAS MUST BE MIGRATED BACK TO THEIR APEXES. (p. 1853)
% - The f-k migration for PWI has been ADAPTED FROM THE STOLT MIGRATION FOR SEISMIC IMAGING. (p. 1853)
% - This migration technique is BASED ON THE EXPLODING REFLECTOR MODEL (ERM), which consists in
%   ASSUMING THAT ALL THE SCATTERERS EXPLODE IN CONCERT AND BECOME ACOUSTIC SOURCES. (p. 1853)
% I. Introduction
% - The f-k MIGRATION FOR PWI is inspired by the
%   ORIGINAL FOURIER MIGRATION INTRODUCED BY STOLT FOR SEISMIC IMAGING [10], [11]. (p. 1854)
% - In this manuscript,
%   WE DERIVE A NEW FOURIER f-k MIGRATION TECHNIQUE FOR PLANE WAVE ULTRASOUND IMAGING by
%   MODIFYING the so-called EXPLODING REFLECTOR MODEL. (p. 1854)
% IV. Discussion
% - In the present paper,
%   A WELL-ESTABLISHED SEISMIC MIGRATION METHOD - the Stolt’s f-k migration - HAS BEEN MODIFIED FOR PWI. (p. 1861)
% - Because we are NOT IN THE SPECIFIC ZERO-OFFSET CONDITION of seismic imaging,
%   WE NEEDED TO ADAPT THE EXPLODING REFLECTOR MODEL (ERM). (p. 1861)
% - We demonstrated that
%   THE ERM CAN BE MADE SUITABLE TO PWI BY FINE-TUNING THE DIFFRACTION HYPERBOLAS PRESENT IN THE RF DATA. (p. 1861)
an \acl{ERM} for
steered \acp{PW}
\cite{article:GarciaITUFFC2013} or
% 7.) variants of the FDT
% article:MoghimiradITUFFC2016: Synthetic Aperture Ultrasound Fourier Beamformation Using Virtual Sources
% II. MULTISTATIC WAVENUMBER ALGORITHM
% - Image reconstruction is an inverse problem to create an IMAGE OF THE MEDIUM REFLECTIVITY FROM THE MEASUREMENTS OF THE ECHO SIGNALS. (p. 2019)
% - Assuming that superposition applies, most algorithms try to invert the system model. (p. 2019)
% - One of the frequency domain reconstruction methods is the WAVENUMBER ALGORITHM, which
%   PERFORMS THE INVERSION USING A SOLUTION TO THE WAVE EQUATION CONSIDERING A CONSTANT SOUND SPEED IN THE MEDIUM. (p. 2019)
% - It has been developed for SAR/SAS based on one-element transmission setup; however,
%   the medical ultrasound imaging is usually performed using focused emissions. (p. 2019)
% - Neglecting attenuation, which can be compensated using a time gain compensation, and assuming that superposition applies,
%   the received echo signal for each TX and RX pair is [28], [29], [32]
%   [ E( u_{rx}, u_{tx}, ω ) = P(ω) \int f(z,x) g( \rho_{in}, ω ) g( \rho_{out}, ω ) dz dx ] (1)
%   where E( u_{rx}, u_{tx}, ω ) denote the temporal spectrum of the received echo signal e( u_{rx}, u_{tx}, t ), and
%   P(ω) is the Fourier transform of the transmitted signal p(t). (p. 2020)
% - f(z, x) is the medium scattering function, which is estimated to form a final image.
% - g( \rho_{in}, ω ) and g( \rho_{out}, ω ) are the spatial responses for
%   the point scatterers in transmit and receive, based on the 2-D free-space Green’s function [28], [29], [32]
%   [ g( \rho, ω ) = -j \hankel{0}{2}( k \rho ) / 4 ] (2)
%   where [...]. (p. 2020)
% - Although the collected data fill a rectangular block in (ω, ku) space, it corresponds to
%   a curved surface of constant temporal frequency, ω, in the (kz, kx ) wavenumber space of the image. (p. 2020)
% - To generate the image estimate from these data via the inverse FFT, the samples should be on a rectangular grid. (p. 2020)
% - The Stolt mapping is then used to interpolate the input samples onto a regular grid. (p. 2020)
% - A 2-D inverse Fourier transform from the interpolated data \hat{f}( k_{z}, k_{x} ) then gives the final image f(z, x). (p. 2020)
% - An inverse Fourier transform then provides an estimate of the medium reflectivity function. (p. 2023)
% V. SIMULATIONS AND EXPERIMENTAL SETUP
% - A full image is beamformed for each emission and a Hanning apodization window is used when combining
%   the low-resolution images in DAS. (p. 2024)
% article:GarciaITUFFC2013: Stolt's f-k Migration for Plane Wave Ultrasound Imaging
% I. Introduction
% - FFT-BASED BEAMFORMING for ultrasonic imaging has been the topic of
%   numerous studies since the early 1980s [12]–[19]. (p. 1854)
% - Most of them are based on the ANGULAR SPECTRUM METHOD, which consists in
%   DECOMPOSING THE REFLECTED WAVEFIELD INTO PLANE WAVES, each propagating at a different angle. (p. 1854)
% - Although originally derived for MONOCHROMATIC WAVES,
%   the ANGULAR SPECTRUM METHOD HAS BEEN EXTENDED TO WIDEBAND SYSTEMS [13], [14]. (p. 1854)
% - To reach very high frame rates,
%   an FFT-based reconstruction of ultrasound images obtained by plane wave insonifications has been successfully addressed by
%   J.-y. lu and his team [23]–[25]. (p. 1854)
%   [23] article:LuITUFFC1997, [24] article:LuITUFFC1998, [25] article:ChengITUFFC2006
% - In Lu’s method,
%   the RF image is essentially remapped in the Fourier domain by interpolating the temporal frequencies. (p. 1854)
% - This approach is based on the assumption that the scatterers all behave as MONOPOLE SOURCES [27]. (p. 1854)
variants of
the \acl{FDT}%
\footnote{
  % a) FDT adapts the Fourier slice theorem underlying the image recovery in CT from rays to diffracting waves
  % book:Devaney2012, Chapter 8: Classical inverse scattering and diffraction tomography
  % Sect. 8.9: Diffraction tomography in two space dimensions / Sect. 8.9.1: The generalized projection-slice theorem
  % - Theorem 8.4 (the generalized projection-slice theorem) (p. 372)
  % book:Natterer2001, Sect. 3.3: Diffraction Tomography
  % - In the following,
  %   WE EXTEND THE “CENTRAL SLICE THEOREM” (Theorem 2.1) TO THE PROPAGATION OPERATOR U_{r} of (3.17). (p. 48)
  % - The theorem shows that f is determined by (U_{r} f)^ on
  %   semispheres around -k \vartheta with vertex at 0 for r > p and vertex at -2k\vartheta for r < -p; see Figure 3.7. (p. 49)
  % book:Kak2001, Chapter 6: Tomographic Imaging with Diffracting Sources
  % - In this chapter, we will show that if the interaction of an object and a field is modeled with the wave equation, then
  %   a tomographic reconstruction approach based on the FOURIER DIFFRACTION THEOREM is possible for WEAKLY DIFFRACTING OBJECTS. (p. 203)
  % - When diffraction effects are included, the FOURIER DIFFRACTION THEOREM says that a “projection” yields
  %   the FOURIER TRANSFORM OF THE OBJECT OVER A SEMICIRCULAR ARC. (p. 203)
  % - THIS RESULT IS FUNDAMENTAL TO DIFFRACTION TOMOGRAPHY. (p. 203)
  % book:Kak2001, Sect. 6.3: The Fourier Diffraction Theorem
  % - Fundamental to diffraction tomography is the FOURIER DIFFRACTION THEOREM, which relates
  %   [1.)] the Fourier transform of the measured forward scattered data with
  %   [2.)] the Fourier transform of the object. (p. 218)
  % - The theorem is valid when the inhomogeneities in the object are only WEAKLY SCATTERING. (p. 218)
  % - The statement of the theorem is as follows:
  %	- When an object, o(x, y), is illuminated with a plane wave as shown in Fig. 6.2,
  %	  the Fourier transform of the forward scattered field measured on line TT' gives
  %	  the values of the 2-D transform, O(w1, w2), of the object along
  %	  a semicircular arc in the frequency domain, as shown in the right half of the figure. (p. 218)
  The \acs{FDT}
  (cf. e.g.
  \cite[Thm. 8.4]{book:Devaney2012},
  \cite[Thm. 3.1]{book:Natterer2001},
  \cite[Sect. 6.3]{book:Kak2001}%
  ), which is also referred to as
  % 1.) generalized projection-slice theorem
  the \term{generalized projection-slice theorem}, adapts
  % 2.) Fourier slice theorem
  the \name{Fourier} slice theorem underlying
  % 3.) image recovery
  the image recovery in
  % 4.) x-ray computed tomography (CT)
  \acl{CT} from
  % 5.) rays
  rays to
  % 6.) diffracting waves
  diffracting waves.
  % b) FDT expands both the incident and scattered waves into steered PWs
  Using
  % 1.) Weyl expansions
  % book:Devaney2012, Chapter 4: Angular-spectrum and multipole expansions / Sect. 4.1: The Weyl expansion
  % - Although there are several ways of deriving the ANGULAR-SPECTRUM EXPANSION of the field U_{+}, the most direct procedure is to expand
  %   the OUTGOING-WAVE GREEN FUNCTION G_{+} in an angular-spectrum expansion and then substitute
  %   this expansion into the Green-function solution for U_{+} given in Eq. (2.23). (pp. 118, 119)
  % - The ANGULAR-SPECTRUM EXPANSION OF THE OUTGOING-WAVE GREEN FUNCTION, originally due to Weyl (Weyl, 1919) and called the WEYL EXPANSION, is derived directly from
  %   the Fourier-integral representation of the outgoing-wave Green function given in Eq. (2.15). (p. 119)
  % - Our goal here, however, is not to obtain a closed-form expression for G_{+} (which we already have) but, rather, to express
  %   G_{+} as a superposition of plane waves all of which satisfy the homogeneous Helmholtz equation. (p. 119)
  the \name{Weyl} expansions of
  % 2.) outgoing free-space Green's functions (two- and three-dimensional Euclidean spaces)
  the outgoing free-space \name{Green}'s functions
  \cite[Sect. 4.1]{book:Devaney2012}, it decomposes
  % 3.) all waves
  all waves into
  % 4.) steered PWs
  steered \acp{PW} and, thus, facilitates
  % 5.) treatment
  their treatment.
} (\acs{FDT})\acused{FDT} for
% 8.) non-diffracting beams (steered PWs, X waves)
% article:ChengITUFFC2006: Extended high-frame rate imaging method with limited-diffraction beams
% Abstract
% - In this paper, THE THEORY [article:LuITUFFC1997] IS EXTENDED TO INCLUDE EXPLICITLY VARIOUS TRANSMISSION SCHEMES such as
%   [1.)] MULTIPLE LIMITED-DIFFRACTION ARRAY BEAMS and
%   [2.)] STEERED PLANE WAVES. (p. 880)
% I. Introduction
% - Forty-six years later, Durnin [2] and Durnin et al. [3] studied the Bessel beam again and termed the beam
%   “NONDIFFRACTING BEAM” or “DIFFRACTION-FREE BEAM”. (p. 880)
% - Because Durnin’s terminologies are controversial in the scientific community and
%   practical beams will eventually diffract,
%   we termed the propagation-invariant beams or waves “LIMITED DIFFRACTION BEAMS” [4]. (p. 880)
% - One class of limited diffraction beams is called X WAVE [44], [45] and has been investigated by many physicists [49]–[56]. (p. 880)
% - In this paper, the theory of high-frame rate imaging [27], [28] is extended to include explicitly
%   various transmission schemes such as
%   [1.)] MULTIPLE LIMITED-DIFFRACTION ARRAY BEAMS and
%   [2.)] STEERED PLANE WAVES [61], [62] (the first report was given in [61]). (p. 880)
non-diffracting beams, e.g.
% 8.a) steered PWs
steered \acp{PW}
\cite{article:KruizingaITUFFC2012,proc:SchiffnerAI2012,article:ChengITUFFC2006,article:LuITUFFC1997} or
% 8.b) X waves
% article:ChengITUFFC2006: Extended high-frame rate imaging method with limited-diffraction beams
% III. Relationships Between Fourier Domains of Echoes and Object Function / A. Image Reconstruction with Limited-Diffraction Array Beams
% - For LIMITED-DIFFRACTION ARRAY BEAM TRANSMISSIONS, BOTH SINE AND COSINE WEIGHTINGS ARE APPLIED; thus
%   the ECHOES NEED TO BE COMBINED using a 2-D version of (30)–(33) to get
%   TWO NEW SETS OF ECHOES before the mapping process above. (p. 887)
\TODO{exact name of wave}
X waves
\cite{article:ChengITUFFC2006}, and
% 9.) outgoing 1-spherical (cylindrical) waves
% article:MoghimiradITUFFC2016: Synthetic Aperture Ultrasound Fourier Beamformation Using Virtual Sources
% III. FOURIER BEAMFORMATION WITH VIRTUAL SOURCES
% - Using one-element transmission limits the penetration depth, due to the low energy transmitted into the medium. (p. 2020)
% - Multielement transmission with focused virtual sources can increase the transmitted energy [13], [45]–[48]. (p. 2020)
% - There are two different placements of the virtual source location: in front of the transducer or behind it. (p. 2020)
% - These configurations have different properties and are chosen depending on the application [49]. (p. 2020)
% - For example, using a virtual source in front of the transducer provides
%   a higher transmitted energy especially in the focused point; however,
%   the energy spreading is not uniform close to the transducer, resulting in
%   some differences between the points above and below the focused point. (pp. 2020, 2021)
% - On the other hand, USING VIRTUAL SOURCE BEHIND THE TRANSDUCER YIELDS
%   LOWER BUT A MORE HOMOGENEOUS ENERGY DISTRIBUTION IN THE IMAGE. (p. 2021)
% - The virtual source is considered a point source, which transmits spherical waves. (p. 2021)
% - Another configuration is to PLACE THE VIRTUAL SOURCES BEHIND THE TRANSDUCER SURFACE, as shown in Fig. 4. (p. 2021)
% V. SIMULATIONS AND EXPERIMENTAL SETUP
% - Each imaging sequence was performed using 182 transmissions with
%   A VIRTUAL SOURCE BEHIND THE TRANSDUCER and 64 elements receiving centered around the TX. (p. 2024)
% VII. CONCLUSION AND DISCUSSION
% - In this paper,
%   an FBV algorithm was introduced for medical ultrasound imaging with
%   a multielement transmit/receive configuration assuming
%   a VIRTUAL SOURCE BEHIND OR IN FRONT OF THE TRANSDUCER. (p. 2028)
outgoing $1$-spherical (cylindrical) waves
\cite{article:MoghimiradITUFFC2016}.
\TODO{\cite{article:ZhangITUFFC2016}}

% article:GarciaITUFFC2013: Stolt's f-k Migration for Plane Wave Ultrasound Imaging
% IV. Discussion / C. PWI Using f-k Migration: Limitations and Perspectives
% - First, the wave equation is not limited to a plane and
%   OUT-OF-PLANE SCATTERERS MAY THUS CONTRIBUTE TO THE RF SIGNALS. (p. 1863)
% - Besides the technical aspects,
%   the f-k MIGRATION MODEL is also subject to restrictions. (p. 1863)
% - This algorithm is based on the 2-D WAVE EQUATION and the BORN APPROXIMATION, assuming
%   a constant speed of sound and
%   only upcoming waves. (pp. 1863, 1864)
% - [48] Q. Chen and J. A. Zagzebski, “Simulation study of effects of speed of sound and attenuation on ultrasound lateral resolution,” Ultrasound Med. Biol., vol. 30, no. 10, pp. 1297–1306, Oct. 2004.

%---------------------------------------------------------------------------------------------------------------
% 3.) benefits of the novel inverse scattering methods
%---------------------------------------------------------------------------------------------------------------
% a) novel inverse scattering methods improve the image quality
% article:BessonITUFFC2018: Ultrafast Ultrasound Imaging as an Inverse Problem: Matrix-Free Sparse Image Reconstruction [Mar. 2018]
% Abstract
% - An alternative to DAS consists in using ITERATIVE TECHNIQUES, which require both
%   [1.)] AN ACCURATE MEASUREMENT MODEL and
%   [2.)] A STRONG PRIOR ON THE IMAGE under scrutiny. (p. 339)
% I. INTRODUCTION
% - Retrieving
%   an image of the medium inhomogeneities from
%   the element raw-data poses
%   an ill-posed inverse problem. (p. 339)
% - These methods are
%   [1.)] BUILT UPON FORWARD MODELS OF THE PROBLEM and
%   [2.)] INTRODUCE ADDITIONAL INFORMATION ON THE SIGNAL UNDER SCRUTINY in order to solve
%   the ill-posed inverse problem, leading to
%   A HIGHER IMAGE QUALITY THAN BACKPROJECTION METHODS. (p. 339)
% article:BerthonPMB2018: Spatiotemporal matrix image formation for programmable ultrasound scanners [Feb. 2018]
% Abstract:
% - In this work, we argue that as the computational power keeps increasing, it is becoming practical to
%   DIRECTLY IMPLEMENT AN APPROXIMATION TO THE MATRIX OPERATOR LINKING
%   [1.)] REFLECTOR POINT TARGETS to
%   [2.)] THE CORRESPONDING RADIOFREQUENCY SIGNALS via
%   [3.)] THOROUGHLY VALIDATED AND WIDELY AVAILABLE SIMULATIONS SOFTWARE. (p. 1)
% - Once such a SPATIOTEMPORAL FORWARD-PROBLEM MATRIX is constructed,
%   STANDARD AND THUS HIGHLY OPTIMIZED INVERSION PROCEDURES can be leveraged to
%   ACHIEVE VERY HIGH QUALITY IMAGES IN REAL TIME. (p. 1)
% - Specifically, we show that
%   [1.)] spatiotemporal matrix image formation produces IMAGES OF SIMILAR OR ENHANCED QUALITY when compared against STANDARD DELAY-AND-SUM APPROACHES IN PHANTOMS AND IN VIVO, and show that
%   [2.)] this APPROACH CAN BE USED TO FORM IMAGES EVEN WHEN USING NON-CONVENTIONAL PROBE DESIGNS for which adapted image formation algorithms are not readily available. (p. 1)
% 1. Introduction
% - Another approach to image reconstruction consists in
%   [1.)] POPULATING A SPATIOTEMPORAL MATRIX APPROXIMATING THE FORWARD MODEL,
%   [2.)] FINDING A COLLECTION OF MEASUREMENTS FROM AN OBJECT TO IMAGE, and
%   [3.)] APPLYING GENERAL MATRIX INVERSION STRATEGIES such as the pseudo-inverse TO FORM IMAGES. (p. 2)
% - In biomedical ultrasound imaging, they [inverse scattering] have been used to
%   [1.)] IMPROVE IMAGE QUALITY (Madore and Can Meral 2012) and
%   [2.)] REDUCE THE REQUIRED AMOUNT OF DATA needed to form an image using compressed sensing
%   (Quinsac et al 2012, Schiffner et al 2012, David et al 2015),
%   [3.)] to simplify real-time graphical-processing-units based implementations of DAS algorithms (Hou et al 2014), or
%   [4.)] to EXTRACT ADDITIONAL INFORMATION FROM THE DATA (Lavarello et al 2006), such as
%         super-resolved imaging in the near field (Viola et al 2008, Ellis et al 2010). (p. 2)
% - Specifically, WE PROPOSE AN IMAGE FORMATION FRAMEWORK based on
%   the EXPLICIT CONSTRUCTION OF THE FORWARD PROBLEM SPATIOTEMPORAL MATRIX, which
%   consists of a DICTIONARY of pixel-specific transmit-receive signals in the form of reference emitted and received signals defined for each pixel. (p. 2)
% - These can be INVERSED to form images via simple, highly optimized and widely available matrix-vector product algorithms. (p. 2)
% - This DICTIONARY CAN INCLUDE ANY AMOUNT OF A PRIORI INFORMATION obtained by leveraging
%   EXISTING, OPEN-SOURCE OR FREEWARE SIMULATION SOFTWARE (for example Matlab-based software such as K-wave or Field II) or
%   EXPERIMENTAL MEASUREMENTS, and thus streamlines the inclusion of all the available a priori knowledge in the image formation process. (p. 2)
% - We show that when used in standard settings, SMIF EQUATES OR SURPASSES STANDARD DAS BEAMFORMING IN TERMS OF RESOLUTION AND CONTRAST. (p. 2)
% - Additionally, we highlight how novel sequences or probe designs can be implemented seamlessly and show the capability of SMIF to produce high quality in vivo images. (p. 2)
Novel inverse scattering methods, which increase both
% 1.) complexity
the complexity and
% 2.) computational costs
% article:BessonITUFFC2018: Ultrafast Ultrasound Imaging as an Inverse Problem: Matrix-Free Sparse Image Reconstruction [Mar. 2018]
% I. INTRODUCTION
% - The main problem of these models resides in their COMPUTATIONAL COMPLEXITY, usually translated in
%   STORAGE REQUIREMENTS OF THE CORRESPONDING MATRIX REPRESENTATION. (p. 340)
% - The models proposed by David et al. [10] and Schiffner and Schmitz [15] require
%   the storage of several hundreds of gigabytes for matrix coefficients in 2-D. (p. 340)
% - Zhang et al. [13] have divided the image in stripes in order to make the problem tractable. (p. 340)
% - This issue severely limits the appeal of ITERATIVE METHODS AGAINST CLASSICAL APPROACHES. (p. 340)
% article:BerthonPMB2018: Spatiotemporal matrix image formation for programmable ultrasound scanners [Feb. 2018]
% 1. Introduction
% - Yet, THIS TYPE OF IMAGE FORMATION STRATEGY [inverse scattering] WAS NOT CONSIDERED PRACTICAL FOR STANDARD IMAGE FORMATION, perhaps because of
%   the LARGE COMPUTATIONAL OVERHEADS THAT LIMITED THEIR REAL-TIME APPLICATION. (p. 2)
% - While constructing the dictionary can be more or less time consuming depending on the level of sophistication of the simulation software or experiments,
%   the inverse problem can be solved in a few milliseconds as it consists of simple matrix-vector products and thus it enables real-time imaging. (p. 2)
the computational costs, improve
% 3.) image quality
the image quality
\cite{article:OzkanITUFFC2018,article:BessonITUFFC2018,article:BerthonPMB2018,article:BessonITUFFC2016,article:DavidJASA2015,article:ZhangUlt2013,proc:SchiffnerIUS2013a,proc:SchiffnerIUS2013b,proc:SchiffnerIUS2012,article:SchiffnerBMT2012,proc:SchiffnerIUS2011}.
% b) novel inverse scattering methods regularize the ill-posed recovery problem given both a set of echo signals and a linear model for its prediction
% article:GarciaITUFFC2013: Stolt's f-k Migration for Plane Wave Ultrasound Imaging
% IV. Discussion / A. Differences Among the Three Migration Methods
% - One must be aware that migration is an ILL-POSED INVERSE PROBLEM. (p. 1862)
% - Available information (acoustic pressure at z = 0, only) is INSUFFICIENT to recover the insonified medium. (p. 1862)
% - Additional assumptions are thus required to close the problem. (p. 1862)
They regularize
% 1.) ill-posed image recovery problem [linear inverse problem]
% book:Hansen2010, Chapter 1: Introduction and Motivation
% - INVERSE PROBLEMS, in turn, belong to the CLASS OF ILL-POSED PROBLEMS. (p. 2)
% - Hadamard’s definition says that a LINEAR PROBLEM IS WELL-POSED if it satisfies the following three requirements:
%   [1.)] EXISTENCE: The problem must have a solution.
%   [2.)] UNIQUENESS: There must be only one solution to the problem.
%   [3.)] STABILITY: The solution must depend continuously on the data. (p. 2)
% - If the problem violates one or more of these requirements, it is said to be ILL-POSED. (p. 2)
% - The stability condition is much harder to “deal with” because a violation implies that
%   arbitrarily small perturbations of data can produce arbitrarily large perturbations of the solution. (p. 3)
% - Again the key is to REFORMULATE THE PROBLEM such that
%   THE SOLUTION TO THE NEW PROBLEM IS LESS SENSITIVE TO THE PERTURBATIONS. (p. 3)
% - We say that
%   WE STABILIZE OR REGULARIZE THE PROBLEM, such that
%   the solution becomes more stable and regular. (p. 3)
the ill-posed recovery problem given both
% 2.) set of echo signals
a set of
echo signals and
% 3.) linear model for its [set of echo signals] prediction
a linear model for
its prediction.
% c) universality optimally complements the programmable UI systems
% article:BerthonPMB2018: Spatiotemporal matrix image formation for programmable ultrasound scanners
% 1. Introduction
% - The NOVELTY OF THIS APPROACH RESIDES IN ITS GENERALITY:
%   SPATIOTEMPORAL MATRIX IMAGE FORMATION (SMIF) can act as
%   [1.)] a REAL-TIME, UNIVERSAL IMAGE FORMATION FRAMEWORK based on an INVERSE PROBLEM FORMULATION that can handle
%   [2.)] ANY TYPE OF EMISSIONS AND PROBE CHARACTERISTICS, regardless of their complexity. (p. 2)
% Merriam-Webster: universal = adapted or adjustable to meet varied requirements (as of use, shape, or size)
Their universality, which stems from
% 1.) arbitrary sophistication
the arbitrary sophistication of
% 2.) linear model for the prediction of the set of echo signals
this model and
% 3.) optional calibration
its optional calibration via
% 4.) experimental measurements
experimental measurements, optimally complements
% 5.) freely programmable UI systems
the programmable \ac{UI} systems.
%The universality of
% 1.) linear model for the prediction of the set of echo signals
%this model, which includes
% 2.) optional calibration
%its optional calibration via
% 3.) experimental measurements
%experimental measurements, optimally complements
% 4.) freely programmable UI systems
%the programmable \ac{UI} systems.
% d) features of the novel inverse scattering methods
% article:BerthonPMB2018: Spatiotemporal matrix image formation for programmable ultrasound scanners
% 1. Introduction
% - While such an approach [implementation of closed-form solution to the inverse problem] is IDEAL TO GENERATE FAST ALGORITHMS, it is impractical to
%   [1.)] GENERALIZE THEM TO NEW TYPES OF ULTRASOUND WAVE EMISSIONS or to
%   [2.)] ENHANCE THEM BY INCLUDING ADDITIONAL A PRIORI KNOWLEDGE. (p. 1)
% - Indeed, in practice, it means that
%   A DIFFERENT IMAGE FORMATION ALGORITHM must be designed, written, tested, and optimized to include, e.g.
%   different types of waves (plane, diverging, focused, or a combination thereof), coherent compounding, chirps,
%   attenuation, different probe geometries, element angular sensitivity, or the lens characteristics. (p. 2)
For instance,
% 1.) novel inverse scattering methods
they theoretically support
% 2.) incident waves
incident waves and
% 3.) array geometries
array geometries of
% 4.) any complexity
any complexity,
% 5.) separate recovery
the separate recovery of
% 6.) multiple acoustic material parameters
multiple acoustic material parameters,
% 7.) efficient spatiotemporal sampling concepts for data rate reduction
% article:ChernyakovaITUFFC2018: Fourier-Domain Beamforming and Structure-Based Reconstruction for Plane-Wave Imaging
% Abstract
% - This technique [coherent plane-wave compounding], however,
%   REQUIRES LARGE COMPUTATIONAL LOADS MOTIVATING METHODS FOR SAMPLING AND PROCESSING RATE REDUCTION. (p. 1)
% I. INTRODUCTION
% - This method [coherent plane-wave compounding], however, is CHALLENGING due to
%   the HIGH DATA TRANSFER RATES AND LARGE COMPUTATIONAL LOAD. (p. 1)
% - To achieve ultrafast imaging,
%   ALL IMAGE LINES ARE COMPUTED IN PARALLEL, TYPICALLY ON A SOFTWARE BASED PLATFORM. (p. 1)
% - This implies that
%   SAMPLED RAW RADIO FREQUENCY SIGNALS, detected at each array element, are
%   DIRECTLY TRANSFERRED TO THE PROCESSING UNIT. (p. 1)
% - Each image line is obtained by STANDARD TIME DOMAIN BEAMFORMING, implying
%   SAMPLING RATES THAT ARE MUCH HIGHER THAN
%   THE NYQUIST RATE OF THE DETECTED SIGNALS. (p. 1)
% - Rates up to 4-10 TIMES THE CENTRAL FREQUENCY OF THE TRANSMITTED PULSE ARE USED in order to
%   eliminate artifacts caused by digital implementation of beamforming in time [4]. (p. 1)
%   [4] article:SteinbergITUFFC1992
% - Taking into account the number of transducer elements,
%   UP TO 10^{7} SAMPLES NEED TO BE TRANSFERRED AND DIGITALLY PROCESSED IN REAL TIME TO OBTAIN AN IMAGE. (p. 1)
% - THE PROCESSING UNIT, therefore, MUST BE POWERFUL ENOUGH TO ALLOW FOR REAL TIME BEAMFORMING. (p. 1)
% - Therefore,
%   ALTERNATIVES FOR SAMPLING AND PROCESSING RATE REDUCTION ARE OF HIGH INTEREST and can
%   LEAD TO SIMPLER AND CHEAPER SYSTEMS. (p. 1)
% - They may also serve as a potential enabler for the concept of a wireless probe and remote processing [9]. (p. 1)
% REDUCTION OF DATA RATES
% - One approach to reduce sampling rate is by QUADRATURE SAMPLING [5]. (p. 1)
% - Here the signals are
%   [1.)] sampled at rates dictated by the RF (radio-frequency) Nyquist condition and then
%   [2.)] DIGITALLY DEMODULATED AND DECIMATED TO THE EFFECTIVE NYQUIST RATE defined by the SIGNALS BANDPASS BANDWIDTH. (p. 1)
% - Digital IQ demodulation and decimation require
%   [1.)] multiplication by complex exponentials and
%   [2.)] low-pass filtering of each sampled signal, thus
%   INCREASING THE OVERALL COMPUTATIONAL LOAD. (p. 1)
% - Moreover,
%   elastography, one of the main applications of ultrafast imaging, requires RF ultrasound data for tissue deformation calculations [6]–[8]. (p. 1; really?)
% article:ProvostPMB2014: 3D ultrafast ultrasound imaging in vivo
% Abstract
% - A customized portable ultrasound system was developed to
%   SAMPLE 1024 INDEPENDENT CHANNELS and to
%   DRIVE A 32 x 32 MATRIX-ARRAY PROBE. (p. L1)
% 1. Methods / 1.1. System infrastructure
% - The 1024 INDEPENDENT CHANNELS COULD BE USED SIMULTANEOUSLY IN TRANSMISSION, whereas
%   RECEIVE CHANNELS WERE MULTIPLEXED to 1 of 2 transducer elements. (p. L3)
% - Therefore, each emission was repeated twice, with
%   the first half of the elements receiving during the first emission, and
%   the second half of the elements receiving during the second emission. (p. L3)
% - Virtual arrays can be tailored to adjust resolu- tion, contrast, signal-to-noise ratio, volume rate, and the field of view in a quasi-continuous
%   fashion, therefore allowing for the selection of the optimal imaging sequence for a specific application
%   (Lockwood et al 1998, Nikolov 2001, Montaldo et al 2009, Nikolov et al 2010, Papadacci et al 2014a). (pp. L4, L5)
% - Indeed, a number of trade-offs between contrast, resolution, volume rate, and field of view exist. (p. L5)
efficient spatiotemporal sampling concepts for
data rate reduction,
% 8.) denoising
denoising, and
% 9.) inclusion of a priori information about the image
the inclusion of
\emph{a priori} information about
the image.
% e) cutting-edge variants have recently adopted CS to disrupt the tradeoff between the image quality and the frame rate
% article:BessonITUFFC2018: Ultrafast Ultrasound Imaging as an Inverse Problem: Matrix-Free Sparse Image Reconstruction
% Abstract
% - We present two different techniques, which
%   [1.)] take advantage of fast and matrix-free formulations derived for the measurement model and its adjoint, and
%   [2.)] rely on sparsity of US images in well-chosen models. (p. 339)
% - Compressed beamforming exploits the compressed sensing framework to restore
%   HIGH-QUALITY IMAGES FROM FEWER RAW DATA than state-of-the-art approaches. (p. 339)
% - Sparse regularization is used for ENHANCED IMAGE RECONSTRUCTION. (p. 339)
% I. INTRODUCTION
% - An ALTERNATIVE TO BACKPROJECTION METHODS consists of
%   SPARSE REGULARIZATION (SR) TECHNIQUES [5]. (p. 339)
% - Medical imaging is well suited to SR methods. (p. 339)
Cutting-edge variants
\cite{article:OzkanITUFFC2018,article:BessonITUFFC2018,article:BessonITUFFC2016,article:DavidJASA2015,proc:SchiffnerIUS2013a,article:ZhangUlt2013,proc:SchiffnerIUS2012,article:SchiffnerBMT2012,proc:SchiffnerIUS2011}
have recently adopted
% 1.) compressed sensing
% book:Foucart2013, Chapter 1: An Invitation to Compressive Sensing / Sect. 1.1: What is Compressive Sensing? (Aug. 2013)
% - Thus, it came as a surprise that under certain assumptions it is actually possible to reconstruct signals when
%   the number m of available measurements is smaller than the signal length N. (p. 2)
% - Even more surprisingly, efficient algorithms do exist for the reconstruction. (p. 2)
% - The UNDERLYING ASSUMPTION which makes all this possible is SPARSITY. (p. 2)
% - THE RESEARCH AREA ASSOCIATED TO THIS PHENOMENON HAS BECOME KNOWN AS
%   COMPRESSIVE SENSING, COMPRESSED SENSING, COMPRESSIVE SAMPLING, OR SPARSE RECOVERY. (p. 2)
% - This whole book is devoted to the mathematics underlying this field. (p. 2)
% book:Eldar2012:
% article:DonohoITIT2006: Compressed Sensing
\term{\acl{CS}}%
\footnote{
  % a) disambiguation of the terms compressed sensing and sparse recovery
  % book:Foucart2013, Chapter 1: An Invitation to Compressive Sensing / Sect. 1.2 Applications,Motivations, and Extensions (Aug. 2013)
  % Sparse Approximation
  % - There are, however, some DIFFERENCES IN PHILOSOPHY compared to the compressive sensing problem. (p. 17)
  %   [1.)]
  % - In the latter, one is often FREE TO DESIGN THE MATRIX A WITH APPROPRIATE PROPERTIES, while
  %   A is usually PRESCRIBED IN THE CONTEXT OF SPARSE APPROXIMATION. (p. 17)
  % - In particular, IT IS NOT REALISTIC TO RELY ON RANDOMNESS AS IN COMPRESSIVE SENSING. (p. 17)
  % - Since it is hard to verify the conditions ensuring sparse recovery in the optimal parameter regime (m linear in s up to logarithmic factors),
  %   the THEORETICAL GUARANTEES FALL SHORT OF THE ONES ENCOUNTERED FOR RANDOM MATRICES. (p. 17)
  %   [2.)]
  % - The second difference between sparse approximation and compressive sensing appears in
  %   the TARGETED ERROR ESTIMATES. (p. 18)
  % - In COMPRESSIVE SENSING, one is interested in the ERROR \norm{ \vect{x} − \vect{x}# } AT THE COEFFICIENT LEVEL, where
  %   \vect{x} and \vect{x}# are the original and reconstructed coefficient vectors, respectively, while
  %   in SPARSE APPROXIMATION, the goal is to approximate a given \vect{y} with a sparse expansion \vect{y}# = \sum_{j} x_{j}# \vect{a}_{j},
  %   so one is rather interested in \norm{ \vect{y} − \vect{y}# }. (p. 18)
  % - An estimate for \norm{ \vect{x} − \vect{x}# } often yields an estimate for
  %   \norm{ \vect{y} − \vect{y}# } = \norm{ A (\vect{x} − \vect{x}#) }, but the converse is not generally true. (p. 18)
  % book:Foucart2013, Chapter 1: An Invitation to Compressive Sensing / Sect. 1.1: What is Compressive Sensing? (Aug. 2013)
  % - THE RESEARCH AREA ASSOCIATED TO THIS PHENOMENON HAS BECOME KNOWN AS
  %   COMPRESSIVE SENSING, COMPRESSED SENSING, COMPRESSIVE SAMPLING, OR SPARSE RECOVERY. (p. 2)
  % article:KutyniokGAMM2013: Theory and applications of compressed sensing (Aug. 2013)
  % 1 Introduction
  % - When the previously mentioned two fundamental papers introducing compressed sensing were published
  %   [18; article:DonohoITIT2006 / 11; article:CandesITIT2006_1],
  %   THE TERM 'COMPRESSED SENSING' WAS INITIALLY UTILIZED FOR RANDOM SENSING MATRICES,
  %   SINCE THOSE ALLOW FOR A MINIMAL NUMBER OF NON-ADAPTIVE, LINEAR MEASUREMENTS. (p. 80)
  % - Nowadays, the terminology 'COMPRESSED SENSING' is
  %   MORE AND MORE OFTEN USED INTERCHANGEABLY WITH 'SPARSE RECOVERY' IN GENERAL, which
  %   is a viewpoint we will also take in this survey paper. (p. 80)
  The initial distinction between
  the terms
  \term{\acl{SR}} and
  \term{\acl{CS}} (\acs{CS}), which was based on
  the usage of either
  % 1.) deterministic sensing matrices
  deterministic or
  % 2.) random sensing matrices
  random sensing matrices and their
  theoretical guarantees, has been abandoned
  \cite[2]{book:Foucart2013},
  \cite{article:KutyniokGAMM2013}.
  % b) additional names include
  Additional names include
  % 1.) compressive sensing
  % article:BaraniukSPM2007: Compressive Sensing [Lecture Notes]
  \term{compressive sensing}
  \cite{article:BaraniukSPM2007} and
  % 2.) compressive sampling
  % article:CandesSPM2008: An Introduction To Compressive Sampling
  \term{compressive sampling}
  \cite{article:CandesSPM2008}.
} (\acs{CS})\acused{CS},
% 2.) data acquisition method
a data acquisition and
% 3.) recovery method
recovery method providing
% 4.) essential benefits
essential benefits in
% 5.) other imaging modalities [magnetic resonance imaging, x-ray computed tomography, photoacoustic tomography]
% article:BessonITUFFC2018: Ultrafast Ultrasound Imaging as an Inverse Problem: Matrix-Free Sparse Image Reconstruction
% I. INTRODUCTION
% - Indeed, in many medical imaging modalities,
%   the IMAGE RECONSTRUCTION PROCESS AMOUNTS TO SOLVING A LINEAR INVERSE PROBLEM. (p. 339)
% 	- In magnetic resonance imaging (MRI), the image is reconstructed from k-space samples and the measurement model is an inverse Fourier transform [6]. (p. 339)
%	- In X-ray-computed tomography (CT), the sinogram is related to the measurements by the Beer–Lambert law, which can be expressed as a linear inverse problem in the discrete domain [7]. (p. 339)
% - Moreover, sparsity priors have been expressed for medical images. (p. 339)
%	- Lustig et al. [6] have exploited sparsity of MRI images in the wavelet domain and under the total-variation (TV) transform. (p. 339)
% 	- In X-ray CT, sparsity priors under the TV-norm have been extensively used [8], [9]. (p. 339)
% article:ProvostITMI2009: The Application of Compressed Sensing for Photo-Acoustic Tomography
% article:ChenMedPhys2008: Prior image constrained compressed sensing (PICCS): A method to accurately reconstruct dynamic CT images from highly undersampled projection data sets
% [article:SongMedPhys2007,article:SidkyJXRST2006]
% article:LustigMRM2007: Sparse MRI: The application of compressed sensing for rapid MR imaging
other modalities
\cite{article:ProvostITMI2009,article:ChenMedPhys2008,article:LustigMRM2007}, to disrupt
% 6.) tradeoff
the tradeoff between
% 7.) image quality
the image quality and
% 8.) frame rate
the frame rate.
% f) cutting-edge variants of the numerical inverse scattering methods iteratively recovered a high-quality image from only a few sequential pulse-echo measurements
% article:BessonITUFFC2018: Ultrafast Ultrasound Imaging as an Inverse Problem: Matrix-Free Sparse Image Reconstruction
% Abstract
% - Using SIMULATED DATA and IN VIVO EXPERIMENTAL ACQUISITIONS, we show that
%   the proposed approach is THREE ORDERS OF MAGNITUDE FASTER than
%   non-DAS state-of-the-art methods, with comparable or better image quality. (p. 339)
They iteratively recover
% 1.) high-quality image
a high-quality image from
% 2.) single pulse-echo measurement
only a single pulse-echo measurement or
% 3.) less echo signals
less echo signals, if
% 4.) known dictionary of structural building blocks represents the image almost sparsely
(i) a known dictionary of
structural building blocks represents
the image almost sparsely, and
% 5.) individual pulse echoes are sufficiently uncorrelated
(ii) their individual pulse echoes, which are predicted by
% 6.) linear model for the prediction of the set of echo signals
the linear model, are
sufficiently uncorrelated.

% article:BessonITUFFC2018: Ultrafast Ultrasound Imaging as an Inverse Problem: Matrix-Free Sparse Image Reconstruction
% I. INTRODUCTION
% - SR methods have raised a growing interest recently in US imaging. (p. 340)
% - They [SR methods] have also been used in the context of the compressed sensing (CS) framework. (p. 340)
% - Three main contributions are proposed in this paper. (p. 340)
% 	- First, parametric, fast, and matrix-free formulations of the measurement model and its adjoint are described for both
%	  PLANE-WAVE (PW) and DIVERGING-WAVE (DW) COMPOUNDING. (p. 340)
%	- Second, a fully parallel implementation of these formulations is included in an SR framework, resulting in
%	  HIGH-QUALITY IMAGING WITH NEAR REAL-TIME CAPABILITY AND NO MEMORY FOOTPRINT, paving the way to SR for 3-D US imaging. (p. 340)
%	- Finally, the proposed model is coupled with INNOVATIVE COMPRESSION STRATEGIES, which outperform existing compressed beamforming (CB) schemes [10], [12]. (p. 340)
% II. PARAMETRIC MATRIX-FREE FORMULATIONS OF THE MEASUREMENT MODEL AND ITS ADJOINT / C. Inverse Problem of Interest
% - In the literature, two different groups of image reconstruction methods may be distinguished, since they aim at solving two different problems. (p. 341)
% - The first group of methods can be denoted as REGULARIZED BEAMFORMING METHODS. (p. 341)
%	- They do not take into account the pulse-echo waveform and
%         retrieve a low-resolution estimate of the TRF γ, usually denoted as the RF image γ^{RF} [11], [12], [23], [24], [32], [33]. (pp. 341, 342)
%	- One reason for this choice resides in the fact that γ^{RF} preserves speckle information (due to the low resolution), which is of interest in many US applications. (p. 342)
% - The second group of methods aims at performing INVERSE SCATTERING, i.e., at retrieving the high-resolution TRF γ from the element raw-data [2], [10], [15]. (p. 342)
%	- This problem is far more complex, highly ill-posed, and usually requires more sophisticated models than the ones used for regularized beamforming methods. (p. 342)
% V. RESULTS:SPARSE REGULARIZATION FOR ENHANCED IMAGE RECONSTRUCTION / C. Reconstruction of the In Vivo Carotids
% - This illustrates the great potential of SR for reducing the number of insonifications required to reach a given image quality. (p. 345)
% V. RESULTS:SPARSE REGULARIZATION FOR ENHANCED IMAGE RECONSTRUCTION / E. Computation Times for Sparse Regularization
% - These timings are two to three orders of magnitude faster than state-of-the-art methods. (p. 346)
% - Indeed, in their latest work, David et al. [46] reported computation times between 3 and 5 min. (pp. 346, 347)
% - Szasz et al. [33] reported reconstruction times between 10 s and 1 min. (p. 347)
% VI. RESULTS:COMPRESSED BEAMFORMING / D. Reconstruction of the In Vivo Carotids
% - However, speckle areas, especially in the far-field, are not well retrieved by the proposed approach, resulting in
%   a darkening of the deepest part of Fig. 9(b) and (d). (p. 348)
% VIII. CONCLUSION
% - This paper presents novel parametric, fast, and matrix-free formulations of the measurement model and its adjoint in the context of ultrafast US imaging. (p. 351)
% - These formulations are included in an SR framework, which is used to achieve high-quality imaging three orders of magnitude faster than existing methods. (p. 351)
% - In addition, new undersampling strategies, more suited to CS than previous ones, are suggested and high-quality reconstruction with a low number of measurements is demonstrated. (p. 351)
%
% [39] R. E. Carrillo, J. D. McEwen, D. Van De Ville, J.-P. Thiran, and Y. Wiaux, “Sparsity averaging for compressive imaging,” IEEE Signal Process. Lett., vol. 20, no. 6, pp. 591–594, Jun. 2013.
% [32] A. Besson et al., “A compressed beamforming framework for ultrafast ultrasound imaging,” in Proc. IEEE Int. Ultrason. Symp., Sep. 2016, pp. 1–4.

%---------------------------------------------------------------------------------------------------------------
% 4.) model inaccuracies and neglect of basic system abilities [GAPS]
%---------------------------------------------------------------------------------------------------------------
% a) linear models currently limit the convergence speed, the image quality, and the potential to meet condition (ii)
% article:BessonITUFFC2018: Ultrafast Ultrasound Imaging as an Inverse Problem: Matrix-Free Sparse Image Reconstruction
% I. INTRODUCTION
% - In US imaging, SEVERAL FORMULATIONS OF FORWARD MODELS have been recently investigated. (p. 339)
% 	- David et al. [10], Wang et al. [11], and Besson et al. [12] have proposed TIME-DOMAIN FORMULATIONS OF THE PROBLEM. (p. 339)
%         [10] article:DavidJASA2015, [11] proc:WangIUS2015, [12] proc:BessonICIP2016
%	- In the Fourier domain, Zhang et al. [13] have suggested a formulation of the model derived by David et al. [10]. (p. 339)
%	  [13] proc:ZhangIUS2015
%	- Besson et al. [14] have presented a forward model in the Fourier domain in which US propagation is seen as a projection on a nonuniform Fourier space. (p. 339)
%	  [14] article:BessonITUFFC2016
%	- Schiffner and Schmitz [15] have proposed a time–frequency model in which each frequency of the transducer-element bandwidth is treated independently,
%         enabling the model to deal with distortion effects. (pp. 339, 340)
The linear models, which are formulated in
% 1.) time domain
% article:OzkanITUFFC2018: Inverse Problem of Ultrasound Beamforming With Sparsity Constraints and Regularization
% article:BessonITUFFC2018: Ultrafast Ultrasound Imaging as an Inverse Problem: Matrix-Free Sparse Image Reconstruction
% article:DavidJASA2015: Time domain compressive beam forming of ultrasound signals
% article:ZhangUlt2013: A measurement-domain adaptive beamforming approach for ultrasound instrument based on distributed compressed sensing: Initial development
the time domain
\cite{article:OzkanITUFFC2018,article:BessonITUFFC2018,article:DavidJASA2015,article:ZhangUlt2013} or
% 2.) temporal Fourier domain
% proc:SchiffnerIUS2013a: Compensating the Combined Effects of Absorption and Dispersion in Plane Wave Pulse-Echo Ultrasound Imaging Using Sparse Recovery
% proc:SchiffnerIUS2013b: The Separate Recovery of Spatial Fluctuations in Compressibility and Mass Density in Plane Wave Pulse-Echo Ultrasound Imaging
% proc:SchiffnerIUS2012: Fast Image Acquisition in Pulse-Echo Ultrasound Imaging Using Compressed Sensing
% article:SchiffnerBMT2012: Compressed Sensing for Fast Image Acquisition in Pulse-Echo Ultrasound
% proc:SchiffnerIUS2011: Fast Pulse-Echo Ultrasound Imaging Employing Compressive Sensing
the temporal
\cite{proc:SchiffnerIUS2013a,proc:SchiffnerIUS2013b,proc:SchiffnerIUS2012,article:SchiffnerBMT2012,proc:SchiffnerIUS2011} and
% 3.) spatiotemporal Fourier domain
% article:BessonITUFFC2016: A Sparse Reconstruction Framework for Fourier-Based Plane-Wave Imaging
spatiotemporal
\cite{article:BessonITUFFC2016} \name{Fourier} domains, however, currently limit
% 4.) convergence speed
the convergence speed,
% 5.) image quality
the image quality, and
% 6.) potential
the potential to meet
% 7.) condition (ii) [ individual pulse echoes are sufficiently uncorrelated ]
condition (ii).
% b) linear models partly neglect diffraction, the combination of frequency-dependent absorption and dispersion, and the specifications of the instrumentation
% article:Schiffner2018, Sect. I: Introduction (sec:introduction)
% - Their [novel inverse scattering methods] UNIVERSALITY, which stems from
%   the arbitrary sophistication of this model including its optional calibration via experimental measurements, OPTIMALLY COMPLEMENTS
%   THE PROGRAMMABLE \ac{UI} SYSTEMS.
% article:Schiffner2018, Sect. I: Introduction (sec:introduction)
% - The ESTABLISHED IMAGE RECOVERY METHODS, which are explicit and computationally efficient, gradually trade
%   the image quality for the high frame rate \cite{article:GarciaITUFFC2013,article:MontaldoITUFFC2009,article:JensenUlt2006,article:ChengITUFFC2006}.
% - Their PHYSICAL MODELS NEGLECT VARIOUS EFFECTS, e.g.
%   the finite number of array elements and their anisotropic directivities, and
%   BASIC ABILITIES OF PROGRAMMABLE \ac{UI} SYSTEMS, e.g. the syntheses of complex incident waves.
Like
% 1.) linear model for the prediction of the set of echo signals
%those underlying
% 2.) established image recovery methods
the established methods,
% 3.) linear model for the prediction of the set of echo signals
they partly neglect
% 4.) diffraction
diffraction,
% 5.) combination of frequency-dependent absorption and dispersion
the combination of
frequency-dependent absorption and
dispersion, and
% 6.) specifications of the instrumentation
the specifications of
the instrumentation, including
% 7.) array geometry
the array geometry,
% 8.) acoustic lens
the acoustic lens, and
% 9.) electromechanical transfer behavior
% article:BessonITUFFC2018: Ultrafast Ultrasound Imaging as an Inverse Problem: Matrix-Free Sparse Image Reconstruction
% II. PARAMETRIC MATRIX-FREE FORMULATIONS OF THE MEASUREMENT MODEL AND ITS ADJOINT / C. Inverse Problem of Interest
% - While the model described in Section II can be used for inverse scattering problems,
%   WE FOCUS, in this paper, ON REGULARIZED BEAMFORMING, which means that
%   WE NEGLECT THE EFFECT OF THE PULSE-ECHO WAVEFORM v_{pe} in the model. (p. 342)
the electromechanical transfer behavior.
% c) linear models partly limit the number of spatial dimensions to only two, intermix results for the two- and three-dimensional Euclidean spaces, and replace the acoustic material parameters by an abstract reflectivity function
% article:BessonITUFFC2018: Ultrafast Ultrasound Imaging as an Inverse Problem: Matrix-Free Sparse Image Reconstruction
% - 
% article:BessonITUFFC2016: A Sparse Reconstruction Framework for Fourier-Based Plane-Wave Imaging
% article:DavidJASA2015: Time domain compressive beam forming of ultrasound signals
% IV. TIME DOMAIN 2D COMPRESSED BEAM FORMING
% - In this section, we introduce a CS approach to the 2D IMAGE FORMATION PARADIGM. (p. 2776)
% IV. TIME DOMAIN 2D COMPRESSED BEAM FORMING / A. 2D beam forming matrix / 1. Linear beam forming operator G
% - The scatterer is assumed to be smaller than \lambda, thus generating
%   a SPHERICAL WAVE which yields to the following backward impulse response that describes
%   the propagation from the scatterer in \vect{r} back to the ith transducer of the array
%   [ h_{bwd}^{i}( t, \vect{r} ) = \delta( t - \norm{ \vect{r} - \vect{r}_{i} }{2} / c ) / ( 2 \pi \norm{ \vect{r} - \vect{r}_{i} }{2} ) ], (17)
%   which is the Green’s function of the homogeneous medium [Fig. 5(b)]. (p. 2778)
% proc:SchiffnerIUS2013a: Compensating the Combined Effects of Absorption and Dispersion in Plane Wave Pulse-Echo Ultrasound Imaging Using Sparse Recovery
% proc:SchiffnerIUS2012: Fast Image Acquisition in Pulse-Echo Ultrasound Imaging Using Compressed Sensing
% article:SchiffnerBMT2012: Compressed Sensing for Fast Image Acquisition in Pulse-Echo Ultrasound
Moreover,
% 1.) limit the number of spatial dimensions to only two
they partly limit
the number of
spatial dimensions to
only two, intermix
% 2.) intermix results for the two- and three-dimensional Euclidean spaces
results for
the two- and
three-dimensional Euclidean spaces, and replace
% 3.) replace the acoustic material parameters by an abstract reflectivity function
% article:OzkanITUFFC2018: Inverse Problem of Ultrasound Beamforming With Sparsity Constraints and Regularization
% - 
% article:BessonITUFFC2018: Ultrafast Ultrasound Imaging as an Inverse Problem: Matrix-Free Sparse Image Reconstruction
% II. PARAMETRIC MATRIX-FREE FORMULATIONS OF THE MEASUREMENT MODEL AND ITS ADJOINT / A. Formulation of the Measurement Model
% - More precisely, let us consider a pulse-echo experiment, shown in Fig. 1, where
%   the propagation medium \Omega ∈ \R^{2} \ {z ≤ 0} [sic!] contains
%   INHOMOGENEITIES AS LOCAL FLUCTUATIONS IN ACOUSTIC VELOCITY AND/OR DENSITY, defining
%   a TISSUE REFLECTIVITY FUNCTION (TRF) γ( \vect{r} ) with r = [ x, z ]^{T} \in \Omega [15], [27]. (p. 340)
%   [15] proc:SchiffnerIUS2011
%   [27] O. V. Michailovich and D. Adam, “A novel approach to the 2-D blind deconvolution problem in medical ultrasound,” IEEE Trans. Med. Imag., vol. 24, no. 1, pp. 86–104, Jan. 2005.
% article:BessonITUFFC2016: A Sparse Reconstruction Framework for Fourier-Based Plane-Wave Imaging
% - 
% article:DavidJASA2015: Time domain compressive beam forming of ultrasound signals
% IV. TIME DOMAIN 2D COMPRESSED BEAM FORMING / A. 2D beam forming matrix / 1. Linear beam forming operator G
% - The reflected amplitude for each scatterer is given by the REFLECTIVITY I( \vect{r} ). (p. 2777)
% article:ZhangUlt2013: A measurement-domain adaptive beamforming approach for ultrasound instrument based on distributed compressed sensing: Initial development
% 3. The measurement-domain adaptive beamforming (MABF) based on DCS / 3.1. Ultrasound imaging model based on DCS
% - The time signal received by the ith element when there is only one point scatterer \vect{\rho} with
%   UNIT INTENSITY in the ROI is denoted by s_{i}( t, \vect{\rho} ). (p. 257)
% - So when the REFLECTION INTENSITY of \vect{\rho} is f( \vect{\rho} ), the backscattering data received by the ith element is
%   s_{i}( t, \vect{\rho} ) f( \vect{\rho} ). (p. 257)
% article:JensenProgBMB2007: Medical ultrasound imaging
% - The ANATOMY can be studied from GRAY-SCALE B-MODE IMAGES, where
%   the REFLECTIVITY and SCATTERING STRENGTH of the TISSUES are DISPLAYED.
% coll:Jensen2002: Ultrasound Imaging and Its Modeling
% ABSTRACT
% - The ANATOMY can be studied from GRAY-SCALE B-MODE IMAGES, where
%   the REFLECTIVITY and SCATTERING STRENGTH of the TISSUES are DISPLAYED. (p. 135)
the acoustic material parameters by
an abstract reflectivity function.
% d) exclusive usage of steered PWs or outgoing cylindrical waves ignores the syntheses of the incident waves by the UI system
% d) exclusive usage of steered PWs or outgoing cylindrical waves ignores basic abilities of programmable \ac{UI} systems
% article:BessonITUFFC2018: Ultrafast Ultrasound Imaging as an Inverse Problem: Matrix-Free Sparse Image Reconstruction
% I. INTRODUCTION
% - Firstly, parametric, fast and matrix-free formulations of the measurement model and its adjoint are described for both
%   PLANE-WAVE (PW) and DIVERGING-WAVE (DW) compounding. (p. 2)
% II. PARAMETRIC MATRIX-FREE FORMULATIONS OF THE MEASUREMENT MODEL AND ITS ADJOINT / A. Formulation of the Measurement Model
% - Ultrafast US imaging involves transmission of either steered PW (SPW) or DWs. (p. 340)
The exclusive usage of
% 1.) steered PWs
% article:OzkanITUFFC2018: Inverse Problem of Ultrasound Beamforming With Sparsity Constraints and Regularization
% Abstract
% - Our proposed method was evaluated for PLANE-WAVE IMAGING (PWI) that transmits unfocused waves, enabling
%   high frame rates with large field of view at the expense of
%   much lower image quality with conventional beamforming techniques. (p. 356)
% article:BessonITUFFC2016: A Sparse Reconstruction Framework for Fourier-Based Plane-Wave Imaging
% VIII. CONCLUSION
% - In this paper, a novel framework for Fourier-based reconstruction of signals obtained with
%   SEVERAL PW INSONIFICATIONS has been proposed. (p. 2103)
% article:DavidJASA2015: Time domain compressive beam forming of ultrasound signals
% Abstract
% - In this paper, it is shown that a valid CS framework can be derived from ultrasound propagation theory, and that
%   this framework can be used to compute images of scatterers USING ONLY ONE PLANE WAVE AS A TRANSMIT BEAM. (p. 2773)
% IV. TIME DOMAIN 2D COMPRESSED BEAM FORMING / A. 2D beam forming matrix / 1. Linearbeam forming operator
% - In the following development, we focus on a SINGLE PLANE WAVE EXCITATION of the medium. (p. 2777)
% VI. CONCLUSION
% - Through simulations and experimentations we showed that
%   an image of point scatterers can be recovered from
%   the INSONIFICATION OF A MEDIUM BY A SINGLE PLANE WAVE, when
%   in the case of conventional DAS, more than a hundred focalized excitation pulses would be necessary and,
%   in the case of plane wave compounding, more than ten excitation pulses would be required. (p. 2783)
% proc:SchiffnerIUS2013a: Compensating the Combined Effects of Absorption and Dispersion in Plane Wave Pulse-Echo Ultrasound Imaging Using Sparse Recovery
% proc:SchiffnerIUS2013b: The Separate Recovery of Spatial Fluctuations in Compressibility and Mass Density in Plane Wave Pulse-Echo Ultrasound Imaging
% article:ZhangUlt2013: A measurement-domain adaptive beamforming approach for ultrasound instrument based on distributed compressed sensing: Initial development
% 3. The measurement-domain adaptive beamforming (MABF) based on DCS / 3.1. Ultrasound imaging model based on DCS
% - We use a SINGLE UNFOCUSED TRANSMISSION, also referred to as a PLANE WAVE TRANSMISSION. (p. 257)
% proc:SchiffnerIUS2012: Fast Image Acquisition in Pulse-Echo Ultrasound Imaging Using Compressed Sensing
% article:SchiffnerBMT2012: Compressed Sensing for Fast Image Acquisition in Pulse-Echo Ultrasound
% proc:SchiffnerIUS2011: Fast Pulse-Echo Ultrasound Imaging Employing Compressive Sensing
steered \acp{PW}
\cite{article:OzkanITUFFC2018,article:BessonITUFFC2018,article:BessonITUFFC2016,article:DavidJASA2015,proc:SchiffnerIUS2013a,proc:SchiffnerIUS2013b,article:ZhangUlt2013,proc:SchiffnerIUS2012,article:SchiffnerBMT2012,proc:SchiffnerIUS2011} or
% 2.) outgoing 1-spherical waves (cylindrical waves)
% article:BessonITUFFC2018: Ultrafast Ultrasound Imaging as an Inverse Problem: Matrix-Free Sparse Image Reconstruction
outgoing cylindrical waves
\cite{article:BessonITUFFC2018} ignores
% 3.) basic abilities
basic abilities of
% 4.) freely programmable UI systems
programmable \ac{UI} systems.
% e) UI system typically specifies pulse shapes, apodization weights, and time delays for the waves emitted by the individual array elements and provides hundreds of degrees of freedom to synthesize alternative types of incident waves
% article:IlovitshNatComBio2018: Acoustical structured illumination for super-resolution ultrasound imaging
% - Advances in ultrasound technologies have now led to user-programmable systems, capable of
%   [1.)] a NEARLY INFINITE VARIETY OF TRANSMITTED PULSE TRAINS, and
%   [2.)] schemes for image reconstruction. (p. 2)
The latter typically specifies
% 1.) pulse shapes
pulse shapes,
% 2.) apodization weights
apodization weights, and
% 3.) time delays
time delays for
% 4.) waves emitted by the individual array elements
the waves emitted by
the individual array elements and provides
% 5.) hundreds
hundreds of
% 6.) degrees of freedom
degrees of
freedom to synthesize
% 7.) alternative types of incident waves
alternative types of
incident waves.
% f) nonconvex recovery methods require less data
% article:FoucartACHA2009: Sparsest solutions of underdetermined linear systems via lq-minimization for 0 < q \leq 1
% 2. Exact recovery via lq-minimization
% - Corollary 2.2.
%     Under the assumption that γ_{ 2s + 2 } < ∞,
%     every s-sparse vector is exactly recovered by solving (P_{q}) for some q > 0 small enough. (p. 397)
% proc:ChartrandICASSP2008: Iteratively reweighted algorithms for compressive sensing
% ABSTRACT
% - In [1], it was shown EMPIRICALLY that using lp minimization with p < 1 can do so [exact sparse signal recovery] with
%   FEWER MEASUREMENTS than with p = 1. (p. 3869)
%   [1] article:ChartrandISPL2007
% 1. INTRODUCTION
% - A THEOREM was also proven in terms of the RESTRICTED ISOMETRY CONSTANTS of Φ,
%   generalizing a result of [17] to show that
%   a CONDITION SUFFICIENT FOR (2) [idealized lp-minimization] TO RECOVER x EXACTLY IS WEAKER FOR SMALLER p. (p. 3869)
% - Thus, the DEPENDENCE OF THE SUFFICIENT NUMBER OF MEASUREMENTS M on the signal size N DECREASES as p → 0. (pp. 3869, 3870)
% article:ChartrandISPL2007: Exact Reconstruction of Sparse Signals via Nonconvex Minimization
% Abstract
% - We show that by replacing the l1 norm with the lp norm with p < 1,
%   EXACT RECONSTRUCTION IS POSSIBLE WITH SUBSTANTIALLY FEWER MEASUREMENTS. (p. 707)
% - We give a THEOREM IN THIS DIRECTION [exact recovery w/ fewer measurements], and [...]. (p. 707)
% I. INTRODUCTION
% - It is natural to ask what happens if the l1 NORM IS REPLACED BY THE lp NORM FOR SOME p \in ( 0; 1 ). (p. 707)
% - We begin with THEORETICAL RESULTS concerning when
%   a GLOBAL MINIMIZER IS GUARANTEED TO BE AN EXACT RECONSTRUCTION. (p. 707)
% - For a given \mat{\Phi}, the restricted isometry condition (5) will hold for larger values of K [sparsity] when p < 1. (p. 708)
% IV. CONCLUSIONS
% - In this letter, we have seen that
%   BY COMPUTING LOCAL lp MINIMIZERS WITH p < 1,
%   FEWER MEASUREMENTS ARE REQUIRED THAN PREVIOUSLY OBSERVED. (p. 710)
Although
% 1.) nonconvex recovery methods
nonconvex recovery methods require
% 2.) less data
less data
\cite{article:FoucartACHA2009,proc:ChartrandICASSP2008,article:ChartrandISPL2007},
% g) convex variants [sparsity-promoting recovery method] are prevalent
% article:BessonITUFFC2018: Ultrafast Ultrasound Imaging as an Inverse Problem: Matrix-Free Sparse Image Reconstruction
% III. SPARSITY-DRIVEN IMAGE RECONSTRUCTION METHODS / A. Quick Tour on Sparse Regularization and Compressed Sensing / 1) Sparse Regularization
% - Problem (18) can also be written as
%   [ \underset{ \vect{s} }{ \min } \norm{ \mat{\Psi}^{\dagger} \vect{s} }{1} subject to \norm{ \vect{y} - \mat{\Phi} \vect{s} }{2} \leq \epsilon ] (19)
%   where \epsilon is a upper bound of the l2-norm of the noise. (p. 342)
% III. SPARSITY-DRIVEN IMAGE RECONSTRUCTION METHODS / B. Sparse Regularization for Enhanced Ultrasound Image Reconstruction
% - The underlying idea consists of the introduction of a sparsity prior on RF images in a well-chosen model, described hereafter,
%   in order to recover a high-quality image by solving Problem (19). (p. 342)
% article:BessonITUFFC2016: A Sparse Reconstruction Framework for Fourier-Based Plane-Wave Imaging
% IV. SPARSE-BASED BEAMFORMING / B. Proposed Sparse-Based Beamforming Method / 4) l1-Minimization Algorithm
% - The proposed imaging method is based on solving the convex problem
%   [ \underset{ \bar{\vect{s}} \in \C^{N} }{ \norm{ \mat{\Psi}^{+} \bar{\vect{s}} }{1} } subject to \norm{ \vect{y} - \mat{\Phi} \bar{\vect{s}} }{2} \leq \epsilon ] (16)
%   where \mat{\Psi}^{+} denotes the adjoint operator of \Psi and \Phi is the NUFT operator. (p. 2096)
% article:DavidJASA2015: Time domain compressive beam forming of ultrasound signals
% proc:SchiffnerIUS2013a: Compensating the Combined Effects of Absorption and Dispersion in Plane Wave Pulse-Echo Ultrasound Imaging Using Sparse Recovery
% proc:SchiffnerIUS2012: Fast Image Acquisition in Pulse-Echo Ultrasound Imaging Using Compressed Sensing
% article:SchiffnerBMT2012: Compressed Sensing for Fast Image Acquisition in Pulse-Echo Ultrasound
% proc:SchiffnerIUS2011: Fast Pulse-Echo Ultrasound Imaging Employing Compressive Sensing
convex variants are
prevalent
\cite{article:BessonITUFFC2018,article:BessonITUFFC2016,article:DavidJASA2015,proc:SchiffnerIUS2013a,proc:SchiffnerIUS2012,article:SchiffnerBMT2012,proc:SchiffnerIUS2011}.
% Besides \acl{MAP} estimates \cite{article:ZhangUlt2013},
% article:ZhangUlt2013: A measurement-domain adaptive beamforming approach for ultrasound instrument based on distributed compressed sensing: Initial development
% 3. The measurement-domain adaptive beamforming (MABF) based on DCS / 3.2. The measurement-domain adaptive beamformer (MABF)
% - The maximum likelihood estimate of f is given by [21] (14). (p. 259)
% - The (14) becomes the MAP optimization problem (15). (p. 259)
% - Therefore, to find the MAP estimate of target vector f, we must maximize the priori PDF p(f) subject to the constraint that y = V_{DCS} f. (p. 259)

%---------------------------------------------------------------------------------------------------------------
% 5.) randomization of the apodization weights and the time delays / random incident waves
%---------------------------------------------------------------------------------------------------------------
% a) randomization of the apodization weights and the time delays in the syntheses of the incident waves potentially improves the conformity with condition (ii)
% article:Ghanbarzadeh-DagheyanSensors2018: Holey-Cavity-Based Compressive Sensing for Ultrasound Imaging
% 1. Introduction
% - When the sensing matrix is built through RANDOM PROJECTIONS in the undersampled signal space,
%   the RESULTING COHERENCE BETWEEN EACH OF TWO OF ITS COLUMNS IS SMALL, with a high probability. (p. 1)
% - This is the reason why
%   MEASUREMENT RANDOMIZATION HAS THE POTENTIAL TO ENHANCE THE ACCURACY OF THE RECOVERED UNKNOWN SIGNAL [2]. (p. 1)
The randomization of
% 1.) apodization weights
the apodization weights and
% 2.) time delays
the time delays in
% 3.) syntheses of the incident waves
the syntheses of
the incident waves, which is motivated by
% 4.) essential theorems underlying CS
the essential theorems underlying
\ac{CS}, potentially improves
% 5.) conformity
the conformity with
% 6.) condition (ii) [ individual pulse echoes are sufficiently uncorrelated ]
condition (ii).
% b) LTI measurement process physically superimposes the weighted and delayed echo signals induced by the individual array elements
In fact,
% 1.) LTI measurement process
the \ac{LTI} measurement process physically superimposes
% 2.) weighted and delayed echo signals
the weighted and
delayed echo signals induced by
% 3.) individual array elements
the individual array elements.
% c) each incident wave randomly projects a complete SA acquisition sequence on a single pulse-echo measurement and strongly compresses both the acquisition time and the data volume
% article:KruizingaSciAdv2017: Compressive 3D ultrasound imaging using a single sensor
% INTRODUCTION
% - Our work follows the concept central to compressive imaging: that of
%   PROJECTING THE OBJECT/IMAGE INFORMATION THROUGH A SET OF INCOHERENT FUNCTIONS ONTO A SINGLE MEASUREMENT. (p. 2)
% DISCUSSION
% - Our work differs in the fact that rather than focusing our ultrasound waves to a single point,
%   WE ENCODE THE WHOLE 3D VOLUME ONTO ONE SPATIOTEMPORAL VARYING ULTRASOUND FIELD, from which
%   we then reconstruct an image based on solving a linear signal model. (p. 8)
% - This has the advantage of requiring only a few measurements to image an entire 3D volume, in contrast to time-reversal work. (p. 8)
% - This difference in acquisition time is crucial for medical imaging where the object changes over time, for example, in a beating heart. (p. 8)
Each incident wave randomly projects
% 1.) complete SA acquisition sequence
a complete \ac{SA} acquisition sequence on
% 2.) single pulse-echo measurement
a single pulse-echo measurement and, thus, strongly compresses both
% 3.) acquisition time
the acquisition time and
% 4.) data volume
the data volume.
% d) few studies of such waves in compressed UI and their origins will now be reviewed
The few studies of
% 1.) random waves
such waves in
% 2.) compressed UI
compressed \ac{UI} and
% 3.) origins
their origins will now be reviewed.

%%%%%%%%%%%%%%%%%%%%%%%%%%%%%%%%%%%%%%%%%%%%%%%%%%%%%%%%%%%%%%%%%%%%%%%%%%%%%%%%%%%%%%%%%%%%%%%%%%%%%%%%%%%%%%%%
% 1.) related work
%%%%%%%%%%%%%%%%%%%%%%%%%%%%%%%%%%%%%%%%%%%%%%%%%%%%%%%%%%%%%%%%%%%%%%%%%%%%%%%%%%%%%%%%%%%%%%%%%%%%%%%%%%%%%%%%
\subsection{Related Work}
\label{subsec:intro_related_work}
%---------------------------------------------------------------------------------------------------------------
% 1.) random waves in computational microwave imaging
%---------------------------------------------------------------------------------------------------------------
% a) syntheses of random waves in UI traces back to computational microwave imaging
% article:Ghanbarzadeh-DagheyanSensors2018: Holey-Cavity-Based Compressive Sensing for Ultrasound Imaging
% 1. Introduction
% - SEVERAL SENSING AND IMAGING APPLICATIONS [4] have been able to take advantage of CS by using
%   [1.)] PSEUDO-RANDOM ILLUMINATION IN THE OUTGOING WAVES from the transmitters and collecting
%   [2.)] PSEUDO-RANDOM MEASUREMENTS FROM THE INCOMING WAVES to the receivers. (pp. 1, 2)
% - Carin et al. used random positions for the elements of a sensing array to make use of CS. (p. 2)
% - They also showed that PLACING SPHERICAL SCATTERING OBJECTS IN FRONT OF THE WAVES increases
%   the RANDOMNESS AND INCOHERENCE of the measurements [5]. (p. 2)
%   [5] Carin, L.; Liu, D.; Guo, B. Coherence, compressive sensing, and random sensor arrays. IEEE Antennas Propag. Mag. 2011, 53, 28–39.
The syntheses of
% 1.) random waves
random waves in
% 2.) ultrasound imaging (UI)
\ac{UI} partly trace back to
% 3.) time-reversal methods
% article:MontaldoITUFFC2005: Building three-dimensional images using a time-reversal chaotic cavity
time-reversal
\cite{article:MontaldoITUFFC2005} and
% 4.) computational microwave imaging
% article:FromentezeOptExp2017: Computational polarimetric microwave imaging (leaky reverberant cavity)
% article:GollubSciRep2017: Large Metasurface Aperture for Millimeter Wave Computational Imaging at the Human-Scale (complex metamaterial)
% article:FromentezeApplPhysLett2015: Computational imaging using a mode-mixing cavity at microwave frequencies (leaky reverberant cavity)
% article:LipworthJOSAA2013: Metamaterial apertures for coherent computational imaging on the physical layer (complex metamaterial)
% article:HuntScience2013: Metamaterial Apertures for Computational Imaging (complex metamaterial)
computational microwave imaging
\cite{article:FromentezeOptExp2017,article:GollubSciRep2017,article:FromentezeApplPhysLett2015,article:LipworthJOSAA2013,article:HuntScience2013}.
% b) computational microwave imaging trades the hardware complexity and costs for the computational complexity of the image recovery
% article:FromentezeOptExp2017: Computational polarimetric microwave imaging (microwave imaging)
% 1. Introduction
% - The FREQUENCY-DIVERSE APERTURE LIMITS
%   THE COMPLEXITY OF THE HARDWARE ARCHITECTURE required for real-time high-resolution imaging, obviating
%   the ACTIVELY CONTROLLED COMPONENTS or
%   the NEED FOR MECHANICAL MOTION that is typically required in conventional systems. (p. 3)
% article:ChewITAP1997: Fast solution methods in electromagnetics
% I. Introduction
% - However,
%   [1.)] the RECENT PHENOMENAL GROWTH IN COMPUTER TECHNOLOGY, coupled with
%   [2.)] the DEVELOPMENT OF FAST ALGORITHMS with reduced computational complexity and memory requirements,
%   have made a RIGOROUS NUMERICAL SOLUTION OF THE PROBLEM OF SCATTERING FROM ELECTRICALLY LARGE OBJECTS FEASIBLE. (p. 533)
The latter trades
% 1.) hardware complexity
the hardware complexity and
% 2.) hardware costs
costs, which are raised by
% 3.) fully-sampled transceiver arrays
transceiver arrays or
% 4.) mechanical scanning
mechanical scans, for
% 5.) computational costs
the computational costs of
% 6.) image recovery
the image recovery.
% c) highly dispersive customized apertures convert excitations at different frequencies into spatially diverse and distinct emitted fields
% article:Ghanbarzadeh-DagheyanSensors2018: Holey-Cavity-Based Compressive Sensing for Ultrasound Imaging
% 1. Introduction
% - Incoherence between each of two measurements can also be achieved by using
%   PHYSICAL STRUCTURES that exhibit DIFFERENT WAVE-MATTER RESPONSES AT DIFFERENT INSTANTANEOUS FREQUENCIES, without
%   the need to CHANGE THE ARRANGEMENT OF THE SENSING ARRAY. (p. 2)
Highly dispersive customized apertures, e.g.
% 1.) complex metamaterials / metasurfaces
% article:Ghanbarzadeh-DagheyanSensors2018: Holey-Cavity-Based Compressive Sensing for Ultrasound Imaging
% 1. Introduction
% - METAMATERIALS HAVE ALSO BEEN USED TO CREATE RANDOMNESS IN THE SENSING SYSTEM for applications such as
%   MICROWAVE IMAGING [8–10], OPTICAL IMAGING [11,12], MILLILITER-WAVE [sic!] IMAGING [13–16] and
%   ACOUSTIC MULTICHANNEL SEPARATION using a single sensor [17]. (p. 2)
% - Similar to CODED MASKS that are commonly used in compressive OPTICAL IMAGING [22–24] as
%   a subgroup of hardware-based methods [...]. (p. 2)
%
%   references from: article:Ghanbarzadeh-DagheyanSensors2018, article:FromentezeOptExp2017
%   microwave imaging:
%   (x)  [8], [25] article:HuntScience2013: Metamaterial Apertures for Computational Imaging [RANDOM IN TX/RX; K-band (18 to 26 gigahertz)]
%		   - The imaging system we present here combines a computational imaging approach with custom aperture hardware that allows compression to be performed on
%		     the PHYSICAL LAYER that is used to do the ILLUMINATION and/or RECORDING.
%	 [9]	   article:HuntJOSAA2014: Metamaterial microwave holographic imaging system
%   (x) [10], [26] article:LipworthJOSAA2013: Metamaterial apertures for coherent computational imaging on the physical layer [RANDOM IN TX/RX; K-band (18 to 26 gigahertz)]
%		   - Furthermore, by randomly distributing the resonance frequencies of its elements, we demonstrated the metaimager can
%		     illuminate a scene with random field patterns well suited for compressive sensing. (p. 1611)
%	[27]	   article:YurdusevenOptExp2016: Frequency-diverse microwave imaging using planar mills-cross cavity apertures
%	[28]	   article:MarksJOSAA2016: Spatially resolving antenna arrays using frequency diversity
%   (x) [29]	   article:GollubSciRep2017: Large Metasurface Aperture for Millimeter Wave Computational Imaging at the Human-Scale
%	[30]	   article:YurdusevenOptExp2017: Millimeter-wave spotlight imager using dynamic holographic metasurface antennas
%
%   optical imaging:
%	[11]	  article:WattsNatPhot2014: Terahertz compressive imaging with metamaterial spatial light modulators
%	[12]	  article:LiutkusSciRep2014: Imaging With Nature: Compressive Imaging Using a Multiply Scattering Medium [RANDOM in RX]
%     coded masks: (metamaterial?)
%	[22]	  Rawat, N.; Hwang, I.C.; Shi, Y.; Lee, B.G. Optical image encryption via photon-counting imaging and compressive sensing based ptychography. J. Opt. 2015, 17, 065704. [CrossRef]
%	[23]	  Spencer, A.P.; Spokoyny, B.; Ray, S.; Sarvari, F.; Harel, E. Mapping multidimensional electronic structure and ultrafast dynamics with single-element detection and compressive sensing. Nat. Commun. 2016, 7, 10434. [CrossRef] [PubMed]
%	[24]	  Ma, R.; Hu, F.; Hao, Q. Active Compressive Sensing via Pyroelectric Infrared Sensor for Human Situation Recognition. IEEE Trans. Syst. Man Cybern. Syst. 2017, 47, 3340–3350. [CrossRef]
%
%   millimeter-wave imaging:
%	[13]	  coll:MolaeiAABF2017: Compressive Reflector Antenna Phased Array
%	[14]	  proc:MolaeiISAP2017: A bilayer ELC metamaterial for multi-resonant spectral coding at mm-Wave frequencies [July 2017]
%	[15]	  proc:MolaeiISAP2016: Active imaging using a metamaterial-based compressive reflector antenna
%	[16]	  article:MolaeiISTHS2017: A 2-bit and 3-bit metamaterial absorber-based compressive reflector antenna for high sensing capacity imaging
%
%   acoustic multichannel separation:
%		  article:XieSciRep2016: Acoustic Holographic Rendering with Two-dimensional Metamaterial-based Passive Phased Array
%	[17]	  article:XiePNAS2015: Single-sensor multispeaker listening with acoustic metamaterials (RANDOM in RX)
%
% article:GollubSciRep2017: Large Metasurface Aperture for Millimeter Wave Computational Imaging at the Human-Scale (microwave imaging)
% Abstract
% - 
% article:LipworthJOSAA2013: Metamaterial apertures for coherent computational imaging on the physical layer (microwave imaging)
% Abstract
% - We introduce the concept of a METAMATERIAL APERTURE, in which
%   an underlying reference mode interacts with a designed metamaterial surface to produce
%   A SERIES OF COMPLEX FIELD PATTERNS. (p. 1603)
% - As the frequency of operation is scanned, different subsets of metamaterial elements become active, in turn varying the field patterns at the scene. (p. 1603)
% - Scene information can thus be indexed by frequency, with the overall effectiveness of the imaging scheme tied to the diversity of the generated field patterns. (p. 1603)
% - In this work we provide the foundation for computational imaging with metamaterial apertures based on FREQUENCY DIVERSITY, and establish that
%   for resonators with physically relevant Q-factors, there are potentially ENOUGH DISTINCT MEASUREMENTS OF A TYPICAL SCENE WITHIN A REASONABLE BANDWIDTH to achieve
%   diffraction-limited reconstructions of physical scenes. (p. 1603)
% 1. INTRODUCTION
% - Imaging systems based on incoherent light typically either populate the image plane with an array of fixed detectors that acquire information in parallel, or mechanically scan a smaller number of detec- tors that acquire scene information serially. (p. 1603)
% - In both cases, a system of optics - often quite complex - is typically used to transmit the scene to the aperture in as pristine a condition as possible. (p. 1603)
% - In fact, since all natural scenes are known to be compressible in some basis, natural scenes can be perfectly recovered with significantly fewer measurement modes than
%   the SBP (N << M)[2,3] provided that a set of optimal measurement modes is utilized. (p. 1603)
% - This statement forms the basis for COMPRESSIVE COMPUTATIONAL IMAGING. (p. 1603)
% - The reality that the N measurement modes are of consequence and not necessarily the methods for forming or detecting them suggests that
%   we may seek NEW COMPUTATIONAL IMAGING MODALITIES for coherent light that can potentially provide SIMILAR FUNCTIONALITY [as phased arrays], but with
%   REDUCED COST AND COMPLEXITY. (p. 1604)
% - Artificial materials exhibit TWO KEY ADVANTAGES:
%   [1.)] first, they offer access to DESIGNED ELECTROMAGNETIC PROPERTIES that may be difficult or impossible to find in NATURALLY OCCURRING MEDIA. (p. 1604)
%   [2.)] A second, potentially more revolutionary, advantage is that artificial materials present the potential for DYNAMIC TUNING [8],
%         which could enable PHASED-ARRAY-LEVEL CONTROL OVER MEASUREMENT MODES in a package with
%	  the LOW COST AND SIMPLICITY OF HOLOGRAPHIC APERTURES. (p. 1604)
% - One particular implementation of HOLOGRAPHIC IMAGING USING METAMATERIALS AT MICROWAVE FREQUENCIES, which serves as the subject of the present analysis, is that of
%   a GUIDED-MODE METAMATERIAL IMAGER (henceforth METAIMAGER) that radiates via coupling a guided wave to a set of
%   RESONANT, METAMATERIAL ELEMENTS DISTRIBUTED ALONG THE PROPAGATION PATH [9]. (p. 1604)
% - As presented, the metaimager is a single-pixel device that performs SEQUENTIAL MEASUREMENTS OF A SCENE using
%   A FREQUENCY-BASED ENCODING OF THE MEASUREMENT MODES. (p. 1604)
% - Because the number of measurement modes is equal to the number of distinct patterns that can be generated over a given frequency bandwidth,
%   THE APERTURE DESIGN STRATEGY IS TO MAXIMIZE FREQUENCY DIVERSITY. (p. 1604)
% - Thus, the metaimager is populated with metamaterial elements whose
%   RESONANCE FREQUENCIES ARE DISTRIBUTED RANDOMLY OVER A GIVEN BANDWIDTH, each with as large a quality (Q-) factor as possible. (p. 1604)
% - The resulting aperture produces a sequence of illumination patterns that
%   VARY RAPIDLY AS A FUNCTION OF FREQUENCY and are well suited for compressive imaging of canonically sparse scenes. (p. 1604)
% - The ADVANTAGE OF IMAGING USING FREQUENCY DIVERSITY is that
%   A SERIES OF MEASUREMENT MODES CAN BE OBTAINED USING A SINGLE FREQUENCY SCAN, avoiding
%   MECHANICAL SCANNING, MULTIPLE DETECTORS, or even RECONFIGURABLE ELEMENTS. (p. 1604)
% - This frequency scanned metamaterial imager provides an important proof of concept that
%   more advanced and novel imaging modalities can be achieved in coherent imaging schemes through the use of
%   COMPLEX, DESIGNED APERTURES. (p. 1604)
% 4. METAMATERIAL APERTURE
% - The use of resonance not only grants us access to the exotic electromagnetic responses metamaterials are known for, but also allows us to use
%   frequency as a convenient parameter by which to index our measurement modes. (p. 1607)
% - The details of the actual complementary metamaterial elements are unimportant to the present discussion; as was stated in the introduction,
%   we model each complementary metamaterial element as a radiating dipole [20,37]. (p. 1607)
% - It is now evident why frequency can serve as a parameter by which to index the measurement modes. (p. 1607)
% - From Eq. (25) we note that by sweeping ω the polarizability of each dipole changes; in addition, the local field U_{GW} at each dipole can change with ω as well. (p. 1607)
% - From Eq. (24) we see that both polarizability and U_{GW} affect the dipole moments, and these in turn can modify the field pattern with which the array illuminates the scene, calculated according to Eq. (26). (p. 1607)
% - The difference in radiation patterns highlights how unique sets of measurement modes can be accessed using only frequency diversity. (p. 1608)
% 8. CONCLUSIONS
% - We have introduced a COMPUTATIONAL IMAGING FRAMEWORK appropriate to a variety of SINGLE-PIXEL COHERENT IMAGERS, and applied it to
%   a specific aperture implementation we termed the metaimager — a 2D guided-wave aperture radiating via an array of complementary metamaterial elements. (p. 1611)
% - We have modeled each element as a RADIATING DIPOLE and showed
%   HOW THEIR DISPERSION ALLOWS THE METAIMAGER TO CONTROL ITS FIELD PATTERNS THROUGH FREQUENCY DIVERSITY. (p. 1611)
% - Furthermore, by RANDOMLY DISTRIBUTING THE RESONANCE FREQUENCIES of its elements, we demonstrated
%   THE METAIMAGER CAN ILLUMINATE A SCENE WITH RANDOM FIELD PATTERNS well suited for compressive sensing. (p. 1611)
% - Lastly, we have presented simulations of 2D and 3D scene reconstructions demonstrating the imaging capabilities of the proposed metaimager. (p. 1611)
% article:HuntScience2013: Metamaterial Apertures for Computational Imaging (microwave imaging)
% Abstract
% - By leveraging metamaterials and compressive imaging, a low-profile aperture capable of microwave imaging without
%   lenses, moving parts, or phase shifters is demonstrated. (p. 1)
% - This designer aperture allows image compression to be performed on the physical hardware layer rather than in the postprocessing stage, thus averting
%   the detector, storage, and transmission costs associated with full diffraction-limited sampling of a scene. (p. 1)
% - A guided-wave metamaterial aperture is used to perform compressive image reconstruction at 10 frames per second of two-dimensional (range and angle) sparse still and
%   video scenes at K-band (18 to 26 gigahertz) frequencies, using frequency diversity to avoid mechanical scanning. (p. 1)
% - Image acquisition is accomplished with a 40:1 compression ratio. (p. 1)
complex metamaterials
\cite{article:GollubSciRep2017,article:LipworthJOSAA2013,article:HuntScience2013} or
% 2.) leaky reverberant cavities
% article:Ghanbarzadeh-DagheyanSensors2018: Holey-Cavity-Based Compressive Sensing for Ultrasound Imaging
% Abstract
% - The use of SOLID CAVITIES AROUND ELECTROMAGNETIC SOURCES has been recently reported as
%   a MECHANISM TO PROVIDE ENHANCED IMAGES AT MICROWAVE FREQUENCIES. (p. 1)
% - These cavities are used as MEASUREMENT RANDOMIZERS; and they COMPRESS THE WAVE FIELDS AT THE PHYSICAL LAYER. (p. 1)
% - As a result of this compression,
%   THE AMOUNT OF INFORMATION COLLECTED BY THE SENSING ARRAY through the different excited modes inside the resonant cavity is increased when
%   compared to that obtained by no-cavity approaches. (p. 1)
% 1. Introduction
% - For instance, Fromenteze et al. used
%   a METAL CAVITY WITH A NUMBER OF HOLES and a wave agitator inside the cavity to RANDOMIZE
%   the ELECTROMAGNETIC WAVE PATTERNS FOR MICROWAVE IMAGING [6]. (p. 2)
% - Later, the same research group fabricated
%   a METALIZED CAVITY WITH HOLES arranged in irises following Fibonacci patterns to CODE
%   the OUTGOING WAVES BASED ON FREQUENCY DIVERSITY [7]. (p. 2)
% - The FABRICATION OF HOLEY CAVITIES IS GENERALLY SIMPLER THAN THAT OF METAMATERIALS, and
%   they do not require any alteration in the sensing array assortment in contrast to the approach adopted in [5]. (p. 2)
% - Other types of ultrasound cavities have also been proposed by Fink et al. to create images using time-reversal techniques [26–29].
%
%   references from: article:Ghanbarzadeh-DagheyanSensors2018, article:FromentezeOptExp2017
%   microwave imaging:
%   (x)		   article:FromentezeOptExp2017: Computational polarimetric microwave imaging
%   (x)  [6], [23] article:FromentezeApplPhysLett2015: Computational imaging using a mode-mixing cavity at microwave frequencies
%	 [7]	   article:YurdusevenIMWCLett2016: Printed aperiodic cavity for computational and microwave imaging
%	      [22] article:CarsenatIAWPLett2012: Uwb antennas beamforming using passive time-reversal device
%	      [24] article:FromentezeIACCESS2016: Single-shot compressive multiple-inputs multiple-outputs radar imaging using a two-port passive device
%
% article:FromentezeOptExp2017: Computational polarimetric microwave imaging (microwave imaging)
% 1. Introduction
% - The potential for efficient, cost-effective, and high-resolution systems that can achieve fast acquisition rates have recently been demonstrated in
%   computational imaging systems based on
%   [1.)] CAVITY-BACKED [22–24] and
%   [2.)] METASURFACE [25–30] apertures. (p. 3)
% 5. Conclusion
% - A POLARIMETRIC MICROWAVE IMAGING TECHNIQUE has been presented in this paper by extending
%   the computational principle of the scalar approaches that have been previously developed in the literature. (p. 17)
% - The LEAKY MULTI-MODAL CAVITY USED TO PRODUCE PSEUDO-ORTHOGONAL FIELD PATTERNS IN FREQUENCY AND POLARIZATION can be generalized to
%   many metasurface aperture paradigms — all of which are capable of generating
%   FIELD PATTERNS WITH LOW CORRELATION AS A FUNCTION OF FREQUENCY or other parameters. (pp. 17, 18)
% - In the present example, measurements were taken between two ports of a multi-modal cavity over the band 17.5 - 26.5 GHz. [K-band] (p. 18)
% - From the FREQUENCY-INDEX MEASUREMENTS, it was possible to reconstruct targets made of copper wires forming the word "DUKE",
%   demonstrating the different modes of operation permitted by the proposed polarimetric approach. (p. 18)
% article:FromentezeApplPhysLett2015: Computational imaging using a mode-mixing cavity at microwave frequencies (microwave imaging)
leaky reverberant cavities
\cite{article:FromentezeOptExp2017,article:FromentezeApplPhysLett2015}, form
% 3.) virtual transceiver arrays
% article:KruizingaSciAdv2017: Compressive 3D ultrasound imaging using a single sensor
% RESULTS / Ultrasound wave field diversity using a coded aperture
% - By MODELING THE APERTURE MASK AS A COLLECTION OF POINT SENSORS, each having different transmit/receive delays,
%   the MASK CAN STILL BE REGARDED AS A SENSOR ARRAY. (p. 3)
virtual transceiver arrays.
% c) highly dispersive customized apertures form virtual transceiver arrays that expand excitations at sufficiently different frequencies into distinct spatial codes and mix the scattered waves in reception
% article:Ghanbarzadeh-DagheyanSensors2018: Holey-Cavity-Based Compressive Sensing for Ultrasound Imaging
% 1. Introduction
% - More specifically, the natural reverberation of the wave propagation inside the medium emulates random illuminations of a scene at different frequencies.
% - These systems exploit the conversion between spatial and spectral degrees of freedom, which are
%   also at the core of the time reversal technique in complex media for spatio-temporal focusing with a single broadband antenna. [29,30]
% article:KruizingaSciAdv2017: Compressive 3D ultrasound imaging using a single sensor
% RESULTS / Ultrasound wave field diversity using a coded aperture
% - However, all these point sensor signals are SUBSEQUENTLY SUMMED BY THE PIEZO SENSOR just after they have passed though the mask, resulting in
%   a SINGLE COMPRESSED MEASUREMENT, as depicted in the right panel of Fig. 1
%   (for a further comparison between a sensor array and a single sensor with a coding mask, see text S2 and figs. S1 and S2). (p. 3)
% article:FromentezeOptExp2017: Computational polarimetric microwave imaging (microwave imaging)
% 1. Introduction
% - These systems RADIATE PSEUDO-ORTHOGONAL FIELD DISTRIBUTIONS IN TRANSMISSION and - by exploitation of the reciprocity principle - IN RECEPTION,
%   to MULTIPLEX INFORMATION AND RECONSTRUCT AN IMAGE. (p. 3)
These expand
% 1.) excitations
excitations at
% 2.) sufficiently different frequencies
sufficiently different frequencies into
% 3.) distinct spatial codes
distinct spatial codes, i.e.
% 4.) spatially erratic incident fields
% TODO: complex erratic spatiotemporal interference patterns
spatially erratic incident fields, and, reciprocally, mix
% 5.) scattered fields
the scattered fields in
% 6.) reception
reception, providing
% 7.) frequency-diverse projections
frequency-diverse projections of
% 8.) field-of-view (FOV)
the \ac{FOV}.
% c)
The knowledge of
% 1.) two-way transfer function
the two-way transfer function enables
% 2.) image recovery
the image recovery.
% d)
% article:TondoYoyaApplPhysLett2017: Computational passive imaging of thermal sources with a leaky chaotic cavity
% article: Single-sensor multispeaker listening with acoustic metamaterials
% article:LiutkusSciRep2014: Imaging With Nature: Compressive Imaging Using a Multiply Scattering Medium
% - In the past few years, SEVERAL HARDWARE IMPLEMENTATIONS capable of performing such random compressive sampling were introduced [5–13]. (p. 1)
% - In optics, these include the SINGLE PIXEL CAMERA [6], which is depicted in Fig. 1(b), and
%   uses a digital array of micromirrors (abbreviated DMD) to sequentially reflect different random portions of the object onto a single photodetector. (p. 1)
% - Other approaches include phase modulation with a SPATIAL LIGHT MODULATOR [10] or a ROTATING OPTICAL DIFFUSER [13]. (p. 1)
% - The idea of RANDOM MULTIPLEXING for imaging has also been considered in other domains of wave propagation. (p. 1)
% - CS holds much promise in areas where detectors are rather complicated and expensive such as the THz or far infrared. (p. 1)
% - In this regards, there have been proposals to implement CS imaging procedures in the THz using random pre-fabricated masks [5],
%   DMD or SLM photo-generated contrast masks on semi-conductors slabs [14] and efforts are also pursued on TUNABLE METAMATERIAL REFLECTORS [15]. (p. 1)
% - Recently, a carefully engineered metamaterial aperture was used to generate complex RF beams at different frequencies [8]. (p. 1)
% - However, these CS implementations come with some limitations. (p. 1)
% - First, these devices include carefully engineered hardware designed to achieve randomization, via a DMD [6], a metamaterial [8] or a coded aperture [11].
% - Second, the acquisition time of most implementations can be large because they require the sequential generation of a large number of random patterns. (p. 1)
% Using natural complex media as random sensing devices
% - In essence, the COMPLEX MEDIUM ACTS AS A HIGHLY EFFICIENT ANALOG MULTIPLEXER FOR LIGHT, with
%   an input-output response characterized by its transmission-matrix [24,25]. (p. 3)
% single pixel camera
% article:RogersJAP2017: Demonstration of acoustic source localization in air using single pixel compressive imaging
% randomness in receive mode: (random multiplexing)
%
% optical imaging:
% 1.) article:DuarteISPM2008: Single-pixel imaging via compressive sampling (x)
%
% terahertz imaging:
% 1.) article:LiutkusSciRep2014: Imaging With Nature: Compressive Imaging Using a Multiply Scattering Medium (x, type of wave?)
% 2.) article:ShrekenhamerOptExp2013: Terahertz single pixel imaging with an optically controlled dynamic spatial light modulator
% 3.) article:WillettOptEng2011: Compressed sensing for practical optical imaging systems: A tutorial (type of wave?)
% 4.) article:ChanApplPhysLett2008: A single-pixel terahertz imaging system based on compressed sensing (x)
%
% microwave imaging
% 1.) 14. L. Wang, L. Li, Y. Li, H. C. Zhang, T. J. Cui, Single-shot and single-sensor high/ super-resolution microwave imaging based on metasurface. Sci. Rep. 6, 26959 (2016).
%
% thermal imaging w/ leaky reverberant cavity:
% 1.) article:TondoYoyaApplPhysLett2017: Computational passive imaging of thermal sources with a leaky chaotic cavity
% - In this article, we demonstrate passive imaging of thermal sources in the X-band frequency regime using a leaky chaotic cavity attached to two ports. (p. 2)
%
% acoustic position detection:
% 1.) article:RogersJAP2017: Demonstration of acoustic source localization in air using single pixel compressive imaging
% - Despite the range of approaches and results for electromagnetic imaging, VERY FEW STUDIES HAVE ATTEMPTED SINGLE-PIXEL IMAGING IN THE ACOUSTIC REGIME. (p. 2)
% 16. N Huynh, E. Zhang, M. Betcke, S. R. Arridge, P. Beard, B. Cox, A real-time ultrasonic field mapping system using a Fabry Pérot single pixel camera for 3D photoacoustic imaging, in Proceedings of SPIE Volume 9323, (International Society for Optics and Photonics, 2015), pp. 93231O.
%
% R. F. Marcia, Z. T. Harmany, and R. M. Willett, “Compressive coded aperture imaging,” Proc. SPIE 7246, 72460G (2009).
% [29] J. Ke and M. Neifeld, “Optical architectures for compresssive imaging,” Appl. Opt., vol. 46, pp. 5293–5303, Aug. 2007.
Unlike
% 1.) random sampling
% article:BessonITUFFC2018: Ultrafast Ultrasound Imaging as an Inverse Problem: Matrix-Free Sparse Image Reconstruction
% VI. RESULTS:COMPRESSED BEAMFORMING / A. Deep Dive Into Coherence
% - Four different sampling strategies are compared:
%   [1.)] uniform selection of transducer elements,
%   [2.)] random selection of transducer elements,
%   [3.)] CMIX, and
%   [4.)] CTMIX. (p. 347)
% - Regarding CMIX and CTMIX, the matrix W is generated with coefficients distributed according to normal and Rademacher distributions. (p. 347)
% - Fig. 6(a) and (b) shows the mutual coherence μ( H_{d} \Psi ) for
%   a number of measurements ranging between 5% and 100%, where H_{d} is square in the case of 100 %, with
%   \Psi being Dirac and Haar bases, respectively. (p. 347)
% - It can be seen that the main benefit of the CMIX and CTMIX strategies reside in their ability to limit
%   the increase of the coherence μ( H_{d} \Psi ) induced by the undersampling of the raw data. (p. 347)
% - In addition, it can be noticed that this effect is more pronounced for the Haar basis than for the Dirac basis. (p. 347)
% - A comparison between CMIX and CTMIX shows that CTMIX has lower coherence than CMIX. (p. 347)
% - This is expected, since CTMIX provides a better mixing than CMIX. (p. 347)
% - Regarding the impact of the probability distribution of the random coefficients on μ( H_{d} \Psi ), Fig. 6(c) shows that
%   there is no significant difference in coherence between Gaussian and Rademacher random coefficients. (p. 347)
% - Regarding CTMIX, Fig. 6(d) shows the values of μ( H_{d} \Psi ) for the different values of D_{t}. (p. 347)
% - It can be seen that the mutual coherence decreases when D_{t} increases. (p. 347)
% - Fig. 6. MUTUAL COHERENCE μ( H_{d} \Psi ) AGAINST THE NUMBER OF MEASUREMENTS for
%   (a) Dirac basis and (b) Haar basis for
%   the uniform selection of transducer elements, the random selection of transducer elements, CMIX, and CTMIX (D_{t} = 10).
%   In addition, the coherence is evaluated for
%   (c) CMIX and CTMIX with mixing coefficients drawn using normal and Rademacher distributions and
%   (d) CTMIX at different depths. (p. 347)
% VI. RESULTS:COMPRESSED BEAMFORMING / B. Reconstruction of the Point-Reflector Phantom
% - It can be concluded that CMIX and CTMIX achieve high-quality reconstructions for all the considered number of measurements. (p. 348)
% - It can also be observed that, in the case of CMIX, the PSNR drops for numbers of measurements lower than 5 %, whereas
%   it remains constant for CTMIX at lower numbers of measurements. (p. 348)
% VI. RESULTS:COMPRESSED BEAMFORMING / E. Computation Times for Compressed Beamforming
% - In addition, the CB method coupled with CMIX and CTMIX is no longer matrix-free, since the random matrices have to be stored in memory. (p. 349)
% VII. DISCUSSION / B. Toward Compressed Sensing in Ultrasound Imaging
% - However, we face ONE MAJOR OBSTACLE, which is the HIGH COHERENCE OF THE MEASUREMENT OPERATOR. (p. 350)
% - CMIX and CTMIX strategies manage to maintain the coherence constant when the number of measurements is decreased, but
%   the COHERENCE REMAINS HIGH RELATIVE TO THE CORRESPONDING WELCH BOUND because of
%   THE COHERENCE INTRINSIC TO THE MEASUREMENT MODEL. (p. 350)
% - Indeed, the high mutual coherence of the measurement model comes from the fact that
%   each projection onto a 1-D-conic, implied by the time-of-flight calculations, involves
%   only a few points in the desired-image space. (p. 350)
% - The natural question that one may ask is whether it is possible to change the nature of the projections in order to involve more points. (p. 350)
% - In fact, this is a relatively difficult task,
%   since projections are a consequence of the expression for the round-trip time-of-flight of US waves in a homogeneous medium. (p. 350)
% - This observation has a deep impact on the design of CS acquisition schemes. (p. 350)
% - Indeed, it means that the attempts to decrease the coherence of the measurement model by playing with
%   random pulses sent in transmit are hopeless, because this does not change the expression for
%   the time-of-flight and thus the fact that echo-samples stem from projections onto 1-D-conics. (p. 350)
% 	- One solution to this issue may reside in dealing with the coherent operator, by exploring
%   	  a recent topic on CS, denoted “constrained adaptive sensing,” which derives sampling theorems and variants of [35, Th. 3.1] where
%   	  the measurement matrix is more constrained than in standard CS [50]. (p. 350)
% 	- Another alternative could be to entirely rethink the measurement process, starting from the requirements of [35, Th. 3.1]. (p. 350)
%	- In terms of acoustic propagation, a random measurement model implies that each sample of
%	  the element raw-data receives contributions from points spread over the entire image space. (p. 350)
%	- In other words, it means that the duality between time and depth, which is at the heart of US imaging, is not valid anymore, since echoes from points at different positions reach the transducer elements at the same time. (p. 350)
%	- Thus, a random measurement model H is unfeasible, in pulse-echo imaging in a homogeneous medium, due to the fact that US waves respect the Helmholtz equation. (p. 350)
%	- One way to address such an issue resides in placing a scattering or heterogeneous medium in front of the US probe. (p. 350)
%	- This principle has been recently developed in optics and gives promising results [51]. (p. 350)
%	- However, such a new design raises many questions regarding the choice and the modeling of the heterogeneous medium. (p. 350)
%	- It can be easily understood that such a medium may not generate a purely random matrix but will be somewhere inbetween
%	  a purely random case and the highly coherent case of the homogeneous medium. (p. 350)
%	- This topic is currently under study and will be the object of further reports or communications. (p. 350)
% proc:BessonICIP2016: Compressed delay-and-sum beamforming for ultrafast ultrasound imaging
the random sampling
\cite{article:BessonITUFFC2018,proc:BessonICIP2016,article:DavidJASA2015} or
% 3.) mixing
mixing
\cite{article:BessonITUFFC2018,article:LiutkusSciRep2014} of
% 4.) scattered waves
the scattered waves, which reduce
the number of
transceivers and measurements,
% 5.)
the random waves improve
%
the conformity with
condition (ii).

% d)
% phased arrays?
% reduce the number of sequential measurements
%Fully-controlled phased arrays, in contrast, increase
%the flexibility at
%higher complexity and costs.


% c) fully-controlled phased arrays require complex systems but are more flexible and reduce the number of sequential measurements
% article:Ghanbarzadeh-DagheyanSensors2018: Holey-Cavity-Based Compressive Sensing for Ultrasound Imaging
% 1. Introduction
% - Specifically for ultrasound imaging,
%   Schiffner introduced a SOFTWARE-BASED TECHNIQUE that uses time delays and apodization weights to generate random incident acoustic fields [18,19]. (p. 2)
% article:ChengIACCESS2017: Near-Field Millimeter-Wave Phased Array Imaging With Compressive Sensing
% ABSTRACT
% - Phased array technology allows for FAST ELECTRONIC BEAM STEERING with high antenna gain for radar imaging systems. (p. 18975)
% - A novel data acquisition methodology has been proposed on the basis of NEAR-FIELD FOCUSING TECHNIQUES. (p. 18975)
% - CS measurements are taken by RANDOMLY FOCUSING BEAMS IN THE NEAR-FIELD REGION. (p. 18975)
% I. INTRODUCTION
% - By integrating the CS theory into the imaging algorithm, ONLY A SMALL NUMBER OF RANDOM SAMPLES ARE NEEDED FOR RECONSTRUCTION. (p. 18976)
% - Our goal is to recover the reflectivity information of the target region from the reflected data. (p. 18976)
% II. NEAR-FIELD PHASED ARRAY IMAGING WITH COMPRESSIVE SENSING / B. PROPOSED NEAR-FIELD IMAGING METHOD
% - Alternatively, we adopt the near-field focusing technique and make sure the sampling points are evenly distributed in the ROI. (p. 18977)
% - In comparison to the far-field focusing method, this scheme can focus at different depths in the ROI and thus offers much more information in 3-D imaging applications. (p. 18977)
% II. NEAR-FIELD PHASED ARRAY IMAGING WITH COMPRESSIVE SENSING / C. COMPRESSIVE SENSING IMPLEMENTATION
% - With CS theory, samplings in the spatial domain and the frequency domain can be greatly reduced while satisfactory reconstruction can still be achieved. (p. 18978)
% - Mathematically, we use matrix A as the undersampling operator. (p. 18978)
% IV. CONCLUSION
% - A new scanning method has also been provided to focus array beams at different spots with various depths such that they can cover the target region with less redundancy. (p. 18984)
% - With the CS theory, far fewer spatial samples are required during data acquisition than FT based methods. (p. 18984)
% article:LipworthJOSAA2013: Metamaterial apertures for coherent computational imaging on the physical layer (microwave imaging)
% 1. INTRODUCTION
% - A PHASED-ARRAY SYSTEM FULFILLS THE DESCRIPTION OF A RECONFIGURABLE APERTURE and can, in principle, provide
%   a LIMITLESS SET OF MEASUREMENT MODES. (p. 1604)
% - However, the drawbacks that have inhibited phased-array prevalence in applications are
%   its significant COST, WEIGHT, AND POWER REQUIREMENTS for implementing the required number of SOURCES, PHASE SHIFTERS, AND ASSOCIATED AMPLIFIERS CIRCUITRY to generate
%   N measurement modes. (p. 1604)
% - [27] A. Massa, P. Rocca, and G. Oliveri, ‘‘Compressive sensing in electromagnetics—A review,’’ IEEE Antennas Propag. Mag., vol. 57, no. 1, pp. 224–238, Feb. 2015.
% article:ChengIACCESS2016: Compressive Millimeter-Wave Phased Array Imaging
% 13. Molaei, A.; Juesas, J.H.; Lorenzo, J.A.M. Compressive Reflector Antenna Phased Array. In Antenna Arrays and Beam-Formation; InTech: London, UK, 2017.

% TODO: acquisition time?

% 1.5-D Sparse Array for Millimeter-Wave Imaging Based on Compressive Sensing Techniques 10.1109/TAP.2018.2800531
% Coded aperture ptychography: uniqueness and reconstruction (Pengwen Chen1 and Albert Fannjiang2,3 Published 10 January 2018 • 2018 Inverse Problems, Volume 34, Number 2)

%   [2] D. J. Brady, K. Choi, D. L. Marks, R. Horisaki, and S. Lim, “Compressive holography,” Opt. Express 17, 13040–13049 (2009).
%   [3] C. F. Cull, D. A. Wikner, J. N. Mait, M. Mattheiss, and D. J. Brady, “Millimeter-wave compressive holography,” Appl. Opt. 49, E67–E82 (2010).

% Subwavelength diffractive acoustics and wavefront manipulation with a reflective acoustic metasurface, https://doi.org/10.1063/1.4967738

% Single-frequency 3D synthetic aperture imaging with dynamic metasurface antennas https://doi.org/10.1364/AO.57.004123 (x)
% Low-cost three-dimensional millimeter-wave holographic imaging system based on a frequency-scanning antenna https://doi.org/10.1364/AO.57.000A65 ->

% W-band sparse synthetic aperture for computational imaging https://doi.org/10.1364/OE.24.008317
% Far-field imaging beyond diffraction limit using single sensor in combination with a resonant aperture https://doi.org/10.1364/OE.23.000401 (x)
% Single-frequency microwave imaging with dynamic metasurface apertures
% X-band SAR imaging with a liquid-crystal-based dynamic metasurface antenna https://doi.org/10.1364/JOSAB.34.000300
% Design considerations for a dynamic metamaterial aperture for computational imaging at microwave frequencies https://doi.org/10.1364/JOSAB.33.001098
% Cavity-backed metasurface antennas and their application to frequency diversity imaging https://doi.org/10.1364/JOSAA.34.000472

%---------------------------------------------------------------------------------------------------------------
% 2.) Kruizinga et al.
%---------------------------------------------------------------------------------------------------------------
% a) Kruizinga et al. equipped a single transducer with a plastic delay mask to enable compressive three-dimensional UI
% article:Ghanbarzadeh-DagheyanSensors2018: Holey-Cavity-Based Compressive Sensing for Ultrasound Imaging
% 1. Introduction
% - Similar to CODED MASKS that are commonly used in COMPRESSIVE OPTICAL IMAGING [22–24] as a subgroup of hardware-based methods,
%   Kruizinga et al. introduced a ROTATING MASK OF RANDOMLY-VARYING THICKNESSES throughout its surface to randomize ultrasound waves, and
%   they were able to retrieve 3D images of objects using a single transducer [25]. (p. 2)
% article:KruizingaSciAdv2017: Compressive 3D ultrasound imaging using a single sensor
% Abstract
% - In this spirit, we have designed a simple ultrasound imaging device that can perform
%   THREE-DIMENSIONAL IMAGING USING JUST A SINGLE ULTRASOUND SENSOR. (p. 1)
% - Our device makes a COMPRESSED MEASUREMENT OF THE SPATIAL ULTRASOUND FIELD using
%   A PLASTIC APERTURE MASK PLACED IN FRONT OF THE ULTRASOUND SENSOR. (p. 1)
% - The need for just one sensor instead of thousands paves the way for
%   CHEAPER, FASTER, SIMPLER, AND SMALLER SENSING DEVICES and possible new clinical applications. (p. 1)
% INTRODUCTION
% - A SUCCESSFUL STRATEGY for obtaining these compressed measurements is to apply a so-called “CODED APERTURE” (17–19). (p. 2)
% - In this case, the INFORMATION — whether it entails different light directions, different frequency bands, or
%   any other information that is conventionally needed to build up an image — IS CODED INTO
%   THE AVAILABLE APERTURE AND ASSOCIATED MEASUREMENT. (p. 2)
% - Smart algorithms are then needed to decode the retrieved measurements to form an image. (p. 2)
% - Along these lines, here we introduce for the first time in ultrasound imaging
%   a SIMPLE COMPRESSIVE IMAGING DEVICE that can create 3D images. (p. 2)
% - Our device contains ONE LARGE PIEZO SENSOR that transmits an ultrasonic wave through A SIMPLE PLASTIC CODING MASK. (p. 2)
% - Using our device, WE AIM TO MITIGATE THE HARDWARE COMPLEXITY associated with conventional 3D ultrasound. (p. 2)
% - The manufacturing costs of our device will be much lower. (p. 2)
% - A simple, cheap, single-element transducer is used, and the plastic coding mask can be produced for less than a euro. (p. 2)
% - These lower costs enable broader use of these 3D compressive imaging devices, for example, for long-term patient monitoring. (p. 2)
% - Because the imaging is 3D, finding and maintaining the proper 2D view does not require a trained operator. (p. 2)
% - One could also envision other applications, such as minimally invasive imaging catheters that are too thin to accommodate
%   the hundreds or thousands of electrical wires currently needed for 3D ultrasound imaging. (p. 2)
% DISCUSSION
% - By contrast [2D LOCALIZATION WORK by Clement et al], we propose to induce signal diversity using
%   LOCAL DELAYS IN BOTH TRANSMIT AND RECEIVE. (p. 8)
% - This type of “wave field coding” can be applied to many other imaging techniques and seems to offer
%   a much higher dynamic range than the slowly varying frequency-dependent wave fields. (p. 8)
% - As a result, we are able to move beyond the localization of isolated point scatterers but perform actual 3D imaging as we show in this paper. (p. 8)
% - We believe that this technique will pave the way for an entirely new means of imaging in which the complexity is shifted away from
%   the hardware side and toward the power of computing. (p. 8)
% proc:VanDerMeulenACSSC2017: Spatial compression in ultrasound imaging
\name{Kruizinga} \emph{et al.} \cite{article:KruizingaSciAdv2017} equipped
a single transducer with
a plastic delay mask to enable
compressive three-dimensional \ac{UI} with
cheap and simple hardware.
% b) mask introduced random time delays into both the emitted and the received waves to decorrelate the pulse echoes received from distinct voxels
% article:KruizingaSciAdv2017: Compressive 3D ultrasound imaging using a single sensor
% Abstract
% - The APERTURE MASK ensures that every pixel in the image is UNIQUELY IDENTIFIABLE in the compressed measurement. (p. 1)
% - We demonstrate that this device can successfully image TWO STRUCTURED OBJECTS placed in water. (p. 1)
% INTRODUCTION
% - Local variations in the mask thickness (Fig. 1C) cause local delays, which scrambles the phase of the wave field. (p. 2)
% - This enables a complex interference pattern to propagate inside the volume,
%   removing ambiguity among echoes from different pixels, as illustrated in Fig. 2. (p. 2)
% - The interference pattern
%   propagates through the medium,
%   scatters from objects within the medium, and then
%   propagates back through the coding mask onto the same ultrasound sensor, providing
%   a single compressed ultrasound measurement of the object. (p. 2)
% RESULTS / Ultrasound wave field diversity using a coded aperture
% - After propagating through the mask, the pulse length increases further due to the distortion of the wavefront by the coding mask (Fig. 4A). (p. 3)
% - These waves will propagate spherically in the medium and will interfere constructively and destructively,
%   a process that stretches out longer in time than an undistorted wavefront does. (p. 3)
% - Consequently, the available spatial bandwidth increases compared to an undistorted burst by a mask-less sensor. (p. 3)
% - This effect is highlighted in Fig. 4B, which shows two spatial frequency spectra: one at the beginning of the pulse and one spectrum a few microseconds later. (p. 3)
% - The delays produced by the mask create complex spatiotemporal interference patterns that ensure that each pixel generates a unique temporal signal in the compressed measurement. (pp. 3, 4)
% - This unique pixel signature enables direct imaging without the need for uncompressed spatial measurements. (p. 4)
% - Because of the limitations in temporal bandwidth, as well as in mask thickness distribution, it is not guaranteed that all pixel echo signals are uncorrelated. (p. 4)
% - To introduce more diversity between the pixels, we chose to rotate the mask in front of the sensor such that the interference pattern is rotated along with it, thereby obtaining additional measurements containing new information. (p. 4)
% RESULTS / Pixel diversity
% - If we model the ultrasound field of the coded aperture by APPROXIMATING THE MASK APERTURE AS A COLLECTION OF POINT SOURCES AND SENSORS
%   (see Materials and Methods section), we can analyze the coding performance of the aperture. (p. 4)
% - To this end, we compute the pulse-echo signals for pixels in the central (xy plane) area at a fixed depth (z) relative to the sensor and
%   then compute their CROSS-CORRELATIONS. (p. 4)
% - Ideally, all signals are uncorrelated, so that any superposition of pulse-echo signals can be uniquely broken down into
%   its composing individual pulse-echo signals (as in Fig. 2), in which case the signals for each pixel can be unambiguously resolved. (p. 4)
% - Because the wave field phase without using a mask is highly uniform, most pulse-echo signals are highly correlated. (p. 4)
% - However, adding a mask to the single sensor causes the distribution to be zero-mean and removes the high correlations around +1 and −1. (p. 4)
% - As expected, these results illustrate that adding more measurements by rotating the mask causes correlations to be distributed more narrowly around zero;
%   pulse-echo signals become more orthogonal. (p. 4)
% - This tells us that the use of rotation removes ambiguities and consequently increases resolvability of scatterers. (p. 4)
% - Here, we do not include the signal correlations over the depth dimension, because generally speaking, the decorrelation in
%   the acoustic propagation direction comes naturally with the pulse-echo delay, or in other words, using only one sensor,
%   we can estimate the depth ofan object, but we cannot see wheth- er the object is located left or right from the sensor. (p. 4)
% - The CRLB is derived for the case of estimating the position of a single scatterer given a pulse-echo measurement with additive Gaussian noise
%   (see the Materials and Methods section for more details). (p. 5)
% - For a 3D problem, this results in estimating three unknowns (x, y,and z coordinates) from N measurements, where N is the total number of samples. (p. 5)
% - Although position estimation and imaging are not the same, one could argue that the best value obtained for positional error spread
%   (for example, 3 SD of error) nevertheless is indicative for the achievable imaging resolution. (p. 5)
% DISCUSSION
% - The plastic mask breaks the phase uniformity of the ultrasound field and causes every pixel to be uniquely identifiable within the received signal. (p. 8)
% - The compressed measurement contains the superimposed signals from all object pixels. (p. 8)
The varying thickness of
the mask introduced random time delays into both
the emitted and
the received waves to decorrelate
the pulse echoes received from
distinct voxels.
% c) sequential emission of these random waves at numerous angles of rotation permitted the recovery of three-dimensional sparse objects
% article:KruizingaSciAdv2017: Compressive 3D ultrasound imaging using a single sensor
% RESULTS / Ultrasound wave field diversity using a coded aperture
% - Here, we assume that most of the acoustic energy is directly transmitted through the coding mask;
%   yet, in reality, some of the energy will be reflected back from the mask-water interface and possibly redirected in
%   the medium through the mask in a second pass after bouncing back from the mask-transducer interface, thereby increasing the spatial variability even more. (p. 3)
% RESULTS / Constructing the system matrix H
% - For computing the column signals of H, it is crucial to KNOW THE ENTIRE SPATIOTEMPORAL WAVE FIELD. (p. 8)
% - To this end, we adopt a SIMPLE CALIBRATION MEASUREMENT in which we SPATIALLY MAP THE IMPULSE SIGNAL
%   (using a small hydrophone and a translation stage) in a plane close to the mask surface and perpendicular to the ultrasound propagation axis (Fig. 6C). (p. 8)
% - This recorded wave field can then be propagated to any plane that is parallel to the recorded plane using the angular spectrum approach (28, 29). (p. 8)
% - Note that THIS PROCEDURE IS REQUIRED ONLY ONCE AND IS UNIQUE FOR THE SPECIFIC SENSOR AND CODING MASK. (p. 8)
% - According to the reciprocity theorem, we are able to obtain the pulse-echo signal by performing an autoconvolution of the propagated hydrophone signal with itself (30). (p. 8)
% - Using these components, we can then compute the pulse-echo ultrasound signal for every point in 3D space and subsequently populate the columns of H. (p. 8)
% DISCUSSION
% - Our device has disadvantages and limitations and does not deliver similar functionality as existing 3D ultrasound arrays. (p. 8)
% - In addition, the mechanical rotation of the mask is expected to introduce a fair amount of error in the prediction of the ultrasound field and limits
%   the usability when it comes down to imaging moving tissue or the application of ultrasound Doppler. (p. 8)
% - Besides, a rotating wave field becomes less effective at pixels closer to the center of rotation— assuming a spatially uniform generation of spatial frequencies. (p. 8)
% - Future systems should consider the incorporation of other techniques to introduce signal variation, such as
%   controlled linearized motion, or a mask that can be controlled electronically, similar to
%   the digital mirror devices used in optical compressive imaging (13). (p. 8)
% - Furthermore, the angular spectrum approach method that predicts the ultrasound field inside the medium is not flawless in cases where
%   the medium contains strong variations in sound speed and density. (p. 8)
% - These issues may possibly be solved, however, by incorporating more advanced ultrasound field prediction tools that
%   can deal with these kinds of variations (38). (p. 8)
% - This will increase the computational burden of the image reconstruction even further but may potentially lead to better results than what we have shown here. (p. 8)
%
% - Another important distinction is the SMALL IMPEDANCE DIFFERENCE BETWEEN OUR CODING MASK AND IMAGING MEDIUM (less than a factor of 2). (p. 8)
% - As a consequence, there is very little loss of acoustic energy as compared to reverberant cavities that inherently require
%   high impedance mismatch to ensure signal diversity (33, 34). (p. 8)
% MATERIALS AND METHODS / Acoustic calibration procedure
% - This calibration step is an essential component in this kind of compressive imaging (40). (p. 9)
% - The better H is known, the better the reconstruction of the image. (p. 9)
% - Note that this calibration is sensor- and mask- specific and, in principle, only needs to be done once. (p. 9)
Adopting
a simple calibration procedure, which measured
the random sound field in
water and, thus, limited
its range of
validity,
% article:KruizingaSciAdv2017: Compressive 3D ultrasound imaging using a single sensor
% RESULTS / Signal model and image reconstruction
% - For the full 3D reconstruction shown in Fig. 6F, we made use of the sparsity of
%   these two letters in water by applying the sparsity-promoting basis pursuit denoising (BPDN) algorithm (27). (p. 6)
% - As can be seen, this prior knowledge about the image could be effectively exploited to improve image quality,
%   significantly improving the dynamic range from 15 to 40 dB. (p. 6)
% - Fig. 6. Compressive 3D ultrasound imaging using a single sensor. [...]
%   The images shown in (B) and (D) were obtained using 72 EVENLY SPACED MASK ROTATIONS, and
%   the full 3D image in (F) was obtained using only 50 EVENLY SPACED ROTATIONS to reduce the total matrix size. (p. 7)
% article:KruizingaSciAdv2017: Compressive 3D ultrasound imaging using a single sensor
% RESULTS / Signal model and image reconstruction
% - Thus, instead of applying a standard geometric operation to multiple sensor observations,
%   we attempt to explain the received signal as a linear combination of point scatterer echo signals [23, 24]. (p. 5)
the recovery of
sparse objects required
$50$ sequential pulse-echo measurements at
evenly spaced angles of
rotation.
%---------------------------------------------------------------------------------------------------------------
% 2.) Ghanbarzadeh-Dagheyan et al.
%---------------------------------------------------------------------------------------------------------------
% a) Ghanbarzadeh-Dagheyan et al. optimized a holey cavity with respect to its opening sizes and the materials to enable compressive two-dimensional UI with only a few transceivers
% article:Ghanbarzadeh-DagheyanSensors2018: Holey-Cavity-Based Compressive Sensing for Ultrasound Imaging
% Abstract
% - In this work,
%   A TWO-DIMENSIONAL CAVITY, having MULTIPLE OPENINGS, is used to perform such a COMPRESSION FOR ULTRASOUND IMAGING. (p. 1)
% - Moreover,
%   COMPRESSIVE SENSING TECHNIQUES are used for SPARSE SIGNAL RETRIEVAL with
%   A LIMITED NUMBER OF OPERATING TRANSCEIVERS. (p. 1)
% - In addition, an analysis of
%   the SENSING CAPACITY and
%   the SHAPE OF THE POINT SPREAD FUNCTION is also carried out for the aforementioned cases. (p. 1)
% - The cavity is designed to have
%   the MAXIMUM SENSING CAPACITY GIVEN DIFFERENT MATERIALS AND OPENING SIZES. (p. 1)
% 1. Introduction
% - In this study, a STATIC 2D CAVITY has been selected as a structure that enables
%   RANDOMIZATION OF THE WAVE FIELDS IN THREE GENERALIZED DIMENSIONS,
%   one spectral and two spatial, thus leading to enhanced ultrasound imaging via compressive sensing. (p. 2)
% - In the succeeding sections,
%   the performance of several 2D holey cavities are studied, showing how the cavity-based imaging performance is enhanced when compared to that of
%   a traditional ultrasound imaging setup. (p. 2)
% 2. Two-Dimensional Cavity
% - As observed, the CAVITY IS CLOSED FROM ALL SIDES EXCEPT THE BOTTOM, where a number of openings is made for the impinging waves to pass through. (p. 2)
% - It is assumed that the OPENINGS ARE UNIFORMLY DISTRIBUTED ALONG THE BOTTOM OF THE CAVITY, and they are symmetric with respect to the y axis. (p. 2)
% 3. Compressive Sensing, Imaging and Performance Metrics
% - Moreover, the SENSING CAPACITY and the POINT SPREAD FUNCTION (PSF) of the system are defined, and
%   they will be used as extra metrics to assess how the ADDITION OF THE CAVITY AFFECTS THE PERFORMANCE OF THE IMAGING SYSTEM. (p. 3)
% 3.5. Sensing Capacity
% - Another metric that will be used to assess the performance of the imaging system is the so-called SENSING CAPACITY. (p. 7)
% - This metric determines the amount of information that can be transferred from the imaging domain into the sensors; and
%   the larger the SENSING CAPACITY is, the better the image reconstruction will be. (p. 7)
% 4. Simulation Results and Discussion / 4.1. The Effect of the Cavity Design on Sensing Capacity
% - In this study, MAXIMIZING THE SENSING CAPACITY was selected as the design goal, and
%   TWO PARAMETERS OF THE CAVITY WERE ADJUSTED to achieve this end:
%   [1.)] the OPENING SIZE and
%   [2.)] the MATERIAL OF THE CAVITY. (p. 8)
% 4. Simulation Results and Discussion / 4.1. The Effect of the Cavity Design on Sensing Capacity / 4.1.1. The Size of the Openings
% - To have the ability to sample all the cavity modes,
%   the SIZE OF THE OPENINGS IN THE CAVITY NEEDS TO BE SMALLER THAN THE MINIMUM GUIDED WAVELENGTH [6]. (p. 8)
% - On the other hand, the HARDSHIPS IN FABRICATION AND MICROMACHINING SET A LIMIT ON HOW SMALL THE OPENINGS CAN BE. (p. 8)
% - Eight different cases are studied, in which the cavity thickness and its material (STEEL) are kept constant as
%   the size of the holes at the bottom of the cavity was changed. (p. 8)
% - The NUMBER OF HOLES IS MAXIMIZED in each case by fitting as many openings as possible at the bottom of the cavity,
%   with the assumption that d_{o} = d_{b}. (p. 8)
% - With this setup, the largest opening size (λmin) has resulted in the largest sensing capacity among other opening sizes, and at the same time,
%   it is the easiest to fabricate in terms of feature size. (p. 8)
% 4. Simulation Results and Discussion / 4.1. The Effect of the Cavity Design on Sensing Capacity / 4.1.2. Material Selection
% - Hence, it is of interest to inspect whether using a 3D printing material such as VeroWhitePlus, which is not as stiff and dense as STEEL,
%   can adequately randomize the wave fields for compressive sensing. (p. 9)
% - Furthermore, another material, ALUMINUM, is tested as the cavity material, and its effect on the sensing capacity is studied, alongside with that of STEEL. (p. 9)
% - Although the addition of the plastic cavity to the domain has increased the sensing capacity,
%   its effect is not as strong as that of the STEEL or the ALUMINUM cavity. (p. 9)
% - Since ALUMINUM is lighter than STEEL, it can be made into thin layers, and its effect in the cavity is close to that of STEEL,
%   so it WAS SELECTED AS THE MATERIAL FOR THE CAVITY. (p. 9)
% - It is easy to observe that the field patterns have a reduced correlation as a result of pseudo-random illumination of the scene. (p. 10)
% 5. Conclusions
% - In this work, a theoretical study of using SOLID CAVITIES ENCLOSING ULTRASOUND SOURCES TO RANDOMIZE THE MEASUREMENTS was presented. (p. 14)
% - Such a novel measurements scheme was combined with CS theory to retrieve the image of two objects using a REDUCED NUMBER OF TRANSMITTERS. (p. 14)
% - The novelty of this work is the introduction of spectral coding cavities into ultrasound imaging. (p. 14)
% - This work was limited by
%   [1.)] the SIMPLIFIED SIMULATION LAYOUT that considered a small two-dimensional imaging domain,
%   [2.)] a REDUCED NUMBER OF TRANSCEIVERS and
%   [3.)] a SMALL NUMBER OF TARGETS to reduce the computational burden and to keep the assumptions valid. (p. 14)
\name{Ghanbarzadeh-Dagheyan} \emph{et al.} \cite{article:Ghanbarzadeh-DagheyanSensors2018} optimized
a holey cavity with
respect to
its opening sizes and
the materials to enable
compressive two-dimensional \ac{UI} with
only a few transceivers.
% b) presence of the cavity significantly improved the lateral resolution of two point-like targets in a lossless homogeneous fluid
% article:Ghanbarzadeh-DagheyanSensors2018: Holey-Cavity-Based Compressive Sensing for Ultrasound Imaging
% Abstract
% - As a proof-of-concept of this theoretical investigation,
%   TWO POINT-LIKE TARGETS LOCATED IN A UNIFORM BACKGROUND MEDIUM ARE IMAGED IN
%   THE PRESENCE AND THE ABSENCE OF THE CAVITY. (p. 1)
% - It is demonstrated that
%   the USE OF A CAVITY, whether it is made of PLASTIC OR METAL, can significantly ENHANCE
%   THE SENSING CAPACITY and THE POINT SPREAD FUNCTION of a focused beam. (p. 1)
% - The imaging performance is also improved in terms CROSS-RANGE RESOLUTION [sic!] when compared to the no-cavity case. (p. 1)
% 4. Simulation Results and Discussion
% - As shown in Figure 1, a SMALL NUMBER OF TRANSCEIVERS (ONLY TWO) are considered here to show
%   the general concept and the imaging capability using compressive sensing. (p. 7)
% - What is more, TWO POINT-LIKE TARGETS ARE SELECTED TO BE IMAGED. (p. 7)
% - The imaging domain has a grid size of n_{x} = 501 and n_{y} = 120 in the x and y direction, in order, leading to
%   a vector signal size of P = 60120 elements. (p. 7)
% - The FREQUENCY BAND IS 2–10 MHz, and the frequency steps in the sweeping are 0.1 MHz. (p. 7)
% - The number of transmitters N_{T}, receivers N_{R}, and frequencies N_{f} used in the simulations is 2, 2 and 81, which yields
%   a total measurement number of M = N_{T} N_{R} N_{f} = 324. (pp. 7, 8)
% - It is lucid that these values for M and P make the system in (9) underdetermined since P >> M. (p. 8)
% 4. Simulation Results and Discussion / 4.2. The Effect of the Cavity on Imaging and Point Spread Function
% - The PSF of the imaging system IMPROVES CONSIDERABLY when the aluminum cavity was used. (p. 10)
% - Specifically, the CROSS-RANGE ALIASING EFFECTS ARE ELIMINATED, and
%   the CROSS-RANGE RESOLUTION IS ENHANCED; nevertheless, this enhancement is not as significant when
%   the plastic cavity is employed. (p. 10)
% - Figure 8 shows that the matrix A∗A, when normalized, is closest to I when the aluminum cavity used, thus leading to the best imaging of all configurations. (p. 11)
% 5. Conclusions
% - It was shown that the SENSING CAPACITY and the PSF of the focused beam were significantly IMPROVED WHEN A CAVITY, MADE OF ALUMINUM OR PLASTIC, WAS UTILIZED. (p. 14)
% - The recovered images of two point-like targets inside a uniform medium showed that the use of the CAVITY ENHANCES THE CROSS-RANGE RESOLUTION, but
%   they might still possess some weak artifacts when the SNR decays below 10 dB. (p. 14)
Encasing only two point-like transceivers that perform
a complete \ac{SA} acquisition sequence,
its presence significantly improved
the lateral resolution of
two point-like targets in
a lossless homogeneous fluid.
% c) time-reversal focusing
% article:Ghanbarzadeh-DagheyanSensors2018: Holey-Cavity-Based Compressive Sensing for Ultrasound Imaging
% 1. Introduction
% - Other types of ULTRASOUND CAVITIES have also been proposed by Fink et al. to create images using
%   TIME-REVERSAL TECHNIQUES [26–29]. (p. 2)
% - However, they have not been used in the scope of compressive sensing; and therefore,
%   THEIR METHOD REQUIRES A LARGE NUMBER OF MEASUREMENTS [25]. (p. 2)
% article:KruizingaSciAdv2017: Compressive 3D ultrasound imaging using a single sensor
% DISCUSSION
% - The calibration procedure that we use and our imaging device both share common ideas with
%   the TIME-REVERSAL WORK conducted by Fink and co-workers (30–34).
% - Time-reversal ultrasound entails the notion that any ultrasound field can be focused back to its source by reemitting
%   the recorded signal back into the same medium. (p. 8)
% - Consequently, the ULTRASONIC WAVES CAN BE FOCUSED ONTO A PARTICULAR POINT IN SPACE AND TIME if
%   THE IMPULSE SIGNAL OF THAT POINT IS KNOWN. (p. 8)
% - Using a REVERBERANT CAVITY TO CREATE IMPULSE SIGNAL DIVERSITY AND A LIMITED NUMBER OF SENSORS,
%   Fink et al. have shown that a 3D medium can be imaged by APPLYING TRANSMIT FOCUSING WITH RESPECT TO EVERY SPATIAL LOCATION. (p. 8)
% - Hence, they need as many measurements as there are pixels, resulting in an unrealistic scenario for real-time imaging. (p. 8)
%
%   references from: article:Ghanbarzadeh-DagheyanSensors2018, article:KruizingaSciAdv2017
%	 [29], [33] article:EtaixJASA2012: Acoustic imaging device with one transducer
%   (x)	       [31] article:MontaldoITUFFC2005: Building three-dimensional images using a time-reversal chaotic cavity
%	 [28]	    article:QuieffinJASA2004: Real-time focusing using an ultrasonic one channel time-reversal mirror coupled to a solid cavity
%	       [34] article:MontaldoApplPhysLett2004: Time reversal kaleidoscope: A smart transducer for three-dimensional ultrasonic imaging
%	 [27], [32] article:DraegerJASA1999: One-channel time-reversal in chaotic cavities: Experimental results
%        [26]	    article:DraegerPhysRevLett1997: One-channel time reversal of elastic waves in a chaotic 2D-silicon cavity
%	       [30] article:FinkITUFFC1992: Time reversal of ultrasonic fields. I. Basic principles
%
% - Besides the time-reversal work, our technique also shares some common ground with
%   the 2D LOCALIZATION WORK by Clement et al. (35–37). (p. 8)
%   [35] P. J. White, G. T. Clement, Two-dimensional localization with a single diffuse ultrasound field excitation. IEEE Trans. Ultrason. Ferroelectr. Freq. Control 54, 2309–2317 (2007).
%   [36] F. C. Meral, M. A. Jafferji, P. J. White, G. T. Clement, Two-dimensional image reconstruction with spectrally-randomized ultrasound signals. IEEE Trans. Ultrason. Ferroelectr. Freq. Control 60, 2501–2510 (2013).
% - Instead of solving a linear system as we propose here,
%   the authors used a dictionary containing measured impulse responses and a CROSS-CORRELATION TECHNIQUE to find the two point scatterers. (p. 8)
%
% article:MontaldoITUFFC2005: Building three-dimensional images using a time-reversal chaotic cavity
% I. Introduction
% - TIME-REVERSAL FOCUSING was studied previously in the field of ultrasound [9], for medical applications [10], and in ocean acoustics [11]. (p. 1489)
%   [9] M. Fink, “Time reversed acoustics,” Phys. Today, vol. 50, pp. 34–40, 1997.
%   [10] M. Fink, G. Montaldo, and M. Tanter, “Time reversal acoustics in biomedical engineering,” Annu. Rev. Biomed. Eng.,vol.5, pp. 465–497, 2003.
% - This technique is based on the REVERSIBILITY OF ACOUSTIC PROPAGATION, which implies that
%   the time-reversed version of an incident pressure field naturally refocuses in space and time on its source,
%   whatever the heterogeneity of the propagation medium. (p. 1489)
% - More precisely, it means that, for every burst of sound emitted from a source and possibly reflected and refracted by multiple boundaries,
%   there exists a set of waves that precisely retraces all the complex path and converges to the original source, as if time were going backward. (p. 1489)
% - Preliminary works on time reversal in 2-D closed cavities was done by Draeger et al. [13], [14] using Bunimovitch billiards. (pp. 1489, 1490)
%   [14] C. Draeger, J.-C. Aime, and M. Fink, “One-channel time-reversal in chaotic cavities: Experimental results,” J. Acoust. Soc. Amer., vol. 105, no. 2, pp. 618–625, 1999.
% - In order to focus a short pulse inside the medium, we use the TIME-REVERSAL PROCESS. (p. 1490)
% - This process achieved in a calibration medium like water allows us to learn
%   the TEMPORAL CODES TO BE APPLIED ON EACH TRANSDUCER IN ORDER TO FOCUS AT A GIVEN LOCATION. (p. 1491)
% - By repeating this process for different initial source locations,
%   we can learn the coded signals h_{i}(−t) allowing us to focus on any specific point of the medium. (p. 1491)
% - The complete calibration of the kaleidoscope consists in
%   recording all the data set of coded signals needed to focus at each point of the calibration medium. (p. 1491)
% - This process was modeled using a NUMERICAL SIMULATION OF THE WAVE PROPAGATION. (p. 1491)
% - This numerical approach greatly helps us to understand the building of the time-reversed focused beam. (p. 1491)
% article:MontaldoITUFFC2005: Building three-dimensional images using a time-reversal chaotic cavity
% Abstract
% - Thousands of transducers are typically needed for focusing and steering in a 3-D volume. (p. 1489)
% - In this article, we propose a different concept allowing us to obtain
%   ELECTRONIC 3-D FOCUSING WITH A SMALL NUMBER OF TRANSDUCERS. (p. 1489)
% - The basic idea is to couple a SMALL NUMBER OF TRANSDUCERS to
%   a CHAOTIC REVERBERATING CAVITY with one face in contact with the body of the patient. (p. 1489)
% - The reverberations of the ultrasonic waves inside the cavity CREATE AT EACH REFLECTION VIRTUAL TRANSDUCERS. (p. 1489)
% - The CAVITY ACTS AS AN ULTRASONIC KALEIDOSCOPE multiplying the small number of transducers and
%   CREATING A MUCH LARGER VIRTUAL TRANSDUCER ARRAY. (p. 1489)
% - By exploiting time-reversal processing, it is possible to use collectively all the virtual transducers to FOCUS A PULSE EVERYWHERE IN A 3-D VOLUME. (p. 1489)
% - The reception process is based on a nonlinear pulse-inversion technique in order to ensure a good contrast. (p. 1489)
% - The feasibility of this concept for the building of 3-D images was demonstrated using a prototype relying only on
%   31 EMISSION TRANSDUCERS AND A SINGLE RECEPTION TRANSDUCER. (p. 1489)
% I. Introduction
% - Here, we present an original approach that REPLACES THE 2-D ARRAY BY A SET OF LESS THAN 100 ELEMENTS. (p. 1489)
% - Our solution combines the use of TIME REVERSAL TECHNOLOGY with
%   a SMALL NUMBER OF PIEZOELECTRIC TRANSDUCERS fastened to a reverberating solid cavity presenting one face in contact with the investigated medium. (p. 1489)
% - Thanks to the multiple reverberations on the waveguide boundaries, waves emitted by each transducer are multiply reflected, creating
%   at each reflection virtual transducers that can be observed from the desired focal point. (p. 1489)
% - Thus, we create a large virtual array from a limited number of transducers. (p. 1489)
% - The result of such an operation is that a small number of transducers is multiplied to create a kaleidoscopic transducer array. (p. 1489)
% - However, symmetries implied in waveguides create periodic kaleidoscopic arrays, resulting in
%   grating lobes that limit the interest of this technique to shock wave generation for lithotripsy. (p. 1489)
% - The solution we propose is to break the waveguide’s sym- metries by introducing reverberating media with chaotic geometries such as chaotic billiards. (p. 1489)
% - Here, we extend this work to 3-D leaky cavities and we select
%   a Sinai billiard geometry to achieve 3-D focusing. (p. 1490)
% III. Image Formation Using the Cavity
% - In most ultrasonic devices, the receiving transducers are the same as the transmit ones. (p. 1495)
% - However, in our case it is very difficult to use the same transducers in both transmit and receive modes. (p. 1495)
% - When the backscattered echoes reach the surface of the cavity,
%   only a few percent of the pressure amplitude penetrates in the cavity because of the strong mechanical impedance mismatch between water and aluminum. (p. 1495)
% - As the reverberation time in the cavity is very long, these very weak backscattered echoes are mixed inside the cavity with
%   the residual reverberation noise of the transmit sequence. (p. 1495)
% - Unfortunately, it is not possible to differentiate them from the latter. (p. 1495)
% - One might be tempted to decrease the impedance mismatch between the cavity and the imaged medium. (p. 1495)
% - However, such a choice would increase the leakage of waves out of the cavity during the transmit mode. (p. 1495)
% - Consequently, it would decrease the time-reversal focusing efficiency as the number of reverberations inside the cavity would be smaller.  (p. 1495)
% - The main advantages of this single receiver at the front face of the cavity is that it overcomes
%   the limit of weak wave transmission at the solid-fluid interface. (p. 1496)
% - Backscattered echoes are recorded before entering the cavity. (p. 1496)
% - Thus, the final procedure in order to obtain an image consists in two steps:
%	[1.)] Calibration.
%	- The kaleidoscope is calibrated in water while learning the data set of transmit codes that allow us to focus pulses at
%         any location in the 3-D volume of interest, as explained in Section II. (p. 1496)
%	- Calibration experiments were carried out for 1600 focal points on a 40 by 40 grid, of a 40 by 40 mm plane placed at a 50 mm focal depth from
%	  the emitting surface (see Fig. 8). (p. 1496)
%	[2.)] Imaging.
%	- The kaleidoscope then is placed in front of the object to image, and we measure the second harmonic component of
%	  the backscattered echoes using the single receive transducer. (p. 1496)
%	- The test objects are tissue phantoms made of gelatin and containing ran- domly distributed scatterers (agar powder). (p. 1496)
%	- The frame rate is limited by the spreading time in the cavity, using signals of 500 µs we need 0.8 seconds to make a 40 by 40 points image. (p. 1496)
% - However, the contrast is not yet high enough to image human organs. (p. 1496)
% - Combined with the time-reversal process, this device exploits the multiple reverberations in a chaotic and leaky cavity to focus a pulse in the medium of interest. (p. 1496)
% - It is important to note that such chaotic cavities are very easy to build; we do not need to use small transducers or specific shapes,
%   we can glue transducers everywhere on the external surface of the cavity. (p. 1496)
% - On the contrary, the mechanical impedance mismatch between the cavity and the imaged medium is responsible for
%   the ability of time-reversal processing to focus waves in a 3-D volume with a very small number of transducers. (p. 1496)
% - Compared to the technological difficulties of a 2-D array of transducers, these simplifications make the construction of the cavities very easy. (p. 1496)
% - The complexity of 2-D arrays design is now transferred into the spatiotemporal coding techniques adapted to the cavity shape and stored into memories. (p. 1496)
Unlike
% 1.) time-reversal technique
the time-reversal technique
\cite{article:MontaldoITUFFC2005}, which uses
% 2.) leaky reverberant cavity
a leaky cavity to generate
% 3.) focused beams
% TODO: Sinai Billiard
focused beams,
% 4.) field-of-view (FOV)
the \ac{FOV} is not progressively scanned.

%---------------------------------------------------------------------------------------------------------------
% 3.) van Sloun et al.
%---------------------------------------------------------------------------------------------------------------
% a) van Sloun et al. proposed randomly-apodized transmissions from a circular array in two-dimensional tomography
% letter:VanSlounITBME2015: Compressed Sensing for Ultrasound Computed Tomography
% Abstract
% - In this letter, we propose a COMPRESSED SENSING SOLUTION FOR UCT. (p. 1660)
% - The adopted measurement scheme is based on COMPRESSED ACQUISITIONS, with
%   CONCURRENT RANDOMISED TRANSMISSIONS IN A CIRCULAR ARRAY CONFIGURATION. (p. 1660)
% - Reconstruction of the image is then obtained by combining
%   the BORN ITERATIVE METHOD and TOTAL VARIATION MINIMIZATION, thereby exploiting
%   VARIATION SPARSITY in the image domain. (p. 1660)
% - Evaluation using simulated UCT scattering measurements shows that
%   the PROPOSED TRANSMISSION SCHEME PERFORMS BETTER than
%   UNIFORM UNDERSAMPLING, and is able to REDUCE ACQUISITION TIME BY ALMOST ONE ORDER OF MAGNITUDE, while maintaining high spatial resolution. (p. 1660)
% I. INTRODUCTION
% - A typical arrangement is one where the BREAST IS ENCLOSED BY A CIRCULAR ARRAY of ultrasound transducer elements,
%   SEQUENTIALLY TRANSMITTING ONE-BY-ONE and receiving the resulting scattered wave fields at all elements [2]. (p. 1660)
% - For a large number of transducer elements, the acquisition time and complexity of this method are high. (p. 1660)
% - Considering a high-resolution ring-shaped ultrasound transducer with
%   a diameter of 20 cm, 1024 transmit elements, and an average speed of sound c0 of 1540 m/s,
%   it takes roughly 130 ms to image a single slice using sequential transmissions. (p. 1660)
% - In the latter [cylindrical matrix configuration for 3D imaging], imaging a full breast consisting of a number of slices in the order of 100, would require
%   an acquisition time in the order of tens of seconds. (p. 1660)
% - One approach is to UNIFORMLY UNDERSAMPLE IN THE TRANSMISSION DOMAIN, while keeping
%   the same number of receivers and employing sparse reconstruction techniques to retain the desired image features. (p. 1660)
% - Instead of applying CS-based reconstruction to sparse-view data,
%   WE CONSIDER COMPRESSIVE ACQUISITIONS WITH RANDOMIZED PARALLEL TRANSMISSIONS FROM THE CIRCULAR ARRAY. (p. 1660)
% - INTRODUCING RANDOMNESS ALLOWS NEAR-OPTIMAL CONDITIONS on the number of measurements in terms of the sparsity [7], facilitating
%   reduction of acquisition time, while keeping high spatial resolution. (p. 1660)
% - We compare the performance of the proposed method with
%   a UNIFORM UNDERSAMPLING APPROACH that uses TV minimization. (p. 1660)
% - In this letter, we show that
%   CS can be applied effectively in UCT to REDUCE ACQUISITION TIME BY ALMOST AN ORDER OF MAGNITUDE, while maintaining high image resolution. (p. 1660)
% II. MEASUREMENT MODEL / B. Inverse CS Problem
% - However, assuming that O is sparse in some domain, CS theory is applied by choosing the matrix \mat{\Phi} such that
%   TRANSDUCERS SIMULTANEOUSLY TRANSMIT PRESSURE WAVES WITH RANDOM AMPLITUDES THAT ARE GAUSSIAN DISTRIBUTED, having mean zero and standard deviation equal to one. (p. 1661)
% - Since we are interested in reducing the amount of transmissions only, we choose to receive with all elements. (p. 1661)
% - The total amount of transmission events is reduced by a factor M / N_{t}^{2}, here referred to as
%   the ACQUISITION REDUCTION FACTOR (ARF). (p. 1661)
% V. RESULTS
% - The CS-based acquisition approach has a lower MNAE [mean normalized absolute error], becoming increasingly significant for higher reduction factors. (p. 1662)
% - The improvement is remarkable when fully exploiting CS theory and using randomized transmission events. (p. 1663)
% VI. CONCLUSION AND DISCUSSION
% - In this letter, we presented a new CS-based approach to diffraction UCT in a circular array configuration. (p. 1663)
% - By combining compressed, randomized transmission events, and sparse reconstruction techniques, the proposed method potentially allows
%   reduction of the acquisition time by an order of magnitude, while preserving high image resolution. (p. 1663)
% - Although this does not influence acquisition time, reconstruction using randomized compressed transmissions required about one iteration more. (p. 1663)
% - A quantitative analysis for both methods showed that using CS results in a lower MNAE for all ARFs, with the difference becoming increasingly significant for higher reductions. (p. 1663)
% - More specifically, the uniform method’s performance declines rapidly up to a reduction factor of 8, whereas the CS method remains more stable. (p. 1663)
% - Reducing the acquisition time using CS, comes at the cost of higher computational expense. (p. 1663)
% - Although solving the l1 minimization problem is about 30–50 times as expensive as solving the least-squares problem [16], the problem size is reduced by ARF. (p. 1663)
\name{Van Sloun} \emph{et al.}
\cite{letter:VanSlounITBME2015} proposed
randomly-apodized monofrequent emissions from
a circular array in
two-dimensional tomography.
% b) randomly-apodized emissions outperformed sparse SA acquisition sequences
These outperformed
sparse \ac{SA} acquisition sequences using
only a few emissions from
random elements.
%---------------------------------------------------------------------------------------------------------------
% 4.) Liu et al
%---------------------------------------------------------------------------------------------------------------
% a) Liu et al. utilize realizations of uniformly-distributed random variables as the apodization weights
% article:LiuITUFFC2018: Compressed Sensing Based Synthetic Transmit Aperture Imaging: Validation in a Convex Array Configuration
% II. THEORY STUDY / B. Compressed Sensing Based Synthetic Transmit Aperture
% - Since \mat{\Phi} in (2) usually obeys a random distribution, to adapt the data acquisition in ultrasound imaging,
%   \mat{\Phi} in (13) can be DESIGNED TO OBEY A CONTINUOUS UNIFORM RANDOM DISTRIBUTION WHOSE ENTRIES RANGE BETWEEN 0 AND 1, i.e.,
%   \mat{\Phi} ∼ U(0, 1). (p. 303)
% article:LiuITMI2017: A Compressed Sensing Strategy for Synthetic Transmit Aperture Ultrasound Imaging
% II. COMPRESSED SENSING BASED SYNTHETIC TRANSMIT APERTURE / B. Compressed Sensing Based Synthetic Transmit Aperture
% - This condition [incoherence] is often satisfied when \mat{\Phi} OBEYS A RANDOM DISTRIBUTION [2], that is why
%   the transmit apodization applied in each PW firing of CS-STA is random. (p. 881)
% - Specifically, in step 1) in this work, the TRANSMIT APODIZATIONS or measurement matrix \mat{\Phi} OBEY
%   A CONTINUOUS UNIFORM DISTRIBUTION WITH
%   the AMPLITUDES BEING RANDOMLY DISTRIBUTED IN THE INTERVAL OF [0, 1] AND THE MEAN VALUE BEING 0.5, i.e.,
%   \mat{\Phi} ∼ U(0, 1). (p. 881)
% VI. DISCUSSIONS / C. Reconstruction Quality
% - In this work, the measurement matrix \mat{\Phi} obeys a CONTINUOUS UNIFORM DISTRIBUTION with
%   AMPLITUDES BEING RANDOMLY DISTRIBUTED IN THE INTERVAL OF [0, 1]. (p. 889)
\name{Liu} \emph{et al.} \cite{article:LiuITUFFC2018,article:LiuITMI2017} realized
uniformly-distributed apodization weights for
linear and convex arrays.
% b) Liu et al. recovered the pulse echoes induced by a complete SA acquisition sequence and subsequently applied the popular DAS method for image formation
% article:LiuITUFFC2018: Compressed Sensing Based Synthetic Transmit Aperture Imaging: Validation in a Convex Array Configuration
% III. SIMULATIONS / A. Simulation Setup
% - To guarantee the sparsity,
%   the SYM8 WAVELET was chosen to construct the sparse basis \mat{\Psi}. (p. 304)
% VI. DISCUSSION / B. CS-STA Performance
% - That is to say, the slow time signal of STA firing has a SPARSER REPRESENTATION WITH SYM8 WAVELET than with Fourier basis. (p. 313)
% - Therefore, SYM8 WAVELET BASIS IS USED IN THIS PAPER. (p. 313)
% - In this paper, the coherence between the measurement matrix \mat{\Phi} and SYM8 WAVELET SPARSE BASIS is about 2.46,
%   close to the coherence 2.2 between noiselets and Daubechies D4 wavelet [2]. (p. 313)
% - It demonstrates that \mat{\Phi} and \mat{\Psi} are incoherent, which guarantees the successful reconstruction of CS-STA. (p. 313)
% article:LiuITMI2017: A Compressed Sensing Strategy for Synthetic Transmit Aperture Ultrasound Imaging
% II. COMPRESSED SENSING BASED SYNTHETIC TRANSMIT APERTURE / B. Compressed Sensing Based Synthetic Transmit Aperture
% - The SYM8 WAVELET [36] IS CHOSEN AS THE SPARSE BASIS \mat{\Psi} and
%   the tolerated error ε is empirically set as 1 × 10−3 in all the simulations, phantom and in vivo experiments in this paper. (p. 881)
% VI. DISCUSSIONS / C. Reconstruction Quality
% - In this work, we analyzed the SPARSITY OF THE SLOW TIME SIGNAL \vect{x} of the STA data in
%   the SYM8 WAVELET [36], Fourier, and wave atoms [40] domains. (p. 888)
% - Both signals from the SIMULATED AND IN VIVO OBJECTS show their sparsity and
%   THEY CAN BE REPRESENTED MORE SPARSELY IN THE SYM8 WAVELET DOMAIN than in the other two bases. (p. 888)
% - That is why THE SYM8 WAVELET WAS CHOSEN AS THE SPARSE BASIS IN THIS STUDY. (pp. 888, 889)
% - As a result, continuous uniform random distribution is used in this work, and
%   the coherence between such a distribution and the SYM8 WAVELET SPARSE BASIS is about 2.46,
%   close to the coherence 2.2 between the noiselets and Daubechies D4 wavelet [2]. (p. 889)
% - It proves that the continuous uniform random distribution is incoherent with the SYM8 WAVELET, which guarantees
%   the successful reconstruction of CS-STA. (p. 889)
Unlike
% 1.) inverse scattering methods
the inverse scattering methods,
% 2.) Liu et al.
they recovered
% 3.) recorded RF voltage signals
the echo signals induced by
% 4.) complete SA acquisition sequence
a complete \ac{SA} acquisition sequence, which were represented almost sparsely by
% 5.) sym8 wavelet basis
a sym8 wavelet basis, and subsequently applied
% 6.) popular DAS method
the popular \ac{DAS} method for
% 7.) image formation
image formation.
% c) large number of unknown temporal samples required on the order of 30 sequential pulse-echo measurements per image
% article:LiuITUFFC2018: Compressed Sensing Based Synthetic Transmit Aperture Imaging: Validation in a Convex Array Configuration
% Abstract
% - The experimental results showed that STA and CS-STA performed better than ME-STA and the focused method at small depths. (p. 300)
% - At the depth of 110 mm, CS-STA, ME-STA, and the focused methods improved the contrast and contrast-to-noise ratio of STA. (p. 300)
% - The improvements in CS-STA are higher than those in ME-STA but lower than those in the focused mode. (p. 300)
% - These results can also be observed qualitatively in the in vivo experiments on the liver of a healthy male volunteer. (p. 300)
% - The CS-STA method is thus proved to increase the frame rate and achieve high image quality at full depth in the convex array configuration. (p. 300)
% article:LiuITMI2017: A Compressed Sensing Strategy for Synthetic Transmit Aperture Ultrasound Imaging
% Abstract
% - In addition, the CONTRAST OF THE STA IMAGE CAN BE IMPROVED at the same time owing to the higher energy of plane wave firing in CS-STA. (p. 878)
% - The results demonstrate that, implemented with the same frame rate, CS-STA achieves HIGHER OR COMPARABLE RESOLUTION AND CONTRAST. (p. 878)
Despite
% 1.) improvements
the improvements in
% 2.) contrast
contrast and
% 3.) spatial resolution
spatial resolution,
% 4.) large number of unknown temporal samples
the large number of
unknown temporal samples required
% 5.) tens of sequential pulse-echo measurements per image
tens of
sequential pulse-echo measurements per
image.


%%%%%%%%%%%%%%%%%%%%%%%%%%%%%%%%%%%%%%%%%%%%%%%%%%%%%%%%%%%%%%%%%%%%%%%%%%%%%%%%%%%%%%%%%%%%%%%%%%%%%%%%%%%%%%%%
% 2.) specific contributions
%%%%%%%%%%%%%%%%%%%%%%%%%%%%%%%%%%%%%%%%%%%%%%%%%%%%%%%%%%%%%%%%%%%%%%%%%%%%%%%%%%%%%%%%%%%%%%%%%%%%%%%%%%%%%%%%
\subsection{Specific Contributions}
%\label{subsec:intro_contributions}
%---------------------------------------------------------------------------------------------------------------
% 1.) three major innovations of the proposed method
%---------------------------------------------------------------------------------------------------------------
% - Be sure to clearly state the purpose and /or hypothesis that you investigated.
% - When you are first learning to write in this format it is okay, and actually preferable, to use a pat statement like,
%   "The purpose of this study was to...." or
%   "We investigated three possible mechanisms to explain the ... (1) blah, blah..(2) etc.
% - It is most usual to place the statement of purpose near
%   THE END OF THE INTRODUCTION, often as the
%   TOPIC SENTENCE OF THE FINAL PARAGRAPH.
%
% - Provide a CLEAR STATEMENT OF THE RATIONALE FOR YOUR APPROACH to the problem studied.
% - State BRIEFLY HOW YOU APPROACHED THE PROBLEM (e.g., you studied oxidative respiration pathways in isolated mitochondria of cauliflower).
%   This will usually follow your statement of purpose in the last paragraph of the Introduction.
% - Do not discuss here the actual techniques or protocols used in your study (this will be done in the Materials and Methods);
%   your readers will be quite familiar with the usual techniques and approaches used in your field.
% - If you are using a novel (new, revolutionary, never used before) technique or methodology,
%   the merits of the new technique/method versus the previously used methods should be presented in the Introduction.
% a) method for the fast compressed acquisition and the subsequent recovery of images is proposed that features three major innovations
A method for
% 1.) fast compressed acquisition
the fast compressed acquisition and
% 2.) subsequent recovery
the subsequent recovery of
images is proposed that features
% 3.) three major innovations
three major innovations.
% b) proposed method significantly enhances current inverse scattering methods by realistic d-dimensional physical models
First,
% 1.) realistic d-dimensional physical models
realistic $d$-dimensional physical models for
% 2.) linear physical model for the pulse-echo measurement process
the pulse-echo measurement process and
% 3.) syntheses of the incident waves
the syntheses of
the incident waves minimize
% 4.) minimize model inaccuracies
inaccuracies and leverage
% 5.) abilities
the abilities of
% 6.) programmable UI systems
programmable \ac{UI} systems.
% c) proposed method recovers the compressibility fluctuations by a sparsity-promoting lq-minimization method
%   What are the SCIENTIFIC MERITS of this particular model system?
They linearly relate
% 1.) spatial compressibility fluctuations
the spatial compressibility fluctuations in
% 2.) weakly-scattering soft tissue structures
weakly-scattering soft tissue structures to
% 3.) recorded RF voltage signals
the \ac{RF} voltage signals provided by
% 4.) instrumentation
the instrumentation.
% d) proposed physical models are universal and permit additional applications beyond ultrafast UI
They readily support
% 1.) calibration procedures
calibration procedures,
% 5.) usage
the usage of
% 6.) measured incident fields
measured incident fields, and
% 7.) applications
applications beyond
% 8.) ultrafast UI
ultrafast \ac{UI}, e.g.
% 5.) conventional UI based on progressive scanning
progressive scanning,
% 6.) structured insonification
structured insonification, or
% 7.) simulation studies
simulation studies.
% a) three innovative types of random incident waves aid in meeting condition (ii)
Second,
% 1.) three innovative types of energy equivalent random waves
three innovative types of
energy equivalent random waves are synthesized using
% 2.) pseudo-random apodization weights
random apodization weights,
% 3.) random time delays
time delays, or
% 4.) combinations thereof
combinations thereof.
% b) associated spatial codes decorrelate the pulse echoes of the structural building blocks defined by an orthonormal basis and facilitate their discrimination in the image recovery
The associated
% 1.) spatial codes
spatial codes decorrelate
% 2.) pulse echoes
the pulse echoes of
% 3.) structural building blocks
the structural building blocks defined by
% 4.) orthonormal basis
an orthonormal basis meeting
% 5.) condition (i) [known dictionary of structural building blocks represents the image almost sparsely]
condition (i) and, thus, improve
% 6.) conformity
the conformity with
% 7.) condition (ii): [individual pulse echoes are sufficiently uncorrelated]
condition (ii).
% c) convex and nonconvex variants of a sparsity-promoting lq-minimization method enable the quantitative recovery of the compressibility fluctuations
Third, both
% 1.) convex
convex and
% 2.) nonconvex
nonconvex variants of
% 3.) sparsity-promoting lq-minimization method
a sparsity-promoting $\ell_{q}$-minimization method, $q \in [ 0; 1 ]$, enable
% 4.) quantitative recovery
the quantitative recovery of
% 5.) compressibility fluctuations
the compressibility fluctuations.

%---------------------------------------------------------------------------------------------------------------
% 2.) results of the simulation study
%---------------------------------------------------------------------------------------------------------------
% - Why did you choose this kind of experiment or experimental design?
% a) numerical simulation of a pulse-echo in the two-dimensional Euclidean space validates the emissions of single random incident waves
Two-dimensional numerical simulations validate
the proposed method using
single realizations of
the random waves for
% 1.) wire phantom
a wire phantom and
% 2.) tissue-mimicking phantom
a tissue-mimicking phantom.
% -> basis transform: Fourier
% d) Liu et al. require on the order of 30 sequential pulse-echo measurements per image, whereas the method proposed in this paper only requires the minimum number of a single wave emission
%In contrast to Liu and Kruizinga,
%the proposed method only requires
%the minimum number of
%a single pulse-echo measurement per
%image and uses additional types of
%random waves.
% b) former phantom permits a sparse representation of its spatial compressibility fluctuations in the canonical basis, whereas the latter phantom requires the Fourier basis
%The former phantom permits
%a sparse representation of
%its spatial compressibility fluctuations in
% 1.) canonical basis
%the canonical basis, whereas
%the latter phantom requires
% 2.) Fourier basis
%the \name{Fourier} basis.
% TODO: insight into the transfer behavior of the UI system
% TODO: amount of information collected by the array through different excited spatial frequencies is increased compared to conventional waves
% c) random incident waves decreased the robustness to
%   What ADVANTAGES DOES IT CONFER in answering the particular question(s) you are posing?
Although
the random waves decrease
% 1.) robustness against additive errors
the robustness against
additive errors for
the wire phantom,
they significantly increase both
% 2.) image quality
the image quality and
% 3.) speed of convergence
the convergence speed for
the tissue-mimicking phantom.
% a) author published two abstracts outlining the fundamental ideas of this paper in connection with oral presentations at two conferences
The study significantly expands
the initial results published in
two abstracts
\cite{proc:SchiffnerIUS2017,article:SchiffnerJASA2017}.
%% c) recovery experiments confirm better performance
%Numerical recovery experiments additionally demonstrate
%% 1.) increased SSIM indices
%increased \ac{SSIM} indices,
%% 2.) reduced relative RMSEs
%reduced relative \acp{RMSE}, and
%% 3.) faster convergence
%faster convergence.

%---------------------------------------------------------------------------------------------------------------
% 3.) results of the experimental validation
%---------------------------------------------------------------------------------------------------------------
% For the experimental validation of
% the random incident waves,
% we acquired pulse-echo measurement data from
% a real phantom consisting of nine wires.


%%%%%%%%%%%%%%%%%%%%%%%%%%%%%%%%%%%%%%%%%%%%%%%%%%%%%%%%%%%%%%%%%%%%%%%%%%%%%%%%%%%%%%%%%%%%%%%%%%%%%%%%%%%%%%%%
% table: summary of the mathematical symbols used throughout the paper
%%%%%%%%%%%%%%%%%%%%%%%%%%%%%%%%%%%%%%%%%%%%%%%%%%%%%%%%%%%%%%%%%%%%%%%%%%%%%%%%%%%%%%%%%%%%%%%%%%%%%%%%%%%%%%%%
\begin{table*}[tb]
 \centering
 \caption{%
  Summary of
  the mathematical symbols used
  throughout the paper.
 }
 \label{tab:list_symbols_math}
 \small
 \begin{tabular}{%
  @{}%
  >{$}l<{$}%		01.) symbol
  p{0.9\textwidth}%	02.) meaning
  @{}%
 }
 \toprule
  \multicolumn{1}{@{}H}{Symbol} &
  \multicolumn{1}{H@{}}{Meaning}\\
  \cmidrule(r){1-1}\cmidrule(l){2-2}
 \addlinespace
 %--------------------------------------------------------------------------------------------------------------
 % a) number sets
 %--------------------------------------------------------------------------------------------------------------
  % 1.) set of consecutive positive integers
  \setcons{ N } &
  Set of consecutive positive integers,
  $\setcons{ N } = \{ 1, 2, \dotsc, N \}$ for $N \in \N$\\
  % 2.) set of consecutive nonnegative integers
  \setconsnonneg{ N } &
  Set of consecutive nonnegative integers,
  $\setconsnonneg{ N } = \{ 0, 1, \dotsc, N \}$ for $N \in \Nnonneg$\\
 %--------------------------------------------------------------------------------------------------------------
 % b) inner product
 %--------------------------------------------------------------------------------------------------------------
  % 1.) inner product
  \inprod{ \vect{a} }{ \vect{b} } &
  Inner product of
  the vectors
  $\vect{a} = \trans{ ( a_{1}, \dotsc, a_{N} ) } \in \C^{ N }$ and
  $\vect{b} = \trans{ ( b_{1}, \dotsc, b_{N} ) } \in \C^{ N }$,
  $\inprod{ \vect{a} }{ \vect{b} } = \sum_{ n = 1 }^{ N } a_{n} \conj{ b }_{n}$\\
 %--------------------------------------------------------------------------------------------------------------
 % c) norms, quasinorms, and NNZC
 %--------------------------------------------------------------------------------------------------------------
  % 1.) lq-norm or lq-quasinorm
  \tnorm{ \vect{a} }{q} &
  $\ell_{q}$-norm, $q \in [ 1; \infty )$, or
  $\ell_{q}$-quasinorm, $q \in ( 0; 1 )$, of
  the vector
  $\vect{a} \in \C^{ N }$,
  $\tnorm{ \vect{a} }{q}^{q} = \sum_{ n = 1 }^{ N } \tabs{ a_{n} }^{q}$\\
  % 2.) number of nonzero components
  % book:Foucart2013: A Mathematical Introduction to Compressive Sensing / Chapter 2: Sparse Solutions of Underdetermined Systems / Sect. 2.1: Sparsity and Compressibility
  % - The customary notation \tnorm{ \vect{x} }{0} - the notation \tnorm{ \vect{x} }{0}^{0} would in fact be more appropriate - comes from the observation that
  %   \norm{ \vect{x} }{p}^{p} = \sum_{ j = 1 }^{ N } \abs{ x_{j} }^{p} \rightarrow \sum_{ j = 1 }^{ N } \indicator{ x_{j} \neq 0 }. (p. 41)
  % - In other words the quantity \tnorm{ \vect{x} }{0} is
  %   the limit as p decreases to zero of the pth power of the \ell_{p}-quasinorm of x. (p. 42)
  \tnorm{ \vect{a} }{0} &
  Number of nonzero components,
  $\tnorm{ \vect{a} }{0} := \tnorm{ \vect{a} }{0}^{0} = \lim_{q \rightarrow 0} \tnorm{ \vect{a} }{q}^{q} = \tabs{ \{ n \in \setcons{ N }: a_{n} \neq 0 \} }$\\
 %--------------------------------------------------------------------------------------------------------------
 % c) spatial coordinates
 %--------------------------------------------------------------------------------------------------------------
  % 1.) spatial position
  \vect{r} &
  Spatial position in
  the $d$-dimensional Euclidean space,
  $\vect{r} = \trans{ ( r_{1}, \dotsc, r_{d} ) } \in \R^{d}$\\
  % 2.) lateral coordinates and axial coordinate
  \vect{r}_{\rho}, r_{d} &
  % 2.1) lateral coordinates
  Lateral coordinates
  $\vect{r}_{\rho} = \trans{ ( r_{1}, \dotsc, r_{d-1} ) } \in \R^{d-1}$ and
  % 2.2) axial coordinate
  axial coordinate
  $r_{d} \in \R$ of
  the spatial position
  $\vect{r} = \trans{ ( \trans{ \vect{r}_{\rho} }, r_{d} ) }$\\
  % 3.) unit (d-1)-sphere
  \usphere{d-1} &
  Unit $(d-1)$-sphere,
  $\usphere{d-1} = \{ \vect{r} \in \R^{d}: \norm{ \vect{r} }{2} = 1 \}$\\
  % 4.) unit (d-1)-hemisphere
  \uhemisphere{d-1} &
  Unit $(d-1)$-hemisphere,
  $\uhemisphere{d-1} = \{ \vect{r} \in \R^{d}: \norm{ \vect{r} }{2} = 1, r_{d} \in \Rplus \}$\\
  % 5.) unit vector indicates the direction of the r_{\delta}-axis in a d-dimensional Cartesian coordinate system
  \uvect{\delta} &
  Unit vector indicating
  the direction of
  the $r_{\delta}$-axis, $\delta \in \setcons{ d }$, in
  a $d$-dimensional Cartesian coordinate system,
  $\uvect{\delta} \in \usphere{d-1}$\\
 %--------------------------------------------------------------------------------------------------------------
 % d) matrices
 %--------------------------------------------------------------------------------------------------------------
  % 1.) superscript H indicates an adjoint matrix
  ^{\hermsymbol} &
  Superscript indicating
  an adjoint (conjugate transpose) matrix\\
  % 2.) I denotes the identity matrix
  \mat{I} &
  Identity matrix\\
 %--------------------------------------------------------------------------------------------------------------
 % e) signal
 %--------------------------------------------------------------------------------------------------------------
  % 1.) tilde symbol identifies time-domain signals
  % TODO: recorded \ac{RF} voltage signals in the time domain.
  \tilde{ \cdot } &
  Tilde accent indicating
  a time-domain signal\\%, e.g. voltage signals
 \addlinespace
 \bottomrule
 \end{tabular}
\end{table*}

%---------------------------------------------------------------------------------------------------------------
% 3.) structure of the presentation and list of mathematical symbols
%---------------------------------------------------------------------------------------------------------------
% a) contributions are organized as follows
These contributions are organized as
follows.
% 2.) compressed sensing
\Cref{sec:compressed_sensing} briefly reviews
the \ac{CS} framework.
% 3.) linear physical model for the pulse-echo measurement process
% 4.) syntheses of the incident waves
\Cref{sec:linear_model,sec:syn_p_in} present
the physical models for
the pulse-echo measurement process and
the syntheses of
the incident waves.
% 5.) image recovery based on compressed sensing
\Cref{sec:recovery} details
the image recovery based on
\ac{CS}, and
% 6.) implementation
\cref{sec:implementation} adds
an efficient matrix-free implementation.
% 7.) simulation study
% 8.) experimental validation
%\cref{sec:experimental_validation}
% 9.) results
\Cref{sec:simulation_study} summarizes
the parameters of
the numerical simulations, and
\cref{sec:results} presents
the results.
% 10.) discussion
\Cref{sec:discussion} discusses
these results and
the proposed method.
% 11.) conclusion and outlook
Eventually,
\cref{sec:conclusion_outlook} concludes
the paper.
% b) table summarizes the mathematical symbols
\Cref{tab:list_symbols_math} summarizes
the mathematical symbols.


%---------------------------------------------------------------------------------------------------------------
% 4.) novel image recovery methods in fast pulse-echo UI
%---------------------------------------------------------------------------------------------------------------
% a) SA imaging and compressed beamforming effectively reduced the high spatial and temporal sampling rates by factors ranging from 4 to 10 to the sub-Nyquist regime
% article:ChernyakovaITUFFC2018: Fourier-Domain Beamforming and Structure-Based Reconstruction for Plane-Wave Imaging
% Abstract
% - In this work we extend
%   the RECENTLY PROPOSED FREQUENCY DOMAIN BEAMFORMING (FDBF) framework to
%   PLANE-WAVE IMAGING. (p. 1)
% - Beamforming in frequency yields
%   the SAME IMAGE QUALITY while
%   using FEWER SAMPLES. (p. 1)
% - It achieves AT LEAST 4 FOLD SAMPLING AND PROCESSING RATE REDUCTION by avoiding
%   oversampling required by standard processing. (p. 1)
% - To further reduce the rate [sampling and processing] we exploit
%   the STRUCTURE OF THE BEAMFORMED SIGNAL and use COMPRESSED SENSING METHODS to
%   RECOVER THE BEAMFORMED SIGNAL from
%   its PARTIAL FREQUENCY DATA obtained at a SUB-NYQUIST RATE. (p. 1)
% - Our approach OBTAINS 10 FOLD RATE REDUCTION compared to standard time domain processing. (p. 1)
% I. INTRODUCTION
% - Thus, in most commercial systems today
%   the NUMBER OF TRANSMISSIONS IS DICTATED BY THE NUMBER OF SCANLINES comprising the image. (p. 1)
% - As a result, the frame rate is limited to several tens of frames per second which is insufficient for a number of applications including
%   echocardiography for heart motion analysis, 3D/4D imaging and elastography. (p. 1)
% - The key to frame rate improvement without compromising image quality is to
%   break the link between the number of transmissions and the number of scanlines. (p. 1)
% - An obvious way to reduce the number of transmissions is to insonify the entire scene with a pulsed plane-wave. (p. 1)
% - The image lines are then obtained in parallel from the acquired data by standard dynamic beamforming upon reception. (p. 1)
% - However, due to inherent lack of focusing upon transmission, this method suffers from reduced contrast and resolution. (p. 1)
% - One approach to overcome this limitation is by SEQUENTIAL TRANSMISSION OF SEVERAL TILTED PLANE-WAVES [3]. (p. 1)
%   [3] article:MontaldoITUFFC2009
% - The images obtained from each insonification are added coherently to yield a final compounded image. (p. 1)
% - The result is characterized by significantly improved resolution and contrast since
%   COHERENT COMPOUNDING EFFECTIVELY GENERATES A POSTERIORI SYNTHETIC FOCUSING IN THE TRANSMISSION [3]. (p. 1)
% FDBF
% - When all the beam’s Fourier coefficients within its bandwidth are computed,
%   the sampling and processing rates are equal to the effective Nyquist rate [10]. (p. 2)
% - In this case the oversampling required by time domain implementation of digital beamforming [4] is avoided leading to 4-10 fold rate reduction. (p. 2)
% - The beam in time is then recovered by an inverse Fourier transform. (p. 2)
% - When further rate reduction is required, only a subset of the beam’s Fourier coefficients is obtained, implying that
%   the detected signals are sampled and processed at a sub-Nyquist rate. (p. 2)
% - Recovery then relies on CS methods that exploit an appropriate model of the beam to compensate for the lack of frequency data. (p. 2)
% - Low-rate data acquisition is based on the ideas of Xampling [12], [24], [25], which
%   obtains the Fourier coefficients of individual detected signals from their low-rate samples. (p. 2)
% proc:SchiffnerIUS2016a: A low-rate parallel Fourier domain beamforming method for ultrafast pulse-echo imaging
%The optimization of
% 1.) sparse arrays
%sparse arrays
%\cite{article:RouxSciRep2018} and
% 2.) compressed beamforming
%compressed beamforming
%\cite{article:ChernyakovaITUFFC2018,proc:SchiffnerIUS2016a}, for example, enabled
% 3.) sub-Nyquist spatiotemporal sampling rates
%sub-\name{Nyquist} spatiotemporal sampling rates, whereas
% c) adaptive beamforming, coded excitation, convolutional neural networks, and deconvolution-based image processing significantly improved the image quality
% [44] F. Varray, M. A. Kalkhoran, and D. Vray, “Adaptive minimum variance coupled with sign and phase coherence factors in IQ domain for plane wave beamforming,” in Proc. IEEE Int. Ultrason. Symp., Sep. 2016, pp. 1–4.
% [45] A. M. Deylami, J. A. Jensen, and B. M. Asl, “An improved minimum variance beamforming applied to plane-wave imaging in medical ultrasound,” in Proc. IEEE Int. Ultrason. Symp., Sep. 2016, pp. 1–4.
% article:HolfortITUFFC2009: Broadband Minimum Variance Beamforming for Ultrasound Imaging
% V. Conclusions
% - An approach for near-field, ADAPTIVE BEAMFORMING OF BROADBAND DATA BASED ON THE MINIMUM VARIANCE (MV) BEAMFORMER has been proposed. (p. 323)
% - The method is validated using Field II simulated synthetic aperture (SA) data and PLANE WAVE (PW) DATA. (p. 323)
% - The performance of the MV beamformer is compared with the DS beamformer using boxcar and Hanning weights. (p. 323)
% - The adaptive subband beamformer provides
%   a SIGNIFICANT INCREASE IN RESOLUTION AND CONTRAST compared with the conventional beamformer, even when
%   USING A SINGLE EMISSION. (pp. 323, 324)
% - It is seen that the resolution does not increase significantly when averaging several single emission images for MV. (p. 324)
% - Thus, the MV beamformer introduces the possibility of imaging the entire region in a single emission using only a single emission. (p. 324)
%adaptive beamforming
%\cite{article:HolfortITUFFC2009},
% 3.) convolutional neural networks
% letter:GasseITUFFC2017: High-Quality Plane Wave Compounding Using Convolutional Neural Networks
% Abstract
% - We propose a new strategy to REDUCE THE NUMBER OF EMITTED PWs BY LEARNING A COMPOUNDING OPERATION FROM DATA, i.e.,
%   by TRAINING A CONVOLUTIONAL NEURAL NETWORK to reconstruct high-quality images using a small number of transmissions. (p. 1637)
% - We present experimental evidence that this approach is promising, as
%   WE WERE ABLE TO PRODUCE HIGH-QUALITY IMAGES FROM ONLY THREE PWs, competing in terms of CONTRAST RATIO AND LATERAL RESOLUTION with
%   the standard compounding of 31 PWs (10× speedup factor). (p. 1637)
% I. INTRODUCTION
% - In this letter, we formulate PW compounding as a supervised learning problem. (p. 1637)
% V. CONCLUSION
% - A methodology for learning
%   a PW compounding operation from
%   data with
%   supervised learning was presented, and experimental evidence was provided that
%   a CNN model is able to exploit information from separate PW acquisitions more efficiently than
%   standard compounding, resulting in a better tradeoff between image quality and frame rate. (p. 1639)
%convolutional neural networks
%\cite{letter:GasseITUFFC2017}, and
% 4.) deconvolution-based image processing
%\TODO{deconvolution}
% article:AlbertiSJAM2017: Mathematical Analysis of Ultrafast Ultrasound Imaging -> no deconvolution
% article:ChenITMI2016: Compressive Deconvolution in Medical Ultrasound Imaging -> not ultrafast
% article:EPFL guy
%deconvolution-based image processing significantly improved
%the image quality.
