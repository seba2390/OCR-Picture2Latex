%---------------------------------------------------------------------------------------------------------------
% 1.) random waves in computational microwave imaging
%---------------------------------------------------------------------------------------------------------------
% a) syntheses of random waves in UI traces back to computational microwave imaging
% article:Ghanbarzadeh-DagheyanSensors2018: Holey-Cavity-Based Compressive Sensing for Ultrasound Imaging
% 1. Introduction
% - SEVERAL SENSING AND IMAGING APPLICATIONS [4] have been able to take advantage of CS by using
%   [1.)] PSEUDO-RANDOM ILLUMINATION IN THE OUTGOING WAVES from the transmitters and collecting
%   [2.)] PSEUDO-RANDOM MEASUREMENTS FROM THE INCOMING WAVES to the receivers. (pp. 1, 2)
% - Carin et al. used random positions for the elements of a sensing array to make use of CS. (p. 2)
% - They also showed that PLACING SPHERICAL SCATTERING OBJECTS IN FRONT OF THE WAVES increases
%   the RANDOMNESS AND INCOHERENCE of the measurements [5]. (p. 2)
%   [5] Carin, L.; Liu, D.; Guo, B. Coherence, compressive sensing, and random sensor arrays. IEEE Antennas Propag. Mag. 2011, 53, 28–39.
The syntheses of
% 1.) random waves
random waves in
% 2.) ultrasound imaging (UI)
\ac{UI} partly trace back to
% 3.) time-reversal methods
% article:MontaldoITUFFC2005: Building three-dimensional images using a time-reversal chaotic cavity
time-reversal
\cite{article:MontaldoITUFFC2005} and
% 4.) computational microwave imaging
% article:FromentezeOptExp2017: Computational polarimetric microwave imaging (leaky reverberant cavity)
% article:GollubSciRep2017: Large Metasurface Aperture for Millimeter Wave Computational Imaging at the Human-Scale (complex metamaterial)
% article:FromentezeApplPhysLett2015: Computational imaging using a mode-mixing cavity at microwave frequencies (leaky reverberant cavity)
% article:LipworthJOSAA2013: Metamaterial apertures for coherent computational imaging on the physical layer (complex metamaterial)
% article:HuntScience2013: Metamaterial Apertures for Computational Imaging (complex metamaterial)
computational microwave imaging
\cite{article:FromentezeOptExp2017,article:GollubSciRep2017,article:FromentezeApplPhysLett2015,article:LipworthJOSAA2013,article:HuntScience2013}.
% b) computational microwave imaging trades the hardware complexity and costs for the computational complexity of the image recovery
% article:FromentezeOptExp2017: Computational polarimetric microwave imaging (microwave imaging)
% 1. Introduction
% - The FREQUENCY-DIVERSE APERTURE LIMITS
%   THE COMPLEXITY OF THE HARDWARE ARCHITECTURE required for real-time high-resolution imaging, obviating
%   the ACTIVELY CONTROLLED COMPONENTS or
%   the NEED FOR MECHANICAL MOTION that is typically required in conventional systems. (p. 3)
% article:ChewITAP1997: Fast solution methods in electromagnetics
% I. Introduction
% - However,
%   [1.)] the RECENT PHENOMENAL GROWTH IN COMPUTER TECHNOLOGY, coupled with
%   [2.)] the DEVELOPMENT OF FAST ALGORITHMS with reduced computational complexity and memory requirements,
%   have made a RIGOROUS NUMERICAL SOLUTION OF THE PROBLEM OF SCATTERING FROM ELECTRICALLY LARGE OBJECTS FEASIBLE. (p. 533)
The latter trades
% 1.) hardware complexity
the hardware complexity and
% 2.) hardware costs
costs, which are raised by
% 3.) fully-sampled transceiver arrays
transceiver arrays or
% 4.) mechanical scanning
mechanical scans, for
% 5.) computational costs
the computational costs of
% 6.) image recovery
the image recovery.
% c) highly dispersive customized apertures convert excitations at different frequencies into spatially diverse and distinct emitted fields
% article:Ghanbarzadeh-DagheyanSensors2018: Holey-Cavity-Based Compressive Sensing for Ultrasound Imaging
% 1. Introduction
% - Incoherence between each of two measurements can also be achieved by using
%   PHYSICAL STRUCTURES that exhibit DIFFERENT WAVE-MATTER RESPONSES AT DIFFERENT INSTANTANEOUS FREQUENCIES, without
%   the need to CHANGE THE ARRANGEMENT OF THE SENSING ARRAY. (p. 2)
Highly dispersive customized apertures, e.g.
% 1.) complex metamaterials / metasurfaces
% article:Ghanbarzadeh-DagheyanSensors2018: Holey-Cavity-Based Compressive Sensing for Ultrasound Imaging
% 1. Introduction
% - METAMATERIALS HAVE ALSO BEEN USED TO CREATE RANDOMNESS IN THE SENSING SYSTEM for applications such as
%   MICROWAVE IMAGING [8–10], OPTICAL IMAGING [11,12], MILLILITER-WAVE [sic!] IMAGING [13–16] and
%   ACOUSTIC MULTICHANNEL SEPARATION using a single sensor [17]. (p. 2)
% - Similar to CODED MASKS that are commonly used in compressive OPTICAL IMAGING [22–24] as
%   a subgroup of hardware-based methods [...]. (p. 2)
%
%   references from: article:Ghanbarzadeh-DagheyanSensors2018, article:FromentezeOptExp2017
%   microwave imaging:
%   (x)  [8], [25] article:HuntScience2013: Metamaterial Apertures for Computational Imaging [RANDOM IN TX/RX; K-band (18 to 26 gigahertz)]
%		   - The imaging system we present here combines a computational imaging approach with custom aperture hardware that allows compression to be performed on
%		     the PHYSICAL LAYER that is used to do the ILLUMINATION and/or RECORDING.
%	 [9]	   article:HuntJOSAA2014: Metamaterial microwave holographic imaging system
%   (x) [10], [26] article:LipworthJOSAA2013: Metamaterial apertures for coherent computational imaging on the physical layer [RANDOM IN TX/RX; K-band (18 to 26 gigahertz)]
%		   - Furthermore, by randomly distributing the resonance frequencies of its elements, we demonstrated the metaimager can
%		     illuminate a scene with random field patterns well suited for compressive sensing. (p. 1611)
%	[27]	   article:YurdusevenOptExp2016: Frequency-diverse microwave imaging using planar mills-cross cavity apertures
%	[28]	   article:MarksJOSAA2016: Spatially resolving antenna arrays using frequency diversity
%   (x) [29]	   article:GollubSciRep2017: Large Metasurface Aperture for Millimeter Wave Computational Imaging at the Human-Scale
%	[30]	   article:YurdusevenOptExp2017: Millimeter-wave spotlight imager using dynamic holographic metasurface antennas
%
%   optical imaging:
%	[11]	  article:WattsNatPhot2014: Terahertz compressive imaging with metamaterial spatial light modulators
%	[12]	  article:LiutkusSciRep2014: Imaging With Nature: Compressive Imaging Using a Multiply Scattering Medium [RANDOM in RX]
%     coded masks: (metamaterial?)
%	[22]	  Rawat, N.; Hwang, I.C.; Shi, Y.; Lee, B.G. Optical image encryption via photon-counting imaging and compressive sensing based ptychography. J. Opt. 2015, 17, 065704. [CrossRef]
%	[23]	  Spencer, A.P.; Spokoyny, B.; Ray, S.; Sarvari, F.; Harel, E. Mapping multidimensional electronic structure and ultrafast dynamics with single-element detection and compressive sensing. Nat. Commun. 2016, 7, 10434. [CrossRef] [PubMed]
%	[24]	  Ma, R.; Hu, F.; Hao, Q. Active Compressive Sensing via Pyroelectric Infrared Sensor for Human Situation Recognition. IEEE Trans. Syst. Man Cybern. Syst. 2017, 47, 3340–3350. [CrossRef]
%
%   millimeter-wave imaging:
%	[13]	  coll:MolaeiAABF2017: Compressive Reflector Antenna Phased Array
%	[14]	  proc:MolaeiISAP2017: A bilayer ELC metamaterial for multi-resonant spectral coding at mm-Wave frequencies [July 2017]
%	[15]	  proc:MolaeiISAP2016: Active imaging using a metamaterial-based compressive reflector antenna
%	[16]	  article:MolaeiISTHS2017: A 2-bit and 3-bit metamaterial absorber-based compressive reflector antenna for high sensing capacity imaging
%
%   acoustic multichannel separation:
%		  article:XieSciRep2016: Acoustic Holographic Rendering with Two-dimensional Metamaterial-based Passive Phased Array
%	[17]	  article:XiePNAS2015: Single-sensor multispeaker listening with acoustic metamaterials (RANDOM in RX)
%
% article:GollubSciRep2017: Large Metasurface Aperture for Millimeter Wave Computational Imaging at the Human-Scale (microwave imaging)
% Abstract
% - 
% article:LipworthJOSAA2013: Metamaterial apertures for coherent computational imaging on the physical layer (microwave imaging)
% Abstract
% - We introduce the concept of a METAMATERIAL APERTURE, in which
%   an underlying reference mode interacts with a designed metamaterial surface to produce
%   A SERIES OF COMPLEX FIELD PATTERNS. (p. 1603)
% - As the frequency of operation is scanned, different subsets of metamaterial elements become active, in turn varying the field patterns at the scene. (p. 1603)
% - Scene information can thus be indexed by frequency, with the overall effectiveness of the imaging scheme tied to the diversity of the generated field patterns. (p. 1603)
% - In this work we provide the foundation for computational imaging with metamaterial apertures based on FREQUENCY DIVERSITY, and establish that
%   for resonators with physically relevant Q-factors, there are potentially ENOUGH DISTINCT MEASUREMENTS OF A TYPICAL SCENE WITHIN A REASONABLE BANDWIDTH to achieve
%   diffraction-limited reconstructions of physical scenes. (p. 1603)
% 1. INTRODUCTION
% - Imaging systems based on incoherent light typically either populate the image plane with an array of fixed detectors that acquire information in parallel, or mechanically scan a smaller number of detec- tors that acquire scene information serially. (p. 1603)
% - In both cases, a system of optics - often quite complex - is typically used to transmit the scene to the aperture in as pristine a condition as possible. (p. 1603)
% - In fact, since all natural scenes are known to be compressible in some basis, natural scenes can be perfectly recovered with significantly fewer measurement modes than
%   the SBP (N << M)[2,3] provided that a set of optimal measurement modes is utilized. (p. 1603)
% - This statement forms the basis for COMPRESSIVE COMPUTATIONAL IMAGING. (p. 1603)
% - The reality that the N measurement modes are of consequence and not necessarily the methods for forming or detecting them suggests that
%   we may seek NEW COMPUTATIONAL IMAGING MODALITIES for coherent light that can potentially provide SIMILAR FUNCTIONALITY [as phased arrays], but with
%   REDUCED COST AND COMPLEXITY. (p. 1604)
% - Artificial materials exhibit TWO KEY ADVANTAGES:
%   [1.)] first, they offer access to DESIGNED ELECTROMAGNETIC PROPERTIES that may be difficult or impossible to find in NATURALLY OCCURRING MEDIA. (p. 1604)
%   [2.)] A second, potentially more revolutionary, advantage is that artificial materials present the potential for DYNAMIC TUNING [8],
%         which could enable PHASED-ARRAY-LEVEL CONTROL OVER MEASUREMENT MODES in a package with
%	  the LOW COST AND SIMPLICITY OF HOLOGRAPHIC APERTURES. (p. 1604)
% - One particular implementation of HOLOGRAPHIC IMAGING USING METAMATERIALS AT MICROWAVE FREQUENCIES, which serves as the subject of the present analysis, is that of
%   a GUIDED-MODE METAMATERIAL IMAGER (henceforth METAIMAGER) that radiates via coupling a guided wave to a set of
%   RESONANT, METAMATERIAL ELEMENTS DISTRIBUTED ALONG THE PROPAGATION PATH [9]. (p. 1604)
% - As presented, the metaimager is a single-pixel device that performs SEQUENTIAL MEASUREMENTS OF A SCENE using
%   A FREQUENCY-BASED ENCODING OF THE MEASUREMENT MODES. (p. 1604)
% - Because the number of measurement modes is equal to the number of distinct patterns that can be generated over a given frequency bandwidth,
%   THE APERTURE DESIGN STRATEGY IS TO MAXIMIZE FREQUENCY DIVERSITY. (p. 1604)
% - Thus, the metaimager is populated with metamaterial elements whose
%   RESONANCE FREQUENCIES ARE DISTRIBUTED RANDOMLY OVER A GIVEN BANDWIDTH, each with as large a quality (Q-) factor as possible. (p. 1604)
% - The resulting aperture produces a sequence of illumination patterns that
%   VARY RAPIDLY AS A FUNCTION OF FREQUENCY and are well suited for compressive imaging of canonically sparse scenes. (p. 1604)
% - The ADVANTAGE OF IMAGING USING FREQUENCY DIVERSITY is that
%   A SERIES OF MEASUREMENT MODES CAN BE OBTAINED USING A SINGLE FREQUENCY SCAN, avoiding
%   MECHANICAL SCANNING, MULTIPLE DETECTORS, or even RECONFIGURABLE ELEMENTS. (p. 1604)
% - This frequency scanned metamaterial imager provides an important proof of concept that
%   more advanced and novel imaging modalities can be achieved in coherent imaging schemes through the use of
%   COMPLEX, DESIGNED APERTURES. (p. 1604)
% 4. METAMATERIAL APERTURE
% - The use of resonance not only grants us access to the exotic electromagnetic responses metamaterials are known for, but also allows us to use
%   frequency as a convenient parameter by which to index our measurement modes. (p. 1607)
% - The details of the actual complementary metamaterial elements are unimportant to the present discussion; as was stated in the introduction,
%   we model each complementary metamaterial element as a radiating dipole [20,37]. (p. 1607)
% - It is now evident why frequency can serve as a parameter by which to index the measurement modes. (p. 1607)
% - From Eq. (25) we note that by sweeping ω the polarizability of each dipole changes; in addition, the local field U_{GW} at each dipole can change with ω as well. (p. 1607)
% - From Eq. (24) we see that both polarizability and U_{GW} affect the dipole moments, and these in turn can modify the field pattern with which the array illuminates the scene, calculated according to Eq. (26). (p. 1607)
% - The difference in radiation patterns highlights how unique sets of measurement modes can be accessed using only frequency diversity. (p. 1608)
% 8. CONCLUSIONS
% - We have introduced a COMPUTATIONAL IMAGING FRAMEWORK appropriate to a variety of SINGLE-PIXEL COHERENT IMAGERS, and applied it to
%   a specific aperture implementation we termed the metaimager — a 2D guided-wave aperture radiating via an array of complementary metamaterial elements. (p. 1611)
% - We have modeled each element as a RADIATING DIPOLE and showed
%   HOW THEIR DISPERSION ALLOWS THE METAIMAGER TO CONTROL ITS FIELD PATTERNS THROUGH FREQUENCY DIVERSITY. (p. 1611)
% - Furthermore, by RANDOMLY DISTRIBUTING THE RESONANCE FREQUENCIES of its elements, we demonstrated
%   THE METAIMAGER CAN ILLUMINATE A SCENE WITH RANDOM FIELD PATTERNS well suited for compressive sensing. (p. 1611)
% - Lastly, we have presented simulations of 2D and 3D scene reconstructions demonstrating the imaging capabilities of the proposed metaimager. (p. 1611)
% article:HuntScience2013: Metamaterial Apertures for Computational Imaging (microwave imaging)
% Abstract
% - By leveraging metamaterials and compressive imaging, a low-profile aperture capable of microwave imaging without
%   lenses, moving parts, or phase shifters is demonstrated. (p. 1)
% - This designer aperture allows image compression to be performed on the physical hardware layer rather than in the postprocessing stage, thus averting
%   the detector, storage, and transmission costs associated with full diffraction-limited sampling of a scene. (p. 1)
% - A guided-wave metamaterial aperture is used to perform compressive image reconstruction at 10 frames per second of two-dimensional (range and angle) sparse still and
%   video scenes at K-band (18 to 26 gigahertz) frequencies, using frequency diversity to avoid mechanical scanning. (p. 1)
% - Image acquisition is accomplished with a 40:1 compression ratio. (p. 1)
complex metamaterials
\cite{article:GollubSciRep2017,article:LipworthJOSAA2013,article:HuntScience2013} or
% 2.) leaky reverberant cavities
% article:Ghanbarzadeh-DagheyanSensors2018: Holey-Cavity-Based Compressive Sensing for Ultrasound Imaging
% Abstract
% - The use of SOLID CAVITIES AROUND ELECTROMAGNETIC SOURCES has been recently reported as
%   a MECHANISM TO PROVIDE ENHANCED IMAGES AT MICROWAVE FREQUENCIES. (p. 1)
% - These cavities are used as MEASUREMENT RANDOMIZERS; and they COMPRESS THE WAVE FIELDS AT THE PHYSICAL LAYER. (p. 1)
% - As a result of this compression,
%   THE AMOUNT OF INFORMATION COLLECTED BY THE SENSING ARRAY through the different excited modes inside the resonant cavity is increased when
%   compared to that obtained by no-cavity approaches. (p. 1)
% 1. Introduction
% - For instance, Fromenteze et al. used
%   a METAL CAVITY WITH A NUMBER OF HOLES and a wave agitator inside the cavity to RANDOMIZE
%   the ELECTROMAGNETIC WAVE PATTERNS FOR MICROWAVE IMAGING [6]. (p. 2)
% - Later, the same research group fabricated
%   a METALIZED CAVITY WITH HOLES arranged in irises following Fibonacci patterns to CODE
%   the OUTGOING WAVES BASED ON FREQUENCY DIVERSITY [7]. (p. 2)
% - The FABRICATION OF HOLEY CAVITIES IS GENERALLY SIMPLER THAN THAT OF METAMATERIALS, and
%   they do not require any alteration in the sensing array assortment in contrast to the approach adopted in [5]. (p. 2)
% - Other types of ultrasound cavities have also been proposed by Fink et al. to create images using time-reversal techniques [26–29].
%
%   references from: article:Ghanbarzadeh-DagheyanSensors2018, article:FromentezeOptExp2017
%   microwave imaging:
%   (x)		   article:FromentezeOptExp2017: Computational polarimetric microwave imaging
%   (x)  [6], [23] article:FromentezeApplPhysLett2015: Computational imaging using a mode-mixing cavity at microwave frequencies
%	 [7]	   article:YurdusevenIMWCLett2016: Printed aperiodic cavity for computational and microwave imaging
%	      [22] article:CarsenatIAWPLett2012: Uwb antennas beamforming using passive time-reversal device
%	      [24] article:FromentezeIACCESS2016: Single-shot compressive multiple-inputs multiple-outputs radar imaging using a two-port passive device
%
% article:FromentezeOptExp2017: Computational polarimetric microwave imaging (microwave imaging)
% 1. Introduction
% - The potential for efficient, cost-effective, and high-resolution systems that can achieve fast acquisition rates have recently been demonstrated in
%   computational imaging systems based on
%   [1.)] CAVITY-BACKED [22–24] and
%   [2.)] METASURFACE [25–30] apertures. (p. 3)
% 5. Conclusion
% - A POLARIMETRIC MICROWAVE IMAGING TECHNIQUE has been presented in this paper by extending
%   the computational principle of the scalar approaches that have been previously developed in the literature. (p. 17)
% - The LEAKY MULTI-MODAL CAVITY USED TO PRODUCE PSEUDO-ORTHOGONAL FIELD PATTERNS IN FREQUENCY AND POLARIZATION can be generalized to
%   many metasurface aperture paradigms — all of which are capable of generating
%   FIELD PATTERNS WITH LOW CORRELATION AS A FUNCTION OF FREQUENCY or other parameters. (pp. 17, 18)
% - In the present example, measurements were taken between two ports of a multi-modal cavity over the band 17.5 - 26.5 GHz. [K-band] (p. 18)
% - From the FREQUENCY-INDEX MEASUREMENTS, it was possible to reconstruct targets made of copper wires forming the word "DUKE",
%   demonstrating the different modes of operation permitted by the proposed polarimetric approach. (p. 18)
% article:FromentezeApplPhysLett2015: Computational imaging using a mode-mixing cavity at microwave frequencies (microwave imaging)
leaky reverberant cavities
\cite{article:FromentezeOptExp2017,article:FromentezeApplPhysLett2015}, form
% 3.) virtual transceiver arrays
% article:KruizingaSciAdv2017: Compressive 3D ultrasound imaging using a single sensor
% RESULTS / Ultrasound wave field diversity using a coded aperture
% - By MODELING THE APERTURE MASK AS A COLLECTION OF POINT SENSORS, each having different transmit/receive delays,
%   the MASK CAN STILL BE REGARDED AS A SENSOR ARRAY. (p. 3)
virtual transceiver arrays.
% c) highly dispersive customized apertures form virtual transceiver arrays that expand excitations at sufficiently different frequencies into distinct spatial codes and mix the scattered waves in reception
% article:Ghanbarzadeh-DagheyanSensors2018: Holey-Cavity-Based Compressive Sensing for Ultrasound Imaging
% 1. Introduction
% - More specifically, the natural reverberation of the wave propagation inside the medium emulates random illuminations of a scene at different frequencies.
% - These systems exploit the conversion between spatial and spectral degrees of freedom, which are
%   also at the core of the time reversal technique in complex media for spatio-temporal focusing with a single broadband antenna. [29,30]
% article:KruizingaSciAdv2017: Compressive 3D ultrasound imaging using a single sensor
% RESULTS / Ultrasound wave field diversity using a coded aperture
% - However, all these point sensor signals are SUBSEQUENTLY SUMMED BY THE PIEZO SENSOR just after they have passed though the mask, resulting in
%   a SINGLE COMPRESSED MEASUREMENT, as depicted in the right panel of Fig. 1
%   (for a further comparison between a sensor array and a single sensor with a coding mask, see text S2 and figs. S1 and S2). (p. 3)
% article:FromentezeOptExp2017: Computational polarimetric microwave imaging (microwave imaging)
% 1. Introduction
% - These systems RADIATE PSEUDO-ORTHOGONAL FIELD DISTRIBUTIONS IN TRANSMISSION and - by exploitation of the reciprocity principle - IN RECEPTION,
%   to MULTIPLEX INFORMATION AND RECONSTRUCT AN IMAGE. (p. 3)
These expand
% 1.) excitations
excitations at
% 2.) sufficiently different frequencies
sufficiently different frequencies into
% 3.) distinct spatial codes
distinct spatial codes, i.e.
% 4.) spatially erratic incident fields
% TODO: complex erratic spatiotemporal interference patterns
spatially erratic incident fields, and, reciprocally, mix
% 5.) scattered fields
the scattered fields in
% 6.) reception
reception, providing
% 7.) frequency-diverse projections
frequency-diverse projections of
% 8.) field-of-view (FOV)
the \ac{FOV}.
% c)
The knowledge of
% 1.) two-way transfer function
the two-way transfer function enables
% 2.) image recovery
the image recovery.
% d)
% article:TondoYoyaApplPhysLett2017: Computational passive imaging of thermal sources with a leaky chaotic cavity
% article: Single-sensor multispeaker listening with acoustic metamaterials
% article:LiutkusSciRep2014: Imaging With Nature: Compressive Imaging Using a Multiply Scattering Medium
% - In the past few years, SEVERAL HARDWARE IMPLEMENTATIONS capable of performing such random compressive sampling were introduced [5–13]. (p. 1)
% - In optics, these include the SINGLE PIXEL CAMERA [6], which is depicted in Fig. 1(b), and
%   uses a digital array of micromirrors (abbreviated DMD) to sequentially reflect different random portions of the object onto a single photodetector. (p. 1)
% - Other approaches include phase modulation with a SPATIAL LIGHT MODULATOR [10] or a ROTATING OPTICAL DIFFUSER [13]. (p. 1)
% - The idea of RANDOM MULTIPLEXING for imaging has also been considered in other domains of wave propagation. (p. 1)
% - CS holds much promise in areas where detectors are rather complicated and expensive such as the THz or far infrared. (p. 1)
% - In this regards, there have been proposals to implement CS imaging procedures in the THz using random pre-fabricated masks [5],
%   DMD or SLM photo-generated contrast masks on semi-conductors slabs [14] and efforts are also pursued on TUNABLE METAMATERIAL REFLECTORS [15]. (p. 1)
% - Recently, a carefully engineered metamaterial aperture was used to generate complex RF beams at different frequencies [8]. (p. 1)
% - However, these CS implementations come with some limitations. (p. 1)
% - First, these devices include carefully engineered hardware designed to achieve randomization, via a DMD [6], a metamaterial [8] or a coded aperture [11].
% - Second, the acquisition time of most implementations can be large because they require the sequential generation of a large number of random patterns. (p. 1)
% Using natural complex media as random sensing devices
% - In essence, the COMPLEX MEDIUM ACTS AS A HIGHLY EFFICIENT ANALOG MULTIPLEXER FOR LIGHT, with
%   an input-output response characterized by its transmission-matrix [24,25]. (p. 3)
% single pixel camera
% article:RogersJAP2017: Demonstration of acoustic source localization in air using single pixel compressive imaging
% randomness in receive mode: (random multiplexing)
%
% optical imaging:
% 1.) article:DuarteISPM2008: Single-pixel imaging via compressive sampling (x)
%
% terahertz imaging:
% 1.) article:LiutkusSciRep2014: Imaging With Nature: Compressive Imaging Using a Multiply Scattering Medium (x, type of wave?)
% 2.) article:ShrekenhamerOptExp2013: Terahertz single pixel imaging with an optically controlled dynamic spatial light modulator
% 3.) article:WillettOptEng2011: Compressed sensing for practical optical imaging systems: A tutorial (type of wave?)
% 4.) article:ChanApplPhysLett2008: A single-pixel terahertz imaging system based on compressed sensing (x)
%
% microwave imaging
% 1.) 14. L. Wang, L. Li, Y. Li, H. C. Zhang, T. J. Cui, Single-shot and single-sensor high/ super-resolution microwave imaging based on metasurface. Sci. Rep. 6, 26959 (2016).
%
% thermal imaging w/ leaky reverberant cavity:
% 1.) article:TondoYoyaApplPhysLett2017: Computational passive imaging of thermal sources with a leaky chaotic cavity
% - In this article, we demonstrate passive imaging of thermal sources in the X-band frequency regime using a leaky chaotic cavity attached to two ports. (p. 2)
%
% acoustic position detection:
% 1.) article:RogersJAP2017: Demonstration of acoustic source localization in air using single pixel compressive imaging
% - Despite the range of approaches and results for electromagnetic imaging, VERY FEW STUDIES HAVE ATTEMPTED SINGLE-PIXEL IMAGING IN THE ACOUSTIC REGIME. (p. 2)
% 16. N Huynh, E. Zhang, M. Betcke, S. R. Arridge, P. Beard, B. Cox, A real-time ultrasonic field mapping system using a Fabry Pérot single pixel camera for 3D photoacoustic imaging, in Proceedings of SPIE Volume 9323, (International Society for Optics and Photonics, 2015), pp. 93231O.
%
% R. F. Marcia, Z. T. Harmany, and R. M. Willett, “Compressive coded aperture imaging,” Proc. SPIE 7246, 72460G (2009).
% [29] J. Ke and M. Neifeld, “Optical architectures for compresssive imaging,” Appl. Opt., vol. 46, pp. 5293–5303, Aug. 2007.
Unlike
% 1.) random sampling
% article:BessonITUFFC2018: Ultrafast Ultrasound Imaging as an Inverse Problem: Matrix-Free Sparse Image Reconstruction
% VI. RESULTS:COMPRESSED BEAMFORMING / A. Deep Dive Into Coherence
% - Four different sampling strategies are compared:
%   [1.)] uniform selection of transducer elements,
%   [2.)] random selection of transducer elements,
%   [3.)] CMIX, and
%   [4.)] CTMIX. (p. 347)
% - Regarding CMIX and CTMIX, the matrix W is generated with coefficients distributed according to normal and Rademacher distributions. (p. 347)
% - Fig. 6(a) and (b) shows the mutual coherence μ( H_{d} \Psi ) for
%   a number of measurements ranging between 5% and 100%, where H_{d} is square in the case of 100 %, with
%   \Psi being Dirac and Haar bases, respectively. (p. 347)
% - It can be seen that the main benefit of the CMIX and CTMIX strategies reside in their ability to limit
%   the increase of the coherence μ( H_{d} \Psi ) induced by the undersampling of the raw data. (p. 347)
% - In addition, it can be noticed that this effect is more pronounced for the Haar basis than for the Dirac basis. (p. 347)
% - A comparison between CMIX and CTMIX shows that CTMIX has lower coherence than CMIX. (p. 347)
% - This is expected, since CTMIX provides a better mixing than CMIX. (p. 347)
% - Regarding the impact of the probability distribution of the random coefficients on μ( H_{d} \Psi ), Fig. 6(c) shows that
%   there is no significant difference in coherence between Gaussian and Rademacher random coefficients. (p. 347)
% - Regarding CTMIX, Fig. 6(d) shows the values of μ( H_{d} \Psi ) for the different values of D_{t}. (p. 347)
% - It can be seen that the mutual coherence decreases when D_{t} increases. (p. 347)
% - Fig. 6. MUTUAL COHERENCE μ( H_{d} \Psi ) AGAINST THE NUMBER OF MEASUREMENTS for
%   (a) Dirac basis and (b) Haar basis for
%   the uniform selection of transducer elements, the random selection of transducer elements, CMIX, and CTMIX (D_{t} = 10).
%   In addition, the coherence is evaluated for
%   (c) CMIX and CTMIX with mixing coefficients drawn using normal and Rademacher distributions and
%   (d) CTMIX at different depths. (p. 347)
% VI. RESULTS:COMPRESSED BEAMFORMING / B. Reconstruction of the Point-Reflector Phantom
% - It can be concluded that CMIX and CTMIX achieve high-quality reconstructions for all the considered number of measurements. (p. 348)
% - It can also be observed that, in the case of CMIX, the PSNR drops for numbers of measurements lower than 5 %, whereas
%   it remains constant for CTMIX at lower numbers of measurements. (p. 348)
% VI. RESULTS:COMPRESSED BEAMFORMING / E. Computation Times for Compressed Beamforming
% - In addition, the CB method coupled with CMIX and CTMIX is no longer matrix-free, since the random matrices have to be stored in memory. (p. 349)
% VII. DISCUSSION / B. Toward Compressed Sensing in Ultrasound Imaging
% - However, we face ONE MAJOR OBSTACLE, which is the HIGH COHERENCE OF THE MEASUREMENT OPERATOR. (p. 350)
% - CMIX and CTMIX strategies manage to maintain the coherence constant when the number of measurements is decreased, but
%   the COHERENCE REMAINS HIGH RELATIVE TO THE CORRESPONDING WELCH BOUND because of
%   THE COHERENCE INTRINSIC TO THE MEASUREMENT MODEL. (p. 350)
% - Indeed, the high mutual coherence of the measurement model comes from the fact that
%   each projection onto a 1-D-conic, implied by the time-of-flight calculations, involves
%   only a few points in the desired-image space. (p. 350)
% - The natural question that one may ask is whether it is possible to change the nature of the projections in order to involve more points. (p. 350)
% - In fact, this is a relatively difficult task,
%   since projections are a consequence of the expression for the round-trip time-of-flight of US waves in a homogeneous medium. (p. 350)
% - This observation has a deep impact on the design of CS acquisition schemes. (p. 350)
% - Indeed, it means that the attempts to decrease the coherence of the measurement model by playing with
%   random pulses sent in transmit are hopeless, because this does not change the expression for
%   the time-of-flight and thus the fact that echo-samples stem from projections onto 1-D-conics. (p. 350)
% 	- One solution to this issue may reside in dealing with the coherent operator, by exploring
%   	  a recent topic on CS, denoted “constrained adaptive sensing,” which derives sampling theorems and variants of [35, Th. 3.1] where
%   	  the measurement matrix is more constrained than in standard CS [50]. (p. 350)
% 	- Another alternative could be to entirely rethink the measurement process, starting from the requirements of [35, Th. 3.1]. (p. 350)
%	- In terms of acoustic propagation, a random measurement model implies that each sample of
%	  the element raw-data receives contributions from points spread over the entire image space. (p. 350)
%	- In other words, it means that the duality between time and depth, which is at the heart of US imaging, is not valid anymore, since echoes from points at different positions reach the transducer elements at the same time. (p. 350)
%	- Thus, a random measurement model H is unfeasible, in pulse-echo imaging in a homogeneous medium, due to the fact that US waves respect the Helmholtz equation. (p. 350)
%	- One way to address such an issue resides in placing a scattering or heterogeneous medium in front of the US probe. (p. 350)
%	- This principle has been recently developed in optics and gives promising results [51]. (p. 350)
%	- However, such a new design raises many questions regarding the choice and the modeling of the heterogeneous medium. (p. 350)
%	- It can be easily understood that such a medium may not generate a purely random matrix but will be somewhere inbetween
%	  a purely random case and the highly coherent case of the homogeneous medium. (p. 350)
%	- This topic is currently under study and will be the object of further reports or communications. (p. 350)
% proc:BessonICIP2016: Compressed delay-and-sum beamforming for ultrafast ultrasound imaging
the random sampling
\cite{article:BessonITUFFC2018,proc:BessonICIP2016,article:DavidJASA2015} or
% 3.) mixing
mixing
\cite{article:BessonITUFFC2018,article:LiutkusSciRep2014} of
% 4.) scattered waves
the scattered waves, which reduce
the number of
transceivers and measurements,
% 5.)
the random waves improve
%
the conformity with
condition (ii).

% d)
% phased arrays?
% reduce the number of sequential measurements
%Fully-controlled phased arrays, in contrast, increase
%the flexibility at
%higher complexity and costs.


% c) fully-controlled phased arrays require complex systems but are more flexible and reduce the number of sequential measurements
% article:Ghanbarzadeh-DagheyanSensors2018: Holey-Cavity-Based Compressive Sensing for Ultrasound Imaging
% 1. Introduction
% - Specifically for ultrasound imaging,
%   Schiffner introduced a SOFTWARE-BASED TECHNIQUE that uses time delays and apodization weights to generate random incident acoustic fields [18,19]. (p. 2)
% article:ChengIACCESS2017: Near-Field Millimeter-Wave Phased Array Imaging With Compressive Sensing
% ABSTRACT
% - Phased array technology allows for FAST ELECTRONIC BEAM STEERING with high antenna gain for radar imaging systems. (p. 18975)
% - A novel data acquisition methodology has been proposed on the basis of NEAR-FIELD FOCUSING TECHNIQUES. (p. 18975)
% - CS measurements are taken by RANDOMLY FOCUSING BEAMS IN THE NEAR-FIELD REGION. (p. 18975)
% I. INTRODUCTION
% - By integrating the CS theory into the imaging algorithm, ONLY A SMALL NUMBER OF RANDOM SAMPLES ARE NEEDED FOR RECONSTRUCTION. (p. 18976)
% - Our goal is to recover the reflectivity information of the target region from the reflected data. (p. 18976)
% II. NEAR-FIELD PHASED ARRAY IMAGING WITH COMPRESSIVE SENSING / B. PROPOSED NEAR-FIELD IMAGING METHOD
% - Alternatively, we adopt the near-field focusing technique and make sure the sampling points are evenly distributed in the ROI. (p. 18977)
% - In comparison to the far-field focusing method, this scheme can focus at different depths in the ROI and thus offers much more information in 3-D imaging applications. (p. 18977)
% II. NEAR-FIELD PHASED ARRAY IMAGING WITH COMPRESSIVE SENSING / C. COMPRESSIVE SENSING IMPLEMENTATION
% - With CS theory, samplings in the spatial domain and the frequency domain can be greatly reduced while satisfactory reconstruction can still be achieved. (p. 18978)
% - Mathematically, we use matrix A as the undersampling operator. (p. 18978)
% IV. CONCLUSION
% - A new scanning method has also been provided to focus array beams at different spots with various depths such that they can cover the target region with less redundancy. (p. 18984)
% - With the CS theory, far fewer spatial samples are required during data acquisition than FT based methods. (p. 18984)
% article:LipworthJOSAA2013: Metamaterial apertures for coherent computational imaging on the physical layer (microwave imaging)
% 1. INTRODUCTION
% - A PHASED-ARRAY SYSTEM FULFILLS THE DESCRIPTION OF A RECONFIGURABLE APERTURE and can, in principle, provide
%   a LIMITLESS SET OF MEASUREMENT MODES. (p. 1604)
% - However, the drawbacks that have inhibited phased-array prevalence in applications are
%   its significant COST, WEIGHT, AND POWER REQUIREMENTS for implementing the required number of SOURCES, PHASE SHIFTERS, AND ASSOCIATED AMPLIFIERS CIRCUITRY to generate
%   N measurement modes. (p. 1604)
% - [27] A. Massa, P. Rocca, and G. Oliveri, ‘‘Compressive sensing in electromagnetics—A review,’’ IEEE Antennas Propag. Mag., vol. 57, no. 1, pp. 224–238, Feb. 2015.
% article:ChengIACCESS2016: Compressive Millimeter-Wave Phased Array Imaging
% 13. Molaei, A.; Juesas, J.H.; Lorenzo, J.A.M. Compressive Reflector Antenna Phased Array. In Antenna Arrays and Beam-Formation; InTech: London, UK, 2017.

% TODO: acquisition time?

% 1.5-D Sparse Array for Millimeter-Wave Imaging Based on Compressive Sensing Techniques 10.1109/TAP.2018.2800531
% Coded aperture ptychography: uniqueness and reconstruction (Pengwen Chen1 and Albert Fannjiang2,3 Published 10 January 2018 • 2018 Inverse Problems, Volume 34, Number 2)

%   [2] D. J. Brady, K. Choi, D. L. Marks, R. Horisaki, and S. Lim, “Compressive holography,” Opt. Express 17, 13040–13049 (2009).
%   [3] C. F. Cull, D. A. Wikner, J. N. Mait, M. Mattheiss, and D. J. Brady, “Millimeter-wave compressive holography,” Appl. Opt. 49, E67–E82 (2010).

% Subwavelength diffractive acoustics and wavefront manipulation with a reflective acoustic metasurface, https://doi.org/10.1063/1.4967738

% Single-frequency 3D synthetic aperture imaging with dynamic metasurface antennas https://doi.org/10.1364/AO.57.004123 (x)
% Low-cost three-dimensional millimeter-wave holographic imaging system based on a frequency-scanning antenna https://doi.org/10.1364/AO.57.000A65 ->

% W-band sparse synthetic aperture for computational imaging https://doi.org/10.1364/OE.24.008317
% Far-field imaging beyond diffraction limit using single sensor in combination with a resonant aperture https://doi.org/10.1364/OE.23.000401 (x)
% Single-frequency microwave imaging with dynamic metasurface apertures
% X-band SAR imaging with a liquid-crystal-based dynamic metasurface antenna https://doi.org/10.1364/JOSAB.34.000300
% Design considerations for a dynamic metamaterial aperture for computational imaging at microwave frequencies https://doi.org/10.1364/JOSAB.33.001098
% Cavity-backed metasurface antennas and their application to frequency diversity imaging https://doi.org/10.1364/JOSAA.34.000472

%---------------------------------------------------------------------------------------------------------------
% 2.) Kruizinga et al.
%---------------------------------------------------------------------------------------------------------------
% a) Kruizinga et al. equipped a single transducer with a plastic delay mask to enable compressive three-dimensional UI
% article:Ghanbarzadeh-DagheyanSensors2018: Holey-Cavity-Based Compressive Sensing for Ultrasound Imaging
% 1. Introduction
% - Similar to CODED MASKS that are commonly used in COMPRESSIVE OPTICAL IMAGING [22–24] as a subgroup of hardware-based methods,
%   Kruizinga et al. introduced a ROTATING MASK OF RANDOMLY-VARYING THICKNESSES throughout its surface to randomize ultrasound waves, and
%   they were able to retrieve 3D images of objects using a single transducer [25]. (p. 2)
% article:KruizingaSciAdv2017: Compressive 3D ultrasound imaging using a single sensor
% Abstract
% - In this spirit, we have designed a simple ultrasound imaging device that can perform
%   THREE-DIMENSIONAL IMAGING USING JUST A SINGLE ULTRASOUND SENSOR. (p. 1)
% - Our device makes a COMPRESSED MEASUREMENT OF THE SPATIAL ULTRASOUND FIELD using
%   A PLASTIC APERTURE MASK PLACED IN FRONT OF THE ULTRASOUND SENSOR. (p. 1)
% - The need for just one sensor instead of thousands paves the way for
%   CHEAPER, FASTER, SIMPLER, AND SMALLER SENSING DEVICES and possible new clinical applications. (p. 1)
% INTRODUCTION
% - A SUCCESSFUL STRATEGY for obtaining these compressed measurements is to apply a so-called “CODED APERTURE” (17–19). (p. 2)
% - In this case, the INFORMATION — whether it entails different light directions, different frequency bands, or
%   any other information that is conventionally needed to build up an image — IS CODED INTO
%   THE AVAILABLE APERTURE AND ASSOCIATED MEASUREMENT. (p. 2)
% - Smart algorithms are then needed to decode the retrieved measurements to form an image. (p. 2)
% - Along these lines, here we introduce for the first time in ultrasound imaging
%   a SIMPLE COMPRESSIVE IMAGING DEVICE that can create 3D images. (p. 2)
% - Our device contains ONE LARGE PIEZO SENSOR that transmits an ultrasonic wave through A SIMPLE PLASTIC CODING MASK. (p. 2)
% - Using our device, WE AIM TO MITIGATE THE HARDWARE COMPLEXITY associated with conventional 3D ultrasound. (p. 2)
% - The manufacturing costs of our device will be much lower. (p. 2)
% - A simple, cheap, single-element transducer is used, and the plastic coding mask can be produced for less than a euro. (p. 2)
% - These lower costs enable broader use of these 3D compressive imaging devices, for example, for long-term patient monitoring. (p. 2)
% - Because the imaging is 3D, finding and maintaining the proper 2D view does not require a trained operator. (p. 2)
% - One could also envision other applications, such as minimally invasive imaging catheters that are too thin to accommodate
%   the hundreds or thousands of electrical wires currently needed for 3D ultrasound imaging. (p. 2)
% DISCUSSION
% - By contrast [2D LOCALIZATION WORK by Clement et al], we propose to induce signal diversity using
%   LOCAL DELAYS IN BOTH TRANSMIT AND RECEIVE. (p. 8)
% - This type of “wave field coding” can be applied to many other imaging techniques and seems to offer
%   a much higher dynamic range than the slowly varying frequency-dependent wave fields. (p. 8)
% - As a result, we are able to move beyond the localization of isolated point scatterers but perform actual 3D imaging as we show in this paper. (p. 8)
% - We believe that this technique will pave the way for an entirely new means of imaging in which the complexity is shifted away from
%   the hardware side and toward the power of computing. (p. 8)
% proc:VanDerMeulenACSSC2017: Spatial compression in ultrasound imaging
\name{Kruizinga} \emph{et al.} \cite{article:KruizingaSciAdv2017} equipped
a single transducer with
a plastic delay mask to enable
compressive three-dimensional \ac{UI} with
cheap and simple hardware.
% b) mask introduced random time delays into both the emitted and the received waves to decorrelate the pulse echoes received from distinct voxels
% article:KruizingaSciAdv2017: Compressive 3D ultrasound imaging using a single sensor
% Abstract
% - The APERTURE MASK ensures that every pixel in the image is UNIQUELY IDENTIFIABLE in the compressed measurement. (p. 1)
% - We demonstrate that this device can successfully image TWO STRUCTURED OBJECTS placed in water. (p. 1)
% INTRODUCTION
% - Local variations in the mask thickness (Fig. 1C) cause local delays, which scrambles the phase of the wave field. (p. 2)
% - This enables a complex interference pattern to propagate inside the volume,
%   removing ambiguity among echoes from different pixels, as illustrated in Fig. 2. (p. 2)
% - The interference pattern
%   propagates through the medium,
%   scatters from objects within the medium, and then
%   propagates back through the coding mask onto the same ultrasound sensor, providing
%   a single compressed ultrasound measurement of the object. (p. 2)
% RESULTS / Ultrasound wave field diversity using a coded aperture
% - After propagating through the mask, the pulse length increases further due to the distortion of the wavefront by the coding mask (Fig. 4A). (p. 3)
% - These waves will propagate spherically in the medium and will interfere constructively and destructively,
%   a process that stretches out longer in time than an undistorted wavefront does. (p. 3)
% - Consequently, the available spatial bandwidth increases compared to an undistorted burst by a mask-less sensor. (p. 3)
% - This effect is highlighted in Fig. 4B, which shows two spatial frequency spectra: one at the beginning of the pulse and one spectrum a few microseconds later. (p. 3)
% - The delays produced by the mask create complex spatiotemporal interference patterns that ensure that each pixel generates a unique temporal signal in the compressed measurement. (pp. 3, 4)
% - This unique pixel signature enables direct imaging without the need for uncompressed spatial measurements. (p. 4)
% - Because of the limitations in temporal bandwidth, as well as in mask thickness distribution, it is not guaranteed that all pixel echo signals are uncorrelated. (p. 4)
% - To introduce more diversity between the pixels, we chose to rotate the mask in front of the sensor such that the interference pattern is rotated along with it, thereby obtaining additional measurements containing new information. (p. 4)
% RESULTS / Pixel diversity
% - If we model the ultrasound field of the coded aperture by APPROXIMATING THE MASK APERTURE AS A COLLECTION OF POINT SOURCES AND SENSORS
%   (see Materials and Methods section), we can analyze the coding performance of the aperture. (p. 4)
% - To this end, we compute the pulse-echo signals for pixels in the central (xy plane) area at a fixed depth (z) relative to the sensor and
%   then compute their CROSS-CORRELATIONS. (p. 4)
% - Ideally, all signals are uncorrelated, so that any superposition of pulse-echo signals can be uniquely broken down into
%   its composing individual pulse-echo signals (as in Fig. 2), in which case the signals for each pixel can be unambiguously resolved. (p. 4)
% - Because the wave field phase without using a mask is highly uniform, most pulse-echo signals are highly correlated. (p. 4)
% - However, adding a mask to the single sensor causes the distribution to be zero-mean and removes the high correlations around +1 and −1. (p. 4)
% - As expected, these results illustrate that adding more measurements by rotating the mask causes correlations to be distributed more narrowly around zero;
%   pulse-echo signals become more orthogonal. (p. 4)
% - This tells us that the use of rotation removes ambiguities and consequently increases resolvability of scatterers. (p. 4)
% - Here, we do not include the signal correlations over the depth dimension, because generally speaking, the decorrelation in
%   the acoustic propagation direction comes naturally with the pulse-echo delay, or in other words, using only one sensor,
%   we can estimate the depth ofan object, but we cannot see wheth- er the object is located left or right from the sensor. (p. 4)
% - The CRLB is derived for the case of estimating the position of a single scatterer given a pulse-echo measurement with additive Gaussian noise
%   (see the Materials and Methods section for more details). (p. 5)
% - For a 3D problem, this results in estimating three unknowns (x, y,and z coordinates) from N measurements, where N is the total number of samples. (p. 5)
% - Although position estimation and imaging are not the same, one could argue that the best value obtained for positional error spread
%   (for example, 3 SD of error) nevertheless is indicative for the achievable imaging resolution. (p. 5)
% DISCUSSION
% - The plastic mask breaks the phase uniformity of the ultrasound field and causes every pixel to be uniquely identifiable within the received signal. (p. 8)
% - The compressed measurement contains the superimposed signals from all object pixels. (p. 8)
The varying thickness of
the mask introduced random time delays into both
the emitted and
the received waves to decorrelate
the pulse echoes received from
distinct voxels.
% c) sequential emission of these random waves at numerous angles of rotation permitted the recovery of three-dimensional sparse objects
% article:KruizingaSciAdv2017: Compressive 3D ultrasound imaging using a single sensor
% RESULTS / Ultrasound wave field diversity using a coded aperture
% - Here, we assume that most of the acoustic energy is directly transmitted through the coding mask;
%   yet, in reality, some of the energy will be reflected back from the mask-water interface and possibly redirected in
%   the medium through the mask in a second pass after bouncing back from the mask-transducer interface, thereby increasing the spatial variability even more. (p. 3)
% RESULTS / Constructing the system matrix H
% - For computing the column signals of H, it is crucial to KNOW THE ENTIRE SPATIOTEMPORAL WAVE FIELD. (p. 8)
% - To this end, we adopt a SIMPLE CALIBRATION MEASUREMENT in which we SPATIALLY MAP THE IMPULSE SIGNAL
%   (using a small hydrophone and a translation stage) in a plane close to the mask surface and perpendicular to the ultrasound propagation axis (Fig. 6C). (p. 8)
% - This recorded wave field can then be propagated to any plane that is parallel to the recorded plane using the angular spectrum approach (28, 29). (p. 8)
% - Note that THIS PROCEDURE IS REQUIRED ONLY ONCE AND IS UNIQUE FOR THE SPECIFIC SENSOR AND CODING MASK. (p. 8)
% - According to the reciprocity theorem, we are able to obtain the pulse-echo signal by performing an autoconvolution of the propagated hydrophone signal with itself (30). (p. 8)
% - Using these components, we can then compute the pulse-echo ultrasound signal for every point in 3D space and subsequently populate the columns of H. (p. 8)
% DISCUSSION
% - Our device has disadvantages and limitations and does not deliver similar functionality as existing 3D ultrasound arrays. (p. 8)
% - In addition, the mechanical rotation of the mask is expected to introduce a fair amount of error in the prediction of the ultrasound field and limits
%   the usability when it comes down to imaging moving tissue or the application of ultrasound Doppler. (p. 8)
% - Besides, a rotating wave field becomes less effective at pixels closer to the center of rotation— assuming a spatially uniform generation of spatial frequencies. (p. 8)
% - Future systems should consider the incorporation of other techniques to introduce signal variation, such as
%   controlled linearized motion, or a mask that can be controlled electronically, similar to
%   the digital mirror devices used in optical compressive imaging (13). (p. 8)
% - Furthermore, the angular spectrum approach method that predicts the ultrasound field inside the medium is not flawless in cases where
%   the medium contains strong variations in sound speed and density. (p. 8)
% - These issues may possibly be solved, however, by incorporating more advanced ultrasound field prediction tools that
%   can deal with these kinds of variations (38). (p. 8)
% - This will increase the computational burden of the image reconstruction even further but may potentially lead to better results than what we have shown here. (p. 8)
%
% - Another important distinction is the SMALL IMPEDANCE DIFFERENCE BETWEEN OUR CODING MASK AND IMAGING MEDIUM (less than a factor of 2). (p. 8)
% - As a consequence, there is very little loss of acoustic energy as compared to reverberant cavities that inherently require
%   high impedance mismatch to ensure signal diversity (33, 34). (p. 8)
% MATERIALS AND METHODS / Acoustic calibration procedure
% - This calibration step is an essential component in this kind of compressive imaging (40). (p. 9)
% - The better H is known, the better the reconstruction of the image. (p. 9)
% - Note that this calibration is sensor- and mask- specific and, in principle, only needs to be done once. (p. 9)
Adopting
a simple calibration procedure, which measured
the random sound field in
water and, thus, limited
its range of
validity,
% article:KruizingaSciAdv2017: Compressive 3D ultrasound imaging using a single sensor
% RESULTS / Signal model and image reconstruction
% - For the full 3D reconstruction shown in Fig. 6F, we made use of the sparsity of
%   these two letters in water by applying the sparsity-promoting basis pursuit denoising (BPDN) algorithm (27). (p. 6)
% - As can be seen, this prior knowledge about the image could be effectively exploited to improve image quality,
%   significantly improving the dynamic range from 15 to 40 dB. (p. 6)
% - Fig. 6. Compressive 3D ultrasound imaging using a single sensor. [...]
%   The images shown in (B) and (D) were obtained using 72 EVENLY SPACED MASK ROTATIONS, and
%   the full 3D image in (F) was obtained using only 50 EVENLY SPACED ROTATIONS to reduce the total matrix size. (p. 7)
% article:KruizingaSciAdv2017: Compressive 3D ultrasound imaging using a single sensor
% RESULTS / Signal model and image reconstruction
% - Thus, instead of applying a standard geometric operation to multiple sensor observations,
%   we attempt to explain the received signal as a linear combination of point scatterer echo signals [23, 24]. (p. 5)
the recovery of
sparse objects required
$50$ sequential pulse-echo measurements at
evenly spaced angles of
rotation.
%---------------------------------------------------------------------------------------------------------------
% 2.) Ghanbarzadeh-Dagheyan et al.
%---------------------------------------------------------------------------------------------------------------
% a) Ghanbarzadeh-Dagheyan et al. optimized a holey cavity with respect to its opening sizes and the materials to enable compressive two-dimensional UI with only a few transceivers
% article:Ghanbarzadeh-DagheyanSensors2018: Holey-Cavity-Based Compressive Sensing for Ultrasound Imaging
% Abstract
% - In this work,
%   A TWO-DIMENSIONAL CAVITY, having MULTIPLE OPENINGS, is used to perform such a COMPRESSION FOR ULTRASOUND IMAGING. (p. 1)
% - Moreover,
%   COMPRESSIVE SENSING TECHNIQUES are used for SPARSE SIGNAL RETRIEVAL with
%   A LIMITED NUMBER OF OPERATING TRANSCEIVERS. (p. 1)
% - In addition, an analysis of
%   the SENSING CAPACITY and
%   the SHAPE OF THE POINT SPREAD FUNCTION is also carried out for the aforementioned cases. (p. 1)
% - The cavity is designed to have
%   the MAXIMUM SENSING CAPACITY GIVEN DIFFERENT MATERIALS AND OPENING SIZES. (p. 1)
% 1. Introduction
% - In this study, a STATIC 2D CAVITY has been selected as a structure that enables
%   RANDOMIZATION OF THE WAVE FIELDS IN THREE GENERALIZED DIMENSIONS,
%   one spectral and two spatial, thus leading to enhanced ultrasound imaging via compressive sensing. (p. 2)
% - In the succeeding sections,
%   the performance of several 2D holey cavities are studied, showing how the cavity-based imaging performance is enhanced when compared to that of
%   a traditional ultrasound imaging setup. (p. 2)
% 2. Two-Dimensional Cavity
% - As observed, the CAVITY IS CLOSED FROM ALL SIDES EXCEPT THE BOTTOM, where a number of openings is made for the impinging waves to pass through. (p. 2)
% - It is assumed that the OPENINGS ARE UNIFORMLY DISTRIBUTED ALONG THE BOTTOM OF THE CAVITY, and they are symmetric with respect to the y axis. (p. 2)
% 3. Compressive Sensing, Imaging and Performance Metrics
% - Moreover, the SENSING CAPACITY and the POINT SPREAD FUNCTION (PSF) of the system are defined, and
%   they will be used as extra metrics to assess how the ADDITION OF THE CAVITY AFFECTS THE PERFORMANCE OF THE IMAGING SYSTEM. (p. 3)
% 3.5. Sensing Capacity
% - Another metric that will be used to assess the performance of the imaging system is the so-called SENSING CAPACITY. (p. 7)
% - This metric determines the amount of information that can be transferred from the imaging domain into the sensors; and
%   the larger the SENSING CAPACITY is, the better the image reconstruction will be. (p. 7)
% 4. Simulation Results and Discussion / 4.1. The Effect of the Cavity Design on Sensing Capacity
% - In this study, MAXIMIZING THE SENSING CAPACITY was selected as the design goal, and
%   TWO PARAMETERS OF THE CAVITY WERE ADJUSTED to achieve this end:
%   [1.)] the OPENING SIZE and
%   [2.)] the MATERIAL OF THE CAVITY. (p. 8)
% 4. Simulation Results and Discussion / 4.1. The Effect of the Cavity Design on Sensing Capacity / 4.1.1. The Size of the Openings
% - To have the ability to sample all the cavity modes,
%   the SIZE OF THE OPENINGS IN THE CAVITY NEEDS TO BE SMALLER THAN THE MINIMUM GUIDED WAVELENGTH [6]. (p. 8)
% - On the other hand, the HARDSHIPS IN FABRICATION AND MICROMACHINING SET A LIMIT ON HOW SMALL THE OPENINGS CAN BE. (p. 8)
% - Eight different cases are studied, in which the cavity thickness and its material (STEEL) are kept constant as
%   the size of the holes at the bottom of the cavity was changed. (p. 8)
% - The NUMBER OF HOLES IS MAXIMIZED in each case by fitting as many openings as possible at the bottom of the cavity,
%   with the assumption that d_{o} = d_{b}. (p. 8)
% - With this setup, the largest opening size (λmin) has resulted in the largest sensing capacity among other opening sizes, and at the same time,
%   it is the easiest to fabricate in terms of feature size. (p. 8)
% 4. Simulation Results and Discussion / 4.1. The Effect of the Cavity Design on Sensing Capacity / 4.1.2. Material Selection
% - Hence, it is of interest to inspect whether using a 3D printing material such as VeroWhitePlus, which is not as stiff and dense as STEEL,
%   can adequately randomize the wave fields for compressive sensing. (p. 9)
% - Furthermore, another material, ALUMINUM, is tested as the cavity material, and its effect on the sensing capacity is studied, alongside with that of STEEL. (p. 9)
% - Although the addition of the plastic cavity to the domain has increased the sensing capacity,
%   its effect is not as strong as that of the STEEL or the ALUMINUM cavity. (p. 9)
% - Since ALUMINUM is lighter than STEEL, it can be made into thin layers, and its effect in the cavity is close to that of STEEL,
%   so it WAS SELECTED AS THE MATERIAL FOR THE CAVITY. (p. 9)
% - It is easy to observe that the field patterns have a reduced correlation as a result of pseudo-random illumination of the scene. (p. 10)
% 5. Conclusions
% - In this work, a theoretical study of using SOLID CAVITIES ENCLOSING ULTRASOUND SOURCES TO RANDOMIZE THE MEASUREMENTS was presented. (p. 14)
% - Such a novel measurements scheme was combined with CS theory to retrieve the image of two objects using a REDUCED NUMBER OF TRANSMITTERS. (p. 14)
% - The novelty of this work is the introduction of spectral coding cavities into ultrasound imaging. (p. 14)
% - This work was limited by
%   [1.)] the SIMPLIFIED SIMULATION LAYOUT that considered a small two-dimensional imaging domain,
%   [2.)] a REDUCED NUMBER OF TRANSCEIVERS and
%   [3.)] a SMALL NUMBER OF TARGETS to reduce the computational burden and to keep the assumptions valid. (p. 14)
\name{Ghanbarzadeh-Dagheyan} \emph{et al.} \cite{article:Ghanbarzadeh-DagheyanSensors2018} optimized
a holey cavity with
respect to
its opening sizes and
the materials to enable
compressive two-dimensional \ac{UI} with
only a few transceivers.
% b) presence of the cavity significantly improved the lateral resolution of two point-like targets in a lossless homogeneous fluid
% article:Ghanbarzadeh-DagheyanSensors2018: Holey-Cavity-Based Compressive Sensing for Ultrasound Imaging
% Abstract
% - As a proof-of-concept of this theoretical investigation,
%   TWO POINT-LIKE TARGETS LOCATED IN A UNIFORM BACKGROUND MEDIUM ARE IMAGED IN
%   THE PRESENCE AND THE ABSENCE OF THE CAVITY. (p. 1)
% - It is demonstrated that
%   the USE OF A CAVITY, whether it is made of PLASTIC OR METAL, can significantly ENHANCE
%   THE SENSING CAPACITY and THE POINT SPREAD FUNCTION of a focused beam. (p. 1)
% - The imaging performance is also improved in terms CROSS-RANGE RESOLUTION [sic!] when compared to the no-cavity case. (p. 1)
% 4. Simulation Results and Discussion
% - As shown in Figure 1, a SMALL NUMBER OF TRANSCEIVERS (ONLY TWO) are considered here to show
%   the general concept and the imaging capability using compressive sensing. (p. 7)
% - What is more, TWO POINT-LIKE TARGETS ARE SELECTED TO BE IMAGED. (p. 7)
% - The imaging domain has a grid size of n_{x} = 501 and n_{y} = 120 in the x and y direction, in order, leading to
%   a vector signal size of P = 60120 elements. (p. 7)
% - The FREQUENCY BAND IS 2–10 MHz, and the frequency steps in the sweeping are 0.1 MHz. (p. 7)
% - The number of transmitters N_{T}, receivers N_{R}, and frequencies N_{f} used in the simulations is 2, 2 and 81, which yields
%   a total measurement number of M = N_{T} N_{R} N_{f} = 324. (pp. 7, 8)
% - It is lucid that these values for M and P make the system in (9) underdetermined since P >> M. (p. 8)
% 4. Simulation Results and Discussion / 4.2. The Effect of the Cavity on Imaging and Point Spread Function
% - The PSF of the imaging system IMPROVES CONSIDERABLY when the aluminum cavity was used. (p. 10)
% - Specifically, the CROSS-RANGE ALIASING EFFECTS ARE ELIMINATED, and
%   the CROSS-RANGE RESOLUTION IS ENHANCED; nevertheless, this enhancement is not as significant when
%   the plastic cavity is employed. (p. 10)
% - Figure 8 shows that the matrix A∗A, when normalized, is closest to I when the aluminum cavity used, thus leading to the best imaging of all configurations. (p. 11)
% 5. Conclusions
% - It was shown that the SENSING CAPACITY and the PSF of the focused beam were significantly IMPROVED WHEN A CAVITY, MADE OF ALUMINUM OR PLASTIC, WAS UTILIZED. (p. 14)
% - The recovered images of two point-like targets inside a uniform medium showed that the use of the CAVITY ENHANCES THE CROSS-RANGE RESOLUTION, but
%   they might still possess some weak artifacts when the SNR decays below 10 dB. (p. 14)
Encasing only two point-like transceivers that perform
a complete \ac{SA} acquisition sequence,
its presence significantly improved
the lateral resolution of
two point-like targets in
a lossless homogeneous fluid.
% c) time-reversal focusing
% article:Ghanbarzadeh-DagheyanSensors2018: Holey-Cavity-Based Compressive Sensing for Ultrasound Imaging
% 1. Introduction
% - Other types of ULTRASOUND CAVITIES have also been proposed by Fink et al. to create images using
%   TIME-REVERSAL TECHNIQUES [26–29]. (p. 2)
% - However, they have not been used in the scope of compressive sensing; and therefore,
%   THEIR METHOD REQUIRES A LARGE NUMBER OF MEASUREMENTS [25]. (p. 2)
% article:KruizingaSciAdv2017: Compressive 3D ultrasound imaging using a single sensor
% DISCUSSION
% - The calibration procedure that we use and our imaging device both share common ideas with
%   the TIME-REVERSAL WORK conducted by Fink and co-workers (30–34).
% - Time-reversal ultrasound entails the notion that any ultrasound field can be focused back to its source by reemitting
%   the recorded signal back into the same medium. (p. 8)
% - Consequently, the ULTRASONIC WAVES CAN BE FOCUSED ONTO A PARTICULAR POINT IN SPACE AND TIME if
%   THE IMPULSE SIGNAL OF THAT POINT IS KNOWN. (p. 8)
% - Using a REVERBERANT CAVITY TO CREATE IMPULSE SIGNAL DIVERSITY AND A LIMITED NUMBER OF SENSORS,
%   Fink et al. have shown that a 3D medium can be imaged by APPLYING TRANSMIT FOCUSING WITH RESPECT TO EVERY SPATIAL LOCATION. (p. 8)
% - Hence, they need as many measurements as there are pixels, resulting in an unrealistic scenario for real-time imaging. (p. 8)
%
%   references from: article:Ghanbarzadeh-DagheyanSensors2018, article:KruizingaSciAdv2017
%	 [29], [33] article:EtaixJASA2012: Acoustic imaging device with one transducer
%   (x)	       [31] article:MontaldoITUFFC2005: Building three-dimensional images using a time-reversal chaotic cavity
%	 [28]	    article:QuieffinJASA2004: Real-time focusing using an ultrasonic one channel time-reversal mirror coupled to a solid cavity
%	       [34] article:MontaldoApplPhysLett2004: Time reversal kaleidoscope: A smart transducer for three-dimensional ultrasonic imaging
%	 [27], [32] article:DraegerJASA1999: One-channel time-reversal in chaotic cavities: Experimental results
%        [26]	    article:DraegerPhysRevLett1997: One-channel time reversal of elastic waves in a chaotic 2D-silicon cavity
%	       [30] article:FinkITUFFC1992: Time reversal of ultrasonic fields. I. Basic principles
%
% - Besides the time-reversal work, our technique also shares some common ground with
%   the 2D LOCALIZATION WORK by Clement et al. (35–37). (p. 8)
%   [35] P. J. White, G. T. Clement, Two-dimensional localization with a single diffuse ultrasound field excitation. IEEE Trans. Ultrason. Ferroelectr. Freq. Control 54, 2309–2317 (2007).
%   [36] F. C. Meral, M. A. Jafferji, P. J. White, G. T. Clement, Two-dimensional image reconstruction with spectrally-randomized ultrasound signals. IEEE Trans. Ultrason. Ferroelectr. Freq. Control 60, 2501–2510 (2013).
% - Instead of solving a linear system as we propose here,
%   the authors used a dictionary containing measured impulse responses and a CROSS-CORRELATION TECHNIQUE to find the two point scatterers. (p. 8)
%
% article:MontaldoITUFFC2005: Building three-dimensional images using a time-reversal chaotic cavity
% I. Introduction
% - TIME-REVERSAL FOCUSING was studied previously in the field of ultrasound [9], for medical applications [10], and in ocean acoustics [11]. (p. 1489)
%   [9] M. Fink, “Time reversed acoustics,” Phys. Today, vol. 50, pp. 34–40, 1997.
%   [10] M. Fink, G. Montaldo, and M. Tanter, “Time reversal acoustics in biomedical engineering,” Annu. Rev. Biomed. Eng.,vol.5, pp. 465–497, 2003.
% - This technique is based on the REVERSIBILITY OF ACOUSTIC PROPAGATION, which implies that
%   the time-reversed version of an incident pressure field naturally refocuses in space and time on its source,
%   whatever the heterogeneity of the propagation medium. (p. 1489)
% - More precisely, it means that, for every burst of sound emitted from a source and possibly reflected and refracted by multiple boundaries,
%   there exists a set of waves that precisely retraces all the complex path and converges to the original source, as if time were going backward. (p. 1489)
% - Preliminary works on time reversal in 2-D closed cavities was done by Draeger et al. [13], [14] using Bunimovitch billiards. (pp. 1489, 1490)
%   [14] C. Draeger, J.-C. Aime, and M. Fink, “One-channel time-reversal in chaotic cavities: Experimental results,” J. Acoust. Soc. Amer., vol. 105, no. 2, pp. 618–625, 1999.
% - In order to focus a short pulse inside the medium, we use the TIME-REVERSAL PROCESS. (p. 1490)
% - This process achieved in a calibration medium like water allows us to learn
%   the TEMPORAL CODES TO BE APPLIED ON EACH TRANSDUCER IN ORDER TO FOCUS AT A GIVEN LOCATION. (p. 1491)
% - By repeating this process for different initial source locations,
%   we can learn the coded signals h_{i}(−t) allowing us to focus on any specific point of the medium. (p. 1491)
% - The complete calibration of the kaleidoscope consists in
%   recording all the data set of coded signals needed to focus at each point of the calibration medium. (p. 1491)
% - This process was modeled using a NUMERICAL SIMULATION OF THE WAVE PROPAGATION. (p. 1491)
% - This numerical approach greatly helps us to understand the building of the time-reversed focused beam. (p. 1491)
% article:MontaldoITUFFC2005: Building three-dimensional images using a time-reversal chaotic cavity
% Abstract
% - Thousands of transducers are typically needed for focusing and steering in a 3-D volume. (p. 1489)
% - In this article, we propose a different concept allowing us to obtain
%   ELECTRONIC 3-D FOCUSING WITH A SMALL NUMBER OF TRANSDUCERS. (p. 1489)
% - The basic idea is to couple a SMALL NUMBER OF TRANSDUCERS to
%   a CHAOTIC REVERBERATING CAVITY with one face in contact with the body of the patient. (p. 1489)
% - The reverberations of the ultrasonic waves inside the cavity CREATE AT EACH REFLECTION VIRTUAL TRANSDUCERS. (p. 1489)
% - The CAVITY ACTS AS AN ULTRASONIC KALEIDOSCOPE multiplying the small number of transducers and
%   CREATING A MUCH LARGER VIRTUAL TRANSDUCER ARRAY. (p. 1489)
% - By exploiting time-reversal processing, it is possible to use collectively all the virtual transducers to FOCUS A PULSE EVERYWHERE IN A 3-D VOLUME. (p. 1489)
% - The reception process is based on a nonlinear pulse-inversion technique in order to ensure a good contrast. (p. 1489)
% - The feasibility of this concept for the building of 3-D images was demonstrated using a prototype relying only on
%   31 EMISSION TRANSDUCERS AND A SINGLE RECEPTION TRANSDUCER. (p. 1489)
% I. Introduction
% - Here, we present an original approach that REPLACES THE 2-D ARRAY BY A SET OF LESS THAN 100 ELEMENTS. (p. 1489)
% - Our solution combines the use of TIME REVERSAL TECHNOLOGY with
%   a SMALL NUMBER OF PIEZOELECTRIC TRANSDUCERS fastened to a reverberating solid cavity presenting one face in contact with the investigated medium. (p. 1489)
% - Thanks to the multiple reverberations on the waveguide boundaries, waves emitted by each transducer are multiply reflected, creating
%   at each reflection virtual transducers that can be observed from the desired focal point. (p. 1489)
% - Thus, we create a large virtual array from a limited number of transducers. (p. 1489)
% - The result of such an operation is that a small number of transducers is multiplied to create a kaleidoscopic transducer array. (p. 1489)
% - However, symmetries implied in waveguides create periodic kaleidoscopic arrays, resulting in
%   grating lobes that limit the interest of this technique to shock wave generation for lithotripsy. (p. 1489)
% - The solution we propose is to break the waveguide’s sym- metries by introducing reverberating media with chaotic geometries such as chaotic billiards. (p. 1489)
% - Here, we extend this work to 3-D leaky cavities and we select
%   a Sinai billiard geometry to achieve 3-D focusing. (p. 1490)
% III. Image Formation Using the Cavity
% - In most ultrasonic devices, the receiving transducers are the same as the transmit ones. (p. 1495)
% - However, in our case it is very difficult to use the same transducers in both transmit and receive modes. (p. 1495)
% - When the backscattered echoes reach the surface of the cavity,
%   only a few percent of the pressure amplitude penetrates in the cavity because of the strong mechanical impedance mismatch between water and aluminum. (p. 1495)
% - As the reverberation time in the cavity is very long, these very weak backscattered echoes are mixed inside the cavity with
%   the residual reverberation noise of the transmit sequence. (p. 1495)
% - Unfortunately, it is not possible to differentiate them from the latter. (p. 1495)
% - One might be tempted to decrease the impedance mismatch between the cavity and the imaged medium. (p. 1495)
% - However, such a choice would increase the leakage of waves out of the cavity during the transmit mode. (p. 1495)
% - Consequently, it would decrease the time-reversal focusing efficiency as the number of reverberations inside the cavity would be smaller.  (p. 1495)
% - The main advantages of this single receiver at the front face of the cavity is that it overcomes
%   the limit of weak wave transmission at the solid-fluid interface. (p. 1496)
% - Backscattered echoes are recorded before entering the cavity. (p. 1496)
% - Thus, the final procedure in order to obtain an image consists in two steps:
%	[1.)] Calibration.
%	- The kaleidoscope is calibrated in water while learning the data set of transmit codes that allow us to focus pulses at
%         any location in the 3-D volume of interest, as explained in Section II. (p. 1496)
%	- Calibration experiments were carried out for 1600 focal points on a 40 by 40 grid, of a 40 by 40 mm plane placed at a 50 mm focal depth from
%	  the emitting surface (see Fig. 8). (p. 1496)
%	[2.)] Imaging.
%	- The kaleidoscope then is placed in front of the object to image, and we measure the second harmonic component of
%	  the backscattered echoes using the single receive transducer. (p. 1496)
%	- The test objects are tissue phantoms made of gelatin and containing ran- domly distributed scatterers (agar powder). (p. 1496)
%	- The frame rate is limited by the spreading time in the cavity, using signals of 500 µs we need 0.8 seconds to make a 40 by 40 points image. (p. 1496)
% - However, the contrast is not yet high enough to image human organs. (p. 1496)
% - Combined with the time-reversal process, this device exploits the multiple reverberations in a chaotic and leaky cavity to focus a pulse in the medium of interest. (p. 1496)
% - It is important to note that such chaotic cavities are very easy to build; we do not need to use small transducers or specific shapes,
%   we can glue transducers everywhere on the external surface of the cavity. (p. 1496)
% - On the contrary, the mechanical impedance mismatch between the cavity and the imaged medium is responsible for
%   the ability of time-reversal processing to focus waves in a 3-D volume with a very small number of transducers. (p. 1496)
% - Compared to the technological difficulties of a 2-D array of transducers, these simplifications make the construction of the cavities very easy. (p. 1496)
% - The complexity of 2-D arrays design is now transferred into the spatiotemporal coding techniques adapted to the cavity shape and stored into memories. (p. 1496)
Unlike
% 1.) time-reversal technique
the time-reversal technique
\cite{article:MontaldoITUFFC2005}, which uses
% 2.) leaky reverberant cavity
a leaky cavity to generate
% 3.) focused beams
% TODO: Sinai Billiard
focused beams,
% 4.) field-of-view (FOV)
the \ac{FOV} is not progressively scanned.

%---------------------------------------------------------------------------------------------------------------
% 3.) van Sloun et al.
%---------------------------------------------------------------------------------------------------------------
% a) van Sloun et al. proposed randomly-apodized transmissions from a circular array in two-dimensional tomography
% letter:VanSlounITBME2015: Compressed Sensing for Ultrasound Computed Tomography
% Abstract
% - In this letter, we propose a COMPRESSED SENSING SOLUTION FOR UCT. (p. 1660)
% - The adopted measurement scheme is based on COMPRESSED ACQUISITIONS, with
%   CONCURRENT RANDOMISED TRANSMISSIONS IN A CIRCULAR ARRAY CONFIGURATION. (p. 1660)
% - Reconstruction of the image is then obtained by combining
%   the BORN ITERATIVE METHOD and TOTAL VARIATION MINIMIZATION, thereby exploiting
%   VARIATION SPARSITY in the image domain. (p. 1660)
% - Evaluation using simulated UCT scattering measurements shows that
%   the PROPOSED TRANSMISSION SCHEME PERFORMS BETTER than
%   UNIFORM UNDERSAMPLING, and is able to REDUCE ACQUISITION TIME BY ALMOST ONE ORDER OF MAGNITUDE, while maintaining high spatial resolution. (p. 1660)
% I. INTRODUCTION
% - A typical arrangement is one where the BREAST IS ENCLOSED BY A CIRCULAR ARRAY of ultrasound transducer elements,
%   SEQUENTIALLY TRANSMITTING ONE-BY-ONE and receiving the resulting scattered wave fields at all elements [2]. (p. 1660)
% - For a large number of transducer elements, the acquisition time and complexity of this method are high. (p. 1660)
% - Considering a high-resolution ring-shaped ultrasound transducer with
%   a diameter of 20 cm, 1024 transmit elements, and an average speed of sound c0 of 1540 m/s,
%   it takes roughly 130 ms to image a single slice using sequential transmissions. (p. 1660)
% - In the latter [cylindrical matrix configuration for 3D imaging], imaging a full breast consisting of a number of slices in the order of 100, would require
%   an acquisition time in the order of tens of seconds. (p. 1660)
% - One approach is to UNIFORMLY UNDERSAMPLE IN THE TRANSMISSION DOMAIN, while keeping
%   the same number of receivers and employing sparse reconstruction techniques to retain the desired image features. (p. 1660)
% - Instead of applying CS-based reconstruction to sparse-view data,
%   WE CONSIDER COMPRESSIVE ACQUISITIONS WITH RANDOMIZED PARALLEL TRANSMISSIONS FROM THE CIRCULAR ARRAY. (p. 1660)
% - INTRODUCING RANDOMNESS ALLOWS NEAR-OPTIMAL CONDITIONS on the number of measurements in terms of the sparsity [7], facilitating
%   reduction of acquisition time, while keeping high spatial resolution. (p. 1660)
% - We compare the performance of the proposed method with
%   a UNIFORM UNDERSAMPLING APPROACH that uses TV minimization. (p. 1660)
% - In this letter, we show that
%   CS can be applied effectively in UCT to REDUCE ACQUISITION TIME BY ALMOST AN ORDER OF MAGNITUDE, while maintaining high image resolution. (p. 1660)
% II. MEASUREMENT MODEL / B. Inverse CS Problem
% - However, assuming that O is sparse in some domain, CS theory is applied by choosing the matrix \mat{\Phi} such that
%   TRANSDUCERS SIMULTANEOUSLY TRANSMIT PRESSURE WAVES WITH RANDOM AMPLITUDES THAT ARE GAUSSIAN DISTRIBUTED, having mean zero and standard deviation equal to one. (p. 1661)
% - Since we are interested in reducing the amount of transmissions only, we choose to receive with all elements. (p. 1661)
% - The total amount of transmission events is reduced by a factor M / N_{t}^{2}, here referred to as
%   the ACQUISITION REDUCTION FACTOR (ARF). (p. 1661)
% V. RESULTS
% - The CS-based acquisition approach has a lower MNAE [mean normalized absolute error], becoming increasingly significant for higher reduction factors. (p. 1662)
% - The improvement is remarkable when fully exploiting CS theory and using randomized transmission events. (p. 1663)
% VI. CONCLUSION AND DISCUSSION
% - In this letter, we presented a new CS-based approach to diffraction UCT in a circular array configuration. (p. 1663)
% - By combining compressed, randomized transmission events, and sparse reconstruction techniques, the proposed method potentially allows
%   reduction of the acquisition time by an order of magnitude, while preserving high image resolution. (p. 1663)
% - Although this does not influence acquisition time, reconstruction using randomized compressed transmissions required about one iteration more. (p. 1663)
% - A quantitative analysis for both methods showed that using CS results in a lower MNAE for all ARFs, with the difference becoming increasingly significant for higher reductions. (p. 1663)
% - More specifically, the uniform method’s performance declines rapidly up to a reduction factor of 8, whereas the CS method remains more stable. (p. 1663)
% - Reducing the acquisition time using CS, comes at the cost of higher computational expense. (p. 1663)
% - Although solving the l1 minimization problem is about 30–50 times as expensive as solving the least-squares problem [16], the problem size is reduced by ARF. (p. 1663)
\name{Van Sloun} \emph{et al.}
\cite{letter:VanSlounITBME2015} proposed
randomly-apodized monofrequent emissions from
a circular array in
two-dimensional tomography.
% b) randomly-apodized emissions outperformed sparse SA acquisition sequences
These outperformed
sparse \ac{SA} acquisition sequences using
only a few emissions from
random elements.
%---------------------------------------------------------------------------------------------------------------
% 4.) Liu et al
%---------------------------------------------------------------------------------------------------------------
% a) Liu et al. utilize realizations of uniformly-distributed random variables as the apodization weights
% article:LiuITUFFC2018: Compressed Sensing Based Synthetic Transmit Aperture Imaging: Validation in a Convex Array Configuration
% II. THEORY STUDY / B. Compressed Sensing Based Synthetic Transmit Aperture
% - Since \mat{\Phi} in (2) usually obeys a random distribution, to adapt the data acquisition in ultrasound imaging,
%   \mat{\Phi} in (13) can be DESIGNED TO OBEY A CONTINUOUS UNIFORM RANDOM DISTRIBUTION WHOSE ENTRIES RANGE BETWEEN 0 AND 1, i.e.,
%   \mat{\Phi} ∼ U(0, 1). (p. 303)
% article:LiuITMI2017: A Compressed Sensing Strategy for Synthetic Transmit Aperture Ultrasound Imaging
% II. COMPRESSED SENSING BASED SYNTHETIC TRANSMIT APERTURE / B. Compressed Sensing Based Synthetic Transmit Aperture
% - This condition [incoherence] is often satisfied when \mat{\Phi} OBEYS A RANDOM DISTRIBUTION [2], that is why
%   the transmit apodization applied in each PW firing of CS-STA is random. (p. 881)
% - Specifically, in step 1) in this work, the TRANSMIT APODIZATIONS or measurement matrix \mat{\Phi} OBEY
%   A CONTINUOUS UNIFORM DISTRIBUTION WITH
%   the AMPLITUDES BEING RANDOMLY DISTRIBUTED IN THE INTERVAL OF [0, 1] AND THE MEAN VALUE BEING 0.5, i.e.,
%   \mat{\Phi} ∼ U(0, 1). (p. 881)
% VI. DISCUSSIONS / C. Reconstruction Quality
% - In this work, the measurement matrix \mat{\Phi} obeys a CONTINUOUS UNIFORM DISTRIBUTION with
%   AMPLITUDES BEING RANDOMLY DISTRIBUTED IN THE INTERVAL OF [0, 1]. (p. 889)
\name{Liu} \emph{et al.} \cite{article:LiuITUFFC2018,article:LiuITMI2017} realized
uniformly-distributed apodization weights for
linear and convex arrays.
% b) Liu et al. recovered the pulse echoes induced by a complete SA acquisition sequence and subsequently applied the popular DAS method for image formation
% article:LiuITUFFC2018: Compressed Sensing Based Synthetic Transmit Aperture Imaging: Validation in a Convex Array Configuration
% III. SIMULATIONS / A. Simulation Setup
% - To guarantee the sparsity,
%   the SYM8 WAVELET was chosen to construct the sparse basis \mat{\Psi}. (p. 304)
% VI. DISCUSSION / B. CS-STA Performance
% - That is to say, the slow time signal of STA firing has a SPARSER REPRESENTATION WITH SYM8 WAVELET than with Fourier basis. (p. 313)
% - Therefore, SYM8 WAVELET BASIS IS USED IN THIS PAPER. (p. 313)
% - In this paper, the coherence between the measurement matrix \mat{\Phi} and SYM8 WAVELET SPARSE BASIS is about 2.46,
%   close to the coherence 2.2 between noiselets and Daubechies D4 wavelet [2]. (p. 313)
% - It demonstrates that \mat{\Phi} and \mat{\Psi} are incoherent, which guarantees the successful reconstruction of CS-STA. (p. 313)
% article:LiuITMI2017: A Compressed Sensing Strategy for Synthetic Transmit Aperture Ultrasound Imaging
% II. COMPRESSED SENSING BASED SYNTHETIC TRANSMIT APERTURE / B. Compressed Sensing Based Synthetic Transmit Aperture
% - The SYM8 WAVELET [36] IS CHOSEN AS THE SPARSE BASIS \mat{\Psi} and
%   the tolerated error ε is empirically set as 1 × 10−3 in all the simulations, phantom and in vivo experiments in this paper. (p. 881)
% VI. DISCUSSIONS / C. Reconstruction Quality
% - In this work, we analyzed the SPARSITY OF THE SLOW TIME SIGNAL \vect{x} of the STA data in
%   the SYM8 WAVELET [36], Fourier, and wave atoms [40] domains. (p. 888)
% - Both signals from the SIMULATED AND IN VIVO OBJECTS show their sparsity and
%   THEY CAN BE REPRESENTED MORE SPARSELY IN THE SYM8 WAVELET DOMAIN than in the other two bases. (p. 888)
% - That is why THE SYM8 WAVELET WAS CHOSEN AS THE SPARSE BASIS IN THIS STUDY. (pp. 888, 889)
% - As a result, continuous uniform random distribution is used in this work, and
%   the coherence between such a distribution and the SYM8 WAVELET SPARSE BASIS is about 2.46,
%   close to the coherence 2.2 between the noiselets and Daubechies D4 wavelet [2]. (p. 889)
% - It proves that the continuous uniform random distribution is incoherent with the SYM8 WAVELET, which guarantees
%   the successful reconstruction of CS-STA. (p. 889)
Unlike
% 1.) inverse scattering methods
the inverse scattering methods,
% 2.) Liu et al.
they recovered
% 3.) recorded RF voltage signals
the echo signals induced by
% 4.) complete SA acquisition sequence
a complete \ac{SA} acquisition sequence, which were represented almost sparsely by
% 5.) sym8 wavelet basis
a sym8 wavelet basis, and subsequently applied
% 6.) popular DAS method
the popular \ac{DAS} method for
% 7.) image formation
image formation.
% c) large number of unknown temporal samples required on the order of 30 sequential pulse-echo measurements per image
% article:LiuITUFFC2018: Compressed Sensing Based Synthetic Transmit Aperture Imaging: Validation in a Convex Array Configuration
% Abstract
% - The experimental results showed that STA and CS-STA performed better than ME-STA and the focused method at small depths. (p. 300)
% - At the depth of 110 mm, CS-STA, ME-STA, and the focused methods improved the contrast and contrast-to-noise ratio of STA. (p. 300)
% - The improvements in CS-STA are higher than those in ME-STA but lower than those in the focused mode. (p. 300)
% - These results can also be observed qualitatively in the in vivo experiments on the liver of a healthy male volunteer. (p. 300)
% - The CS-STA method is thus proved to increase the frame rate and achieve high image quality at full depth in the convex array configuration. (p. 300)
% article:LiuITMI2017: A Compressed Sensing Strategy for Synthetic Transmit Aperture Ultrasound Imaging
% Abstract
% - In addition, the CONTRAST OF THE STA IMAGE CAN BE IMPROVED at the same time owing to the higher energy of plane wave firing in CS-STA. (p. 878)
% - The results demonstrate that, implemented with the same frame rate, CS-STA achieves HIGHER OR COMPARABLE RESOLUTION AND CONTRAST. (p. 878)
Despite
% 1.) improvements
the improvements in
% 2.) contrast
contrast and
% 3.) spatial resolution
spatial resolution,
% 4.) large number of unknown temporal samples
the large number of
unknown temporal samples required
% 5.) tens of sequential pulse-echo measurements per image
tens of
sequential pulse-echo measurements per
image.
