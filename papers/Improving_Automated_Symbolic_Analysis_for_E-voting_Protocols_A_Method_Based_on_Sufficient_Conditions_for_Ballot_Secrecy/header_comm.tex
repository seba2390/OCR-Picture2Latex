\usepackage[noamssymb]{dlfltxbcodetips}
%
% \lrstep
%\newcommand{\lrstep}{\xrightarrow}
\newcommand{\lrstep}[1]{\ensuremath{\raisebox{-2pt}{$\xrightarrow{#1}$}}}  
%%%%%%%%%%%% COMMON %%%%%%%%%%%%%%%
% Mathcal
\newcommand{\C}[1]{\mathcal{#1}}
\renewcommand{\S}{\mathcal{S}}
\newcommand{\V}{\mathcal{V}}
\newcommand{\X}{\mathcal{X}}
\newcommand{\W}{\mathcal{W}}
\newcommand{\T}{\mathcal{T}}
\newcommand{\N}{\mathcal{N}}
\newcommand{\q}{\mathcal{Q}}
\newcommand{\p}{\mathcal{P}}
\newcommand{\E}{\mathsf{E}}
% Abrevv
\newcommand{\ie}{\emph{i.e.}\xspace}
\newcommand{\eg}{\emph{e.g.}\xspace}
\newcommand{\etc}{etc.\xspace}
\newcommand{\wrt}{w.r.t.\xspace}
\newcommand{\eqdefi}{\stackrel{\mathsf{def}}{=}} % equality-definition
% Font & presentaion %
\newcommand{\bff}[1]{\boldsymbol{#1}} % for code and algo.
\newcommand{\sep}{\ |\ }              % for inductive def
% Math %
\newcommand{\pws}[1]{\mathcal{P}\left(#1\right)} % powerset
\newcommand{\multiset}[1]{\{#1\}^{\#}}  
%% Vectors
\usepackage{esvect}
\newcommand{\vect}[1]{\overline{#1}}


%%%%%%%%%%% APPLIED PI %%%%%%%%%%%%%
%%% MESSAGES %%%
% constantes
\newcommand{\ok}{\mathsf{ok}}
\newcommand{\no}{\mathsf{no}}
\newcommand{\error}{\mathsf{error}}
% Function Symbols
\newcommand{\priv}{\mathsf{priv}} % private function symbols
\newcommand{\aenc}[0]{\mathsf{aenc}}
\newcommand{\renc}[0]{\mathsf{reEnc}}
\newcommand{\adec}[0]{\mathsf{adec}}
\newcommand{\h}{\mathsf{h}}
\newcommand{\pub}{\mathsf{pub}}
\newcommand{\senc}[0]{\mathsf{enc}}
\newcommand{\sdec}[0]{\mathsf{dec}}
\newcommand{\sk}[0]{\mathsf{sk}}  % secret key for aenc ---> to remove, we want adec(aenc(x,pk(k)),k) -> x
\newcommand{\pk}[0]{\mathsf{pk}}  % public ket for aenc
\newcommand{\eq}[0]{\mathsf{eq}}  % equality (test)
\newcommand{\AND}[0]{\mathsf{and}}  % equality (test)
\newcommand{\proj}{\pi}
\newcommand{\projl}[0]{\proj_1} % projections
\newcommand{\projr}[0]{\proj_2}
\newcommand{\theo}{=_\E}        % equational theory
\newcommand{\nottheo}{\not=_\E}        % equational theory inequality
\newcommand{\messS}{\T(\Sigma_c,\N)}
\newcommand{\messShole}{\T(\Sigma_t,\{\hole\})}
\newcommand{\terms}{\T(\Sigma,\N\cup\X)} % terms
\newcommand{\recipes}{\T(\Sigma_\pub,\W)} % recipes
\newcommand{\ffun}{\mathsf{f}}  % consructor
\newcommand{\gfun}{\mathsf{g}}  % destructor
\newcommand{\efun}{\mathsf{e}}  % any function symbol 
\newcommand{\subt}{\sqsubseteq_s} %sub-term relation
% computations
\newcommand{\redd}{\rightarrow}          % reduction rules
\newcommand{\redc}{\!\downarrow\!} % computation relation
\newcommand{\redcb}{\!\ndownarrow\!} % failure of computation relation
%%% MESSAGES FOR EVOTING
\newcommand{\id}{\mathrm{id}}
\newcommand{\tcom}{\mathsf{trapCommit}}
\newcommand{\com}{\mathsf{commit}}
\newcommand{\open}{\mathsf{open}}
\newcommand{\bl}{\mathsf{blind}}
\newcommand{\sign}{\mathsf{sign}}
\newcommand{\pkv}[0]{\mathsf{pkv}}  % public ket for sign (versign(sign(x,k),pkv(k)) = x)
\newcommand{\versign}{\mathsf{verSign}}
\newcommand{\key}{\mathsf{key}}
\newcommand{\getMess}{\mathsf{getMess}}
\newcommand{\unbl}{\mathsf{unblind}}
\newcommand{\ZK}{\mathsf{ZK}}
\newcommand{\checkZK}{\mathsf{checkZK}}
%%% Syntax & Actions %%%%
\newcommand{\Ch}{\mathcal{C}}
\newcommand{\phase}[1]{#1:} %{\mathrm{phase}_{#1}}
\newcommand{\Phase}[1]{\mathtt{phase}(#1)} %{\mathrm{phase}_{#1}}
\newcommand{\pair}[2]{\langle #1,#2 \rangle}
\newcommand{\trip}[3]{\langle #1,#2,#3 \rangle}
\newcommand{\Out}{\mathtt{out}}
\newcommand{\In} {\mathtt{in}}
\newcommand{\Par} {~|~}
\newcommand{\Bang}{!\ }
\newcommand{\new}{\nu}
\newcommand{\choice}[2]{\mathsf{choice}[#1,#2]}
\newcommand{\fst}[0]{\mathsf{fst}}
\newcommand{\snd}[0]{\mathsf{snd}}
\newcommand{\diff}[2]{\mathsf{diff}[#1;#2]}
\newcommand{\Let}{\mathsf{let}}
\newcommand{\Else}{\mathsf{else}}
\newcommand{\If}{\mathsf{if}}
\newcommand{\Then}{\mathsf{then}}
\newcommand{\monthen}{\mathsf{then}} % stef
\newcommand{\monelse}{\mathsf{else}} % stef
\newcommand{\test}[3]{\Let\; #1\; \In\; #2\; \Else\; #3}
\newcommand{\testt}[2]{\mathtt{if}\ #1\ \mathtt{then}\ #2}
% PROCESS
\newcommand{\refer}{\mapsto} % \triangleright
\newcommand{\proc}[2]{(#1;#2)}
% Variables &
\newcommand{\vars}{{\mathrm{vars}}}  % variables in terms/messages
\newcommand{\fv}{\mathrm{fv}}    % free variables in processes
\newcommand{\bc}{\mathit{bc}}    % bound channels
\newcommand{\fc}{\mathit{fc}}    % free channels
\newcommand{\fn}{\mathit{fn}}    % free names
% SEMANTIC
\newcommand{\sint}[1]{\lrstep{\ensuremath{#1}}}          % ???
\newcommand{\internRed}{\rightsquigarrow}
% \newcommand{\sint}[1]{\lrstep{#1}}         
\newcommand{\subst}[2]{\{{#1}/{#2}\}} %subst for input
\newcommand{\taut}{\tau_\monthen}
\newcommand{\taue}{\tau_\monelse}
% TRACES
\newcommand{\dom}{\mathrm{dom}}
\newcommand{\tr}{\mathsf{tr}}
% EQUIVALENCES
\newcommand{\obs}{\mathsf{obs}}
\newcommand{\trace}{\mathsf{trace}}
\newcommand{\estat}{\sim}
\newcommand{\eint}{\approx}         %équivalence de traces
\newcommand{\eintm}{\approx_m}         %équivalence de traces with only minimal traces

%%%%%%%%%%% Annotations %%%%%%%%%%%
\renewcommand{\th}{\mathsf{th}}       %honest trace
\newcommand{\tha}{\ta_h}       %honest annotated trace
% \newcommand{\ian}[0]{\ini}
% \newcommand{\ran}[0]{\res}
\newcommand{\rename}[1]{[#1]}
\newcommand{\swap}[3]{#1\left\{#3\slash#2\right\}}
\newcommand{\restrict}[2]{#1|_{#2}}


%%%%%%%%%%%% TOOLS %%%%%%%%%%%%%%%%%
% TOOLS
\newcommand{\tamarin}{Tamarin\xspace}
\newcommand{\proverif}{ProVerif\xspace}%\textsf{ProVerif}\xspace}
\newcommand{\spec}{Spec\xspace}
\newcommand{\apte}{Apte\xspace}
\newcommand{\ukano}{\textsf{UKano}}


%%%%%%%%%% CUSTOM COUNTERS and ENV %%%%%%%%%
\newcounter{condiC}             % conditions
\setcounter{condiC}{0}
%Numbered environment
\newenvironment{condi}[1][]{\refstepcounter{condiC}\par\medskip
   \noindent \textbf{Condition~\thecondiC. (#1)} \rmfamily}{\medskip}
\newcommand{\condiR}[1]{(C\ref{#1})}

%\newcounter{problemC}
%\setcounter{problemC}{0}
%\renewenvironment{problem}[1][]{\refstepcounter{problemC}\par\medskip
 %  \noindent \textbf{Problem~\theproblemC.} \rmfamily}{\medskip}

%\renewenvironment{proof}{\paragraph{Proof:}}{\hfill$\qed$\\}

% \newenvironment{proof}[1][Proof]{\par{\emph{#1}.} }{\hfill$\Box$}

\newcommand{\lucca}[1]{\textcolor{red}{#1}}
\newcommand{\cas}[1]{\textcolor{blue}{#1}}
\newcommand{\lum}[1]{\marginpar{\lucca{\scriptsize #1\normalsize}}}
\newcommand{\cam}[1]{\marginpar{\cas{\footnotesize #1\normalsize}}}
\newcommand{\lme}[1]{{\tiny \textcolor{black}{#1} \normalsize}}
\newcommand{\lumF}[1]{\marginpar{\color{brown}{\scriptsize #1\normalsize}}}
\newcommand{\camF}[1]{\marginpar{\color{green}{\footnotesize #1\normalsize}}}
%%% NEW COMMENTS (CCS)
\newcommand{\luccaN}[1]{#1} %{\textcolor{red}{#1}}
\newcommand{\toRM}[1]{\textcolor{gray}{#1}}
\newcommand{\casN}[1]{\textcolor{yellow}{#1}}
\newcommand{\lumN}[1]{}%\marginpar{\luccaN{\scriptsize #1\normalsize}}}
\newcommand{\camN}[1]{\marginpar{\casN{\footnotesize #1\normalsize}}}


%% UNCOMENT THE BELOW for REMOVING old comments
% \renewcommand{\lucca}[1]{}%\textcolor{red}{#1}}
% \renewcommand{\cas}[1]{}%\textcolor{blue}{#1}}
% \renewcommand{\lum}[1]{}%\marginpar{\lucca{\scriptsize #1\normalsize}}}
% \renewcommand{\cam}[1]{}%\marginpar{\cas{\footnotesize #1\normalsize}}}
% \renewcommand{\lme}[1]{}%{\tiny \textcolor{black}{#1} \normalsize}}
% \renewcommand{\lumF}[1]{}%\marginpar{\color{brown}{\scriptsize #1\normalsize}}}
% \renewcommand{\camF}[1]{}%\marginpar{\color{green}{\footnotesize #1\normalsize}}}

%%% UNCOMMENT BELOW FOR REMOVING NEW COMMENTS
% \renewcommand{\luccaN}[1]{}
%\renewcommand{\casN}[1]{}
% \renewcommand{\lumN}[1]{}
% \renewcommand{\camN}[1]{}

%%% UNCOMMENT BELOW FOR REMOVING "toRM" texts
\renewcommand{\toRM}[1]{}



%%%% CLASS
\newcommand{\Res}{\mathrm{Res}}
\newcommand{\BB}{\mathrm{BB}}
\newcommand{\roles}{\mathcal{R}}
%%%% SUB-PARTS
\newcommand{\nID}[1]{{\mathbf{n^{id}_{#1}}}}
\newcommand{\nV}[1]{{\mathbf{n^{v}_{#1}}}}
\renewcommand{\vect}[1]{{\mathbf{#1}}}
\newcommand{\biproc}{\mathcal{B}}
\newcommand{\biprocID}{\biproc^\id}
\newcommand{\biprocVO}{\biproc^v}
\newcommand{\nameL}[1]{v_{#1},k_{#1},k_{#1'}}
\newcommand{\nameS}[1]{\vect{n^v_{#1}}}
\newcommand{\nameLid}[1]{\id_{#1}}
\newcommand{\nameSid}[1]{\vect{n^{\id}_{#1}}}
\newcommand{\Extract}[0]{\mathrm{Extract}}

%%% MISC
\newcommand{\cmark}{\ding{51}}%
\newcommand{\xmark}{\ding{55}}%


\newcommand\scalemath[2]{\scalebox{#1}{\mbox{\ensuremath{\displaystyle #2}}}}



\theoremstyle{definition}
\newtheorem{example}{Example}
\newtheorem{definition}{Definition}
