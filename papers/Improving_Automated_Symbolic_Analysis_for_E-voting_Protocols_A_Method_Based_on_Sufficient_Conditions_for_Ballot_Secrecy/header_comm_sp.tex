\usepackage{graphicx}

\usepackage[usenames,dvipsnames,svgnames,table]{xcolor}

% Pour flusher les figures
\usepackage{placeins}
\usepackage{ wasysym }

\newcommand{\Choice}[2]{\mathtt{choice}[#1,#2]}


\newcommand{\ideal}{\mathsf{ideal}}
\newcommand{\noncee}{\mathsf{nonce}}

\newcommand{\con}{\C{C}\mathrm{o}}
\newcommand{\freshOf}[2]{\nu#1[#2]}


\newcommand{\cmark}{\ding{51}}%
\newcommand{\xmark}{\ding{53}}%
% CELUI-ci est cool pour la croix aussi (mais trop zépais je trouve):
%\newcommand{\xmark}{\ding{55}}%

\newcommand{\verif}{\cmark}
\newcommand{\holds}{\bf safe}
\newcommand{\nope}{{\xmark}}
\newcommand{\attaque}{{attack}}


%%%%%%%%%%%% NEW SYMBOLS %%%%%%%%%%%%%%
% \lrstep
\makeatletter
\def\rightarrowfillstar@{\arrowfill@\relbar\relbar{\rightarrow\smash{^*}}}
\newcommand{\xrightarrowstar}[2][]{\ext@arrow
  0{13}{15}8\rightarrowfillstar@{#1}{#2}}
\newcommand{\lrstep}{\@ifstar{\xrightarrowstar}{\xrightarrow}}
\makeatother
% \LRstep
\makeatletter
\def\Rightarrowfillstar@{\arrowfill@=={\Rightarrow\smash{^*}}}
%\def\Rightarrowfill@{\arrowfill@=={\Rightarrow}}
\newcommand{\NewxRightarrowstar}[2][]{\ext@arrow
  0{13}{15}8\Rightarrowfillstar@{#1}{#2}}
\newcommand{\NewxRightarrow}[2][]{\ext@arrow
  0{13}{15}8\Rightarrowfill@{#1}{#2}}
\newcommand{\LRstep}{\@ifstar{\NewxRightarrowstar}{\NewxRightarrow}}
\makeatother
% \xRightarrow
\makeatletter
\newcommand{\xRightarrow}[2][]{\ext@arrow 0359\Rightarrowfill@{#1}{#2}}
\makeatother
% Les nouvelles fleches de Nico
\makeatletter
\newcommand{\rightdoublearrow}{%
  \rightarrow\mkern-10mu\protect\joinrel\rightarrow}
\newcommand{\longrightdoublearrow}{%
  \DOTSB\protect\relbar\protect\joinrel\rightdoublearrow}
\newcommand{\rightdoublearrowfill@}
  {\arrowfill@\relbar\relbar\rightdoublearrow}
\newcommand{\mapstodoublearrowfill@}
  {\arrowfill@{\mapstochar\relbar}\relbar\rightdoublearrow}
\newcommand{\xrightdoublearrow}[2][]
  {\ext@arrow 3{15}59\rightdoublearrowfill@{#1}{#2}}
\newcommand{\xmapstodoublearrow}[2][]
  {\ext@arrow 3{15}59\mapstodoublearrowfill@{#1}{#2}}
\makeatother
% xMapsto
\makeatletter
\newcommand{\xMapsto}[2][]{\ext@arrow 0599{\Mapstofill@}{#1}{#2}}
\def\Mapstofill@{\arrowfill@{\Mapstochar\Relbar}\Relbar\Rightarrow}
\makeatother

%%%%%%%%%%%% COMMON %%%%%%%%%%%%%%%
% Mathcal
\newcommand{\C}[1]{\mathcal{#1}}
% Lettres
\newcommand{\V}{\mathcal{V}}
\newcommand{\U}{\mathcal{U}}
\newcommand{\X}{\mathcal{X}}
\newcommand{\W}{\mathcal{W}}
\newcommand{\T}{\mathcal{T}}
\newcommand{\N}{\mathcal{N}}
\newcommand{\q}{\mathcal{Q}}
\newcommand{\p}{\mathcal{P}}
\newcommand{\E}{\mathsf{E}}
\newcommand{\defd}{\mathsf{def}}
\newcommand{\redd}{\rightarrow} % règles de réduction
\newcommand{\redc}{\mathrel{\Downarrow}} % relation calcul
\newcommand{\redcb}{\mathrel\nDownarrow} % relation calcul fail
% Abrevv
\newcommand{\ie}{\emph{i.e.}\xspace}
\newcommand{\eg}{\emph{e.g.}\xspace}
\newcommand{\etc}{etc.\xspace}
\newcommand{\eqdefi}{\stackrel{\mathsf{def}}{=}}
\newcommand{\df}{\mathsf{dh}}
% Font & presentaion %
\newcommand{\bff}[1]{\boldsymbol{#1}} %Pour code et algos
\newcommand{\undl}[1]{\underline{#1}} %Pour def.
\newcommand{\sep}{\ |\ }              %Pour inductive & ensemble
% Math %
\newcommand{\pws}[1]{\mathcal{P}\left(#1\right)} %puissance
\newcommand{\pwsf}[1]{\mathcal{P}_{\mathrm{f}\slash\mathrm{cf}}\left(#1\right)} %puissance finie
%% VECTEURS
\usepackage{esvect}
\newcommand{\vect}[1]{\overline{#1}}
% %% PROOF AVEC QED
% \renewenvironment{proof}[1][Proof]{{\noindent\emph{#1.}\;}}{{\qed}\medskip}


%%%%%%%%%%% APPLIED PI %%%%%%%%%%%%%
%% MESSAGES
% constantes
\newcommand{\ok}{\mathsf{ok}}
\newcommand{\no}{\mathsf{no}}
\newcommand{\error}{\mathsf{error}}
\newcommand{\hello}{\mathsf{hello}}
\newcommand{\ack}{\mathsf{ack}}
\newcommand{\start}{\mathsf{start}}
% symboles de fonctions
\newcommand{\eq}{\mathsf{eq}}
\newcommand{\monneq}{\mathsf{neq}}

\newcommand{\priv}{\mathsf{priv}}
\newcommand{\senc}{\mathsf{enc}}
\newcommand{\sdec}{\mathsf{dec}}
\newcommand{\aenc}[2]{\mathsf{aenc}(#1,#2)}
\newcommand{\adec}[2]{\mathsf{adec}(#1,#2)}
\newcommand{\h}{\mathsf{h}}
\newcommand{\pub}{\mathsf{pub}}
\newcommand{\enc}[2]{\mathsf{enc}(#1,#2)}
\newcommand{\dec}[2]{\mathsf{dec}(#1,#2)}
\newcommand{\sk}[1]{\mathsf{sk}(#1)}
\newcommand{\pk}[1]{\mathsf{pk}(#1)}
\newcommand{\projl}[1]{\pi_1(#1)}
\newcommand{\projr}[1]{\pi_2(#1)}
\newcommand{\theo}{=_\E}        %théorie équationnelle
\newcommand{\theoN}{\equiv}     %théorie équiationnele modulo substitution bijective de noms
\newcommand{\mess}{\T(\Sigma_c,\N)}
\newcommand{\messhole}{\T(\Sigma_s,\{\hole\})}
\newcommand{\terms}{\T(\Sigma,\N\cup\X)}
\newcommand{\recipes}{\T(\Sigma_\pub,\W)}
\newcommand{\ffun}{\mathsf{f}}  % consructor
\newcommand{\gfun}{\mathsf{g}}  % destructor
\newcommand{\efun}{\mathsf{e}}  % any function symbol 
\newcommand{\proj}{\pi}
\newcommand{\w}{\mathsf{w}}
\newcommand{\subt}{\sqsubseteq_s} %sub-term relation
% Syntax & Actions %
\newcommand{\Ch}{\mathcal{C}}
\newcommand{\pair}[2]{\langle #1,#2 \rangle}
\newcommand{\trip}[3]{\langle #1,#2,#3 \rangle}
\newcommand{\Out}{\mathtt{out}}
\newcommand{\In} {\mathtt{in}}
\newcommand{\Par} {~|~}
\newcommand{\Bang}{!\ }
\newcommand{\new}{\nu}
\newcommand{\rep}{\mathrel{\raisebox{.5ex}{!`}}\hspace{-2pt}} %sequential replication
\newcommand{\inp}[2]{\In(#1,#2)} %input
\newcommand{\out}[2]{\Out(#1,#2)} %output
\newcommand{\cond}{\varphi} %conditional
\newcommand{\choice}[2]{\mathsf{choice}[#1,#2]}
\newcommand{\hole}{\Box}
\newcommand{\Let}{\mathsf{let}}
\newcommand{\Else}{\mathsf{else}}
\newcommand{\If}{\mathsf{if}}
\newcommand{\Then}{\mathsf{then}}
\newcommand{\LetIf}{\mathsf{letif}}  % letif:  début de extended conditional
\newcommand{\monthen}{\mathsf{then}} % stef
\newcommand{\monelse}{\mathsf{else}} % stef
\newcommand{\eneq}{\boxdot}
\newcommand{\test}[3]{\Let\; #1\; \In\; #2\; \Else\; #3}
\newcommand{\testt}[2]{\mathtt{if}\ #1\ \mathtt{then}\ #2}
% PROCESSUS
\newcommand{\refer}{\mapsto} % \triangleright
\newcommand{\proc}[2]{(#1;#2)}
%% EXAMPLES DE PROCESS
\newcommand{\bacsys}{\C M_{\mathsf{BAC}}}
% Variables &
\newcommand{\vars}{{\mathrm{vars}}}  % variables dans messages
\newcommand{\fv}{\mathrm{vars}} %variables libres dans process
\newcommand{\fvs}{\mathit{fv}^2} %symboles libres
\newcommand{\fvp}{\mathit{fv}^1} %variables premier ordre libres
\newcommand{\bc}{\mathit{bc}}    %bound channels
\newcommand{\fc}{\mathit{fc}}    %free channels
\newcommand{\fn}{\mathit{fn}}    %free names
% SEMANTIC
\newcommand{\sint}[1]{\lrstep{#1}}                  %sém. concrète
%\newcommand{\sintm}[1]{\lrstep{#1}}                  %sém. concrète avec trace maximale (pour les tests)
\newcommand{\internRed}{\rightsquigarrow}
\newcommand{\subst}[2]{\{{#1}/{#2}\}} %subst for input
\newcommand{\taut}{\tau_\monthen}
\newcommand{\taue}{\tau_\monelse}
\newcommand{\lastt}[1]{#1}   % {{#1}^*} 
% TRACES
\newcommand{\dom}{\mathrm{dom}}
\newcommand{\tr}{\mathsf{tr}}
\newcommand{\ta}{\mathsf{ta}}
% EQUIVALENCES
\newcommand{\obs}{\mathsf{obs}}
\newcommand{\trace}{\mathsf{trace}}
\newcommand{\etr}{\stackrel{\obs}{=}}  % égalité de traces modulo obs
\newcommand{\statequiv}{\sim}
\newcommand{\estat}{\statequiv}
\newcommand{\eint}{\approx}         %équivalence de traces
%% ANNOTATIONS
\newcommand{\annot}[1]{[#1]}
\newcommand{\trunc}[1]{[#1]}

%%%%%%%%%%%% Agents %%%%%%%%%%%%%%%
\newcommand{\dual}[1]{\overline{#1}}
\newcommand{\ini}{\mathcal{I}}
\newcommand{\res}{\mathcal{R}}
\newcommand{\aini}{I}
\newcommand{\ares}{R}
\newcommand{\agent}{\mathcal{A}}
\newcommand{\aagent}{A} % denotes \aini or \ares
\newcommand{\agents}{\mathcal{A}\mathrm{gents}} % set of all agents (anotations)

%%%%%%%%%%% Annotations %%%%%%%%%%%
\renewcommand{\th}{\tr_h}       %honest trace
\newcommand{\tha}{\ta_h}       %honest annotated trace
% \newcommand{\ian}[0]{\ini}
% \newcommand{\ran}[0]{\res}
\newcommand{\rename}[1]{[#1]}
\newcommand{\swap}[3]{#1\left\{#3\slash#2\right\}}
\newcommand{\restrict}[2]{#1|_{#2}}

%%%%%%%%%%%% Transformations %%%%%%%%%
\newcommand{\nonce}[1]{[#1]^{\mathsf{nonce}}} % transfo nonces (un ensemble en gros \inst(\nonce{#1)))
\newcommand{\holes}[1]{[#1]^{\mathsf{ideal}}}          % transfo avec trous 
\newcommand{\concr}[1]{\mathsf{inst}(#1)}      % transfo nonces à partir d'un terme avec trous
\newcommand{\tof}[1]{[#1]^t}
\newcommand{\nof}[1]{[#1]^n}
\newcommand{\ideaf}[0]{\Phi_s}% idealizations of frames
\newcommand{\ideam}[1]{\mathrm{idea}(#1)}% idealizations of messages
\newcommand{\inpu}[0]{\mathrm{inp}}% idealizations of messages

%%%%%%%%%%%% TOOLS %%%%%%%%%%%%%%%%%
\newcommand{\Spec}{$\mathsf{Spec}$\xspace}
\newcommand{\Apte}{$\mathsf{Apte}$\xspace}
\newcommand{\ukano}{$\mathsf{UKano}$\xspace}
\newcommand{\proverif}{$\mathsf{ProVerif}$\xspace}
\newcommand{\tamarin}{$\mathsf{Tamarin}$\xspace}
\newcommand{\proveriff}{\textbf{$\textsf{ProVerif}$}\xspace}


%%%%%%%%%% PROTOCOLS %%%%%%%%%%
\newcommand{\proto}{\Pi}    % a protocol (tuple (k,n,I,R)
\newcommand{\pM}{\C M_\Pi}    % whole system multiple session
\newcommand{\pS}{\C S_\Pi}    % whole system, once session per identity
\newcommand{\pMa}{\C M_{\Pi,{\mathsf{id}}}} % whole system multiple sessions + known identity


\newcommand{\Feldhofer}{\mathsf{Fh}}
%%%%%%%%%% CUSTOM COUNTERS and ENV %%%%%%%%%
\newcounter{condiC}             % conditions
\setcounter{condiC}{0}
%Numbered environment
\newenvironment{condi}[1][]{\refstepcounter{condiC}\par\medskip
   \noindent \textbf{Condition~\thecondiC. (#1)} \rmfamily}{\medskip}
\newcommand{\condiR}[1]{(C\ref{#1})}

%\newcounter{problemC}
%\setcounter{problemC}{0}
%\renewenvironment{problem}[1][]{\refstepcounter{problemC}\par\medskip
 %  \noindent \textbf{Problem~\theproblemC.} \rmfamily}{\medskip}

%\renewenvironment{proof}{\paragraph{Proof:}}{\hfill$\qed$\\}

\newenvironment{proof}[1][Proof]{\par{\emph{#1}.} }{\hfill$\Box$}

\newcommand{\ar}{\mathrm{ar}}

\newcommand{\lucca}[1]{#1}
\newcommand{\david}[1]{#1}
\newcommand{\lu}[1]{{\color{blue}{#1}}}
\newcommand{\lum}[1]{\marginpar{\lu{#1}}}

% \newcommand{\dam}[1]{\marginpar{\david{#1}}}
% TOOLS

\newcommand{\stef}[1]{#1} % 
\newcommand{\newP}[1]{#1} % \textcolor{Sepia}{#1}}  %NEW : PACE
\newcommand{\newA}[1]{#1} % \textcolor{red}{#1}}    %NEW : def UK
\newcommand{\newT}[1]{#1} % \textcolor{Cyan%Violet
%        }{#1}} %NEW : fix THM
\newcommand{\newR}[1]{#1} % \textcolor{Green}{#1}} %NEW: some comments from reviewers


%%% A SUUPRIMER:
\newcommand{\rmT}[1]{}
%\textcolor{Gray}{#1}} %REMOVED : fix THM
%\newcommand{\stef}[1]{\textcolor{green}{#1}}
\definecolor{orange}{rgb}{1,0.7,0}
% \newcommand{\stef}[1]{\textcolor{orange}{#1}}
% \newcommand{\david}[1]{\textcolor{blue}{#1}}
% \definecolor{newC}{rgb}{0.7,0,1}
% \newcommand{\newC}[1]{\textcolor{newC}{#1}} %couleur à supprimer, c'est juste pour bien mettre en évidence

\newcommand{\jou}[1]{\textcolor{orange}{#1}}
\newcommand{\newGen}[1]{{\color{blue}{#1}}}
\newcommand{\newI}[1]{\textcolor{Olive}{#1}}
\newcommand{\journ}[1]{\textcolor{red}{#1}}
