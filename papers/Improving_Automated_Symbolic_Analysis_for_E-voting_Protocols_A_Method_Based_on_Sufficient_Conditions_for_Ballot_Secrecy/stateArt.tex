\section{(OLD) State of the Art}
For more in-depth description of papers, see Section~\ref{sec:refs}. When I cite paper, I use that section
(not fetch all bibtex data yet) like this: (M1,CSF'16) first  paper in the listing ``Method'' (M) published at CSF'16.
I only describe works in symbolic model with a strong focus on mechanized verification for privacy.
I do not know much about early work (<2005).
I may miss some important works (see Section~\ref{sec:refs} to see papers).


%\subsection*{Summary of Section~\ref{sec:refs}}

\subsection{Security Definitions}
In symbolic model, there is almost a consensus on security definitions for eVoting (this is not the case in computational model, see (O1,S\&P'15)).
I list below all definitions. We are mostly interested in privacy definitions (and with a focus on ballot secrecy).

\paragraph*{Privacy}
\begin{enumerate}
\item Ballot Secrecy (also called vote privacy): voter's vote are never revelead
\item Receipt Freeness: a coter cannot gain information that can be used to prove to a coercer how she voted
\item Coercion-resistance: a voter cannot collaborate with a coercer to gain information that can be used to prove to him how she voted
  (example of difference with receipt-freeness: here the attaker can force the voter to cast a ballot he specifically designed)
\end{enumerate}
Remark: it has been proved that those three properties are strictly stronger taken in the reversed order (for instance, coercion-resistance
always implies receipt-freeness and there is at least one protocol satisfying the latter but not the former).

\paragraph*{Verifiability}
\begin{enumerate}
\item Individual Verifiability: voter can verify that her vote is in the BB (bulletin board)
\item Universal Verifiability: anyone can verify that the tally's outcome corresponds to the result for the public BB
\item Eligibility Verifiability (not that common): anyone can verify that all ballots in BB have been cast by an eligible voter
  (for example via a ZK: this ballot correspond to a credential in this set with no explicit link)
\end{enumerate}

\paragraph*{Fairness}
\begin{enumerate}
\item Ballot Independence: the attacker cannot cast a ballot meaningfully related to an honest voter's ballot.
  Note: this property is actually needed to have ballot secrecy (see. (M5,JCS'13).
\item No Early Result: a voter cannot change his vote once partial results (outcomes) are available
\item Pulling Out: one partial results (outcome) are available, a voter cannot abort
\end{enumerate}


\subsection{Automatic Verification}

Several papers (M2,ESORICS'16), (M4,IJCAR'14) acknowledge that current scientific challenges for the verification
of eVtoting protocols are:
 \begin{enumerate}
 \item shift from reachability to equivalence properties,
 \item less standard primitives (\eg homorphic enc., ZK proofs, reenc. mixntes),
 \item eVoting schemes are {\em parametrized} by the number of voters. And Bulletin Board (BB) or\slash and Tally have to process
   as many ballots as the nb. of voters.
 \end{enumerate}

This is why until recently (I mean June 2016), we were able to automatically verify privacy for only a very few eVoting protocols
and often with strong assumptions (exception is the simple protocol FOO\footnote{I mean that FOO was {\em completely} verified in (M7,IFIPTM'08) using ProVerif.}).
However, two recent papers (M1,CSF'16) and (M2, ESORICS'16) are improving this state of the art a lot.
They respectively (partly) address problems (1) and (3).
Problem (2) is obviously more general than eVoting and quite orthogonal to other problems.
I know that Delaune, Baelde and Gazeau work on AC functions in Akiss for instance.

(M2, ESORICS'16) basically proves a {\it small-attack} property for ballot secrecy. It addresses problem (3) since
it remains to verify the property for a fixed number of voters.
They were able to verify mixnets variants of Helios, Belenios and Prêt-à-voter.
It does not entirely solve the problem though mainly because of
the quite constrained class of eVoting schemes they consider and because one still has to verify privacy on the
scheme for a fixed number of voters (\eg they typically encountered the state explosion problem \& problem (2)).
It should also be noted that they have simplified Helios and Belenios in their case studies (\eg they only consider
the referendum case that cannot suffer from permutation attacks described in (M5,JCS'13)).

(M1, CSF'16) pushes quite a bit the diff-equivalence that ProVerif can verify towards trace equivalence.
Before that work, diff-equivalence was generally not appropriate to verify
eVoting schemes because after the voting phase, one has to permute two processes in parallel before continuing.
The problem is well-known and here is a minimal example: verification of vote privacy for
$V(id,v)=out(id);  \texttt{phase}_1; out(v)$ (see Section~\ref{sec:refs},M1 for a full explanation).
Thanks to a sound compilation, they are now able swap processes at beginning of phases. 
This allows for the first automated verification of ballot secrecy and receipt-freeness of
schemes like FOO and Lee {\textit et al.}.
Again, it does not kill the problem because: there is still a gap between trace equivalence and diff-equivalence that
may the source of problems (for ex. they had to modify how swaps are performed for Lee {\it et al.}),
they can deal only with a constrained form of replication that forbids for instance unbounded number of sessions corresponding to revote,
problem (2),
their compiler introduces an overhead (\eg for 3 phases and 3 agents to swap, the compiler generates 216 biprocesses to be verified),
\etc


To sum up, the state of the art has been improved a lot recently but many challenges remain.
I think that we are taking a direction that is orthogonal to (M1, CSF'16) but we can builds upon (M2,ESORICS'16)
in the sense that when the scheme satisfies all the hypothesis of that work then we just have to verify our conditions for
a finite version of the scheme (and that may help) and otherwise we have to verify the conditions for the unbounded version.


% \subsection{Definitions}
% \subsection{eVoting Protocols}
% \subsection{Verification}


%%% Local Variables:
%%% mode: latex
%%% TeX-master: "main"
%%% End:
