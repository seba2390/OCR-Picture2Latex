\section{Background}
\label{sec:background}

%\SHAN{I think the background section is too long, maybe we can remove the SO3, SE3 parts, only keep SIM3, ellipsoid, SDF}

Rigid body orientation, pose, and similarity are represented using the $\text{SO}(3)$, $\text{SE}(3)$, and $\text{SIM}(3)$ Lie groups, respectively, defined as:
%
\begin{gather}
\label{eq:LieGroups}
\scaleMathLine{\begin{aligned}
\text{SO}(3) &\triangleq \crl{ \bfR \in \bbR^{3 \times 3} \mid \bfR^\top\bfR = \bfI, \det(\bfR) = 1},\\
\text{SE}(3) &\triangleq \crl{ \begin{bmatrix} \bfR & \bft\\\mathbf{0}^\top & 1 \end{bmatrix} \in \bbR^{4 \times 4} \,\bigg\vert\, \bfR \in SO(3), \bft \in \bbR^3}, \\
\text{SIM}(3) &\triangleq \crl{ \begin{bmatrix} s\bfR & \bft\\\mathbf{0}^\top & 1 \end{bmatrix} \in \bbR^{4 \times 4} \,\bigg\vert\, \bfR \in SO(3), \bft \in \bbR^3, s \in \bbR}.
\end{aligned}}
\raisetag{10ex}
\end{gather}
%
We overload $\bfxi_\times$ to denote a mapping from a vector in $\bbR^3$ or $\bbR^6$ or $\bbR^7$ to the Lie algebra $\mathfrak{so}(3)$, $\mathfrak{se}(3)$, or $\mathfrak{sim}(3)$, associated with the Lie groups in \eqref{eq:LieGroups}, defined as:
%
\begin{gather}
\label{eq:LieAlgebras}
\begin{aligned}
\mathfrak{so}(3) &\triangleq \crl{\bfxi_\times = \begin{bmatrix}0 & -\xi_3 & \xi_2\\\xi_3 & 0 & -\xi_1\\-\xi_2 & \xi_1 & 0 \end{bmatrix} \,\bigg\vert\, \bfxi \in \bbR^3},\\
\mathfrak{se}(3) &\triangleq \crl{\bfxi_\times = \begin{bmatrix} \bftheta_\times & \bfrho \\ \mathbf{0}^\top & 0 \end{bmatrix}\,\bigg\vert\, \bfxi = \begin{bmatrix} \bfrho \\ \bftheta\end{bmatrix} \in \bbR^6},\\
\mathfrak{sim}(3) &\triangleq \crl{\bfxi_\times = \begin{bmatrix} \sigma \bfI + \bftheta_\times & \bfrho \\ \mathbf{0}^\top & 0 \end{bmatrix} \,\bigg\vert\, \bfxi = \begin{bmatrix} \bfrho \\ \bftheta \\ \sigma \end{bmatrix} \in \bbR^7}.
\end{aligned}
\raisetag{12ex}
\end{gather}
%
We define an infinitesimal change of a Lie group element $\bfT$ via a left perturbation $\exp\prl{\bfxi_\times}\bfT$, using the exponential map $\exp\prl{\bfxi_\times}$ to retract a Lie algebra element $\bfxi_\times$ to the Lie group. Please refer to \cite[Ch.7]{BarfootBook} or \cite{Gao2017SLAM} for details. 

The coarse shape of a rigid body is represented using a \emph{quadric shape} \cite[Ch.3]{MVGBook}, $\crl{ \bfx \in \bbR^3 \mid \underline{\bfx}^\top \bfQ \underline{\bfx} \leq 0}$, where $\underline{\bfx} \triangleq [\bfx^\top, 1]^\top$ denotes the homogeneous coordinates of $\bfx$ and $\bfQ \in \bbR^{4 \times 4}$ is a symmetric matrix. An axis-aligned ellipsoid centered at the origin:
%
\begin{equation}
\label{eq:ellipsoid}
\mathcal{E}_{\bfu} \triangleq \crl{\bfx \in \bbR^3 \mid \bfx^\top \bfU^{-\top}\bfU^{-1}\bfx \leq 1},
\end{equation}
%
where $\bfU \triangleq \diag(\bfu)$ and the elements of the vector $\bfu \in \bbR^3$ specify the lengths of the semi-axes of $\mathcal{E}_{\bfu}$. An ellipsoid $\mathcal{E}_{\bfu}$ is a special case of a quadric shape with $\bfQ = \diag(\bfU^{-2},-1)$. 
% Instead of as the collection of points $\bfx$ contained in it, 
A quadric shape can also be defined in dual form, as the set of planes $\underline{\boldsymbol{\pi}} = \bfQ\underline{\bfx}$ that are tangent to the shape surface at each $\bfx$. This dual quadric surface definition is $\crl{ \bfpi \in \bbR^3 \mid \underline{\bfpi}^\top \bfQ^* \underline{\bfpi} = 0}$, where $\bfQ^* = \adj(\bfQ)$ is the adjugate of $\bfQ$.
%\footnote{If $\bfQ$ is invertible, $\bfQ^* = \adj(\bfQ) \triangleq \det(\bfQ)\bfQ^{-1}$ can be simplified to $\bfQ^* = \bfQ^{-1} = \diag(\bfU^2, -1)$ due to the scale-invariance of the dual quadric definition.}.
A dual quadric defined by $\bfQ^*$ can be scaled, rotated, or translated by a similarity transform $\bfT \in \text{SIM}(3)$ as $\bfT \bfQ^* \bfT^\top$. Similarity, a dual quadric can be projected to a lower-dimensional space by a projection matrix $\bfP = \begin{bmatrix} \bfI & \mathbf{0} \end{bmatrix}$ as $\bfP \bfQ^* \bfP^\top$.

The fine shape of a rigid body is represented as $\crl{\bfx \in \bbR^3 \mid f(\bfx) \leq 0}$ using the \emph{signed distance field} of a set $\calS \subset \bbR^3$:
%
\begin{equation}
f(\bfx) = \begin{cases}
  -d(\bfx,\partial\calS), & \bfx \in \calS,\\
  \phantom{-}d(\bfx,\partial\calS), & \bfx \notin \calS,
\end{cases}
\end{equation}
%
where $d(\bfx,\partial\calS)$ denotes the Euclidean distance from a point $\bfx \in \bbR^3$ to the boundary $\partial\calS$ of $\calS$.























%% \subsection{SE3}

%This section introduces the necessary mathmatical background. 
%The transformation in $\text{SE}(3)$ can be expressed as:
%\begin{equation}
%\bfT \triangleq \begin{bmatrix} \bfR & \bft\\\mathbf{0}^\top & 1 \end{bmatrix} \in \text{SE}(3)
%\end{equation}
%We overload $\bftheta_\times$ to denote the mapping from an axis-angle vector $\bftheta \in \mathbb{R}^3$ to a $3 \times 3$ skew-symmetric matrix $\bftheta_\times \in \mathfrak{so}(3)$ and the mapping from a position-rotation vector $\bfxi \in \mathbb{R}^6$ to a $4 \times 4$ twist matrix $\bfxi_\times \in \mathfrak{se}(3)$. We define an infinitesimal change of pose $\bfT \in SE(3)$ using a left perturbation $\exp\prl{\bfxi_\times}\bfT \in \text{SE}(3)$ (see~\cite[Ch.7]{BarfootBook}).

%% \subsection{SIM(3)}

%We use the space $\text{SIM}(3)$ of similarity transformations to capture scale $s$ in addition to pose: 
%\begin{equation}
%\bfT \triangleq \begin{bmatrix} s\bfR & \bft\\\mathbf{0}^\top & 1 \end{bmatrix} \in \text{SIM}(3).
%\end{equation}
%We also use $\bfxi$ to denote the corresponding Lie algebra $\mathfrak{sim}(3)$, as in \cite{Gao2017SLAM}:
%\begin{equation}
%\label{eq:sim3_to_SIM3}
%\scaleMathLine[0.91]{
%\begin{aligned}
%\mathfrak{sim}(3) \triangleq
%\left\{\bfxi_\times=
%\left[\begin{array}{cc}
%{\sigma \bfI+\bfphi_\times} & {\bfrho} \\
%\mathbf{0}^\top & {0}
%\end{array}\right] \in \mathbb{R}^{4 \times 4} \;\bigg\vert\; \bfxi = \left[\begin{array}{c}
%{\bfrho} \\
%{\bfphi} \\
%{\sigma}
%\end{array}\right] \in \mathbb{R}^{7}\right\}
%\end{aligned}
%}
%\end{equation}
%We define the operator:
%%\NA{what is the operator $\underline{\bfx}^\circledcirc$?}
%%\SHAN{I don't think we will use $\underline{\bfx}^\circledcirc$ anywhere}
%\begin{equation}
%\underline{\bfx}^\odot \triangleq \begin{bmatrix} \bfI_3 & -\bfx_\times & \bfx\\ \mathbf{0}^\top & \mathbf{0}^\top & 0 \end{bmatrix} \in \mathbb{R}^{4 \times 7}
%\end{equation}




%\begin{definition*}
%The \textit{signed distance field} of a set $\calS \subset \mathbb{R}^3$ is
%\begin{equation}
%f(\bfx) = \begin{cases}
%  -d(\bfx,\partial\calS) & \bfx \in \calS\\
%  d(\bfx,\partial\calS) & \bfx \notin \calS
%\end{cases}
%\end{equation}
%where $d(\bfx,\partial\calS)$ is the distance from a point $\bfx \in \mathbb{R}^3$ to the set boundary $\partial\calS$, and we use $d$ as a shorthand notation to denote this distance.
%\end{definition*}


%\begin{definition*}
%\textit{Huber error function} \cite{Huber1964Robust} with parameter $\delta > 0$ is:
%\begin{equation}
%\rho(r) \triangleq 
%\begin{cases}
%\frac{1}{2}r^2 & |r|\leq \delta,\\
%\delta(|r|-\frac{1}{2}\delta) & \text{else}.
%\end{cases}
%\end{equation}
%\end{definition*}
%whose gradient can be computed as 
%\[
%  \frac{\partial \rho(r)}{\partial r}
%  =\left\{\begin{array}{ll}
%    r & |r| \leq \delta \\
%    \text{sign}(r)\delta  & \text{ else}. 
%    \end{array}\right.
%\] 

%An axis-aligned ellipsoid centered at the origin can be described as:
%\begin{equation}
%\mathcal{E}_{\bfu} \triangleq \crl{\bfx \mid \bfx^\top \bfU^{-\top}\bfU^{-1}\bfx \leq 1},
%\end{equation}
%where $\bfU \triangleq \diag(\bfu)$ and the elements of the vector $\bfu = [a,b,c]^{\top}$ are the lengths of the semi-axes of $\mathcal{E}_{\bfu}$. In homogeneous coordinates, $\mathcal{E}_{\bfu}$ can be represented as a quadric surface~\cite[Ch.3]{MVGBook}, $\crl{ \bfx \mid \underline{\bfx}^\top \bfQ_{\bfu} \underline{\bfx} \leq 0}$, where $\bfQ_{\bfu} = \mathbf{diag}(\bfU^{-2},-1)$. This describes the ellipsoid as a collection of points lying on its surface. 

%Alternatively, a quadric can be defined by the set of planes $\underline{\boldsymbol{\pi}} = \bfQ_{\bfu}\underline{\bfx}$ tangent to its surface at $\underline{\bfx}$. This dual quadric surface is defined as $\crl{ \bfpi \mid \underline{\bfpi}^\top \bfQ_{\bfu}^* \underline{\bfpi} = 0}$, where $\bfQ_{\bfu}^* = \mathbf{adj}(\bfQ_{\bfu})$~\footnote{If $\bfQ_{\bfu}$ is invertible, $\bfQ_{\bfu}^* = \mathbf{adj}(\bfQ_{\bfu}) = \det(\bfQ_{\bfu})\bfQ_{\bfu}^{-1}$ can be simplified to $\bfQ_{\bfu}^* = \bfQ_{\bfu}^{-1} = \diag(\bfU^2, -1)$ due to the scale-invariance of the dual quadric definition.}. A dual quadric defined by $\bfQ_{\bfu}^*$ can be transformed by $\bfT \in SE(3)$ to another reference frame as $\bfQ^* = \bfT \bfQ_{\bfu}^* \bfT^\top$, which can be projected to a lower-dimensional space by $\bfP \in \mathbb{R}^{3\times 4}$ as $\bfP \bfQ^* \bfP^\top$. 
%$\bfQ^*$ can be parameterized as:
%\begin{equation}
%\label{eq:ellipsoid}
%\begin{aligned}
%\bfQ^*
%&=
%\mathbf{T} \bfQ_{\bfu}^* \mathbf{T}^{\top}=
%\left[\begin{array}{cc}
%\mathbf{R} & \bft \\
%\mathbf{0}^{\top} & 1
%\end{array}\right]\left[\begin{array}{cc}
%\mathbf{U} \mathbf{U}^{\top} & \mathbf{0} \\
%\mathbf{0} & -1
%\end{array}\right]\left[\begin{array}{ll}
%\mathbf{R}^{\top} & \mathbf{0} \\
%\bft^{\top} & 1
%\end{array}\right] \\ 
%&=
%\begin{pmatrix} 
%\mathbf{R} \mathbf{U} \mathbf{U}^\top \mathbf{R}^\top -  \bft \bft^\top & - \bft \\ -\bft^\top & -1
%\end{pmatrix}
%\end{aligned}
%\end{equation}



