\documentclass[10pt,twocolumn,letterpaper]{article}

\usepackage{iccv}
\usepackage{times}
\usepackage{epsfig}

% Include other packages here, before hyperref.
\usepackage{amsmath,amssymb,amsfonts,amsthm,dsfont} % math
\usepackage{algorithm,algorithmicx,listings}        % algorithms
\usepackage{graphicx,tabularx,adjustbox}            % figures

\usepackage{multirow} %for multi-row in table
\usepackage{booktabs} % for midrule
\usepackage{enumitem} % for nolistsep, noitemsep,...

% If you comment hyperref and then uncomment it, you should deletayaka miyoshie
% egpaper.aux before re-running latex.  (Or just hit 'q' on the first latex
% run, let it finish, and you should be clear).
% \usepackage[pagebackref=true,breaklinks=true,letterpaper=true,colorlinks,bookmarks=false]{hyperref}

% \usepackage[accsupp]{axessibility}  % Improves PDF readability for those with disabilities.

% Commands
\def\argmin{\mathop{\arg\min}\limits}
\def\argmax{\mathop{\arg\max}\limits}
\newcommand{\longeq}[2]{\xlongequal[\!#2\!]{\!#1\!}}
% #1 = top; #2 = bottom; #3 = inequality (<,>,\leq,\geq)
\newcommand{\longineq}[3]{\overset{#1}{\underset{#2}{#3}}}
\newcommand{\indicator}{\mathds{1}}
\newcommand{\ceil}[1]{\left\lceil#1\right\rceil}
\newcommand{\floor}[1]{\left\lfloor#1\right\rfloor}
\DeclareMathOperator{\tr}{tr}
\DeclareMathOperator{\diag}{diag}
\DeclareMathOperator{\adj}{adj}
\newcommand{\TODO}[1]{{\color{red}#1}}
\newcommand{\scaleMathLine}[2][1]{\resizebox{#1\linewidth}{!}{$\displaystyle{#2}$}}
\newcommand{\scaleLine}[2]{\begingroup\fontsize{#1}{10pt}\selectfont#2\endgroup}

\newcommand{\prl}[1]{\left(#1\right)}
\newcommand{\brl}[1]{\left[#1\right]}
\newcommand{\crl}[1]{\left\{#1\right\}}
\newcommand{\bm}[1]{\boldsymbol{#1}}
\newcommand{\aug}{\fboxsep=-\fboxrule\!\!\!\fbox{\strut}\!\!\!}

% ref
% \DeclareRobustCommand\onedot{\futurelet\@let@token\@onedot}
% \def\onedot{\ifx\@let@token.\else.\null\fi\xspace}
\def\onedot{.}
\newcommand{\figref}[1]{Fig\onedot~\ref{#1}}
\newcommand{\equref}[1]{Eq\onedot~\eqref{#1}}
\newcommand{\secref}[1]{Sec\onedot~\ref{#1}}
\newcommand{\tabref}[1]{Tab\onedot~\ref{#1}}
\newcommand{\thmref}[1]{Theorem~\ref{#1}}
\newcommand{\prgref}[1]{Program~\ref{#1}}
% \newcommand{\algref}[1]{Alg\onedot~\ref{#1}}
\newcommand{\clmref}[1]{Claim~\ref{#1}}
\newcommand{\lemref}[1]{Lemma~\ref{#1}}
\newcommand{\ptyref}[1]{Property\onedot~\ref{#1}}
\newcommand{\ve}[1]{{\mathbf #1}} % for displaying a vector or matrix
\newcommand{\hua}[1]{{\mathcal #1}}
\newcommand{\scr}[1]{{\mathcal #1}}
\newcommand{\by}[2]{\ensuremath{#1 \! \times \! #2}}
\newcommand{\thickhline}{%
    \noalign {\ifnum 0=`}\fi \hrule height 1pt
    \futurelet \reserved@a \@xhline
}
\def\etal{\emph{et al.}}


% Comments
\newcommand{\NAA}[1]{\textbf{\textcolor{red}{(Nikolay: #1)}}}
\newcommand{\SHAN}[1]{\textbf{\textcolor{blue}{(shan: #1)}}}
\newcommand{\QJ}[1]{\textbf{\textcolor{blue}{(QJ: #1)}}}
\newcommand{\YYJ}[1]{\textbf{\textcolor{brown}{(YY: #1)}}}

\newcommand{\NA}[1]{$\clubsuit$\footnote{\color{red}{Nikolay}: #1}}
\newcommand{\QF}[1]{$\diamondsuit$\footnote{\color{blue}{Qiaojun}: #1}}
\newcommand{\YY}[1]{$\spadesuit$\footnote{YY: #1}}
\newcommand{\ZZ}[1]{$\heartsuit$\footnote{ZZ: #1}}

% Environments:
\theoremstyle{definition}
\newtheorem{definition}{Definition}
\newtheorem*{definition*}{Definition}
\newtheorem*{problem*}{Problem}
\newtheorem{problem}{Problem}
\newtheorem*{proposition*}{Proposition}
\newtheorem{proposition}{Proposition}


\makeatletter
\newenvironment{proof*}[1][\proofname]{\par
%   \pushQED{\qedhere}%
  \pushQED{\qed}%
  \normalfont \partopsep=\z@skip \topsep=\z@skip
  \trivlist
  \item[\hskip\labelsep
        \itshape
    #1\@addpunct{.}]\ignorespaces
}{%
  \popQED\endtrivlist\@endpefalse
}
\makeatother


%%%%%%%%%%%%%%%%%%%%%%%%%%%%%%%%%%%%%%%%%%%%%%%
% Calligraphic fonts
\newcommand{\calA}{{\cal A}}
\newcommand{\calB}{{\cal B}}
\newcommand{\calC}{{\cal C}}
\newcommand{\calD}{{\cal D}}
\newcommand{\calE}{{\cal E}}
\newcommand{\calF}{{\cal F}}
\newcommand{\calG}{{\cal G}}
\newcommand{\calH}{{\cal H}}
\newcommand{\calI}{{\cal I}}
\newcommand{\calJ}{{\cal J}}
\newcommand{\calK}{{\cal K}}
\newcommand{\calL}{{\cal L}}
\newcommand{\calM}{{\cal M}}
\newcommand{\calN}{{\cal N}}
\newcommand{\calO}{{\cal O}}
\newcommand{\calP}{{\cal P}}
\newcommand{\calQ}{{\cal Q}}
\newcommand{\calR}{{\cal R}}
\newcommand{\calS}{{\cal S}}
\newcommand{\calT}{{\cal T}}
\newcommand{\calU}{{\cal U}}
\newcommand{\calV}{{\cal V}}
\newcommand{\calW}{{\cal W}}
\newcommand{\calX}{{\cal X}}
\newcommand{\calY}{{\cal Y}}
\newcommand{\calZ}{{\cal Z}}

% Sets:
\newcommand{\setA}{\textsf{A}}
\newcommand{\setB}{\textsf{B}}
\newcommand{\setC}{\textsf{C}}
\newcommand{\setD}{\textsf{D}}
\newcommand{\setE}{\textsf{E}}
\newcommand{\setF}{\textsf{F}}
\newcommand{\setG}{\textsf{G}}
\newcommand{\setH}{\textsf{H}}
\newcommand{\setI}{\textsf{I}}
\newcommand{\setJ}{\textsf{J}}
\newcommand{\setK}{\textsf{K}}
\newcommand{\setL}{\textsf{L}}
\newcommand{\setM}{\textsf{M}}
\newcommand{\setN}{\textsf{N}}
\newcommand{\setO}{\textsf{O}}
\newcommand{\setP}{\textsf{P}}
\newcommand{\setQ}{\textsf{Q}}
\newcommand{\setR}{\textsf{R}}
\newcommand{\setS}{\textsf{S}}
\newcommand{\setT}{\textsf{T}}
\newcommand{\setU}{\textsf{U}}
\newcommand{\setV}{\textsf{V}}
\newcommand{\setW}{\textsf{W}}
\newcommand{\setX}{\textsf{X}}
\newcommand{\setY}{\textsf{Y}}
\newcommand{\setZ}{\textsf{Z}}

% Vectors
\newcommand{\bfa}{\mathbf{a}}
\newcommand{\bfb}{\mathbf{b}}
\newcommand{\bfc}{\mathbf{c}}
\newcommand{\bfd}{\mathbf{d}}
\newcommand{\bfe}{\mathbf{e}}
\newcommand{\bff}{\mathbf{f}}
\newcommand{\bfg}{\mathbf{g}}
\newcommand{\bfh}{\mathbf{h}}
\newcommand{\bfi}{\mathbf{i}}
\newcommand{\bfj}{\mathbf{j}}
\newcommand{\bfk}{\mathbf{k}}
\newcommand{\bfl}{\mathbf{l}}
\newcommand{\bfm}{\mathbf{m}}
\newcommand{\bfn}{\mathbf{n}}
\newcommand{\bfo}{\mathbf{o}}
\newcommand{\bfp}{\mathbf{p}}
\newcommand{\bfq}{\mathbf{q}}
\newcommand{\bfr}{\mathbf{r}}
\newcommand{\bfs}{\mathbf{s}}
\newcommand{\bft}{\mathbf{t}}
\newcommand{\bfu}{\mathbf{u}}
\newcommand{\bfv}{\mathbf{v}}
\newcommand{\bfw}{\mathbf{w}}
\newcommand{\bfx}{\mathbf{x}}
\newcommand{\bfy}{\mathbf{y}}
\newcommand{\bfz}{\mathbf{z}}


\newcommand{\bfalpha}{\boldsymbol{\alpha}}
\newcommand{\bfbeta}{\boldsymbol{\beta}}
\newcommand{\bfgamma}{\boldsymbol{\gamma}}
\newcommand{\bfdelta}{\boldsymbol{\delta}}
\newcommand{\bfepsilon}{\boldsymbol{\epsilon}}
\newcommand{\bfzeta}{\boldsymbol{\zeta}}
\newcommand{\bfeta}{\boldsymbol{\eta}}
\newcommand{\bftheta}{\boldsymbol{\theta}}
\newcommand{\bfiota}{\boldsymbol{\iota}}
\newcommand{\bfkappa}{\boldsymbol{\kappa}}
\newcommand{\bflambda}{\boldsymbol{\lambda}}
\newcommand{\bfmu}{\boldsymbol{\mu}}
\newcommand{\bfnu}{\boldsymbol{\nu}}
\newcommand{\bfomicron}{\boldsymbol{\omicron}}
\newcommand{\bfpi}{\boldsymbol{\pi}}
\newcommand{\bfrho}{\boldsymbol{\rho}}
\newcommand{\bfsigma}{\boldsymbol{\sigma}}
\newcommand{\bftau}{\boldsymbol{\tau}}
\newcommand{\bfupsilon}{\boldsymbol{\upsilon}}
\newcommand{\bfphi}{\boldsymbol{\phi}}
\newcommand{\bfchi}{\boldsymbol{\chi}}
\newcommand{\bfpsi}{\boldsymbol{\psi}}
\newcommand{\bfomega}{\boldsymbol{\omega}}
\newcommand{\bfxi}{\boldsymbol{\xi}}

% Matrices
\newcommand{\bfA}{\mathbf{A}}
\newcommand{\bfB}{\mathbf{B}}
\newcommand{\bfC}{\mathbf{C}}
\newcommand{\bfD}{\mathbf{D}}
\newcommand{\bfE}{\mathbf{E}}
\newcommand{\bfF}{\mathbf{F}}
\newcommand{\bfG}{\mathbf{G}}
\newcommand{\bfH}{\mathbf{H}}
\newcommand{\bfI}{\mathbf{I}}
\newcommand{\bfJ}{\mathbf{J}}
\newcommand{\bfK}{\mathbf{K}}
\newcommand{\bfL}{\mathbf{L}}
\newcommand{\bfM}{\mathbf{M}}
\newcommand{\bfN}{\mathbf{N}}
\newcommand{\bfO}{\mathbf{O}}
\newcommand{\bfP}{\mathbf{P}}
\newcommand{\bfQ}{\mathbf{Q}}
\newcommand{\bfR}{\mathbf{R}}
\newcommand{\bfS}{\mathbf{S}}
\newcommand{\bfT}{\mathbf{T}}
\newcommand{\bfU}{\mathbf{U}}
\newcommand{\bfV}{\mathbf{V}}
\newcommand{\bfW}{\mathbf{W}}
\newcommand{\bfX}{\mathbf{X}}
\newcommand{\bfY}{\mathbf{Y}}
\newcommand{\bfZ}{\mathbf{Z}}


\newcommand{\bfGamma}{\boldsymbol{\Gamma}}
\newcommand{\bfDelta}{\boldsymbol{\Delta}}
\newcommand{\bfTheta}{\boldsymbol{\Theta}}
\newcommand{\bfLambda}{\boldsymbol{\Lambda}}
\newcommand{\bfPi}{\boldsymbol{\Pi}}
\newcommand{\bfSigma}{\boldsymbol{\Sigma}}
\newcommand{\bfUpsilon}{\boldsymbol{\Upsilon}}
\newcommand{\bfPhi}{\boldsymbol{\Phi}}
\newcommand{\bfPsi}{\boldsymbol{\Psi}}
\newcommand{\bfOmega}{\boldsymbol{\Omega}}


% Blackboard Bold:
\newcommand{\bbA}{\mathbb{A}}
\newcommand{\bbB}{\mathbb{B}}
\newcommand{\bbC}{\mathbb{C}}
\newcommand{\bbD}{\mathbb{D}}
\newcommand{\bbE}{\mathbb{E}}
\newcommand{\bbF}{\mathbb{F}}
\newcommand{\bbG}{\mathbb{G}}
\newcommand{\bbH}{\mathbb{H}}
\newcommand{\bbI}{\mathbb{I}}
\newcommand{\bbJ}{\mathbb{J}}
\newcommand{\bbK}{\mathbb{K}}
\newcommand{\bbL}{\mathbb{L}}
\newcommand{\bbM}{\mathbb{M}}
\newcommand{\bbN}{\mathbb{N}}
\newcommand{\bbO}{\mathbb{O}}
\newcommand{\bbP}{\mathbb{P}}
\newcommand{\bbQ}{\mathbb{Q}}
\newcommand{\bbR}{\mathbb{R}}
\newcommand{\bbS}{\mathbb{S}}
\newcommand{\bbT}{\mathbb{T}}
\newcommand{\bbU}{\mathbb{U}}
\newcommand{\bbV}{\mathbb{V}}
\newcommand{\bbW}{\mathbb{W}}
\newcommand{\bbX}{\mathbb{X}}
\newcommand{\bbY}{\mathbb{Y}}
\newcommand{\bbZ}{\mathbb{Z}}

% This paper commands
\newcommand{\ubfu}{\underline{\bfu}}
\newcommand{\Var}{\textit{Var}}
\newcommand{\Cov}{\textit{Cov}}

%%%%%%%%%%%%%%%%%%%%%%%%%%%%%%%%%%%%%%%%%%%%%%%

\iccvfinalcopy % *** Uncomment this line for the final submission

\def\iccvPaperID{2741} % *** Enter the ICCV Paper ID here
\def\httilde{\mbox{\tt\raisebox{-.5ex}{\symbol{126}}}}

\makeatletter
\def\thanks#1{\protected@xdef\@thanks{\@thanks
        \protect\footnotetext{#1}}}
\makeatother

% Pages are numbered in submission mode, and unnumbered in camera-ready
\ificcvfinal\pagestyle{empty}\fi
\begin{document}

%%%%%%%%% TITLE
%\title{ELLIPSDF: Optimizing Encoding and Similarity Transformation of Object Signed Distance Functions from Multi-view RGB-D Sequences}
\title{ELLIPSDF: Joint Object Pose and Shape Optimization with a Bi-level Ellipsoid and Signed Distance Function Description\thanks{We gratefully acknowledge support from ARL DCIST CRA W911NF-17-2-0181 and NSF RI IIS-2007141.}}

\author{Mo Shan, Qiaojun Feng, You-Yi Jau, Nikolay Atanasov\\
University of California San Diego\\
{\tt\small \{moshan,qjfeng,yjau,natanasov\}@ucsd.edu}
}

\maketitle
\thispagestyle{empty}


%%%%%%%%% ABSTRACT
\begin{abstract}
Autonomous systems need to understand the semantics and geometry of their surroundings in order to comprehend and safely execute object-level task specifications. 
This paper proposes an expressive yet compact model for joint object pose and shape optimization, 
and an associated optimization algorithm to infer an object-level map from multi-view RGB-D camera observations. 
The model is expressive because it captures the identities, positions, orientations, and shapes of objects in the environment. 
It is compact because it relies on a low-dimensional latent representation of implicit object shape, allowing onboard storage of large multi-category object maps. 
Different from other works that rely on a single object representation format, our approach has a bi-level object model that captures both the coarse level scale as well as the fine level shape details. 
Our approach is evaluated on the large-scale real-world ScanNet dataset and compared against state-of-the-art methods.
\end{abstract}

% Sections
\IEEEraisesectionheading{\section{Introduction}}

\IEEEPARstart{V}{ision} system is studied in orthogonal disciplines spanning from neurophysiology and psychophysics to computer science all with uniform objective: understand the vision system and develop it into an integrated theory of vision. In general, vision or visual perception is the ability of information acquisition from environment, and it's interpretation. According to Gestalt theory, visual elements are perceived as patterns of wholes rather than the sum of constituent parts~\cite{koffka2013principles}. The Gestalt theory through \textit{emergence}, \textit{invariance}, \textit{multistability}, and \textit{reification} properties (aka Gestalt principles), describes how vision recognizes an object as a \textit{whole} from constituent parts. There is an increasing interested to model the cognitive aptitude of visual perception; however, the process is challenging. In the following, a challenge (as an example) per object and motion perception is discussed. 



\subsection{Why do things look as they do?}
In addition to Gestalt principles, an object is characterized with its spatial parameters and material properties. Despite of the novel approaches proposed for material recognition (e.g.,~\cite{sharan2013recognizing}), objects tend to get the attention. Leveraging on an object's spatial properties, material, illumination, and background; the mapping from real world 3D patterns (distal stimulus) to 2D patterns onto retina (proximal stimulus) is many-to-one non-uniquely-invertible mapping~\cite{dicarlo2007untangling,horn1986robot}. There have been novel biology-driven studies for constructing computational models to emulate anatomy and physiology of the brain for real world object recognition (e.g.,~\cite{lowe2004distinctive,serre2007robust,zhang2006svm}), and some studies lead to impressive accuracy. For instance, testing such computational models on gold standard controlled shape sets such as Caltech101 and Caltech256, some methods resulted $<$60\% true-positives~\cite{zhang2006svm,lazebnik2006beyond,mutch2006multiclass,wang2006using}. However, Pinto et al.~\cite{pinto2008real} raised a caution against the pervasiveness of such shape sets by highlighting the unsystematic variations in objects features such as spatial aspects, both between and within object categories. For instance, using a V1-like model (a neuroscientist's null model) with two categories of systematically variant objects, a rapid derogate of performance to 50\% (chance level) is observed~\cite{zhang2006svm}. This observation accentuates the challenges that the infinite number of 2D shapes casted on retina from 3D objects introduces to object recognition. 

Material recognition of an object requires in-depth features to be determined. A mineralogist may describe the luster (i.e., optical quality of the surface) with a vocabulary like greasy, pearly, vitreous, resinous or submetallic; he may describe rocks and minerals with their typical forms such as acicular, dendritic, porous, nodular, or oolitic. We perceive materials from early age even though many of us lack such a rich visual vocabulary as formalized as the mineralogists~\cite{adelson2001seeing}. However, methodizing material perception can be far from trivial. For instance, consider a chrome sphere with every pixel having a correspondence in the environment; hence, the material of the sphere is hidden and shall be inferred implicitly~\cite{shafer2000color,adelson2001seeing}. Therefore, considering object material, object recognition requires surface reflectance, various light sources, and observer's point-of-view to be taken into consideration.


\subsection{What went where?}
Motion is an important aspect in interpreting the interaction with subjects, making the visual perception of movement a critical cognitive ability that helps us with complex tasks such as discriminating moving objects from background, or depth perception by motion parallax. Cognitive susceptibility enables the inference of 2D/3D motion from a sequence of 2D shapes (e.g., movies~\cite{niyogi1994analyzing,little1998recognizing,hayfron2003automatic}), or from a single image frame (e.g., the pose of an athlete runner~\cite{wang2013learning,ramanan2006learning}). However, its challenging to model the susceptibility because of many-to-one relation between distal and proximal stimulus, which makes the local measurements of proximal stimulus inadequate to reason the proper global interpretation. One of the various challenges is called \textit{motion correspondence problem}~\cite{attneave1974apparent,ullman1979interpretation,ramachandran1986perception,dawson1991and}, which refers to recognition of any individual component of proximal stimulus in frame-1 and another component in frame-2 as constituting different glimpses of the same moving component. If one-to-one mapping is intended, $n!$ correspondence matches between $n$ components of two frames exist, which is increased to $2^n$  for one-to-any mappings. To address the challenge, Ullman~\cite{ullman1979interpretation} proposed a method based on nearest neighbor principle, and Dawson~\cite{dawson1991and} introduced an auto associative network model. Dawson's network model~\cite{dawson1991and} iteratively modifies the activation pattern of local measurements to achieve a stable global interpretation. In general, his model applies three constraints as it follows
\begin{inlinelist}
	\item \textit{nearest neighbor principle} (shorter motion correspondence matches are assigned lower costs)
	\item \textit{relative velocity principle} (differences between two motion correspondence matches)
	\item \textit{element integrity principle} (physical coherence of surfaces)
\end{inlinelist}.
According to experimental evaluations (e.g.,~\cite{ullman1979interpretation,ramachandran1986perception,cutting1982minimum}), these three constraints are the aspects of how human visual system solves the motion correspondence problem. Eom et al.~\cite{eom2012heuristic} tackled the motion correspondence problem by considering the relative velocity and the element integrity principles. They studied one-to-any mapping between elements of corresponding fuzzy clusters of two consecutive frames. They have obtained a ranked list of all possible mappings by performing a state-space search. 



\subsection{How a stimuli is recognized in the environment?}

Human subjects are often able to recognize a 3D object from its 2D projections in different orientations~\cite{bartoshuk1960mental}. A common hypothesis for this \textit{spatial ability} is that, an object is represented in memory in its canonical orientation, and a \textit{mental rotation} transformation is applied on the input image, and the transformed image is compared with the object in its canonical orientation~\cite{bartoshuk1960mental}. The time to determine whether two projections portray the same 3D object
\begin{inlinelist}
	\item increase linearly with respect to the angular disparity~\cite{bartoshuk1960mental,cooperau1973time,cooper1976demonstration}
	\item is independent from the complexity of the 3D object~\cite{cooper1973chronometric}
\end{inlinelist}.
Shepard and Metzler~\cite{shepard1971mental} interpreted this finding as it follows: \textit{human subjects mentally rotate one portray at a constant speed until it is aligned with the other portray.}



\subsection{State of the Art}

The linear mapping transformation determination between two objects is generalized as determining optimal linear transformation matrix for a set of observed vectors, which is first proposed by Grace Wahba in 1965~\cite{wahba1965least} as it follows. 
\textit{Given two sets of $n$ points $\{v_1, v_2, \dots v_n\}$, and $\{v_1^*, v_2^* \dots v_n^*\}$, where $n \geq 2$, find the rotation matrix $M$ (i.e., the orthogonal matrix with determinant +1) which brings the first set into the best least squares coincidence with the second. That is, find $M$ matrix which minimizes}
\begin{equation}
	\sum_{j=1}^{n} \vert v_j^* - Mv_j \vert^2
\end{equation}

Multiple solutions for the \textit{Wahba's problem} have been published, such as Paul Davenport's q-method. Some notable algorithms after Davenport's q-method were published; of that QUaternion ESTimator (QU\-EST)~\cite{shuster2012three}, Fast Optimal Attitude Matrix \-(FOAM)~\cite{markley1993attitude} and Slower Optimal Matrix Algorithm (SOMA)~\cite{markley1993attitude}, and singular value decomposition (SVD) based algorithms, such as Markley’s SVD-based method~\cite{markley1988attitude}. 

In statistical shape analysis, the linear mapping transformation determination challenge is studied as Procrustes problem. Procrustes analysis finds a transformation matrix that maps two input shapes closest possible on each other. Solutions for Procrustes problem are reviewed in~\cite{gower2004procrustes,viklands2006algorithms}. For orthogonal Procrustes problem, Wolfgang Kabsch proposed a SVD-based method~\cite{kabsch1976solution} by minimizing the root mean squared deviation of two input sets when the determinant of rotation matrix is $1$. In addition to Kabsch’s partial Procrustes superimposition (covers translation and rotation), other full Procrustes superimpositions (covers translation, uniform scaling, rotation/reflection) have been proposed~\cite{gower2004procrustes,viklands2006algorithms}. The determination of optimal linear mapping transformation matrix using different approaches of Procrustes analysis has wide range of applications, spanning from forging human hand mimics in anthropomorphic robotic hand~\cite{xu2012design}, to the assessment of two-dimensional perimeter spread models such as fire~\cite{duff2012procrustes}, and the analysis of MRI scans in brain morphology studies~\cite{martin2013correlation}.

\subsection{Our Contribution}

The present study methodizes the aforementioned mentioned cognitive susceptibilities into a cognitive-driven linear mapping transformation determination algorithm. The method leverages on mental rotation cognitive stages~\cite{johnson1990speed} which are defined as it follows
\begin{inlinelist}
	\item a mental image of the object is created
	\item object is mentally rotated until a comparison is made
	\item objects are assessed whether they are the same
	\item the decision is reported
\end{inlinelist}.
Accordingly, the proposed method creates hierarchical abstractions of shapes~\cite{greene2009briefest} with increasing level of details~\cite{konkle2010scene}. The abstractions are presented in a vector space. A graph of linear transformations is created by circular-shift permutations (i.e., rotation superimposition) of vectors. The graph is then hierarchically traversed for closest mapping linear transformation determination. 

Despite of numerous novel algorithms to calculate linear mapping transformation, such as those proposed for Procrustes analysis, the novelty of the presented method is being a cognitive-driven approach. This method augments promising discoveries on motion/object perception into a linear mapping transformation determination algorithm.



\section{Related Work}
\label{sec:review}

% Subsequent works \cite{bloesch2018codeslam, zhi2019scenecode, sucar2020nodeslam} adopt a probabilistic formulation leading to a keyframe-based semantic mapping system that jointly estimates motion, geometry and semantics via a latent space representation\NA{latent space of what?}. 

Several RGB-D SLAM approaches \cite{newcombe2011kinectfusion, endres2012evaluation, kerl2013dense, endres20133, whelan2016elasticfusion} are able to generate accurate trajectory and a dense 3D model of the environment. However, early RGB-D SLAM techniques focus on obtaining a geometric map and overlook the semantics. 
Later, object-level SLAM approaches \cite{nicholson2018quadricslam, yang2019cubeslam} are proposed to model both geometry and semantics. Those works focus on estimating the object pose accurately, but have limited capabilities to model object shape details due to the very simple geometric shape models used, such as cuboids and quadrics.  

% commented out due to page limit 
% Subsequent attempts \cite{salas2013slam++, pillai2015monocular, slavcheva2016sdf, sunderhauf2017meaningful, nicholson2018quadricslam, xu2019mid, hu2019deep, nellithimaru2019rols, fenglocalization} model objects explicitly in RGB-D SLAM to attain both compactness and expressiveness. 
% For instance, SLAM++ \cite{salas2013slam++} enforces camera-object constraints based on prior object CAD models in the pose-graph optimization for SLAM, and the object-graph enables camera tracking via Iterative closest point (ICP). 
% An object-centric RGB-D SLAM algorithm capable of generating instance-level segmentation is proposed in \cite{sunderhauf2017meaningful}, which does not rely on a priori object models used in SLAM++. This work uses the RGB-D version of ORB-SLAM \cite{mur2017orb} for camera pose tracking and mapping geometric entities. To segment out objects from RGB-D measurements, an adjacency graph connecting nearby supervoxels are constructed, which is then partitioned into connected components to identify objects.
% Deep-SLAM++ \cite{hu2019deep} presents the first room-scale object SLAM which integrates an object shape prior into the estimation, using a Kinect V2 sensor. Instead of point clouds the paper uses scalable binary occupancy grids, but those have a severely limited resolution. 

Compared with other similar works~\cite{Mescheder_2019_CVPR, Chen_2019_CVPR} on learning implicit function for surface, DeepSDF \cite{park2019deepsdf} learns a continuous metric function of distance instead of binary classification function dividing inside or outside, which makes it suitable for gradient-based optimization in SLAM. 
Subsequent works along the direction of DeepSDF include FroDO \cite{runz2020frodo}, MOLTR \cite{li2020mo}, and DualSDF \cite{hao2020dualsdf}. 
FroDO leverages both point cloud and SDF representations, which defines sparse and dense losses to optimize the object shape. 
An extension of FroDO is MOLTR, which reconstructs an object shape by fusing multiple single-view shape codes to handle both static and dynamic objects. 
Similar to the coarse-to-fine shape estimation in FroDO and MOLTR, DualSDF uses two levels of granularity to represent 3D shapes. A shared latent space is employed to tightly couple the two levels, and a Gaussian prior is imposed on the latent space to enable sampling, interpolation, and optimization-based manipulation.  
DeepSDF and the derivatives offer models for accurate shape modeling but few of them consider object pose estimation. 

Object pose estimation is a critical step in the construction of an object level map. 
To estimate the transformation between world frame and the object frame, Scan2CAD \cite{avetisyan2019scan2cad} estimates the object pose and scale by establishing keypoint correspondences between the objects in the scene and their 3D CAD models. The keypoints are annotated for the CAD models and predicted by CNNs during inference. The Harris keypoints are detected from the 3D scan to be matched with those keypoints on the CAD models. However, both keypoint annotation and model retrieval take a long time for objects with complicated shapes, such as sofa. Later on Avetisyan \etal\cite{avetisyan2019end} dramatically increased the efficiency of the alignment process by utilizing a novel differentiable Procrustes alignment loss. Firstly, a proposed 3D CNN is used to identify objects in the 3D scan. Secondly, object bounding boxes are used to establish correspondence between scan objects and the CAD models. Lastly, alignment-informed correspondences are learnt via the differentiable Procrustes alignment loss.
Furthermore, multi-view constraints are introduced in Vid2CAD \cite{maninis2020vid2cad}. 
% Object pose estimation could also be achieved using generic geometric representations, such as cuboids in CubeSLAM \cite{yang2019cubeslam} or ellipsoids in QuadricSLAM \cite{rubino20173d, nicholson2018quadricslam}. 
% FroDO uses \cite{nicholson2018quadricslam} to initialize the object pose, but it does not take object shape prior or occlusion into account. 

In the proposed ELLIPSDF, a learnt continuous SDF is used to reconstruct the object at arbitrary resolutions, and thus our approach has a more expressive object model in comparison to \cite{hu2019deep, sucar2020nodeslam}. 
Furthermore, our model has two levels of granularity that provide a coarse object prior to optimize the object scale, which is different from FroDO or \cite{afolabi2020extending}. 
Our system is online and more efficient, and unlike prior works that focus on single object estimation, we also present a large-scale, quantitative evaluation using a public benchmark that has multiple objects. 

\section{Background and Related Work}~\label{sec:background}
%This section presents the background on MDE, ML, and MDE for systems with ML components. We further present the related work on the existing secondary and relevant studies.
\subsection{Model-driven Engineering}~\label{subsec:MDEBackground}
%The word \textit{model} originates from the Latin word \textit{modulus}, which means a measure, pattern, or example to follow~\cite{ludewig2003models}. 
%While modeling is relatively new to software engineering, it has been successfully applied for a long time in several traditional engineering domains~\cite{selic2012will,bucchiarone2020grand}. 
Model-driven Engineering (MDE) is a software development methodology that relies on models as the primary artifacts that drive the development process~\cite{ciccozzi2019execution, almonte2021recommender,hutchinson2011model}. This differs from traditional software development processes such as waterfall and agile, where the focus is on development phases like requirements engineering, design, and implementation, and models are only used as auxiliary artifacts to support these activities and serve as documentation~\cite{ciccozzi2019execution}. 
%In contrast to traditional software engineering using waterfall or agile methodology, where the focus is on the different phases of development, e.g., requirements engineering, design, implementation, and quality assurance, and the models are used to aid in requirements analysis or design, in MDE models are the primary artifact. 
The focus of MDE is on the continual refinement and transformation of models, beginning with computation-independent models (CIMs), to platform-independent models (PIMs) and then platform-specific models (PSMs)~\cite{brambilla2017model}. Finally, these models are transformed into code, documentation, configurations, and tests for the software system.

MDE relies on two key aspects: abstraction and automation~\cite{mohagheghi2009mde}. Models are abstractions of complex entities; they hide unwanted information so modelers can easily focus on areas of interest~\cite{schmidt2006model, brambilla2017model}. 
%Currently, MDE is the state-of-the-art in software abstraction~\cite{hutchinson2011model} by reducing complexity and offering a more intuitive and natural way to define software compared to programming languages~\cite{ciccozzi2019execution}. 
In MDE, models are automatically transformed into artifacts such as code, documentation, and other models to achieve various goals such as merging, translation, refinement, refactoring, or alignment~\cite{brambilla2017model}. These transformations help reduce developers' manual effort and production time by generating executable artifacts -- leading to improved software quality, reduced complexity, and decreased development time and effort~\cite{kelly2008domain}. There are two types of transformations in MDE: 1) Model-to-Text (M2T) transformations, for a given input model a M2T transformation produces a textual artifact such as code or documentation as output; and ) Model-to-model (M2M) transformations, for a given input model an M2M transformation produces a different kind of model, for example translating a model from one language to another~\cite{brambilla2017model}.

A model is created in a modeling language, conforming to a meta-model that defines the syntax and semantics of that language. There are two types of modeling languages: general-purpose languages (GPL) and domain-specific languages (DSL). GPLs are intended for modeling generic concepts applicable to multiple domains; some examples include the Unified Modeling Language (UML)~\cite{eriksson2003uml}, Petri-nets~\cite{peterson1977petri} and finite state machines~\cite{wagner2006modeling}. On the other hand, a DSL has modeling concepts tailored to a specific domain or context, like SysML for embedded systems, HTML for web page development, and SQL for database queries~\cite{brambilla2017model}.

While exploring the literature, one encounters terms similar to MDE: examples include model-driven architecture (MDA), model-driven development (MDD), and model-based engineering (MBE). MDA is an architectural standard~\cite{mda} developed by the Object Management Group (OMG) \cite{omg} for MDD. MDD refers to automatically generating artifacts from models, whereas MDE has a broader scope and includes analysis, validation~\cite{almonte2021recommender}, interoperability of artifacts and reverse engineering \cite{brambilla2017model}. MBE is a lighter version of MDE, where models are not necessarily the central focus of the engineering process; however, they provide critical support~\cite{brambilla2017model}. This SLR primarily focuses on MDE.

\subsection{Machine Learning}
Machine Learning (ML) is a branch of Artificial Intelligence (AI) that enables machines to learn patterns from data without being explicitly programmed~\cite{samuel1959machine}. ML algorithms are fed with existing data to \textit{train} them and produce an ML model. This trained ML model then has the capability to \textit{infer}, i.e., predict outcomes for new data inputs or also commonly known as \emph{ML model inference}~\cite{mueller2021machine}. For example, an ML model trained on stock prices for a company till September 2023 can predict stock prices in the following months. ML is preferable when solving problems that would require very complex and difficult-to-maintain traditional algorithms~\cite{geron2022hands}. Since ML algorithms can learn autonomously, they reduce complexity and facilitate easier maintenance~\cite{geron2022hands}. This ability of ML to minimise complexity, learn from changing data, and make future predictions is immensely valuable for businesses~\cite{lee2020machine}. According to a recent survey~\cite{rackspace2023report}, organizations report that applying ML increases employee efficiency by 20\%, innovation by 17\%, and lowers costs by 16\% -- leading to increased adoption of ML in practical settings~\cite{rackspace2023report}.

ML can further be divided into three broad categories: supervised learning, unsupervised learning, and reinforcement learning. The most suitable ML approach depends on the specific problem and data.
%
Supervised learning is when an ML algorithm is trained on a labeled dataset that has labels to define the meaning of data~\cite{mueller2021machine}. For example, a dataset with images labeled as ``cat'' or ``not cat'' images. Supervised learning algorithms learn to make classifications or predictions by learning patterns and relationships in labeled data~\cite{lee2020machine,mueller2021machine}. When the labels are discrete, this is known as \textit{classification} and when labels are continuous, this is known as \textit{regression}~\cite{mueller2021machine}. Once the algorithm is trained, the performance is evaluated on unseen or test data. Some popular supervised learning algorithms include linear regression, decision trees, naive Bayes classifier, support vector machines (SVM), random forest, and artificial neural networks (ANNs)~\cite{lee2020machine}. Supervised model applications include fraud detection and recommender systems~\cite{mueller2021machine}. 

Unsupervised learning is when an ML algorithm is trained on an unlabeled dataset with few or no labels to define the meaning of data~\cite{mueller2021machine,lee2020machine}. Unsupervised learning algorithms attempt to understand hidden patterns in data and group similar data together creating a classification of the data~\cite{mueller2021machine}. Unsupervised learning works without any guidance, hence it is most suitable for large volumes of data when classifications are unknown and data cannot be labeled~\cite{mueller2021machine}. Evaluating the performance of such algorithms can be challenging due to the lack of ground truth. Some popular unsupervised techniques include clustering, k-means, principal component analysis, and association rules~\cite{lee2020machine}. Applications of unsupervised models include customer segmentation and clustering user reviews~\cite{mueller2021machine}.

 Reinforcement learning is when an ML algorithm receives feedback on actions to guide the behavior toward an optimal outcome~\cite{mueller2021machine, lee2020machine}. Reinforcement learning algorithms are not trained with datasets; instead, they learn from trial and error in a simulated environment or a real-world environment~\cite{mueller2021machine}. Desired behaviors are rewarded and reinforcement learning algorithms attempt to maximize rewards through successful decisions~\cite{lee2020machine,mueller2021machine}. These algorithms are most suitable when sequential decision-making is required, interaction with an environment is possible and feedback is available. %However, reinforcement learning can be expensive since the algorithms require a large number of interactions with the environment to learn effectively. 
 Some popular reinforcement learning algorithms are Q-learning, temporal difference learning, hierarchal reinforcement learning, and policy gradient~\cite{lee2020machine}. Applications of reinforcement learning include robotics, self-driving cars, and game playing~\cite{lee2020machine}.
 
\subsection{Model-driven Engineering for Machine Learning (MDE4ML)}
%Models are a significant element of both MDE and ML. In MDE, models describe software systems in all phases of their life-cycle: requirements, design, implementation, testing and evolution~\cite{moin2022model}. ML models are mathematical models that learn patterns in data to make predictions~\cite{moin2022model}. 
Developing and managing systems with ML models and components is challenging. 
Some aspects of this complexity are immature requirements specification~\cite{kuwajima2020engineering, ahmad2023requirements}, constantly evolving data~\cite{baumann2022dynamic}, lack of ML domain knowledge~\cite{yohannis2022towards}, integration with traditional software \cite{atouani2021artifact}, responsible use of ML~\cite{yohannis2022towards}, and deployment and maintenance of ML models~\cite{kourouklidis2021model, langford2021modalas}. 

These complexities introduce several challenges. For example, Nils Baumann et al.~\cite{baumann2022dynamic} describe how challenging it is to handle changing datasets; ML engineers have to manually merge new and old datasets and re-train the entire ML model;  Benjamin Jahi et al. \cite{jahic2023semkis} point out how challenging it is to describe the dataset and neural network requirements to satisfy customer expectations;  Benjamin Benni et al. \cite{benni2019devops} state how the development of a correct ML pipeline is a highly demanding task, data scientists must have knowledge and experience to go through numerous data pre-processing and ML models to select the best one; and Kaan Koseler et al. \cite{koseler2019realization} mention the difficulties developers face when attempting to use ML techniques with big data, developers need to acquire knowledge of the problem space, domain and ML concepts. There is a need for solutions to efficiently and effectively address these challenges~\cite{raedler2023model}.
%All these challenges point towards the need for a technique that can efficiently and effectively address them.

A synergy between MDE and ML development exists, where software models are leveraged to drive the development and management of ML components~\cite{safdar2022modlf, yohannis2022towards, kourouklidis2021model}. This should not to be confused with AI or ML for MDE (AI4MDE), where intelligent agents and recommenders support users in modeling and related activities \cite{almonte2021recommender, gil2021artificial, boubekeur2020towards, saini2019teaching}. The application of MDE for ML-based systems (MDE4ML) offers many potential benefits to developers, such as reduced complexity~\cite{kourouklidis2021model, bucchiarone2020grand}, development effort, and time~\cite{yohannis2022towards,gatto2019modeling}. Domain experts, software engineers and ML novices can also take advantage of ML through the abstraction and automation of MDE \cite{shi2022feature,moin2022supporting, bucchiarone2020grand}. Additionally, MDE can also improve the quality of the ML-based system through easier maintainability, scalability~\cite{selic2003pragmatics}, reusability, and interoperability~\cite{brambilla2017model}.

\subsection{Key MDE4ML Related Work}
MDE4ML has received growing attention from researchers in recent years. We found six relevant secondary studies comprising SLRs, scoping reviews, and surveys. In their SLR \cite{raedler2023model}, the authors identify 15 primary studies on MDE for AI and analyze them with respect to MDE practices for the development of AI systems and the stages of AI development aligned with CRoss Industry Standard Process for Data Mining (CRISP-DM) \cite{wirth2000crisp} methodology. However, this study only considers a small subset of studies and performs a shallow analysis with no details about goals, end-users, types of models, implemented tools, and evaluation. A second SLR \cite{zafar2017systematic} reviews 24 papers on MDE for ML in the context of Big data analytics. This study has a narrower scope compared to ours and provides only a brief overview of the models, approaches, tools, and frameworks in the studies. In a third SLR \cite{li2022can}, 31 studies on no/low code platforms for ML applications are reviewed. This study is limited to no/low code approaches and therefore misses out on many other MDE for ML studies. A scoping review is presented in \cite{mardani2023model} on MDE for ML in IoT applications. The study examines 68 studies in depth; however, the review focuses more on MDE for IoT applications and only four of the selected studies apply ML techniques. A preliminary survey on DSLs for ML in Big data is presented in \cite{portugal2016preliminary}, with an extended version in \cite{portugal2016survey}. These surveys do not follow a systematic review process, include studies only for big data, and briefly highlight the DSLs and frameworks in the studies. From the analysis of existing literature, we found that the available secondary studies consist of limited subsets of papers on MDE for ML, lack analysis of key areas like goals, end-users, ML aspects, MDE approach details, evaluation methods, and limitations, and often do not follow a systematic and rigorous review process. Therefore, we aim to address these gaps in this SLR.


\section{Problem Formulation}
\label{sec:problem}


Consider an RGB-D camera whose optical frame has pose $\bfC_k \in \text{SE}(3)$ with respect to the global frame at discrete time steps $k = 1,\ldots,K$. Assume that the camera is calibrated and its pose trajectory $\crl{\bfC_k}_k$ is known, e.g., from a SLAM or SfM algorithm. At time $k$, the camera provides an RGB image $I_k : \Omega^2 \mapsto \mathbb{R}_{\geq 0}^3$ and a depth image $D_k : \Omega^2 \mapsto \mathbb{R}_{\geq 0}$ such that $I_k(\bfp)$ and $D_k(\bfp)$ are the color and depth of a pixel $\bfp \in \Omega^2$ in normalized pixel coordinates. The camera moves in an unknown environment that contains $N$ objects $\calO \triangleq \crl{\bfo_n}_{n=1}^N$. Each object $\bfo_n = (\bfc_n,\bfi_n)$ is an instance $\bfi_n$ of class $\bfc_n$, defined below.


\begin{definition*}
An \emph{object class} is a tuple $\bfc \triangleq \prl{\nu, \bfz, f_{\bftheta}, g_{\bfphi}}$, where $\nu \in \mathbb{N}$ is the class identity, e.g., chair, table, sofa, and $\bfz \in \mathbb{R}^d$ is a latent code vector, encoding the average class shape. The class shape is represented in a canonical coordinate frame at two levels of granularity: coarse and fine. The coarse shape is specified by an ellipsoid $\calE_\bfu$ in \eqref{eq:ellipsoid} with semi-axis lengths $\bfu = g_{\bfphi}(\bfz)$ decoded from the latent code $\bfz$ via a function $g_{\bfphi} : \bbR^d \mapsto \bbR^3$ with parameters $\bfphi$. The fine shape is specified by the signed distance $f_{\bftheta}(\bfx,\bfz)$ from any $\bfx \in \bbR^3$ to the average shape surface, decoded from the latent code $\bfz$ via a function $f_{\bftheta} : \bbR^3 \times \bbR^d \mapsto \bbR$ with parameters $\bftheta$.
\end{definition*}

\begin{definition*}
An \emph{object instance} of class $\bfc$ is a tuple $\bfi \triangleq \prl{\bfT, \delta\bfz}$, where $\bfT \in \text{SIM}(3)$ specifies the transformation from the global frame to the object instance frame, and $\delta\bfz \in \bbR^d$ is a deformation of the latent code $\bfz$, specifying the average shape of class $\bfc$.
\end{definition*}


\begin{figure*}[t] 
  \centering
  \includegraphics[width=\linewidth]{framework_new.jpg}
  \caption{ELLIPSDF Overview: A point cloud and initial pose (\textit{green}) are obtained from RGB-D detections of a chair instance from known camera poses (\textit{blue}). A bi-level category shape description, consisting of a latent shape code, a coarse shape decoder, and a fine shape decoder (\textit{orange}), is trained offline using a dataset of mesh models. Given the observed point cloud, the pose and shape deformation of the newly seen instance are optimized jointly online, achieving shape reconstruction in the global frame (\textit{red}).}
%   allowing shape reconstruction in the global frame (\textit{red}).}
  \label{fig:framework}
  %The point cloud of the object and the initial object pose (in the green rectange) are obtained from the RGB-D detections, including color image, depth image, instance segmentation, and fitted ellipse, and camera poses (in the blue rectangles). Then leveraging the observed object point cloud and the initial pose, \text{SIM}(3) object pose and the latent code of a two-level object model (in the orange rectangle) trained from a dataset (in the blue rectangle) are jointly optimized. The object in the world frame (in the red rectangle) could be reconstructed from the optimized object pose and latent code.}
\end{figure*}


We assume that object detection (e.g., \cite{Cai2019Cascade}) and tracking (e.g., \cite{bewley2016simple}) algorithms are available to provide the class $\bfc_n$ and pixel-wise segmentation $\Omega_{n,k}^2 \subseteq \Omega^2$ of any object $n$ observed by the camera at time $k$. %The segmentation $\Omega_{n,k}^2\triangleq \crl{ \bfp \in \Omega^2 \mid  \Delta_k(\bfp) = n}$ of object $n$ at time $k$ is obtained from labeling $\Delta_k : \Omega^2 \rightarrow \{0,\ldots,N\}$ of each pixel $\bfp \in \Omega^2$ with an object id $\Delta_k(\bfp) \in \crl{1,\ldots,N}$ or the scene background $\Delta_k(\bfp) = 0$.
Our goal is to estimate the transformation and shape $\bfi_n := (\bfT_n, \delta \bfz_n)$ of each observed object $n$. We consider object instances independently and drop the subscript $n$ when it is clear from the context. 
%\SHAN{n is still used below, do we need to say the sentence below?}
%in the reminder for clarity.

%  and the camera pose $\bfC_k$

Given an object with multi-view segmentation $\Omega_{k}^2$, we use the depth $D_k(\bfp)$ of each pixel $\bfp \in \Omega_{k}^2$ to obtain a set of points $\calX_k(\bfp)$ along the ray starting from the camera optical center and passing through $\bfp$. The sets $\calX_k(\bfp)$ is used to optimize the pose and shape of the object instance. For each ray, we choose three points, one lying on the observed surface, one a small distance $\epsilon>0$ in front of the surface, and one a small distance $\epsilon$ behind. Given $d \in \{0,\pm \epsilon\}$, we obtain points $\bfy \in \bbR^3$ in the optical frame corresponding to the pixels $\bfp \in \Omega_{k}^2$:
%
\begin{equation*}
\scaleMathLine{\calY_k(\bfp) \triangleq \crl{(\bfy, d) \,\bigg\vert\, \bfy = \prl{D_k(\bfp) + \frac{d}{\|\underline{\bfp}\|}}\underline{\bfp}, \; d \in \{0,\pm \epsilon\}},}
\end{equation*}
%
and project them to the global frame using the known camera pose $\bfC_k$:
%
\begin{equation}\label{eq:distance_measurements}
\calX_k(\bfp) \triangleq \crl{(\bfx, d) \,\bigg\vert\, \bfx = \bfP \bfC_k \underline{\bfy}, \; (\bfy,d) \in \calY_k(\bfp)}.
\end{equation}
%
%At training time, distance measurements like in \eqref{eq:distance_measurements} are obtained from several object instances of the same class and are used to optimize the shape model parameters $\bfz$, $\bftheta$, $\bfphi$, as described in Sec.~\ref{sec:train_code}. 

We define an error function $e_{\bfphi}$ to measure the discrepancy between a distance-labelled point $(\bfx,d) \in \calX_{k}(\bfp)$ observed close to the instance surface and the coarse shape $\calE_{\bfu}$ provided by $\bfu = g_{\bfphi}(\bfz)$. Another error function $e_{\bftheta}$ is used for the difference between $(\bfx,d)$ and the SDF value $f_{\bftheta}(\bfx, \bfz)$ predicted by the fine shape model. The overall error function is defined as: 
\begin{align}
\label{eq:cost_function}
&e(\bfT,\delta \bfz, \bftheta, \bfphi ; \crl{\calX_k(\bfp)}) \triangleq \alpha e_r(\delta \bfz) \\
&+ \sum_{k=1}^K 
      \sum_{\bfp \in \Omega_{k}^2}
      \sum_{(\bfx,d) \in \calX_k(\bfp)} \!\!\!\beta e_{\bftheta}(\bfx,d,\bfT,\delta \bfz) + \gamma  e_{\bfphi}(\bfx,d,\bfT,\delta \bfz)\notag,
\end{align}
where $e_r(\delta \bfz)$ is a shape deformation regularization term. The coarse-shape error, $e_{\bftheta}$, fine-shape error, $e_{\bfphi}$, and the regularization, $e_r$ are defined precisely in Sec. \ref{sec:train_code}.


We distinguish between a training phase, where we optimize the parameters $\bfz$, $\bftheta$, $\bfphi$ of an object class using offline data from instances with known mesh shapes, and a testing phase, where we optimize the pose $\bfT$ and shape deformation $\delta \bfz$ of a previously unseen instance from the same category using online distance data from an RGB-D camera.

In training, we generate points $\crl{\calX_{n,k}(\bfp)}$ close to the surface of each available mesh model $n$ in a canonical coordinate frame (with identity pose $\bfI_4$) and optimize the class shape parameters via:
%
\begin{equation}
\min_{\crl{\delta \bfz_n}, \bftheta, \bfphi} \sum_n e(\bfI_4,\delta \bfz_n, \bftheta, \bfphi ; \crl{\calX_{n,k}(\bfp)}).
\end{equation}

%During the training phase, we generate distance measurements sampled from the mesh model of an object with a canonical pose $\bfI_4$ and learn the latent code $\bfz$ as well as the decoder parameters $\bfphi$, $\bftheta$ by minimizing the cost function: 

In testing, we receive points $\crl{\calX_{k}(\bfp)}$ in the global frame, generated by the RGB-D camera from the surface of a previously unseen instance. Assuming known object class, we fix the trained shape parameters $\bfz^*$, $\bftheta^*$, $\bfphi^*$ and optimize the unknown instance transform $\bfT \in \text{SIM}(3)$ and shape deformation $\delta \bfz \in \bbR^d$:
%
\begin{equation} \label{eq:test_optimization}
\min_{\bfT, \delta \bfz} \; e(\bfT,\delta \bfz, \bftheta^*, \bfphi^* ; \crl{\calX_k(\bfp)}).
\end{equation}

























\section{Object Pose and Shape Optimization}
\label{sec:reconstruction}

This section develops ELLIPSDF, an autodecoder model for bi-level object shape representation. Sec.~\ref{sec:train_code} presents the model and defines the error functions for its parameter optimization. Sec.~\ref{sec:shape_pose_inference} describes how a trained ELLIPSDF model is used at test time for multi-view joint optimization of object pose and shape. An overview is shown in Fig.~\ref{fig:framework}.



%ELLIPSDF includes learning shape representation as well as joint pose and shape optimization, as shown in Fig.~\ref{fig:framework}. In Sec.~\ref{sec:train_code}, we introduce the two-level object model (Fig.~\ref{fig:two_level_model}). In Sec.~\ref{sec:shape_pose_inference}, we introduce our pose initialization with ellipsoids (Sec.~\ref{sec:pose_init}). Then, the object model is utilized for joint optimization with gradient descent (Sec.~\ref{sec:pose_shape_opt}). 
% uses neural networks for shape representation

% This section provides the details about how we train the ELLIPSDF two-level object model, and then covers the inference framework for multi-view object shape and pose initialization as well as optimization. Fig. \ref{fig:framework} demonstrates the framework of our approach.

%\subsection{Learning a Shared Shape Latent Space}

\subsection{Training an ELLIPSDF Model}
\label{sec:train_code}

%\subsubsection{Bi-level Shape Representation}
%\label{sec:model}

{\vspace{1ex}\bf \noindent Bi-level Shape Representation: }%
The ELLIPSDF shape model consists of two autodecoders $g_{\bfphi}(\bfz)$ and $f_{\bftheta}(\bfx,\bfz)$, using a shared latent code $\bfz \in \bbR^d$. The first autodecoder provides a \emph{coarse} shape representation with parameters $\bfphi$, as an axis-aligned ellipsoid $\calE_{\bfu}$ in a canonical coordinate frame with semi-axis lengths $\bfu = g_{\bfphi}(\bfz)$. The second autoencoder provides a \emph{fine} shape representation with parameters $\bftheta$, as an implicit SDF surface $\crl{\bfx \in \bbR^3 \mid f_{\bftheta}(\bfx,\bfz) \leq 0}$ in the same canonical coordinate frame. We implement $g_{\bfphi}(\bfz)$ and $f_{\bftheta}(\bfx,\bfz)$ as $8$-layer perceptrons with one cross-connection, as described in Sec.~D in the supplementary material of DualSDF \cite{hao2020dualsdf}. The reparametrization trick \cite{kingma2013auto} is used to maintain a Gaussian distribution $\bfz = \bfmu + \diag(\bfsigma) \bfepsilon$ over the latent code with $\bfepsilon \sim \mathcal{N}(\bf{0},\bf{I})$. Thus, at training time, the ELLIPSDF model parameters are the mean $\bfmu \in \bbR^d$ and standard deviation $\bfsigma \in \bbR^d$ of the latent shape code and the coarse and fine shape autodecoder parameters $\bfphi$ and $\bftheta$. The model is visualized in Fig.~\ref{fig:two_level_model}. 

% \TODO{The input for $g_{\bfphi}(\bfz)$ is the latent code $\bfz$. The input for $f_{\bftheta}(\bfx,\bfz)$ is the concatenation of one 3D vector $\bfx$ and the latent code $\bfz$}\NA{Perhaps we can comment this out. Seems pretty obvious.}.

%by sampling $\bfepsilon \sim \mathcal{N}(\bf{0},\bf{I})$, and setting $\bfz = \bfmu + \bfsigma \odot \bfepsilon$ in order to optimize the parameters $\bfmu$ and $\bfsigma$ via gradient descent.

%Our object model employs a two-level representation for compact object shape modeling using a shared latent code. The fine level provides shape details where as the coarse level restrains the shape scale and pose. Both levels are linked to the shared latent code such that the shape and pose can be optimized jointly from the multiview observations. A diagram visualising the framework is Fig. \ref{fig:two_level_model}.



%\subsubsection{Error Functions}
%\label{sec:errors}

{\vspace{1ex}\bf \noindent Error Functions: }%
We introduce error terms that play a key role for optimizing the category-level latent code $\bfz$ and decoder parameters $\bftheta$, $\bfphi$, during training time, as well as the transformation $\bfT$ from the global frame to the canonical object frame and the latent code deformation $\delta\bfz$ of a particular instance during test time. The training data for an ELLIPSDF model consists of distance-labeled point clouds $\calX_{n,k}(\bfp)$ associated with instances $n$ from the same class, as introduced in Sec.~\ref{sec:problem}. A different latent code $\bfz_n$ is optimized for each instance $n$, while the decoder parameters $\bftheta$ and $\bfphi$ are common for all instances of the same class. 

% The input for the error functions are generated from a training set that consists of distance measurements $\calX_{n,k}(\bfp)$ associated with object instances $n$ from the same object class, as introduced in Sec.~\ref{sec:problem}, and described in details in Sec.~\ref{sec:training_details}. 

% The distance measurements $\calX_{n,k}(\bfp)$ of instance $n$ at time $k$ are obtained from the pixel-wise instance segmentation $\bfp \in \Omega_{n,k}^2$ observed with known camera pose $\bfC_k \in \text{SE}(3)$. 

% We define an error function $e^g_{\bfphi}$ to measure the discrepancy between a distance-labelled point $(\bfx,d) \in \calX_{n,k}(\bfp)$ observed close to the instance surface and the coarse shape $\calE_{\bfu}$ provided by $\hat{\bfu} = e^g_{\bfphi}(\bfz)$.
% Another error function $e^f_{\bftheta}$ is used for the difference between a distance-labelled point $(\bfx,d) \in \calX_{n,k}(\bfp)$ observed close to the surface and the SDF value provided by $\hat{d} = f_{\bftheta}(\bfx, \bfz)$.
% The overall cost function is thus $e(\bfx,\bfphi,\bftheta,\bfT,\delta \bfz) = \beta e^g_{\bfphi}(\bfx, \bfT, \bfz + \delta \bfz) + \gamma e^f_{\bftheta}(\bfx, \bfT, \bfz + \delta \bfz)$, where $\beta, \gamma \geq 0$ are the weights. 

% of the object with respect to point $\bfx$, whose signed distance to the object surface is $d$

The fine-level shape error function $e_{\bftheta}(\bfx,d,\bfT,\delta\bfz)$ of a point $\bfx$ in global coordinates  with signed distance label $d$ is defined as:
%
\begin{equation}
  \label{eq:e_k}
  e_{\bftheta}(\bfx,d,\bfT,\delta\bfz) \triangleq \rho(s f_{\bftheta}(\bfP\bfT \underline{\bfx}; \bfz+\delta\bfz) - d).
\end{equation}
%
In the definition above, the point $\bfx$ is first transformed to the object coordinate frame via $\bfP\bfT \underline{\bfx}$ and the fine-shape model $f_{\bftheta}$ is queried with the instance shape code $\bfz + \delta\bfz$ to predict the SDF to the object surface. Since SDF values vary proportionally with scaling \cite{afolabi2020extending}, the returned value is scaled back by $s$ before measuring its discrepancy with the label $d$. Instead of measuring the difference between $s f_{\bftheta}$ and $d$ in absolute value, we employ a Huber term \cite{Huber1964Robust} to make the error function robust against outliers:
%
\begin{equation}
\label{eq:huber_loss}
\rho(r) \triangleq 
\begin{cases}
\frac{1}{2}r^2 & |r|\leq \delta,\\
\delta(|r|-\frac{1}{2}\delta) & \text{else}.
\end{cases}
\end{equation}
%
Note that the error $e_{\bftheta}$ relates both the object pose and shape to the SDF residual, which is unique to our formulation and enables their joint optimization.

The coarse-level shape error function $e_{\bfphi}(\bfx,d,\bfT,\delta\bfz)$ is defined similarly, using a signed distance function for the coarse shape. Since the coarse shape decoder, $\bfu = g_{\bfphi}(\bfz)$, provides an explicit ellipsoid description, we first need a conversion to SDF before we can define the error term. An approximation of the SDF of an ellipsoid $\calE_{\bfu}$ with semi-axis lengths $\bfu$ can be obtained as:
%
\begin{equation}
  \label{eq:ellpsoid_sdf}
  h\left(\bfx, \bfu\right)
  =
  \frac{\left\|\bfU^{-1}\bfx\right\|_{2}\left(\left\|\bfU^{-1}\bfx\right\|_{2}-1\right)}
  {\left\|\bf{U}^{-2}\bfx\right\|_{2}}.
\end{equation}
%
Then, the coarse-level shape error of a point $\bfx$ in global coordinates  with signed distance label $d$ is defined as:
%
\begin{equation}
  \label{eq:e_g}
  e_{\bfphi}(\bfx,d,\bfT,\delta\bfz) \triangleq \rho(s h(\bfP\bfT \underline{\bfx}, g_{\bfphi}(\bfz+\delta\bfz)) - d).
\end{equation} 
%
%where $\bfP = [\bfI\;\mathbf{0}] \in \mathbb{R}^{3 \times 4}$ is a projection matrix, $\bfT \in \text{SIM}(3)$ is the transformation from world frame to object frame, and $s$ is the scale. 
%Furthermore $e_{\bfphi}$ is a novel error function that has not been used to jointly optimize object pose and shape before. 

During training, the object transformation is fixed to be the canonical coordinate frame $\bfT = \bfI_4$ because the training point-cloud data is collected directly in the object frame. The regularization term $e_r(\delta\bfz)$ in \eqref{eq:cost_function} is defined as the KL divergence between the distribution of $\delta\bfz$ and a standard normal distribution \cite{hao2020dualsdf}.



%Note that during training, $\bfT = \bfI_4$ since each object is already in its canonical frame and does not require any transformation to the world frame.  


%During training, we also add the regularization term $e_r(\delta\bfz)$ as the KL divergence between the distribution of $\delta\bfz$ and a standard normal distribution. 



%Note that SDF is invariant to rotation, translation but vary proportionally with scaling \cite{afolabi2020extending}. 

%The error function $e_{\bftheta}$ relates both the object pose and shape to the SDF residual, which enables the joint optimization of both pose and shape via SDF residuals, and is different from the error functions used in DeepSDF~\cite{park2019deepsdf} and its derivatives that only depend on the shape. 

% Before introducing the coarse level error function $e_{\bfphi}$, we need to define the function $h(\cdot, \cdot)$ that computes an inexact SDF value for a point $\bfx$ with respect to an ellipsoid with shape $\bfu$ centered at origin:
% \begin{equation}
%   \label{eq:ellpsoid_sdf}
%   h\left(\bfx, \bfu\right)
%   =
%   \frac{\left\|\bfU^{-1}\bfx\right\|_{2}\left(\left\|\bfU^{-1}\bfx\right\|_{2}-1\right)}
%   {\left\|\bf{U}^{-2}\bfx\right\|_{2}}.
% \end{equation}
% Then we present the term $e_{\bfphi}(\bfx,d,\bfT,\delta\bfz)$ to depict the SDF for the coarse level shape, and is defined as 
% \begin{equation}
%   \label{eq:e_g}
%   e_{\bfphi}((\bfx,d,\bfT,\delta\bfz) \triangleq \rho(s h(\bfP\bfT \underline{\bfx}, g_{\bfphi}(\bfz+\delta\bfz)) - d),
% \end{equation} 
% where $\bfP = [\bfI\;\mathbf{0}] \in \mathbb{R}^{3 \times 4}$ is a projection matrix, $\bfT \in \text{SIM}(3)$ is the transformation from world frame to object frame, and $s$ is the scale. Note that during training, $\bfT = \bfI_4$ since each object is already in its canonical frame and does not require any transformation to the world frame.  
% Furthermore $e_{\bfphi}$ is a novel error function that has not been used to jointly optimize object pose and shape before. 

%Lastly, we employ the Huber loss \eqref{eq:huber_loss} to make the error functions robust against outliers
%\begin{equation}
%\label{eq:huber_loss}
%\rho(r) \triangleq 
%\begin{cases}
%\frac{1}{2}r^2 & |r|\leq \delta,\\
%\delta(|r|-\frac{1}{2}\delta) & \text{else}.
%\end{cases}
%\end{equation}
%During training, we also add the regularization term $e_r(\delta\bfz)$ as the KL divergence between the distribution of $\delta\bfz$ and a standard normal distribution. 
% such that the $\delta\bfz$ is generated from a compact distribution.






\begin{figure}[t]
    \centering
    \includegraphics[width=\linewidth]{dualSDF_ellipsoid.png}
    \caption{Overview of our ELLIPSDF bi-level object shape model. A latent shape code, $\bfz$, with distribution $\calN(\bfmu,\diag(\bfsigma)^2)$ is shared by a coarse shape decoder $g_\phi$, providing an ellipsoid shape description, and a fine shape decoder $f_\theta$, providing an SDF shape description. During training, the decoder parameters $\phi$ and $\theta$ are optimized by minimizing the errors between the SDF values of the training points $\bfx$, obtained close to the object surface, and the coarse and fine shape models.}
    %decoded by $g_\phi$ for the ellipsoid shape $\bfu$ and $f_\theta$ for the object SDF. The network parameters $\phi$, $\theta$ are optimized by minimizing the loss between the predicted and ground truth SDF for each sampled point $\bfx$.}
    \label{fig:two_level_model}
\end{figure}



% The latent code can be learnt by minimizing the Huber loss defined in \eqref{eq:huber_loss} between the predicted and the real SDF in \eqref{eq:cost_function}. 

% \begin{definition}
%   \textit{Huber error function} \cite{Huber1964Robust} with parameter $\delta > 0$ is:
%   \begin{equation}
%   \label{eq:huber_loss}
%   \rho(r) \triangleq 
%   \begin{cases}
%   \frac{1}{2}r^2 & |r|\leq \delta,\\
%   \delta(|r|-\frac{1}{2}\delta) & \text{else}.
%   \end{cases}
%   \end{equation}
% \end{definition}
% whose gradient can be computed as 
% \[
%   \frac{\partial \rho(r)}{\partial r}
%   =\left\{\begin{array}{ll}
%     r & |r| \leq \delta \\
%     \text{sign}(r)\delta  & \text{ else}. 
%     \end{array}\right.
% \] 







% \subsubsection{Training Process}
% \label{sec:training}

% % The shared latent space is represented by a latent code vector $\mathbf{z} \in \mathbb{R}^{d}$, which can be decoded by two generative neural networks $\bfu = g_{\bfphi}\left(\mathbf{z}\right)$, and $f_{\bftheta}\left(\bfx; \mathbf{z}\right) \approx f(\bfx), \forall x \in \calS$, where $\bfu$ represents an ellipsoid shape and $f_{\bftheta}$ is an approximator of a given SDF $f(\bfx)$\NA{some of this stuff is repeated with the problem statement section.}.
% % The ellipsoid constrains the scale of the SDF model, whereas the SDF model provides details for the object shape. 
% % A diagram visualising the framework is Fig. \ref{fig:two_level_model}. 

% This section describes how to train the two-level shape code. 
% To prepare the training data, we sample pairs of 3D points $\bfX = \{\bfx_t\}_t$ and the corresponding groundtruth SDF values $\{ d_t \}_t$ from the object meshes in a database such as ShapeNet \cite{chang2015shapenet}. 


% The training loss $L_t(\delta\bfz)$ is formulated as\NA{Define this term only once in the problem statement section and reuse it here} 
% \begin{equation}
%   \label{eq:train_loss}
%   \begin{aligned}
%     L_t(\delta\bfz) = 
%     \alpha e_r(\delta\bfz) & + 
%     \beta \sum_{\bfx_t, d_t} 
%     e^g_{k}(\bfx_t,d_t,\bfI_4,\delta\bfz) 
%     \\ 
%     &+  \gamma \sum_{\bfx_t, d_t} e_{\bftheta}(\bfx_t,d_t,\bfI_4,\delta\bfz)
%   \end{aligned}
% \end{equation}
% in which $\bfT = \bfI_4$ since during training the canonical coordinate frame of the object class is known, and thus the training focuses on the coarse and fine shape decoders. Here we define $e_r(\delta\bfz)$ as the KL divergence between the distribution of $\delta\bfz$ and a standard normal distribution.





\subsection{Joint Pose and Shape Optimization with an ELLIPSDF Model}
\label{sec:shape_pose_inference}

% (Sec.~\ref{sec:pose_init}) 
% (Sec.~\ref{sec:pose_shape_opt}) 
This section describes how a trained ELLIPSDF model is used to initialize and optimize the pose and shape of a new object instance at test time.


%\subsubsection{Initialization}
%\label{sec:pose_init}

{\vspace{1ex}\bf \noindent Initialization: }%
We follow \cite{crocco2016structure, rubino20173d, gay2018visual} to initialize the $\text{SIM}(3)$ scale and pose of an observed object, relying on its coarse ellipsoid shape representation. We fit ellipses to the pixel-wise segmentation $\Omega_k^2$ of an object at each time $k$:
%
\begin{equation} \label{eq:ellipse_fit}
    \crl{ \bfq \in \Omega^2 \mid (\bfq - \bfc_k)^\top \bfE_k^{-1} (\bfq - \bfc_k) \leq 1},
\end{equation}
%
where the center and symmetric matrix are obtained as $\bfc_k = \frac{1}{|\Omega_k^2|}\sum_{\bfp \in \Omega_k^2} \bfp$ and $\bfE_k = \frac{2}{|\Omega_k^2|}\sum_{\bfp \in \Omega_k^2} (\bfp-\bfc_k)(\bfp-\bfc_k)^\top$. The axes lengths are the eigenvalues $\lambda_{0}$, $\lambda_{1}$ of $\bfE_k$. The 2D quadric surface corresponding to the ellipse in \eqref{eq:ellipse_fit} and its dual are defined by the matrix $\bfH_k$ and its inverse $\bfH_k^*$:
%
\begin{equation*}
    \scaleMathLine{\bfH_k = \begin{bmatrix} \bfE_k^{-1} & -\bfE_k^{-1}\bfc_k \\ -\bfc_k^\top \bfE_k^{-1} & \bfc_k^\top \bfE_k^{-1} \bfc_k - 1\end{bmatrix}, \;\; \bfH_k^* = \begin{bmatrix}\bfE_k - \bfc_k\bfc_k^\top & -\bfc_k \\ -\bfc_k^\top & -1\end{bmatrix}.}
\end{equation*}




% The dual form of the ellipse is 
% $
% \bfH_k = 
% \left(\begin{array}{cc}
% \bfE_k & \bf{0} \\
% \bf{0}^\top & -1  
% \end{array}\right) 
% \in \mathbb{R}^{3\times3}
% $.

%Given a set of pixels with image coordinates $\left\{\left(u_{p}, v_{p}\right) \in \Omega_k^2 \right\}$, its center of mass is $\bfc_k = \frac{1}{|\Omega_k^2|}\sum_{\bfp \in \Omega_k^2} \bfp$. We could fit an ellipse represented by $\bfE_k = \frac{2}{|\Omega_k^2|}\sum_{\bfp \in \Omega_k^2} (\bfp-\bfc_k)(\bfp-\bfc_k)^\top$. 
% \[
%   \scaleMathLine[0.9]{
%     \left\{(u, v) \in \mathbb{R}^{2} \mid\left[\begin{array}{l}
%       u-c_{u} \\
%       v-c_{v}
%       \end{array}\right]^{\top}\left[\begin{array}{ll}
%       E_{u u} & E_{u v} \\
%       E_{u v} & E_{v v}
%       \end{array}\right]^{-1}\left[\begin{array}{l}
%       u-c_{u} \\
%       v-c_{v}
%       \end{array}\right] \leq 1
%     \right\}
%   }
% \]  
% where $E_{u u}=\frac{2}{N_{p}} \sum_{p}\left(u_{p}-c_{u}\right)^{2}$, $E_{v v}=\frac{2}{N_{p}} \sum_{p}\left(v_{p}-c_{v}\right)^{2}$, $E_{u v}=\frac{2}{N_{p}} \sum_{p}\left(u_{p}-c_{u}\right)\left(v_{p}-c_{v}\right)$. 
%The axes lengths are the eigenvalues $\lambda_{0}, \lambda_{1}$ of $\bfE_k$. The dual form of the ellipse is $\bfH_k = \text{adj}(\bfE_k)$\NA{This is not true. Look at how the ellipse is defined using $\bfE_k^{-1}$ which is both inverted and a $2 \times 2$ matrix.}
% $
% \left[\begin{array}{ll}
%   E_{u u} & E_{u v} \\
%   E_{u v} & E_{v v}
% \end{array}\right]
% $.
% The dual form of ellipse is the conic and can be obtained by the adjoint operator.\NA{I am not sure what this sentence is saying. How is $\bfH_k$ related to $\bfE_k$?} 
% , \beta = (\beta_{1}, \beta_{2},..., \beta_{k})^\top

An ellipsoid in dual quadric form $\bfQ^*$ in global coordinates and its conic projection $\bfH_k^*$ in image $k$ are related by $\beta_{k} \mathbf{H}_{k}^*=\mathbf{P} \bfC_k^{-1} \mathbf{Q}^* \bfC_k^{-\top} \mathbf{P}^{\top}$ defined up to a scale factor $\beta_{k}$. This equation can be rearranged to $\beta_{k} \mathbf{h}_{k} = \mathbf{G}_k\mathbf{v}$, where $\mathbf{h}_{k} = \operatorname{vech}(\mathbf{H}_{k}^*)$, $\mathbf{h}_{k} \in \mathbb{R}^6$, $\mathbf{v} = \operatorname{vech}(\mathbf{Q}^*)$ and $\mathbf{v} \in \mathbb{R}^{10}$. The operator $\operatorname{vech}$ serializes the lower triangular part of a symmetric matrix, and $\mathbf{G}_k$ is a matrix that depends on $\mathbf{P} \bfC_k^{-1}$. The explicit form of $\mathbf{G}_k$ is derived in (5) in \cite{rubino20173d}. 
Next, a least squares system is constructed from the multi-view observations. By stacking all observations, we obtain $\mathbf{M} \mathbf{w} = \bf0$, where $\mathbf{w} = (\mathbf{v}, \beta_1,\ldots,\beta_k)^\top$, and $\bfM$ is defined in (8) in  \cite{rubino20173d}.  
This leads to a least squares system:
%
\begin{equation}
\label{eq:ellipsoid_lsq}
\hat{\mathbf{w}}=\arg \min _{\bfw}\left\|\mathbf{M} \mathbf{w}\right\|_{2}^{2} \quad \text { s.t. } \quad\|\mathbf{w}\|_{2}^{2}=1,
\end{equation}
%
which can be solved by applying SVD to $\mathbf{M}$, and taking the right singular vector associated to the minimum singular value. The constraint $\|\mathbf{w}\|_{2}^{2}=1$ avoids a trivial solution. The first $10$ entries of $\hat{\mathbf{w}}$ are $\hat{\mathbf{v}}$, which is a vectorized version of the dual ellipsoid $\hat{\mathbf{Q}}^*$ in the global frame. To avoid degenerate quadrics, a variant of the least squares system in \eqref{eq:ellipsoid_lsq} is proposed in \cite{gay2018visual}, which constrains the center of the ellipse and the reprojection of the center of the 3D ellipsoid to be close. Thus, we modify $\bfM$ using the version in (9) in \cite{gay2018visual} to improve the estimation.

The object pose $\hat{\bfT}^{-1}$ can be recovered by relating the estimated ellipsoid $\hat{\mathbf{Q}}^*$ in global coordinates to the ellipsoid $\bfQ^*_{\bfu}$ in the canonical coordinate frame predicted by the coarse shape decoder $\bfu = g_{\bfphi}(\bfz)$ using the average class shape $\bfz$: 
%from the estimated ellipsoid $\hat{\mathbf{Q}}^*$ using \eqref{eq:ellipsoid}. Since $\bfQ^*$ can be parameterized as a centered ellipsoid $\bfQ^*_{\bfu}$ transformed by the object pose:
\begin{equation*}
\begin{aligned}
\hat{\bfQ}^*\! =
\hat{\mathbf{T}}^{-1} \bfQ_{\bfu}^* \hat{\mathbf{T}}^{-\top}\!\!=
% \left[\begin{array}{cc}
% \mathbf{R} & \bft \\
% \mathbf{0}^{\top} & 1
% \end{array}\right]\left[\begin{array}{cc}
% \mathbf{U} \mathbf{U}^{\top} & \mathbf{0} \\
% \mathbf{0} & -1
% \end{array}\right]\left[\begin{array}{ll}
% \mathbf{R}^{\top} & \mathbf{0} \\
% \bft^{\top} & 1
% \end{array}\right] \\ 
% &=
\begin{bmatrix} 
\hat{s}^2 \hat{\mathbf{R}} \bfU\bfU^\top \hat{\mathbf{R}}^\top -  \hat{\bft} \hat{\bft}^\top & - \hat{\bft} \\ -\hat{\bft}^\top & -1
\end{bmatrix}.
\end{aligned}
\end{equation*}
The translation $\hat{\bft}$ can be recovered from the last column of $\hat{\bfQ}^*$.
%$\hat{\bft} = -\bfP \hat{\mathbf{Q}} [0,0,0,1]^\top$. 
To recover the rotation, note that $\bfA \triangleq \bfP\hat{\mathbf{Q}}^*\bfP^\top  + \hat{\bft} \hat{\bft}^\top = \hat{s}^2\hat{\bfR}\bfU\bfU^\top\hat{\bfR}^\top$ is a positive semidefinite matrix. Let its eigen-decomposition be $\bfA = \bfV\bfY\bfV^\top$, where $\bfY$ is a diagonal matrix containing the eigenvalues of $\bfA$. Since $\bfU\bfU^\top$ is diagonal, it follows that $\hat{\bfR} = \bfV$, while the scale $\hat{s}$ is obtained as $\hat{s} = \frac{1}{3} \sqrt{\tr(\bfU^{-1}\bfY \bfU^{-\top})}$.
%
% \begin{equation}
%   \label{eq:ellipsoid_scale}
%   \hat{s} = \frac{1}{3} \sqrt{\tr(\bfU^{-1}\bfY \bfU^{-\top})}
%   %\hat{s} = \frac{1}{3} (\frac{\sqrt{s_1}}{u_1} + \frac{\sqrt{s_2}}{u_2} + \frac{\sqrt{s_3}}{u_3}) 
% \end{equation}
%
%and $\bfU = \diag(\bfu)$ is a prior object shape obtained as the mean shape from the training set. 
% Finally, the ellipsoid $\hat{\mathbf{Q}}$ represented by $\hat{\bfU}, \hat{s}, \hat{\bfR}, \hat{\bft}$ is refined by the tangent plane residual. 
Note that although the $\text{SIM}(3)$ pose could also be recovered from the object point cloud, other outlier rejection methods are required \cite{wu2020eao} when the point cloud is noisy. 



%\subsubsection{Optimization}
%\label{sec:pose_shape_opt}

{\vspace{1ex}\bf \noindent Optimization: }%
Given the initialized instance transformation $\hat{\bfT}$ and initial shape deformation $\delta\hat{\bfz} = \mathbf{0}$, we solve the joint pose and shape optimization in \eqref{eq:test_optimization} via gradient descent. Note that the decoder parameters $\bftheta$, $\bfphi$ and the mean category shape code $\bfz$ are fixed during online inference. Since $\bfT$ is an element of the $\text{SIM}(3)$ manifold, the gradients and gradient steps need to be computed by projecting to the tangent $\text{sim}(3)$ vector space and retracting back to $\text{SIM}(3)$. We introduce local perturbations $\bfT = \exp\prl{\bfxi_\times} \hat{\bfT}$, $\delta\bfz = \delta\tilde{\bfz} + \delta\hat{\bfz}$ and derive the Jacobians of the error function in \eqref{eq:cost_function} with respect to $\bfxi$ and $\delta\tilde{\bfz}$. 



%To incorporate the proposed error functions in \eqref{eq:cost_function} in a tightly coupled localization and mapping algorithm such as~\cite{Atanasov_SemanticLocalization_IJRR15, sunderhauf2017meaningful, mur2017orb}, we could linearize the terms around the estimated camera and object states. However, to keep the presentation clear we leave this for future work, and use ground truth camera poses. Hence, in this paper we only linearize around the object instance $\hat{\bfi}$ using perturbation $\tilde{\bfi}$: $\bfT = \exp\prl{\bfxi_\times} \hat{\bfT}$, $\delta\bfz = \delta\tilde{\bfz} + \delta\hat{\bfz}$.
% \begin{equation}
% \label{eq:perturbations}
% \begin{aligned}
%   \bfT = \exp\prl{\bfxi_\times} \hat{\bfT} \quad 
% \delta\bfz = \delta\tilde{\bfz} + \delta\hat{\bfz}
% \end{aligned}
% \end{equation}
%We define the Jacobians of the pose and latent code perturbations with respect to the error functions next. Note that the parameters $\bftheta$, $\bfphi$ of the decoders, and the mean shape latent code $\bfz$ remain fixed during inference. 

\begin{proposition}
\label{prop:sdf-sim3-jacobians}
The Jacobian of $e_{\bftheta}$ in \eqref{eq:e_k} with respect to the transformation perturbation $\bfxi \in \mathfrak{sim}(3)$ is:
\begin{equation*}
% \label{eq:gk-jacobians}
\scaleMathLine[1]
{
  \begin{aligned}
    \frac{\partial e_{\bftheta}}{\partial \bfxi} 
    &= 
    \frac{\partial \rho(r)}{\partial r}
    \prl{
      \hat{s} [{\bf0}_6,1] f_{\bftheta} (\bfx,\delta\hat{\bfz})
      + 
      \hat{s} \nabla_{\bfx} f_{\bftheta}(\bfx,\delta\hat{\bfz})^\top
      \bfP \brl{\hat{\bfT} \underline{\bfx}}^\odot 
    }
    \\
    \frac{\partial e_{\bftheta}}{\partial \delta\tilde{\bfz}} 
    &= 
    \frac{\partial \rho(r)}{\partial r}
    \hat{s} \nabla_{\bfz} f_{\bftheta}(\bfx,\delta\hat{\bfz}), 
    \end{aligned}
}
\end{equation*}
where $f_{\bftheta}(\bfx,\delta\hat{\bfz}) = f_{\bftheta}(\bfP\hat{\bfT} \underline{\bfx}; \bfz+\delta\hat{\bfz})$ is defined in \eqref{eq:e_k} and $\frac{\partial \rho(r)}{\partial r}$ is the derivative of the Huber term in \eqref{eq:huber_loss} evaluated at $r = \hat{s} f_{\bftheta}(\bfx,\delta\hat{\bfz}) - d$:
\[
  \frac{\partial \rho(r)}{\partial r}
  =\left\{\begin{array}{ll}
    r & |r| \leq \delta \\
    \text{sign}(r)\delta  & \text{ else}. 
    \end{array}\right.
\] 
%The fine level autodecoder $f_{\bftheta}(\bfx,\delta\hat{\bfz}) = f_{\bftheta}(\bfP\hat{\bfT} \underline{\bfx}; \bfz+\delta\hat{\bfz})$ is defined in \eqref{eq:e_k}. 
The operator $\underline{\bfx}^\odot$ is defined as:
\begin{equation*}
\underline{\bfx}^\odot \triangleq \begin{bmatrix} \bfI_3 & -\bfx_\times & \bfx\\ \mathbf{0}^\top & \mathbf{0}^\top & 0 \end{bmatrix} \in \mathbb{R}^{4 \times 7}. 
\end{equation*}
% $r = \hat{s} f_{\bftheta}(\bfP\hat{\bfT} \bfC_k\underline{\bfx}; \bfz+\delta\hat{\bfz}) - d$, \YYJ{ abbreviated as $f_{\bftheta}(\bfx,\delta\hat{\bfz})$}.
\end{proposition}

%The first equality is proved as follows.
\begin{proof}
Using the chain rule and the product rule:
\begin{equation*}
\begin{aligned}
  \frac{\partial e_{\bftheta}}{\partial\bfxi} 
  = 
  \frac{\partial e_{\bftheta}}{\partial r}
  \frac{\partial r}{\partial\bfxi}
  = 
    \frac{\partial e_{\bftheta}}{\partial r}
    \prl{
      \frac{\partial s}{\partial\bfxi}
      f_{\bftheta} (\bfx,\delta\bfz)
      +
      s
      \frac{\partial f_{\bftheta}}{\partial{}^{}_{O}\bfx}
      \frac{\partial {}^{}_{O}\bfx}{\partial\bfxi}
    }, 
\end{aligned}
\end{equation*}
where ${}^{}_{O}\bfx = \bfP\bfT \underline{\bfx}$ is a point in the object frame. We have $
\frac{\partial s}{\partial\bfxi}
  = 
  e^\sigma [{\bf0}_6,1]
  = s [{\bf0}_6,1]
$. 
The term $s\frac{\partial f_{\bftheta}}{\partial{}^{}_{O}\bfx}$ is the gradient of the fine-level SDF decoder with respect to the input $s \nabla_{\bfx} f_{\bftheta}(\bfx, \delta\bfz)$, which could be obtained from auto-differentiation. Finally, we have:
%
\begin{equation*}
% \label{eq:sdf-sim3-jacobians-proof2b}
\begin{aligned}
{}^{}_{O} \bfx &= \bfP \bfT \underline{\bfx} \approx 
\bfP (\bfI + \bfxi_\times) \hat\bfT \underline{\bfx} \\
&=
\bfP \hat\bfT \underline{\bfx}
+
\bfP \bfxi_\times \hat\bfT \underline{\bfx} \\ 
&= 
\bfP \hat\bfT \underline{\bfx}
+
\underbrace{\bfP [\hat\bfT \underline{\bfx}]^\odot}_{\frac{\partial {}^{}_{O} \bfx}{\partial \bfxi}} \bfxi. 
% \underbrace{\bfP [\hat\bfT \underline{\bfx}]^\odot}_{Jacobian} \bfxi
\end{aligned}\qedhere
\end{equation*}
\end{proof}


In the second equality in Prop.~\ref{prop:sdf-sim3-jacobians}, the term $\frac{\partial \rho(r)}{\partial r} \hat{s} \nabla_{\bfz} f_{\bftheta}(\bfx, \delta\hat{\bfz})$ is the gradient of the fine-level SDF loss with respect to the input $\bfz$ and can be obtained via auto-differentiation. The Jacobians of the coarse-level SDF error $\frac{\partial e_{\bfphi}}{\partial \bfxi}$, $\frac{\partial e_{\bfphi}}{\partial \delta \tilde{\bfz}}$ can be obtained in a similar way. 

%Lastly, the object pose and the shape latent codes are optimized via gradient descent.
%The initialization step provides initial estimates $\bfT^0$, $\delta \bfz^0$. 

After obtaining the Jacobians, the pose and latent shape code can be optimized via:
\begin{equation*}
\begin{aligned}
\bfT^{i+1} &\triangleq 
\exp \prl{- \eta_1 
\frac{\partial e(\bfT,\delta \bfz, \bftheta^*, \bfphi^* ; \crl{\calX_k(\bfp)})}{\partial \bfxi}}
\bfT^{i}
\\
\delta \bfz^{i+1} &\triangleq 
\delta \bfz^{i} - \eta_2 
\prl{\frac{\partial e(\bfT,\delta \bfz, \bftheta^*, \bfphi^* ; \crl{\calX_k(\bfp)})}{\partial \delta \bfz}}, 
\end{aligned}
\end{equation*}
where $\eta_1, \eta_2$ are step sizes, $\delta \bfz^0 = \mathbf{0}$, and $\bfT^0 = \hat{\bfT}$ is obtained from the initialization. During optimization, we add regularization $e_r(\delta\bfz) = \|\delta\bfz\|_2^2$ to restrict the amount of latent code deformation.


\section{Evaluation} 
\label{sec.evaluation}

\subsection{Experimental setup}\label{sec.expsetting}

\textbf{Training sources.} We use the following four sources of training data to verify the effectiveness of our methods:
\begin{itemize}[leftmargin=*]
\item[-] \textbf{Query Log (AOL, ranking-based, $100k$ queries).}
This source uses the AOL query log~\cite{pass2006} as the basis for a ranking-based source, following the approach of~\cite{dehghani2017neural}.\footnote{
Distinct non-navigational queries from the AOL query log from March 1, 2006 to May 31, 2006 are selected. We randomly sample $100k$ of queries with length of at least 4. While \citeauthor{dehghani2017neural}~\cite{dehghani2017neural} used a larger number of queries to train their model, the state-of-the-art relevance matching models we evaluate do not learn term embeddings (as \cite{dehghani2017neural} does) and thus converge with fewer than $100k$ training samples.}
We retrieve ClueWeb09 documents for each query using the Indri\footnote{https://www.lemurproject.org/indri/} query likelihood~(QL) model. We fix $c^+=1$ and $c^-=10$ due to the expense of sampling documents from ClueWeb.

\item[-] \textbf{Newswire (NYT, content-based, $1.8m$ pairs).} 
We use the New York Times corpus~\cite{sandhaus2008new} as a content-based source, using headlines as pseudo queries and the corresponding content as pseudo relevant documents. We use BM25 to select the negative articles, retaining top $c^-=100$ articles for individual headlines.

\item[-] 
\textbf{Wikipedia (Wiki, content-based, $1.1m$ pairs).} 
Wikipedia article heading hierarchies and their corresponding paragraphs have been employed as a training set for the \textsc{Trec} Complex Answer Retrieval (CAR) task~\cite{Nanni2017BenchmarkFC,macavaney2018overcoming}.
We use these pairs as a content-based source, assuming that the hierarchy of headings is a relevant query for the paragraphs under the given heading.
Heading-paragraph pairs from train fold 1 of the \textsc{Trec} CAR dataset~\cite{Dietz2017} (v1.5) are used. We generate negative heading-paragraph pairs for each heading using BM25 ($c^-=100$).

\item[-] \textbf{Manual relevance judgments (WT10).}
We compare the ranking-based and content-based sources with a data source that consists of relevance judgments generated by human assessors. In particular, manual judgments from 2010 \textsc{Trec} Web Track ad-hoc task (WT10) are employed, which includes $25k$ manual relevance judgments ($5.2k$ relevant) for 50 queries (topics + descriptions, in line with~\cite{hui2017pacrr,guo2016deep}). This setting represents a new target domain, with limited (yet still substantial) manually-labeled data.

\end{itemize}


\textbf{Training neural IR models.}
We test our method using several state-of-the-art neural IR models (introduced in Section~\ref{sec.background.nir}):
PACRR~\cite{hui2017pacrr},
Conv-KNRM~\cite{convknrm}, and
KNRM~\cite{xiong2017end}.\footnote{By using these stat-of-the-art architectures, we are using stronger baselines than those used in~\cite{dehghani2017neural,Li2018JointLF}.}
We use the model architectures and hyper-parameters (e.g., kernel sizes) from the best-performing configurations presented in the original papers for all models.
All models are trained using pairwise loss for 200 iterations with 512 training samples each iteration.
We use Web Track 2011 (WT11) manual relevance judgments as validation data to select the best iteration via nDCG@20. This acts as a way of fine-tuning the model to the particular domain, and is the only place that manual relevance judgments are used during the weak supervision training process. At test time, we re-rank the top 100 Indri QL results for each query.

\textbf{Interaction filters.}
We use the 2-maximum and discriminator filters for each ranking architecture to evaluate the effectiveness of the interaction filters.
We use queries from the target domain (\textsc{Trec} Web Track 2009--14) to generate the template pair set for the target domain $T_D$.
To generate pairs for $T_D$, the top 20 results from query likelihood (QL) for individual queries on ClueWeb09 and ClueWeb12\footnote{\url{https://lemurproject.org/clueweb09.php}, \url{https://lemurproject.org/clueweb12.php}} are used to construct query-document pairs.
Note that this approach makes no use of manual relevance judgments because only query-document pairs from the QL search results are used (without regard for relevance).
We do not use query-document pairs from the target year to avoid any latent query signals from the test set. The supervised discriminator filter is validated using a held-out set of 1000 pairs. To prevent overfitting the training data, we reduce the convolutional filter sizes of PACRR and ConvKNRM to 4 and 32, respectively. We tune $c_{max}$ with the validation dataset (WT11) for each model ($100k$ to $900k$, $100k$ intervals).

\textbf{Baselines and benchmarks.}
As baselines, we use the AOL ranking-based source as a weakly supervised baseline~\cite{dehghani2017neural}, WT10 as a manual relevance judgment baseline, and BM25 as an unsupervised baseline. The two supervised baselines are trained using the same conditions as our approach, and the BM25 baselines is tuned on each testing set with Anserini~\cite{Yang2017AnseriniET}, representing the best-case performance of BM25.\footnote{Grid search: $b\in[0.05,1]$ (0.05 interval), and $k_1\in[0.2,4]$ (0.2 interval)}
We measure the performance of the models using
the \textsc{Trec} Web Track 2012--2014 (WT12--14) queries (topics + descriptions) and manual relevance judgments. These cover two target collections: ClueWeb09 and ClueWeb12.
Akin to~\cite{dehghani2017neural}, the trained models are used to
re-rank the top 100 results from a query-likelihood model (QL, Indri~\cite{strohman2005indri} version).
Following the \textsc{Trec} Web Track, we use
nDCG@20 and ERR@20 for evaluation.

\begin{table}
\scriptsize
\caption{Ranking performance when trained using content-based sources (NYT and Wiki). Significant differences compared to the baselines ([B]M25, [W]T10, [A]OL) are indicated with $\uparrow$ and $\downarrow$ (paired t-test, $p<0.05$).}\label{tab.results}
\vspace{-1em}
\begin{tabular}{llrrrrrr}
\toprule
&&\multicolumn{3}{c}{nDCG@20} \\\cmidrule(lr){3-5}
        Model &   Training & WT12 & WT13 & WT14 \\
\midrule


\multicolumn{2}{l}{BM25 (tuned w/~\cite{Yang2017AnseriniET})} & 0.1087  & 0.2176  & 0.2646 \\
\midrule
PACRR & WT10 & B$\uparrow$ 0.1628  & 0.2513  & 0.2676 \\
 & AOL & 0.1910  & 0.2608  & 0.2802 \\
\greyrule
 & NYT & \bf W$\uparrow$ B$\uparrow$ 0.2135  & \bf A$\uparrow$ W$\uparrow$ B$\uparrow$ 0.2919  & \bf W$\uparrow$ 0.3016 \\
 & Wiki & W$\uparrow$ B$\uparrow$ 0.1955  & A$\uparrow$ B$\uparrow$ 0.2881  & W$\uparrow$ 0.3002 \\
\midrule
Conv-KNRM & WT10 & B$\uparrow$ 0.1580  & 0.2398  & B$\uparrow$ 0.3197 \\
 & AOL & 0.1498  & 0.2155  & 0.2889 \\
\greyrule
 & NYT & \bf A$\uparrow$ B$\uparrow$ 0.1792  & \bf A$\uparrow$ W$\uparrow$ B$\uparrow$ 0.2904  & \bf B$\uparrow$ 0.3215 \\
 & Wiki & 0.1536  & A$\uparrow$ 0.2680  & B$\uparrow$ 0.3206 \\
\midrule
KNRM & WT10 & B$\uparrow$ 0.1764  & \bf 0.2671  & 0.2961 \\
 & AOL & \bf B$\uparrow$ 0.1782  & 0.2648  & \bf 0.2998 \\
\greyrule
 & NYT & W$\downarrow$ 0.1455  & A$\downarrow$ 0.2340  & 0.2865 \\
 & Wiki & A$\downarrow$ W$\downarrow$ 0.1417  & 0.2409  & 0.2959 \\


\bottomrule
\end{tabular}
\vspace{-2em}
\end{table}










\renewcommand{\arraystretch}{1}








\subsection{Results}\label{sec.results}
In Table~\ref{tab.results}, we present the performance of the rankers when trained using content-based sources without filtering.
In terms of absolute score, we observe that the two n-gram models (PACRR and ConvKNRM) always perform better when trained on content-based sources than when trained on the limited sample of in-domain data. When trained on NYT, PACRR performs significantly better. KNRM performs worse when trained using the content-based sources, sometimes significantly. These results suggest that these content-based training sources contain relevance signals where n-grams are useful, and it is valuable for these models to see a wide variety of n-gram relevance signals when training. The n-gram models also often perform significantly better than the ranking-based AOL query log baseline. This makes sense because BM25's rankings do not consider term position, and thus cannot capture this important indicator of relevance. This provides further evidence that content-based sources do a better job providing samples that include various notions of relevance than ranking-based sources.

When comparing the performance of the content-based training sources, we observe that the NYT source usually performs better than Wiki. We suspect that this is due to the web domain being more similar to the newswire domain than the complex answer retrieval domain. For instance, the document lengths of news articles are more similar to web documents, and precise term matches are less common in the complex answer retrieval domain~\cite{macavaney2018overcoming}.

We present filtering performance on NYT and Wiki for each ranking architecture in Table~\ref{tab:filter_results}. In terms of absolute score, the filters almost always improve the content-based data sources, and in many cases this difference is statistically significant. The one exception is for Conv-KNRM on NYT. One possible explanation is that the filters caused the training data to become too homogeneous, reducing the ranker's ability to generalize. We suspect that Conv-KNRM is particularly susceptible to this problem because of language-dependent convolutional filters; the other two models rely only on term similarity scores. We note that Wiki tends to do better with the 2max filter, with significant improvements seen for Conv-KNRM and KNRM. In thse models, the discriminator filter may be learning surface characteristics of the dataset,
rather than more valuable notions of relevance. We also note that $c_{max}$ is an important (yet easy) hyper-parameter to tune, as the optimal value varies considerably between systems and datasets.

\section{Conclusion}
We have presented a neural performance rendering system to generate high-quality geometry and photo-realistic textures of human-object interaction activities in novel views using sparse RGB cameras only. 
%
Our layer-wise scene decoupling strategy enables explicit disentanglement of human and object for robust reconstruction and photo-realistic rendering under challenging occlusion caused by interactions. 
%
Specifically, the proposed implicit human-object capture scheme with occlusion-aware human implicit regression and human-aware object tracking enables consistent 4D human-object dynamic geometry reconstruction.
%
Additionally, our layer-wise human-object rendering scheme encodes the occlusion information and human motion priors to provide high-resolution and photo-realistic texture results of interaction activities in the novel views.
%
Extensive experimental results demonstrate the effectiveness of our approach for compelling performance capture and rendering in various challenging scenarios with human-object interactions under the sparse setting.
%
We believe that it is a critical step for dynamic reconstruction under human-object interactions and neural human performance analysis, with many potential applications in VR/AR, entertainment,  human behavior analysis and immersive telepresence.





{\small
\bibliographystyle{ieee}
\bibliography{egbib}
}

\newpage 
\section*{Supplementary Material}

\subsection*{Trained Object Models}

This section provides additional visualizations for the trained object models. Training loss for the chair category is visualize in Fig.~\ref{fig:training_loss_chair}, which shows the loss is decreasing and stabilizes around 40,000 epochs. 

Fig.~\ref{fig:trained_model_chair} visualizes the rendering results for some chairs in the training set. It shows that the scale of the primitive-based representation varies proportionally with the high-resolution representation. 

\begin{figure}[thp!]
    \centering
    \includegraphics[width=\linewidth]{loss_chairs.jpg}
    \caption{Visualization of the training loss for chairs.}
    \label{fig:training_loss_chair}
\end{figure}


Fig.~\ref{fig:trained_model_sofa} visualizes the rendering results for sofas in the training set. There is a lack of shape variation since the majority of sofas have similar structure. Nevertheless, the ellispoid for the angle sofa is still different with that of other sofas. 

\begin{figure}[thp!]
    \centering
    \includegraphics[width=\linewidth]{trained_model_chair.jpg}
    \caption{Visualization of the trained object model for chairs. Upper row: coarse ellipsoid shapes regressed from $g_{\bfphi}$ and $\bfz$. Lower row: SDF object model from $f_{\bftheta}$ and $\bfz$.}
    \label{fig:trained_model_chair}
\end{figure}





\begin{figure}[thp!]
    \centering
    \includegraphics[width=\linewidth]{trained_model_sofa.jpg}
    \caption{Visualization of the trained object model for sofas. Upper row: coarse ellipsoid shapes regressed from $g_{\bfphi}$ and $\bfz$. Lower row: SDF object model from $f_{\bftheta}$ and $\bfz$.}
    \label{fig:trained_model_sofa}
\end{figure}



Fig.~\ref{fig:trained_model_table} visualizes the rendering results for tables in the training set. Similar to sofas, the variation is limited due to similar table shapes. Nonetheless, the ellipsoid for the rounded table is different from the rest. 


\begin{figure}[thp!]
    \centering
    \includegraphics[width=\linewidth]{trained_model_table.jpg}
    \caption{Visualization of the trained object model for tables. Upper row: coarse ellipsoid shapes regressed from $g_{\bfphi}$ and $\bfz$. Lower row: SDF object model from $f_{\bftheta}$ and $\bfz$.}
    \label{fig:trained_model_table}
\end{figure}



Fig.~\ref{fig:trained_model_trashbin} visualizes the rendering results for trashbins in the training set. It could be observed that the ellipsoid shape varies based on the object shape, for instance, the ellipsoid is enlongated for a tall trashbin. 


\begin{figure}[thp!]
    \centering
    \includegraphics[width=\linewidth]{trained_model_trashbin.jpg}
    \caption{Visualization of the trained object model for trashbins. Upper row: coarse ellipsoid shapes regressed from $g_{\bfphi}$ and $\bfz$. Lower row: SDF object model from $f_{\bftheta}$ and $\bfz$.}
    \label{fig:trained_model_trashbin}
\end{figure}


\begin{figure}[thp!]
    \centering
    \includegraphics[width=\linewidth]{trained_model_display.jpg}
    \caption{Visualization of the trained object model for displays. Upper row: coarse ellipsoid shapes regressed from $g_{\bfphi}$ and $\bfz$. Lower row: SDF object model from $f_{\bftheta}$ and $\bfz$.}
    \label{fig:trained_model_display}
\end{figure}

\begin{figure}[thp!]
    \centering
    \includegraphics[width=\linewidth]{trained_model_cabinet.jpg}
    \caption{Visualization of the trained object model for cabinets. Upper row: coarse ellipsoid shapes regressed from $g_{\bfphi}$ and $\bfz$. Lower row: SDF object model from $f_{\bftheta}$ and $\bfz$.}
    \label{fig:trained_model_cabinet}
\end{figure}

Fig.~\ref{fig:trained_model_display} visualizes the rendering results for displays in the training set. The ellipsoid is rounded for the thicker display and is very thin for the rest. 

Fig.~\ref{fig:trained_model_cabinet} visualizes the rendering results for cabinets in the training set. The ellipsoid varies according to the different cabinet shapes. 



\subsection*{More Qualitative Results on ScanNet}

This section presents more qualitative results on ScanNet~\cite{dai2017scannet}. 
Fig.~\ref{fig:scannet_qualitative_0077_01} shows a reconstruction with table, trashbins, and cabinet. The cabinet and trashbins are reconstructed well, as can be seen from the resulting meshes which resemble the original object shapes. However, the table is poorly reconstructed, since the shape is quite different and the pose is inaccurate. This is because the available observation in the scene for the table is very limited, as can be seen in the segmented mesh, which is insufficient for optimization. 



A ScanNet scene with bookshelves and tables are shown in Fig.~\ref{fig:scannet_qualitative_0208_00}, to demonstrate the usefulness of the coarse and fine level residuals. The figure illustrates that the initialized object pose and shape are different from the actual scene, since the two bookshelves in the center are not parallel and are too small compared to the observation. In contrast, the bookshelves become larger after applying the fine level residual, which is more consistent with the observations. The reconstructions are further improved with both the coarse and fine level residuals, where the bookshelves become parallel. Moreover, the bottom bookshelf and the top right table also become thinner, which agrees more with the observation. 
This example clearly shows the effectiveness of the proposed bi-level model for joint object pose and shape optimization.  

\begin{figure}[thp!]
    \centering
    \includegraphics[width=\linewidth]{qualitive_0077_01.jpg}
    \caption{Visualization of the original scene and reconstructed objects for ScanNet scene $0077$. The green arrows point to the segmented mesh of the objects.}
    \label{fig:scannet_qualitative_0077_01}
\end{figure}

\begin{figure*}[thp!]
    \centering
    \includegraphics[width=\linewidth]{qualitive_0208_00.jpg}
    \caption{Visualization of the original scene and reconstructed objects for ScanNet scene $0208$. First row from left to right: original scene, reconstruction using initialized pose and mean categorical object shape, reconstruction using optimized pose and shape with fine level residual only, reconstruction using optimized pose and shape with both coarse and fine level residuals. Second row from left to right: original scene with bookshelves and tables highlighted in light blue and beige, the rest are reconstructions overlaid with object point clouds and added pseudo points.}
    \label{fig:scannet_qualitative_0208_00}
\end{figure*}

\subsection*{Pose Estimation Metric}

This section presents the metric used to evaluate the object pose, which follows Scan2CAD~\cite{avetisyan2019scan2cad}. 
We introduce the details on how to decompose a pose $\bfT \in \text{SIM}(3)$ into rotation $\bfq$, translation $\bfp$ and scale $\bfs$ and the error functions for each element separately.
For rotation and scale, $\bfR_s = \bfP\bfT\bfP^\top$:
\begin{equation}
\label{eq:pose_error}
\begin{aligned}
s_1 &= \| \bfR_s\bfe_1 \|_2 \quad 
s_2 = \| \bfR_s\bfe_2 \|_2 \quad 
s_3 = \| \bfR_s\bfe_3 \|_2, \\
\bfR\bfe_1 &= \frac{\bfR_s\bfe_1}{s_1} \quad 
\bfR\bfe_2 = \frac{\bfR_s\bfe_2}{s_2}
\quad 
\bfR\bfe_3 = \frac{\bfR_s\bfe_3}{s_3}. 
\end{aligned}
\end{equation}
Suppose $\boldsymbol{R}=\left\{m_{i j}\right\}, i, j \in[1,2,3]$, we transform it to quaternion $\bfq$ by 
\begin{equation}
\scaleMathLine[0.9]
{
\begin{aligned}
q_{0}=\frac{\sqrt{\operatorname{tr}(R)+1}}{2}, q_{1}=\frac{m_{23}-m_{32}}{4 q_{0}}, q_{2}=\frac{m_{31}-m_{13}}{4 q_{0}}, q_{3}=\frac{m_{12}-m_{21}}{4 q_{0}}. 
\end{aligned}
}
\end{equation}
Suppose the prediction and groundtruth are $\bfq_{pred}, \bfq_{gt}$, we compute the difference by 
\begin{equation}
\begin{aligned}
e_{\text{SO(3)}}(\bfq,\hat{\bfq}) := 2 \arccos (| \bfq_{gt}^\top \bfq_{pred} |). 
\end{aligned}
\end{equation}
Translation is $\bfp = \bfT[1:3, 4]$, and we compare the difference between prediction and groundtruth by 
\begin{equation}
\| \bfp_{pred} - \bfp_{gt} \|_2. 
\end{equation}
For scale percentage error, we compute it by 
\begin{equation}
100\times | \frac{1}{3} \sum_{i=1}^{3} \bar{s}_i - 1 |,
\end{equation}
where $\bar{s}_i = \frac{s_{pred}}{s_{gt}}$ for each of $s_1, s_2, s_3$ recovered from the $\text{SIM}(3)$ matrix. 

\subsection*{Timing}

\begin{table}[tph!]
    \centering
    \caption{ELLIPSDF timing breakdown (sec)}
    \label{tab:time}
    % \vspace*{-1ex}
    \scalebox{0.78}{
        \begin{tabular}{c|c|c|c|c} % <-- Alignments: 1st column left, 2nd middle and 3rd right, with vertical lines in between
        \hline
        Init & Latent Code Opt & SIM(3) Opt & SDF Decoding & Meshing \\
        \hline
        0.04 & 0.13 & 0.58 & 1.38 & 2.34 \\
        \hline
        \end{tabular}}
    % \vspace*{-1.2ex}
\end{table}

Timing for one instance is provided in Table~\ref{tab:time}. \textit{Init} is the pose initialization in (14) for 100 views. \textit{Latent Code Opt} and \textit{SIM(3) Opt} are a single SGD step with respect to $\delta \bfz$ and $\bfT$ respectively using 10000 points as batch size. \textit{SDF Decoding} and \textit{Meshing} are optional steps that generate SDF predictions over $256^3$ points and apply Marching Cubes to generate a mesh. Our approach does not currently operate in real-time but it is more efficient than existing work. We will investigate how to accelerate the current slow python SIM(3) optimization.


\section*{Acknowledgments}
The first author would like to thank Kejie Li at University of Adelaide for helpful discussions.  

\end{document}



