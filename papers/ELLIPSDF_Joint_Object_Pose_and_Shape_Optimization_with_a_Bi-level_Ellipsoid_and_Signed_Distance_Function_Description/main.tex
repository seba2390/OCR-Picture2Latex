\documentclass[10pt,twocolumn,letterpaper]{article}

\usepackage{iccv}
\usepackage{times}
\usepackage{epsfig}

% Include other packages here, before hyperref.
\usepackage{amsmath,amssymb,amsfonts,amsthm,dsfont} % math
\usepackage{algorithm,algorithmicx,listings}        % algorithms
\usepackage{graphicx,tabularx,adjustbox}            % figures

\usepackage{multirow} %for multi-row in table
\usepackage{booktabs} % for midrule
\usepackage{enumitem} % for nolistsep, noitemsep,...

% If you comment hyperref and then uncomment it, you should deletayaka miyoshie
% egpaper.aux before re-running latex.  (Or just hit 'q' on the first latex
% run, let it finish, and you should be clear).
% \usepackage[pagebackref=true,breaklinks=true,letterpaper=true,colorlinks,bookmarks=false]{hyperref}

% \usepackage[accsupp]{axessibility}  % Improves PDF readability for those with disabilities.

% Commands
\def\argmin{\mathop{\arg\min}\limits}
\def\argmax{\mathop{\arg\max}\limits}
\newcommand{\longeq}[2]{\xlongequal[\!#2\!]{\!#1\!}}
% #1 = top; #2 = bottom; #3 = inequality (<,>,\leq,\geq)
\newcommand{\longineq}[3]{\overset{#1}{\underset{#2}{#3}}}
\newcommand{\indicator}{\mathds{1}}
\newcommand{\ceil}[1]{\left\lceil#1\right\rceil}
\newcommand{\floor}[1]{\left\lfloor#1\right\rfloor}
\DeclareMathOperator{\tr}{tr}
\DeclareMathOperator{\diag}{diag}
\DeclareMathOperator{\adj}{adj}
\newcommand{\TODO}[1]{{\color{red}#1}}
\newcommand{\scaleMathLine}[2][1]{\resizebox{#1\linewidth}{!}{$\displaystyle{#2}$}}
\newcommand{\scaleLine}[2]{\begingroup\fontsize{#1}{10pt}\selectfont#2\endgroup}

\newcommand{\prl}[1]{\left(#1\right)}
\newcommand{\brl}[1]{\left[#1\right]}
\newcommand{\crl}[1]{\left\{#1\right\}}
\newcommand{\bm}[1]{\boldsymbol{#1}}
\newcommand{\aug}{\fboxsep=-\fboxrule\!\!\!\fbox{\strut}\!\!\!}

% ref
% \DeclareRobustCommand\onedot{\futurelet\@let@token\@onedot}
% \def\onedot{\ifx\@let@token.\else.\null\fi\xspace}
\def\onedot{.}
\newcommand{\figref}[1]{Fig\onedot~\ref{#1}}
\newcommand{\equref}[1]{Eq\onedot~\eqref{#1}}
\newcommand{\secref}[1]{Sec\onedot~\ref{#1}}
\newcommand{\tabref}[1]{Tab\onedot~\ref{#1}}
\newcommand{\thmref}[1]{Theorem~\ref{#1}}
\newcommand{\prgref}[1]{Program~\ref{#1}}
% \newcommand{\algref}[1]{Alg\onedot~\ref{#1}}
\newcommand{\clmref}[1]{Claim~\ref{#1}}
\newcommand{\lemref}[1]{Lemma~\ref{#1}}
\newcommand{\ptyref}[1]{Property\onedot~\ref{#1}}
\newcommand{\ve}[1]{{\mathbf #1}} % for displaying a vector or matrix
\newcommand{\hua}[1]{{\mathcal #1}}
\newcommand{\scr}[1]{{\mathcal #1}}
\newcommand{\by}[2]{\ensuremath{#1 \! \times \! #2}}
\newcommand{\thickhline}{%
    \noalign {\ifnum 0=`}\fi \hrule height 1pt
    \futurelet \reserved@a \@xhline
}
\def\etal{\emph{et al.}}


% Comments
\newcommand{\NAA}[1]{\textbf{\textcolor{red}{(Nikolay: #1)}}}
\newcommand{\SHAN}[1]{\textbf{\textcolor{blue}{(shan: #1)}}}
\newcommand{\QJ}[1]{\textbf{\textcolor{blue}{(QJ: #1)}}}
\newcommand{\YYJ}[1]{\textbf{\textcolor{brown}{(YY: #1)}}}

\newcommand{\NA}[1]{$\clubsuit$\footnote{\color{red}{Nikolay}: #1}}
\newcommand{\QF}[1]{$\diamondsuit$\footnote{\color{blue}{Qiaojun}: #1}}
\newcommand{\YY}[1]{$\spadesuit$\footnote{YY: #1}}
\newcommand{\ZZ}[1]{$\heartsuit$\footnote{ZZ: #1}}

% Environments:
\theoremstyle{definition}
\newtheorem{definition}{Definition}
\newtheorem*{definition*}{Definition}
\newtheorem*{problem*}{Problem}
\newtheorem{problem}{Problem}
\newtheorem*{proposition*}{Proposition}
\newtheorem{proposition}{Proposition}


\makeatletter
\newenvironment{proof*}[1][\proofname]{\par
%   \pushQED{\qedhere}%
  \pushQED{\qed}%
  \normalfont \partopsep=\z@skip \topsep=\z@skip
  \trivlist
  \item[\hskip\labelsep
        \itshape
    #1\@addpunct{.}]\ignorespaces
}{%
  \popQED\endtrivlist\@endpefalse
}
\makeatother


%%%%%%%%%%%%%%%%%%%%%%%%%%%%%%%%%%%%%%%%%%%%%%%
\input{sym.tex}
%%%%%%%%%%%%%%%%%%%%%%%%%%%%%%%%%%%%%%%%%%%%%%%

\iccvfinalcopy % *** Uncomment this line for the final submission

\def\iccvPaperID{2741} % *** Enter the ICCV Paper ID here
\def\httilde{\mbox{\tt\raisebox{-.5ex}{\symbol{126}}}}

\makeatletter
\def\thanks#1{\protected@xdef\@thanks{\@thanks
        \protect\footnotetext{#1}}}
\makeatother

% Pages are numbered in submission mode, and unnumbered in camera-ready
\ificcvfinal\pagestyle{empty}\fi
\begin{document}

%%%%%%%%% TITLE
%\title{ELLIPSDF: Optimizing Encoding and Similarity Transformation of Object Signed Distance Functions from Multi-view RGB-D Sequences}
\title{ELLIPSDF: Joint Object Pose and Shape Optimization with a Bi-level Ellipsoid and Signed Distance Function Description\thanks{We gratefully acknowledge support from ARL DCIST CRA W911NF-17-2-0181 and NSF RI IIS-2007141.}}

\author{Mo Shan, Qiaojun Feng, You-Yi Jau, Nikolay Atanasov\\
University of California San Diego\\
{\tt\small \{moshan,qjfeng,yjau,natanasov\}@ucsd.edu}
}

\maketitle
\thispagestyle{empty}


%%%%%%%%% ABSTRACT
\begin{abstract}
Autonomous systems need to understand the semantics and geometry of their surroundings in order to comprehend and safely execute object-level task specifications. 
This paper proposes an expressive yet compact model for joint object pose and shape optimization, 
and an associated optimization algorithm to infer an object-level map from multi-view RGB-D camera observations. 
The model is expressive because it captures the identities, positions, orientations, and shapes of objects in the environment. 
It is compact because it relies on a low-dimensional latent representation of implicit object shape, allowing onboard storage of large multi-category object maps. 
Different from other works that rely on a single object representation format, our approach has a bi-level object model that captures both the coarse level scale as well as the fine level shape details. 
Our approach is evaluated on the large-scale real-world ScanNet dataset and compared against state-of-the-art methods.
\end{abstract}

% Sections
\section{Introduction}
\label{sec:Introduction}


The goal in top-$\size$ recommendation is to recommend to each
consumer a small set of $\size$ items from a large collection of
items~\cite{cremonesi2010performance}.  For example, Netflix may want
to recommend $\size$ appealing movies to each consumer.  Collaborative
Filtering (CF)~\cite{herlocker2002empirical,lee2012comparative} is a
common top-$\size$ recommendation method.  CF infers user interests by
analyzing partially observed user-item interaction data, such as user
ratings on movies or historical purchase
logs~\cite{kanagal2012supercharging}. The main assumption in CF is that
users with similar interaction patterns have similar interests.


Standard CF methods for top-$\size$ recommendation focus on making  suggestions  that accurately reflect the user's preference history. However, as  observed in previous work,  CF recommendations are generally biased toward  popular items, leading to a rich get richer effect~\cite{vargas2014improving,steck2011item}.  The major reasons for this are \textit{popularity bias} and \textit{sparsity} of CF interaction data (detailed in Section~\ref{sec:related-work}). In a nutshell, to maintain  accuracy, recommendations are generated from the dense regions of the data,  where the popular items lie.  

However,  accurately suggesting popular items, may not be satisfactory for the consumers. For example, in Netflix, an accuracy-focused movie recommender may recommend ``Star Wars: The Force Awakens'' to users who have seen ``Star Wars: Rogue One''.  But, those users are probably already aware of ``The Force Awakens''. Considering additional factors, such as novelty of recommendations,  can lead to more effective suggestions~\cite{cremonesi2010performance,Castells2015,zhang2008avoiding,ziegler2005improving,zhang2012auralist}. 
%Second, accuracy-focused models typically achieve a   overall item-space coverage across their recommendations,  whereas high item-space coverage helps providers of the items increase revenue
%, users satisfaction since they are  likely already aware of or can find these items on their own.  

Focusing on popular items also adversely affects the satisfaction of  the providers of the items. This is because  accuracy-focused models typically achieve a  low overall item space coverage across their recommendations, whereas   high item space coverage helps providers of the items increase their revenue~\cite{vargas2014improving,Castells2015,adomavicius2011maximizing,anderson2006thelongtail, yin2012challenging,adomavicius2012improving}.
%accuracy-focused models typically achieve a

In contrast to the relatively small number of popular items, there are copious  {\it long-tail\/} items that have fewer observations (e.g., ratings) available. More precisely,  using the Pareto  principle (i.e.,~the $80/20$ rule),  long-tail items can be defined as items that generate the lower $20\%$ of observations~\cite{yin2012challenging}. Experimentally we found that these items correspond to almost $85\%$ of the items in several datasets (Sections~\ref{sec:Notation} and \ref{sec:Experiments}). %Table~\ref{tab:DatasetStatsticsSmall})


As previously shown, one way to improve the novelty of top-$\size$ sets is to recommend interesting long-tail items~\cite{cremonesi2010performance,ge2010beyond}.  The intuition  is that since they have fewer observations available,  they are more likely to be unseen~\cite{Kaminskas:2016:DSN:3028254.2926720}.  
 %For example, in online commerce,  newly added items are long-tail items that are yet to be discovered.  
Moreover, long-tail item promotion also results in higher overall coverage of the item space%, which increases profits for providers of the items
~\cite{vargas2014improving,Castells2015,zhang2008avoiding,zhang2012auralist,adomavicius2011maximizing,anderson2006thelongtail,yin2012challenging,jambor2010optimizing}. Because long-tail promotion reduces accuracy~\cite{steck2011item}, there are trade-offs to be explored.


%original submitted to ICDE
%This work studies three aspects of top-$\size$ recommendation: accuracy, novelty, and item-space coverage, and examines their trade-offs. In most previous work, predictions of a base recommendation system are re-ranked to handle their trade-offs~\cite{adomavicius2012improving,jambor2010optimizing,zhang2013personalize,wang2009portfolio}. Due to performance considerations, however, these techniques are not customized per user. For example,  parameters that balance the trade-off between novelty and accuracy are cross-validated at a global level.  This can be detrimental since users have varying preferences for  objectives such as long-tail novelty. We explore how to  automatically infer  user  preference for long-tail novelty, and how to leverage  it to correct  the popularity bias in standard recommender models. Our work does not rely on any additional contextual data, although such data, if available, can help promote newly-added long-tail items~\cite{agarwal2009regression,Saveski:2014:ICR:2645710.2645751}.

This work studies three aspects of top-$\size$ recommendation: accuracy, novelty, and item space coverage, and examines their trade-offs. In most previous work, predictions of a base recommendation algorithm are \textit{re-ranked} to handle these trade-offs~\cite{adomavicius2012improving,jambor2010optimizing,zhang2013personalize,wang2009portfolio}. The re-ranking models are computationally efficient but suffer from two drawbacks. First, due to performance considerations,  parameters that balance the trade-off between novelty and accuracy  are not customized per user. Instead they are cross-validated at a global level.  This can be detrimental since users have varying preferences for  objectives such as long-tail novelty. Second,  the re-ranking methods are often limited to a specific base recommender  that may be sensitive to dataset density. 
As a result, the datasets are pruned and the problem is studied in dense settings~\cite{adomavicius2012improving,ho2014likes}; but real world  scenarios are often sparse~\cite{kanagal2012supercharging,liu2017experimental}.   
% Because  dataset density can impact the performance of most base recommenders (like R-SVD), which in turn affects the performance of the re-ranking model, 

\iffalse
We address these limitations by directly inferring  user  preference for long-tail novelty  from interaction data.  This  allows us to customize the re-ranking  per user, and design a \textit{generic} framework, which resolves the second problem. In particular, since the long-tail novelty preferences are estimated independently of any base  recommender model, we can  plug-in an appropriate base recommender w.r.t. the dataset sparsity.% including ones that are more suitable for sparse settings.  

Modelling  user  preference for  long-tail novelty using only item popularity statistics, e.g., the average popularity of rated items as in~\cite{jugovac2017efficient}, disregards additional information like whether the user found the item interesting and the long-tail preferences of other users  of the items. \iffalse To incorporate them, we introduce the notion of  \emph{item long-tail importance}. Both  user long-tail preferences and item long-tail importance are dependent:  a user has high preference for discovering long-tail items if she is interested in important long-tail items, and an item that is associated with many of these kinds of users is likely to be more important.  We propose a joint optimization framework to directly learn,  from interaction data, both the users' long-tail preferences and the  items' long-tail importance. \fi
We propose an optimization approach that  incorporates  this information and  directly learns,  from interaction data, the users' long-tail novelty preferences.

Next, we use these learned preferences  to design a  top-$\size$ recommendation framework thats is generic, and provides customized balance between accuracy, novelty, and coverage. We refer to it as framework as GANC.  Using GANC, we design a novel algorithm, {\it Ordered Sampling-based Locally Greedy (OSLG)\/}, that relies on the learned long-tail novelty preferences  to scalably correct for popularity bias. Our work does not rely on any additional contextual data, although such data, if available, can help promote newly-added long-tail items~\cite{agarwal2009regression,Saveski:2014:ICR:2645710.2645751}. In summary:
\fi

We address the first limitation by directly inferring  user  preference for long-tail novelty  from interaction data.   Estimating these  preferences  using only item popularity statistics, e.g., the average popularity of rated items as in~\cite{jugovac2017efficient}, disregards additional information, like whether the user found the item interesting or the long-tail preferences of other users  of the items. We propose an approach that  incorporates  this information and  learns the users' long-tail novelty preferences from interaction data.

This approach allows us to customize the re-ranking  per user, and  design a \textit{generic} re-ranking framework, which resolves the second limitation of prior work. In particular, since the long-tail novelty preferences are estimated independently of any base recommender, we can  plug-in an appropriate one w.r.t. different factors, such as the dataset sparsity.

Our top-$\size$ recommendation framework, \textbf{GANC}, is \textbf{G}eneric, and provides customized balance between \textbf{A}ccuracy, \textbf{N}ovelty, and \textbf{C}overage. % Moreover, based on the learned long-tail novelty preferences, we also design a novel algorithm, {\it Ordered Sampling-based Locally Greedy (OSLG)\/}, that relies on the learned long-tail novelty preferences  to scalably correct for popularity bias. 
Our work does not rely on any additional contextual data, although such data, if available, can help promote newly-added long-tail items~\cite{agarwal2009regression,Saveski:2014:ICR:2645710.2645751}. In summary:

%Consider  the following toy example:
\vspace{-0.2cm}
\begin{table}[htb]
\centering
\scriptsize
%\small
\begin{tabular}{ccccccc} 
%\toprule
%&\multirow{2}{*}{}&\multicolumn{7}{c}{Ratings}\\
& & \cellcolor{blue!35}$w_1$ &\cellcolor{blue!18} $w_2$ & $\dots$ &\cellcolor{blue!8} $w_{89}$  &\cellcolor{blue!8} $w_{99}$   
\\
&   &$i_1$&$i_2$&$\dots$&$i_{89}$&$i_{90}$\\ 
\cmidrule(r){3-7} 	 
%\midrule
\cellcolor{red!35}$\theta_1$  &$u_1 $   &5 &   & $\dots$ &  &   \\
\cellcolor{red!28}$\theta_2$  &$u_2$     &5 &    & $\dots$ &  &  \\
 $\theta_3=?$  &$\bf u_3$  &5 &  &   $\dots$ &  &  \\
\cellcolor{red!10}$\theta_4$ & $u_4$  &  &5   & $\dots$ & &\\ 
\cellcolor{red!10}$\theta_5$ & $u_5$  &  & 5  & $\dots$ & &\\ 
$\theta_6=?$  & $\bf u_6$ & &5  &      $\dots$& &  \\ 
 & & $\hdots$  &$\hdots$   &$\hdots$   &$\hdots$   &$\hdots$  \\
%\midrule 
\cmidrule(r){3-7} 	 
\multicolumn{2}{c}{item pop.}  & 3  & 3  & $\dots$ &50&60\\  
%\bottomrule
%$ f_i$    &3  &3  &1  &3  &1  &2  \\  \hline
\end{tabular}
%#.
\caption{Simplified user-item interaction data. The user long-tail novelty preference ($\theta_u$), item long-tail importance weight ($w_i$) are highlighted. Darker colors indicate larger values. } \label{tab:example}
\end{table} 
\vspace{-0.2cm}
\begin{example}  
In Table~\ref{tab:example}, we are interested in estimating $\theta_3$ and $\theta_6$,  the long-tail preference of users $u_3$ and $u_6$ who have each rated a single movie. Additional ratings for other users  are not included here.  Considering only rating information, we observe $i_1$ and $i_2$ are  equally popular $|\mathcal{U}_{i_1}^{\trainset}| = |\mathcal{U}_{i_2}^{\trainset}|=3$, and $r_{31}=5$ and $r_{62}=5$. Using Eq.~\ref{eq:tfidf-risk}  we have $\theta_3 = \theta_6$. However, if we were given the long-tail preferences of the each item's user set, specifically that $u_1$ and $u_2$ have high long-tail preference (darker red), while $u_4$ and $u_5$ have lower long-tail preference (lighter red), we could conclude $i_1$ is a more important long-tail item compared to $i_2$ (indicated by a darker blue shade for $w_1$), and we expect  $\theta_3 \geq \theta_6$.

% On the other hand, if we knew that $u_4$ and $u_5$ have lower long-tail preference, we could conclude $i_2$ is a  less significant long-tail item. Therefore, However, if we  consider the long-tail preferences of other users, we may reason differently.    We need another variable $w_i$ which captures this information. 
%we would conclude that $u_3$ has higher long-tail preference compared to $u_6$, since the users $i_1$ is a more prominent long-tail item. 

% Relying only  on item popularity information, we would  conclude   $u_3$ and $u_6$ have equal long-tail preference, since $i_1$ and $i_2$ are  equally popular. However, considering  the second column,  long-tail preference of users,  long-tail importance for each item,  which captures the long-tail preference of its users. Since  that  both users of $i_1$ have high long-tail preference while  the users of $i_2$ have lower preference,  we may conclude $i_1$ is a more important long-tail item compared to $i_2$. Therefore, $u_3$'s long-tail preference should be at least as large as $u_6$'s preference. Specifically, consider two  items $i_1$ and $i_2$, with the following rating data: $i_1=\{u_1:5, u_2:5, u_3:5 \}$, $i_2=\{u_4:5, u_5:5, u_6:5\}$.  

%Table~\ref{tab:example} shows  simplified rating data. We want an estimate of the long-tail preference of $u_3$ and $u_6$, who have each  rated a single movie.  Relying only  on movie popularity information, we would  conclude   $u_3$ and $u_6$ have similar long-tail preference, since $m_1$ and $m_2$ are  equally popular. However, considering the long-tail preferences of other users of those movies, we may reason differently: since $u_1$ and $u_2$ have high long-tail preference, and $u_4$ and $u_5$ have low long-tail preference, $m_1$ is a more prominent long-tail item compared to $m_2$. Therefore, it is likely that $u_3$ has higher long-tail preference compared to $u_6$.considering the long-tail preferences of other users of those movies, we may reason differently.  For example, 
\label{ex:running}
\end{example}



%------------------------------

\iffalse
\begin{example}
Table~\ref{tab:example} shows rating data for a simplified system. %Note the user-item interaction matrix is sparse.
For this example, we define popular movies as those that have received  three or more ratings; $\{m_1, m_2, m_4\}$ are popular and  $\{m_3, m_5, m_6\}$ are niche movies. We observe $u_1$ and $u_3$  have rated relatively popular movies (risk-averse) while $u_2$ and $u_4$ have rated niche movies (risk-loving). 
\label{ex:running}
\end{example}

\begin{table}[htb]
\centering
\scriptsize
\begin{tabular}{ccccccc} 
\toprule
			&$m_1$ &$m_2$   &$m_3$    &$m_4$   &$m_5$ &$m_6$  \\ \hline 
$u_1 $ &5  &4  & - &-  &-  &-   \\
$u_2$  &-  &-  &-  &-  &5  &5   \\
$u_3$  &-  &4  &-  &5  &-  &-   \\
$u_4$  &-  &-  &3  &-  &-  &4   \\ 
$u_5$  &5  &-  &-  &3  &-  &-   \\ 
$u_6$  &4  &2  &-  &4  &-  &-   \\ 
\bottomrule
%$ f_i$    &3  &3  &1  &3  &1  &2  \\  \hline
\end{tabular}
\caption{User-Movie rating data} \label{tab:example}
\end{table}

It is essential to consider consumer characteristics in designing recommender systems so that they promote long-tail items to the right group of users and spread demand evenly between hit and niche items.  

\fi





%------------------------------
\iffalse
\begin{table}[htb]
\centering
\scriptsize
\begin{tabular}{ccccccc} 
\toprule
			&$m_1$ &$m_2$   &$m_3$    &$m_4$   &$m_5$ &$m_6$  \\ \hline 
$u_1 $ &\textbf{5}  & \textbf{4}  &\textcolor{gray}{ 1.2} &-  &-  &-   \\
$u_2$  &-  &-  &-  &-  & \textbf{5}  &\textbf{5}   \\
$u_3$  &-  &\textbf{4}  &-  &\textbf{5}  &-  &-   \\
$u_4$  &-  &-  &\textbf{3}  &-  &-  &\textbf{4}   \\ 
$u_5$  &\textbf{5}  &-  &-  &\textbf{3}  &-  &-   \\ 
$u_6$  &\textbf{4}  &\textbf{2}  &-  &\textbf{4}  &-  &-   \\ 
\bottomrule
%$ f_i$    &3  &3  &1  &3  &1  &2  \\  \hline
\end{tabular}
\caption{User-Movie rating data} \label{tab:example}
\end{table}
% $\mathcal{P}^1= \{ \mathcal{P}_1^1 \{i_1,i_2,i_3\}, \mathcal{P}_2^1:\{i_2,i_3,i_5\}  \}$
 %$\mathcal{P}^2= \{ \mathcal{P}_1^2: \{i_1,i_2,i_3\}, \mathcal{P}_2^2:\{i_2,i_5,i_6\}  \}$
 %$\mathcal{P}^3= \{ \mathcal{P}_1^3: \{i_7,i_8,i_9\}, \mathcal{P}_2^3:\{i_{10},i_{11},i_{12}\}  \}$
\begin{table}[htb]
\centering
\tiny
\begin{tabular}{ccc} 
\toprule
		&$u_1$&$u_2$  \\ \hline 
$\mathcal{P}^1 $ & $\{i_1,i_2,i_3\}$ & $\{i_2,i_3,i_5\} $ \\
$\mathcal{P}^2$ & $\{i_1,i_2,i_3\}$ & $\{i_2,i_5,i_6\} $ \\
$\mathcal{P}^3$ & $\{i_7,i_8,i_9\}$ & $\{i_{10},i_{11},i_{12} \}$ \\
\bottomrule
%$ f_i$    &3  &3  &1  &3  &1  &2  \\  \hline
\end{tabular}
\caption{Top-$\size$ allocations to users.} \label{tab:paretoExamples}
\end{table}
\fi


\iffalse
When considering long-tail items, it is important to consider consumers' willingness  to explore niche or unpopular items and their propensity towards similar items. In particular, they can be characterized by their  {\it risk degree\/} and {\it focusing degree\/}, respectively.  We compute these estimates  based on historical rating information. The following example further describes these notions in the context of movie rating data. 

\begin{example}  
Table~\ref{tab:example} shows rating data for a simplified system with $6$ users, $6$ movies, and $3$ genres. $m_i^{j}$ implies that movie $m_i$ belongs to genre $j$. Note the user-item interaction matrix is sparse. 
  For this setting, we define popular movies as those that have received  three or more ratings; $\{m_1, m_2, m_4\}$ are popular and  $\{m_3, m_5, m_6\}$ are niche movies. We now profile the users according to their risk and focusing degree. E.g., $u_1$ has rated relatively popular movies belonging to the same genre (risk-averse, high focusing degree); $u_2$ has rated niches movies in the same genre (risk-loving, high focusing degree); $u_3$ has rated popular movies in two different genres (risk-averse, low focusing degree), and $u_4$ has rated niches movies in two different genres (risk-loving, low focusing degree). 
\label{ex:running}
\end{example}
\begin{table}[htb]
\centering
\tiny
\begin{tabular}{ccccccc} 
\toprule
			&$m_1^{1}$ &$m_2^{1}$   &$m_3^{2}$    &$m_4^{3}$   &$m_5^{3}$ &$m_6^{3}$  \\ \hline 
$u_1 $ &5  &4  &-  &-  &-  &-   \\
$u_2$  &-  &-  &-  &-  &5  &5   \\
$u_3$  &-  &4  &-  &5  &-  &-   \\
$u_4$  &-  &-  &3  &-  &-  &4   \\ 
$u_5$  &5  &-  &-  &3  &-  &-   \\ 
$u_6$  &4  &2  &-  &4  &-  &-   \\ 
\bottomrule
%$ f_i$    &3  &3  &1  &3  &1  &2  \\  \hline
\end{tabular}
\caption{User-Movie rating data} \label{tab:example}
\end{table}
It is essential to consider these consumer characteristics in designing recommender systems so that they promote long-tail items to the right group of users and spread demand evenly between the hit and niche items.  
\fi
\iffalse
\begin{center}
\begin{figure*}[tp]
%\scalebox{0.5}{%
\resizebox{1\textwidth}{!}{%
%\small%\addtolength{\tabcolsep}{5pt}% below sums to 8
\begin{tabularx}{1.5\textwidth}{>{\hsize=2.5\hsize}X>{\hsize=2.5\hsize}X>{\hsize=0.5\hsize}X>{\hsize=0.5\hsize}X>{\hsize=0.5\hsize}X>{\hsize=0.5\hsize}X>{\hsize=0.5\hsize}X>{\hsize=0.5\hsize}X}
    \multirow{12}{*}{\includegraphics[scale=0.3]{codeForExample/popularity-movie.png}} & \multirow{12}{*}{\includegraphics[scale=0.3]{codeForExample/scatterplot.png}} & & & & & & \\
%   & &               &       &       &       &       &       \\
    & &\multicolumn{1}{l|}{}               &$m_1^{g1}$   	&$m_2^{g1}$    	&$m_3^{g2}$    &$m_4^{g2}$      &$m_5^{g3}$    \\ \cline{3-8}%\hline
    & &\multicolumn{1}{l|}{u1}          &5  &5  &-  &-   &-  \\
    & &\multicolumn{1}{l|}{u2}    		&-  &-  &4  &4  &5  \\
    & &\multicolumn{1}{l|}{u3}   			&1  &2  &1  &-  &-   \\
    & &\multicolumn{1}{l|}{u4}     		&1  &-  &-  &-  &-  \\
    & &               &       &       &       &       &       \\
    & &               &       &       &       &       &       \\
    & &               &       &       &       &       &       \\
    & &               &       &       &       &       &	\\
    \\
\end{tabularx}}
\caption{User-Movie interaction data a) Popularity-Movie histogram b)Movie genres/clusters c) User-Movie rating data} \label{fig:example}
\end{figure*}
\end{center}
\fi



%We propose a novel approach that allows us to  promote long-tail items in a targeted manner, thereby improving the novelty of top-$\size$ sets, the overall item-space coverage across recommendations, while maintaining reasonable levels of accuracy.

%Next, we integrate these learned preferences  in a generic  top-$\size$ recommendation framework to provide customized balance between accuracy and coverage.

%sequentially make recommendations, while adjusting its parameters with regard to the set of top-$\size$ recommendations made so far. However, since  sequential parameter updates  cause  scalability issues, we propose a sampling based algorithm. This variant of our framework, called {\it Ordered Sampling-based Locally Greedy (OSLG)\/},  allows us to  correct for the popularity bias in recommendations with regard to individual user long-tail preferences. 

%ICDE submission
%Our framework differs with  prior work in the following aspects:  unlike~\cite{adomavicius2011maximizing,adomavicius2012improving,zhang2013personalize,ho2014likes},  the long-tail preference personalization in our framework is learned rather than optimized using cross-validation or parameter tuning. In other words, our personalization method is independent of the underlying base  recommendation models.  Moreover, our framework is  generic. This enables us to  plug-in several base recommenders, and evaluate their  effectiveness without requiring  extensive tuning for the accuracy and coverage trade-off. 


%\vspace{-2.8pt}
\begin{itemize}

\item  We examine various measures for estimating user long-tail novelty preference in Section~\ref{sec:lt-pref} and formulate an optimization problem  to directly learn users' preferences for long-tail  items from interaction data in Section~\ref{sec:learning-lt-pref}. %In addition, we introduce several heuristics for measuring the user preference for less common items from historical rating data.% 

\item  We integrate the user preference estimates into GANC %, a generic re-ranking framework that provides customized balance between accuracy, novelty, and coverage 
(Section~\ref{sec:RiskbasedReranking}), and  introduce {\it Ordered Sampling-based Locally Greedy (OSLG)\/}, a scalable algorithm that relies  on user long-tail preferences to correct the popularity bias (Section~\ref{sec:optimizationAlgorithm}).
%We introduce OSLG, a scalable algorithm that relies  on user long-tail preferences to  maximize item space coverage \textcolor{red}{while maintaining acceptable levels of accuracy} (Section~\ref{sec:optimizationAlgorithm}).

\item   We conduct an extensive empirical study and evaluate performance from  accuracy, novelty, and coverage perspectives (Section~\ref{sec:Experiments}).  We use five  datasets with varying density and difficulty levels. %:  Netflix, MovieTweetings, and MovieLens (100K, 1M, 10M). 
  In contrast to most related work,  our evaluation considers realistic settings that include a large number of infrequent  items and users. %This enables us to study the impact of  data density on the performance trade-offs of several  state of the art top-$\size$ recommendation algorithms. %   %,  and use the all-items ranking protocol~\cite{steck2013evaluation,vargas2014improving}, where performance is measured using all items with train data. to evaluate the performance of several  state of the art top-$\size$ recommendation algorithms 
 
\item Our empirical results confirm that the performance of re-ranking models is impacted by the underlying   base recommender and the dataset density. Our generic approach enables us to easily incorporate a suitable base recommender to devise an effective solution for both dense and sparse settings. In dense settings, we use the same base recommender as existing re-ranking approaches, and we outperform them in accuracy and coverage metrics. For sparse settings, we plug-in a more suitable base recommender, and devise an effective solution that is competitive with existing top-$\size$ recommendation methods in accuracy and novelty. 

%Directly estimating the long-tail novelty preferences allows us to customize re-ranking per user, and  devise a generic framework.   
 
\end{itemize}

Section~\ref{sec:related-work} describes related work. Section~\ref{sec:conclusion} concludes.

\section{Related Work}
\label{sec:review}

% Subsequent works \cite{bloesch2018codeslam, zhi2019scenecode, sucar2020nodeslam} adopt a probabilistic formulation leading to a keyframe-based semantic mapping system that jointly estimates motion, geometry and semantics via a latent space representation\NA{latent space of what?}. 

Several RGB-D SLAM approaches \cite{newcombe2011kinectfusion, endres2012evaluation, kerl2013dense, endres20133, whelan2016elasticfusion} are able to generate accurate trajectory and a dense 3D model of the environment. However, early RGB-D SLAM techniques focus on obtaining a geometric map and overlook the semantics. 
Later, object-level SLAM approaches \cite{nicholson2018quadricslam, yang2019cubeslam} are proposed to model both geometry and semantics. Those works focus on estimating the object pose accurately, but have limited capabilities to model object shape details due to the very simple geometric shape models used, such as cuboids and quadrics.  

% commented out due to page limit 
% Subsequent attempts \cite{salas2013slam++, pillai2015monocular, slavcheva2016sdf, sunderhauf2017meaningful, nicholson2018quadricslam, xu2019mid, hu2019deep, nellithimaru2019rols, fenglocalization} model objects explicitly in RGB-D SLAM to attain both compactness and expressiveness. 
% For instance, SLAM++ \cite{salas2013slam++} enforces camera-object constraints based on prior object CAD models in the pose-graph optimization for SLAM, and the object-graph enables camera tracking via Iterative closest point (ICP). 
% An object-centric RGB-D SLAM algorithm capable of generating instance-level segmentation is proposed in \cite{sunderhauf2017meaningful}, which does not rely on a priori object models used in SLAM++. This work uses the RGB-D version of ORB-SLAM \cite{mur2017orb} for camera pose tracking and mapping geometric entities. To segment out objects from RGB-D measurements, an adjacency graph connecting nearby supervoxels are constructed, which is then partitioned into connected components to identify objects.
% Deep-SLAM++ \cite{hu2019deep} presents the first room-scale object SLAM which integrates an object shape prior into the estimation, using a Kinect V2 sensor. Instead of point clouds the paper uses scalable binary occupancy grids, but those have a severely limited resolution. 

Compared with other similar works~\cite{Mescheder_2019_CVPR, Chen_2019_CVPR} on learning implicit function for surface, DeepSDF \cite{park2019deepsdf} learns a continuous metric function of distance instead of binary classification function dividing inside or outside, which makes it suitable for gradient-based optimization in SLAM. 
Subsequent works along the direction of DeepSDF include FroDO \cite{runz2020frodo}, MOLTR \cite{li2020mo}, and DualSDF \cite{hao2020dualsdf}. 
FroDO leverages both point cloud and SDF representations, which defines sparse and dense losses to optimize the object shape. 
An extension of FroDO is MOLTR, which reconstructs an object shape by fusing multiple single-view shape codes to handle both static and dynamic objects. 
Similar to the coarse-to-fine shape estimation in FroDO and MOLTR, DualSDF uses two levels of granularity to represent 3D shapes. A shared latent space is employed to tightly couple the two levels, and a Gaussian prior is imposed on the latent space to enable sampling, interpolation, and optimization-based manipulation.  
DeepSDF and the derivatives offer models for accurate shape modeling but few of them consider object pose estimation. 

Object pose estimation is a critical step in the construction of an object level map. 
To estimate the transformation between world frame and the object frame, Scan2CAD \cite{avetisyan2019scan2cad} estimates the object pose and scale by establishing keypoint correspondences between the objects in the scene and their 3D CAD models. The keypoints are annotated for the CAD models and predicted by CNNs during inference. The Harris keypoints are detected from the 3D scan to be matched with those keypoints on the CAD models. However, both keypoint annotation and model retrieval take a long time for objects with complicated shapes, such as sofa. Later on Avetisyan \etal\cite{avetisyan2019end} dramatically increased the efficiency of the alignment process by utilizing a novel differentiable Procrustes alignment loss. Firstly, a proposed 3D CNN is used to identify objects in the 3D scan. Secondly, object bounding boxes are used to establish correspondence between scan objects and the CAD models. Lastly, alignment-informed correspondences are learnt via the differentiable Procrustes alignment loss.
Furthermore, multi-view constraints are introduced in Vid2CAD \cite{maninis2020vid2cad}. 
% Object pose estimation could also be achieved using generic geometric representations, such as cuboids in CubeSLAM \cite{yang2019cubeslam} or ellipsoids in QuadricSLAM \cite{rubino20173d, nicholson2018quadricslam}. 
% FroDO uses \cite{nicholson2018quadricslam} to initialize the object pose, but it does not take object shape prior or occlusion into account. 

In the proposed ELLIPSDF, a learnt continuous SDF is used to reconstruct the object at arbitrary resolutions, and thus our approach has a more expressive object model in comparison to \cite{hu2019deep, sucar2020nodeslam}. 
Furthermore, our model has two levels of granularity that provide a coarse object prior to optimize the object scale, which is different from FroDO or \cite{afolabi2020extending}. 
Our system is online and more efficient, and unlike prior works that focus on single object estimation, we also present a large-scale, quantitative evaluation using a public benchmark that has multiple objects. 

\section{Background}
\label{sec:background}

%\SHAN{I think the background section is too long, maybe we can remove the SO3, SE3 parts, only keep SIM3, ellipsoid, SDF}

Rigid body orientation, pose, and similarity are represented using the $\text{SO}(3)$, $\text{SE}(3)$, and $\text{SIM}(3)$ Lie groups, respectively, defined as:
%
\begin{gather}
\label{eq:LieGroups}
\scaleMathLine{\begin{aligned}
\text{SO}(3) &\triangleq \crl{ \bfR \in \bbR^{3 \times 3} \mid \bfR^\top\bfR = \bfI, \det(\bfR) = 1},\\
\text{SE}(3) &\triangleq \crl{ \begin{bmatrix} \bfR & \bft\\\mathbf{0}^\top & 1 \end{bmatrix} \in \bbR^{4 \times 4} \,\bigg\vert\, \bfR \in SO(3), \bft \in \bbR^3}, \\
\text{SIM}(3) &\triangleq \crl{ \begin{bmatrix} s\bfR & \bft\\\mathbf{0}^\top & 1 \end{bmatrix} \in \bbR^{4 \times 4} \,\bigg\vert\, \bfR \in SO(3), \bft \in \bbR^3, s \in \bbR}.
\end{aligned}}
\raisetag{10ex}
\end{gather}
%
We overload $\bfxi_\times$ to denote a mapping from a vector in $\bbR^3$ or $\bbR^6$ or $\bbR^7$ to the Lie algebra $\mathfrak{so}(3)$, $\mathfrak{se}(3)$, or $\mathfrak{sim}(3)$, associated with the Lie groups in \eqref{eq:LieGroups}, defined as:
%
\begin{gather}
\label{eq:LieAlgebras}
\begin{aligned}
\mathfrak{so}(3) &\triangleq \crl{\bfxi_\times = \begin{bmatrix}0 & -\xi_3 & \xi_2\\\xi_3 & 0 & -\xi_1\\-\xi_2 & \xi_1 & 0 \end{bmatrix} \,\bigg\vert\, \bfxi \in \bbR^3},\\
\mathfrak{se}(3) &\triangleq \crl{\bfxi_\times = \begin{bmatrix} \bftheta_\times & \bfrho \\ \mathbf{0}^\top & 0 \end{bmatrix}\,\bigg\vert\, \bfxi = \begin{bmatrix} \bfrho \\ \bftheta\end{bmatrix} \in \bbR^6},\\
\mathfrak{sim}(3) &\triangleq \crl{\bfxi_\times = \begin{bmatrix} \sigma \bfI + \bftheta_\times & \bfrho \\ \mathbf{0}^\top & 0 \end{bmatrix} \,\bigg\vert\, \bfxi = \begin{bmatrix} \bfrho \\ \bftheta \\ \sigma \end{bmatrix} \in \bbR^7}.
\end{aligned}
\raisetag{12ex}
\end{gather}
%
We define an infinitesimal change of a Lie group element $\bfT$ via a left perturbation $\exp\prl{\bfxi_\times}\bfT$, using the exponential map $\exp\prl{\bfxi_\times}$ to retract a Lie algebra element $\bfxi_\times$ to the Lie group. Please refer to \cite[Ch.7]{BarfootBook} or \cite{Gao2017SLAM} for details. 

The coarse shape of a rigid body is represented using a \emph{quadric shape} \cite[Ch.3]{MVGBook}, $\crl{ \bfx \in \bbR^3 \mid \underline{\bfx}^\top \bfQ \underline{\bfx} \leq 0}$, where $\underline{\bfx} \triangleq [\bfx^\top, 1]^\top$ denotes the homogeneous coordinates of $\bfx$ and $\bfQ \in \bbR^{4 \times 4}$ is a symmetric matrix. An axis-aligned ellipsoid centered at the origin:
%
\begin{equation}
\label{eq:ellipsoid}
\mathcal{E}_{\bfu} \triangleq \crl{\bfx \in \bbR^3 \mid \bfx^\top \bfU^{-\top}\bfU^{-1}\bfx \leq 1},
\end{equation}
%
where $\bfU \triangleq \diag(\bfu)$ and the elements of the vector $\bfu \in \bbR^3$ specify the lengths of the semi-axes of $\mathcal{E}_{\bfu}$. An ellipsoid $\mathcal{E}_{\bfu}$ is a special case of a quadric shape with $\bfQ = \diag(\bfU^{-2},-1)$. 
% Instead of as the collection of points $\bfx$ contained in it, 
A quadric shape can also be defined in dual form, as the set of planes $\underline{\boldsymbol{\pi}} = \bfQ\underline{\bfx}$ that are tangent to the shape surface at each $\bfx$. This dual quadric surface definition is $\crl{ \bfpi \in \bbR^3 \mid \underline{\bfpi}^\top \bfQ^* \underline{\bfpi} = 0}$, where $\bfQ^* = \adj(\bfQ)$ is the adjugate of $\bfQ$.
%\footnote{If $\bfQ$ is invertible, $\bfQ^* = \adj(\bfQ) \triangleq \det(\bfQ)\bfQ^{-1}$ can be simplified to $\bfQ^* = \bfQ^{-1} = \diag(\bfU^2, -1)$ due to the scale-invariance of the dual quadric definition.}.
A dual quadric defined by $\bfQ^*$ can be scaled, rotated, or translated by a similarity transform $\bfT \in \text{SIM}(3)$ as $\bfT \bfQ^* \bfT^\top$. Similarity, a dual quadric can be projected to a lower-dimensional space by a projection matrix $\bfP = \begin{bmatrix} \bfI & \mathbf{0} \end{bmatrix}$ as $\bfP \bfQ^* \bfP^\top$.

The fine shape of a rigid body is represented as $\crl{\bfx \in \bbR^3 \mid f(\bfx) \leq 0}$ using the \emph{signed distance field} of a set $\calS \subset \bbR^3$:
%
\begin{equation}
f(\bfx) = \begin{cases}
  -d(\bfx,\partial\calS), & \bfx \in \calS,\\
  \phantom{-}d(\bfx,\partial\calS), & \bfx \notin \calS,
\end{cases}
\end{equation}
%
where $d(\bfx,\partial\calS)$ denotes the Euclidean distance from a point $\bfx \in \bbR^3$ to the boundary $\partial\calS$ of $\calS$.























%% \subsection{SE3}

%This section introduces the necessary mathmatical background. 
%The transformation in $\text{SE}(3)$ can be expressed as:
%\begin{equation}
%\bfT \triangleq \begin{bmatrix} \bfR & \bft\\\mathbf{0}^\top & 1 \end{bmatrix} \in \text{SE}(3)
%\end{equation}
%We overload $\bftheta_\times$ to denote the mapping from an axis-angle vector $\bftheta \in \mathbb{R}^3$ to a $3 \times 3$ skew-symmetric matrix $\bftheta_\times \in \mathfrak{so}(3)$ and the mapping from a position-rotation vector $\bfxi \in \mathbb{R}^6$ to a $4 \times 4$ twist matrix $\bfxi_\times \in \mathfrak{se}(3)$. We define an infinitesimal change of pose $\bfT \in SE(3)$ using a left perturbation $\exp\prl{\bfxi_\times}\bfT \in \text{SE}(3)$ (see~\cite[Ch.7]{BarfootBook}).

%% \subsection{SIM(3)}

%We use the space $\text{SIM}(3)$ of similarity transformations to capture scale $s$ in addition to pose: 
%\begin{equation}
%\bfT \triangleq \begin{bmatrix} s\bfR & \bft\\\mathbf{0}^\top & 1 \end{bmatrix} \in \text{SIM}(3).
%\end{equation}
%We also use $\bfxi$ to denote the corresponding Lie algebra $\mathfrak{sim}(3)$, as in \cite{Gao2017SLAM}:
%\begin{equation}
%\label{eq:sim3_to_SIM3}
%\scaleMathLine[0.91]{
%\begin{aligned}
%\mathfrak{sim}(3) \triangleq
%\left\{\bfxi_\times=
%\left[\begin{array}{cc}
%{\sigma \bfI+\bfphi_\times} & {\bfrho} \\
%\mathbf{0}^\top & {0}
%\end{array}\right] \in \mathbb{R}^{4 \times 4} \;\bigg\vert\; \bfxi = \left[\begin{array}{c}
%{\bfrho} \\
%{\bfphi} \\
%{\sigma}
%\end{array}\right] \in \mathbb{R}^{7}\right\}
%\end{aligned}
%}
%\end{equation}
%We define the operator:
%%\NA{what is the operator $\underline{\bfx}^\circledcirc$?}
%%\SHAN{I don't think we will use $\underline{\bfx}^\circledcirc$ anywhere}
%\begin{equation}
%\underline{\bfx}^\odot \triangleq \begin{bmatrix} \bfI_3 & -\bfx_\times & \bfx\\ \mathbf{0}^\top & \mathbf{0}^\top & 0 \end{bmatrix} \in \mathbb{R}^{4 \times 7}
%\end{equation}




%\begin{definition*}
%The \textit{signed distance field} of a set $\calS \subset \mathbb{R}^3$ is
%\begin{equation}
%f(\bfx) = \begin{cases}
%  -d(\bfx,\partial\calS) & \bfx \in \calS\\
%  d(\bfx,\partial\calS) & \bfx \notin \calS
%\end{cases}
%\end{equation}
%where $d(\bfx,\partial\calS)$ is the distance from a point $\bfx \in \mathbb{R}^3$ to the set boundary $\partial\calS$, and we use $d$ as a shorthand notation to denote this distance.
%\end{definition*}


%\begin{definition*}
%\textit{Huber error function} \cite{Huber1964Robust} with parameter $\delta > 0$ is:
%\begin{equation}
%\rho(r) \triangleq 
%\begin{cases}
%\frac{1}{2}r^2 & |r|\leq \delta,\\
%\delta(|r|-\frac{1}{2}\delta) & \text{else}.
%\end{cases}
%\end{equation}
%\end{definition*}
%whose gradient can be computed as 
%\[
%  \frac{\partial \rho(r)}{\partial r}
%  =\left\{\begin{array}{ll}
%    r & |r| \leq \delta \\
%    \text{sign}(r)\delta  & \text{ else}. 
%    \end{array}\right.
%\] 

%An axis-aligned ellipsoid centered at the origin can be described as:
%\begin{equation}
%\mathcal{E}_{\bfu} \triangleq \crl{\bfx \mid \bfx^\top \bfU^{-\top}\bfU^{-1}\bfx \leq 1},
%\end{equation}
%where $\bfU \triangleq \diag(\bfu)$ and the elements of the vector $\bfu = [a,b,c]^{\top}$ are the lengths of the semi-axes of $\mathcal{E}_{\bfu}$. In homogeneous coordinates, $\mathcal{E}_{\bfu}$ can be represented as a quadric surface~\cite[Ch.3]{MVGBook}, $\crl{ \bfx \mid \underline{\bfx}^\top \bfQ_{\bfu} \underline{\bfx} \leq 0}$, where $\bfQ_{\bfu} = \mathbf{diag}(\bfU^{-2},-1)$. This describes the ellipsoid as a collection of points lying on its surface. 

%Alternatively, a quadric can be defined by the set of planes $\underline{\boldsymbol{\pi}} = \bfQ_{\bfu}\underline{\bfx}$ tangent to its surface at $\underline{\bfx}$. This dual quadric surface is defined as $\crl{ \bfpi \mid \underline{\bfpi}^\top \bfQ_{\bfu}^* \underline{\bfpi} = 0}$, where $\bfQ_{\bfu}^* = \mathbf{adj}(\bfQ_{\bfu})$~\footnote{If $\bfQ_{\bfu}$ is invertible, $\bfQ_{\bfu}^* = \mathbf{adj}(\bfQ_{\bfu}) = \det(\bfQ_{\bfu})\bfQ_{\bfu}^{-1}$ can be simplified to $\bfQ_{\bfu}^* = \bfQ_{\bfu}^{-1} = \diag(\bfU^2, -1)$ due to the scale-invariance of the dual quadric definition.}. A dual quadric defined by $\bfQ_{\bfu}^*$ can be transformed by $\bfT \in SE(3)$ to another reference frame as $\bfQ^* = \bfT \bfQ_{\bfu}^* \bfT^\top$, which can be projected to a lower-dimensional space by $\bfP \in \mathbb{R}^{3\times 4}$ as $\bfP \bfQ^* \bfP^\top$. 
%$\bfQ^*$ can be parameterized as:
%\begin{equation}
%\label{eq:ellipsoid}
%\begin{aligned}
%\bfQ^*
%&=
%\mathbf{T} \bfQ_{\bfu}^* \mathbf{T}^{\top}=
%\left[\begin{array}{cc}
%\mathbf{R} & \bft \\
%\mathbf{0}^{\top} & 1
%\end{array}\right]\left[\begin{array}{cc}
%\mathbf{U} \mathbf{U}^{\top} & \mathbf{0} \\
%\mathbf{0} & -1
%\end{array}\right]\left[\begin{array}{ll}
%\mathbf{R}^{\top} & \mathbf{0} \\
%\bft^{\top} & 1
%\end{array}\right] \\ 
%&=
%\begin{pmatrix} 
%\mathbf{R} \mathbf{U} \mathbf{U}^\top \mathbf{R}^\top -  \bft \bft^\top & - \bft \\ -\bft^\top & -1
%\end{pmatrix}
%\end{aligned}
%\end{equation}




\section{Problem Formulation}
\label{sec:problem}


Consider an RGB-D camera whose optical frame has pose $\bfC_k \in \text{SE}(3)$ with respect to the global frame at discrete time steps $k = 1,\ldots,K$. Assume that the camera is calibrated and its pose trajectory $\crl{\bfC_k}_k$ is known, e.g., from a SLAM or SfM algorithm. At time $k$, the camera provides an RGB image $I_k : \Omega^2 \mapsto \mathbb{R}_{\geq 0}^3$ and a depth image $D_k : \Omega^2 \mapsto \mathbb{R}_{\geq 0}$ such that $I_k(\bfp)$ and $D_k(\bfp)$ are the color and depth of a pixel $\bfp \in \Omega^2$ in normalized pixel coordinates. The camera moves in an unknown environment that contains $N$ objects $\calO \triangleq \crl{\bfo_n}_{n=1}^N$. Each object $\bfo_n = (\bfc_n,\bfi_n)$ is an instance $\bfi_n$ of class $\bfc_n$, defined below.


\begin{definition*}
An \emph{object class} is a tuple $\bfc \triangleq \prl{\nu, \bfz, f_{\bftheta}, g_{\bfphi}}$, where $\nu \in \mathbb{N}$ is the class identity, e.g., chair, table, sofa, and $\bfz \in \mathbb{R}^d$ is a latent code vector, encoding the average class shape. The class shape is represented in a canonical coordinate frame at two levels of granularity: coarse and fine. The coarse shape is specified by an ellipsoid $\calE_\bfu$ in \eqref{eq:ellipsoid} with semi-axis lengths $\bfu = g_{\bfphi}(\bfz)$ decoded from the latent code $\bfz$ via a function $g_{\bfphi} : \bbR^d \mapsto \bbR^3$ with parameters $\bfphi$. The fine shape is specified by the signed distance $f_{\bftheta}(\bfx,\bfz)$ from any $\bfx \in \bbR^3$ to the average shape surface, decoded from the latent code $\bfz$ via a function $f_{\bftheta} : \bbR^3 \times \bbR^d \mapsto \bbR$ with parameters $\bftheta$.
\end{definition*}

\begin{definition*}
An \emph{object instance} of class $\bfc$ is a tuple $\bfi \triangleq \prl{\bfT, \delta\bfz}$, where $\bfT \in \text{SIM}(3)$ specifies the transformation from the global frame to the object instance frame, and $\delta\bfz \in \bbR^d$ is a deformation of the latent code $\bfz$, specifying the average shape of class $\bfc$.
\end{definition*}


\begin{figure*}[t] 
  \centering
  \includegraphics[width=\linewidth]{framework_new.jpg}
  \caption{ELLIPSDF Overview: A point cloud and initial pose (\textit{green}) are obtained from RGB-D detections of a chair instance from known camera poses (\textit{blue}). A bi-level category shape description, consisting of a latent shape code, a coarse shape decoder, and a fine shape decoder (\textit{orange}), is trained offline using a dataset of mesh models. Given the observed point cloud, the pose and shape deformation of the newly seen instance are optimized jointly online, achieving shape reconstruction in the global frame (\textit{red}).}
%   allowing shape reconstruction in the global frame (\textit{red}).}
  \label{fig:framework}
  %The point cloud of the object and the initial object pose (in the green rectange) are obtained from the RGB-D detections, including color image, depth image, instance segmentation, and fitted ellipse, and camera poses (in the blue rectangles). Then leveraging the observed object point cloud and the initial pose, \text{SIM}(3) object pose and the latent code of a two-level object model (in the orange rectangle) trained from a dataset (in the blue rectangle) are jointly optimized. The object in the world frame (in the red rectangle) could be reconstructed from the optimized object pose and latent code.}
\end{figure*}


We assume that object detection (e.g., \cite{Cai2019Cascade}) and tracking (e.g., \cite{bewley2016simple}) algorithms are available to provide the class $\bfc_n$ and pixel-wise segmentation $\Omega_{n,k}^2 \subseteq \Omega^2$ of any object $n$ observed by the camera at time $k$. %The segmentation $\Omega_{n,k}^2\triangleq \crl{ \bfp \in \Omega^2 \mid  \Delta_k(\bfp) = n}$ of object $n$ at time $k$ is obtained from labeling $\Delta_k : \Omega^2 \rightarrow \{0,\ldots,N\}$ of each pixel $\bfp \in \Omega^2$ with an object id $\Delta_k(\bfp) \in \crl{1,\ldots,N}$ or the scene background $\Delta_k(\bfp) = 0$.
Our goal is to estimate the transformation and shape $\bfi_n := (\bfT_n, \delta \bfz_n)$ of each observed object $n$. We consider object instances independently and drop the subscript $n$ when it is clear from the context. 
%\SHAN{n is still used below, do we need to say the sentence below?}
%in the reminder for clarity.

%  and the camera pose $\bfC_k$

Given an object with multi-view segmentation $\Omega_{k}^2$, we use the depth $D_k(\bfp)$ of each pixel $\bfp \in \Omega_{k}^2$ to obtain a set of points $\calX_k(\bfp)$ along the ray starting from the camera optical center and passing through $\bfp$. The sets $\calX_k(\bfp)$ is used to optimize the pose and shape of the object instance. For each ray, we choose three points, one lying on the observed surface, one a small distance $\epsilon>0$ in front of the surface, and one a small distance $\epsilon$ behind. Given $d \in \{0,\pm \epsilon\}$, we obtain points $\bfy \in \bbR^3$ in the optical frame corresponding to the pixels $\bfp \in \Omega_{k}^2$:
%
\begin{equation*}
\scaleMathLine{\calY_k(\bfp) \triangleq \crl{(\bfy, d) \,\bigg\vert\, \bfy = \prl{D_k(\bfp) + \frac{d}{\|\underline{\bfp}\|}}\underline{\bfp}, \; d \in \{0,\pm \epsilon\}},}
\end{equation*}
%
and project them to the global frame using the known camera pose $\bfC_k$:
%
\begin{equation}\label{eq:distance_measurements}
\calX_k(\bfp) \triangleq \crl{(\bfx, d) \,\bigg\vert\, \bfx = \bfP \bfC_k \underline{\bfy}, \; (\bfy,d) \in \calY_k(\bfp)}.
\end{equation}
%
%At training time, distance measurements like in \eqref{eq:distance_measurements} are obtained from several object instances of the same class and are used to optimize the shape model parameters $\bfz$, $\bftheta$, $\bfphi$, as described in Sec.~\ref{sec:train_code}. 

We define an error function $e_{\bfphi}$ to measure the discrepancy between a distance-labelled point $(\bfx,d) \in \calX_{k}(\bfp)$ observed close to the instance surface and the coarse shape $\calE_{\bfu}$ provided by $\bfu = g_{\bfphi}(\bfz)$. Another error function $e_{\bftheta}$ is used for the difference between $(\bfx,d)$ and the SDF value $f_{\bftheta}(\bfx, \bfz)$ predicted by the fine shape model. The overall error function is defined as: 
\begin{align}
\label{eq:cost_function}
&e(\bfT,\delta \bfz, \bftheta, \bfphi ; \crl{\calX_k(\bfp)}) \triangleq \alpha e_r(\delta \bfz) \\
&+ \sum_{k=1}^K 
      \sum_{\bfp \in \Omega_{k}^2}
      \sum_{(\bfx,d) \in \calX_k(\bfp)} \!\!\!\beta e_{\bftheta}(\bfx,d,\bfT,\delta \bfz) + \gamma  e_{\bfphi}(\bfx,d,\bfT,\delta \bfz)\notag,
\end{align}
where $e_r(\delta \bfz)$ is a shape deformation regularization term. The coarse-shape error, $e_{\bftheta}$, fine-shape error, $e_{\bfphi}$, and the regularization, $e_r$ are defined precisely in Sec. \ref{sec:train_code}.


We distinguish between a training phase, where we optimize the parameters $\bfz$, $\bftheta$, $\bfphi$ of an object class using offline data from instances with known mesh shapes, and a testing phase, where we optimize the pose $\bfT$ and shape deformation $\delta \bfz$ of a previously unseen instance from the same category using online distance data from an RGB-D camera.

In training, we generate points $\crl{\calX_{n,k}(\bfp)}$ close to the surface of each available mesh model $n$ in a canonical coordinate frame (with identity pose $\bfI_4$) and optimize the class shape parameters via:
%
\begin{equation}
\min_{\crl{\delta \bfz_n}, \bftheta, \bfphi} \sum_n e(\bfI_4,\delta \bfz_n, \bftheta, \bfphi ; \crl{\calX_{n,k}(\bfp)}).
\end{equation}

%During the training phase, we generate distance measurements sampled from the mesh model of an object with a canonical pose $\bfI_4$ and learn the latent code $\bfz$ as well as the decoder parameters $\bfphi$, $\bftheta$ by minimizing the cost function: 

In testing, we receive points $\crl{\calX_{k}(\bfp)}$ in the global frame, generated by the RGB-D camera from the surface of a previously unseen instance. Assuming known object class, we fix the trained shape parameters $\bfz^*$, $\bftheta^*$, $\bfphi^*$ and optimize the unknown instance transform $\bfT \in \text{SIM}(3)$ and shape deformation $\delta \bfz \in \bbR^d$:
%
\begin{equation} \label{eq:test_optimization}
\min_{\bfT, \delta \bfz} \; e(\bfT,\delta \bfz, \bftheta^*, \bfphi^* ; \crl{\calX_k(\bfp)}).
\end{equation}

























\section{Object Pose and Shape Optimization}
\label{sec:reconstruction}

This section develops ELLIPSDF, an autodecoder model for bi-level object shape representation. Sec.~\ref{sec:train_code} presents the model and defines the error functions for its parameter optimization. Sec.~\ref{sec:shape_pose_inference} describes how a trained ELLIPSDF model is used at test time for multi-view joint optimization of object pose and shape. An overview is shown in Fig.~\ref{fig:framework}.



%ELLIPSDF includes learning shape representation as well as joint pose and shape optimization, as shown in Fig.~\ref{fig:framework}. In Sec.~\ref{sec:train_code}, we introduce the two-level object model (Fig.~\ref{fig:two_level_model}). In Sec.~\ref{sec:shape_pose_inference}, we introduce our pose initialization with ellipsoids (Sec.~\ref{sec:pose_init}). Then, the object model is utilized for joint optimization with gradient descent (Sec.~\ref{sec:pose_shape_opt}). 
% uses neural networks for shape representation

% This section provides the details about how we train the ELLIPSDF two-level object model, and then covers the inference framework for multi-view object shape and pose initialization as well as optimization. Fig. \ref{fig:framework} demonstrates the framework of our approach.

%\subsection{Learning a Shared Shape Latent Space}

\subsection{Training an ELLIPSDF Model}
\label{sec:train_code}

%\subsubsection{Bi-level Shape Representation}
%\label{sec:model}

{\vspace{1ex}\bf \noindent Bi-level Shape Representation: }%
The ELLIPSDF shape model consists of two autodecoders $g_{\bfphi}(\bfz)$ and $f_{\bftheta}(\bfx,\bfz)$, using a shared latent code $\bfz \in \bbR^d$. The first autodecoder provides a \emph{coarse} shape representation with parameters $\bfphi$, as an axis-aligned ellipsoid $\calE_{\bfu}$ in a canonical coordinate frame with semi-axis lengths $\bfu = g_{\bfphi}(\bfz)$. The second autoencoder provides a \emph{fine} shape representation with parameters $\bftheta$, as an implicit SDF surface $\crl{\bfx \in \bbR^3 \mid f_{\bftheta}(\bfx,\bfz) \leq 0}$ in the same canonical coordinate frame. We implement $g_{\bfphi}(\bfz)$ and $f_{\bftheta}(\bfx,\bfz)$ as $8$-layer perceptrons with one cross-connection, as described in Sec.~D in the supplementary material of DualSDF \cite{hao2020dualsdf}. The reparametrization trick \cite{kingma2013auto} is used to maintain a Gaussian distribution $\bfz = \bfmu + \diag(\bfsigma) \bfepsilon$ over the latent code with $\bfepsilon \sim \mathcal{N}(\bf{0},\bf{I})$. Thus, at training time, the ELLIPSDF model parameters are the mean $\bfmu \in \bbR^d$ and standard deviation $\bfsigma \in \bbR^d$ of the latent shape code and the coarse and fine shape autodecoder parameters $\bfphi$ and $\bftheta$. The model is visualized in Fig.~\ref{fig:two_level_model}. 

% \TODO{The input for $g_{\bfphi}(\bfz)$ is the latent code $\bfz$. The input for $f_{\bftheta}(\bfx,\bfz)$ is the concatenation of one 3D vector $\bfx$ and the latent code $\bfz$}\NA{Perhaps we can comment this out. Seems pretty obvious.}.

%by sampling $\bfepsilon \sim \mathcal{N}(\bf{0},\bf{I})$, and setting $\bfz = \bfmu + \bfsigma \odot \bfepsilon$ in order to optimize the parameters $\bfmu$ and $\bfsigma$ via gradient descent.

%Our object model employs a two-level representation for compact object shape modeling using a shared latent code. The fine level provides shape details where as the coarse level restrains the shape scale and pose. Both levels are linked to the shared latent code such that the shape and pose can be optimized jointly from the multiview observations. A diagram visualising the framework is Fig. \ref{fig:two_level_model}.



%\subsubsection{Error Functions}
%\label{sec:errors}

{\vspace{1ex}\bf \noindent Error Functions: }%
We introduce error terms that play a key role for optimizing the category-level latent code $\bfz$ and decoder parameters $\bftheta$, $\bfphi$, during training time, as well as the transformation $\bfT$ from the global frame to the canonical object frame and the latent code deformation $\delta\bfz$ of a particular instance during test time. The training data for an ELLIPSDF model consists of distance-labeled point clouds $\calX_{n,k}(\bfp)$ associated with instances $n$ from the same class, as introduced in Sec.~\ref{sec:problem}. A different latent code $\bfz_n$ is optimized for each instance $n$, while the decoder parameters $\bftheta$ and $\bfphi$ are common for all instances of the same class. 

% The input for the error functions are generated from a training set that consists of distance measurements $\calX_{n,k}(\bfp)$ associated with object instances $n$ from the same object class, as introduced in Sec.~\ref{sec:problem}, and described in details in Sec.~\ref{sec:training_details}. 

% The distance measurements $\calX_{n,k}(\bfp)$ of instance $n$ at time $k$ are obtained from the pixel-wise instance segmentation $\bfp \in \Omega_{n,k}^2$ observed with known camera pose $\bfC_k \in \text{SE}(3)$. 

% We define an error function $e^g_{\bfphi}$ to measure the discrepancy between a distance-labelled point $(\bfx,d) \in \calX_{n,k}(\bfp)$ observed close to the instance surface and the coarse shape $\calE_{\bfu}$ provided by $\hat{\bfu} = e^g_{\bfphi}(\bfz)$.
% Another error function $e^f_{\bftheta}$ is used for the difference between a distance-labelled point $(\bfx,d) \in \calX_{n,k}(\bfp)$ observed close to the surface and the SDF value provided by $\hat{d} = f_{\bftheta}(\bfx, \bfz)$.
% The overall cost function is thus $e(\bfx,\bfphi,\bftheta,\bfT,\delta \bfz) = \beta e^g_{\bfphi}(\bfx, \bfT, \bfz + \delta \bfz) + \gamma e^f_{\bftheta}(\bfx, \bfT, \bfz + \delta \bfz)$, where $\beta, \gamma \geq 0$ are the weights. 

% of the object with respect to point $\bfx$, whose signed distance to the object surface is $d$

The fine-level shape error function $e_{\bftheta}(\bfx,d,\bfT,\delta\bfz)$ of a point $\bfx$ in global coordinates  with signed distance label $d$ is defined as:
%
\begin{equation}
  \label{eq:e_k}
  e_{\bftheta}(\bfx,d,\bfT,\delta\bfz) \triangleq \rho(s f_{\bftheta}(\bfP\bfT \underline{\bfx}; \bfz+\delta\bfz) - d).
\end{equation}
%
In the definition above, the point $\bfx$ is first transformed to the object coordinate frame via $\bfP\bfT \underline{\bfx}$ and the fine-shape model $f_{\bftheta}$ is queried with the instance shape code $\bfz + \delta\bfz$ to predict the SDF to the object surface. Since SDF values vary proportionally with scaling \cite{afolabi2020extending}, the returned value is scaled back by $s$ before measuring its discrepancy with the label $d$. Instead of measuring the difference between $s f_{\bftheta}$ and $d$ in absolute value, we employ a Huber term \cite{Huber1964Robust} to make the error function robust against outliers:
%
\begin{equation}
\label{eq:huber_loss}
\rho(r) \triangleq 
\begin{cases}
\frac{1}{2}r^2 & |r|\leq \delta,\\
\delta(|r|-\frac{1}{2}\delta) & \text{else}.
\end{cases}
\end{equation}
%
Note that the error $e_{\bftheta}$ relates both the object pose and shape to the SDF residual, which is unique to our formulation and enables their joint optimization.

The coarse-level shape error function $e_{\bfphi}(\bfx,d,\bfT,\delta\bfz)$ is defined similarly, using a signed distance function for the coarse shape. Since the coarse shape decoder, $\bfu = g_{\bfphi}(\bfz)$, provides an explicit ellipsoid description, we first need a conversion to SDF before we can define the error term. An approximation of the SDF of an ellipsoid $\calE_{\bfu}$ with semi-axis lengths $\bfu$ can be obtained as:
%
\begin{equation}
  \label{eq:ellpsoid_sdf}
  h\left(\bfx, \bfu\right)
  =
  \frac{\left\|\bfU^{-1}\bfx\right\|_{2}\left(\left\|\bfU^{-1}\bfx\right\|_{2}-1\right)}
  {\left\|\bf{U}^{-2}\bfx\right\|_{2}}.
\end{equation}
%
Then, the coarse-level shape error of a point $\bfx$ in global coordinates  with signed distance label $d$ is defined as:
%
\begin{equation}
  \label{eq:e_g}
  e_{\bfphi}(\bfx,d,\bfT,\delta\bfz) \triangleq \rho(s h(\bfP\bfT \underline{\bfx}, g_{\bfphi}(\bfz+\delta\bfz)) - d).
\end{equation} 
%
%where $\bfP = [\bfI\;\mathbf{0}] \in \mathbb{R}^{3 \times 4}$ is a projection matrix, $\bfT \in \text{SIM}(3)$ is the transformation from world frame to object frame, and $s$ is the scale. 
%Furthermore $e_{\bfphi}$ is a novel error function that has not been used to jointly optimize object pose and shape before. 

During training, the object transformation is fixed to be the canonical coordinate frame $\bfT = \bfI_4$ because the training point-cloud data is collected directly in the object frame. The regularization term $e_r(\delta\bfz)$ in \eqref{eq:cost_function} is defined as the KL divergence between the distribution of $\delta\bfz$ and a standard normal distribution \cite{hao2020dualsdf}.



%Note that during training, $\bfT = \bfI_4$ since each object is already in its canonical frame and does not require any transformation to the world frame.  


%During training, we also add the regularization term $e_r(\delta\bfz)$ as the KL divergence between the distribution of $\delta\bfz$ and a standard normal distribution. 



%Note that SDF is invariant to rotation, translation but vary proportionally with scaling \cite{afolabi2020extending}. 

%The error function $e_{\bftheta}$ relates both the object pose and shape to the SDF residual, which enables the joint optimization of both pose and shape via SDF residuals, and is different from the error functions used in DeepSDF~\cite{park2019deepsdf} and its derivatives that only depend on the shape. 

% Before introducing the coarse level error function $e_{\bfphi}$, we need to define the function $h(\cdot, \cdot)$ that computes an inexact SDF value for a point $\bfx$ with respect to an ellipsoid with shape $\bfu$ centered at origin:
% \begin{equation}
%   \label{eq:ellpsoid_sdf}
%   h\left(\bfx, \bfu\right)
%   =
%   \frac{\left\|\bfU^{-1}\bfx\right\|_{2}\left(\left\|\bfU^{-1}\bfx\right\|_{2}-1\right)}
%   {\left\|\bf{U}^{-2}\bfx\right\|_{2}}.
% \end{equation}
% Then we present the term $e_{\bfphi}(\bfx,d,\bfT,\delta\bfz)$ to depict the SDF for the coarse level shape, and is defined as 
% \begin{equation}
%   \label{eq:e_g}
%   e_{\bfphi}((\bfx,d,\bfT,\delta\bfz) \triangleq \rho(s h(\bfP\bfT \underline{\bfx}, g_{\bfphi}(\bfz+\delta\bfz)) - d),
% \end{equation} 
% where $\bfP = [\bfI\;\mathbf{0}] \in \mathbb{R}^{3 \times 4}$ is a projection matrix, $\bfT \in \text{SIM}(3)$ is the transformation from world frame to object frame, and $s$ is the scale. Note that during training, $\bfT = \bfI_4$ since each object is already in its canonical frame and does not require any transformation to the world frame.  
% Furthermore $e_{\bfphi}$ is a novel error function that has not been used to jointly optimize object pose and shape before. 

%Lastly, we employ the Huber loss \eqref{eq:huber_loss} to make the error functions robust against outliers
%\begin{equation}
%\label{eq:huber_loss}
%\rho(r) \triangleq 
%\begin{cases}
%\frac{1}{2}r^2 & |r|\leq \delta,\\
%\delta(|r|-\frac{1}{2}\delta) & \text{else}.
%\end{cases}
%\end{equation}
%During training, we also add the regularization term $e_r(\delta\bfz)$ as the KL divergence between the distribution of $\delta\bfz$ and a standard normal distribution. 
% such that the $\delta\bfz$ is generated from a compact distribution.






\begin{figure}[t]
    \centering
    \includegraphics[width=\linewidth]{dualSDF_ellipsoid.png}
    \caption{Overview of our ELLIPSDF bi-level object shape model. A latent shape code, $\bfz$, with distribution $\calN(\bfmu,\diag(\bfsigma)^2)$ is shared by a coarse shape decoder $g_\phi$, providing an ellipsoid shape description, and a fine shape decoder $f_\theta$, providing an SDF shape description. During training, the decoder parameters $\phi$ and $\theta$ are optimized by minimizing the errors between the SDF values of the training points $\bfx$, obtained close to the object surface, and the coarse and fine shape models.}
    %decoded by $g_\phi$ for the ellipsoid shape $\bfu$ and $f_\theta$ for the object SDF. The network parameters $\phi$, $\theta$ are optimized by minimizing the loss between the predicted and ground truth SDF for each sampled point $\bfx$.}
    \label{fig:two_level_model}
\end{figure}



% The latent code can be learnt by minimizing the Huber loss defined in \eqref{eq:huber_loss} between the predicted and the real SDF in \eqref{eq:cost_function}. 

% \begin{definition}
%   \textit{Huber error function} \cite{Huber1964Robust} with parameter $\delta > 0$ is:
%   \begin{equation}
%   \label{eq:huber_loss}
%   \rho(r) \triangleq 
%   \begin{cases}
%   \frac{1}{2}r^2 & |r|\leq \delta,\\
%   \delta(|r|-\frac{1}{2}\delta) & \text{else}.
%   \end{cases}
%   \end{equation}
% \end{definition}
% whose gradient can be computed as 
% \[
%   \frac{\partial \rho(r)}{\partial r}
%   =\left\{\begin{array}{ll}
%     r & |r| \leq \delta \\
%     \text{sign}(r)\delta  & \text{ else}. 
%     \end{array}\right.
% \] 







% \subsubsection{Training Process}
% \label{sec:training}

% % The shared latent space is represented by a latent code vector $\mathbf{z} \in \mathbb{R}^{d}$, which can be decoded by two generative neural networks $\bfu = g_{\bfphi}\left(\mathbf{z}\right)$, and $f_{\bftheta}\left(\bfx; \mathbf{z}\right) \approx f(\bfx), \forall x \in \calS$, where $\bfu$ represents an ellipsoid shape and $f_{\bftheta}$ is an approximator of a given SDF $f(\bfx)$\NA{some of this stuff is repeated with the problem statement section.}.
% % The ellipsoid constrains the scale of the SDF model, whereas the SDF model provides details for the object shape. 
% % A diagram visualising the framework is Fig. \ref{fig:two_level_model}. 

% This section describes how to train the two-level shape code. 
% To prepare the training data, we sample pairs of 3D points $\bfX = \{\bfx_t\}_t$ and the corresponding groundtruth SDF values $\{ d_t \}_t$ from the object meshes in a database such as ShapeNet \cite{chang2015shapenet}. 


% The training loss $L_t(\delta\bfz)$ is formulated as\NA{Define this term only once in the problem statement section and reuse it here} 
% \begin{equation}
%   \label{eq:train_loss}
%   \begin{aligned}
%     L_t(\delta\bfz) = 
%     \alpha e_r(\delta\bfz) & + 
%     \beta \sum_{\bfx_t, d_t} 
%     e^g_{k}(\bfx_t,d_t,\bfI_4,\delta\bfz) 
%     \\ 
%     &+  \gamma \sum_{\bfx_t, d_t} e_{\bftheta}(\bfx_t,d_t,\bfI_4,\delta\bfz)
%   \end{aligned}
% \end{equation}
% in which $\bfT = \bfI_4$ since during training the canonical coordinate frame of the object class is known, and thus the training focuses on the coarse and fine shape decoders. Here we define $e_r(\delta\bfz)$ as the KL divergence between the distribution of $\delta\bfz$ and a standard normal distribution.





\subsection{Joint Pose and Shape Optimization with an ELLIPSDF Model}
\label{sec:shape_pose_inference}

% (Sec.~\ref{sec:pose_init}) 
% (Sec.~\ref{sec:pose_shape_opt}) 
This section describes how a trained ELLIPSDF model is used to initialize and optimize the pose and shape of a new object instance at test time.


%\subsubsection{Initialization}
%\label{sec:pose_init}

{\vspace{1ex}\bf \noindent Initialization: }%
We follow \cite{crocco2016structure, rubino20173d, gay2018visual} to initialize the $\text{SIM}(3)$ scale and pose of an observed object, relying on its coarse ellipsoid shape representation. We fit ellipses to the pixel-wise segmentation $\Omega_k^2$ of an object at each time $k$:
%
\begin{equation} \label{eq:ellipse_fit}
    \crl{ \bfq \in \Omega^2 \mid (\bfq - \bfc_k)^\top \bfE_k^{-1} (\bfq - \bfc_k) \leq 1},
\end{equation}
%
where the center and symmetric matrix are obtained as $\bfc_k = \frac{1}{|\Omega_k^2|}\sum_{\bfp \in \Omega_k^2} \bfp$ and $\bfE_k = \frac{2}{|\Omega_k^2|}\sum_{\bfp \in \Omega_k^2} (\bfp-\bfc_k)(\bfp-\bfc_k)^\top$. The axes lengths are the eigenvalues $\lambda_{0}$, $\lambda_{1}$ of $\bfE_k$. The 2D quadric surface corresponding to the ellipse in \eqref{eq:ellipse_fit} and its dual are defined by the matrix $\bfH_k$ and its inverse $\bfH_k^*$:
%
\begin{equation*}
    \scaleMathLine{\bfH_k = \begin{bmatrix} \bfE_k^{-1} & -\bfE_k^{-1}\bfc_k \\ -\bfc_k^\top \bfE_k^{-1} & \bfc_k^\top \bfE_k^{-1} \bfc_k - 1\end{bmatrix}, \;\; \bfH_k^* = \begin{bmatrix}\bfE_k - \bfc_k\bfc_k^\top & -\bfc_k \\ -\bfc_k^\top & -1\end{bmatrix}.}
\end{equation*}




% The dual form of the ellipse is 
% $
% \bfH_k = 
% \left(\begin{array}{cc}
% \bfE_k & \bf{0} \\
% \bf{0}^\top & -1  
% \end{array}\right) 
% \in \mathbb{R}^{3\times3}
% $.

%Given a set of pixels with image coordinates $\left\{\left(u_{p}, v_{p}\right) \in \Omega_k^2 \right\}$, its center of mass is $\bfc_k = \frac{1}{|\Omega_k^2|}\sum_{\bfp \in \Omega_k^2} \bfp$. We could fit an ellipse represented by $\bfE_k = \frac{2}{|\Omega_k^2|}\sum_{\bfp \in \Omega_k^2} (\bfp-\bfc_k)(\bfp-\bfc_k)^\top$. 
% \[
%   \scaleMathLine[0.9]{
%     \left\{(u, v) \in \mathbb{R}^{2} \mid\left[\begin{array}{l}
%       u-c_{u} \\
%       v-c_{v}
%       \end{array}\right]^{\top}\left[\begin{array}{ll}
%       E_{u u} & E_{u v} \\
%       E_{u v} & E_{v v}
%       \end{array}\right]^{-1}\left[\begin{array}{l}
%       u-c_{u} \\
%       v-c_{v}
%       \end{array}\right] \leq 1
%     \right\}
%   }
% \]  
% where $E_{u u}=\frac{2}{N_{p}} \sum_{p}\left(u_{p}-c_{u}\right)^{2}$, $E_{v v}=\frac{2}{N_{p}} \sum_{p}\left(v_{p}-c_{v}\right)^{2}$, $E_{u v}=\frac{2}{N_{p}} \sum_{p}\left(u_{p}-c_{u}\right)\left(v_{p}-c_{v}\right)$. 
%The axes lengths are the eigenvalues $\lambda_{0}, \lambda_{1}$ of $\bfE_k$. The dual form of the ellipse is $\bfH_k = \text{adj}(\bfE_k)$\NA{This is not true. Look at how the ellipse is defined using $\bfE_k^{-1}$ which is both inverted and a $2 \times 2$ matrix.}
% $
% \left[\begin{array}{ll}
%   E_{u u} & E_{u v} \\
%   E_{u v} & E_{v v}
% \end{array}\right]
% $.
% The dual form of ellipse is the conic and can be obtained by the adjoint operator.\NA{I am not sure what this sentence is saying. How is $\bfH_k$ related to $\bfE_k$?} 
% , \beta = (\beta_{1}, \beta_{2},..., \beta_{k})^\top

An ellipsoid in dual quadric form $\bfQ^*$ in global coordinates and its conic projection $\bfH_k^*$ in image $k$ are related by $\beta_{k} \mathbf{H}_{k}^*=\mathbf{P} \bfC_k^{-1} \mathbf{Q}^* \bfC_k^{-\top} \mathbf{P}^{\top}$ defined up to a scale factor $\beta_{k}$. This equation can be rearranged to $\beta_{k} \mathbf{h}_{k} = \mathbf{G}_k\mathbf{v}$, where $\mathbf{h}_{k} = \operatorname{vech}(\mathbf{H}_{k}^*)$, $\mathbf{h}_{k} \in \mathbb{R}^6$, $\mathbf{v} = \operatorname{vech}(\mathbf{Q}^*)$ and $\mathbf{v} \in \mathbb{R}^{10}$. The operator $\operatorname{vech}$ serializes the lower triangular part of a symmetric matrix, and $\mathbf{G}_k$ is a matrix that depends on $\mathbf{P} \bfC_k^{-1}$. The explicit form of $\mathbf{G}_k$ is derived in (5) in \cite{rubino20173d}. 
Next, a least squares system is constructed from the multi-view observations. By stacking all observations, we obtain $\mathbf{M} \mathbf{w} = \bf0$, where $\mathbf{w} = (\mathbf{v}, \beta_1,\ldots,\beta_k)^\top$, and $\bfM$ is defined in (8) in  \cite{rubino20173d}.  
This leads to a least squares system:
%
\begin{equation}
\label{eq:ellipsoid_lsq}
\hat{\mathbf{w}}=\arg \min _{\bfw}\left\|\mathbf{M} \mathbf{w}\right\|_{2}^{2} \quad \text { s.t. } \quad\|\mathbf{w}\|_{2}^{2}=1,
\end{equation}
%
which can be solved by applying SVD to $\mathbf{M}$, and taking the right singular vector associated to the minimum singular value. The constraint $\|\mathbf{w}\|_{2}^{2}=1$ avoids a trivial solution. The first $10$ entries of $\hat{\mathbf{w}}$ are $\hat{\mathbf{v}}$, which is a vectorized version of the dual ellipsoid $\hat{\mathbf{Q}}^*$ in the global frame. To avoid degenerate quadrics, a variant of the least squares system in \eqref{eq:ellipsoid_lsq} is proposed in \cite{gay2018visual}, which constrains the center of the ellipse and the reprojection of the center of the 3D ellipsoid to be close. Thus, we modify $\bfM$ using the version in (9) in \cite{gay2018visual} to improve the estimation.

The object pose $\hat{\bfT}^{-1}$ can be recovered by relating the estimated ellipsoid $\hat{\mathbf{Q}}^*$ in global coordinates to the ellipsoid $\bfQ^*_{\bfu}$ in the canonical coordinate frame predicted by the coarse shape decoder $\bfu = g_{\bfphi}(\bfz)$ using the average class shape $\bfz$: 
%from the estimated ellipsoid $\hat{\mathbf{Q}}^*$ using \eqref{eq:ellipsoid}. Since $\bfQ^*$ can be parameterized as a centered ellipsoid $\bfQ^*_{\bfu}$ transformed by the object pose:
\begin{equation*}
\begin{aligned}
\hat{\bfQ}^*\! =
\hat{\mathbf{T}}^{-1} \bfQ_{\bfu}^* \hat{\mathbf{T}}^{-\top}\!\!=
% \left[\begin{array}{cc}
% \mathbf{R} & \bft \\
% \mathbf{0}^{\top} & 1
% \end{array}\right]\left[\begin{array}{cc}
% \mathbf{U} \mathbf{U}^{\top} & \mathbf{0} \\
% \mathbf{0} & -1
% \end{array}\right]\left[\begin{array}{ll}
% \mathbf{R}^{\top} & \mathbf{0} \\
% \bft^{\top} & 1
% \end{array}\right] \\ 
% &=
\begin{bmatrix} 
\hat{s}^2 \hat{\mathbf{R}} \bfU\bfU^\top \hat{\mathbf{R}}^\top -  \hat{\bft} \hat{\bft}^\top & - \hat{\bft} \\ -\hat{\bft}^\top & -1
\end{bmatrix}.
\end{aligned}
\end{equation*}
The translation $\hat{\bft}$ can be recovered from the last column of $\hat{\bfQ}^*$.
%$\hat{\bft} = -\bfP \hat{\mathbf{Q}} [0,0,0,1]^\top$. 
To recover the rotation, note that $\bfA \triangleq \bfP\hat{\mathbf{Q}}^*\bfP^\top  + \hat{\bft} \hat{\bft}^\top = \hat{s}^2\hat{\bfR}\bfU\bfU^\top\hat{\bfR}^\top$ is a positive semidefinite matrix. Let its eigen-decomposition be $\bfA = \bfV\bfY\bfV^\top$, where $\bfY$ is a diagonal matrix containing the eigenvalues of $\bfA$. Since $\bfU\bfU^\top$ is diagonal, it follows that $\hat{\bfR} = \bfV$, while the scale $\hat{s}$ is obtained as $\hat{s} = \frac{1}{3} \sqrt{\tr(\bfU^{-1}\bfY \bfU^{-\top})}$.
%
% \begin{equation}
%   \label{eq:ellipsoid_scale}
%   \hat{s} = \frac{1}{3} \sqrt{\tr(\bfU^{-1}\bfY \bfU^{-\top})}
%   %\hat{s} = \frac{1}{3} (\frac{\sqrt{s_1}}{u_1} + \frac{\sqrt{s_2}}{u_2} + \frac{\sqrt{s_3}}{u_3}) 
% \end{equation}
%
%and $\bfU = \diag(\bfu)$ is a prior object shape obtained as the mean shape from the training set. 
% Finally, the ellipsoid $\hat{\mathbf{Q}}$ represented by $\hat{\bfU}, \hat{s}, \hat{\bfR}, \hat{\bft}$ is refined by the tangent plane residual. 
Note that although the $\text{SIM}(3)$ pose could also be recovered from the object point cloud, other outlier rejection methods are required \cite{wu2020eao} when the point cloud is noisy. 



%\subsubsection{Optimization}
%\label{sec:pose_shape_opt}

{\vspace{1ex}\bf \noindent Optimization: }%
Given the initialized instance transformation $\hat{\bfT}$ and initial shape deformation $\delta\hat{\bfz} = \mathbf{0}$, we solve the joint pose and shape optimization in \eqref{eq:test_optimization} via gradient descent. Note that the decoder parameters $\bftheta$, $\bfphi$ and the mean category shape code $\bfz$ are fixed during online inference. Since $\bfT$ is an element of the $\text{SIM}(3)$ manifold, the gradients and gradient steps need to be computed by projecting to the tangent $\text{sim}(3)$ vector space and retracting back to $\text{SIM}(3)$. We introduce local perturbations $\bfT = \exp\prl{\bfxi_\times} \hat{\bfT}$, $\delta\bfz = \delta\tilde{\bfz} + \delta\hat{\bfz}$ and derive the Jacobians of the error function in \eqref{eq:cost_function} with respect to $\bfxi$ and $\delta\tilde{\bfz}$. 



%To incorporate the proposed error functions in \eqref{eq:cost_function} in a tightly coupled localization and mapping algorithm such as~\cite{Atanasov_SemanticLocalization_IJRR15, sunderhauf2017meaningful, mur2017orb}, we could linearize the terms around the estimated camera and object states. However, to keep the presentation clear we leave this for future work, and use ground truth camera poses. Hence, in this paper we only linearize around the object instance $\hat{\bfi}$ using perturbation $\tilde{\bfi}$: $\bfT = \exp\prl{\bfxi_\times} \hat{\bfT}$, $\delta\bfz = \delta\tilde{\bfz} + \delta\hat{\bfz}$.
% \begin{equation}
% \label{eq:perturbations}
% \begin{aligned}
%   \bfT = \exp\prl{\bfxi_\times} \hat{\bfT} \quad 
% \delta\bfz = \delta\tilde{\bfz} + \delta\hat{\bfz}
% \end{aligned}
% \end{equation}
%We define the Jacobians of the pose and latent code perturbations with respect to the error functions next. Note that the parameters $\bftheta$, $\bfphi$ of the decoders, and the mean shape latent code $\bfz$ remain fixed during inference. 

\begin{proposition}
\label{prop:sdf-sim3-jacobians}
The Jacobian of $e_{\bftheta}$ in \eqref{eq:e_k} with respect to the transformation perturbation $\bfxi \in \mathfrak{sim}(3)$ is:
\begin{equation*}
% \label{eq:gk-jacobians}
\scaleMathLine[1]
{
  \begin{aligned}
    \frac{\partial e_{\bftheta}}{\partial \bfxi} 
    &= 
    \frac{\partial \rho(r)}{\partial r}
    \prl{
      \hat{s} [{\bf0}_6,1] f_{\bftheta} (\bfx,\delta\hat{\bfz})
      + 
      \hat{s} \nabla_{\bfx} f_{\bftheta}(\bfx,\delta\hat{\bfz})^\top
      \bfP \brl{\hat{\bfT} \underline{\bfx}}^\odot 
    }
    \\
    \frac{\partial e_{\bftheta}}{\partial \delta\tilde{\bfz}} 
    &= 
    \frac{\partial \rho(r)}{\partial r}
    \hat{s} \nabla_{\bfz} f_{\bftheta}(\bfx,\delta\hat{\bfz}), 
    \end{aligned}
}
\end{equation*}
where $f_{\bftheta}(\bfx,\delta\hat{\bfz}) = f_{\bftheta}(\bfP\hat{\bfT} \underline{\bfx}; \bfz+\delta\hat{\bfz})$ is defined in \eqref{eq:e_k} and $\frac{\partial \rho(r)}{\partial r}$ is the derivative of the Huber term in \eqref{eq:huber_loss} evaluated at $r = \hat{s} f_{\bftheta}(\bfx,\delta\hat{\bfz}) - d$:
\[
  \frac{\partial \rho(r)}{\partial r}
  =\left\{\begin{array}{ll}
    r & |r| \leq \delta \\
    \text{sign}(r)\delta  & \text{ else}. 
    \end{array}\right.
\] 
%The fine level autodecoder $f_{\bftheta}(\bfx,\delta\hat{\bfz}) = f_{\bftheta}(\bfP\hat{\bfT} \underline{\bfx}; \bfz+\delta\hat{\bfz})$ is defined in \eqref{eq:e_k}. 
The operator $\underline{\bfx}^\odot$ is defined as:
\begin{equation*}
\underline{\bfx}^\odot \triangleq \begin{bmatrix} \bfI_3 & -\bfx_\times & \bfx\\ \mathbf{0}^\top & \mathbf{0}^\top & 0 \end{bmatrix} \in \mathbb{R}^{4 \times 7}. 
\end{equation*}
% $r = \hat{s} f_{\bftheta}(\bfP\hat{\bfT} \bfC_k\underline{\bfx}; \bfz+\delta\hat{\bfz}) - d$, \YYJ{ abbreviated as $f_{\bftheta}(\bfx,\delta\hat{\bfz})$}.
\end{proposition}

%The first equality is proved as follows.
\begin{proof}
Using the chain rule and the product rule:
\begin{equation*}
\begin{aligned}
  \frac{\partial e_{\bftheta}}{\partial\bfxi} 
  = 
  \frac{\partial e_{\bftheta}}{\partial r}
  \frac{\partial r}{\partial\bfxi}
  = 
    \frac{\partial e_{\bftheta}}{\partial r}
    \prl{
      \frac{\partial s}{\partial\bfxi}
      f_{\bftheta} (\bfx,\delta\bfz)
      +
      s
      \frac{\partial f_{\bftheta}}{\partial{}^{}_{O}\bfx}
      \frac{\partial {}^{}_{O}\bfx}{\partial\bfxi}
    }, 
\end{aligned}
\end{equation*}
where ${}^{}_{O}\bfx = \bfP\bfT \underline{\bfx}$ is a point in the object frame. We have $
\frac{\partial s}{\partial\bfxi}
  = 
  e^\sigma [{\bf0}_6,1]
  = s [{\bf0}_6,1]
$. 
The term $s\frac{\partial f_{\bftheta}}{\partial{}^{}_{O}\bfx}$ is the gradient of the fine-level SDF decoder with respect to the input $s \nabla_{\bfx} f_{\bftheta}(\bfx, \delta\bfz)$, which could be obtained from auto-differentiation. Finally, we have:
%
\begin{equation*}
% \label{eq:sdf-sim3-jacobians-proof2b}
\begin{aligned}
{}^{}_{O} \bfx &= \bfP \bfT \underline{\bfx} \approx 
\bfP (\bfI + \bfxi_\times) \hat\bfT \underline{\bfx} \\
&=
\bfP \hat\bfT \underline{\bfx}
+
\bfP \bfxi_\times \hat\bfT \underline{\bfx} \\ 
&= 
\bfP \hat\bfT \underline{\bfx}
+
\underbrace{\bfP [\hat\bfT \underline{\bfx}]^\odot}_{\frac{\partial {}^{}_{O} \bfx}{\partial \bfxi}} \bfxi. 
% \underbrace{\bfP [\hat\bfT \underline{\bfx}]^\odot}_{Jacobian} \bfxi
\end{aligned}\qedhere
\end{equation*}
\end{proof}


In the second equality in Prop.~\ref{prop:sdf-sim3-jacobians}, the term $\frac{\partial \rho(r)}{\partial r} \hat{s} \nabla_{\bfz} f_{\bftheta}(\bfx, \delta\hat{\bfz})$ is the gradient of the fine-level SDF loss with respect to the input $\bfz$ and can be obtained via auto-differentiation. The Jacobians of the coarse-level SDF error $\frac{\partial e_{\bfphi}}{\partial \bfxi}$, $\frac{\partial e_{\bfphi}}{\partial \delta \tilde{\bfz}}$ can be obtained in a similar way. 

%Lastly, the object pose and the shape latent codes are optimized via gradient descent.
%The initialization step provides initial estimates $\bfT^0$, $\delta \bfz^0$. 

After obtaining the Jacobians, the pose and latent shape code can be optimized via:
\begin{equation*}
\begin{aligned}
\bfT^{i+1} &\triangleq 
\exp \prl{- \eta_1 
\frac{\partial e(\bfT,\delta \bfz, \bftheta^*, \bfphi^* ; \crl{\calX_k(\bfp)})}{\partial \bfxi}}
\bfT^{i}
\\
\delta \bfz^{i+1} &\triangleq 
\delta \bfz^{i} - \eta_2 
\prl{\frac{\partial e(\bfT,\delta \bfz, \bftheta^*, \bfphi^* ; \crl{\calX_k(\bfp)})}{\partial \delta \bfz}}, 
\end{aligned}
\end{equation*}
where $\eta_1, \eta_2$ are step sizes, $\delta \bfz^0 = \mathbf{0}$, and $\bfT^0 = \hat{\bfT}$ is obtained from the initialization. During optimization, we add regularization $e_r(\delta\bfz) = \|\delta\bfz\|_2^2$ to restrict the amount of latent code deformation.


\section{Evaluation}
\label{sec: evaluate}


% \begin{figure*}[h]
%     \centering
%     % First row
%     \includegraphics[width=0.19\textwidth, trim={0.5cm 0.2cm 2cm 0.0cm},clip]{fig/SE(2)_R2_compare/distance_plot_0.00.png}
%     \includegraphics[width=0.19\textwidth, trim={0.5cm 0.2cm 2cm 0.0cm},clip]{fig/SE(2)_R2_compare/distance_plot_0.79.png}
%     \includegraphics[width=0.19\textwidth, trim={0.5cm 0.2cm 2cm 0.0cm},clip]{fig/SE(2)_R2_compare/distance_plot_1.57.png}
%     \includegraphics[width=0.19\textwidth, trim={0.5cm 0.2cm 2cm 0.0cm},clip]{fig/SE(2)_R2_compare/distance_plot_3.14.png}
%     \includegraphics[width=0.19\textwidth, trim={0.5cm 0.2cm 2cm 0.0cm},clip]{fig/SE(2)_R2_compare/distance_plot_4.71.png}
%     \caption{Comparative analysis of the $SE(2)$ and $\bbR^2$ signed distance functions for elliptical obstacles. The cyan triangle represents the rigid-body robot, with its orientation varying across the sequence. The importance of considering robot orientation in distance computations becomes evident: while the $SE(2)$ function accounts for this orientation, the $\bbR^2$ approximation treats the robot as an encapsulating circle with radius $1$. Level sets at distances $0.2$ and $2$ are depicted for both functions.}
%     \label{fig:se2_r2_compare}
% \end{figure*}

% \begin{figure*}[h]
%     \centering
%     % First row
%     \subcaptionbox{Initial Pose \label{fig:2a}}
%     {\includegraphics[width=0.19\textwidth, trim={3cm 0.2cm 3cm 0.0cm},clip]{fig/unicycle/snapshot_step_0.png}}
%     \subcaptionbox{Time t = 1.66 sec \label{fig:2b}}
%     {\includegraphics[width=0.19\textwidth, trim={3cm 0.2cm 3cm 0.0cm},clip]{fig/unicycle/snapshot_step_83.png}}
%     \subcaptionbox{Time t = 4.12 sec \label{fig:2c}}
%     {\includegraphics[width=0.19\textwidth, trim={3cm 0.2cm 3cm 0.0cm},clip]{fig/unicycle/snapshot_step_206.png}}
%     \subcaptionbox{Final Pose \label{fig:2d}}
%     {\includegraphics[width=0.19\textwidth, trim={3cm 0.2cm 3cm 0.0cm},clip]{fig/unicycle/snapshot_step_305.png}}
%     \subcaptionbox{Circular Robot \label{fig:2e}}
%     {\includegraphics[width=0.19\textwidth, trim={3cm 0.2cm 3cm 0.0cm},clip]{fig/unicycle/circular_snapshot_step_451.png}}
    
%     % Space between rows
%     %\vspace{-1mm}
    
%     % Second row
%     % \includegraphics[width=0.19\textwidth, trim={3cm 0.2cm 3cm 0.0cm},clip]{fig/unicycle/circular_snapshot_0.png}
%     % \includegraphics[width=0.19\textwidth, trim={3cm 0.2cm 3cm 0.0cm},clip]{fig/unicycle/circular_snapshot_40.png}
%     % \includegraphics[width=0.19\textwidth, trim={3cm 0.2cm 3cm 0.0cm},clip]{fig/unicycle/circular_snapshot_82.png}
%     % \includegraphics[width=0.19\textwidth, trim={3cm 0.2cm 3cm 0.0cm},clip]{fig/unicycle/circular_snapshot_124.png}
%     % \includegraphics[width=0.19\textwidth, trim={3cm 0.2cm 3cm 0.0cm},clip]{fig/unicycle/circular_snapshot_176.png}
    
%     \caption{Safe navigation in a dynamical elliptical environment. (a) shows the initial pose of the triangular robot and the environment. (b) shows the triangular robot passing through the narrow space between two moving ellipses. (c) shows the robot adjusts its pose to avoid the moving obstacle. (d) shows the final pose of the robot that reaches the goal region and the current environment. In (e), we plot the trajectory of navigating a circular robot in the same environment.}
%     \label{fig:safe_navigation}
% \end{figure*}

In this section, we show the efficacy of our proposed CBF construction techniques using simulation examples, focusing on ground-robot navigation and 2-D robot arm control.


Fig.~\ref{fig:se2_r2_compare} contrasts the $SE(2)$ distance function with the $\mathbb{R}^2$ counterpart by visualizing their level sets. Our proposed $SE(2)$ approach incorporates the orientation of the rigid-body robot, yielding notably improved results, particularly when the robot is close to obstacles.







To highlight the significance of accurate robot shape representation, we draw a comparison with a baseline circular robot CBF formulation. In Fig.~\ref{fig:safe_navigation}, we compare safe navigation using our proposed $SE(2)$ CBF approach with a regular $\mathbb{R}^2$ CBF approach. For both methods,  we set ${\bfk}(\bfx) = [v_{\max}, 0]^\top$ where $v_{\max} = 3.0$ is the maximum linear velocity. The remaining parameters were $\lambda = 100$, $\alpha_V(V(\bfx)) = 2V(\bfx)$, and $\alpha_h(h(\bfx, t)) = 3 h(\bfx, t)$.

We demonstrate safe navigation to a goal state. In Fig.~\ref{fig:2a}, the triangular robot starts the navigation with position centered at $(0,0)$ and orientation $\theta = \pi / 4$. In Fig.~\ref{fig:2b}, the robot adeptly navigates the narrow passage between two dynamic obstacles. In Fig.~\ref{fig:2d}, we see that the robot is able to reach the goal region without collision. In Fig.~\ref{fig:2e}, when the robot is conservatively modeled as a circle navigating the identical environment, it is evident that the robot has to opt for a more circuitous route to circumvent obstacles. This is due to its inability to traverse certain constricted spaces, as illustrated in Fig.~\ref{fig:2b}. These outcomes underscore the superior performance of our $SE(2)$ CBF methodology. Another advantage of the $SE(2)$ formulation lies in its assurance of a uniformly relative degree of $1$ for the constructed CBF, obviating the need to model a point off the wheel axis~\cite{cortes2017coordinated}.

% \begin{figure*}[h]
%     \centering
%     % First row
%     \subcaptionbox{Initial Pose \label{fig:3a}}
%     {\includegraphics[width=0.19\textwidth, trim={0.5cm 0.2cm 2cm 0.0cm},clip]{fig/robot_arm/arm_snapshot_0.png}}
%     \subcaptionbox{Time t = 4.12 sec \label{fig:3b}}
%     {\includegraphics[width=0.19\textwidth, trim={0.5cm 0.2cm 2cm 0.0cm},clip]{fig/robot_arm/arm_snapshot_206.png}}
%     \subcaptionbox{Time t = 4.92 sec \label{fig:3c}}
%     {\includegraphics[width=0.19\textwidth, trim={0.5cm 0.2cm 2cm 0.0cm},clip]{fig/robot_arm/arm_snapshot_246.png}}
%     \subcaptionbox{Time t = 6.28 sec  \label{fig:3d}}
%     {\includegraphics[width=0.19\textwidth, trim={0.5cm 0.2cm 2cm 0.0cm},clip]{fig/robot_arm/arm_snapshot_314.png}}
%     \subcaptionbox{Final Pose \label{fig:3e}}
%     {\includegraphics[width=0.19\textwidth, trim={0.5cm 0.2cm 2cm 0.0cm},clip]{fig/robot_arm/arm_snapshot_383.png}}
%     \caption{Safe stabilization of a 3-joint robot arm. The green circle denotes the goal region, and the gray box denotes the base of the arm. The arm is shown in blue and the trajectory of its end-effector is shown in red. The trajectories of the moving elliptical obstacles are shown in purple.}
%     \label{fig:robot_arm_safety}
% \end{figure*}

In the following set of experiments, we consider safe stabilization of a 3-joint robot arm in a dynamical elliptical environment. We set ${\bfk}(\bfx) = [0, 0, 0]^\top$ and restrict the joint control bounds with $|\omega_i | \leq 3$. In Fig.~\ref{fig:robot_arm_safety}, the robot arm is able to elude the mobile ellipses by nimbly adjusting its pose. In Fig.~\ref{fig:robot_arm_control_input}, we show the control inputs of each joint over time. We see that when the robot arm is close to the obstacles, it is able to take large control inputs in adjusting its pose. In Fig.~\ref{fig:robot_arm_LF_BF}, we show the CLF and CBF values over time. A consistently positive CBF value throughout the trajectory signifies safety assurance, while the decreasing CLF values indicates the convergence to the desired state. Moreover, the CLF value may increase when the arm is close to obstacles (i.e. CBF value is low), this comes from the relaxation of the CLF-CBF QP to ensure the feasibility of the program. 





\begin{figure}
    \centering
    \includegraphics[width = 0.48\textwidth]{fig/robot_arm/control_input_history_3_links.png}
    \caption{Control input of the 3-joint robot arm. }
    \label{fig:robot_arm_control_input}
\end{figure}

\begin{figure}
    \centering
    \includegraphics[width = 0.48\textwidth]{fig/robot_arm/lyapunov_and_barrier_history_3_links.png}
    \caption{Lyapunov function and barrier function values over time.}
    \label{fig:robot_arm_LF_BF}
\end{figure}
\section{Conclusion}
\label{sec:conclusion}
This paper presents a generic top-$\size$ recommendation framework for  trading-off accuracy, novelty, and coverage. To achieve this, we profile the users according to their preference for long-tail novelty. We examine various measures, and formulate an optimization problem to learn these user preferences from interaction data.  We then integrate the user preference estimates in our generic framework, GANC.  Extensive experiments on several datasets confirm that there are trade-offs between accuracy, coverage, and novelty. Almost all re-ranking models increase coverage and novelty at the cost of accuracy. However, existing re-ranking models typically rely on rating prediction models, and are hence more effective in dense settings. Using a generic approach, we can easily incorporate a suitable base accuracy recommender to devise an effective solution for both sparse and dense settings.  %Our results  also indicate there is no single method that outperforms other methods in all metrics. However our techniques obtain a significant improvement in coverage, while  . 
Although we integrated the  long-tail novelty preference estimates into a re-ranking framework, their use-case is not limited to these frameworks. In  the future, we intend to explore the temporal and topical dynamics of long-tail novelty preference, particularly in settings where contextual information is  available.  
%We achieve these objectives without using any additional contextual information.


\iffalse
While we focused on promoting long-tail items across users, we did not consider diversity of individual top-$\size$ recommendations, a factor that has been shown to positively affect consumer satisfaction. This is one direction for future work. Moreover, the sequential setting  in our work, creates a dependency between different batches, where,  the items recommended to a batch of users, depends on those recommended to previous batches. This dependency is created through the parameter $\mathbf{f}$, that is updated every time a top-$\size$ set  is allocated to a batch of users. A future direction for our work is to estimate a distribution over $\mathbf{f}$ that allows us to independently solve the problem for each user, leading to improvements across all performance metrics, including recommendation time. 

We design algorithms that take advantage of the structure in the value functions to obtain both efficient and scalable solutions. 
We design an algorithm that takes advantage of the structure in the value functions to obtain both efficient and scalable solutions. 

\textcolor{red}{Our  sequential  algorithms can be applied for batch recommendation contexts,~e.g., personalized email marketing, where based on prior interaction data between users and items,  a new round of recommendations must be sent to all users in the system.  However, the independent coverage algorithms lift the sequential setting restrictions and allow it be applied for re-ranking the output of base recommender in any setting. }A future direction for our work is to incorporate explicit diversity metrics in the framework. 
\fi


%We have a presented a submodular maximization framework to systematically trade-off relevance and diversity in recommendations to individual users and coverage across the item-space. This ensures both consumer and producer satisfaction. We model users according to their risk and focusing degrees and promote long-tail items to the right group of consumers. Consequently, we obtain a significant improvement in coverage while maintaining reasonable levels of user satisfaction. Furthermore, our methods are able to achieve a more balanced distribution across the set of recommended items. In the future, we plan to investigate the effect of using alternative base recommender systems. 

%Future Work
%However most of these methods assume that the ratings are missing at random (MAR). Since our method of generating recommendations is based on the completed matrix, assuming MAR might introduce additional bias, we will use methods which assume that the ratings at missing not at random (MNAR),explored in~\cite{steck2010training, icml2014c2_hernandez-lobatob14}. 	 
%Long Tail %Recently, authors in~\cite{cremonesi2010performance} conducted extensive experiments to evaluate the performances of various matrix factorization-based algorithms and neighborhood models on the task of recommending long tail items. Their experimental results show that long tail recommendation leads to a decrease in accuracy for all algorithms. They also showed that for this task, SVD outperforms other state-of-the-art algorithms. 


{\small
\bibliographystyle{ieee}
\bibliography{egbib}
}

\newpage 
\section*{Supplementary Material}

\subsection*{Trained Object Models}

This section provides additional visualizations for the trained object models. Training loss for the chair category is visualize in Fig.~\ref{fig:training_loss_chair}, which shows the loss is decreasing and stabilizes around 40,000 epochs. 

Fig.~\ref{fig:trained_model_chair} visualizes the rendering results for some chairs in the training set. It shows that the scale of the primitive-based representation varies proportionally with the high-resolution representation. 

\begin{figure}[thp!]
    \centering
    \includegraphics[width=\linewidth]{loss_chairs.jpg}
    \caption{Visualization of the training loss for chairs.}
    \label{fig:training_loss_chair}
\end{figure}


Fig.~\ref{fig:trained_model_sofa} visualizes the rendering results for sofas in the training set. There is a lack of shape variation since the majority of sofas have similar structure. Nevertheless, the ellispoid for the angle sofa is still different with that of other sofas. 

\begin{figure}[thp!]
    \centering
    \includegraphics[width=\linewidth]{trained_model_chair.jpg}
    \caption{Visualization of the trained object model for chairs. Upper row: coarse ellipsoid shapes regressed from $g_{\bfphi}$ and $\bfz$. Lower row: SDF object model from $f_{\bftheta}$ and $\bfz$.}
    \label{fig:trained_model_chair}
\end{figure}





\begin{figure}[thp!]
    \centering
    \includegraphics[width=\linewidth]{trained_model_sofa.jpg}
    \caption{Visualization of the trained object model for sofas. Upper row: coarse ellipsoid shapes regressed from $g_{\bfphi}$ and $\bfz$. Lower row: SDF object model from $f_{\bftheta}$ and $\bfz$.}
    \label{fig:trained_model_sofa}
\end{figure}



Fig.~\ref{fig:trained_model_table} visualizes the rendering results for tables in the training set. Similar to sofas, the variation is limited due to similar table shapes. Nonetheless, the ellipsoid for the rounded table is different from the rest. 


\begin{figure}[thp!]
    \centering
    \includegraphics[width=\linewidth]{trained_model_table.jpg}
    \caption{Visualization of the trained object model for tables. Upper row: coarse ellipsoid shapes regressed from $g_{\bfphi}$ and $\bfz$. Lower row: SDF object model from $f_{\bftheta}$ and $\bfz$.}
    \label{fig:trained_model_table}
\end{figure}



Fig.~\ref{fig:trained_model_trashbin} visualizes the rendering results for trashbins in the training set. It could be observed that the ellipsoid shape varies based on the object shape, for instance, the ellipsoid is enlongated for a tall trashbin. 


\begin{figure}[thp!]
    \centering
    \includegraphics[width=\linewidth]{trained_model_trashbin.jpg}
    \caption{Visualization of the trained object model for trashbins. Upper row: coarse ellipsoid shapes regressed from $g_{\bfphi}$ and $\bfz$. Lower row: SDF object model from $f_{\bftheta}$ and $\bfz$.}
    \label{fig:trained_model_trashbin}
\end{figure}


\begin{figure}[thp!]
    \centering
    \includegraphics[width=\linewidth]{trained_model_display.jpg}
    \caption{Visualization of the trained object model for displays. Upper row: coarse ellipsoid shapes regressed from $g_{\bfphi}$ and $\bfz$. Lower row: SDF object model from $f_{\bftheta}$ and $\bfz$.}
    \label{fig:trained_model_display}
\end{figure}

\begin{figure}[thp!]
    \centering
    \includegraphics[width=\linewidth]{trained_model_cabinet.jpg}
    \caption{Visualization of the trained object model for cabinets. Upper row: coarse ellipsoid shapes regressed from $g_{\bfphi}$ and $\bfz$. Lower row: SDF object model from $f_{\bftheta}$ and $\bfz$.}
    \label{fig:trained_model_cabinet}
\end{figure}

Fig.~\ref{fig:trained_model_display} visualizes the rendering results for displays in the training set. The ellipsoid is rounded for the thicker display and is very thin for the rest. 

Fig.~\ref{fig:trained_model_cabinet} visualizes the rendering results for cabinets in the training set. The ellipsoid varies according to the different cabinet shapes. 



\subsection*{More Qualitative Results on ScanNet}

This section presents more qualitative results on ScanNet~\cite{dai2017scannet}. 
Fig.~\ref{fig:scannet_qualitative_0077_01} shows a reconstruction with table, trashbins, and cabinet. The cabinet and trashbins are reconstructed well, as can be seen from the resulting meshes which resemble the original object shapes. However, the table is poorly reconstructed, since the shape is quite different and the pose is inaccurate. This is because the available observation in the scene for the table is very limited, as can be seen in the segmented mesh, which is insufficient for optimization. 



A ScanNet scene with bookshelves and tables are shown in Fig.~\ref{fig:scannet_qualitative_0208_00}, to demonstrate the usefulness of the coarse and fine level residuals. The figure illustrates that the initialized object pose and shape are different from the actual scene, since the two bookshelves in the center are not parallel and are too small compared to the observation. In contrast, the bookshelves become larger after applying the fine level residual, which is more consistent with the observations. The reconstructions are further improved with both the coarse and fine level residuals, where the bookshelves become parallel. Moreover, the bottom bookshelf and the top right table also become thinner, which agrees more with the observation. 
This example clearly shows the effectiveness of the proposed bi-level model for joint object pose and shape optimization.  

\begin{figure}[thp!]
    \centering
    \includegraphics[width=\linewidth]{qualitive_0077_01.jpg}
    \caption{Visualization of the original scene and reconstructed objects for ScanNet scene $0077$. The green arrows point to the segmented mesh of the objects.}
    \label{fig:scannet_qualitative_0077_01}
\end{figure}

\begin{figure*}[thp!]
    \centering
    \includegraphics[width=\linewidth]{qualitive_0208_00.jpg}
    \caption{Visualization of the original scene and reconstructed objects for ScanNet scene $0208$. First row from left to right: original scene, reconstruction using initialized pose and mean categorical object shape, reconstruction using optimized pose and shape with fine level residual only, reconstruction using optimized pose and shape with both coarse and fine level residuals. Second row from left to right: original scene with bookshelves and tables highlighted in light blue and beige, the rest are reconstructions overlaid with object point clouds and added pseudo points.}
    \label{fig:scannet_qualitative_0208_00}
\end{figure*}

\subsection*{Pose Estimation Metric}

This section presents the metric used to evaluate the object pose, which follows Scan2CAD~\cite{avetisyan2019scan2cad}. 
We introduce the details on how to decompose a pose $\bfT \in \text{SIM}(3)$ into rotation $\bfq$, translation $\bfp$ and scale $\bfs$ and the error functions for each element separately.
For rotation and scale, $\bfR_s = \bfP\bfT\bfP^\top$:
\begin{equation}
\label{eq:pose_error}
\begin{aligned}
s_1 &= \| \bfR_s\bfe_1 \|_2 \quad 
s_2 = \| \bfR_s\bfe_2 \|_2 \quad 
s_3 = \| \bfR_s\bfe_3 \|_2, \\
\bfR\bfe_1 &= \frac{\bfR_s\bfe_1}{s_1} \quad 
\bfR\bfe_2 = \frac{\bfR_s\bfe_2}{s_2}
\quad 
\bfR\bfe_3 = \frac{\bfR_s\bfe_3}{s_3}. 
\end{aligned}
\end{equation}
Suppose $\boldsymbol{R}=\left\{m_{i j}\right\}, i, j \in[1,2,3]$, we transform it to quaternion $\bfq$ by 
\begin{equation}
\scaleMathLine[0.9]
{
\begin{aligned}
q_{0}=\frac{\sqrt{\operatorname{tr}(R)+1}}{2}, q_{1}=\frac{m_{23}-m_{32}}{4 q_{0}}, q_{2}=\frac{m_{31}-m_{13}}{4 q_{0}}, q_{3}=\frac{m_{12}-m_{21}}{4 q_{0}}. 
\end{aligned}
}
\end{equation}
Suppose the prediction and groundtruth are $\bfq_{pred}, \bfq_{gt}$, we compute the difference by 
\begin{equation}
\begin{aligned}
e_{\text{SO(3)}}(\bfq,\hat{\bfq}) := 2 \arccos (| \bfq_{gt}^\top \bfq_{pred} |). 
\end{aligned}
\end{equation}
Translation is $\bfp = \bfT[1:3, 4]$, and we compare the difference between prediction and groundtruth by 
\begin{equation}
\| \bfp_{pred} - \bfp_{gt} \|_2. 
\end{equation}
For scale percentage error, we compute it by 
\begin{equation}
100\times | \frac{1}{3} \sum_{i=1}^{3} \bar{s}_i - 1 |,
\end{equation}
where $\bar{s}_i = \frac{s_{pred}}{s_{gt}}$ for each of $s_1, s_2, s_3$ recovered from the $\text{SIM}(3)$ matrix. 

\subsection*{Timing}

\begin{table}[tph!]
    \centering
    \caption{ELLIPSDF timing breakdown (sec)}
    \label{tab:time}
    % \vspace*{-1ex}
    \scalebox{0.78}{
        \begin{tabular}{c|c|c|c|c} % <-- Alignments: 1st column left, 2nd middle and 3rd right, with vertical lines in between
        \hline
        Init & Latent Code Opt & SIM(3) Opt & SDF Decoding & Meshing \\
        \hline
        0.04 & 0.13 & 0.58 & 1.38 & 2.34 \\
        \hline
        \end{tabular}}
    % \vspace*{-1.2ex}
\end{table}

Timing for one instance is provided in Table~\ref{tab:time}. \textit{Init} is the pose initialization in (14) for 100 views. \textit{Latent Code Opt} and \textit{SIM(3) Opt} are a single SGD step with respect to $\delta \bfz$ and $\bfT$ respectively using 10000 points as batch size. \textit{SDF Decoding} and \textit{Meshing} are optional steps that generate SDF predictions over $256^3$ points and apply Marching Cubes to generate a mesh. Our approach does not currently operate in real-time but it is more efficient than existing work. We will investigate how to accelerate the current slow python SIM(3) optimization.


\section*{Acknowledgments}
The first author would like to thank Kejie Li at University of Adelaide for helpful discussions.  

\end{document}



