\documentclass[sigconf]{acmart}

\usepackage{booktabs} % For formal tables


\usepackage{amssymb}
\usepackage{amsthm, amsmath}
\usepackage{algorithmic}
\usepackage[ruled]{algorithm2e}
\usepackage{color}
\usepackage{url}
\usepackage{xfrac}

\usepackage{subfigure}
\usepackage{tablefootnote}
 
\newcommand{\red}[1]{\textcolor{red}{#1}}
\newcommand{\blue}[1]{\textcolor{blue}{#1}}

% Copyright
%\setcopyright{none}
%\setcopyright{acmcopyright}
%\setcopyright{acmlicensed}
%\setcopyright{rightsretained}
%\setcopyright{usgov}
%\setcopyright{usgovmixed}
%\setcopyright{cagov}
%\setcopyright{cagovmixed}

%%%%%% Added by Bang begin
\usepackage{tabularx}
%\usepackage{multicol} % use this will case single column title!!
\usepackage{multirow}
\usepackage{balance}
%%%%%% Added by Bang end



%Conference
%\acmConference[WOODSTOCK'97]{ACM Woodstock conference}{July 1997}{El
%  Paso, Texas USA} 
%\acmYear{1997}
%\copyrightyear{2016}

%\acmPrice{15.00}
\fancyhead{}  


\begin{document}
\title{Matching Natural Language Sentences \\with Hierarchical Sentence Factorization}

\copyrightyear{2018}
\acmYear{2018} 
\setcopyright{iw3c2w3}
\acmConference[WWW 2018]{The 2018 Web Conference}{April 23--27, 2018}{Lyon, France}
\acmBooktitle{WWW 2018: The 2018 Web Conference, April 23--27, 2018, Lyon, France}
\acmPrice{}
\acmDOI{10.1145/3178876.3186022}
\acmISBN{978-1-4503-5639-8}

% \titlenote{Produces the permission block, and
%   copyright information}
% \subtitle{Extended Abstract}
% \subtitlenote{The full version of the author's guide is available as
%   \texttt{acmart.pdf} document}

% \copyrightyear{2017} 
% \acmYear{2017} 
% \setcopyright{acmcopyright}
% \acmConference{CIKM'17}{}{November 6--10, 2017, Singapore.}
% \acmPrice{15.00}
% \acmDOI{https://doi.org/10.1145/3132847.3132852}
% \acmISBN{ISBN 978-1-4503-4918-5/17/11}

% \fancyhead{}
% \settopmatter{printacmref=false, printfolios=false}

% \author{Bang Liu}
% %\authornote{Dr.~Trovato insisted his name be first.}
% %\orcid{1234-5678-9012}
% \affiliation{%
%   \institution{Department of Electrical and Computer Engineering}
%   \streetaddress{University of Alberta}
%   \city{Edmonton} 
%   \state{Alberta, Canada} 
%   %\postcode{43017-6221}
% }
% %\email{bang3@ualberta.ca}

% \author{Di Niu}
% %\authornote{The secretary disavows any knowledge of this author's actions.}
% \affiliation{%
%   \institution{Department of Electrical and Computer Engineering}
%   \streetaddress{University of Alberta}
%   \city{Edmonton} 
%   \state{Alberta, Canada} 
%   %\postcode{43017-6221}
% }
% %\email{}

% \author{Kunfeng Lai}
% %\authornote{This author is the one who did all the really hard work.}
% \affiliation{%
%   \institution{Mobile Internet Group}
%   \streetaddress{Tencent}
%   \city{Shenzhen} 
%   \country{China}}
% %\email{}

% \author{Linglong Kong}
% \affiliation{%
%   \institution{Department of Mathematical and Statistical Sciences}
%   \streetaddress{University of Alberta}
%   \city{Edmonton} 
%   \state{Alberta, Canada} 
%   %\postcode{43017-6221}
% }
% %\email{}

% \author{Yu Xu}
% \affiliation{%
%   \institution{Mobile Internet Group}
%   \streetaddress{Tencent}
%   \city{Shenzhen} 
%   \country{China}}
% %\email{}




% \author{Charles Palmer}
% \affiliation{%
%   \institution{Palmer Research Laboratories}
%   \streetaddress{8600 Datapoint Drive}
%   \city{San Antonio}
%   \state{Texas} 
%   \postcode{78229}}
% \email{cpalmer@prl.com}

% \author{John Smith}
% \affiliation{\institution{The Th{\o}rv{\"a}ld Group}}
% \email{jsmith@affiliation.org}

% \author{Julius P.~Kumquat}
% \affiliation{\institution{The Kumquat Consortium}}
% \email{jpkumquat@consortium.net}

% % The default list of authors is too long for headers}
% \renewcommand{\shortauthors}{B. Trovato et al.}

\author{Bang Liu$^1$, Ting Zhang$^1$, Fred X. Han$^1$, Di Niu$^1$, Kunfeng Lai$^2$, Yu Xu$^2$}
       \affiliation{$^1$University of Alberta, Edmonton, AB, Canada}
 \affiliation{$^2$Mobile Internet Group, Tencent, Shenzhen, China}
       %\email{haolan@ualberta.ca}

%\email{bang3@ualberta.ca}


% \numberofauthors{5}
% \author{
% % You can go ahead and credit any number of authors here,
% % e.g. one 'row of three' or two rows (consisting of one row of three
% % and a second row of one, two or three).
% %
% % The command \alignauthor (no curly braces needed) should
% % precede each author name, affiliation/snail-mail address and
% % e-mail address. Additionally, tag each line of
% % affiliation/address with \affaddr, and tag the
% % e-mail address with \email.
% %
% % 1st. author
% %\alignauthor
% Bang Liu$^1$, Di Niu$^1$, Kunfeng Lai$^2$, Linglong Kong$^1$, Yu Xu$^$\\
%        \affaddr{$^1$University of Alberta, Edmonton, AB, Canada}\\
%  \affaddr{$^2$Mobile Internet Group, Tencent Inc., Shenzhen, China}\\
%        %\email{haolan@ualberta.ca}
% }



\begin{abstract}
Semantic matching of natural language sentences or identifying the relationship between two sentences is a core research problem underlying many natural language tasks.
%Prior research has proposed both unsupervised distance-based schemes, when training data is not available, and deep learning schemes for sentence matching, given training data. 
Depending on whether training data is available, prior research has proposed both unsupervised distance-based schemes and supervised deep learning schemes for sentence matching.
However, previous approaches either omit or fail to fully utilize the ordered, hierarchical, and flexible structures of language objects, as well as the interactions between them.
In this paper, we propose \textit{Hierarchical Sentence Factorization}---a technique %that is able 
to factorize a sentence into a hierarchical representation, with the components at each different scale reordered into a ``predicate-argument'' form. The proposed sentence factorization technique leads to the invention of: 1) a new unsupervised distance metric which calculates the semantic distance between a pair of text snippets by solving a penalized optimal transport problem while preserving the logical relationship of words in the reordered sentences, and 2) new multi-scale deep learning models for supervised semantic training, based on factorized sentence hierarchies.
We apply our techniques to text-pair similarity estimation and text-pair relationship classification tasks, based on multiple datasets such as STSbenchmark, the Microsoft Research paraphrase identification (MSRP) dataset, the SICK dataset, etc. Extensive experiments show that the proposed hierarchical sentence factorization can be used to significantly improve the performance of existing unsupervised distance-based metrics as well as multiple supervised deep learning models based on the convolutional neural network (CNN) and long short-term memory (LSTM).
% \footnote{This is an abstract footnote}. 
\end{abstract}

%
% The code below should be generated by the tool at
% http://dl.acm.org/ccs.cfm
% Please copy and paste the code instead of the example below. 
%
% \begin{CCSXML}
% <ccs2012>
%  <concept>
%   <concept_id>10010520.10010553.10010562</concept_id>
%   <concept_desc>Computer systems organization~Embedded systems</concept_desc>
%   <concept_significance>500</concept_significance>
%  </concept>
%  <concept>
%   <concept_id>10010520.10010575.10010755</concept_id>
%   <concept_desc>Computer systems organization~Redundancy</concept_desc>
%   <concept_significance>300</concept_significance>
%  </concept>
%  <concept>
%   <concept_id>10010520.10010553.10010554</concept_id>
%   <concept_desc>Computer systems organization~Robotics</concept_desc>
%   <concept_significance>100</concept_significance>
%  </concept>
%  <concept>
%   <concept_id>10003033.10003083.10003095</concept_id>
%   <concept_desc>Networks~Network reliability</concept_desc>
%   <concept_significance>100</concept_significance>
%  </concept>
% </ccs2012>  
% \end{CCSXML}

% \ccsdesc[500]{Computer systems organization~Embedded systems}
% \ccsdesc[300]{Computer systems organization~Redundancy}
% \ccsdesc{Computer systems organization~Robotics}
% \ccsdesc[100]{Networks~Network reliability}

% We no longer use \terms command
%\terms{Theory}

\settopmatter{printacmref=false}

% \keywords{Hierarchical Sentence Factorization; Sentence Reordering; Ordered Word Mover's Distance; Abstract Meaning Representation}
\maketitle

% \leavevmode
% \\
% \\
% \\
% \\
% \\
\section{Introduction}
\label{introduction}

AutoML is the process by which machine learning models are built automatically for a new dataset. Given a dataset, AutoML systems perform a search over valid data transformations and learners, along with hyper-parameter optimization for each learner~\cite{VolcanoML}. Choosing the transformations and learners over which to search is our focus.
A significant number of systems mine from prior runs of pipelines over a set of datasets to choose transformers and learners that are effective with different types of datasets (e.g. \cite{NEURIPS2018_b59a51a3}, \cite{10.14778/3415478.3415542}, \cite{autosklearn}). Thus, they build a database by actually running different pipelines with a diverse set of datasets to estimate the accuracy of potential pipelines. Hence, they can be used to effectively reduce the search space. A new dataset, based on a set of features (meta-features) is then matched to this database to find the most plausible candidates for both learner selection and hyper-parameter tuning. This process of choosing starting points in the search space is called meta-learning for the cold start problem.  

Other meta-learning approaches include mining existing data science code and their associated datasets to learn from human expertise. The AL~\cite{al} system mined existing Kaggle notebooks using dynamic analysis, i.e., actually running the scripts, and showed that such a system has promise.  However, this meta-learning approach does not scale because it is onerous to execute a large number of pipeline scripts on datasets, preprocessing datasets is never trivial, and older scripts cease to run at all as software evolves. It is not surprising that AL therefore performed dynamic analysis on just nine datasets.

Our system, {\sysname}, provides a scalable meta-learning approach to leverage human expertise, using static analysis to mine pipelines from large repositories of scripts. Static analysis has the advantage of scaling to thousands or millions of scripts \cite{graph4code} easily, but lacks the performance data gathered by dynamic analysis. The {\sysname} meta-learning approach guides the learning process by a scalable dataset similarity search, based on dataset embeddings, to find the most similar datasets and the semantics of ML pipelines applied on them.  Many existing systems, such as Auto-Sklearn \cite{autosklearn} and AL \cite{al}, compute a set of meta-features for each dataset. We developed a deep neural network model to generate embeddings at the granularity of a dataset, e.g., a table or CSV file, to capture similarity at the level of an entire dataset rather than relying on a set of meta-features.
 
Because we use static analysis to capture the semantics of the meta-learning process, we have no mechanism to choose the \textbf{best} pipeline from many seen pipelines, unlike the dynamic execution case where one can rely on runtime to choose the best performing pipeline.  Observing that pipelines are basically workflow graphs, we use graph generator neural models to succinctly capture the statically-observed pipelines for a single dataset. In {\sysname}, we formulate learner selection as a graph generation problem to predict optimized pipelines based on pipelines seen in actual notebooks.

%. This formulation enables {\sysname} for effective pruning of the AutoML search space to predict optimized pipelines based on pipelines seen in actual notebooks.}
%We note that increasingly, state-of-the-art performance in AutoML systems is being generated by more complex pipelines such as Directed Acyclic Graphs (DAGs) \cite{piper} rather than the linear pipelines used in earlier systems.  
 
{\sysname} does learner and transformation selection, and hence is a component of an AutoML systems. To evaluate this component, we integrated it into two existing AutoML systems, FLAML \cite{flaml} and Auto-Sklearn \cite{autosklearn}.  
% We evaluate each system with and without {\sysname}.  
We chose FLAML because it does not yet have any meta-learning component for the cold start problem and instead allows user selection of learners and transformers. The authors of FLAML explicitly pointed to the fact that FLAML might benefit from a meta-learning component and pointed to it as a possibility for future work. For FLAML, if mining historical pipelines provides an advantage, we should improve its performance. We also picked Auto-Sklearn as it does have a learner selection component based on meta-features, as described earlier~\cite{autosklearn2}. For Auto-Sklearn, we should at least match performance if our static mining of pipelines can match their extensive database. For context, we also compared {\sysname} with the recent VolcanoML~\cite{VolcanoML}, which provides an efficient decomposition and execution strategy for the AutoML search space. In contrast, {\sysname} prunes the search space using our meta-learning model to perform hyperparameter optimization only for the most promising candidates. 

The contributions of this paper are the following:
\begin{itemize}
    \item Section ~\ref{sec:mining} defines a scalable meta-learning approach based on representation learning of mined ML pipeline semantics and datasets for over 100 datasets and ~11K Python scripts.  
    \newline
    \item Sections~\ref{sec:kgpipGen} formulates AutoML pipeline generation as a graph generation problem. {\sysname} predicts efficiently an optimized ML pipeline for an unseen dataset based on our meta-learning model.  To the best of our knowledge, {\sysname} is the first approach to formulate  AutoML pipeline generation in such a way.
    \newline
    \item Section~\ref{sec:eval} presents a comprehensive evaluation using a large collection of 121 datasets from major AutoML benchmarks and Kaggle. Our experimental results show that {\sysname} outperforms all existing AutoML systems and achieves state-of-the-art results on the majority of these datasets. {\sysname} significantly improves the performance of both FLAML and Auto-Sklearn in classification and regression tasks. We also outperformed AL in 75 out of 77 datasets and VolcanoML in 75  out of 121 datasets, including 44 datasets used only by VolcanoML~\cite{VolcanoML}.  On average, {\sysname} achieves scores that are statistically better than the means of all other systems. 
\end{itemize}


%This approach does not need to apply cleaning or transformation methods to handle different variances among datasets. Moreover, we do not need to deal with complex analysis, such as dynamic code analysis. Thus, our approach proved to be scalable, as discussed in Sections~\ref{sec:mining}.
%!TEX root = main.tex
\section{Hierarchical Sentence Factorization and Reordering}
\label{sec:sentence}


\begin{figure*}[tb]
\centering
\includegraphics[width=\textwidth]{figure/CaseStudy}
\vspace{-3mm}
\caption{An example of the sentence factorization process. Here we show: A. The original sentence pair; B. The procedures of creating sentence factorization trees; C. The predicate-argument form of original sentence pair; D. The alignment of semantic units with the reordered form.}
\label{fig:casestudy}
\vspace{-1mm}
\end{figure*}

In this section, we present our \textit{Hierarchical Sentence Factorization} techniques to transform a sentence into a hierarchical tree structure, which also naturally produces a reordering of the sentence at the root node. 
%Such multi-scaled sentence factorization and reordering 
This multi-scaled representation form proves to be effective at improving both unsupervised and supervised semantic matching, which will be discussed in Sec.~\ref{sec:owmd} and Sec.~\ref{sec:multi-layer}, respectively. 

We first describe our desired factorization tree structure before presenting the steps to obtain it. Given a natural language sentence $S$, our objective is to transform it into a semantic factorization tree denoted by $T^f_S$. Each node in $T^f_S$ is called a \textit{semantic unit}, which contains one or a few tokens (tokenized words) from the original sentence $S$, as illustrated in Fig.~\ref{fig:casestudy} $(a4)$, $(b4)$.
The tokens in every semantic unit in $T^f_S$ is re-organized into a ``predicate-argument'' form. For example, a semantic unit for ``Tom catches Jerry'' in the ``predicate-argument'' form will be ``catch Tom Jerry''.

Our proposed factorization tree recursively factorizes a sentence into a hierarchy of semantic units at different granularities to represent the semantic structure of that sentence.
%In different depths of a factorization tree, a sentence will be factorized into semantic units in different granularities.
The root node in a factorization tree contains the entire sentence reordered in the predicate-argument form, thus providing a ``normalized'' representation for sentences expressed in different ways (e.g., passive vs. active tenses). Moreover, each semantic unit at depth $d$ will be further split into several child nodes at depth $d + 1$, which are smaller semantic sub-units. Each sub-unit also follows the predicate-argument form.

For example, in Fig.~\ref{fig:casestudy}, we convert sentence $A$ into a hierarchical factorization tree $(a4)$ using a series of operations. The root node of the tree contains the semantic unit ``chase Tom Jerry little yard big'', which is the reordered representation of the original sentence ``The little Jerry is being chased by Tom in the big yard'' in a semantically normalized form. 
Moreover, the semantic unit at depth $0$ is factorized into four sub-units at depth $1$: ``chase'', ``Tom'', ``Jerry little'' and ``yard big'', each in the ``predicate-argument'' form. And at depth $2$, the semantic sub-unit ``Jerry little'' is further factorized into two sub-units ``Jerry'' and ``little''. Finally, a semantic unit that contains only one token (e.g., ``chase'' and ``Tom'' at depth $1$) can not be further decomposed. Therefore, it only has one child node at the next depth through self-duplication.

We can observe that each depth of the tree contains all the tokens (except meaningless ones) in the original sentence, but re-organizes these tokens into semantic units of different granularities. %In Sec.~\ref{sec:owmd} and Sec.~\ref{sec:multi-layer}, we will show that such a specially designed sentence factorization tree together with the naturally generated semantic reordering can be used to effectively improve both unsupervised and supervised semantic matching methods.


% Our approach is mainly based on the Abstract Meaning Representation (AMR) of a sentence \cite{banarescu2013abstract, flanigan2014discriminative}.

\subsection{Hierarchical Sentence Factorization}
\label{subsec:sf}

We now describe our detailed procedure to transform a natural language sentence to the desired factorization tree mentioned above.
Our Hierarchical Sentence Factorization algorithm mainly consists of five steps: 1) AMR parsing and alignment, 2) AMR purification, 3) index mapping, 4) node completion, and 5) node traversal. The latter four steps are illustrated in the example in Fig.~\ref{fig:casestudy} from left to right.%We will explain each step in detail in the following.


\begin{figure}[tb]
\centering
\includegraphics[width=2.7in]{figure/amr}
\vspace{0mm}
\caption{An example of a sentence and its Abstract Meaning Representation (AMR), as well as the alignment between the words in the sentence and the nodes in AMR.}
\label{fig:amr}
\vspace{-3mm}
\end{figure}

\textbf{AMR parsing and alignment.}
Given an input sentence, the first step of our hierarchical sentence factorization algorithm is to acquire its Abstract Meaning Representation (AMR), as well as perform AMR-Sentence alignment to align the concepts in AMR with the tokens in the original sentence.
 
Semantic parsing \cite{baker1998berkeley,kingsbury2002treebank,berant2014semantic,banarescu2013abstract, damonte2016incremental} can be performed to generate the formal semantic representation of a sentence.
Abstract Meaning Representation (AMR) \cite{banarescu2013abstract} is a semantic parsing language that represents a sentence by a directed acyclic graph (DAG).
Each AMR graph can be converted into an AMR tree by duplicating the nodes that have more than one parent.

Fig.~\ref{fig:amr} shows the AMR of the sentence ``I observed that the army moved quickly.''
In an AMR graph, leaves are labeled with concepts, which represent either English words (e.g., ``army''), PropBank framesets (e.g., ``observe-01'') \cite{kingsbury2002treebank}, or special keywords (e.g., dates, quantities, world regions, etc.). For example, ``(a / army)'' refers to an instance of the concept army, where ``a'' is the variable name of army (each entity in AMR has a variable name).  ``ARG0'', ``ARG1'', ``:manner'' are different kinds of relations defined in AMR. Relations are used to link entities. For example, ``:manner'' links ``m / move-01'' and ``q / quick'', which means ``move in a quick manner''. Similarly, ``:ARG0'' links ``m / move-01'' and ``a / army'', which means that ``army'' is the first argument of ``move''.


%Given an input sentence, a semantic parser extracts a semantic representation of the sentence \cite{damonte2016incremental}.
% AMR aims to \red{normalize (??canonicalize)} different ways of stating the same thing and assigns the same AMR to sentences with the same meaning.


Each leaf in AMR is a concept rather than the original token in a sentence. 
The alignment between a sentence and its AMR graph is not given in the AMR annotation. Therefore, AMR alignment \cite{pourdamghani2014aligning} needs to be performed to link the leaf nodes in the AMR to tokens in the original sentence.
Fig.~\ref{fig:amr} shows the alignment between sentence tokens and AMR concepts by the alignment indexes.
The alignment index $0$ is for the root node, 0.0 for the first child of the root node, 0.1 for the second child of the root node, and so forth. For example, in Fig.~\ref{fig:amr}, the word ``army'' in sentence is linked with index ``0.1.0'', which represents the concept node ``a / army'' in its AMR.
% The numbering system will skip variable \red{re-entrancies} of duplicated nodes.
% AMR strives to achieve a more logical representation for sentences. It attempts to assign the same AMR to sentences that have the same meaning. For example, ``I observed the quick movement of the army'' and ``I observed the army moved quickly'' will have the same AMR representation that is shown in Fig.~\ref{fig:amr}.
We refer interested readers to \cite{banarescu2013abstract,banarescu2012abstract} for more detailed description about AMR. 

% We utilize the AMR representation of sentences to get a unified, hierarchical and reordered sentence representation.
%Alg.~.... shows the steps of sentence factorization.
Various parsers have been proposed for AMR parsing and alignment \cite{flanigan2014discriminative,wang2015boosting}. We choose the JAMR parser \cite{flanigan2014discriminative} in our algorithm implementation. 

\begin{figure}[tb]
\centering
\includegraphics[width=3.0in]{figure/amr2}
\vspace{0mm}
\caption{An example to show the operation of AMR purification.}
\label{fig:amr2}
\vspace{-3mm}
\end{figure}


\textbf{AMR purification.} Unfortunately, AMR itself cannot be used to form the desired factorization tree. 
First, it is likely that multiple concepts in AMR may link to the same token in the sentence.
For example, Fig.~\ref{fig:amr2} shows AMR and its alignment for the sentence ``Three Asian kids are dancing.''.
The token ``Asian'' is linked to four concepts in the AMR graph: `` continent (0.0.0)'', ``name (0.0.0.0)'', ``Asia (0.0.0.0.0)'' and ``wiki Asia (0.0.0.1)''.
This is because AMR will match a named entity with predefined concepts which it belongs to, such as ``c / continent'' for ``Asia'', and form a compound representation of the entity. For example, in Fig.\ref{fig:amr2}, the token ``Asian'' is represented as a continent whose name is Asia, and its Wikipedia entity name is also Asia.

In this case, we select the link index with the smallest tree depth as the token's position in the tree.
Suppose $\mathcal{P}_w = \{p_1, p_2, \cdots, p_{|\mathcal{P}|}\}$ denotes the set of alignment indexes of token $w$.
We can get the desired alignment index of $w$ by calculating the longest common prefix of all the index strings in  $\mathcal{P}_w$.
After getting the alignment index for each token, we then replace the concepts in AMR with the tokens in sentence by the alignment indexes, and remove relation names (such as ``:ARG0'') in AMR, resulting into a compact tree representation of the original sentence, as shown in the right part of Fig.~\ref{fig:amr2}.


\textbf{Index mapping.}
A purified AMR tree for a sentence obtained in the previous step is still not in our desired form.
To transform it into a hierarchical sentence factorization tree, we perform index mapping and calculate a new position (or index) for each token in the desired factorization tree given its position (or index) in the purified AMR tree.
Fig.~\ref{fig:casestudy} illustrates the process of index mapping. After this step, for example, the purified AMR trees in Fig.~\ref{fig:casestudy} $(a1)$ and $(b1)$ will be transformed into $(a2)$ and $(b2)$.

Specifically, let $T^p_S$ denote a purified AMR tree of sentence $S$, and $T^f_S$ our desired sentence factorization tree of $S$.
Let $I^p_N = \overline{i_0.i_1.i_2.\cdots.i_d}$ denote the index of node $N$ in $T^p_S$, where $d$ is the depth of $N$ in $T^p_S$ (where depth 0 represents the root of a tree).
Then, the index $I^f_N$ of node $N$ in our desired factorization tree $T^f_S$ will be calculated as follows:
\begin{equation}
\label{eqn:transform-index}
I^f_N := \begin{cases}
 		     \overline{0.0} & \quad \text{if } d=0, \\ 
 			 \overline{i_0.(i_1+1).(i_2+1).\cdots.(i_d + 1)} & \quad \text{otherwise}. 
		 \end{cases}
\end{equation}
After index mapping, we add an empty root node with index $0$ in the new factorization tree, and link all nodes at depth $1$ to it as its child nodes. Note that the $i_0$ in every node index will always be 0.


\textbf{Node completion.}
We then perform node completion to make sure each branch of the factorization tree have the same maximum depth and to fill in the missing nodes caused by index mapping, illustrated by Fig.~\ref{fig:casestudy} $(a3)$ and $(b3)$.

First, given a pre-defined maximum depth $D$, for each leaf node $N^l$ with depth $d < D$ in the current $T^f_S$ after index mapping, we duplicate it for $D - d$ times and append all of them sequentially to $N^l$, as shown in Fig.~\ref{fig:casestudy} $(a3)$, $(b3)$, such that the depths of the ending nodes will always be $D$. For example, in Fig.~\ref{fig:casestudy} with $D = 2$, the node ``chase (0.0)'' and ``Tom (0.1)'' will be extended to reach depth $2$ via self-duplication.

Second, after index mapping, the children of all the non-leaf nodes, except the root node, will be indexed starting from 1 rather than 0. For example, in Fig.~\ref{fig:casestudy} $(a2)$, the first child node of ``Jerry (0.2)'' is ``little (0.2.1)''. %However, the first child node according to our indexing rule should have the index of ``0.2.0'', which is missed at present. 
In this case, we duplicate ``Jerry (0.2)'' itself to ``Jerry (0.2.0)'' to fill in the missing first child of ``Jerry (0.2)''. Similar filling operations are done for other non-leaf nodes after index mapping as well.

\textbf{Node traversal to complete semantic units}.
Finally, we complete each semantic unit in the formed factorization tree via node traversal, as shown in Fig.~\ref{fig:casestudy} $(a4)$, $(b4)$.
For each non-leaf node $N$, we traverse its sub-tree by Depth First Search (DFS). The original semantic unit in $N$ will then be replaced by the concatenation of the semantic units of all the nodes in the sub-tree rooted at $N$, following the order of traversal.
%Notice that the traverse process should skip all the nodes completed in the previous step of node completion.

For example, for sentence $A$ in Fig.~\ref{fig:casestudy}, after node traversal, the root node of the factorization tree becomes ``chase Tom Jerry little yard big'' with index ``0''. We can see that the original sentence has been reordered into a predicate-argument structure. A similar structure is generated for the other nodes at different depths. 
Until now, each depth of the factorization tree $T^f_S$ can express the full sentence $S$  in terms of  semantic units at different granularity.




%!TEX root = main.tex
\section{Ordered Word Mover's Distance}
\label{sec:owmd}

\begin{figure}[tb]
\centering
\includegraphics[width=2.5in]{figure/SentenceMatch}
\vspace{0mm}
\caption{Compare the sentence matching results given by Word Mover's Distance and Ordered Word Mover's Distance.}
\label{fig:match}
\vspace{-3mm}
\end{figure}

The proposed hierarchical sentence factorization technique naturally reorders an input sentence into a unified format at the root node. In this section, we introduce the \textit{Ordered Word Mover's Distance} metric which measures the semantic distance between two input sentences based on the unified representation of reordered sentences.

Assume $X\in \mathbb{R}^{d\times n}$ is a \textit{word2vec} embedding matrix for a vocabulary of $n$ words, and the $i$-th column $\mathbf{x}_i \in \mathbb{R}^d$ represents the $d$-dimensional embedding vector of $i$-th word in vocabulary.
Denote a sentence $S = \overline{a_1a_2\cdots a_K}$ where $a_i$ represents the $i$-th word (or the word embedding vector).
The Word Mover's Distance considers a sentence $S$ as its normalized bag-of-words (nBOW) vectors where the weights of the words in $S$ is $\mathbf \alpha = \{\alpha_1, \alpha_2, \cdots, \alpha_K\}$. Specifically, if word $a_i$ appears $c_i$ times in $S$, then $\alpha_i = \frac{c_i}{\sum_{j=1}^K c_j}$.

The Word Mover's Distance metric combines the normalized bag-of-words representation of sentences with Wasserstein distance (also known as Earth Mover's Distance \cite{rubner2000earth}) to measure the semantic distance between two sentences. 
Given a pair of sentences $S_1 = \overline{a_1a_2\cdots a_M}$ and $S_2 = \overline{b_1b_2\cdots b_N}$, where %$a_i \in \mathbb{R}^d$ is the embedding vector of the $i$-th word in $S_1$ and 
$b_j \in \mathbb{R}^d$ is the embedding vector of the $j$-th word in $S_2$. Let $\mathbf \alpha = \{\alpha_1, \cdots, \alpha_M\}$ and $\mathbf \beta = \{\beta_1, \cdots, \beta_N\}$ represents the normalized bag-of-words vectors of $S_1$ and $S_2$. We can calculate a distance matrix $D \in \mathbb{R}^{M\times N}$ where each element
$D_{ij} = \|a_i - b_j\|_2$ measures the distance between word $a_i$ and $b_j$ (we use the same notation to denote the word itself or its word vector representation).
Let $T \in \mathbb{R}^{M\times N}$ be a non-negative sparse transport matrix where $T_{ij}$ denotes the portion of word $a_i \in S_1$ that transports to word $b_j \in S_2$.
The Word Mover's Distance between sentences $S_1$ and $S_2$ is given by $\sum_{i,j} T_{ij} D_{ij}$. The transport matrix $T$ is computed solving the following constrained optimization problem:
\begin{equation}
\label{eq:wmd}
\begin{split}
	\underset{T \in \mathbb{R}_{+}^{M\times N}}{\mbox{minimize}}\quad 		& \sum_{i,j} T_{ij} D_{ij} \\
	\mbox{subject to}\quad & \sum\limits_{i = 1}^{M}  T_{ij} = \beta_j \quad 1\leq j \leq N,\\
			   & \sum\limits_{j = 1}^{N}  T_{ij} = \alpha_i \quad 1\leq i \leq M.
\end{split}
\end{equation}
Where the minimum ``word travel cost'' between two bags of words for a pair of sentences is calculated to measure the their semantic distance.

However, the Word Mover's Distance fails to consider a few aspects of natural language. First, it omits the sequential structure. For example, in Fig.~\ref{fig:match}, the pair of sentences ``Morty is laughing at Rick'' and ``Rick is laughing at Morty'' only differ in the order of words. The Word Mover's Distance metric will then find an exact match between the two sentences and estimate the semantic distance as zero, which is obviously false. Second, the normalized bag-of-words representation of a sentence can not distinguish duplicated words shown in multiple positions of a sentence.

To overcome the above challenges, we propose a new kind of semantic distance metric named Ordered Word Mover's Distance (OWMD). The Ordered Word Mover's Distance combines our sentence factorization technique with Order-preserving Wasserstein Distance proposed in \cite{su2017order}. It casts the calculation of semantic distance between texts as an optimal transport problem while preserving the sequential structure of words in sentences. The Ordered Word Mover's Distance differs from the Word Mover's Distance in multiple aspects.

First, rather than using normalized bag-of-words vector to represent a sentence, we decompose and re-organize a sentence using the sentence factorization algorithm described in Sec.~\ref{sec:sentence}. Given a sentence $S$, we represent it by the reordered word sequence $S'$ in the root node of its sentence factorization tree. Such representation normalizes a sentence into ``predicate-argument'' structure to better handle syntactic variations.
%process sentences with different syntactic structures that express the same meaning. 
For example, after performing sentence factorization, sentences ``Tom is chasing Jerry'' and ``Jerry is being chased by Tom'' will both be normalized as ``chase Tom Jerry''.

Second, we calculate a new transport matrix $T$ by solving the following optimization problem
\begin{equation}
\label{eq:owmd}
\begin{split}
	\underset{T \in \mathbb{R}_{+}^{M\times N}}{\mbox{minimize}}\quad 		& \sum_{i,j} T_{ij} D_{ij} - \lambda_1 I(T) + \lambda_2 KL(T||P)\\
	\mbox{subject to}\quad & \sum\limits_{i = 1}^{M}  T_{ij} = \beta_j' \quad 1\leq j \leq N',\\
			   & \sum\limits_{j = 1}^{N}  T_{ij} = \alpha_i' \quad 1\leq i \leq M',
\end{split}
\end{equation}
where $\lambda_1 > 0$ and $\lambda_2 > 0$ are two hyper parameters.
$M'$ and $N'$ denotes the number of words in $S_1'$ and $S_2'$.
$\alpha_i'$ denotes the weight of the $i$-th word in normalized sentence $S_1'$ and $\beta_j'$ denotes the weight of the $j$-th word in normalized sentence $S_2'$. Usually we can set $\mathbf{\alpha'} = (\frac{1}{M'}, \cdots, \frac{1}{M'})$ and $\mathbf{\beta'} = (\frac{1}{N'}, \cdots, \frac{1}{N'})$ without any prior knowledge of word differences.

The first penalty term $I(T)$ is the inverse difference moment \cite{albregtsen2008statistical} of the transport matrix $T$ that measures local homogeneity of $T$. It is defined as:
\begin{equation}
\label{eq:IT}
\begin{split}
	I(T) = \sum\limits_{i=1}^{M'} \sum\limits_{j=1}^{N'} \frac{T_{ij}}{(\frac{i}{M'} - \frac{j}{N'})^2 + 1}.
\end{split}
\end{equation}
$I(T)$ will have a relatively large value if the large values of $T$ mainly appear near its diagonal.

Another penalty term $KL(T||P)$ denotes the KL-divergence between $T$ and $P$. 
$P$ is a two-dimensional distribution used as the prior distribution for values in $T$. It is defined as
\begin{equation}
\label{eq:P}
\begin{split}
	P_{ij} = \frac{1}{\sigma \sqrt{2\pi}}e^{- \frac{l^2(i,j)}{2\sigma^2}}
\end{split}
\end{equation}
where $l(i, j)$ is the distance from position $(i, j)$ to the diagonal line, which is calculated as
\begin{equation}
\label{eq:l}
\begin{split}
	l(i, j) = \frac{|i/M' - j/N'|}{\sqrt{1/M'^2 + 1/N'^2}}.
\end{split}
\end{equation}
As we can see, the farther a word in one sentence is from the other word in another sentence in terms of word orders, the less likely it will be transported to that word. Therefore, by introducing the two penalty terms $I(T)$ and $KL(T||P)$ into problem~(\ref{eq:owmd}), 
we encourage words at similar positions in two sentences to be matched.
%we encourage the words in two sentences with similar positions be matched. 
Words at distant positions are less likely to be matched by $T$.

The problem (\ref{eq:owmd}) has a unique optimal solution $T^{\lambda_1, \lambda_2}$ since both the objective and the feasible set are convex. It has been proved in \cite{su2017order} that the optimal $T^{\lambda_1, \lambda_2}$ has the same form with $diag(\mathbf{k}_1) \cdot \mathbf{K} \cdot diag(\mathbf{k}_2)$, where $diag(\mathbf{k}_1) \in \mathbb{R}^{M'}$ and $diag(\mathbf{k}_2) \in \mathbb{R}^{N'}$ are two diagonal matrices with strictly positive diagonal elements. $\mathbf{K} \in \mathbb{R}^{M'\times N'}$ is a matrix defined as
\begin{equation}
\label{eq:K}
\begin{split}
	K_{ij} = P_{ij} e^{\frac{1}{\lambda_2}(S_{ij}^{\lambda_1} - D_{ij})},
\end{split}
\end{equation}
where
\begin{equation}
\label{eq:S}
\begin{split}
	S_{ij} = \frac{\lambda_1}{(\frac{i}{M'} - \frac{j}{N'})^2 + 1}.
\end{split}
\end{equation}
The two matrices $\mathbf{k}_1$ and $\mathbf{k}_2$ can be efficiently obtained by the Sinkhorn-Knopp iterative matrix scaling algorithm \cite{knight2008sinkhorn}:
\begin{equation}
\label{eq:k1}
\begin{split}
	\mathbf{k}_1 &\leftarrow \mathbf{\alpha'} ./ K \mathbf{k}_2, \\
	\mathbf{k}_2 &\leftarrow \mathbf{\beta'} ./ K^T \mathbf{k}_1.
\end{split}
\end{equation}
where $./$ is the element-wise division operation.
%Until now, we have introduced the sentence representation and the way we calculate the transport matrix $T$ for the Ordered Word Mover's Distance metric.
Compared with Word Mover's Distance, the Ordered Word Mover's Distance 
considers the positions of words in a sentence,
%takes the positions of words in sentences into account, 
and is able to distinguish duplicated words at different locations. For example, in Fig.~\ref{fig:match}, while the WMD finds an exact match and get a semantic distance of zero for the sentence pair ``Morty is laughing at Rick'' and ``Rick is laughing at Morty'', the OWMD metric is able to find a better match relying on the penalty terms, and gives a semantic distance greater than zero.

The computational complexity of OWMD is also effectively reduced compared to WMD. With the additional constraints, the time complexity is $O(dM'N')$ where $d$ is the dimension of word vectors \cite{su2017order}, while it is $O(dp^3\log p)$ for WMD, where $p$ denotes the number of uniques words in sentences or documents \cite{kusner2015word}.


%!TEX root = main.tex
\section{Multi-scale Sentence Matching}
\label{sec:multi-layer}


\begin{figure*}[!htb]
\centering
\includegraphics[width=5.5in]{figure/network}
\vspace{0mm}
\caption{Extend the Siamese network architecture for sentence matching by feeding into the multi-scale representations of sentence pairs.}
\label{fig:network}
\vspace{-2mm}
\end{figure*}


Our sentence factorization algorithm parses a sentence $S$ into a hierarchical factorization tree $T^f_S$, where each depth of $T^f_S$ contains the semantic units of the sentence at a different granularity.
% We proposed a new unsupervised semantic distance metric using the reordered sentence representation in depth $0$ (the root node) of $T^f_S$.
In this section, we exploit this multi-scaled representation of $S$ present in $T^f_S$ to propose a multi-scaled Siamese network architecture that can extend any existing CNN or RNN-based Siamese architectures to leverage the hierarchical representation of sentence semantics.


Fig.~\ref{fig:network} (a) shows the network architecture of the popular Siamese ``matching-aggregation'' framework \cite{wang2016compare,mueller2016siamese,severyn2015learning,neculoiu2016learning,baudivs2016sentence} for sentence matching tasks. The matching process is usually performed as follows: First, the sequence of word embeddings in two sentences will be encoded by a context representation layer, which usually contains one or multiple layers of LSTM, bi-directional LSTM (BiLSTM), or CNN with max pooling layers.
The goal is to capture the contextual information of each sentence into a context vector. 
In a Siamese network, every sentence is encoded by the same context representation layer.
%The left part and the right part for a pair of sentences share the same context representation layer.
Second, the context vectors of two sentences will be concatenated in the aggregation layer. They may be further transformed by more layers of neural network
%one or a few neural network layers (such as LSTM) 
to get a fixed length matching vector.
Finally, a prediction layer will take in the matching vector and outputs a similarity score for the two sentences or the probability distribution over different sentence-pair relationships.


Compared with the typical Siamese network shown in Fig.~\ref{fig:network} (a), our proposed architecture shown in Fig.~\ref{fig:network} (b) differs in two aspects.
First, our network contains three Siamese sub-modules that are similar to (a). They correspond to the factorized representations from depth $0$ (the root layer) to depth $2$. We only select the semantic units from the top $3$ depths of the factorization tree as our input, because usually most semantic units at depth $2$ are already single words and can not be further factorized. Second, for each Siamese sub-module in our network architecture, the input is not the embedding vectors of words from the original sentences. Instead, we use semantic units at different depths of sentence factorization tree 
for matching.
%as the basic sentence matching unit.
We sum up the embedding vectors of the words contained in a semantic unit to represent that unit. Assuming each semantic unit at depth $d$ can be factorized into $k$ semantic sub-units at depth $d + 1$. If a semantic unit has less than $k$ sub-units, we add empty units as its child node to make each non-leaf node in a factorization tree has exactly $k$ child nodes. The empty units are embedded with a vector of zeros. After this procedure, the number of semantic units at depth $d$ of a sentence factorization tree is $k^d$.

Taking Fig.~\ref{fig:casestudy} as an example. We set $k = 4$ in Fig.~\ref{fig:casestudy}. For sentence A ``The little Jerry is being chased by Tom in the big yard'', the input at depth $0$ is the sum of word embedding $\{$chase, Tom, Jerry, little, yard, big$\}$. The input at depth $1$ are the embedding vectors of four semantic units: 
$\{$chase, Tome, Jerry little, yard big$\}$. Finally, at depth $2$, the semantic units are $\{$chase, -, -, -, Tom, -, -, -, Jerry, little, -, -, yard, big, -, -$\}$, where ``$-$'' denotes an empty unit.

As we can see, based on this factorized sentence representation, our network architecture explicitly matches a pair of sentences at several semantic granularities. 
%In addition, we align the semantic units in two sentences by mapping the position of semantic units in the tree to its input index in the input layer of neural network. 
In addition, we align the semantic units in two sentences by mapping their positions in the tree to the corresponding indices in the input layer of the neural network.
For example, as shown in Fig.~\ref{fig:casestudy}, the semantic units at depth $2$ are aligned according to their unit indices: ``chase'' matches with ``catch'', ``Tom'' matches with ``cat blue'', ``Jerry little'' matches with ``mouse brown'', and ``yard big'' matches with ``forecourt''.




\section{Evaluation}
\label{sec:evaluation}
\begin{table*}[!t]
\begin{center}
%\small
\caption {Benchmarks and applications for the study of the application-level resilience}
\vspace{-5pt}
\label{tab:benchmark}
\tiny
\begin{tabular}{|p{1.7cm}|p{7.5cm}|p{4cm}|p{2.5cm}|}
\hline
\textbf{Name} 	& \textbf{Benchmark description} 		& \textbf{Execution phase for evaluation}  			& \textbf{Target data objects}             \\ \hline \hline
CG (NPB)             & Conjugate Gradient, irregular memory access (input class S)   & The routine conj\_grad in the main computation loop  & The arrays $r$ and $colidx$     \\\hline
MG (NPB)    	       & Multi-Grid on a sequence of meshes (input class S)             & The routine mg3P in the main computation loop & The arrays $u$ and $r$ 	\\ \hline
FT (NPB)             & Discrete 3D fast Fourier Transform (input class S)            & The routine fftXYZ in the main computation loop  & The arrays $plane$ and $exp1$    \\ \hline
BT (NPB)             & Block Tri-diagonal solver (input class S)         		& The routine x\_solve in the main computation loop & The arrays $grid\_points$ and $u$	\\ \hline
SP (NPB)             & Scalar Penta-diagonal solver (input class S)         		& The routine x\_solve in the main computation loop & The arrays $rhoi$ and $grid\_points$  \\ \hline
LU (NPB)            & Lower-Upper Gauss-Seidel solver (input class S)        	& The routine ssor 	& The arrays $u$ and $rsd$ \\ \hline \hline
LULESH~\cite{IPDPS13:LULESH} & Unstructured Lagrangian explicit shock hydrodynamics (input 5x5x5) & 
The routine CalcMonotonicQRegionForElems 
& The arrays $m\_elemBC$ and $m\_delv\_zeta$ \\ \hline
AMG2013~\cite{anm02:amg} & An algebraic multigrid solver for linear systems arising from problems on unstructured grids (we use  GMRES(10) with AMG preconditioner). We use a compact version from LLNL with input matrix $aniso$. & The routine hypre\_GMRESSolve & The arrays $ipiv$ and $A$   \\ \hline
%$hierarchy.levels[0].R.V$ \\ \hline
\end{tabular}
\end{center}
\vspace{-5pt}
\end{table*}

%We evaluate the effectiveness of ARAT, and 
%We use ARAT to study the application-level resilience.
%The goal is to demonstrate 
%that aDVF can be a very useful metric to quantify the resilience of data objects
%at the application level. 
We study 12 data objects from six benchmarks of the NAS parallel benchmark (NPB) suite (we use SNU\_NPB-1.0.3) and 4 data objects from two scientific applications. 
%which is a c version of NPB 3.3, but ARAT can work for Fortran.
Those data objects are chosen to be representative: they have various data access patterns and participate in various execution phases.  
%For the benchmarks, we use CLASS S as the input problems and use the default compiler options of NPB.
For those benchmarks and applications, we use their default compiler options, and use gcc 4.7.3 and LLVM 3.4.2 for trace generation.
To count the algorithm-level fault masking, we use the default convergence thresholds (or the fault tolerance levels) for those benchmarks.
Table~\ref{tab:benchmark} gives 
%for->on by anzheng
detailed information on the benchmarks and applications.
The maximum fault propagation path for aDVF analysis is set to 10 by default.
%the value shadowing threshold is set as 0.01 (except for BT, we use $1 \times 10^{-6}$).
%These value shadowing thresholds are chosen such that any error corruption
%that results in the operand's value variance less than 1\% (for the threshold 0.01) or 0.0001\% (for the threshold $1 \times 10^{-6}$) during the 
%trace analysis does not impact the outcome correctness of six benchmarks.
%LU: check the newton-iteration residuals against the tolerance levels
%SP: check the newton-iteration residuals against the tolerance levels
%BT: check the newton-iteration residuals against the tolerance levels

\subsection{Resilience Modeling Results}
%We use ARAT to calculate aDVF values of 16 data objects. 
Figure~\ref{fig:aDVF_3tiers_profiling}
shows the aDVF results and breaks them down into the three levels 
(i.e., the operation-level, fault propagation level, and algorithm-level).
Figure~\ref{fig:aDVF_3classes_profiling} shows the 
%for->of by anzheng
results for the analyses at the levels of the operation and fault propagation,
and further breaks down the results into 
the three classes (i.e., the value overwriting, logical and comparison operations,
and value shadowing). %based on the reasons of the fault masking.
We have multiple interesting findings from the results.

\begin{figure*}
	\centering
        \includegraphics[width=0.8\textwidth]{three_tiers_gray.pdf}
% * <azguolu@gmail.com> 2017-03-23T03:20:28.808Z:
%
% ^.
        \vspace{-5pt}
        \caption{The breakdown of aDVF results based on the three level analysis. The $x$ axis is the data object name.}
        \vspace{-8pt}
        \label{fig:aDVF_3tiers_profiling}
\end{figure*}


\begin{figure*}
	\centering
	\includegraphics[width=0.8\textwidth]{three_types_gray.pdf}
	\vspace{-5pt}
	\caption{The breakdown of aDVF results based on the three classes of fault masking. The $x$ axis is the data object name. \textit{zeta} and \textit{elemBC} in LULESH are \textit{m\_delv\_zeta} and \textit{m\_elemBC} respectively.} % Anzheng
	\vspace{-5pt}
	\label{fig:aDVF_3classes_profiling}
    %\vspace{-5pt}
\end{figure*}

(1) Fault masking is common across benchmarks and applications.
Several data objects (e.g., $r$ in CG, and $exp1$ and $plane$ in FT)
have aDVF values close to 1 in Figure~\ref{fig:aDVF_3tiers_profiling}, 
which indicates that most of operations working on these data objects
have fault masking.
However, a couple of data objects have much less intensive fault masking.
For example, the aDVF value of $colidx$ in CG is 0.28 (Figure~\ref{fig:aDVF_3tiers_profiling}). 
Further study reveals that $colidx$ is an array to store column indexes of sparse matrices, and there is few operation-level or fault propagation-level fault masking  (Figure~\ref{fig:aDVF_3classes_profiling}).
The corruption of it can easily cause segmentation fault caught by the
algorithm-level analysis. 
$grid\_points$ in SP and BT also have a relatively small aDVF value (0.14 and 0.38 for SP and BT respectively in Figure~\ref{fig:aDVF_3tiers_profiling}).
Further study reveals that $grid\_points$ defines input problems for SP and BT. 
A small corruption of $grid\_points$ 
%change->changes by anzheng
can easily cause major changes in computation
caught by the fault propagation analysis. 

The data object $u$ in BT also has a relatively small aDVF value (0.82 in Figure~\ref{fig:aDVF_3tiers_profiling}).
Further study reveals that $u$ is read-only in our target code region
for matrix factorization and Jacobian, neither of which is friendly
for fault masking.
Furthermore, the major fault masking for $u$ comes from value shadowing,
and value shadowing only happens in a couple of the least significant bits 
of the operands that reference $u$, which further reduces the value of aDVF.
%also reduces fault masking.

(2) The data type is strongly correlated with fault masking.
Figure~\ref{fig:aDVF_3tiers_profiling} reveals that the integer data objects ($colidx$ in CG, $grid\_points$ in BT and SP, $m\_elemBC$ in LULESH) appear to be 
more sensitive to faults than the floating point data objects 
($u$ and $r$ in MG, $exp1$ and $plane$ in FT, $u$ and $rsd$ in LU, $m\_delv\_zeta$ in LULESH, and $rhoi$ in SP).
In HPC applications, the integer data objects are commonly employed to
define input problems and bound computation boundaries (e.g., $colidx$ in CG and $grid\_points$ in BT), 
or track computation status (e.g., $m\_elemBC$ in LULESH). Their corruption 
%these integer data objects
is very detrimental to the application correctness. 

(3) Operation-level fault masking is very common.
For many data objects, the operation-level fault masking contributes 
more than 70\% of the aDVF values. For $r$ in CG, $exp1$ in FT, and $rhoi$ in SP,
the contribution of the operation-level fault masking is close to 99\% (Figure~\ref{fig:aDVF_3tiers_profiling}).

Furthermore, the value shadowing is a very common operation level fault masking,
especially for floating point data objects (e.g., $u$ and $r$ in BT, $m\_delv\_zeta$ in LULESH, and $rhoi$ in SP in Figure~\ref{fig:aDVF_3classes_profiling}).
This finding has a very important indication for studying the application resilience.
In particular, the values of a data object can be different across different input problems. If the values of the data object are different, 
then the number of fault masking events due to the value shadowing will be different. 
Hence, we deduce that the application resilience
can be correlated with the input problems,
because of the correlation between the value shadowing and input problems. 
We must consider the input problems when studying the application resilience.
This conclusion is consistent with a very recent work~\cite{sc16:guo}.

(4) The contribution of the algorithm-level fault masking to the application resilience can be nontrivial.
For example, the algorithm-level fault masking contributes 19\% of the aDVF value for $u$ in MG and 27\% for $plane$ in FT (Figure~\ref{fig:aDVF_3tiers_profiling}).
The large contribution of algorithm-level fault masking in MG is consistent with
the results of existing work~\cite{mg_ics12}. 
For FT (particularly 3D FFT), the large contribution of algorithm-level fault masking in $plane$ (Figure~\ref{fig:aDVF_3tiers_profiling})
comes from frequent transpose and 1D FFT computations that average out 
or overwrite the data corruption.
CG, as an iterative solver, is known to have the algorithm-level fault masking
because of the iterative nature~\cite{2-shantharam2011characterizing}.
Interestingly, the algorithm-level fault masking in CG contributes most to the resilience of $colidx$ which is a vulnerable integer data object (Figure~\ref{fig:aDVF_3tiers_profiling}).

%Our study reveals the algorithm-level fault masking of CG from
%two perspectives. First, $a$ in CG, which is an array for intermediate results,
%has few algorithm-level fault masking (0.008\%);
%Second, $x$ in CG, which is a result vector, has 5.4\% of the aDVF value coming from the algorithm-level fault masking.
%This result indicates that the effects of the algorithm-level fault masking
%are not uniform across data objects. 

(5) Fault masking at the fault propagation level is small.
For all data objects, the contribution of the fault masking at the level of fault propagation is less than 5\% (Figure~\ref{fig:aDVF_3tiers_profiling}).
For 6 data objects ($r$ and $colidx$ in CG, $grid\_points$ and $u$ in BT, and 
$grid\_points$ and $rhoi$ in SP),  there is no fault masking at the level of fault propagation.
In combination with the finding 4, we conclude that once the fault
is propagated, it is difficult to mask it because of the contamination of
more data objects after fault propagation, and only the algorithm semantics can tolerate  propagated faults well. 
%This finding is consistent with our sensitivity analysis. 

(6) Fault masking by logical and comparison operations is small,
%For all data objects, the fault masking contributions due to logical and comparison operations are very small, 
comparing with the contributions of value shadowing and overwriting (Figure~\ref{fig:aDVF_3classes_profiling}). 
Among all data objects, 
the logical and comparison operations in $grid\_points$ in BT contribute the most (25\% contribution in Figure~\ref{fig:aDVF_fine_profiling}), 
because of intensive ICmp operations (integer comparison). %logical OR and SHL (left shifting).


(7) The resilience varies across data objects. %within the same application.
This fact is especially pronounced in two data objects $colidx$ and $r$ in CG (Figure~\ref{fig:aDVF_3tiers_profiling}).
 $colidx$ has aDVF much smaller than $r$, which means $colidx$ is much less resilient than $r$ (see finding 1 for a detailed analysis on $colidx$). 
Furthermore, $colidx$ and $r$ have different algorithm-level
fault masking (see finding 4 for a detailed analysis).

\begin{comment}
\textbf{Finding 7: The resilience of the same data objects varies across different applications.}
This fact is especially pronounced in BT and SP.
BT and SP address the same numerical problem but with different algorithms.
BT and SP have the same data objects, $qs$ and $rhoi$, but
$qs$ manifests different resilience in BT and SP.
This result is interesting, because it indicates that by using
different algorithms, we have opportunities to
improve the resilience of data objects.
\end{comment}

To further investigate the reasons for fault masking, 
we break down the aDVF results at the granularity of LLVM instructions,
based on the analyses at the levels of operation and fault propagation.
The results are shown in Figure~\ref{fig:aDVF_fine_profiling}.
%Because of the space limitation, 
%we only show one data object per benchmark, but each selected data object has the most diverse fault masking events within the corresponding benchmark.
%Based on Figure~\ref{fig:aDVF_fine_profiling}, we have another interesting finding.

(8) Arithmetic operations make a lot of contributions to fault masking.
%For $r$ in CG, $r$ in MG, $exp1$ in FT, $u$ in BT, $qs$ in SP, and $u$ in LU,
%the arithmetic operations, FMul (100\%), Add (16\%), FMul (85\%), 
%FMul (94\%), FMul (28\%), and FAdd (50\%)
For $r$ in CG, $u$ in BT, $plane$ and $exp1$ in FT, $m\_elemBC$ in LULESH, 
arithmetic operations (addition, multiplication, and division) contribute to almost 100\% of the fault masking (Figure~\ref{fig:aDVF_fine_profiling}).  
%(at the operation level and the fault propagation level).
%For $qs$ in SP and $u$ in LU, the store operation also makes
%important contributions as the arithmetic operations because of value overwriting.

\begin{figure*}
	\centering
	\includegraphics[width=0.77\textheight, height=0.23\textheight]{pie_chart.pdf}
	\vspace{-10pt}
	\caption{Breakdown of the aDVF results based on the analyses at the levels of operation and fault propagation}
    \vspace{-10pt}
	\label{fig:aDVF_fine_profiling}
\end{figure*}


\subsection{Sensitivity Study}
\label{sec:eval_sen}
%\textbf{change the fault propagation threshold and study the sensitivity of analysis to the threshold}
ARAT uses 10 as the default fault propagation analysis threshold. 
The fault propagation analysis will not go beyond 10 operations. Instead,
we will use deterministic fault injection after 10 operations. 
In this section, we study the impact of this threshold on the modeling accuracy. We use a range of threshold values and examine how the aDVF value varies and whether
the identification of fault masking varies. 
Figure~\ref{fig:sensitivity_error_propagation} shows the results for 
%add , after BT by anzheng
multiple data objects in CG, BT, and SP.
We perform the sensitivity study for all 16 data objects.
%in six benchmarks and two applications.
Due to the page space limitation, we only show the results for three data objects,
but we summarize the sensitivity study results for all data objects in this section.
%but other data objects in all benchmarks have the same trend.

Our results reveal that the identification of fault masking by tracking fault propagation is not significantly 
affected by the fault propagation analysis threshold. Even if we use a rather large threshold (50), 
the variation of aDVF values is 4.48\% on average among all data objects,
and the variation at each of the three levels of analysis (the operation level, fault propagation level,  and algorithm level) is less than 5.2\% on average. 
In fact, using a threshold value of 5 is sufficiently accurate in most of the cases (14 out of 16 data objects).
This result is consistent with our finding 5 (i.e., fault masking at the fault propagation level is small). %in most benchmarks).
However, we do find a data object ($m\_elementBC$ in LULESH) %and $exp1$ in FT) 
showing relatively high-sensitive (up to 15\% variation) to the threshold. For this uncommon data object, using 50 as the fault propagation path is sufficient. 

%In other words, even though using a larger threshold value can identify more error masking by tracking error 
%propagation, the implicit error masking induced by the error propagation is very limited.

\begin{figure}
		\begin{center}
		\includegraphics[width=0.48\textwidth,height=0.11\textheight]{sensi_study_gray.pdf}
		\vspace{-15pt}
		\caption{Sensitivity study for fault propagation threshold}
		\label{fig:sensitivity_error_propagation}
		\end{center}
\vspace{-15pt}
\end{figure}


\begin{comment}
\subsection{Comparison with the Traditional Random Fault Injection}
%\textbf{compare with the traditional fault injection to verify accuracy}
To show the effectiveness of our resilience modeling, we compare traditional random fault injection
and our analytical modeling. Figure~\ref{fig:comparison_fi} and Table~\ref{tab:comparison} show the results.
The figure shows the success rate of all random fault injection. The ``success'' means the application
outcome is verified successfully by the benchmarks and the execution does not have any segfault. The success rate is used as a metric
to evaluate the application resilience.

We use a data-oriented approach to perform random fault injection.
In particular, given a data object, for each fault injection test we trigger a bit flip
in an operand of a random instruction, and this operand must be a reference to the
target data object. We develop a tool based on PIN~\cite{pintool} to implement the above fault injection functionality.
For each data object, we conduct five sets of random fault injection tests, 
and each set has 200 tests (in total 1000 tests per data object). 
We show the results for CG and FT in this section, but we find that
the conclusions we draw from CG and FT are also valid for the other four benchmarks.


%\begin{table*}
%\label{tab:success_rate}
%\begin{centering}
%\renewcommand\arraystretch{1.1}
%\begin{tabular}{|c|c|c|c|c|c|c|}
%\hline 
%Success Rate (Difference) & Test set 1 & Test set 2 & Test set 3 & Test set 4 & Test set 5 & Average\tabularnewline
%\hline 
%\hline 
%CG-a & 66.1\% (11.7\%) & 68.5\% (15.7\%) & 56.7\% (4.21\%) & 61.3\% (3.57\%) & 43.3\% (26.8\%) & 59.2\%\tabularnewline
%\hline 
%CG-x & 99.2\% (2.2\%) & 98.6\% (1.5\%) & 96.5\% (0.63\%) & 97.8\% (0.64\%) & 93.6\% (3.7\%) & 97.1\%\tabularnewline
%\hline 
%CG-colidx & 36.8\% (12.7\%) & 49.6\% (17.8\%) & 40.2\% (4.6\%) & 52.6\% (24.9\%) & 31.4\% (25.4\%) & 42.1\%\tabularnewline
%\hline 
%FT-exp1 & 52.7\% (1.4\%) & 22.6\% (56.5\%) & 78.5\% (51.0\%) & 60.7\% (16.7\%) & 45.4\% (12.7\%) & 51.9\%\tabularnewline
%\hline 
%FT-plane & 82.1\% (2.5\%) & 79.3\% (5.6\%) & 99.5\% (18.2\%) & 93.2\% (10.7\%) & 66.8\% (20.6\%) & 84.2\%\tabularnewline
%\hline 
%\end{tabular}
%\par\end{centering}
%\caption{XXXXX}
%\end{table*}


\begin{table*}
\begin{centering}
\caption{\small The results for random fault injection. The numbers in parentheses for each set of tests (200 tests per set) are the success rate difference from the average success rate of 1000 fault injection tests.}
\label{tab:comparison}
\renewcommand\arraystretch{1.1}
\begin{tabular}{|c|p{2.2cm}|p{2.2cm}|p{2.2cm}|p{2.2cm}|p{2.2cm}|p{1.8cm}|}
\hline 
       %& Test set 1 & Test set 2 & Test set 3 & Test set 4 & Test set 5 & Average\tabularnewline
       & \hspace{13pt} Test set 1 \hspace{1pt}/  & \hspace{13pt} Test set 2 \hspace{1pt}/ & \hspace{13pt} Test set 3 \hspace{1pt}/ & \hspace{13pt} Test set 4 \hspace{1pt}/ & \hspace{13pt} Test set 5 \hspace{1pt}/ & Ave. of all test / \\
       & success rate (diff.) & success rate (diff.) & success rate (diff.) & success rate (diff.) & success rate (diff.) & \hspace{5pt} success rate \\
\hline 
\hline 
CG-a & 66.1\% (6.9\%) & 68.5\% (9.3\%) & 56.7\% (-2.5\%) & 61.3\% (2.1\%) & 43.3\% (-15.9\%) & 59.2\%\tabularnewline
\hline 
CG-x & 99.2\% (2.1\%) & 98.6\% (1.5\%) & 96.5\% (-0.6\%) & 97.8\% (0.7\%) & 93.6\% (-3.5\%) & 97.1\%\tabularnewline
\hline 
CG-colidx & 36.8\% (-5.3\%) & 49.6\% (7.5\%) & 40.2\% (-2.0\%) & 52.6\% (10.5\%) & 31.4\% (-10.7\%) & 42.1\%\tabularnewline
\hline 
FT-exp1 & 52.7\% (0.8\%) & 22.6\% (-29.3\%) & 78.5\% (26.6\%) & 60.7\% (8.8\%) & 45.4\% (-6.5\%) & 51.9\%\tabularnewline
\hline 
FT-plane & 82.1\% (-2.1\%) & 79.3\% (-4.9\%) & 99.5\% (15.3\%) & 93.2\% (9.0\%) & 66.8\% (-17.4\%) & 84.2\%\tabularnewline
\hline 
\end{tabular}
\par\end{centering}
\vspace{-0.4cm}
\end{table*}

\begin{figure}
	\begin{center}
		\includegraphics[width=0.48\textwidth,keepaspectratio]{verifi-study.png}
		\caption{The traditional random fault injection vs. ARAT}
		\label{fig:comparison_fi}
	\end{center}
\vspace{-0.7cm}
\end{figure}


We first notice from Table~\ref{tab:comparison} that 
%across 5 sets of random fault injection tests, there are big variances (up to 55.9\% in $exp1$ of FT) in terms of the success rate. 
the results of 5 test sets can be quite different from each other and from 1000 random fault inject tests (up to 29.3\%).
1000 fault injection tests provide better statistical significance than 200 fault injection tests.
We expect 1000 fault injection tests potentially provide higher accuracy to quantify the application resilience.
The above result difference is clearly an indication to the randomness of fault injection, and there
is no guarantee on the random fault injection accuracy.

%In Figure~\ref{fig:comparison_fi}, 
We compare the success rate of 1000 fault inject tests with the aDVF value (Fig.~\ref{fig:comparison_fi}). 
We find that the order of the success rate of the three data objects in CG (i.e., $colidx < a < x$) and the two data objects in FT 
(i.e., $exp1 < plane$) is the same as the order of the aDVF values of these data objects. 
%In fact, 1000 fault injection tests
%account for \textcolor{blue}{\textbf{xxx\%}} of total memory references to the data object,
%and provide better resilience quantification than 200 fault injection tests.
The same order (or the same resilience trend)
%between our approach and the random fault injection based on a large number of tests 
is a demonstration of the effectiveness of our approach.
Note that the values of the aDVF and success rate %for a data object
cannot be exactly the same (even if we have sufficiently large numbers of random fault injection), 
because aDVF and random fault injection quantify
the resilience based on different metrics.
Also, the random fault injection can miss some fault masking events that can be captured by our approach.

\end{comment}
\section{Related Work}\label{sec:related}
 
The authors in \cite{humphreys2007noncontact} showed that it is possible to extract the PPG signal from the video using a complementary metal-oxide semiconductor camera by illuminating a region of tissue using through external light-emitting diodes at dual-wavelength (760nm and 880nm).  Further, the authors of  \cite{verkruysse2008remote} demonstrated that the PPG signal can be estimated by just using ambient light as a source of illumination along with a simple digital camera.  Further in \cite{poh2011advancements}, the PPG waveform was estimated from the videos recorded using a low-cost webcam. The red, green, and blue channels of the images were decomposed into independent sources using independent component analysis. One of the independent sources was selected to estimate PPG and further calculate HR, and HRV. All these works showed the possibility of extracting PPG signals from the videos and proved the similarity of this signal with the one obtained using a contact device. Further, the authors in \cite{10.1109/CVPR.2013.440} showed that heart rate can be extracted from features from the head as well by capturing the subtle head movements that happen due to blood flow.

%
The authors of \cite{kumar2015distanceppg} proposed a methodology that overcomes a challenge in extracting PPG for people with darker skin tones. The challenge due to slight movement and low lighting conditions during recording a video was also addressed. They implemented the method where PPG signal is extracted from different regions of the face and signal from each region is combined using their weighted average making weights different for different people depending on their skin color. 
%

There are other attempts where authors of \cite{6523142,6909939, 7410772, 7412627} have introduced different methodologies to make algorithms for estimating pulse rate robust to illumination variation and motion of the subjects. The paper \cite{6523142} introduces a chrominance-based method to reduce the effect of motion in estimating pulse rate. The authors of \cite{6909939} used a technique in which face tracking and normalized least square adaptive filtering is used to counter the effects of variations due to illumination and subject movement. 
The paper \cite{7410772} resolves the issue of subject movement by choosing the rectangular ROI's on the face relative to the facial landmarks and facial landmarks are tracked in the video using pose-free facial landmark fitting tracker discussed in \cite{yu2016face} followed by the removal of noise due to illumination to extract noise-free PPG signal for estimating pulse rate. 

Recently, the use of machine learning in the prediction of health parameters have gained attention. The paper \cite{osman2015supervised} used a supervised learning methodology to predict the pulse rate from the videos taken from any off-the-shelf camera. Their model showed the possibility of using machine learning methods to estimate the pulse rate. However, our method outperforms their results when the root mean squared error of the predicted pulse rate is compared. The authors in \cite{hsu2017deep} proposed a deep learning methodology to predict the pulse rate from the facial videos. The researchers trained a convolutional neural network (CNN) on the images generated using Short-Time Fourier Transform (STFT) applied on the R, G, \& B channels from the facial region of interests.
The authors of \cite{osman2015supervised, hsu2017deep} only predicted pulse rate, and we extended our work in predicting variance in the pulse rate measurements as well.

All the related work discussed above utilizes filtering and digital signal processing to extract PPG signals from the video which is further used to estimate the PR and PRV.  %
The method proposed in \cite{kumar2015distanceppg} is person dependent since the weights will be different for people with different skin tone. In contrast, we propose a deep learning model to predict the PR which is independent of the person who is being trained. Thus, the model would work even if there is no prior training model built for that individual and hence, making our model robust. 

%
% \vspace{-0.5em}
\section{Conclusion}
% \vspace{-0.5em}
Recent advances in multimodal single-cell technology have enabled the simultaneous profiling of the transcriptome alongside other cellular modalities, leading to an increase in the availability of multimodal single-cell data. In this paper, we present \method{}, a multimodal transformer model for single-cell surface protein abundance from gene expression measurements. We combined the data with prior biological interaction knowledge from the STRING database into a richly connected heterogeneous graph and leveraged the transformer architectures to learn an accurate mapping between gene expression and surface protein abundance. Remarkably, \method{} achieves superior and more stable performance than other baselines on both 2021 and 2022 NeurIPS single-cell datasets.

\noindent\textbf{Future Work.}
% Our work is an extension of the model we implemented in the NeurIPS 2022 competition. 
Our framework of multimodal transformers with the cross-modality heterogeneous graph goes far beyond the specific downstream task of modality prediction, and there are lots of potentials to be further explored. Our graph contains three types of nodes. While the cell embeddings are used for predictions, the remaining protein embeddings and gene embeddings may be further interpreted for other tasks. The similarities between proteins may show data-specific protein-protein relationships, while the attention matrix of the gene transformer may help to identify marker genes of each cell type. Additionally, we may achieve gene interaction prediction using the attention mechanism.
% under adequate regulations. 
% We expect \method{} to be capable of much more than just modality prediction. Note that currently, we fuse information from different transformers with message-passing GNNs. 
To extend more on transformers, a potential next step is implementing cross-attention cross-modalities. Ideally, all three types of nodes, namely genes, proteins, and cells, would be jointly modeled using a large transformer that includes specific regulations for each modality. 

% insight of protein and gene embedding (diff task)

% all in one transformer

% \noindent\textbf{Limitations and future work}
% Despite the noticeable performance improvement by utilizing transformers with the cross-modality heterogeneous graph, there are still bottlenecks in the current settings. To begin with, we noticed that the performance variations of all methods are consistently higher in the ``CITE'' dataset compared to the ``GEX2ADT'' dataset. We hypothesized that the increased variability in ``CITE'' was due to both less number of training samples (43k vs. 66k cells) and a significantly more number of testing samples used (28k vs. 1k cells). One straightforward solution to alleviate the high variation issue is to include more training samples, which is not always possible given the training data availability. Nevertheless, publicly available single-cell datasets have been accumulated over the past decades and are still being collected on an ever-increasing scale. Taking advantage of these large-scale atlases is the key to a more stable and well-performing model, as some of the intra-cell variations could be common across different datasets. For example, reference-based methods are commonly used to identify the cell identity of a single cell, or cell-type compositions of a mixture of cells. (other examples for pretrained, e.g., scbert)


%\noindent\textbf{Future work.}
% Our work is an extension of the model we implemented in the NeurIPS 2022 competition. Now our framework of multimodal transformers with the cross-modality heterogeneous graph goes far beyond the specific downstream task of modality prediction, and there are lots of potentials to be further explored. Our graph contains three types of nodes. while the cell embeddings are used for predictions, the remaining protein embeddings and gene embeddings may be further interpreted for other tasks. The similarities between proteins may show data-specific protein-protein relationships, while the attention matrix of the gene transformer may help to identify marker genes of each cell type. Additionally, we may achieve gene interaction prediction using the attention mechanism under adequate regulations. We expect \method{} to be capable of much more than just modality prediction. Note that currently, we fuse information from different transformers with message-passing GNNs. To extend more on transformers, a potential next step is implementing cross-attention cross-modalities. Ideally, all three types of nodes, namely genes, proteins, and cells, would be jointly modeled using a large transformer that includes specific regulations for each modality. The self-attention within each modality would reconstruct the prior interaction network, while the cross-attention between modalities would be supervised by the data observations. Then, The attention matrix will provide insights into all the internal interactions and cross-relationships. With the linearized transformer, this idea would be both practical and versatile.

% \begin{acks}
% This research is supported by the National Science Foundation (NSF) and Johnson \& Johnson.
% \end{acks}

\bibliographystyle{ACM-Reference-Format}
\balance
\bibliography{main} 

\end{document}
