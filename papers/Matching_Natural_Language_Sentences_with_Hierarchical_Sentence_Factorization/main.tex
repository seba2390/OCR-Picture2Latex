\documentclass[sigconf]{acmart}

\usepackage{booktabs} % For formal tables


\usepackage{amssymb}
\usepackage{amsthm, amsmath}
\usepackage{algorithmic}
\usepackage[ruled]{algorithm2e}
\usepackage{color}
\usepackage{url}
\usepackage{xfrac}

\usepackage{subfigure}
\usepackage{tablefootnote}
 
\newcommand{\red}[1]{\textcolor{red}{#1}}
\newcommand{\blue}[1]{\textcolor{blue}{#1}}

% Copyright
%\setcopyright{none}
%\setcopyright{acmcopyright}
%\setcopyright{acmlicensed}
%\setcopyright{rightsretained}
%\setcopyright{usgov}
%\setcopyright{usgovmixed}
%\setcopyright{cagov}
%\setcopyright{cagovmixed}

%%%%%% Added by Bang begin
\usepackage{tabularx}
%\usepackage{multicol} % use this will case single column title!!
\usepackage{multirow}
\usepackage{balance}
%%%%%% Added by Bang end



%Conference
%\acmConference[WOODSTOCK'97]{ACM Woodstock conference}{July 1997}{El
%  Paso, Texas USA} 
%\acmYear{1997}
%\copyrightyear{2016}

%\acmPrice{15.00}
\fancyhead{}  


\begin{document}
\title{Matching Natural Language Sentences \\with Hierarchical Sentence Factorization}

\copyrightyear{2018}
\acmYear{2018} 
\setcopyright{iw3c2w3}
\acmConference[WWW 2018]{The 2018 Web Conference}{April 23--27, 2018}{Lyon, France}
\acmBooktitle{WWW 2018: The 2018 Web Conference, April 23--27, 2018, Lyon, France}
\acmPrice{}
\acmDOI{10.1145/3178876.3186022}
\acmISBN{978-1-4503-5639-8}

% \titlenote{Produces the permission block, and
%   copyright information}
% \subtitle{Extended Abstract}
% \subtitlenote{The full version of the author's guide is available as
%   \texttt{acmart.pdf} document}

% \copyrightyear{2017} 
% \acmYear{2017} 
% \setcopyright{acmcopyright}
% \acmConference{CIKM'17}{}{November 6--10, 2017, Singapore.}
% \acmPrice{15.00}
% \acmDOI{https://doi.org/10.1145/3132847.3132852}
% \acmISBN{ISBN 978-1-4503-4918-5/17/11}

% \fancyhead{}
% \settopmatter{printacmref=false, printfolios=false}

% \author{Bang Liu}
% %\authornote{Dr.~Trovato insisted his name be first.}
% %\orcid{1234-5678-9012}
% \affiliation{%
%   \institution{Department of Electrical and Computer Engineering}
%   \streetaddress{University of Alberta}
%   \city{Edmonton} 
%   \state{Alberta, Canada} 
%   %\postcode{43017-6221}
% }
% %\email{bang3@ualberta.ca}

% \author{Di Niu}
% %\authornote{The secretary disavows any knowledge of this author's actions.}
% \affiliation{%
%   \institution{Department of Electrical and Computer Engineering}
%   \streetaddress{University of Alberta}
%   \city{Edmonton} 
%   \state{Alberta, Canada} 
%   %\postcode{43017-6221}
% }
% %\email{}

% \author{Kunfeng Lai}
% %\authornote{This author is the one who did all the really hard work.}
% \affiliation{%
%   \institution{Mobile Internet Group}
%   \streetaddress{Tencent}
%   \city{Shenzhen} 
%   \country{China}}
% %\email{}

% \author{Linglong Kong}
% \affiliation{%
%   \institution{Department of Mathematical and Statistical Sciences}
%   \streetaddress{University of Alberta}
%   \city{Edmonton} 
%   \state{Alberta, Canada} 
%   %\postcode{43017-6221}
% }
% %\email{}

% \author{Yu Xu}
% \affiliation{%
%   \institution{Mobile Internet Group}
%   \streetaddress{Tencent}
%   \city{Shenzhen} 
%   \country{China}}
% %\email{}




% \author{Charles Palmer}
% \affiliation{%
%   \institution{Palmer Research Laboratories}
%   \streetaddress{8600 Datapoint Drive}
%   \city{San Antonio}
%   \state{Texas} 
%   \postcode{78229}}
% \email{cpalmer@prl.com}

% \author{John Smith}
% \affiliation{\institution{The Th{\o}rv{\"a}ld Group}}
% \email{jsmith@affiliation.org}

% \author{Julius P.~Kumquat}
% \affiliation{\institution{The Kumquat Consortium}}
% \email{jpkumquat@consortium.net}

% % The default list of authors is too long for headers}
% \renewcommand{\shortauthors}{B. Trovato et al.}

\author{Bang Liu$^1$, Ting Zhang$^1$, Fred X. Han$^1$, Di Niu$^1$, Kunfeng Lai$^2$, Yu Xu$^2$}
       \affiliation{$^1$University of Alberta, Edmonton, AB, Canada}
 \affiliation{$^2$Mobile Internet Group, Tencent, Shenzhen, China}
       %\email{haolan@ualberta.ca}

%\email{bang3@ualberta.ca}


% \numberofauthors{5}
% \author{
% % You can go ahead and credit any number of authors here,
% % e.g. one 'row of three' or two rows (consisting of one row of three
% % and a second row of one, two or three).
% %
% % The command \alignauthor (no curly braces needed) should
% % precede each author name, affiliation/snail-mail address and
% % e-mail address. Additionally, tag each line of
% % affiliation/address with \affaddr, and tag the
% % e-mail address with \email.
% %
% % 1st. author
% %\alignauthor
% Bang Liu$^1$, Di Niu$^1$, Kunfeng Lai$^2$, Linglong Kong$^1$, Yu Xu$^$\\
%        \affaddr{$^1$University of Alberta, Edmonton, AB, Canada}\\
%  \affaddr{$^2$Mobile Internet Group, Tencent Inc., Shenzhen, China}\\
%        %\email{haolan@ualberta.ca}
% }



\begin{abstract}
Semantic matching of natural language sentences or identifying the relationship between two sentences is a core research problem underlying many natural language tasks.
%Prior research has proposed both unsupervised distance-based schemes, when training data is not available, and deep learning schemes for sentence matching, given training data. 
Depending on whether training data is available, prior research has proposed both unsupervised distance-based schemes and supervised deep learning schemes for sentence matching.
However, previous approaches either omit or fail to fully utilize the ordered, hierarchical, and flexible structures of language objects, as well as the interactions between them.
In this paper, we propose \textit{Hierarchical Sentence Factorization}---a technique %that is able 
to factorize a sentence into a hierarchical representation, with the components at each different scale reordered into a ``predicate-argument'' form. The proposed sentence factorization technique leads to the invention of: 1) a new unsupervised distance metric which calculates the semantic distance between a pair of text snippets by solving a penalized optimal transport problem while preserving the logical relationship of words in the reordered sentences, and 2) new multi-scale deep learning models for supervised semantic training, based on factorized sentence hierarchies.
We apply our techniques to text-pair similarity estimation and text-pair relationship classification tasks, based on multiple datasets such as STSbenchmark, the Microsoft Research paraphrase identification (MSRP) dataset, the SICK dataset, etc. Extensive experiments show that the proposed hierarchical sentence factorization can be used to significantly improve the performance of existing unsupervised distance-based metrics as well as multiple supervised deep learning models based on the convolutional neural network (CNN) and long short-term memory (LSTM).
% \footnote{This is an abstract footnote}. 
\end{abstract}

%
% The code below should be generated by the tool at
% http://dl.acm.org/ccs.cfm
% Please copy and paste the code instead of the example below. 
%
% \begin{CCSXML}
% <ccs2012>
%  <concept>
%   <concept_id>10010520.10010553.10010562</concept_id>
%   <concept_desc>Computer systems organization~Embedded systems</concept_desc>
%   <concept_significance>500</concept_significance>
%  </concept>
%  <concept>
%   <concept_id>10010520.10010575.10010755</concept_id>
%   <concept_desc>Computer systems organization~Redundancy</concept_desc>
%   <concept_significance>300</concept_significance>
%  </concept>
%  <concept>
%   <concept_id>10010520.10010553.10010554</concept_id>
%   <concept_desc>Computer systems organization~Robotics</concept_desc>
%   <concept_significance>100</concept_significance>
%  </concept>
%  <concept>
%   <concept_id>10003033.10003083.10003095</concept_id>
%   <concept_desc>Networks~Network reliability</concept_desc>
%   <concept_significance>100</concept_significance>
%  </concept>
% </ccs2012>  
% \end{CCSXML}

% \ccsdesc[500]{Computer systems organization~Embedded systems}
% \ccsdesc[300]{Computer systems organization~Redundancy}
% \ccsdesc{Computer systems organization~Robotics}
% \ccsdesc[100]{Networks~Network reliability}

% We no longer use \terms command
%\terms{Theory}

\settopmatter{printacmref=false}

% \keywords{Hierarchical Sentence Factorization; Sentence Reordering; Ordered Word Mover's Distance; Abstract Meaning Representation}
\maketitle

\section{Introduction}  \label{sec:introduction}

\newcommand\inexpIntro[3]{#1?(#2,#3).}
\newcommand\rinexpIntro[3]{*#1?(#2,#3).}
\newcommand\outexpIntro[3]{#1!(#2,#3).}
\newcommand\outatomIntro[3]{#1!(#2,#3)}

We propose a fully automated method for proving termination of \(\pi\)-calculus processes.
Although there have been a lot of studies on termination analysis for the \(\pi\)-calculus
and related calculi~\cite{Deng06IC,Demangeon07,SangiorgiTermination,KobayashiHybrid,Yoshida04IC,DBLP:journals/jlp/DemangeonHS10,Venet98SAS}, most of them have been rather theoretical,
and there have been surprisingly little efforts in developing  fully automated termination
verification methods and tools based on them. To our knowledge,
Kobayashi's \typical{}~\cite{TyPiCal,KobayashiHybrid} is the only exception that
can prove termination of \(\pi\)-calculus processes (extended with natural numbers)
fully automatically, but its termination analysis is quite limited (see Section~\ref{sec:relatedwork}).

Our method is based on a reduction to termination analysis for sequential programs:
we translate a \(\pi\)-calculus process \(P\) to a sequential program \(S_P\), so that
if \(S_P\) is terminating, so is \(P\). The reduction allows us to use
powerful, mature methods and tools
for termination analysis of sequential programs~\cite{heizmann2016ultimate,freqterm,DBLP:conf/lics/PodelskiR04,Kuwahara2014Termination,DBLP:journals/cacm/CookPR11}.

The idea of the translation is to convert a chain of communications on replicated input
channels to a chain of recursive function calls of the target sequential program.
Let us consider the following Fibonacci process:
\begin{align*}
    & \rinexpIntro{\fib}{n}{r}
        \ifexp{n<2}{ \soutatom{r}{1} \\ &\quad}
                   { \nuexp{s_1} \nuexp{s_2} (\outatomIntro{\fib}{n-1}{s_1} \PAR \outatomIntro{\fib}{n-2}{s_2} \PAR \sinexp{s_1}{x}\sinexp{s_2}{y}\soutatom{r}{x+y}) \\}
    & \PAR \outatomIntro{\fib}{m}{r}
\end{align*}
Here, the process
$\rinexpIntro{\fib}{n}{r} \ldots$ is a function server that computes the \(n\)-th Fibonacci number
in parallel and returns the result to \(r\),
and $\outatom{\fib}{m}{r}$ sends a request for computing the \(m\)-th Fibonacci number;
those who are not familiar with the syntax of the \(\pi\)-calculus may wish to consult
Section~\ref{sec:targetlanguage} first.
To prove that the process above is terminating for any integer \(m\),
it suffices to show that there is no infinite chain of communications on $\fib$:
\[
    \fib(m,r) \to \fib(m_1,r_1) \to \fib(m_2,r_2) \to \cdots.
\]
We convert the process above to the following program:\footnote{The actual translation
  given later is a little more complex.}
\begin{verbatim}
 let rec fib(n) = if n<2 then () else (fib(n-1) [] fib(n-2)) in
 fib(m)
\end{verbatim}
Here, \texttt{[]} represents the non-deterministic choice.
Note that, although the calculation of Fibonacci numbers is not preserved,
for each chain of communications on \texttt{fib}, there is a corresponding
sequence of recursive calls:
\[
\mathtt{fib}(m) \to \mathtt{fib}(m_1) \to \mathtt{fib}(m_2) \to \cdots.
\]
Thus, the termination of the sequential program above implies the termination of
the original process.
As shown in the example above, (i) each communication on a replicated input channel
is converted to a function call, (ii) each communication on a non-replicated input
channel is just removed (or, in the actual translation, replaced by a call of
a trivial function defined by \(f(\seq{x})=(\,)\)), and (iii) parallel composition
is replaced by a non-deterministic choice.
We formalize the translation outlined above and prove its correctness.

The basic translation sketched above sometimes loses too much information.
For example, consider the following process:
\begin{align*}
    & \rinexpIntro{\pre}{n}{r} \soutatom{r}{n-1} \\
    & \PAR \rinexpIntro{f}{n}{r} \ifexp{n<0}{ \soutatom{r}{1} }
                                       { \nuexp{s} (\outatomIntro{\pre}{n}{s} \PAR \sinexp{s}{x}\outatomIntro{f}{x}{r}) } \\
    & \PAR \outatomIntro{f}{m}{r}
\end{align*}
The translation sketched above would yield:
\begin{verbatim}
  let pred(n) = n-1 in
  let rec f(n) = if n<0 then () else (pred(n) [] f(*)) in
  f(m)
\end{verbatim}
Here, \texttt{*} represents a non-deterministic integer: since we have removed
the input $\sinatom{s}{x}$, we do not have information about the value of \( x \).
As a result, the sequential program above is non-terminating, although the original
process is terminating.
To remedy this problem, we also refine the basic translation above by using a refinement
type system for the \(\pi\)-calculus. Using the refinement type system,
we can infer that the value of \(x\) in the original process is less than \(n\),
so that we can refine the definition of \texttt{f} to:
\begin{verbatim}
 let rec f(n) = ... else (pred(n) [] let x=* in assume(x<n);f(x))
\end{verbatim}
The target program is now terminating, from which
we can deduce that the original process is also terminating.
We have implemented an automated tool based on the refined translation above.

The contributions of this paper are summarized as follows.
\begin{itemize}
\item The formalization of the basic translation from the \(\pi\)-calculus
  (extended with integers) to sequential programs, and a proof of its correctness.
\item The formalization of a refined translation based on a refinement type system.
\item An implementation of the refined translation, including automated refinement type
  inference based on CHC solving, and experiments to evaluate the effectiveness of
  our method.
\end{itemize}

The rest of this paper is structured as follows.
Section~\ref{sec:targetlanguage} introduces the source and target languages
of our translation.
Section~\ref{sec:approach} 
formalizes the basic translation, and proves its correctness.
Section~\ref{sec:refinement} refines the basic translation by using a refinement type system.
Section~\ref{sec:implementation} reports an implementation and experiments.
Section~\ref{sec:relatedwork} discusses related work,
and Section~\ref{sec:conclusion} concludes the paper.

%!TEX root = main.tex
\section{Hierarchical Sentence Factorization and Reordering}
\label{sec:sentence}


\begin{figure*}[tb]
\centering
\includegraphics[width=\textwidth]{figure/CaseStudy}
\vspace{-3mm}
\caption{An example of the sentence factorization process. Here we show: A. The original sentence pair; B. The procedures of creating sentence factorization trees; C. The predicate-argument form of original sentence pair; D. The alignment of semantic units with the reordered form.}
\label{fig:casestudy}
\vspace{-1mm}
\end{figure*}

In this section, we present our \textit{Hierarchical Sentence Factorization} techniques to transform a sentence into a hierarchical tree structure, which also naturally produces a reordering of the sentence at the root node. 
%Such multi-scaled sentence factorization and reordering 
This multi-scaled representation form proves to be effective at improving both unsupervised and supervised semantic matching, which will be discussed in Sec.~\ref{sec:owmd} and Sec.~\ref{sec:multi-layer}, respectively. 

We first describe our desired factorization tree structure before presenting the steps to obtain it. Given a natural language sentence $S$, our objective is to transform it into a semantic factorization tree denoted by $T^f_S$. Each node in $T^f_S$ is called a \textit{semantic unit}, which contains one or a few tokens (tokenized words) from the original sentence $S$, as illustrated in Fig.~\ref{fig:casestudy} $(a4)$, $(b4)$.
The tokens in every semantic unit in $T^f_S$ is re-organized into a ``predicate-argument'' form. For example, a semantic unit for ``Tom catches Jerry'' in the ``predicate-argument'' form will be ``catch Tom Jerry''.

Our proposed factorization tree recursively factorizes a sentence into a hierarchy of semantic units at different granularities to represent the semantic structure of that sentence.
%In different depths of a factorization tree, a sentence will be factorized into semantic units in different granularities.
The root node in a factorization tree contains the entire sentence reordered in the predicate-argument form, thus providing a ``normalized'' representation for sentences expressed in different ways (e.g., passive vs. active tenses). Moreover, each semantic unit at depth $d$ will be further split into several child nodes at depth $d + 1$, which are smaller semantic sub-units. Each sub-unit also follows the predicate-argument form.

For example, in Fig.~\ref{fig:casestudy}, we convert sentence $A$ into a hierarchical factorization tree $(a4)$ using a series of operations. The root node of the tree contains the semantic unit ``chase Tom Jerry little yard big'', which is the reordered representation of the original sentence ``The little Jerry is being chased by Tom in the big yard'' in a semantically normalized form. 
Moreover, the semantic unit at depth $0$ is factorized into four sub-units at depth $1$: ``chase'', ``Tom'', ``Jerry little'' and ``yard big'', each in the ``predicate-argument'' form. And at depth $2$, the semantic sub-unit ``Jerry little'' is further factorized into two sub-units ``Jerry'' and ``little''. Finally, a semantic unit that contains only one token (e.g., ``chase'' and ``Tom'' at depth $1$) can not be further decomposed. Therefore, it only has one child node at the next depth through self-duplication.

We can observe that each depth of the tree contains all the tokens (except meaningless ones) in the original sentence, but re-organizes these tokens into semantic units of different granularities. %In Sec.~\ref{sec:owmd} and Sec.~\ref{sec:multi-layer}, we will show that such a specially designed sentence factorization tree together with the naturally generated semantic reordering can be used to effectively improve both unsupervised and supervised semantic matching methods.


% Our approach is mainly based on the Abstract Meaning Representation (AMR) of a sentence \cite{banarescu2013abstract, flanigan2014discriminative}.

\subsection{Hierarchical Sentence Factorization}
\label{subsec:sf}

We now describe our detailed procedure to transform a natural language sentence to the desired factorization tree mentioned above.
Our Hierarchical Sentence Factorization algorithm mainly consists of five steps: 1) AMR parsing and alignment, 2) AMR purification, 3) index mapping, 4) node completion, and 5) node traversal. The latter four steps are illustrated in the example in Fig.~\ref{fig:casestudy} from left to right.%We will explain each step in detail in the following.


\begin{figure}[tb]
\centering
\includegraphics[width=2.7in]{figure/amr}
\vspace{0mm}
\caption{An example of a sentence and its Abstract Meaning Representation (AMR), as well as the alignment between the words in the sentence and the nodes in AMR.}
\label{fig:amr}
\vspace{-3mm}
\end{figure}

\textbf{AMR parsing and alignment.}
Given an input sentence, the first step of our hierarchical sentence factorization algorithm is to acquire its Abstract Meaning Representation (AMR), as well as perform AMR-Sentence alignment to align the concepts in AMR with the tokens in the original sentence.
 
Semantic parsing \cite{baker1998berkeley,kingsbury2002treebank,berant2014semantic,banarescu2013abstract, damonte2016incremental} can be performed to generate the formal semantic representation of a sentence.
Abstract Meaning Representation (AMR) \cite{banarescu2013abstract} is a semantic parsing language that represents a sentence by a directed acyclic graph (DAG).
Each AMR graph can be converted into an AMR tree by duplicating the nodes that have more than one parent.

Fig.~\ref{fig:amr} shows the AMR of the sentence ``I observed that the army moved quickly.''
In an AMR graph, leaves are labeled with concepts, which represent either English words (e.g., ``army''), PropBank framesets (e.g., ``observe-01'') \cite{kingsbury2002treebank}, or special keywords (e.g., dates, quantities, world regions, etc.). For example, ``(a / army)'' refers to an instance of the concept army, where ``a'' is the variable name of army (each entity in AMR has a variable name).  ``ARG0'', ``ARG1'', ``:manner'' are different kinds of relations defined in AMR. Relations are used to link entities. For example, ``:manner'' links ``m / move-01'' and ``q / quick'', which means ``move in a quick manner''. Similarly, ``:ARG0'' links ``m / move-01'' and ``a / army'', which means that ``army'' is the first argument of ``move''.


%Given an input sentence, a semantic parser extracts a semantic representation of the sentence \cite{damonte2016incremental}.
% AMR aims to \red{normalize (??canonicalize)} different ways of stating the same thing and assigns the same AMR to sentences with the same meaning.


Each leaf in AMR is a concept rather than the original token in a sentence. 
The alignment between a sentence and its AMR graph is not given in the AMR annotation. Therefore, AMR alignment \cite{pourdamghani2014aligning} needs to be performed to link the leaf nodes in the AMR to tokens in the original sentence.
Fig.~\ref{fig:amr} shows the alignment between sentence tokens and AMR concepts by the alignment indexes.
The alignment index $0$ is for the root node, 0.0 for the first child of the root node, 0.1 for the second child of the root node, and so forth. For example, in Fig.~\ref{fig:amr}, the word ``army'' in sentence is linked with index ``0.1.0'', which represents the concept node ``a / army'' in its AMR.
% The numbering system will skip variable \red{re-entrancies} of duplicated nodes.
% AMR strives to achieve a more logical representation for sentences. It attempts to assign the same AMR to sentences that have the same meaning. For example, ``I observed the quick movement of the army'' and ``I observed the army moved quickly'' will have the same AMR representation that is shown in Fig.~\ref{fig:amr}.
We refer interested readers to \cite{banarescu2013abstract,banarescu2012abstract} for more detailed description about AMR. 

% We utilize the AMR representation of sentences to get a unified, hierarchical and reordered sentence representation.
%Alg.~.... shows the steps of sentence factorization.
Various parsers have been proposed for AMR parsing and alignment \cite{flanigan2014discriminative,wang2015boosting}. We choose the JAMR parser \cite{flanigan2014discriminative} in our algorithm implementation. 

\begin{figure}[tb]
\centering
\includegraphics[width=3.0in]{figure/amr2}
\vspace{0mm}
\caption{An example to show the operation of AMR purification.}
\label{fig:amr2}
\vspace{-3mm}
\end{figure}


\textbf{AMR purification.} Unfortunately, AMR itself cannot be used to form the desired factorization tree. 
First, it is likely that multiple concepts in AMR may link to the same token in the sentence.
For example, Fig.~\ref{fig:amr2} shows AMR and its alignment for the sentence ``Three Asian kids are dancing.''.
The token ``Asian'' is linked to four concepts in the AMR graph: `` continent (0.0.0)'', ``name (0.0.0.0)'', ``Asia (0.0.0.0.0)'' and ``wiki Asia (0.0.0.1)''.
This is because AMR will match a named entity with predefined concepts which it belongs to, such as ``c / continent'' for ``Asia'', and form a compound representation of the entity. For example, in Fig.\ref{fig:amr2}, the token ``Asian'' is represented as a continent whose name is Asia, and its Wikipedia entity name is also Asia.

In this case, we select the link index with the smallest tree depth as the token's position in the tree.
Suppose $\mathcal{P}_w = \{p_1, p_2, \cdots, p_{|\mathcal{P}|}\}$ denotes the set of alignment indexes of token $w$.
We can get the desired alignment index of $w$ by calculating the longest common prefix of all the index strings in  $\mathcal{P}_w$.
After getting the alignment index for each token, we then replace the concepts in AMR with the tokens in sentence by the alignment indexes, and remove relation names (such as ``:ARG0'') in AMR, resulting into a compact tree representation of the original sentence, as shown in the right part of Fig.~\ref{fig:amr2}.


\textbf{Index mapping.}
A purified AMR tree for a sentence obtained in the previous step is still not in our desired form.
To transform it into a hierarchical sentence factorization tree, we perform index mapping and calculate a new position (or index) for each token in the desired factorization tree given its position (or index) in the purified AMR tree.
Fig.~\ref{fig:casestudy} illustrates the process of index mapping. After this step, for example, the purified AMR trees in Fig.~\ref{fig:casestudy} $(a1)$ and $(b1)$ will be transformed into $(a2)$ and $(b2)$.

Specifically, let $T^p_S$ denote a purified AMR tree of sentence $S$, and $T^f_S$ our desired sentence factorization tree of $S$.
Let $I^p_N = \overline{i_0.i_1.i_2.\cdots.i_d}$ denote the index of node $N$ in $T^p_S$, where $d$ is the depth of $N$ in $T^p_S$ (where depth 0 represents the root of a tree).
Then, the index $I^f_N$ of node $N$ in our desired factorization tree $T^f_S$ will be calculated as follows:
\begin{equation}
\label{eqn:transform-index}
I^f_N := \begin{cases}
 		     \overline{0.0} & \quad \text{if } d=0, \\ 
 			 \overline{i_0.(i_1+1).(i_2+1).\cdots.(i_d + 1)} & \quad \text{otherwise}. 
		 \end{cases}
\end{equation}
After index mapping, we add an empty root node with index $0$ in the new factorization tree, and link all nodes at depth $1$ to it as its child nodes. Note that the $i_0$ in every node index will always be 0.


\textbf{Node completion.}
We then perform node completion to make sure each branch of the factorization tree have the same maximum depth and to fill in the missing nodes caused by index mapping, illustrated by Fig.~\ref{fig:casestudy} $(a3)$ and $(b3)$.

First, given a pre-defined maximum depth $D$, for each leaf node $N^l$ with depth $d < D$ in the current $T^f_S$ after index mapping, we duplicate it for $D - d$ times and append all of them sequentially to $N^l$, as shown in Fig.~\ref{fig:casestudy} $(a3)$, $(b3)$, such that the depths of the ending nodes will always be $D$. For example, in Fig.~\ref{fig:casestudy} with $D = 2$, the node ``chase (0.0)'' and ``Tom (0.1)'' will be extended to reach depth $2$ via self-duplication.

Second, after index mapping, the children of all the non-leaf nodes, except the root node, will be indexed starting from 1 rather than 0. For example, in Fig.~\ref{fig:casestudy} $(a2)$, the first child node of ``Jerry (0.2)'' is ``little (0.2.1)''. %However, the first child node according to our indexing rule should have the index of ``0.2.0'', which is missed at present. 
In this case, we duplicate ``Jerry (0.2)'' itself to ``Jerry (0.2.0)'' to fill in the missing first child of ``Jerry (0.2)''. Similar filling operations are done for other non-leaf nodes after index mapping as well.

\textbf{Node traversal to complete semantic units}.
Finally, we complete each semantic unit in the formed factorization tree via node traversal, as shown in Fig.~\ref{fig:casestudy} $(a4)$, $(b4)$.
For each non-leaf node $N$, we traverse its sub-tree by Depth First Search (DFS). The original semantic unit in $N$ will then be replaced by the concatenation of the semantic units of all the nodes in the sub-tree rooted at $N$, following the order of traversal.
%Notice that the traverse process should skip all the nodes completed in the previous step of node completion.

For example, for sentence $A$ in Fig.~\ref{fig:casestudy}, after node traversal, the root node of the factorization tree becomes ``chase Tom Jerry little yard big'' with index ``0''. We can see that the original sentence has been reordered into a predicate-argument structure. A similar structure is generated for the other nodes at different depths. 
Until now, each depth of the factorization tree $T^f_S$ can express the full sentence $S$  in terms of  semantic units at different granularity.




%!TEX root = main.tex
\section{Ordered Word Mover's Distance}
\label{sec:owmd}

\begin{figure}[tb]
\centering
\includegraphics[width=2.5in]{figure/SentenceMatch}
\vspace{0mm}
\caption{Compare the sentence matching results given by Word Mover's Distance and Ordered Word Mover's Distance.}
\label{fig:match}
\vspace{-3mm}
\end{figure}

The proposed hierarchical sentence factorization technique naturally reorders an input sentence into a unified format at the root node. In this section, we introduce the \textit{Ordered Word Mover's Distance} metric which measures the semantic distance between two input sentences based on the unified representation of reordered sentences.

Assume $X\in \mathbb{R}^{d\times n}$ is a \textit{word2vec} embedding matrix for a vocabulary of $n$ words, and the $i$-th column $\mathbf{x}_i \in \mathbb{R}^d$ represents the $d$-dimensional embedding vector of $i$-th word in vocabulary.
Denote a sentence $S = \overline{a_1a_2\cdots a_K}$ where $a_i$ represents the $i$-th word (or the word embedding vector).
The Word Mover's Distance considers a sentence $S$ as its normalized bag-of-words (nBOW) vectors where the weights of the words in $S$ is $\mathbf \alpha = \{\alpha_1, \alpha_2, \cdots, \alpha_K\}$. Specifically, if word $a_i$ appears $c_i$ times in $S$, then $\alpha_i = \frac{c_i}{\sum_{j=1}^K c_j}$.

The Word Mover's Distance metric combines the normalized bag-of-words representation of sentences with Wasserstein distance (also known as Earth Mover's Distance \cite{rubner2000earth}) to measure the semantic distance between two sentences. 
Given a pair of sentences $S_1 = \overline{a_1a_2\cdots a_M}$ and $S_2 = \overline{b_1b_2\cdots b_N}$, where %$a_i \in \mathbb{R}^d$ is the embedding vector of the $i$-th word in $S_1$ and 
$b_j \in \mathbb{R}^d$ is the embedding vector of the $j$-th word in $S_2$. Let $\mathbf \alpha = \{\alpha_1, \cdots, \alpha_M\}$ and $\mathbf \beta = \{\beta_1, \cdots, \beta_N\}$ represents the normalized bag-of-words vectors of $S_1$ and $S_2$. We can calculate a distance matrix $D \in \mathbb{R}^{M\times N}$ where each element
$D_{ij} = \|a_i - b_j\|_2$ measures the distance between word $a_i$ and $b_j$ (we use the same notation to denote the word itself or its word vector representation).
Let $T \in \mathbb{R}^{M\times N}$ be a non-negative sparse transport matrix where $T_{ij}$ denotes the portion of word $a_i \in S_1$ that transports to word $b_j \in S_2$.
The Word Mover's Distance between sentences $S_1$ and $S_2$ is given by $\sum_{i,j} T_{ij} D_{ij}$. The transport matrix $T$ is computed solving the following constrained optimization problem:
\begin{equation}
\label{eq:wmd}
\begin{split}
	\underset{T \in \mathbb{R}_{+}^{M\times N}}{\mbox{minimize}}\quad 		& \sum_{i,j} T_{ij} D_{ij} \\
	\mbox{subject to}\quad & \sum\limits_{i = 1}^{M}  T_{ij} = \beta_j \quad 1\leq j \leq N,\\
			   & \sum\limits_{j = 1}^{N}  T_{ij} = \alpha_i \quad 1\leq i \leq M.
\end{split}
\end{equation}
Where the minimum ``word travel cost'' between two bags of words for a pair of sentences is calculated to measure the their semantic distance.

However, the Word Mover's Distance fails to consider a few aspects of natural language. First, it omits the sequential structure. For example, in Fig.~\ref{fig:match}, the pair of sentences ``Morty is laughing at Rick'' and ``Rick is laughing at Morty'' only differ in the order of words. The Word Mover's Distance metric will then find an exact match between the two sentences and estimate the semantic distance as zero, which is obviously false. Second, the normalized bag-of-words representation of a sentence can not distinguish duplicated words shown in multiple positions of a sentence.

To overcome the above challenges, we propose a new kind of semantic distance metric named Ordered Word Mover's Distance (OWMD). The Ordered Word Mover's Distance combines our sentence factorization technique with Order-preserving Wasserstein Distance proposed in \cite{su2017order}. It casts the calculation of semantic distance between texts as an optimal transport problem while preserving the sequential structure of words in sentences. The Ordered Word Mover's Distance differs from the Word Mover's Distance in multiple aspects.

First, rather than using normalized bag-of-words vector to represent a sentence, we decompose and re-organize a sentence using the sentence factorization algorithm described in Sec.~\ref{sec:sentence}. Given a sentence $S$, we represent it by the reordered word sequence $S'$ in the root node of its sentence factorization tree. Such representation normalizes a sentence into ``predicate-argument'' structure to better handle syntactic variations.
%process sentences with different syntactic structures that express the same meaning. 
For example, after performing sentence factorization, sentences ``Tom is chasing Jerry'' and ``Jerry is being chased by Tom'' will both be normalized as ``chase Tom Jerry''.

Second, we calculate a new transport matrix $T$ by solving the following optimization problem
\begin{equation}
\label{eq:owmd}
\begin{split}
	\underset{T \in \mathbb{R}_{+}^{M\times N}}{\mbox{minimize}}\quad 		& \sum_{i,j} T_{ij} D_{ij} - \lambda_1 I(T) + \lambda_2 KL(T||P)\\
	\mbox{subject to}\quad & \sum\limits_{i = 1}^{M}  T_{ij} = \beta_j' \quad 1\leq j \leq N',\\
			   & \sum\limits_{j = 1}^{N}  T_{ij} = \alpha_i' \quad 1\leq i \leq M',
\end{split}
\end{equation}
where $\lambda_1 > 0$ and $\lambda_2 > 0$ are two hyper parameters.
$M'$ and $N'$ denotes the number of words in $S_1'$ and $S_2'$.
$\alpha_i'$ denotes the weight of the $i$-th word in normalized sentence $S_1'$ and $\beta_j'$ denotes the weight of the $j$-th word in normalized sentence $S_2'$. Usually we can set $\mathbf{\alpha'} = (\frac{1}{M'}, \cdots, \frac{1}{M'})$ and $\mathbf{\beta'} = (\frac{1}{N'}, \cdots, \frac{1}{N'})$ without any prior knowledge of word differences.

The first penalty term $I(T)$ is the inverse difference moment \cite{albregtsen2008statistical} of the transport matrix $T$ that measures local homogeneity of $T$. It is defined as:
\begin{equation}
\label{eq:IT}
\begin{split}
	I(T) = \sum\limits_{i=1}^{M'} \sum\limits_{j=1}^{N'} \frac{T_{ij}}{(\frac{i}{M'} - \frac{j}{N'})^2 + 1}.
\end{split}
\end{equation}
$I(T)$ will have a relatively large value if the large values of $T$ mainly appear near its diagonal.

Another penalty term $KL(T||P)$ denotes the KL-divergence between $T$ and $P$. 
$P$ is a two-dimensional distribution used as the prior distribution for values in $T$. It is defined as
\begin{equation}
\label{eq:P}
\begin{split}
	P_{ij} = \frac{1}{\sigma \sqrt{2\pi}}e^{- \frac{l^2(i,j)}{2\sigma^2}}
\end{split}
\end{equation}
where $l(i, j)$ is the distance from position $(i, j)$ to the diagonal line, which is calculated as
\begin{equation}
\label{eq:l}
\begin{split}
	l(i, j) = \frac{|i/M' - j/N'|}{\sqrt{1/M'^2 + 1/N'^2}}.
\end{split}
\end{equation}
As we can see, the farther a word in one sentence is from the other word in another sentence in terms of word orders, the less likely it will be transported to that word. Therefore, by introducing the two penalty terms $I(T)$ and $KL(T||P)$ into problem~(\ref{eq:owmd}), 
we encourage words at similar positions in two sentences to be matched.
%we encourage the words in two sentences with similar positions be matched. 
Words at distant positions are less likely to be matched by $T$.

The problem (\ref{eq:owmd}) has a unique optimal solution $T^{\lambda_1, \lambda_2}$ since both the objective and the feasible set are convex. It has been proved in \cite{su2017order} that the optimal $T^{\lambda_1, \lambda_2}$ has the same form with $diag(\mathbf{k}_1) \cdot \mathbf{K} \cdot diag(\mathbf{k}_2)$, where $diag(\mathbf{k}_1) \in \mathbb{R}^{M'}$ and $diag(\mathbf{k}_2) \in \mathbb{R}^{N'}$ are two diagonal matrices with strictly positive diagonal elements. $\mathbf{K} \in \mathbb{R}^{M'\times N'}$ is a matrix defined as
\begin{equation}
\label{eq:K}
\begin{split}
	K_{ij} = P_{ij} e^{\frac{1}{\lambda_2}(S_{ij}^{\lambda_1} - D_{ij})},
\end{split}
\end{equation}
where
\begin{equation}
\label{eq:S}
\begin{split}
	S_{ij} = \frac{\lambda_1}{(\frac{i}{M'} - \frac{j}{N'})^2 + 1}.
\end{split}
\end{equation}
The two matrices $\mathbf{k}_1$ and $\mathbf{k}_2$ can be efficiently obtained by the Sinkhorn-Knopp iterative matrix scaling algorithm \cite{knight2008sinkhorn}:
\begin{equation}
\label{eq:k1}
\begin{split}
	\mathbf{k}_1 &\leftarrow \mathbf{\alpha'} ./ K \mathbf{k}_2, \\
	\mathbf{k}_2 &\leftarrow \mathbf{\beta'} ./ K^T \mathbf{k}_1.
\end{split}
\end{equation}
where $./$ is the element-wise division operation.
%Until now, we have introduced the sentence representation and the way we calculate the transport matrix $T$ for the Ordered Word Mover's Distance metric.
Compared with Word Mover's Distance, the Ordered Word Mover's Distance 
considers the positions of words in a sentence,
%takes the positions of words in sentences into account, 
and is able to distinguish duplicated words at different locations. For example, in Fig.~\ref{fig:match}, while the WMD finds an exact match and get a semantic distance of zero for the sentence pair ``Morty is laughing at Rick'' and ``Rick is laughing at Morty'', the OWMD metric is able to find a better match relying on the penalty terms, and gives a semantic distance greater than zero.

The computational complexity of OWMD is also effectively reduced compared to WMD. With the additional constraints, the time complexity is $O(dM'N')$ where $d$ is the dimension of word vectors \cite{su2017order}, while it is $O(dp^3\log p)$ for WMD, where $p$ denotes the number of uniques words in sentences or documents \cite{kusner2015word}.


%!TEX root = main.tex
\section{Multi-scale Sentence Matching}
\label{sec:multi-layer}


\begin{figure*}[!htb]
\centering
\includegraphics[width=5.5in]{figure/network}
\vspace{0mm}
\caption{Extend the Siamese network architecture for sentence matching by feeding into the multi-scale representations of sentence pairs.}
\label{fig:network}
\vspace{-2mm}
\end{figure*}


Our sentence factorization algorithm parses a sentence $S$ into a hierarchical factorization tree $T^f_S$, where each depth of $T^f_S$ contains the semantic units of the sentence at a different granularity.
% We proposed a new unsupervised semantic distance metric using the reordered sentence representation in depth $0$ (the root node) of $T^f_S$.
In this section, we exploit this multi-scaled representation of $S$ present in $T^f_S$ to propose a multi-scaled Siamese network architecture that can extend any existing CNN or RNN-based Siamese architectures to leverage the hierarchical representation of sentence semantics.


Fig.~\ref{fig:network} (a) shows the network architecture of the popular Siamese ``matching-aggregation'' framework \cite{wang2016compare,mueller2016siamese,severyn2015learning,neculoiu2016learning,baudivs2016sentence} for sentence matching tasks. The matching process is usually performed as follows: First, the sequence of word embeddings in two sentences will be encoded by a context representation layer, which usually contains one or multiple layers of LSTM, bi-directional LSTM (BiLSTM), or CNN with max pooling layers.
The goal is to capture the contextual information of each sentence into a context vector. 
In a Siamese network, every sentence is encoded by the same context representation layer.
%The left part and the right part for a pair of sentences share the same context representation layer.
Second, the context vectors of two sentences will be concatenated in the aggregation layer. They may be further transformed by more layers of neural network
%one or a few neural network layers (such as LSTM) 
to get a fixed length matching vector.
Finally, a prediction layer will take in the matching vector and outputs a similarity score for the two sentences or the probability distribution over different sentence-pair relationships.


Compared with the typical Siamese network shown in Fig.~\ref{fig:network} (a), our proposed architecture shown in Fig.~\ref{fig:network} (b) differs in two aspects.
First, our network contains three Siamese sub-modules that are similar to (a). They correspond to the factorized representations from depth $0$ (the root layer) to depth $2$. We only select the semantic units from the top $3$ depths of the factorization tree as our input, because usually most semantic units at depth $2$ are already single words and can not be further factorized. Second, for each Siamese sub-module in our network architecture, the input is not the embedding vectors of words from the original sentences. Instead, we use semantic units at different depths of sentence factorization tree 
for matching.
%as the basic sentence matching unit.
We sum up the embedding vectors of the words contained in a semantic unit to represent that unit. Assuming each semantic unit at depth $d$ can be factorized into $k$ semantic sub-units at depth $d + 1$. If a semantic unit has less than $k$ sub-units, we add empty units as its child node to make each non-leaf node in a factorization tree has exactly $k$ child nodes. The empty units are embedded with a vector of zeros. After this procedure, the number of semantic units at depth $d$ of a sentence factorization tree is $k^d$.

Taking Fig.~\ref{fig:casestudy} as an example. We set $k = 4$ in Fig.~\ref{fig:casestudy}. For sentence A ``The little Jerry is being chased by Tom in the big yard'', the input at depth $0$ is the sum of word embedding $\{$chase, Tom, Jerry, little, yard, big$\}$. The input at depth $1$ are the embedding vectors of four semantic units: 
$\{$chase, Tome, Jerry little, yard big$\}$. Finally, at depth $2$, the semantic units are $\{$chase, -, -, -, Tom, -, -, -, Jerry, little, -, -, yard, big, -, -$\}$, where ``$-$'' denotes an empty unit.

As we can see, based on this factorized sentence representation, our network architecture explicitly matches a pair of sentences at several semantic granularities. 
%In addition, we align the semantic units in two sentences by mapping the position of semantic units in the tree to its input index in the input layer of neural network. 
In addition, we align the semantic units in two sentences by mapping their positions in the tree to the corresponding indices in the input layer of the neural network.
For example, as shown in Fig.~\ref{fig:casestudy}, the semantic units at depth $2$ are aligned according to their unit indices: ``chase'' matches with ``catch'', ``Tom'' matches with ``cat blue'', ``Jerry little'' matches with ``mouse brown'', and ``yard big'' matches with ``forecourt''.




%!TEX root = main.tex
\section{Evaluation}
\label{sec:eval}

In this section, we evaluate the performance of our unsupervised Ordered Word Mover's Distance metric and supervised Multi-scale Sentence Matching model with factorized sentences as input. We apply our algorithms to semantic textual similarity estimation tasks and sentence pair paraphrase identification tasks, based on four datasets: STSbenchmark, SICK, MSRP and MSRvid. 

\subsection{Experimental Setup}
\label{subsec:setup}


\begin{table}[tb]
  \caption{Description of evaluation datasets.}
  \label{tab:datasets}
  \begin{tabular}{lllll}
    \toprule
    Dataset & Task & Train & Dev & Test\\
    \midrule
    STSbenchmark & Similarity scoring & $5748$ & $1500$ & $1378$ \\
    SICK & Similarity scoring & $4500$ & $500$ & $4927$ \\
    MSRP & Paraphrase identification & $4076$ & - & $1725$ \\
    MSRvid & Similarity scoring & $750$ & - & $750$ \\
    \bottomrule
  \end{tabular}
  \vspace{-2mm}
\end{table}

We will start with a brief description for each dataset:
\begin{itemize}
\item \textbf{STSbenchmark}\cite{cer2017semeval}: it is a dataset for semantic textual similarity (STS) estimation. The task is to assign a similarity score to each sentence pair on a scale of 0.0 to 5.0, with 5.0 being the most similar.

\item \textbf{SICK}\cite{marelli2014sick}: it is another STS dataset from the SemEval 2014 task 1. It has the same scoring mechanism as STSbenchmark, where 0.0 represents the least amount of relatedness and 5.0 represents the most.

\item \textbf{MSRvid}: the Microsoft Research Video Description Corpus contains 1500 sentences that are concise summaries on the content of a short video. Each pair of sentences is also assigned a semantic similarity score between 0.0 and 5.0. 

\item \textbf{MSRP}\cite{quirk2004monolingual}: the Microsoft Research Paraphrase Corpus is a set of 5800 sentence pairs collected from news articles on the Internet. Each sentence pair is labeled 0 or 1, with 1 indicating that the two sentences are paraphrases of each other.
\end{itemize}

Table \ref{tab:datasets} shows a detailed breakdown of the datasets used in evaluation.
For STSbenchmark dataset we use the provided train/dev/test split.
The SICK dataset does not provide development set out of the box, so we extracted 500 instances from the training set as the development set.
For MSRP and MSRvid, since their sizes are relatively small to begin with, we did not create any development set for them.

One metric we used to evaluate the performance of our proposed models on the task of semantic textual similarity estimation is the Pearson Correlation coefficient, commonly denoted by $r$. Pearson Correlation is defined as:
\begin{equation}
\label{eq:pearson}
 r = cov(X,Y) /( \sigma_X \sigma_Y),
\end{equation}
where $cov(X,Y)$ is the co-variance between distributions X and Y, and $\sigma_X$, $\sigma_Y$ are the standard deviations of X and Y.
The Pearson Correlation coefficient can be thought as a measure of how well two distributions fit on a straight line. Its value has range [-1, 1], where a value of 1 indicates that data points from two distribution lie on the same line with a positive slope.
% Due to this unique property, we believe the Pearson Correlation coefficient is a strong indicator of the performance of our metric. 

Another metric we utilized is the Spearman's Rank Correlation coefficient. Commonly denoted by $r_s$, the Spearman's Rank Correlation coefficient shares a similar mathematical expression with the Pearson Correlation coefficient, but it is applied to ranked variables.
Formally it is defined as \cite{wiki:spearman}:
\begin{equation}
\label{eq:spearman}
 \rho = cov(rg_X, rg_Y) / (\sigma_{rg_X} \sigma_{rg_Y}),
\end{equation}
where $rg_X$, $rg_Y$ denotes the ranked variables derived from $X$ and $Y$. $cov(rg_X,rg_Y)$, $\sigma_{rg_X}$, $\sigma_{rg_Y}$ corresponds to the co-variance and standard deviations of the rank variables. The term ranked simply means that each instance in X is ranked higher or lower against every other instances in X and the same for Y. We then compare the rank values of X and Y with \ref{eq:spearman}. Like the Pearson Correlation coefficient, the Spearman's Rank Correlation coefficient has an output range of [-1, 1], and it measures the monotonic relationship between X and Y. A Spearman's Rank Correlation value of 1 implies that as X increases, Y is guaranteed to increase as well.
The Spearman's Rank Correlation is also less sensitive to noise created by outliers compared to the Pearson Correlation.

For the task of paraphrase identification, the classification accuracy of label $1$ and the F1 score are used as metrics. 

In the supervised learning portion, we conduct the experiments on the aforementioned four datasets. We use training sets to train the models, development set to tune the hyper-parameters and each test set is only used once in the final evaluation. For datasets without any development set, we will use cross-validation in the training process to prevent overfitting, that is, use $10\%$ of the training data for validation and the rest is used in training. For each model, we carry out training for 10 epochs. We then choose the model with the best validation performance to be evaluated on the test set.  


\subsection{Unsupervised Matching with OWMD}
\label{subsec:eval-owmd}

To evaluate the effectiveness of our Ordered Word Mover's Distance metric, we first take an unsupervised approach towards the similarity estimation task on the STSbenchmark, SICK and MSRvid datasets. Using the distance metrics listed in Table \ref{tab:compare-pearson} and \ref{tab:compare-spearman}, we first computed the distance between two sentences, then calculated the Pearson Correlation coefficients and the Spearman's Rank Correlation coefficients between all pair's distances and their labeled scores. We did not use the MSRP dataset since it is a binary classification problem.


In our proposed Ordered Word Mover's Distance metric, distance between two sentences is calculated using the order preserving Word Mover's Distance algorithm. For all three datasets, we performed hyper-parameter tuning using the training set and calculated the Pearson Correlation coefficients on the test and development set. We found that for the STSbenchmark dataset, setting $\lambda_1=10$, $\lambda_2=0.03$ produces the most optimal result. For the SICK dataset, a combination of $\lambda_1=3.5$, $\lambda_2=0.015$ works best. And for the MSRvid dataset, the highest Pearson Correlation is attained when $\lambda_1=0.01$, $\lambda_2=0.02$.
We maintain a max iteration of 20 since in our experiments we found that it is sufficient for the correlation result to converge.
During hyper-parameter tuning we discovered that using the Euclidean metric along with $\sigma=10$ produces better results, so all OWMD results summarized in Table \ref{tab:compare-pearson} and \ref{tab:compare-spearman} are acquired under these parameter settings. Finally, it is worth mentioning that our OWMD metric calculates the distances using factorized versions of sentences, while all other metrics use the original sentences. Sentence factorization is a necessary preprocessing step for the OWMD metric.


We compared the performance of Ordered Word Mover's Distance metric with the following methods:

\begin{itemize}
\item \textbf{Bag-of-Words (BoW)}: in the Bag-of-Words metric, distance between two sentences is computed as the cosine similarity between the word counts of the sentences.

\item \textbf{LexVec}~\cite{salle2016enhancing}: calculate the cosine similarity between the  averaged 300-dimensional LexVec word embedding of the two sentences. 

\item \textbf{GloVe}~\cite{pennington2014glove}: calculate the cosine similarity between the averaged 300-dimensional GloVe 6B word embedding of the two sentences. 

\item \textbf{Fastext}~\cite{joulin2016bag}: calculate the cosine similarity between the averaged 300-dimensional Fastext word embedding of the two sentences. 

\item \textbf{Word2vec}~\cite{mikolov2013efficient}: calculate the cosine similarity between the averaged 300-dimensional Word2vec word embedding of the two sentences.

\item \textbf{Word Mover's Distance (WMD)}~\cite{kusner2015word}: estimating the semantic distance between two sentences by WMD introduced in Sec.~\ref{sec:owmd}.
\end{itemize} 


\begin{table}[tb]
  \caption{Pearson Correlation results on different distance metrics.}
  \label{tab:compare-pearson}
  \begin{tabular}{c|cc|cc|c}
    \toprule
    \multirow{2}{*}{Algorithm} & \multicolumn{2}{c}{STSbenchmark} & \multicolumn{2}{c}{SICK} & MSRvid\\ 
     & Test & Dev & Test & Dev & Test\\
    \midrule
    BoW & $0.5705$ & $0.6561$ & $0.6114$ & $0.6087$ & $0.5044$ \\
    LexVec & $0.5759$ & $0.6852$ & $0.6948$ & $\mathbf{0.6811}$ & $0.7318$\\
    GloVe & $0.4064$ & $0.5207$ & $0.6297$ & $0.5892$  & $0.5481$ \\
    Fastext & $0.5079$ & $0.6247$ & $0.6517$ & $0.6421$  & $0.5517$  \\
    Word2vec & $0.5550$ & $0.6911$ & $\mathbf{0.7021}$ & $0.6730$  & $0.7209$  \\
    WMD & $0.4241$ & $0.5679$ & $0.5962$ & $0.5953$  & $0.3430$  \\
    OWMD & $\mathbf{0.6144}$ & $\mathbf{0.7240}$ & $0.6797$ & $0.6772$  & $\mathbf{0.7519}$  \\
    \bottomrule
  \end{tabular}
  \vspace{-4mm}
\end{table}

\begin{table}[tb]
  \caption{Spearman's Rank Correlation results on different distance metrics.}
  \label{tab:compare-spearman}
  \begin{tabular}{c|cc|cc|c}
    \toprule
    \multirow{2}{*}{Algorithm} & \multicolumn{2}{c}{STSbenchmark} & \multicolumn{2}{c}{SICK} & MSRvid\\ 
     & Test & Dev & Test & Dev & Test\\
    \midrule
    BoW & $0.5592$ & $0.6572$ & $0.5727$ & $0.5894$ & $0.5233$ \\
    LexVec & $0.5472$ & $0.7032$ & $0.5872$ & $0.5879$ & $0.7311$\\
    GloVe & $0.4268$ & $0.5862$ & $0.5505$ & $0.5490$  & $0.5828$ \\
    Fastext & $0.4874$ & $0.6424$ & $0.5739$ & $0.5941$  & $0.5634$  \\
    Word2vec & $0.5184$ & $0.7021$ & $0.6082$ & $0.6056$  & $0.7175$  \\
    WMD & $0.4270$ & $0.5781$ & $0.5488$ & $0.5612$  & $0.3699$  \\
    OWMD & $\mathbf{0.5855}$ & $\mathbf{0.7253}$ & $\mathbf{0.6133}$ & $\mathbf{0.6188}$  & $\mathbf{0.7543}$  \\
    \bottomrule
  \end{tabular}
  \vspace{-2mm}
\end{table}


Table \ref{tab:compare-pearson} and Table \ref{tab:compare-spearman} compare the performance of different metrics in terms of the Pearson Correlation coefficients and the Spearman's Rank Correlation coefficients.
We can see that the result of our OWMD metric achieves the best performance on all the datasets in terms of the Spearman's Rank Correlation coefficients.
It also produced the best Pearson Correlation results on the STSbenchmark and the MSRvid dataset, while the performance on SICK datasets are close to the best.
This can be attributed to the two characteristics of OWMD. First, the input sentence is re-organized into a predicate-argument structure using the sentence factorization tree. Therefore, corresponding semantic units in the two sentences will be aligned roughly in order. Second, our OWMD metric takes word positions into consideration and penalizes disordered matches. Therefore, it will produce less mismatches compared with the WMD metric.

% On the SICK dataset, although the result of our metric falls slightly behind Word2vec, LexVec on the test set and Word2vec on the development set, we still believe that it is a superior metric because it produced competitive results across multiple datasets. 

% Table \ref{tab:compare-spearman} presents the Spearman's Rank Correlation coefficients acquired with the same distance metrics. We can observe that our OWMD metric achieves the highest correlation scores on all three datasets. Which proves once again that OWMD is a better distance metric for the task of semantic similarity detection.

\subsection{Supervised Multi-scale Semantic Matching}
\label{subsec:eval-multilayer}

\begin{table*}[tb]
  \caption{A comparison among different supervised learning models in terms of accuracy, F1 score, Pearson's $r$ and Spearman's $\rho$ on various test sets.}
  \label{tab:sts}
  \begin{tabular}{c|cc|cc|cc|cc}
    \toprule
    \multirow{2}{*}{Model} & \multicolumn{2}{c}{MSRP} & \multicolumn{2}{c}{SICK} & \multicolumn{2}{c}{MSRvid} & \multicolumn{2}{c}{STSbenchmark}\\ 
     & Acc.(\%) & F1(\%) & $r$ & $\rho$ & $r$ & $\rho$ & $r$ & $\rho$ \\
    \midrule
    MaLSTM & $66.95$ & $73.95$ & $0.7824$ & $0.71843$ & $0.7325$ & $0.7193$ & $0.5739$ & $0.5558$\\
    Multi-scale MaLSTM & $\mathbf{74.09}$ & $\mathbf{82.18}$ & $\mathbf{0.8168}$ & $\mathbf{0.74226}$ & $\mathbf{0.8236}$ & $\mathbf{0.8188}$ & $\mathbf{0.6839}$ & $\mathbf{0.6575}$\\
    \midrule
    HCTI & $73.80$ & $80.85$ & $0.8408$ & $0.7698$ & $\mathbf{0.8848}$ & $\mathbf{0.8763}$  & $\mathbf{0.7697}$ & $\mathbf{0.7549}$ \\
    Multi-scale HCTI & $\mathbf{74.03}$ & $\mathbf{81.76}$ & $\mathbf{0.8437}$ & $\mathbf{0.7729}$ & $0.8763$ & $0.8686$  & $0.7269$ & $0.7033$  \\
    \bottomrule
  \end{tabular}
  \vspace{-2mm}
\end{table*}

The use of sentence factorization can improve both existing unsupervised metrics and existing supervised models. 
% We extend the normal Siamese model to Fig. \ref{fig:network} to take advantage of different level of information in the factorized sentence. 
To evaluate how the performance of existing Siamese neural networks can be improved by our sentence factorization technique and the multi-scale Siamese architecture, we implemented two types of Siamese sentence matching models, HCTI \cite{mueller2016siamese} and MaLSTM \cite{shao2017hcti}. HCTI is a Convolutional Neural Network (CNN) based Siamese model, which achieves the best Pearson Correlation coefficient on STSbenchmark dataset in SemEval2017 competition (compared with all the other neural network models). MaLSTM is a Siamese adaptation of the Long Short-Term Memory (LSTM) network for learning sentence similarity. As the source code of HCTI is not released in public, we implemented it according to \cite{shao2017hcti} by Keras \cite{chollet2015keras}. With the same parameter settings listed in paper \cite{shao2017hcti} and tried our best to optimize the model, we got a Pearson correlation of 0.7697 (0.7833 in paper \cite{shao2017hcti}) in STSbencmark test dataset.

We extended HCTI and MaLSTM to our proposed Siamese architecture in Fig. \ref{fig:network}, namely the Multi-scale MaLSTM and the Multi-scale HCTI. To evaluate the performance of our models, the experiment is conducted on two tasks: 1) semantic textual similarity estimation based on the STSbenchmark, MSRvid, and SICK2014 datasets; 2) paraphrase identification based on the MSRP dataset.

Table \ref{tab:sts} shows the results of HCTI, MaLSTM and our multi-scale models on different datasets. Compared with the original models, our models with multi-scale semantic units of the input sentences as network inputs significantly improved the performance on most datasets. 
Furthermore, the improvements on different tasks and datasets also proved the general applicability of our proposed architecture.

Compared with MaLSTM, our multi-scaled Siamese models with factorized sentences as input perform much better on each dataset. For MSRvid and STSbenmark dataset, both Pearson's $r$ and Spearman's $\rho$ increase about $10\%$ with Multi-scale MaLSTM. Moreover, the Multi-scale MaLSTM achieves the highest accuracy and F1 score on the MSRP dataset compared with other models listed in Table \ref{tab:sts}.

There are two reasons why our Multi-scale MaLSTM significantly outperforms MaLSTM model. First, for an input sentence pair, 
we explicitly model their semantic units with the factorization algorithm.
%we explicitly model the different scales of semantics of them with the semantic units produced by our sentence factorization algorithm. 
Second, our multi-scaled network architecture is 
specifically designed
%specially adapted to 
for multi-scaled sentences representations. Therefore, it is able to explicitly match a pair of sentences at different granularities.

We also report the results of HCTI and Multi-scale HCTI in Table \ref{tab:sts}. For the paraphrase identification task, our model shows better accuracy and F1 score on MSRP dataset. For the semantic textual similarity estimation task, the performance varies across datasets. On the SICK dataset, the performance of Multi-scale HCTI is close to HCTI with slightly better Pearson' $r$ and Spearman's $\rho$. However, the Multi-scale HCTI is not able to outperform HCTI on MSRvid and STSbenchmark. HCTI is still the best neural network model on the STSbenchmark dataset, and the MSRvid dataset is a subset of STSbenchmark.
Although HCTI has strong performance on these two datasets, it performs worse than our model on other datasets.
% Overall, the experimental results demonstrated the superior applicability and generalizability of our proposed models.
Overall, the experimental results demonstrated the general applicability of our proposed model architecture, which performs well on various semantic matching tasks.

% \begin{table}[tb]
%   \caption{Results of Accuracy and F1 score on MSRP test dataset.}
%   \label{tab:MSRP result}
%   \begin{tabular}{lllll}
%     \toprule
%     Model & Acc.(\%) & F1(\%)  \\
%     \midrule
%     MaLSTM & $66.95$ & $73.95$ \\
%     Factorized MaLSTM & $\mathbf{74.09}$ & $\mathbf{82.18}$ \\
%     HCTI & $73.80$ & $80.85$ \\
%     Factorized HCTI & $\mathbf{74.03}$ & $\mathbf{81.76}$ \\
%     \bottomrule
%   \end{tabular}
%   \vspace{0mm}
% \end{table}


% \begin{table}[tb]
%   \caption{Results of Pearson's $r$ and Spearman's $\rho$ on SICK test dataset.}
%   \label{tab:SICK result}
%   \begin{tabular}{lllll}
%     \toprule
%     Model & r & $\rho$ \\
%     \midrule
%     MaLSTM & $0.7824$ & $0.71843$ \\
%     Factorized MaLSTM & $\mathbf{0.8168}$ & $\mathbf{0.74226}$ \\
%     HCTI & $0.8408$ & $\mathbf{0.7698}$ \\
%     Factorized HCTI & $\mathbf{0.8429}$ & $0.7676$ \\
%     \bottomrule
%   \end{tabular}
%   \vspace{0mm}
% \end{table}


% \begin{table}[tb]
%   \caption{Results of Pearson's $r$ and Spearman's $\rho$ on MSRvid test dataset.}
%   \label{tab:MSRvid result}
%   \begin{tabular}{lll}
%     \toprule
%     Model & r & $\rho$  \\
%     \midrule
%     MaLSTM & $0.7325$ & $0.7193$ \\
%     Factorized MaLSTM & $\mathbf{0.8236}$ & $\mathbf{0.8188}$ \\
%     HCTI & $\mathbf{0.8848}$ & $\mathbf{0.8763}$ \\
%     Factorized HCTI & $0.8763$ & $0.8686$ \\
%     \bottomrule
%   \end{tabular}
%   \vspace{0mm}
% \end{table}



% \begin{table}[tb]
%   \caption{Results of Pearson's $r$ and Spearman's $\rho$ on STSbenchmark test dataset.}
%   \label{tab:STSbenchmark result}
%   \begin{tabular}{lllll}
%     \toprule
%     Model & r & $\rho$ \\
%     \midrule
%     MaLSTM & $0.5739$ & $0.5558$ \\
%     Factorized MaLSTM & $\mathbf{0.6839}$ & $\mathbf{0.6575}$ \\
%     HCTI & $\mathbf{0.7697}$ & $\mathbf{0.7549}$ \\
%     Factorized HCTI & $0.7269$ & $0.7033$ \\
%     \bottomrule
%   \end{tabular}
%   \vspace{0mm}
% \end{table}




\textbf{Related work}:
% Object detection related datasets/algo in non-medical domain
% Locally labeled CXR dataset
A few CXR datasets have localized abnormality annotations \cite{shih2019augmenting,filice2020crowdsourcing,jaeger2014two} that are curated manually. These are high quality gold standard ground truth datasets but tend to be smaller in scale (< 30,000 images) and have a narrow coverage, with typically only 1-2 labels. In addition, since most labeling efforts only have abnormality semantics attached, no direct relationships with the affected anatomical locations are available. 

%MEHDI: repeated concepts from above. I am removing the following: 

%The lack of anatomic semantics in the annotation is a limitation for complex multi-modal clinical reasoning work, e.g., differential diagnosis, since clinicians often integrate information along anatomical lines, and for downstream report generation tasks, which often requires describing not only the abnormality but also correctly communicate the location of the abnormalities (and medical devices) to the receiving clinicians. 

Two recent CXR datasets have labels for anatomies described in the reports. In \cite{datta2020dataset}, a small manually annotated dataset (2000 reports) included 10 abnormalities that are individually associated with 29 unique spatial locations (anatomies) at the report level. Another CXR dataset has automatically extracted abnormality and anatomy labels as disconnected concepts that are only correlated at the study level from  160,000 reports using a supervised NLP algorithm \cite{bustos2020padchest}. This was trained on a smaller set of manually annotated data. Neither datasets contain localized annotations for the associated CXR images, nor any comparison relation annotations between sequential exams, both of which are available in the Chest ImaGenome dataset. In Table \ref{tab:related}, we present a comparison of our Chest ImagGenome dataset with other datasets available in the literature.

% Table -- Kashyap

% MEdical imaging datasets to go here: Discussed that we will only focus on cxr datasets that are available for this paper. 
% \caption{\color{red} Kashyap, feel free to continue with the table. We should remove the questionmarks and add a line for our dataset (since all others are not graph). For longer text, using abbreviations and explaining them in the caption often works better. If fill in the values is not possible, it is better to remove the table altogether.}


\begin{table}[t!]
\caption{Summary of existing chest X-ray datasets}
\resizebox{\textwidth}{!}{%
\begin{tabular}{@{}lllllllll@{}}
\toprule
\textbf{Dataset} & \textbf{Annotation Level} & \textbf{Annotation Method} & \textbf{Num Labels} & \textbf{Anatomy Labeled} & \textbf{Graph} & \textbf{Dataset Size} & \textbf{Temporal Labels} & \textbf{Reports} \\ \midrule
SIIM-ACR Pneumothorax Segmentation \cite{filice2020crowdsourcing} & Segmentation & Manual + augmented & 1 & No & No & 12,047 & No & No \\
RSNA Pneumonia Detection Challenge   \cite{shih2019augmenting} & Bounding Boxes & Manual & 1 & No & No & 30,000 & No & No \\
Indiana University Chest X-ray collection \cite{demner2016preparing} & Global & Automated & 10 & No & No & 3,813 & No & Yes \\
NIH CXR dataset \cite{wang2017chestx} & Global & Automated & 14 & No & No & 112,120 & No & No \\
PLCO \cite{team2000prostate} & Global & Automated & 24 & Yes & No & 236,000 & Yes & No \\
Stanford CheXpert \cite{irvin2019chexpert} & Global & Automated & 14 & No & No & 224,316 & No & No \\
MIMIC-CXR \cite{johnson2019mimic} & Global & Automated & 14 & No & No & 377,110 & No & Yes \\
Dutta \cite{datta2020dataset} & Global & Manual & 10 & Yes & Yes & 2,000 & No & Yes \\
PadChest \cite{bustos2020padchest} & Global & Manual + automated & 297 & Yes & No & 160,868 & No & Yes \\
Montgomery County Chest X-ray   \cite{jaeger2014two} & Segmentation & Manual & 1 & Yes & No & 138 & No & No \\
Shenzen Hospital Chest X-ray   \cite{jaeger2014two} & Segmentation & Manual & 1 & Yes & No & 662 & No & No \\  \hline \hline
\textbf{Chest ImaGenome} & Bounding Boxes & Automated & 131 & Yes & Yes & 242,072 & Yes & Yes \\
\bottomrule
\end{tabular}%
}
\label{tab:related}
\vspace{-0.4cm}
\end{table}
% removed (Derived from MIMIC-CXR \cite{johnson2019mimic}) % makes table really small


\begin{comment}
\begin{figure}
\includegraphics[width=\linewidth]{figs/beyond_tss_lesion.pdf}
\caption[]{End-to-End runtime lesion study of the entire MNIST dataset and the FMA featurized music dataset. Each of DROP's contributions provides a runtime improvement.}
\label{fig:beyond_lesion}
\end{figure}
\end{comment}



\section{Conclusion}
\label{sec:conclusion}

Advanced data analytics techniques must scale to rising data volumes. 
DR techniques offer a powerful toolkit when processing these datasets, with PCA frequently outperforming popular techniques in exchange for high computational cost. 
In response, we propose DROP, a new dimensionality reduction optimizer. 
DROP combines progressive sampling, progress estimation, and online aggregation to identify high quality low dimensional bases via PCA without processing the entire dataset by balancing the runtime of downstream tasks and achieved dimensionality. 
Thus, DROP provides a first step in bridging the gap between quality and efficiency in end-to-end DR for downstream \red{analytics}. 

%We revisit canonical operators for time series dimensionality reduction and the measurement study of~\cite{keogh-study}, and show that PCA is more effective than popular alternatives in the data mining literature often by a margin of over $2\times$ on average on gold-standard time series benchmark data sets with respect to output data dimension. More surprisingly, we empirically demonstrate that a small number of samples are sufficient to accurately characterize directions of maximum variance and obtain a high-quality low-dimensional transformation.




\bibliographystyle{ACM-Reference-Format}
\balance
\bibliography{main} 

\end{document}
