\documentclass[11pt]{article}
\usepackage[utf8]{inputenc}
\usepackage{blindtext}
\usepackage{soul}
\usepackage{latexsym} % special \LaTeX symbols
\usepackage{amsmath,amssymb,amsfonts,amsthm} % AMS math
\usepackage{graphics,psfrag} % graphics support 
\usepackage{fancyhdr,lastpage} % for making nice headers
\usepackage{color,verbatim} % text markup
\usepackage[hidelinks]{hyperref} % linking and navigation
\usepackage{hyperref} % linking and navigation
\usepackage[table]{xcolor} % make colored tables
\usepackage{metalogo}
\usepackage{etoolbox}
\usepackage{cite}
\usepackage{mathrsfs}
\usepackage[affil-it]{authblk}

\usepackage{xfrac}
\usepackage{bbm}
\patchcmd{\thebibliography}{\section*{\refname}}{}{}{}

\newcommand{\mathcolorbox}[2]{\colorbox{#1}{$\displaystyle #2$}}
\newtheorem{theorem}{Theorem}[section]
\newtheorem{corollary}{Corollary}[theorem]
\newtheorem{lemma}[theorem]{Lemma}
\newtheorem{assumption}[theorem]{Assumption}
\newtheorem{definition}[theorem]{Definition}
\newtheorem{example}[theorem]{Example}
\newtheorem{remark}[theorem]{Remark}
\newtheorem{conjecture}[theorem]{Conjecture}

\hypersetup{
    colorlinks=false,
    pdfborder={0 0 0},
}

\usepackage{calc}

% general formatting of pages
\usepackage[a4paper,top=1in,bottom=0.8in,right=0.8in,left=0.8in]{geometry}
\setlength{\headheight}{13.6pt}

% line spacing
\renewcommand{\baselinestretch}{1.0}

% squeeze the bibliography a bit, should be a better way and a way to get 
% rid of some more white spaces between [...] ____ Ref
\let\oldbibliography\thebibliography
\renewcommand{\thebibliography}[1]{%
  \oldbibliography{#1}%
  \setlength{\itemsep}{-1.5mm}%
}

% common notation
\def\R{\mathbb{R}}
\def\C{\mathbb{C}}
\def\N{\mathbb{N}}
\def\P{\mathbb{P}}
\def\Z{\mathbb{Z}}
\def\E{\mathbb{E}}
\def\I{\infty}

% text letters in mathmode
\def\txtb{{\textnormal{b}}}
\def\txtc{{\textnormal{c}}}
\def\txtd{{\textnormal{d}}}
\def\txte{{\textnormal{e}}}
\def\txti{{\textnormal{i}}}
\def\txts{{\textnormal{s}}}
\def\txtu{{\textnormal{u}}}
\def\txtD{{\textnormal{D}}}

% equation abbreviations
\newcommand{\be}{\begin{equation}}
\newcommand{\ee}{\end{equation}}
\newcommand{\bea}{\begin{eqnarray}}
\newcommand{\eea}{\end{eqnarray}}
\newcommand{\beann}{\begin{eqnarray*}}
\newcommand{\eeann}{\end{eqnarray*}}
\newcommand{\benn}{\begin{equation*}}
\newcommand{\eenn}{\end{equation*}}

% special math shortcuts
\def\ra{\rightarrow}
\def\I{\infty}

% Calligraphic letters
\newcommand{\cA}{{\mathcal A}}  % calligraphic A
\newcommand{\cB}{{\mathcal B}}  % calligraphic B
\newcommand{\cC}{{\mathcal C}}  % calligraphic C
\newcommand{\cD}{{\mathcal D}}  % calligraphic D
\newcommand{\cE}{{\mathcal E}}  % calligraphic E
\newcommand{\cF}{{\mathcal F}}  % calligraphic F
\newcommand{\cG}{{\mathcal G}}  % calligraphic G
\newcommand{\cH}{{\mathcal H}}  % calligraphic H
\newcommand{\cI}{{\mathcal I}}  % calligraphic J
\newcommand{\cJ}{{\mathcal J}}  % calligraphic J
\newcommand{\cK}{{\mathcal K}}  % calligraphic K
\newcommand{\cL}{{\mathcal L}}  % calligraphic L
\newcommand{\cM}{{\mathcal M}}  % calligraphic M
\newcommand{\cN}{{\mathcal N}}  % calligraphic N
\newcommand{\cO}{{\mathcal O}}  % calligraphic O
\newcommand{\cP}{{\mathcal P}}  % calligraphic P
\newcommand{\cQ}{{\mathcal Q}}  % calligraphic Q
\newcommand{\cR}{{\mathcal R}}  % calligraphic R
\newcommand{\cS}{{\mathcal S}}  % calligraphic S
\newcommand{\cT}{{\mathcal T}}  % calligraphic T
\newcommand{\cU}{{\mathcal U}}  % calligraphic U
\newcommand{\cV}{{\mathcal V}}  % calligraphic V
\newcommand{\cW}{{\mathcal W}}  % calligraphic W
\newcommand{\cX}{{\mathcal X}}  % calligraphic X
\newcommand{\cY}{{\mathcal Y}}  % calligraphic Y
\newcommand{\cZ}{{\mathcal Z}}  % calligraphic Z

% make color comments for participants

% Comment color for Giulio Zucal
\newcommand{\CK}[1]{\color{red} #1 \color{black} }
\newcommand{\CKcomment}[1]{\color{red} [#1] \color{black} }
\newcommand{\james}[1]{\color{green} #1 \color{black} }

% erase indent
\setlength\parindent{0pt}

\fancypagestyle{plain}{%
  \renewcommand{\headrulewidth}{0pt}%
  \fancyhf{}%
  \fancyfoot[C]{\footnotesize Page \thepage\ of \pageref{LastPage}}%
}

%%%%%%%%%%%%%%%%% Start of the real document... %%%%%%%%%%%%%%%%%%%%%%%%%%%%%%%%%


\title{Action convergence of general hypergraphs and tensors}
\author{Giulio Zucal\thanks{giulio.zucal@mis.mpg.de}}
\affil[1]{Max Planck Institute for Mathematics in the Sciences, Leipzig, Germany}


\date{\today}

\begin{document}
\maketitle

\begin{abstract}


Action convergence provides a limit theory for linear bounded operators $A_n:L^{\infty}(\Omega_n)\longrightarrow L^1(\Omega_n)$ where $\Omega_n$ are potentially different probability spaces. This notion of convergence emerged in graph limits theory as it unifies and generalizes many notions of graph limits. We generalize the theory of action convergence to sequences of multi-linear bounded operators $A_n:L^{\infty}(\Omega_n)\times \ldots \times L^{\infty}(\Omega_n)\longrightarrow L^1(\Omega_n)$. Similarly to the linear case, we obtain that for a uniformly bounded (under an appropriate norm) sequence of multi-linear operators there exists an action convergent subsequence. Additionally, we explain how to associate different types of multi-linear operators to a tensor and we study the different notions of convergence that we obtain for tensors and in particular for adjacency tensors of hypergraphs. We obtain several hypergraphs convergence notions and we link these with the gerarchy of notions of quasirandomness for hypergraph sequences. This convergence covers also sparse and inhomogeneous hypergraph sequences and it preserves many properties of adjacency tensors of hypergraphs. Moreover, we explain how to obtain a meaningful convergence for sequences of non-uniform hypergraphs and, therefore, also for simplicial complexes. Additionally, we highlight many connections with the theory of  dense uniform hypergraph limits (hypergraphons) and we conjecture the equivalence of this theory with a special case of multi-linear action convergence. 


    \vspace{0.2cm}
\noindent {\bf Keywords:}  Graph limits, Hypergraphs, Action convergence, Tensors, Higher-order interactions
\end{abstract}

\section{Introduction}


In the last 20 years the study of complex networks has permeated many areas of social and natural sciences. Important examples are computer, telecommunication, biological, cognitive, semantic and social networks. In particular, in all of these areas, understanding of large networks is a fundamental problem.

Network structures are usually modeled using graph theory to represent pairwise interactions between the elements of the network. However, for very large networks, as the internet, the brain and social networks among others, exact information about the number of nodes and other specific features of the underlying graph are not available. For this reason, there is the need for a mathematical definition of synthetic structures containing only the relevant information for a very large graph. This is equivalent to assuming that the number of nodes is so big that the graph can be well approximated with a “graph-like” object with infinite number of nodes. This motivated the development of graph limits theory, the study of graph sequences, their convergence and their limit objects. In mathematical terms, one is interested in finding a metric on the space of graphs and a completion of this space with respect to this metric. This is a very active field of mathematics that connects graph theory with many other mathematical areas as stochastic processes, ergodic theory, spectral theory and several branches of analysis and topology.  

From the rise of graph limits theory two different cases have been mostly considered. The first case is the limits of dense graphs, i.e.\ when the graphs considered in the sequence have asymptotically number of edges proportional to the square of the number of vertices. This case, where the limit objects are called graphons (from graph functions), is now very well understood thanks to the contributions of L.\ Lovász, B.\ Szegedy, C.\ Borgs and J.\ Chayes among others  \cite{BORGS20081801,Lovsz2007SzemerdisLF, LOVASZ2006933}. The dense graph limit convergence is metrized by the so called “cut-metric” and is equivalent to the convergence of homomorphism densities. The completion of the set of all graphs in this metric is compact, i.e. every graph sequence has a convergent sub-sequence, which is a very useful property. A shortcoming of the dense graph limit theory is that it has not enough expressive power to study graphs in which the number of edges is sub-quadratic in the number of vertices. In fact, every sparse graph is considered to be similar to the empty graph in this metric. An important generalization of this theory are $L^p-$graphons \cite{LpGraphons1,Lpgraphon2}. The other case that has been well studied are graph sequences with uniformly bounded degree and the associated notion of convergence was introduced by I.\ Benjamini and O.\ Schramm \cite{BenjaminiLimit} and it has a stronger version called local-global convergence \cite{local-global1,Hatami2014LimitsOL}. The limits of such convergent sequences can be represented as objects called graphings. For a complete treatment of these topics see the monograph by L.\ Lovász \cite{LovaszGraphLimits}. 

Unfortunately, for most applications, the really interesting case is the intermediate degree case, not covered by the previously presented theories. Real networks are usually sparse but not very sparse and heterogeneous. For this reason the intermediate case attracted a lot of attention recently, see for example \cite{MarkovSpaces,KUNSZENTIKOVACS20191,Kunszenti_Kov_cs_2022}.  In particular, in a recent work Á.\ Backhausz and B.\ Szegedy introduced a new functional analytic/measure theoretic notion of convergence \cite{backhausz2018action}, that not only covers the intermediate degree case but also unifies the graph limit theories previously presented. This notion of convergence is called action convergence and the limit objects for graph sequences are called graphops (from graph operators). More generally this is a notion of convergence for $P-$operators, i.e.\ linear bounded operators
$$
A:L^{\infty}(\Omega)\longrightarrow L^1(\Omega)
$$
where $\Omega$ is a probability space. As a matrix can be naturally interpreted as a $P-$operator, we obtain as a special case a notion of convergence for matrices. The notions of convergence for graphs are derived associating to graphs (properly normalized) matrices, for example, adjacency matrices or Laplacian matrices. 
In this work we extend the notion of action convergence to multi-linear operators. More specifically, we consider multi-linear operators of the form 
$$
A:L^{\infty}(\Omega)^r\longrightarrow L^1(\Omega)
$$
where $\Omega$ is a probability space and $L^{\infty}(\Omega)^r=L^{\infty}(\Omega)\times\ldots\times L^{\infty}(\Omega)$ is the cartesian product of $L^{\infty}(\Omega)$ with itself $r$ times. We name such operators multi-$P-$operators. 

This convergence notion comes with an associated pseudo-metric $d_M$. We therefore say that two multi-$P-$operators $A$ and $B$ are isomorphic if $d_M(A,B)=0$. The space of classes of isomorphism of multi-$P-$operators equipped with $d_M$ is a metric space.

We obtain a compactness result for multi-$P-$operators analogous to the compactness result for the case of $P-$operators: Sequences of multi-$P-$operators $(A_n)_{n}$ that have a uniform bound $C>0$ on the quantity
$$
\|A_n\|_{p_1,\ldots,p_{r}\rightarrow q}=\sup_{0\neq f_1,\ldots, f_r\in L^\infty(\Omega_n)}\frac{\|A_n[f_1,\ldots,f_r]\|_q}{\|f_1\|_{p_1}\ldots \|f_r\|_{p_r}}\leq C
$$
for all $n\in \mathbb{N}$ have a convergent subsequence in the space of isomorphism classes of multi-$P-$operators equipped with the metric $d_M$.  Moreover,
$$
\|A\|_{p_1,\ldots,p_{r}\rightarrow q}\leq\lim_{n\rightarrow \infty}\|A_n\|_{p_1,\ldots,p_{r}\rightarrow q}\leq C.
$$
if the sequence is convergent with limit multi-$P-$operator $A$. 

We focus on using multi-linear action convergence to define meaningful convergence notions for tensors and hypergraphs.

\begin{definition}
Let $r,n\geq 2$. An $r$-th order $n$-dimensional \emph{tensor} $T$ consists of $n^r$ entries
\begin{equation*}
    T_{i_1,\ldots,i_r}\in \mathbb{R},
\end{equation*}
where $i_1,\dots,i_r\in[n]$.%\newline Given $i\in [n]$, we define the \emph{$i$-th row} of $T$ as the $(r-1)$-order $n$-dimensional tensor $T_i$ obtained by setting the first index of $T$ equal to $i$.
\newline The tensor $T$ is \emph{symmetric} if its entries are invariant under any permutation of their indices.\newline
\end{definition}
First of all, we explain how symmetric tensors can be associated with multi-$P-$operators in multiple ways. For example, for a $3-$rd order symmetric tensor 
$$
T_{i_1,i_2, i_3}
$$
we can consider the operator
$$
T_1[\mathrm{v},\mathrm{w}]=\sum^n_{i_1,i_2=1}T_{i_1,i_2, i_3}v_{i_1}w_{i_2}
$$
where $\mathrm{v}=(v_i)_i,\mathrm{w}=(w_i)_i\in \R^n$ are vectors or alternatively

$$
T_2[f,g]=\frac{1}{2}(\sum^n_{i_2=1}T_{i_1,i_2,i_3}f_{i_1,i_2}g_{i_2,i_3} +\sum^n_{i_2=1}T_{i_1,i_2,i_3}f_{i_3,i_2}g_{i_2,i_1})
$$
where $f=(f_{ij})_{ij},g=(g_{ij})_{ij}$ are symmetric matrices. These different choices of associating a multi-$P-$operator to a tensor give rise in general to different convergence notions for tensors. In the case of $3-$rd order symmetric tensors the second choice presented seems to be in many cases more appropriate. However, one can require action convergence of both multi-$P-$operators $(T)_1$ and $(T)_2$ associated to the tensor $T$ at the same time.

Recently, in network sciences, a lot of interest has been generated by higher-order interactions (interactions that are beyond pairwise) and the phenomena generated by them \cite{HypergraphsNetworks,HigerOrdIntBook,carletti2020dynamical,majhi2022dynamics,MHJ,HypergraphsDynamics,Bohle_2021,JOST2019870,JostMulasBook}. Hypergraphs are the natural mathematical/combinatorial structure to represent higher-order interactions. 

\begin{definition}
 An \emph{hypergraph} is a pair $H=(V,E)$ where $V=\{v_1,\ldots,v_n\}$ is the set of \emph{vertices}, $E=\{e_1,\ldots,e_m\}$ is the set of \emph{edges} and $\emptyset \neq e \subseteq V$ for each $e\in E$.\newline
 An hypergraph $H$ is $k$\emph{-uniform} if $|e|=k$ for every $e\in E$.
 \end{definition}

Limit theories for hypergraphs are much less developed than the ones for graphs due to the bigger combinatorial complexity and for this reason very limited to the uniform and dense hypergraph sequences case. The first contributions on hypergraph limits,\cite{hypergrELEK20121731,HypergraphsSzegedy2} by Elek and Szegedy used techniques from nonstandard analysis/model theory to define uniform and dense hypergraph limit objects. This approach using “ultralimits” is well explained in the recent book \cite{RandomneLimitTow} by Towsner. A more classical approach using “quotients” and regularity partitions obtaining the same type of limits has been developed by Zhao in  \cite{HypergraphonsZhao}. The limit objects of this convergence notions appeared earlier in the context of exchangeable arrays of random variables\cite{kallenberg1992symmetries, hoover1979relations, aldous1981representations, aldous2010exchangeability,austin2008exchangeable,diaconis2007graph}. For sparse uniform hypergraph sequences, a removal lemma is obtained in \cite{SparseHypLogicLimit} using again techniques from logic but no limit theory/convergence for sparse hypergraphs is developed to the best of our knowledge. Our hypergraphs convergence based on multi-linear action convergence instead is based on functional analytic and measure-theoretic techniques and can be applied to any hypergraph sequence, also for non-uniform, very sparse and heterogeneous hypergraphs sequences. 

To apply action convergence we associate to hypergraphs their adjacency tensor

\begin{definition}
Let $H=(V,E)$ be a hypergraph on $n$ nodes with largest edge cardinality $r$. The \emph{adjacency tensor} of $H$ is the $r$-th order $n$-dimensional tensor $A=A(H)$ with entries
\begin{equation*}
A_{i_1,\ldots,i_r}:=\begin{cases}
0 & \text{ if }\{v_{i_1},\ldots,v_{i_r}\}\notin E\\
1 &  \text{ if }\{v_{i_1},\ldots,v_{i_r}\}\in E.\\
\end{cases}
\end{equation*}
\end{definition}

(possibly multiplied by some normalizing constant) and as already explained we can associate a tensor with different multi-$P-$operators and therefore different convergence notions with a relationship between them. These different types of convergence are related to the different types of quasi-randomness for sequences of hypergraphs. In particular, we focus our attention on one notion of convergence obtained in such a way, which we consider being in many cases the most appropriate, and we compare it with the existing notion of convergence for dense hypergraphs (hypergraphon convergence). We underline many similarities in the two theories and we look at some motivating examples that bring us to conjecture that action convergence of the normalized adjacency tensor and hypergraphon convergence are equivalent. 

The generalization of action convergence to multi-linear operators allows us to study hypergraph limits and therefore to represent conveniently large hypernetworks with objects that we will call hypergraphops. Hypergraphops are symmetric and positivity-preserving multi-$P-$operators and hypergraphs are obviously special cases of hypergraphops. We show that the space of hypergraphops is closed. In fact, symmetry and positivity of multi-$P-$operators are preserved under action convergence, i.e.\ the limit of an action convergent sequence of symmetric, respectively positivity-preserving, multi-$P-$operators is again symmetric, respectively positivity-preserving.

We also study other possible tensors associated with hypergraphs and their associated action convergence. In particular, we present possible choices to obtain meaningful limit objects for inhomogeneous and non-uniform hypergraph sequences. In particular, to the best of our knowledge, we are the first to introduce a meaningful convergence for non-uniform hypergraphs. Covering the case of non-uniform hypergraphs, our limit theory gives us a convergence for simplicial complexes as a special case answering a question in \cite{Bobrowski2022}.

%In particular, we want to contribute in developing sound mathematical foundations of action convergence, generalize it and use it in applications. Action convergence is in fact a very general notion of convergence for linear operators and this makes it very suitable for studying random graphs, random matrices and mean field limits of network dynamics.
 %For this reason defining a notion of convergence for the intermediate case is a major direction in  
Generalising the results of \cite{backhausz2018action} is technically challenging as it requires us to use multi-linear operators and tensors instead of linear operators and matrices, which is technically challenging as there are fewer results. Furthermore, this generalisation also significantly complicates the associated notation. However, on a more conceptual level, the main challenge of understanding the limit objects of hypergraphs requires a deeper understanding of Action convergence to which we contribute here.
%In generalizing the results of \cite{backhausz2018action} we encounter several challenges. First of all, working with multi-linear operators and tensors, instead of linear operators and matrices, presents several problems. A non-trivial difficulty is also presented by dealing with a much heavier notation and more complex combinatorics. However, on a more conceptual level, to understand the limit objects of hypergraphs requires much       
% This work based on


\subsection*{Structure of the paper}

In Section \ref{SecNotation} we introduce the notation and basic definitions from functional analysis and probability theory. In Section \ref{SecActionConve} we briefly recall the theory of action convergence and in Section \ref{SecTensHyp} we introduce relevant notions for hypergraphs and tensors. In Section \ref{SecMultiActConv} we finally introduce the generalization of  action convergence to multi-linear operators and in Section \ref{SecConstrLimObj} we prove the main compactness result. Moreover, in Section \ref{SecPropLimObje} we study several properties of multi-linear action convergence and of the related limit objects. In Section \ref{SecMultActTensHyp} and \ref{NonUnifHypSect} we investigate how action convergence for multi-linear operators can be specialized to tensors and hypergraphs in different ways and we study the different convergence notions obtained. In Section \ref{SectHypergraphons} we point out many relationships between the strongest version of hypergraph convergence obtained from action convergence and hypergraphon convergence in the context of dense hypergraph sequences and we conjecture the equivalence of these two notions.  

\section{Notation}\label{SecNotation}

In the following, we will denote with $(\Omega,\P,\mathcal{F})$ a standard probability space where $\mathcal{F}$ is a  $\sigma-$algebra and $\P$ is a probability measure on $(\Omega,\mathcal{F})$. We will denote with $\mathcal{P}(\Omega,\mathcal{F})$ or shortened $\mathcal{P}(\Omega)$ the set of probability measures on $(\Omega,\mathcal{F})$. Moreover, we will indicate the expectation of a real-valued measurable function (or in probabilistic language a random variable) $f$ on $(\Omega,\P,\mathcal{F})$ with $\E[f]$. We indicate the (possibly infinite) $L^p-$norm of a real-valued measurable function $f$ with  
$$
\|f\|_p=\int_{\Omega}|f(\omega)|^pd\P(\omega)=\E[|f|^p].
$$ If a measurable function $f$ has finite $L^p-$norm we say that $f$ is $p-$integrable (or has finite $p-$moment). We denote with $L^p(\Omega,\P,\mathcal{F})$ the usual Banach space of the real-valued measurable $p-$integrable functions (identified if they are equal almost everywhere) on $(\Omega,\P,\mathcal{F})$ equipped with the usual $L^p-$norm or equivalently, in probabilistic language, the space of random variables with finite $p-$moment. We will use a lot of times the shortened notations $L^p(\Omega)$ or $L^p$ when there is no risk of confusion. For a set $S\subset \R$ we will also denote with $L_S^p(\Omega)$ the space of the $p-$integrable random variables taking values in $S$.\newline 

For a linear operator 
$$\begin{aligned}
    A:L^p(\Omega, \P,\mathcal{F})&\longrightarrow L^q(\Omega,\P,\mathcal{F})\\
& f\mapsto Af
\end{aligned}$$ we define the $(p,q)-$operator norm $$
\|A\|_{p\rightarrow q}=\sup_{f\in L^p,\ f\neq 0} \frac{\|Af\|_q}{\|f\|_p}.
$$
The linear operator $A$ is said to be bounded (or equivalently continuous) if the operator norm is finite. We denote with $L(p\rightarrow q)$ the Banach space of linear bounded operators from $L^p(\Omega)$ to $L^q(\Omega)$ equipped with the $(p,q)-$operator norm.\newline
A $k-$dimensional random vector is a measurable function $\mathbf{f}$ from a probability space $(\Omega,\P,\mathcal{F})$ to $\R^k$ and we can naturally represent it as $$
\mathbf{f}=(f_1,\ldots,f_k)
,$$ where $f_1,\ldots,f_k$ are real-valued random variables on $(\Omega,\P,\mathcal{F})$. Therefore, a real-valued random variable is a $1-$random vector. For a $k-$dimensional random vector $\mathbf{f}$, we denote with $\mathcal{L}(\mathbf{f})=\mathcal{L}(f_1,\ldots,f_k)$ its distribution (or law), that is the measure on $\R^k$ defined as  
$$
\mathcal{L}(\mathbf{f})(A)=\P(\mathbf{f}^{-1}(A))
$$
where $A$ is a set in the Borel sigma-algebra of $\R^k$. \newline
Given $n\in\N$,
we denote by $[n]$ the set $\{1,\ldots,n\}$. In the case of a finite probability space, the law of a random vector has a particularly easy representation. We show this with the following example that will be important in the next sections. 
\begin{example}
Let's consider the probability space $([n],\mathcal{U},\mathcal{D})$ where we denoted with $\mathcal{U}$ the uniform probability measure on $[n]$ and with $\mathcal{B}$ the discrete sigma algebra on $[n]$. Then for any $k-$dimensional random vector $$
\mathbf{f}=(f_1,\ldots,f_k)$$
\end{example}
the law $\mathcal{L}(\mathbf{f})$ is 
$$
\mathcal{L}(\mathbf{f})=\frac{1}{n}\sum^n_{i=1}\delta_{(f_1(i),\ldots,f_k(i))}
$$
where $\delta_{(x_1,\ldots,x_k)}$ is the Dirac measure centered in $(x_1,\ldots,x_k)\in \R^k$.


\section{Action convergence}\label{SecActionConve}
We briefly recall here, following \cite{backhausz2018action}, the notion of action convergence of operators, a very general notion of convergence for operators acting on $L^p$ spaces defined on different probability spaces, introduced in the context of graph limit theory. Other related works to this limit notion are \cite{ArankaAction2022,MeasTheorActionZucal}.\newline
We start giving the following
\begin{definition}
A $P-$\emph{operator} is a linear bounded operator $$\begin{aligned}
     A:L^\infty(\Omega, \P,\mathcal{F})&\longrightarrow L^1(\Omega,\P,\mathcal{F})
\end{aligned}
$$
for any probability space $(\Omega, \P,\mathcal{F})$.\newline
A $P-$operator $A$ is acting on the probability space $(\Omega, \P,\mathcal{F})$ if the $L^1$ and $L^\infty$ spaces are defined on $(\Omega, \P,\mathcal{F})$. We denote the set of all $P-$operators with $\mathcal{B}$ and the set of all $P-$operators acting on $(\Omega, \P,\mathcal{F})$ with $\mathcal{B}(\Omega, \P,\mathcal{F})$.
\end{definition}

We give here an example that will be important in the following.
\begin{example}
A matrix $A=(A_{i,j})_{i,j\in [n]}$ can be interpreted as a $P-$operator acting on the probability space $\Omega=([n],\mathcal{U},\mathcal{D})$ where we denoted with $\mathcal{U}$ the uniform probability measure on $[n]$ and with $\mathcal{B}$ the discrete sigma-algebra on $[n]$. In fact, for $\mathrm{v}=(v_i)_{i\in [n]}\in \R^n\cong L^{\infty}(\Omega)\cong L^1(\Omega)$
$$
A:L^{\infty}(\Omega)\longrightarrow L^1(\Omega)
$$

$$
(A\mathrm{v})_i=\sum^n_{j=1}A_{ij}v_j.
$$ In particular, a graph can be associated to its adjacency matrix (or its Laplacian matrix) and therefore it can be interpreted as a $P-$operator.
\end{example}

We would now like to introduce a metric on $P-$operators possibly acting on different probability spaces. This means that we would like to equip $\mathcal{B}$ with a metric and, therefore, with a natural notion of convergence. In reality, we will equip $\mathcal{B}$ with a pseudo-metric and then quotient over equivalent classes (elements at distance $0$) of $\mathcal{B}$ to obtain a proper metric space.  We will see that for graphs (adjacency matrices of graphs) this identification of elements is exactly what we want as it identifies isomorphic graphs.\newline
By definition, an element $f$ of $L^\infty(\Omega,\P,\mathcal{F})$ is a real-valued bounded random variable on $(\Omega,\P,\mathcal{F})$. Therefore, for a $P-$operator $A$ acting on $(\Omega,\P,\mathcal{F})$ , 
$$
Af\in L^1(\Omega,\P,\mathcal{F}) $$is, by definition, a real-valued random variable with finite expectation. Therefore, for functions $f_1,\ldots,f_k\in L^{\infty}(\Omega)$ we can consider the $2k-$dimensional random vector
$$
(f_1,Af_1,\ldots , f_k,Af_k)
$$and in particular its distribution $\mathcal{L}(f_1,Af_1,\ldots , f_k,Af_k)\in \mathcal{P}(\R^{2k})$. For a $P-$operator $A$, if a measure $\mu\in\mathcal{P}(\R^{2k})$ is such that $$
\mu= \mathcal{L}(f_1,Af_1,\ldots , f_k,Af_k)
$$
for some functions $f_1,\ldots, f_k\in L^{\infty}(\Omega)$ we say that $\mu$ is a measure generated by $A$ through $f_1,\ldots, f_k$.
We now define the set of measures generated by $A$. For reasons that will be clear in the following, it will be convenient to allow in our sets only measures generated by functions in $L_{[-1,1]}^{\infty}(\Omega)$, i.e. functions taking values between $+1$ and $-1$ almost everywhere. Therefore, we define the $k-$\emph{profile} of $A$,  $S_k(A)$, as the set of measures generated by $A$ through functions in  $L_{[-1,1]}^{\infty}(\Omega)$, i.e.\
$$S_k(A)=\bigcup_{f_1,\ldots, f_k\in L_{[-1,1]}^{\infty}(\Omega)}\{\mathcal{L}(f_1,Af_1,\ldots , f_k,Af_k)\}
$$This is a set of measures. To compare sets of measures we will therefore need  a metric on the space of measures.  For this reason, we recall the following well-known metric:

\begin{definition}[Lévy-Prokhorov metric]\label{LevyProk}
 The \emph{Lévy-Prokhorov Metric} $d_{\mathrm{LP}}$ on the space of probability measures $\mathcal{P}\left(\mathbb{R}^{k}\right)$ is for $\eta_1,\eta_2\in \mathcal{P}\left(\mathbb{R}^{k}\right)$
$$\begin{aligned}
d_{\mathrm{LP}}\left(\eta_{1}, \eta_{2}\right)=&\inf \left\{\varepsilon>0: \eta_{1}(U) \leq \eta_{2}\left(U^{\varepsilon}\right)+\varepsilon \text{ and } \right.\\
&\left.\eta_{2}(U) \leq \eta_{1}\left(U^{\varepsilon}\right)+\varepsilon  \text{ for all } U \in \mathcal{B}_{k}\right\},
\end{aligned}$$

where $\mathcal{B}_{k}$ is the Borel $\sigma$-algebra on $\mathbb{R}^{k}$, $U^{\varepsilon}$ is the set of points that have Euclidean distance smaller than $\varepsilon$ from $U$.
\end{definition}

The above metric metrizes the weak/narrow convergence for measures. \newline
Now, we want to be able to compare sets of measures. We, therefore, introduce the following

\begin{definition}[Hausdorff metric]\label{HausdorffDist}
 Given $X, Y\subset \mathcal{P}\left(\mathbb{R}^{k}\right)$, their \emph{Hausdorff distance} 
$$
d_{H}(X, Y):=\max \left\{\sup _{x \in X} \inf _{y \in Y} d_{\mathrm{LP}}(x, y), \sup _{y \in Y} \inf _{x \in X} d_{\mathrm{LP}}(x, y)\right\}
$$
Note that $d_{H}(X, Y)=0$ if and only if $\operatorname{cl}(X)=\operatorname{cl}(Y)$, where $\operatorname{cl}$ is the closure in $d_{\mathrm{LP}} .$ It follows that $d_{H}$ is a pseudometric for all subsets in $\mathcal{P}\left(\mathbb{R}^{k}\right)$, and it is a metric for closed sets.
\end{definition}
By definition, the Lévy-Prokhorov distance between probability measures is upper-bounded by $1$ and, therefore, the Hausdorff metric for sets of measures is upper-bounded by $1$.\newline

We are now ready to define the pseudo-metric we are interested in.
Consider two $P-$operators
$$
A: L^{\infty}(\Omega_1,\P_1,\mathcal{F}_1)\rightarrow L^1(\Omega_1,\P_1,\mathcal{F}_1)
$$and
$$
B: L^{\infty}(\Omega_2,\P_2,\mathcal{F}_2)\rightarrow L^1(\Omega_2,\P_2,\mathcal{F}_2).
$$
\begin{definition}[Metrization of action convergence] For the two $P$-operators $A, B$ the \emph{action convergence metric} is
$$
d_{M}(A, B):=\sum_{k=1}^{\infty} 2^{-k} d_{H}\left(\mathcal{S}_{k}(A), \mathcal{S}_{k}(B)\right)
$$
\end{definition}
As the Hausdorff metric is bounded by 1, we have that also the action convergence distance is bounded by 1.

We now define convergence in the action convergence metric. Differently from the terminology in \cite{backhausz2018action} we will use distinct terms to distinguish the convergence in the metric and the convergence with the additional requirement that a limit $P-$operator exists.\newline


We will say that a sequence of $P$-operators $\left\{A_i \in \mathcal{B}\left(\Omega_i\right)\right\}_{i=1}^{\infty}$ is a Cauchy sequence if the sequence is Cauchy in $d_M$. \newline


We observe that a sequence $\left\{A_i \in \mathcal{B}\left(\Omega_i\right)\right\}_{i=1}^{\infty}$ is a Cauchy sequence  if and only if for every $k \in \mathbb{N}$ the sequence $\left\{\mathcal{S}_k\left(A_i\right)\right\}_{i=1}^{\infty}$ is a Cauchy sequence in $d_H$.
 

\begin{remark} The completeness of $\left(\mathcal{P}\left(\mathbb{R}^k\right), d_{L P}\right)$ implies that the induced Hausdorff topology is also complete \cite{RealAnalysisGordon}. Therefore, a sequence $\left\{A_i\right\}_{i=1}^{\infty}$ is a Cauchy sequence if and only if for every $k \in \mathbb{N}$ there is a closed set of measures $X_k$ such that $\lim _{i \rightarrow \infty} d_H\left(S_k\left(A_i\right), X_k\right)=0$.
 \end{remark}
 
The following lemma is a consequence of Prokhorov theorem and guarantees that a subsequence $\{{S_k\left(A_i\right)}\}^{\infty}_{i=1}$ converges in $d_H$ to a closed set of measures $X_k$ under a uniform bound assumption on the $\|\cdot\|_{\infty\rightarrow 1}$ norm.

\begin{lemma}[Lemma 2.6 in \cite{backhausz2018action}]
Let $\{A_i\}^{+\infty}_{i=1}$ be a sequence of $P-$operators with uniformly bounded $(\infty,1)-$norms. Then, it has a subsequence that is a Cauchy sequence.
\end{lemma}

Untill now, for a sequence of $P-$operators we just looked at the convergence of the sequences of $k-$profiles $\{S_k(A_i)\}^{+\infty}_{i=1}$ but not to the actual existence of a $P-$operator as limit object. This is actually the convergence we are interested in.

\begin{definition}[Action convergence of $P$-operators] 
We say that the sequence  $\left\{A_i \in \mathcal{B}\left(\Omega_i\right)\right\}_{i=1}^{\infty}$ is \emph{action convergent} to the $P-$operator $A\in\mathcal{B}\left(\Omega\right)$  if it is a Cauchy sequence and it is such that for every $k\in \N$ the $k-$profile $S_k(A)$ is the weak limit of the $k$-profiles sequence $\{S_k(A_i)\}^{\infty}_{i}$. The $P-$operator $A$ is the limit of the sequence  $\left\{A_i \right\}_{i=1}^{\infty}$.
\end{definition}

Additionally, we will say that a sequence of $P-$operators $\left\{A_i \in \mathcal{B}\left(\Omega_i\right)\right\}_{i=1}^{\infty}$ is action convergent if there exists a limit $P-$operator.

\begin{remark}
We will often use the following consequence of the definition of action convergence. For an action convergent sequence of operators $\left\{A_i\right\}_{i=1}^{\infty}$ to a $P-$operator $A$ and for every $v \in L_{[-1,1]}^{\infty}(\Omega)$, there are elements $v_i \in L_{[-1,1]}^{\infty}\left(\Omega_i\right)$ such that $\mathcal{L}\left(v_i,Av_i\right)$ weakly converges to $\mathcal{L}(v,Av)$ as $i$ goes to infinity.
\end{remark}

The following theorem gives us that sets of $P-$operators with uniformly bounded $(p,q)-$norm with $p\neq \infty$ are pre-compact in the action convergence metric.

\begin{theorem}[Theorem 2.14 in \cite{backhausz2018action}] Let $p\in [1,\infty)$ and $q\in [1,\infty]$. Let $\{A_i\}_{i=1}^\infty$ be a Cauchy sequence of $P$-operators with uniformly bounded $\|\cdot\|_{p\to q}$ norms. Then there is a $P$-operator $A$ such that $\lim_{i\to\infty} d_M(A_i,A)=0$, and $\|A\|_{p\to q}\leq\sup_{i\in\mathbb{N}}\|A_i\|_{p\to q}$. Therefore, the sequence $\{A_i\}_{i=1}^\infty$ is action convergent. 
\end{theorem}

In the following, we will sometimes also use the following terminology. For a $P$-operator $A$ and $k\in\mathbb{N}$ let $cl(\mathcal{S}_k(A))$ denote the closure of $\mathcal{S}_k(A)$ in the space $(\mathcal{P}(\mathbb{R}^{2k}),d_{\rm LP})$.

\begin{definition} [Weak equivalence and weak containment] Let $A$ and $B$ be two $P$-operators. We say that $A$ and $B$ are  weakly equivalent if $d_M(A,B)=0$. We have that $A$ and $B$ are weakly equivalent if and only if $cl(\mathcal{S}_k(A))=cl(\mathcal{S}_k(B))$ holds for every $k\in\mathbb{N}$. We say that $A$ is weakly contained in $B$ if $cl(\mathcal{S}_k(A))\subseteq cl(\mathcal{S}_k(B))$ holds for every $k\in\mathbb{N}$. We denote weak containment by $A\prec B$.
\end{definition}

The action convergence metric can be compared with many other norms and metrics.

We start with an inequality relating the action convergence metric and the $(\infty,1)-$norm.

\begin{lemma}[Lemma 2.19 in \cite{backhausz2018action}]  Assume that $A,B$ are $P$-operators acting on the same probability space $\Omega$. We have  $d_M(A,B)\leq 3\|A-B\|_{\infty\to 1}^{1/2}$.
\end{lemma}

We also have an inequality relating the action convergence metric and the cut-metric (see \eqref{eqn:cutdist} or \cite{LovaszGraphLimits}). 

\begin{lemma} [Lemma 2.20 in \cite{backhausz2018action}]Assume that $A,B$ are $P$-operators acting on the same space $L^\infty(\Omega)$. Then $d_M(A,B)\leq 12\delta_\square(A,B)^{1/2}$.
\end{lemma}

Importantly, we can relate action convergence to the notions of convergence arising in the cases of dense graph sequences convergence (cut-metric, graphons) and uniformly bounded degree graph sequences convergence (local-global convergence). We state here the two results

\begin{theorem}[Theorem 8.2 in \cite{backhausz2018action}]\label{TheoremEqCutMetr} The two pseudometrics $\delta_\square$ and $d_M$ are equivalent on the space of graphons. 
\end{theorem}

\begin{theorem}[Theorem 9.4 \cite{backhausz2018action}] A sequence of graphings is local-global convergent if and only if it is convergent in the metric $d_M$.
\end{theorem}

We refer to \cite{backhausz2018action} for more details.

A lot of properties of matrices and graphs can be directly translated in the language of $P-$operators.

\begin{definition} Let $A\in\mathcal{B}(\Omega)$ be a $P$-operator. 
\begin{itemize}
\item $A$ is {\bf self-adjoint} if $\mathbb{E}[(Av)w]=\mathbb{E}[v(Aw)]$ holds for every $v,w\in L^\infty(\Omega)$.
\item $A$ is {\bf positive} if $\mathbb{E}[vAv]\geq 0$ holds for every $v\in L^\infty(\Omega)$.
\item $A$ is {\bf positivity-preserving} if for every $v\in L^\infty(\Omega)$ with $v(x)\geq 0$ for almost every $x\in\Omega$, we have that $(Av)(x)\geq 0$ holds for almost every $x\in\Omega$.
\item $A$ is {\bf $c$-regular} if $1_\Omega A=c1_\Omega$ for some $c\in\mathbb{R}$.
\item $A$ is a {\bf graphop} if it is positivity-preserving and self-adjoint.
\item $A$ is a {\bf Markov graphop} if $A$ is a $1$-regular graphop.
\item $A$ is {\bf atomless} if $\Omega$ is atomless.
\end{itemize}
\end{definition}

In particular, a graphop is the natural translation in the language of $P-$operators of the adjacency matrix of a graph when we consider the uniform probability measure on the nodes, i.e.\ a self-adjoint and positivity-preserving $P-$operator. The Markov probability matrix is instead a Markov grahop if considered with respect to the probability measure weighted on the degrees.\newline

All these properties carry over in the limit if we assume a uniform bound on the $(p,q)-$norm with $p,q\notin\{1,\infty\}$. We briefly state here the relative results.

\begin{lemma}[Lemma 3.2 in \cite{backhausz2018action}]Atomless $P$-operators are closed with respect to $d_M$.
\end{lemma}

\begin{lemma}[Corollary 2.2 in \cite{ArankaAction2022}]Let $(p,q)\in[1,\infty]\times[1,\infty]\setminus\{(\infty,1)\}$. Let $\{A_i\in\mathcal{B}(\Omega_i)\}_{i=1}^\infty$ be a sequence of uniformly $(p,q)$-bounded  $P$-operators converging to a $P$-operator $A\in\mathcal{B}(\Omega)$. Then we have the following two statements.
\begin{enumerate}
\item If $A_i$ is positive for every $i$, then $A$ is also positive.
\item If $A_i$ is self-adjoint for every $i$, then $A$ is also self-adjoint.
\end{enumerate}
\end{lemma}

In \cite{ArankaAction2022} it is also shown that the previous Lemma fails for $(p,q)$ different from the assumption.

We also state
\begin{lemma}[Proposition 3.4 in \cite{backhausz2018action}] Let $p\in [1,\infty),q\in [1,\infty],c\in\mathbb{R}$ and let $\{A_i\in\mathcal{B}(\Omega_i)\}_{i=1}^\infty$ be a sequence of uniformly $(p,q)$-bounded  $P$-operators converging to a $P$-operator $A\in\mathcal{B}(\Omega)$. Then we have the following two statements.
\begin{enumerate}
\item If $A_i$ is positivity-preserving for every $i$, then $A$ is also positivity-preserving.
\item If $A_i$ is $c$-regular for every $i$, then $A$ is also $c$-regular.
\end{enumerate}
\end{lemma}

%\section{Some heuristics about action convergence}\label{secHeuristics}

\section{Tensors and hypergraphs}\label{SecTensHyp}

We start by giving some preliminary definitions and notations on tensors and hypergraphs.\newline 

We indicate a vector in $\R^n$ by ${\bf x}=(x_1,\dots,x_n)$. For a set $A$ we denote with $|A|$ the cardinality of $A$ and with $2^A$ the powerset of A.\newline
\begin{definition}
The \emph{symmetrization} of an  $r-$th order tensor $T$ is the $r$-th order tensor $Sym(T)$ where
$$
(Sym(T))_{i_1\ldots,i_r}=\frac{1}{r!}\sum_{\sigma \in \Sigma}T_{i_{\sigma(1)},\ldots,i_{\sigma(r)}}
$$
where $\Sigma$ is the set of all permutations of $[r]$.
\end{definition}

It will be very convenient in the following to consider a tensor as many different possible operators. 
\begin{definition}\label{DefAction}
For an $r-$th order $n-$dimensional symmetric tensor $T$ and for $s\in [r-1]$, the $s-$action of $T$ on the $s-$th order $n-$dimensional $f^{(1)},\ldots, f^{(r-1)}$ is the operation 
\begin{equation*}
\begin{aligned}
&(T[f^{(1)},\ldots,f^{(r-1)}])_{i_1,\ldots,i_s}=\\
Sym(\sum^n_{j_1,\ldots,j_{r-s}=1}T_{j_1,\ldots,j_{r-s},i_1,\ldots,i_{s}}&f^{(1)}_{i_2,\ldots,i_{s},j_1}f^{(2)}_{i_3,\ldots,i_{s},j_1,j_2}\ldots f^{(r-2s+1)}_{j_{r-2s+2},\ldots,j_{r-s},i_1}\ldots f^{(r-1)}_{j_{r-s},i_1,\ldots,i_{s-1}})
\end{aligned}
\end{equation*}
%%f^{(r-1)}_{j_{r-2s+2},\ldots,j_{r-s},i_1,\ldots,i_{s-1}} 
%\ldots,f^{(s)}_{j_1,\ldots,j_s},f^{(s+1)}_{j_2,\ldots j_{s+1}}
%,

\end{definition}
The $s-$action of $T$ is an operator that sends $k-1$ $s-$th order $n-$dimensional  symmetric tensors in a $s-$th order $n-$dimensional symmetric tensor. Therefore, the $s-action$ is an operator acting on real-valued functions with domain the set of subsets of cardinality $s$ of $[n]$. 

To make the definition more clear, we give here some examples of $s-$action of a tensor that we will also use in the following.

\begin{example}
For a $r-$th order $n-$dimensional symmetric tensor $T$ the $1-$action of $T$ on the $n-$dimensional vectors  (first-order tensors) $f^1,\ldots, f^{r-1}\in \R^n$  is the operation
\begin{equation*}
(T[f^{(1)},\ldots,f^{(r-1)}])_{i}=\sum^n_{j_2,\ldots j_{r-1}=1}T_{i,j_2,\ldots,j_{r-1}}f^{(1)}_{j_2}\ldots f^{(r-1)}_{j_{r-1}}
\end{equation*}

In the case of $r=2$, a second-order tensor $T$ is a matrix and the $1-$action on a vector $f$ is just  the classical matrix multiplication with the vector $f\in \R^n$, i.e.
\begin{equation*}
(Tf)_{i}=\sum^n_{j=1}T_{ij}f_{j}
\end{equation*}

\end{example}
%For a third order $n-$dimensional symmetric tensor $T$ the $1-$action of $T$ on the $n-$dimensional vectors  $f^{(1)},\ldots, f^{(r-1)}$  is the operation
%\begin{equation*}
%T[f^{(1)},\ldots,f^{(r-1)}]_{i}=\sum^n_{i_2,\ldots i_{r-1}=1}T_{i,i_2,\ldots,i_{r-1}}f^{(1)}_{i_2},\ldots, f^{(r-1)}_{i_{r-1}}
%\end{equation*}
\begin{example}
For a $r-$th order $n-$dimensional symmetric tensor $T$ and the $(r-1)-$action of $T$ on the $s-$th order $n-$dimensional $f^{(1)},\ldots, f^{(r-1)}$ is the operation 
\begin{equation*}
(T[f^{(1)},\ldots,f^{(r-1)}])_{i_1,\ldots,i_{r-1}}=Sym(\sum^n_{j=1}T_{j,i_2,\ldots,i_{r}}f^{(1)}_{j,i_1,i_2,\ldots,i_{s-1},\hat{i}_s}\ldots f^{(p)}_{j,i_1,i_2,\ldots,  \hat{i}_p\ldots i_{s}} \ldots f^{(r-1)}_{j,\hat{i}_1,i_2,i_3\ldots,i_{s}})
\end{equation*}

In particular, for a third-order $n-$dimensional symmetric tensor $T$ the $2-$action of $T$ on the $n-$dimensional vectors  $f,g\in \R^n$  is the operation
\begin{equation*}
(T[f,g])_{i,k}=\frac{1}{2}(\sum^n_{j=1}T_{j,i,k}f_{j,i}g_{j,k}+\sum^n_{j=1}T_{j,k,i}f_{j,k}g_{j,i} )
\end{equation*}
\end{example}

We now introduce some notation and notions for hypergraphs.
%We briefly recover the notion of hypergraph here, a generalization of graph.
%\begin{definition}
 %An \emph{hypergraph} is a pair $H=(V,E)$ where $V=\{v_1,\ldots,v_n\}$ is the set of \emph{vertices}, $E=\{e_1,\ldots,e_m\}$ is the set of \emph{edges} and $\emptyset \neq e \subseteq V$ for each $e\in E$.\newline
% An hypergraph $H$ is $k$\emph{-uniform} if $|e|=k$ for every $e\in E$.
% Given $v\in V$, its \emph{degree} is
% \begin{equation*}
 %    \deg v:=|e\in E\,:\,v\in e|.
% \end{equation*}
Given an edge $e\in E$, we recall that we denote its cardinality by $|e|$, and in the following we will denote with $r$ the maximal edge cardinality, i.e.\
\begin{equation*}
  r:=\max_{e\in E}|e|.
\end{equation*}
%Moreover, given $s\leq k$, we let 
%\begin{equation*}
 %   E_s=\{e\in E:\, |e|=s\} \quad \text{and}\quad E_{< s}:=\{e\in E:\, |e|< s\}.
%\end{equation*}
%\end{definition}
%Additionally, we define the $r-$\emph{uniform degree}
%  \begin{equation*}
  %   \deg_s v:=|e\in E_s\,:\,v\in e|.
 %\end{equation*}
%From here on we fix a hypergraph $G=(V,E)$, we assume that each edge contains at least two vertices and we assume that $e_i\neq e_j$ if $i\neq j$.

%We will now use the number 

%$$N(s,r)=\left\{\begin{matrix}
%s\\
%r
%\end{matrix}\right\}(r-1)!$$
%where
%$$
%\left\{\begin{matrix}
%s\\
%r
%\end{matrix}\right\}=\frac{1}{s!}\sum^s_{j=1}(-1)^j\left(\begin{matrix}
%r\\
%j
%\end{matrix}\right)(s-j)^r$$
%
%is the \emph{Stirling number of the second kind}. In particular, we have$N(r,r)=(r-1)!$. \newline
%We now present a tensor naturally associated to an hypergraph, as the adjacency matrix is naturally associated to a graph. This tensor has been introduced in \cite{TensorsHypergraphs}.
Moreover, we observe that for the adjacency tensor of an hypergraph $H=(V,E)$ on $n$ nodes with largest edge cardinality $r$ its adjacency tensor
\begin{equation*}
A_{i_1,\ldots,i_r}:=\begin{cases}
0 & \text{ if }\{v_{i_1},\ldots,v_{i_r}\}\notin E\\
1 &  \text{ if }\{v_{i_1},\ldots,v_{i_r}\}\in E.\\
\end{cases}
\end{equation*}
is a standard notion for $r-$uniform hypergraphs. However, also edges with non-maximal cardinality are incorporated as repeated indices correspond to sets of lower cardinality. 

We give here some examples of deterministic and random hypergraphs. We will use these examples in the following.

\begin{example}
The \emph{complete} hypergraph on $n$ vertices is the hyepergraph on $[n]$ and with $E=2^V\setminus\emptyset$.  
\end{example}


\begin{example}
 The \emph{complete $r-$uniform} hypergraph on $n$ vertices is the hypergraph with $[n]$ and such that $E$ is the set of all ${n \choose r}$  subsets of $V$ with cardinality $r$. 
\end{example}

\begin{example}
  Graphs are the $2-$uniform hypergraphs.  
\end{example}

Therefore a random graph is a $2-$uniform random hypergraph. A very common random graph model is the following 

\begin{example}[Erd\"os-Renyi graph]
Consider the vertex set $V=[n]$ and we connect every pair of the ${n \choose 2}$ possible independently with probability $p$, i.e.\ following the law of independent Bernoulli random variables. This is the Erd\"os-Renyi random graph and we will denote it with $G(n,p)$.%\left(\begin{matrix}
%n\\
%2
%\end{matrix}\right)$

\end{example}

A very common random uniform hypergraph that we will use to construct other random hypergraphs is the following

\begin{example}[$r-$uniform Erdős–Rényi random hypergraph] \label{ERRandomHypergraph}We denote with $G(n,p,r)$ the $r-$uniform random hypergraph with vertex set $V=[n]$ and with edge set $E$ defined as follows: For every $e$, set of vertices of cardinality $r$,  $e$ is in $E$ with probability $p$, i.e.\ every edge of cardinality $r$ is in $E$ following independent Bernoulli random variables with parameter p. In the case r=2, $G(n,p,2)$ corresponds with the Erdős–Rényi random graph $G(n,p)$.
\end{example}

In the case $p=1$ the $r-$uniform Erdős–Rényi random hypergraph $G(n,1,r)$ corresponds with the complete $r-$uniform hypergraph.

We give also another example of uniform random hypergraph:
\begin{example}\label{TriangRandHyp}
We denote with $T(n,p)$ the random $3-$uniform hypergraph constructed taking the vertex set $V=[n]$  and as edges the triangles of the  Erdős–Rényi random graph $G(n,p)$ on the same vertex set $V=[n]$.  
\end{example}

We can generalize naturally this random hypergraph model
\begin{example}\label{RandHypergGener}
We denote with $R(n,p_1,\ldots p_{r-1},r)$ the $r-$uniform random hypergraph on the vertex set $V=[n]$ constructed inductively on $k$ as follows:
\begin{itemize}
\item for $r=2$ we define $R(n,p,2)=G(n,p,2)$.  
\item for $r>2$ we define $R(n,p_1,\ldots,p_{r-1},r) $  as the $r-$uniform hypergraph constructed selecting as edges independently with probability $p_{r-1}$ the sets of  $r$ vertices such that $R(n,p_1,\ldots,p_{r-2},r-1) $ restricted to these $r$ vertices is the $(r-1)-$uniform complete hypergraph on $r$ vertices. 
 \end{itemize}
\end{example}

We notice that the random $3-$uniform hypergraph $T(n,p)$ is the same as the random $3-$uniform hypergraph $R(n,1,p,3)$.

We now define a simple example of random oriented graph.

\begin{example}[Tournament]\label{RandtournGraph}
A tournament $Tor(n)$ is a random directed graph model in which for every edge $\{i,j\}$ of the complete graph on $[n]$ we give a direction $(i,j)$ or $(j,i)$ with equal probability.
\end{example}

We use the notion of tournament to define one last random hypergraph model.

\begin{example}\label{RandHyperTourn}
A random hypergraph $TH(n)$ is the $3-$uniform hypergraph with vertex set $[n]$ and with edge set consisting of the triples of vertices that constitutes an oriented cycle of length 3 of the tournament $Tor(n)$ on the same vertex set. 
\end{example}
\section{Multi-action convergence for multi-linear operators}\label{SecMultiActConv}

In the previous section, we have seen how an hypergraph can be interpreted as a tensor and how there are various ways to interpret tensors as multi-linear operators. Therefore, we now want to generalize action convergence to general multi-linear operators.  

\begin{definition}
An $r-th$ order $P-$multi-operator  is a multi-linear operator $A: L^{\infty}(\Omega) ^{r-1}\rightarrow$ $L^{1}(\Omega)$ such that the $\infty\rightarrow 1$ multi-linear operator norm 
\begin{equation*}
\|A\|_{\infty\rightarrow 1}:=\sup_{f^{(i)}\in L^{\infty}(\Omega), \, f^{(i)}\neq 0}\frac{\|A[f^{(1)},\ldots,f^{(r-1)}]\|_{ 1}}{\|f^{(1)}\|_{ \infty}\cdots\|f^{(r-1)}\|_{ \infty}}
\end{equation*}is finite. A multi-$P-$operator $A$ is acting on the probability space $(\Omega, \P,\mathcal{F})$ if the $L^1$ and $L^\infty$ spaces are defined on $(\Omega, \P,\mathcal{F})$. We denote the set of all $r-$th order multi-$P-$operators with $\mathcal{B}_r$ and the set of all $r-$th order $P-$operators acting on $(\Omega, \P,\mathcal{F})$ with $\mathcal{B}_r(\Omega, \P,\mathcal{F})$.
\end{definition}


We can relate multi-$P$ operators and tensors in multiple ways as in the following
\begin{example}
    We can interpret the $s-$action of a $r-$th order tensor as a multi-$P-$operator 
$$\widetilde{T}:L^{\infty}([n]^s,Sym)^{r-1}\longrightarrow L^1([n]^s,Sym)$$
where $Sym$ is the symmetric $\sigma-$algebra on $[n]^s$ and we consider the uniform probability measure on $[n]^s$, i.e.\ $$\begin{aligned}
&\mathbb{P}(\{(j_1,\ldots,j_s) \text{ such that } (i_{\sigma(1)},\ldots,i_{\sigma(s)}) \text{ where } \sigma \text{ is a permutation of $[s]$}\})\\
&=\frac{|\{(j_1,\ldots,j_s) \text{ s.t } (i_{\sigma(1)},\ldots,i_{\sigma(s)}) \text{ where } \sigma \in \mathcal{P}\}|}{n^s}
\end{aligned}$$
for all $i_1,\ldots, i_s \in [n]$. We just have to identify the set of $s-$th order symmetric tensors with $L^{\infty}([n]^s,Sym)\cong L^1([n]^s,Sym)$ in the canonical way.
\end{example}


For functions $f^{(1)}_1,\ldots, f^{(r-1)}_1,\ldots, f^{(1)}_k,\ldots,f^{(r-1)}_k \in L^{\infty}(\Omega)$ we consider the $rk-$dimensional random vector
$$
(f^{(1)}_1,\ldots, f^{(r-1)}_1,A[f^{(1)}_1,\ldots ,f^{(r-1)}_1],\ldots,f^{(1)}_k,\ldots,f^{(r-1)}_k,A[f^{(1)}_k,\ldots,f^{(r-1)}_k])
$$
for a $P-$multi-operator $A$ and we call the distribution of this random vector 
\begin{equation}
\begin{aligned}
\mathcal{L}
(f^{(1)}_1,\ldots, f^{(r-1)}_1,A[f^{(1)}_1,\ldots f^{(r-1)}_1],\ldots,f^{(1)}_k,\ldots,f^{(r-1)}_k,A[f^{(1)}_k,\ldots,f^{(r-1)}_k])
\in \mathcal{P}(\mathbb{R}^{rk})
\end{aligned}
\end{equation}
the measure generated by the multi-$P-$operator $A$ through the ordered sequence of functions $f^{(1)}_1,$$\ldots,$\\$ f^{(r-1)}_1,$$\ldots,  f^{(1)}_k,\ldots,f^{(r-1)}_k $$\in L^{\infty}(\Omega)$.
Sometimes, we will use the abbreviation 
$$
\begin{aligned}
 &S_A(f^{(1)}_1,\ldots, f^{(r-1)}_1\ldots,f^{(1)}_k,\ldots,f^{(r-1)}_k)\\ &=\mathcal{L}
(f^{(1)}_1,\ldots, f^{(r-1)}_1,A[f^{(1)}_1,\ldots, f^{(r-1)}_1],\ldots, f^{(1)}_k,\ldots,f^{(r-1)}_k,A[f^{(1)}_k,\ldots,f^{(r-1)}_k])
\in \mathcal{P}(\mathbb{R}^{rk})   
\end{aligned}
$$

We now define the set of measures generated by $A$. Similarly to the action convergence in the linear case, it is convenient to allow in our sets only measures generated by functions in $L_{[-1,1]}^{\infty}(\Omega)$, i.e. functions taking values between $-1$ and $+1$ almost everywhere. Therefore, we define the $k-$\emph{profile} of $A$,  $S_k(A)$, as the set of measures generated by $A$ by functions in  $L_{[-1,1]}^{\infty}(\Omega)$.
$$
\begin{aligned}
S_k(A)=\bigcup_{f^{(1)}_1,\ldots, f^{(r-1)}_1,\ldots, f^{(1)}_k,\ldots,f^{(r-1)}_k\in L_{[-1,1]}^{\infty}(\Omega)}&\{\mathcal{L}
(f^{(1)}_1,\ldots, f^{(r-1)}_1,A[f^{(1)}_1,\ldots f^{(r-1)}_1],\ldots,\\
&f^{(1)}_k,\ldots,f^{(r-1)}_k,A[f^{(1)}_k,\ldots,f^{(r-1)}_k])\}
\end{aligned}
$$This is a set of measures. To compare two different sets of measures we will use the Hausdorff metric $d_H$ (Definition	\ref{HausdorffDist}) on sets of the space of probability measures $\mathcal{P}(\mathbb{R}^{rk})$ equipped with the Levy-Prokhorov metric (Definition \ref{LevyProk}) $d_{LP}$. 

We are now ready to define the pseudo-metric we are interested in.
Consider two $P-$operators 
$$
A: L^{\infty}(\Omega_1)^{r-1}\rightarrow L^1(\Omega_1)
$$and
$$
B: L^{\infty}(\Omega_2)^{r-1}\rightarrow L^1(\Omega_2).
$$
\begin{definition}[Metrization of action convergence] For the two $r-th$ order multi-$P$-operators $A, B$ the \emph{action convergence metric} is
$$
d_{M}(A, B):=\sum_{k=1}^{\infty} 2^{-k} d_{H}\left(\mathcal{S}_{k}(A), \mathcal{S}_{k}(B)\right)
$$
\end{definition}

As the Hausdorff metric is bounded by 1, we have that also the action convergence distance is bounded by 1.


We will say that a sequence of $P$-operators $\left\{A_i \in \mathcal{B}_r\left(\Omega_i\right)\right\}_{i=1}^{\infty}$ is a Cauchy sequence if the sequence is Cauchy in $d_M$. \newline


We notice that a sequence $\left\{A_i \in \mathcal{B}_r\left(\Omega_i\right)\right\}_{i=1}^{\infty}$ is a Cauchy sequence  if and only if for every $k \in \mathbb{N}$ the sequence $\left\{\mathcal{S}_k\left(A_i\right)\right\}_{i=1}^{\infty}$ is a Cauchy sequence in $d_H$.
 

\begin{remark} The completeness of $\left(\mathcal{P}\left(\mathbb{R}^k\right), d_{L P}\right)$ implies that the induced Hausdorff topology is also complete \cite{RealAnalysisGordon}. Therefore, a sequence $\left\{A_i\right\}_{i=1}^{\infty}$ is a Cauchy sequence if and only if for every $k \in \mathbb{N}$ there is a closed set of measures $X_k$ such that $\lim _{i \rightarrow \infty} d_H\left(S_k\left(A_i\right), X_k\right)=0$.
 \end{remark}
 
The following lemma is an equivalent of Lemma 2.6 in \cite{backhausz2018action} for multi-$P-$operators and guarantees that a subsequence $\{{S_k\left(A_i\right)}\}^{\infty}_{i=1}$ converges in $d_H$ to a closed set of measures $X_k$ under a uniform bound assumption on the $\|\cdot\|_{\infty\rightarrow 1}$ norm.
 
\begin{lemma}
Let $\{A_i\}^{+\infty}_{i=1}$ be a sequence of $r-$th order multi-$P-$operators with uniformly bounded $(\infty,1)-$norms. Then, it has a subsequence that is a Cauchy sequence.
\end{lemma}

This lemma follows directly from the same standard arguments that we summarize here for completeness.

For a probability measure $\mu$ on $\mathbb{R}^k$ let $\tau(\mu)\in [0,\infty]$ denote the maximal expectation of the marginals of $\mu$,
\begin{equation}\label{eqn:tau}\tau(\mu)=\max_{1\leq i\leq k} \int_{(x_1,x_2,\dots,x_k)\in\mathbb{R}^k}|x_i|~d\mu.\end{equation} For $c\in\mathbb{R}^+$ and $k\in\mathbb{N}$ let 
$$\mathcal{P}_c(\mathbb{R}^k):=\{\mu : \mu\in\mathcal{P}(\mathbb{R}^k) , \tau(\mu)\leq c\}.$$
Let furthermore $\mathcal{Q}_c(\mathbb{R}^k)$ denote the set of closed sets in the metric space $(\mathcal{P}_c(\mathbb{R}^k),d_{\rm LP})$. 

\begin{lemma} The metric spaces $(\mathcal{P}_c(\mathbb{R}^k),d_{\rm LP})$ and $(\mathcal{Q}_c(\mathbb{R}^k),d_H)$ are both compact and complete metric spaces.
\end{lemma}

\begin{proof} Markov's inequality gives uniform tightness in $\mathcal{P}_c(\mathbb{R}^k)$, which implies the compactness of $(\mathcal{P}_c(\mathbb{R}^k),d_{\rm LP})$ for Prokhorov's theorem. It is known that the set of closed subsets of a compact Polish space equipped with the Hausdorff metric is again compact.
\end{proof}


\begin{lemma} Let $A\in\mathcal{B}(\Omega)$ and let $c:=\max(\|A\|_{\infty\to1},1)$. Then for every $k\in\mathbb{N}$ we have that $\mathcal{S}_k(A)\in \mathcal{Q}_c(\mathbb{R}^{rk})$.
\end{lemma}

\begin{proof} Let $\{v^{(1)}_i,\ldots,v_i^{(r-1)}\}_{i=1}^k$ be a sequence of functions in $L^\infty_{[-1,1]}(\Omega)$. We have that $\|v^{(j)}_i\|_1\leq \|v^{(j)}_i\|_\infty\leq 1$ for every $j\in[r-1]$ and $\|A[v^{(1)}_i,\ldots,v^{(r-1)}_i ]\|_1\leq \|A\|_{\infty\to 1}$  holds for $1\leq i\leq k$. The result follows as the first moments of the absolute values of the coordinates in $\tau$, \eqref{eqn:tau}, are given by $$\{\|v^{(j)}_i\|_1\}_{i=1}^k$$ for $j\in [r-1]$ and $$\{\|A[v^{(1)}_i,\ldots,v^{(r-1)}_i]\|_1\}_{i=1}^k.$$
\end{proof}

As in the linear case, for a sequence of multi-$P-$operators, we will not be interested only in the convergence of the sequences of $k-$profiles $\{S_k(A_i)\}^{+\infty}_{i=1}$ but also in the existence of a multi-$P-$operator as limit object. This will actually be the convergence we are interested in.

\begin{definition}[Action convergence of $P$-operators] 
We say that the sequence  $\left\{A_i \in \mathcal{B}\left(\Omega_i\right)\right\}_{i=1}^{\infty}$ is \emph{action convergent}  to the $r-$th order multi-$P-$operator $A\in\mathcal{B}\left(\Omega\right)$  if it is a Cauchy sequence and it is such that for every $k\in \N$ the $k-$profile $S_k(A)$ is the weak limit of the k-profiles sequence $\{S_k(A_i)\}^{\infty}_{i}$. The multi-$P-$operator $A$ is the limit of the sequence  $\left\{A_i \right\}_{i=1}^{\infty}$.
\end{definition}
Additionally, we will say that a sequence of multi-$P-$operators $\left\{A_i \in \mathcal{B}\left(\Omega_i\right)\right\}_{i=1}^{\infty}$ is action convergent if there exists a limit multi-$P-$operator.

 \begin{remark}
 We will often use the following consequence of the definition of action convergence. For an action convergent sequence of operators $\left\{A_i\right\}_{i=1}^{\infty}$ to a multi-$P-$operator $A$ and for every $v^{(1)},\ldots, v^{(r-1)}\in L_{[-1,1]}^{\infty}(\Omega)$, there are elements $v^{(1)}_i,\ldots, v^{(r-1)}_i \in L_{[-1,1]}^{\infty}\left(\Omega_i\right)$ such that $$\mathcal{L}\left(v^{(1)}_i,\ldots, v^{(r-1)}_i,A_i[v^{(1)}_i,\ldots, v^{(r-1)}_i]\right)$$ weakly converges to $$\mathcal{L}(v^{(1)},\ldots, v^{(r-1)},A[v^{(1)},\ldots, v^{(r-1)}])$$ as $i$ goes to infinity.
 \end{remark}

We introduce now a multi-operator norm for $L^p$ spaces that is a natural generalization of the linear operator norm
\begin{definition}[Multi-linear operator norm]
For an $r-$th order multi-$P-$operator $A$ %$$A:L^{p_1}(\Omega)\times \ldots \times L^{p_r}(\Omega)\rightarrow L^{q}(\Omega)$$ 
the multi-linear operator $(p_1,\ldots,p_{r-1},q)-$norm is 
\begin{equation*}
\|A\|_{p_1,\ldots,p_{r-1}\rightarrow q}:=\sup_{f^{(i)}\in L^{\infty}(\Omega), \, f^{(i)}\neq 0}\frac{\|A[f^{(1)},\ldots,f^{(r-1)}]\|_{q}}{\|f^{(1)}\|_{ p_1}\cdots\|f^{(r-1)}\|_{ p_r}}.
\end{equation*}

We denote the set of all $r-$th order multi-$P-$operators with finite $(p_1,\ldots,p_{r},q)-$norm with $\mathcal{B}_{p_1,\ldots,p_r,q}$ and the set of all $r-$th order $P-$operators acting on $(\Omega, \P,\mathcal{F})$ with finite $(p_1,\ldots,p_{r},q)-$norm with $\mathcal{B}_{p_1,\ldots,p_r,q}(\Omega, \P,\mathcal{F})$.
\end{definition}

\begin{remark}
    With an abuse of notation, we can think of a multi-$P-$operator $A$ with bounded $(p_1,\ldots,p_{r-1},q)-$norm as a multi-linear bounded operator $$A:L^{p_1}(\Omega)\times \ldots \times L^{p_r}(\Omega)\rightarrow L^{q}(\Omega)$$ by Lemma \ref{LemmMultBoundExt}.
\end{remark}

The following theorem is the generalization of Theorem 2.14 in \cite{backhausz2018action} to the multi-linear case and it states that sets of multi-$P-$operators with uniformly bounded $(p_1,\ldots,p_{r-1},q)-$norm with $p_1,\ldots,p_{r-1}\neq \infty$ are pre-compact in the action convergence metric.

\begin{theorem} \label{CompactnesMultiActConv} For $p\in [1,\infty)$ and $q\in [1,\infty]$, let $\{A_i\}_{i=1}^\infty$ be a Cauchy sequence of $P$-operators with uniformly bounded $\|.\|_{p\to q}$ norms. Then there is a multi-$P$-operator $A$ such that $\lim_{i\to\infty} d_M(A_i,A)=0$, and $\|A\|_{p\to q}\leq\sup_{i\in\mathbb{N}}\|A_i\|_{p\to q}$. Therefore, the sequence $\{A_i\}_{i=1}^\infty$ is action convergent. 
\end{theorem}

We give the technical proof of this theorem in the next section that is an adaptation of the proof of Theorem 2.9 in \cite{backhausz2018action} to the multi-linear case.

In the following, we will sometimes also use the following terminology.

For a $r-$th order multi-$P$-operator $A$ and $k\in\mathbb{N}$ let $cl(\mathcal{S}_k(A))$ denote the closure of $\mathcal{S}_k(A)$ in the space $(\mathcal{P}(\mathbb{R}^{rk}),d_{\rm LP})$.

\begin{definition} [Weak equivalence and weak containment] Let $A$ and $B$ be two $r-$th order multi-$P$-operators. We say that $A$ and $B$ are  weakly equivalent if $d_M(A,B)=0$. We have that $A$ and $B$ are weakly equivalent if and only if $cl(\mathcal{S}_k(A))=cl(\mathcal{S}_k(B))$ holds for every $k\in\mathbb{N}$. We say that $A$ is weakly contained in $B$ if $cl(\mathcal{S}_k(A))\subseteq cl(\mathcal{S}_k^*(B))$ holds for every $k\in\mathbb{N}$. We denote weak containment by $A\prec B$.
\end{definition}

\section{Construction of the limit object}\label{SecConstrLimObj}

In this section, we prove Theorem \ref{CompactnesMultiActConv}. This proof is a generalization of the proof of Theorem 2.9 in \cite{backhausz2018action} to the multi-linear case. Let $\left\{\left(\Omega_{i}, \mathcal{A}_{i}, \mu_{i}\right)\right\}_{i=1}^{\infty}$ be a sequence of probability spaces and assume that $\left\{A_{i} \in \mathcal{B}_{p_1,\ldots,p_{r-1}, q}\left(\Omega_{i}\right)\right\}_{i=1}^{\infty}$ is a Cauchy convergent sequence of $P$-operators such that $\sup_i \left\|A_{i}\right\|_{p_1,\ldots, p_{r-1}\rightarrow q} \leq c$ for every $i\in \N$ for some $c \in \mathbb{R}^{+}$. For every $k \in \mathbb{N}$, we define
$$
X_{k}:=\lim _{i \rightarrow \infty} cl(\mathcal{S}_{k}\left(A_{i}\right))
$$
Our goal is to construct a multi-$P$-operator such that its $k$-profile is the limit of the $k$-profiles of the operators in a given convergent sequence of operators for every fixed $k$, i.e.\ we will prove that there is a $P$-operator $A \in \mathcal{B}_{p_1,\ldots,p_{r-1}, q}(\Omega)$ for some probability space $(\Omega, \mathcal{A}, \mu)$ such that for every $k \in \mathbb{N}$ we have that
$$
\lim _{i \rightarrow \infty} cl(\mathcal{S}_{k}\left(A_{i}\right))=cl(\mathcal{S}_{k}(A)) .
$$
Before the technical proof, we explain the main idea. For every $k\in \N$ we consider the limit of the $k$-profiles $S_k(A_i)$ of the sequence of operators $A_i$, which is a set of measures, and we take a dense countable subset of this set. In this way, we have that each point in this dense subset can be approximated by elements in the $k$-profiles of the sequence of operators $A_{i}$. Moreover, every element in the $k$-profile of $A_{i}$ involves $r k$ measurable functions on $\Omega_{i}$ (in the terminology used before the measure is generated through those functions). In probabilistic language, these functions are random variables, since $\Omega_{i}$ is a probability space. Very roughly speaking, the main idea is to take, for every $k$, enough functions needed to generate enough measures (contained in the $k-$profiles of the operators $A_{i}$) to approximate a dense countable subset of the limiting $k-$profile. These are countably many functions for each $i$. By passing to a subsequence, we can assume that the joint distributions of these countably many functions (random variables) converge weakly and the limit is some probability distribution on $\Omega:=\mathbb{R}^{\infty}$. %As a first attempt, we could try to produce the limiting operator on the function space of this probability space. 
Each coordinate function in the probability space on $\mathbb{R}^{\infty}$ corresponds to a function involved in a $k$-profile for some $k$. Since every $k$-profile comes from $(r-1)k$ functions and their $k$ images, we obtain some information on a possible limiting operator. More precisely, we obtain that certain coordinate functions are the images of some other coordinate functions under the action of the candidate limit multi-linear operator. However, it is not clear that it is possible to extend the obtained multi-linear operator to the full function space on $\Omega$ and so we need to refine the above idea. 

We now make the above idea rigorous. %Instead of just working with functions involved in profile points, 
We need to work with enough functions to represent the function space of a whole $\sigma$-algebra to be able to extend the candidate limit multi-linear operator to the whole function space on $\Omega$. To do this, we extend the above function systems by new functions obtained by some natural operations. In order to do this, we introduce an abstract algebraic formalism involving semigroups. The main challenge in the proof is to show that, at the end of this process, we obtain a well-defined operator with the desired limiting properties.

First, we will need the next algebraic notion.
\begin{definition}[Free semigroup with $r-$multi-operators]\label{FreeSemigroupDef}
Let $G$ and $L$ be sets. We denote by $F(G, L)$ the free semigroup with generator set $G$ and $r-$multioperator set $L$ (freely acting on $F(G, L))$. More precisely, we have that $F(G, L)$ is the smallest set of abstract words satisfying the following properties.
(1) $G \subseteq F(G, L)$.
(2) If $w_{1}, w_{2} \in F(G, L)$, then $w_{1} w_{2} \in F(G, L)$.
(3) If $w_1,\ldots, w_{r-1} \in F(G, L), l \in L$, then $l(w_1,\ldots,w_{r-1}) \in F(G, L)$. There is a unique length function $m: F(G, L) \rightarrow \mathbb{N}$ such that $m(g)=1$ for $g \in G$, $m\left(w_{1} w_{2}\right)=m\left(w_{1}\right)+m\left(w_{2}\right)$ and $m(l(w_1,\ldots,w_{r-1}))=\max_{s\in [r-1] }m(w_s)+1 .$
\end{definition} 

An example for a word in $F(G, L)$ with $L$ set of  $2-$multioperators is $l_{3}\left(l_{1}\left(g_{1} ,g_{2}\right), l_{2}\left(g_{2} ,g_{2}g_{3}\right)\right) l_{3}\left(g_{1},g_{2}\right) $, where $g_{1}, g_{2}, g_{3} \in G$ and $l_{1}, l_{2}, l_{3} \in L$. The length of this word is $\max\{\max\{1,1\}+1,\max\{1,1+1\}+1\}+1+\max\{1,1\}+1=6$. Note that if both $G$ and $L$ are countable sets, then so is $F(G, L)$.\newline

Construction of a function system: In this technical part of the proof, we construct a function system $\left\{v_{i, f} \in L^{\infty}\left(\Omega_{i}\right)\right\}, i \in \mathbb{N}, f \in F$ for some countable index set $F$. Later, we will construct a probability distribution $\kappa \in \mathcal{P}\left(\mathbb{R}^{F \times[r]}\right)$ and an operator $A \in \mathcal{B}_{p_1,\ldots,p_{r-1}, q}\left(\mathbb{R}^{F \times[r]}, \kappa\right)$ using this function system. In the end, we will show that $A$ is an appropriate limit object for the sequence $\left\{A_{i}\right\}_{i=1}^{\infty}$.\newline

First, we define the index set $F$. For every $k \in \mathbb{N}$, let $X_{k}^{\prime} \subseteq X_{k}$ be a countable dense subset in the metric space $\left(X_{k}, d_{\mathrm{LP}}\right)$. Let $G:=\bigcup_{k=1}^{\infty} X_{k}^{\prime} \times[k]\times [r-1]$. The index set $F$ will be the free semigroup generated by $G$ and an appropriate set of nonlinear $(r-1)-$multi-operators on function spaces. For $y \in \mathbb{Q}$ and $z \in \mathbb{Q}^{+}$ let $h_{y, z}: \mathbb{R} \rightarrow \mathbb{R}$ be the bounded, continuous function defined by $h_{y, z}(x)=0$ if $x \notin(y-z, y+z)$ and $h_{y, z}(x)=1-|x-y| / z$ if $x \in$ $(y-z, y+z)$. For every $i \in \mathbb{N}, l \in L$ and $v_1,\ldots, v_{r-1} \in L^{\infty}\left(\Omega_{i}\right)$ we define $l(v_1,\ldots,v_{r-1}):=h_{y, z} \circ\left( A_{i}[v_1,\ldots,v_{r-1}]\right)$, where $l$ is given by the pair $(y, z) \in \mathbb{Q} \times \mathbb{Q}^{+}$. Observe that by definition, $\|l(v_1,\ldots,v_{r-1})\|_{\infty} \leq 1$. These functions can be naturally identified with $\mathbb{Q} \times \mathbb{Q}^{+}$, therefore, we will denote $L=\mathbb{Q} \times \mathbb{Q}^{+}$. Furthermore, let $F:=F(G, L)$ be as in Definition \ref{FreeSemigroupDef}. We have that $F$ is countable.
Now we describe the functions $\left\{v_{i, g}\right\}_{i \in \mathbb{N}, g \in G}$. For every $i, k \in \mathbb{N}$, and $t \in X_{k}^{\prime}$ let $\left\{v_{i,(t, j,s)}\right\}_{j\in [k],s\in [r-1]}$ be functions in $L_{[-1,1]}^{\infty}\left(\Omega_{i}\right)$ such that the joint distribution of
$$
\begin{aligned}
&(v_{i,(t, 1,1)},\ldots,v_{i,(t, 1,(r-1))},A_{i}[v_{i,(t, 1,1)},\ldots, v_{i,(t, 1,(r-1))}], v_{i,(t, 2,1)},\ldots,\\
&v_{i,(t, k,1)},\ldots,v_{i,(t, k,(r-1))},A_{i}[v_{i,(t, k,1)},\ldots, v_{i,(t, k,({r-1}))}])
\end{aligned}
$$%v_{i,(t, 2)}, %\ldots, v_{i,(t, k)}, v_{i,(t, 1)} A_{i}, v_{i,(t, 2)} A_{i}, \ldots, v_{i,(t, k)} A_{i}\right)
converges to $t$ as $i$ goes to infinity.

Now we construct the functions $\left\{v_{i, w}\right\}_{i \in \mathbb{N}, w \in F}$ recursively to the length of $m(w)$. For words of length 1, the functions have been constructed above. Assume that we have already constructed all the functions $v_{i, w}$ with $m(w) \leq k$ for some $k \in \mathbb{N}$. Let $w \in F$ such that $m(w)=k+1$. If $w=w_{1} w_{2}$ for some $w_{1}, w_{2} \in F$, then $v_{i, w}:=v_{i, w_{1}} v_{i, w_{2}}$. If $w=l\left(w_{1},w_2,\ldots,w_{(r-1)}\right)$, then $v_{i, w}:=l\left(v_{i, w_{1}},v_{i, w_{2}},\ldots,v_{i, w_{r-1}}\right)$.

Construction of the probability space: Let $\xi_{i}: \Omega_{i} \rightarrow \mathbb{R}^{F^{(r-1)} \times[r]}$ be the function such that for $f_1,\ldots,f_{(r-1)} \in F, e \in[r]$, and $\omega_{i} \in \Omega_{i}$ the $(f_1,\ldots,f_{(r-1)}, e)$ coordinate of $\xi_{i}\left(\omega_{i}\right)$ is equal to $$\left(A_{i}^{e}[v_{i, f_1},\ldots, v_{i, f_{(r-1)} }]\right)\left(\omega_{i}\right),$$ where $A_{i}^{s}$ for $s\in [r-1]$ is defined to be the projection on the $s-$th variable and $A_i^r=A_i$. Let $\kappa_{i} \in \mathcal{P}\left(\mathbb{R}^{F^{(r-1)} \times [r]}\right)$ denote the distribution of the random variable $\xi_{i}$, i.e.\ $\kappa_{i}$ is the joint distribution of the functions $\left\{v_{i, f_1}\right\}_{f_{1} \in F},\ldots,\left\{v_{i, f_{r-1}}\right\}_{f_{r-1} \in F}$ and $\left\{A_{i}[v_{i, f_1},\ldots ,v_{i, f_{r-1}} ]\right\}_{f_1,\ldots,f_{(r-1)} \in F}$. Since $\tau\left(\kappa_{i}\right) \leq c$ holds (recall equation \eqref{eqn:tau} for the definition of $\tau$), there exists a strictly growing sequence $\left\{n_{i}\right\}_{i=1}^{\infty}$ of natural numbers such that $\kappa_{n_{i}}$ is weakly convergent with limit $\kappa$ as $i$ goes to infinity. Let $\Omega:=\mathbb{R}^{F^{(r-1)} \times[r]}$ be the probability space with the Borel $\sigma$-algebra $\mathcal{A}$ and probability measure $\kappa$. We will consider $\Omega$ as a topological space, equipped with the product topology. Moreover, $\kappa$ is a probability measure as it is the weak limit of the probability distributions $\kappa_{i_{n}}$.

Construction of the operator: We now define an operator $A \in \mathcal{B}_{p_1,\ldots,p_{(r-1)}, q}(\Omega)$ with $\Omega$ defined above. For $(f_1,\ldots,f_{r-1}, e) \in F^{(r-1)} \times [r]$ we denote with $\pi_{(f_1,\ldots,f_{(r-1)}, e)}: \mathbb{R}^{F^{(r-1)} \times [r]} \rightarrow \mathbb{R}$ the projection function to the $(f_1,\ldots,f_{(r-1)}, e)$ coordinate. Notice that \begin{equation}\label{eqn:EqoperatorLin}
\pi_{(f_1,\ldots,f_{(r-1)}, e)} \circ \xi_{i}= A_{i}^{e}[v_{i, f_1},\ldots ,v_{i, f_{r-1}} ] \quad(i \in \mathbb{N},(f_1,\ldots,f_{(r-1)}, e) \in F^{(r-1)}) \times[r] .\end{equation}
In particular, by the definition of $\kappa$, we also have $\pi_{(f_1,\ldots,f_{(r-1)}, e)} \in L_{[-1,1]}^{\infty}(\Omega)$ for $f_1,\ldots,f_{(r-1)} \in F$ and $e\in [r-1]$. Our goal is now to show that there is a unique $(p_1,\ldots,p_{(r-1)}, q)$-bounded linear operator $A$ from $L^{\infty}(\Omega)\times \ldots \times L^{\infty}(\Omega)$ to $L^{1}(\Omega)$ with $\|A\|_{p_1,\ldots,p_{(r-1)} \rightarrow q} \leq c$ such that $\pi_{(f_1,\ldots, f_{r-1}, e)} A=\pi_{(\cdot,\ldots ,l(f_1,\ldots, f_{r-1}),\ldots,\cdot, r)}$ holds for every $e\in [r-1]$ and $f_1,\ldots, f_{(r-1)} \in F$.

\begin{lemma}\label{LemmaPropertiesCoordinateFunct}
 The coordinate functions on $\mathbb{R}^{F^{(r-1)} \times[r]}$ have the following properties.
 \begin{enumerate}
\item  If $e\in [r-1]$ and $f_{1}, f_{2} \in F$, then $\pi_{\left(\cdot,\ldots,\cdot ,f_{1} f_{2},\cdot,\ldots,\cdot,  e\right)}=\pi_{\left(\cdot,\ldots,\cdot ,f_{1},\cdot,\ldots,\cdot , e\right)} \pi_{(\cdot,\ldots,\cdot ,f_{2},\cdot,\ldots,\cdot ,e)}$ holds in $L^{\infty}(\Omega)$.
\item  If $f \in F$ and $l=(y, z)$ holds for some $y, z$, then $\pi_{(\cdot,\ldots,\cdot ,l(f_1,\ldots,f_{r-1}),\cdot,\ldots,\cdot, e)}=h_{y, z} \circ \pi_{(f_1,\ldots,f_{(r-1)}, r)}$ holds in $L^{\infty}(\Omega)$.
\item  If $a^{(1)}_{s}, a^{(2)}_{s}, \ldots, a_{s}^{(d_s)} \in F, \lambda_s^{(1)}, \lambda_s^{(2)}, \ldots, \lambda_s^{(d_s)} \in \mathbb{R}$, for every $s\in [r-1]$ then
$$\begin{aligned}
&\left\|\sum_{j_1,\ldots j_{(r-1)}=1}^{d_1,\ldots,d_{(r-1)}} \lambda_1^{(j_1)} \lambda_2^{(j_2)}\ldots \lambda_{(r-1)}^{(j_{(r-1)})} \pi_{\left(a^{(j_1)}_{1}, a^{(j_2)}_{2}, \ldots, a_{(r-1)}^{(j_s)}, r\right)}\right\|_{q} \leq \\
& c\left\|\sum_{j_1=1}^{d_1} \lambda^{(j_1)} \pi_{\left(a^{(j_1)}_{1},\cdot, \ldots, \cdot, 1\right)}\right\|_{p_1}\ldots\left\|\sum_{j_{(r-1)}=1}^{d_{(r-1)}} \lambda^{(j_{(r-1)})} \pi_{\left(\cdot, \ldots, \cdot,a^{(j_{r-1})}_{1}, r-1\right)}\right\|_{p_{r-1}}
\end{aligned}
$$
\item  For all $e\in [r-1]$, the linear span of the functions $\left\{\pi_{(\cdot,\ldots , \cdot,f,\cdot, \ldots,\cdot, e)}\right\}_{f \in F}$ is dense in the space $L^{p_e}(\Omega)$.
\item  Assume that $k \in \mathbb{N}$ and $t \in X_{k}^{\prime}$. Then $(t, j) \in G \subset F$ holds for $1 \leq j \leq k$ and
\begin{equation*}\begin{aligned}
& \mathcal{L}\left(\pi_{((t, 1,1),\ldots, 1)}, \pi_{((t, 2,1),\ldots, 1)}, \ldots, \pi_{((t, k,1),\ldots,1)}, \pi_{(\cdot,(t, 1,2),\ldots, 2)}, \pi_{(\cdot, (t, 2,2),\ldots, 2)}, \ldots, \pi_{(\cdot,(t, k,2),\ldots, 2)},\right. \\ & \left.
 \ldots, \pi_{(\ldots,(t, k,r-1), r-1)},\ldots,\pi_{((t, k,1),\ldots,(t, k,r-1), r)}\right)=t
 \end{aligned}
\end{equation*}

 \end{enumerate}
\end{lemma}
\begin{remark} When functions on $\Omega$ are treated as functions in $L^r(\Omega)$ for some $r\in [1,\infty]$, they are identified if they differ on a set of measure zero. This standard identification of functions allows the correspondence between different coordinate functions. For example let us consider the uniform measure $\mu$ on $\{(x,x):x\in [0,1]\}$ which is a Borel measure on $\mathbb{R}^2$. The $x$-coordinate function $(x,y)\mapsto x$ and the $y$-coordinate function $(x,y)\mapsto y$ coincide in the space $L^r(\mathbb{R}^2,\mu)$, as they agree on the support of $\mu$. We will heavily exploit this fact in the rest of our proof.
\end{remark}

For the proof of Lemma \ref{LemmaPropertiesCoordinateFunct} we will need the following two lemmas.
\begin{lemma}[Lemma 4.3 in \cite{backhausz2018action}]
 Let $r \in[1, \infty)$. For every $v \in L^{r}(\Omega)$ we have that
$$
\lim _{n \rightarrow \infty}\left\|v-\sum_{j=-n^{2}}^{n^{2}}(j / n) h_{j / n, 1 / n} \circ v\right\|_{\mathrm{r}}=0 .
$$\end{lemma}

The following lemma is easy to prove, see, e.g., Theorem 22.4 in the lecture notes \cite{driver2004analysis}.

\begin{lemma}
Let $r \in[1, \infty)$. Let $\left\{v_{i} \in L^{\infty}(\Omega)\right\}_{i \in I}$ be a system of functions for some countable index set I such that for every $a, b \in I$ there is $c \in I$ with $v_{a} v_{b}=v_{c}$. Let $\mathcal{A}_{0}$ be the $\sigma$-algebra generated by the functions $\left\{v_{i}\right\}_{i \in I}$. Suppose that the constant 1 function on $\Omega$ can be approximated by a uniformly bounded family of finite linear combinations of $\left\{v_{i}\right\}_{i \in I}$. Then the $L^{r}$-closure of the linear span of $\left\{v_{i} \in L^{\infty}(\Omega)\right\}_{i \in I}$ is $L^{r}\left(\Omega, \mathcal{A}_{0}, \kappa\right)$.
\end{lemma} 

Now we return to the proof of Lemma \ref{LemmaPropertiesCoordinateFunct}.

\proof To show the first statement of the lemma we observe that, by the construction of the function system, we have for every $i \in \mathbb{N}$ and $f_{1}, f_{2} \in F$ that $v_{i, f_{1} f_{2}}=v_{i, f_{1}} v_{i, f_{2}}$ holds. Therefore, by equation \eqref{eqn:EqoperatorLin} and the continuity of $\pi$, it follows that each $\kappa_{i}$ is supported inside the closed set
$$
\bigcap_{e\in [r-1] }\left\{\omega: \omega \in \mathbb{R}^{F \times[r-1]}, \pi_{\left(\ldots,f_{1} f_{2},\ldots, e\right)}(\omega)=\pi_{\left(\ldots,f_{1},\ldots, e\right)}(\omega) \pi_{\left(\ldots,f_{2},\ldots, e\right)}(\omega)\right\} .
$$
Therefore, $\kappa$ is also supported inside this set and thus $\pi_{\left(\ldots,f_{1} f_{2},\ldots, e\right)}=\pi_{\left(\ldots,f_{1},\ldots, e\right)} \pi_{\left(\ldots,f_{2},\ldots, e\right)}$ holds $\kappa$-almost everywhere for every $e\in [r-1]$.

The proof of the second statement is similar to the first one. Again, by the construction of the function system, we have for every $i \in \mathbb{N}$ and $f \in F, l=(p, q) \in L$ that $\left.l(f_1,\ldots,f_{r-1}\right)=l\left(v_{i, f_1},\ldots,v_{i, f_{r-1}}\right)=h_{p, q} \circ\left( A_{i}[v_{i,f_1},\ldots, v_{i,f_{r-1}}]\right)$. This means by the definition of $\kappa_{i}$, equation \eqref{eqn:EqoperatorLin} and the continuity of $\pi$ that $\kappa_{i}$ is supported on the closed set
$$
\bigcap_{e\in [r-1] } \left\{\omega: \omega \in \mathbb{R}^{F \times [r-1]}, \pi_{(\ldots, l(f_1,\ldots,f_{r-1}),\ldots, e)}(\omega)=h_{p, q}\left(\pi_{(f_1,\ldots,f_{r-1}, r)}(\omega)\right)\right\}
$$
for every $i \in \mathbb{N}$. Thus, $\pi_{(\ldots,l(f_1,\ldots, f_{r-1}),\ldots, e)}=h_{p, q} \circ \pi_{(f_1,\ldots,f_{r-1}, r)}$ holds $\kappa$-almost everywhere for every $e\in [r-1]$.


To show the third statement, we recall that $\left\|A_{i}\right\|_{p_1,\ldots, p_{r-1} \rightarrow q} \leq c$ holds for every $i \in \mathbb{N}$ and thus
$$
\left\|\sum_{j_1,\ldots, j_{(r-1)}=1}^{d_1,\ldots,d_{(r-1)}} \lambda_1^{j_1}\ldots\lambda_{(r-1)}^{j_{(r-1)}} A_{i}[v_{i, a_{j_1}},\ldots, v_{i, a_{j_{(r-1)}}}]\right\|_{q} \leq c\left\|\sum_{j_1=1}^{d_1} \lambda_1^{j_1} v_{i, a_{j_1}}\right\|_{p_1}\ldots\left\|\sum_{j_{(r-1)}=1}^{d_{(r-1)}} \lambda_{(r-1)}^{j_{(r-1)}} v_{i, a_{j_{r-1}}}\right\|_{p_{r-1}}
$$The sums in the multiplicands on the right-hand side are functions in $L^{\infty}\left(\Omega_{i}\right)$ whose values for the respective $e\in [r-1]$ are in the compact intervals $[-\lambda^{(e)}, \lambda^{(e)}]$ for $\lambda:=\sum_{j_e=1}^{d_e}\left|\lambda^{(e)}_{j_e}\right|$, therefore, we obtain that $\sum_{j_e=1}^{d_e} \lambda^{(e)}_{j_e} \pi_{\left(\ldots,a_{j_e},\ldots , e\right)}$ is a bounded, continuous function on the support of $\kappa$. Therefore, using that  $\kappa_{i} $ converges to $\kappa$ weakly and equation \eqref{eqn:EqoperatorLin} again (in particular, integrating the $p-$th power of the absolute values with respect to $\kappa_{i}$), we obtain that
$$
\lim _{i \rightarrow \infty}\left\|\sum_{j_e=1}^{d_e} \lambda^e_{j_e} v_{i, a_{j_e}}\right\|_{p_e}=\left\|\sum_{j_e=1}^{d_e} \lambda^e_{j_e} \pi_{\left(\ldots,a_{j_e},\ldots, e\right)}\right\|_{p_e}.
$$
But on the other hand, $\left|\sum_{j_1,\ldots,j_{(r-1)}=1}^{d_1,\ldots,d_{(r-1)}} \lambda^1_{j_1} \ldots \lambda^{(r-1)}_{j_{(r-1)}}\pi_{(a_1,\ldots,a_{(r-1)}, r)}\right|^{q}$ is a continuous non-negative function, thus weak convergence implies the following inequality:
$$
\left\|\sum_{j_1,\ldots,j_{(r-1)}=1}^{d_1,\ldots,d_{(r-1)}} \lambda^1_{j_1} \ldots \lambda^{(r-1)}_{j_{(r-1)}}\pi_{(a_1,\ldots,a_{(r-1)}, r)}\right\|_{q} \leq \limsup _{i \rightarrow \infty}\left\|\sum_{j_1,\ldots,j_{(r-1)}=1}^{d_1,\ldots,d_{(r-1)}}\lambda^1_{j_1} \ldots \lambda^{(r-1)}_{j_{(r-1)}}A_{i}[v_{i, a_{j_1}},\ldots, v_{i, a_{j_{(r-1)}}}]\right\|_{q} .
$$These inequalities together yield the third statement.
For the fourth statement, let $\mathcal{H}^{(e)}_{s}$ denote the $L^{s}$-closure of the linear span of the functions $\left\{\pi_{(f_1,\ldots,f_{(r-1)}, e)}\right\}_{f_1,\ldots, f_{(r-1)} \in F}$ for $e\in [r-1]$  and $s \in[1, \infty)$. 

First of all we notice that
$$
\pi(f,\ldots,1) =\pi(\ldots,f,\ldots,e)=\pi(\ldots,f,r-1)
$$for all $f\in F$ and $e\in [r-1]$ and, therefore, $\mathcal{H}^{(1)}_{s}=\ldots=\mathcal{H}^{(e)}_{s}=\ldots=\mathcal{H}^{(r-1)}_{s}$. From now on we will write $\mathcal{H}_s=\mathcal{H}^{(e)}_s$ as it does not depend on $e$.

Now we prove that $\pi_{(f_1,\ldots,f_{(r-1)}, r)} \in \mathcal{H}_{q}$ holds for every $f_1,\ldots, f_{(r-1)} \in F$. By the second statement of the lemma, we have that
 \begin{equation}\label{eqn:stepCompEqual}
\sum_{j=-n^{2}}^{n^{2}}(j / n) h_{j / n, 1 / n} \circ \pi_{(f_1,\ldots, f_{(r-1)}, r)}=\sum_{j=-n^{2}}^{n^{2}}(j / n) \pi_{(\ldots,l_j(f_1,\ldots,f_{(r-1)}), e)},
\end{equation}
where $l_{j}$ is given by the pair $(j / n, 1 / n)$ for $-n^{2} \leq j \leq n^{2}$. 

Since the right-hand side of \eqref{eqn:stepCompEqual} is in $\mathcal{H}_{q}$, we notice that the left-hand side is in $\mathcal{H}_{q}$ too. On the other hand, $\pi_{(f_1, \ldots,f_{(r-1)},r)} \in L^{q}(\Omega)$ by the third statement. Hence by Lemma \ref{LemmaPropertiesCoordinateFunct}, we have that,
as $n$ goes to infinity, the left-hand side of \eqref{eqn:stepCompEqual} converges to $\pi_{(f_1,\ldots,f_{(r-1)}, r)}$ in $L^{q}(\Omega)$ and thus $\pi_{(f_1,\ldots,f_{r-1}, r)} \in \mathcal{H}_{q} .$

Let $\mathcal{A}_{0}$ be the $\sigma$-algebra generated by the functions $\left\{\pi_{(f_1,\ldots,f_{r-1}, e)}\right\}_{f_1,\ldots,f_{r-1} \in F}$ for a fixed $e\in [r-1]$. Observe that the constant function $1$ on $\Omega$ can be approximated already in $X_{1}^{\prime}$. Therefore, we obtain by the first statement in this lemma and Lemma \ref{LemmaPropertiesCoordinateFunct} that $\mathcal{H}_{r}=L^{r}\left(\Omega, \mathcal{A}_{0}, \kappa\right)$ holds for every $e\in [r-1]$ and $r \in$ $[1, \infty)$. As we have shown, we have for every $f_1,\ldots, f_{r-1}\in F$ that $\pi_{(f_1,\ldots,f_{r-1}, r)} \in \mathcal{H}_{q}=L^{q}\left(\Omega, \mathcal{A}_{0}, \kappa\right)$ and thus all coordinate functions on $\mathbb{R}^{F^{r-1} \times[r]}$ are measurable in $\mathcal{A}_{0}$. This shows that $\mathcal{H}_{r}=L^{r}\left(\Omega, \mathcal{A}_{0}, \kappa\right)=L^{r}(\Omega, \mathcal{A}, \kappa)=L^{r}(\Omega)$ holds for every $r \in[1, \infty)$.

The last statement of the lemma follows directly from the definition of $\kappa$ and the definition of the functions $\left\{v_{i,(t, j,e)}\right\}_{i \in \mathbb{N}, j \in[k]}$.
\endproof

We will need also the following lemma to prove the existence of the multi-$P-$operator. This is the multi-linear version of a classical result about the extension of linear bounded operators defined on a dense set.

\begin{lemma}\label{LemmMultBoundExt}
Let $V_1,\ldots,V_r$ and $F$ be Banach spaces and $W_1,\ldots, W_r$ where, for every $i\in [r]$, $W_i$ is a dense subspace of $V_i$.
For a multi-linear bounded operator
$$
  T_0:W_1\times \ldots \times W_r \longrightarrow F     
$$
$$
(x_1,\ldots,x_r)\mapsto T[x_1,\ldots,x_r]
$$
there exists a unique multi-linear bounded operator 

$$
  T:V_1\times \ldots \times V_r \longrightarrow F     
$$

and 
$$
\|T_0\|=\|T\|
$$
\end{lemma}
\proof
For every $(x_1,\ldots,x_r)\in V_1\times \ldots \times V_r$ we define
$$T[x_1,\ldots,x_r ]=\lim_{n\rightarrow \infty}T[x_{1,n},\ldots,x_{r,n}]$$
where $(x_{1,n},\ldots,x_{r,n})\rightarrow (x_1,\ldots,x_r)$ as $n\rightarrow \infty$. We show that this definition is independent of the sequence we choose. We consider two sequences

$$
(x_{1,n},\ldots,x_{r,n})\rightarrow (x_1,\ldots,x_r)
$$
$$
(y_{1,n},\ldots,y_{r,n})\rightarrow (x_1,\ldots,x_r)
$$

$$
\begin{aligned}
&\|T[x_{1,n},\ldots,x_{r,n}]-T[y_{1,n},\ldots,y_{r,n}]\|\leq\\
& \|T[x_{1,n},\ldots,x_{r,n}]-T[y_{1,n},y_{r-1,n},\ldots,x_{r,n}]+\ldots + T[x_{1,n},y_{2,n},\ldots,y_{r,n}]-T[y_{1,n},\ldots,y_{r,n}]\|\leq\\
&
C \sum^r_{i=1}\left(\prod^{i-1}_{j=1}\|x_{j,n}\|\right)\|x_{i,n}-y_{i,n}\|\left(\prod^{r}_{j=i}\|y_{j,n}\|\right)\leq \\
&
K\sum^r_{i=1}\|x_{i,n}-y_{i,n}\|\rightarrow 0
\end{aligned}
$$
as $n\rightarrow 0$. Moreover,
$$
\begin{aligned}
&\|T_0\|=\sup_{x_1\in W_1,\ldots, x_r\in W_r, \
x_1,\ldots,x_r\neq 0}\frac{\|T[x_1,\ldots,x_r]\|}{\|x_1\|\ldots\|x_r\|}=\\
&\sup_{x_1\in V_1,\ldots, x_r\in V_r, \
x_1,\ldots,x_r\neq 0}\frac{\|T[x_1,\ldots,x_r]\|}{\|x_1\|\ldots\|x_r\|}=\|T\|&
\end{aligned}
$$

as the sets $W_i$ are dense in $V_i$ and therefore $W_1\times \ldots \times W_r$ is dense in $V_1\times \ldots \times V_r$.


%$$
%\begin{aligned}
%&\|T[x_1,\ldots, x_r]\|=\lim_{n\rightarrow\infty}\|T[x_{1,n},\ldots,x_{r,n}]\|\leq\\
%&
%\end{aligned}
%$$

\endproof

We now finally define the limit operator $A \in \mathcal{B}_{p_1,\ldots,p_{r-1}, q}(\Omega)$. For $f_1,\ldots,f_{r-1} \in F$, let 
$$ A[\pi_{(f_1,\ldots, 1)},\ldots,\pi_{(\ldots,f_e,\ldots, e)},\ldots, \pi_{(\ldots,f_{r-1}, r-1)}]=\pi_{(f_1,\ldots,f_{r-1},r)}.$$This defines a multi-linear operator on the linear span of $\left\{\pi_{(f_1,\ldots,f_{r-1}, e)}\right\}_{f_1,\ldots,f_{r-1} \in F}$, which is bounded due to the third statement of Lemma \ref{LemmaPropertiesCoordinateFunct}. 
Hence it has a unique continuous linear extension on its $L^{p_1}\times \ldots \times L^{p_{r-1}}$-closure. 
By the fourth statement of Lemma \ref{LemmaPropertiesCoordinateFunct} and Lemma \ref{LemmMultBoundExt}, we get that there is a unique operator $A \in \mathcal{B}_{p_1,\ldots,p_{r-1}, q}(\Omega)$ with $\|A\|_{p_1,\ldots,p_{r-1} \rightarrow q} \leq c$ such that $ A[\pi_{(f_1,\ldots, 1)},\ldots,\pi_{(\ldots,f_{r-1} ,r-1)}]=\pi_{(f_1,\ldots,f_{r-1}, r)}$ holds for every $f_1,\ldots,f_{r-1} \in F$.\newline
Last part of the proof: From the last statement of Lemma \ref{LemmaPropertiesCoordinateFunct} together with the equality $$A[\pi_{((t, j,1),\ldots, 1)},\ldots, \pi_{(\ldots, (t, j,r-1), r-1)}] =\pi_{((t, j, 1),\ldots,(t, j, r-1), r)}$$we obtain that for every $k \in \mathbb{N}$ and $t \in X_{k}^{\prime}$ we have $t \in \mathcal{S}_{k}(A)$. Therefore, for every $k \in \mathbb{N}$ we observe that $X_{k} \subseteq cl(\mathcal{S}_{k}(A))$. Our goal is to show that $X_{k}=cl(\mathcal{S}_{k}(A))$ for every $k \in \mathbb{N}$ and thus we still need to prove that $cl(\mathcal{S}_{k}(A) )\subseteq X_{k}$. Let $k \in \mathbb{N}$ and let $v_{1,1}, v_{2,1}, \ldots, v_{k,1}, \ldots ,v_{1,r-1}, v_{2,r-1}, \ldots, v_{k,r-1} \in L_{[-1,1]}^{\infty}(\Omega)$. We aim to show that $$\alpha:=\mathcal{D}_{A}\left(\left\{v_{j,s}\right\}_{j\in[k],s\in [r-1]}\right)\in X_{k}.$$ Let $\varepsilon>0$ be arbitrary. It follows by the fourth statement of Lemma \ref{LemmaPropertiesCoordinateFunct} that for some large enough natural number $m$ there are elements $f_{1,s}, f_{2,s}, \ldots, f_{m,s} \in F$ and real numbers $\left\{\lambda_{a, j,s}\right\}_{a \in[m], j \in[k] , s\in [r-1]}$ such that for every $j \in[k]$ we have $\left\|w_{j,s}-v_{j,s}\right\|_{p} \leq \varepsilon$, where $w_{j,s}:=$ $\sum_{a=1}^{m}$\ $ \lambda_{a, j,s}$  $ \pi_{\left(\ldots, f_{a,s},\ldots, s\right) }$ for $j \in[k], s\in [r-1]$.

We recall that only vectors with $\infty-$norm at most $1$ are considered in the profiles. For this reason, we will need to use a truncating function. Let $\tilde{h}: \mathbb{R} \rightarrow \mathbb{R}$ be the continuous function with $\tilde{h}(x)=x$ for $x \in[-1,1], \tilde{h}(x)=-1$ for $x \in(-\infty,-1]$ and $\tilde{h}(x)=1$ for $x \in[1, \infty)$. Since $\left\|v_{j}\right\|_{\infty} \leq 1$ holds for $j \in[k]$, it follows that $\left|w_{j,s}(x)-v_{j,s}(x)\right| \geq\left|\tilde{h} \circ w_{j,s}(x)-v_{j,s}(x)\right|$ holds almost everywhere. Thus by $\left\|w_{j,s}-v_{j,s}\right\|_{p} \leq \varepsilon$, we obtain $\left\|\tilde{h} \circ w_{j,s}-v_{j,s}\right\|_{p} \leq \varepsilon$ for $j \in [k]$. Therefore,  by the triangle inequality
\begin{equation}\label{eqn:EqIneqLp}
\left\|\tilde{h} \circ w_{j,s}-w_{j,s}\right\|_{p} \leq\left\|\tilde{h} \circ w_{j,s}-v_{j,s}\right\|_{p}+\left\|v_{j,s}-w_{j,s}\right\|_{p} \leq 2 \varepsilon\end{equation}
holds for $j \in[k],s\in[r-1]$.
For $i \in \mathbb{N}, \ e\in [r-1]$ and $j \in[k]$ let $z_{i, j,s}:=\sum_{a=1}^{m} \lambda_{a, j,s} v_{i, f_{a,j,s}}$. Let $$\beta_{i}:=\mathcal{D}_{A_{i}}\left(\left\{z_{i, j,s}\right\}_{j\in [k], s\in [r-1]}\right).$$ By the definition of $\kappa$ and the properties of convergence in distribution of random vectors (linear combinations of entries converges in distribution to the same linear combination of the entries of the limit random vector), we have that
$$
\beta:=\lim _{i \rightarrow \infty} \beta_{n_{i}}=\mathcal{D}_{A}\left(\left\{w_{j,s}\right\}_{j\in [k],s\in [r-1]]}\right)
$$
holds in $d_{\text {LP  }}$. Since  $w_{j,s} \in L^{\infty}(\Omega)$ we have $$\begin{aligned}
&\left\| A[v_{j,1},\ldots, v_{j,r-1}]- A[w_{j,1}, \ldots , w_{j,r-1}]\right\|_{1} \leq\left\| A[v_{j,1},\ldots, v_{j,r-1}]- A[w_{j,1}, \ldots , w_{j,r-1}]\right\|_{q} \leq \\
&\sum^{r-1}_{e=1}c \left(\prod^{e-1}_{s=1}\| v_{j,s}\|_{p_s}\right)\| v_{j,e}-w_{j,e} \|_{p_e}\left(\prod^{r-1}_{s=e+1}\| w_{j,s}\|_{p_s}\right)\\
%&\leq ........c (r-1)\max_{s\in [r-1]}\{\|v_{j,s}\|_{p_s}\}\max_{s\in [r-1]}\{\|w_{j,s}\|_{p_s}\}\varepsilon............
%\\
& \leq c (r-1)\max_{s\in [r-1]}\ \{\|v_{j,s}\|_{p_s}+\varepsilon\}^{2(r-1)}\varepsilon\\
& \leq c (r-1)\{1+\varepsilon\}^{2(r-1)}\varepsilon\leq C\ \varepsilon\end{aligned} $$
where the second last inequality follows from $\|v_{j,s}\|_{\infty}\leq 1$.
From Lemma \ref{coupdist2} we get that $d_{\mathrm{LP}}(\alpha, \beta) \leq(r k)^{3 / 4}\left(C^{\prime} \varepsilon\right)^{1 / 2}$, where $C^{\prime}:=$ $\max (C, 1)$.
Let
$$
\beta_{i}^{\prime}:=\mathcal{D}_{A_{i}}\left(\left\{\tilde{h} \circ z_{i, j,s}\right\}_{j\in [k],s\in [r-1]}\right)
$$Notice that the function $$f:\R\longrightarrow \R$$ $$f(x)=\tilde{h}(x)-x$$ is continuous. %Therefore, for the Mann-Wald theorem the distribution of $f(z_{i,j,s})$ converges weakly to the distribution of $f(w_{j,s})$ as $z_{i,j,s}$ converges in distribution to $w_{j,s}$.
Moreover, the functions $z_{i,j,s}$ take all values in the compact interval $[-m \tilde\lambda,m\tilde\lambda]$ where $\tilde \lambda= \max_{a\in[m],j\in[k] ,s\in [r-1]}{|\lambda_{a,j,s}|}$. Therefore, it follows that 
$$
\|f(z_{i,j,s})\|_p\rightarrow\|f(w_{j,s})\|_p\leq 2 \varepsilon
$$
for $i\rightarrow \infty$ as $f$ is continuous, $z_{i,j,s}$ converge in distribution to $w_{j,s}$, $z_{i,j,s}$ are uniformly bounded and the inequality in \eqref{eqn:EqIneqLp}.
Hence, if $i$ is large enough, then $\left\|f(z_{i, j,s})\right\|_{p}=\left\|\tilde{h} \circ z_{i, j,s}-z_{i, j,s}\right\|_{p} \leq 3 \varepsilon$ holds for $j \in[k]$ and thus $d_{\mathrm{LP}}\left(\beta_{i}^{\prime}, \beta_{i}\right) \leq$ $(r k)^{3 / 4}\left(3 C^{\prime} \varepsilon\right)^{1 / 2}$ by Lemma \ref{coupdist2}.

Let $\left\{n_{i}^{\prime}\right\}_{i=1}^{\infty}$ be a subsequence of $\left\{n_{i}\right\}_{i=1}^{\infty}$ such that $\beta^{\prime}:=\lim _{i \rightarrow \infty} \beta_{n_{i}^{\prime}}^{\prime}$ exists. Noticing that $\beta^{\prime} \in X_{k}$ and $d_{\mathrm{LP}}\left(\beta^{\prime}, \beta\right) \leq(r k)^{3 / 4}\left(3 C^{\prime} \varepsilon\right)^{1 / 2}$, we get that
$$
d_{\mathrm{LP}}\left(X_{k}, \alpha\right) \leq d_{\mathrm{LP}}\left(\beta^{\prime}, \alpha\right) \leq d_{\mathrm{LP}}\left(\beta^{\prime}, \beta\right)+d_{\mathrm{LP}}(\beta, \alpha) \leq 3(r k)^{3 / 4}\left(C^{\prime} \varepsilon\right)^{1 / 2} .
$$
This holds for every $\varepsilon>0$ and thus we finally obtain $\alpha \in X_{k}$.

\begin{remark}
    This proof works generally for any sequence of multi-$P-$operators with a uniform  bound on their order. However, this proof cannot work for sequences of multi-$P-$operators in which the order of the multi-$P-$operators is diverging.  
\end{remark}


\section{Properties of limit objects}\label{SecPropLimObje}
In this section, we discuss some properties of multi-$P-$operators that are preserved under action convergence. 

\begin{definition}\label{defprops} Let $A\in\mathcal{B}_{r-1}(\Omega)$ be a multi-$P$-operator. 
\begin{itemize}
\item $A$ is {\bf symmetric} if $$\E[A[v_1,\ldots,v_{r-1}]v_r]=\E[A[v_{\pi(1)}\ldots,v_{\pi(r-1)}]v_{\pi(r)}]$$ holds for every $v_1,\ldots, v_{r}\in L^\infty(\Omega)$ and for every $\pi$ permutation of $[r]$.
\item $A$ is {\bf positive} if $\E[A[v,\ldots,v]v]\geq 0$ holds for every $v\in L^\infty(\Omega)$.
\item $A$ is {\bf positivity-preserving} if for every $v_1,\ldots,v_{r-1}\in L^\infty(\Omega)$ with $v_1(x),\ldots,v_{r-1}(x)\geq 0$ for almost every $x\in\Omega$, we have that $(A[v_1,\ldots,v_{r-1}])(x)\geq 0$ holds for almost every $x\in\Omega$.
\item $A$ is {\bf $c$-regular} if $ A[1_\Omega,\ldots,1_\Omega]=c1_\Omega$ for some $c\in\mathbb{R}$.
\item $A$ is a {\bf hypergraphop} if it is positivity-preserving and symmetric.%$A$ is a {\bf Markov hypergraphop} if $A$ is a $1$-regular hypergraphop.
\item $A$ is {\bf atomless} if $\Omega$ is atomless.
\end{itemize}
\end{definition}

In particular, we notice that the $s-$action of the adjacency tensor of a hypergraph is positivity-preserving and symmetric, i.e.\ a hypergraphop.

The following lemmas are generalizations to the multi-linear case of the results from Section 3 in \cite{backhausz2018action} for action convergence and the proofs are similar. 

\begin{lemma}Atomless multi-$P$-operators are closed with respect to $d_M$.
\end{lemma}

\begin{proof} %The proof of this Lemma is similar to the proof of Lemma 3.1 in \cite{backhausz2018action}.
Let $A\in\mathcal{B}_{r-1}(\Omega)$ and $B\in\mathcal{B}_{r-1}(\Omega_2)$ be two multi-$P$-operators with $d_M(B,A)=d$. Let's suppose $A\in\mathcal{B}_{r-1}(\Omega)$ to be atomless. Therefore, there exists a function $v\in L^{\infty}_{[-1,1]}(\Omega)$ such that the distribution of $v$ is uniform on $[-1,1]$. Let $\alpha:=\mathcal{D}_A(v,\ldots,v)$.  As $d_H(\mathcal{S}_1(A),\mathcal{S}_1(B))\leq 2d$ it follows that $\beta=\mathcal{D}_B(w,w^{(2)},\ldots,w^{(r-1)})\in\mathcal{S}_1(B)$ with $d_{\rm LP}(\beta,\alpha)\leq 3d$ and thus $d_{\rm LP}(\alpha_1,\beta_1)\leq3d$, where $\alpha_1=\mathcal{L}(v)=\text{Unif}_{[-1,1]}$ and $\beta_1=\mathcal{L}(w)$ are the marginals of $\alpha$ and $\beta$ on the first coordinate. Thus, $\beta_1$ is at most $3d$ far from the uniform distribution in $d_{\rm LP}$, and, therefore, the largest atom in $\beta_1$ is at most $10d$ as by the definition of Levy-Prokhorov distance
$$
\inf\{\delta:\ \beta_1(\{x_0\})\leq \alpha_1(B_{\delta}(x_0))+\delta\}\leq d_{LP}(\alpha_1,\beta_1)\leq 3\varepsilon
$$
and $\alpha_1(B_{\delta}(x_0))=2\delta$. Hence the largest atom in $\Omega_2$ has weight at most $10d=10d_M(B,A)$. For this reason, if $B$ is the limit of atomless operators, then $B$ is atomless. 
\end{proof}

Under uniform boundedness conditions, positivity and symmetry of multi-$P-$operators are preserved under action convergence.

\begin{lemma} \label{LemmaClosedSymmetric}Let $p\in[1,\infty]$ and $q\in (1,\infty)$. Let $\{A_i\in\mathcal{B}(\Omega_i)\}_{i=1}^\infty$ be a sequence of uniformly $(p,q)$-bounded  $P$-operators converging to a multi-$P$-operator $A\in\mathcal{B}(\Omega)$. Then we have the following two statements.
\begin{enumerate}
\item If $A_i$ is positive for every $i$, then $A$ is also positive.
\item If $A_i$ is symmetric for every $i$, then $A$ is also symmetric.
\end{enumerate}
\end{lemma}

\begin{proof} To prove the first claim let $v\in L^\infty_{[-1,1]}(\Omega)$. By the definition of Action convergence, there are elements $v_i\in L^\infty_{[-1,1]}(\Omega_i)$  such that $\mathcal{D}_A(v_i,\ldots,v_i)$ weakly converges to $\mathcal{D}_A(v\ldots, v)$ as $i$ goes to infinity. Using Lemma \ref{closedlem2}, we obtain that $\mathbb{E}(v_i(A_i[v_i,\ldots, v_i ]))$ converges to $\mathbb{E}(v(A[v,\ldots,v]))$. By assumption, $\mathbb{E}(v_i(A_i[v_i,\ldots, v_i ]))\geq 0$ for every $i$, and thus $\mathbb{E}(v(A[v,\ldots,v]))\geq 0$. 

To prove the second claim let $v_1,\ldots, v_{r}\in L^\infty_{[-1,1]}(\Omega)$ and let $\mu:=\mathcal{D}_A(v_1,\ldots,v_r)$. By the definition of Action convergence, it follows that for every $i\in\mathbb{N}$ there exist functions $v_{i,1},\ldots, v_{i,r}\in L_{[-1,1]}^\infty(\Omega_i)$ such that $\mu_i:=\mathcal{D}_{A_i}(v_{i,1},\ldots,v_{i,r},v_{i,\pi(1)},\ldots,v_{i,\pi(r)})$ weakly converges to $\mu$.  By Lemma \ref{closedlem2}, we have that $\mathbb{E}(v_{i,r}(A_i[v_{i,1},\ldots,v_{i,r-1}]))$ converges to $\mathbb{E}(v_r(A[v_1,\ldots,v_{r-1}]))$ and $\mathbb{E}(v_{i,\pi(r)}(A_i[v_{i,\pi(1)},\ldots,v_{i,\pi(r-1)}]))$ converges to $\mathbb{E}(v_{\pi(r)}(A[v_{\pi(1)},\ldots,v_{\pi(r-1)}]))$ as $i$ goes to infinity. But we notice that 
$$\mathbb{E}(v_{i,r}(A_i[v_{i,1},\ldots,v_{i,r-1}]))=\mathbb{E}(v_{i,\pi(r)}(A_i[v_{i,\pi(1)},\ldots,v_{i,\pi(r-1)}]))$$
and therefore $$
\mathbb{E}(v_r(A[v_1,\ldots,v_{r-1}]))=\mathbb{E}(v_{\pi(r)}(A[v_{\pi(1)},\ldots,v_{\pi(r-1)}]))
$$\end{proof} 

\begin{remark}
The $s-$action of the adjacency tensor of a hypergraph is positive and symmetric. 
\end{remark}
Therefore, positivity-preserving and $c-$regular multi-$P-$operators are closed under action convergence, under slightly different uniform boundedness conditions.

\begin{lemma}\label{LemmaClosedPosPres} Let $p\in [1,\infty),q\in [1,\infty],c\in\mathbb{R}$ and let $\{A_i\in\mathcal{B}(\Omega_i)\}_{i=1}^\infty$ be a sequence of uniformly $(p,q)$-bounded  multi-$P$-operators converging to a $P$-operator $A\in\mathcal{B}(\Omega)$. Then we have the following two statements.
\begin{enumerate}
\item If $A_i$ is positivity-preserving for every $i$, then $A$ is also positivity-preserving.
\item If $A_i$ is $c$-regular for every $i$, then $A$ is also $c$-regular.
\end{enumerate}
\end{lemma}
\begin{proof}
To show the first statement, let $v_1,\ldots,v_{r-1}\in L^\infty_{[0,1]}(\Omega)$. By the definition of Action convergence, there is a sequence $\{v_{i,1},\ldots,$\ $ v_{i,r-1}\in L_{[-1,1]}^\infty(\Omega_i)\}_{i=1}^\infty$  such that $\mathcal{D}_{A_i}(v_{i,1},\ldots,v_{i,r-1})$ weakly converges to $\mathcal{D}_A(v_1,\ldots,v_{r-1})$. As $\mathcal{L}(v_{i,1},\ldots,v_{i,r-1})$ weakly converges to the non-negative distribution $\mathcal{L}(v_1,\ldots,v_{r-1})$ we have that $\mathcal{L}(v_{i,1}-|v_{i,{1}}|,\ldots,v_{i,r-1}-|v_{i,{r-1}}|)$ weakly converges to $\delta_0$. Thus, by Lemma \ref{applem2}, it follows that 
$$d_{\rm LP}(\mathcal{D}_{A_i}(v_{i,1},\ldots,v_{i,r-1}),\mathcal{D}_{A_i}(|v_{i,1}|,\ldots,|v_{i,r-1}|))\rightarrow 0$$
for $i\rightarrow \infty$ and, for  this reason, $(v_1,\ldots,v_{r-1},A[v_1,\ldots,v_{r-1}])$ is the weak limit of $\mathcal{D}_{A_i}(|v_{i,1}|,\ldots, |v_{i,r-1}|)$. The fact that $A_i[|v_{i,1}|,\ldots, |v_{i,r-1}|]$ is non-negative for every $i$  implies that $A[v_1,\ldots,v_{r-1}]$ is non-negative. 

To show the second statement, let $v_{i,1},\ldots, v_{i,r-1}\in L_{[-1,1]}^\infty(\Omega_i)$ be a sequence of functions such that $\mathcal{D}_{A_i}(v_{i,1},\ldots, v_{i,r-1})$ weakly converges to $\mathcal{D}_A(1_\Omega,\ldots, 1_\Omega)$. We notice that $\mathcal{D}(v_{i,1}-1_{\Omega_i},\ldots,v_{i,r-1}-1_{\Omega_i})$ weakly converges to $\delta_0$ and, for this reason, by Lemma \ref{applem2} we have that 
$$d_{\rm LP}(\mathcal{D}_{A_i}(1_{\Omega_i},\ldots,1_{\Omega_i}),\mathcal{D}_{A_i}(v_{i,1},\ldots,v_{i,r-1}))\rightarrow 0$$ as $i\rightarrow \infty$. Hence, it follows that $\mathcal{D}_{A_i}(1_{\Omega_i},\ldots,1_{\Omega_i})$ weakly converges to $\mathcal{D}_A(1_\Omega,\ldots,1_\Omega)$. The result directly follows now by the fact that $ A_i[1_{\Omega_i},\ldots,1_{\Omega_i}]=c1_{\Omega_i}$.
\end{proof}

\begin{remark}
The $s-$action of the adjacency tensor of an hypergraph is positivity-preserving. 
\end{remark}
\section{Multi-action convergence of hypergraphs and tensors}\label{SecMultActTensHyp}

We have seen in the previous sections that hypergraphs can be naturally associated with symmetric tensors through, for example, the adjacency tensor. We can therefore study the convergence of sequences of tensors and see the convergence of hypergraphs as a particular case. Moreover, in the previous sections, we noticed that tensors can be associated with multi-linear operators in many different ways. We will compare the notions of convergence induced by the different operators associated to the same tensor. We mainly focus on symmetric tensors as we are originally motivated by undirected hypergraphs. We notice that the obtained notions of convergence are not equivalent. For simplicity, we will mainly present the convergence in the case of $3-$rd order symmetric tensors and therefore hypergraphs with maximal edge cardinality 3. The notions of convergence are anyway general and cover tensors of any finite order and hypergraphs with any finite maximal edge cardinality. These convergence notions particularly make sense for uniform hypergraphs. However, we will explain, in the next section, how one can extend these notions to not lose information regarding the non-maximal cardinality edges in non-uniform hypergraphs.

We recall the notion of $s-$action of a tensor from Section \ref{SecTensHyp}. For a $3$rd-order tensor $T=(T_{i,j,k})_{i,j,k\in [n]}$ the 1-action and the $2-$action of the tensor are respectively 

\begin{equation*}
(T_1[f,g])_{i}=\sum^n_{j,k=1}T_{j,i,k}f_{i}g_{k} 
\end{equation*}and
$$
\begin{aligned}
& T_2[f,g]=\frac{1}{2}(\sum^n_{j=1}T_{j,i,k}f_{j,i}g_{j,k} +\sum^n_{j=1}T_{j,i,k}f_{j,k}g_{j,i})
\end{aligned}
$$
%\begin{equation*}
%(T_2[f,g])_{i,k}=\sum^n_{j=1}T_{j,i,k}f_{j,i},g_{j,k} 
%\end{equation*}
Therefore, we can interpret the $1-$action as an operator 
$$
\begin{aligned}
T_1:(L^{\infty}([n]))^2\longrightarrow L^{1}([n])
\end{aligned}
$$
and the $2-$action as
$$T_2:(L^{\infty}([n]\times [n],Sym))^2\longrightarrow L^{1}([n]\times [n],Sym)$$
where $Sym$ is the symmetric $\sigma-$algebra on $[n]\times [n]$.

\begin{remark}
More generally the $s-$action of a $r-$th order symmetric tensor is acting on $r-1$ symmetric $s-$th order tensors and gives as an output another symmetric $s-$th order tensor. For this reason, this $s-$action can be interpreted as an operator
$$
T_s:(L^{\infty}([n]^s,Sym))^{r-1}\longrightarrow L^{1}([n]^s,Sym)
$$
where  $Sym$ is the symmetric $\sigma-$algebra on $[n]^s$.
\end{remark}

In such a way, we obtain two notions of convergence for sequences of $3$rd-order tensors $T_n=(T_{i,j,k})_{i,j,k\in [n]}$, the one obtained by the action convergence of the sequence of multi-linear operators $(T_1)_n$ and the one obtained by the action convergence of the sequence of multi-linear operators $(T_2)_n$.

\begin{remark}
As already pointed out we can associate to a $r-$th order symmetric tensor its $s-$action for $s\in[r-1]$. These different actions can be interpreted as $r-1$ different multi-$P-$operators. Therefore, for a sequence of $r-$th order symmetric tensors, we obtain $r-1$ different notions of convergence. %with the convergence of the $s-$action finer than the convergence of the $\tilde{s}-$
\end{remark}

%An additional observation from an analytical perspective regarding the appearance of the additional coordinates in hypergraph limits is that in the case of dense graph limits we consider $P-$operators 
We recall that in the case of graphs, we typically consider the $1-$action of normalized adjacency matrices. In particular for dense graphs, we consider

$$
\widetilde{A}(G):L^{\infty}([n])\longrightarrow L^{1}([n])
$$
$$
\widetilde{A}(G)[f]=\sum_{j}\frac{A(G)_{i,j}}{n}f_j.
$$
for a graph $G$ on the vertex set $[n]$. These linear bounded operators can be easily extended to linear bounded operators 
$$
\widetilde{A}(G):L^1{([n])}\longrightarrow L^{\infty}([n])
$$
and we have that these operators are uniformly bounded (independently by the cardinality of the vertex set $n$) as 
$$
\begin{aligned}
&\|\widetilde{A}(G)[f]\|_{1\rightarrow\infty}=\max_i|\sum_{j}\frac{A(G)_{i,j}}{n}f_j|\leq\\&
\max_i\sum_{j}\frac{A(G)_{i,j}}{n}|f_j|\leq\sum_{j}\frac{|f_j|}{n}= \|f\|_1 \\ &
\end{aligned}.
$$
We can observe that for hypergraphs with maximal edge cardinality $r>2$ this is obviously not true. For example, consider a (dense) hypergraph $H$, its adjacency tensor $(A_{i,j,k})_{i,j,k\in [n]}$ and the associated multi-$P-$operator 
$$\widetilde{A}(H)[f,g]=\widetilde{A}[f,g]=\frac{1}{2}(\sum^n_{j=1}\frac{A_{j,i,k}}{n}f_{j,i},g_{j,k} +\sum^n_{j=1}\frac{A_{j,i,k}}{n}f_{j,k},g_{j,i})
$$
 and consider the matrices $f,g$ such $$f_{i,j}=g_{i,j}=\begin{cases}
f_{i,j}=0 & \text{ if } i,j\neq 1 \\
f_{i,j}=1 & \text{ if } i=1 \text{ or }j=1
\end{cases}.
$$

However, we can consider a smaller extension.

\begin{lemma}\label{LemmaUnifBoundNormHyp}
    The sequence of operators $$
\widetilde{A}(G_n):L^2{([n]\times [n],Sym)}\times L^2{([n]\times [n],Sym)}\longrightarrow L^{2}([n]\times [n], Sym)
$$
$$\widetilde{A}(G_n)[f,g]=\widetilde{A}[f,g]=\frac{1}{2}(\sum^n_{j=1}\frac{A_{j,i,k}}{n}f_{j,i},g_{j,k} +\sum^n_{j=1}\frac{A_{j,i,k}}{n}f_{j,k},g_{j,i})
$$
is uniformly bounded in $L^2-$operator norm.
\end{lemma}
\proof
In these spaces, we have a uniform bound, in fact
$$
\begin{aligned}
&
|\widetilde{A}[f,g]_{i,k}|\leq
\\
&\frac{1}{2}(\sum^n_{j=1}\frac{A_{j,i,k}}{n}|f_{j,i}g_{j,k}| +\sum^n_{j=1}\frac{A_{j,i,k}}{n}|f_{j,k}g_{j,i}|)\leq\\&
\frac{1}{2}(\sum^n_{j=1}\frac{1}{n}|f_{j,i}g_{j,k}| +\sum^n_{j=1}\frac{1}{n}|f_{j,k}g_{j,i}|)\leq 
\\ &
\frac{1}{2}((\sum^n_{j=1}\frac{1}{n}|f_{j,i}|^2)^{\frac{1}{2}}(\sum^n_{j=1}\frac{1}{n}|g_{j,k}|^2 )^{\frac{1}{2}}+(\sum^n_{j=1}\frac{1}{n}|f_{j,k}|^2)^{\frac{1}{2}}(\sum^n_{j=1}\frac{1}{n}|g_{j,i}|^2)^{\frac{1}{2}})
\end{aligned}
$$

where the last inequality follows by Cauchy-Schwartz inequality. Therefore, we obtain
$$
\begin{aligned}
&
\|\widetilde{A}[f,g]\|_2\leq
(\frac{1}{n^2}\sum_{i,k=1}
|\widetilde{A}[f,g]_{i,k}|^2)^{\frac{1}{2}}\leq
\\&
\frac{1}{2}(\frac{1}{n^2}\sum_{i,k=1}(\sum^n_{j=1}\frac{1}{n}|f_{j,i}|^2)(\frac{1}{n}\sum^n_{j=1}|g_{j,k}|^2 )+\frac{1}{n^2}\sum_{i,k=1}(\sum^n_{j=1}\frac{1}{n}|f_{j,k}|^2)(\sum^n_{j=1}\frac{1}{n}|g_{j,i}|^2))^{\frac{1}{2}}\leq
\\&
\frac{1}{2}((\frac{1}{n^2}\sum^n_{j,i=1}|f_{j,i}|^2)^{\frac{1}{2}}(\frac{1}{n^2}\sum^n_{j,k=1}|g_{j,k}|^2 )^{\frac{1}{2}}+(\frac{1}{n^2}\sum^n_{j,k=1}|f_{j,k}|^2)^{\frac{1}{2}}(\frac{1}{n^2}\sum^n_{j,i=1}|g_{j,i}|^2))^{\frac{1}{2}}=\\
&  \|f\|_{2}\|g\|_{2}
\end{aligned}
$$

where we used Minkowski inequality in the fourth inequality.

\endproof
\begin{remark}
More in general, for $r>2$, for a sequence of dense hypergraphs the sequence of $(r-1)-$actions of the relative normalized adjacency tensors cannot be extended/interpreted as a uniformly bounded sequence of linear operators from $L^1\times \ldots \times L^1$ to $L^{\infty}$. Therefore, one has to consider them as operators from $L^{p_1}\times \ldots \times L^{p_{r-1}}$ to $L^q$ with $p_1,\ldots,p_{r-1}\neq 1$ and $q\neq \infty$. This happens already in the case of graph limits for sparse graph sequences, and we know that this translates in larger classes of measures admitting also more irregular measures, possibly not absolutely continuous with respect to the Lebesgue measure on the interval $[0,1]$. This could be also the reason for the additional variables needed to represent hypergraphons. 
Instead, for every $r$ the sequence of $1-$actions of the normalized adjacency matrices of dense graphs is a uniformly bounded sequence of linear operators from $L^1\times \ldots \times L^1$ to $L^{\infty}$.
\end{remark}



The emergence of multiple operators and therefore of different notions of convergence of symmetric tensors is strictly related to the emergence of different levels of quasi-randomness for sequences of hypergraphs \cite{RandomneLimitTow,QuasirandHyp1,QuasirandHyp2}.

We illustrate here this relationship with some examples.

\begin{example}\label{ExamCompERHyp}
    Let's consider the $3-$uniform Erdős–Rényi hypergraph $G(n,\frac{1}{8},3)$ from Example \ref{ERRandomHypergraph} and the $3-$uniform hypergraph $T(n,\frac{1}{2})$, i.e. the $3-$uniform hypergraph with edges the triangles of an Erdős–Rényi graph $G(n,\frac{1}{2},2)$ from Example \ref{TriangRandHyp}. We now consider the sequence $(A_n)_{n\in \mathbb{N}}$ of the normalized adjacency tensors associated with $G(n,\frac{1}{8},3)$, i.e.\  $A_n=\sfrac{A(G(n,\frac{1}{8},3))}{n}$, and the sequence $(B_n)_{n\in \N}$ of the normalized adjacency tensors associated with $T(n,\frac{1}{2})$, i.e.\ $B_n=\sfrac{A(T(n,\frac{1}{2}))}{n}$. We remark that the normalization of the adjacency tensors is necessary to satisfy the hypothesis of Theorem \ref{CompactnesMultiActConv} and, therefore, to ensure a convergent (sub)sequence as shown in \ref{LemmaUnifBoundNormHyp}. However, different normalizations could be chosen as we will explore later.

If we now consider the sequences of multi-$P-$operators $(A_n)_1$ and $(B_n)_1$ they both action converge to the same limit object also if the two random hypergraph models are very different. To see the combinatorial difference between these two random hypergraph models consider how many edges can be present in an induced hypergraph on $4$ vertices. In $T(n,\frac{1}{2})$ there cannot be exactly three edges but in $G(n,\frac{1}{8},3)$ this happens with probability $$4\cdot \frac{1}{8}\cdot \frac{1}{8}\cdot \frac{1}{8}\cdot \frac{7}{8}=\frac{7}{1024}.$$

Instead, if we now consider the sequences of multi-$P-$operators $(A_n)_2$ and $(B_n)_2$ the two sequences are now converging to two different limits as we show in the following Lemma. 
\end{example}

\begin{lemma}\label{LemmaDiffLimitERHTriangER}
The (sub-)sequences of the multi-$P-$operators $(A_n)_2$ and $(B_n)_2$, as defined in Example \ref{ExamCompERHyp}, have different action convergence limits.
\end{lemma}


\proof
Let's denote with $E_n$ the set of edges of the Erdős–Rényi graph $G(n,\frac{1}{2})$ from which $T(n,\frac{1}{2})$ is generated, that is the Erdős–Rényi graph from which the triangles are taken to create the edges of $T(n,\frac{1}{2})$.
Let $\mathbbm{1}_{\Omega_n}$ be the $n \times n$ matrix with every entry equal to $1$.  We can observe that the distribution

$$\mathcal{L}(\mathbbm{1}_{\Omega_n},\mathbbm{1}_{\Omega_n},(A_n)_2[\mathbbm{1}_{\Omega_n},\mathbbm{1}_{\Omega_n}])$$

of the $3-$random vector 
$$(\mathbbm{1}_{\Omega_n},\mathbbm{1}_{\Omega_n},(A_n)_2[\mathbbm{1}_{\Omega_n},\mathbbm{1}_{\Omega_n}])
$$where 
$$(A_n)_2[\mathbbm{1}_{\Omega_n},\mathbbm{1}_{\Omega_n}]_{i,k}=\frac{1}{2}(\sum^n_{j=1}(A_n)_{j,i,k}(\mathbbm{1}_{\Omega_n})_{j,i},(\mathbbm{1}_{\Omega_n})_{j,k} +\sum^n_{j=1}(A_n)_{j,i,k}(\mathbbm{1}_{\Omega_n})_{j,k},(\mathbbm{1}_{\Omega_n})_{j,i} )$$
and the distribution
$$\mathcal{L}(\mathbbm{1}_{\Omega_n},\mathbbm{1}_{\Omega_n},(B_n)_2[\mathbbm{1}_{\Omega_n},\mathbbm{1}_{\Omega_n}])$$of the $3-$random vector 
$$(\mathbbm{1}_{\Omega_n},\mathbbm{1}_{\Omega_n},(B_n)_2[\mathbbm{1}_{\Omega_n},\mathbbm{1}_{\Omega_n}])
$$where 
$$(B_n)_2[\mathbbm{1}_{\Omega_n},\mathbbm{1}_{\Omega_n}]_{i,k}=\frac{1}{2}(\sum^n_{j=1}(B_n)_{j,i,k}(\mathbbm{1}_{\Omega_n})_{j,i},(\mathbbm{1}_{\Omega_n})_{j,k} +\sum^n_{j=1}(B_n)_{j,i,k}(\mathbbm{1}_{\Omega_n})_{j,k},(\mathbbm{1}_{\Omega_n})_{j,i} )$$are very different. In fact, for $n\rightarrow \infty$ we have

$$\mathcal{L}(\mathbbm{1}_{\Omega_n},\mathbbm{1}_{\Omega_n},(A_n)_2[\mathbbm{1}_{\Omega_n},\mathbbm{1}_{\Omega_n}])\rightarrow\delta_{(1,1,\frac{1}{8})}$$
as for any $(i,k)\in [n]\times [n]\setminus \{(i,i):\ i\in[n]\}$ for any $j\in [n]$ the probability that $\{i,j,k\}$ is an edge of $G(n,\frac{1}{8},3)$ is $\frac{1}{8}$.
However,
$$\mathcal{L}(\mathbbm{1}_{\Omega_n},\mathbbm{1}_{\Omega_n},(B_n)_2[\mathbbm{1}_{\Omega_n},\mathbbm{1}_{\Omega_n}])\rightarrow\frac{1}{2}\delta_{(1,1,0)}+\frac{1}{2}\delta_{(1,1,\frac{1}{4})}$$
as if $(i,k)\in E_n$ then for any $j\in [n]$ the probability that $\{i,j,k\}$ is an edge of  $T(n,\frac{1}{2})$ is $\frac{1}{4}$ but if $(i,k)\notin E_n$ then there is no $j\in [n]$ such that $\{i,j,k\}$ is an edge of  $T(n,\frac{1}{2})$.
Therefore, the $1-$profiles $S_1(A)$ and $S_1(B)$ of the action convergence limits  $A$ and $B$ (passing to subsequences if it is necessary) of the sequences $((A_n)_2)_n$ and $((B_n)_2)_n$ are at Hausdorff distance bigger than a constant $c>0$. Let's suppose by contradiction that there exists $f,g\in L_{[-1,1]}^{\infty}(\Omega)$ such that for every $\varepsilon>0$ $$d_{LP}(\mathcal{L}(f,g,A[f,g]),\mathcal{L}(\mathbbm{1}_{\Omega},\mathbbm{1}_{\Omega},B[\mathbbm{1}_{\Omega},\mathbbm{1}_{\Omega}]))\leq \varepsilon.$$
We observe that convergence in distribution to a constant and convergence in probability to the same constant are equivalent and, as the random variables are bounded between $1$ and $-1$, convergence in probability is equivalent to the convergence of the $p-$th moment. Therefore, for any $\delta >0$, we can choose $\varepsilon$ small enough such that 
$$
\|\mathbbm{1}_{\Omega}- f\|_1\leq\|\mathbbm{1}_{\Omega}- f\|_{p_1}< \delta \text{ and } \|\mathbbm{1}_{\Omega}- g\|_1\leq\|\mathbbm{1}_{\Omega}- g\|_{p_2}< \delta 
$$and, therefore, 

$$\|A[\mathbbm{1}_{\Omega},\mathbbm{1}_{\Omega}]-A[f,g]\|_1\leq\|A[\mathbbm{1}_{\Omega},\mathbbm{1}_{\Omega}]-A[f,g]\|_q< 2C\delta
$$
Using Lemma \ref{coupdist}, we obtain that 
 $$d_{LP}(\mathcal{L}(f,g,A[f,g]),\mathcal{L}(\mathbbm{1}_{\Omega},\mathbbm{1}_{\Omega},A[\mathbbm{1}_{\Omega},\mathbbm{1}_{\Omega}]))\leq 3^{\frac{3}{4}}\delta^{\frac{1}{2}}\max\{(2C)^{\frac{1}{2}},1\}$$
Therefore, for the triangular inequality we have 
$$\begin{aligned}
&d_{LP}(\mathcal{L}(f,g,A[f,g]),\mathcal{L}(\mathbbm{1}_{\Omega},\mathbbm{1}_{\Omega},B[\mathbbm{1}_{\Omega},\mathbbm{1}_{\Omega}]))
\geq  \\
&|d_{LP}(\mathcal{L}(\mathbbm{1}_{\Omega},\mathbbm{1}_{\Omega},B[\mathbbm{1}_{\Omega},\mathbbm{1}_{\Omega}]),\mathcal{L}(\mathbbm{1}_{\Omega},\mathbbm{1}_{\Omega},A[\mathbbm{1}_{\Omega},\mathbbm{1}_{\Omega}]))-d_{LP}(\mathcal{L}(f,g,A[f,g]),\mathcal{L}(\mathbbm{1}_{\Omega},\mathbbm{1}_{\Omega},A[\mathbbm{1}_{\Omega},\mathbbm{1}_{\Omega}]))|\\
&\geq K - 3^{\frac{3}{4}}\delta^{\frac{1}{2}}\max\{(2C)^{\frac{1}{2}},1\}\geq c >0
\end{aligned}$$
where $K>0$ and $\delta \rightarrow 0$ as $\varepsilon \rightarrow 0 $. But this is in contradiction with 
$$d_{LP}(\mathcal{L}(f,g,A[f,g]),\mathcal{L}(\mathbbm{1}_{\Omega},\mathbbm{1}_{\Omega},B[\mathbbm{1}_{\Omega},\mathbbm{1}_{\Omega}]))\leq \varepsilon.$$
\endproof

Similarly, we can consider other types of realizations of random hypergraphs and try to understand some specific measures in the associated profiles.

\begin{example}\label{ExampleTrianTourn}
We can consider the random hypergraph $TH(n)$ produced considering the triangles of a random tournament, a realization of $Tor(n)$ (Example \ref{RandtournGraph}), as in Example \ref{RandHyperTourn} with ordered vertex set $[n]$. Now, we can consider the sequences of multi-operators $(A_n)_2$ and $(B_n)_2$, the operators obtained by the normalized adjacency tensor of the  $3-$uniform Erdős–Rényi hypergraph $G(n,\frac{1}{8},3)$  and the random hypergraph $T(n,\frac{1}{2})$,% i.e.\ the $3-$uniform hypergraph with edges the triangles of an Erdős–Rényi graph $G(n,\frac{1}{2},2)$, 
respectively as in Example \ref{ExamCompERHyp}. Additionally, we will denote with $(C_n)_2$ the sequence of multi-operators obtained by the normalized adjacency tensor of  the random hypergraph $TH(n)$, i.e.\ from the adjacency tensor $C_n=\sfrac{A(TH(n))}{n}$.  Again, it is easy to see that the random graph $TH(n)$ is combinatorially very different from $G(n,\frac{1}{8},3)$ and $T(n,\frac{1}{2})$. In fact, for every four vertices in $TH(n)$ it is impossible to have the complete $3-$uniform hypergraph as an induced subgraph differently from the cases of $G(n,\frac{1}{8},3)$ and $T(n,\frac{1}{2})$. We have already shown in Example \ref{ExamCompERHyp} and Lemma \ref{LemmaDiffLimitERHTriangER} that the sequences $(A_n)_2$ and $(B_n)_2$ have different limits. Moreover, $(C_n)_2$ has a different limit than both $(A_n)_2$ and $(B_n)_2$ as we show in the following lemma.
 \end{example}


\begin{lemma}
The (sub-)sequences of the multi-$P-$operators  $(A_n)_2$, $(B_n)_2$ and $(C_n)_2$, defined in Example \ref{ExampleTrianTourn}, have different action convergence limits.
\end{lemma}

 
 \proof
If there is an oriented edge of the realization of $Tor(n)$ from $i\in [n]$ to $j\in [n]$ we write $i\rightarrow j$. We define the symmetric set constructed from the oriented edges of the tournament $Tor(n)$
$$
\begin{aligned}
S_n&=\{(i,j) \ s.t. \ i\rightarrow j \text{ and } j<i, \ \text{ and } i,j \in [n]\}\cup \\
&
\{(j,i) \ s.t.\ i\rightarrow j \text{ and } j<i, \ \text{ and } i,j \in [n]\}\end{aligned}
$$ and also the set
$$
\begin{aligned}
U_n=(S_n)^c&=\{(i,j) \ s.t.\ i\rightarrow j \text{ and } j>i, \ \text{ and } i,j \in [n]\}\cup \\
&
\{(j,i) \ s.t.\ i\rightarrow j \text{ and } j>i, \ \text{ and } i,j \in [n]\}.\end{aligned}
$$
We can now compare the distributions of the random vectors 
$$
(\mathbbm1_{\Omega_n},\mathbbm1_{\Omega_n},(A_n)_2[\mathbbm1_{\Omega_n},\mathbbm1_{\Omega_n}]),
$$
$$
(\mathbbm1_{\Omega_n},\mathbbm1_{\Omega_n},(B_n)_2[\mathbbm1_{\Omega_n},\mathbbm1_{\Omega_n}])
$$

with the distribution of the random vector

$$
(\mathbbm1_{\Omega_n},\mathbbm1_{\Omega_n},(C_n)_2[\mathbbm1_{\Omega_n},\mathbbm1_{\Omega_n}])
$$
and similarly as before we can show that we obtain different limits.

As already observed in the proof of Lemma \ref{LemmaDiffLimitERHTriangER}, 
$$
\mathcal{L}(\mathbbm1_{\Omega_n},\mathbbm1_{\Omega_n},(A_n)_2[\mathbbm1_{\Omega_n},\mathbbm1_{\Omega_n}])\rightarrow \delta_{(1,1,\frac{1}{8})}
$$

$$
\mathcal{L}(\mathbbm1_{\Omega_n},\mathbbm1_{\Omega_n},(B_n)_2[\mathbbm1_{\Omega_n},\mathbbm1_{\Omega_n}])\rightarrow \frac{1}{2}\delta_{(1,1,0)}+\frac{1}{2}\delta_{(1,1,\frac{1}{4})}.
$$

Moreover, we observe  

$$
\mathcal{L}(\mathbbm1_{\Omega_n},\mathbbm1_{\Omega_n},(C_n)_2[\mathbbm1_{\Omega_n},\mathbbm1_{\Omega_n}])\rightarrow \delta_{(1,1,\frac{1}{4})}.
$$
This can be observed as if $(i,k)\in S_n$ and, without loss of generality, we assume $i<k$, then for every $j \in [n] \setminus \{i,k\}$ the probability that $\{i,j,k\}$ is an edge of $TH(n)$, i.e.\ it is a triangle of $Tor(n)$ is $\frac{1}{4}$. This follows simply by checking separately the cases: $j<i$, $i<j<k$ and $k<j$. The same argument applies for $(i,k)\in U_n=(S_n)^c$ too.
From this, it follows with a similar argument as in the proof of Lemma \ref{LemmaDiffLimitERHTriangER} that the sequence $(C_n)_2$ (or its sub-sequences) action converges to different limits than $(A_n)_2$ and $(B_n)_2$.

%Instead, to show that $((A_n)_2)_{n\in \mathbb{N}}$ and $((C_n)_2)_{n\in \mathbb{N}}$ have different limits we will use the test functions $\mathbbm{1}_{S_n}$ and $\mathbbm{1}_{U_n}$. 

%The sequence of the distributions of the random vectors 

%$$
%(\mathbbm{1}_{U_n},\mathbbm{1}_{U_n},(C_n)_2[\mathbbm1_{U_n},\mathbbm1_{U_n}])
%$$
%weakly converges (almost surely) to 

%$$
%\frac{1}{2}\delta_{(1,1,0)} +\frac{1}{2}\delta_{(0,0,\frac{1}{4})}$$
%while the sequence of distributions of the random vectors 

%$$
%(\mathbbm{1}_{T_n},\mathbbm{1}_{T_n},(A_n)_2[\mathbbm1_{T_n},\mathbbm1_{T_n}])
%$$
%weakly converges (almost surely) to

%$$
%\frac{1}{2}\delta_{(1,1,\frac{1}{8})}+\frac{1}{2}\delta_{(0,0,\frac{1}{8})}
%$$
%for every sequence of vectors $\mathbbm{1}_{T_n}$
%where the set $T_n\subset [n]$ is such that $$
%\frac{|T_n|}{n}\rightarrow\frac{1}{2}$$ as $n\rightarrow\infty$. Moreover, using similar arguments as before we can show that for every sequence of functions $f,g$ such that 

%$$
%d_{LP}(\mathcal{L}(f,g), \frac{1}{2}\delta(1,1)+\frac{1}{2}\delta(0,0))<\varepsilon
%$$
%then it exists a set $T$ such that $\mathbb{P}(T)=\frac{1}{2}$ 
%$$
%|f(\omega)-\mathbbm{1}_T(\omega)|<\delta
%$$

%$$
%|g(\omega)-\mathbbm{1}_T(\omega)|<\delta
%$$
%except for a set $K\subset \Omega$ of measure $\mathbb{P}(K)$ small.
%Therefore, 

%$$
%\begin{aligned}
%&\|f-\mathbbm{1}_T\|_1 \leq \|f-\mathbbm{1}_T\|_{p_1}
%\leq (\int_{K}|f(\omega)-\mathbbm{1}_T(\omega)|^{p_1}d\mathbb{P}(\omega) +\int_{\Omega\setminus K}|f(\omega)-\mathbbm{1}_T(\omega)|^{p_1}d\mathbb{P}(\omega))^{\frac{1}{p_1}}\\
%& \leq (2^{p_1}\mathbb{P}(K)+ \delta^{p_1})^{\frac{1}{p_1}}<\tilde{\delta}_1
%\end{aligned}
%$$
%and in the same way 
%$$
%\|g-\mathbbm{1}_T\|_1 \leq \|g-\mathbbm{1}_T\|_{p_1}<\tilde{\delta}_2
%$$
%Therefore, it follows that 
%$$%\begin{aligned}
%\|(A_n)_2[\mathbbm{1}_{T},\mathbbm{1}_{T}]-(A_n)_2[f,g]\|_1&\\
%\leq\|(A_n)_2[\mathbbm{1}_{T},\mathbbm{1}_{T}]-(A_n)_2[f,g]\|_q
%&
%\\< C(\tilde{\delta}_1+\tilde{\delta}_1)
%\end{aligned}
%$$
%Therefore, similarly as before, for every $\tilde\varepsilon>0$, there exists an $\varepsilon>0$ such that 
%$$
%d_{LP}(\mathcal{L}(f,g,(A_n)_2[f,g]),\frac{1}{2}\delta_{(1,1,\frac{1}{8})}+\frac{1}{2}\delta_{(0,0,\frac{1}{8})})<\tilde\varepsilon
%$$
%for every $f,g\in L^{\infty}_{[-1,1]}(\Omega)$ such that 
%$$
%d_{LP}(\mathcal{L}(f,g), \frac{1}{2}\delta(1,1)+\frac{1}{2}\delta(0,0))<\varepsilon.
%$$% Therefore, using a similar argument as in the proof of Lemma \ref{LemmaDiffLimitERHTriangER} we obtain 

%$$
%d_M((A_n)_2,(C_n)_2)> \delta
%$$
%for a fixed $\delta>0$ for every $n$ big enough.
\endproof


We give one last example where we look at specific limit measures for specific test functions.


\begin{example}
    We can also consider the $3-$uniform hypergraph stochastic block model in which we consider a partition of the vertices $V(G_n)=[n]=V_1(G_n)\dot{\cup}V_2(G_n)$ where $|V_1(G_n)|=\left \lceil{\frac{n}{2}}\right \rceil $ and $|V_2(G_n)|=\left \lfloor{\frac{n}{2}}\right \rfloor$ and we add the edge $e=\{v_1,v_2,v_3\}$ for every three distinct vertices with probability $p_{111}$ if all the three vertices are in $V_1$, with probability $p_{222}$ if all of the three vertices are in $V_2$ and with probability $p_{211}$, respectively $p_{221}$, if two vertices are in $V_1$ and the other one in $V_2$, respectively two vertices are in $V_2$ and one is in $V_1$. In the limit, we obtain that 
$$
\mathcal{L}((\mathbbm{1}_{\Omega},\mathbbm{1}_{\Omega},A[\mathbbm{1}_{\Omega},\mathbbm{1}_{\Omega}]))=\frac{1}{4}\delta_{(1,1,\frac{p_{111}+p_{112}}{2})}+\frac{1}{4}\delta_{(1,1,\frac{p_{221}+p_{222}}{2})}+\frac{1}{2}\delta_{(1,1,\frac{p_{221}+p_{211}}{2})}
$$
Another $3-$uniform random hypergraph model is the model in which we assign at random a different colour out of three colours, say black, white, or grey to every pair of vertices. Now, for every triple of distinct vertices $v_1,v_2,v_3$ we add the edge $e=\{v_1,v_2,v_3\}$,  with probability $1$ if the three pairs $\{v_1,v_2\}$,$\{v_2,v_3\}$ and $\{v_1,v_3\}$ are all white, with probability $0$ if the three pairs $\{v_1,v_2\}$,$\{v_2,v_3\}$ and $\{v_1,v_3\}$ are all black and with probability $p$ if the three pairs $\{v_1,v_2\}$,$\{v_2,v_3\}$ and $\{v_1,v_3\}$ are all grey. Then in the limit we have
$$
\mathcal{L}((\mathbbm{1}_{\Omega},\mathbbm{1}_{\Omega},A[\mathbbm{1}_{\Omega},\mathbbm{1}_{\Omega}]))=\frac{1}{3}\delta_{(1,1,0)}+\frac{1}{3}\delta_{(1,1,\frac{1}{3^2})}+\frac{1}{3}\delta_{(1,1,\frac{p}{3^2})}
$$
\end{example}%Another $3-$uniform random hypergraph is the model where we consider the triangles of a graph stochastic block model:

%$$
%\tilde{p}\delta_{(1,1,0)}+\frac{1}{4}p_{11}\delta_{(1,1,\frac{p_{11}^2+p_{12}^2}{2})}+\frac{1}{4}p_{22}\delta_{(1,1,\frac{p_{22}^2+p_{12}^2}{2})}+\frac{1}{2}p_{12}\delta_{(1,1,\frac{p_{12}}{2}(p_{11}+p_{22}))}
%$$
Now that we have given some motivating examples for sequences of hypergraphs with diverging number of vertices we study what action convergence and the $k-$profiles capture for finite tensors and hypergraphs.  

The following theorem states that finite tensors are completely determined by the action convergence distance, up to relabelling of the indices. This is particularly interesting for adjacency tensors of hypergraphs because the following result implies that two adjacency tensors of two hypergraphs are identified if and only if the two hypergraphs are isomorphic.    

\begin{theorem}\label{IdentificationTensors}
For two $3-$rd order $n-$dimensional symmetric tensors $T=(T_{i,j})_{i,j\in [n]}$ and $(\widetilde{T})_{i,j\in [n]}$, the $2-$actions $T_2$ and $\widetilde{T}_2$ are at distance zero in action convergence distance $d_M$ if and only if there exists a bijective map $$\psi:[n]\rightarrow [n]$$
  such that 
   $$T_{i,j,k}=\widetilde{T}_{\psi(i),\psi(j),\psi(k)}.$$  \end{theorem}
\proof
The only non-trivial implication is the “only if” part.
We observe that, in the finite case, it must exist a bijective and measure-preserving function
$$\begin{aligned}
&\phi: ([n]\times [n],Sym)\longrightarrow ([n]\times [n],Sym)
\\ & (i,k)\mapsto\phi(i,k)=(\phi_1(i,k),\phi_2(i,k))
\end{aligned}$$
such that 
$$
(T_2[f,g])^{\phi}=\widetilde{T}_2[f^{\phi},g^{\phi}])
$$
for all symmetric matrices $f,g$ on $[n]\times[n]$. 
%$T_{i,j}=\widetilde{T}_{\phi_1(i,j),\phi_2(i,j)}$
%This follows just from choosing $m$ big enough and a measure $\mu_m$ in the $m-$profile such that there is an 
%$(n+{n \choose 2})!-$dimensional marginal of the measure supported on $(n+{n \choose 2})!$ different $((n+{n \choose 2})!)-$tuples and every $1-$dimensional marginal of this $(n+{n \choose 2})!-$dimensional marginal is the same set of $(n+{n \choose 2})$ distinct elements. For every $f,g$ we have that 
%$$\mathcal{L}(f_1,\ldots,f_{m},f,g)=\mathcal{L}(h_1,\ldots,h_m,\tilde{f},\tilde{g})$$
%such that  
%$$\mu_m=\mathcal{L}(f_1,\ldots,f_{m})=\mathcal{L}(h_1,\ldots,h_m)$$ if and only if for the same $\phi$ bijective and measure-preserving: $$\tilde{f}=f^{\phi}$$ and $$\tilde{g}=g^{\phi}$$%$$
%h_i=f_i^{\phi}$$
%for every $i\in [m]$ 
%where for a symmetric matrix $f$ on $[n]\times [n]$ we denote $f^{\phi }(i,k)=f(\phi_1(i,k),\phi_2(i,k))=f_{\phi_1(i,k),\phi_2(i,k)}$.
%$$
%(f,g)=^d (f^{\phi},g^{\phi})
%$$%and $g^{\phi }(i,k)=g(\phi_1(i,k),\phi_2(i,k))=g_{\phi_1(i,k),\phi_2(i,k)}$. 

%Notice that for a function $$\begin{aligned}
%&\phi: ([n]\times [n],Sym)\longrightarrow ([n]\times [n],Sym)
%\\ & (i,k)\mapsto\phi(i,k)=(\phi_1(i,k),\phi_2(i,k))
%\end{aligned}$$
Because, in general, to have 

$$
(f,g, T_2[f,g])=^d (f^{\phi},g^{\phi},  (\widetilde{T}_2[f^{\phi},g^{\phi}]))
$$
we need 
$$(T_2[f,g])^{\phi}=\widetilde{T}_2[f^{\phi},g^{\phi}]).$$

Therefore, we can compare the two terms
$$
\begin{aligned}
(T_2[f,g])_{i,k}^{\phi}=\frac{1}{2}(\sum^n_{j=1}T_{j,\phi_1(i,k),\phi_2(i,k)}f_{j,\phi_1(i,k)}g_{j,\phi_2(i,k)} +\sum^n_{j=1}T_{j,\phi_2(i,k),\phi_1(i,k)}f_{j,\phi_2(i,k)}g_{j,\phi_1(i,k)})
\end{aligned}
$$and
$$
\widetilde{T_2}[f^{\phi},g^{\phi}]_{i,k}=\frac{1}{2}(\sum^n_{j=1}\widetilde{T}_{j,i,k}f_{\phi_1(j,i),\phi_2(j,i)}g_{\phi_1(j,k),\phi_2(j,k)} +\sum^n_{j=1}\widetilde{T}_{j,i,k}f_{\phi_1(j,k),\phi_2(j,k)}g_{\phi_1(j,i),\phi_2(j,i)})
$$
Now, we choose $f=\mathbbm{1}_{\{\phi_1(i,k),a\}}$ and $g=\mathbbm{1}_{\{a,\phi_2(i,k)\}}$ where $\mathbbm{1}_{\{c,d\}}$ is the indicator function of the set $\{(c,d),(d,c)\}$. Then we have 
$$
\begin{aligned}
&(T_2[f,g])_{i,k}^{\phi}=\frac{1}{2}(\sum^n_{j=1}T_{j,\phi_1(i,k),\phi_2(i,k)}{\mathbbm{1}_{\{\phi_1(i,k),a\}}}_{j,\phi_1(i,k)}{\mathbbm{1}_{\{a,\phi_2(i,k)\}}}_{j,\phi_2(i,k)} +\\
&\sum^n_{j=1}T_{j,\phi_2(i,k),\phi_1(i,k)}{\mathbbm{1}_{\{\phi_1(i,k),a\}}}_{j,\phi_2(i,k)}{\mathbbm{1}_{\{a,\phi_2(i,k)\}}}_{j,\phi_1(i,k)})=\\
&\frac{1}{2}T_{a,\phi_1(i,k),\phi_2(i,k)}
\end{aligned}
$$and
\begin{equation*}
\begin{aligned}
&\widetilde{T_2}[f^{\phi},g^{\phi}]_{i,k}=\frac{1}{2}(\sum^n_{j=1}\widetilde{T}_{j,i,k}{\mathbbm{1}_{\{\phi_1(i,k),a\}}}_{\phi_1(j,i),\phi_2(j,i)}{\mathbbm{1}_{\{a,\phi_2(i,k)\}}}_{\phi_1(j,k),\phi_2(j,k)} + \\ &\sum^n_{j=1}\widetilde{T}_{j,i,k}{\mathbbm{1}_{\{\phi_1(i,k),a\}}}_{\phi_1(j,k),\phi_2(j,k)}{\mathbbm{1}_{\{a,\phi_2(i,k)\}}}_{\phi_1(j,i),\phi_2(j,i)})
\end{aligned}
\end{equation*}
From the second expression, we can notice that for an element of the sum to be non-zero it is necessary that one of the following sets of conditions is satisfied:

\begin{equation}\label{eqn:Cond1nonzero}
\begin{aligned}
&\phi_1(i,k)=\phi_1(d,i)\\
&a=\phi_2(d,i)=\phi_2(d,k)\\
&\phi_2(i,k)=\phi_1(d,k)
\end{aligned}
\end{equation}
\begin{equation}\label{eqn:Cond2nonzero}
\begin{aligned}
&\phi_1(i,k)=\phi_2(d,i)\\
&a=\phi_1(d,i)=\phi_2(d,k)\\
&\phi_2(i,k)=\phi_1(d,k)
\end{aligned}
\end{equation}

\begin{equation}\label{eqn:Cond3nonzero}
\begin{aligned}
&\phi_1(i,k)=\phi_1(d,i)\\
&a=\phi_2(d,i)=\phi_1(d,k)\\
&\phi_2(i,k)=\phi_2(d,k)
\end{aligned}
\end{equation}
\begin{equation}\label{eqn:Cond4nonzero}
\begin{aligned}
&\phi_1(i,k)=\phi_2(d,i)\\
&a=\phi_1(d,i)=\phi_1(d,k)\\
&\phi_2(i,k)=\phi_2(d,k)
\end{aligned}
\end{equation} 
We observe that varying $a$ we accordingly vary $d$ as $\phi$ is bijective. In fact, for all conditions \eqref{eqn:Cond1nonzero}, \eqref{eqn:Cond2nonzero}, \eqref{eqn:Cond3nonzero} and \eqref{eqn:Cond4nonzero} if there would be two distinct $d$ and $\tilde{d}$ in $[n]$ corresponding to the same $a$ then $\phi$ would fail to be bijective. For this reason, we obtain from the conditions \eqref{eqn:Cond1nonzero},\eqref{eqn:Cond2nonzero},\eqref{eqn:Cond3nonzero} and \eqref{eqn:Cond4nonzero} that $\phi_1$ (respectively $\phi_2$) depend only on the second variable. Moreover, we notice that a necessary condition to be bijective and measure-preserving (measurable) for $\phi$ is 
\begin{equation}\label{eqn:condMeasPreBiject}
\begin{aligned}
&\phi_1(i,k)=\phi_2(k,i).\\
%&\phi_2(i,k)=\phi_1(k,i)
\end{aligned}
\end{equation}


Therefore, we notice that conditions \eqref{eqn:Cond2nonzero} and \eqref{eqn:Cond3nonzero} would contradict condition \eqref{eqn:condMeasPreBiject}. In conclusion, we can only have from \eqref{eqn:Cond1nonzero} and \eqref{eqn:condMeasPreBiject} that 

$$
\phi_1(i,j)=\psi(j)
$$
$$
\phi_2(i,j)=\psi(i)
$$
or from \eqref{eqn:Cond4nonzero} and \eqref{eqn:condMeasPreBiject} that 
$$
\phi_1(i,j)=\psi(i)$$
$$
\phi_2(i,j)=\psi(j).
$$
Therefore, substituting and requiring that $$
(T_2[f,g])^{\phi}=\widetilde{T_2}[f^{\phi},g^{\phi}]
$$
we obtain that 

$$T_{\psi(d),\psi(i),\psi(k)}=T_{a,\psi(i),\psi(k)}=T_{a,\phi_1(i,k),\phi_2(i,k)}=2(T_2[f,g])_{i,k}^{\phi}=2\widetilde{T_2}[f^{\phi},g^{\phi}]_{i,k}=\widetilde{T}_{d,i,k}.$$
\endproof
This result holds more generally as explained in the following remark.
\begin{remark}
We can use the same reasoning as in the proof of Theorem \ref{IdentificationTensors} to show more generally that the $r-1$-actions of two $r-$th order symmetric tensors $T=(T_{i_1,\ldots,i_r})_{i_1,\ldots,i_r\in [n]}$ and $\widetilde{T}=(\widetilde{T}_{i_1,\ldots,i_r})_{i_1,\ldots,i_r\in [n]}$ are completely determined by the action convergence distance, i.e.\ their $(r-1)-$actions are at action convergence distance $d_M$ zero %generate exactly the same measures (with multiplicity), i.e.  for every measure %\mu\in \mathcal{P}(\mathbb{R}^r)$ there are exactly $t_{\mu}\in\mathbb{N}$  different $r-1$-tuples of functions $(f_1,\ldots,f_{r-1})$ and there are exactly $t_{\mu}\in\mathbb{N}$  different $r-1$-tuples of functions  $(g_1,\ldots,g_{r-1})$ such that the distribution of the random vectors
%$$
   % \mathcal{L}(f_1,\ldots,f_{r-1},(T)_2[f_1,\ldots,f_{r-1}])=\mu=\mathcal{L}(g_1,\ldots,g_{r-1},(\widetilde{T})_2[g_1,\ldots,g_{r-1}])
%$$
if and only if  

$$
T_{\psi(i_1),\ldots,\psi(i_{r})}=\widetilde{T}_{i_1,\ldots,i_r}.
$$
In fact, similarly to the case $r=3$, 
%$$
% \mathcal{L}(f^{(1)}_1,\ldots,f^{(1)}_{r-1},\ldots,f^{(n!)}_1,\ldots,f^{(n!)}_{r-1})=\mathcal{L}(g^{(n!)}_1,\ldots,g^{(1)}_{r-1},\ldots,g^{(1)}_1,\ldots,g^{(n!)}_{r-1})
%$$
there must exist a  bijective and measure-preserving transformation 
$$\phi=(\phi_1,\ldots,\phi_{r}):([n]^{r-1},Sym)\longrightarrow ([n]^{r-1},Sym)$$
such that $$
((T)_{r-1}[f_1,\ldots,f_{r-1}])^{\phi}=(\widetilde{T})_{r-1}[f^{\phi}_1,\ldots,f^{\phi}_{r-1}]
$$
for all $f_1,\ldots,f_{r-1}$ symmetric $(r-1)-$th order tensors and where for a symmetric $(r-1)-$th order tensor $f$ we define $$f^{\phi}(i_1,\ldots,i_{r-1})=f(\phi_1(i_1),\ldots,\phi_{r-1}(i_{r-1})).$$
Moreover, using the test functions $f_s=\mathbbm{1}_{\{a,\phi_1(i_1,\ldots,i_{r-1}),\ldots,\hat{\phi}_s(i_1,\ldots,\phi_{r-1}),\ldots, \phi_{r-1}(i_1,\ldots ,\ldots,i_{r-1})\}}$, where $\mathbbm1_{\{a_1,\ldots,a_r\}}$ represents the indicator function of the set $$\{(a_{\sigma(1)},\ldots,a_{\sigma(r)})\in [n]^{r-1}: \sigma \text{ is a permutation of } [r-1]\},$$ the conditions on the $\phi_i$ imposed by the fact that $\phi$ is measure-preserving and bijective we obtain that  for a permutation $\sigma$ of $[r-1]$ we have $$\phi(i_1,\ldots,i_{r-1})=%(\psi\otimes \ldots \otimes \psi )(i_{\sigma(1)},\ldots,i_{\sigma(r-1)})=
(\psi(i_{\sigma(1)}),\ldots,\psi(i_{\sigma(r-1)}))$$
where $$\psi:[n]\rightarrow[n]$$
is a bijective map. 
\end{remark}

The previous theorem has the following direct important corollary:

\begin{corollary}
For two hypergraphs $H_1$ and $H_2$ with maximal edge cardinality $r$ the $(r-1)-$actions of their adjacency tensors $A(H_1)$ and $A(H_2)$ (that are $r-$th order tensors) are identified by the action convergence metric $d_M$ if and only if the hypergraphs $H_1$ and $H_2$ are isomorphic.
\end{corollary}

We conjecture that the previous theorem and remark can be generalized to any $s-$action ($s\in [r-1]$) of an $r-$th order tensor. The $1-$action case is trivial and we showed in the previous theorem and remark the $(r-1)-$action case.

We underline here a few properties of the action of (normalized) adjacency matrices of hypergraphs and, therefore, also of their limits by Lemma \ref{LemmaClosedSymmetric} and Lemma \ref{LemmaClosedPosPres}. 

First of all, we notice that the actions of (normalized) adjacency tensors are obviously positivity-preserving multi-$P-$operators and, therefore, their action convergence limits are too.
The following lemma and remark state that the action of a symmetric tensor is a symmetric multi-$P-$operator. 
\begin{lemma}
For a $3-$rd order symmetric tensor $T=((T)_{i,j,k})_{i,j,k\in [n]}$ the multi-$P-$operator $(T)_2$ is symmetric. 
\end{lemma}
\proof
The result follows from the following equality
\begin{equation}
\begin{aligned}
\mathbb{E}[(T)_2[f,g]h]&=\frac{1}{n^2}\sum^n_{i,k=1}\frac{1}{2}(\sum^n_{j=1}T_{i,j,k}f_{i,j}g_{j,k}+\sum^n_{j=1}T_{i,j,k}g_{i,j}f_{j,k})h_{i,k}\\
&=\frac{1}{n^2}\sum^n_{i,j,k=1}T_{i,j,k}f_{i,j}g_{j,k}h_{i,k}\\
&=\mathbb{E}[(T)_2[h,g]f]\\
&=\mathbb{E}[(T)_2[f,h]g]
\end{aligned}
\end{equation}
\endproof
\begin{remark}
Similarly, the $s-$action of a symmetric $r-$th order $n-$dimensional symmetric tensor $T$ is symmetric for every $s\in [r-1]$ by a similar computation.
\end{remark}

Therefore, the limit of the sequence of symmetric tensors will also be symmetric for Lemma \ref{LemmaClosedSymmetric}.


\section{Sparse and non-uniform hypergraphs and different tensors}\label{NonUnifHypSect}
In this section, we study how one can use action convergence for sparse hypergraphs and for hypergraphs with different edge cardinalities (non-uniform hypergraphs), without losing information about edges with non-maximal cardinality. %Considering the multi-$P-$operator $(T)_2$ associated with a $3-$rd order tensor $T$ we introduced in the previous sections, i.\ e.\ 
First of all, we discuss here how the sparseness of the hypergraphs interacts with our notions of Action convergence.
We underline that the $2-$action for $3-$uniform hypergraphs might not be the best choice for sparser hypergraphs and the $1-$action might be sometimes more appropriate as the following example shows.

\begin{example}
Consider the $3-$uniform hypergraph $T(n,s_n)$ given by the triangles of the sparse Erdős–Rényi random graph $G(n,s_n)$ where $s_n \rightarrow 0 $ and $s_n n\rightarrow \infty$. Let's denote with $E_n$ the set of edges of $G(n,s_n)$. In this case, for every $f,g$ symmetric matrices $\mathcal{L}(A(H)[f,g])=\delta_0$.  In fact, if we consider the multi-$P-$operator
$$\widetilde{A}:(L^{\infty}([n]\times [n],Sym,\mathbb{P}_n))^2\longrightarrow L^{1}([n]\times [n],Sym,\mathbb{P}_n)$$
$$\widetilde{A}[f,g]_{i,k}=\frac{1}{2}(\sum^n_{j=1}\frac{A_{j,i,k}}{s_n}f_{j,i},g_{j,k} +\sum^n_{j=1}\frac{A_{j,i,k}}{s_n}f_{j,k},g_{j,i})
$$
where $\mathbb{P}_n$ is the uniform measure on $[n]\times [n]$, $\widetilde{A}[f,g]_{i,k}\neq 0$ if and only if $(i,k)\in E_n$. But as $n\rightarrow \infty$, $\mathbb{P}_n(E_n)\rightarrow 0$. For this reason, it might be appropriate to consider the $1-$action or change the probability measures $\mathbb{P}_n$ in such a way that $\mathbb{P}_n$ converges to some positive constant (for example choose $\mathbb{P}_n$ as the uniform probability measures on $E_n$).
\end{example}

Moreover, in some contexts, one might be specifically interested in the limit of the $1-$actions of the adjacency tensors, see for example \cite{kuehn2023vlasov} in the context of dynamical systems of networks.

\begin{remark}
    However, we remark that there are very sparse $3-$uniform hypergraphs for which the $2-$action works without any problem, i.e. the limit measures do not trivialize. Consider for example the $3-$uniform hypergraph $H$ with vertex set $[n]$ and as edges the sets $\{\{i,i+1,i+2\}: \ i\in [n-2] \}$. The requirement is that (almost) every pair of vertices is covered by at least one edge. 
\end{remark}  
Now, we present some possible choices to adapt Action convergence to non-uniform hypergraphs.

In fact, considering the $2-$action (Definition \ref{DefAction}) associated with a (normalized) adjacency tensor of a hypergraph $H$ 

$$\widetilde{A}:(L^{\infty}([n]\times [n],Sym,\mathbb{P}_n))^2\longrightarrow L^{1}([n]\times [n],Sym,\mathbb{P}_n)$$
$$\widetilde{A}[f,g]_{i,k}=\frac{1}{2}(\sum^n_{j=1}\frac{A_{j,i,k}}{n}f_{j,i},g_{j,k} +\sum^n_{j=1}\frac{A_{j,i,k}}{n}f_{j,k},g_{j,i})
$$

we notice that considering the probability space $[n]\times [n]$ with uniform probability $\mathbb{P}_n$ (and the symmetric $\sigma-$algebra) the diagonal, i.e. the set 
$$
D_n=\{(i,i): i\in [n]\}\subset [n]\times [n]
$$
has probability $\mathbb{P}_n(D_n)=\frac{n}{n^2}=\frac{1}{n}$. Therefore, in the limit $n\rightarrow \infty$ we have that the edges of cardinality $2$ do not play any role in the profile measures of the multi-linear operator. However, we can choose other probability measures $\mathbb{P}_n$ different from the uniform distribution so that the information from the edges with lower cardinality is not lost. A natural choice for $\mathbb{P}_n$ is the discrete measure defined by $\mathbb{P}_n(\{(i,i)\})=\frac{1}{2n}$ and $\mathbb{P}_n(\{(i,j), (j,i)\}) =\frac{1}{2n(n-1)}$. This obviously characterizes uniquely the discrete probability measure $\mathbb{P}_n$. In this case, $\mathbb{P}_n(D_n)=\frac{1}{2}$ and, therefore, the lower cardinality edges play a role in the construction of the profiles and therefore of the limit object.


\begin{remark}
This construction of this  probability measure can be naturally generalized for the case $k>3$ where $\Omega=[n]^k$ with the symmetric sigma-algebra.
\end{remark}

As simplicial complexes are a special case of general hypergraphs we obtain in such a way a notion of convergence for dense simplicial complexes.  Interest in a notion of convergence for dense simplicial complexes, similar to the one for dense graphs (graphons), has been expressed in  \cite{Bobrowski2022} describing it as a “potentially very interesting direction of future research in mathematics of random complexes”. Therefore, the study of this convergence and the relative limit objects might be of special interest.

We have seen that we have different possible choices for the probability measures $\mathbb{P}_n$. We obviously have also many possible options for choosing different tensors and different normalizations of these tensors.

In fact, the (normalized) adjacency tensor is not the only tensor we can associate with a hypergraph. One possibility is to normalize dividing  every entry of the adjacency tensor by the quantity 

$$
\begin{aligned}
deg(i_1,\ldots, i_{k-1})=|\{e\in E \ \text{ s.t. }\ i_1,\ldots, i_{k-1}\in e  \\
\text{ and } |e|=|\{i_1,\ldots, i_{k-1}\}|+1\}|
\end{aligned}
$$in the following way 


$$
\widetilde{A}_{i_1,\ldots, i_{k}}=\frac{A_{i_1,\ldots, i_{k}}}{deg(i_1,\ldots, i_{k-1})}.
$$
It is easy to notice that $$deg(i_1,\ldots, i_{k-1})\leq |V|-k+1\leq |V|$$

In the particular case $k=3$ we have

$$
\widetilde{A}_{i,j,k}=\frac{A_{i,j,k}}{deg(i,k)}
$$

This is interesting for inhomogeneous hypergraphs and for hypergraphs with different edge cardinality.
In fact, we can define on $\Omega=[n]\times [n]$ the probability measure $$\mathbb{P}_n(\{(i,j)\})=\frac{deg(i,j)}{2\sum^n_{i,j=1,\ i\neq j}deg(i,j)}$$ if $i\neq j$ and $$\mathbb{P}_n(\{(i,i)\})=\frac{deg(i,i)}{2\sum^n_{i=1}deg(i,i)}.$$


These operators are also symmetric with respect to the right probability measure.

\begin{lemma}
    The operator $(\widetilde{A})_2$  is symmetric with respect to the probability measure $\mathbb{P}_n$. 
\end{lemma}
\proof
The lemma follows from the following equality
$$\begin{aligned}
&\mathbb{E}[(\widetilde{A})_2[f,g]h]=\\
&\frac{1}{2}(\sum^n_{i,k=1, \ i\neq k}(\sum^n_{j=1}\frac{A_{i,j,k}}{deg(i,k)}f_{i,j}g_{j,k}+\sum^n_{j=1}\frac{A_{i,j,k}}{deg(i,k)}g_{i,j}f_{j,k}))h_{i,k}\frac{deg(i,k)}{2\sum^n_{i,k=1,\ i\neq k}deg(i,k)}+\\
& \frac{1}{2}(\sum^n_{i}(\sum^n_{j=1}\frac{A_{i,j,i}}{deg(i,i)}f_{i,j}g_{j,i}+\sum^n_{j=1}\frac{A_{i,j,i}}{deg(i,i)}g_{i,j}f_{j,i}))h_{i,i}\frac{deg(i,i)}{2\sum^n_{i=1}deg(i,i)})=\\&
\frac{1}{2}(\sum^n_{i,k=1, \ i\neq k}(\sum^n_{j=1}A_{i,j,k}f_{i,j}g_{j,k}+\sum^n_{j=1}A_{i,j,k}g_{i,j}f_{j,k}))h_{i,k}\frac{1}{2\sum^n_{i,k=1,\ i\neq k}deg(i,k)}+\\
& \frac{1}{2}(\sum^n_{i}(\sum^n_{j=1}A_{i,j,i}f_{i,j}g_{j,i}+\sum^n_{j=1}A_{i,j,i}g_{i,j}f_{j,i}))h_{i,i}\frac{1}{2\sum^n_{i=1}deg(i,i)})=\\&
\mathbb{E}[(\widetilde{A})_2[f,h]g]=\mathbb{E}(\widetilde{A})_2[h,g]f].
\end{aligned}
$$\endproof
Therefore, the limit of a sequence of such operators will be also symmetric and positivity-preserving by Lemma \ref{LemmaClosedSymmetric} and Lemma \ref{LemmaClosedPosPres}.


\begin{remark}
    The previous lemma can be easily generalized for the case $k>3.$
\end{remark}

\section{Multi-action convergence and hypergraphons}\label{SectHypergraphons}
We recall that a graphon is a measurable function 

$$
W:[0,1]^2\longrightarrow[0,1]$$
where  $[0,1]$ is equipped with the Lebesgue measure. As already explained, we can interpret the graphon $W$ as a $P-$operator 
$$\widetilde{W}f(y)=\int_{[0,1]}W(x,y)f(x)\mathrm{d}x.$$
There are several equivalent ways to define a convergence notion for (isomorphism classes of) graphons and therefore for dense graphs. For a complete treatment of the topic see the monography \cite{LovaszGraphLimits}.

One possibility is the convergence of the homomorphism densities, also called left-convergence.

We can naturally identify a graph $G$ with a graphon $W_G$ as the $\{0,1\}-$valued  step function  
$$
W_G(x,y)= \sum^{|V(G)|}_{i,j=1}A(G)_{i,j}\mathbbm{1}_{S_i}(x)\mathbbm{1}_{S_j}(y)
$$
where $\{S_i\}_{i\in V(G)}$ is a partition of the unit interval $[0,1]$ in intervals $S_i$ of size $\frac{1}{|V(G)|}$.
Moreover, for a graphon $W$ we can define the homomorphism densities  

$$
t(F,W)=\int_{[0,1]^{|V(F)|}}\prod_{(i,j)\in E(F)}W(x_i,x_j)\prod_{i\in V(F)}\mathrm{d}x_i
$$
for every finite graph $F$.

For a graph $G$ and its associated graphon $W_G$ we have that 
$$
t(F,W_G)=t(F,G)=\frac{|\{\phi : F\rightarrow G \text{ is an homomorphism}\}|}{|V(G)|^{|V(F)|}}
$$
i.e.\ it is the probability that a map from $F$ to $G$ chosen uniformly at random is a graph homomorphism.

We say that a sequence of graphons $(W_n)_n$ is left-convergent if the sequence of the homomorphism densities 
$$(t(F,W_n))_n$$ converges for every finite graph $F$.

Another possibility to define convergence for graphons is through the convergence in cut distance, i.e.\ the convergence in the (pseudo-)metric 
\begin{equation}\label{eqn:cutdist}
    \delta_{\square}(W,U)=\inf_{\phi\in \mathcal{MB}([0,1])}\|W-U^{\phi}\|_{\square}
\end{equation}
for two graphons $U$ and $W$

where 
$$\mathcal{MB}([0,1])=\{\phi:[0,1]\rightarrow[0,1] \text{ bijective and measure-preserving }\}$$
  and $U^{\phi}(x,y)=U(\phi(x),\phi(y))$.

  and 
$$\|W\|_{\square}=|\int_{[0,1]^2}W(x,y)\mathrm{d}x\mathrm{d}y|$$ 
is the cut norm. Moreover, from Theorem \ref{TheoremEqCutMetr} (Theorem 3.13 in \cite{backhausz2018action}) we have that these convergence notions are equivalent to the action convergence of the sequence of the normalized adjacency matrices 

$$
\frac{A(G_n)}{|V(G_n)|}.
$$

The theory of dense $r-$uniform hypergraph limits (hypergraphons) has been developed in \cite{hypergrELEK20121731} using techniques from model theory (ultralimits, ultraproducts) and successively translated in a more standard graph limit  language in \cite{HypergraphonsZhao}. A good presentation of the model-theoretic approach is given in \cite{RandomneLimitTow}. We briefly present here the theory of dense hypergraph limits, highlighting the similarities with action convergence, following the analytic presentation in \cite{HypergraphonsZhao}. Moreover, we conjecture that the action convergence of the normalized adjacency matrices of a sequence of dense $r-$uniform hypergraphs is equivalent to the convergence of dense hypergraphs (hypergraphons).

We start with some notation. For any subset $A\subset [n]$, define $r(A)$ to be the collection of all nonempty subsets of $A$, and $r_{<}(A)$ to be the collection of all nonempty proper subsets of $A$. More generally, let $r(A, m)$ denote the collection of all nonempty subsets of $A$ of size at most $m$. So for instance, $r_{<}([k])=r([k], k-1)$. We will also use the shorthand $r[k]$ and $r_{<}[k]$ to mean $r([k])$ and $r_{<}([k])$ respectively.

Any permutation $\sigma$ of a set $A$ induces a permutation on $r(A, m)$. For a set $A=\{v_1,\ldots v_t\}\subset [k]$ of cardinality $t$ where $v_1<\ldots <v_t$, we indicate with $\mathrm{x}_A=(x_{v_1},\ldots,x_{v_t},x_{v_1v_2}\ldots,x_{v_1\ldots v_t} )$.   %We say that a function $W:[0,1]^{r([k], m)} \rightarrow[0,1]$ is symmetric if it remains invariant under any permutation of the coordinates induced by any permutation of $[k]$. For example, $W:[0,1]^{r<[3]} \rightarrow[0,1]$ being symmetric means that

Quite surprisingly, the limit object of a sequence of $k-$uniform hypergraphs, i.e. an hypergraphon, is a symmetric measurable function
$$
W:[0,1]^{2^r-2}\longrightarrow [0,1].
$$
$$
W(x_1,\ldots, x_r,x_{12},\ldots,x_{(r-1)r},\ldots x_{12\ldots r-1},\ldots,x_{2\ldots,r})
$$
where symmetric means that 
\begin{equation*}
\begin{aligned}
   & W(x_1,\ldots, x_r,x_{12},\ldots,x_{(r-1)r},\ldots x_{12\ldots r-1},\ldots,x_{2\ldots r})=\\
  &  W(x_{\sigma(1)},\ldots, x_{\sigma{(r)}},x_{\sigma{(1)}\sigma{(2)}},\ldots,x_{\sigma(r-1)\sigma(r)},\ldots x_{\sigma(1)\sigma(2)\ldots \sigma(r-1)},\ldots,x_{\sigma(2)\ldots \sigma(r)})
    \end{aligned}
\end{equation*}

for every permutation $\sigma$ of $[r]$.

This might be surprising because, differently from the case of graphs ($r=2$), for $r>2$ the dimensionality of the $r-$th order adjacency tensor associated to a $r-$uniform hypergraph, $r$,  does not coincide with the dimensionality of the hypergraphon, $2^r-2$.

The need for the additional coordinates, representing all proper subsets of $[r]$, is related to the need for suitable regularity partitions for hypergraphs \cite{GowersHypRegularity,RodlHyperReg1, RodlHyperReg2}  and it is moreover related to the hierarchy of notions of quasi-randomness in the case of $r-$uniform hypergraphs for $r>2$  \cite{RandomneLimitTow}.

This is also intuitively related to the various multi-$P$-operators associated with a tensor through its $s-$actions. In fact for $r=3$ the additional coordinates are again needed, for example, to differentiate the limits of the sequence of the Erdős–Rényi $3-$uniform hypergraphs $G(n,\frac{1}{8},2)$ (Example \ref{ERRandomHypergraph}) and the sequence of the $3-$uniform hypergraphs $T(n,\frac{1}{2})$ given by the triangles of the Erdős–Rényi graph (Example \ref{TriangRandHyp}).



We notice that similarly to how we associated graphons to $P-$operators we can associate hypergraphons to multi-$P-$operators:

$$
\widetilde{W}:L^{\infty}([0,1]^{2^{r-1}-2},Sym)\times \ldots \times L^{\infty}([0,1]^{2^{r-1}-2},Sym)\longrightarrow L^1([0,1]^{2^{r-1}-2},Sym)
$$
$$
\widetilde{W}(\mathrm{x}_{r([r]\setminus \{r\})})=\frac{1}{(r-1)!}\sum_{\sigma}\int_{[0,1]^{2^r-2^{r-1}}}W(\mathrm{x}_{r[r]})\prod^{r-1}_{i=1} f_{\sigma(i)}(\mathrm{x}_{r_{[r]\setminus\{i\}}})\mathrm{d\mathrm{x}_{r([r]\setminus\{r\})}}
$$
where $\sigma$ here is a permutation of $[r-1]$ and $Sym$ is the symmetric $\sigma-$algebra (i.e.\ the $\sigma-$algebra generated by the subsets of  $[0,1]^{2^{r-1}-2}$ that are invariant under the action of all permutations of $[r-1]$).

We notice that there are promising similarities between Action convergence of hypergraphons and the $k-1$th action multi-convergence of the adjacency tensor.

Let's consider for example the hypergraphon,
$$
W(x_1,x_2,x_{12})=\begin{cases}
    1 \ \text{ if } 0\leq x_{12}\leq \frac{1}{2}\\
    0 \ \text{ else}
\end{cases}
$$

that is the limit of the sequence of hypergraphs $T(n,\frac{1}{2})$ given by the triangles of the Erdős–Rényi random graph and the action convergence limit of the $2-$action $(A_n)_2$ of the sequence of tensors $(A_n)_n$ obtained normalizing the adjacency tensors of the same hypergraphs, i.e.\ $A_n=\frac{A(T(n,\frac{1}{2}))}{n}$,  we have, for example, that
$$ 
\mathcal{L}(\mathbbm{1}_{\Omega_n},\mathbbm{1}_{\Omega_n},(A_n)_2[\mathbbm{1}_{\Omega_n},\mathbbm{1}_{\Omega_n}])\rightarrow  \frac{1}{2}\delta_{(1,1,0)}+\frac{1}{2}\delta_{(1,1,\frac{1}{4})}=\mathcal{L}(\mathbbm{1}_{\Omega},\mathbbm{1}_{\Omega},\widetilde{W}[\mathbbm{1}_{\Omega},\mathbbm{1}_{\Omega}])
$$ also if we take the set $S_n$ to be the pairs that correspond to edges of $G(n,\frac{1}{2})$. Then also   
$$ \mathcal{L}(\mathbbm{1}_{S_n},\mathbbm{1}_{S_n},(A_n)_2[\mathbbm{1}_{S_n},\mathbbm{1}_{S_n}])\rightarrow  \mathcal{L}(\mathbbm{1}_{[0,1]^2\times [0,\frac{1}{2}]},\mathbbm{1}_{[0,1]^2\times [0,\frac{1}{2}]},\widetilde{W}[\mathbbm{1}_{[0,1]^2\times [0,\frac{1}{2}]},\mathbbm{1}_{[0,1]^2\times [0,\frac{1}{2}]}])
$$
and, similarly,

$$\mathcal{L}(\mathbbm{1}_{S^c_n},\mathbbm{1}_{S^c_n},(A_n)_2[\mathbbm{1}_{S^c_n},\mathbbm{1}_{S^c_n}])\rightarrow \mathcal{L}(\mathbbm{1}_{[0,1]^2\times [\frac{1}{2},1]},\mathbbm{1}_{[0,1]^2\times [\frac{1}{2},1]},\widetilde{W}[\mathbbm{1}_{[0,1]^2\times [\frac{1}{2},1]},\mathbbm{1}_{[0,1]^2\times [\frac{1}{2},1]}])$$
and 

$$\mathcal{L}(\mathbbm{1}_{S_n},\mathbbm{1}_{S^c_n},(A_n)_2[\mathbbm{1}_{S_n},\mathbbm{1}_{S^c_n}])\rightarrow \mathcal{L}(\mathbbm{1}_{[0,1]^2\times [0,\frac{1}{2}]},\mathbbm{1}_{[0,1]^2\times [\frac{1}{2},1]},\widetilde{W}[\mathbbm{1}_{[0,1]^2\times [0,\frac{1}{2}]},\mathbbm{1}_{[0,1]^2\times [\frac{1}{2},1]}])$$

We briefly present now the results in \cite{HypergraphonsZhao}. 

\begin{definition} [Symmetric sets and partitions] A symmetric (measurable) subset of $[0,1]^{r[r]}$ is one which is invariant under the action of all permutations of $[r]$. A symmetric (measurable) partition of $[0,1]^{r[r]}$ is a partition of $[0,1]^{r[r]}$ into a finite collection of symmetric subsets.
\end{definition}

A symmetric subset $P \subseteq[0,1]^{r[k]}$ is associated to a $k$-hypergraphon $W^P:[0,1]^{r_<[k]} \rightarrow[0,1]$ by integrating out the top coordinate:
$$
W^P\left(\mathrm{x}_{r_<[k]}\right):=\int_0^1 1_P\left(\mathrm{x}_{r[k]}\right) d x_{1\ldots k} .
$$
For example, for $k=3$, we have $P \subseteq[0,1]^3$, with coordinates indexed by $r[2]=\{1,2,12\}$, and
$$
W^P\left(x_1, x_2\right)=\int_0^1 1_P\left(x_1, x_2, x_{12}\right) d x_{12} .
$$
This operation collapses the final coordinate in $P$. It is helpful to think of $P$ and $W^P$ as representing the same object. For example, when $k=2$ this means it does not matter how $P$ is placed along the $x_{12}$ coordinate, as we only care about how much $P$ intersects line segments of the form $\left\{x_1\right\} \times\left\{x_2\right\} \times[0,1]$. And conversely, given a $W:[0,1]^2 \rightarrow[0,1]$, there are many $P \subseteq[0,1]^3$ satisfying $W^P=W$, for example, any set of the form $P=\{(x, y, z): a(x, y) \leq z \leq b(x, y)\}$ where $b(x, y)-a(x, y)=W(x, y)$. 

For example, we can represent the limit of a sequence of Erdős–Rényi graph realizations $G(n,p)$ as $P=[0,1]\times [0,1]\times [0,\frac{1}{2}]$ and the limit of the sequence of the complement graphs of these realizations  as $P^{c}= [0,1]\times [0,1]\times [\frac{1}{2},1]$. However, both limits have graphon representation $W(x,y)=\frac{1}{2}$.

We  define homomorphism densities not only for hypergraphons but more generally for $m-$tuples of hypergraphons. This more general definition is necessary for the constructions in the following.  

For any tuple of $k-$uniform hypergraphons $W=(W_1,\ldots,W_m)$, any $k-$uniform hypergraph $F$, and any map $\alpha: F\rightarrow [m]$, define the homomorphism density
\begin{equation}\label{eqn:homdensHyp}
t_{\alpha}(F,\mathbf{W})=\int_{[0,1]^{r(V(F),k-1)}}\prod_{e\in F}\mathbf{W}_{\alpha(e)}(x_{r<(e)})dx
\end{equation}
\begin{example}
If $k=2, \ F=K_3=\{12, 13, 23\}, \ \alpha=(12\mapsto 1,\ 13\mapsto 2,\ 23\mapsto 3)$, then
$$
t_{\alpha}(F,\mathbf{W})=\int_{[0,1]^3}W_1(x_1,x_2)W_2(x_1,x_3)W_3(x_2,x_3)dx_1 dx_2 dx_3
$$
\end{example}

For any symmetric partition $\mathcal{P}=(P_1,\ldots,P_m)$ of $[0,1]^{r[k]}$, define 
$$
\mathbf{W}^{\mathcal{P}}=(W^{P_1},\ldots,W^{P_m})\ \text{           and          } \ t_{\alpha}(F,\mathcal{P})=t_{\alpha}(F,\mathbf{W}^{\mathcal{P}})
$$ Let $W:[0,1]^{r<[k]} \rightarrow[0,1]$ be a $k$-uniform hypergraphon and $\mathcal{Q}$ a symmetric partition of $[0,1]^{r[k-1]}$ into $q$ parts $Q_1, Q_2, \ldots, Q_q \subseteq[0,1]^{r[k-1]}$. The quotient $W / \mathcal{Q}$ is a $2 q^k$-tuple of numbers in $[0,1]$ defined by assigning to each $k$-tuple $f=\left(f_1, \ldots, f_k\right) \in[q]^k$ a pair $\left(v_f, w_f\right)$, referred to as (volume, average), as follows:
\begin{itemize}
\item Volume: $v_f$ equals the integral

$$
v_f:=\int_{\mathbf{x} \in[0,1]^{r<[k]}} 1_{Q_{f_1}}\left(\mathbf{x}_{r([k] \backslash\{1\})}\right) 1_{Q_{f_2}}\left(\mathbf{x}_{r([k] \backslash\{2\})}\right) \cdots 1_{Q_{f_k}}\left(\mathbf{x}_{r([k] \backslash\{k\})}\right) d \mathbf{x} .
$$

\item Average: If $v_f=0$, then we set $w_f=0$. Otherwise, $w_f$ is defined to be

$$
w_f:=\frac{1}{v_f} \int_{\mathbf{x} \in[0,1]^{r<[k]}} W\left(\mathbf{x}_{r_<[k]}\right) 1_{Q_{f_1}}\left(\mathbf{x}_{r([k] \backslash\{1\})}\right) 1_{Q_{f_2}}\left(\mathbf{x}_{r([k] \backslash\{2\})}\right) \cdots 1_{Q_{f_k}}\left(\mathbf{x}_{r([k] \backslash\{k\})}\right) d \mathbf{x}
$$
\end{itemize}

Intuitively, the partition $\mathcal{Q}$ induces a partition $\mathcal{Q}^*$ of $[0,1]^{r[k]}$ into parts enumerated by $f \in[q]^k$. Each cell of $\mathcal{Q}^*$ has a volume $v_f$ and an average value $w_f$ of $W$ on the cell.

Consider now the random vector

$$( 1_{Q_{1}},\ldots,1_{Q_{q}},\widetilde{W}[1_{Q_{f_2}},\ldots,1_{Q_{f_{r}}}])$$
and its distribution $\mathcal{L}( 1_{Q_{1}},\ldots,1_{Q_{q}},\widetilde{W}[1_{Q_{f_2}},\ldots,1_{Q_{f_{r}}}])$. The marginal of $\mathcal{L}( 1_{Q_{f_1}},\ldots,1_{Q_{f_{r}}},\widetilde{W}[1_{Q_{f_2}},\ldots,1_{Q_{f_{r}}}])$ on the first $q$ coordinates is
$$
\mathcal{L}( 1_{Q_{1}},\ldots,1_{Q_{q}})=\sum^q_{i=1}\mathcal{L}(Q_i)\delta_{(0,\ldots,\underbrace{1}_{i-th},\ldots,0)}
$$
In general, for a probability space $\Omega$  a measure $\mu \in \mathcal{P}(\R^{r})$ is the distribution of the random vector
$$
( 1_{P_{1}},\ldots,1_{P_{q}}), 
$$
i.e.\ $\mu=\mathcal{L}( 1_{P_{1}},\ldots,1_{P_{q}})$, where $P_1,\ldots,P_q$ is a partition of $\Omega$ if and only if 

$$
\mu=\sum^q_{i=1}p_i\delta_{(0,\ldots,\underbrace{1}_{i-th},\ldots,0)} 
$$

Therefore, we can easily extract the distribution of  

$$( 1_{Q_{1}},\ldots,1_{Q_{q}},\widetilde{W}[1_{Q_{f_2}},\ldots,1_{Q_{f_{r}}}])$$
from the profile $S_k(\widetilde{W})$ with $k$ large enough.
We remark that the quantities $v_f$ and $w_f$ can be expressed from the measure 
$$\mathcal{L}( 1_{Q_{f_1}},\ldots,1_{Q_{f_{r}}},\widetilde{W}[1_{Q_{f_2}},\ldots,1_{Q_{f_{r}}}])$$
in fact, 
$$
v_f=\mathbb{E}[1_{Q_{f_1}}\ldots1_{Q_{f_{r}}}]
$$
and 
$$
w_f=\frac{\mathbb{E}[\widetilde{W}[1_{Q_{f_1}},\ldots,1_{Q_{f_{r-1}}}]1_{Q_{f_{r}}}]}{v_f}.
$$



If we have another $k$-uniform hypergraphon $W^{\prime}$, and a symmetric partition $\mathcal{Q}^{\prime}$ of $[0,1]^{r[k-1]}$ into $q$ parts ($\mathcal{Q}$ and $\mathcal{Q}^{\prime}$ have the same number of parts) with volumes and weights $\left(v_f^{\prime}, w_f^{\prime}\right)$, we define
$$
d_1\left(W / \mathcal{Q}, W^{\prime} / \mathcal{Q}^{\prime}\right):=\sum_{f \in[q]^k}\left(\left|v_f-v_f^{\prime}\right|+\left|v_f w_f-v_f^{\prime} w_f^{\prime}\right|\right) .
$$
From the previous observation, it follows that if two $k-$uniform hypergraphons are close in the multi-action convergence distance, i.e.\
$$
d_M(W,W^{\prime})<\delta,
$$
then for every partition $\mathcal{Q}$ it exists a partition $\mathcal{Q}^{\prime}$ such that 
$$
d_1\left(W / \mathcal{Q}, W^{\prime} / \mathcal{Q}^{\prime}\right)<\varepsilon.$$




For any symmetric subset $P \subseteq[0,1]^{r \mid k]}$, we write
$$
P / \mathcal{Q}:=W^P / \mathcal{Q} \text {. }
$$
A $\mathcal{Q}$-step function $U:[0,1]^{r_{<}[k]} \rightarrow \mathbb{R}$ is a function of the form
\begin{equation}\label{eqn:HypGraIndicPart}
W_{\mathcal{Q}}(\mathbf{x}):=\sum_{f=\left(f_1, \ldots . f_k\right) \in[q]^k} u_f 1_{Q_{f_1}}\left(\mathbf{x}_{r([k] \backslash\{1\})}\right) 1_{Q_{f_2}}\left(\mathbf{x}_{r([k] \backslash\{2\})}\right) \cdots 1_{Q_{f_k}}\left(\mathbf{x}_{r([k] \backslash\{k\})}\right)
\end{equation}
for some real values $u_f$. 

Since $\mathcal{Q}$ is a partition, the indicator functions in \eqref{eqn:HypGraIndicPart} all have disjoint support, which together partition the domain $[0,1]^{r_{<}[ k]}$. Usually, $U$ is a symmetric function, which is equivalent to having an additional symmetry constraint on $u_f$, namely that $u_f=u_{f'}$ whenever $f'$ is obtained from $f$ by a permutation of the coordinates.

The $\mathcal{Q}$-stepping operator, denoted by a subscript $\mathcal{Q}$, turns a $k$-uniform hypergraphon $W$ into a symmetric $\mathcal{Q}$-step function $W_{\mathcal{Q}}$ by averaging over each induced cell of $\mathcal{Q}^*$. More precisely, we define $W_{\mathcal{Q}}:[0,1]^{r_{\leq}[k]} \rightarrow[0,1]$ to be (using $v_f$ and $w_f$ from $W / \mathcal{Q}$ defined earlier)
$$
W_{\mathcal{Q}}(\mathbf{x}):=\sum_{f=\left(f_1, \ldots, f_k\right) \in [q]^k} w_f 1_{Q_{f 1}}\left(\mathbf{x}_{r([k] \backslash\{1\}))}\right) 1_{Q_{f_2}}\left(\mathbf{x}_{r([k] \backslash \{2\} )}\right) \cdots 1_{Q_{f_k}}\left(\mathbf{x}_{r([k] \backslash [k])}\right)
$$
We can also apply the stepping operator to a tuple of hypergraphons. If $\mathbf{W}=$ $\left(W_1, \ldots, W_m\right)$, then
$$
\mathbf{W}_{\mathcal{Q}}:=\left(\left(W_1\right)_{\mathcal{Q}}, \ldots,\left(W_m\right)_{\mathcal{Q}}\right)
$$
In particular, if $\mathcal{P}=\left\{P_1, \ldots, P_m\right\}$ is a partition of $[0,1]^{r [ k]}$, then we write
$$
\mathbf{W}_{\mathcal{Q}}^{\mathcal{P}}:=\left(\left(W^{P_1}\right)_{\mathcal{Q}}, \ldots,\left(W^{P_m}\right)_{\mathcal{Q}}\right)=\left(W_{\mathcal{Q}}^{P_1}, \ldots, W_{\mathcal{Q}}^{P_m}\right)
$$
\begin{definition}
 For any symmetric function $W:[0,1]^{r[k]} \rightarrow \mathbb{R}$, define
\begin{equation}\label{HypCutNorm}
\|W\|_{\square^{k-1}}:=\sup _{\substack{u_1, \ldots, u_k:[0,1]^{r|k-1|} \rightarrow[0,1] \\ \text { symmetric }}}\left|\int_{[0,1]^{r<:|k|}} W\left(\mathbf{x}_{r_<[k]}\right) \prod_{i=1}^k u_i\left(\mathbf{x}_{r(|k| \backslash\{i\})}\right) d \mathbf{x}\right|\end{equation}
\end{definition}
Note that by the linearity of the expression inside the absolute value in \eqref{HypCutNorm}, it suffices to consider functions $u_i$'s which are indicator functions $1_{B_i}$ of symmetric subsets $B_i \subseteq$ $[0,1]^{r [ k-1]}$. The usual cut norm corresponds to the case $k=2$. The following example shows $k=3$.
\begin{example}
For any symmetric function $W:[0,1]^{r_{<}[3]} \rightarrow \mathbb{R},\|W\|_{\square^2}$ equals to
$$
\begin{aligned}
&\sup _{u_1, u_2, u_3} \mid \int_{10,\left.1\right|^6} W\left(x_1, x_2, x_3, x_{12}, x_{13}, x_{23}\right) u_1\left(x_2, x_3, x_{23}\right) u_2\left(x_1, x_3, x_{13}\right) u_3\left(x_1, x_2, x_{12}\right) \\
&d x_1 d x_2 d x_3 d x_{12} d x_{13} d x_{23} \mid
\end{aligned}
$$
where $u_1, u_2, u_3$ vary over all symmetric functions $[0,1]^{r[2]} \rightarrow[0,1]$.
\end{example} 

 We continue with the following
\begin{definition}\label{RegulDef1}
Let $W$ be a $k$-uniform hypergraphon and $\mathcal{Q}$ a symmetric partition of $[0,1]^{r [ k-1]}$. We say that $(W, \mathcal{Q})$ is weakly $\varepsilon$-regular if $\left\|W-W_{\mathcal{Q}}\right\|_{\square_{k-1}} \leq \varepsilon$.
For a symmetric subset $P \subseteq[0,1]^{r [ k]}$, we say that $(P, \mathcal{Q})$ is weakly $\varepsilon$-regular if $\left(W^P, \mathcal{Q}\right)$ is.
\end{definition} 

We recall here a regularity lemma for (sequences) of  hypergraphons from \cite{HypergraphonsZhao}.

\begin{lemma}\label{RegLemmaHyp}
 (Weak regularity lemma, 
 Lemma 5.2 in \cite{HypergraphonsZhao}). Let $k \geq 2$ and $\varepsilon>0$. Let $\mathbf{W}=\left(W_1, \ldots . W_m\right)$ be a tuple of $k$-uniform hypergraphons. Let $\mathcal{Q}$ be a symmetric partition of $[0,1]^{r[k-1]}$. Then there exists a partition $\mathcal{Q}^{\prime}$ refining $\mathcal{Q}$ so that every part of $\mathcal{Q}$ is refined into exactly $\left[2^{\frac{km}{\varepsilon^2}}\right]$ parts (allowing empty parts) so that $\left(W_i, \mathcal{Q}^{\prime}\right.$ ) is weakly $\varepsilon$-regular for every $1 \leq i \leq m$.
\end{lemma}

Instead of defining a convergence notion for hypergraphons it is needed to define a convergence notion for more general objects: 

\begin{definition}
A degree $p=\left(p_1, p_2, \ldots \right ) \in \N^{\N}$ (symmetric) branching partition $\mathscr{P}$ of $[0,1]^{r[k]}$ is a collection of symmetric subsets $P_i$ of $[0,1]^{r[k]}$, collected into levels, where each level $\mathcal{P}_l$ is a symmetric partition of $[0,1]^{r[k]}$:
\begin{itemize}
\item Level 0: $\mathcal{P}_0=\left\{[0,1]^{r[k]}\right\}$
\item Level 1: $\mathcal{P}_1=\left\{P_1, P_2, \ldots, P_{p_1}\right\}$ is a symmetric partition of $[0,1]^{r \mid k]}$.
\item Level $l(l \geq 2): \mathcal{P}_l$ is a refinement of $\mathcal{P}_{l-1}$, where each part of $\mathcal{P}_{l-1}$ gets further refined into exactly $p_l$ parts.
\end{itemize}
\end{definition} 

An index at level $l$ is a tuple $i=\left(i_1, i_2, \ldots, i_l\right) \in\left[p_1\right] \times\left[p_2\right] \times \cdots \times\left[p_l\right]$, which points to the symmetric subset $P_i=P_{i_1, \ldots, i_l} \in \mathcal{P}_l$ at level $l$, where $P_i$ is the $i_l$-th part in the refinement of the part $P_{i_1, \ldots, i_{l-1}}$ at level $l-1$, whenever $l \geq 2$ (all partitions are ordered).


Font convention. $\mathscr{P}$ is a branching partition, $\mathcal{P}$ is a partition, and $P$ is a subset of $[0,1]^{r[k]}$.

As already remarked the concept of branching partition generalizes the concept of hypergraphon as the following example shows.

\begin{example}\label{ExBranParHyper} A symmetric subset $P \subseteq[0,1]^{r[k]}$ or a $k$-uniform hypergraphon $W$ can be thought of as a degree $(2,1,1,1, \ldots)$ branching partition: level 1 is $P$ and $P^c$ (the complement of $P$ in $[0,1]^{r[k]}$ ) and all subsequent levels are trivial refinements.
\end{example}

We can generalize the notion of regularity from sequences of hypergraphons to branching partitions as follows

\begin{definition}
 Let $\boldsymbol{P}$ be a branching partition of $[0,1]^{r[k]}$ and $\mathscr{Q}$ a branching partition of $[0,1]^{r[k-1]}$. We say that $(\mathscr{P}, \mathscr{Q})$ is weakly $\left(\varepsilon_1, \varepsilon_2, \ldots\right)$-regular if for every $s \geq 1$, whenever $P \subseteq[0,1]^{r[k]}$ is a member of $\boldsymbol{P}$ of level at most s, and $\mathcal{Q}_s$ is the level s partition of $[0,1]^{r[k-1]}$ in $\mathscr{Q}$, the pair $\left(P, \mathcal{Q}_s\right)$ is weakly $\varepsilon_s$-regular.   
\end{definition}

and an analogous regularity lemma  for branching partitions follows simply from the regularity Lemma \ref{RegLemmaHyp} for sequences of hypergraphons.

\begin{lemma}[Weak regularity lemma for branching partitions, Lemma 6.4 in \cite{HypergraphonsZhao}] For every $k \geq 2, p=$ $\left(p_1, p_2, \ldots\right) \in \mathbb{N}^{\mathbb{N}}$ and $\varepsilon=\left(\varepsilon_1, \varepsilon_2, \ldots\right) \in \mathbb{R}_{>0}^{\mathbb{N}}$, we can find a $q=\left(q_1, q_2, \ldots\right) \in \mathbb{N}^{\mathbb{N}}$ so that the following holds: for every degree $p$ branching partition $\boldsymbol{P}$ of $[0,1]^{r[k]}$, there exists a degree $q$ branching partition $\mathscr{Q}$ of $[0,1]^{r[k-1]}$ so that $(\mathscr{P}, \mathscr{Q})$ is weakly $\varepsilon$-regular.
\end{lemma}

Now we introduce two notions of convergence for branching partitions. The first notion, called left-convergence, is based on the convergence of homomorphism densities. The second notion, called partitionable convergence, is based on the convergence of regularity partitions. We will show, using our counting lemmas, that partitionable convergence implies left-convergence.

Notation: Given degree $p=\left(p_1, p_2, \ldots\right)$ branching partitions $\mathscr{P}_1, \mathscr{P}_2, \ldots$ and $\tilde{\mathscr{P}}$ of $[0,1]^{r[k]}$ and degree $q=\left(q_1, q_2, \ldots\right)$ branching partitions $\mathscr{Q}_1, \mathscr{Q}_2, \ldots$ and $\tilde{\mathscr{Q}}$ of $[0,1]^{r[k-1]}$, we use the following notation to refer to the partitions and parts in these branching partitions.
\begin{itemize}
\item For each $l \geq 1, \mathcal{P}_{n, l}$ is the level $l$ partition in $\mathscr{P}_n$, and $\tilde{\mathcal{P}}_l$ is the level $l$ partition in $\tilde{\mathscr{P}}$.
\item For each $s \geq 1, \mathcal{Q}_{n, s}$ is the level $s$ partition in $\mathscr{Q}_n$, and $\tilde{\mathcal{Q}}_s$ is the level $s$ partition in $\tilde{\mathscr{Q}}$.
\item For each index $i=\left(i_1, i_2, \ldots, i_l\right) \in\left[p_1\right] \times \cdots \times\left[p_l\right], P_{n, i}$ is the index $i$ element of $\mathscr{P}_n$ and $\tilde{P}_i$ is the index $i$ element of $\tilde{\mathscr{P}}$.
\end{itemize}

\begin{definition}
[Left-convergence: $\mathscr{P}_n \rightarrow \tilde{\mathscr{P}}$] We say that a sequence $\mathscr{P}_1, \mathscr{P}_2, \ldots$ of degree $p$ branching partitions of $[0,1]^{r[k]}$ left-converges to another degree p branching partition $\tilde{\mathscr{P}}$ of $[0,1]^{[k]}$, written $\mathscr{P}_n \rightarrow \tilde{\mathscr{P}}$, if
$$
\lim _{n \rightarrow \infty} t_\alpha\left(F, \mathcal{P}_{n, l}\right)=t_\alpha\left(F, \tilde{\mathcal{P}}_l\right) \quad \text { for all } F, l, \alpha
$$
where $F$ ranges over all $k$-uniform hypergraphs, $l$ ranges over all positive integers, and $\alpha$ ranges over all maps $F \rightarrow\left[p_1 \cdots p_l\right]$. Recall from \eqref{eqn:homdensHyp} that $t_\alpha(F, \mathcal{P}):=t_\alpha\left(F, \mathbf{W}^{\mathcal{P}}\right)$ for a partition $\mathcal{P}$.
\end{definition}

For $k=2$, this is equivalent to left-convergence for  graphons \cite{LovaszGraphLimits}, where graphons are considered branching partitions as in Example \ref{ExBranParHyper}. We continue with the second definition:

\begin{definition}[Partitionable convergence: $\mathscr{P}_n-\rightarrow \rightarrow \tilde{\mathscr{P}}$ ]
We say that a sequence $\mathscr{P}_1, \mathscr{P}_2, \ldots$ of degree $p=\left(p_1, p_2, \ldots\right)$ branching partitions of $[0,1]^{r[k]}$ partitionably converges to another degree $p$ branching partition $\tilde{\mathscr{P}}$ of $[0,1]^{r[k]}$, written $\mathscr{P}_n-\rightarrow \rightarrow \tilde{\mathscr{P}}$, if the following is satisfied (the definition is inductive on $k$).
When $k=1$, for every index $i=\left(i_1, \ldots, i_l\right) \in\left[p_1\right] \times \cdots \times\left[p_l\right]$, we have $\lim _{n \rightarrow \infty} \lambda\left(P_{n, i}\right)=$ $\lambda\left(\tilde{P}_i\right)$, where $\lambda$ is the Lebesgue measure on $[0,1]$.

When $k \geq 2$, there exists some $q \in \mathbb{N}^{\mathbb{N}}$ and degree $q$ branching partitions $\mathscr{Q}_1, \mathscr{Q}_2, \ldots$  and $\tilde{\mathscr{Q}}$ of $[0,1]^{r k-1]}$ satisfying:\newline
a. $\left(\mathscr{P}_n, \mathscr{Q}_n\right)$ is weakly $(1,1 / 2,1 / 3, \ldots)$-regular for every $n$;\newline
b. $\mathscr{Q}_n--\rightarrow \tilde{\mathscr{Q}}$ as $n \rightarrow \infty$ (defined inductively); \newline
c. For every $s \geq 1$ and every index $i \in\left[p_1\right] \times \cdots \times\left[p_l\right]$, one has $\lim _{n \rightarrow \infty} d_1\left(P_{n, i} / \mathcal{Q}_s, \tilde{P}_i / \tilde{\mathcal{Q}}_s\right)=0$ \newline
d. For every member $\tilde{P} \subseteq[0,1]^{r[k]}$ of $\tilde{\mathscr{P}}$, one has $\left(W^{\tilde{P}}\right)_{\mathcal{Q}_s} \rightarrow W^{\bar{P}}$ pointwise almost everywhere as $s \rightarrow \infty$.
\end{definition}

Also for this type of convergence, we have compactness-type results:
 \begin{lemma}[Proposition 6.8 in\cite{HypergraphonsZhao}]
 Let $p\in \N^{\N}$. Let $\mathscr{P}_1, \mathscr{P}_2, \ldots$ be a sequence of degree p branching partitions of $[0,1]^{r[k]}$. Then there exists another degree $p$ branching partition $\tilde{\mathscr{P}}$ of $[0,1]^{r[k]}$ so that $\mathscr{P}_n--\rightarrow \tilde{\mathscr{P}}$ as $n\rightarrow \infty$ along some infinite subsequence.
 \end{lemma}
 
Moreover, the following result relates left-convergence and partitionable convergence.

\begin{lemma}[Partitionable convergence implies left-convergence, Proposition 6.7 in \cite{HypergraphonsZhao}]\label{PropImplPartLeftConv} If $\mathscr{P}_n--\rightarrow \tilde{\mathscr{P}}$ then $\mathscr{P}_n \rightarrow \tilde{\mathscr{P}}$. 
 \end{lemma}

 We conclude this work with the following conjecture:
 
\begin{conjecture}
Left-convergence, partitionable convergence and action convergence of the $(r-1)-$action of the sequence of normalized adjacency tensors
\begin{equation*}
    \frac{A(H_n)}{|V(H_n)|}
\end{equation*}
are equivalent for a sequence of $r-$uniform hypergraphs $H_n=(V(H_n),E(H_n))$.
\end{conjecture}

Sufficiency of left-convergence for partitionable convergence, and therefore equivalence between the two notions of convergence by Lemma \ref{PropImplPartLeftConv}, has already been conjectured in \cite{HypergraphonsZhao}. We pointed out many similarities between action convergence and partitionable convergence, in particular, from our discussion, it seems very natural that action convergence implies quotient convergence, and therefore by Lemma \ref{PropImplPartLeftConv} implies also left-convergence. 
%\section{Examples}\label{SecExamples}
%[to do]
%\section{Conclusions and Outlook}\label{SecConcOutlook}
%[to do]


\section*{Appendix (technical lemmas)}\label{SecAppendixTech}

For completeness we collect here a series of lemmas proven in  \cite{backhausz2018action} that we used extensevely through our work.

We start with an upper-bound on the Lévy–Prokhorov distance of the distribution of two random variables

\begin{lemma}[Lemma 13.1 in \cite{backhausz2018action}]\label{coupdist} Let $X,Y$ be two jointly distributed $\mathbb{R}^k$-valued random variables. Then $$d_{\rm LP}(\mathcal{L}(X),\mathcal{L}(Y))\leq \tau(X-Y)^{1/2}k^{3/4},$$

where $\tau$ is defined as in \eqref{eqn:tau}.
\end{lemma}

and its direct consequence

\begin{lemma}[Lemma 13.2 in \cite{backhausz2018action}\label{coupdist2}] Let $v_1,v_2,\dots,v_k$ and $w_1,w_2,\dots,w_k$ be in $L^1(\Omega)$ for some probability space $\Omega$. Let $m:=\max_{i\in [k]} \|v_i-w_i\|_1$. Then
$$d_{\rm LP}(\mathcal{L}(v_1,v_2,\dots,v_k),\mathcal{L}(w_1,w_2,\dots,w_k))\leq m^{1/2}k^{3/4}.$$
\end{lemma} 

%For a real number $z\in\mathbb{R}^+$ let $f_z:\mathbb{R}\to\mathbb{R}$ denote the function such that $f_z(x)=0$ for $|x|\geq 2z$, $f_z(x)=|x+z|-z$ for $x\in [-2z,0]$ and $f_z(x)=-|x-z|+z$ for $x\in [0,2z]$.

%\begin{lemma}(Lemma 13.3\cite{backhausz2018action})\label{closedlem1} Let $q\in (1,\infty)$ and let $X$ be a real-valued random variable with $\mathbb{E}(|X|^q)=c<\infty$. Then for $z\in\mathbb{R}^+$ we have that $\mathbb{E} |f_z(X)-X|\leq cz^{1-q}$.
%\end{lemma}

The next lemma is a general probabilistic result about limits of random variables, products and expectations.

\begin{lemma}[Lemma 13.4 in \cite{backhausz2018action}]\label{closedlem2} Let $q\in (1,\infty)$.  Let $\{(X_i,Y_i)\}_{i=1}^\infty$ be a sequence of pairs of jointly distributed real-valued random variables such that $X_i\in [-1,1]$ and $\mathbb{E}(|Y_i|^q)\leq c<\infty$ for some $c\in\mathbb{R}^+$. Assume that the distributions of $(X_i,Y_i)$ weakly converge to some probability distribution $(X,Y)$ as $i$ goes to infinity. Then $\mathbb{E}(|Y|^q)\leq c$ and $$\lim_{i\to\infty} \mathbb{E}(X_iY_i)=\mathbb{E}(XY).$$ 
\end{lemma}



%\begin{lemma}(Lemma 13.5 in \cite{backhausz2018action}\label{applem1}) Let $\mu$ be a probability measure on $[-c,c]$ for some $c\in\mathbb{R}^+$. Let $p\in [1,\infty)$. Then $\int_{\mathbb{R}} |x|^p~d\mu\leq (2d_{\rm LP}(\mu,\delta_{0}))^p+2d_{\rm LP}(\mu,\delta_{0})c^p$
%\end{lemma}
We give a last technical upper bound for the Lévy–Prokhorov distance of measures generated by a $P-$operator through specific random variables.

\begin{lemma}[Lemma 13.6 in \cite{backhausz2018action}]\label{applem2} Let $p\in [1,\infty)$ and let $A\in\mathcal{B}(\Omega)$ be a $P$-operator. Let  $v_i$ and $w_i$ be elements in $L^\infty(\Omega)$ with values in $[-1,1]$ for $i\in [k]$. Then we have
$$d_{\rm LP}(\mathcal{D}_A(\{v_i\}_{i=1}^k),\mathcal{D}_A(\{w_i\}_{i=1}^k))\leq m^{1/2}((2d)^p+2^{p+1}d)^{1/(2p)}(2k)^{3/4},$$ where $m=\max\{1,\|A\|_{p\to 1}\}$ and $d=\max_{i\in [k]}\{d_{\rm LP}(\mathcal{D}(v_i-w_i),\delta_0)\}$.
\end{lemma}






\textbf{Acknowledgements:} The author thanks Ágnes Backhausz, Tobias B\"ohle, Christian K\"uhn, Raffaella Mulas, Florentin M\"unch, Balázs Szegedy and Chuang Xu for useful discussions.
\medskip


\section*{References}

\bibliographystyle{plain}
\bibliography{biblio}


\end{document}
