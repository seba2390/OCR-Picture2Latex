\section{Introduction}\label{sec:intro}

%

Regular and non-invasive measurement of vital physiological attributes such as pulse rate (PR),  pulse rate variability (PRV), and blood pressure (BP) are important due to their fundamental role in tracking one's fitness level, diagnosis of cardiovascular diseases, and monitoring of well-being. In-office and home environments, passive non-contact measurements are essential to monitor warning signs for cardiovascular diseases, stress, and anxiety. This paper explores the use of facial features from the videos to predict these vital health attributes.


%
%
	
%

%
%

%
%




%
Currently, the gold standard techniques for measuring such vital health attributes include using intrusive contact devices such as an electrocardiogram (ECG), chest straps, and pulse oximeters. Traditionally, ECG was extensively used for such measurement, but the recent trend has been shifted towards using pulse oximeters because of its low cost. 

Although pulse oximeters are easy to use, they have limitations for frequent measurements. First, it requires the purchase of equipment and needs either the health provider or the user to manually perform the measurements. Second, the device needs to be carried to the different places that the user goes, limiting its use. Third, the finger clip-on and earlobe clip-on may not always fit well on every individual due to the varying size of fingers and earlobes. The improper fitting of the device may lead to estimation errors \cite{haynes2007ear}. Fourth, using clip-on may be potentially uncomfortable during long use. Thus, this paper considers the use of a non-contact based approach where the passive video of the face can be used to estimate the health metrics. 
%


Monitoring of health parameters using non-contact methods like videos from commercially-available camera has been recently considered in \cite{verkruysse2008remote,poh2011advancements,sun2011motion,kumar2015distanceppg}, showing that the photoplethysmogram (PPG) signal can be extracted from the videos of the face.These techniques required no dedicated source of light, and a  low-cost digital camera can be used. The non-contact measurement using camera video has many applications including determination of health parameters of people working in an office environment, shop-floor, newborn infants in the hospital where using contact probes may not be possible. The non-contact method can also replace current contact methods deployed on treadmills for measurement of pulse rate.  In these works, the PPG signal is extracted from each individual, and thus the coefficients of the video features that provide the PPG signal are dependent on the individual. In contrast, we do not consider individual characteristics in the prediction. The proposed method can thus help predict health metrics of an individual for which no training sample has been collected in the past, making our methodology robust. Also, the use of non-contact methodology has an additional advantage of being scalable, and portable since cameras are ubiquitous. 
%

The proposed approach has two key steps. The first step considers capturing the video and extracting the face from the video. The features corresponding to the face in each frame are obtained. The second step includes training a  neural network to learn the health parameters from the above obtained features. Our results obtained from the deep learning model has a mean absolute percentage error of 4.6\% for predicting pulse rate. The appendix contains initial results on predicting indicators of the variance in pulse rate.

%

%

%

%


%
%



	




