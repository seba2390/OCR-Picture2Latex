\section{Data Collection}\label{sec:data}
We designed our own experiment to collect the data for training the model. Twenty healthy volunteers participated in this study. The participants were recruited from a university population through email including a description of the study. This study was reviewed by the university's Institutional Review Board, and all participants provided informed consent. The details of our experiment are given below:
\subsection{Set-Up}
To predict the vital health metrics, we used the face video of the person. The video is obtained using a 5 MP front-facing Hello face-authentication camera (1080p HD) from Microsoft Surface Book, having 30 frames per second.  %
The camera is capable of capturing red, green and blue color channels. The authors of \cite{verkruysse2008remote} suggested that green channel of the video outperforms blue and red channel in estimating the health parameters. Therefore, we also utilize the green channel of the camera. The features obtained from the video will be used to predict the health metrics. 

To train the data, true values of PR is calculated using a contact measurement device, Shimmer3 GSR+, that records the ground truth PPG signal. We chose earlobe as the suitable position for recording PPG since it is close to the face. We care about this proximity since face is used for our video recordings as well. Subjects were asked to sit still for a 50s video, facing towards the camera at a distance of approximately 0.5m, and the PPG signal was recorded simultaneously through the Shimmer3 GSR+ device. The experimental set-up for conducting the experiments is shown in Figure \ref{fig:experiment} in appendix.  

\begin{figure*}[htbp]
	\centering
	\includegraphics[width=\textwidth]{Process}
	
	\caption{The steps followed for feature extraction from each frame of the video. (a) The actual image (one of the many frame) from the video captured during the experiment. (b) The detected and aligned face using DeepFace along with landmark points. (c) The face is cropped using landmark points to get only required face features. (d) Each frame is downsampled to 20x20 image}
	%
		\vspace{-.2in}
	
	\label{fig:process}
\end{figure*}

\subsection{Experiments}
%
%

%

The study involved people of different skin colors and ethnicities. For each subject, the measurements were performed after different activity levels, thus providing a variation in the heart rate of each subject. The different activity levels at which the measurements were collected were:  Rest Position, Brisk Walk, and Exercise.
%


%

\textbf{Rest Position}: The first experiment was conducted when each participant was in rest condition. Each subject was asked to relax and sit in front of the camera. The video and Shimmer device recordings were collected simultaneously.


\textbf{Brisk Walk}: The next experiment involved data collection of the same participants after they were asked to do a brisk walk for 0.25 miles at a speed of 3-5 mph on the treadmill.  The video and shimmer recordings were captured immediately after the subject complete the brisk walk. 

\textbf{Exercise}:  The last experiment involved more challenging physical tasks. All the subjects were asked to perform as many push-ups or sit-ups as they can such that they exert themselves to their full capacity. This activity was designed to elicit a high pulse rate since the individual was working out at their full capacity. 


The mean pulse rate for rest, walk, and exercise conditions were 72.9, 79.6, and 98.5 respectively.  PPG and facial data were videos recorded immediately after each activity to minimize recovery effects on the physiological data. Each subject was given rest of 10 minutes before each activity so they can recover prior to the next activity.


%
%
%

%
We acknowledge that the heart rate will dynamically change during the video capture; however, the purpose of this study was to compare device-free sensing to the gold standard continuous measurement. The success of capturing these dynamic behaviors may further show the promising capability of the technique in addressing the complexity of dynamic changes commonly seen in the real world.