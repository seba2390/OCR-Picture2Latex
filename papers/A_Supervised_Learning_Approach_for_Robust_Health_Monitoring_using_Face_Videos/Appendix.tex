%
\section{Experimental Setup and Neural Network}

Figure \ref{fig:experiment} depicts the experimental setup. Further, Fig. \ref{fig:arch} depicts the used fully connected neural network architecture. 

\begin{figure}[htbp]
	\centering\
	\includegraphics[width=0.48\textwidth]{Experiment}
	\vspace{-.2in}
	\caption{Experimental set-up. The contact probe of the Shimmer3 GSR+ is attached to the earlobe and laptop camera is placed around 0.5 m away from the subject.}
	\label{fig:experiment}
		\vspace{-.1in}
\end{figure}

\begin{figure}[htbp]
	\centering
	\includegraphics[width=0.48\textwidth]{Architecture}
		\vspace{-.2in}
	\caption{The architecture of a fully connected neural network with three hidden layers}
		\vspace{-.1in}
	\label{fig:arch}
\end{figure} 

\begin{figure}[htbp]
	\centering
	\includegraphics[trim=.2in .5in .2in .5in, clip, width=.48\textwidth]{LF_scatter.eps}
	\vspace{-.2in}
	\caption{Scatter plot of the predicted LF value vs. the ground truth LF value.}
	\label{fig:scatter_LF}
	\vspace{-.1in}
\end{figure}
\section{Pulse Rate Variability}\label{sec:appendix}


%


\begin{figure}[htbp]
	\centering
	\includegraphics[width=0.48\textwidth]{Ivan_Loss2}
	\vspace{-.1in}
	\caption{Training loss and test loss for predicting Low Frequency (LF) component of PRV. }
	\label{fig:LF}
	\vspace{-.1in}
\end{figure}



%

%

%


%
%
%
%
%
%
%
%
%
%
%
%
%
%
%




\begin{figure}[htbp]
	\centering
	\includegraphics[width=.485\textwidth]{Mayank_Loss2.eps}
	\vspace{-.1in}
	\caption{Training loss and test loss for predicting High Frequency (HF) component of PRV.}
	\label{fig:HF}
	\vspace{-.1in}
\end{figure}

\begin{figure}
	\centering
	\includegraphics[trim=.2in .5in .2in .5in, clip, width=.48\textwidth]{HF_scatter.eps}
	\vspace{-.1in}
	\caption{Scatter plot of the predicted HF value vs. the ground truth HF value.}
	\label{fig:scatter_HF}
	\vspace{-.1in}
\end{figure}



Pulse  rate  variability  is  the  variation  in  the  time  interval between two expansions of the artery. It is usually measured by the variation in beat-to-beat interval. This metric is considered as a non-invasive technique for measuring autonomic nervous system  (ANS)  activity.  The  autonomic  nervous  system has  two  branches;  sympathetic  nervous  system  (SNS)  and parasympathetic  nervous  system  (PNS)  and  is  regulated  by hypothalamus.  Its  function  includes  control  of  respiration, cardiac regulation, vasometer activity and certain reflex actions like  coughing,  sneezing,  swallowing,  and  vomiting.  High-frequency  (HF)  component  of  PRV  is  affected  by  efferent vagal  (parasympathetic)  activity  and  it  decreases  during  the conditions  of  acute  time  pressure,  emotional  strain,  mental stress,  and  elevated  anxiety.  The  low-frequency (LF) component of PRV is known to contain both sympathetic and  vagal  influences.  Thus,  frequent  and  accurate  measurement  of  PR  and  PRV  can  provide  critical  signs  of  one's well being and any abnormality could lead to potential health problems.

In this study, the LF and HF components of PRV were roughly estimated by computing the area under the PSD curve between specific frequency range. For LF, the frequency range is 0.04-0.15Hz and for HF, it is 0.15-0.4Hz. We used these rough estimates of HF, and LF as a response variable to train our model.   We trained our model using High-Frequency component and Low-Frequency component of the PPG signal. We designed separate models to train our model to predict  the HF component and the LF component. 

 We  tested our model using leave-one-out cross validation method similar to the pulse rate. Our model makes predictions on the user who has not been seen before in the training data. We choose the  number of iterations to run our model as 200. The test and training losses with iterations for the LF and the HF component of the PPG signal are depicted in  Fig. \ref{fig:LF} and Fig. \ref{fig:HF}, respectively.  

For predicting LF component, mean absolute percentage error on test set is 4.58\%, and  the root mean squared error is 3.49. On the other hand, the MAPE for  HF component is found to be 10.2\%,  and the RMSE is 4.96 for test data.  The mean RMSE for our model is 4.3 whereas the mean RMSE taken over different skin colored people in \cite{kumar2015distanceppg} is 25.3 thus providing 83\% decrease in the RMSE. Figures \ref{fig:scatter_LF} and \ref{fig:scatter_HF} depict the comparison between the actual values and the predicted values for the two components of the PPG signal, respectively. We note that dataset in \cite{kumar2015distanceppg} \& \cite{osman2015supervised} is different from the dataset used in this paper, thus not providing comparison on the same dataset. However, since the code or data of the prior work is not public, the comparison is made on the aggregate prediction accuracy.  





%



