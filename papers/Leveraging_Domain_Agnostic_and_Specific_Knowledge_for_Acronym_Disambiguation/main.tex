\def\year{2021}\relax
%File: formatting-instructions-latex-2021.tex
%release 2021.2
\documentclass[letterpaper]{article} % DO NOT CHANGE THIS
\usepackage{aaai21}  % DO NOT CHANGE THIS
\usepackage{times}  % DO NOT CHANGE THIS
\usepackage{helvet} % DO NOT CHANGE THIS
\usepackage{courier}  % DO NOT CHANGE THIS
\usepackage[hyphens]{url}  % DO NOT CHANGE THIS
%\usepackage{hyperref}
\usepackage{bm}
\usepackage{multirow}
%\usepackage{footnote}
%\usepackage{tablefootnote}
\usepackage{threeparttable}
\usepackage{graphicx} % DO NOT CHANGE THIS
\urlstyle{rm} % DO NOT CHANGE THIS
\def\UrlFont{\rm}  % DO NOT CHANGE THIS
\usepackage{natbib}  % DO NOT CHANGE THIS AND DO NOT ADD ANY OPTIONS TO IT
\usepackage{caption} % DO NOT CHANGE THIS AND DO NOT ADD ANY OPTIONS TO IT
\usepackage{color}
\newcommand{\reminder}[1]{ (((\mbox{$\Longleftarrow \star$}{\textbf{#1}} )))}
\newcommand{\redfont}[1]{ {\color{red}{#1}}}
\newcommand{\bluefont}[1]{ {\color{blue}{#1}}}
%%%%%%%%%%%%%%%
%%Remove Conference Information from ACM 2017 SIGCONF Template
%%yunwei.zqw
%\settopmatter{printacmref=false} % Removes citation information below abstract
%\renewcommand\footnotetextcopyrightpermission[1]{} % removes footnote with conference information in first column
%\pagestyle{empty} % removes running headers
%%%%%%%%%%%%%%%

\frenchspacing  % DO NOT CHANGE THIS
\setlength{\pdfpagewidth}{8.5in}  % DO NOT CHANGE THIS
\setlength{\pdfpageheight}{11in}  % DO NOT CHANGE THIS
%\nocopyright
%PDF Info Is REQUIRED.
% For /Author, add all authors within the parentheses, separated by commas. No accents or commands.
% For /Title, add Title in Mixed Case. No accents or commands. Retain the parentheses.
\pdfinfo{
/Title (AAAI Press Formatting Instructions for Authors Using LaTeX -- A Guide)
/Author (AAAI Press Staff, Pater Patel Schneider, Sunil Issar, J. Scott Penberthy, George Ferguson, Hans Guesgen, Francisco Cruz, Marc Pujol-Gonzalez)
/TemplateVersion (2021.2)
} %Leave this
% /Title ()
% Put your actual complete title (no codes, scripts, shortcuts, or LaTeX commands) within the parentheses in mixed case
% Leave the space between \Title and the beginning parenthesis alone
% /Author ()
% Put your actual complete list of authors (no codes, scripts, shortcuts, or LaTeX commands) within the parentheses in mixed case.
% Each author should be only by a comma. If the name contains accents, remove them. If there are any LaTeX commands,
% remove them.

% DISALLOWED PACKAGES
% \usepackage{authblk} -- This package is specifically forbidden
% \usepackage{balance} -- This package is specifically forbidden
% \usepackage{color (if used in text)
% \usepackage{CJK} -- This package is specifically forbidden
% \usepackage{float} -- This package is specifically forbidden
% \usepackage{flushend} -- This package is specifically forbidden
% \usepackage{fontenc} -- This package is specifically forbidden
% \usepackage{fullpage} -- This package is specifically forbidden
% \usepackage{geometry} -- This package is specifically forbidden
% \usepackage{grffile} -- This package is specifically forbidden
% \usepackage{hyperref} -- This package is specifically forbidden
% \usepackage{navigator} -- This package is specifically forbidden
% (or any other package that embeds links such as navigator or hyperref)
% \indentfirst} -- This package is specifically forbidden
% \layout} -- This package is specifically forbidden
% \multicol} -- This package is specifically forbidden
% \nameref} -- This package is specifically forbidden
% \usepackage{savetrees} -- This package is specifically forbidden
% \usepackage{setspace} -- This package is specifically forbidden
% \usepackage{stfloats} -- This package is specifically forbidden
% \usepackage{tabu} -- This package is specifically forbidden
% \usepackage{titlesec} -- This package is specifically forbidden
% \usepackage{tocbibind} -- This package is specifically forbidden
% \usepackage{ulem} -- This package is specifically forbidden
% \usepackage{wrapfig} -- This package is specifically forbidden
% DISALLOWED COMMANDS
% \nocopyright -- Your paper will not be published if you use this command
% \addtolength -- This command may not be used
% \balance -- This command may not be used
% \baselinestretch -- Your paper will not be published if you use this command
% \clearpage -- No page breaks of any kind may be used for the final version of your paper
% \columnsep -- This command may not be used
% \newpage -- No page breaks of any kind may be used for the final version of your paper
% \pagebreak -- No page breaks of any kind may be used for the final version of your paperr
% \pagestyle -- This command may not be used
% \tiny -- This is not an acceptable font size.
% \vspace{- -- No negative value may be used in proximity of a caption, figure, table, section, subsection, subsubsection, or reference
% \vskip{- -- No negative value may be used to alter spacing above or below a caption, figure, table, section, subsection, subsubsection, or reference

\setcounter{secnumdepth}{0} %May be changed to 1 or 2 if section numbers are desired.

% The file aaai21.sty is the style file for AAAI Press
% proceedings, working notes, and technical reports.
%

% Title

% Your title must be in mixed case, not sentence case.
% That means all verbs (including short verbs like be, is, using,and go),
% nouns, adverbs, adjectives should be capitalized, including both words in hyphenated terms, while
% articles, conjunctions, and prepositions are lower case unless they
% directly follow a colon or long dash

\title{
Leveraging Domain Agnostic and Specific Knowledge for Acronym Disambiguation
%Dual-path Hierarchical Bert on Ambiguity Elimination in Scientific Domain
}
\author{
    %Authors
    % All authors must be in the same font size and format.
    Qiwei Zhong, Guanxiong Zeng, Danqing Zhu, Yang Zhang, Wangli Lin, Ben Chen, Jiayu Tang\\
}
\affiliations{
    %Afiliations
    Alibaba Group, Hangzhou, China\\
    %If you have multiple authors and multiple affiliations
    % use superscripts in text and roman font to identify them.
    %For example,

    % Sunil Issar, \textsuperscript{\rm 2}
    % J. Scott Penberthy, \textsuperscript{\rm 3}
    % George Ferguson,\textsuperscript{\rm 4}
    % Hans Guesgen, \textsuperscript{\rm 5}.
    % Note that the comma should be placed BEFORE the superscript for optimum readability
    % email address must be in roman text type, not monospace or sans serif
    \{yunwei.zqw, moshi.zgx, danqing.zdq, zy142206, wangli.lwl, chenben.cb, jiayu.tangjy\}@alibaba-inc.com

    % See more examples next
}
\begin{document}

\maketitle
\begin{abstract}
An obstacle to scientific document understanding is the extensive use of acronyms which are shortened forms of long technical phrases. Acronym disambiguation aims to find the correct meaning of an ambiguous acronym in a given text. Recent efforts attempted to incorporate word embeddings and deep learning architectures, and achieved significant effects in this task.
In general domains, kinds of fine-grained pretrained language models
have sprung up, thanks to the large-scale corpora which can usually be obtained through crowd-sourcing.
However, these models based on domain agnostic knowledge might achieve insufficient performance when directly applied to the scientific domain. Moreover, obtaining large-scale high-quality annotated data and representing high-level semantics in the scientific domain is challenging and expensive.
In this paper, we consider both the domain agnostic and specific knowledge, and propose a \underline{H}ierarchical \underline{D}ual-path \underline{BERT} method coined \textbf{hdBERT} to capture the general fine-grained and high-level specific representations for acronym disambiguation. First, the context-based pretrained models, RoBERTa and SciBERT, are elaborately involved in encoding these two kinds of knowledge respectively. Second, multiple layer perceptron is devised to integrate the dual-path representations simultaneously and outputs the prediction. With a widely adopted SciAD dataset contained 62,441 sentences, we investigate the effectiveness of hdBERT. The experimental results exhibit that the proposed approach outperforms state-of-the-art methods among various evaluation metrics. Specifically, its macro F1 achieves 93.73\%.



%As acronyms  might  be  defined  either  locally  in  the  same document or globally in an external dictionary with multiple meanings, a successful document understanding model should  both capture local  definitions  and disambiguate acronyms which are not defined in documents. 

%Acronyms are short forms of phrases that help to convey lengthy sentences in documents.
%And as one of the subjects of writing. Because of their importance, please identify abbreviations and It is very important to understand the text to find the correct meaning of each phrase and find the initial abbreviation (namely, the ambiguity elimination of the initial abbreviation (AD)). 
%In this article, our task is to find out the correct meaning of the ambiguous acronym in a given sentence. That is to say, given a sentence with an ambiguous acronym and a dictionary with a possible extension (I .e., long form) of the acronym, we need to determine the true meaning of the acronym as correctly as possible. 
%In the final online evaluation in the Acronym Disambiguation task of AAAI-21 scientific document understanding competition~\footnote{\url{https://competitions.codalab.org/competitions/26611}}, the \textbf{D}ual-\textbf{p}ath \textbf{H}ierarchical \textbf{BERT}~(DPHBERT) we proposed achieved excellent results. 
%For this task, we are provided with the training and development datasets consisting of 62,441 sentences and a dictionary of 732 ambiguous acronyms. 
%The dataset and dictionary are created from 6,786 English scientific papers published at arXiv. 
%With the a series of experiments, we investigate the effectiveness of the DPHBERT, which can extract the meaning of the target abbreviation from the word level to the byte level, and finally provide a more appropriate explanation on the AD task.
\end{abstract}

\section{Introduction}

Scientific literature is most commonly available in the form of PDFs, which pose challenges for accessibility \citep{NielsenPDFStillUnfit, Bigham2016AnUT}. When researchers, students, and other individuals who are blind or low vision (BLV) interact with scientific PDFs through screen readers, the availability of document structure tags, labeled reading order, labeled headers, and image alt-text are necessary to facilitate these interactions. However, these features must be painstakingly added by authors using proprietary software tools, and as a result, are often missing from papers. Low vision or dyslexic readers who interact with PDFs through screen magnification or text-to-speech may also find the complexity of certain academic paper PDF formats challenging, e.g., non-linear layout can interrupt the flow of text in a magnifying tool. Inaccessible paper PDFs can lead to high cognitive overload, frustration, and abandonment of reading for BLV readers. 

Unfortunately, we find that the majority of scientific PDFs lack basic accessibility features. We estimate based on a sample of \numpdfs PDFs from multiple fields of study that only around \percaccessible of paper PDFs released in the last decade satisfy all of the aforementioned accessibility requirements. 
Accessibility challenges for academic PDFs are largely due to three factors: (1) the complexity of the PDF file format, which make it less amenable to certain accessibility features, (2) the dearth of tools, especially non-proprietary tools, for creating accessible PDFs, and (3) the dependency on volunteerism from the community with minimal support or enforcement \citep{Bigham2016AnUT}. The intent of the PDF file format is to support faithful visual representation of a document for printing, a goal that is inherently divergent from that of document representation for the purposes of accessibility. Though some professional organizations like the Association for Computing Machinery (ACM) have encouraged PDF accessibility through standards and writing guidelines,\footnote{\href{https://www.acm.org/publications/authors/submissions}{https://www.acm.org/publications/authors/submissions}} uptake among academic publishers and disciplines more broadly has been limited. 

While policy changes help, the fact remains that most academic PDFs produced today, and historically, are inaccessible, yet remain as the dominant way to read those papers. A long-range solution will necessitate buy-in from multiple stakeholders---publishers, authors, readers, technologists, granting agencies, and the like. But in the interim, there are technological solutions that can be offered as a sort of ``band-aid'' to the problem. We use this paper to offer an in-depth qualitative and quantitative description of the problem as it stands, and to introduce one such technological solution: the \scially system that automatically extracts semantic information from paper PDFs and re-renders this content in the form of an accessible HTML document. Though the process is imperfect and can introduce errors, we demonstrate the ability of the rendered HTMLs to reduce cognitive load and facilitate in-paper navigation and interactions for BLV users. 

The goals and contributions of this paper are three-fold:

\begin{enumerate}
    \item We characterize the state of academic-paper PDF accessibility by estimating the degree of adherence to accessibility criteria for papers published in the last decade (2010--2019), and describe correlations between year, field of study, PDF typesetting software, and PDF accessibility.
    \item We propose an automated approach for extracting the content of academic PDFs and displaying this content in a more accessible HTML document format. We build a prototype that re-renders 12 million PDFs in HTML, and describe the design decisions, features, and quality of the renders (assessed as faithfulness to the source PDF). We perform expert grading of the rendered HTML and report an error analysis. A demo of our system is available at \href{https://scia11y.org/}{scia11y.org}, which makes available 1.5M HTML renders of open access PDFs.
    \item We conduct an exploratory user study with \numusers BLV scholars to better understand the challenges they experience when reading academic papers and how our proposed tool might augment their current workflow. During the study, we ask users to interact with the prototype and offer feedback for its improvement. We perform open coding of interviews to identify existing reading challenges, coping mechanisms, as well as positive and negative responses to prototype features. We summarize the findings of this user study into a set of design recommendations.
\end{enumerate}

Our analysis reveals that PDF accessibility adherence is low across all fields of study. Of the five accessibility criteria we assess, only \percaccessible of the PDFs we assess demonstrate full compliance. Though compliance for several criteria seems to be increasing over time, author awareness and contribution to accessibility remains low, as Alt-text has the lowest compliance of the five criteria at between 5--10\% (Alt-text is the only criterion of the five that \textit{requires} author intervention in all cases using current tools). We also find that typesetting software is strongly associated with accessibility compliance, with LaTeX and publishing software like Arbortext APP producing low compliance PDFs, while Microsoft Word is generally associated with higher compliance.


\begin{figure}[t!]
    \centering
    \includegraphics[width=\textwidth]{figures/pipeline.png}
    \caption{A schematic for creating the \scially HTML render from a paper PDF. Starting with the raw two-column PDF on the left, S2ORC \citep{lo-wang-2020-s2orc} is used to extract title, authors, abstract, section headers, body text, and references. S2ORC also identifies links between inline citations and references to figures and table objects. DeepFigures \citep{Siegel2018ExtractingSF} is used to extract figures and tables, along with their captions. The output of these two models are merged with metadata from the Semantic Scholar API. Heuristics are used to construct a table of contents, to insert figures and tables in the appropriate places in the text, and to repair broken URLs. We add HTML headers as illustrated (header tags for sections, paragraph tags for body text, and figure tags for figures and tables); highlighted components (table of contents and links in references) are not in the PDF and novel navigational features that we introduce to the HTML render. An example HTML render of parts of a paper document is show to the right (actual render is single column, which is split here for presentation).}
    \label{fig:pipeline}
    \Description{A schematic diagram showing the components of the SciA11y pipeline. An image of a paper PDF is on the left. Red boxes on the PDF image highlight the text components from the paper, with an arrow pointing to a box that says "S2ORC extracts: title, authors, abstract, section headers, body paragraphs, and references." A blue box on the PDF image highlights a figure, with an arrow pointing to a box that says "DeepFigures extracts: figures, figure captions, tables, and table titles/captions." A box below "S2ORC extracts" and "DeepFigures extracts" says "Additional content: metadata from Semantic Scholar API, table of contents, figures and tables inserted at first mention, and links between references and text." Arrows from all three boxes point into a larger box that describes the SciA11y prototype, where HTML tags are inserted around various blocks of text extracted from the PDF. On the right of all this is a screen capture of an example HTML render, showing how the semantic content from the PDF is represented as a single-column HTML page for easy reading.}
\end{figure}

To offset the reading challenges of inaccessible papers for BLV researchers, we propose and test the \scially system for rendering academic PDFs into accessible HTML documents. As shown in Figure~\ref{fig:pipeline}, our prototype integrates several machine learning text and vision models to extract the structure and semantic content of papers. The content is represented as an HTML document with headings and links for navigation, figures and tables, as well as other novel features to assist in document structure understanding. Our evaluation of the \scially system identifies common classes of extraction problems, and finds that though many papers exhibit some extraction errors, the majority (55\%) have no major problems that impact readability, and another 32\% have only some problems that impact readability.

Through our user study, we identify numerous challenges faced by BLV users when reading paper PDFs, including some that affect the whole document or limit navigation, and many that affect the ability of the reader to understand text or various elements of a paper like math content or tables. Responses to \scially were positive; participants especially liked navigation features such as headings, the table of contents, and bidirectional links between inline citations and references. Of the extraction errors in \scially, missed or incorrectly extracted headings were the most problematic, as these impact the user's ability to navigate between sections and fully trust the system. All users reported being likely to use the system in the future. When asked how the system might be integrated into their workflow, one participant replied ``I think it would become the workflow.'' Another participant said, ``for unaccessible PDFs, this is life-changing.'' We condense these findings into a set of recommendations for designing and engineering accessible reading systems (Section~\ref{sec:designrecs}). Most importantly, documents should be structured to match a reader's mental model, objects should be properly tagged, and care should be taken to reduce the reader's cognitive load and increase trust in the system. Features that emulate the external memory that visual layout provides to sighted users can be especially beneficial.

This paper is organized as follows. Following a description of related work in Section \ref{sec:related_work}, we first provide a meta-scientific analysis of the current state of academic PDF accessibility in Section \ref{sec:sos}. In Section \ref{sec:pdf2html}, we document our pipeline for converting PDF to HTML and describe the \scially prototype for rendering papers. An evaluation of HTML render quality and faithfulness is provided in Section \ref{sec:evaluation}. Section \ref{sec:user_study} describes our user study and findings. 
We recognize that no PDF extraction system is perfect, and many open research challenges remain in improving these systems. However, based on our findings, we believe \scially can dramatically improve screen reader navigation of most papers compared to PDFs, and is well-positioned to assist BLV researchers with many of their most common reading use cases. Our hope is that a system such as \scially can improve BLV researcher access to the content of academic papers, and that these design recommendations can be leveraged by others to create better, more faithful, and ultimately more usable tools and systems for scholars in the BLV community.

With ProgressLabeller, a user can scalably label new datasets with camera world pose, scene object poses and scene object segmentations. This process is enabled by fusing streaming RGB (or RGB-D) inputs into a single scene-wide representation, and then allowing a human user to input relevant 6-DoF information via 3D modelling interfaces (such as those provided by Blender \cite{blender}). This process demonstrates label stability even over long input video streams, and due to its functionality with direct RGB inputs, can label even difficult objects such as transparent cups.  We discuss below methods related to ProgressLabeller.

\subsection{Direct \& Human-in-the-loop labelling}
The creation of 2D segmentation data is analogous to the object detection, keypoint detection, or semantic segmentation tasks (depending on desired output labels). Tools such as LabelMe \cite{russell2008labelme} required users to directly interact with the underlying data to be labelled. This manual process was improved by model-assisted approaches such as Deep Extreme Cut \cite{maninis2018deep} which decreases the amount of user effort necessary to label images.
Shared autonomy and mixed-initiative methods have also been used in this approach, in which the user provides coarse pose or other estimations which are fine-tuned via a model-informed approach \cite{ye2021human}.

%%%%%%%%%%%%%%%%%%%%%%%%%%%%%%%
 
\subsection{End-to-End Labellers}
Previous tools have been created to enable this style of learning process. LabelFusion \cite{marion2018label} is perhaps the most commonly utilized example. LabelFusion utilizes streaming RGB-D inputs to create a dense reconstruction of the scene, which is then labelled semi-manually by aligning 3D object models to the 3D reconstruction. While this approach is typically robust, it relies on RGB-D input for reconstruction, and experiences difficulties under certain regimes. In particular, transparent objects cause problems for commonly employed depth sensor technologies, and long-running input streams typically result in large amounts of 'drift'. 

Some methods have been introduced to eliminate the need for CAD models in the labelling process. Singh et al. \cite{singh2021rapid} proposed a method which utilizes user labelled keypoints and bounding boxes to generate pose and segmentation labels. This frees the system from dependency on CAD models, but requires user interaction directly with the images. SALT \cite{stumpf2021salt} proposed utilizing GrabCut to generate 3D bounding boxes and image segmentation labels for relevant scenes. This allows removing the dependency on object masks while also allows the labelling of dynamic scenes such as human gait videos. 
% This method requires point cloud inputs however, which precludes the labelling of RGB only image streams.

Other works sought to improve the labelling procedure itself. EasyLabel \cite{suchi2019easylabel} allows for semi-automatic labelling of scenes via sequentially added objects.  This process is scalable, and generates high quality labels. However, it requires tight physical control over the scene to be labelled, which is not always feasible to obtain. Objectron \cite{ahmadyan2021objectron} utilized modern smartphone's AR capabilities combined with human-labelled 3D bounding boxes to scalably create a large scale dataset. This method however is susceptible to label drift during long-duration input videos. KeyPose \cite{liu2020keypose} specifically sought to generate labelled datasets for transparent objects. This method utilized stereoscopic images taken from a robot armature in order to avoid the problems of typical depth cameras have with transparent objects.
% StereOBJ-1M \cite{liu2021stereobj} improves the data collection efficiency in a setting with two more static cameras and more fiducial markers in the scene.
\section{Method}
Fig.~\ref{fig:framework} presents the illustration of the proposed \frameworkName.
In this section,  
we start by providing the problem definition of online CIL. Then, we describe the definition of the online prototype, the proposed online prototype equilibrium, and the proposed adaptive prototypical feedback. Finally, we propose an online prototype learning framework.

\subsection{Problem Definition}
Formally, online CIL considers a continuous sequence of tasks from a single-pass data stream $\mathfrak{D}=\left\{\mathcal{D}_1, \ldots, \mathcal{D}_T \right\} $, where $\mathcal{D}_t = \left\{ x_{i}, y_{i} \right\} ^{N_t}_{i=1} $ is the dataset of task $t$, and $T$ is the total number of tasks. Dataset $\mathcal{D}_t$ contains $N_t$ labeled samples, $y_{i}$ is the class label of sample $x_{i}$ and $y_{i} \in \mathcal{C}_t$, where $\mathcal{C}_t$ is the class set of task $t$ and the class sets of different tasks are disjoint. 
For replay-based methods, a memory bank is used to store a small subset of seen data, and we also maintain a memory bank $\mathcal{M}$ in our method.
At each time step of task $t$, the model receives a mini-batch data $X \cup X^\mathrm{b}$ for training, where $X$ and $X^\mathrm{b}$ are drawn from the i.i.d distribution $\mathcal{D}_t$ and the memory bank $\mathcal{M}$, respectively. 
Moreover, we adopt the single-head evaluation setup~\cite{EWC}, where a unified classifier must choose labels from all seen classes at inference due to unavailable task identifiers. 
The goal of online CIL is to train a unified model on data seen only once while predicting well on both new and old classes.

\subsection{Online Prototype Definition}
Prior to introducing the online prototypes, we first present the network architecture of our \frameworkName. Suppose that the model consists of three components: an encoder network $f$, a projection head $g$, and a classifier $\varphi$. Each sample $x$ in incoming data $X$ (a mini-batch data from new classes) is mapped to a projected vectorial embedding (representation) $\mathbf{z}$ by encoder $f$ and projector $g$:
\begin{align}
\label{eq:cal_z}
    \mathbf{z} = g(f(\operatorname{aug}(x);\theta_f);\theta_g),
\end{align}
where $\operatorname{aug}$ represents the data augmentation operation, $\theta_f$ and $\theta_g$ represent the parameters of $f$ and $g$, respectively, and $\mathbf{z}$ is $\ell_2$-normalized. 
Similar to Eq.~\eqref{eq:cal_z}, we use $\mathbf{z}^\mathrm{b}$ to denote the representation of replay data $X^\mathrm{b}$ (a mini-batch data from seen classes in the memory bank). 

At each time step of task $t$, the online prototype of each class is defined as the mean representation in a mini-batch:
\begin{align}
\label{eq:cal_p}
    \mathbf{p}_i = \frac{1}{n_i}\sum\nolimits_j\mathbf{z}_j\cdot \mathbbm{1}\{y_j = i\},
\end{align}
where $n_i$ is the number of samples for class $i$ in a mini-batch, and $\mathbbm{1}$ is the indicator function. 
We can get a set of $K$ online prototypes  in $X$, $\mathcal{P} = \left\{ \mathbf{p}_{i} \right\} ^{K}_{i=1}$, and a set of $K^\mathrm{b}$ online prototypes in $X^\mathrm{b}$, $\mathcal{P}^\mathrm{b} = \left\{ \mathbf{p}_i^\mathrm{b} \right\} ^{K^\mathrm{b}}_{i=1}$.
Note that $K = |\mathcal{P}| \leq |\mathcal{C}_t|$ and $K^\mathrm{b} = |\mathcal{P}^\mathrm{b}| \leq \sum_{i=1}^{t}|\mathcal{C}_i| $, where $|\cdot|$ denotes the cardinal number.



\subsection{Online Prototype Equilibrium}
The introduced online prototypes can provide representative features and avoid class-unrelated information.  
These characteristics are exactly the key to counteracting shortcut learning in online CL.
Besides, maintaining the discrimination among seen classes is also essential to mitigate catastrophic forgetting.
Based on these, we attempt to learn representative features of each class by pulling online prototypes $\mathcal{P}$ and their augmented views $\widehat{\mathcal{P}}$ closer in the embedding space, and learn discriminative features between classes by pushing online prototypes of different classes away, formally defined as a contrastive loss:
\begin{align}
\label{eq:proto_infoNCE}
    \ell(\mathcal{P},\widehat{\mathcal{P}})\!=\!
    % \frac{-1}{K}
    \frac{-1}{|\mathcal{P}|}\sum_{i=1}^{|\mathcal{P}|}\!\log\! 
    \tfrac
    {\exp \big(\tfrac{{\mathbf{p}_i^\mathrm{T} \widehat{\mathbf{p}}_i}}{\tau}\big)}
    {
    \sum\limits_{j} \exp \big(\tfrac{{\mathbf{p}_i^\mathrm{T} \widehat{\mathbf{p}}_j}}{\tau}\big)
    +\!
    \sum\limits_{\substack{j \neq i}} \exp \big(\tfrac{{\mathbf{p}_i^\mathrm{T} \mathbf{p}_j}}{\tau}\big) 
    },
\end{align}
where 
$\tau$ is the temperature hyper-parameter, 
$\mathcal{P}$ and $\widehat{\mathcal{P}}$ are $\ell_2$-normalized. To compute the contrastive loss across all positive pairs in both $(\mathcal{P}, \widehat{\mathcal{P}})$ and $(\widehat{\mathcal{P}}, \mathcal{P})$, we define $\mathcal{L}_{\mathrm{pro}}$ as the final contrastive loss over online prototypes:
\begin{align}
    \mathcal{L}_{\mathrm{pro}}(\mathcal{P},\widehat{\mathcal{P}}) = 
    \frac{1}{2}
    \left[\ell(\mathcal{P}, \widehat{\mathcal{P}}) + \ell(\widehat{\mathcal{P}}, \mathcal{P})\right].
\end{align}



Considering the learning of new classes and the consolidation of learned knowledge simultaneously in online CL, we propose Online Prototype Equilibrium (\methodname) to 
learn representative and discriminative features on both new and seen classes by employing $\mathcal{L}_{\mathrm{pro}}$:
\begin{equation}
    \begin{aligned}
    \mathcal{L}_{\mathrm{\methodname}}
    &=
    \mathcal{L}^{\mathrm{new}}_{\mathrm{pro}}(\mathcal{P},\widehat{\mathcal{P}}) + \mathcal{L}^{\mathrm{seen}}_{\mathrm{pro}}(\mathcal{P}^\mathrm{b},\widehat{\mathcal{P}}^\mathrm{b}),
    \end{aligned}
\end{equation}
where
$\mathcal{L}^{\mathrm{new}}_{\mathrm{pro}}$ focuses on learning knowledge from \emph{new} classes, and $\mathcal{L}^{\mathrm{seen}}_{\mathrm{pro}}$ is dedicated to preserving learned knowledge of all \emph{seen} classes.
\textit{This process is similar to a zero-sum game, 
and \methodname aims to achieve the equilibrium to play a win-win game.}
Concretely,
as the model learns, the knowledge of new classes is gained and added to the prototypes over the memory bank $\mathcal{M}$, causing $\mathcal{L}^{\mathrm{seen}}_{\mathrm{pro}}$ gradually changes to the equilibrium that separates all seen classes well, including new ones. 
This variation is crucial to mitigate forgetting and is consistent with the goal of CIL.



\subsection{Adaptive Prototypical Feedback} 
Although \methodname can bring an overall equilibrium, it tends to treat each class \emph{equally}. 
In fact, the degree of confusion varies among classes, 
and the model should focus purposefully on confused classes to consolidate learned knowledge. 
To this end, we propose Adaptive Prototypical Feedback (\dataaugname) with the feedback of online prototypes to sense the classes that are prone to be misclassified and then enhance their decision boundaries.
 
For each class pair in the memory bank $\mathcal{M}$, \dataaugname calculates the distances between online prototypes of all seen classes from the previous time step, showing the class confusion status by these distances. The closer the two prototypes are, the easier to be misclassified. 
Based on this analysis, 
our idea is to enhance the boundaries for those classes. Therefore, we convert the prototype distance matrix to a probability distribution $P$ over the classes via a symmetric Gaussian kernel, defined as follows:
\begin{align}
\label{eq:cal_d}
    P_{i, j} \propto \exp (-\| \mathbf{p}_i^\mathrm{b} - \mathbf{p}_j^\mathrm{b} \|_2^2),
\end{align}
where $i,j \in \{ 1, \ldots, |\mathcal{P}^\mathrm{b}| \}$ and $i \neq j$. 
Then, 
all probabilities are normalized to a probability mass function that sums to one.
\dataaugname returns probabilities to $\mathcal{M}$ for guiding the next sampling process and enhancing decision boundaries of easily misclassified classes. 


Our adaptive prototypical feedback is implemented as a sampling-based mixup. Specifically, 
\dataaugname adaptively selects more samples from easily misclassified classes in $\mathcal{M}$ for mixup~\cite{Mixup} according to the probability distribution $P$. 
Considering not over-penalizing the equilibrium of current online prototypes, we introduce a two-stage sampling strategy for replay data $X^\mathrm{b}$ of size $m$. 
First, we select $n_{\mathrm{\dataaugname}}$ samples  
with $P$, and a larger $P_{a,b}$ means more sampling from classes $a$ and $b$. Here, $n_{\mathrm{\dataaugname}} = \alpha \cdot m$, and $\alpha$ is the ratio of \dataaugname.
Second, the remaining $m-n_{\mathrm{\dataaugname}}$ samples are uniformly randomly selected from the entire memory bank to avoid the model only focusing on easily misclassified classes and disrupting the established equilibrium. 




\subsection{Overall Framework of \frameworkName}
The overall structure of \frameworkName is shown in Fig.~\ref{fig:framework}. \frameworkName comprises two key components based on proposed online prototypes: Online Prototype Equilibrium (\methodname) and Adaptive Prototypical Feedback (\dataaugname). 
With the two components, 
the model can learn representative features against shortcut learning, and 
all seen classes maintain discriminative when learning new classes. 
However, classes may not be compact, because the online prototypes cannot cover full instance-level information.
To further achieve intra-class compactness, 
we employ supervised contrastive learning~\cite{SupCL} to learn instance-wise representations:
\begin{equation}
\begin{aligned}
    \mathcal{L}_{\mathrm{INS}}
    &=
    \sum_{i=1}^{2N} \frac{-1}{\left|I_i\right|} \sum_{j \in I_i} \log \frac{\exp \left(\mathrm{sim}(\mathbf{z}_i, \mathbf{z}_j) / \tau^{\prime}\right)}{\sum\limits_{k \neq i} \exp \left(\mathrm{sim}(\mathbf{z}_i, \mathbf{z}_k) / \tau^{\prime}\right)}
    \\
    &+
    \sum_{i=1}^{2m} \frac{-1}{\left|I_i^{\mathrm{b}}\right|} \sum_{j \in I_i^{\mathrm{b}}} \log \frac{\exp (\mathrm{sim}(\mathbf{z}_i^{\mathrm{b}}, \mathbf{z}_j^{\mathrm{b}}) / \tau^{\prime})}{\sum\limits_{k \neq i} \exp \left(\mathrm{sim}(\mathbf{z}_i^{\mathrm{b}}, \mathbf{z}_k^{\mathrm{b}}) / \tau^{\prime}\right)},
\end{aligned}
\end{equation}
where $I_i=\left\{j \in\{1, \ldots, 2 N\} \mid j \neq i, y_j=y_i\right\}$ and $I_i^\mathrm{{b}}=\left\{j \in\{1, \ldots, 2m\} \mid j \neq i, y_j^\mathrm{b}=y_i^\mathrm{b}\right\}$ are the set of positive samples indexes to $\mathbf{z}_i$ and $\mathbf{z}_i^\mathrm{{b}}$, respectively. $y_i^\mathrm{b}$ is the class label of input $x_i^\mathrm{b}$ from $X^\mathrm{b}$. $N$ is the batch size of $X$. $\tau^{\prime}$ is the temperature hyperparameter.
The similarity function $\mathrm{sim}$ is computed in the same way as Eq.~(9) in OCM~\cite{OCM}.

Thus, the total loss of our \frameworkName framework is given as:
\begin{align}
    \mathcal{L}_{\mathrm{\frameworkName}}=\mathcal{L}_{\mathrm{\methodname}} + \mathcal{L}_{\mathrm{INS}} + \mathcal{L}_{\mathrm{CE}},
\end{align}
where $\mathcal{L}_{\mathrm{CE}} = \mathrm{CE}(y^\mathrm{b}, \varphi(f(\operatorname{aug}(x^\mathrm{b}))))$ is the cross-entropy loss; see Appendix~\ref{appendix:algorithm} for detailed training algorithms.

Following other replay-based methods~\cite{ER, SCR, OCM}, we update the memory bank in each time step by uniformly randomly selecting samples from $X$ to push into $\mathcal{M}$ and, if $\mathcal{M}$ is full, pulling an equal number of samples out of $\mathcal{M}$.


\section{Experiments}
\subsection{Datasets and Metrics}

%dialogue task分四种, intent prediction, slot-filling, semantic parsing, and dialogue state tracking. 然后我们完成的数据集任务是relation extraction, emotional recognition, speech act classification,不太确定是不是都能解决

\textbf{DialogRE} \citep{yu-etal-2020-dialogue} is a relation extraction task based on 1,788 dialogues from the Friends transcript. Each pair of arguments can be classified as one of 36 possible relation types. For each of the 10,168 human-annotated entity pairs, the trigger words are also provided. % such as neighbours or siblings. (subject, object, relation type) triplets

\textbf{EmoryNLP} \citep{zahiri:18a} is an emotion detection task based on 12,606 utterances from the Friends transcript. Each utterance can be classified as one of seven emotions, e.g., joyful, scared. 
%The label is human-annotated based on the dialogue context. 

\textbf{DailyDialog} \citep{DailyDialog} is a dialogue database containing 13,118 simple English dialogues. Each utterance can be assigned an emotion label from seven categories (anger, surprise, etc.). 
%targets both emotion detection and act classification. DailyDialog The label is human-annotated based on the dialogue context.  (Inform, Questions, Directives, Commissive) 

\textbf{MELD} \citep{poria-etal-2019-meld} is an emotion detection task based on 13,000 sentences from the Friends transcript. Each utterance can be classified as one of eight emotions, such as sad, disgust. % or neutral. 

%The label is human-annotated based on the multimodal cues of the dialogue, including visual, audio and textual information. 
\begin{table}[t]
  % \setlength{\tabcolsep}{3.5pt}
% \footnotesize
% \scriptsize
  \centering
  \vspace{-0.35cm}
  \begin{tabular}{lllllll}
    \toprule
  \multirow{2}{*}{\textbf{Method}} &\multirow{2}{*}{\textbf{MELD}} & \multirow{2}{*}{\textbf{ENLP}} & \multirow{2}{*}{\textbf{DDialog}} & \multirow{2}{*}{\textbf{MRDA}} & \multicolumn{2}{c}{\textbf{DialogRE-Test}}      \\
    % \textbf{Method} &\textbf{MELD} & \textbf{ENLP} & \textbf{DDialog} &\textbf{MRDA} & \multicolumn{2}{c}{\textbf{DialogRE}}      \\
    %\cmidrule(r){2-5}
    \cmidrule(l){6-7}   
    % \cmidrule(l){2-2}  \cmidrule(l){3-3} \cmidrule(l){4-4} \cmidrule(l){5-5}  \cmidrule(l){6-7}  
      &  & & & & $F1$  & $F1_c$  \\
    \midrule
    % BERT &61.50	& 34.17	& 54.09& 91.0 & 58.5	& 53.2\\
    % +HiDialog & + 1.78	& +0.63 & +5.55 & +0.3 & + &\textbf{59.8}\\
    PHT  &61.90&-&	60.14&	\textbf{92.4}&- &- \\
    DialogXL  & 62.41 & 34.73 & 54.93 & -& - & -  \\
    % \midrule
    RoBERTa$_s$& 64.19&38.03	&61.65	&91.3	&71.3 & 63.7\\
+Intra-turn &\textbf{65.64}\textsubscript{\textcolor{green}{+1.45}}& \textbf{38.13}\textsubscript{\textcolor{green}{+0.1}} &\textbf{61.83}\textsubscript{\textcolor{green}{+0.28}}&91.5\textsubscript{\textcolor{green}{+0.2}} & \textbf{74.4}\textsubscript{\textcolor{green}{+3.1}}&\textbf{66.6}\textsubscript{\textcolor{green}{+2.9}}\\
    \bottomrule
  \end{tabular}
  % \vspace{-0.2cm}
  % \caption{All methods performance on 5 multi-turn dialogue-based understanding datasets: MELD, EmoryNLP (Weighted-F1), DailyDialog (Micro-F1), MRDA (Top-1 Acc.), DialogRE (F1 and F1$_c$), averaged over five runs. Performance gains over the RoBERTa$_s$ are highlighted in green.}
    \caption{All methods performance on 5 multi-turn dialogue-based understanding datasets: MELD, EmoryNLP, DailyDialog, MRDA, DialogRE, averaged over five runs. Performance gains over the RoBERTa$_s$ are highlighted in green.}
  \label{tab:exp-mtr}
  \vspace{-0.5cm}
\end{table}


\begin{minipage}[]{0.48\linewidth}
\footnotesize
\setlength{\tabcolsep}{7pt}
\begin{center}
\begin{tabular}{l c c} 
 \toprule
  {\textbf{Method}} & \textbf{F1} & \textbf{F1$_c$} \\
 \midrule
HiDialog                      & 77.1        & 68.2       \\ 
w/o attention mask & 76.5 (-0.6) & 67.9 (-0.3) \\
w/o special tokens & 75.6 (-1.5) & 67.4 (-0.8) \\
only intra-turn     & 74.4 (-2.7) & 66.6 (-1.6) \\
\bottomrule
\end{tabular}
\end{center}
\captionof{table}{Ablation Study on HiDialog components on DialogRE to evaluate the individual effect of turn-level attention, turn-level special tokens, and graph module. } %\textit{Ablation study. Turn-level} is omitted for brevity.}
\label{tab:ablation_structure}
\end{minipage}
\hspace*{0.1cm}
\begin{minipage}[]{0.48\linewidth}
\footnotesize
\setlength{\tabcolsep}{7pt}
\begin{center}
% \vspace{-0.5cm}
\begin{tabular}{lccc} 
\toprule
\textbf{Method} & \textbf{I} &  \textbf{II} &  \textbf{III}\\
\midrule
BERT & 42.5  & 60.7 & 65.6 \\
% BERT$_s$ & 46.5 & 61.5 & 69.4 \\
GDPNet  & 47.4  &59.8  &68.1 \\
RoBERTa$_s$ & 57.4 & 69.3 & 79.6 \\
TUCORE-GCN  &62.3  &\textbf{71.3}  &79.9 \\
\midrule
HiDialog & \textbf{76.6}  & 70.5	& \textbf{80.9}  \\
w/o graph module &65.5 & 69.9 & 79.4\\
\bottomrule
\end{tabular}
\end{center}
% \vspace{-0.3cm}
\captionof{table}{All methods performance on DialogRE. We break down the performance into three groups (I) asymmetric inverse relations, (II) symmetric inverse relations, and (III) others.}
\label{tab:symmetric}
\vspace{0.5cm}
\end{minipage}

\textbf{MRDA} \citep{MRDA} is a dialogue act task based on 75 hours of real-life meeting transcript. Each sentence is assigned a general dialogue act (topic change, repeat, etc.) and a specific dialogue act (apology, suggestion, etc.).  % explanation sympathy

\textbf{Metrics}. For DialogRE, F1 and F1$_c$ are used as evaluation metrics. F1$_c$ modifies F1 by taking an early part of the dialogue as input \cite{yu-etal-2020-dialogue}. For MELD and EmoryNLP, we use weighted-F1 as metrics. For DailyDialog, the Micro-F1 score excluding the neutral class is used as the metric. 
\subsection{Results and Analysis}
\textbf{Overall Performance}. We first evaluated HiDialog on the Dialogue Relation Extract (DRE) dataset, DialogRE \citep{yu-etal-2020-dialogue} and the Emotion Recognition in Conversation (ERC) dataset, MELD \citep{poria-etal-2019-meld}. We selected BERT \citep{bertbase}, GDPNet \citep{xue2021gdpnet}, RoBERTa$_s$ \citep{yu-etal-2020-dialogue}, SimpleRE \citep{SimpleRE}, and TUCORE-GCN \citep{lee2021graph} as baselines. As reported in Table \ref{tab:exp-re}, HiDialog established new state-of-the-art results on both datasets. 
On the DialogRE test set, HiDialog surpassed the previous SOTA, TUCORE-GCN, by 4\% in F1 and 2.3\% in F1$_c$. On the MELD dataset, HiDialog outperformed TUCORE-GCN by 1.5\% in weighted F1. 
%, surpassing the previous SOTA by 4\% in F1 and 2.3\% in F1$_c$ on the DialogRE test set, by 1.5\% in weighted F1 score on the MELD test set. 


\textbf{Towards Generality}.
% In view of the simplicity and effectiveness of the intra-turn modeling, it is expected to have a general use for further work on dialogue understanding. To validate this idea, we incorporate our intra-turn modeling into the baseline encoder, without any extra components such as the inter-turn module or speaker embeddings. For a fair comparison, only the global $[CLS]$ token embedding from the encoder output is fed into a softmax classifier to make a prediction. 
Our intra-turn modeling's simplicity suggests its potential as a valuable solution for enhancing dialogue understanding without the need for extra pre-training. To assess this claim, we integrated it into the baseline encoder without any additional components, such as an inter-turn module or speaker embeddings. For fair comparisons, only the encoder's global $[CLS]$ token was used in a softmax classifier for prediction.
%Our intra-turn modeling's simplicity and effectiveness suggest its potential as a valuable solution for enhancing dialogue understanding, without additional pre-training. 

% \begin{wraptable}{r}{8.2cm}
%   \setlength{\tabcolsep}{3.5pt}
% % \footnotesize
% \scriptsize
%   \centering
%   \vspace{-0.35cm}
%   \begin{tabular}{lllllll}
%     \toprule
%   \multirow{2}{*}{\textbf{Method}} &\multirow{2}{*}{\textbf{MELD}} & \multirow{2}{*}{\textbf{ENLP}} & \multirow{2}{*}{\textbf{DDialog}} & \multirow{2}{*}{\textbf{MRDA}} & \multicolumn{2}{c}{\textbf{DialogRE}}      \\
%     % \textbf{Method} &\textbf{MELD} & \textbf{ENLP} & \textbf{DDialog} &\textbf{MRDA} & \multicolumn{2}{c}{\textbf{DialogRE}}      \\
%     %\cmidrule(r){2-5}
%     \cmidrule(l){6-7}   
%     % \cmidrule(l){2-2}  \cmidrule(l){3-3} \cmidrule(l){4-4} \cmidrule(l){5-5}  \cmidrule(l){6-7}  
%       &  & & & & $F1$  & $F1_c$  \\
%     \midrule
%     % BERT &61.50	& 34.17	& 54.09& 91.0 & 58.5	& 53.2\\
%     % +HiDialog & + 1.78	& +0.63 & +5.55 & +0.3 & + &\textbf{59.8}\\
%     PHT  &61.90&-&	60.14&	\textbf{92.4}&- &- \\
%     DialogXL  & 62.41 & 34.73 & 54.93 & -& - & -  \\
%     % \midrule
%     RoBERTa$_s$& 64.19&38.03	&61.65	&91.3	&71.3 & 63.7\\
% +Intra-turn &\textbf{65.64}\textsubscript{\textcolor{green}{+1.45}}& \textbf{38.13}\textsubscript{\textcolor{green}{+0.1}} &\textbf{61.83}\textsubscript{\textcolor{green}{+0.28}}&91.5\textsubscript{\textcolor{green}{+0.2}} & \textbf{74.4}\textsubscript{\textcolor{green}{+3.1}}&\textbf{66.6}\textsubscript{\textcolor{green}{+2.9}}\\
%     \bottomrule
%   \end{tabular}
%   \vspace{-0.2cm}
%   % \caption{All methods performance on 5 multi-turn dialogue-based understanding datasets: MELD, EmoryNLP (Weighted-F1), DailyDialog (Micro-F1), MRDA (Top-1 Acc.), DialogRE (F1 and F1$_c$), averaged over five runs. Performance gains over the RoBERTa$_s$ are highlighted in green.}
%     \caption{All methods performance on 5 multi-turn dialogue-based understanding datasets: MELD, EmoryNLP, DailyDialog, MRDA, DialogRE, averaged over five runs. Performance gains over the RoBERTa$_s$ are highlighted in green.}
%   \label{tab:exp-mtr}
%   \vspace{-0.35cm}
% \end{wraptable}



% \begin{minipage}[]{0.5\linewidth}
% % \begin{figure}[h]
% % \vspace{-0.2cm}
% \footnotesize
% \centering
% \includegraphics[width=7cm]{sections/length.pdf}
% \vspace{-0.5cm}
% \captionof{figure}{Analysis of robustness of HiDialog tackling increasing utterance length compared to baseline TUCORE-GCN on DialogRE dataset.}
% \vspace{0.6cm}
% \label{fig:length}
% % \end{figure}
% \end{minipage}



We conducted the experiment on 5 datasets from 3 different tasks: DRE (DialogRE), ERC (MELD, EmoryNLP \citep{zahiri:18a}, DailyDialog \citep{DailyDialog}), and Dialogue Act Classification (MRDA \citep{MRDA}). We chose RoBERTa$_s$, Pretrained Hierarchical Transformer (PHT) \citep{chapuis2020hierarchical}, and DialogXL \citep{DialogXL} as baselines. Compared to PHT and DialogXL, both of which require additional pre-training to address the domain adaption gap, the performance of proposed intra-turn modeling is surprisingly good in all 5 datasets (Table \ref{tab:exp-mtr}). 

% Moreover, we conducted an ablation study and analysis, which further reveals HiDialog is good at handling asymmetric relations and robust against increasing utterance length (see Appendix \ref{ap:ablation}). 

% \begin{wraptable}{r}{7cm}
% % \scriptsize
% % \vspace{-0.5cm}
% % \setlength{\tabcolsep}{3.5pt}
% \begin{center}

% \begin{tabular}{l c c} 
%  \toprule
%   {\textbf{Method}} & \textbf{F1} & \textbf{F1$_c$} \\
%  \midrule
% HiDialog                      & 77.1        & 68.2       \\ 
% w/o attention mask & 76.5 (-0.6) & 67.9 (-0.3) \\
% w/o special tokens & 75.6 (-1.5) & 67.4 (-0.8) \\
% only intra-turn     & 74.4 (-2.7) & 66.6 (-1.6) \\
% \bottomrule
% \end{tabular}
% \end{center}
% % \vspace{-0.2cm}
% \caption{Ablation Study on HiDialog components on DialogRE to evaluate the individual effect of turn-level attention, turn-level special tokens, and graph module. } %\textit{Ablation study. Turn-level} is omitted for brevity.}
% \label{tab:ablation_structure}
% % \vspace{-0.2cm}
% \end{wraptable}



\textbf{Ablation study on components.} We conducted an ablation study on DialogRE to evaluate key components in HiDialog: turn-level attention, turn-level special tokens, and inter-turn module (Table \ref{tab:ablation_structure}). First, after we removed the turn-level attention mask, the performance slightly dropped. In this case, these special tokens are able to aggregate information from the entire sequence, thus they are not context-aware at the turn level. We experimented with removing intra-turn modeling, resulting in only one difference from the final HiDialog: here we used an average of corresponding token embeddings for initialization. The $F1$ score decreases by 1.5\% and the $F1_c$ score declines by 0.8\%.

\textbf{Analysis of relations}. We grouped the test set of DialogRE according to the relation types into three subsets: (I) asymmetric, when a relation type differs from its inversion (e.g. \textit{children} and \textit{parents});  (II) symmetric, when a relation type is the same as its inversion (e.g. \textit{spouse}); (III) other, when a relation type does not have inversion (e.g. \textit{age}). We compared the performance of our model with baselines and report the results in Table \ref{tab:symmetric}. As we can observe, there is a great performance increase in the asymmetric subset while the F1 score drops moderately for symmetric relations. This trend reverses when we remove the graph module in our method (i.e. symmetric $>$ asymmetric). 
% \clearpage

\textbf{Analysis of robustness against increasing utterance length.} With the hierarchical aggregation in HiDialog, each turn-level special token is enforced to capture intra-turn critical information regardless of the whole dialogue. This nature enables our method to handle dialogues of various lengths. Thus, we further divided the samples in the DialogRE test set into six groups according to their lengths and compared HiDialog against the previous SOTA, TUCORE-GCN. As shown in Figure \ref{fig:length}, our method consistently outperforms TUCORE-GCN in all groups, where the largest performance gap can be found in the group with less than 100 tokens. Moreover, TUCORE-GCN shows a great drop with an increase of length (i.e., from $[400,500)$ to $[500,+\infty)$), while HiDialog maintains decent performance for long sequences.

 \begin{figure}[t]
\centering
% \vspace{-0.2cm}
\footnotesize
\centering
\includegraphics[width=7cm]{sections/length.pdf}
% \vspace{-0.5cm}
\captionof{figure}{Analysis of robustness of HiDialog tackling increasing utterance length compared to baseline TUCORE-GCN on DialogRE dataset.}
% \vspace{0.6cm}
\label{fig:length}
\end{figure}


%Note that both PHT and DialogXL are pre-trained methods that require extra computation
%Moreover, it outperforms the current SOTA by 1.3\% in $F1$ and 0.7\% in $F1_c$ on the DialogRE dataset, by 0.3\% in weighted F1 on MELD.
%HiDialog performs well in managing asymmetric relations and handling longer utterances, as revealed in our ablation study (see Appendix \ref{ap:ablation}). It provides an effective solution for bridging the gap between pre-training on general corpora and dialogue understanding without additional computational costs or training data while maintaining good performance. 
%Considering that our turn-level attention is easy to adopt and does not introduce any parameters to the base encoder, we believe it can be used as a strong baseline or plug-in module for future work in the community.
%Our turn-level attention mechanism can be effortlessly integrated into the base encoder without introducing any new parameters. Thus, we anticipate that it will serve as a compelling baseline or plug-in module for upcoming research in this field.
In this work, we demonstrate that it's possible to distill huge models trained on large datasets to obtain much smaller models that perform well on paralinguistic speech tasks.
The distillation uses only \textbf{7\% of the training data} and is entirely from public sources. The models we obtain are between 22MB and 314MB, and achieve between \textbf{90\% and 96\% of the larger CAP12 accuracy on 6 of 7 tasks}. These models are between \textbf{1\% and 15\% the size} of the original model. We release the model to allow the research community to benefit from the practical applications of self-supervised representations for paralinguistic speech.
\section{Acknowledgments}
\label{sec:acknowledgments}
We thank the organizers of acronym identification and disambiguation competitions and the reviewers for their valuable comments and suggestions.

% \clearpage
\bibliography{references}

\end{document}
