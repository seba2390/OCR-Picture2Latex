\documentclass[aps,floatfix,prl]{revtex4}
\usepackage{graphicx,subfigure,amsmath,bbm}

%\usepackage{babel}
%\addto{\captionsenglish}{\renewcommand*{\appendixname}{}}


\usepackage{graphicx}% Include figure files
\usepackage{dcolumn}% Align table columns on decimal point
%\usepackage{bm}% bold math
\usepackage{hyperref}
\usepackage{multirow}
\usepackage{array}
%\usepackage{booktabs}
%\usepackage{ctable}
%\usepackage{upgreek}
\usepackage{epsfig}
\usepackage{mathrsfs}
\usepackage{amssymb}
\usepackage{amsbsy}
\usepackage{color}
%\usepackage{cancel}
%\usepackage{epsf}
\usepackage{pifont}
%\usepackage{marginnote}
\usepackage{float}
%\usepackage{eufrak}
\usepackage[normalem]{ulem}
\newcommand{\mylabel}[1]{\label{#1}} 

%%
\newcommand{\myonlinecite}[1]{[\onlinecite{#1}]}
%%%%%\newcommand{\mycite}[1]{{\tt[#1]}\cite{#1}}
\newcommand{\mycite}[1]{\cite{#1}}
\newcommand{\red}{\textcolor{red}}

\newcommand{\titlename}{TlCuCl$_3$ supplimentary}

\begin{document}
\relax



\title{\titlename}
\title{Supplementary Material: Higgs-Axion conversion and anomalous magnetic phase diagram in TlCuCl$_3$}

\author{Gaurav Kumar Gupta$^1$}
%\email[]{ggaurav@iisc.ac.in}
\author{Kapildeb Dolui$^1$}
\author{Abhinav Kumar$^2$}
\author{D. D. Sarma$^2$}
\author{Tanmoy Das$^1$}
%\email[]{tnmydas@gmail.com}
\affiliation{$^1$ Department of Physics, Indian Institute of Science, Bangalore, India - 560012}
\affiliation{$^2$ Solid State and Structural Chemistry Unit, Indian Institute of Science, Bangalore, India - 560012}



\date{\today}

\maketitle
In this supplementary material, we give details of the derivations corresponding to various terms presented in the main text.  In Sec. I, we show DFT calculations including spin-orbit coupling in the paramagnetic and AFM phases. Next we derive the Ginzburg-Landau-Chern-Simons theory. Subsequently, we calculate the Lagrangian minima and AFM transition temperature  for the CSGL theory. In the last section, calculations of Higgs mass and corresponding lifetime are presented.

\section{DFT calculation W/ SOC}
We compute the DFT band structure using the Local  Density Approximation (LDA) exchange correlation as implemented in the Vienna  ab-initio  simulation  package  (VASP)\cite{DFT1}. The results remain characteristically the same for the GGA and other functionals. The DFT band structure also agrees well with a previous LMTO calculation.\cite{LMTODFT} In our LDA+U calculation, the electronic wave function is expanded using plane wave up to a cutoff energy of 500 eV. Brillouin zone sampling is done by using a ($8\times8\times8$) Monkhorst-Pack k-grid. Projected augmented-wave (PAW) pseudo-potentials are used to describe the core electron in the calculation\cite{DFT2}. 


\begin{table}[H]
\caption{Table to show the experimantal and relaxed lattice parameters}
\begin{center}
	\begin{tabular}{|c|c|c|c|c|c|c|}
		\hline
		& a (\AA) & b (\AA) & c(\AA) & $\alpha$($^o$) & $\beta$($^o$)&$\gamma$($^o$) \\
		\hline
		Experimental & 14.144 & 8.890 & 3.982 & 83.68 & 90  & 90\\
		\hline
		Relaxed &13.587 &8.651 &3.886 &84.261 &89.983 &90.01 \\
		\hline
	\end{tabular}
\end{center}
\end{table}
We found in the DFT band structure that the magnetic gap is larger than the crystal field splitting (CFS), and thus one obtains a band insulating behavior in the electronic properties. As one approaches the magnetic critical point, the magnetic gap takes over for $\Delta<\Delta_{\rm CFS}$.


\begin{figure}
	\begin{center}
	\includegraphics[scale=0.7]{relaxed_experimental.png}
	\caption{Comparison of band structure with relaxed (a) and experimental (b) lattices constant. In the relaxed parameter case, we find that the bands are inverted at Z-point ($t^\prime<t$), while it gives a Dirac cone ($t^\prime\sim t$). Therefore, we anticipate the topological nature should be persistent to ambient pressure, however the magnetic order disappears here.}
		\end{center}
\end{figure}

\begin{figure}[h]
	\includegraphics[scale=0.7]{fig_with_different_u.png}
	\caption{\label{phase}Computed {\it ab-inito} band structure of TlCuCl$_3$ with experimental coordinates in presence of spin-orbit coupling for different Hubbard $U$ for Cu 3d orbitals. }
\end{figure}
%\begin{figure}[h]
%	\includegraphics[scale=0.7]{SO.png}
%	\caption{\label{phase}Computed {\it ab-inito} band structure of TlCuCl$_3$ in presence of spin-orbit coupling (a) with magnetism and (b) without magnetism. }
%\end{figure}



\section{SSH term}
The expression for SSH like term in our 3D model looks like
\begin{eqnarray}
H_{AA}&=C_{10} \cos((k_x+k_z)/2)+C_{11}\cos(k_y)+C_{12}\cos((k_x-k_z)/2)+C_{13}\cos(2*k_y)\\
&+C_{14}\cos(k_x)+C_{15}\cos(k_z)+C_{16}\cos(k_x+k_z)+C_{17}\cos(k_x-k_z)+C_{18}\nonumber\\
H_{AB}&=e^{ik_z/2}\left(C_0+C_1e^{-ik_z}+\left(C_2+C_3e^{-ik_z}\right)\left(C_4e^{i(k_x+k_y)/2}+C_5e^{-i(k_x+k_y)/2}\right)\right.\\
&\left.+\left(C_6+C_7e^{-ik_z}\right)\left(C_8e^{i(k_x-k_y)/2}+C_9e^{-i(k_x-k_y)/2}\right)\right)\nonumber
\end{eqnarray}
where C's are the fitting parameters. The above equation for inter-sublattice hopping can be rewritten in the form,
\begin{eqnarray}
H_{AB}=e^{i k_z/2}T_{\bf k\perp}(1+\frac{T'_{\bf k\perp}}{T_{\bf k\perp}}e^{ik_z})\nonumber
\end{eqnarray}

where
\begin{eqnarray}
T_{\bf k\perp}=C_0+C_2C_4e^{i(k_x+k_y)/2}+C_2C_5e^{-i(k_x+k_y)/2}+C_6C_8e^{i(k_x-k_y)/2}+C_6C_9e^{-i(k_x-k_y)/2}\nonumber\\
T'_{\bf k\perp}=C_1+C_3C_4e^{i(k_x+k_y)/2}+C_3C_5e^{-i(k_x+k_y)/2}+C_7C_8e^{i(k_x-k_y)/2}+C_7C_9e^{-i(k_x-k_y)/2}\nonumber\\
\end{eqnarray}
The fitting parameters are $C_{0-18}$=[0.156, 0.209, -0.076, 0.223, -0.135, -0.325, 1.466, -0.019, -0.030, 0.045, 0.002, -0.033, -0.003, 0.027, -0.002, 0.095, -0.105, -0.019] in eV.

\begin{figure}[h]
	\includegraphics[width=70mm,scale=3.5]{hopping.png}
	\caption{\label{phase}Figure showing effective hopping in (x+y,z) plane. Here $C_0^{'},C_0^{''},C_1^{'}$ and $C_1^{''}$ correspond to $C_2 C_4, C_2 C_5, C_3 C_4 $ and $C_3 C_6$ respectively. We have similar kind of hopping terme in (x-y,z) plane as well.}
\end{figure}

{
\section{Full form of SOC}
Although SOC is weak (as seen from the DFT calculations), however, it introduces chirality in the low energy spectrum across ${\bf k}^*$. Due to anisotropy between the $ab$-plane and the $c$-axis, we can conveniently split the SOC Hamiltonian into in-plane and out-of-plane, as 
%
\begin{eqnarray}
H_{\rm SO}&=&\sum_{i,j\in (A,B)}\sum_{{\bf k},ss'}\left[\psi^{\dag}_{i,s}({\bf k})\left({\bm \alpha}^{ij}_{\bf k}\times {\bm \sigma}_{ss'}\right)\psi_{j,s'}({\bf k}) + \psi^{\dag}_{i,s}({\bf k})\left({\bm \beta}^{ij}_{\bf k}\cdot\sigma^z_{ss'}\right)\psi_{j,s'}({\bf k})\right].
\label{Eq:HSO}
\end{eqnarray}
In the first term, the in-plane spin is locked to its transverse velocity matrix ${\bm \alpha}^{ij}_{\bf k}$, while the out-of-plane spin is locked to the longitudinal one ${\bm \beta}^{ij}_{\bf k}$. The components of the velicity operators are
%
\begin{eqnarray}
{\bm \alpha}^{ij}_{\bf k}& = &\alpha^{ij}_0\left(-\frac{\partial \xi^{ij}_{\bf k}}{\partial k_y},\frac{\partial \xi^{ij}_{\bf k}}{\partial  k_x},0\right), ~{\bm \beta}^{ij}_{\bf k}=\beta^{ij}_0\left(0,0,\frac{\partial \xi^{ij}_{\bf k}}{\partial k_z}\right). 
\label{Eq:SOC}
\end{eqnarray}
%
$\alpha^{ij}_0$, and $\beta^{ij}_0$ are the corresponding SOC strengths. Eq.~\ref{Eq:HSO} allows several SOC terms, however, fitting to DFT results indicate that $\beta^{ij}_0\rightarrow 0$, implying that the spins are aligned perpendicular to the SSH chain. $\alpha^{\rm AB}_0$ is the second nearest neighbor SOC term, and is negligibly small, while $\alpha^{\rm AA}_0=5$ meV.

}
\section{Chern-Simons-Ginzburg-Landau theory}
\subsection{Ginzburg-Landau theory}
Here we develop the Ginzburg Landau theory of our Hamiltonian around the antiferromagnetic order parameter.  We write the partition function for the total Hamiltonian $H+H_{\rm I}$ written in terms of the Dirac matrices in Eq.~(4) in the main text as
\begin{eqnarray}
Z=\int \mathcal{D}[\psi,\bar{\psi}]\exp\left[-\int_{0}^{\beta}d\tau\left(\bar{\psi}(\partial_\tau\mathbb{I}_{4\times 4}-H_{\bf k}) \psi - J\sum_{\langle{i,j}\rangle}S^z_i.S^z_j\right)\right],
\label{Eq:Z1}
\end{eqnarray}
where $\psi$ are 4 component Grassman variables $\psi=(\psi_{A\uparrow}, \psi_{B\uparrow},\psi_{A\downarrow},\psi_{B\downarrow})^T$ (equivalent to the Dirac spinor used in the main text), and $\bar{\psi}$ is the conjugate of $\psi$. $i$,$j$ denote `A', `B' sublattices. ${\bf S}_i$ are the corresponding spin operators. We orient the spin-quantization axis along $\sigma^z$, i.e., we only consider $S_i^z$ component. We define the antiferromagnetic (AF) field as $\phi = \left( S^z_{\rm A} - S^z_{\rm B}\right)/2$. Using the Hubbard Stratonovich transformation for $H_{\rm I}$ in terms of the FM fields in the last term of Eq.~\eqref{Eq:Z1}, we obtain
\begin{eqnarray}
\int \mathcal{D}[\psi,\bar{\psi}]\exp{\left(-J\sum_{\langle{i,j}\rangle}{\bf S}_i.{\bf S}_j\right)}&=&\int \mathcal{D}[\psi,\bar{\psi},\phi,\bar{\phi}]\exp\left[-J\phi(S^z_{{\rm A}}-S^z_{\rm B})-\frac{\phi^2}{4J}\right].
\label{Eq:Z1HI}
\end{eqnarray}

Now we express $S^z_i$ in terms of the Grassman variables as $S^z_i=(\bar{\psi}_{i\uparrow}\psi_{i\uparrow}-\bar{\psi}_{i\downarrow}\psi_{i\downarrow})/2$. In doing so, we can write the AF  term in terms of the Grassman spinor $\phi$ as $J\phi (S^z_{{\rm A}}-S^z_{\rm B}) = \bar{\psi}\left(J\phi\Gamma_5 \right)\psi$, where $\Gamma_5=\sigma_z\otimes\tau_z$, as defined in the main text. Substituting this identity in Eq.~\eqref{Eq:Z1HI}, and then inserting it back to Eq.~\eqref{Eq:Z1}, we get
\begin{eqnarray}
Z=\int \mathcal{D}[\psi,\bar{\psi},\phi,\bar{\phi}]\exp\left[-\int_{0}^{\beta}d\tau\bar{\psi}\left(\mathbf{G}_0^{-1}(\tau,{\bf k}) - \mathbf{M}(\phi)\right) \psi\right].
\label{Eq:Z2}
\end{eqnarray}
%
Here we have defined the non-interacting Green's function matrix $\mathbf{G}_0^{-1}({\bf k},\tau)=\partial_\tau\mathbb{I}_{4\times 4}-H_{\bf k}$, and the magnetization matrix as $\mathbf{M}(\phi)= J\phi\Gamma_5 $. Now we can go to the Matsubara frequency $i\omega_n$ domain and integrate out fermion variables ($\psi,\bar{\psi}$) to get the effective Lagrangian density as
\begin{eqnarray}
\mathcal{L}={\rm Log} \left[{\rm Det} \left(\sum_{{i\omega_n},{\bf k}} \mathbf{G}_0^{-1}(i\omega_n,{\bf k})-\mathbf{M}(\phi)\right)\right]-\frac{\phi^2 }{4J}.
\end{eqnarray}

Under the saddle point approximation around the AFM, using the identity $\text{Log[Det}[..]]=\text{Tr[Log}[..]]$ and $\text{Log}[x]=-\sum_{n=1}^{\infty}(-1)^nx^n/n$, we get the GL Lagrangian potential
\begin{eqnarray}
\mathcal{L}_{\rm GL}= -\alpha|\phi|^2-\beta|\phi|^4+\mathcal{O}(|\phi|^6),
\label{LGL}
\end{eqnarray}
where 
\begin{eqnarray}
\alpha &=&-\frac{1}{4J}+{\rm Tr}\sum_{k,k'} {\bf G}_0(k)\Gamma_5 {\bf G}_0(k')\Gamma_5,\\
\beta &=& {\rm Tr}\sum_{k,k',k'',k'''}{\bf G}_0(k)\Gamma_5 {\bf G}_0(k')\Gamma_5 {\bf G}_0(k'')\Gamma_5 {\bf G}_0(k''')\Gamma_5,
\end{eqnarray}
where we define $k=({\bf k},i\omega_n)$. Exact computation of $\alpha$, and $\beta$ variables are difficult, but we we can already grasp the essence that $\beta>0$, and $\alpha\rightarrow 0$ when the particle-hole bubble compensates the interaction terms. These results are typical for the GL theory.

\subsection{The Chern-Simons term}
Chern-Simons term arises in the presence of electromagnetic (EM) fields. We assume the probe electromagnetic fields as $(A_0, {\bf A})$. In addition to probe fields, there may arise intrinsic `statistical' gauge fields $(a_0, {\bf a})$ due to fluctuations of the bosonic fields $\phi$. This can be seen easily. The statistical gauge field arises due to fluctuations of the order parameter, so we can write $a_0\propto \partial_t (\delta \phi\delta \phi)$, and ${\bf a}\propto \nabla (\delta \phi\delta \phi)$, where $\delta \phi$ is the fluctuation of the AFM field around its saddle point $\phi_0$. Such intrinsic gauge clearly arises from the $|\phi|^4$ term in the GL potential in Eq.~\eqref{LGL}, and persists above the AFM critical point. We are not particularly interested in the details of the origin of the intrinsic guage field, except it conveys an important message that such due to spin-fluctuations in space-time dimensions, there can be CS term even in the absence of any external EM field. Readers interested in the details of the origin of such statistical gauge field can refer to Refs.~\cite{CSGL_Zhang,axionJPSJ} and references therein

Thanks to the linear combination form of the intrinsic and probe gauge fields in the Lagrangian, we can combine their effects in a total gauge field as $\mathcal{A}_0=a_0+A_0$, and $\bm{\mathcal{A}}={\bf a}+{\bf A}$. Due to the total EM field, we have a typical Maxwell term ($\mathcal{L}_{\rm MW}$), and the Chern-Simons term $\mathcal{L}_{\rm CS}$ as defined in 3+1 dimensions as\cite{zhang_prb,JMoore,CSGL_rest}
\begin{eqnarray}
\mathcal{L}_{\rm MW}&=&-\frac{1}{4}\mathcal{F}_{\mu\nu}\mathcal{F}^{\mu\nu}-\mathcal{A}_{\mu}\mathcal{J}^{\mu},\\
\mathcal{L}_{\rm CS}&=& \theta\frac{\hbar}{\Phi_0^2}\epsilon^{\mu\nu\sigma\tau}\partial_{\mu}\mathcal{A}_{\nu}\partial_{\sigma}\mathcal{A}_{\tau}-\mathcal{A}_{\mu}\mathcal{J}^{\mu}.
\label{CS}
\end{eqnarray}
where the Einsteins summation convention is implied. $\mathcal{F}_{\mu\nu}=\partial_{\mu}\mathcal{A}_{\nu}-\partial_{\nu}\mathcal{A}_{\mu}$, current density $\mathcal{J}^{\mu}$ is included by conservation principles and can be eliminated for the Lagrangian minimization problem of our interest. $\theta$ is the axion angle which is related to the momentum-space non-Abelian Berry connection $\mathscr{A}^{st}_{\mu}=-i\langle u^{s}_{\bf k}|\partial_{k_{\mu}}u^{t}_{\bf k}\rangle$, where $|u^s_{\bf k}\rangle$ is the $s^{\rm th}$-eigenstate of the mean-field Hamiltonian,\cite{zhang_nat} as
\begin{eqnarray}
\theta=\frac{1}{4\pi}\int_{\rm BZ}d^3k \epsilon^{\mu\nu\sigma}{\rm Tr}\left[{\mathscr{A}}_{\mu}\partial_{\nu}{\mathscr{A}}_{\sigma}+i\frac{2}{3}{\mathscr{A}}_{\mu}{\mathscr{ A}}_{\mu}{\mathscr{A}}_{\mu}\right].
\end{eqnarray}
By evaluating the eigenvectors of our Hamiltonian in the main text, we can obtain an algebric, gauge independent form of the axion angle can be deduced to be:
\begin{eqnarray}
\theta=\int_{\rm BZ}\frac{d^3k}{4\pi} \frac{2|d|+d_4}{(|d|+d_4)^2|d|^3}\epsilon^{ijkl}d_i\partial_{k_x} d_j\partial_{k_y} d_k\partial_{k_z} d_l.
\label{ax_2}
\end{eqnarray}
where $|d|^2=\sum_{i=1}^5|d_i|^2$, and $d_5=J\phi$, and $i,j,k,l$ runs from 1,2,4,5. The above integral evaluates the solid angle enclosed in the $d$-space as one encircles the entire 3D Brillouin zone in the $k$-space. Reminiscence to the topological phase transition in a single SSH chain, here also we can show that $\theta$ acquires finite value where the zeros of $d_3({\bf k})$ term lies inside the solid angle, giving the condition that $T_{{\bf k}_{\perp}}\le T'_{{\bf k}_{\perp}}$, for ${\bf k}\in {\rm BZ}$. Having a Dirac cone in the SOC band structure, we ensure that such a condition is automatically satisfied in the non-interacting phase ($d_5=0$). Axion angle can be calculated numerically. We are interested in studying its behavior as a function AFM field $\phi$, which yields an exponential function $\pi e^{-\lambda|\phi|}$, where $\lambda$ is a fitting parameter. For both signs of $\phi$, $\theta$ decreases from $\pi$ at $\phi\rightarrow0$. Absorbing the remaining factors in the CS term into  $\gamma=\frac{\pi\hbar}{\Phi_0^2}{\bf E}\cdot{\bf B}$, we obtain $\mathcal{L}_{\rm CS}=\gamma e^{-\lambda|\phi|}$. $\gamma>0$ ($\gamma<0$) if ${\bf E}$ and ${\bf B}$ are parallel (antiparallel) to each other.    

\subsection{Chern-Simons-Ginzburg-Landau theory} 
The Kinetic energy term due to the AFM field is 
\begin{eqnarray}
\mathcal{L}_{\rm KE} &=& i\phi^*\mathcal{D}_0\phi + \frac{1}{2m}\phi^*\bm{\mathcal{D}}^2\phi.
\label{Eq:KE}
\end{eqnarray}
Here the covariant derivative operators are $\mathcal{D}_0=\partial_t +ie\mathcal{A}_0$, and $\bm{\mathcal{D}}=i{\bm \nabla}+e\bm{\mathcal{A}}$. Therefore the total Lagrangian density becomes\cite{CSGL_rest,CSGL_Zhang} $\mathcal{L}_{\rm total}=\mathcal{L}_{\rm KE} + \mathcal{L}_{\rm MW} + \mathcal{L}_{\rm GL} + \mathcal{L}_{\rm CS}+ \mathcal{L}_{\rm AN}$. Here $\mathcal{L}_{\rm AN}$ represents the contribution from anyons arising from the fluctuation of the order parameters. The Maxwell term does not involve the order parameter or axion term, and thus also can be neglected. Neglecting space-time dependence of the order parameter, we otain the effect CSGL term in terms of the AFM field $\phi$ as
{
\begin{eqnarray}
\mathcal{L}_{\rm CSGL}&=&-\alpha|\phi|^2 - \beta|\phi|^4 - \gamma e^{-\lambda|\phi|}+\gamma.
\label{CSGL}
\end{eqnarray}
}

\section{Free energy minima and Transition temperature}
The Free energy is minimum where the Lagrangian in Eq.~\eqref{CSGL} is maximum. Solving for $\frac{\partial F[\phi]}{\partial \phi}|_{\phi_0}=0$, we get $2\left(\alpha+2\beta|\phi_0|^2\right)|\phi_0| = \gamma\lambda e^{-\lambda|\phi_0|}$. This equation cannot be solved analytically, but for small $\lambda$, we can expand the exponential up to third power in $\phi$ to get 
\begin{eqnarray}
2\alpha|\phi_0|+4\beta|\phi_0|^3-\lambda \gamma (1-\lambda|\phi_0|+\frac{\lambda^2|\phi_0|^2}{2!}-\frac{\lambda^3|\phi_0|^3}{3!})=0.
\label{saddlept2}
\end{eqnarray}
It turns out that any arbitrarily small value of $\gamma$, this leads to a minima in free energy away from $\phi=0$ but very close. At high temperature, it goes arbitrarily close to 0. 

The axion term leads to a correction to the N\'eel temperature $T_{\rm N}$. If we assume $T_{\rm N,0}$ is the N\'eel temperature without the axion term, then for a second order phase transition, we can write { $\alpha=\alpha_0(1-T/T_{\rm N,0})$}, where $\alpha_0>0$ is a constant. This coefficient is modified to  
$\alpha'=\alpha+\gamma\lambda^2/2$. $\alpha'=0$ gives the AFM transition. Therefore, the phase transition condition becomes $\alpha_0\left(1-\frac{T_{\rm N}}{T_{\rm N,0}}\right) + \gamma\lambda^2/2 = 0$, which gives $T_{\rm N}=T_{\rm N,0}\left(1+\frac{\gamma\lambda^2}{2 \alpha_0}\right)$.

\section{ Higgs Mass and Lifetime}
Using CSGL Lagrangian, we can calculate the mass of the Higgs mode. This can be done by substituting an amplitude fluctuation term $\delta\phi$ in the Lagrangian, and calculating the coefficient of the quadratic term in fluctuations. (We do not worry about the Goldstone modes here since they are guaged awayout from the Lagrangian.) Replacing $|\phi|\to |\phi_0|+\delta \phi$ in Eq.~\eqref{CSGL} (assuming $\phi$ is positive), evaluating $\partial^2 \mathcal{L}_{\rm CSGL}[|\phi_0|+\delta \phi]/\partial^2 \delta \phi$ at $\delta \phi\to 0$, we get the Higgs mass as
{
\begin{eqnarray}
M=\frac{\partial^2 \mathcal{L}_{\rm CSGL}}{\partial^2 \delta \phi}\Big|_{\delta\phi=0}&=&-2\alpha-12\beta|\phi_0|^2+\gamma\lambda^2e^{-\lambda |\phi_0|},\nonumber\\
&=&-2\alpha  -2\lambda\alpha|\phi_0|-12\beta|\phi_0|^2 -4\lambda\beta|\phi_0|^3.
\end{eqnarray}
}
In the last equation, we have substituted the condition for the saddle point, given above Eq.~\eqref{saddlept2}.

Unlike Higgs mass, its lifetime cannot be estimated exactly. One source of lifetime to the Higgs boson is the interaction term, i.e. quartic term in the Lagrangian. In this spirit, the leading term in the inverse lifetime ($\tau$) of the Higgs mass is proportional to the coefficient of the $\delta\phi^4$, which can be calculated by setting $\delta\phi=0$ in $\partial^4 \mathcal{L}_{\rm CSGL}[|\phi_0|+\delta \phi]/\partial \delta \phi^4$ which leads to 
{
\begin{eqnarray}
\frac{1}{\tau}&\propto& 24 \beta|\phi_0|+\lambda^4\gamma e^{-\lambda|\phi_0|},\nonumber\\
&=& 24\beta+2\lambda^3\phi_0\left(\alpha+4\beta\phi_0^3\right).
\end{eqnarray}
}
Here one should take the absolute value of the right-hand side.


\begin{thebibliography}{10}
\bibitem{DFT1}
	\bibinfo{author}{ G. Kresse and J. Furthm\"uller}
	\newblock \emph{\bibinfo{journal} {Phys. Rev. Lett.}}
	\textbf{\bibinfo{volume}{54}}, \bibinfo{pages}{11169} (\bibinfo{year}{1996})

	\bibitem{DFT2}
	\bibinfo{author}{G. Kresse and D. Joubert}
	\newblock \emph{\bibinfo{journal} {Phys. Rev. B}}
	\textbf{\bibinfo{volume}{59}}, \bibinfo{pages}{1758} (\bibinfo{year}{1999})

	\bibitem{LMTODFT}
	\bibinfo{author}{T. Saha-Dasgupta and R. Valen,}
	\newblock \emph{\bibinfo{journal} {EPL}}
	\textbf{\bibinfo{volume}{60}}, \bibinfo{pages}{309} (\bibinfo{year}{2002})

 \bibitem{CSGL_Zhang} S. C. Chang, T. H. Hansson, S. Kivelson, Phys. Rev. Letts. {\bf 62}, 82 (1989); S. C. Chang, Int. J. Mod. Phys. {\bf 6}, 25-58 (1992).

 \bibitem{CSGL_rest} S. M. Girvin, A. H. MacDonald, P. M. Platzman, Phys. Rev. Lett.  {\bf 54}, 581 (1985);  Phys. Rev. B {\bf 33}, 2481 (1986); S. M. Girvin, A. H. MacDonald, Phys. Rev. Lett. {\bf 58}, 1252 (1987); N. Read, Phys. Rev. Letts. {\bf 62}, 86 (1988).

\bibitem{zhang_prb}
	\bibinfo{author}{ X-L. Qi, T. L. Hughes and S-C. Zhang}
	\newblock \emph{\bibinfo{journal} {Phys. Rev. B}}
	\textbf{\bibinfo{volume}{78}}, \bibinfo{pages}{195424} (\bibinfo{year}{2008})
	
	\bibitem{zhang_nat}
	\bibinfo{author}{R. Li, J. Wang, X-L. Qi, S-C. Zhang}
	\newblock \emph{\bibinfo{journal} {Nat. Phys.}}
	\textbf{\bibinfo{volume}{6}}, \bibinfo{pages}{284–288} (\bibinfo{year}{2010})	
	
       \bibitem{JMoore}R. S. K. Mong, A. M. Essin, and J. E. Moore, Phys. Rev. B {\bf 81}, 245209 (2010); M. M. Vazifeh and M. Franz, Phys. Rev. B {\bf 82}, 233103 (2010).

\bibitem{axionJPSJ}A. Sekine, K. Nomura, J. Phys. Soc. Jpn. {\bf 83}, 104709 (2014).

\end{thebibliography}

\end{document}