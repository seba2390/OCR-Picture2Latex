%\documentclass[fleqn,10pt]{wlscirepsi}
%\usepackage[utf8]{inputenc}
%\usepackage[T1]{fontenc}
%\title{Differential imprints of distinct ENSO flavors in global patterns of very low and high seasonal precipitation --- Supplementary Information}

%\author[1,*]{Marc Wiedermann}
%\author[1]{Jonathan F. Siegmund}
%\author[2,3]{Jonathan F. Donges}
%\author[1,4]{J\"urgen Kurths}
%\author[1,5]{Reik V. Donner}

%\affil[]{Affiliation, department, city, postcode, country}
%\affil[1]{Potsdam Institute for Climate Impact Research, Complexity Science,
%  14473 Potsdam, Germany}
%\affil[2]{Potsdam Institute for Climate Impact Research, Earth System Analysis,
%  14473 Potsdam, Germany}
%\affil[3]{Stockholm Resilience
%  Centre, Stockholm University, Kr\"aftriket 2B, 114 19 Stockholm, Sweden}
%\affil[4]{Humboldt University of Berlin,  Department of Physics, 12489 Berlin,
%  Germany}
%\affil[3]{University of Potsdam, Institute of Earth and Environmental Science, 
%  14476 Potsdam-Golm, Germany} 
%\affil[5]{Magdeburg-Stendal University of Applied Sciences, Department of
%  Water, Environment, Construction and Safety, 39114 Magdeburg, Germany}
 
%\affil[*]{marcwie@pik-potsdam.de}

%\begin{abstract}
%\end{abstract}
\documentclass[utf8]{frontiers_suppmat} % for all articles
\usepackage{url,hyperref,lineno,microtype}
\usepackage[onehalfspacing]{setspace}

\begin{document}

%\flushbottom
%\maketitle
\onecolumn
\firstpage{1}

\title[Supplementary Material]{{\helveticaitalic{Supplementary Material}}}


\maketitle

\section{Event coincidence rates for different thresholds of strong and weak seasonal precipitation}
 
To demonstrate the robustness of our results, we repeat the analysis presented in the main manuscript for seasonal precipitation sums falling below/exceeding a threshold defined by the empirical 30/70\% or 10/90\% quantiles, respectively. The corresponding significant event coincidence rates between EP/CP El Ni\~no/La Ni\~na periods and outstanding seasonal precipitation sums above (below) the 90th (10th) percentile are shown in Fig.~\ref{fig:eca_el_nino90} and Fig.~\ref{fig:eca_la_nina90}. Figure~\ref{fig:eca_el_nino70} and~\ref{fig:eca_la_nina70} show the same for strong and weak precipitation defined as values above (below) the 70th (30th) percentile and the respective ENSO periods. The arrangements of the panels corresponds to Fig.\ 2 and\ 3 in the main manuscript. 

We observe that changing the percentile threshold such that only the seasons with the 10\% highest or lowest precipitation sums are considered as events yields an expected reduction in spatially coherent coincidences for both El Ni\~no (Fig.~\ref{fig:eca_el_nino90}) and La Ni\~na periods (Fig.~\ref{fig:eca_la_nina90}). However, the dominant large scale patterns, such as reduced rainfall over Indonesia and the Philippines in SON during EP El Ni\~no phases or strong precipitation over North Australia in SON during EP (and less pronounced also for CP) La Ni\~na phases, persist even for such a stricter definition of precipitation events. Generally, we observe that the patterns displayed in Fig.~\ref{fig:eca_el_nino90} and Fig.~\ref{fig:eca_la_nina90} are also present in Fig.\ 2 and\ 3 of the main manuscript.

Similarly, we observe significant event coincidence rates with the 30\% highest and lowest seasonal precipitation sums being considered as events, Fig.~\ref{fig:eca_el_nino70} and~\ref{fig:eca_la_nina70}. We observe similar patterns as displayed in Fig.\ 2 and\ 3 of the main manuscript, but note that the event coincidence rates (especially for the CP phases of El Ni\~no and La Ni\~na) show a tendency to increase. This is because counting less strong or weak signals leads to an increase in the number of precipitation events as compared to the stricter percentile thresholds chosen in the main manuscript. Since the number of ENSO periods is the same for every choice of precipitation threshold, the event coincidence rate is likely to increase with more precipitation events, thus yielding the increased numbers in Fig.~\ref{fig:eca_el_nino70} and Fig.~\ref{fig:eca_la_nina70}. Generally, we find that the spatial patterns which were observed in Fig.\ 2 and Fig.\ 3 of the main manuscript persist also for less rigid definitions of very high and very low seasonal precipitation. 

In summary, our results generally vary smoothly with the actual choice of percentile thresholds above and below which seasons are considered as an event according to their respective precipitation sum, and we therefore consider the analysis presented in the main manuscript to be sufficiently robust against its actual choice. 

\section{Figures}
\begin{figure*}[t]
  \centering
  \includegraphics[width=\linewidth]{figs/en_trigger_coincidences_90.pdf}
  \caption{Same as Fig.\ 3 of the main manuscript but for strong (weak) precipitation
    events defined as values that exceed (fall below) the 90th (10th) percentile of each
    individual time series.}    
  \label{fig:eca_el_nino90}
\end{figure*}

\begin{figure*}[t]
  \centering
  \includegraphics[width=\linewidth]{figs/ln_trigger_coincidences_90.pdf}
  \caption{Same as Fig.~\ref{fig:eca_la_nina90} for La Ni\~na periods.}
  \label{fig:eca_la_nina90}
\end{figure*}

\begin{figure*}[t]
  \centering
  \includegraphics[width=\linewidth]{figs/en_trigger_coincidences_70.pdf}
  \caption{Same as Fig.\ 3 of the main manuscript but for strong (weak) precipitation
    events defined as values that exceed (fall below) the 70th (30th) percentile of each
    individual time series.}    
  \label{fig:eca_el_nino70}
\end{figure*}

\begin{figure*}[t]
  \centering
  \includegraphics[width=\linewidth]{figs/ln_trigger_coincidences_70.pdf}
  \caption{Same as Fig.~\ref{fig:eca_el_nino70} for La Ni\~na periods.}
  \label{fig:eca_la_nina70}
\end{figure*}

\begin{figure*}[t]
  \centering
  \includegraphics[width=.65\linewidth]{figs/valid_stations.pdf}
  \caption{Valid grid cells (yellow) in the GPCC dataset for SON (a), DJF (b) and MAM (c) periods. Grid cells are considered valid if at least one measurement station is present for 95\% of the study period and the average seasonal precipitation sum exceeds 3~cm (see Sec. 2.1 of the main manuscript for details).
  }
\end{figure*}

\end{document}
