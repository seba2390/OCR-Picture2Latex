
 \documentclass[journal,12pt,draftclsnofoot, onecolumn]{IEEEtran}
\usepackage{ifpdf}
\usepackage{cite}
\usepackage[pdftex]{graphicx}
\usepackage{amsmath}
\usepackage{algorithmic}
\usepackage{array}
\usepackage{subfigure}
%\usepackage{mathtools}
\usepackage{caption}
\usepackage[caption=false,font=footnotesize]{subfig}
\usepackage{fixltx2e}
\usepackage{stfloats}
\usepackage{url}
\usepackage{amsthm}
\usepackage{amsmath}
\usepackage{amsfonts}
\usepackage{amssymb}
\usepackage[T1]{fontenc} % optional
\usepackage{amsmath}
\usepackage[cmintegrals]{newtxmath}
\usepackage{bm} % optional
\usepackage[all]{xy}
\usepackage{enumitem}
\usepackage{float}
\usepackage{amsmath}
\interdisplaylinepenalty=2500
\usepackage[usenames, dvipsnames]{color}
\usepackage{ifpdf}
\usepackage{cite}
\usepackage[pdftex]{graphicx}
\usepackage{amsmath}
\usepackage{algorithmic}
\usepackage{array}
\usepackage{mathtools}
\DeclarePairedDelimiter{\abs}{\lvert}{\rvert}

\usepackage{caption}
\usepackage[caption=false,font=footnotesize]{subfig}
\usepackage{fixltx2e}
\usepackage{stfloats}
\usepackage{url}
\usepackage{mathtools}
\usepackage{amsthm}
\usepackage{amsmath}
\usepackage{amsfonts}
\usepackage{amssymb}
\usepackage[T1]{fontenc} % optional
\usepackage{amsmath}
\usepackage[cmintegrals]{newtxmath}
\usepackage{bm} % optional
\usepackage[all]{xy}
\usepackage{enumitem}
\usepackage{float}
\usepackage{amsmath}
\interdisplaylinepenalty=2500
%\usepackage[usenames, dvipsnames]{color}
\usepackage[dvipsnames]{xcolor}


\theoremstyle{definition}
\newtheorem{prop}{Proposition}
\newtheorem{thm}{Theorem}
\newtheorem{cor}{Corollary}
\newtheorem{fact}{Fact}
\newtheorem{remark}{Remark}
\newtheorem{example}{Example}
\newtheorem{lem}{Lemma}
\newtheorem{con}{Conjecture}
\newtheorem{define}{Definition}
\newcommand{\cov}{\textrm{Cov}}
\newcommand{\var}{\textrm{Var}}
\newcommand{\sol}{\textit{Solution:} }
\newcommand{\hs}{\hspace{5pt}}
\newcommand{\hsa}{\hspace{10pt}}
\newcommand{\hsb}{\hspace{20pt}}
\newcommand{\no}{\nonumber}

\newcommand{\amir}[1]{\textcolor{red}{{\bf AMIR: #1}}}
\newcommand{\hoosein}[1]{\textcolor{blue}{{\bf Hossein: #1}}}
%\newcommand{\amir}[1]{\textcolor{red}{{}}}


% correct bad hyphenation here
\hyphenation{op-tical net-works semi-conduc-tor}





\begin{document}
%
\title{Matching Anonymized and Obfuscated\\ Time Series to Users' Profiles}

\author{Nazanin~Takbiri~\IEEEmembership{Student Member,~IEEE,}
        Amir~Houmansadr~\IEEEmembership{Member,~IEEE,}
        Dennis~Goeckel~\IEEEmembership{Fellow,~IEEE,}
        Hossein~Pishro-Nik~\IEEEmembership{Member,~IEEE}
        %\thanks{Manuscript received April 19, 2005; revised August 26, 2015.}
        \thanks{N. Takbiri is with the Department
        of Electrical and Computer Engineering, University of Massachusetts, Amherst,
        MA, 01003 USA e-mail: (ntakbiri@umass.edu).}% <-this % stops a space
        \thanks{A. Houmansadr is with the College of Information and Computer Sciences, University of Massachusetts, Amherst,
        MA, 01003 USA e-mail:(amir@cs.umass.edu)}
        \thanks{H. Pishro-Nik and D. Goeckel are with the Department
        of Electrical and Computer Engineering, University of Massachusetts, Amherst,
        MA, 01003 USA e-mail:(pishro@engin.umass.edu)}
        \thanks{This work was supported by National Science Foundation under grants CCF--0844725, CCF--1421957 and CNS1739462. 
        
        This work was presented in part in IEEE International Symposium on Information Theory (ISIT 2017) \cite{nazanin_ISIT2017}.}}

%\markboth{IEEE TRANSACTIONS ON WIRELESS COMMUNICATIONS}
%\IEEEpubid{0000--0000/00\$00.00~\copyright~2015 IEEE}
\maketitle

\begin{abstract}
%Many popular applications use traces of user data
%to offer various services to  their users; example applications include
 %driver-assistance systems and smart home services.
%However, revealing user information to such applications
%puts users' privacy at stake, as adversaries can
%infer  sensitive
%private information about the users  such as their behaviors, interests, and locations.
%Recent research shows that adversaries can  compromise users' privacy when they use such applications even
%when the traces of
%users' information are protected by mechanisms like anonymization and obfuscation.

Many popular applications use traces of user data to offer various services to their users.  However, even if user data is anonymized and obfuscated, a user's privacy can be compromised through the use of statistical matching techniques that match a user trace to prior user behavior. In this work, we derive the theoretical bounds on the privacy of users in such a scenario. We build on our recent study in the area of location privacy,
in which we introduced formal notions of location privacy for anonymization-based location privacy-protection mechanisms.
Here we derive the fundamental limits of user privacy when both anonymization and obfuscation-based protection mechanisms are applied to users' time series of data.
%    A recent effort towards a fundamental analysis of this risk introduced ``perfect location privacy'' for anonymization-based location-based services (LBS).
%    Here we generalize the application context and, more importantly, consider the fundamental limits of user privacy when anonymization and obfuscation privacy-preserving techniques are applied together.
We investigate the impact of such mechanisms on the trade-off between privacy protection and user utility. We first study achievability results for the case where the time-series of users are governed by an i.i.d.\ process. The converse results are proved both for the i.i.d.\ case as well as the more general Markov chain model. We demonstrate that as the number of users in the network grows, the obfuscation-anonymization plane can be divided into two regions: in the first region, all users have perfect privacy; and, in the second region, no user has privacy.%~\cite{tifs2016,KeConferance2017}.
\end{abstract}

\begin{IEEEkeywords}
Anonymization, Obfuscation, Information theoretic privacy, Privacy-Protection Mechanism (PPM), User-Data Driven Services (UDD).
\end{IEEEkeywords}

%\IEEEpeerreviewmaketitle

\section{Introduction}  \label{sec:introduction}

\newcommand\inexpIntro[3]{#1?(#2,#3).}
\newcommand\rinexpIntro[3]{*#1?(#2,#3).}
\newcommand\outexpIntro[3]{#1!(#2,#3).}
\newcommand\outatomIntro[3]{#1!(#2,#3)}

We propose a fully automated method for proving termination of \(\pi\)-calculus processes.
Although there have been a lot of studies on termination analysis for the \(\pi\)-calculus
and related calculi~\cite{Deng06IC,Demangeon07,SangiorgiTermination,KobayashiHybrid,Yoshida04IC,DBLP:journals/jlp/DemangeonHS10,Venet98SAS}, most of them have been rather theoretical,
and there have been surprisingly little efforts in developing  fully automated termination
verification methods and tools based on them. To our knowledge,
Kobayashi's \typical{}~\cite{TyPiCal,KobayashiHybrid} is the only exception that
can prove termination of \(\pi\)-calculus processes (extended with natural numbers)
fully automatically, but its termination analysis is quite limited (see Section~\ref{sec:relatedwork}).

Our method is based on a reduction to termination analysis for sequential programs:
we translate a \(\pi\)-calculus process \(P\) to a sequential program \(S_P\), so that
if \(S_P\) is terminating, so is \(P\). The reduction allows us to use
powerful, mature methods and tools
for termination analysis of sequential programs~\cite{heizmann2016ultimate,freqterm,DBLP:conf/lics/PodelskiR04,Kuwahara2014Termination,DBLP:journals/cacm/CookPR11}.

The idea of the translation is to convert a chain of communications on replicated input
channels to a chain of recursive function calls of the target sequential program.
Let us consider the following Fibonacci process:
\begin{align*}
    & \rinexpIntro{\fib}{n}{r}
        \ifexp{n<2}{ \soutatom{r}{1} \\ &\quad}
                   { \nuexp{s_1} \nuexp{s_2} (\outatomIntro{\fib}{n-1}{s_1} \PAR \outatomIntro{\fib}{n-2}{s_2} \PAR \sinexp{s_1}{x}\sinexp{s_2}{y}\soutatom{r}{x+y}) \\}
    & \PAR \outatomIntro{\fib}{m}{r}
\end{align*}
Here, the process
$\rinexpIntro{\fib}{n}{r} \ldots$ is a function server that computes the \(n\)-th Fibonacci number
in parallel and returns the result to \(r\),
and $\outatom{\fib}{m}{r}$ sends a request for computing the \(m\)-th Fibonacci number;
those who are not familiar with the syntax of the \(\pi\)-calculus may wish to consult
Section~\ref{sec:targetlanguage} first.
To prove that the process above is terminating for any integer \(m\),
it suffices to show that there is no infinite chain of communications on $\fib$:
\[
    \fib(m,r) \to \fib(m_1,r_1) \to \fib(m_2,r_2) \to \cdots.
\]
We convert the process above to the following program:\footnote{The actual translation
  given later is a little more complex.}
\begin{verbatim}
 let rec fib(n) = if n<2 then () else (fib(n-1) [] fib(n-2)) in
 fib(m)
\end{verbatim}
Here, \texttt{[]} represents the non-deterministic choice.
Note that, although the calculation of Fibonacci numbers is not preserved,
for each chain of communications on \texttt{fib}, there is a corresponding
sequence of recursive calls:
\[
\mathtt{fib}(m) \to \mathtt{fib}(m_1) \to \mathtt{fib}(m_2) \to \cdots.
\]
Thus, the termination of the sequential program above implies the termination of
the original process.
As shown in the example above, (i) each communication on a replicated input channel
is converted to a function call, (ii) each communication on a non-replicated input
channel is just removed (or, in the actual translation, replaced by a call of
a trivial function defined by \(f(\seq{x})=(\,)\)), and (iii) parallel composition
is replaced by a non-deterministic choice.
We formalize the translation outlined above and prove its correctness.

The basic translation sketched above sometimes loses too much information.
For example, consider the following process:
\begin{align*}
    & \rinexpIntro{\pre}{n}{r} \soutatom{r}{n-1} \\
    & \PAR \rinexpIntro{f}{n}{r} \ifexp{n<0}{ \soutatom{r}{1} }
                                       { \nuexp{s} (\outatomIntro{\pre}{n}{s} \PAR \sinexp{s}{x}\outatomIntro{f}{x}{r}) } \\
    & \PAR \outatomIntro{f}{m}{r}
\end{align*}
The translation sketched above would yield:
\begin{verbatim}
  let pred(n) = n-1 in
  let rec f(n) = if n<0 then () else (pred(n) [] f(*)) in
  f(m)
\end{verbatim}
Here, \texttt{*} represents a non-deterministic integer: since we have removed
the input $\sinatom{s}{x}$, we do not have information about the value of \( x \).
As a result, the sequential program above is non-terminating, although the original
process is terminating.
To remedy this problem, we also refine the basic translation above by using a refinement
type system for the \(\pi\)-calculus. Using the refinement type system,
we can infer that the value of \(x\) in the original process is less than \(n\),
so that we can refine the definition of \texttt{f} to:
\begin{verbatim}
 let rec f(n) = ... else (pred(n) [] let x=* in assume(x<n);f(x))
\end{verbatim}
The target program is now terminating, from which
we can deduce that the original process is also terminating.
We have implemented an automated tool based on the refined translation above.

The contributions of this paper are summarized as follows.
\begin{itemize}
\item The formalization of the basic translation from the \(\pi\)-calculus
  (extended with integers) to sequential programs, and a proof of its correctness.
\item The formalization of a refined translation based on a refinement type system.
\item An implementation of the refined translation, including automated refinement type
  inference based on CHC solving, and experiments to evaluate the effectiveness of
  our method.
\end{itemize}

The rest of this paper is structured as follows.
Section~\ref{sec:targetlanguage} introduces the source and target languages
of our translation.
Section~\ref{sec:approach} 
formalizes the basic translation, and proves its correctness.
Section~\ref{sec:refinement} refines the basic translation by using a refinement type system.
Section~\ref{sec:implementation} reports an implementation and experiments.
Section~\ref{sec:relatedwork} discusses related work,
and Section~\ref{sec:conclusion} concludes the paper.

%%%%%%%%%%%%%%%%%%%%%%%%%%%%%%%%%%%%%%%%%%%%%
%\subsection{Multitask Learning}

MTL has been succesfully used in different domains, including CV \cite{UberNet,MaskRCNN}. Some challenges appear when applying it \cite{Caruana}: \textit{learning speed} differences between tasks and  deciding \textit{what to share} according to the \textit{relatedness} between tasks in the multitask architecture \cite{Stitch, AdaptiveFeatureSharing}.

\subsection{Semantic Segmentation}

Semantic segmentation aims at partitioning parts of images belonging to the same semantic class, typically via pixel-wise classification. Fully convolutional networks (FCN) \cite{FCN} have improved both accuracy and speed for dense prediction problems by using only convolutional layers. Upsampling layers allow a segmentation output size equal to the input and skip connections add finer details. Other approaches add post-processing steps \cite{DeeplabCRF}, learnable \textit{deconvolution} layers \cite{ Deconv} or global context \cite{ParseNet}.

\subsection{Object Detection}

Object detection aims at finding in an image all instances of objects and classifying them in a number of classes. Faster R-CNN \cite{FasterRCNN} was the first to give close to real-time performance. YOLO \cite{YOLO} avoids the generation of region proposals for increased speed. SSD \cite{SSD} avoids fully-connected layers for speed and takes features at different levels for improved accuracy. 

%\cite{SpeedAccuracy} reviews the speed vs. accuracy trade-off for different object detectors.
%%%%%%%%%%%%%%%%%%%%%%%%%%%%%%%%%%%%%%%%%%%%%%%%%%%%
\section{Framework}
\label{sec:framework}
In this paper, we adopt a similar framework to that employed in~\cite{tifs2016,ciss2017}.  The general set up is provided here, and the refinement to the precise models for this paper will be presented in the following sections.  We assume a system with $n$ users with $X_u(k)$ denoting a sample of the data of user $u$ at time $k$, which we would like to protect from an interested adversary $\mathcal{A}$. We consider a strong adversary $\mathcal{A}$ that has complete statistical knowledge of the users' data patterns based on the previous observations or other resources. In order to secure data privacy of users, both obfuscation and anonymization techniques are used as shown in Figure \ref{fig:xyz}. In Figure \ref{fig:xyz}, $Z_u(k)$ shows the (reported) sample of the data of user $u$ at time $k$ after applying obfuscation, and $Y_u(k)$ shows the (reported) sample of the data of user $u$ at time $k$ after applying anonymization. The adversary observes only $Y_u(k)$, $k=1,2,\cdots, m(n)$, where $m(n)$ is the number of observations of each user before the identities are permuted. The adversary then tries to estimate $X_u(k)$ by using those observations.
\begin{figure}[h]
	\centering
	\includegraphics[width = 0.75\linewidth]{fig/xyz}
	\caption{Applying obfuscation and anonymization techniques to users' data samples.}
	\label{fig:xyz}
\end{figure}

Let $\textbf{X}_u$ be the $m(n) \times 1$ vector containing the sample of the data of user $u$, and $\textbf{X}$ be the $m(n) \times n$ matrix with $u^{th}$ column equal to $\textbf{X}_u$;
\[\textbf{X}_u = \begin{bmatrix}
X_u(1) \\ X_u(2) \\ \vdots \\X_u(m) \end{bmatrix} , \ \ \  \textbf{X} =\left[\textbf{X}_{1}, \textbf{X}_{2}, \cdots,  \textbf{X}_{n}\right].
\]


\textit{Data Samples Model:}
We assume there are $r \geq 2$ possible values ($0,1, \cdots, r-1$) for each sample of the users' data. In the first part of the paper (perfect privacy analysis), we assume an i.i.d.\ model as motivated in Section \ref{intro}. In the second part of the paper (converse results: no privacy region), the users' datasets are governed by irreducible and aperiodic Markov chains. At any time, $X_u(k)$ is equal to a value in $\left\{0,1, \cdots, r-1 \right\}$ according to a user-specific probability distribution. The collection of user distributions, which satisfy some mild regularity conditions discussed below, is known to the adversary $\mathcal{A}$, and he/she employs such to distinguish different users based on statistical matching of those user distributions to traces of user activity of length $m(n)$.%In any case, any dependency can only favor the adversary, so our results provide lower bounds on the achievable privacy in those settings, which would be desirable if the dependency is poorly understood.  Second, understanding the i.i.d.\ case can be considered the first step toward understanding the more complicated case where there is a dependency, as was done for anonymization-only LPPMs in \cite{tifs2016} to consider perfect privacy when users' data pattern are governed by Markov chains. In the second part of the paper, in addition to the above model, we consider the case where users' data patterns are modeled by Markov chains as discussed in Section \ref{subsec:markov}.%





\textit{Obfuscation Model:} The first step in obtaining privacy is to apply the obfuscation operation in order to perturb the users' data samples. In this paper, we assume that each user has only limited knowledge of the characteristics of the overall population and thus we employ a simple distributed method in which the samples of the data of each user are reported with error with a certain probability, where that probability itself is generated randomly for each user. In other words, the obfuscated data is obtained by passing the users' data through an $r$-ary symmetric channel with a random error probability. More precisely, let $\textbf{Z}_u$ be the vector which contains
%$m$ number of
the obfuscated versions of user $u$'s data samples, and $\textbf{Z}$ is the collection of $\textbf{Z}_u$ for all users,
\[\textbf{Z}_u = \begin{bmatrix}
Z_u(1) \\ Z_u(2) \\ \vdots \\Z_u(m) \end{bmatrix} , \ \ \  \textbf{Z} =\left[ \textbf{Z}_{1}, \textbf{Z}_{2}, \cdots,  \textbf{Z}_{n}\right].
\]
To create a noisy version of data samples, for each user $u$, we independently generate a random variable $R_u$ that is uniformly distributed between $0$ and $a_n$, where $a_n \in (0,1]$. The value of $R_u$ gives the probability that a user's data sample is changed to a different data sample by obfuscation, and $a_n$ is termed the ``noise level'' of the system. For the case of $r=2$ where there are two states for users' data (state $0$ and state $1$), the obfuscated data is obtained by passing users' data through a Binary Symmetric Channel (BSC) with a small error probability~\cite{randomizedresponse}. Thus, we can write
\[
{Z}_{u}(k)=\begin{cases}
{X}_{u}(k), & \textrm{with probability } 1-R_u.\\
1-{X}_{u}(k),& \textrm{with probability } R_u.
\end{cases}
\]
When $r>2$, for $l \in \{0,1,\cdots, r-1\}$:
\[
P({Z}_{u}(k)=l| X_{u}(k)=i) =\begin{cases}
1-R_u, & \textrm{for } l=i.\\
\frac{R_u}{r-1}, & \textrm{for } l \neq i.
\end{cases}
\]
Note that the effect of the obfuscation is to alter the probability distribution function of each user across the $r$ possibilities in a way that is unknown to the adversary, since it is independent of all past activity of the user, and hence the obfuscation inhibits user identification. For each user, $R_u$ is generated once and is kept constant for the collection of samples of length $m(n)$, thus, providing a very low-weight obfuscation algorithm. We will discuss the extension to the case where $R_u$ is regenerated independently over time in Section \ref{sec:perfect-MC}. There, we will also provide a discussion about obfuscation using continuous noise distributions (e.g., Gaussian noise).

\textit{Anonymization Model:} Anonymization is modeled by a random permutation $\Pi$ on the set of $n$ users. The user $u$ is assigned the pseudonym $\Pi(u)$. $\textbf{Y}$ is the anonymized version of $\textbf{Z}$; thus,
\begin{align}
\no \textbf{Y} &=\textrm{Perm}\left(\textbf{Z}_{1}, \textbf{Z}_{2}, \cdots,  \textbf{Z}_{n}; \Pi \right) \\
\nonumber &=\left[ \textbf{Z}_{\Pi^{-1}(1)}, \textbf{Z}_{\Pi^{-1}(2)}, \cdots,  \textbf{Z}_{\Pi^{-1}(n)}\right ] \\
\nonumber &=\left[ \textbf{Y}_{1}, \textbf{Y}_{2}, \cdots, \textbf{Y}_{n}\right], \ \
\end{align}
where $\textrm{Perm}( \ . \ , \Pi)$ is permutation operation with permutation function $\Pi$. As a result, $\textbf{Y}_{u} = \textbf{Z}_{\Pi^{-1}(u)}$ and $\textbf{Y}_{\Pi(u)} = \textbf{Z}_{u}$.

\textit{Adversary Model:} We protect against the strongest reasonable adversary. Through past observations or some other sources, the adversary is assumed to have complete statistical knowledge of the users' patterns; in other words, he/she knows the probability distribution for each user on the set of data samples $\{0,1,\ldots,r-1\}$. As discussed in the model for the data samples, the parameters $\textbf{p}_u$, $u=1, 2, \cdots, n$ are drawn independently from a continuous density function, $f_\textbf{P}(\textbf{p}_u)$, which has support on a subset of a defined hypercube. The density $f_\textbf{P}(\textbf{p}_u)$ might be unknown to the adversary, as all that is assumed here is that such a density exists, and it will be evident from our results that knowing or not knowing $f_\textbf{P}(\textbf{p}_u)$ does not change the results asymptotically. Specifically, from the results of Section \ref{perfectsec}, we conclude that user $u$ has perfect privacy even if the adversary knows $f_\textbf{P}(\textbf{p}_u)$. In addition, in Section \ref{converse}, it is shown that the adversary can recover the true data of user $u$ at time $k$ without using the specific density function of $f_\textbf{P}(\textbf{p}_u)$, and as result, users have no privacy even if the adversary does not know $f_\textbf{P}(\textbf{p}_u)$.

The adversary also knows the value of $a_n$ as it is a design parameter. However, the adversary does not know the realization of the random permutation $\Pi$ or the realizations of the random variables $R_u$, as these are independent of the past behavior of the users. It is critical to note that we assume the adversary does not have any auxiliary information or side information about users' data.


In \cite{tifs2016}, perfect privacy is defined as follows:
\begin{define}
User $u$ has \emph{perfect privacy} at time $k$, if and only if
\begin{align}%\label{}
\no  \forall k\in \mathbb{N}, \ \ \ \lim\limits_{n\rightarrow \infty} I \left(X_u(k);{\textbf{Y}}\right) =0,
\end{align}
where $I(X;Y)$ denotes the mutual information between random variables (vectors) $X$ and $Y$.
\end{define}

\noindent In this paper, we also consider the situation in which there is no privacy. 

\begin{define}
For an algorithm for the adversary that tries to estimate the actual sample of data of user $u$ at time $k$, define
\[P_e(u,k)\triangleq P\left(\widetilde{X_u(k)} \neq X_u(k)\right),\]
where $X_u(k)$ is the actual sample of the data of user $u$ at time $k$, $\widetilde{X_u(k)}$ is the adversary's estimated sample of the data of user $u$ at time $k$, and $P_e(u,k)$ is the error probability. Now, define ${\cal E}$ as the set of all possible adversary's estimators; then, user $u$ has \emph{no privacy} at time $k$, if and only if for large enough $n$,
\[
\forall k\in \mathbb{N}, \ \ \ P^{*}_e(u,k)\triangleq \inf_{\cal E} {P\left(\widetilde{X_u(k)} \neq X_u(k)\right)} \rightarrow 0.
\]
Hence, a user has no privacy if there exists an algorithm for the adversary to estimate $X_u(k)$ with diminishing error probability as $n$ goes to infinity.
\end{define}

\textbf{\textit{Discussion:}} Both of the privacy definitions given above (perfect privacy and no privacy) are asymptotic in the number of users $(n \to \infty)$, which allows us to find clean analytical results for the fundamental limits. Moreover, in many IoT applications, such as ride sharing and dining recommendation applications, the number of users is large. %And finally, we show in the simulation section that our results provide good predictions for the behavior observed when there is a finite number of users in the system. 

\textbf{\textit{Notation:}} Note that the sample of data of user $u$ at time $k$ after applying obfuscation $\left(Z_u(k)\right)$ and the sample of data of user $u$ at time $k$ after applying anonymization $\left(Y_u(k)\right)$ depend on the number of users in the network $(n)$, while the actual sample of data of user $u$ at time $k$ is independent of the number of users $(n)$.  Despite the dependency in the former cases, we omit this subscript $(n)$ on $\left(Z_u^{(n)}(k), Y_u^{(n)}(k) \right)$ to avoid confusion and make the notation consistent.

\textbf{\textit{Notation:}} Throughout the paper, $X_n \xrightarrow{d} X$ denotes convergence in distribution. Also, We use $P\left(X=x\bigg{|} Y=y\right)$ for the conditional probability of $X=x$ given $Y=y$. When we write $P\left(X=x \bigg{|}Y\right)$, we are referring to a random variable that is defined as a function of $Y$.


\section{Perfect Privacy Analysis: I.I.D.\ Case}
\label{perfectsec}

\subsection{Two-States Model}

We first consider the two-states case $(r=2)$ which captures the salient aspects of the problem. For the two-states case, the sample of the data of user $u$ at any time is a Bernoulli random variable with parameter $p_u$, which is the probability of user $u$ having data sample $1$. Thus,
\[X_u(k) \sim Bernoulli \left(p_u\right).\]
Per Section \ref{sec:framework}, the parameters $p_u$, $u=1, 2, \cdots, n$ are drawn independently from a continuous density function, $f_P(p_u)$, on the $(0,1)$ interval. We assume there are $\delta_1, \delta_2>0$ such that:\footnote{The condition $\delta_1<f_P(p_u) <\delta_2$ is not actually necessary for the results and can be relaxed; however, we keep it here to avoid unnecessary technicalities.}
\begin{equation}
\no\begin{cases}
    \delta_1<f_P(p_u) <\delta_2, & p_u \in (0,1).\\
    f_P(p_u)=0, &  p _u\notin (0,1).
\end{cases}
\end{equation}

The adversary knows the values of $p_u$, $u=1, 2, \cdots, n$ and uses this knowledge to identify users. We will use capital letters (i.e., $P_u$) when we are referring to the random variable, and use lower case (i.e., $p_u$) to refer to the realization of $P_u$.

In addition, since the user data $\left(X_u(k)\right)$ are i.i.d.\ and have a Bernoulli distribution, the obfuscated data $\left(Z_u(k)\right)$ are also i.i.d.\ with  a Bernoulli distribution. Specifically,
\[Z_u(k) \sim Bernoulli\left(Q_u\right),\]
where
\begin{align}
\no {Q}_u &=P_u(1-R_u)+(1-P_u)R_u  \\
\nonumber &= P_u+\left(1-2P_u\right)R_u,\ \
\end{align}
and recall that $R_u$ is the probability that user $u$'s data sample is altered at any time. For convenience, define a vector where element $Q_u$ is the probability that an obfuscated data sample of user $u$ is equal to one, and
\[\textbf{Q} =\left[{Q}_{1},{Q}_{2}, \cdots,{Q}_{n}\right].\]
Thus, a vector containing the permutation of those probabilities after anonymization is given by:
\begin{align}
\no \textbf{V} &=\textrm{Perm}\left({Q}_{1}, {Q}_{2}, \cdots,  {Q}_{n}; \Pi \right) \\
\nonumber &=\left[ {Q}_{\Pi^{-1}(1)}, {Q}_{\Pi^{-1}(2)}, \cdots, {Q}_{\Pi^{-1}(n)}\right ] \\
\nonumber &=\left[{V}_{1}, {V}_{2}, \cdots, {V}_{n}\right ] ,\ \
\end{align}
where ${V}_{u} = {Q}_{\Pi^{-1}(u)}$ and  ${V}_{\Pi(u)} = {Q}_{u}$. As a result, for $u=1,2,..., n$, the distribution of the data symbols for the user with pseudonym $u$ is given by:
\[ Y_u(k) \sim Bernoulli \left(V_u\right) \sim Bernoulli\left(Q_{\Pi^{-1}(u)}\right)  .\]

The following theorem states that if $a_n$ is significantly larger than $\frac{1}{n}$ in this two-states model, then all users have perfect privacy independent of the value of $m(n)$.
\begin{thm}\label{two_state_thm}
 For the above two-states model, if $\textbf{Z}$ is the obfuscated version of $\textbf{X}$, and $\textbf{Y}$ is the anonymized version of $\textbf{Z}$  as defined above, and
\begin{itemize}
	\item $m=m(n)$ is arbitrary;
	\item $R_u \sim Uniform [0, a_n]$, where $a_n \triangleq c'n^{-\left(1-\beta\right)}$ for any $c'>0$ and $0<\beta<1$;
\end{itemize}
then, user 1 has perfect privacy. That is,
\begin{align}%\label{}
\no  \forall k\in \mathbb{N}, \ \ \ \lim\limits_{n\rightarrow \infty} I \left(X_1(k);{\textbf{Y}}\right) =0.
\end{align}
\end{thm}

The proof of Theorem \ref{two_state_thm} will be provided for the case $0\leq p_1<\frac{1}{2}$, as the proof for the case $\frac{1}{2}\leq p_1\leq1$ is analogous and is thus omitted.
\\

\noindent \textbf{Intuition behind the Proof of Theorem \ref{two_state_thm}:} 

Since $m(n)$ is arbitrary, the adversary is able to estimate very accurately (in the limit, perfectly) the distribution from which each data sequence $\textbf{Y}_u$, $u= 1, 2, \cdots, n$ is drawn; that is, the adversary is able to accurately estimate the probability $V_u$, $u= 1, 2, \cdots, n$. Clearly, if there were no obfuscation for each user $u$, the adversary would then simply look for the $j$ such that $p_j$ is very close to $V_u$ and set $\widetilde{X_j(k)}=Y_u(k)$, resulting in no privacy for any user.

We want to make certain that the adversary obtains no information about $X_1(k)$, the sample of data of user $1$ at time $k$. To do such, we will establish that there are a large number of users whom have a probability $p_u$ that when obfuscated could have resulted in a probability consistent with $p_1$. Consider asking whether another probability $p_2$ is sufficiently close enough to be confused with $p_1$ after obfuscation; in particular, we will look for $p_2$ such that, even if the adversary is given the obfuscated probabilities $V_{\Pi(1)}$ and $V_{\Pi(2)}$, he/she cannot associate these probabilities with $p_1$ and $p_2$. This requires that the distributions $Q_{1}$ and $Q_{2}$ of the obfuscated data of user $1$ and user $2$ have significant overlap; we explore this next.

Recall that $Q_u=P_u+ (1-2P_u)R_u$, and $R_u\sim Uniform [0, a_n]$. Thus, we know $Q_u {|} P_u=p_u$ has a uniform distribution with length $(1-2p_u)a_n$. Specifically,
\[Q_u\bigg{|}P_u=p_u \sim Uniform \left[p_u,p_u+(1-2p_u)a_n\right].\]
Figure \ref{fig:piqi_a} shows the distribution of $Q_u$ given $P_u=p_u$.

\begin{figure}[h]
	\centering
	\includegraphics[width = 0.8\linewidth]{fig/piqi_a}
	\caption{Distribution of $Q_u$ given $P_u=p_u$.}
	\label{fig:piqi_a}
\end{figure}

Consider two cases: In the first case, the support of the distributions $Q_1\big{|} P_1=p_1$ and $Q_2\big{|} P_2=p_2$ are small relative to the difference between $p_1$ and $p_2$ (Figure \ref{fig:case1}); in this case, given the probabilities $V_{\Pi(1)}$ and $V_{\Pi(2)}$ of the anonymized data sequences, the adversary can associate those with $p_1$ and $p_2$ without error. In the second case, the support of the distributions $Q_1\big{|} P_1=p_1$ and $Q_2\big{|} P_2=p_2$ is large relative to the difference between $p_1$ and $p_2$ (Figure \ref{fig:case2}), so it is difficult for the adversary to associate the probabilities $V_{\Pi(1)}$ and $V_{\Pi(2)}$ of the anonymized data sequences with $p_1$ and $p_2$. In particular, if $V_{\Pi(1)}$ and $V_{\Pi(2)}$ fall into the overlap of the support of $Q_1$ and $Q_2$, we will show the adversary can only guess randomly how to de-anonymize the data. Thus, if the ratio of the support of the distributions to $\big{|}p_1-p_2\big{|}$ goes to infinity, the adversary's posterior probability for each user converges to $\frac{1}{2}$, thus, implying no information leakage on the user identities. More generally, if we can guarantee that there will be a large set of users with $p_u$'s very close to $p_1$ compared to the support of $Q_1\big{|} P_1=p_1$, we will be able to obtain perfect privacy as demonstrated rigorously below.

\begin{figure}[h]
	\centering
	\includegraphics[width = 0.8\linewidth]{fig/case1}
	\caption{Case 1: The support of the distributions is small relative to the difference between $p_1$ and $p_2$.}
	\label{fig:case1}
\end{figure}

\begin{figure}[h]
	\centering
	\includegraphics[width = 0.8\linewidth]{fig/case2}
	\caption{Case 2: The support of the distributions is large relative to the difference between $p_1$ and $p_2$.}
	\label{fig:case2}
\end{figure}

Given this intuition, the formal proof proceeds as follows. Given $p_1$, we define a set $J^{(n)}$ of users whose parameter $p_u$ of their data distributions is sufficiently close to $p_1$ (Figure \ref{fig:case2}; case 2), so that it is likely that $Q_1$ and $Q_u$ cannot be readily associated with $p_1$ and $p_u$.

The purpose of Lemmas \ref{lemOnePointFive}, \ref{lem2}, and \ref{lem3} is to show that, from the adversary's perspective, the users in set $J^{(n)}$ are indistinguishable. More specifically, the goal is to show that the obfuscated data corresponding to each of these users could have been generated by any other users in $J^{(n)}$ in an equally likely manner. To show this, Lemma \ref{lemOnePointFive} employs the fact that, if the observed values of $N$ uniformly distributed random variables ($N$ is size of set $J^{(n)}$) are within the intersection of their ranges, it is impossible to infer any information about the matching between the observed values and the distributions. That is, all possible $N!$ matchings are equally likely. Lemmas \ref{lem2} and \ref{lem3} leverage Lemma \ref{lemOnePointFive} to show that even if the adversary is given a set that includes all of the pseudonyms of the users in set $J^{(n)}$  (i.e., $\Pi(J^{(n)})\overset{\Delta}{=} \left\{\Pi^{-1}(u) \in J^{(n)}\right\}$) he/she still will not be able to infer any information about the matching of each specific user in set $J^{(n)}$ and his pseudonym. Then Lemma \ref{lem4} uses the above fact to show that the mutual information between the data set of user $1$ at time $k$ and the observed data sets of the adversary converges to zero for large enough $n$.

%Remember that the adversary does not know the realization of the random permutation $\Pi$ as well as the realizations of random variables  $R_u^{(n)}$. Therefore, the adversary does not know the realizations of random variables ${V}_{u}^{(n)}$ defined above.
%Then by using lemma \ref{lem2}, \ref{lem3}, and \ref{lem4} we try to prove that the the mutual information between data set of user $1$ at time $k$ and the observed data sets by the adversary will  go to zero for large enough $n$.

\noindent \textbf{Proof of Theorem \ref{two_state_thm}:}

\begin{proof}
Note, per Lemma~\ref{lemx} of Appendix \ref{sec:app_a}, it is sufficient to establish the results on a sequence of sets with high probability. That is, we can condition on high-probability events.

Now, define the critical set $J^{(n)}$ with size $N^{(n)}=\big{|}J^{(n)}\big{|}$ for $0\leq p_1<\frac{1}{2}$ as follows:
\[J^{(n)}=
\left\{  u \in \{1, 2, \dots, n\}: p_1 \leq P_u\leq p_1+\epsilon_n;  p_1+\epsilon_n\leq Q_u\leq p_1+(1-2p_1)a_n\right\},
\]
where $\epsilon_n \triangleq \frac{1}{n^{1-\frac{\beta}{2}}}$, $a_n= c'n^{-\left(1-\beta\right)}$ ,and $\beta$ is defined in the statement of Theorem \ref{two_state_thm}.


Note for large enough $n$, if $0\leq p_1<\frac{1}{2}$, we have $0\leq p_u<\frac{1}{2}$. As a result,
\[Q_u\bigg{|}P_u=p_u \sim Uniform \left(p_u,p_u+(1-2p_u)a_n\right).\]
We can prove that with high probability, $1 \in J^{(n)}$ for large enough $n$, as follows. First, Note that
\[Q_1\bigg{|}P_1=p_1 \sim Uniform \left(p_1,p_1+(1-2p_1)a_n\right).\]
Now, according to Figure \ref{fig:piqi_c},
	\begin{align}
	\no P\left(1 \in J^{(n)} \right) &= 1- \frac{\epsilon_n}{ \left(1-2p_1 \right)a_n}\\
	\nonumber &= 1- \frac{1}{ \left(1-2p_1 \right)c'n^{\frac{\beta}{2}}}, \ \
	\end{align}
thus, for any $c'>0$ and large enough $n$,
\begin{align}
	\no P\left(1 \in J^{(n)} \right) \to 1.
\end{align}
\begin{figure}[h]
	\centering
	\includegraphics[width = 0.8\linewidth]{fig/piqi_b}
	\caption{Range of $P_u$ and $Q_u$ for elements of set $J^{(n)}$ and probability density function of $Q_u\bigg{|}P_u=p_u$.}
	\label{fig:piqi_b}
\end{figure}

\begin{figure}[h]
	\centering
	\includegraphics[width = 0.8\linewidth]{fig/piqi_c}
	\caption{Range of $P_u$ and $Q_u$ for elements of set $J^{(n)}$ and probability density function of $Q_1\bigg{|}P_1=p_1$.}
	\label{fig:piqi_c}
\end{figure}




Now in the second step, we define the probability $W_j^{(n)}$ for any $j \in \Pi(J^{(n)})=\{\Pi(u): u \in J^{(n)} \}$ as
\[W_j^{(n)}= P\left(\Pi(1)=j \bigg{|} \textbf{V}, \Pi (J^{(n)})\right).\]
$W_j^{(n)}$ is the conditional probability that $\Pi(1)=j$ after perfectly observing the values of the permuted version of obfuscated probabilities ($\textbf{V}$) and set including all of the pseudonyms of the users in set $J^{(n)}$ $\left(\Pi(J^{(n)})\right)$. Since $\textbf{V}$ and $\Pi (J^{(n)})$ are random, $W_j^{(n)}$ is a random variable. However, we will prove shortly that in fact $W_j^{(n)}=\frac{1}{N^{(n)}}$, for all $j \in \Pi (J^{(n)})$.

Note: Since we are looking from the adversary's point of view, the assumption is that all the values of $P_u$, $u \in \{1,2,\cdots,n\}$ are known, so all of the probabilities are conditioned on the values of $P_1=p_1, P_2=p_2, \cdots, P_n=p_n$. Thus, to be accurate, we should write
\[W_j^{(n)}= P\left(\Pi(1)=j \bigg{|} \textbf{V}, \Pi (J^{(n)}), P_1, P_2, \cdots, P_n\right).\]
Nevertheless, for simplicity of notation, we often omit the conditioning on $P_1, P_2, \cdots, P_n$.

First, we need a lemma from elementary probability.

\begin{lem}
	\label{lemOnePointFive}
	Let $N$ be a positive integer, and let $a_1, a_2, \cdots, a_N$ and $b_1, b_2, \cdots, b_N$ be real numbers such that $a_u \leq b_u$ for all $u$. Assume that $X_1, X_2, \cdots, X_N$ are independent random variables such that
\[X_u \sim Uniform [a_u,b_u]. \]
Let also $\gamma_1, \gamma_2, \cdots, \gamma_N$ be distinct real numbers such that
\[ \gamma_j \in \bigcap_{u=1}^{N} [a_u, b_u] \ \ \textrm{for all }j \in \{1,2,..,N\}.\]
Suppose that we know the event $E$ has occurred, meaning that the observed values of $X_u$'s are equal to the set of $\gamma_j$'s (but with unknown ordering), i.e.,
\[E \ \ \equiv \ \ \{X_1, X_2, \cdots, X_N\}= \{ \gamma_1, \gamma_2, \cdots, \gamma_N \},\] then
\[P\left(X_1=\gamma_j |E\right)=\frac{1}{N}. \]
\end{lem}

\begin{proof}
Lemma \ref{lemOnePointFive} is proved in Appendix \ref{sec:app_b}.
\end{proof}


%We now state a slightly modified version of Lemma \ref{lemOnePointFive} that is suitable for our purpose. The difference here is that $N$  is a random variable rather than a fixed integer (so are $a_i, b_i$).

%\begin{lem}
	%\label{lemOnePointSix}
	%Let $N$ be a positive integer-valued random variable, and let $A_1, A_2, \cdots$ and $B_1, B_2, \cdots$ be a sequence of real-valued random variables such that $A_i \leq B_i$ for all $i$, almost surely. Assume that $X_1, X_2, \cdots$ are independent random variables such that for all $n' \in \mathbb{N}$, given $N=n'$, $A_i=a_i$ and $B_i=b_i$ for all $i$,
%\[X_i \sim Uniform [a_i,b_i]. \]
%Let also $\gamma_1, \gamma_2, \cdots$ be distinct real numbers such that for all $n' \in \mathbb{N}$
%\[ \gamma_j \in \bigcap_{i=1}^{n'} [A_i, B_i] \ \ \textrm{for all }j \in \{1,2,..,n'\} \]
%Let $E$ be the event that the observed values of $X_i$'s is equal to the set of $\gamma_j$'s (but with unknown ordering), i.e.,
%\[E \ \ \equiv \ \ \{X_1, X_2, \cdots, X_N\}= \{ \gamma_1, \gamma_2, \cdots, \gamma_N \}. \]
%Then
%\[P\left(X_1=\gamma_j |E, A_i, B_i, i=1,2\cdots, N \right)=\frac{1}{N}. \]
%\end{lem}
%\begin{proof}
%Lemma \ref{lemOnePointSix} follows directly from Lemma \ref{lemOnePointFive}. In particular, assuming $N=n'$ is observed, and given $A_i=a_i$ and $B_i=b_i$, we can apply Lemma \ref{lemOnePointFive} to write
%\[P\left(X_1=\gamma_j |E,  A_i=a_i, B_i=b_i, i=1,2\cdots, N\right)=\frac{1}{n'}. \]
%Treating $N$, $A_i$, and $B_i$ as random variables,  we can write
%\[P\left(X_1=\gamma_j |E, A_i, B_i, i=1,2\cdots, N \right)=\frac{1}{N}. \]
%\end{proof}

Using the above lemma, we can state our desired result for $W_j^{(n)}$.

\begin{lem}
	\label{lem2}
	For all $j \in \Pi (J^{(n)})$, $W_j^{(n)}=\frac{1}{N^{(n)}}.$
\end{lem}
\begin{proof}
	
%We have
%	\begin{align}
%	\no P\left(\Pi^{(n)}(1)=j \bigg{|} \Pi (J^{(n)})\right)&= \frac{P\left(\Pi^{(n)}(1)=j \bigg{|} \Pi (J^{(n)})\right)}{P\left(\Pi (J^{(n)}\right)} \\
%	\nonumber &=\frac{\left(N^{(n)}-1\right)!}{\left(N^{(n)}\right)!}\\
%		\nonumber &=\frac{1}{N^{(n)}}.\ \
%	\end{align}

We argue that the setting of this lemma is essentially equivalent to the assumptions in Lemma \ref{lemOnePointFive}. First, remember that
\[W_j^{(n)}= P\left(\Pi(1)=j \bigg{|} \textbf{V}, \Pi (J^{(n)})\right).\]

Note that ${Q}_u= P_u+(1-2P_u)R_u$, and since $R_u$ is uniformly distributed, ${Q}_u$ conditioned on $P_u$ is also uniformly distributed in the appropriate intervals. Moreover, since ${V}_{u} = {Q}_{\Pi^{-1}(u)}$, we conclude ${V}_{u}$ is also uniformly distributed. So, looking at the definition of $W_j^{(n)}$, we can say the following: given the values of the uniformly distributed random variables ${Q}_u$, we would like to know which one of the values in  $\textbf{V}$ is the actual value of ${Q}_1={V}_{\Pi(1)}$, i.e., is $\Pi(1)=j$? This is equivalent to the setting of Lemma \ref{lemOnePointFive} as described further below. 

Note that since $1 \in J^{(n)}$, $\Pi(1) \in \Pi (J^{(n)})$. Therefore, when searching for the value of $\Pi(1)$, it is sufficient to look inside set $\Pi (J^{(n)})$. Therefore, instead of looking among all the values of ${V}_{j}$, it is sufficient to look at ${V}_{j}$ for $j \in  \Pi (J^{(n)})$. Let's show these values by $\textbf{V}_{\Pi} =\{v_1, v_2, \cdots, v_{N^{(n)}} \}$, so,
\[W_j^{(n)}= P\left(\Pi(1)=j \bigg{|} \textbf{V}_{\Pi}, \Pi (J^{(n)})\right).\]

Thus, we have the following scenario: $Q_u, u \in  J^{(n)}$ are independent random variables, and
\[Q_u\big{|} P_u=p_u \sim Uniform [p_u, p_u+(1-2p_u)a_n]. \]
Also, $v_1, v_2, \cdots, v_{N^{(n)}}$ are the observed values of $Q_u$ with unknown ordering (unknown mapping $\Pi$). We also know from the definition of set $J^{(n)}$ that
\[P_u \leq p_1+\epsilon_n \leq Q_u,\]
\[Q_u \leq p_1(1-2a_n)+an \leq P_u(1-2a_n)+a_n,\]
so, we can conclude
\[ v_j \in \bigcap_{u=1}^{N^{(n)}} [p_u, p_u+(1-2p_u)a_n] \ \ \textrm{for all }j \in \{1,2,..,N^{(n)}\}. \]
We know the event $E$ has occurred, meaning that the observed values of $Q_u$'s are equal to set of $v_j$'s (but with unknown ordering), i.e.,
\[E \ \ \equiv \ \ \{Q_u, u \in  J^{(n)}\}= \{ v_1, v_2, \cdots, v_{N^{(n)}} \}. \]
Then, according to Lemma \ref{lemOnePointFive},
\[P\left(Q_1=v_j |E, P_1, P_2, \cdots, P_n \right)=\frac{1}{N^{(n)}}. \]
Note that there is a subtle difference between this lemma and Lemma \ref{lemOnePointFive}. Here $N^{(n)}$ is a random variable while $N$ is a fixed number in Lemma \ref{lemOnePointFive}. Nevertheless, since the assertion holds for every fixed $N$, it also holds for the case where $N$ is a random variable. Now, note that
\begin{align*}
P\left(Q_1=v_j |E, P_1, P_2, \cdots, P_n \right) &= P\left(\Pi(1)=j \bigg{|} E, P_1, P_2, \cdots, P_n \right)\\
&=P\left(\Pi(1)=j \bigg{|} \textbf{V}_{\Pi}, \Pi (J^{(n)}), P_1, P_2, \cdots, P_n \right)\\
&=W_j^{(n)}.
\end{align*}
Thus, we can conclude
		\[W_j^{(n)}=\frac{1}{N^{(n)}}.\]
\end{proof}	
%It is worth noting that the above lemma in fact implies that  that given $\Pi (J^{(n)})$, $\Pi (1)$ and $V^{(n)}$ are independent. Specifically,
%\[P\left(\Pi^{(n)}(1)=j \bigg{|} V^{(n)}, \Pi (J^{(n)})\right)= P\left(\Pi^{(n)}(1)=j \bigg{|} \Pi (J^{(n)})\right).\]
%
%
In the third step, we define $\widetilde{W_j^{(n)}}$ for any $j \in \Pi (J^{(n)})$ as
\[\widetilde{W_j^{(n)}} = P\left(\Pi(1)=j \bigg{|} \textbf{Y}, \Pi (J^{(n)})\right).\]

$\widetilde{W_j^{(n)}}$ is the conditional probability that $\Pi(1)=j$ after observing the values of the anonymized version of the obfuscated samples of the users' data ($\textbf{Y}$) and the aggregate set including all the pseudonyms of the users in set $J^{(n)}$ (i.e., $\Pi(J^{(n)})\overset{\Delta}{=} \left\{\Pi^{-1}(j) \in J^{(n)}\right\}$). Since $\textbf{Y}$ and $\Pi (J^{(n)})$ are random, $\widetilde{W_j^{(n)}} $ is a random variable. Now, in the following lemma, we will prove $\widetilde{W_j^{(n)}} =\frac{1}{N^{(n)}}$, for all $j \in \Pi (J^{(n)})$ by using Lemma \ref{lem3}.

Note in the following lemma, we want to show that even if the adversary is given a set including all of the pseudonyms of the users in set $J^{(n)}$, he/she cannot match each specific user in set $J^{(n)}$ and his pseudonym.

\begin{lem}
	\label{lem3}
	For all $j \in \Pi (J^{(n)})$, $\widetilde{W_j^{(n)}}=\frac{1}{N^{(n)}}.$
\end{lem}

\begin{proof}
%	Consider random variable $\Pi$  while
%	\[Range(\Pi)=\{1, 2, \cdots ,N\}.\]
First, note that

%\begin{multline*}
%\widetilde{W_j^{(n)}} = \sum_{\text{for all v}} P\left(\Pi^{(n)}(1)=j \bigg{|} Y^{(n)}, \Pi \left(J^{(n)}\right), V^{(n)}=v\right)\cdot \\
%P\left(V^{(n)}=v \bigg{|} Y^{(n)}, \Pi \left(J^{(n)}\right)\right)
%\end{multline*}

\begin{align*}
\widetilde{W_j^{(n)}} = \sum_{\text{for all v}} P\left(\Pi(1)=j \bigg{|} \textbf{Y}, \Pi \left(J^{(n)}\right), \textbf{V}=\textbf{v}\right) P\left(\textbf{V}=\textbf{v} \bigg{|} \textbf{Y}, \Pi \left(J^{(n)}\right)\right).
\end{align*}

Also, we note that given $\textbf{V}$, $\Pi(J^{(n)})$, and $\textbf{Y}$ are independent. Intuitively, this is because when observing $\textbf{Y}$, any information regarding $\Pi(J^{(n)})$ is leaked through estimating $\textbf{V}$. This can be rigorously proved similar to the proof of Lemma 1 in \cite{tifs2016}. We can state this fact as
\[
	P\left(Y_u(k)\  \bigg{ | }  \ {V}_u=v_u, \Pi(J^{(n)}) \right)  = P\left(Y_u(k)\  \bigg{ | } \ {V}_u=v_u\right)=v_u.
\]
The right and left hand side are given by $Bernoulli (v_u)$ distributions.

As a result,
\[
\widetilde{W_j^{(n)}} = \sum_{\text{for all \textbf{v}}} P\left(\Pi(1)=j \bigg{|} \Pi (J^{(n)}), \textbf{V}=\textbf{v}\right) P\left(\textbf{V}=\textbf{v} \bigg{|} \textbf{Y}, \Pi \left(J^{(n)}\right)\right).
\]
Note $W_j^{(n)}=P\left(\Pi(1)=j \bigg{|} \Pi (J^{(n)}), \textbf{V}\right)$, so
\begin{align}
\no \widetilde{W_j^{(n)}} &= \sum_{\text{for all \textbf{v}}}W_j^{(n)}  P\left(\textbf{V}=\textbf{v} \bigg{|} \textbf{Y}, \Pi \left(J^{(n)}\right) \right) \\
\nonumber &= \frac{1}{N^{(n)}}\sum_{\text{for all \textbf{v}}} P\left(\textbf{V}=\textbf{v}\bigg{|} \textbf{Y}, \Pi \left(J^{(n)}\right)\right) \\
\nonumber &= \frac {1}{N^{(n)}}.\ \
\end{align}
\end{proof}

To show that no information is leaked, we need to show that the size of set $J^{(n)}$ goes to infinity. This is established in Lemma \ref{lem1}.

\begin{lem}
	\label{lem1}
	If $N^{(n)} \triangleq |J^{(n)}| $, then $N^{(n)} \rightarrow \infty$ with high probability as $n \rightarrow \infty$.  More specifically, there exists $\lambda>0$ such that
\[
	P\left(N^{(n)} > \frac{\lambda}{2}n^{\frac{\beta}{2}}\right) \rightarrow 1.
	\]
\end{lem}

\begin{proof}
Lemma \ref{lem1} is proved in Appendix \ref{sec:app_c}.
\end{proof}

In the final step, we define $\widehat{W_j^{(n)}}$ for any $j \in \Pi (J^{(n)})$ as
\[\widehat{W_j^{(n)}}=P\left(X_1(k)=1 \bigg{|} \textbf{Y}, \Pi (J^{(n)})\right).\]
$\widehat{W_j^{(n)}}$ is the conditional probability that $X_1(k)=1$ after observing the values of the anonymized version of the obfuscated samples of the users' data ($\textbf{Y}$) and the aggregate set including all of the pseudonyms of the users in set $J^{(n)}$ ($\Pi (J^{(n)})$). $\widehat{W_j^{(n)}} $ is a random variable because $\textbf{Y}$ and $\Pi (J^{(n)})$ are random. Now, in the following lemma, we will prove $\widehat{W_j^{(n)}}$ converges in distribution to $p_1$.

Note that this is the probability from the adversary's point of view. That is, given that the adversary has observed $\textbf{Y}$ as well as the extra information $ \Pi (J^{(n)})$, what can he/she infer about $X_1(k)$?
\begin{lem}
	\label{lem4}
	For all $j \in \Pi (J^{(n)})$, $\widehat{W_j^{(n)}} \xrightarrow{d} p_1.$
\end{lem}

\begin{proof}
We know
\begin{align*}
\widehat{W_j^{(n)}}= \sum_{j \in \Pi(J^{(n)})} P\left(X_1(k)=1 \bigg{|} \Pi(1)=j, \textbf{Y}, \Pi (J^{(n)})\right) P\left(\Pi(1)=j \bigg{|} \textbf{Y}, \Pi (J^{(n)})\right),
\end{align*}
and according to the definition $\widetilde{W_j^{(n)}}=P \left(\Pi(1)=j \bigg{|} \textbf{Y}, \Pi (J^{(n)})\right)$, we have
	\begin{align}
	\no \widehat{W_j^{(n)}} &= \sum_{j \in \Pi(J^{(n)})} P\left(X_1(k)=1 \bigg{|} \Pi(1)=j, \textbf{Y}, \Pi (J^{(n)})\right) \widetilde{W_j^{(n)}}\\
	\nonumber &= \frac{1}{N^{(n)}} \sum_{j \in \Pi(J^{(n)})} P\left(X_1(k)=1 \bigg{|} \Pi(1)=j, \textbf{Y}, \Pi (J^{(n)})\right).
	\end{align}
We now claim that
\[
 P\left(X_1(k)=1 \bigg{|} \Pi(1)=j, \textbf{Y}, \Pi (J^{(n)})\right)=p_1+o(1).
\]
The reasoning goes as follows. Given $\Pi(1)=j$ and knowing $\textbf{Y}$, we know that

\[
Y_{\Pi(1)}(k)={Z}_{1}(k)=\begin{cases}
{X}_{1}(k), & \textrm{with probability } 1-R_1.\\
1-{X}_{1}(k), & \textrm{with probability } R_1.
\end{cases}
\]

Thus, given $Y_{j}(k)=1$, Bayes' rule yields:
\begin{align*}
 P\left(X_1(k)=1 \bigg{|} \Pi(1)=j, \textbf{Y}, \Pi (J^{(n)})\right)&= \left(1- R_1 \right) \frac{P(X_1(k)=1)}{P(Y_{\Pi(1)}(k)=1)}\\
 &=\left(1-R_1 \right) \frac{p_1}{p_1 (1- R_1)+(1-p_1)R_1}\\
 &=1-o(1),
\end{align*}
and similarly, given $Y_{j}(k)=0$,
\begin{align*}
P\left(X_1(k)=1 \bigg{|} \Pi(1)=j, \textbf{Y}, \Pi (J^{(n)})\right)&= R_1 \frac{P(X_1(k)=1)}{P(Y_{\Pi(1)}(k)=0)}\\
&=R_1 \frac{p_1}{p_1 (1- R_1)+(1-p_1)R_1}\\
&=o(1).
\end{align*}
Note that by the independence assumption, the above probabilities do not depend on the other values of $Y_{u}(k)$ (as we are conditioning on $\Pi(1)=j$ ).
Thus, we can write
\begin{align}
	\no \widehat{W_j^{(n)}} &= \frac{1}{N^{(n)}} \sum_{j \in \Pi(J^{(n)})} P\left(X_1(k)=1 \bigg{|} \Pi(1)=j, \textbf{Y}, \Pi (J^{(n)})\right)\\
\no &=\frac{1}{N^{(n)}} \sum_{j \in \Pi(J^{(n)}), Y_{j}(k)=1}  (1-o(1)) + \frac{1}{N^{(n)}} \sum_{j \in \Pi(J^{(n)}), Y_{j}(k)=0}  o(1).
\end{align}
First, note that since $\abs*{\left\{j \in \Pi(J^{(n)}), Y_{j}(k)=0\right\}} \leq N^{(n)}$, the second term above converges to zero, thus,
\begin{align}
	\no \widehat{W_j^{(n)}}  \rightarrow \frac{\abs*{\left\{ j \in \Pi(J^{(n)}), Y_{\Pi(1)}(k)=1\right\} }}{N^{(n)}}.
\end{align}
Since for all $j \in \Pi(J^{(n)})$, $ Y_{j}(k)  \sim Bernoulli \left(p_1+o(1)\right)$, by a simple application of Chebyshev's inequality, we can conclude $\widehat{W_j^{(n)}}\rightarrow p_1$. Appendix \ref{sec:app_d} provides the detail.
\end{proof}	

As a result,
\begin{align*}
X_1(k) {|} \textbf{Y}, \Pi (J^{(n)})\rightarrow \textit{Bernoulli} (p_1),
\end{align*}
thus,
\[H\left(X_1(k) \bigg{|} \textbf{Y}, \Pi (J^{(n)})\right)\rightarrow H\left(X_1(k)\right).\]
Since conditioning reduces entropy,
\[H\left(X_1(k) \bigg{|} \textbf{Y}, \Pi (J^{(n)})\right)\leq H\left(X_1(k) \bigg{|} \textbf{Y}\right),\]
and as a result,
\[\lim_{n \rightarrow \infty}   H\left(X_1(k)\right)-H\left(X_1(k) \bigg{|} \textbf{Y}\right) \leq 0,\]
and 
\[\lim_{n \rightarrow \infty}   I\left(X_1(k);\textbf{Y}\right)\leq 0.\]
By knowing that $I\left(X_1(k);\textbf{Y}\right)$ cannot take any negative value, we can conclude that
\[I\left(X_1(k);\textbf{Y}\right)\rightarrow 0.\]
\end{proof}

\subsection{Extension to $r$-States}
Now, assume users' data samples can have $r$ possibilities $\left(0, 1, \cdots, r-1\right)$, and $p_u(i)$ shows the probability of user $u$ having data sample $i$. We define the vector $\textbf{p}_u$ and the matrix $\textbf{p}$ as
\[\textbf{p}_u= \begin{bmatrix}
p_u(1) \\ p_u(2) \\ \vdots \\p_u(r-1) \end{bmatrix} , \ \ \  \textbf{p} =\left[ \textbf{p}_{1}, \textbf{p}_{2}, \cdots,  \textbf{p}_{n}\right].
\]
We assume $p_u(i)$'s are drawn independently from some continuous density function, $f_\textbf{P}(\textbf{p}_u)$, which has support on a subset of the $(0,1)^{r-1}$ hypercube (Note that the $p_u(i)$'s sum to one, so one of them can be considered as the dependent value and the dimension is $r-1$). In particular, define the range of the distribution as
\begin{align}
\no  \mathcal{R}_{\textbf{p}} &= \{ (x_1,x_2,  \cdots, x_{r-1}) \in (0,1)^{r-1}:  x_i > 0 , x_1+x_2+\cdots+x_{r-1} < 1,\ i=1, 2, \cdots, r-1\}.
\end{align}
Figure~\ref{fig:rp} shows the range $\mathcal{R}_{\textbf{p}}$ for the case where $r=3$.
\begin{figure}[h]
	\centering
	\includegraphics[width = 0.5\linewidth]{fig/rp}
	\caption{$\mathcal{R}_{\textbf{p}}$ for case $r=3$.}
	\label{fig:rp}
\end{figure}


Then, we assume there are $\delta_1, \delta_2>0$ such that:
\begin{equation}
\begin{cases}
\no    \delta_1<f_{\textbf{P}}(\mathbf{p}_u) <\delta_2, & \textbf{p}_u \in \mathcal{R}_{\textbf{p}}.\\
    f_{\textbf{P}}(\mathbf{p}_u)=0, &  \textbf{p}_u \notin \mathcal{R}_{\textbf{p}}.
\end{cases}
\end{equation}

The obfuscation is similar to the two-states case. Specifically, for $l \in \{0,1,\cdots, r-1\}$, we can write
\[
P({Z}_{u}(k)=l| X_{u}(k)=i) =\begin{cases}
1-R_u, & \textrm{for } l=i.\\
\frac{R_u}{r-1}, & \textrm{for } l \neq i.
\end{cases}
\]



%


\begin{thm}\label{r_state_thm}
For the above $r$-states model, if $\textbf{Z}$ is the obfuscated version of $\textbf{X}$, and $\textbf{Y}$ is the anonymized version of $\textbf{Z}$ as defined previously, and
\begin{itemize}
	 \item $m=m(n)$ is arbitrary;
	\item $R_u \sim Uniform [0, a_n]$, where $a_n \triangleq c'n^{-\left(\frac{1}{r-1}-\beta\right)}$ for any $c'>0$ and $0<\beta<\frac{1}{r-1}$;
\end{itemize}
then, user 1 has perfect privacy. That is,
\begin{align}%\label{}
\no  \forall k\in \mathbb{N}, \ \ \ \lim\limits_{n\rightarrow \infty} I \left(X_1(k);{\textbf{Y}}\right) =0.
\end{align}\end{thm}

The proof of Theorem \ref{r_state_thm} is similar to the proof of Theorem \ref{two_state_thm}. The major difference is that instead of the random variables $P_u, Q_u, V_u$, we need to consider the random vectors $\textbf{P}_u, \textbf{Q}_u, \textbf{V}_u$.  Similarly, for user $u$, we define the vector $\textbf{Q}_u$ as
\[\textbf{Q}_u= \begin{bmatrix}
Q_u(1) \\ Q_u(2) \\ \vdots \\Q_u(r-1) \end{bmatrix}.
\]

In the $r$-states case,
\begin{align}
\no {Q}_u(i) &=P_u(i)\bigg(1-R_u(i) \bigg)+\bigg(1-P_u(i)\bigg)\frac{R_u}{r-1}  \\
\nonumber &= P_u+\bigg(1-r P_u\bigg)\frac{R_u}{r-1}.\ \
\end{align}
%Note that we have dropped the superscript $n$ for simplicity of notation (we need to write $Q_u^{(n)}(1)$ instead of $Q_u(1)$).
We also need to define the critical set $J^{(n)}$.  First, for $i=0,1, \cdots, r-1$, define set $J_i^{(n)}$ as follows. If $0\leq p_1(i)<\frac{1}{r}$, then,
\begin{align*}
&J_i^{(n)}= \\
&\left\{u \in \{1, 2, \dots, n\}: p_1(i) \leq P_u(i)\leq p_1(i)+\epsilon_n; p_1(i)+\epsilon_n\leq Q_u(i)\leq p_1(i)+(1-r p_1(i))\frac{a_n}{r-1}\right\},
\end{align*}
where $\epsilon_n \triangleq \frac{1}{n^{\frac{1}{r-1}-\frac{\beta}{2}}}$,  $a_n = c'n^{-\left(\frac{1}{r-1}-\beta\right)}$, and $\beta$ is defined in the statement of Theorem \ref{r_state_thm}.

We then define the critical set $J^{(n)}$ as:
\[
J^{(n)}=\bigcap_{l=0}^{r-1} J_i^{(n)}.
\]
We can then repeat the same arguments in the proof of Theorem \ref{two_state_thm} to complete the proof.

%\[J_i^{(n)}=
%\left\{  u \in \{1, 2, \dots, n\}: p_1(i) \leq P_u(i)\leq p_1(1)+\epsilon_n;  p_1(1)+\epsilon_n\leq Q_u(i)\leq p_1(1)+(1-2p_1(1)).a_n\right\}, \ \   \text{for}\ \ 0\leq p_1(1)<\frac{1}{2}\\
%\left\{  u \in \{1, 2, \dots, n\}: p_1(i)-\epsilon_n \leq P_u(i)\leq p_1(1); p_1(1)+(1-2p_1(1)).a_n\leq Q_u(i)\leq p_1-\epsilon_n\right\}, \ \  \text{for}\ \ \frac{1}{2}\leq p_1\leq1
%\end{cases}\]




\section{Converse Results: No Privacy Region}
\label{converse}

In this section, we prove that if the number of observations by the adversary is larger than its critical value and the noise level is less than its critical value, then the adversary can find an algorithm to successfully estimate users' data samples with arbitrarily small error probability. Combined with the results of the previous section, this implies that asymptotically (as $n \rightarrow \infty$), privacy can be achieved \emph{if and only if} at least one of  the two techniques (obfuscation or anonymization) are used above their thresholds. This statement needs a clarification as follows:  Looking at the results of \cite{tifs2016}, we notice that anonymization alone can provide perfect privacy if $m(n)$ is below its threshold. On the other hand, the threshold for obfuscation requires some anonymization: In particular, the identities of the users must be permuted once to prevent the adversary from readily identifying the users.



%is valid when both anonymization and obfuscation are simultaneously used as presented above (Figure \ref{fig:xyz}). In other words, if only obfuscation is used, then clearly the limit that we obtained above for $a_n$ is not nearly enough for privacy as it approaches zero as $n$ goes to infinity.

\subsection{Two-States Model}
Again, we start with the i.i.d.\ two-states model. The data sample of user $u$ at any time is a Bernoulli random variable with parameter $p_u$.

As before, we assume that $p_u$'s are drawn independently from some continuous density function, $f_P(p_u)$, on the $(0,1)$ interval. Specifically, there are $\delta_1, \delta_2>0$ such that:%\footnote{The condition $\delta_1<f_P(p_u) <\delta_2$ is not actually necessary for the results and can be relaxed; however, we keep it here to avoid unnecessary technicalities.}:
\begin{equation}
\no\begin{cases}
    \delta_1<f_P(p_u) <\delta_2, & p_u \in (0,1).\\
    f_P(p_u)=0, &  p _u\notin (0,1).
\end{cases}
\end{equation}

\begin{thm}\label{two_state_thm_converse}
For the above two-states mode, if $\textbf{Z}$ is the obfuscated version of $\textbf{X}$, and $\textbf{Y}$ is the anonymized version of $\textbf{Z}$ as defined, and
\begin{itemize}
	\item $m =cn^{2 +  \alpha}$ for any $c>0$ and $\alpha>0$;
	\item $R_u \sim Uniform [0, a_n]$, where $a_n \triangleq c'n^{-\left(1+\beta\right)}$ for any $c'>0$ and $\beta>\frac{\alpha}{4}$;
\end{itemize}
then, user $1$ has no privacy as $n$ goes to infinity.
\end{thm}

Since this is a converse result, we give an explicit detector at the adversary and show that it can be used by the adversary to recover the true data of user $1$.

\begin{proof}
The adversary first inverts the anonymization mapping $\Pi$ to obtain $Z_1(k)$, and then estimates the value of $X_1(k)$ from that. To invert the anonymization, the adversary calculates the empirical probability that each string is in state $1$ and then assigns the string with the empirical probability closest to $p_1$ to user 1.

\begin{figure}
	\centering
	\includegraphics[width=.8\linewidth]{fig/converse.jpg}
	\centering
	\caption{$p_1$, sets $B^{(n)}$ and $C^{(n)}$ for case $r=2$.}
	\label{fig:converse}
\end{figure}


Formally, for $u=1, 2, \cdots, n$, the adversary computes $\overline{Y_u}$, the empirical probability of user $u$ being in state $1$, as follows:
\[
\overline{Y_u}=\frac{Y_u(1)+Y_u(2)+ \cdots +Y_u(m)}{m},
\]
thus,
\[
\overline{Y_{\Pi(u)}}=\frac{Z_u(1)+Z_u(2)+ \cdots +Z_u(m)}{m}.
\]

As shown in Figure \ref{fig:converse}, define
\[B^{(n)}\triangleq \left\{x \in (0,1); p_1-\Delta_n \leq x \leq p_1+\Delta_n\right\},\]
where $\Delta_n = \frac{1}{n^{1+\frac{\alpha}{4}}}$ and $\alpha $ is defined in the statement of Theorem \ref{two_state_thm_converse}. We claim that for $m =cn^{2 +  \alpha}$, $a_n=c'n^{-(1 +  \beta)}$, and large enough $n$,
\begin{enumerate}
\item $P\left( \overline{Y_{\Pi(1) }}\in B^{(n)}\right) \rightarrow 1.$
\item $P\left( \bigcup\limits_{u=2}^{n} \left(\overline{Y_{\Pi(u)}}\in B^{(n)}\right)\right) \rightarrow 0.$
\end{enumerate}
As a result, the adversary can identify $\Pi(1)$ by examining  $\overline{Y_u}$'s and assigning the one in $B^{(n)}$ to user $1$. Note that $\overline{Y_{\Pi(u) }} \in B^{(n)}$ is a set (event) in the underlying probability space and can be written as $\left\{\omega \in \Omega: \overline{Y_{\Pi(u) }}(\omega) \in B^{(n)}\right\}$.

First, we show that as $n$ goes to infinity,
\[P\left( \overline{Y_{\Pi(1) }}\in B^{(n)}\right) \rightarrow 1.\]
We can write
\begin{align}
\no P\left(\overline{Y_{\Pi(1)}} \in B^{(n)}\right) &= P\left(\frac{\sum\limits_{k=1}^{m}Z_1(k)}{m} \in  B^{(n)} \right)\\
\nonumber &= P\left(p_1-\Delta_n \leq\frac{\sum\limits_{k=1}^{m}Z_1(k)}{m}\leq p_1+\Delta_n \right)\\
\nonumber &= P\left(mp_1-m\Delta_n-mQ_1\leq \sum\limits_{k=1}^{m}Z_1(k)-mQ_1\leq  mp_1+m\Delta_n-mQ_1 \right).\ \
\end{align}
%Note as n becomes large, $a_n\ll \Delta_n$, as a result,
%\[P\left(\overline{Y_{\Pi(1)}^{(n)}} \in B^{(n)}\right)=P\left(\abs*{\sum\limits_{i=1}^{m}Z_1^{(n)}(i)-mQ_1^{(n)}}< m\Delta_n \right).\]
Note that for any $u \in \{1,2,\cdots, n \}$, we have
\begin{align}
\no |p_u-{Q}_u| &=|1-2p_u|R_u \\
\no & \leq R_u \leq a_n,
\end{align}
so, we can conclude
\begin{align}
\no P\left(\overline{Y_{\Pi(1)}} \in B^{(n)}\right) &= P\left(mp_1-m\Delta_n-mQ_1\leq \sum\limits_{k=1}^{m}Z_1(k)-mQ_1 \leq mp_1+m\Delta_n-mQ_1 \right)\\
\nonumber  &\geq P\left(-m\Delta_n+m a_n\leq \sum\limits_{k=1}^{m}Z_1(k)-mQ_1\leq -ma_n+m\Delta_n \right)\\
\nonumber &= P\left(\abs*{\sum\limits_{k=1}^{m}Z_1(k)-mQ_1}\leq m (\Delta_n-a_n) \right).\ \
\end{align}
Since $a_n \rightarrow 0$, for $p_1 \in (0,1)$ and large enough $n$, we can say $p_1+a_n < 2 p_1$. From Chernoff bound, for any $c,c',\alpha>0$ and $\beta>\frac{\alpha}{4}$,
\begin{align}
%\nonumber
\no P\left(\abs*{\sum\limits_{k=1}^{m}Z_1(k)-mQ_1}\leq m (\Delta_n-a_n) \right) &\geq 1-2e^{-\frac{m(\Delta_n-a_n)^2}{3Q_1}} \\
\nonumber &\geq 1-2e^{-\frac{1}{3(p_1+a_n)}cn^{2+\alpha}\left(\frac{1}{n^{1+\frac{\alpha}{4}}}- \frac{c'}{n^{1 +  \beta}}\right)^2}\\
%\nonumber &\geq 1-2e^{-\frac{c}{3c'(1-2p_1)}n^{1+\frac{3\alpha}{2}}}\\
\nonumber &\geq 1-2e^{-\frac{c''}{6 p_1}n^{\frac{\alpha}{2}}}\rightarrow 1. \ \
\end{align}
As a result, as $n$ becomes large,
\[P\left(\overline{Y_{\Pi(1)}} \in B^{(n)}\right) \rightarrow 1.\]

Now, we need to show that as $n$ goes to infinity,
\[P\left( \bigcup\limits_{u=2}^{n} \left(\overline{Y_{\Pi(u)}}\in B^{(n)}\right)\right) \rightarrow 0.\]
First, we define
\[
C^{(n)}=\left\{x\in (0,1); p_1-2\Delta_n \leq x \leq p_1+2\Delta_n\right\}
,\]
and claim as $n$ goes to infinity,
\[P\left(\bigcup\limits_{u=2}^n \left(P_u \in C^{(n)}\right) \right)\rightarrow 0. \]

Note
\[4\Delta_n\delta_1< P\left( P_u\in C^{(n)}\right) < 4 \Delta_n\delta_2,\]
and according to the union bound, for large enough $n$,
\begin{align}
\no P\left( \bigcup\limits_{u=2}^n \left(P_u \in C^{(n)}\right) \right) &\leq \sum\limits_{u=2}^n P\left( P_u \in C^{(n)}\right) \\
\nonumber &\leq 4n \Delta_n \delta_2\\
\nonumber &= 4n \frac{1}{n^{1+{\frac{\alpha}{4}}}} \delta_2\\
\nonumber &= 4n^{-\frac{\alpha}{4}}\delta_2 \rightarrow 0. \ \
%\nonumber &\leq 4n^{-\frac{\alpha}{4}}\delta_2 \rightarrow 0. \ \
\end{align}
As a result, we can conclude that all $p_u$'s are outside of $C^{(n)}$ for $u \in \left\{2,3, \cdots, n\right\}$ with high probability.

Now, we claim that given all $p_u$'s are outside of $C^{(n)}$, $P\left(\overline{Y_{\Pi (u)}} \in B^{(n)}\right)$ is small. Remember that for any $u \in \{1,2,\cdots, n \}$, we have
\begin{align}
\no |p_u-{Q}_u| \leq a_n.
\end{align}
Now, noting the definitions of sets $B^{(n)}$ and $C^{(n)}$, we can write for $u \in \left\{2,3,  \cdots , n\right\}$,
\begin{align}
\nonumber
\no P\left(\overline{Y_{\Pi(u)}} \in B^{(n)}\right) &\leq P\left(\abs*{\overline{Y_{\Pi(u)}}-Q_u}\geq  (\Delta_n-a_n) \right)\\
%\nonumber &\hspace{-0.2 in}\leq  P\left(\frac{\sum\limits_{i=1}^{m}Z_j^{(n)}(i)}{m}<p_j-\Delta_n ,\frac{\sum\limits_{i=1}^{m}Z_j^{(n)}(i)}{m}>p_j+\Delta_n \right)\\
%\nonumber &\hspace{-0.2 in}\leq P\left(\sum\limits_{i=1}^{m}Z_j^{(n)}(i)-mQ_j^{(n)}<mp_j-m\Delta_n-mQ_j^{(n)}, \sum\limits_{i=1}^{m}Z_j^{(n)}(i)-mQ_j^{(n)}> mp_j+m\Delta_n-mQ_j^{(n)} \right)\\
%\nonumber &\hspace{-0.2 in}\leq P\left(\sum\limits_{i=1}^{m}Z_j^{(n)}(i)-mQ_j^{(n)}<ma_n-m\Delta_n, \sum\limits_{i=1}^{m}Z_j^{(n)}(i)-mQ_j^{(n)}>-ma_n+m\Delta_n \right)\\
\nonumber &= P\left(\abs*{\sum\limits_{k=1}^{m}Z_u(k)-mQ_u}> m(\Delta_n-a_n) \right).\ \
\end{align}
%We also know that as n becomes large, $a_n\ll \Delta_n$, as a result,
%\[P\left(\overline{Y_{\Pi(j)}^{(n)}} \in B^{(n)}\right)\leq P\left(\abs*{\sum\limits_{i=1}^{m}Z_j^{(n)}(i)-mQ_j^{(n)}}> m\Delta_n \right).\]
According to the Chernoff bound, for any $c,c',\alpha>0$ and $\beta>\frac{\alpha}{4}$,
\begin{align}
%\nonumber
\no P\left(\abs*{\sum\limits_{k=1}^{m}Z_u(k)-mQ_u}> m(\Delta_n-a_n) \right) &\leq 2e^{-\frac{m(\Delta_n-a_n)^2}{3Q_1}} \\
\nonumber &\leq 2e^{-\frac{1}{3(p_1+a_n)}cn^{2+\alpha}\left(\frac{1}{n^{1+\frac{\alpha}{4}}}- \frac{c'}{n^{1 +  \beta}}\right)^2}\\
%\nonumber &\leq 2e^{-\frac{c}{3c'(1-2p_1)}n^{1+\frac{3\alpha}{2}}}\\
\nonumber &\leq 2e^{-\frac{c''}{6 p_1}n^{\frac{\alpha}{2}}}. \ \
\end{align}
Now, by using a union bound, we have
\begin{align}
\no P\left( \bigcup\limits_{u=2}^n \left(\overline{Y_{\Pi(u)}}\in B^{(n)}\right)\right)&\leq \sum\limits_{u=2}^{n}P\left(\overline{Y_{\Pi(u)}}\in B^{(n)}\right)\\
\nonumber &\leq n\left(2e^{-\frac{c''}{6 p_1}n^{\frac{\alpha}{2}}}\right),\ \
\end{align}
and thus, as $n$ goes to infinity,
\[P\left( \bigcup\limits_{u=2}^n \left(\overline{Y_{\Pi(u)}}\in B^{(n)}\right)\right) \rightarrow 0.\]

So, the adversary can successfully recover $Z_1(k)$. Since $Z_{1}(k)=X_1(k)$ with probability $1-R_1=1-o(1)$, the adversary can recover $X_{1}(k)$ with vanishing error probability for large enough $n$.
\end{proof}

\subsection{Extension to $r$-States}
Now, assume users' data samples can have $r$ possibilities $\left(0, 1, \cdots, r-1\right)$, and $p_u(i)$ shows the probability of user $u$ having data sample $i$. We define the vector $\textbf{p}_u$ and the matrix $\textbf{p}$ as
\[\textbf{p}_u= \begin{bmatrix}
p_u(1) \\ p_u(2) \\ \vdots \\p_u(r-1) \end{bmatrix} , \ \ \  \textbf{p} =\left[ \textbf{p}_{1}, \textbf{p}_{2}, \cdots,  \textbf{p}_{n}\right].
\]
We also assume $\textbf{p}_u$'s are drawn independently from some continuous density function, $f_P(\textbf{p}_u)$, which has support on a subset of the $(0,1)^{r-1}$ hypercube. In particular, define the range of distribution as
\begin{align}
\no  \mathcal{R}_{\textbf{p}} &= \left\{ (x_1, x_2, \cdots, x_{r-1}) \in (0,1)^{r-1}: x_i > 0 , x_1+ x_2+\cdots+ x_{r-1} < 1,\ \ i=1, 2,\cdots, r-1\right\}.
\end{align}
Then, we assume there are $\delta_1, \delta_2>0$ such that:
\begin{equation}
\begin{cases}
\no    \delta_1<f_{\textbf{P}}(\mathbf{p}_u) <\delta_2, & \textbf{p}_u \in \mathcal{R}_{\textbf{p}}.\\
    f_{\textbf{P}}(\mathbf{p}_u)=0, &  \textbf{p}_u \notin \mathcal{R}_{\textbf{p}}.
\end{cases}
\end{equation}

\begin{thm}\label{r_state_thm_converse}
For the above $r$-states mode, if $\textbf{Z}$ is the obfuscated version of $\textbf{X}$, and $\textbf{Y}$ is the anonymized version of $\textbf{Z}$ as defined, and
\begin{itemize}
	\item $m =cn^{\frac{2}{r-1} +  \alpha}$ for any $c>0$ and $0<\alpha<1$;
	\item $R_u \sim Uniform [0, a_n]$, where $a_n \triangleq c'n^{-\left(\frac{1}{r-1}+\beta\right)}$ for any $c'>0$ and $\beta>\frac{\alpha}{4}$;
\end{itemize}
then, user $1$ has no privacy as $n$ goes to infinity. 
\end{thm}


The proof of Theorem \ref{r_state_thm_converse} is similar to the proof of Theorem \ref{two_state_thm_converse}, so we just provide the general idea. We similarly define the empirical probability that the user with pseudonym $u$ has data sample $i$ $\left(\overline{{Y}_{u}}(i)\right)$ as follows:

\[
\overline{Y_u}(i)=\frac{\abs {\left\{k \in \{1, 2, \cdots, m\}:Y_u(k)=i\right\}}}{m},
\]
thus,
\[
\overline{Y_{\Pi(u)}}(i)=\frac{\abs {\left\{k \in \{1, 2, \cdots, m\}:Y_u(k)=i\right\}}}{m}.
\]

The difference is that now for each $u \in \{1,2,\cdots, n \}$, $\overline{\textbf{Y}_{u}}$ is a vector of size $r-1$. In other words,
\[\overline{\textbf{Y}_{u}}=\begin{bmatrix}
\overline{Y_u}(1) \\ \overline{Y_u}(2) \\ \vdots \\\overline{Y_u}(r-1) \end{bmatrix}.\]


\begin{figure}
  \centering
  \includegraphics[width=.5\linewidth, height=0.5 \linewidth]{fig/R.jpg}
  \centering
  \caption{$\textbf{p}_1$, sets $B'^{(n)}$ and $C'^{(n)}$ in $\mathcal{R}_\textbf{p}$ for case $r=3$.}
  \label{fig:rpp}
\end{figure}

Define sets $B'^{(n)}$ and $C'^{(n)}$ as
\begin{align}
 \no B'^{(n)}\triangleq & \left\{(x_1,x_2, \cdots ,x_{r-1}) \in \mathcal{R}_{\textbf{p}}: p_1(i)-\Delta'_n \leq x_i \leq p_1(i)+\Delta'_n,\ i=1,2, \cdots, r-1\right\},
\end{align}
\begin{align}
\no C'^{(n)}\triangleq &\left\{(x_1,x_2, \cdots ,x_{r-1}) \in \mathcal{R}_{\textbf{p}}: p_1(i)-2 \Delta'_n \leq x_i \leq p_1(i)+2 \Delta'_n,\ i=1,2, \cdots ,r-1\right\},
\end{align}
where $\Delta'_n = \frac{1}{n^{\frac{1}{r-1}+\frac{\alpha}{4}}}.$ Figure \ref{fig:rpp} shows $\textbf{p}_1$ and sets  $B'^{(n)}$ and $C'^{(n)}$ for the case $r=3.$

We claim for $m =cn^{\frac{2}{r-1} +  \alpha}$ and large enough $n$,
\begin{enumerate}
\item $P\left( \overline{\textbf{Y}_{\Pi(1) }}\in B'^{(n)}\right) \rightarrow 1$.
\item $P\left( \bigcup\limits_{u=2}^{n} \left(\overline{\textbf{Y}_{\Pi(u)}}\in B'^{(n)}\right)\right) \rightarrow 0.$
\end{enumerate}
The proof follows that for the two-states case. Thus, the adversary can de-anonymize the data and then recover $X_1(k)$ with vanishing error probability in the $r$-states model.



\subsection{Markov Chain Model} 
\label{subsec:markov}

So far, we have assumed users' data samples can have $r$ possibilities $\left(0, 1, \cdots, r-1\right)$ and users' pattern are i.i.d.\ . Here we model users' pattern using Markov chains to capture the dependency of the users' pattern over time. Again, we assume there are $r$ possibilities (the number of states in the Markov chains). Let $E$ be the set of edges. More specifically, $(i, l) \in E$ if there exists an edge from $i$ to $l$ with probability $ p(i,l)>0 $. What distinguishes different users is their transition probabilities $p_u(i,l)$ (the probability that user $u$ jumps from state $i$ to state $l$). The adversary knows the transition probabilities of all users. The model for obfuscation and anonymization is exactly the same as before.

We show that the adversary will be able to estimate the data samples of the users with low error probability if $m(n)$ and $a_n$ are in the appropriate range. The key idea is that the adversary can focus on a subset of the transition probabilities that are sufficient for recovering the entire transition probability matrix. By estimating those transition probabilities from the observed data and matching with the known transition probabilities of the users, the adversary will be able to first de-anonymize the data, and then estimate the actual samples of users' data. In particular, note that for each state $i$, we must have
\[\sum\limits_{l=1}^{r} p_u(i,l)=1,   \  \  \textrm{ for each }u \in \{1,2,\cdots, n \}, \]
so, the Markov chain of user $u$ is completely determined by a subset of size $d=|E|-r$ of transition probabilities. We define the vector $\textbf{p}_u$ and the matrix $\textbf{p}$ as
\[\textbf{p}_u= \begin{bmatrix}
p_u(1) \\ p_u(2) \\ \vdots \\p_u(|E|-r) \end{bmatrix} , \ \ \  \textbf{p} =\left[ \textbf{p}_{1}, \textbf{p}_{2}, \cdots,  \textbf{p}_{n}\right].
\]


We also consider $\textbf{p}_u$'s are drawn independently from some continuous density function, $f_P(\textbf{p}_u)$, which has support on a subset of the $(0,1)^{|E|-r}$ hypercube. Let $\mathcal{R}_{\textbf{p}} \subset \mathbb{R}^{d}$ be the range of acceptable values for $\textbf{p}_{u}$, so we have
\begin{align}
\no  \mathcal{R}_{\textbf{P}} &= \left\{ (x_1,x_2 \cdots, x_{d}) \in (0,1)^{d}: x_i > 0 , x_1+x_2+\cdots+x_{d} < 1,\ \ i=1,2,\cdots, d\right\}.
\end{align}
As before, we assume there are $ \delta_1, \delta_2 >0$, such that:
\begin{equation}
\begin{cases}
\no    \delta_1<f_{\textbf{P}}(\textbf{p}_u) <\delta_2, & \textbf{p}_u \in \mathcal{R}_{\textbf{p}}.\\
    f_{\textbf{P}}(\textbf{p}_u)=0, &  \textbf{p}_u \notin \mathcal{R}_{\textbf{p}}.
\end{cases}
\end{equation}

Using the above observations, we can establish the following theorem.


\begin{thm}\label{markov_thm}
For an irreducible, aperiodic Markov chain with $r$ states and $|E|$ edges as defined above, if $\textbf{Z}$ is the obfuscated version of $\textbf{X}$, and $\textbf{Y}$ is the anonymized version of $\textbf{Z}$, and
\begin{itemize}
 \item $m =cn^{\frac{2}{|E|-r} +  \alpha}$ for any $c>0$ and $\alpha>0$;
\item $R_u \sim Uniform [0, a_n]$, where $a_n \triangleq c'n^{-\left(\frac{1}{|E|-r}+\beta \right)}$ for any $c'>0$ and $\beta>\frac{\alpha}{4}$;
\end{itemize}
then, the adversary can successfully identify the data of user $1$ as $n$ goes to infinity. 
\end{thm}
The proof has a lot of similarity to the i.i.d.\ case, so we provide a sketch, mainly focusing on the differences. We argue as follows. If the total number of observations per user is $m=m(n)$, then define $M_i(u)$ to be the total number of visits by user $u$ to state $i$, for $i=0, 1, \cdots, r-1$. Since the Markov chain is irreducible and aperiodic, and $m(n) \rightarrow \infty$, all $\frac{M_i(u)}{m(n)}$ converge to their stationary values. Now conditioned on $M_i(u)=m_i(u)$, the transitions from state $i$ to state $l$ for user $u$ follow a multinomial distribution with probabilities $p_u(i,l)$.

Given the above, the setting is now very similar to the i.i.d.\ case. Each user is uniquely characterized by a vector $\textbf{p}_u$ of size $|E|-r$. We define sets $B^{''(n)}$ and $C^{''(n)}$ as
\[
 B^{''(n)}\triangleq \{(x_1, x_2, \cdots ,x_{d}) \in \mathcal{R}_{\textbf{p}}: p_1(i)-\Delta''_n \leq x_i \leq p_1(i)+\Delta''_n, i=1,2 , \cdots ,d\},
\]
\[
 C^{''(n)}\triangleq \{(x_1, x_2, \cdots ,x_{d}) \in \mathcal{R}_{\textbf{p}}:
  p_1(i)-2 \Delta''_n < x_i < p_1(i)+2 \Delta''_n, i=1,2, \cdots, d\},
\]
where $\Delta''_n= \frac{1}{n^{\frac{1}{|E|-r}+\frac{\alpha}{4}}}$, and $d= |E|-r$. Then, we can show that for the stated values of $m(n)$ and $a_n$, as $n$ becomes large:
\begin{enumerate}
\item $P\left( \overline{\textbf{Y}_{\Pi(1) }}\in B^{''(n)}\right) \rightarrow 1$,
\item $P\left( \bigcup\limits_{u=2}^{n} \left(\overline{\textbf{Y}_{\Pi(u)}}\in B{''}^{(n)}\right)\right) \rightarrow 0$,
\end{enumerate}
which means that the adversary can estimate the data of user $1$ with vanishing error probability. The proof is very similar to the proof of the i.i.d.\ case; however, there are two differences that need to be addressed:

First, the probability of observing an erroneous observation is not exactly given by $R_u$. In fact, a transition is distorted if at least one of its nodes is distorted. So, if the actual transition is from state $i$ to state $l$, then the probability of an erroneous observation is equal to
\begin{align}
	\no R'_u&=R_u+R_u-R_uR_u =R_u(2-R_u).
\end{align}


Nevertheless, here the order only matters, and the above expression is still in the order of $a_n =O \left( n^{-\left(\frac{1}{|E|-r}+\beta \right)} \right)$.

The second difference is more subtle. As opposed to the i.i.d.\ case, the error probabilities are not completely independent. In particular, if $X_u(k)$ is reported in error, then both the transition to that state and from that state are reported in error. This means that there is a dependency between errors of adjacent transitions. We can address this issue in the following way:  The adversary makes his decision only based on a subset of the observations. More specifically, the adversary looks at only odd-numbered transitions: First, third, fifth, etc., and ignores the even-numbered transitions.  In this way, the number of observations is effectively reduced from $m$ to $\frac{m}{2}$ which again does not impact the order of the result (recall that the Markov chain is aperiodic). However, the adversary now has access to observations with independent errors.



\section{Perfect Privacy Analysis: Markov Chain Model}\label{sec:perfect-MC}
So far, we have provided both achievability and converse results for the i.i.d.\ case. However, we have only provided the converse results for the Markov chain case. Here, we investigate achievability for Markov chain models. It turns out that for this case, the assumed obfuscation technique is not sufficient to achieve a reasonable level of privacy. Loosely speaking, we can state that if the adversary can make enough observations, then he can break the anonymity. The culprit is the fact that the sequence observed by the adversary is no longer modeled by a Markov chain; rather, it can be modeled by a hidden Markov chain. This allows the adversary to successfully estimate the obfuscation random variable $R_u$ as well as the $ p_u(i,l)$ values for each sequence, and hence successfully de-anonymize the sequences.


More specifically, as we will see below, there is a fundamental difference between the i.i.d.\ case and the Markov chain case. In the i.i.d.\ case, if the noise level is beyond a relatively small threshold, the adversary will be unable to de-anonymize the data and unable to recover the actual values of the data sets for users, \emph{regardless of the (large) size of $m=m(n)$}. On the other hand, in the Markov chain case, if $m=m(n)$ is large enough, then the adversary can easily de-anonymize the data. To better illustrate this, let's consider a simple example.

\begin{example}
Consider the scenario where there are only two states and the users' data samples change between the two states according to the Markov chain shown in Figure \ref{fig:MC-diagram}. What distinguishes the users is their different values of $p$. Now, suppose we use the same obfuscation method as before. That is, to create a noisy version of the sequences of data samples, for each user $u$, we generate the random variable $R_u$ that is the probability that the data sample of the user is changed to a different data sample by obfuscation. Specifically,
\[
{Z}_{u}(k)=\begin{cases}
{X}_{u}(k), & \textrm{with probability } 1-R_u.\\
1-{X}_{u}(k),& \textrm{with probability } R_u.
\end{cases}
\]

\begin{figure}[H]
\begin{center}
\[
\SelectTips {lu}{12 scaled 2500}
\xymatrixcolsep{6pc}\xymatrixrowsep{5pc}\xymatrix{
*++[o][F]{0}  \ar@/^1pc/[r]^{1}
& *++[o][F]{1} \ar@(dr,ur)[]_{1-p}  \ar@/^1pc/[l]^{p}
}
\]
\caption{A state transition diagram.}\label{fig:MC-diagram}
\end{center}
\end{figure}
To analyze this problem, we can construct the underlying Markov chain as follows. Each state in this Markov chain is identified by two values: the real state of the user, and the observed value by the adversary. In particular, we can write
\[\left(\text{Real value}, \text{Observed value}\right) \in \left\{\right(0,0), (0,1), (1,0), (1,1)\}.\]
Figure \ref{fig:MC-diagram2} shows the state transition diagram of this new Markov chain.

\begin{figure}[H]
\begin{center}
\[
\SelectTips {lu}{12 scaled 2000}
\xymatrixcolsep{10pc}\xymatrixrowsep{8pc}\xymatrix{
*++[o][F]{00} \ar@//[d]^{R} \ar@/_1pc/[dr]_>>>>>{1-R}
& *++[o][F]{01} \ar@//[d]^{1-R} \ar@/^1pc/[dl]^>>>>>{R} \\
*++[o][F]{10} \ar@(d,l)[]^{(1-p)R} \ar@/_2pc/[r]_{(1-p)(1-R)} \ar@/^2pc/[u]^{p(1-R)} \ar@/^1pc/[ur]^>>>>>{pR}
& *++[o][F]{11} \ar@(d,r)[]_{(1-p)(1-R)}  \ar@//[l]^{(1-p)R} \ar@/_2pc/[u]_{pR} \ar@/_1pc/[ul]_>>>>>{p(1-R)}
}
\]
\caption{The state transition diagram of the new Markov chain.}\label{fig:MC-diagram2}
\end{center}
\end{figure}
 We know
 \[ \pi_{00}=\pi_0(1-R)=\frac{p}{1+p}(1-R).\]
 \[ \pi_{01}=\pi_0R= \frac{p}{1+p}R.\]
 \[ \pi_{10}=\pi_1R=\frac{1}{1+p}R.\]
 \[ \pi_{11}=\pi_1(1-R)=\frac{1}{1+p}(1-R).\]

The observed process by the adversary is not a Markov chain; nevertheless, we can define limiting probabilities. In particular, let $\theta_0$ be the limiting probability of observing a zero. That is, we have
\[
\frac{M_0}{m} \xrightarrow{d} \theta_0,  \ \ \textrm{ as }n \rightarrow \infty,
\]
where $m$ is the total number of observations by the adversary, and $M_0$ is the number of $0$'s observed. Then,
\[\theta_0= \pi_{00}+\pi_{10} =\frac{(1-R)p+R}{1+p}.\]
Also, let $\theta_1$ be the limiting probability of observing a one, so
\[\theta_1= \pi_{01}+\pi_{11} =\frac{pR+(1-R)}{1+p}=1-\theta_0.\]


Now the adversary's estimate of $\theta_0$ is given by:
\begin{align}\label{eq1}
\hat{\theta}_0= \frac{(1-R)p+R}{1+p}.
\end{align}
Note that if the number of observations by the adversary can be arbitrarily large, the adversary can obtain an arbitrarily accurate estimate of $\theta_0$.
The adversary can obtain another equation easily, as follows.   Let $\theta_{01}$ be the limiting value of the portion of transitions from state $0$ to $1$ in the chain observed by the adversary. We can write
\begin{align}
%\nonumber
\no \theta_{01} &=P \left\{(00\rightarrow 01), (00\rightarrow 11), (10 \rightarrow 01), (10 \rightarrow 11) \right\}\\\
\nonumber &= \pi_{00}(1-R)+\pi_{10}PR+ \pi_{10}(1-p)(1-R).\ \
\end{align}
As a result,
\begin{align}\label{eq2}
\hat{\theta}_{01}= \frac{p(1-R)^2+R\left(PR(1-R)(1-p)\right)}{1+p}.
\end{align}
Again, if the number of observations can be arbitrarily large, the adversary can obtain an arbitrarily accurate estimate of $\theta_{01}$.
By solving the Equations \ref{eq1} and \ref{eq2}, the adversary can successfully recover $R$ and $p$; thus, he/she can successfully determine the users' data values.

\end{example}



%%%%%%%%%%%%%%%%%%%%%%%%%%%%%%%%%%%%%%%%%%%%%%%%%%%%%

%%%%%%%%%%%%%%%%%%%%%%%%%%%%%%%%%%%%%

%%%%%%%%%%%%%%%%%%%%%%%%%%%%%%%
%\section{Simulation}

\subsection{motivation} %
The most efficient way to figure out the answers to the questions we posed in the introduction is to deploy the proposed framework on a real-world platform and analyze how users adopt different and complex privacy policies to optimize their rewards.
However, direct deployment of these strategies and investments is currently impractical due to the following reasons.


Firstly, the most important reason is that such an online experiment may lead to the decline of the recommendation performances as well as the user experience, which harms the platform's revenue.
In the real world, nearly all the companies determine their platform mechanism driven by interest, and the revenues of the platforms are highly correlated with the recommendation performances. 
Therefore, it's nearly impossible to persuade any platform to directly deploy proposed strategies and mechanism online without other benefits.  



Secondly, the experiments are \czq{built} upon several simplifications, mentioned in \cref{assumptions}, which poses challenges towards recommendation model training process.
For example, we assume when a user \czq{adjusts} his data disclosure policy, the recommendation system will forget his un-disclosed data. 
To facilitate such challenges, model unlearning or other privacy-preserving technologies are imposed.
However, in real-world applications, very few the e-commercial platforms have deployed these privacy-preserving technologies during the deep recommendation model training and evaluation processing.
As a result, we may still fail to guarantee the assumptions and simulation methodology becomes a substitution.










In summary, inspired by the success of simulation study on dynamic interactive problems in real-world applications~\cite{Ie:arxiv19:RecSim,krauth2020offline,lucherini2021t,yao2021measuring},  we employ the simulation to study the effects of the proposed framework and the possible game between users and the platform.















\subsection{Simplified Assumptions}
\label{assumptions}

To simplify the simulation process for easier analysis, we make some necessary assumptions to simplify the problem.

\begin{assumption}[Static Assumption] User $i$ optimizes her/his policy on the fixed data $\di{}$ which is not affected by user policy $\pi_i$.
\label{assumption:static}
\end{assumption}

Here static means the user data $\di{}$ is fixed during the simulation, but the disclosed data $\si{}$ produced by different user policies is dynamic. 
It is also the most common setting for recommendation task in research papers~\cite{Rendle:www10:Factorizing,Hidasi:ICLR2016:gru4rec,NCF,kang2018self,Sun:cikm19:BERT4Rec}.
In the simulation, we train the recommendation system $\texttt{M}_{{\scriptscriptstyle \mathcal{S}}}$ on the collected dynamic data $\mathcal{S}$ and validate the recommendation efficiency on a fixed test set. 
In real-world applications, the data $\di{}$, which contains the behavior data from the interaction with the recommender $\texttt{M}_{{\scriptscriptstyle \mathcal{S}}}$, is also dynamically changing with the user's policy $\pi_i$.
It is beyond the scope of this paper and we leave it as the future work. 


\begin{assumption}[Immediate Assumption] The recommendation model $\texttt{M}_{{\scriptscriptstyle \mathcal{S}}}$ can only use the data $\si{}$ currently disclosed by each user $i$.
\label{assumption:forget}
\end{assumption}
The motivation of this assumption is that an untrusted platform can leverage user $i$' all data $\di{}$ if it can use the data disclosed in previous actions.
Without this constraint, the privacy right discussed in this paper is meaningless.
To achieve this, the platform can retrain the model from scratch with new data $\si{'}$ or quick unlearn the data in $\si{}$ then finetune with data $\si{'}$~\cite{cao2015towards,bourtoule2021machine,chen2022recommendation}.


However, the \cref{assumption:forget} also raises a new challenge that the asynchronous changes of user policy bring intractable computation costs for the platform since each time the user changes the disclosed data, the platform needs to update the model.
Here, we make an assumption for simplifying the simulation, assuming all users realize that the platform will cyclically (e.g., once a day) collect their privacy decisions and update recommender systems.
\begin{assumption}
[%
Cyclical Assumption]
Platform cyclically collects user privacy choices, and then the platform updates the model using all newly disclosed data. 
\label{assumption:synchronization}
\end{assumption}



In summary, for easy analysis in simulations, we introduce these assumptions to ignore the time and dynamic effects in this feedback system, just like the traditional recommendation task formulation.
























\subsection{Platform Mechanism Simulation}
\label{sec:plat_mech}
In order to validate the effect of the platform mechanism, we adopt several mechanisms during simulation. 
For easy comparison, we utilize one mechanism at each experiment. %


\subsubsection{\textbf{Data Split Rule}}

\czq{In our simulation, we do not split the profile attribution and the user can determine whether to share all of their attributes.}
For behavior data, we apply ``percentage split'' as $\delta_b$ with different split granularity $p$ (e.g., 1/3) to split the behavior sequence into $1/p$ parts. 
One obvious advantage of ``percentage split'' is that it can normalize the size of user action space and decrease the inconvenience of the interaction between the user and the platform.

\subsubsection{\textbf{Data Disclosure Strategy}}
\label{sec:data_disclose_choice}

As the platforms have certain flexibility to implement different data disclosure strategies, we discuss three representative disclosure strategies used in our study for behavior data in this subsection.
These strategies determine the data disclosure action space $\Pi$ the user can choose.
For profile attributes, we found that all users tend not to disclose them in the experiments since these features are negligible for improving recommendation utility in the presence of behavior data.
Similar result that user profile features contribute very marginal to the recommendation results in the case of strong user behavior modeling on public benchmark datasets has also been reported in other works~\cite{kang2018self,Sun:cikm19:BERT4Rec}.
Thus, in the following study, we mainly focus on modeling only  behavior data.

The ``\textit{separate}'' rule gives the users the control to freely disclose any split personal data.
For this rule, the size of user $i$'s the action space is exponentially expended on the size of the spilt data set $|\delta_b ({\scriptstyle \mathcal{D}_{i,b}})|$, denoted as $2^{|\delta_b ({\scriptstyle \mathcal{D}_{i,b}})|}$. 
However, too many choices might make it difficult for users to make better data disclosure decisions.

Another data disclosure strategy named ``\textit{oldest continuous}'' provides users the choices to disclose continuous behavior data from the beginning time, such as selecting ``the oldest 33\% data''.
In this strategy, to disclose newer behavior data ${\scriptstyle \mathcal{S}_{i,bj}}$, users must also disclose all behavior data before it.
Take an already split data $\delta_b ({\scriptstyle \mathcal{D}_{i,b}}) = \{\scriptstyle \mathcal{S}_{i,b1}, \scriptstyle \mathcal{S}_{i,b2}, \scriptstyle \mathcal{S}_{i,b3}\}$ as an example, the action space provided by oldest continuous strategy is $\Pi = \{[0,0,0], [1,0,0], [1,1,0], [1,1,1]\}$, and its corresponding disclosed data is $\{\varnothing ,
    \{\! {\scriptstyle \mathcal{S}_{i,b1}} \!\},
    \{\! {\scriptstyle \mathcal{S}_{i,b1}} , {\scriptstyle \mathcal{S}_{i,b2}}\! \},$
    $\{ {\scriptstyle \mathcal{S}_{i,b1}}, {\!\scriptstyle \mathcal{S}_{i,b2}},$ ${\scriptstyle \mathcal{S}_{i,b3}}\} \}$.
``\textit{Latest continuous}'' mechanism is similar to ``oldest continuous'', with the only difference in the opposite direction.
The size of these two mechanisms' action spaces is $|\delta_b ({\scriptstyle \mathcal{D}_{i,b}})|$.









\subsection{User Policy Simulation}
\label{sec:user}


In this subsection, we introduce the simulation of user policy in our proposed framework.
As defined in \cref{eq:S_i}, the disclosed data $\si{}$ is result of the platform mechanism $\mathrm{G}$ and user's disclosure policy $\pi_i$. 
Meanwhile, in \cref{eq:updated_rec}, the recommendation utility $\texttt{U}_i( \si{})=\texttt{U}'(\si{},\, \texttt{M}_{{\scriptscriptstyle \mathcal{S}}})$ is also determined by the recommendation model $\texttt{M}_{{\scriptscriptstyle \mathcal{S}}}$, which is \czq{built} upon the all users' disclosed data $\mathcal{S}$. 
The reward of user $i$ may be varied even when $i$ keeps the disclosed data $\si{}$ unchanged since other users might change their disclosed data and the recommender system is changed.
Thus, the expectation rewards are considered rather than immediate value defined in \cref{eq:framework} and we assume all the users are rational and seek for the optimal privacy disclosure action  $\alpha_i^*$ to the optimal expected reward $E[ \texttt{R}_i | \alpha_i ] $ as his policy, i.e., %
\begin{equation}
    \begin{aligned}
    \alpha_i^{*} &= \argmax_{\alpha_i \in \Pi} E[ \texttt{R}_i | \alpha_i ]=  \argmax_{\si{\in [ \Pi \otimes \mathrm{\delta}(\di{}) ] }} E[ \texttt{R}_i(\si{)} ] \\
& =\argmax_{\alpha_i \in \Pi } E\Bigl[ -\lambda_i \texttt{C}_i\bigl( \alpha_i \otimes \mathrm{\delta}(\di{}) \bigr) + \texttt{U}_i\bigl( \alpha_i \otimes \mathrm{\delta}(\di{}) )\bigr) \Bigr].
    \end{aligned}
\label{eq:opt_pi}
\end{equation}

As mentioned before, recommendation utility $\texttt{U}_i$ has been discussed in \cref{sec:platform_obj}.
To study this objective, we need to define the privacy cost function $\texttt{C}_i$ and sensitive weight $\lambda_i$.



\subsubsection{\textbf{Privacy Cost Function}}
\label{sec:privacy_cost}
We simulate every user with the same cost function $\texttt{C}$ and leave the diversity of user privacy sensitivity to the parameter $\lambda_i$. 
Following current experiment specifications in the economics literature~\cite{lin2019valuing,tang2019value}, we model the privacy cost function as a linear summation\footnote{See the Eq. 2 in \cite{lin2019valuing} and the dis-utility from disclosure in the econometric specification session in \cite{tang2019value}.} of disclosed personal data.


\czq{According to the comprehensive survey on privacy value definition \cite{MKT-053}, people will measure the value of their privacy into the intrinsic value of privacy and the instrumental value of privacy.}
\czq{
The intrinsic loss indicates the sake of protecting their intrinsic private data, which measures the valuation on the intrinsic properties such as the education or the income levels. }
\czq{In this work, we denote the intrinsic loss towards the privacy cost on amount of the sharing user profile attributes.}
\czq{The instrumental value of privacy indicates how the transaction efficiency would be affected by sharing user data, especially the data generated in the applications. 
In this work we denote the privacy cost towards the percentage of shard user historical behavior data. 
Therefore, the privacy cost function is described below,}
\begin{equation}
    \texttt{C}_i(\si{})= \beta_i * | {\scriptstyle \mathcal{S}_{i,a}} | + \frac{| {\scriptstyle \mathcal{S}_{i,b}} |}{ | {\scriptstyle \mathcal{D}_{i,b}} |}
    \label{eq:cost_function0}
\end{equation}
\czq{where the first term indicates the intrinsic loss and the second term indicates the instrumental loss.} 
\czq{If user does not tend to disclose profile attribute, such privacy cost function can be simplified to the following format with the instrumental value alone.}
As mentioned in \cref{sec:plat_mech}, user tends not to disclose profile attributes $\scriptstyle \mathcal{D}_{i,a}$ due to no gains in our experiments, so we only consider behavior data here, i.e.,
\begin{equation}
    \texttt{C}_i(\si{})=\texttt{C}(\si{}) =  \frac{| {\scriptstyle \mathcal{S}_{i,b}} |}{ | {\scriptstyle \mathcal{D}_{i,b}} |}
    , %
    \label{eq:cost_function}
\end{equation}
where the $|x|$ is the number of elements in $x$.
Here, the percentage based measurement regards different amount of users' data equally. %


This reduced form specification is not unrealistic as it captures the substitution effect among personal data and incorporates the idea of constant marginal privacy cost. 
One might argue for a higher order functional to capture richer implications. 
However, there is little experimental evidence that the higher order form for privacy cost exists and how the functional form looks like.






\subsubsection{\textbf{Privacy Sensitive Weight}}
\label{sec:user_type}
For user $i$ who disclosed all her/his data (i.e., $\si{} = \di{}$), her/his privacy cost compared to not sharing any data (i.e., $\si{} = \varnothing $) is 
\begin{equation}
    \texttt{C}(\di{}) - \texttt{C}(\varnothing).
\label{eq:privacy_diff}
\end{equation}
Meanwhile, her/his anticipated recommendation utility compared to not sharing any data is:
\begin{equation}
    \texttt{U}(\di{}) - \texttt{U}(\varnothing).
\label{eq:utility_diff}
\end{equation}
We assume all users have accessed to the recommendation utility $\texttt{U}(\di{})=\texttt{U}'(\di{},\texttt{M}_{{\scriptscriptstyle \mathcal{D}}})$ computed on all the data $\di{}$ and the recommendation utility without their data $\texttt{U}(\varnothing)$ before they can take data disclosing actions, which can be regard as a prior knowledge, like the experiences before the platform adopted our framework.
With \cref{eq:privacy_diff} and \cref{eq:utility_diff}, we define the privacy sensitive weight $\lambda_i$ as: 
\begin{equation}
    \lambda_i = w_i  * \frac{\texttt{U}(\di{})  - \texttt{U}(\varnothing) } {  \texttt{C}(\di{}) - \texttt{C}(\varnothing) },
    \label{marginal_define}
\end{equation}
where $w_i$ indicates the diversity of user types towards privacy sensitivity.
The users with $w_i > 1$ is privacy sensitive users, as they will not be willing to disclose the corresponding data $\di{}$ if they only get $\texttt{U}(\di{})$ as before.
While users  with $w_i < 1$ are just the opposite. %
Therefore, the user's privacy sensitive weight is pre-computed, and the $\texttt{U}(\di{})$ can be regarded as the benchmark expectation of the platform.
The formulation of the privacy sensitive weight $\lambda_i$ also meets the idea from \cite{lin2019valuing}, where the heterogeneity from users' social demographic variety should also be explicitly characterized. %



\subsubsection{\textbf{Simulation Algorithm}}

As users behave rationally to find the optimal strategy with a trade-off of exploration and exploitation, it just meets the idea of the reinforcement learning algorithm. 
Therefore, we model each user as a unique agent and apply a multi-agent reinforcement learning method to simulate user possible policy adaptation. 
The recommender system is regarded as the environment to provide feedback, which is built upon the disclosed user data.
All agents' policies are optimized simultaneously by determining their actions, i.e., the disclosed data $\scriptstyle \mathcal{S}^t$ at simulation epoch $t$, which is used to train the recommendation model $\texttt{M}_{{\scriptscriptstyle \mathcal{S}}^{t}}$.
As mentioned before, users tend to find an optimal action over possible action space $\Pi$ to maximize his expected reward, which is determined by all agents in this dynamic MARL environment. 


We assume each user (agent) realizes this situation that the immediate reward is the result of all agents, but no communication or observation among agents is permitted. 
Then, each agent is concerned about her/his own utility and regards the environment as a dynamic system that is partially correlated to herself/himself. 
Now, it is simplified to a Multi-Armed Bandit problem~\cite{katehakis1987multi}.


However, the challenge of the exploration and explication problem also exists in our simulation. 
To address it, we adopt a simple but efficient method, Epsilon Greedy~\cite{sutton2018reinforcement} algorithm, to simulate user's policy $\pi_i$ as following, %
\begin{equation}
     \alpha_i^{t+1} = \left\{
\begin{array}{l l }
\alpha_i \sim \texttt{P}^t_i,
& \text{with possibility } \epsilon \\
\argmax_{\alpha_i} Q_i^t(\alpha_i), & \text{with possibility }  1-\epsilon \\
\end{array} \right. 
    \label{epsilon_greedy}
\end{equation}
where $Q_i^t(\alpha)$ is the user $i$'s estimation value at simulation epoch $t$ on action $\alpha$, and $\texttt{P}^t_i$ denotes a random sample policy.
To conduct an efficient policy exploration, we sample a less explored action with a higher possibility as following,
\begin{equation}
    \texttt{P}^t_i(\alpha) = \frac{ 1/ (N^{t-1}_i(\alpha) +1) } { \sum_{x \in \Pi} 1/(N^{t-1}_i(x) +1) },
    \label{random_rule}
\end{equation}
where $N^{t-1}_i(\alpha)$ represents the total number of action $\alpha$ was taken by user $i$ from start to the last simulation epoch $t{-}1$.
In convenience, we adopt the approximated expected estimation results and 
update it with the residual between the estimation $Q_i^{t-1}(\alpha_i^{t-1})$ and immediate reward  $\texttt{R}_i^{t-1}$ when she/he takes action $\alpha_i^{t-1}$ as following.
\begin{equation*}
       Q_i^t(\alpha) {=} \left\{\!\!\!
\begin{array}{l l}
Q_i^{t-1}(\alpha), &  \text{if } \alpha_i^{t-1} {\neq} \alpha \\
Q_i^{t{-}1}(\alpha) {+} \frac{1}{N^t_i(\alpha)} \bigl(\texttt{R}_i^{t-1}(\! \alpha {\otimes} \mathrm{\delta}(\di{})  ) {-} Q_i^{t{-}1}(\alpha) \bigr), &  \text{if }  \alpha_i^{t{-}1} {=} \alpha \\
\end{array} \right. 
\end{equation*}
where $\texttt{R}_i^{t-1}$ is user $i$-th immediate objective at simulation epoch $t{-}1$, computed by \cref{eq:framework}. 
$Q_i^0(\alpha) $ is the user $i$'s initial expected reward if she/he takes action $\alpha$. 
which is initialized to $0$ as users have no prior about their behaviors on the new dynamic environment.


In our simulation, we set initial $\epsilon=0.5$ for all agents and decay a half during the MARL training processing. The detailed decay epoch is co-related to the size of possible action space $\Pi$.
Here, we define it as $\epsilon = 0.5^{ t /(3 * |\Pi |) }  $, 
where $t$ is the epoch during the reinforcement learning training processing. 

\subsection{\textbf{Discussion}}
To figure out how the platform mechanism affects users' behavior, we turn to the simulation built upon several simplified assumptions. 
One fundamental assumption is the hypothesis of rational man, where users will seek their optimal policies to maximize their objectives. 
However, in the real-world scenarios, human behaviors are also affected by psychological factors, which should also be modeled in future work.
One detailed example is that some users may feel exhausted digging out all the potential privacy choices with the provided platform mechanism.
In our simulation, we assume there remains no mental cost when a user adjusts his policy. 
However, in the reality, some users may refuse to change their policy frequently, especially in complex user interaction applications.
For such situation, a convenient user interface (UI) could be a potential solution to mitigate users' fatigues. 
\czq{
Another important factor is that users may adjust their trust level towards the platform during their exploration. One detailed example is that if the platform or even the recommender system \cite{zhang2022pipattack} is easy to be attacked or the platform will abuse their disclosed data to other applications, they may re-consider their privacy sensitivity. 
Though some works have discussed the utilization of trusted platform or the privacy-preserved recommendation model, the possible effects on user psychological factors might be tackled by a dynamic modeling on the user privacy sensitive weights, which is out of the scope of this work.
}
We simplify the influences of the psychological factors in this work and leave the exploration of psychological effects in mechanism designs and UI designs for future works. 





%%%%%%%%%%%%%%%%%%%%%%%%%%

\begin{comment}
\begin{figure}
\includegraphics[width=\linewidth]{figs/beyond_tss_lesion.pdf}
\caption[]{End-to-End runtime lesion study of the entire MNIST dataset and the FMA featurized music dataset. Each of DROP's contributions provides a runtime improvement.}
\label{fig:beyond_lesion}
\end{figure}
\end{comment}



\section{Conclusion}
\label{sec:conclusion}

Advanced data analytics techniques must scale to rising data volumes. 
DR techniques offer a powerful toolkit when processing these datasets, with PCA frequently outperforming popular techniques in exchange for high computational cost. 
In response, we propose DROP, a new dimensionality reduction optimizer. 
DROP combines progressive sampling, progress estimation, and online aggregation to identify high quality low dimensional bases via PCA without processing the entire dataset by balancing the runtime of downstream tasks and achieved dimensionality. 
Thus, DROP provides a first step in bridging the gap between quality and efficiency in end-to-end DR for downstream \red{analytics}. 

%We revisit canonical operators for time series dimensionality reduction and the measurement study of~\cite{keogh-study}, and show that PCA is more effective than popular alternatives in the data mining literature often by a margin of over $2\times$ on average on gold-standard time series benchmark data sets with respect to output data dimension. More surprisingly, we empirically demonstrate that a small number of samples are sufficient to accurately characterize directions of maximum variance and obtain a high-quality low-dimensional transformation.




\appendices

%\section{The FELT framework's components}
\label{app:felt_interactions}

The FELT framework introduced in Section \ref{sec:framework} presents an important overview of all the factors that influence feedback and are in turn influenced by it. Figure \ref{fig:felt_framework} showcased a schematic overview of the FELT framework, integrating four distinct components: Feedback, Errors, Learner, and Task. In this appendix, we will outline more precisely each of the components of the FELT Framework, as well as the interactions between them.

\subsection{Task}
Typically, the task will be the first element to be defined. 

\paragraph{Nature of Task} In this paper, we have limited tha nature of the task to the answer type. Understanding itis fairly easy -- a task has a closed-answer if there is a finite set of correct answers, and an open-answer otherwise. Notably, tasks can contain both elements. For example the task "\textit{Write a quality 4-paragraph short-story}" has both open- and closed-answer elements. There is no finite set of answer of what a quality story is, but whether a story has 4 paragraphs, or not, is a binary closed-answer, as seen in Appendix \ref{app:feedback_examples}.

\paragraph{Complexity} The difficulty level of a task is harder to define as some measure of relativity is involved. We suggest anchoring this measurement to the average adult human capabilities. A simple arithmetic task will thus be considered very easy, whereas researching and writing a doctoral thesis would be seen as hard.

\paragraph{Prompt Instructions} The task instructions will be presented to the model at two distinct points in time: when first assigning the model this task, and when later providing feedback. With regards to the former, this element captures the degree to which the task is explained -- is the model explicitly aware of all criteria it should satisfy? With regards to the second pass, when feedback is provided, this dimension pertains instead with the degree of freedom it gives the LLM -- is the model forced to take the feedback into account, or can it consider only part of it, or even disregard it altogether if it deems it useless?

\subsection{Learner}
Either at the same time the task is defined or immediately after, the model to be tested will be chosen. The model choice influences two important features.

\paragraph{Prior Knowledge} The prior knowledge captures the LLM's abilities as a direct result of its size, training data, and training method. These, in turn, also reflect the model's purpose (e.g., was it designed to be helpful, harmless, entertaining, etc.). The prior knowledge thus captures the model's  representation of the learner, and in its architecture and parametric knowledge, it encodes the LLM's current abilities -- or its proficiency -- both in general and with regards to the specific task.

\paragraph{Feedback Processing Mechanism} Mainly defined by the experimental setup, the mechanism by which the model process feedback can vary significantly, and not all of them are able to leverage the same level of information. Imitation learning, for example, can only leverage information which was positively evaluated. As stated in Section \ref{sec:framework}, we identify 4 main processing mechanisms, 3 of which alter the model's parametric state -- feedback-based imitation learning, joint-feedback modeling, and reinforcement learning, as defined in \citet{fernandes_bridging_2023} -- and a fourth, non-parametric mode: in-context learning \citep{brown2020language}.

\subsection{Errors}
After both the task and learner are in place, the first pass of the experiment can be run, where the model will have its first attempt at solving the task. In this attempt, it is expected that the model will make some degree of mistakes -- which have two important characteristics.

\paragraph{Error Type} There are several possible types of errors, and their differences are significant. For example, an error made due to a guess only needs to provide the learner with the right information for it be be corrected, whereas a systematic error (for example, the mixing of British and American English spellings) will require a different, much more insistent, intervention. ROSCOE \citep{golovneva2023roscoe} proposes a taxonomy of step-by-step reasoning errors. While task dependent (i.e., there are grammar errors and arithmetic errors, rather than fully task independent failure modes), this taxonomy provides a good starting ground for the exploration of error types in NLP.

\paragraph{Error Severity} Besides the type of error, it is also important to take the severity of the error into account. Stating that Marie Curie was a German philosopher and stating that she won one Nobel Prize in her lifetime are both factually inaccurate -- but one is a severe, complete hallucination, while the other omitted she actually won the Nobel Prize twice. The more severe the error, the stronger, more insistent, and more corrective the feedback should be.

\subsection{Feedback}
Finally, after the model has finished its first attempt at the task, producing some number of errors, feedback can be provided on this attempt.

\paragraph{Timing} One easy to neglect aspect of feedback that pedagogy has shown to be impactful is timing -- whether the feedback is provided immediately after a task is attempted or whether there is a delay between the two actions. There are differing opinions amongst education researchers, but how to make feedback content more effective through timing merit research in LLMs. For example, in line with \citet{Mathan2005FosteringTI} and \citet{narciss_feedback_2008}'s take on timing -- delay feedback if the learner possesses metacognitive abilities that allow them to identify and possibly correct mistakes -- we posit feedback will be more effective if, content-wise, it is preceded by information on the answer's correctness and mistakes' existence and only after this metacognitive priming is the rest of the information presented.

\paragraph{Content} Section \ref{sec:taxonomy} explores feedback content in depth, presenting 10 impactful axes on which it can vary: length, granularity, applicability of instructions, answer coverage, criteria, information novelty, purpose, style, valence, and mode. It also presents a set of 9 emergent categories which, based on pedagogical research, we estimate to be the most promising one with regards to impact on revised model generations, and thus most deserving of further study.

\paragraph{Source} Finally, it is also important to consider the source of feedback, which might be an authority, such as an expert, an average human, another LLM, a rule-based system, among others. Different sources will reflect different authority and reliability levels.

\subsection{Interactions}
With a clear understanding of all the components and sub-components of the FELT framework, we can explore the influences that exist between them.

Both the task complexity and the learner's prior knowledge can impact the ideal feedback timing -- be it delayed when the learner has metacognitive skills \citep{narciss_feedback_2008} or enough task proficiency \citep{mason_providing_2001} they can identify where the mistake occurred, or, for example, immediate if they don't \citep{narciss_feedback_2008} or the task difficulty is low \citep{mason_providing_2001}.

With regards to the feedback content, the type of task \citep{butler_feedback_1995, kluger_effects_1996, mason_providing_2001, anastasiya_a_lipnevich_david_a_g_berg_jeffrey_k_smith_toward_2016} and both the error type and severity will have an impact \citep{narciss_how_2004, narciss_feedback_2008}. The nature of the task (open or closed answer) will directly condition the feedback that can be given in response to the model's answer, as well as how difficult it will be to produce it. For example, generating the correct answer for a multiple choice quiz or a story writing task will be two very different endeavors. Similarly, it is impossible to provide response elaboration feedback on a single multiple choice question.
The error type and severity will also influence the feedback content, as apart from directly dictating what mistakes verification and elaboration feedback can be given, they will also condition the ideal amount of detail and explanations to address the mistake at the most efficient level.

Finally, all aspects of feedback will influence the learner's feedback processing mechanism \citep{kulhavy_feedback_1989, sadler_formative_1989, bangert-drowns_instructional_1991, butler_feedback_1995, kluger_effects_1996, narciss_how_2004, nicol_formative_2006, narciss_feedback_2008, anastasiya_a_lipnevich_david_a_g_berg_jeffrey_k_smith_toward_2016, carless_development_2018}. All three dimensions of feedback have evident potential to directly influence how the model processes them. The instruction's permissiveness to consider or discard feedback will also impact the learner's feedback processing mechanism. This processing is, of course, dependent on the specific processing mechanism employed, and while some might be indifferent to some of these components -- like imitation learning, for example, which focuses exclusively on the feedback content -- others will be sensitive to all, including the task's prompt instructions -- such as in-context learning.


%\section{Different feedback content types}
\label{app:feedback_examples}

In this appendix, we first present different examples of the nine feedback content categories for the summarization task used in the case study. Second, we present a more detailed description of each feedback type, to facilitate feedback categorization according to the categorical taxonomy.

\subsection{Task Introduction}
In this task, the model is instructed to summarize a news article about a research finding into a summary that can be understood by an adult who is not particularly familiar with the field. As more thoroughly described in Appendix \ref{app:case_study}, the summary should be engaging while also describing the problem, the finding and the venue of publication.

\subsection{Feedback Examples}
\label{app:categorial_examples}

Let us consider the following generated summary: 

\texttt{"Scientists at the University of Bern have used 3D computer simulations to reconstruct how protoplanet Vesta collided with other asteroids around a billion years ago, using data from NASA's 2011 Dawn probe. The simulations also reveal detailed information about Vesta's composition and properties, adding to our understanding of the solar system's evolution. The findings were presented in a study published in Nature magazine."}

Below is a possible example of each type of feedback that could be given in response:

\begin{enumerate}[label=\textbf{\arabic*.}]
\item \textbf{Global Verification:} Overall grade of 85\%.

\item \textbf{Response Verification:} Describes the approach and the motivation, and mentions the venue, but not particularly engaging and incomplete about authorship.

\item \textbf{Mistakes Verification:} No initial attempt at grabbing the reader's attention; inaccurate authorship attribution (incomplete).

\item \textbf{Correct Answer:} The correct answer is: Models boost the significance of image and measurement data from space missions and help to understand our solar system. A simulation of a double impact that occurred on the proto-planet Vesta one billion years ago allowed scientists to describe precisely the inner structure of the asteroid. A joint research from EPFL, Bern University, France and the United States is on the cover of Nature this week.

\item \textbf{Response Elaboration:} The summary is accurate overall, and the language employed is adequate for the target audience. It could, however, use improvement in several areas, such as being more attention grabbing, especially in its first sentence, and providing more detail about what was actually found, rather than generic phrases like "adding to our understanding of the solar system's evolution.". It also failed to capture it was a joint research project.

\item \textbf{Mistakes Elaboration:} The summary does not attempt to engage with the user and capture their attention, in order to make them curious to read it to its completion. Furthermore, it only mentions one of the universities collaborating in this research, in which several labs from different universities joined efforts.

\item \textbf{Task Elaboration:} A good and captivating summary should first grab the reader's attention and make them curious to learn more. It is then important to factually and precisely state what the problem is, why it is important, the proposed solution, and, if published or being divulged, disclose where the reader can find it.

\item \textbf{Procedural Elaboration:} While reading the article, it is important to identify the key aspects of the research -- what problem or research question was being studied, what approach was used to solve it, and what its contributions or applications are. It is also important to register who was behind the findings and where they were published. Finally, it is important to think about how to present this idea in the first sentence -- in a way that is engaging for the reader, getting their attention and making them curious to read the rest. 

\item \textbf{Metacognition Elaboration:} To achieve a given task, it is first necessary to understand it and the concepts involved. With this understanding, it is possible to identify the task’s goal. Then, one must determine the steps needed to achieve this goal. If needed, each of these steps can be further split into even smaller tasks. Finally, it can be helpful to establish timelines and deadlines for each of these tasks, so the goal is achieved in time.

\end{enumerate}

\subsection{Observations}

\paragraph{Information Overlap}
It is possible to observe some of these feedback types share some information, that is, that there is some overlap between different categories. This is a natural consequence from the task type. Indeed, this summarization task is not a fully open answer question. While there is significant answer space, there are some hard requirements, such as mentioning the research finding, its motivation, the publication venue, being engaging, and using an accessible language. Almost all these are binary requirements, which a summary either fulfills or fails to address. As such, there will naturally be some overlap with regards to them in the feedback information, as it is only possible to fully prevent it in a truly open-answer setting.


\paragraph{Feedback Effectiveness}
In Section \ref{sec:background_pedagogy}, \citet{hattie_power_2007}'s definition of effective feedback was presented. According to the authors, it should address three different information needs: where the learner is going (\textit{feed up}), how they are going (\textit{feed back}), and where to next (\textit{feed forward}). It is possible to relate these questions to the feedback categories exemplified above. 

The \textit{feed up} question, that is, the goal performance, can be implicitly derived from all feedback types that describe flaws of the current answer (by exclusion) or its merits (by inclusion). However, it is the \textit{Correct Answer} feedback category that directly and explicitly presents this information.

The \textit{feed back} question -- the learner's current performance -- is derived from all the verification feedback categories as well as from \textit{Response Elaboration} and \textit{Mistakes Elaboration}.

Finally, the \textit{feed forward} question, how the learner should proceed, is directly tackled by \textit{Procedural Elaboration} feedback.

This definition would then, disregard \textit{Task Elaboration} and \textit{Metacognition Elaboration} feedback categories as inefficient feedback. However, as the case study presented in Section \ref{sec:case_study} demonstrates, at least in the NLP domain, we cannot be so quick to dismiss them -- as in it, \textit{Task Elaboration} actually outperformed \textit{Correct Answer} feedback. Consequently, while the definition of effective feedback proposed by \citet{hattie_power_2007} might help researchers consider promising feedback types, it might also dismiss other pertinent pieces of information. For that reason, both \textit{Task Elaboration} and \textit{Metacognition Elaboration} are present in the categorical taxonomy proposed by this paper, despite their lack of clear mapping to any of \citet{hattie_power_2007}'s three questions.

\subsection{Mapping Feedback to a Categorical Type}
\label{app:categorial_mapping}

While examples of the nine feedback categories might be easy to understand, classifying novel pieces of feedback might prove challenging. Below, we provide a more exhaustive overview of the content of each type of feedback:


\begin{enumerate}[label=\textbf{\arabic*.}]
\item \textbf{Global Verification:} An aggregate score for the task as a whole. Cannot contain more than a single data point of information. The score need not be numeric (\eg ``\texttt{Satisfactory},'' ``\texttt{Grade: 65\%},'' and ``\texttt{C}'' are all valid examples of Global Verification feedback).

\item \textbf{Response Verification:} Granular response-classification feedback. Can either provide a score for several answer segments (\eg a unique score for each question on a quiz, or for each paragraph on a written text) or a score for several evaluative criteria (\eg evaluate the entire written text on readability, engagement, etc.). 

\item \textbf{Mistakes Verification:} Granular error-classification feedback. Can either simply state errors were committed or identify which types of errors are present in the submitted answer (it can mention the number of mistakes). 

\item \textbf{Correct Answer:} The correct answer, an expected solution or, for an open-ended task, a rewritten version of the submitted answer that fulfills the evaluative criteria (ideally with as few changes as possible).

\item \textbf{Response Elaboration:} An overview of the answer as a whole, incorporating feedback about the current level of performance of the student. It can choose to only mention part of it, focusing only on the learner's positive accomplishments or their shortcomings. It differs from mistakes elaboration as, though it can discuss shortcomings, it does not directly address mistakes.

\item \textbf{Mistakes Elaboration:} Detailed feedback about mistakes, including their location, thorough descriptions of their type and of their possible sources (\eg information about common mistakes and what misconceptions might lead to them). Can either be explicit or presented through hints or guiding questions. Note that no information on correcting mistakes is included as part of this feedback type, as these belong to the Procedural Elaboration feedback instead.

\item \textbf{Task Elaboration:} Clarifications about the task --- its type, requirements, constraints, sub-processes --- and relevant concepts and technical terms. Can either be explicit or presented through hints or guiding questions. Note, however, no information on task solving strategies is considered Task Elaboration type of feedback, as these belong to the Procedural Elaboration feedback instead.

\item \textbf{Procedural Elaboration:} Instructions on how to improve performance, be it through worked out examples, explanations on error correcting, or strategies for processing and solving the task. Can either be explicit or presented through hints or guiding questions.

\item \textbf{Metacognition Elaboration:} General strategies for learning and problem solving. This feedback cannot be directly related to the task being attempted by the learner. Can either be explicit or presented through hints or guiding questions.

\end{enumerate}

\section{Pedagogical models of feedback}
\label{app:feedback_definitions}

\subsection{Defining feedback}

Table \ref{tab:feedback_defintions} presents an overview of the various definitions of feedback put forward by several pedagogical works.


\begin{table*}[h]
  \centering
  \begin{tabularx}{\textwidth}{p{3.95cm}|X}
    \toprule
    Work  & Feedback Definition  \\
    \midrule
    \citet{ramaprasad_definition_1983} & Information which changes the gap between "the actual level and the reference level of a system parameter." This is quite a strict definition -- if the information does not change the gap, it is not considered feedback, and information about the actual level, the reference level and their comparison is needed beforehand. \\
    \midrule
    \citet{kulhavy_feedback_1989} & Refer to a previous definition of feedback, whereby feedback is considered "any of the numerous procedures that are used to tell a learner if an instructional response is right or wrong" \citep{kulhavy_feedback_1977}.
    \\
    \midrule
    \citet{sadler_formative_1989} & "Information about how successfully something has been or is being done." \\
    \midrule
    \citet{butler_feedback_1995} & A way to update the learner's internal state and knowledge, and subsequently task execution (a more learner-centric model of feedback). \\
    \midrule
    \citet{kluger_effects_1996} & The information provided by an external agent on one or more aspects of task performance. Note this excludes the learner as a possible source of feedback.  \\
    \midrule
    \citet{mason_providing_2001} & Feedback "is any message generated in response to a learner's action." \\
    \midrule
    \citet{narciss_how_2004, narciss_feedback_2008} & "All post-response information which informs the learner on his/her actual state of learning or performance in order to regulate the further process of learning in the direction of the learning standards strived for." \\
    \midrule
    \citet{nicol_formative_2006} & Information relating the learner's current state to the goal state (both with regards to learning as well as performance). Importantly, they consider students generate internal feedback and that the better they are at self-regulation, the better they will be at either generating or leveraging feedback. \\
    \midrule
    \citet{hattie_power_2007} & Information generated by an agent about the learner's understanding or their performance. \\
    \midrule
    \citet{evans_making_2013} & Feedback "includes all feedback exchanges generated within assessment design, occurring within and beyond the immediate learning context, being overt or covert (actively and/or passively sought and/or received), and importantly, drawing from a range of sources." \\
    \midrule
    \citet{anastasiya_a_lipnevich_david_a_g_berg_jeffrey_k_smith_toward_2016} & Feedback is information transmitted to the learner with the intent of changing their understanding and execution, in order to improve learning. \\
    \midrule
    \citet{carless_development_2018} & Feedback as the process through which the student understands and integrates information from various sources in order to improve their learning or performance (a more learner-centric perspective). \\
    \midrule
    \citet{lipnevich_review_2021} & Feedback "is information that includes all or several components: students’ current state, information about where they are, where they are headed and how to get there, and can be presented by different agents (i.e., peer, teacher, self, task itself, computer). This information is expected to have a stronger effect on performance and learning if it encourages students to engage in active processing." \\
    \bottomrule
  \end{tabularx}
  \caption{Different pedagogical works' definitions of feedback.}
  \label{tab:feedback_defintions}
\end{table*}


\subsection{Categorizing feedback}

\citet{kulhavy_feedback_1989} model feedback as having two components: the verification component, $f_v$, which is a simple discrete classification of the answer as correct or incorrect, and the elaboration component, $f_e$, consists of three elements:
\begin{enumerate}[itemsep=0.05em,label=(\roman*)]
    \item \textit{type}, whether the feedback contains information derived from the current task (task-specific), not from the task but from the relevant lesson (instruction-based), or beyond the relevant lesson, such as new information, examples or analogies not previously introduced (extra-instructional), 
    \item \textit{form}, the difference in structure between the feedback and instruction or task specification messages, requiring increased processing the less similar it is\footnote{The \textit{form} element does not apply to \textit{extra-instructional type} feedback, as there is no structural comparison point possible}, and
    \item \textit{load}, the total amount of information in the feedback - from a single "correct/incorrect" bit to including the correct answer to even more informative feedback accompanying it with an explanation, for example.
\end{enumerate}

\citet{mason_providing_2001} propose 8 feedback categories, arguing different types of feedback are best suited for different learner characteristics, taking into account the students' proficiency and prior knowledge, as well as the task difficulty.
The eight categories are:
\begin{enumerate}[itemsep=0.05em,label=(\roman*)]
    \item \textit{no-feedback}, which presents a single grade, 
    \item \textit{knowledge-of-response}, which analogously to the aforementioned verification component, indicates whether the given answer is correct or incorrect, 
    \item \textit{answer-until-correct}, an iterative variant of knowledge-of-response feedback, not allowing the student to progress until they have provided the correct answer, 
    \item \textit{knowledge-of-correct-response}, which provides the correct answer, 
    \item \textit{topic-contingent}, which provides both knowledge-of-response feedback and, analogously to \citet{kulhavy_feedback_1989}'s instruction-based type of feedback, provides general information about the topic of the task, where the learner might locate the correct answer, 
    \item \textit{response-contingent}, which similarly provides knowledge-of-response feedback as well as an explanation of why the answer is wrong or right (mapping it to \citet{kulhavy_feedback_1989}'s extra-instructional type of feedback), 
    \item \textit{bug-related}, providing knowledge-of-response feedback and bug-related feedback, which relies on rule sets to identify procedural errors, and
    \item \textit{attribute-isolation}, which provides knowledge-of-response feedback as well as information on the essential attributes of the relevant concept, focusing the learner on its key components.
\end{enumerate}

\citet{narciss_how_2004, narciss_feedback_2008} present a detailed and comprehensive feedback model, taking into account many learner and task characteristics. They also present a content-related feedback classification scheme, with eight categories: 
\begin{enumerate}[itemsep=0.05em,label=(\roman*)]
    \item \textit{Knowledge of performance (KP)}, analogous to \citet{mason_providing_2001}'s no-feedback and \citet{kulhavy_feedback_1989}'s verification component for a multiple-question task, presents the learner with an aggregate score (e.g., percentage or number of correct answers out of the total number of questions), 
    \item \textit{Knowledge of result/response (KR)}, directly mirrors \citet{mason_providing_2001}'s knowledge-of-response and \citet{kulhavy_feedback_1989}'s verification component for each question or task, classifying it as either correct or incorrect,
    \item \textit{Knowledge of the correct results (KCR)}, equivalent to \citet{mason_providing_2001}'s knowledge-of-correct-response, indicating the correct answer to the learner,
    \item \textit{Knowledge about task constraints (KTC)}, somewhat similar to \citet{mason_providing_2001}'s topic-contingent feedback, is elaboration feedback about the task, containing hints, examples or explanations about the type of task, its rules, sub-tasks, requirements and other constraints, 
    \item \textit{Knowledge about concepts (KC)}, containing some resemblance to \citet{mason_providing_2001}'s attribute-isolation feedback, is elaboration feedback on the relevant concepts, providing hints, examples or explanations on technical terms, the concept or its context, attributes, or key components, 
    \item \textit{Knowledge about mistakes (KM)}, which parallels \citet{mason_providing_2001}'s bug-related feedback, provides elaboration feedback containing the number of mistakes, their location, and hints, examples or explanations on error types and sources, 
    \item \textit{Knowledge about how to proceed (KH)}, elaboration feedback on the general know-how of the task, containing hints, examples or explanations on error correction, task-specific solving strategies or processing steps, guiding questions and worked-out examples, and
    \item \textit{Knowledge about metacognition (KMC)}, elaboration feedback going beyond the context of the current task, and providing hints, examples, explanations, or guiding questions on metacognitive strategies.
\end{enumerate}

\noindent \citet{hattie_power_2007} present a small typology about the information being conveyed about the learner in the feedback message, presenting 3 questions feedback can answer: 
\begin{enumerate}[itemsep=0.05em,label=(\roman*)]
    \item where the learner is going (\textit{feed up}), 
    \item how they are going (\textit{feed back}), and 
    \item where to next (\textit{feed forward})
\end{enumerate} 
and argue feedback is effective if it answers all three. 
\section{Different feedback content types}
\label{app:feedback_examples}

In this appendix, we first present different examples of the nine feedback content categories for the summarization task used in the case study. Second, we present a more detailed description of each feedback type, to facilitate feedback categorization according to the categorical taxonomy.

\subsection{Task Introduction}
In this task, the model is instructed to summarize a news article about a research finding into a summary that can be understood by an adult who is not particularly familiar with the field. As more thoroughly described in Appendix \ref{app:case_study}, the summary should be engaging while also describing the problem, the finding and the venue of publication.

\subsection{Feedback Examples}
\label{app:categorial_examples}

Let us consider the following generated summary: 

\texttt{"Scientists at the University of Bern have used 3D computer simulations to reconstruct how protoplanet Vesta collided with other asteroids around a billion years ago, using data from NASA's 2011 Dawn probe. The simulations also reveal detailed information about Vesta's composition and properties, adding to our understanding of the solar system's evolution. The findings were presented in a study published in Nature magazine."}

Below is a possible example of each type of feedback that could be given in response:

\begin{enumerate}[label=\textbf{\arabic*.}]
\item \textbf{Global Verification:} Overall grade of 85\%.

\item \textbf{Response Verification:} Describes the approach and the motivation, and mentions the venue, but not particularly engaging and incomplete about authorship.

\item \textbf{Mistakes Verification:} No initial attempt at grabbing the reader's attention; inaccurate authorship attribution (incomplete).

\item \textbf{Correct Answer:} The correct answer is: Models boost the significance of image and measurement data from space missions and help to understand our solar system. A simulation of a double impact that occurred on the proto-planet Vesta one billion years ago allowed scientists to describe precisely the inner structure of the asteroid. A joint research from EPFL, Bern University, France and the United States is on the cover of Nature this week.

\item \textbf{Response Elaboration:} The summary is accurate overall, and the language employed is adequate for the target audience. It could, however, use improvement in several areas, such as being more attention grabbing, especially in its first sentence, and providing more detail about what was actually found, rather than generic phrases like "adding to our understanding of the solar system's evolution.". It also failed to capture it was a joint research project.

\item \textbf{Mistakes Elaboration:} The summary does not attempt to engage with the user and capture their attention, in order to make them curious to read it to its completion. Furthermore, it only mentions one of the universities collaborating in this research, in which several labs from different universities joined efforts.

\item \textbf{Task Elaboration:} A good and captivating summary should first grab the reader's attention and make them curious to learn more. It is then important to factually and precisely state what the problem is, why it is important, the proposed solution, and, if published or being divulged, disclose where the reader can find it.

\item \textbf{Procedural Elaboration:} While reading the article, it is important to identify the key aspects of the research -- what problem or research question was being studied, what approach was used to solve it, and what its contributions or applications are. It is also important to register who was behind the findings and where they were published. Finally, it is important to think about how to present this idea in the first sentence -- in a way that is engaging for the reader, getting their attention and making them curious to read the rest. 

\item \textbf{Metacognition Elaboration:} To achieve a given task, it is first necessary to understand it and the concepts involved. With this understanding, it is possible to identify the task’s goal. Then, one must determine the steps needed to achieve this goal. If needed, each of these steps can be further split into even smaller tasks. Finally, it can be helpful to establish timelines and deadlines for each of these tasks, so the goal is achieved in time.

\end{enumerate}

\subsection{Observations}

\paragraph{Information Overlap}
It is possible to observe some of these feedback types share some information, that is, that there is some overlap between different categories. This is a natural consequence from the task type. Indeed, this summarization task is not a fully open answer question. While there is significant answer space, there are some hard requirements, such as mentioning the research finding, its motivation, the publication venue, being engaging, and using an accessible language. Almost all these are binary requirements, which a summary either fulfills or fails to address. As such, there will naturally be some overlap with regards to them in the feedback information, as it is only possible to fully prevent it in a truly open-answer setting.


\paragraph{Feedback Effectiveness}
In Section \ref{sec:background_pedagogy}, \citet{hattie_power_2007}'s definition of effective feedback was presented. According to the authors, it should address three different information needs: where the learner is going (\textit{feed up}), how they are going (\textit{feed back}), and where to next (\textit{feed forward}). It is possible to relate these questions to the feedback categories exemplified above. 

The \textit{feed up} question, that is, the goal performance, can be implicitly derived from all feedback types that describe flaws of the current answer (by exclusion) or its merits (by inclusion). However, it is the \textit{Correct Answer} feedback category that directly and explicitly presents this information.

The \textit{feed back} question -- the learner's current performance -- is derived from all the verification feedback categories as well as from \textit{Response Elaboration} and \textit{Mistakes Elaboration}.

Finally, the \textit{feed forward} question, how the learner should proceed, is directly tackled by \textit{Procedural Elaboration} feedback.

This definition would then, disregard \textit{Task Elaboration} and \textit{Metacognition Elaboration} feedback categories as inefficient feedback. However, as the case study presented in Section \ref{sec:case_study} demonstrates, at least in the NLP domain, we cannot be so quick to dismiss them -- as in it, \textit{Task Elaboration} actually outperformed \textit{Correct Answer} feedback. Consequently, while the definition of effective feedback proposed by \citet{hattie_power_2007} might help researchers consider promising feedback types, it might also dismiss other pertinent pieces of information. For that reason, both \textit{Task Elaboration} and \textit{Metacognition Elaboration} are present in the categorical taxonomy proposed by this paper, despite their lack of clear mapping to any of \citet{hattie_power_2007}'s three questions.

\subsection{Mapping Feedback to a Categorical Type}
\label{app:categorial_mapping}

While examples of the nine feedback categories might be easy to understand, classifying novel pieces of feedback might prove challenging. Below, we provide a more exhaustive overview of the content of each type of feedback:


\begin{enumerate}[label=\textbf{\arabic*.}]
\item \textbf{Global Verification:} An aggregate score for the task as a whole. Cannot contain more than a single data point of information. The score need not be numeric (\eg ``\texttt{Satisfactory},'' ``\texttt{Grade: 65\%},'' and ``\texttt{C}'' are all valid examples of Global Verification feedback).

\item \textbf{Response Verification:} Granular response-classification feedback. Can either provide a score for several answer segments (\eg a unique score for each question on a quiz, or for each paragraph on a written text) or a score for several evaluative criteria (\eg evaluate the entire written text on readability, engagement, etc.). 

\item \textbf{Mistakes Verification:} Granular error-classification feedback. Can either simply state errors were committed or identify which types of errors are present in the submitted answer (it can mention the number of mistakes). 

\item \textbf{Correct Answer:} The correct answer, an expected solution or, for an open-ended task, a rewritten version of the submitted answer that fulfills the evaluative criteria (ideally with as few changes as possible).

\item \textbf{Response Elaboration:} An overview of the answer as a whole, incorporating feedback about the current level of performance of the student. It can choose to only mention part of it, focusing only on the learner's positive accomplishments or their shortcomings. It differs from mistakes elaboration as, though it can discuss shortcomings, it does not directly address mistakes.

\item \textbf{Mistakes Elaboration:} Detailed feedback about mistakes, including their location, thorough descriptions of their type and of their possible sources (\eg information about common mistakes and what misconceptions might lead to them). Can either be explicit or presented through hints or guiding questions. Note that no information on correcting mistakes is included as part of this feedback type, as these belong to the Procedural Elaboration feedback instead.

\item \textbf{Task Elaboration:} Clarifications about the task --- its type, requirements, constraints, sub-processes --- and relevant concepts and technical terms. Can either be explicit or presented through hints or guiding questions. Note, however, no information on task solving strategies is considered Task Elaboration type of feedback, as these belong to the Procedural Elaboration feedback instead.

\item \textbf{Procedural Elaboration:} Instructions on how to improve performance, be it through worked out examples, explanations on error correcting, or strategies for processing and solving the task. Can either be explicit or presented through hints or guiding questions.

\item \textbf{Metacognition Elaboration:} General strategies for learning and problem solving. This feedback cannot be directly related to the task being attempted by the learner. Can either be explicit or presented through hints or guiding questions.

\end{enumerate}

\section{The FELT framework's components}
\label{app:felt_interactions}

The FELT framework introduced in Section \ref{sec:framework} presents an important overview of all the factors that influence feedback and are in turn influenced by it. Figure \ref{fig:felt_framework} showcased a schematic overview of the FELT framework, integrating four distinct components: Feedback, Errors, Learner, and Task. In this appendix, we will outline more precisely each of the components of the FELT Framework, as well as the interactions between them.

\subsection{Task}
Typically, the task will be the first element to be defined. 

\paragraph{Nature of Task} In this paper, we have limited tha nature of the task to the answer type. Understanding itis fairly easy -- a task has a closed-answer if there is a finite set of correct answers, and an open-answer otherwise. Notably, tasks can contain both elements. For example the task "\textit{Write a quality 4-paragraph short-story}" has both open- and closed-answer elements. There is no finite set of answer of what a quality story is, but whether a story has 4 paragraphs, or not, is a binary closed-answer, as seen in Appendix \ref{app:feedback_examples}.

\paragraph{Complexity} The difficulty level of a task is harder to define as some measure of relativity is involved. We suggest anchoring this measurement to the average adult human capabilities. A simple arithmetic task will thus be considered very easy, whereas researching and writing a doctoral thesis would be seen as hard.

\paragraph{Prompt Instructions} The task instructions will be presented to the model at two distinct points in time: when first assigning the model this task, and when later providing feedback. With regards to the former, this element captures the degree to which the task is explained -- is the model explicitly aware of all criteria it should satisfy? With regards to the second pass, when feedback is provided, this dimension pertains instead with the degree of freedom it gives the LLM -- is the model forced to take the feedback into account, or can it consider only part of it, or even disregard it altogether if it deems it useless?

\subsection{Learner}
Either at the same time the task is defined or immediately after, the model to be tested will be chosen. The model choice influences two important features.

\paragraph{Prior Knowledge} The prior knowledge captures the LLM's abilities as a direct result of its size, training data, and training method. These, in turn, also reflect the model's purpose (e.g., was it designed to be helpful, harmless, entertaining, etc.). The prior knowledge thus captures the model's  representation of the learner, and in its architecture and parametric knowledge, it encodes the LLM's current abilities -- or its proficiency -- both in general and with regards to the specific task.

\paragraph{Feedback Processing Mechanism} Mainly defined by the experimental setup, the mechanism by which the model process feedback can vary significantly, and not all of them are able to leverage the same level of information. Imitation learning, for example, can only leverage information which was positively evaluated. As stated in Section \ref{sec:framework}, we identify 4 main processing mechanisms, 3 of which alter the model's parametric state -- feedback-based imitation learning, joint-feedback modeling, and reinforcement learning, as defined in \citet{fernandes_bridging_2023} -- and a fourth, non-parametric mode: in-context learning \citep{brown2020language}.

\subsection{Errors}
After both the task and learner are in place, the first pass of the experiment can be run, where the model will have its first attempt at solving the task. In this attempt, it is expected that the model will make some degree of mistakes -- which have two important characteristics.

\paragraph{Error Type} There are several possible types of errors, and their differences are significant. For example, an error made due to a guess only needs to provide the learner with the right information for it be be corrected, whereas a systematic error (for example, the mixing of British and American English spellings) will require a different, much more insistent, intervention. ROSCOE \citep{golovneva2023roscoe} proposes a taxonomy of step-by-step reasoning errors. While task dependent (i.e., there are grammar errors and arithmetic errors, rather than fully task independent failure modes), this taxonomy provides a good starting ground for the exploration of error types in NLP.

\paragraph{Error Severity} Besides the type of error, it is also important to take the severity of the error into account. Stating that Marie Curie was a German philosopher and stating that she won one Nobel Prize in her lifetime are both factually inaccurate -- but one is a severe, complete hallucination, while the other omitted she actually won the Nobel Prize twice. The more severe the error, the stronger, more insistent, and more corrective the feedback should be.

\subsection{Feedback}
Finally, after the model has finished its first attempt at the task, producing some number of errors, feedback can be provided on this attempt.

\paragraph{Timing} One easy to neglect aspect of feedback that pedagogy has shown to be impactful is timing -- whether the feedback is provided immediately after a task is attempted or whether there is a delay between the two actions. There are differing opinions amongst education researchers, but how to make feedback content more effective through timing merit research in LLMs. For example, in line with \citet{Mathan2005FosteringTI} and \citet{narciss_feedback_2008}'s take on timing -- delay feedback if the learner possesses metacognitive abilities that allow them to identify and possibly correct mistakes -- we posit feedback will be more effective if, content-wise, it is preceded by information on the answer's correctness and mistakes' existence and only after this metacognitive priming is the rest of the information presented.

\paragraph{Content} Section \ref{sec:taxonomy} explores feedback content in depth, presenting 10 impactful axes on which it can vary: length, granularity, applicability of instructions, answer coverage, criteria, information novelty, purpose, style, valence, and mode. It also presents a set of 9 emergent categories which, based on pedagogical research, we estimate to be the most promising one with regards to impact on revised model generations, and thus most deserving of further study.

\paragraph{Source} Finally, it is also important to consider the source of feedback, which might be an authority, such as an expert, an average human, another LLM, a rule-based system, among others. Different sources will reflect different authority and reliability levels.

\subsection{Interactions}
With a clear understanding of all the components and sub-components of the FELT framework, we can explore the influences that exist between them.

Both the task complexity and the learner's prior knowledge can impact the ideal feedback timing -- be it delayed when the learner has metacognitive skills \citep{narciss_feedback_2008} or enough task proficiency \citep{mason_providing_2001} they can identify where the mistake occurred, or, for example, immediate if they don't \citep{narciss_feedback_2008} or the task difficulty is low \citep{mason_providing_2001}.

With regards to the feedback content, the type of task \citep{butler_feedback_1995, kluger_effects_1996, mason_providing_2001, anastasiya_a_lipnevich_david_a_g_berg_jeffrey_k_smith_toward_2016} and both the error type and severity will have an impact \citep{narciss_how_2004, narciss_feedback_2008}. The nature of the task (open or closed answer) will directly condition the feedback that can be given in response to the model's answer, as well as how difficult it will be to produce it. For example, generating the correct answer for a multiple choice quiz or a story writing task will be two very different endeavors. Similarly, it is impossible to provide response elaboration feedback on a single multiple choice question.
The error type and severity will also influence the feedback content, as apart from directly dictating what mistakes verification and elaboration feedback can be given, they will also condition the ideal amount of detail and explanations to address the mistake at the most efficient level.

Finally, all aspects of feedback will influence the learner's feedback processing mechanism \citep{kulhavy_feedback_1989, sadler_formative_1989, bangert-drowns_instructional_1991, butler_feedback_1995, kluger_effects_1996, narciss_how_2004, nicol_formative_2006, narciss_feedback_2008, anastasiya_a_lipnevich_david_a_g_berg_jeffrey_k_smith_toward_2016, carless_development_2018}. All three dimensions of feedback have evident potential to directly influence how the model processes them. The instruction's permissiveness to consider or discard feedback will also impact the learner's feedback processing mechanism. This processing is, of course, dependent on the specific processing mechanism employed, and while some might be indifferent to some of these components -- like imitation learning, for example, which focuses exclusively on the feedback content -- others will be sensitive to all, including the task's prompt instructions -- such as in-context learning.

\section{Case Study Implementation Details}
\label{app:case_study}

\subsection{Dataset and Model}
The task presented in section \ref{sec:case_study} involved the summarizing of a news article describing a research development at EPFL in more approachable language and terminology. The summary had as an objective to be even more approachable to people outside the field. 

\paragraph{Dataset} The data for this task came from EPFL's Mediacom department, where they provided the authors with a set of 2370 entries of articles, summaries, and extra information (title, author, date, etc.). Out of these, 50 were chosen so that all articles relayed a newly published work. This was the only criteria for selection. All articles and summaries were pre-processed so as to remove HTML tags.

\paragraph{Model} The model used for this task was GPT-4 \citep{openai2023gpt4}, with its default hyperparameters, called through OpenAI's Chat Completions API. Thus, the model generated a single completion for each prompt, with a temperature of $1$, and with no limitation on the maximum number of tokens (beyond the model's own context length).

\subsection{Experiment Execution}
The experiment was run in two stages. In the initial phase, the model is asked to generate a first summary. It is then provided with feedback and asked to revise its original summary. In this experiment, two distinct types of feedback were provided: \textit{Task Elaboration} (TE) and \textit{Correct Answer} (CA).

\paragraph{Initial Generation}
Following OpenAI's Chat Completions API, the model prompting is done under a chat format. In this setting, the first \textit{message} is a system message stating \texttt{You are a helpful assistant.} This is then followed by an user message, with the following prompt: 

\texttt{Summarize the following article into a short but captivating snippet under around 100 tokens. It must describe both the problem and the approach used to solve it, as well as the venue where these findings were presented, whenever this information is available. \\
Article: [article body]}

The model's response message to this prompt is considered its original summary.

\paragraph{Revised Generation} The revised generation prompt is shares the chat prompting format. It contains the previous chat history, which includes not only the two messages outlined above but also the model's answer as an assistant message. To these three messages, a new user message is added, with the following content:

\texttt{Feedback: [feedback] \\
Please revise your original summary taking the feedback into consideration. If you feel the feedback is not appropriate or useful, you can disregard it.}

The \texttt{[feedback]} placeholder will have one of two different values, depending on the feedback type being provided:
\begin{itemize}
    \item \textbf{Correct Answer:} The feedback will be of the form \\
    \texttt{The correct answer is: [gold\_summary]} \\
    where \texttt{[gold\_summary]} is the summary provided by the Mediacom dataset,
    
    \item \textbf{Task Elaboration:} The feedback will be of the form \\
    \texttt{A good and captivating summary should first grab the reader's attention and make them curious to learn more. It is then important to factually and precisely state what the problem is, why it is important, the proposed solution, and, if published or being divulged, disclose where the reader can find it.}
\end{itemize}

Finally, as in the first stage, the model's response message to this prompt is considered its revised summary.

\subsection{Example Outputs}
In this subsection, we present a few examples outputs from the case study.

\subsubsection{Example 1}
\paragraph{\colorbox{YellowOrange}{Original Article}}
\texttt{The International Consortium of Investigative Journalists (ICIJ), which has over 200 members in 70 countries, has broken a number of important stories, particularly ones that expose medical fraud and tax evasion. One of its most famous investigations was the Panama Papers, a trove of millions of documents that revealed the existence of several hundred thousand shell companies whose owners included cultural figures, politicians, businesspeople and sports personalities. To complete an investigation of this size is only possible through international cooperation between journalists. When sharing such sensitive files, however, a leak can jeopardize not only the story’s publication, but also the safety of the journalists and sources involved. At the ICIJ’s behest, EPFL’s Security and Privacy Engineering (SPRING) Lab recently developed Datashare Network, a fully anonymous, decentralized system for searching and exchanging information. A paper about it will be presented during the Usenix Security Symposium, a worldwide reference for specialists, which will be held online from 12 to 14 August. \\
Anonymity at every stage \\
Anonymity is the backbone of the system. Users can search and exchange information without revealing their identity, or the content of their queries, either to colleagues or to the ICIJ. The Consortium ensures that the system is running properly but remains unaware of any information exchange. It issues virtual secure tokens that journalists can attach to their messages and documents to prove to others that they are Consortium members. A centralized file management system would be too conspicuous a target for hackers; since the ICIJ does not have servers in various jurisdictions, documents are typically stored on its members’ servers or computers. Users provide only the elements that enable others to link to their investigation. \\
Users searching for information enter keywords in the search engine. If the search produces hits, they can then contact colleagues – whose identity remains protected – who are in possession of potentially relevant documents. Search queries are sent encrypted to all users, if there is a macth the querier gets an alert and can decide whether they wish to enter in contact and share information. “Given the fact that users work in different time zones, some with only a few hours of internet access per day, it was critical that searches and responses could take place asynchronously,” notes Carmela Troncoso, who runs the SPRING Lab at the School of Computer and Communication Sciences (IC). Another messaging system, also secure and anonymous, is subsequently used for two-way exchanges. \\
Two completely new secure applications \\
“This system, which addresses real-world needs, has enabled SPRING to tackle some interesting challenges,” notes Troncoso. The research team drew on existing authentication mechanisms and anonymous communication primitives, which they then optimized. They also developed two completely new secure building blocks that are described in the paper being published today in connection with the Usenix Security Symposium – an asynchronous search engine and a messaging system. A new protocol, known as “multi-set private set intersection” (MS-PSI), ensures the security of the search engine, allowing users to easily search a large number of databases without increasing the risk of leaks. The messaging system relies on a large number of single-use virtual mailboxes and is based on the well-known “pigeonhole” system, which chooses one option at random, in this case one of the mailboxes. Currently, the system does not allow users to exchange documents. “At this stage in the process, journalists are using other secure messaging systems,” Troncoso says. \\
Working with the Consortium has allowed SPRING to frame new requirements that are rarely examined in the scientific literature. Datashare can be scaled to thousands of users and millions of documents while encrypting all communications. “The hurdles we encountered during the development process, however, have paved the way to a new area of research with significant potential for other fields,” Troncoso concludes.}


\paragraph{\colorbox{yellow}{Golden Summary}}
\texttt{An EPFL laboratory has developed Datashare Network, a decentralized search engine paired with a secure messaging system that allows investigative journalists to exchange information securely and anonymously. An scientific article on this subject will be presented during the Usenix Security Symposium which will be held online from August 12 to 14.}


\paragraph{\colorbox{pink}{GPT4 Original Summary}}
\texttt{EPFL's Security and Privacy Engineering Lab developed an anonymous, decentralized system called Datashare Network to aid the International Consortium of Investigative Journalists securely exchange information. Offering user anonymity, this system will be presented at the Usenix Security Symposium held online from 12-14 August.}

\paragraph{\colorbox{SkyBlue}{\textit{Correct Answer} Revision}}
\texttt{EPFL's Security and Privacy Engineering Lab created Datashare Network, a decentralized search engine with a secure messaging system that enables investigative journalists to exchange information anonymously and securely. A scientific paper on this solution will be showcased at the online Usenix Security Symposium from August 12 to 14.}

\paragraph{\colorbox{SpringGreen}{\textit{Task Elaboration} Revision}}
\texttt{Investigative journalism goes undercover with Datashare Network, a breakthrough anonymous, decentralized information-sharing system. Developed by EPFL's Security and Privacy Engineering Lab, the network protects journalists from exposure while tackling global issues like tax evasion and medical fraud. Discover how it secures data-sharing at the Usenix Security Symposium, held online from 12-14 August.}



\subsubsection{Example 2}
\paragraph{\colorbox{YellowOrange}{Original Article}}
\texttt{Thermoelectric materials in particular hold vast potential for use in energy applications because they generate electricity from waste heat, such as that generated by industrial processes, by car and truck engines, or simply by the sun. Reducing the thermal conductivity of these materials by a factor of three, for example, would completely revolutionize existing waste-heat recovery, and also all refrigeration and air-cooling technology. \\
A unique theory for all insulating materials \\
In the paper Unified theory of thermal transport in crystals and glasses, out in Nature Physics, Michele Simoncelli, a PhD student at EPFL’s Theory and Simulation of Materials (THEOS) Laboratory – together with Nicola Marzari, a professor at EPFL’s School of Engineering and head of THEOS and of the MARVEL NCCR, and Francesco Mauri, a professor at the University of Rome–Sapienza – present a novel theory that finally decodes the fundamental, atomistic origin of heat conduction. Up to now, different formulations needed to be used depending on the systems studied (e.g., ordered materials, like a silicon chip, or disordered, like in a glass), and there wasn’t a unified picture covering all possible cases. \\
This has now been made possible by deriving directly from the quantum mechanics of dissipative systems a transport equation that covers on equal footing diffusion, hopping, and tunneling of heat.
Waste heat recovery \\
This fundamental understanding will allow scientists and engineers to accurately predict the thermal conductivity of any insulating material (in metals, the heat is carried by the electrons, and that is well understood) – this is exceedingly important for thermoelectrics (i.e. materials that can convert heat into electricity), since these have both crystal- and glass-like properties, and are much needed for waste-heat recovery, or for refrigeration without greenhouse gases (and if you think refrigeration is boring, it is worth remembering that Albert Einstein spent many years trying to invent a new form of refrigerator). \\
In order to develop such next-generation technology, however, scientists first need to understand how and to what extent materials conduct heat. “Up to now, two different equations have been used for calculating thermal properties: one describes perfectly crystalline materials – that is, materials with highly ordered atomic structures – and the other one completely amorphous materials like glass, whose atoms do not follow an ordered pattern,” says Michele Simoncelli. These equations happened to work well in those special cases. “But between these two extremes lie a plethora of interesting cases, and neither equation worked – this is really where our contribution makes a profound difference” \\
* Illustration : Heat is the energy due to vibrations of atoms. The ridges represent the quantum vibrational energy levels of CsPbBr3, a promising thermoelectric material. Heat conduction originates from both diffusion of vibrational excitations (blurred spheres, following the energy level) and quantum tunnelling (jump between different energy levels, represented by blue waves).}

\paragraph{\colorbox{yellow}{Golden Summary}}
\texttt{EPFL and MARVEL researchers have developed a new theory for heat conduction that can finally describe and predict the thermal conductivity of any insulating material. This new formulation will let scientists make accurate predictions of thermal conductivity in a range of materials for critical engineering applications – from electronics to lasers to waste-heat recovery.}

\paragraph{\colorbox{pink}{GPT4 Original Summary}}
\texttt{Scientists at EPFL's Theory and Simulation of Materials Laboratory have developed a unified theory to predict the thermal conductivity of insulating materials, opening up possibilities for more efficient energy applications. The research breakthrough, published in Nature Physics, could lead to waste-heat recovery, revolutionising refrigeration and air-cooling technology without greenhouse gases. Understanding how materials conduct heat is key to developing next-generation thermoelectric technology.}

\paragraph{\colorbox{SkyBlue}{\textit{Correct Answer} Revision}}
\texttt{Researchers at EPFL and MARVEL have developed a groundbreaking theory for heat conduction, enabling the description and prediction of thermal conductivity in any insulating material. Published in Nature Physics, the new formulation offers potential for critical engineering applications such as waste-heat recovery, electronics, and lasers, enhancing energy efficiency in various industries.}

\paragraph{\colorbox{SpringGreen}{\textit{Task Elaboration} Revision}}
\texttt{Researchers have unlocked the secret to more efficient energy applications by developing a unified theory for predicting the thermal conductivity of insulating materials, transforming waste-heat recovery and revolutionizing refrigeration technology. With this innovative approach, published in Nature Physics, scientists can now accurately predict and optimize thermoelectric materials, paving the way for energy-efficient solutions without greenhouse gas emissions. Discover the full potential of this game-changing breakthrough at EPFL's Theory and Simulation of Materials Laboratory.}




\subsubsection{Example 3}
\paragraph{\colorbox{YellowOrange}{Original Article}}
\texttt{With its 500 km diameter, the asteroid Vesta is one of the largest known planet embryos. It came into existence at the same time as the Solar System. Spurring scientific interest, NASA sent the Dawn spacecraft on Vesta’s orbit for one year between July 2011 and July 2012. \\
Data gathered by Dawn were analyzed by a team of researchers from EPFL as well as the Universities of Bern (Switzerlanf), Brittany (France) and Arizona (USA). Conclusion : the asteroid's crust is almost three times thicker than expected. The study does not only have implications for the structure of this celestial object, located between Mars and Jupiter. Their results challenge a fundamental component in planet formation models, namely the composition of the original cloud of matter that aggregated together, heated, melted and then crystallized to form planets. \\
At EPFL’s Earth and Planetary Science Laboratory (EPSL), led by Philippe Gillet, Harold Clenet had a look at the composition of the rocks scattered across Vesta’s ground. "What is striking is the absence of a particular mineral, olivine, on the asteroid’s surface," said the researcher. Olivine is a main component of planetary mantles and should have been found in large quantities on the surface of Vesta, due to a double meteorite impact which, according to computer simulations, "dug" the celestial body’s southern pole to a depth of 80 km, catapulting large amounts of materials to the surface. \\
The two impacts were so powerful that more than 5 \% of Earth's meteorites come from Vesta. « But these cataclysms were not strong enough to pierce through the crust and reach the asteroid's mantle,» Harold Clenet continued. The meteorites originating from Vesta and found on Earth confirm this since they generally lack Olivine, or contain only minute amounts compared to the amount observed in planetary mantles. Also, the spacecraft Dawn did not find olivine in the vicinity of the two impact craters. « This means that the crust of the asteroid is not 30 km thick, as suggested by the models, but more than 80 km. » \\
Composition of planets \\
These discoveries challenge models that describe the formation of Vesta, and consequently the formation of rocky planets in the Solar System including planet Earth. Cooling theory and "re- melting" phenomena in the depths of previously solidified elements would also need to be reviewed. "The crust might have been thickened by the formation of "plutons", that is: igneous rock intrusions, hundreds of meters large, some of which emerged to the surface," explained the scientist. \\
If Vesta has less of a (olivine-rich) mantle and more of a (pyroxene-rich) crust, then the proportion of materials making up Vesta, and probably the Earth and other telluric planets (Mars, Venus, Mercury), is different from what was previously expected. \\
A more complex model of planet formation therefore has to be considered, one that takes into account not only the original composition of planets, but also their orbits, sizes and related cooling times. Vesta is the only known asteroid that has an earth-like structure - with a core, mantle and crust – making it an incredible laboratory for testing hypotheses and theories.}

\paragraph{\colorbox{yellow}{Golden Summary}}
\texttt{EPFL researchers have a better understanding of the asteroid Vesta and its internal structure, thanks to numerical simulations and data from the space mission Dawn. Their findings, published today in Nature, question contemporary models of rocky planet formation, including that of Earth.}

\paragraph{\colorbox{pink}{GPT4 Original Summary}}
\texttt{Data collected from asteroid Vesta poses challenges to planet formation models as it reveals the asteroid's crust is almost three times thicker than expected. The discovery, made by researchers from the Earth and Planetary Science Laboratory and partner universities, suggests that original assumptions about the composition of the cloud of matter that formed planets may be incorrect. The findings could lead to a more complex model of planet formation, taking into account composition, orbit, size, and cooling times.}

\paragraph{\colorbox{SkyBlue}{\textit{Correct Answer} Revision}}
\texttt{EPFL researchers have gained insights into asteroid Vesta's internal structure, utilizing numerical simulations and data from the Dawn space mission. The findings, published in Nature, indicate Vesta's crust is almost three times thicker than expected, questioning current models of rocky planet formation, including Earth's. This could lead to more complex planet formation models that consider composition, orbit, size, and cooling times.}

\paragraph{\colorbox{SpringGreen}{\textit{Task Elaboration} Revision}}
\texttt{Astonishing findings from asteroid Vesta are challenging our understanding of planet formation, as researchers discovered its crust is nearly triple the expected thickness. This breakthrough, made by EPFL's Earth and Planetary Science Laboratory and partner universities, questions key elements of planet formation models and paves the way for a more complex approach, considering composition, orbits, sizes, and cooling times. With Vesta being the only known asteroid with an Earth-like structure, this revelation creates a compelling lure for further explorations.}




\bibliographystyle{IEEEtran}
\bibliography{REF}
%\begin{IEEEbiography}{Zarrin Montazeri}
%Biography text here.
%\end{IEEEbiography}
%
%
%
%\begin{IEEEbiography}{Amir Houmansadr}
%Biography text here.
%\end{IEEEbiography}
%
%\begin{IEEEbiography}{Hossein Pishro-Nik}
%Biography text here.
%\end{IEEEbiography}







% that's all folks
 \end{document}
