\section{Lemma \ref{lemx} and its Proof}
\label{sec:app_a}
Here we state that we can condition on high-probability events.

\begin{lem}
	\label{lemx}
	Let $p \in (0,1)$, and $X \sim Bernoulli (p)$ be defined on a probability space $(\Omega, \mathcal{F}, P)$. Consider $B_1, B_2, \cdots $ be a sequence of events defined on the same probability space such that $P(B_n) \rightarrow 1$ as $n$ goes to infinity. Also, let $\textbf{Y}$ be a random vector (matrix) in the same probability space, then:
	\[I(X; \textbf{Y}) \rightarrow 0\ \ \text{iff}\ \  I(X; \textbf{Y} {|} B_n) \rightarrow 0. \]
\end{lem}

\begin{proof}
First, we prove that as n becomes large,
\begin{align}\label{eq:H1}
H(X {|} B_n)- H(X) \rightarrow 0.
\end{align}

Note that as $n$ goes to infinity,
\begin{align}
\no P\left(X=1\right) &=P\left(X=1 \bigg{|} B_n\right) P\left(B_n\right) + P\left(X=1 \bigg{|} \overline{B_n}\right) P\left(\overline{B_n}\right)\\
%\no   &=P\left(X=1 \bigg{|} B_n\right)\left(1-o(1)\right) + P\left(X=1 \bigg{|} \overline{B_n}\right)o(1)\\
%\no   & \rightarrow  P\left(X=1 \bigg{|} B_n\right)+ o(1)  \ \ (\textrm{as } n \rightarrow \infty) \\
\no &=P\left(X=1 \bigg{|} B_n\right),\ \
\end{align}
thus,
$\left(X \bigg{|} B_n\right) \xrightarrow{d} X$, and as $n$ goes to infinity,
 \[H\left(X {|} B_n\right)- H(X) \rightarrow 0.\]
Similarly, as $n$ becomes large,
\[ P\left(X=1 \bigg{|} \textbf{Y}=\textbf{y}\right) \rightarrow P\left(X=1 \bigg{|} \textbf{Y}=\textbf{y}, B_n\right),\ \
\]
and
\begin{align}\label{eq:H2}
H\left(X {|}  \textbf{Y}=\textbf{y},B_n\right)- H\left(X {|} \textbf{Y}=\textbf{y}\right) \rightarrow 0.
\end{align}
%Thus $(X {|}  Y_n=y,B_n) \xrightarrow{D} (X {|} Y_n=y)$, and so $H(X {|}  Y_n=y,B_n)- H(X {|} Y_n=y) \rightarrow 0 $ as $n \rightarrow \infty$.
Remembering that
\begin{align}\label{eq:H3}
I\left(X; \textbf{Y}\right)=H(X)-H(X {|} \textbf{Y}),
\end{align}
and using (\ref{eq:H1}), (\ref{eq:H2}), and (\ref{eq:H3}), we can conclude that as  $n$ goes to infinity,
\[I\left(X;\textbf{Y} {|} B_n\right) - I\left(X,\textbf{Y}\right) \rightarrow 0.\]
As a result, for large enough $n$,
\[I\left(X; \textbf{Y}\right) \rightarrow 0 \Longleftrightarrow I\left(X; \textbf{Y} {|} B_n\right) \rightarrow 0. \]
\end{proof}
