
 \documentclass[journal,12pt,draftclsnofoot, onecolumn]{IEEEtran}
\usepackage{ifpdf}
\usepackage{cite}
\usepackage[pdftex]{graphicx}
\usepackage{amsmath}
\usepackage{algorithmic}
\usepackage{array}
\usepackage{subfigure}
%\usepackage{mathtools}
\usepackage{caption}
\usepackage[caption=false,font=footnotesize]{subfig}
\usepackage{fixltx2e}
\usepackage{stfloats}
\usepackage{url}
\usepackage{amsthm}
\usepackage{amsmath}
\usepackage{amsfonts}
\usepackage{amssymb}
\usepackage[T1]{fontenc} % optional
\usepackage{amsmath}
\usepackage[cmintegrals]{newtxmath}
\usepackage{bm} % optional
\usepackage[all]{xy}
\usepackage{enumitem}
\usepackage{float}
\usepackage{amsmath}
\interdisplaylinepenalty=2500
\usepackage[usenames, dvipsnames]{color}
\usepackage{ifpdf}
\usepackage{cite}
\usepackage[pdftex]{graphicx}
\usepackage{amsmath}
\usepackage{algorithmic}
\usepackage{array}
\usepackage{mathtools}
\DeclarePairedDelimiter{\abs}{\lvert}{\rvert}

\usepackage{caption}
\usepackage[caption=false,font=footnotesize]{subfig}
\usepackage{fixltx2e}
\usepackage{stfloats}
\usepackage{url}
\usepackage{mathtools}
\usepackage{amsthm}
\usepackage{amsmath}
\usepackage{amsfonts}
\usepackage{amssymb}
\usepackage[T1]{fontenc} % optional
\usepackage{amsmath}
\usepackage[cmintegrals]{newtxmath}
\usepackage{bm} % optional
\usepackage[all]{xy}
\usepackage{enumitem}
\usepackage{float}
\usepackage{amsmath}
\interdisplaylinepenalty=2500
%\usepackage[usenames, dvipsnames]{color}
\usepackage[dvipsnames]{xcolor}


\theoremstyle{definition}
\newtheorem{prop}{Proposition}
\newtheorem{thm}{Theorem}
\newtheorem{cor}{Corollary}
\newtheorem{fact}{Fact}
\newtheorem{remark}{Remark}
\newtheorem{example}{Example}
\newtheorem{lem}{Lemma}
\newtheorem{con}{Conjecture}
\newtheorem{define}{Definition}
\newcommand{\cov}{\textrm{Cov}}
\newcommand{\var}{\textrm{Var}}
\newcommand{\sol}{\textit{Solution:} }
\newcommand{\hs}{\hspace{5pt}}
\newcommand{\hsa}{\hspace{10pt}}
\newcommand{\hsb}{\hspace{20pt}}
\newcommand{\no}{\nonumber}

\newcommand{\amir}[1]{\textcolor{red}{{\bf AMIR: #1}}}
\newcommand{\hoosein}[1]{\textcolor{blue}{{\bf Hossein: #1}}}
%\newcommand{\amir}[1]{\textcolor{red}{{}}}


% correct bad hyphenation here
\hyphenation{op-tical net-works semi-conduc-tor}





\begin{document}
%
\title{Matching Anonymized and Obfuscated\\ Time Series to Users' Profiles}

\author{Nazanin~Takbiri~\IEEEmembership{Student Member,~IEEE,}
        Amir~Houmansadr~\IEEEmembership{Member,~IEEE,}
        Dennis~Goeckel~\IEEEmembership{Fellow,~IEEE,}
        Hossein~Pishro-Nik~\IEEEmembership{Member,~IEEE}
        %\thanks{Manuscript received April 19, 2005; revised August 26, 2015.}
        \thanks{N. Takbiri is with the Department
        of Electrical and Computer Engineering, University of Massachusetts, Amherst,
        MA, 01003 USA e-mail: (ntakbiri@umass.edu).}% <-this % stops a space
        \thanks{A. Houmansadr is with the College of Information and Computer Sciences, University of Massachusetts, Amherst,
        MA, 01003 USA e-mail:(amir@cs.umass.edu)}
        \thanks{H. Pishro-Nik and D. Goeckel are with the Department
        of Electrical and Computer Engineering, University of Massachusetts, Amherst,
        MA, 01003 USA e-mail:(pishro@engin.umass.edu)}
        \thanks{This work was supported by National Science Foundation under grants CCF--0844725, CCF--1421957 and CNS1739462. 
        
        This work was presented in part in IEEE International Symposium on Information Theory (ISIT 2017) \cite{nazanin_ISIT2017}.}}

%\markboth{IEEE TRANSACTIONS ON WIRELESS COMMUNICATIONS}
%\IEEEpubid{0000--0000/00\$00.00~\copyright~2015 IEEE}
\maketitle

\begin{abstract}
%Many popular applications use traces of user data
%to offer various services to  their users; example applications include
 %driver-assistance systems and smart home services.
%However, revealing user information to such applications
%puts users' privacy at stake, as adversaries can
%infer  sensitive
%private information about the users  such as their behaviors, interests, and locations.
%Recent research shows that adversaries can  compromise users' privacy when they use such applications even
%when the traces of
%users' information are protected by mechanisms like anonymization and obfuscation.

Many popular applications use traces of user data to offer various services to their users.  However, even if user data is anonymized and obfuscated, a user's privacy can be compromised through the use of statistical matching techniques that match a user trace to prior user behavior. In this work, we derive the theoretical bounds on the privacy of users in such a scenario. We build on our recent study in the area of location privacy,
in which we introduced formal notions of location privacy for anonymization-based location privacy-protection mechanisms.
Here we derive the fundamental limits of user privacy when both anonymization and obfuscation-based protection mechanisms are applied to users' time series of data.
%    A recent effort towards a fundamental analysis of this risk introduced ``perfect location privacy'' for anonymization-based location-based services (LBS).
%    Here we generalize the application context and, more importantly, consider the fundamental limits of user privacy when anonymization and obfuscation privacy-preserving techniques are applied together.
We investigate the impact of such mechanisms on the trade-off between privacy protection and user utility. We first study achievability results for the case where the time-series of users are governed by an i.i.d.\ process. The converse results are proved both for the i.i.d.\ case as well as the more general Markov chain model. We demonstrate that as the number of users in the network grows, the obfuscation-anonymization plane can be divided into two regions: in the first region, all users have perfect privacy; and, in the second region, no user has privacy.%~\cite{tifs2016,KeConferance2017}.
\end{abstract}

\begin{IEEEkeywords}
Anonymization, Obfuscation, Information theoretic privacy, Privacy-Protection Mechanism (PPM), User-Data Driven Services (UDD).
\end{IEEEkeywords}

%\IEEEpeerreviewmaketitle

% \leavevmode
% \\
% \\
% \\
% \\
% \\
\section{Introduction}
\label{introduction}

AutoML is the process by which machine learning models are built automatically for a new dataset. Given a dataset, AutoML systems perform a search over valid data transformations and learners, along with hyper-parameter optimization for each learner~\cite{VolcanoML}. Choosing the transformations and learners over which to search is our focus.
A significant number of systems mine from prior runs of pipelines over a set of datasets to choose transformers and learners that are effective with different types of datasets (e.g. \cite{NEURIPS2018_b59a51a3}, \cite{10.14778/3415478.3415542}, \cite{autosklearn}). Thus, they build a database by actually running different pipelines with a diverse set of datasets to estimate the accuracy of potential pipelines. Hence, they can be used to effectively reduce the search space. A new dataset, based on a set of features (meta-features) is then matched to this database to find the most plausible candidates for both learner selection and hyper-parameter tuning. This process of choosing starting points in the search space is called meta-learning for the cold start problem.  

Other meta-learning approaches include mining existing data science code and their associated datasets to learn from human expertise. The AL~\cite{al} system mined existing Kaggle notebooks using dynamic analysis, i.e., actually running the scripts, and showed that such a system has promise.  However, this meta-learning approach does not scale because it is onerous to execute a large number of pipeline scripts on datasets, preprocessing datasets is never trivial, and older scripts cease to run at all as software evolves. It is not surprising that AL therefore performed dynamic analysis on just nine datasets.

Our system, {\sysname}, provides a scalable meta-learning approach to leverage human expertise, using static analysis to mine pipelines from large repositories of scripts. Static analysis has the advantage of scaling to thousands or millions of scripts \cite{graph4code} easily, but lacks the performance data gathered by dynamic analysis. The {\sysname} meta-learning approach guides the learning process by a scalable dataset similarity search, based on dataset embeddings, to find the most similar datasets and the semantics of ML pipelines applied on them.  Many existing systems, such as Auto-Sklearn \cite{autosklearn} and AL \cite{al}, compute a set of meta-features for each dataset. We developed a deep neural network model to generate embeddings at the granularity of a dataset, e.g., a table or CSV file, to capture similarity at the level of an entire dataset rather than relying on a set of meta-features.
 
Because we use static analysis to capture the semantics of the meta-learning process, we have no mechanism to choose the \textbf{best} pipeline from many seen pipelines, unlike the dynamic execution case where one can rely on runtime to choose the best performing pipeline.  Observing that pipelines are basically workflow graphs, we use graph generator neural models to succinctly capture the statically-observed pipelines for a single dataset. In {\sysname}, we formulate learner selection as a graph generation problem to predict optimized pipelines based on pipelines seen in actual notebooks.

%. This formulation enables {\sysname} for effective pruning of the AutoML search space to predict optimized pipelines based on pipelines seen in actual notebooks.}
%We note that increasingly, state-of-the-art performance in AutoML systems is being generated by more complex pipelines such as Directed Acyclic Graphs (DAGs) \cite{piper} rather than the linear pipelines used in earlier systems.  
 
{\sysname} does learner and transformation selection, and hence is a component of an AutoML systems. To evaluate this component, we integrated it into two existing AutoML systems, FLAML \cite{flaml} and Auto-Sklearn \cite{autosklearn}.  
% We evaluate each system with and without {\sysname}.  
We chose FLAML because it does not yet have any meta-learning component for the cold start problem and instead allows user selection of learners and transformers. The authors of FLAML explicitly pointed to the fact that FLAML might benefit from a meta-learning component and pointed to it as a possibility for future work. For FLAML, if mining historical pipelines provides an advantage, we should improve its performance. We also picked Auto-Sklearn as it does have a learner selection component based on meta-features, as described earlier~\cite{autosklearn2}. For Auto-Sklearn, we should at least match performance if our static mining of pipelines can match their extensive database. For context, we also compared {\sysname} with the recent VolcanoML~\cite{VolcanoML}, which provides an efficient decomposition and execution strategy for the AutoML search space. In contrast, {\sysname} prunes the search space using our meta-learning model to perform hyperparameter optimization only for the most promising candidates. 

The contributions of this paper are the following:
\begin{itemize}
    \item Section ~\ref{sec:mining} defines a scalable meta-learning approach based on representation learning of mined ML pipeline semantics and datasets for over 100 datasets and ~11K Python scripts.  
    \newline
    \item Sections~\ref{sec:kgpipGen} formulates AutoML pipeline generation as a graph generation problem. {\sysname} predicts efficiently an optimized ML pipeline for an unseen dataset based on our meta-learning model.  To the best of our knowledge, {\sysname} is the first approach to formulate  AutoML pipeline generation in such a way.
    \newline
    \item Section~\ref{sec:eval} presents a comprehensive evaluation using a large collection of 121 datasets from major AutoML benchmarks and Kaggle. Our experimental results show that {\sysname} outperforms all existing AutoML systems and achieves state-of-the-art results on the majority of these datasets. {\sysname} significantly improves the performance of both FLAML and Auto-Sklearn in classification and regression tasks. We also outperformed AL in 75 out of 77 datasets and VolcanoML in 75  out of 121 datasets, including 44 datasets used only by VolcanoML~\cite{VolcanoML}.  On average, {\sysname} achieves scores that are statistically better than the means of all other systems. 
\end{itemize}


%This approach does not need to apply cleaning or transformation methods to handle different variances among datasets. Moreover, we do not need to deal with complex analysis, such as dynamic code analysis. Thus, our approach proved to be scalable, as discussed in Sections~\ref{sec:mining}.
%%%%%%%%%%%%%%%%%%%%%%%%%%%%%%%%%%%%%%%%%%%%%
%\section{Related Work}
\label{sec:related_work}
We now provide a brief overview of related work in the areas of language grounding and transfer for reinforcement learning.
%There has been work on learning to make optimal local decisions for structured prediction problems~\cite{daume2006searn}.
%
%\newcite{branavan2010reading} looked at a similar task of building a partial model of the environment while following instructions. The differences with our work are (1) the text in their case is instructions, while we only have text describing the environment, and (2) their environment is deterministic, hence the transition function can be learned more easily. 
%
%TODO - model-based RL, value iteration, predictron.


\subsection{Grounding Language in Interactive Environments}
In recent years, there has been increasing interest in systems that can utilize textual knowledge to learn control policies. Such applications include interpreting help documentation~\fullcite{branavan2010reading}, instruction following~\fullcite{vogel2010learning,kollar2010toward,artzi2013weakly,matuszek2013learning,Andreas15Instructions} and learning to play computer games~\fullcite{branavan2011nonlinear,branavan2012learning,narasimhan2015language,he2016deep}. In all these applications, the models are trained and tested on the same domain.

Our work represents two departures from prior work on grounding. First, rather than optimizing control performance for a single domain,
we are interested in the multi-domain transfer scenario, where language 
descriptions drive generalization. Second, prior work used text in the form of strategy advice to directly learn the policy. Since the policies are typically optimized for a specific task, they may be harder to transfer across domains. Instead, we utilize text to bootstrap the induction of the environment dynamics, moving beyond task-specific strategies. 

%Previous work has explored the use of text manuals in game playing, %ranging from constructing useful features by mining patterns in %text~\cite{eisenstein2009reading}, learning a semantic interpreter %with access to limited gameplay examples~\cite{goldwasser2014learning} %to learning through reinforcement from in-game %rewards~\cite{branavan2011learning}. These efforts have demonstrated %the usefulness of exploiting domain knowledge encoded in text to learn %effective policies. However, these methods use the text to infer %directly the best strategy to perform a task. In contrast, our goal is %to learn mappings from the text to the dynamics of an environment and %separate out the learning of the strategy/motives. 

Another related line of work consists of systems that learn spatial and topographical maps of the environment for robot navigation using natural language descriptions~\fullcite{walter2013learning,hemachandra2014learning}. These approaches use text mainly containing appearance and positional information, and integrate it with other semantic sources (such as appearance models) to obtain more accurate maps. In contrast, our work uses language describing the dynamics of the environment, such as entity movements and interactions, which 
is complementary to static positional information received through state observations. Further, our goal is to help an agent learn policies that generalize over different stochastic domains, while their works consider a single domain.

%karthik: I don't see the direct relevance
%Another line of work explores using textual interactive %environments~\cite{narasimhan2015language,he2016deep} to ground %language understanding into actions taken by the system in the %environment. In these tasks, understanding of language is crucial, %without which a system would not be able to take reasonable actions. %Our motivation is different -- we take an existing set of tasks and %domains which are amenable to learning through reinforcement, and %demonstrate how to utilize textual knowledge to learn faster and more %optimal policies in both multitask and transfer setups.

\subsection{Transfer in Reinforcement Learning}
Transferring policies across domains is a challenging problem in reinforcement learning~\fullcite{konidaris2006framework,taylor2009transfer}. The main hurdle lies in learning a good mapping between the state and action spaces of different domains to enable effective transfer. Most previous approaches have either explored skill transfer~\fullcite{konidaris2007building,konidaris2012transfer} or value function/policy transfer~\fullcite{liu2006value,taylor2007transfer,taylor2007cross}. There have also been attempts at model-based transfer for RL~\fullcite{taylor2008transferring,nguyen2012transferring,gavsic2013pomdp,wang2015learning,joshi2018cross} but these methods either rely on hand-coded inter-task mappings for state and actions spaces or require significant interactions in the target task to learn an effective mapping. Our approach doesn't use any explicit mappings and can learn to predict the dynamics of a target task using its descriptions.

% Work by \newcite{konidaris2006autonomous} look at knowledge transfer by learning a mapping from sensory signals to reward functions.

A closely related line of work concerns transfer methods for deep reinforcement learning. \citeA{parisotto2016actor}  train a deep network to mimic pre-trained experts on source tasks using policy distillation. The learned parameters are then used to initialize a network on a target task to perform transfer. Rusu et al.~\citeyear{rusu2016progressive} facilitate transfer by freezing parameters learned on source tasks and adding a new set of parameters for every new target task, while using both sets to learn the new policy. Work by Rajendran et al.~\citeyear{rajendran20172t} uses attention networks to selectively transfer from a set of expert policies to a new task. \textcolor{black}{Barreto et al.~\citeyear{barreto2017successor} use features based on successor representations~\fullcite{dayan1993improving} for transfer across tasks in the same domain. Kansky~et~al.~\citeyear{kansky2017schema} learn a generative model of causal physics in order to help zero-shot transfer learning.} Our approach is orthogonal to all these directions since we use text to bootstrap transfer, and can potentially be combined with these methods to achieve more effective transfer. 

\textcolor{black}{There has also been prior work on zero-shot policy generalization for tasks with input goal specifications. \fullciteA{schaul2015universal} learn a universal value function approximator that can generalize across both states and goals. \fullcite{andreas2016modular} use policy sketches, which are annotated sequences of subgoals, in order to learn a hierarchical policy that can generalize to new goals. \fullciteA{oh2017zero} investigate zero-shot transfer for instruction following tasks, aiming to generalize to unseen instructions in the same domain. The main departure of our work compared to these is in the use of environment descriptions for generalization across domains rather than generalizing across text instructions.}

Perhaps closest in spirit to our hypothesis is the recent work by~\fullcite{harrison2017guiding}. Their approach makes use of paired instances of text descriptions and state-action information from human gameplay to learn a machine translation model. This model is incorporated into a policy shaping algorithm to better guide agent exploration. Although the motivation of language-guided control policies is similar to ours, their work considers transfer across tasks in a single domain, and requires human demonstrations to learn a policy.

\textcolor{black}{
\subsection{Using Task Features for Transfer}
The idea of using task features/dictionaries for zero-shot generalization has been explored previously in the context of image classification. \fullciteA{kodirov2015unsupervised} learn a joint feature embedding space between domains and also induce the corresponding projections onto this space from different class labels. 
\fullciteA{kolouri2018joint} learn a joint dictionary across visual features and class attributes using sparse coding techniques. \fullciteA{romera2015embarrassingly} model the relationship between input features, task attributes and classes as a linear model to achieve efficient yet simple zero-shot transfer for classification. \fullciteA{socher2013zero} learn a joint semantic representation space for images and associated words to perform zero-shot transfer.}

\textcolor{black}{
Task descriptors have also been explored in zero-shot generalization for control policies. \fullciteA{sinapov2015learning} use task meta-data as features to learn a mapping between pairs of tasks. This mapping is then used to select the most relevant source task to transfer a policy from. \fullciteA{isele2016using} build on the ELLA framework~\fullcite{ruvolo2013ella,ammar2014online}, and their key idea is to maintain two shared linear bases across tasks -- one for the policy ($L$) and the other for task descriptors ($D$). Once these bases are learned on a set of source tasks, they can be used to predict policy parameters for a new task given its corresponding descriptor. 
% The training scheme is similar to Actor-mimic scheme~\cite{parisotto2016actor} -- for each task, an expert policy is trained separately and then distilled into policy parameters dependent on the shared basis $L$. 
In these lines of work, the task features were either manually engineered or directly taken from the underlying system parameters defining the dynamics of the environment. In contrast, our framework only requires access to crowd-sourced textual descriptions, alleviating the need for expert domain knowledge.}





% A major difference in our work is that we utilize natural language descriptions of different environments to bootstrap transfer, requiring less exploration in the new task.

% using a policy distillation~\cite{parisotto2016actor,rusu2016progressive,yin2017knowledge} or selective attention over expert networks learnt in the source tasks~\cite{rajendran20172t}. Though these approaches provide some benefits, they still suffer from the requirement of efficiently exploring the new environment to learn how to transfer their existing policies. In contrast, we utilize natural language descriptions of different environments to bootstrap transfer, leading to more focused exploration in the target task. 


% Describe amn in detail





%%%%%%%%%%%%%%%%%%%%%%%%%%%%%%%%%%%%%%%%%%%%%%%%%%%%
\section{Framework}
\label{sec:framework}
In this paper, we adopt a similar framework to that employed in~\cite{tifs2016,ciss2017}.  The general set up is provided here, and the refinement to the precise models for this paper will be presented in the following sections.  We assume a system with $n$ users with $X_u(k)$ denoting a sample of the data of user $u$ at time $k$, which we would like to protect from an interested adversary $\mathcal{A}$. We consider a strong adversary $\mathcal{A}$ that has complete statistical knowledge of the users' data patterns based on the previous observations or other resources. In order to secure data privacy of users, both obfuscation and anonymization techniques are used as shown in Figure \ref{fig:xyz}. In Figure \ref{fig:xyz}, $Z_u(k)$ shows the (reported) sample of the data of user $u$ at time $k$ after applying obfuscation, and $Y_u(k)$ shows the (reported) sample of the data of user $u$ at time $k$ after applying anonymization. The adversary observes only $Y_u(k)$, $k=1,2,\cdots, m(n)$, where $m(n)$ is the number of observations of each user before the identities are permuted. The adversary then tries to estimate $X_u(k)$ by using those observations.
\begin{figure}[h]
	\centering
	\includegraphics[width = 0.75\linewidth]{fig/xyz}
	\caption{Applying obfuscation and anonymization techniques to users' data samples.}
	\label{fig:xyz}
\end{figure}

Let $\textbf{X}_u$ be the $m(n) \times 1$ vector containing the sample of the data of user $u$, and $\textbf{X}$ be the $m(n) \times n$ matrix with $u^{th}$ column equal to $\textbf{X}_u$;
\[\textbf{X}_u = \begin{bmatrix}
X_u(1) \\ X_u(2) \\ \vdots \\X_u(m) \end{bmatrix} , \ \ \  \textbf{X} =\left[\textbf{X}_{1}, \textbf{X}_{2}, \cdots,  \textbf{X}_{n}\right].
\]


\textit{Data Samples Model:}
We assume there are $r \geq 2$ possible values ($0,1, \cdots, r-1$) for each sample of the users' data. In the first part of the paper (perfect privacy analysis), we assume an i.i.d.\ model as motivated in Section \ref{intro}. In the second part of the paper (converse results: no privacy region), the users' datasets are governed by irreducible and aperiodic Markov chains. At any time, $X_u(k)$ is equal to a value in $\left\{0,1, \cdots, r-1 \right\}$ according to a user-specific probability distribution. The collection of user distributions, which satisfy some mild regularity conditions discussed below, is known to the adversary $\mathcal{A}$, and he/she employs such to distinguish different users based on statistical matching of those user distributions to traces of user activity of length $m(n)$.%In any case, any dependency can only favor the adversary, so our results provide lower bounds on the achievable privacy in those settings, which would be desirable if the dependency is poorly understood.  Second, understanding the i.i.d.\ case can be considered the first step toward understanding the more complicated case where there is a dependency, as was done for anonymization-only LPPMs in \cite{tifs2016} to consider perfect privacy when users' data pattern are governed by Markov chains. In the second part of the paper, in addition to the above model, we consider the case where users' data patterns are modeled by Markov chains as discussed in Section \ref{subsec:markov}.%





\textit{Obfuscation Model:} The first step in obtaining privacy is to apply the obfuscation operation in order to perturb the users' data samples. In this paper, we assume that each user has only limited knowledge of the characteristics of the overall population and thus we employ a simple distributed method in which the samples of the data of each user are reported with error with a certain probability, where that probability itself is generated randomly for each user. In other words, the obfuscated data is obtained by passing the users' data through an $r$-ary symmetric channel with a random error probability. More precisely, let $\textbf{Z}_u$ be the vector which contains
%$m$ number of
the obfuscated versions of user $u$'s data samples, and $\textbf{Z}$ is the collection of $\textbf{Z}_u$ for all users,
\[\textbf{Z}_u = \begin{bmatrix}
Z_u(1) \\ Z_u(2) \\ \vdots \\Z_u(m) \end{bmatrix} , \ \ \  \textbf{Z} =\left[ \textbf{Z}_{1}, \textbf{Z}_{2}, \cdots,  \textbf{Z}_{n}\right].
\]
To create a noisy version of data samples, for each user $u$, we independently generate a random variable $R_u$ that is uniformly distributed between $0$ and $a_n$, where $a_n \in (0,1]$. The value of $R_u$ gives the probability that a user's data sample is changed to a different data sample by obfuscation, and $a_n$ is termed the ``noise level'' of the system. For the case of $r=2$ where there are two states for users' data (state $0$ and state $1$), the obfuscated data is obtained by passing users' data through a Binary Symmetric Channel (BSC) with a small error probability~\cite{randomizedresponse}. Thus, we can write
\[
{Z}_{u}(k)=\begin{cases}
{X}_{u}(k), & \textrm{with probability } 1-R_u.\\
1-{X}_{u}(k),& \textrm{with probability } R_u.
\end{cases}
\]
When $r>2$, for $l \in \{0,1,\cdots, r-1\}$:
\[
P({Z}_{u}(k)=l| X_{u}(k)=i) =\begin{cases}
1-R_u, & \textrm{for } l=i.\\
\frac{R_u}{r-1}, & \textrm{for } l \neq i.
\end{cases}
\]
Note that the effect of the obfuscation is to alter the probability distribution function of each user across the $r$ possibilities in a way that is unknown to the adversary, since it is independent of all past activity of the user, and hence the obfuscation inhibits user identification. For each user, $R_u$ is generated once and is kept constant for the collection of samples of length $m(n)$, thus, providing a very low-weight obfuscation algorithm. We will discuss the extension to the case where $R_u$ is regenerated independently over time in Section \ref{sec:perfect-MC}. There, we will also provide a discussion about obfuscation using continuous noise distributions (e.g., Gaussian noise).

\textit{Anonymization Model:} Anonymization is modeled by a random permutation $\Pi$ on the set of $n$ users. The user $u$ is assigned the pseudonym $\Pi(u)$. $\textbf{Y}$ is the anonymized version of $\textbf{Z}$; thus,
\begin{align}
\no \textbf{Y} &=\textrm{Perm}\left(\textbf{Z}_{1}, \textbf{Z}_{2}, \cdots,  \textbf{Z}_{n}; \Pi \right) \\
\nonumber &=\left[ \textbf{Z}_{\Pi^{-1}(1)}, \textbf{Z}_{\Pi^{-1}(2)}, \cdots,  \textbf{Z}_{\Pi^{-1}(n)}\right ] \\
\nonumber &=\left[ \textbf{Y}_{1}, \textbf{Y}_{2}, \cdots, \textbf{Y}_{n}\right], \ \
\end{align}
where $\textrm{Perm}( \ . \ , \Pi)$ is permutation operation with permutation function $\Pi$. As a result, $\textbf{Y}_{u} = \textbf{Z}_{\Pi^{-1}(u)}$ and $\textbf{Y}_{\Pi(u)} = \textbf{Z}_{u}$.

\textit{Adversary Model:} We protect against the strongest reasonable adversary. Through past observations or some other sources, the adversary is assumed to have complete statistical knowledge of the users' patterns; in other words, he/she knows the probability distribution for each user on the set of data samples $\{0,1,\ldots,r-1\}$. As discussed in the model for the data samples, the parameters $\textbf{p}_u$, $u=1, 2, \cdots, n$ are drawn independently from a continuous density function, $f_\textbf{P}(\textbf{p}_u)$, which has support on a subset of a defined hypercube. The density $f_\textbf{P}(\textbf{p}_u)$ might be unknown to the adversary, as all that is assumed here is that such a density exists, and it will be evident from our results that knowing or not knowing $f_\textbf{P}(\textbf{p}_u)$ does not change the results asymptotically. Specifically, from the results of Section \ref{perfectsec}, we conclude that user $u$ has perfect privacy even if the adversary knows $f_\textbf{P}(\textbf{p}_u)$. In addition, in Section \ref{converse}, it is shown that the adversary can recover the true data of user $u$ at time $k$ without using the specific density function of $f_\textbf{P}(\textbf{p}_u)$, and as result, users have no privacy even if the adversary does not know $f_\textbf{P}(\textbf{p}_u)$.

The adversary also knows the value of $a_n$ as it is a design parameter. However, the adversary does not know the realization of the random permutation $\Pi$ or the realizations of the random variables $R_u$, as these are independent of the past behavior of the users. It is critical to note that we assume the adversary does not have any auxiliary information or side information about users' data.


In \cite{tifs2016}, perfect privacy is defined as follows:
\begin{define}
User $u$ has \emph{perfect privacy} at time $k$, if and only if
\begin{align}%\label{}
\no  \forall k\in \mathbb{N}, \ \ \ \lim\limits_{n\rightarrow \infty} I \left(X_u(k);{\textbf{Y}}\right) =0,
\end{align}
where $I(X;Y)$ denotes the mutual information between random variables (vectors) $X$ and $Y$.
\end{define}

\noindent In this paper, we also consider the situation in which there is no privacy. 

\begin{define}
For an algorithm for the adversary that tries to estimate the actual sample of data of user $u$ at time $k$, define
\[P_e(u,k)\triangleq P\left(\widetilde{X_u(k)} \neq X_u(k)\right),\]
where $X_u(k)$ is the actual sample of the data of user $u$ at time $k$, $\widetilde{X_u(k)}$ is the adversary's estimated sample of the data of user $u$ at time $k$, and $P_e(u,k)$ is the error probability. Now, define ${\cal E}$ as the set of all possible adversary's estimators; then, user $u$ has \emph{no privacy} at time $k$, if and only if for large enough $n$,
\[
\forall k\in \mathbb{N}, \ \ \ P^{*}_e(u,k)\triangleq \inf_{\cal E} {P\left(\widetilde{X_u(k)} \neq X_u(k)\right)} \rightarrow 0.
\]
Hence, a user has no privacy if there exists an algorithm for the adversary to estimate $X_u(k)$ with diminishing error probability as $n$ goes to infinity.
\end{define}

\textbf{\textit{Discussion:}} Both of the privacy definitions given above (perfect privacy and no privacy) are asymptotic in the number of users $(n \to \infty)$, which allows us to find clean analytical results for the fundamental limits. Moreover, in many IoT applications, such as ride sharing and dining recommendation applications, the number of users is large. %And finally, we show in the simulation section that our results provide good predictions for the behavior observed when there is a finite number of users in the system. 

\textbf{\textit{Notation:}} Note that the sample of data of user $u$ at time $k$ after applying obfuscation $\left(Z_u(k)\right)$ and the sample of data of user $u$ at time $k$ after applying anonymization $\left(Y_u(k)\right)$ depend on the number of users in the network $(n)$, while the actual sample of data of user $u$ at time $k$ is independent of the number of users $(n)$.  Despite the dependency in the former cases, we omit this subscript $(n)$ on $\left(Z_u^{(n)}(k), Y_u^{(n)}(k) \right)$ to avoid confusion and make the notation consistent.

\textbf{\textit{Notation:}} Throughout the paper, $X_n \xrightarrow{d} X$ denotes convergence in distribution. Also, We use $P\left(X=x\bigg{|} Y=y\right)$ for the conditional probability of $X=x$ given $Y=y$. When we write $P\left(X=x \bigg{|}Y\right)$, we are referring to a random variable that is defined as a function of $Y$.


\section{Perfect Privacy Analysis: I.I.D.\ Case}
\label{perfectsec}

\subsection{Two-States Model}

We first consider the two-states case $(r=2)$ which captures the salient aspects of the problem. For the two-states case, the sample of the data of user $u$ at any time is a Bernoulli random variable with parameter $p_u$, which is the probability of user $u$ having data sample $1$. Thus,
\[X_u(k) \sim Bernoulli \left(p_u\right).\]
Per Section \ref{sec:framework}, the parameters $p_u$, $u=1, 2, \cdots, n$ are drawn independently from a continuous density function, $f_P(p_u)$, on the $(0,1)$ interval. We assume there are $\delta_1, \delta_2>0$ such that:\footnote{The condition $\delta_1<f_P(p_u) <\delta_2$ is not actually necessary for the results and can be relaxed; however, we keep it here to avoid unnecessary technicalities.}
\begin{equation}
\no\begin{cases}
    \delta_1<f_P(p_u) <\delta_2, & p_u \in (0,1).\\
    f_P(p_u)=0, &  p _u\notin (0,1).
\end{cases}
\end{equation}

The adversary knows the values of $p_u$, $u=1, 2, \cdots, n$ and uses this knowledge to identify users. We will use capital letters (i.e., $P_u$) when we are referring to the random variable, and use lower case (i.e., $p_u$) to refer to the realization of $P_u$.

In addition, since the user data $\left(X_u(k)\right)$ are i.i.d.\ and have a Bernoulli distribution, the obfuscated data $\left(Z_u(k)\right)$ are also i.i.d.\ with  a Bernoulli distribution. Specifically,
\[Z_u(k) \sim Bernoulli\left(Q_u\right),\]
where
\begin{align}
\no {Q}_u &=P_u(1-R_u)+(1-P_u)R_u  \\
\nonumber &= P_u+\left(1-2P_u\right)R_u,\ \
\end{align}
and recall that $R_u$ is the probability that user $u$'s data sample is altered at any time. For convenience, define a vector where element $Q_u$ is the probability that an obfuscated data sample of user $u$ is equal to one, and
\[\textbf{Q} =\left[{Q}_{1},{Q}_{2}, \cdots,{Q}_{n}\right].\]
Thus, a vector containing the permutation of those probabilities after anonymization is given by:
\begin{align}
\no \textbf{V} &=\textrm{Perm}\left({Q}_{1}, {Q}_{2}, \cdots,  {Q}_{n}; \Pi \right) \\
\nonumber &=\left[ {Q}_{\Pi^{-1}(1)}, {Q}_{\Pi^{-1}(2)}, \cdots, {Q}_{\Pi^{-1}(n)}\right ] \\
\nonumber &=\left[{V}_{1}, {V}_{2}, \cdots, {V}_{n}\right ] ,\ \
\end{align}
where ${V}_{u} = {Q}_{\Pi^{-1}(u)}$ and  ${V}_{\Pi(u)} = {Q}_{u}$. As a result, for $u=1,2,..., n$, the distribution of the data symbols for the user with pseudonym $u$ is given by:
\[ Y_u(k) \sim Bernoulli \left(V_u\right) \sim Bernoulli\left(Q_{\Pi^{-1}(u)}\right)  .\]

The following theorem states that if $a_n$ is significantly larger than $\frac{1}{n}$ in this two-states model, then all users have perfect privacy independent of the value of $m(n)$.
\begin{thm}\label{two_state_thm}
 For the above two-states model, if $\textbf{Z}$ is the obfuscated version of $\textbf{X}$, and $\textbf{Y}$ is the anonymized version of $\textbf{Z}$  as defined above, and
\begin{itemize}
	\item $m=m(n)$ is arbitrary;
	\item $R_u \sim Uniform [0, a_n]$, where $a_n \triangleq c'n^{-\left(1-\beta\right)}$ for any $c'>0$ and $0<\beta<1$;
\end{itemize}
then, user 1 has perfect privacy. That is,
\begin{align}%\label{}
\no  \forall k\in \mathbb{N}, \ \ \ \lim\limits_{n\rightarrow \infty} I \left(X_1(k);{\textbf{Y}}\right) =0.
\end{align}
\end{thm}

The proof of Theorem \ref{two_state_thm} will be provided for the case $0\leq p_1<\frac{1}{2}$, as the proof for the case $\frac{1}{2}\leq p_1\leq1$ is analogous and is thus omitted.
\\

\noindent \textbf{Intuition behind the Proof of Theorem \ref{two_state_thm}:} 

Since $m(n)$ is arbitrary, the adversary is able to estimate very accurately (in the limit, perfectly) the distribution from which each data sequence $\textbf{Y}_u$, $u= 1, 2, \cdots, n$ is drawn; that is, the adversary is able to accurately estimate the probability $V_u$, $u= 1, 2, \cdots, n$. Clearly, if there were no obfuscation for each user $u$, the adversary would then simply look for the $j$ such that $p_j$ is very close to $V_u$ and set $\widetilde{X_j(k)}=Y_u(k)$, resulting in no privacy for any user.

We want to make certain that the adversary obtains no information about $X_1(k)$, the sample of data of user $1$ at time $k$. To do such, we will establish that there are a large number of users whom have a probability $p_u$ that when obfuscated could have resulted in a probability consistent with $p_1$. Consider asking whether another probability $p_2$ is sufficiently close enough to be confused with $p_1$ after obfuscation; in particular, we will look for $p_2$ such that, even if the adversary is given the obfuscated probabilities $V_{\Pi(1)}$ and $V_{\Pi(2)}$, he/she cannot associate these probabilities with $p_1$ and $p_2$. This requires that the distributions $Q_{1}$ and $Q_{2}$ of the obfuscated data of user $1$ and user $2$ have significant overlap; we explore this next.

Recall that $Q_u=P_u+ (1-2P_u)R_u$, and $R_u\sim Uniform [0, a_n]$. Thus, we know $Q_u {|} P_u=p_u$ has a uniform distribution with length $(1-2p_u)a_n$. Specifically,
\[Q_u\bigg{|}P_u=p_u \sim Uniform \left[p_u,p_u+(1-2p_u)a_n\right].\]
Figure \ref{fig:piqi_a} shows the distribution of $Q_u$ given $P_u=p_u$.

\begin{figure}[h]
	\centering
	\includegraphics[width = 0.8\linewidth]{fig/piqi_a}
	\caption{Distribution of $Q_u$ given $P_u=p_u$.}
	\label{fig:piqi_a}
\end{figure}

Consider two cases: In the first case, the support of the distributions $Q_1\big{|} P_1=p_1$ and $Q_2\big{|} P_2=p_2$ are small relative to the difference between $p_1$ and $p_2$ (Figure \ref{fig:case1}); in this case, given the probabilities $V_{\Pi(1)}$ and $V_{\Pi(2)}$ of the anonymized data sequences, the adversary can associate those with $p_1$ and $p_2$ without error. In the second case, the support of the distributions $Q_1\big{|} P_1=p_1$ and $Q_2\big{|} P_2=p_2$ is large relative to the difference between $p_1$ and $p_2$ (Figure \ref{fig:case2}), so it is difficult for the adversary to associate the probabilities $V_{\Pi(1)}$ and $V_{\Pi(2)}$ of the anonymized data sequences with $p_1$ and $p_2$. In particular, if $V_{\Pi(1)}$ and $V_{\Pi(2)}$ fall into the overlap of the support of $Q_1$ and $Q_2$, we will show the adversary can only guess randomly how to de-anonymize the data. Thus, if the ratio of the support of the distributions to $\big{|}p_1-p_2\big{|}$ goes to infinity, the adversary's posterior probability for each user converges to $\frac{1}{2}$, thus, implying no information leakage on the user identities. More generally, if we can guarantee that there will be a large set of users with $p_u$'s very close to $p_1$ compared to the support of $Q_1\big{|} P_1=p_1$, we will be able to obtain perfect privacy as demonstrated rigorously below.

\begin{figure}[h]
	\centering
	\includegraphics[width = 0.8\linewidth]{fig/case1}
	\caption{Case 1: The support of the distributions is small relative to the difference between $p_1$ and $p_2$.}
	\label{fig:case1}
\end{figure}

\begin{figure}[h]
	\centering
	\includegraphics[width = 0.8\linewidth]{fig/case2}
	\caption{Case 2: The support of the distributions is large relative to the difference between $p_1$ and $p_2$.}
	\label{fig:case2}
\end{figure}

Given this intuition, the formal proof proceeds as follows. Given $p_1$, we define a set $J^{(n)}$ of users whose parameter $p_u$ of their data distributions is sufficiently close to $p_1$ (Figure \ref{fig:case2}; case 2), so that it is likely that $Q_1$ and $Q_u$ cannot be readily associated with $p_1$ and $p_u$.

The purpose of Lemmas \ref{lemOnePointFive}, \ref{lem2}, and \ref{lem3} is to show that, from the adversary's perspective, the users in set $J^{(n)}$ are indistinguishable. More specifically, the goal is to show that the obfuscated data corresponding to each of these users could have been generated by any other users in $J^{(n)}$ in an equally likely manner. To show this, Lemma \ref{lemOnePointFive} employs the fact that, if the observed values of $N$ uniformly distributed random variables ($N$ is size of set $J^{(n)}$) are within the intersection of their ranges, it is impossible to infer any information about the matching between the observed values and the distributions. That is, all possible $N!$ matchings are equally likely. Lemmas \ref{lem2} and \ref{lem3} leverage Lemma \ref{lemOnePointFive} to show that even if the adversary is given a set that includes all of the pseudonyms of the users in set $J^{(n)}$  (i.e., $\Pi(J^{(n)})\overset{\Delta}{=} \left\{\Pi^{-1}(u) \in J^{(n)}\right\}$) he/she still will not be able to infer any information about the matching of each specific user in set $J^{(n)}$ and his pseudonym. Then Lemma \ref{lem4} uses the above fact to show that the mutual information between the data set of user $1$ at time $k$ and the observed data sets of the adversary converges to zero for large enough $n$.

%Remember that the adversary does not know the realization of the random permutation $\Pi$ as well as the realizations of random variables  $R_u^{(n)}$. Therefore, the adversary does not know the realizations of random variables ${V}_{u}^{(n)}$ defined above.
%Then by using lemma \ref{lem2}, \ref{lem3}, and \ref{lem4} we try to prove that the the mutual information between data set of user $1$ at time $k$ and the observed data sets by the adversary will  go to zero for large enough $n$.

\noindent \textbf{Proof of Theorem \ref{two_state_thm}:}

\begin{proof}
Note, per Lemma~\ref{lemx} of Appendix \ref{sec:app_a}, it is sufficient to establish the results on a sequence of sets with high probability. That is, we can condition on high-probability events.

Now, define the critical set $J^{(n)}$ with size $N^{(n)}=\big{|}J^{(n)}\big{|}$ for $0\leq p_1<\frac{1}{2}$ as follows:
\[J^{(n)}=
\left\{  u \in \{1, 2, \dots, n\}: p_1 \leq P_u\leq p_1+\epsilon_n;  p_1+\epsilon_n\leq Q_u\leq p_1+(1-2p_1)a_n\right\},
\]
where $\epsilon_n \triangleq \frac{1}{n^{1-\frac{\beta}{2}}}$, $a_n= c'n^{-\left(1-\beta\right)}$ ,and $\beta$ is defined in the statement of Theorem \ref{two_state_thm}.


Note for large enough $n$, if $0\leq p_1<\frac{1}{2}$, we have $0\leq p_u<\frac{1}{2}$. As a result,
\[Q_u\bigg{|}P_u=p_u \sim Uniform \left(p_u,p_u+(1-2p_u)a_n\right).\]
We can prove that with high probability, $1 \in J^{(n)}$ for large enough $n$, as follows. First, Note that
\[Q_1\bigg{|}P_1=p_1 \sim Uniform \left(p_1,p_1+(1-2p_1)a_n\right).\]
Now, according to Figure \ref{fig:piqi_c},
	\begin{align}
	\no P\left(1 \in J^{(n)} \right) &= 1- \frac{\epsilon_n}{ \left(1-2p_1 \right)a_n}\\
	\nonumber &= 1- \frac{1}{ \left(1-2p_1 \right)c'n^{\frac{\beta}{2}}}, \ \
	\end{align}
thus, for any $c'>0$ and large enough $n$,
\begin{align}
	\no P\left(1 \in J^{(n)} \right) \to 1.
\end{align}
\begin{figure}[h]
	\centering
	\includegraphics[width = 0.8\linewidth]{fig/piqi_b}
	\caption{Range of $P_u$ and $Q_u$ for elements of set $J^{(n)}$ and probability density function of $Q_u\bigg{|}P_u=p_u$.}
	\label{fig:piqi_b}
\end{figure}

\begin{figure}[h]
	\centering
	\includegraphics[width = 0.8\linewidth]{fig/piqi_c}
	\caption{Range of $P_u$ and $Q_u$ for elements of set $J^{(n)}$ and probability density function of $Q_1\bigg{|}P_1=p_1$.}
	\label{fig:piqi_c}
\end{figure}




Now in the second step, we define the probability $W_j^{(n)}$ for any $j \in \Pi(J^{(n)})=\{\Pi(u): u \in J^{(n)} \}$ as
\[W_j^{(n)}= P\left(\Pi(1)=j \bigg{|} \textbf{V}, \Pi (J^{(n)})\right).\]
$W_j^{(n)}$ is the conditional probability that $\Pi(1)=j$ after perfectly observing the values of the permuted version of obfuscated probabilities ($\textbf{V}$) and set including all of the pseudonyms of the users in set $J^{(n)}$ $\left(\Pi(J^{(n)})\right)$. Since $\textbf{V}$ and $\Pi (J^{(n)})$ are random, $W_j^{(n)}$ is a random variable. However, we will prove shortly that in fact $W_j^{(n)}=\frac{1}{N^{(n)}}$, for all $j \in \Pi (J^{(n)})$.

Note: Since we are looking from the adversary's point of view, the assumption is that all the values of $P_u$, $u \in \{1,2,\cdots,n\}$ are known, so all of the probabilities are conditioned on the values of $P_1=p_1, P_2=p_2, \cdots, P_n=p_n$. Thus, to be accurate, we should write
\[W_j^{(n)}= P\left(\Pi(1)=j \bigg{|} \textbf{V}, \Pi (J^{(n)}), P_1, P_2, \cdots, P_n\right).\]
Nevertheless, for simplicity of notation, we often omit the conditioning on $P_1, P_2, \cdots, P_n$.

First, we need a lemma from elementary probability.

\begin{lem}
	\label{lemOnePointFive}
	Let $N$ be a positive integer, and let $a_1, a_2, \cdots, a_N$ and $b_1, b_2, \cdots, b_N$ be real numbers such that $a_u \leq b_u$ for all $u$. Assume that $X_1, X_2, \cdots, X_N$ are independent random variables such that
\[X_u \sim Uniform [a_u,b_u]. \]
Let also $\gamma_1, \gamma_2, \cdots, \gamma_N$ be distinct real numbers such that
\[ \gamma_j \in \bigcap_{u=1}^{N} [a_u, b_u] \ \ \textrm{for all }j \in \{1,2,..,N\}.\]
Suppose that we know the event $E$ has occurred, meaning that the observed values of $X_u$'s are equal to the set of $\gamma_j$'s (but with unknown ordering), i.e.,
\[E \ \ \equiv \ \ \{X_1, X_2, \cdots, X_N\}= \{ \gamma_1, \gamma_2, \cdots, \gamma_N \},\] then
\[P\left(X_1=\gamma_j |E\right)=\frac{1}{N}. \]
\end{lem}

\begin{proof}
Lemma \ref{lemOnePointFive} is proved in Appendix \ref{sec:app_b}.
\end{proof}


%We now state a slightly modified version of Lemma \ref{lemOnePointFive} that is suitable for our purpose. The difference here is that $N$  is a random variable rather than a fixed integer (so are $a_i, b_i$).

%\begin{lem}
	%\label{lemOnePointSix}
	%Let $N$ be a positive integer-valued random variable, and let $A_1, A_2, \cdots$ and $B_1, B_2, \cdots$ be a sequence of real-valued random variables such that $A_i \leq B_i$ for all $i$, almost surely. Assume that $X_1, X_2, \cdots$ are independent random variables such that for all $n' \in \mathbb{N}$, given $N=n'$, $A_i=a_i$ and $B_i=b_i$ for all $i$,
%\[X_i \sim Uniform [a_i,b_i]. \]
%Let also $\gamma_1, \gamma_2, \cdots$ be distinct real numbers such that for all $n' \in \mathbb{N}$
%\[ \gamma_j \in \bigcap_{i=1}^{n'} [A_i, B_i] \ \ \textrm{for all }j \in \{1,2,..,n'\} \]
%Let $E$ be the event that the observed values of $X_i$'s is equal to the set of $\gamma_j$'s (but with unknown ordering), i.e.,
%\[E \ \ \equiv \ \ \{X_1, X_2, \cdots, X_N\}= \{ \gamma_1, \gamma_2, \cdots, \gamma_N \}. \]
%Then
%\[P\left(X_1=\gamma_j |E, A_i, B_i, i=1,2\cdots, N \right)=\frac{1}{N}. \]
%\end{lem}
%\begin{proof}
%Lemma \ref{lemOnePointSix} follows directly from Lemma \ref{lemOnePointFive}. In particular, assuming $N=n'$ is observed, and given $A_i=a_i$ and $B_i=b_i$, we can apply Lemma \ref{lemOnePointFive} to write
%\[P\left(X_1=\gamma_j |E,  A_i=a_i, B_i=b_i, i=1,2\cdots, N\right)=\frac{1}{n'}. \]
%Treating $N$, $A_i$, and $B_i$ as random variables,  we can write
%\[P\left(X_1=\gamma_j |E, A_i, B_i, i=1,2\cdots, N \right)=\frac{1}{N}. \]
%\end{proof}

Using the above lemma, we can state our desired result for $W_j^{(n)}$.

\begin{lem}
	\label{lem2}
	For all $j \in \Pi (J^{(n)})$, $W_j^{(n)}=\frac{1}{N^{(n)}}.$
\end{lem}
\begin{proof}
	
%We have
%	\begin{align}
%	\no P\left(\Pi^{(n)}(1)=j \bigg{|} \Pi (J^{(n)})\right)&= \frac{P\left(\Pi^{(n)}(1)=j \bigg{|} \Pi (J^{(n)})\right)}{P\left(\Pi (J^{(n)}\right)} \\
%	\nonumber &=\frac{\left(N^{(n)}-1\right)!}{\left(N^{(n)}\right)!}\\
%		\nonumber &=\frac{1}{N^{(n)}}.\ \
%	\end{align}

We argue that the setting of this lemma is essentially equivalent to the assumptions in Lemma \ref{lemOnePointFive}. First, remember that
\[W_j^{(n)}= P\left(\Pi(1)=j \bigg{|} \textbf{V}, \Pi (J^{(n)})\right).\]

Note that ${Q}_u= P_u+(1-2P_u)R_u$, and since $R_u$ is uniformly distributed, ${Q}_u$ conditioned on $P_u$ is also uniformly distributed in the appropriate intervals. Moreover, since ${V}_{u} = {Q}_{\Pi^{-1}(u)}$, we conclude ${V}_{u}$ is also uniformly distributed. So, looking at the definition of $W_j^{(n)}$, we can say the following: given the values of the uniformly distributed random variables ${Q}_u$, we would like to know which one of the values in  $\textbf{V}$ is the actual value of ${Q}_1={V}_{\Pi(1)}$, i.e., is $\Pi(1)=j$? This is equivalent to the setting of Lemma \ref{lemOnePointFive} as described further below. 

Note that since $1 \in J^{(n)}$, $\Pi(1) \in \Pi (J^{(n)})$. Therefore, when searching for the value of $\Pi(1)$, it is sufficient to look inside set $\Pi (J^{(n)})$. Therefore, instead of looking among all the values of ${V}_{j}$, it is sufficient to look at ${V}_{j}$ for $j \in  \Pi (J^{(n)})$. Let's show these values by $\textbf{V}_{\Pi} =\{v_1, v_2, \cdots, v_{N^{(n)}} \}$, so,
\[W_j^{(n)}= P\left(\Pi(1)=j \bigg{|} \textbf{V}_{\Pi}, \Pi (J^{(n)})\right).\]

Thus, we have the following scenario: $Q_u, u \in  J^{(n)}$ are independent random variables, and
\[Q_u\big{|} P_u=p_u \sim Uniform [p_u, p_u+(1-2p_u)a_n]. \]
Also, $v_1, v_2, \cdots, v_{N^{(n)}}$ are the observed values of $Q_u$ with unknown ordering (unknown mapping $\Pi$). We also know from the definition of set $J^{(n)}$ that
\[P_u \leq p_1+\epsilon_n \leq Q_u,\]
\[Q_u \leq p_1(1-2a_n)+an \leq P_u(1-2a_n)+a_n,\]
so, we can conclude
\[ v_j \in \bigcap_{u=1}^{N^{(n)}} [p_u, p_u+(1-2p_u)a_n] \ \ \textrm{for all }j \in \{1,2,..,N^{(n)}\}. \]
We know the event $E$ has occurred, meaning that the observed values of $Q_u$'s are equal to set of $v_j$'s (but with unknown ordering), i.e.,
\[E \ \ \equiv \ \ \{Q_u, u \in  J^{(n)}\}= \{ v_1, v_2, \cdots, v_{N^{(n)}} \}. \]
Then, according to Lemma \ref{lemOnePointFive},
\[P\left(Q_1=v_j |E, P_1, P_2, \cdots, P_n \right)=\frac{1}{N^{(n)}}. \]
Note that there is a subtle difference between this lemma and Lemma \ref{lemOnePointFive}. Here $N^{(n)}$ is a random variable while $N$ is a fixed number in Lemma \ref{lemOnePointFive}. Nevertheless, since the assertion holds for every fixed $N$, it also holds for the case where $N$ is a random variable. Now, note that
\begin{align*}
P\left(Q_1=v_j |E, P_1, P_2, \cdots, P_n \right) &= P\left(\Pi(1)=j \bigg{|} E, P_1, P_2, \cdots, P_n \right)\\
&=P\left(\Pi(1)=j \bigg{|} \textbf{V}_{\Pi}, \Pi (J^{(n)}), P_1, P_2, \cdots, P_n \right)\\
&=W_j^{(n)}.
\end{align*}
Thus, we can conclude
		\[W_j^{(n)}=\frac{1}{N^{(n)}}.\]
\end{proof}	
%It is worth noting that the above lemma in fact implies that  that given $\Pi (J^{(n)})$, $\Pi (1)$ and $V^{(n)}$ are independent. Specifically,
%\[P\left(\Pi^{(n)}(1)=j \bigg{|} V^{(n)}, \Pi (J^{(n)})\right)= P\left(\Pi^{(n)}(1)=j \bigg{|} \Pi (J^{(n)})\right).\]
%
%
In the third step, we define $\widetilde{W_j^{(n)}}$ for any $j \in \Pi (J^{(n)})$ as
\[\widetilde{W_j^{(n)}} = P\left(\Pi(1)=j \bigg{|} \textbf{Y}, \Pi (J^{(n)})\right).\]

$\widetilde{W_j^{(n)}}$ is the conditional probability that $\Pi(1)=j$ after observing the values of the anonymized version of the obfuscated samples of the users' data ($\textbf{Y}$) and the aggregate set including all the pseudonyms of the users in set $J^{(n)}$ (i.e., $\Pi(J^{(n)})\overset{\Delta}{=} \left\{\Pi^{-1}(j) \in J^{(n)}\right\}$). Since $\textbf{Y}$ and $\Pi (J^{(n)})$ are random, $\widetilde{W_j^{(n)}} $ is a random variable. Now, in the following lemma, we will prove $\widetilde{W_j^{(n)}} =\frac{1}{N^{(n)}}$, for all $j \in \Pi (J^{(n)})$ by using Lemma \ref{lem3}.

Note in the following lemma, we want to show that even if the adversary is given a set including all of the pseudonyms of the users in set $J^{(n)}$, he/she cannot match each specific user in set $J^{(n)}$ and his pseudonym.

\begin{lem}
	\label{lem3}
	For all $j \in \Pi (J^{(n)})$, $\widetilde{W_j^{(n)}}=\frac{1}{N^{(n)}}.$
\end{lem}

\begin{proof}
%	Consider random variable $\Pi$  while
%	\[Range(\Pi)=\{1, 2, \cdots ,N\}.\]
First, note that

%\begin{multline*}
%\widetilde{W_j^{(n)}} = \sum_{\text{for all v}} P\left(\Pi^{(n)}(1)=j \bigg{|} Y^{(n)}, \Pi \left(J^{(n)}\right), V^{(n)}=v\right)\cdot \\
%P\left(V^{(n)}=v \bigg{|} Y^{(n)}, \Pi \left(J^{(n)}\right)\right)
%\end{multline*}

\begin{align*}
\widetilde{W_j^{(n)}} = \sum_{\text{for all v}} P\left(\Pi(1)=j \bigg{|} \textbf{Y}, \Pi \left(J^{(n)}\right), \textbf{V}=\textbf{v}\right) P\left(\textbf{V}=\textbf{v} \bigg{|} \textbf{Y}, \Pi \left(J^{(n)}\right)\right).
\end{align*}

Also, we note that given $\textbf{V}$, $\Pi(J^{(n)})$, and $\textbf{Y}$ are independent. Intuitively, this is because when observing $\textbf{Y}$, any information regarding $\Pi(J^{(n)})$ is leaked through estimating $\textbf{V}$. This can be rigorously proved similar to the proof of Lemma 1 in \cite{tifs2016}. We can state this fact as
\[
	P\left(Y_u(k)\  \bigg{ | }  \ {V}_u=v_u, \Pi(J^{(n)}) \right)  = P\left(Y_u(k)\  \bigg{ | } \ {V}_u=v_u\right)=v_u.
\]
The right and left hand side are given by $Bernoulli (v_u)$ distributions.

As a result,
\[
\widetilde{W_j^{(n)}} = \sum_{\text{for all \textbf{v}}} P\left(\Pi(1)=j \bigg{|} \Pi (J^{(n)}), \textbf{V}=\textbf{v}\right) P\left(\textbf{V}=\textbf{v} \bigg{|} \textbf{Y}, \Pi \left(J^{(n)}\right)\right).
\]
Note $W_j^{(n)}=P\left(\Pi(1)=j \bigg{|} \Pi (J^{(n)}), \textbf{V}\right)$, so
\begin{align}
\no \widetilde{W_j^{(n)}} &= \sum_{\text{for all \textbf{v}}}W_j^{(n)}  P\left(\textbf{V}=\textbf{v} \bigg{|} \textbf{Y}, \Pi \left(J^{(n)}\right) \right) \\
\nonumber &= \frac{1}{N^{(n)}}\sum_{\text{for all \textbf{v}}} P\left(\textbf{V}=\textbf{v}\bigg{|} \textbf{Y}, \Pi \left(J^{(n)}\right)\right) \\
\nonumber &= \frac {1}{N^{(n)}}.\ \
\end{align}
\end{proof}

To show that no information is leaked, we need to show that the size of set $J^{(n)}$ goes to infinity. This is established in Lemma \ref{lem1}.

\begin{lem}
	\label{lem1}
	If $N^{(n)} \triangleq |J^{(n)}| $, then $N^{(n)} \rightarrow \infty$ with high probability as $n \rightarrow \infty$.  More specifically, there exists $\lambda>0$ such that
\[
	P\left(N^{(n)} > \frac{\lambda}{2}n^{\frac{\beta}{2}}\right) \rightarrow 1.
	\]
\end{lem}

\begin{proof}
Lemma \ref{lem1} is proved in Appendix \ref{sec:app_c}.
\end{proof}

In the final step, we define $\widehat{W_j^{(n)}}$ for any $j \in \Pi (J^{(n)})$ as
\[\widehat{W_j^{(n)}}=P\left(X_1(k)=1 \bigg{|} \textbf{Y}, \Pi (J^{(n)})\right).\]
$\widehat{W_j^{(n)}}$ is the conditional probability that $X_1(k)=1$ after observing the values of the anonymized version of the obfuscated samples of the users' data ($\textbf{Y}$) and the aggregate set including all of the pseudonyms of the users in set $J^{(n)}$ ($\Pi (J^{(n)})$). $\widehat{W_j^{(n)}} $ is a random variable because $\textbf{Y}$ and $\Pi (J^{(n)})$ are random. Now, in the following lemma, we will prove $\widehat{W_j^{(n)}}$ converges in distribution to $p_1$.

Note that this is the probability from the adversary's point of view. That is, given that the adversary has observed $\textbf{Y}$ as well as the extra information $ \Pi (J^{(n)})$, what can he/she infer about $X_1(k)$?
\begin{lem}
	\label{lem4}
	For all $j \in \Pi (J^{(n)})$, $\widehat{W_j^{(n)}} \xrightarrow{d} p_1.$
\end{lem}

\begin{proof}
We know
\begin{align*}
\widehat{W_j^{(n)}}= \sum_{j \in \Pi(J^{(n)})} P\left(X_1(k)=1 \bigg{|} \Pi(1)=j, \textbf{Y}, \Pi (J^{(n)})\right) P\left(\Pi(1)=j \bigg{|} \textbf{Y}, \Pi (J^{(n)})\right),
\end{align*}
and according to the definition $\widetilde{W_j^{(n)}}=P \left(\Pi(1)=j \bigg{|} \textbf{Y}, \Pi (J^{(n)})\right)$, we have
	\begin{align}
	\no \widehat{W_j^{(n)}} &= \sum_{j \in \Pi(J^{(n)})} P\left(X_1(k)=1 \bigg{|} \Pi(1)=j, \textbf{Y}, \Pi (J^{(n)})\right) \widetilde{W_j^{(n)}}\\
	\nonumber &= \frac{1}{N^{(n)}} \sum_{j \in \Pi(J^{(n)})} P\left(X_1(k)=1 \bigg{|} \Pi(1)=j, \textbf{Y}, \Pi (J^{(n)})\right).
	\end{align}
We now claim that
\[
 P\left(X_1(k)=1 \bigg{|} \Pi(1)=j, \textbf{Y}, \Pi (J^{(n)})\right)=p_1+o(1).
\]
The reasoning goes as follows. Given $\Pi(1)=j$ and knowing $\textbf{Y}$, we know that

\[
Y_{\Pi(1)}(k)={Z}_{1}(k)=\begin{cases}
{X}_{1}(k), & \textrm{with probability } 1-R_1.\\
1-{X}_{1}(k), & \textrm{with probability } R_1.
\end{cases}
\]

Thus, given $Y_{j}(k)=1$, Bayes' rule yields:
\begin{align*}
 P\left(X_1(k)=1 \bigg{|} \Pi(1)=j, \textbf{Y}, \Pi (J^{(n)})\right)&= \left(1- R_1 \right) \frac{P(X_1(k)=1)}{P(Y_{\Pi(1)}(k)=1)}\\
 &=\left(1-R_1 \right) \frac{p_1}{p_1 (1- R_1)+(1-p_1)R_1}\\
 &=1-o(1),
\end{align*}
and similarly, given $Y_{j}(k)=0$,
\begin{align*}
P\left(X_1(k)=1 \bigg{|} \Pi(1)=j, \textbf{Y}, \Pi (J^{(n)})\right)&= R_1 \frac{P(X_1(k)=1)}{P(Y_{\Pi(1)}(k)=0)}\\
&=R_1 \frac{p_1}{p_1 (1- R_1)+(1-p_1)R_1}\\
&=o(1).
\end{align*}
Note that by the independence assumption, the above probabilities do not depend on the other values of $Y_{u}(k)$ (as we are conditioning on $\Pi(1)=j$ ).
Thus, we can write
\begin{align}
	\no \widehat{W_j^{(n)}} &= \frac{1}{N^{(n)}} \sum_{j \in \Pi(J^{(n)})} P\left(X_1(k)=1 \bigg{|} \Pi(1)=j, \textbf{Y}, \Pi (J^{(n)})\right)\\
\no &=\frac{1}{N^{(n)}} \sum_{j \in \Pi(J^{(n)}), Y_{j}(k)=1}  (1-o(1)) + \frac{1}{N^{(n)}} \sum_{j \in \Pi(J^{(n)}), Y_{j}(k)=0}  o(1).
\end{align}
First, note that since $\abs*{\left\{j \in \Pi(J^{(n)}), Y_{j}(k)=0\right\}} \leq N^{(n)}$, the second term above converges to zero, thus,
\begin{align}
	\no \widehat{W_j^{(n)}}  \rightarrow \frac{\abs*{\left\{ j \in \Pi(J^{(n)}), Y_{\Pi(1)}(k)=1\right\} }}{N^{(n)}}.
\end{align}
Since for all $j \in \Pi(J^{(n)})$, $ Y_{j}(k)  \sim Bernoulli \left(p_1+o(1)\right)$, by a simple application of Chebyshev's inequality, we can conclude $\widehat{W_j^{(n)}}\rightarrow p_1$. Appendix \ref{sec:app_d} provides the detail.
\end{proof}	

As a result,
\begin{align*}
X_1(k) {|} \textbf{Y}, \Pi (J^{(n)})\rightarrow \textit{Bernoulli} (p_1),
\end{align*}
thus,
\[H\left(X_1(k) \bigg{|} \textbf{Y}, \Pi (J^{(n)})\right)\rightarrow H\left(X_1(k)\right).\]
Since conditioning reduces entropy,
\[H\left(X_1(k) \bigg{|} \textbf{Y}, \Pi (J^{(n)})\right)\leq H\left(X_1(k) \bigg{|} \textbf{Y}\right),\]
and as a result,
\[\lim_{n \rightarrow \infty}   H\left(X_1(k)\right)-H\left(X_1(k) \bigg{|} \textbf{Y}\right) \leq 0,\]
and 
\[\lim_{n \rightarrow \infty}   I\left(X_1(k);\textbf{Y}\right)\leq 0.\]
By knowing that $I\left(X_1(k);\textbf{Y}\right)$ cannot take any negative value, we can conclude that
\[I\left(X_1(k);\textbf{Y}\right)\rightarrow 0.\]
\end{proof}

\subsection{Extension to $r$-States}
Now, assume users' data samples can have $r$ possibilities $\left(0, 1, \cdots, r-1\right)$, and $p_u(i)$ shows the probability of user $u$ having data sample $i$. We define the vector $\textbf{p}_u$ and the matrix $\textbf{p}$ as
\[\textbf{p}_u= \begin{bmatrix}
p_u(1) \\ p_u(2) \\ \vdots \\p_u(r-1) \end{bmatrix} , \ \ \  \textbf{p} =\left[ \textbf{p}_{1}, \textbf{p}_{2}, \cdots,  \textbf{p}_{n}\right].
\]
We assume $p_u(i)$'s are drawn independently from some continuous density function, $f_\textbf{P}(\textbf{p}_u)$, which has support on a subset of the $(0,1)^{r-1}$ hypercube (Note that the $p_u(i)$'s sum to one, so one of them can be considered as the dependent value and the dimension is $r-1$). In particular, define the range of the distribution as
\begin{align}
\no  \mathcal{R}_{\textbf{p}} &= \{ (x_1,x_2,  \cdots, x_{r-1}) \in (0,1)^{r-1}:  x_i > 0 , x_1+x_2+\cdots+x_{r-1} < 1,\ i=1, 2, \cdots, r-1\}.
\end{align}
Figure~\ref{fig:rp} shows the range $\mathcal{R}_{\textbf{p}}$ for the case where $r=3$.
\begin{figure}[h]
	\centering
	\includegraphics[width = 0.5\linewidth]{fig/rp}
	\caption{$\mathcal{R}_{\textbf{p}}$ for case $r=3$.}
	\label{fig:rp}
\end{figure}


Then, we assume there are $\delta_1, \delta_2>0$ such that:
\begin{equation}
\begin{cases}
\no    \delta_1<f_{\textbf{P}}(\mathbf{p}_u) <\delta_2, & \textbf{p}_u \in \mathcal{R}_{\textbf{p}}.\\
    f_{\textbf{P}}(\mathbf{p}_u)=0, &  \textbf{p}_u \notin \mathcal{R}_{\textbf{p}}.
\end{cases}
\end{equation}

The obfuscation is similar to the two-states case. Specifically, for $l \in \{0,1,\cdots, r-1\}$, we can write
\[
P({Z}_{u}(k)=l| X_{u}(k)=i) =\begin{cases}
1-R_u, & \textrm{for } l=i.\\
\frac{R_u}{r-1}, & \textrm{for } l \neq i.
\end{cases}
\]



%


\begin{thm}\label{r_state_thm}
For the above $r$-states model, if $\textbf{Z}$ is the obfuscated version of $\textbf{X}$, and $\textbf{Y}$ is the anonymized version of $\textbf{Z}$ as defined previously, and
\begin{itemize}
	 \item $m=m(n)$ is arbitrary;
	\item $R_u \sim Uniform [0, a_n]$, where $a_n \triangleq c'n^{-\left(\frac{1}{r-1}-\beta\right)}$ for any $c'>0$ and $0<\beta<\frac{1}{r-1}$;
\end{itemize}
then, user 1 has perfect privacy. That is,
\begin{align}%\label{}
\no  \forall k\in \mathbb{N}, \ \ \ \lim\limits_{n\rightarrow \infty} I \left(X_1(k);{\textbf{Y}}\right) =0.
\end{align}\end{thm}

The proof of Theorem \ref{r_state_thm} is similar to the proof of Theorem \ref{two_state_thm}. The major difference is that instead of the random variables $P_u, Q_u, V_u$, we need to consider the random vectors $\textbf{P}_u, \textbf{Q}_u, \textbf{V}_u$.  Similarly, for user $u$, we define the vector $\textbf{Q}_u$ as
\[\textbf{Q}_u= \begin{bmatrix}
Q_u(1) \\ Q_u(2) \\ \vdots \\Q_u(r-1) \end{bmatrix}.
\]

In the $r$-states case,
\begin{align}
\no {Q}_u(i) &=P_u(i)\bigg(1-R_u(i) \bigg)+\bigg(1-P_u(i)\bigg)\frac{R_u}{r-1}  \\
\nonumber &= P_u+\bigg(1-r P_u\bigg)\frac{R_u}{r-1}.\ \
\end{align}
%Note that we have dropped the superscript $n$ for simplicity of notation (we need to write $Q_u^{(n)}(1)$ instead of $Q_u(1)$).
We also need to define the critical set $J^{(n)}$.  First, for $i=0,1, \cdots, r-1$, define set $J_i^{(n)}$ as follows. If $0\leq p_1(i)<\frac{1}{r}$, then,
\begin{align*}
&J_i^{(n)}= \\
&\left\{u \in \{1, 2, \dots, n\}: p_1(i) \leq P_u(i)\leq p_1(i)+\epsilon_n; p_1(i)+\epsilon_n\leq Q_u(i)\leq p_1(i)+(1-r p_1(i))\frac{a_n}{r-1}\right\},
\end{align*}
where $\epsilon_n \triangleq \frac{1}{n^{\frac{1}{r-1}-\frac{\beta}{2}}}$,  $a_n = c'n^{-\left(\frac{1}{r-1}-\beta\right)}$, and $\beta$ is defined in the statement of Theorem \ref{r_state_thm}.

We then define the critical set $J^{(n)}$ as:
\[
J^{(n)}=\bigcap_{l=0}^{r-1} J_i^{(n)}.
\]
We can then repeat the same arguments in the proof of Theorem \ref{two_state_thm} to complete the proof.

%\[J_i^{(n)}=
%\left\{  u \in \{1, 2, \dots, n\}: p_1(i) \leq P_u(i)\leq p_1(1)+\epsilon_n;  p_1(1)+\epsilon_n\leq Q_u(i)\leq p_1(1)+(1-2p_1(1)).a_n\right\}, \ \   \text{for}\ \ 0\leq p_1(1)<\frac{1}{2}\\
%\left\{  u \in \{1, 2, \dots, n\}: p_1(i)-\epsilon_n \leq P_u(i)\leq p_1(1); p_1(1)+(1-2p_1(1)).a_n\leq Q_u(i)\leq p_1-\epsilon_n\right\}, \ \  \text{for}\ \ \frac{1}{2}\leq p_1\leq1
%\end{cases}\]




\section{Converse Results: No Privacy Region}
\label{converse}

In this section, we prove that if the number of observations by the adversary is larger than its critical value and the noise level is less than its critical value, then the adversary can find an algorithm to successfully estimate users' data samples with arbitrarily small error probability. Combined with the results of the previous section, this implies that asymptotically (as $n \rightarrow \infty$), privacy can be achieved \emph{if and only if} at least one of  the two techniques (obfuscation or anonymization) are used above their thresholds. This statement needs a clarification as follows:  Looking at the results of \cite{tifs2016}, we notice that anonymization alone can provide perfect privacy if $m(n)$ is below its threshold. On the other hand, the threshold for obfuscation requires some anonymization: In particular, the identities of the users must be permuted once to prevent the adversary from readily identifying the users.



%is valid when both anonymization and obfuscation are simultaneously used as presented above (Figure \ref{fig:xyz}). In other words, if only obfuscation is used, then clearly the limit that we obtained above for $a_n$ is not nearly enough for privacy as it approaches zero as $n$ goes to infinity.

\subsection{Two-States Model}
Again, we start with the i.i.d.\ two-states model. The data sample of user $u$ at any time is a Bernoulli random variable with parameter $p_u$.

As before, we assume that $p_u$'s are drawn independently from some continuous density function, $f_P(p_u)$, on the $(0,1)$ interval. Specifically, there are $\delta_1, \delta_2>0$ such that:%\footnote{The condition $\delta_1<f_P(p_u) <\delta_2$ is not actually necessary for the results and can be relaxed; however, we keep it here to avoid unnecessary technicalities.}:
\begin{equation}
\no\begin{cases}
    \delta_1<f_P(p_u) <\delta_2, & p_u \in (0,1).\\
    f_P(p_u)=0, &  p _u\notin (0,1).
\end{cases}
\end{equation}

\begin{thm}\label{two_state_thm_converse}
For the above two-states mode, if $\textbf{Z}$ is the obfuscated version of $\textbf{X}$, and $\textbf{Y}$ is the anonymized version of $\textbf{Z}$ as defined, and
\begin{itemize}
	\item $m =cn^{2 +  \alpha}$ for any $c>0$ and $\alpha>0$;
	\item $R_u \sim Uniform [0, a_n]$, where $a_n \triangleq c'n^{-\left(1+\beta\right)}$ for any $c'>0$ and $\beta>\frac{\alpha}{4}$;
\end{itemize}
then, user $1$ has no privacy as $n$ goes to infinity.
\end{thm}

Since this is a converse result, we give an explicit detector at the adversary and show that it can be used by the adversary to recover the true data of user $1$.

\begin{proof}
The adversary first inverts the anonymization mapping $\Pi$ to obtain $Z_1(k)$, and then estimates the value of $X_1(k)$ from that. To invert the anonymization, the adversary calculates the empirical probability that each string is in state $1$ and then assigns the string with the empirical probability closest to $p_1$ to user 1.

\begin{figure}
	\centering
	\includegraphics[width=.8\linewidth]{fig/converse.jpg}
	\centering
	\caption{$p_1$, sets $B^{(n)}$ and $C^{(n)}$ for case $r=2$.}
	\label{fig:converse}
\end{figure}


Formally, for $u=1, 2, \cdots, n$, the adversary computes $\overline{Y_u}$, the empirical probability of user $u$ being in state $1$, as follows:
\[
\overline{Y_u}=\frac{Y_u(1)+Y_u(2)+ \cdots +Y_u(m)}{m},
\]
thus,
\[
\overline{Y_{\Pi(u)}}=\frac{Z_u(1)+Z_u(2)+ \cdots +Z_u(m)}{m}.
\]

As shown in Figure \ref{fig:converse}, define
\[B^{(n)}\triangleq \left\{x \in (0,1); p_1-\Delta_n \leq x \leq p_1+\Delta_n\right\},\]
where $\Delta_n = \frac{1}{n^{1+\frac{\alpha}{4}}}$ and $\alpha $ is defined in the statement of Theorem \ref{two_state_thm_converse}. We claim that for $m =cn^{2 +  \alpha}$, $a_n=c'n^{-(1 +  \beta)}$, and large enough $n$,
\begin{enumerate}
\item $P\left( \overline{Y_{\Pi(1) }}\in B^{(n)}\right) \rightarrow 1.$
\item $P\left( \bigcup\limits_{u=2}^{n} \left(\overline{Y_{\Pi(u)}}\in B^{(n)}\right)\right) \rightarrow 0.$
\end{enumerate}
As a result, the adversary can identify $\Pi(1)$ by examining  $\overline{Y_u}$'s and assigning the one in $B^{(n)}$ to user $1$. Note that $\overline{Y_{\Pi(u) }} \in B^{(n)}$ is a set (event) in the underlying probability space and can be written as $\left\{\omega \in \Omega: \overline{Y_{\Pi(u) }}(\omega) \in B^{(n)}\right\}$.

First, we show that as $n$ goes to infinity,
\[P\left( \overline{Y_{\Pi(1) }}\in B^{(n)}\right) \rightarrow 1.\]
We can write
\begin{align}
\no P\left(\overline{Y_{\Pi(1)}} \in B^{(n)}\right) &= P\left(\frac{\sum\limits_{k=1}^{m}Z_1(k)}{m} \in  B^{(n)} \right)\\
\nonumber &= P\left(p_1-\Delta_n \leq\frac{\sum\limits_{k=1}^{m}Z_1(k)}{m}\leq p_1+\Delta_n \right)\\
\nonumber &= P\left(mp_1-m\Delta_n-mQ_1\leq \sum\limits_{k=1}^{m}Z_1(k)-mQ_1\leq  mp_1+m\Delta_n-mQ_1 \right).\ \
\end{align}
%Note as n becomes large, $a_n\ll \Delta_n$, as a result,
%\[P\left(\overline{Y_{\Pi(1)}^{(n)}} \in B^{(n)}\right)=P\left(\abs*{\sum\limits_{i=1}^{m}Z_1^{(n)}(i)-mQ_1^{(n)}}< m\Delta_n \right).\]
Note that for any $u \in \{1,2,\cdots, n \}$, we have
\begin{align}
\no |p_u-{Q}_u| &=|1-2p_u|R_u \\
\no & \leq R_u \leq a_n,
\end{align}
so, we can conclude
\begin{align}
\no P\left(\overline{Y_{\Pi(1)}} \in B^{(n)}\right) &= P\left(mp_1-m\Delta_n-mQ_1\leq \sum\limits_{k=1}^{m}Z_1(k)-mQ_1 \leq mp_1+m\Delta_n-mQ_1 \right)\\
\nonumber  &\geq P\left(-m\Delta_n+m a_n\leq \sum\limits_{k=1}^{m}Z_1(k)-mQ_1\leq -ma_n+m\Delta_n \right)\\
\nonumber &= P\left(\abs*{\sum\limits_{k=1}^{m}Z_1(k)-mQ_1}\leq m (\Delta_n-a_n) \right).\ \
\end{align}
Since $a_n \rightarrow 0$, for $p_1 \in (0,1)$ and large enough $n$, we can say $p_1+a_n < 2 p_1$. From Chernoff bound, for any $c,c',\alpha>0$ and $\beta>\frac{\alpha}{4}$,
\begin{align}
%\nonumber
\no P\left(\abs*{\sum\limits_{k=1}^{m}Z_1(k)-mQ_1}\leq m (\Delta_n-a_n) \right) &\geq 1-2e^{-\frac{m(\Delta_n-a_n)^2}{3Q_1}} \\
\nonumber &\geq 1-2e^{-\frac{1}{3(p_1+a_n)}cn^{2+\alpha}\left(\frac{1}{n^{1+\frac{\alpha}{4}}}- \frac{c'}{n^{1 +  \beta}}\right)^2}\\
%\nonumber &\geq 1-2e^{-\frac{c}{3c'(1-2p_1)}n^{1+\frac{3\alpha}{2}}}\\
\nonumber &\geq 1-2e^{-\frac{c''}{6 p_1}n^{\frac{\alpha}{2}}}\rightarrow 1. \ \
\end{align}
As a result, as $n$ becomes large,
\[P\left(\overline{Y_{\Pi(1)}} \in B^{(n)}\right) \rightarrow 1.\]

Now, we need to show that as $n$ goes to infinity,
\[P\left( \bigcup\limits_{u=2}^{n} \left(\overline{Y_{\Pi(u)}}\in B^{(n)}\right)\right) \rightarrow 0.\]
First, we define
\[
C^{(n)}=\left\{x\in (0,1); p_1-2\Delta_n \leq x \leq p_1+2\Delta_n\right\}
,\]
and claim as $n$ goes to infinity,
\[P\left(\bigcup\limits_{u=2}^n \left(P_u \in C^{(n)}\right) \right)\rightarrow 0. \]

Note
\[4\Delta_n\delta_1< P\left( P_u\in C^{(n)}\right) < 4 \Delta_n\delta_2,\]
and according to the union bound, for large enough $n$,
\begin{align}
\no P\left( \bigcup\limits_{u=2}^n \left(P_u \in C^{(n)}\right) \right) &\leq \sum\limits_{u=2}^n P\left( P_u \in C^{(n)}\right) \\
\nonumber &\leq 4n \Delta_n \delta_2\\
\nonumber &= 4n \frac{1}{n^{1+{\frac{\alpha}{4}}}} \delta_2\\
\nonumber &= 4n^{-\frac{\alpha}{4}}\delta_2 \rightarrow 0. \ \
%\nonumber &\leq 4n^{-\frac{\alpha}{4}}\delta_2 \rightarrow 0. \ \
\end{align}
As a result, we can conclude that all $p_u$'s are outside of $C^{(n)}$ for $u \in \left\{2,3, \cdots, n\right\}$ with high probability.

Now, we claim that given all $p_u$'s are outside of $C^{(n)}$, $P\left(\overline{Y_{\Pi (u)}} \in B^{(n)}\right)$ is small. Remember that for any $u \in \{1,2,\cdots, n \}$, we have
\begin{align}
\no |p_u-{Q}_u| \leq a_n.
\end{align}
Now, noting the definitions of sets $B^{(n)}$ and $C^{(n)}$, we can write for $u \in \left\{2,3,  \cdots , n\right\}$,
\begin{align}
\nonumber
\no P\left(\overline{Y_{\Pi(u)}} \in B^{(n)}\right) &\leq P\left(\abs*{\overline{Y_{\Pi(u)}}-Q_u}\geq  (\Delta_n-a_n) \right)\\
%\nonumber &\hspace{-0.2 in}\leq  P\left(\frac{\sum\limits_{i=1}^{m}Z_j^{(n)}(i)}{m}<p_j-\Delta_n ,\frac{\sum\limits_{i=1}^{m}Z_j^{(n)}(i)}{m}>p_j+\Delta_n \right)\\
%\nonumber &\hspace{-0.2 in}\leq P\left(\sum\limits_{i=1}^{m}Z_j^{(n)}(i)-mQ_j^{(n)}<mp_j-m\Delta_n-mQ_j^{(n)}, \sum\limits_{i=1}^{m}Z_j^{(n)}(i)-mQ_j^{(n)}> mp_j+m\Delta_n-mQ_j^{(n)} \right)\\
%\nonumber &\hspace{-0.2 in}\leq P\left(\sum\limits_{i=1}^{m}Z_j^{(n)}(i)-mQ_j^{(n)}<ma_n-m\Delta_n, \sum\limits_{i=1}^{m}Z_j^{(n)}(i)-mQ_j^{(n)}>-ma_n+m\Delta_n \right)\\
\nonumber &= P\left(\abs*{\sum\limits_{k=1}^{m}Z_u(k)-mQ_u}> m(\Delta_n-a_n) \right).\ \
\end{align}
%We also know that as n becomes large, $a_n\ll \Delta_n$, as a result,
%\[P\left(\overline{Y_{\Pi(j)}^{(n)}} \in B^{(n)}\right)\leq P\left(\abs*{\sum\limits_{i=1}^{m}Z_j^{(n)}(i)-mQ_j^{(n)}}> m\Delta_n \right).\]
According to the Chernoff bound, for any $c,c',\alpha>0$ and $\beta>\frac{\alpha}{4}$,
\begin{align}
%\nonumber
\no P\left(\abs*{\sum\limits_{k=1}^{m}Z_u(k)-mQ_u}> m(\Delta_n-a_n) \right) &\leq 2e^{-\frac{m(\Delta_n-a_n)^2}{3Q_1}} \\
\nonumber &\leq 2e^{-\frac{1}{3(p_1+a_n)}cn^{2+\alpha}\left(\frac{1}{n^{1+\frac{\alpha}{4}}}- \frac{c'}{n^{1 +  \beta}}\right)^2}\\
%\nonumber &\leq 2e^{-\frac{c}{3c'(1-2p_1)}n^{1+\frac{3\alpha}{2}}}\\
\nonumber &\leq 2e^{-\frac{c''}{6 p_1}n^{\frac{\alpha}{2}}}. \ \
\end{align}
Now, by using a union bound, we have
\begin{align}
\no P\left( \bigcup\limits_{u=2}^n \left(\overline{Y_{\Pi(u)}}\in B^{(n)}\right)\right)&\leq \sum\limits_{u=2}^{n}P\left(\overline{Y_{\Pi(u)}}\in B^{(n)}\right)\\
\nonumber &\leq n\left(2e^{-\frac{c''}{6 p_1}n^{\frac{\alpha}{2}}}\right),\ \
\end{align}
and thus, as $n$ goes to infinity,
\[P\left( \bigcup\limits_{u=2}^n \left(\overline{Y_{\Pi(u)}}\in B^{(n)}\right)\right) \rightarrow 0.\]

So, the adversary can successfully recover $Z_1(k)$. Since $Z_{1}(k)=X_1(k)$ with probability $1-R_1=1-o(1)$, the adversary can recover $X_{1}(k)$ with vanishing error probability for large enough $n$.
\end{proof}

\subsection{Extension to $r$-States}
Now, assume users' data samples can have $r$ possibilities $\left(0, 1, \cdots, r-1\right)$, and $p_u(i)$ shows the probability of user $u$ having data sample $i$. We define the vector $\textbf{p}_u$ and the matrix $\textbf{p}$ as
\[\textbf{p}_u= \begin{bmatrix}
p_u(1) \\ p_u(2) \\ \vdots \\p_u(r-1) \end{bmatrix} , \ \ \  \textbf{p} =\left[ \textbf{p}_{1}, \textbf{p}_{2}, \cdots,  \textbf{p}_{n}\right].
\]
We also assume $\textbf{p}_u$'s are drawn independently from some continuous density function, $f_P(\textbf{p}_u)$, which has support on a subset of the $(0,1)^{r-1}$ hypercube. In particular, define the range of distribution as
\begin{align}
\no  \mathcal{R}_{\textbf{p}} &= \left\{ (x_1, x_2, \cdots, x_{r-1}) \in (0,1)^{r-1}: x_i > 0 , x_1+ x_2+\cdots+ x_{r-1} < 1,\ \ i=1, 2,\cdots, r-1\right\}.
\end{align}
Then, we assume there are $\delta_1, \delta_2>0$ such that:
\begin{equation}
\begin{cases}
\no    \delta_1<f_{\textbf{P}}(\mathbf{p}_u) <\delta_2, & \textbf{p}_u \in \mathcal{R}_{\textbf{p}}.\\
    f_{\textbf{P}}(\mathbf{p}_u)=0, &  \textbf{p}_u \notin \mathcal{R}_{\textbf{p}}.
\end{cases}
\end{equation}

\begin{thm}\label{r_state_thm_converse}
For the above $r$-states mode, if $\textbf{Z}$ is the obfuscated version of $\textbf{X}$, and $\textbf{Y}$ is the anonymized version of $\textbf{Z}$ as defined, and
\begin{itemize}
	\item $m =cn^{\frac{2}{r-1} +  \alpha}$ for any $c>0$ and $0<\alpha<1$;
	\item $R_u \sim Uniform [0, a_n]$, where $a_n \triangleq c'n^{-\left(\frac{1}{r-1}+\beta\right)}$ for any $c'>0$ and $\beta>\frac{\alpha}{4}$;
\end{itemize}
then, user $1$ has no privacy as $n$ goes to infinity. 
\end{thm}


The proof of Theorem \ref{r_state_thm_converse} is similar to the proof of Theorem \ref{two_state_thm_converse}, so we just provide the general idea. We similarly define the empirical probability that the user with pseudonym $u$ has data sample $i$ $\left(\overline{{Y}_{u}}(i)\right)$ as follows:

\[
\overline{Y_u}(i)=\frac{\abs {\left\{k \in \{1, 2, \cdots, m\}:Y_u(k)=i\right\}}}{m},
\]
thus,
\[
\overline{Y_{\Pi(u)}}(i)=\frac{\abs {\left\{k \in \{1, 2, \cdots, m\}:Y_u(k)=i\right\}}}{m}.
\]

The difference is that now for each $u \in \{1,2,\cdots, n \}$, $\overline{\textbf{Y}_{u}}$ is a vector of size $r-1$. In other words,
\[\overline{\textbf{Y}_{u}}=\begin{bmatrix}
\overline{Y_u}(1) \\ \overline{Y_u}(2) \\ \vdots \\\overline{Y_u}(r-1) \end{bmatrix}.\]


\begin{figure}
  \centering
  \includegraphics[width=.5\linewidth, height=0.5 \linewidth]{fig/R.jpg}
  \centering
  \caption{$\textbf{p}_1$, sets $B'^{(n)}$ and $C'^{(n)}$ in $\mathcal{R}_\textbf{p}$ for case $r=3$.}
  \label{fig:rpp}
\end{figure}

Define sets $B'^{(n)}$ and $C'^{(n)}$ as
\begin{align}
 \no B'^{(n)}\triangleq & \left\{(x_1,x_2, \cdots ,x_{r-1}) \in \mathcal{R}_{\textbf{p}}: p_1(i)-\Delta'_n \leq x_i \leq p_1(i)+\Delta'_n,\ i=1,2, \cdots, r-1\right\},
\end{align}
\begin{align}
\no C'^{(n)}\triangleq &\left\{(x_1,x_2, \cdots ,x_{r-1}) \in \mathcal{R}_{\textbf{p}}: p_1(i)-2 \Delta'_n \leq x_i \leq p_1(i)+2 \Delta'_n,\ i=1,2, \cdots ,r-1\right\},
\end{align}
where $\Delta'_n = \frac{1}{n^{\frac{1}{r-1}+\frac{\alpha}{4}}}.$ Figure \ref{fig:rpp} shows $\textbf{p}_1$ and sets  $B'^{(n)}$ and $C'^{(n)}$ for the case $r=3.$

We claim for $m =cn^{\frac{2}{r-1} +  \alpha}$ and large enough $n$,
\begin{enumerate}
\item $P\left( \overline{\textbf{Y}_{\Pi(1) }}\in B'^{(n)}\right) \rightarrow 1$.
\item $P\left( \bigcup\limits_{u=2}^{n} \left(\overline{\textbf{Y}_{\Pi(u)}}\in B'^{(n)}\right)\right) \rightarrow 0.$
\end{enumerate}
The proof follows that for the two-states case. Thus, the adversary can de-anonymize the data and then recover $X_1(k)$ with vanishing error probability in the $r$-states model.



\subsection{Markov Chain Model} 
\label{subsec:markov}

So far, we have assumed users' data samples can have $r$ possibilities $\left(0, 1, \cdots, r-1\right)$ and users' pattern are i.i.d.\ . Here we model users' pattern using Markov chains to capture the dependency of the users' pattern over time. Again, we assume there are $r$ possibilities (the number of states in the Markov chains). Let $E$ be the set of edges. More specifically, $(i, l) \in E$ if there exists an edge from $i$ to $l$ with probability $ p(i,l)>0 $. What distinguishes different users is their transition probabilities $p_u(i,l)$ (the probability that user $u$ jumps from state $i$ to state $l$). The adversary knows the transition probabilities of all users. The model for obfuscation and anonymization is exactly the same as before.

We show that the adversary will be able to estimate the data samples of the users with low error probability if $m(n)$ and $a_n$ are in the appropriate range. The key idea is that the adversary can focus on a subset of the transition probabilities that are sufficient for recovering the entire transition probability matrix. By estimating those transition probabilities from the observed data and matching with the known transition probabilities of the users, the adversary will be able to first de-anonymize the data, and then estimate the actual samples of users' data. In particular, note that for each state $i$, we must have
\[\sum\limits_{l=1}^{r} p_u(i,l)=1,   \  \  \textrm{ for each }u \in \{1,2,\cdots, n \}, \]
so, the Markov chain of user $u$ is completely determined by a subset of size $d=|E|-r$ of transition probabilities. We define the vector $\textbf{p}_u$ and the matrix $\textbf{p}$ as
\[\textbf{p}_u= \begin{bmatrix}
p_u(1) \\ p_u(2) \\ \vdots \\p_u(|E|-r) \end{bmatrix} , \ \ \  \textbf{p} =\left[ \textbf{p}_{1}, \textbf{p}_{2}, \cdots,  \textbf{p}_{n}\right].
\]


We also consider $\textbf{p}_u$'s are drawn independently from some continuous density function, $f_P(\textbf{p}_u)$, which has support on a subset of the $(0,1)^{|E|-r}$ hypercube. Let $\mathcal{R}_{\textbf{p}} \subset \mathbb{R}^{d}$ be the range of acceptable values for $\textbf{p}_{u}$, so we have
\begin{align}
\no  \mathcal{R}_{\textbf{P}} &= \left\{ (x_1,x_2 \cdots, x_{d}) \in (0,1)^{d}: x_i > 0 , x_1+x_2+\cdots+x_{d} < 1,\ \ i=1,2,\cdots, d\right\}.
\end{align}
As before, we assume there are $ \delta_1, \delta_2 >0$, such that:
\begin{equation}
\begin{cases}
\no    \delta_1<f_{\textbf{P}}(\textbf{p}_u) <\delta_2, & \textbf{p}_u \in \mathcal{R}_{\textbf{p}}.\\
    f_{\textbf{P}}(\textbf{p}_u)=0, &  \textbf{p}_u \notin \mathcal{R}_{\textbf{p}}.
\end{cases}
\end{equation}

Using the above observations, we can establish the following theorem.


\begin{thm}\label{markov_thm}
For an irreducible, aperiodic Markov chain with $r$ states and $|E|$ edges as defined above, if $\textbf{Z}$ is the obfuscated version of $\textbf{X}$, and $\textbf{Y}$ is the anonymized version of $\textbf{Z}$, and
\begin{itemize}
 \item $m =cn^{\frac{2}{|E|-r} +  \alpha}$ for any $c>0$ and $\alpha>0$;
\item $R_u \sim Uniform [0, a_n]$, where $a_n \triangleq c'n^{-\left(\frac{1}{|E|-r}+\beta \right)}$ for any $c'>0$ and $\beta>\frac{\alpha}{4}$;
\end{itemize}
then, the adversary can successfully identify the data of user $1$ as $n$ goes to infinity. 
\end{thm}
The proof has a lot of similarity to the i.i.d.\ case, so we provide a sketch, mainly focusing on the differences. We argue as follows. If the total number of observations per user is $m=m(n)$, then define $M_i(u)$ to be the total number of visits by user $u$ to state $i$, for $i=0, 1, \cdots, r-1$. Since the Markov chain is irreducible and aperiodic, and $m(n) \rightarrow \infty$, all $\frac{M_i(u)}{m(n)}$ converge to their stationary values. Now conditioned on $M_i(u)=m_i(u)$, the transitions from state $i$ to state $l$ for user $u$ follow a multinomial distribution with probabilities $p_u(i,l)$.

Given the above, the setting is now very similar to the i.i.d.\ case. Each user is uniquely characterized by a vector $\textbf{p}_u$ of size $|E|-r$. We define sets $B^{''(n)}$ and $C^{''(n)}$ as
\[
 B^{''(n)}\triangleq \{(x_1, x_2, \cdots ,x_{d}) \in \mathcal{R}_{\textbf{p}}: p_1(i)-\Delta''_n \leq x_i \leq p_1(i)+\Delta''_n, i=1,2 , \cdots ,d\},
\]
\[
 C^{''(n)}\triangleq \{(x_1, x_2, \cdots ,x_{d}) \in \mathcal{R}_{\textbf{p}}:
  p_1(i)-2 \Delta''_n < x_i < p_1(i)+2 \Delta''_n, i=1,2, \cdots, d\},
\]
where $\Delta''_n= \frac{1}{n^{\frac{1}{|E|-r}+\frac{\alpha}{4}}}$, and $d= |E|-r$. Then, we can show that for the stated values of $m(n)$ and $a_n$, as $n$ becomes large:
\begin{enumerate}
\item $P\left( \overline{\textbf{Y}_{\Pi(1) }}\in B^{''(n)}\right) \rightarrow 1$,
\item $P\left( \bigcup\limits_{u=2}^{n} \left(\overline{\textbf{Y}_{\Pi(u)}}\in B{''}^{(n)}\right)\right) \rightarrow 0$,
\end{enumerate}
which means that the adversary can estimate the data of user $1$ with vanishing error probability. The proof is very similar to the proof of the i.i.d.\ case; however, there are two differences that need to be addressed:

First, the probability of observing an erroneous observation is not exactly given by $R_u$. In fact, a transition is distorted if at least one of its nodes is distorted. So, if the actual transition is from state $i$ to state $l$, then the probability of an erroneous observation is equal to
\begin{align}
	\no R'_u&=R_u+R_u-R_uR_u =R_u(2-R_u).
\end{align}


Nevertheless, here the order only matters, and the above expression is still in the order of $a_n =O \left( n^{-\left(\frac{1}{|E|-r}+\beta \right)} \right)$.

The second difference is more subtle. As opposed to the i.i.d.\ case, the error probabilities are not completely independent. In particular, if $X_u(k)$ is reported in error, then both the transition to that state and from that state are reported in error. This means that there is a dependency between errors of adjacent transitions. We can address this issue in the following way:  The adversary makes his decision only based on a subset of the observations. More specifically, the adversary looks at only odd-numbered transitions: First, third, fifth, etc., and ignores the even-numbered transitions.  In this way, the number of observations is effectively reduced from $m$ to $\frac{m}{2}$ which again does not impact the order of the result (recall that the Markov chain is aperiodic). However, the adversary now has access to observations with independent errors.



\section{Perfect Privacy Analysis: Markov Chain Model}\label{sec:perfect-MC}
So far, we have provided both achievability and converse results for the i.i.d.\ case. However, we have only provided the converse results for the Markov chain case. Here, we investigate achievability for Markov chain models. It turns out that for this case, the assumed obfuscation technique is not sufficient to achieve a reasonable level of privacy. Loosely speaking, we can state that if the adversary can make enough observations, then he can break the anonymity. The culprit is the fact that the sequence observed by the adversary is no longer modeled by a Markov chain; rather, it can be modeled by a hidden Markov chain. This allows the adversary to successfully estimate the obfuscation random variable $R_u$ as well as the $ p_u(i,l)$ values for each sequence, and hence successfully de-anonymize the sequences.


More specifically, as we will see below, there is a fundamental difference between the i.i.d.\ case and the Markov chain case. In the i.i.d.\ case, if the noise level is beyond a relatively small threshold, the adversary will be unable to de-anonymize the data and unable to recover the actual values of the data sets for users, \emph{regardless of the (large) size of $m=m(n)$}. On the other hand, in the Markov chain case, if $m=m(n)$ is large enough, then the adversary can easily de-anonymize the data. To better illustrate this, let's consider a simple example.

\begin{example}
Consider the scenario where there are only two states and the users' data samples change between the two states according to the Markov chain shown in Figure \ref{fig:MC-diagram}. What distinguishes the users is their different values of $p$. Now, suppose we use the same obfuscation method as before. That is, to create a noisy version of the sequences of data samples, for each user $u$, we generate the random variable $R_u$ that is the probability that the data sample of the user is changed to a different data sample by obfuscation. Specifically,
\[
{Z}_{u}(k)=\begin{cases}
{X}_{u}(k), & \textrm{with probability } 1-R_u.\\
1-{X}_{u}(k),& \textrm{with probability } R_u.
\end{cases}
\]

\begin{figure}[H]
\begin{center}
\[
\SelectTips {lu}{12 scaled 2500}
\xymatrixcolsep{6pc}\xymatrixrowsep{5pc}\xymatrix{
*++[o][F]{0}  \ar@/^1pc/[r]^{1}
& *++[o][F]{1} \ar@(dr,ur)[]_{1-p}  \ar@/^1pc/[l]^{p}
}
\]
\caption{A state transition diagram.}\label{fig:MC-diagram}
\end{center}
\end{figure}
To analyze this problem, we can construct the underlying Markov chain as follows. Each state in this Markov chain is identified by two values: the real state of the user, and the observed value by the adversary. In particular, we can write
\[\left(\text{Real value}, \text{Observed value}\right) \in \left\{\right(0,0), (0,1), (1,0), (1,1)\}.\]
Figure \ref{fig:MC-diagram2} shows the state transition diagram of this new Markov chain.

\begin{figure}[H]
\begin{center}
\[
\SelectTips {lu}{12 scaled 2000}
\xymatrixcolsep{10pc}\xymatrixrowsep{8pc}\xymatrix{
*++[o][F]{00} \ar@//[d]^{R} \ar@/_1pc/[dr]_>>>>>{1-R}
& *++[o][F]{01} \ar@//[d]^{1-R} \ar@/^1pc/[dl]^>>>>>{R} \\
*++[o][F]{10} \ar@(d,l)[]^{(1-p)R} \ar@/_2pc/[r]_{(1-p)(1-R)} \ar@/^2pc/[u]^{p(1-R)} \ar@/^1pc/[ur]^>>>>>{pR}
& *++[o][F]{11} \ar@(d,r)[]_{(1-p)(1-R)}  \ar@//[l]^{(1-p)R} \ar@/_2pc/[u]_{pR} \ar@/_1pc/[ul]_>>>>>{p(1-R)}
}
\]
\caption{The state transition diagram of the new Markov chain.}\label{fig:MC-diagram2}
\end{center}
\end{figure}
 We know
 \[ \pi_{00}=\pi_0(1-R)=\frac{p}{1+p}(1-R).\]
 \[ \pi_{01}=\pi_0R= \frac{p}{1+p}R.\]
 \[ \pi_{10}=\pi_1R=\frac{1}{1+p}R.\]
 \[ \pi_{11}=\pi_1(1-R)=\frac{1}{1+p}(1-R).\]

The observed process by the adversary is not a Markov chain; nevertheless, we can define limiting probabilities. In particular, let $\theta_0$ be the limiting probability of observing a zero. That is, we have
\[
\frac{M_0}{m} \xrightarrow{d} \theta_0,  \ \ \textrm{ as }n \rightarrow \infty,
\]
where $m$ is the total number of observations by the adversary, and $M_0$ is the number of $0$'s observed. Then,
\[\theta_0= \pi_{00}+\pi_{10} =\frac{(1-R)p+R}{1+p}.\]
Also, let $\theta_1$ be the limiting probability of observing a one, so
\[\theta_1= \pi_{01}+\pi_{11} =\frac{pR+(1-R)}{1+p}=1-\theta_0.\]


Now the adversary's estimate of $\theta_0$ is given by:
\begin{align}\label{eq1}
\hat{\theta}_0= \frac{(1-R)p+R}{1+p}.
\end{align}
Note that if the number of observations by the adversary can be arbitrarily large, the adversary can obtain an arbitrarily accurate estimate of $\theta_0$.
The adversary can obtain another equation easily, as follows.   Let $\theta_{01}$ be the limiting value of the portion of transitions from state $0$ to $1$ in the chain observed by the adversary. We can write
\begin{align}
%\nonumber
\no \theta_{01} &=P \left\{(00\rightarrow 01), (00\rightarrow 11), (10 \rightarrow 01), (10 \rightarrow 11) \right\}\\\
\nonumber &= \pi_{00}(1-R)+\pi_{10}PR+ \pi_{10}(1-p)(1-R).\ \
\end{align}
As a result,
\begin{align}\label{eq2}
\hat{\theta}_{01}= \frac{p(1-R)^2+R\left(PR(1-R)(1-p)\right)}{1+p}.
\end{align}
Again, if the number of observations can be arbitrarily large, the adversary can obtain an arbitrarily accurate estimate of $\theta_{01}$.
By solving the Equations \ref{eq1} and \ref{eq2}, the adversary can successfully recover $R$ and $p$; thus, he/she can successfully determine the users' data values.

\end{example}



%%%%%%%%%%%%%%%%%%%%%%%%%%%%%%%%%%%%%%%%%%%%%%%%%%%%%

%%%%%%%%%%%%%%%%%%%%%%%%%%%%%%%%%%%%%

%%%%%%%%%%%%%%%%%%%%%%%%%%%%%%%
%\section{Simulation Studies}\label{sec:simulation}
In this section, we are mainly interested in the empirical performance of the ABESS algorithm on logistic regression and Poisson regression.
Logistic regression is widely used for classification tasks, and Poisson regression is appropriate when the response is a count.
\if0\informsMOR{In ``Additional Simulation'' of Supplementary Material, we }\else{We }\fi
also consider the performance of ABESS algorithm on multi-response linear regression (a.k.a., multi-task learning).
Before formally analyzing the simulation results,
we illustrate our simulation settings in Section~\ref{subsec:setup}.
% This subsection develops parallel with Section \ref{subsec:logistic}.

%In this section, we study the empirical performance of ABESS for GLM on two generalized linear models,
%logistic regression and gamma regression,
%where logistic regression is widely used for classification and
%gamma regression model is useful for modeling positive continuous response variables.
%Before formally studying logistic regression and gamma regression in Section~\ref{subsec:logistic} and Section~\ref{subsec:gamma}, respectively, we illustrate our simulation setting in Section~\ref{subsec:setup}.

\subsection{Setup}\label{subsec:setup}
To synthesize a dataset, we generate multivariate Gaussian realizations $\boldsymbol{x}_1, \ldots, \boldsymbol{x}_n \overset{i.i.d.}{\sim} \mathcal{MVN}(0,\Sigma)$,
where $\Sigma$ is a $p$-by-$p$ covariance matrix.
%We generate i.i.d error $\epsilon\sim N(0,\sigma^2)$.
%Define the signal to noise ratio (SNR) by $SNR = \frac{\beta^{\top}\Sigma\beta}{\sigma^2}$.
We consider two covariance structures for $\Sigma$: the independent structure ($\Sigma$ is an identity matrix)
and the constant structure ($\Sigma_{ij} = \rho^{I(i\neq j)}$ for some positive constant $\rho$). The value of $\rho$ and $p$ will be specified later.
We set the true regression coefficient $\boldsymbol{\beta}^*$ as a sparse vector with $k$ non-zero entries that have equi-spaced indices in $\{1, \ldots, p\}$.
Finally, given a design matrix $\mathbf{X} = (\boldsymbol{x}_1, \ldots, \boldsymbol{x}_n)^\top$ and $\boldsymbol{\beta}^*$,
we draw response realizations $\{y_i\}_{i=1}^n$ according to the GLMs.

We assess our proposal via the following criteria.
First, to measure the performance of subset selection,
we consider the probabilities of covering true active and inactive sets: $\mathbb{P}(\mathcal{A}^* \subseteq \hat{\mathcal{A}})$ and
$\mathbb{P}(\mathcal{I}^* \subseteq \hat{\mathcal{I}})$ (here, $\mathcal{I}^* = (\mathcal{A}^*)^c$).
We also consider exact support recover probability as $\mathbb{P}(\mathcal{A}^* = \hat{\mathcal{A}})$.
Since the probability is unknown, we empirically examine the proportion of recovery for the active set, inactive set, and exact recovery in 200 replications for instead.
As for parameter estimation performance, we examine relative error (ReErr) on parameter estimations:
$\|\hat{\boldsymbol{\beta}}-\boldsymbol{\beta}^*\|_{2} /\|\boldsymbol{\beta}^*\|_{2}$.
Finally, computational efficiency is directly measured by the runtime.

In addition to our proposed algorithms, we compare classical variable selection methods: LASSO \citep{tibshirani1996regression}, SCAD \citep{fan2001variable}, and MCP \citep{zhang2010nearly}.
%, and a recently proposed coordinate descent (CD) method for $\ell_0$-regularized classification \citep{antoine2021l0learn}.
For all these methods, we apply 10-fold cross-validation (CV) and the GIC to select the tuning parameter, respectively.
% For all these methods, we apply 10-fold cross-validation (CV) to select the tuning parameter.
% ABESS also uses generalized information criterion (GIC) \citep{fan2013tuning} because,
% by combining GIC, ABESS can consistently recover $\mathcal{A}^*$ under linear models \citep{zhu2020polynomial}.
The software for these methods is available at R CRAN (\url{https://cran.r-project.org}).
The software of all methods is summarized in Table~\ref{tab:implementation-details}.
All experiments are carried out on an R environment in a Linux platform with Intel(R) Xeon(R) Gold 6248 CPU @ 2.50GHz. 
% Note that, all experiments result are based on 200 random synthetic datasets.
%Ubuntu platform with Intel(R) Xeon(R) Gold 6248 CPU @ 2.50GHz.

% Model selection methods such as cross-validation and information criteria are widely used.
% Recently, \citet{fan2013tuning} explored generalized information criterion (GIC) in tuning parameter selection for
% penalized likelihood methods under GLM.
% Here, we use a GIC-type information criterion to recovery support size, which is defined as:
% $\mathrm{F}(\hat{\boldsymbol \beta}) = l_n( \hat{\boldsymbol \beta} ) + |\text{supp}(\hat{\boldsymbol \beta})| \log(p) \log\log n.$
% Intuitively speaking, the model complexity penalty term $|\text{supp}(\hat{\boldsymbol \beta})| \log p \log\log n$ is set to prevent over-fitting,
% where the term $\log\log n$ with a slow diverging rate is used to prevent under-fitting.
% Combining the Algorithm~\ref{alg:fbess} with GIC, we select the support size that minimizes the $F(\hat{\boldsymbol{\beta}})$.}

% \begin{table}[htbp]
% \caption{Implementation details for all methods.
% The values in the parentheses indicate the version number of R packages.}\label{tab:implementation-details}
% \centering
% \begin{tabular}{ccc}
% \toprule
% Method & Software & Tuning method \\
% \midrule
% ABESS-GIC & abess (0.4.0) & GIC \\
% LASSO-GIC & glmnet (4.1-3) & GIC \\
% SCAD-GIC & ncvreg (3.13.0) & GIC \\
% MCP-GIC & ncvreg (3.13.0) & GIC \\
% CD-GIC & L0Learn (2.0.3) & GIC \\
% ABESS-CV & abess (0.4.0) & 10-folds CV \\
% LASSO-CV & glmnet (4.1-3) & 10-folds CV \\
% SCAD-CV & ncvreg (3.13.0) & 10-folds CV \\
% MCP-CV & ncvreg (3.13.0) & 10-folds CV \\
% CD-CV & L0Learn (2.0.3) & 10-folds CV \\
% \bottomrule
% \end{tabular}
% \end{table}
\begin{table}[htbp]
\caption{Software for all methods.
The values in the parentheses indicate the version number of R packages.The tuning parameter within the MCP/SCAD penalty is fixed at 3/3.7.}\label{tab:implementation-details}
\centering
\if0\informsMOR{
% \begin{tabular}{ccccccc}
% \toprule
% Method & ABESS & LASSO & SCAD & MCP & CD \\
% \midrule
% Software & \textsf{abess} (0.4.0) & \textsf{glmnet} (4.1-3) & \textsf{ncvreg} (3.13.0) & \textsf{ncvreg} (3.13.0) & \textsf{L0Learn} (2.0.3) \\
% Tuning & sparsity $s$ & $\ell_1$ penalty & $\lambda$ & $\lambda$& $\lambda$ \\
% \bottomrule
% \end{tabular}
\begin{tabular}{cccccc}
    \toprule
    Method & ABESS & LASSO & SCAD & MCP \\
    \midrule
    Software & \textsf{abess} (0.4.0) & \textsf{glmnet} (4.1-3) & \textsf{ncvreg} (3.13.0) & \textsf{ncvreg} (3.13.0) \\
    Tuning & sparsity $s$ & $\ell_1$ penalty & $\lambda$ & $\lambda$ \\
    \bottomrule
    \end{tabular}
}\else{
% \begin{tabular}{ccccccc}
% \hline
% Method & ABESS & LASSO & SCAD & MCP & CD \\
% \hline
% Software & \textsf{abess} (0.4.0) & \textsf{glmnet} (4.1-3) & \textsf{ncvreg} (3.13.0) & \textsf{ncvreg} (3.13.0) & \textsf{L0Learn} (2.0.3) \\
% Tuning & sparsity $s$ & $\ell_1$ penalty & {\color{red}SCAD penalty} & {\color{red}MCP penalty} & $\ell_0$ penalty \\
% \hline
% \end{tabular}
\begin{tabular}{cccccc}
\hline
Method & ABESS & LASSO & SCAD & MCP \\
\hline
Software & \textsf{abess} (0.4.0) & \textsf{glmnet} (4.1-3) & \textsf{ncvreg} (3.13.0) & \textsf{ncvreg} (3.13.0) \\
Tuning & sparsity $s$ & $\ell_1$ penalty & {\color{red}SCAD penalty} & {\color{red}MCP penalty} \\
\hline
\end{tabular}
}\fi
\end{table}
% We implement our proposal in an R package abess \citep{zhu-abess-arxiv}.

\subsection{Logistic Regression}\label{subsec:logistic}
% In this subsection, we illustrate the power of ABESS on logistic regression, which is one of the most popular GLMs widely used for classification tasks.
% In terms of logistic regression, the response $y_i$ is a binary variable following a Bernoulli distribution $B(1, p_i)$,
% where $p_i \coloneqq \mathbb{P}(y_i=1)$ is determined by $\log(\frac{p_i}{1-p_i}) = \boldsymbol x_i^\top \boldsymbol{\beta }$.
% Here, the link function is known as the logit function, defined by $logit(p) = \log(\frac{p}{1-p})$.
% As a result, the negative log-likelihood is given by
% \begin{equation*}
% l_n(\boldsymbol\beta) = -\sum_{i=1}^{N}\left\{y_{i} \boldsymbol {x}_i^\top \boldsymbol \beta-\log \left(1+e^{\boldsymbol {x}_i^\top \boldsymbol \beta}\right)\right\}.
% \end{equation*}
% Empirically, we generate $x_i$ and $\beta$ as described in Section \ref{subsec:setup}.
% Binary response $y_i$ is then drawn from the Bernoulli distribution according to (\ref{eqn:formula_binomial}).
% Let $H_j = \sum\limits_{i=1}^{n} \frac{e^{\boldsymbol {x}_i^\top \hat{\boldsymbol \beta}}}{(1 + e^{\boldsymbol {x}_i^\top \hat{\boldsymbol \beta}})^2} x_{ij}^2$ and
% the gradient of $l_n(\boldsymbol{\beta})$ at $\hat{\boldsymbol{\beta}}$ be $\hat{\boldsymbol d} = -\sum\limits_{i=1}^{n}(y_i - \frac{e^{\boldsymbol {x}_i^\top \hat{\boldsymbol \beta}}}{1 + e^{\boldsymbol {x}_i^\top \hat{\boldsymbol \beta}}}) \boldsymbol {x}_i$,
% \eqref{eqn:approx_sacrifice} can be explicit expressed as:
% $\xi_j = H_j (\hat{\boldsymbol{\beta}}_j)^2$ for $j\in \mathcal{A}$ and
% $\zeta_j = H_j^{-1}( \hat{\boldsymbol d}_j )^2$ for $j\in \mathcal{I}$.
% \begin{equation*}
% \begin{aligned}
% % \hat{\boldsymbol d} &= -\sum_{i=1}^{n}(y_i - \frac{e^{\boldsymbol {x}_i^\top \boldsymbol \beta}}{1 + e^{\boldsymbol {x}_i^\top \boldsymbol \beta}}) \boldsymbol {x}_i. \\
% \xi_j
% & = H_j (\hat{\boldsymbol{\beta}}_j)^2, j\in \mathcal{A},\\
% \zeta_j
% & = H_j^{-1}
% ( \hat{\boldsymbol d}_j )^2, j\in \mathcal{I}.
% \end{aligned}
% \end{equation*}
% Given the explicit expression of \eqref{eqn:approx_sacrifice},
% we can conduct Algorithm~\ref{alg:abess} to estimate $\boldsymbol{\beta}$.

The dimension $p$ is fixed as 500 for the logistic regression model. For the constant correlation case, we set $\rho = 0.4$.
The non-zero coefficients $\boldsymbol{\beta}^*_{\mathcal{A}^*}$ are set to be $(2,2,8,8,8,8,10,10,10,10)^\top$. 
Now we compare methods listed in Table~\ref{tab:implementation-details}.
Figures~\ref{fig:rate_binomial} and \ref{fig:ReErr_binomial} present the results on subset selection and parameter estimation when the sample size increases. Out of clarity, we omit the CV results here and defer these results to the Additional Figures in Supplementary Material.


\begin{figure}[htbp]
\centering
\includegraphics[width=1.0\textwidth]{figure/rate_binomial_gic.pdf}
\if0\informsMOR{
\vspace{-30pt}
}\fi
\caption{Performance on subset selection under logistic regression when covariates have independent correlation structure (Upper) and constant correlation structure (Lower), measured by three kinds probabilities: $\mathbb{P}(\mathcal{A}^* \subseteq \hat{\mathcal{A}})$, $\mathbb{P}(\mathcal{I}^* \subseteq \hat{\mathcal{I}})$, and $\mathbb{P}(\mathcal{A}^* = \hat{\mathcal{A}})$ that are presented in Left, Middle and Right panels, respectively.
}
\label{fig:rate_binomial}
\end{figure}
\begin{figure}[htbp]
\centering
\includegraphics[width=0.8\textwidth]{figure/ReErr_binomial_gic.pdf}
\if0\informsMOR{
\vspace{-10pt}
}\fi
\caption{Performance on parameter estimation under logistic regression models when covariance matrices have independent correlation structure (Left) and exponential correlation structure (Right). The $y$-axis is the median of ReErr in a log scale.}
\label{fig:ReErr_binomial}
\end{figure}

As depicted in the left panel of Figure~\ref{fig:rate_binomial}, the probability $\mathbb{P}(\mathcal{A}^* \subseteq \hat{\mathcal{A}})$ approaches 1 as the sample size increases, indicating that all methods, except LASSO in the high correlation setting, can provide a no-false-exclusion estimator when the sample size is sufficiently large. However, when considering $\mathbb{P}(\mathcal{I}^* \subseteq \hat{\mathcal{I}})$, as observed in the middle panel of Figure~\ref{fig:rate_binomial}, the LASSO estimator consistently exhibits false inclusions, and the SCAD/MCP estimator shows false inclusions when the covariates are highly correlated. In contrast, only ABESS guarantees that $\mathbb{P}(\mathcal{I}^* \subseteq \hat{\mathcal{I}})$ approaches 1 for large sample sizes. 

Furthermore, as evident from the right panel of Figure~\ref{fig:rate_binomial}, ABESS accurately recovers the true subset under both correlation settings. While SCAD and MCP can also achieve exact support recovery given a sufficient sample size, ABESS demonstrates support recovery consistency with the smallest sample size, particularly when variables are correlated. It is important to note that although our theory imposes restrictions on the correlation among a small subset of variables (see Assumption~\ref{con:technical-assumption}), our algorithm still performs effectively in the constant correlation setting. This setting (i.e., $\rho=0.4$) violates Assumption~\ref{con:technical-assumption} as the correlation between any two variables exceeds 0.183, which is the maximum acceptable pairwise correlation satisfying Assumption~\ref{con:technical-assumption}.

Moving on to Figure~\ref{fig:ReErr_binomial}, it illustrates the superiority of ABESS in parameter estimation. ABESS visibly outperforms other methods in the small sample size regime and maintains highly competitive performance as the sample size increases. This superiority in parameter estimation is not surprising, as ABESS yields an oracle estimator when the support set is correctly identified. Although SCAD and MCP do not provide algorithmic guarantees for finding the local minimum, they exhibit competitive parameter estimation performance due to their asymptotic unbiasedness. Conversely, the LASSO estimator is biased and performs the worst among all the methods.

%\begin{figure}
%	\centering
%	\includegraphics[width=\textwidth]{figure/Performance_binomial.pdf}
%	\caption{Performance comparison under two correlation structures: independent and exponential. (A) Performance for subset selection, measured by support recover probability. (B) Performance for parameter estimation, measured by median ReErr. (C) Average runtime, measured in seconds. L0Learn is omitted since its runtime is far longer than others.}
%	\label{fig:Performance_binomial}
%\end{figure}

\subsection{Poisson Regression}\label{seubsec:poisson}
% As regard to Poisson regression, the response $y_i$ is a integer variable following a Poisson distribution $\mathcal{P}(\lambda_i)$ where $\lambda_i = \exp(\boldsymbol x_i^\top \boldsymbol{\beta})$.
% As a result, the negative log-likelihood is given by
% \begin{equation*}
% l_n(\boldsymbol\beta) = -\sum_{i=1}^{N}\left\{y_{i} \boldsymbol {x}_i^\top \boldsymbol\beta - e^{\boldsymbol {x}_i^\top \boldsymbol \beta} -\log(y_i!)\right\}.
% \end{equation*}
% Empirically, we generate $x_i$ and $\beta$ as described in Section \ref{subsec:setup}.
% Binary response $y_i$ is then drawn from the Bernoulli distribution according to (\ref{eqn:formula_binomial}).
% Let $H_j = \sum\limits_{i=1}^{n} \exp(\boldsymbol {x}_i^\top \hat{\boldsymbol \beta}) x_{ij}^2$ and
% the gradient of $l_n(\boldsymbol{\beta})$ at $\hat{\boldsymbol{\beta}}$ be $\hat{\boldsymbol d} = -\sum\limits_{i=1}^{n}(y_i - \exp(\boldsymbol {x}_i^\top \hat{\boldsymbol \beta})) \boldsymbol {x}_i$,
% \eqref{eqn:approx_sacrifice} can be explicit expressed as:
% $\xi_j = H_j (\hat{\boldsymbol{\beta}}_j)^2$ for $j\in \mathcal{A}$ and
% $\zeta_j = H_j^{-1}( \hat{\boldsymbol d}_j )^2$ for $j\in \mathcal{I}$.
% Given the explicit expression of \eqref{eqn:approx_sacrifice},
% we can conduct Algorithm~\ref{alg:abess} to estimate $\boldsymbol{\beta}$.


For the Poisson regression model, we consider a fixed $p$ value of 500, and set $\rho = 0.2$ for the constant correlation case. The non-zero coefficients $\boldsymbol{\beta}^*_{\mathcal{A}^*}$ are specified as $(1, 1, 1)^\top$. Figures~\ref{fig:rate_poisson_gic}-\ref{fig:ReErr_poisson_gic} present the evaluation of subset selection and parameter estimation quality. Examining Figures~\ref{fig:rate_poisson_gic}, we observe that for ABESS/SCAD/MCP, the probabilities $\mathbb{P}(\mathcal{A}^* \subseteq \hat{\mathcal{A}})$, $\mathbb{P}(\mathcal{I}^* \subseteq \hat{\mathcal{I}})$, and $\mathbb{P}(\mathcal{A}^* = \hat{\mathcal{A}})$ gradually approach 1 as the sample size $n$ increases. In contrary, the LASSO, regardless of the highest inclusion probability for $\mathcal{A}^*$, still has a chance of including ineffective variables, especially when variables are correlated. Comparing ABESS, SCAD, and MCP, it is evident that ABESS achieves the highest exact selection probability, followed by SCAD and MCP. Similar to the results in logistic regression, ABESS achieves exact selection of the effective variables with the smallest sample size under the constant correlation structure.
Regarding the quality of parameter estimation, the ReErr of all methods reasonably decreases as the sample size $n$ increases. Again, ABESS exhibits the least estimation error in terms of the $\ell_2$-norm, which coincides with the results on logistic regression. It is worth noting that our method demonstrates consistency and polynomial complexity under Poisson regression, even though it violates the sub-Gaussian assumption. This is because the current framework of proofs allows for the relaxation of Assumption~\ref{con:subgaussian} to a sub-exponential distribution assumption, enabling the establishment of similar theoretical properties.

\begin{figure}[htbp]
\centering
\includegraphics[width=1.0\textwidth]{figure/rate_poisson_gic.pdf}
\if0\informsMOR{
\vspace{-30pt}
}\fi
\caption{Performance on subset selection under Poisson regression when covariates have independent correlation structure (Upper) and constant correlation structure (Lower), measured by three kinds probabilities: $\mathbb{P}(\mathcal{A}^* \subseteq \hat{\mathcal{A}})$, $\mathbb{P}(\mathcal{I}^* \subseteq \hat{\mathcal{I}})$, and $\mathbb{P}(\mathcal{A}^* = \hat{\mathcal{A}})$ that are presented in Left, Middle and Right panels, respectively.}
\label{fig:rate_poisson_gic}
\end{figure}
\begin{figure}[htbp]
\centering
\includegraphics[width=0.8\textwidth]{figure/ReErr_poisson_gic.pdf}
\if0\informsMOR{
\vspace{-5pt}
}\fi
\caption{Performance on parameter estimation under Poisson regression models when covariance matrices have independent correlation structure (Left) and exponential correlation structure (Right). The $y$-axis is the median of ReErr in a log scale.}
\label{fig:ReErr_poisson_gic}
\end{figure}

\subsection{Computational analysis}

We compare the runtime of different methods in Table~\ref{tab:implementation-details} for the logistic regression and Poisson regression models in Sections~\ref{subsec:logistic} to \ref{seubsec:poisson}. The runtime results are summarized in Figure~\ref{fig:simu_runtime}, indicating that ABESS demonstrates superior computational efficiency compared to state-of-the-art variable selection methods. For instance, when $n = 3000$, ABESS is at least four times faster than its competitors in logistic regression under an independent correlation structure. Furthermore, regardless of logistic regression or Poisson regression, ABESS exhibits similar computational performance, while other competitors run much faster when the pairwise correlation is higher. Lastly, it is important to note that the runtime of ABESS scales polynomially with sample sizes, aligning with the complexity presented in Theorem~\ref{thm:complexity}.
%In contrast, the runtime of other methods grows more rapidly as the sample size increases
%and appears like a quadratic function of the sample size in the independent scenario.
%Increasing iteration numbers for convergence may lead to this result.
%Moreover, ABESS-GIC is faster than ABESS-CV, demonstrating the superiority of the proposed adaptive parameter tuning procedure.
% Finally, according to the computational comparison presented in Figure~\ref{fig runtime_poisson_gic}, the ABESS has the least runtime and is much faster than the MCP and SCAD when variables are independent.

\begin{figure}[htbp]
\centering
\includegraphics[width=0.8\textwidth]{figure/runtime_binomial_gic.pdf}
\includegraphics[width=0.8\textwidth]{figure/runtime_poisson_gic.pdf}
\if0\informsMOR{
\vspace{-10pt}
}\fi
\caption{Average runtime (measured in seconds) on logistic regression (Upper panel) and Poisson regression (Lower panel). The results on two types of covariances matrix $\Sigma$, the independent correlation structure and constant correlation structure, are presented in the left and right panels, respectively. The error bars represent two times the standard errors.
}
\label{fig:simu_runtime}
\end{figure}

%% For the MOR template, uncomment this line and comment on the code blocks

\if1\informsMOR
{
\input{../appendix_numerical}
}\fi

%%%%%%%%%%%%%%%%%%%%%%%%%%
% \vspace{-0.5em}
\section{Conclusion}
% \vspace{-0.5em}
Recent advances in multimodal single-cell technology have enabled the simultaneous profiling of the transcriptome alongside other cellular modalities, leading to an increase in the availability of multimodal single-cell data. In this paper, we present \method{}, a multimodal transformer model for single-cell surface protein abundance from gene expression measurements. We combined the data with prior biological interaction knowledge from the STRING database into a richly connected heterogeneous graph and leveraged the transformer architectures to learn an accurate mapping between gene expression and surface protein abundance. Remarkably, \method{} achieves superior and more stable performance than other baselines on both 2021 and 2022 NeurIPS single-cell datasets.

\noindent\textbf{Future Work.}
% Our work is an extension of the model we implemented in the NeurIPS 2022 competition. 
Our framework of multimodal transformers with the cross-modality heterogeneous graph goes far beyond the specific downstream task of modality prediction, and there are lots of potentials to be further explored. Our graph contains three types of nodes. While the cell embeddings are used for predictions, the remaining protein embeddings and gene embeddings may be further interpreted for other tasks. The similarities between proteins may show data-specific protein-protein relationships, while the attention matrix of the gene transformer may help to identify marker genes of each cell type. Additionally, we may achieve gene interaction prediction using the attention mechanism.
% under adequate regulations. 
% We expect \method{} to be capable of much more than just modality prediction. Note that currently, we fuse information from different transformers with message-passing GNNs. 
To extend more on transformers, a potential next step is implementing cross-attention cross-modalities. Ideally, all three types of nodes, namely genes, proteins, and cells, would be jointly modeled using a large transformer that includes specific regulations for each modality. 

% insight of protein and gene embedding (diff task)

% all in one transformer

% \noindent\textbf{Limitations and future work}
% Despite the noticeable performance improvement by utilizing transformers with the cross-modality heterogeneous graph, there are still bottlenecks in the current settings. To begin with, we noticed that the performance variations of all methods are consistently higher in the ``CITE'' dataset compared to the ``GEX2ADT'' dataset. We hypothesized that the increased variability in ``CITE'' was due to both less number of training samples (43k vs. 66k cells) and a significantly more number of testing samples used (28k vs. 1k cells). One straightforward solution to alleviate the high variation issue is to include more training samples, which is not always possible given the training data availability. Nevertheless, publicly available single-cell datasets have been accumulated over the past decades and are still being collected on an ever-increasing scale. Taking advantage of these large-scale atlases is the key to a more stable and well-performing model, as some of the intra-cell variations could be common across different datasets. For example, reference-based methods are commonly used to identify the cell identity of a single cell, or cell-type compositions of a mixture of cells. (other examples for pretrained, e.g., scbert)


%\noindent\textbf{Future work.}
% Our work is an extension of the model we implemented in the NeurIPS 2022 competition. Now our framework of multimodal transformers with the cross-modality heterogeneous graph goes far beyond the specific downstream task of modality prediction, and there are lots of potentials to be further explored. Our graph contains three types of nodes. while the cell embeddings are used for predictions, the remaining protein embeddings and gene embeddings may be further interpreted for other tasks. The similarities between proteins may show data-specific protein-protein relationships, while the attention matrix of the gene transformer may help to identify marker genes of each cell type. Additionally, we may achieve gene interaction prediction using the attention mechanism under adequate regulations. We expect \method{} to be capable of much more than just modality prediction. Note that currently, we fuse information from different transformers with message-passing GNNs. To extend more on transformers, a potential next step is implementing cross-attention cross-modalities. Ideally, all three types of nodes, namely genes, proteins, and cells, would be jointly modeled using a large transformer that includes specific regulations for each modality. The self-attention within each modality would reconstruct the prior interaction network, while the cross-attention between modalities would be supervised by the data observations. Then, The attention matrix will provide insights into all the internal interactions and cross-relationships. With the linearized transformer, this idea would be both practical and versatile.

% \begin{acks}
% This research is supported by the National Science Foundation (NSF) and Johnson \& Johnson.
% \end{acks}

\appendices

%\section{Proof of Lemma \ref{lem1}}
\label{sec:app_c}
Here, we provide a formal proof for Lemma \ref{lem1} which we restate as follows. The following lemma confirms that the number of elements in $J^{(n)}$ goes to infinity as $n$ becomes large.

	If $N^{(n)} \triangleq |J^{(n)}| $, then $N^{(n)} \rightarrow \infty$ with high probability as $n \rightarrow \infty$.  More specifically, there exists $\lambda>0$ such that
\[
	P\left(N^{(n)} > \frac{\lambda}{2}n^{\frac{\beta}{2}}\right) \rightarrow 1.
	\]

\begin{proof}
Define the events $A$, $B$ as
\[A \equiv  p_1\leq P_u\leq p_1+\epsilon_n\]
\[B \equiv p_1+\epsilon_n\leq Q_u\leq p_1+(1-2p_1)a_n.\]
Then, for $u \in \{1, 2, \dots, n\}$ and $0\leq p_1<\frac{1}{2}$:
	\begin{align}
	\no P\left(u\in J^{(n)}\right) &= P\left(A\ \cap \ B\right)\\
	\nonumber &= P\left(A\right) P\left(B \big{|}A \right). \ \
	\end{align}		
So, given $p_1 \in (0,1)$ and the assumption $0<\delta_1<f_p<\delta_2$, for $n$ large enough, we have
	\[
	P(A)  = \int_{p_1}^{ p_1+\epsilon_n}f_P(p) dp,
	\]
so, we can conclude that
	\[
	\epsilon_n\delta_1<P(A) <\epsilon_n\delta_2.
	\]
	We can find a $\delta$ such that $\delta_1<\delta<\delta_2$ and
\begin{equation}\label{eq:5}
	P( A) = \epsilon_n\delta.
\end{equation}
We know
	\[Q_u\bigg{|}P_u=p_u \sim Uniform \left[p_u,p_u+(1-2p_u)a_n\right],\]
	so, according to Figure \ref{fig:piqi_b}, for $p_1\leq p_u\leq p_1+\epsilon_n$,
	\begin{align}
	\no P\left(B | P_u=p_u\right ) &= \frac{p_1+(1-2p_1)a_n-p_1-\epsilon_n}{p_u+(1-2p_u)a_n-p_u} \\
	\nonumber &= \frac{(1-2p_1)a_n-\epsilon_n}{(1-2p_u)a_n}\\
	\nonumber &\geq \frac{(1-2p_1)a_n-\epsilon_n}{(1-2p_1)a_n} \\
	\nonumber &= 1- \frac{\epsilon_n}{(1-2p_1)a_n}, \ \
	\end{align}
which implies
\begin{align}
P\left(B | A\right ) \geq 1- \frac{\epsilon_n}{(1-2p_1)a_n}. \label{eq:6}
\end{align}
Using (\ref{eq:5}) and (\ref{eq:6}), we can conclude
	\[P\left(u\in J^{(n)}\right)\geq \epsilon_n\delta \left(1- \frac{\epsilon_n}{(1-2p_1)a_n}\right).\]
Then, we can say that $N^{(n)}$ has a binomial distribution with expected value of $N^{(n)}$ greater than $n\epsilon_n\delta \left(1- \frac{\epsilon_n}{(1-2p_1)a_n}\right)$, and by substituting $\epsilon_n$ and $a_n$, for any $c'>0$, we get
\[E\left[N^{(n)}\right] \geq \delta\left(n^{\frac{\beta}{2}}- \frac{1}{{c'(1-2p_1)}}\right)   \geq  \lambda n^{\frac{\beta}{2}}.\]

Now by using Chernoff bound, we have
\[P\left(N^{(n)} \leq (1- \theta) E\left[N^{(n)}\right]\right) \leq e^{-\frac{\theta^2}{2}E\left[N^{(n)}\right]},\]
so, if we assume $\theta=\frac{1}{2}$, we can conclude for large enough $n$,
\begin{align}
\nonumber P\left(N^{(n)} \leq \frac{\lambda}{2}n^{\frac{\beta}{2}}\right) &\leq P\left(N^{(n)} \leq \frac{E\left[N^{(n)}\right]}{2}\right)\\
\nonumber &\leq e^{-\frac{E[N^{(n)}]}{8}}\\
\nonumber &\leq e^{-\frac{\lambda n^{\frac{\beta}{2}}}{8}} \rightarrow 0.
\end{align}
As a result, $N^{(n)} \rightarrow \infty$ with high probability for large enough $n.$

\end{proof}

%\section{Proof of Lemma \ref{lemOnePointFive}}
\label{sec:app_b}
Here we provide a formal proof for Lemma \ref{lemOnePointFive} which we restate as follows.

Let $N$ be a positive integer, and let $a_1, a_2, \cdots, a_N$ and $b_1, b_2, \cdots, b_N$ be real numbers such that $a_u \leq b_u$ for all $u$. Assume that $X_1, X_2, \cdots, X_N$ are $N$ independent random variables such that
\[X_u \sim Uniform[a_u,b_u]. \]
Let also $\gamma_1, \gamma_2, \cdots, \gamma_N$ be real numbers such that
\[ \gamma_j \in \bigcap_{u=1}^{N} [a_u, b_u] \ \ \textrm{for all }j \in \{1,2,\cdots,N\}. \]
Suppose that we know the event $E$ has occurred, meaning that the observed values of $X_u$'s is equal to the set of $\gamma_j$'s (but with unknown ordering), i.e.,
\[E \ \ \equiv \ \ \{X_1,X_2,\cdots,X_N\}= \{ \gamma_1, \gamma_2, \cdots, \gamma_N \}, \] then
\[P\left(X_1=\gamma_j |E\right)=\frac{1}{N}. \]

\begin{proof}
Define sets $\mathfrak{P}$ and $\mathfrak{P}_j$ as follows:
\[\mathfrak{P}= \textrm{The set of all permutations $\Pi$ on }\{1,2,\cdots,N\}. \]
\[\mathfrak{P}_j= \textrm{The set of all permutations $\Pi$ on }\{1,2,\cdots,N\} \textrm{ such that } \Pi(1)=j. \]
We have $|\mathfrak{P}|=N!$ and $|\mathfrak{P}|=(N-1)!$. Then
\begin{align*}
 P(X_1=\alpha_j |E)&=\frac{\sum_{\pi \in \mathfrak{P}_j} f_{X_1,X_2, \cdots,X_N} (\gamma_{\pi(1)}, \gamma_{\pi(2)}, \cdots, \gamma_{\pi(N)})} {\sum_{\pi \in \mathfrak{P}} f_{X_1,X_2, \cdots,X_N} (\gamma_{\pi(1)}, \gamma_{\pi(2)}, \cdots, \gamma_{\pi(N)})}\\
 &=\frac{(N-1)! \prod\limits_{u=1}^{N} \frac{1}{b_u-a_u}}{N! \prod\limits_{u=1}^{N} \frac{1}{b_u-a_u}}\\
 &=\frac{1}{N}.
\end{align*}
\end{proof}
\section{Lemma \ref{lemx} and its Proof}
\label{sec:app_a}
Here we state that we can condition on high-probability events.

\begin{lem}
	\label{lemx}
	Let $p \in (0,1)$, and $X \sim Bernoulli (p)$ be defined on a probability space $(\Omega, \mathcal{F}, P)$. Consider $B_1, B_2, \cdots $ be a sequence of events defined on the same probability space such that $P(B_n) \rightarrow 1$ as $n$ goes to infinity. Also, let $\textbf{Y}$ be a random vector (matrix) in the same probability space, then:
	\[I(X; \textbf{Y}) \rightarrow 0\ \ \text{iff}\ \  I(X; \textbf{Y} {|} B_n) \rightarrow 0. \]
\end{lem}

\begin{proof}
First, we prove that as n becomes large,
\begin{align}\label{eq:H1}
H(X {|} B_n)- H(X) \rightarrow 0.
\end{align}

Note that as $n$ goes to infinity,
\begin{align}
\no P\left(X=1\right) &=P\left(X=1 \bigg{|} B_n\right) P\left(B_n\right) + P\left(X=1 \bigg{|} \overline{B_n}\right) P\left(\overline{B_n}\right)\\
%\no   &=P\left(X=1 \bigg{|} B_n\right)\left(1-o(1)\right) + P\left(X=1 \bigg{|} \overline{B_n}\right)o(1)\\
%\no   & \rightarrow  P\left(X=1 \bigg{|} B_n\right)+ o(1)  \ \ (\textrm{as } n \rightarrow \infty) \\
\no &=P\left(X=1 \bigg{|} B_n\right),\ \
\end{align}
thus,
$\left(X \bigg{|} B_n\right) \xrightarrow{d} X$, and as $n$ goes to infinity,
 \[H\left(X {|} B_n\right)- H(X) \rightarrow 0.\]
Similarly, as $n$ becomes large,
\[ P\left(X=1 \bigg{|} \textbf{Y}=\textbf{y}\right) \rightarrow P\left(X=1 \bigg{|} \textbf{Y}=\textbf{y}, B_n\right),\ \
\]
and
\begin{align}\label{eq:H2}
H\left(X {|}  \textbf{Y}=\textbf{y},B_n\right)- H\left(X {|} \textbf{Y}=\textbf{y}\right) \rightarrow 0.
\end{align}
%Thus $(X {|}  Y_n=y,B_n) \xrightarrow{D} (X {|} Y_n=y)$, and so $H(X {|}  Y_n=y,B_n)- H(X {|} Y_n=y) \rightarrow 0 $ as $n \rightarrow \infty$.
Remembering that
\begin{align}\label{eq:H3}
I\left(X; \textbf{Y}\right)=H(X)-H(X {|} \textbf{Y}),
\end{align}
and using (\ref{eq:H1}), (\ref{eq:H2}), and (\ref{eq:H3}), we can conclude that as  $n$ goes to infinity,
\[I\left(X;\textbf{Y} {|} B_n\right) - I\left(X,\textbf{Y}\right) \rightarrow 0.\]
As a result, for large enough $n$,
\[I\left(X; \textbf{Y}\right) \rightarrow 0 \Longleftrightarrow I\left(X; \textbf{Y} {|} B_n\right) \rightarrow 0. \]
\end{proof}

\section{Proof of Lemma \ref{lemOnePointFive}}
\label{sec:app_b}
Here we provide a formal proof for Lemma \ref{lemOnePointFive} which we restate as follows.

Let $N$ be a positive integer, and let $a_1, a_2, \cdots, a_N$ and $b_1, b_2, \cdots, b_N$ be real numbers such that $a_u \leq b_u$ for all $u$. Assume that $X_1, X_2, \cdots, X_N$ are $N$ independent random variables such that
\[X_u \sim Uniform[a_u,b_u]. \]
Let also $\gamma_1, \gamma_2, \cdots, \gamma_N$ be real numbers such that
\[ \gamma_j \in \bigcap_{u=1}^{N} [a_u, b_u] \ \ \textrm{for all }j \in \{1,2,\cdots,N\}. \]
Suppose that we know the event $E$ has occurred, meaning that the observed values of $X_u$'s is equal to the set of $\gamma_j$'s (but with unknown ordering), i.e.,
\[E \ \ \equiv \ \ \{X_1,X_2,\cdots,X_N\}= \{ \gamma_1, \gamma_2, \cdots, \gamma_N \}, \] then
\[P\left(X_1=\gamma_j |E\right)=\frac{1}{N}. \]

\begin{proof}
Define sets $\mathfrak{P}$ and $\mathfrak{P}_j$ as follows:
\[\mathfrak{P}= \textrm{The set of all permutations $\Pi$ on }\{1,2,\cdots,N\}. \]
\[\mathfrak{P}_j= \textrm{The set of all permutations $\Pi$ on }\{1,2,\cdots,N\} \textrm{ such that } \Pi(1)=j. \]
We have $|\mathfrak{P}|=N!$ and $|\mathfrak{P}|=(N-1)!$. Then
\begin{align*}
 P(X_1=\alpha_j |E)&=\frac{\sum_{\pi \in \mathfrak{P}_j} f_{X_1,X_2, \cdots,X_N} (\gamma_{\pi(1)}, \gamma_{\pi(2)}, \cdots, \gamma_{\pi(N)})} {\sum_{\pi \in \mathfrak{P}} f_{X_1,X_2, \cdots,X_N} (\gamma_{\pi(1)}, \gamma_{\pi(2)}, \cdots, \gamma_{\pi(N)})}\\
 &=\frac{(N-1)! \prod\limits_{u=1}^{N} \frac{1}{b_u-a_u}}{N! \prod\limits_{u=1}^{N} \frac{1}{b_u-a_u}}\\
 &=\frac{1}{N}.
\end{align*}
\end{proof}
\section{Proof of Lemma \ref{lem1}}
\label{sec:app_c}
Here, we provide a formal proof for Lemma \ref{lem1} which we restate as follows. The following lemma confirms that the number of elements in $J^{(n)}$ goes to infinity as $n$ becomes large.

	If $N^{(n)} \triangleq |J^{(n)}| $, then $N^{(n)} \rightarrow \infty$ with high probability as $n \rightarrow \infty$.  More specifically, there exists $\lambda>0$ such that
\[
	P\left(N^{(n)} > \frac{\lambda}{2}n^{\frac{\beta}{2}}\right) \rightarrow 1.
	\]

\begin{proof}
Define the events $A$, $B$ as
\[A \equiv  p_1\leq P_u\leq p_1+\epsilon_n\]
\[B \equiv p_1+\epsilon_n\leq Q_u\leq p_1+(1-2p_1)a_n.\]
Then, for $u \in \{1, 2, \dots, n\}$ and $0\leq p_1<\frac{1}{2}$:
	\begin{align}
	\no P\left(u\in J^{(n)}\right) &= P\left(A\ \cap \ B\right)\\
	\nonumber &= P\left(A\right) P\left(B \big{|}A \right). \ \
	\end{align}		
So, given $p_1 \in (0,1)$ and the assumption $0<\delta_1<f_p<\delta_2$, for $n$ large enough, we have
	\[
	P(A)  = \int_{p_1}^{ p_1+\epsilon_n}f_P(p) dp,
	\]
so, we can conclude that
	\[
	\epsilon_n\delta_1<P(A) <\epsilon_n\delta_2.
	\]
	We can find a $\delta$ such that $\delta_1<\delta<\delta_2$ and
\begin{equation}\label{eq:5}
	P( A) = \epsilon_n\delta.
\end{equation}
We know
	\[Q_u\bigg{|}P_u=p_u \sim Uniform \left[p_u,p_u+(1-2p_u)a_n\right],\]
	so, according to Figure \ref{fig:piqi_b}, for $p_1\leq p_u\leq p_1+\epsilon_n$,
	\begin{align}
	\no P\left(B | P_u=p_u\right ) &= \frac{p_1+(1-2p_1)a_n-p_1-\epsilon_n}{p_u+(1-2p_u)a_n-p_u} \\
	\nonumber &= \frac{(1-2p_1)a_n-\epsilon_n}{(1-2p_u)a_n}\\
	\nonumber &\geq \frac{(1-2p_1)a_n-\epsilon_n}{(1-2p_1)a_n} \\
	\nonumber &= 1- \frac{\epsilon_n}{(1-2p_1)a_n}, \ \
	\end{align}
which implies
\begin{align}
P\left(B | A\right ) \geq 1- \frac{\epsilon_n}{(1-2p_1)a_n}. \label{eq:6}
\end{align}
Using (\ref{eq:5}) and (\ref{eq:6}), we can conclude
	\[P\left(u\in J^{(n)}\right)\geq \epsilon_n\delta \left(1- \frac{\epsilon_n}{(1-2p_1)a_n}\right).\]
Then, we can say that $N^{(n)}$ has a binomial distribution with expected value of $N^{(n)}$ greater than $n\epsilon_n\delta \left(1- \frac{\epsilon_n}{(1-2p_1)a_n}\right)$, and by substituting $\epsilon_n$ and $a_n$, for any $c'>0$, we get
\[E\left[N^{(n)}\right] \geq \delta\left(n^{\frac{\beta}{2}}- \frac{1}{{c'(1-2p_1)}}\right)   \geq  \lambda n^{\frac{\beta}{2}}.\]

Now by using Chernoff bound, we have
\[P\left(N^{(n)} \leq (1- \theta) E\left[N^{(n)}\right]\right) \leq e^{-\frac{\theta^2}{2}E\left[N^{(n)}\right]},\]
so, if we assume $\theta=\frac{1}{2}$, we can conclude for large enough $n$,
\begin{align}
\nonumber P\left(N^{(n)} \leq \frac{\lambda}{2}n^{\frac{\beta}{2}}\right) &\leq P\left(N^{(n)} \leq \frac{E\left[N^{(n)}\right]}{2}\right)\\
\nonumber &\leq e^{-\frac{E[N^{(n)}]}{8}}\\
\nonumber &\leq e^{-\frac{\lambda n^{\frac{\beta}{2}}}{8}} \rightarrow 0.
\end{align}
As a result, $N^{(n)} \rightarrow \infty$ with high probability for large enough $n.$

\end{proof}
\section{Case Study Implementation Details}
\label{app:case_study}

\subsection{Dataset and Model}
The task presented in section \ref{sec:case_study} involved the summarizing of a news article describing a research development at EPFL in more approachable language and terminology. The summary had as an objective to be even more approachable to people outside the field. 

\paragraph{Dataset} The data for this task came from EPFL's Mediacom department, where they provided the authors with a set of 2370 entries of articles, summaries, and extra information (title, author, date, etc.). Out of these, 50 were chosen so that all articles relayed a newly published work. This was the only criteria for selection. All articles and summaries were pre-processed so as to remove HTML tags.

\paragraph{Model} The model used for this task was GPT-4 \citep{openai2023gpt4}, with its default hyperparameters, called through OpenAI's Chat Completions API. Thus, the model generated a single completion for each prompt, with a temperature of $1$, and with no limitation on the maximum number of tokens (beyond the model's own context length).

\subsection{Experiment Execution}
The experiment was run in two stages. In the initial phase, the model is asked to generate a first summary. It is then provided with feedback and asked to revise its original summary. In this experiment, two distinct types of feedback were provided: \textit{Task Elaboration} (TE) and \textit{Correct Answer} (CA).

\paragraph{Initial Generation}
Following OpenAI's Chat Completions API, the model prompting is done under a chat format. In this setting, the first \textit{message} is a system message stating \texttt{You are a helpful assistant.} This is then followed by an user message, with the following prompt: 

\texttt{Summarize the following article into a short but captivating snippet under around 100 tokens. It must describe both the problem and the approach used to solve it, as well as the venue where these findings were presented, whenever this information is available. \\
Article: [article body]}

The model's response message to this prompt is considered its original summary.

\paragraph{Revised Generation} The revised generation prompt is shares the chat prompting format. It contains the previous chat history, which includes not only the two messages outlined above but also the model's answer as an assistant message. To these three messages, a new user message is added, with the following content:

\texttt{Feedback: [feedback] \\
Please revise your original summary taking the feedback into consideration. If you feel the feedback is not appropriate or useful, you can disregard it.}

The \texttt{[feedback]} placeholder will have one of two different values, depending on the feedback type being provided:
\begin{itemize}
    \item \textbf{Correct Answer:} The feedback will be of the form \\
    \texttt{The correct answer is: [gold\_summary]} \\
    where \texttt{[gold\_summary]} is the summary provided by the Mediacom dataset,
    
    \item \textbf{Task Elaboration:} The feedback will be of the form \\
    \texttt{A good and captivating summary should first grab the reader's attention and make them curious to learn more. It is then important to factually and precisely state what the problem is, why it is important, the proposed solution, and, if published or being divulged, disclose where the reader can find it.}
\end{itemize}

Finally, as in the first stage, the model's response message to this prompt is considered its revised summary.

\subsection{Example Outputs}
In this subsection, we present a few examples outputs from the case study.

\subsubsection{Example 1}
\paragraph{\colorbox{YellowOrange}{Original Article}}
\texttt{The International Consortium of Investigative Journalists (ICIJ), which has over 200 members in 70 countries, has broken a number of important stories, particularly ones that expose medical fraud and tax evasion. One of its most famous investigations was the Panama Papers, a trove of millions of documents that revealed the existence of several hundred thousand shell companies whose owners included cultural figures, politicians, businesspeople and sports personalities. To complete an investigation of this size is only possible through international cooperation between journalists. When sharing such sensitive files, however, a leak can jeopardize not only the story’s publication, but also the safety of the journalists and sources involved. At the ICIJ’s behest, EPFL’s Security and Privacy Engineering (SPRING) Lab recently developed Datashare Network, a fully anonymous, decentralized system for searching and exchanging information. A paper about it will be presented during the Usenix Security Symposium, a worldwide reference for specialists, which will be held online from 12 to 14 August. \\
Anonymity at every stage \\
Anonymity is the backbone of the system. Users can search and exchange information without revealing their identity, or the content of their queries, either to colleagues or to the ICIJ. The Consortium ensures that the system is running properly but remains unaware of any information exchange. It issues virtual secure tokens that journalists can attach to their messages and documents to prove to others that they are Consortium members. A centralized file management system would be too conspicuous a target for hackers; since the ICIJ does not have servers in various jurisdictions, documents are typically stored on its members’ servers or computers. Users provide only the elements that enable others to link to their investigation. \\
Users searching for information enter keywords in the search engine. If the search produces hits, they can then contact colleagues – whose identity remains protected – who are in possession of potentially relevant documents. Search queries are sent encrypted to all users, if there is a macth the querier gets an alert and can decide whether they wish to enter in contact and share information. “Given the fact that users work in different time zones, some with only a few hours of internet access per day, it was critical that searches and responses could take place asynchronously,” notes Carmela Troncoso, who runs the SPRING Lab at the School of Computer and Communication Sciences (IC). Another messaging system, also secure and anonymous, is subsequently used for two-way exchanges. \\
Two completely new secure applications \\
“This system, which addresses real-world needs, has enabled SPRING to tackle some interesting challenges,” notes Troncoso. The research team drew on existing authentication mechanisms and anonymous communication primitives, which they then optimized. They also developed two completely new secure building blocks that are described in the paper being published today in connection with the Usenix Security Symposium – an asynchronous search engine and a messaging system. A new protocol, known as “multi-set private set intersection” (MS-PSI), ensures the security of the search engine, allowing users to easily search a large number of databases without increasing the risk of leaks. The messaging system relies on a large number of single-use virtual mailboxes and is based on the well-known “pigeonhole” system, which chooses one option at random, in this case one of the mailboxes. Currently, the system does not allow users to exchange documents. “At this stage in the process, journalists are using other secure messaging systems,” Troncoso says. \\
Working with the Consortium has allowed SPRING to frame new requirements that are rarely examined in the scientific literature. Datashare can be scaled to thousands of users and millions of documents while encrypting all communications. “The hurdles we encountered during the development process, however, have paved the way to a new area of research with significant potential for other fields,” Troncoso concludes.}


\paragraph{\colorbox{yellow}{Golden Summary}}
\texttt{An EPFL laboratory has developed Datashare Network, a decentralized search engine paired with a secure messaging system that allows investigative journalists to exchange information securely and anonymously. An scientific article on this subject will be presented during the Usenix Security Symposium which will be held online from August 12 to 14.}


\paragraph{\colorbox{pink}{GPT4 Original Summary}}
\texttt{EPFL's Security and Privacy Engineering Lab developed an anonymous, decentralized system called Datashare Network to aid the International Consortium of Investigative Journalists securely exchange information. Offering user anonymity, this system will be presented at the Usenix Security Symposium held online from 12-14 August.}

\paragraph{\colorbox{SkyBlue}{\textit{Correct Answer} Revision}}
\texttt{EPFL's Security and Privacy Engineering Lab created Datashare Network, a decentralized search engine with a secure messaging system that enables investigative journalists to exchange information anonymously and securely. A scientific paper on this solution will be showcased at the online Usenix Security Symposium from August 12 to 14.}

\paragraph{\colorbox{SpringGreen}{\textit{Task Elaboration} Revision}}
\texttt{Investigative journalism goes undercover with Datashare Network, a breakthrough anonymous, decentralized information-sharing system. Developed by EPFL's Security and Privacy Engineering Lab, the network protects journalists from exposure while tackling global issues like tax evasion and medical fraud. Discover how it secures data-sharing at the Usenix Security Symposium, held online from 12-14 August.}



\subsubsection{Example 2}
\paragraph{\colorbox{YellowOrange}{Original Article}}
\texttt{Thermoelectric materials in particular hold vast potential for use in energy applications because they generate electricity from waste heat, such as that generated by industrial processes, by car and truck engines, or simply by the sun. Reducing the thermal conductivity of these materials by a factor of three, for example, would completely revolutionize existing waste-heat recovery, and also all refrigeration and air-cooling technology. \\
A unique theory for all insulating materials \\
In the paper Unified theory of thermal transport in crystals and glasses, out in Nature Physics, Michele Simoncelli, a PhD student at EPFL’s Theory and Simulation of Materials (THEOS) Laboratory – together with Nicola Marzari, a professor at EPFL’s School of Engineering and head of THEOS and of the MARVEL NCCR, and Francesco Mauri, a professor at the University of Rome–Sapienza – present a novel theory that finally decodes the fundamental, atomistic origin of heat conduction. Up to now, different formulations needed to be used depending on the systems studied (e.g., ordered materials, like a silicon chip, or disordered, like in a glass), and there wasn’t a unified picture covering all possible cases. \\
This has now been made possible by deriving directly from the quantum mechanics of dissipative systems a transport equation that covers on equal footing diffusion, hopping, and tunneling of heat.
Waste heat recovery \\
This fundamental understanding will allow scientists and engineers to accurately predict the thermal conductivity of any insulating material (in metals, the heat is carried by the electrons, and that is well understood) – this is exceedingly important for thermoelectrics (i.e. materials that can convert heat into electricity), since these have both crystal- and glass-like properties, and are much needed for waste-heat recovery, or for refrigeration without greenhouse gases (and if you think refrigeration is boring, it is worth remembering that Albert Einstein spent many years trying to invent a new form of refrigerator). \\
In order to develop such next-generation technology, however, scientists first need to understand how and to what extent materials conduct heat. “Up to now, two different equations have been used for calculating thermal properties: one describes perfectly crystalline materials – that is, materials with highly ordered atomic structures – and the other one completely amorphous materials like glass, whose atoms do not follow an ordered pattern,” says Michele Simoncelli. These equations happened to work well in those special cases. “But between these two extremes lie a plethora of interesting cases, and neither equation worked – this is really where our contribution makes a profound difference” \\
* Illustration : Heat is the energy due to vibrations of atoms. The ridges represent the quantum vibrational energy levels of CsPbBr3, a promising thermoelectric material. Heat conduction originates from both diffusion of vibrational excitations (blurred spheres, following the energy level) and quantum tunnelling (jump between different energy levels, represented by blue waves).}

\paragraph{\colorbox{yellow}{Golden Summary}}
\texttt{EPFL and MARVEL researchers have developed a new theory for heat conduction that can finally describe and predict the thermal conductivity of any insulating material. This new formulation will let scientists make accurate predictions of thermal conductivity in a range of materials for critical engineering applications – from electronics to lasers to waste-heat recovery.}

\paragraph{\colorbox{pink}{GPT4 Original Summary}}
\texttt{Scientists at EPFL's Theory and Simulation of Materials Laboratory have developed a unified theory to predict the thermal conductivity of insulating materials, opening up possibilities for more efficient energy applications. The research breakthrough, published in Nature Physics, could lead to waste-heat recovery, revolutionising refrigeration and air-cooling technology without greenhouse gases. Understanding how materials conduct heat is key to developing next-generation thermoelectric technology.}

\paragraph{\colorbox{SkyBlue}{\textit{Correct Answer} Revision}}
\texttt{Researchers at EPFL and MARVEL have developed a groundbreaking theory for heat conduction, enabling the description and prediction of thermal conductivity in any insulating material. Published in Nature Physics, the new formulation offers potential for critical engineering applications such as waste-heat recovery, electronics, and lasers, enhancing energy efficiency in various industries.}

\paragraph{\colorbox{SpringGreen}{\textit{Task Elaboration} Revision}}
\texttt{Researchers have unlocked the secret to more efficient energy applications by developing a unified theory for predicting the thermal conductivity of insulating materials, transforming waste-heat recovery and revolutionizing refrigeration technology. With this innovative approach, published in Nature Physics, scientists can now accurately predict and optimize thermoelectric materials, paving the way for energy-efficient solutions without greenhouse gas emissions. Discover the full potential of this game-changing breakthrough at EPFL's Theory and Simulation of Materials Laboratory.}




\subsubsection{Example 3}
\paragraph{\colorbox{YellowOrange}{Original Article}}
\texttt{With its 500 km diameter, the asteroid Vesta is one of the largest known planet embryos. It came into existence at the same time as the Solar System. Spurring scientific interest, NASA sent the Dawn spacecraft on Vesta’s orbit for one year between July 2011 and July 2012. \\
Data gathered by Dawn were analyzed by a team of researchers from EPFL as well as the Universities of Bern (Switzerlanf), Brittany (France) and Arizona (USA). Conclusion : the asteroid's crust is almost three times thicker than expected. The study does not only have implications for the structure of this celestial object, located between Mars and Jupiter. Their results challenge a fundamental component in planet formation models, namely the composition of the original cloud of matter that aggregated together, heated, melted and then crystallized to form planets. \\
At EPFL’s Earth and Planetary Science Laboratory (EPSL), led by Philippe Gillet, Harold Clenet had a look at the composition of the rocks scattered across Vesta’s ground. "What is striking is the absence of a particular mineral, olivine, on the asteroid’s surface," said the researcher. Olivine is a main component of planetary mantles and should have been found in large quantities on the surface of Vesta, due to a double meteorite impact which, according to computer simulations, "dug" the celestial body’s southern pole to a depth of 80 km, catapulting large amounts of materials to the surface. \\
The two impacts were so powerful that more than 5 \% of Earth's meteorites come from Vesta. « But these cataclysms were not strong enough to pierce through the crust and reach the asteroid's mantle,» Harold Clenet continued. The meteorites originating from Vesta and found on Earth confirm this since they generally lack Olivine, or contain only minute amounts compared to the amount observed in planetary mantles. Also, the spacecraft Dawn did not find olivine in the vicinity of the two impact craters. « This means that the crust of the asteroid is not 30 km thick, as suggested by the models, but more than 80 km. » \\
Composition of planets \\
These discoveries challenge models that describe the formation of Vesta, and consequently the formation of rocky planets in the Solar System including planet Earth. Cooling theory and "re- melting" phenomena in the depths of previously solidified elements would also need to be reviewed. "The crust might have been thickened by the formation of "plutons", that is: igneous rock intrusions, hundreds of meters large, some of which emerged to the surface," explained the scientist. \\
If Vesta has less of a (olivine-rich) mantle and more of a (pyroxene-rich) crust, then the proportion of materials making up Vesta, and probably the Earth and other telluric planets (Mars, Venus, Mercury), is different from what was previously expected. \\
A more complex model of planet formation therefore has to be considered, one that takes into account not only the original composition of planets, but also their orbits, sizes and related cooling times. Vesta is the only known asteroid that has an earth-like structure - with a core, mantle and crust – making it an incredible laboratory for testing hypotheses and theories.}

\paragraph{\colorbox{yellow}{Golden Summary}}
\texttt{EPFL researchers have a better understanding of the asteroid Vesta and its internal structure, thanks to numerical simulations and data from the space mission Dawn. Their findings, published today in Nature, question contemporary models of rocky planet formation, including that of Earth.}

\paragraph{\colorbox{pink}{GPT4 Original Summary}}
\texttt{Data collected from asteroid Vesta poses challenges to planet formation models as it reveals the asteroid's crust is almost three times thicker than expected. The discovery, made by researchers from the Earth and Planetary Science Laboratory and partner universities, suggests that original assumptions about the composition of the cloud of matter that formed planets may be incorrect. The findings could lead to a more complex model of planet formation, taking into account composition, orbit, size, and cooling times.}

\paragraph{\colorbox{SkyBlue}{\textit{Correct Answer} Revision}}
\texttt{EPFL researchers have gained insights into asteroid Vesta's internal structure, utilizing numerical simulations and data from the Dawn space mission. The findings, published in Nature, indicate Vesta's crust is almost three times thicker than expected, questioning current models of rocky planet formation, including Earth's. This could lead to more complex planet formation models that consider composition, orbit, size, and cooling times.}

\paragraph{\colorbox{SpringGreen}{\textit{Task Elaboration} Revision}}
\texttt{Astonishing findings from asteroid Vesta are challenging our understanding of planet formation, as researchers discovered its crust is nearly triple the expected thickness. This breakthrough, made by EPFL's Earth and Planetary Science Laboratory and partner universities, questions key elements of planet formation models and paves the way for a more complex approach, considering composition, orbits, sizes, and cooling times. With Vesta being the only known asteroid with an Earth-like structure, this revelation creates a compelling lure for further explorations.}




\bibliographystyle{IEEEtran}
\bibliography{REF}
%\begin{IEEEbiography}{Zarrin Montazeri}
%Biography text here.
%\end{IEEEbiography}
%
%
%
%\begin{IEEEbiography}{Amir Houmansadr}
%Biography text here.
%\end{IEEEbiography}
%
%\begin{IEEEbiography}{Hossein Pishro-Nik}
%Biography text here.
%\end{IEEEbiography}







% that's all folks
 \end{document}
