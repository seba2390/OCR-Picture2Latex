In this thesis, two types of adaptive control was proposed for two operational scenarios of a quadrotor: 
\begin{itemize}
    \item \textbf{Aerial Transportation of Unknown Payloads:} In this scenario, the payload does not change during the flight. Under such a case, an adaptive controller for quadrotors was proposed, which can tackle parametric uncertainties and external disturbances without their a priori knowledge for the application of carrying unknown payloads. The control framework was built to negotiate possibly a priori unbounded state-dependent uncertainties. Closed-loop system stability was established via the notion of uniformly ultimately boundedness. The performance of the proposed controller was verified using Gazebo simulation using RotorS Simulator framework, and comparative simulations confirmed the effectiveness of the proposed scheme against state-of-the-art methods under various scenarios.

\item \textbf{Dynamic Payload Lifting Operations:} Contrary to the first scenario, the payload mass was considered to change for this scenario. Accordingly, a new concept of adaptive control design for quadrotors was introduced for vertical operations, using the framework of average dwell time based switched dynamics. The proposed design did not require any a priori knowledge of the structure and bound of uncertainty. The effectiveness of the concept was validated via simulations with dynamically varying payloads.
\end{itemize}




\section{Future Works}
\begin{itemize}
    \item The current work on dynamic payload lifting operations uses the hovering control dynamics. A more realistic scenario would be to operate in six degrees-of-freedom. In order to solve this problem, a future controller needs to utilise the partly decoupled dynamics for the switched mode control approach.
    
    \item The adaptive controllers developed in this thesis do not tackle the situations of manoeuvring under various motion constraints (e.g., moving through a window, pipeline interior inspection). Therefore, an interesting future work would be to extend the proposed adaptive designs under position and velocity constraints. %system would also be taken as a research statement.
    
    \item Throughout the thesis, the external payloads are considered to be rigidly attached to the quadrotor platform. The case of suspended payload thus needs to be considered in future. 
\end{itemize}
 