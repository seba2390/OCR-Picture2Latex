Robotics has been one of the significant fields of interest owing to science fiction and fantasies. The idea of machines replacing a human in everyday tasks is so beneficial because of two main reasons. Firstly, the inefficiency of humans in doing mundane tasks. A free-thinking human is inefficient in performing repetitive tasks due to constraints such as boredom, distraction etc. The other reason is that machines (or robots in specific) can transcend the mechanical limitations of a human. Humans somehow wanted to enslave machines for these two reasons, and hence these machines were named robots (meaning `forced labours'). 

Much similar to the evolution of living beings, there has been a notable evolution of robots too. Humans invented robots for specific operations when humanoid fantasies were far-fetched and impractical with the technologies of their time. Hence, robots came in numerous feasible mechanical forms with task-specific features very similar to various living things. One significant step of robotic evolution was the autonomy of a robotic system, which relieves humans from the loop and drastically reduces the need for manual labour.

The autonomy of robots (or autonomous systems in general) posed a need for a new science stream called control. A robotic system can be defined as an integrated unit of machinery that coordinates to perform a task. In this context, an autonomous system is a robot that can perform tasks without being manually controlled by a human.

Autonomous vehicles/ mobile robots are fantasies brought to life with the advancement in control techniques. Autonomous vehicles are extensively used not only for research purposes but also for commercial applications. Autonomous vehicles are of different forms ranging from cars to ships to quadrotors. While the use cases of aerial and marine robots are limited due to safety constraints, road vehicles (carrying humans) are equipped with an autopilot feature already. So, every major automobile manufacturer is shifting its focus towards autonomous vehicles.

Control of autonomous vehicles is a broad category ranging from a high-level control in choosing an optimal trajectory to the low-level control in deciding the appropriate actuator inputs. The problem statement of each level of control is different from one another. For example, the high-level control might be a Model Predictive Control, which would have to give desired setpoint velocities to actuators, assuming an ideal actuator model. However, when it comes to low-level control, the goal would be to calculate the desired forces and torques required to follow the velocity or position trajectory that has been given as an input to the controller while relying on prior information about the dynamics of the system. An even lower level controller would translate the force and torque inputs into current or fuel output.





\section{Aerial Robotics}
Aerial robots are one of the best examples to show how humans invented machines to surpass their physical limitations. Aerial transportation has allowed a species that perceives the universe in three dimensions to exploit the third dimension for efficient travelling. As modern-day researchers on autonomous vehicles rightfully say, ``the additional dimension would solve the exponentially increasing road traffic problems".

Aerial vehicles are broadly classified as fixed-wing and rotary-wing quadrotors based on their mechanical design. Unmanned aerial vehicles (UAV) is a breakthrough in the applications of aerial vehicles. UAVs are not necessarily autonomous. They can also be teleoperated. In both cases, control plays an incredible role. Similar to other applications, the control problem of UAVs is also a broad category. The focus of this particular research is on low-level control of quadrotors. Control of a UAV is exceptionally critical compared to that of rovers (ground robots) because there is no braking mechanism in aerial robots when the system goes unstable.

The mechanical advantage in the structure of the fixed-wing aircraft makes it convenient for the transport of cargo and humans. It is widely used for personal, commercial, research, and defence purposes. However, the major disadvantage of fixed-wing aircraft is its take-off mechanism, which needs a runway, non-holonomic design, and its inability to hover on a position in the air. These disadvantages make them infeasible in many applications such as indoor goods transfer, disaster relief, surveying etc. and make room for rotary-wing quadrotors.

\section{Dynamics of Quadrotors}
Modelling any robotic system can be divided into four sub-sections: model representation, reference frames, representation of state variables, and dynamic equations of the system. Though there are different methods to model the dynamics in each subsection, I would focus on the method used in my works.

\subsection{Model Representation}
For complete dynamic representation, including non-linearities, a set of differential equations is used. For simplified modelling to use a linear controller, state-space representation might be convenient. However, it would not be optimal to design nonlinear controllers. In general, the UAV dynamics can be of a time-invariant differential form, as shown in the following equation

\begin{align*}
    a_n\frac{d^n y}{dt^n} + a_{n-1}\frac{d^{n-1} y}{dt^{n-1}} + a_{n-2}\frac{d^{n-2} y}{dt^{n-2}} +...+ a_1\frac{dy}{dt} + a_0y = b u
\end{align*}
where the left hand side represents the state variable (the variable that needs to be controlled) $y$ multiplied by gains $a_i$ and its derivatives, and the right hand side represents the control variable $u$ scaled by a factor of $b$. Depending on which degree of control is required, the differential equation of sufficient order is chosen. For a robotic system, typically the control inputs are forces and torques to control position and velocity of each degree-of-freedom. Hence, a second order differential equation is chosen.

\subsection{Reference Frames}
A crucial step in modelling the dynamics is to identify the frames of reference. For quadrotor applications, two reference frames are considered for ease of representation: an inertial frame to represent the absolute position and orientation of the quadrotor and a body-fixed frame to represent the body forces and torques.

\subsubsection{Inertial Frame}
The inertial reference frame is where a body with zero net force acting on it is either at rest or in motion along a straight line. It is generally taken the ground fixed plane. So the frame is stationary if we neglect the movement of the Earth. It is also known as the World frame and is taken such that its X-axis is aligned with the forward direction, Z-axis in the upward direction and Y-axis is chosen based on right-hand rule. Differential equations to represent the system dynamics is written inertial frame reference.

\subsubsection{Body-Fixed Frame}
The body-fixed frame is fixed to the quadrotor with the same axes convention as in the inertial frame.  Every frame has three orthonormal axes passing through the center-of-mass (CoM). For convenience, these axes are chosen as the principal axes for the body-fixed frame similar to the inertial frame. Forces from the propellers act on the body-fixed frame. Considering the example of a quadrotor, the thrusts from the motors produce forces and torques in the body-fixed frame. So, the angular velocities are represented in the body-fixed frames. 

Figure \ref{fig:frames} shows the inertial and body-fixed frames with origins at $O_W$ and $O_B$, respectively. The position vector $p$ gives the relative position of CoM of the quadrotor (origin of body-fixed frame) in the inertial frame.

\begin{figure}
	\centering
	\includegraphics[width=0.5\textwidth]{LaTex/figures/frames_axes.jpg}
	\caption{{A schematic of reference frames.}}\label{fig:frames}
\end{figure}

\subsection{Representation of State Variables}
Typically the pose of a quadrotor is represented by using the position vector $[x, y, z]^T$ and the orientation $[\phi, \theta, \psi]^T$, where $(\phi, \theta, \psi)$ represent roll, pitch and yaw angles respective. Though there are other conventions for representing the pose of a quadrotor, we would see a few methods used in the controller design. 

%\subsubsection{Euler Angles}
%Euler angle represents the orientation of one frame with respect to another frame using three cascaded rotations of angles ($\phi, \theta$ and $\psi$) along X, Y and Z axes. This representation is the minimalistic representation of orientation. However, this representation is not unique. There can be three possible sets of Euler-angles to represent the same rotation based on the order of rotations. Also, they have singularities and discontinuities. Hence, these are not used in the equations. Though Z-Y-Z sequence is also used to find the Euler angles, we use the sequence Z-Y-X to find $\psi$, $\theta$ and $\phi$ respectively.

%\subsubsection{Rotation Matrices}
%The most convenient way to represent the orientation of a frame with respect to another is using a rotation matrix. A rotation matrix has the unit vector of a given frame's principal axes with respect to the reference frame as its columns. The rotation matrix forms a special orthogonal set of $n$ dimensions.

%Euler angles can be converted into rotation matrices as shown below:

%\begin{align*}
%    R_x(\phi) = 
%    \begin{bmatrix}
%    1 & 0 & 0 \\
%    0 & cos(\phi) & -sin(\phi) \\
%    0 & sin(\phi) & cos(\phi)
%    \end{bmatrix}
%\end{align*}

%\begin{align*}
%    R_y(\theta) = 
%    \begin{bmatrix}
%    cos(\theta) & 0 & sin(\theta) \\
%    0 & 1 & 0 \\
%    -sin(\theta) & 0 & cos(\theta)
%    \end{bmatrix}
%\end{align*}

%\begin{align*}
 %   R_z(\psi) = 
 %   \begin{bmatrix}
%    cos(\psi) & -sin(\psi) & 0 \\
%    sin(\psi) & cos(\psi) & 0 \\
%    0 & 0 & 1
%    \end{bmatrix}
%\end{align*}

%A complete rotation in 3-dimension mentioned using Z-Y-X Euler angles form can be represented by cascading rotation matrices as given below:
%\begin{align*}
%R(\Theta) &= R_x(\phi)R_y(\theta)R_z(\psi)
%\end{align*}

%This cascaded form makes representation of sequential rotations simpler. %Though Euler angles have lesser number of parameters, they misinterpret the UAV's orientation in case of a gimbal lock.

%\subsection{Dynamic Equations}
%The last part of the modelling is to derive the dynamic equations of the system. The dynamic equations are a set of differential equations that represent the relationships between the state variables. These equations define how the system responds to different inputs and state over time. They can be derived based on Newton's second law. There are two popular methods to derive the equations: Newton-Euler's method and Euler-Lagrange method. They both work under the assumption that the body of the UAV is rigid.

%The disadvantage of Newton's laws of motion is that they are inconvenient for curved coordinate systems such as radial or angular coordinates can be surpassed in Euler-Langrangian equations as they have the generalised form for all standard coordinate systems. Equation (1) represents the Euler-Lagrangian equations.

%\begin{equation*}
%\frac{d}{dt}(\frac{\partial{L}}{\partial{\dot{q}_j}}) - \frac{\partial{L}}{\partial{q}_j} = 0
%\end{equation*}

%where $q_j$ is the $j$th state variable, $L$ is called the Lagrangian, and $t$ is time. The Lagragian $L$ is the difference between the total kinetic energy, $T$ and total potential energy, $V$ of the system.
%\begin{equation*}
%L = T - V
%\end{equation*}
%Given the system's total kinetic and potential energies, its mechanics can be represented in any standardised coordinate system.

% Any mechanical system (robot) can be represented by the following equation
% \begin{align}
% & \mathbf M( {\mathbf r}(t))\ddot{ \mathbf r}(t)+\mathbf C({\mathbf r}(t),\dot{\mathbf r}(t))\dot{\mathbf r}(t)+ \nonumber
% \mathbf G(\mathbf r(t))+ \mathbf F(\dot{\mathbf r}(t))+\mathbf{ d}(t)=\boldsymbol \tau(t), \label{sys_1}
% \end{align} 
% where $\mathbf{q}, \dot{\mathbf r} \in\mathbb{R}^{n}$ are the system states; $\mathbf{M(q)}\in\mathbb{R}^{n\times n}$ is the mass/inertia matrix; $\mathbf C(\mathbf r,\dot{\mathbf r})\in\mathbb{R}^{n\times n}$ denotes the Coriolis, centripetal terms; $\mathbf G(\mathbf r)\in\mathbb{R}^{n}$ denotes the gravity vector; $\mathbf F(\dot{\mathbf r})\in\mathbb{R}^{n}$ represents the vector of damping and friction forces; $\mathbf{ d}(t) \in \mathbb{R}^n$ denotes an external disturbance and $\boldsymbol \tau \in \mathbb{R}^n$ is the generalized control input. 


% For most EL systems of practical interest, (\ref{sys_1}) presents a few interesting properties {(cf. \cite[Sect. 9.5]{spong2008robot})}, which are later exploited for control design and stability analysis:

% \noindent \textbf{Property 1:} $\exists \overline c, \overline{g}, \overline f, \overline{d} \in \mathbb{R}^{+}$ such that $||\mathbf{C(q},\dot{\mathbf r})|| \leq \overline c ||\dot{\mathbf r}||$, $||\mathbf G(\mathbf r)|| \leq \overline g$, $||\mathbf{F}(\dot{\mathbf r})|| \leq \overline f ||\dot{\mathbf r}||$ and $||\mathbf d(t)|| \leq \overline{d}$, $\forall \mathbf r,\dot{\mathbf r}$, $\forall t \geq 0$.\\
% \textbf{Property 2:} The matrix $\mathbf{M(q)}$ is symmetric and uniformly positive definite in $\mathbf{q}$, i.e. %with respect to $\mathbf{q}$. This implies that 
% $\exists \underline m, \overline m \in \mathbb{R}^{+}$ such that
% \begin{equation}\label{prop 3}
% 0 < \underline m \mathbf I \leq \mathbf{M(q)} \leq \overline m \mathbf I .
% \end{equation}
% \textbf{Property 3:} The matrix $(\dot{\mathbf M}(\mathbf r) -2{\mathbf C( \mathbf r,\dot{\mathbf r})})$ is skew symmetric, i.e., for any non-zero vector $\mathbf z$, we have $\mathbf z^T(\dot{\mathbf M}(\mathbf r)\-2\mathbf{C(r},\dot{\mathbf r}))\mathbf{z}=0$.

% The interested reader can check  \cite{5711693,7968486,8732468,SKJETNE2004203,SKJETNE2005289,4287130,8624443} and references therein to see how \eqref{sys_1} apply to many under-actuated robotic systems spanning from cranes to vehicles, or to standard benchmarks such as cart-poles and Acrobots.

\subsection{Quadrotor Dynamics}
The last part of the modelling is to derive the dynamic equations of the system.

The forces and torques in the quadrotor are only caused by the rotation of the propellers. Each propeller produces an upward force along the $Z_B$ axis in the body-frame and two different torques: one due to the upward force acting along $X_B$ and $Y_B$ axes, and the other is due to the spinning of each propeller, which causes a torque along the $Z_B$ axis. The quadrotor is designed symmetrically about the origin in the $X_BY_B$ plane to reduce complexity in dynamics. Also, the CoM, $O_B$ is maintained on the plane of the propeller.

The quadrotor's body frame is designed with its origin at the CoM of the quadrotor and $X_B$ and $Y_B$ axes along two adjacent arms (the connecting rod between the centre and the rotor), or 45 degrees on either side of an arm. The first configuration is called the "+" (plus) configuration, and the latter is called the "x" configuration (Fig. \ref{fig:q450_quad}). Here we demonstrate the dynamics of the quadrotor in the ``x" configuration.


\begin{figure}
	\centering
	\includegraphics[width=\textwidth]{LaTex/figures/q450_quad.jpg}
	\caption{{A picture of Q450 quadrotor with ``x" configuration.}}\label{fig:q450_quad}
\end{figure}

When all propellers are spinning in the same direction with equal speed, the opposite propellers cancel out the torques along the $X_B$ and $Y_B$ directions. However, the torque along the $Z_B$-axis scales up, and the quadrotor starts to spin along the $Z_B$-axis while holding its position. So, the opposite propellers spin in the direction, while the adjacent ones spin in the opposite directions.

As the rotors are always facing downward, the quadrotor's thrust is always along the negative $Z_B$-axis. This essentially causes the quadrotor to move upwards against gravity when the body frame $Z_B$-axis is aligned with the world frame $Z_W$-axis. The velocity in $X_B$ and $Y_B$ directions are caused by pitching and rolling the quadrotor. To make the quadrotor pitch forward, the motors in the forward direction have to run slower than those in the backward direction. Similarly, to roll on the right side, the motors on the left side have to run faster than the ones on the left side and vice versa. Mismatch in the speeds of rotors can cause a difference between the torques in clockwise and anti-clockwise directions. This difference would make the quadrotor yaw. So, the sums of the squares of the speeds of the opposite motors are maintained to be equal.

In this work we use the dynamic model of the quadrotor that is popularly used in works such as, \cite{lee2010geometric}, \cite{rotors_sim_2016} and \cite{mellinger2014traj}. The relationships between the thrust, $T_{th}$, torques $\tau_{\phi}, \tau_{\theta}, \tau_{\psi}$ and the rotor speeds $\omega_i$, where $i = 1,2,3,4$ are given by,

\begin{align}
    \begin{bmatrix}
        T_{th} \\ \tau_{\phi} \\ \tau_{\theta} \\ \tau_{\psi}
    \end{bmatrix} =
    \begin{bmatrix}
        C_T & C_T & C_T & C_T \\
        0 & lC_T & 0 & -lC_T \\
        -lC_T & 0 & lC_T & 0 \\
        -C_TC_M & C_TC_M & -C_TC_M & C_TC_M
    \end{bmatrix}
    \begin{bmatrix}
        \omega^2_1 \\
        \omega^2_2 \\
        \omega^2_3 \\
        \omega^2_4 
    \end{bmatrix}
\end{align}

where $C_T$ and $C_M$ are thrust constant and moment constant respectively, and $l$ is the arm-length (please refer to Fig \ref{fig:quad_model}).

\begin{figure}
	\centering
	\includegraphics[width=\textwidth]{LaTex/figures/quad_model.jpg}
	\caption{{A schematic representing a quadrotor's dynamics.}}\label{fig:quad_model}
\end{figure}

When there are no external disturbances, these are the only forces and torques the quadrotor would experience other than gravity. As seen from the equations, there are only four control parameters that can be controlled, the angular velocities of the rotors. The control of these velocities would facilitate the control of upward thrust, roll, pitch and yaw. The linear motion in other directions can be controlled only using the control of roll and pitch. So, the quadrotor is an underactuated system with 4 control inputs for 6 DoF.

The linear forces acting on the system are gravitational force, $G$ along the negative $Z_W$-axis in the inertial frame and the thrust $T_{th}$, acting along the positive $Z_B$-axis in the body frame. Hence we use the linear dynamics of the system as given in \ref{pos_dyn}.

\begin{align}\label{pos_dyn}
    &m 
    \begin{bmatrix}
        \ddot{x}\\ 
        \ddot{y}\\ 
        \ddot{z}
    \end{bmatrix} + 
    \begin{bmatrix}
        0\\ 
        0\\ 
        mg
    \end{bmatrix} = \mathbf{R}^T 
    \begin{bmatrix}
        0\\ 
        0\\
        T_{th}
    \end{bmatrix}
\end{align}\\

where $g$ is the acceleration due to gravity, $m$ is the total mass of the quadrotor, $R$ is the rotational matrix that is used to map the forces from the inertial frame to the body fixed frame. As per properties of orthonormal matrices, $R^T$ maps vectors in body frame to the inertial frame. The rotational dynamics are given by the form,
\begin{align}
    &\frac{I_{xx}}{l}\ddot{\phi} + \frac{I_{zz}-I_{yy}}{l}\dot{\varphi}\dot{\psi} = \tau_\phi,\nonumber\\
&\frac{I_{yy}}{l}\ddot{\varphi} + \frac{I_{xx}-I_{zz}}{l}\dot{\phi}\dot{\psi}= \tau_\varphi, \nonumber\\
& I_{zz}\ddot{\psi} + (I_{yy}-I_{xx})\dot{\varphi}\dot{\phi}= \tau_\psi, 
\end{align} \label{att_sub}

where $l$ is arm-length of the rotor units; $I_{xx}, I_{yy},I_{zz}  $ are the inertia terms in $x,y$ and $z$ directions respectively. The rotation matrix $R$ is given by 
$$\mathbf{R} =\begin{bmatrix}
		c_\psi c_\varphi & s_\psi c_\varphi & -s_\varphi \\
		c_\psi s_\varphi s_\phi  - s_\psi c_\phi & s_\psi s_\varphi s_\phi  + c_\psi c_\phi & s_\phi  c_\varphi\\
		 c_\psi s_\varphi c_\phi + s_\psi s_\phi  &  s_\psi s_\varphi c_\phi - c_\psi s_\phi  & c_\varphi c_\phi
		\end{bmatrix}$$
where, $c_{(\cdot)},s_{(\cdot)}$ denote $\cos_{(\cdot)},\sin_{(\cdot)}$; $m$ is the mass of the overall system.

In this work, we have used the following quadrotor dynamics.

\begin{align} \label{el_pos}
   \mathbf{ M_p\ddot{p}} + \mathbf{G} &= \mathbf{\mathcal{T}_p}  \nonumber \\
    \mathbf{M_q(q)\ddot{q}} + \mathbf{C_q(\dot{q})\dot{q}} &= \mathbf{\mathcal{T}_q}
\end{align}

with

\begin{align}
    \mathbf{M_p} & = \begin{bmatrix}
        m & 0 & 0 \\
        0 & m & 0 \\
        0 & 0 & m
    \end{bmatrix}, ~
    \mathbf{M_q} = \begin{bmatrix}
        \frac{I_{xx}}{l} & 0 & 0 \\
        0 & \frac{I_{yy}}{l} & 0 \\
        0 & 0 & I_{zz}
    \end{bmatrix}, ~
    \mathbf{G} = \begin{bmatrix}
        0 \\
        0 \\
        mg
    \end{bmatrix}\nonumber \\
    \mathbf{C_q} &= \begin{bmatrix}
        0 & 0 & \frac{I_{zz} - I_{yy}}{l}\dot{\varphi} \\
        \frac{I_{xx} - I{zz}}{l}\dot{\psi} & 0 & 0 \\
        0 & \frac{I_{yy} - I{xx}}{l}\dot{\phi} & 0
    \end{bmatrix}, ~
    \mathbf{p} = \begin{bmatrix}
        x \\
        y \\
        z
    \end{bmatrix}, ~
    \mathbf{q} = \begin{bmatrix}
        \phi \\
        \varphi \\
        \psi
    \end{bmatrix}\nonumber \\
    \mathbf{\mathcal{T}_p} & = \begin{bmatrix}
        \tau_x \\
        \tau_y \\
        \tau_z
    \end{bmatrix}, ~
    \mathbf{\mathcal{T}_q} = \begin{bmatrix}
        \tau_{\phi} \\
        \tau_{\varphi} \\
        \tau_{\psi}
    \end{bmatrix}\nonumber \\
\end{align}

where, $\mathbf{M_p}$ and $\mathbf{M_q}$ are diagonal matrices that represent the mass and inertia of the quadrotor respectively; $\mathbf{C_q}$ is the Coriolis matrix containing the cross-coupling terms; $\mathbf{p}$ and $\mathbf{q}$ are the position and orientation vectors of the quadrotor in the inertial frame, $\mathbf{G}$ is the gravity vector and $\mathbf{\mathcal{T}_p}$ and $\mathbf{\mathcal{T}_q}$ are the force and moment vectors acting on the CoM of the quadrotor.


%Depending on the applications, the dynamic equations can be chosen. Interested readers may refer to \cite{zhangsurvey} for the different types of quadrotor modelling and system identification techniques.




\section{Control of Quadrotors}
%UAVs, especially quadrotors, has been one of the main reasons for the advancements in control theory. Research in quadrotor control started as a reliable and stable control of quadrotors. However, modern-day research is on a wide range of topics to improve the performance of quadrotor in various applications. 
The initial works of quadrotor control were focused on reducing the complexity of the dynamics of the quadrotor. The simplest of all is the hovering mode control, where the quadrotor is assumed to have no disturbances across the $X_W$ and $Y_W$ axes. This type of control can be used to hover the quadrotor at any given height, and thus, only controlling the altitude of the quadrotor was sufficient. The earliest of control designs was well-proven Proportional-Integral-Derivative (PID) controller \cite{1302409,1389776, boua_pid}. %A block diagram of PID controller is shown in Fig: \ref{fig:pid_con}.

%\begin{figure}[!h]
%	\centering
%	\includegraphics[width=\textwidth]{LaTex/figures/pid_controller.jpg}
%	\caption{{A schematic of PID controller.}}\label{fig:pid_con}
%\end{figure}

Due to the limited applications of hovering mode, the next step in the development of quadrotor control was feedback linearization \cite{280180, smallanglemodel2, smallanglemodel3}: this approach is also known as small-angle variation control as it assumes that the variation in roll and pitch to be minimal.
This method is ideal for near hovering flying modes, and the adaptive controller is also designed for this specific condition \cite{lee2009asmc}. Though small-angle approximation based control is easy to implement, it might be unstable when the quadrotors need to perform aggressive manoeuvres.

 One of the significant pieces of research on the control of quadrotors is the geometric controller introduced by \cite{lee2010geometric}. This work proposed a partially decoupled dynamic model of the quadrotor (cf. Chapter 2 for details) and a dual-loop control system for the same. The geometric control in the manifold of rotation matrices makes it suitable for aggressive manoeuvres. Another notable work in this line came from \cite{mellinger2011minimum} and their group thereafter. The completely decoupled dynamic model of quadrotors, however is still in the theoretical phase and needs additional mechanical systems to realize it in practice (cf. tiltrotor mechanism \cite{8688053, 8815013}. An exceptional survey on the existing control strategies for autonomous quadrotors can be found in \cite{kimsurvey}.

Nevertheless, the aforementioned works (and references therein) depends heavily on the accuracy on system modelling and dynamics. In reality, it is tedious and almost impossible to model the exact dynamics of the quadrotor due to factors such as nonlinearities, inaccurate parametric knowledge, imbalance in the mass distribution, external disturbances etc. When quadrotors are deployed for applications such as surveying, manipulation, or payload transport \cite{aerial_mani, tang2015mixed, yang2019energy}, their dynamics keep changing due to payload variations. In this thesis, we particularly look into two types of aerial transportation and the control problems and solutions for them. We discuss these two scenarios below:

\subsection{Adaptive Path Tracking of Quadrotors for Aerial Transportation of Unknown Payloads} \label{motive_1}
Over the past two decades, quadrotors have been a source of considerable research interest owing to their advantages such as simple structure, vertical taking off and landing, rapid manoeuvring etc. \cite{fusini2018nonlinear, kapoutsis2013autonomous, nazaruddin2018communication}. Such advantages are crucial in various military and civil applications such as surveillance, fire fighting, environmental monitoring, to name a few \cite{invernizzi2019dynamic, invernizzi2019integral}. Most recently, global research is more and more interested in smart transport systems \cite{tang2015mixed, yang2019energy}.

One of the most common ways to attach a payload to a quadrotor is via a suspended cable, which allows the quadrotor to retain most of its agility \cite{tang2015mixed, sreenath2013trajectory, yang2019energy}. However, in indoor scenarios, especially in disaster sites, the quadrotor might need to manoeuvre through constrained altitudes, where using a suspended payload may not be optimal/desirable. In such scenarios, rigidly attaching a payload would be a preferred mode, which also provides the flexibility to autonomously pick up and drop the payload. From a research point of view, crucial control challenges arise when the mass and inertia of the quadrotor vary due to unknown payload, imprecise knowledge of the quadrotor parameters and unknown external disturbances: these uncertainties make the control design for a payload-carrying quadrotor challenging. %In the following, we present state-of-the-art control designs for quadrotors, together with existing challenges and the contributions brought out by this research work.

\subsection{Dynamic Payload Lifting Operations: In-flight Payload Variation}\label{motive_2}
\begin{figure}[!h]
	\centering
	\includegraphics[width=\textwidth]{LaTex/figures/exp_1.jpg}%{FIG5_16-TIE-3042.eps}
	\caption{{A schematic representing a quadrotor ascending with dynamic payload.}}\label{fig:mot} % (the response of link 1 is omitted due to the limited space)
\end{figure}
In the previous scenario, the payload mass does not vary during the flight. However, in a typical construction assignment or emergency evacuation, a quadrotor might be required to lift or drop varying payloads (construction and disaster relief materials), emergency evacuation from a high rise building, where humans are rescued from various floors. Dynamic variations in payload are orchestrated from these situations. One such lifting operation is sketched in Fig. \ref{fig:mot} via three phases: (i) in the first phase, the quadrotor rests on the ground without any payload; (ii) in the second phase, it starts to ascend with an initial payload; (iii) and in the third phase, a new payload is attached to the quadrotor as it ascends to its desired height. Similar phases can be observed during a construction scenario. In a reverse scenario to Fig. \ref{fig:mot}, a quadrotor may descend from a height while releasing payloads at different heights (e.g., dropping/throwing fire extinguishing materials at different floors). Clearly, the overall mass/inertia of the quadrotor changes (switches) during the interchange of these phases. 

Though robust controllers \cite{zhao2015nonlinear, nersesov2014estimation, sankaranarayanan2009control, xu2008sliding, Ref:17} might solve the problem of variation in dynamics to an extent, they also suffer when bounds of uncertainties are unknown. Hence, an adaptive controller is typically employed to solve these issues \cite{nicol2011robust, bialy2013lyapunov, dydek2012adaptive, ha2014passivity, tran2018adaptive, tian2019adaptive, zhao2014nonlinear, yang2019energy, roy2019adaptive, roy2020adaptive}. However, adaptive control typically requires structural knowledge of the system and cannot handle unmodelled dynamics, and such prerequisites are difficult to meet in the cases of unknown payload transport. Further, it is well-known that under a switched dynamics, typically arising during dynamic payload variation, conventional controllers are not suitable \cite{lai2018adaptive,yuan2018robust,yuan2017adaptive,lou2018immersion, chen2018global, ye2021robustifying}. In addition, to the best of our knowledge, no control solution exists to tackle the interchanging/switched dynamics of a quadrotor in an adaptive setting. This thesis aims to solve these control challenges.


\section{Preliminaries}
\subsection{Stability Notions}

An autonomous nonlinear system represented by the following dynamic equation
\begin{align*}
{\dot {x}}=f(x(t)),\;\;\;\;x(0)=x_{0}    
\end{align*}

where ${\displaystyle x(t)\in  \mathbb {R}^n}$ represents the system state vector is said to be stable, the function $f$ attains an equilibrium at the point $x = x_e$, i.e, $f(x_e) = 0$ and $ f'(x_e) = 0$.

The equilibrium point is said to be:
\begin{itemize}
    \item \textbf{Stable}, if there exists a bound $\epsilon > 0, \delta > 0$ such that $|x(0) - x_e| < \delta \implies |x(t) - x_e| < \epsilon, \quad \forall t \geq 0$. Stability makes sure the states starting from close enough to a bound $\delta$ will remain within the bound $\epsilon$.
    \item \textbf{Asymptotically stable}, if in addition to being stable, $\lim _{t\rightarrow \infty }\|x(t)-x_{e}\|=0$. Therefore, the system not only remains within the bound $\epsilon$ but also converges to the equilibrium point.
    \item \textbf{Exponentially stable}, if in addition with asymptotic stability there exists $ \alpha >0,\beta >0,\delta >0$ such that $\|x(t)-x_{e}\|\leq \alpha \|x(0)-x_{e}\|e^{-\beta t}, \quad \forall  t\geq 0$. Hence, system converges to the equilibrium at an exponential rate.
    \item \textbf{Uniformly Ultimately Bound (UUB):}
The solutions of $\dot{x} = f(x(t))$ is said to be uniformly ultimately bound with ultimate bound $b$ if $\exists b > 0,$ $c > 0$ and for every $0 < a < c,$ $\exists T = T(a,b) > 0$ such that
\begin{align}
    ||x(t_0)|| \leq a \implies  ||x(t)|| \leq b, \quad \forall t\geq t_o + T.
\end{align}
    
\end{itemize}

%The generic method to verify the stability is of a dynamic system is to form an energy function, $V(x)$ using the state variables and choosing a controller to ensure the following properties:
%\begin{itemize}
%    \item The function is positive definite, $V(x) > 0, \quad \forall x>0$.
%    \item The derivative of the function is negative definite, $\dot{V}(x) < 0, \quad \forall x>0$. This is required for asymptotic stability. For Lyapunov stability, the derivative can be negative semi-definite.
%    \item The function is unbounded, $V(x) \rightarrow \infty$ as $x \rightarrow \infty$. This is required for global stability of the system. If the system is bounded, then the stability can be achieved in a given locality, i.e., the system is locally stable.
%\end{itemize}



%\subsection{Sliding Mode Controller}
%Though PD controller is the simplest to implement, it is not preferable in autonomous mobile robots because it needs complete knowledge about the system dynamics. Any uncertainty in the dynamics or external disturbance will completely ruin the performance of the controller. In most cases, an Integral component is added to the PD controller to make it a PID controller. The integral term suppresses all the steady-state offsets, but it performs poorly if the offset is not static.

%In many applications of autonomous vehicles, such as transportation of multiple cargoes, travelling over different surfaces etc., the variation in dynamics is not static. Hence, it needs a nonlinear controller to take care of the uncertainties and disturbances. The simplest form of nonlinear control is the sliding mode controller.

%The sliding mode controller creates a surface in the phase-plane graph passing through the origin. Every other point in the phase-plane converges to the closest point in the sliding surface. Once the states reach the sliding surface, they tend to converge to the origin asymptotically.



%\begin{figure}
%	\centering
%	\includegraphics[width=0.6\textwidth]{LaTex/figures/sliding_mode_controller.jpg}%{FIG5_16-TIE-3042.eps}
%	\caption{{A phase portrait showing sliding surface and phase trajectory.}}\label{fig:sliding_surface}
%\end{figure}

%A Phase-plane graph or phase portrait is a two-dimensional graph with the axes being the values of two state variables. It is used to represent the vector field of the states graphically. In the phase portrait, a particular path taken by the state flow-line is called a phase path. In Fig $\ref{fig:sliding_surface}$, we can see a sample phase portrait with a straight line passing through the origin. This phase path is an example of a sliding surface. A sliding surface is a linear combination of two or more states present in the system. The phase path shows the points in state space, where the value of the sliding variable is zero. Since this phase path passes through the origin, the value of the sliding variable is zero when the value of both the states are zeros.

%A sliding mode controller ensures that the points away from the sliding surface converge to the sliding phase path and follow the path to the origin, where all the states are at zero. It uses discontinuous switching between positive and negative sides to maintain the state trajectory on the sliding surface. This discontinuity introduces oscillations in the system called chattering. The effect of chattering is reduced by using smoothing functions such as sigmoid or arctan.

%Typically, position error and its derivative would be chosen as the state variables. Since the velocity error becomes zero when position error is zero on the sliding surface, the states attain an equilibrium. The sliding mode controller is robust to uncertainties and disturbances in a given range. This property makes it ideal for the control of applications such as aerial robots.




\section{Contributions of the Thesis}
Two different problems were highlighted earlier while a quadrotor is carrying a payload of unknown mass and inertia. While suspended payloads retain the dynamic properties of the quadrotor, they cannot be used in indoor scenarios and safety-critical operations. So, the following contributions are made in this thesis to solve the dynamic problems in quadrotors with rigidly attached payloads for their respective applications:
\begin{itemize}
    \item Based on the partly decoupled six degrees-of-freedom dynamic model, an adaptive sliding mode controller is proposed that tackles variation in quadrotor dynamics caused by uncertainties while carrying payloads of unknown mass and inertia, and external disturbances. The proposed controller does not require any a priori knowledge of the system dynamics and of uncertainty bound. The closed-loop system stability is analysed via the Lyapunov-based method. The performance of the proposed scheme is verified using a simulated quadrotor on the Gazebo framework and in comparison with the state-of-the-art controllers.
    \item In the previous case, the case of payload variation during the flight (e.g., payload drop during construction, disaster management scenario etc.) was not considered. Such in-flight payload variations give rise to the switched dynamics in the quadrotor system. A suitable switched adaptive controller, which does not require any a priori knowledge of the system dynamics and of uncertainties, is introduced for tracking any arbitrary altitude trajectory of the quadrotor. The closed-loop stability is analysed using multiple Lyapunov functions, and the effectiveness of the proposed controller is verified in simulation.
\end{itemize}


\section{Organization of the Thesis}
The thesis is organized into four chapters. A brief summary of each chapter is mentioned below.

\begin{itemize}
    \item \textbf{Chapter 1:} This introductory chapter gives an overview of aerial robotics, modelling the dynamics of a quadrotor and state-of-the-art control strategies. It briefly describes the motivation for this research, the problem orientation, the pertaining gaps in the literature, the main contributions and an outline of the thesis.
    \item \textbf{Chapter 2:} This chapter introduces the adaptive sliding mode controller for a quadrotor to address the problems involved while carrying payloads of unknown masses. The quadrotor dynamics are partly decoupled into two loops, with position tracking as the outer loop and attitude tracking as the inner loop. The uncertainties in both the loops are modelled and tackled using the adaptive sliding mode controller strategies. The stability of the controller is proved using the Lyapunov stability method and verified using a simulated quadrotor. The results are compared with the state-of-the-art controllers to show the superiority in the performance.
    \item \textbf{Chapter 3:} This chapter introduces the fundamentals of the switched mode controller for in-flight switched dynamics stemming from the addition and removal of unknown payloads. The quadrotor is modelled using hovering dynamics with control only on roll, pitch, yaw and altitude. An swtiched adaptive controller is designed to tackle the uncertainty. The closed-loop system is proved using dwell-time based multiple Lyapunov method and verified via simulation settings.
    \item \textbf{Chapter 4:} This chapter concludes the thesis by summarizing the various contributions brought out by this thesis. It also provides a brief discussion about the future work in this line of research.
    
\end{itemize}

\section{Symbols and Notations:}
The symbols and notations used in the following chapters are as shown in Table \ref{tab:notations}
\begin{table}[ht!]
    \centering
    \begin{tabular}{l l l}
        $\mathbb{R}$ & Real line  \\
        $\mathbb{R}^+$ & Real line of positive numbers \\
        $\mathbb{R}^n$ & Real space of dimension $n$ \\
        $\mathbb{R}^n$ & Real matrix of dimension $n \times n$ \\
        $\exists$ & there exists \\
        $\forall$ & for all \\
        $\mathbf{I}$ & Identity matrix \\ 
        $\mathbf{\Xi} > 0 (<0)$ & Positive (negative) definite matrix \\
        $\lambda_{\min}(\mathbf{\Xi})$  & Minimum eigen value of the matrix $\mathbf{\Xi}$ \\ 
        $\lambda_{\max}(\mathbf{\Xi})$  & Maximum eigen value of the matrix $\mathbf{\Xi}$ \\
        $||\mathbf{\Xi}||$ & Euclidean norm of the matrix $\mathbf{\Xi}$ \\
        $\sgn(x)$ & Signum of $x$ = $x/||x||$\\
        $\sat(x, \varpi)$ & Saturation of $x$ = 
        $\begin{cases}
            x/||x||, ~ if ~ ||x|| \geq \varpi \\
            x/ \varpi, ~ otherwise \\
        \end{cases}$
        
    \end{tabular}
    \caption{Nomenclature of various symbols and notations used in the thesis.}
    \label{tab:notations}
\end{table}