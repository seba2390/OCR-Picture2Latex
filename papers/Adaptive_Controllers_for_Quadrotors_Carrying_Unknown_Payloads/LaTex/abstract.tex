With the advent of intelligent transport, quadrotors are becoming an attractive solution while lifting or dropping payloads during emergency evacuations, construction works, etc. During such operations, dynamic variations in (possibly unknown) payload cause considerable changes in the system dynamics. However, a systematic control solution to tackle such varying dynamical behaviour is still missing. In this work, two control solutions are proposed to solve two specific problems in aerial transportation of payload, as mentioned below.

In the first work, we explore the tracking control problem for a six degrees-of-freedom quadrotor carrying different unknown payloads. Due to the state-dependent nature of the uncertainties caused by variation in the dynamics, the state-of-the-art adaptive control solutions would be ineffective against these uncertainties that can be completely unknown and possibly unbounded a priori. In addition, external disturbances such as wind while following a trajectory in all three positions and attitude angles. Hence, an adaptive control solution for quadrotors is proposed, which does not require any a priori knowledge of the parameters of quadrotor dynamics and external disturbances.  

The second work is focused on the interchanging dynamic behaviour of a quadrotor while loading and unloading different payloads during vertical operations. %A quadrotor is modelled by decoupling the linear and angular dynamics.
The control problem to maintain the desired altitude is formulated via a switched dynamical framework to capture the interchanging dynamics of the quadrotor during such vertical operations, and a robust adaptive control solution is proposed to tackle such dynamics when it is unknown. 


The stability of the closed-loop system employing both the proposed controllers are studied analytically via Lyapunov theory. Further, the effectiveness of the proposed solutions is verified by deploying the controller on a realistic simulator under various scenarios and by testing the performance by comparing them with state-of-the-art controllers.

%The stability of the closed-loop system is studied analytically and the effectiveness of the proposed solution is verified via simulations.