\section{Introduction}
In the quest to operate a quadrotor under various sources of uncertainties, the control community has inevitably looked into robust control \cite{xu2008sliding, sanchez2012continuous, derafa2012super, madani2007sliding} and adaptive control methods \cite{nicol2011robust, bialy2013lyapunov, dydek2012adaptive, ha2014passivity, mofid2018adaptive, tran2018adaptive, tian2019adaptive, zhao2014nonlinear, yang2019energy}. However, robust control methods \cite{xu2008sliding, sanchez2012continuous, derafa2012super, madani2007sliding} rely on a priori knowledge of bounds on uncertainties; on the other hand, the adaptive control designs \cite{nicol2011robust, bialy2013lyapunov, dydek2012adaptive, ha2014passivity, tran2018adaptive, tian2019adaptive, zhao2014nonlinear, yang2019energy} require a priori knowledge of system structures. Such constraints are often difficult to be satisfied in practice due to unknown parametric variations and external disturbances. 

To avoid a priori knowledge of bounds on uncertainties, researchers have recently applied adaptive sliding mode designs \cite{mofid2018adaptive, shtessel2012novel, utkin2013adaptive, obeid2018barrier} considering uncertainties to be a priori bounded. Nevertheless, when such assumption is not satisfied (e.g. state-dependent uncertainty), instability cannot be ruled out (cf. \cite{roy2020adaptive, roy2020towards}): in the attitude dynamics of a quadrotor, state-dependent uncertainty naturally occurs owing to the Coriolis terms.

 In view of the above discussions, an adaptive control solution for the quadrotor is still missing in the presence of \textit{completely unknown state-dependent uncertainty and external disturbances}.
Toward this direction, the proposed adaptive solution has the following major contributions:

\begin{itemize}
    \item An adaptive controller for quadrotor is formulated, which does not require any a priori knowledge of the system dynamic paramters, payload and of disturbances. 
    \item Differently from \cite{nicol2011robust, bialy2013lyapunov, dydek2012adaptive, ha2014passivity, mofid2018adaptive}, the control framework considers six degrees-of-freedom (DoF) dynamics with unknown external perturbations affecting both actuated and non-actuated sub-dynamics.
    \item The closed-loop stability of the system is analysed via Lyapunov-based method and comparative simulation results suggest significant improvement in tracking accuracy of the proposed scheme compared to the state of the art.
\end{itemize}
%It is worth mentioning here that the present work is different from our previous works \cite{roy2020adaptive, roy2020towards} on the following directions: while \cite{roy2020adaptive} is designed for fully actuated systems, the present being developed for underactuated quadrotor system; further, compared to the adaptive design for a class of underactuated systems as in \cite{ roy2020towards}, the present work does not require any knowledge of mass/inertia matrix.

The rest of the chapter is organised as follows: Section \ref{sec:asmc_dyn} describes the dynamics of quadrotor; Section \ref{sec:asmc_cont} details the proposed control framework, while corresponding stability analysis is provided in Section \ref{sec:asmc_stab}; comparative simulation results are provided in Section \ref{sec:asmc_sim}.


\section{Quadrotor System Dynamics and Problem Formulation}\label{sec:asmc_dyn}
\begin{figure}[!h]
    \includegraphics[width=2.0 in, height=2in]{LaTex/figures/drone_model_BodyWorld.png}
    \centering
    \caption{Schematic of quadrotor with coordinate frames}
    \label{fig:quad_axis}
\end{figure}
	Let us consider the Euler-Lagrange system dynamics of a quadrotor model (cf. Fig. \ref{fig:quad_axis}), widely used in literature (\cite{bialy2013lyapunov})
	\begin{align}	
		m \ddot{p} + G + d_p &= \tau_p, \label{p_tau} \\
		J (q)\ddot{q} + C_q(q,\dot{q})\dot{q} + d_q &= \tau_q, \label{q_tau} \\
		\tau_p &= R_B^WU, \label{tau_conv}
	\end{align}
	where (\ref{p_tau}) and (\ref{q_tau}) are the position and attitude dynamics of the quadrotor, respectively. Various symbols in the dynamics (\ref{p_tau}) and (\ref{q_tau}) are described as follows: $m \in \mathbb{R}^{+}$ and $J(q) \in \mathbb{R}^{3 \times 3}$ represent the mass and inertia matrix, respectively; $p(t)\triangleq
	\begin{bmatrix}
	x(t) & y(t) & z(t)
	\end{bmatrix}^T \in\mathbb{R}^3$ is the position vector of the center of mass of the quadrotor; $q(t)\triangleq \begin{bmatrix}
	\phi(t) & \theta(t) & \psi(t) 
	\end{bmatrix}^T\in \mathbb{R}^3$ is the orientation/attitude (roll, pitch, yaw angles respectively); $G\triangleq
	\begin{bmatrix}
	0 & 0 & mg
	\end{bmatrix}^T\in \mathbb{R}^3$ is the gravitational force; $C_q(q, \dot{q}) \in \mathbb{R}^{3 \times3}$ represents the Coriolis matrix; $d_p(t)$ and $d_q(t)$ denote the unknown disturbances in position and attitude dynamics, respectively; $\tau_q \triangleq 
	\begin{bmatrix}
	u_2(t) & u_3(t) & u_4(t)
	\end{bmatrix}^T\in\mathbb{R}^3$ denotes the control inputs for roll, pitch and yaw; $\tau_p(t) \in \mathbb{R}^3$ is the generalized control input for position tracking in Earth-fixed frame, with $U(t)\triangleq
	\begin{bmatrix}
	0 & 0 & u_1(t)
	\end{bmatrix}^T\in \mathbb{R}^3$ being the force vector in body-fixed frame and $R_B^W \in\mathbb{R}^{3\times3}$ being the $Z-Y-X$ Euler angle rotation matrix describing the rotation from the body-fixed coordinate frame to the Earth-fixed frame, given by
	\begin{align}
	R_B^W =
	\begin{bmatrix}
	c_{\psi}c_{\theta} & c_{\psi}s_{\theta}s_{\phi} - s_{\psi}c_{\phi} & c_{\psi}s_{\theta}c_{\phi} + s_{\psi}s_{\phi} \\
	s_{\psi}c_{\theta} & s_{\psi}s_{\theta}s_{\phi} + c_{\psi}c_{\phi} & s_{\psi}s_{\theta}c_{\phi} - c_{\psi}s_{\phi} \\
	-s_{\theta} & s_{\phi}c_{\theta} & c_{\theta}c_{\phi}
	\end{bmatrix}, \label{rot_matrix}
	\end{align}
	where $c_{(\cdot)} , s_{(\cdot)}$ and denote $\cos{(\cdot)} , \sin{(\cdot)}$ respectively. 
	
	As per the celebrated Euler-Lagrange dynamics, the attitude dynamics of quadrotor satisfy the following properties \cite{bialy2013lyapunov}:
	
\noindent\textbf{Property 1.} The inertia matrix $J (q)$ is uniformly positive definite $\forall q$ and there exist $\underline{j},\overline{j} \in{R}^{+}$ such that $ \underline{j}I \leq J(q) \leq \overline{j}I$. \\
\textbf{Property 2.} $\exists \overline{c}_q, \overline{d}_p, \overline{d}_q \in\mathbb{R}^{+}$ such that $||C_q(q,\dot{q})|| \leq \overline{c}_q||\dot{q}||$, $||d_q|| \leq \overline{d}_q$, $||d_p|| \leq \overline{d}_p$. \\ %{\color{red} $||G|| \leq m^*g$, where $m$ is the unknown total mass bounded by $0 \leq \underline{m} \leq m \leq \overline{m}$}.
\textbf{Property 3.} The matrix $(\dot{J} - 2C_q)$ is skew symmetric, i.e., for any non-zero vector $r$, we have $r^T(\dot{J} - 2C_q)r = 0$.

The following assumption highlights the available knowledge of various system parameters for control design:
\begin{assum}
The system dynamics terms $m,J,C_q,d_p,d_q$ and their bounds $\overline{c}_q, \overline{d}_p, \overline{d}_q$ are unknown for control design.
\end{assum}
\begin{remark}[Knowledge of system dynamics]
Assumption 1 is indeed a design challenge: the objective is to formulate a control framework without any knowledge of system parametric uncertainties and external disturbances. Assumption 1 overcomes the need for either precise knowledge of mass/inertia matrix (cf. \cite{xu2008sliding, sanchez2012continuous, derafa2012super, madani2007sliding}), or a priori knowledge of system structures and of bounds on external disturbances (cf. \cite{nicol2011robust, bialy2013lyapunov, dydek2012adaptive, ha2014passivity, tran2018adaptive, tian2019adaptive, zhao2014nonlinear, yang2019energy}).
\end{remark}
\begin{remark}[Position and attitude tracking co-design]\label{rem_co}
Tracking control problem of quadrotors can be broadly classified under two categories: (i) reduced-order model based design (cf. \cite{nicol2011robust, dydek2012adaptive, ha2014passivity, mofid2018adaptive}) and (ii) position and attitude tracking co-design (cf. \cite{bialy2013lyapunov, mellinger2011minimum}). Control designs relying on the first category ignore the non-actuated $(x,y)$ dynamics and define the control problem as tracking of only the actuated degrees-of-freedom (DoF) i.e. of $z$ and $(\phi, \theta ,\psi)$. Such approach is generally termed as `collocated design' in the literature of underactuated systems (cf. \cite{roy2020towards}). The second category relies on co-designing position and attitude tracking controller for the six DoF dynamics (\ref{p_tau})-(\ref{q_tau}). 
\end{remark}

In this work, we shall follow the co-design approach as the reduced-order based approach is conservative for relying on a priori boundedness of $(x,y)$ dynamics. Therefore, we take the following standard assumption:
\begin{assum}[\cite{bialy2013lyapunov, mellinger2011minimum, tang2015mixed}]\label{assum_des}
Let $p_d (t) \triangleq
	\begin{bmatrix}
	x_d (t) & y_d (t) & z_d (t)
	\end{bmatrix}^T$ and $\psi _d(t) $ be the desired position and yaw trajectories to be tracked, which are designed to be sufficiently smooth and bounded. Further, it is assumed that the following inequality holds for all time: $ - \frac{\pi}{2} < (\phi_d, \theta_d) < \frac{\pi}{2}$ radians.
\end{assum}
\begin{remark}[Desired roll and pitch]
In position and attitude tracking co-design, the desired roll ($\phi_d$) and pitch $(\theta_d$) trajectories are derived based on the computed position control input $\tau_p$ and the desired yaw trajectory $\psi_d$ (cf. \cite{mellinger2011minimum}) and it is discussed later.
\end{remark}
\textbf{Control Problem:} Under Properties 1-3, to design an adaptive controller to track a desired trajectory (cf. Assumption 2) without any knowledge of system dynamics parameters (cf. Assumption 1).

The following section provides a solution to this control problem.
\section{Controller Co-design}\label{sec:asmc_cont}
%
     The position and attitude tracking co-design approach consists of simultaneous design of an outer loop controller for position dynamics (\ref{p_tau}) and of an inner loop controller for attitude dynamics (\ref{q_tau}). Following this approach, the proposed control solution is elaborated in the following subsections. % to thecontrol problem is solved
     
     \subsection{Outer Loop Controller}
Let us define the position tracking error as $e_p \triangleq p - p_d$ and
% \begin{align}
%     e_p &= p - p_d \label{p_err}
%\end{align} 
a sliding variable as
\begin{align}
     s_p &= \dot{e}_p + \Phi_p e_p \label{p_s}
 \end{align}
where $\Phi_{p}$ is a positive definite gain matrix. Multiplying the time derivative of (\ref{p_s}) by $m$ and using (\ref{p_tau}) yields
\begin{align}
     m\dot{s}_p = m(\ddot{p}-\ddot{p}_d + \Phi_p \dot{e}_p) = \tau_p + \varphi_p - G \label{dot_s_p}
\end{align}
where $\varphi_p
\triangleq -(d_p + m\ddot{p}_d - m\Phi_p\dot{e}_p)$. Using Property 2, we have
\begin{align}
 ||\varphi_p|| &\leq \overline{d}_p + {m}(||\ddot{p}_d|| + ||\Phi_p||  ||\dot{e}_p||).
\end{align}
Further, let us define $\xi_p \triangleq
\begin{bmatrix}
    e_p^T & \dot{e}_p^T
\end{bmatrix}^T$. Then using the inequality $||\xi_p|| \geq ||\dot{e}_p||$ and boundedness of the desired trajectories, we have
\begin{align}
    ||\varphi_p|| &\leq K_{p0}^* + K_{p1}^*||\xi_p||  \label{p_up_bound_K}
\end{align}
where $K_{p0}^* \triangleq {\overline{d}_p + m||\ddot{p}_d||}$ , $K_{p1}^* \triangleq m||\Phi_p||$ are \textit{unknown} finite scalars. The outer loop control law is designed as
\begin{subequations}\label{ip_out}
\begin{align}
    \tau_p (t) &= -\Lambda_p s_p (t) - \rho_p (t) \sgn(s_p(t)) + \hat{m}(t) g_p \label{p_con_tau}\\
    \rho_p(t) &= \hat{K}_{p0}(t) + \hat{K}_{p1}||\xi_p|| \label{p_con_rho}
\end{align}
\end{subequations}
where $\Lambda_p$ is a positive definite user-defined gain matrix; $g_p =  
    \begin{bmatrix}
        0 & 0 & g
    \end{bmatrix}^T$ is the gravity vector; $\hat{K}_{pi}$ and $\hat{m}$ are the estimates of $K_{pi}^*$ and ${m}$, $i=0,1$ respectively and they are evaluated via the following adaptive laws
    \begin{subequations}\label{adap_out}
\begin{align}
    \dot{\hat{K}}_{pi}(t) &= ||s_p(t)|| ||\xi_p(t)||^i - \alpha_{pi} \hat{K}_{pi}(t), ~\hat{K}_{pi}(0) > 0  \label{p_adap_K} \\
    \dot{\hat{m}}(t) &= - s_p(t)^T g_p - \alpha_m \hat{m}(t), ~ \hat{m}(0) > 0\label{p_adap_m}
\end{align}
\end{subequations}
where $\alpha_{pi}, \alpha_m  \in \mathbb{R}^{+}$ are user-defined design scalars. Note that eventually $U$ is applied to the system by transforming $\tau_p$ via the relation (\ref{tau_conv}) ($R^W_B$ is invertible rotational matrix), giving non-zero input only in $z$ direction.

\subsection{Inner Loop Controller}
To realize the inner loop controller, it is important to generate the desired roll ($\phi_d$) and pitch ($\theta_d$) angles. This process involves defining an intermediate coordinate frame as the first step (cf. \cite{mellinger2011minimum}):
\begin{subequations}\label{int_co}
\begin{align}
    z_B &= \frac{\tau_p}{||\tau_p||} \\
    y_A &= \begin{bmatrix}
    -s_{\psi_d} & c_{\psi_d} & 0
\end{bmatrix}^T \\
    x_B &= \frac{y_A \times z_B}{||y_A \times z_B||} \\
    y_B &= z_B \times x_B
\end{align}
\end{subequations}
where $y_A$ is the $y$-axis of the intermediate coordinate frame $A$; $x_B$, $y_B$ and $z_B$ are the $x$-axis, $y$-axis and $z$-axis of the body fixed coordinate frame. Given the desired yaw angle $\psi_d (t)$ and based on the computed intermediate axes as in (\ref{int_co}), $\phi_d (t)$ and $\theta_d (t)$ can be determined using the desired body frame axes as described in \cite{mellinger2011minimum}. 

Further, to achieve the attitude tracking control objective, the error in orientation/attitude is defined as \cite{mellinger2011minimum}
\begin{align}
    e_q &= {((R_d)^T R_B^W - (R_B^W)^T R_d)}^{v} \label{q_err} \\
    \dot{e}_q & = \dot{q} - R_d^T R_B^W \dot{q}_d
\end{align}
where $(.)^v$ represents \textit{vee} map, which converts elements of $SO(3)$ to $\in{\mathbb{R}^3}$ \cite{mellinger2011minimum} and $R_d$ is the rotation matrix as in (\ref{rot_matrix}) evaluated at ($\phi_d, \theta_d, \psi_d$).

The sliding variable $s_q \in{\mathbb{R}^3}$ for inner loop control is defined as
\begin{align}
     s_q = \dot{e}_q + \Phi_q e_q \label{q_s}
\end{align}
where $\Phi_{q}$ is a positive definite gain matrix. Multiplying the derivative of (\ref{q_s}) by $J$ and using (\ref{q_tau}) yield
\begin{align}
     J\dot{s}_q = J(\ddot{q}-\ddot{q}_d + \Phi_q{\dot{e}_q}) = \tau_q - C_q s_q + \varphi_q \label{dot_s_q}
 \end{align}
 where $\varphi_q 
 \triangleq -(C_q \dot{q} + d_q + J\ddot{q}_d - J\Phi_q\dot{e}_q - C_q s_q)$ represents the overall uncertainties in attitude dynamics. Using (28) and Properties 1 and 2 we have
 \begin{align}
     ||\varphi_{q}|| &\leq \overline{c}_q||\dot{q}||^2 + \overline{d_{q}} + \overline{j}(||\ddot{q}_d|| + ||\Phi_{q}|| ||\dot{e}_{q}||)  + \overline{c}_q||\dot{q}||(||\dot{e}_{q}|| + ||\Phi_{q}||||q||). \label{q_phi_bound}
 \end{align}
 Further, let us define $\xi_q \triangleq
	\begin{bmatrix}
	e_q^T & \dot{e}_q^T
	\end{bmatrix}^T$. Then using the inequalities $||\xi_q|| \geq ||e_q||$, $||\xi_q|| \geq ||\dot{e}_q||$, boundedness of the desired trajectories, and substituting $\dot{q} = \dot{e}_q + \dot{q}_d$ in $(\ref{q_phi_bound})$ yield
\begin{align}
    ||\varphi_{q}|| &\leq K_{q0}^* + K_{q1}^*||\xi_{q}|| + K_{q2}^*||\xi_{q}||^2, \label{q_up_bound_K}
\end{align}
where $K_{q0}^* \triangleq \overline{c}_q||\dot{q}_d||^2 + \overline{d}_q + \overline{j}||\ddot{q}_d||$, $K_{q1}^* \triangleq \overline{c}_q||\dot{q}_d||(3 + ||\Phi_q||) + \overline{j}||\Phi_q||$, $K_{q2}^* \triangleq {\overline{c}_q||\dot{q}_d||(2 + ||\Phi_q||)}$ are \textit{unknown} finite scalars. Hence, a state-dependent upper bound occurs naturally to the attitude tracking system. So, the inner loop control law is designed as
\begin{subequations}\label{ip_inn}
\begin{align}
	    \tau_q (t) &= -\Lambda_q s_q (t) - \rho_q (t) \sgn(s_q(t)), \label{q_con_tau} \\
	    \rho_{q}(t) &= \hat{K}_{q0}(t) + \hat{K}_{q1}(t)||\xi_q (t)|| + \hat{K}_{q2}(t)||\xi_q(t)||^2, \label{q_con_rho}
\end{align}
\end{subequations}
where $\Lambda_q$ is a positive definite user-defined gain matrix and $\hat{K}_{qi}$, $i=0,1,2$ are the estimates of ${K}_{qi}^*$ adapted via the following law:
\begin{align}
    \dot{\hat{K}}_{qi}(t) = ||s_q(t)||||\xi_q||^i - \alpha_{qi} \hat{K}_{qi}(t),~\hat{K}_{qi} (0) > 0, \label{q_adap_K}
\end{align}
%with \begin{align}
%     ~\alpha_{qi} > 0~ \label{q_init_K}
%\end{align}
where $\alpha_{qi} \in\mathbb{R}^{+}$, $i=0,1,2$ are user-defined scalars.
\begin{remark}[On state-dependent uncertainty] The inequalities (\ref{p_up_bound_K}) and (\ref{q_up_bound_K}) reveal that the state-dependencies occur inherently in the upper bound structures of uncertainties via $\xi_p$ and $\xi_q$. Therefore, conventional adaptive sliding mode designs such as \cite{mofid2018adaptive, shtessel2012novel, utkin2013adaptive, obeid2018barrier} are not feasible as state-dependent uncertainty cannot be bounded a priori. On the other hand, the gains $\rho_p$ in (\ref{p_con_rho}) and $\rho_q$ in (\ref{q_con_rho}) are designed according to the state-dependent uncertainty structures (\ref{p_up_bound_K}) and (\ref{q_up_bound_K}) respectively.
\end{remark}
\begin{remark}[On co-design of outer and inner loop]
Note that the outer and inner loop designs are not fully decoupled process: rather, the inner loop design process relies on $\tau_p$ from the outer loop. Therefore, these loop designs are coupled and simultaneously affect the quadrotor system: this makes the proposed design a more realistic one compared to decoupled dynamics based approaches (cf. \cite{nicol2011robust, dydek2012adaptive, ha2014passivity, mofid2018adaptive}) or autopilot based designs where often linearized or decouple dynamics are considered (cf. \cite{fari2019addressing,yang2019software}).
\end{remark}
\section{Stability Analysis of The Proposed Controller}\label{sec:asmc_stab}
\begin{theorem}
Under Properties 1-3 and Assumptions 1-2, the trajectories of the closed-loop systems (\ref{dot_s_p}) and (\ref{dot_s_q}) using the control laws (\ref{ip_out}) and (\ref{ip_inn}), in conjunction with the adaptive laws (\ref{adap_out}) and (\ref{q_adap_K}) are Uniformly Ultimately Bounded (UUB).
\end{theorem}
\textit{Proof.}
	Note that the solutions of the adaptive gains $( \ref{p_adap_K} )$ and $(\ref{q_adap_K})$ are given by \cite{roy2020adaptive}
\begin{align}
\vspace{-0.8cm}
    \hat{K}_{pi}(t) &= \underbrace{\exp(-\alpha_{pi} t) \hat{K}_{pi} (0)}_{\geq 0} + \underbrace{\int_{0}^{t} \exp(-\alpha_{pi}(t-\vartheta)) (|| {s_p} (\vartheta)|| ||\xi_p (\vartheta)||^{i}) \mathrm{d}\vartheta}_{\geq 0} \nonumber \\
    \Rightarrow~\hat{K}_{pi} (t) &\geq 0,~ i=0,1 ~\forall t\geq 0 \label{p_low_bound} \\
%    \hat{m} (t) &= \underbrace{exp(-\alpha_m)\hat{m} (0)}_{\geq 0} \nonumber\\
%    &+ \underbrace{\int_{0}^{t} \exp(-\alpha_{m}(t-\tau)) ({-||s_p||} (\tau) g) \mathrm{d}\tau}_{\geq 0}\\
%    \Rightarrow~\hat{m} (t) &\geq 0, \, \forall t \geq 0 \label{m_low_bound}\\
    \hat{K}_{qi}(t) &= \underbrace{\exp(-\alpha_{qi} t) \hat{K}_{qi} (0)}_{\geq 0}  + \underbrace{\int_{0}^{t} \exp(-\alpha_{qi}(t-\vartheta)) (|| {s_q} (\vartheta)|| ||\xi_q (\vartheta)||^{i}) \mathrm{d}\vartheta}_{\geq 0} \nonumber\\
    \Rightarrow~\hat{K}_{qi} (t) &\geq 0,~ i=0,1,2 ~\forall t\geq 0. \label{q_low_bound}
\end{align}
Closed-loop stability is analysed using the following Lyapunov function
\begin{align}
 V &= V_p + V_q, \label{overall_lyap}\\
\text{where}~    V_p &= \frac{1}{2}s_p^T m s_p + \frac{1}{2} \sum \limits_{i=0}^{1} (\hat{K}_{pi} - K_{pi}^*)^2 + \frac{1}{2} (\hat{m} - m)^2 \label{p_lyap}\\
    V_q &= \frac{1}{2}s_q^T J s_q + \frac{1}{2} \sum \limits_{i=0}^{2} (\hat{K}_{qi} - K_{qi}^*)^2 .\label{q_lyap}    
\vspace{-0.5cm}
\end{align}
Using $(\ref{dot_s_p})$ and $(\ref{dot_s_q})$, the time derivatives of $(\ref{p_lyap})$ and $(\ref{q_lyap})$ yield
\begin{align}
    \dot{V}_p &= s_p^T m \dot{s}_p + \sum \limits_{i=0}^{1} (\hat{K}_{pi} - K_{pi}^*) \dot{\hat{K}}_{pi} + (\hat{m} - m)\dot{\hat{m}} \nonumber \\
    &= s_p^T (\tau_p + \varphi_p + G )  + \sum \limits_{i=0}^{1} (\hat{K}_{pi} - K_{pi}^*) \dot{\hat{K}}_{pi} + (\hat{m} - m)\dot{\hat{m}} \label{p_lyap_dot_2} \\
    \dot{V}_q &= s_q^T J \dot{s}_q + \frac{1}{2}s_q^T \dot{J} s_q + \sum \limits_{i=0}^{2} (\hat{K}_{qi} - K_{qi}^*) \dot{\hat{K}}_{qi} \nonumber\\
    &= s_q^T (\tau_q - C_q s_q + \varphi_q) + \frac{1}{2}s_q^T \dot{J} s_q   + \sum \limits_{i=0}^{2} (\hat{K}_{qi} - K_{qi}^*) \dot{\hat{K}}_{qi}. \label{q_lyap_dot_2}
\end{align}
Utilizing the upper bound structure given by (\ref{p_up_bound_K}) and Property 2, and the control laws (\ref{ip_out}), (\ref{ip_inn}), the terms $\dot{V}_p$ in (\ref{p_lyap_dot_2}) and $\dot{V}_q$ in (\ref{q_lyap_dot_2}) are simplified to
\begin{align}
    \dot{V}_p &= s_p^T (- \Lambda_p s_p - \rho_p \sgn(s_p) + \hat{m} g_p + \varphi_p - G )  + \sum \limits_{i=0}^{1} (\hat{K}_{pi} - K_{pi}^*) \dot{\hat{K}}_{pi} + (\hat{m} - m)\dot{\hat{m}} \label{p_lyap_dot_3} \\
    \dot{V}_q &= s_q^T (- \Lambda_p s_p - \rho_p \sgn(s_p) + \varphi_q) + \frac{1}{2}s_q^T (\dot{J} - 2C_q) s_q + \sum \limits_{i=0}^{2} (\hat{K}_{qi} - K_{qi}^*) \dot{\hat{K}}_{qi}. \label{q_lyap_dot_3}
\end{align}
Property 3 implies that $s_q^T (J-2C_q) s_q = 0$. Then utilizing the upper bound structure (\ref{p_up_bound_K}) and (\ref{q_up_bound_K}), and the facts that $G=mg_p$, $\rho_p \geq 0, \rho_q \geq 0$ from (\ref{p_low_bound}) and (\ref{q_low_bound}) we have
\begin{align}
    \dot{V}_p &= - s_p^T \Lambda_p s_p - \sum \limits_{i=0}^{1} (\hat{K}_{pi} - K_{pi}^*) (||\xi_p||^i ||s_p|| - \dot{\hat{K}}_{pi}) + (\hat{m} - m)(s_p^Tg_p + \dot{\hat{m}}) \label{p_lyap_dot_4} \\ 
    \dot{V}_q &= - s_q^T \Lambda_q s_q - \sum \limits_{i=0}^{2} (\hat{K}_{qi} - K_{qi}^*) (||\xi_q||^i ||s_q|| -\dot{\hat{K}}_{qi}). \label{q_lyap_dot_4}
\end{align}
Using (\ref{adap_out}) and (\ref{q_adap_K}), we have
\begin{align}
    (\hat{K}_{pi} - K_{pi}^*) \dot{\hat{K}}_{pi} &= ||s_p|| (\hat{K}_{pi} - K_{pi}^*) ||\xi_p||^i  + \alpha_{pi} \hat{K}_{pi} K_{pi}^* - \alpha_{pi} \hat{K}_{pi}^2 \label{Kp_dot_exp}\\
    (\hat{m} - m) \dot{\hat{m}} &= - (\hat{m} - m) s_p^Tg_p  + \alpha_{m} \hat{m} m - \alpha_{m} \hat{m}^2 \label{m_dot_exp}\\
    (\hat{K}_{qi} - K_{qi}^*) \dot{\hat{K}}_{qi} &= ||s_q|| (\hat{K}_{qi} - K_{qi}^*) ||\xi_q||^i + \alpha_{qi} \hat{K}_{qi} K_{qi}^* - \alpha_{qi} \hat{K}_{qi}^2 \label{Kq_dot_exp}
\end{align}
Substituting (\ref{Kp_dot_exp})-(\ref{Kq_dot_exp}) into (\ref{p_lyap_dot_4}) and (\ref{q_lyap_dot_4}) yield
\begin{align}
    \dot{V}_p & \leq - \frac{\lambda_{\min}(\Lambda_p)||s_p||^2}{3}  + \sum \limits_{i=0}^{1} (\alpha_{pi} \hat{K}_{pi} K_{pi}^* - \alpha_{pi} \hat{K}_{pi}^2) + \alpha_{m} \hat{m} m - \alpha_{m} \hat{m}^2 \nonumber \\
    & \leq - \frac{\lambda_{\min}(\Lambda_p)||s_p||^2}{3}  - \sum \limits_{i=0}^{1} \left(\frac{\alpha_{pi}(\hat{K}_{pi} -  K_{pi}^*)^2}{2} - \frac{\alpha_{pi} {K_{pi}^*}^2}{2} \right)  \nonumber \\
    & \quad - \left(\frac{ \alpha_{m} (\hat{m} - m)^2}{2} - \frac{\alpha_{m} {m}^2}{2} \right) \label{p_lyap_dot_5} \\
    \dot{V}_q & \leq - \frac{\lambda_{\min}(\Lambda_q)||s_q||^2}{3} + \sum \limits_{i=0}^{2} (\alpha_{qi} \hat{K}_{qi} K_{qi}^* - \alpha_{qi} \hat{K}_{qi}^2) \nonumber \\
    & \leq - \frac{\lambda_{\min}(\Lambda_q)||s_q||^2}{3}  - \sum \limits_{i=0}^{2} \left(\frac{\alpha_{qi}(\hat{K}_{qi} -  K_{qi}^*)^2}{2} - \frac{\alpha_{qi} {K_{qi}^*}^2}{2} \right). \label{q_lyap_dot_5}
\end{align}
Further the definition of Lyapunov function yields
\begin{align}
    V_p & \leq \frac{{m}}{2} ||s_p||^2 + \frac{1}{2} \sum \limits_{i=0}^{1} (\hat{K}_{pi} - K_{pi}^*)^2 + \frac{1}{2} (\hat{m} - m)^2 \label{p_lyap_upbound} \\
    V_q & \leq \frac{\overline{j}}{2} ||s_q||^2 + \frac{1}{2} \sum \limits_{i=0}^{2} (\hat{K}_{qi} - K_{qi}^*)^2. \label{q_lyap_upbound}
\end{align}
From $(\ref{p_lyap_upbound})$ and $(\ref{q_lyap_upbound})$, the conditions $(\ref{p_lyap_dot_5})$ and $(\ref{q_lyap_dot_5})$ can be further simplified to
\begin{align}
    \dot{V}_p & \leq -\varrho_p V_p + \frac{1}{2}\left( \sum \limits_{i=0}^{1} \alpha_{pi} {K^*_{pi}}^2 + \alpha_{m} {m}^2  \right) \label{p_lyap_dot_6} \\
    \dot{V}_q & \leq -\varrho_q V_q + \frac{1}{2}\left( \sum \limits_{i=0}^{2} \alpha_{qi} {K^*_{qi}}^2 \right) \label{q_lyap_dot_6}
\end{align}
where $\varrho_p \triangleq  \frac{\min_i\{ \lambda_{\min}(\Lambda_p)/3, \alpha_{pi}, \alpha_{m} \}}{\max \{ {m}/2, 1/2 \}}$, $\varrho_q \triangleq  \frac{\min_i \lbrace \lambda_{min}(\Lambda_q)/3,  \alpha_{qi} \rbrace }{\max \lbrace \overline{j}/2, 1/2 \rbrace} > 0$ can be designed via (\ref{p_con_tau}), (\ref{adap_out}), (\ref{q_con_tau}), and (\ref{q_adap_K}). From (\ref{p_lyap_dot_6}) and (\ref{q_lyap_dot_6}), the upper bound for the time derivative of the overall Lyapunov function $\dot{V}$ can be obtained as
\begin{align}
    \dot{V} & \leq - \varrho V + \frac{1}{2} \left( \sum \limits_{i=0}^{1} \alpha_{pi} {K^*_{pi}}^2  \right) + \frac{1}{2} \left( \alpha_{m} {m}^2 \right)  + \frac{1}{2}\left( \sum \limits_{i=0}^{2} \alpha_{qi} {K^*_{qi}}^2 \right) \label{overall_V_dot_1}
\end{align}
where $\varrho = \min \{ \varrho_p, \, \varrho_q\}$. Defining a scalar $\kappa$ such that $0 < \kappa < \varrho$, $(\ref{overall_V_dot_1})$ is simplified to
\begin{align}
    \dot{V} &= - \kappa V - (\varrho - \kappa) V  + \frac{1}{2} \left( \sum \limits_{i=0}^{1} \alpha_{pi} {K^*_{pi}}^2  +  \alpha_{m} {m}^2 + \sum \limits_{i=0}^{2} \alpha_{qi} {K^*_{qi}}^2 \right). \label{overall_V_dot_upbound}
\end{align}
Defining a scalar $\mathcal{ \bar B} \triangleq \frac{\left( \sum \limits_{i=0}^{1} \alpha_{pi} {K^*_{pi}}^2  +  \alpha_{m} {m}^2 + \sum \limits_{i=0}^{2} \alpha_{qi} {K^*_{qi}}^2 \right)}{2(\varrho - \kappa)}$, it can be seen that $\dot{V} (t) < - \kappa V (t)$ when $V (t) \geq \mathcal{ \bar B}$, so that
\begin{align}
    V & \leq max \{ V(0), \mathcal{ \bar B} \}, \forall t \geq 0,
\end{align}
and the closed-loop system remains UUB.

\begin{remark}
 Control laws (\ref{p_con_tau}) and (\ref{q_con_tau}) are discontinuous in nature, which may cause chattering. Therefore, as standard for sliding mode designs, the control laws can be made continuous by replacing the `signum' function by a `saturation' function defined as $\sat(s_p,\varpi_p)=s_p/||s_p||$ (resp. $s_p/\varpi_p$) if $|| s_p || \geq \varpi_p$ (resp. $|| s_p || <\varpi_p$) where $\varpi_p \in \mathbb{R}^{+}$ is a small scalar used to avoid chattering. Similarly, $\sgn(s_q)$ can be modified. Such modifications do not change the closed-loop stability result albeit some minor modifications in stability analysis, which can be carried out in the same line as in \cite{roy2019overcoming} and thus, repetition is avoided.
\end{remark}
\section{Simulation Results and Analysis}\label{sec:asmc_sim}
The performance of the proposed controller is tested on a Gazebo simulation platform using the RotorS Simulator framework \cite{rotors_sim_2016} for ROS with the Pelican quadrotor model (mass of the quadrotor without any payload is set to be $2$ kg). The results of the proposed adaptive sliding mode controller (ASMC) is compared with the geometric controller \cite{lee2010geometric} (referred to as PD control hereafter) available in the RotorS framework, and sliding mode controller (SMC) \cite{xu2008sliding} (note that existing ASMC designs such as \cite{mofid2018adaptive} cannot be applied as, aside considering the reduced-order model, they are designed for a priori bounded uncertainty, which is not valid for a quadrotor).

The Gazebo model of the quadrotor can be actuated by commanding the angular velocities for the rotors. The angular velocities are calculated by solving the following relationship between thrust, moments and angular velocities \cite{mellinger2011minimum}:
    \begin{align}
        \begin{bmatrix}
            \omega_1\\ \omega_2 \\ \omega_3 \\ \omega_4
        \end{bmatrix} &= 
        K^{-1} 
        \begin{bmatrix}
            u_1 \\ u_2 \\ u_3 \\ u_4
        \end{bmatrix} \\
        K &= \begin{bmatrix}
            k_f & k_f & k_f & k_f\\ -l k_f & 0 & l k_f & 0 \\ 0 & -l k_f & 0 & l k_f \\ k_{\tau} & -k_{\tau} & k_{\tau} & -k_{\tau} 
        \end{bmatrix}
    \end{align}
    
where $k_f$ and $k_{\tau}$ are force and moment constants respectively of the motors; $l$ is the arm length of the quadrotor and $\omega_i$ is the angular velocity of $i^{th}$ motor.

To properly judge the performance of the proposed design, two different simulation scenarios are created in the following subsections. For both the scenarios, the control parameters of the proposed ASMC are selected to be:
$\Phi_p = diag \lbrace
    2.4, 2.4, 16 \rbrace$, $\Phi_q = diag \lbrace
    0.22 , 0.22 , 0.01 \rbrace$, $\Lambda_p = diag \lbrace
    1.2 , 1.2 , 0.8 \rbrace$, $\Lambda_q = diag \lbrace 
    4.5 , 4.5 , 3.5 \rbrace$, $\hat{K}_{p0}(0) = \hat{K}_{p1}(0) = 0.01$, $\hat{K}_{q0}(0) = \hat{K}_{q1}(0) = \hat{K}_{q2}(0) = 0.0001$, $\alpha_{p0} = \alpha_{p1} = 3$, $\alpha_{q0} = \alpha_{q1} = \alpha_{q2} = 50$, $\hat{m}(0) = 0.01$ and $\alpha_m = 0.1$, $\varpi_p=0.1$, $\varpi_q=1$. For a fair comparison, similar sliding surfaces are selected for SMC as in (\ref{p_s}) and (\ref{q_s}). The gains for geometric controller \cite{lee2010geometric} is selected to be same as in the RotorS package, which are optimized for the Pelican model. Initial position and attitude for the quadrotor are selected to be $x(0)=y(0)=0$, $z(0)=0.1$ and $\phi(0)=\theta(0)=\psi(0)=0$.

To perform the experiments, initial values of positions, angles and their derivatives are taken to be zero. The control parameters chosen for the geometric controller is their default parameters for Pelican drone, i.e., position gain $k_p = 4$, linear velocity gain $k_d = 2.7$; for pitch and roll, angular gain and angular rate gain are chosen $1$ and  $0.22$  respectively, and $0.035, 0.01$ are chosen as angular gain and angular rate gain for yaw, respectively. 

\subsection{Scenario 1: Aggressive Manoeuvre}
The objective of this scenario is to test the performance of various controllers when a quadrotor is to make aggressive manoeuvres. To this end, a set of way-points $(x_d, y_d, z_d)$ and $ \psi_d$ are commanded as input to the controllers (cf. Fig. \ref{fig:traj_exp1}). The list of waypoints provided is mentioned in the table \ref{tab:waypoints}. Once the last waypoint is reached, the quadrotor altitude is reduced to the height of the payload, and the payload is attached. Once again, the set of waypoints is given in the same time interval. These abrupt transitions between the positions demand aggressive manoeuvres for a fast response. For time duration $0\leq t<48$s, the set of waypoints are commanded without any payload attached to the quadrotor. Then, at $t=48$s, a payload of $0.5$ kg is added to the quadrotor, and the same set of waypoints are repeated.

\begin{table}[h!]
\renewcommand{\arraystretch}{1.1}
\caption{{List of waypoints for Scenario 1}}
\label{tab:waypoints}
		\centering
{
{	\begin{tabular}{c c c c c}
		\hline
		\hline
		
		 time (s) & $x$ (m) & $y$ (m) & $z$ (m)  & $\psi$ (deg) \\
		 \hline
		0.0 & 2.0 & 2.0  & 2.0 & 0.0 \\
		\hline
		6.0 & 0.0 & 0.0 & 3.0 & 0.0 \\
		\hline
		12.0 & 0.0 & 2.0 & 2.0 & 0.0 \\
		\hline
		18.0 & 2.0 & 0.0 & 2.0 & 90.0 \\
		\hline
		24.0 & 2.0 & 2.0 & 4.0 & 0.0 \\
		\hline
		\hline
\end{tabular}}}
\end{table}

Figures \ref{fig:traj_exp1}-\ref{fig:error_att_exp1} depict the position tracking with various controllers and the corresponding errors incurred by them in position and attitude. It can be noted at $t\geq 72$s, the altitude ($z$ position) of the quadrotor using the geometric controller (PD) has become zero, i.e., the quadrotor has crashed on the floor during the manoeuvres after adding the payload. Reduction in the $z$ position error for the PD controller at $t>82$s should not be confused to be performance improvement: it is a result of reduction in the desired altitude $z_d$ (cf. Fig. \ref{fig:traj_exp1}), whereas the quadrotor still remains crashed on the ground. 

To demonstrate the performance of the controllers while carrying a payload, a performance comparison via root-mean-squared (RMS) error between SMC and ASMC is provided in Table \ref{table 1} after the payload is attached, i.e., for $t \geq 48$s (since the quadrotor crashed during the scenario for the geometric controller, its RMS error data is not tabulated). It clearly demonstrates the superior position tracking performance of the proposed scheme, where it has delivered a minimum performance improvement of $14.5 \%$. Despite position tracking is of more importance for transportation, it is also worth mentioning that attitude tracking of both the controllers is, however, almost similar. This is caused by the higher transients for ASMC compared to SMC during the sharp changes in the desired trajectory. This happens because the gains of ASMC need to re-adapt itself every time with the sharp changes in trajectories, while gains for SMC are fixed and less sensitive to abrupt changes. These transients are inherent for any adaptive design, but it helps them to adjust with unknown changes, while fixed-gain designs can fail when uncertainties lie beyond the a priori knowledge: such a situation is studied in the next scenario.  

% ($14.9 \%$) , 24.6% 23.6% 
 %various position Though the sliding mode controller is stable, it also shows a drop in the altitude after the payload is loaded, whereas the altitude trajectory followed by the drone with the proposed controller so similar for both loaded and unloaded cases.

\begin{figure}[t!]
    \includegraphics[width=\textwidth]{LaTex/figures/traj_exp1.png}
    \centering
    \caption{Position tracking comparison for aggressive manoeuvre}
    \label{fig:traj_exp1}
\end{figure}
\begin{figure}[t!]
    \includegraphics[width=\textwidth]{LaTex/figures/error_exp1.png}
    \centering
    \caption{Position tracking error comparison for aggressive manoeuvre}
    \label{fig:error_exp1}
\end{figure}
\begin{figure}[t!]
    \includegraphics[width=\textwidth]{LaTex/figures/error_att_exp1.png}
    \centering
    \caption{Attitude tracking error comparison for aggressive manoeuvre}
    \label{fig:error_att_exp1}
\end{figure}

Table shows the Root Mean Squared Error in the states $(x,y,z)$ and the overall position for the controllers. It can be observed that the proposed controller performs better than the other controllers.

\begin{table}[h!]
\renewcommand{\arraystretch}{1.1}
\caption{{Performance comparison for Scenario 1}}
\label{table 1}
		\centering
{
{	\begin{tabular}{c c c c c c c}
		\hline
		\hline
		& \multicolumn{3}{c}{RMS error (m)} & \multicolumn{3}{c}{RMS error (degree)}  \\ \cline{1-7}
		 position & $x$ & $y$  & $z$  & $\phi$ & $\theta$  & $\psi$  \\
		 \hline
		SMC & 0.47& 0.69  & 0.65 & 3.70 & 4.06  & 9.90 \\
		\hline
		ASMC (proposed)& {0.40} & {0.48}  & {0.42} & 3.80 & 2.67  & 10.06 \\
		\hline
		\hline
\end{tabular}}}
\end{table}
\clearpage

\subsection{Scenario 2: Dynamic payload}
\begin{figure*}
    \includegraphics[width=\textwidth]{LaTex/figures/scenario_2.jpg}
    \centering
    \caption{Snapshots from experimental scenario 2 (with ASMC): quadrotor (a) starting from initial position (b) picking up first payload ($0.4$ kg) (c) moving over a wall with first payload (d) dropping the first payload (e) picking up second payload ($1$ kg) (f) moving over the wall with second payload (g) dropping the second payload at initial position.}
    \label{fig:scn_exp2}
\end{figure*}

In a real-life aerial transport scenario, payloads can vary according to application requirements (e.g., human evacuations in an emergency). Accordingly, a second experimental scenario is created in the `indoor environment' available in the RotorS simulator, wherein performance of the various controllers under dynamic payload variation is tested while following a minimum snap trajectory \cite{mellinger2011minimum}. In this scenario, the quadrotor is required to perform the following actions sequentially (cf. Fig. \ref{fig:scn_exp2}):
\begin{itemize}
    \item From an initial position the quadrotor (unloaded) moves to the first payload location;
    \item it picks up a payload of $0.4$ kg (at approx $t=32$s) and an altitude of 1.0 m is given as the setpoint;
    \item it moves over a wall to the other side;
    \item it delivers the payload (at approx $t=58$s) and holds its altitude of 0.6 m;
    \item then, it moves towards a second payload of $1.0$ kg;
    \item it picks up the payload (at approx $t=83$s) and holds an altitude of 1.0 m;
    \item then, it moves over the wall back to the initial position;
    \item it delivers the payload (at approx $t=123$s) and holds its altitude of 0.5 m.
\end{itemize}
Moving over the walls with payload is an important phase for this experiment as it tests the capability of the controllers to maintain a designated altitude: such test cases are important for aerial transport in urban scenarios.
Masses of the first and second payload are chosen to be $0.4$ kg and $1.0$ kg. Figures \ref{fig:traj_exp2}, \ref{fig:error_exp2} show the states $(x, y, z)$ for different controllers with respect to time along with the commanded trajectory and their tracking errors respectively. The first payload is loaded at $t=32$s and unloaded at $t=58$s, while the second payload is loaded at $t=83$s and dropped at $t=123$s.

\begin{table}[!h]
\renewcommand{\arraystretch}{1.1}
\caption{{Performance of ASMC for Scenario 2}}
\label{table 2}
		\centering
{
{	\begin{tabular}{c c c c c c c}
		\hline
		\hline
		& \multicolumn{3}{c}{RMS error (m)} & \multicolumn{3}{c}{RMS error (degree)}  \\ \cline{1-7}
		 position & $x$ & $y$  & $z$  & $\phi$ & $\theta$  & $\psi$  \\
%		 \hline
%		SMC & 3.953& 0.735  & 0.567 & TBD & TBD  & TBD \\
		\hline
	& {0.159} & {0.162}  & {0.159} & 1.278 & 1.072  & 0.311 \\
		\hline
		\hline
\end{tabular}}}
\end{table}

Figures \ref{fig:traj_exp2}-\ref{fig:error_att_exp2} show the position responses with various controllers and their corresponding tracking errors. The position responses in the geometric controller become constant for $t>45$s, the quadrotor topples while moving with the first payload. A similar incident can be observed with SMC for $t> 92 $s: though the quadrotor could manage to lift the first payload, it could not lift the second payload high enough to cross the wall. This happened because SMC relies on a priori fixed gains based on predefined uncertainty bounds; when the payload becomes heavier, the controller cannot adjust its gain to adapt with higher uncertainty. In such a scenario, thanks to the adaptive gains as in (\ref{adap_out}) and (\ref{q_adap_K}), the proposed ASMC can tackle uncertainties without any a priori knowledge and thus, outperforms other controllers. The performance of ASMC is compiled in Table \ref{table 2}. The desired trajectories in this scenario being a smoother one (cf. \cite{mellinger2011minimum}) compared to the one in Scenario 1, transients in the attitude response for ASMC are visibly absent except at the instants when payloads are added. Similar to Scenario 1, since the quadrotor crashed during the scenario for SMC and PD, their performances are not tabulated.



\begin{figure}
    \includegraphics[width=\textwidth]{LaTex/figures/traj_exp2.png}
    \centering
    \caption{Position tracking by various controllers with variable payloads}
    \label{fig:traj_exp2}
\end{figure}
\begin{figure}
    \includegraphics[width=\textwidth]{LaTex/figures/error_exp2.png}
    \centering
    \caption{Position Tracking error comparison with variable payloads}
    \label{fig:error_exp2}
\end{figure}
\begin{figure}
    \includegraphics[width=\textwidth]{LaTex/figures/error_att_exp2.png}
    \centering
    \caption{Attitude tracking error comparison with variable payloads}
    \label{fig:error_att_exp2}
\end{figure}

%\begin{table}[!t]
%\renewcommand{\arraystretch}{1.1}
%\caption{{Performance of ASMC for Scenario 2}}
%\label{table 2}
%\centering
%{
%{	\begin{tabular}{c c c c c c c}
%		\hline
%		\hline
%		& \multicolumn{3}{c}{RMS error (m)} & \multicolumn{3}{c}{RMS error (degree)}  \\ \cline{1-7}
%		& $x$ & $y$  & $z$  & $\phi$ & $\theta$  & $\psi$  \\
%		\hline
%		& 0.160& 0.168  & 0.161 & TBD & TBD  & TBD \\
%		\hline
%		\hline
%\end{tabular}}}
%\end{table}

