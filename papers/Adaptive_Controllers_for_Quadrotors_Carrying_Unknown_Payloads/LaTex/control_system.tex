
A control system is used to establish a bridge between what we have (current outputs) and what we need (setpoints or desired outputs). It is mainly classified into two types: open-loop and closed-loop. An open-loop control system is a direct mapping between expected outputs and desired control inputs. This mapping is defined as the dynamics of the robotic system or commonly mentioned as a plant.

A feedback element is added in a closed-loop control system to compensate for any error in the mapping. The block diagram Fig: \ref{fig:closed_loop} represents a closed-loop control system. As mentioned earlier, the plant represents the dynamics of the robot. It consists of one or more actuators to perform a given task. A controller receives the error signal, which is the difference between the expected output and the obtained output. Based on the error signal, the controller gives out a control signal (or the actuator signal) to set the actuators to a particular value. A feedback element is a set of sensors that measure the current output and send the feedback signal to the error block.


\begin{figure}
	\centering
	\includegraphics[width=\textwidth]{LaTex/figures/control_system.jpg}%{FIG5_16-TIE-3042.eps}
	\caption{{A schematic representing a closed-loop control system.}}\label{fig:closed_loop}
\end{figure}


Conventionally the controller has to be aware of the dynamics of the robotic system to generate an appropriate actuator signal for a given setpoint. Hence the dynamics of the robot has to be modelled mathematically using forces and torques. The commonly used way of modelling a mechanical system (robot) is using Euler-Lagrangian dynamics.