\section{Details of velocity mapping}
\label{sec:details}

\subsection{Accounting for PSF scattering}
\label{sec:xcm}

We now describe how we accounted for PSF scattering in section~\ref{sec:velocity} in further detail. In the presence of steep X-ray surface brightness gradients, such as those in the cluster cool cores, the X-ray mirror PSF with a sharp core and broad wings \citep{okajima16} can cause significant flux contamination from the bright cluster peak into the lower-brightness regions at distances much greater than the nominal HPD of 1.2~arcmin. It is therefore essential to take PSF scattering into account even if the regions of interest are much wider than $\sim 1$~arcmin.

To map the bulk velocities and velocity dispersions, we employ forward model fitting for pre-selected sky regions (as opposed to ``backward'' image deconvolution), adopting a method first used by \cite{markevitch96a} and \cite{markevitch96b} to derive cluster temperature profiles and maps using {\it ASCA} data that was similarly affected by a broad PSF. We divided the Perseus core into seven sky regions (Sky~0 to Sky~6) as shown in figure~\ref{fig:region} right. Their combined outline extends beyond the combined outline of the FOVs of the three SXS observations as described in section~\ref{sec:velocity}, in order to keep the scattered flux from {\em outside}\/ that sky area into the FOV negligible, which is easily achieved given the cluster's sharply declining X-ray brightness profile.

We assume that the X-ray emission in each sky region is represented by a single-temperature, single-velocity thermal plasma model $M_{j}\;(j=0, 1,\dots, 6)$. As the X-ray emission from each region passes thorough the X-ray telescope, it is spread among the detector pixels because of the PSF as well as the slight drift of the satellite pointing direction during each observation. The spectra are collected in several detector regions shown in figure~\ref{fig:region} left for each of the 3 observations. The detector regions are selected to follow the sky regions as close as possible, but because of the $0.5$~arcmin pixel size and the pointing offsets, they are not the same for different observations. With Obs~1 covering only two sky Reg~ 5 and 6, we have a total of 12 spectra $S_i\;(i=1,\dots,12)$ for all regions and all pointings. Each of those spectra is the sum of the contributions from all sky regions $j$:
\begin{equation}
 S_i = R_i \sum_{j=0,6} P_{j\rightarrow i} M_{j},
\label{eq:modelsum}
\end{equation}
where $P_{j\rightarrow i}$ contains the relative flux contributions of the $j$-th sky region into the $i$-th detector region, and $R_i$\/ is the spectral redistribution matrix for the $i$-th detector region.

To calculate these relative flux contributions, we use external data --- Chandra ACIS images with a much better angular resolution. We combined Chandra ObsIDs 11713, 11714, 11715, 11716, 12025, 12033, 12036, 12037, 3209, 4289, 4946, 4947, 4948, 4949, 4950, 4951, 4952, 4953, 6139, 6145, and 6146, which include both ACIS-I and ACIS-S pointings. We used the standard Chandra data reduction techniques \citep[see, e.g.,][for details]{wang16}, including subtracting the blank-sky background after normalizing it at high energies, and modeling and subtracting the CCD readout artifact. The central AGN is a bright X-ray source affected by pileup in the ACIS image, and for our current purposes of modeling the ICM emission, we masked the central source and replaced it with the average brightness for the adjacent pixels. Other areas of the image are not affected by pileup. Point sources other than the AGN were left in the image; their flux is negligible compared to the ICM emission. We constructed two images, one in the broad 1.8--9~keV energy band and another containing only the 6.7~keV line flux --- to the accuracy possible with a CCD resolution. For the latter, we first extracted an image in the 6.4--6.7~keV band containing the redshifted 6.7~keV line complex (including the CCD line broadening). We then modeled the underlying continuum in this band by linear interpolation between images in the line-free intervals of 6.0--6.3~keV and 7.0--7.3~keV, with a small normalization correction to reflect the deviation of the spectrum from linear in this interval. The continuum image was then subtracted to result in a map of the 6.7 keV~line emission.

These Chandra images were divided into the sky regions, the image for each region was then multiplied by the mirror effective area and vignetting and convolved with the PSF for each of the 3 Hitomi pointings and at each energy of interest (the effective area, vignetting and the PSF depend on the photon energy). For each of these sky region images $j$, the flux that falls into each of the detector regions $i$\/ was collected. Technically, this was done using the Hitomi raytracing tool \verb+aharfgen+, which generates an ARF containing the values $P_{j\rightarrow i}$ in the expression above. These values for the energy of the 6.7~keV line (redshifted), are given in table~\ref{tab:psf} as fractions of the sum for all regions.

In addition to the ICM emission, the spectra have a contribution from the central AGN scattered into each integration region. Therefore, a similar calculation of the scattered contributions for a point source representing the AGN was done as above. Its normalization is determined separately using the Hitomi data (AGN~paper).

We can now derive the velocities and velocity dispersions for the 7 sky regions by fitting all 12 spectra simultaneously using the model $S_i$ that includes the ICM and AGN components. This can be done using two different technical approaches, both of which we used in this work and described them in the following section.

\subsection{The ARF method}
\label{sec:arfs}

\begin{figure*}
 \begin{center}
  \includegraphics[width=15cm]{figures/12spec_psfcor.eps}
 \end{center}
 \caption{Fits and residuals for the PSF-corrected velocity mapping.}
 \label{fig:12spec_psfcor}
\end{figure*}

In our main approach, whose results are described in section~\ref{sec:velocity}, the PSF effects are taken into account in {\small XSPEC} fitting by using the cross-region ARFs calculated as described above. This approach is general enough to allow fitting of various quantities such as temperatures and metallicities in addition to velocities. It also allows us to use different lines for velocity fitting --- e.g., excluding the resonant ({\it w}) line and using only the remaining lines of the 6.76~keV complex unaffected by resonant scattering. Because the ARF values are applied by {\small XSPEC} to the APEC normalization (as opposed to model flux), the ARF should contain values calculated using an image of the projected emission measure rather than the X-ray brightness \citep{markevitch96b}. Our broad-band Chandra image is an adequate approximation for this purpose. Furthermore, while the Chandra image contains information on the relative normalizations between various regions, given the calibration uncertainties, we let the overall model normalizations be free parameters for each spectrum. Thus we use external information only for the regions' relative contributions into each spectrum. Fitting was done in 2 steps --- first, temperatures were fit in a broad energy band excluding the 6.7~keV complex, then those temperatures were fixed, while the abundances and velocities were fit using the 6.7~keV complex (6.4--6.7~keV band excluding the {\it w} line). The best fit models and residuals for the velocity-fitting step of the above procedure are shown in figure~\ref{fig:12spec_psfcor}.

To give a clearer idea of the procedure for joint fitting of 12 spectra with 8 model components (7 plasma models and an AGN model), we show below a part of the {\small XSPEC} command file used in the velocity fitting. Note that in the current {\small XSPEC} implementation, the spectral redistribution matrix (the 'response' commands below) is specified for each of the sky region contributions, even though it is the same file for all the components within the same spectrum and could be applied after summing the model components, as shown in eq.\ (\ref{eq:modelsum}) --- this may change in the future.

\begin{lstlisting}[basicstyle=\ttfamily\scriptsize, frame=single]
 # Spectrum 1 (observation 3, approximating sky region 0):
 data 1 reg0obs3.pi

 # Contribution into this spectrum from the central point source:
 response 1:1 reg0obs3.rmf
 arf 1:1 AGN_reg0obs3.arf

 # Contribution into this spectrum from sky region 0:
 response 2:1 reg0obs3.rmf
 arf 2:1 sky0_to_reg0obs3.arf

 # Contribution into this spectrum from sky region 1:
 response 3:1 reg0obs3.rmf
 arf 3:1 sky1_to_reg0obs3.arf
 ... 
 response 8:1 reg0obs3.rmf
 arf 8:1 sky6_to_reg0obs3.arf
 
 # Spectrum 2 (observation 4, approximating sky region 0):
 data 2 reg0obs4.pi
 ...
 
 # Spectrum 12 (observation 1, approximating sky region 6):
 data 12 reg6obs1.pi
 response 1:12 reg6obs1.rmf
 arf 1:12 AGN_reg6obs1.arf
 arf 2:12 sky0_to_reg6obs1.arf
 ...
 arf 8:12 sky6_to_reg6obs1.arf

 # Spectral models: component 1 for AGN, components 2-8 for sky regions 0-6:
 model 1:agn TBabs(pegpwrlw+zgauss+zgauss)
 model 2:plasma0 TBabs*bapec
 model 3:plasma1 TBabs*bapec
 ...
 model 8:plasma6 TBabs*bapec
\end{lstlisting}

\begin{figure*}
 \begin{center}
  \includegraphics[width=15cm]{figures/plot_comp_ai-01.eps}
 \end{center}
 \caption{Best-fit velocity and dispersion values for the sky regions. Black and red crosses show the PSF corrected and uncorrected fits to the real spectra, while blue crosses show the PSF-uncorrected fits to the simulated spectra, using the PSF-corrected values as input for the simulation. The agreement between blue and red crosses shows that the fitting method has found a self-consistent solution.}
 \label{fig:forward}
\end{figure*}

For a check of the results, we used the best-fit PSF-corrected values of the temperatures, abundances, velocities and dispersions, and applied the PSF blending (table~\ref{tab:psf}) and detector response to generate simulated spectra for 12 detector regions. We then fitted the simulated spectra for individual regions (without the PSF correction, but simultaneously fitting the spectra for the same sky region from different pointings). We reproduced the fits for the real spectra within the statistical errors, as shown in figure~\ref{fig:forward}.

\subsection{Simplified velocity analysis using the {\it w}\/ line}
\label{sec:maxim}

While the shape of the brightest ({\it w}) line of the He$\alpha$ triplet can be significantly affected by resonant scattering in the dense cluster core (which is indeed observed, see RS~paper), the line centroid should be less sensitive to scattering than its width. Thus, the {\it w} line can offer a useful test of the bulk velocity results derived above using the other lines of the triplet. Its width should also give an upper limit on turbulent broadening. This may be accomplished using the above ARF method, limiting the last step (fitting the velocities) to the narrow interval including only the {\it w} line. However, if we choose to fit only the {\it w} line, we can use a simpler and faster fitting approach, which is also less model-dependent, since it removes (to a good approximation) the effects of the dependence of the line flux ratios in the He$\alpha$ complex on the gas temperature.

\begin{figure*}
 \begin{center}
  \includegraphics[width=8.5cm]{figures/ah_per_2zgauss_reg43.ps}
 \end{center}  
 \vspace{15mm}
 \caption{A fit to the resonance line only, using a model consisting of a power law plus two Gaussians representing the resonance line (red curve) and a combination of its nearest satellites (blue curve). The other lines of the He$\alpha$ triplet are excluded from the fit. One spectrum is shown for illustration.}
 \label{fig:2gausreg43}
\end{figure*}

To model the {\it w} line and the underlying continuum, we fit a Gaussian plus a power law in the eneregy intervals 6.42--6.49~keV and 6.575--6.65~keV (observer frame), see figure~\ref{fig:2gausreg43}. The nearest bright component of the line complex, the {\it x} line, is 18~eV away from the {\it w} line (6.7004~keV rest frame) and is excluded using the above interval. However, there is a large number of faint satellites within $\Delta E=10$~eV of the {\it w} component, which cumulatively account for 10--15\% of the {\it w} flux (for a $T=4$~keV plasma). If not included in the model, they would bias the {\it w} line position and width. We found that these satellites can be adequately modeled by adding one Gaussian component at $E=6.695$~keV (rest frame) with an intrinsic width $\sigma=3.7$~eV and an intensity 0.138 times that of the {\it w} line. To further simplify the model, we add in quadrature the typical expected velocity dispersion of 160~km~s$^{-1}$ to this component, which gives a total width of $\sigma=5.2$~eV. While the satellite line fluxes depend on plasma temperature, and the velocty broadening is of course different in different spectra, this simplification proves to be adequate. Fitting simulated APEC spectra for a relevant range of plasma temperatures ($T=3-5$~keV) and velocity broadening ($\sigma_\mathrm{v}=100-200$~km~s$^{-1}$) using the above energy interval and a model consisting of a power law with a slope fixed at 5.0 (the local slope of the thermal spectrum for $T=4$~keV), the {\it w} line represented by a Gaussian with free redshift and width, and the satellite Gaussian with the width and relative flux fixed as above and the same redshift, we were able to recover the redshift to within 15~km~s$^{-1}$ and the line width to within 10~km~s$^{-1}$. This redshift error is acceptable given the other uncertainties, e.g., $\sim$10~km~s$^{-1}$ systematic uncertainty due to the difference between the measured \citep[e.g.][]{beiersdorfer93} and theoretical (Atomic~paper) {\it w} line energies of up to $\sim$0.3~eV. A fit to one of the spectra is shown in figure~\ref{fig:2gausreg43}, where red shows the {\it w} component and blue the satellite component. Freeing the slope of the power law does not affect the best-fit line parameters, because with our choice of the energy intervals, the continuum fit straddles the line. We also verified that fits to the real spectra using this model or full APEC in the same energy interval agree within the above errors.

We model each of the 12 spectra with a sum of 6 two-Gaussian models (one for each sky region) constructed as above. Redshifts and velocity dispersions for each sky region are tied between the 12 spectra, and the relative normalizations of the 6 main Gaussians within each spectrum are fixed to the PSF-scattered fractions given in table~\ref{tab:psf}. Here we use the fractions computed using the Chandra image of the 6.7~keV line emission (see above), which are directly applicable to our Gaussian line normalizations. Thus, instead of using 6 ARFs for each spectrum to represent the PSF contributions from each of the 6 regions, as is done in our main method (section~\ref{sec:arfs}), in this method we account for the PSF mixing within the model for each spectrum. We use only one ARF and RMF for each spectrum (we used an ARF generated for a point source in the middle of each region, but it does not matter). For reasons related to XSPEC technical implementation, this fitting method is much faster --- provided the approximations used in it are acceptable. As in the ARF method (section~\ref{sec:arfs}), we allow the overall model normalization for each spectrum to be a free parameter (even though the normalizations for each sky region can be computed from the Chandra image) to account for calibration uncertainties. The power law component for each spectrum, which represents the sum of the thermal continuum and the AGN contribution, was allowed to be a free parameter, because we are interested in the line components only and must model the underlying continuum well. It is also  possible to use APEC as a model for the {\it w} line, using the same relative model normalization scheme (though care should be taken to apply the PSF mixing fractions to {\it line fluxes} rather than the APEC normalizations), but it is much slower.

A subset of {\small XSPEC} commands for this method and a printout of the model for one of 12 spectra are given below to provide a clearer view of the procedure.

\begin{lstlisting}[basicstyle=\ttfamily\scriptsize, frame=single]

# Spectrum 1 (observation 3, approximating region 0):
data 1:1 reg0_obs3_HP_gr1.pi
response  1:1 reg0_obs3_HP_l.rmf
arf 1:1 reg0_obs3_HP_ps1890.arf

# Spectrum 2 (observation 4, approximating region 0):
data 1:2 reg0_obs4_HP_gr1.pi
response  2:2 reg0_obs4_HP_l.rmf
arf 2:2 reg0_obs4_HP_ps1890.arf
...
# Spectrum 12 (observation 1, approximating region 6)
data 1:12 reg6_obs1_HP_gr1.pi
response  12:12 reg6_obs1_HP_l.rmf
arf 12:12 reg6_obs1_HP_ps1890.arf

# Model for spectrum 1: a pair of Gaussians (a w line and the sum of the
# nearby satellites  with a normalization 0.138*w) for each of the 7 sky
# regions, plus a power law:
model 1:reg03 zgauss + zgauss + zgauss + zgauss + zgauss + zgauss + zgauss
            + zgauss + zgauss + zgauss + zgauss + zgauss + zgauss + zgauss
            + powerlaw

# Printout of the model for one spectrum (reg03), showing  parameter
# dependencies. The normalization of the `diagonal' (i=j) Gaussian 
# (parameter 4 for this spectrum) is free, while normalizations of the
# w components from other sky regions are tied to it via the relative 
# PSF contributions (for simplicity we use 0 for the fractions <5%):
   1    1   zgauss     LineE      keV      6.70040      frozen
   2    1   zgauss     Sigma      keV      5.41827E-03  +/-  4.01105E-04  
   3    1   zgauss     Redshift            1.76995E-02  +/-  5.97538E-05  
   4    1   zgauss     norm                5.19561E-05  +/-  1.72050E-06  
   5    2   zgauss     LineE      keV      6.69500      frozen
   6    2   zgauss     Sigma      keV      5.20000E-03  frozen
   7    2   zgauss     Redshift            1.76995E-02  = reg03:p3
   8    2   zgauss     norm                7.16994E-06  = reg03:p4*0.138
   9    3   zgauss     LineE      keV      6.70040      frozen
  10    3   zgauss     Sigma      keV      3.39663E-03  +/-  2.76307E-04  
  11    3   zgauss     Redshift            1.74271E-02  +/-  4.28597E-05  
  12    3   zgauss     norm                8.41689E-06  = reg03:p4*0.162
  13    4   zgauss     LineE      keV      6.69500      frozen
  14    4   zgauss     Sigma      keV      5.20000E-03  frozen
  15    4   zgauss     Redshift            1.74271E-02  = reg03:p11
  16    4   zgauss     norm                1.16153E-06  = reg03:p12*0.138
  17    5   zgauss     LineE      keV      6.70040      frozen
  18    5   zgauss     Sigma      keV      3.48743E-03  +/-  2.36872E-04  
  19    5   zgauss     Redshift            1.75454E-02  +/-  3.76188E-05  
  20    5   zgauss     norm                1.15343E-05  = reg03:p4*0.222
  21    6   zgauss     LineE      keV      6.69500      frozen
  22    6   zgauss     Sigma      keV      5.20000E-03  frozen
  23    6   zgauss     Redshift            1.75454E-02  = reg03:p19
  24    6   zgauss     norm                1.59173E-06  = reg03:p20*0.138
  25    7   zgauss     LineE      keV      6.70040      frozen
  26    7   zgauss     Sigma      keV      4.65027E-03  +/-  2.67985E-04  
  27    7   zgauss     Redshift            1.71175E-02  +/-  4.26310E-05  
  28    7   zgauss     norm                6.18277E-06  = reg03:p4*0.119
  29    8   zgauss     LineE      keV      6.69500      frozen
  30    8   zgauss     Sigma      keV      5.20000E-03  frozen
  31    8   zgauss     Redshift            1.71175E-02  = reg03:p27
  32    8   zgauss     norm                8.53223E-07  = reg03:p28*0.138
  33    9   zgauss     LineE      keV      6.70040      frozen
  34    9   zgauss     Sigma      keV      3.95278E-03  +/-  2.31365E-04  
  35    9   zgauss     Redshift            1.71612E-02  +/-  3.71691E-05  
  36    9   zgauss     norm                5.09170E-06  = reg03:p4*0.098
  37   10   zgauss     LineE      keV      6.69500      frozen
  38   10   zgauss     Sigma      keV      5.20000E-03  frozen
  39   10   zgauss     Redshift            1.71612E-02  = reg03:p35
  40   10   zgauss     norm                7.02654E-07  = reg03:p36*0.138
  41   11   zgauss     LineE      keV      6.70040      frozen
  42   11   zgauss     Sigma      keV      4.61808E-03  +/-  6.04779E-04  
  43   11   zgauss     Redshift            1.73606E-02  +/-  9.80638E-05  
  44   11   zgauss     norm                0.0          = reg03:p4*0.
  45   12   zgauss     LineE      keV      6.69500      frozen
  46   12   zgauss     Sigma      keV      5.20000E-03  frozen
  47   12   zgauss     Redshift            1.73606E-02  = reg03:p43
  48   12   zgauss     norm                0.0          = reg03:p44*0.138
  49   13   zgauss     LineE      keV      6.70040      frozen
  50   13   zgauss     Sigma      keV      3.92239E-03  +/-  4.30404E-04  
  51   13   zgauss     Redshift            1.70599E-02  +/-  6.30424E-05  
  52   13   zgauss     norm                0.0          = reg03:p4*0.
  53   14   zgauss     LineE      keV      6.69500      frozen
  54   14   zgauss     Sigma      keV      5.20000E-03  frozen
  55   14   zgauss     Redshift            1.70599E-02  = reg03:p51
  56   14   zgauss     norm                0.0          = reg03:p52*0.138
  57   15   powerlaw   PhoIndex            5.00000      frozen
  58   15   powerlaw   norm                7.22371      +/-  0.208572     

# In models for other spectra, redshifts and widths of the lines are
# tied to the values for the respective sky region in this model. 
# A total of 38 parameters are being fit.
\end{lstlisting}

\begin{table*}
  \tbl{Best-fit bulk velocity and LOS velocity dispersion values. Values of
    $v$ and $\sigma_\mathrm{v}$ are km\,s$^{-1}$.}{%
  \begin{tabular}{crrrrrrrrrrr}
   \hline
   & \multicolumn{5}{c}{{\it w}\/ line excluded, ARF method} & 
   & \multicolumn{5}{c}{Fit {\it w}\/ line only, model mixing method}\\
   \cline{2-6} \cline{8-12}
   & \multicolumn{2}{c}{PSF uncorrected} && \multicolumn{2}{c}{PSF corrected} & & \multicolumn{2}{c}{PSF uncorrected} && \multicolumn{2}{c}{PSF corrected}\\
  region & $v_\mathrm{bulk}$ & $\sigma_\mathrm{v}$ & & $v_\mathrm{bulk}$ & $\sigma_\mathrm{v}$ & & $v_\mathrm{bulk}$ & $\sigma_\mathrm{v}$ & & $v_\mathrm{bulk}$ & $\sigma_\mathrm{v}$ \\
  \hline
  0  & $43_{-13}^{+12}$  & $163_{-10}^{+10}$ &~~~~~~& $75_{-28}^{+26}$  & $189_{-18}^{+19}$ &~~~~~~~~~&$50_{-8}^{+8}$&$194_{-8}^{+9}$&~~~~~~&$98_{-18}^{+20}$  & $228_{-20}^{+21}$\\
  1  & $42_{-12}^{+12}$  & $131_{-11}^{+11}$ &      & $46_{-19}^{+19}$  & $103_{-20}^{+19}$ & & $33_{-8}^{+8}$  &   $163_{-9}^{+9}$   & & $16_{-13}^{+13}$   & $127_{-16}^{+17}$ \\
  2  & $39_{-11}^{+11}$  & $126_{-12}^{+12}$ &      & $47_{-14}^{+14}$  & $98_{-17}^{+17}$  & & $39_{-7}^{+8}$    & $158_{-9}^{+9}$   & & $52_{-12}^{+11}$   & $132_{-14}^{+15}$ \\
  3  & $-19_{-11}^{+11}$ & $138_{-12}^{+12}$ &      & $-39_{-16}^{+15}$ & $106_{-20}^{+20}$ & & $-43_{-8}^{+8}$   & $193_{-8}^{+9}$   & & $-76_{-13}^{+13}$  & $191_{-16}^{+16}$ \\
  4  & $-35_{-14}^{+15}$ & $186_{-12}^{+12}$ &      & $-77_{-28}^{+29}$ & $218_{-21}^{+21}$ & & $-30_{-7}^{+7}$   & $175_{-7}^{+8}$   & & $-63_{-11}^{+11}$  & $156_{-14}^{+14}$\\
  5  & $-6_{-26}^{+25}$  & $125_{-28}^{+28}$ &      & $-9_{-56}^{+55}$  & $117_{-73}^{+62}$ & & $-9_{-17}^{+17}$  & $175_{-18}^{+20}$ & & $-3_{-29}^{+30}$  & $189_{-35}^{+37}$ \\
  6  & $-35_{-22}^{+22}$ & $99_{-32}^{+31}$  &      & $-45_{-29}^{+29}$ & $84_{-54}^{+44}$  & & $-78_{-15}^{+15}$ & $164_{-16}^{+17}$ & & $-93_{-19}^{+19}$ & $154_{-20}^{+21}$ \\
     \hline
  \end{tabular}}\label{tab:velocity_modelmix}
\end{table*}

\begin{figure*}
 \begin{center}
  \includegraphics[width=15cm]{figures/ah_per_vmap_12spec_w2zgauss.ps}
 \end{center}
 \vspace{30mm}
 \caption{Fits and residuals for the joint fit of all spectra using the {\it w}-line method (figure~\ref{fig:2gausreg43}). The continuum energy interval to the left of the line complex (see figure~\ref{fig:2gausreg43}) is not not shown.}
 \label{fig:12spec}
\end{figure*}

The resulting LOS velocities (in the same reference frame as above) and velocity dispersions are given in table~\ref{tab:velocity_modelmix}. It gives both the individual, PSF-uncorrected fits for each sky region (fitting together either two or one spectra for each region, as above) and the joint PSF-corrected fit to all spectra. The joint fit is good, with C statistic of 1390 for 1484 d.o.f. Residuals for individual spectra are shown in figure~\ref{fig:12spec}.

We note that these velocities and dispersions are derived using both a different fitting method and the independent data excluded from our main fit. It thus provides a good check of that fit. The PSF-corrected velocities from both methods are in good statistical agreement and show the same large-scale velocity gradient. The velocity dispersions from the {\it w} line method show approximately similar spatial pattern, but most values are higher (though they are statistically inconsistent only in Reg~3). Higher widths for the {\it w} line are expected in the presence of resonant scattering. We also fit the {\it w} line in the same energy interval using the ARF method and obtained results very close to those from the simplified method.


\subsection{Velocity analysis using He$\beta$ lines}
\label{sec:heb}

Fe He$\beta$ lines are optically thin and thus provide another consistency check of our main result in section~\ref{sec:velocity}. We focused our comparison only on the PSF uncorrected $\sigma_\mathrm{v}$ values, because the statistics of the He$\beta$ lines is not as good as that of the He$\alpha$ complex and the gain is not well calibrated compared to the He$\alpha$ complex.

We fit the spectra in the energy range of 7.7--7.8~keV or 7.6--7.9~keV (observer frame) using \verb+bapec+, with all parameters allowed to vary. The narrower energy range includes only the He$\beta$ lines and the results are not significantly affected by other lines. The results obtained using the wider energy range are not as clean as the former ones, because the energy range also covers the Ni He$\alpha$ line, but the errorbars are small because of the higher statistics for the continuum determination. As in above PSF-uncorrected fits, we fit spectra from different sky regions independently, while fitting simultaneously the spectra for the same sky region from different pointings.

\begin{table*}
 \tbl{Best-fit LOS velocity dispersion ($\sigma_\mathrm{v}$) in the unit of km~s$^{-1}$.}{%
 \begin{tabular}{crrr}
  \hline
  & He$\alpha$, {\it w}\/ excluded & He$\beta$, 7.7--7.8~keV & He$\beta$, 7.6--7.9~keV\\
  \hline
  0  & $163_{-10}^{+10}$ &  $145_{-32}^{+36}$ & $160_{-29}^{+31}$\\
  1  & $131_{-11}^{+11}$ & $112_{-112}^{+31}$ & $109_{-26}^{+25}$\\
  2  & $126_{-12}^{+12}$ &   $95_{-83}^{+62}$ & $154_{-28}^{+30}$\\
  3  & $138_{-12}^{+12}$ &  $152_{-32}^{+31}$ & $138_{-30}^{+31}$\\
  4  & $186_{-12}^{+12}$ &  $196_{-25}^{+27}$ & $184_{-23}^{+24}$\\
  5  & $125_{-28}^{+28}$ &  $42_{-42}^{+128}$ & $151_{-77}^{+96}$\\
  6  & $99_{-32}^{+31}$  &   $96_{-96}^{+95}$ & $171_{-79}^{+70}$\\
  \hline
 \end{tabular}}\label{tab:heb}
\end{table*}

The resulting best-fit $\sigma_\mathrm{v}$ are shown in table~\ref{tab:heb}. The PSF-uncorrected $\sigma_\mathrm{v}$ values obtained from the optically thin lines (He$\alpha$ {\it x}+{\it y}+{\it z} or He$\beta$) are consistent with each other.
