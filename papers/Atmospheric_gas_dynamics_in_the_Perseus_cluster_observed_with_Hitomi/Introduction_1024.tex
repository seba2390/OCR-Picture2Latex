\section{Introduction}
\label{sec:intro}

Clusters of galaxies are the most massive bound and virialized structures in the Universe. Their peripheries are dynamically young as clusters continue to grow through the accretion of surrounding matter. Disturbances due to subcluster mergers are found even in relaxed clusters with cool cores \citep[e.g.,][]{markevitch01,Churazov03,Clarke04,Blanton11,Ueda17}. Mergers are expected to drive shocks, bulk shear, and turbulence in the intracluster medium (ICM). Clusters with cool cores also host active galactic nuclei \citep[AGN;][]{Burns90,Sun09} which inject mechanical energy and magnetic fields into the gas of the cluster cores that drive its motions \citep[e.g.,][]{Boehringer93,Carilli94,Churazov00,McNamara00,Fabian03,Werner10}. Such AGN feedback may play a major role in preventing runaway cooling in cluster cores \citep[see][for reviews]{McNamara07,Fabian12}. Knowledge of the dynamics of the ICM will be crucial for understanding the physics of galaxy clusters such as heating and thermalization of the gas, acceleration of relativistic particles, and the level of atmospheric viscosity. It also probes the degree to which hot atmospheres are in hydrostatic balance, which has been widely assumed in cosmological studies using galaxy clusters \citep[see][for review]{Allen11}.

Bulk and turbulent motions have been difficult to measure owing to the lack of non-dispersive X-ray spectrometers with sufficient energy resolution to resolve line-of-sight (LOS) velocities. For example, a LOS bulk velocity of 500~km~s$^{-1}$ produces a Doppler shift of 11~eV for the Fe~XXV He$\alpha$ line at 6.7~keV. Most of the previous attempts using X-ray charge-coupled device (CCD) cameras, with typical energy resolutions of $\sim$150~eV, lead to upper limits or low significance ($< 3\sigma$) detections of bulk motions \citep[e.g.,][]{Dupke06,Ota07,Dupke07,Fujita08,Sato08,Sato11,Sugawara09,Nishino12,Tamura14,Ota16}; higher significance measurements were reported only in a few merging clusters \citep{Tamura11,Liu16}.

Upper limits on Doppler broadening were also obtained using the Reflection Grating Spectrometer on board XMM-Newton \citep[RGS;][]{den_Herder01} with typical values of 200--600~km~s$^{-1}$ at the 68\% confidence level \citep{Sanders10,Sanders11,bulbul12,Sanders13,Pinto15}. As the RGS is slitless, spectral lines are broadened by the spatial extent of the ICM, making it challenging to separate and spatially map the Doppler widths.

The Soft X-ray Spectrometer \citep[SXS;][]{Kelley16} on board Hitomi \citep{Takahashi16} is the first X-ray instrument in orbit capable of resolving the emission lines in extended sources and measuring their Doppler broadening and shifts. The SXS is a non-dispersive spectrometer with an energy resolution of 4.9~eV full-width at half-maximum (FWHM) at 6~keV \citep{porter16}. The SXS imaged the core of the Perseus cluster, the brightest galaxy cluster in the X-ray sky. Previous X-ray observations of this region revealed a series of faint, X-ray cavities around the AGN in the central galaxy NGC~1275 \citep{Boehringer93,McNamara96,Churazov00,Fabian00} as well as weak shocks and ripples \citep{Fabian03,Fabian06,Fabian11,Sanders07}, both suggestive of the presence of gas motions. The SXS performed four pointings in total with a field of view (FOV) of 60~kpc $\times$ 60~kpc each and a total exposure time of 320~ks as shown in figure~\ref{fig:fov} and table~\ref{tab:obs}. Early results based on two pointings toward nearly the same sky region (Obs~2 and Obs~3) were published in \citet[][hereafter H16]{hitomi16}. H16 reported that the LOS velocity dispersion in a region 30--60~kpc from the central AGN is $164 \pm 10$~km~s$^{-1}$ and the gradient in the LOS bulk velocity across the image is $150 \pm 70$~km~s$^{-1}$, where the quoted errors denote 90\% statistical uncertainties.

In this paper, we present a thorough analysis of gas motions in the Perseus cluster measured with Hitomi. Updates from H16 include; (i) the full dataset including remaining two offset pointings (Obs~1 and Obs~4) are analyzed to probe the gas motions out to 100~kpc from the central AGN; (ii) the effects of the point spread function (PSF) of the telescope with the half power diameter (HPD) of 1.2~arcmin \citep{okajima16} are taken into account in deriving the velocity maps; (iii) the absolute gas velocities are compared to a new recession velocity of NGC~1275 based on stellar absorption lines; (iv) detailed shapes of bright emission lines are examined to search for non-Gaussianity of the distribution function of the gas velocity; (v) constraints on the thermal motion of ions in the ICM are derived combining the widths of the lines originating from various elements; and (vi) revised calibration and improved estimation for the systematic errors \citep{eckart17} are adopted.

This paper is organized as follows. Section~\ref{sec:data} describes observations and data reduction. Section~\ref{sec:analysis} presents details of analysis and results. Implications of our results on the physics of galaxy clusters are discussed in section~\ref{sec:discussion}. Section~\ref{sec:conclusions} summarizes our conclusions. A new redshift measurement of the central galaxy NGC~1275 is presented in appendix~\ref{sec:redshift} and various systematic uncertainties of our results are discussed in appendix~\ref{sec:systematic}. The details of the velocity mapping are shown in appendix~\ref{sec:details}. Throughout the paper, we adopt standard values of cosmological density parameters, $\Omega_{\rm M}=0.3$ and $\Omega_{\Lambda}=0.7$, and the Hubble constant $H_0 = 70$~km~s$^{-1}$~Mpc$^{-1}$. In this cosmology, the angular size of 1~arcmin corresponds to the physical scale of 21~kpc at the updated redshift of NGC~1275, $z=0.017284$. Unless stated otherwise, errors are given at 68\% confidence levels.
