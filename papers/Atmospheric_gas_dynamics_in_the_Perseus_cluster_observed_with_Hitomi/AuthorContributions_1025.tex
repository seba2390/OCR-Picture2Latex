\section*{Author Contributions}
Y. Ichinohe and S. Ueda led this study and wrote the final manuscript along with T. Kitayama, B. McNamara, N. Werner, R. Fujimoto, S. Inoue, M. Markevitch, and C. Kilbourne.
Y. Ichinohe and S. Ueda performed the analysis of section~\ref{sec:velocity} and appendices~\ref{sec:systematic}, \ref{sec:xcm}, \ref{sec:arfs}, and \ref{sec:heb}.
R. Fujimoto and K. Tanaka conducted the analysis of sections~\ref{sec:spectra} and \ref{sec:nongaussianity}.
S. Inoue and T. Kitayama performed the analysis of section~\ref{sec:iontemperature}.
N. Werner, B. McNamara, and I. Zhuravleva provided various inputs to section \ref{sec:discussion}.
R. Canning measured the new redshift of the central galaxy NGC~1275 presented in appendix \ref{sec:redshift}.
M. Markevitch performed the analysis of appendix~\ref{sec:maxim}.
Q. Wang contributed to the analysis of appendix~\ref{sec:xcm}.
T. Tamura, N. Ota, M. Tsujimoto, K. Sato, and S. Nakashima contributed to the velocity mapping analysis and studies on systematic uncertainties.
R. Fujimoto, C. Kilbourne, and S. Porter achieved the development, integration tests, and in-orbit operation of the SXS.
Y. Maeda supported the evaluation of the PSF scattering effect.
T. Hayashi, S. Kitamoto, and I. Zhuravleva evaluated the impact of the gravitational redshift.
The science goals of Hitomi were discussed and developed over more than 10 years by the ASTRO-H Science Working Group (SWG), all members of which are authors of this manuscript. All the instruments were prepared by joint efforts of the team. The manuscript was subject to an internal collaboration-wide review process. All authors reviewed and approved the final version of the manuscript.
