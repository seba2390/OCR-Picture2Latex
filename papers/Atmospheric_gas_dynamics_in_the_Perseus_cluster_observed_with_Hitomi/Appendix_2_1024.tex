\section{Systematic uncertainty}
\label{sec:systematic}

\subsection{Gain uncertainty}
\label{sec:gain}

We achieved the systematic gain difference between Obs~3 and Obs~4 of $\lesssim0.3$~eV at 6.586~keV (the line centroid of Fe He$\alpha$ {\it w} in observer frame) with the standard pipeline gain correction processes alone. As the pointings of Obs~1 and Obs~2 were performed before the temperature of the helium tank reached near thermal equilibrium, an additional energy scale adjustment (\verb+sxsperseus+\footnote{https://heasarc.gsfc.nasa.gov/docs/hitomi/analysis/ahhelp/sxsperseus.html}), in addition to the standard pipeline gain correction, was applied to these datasets. As the FOV of Obs~2 overlaps with those of Obs~3 or Obs~4, we are able to compare the gain among these observations directly. After the gain adjustment, the data of Obs~2 have a gain offset of $\sim$2~eV at 6.586~keV, compared to Obs~3 (and Obs~4). As the FOV of Obs~1 does not overlap with those of Obs~2, 3 or 4, the absolute gain scale of Obs~1 is difficult to estimate. Considering the $\sim$2~eV gain offset of Obs~2, we think that the systematic uncertainty of the energy scale of Obs~1, is at least $\sim$2~eV relative to Obs~3. The pixel-to-pixel relative gain uncertainty within each single pointing is $\sim$0.5~eV. More details are described in \citet[][]{eckart17}.

\subsubsection{Effect of the gain uncertainty}
\label{sec:effect}

We investigated the effect of the gain uncertainty described in section~\ref{sec:gain} on the velocity measurements. We manually shifted the gain\footnote{We used \texttt{rmodel gain} command available in XSPEC, with \texttt{slope} $=1$ and \texttt{intercept} $=\Delta E$ where $\Delta E$ is the gain shift. We used the energy range of 6.4--6.7~keV.} and followed the same velocity fitting described in section~\ref{sec:velocity} to see how the result changes by the systematic gain difference. We shifted (1) the gain of all the Obs~1 data by $\pm$2~eV to account for the uncertainty of Obs~1 gain relative to Obs~3 and Obs~4 gain. (2) the gain of Reg~5 Obs~1 by $\pm$0.5~eV and at the same time the gain of Reg~6 Obs~1 by $\mp$0.5~eV for the pixel-to-pixel gain uncertainties within Obs~1. (3) the gain of Reg~0 Obs~3, Reg~0 Obs~4, Reg~1 Obs~3, Reg~1 Obs~4, Reg~2 Obs~3, Reg~2 Obs~4, Reg~3 Obs~3, Reg~3 Obs~4, Reg~4 Obs~3, and Reg~4 Obs~4 by $0.5~\mathrm{eV}/\sqrt{n}$, where $n$ is the number of pixels of each single region, twenty times with random signs in each trial, for relative gain uncertainties within Obs~3 and Obs~4.

\begin{figure*}
 \begin{center}
 \includegraphics[width=15cm]{figures/plot_gain_ai-01.eps} 
 \end{center}
 \caption{The best-fitting bulk velocities and LOS velocity dispersions after manually shifting the energy gain. {\it Top:} the effect of the uncertainty of Obs~1 gain relative to Obs~3 and Obs~4. {\it Middle:} the effect of the pixel-to-pixel gain uncertainty within Obs~1. {\it Bottom:} the effect of the pixel-to-pixel gain uncertainties within Obs~3 and Obs~4. The red crosses are the best-fitting values shown in table~\ref{tab:velocity}, and the grey crosses and dashed lines represent the best-fitting values after the gain shifts.}
\label{fig:plotgain}
\end{figure*}

The best-fitting bulk velocities and LOS velocity dispersions after the above mentioned gain shifts are shown in figure~\ref{fig:plotgain}. We found in every case that the LOS velocity dispersion does not change significantly from the nominal value ($\lesssim$20~km~s$^{-1}$ except for one case in Reg~5), although the best-fitting bulk velocity changes in proportion to the gain shifts.

\subsection{Effect of the line spread function uncertainty}
\label{sec:lsf}

We examined the uncertainty of line spread function (LSF) of the SXS and its effect on the measurement of LOS velocity dispersion. Due to the incomplete state of the SXS calibration at the time of these observations, it is not possible to determine a robust estimate of the uncertainty on the instrumental broadening. In H16, we conservatively estimated the range of possible FWHM values as 5$\pm$0.5~eV and set that as the 90\% confidence level. This estimate was based on variation in the calibration pixel LSF over time, how the array resolution compared with the calibration-pixel resolution during the later calibration measurement, and the difference in apparent line widths between Obs~2 and Obs~3. Even this conservative value corresponded to a smaller uncertainty on the velocity broadening at the Fe He-alpha lines than that due to the statistical uncertainty. For the current paper, we would like to be able to use a less conservative value, and to assess the impact of both estimates on our results. For the more optimistic estimate, we have chosen $\pm$0.15~eV, based on the dispersion of the resolutions of the individual pixels across the array during the later in-orbit calibration with 55Fe, and the premise that this dispersion represents pixel-dependent temporal variation more than intrinsic differences in the resolution.

The effect of its uncertainty on LOS velocity dispersion is expressed by
\begin{eqnarray}
\Delta \sigma_{\rm v} &\simeq& 3~{\rm km~s}^{-1}
\left(\frac{\sigma_{\rm v}}{100~{\rm km~s}^{-1}} \right)^{-1}
\left(\frac{W_{\rm inst}}{5~{\rm eV}} \right)
\left(\frac{\Delta W_{\rm inst}}{0.15~{\rm eV}}\right)
\left(\frac{E_{\rm obs}}{6.7~{\rm keV}} \right)^{-2},
\label{eq-deltavt2}
\end{eqnarray}
where $W_{\rm inst}$ is the FWHM of instrumental broadening and $\Delta W_{\rm inst}$ is an uncertainty of instrumental broadening in FWHM, assuming $\Delta W_{\rm inst} \ll W_{\rm inst}$ \citep[more details are shown in appendix of][]{kitayama-wp}. The effect is negligible.

\subsection{Effect of the PSF shape uncertainty}

We examined systematic uncertainties of LOS velocity dispersion introduced by the uncertainty of the PSF shape. As indicated in table~\ref{tab:psf}, the cross-term contribution from Sky~0 to Reg~1 Obs~3 is the largest among cross-term contributions. We found that the difference is typically $\lesssim$5~km~s$^{-1}$, except for Reg~1 ($\sim$10~km~s$^{-1}$) even when this cross-term was changed by $\pm30$\%, which is the maximum calibration uncertainty of the off-axis PSF normalizations between in the ground and in orbit \citep[][]{maeda17}. We also checked the effect of PSF uncertainty on the results of Obs~1 (Reg~5 and Reg~6). By changing the contribution from Sky~2 by $\pm 30$\% where is the largest contribution among the sky regions, we found that the difference is $\sim$5~km~s$^{-1}$, except for Reg~5 ($\sim$20~km~s$^{-1}$).

\subsection{Effect of the modeling uncertainty}
\label{sec:modeling}

We investigated the systematic uncertainty originating from plasma emission modeling. We examined the change of the best-fit redshift by fitting only Fe He$\alpha$ {\it w} line, which is not used in the velocity fitting in section~\ref{sec:velocity}. The analysis details are shown in appendix~\ref{sec:maxim}. This line has the highest counts among the He$\alpha$ complex. While the shape of this line can be affected by resonance scattering, the line centroid position is expected to be nearly unchanged. We obtained that the PSF-uncorrected bulk velocity of each regions are consistent between two methods except for in Reg~3 and Reg~6 (see table~\ref{tab:velocity_modelmix}). The offset of bulk velocity in these two regions is $\lesssim45$~km~s$^{-1}$. However, in Reg~6, when we modeled the {\it w} line using \verb+bapec+, we obtained a consistent bulk velocity with that shown in table~\ref{tab:velocity}. This suggests that the discrepancy originates from the emission modelling uncertainties. The effect of the modeling uncertainty on the bulk velocity measurements is therefore $\lesssim45$~km~s$^{-1}$.
