\section{New redshift measurement of NGC~1275 using absorption lines}
\label{sec:redshift}

\begin{figure*}
\begin{center}
\includegraphics[width=8cm]{figures/slitplacement.eps}
\end{center}
\caption{The slit placement on NGC~1275. Our spectral extraction region is indicated with the cyan box on the red slit. Only the brighter parts of the low velocity system are extracted to avoid contamination by both a bright star and the high velocity system.}
\label{fig:ApA1}
\end{figure*}

Long slit spectroscopy was performed using the Intermediate dispersion Spectrograph and Imaging System (ISIS) at the 4.2~m William Herschel Telescope on the island of La Palma on 2007 December 29. The data were reduced using tailored IDL routines (adapted from the KRISIS IDL scripts by J.R. Mullaney 2008) for standard bias, flat field correction and wavelength calibration. The spectra were then traced and extracted separately on each frame using Gaussian and Lorentz profile fits in the cross-dispersion direction. Only the brighter parts of the low velocity system are extracted to avoid contamination by both a bright star which is in the slit and the high velocity system (see figure~\ref{fig:ApA1}). The spectra are median-combined. The wavelength calibration was checked and refined using bright sky Hg lines at air wavelengths of 4046.565\AA\ and 4358.335\AA. These features, especially at 4358\AA, are strong in our spectra and allow a finer, more precise wavelength calibration.

We fit the median combined R300B arm spectra using pPXF, which is an IDL program to extract the stellar kinematics or stellar population from absorption-line spectra of galaxies using the Penalized Pixel-Fitting method \citep[pPXF;][]{Cappellari04,cappellari17}. We fit Miles stellar population synthesis models with an IMF slope of 1.3 and metallicity values ranging from $-2.32$ to $+0.22$. The stellar kinematics is fit with the emission lines masked out. We obtain a best fit redshift of $z=0.017284\pm0.000039$ with only the statistical fitting uncertainties included. Including the upper and lower limits on wavelength calibration, we obtain $z=0.017284\pm0.00005$. For comparison, fitting the [O II] emission line doublet in the same region as the absorption lines gives $z=0.01697\pm0.00003$.
