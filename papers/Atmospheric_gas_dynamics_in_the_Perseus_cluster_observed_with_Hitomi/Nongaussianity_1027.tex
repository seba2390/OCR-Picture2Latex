\subsection{Limits on non-Gaussianity of line shapes}
\label{sec:nongaussianity}

As shown in section~\ref{sec:spectra}, the observed widths of the Fe lines ($\sigma\sim 4$~eV) are much broader than those expected by the convolution of the line spread function of the SXS (FWHM $\sim5$~eV or $\sigma\sim2$~eV) with the thermal width ($\sigma_{\rm th}\sim2$~eV for Fe at $kT\sim4$~keV). Note also that uncertainties of instrumental energy scale and the line spread function at around 6~keV are smaller than the observed widths, as shown in appendix~\ref{sec:systematic}. They are instead governed by hydrodynamic motion of the gas. We thus aim to obtain further information on the gas velocity distribution by examining the line shapes in detail. In figures~\ref{fig:FeHea} and \ref{fig:otherlines}, fitting results of S Ly$\alpha$, Fe He$\alpha$, Ly$\alpha$, and He$\beta$ lines from Obs~3 and 4 are shown with residuals (ratios of the data to the best-fit model). In what follows, we make use of Obs~2 to improve the statistics and further investigate the line shapes.

\begin{table*}
\tbl{Centroid energy in the observer frame, width, significance, and goodness-of-fit of lines detected at $> 5\sigma$.}{
\begin{tabular}{@{\extracolsep{3pt}}cccccccc@{}}
\hline
     & \multicolumn{3}{c}{Line information} & \multicolumn{3}{c}{Fitting information$^*$} & \\\cline{2-4} \cline{5-7}
Line & Centroid energy$^\dagger$ & $\sigma_{\rm v+th}$ & Significance$^\ddagger$ & Energy band & C-statistic & d.o.f. & Note$^{**}$\\
     & (eV)   & (km\,s$^{-1}$)  &                  & (keV)       &             &        &\\
\hline
Si Ly$\alpha$ & $1969.32\pm 0.21$ & $224^{+49}_{-54}$ & 12.9 & 1.945--1.995 & 40.27 & 45&(1)\\
Si Ly$\beta$ & $2333.73\pm 0.49$ & $327^{+71}_{-68}$ & 7.4 & 2.28--2.38 & 71.66 & 94&\\
S He$\alpha$ & $2417.05\pm 0.38$ & $256^{+59}_{-57}$ & 8.1 & 2.355--2.45 & 73.47 & 90&\\
S Ly$\alpha$ & $2575.83\pm 0.11$ & $192^{+21}_{-22}$ & 27.2 & 2.53--2.62 & 117.17 & 85&\\
S Ly$\beta$ & $3052.33\pm 0.26$ & $198^{+39}_{-38}$ & 10.9 & 3.00--3.14 & 116.85 & 132&\\
Ar He$\alpha$ & $3084.46\pm 0.34$ & $150^{+47}_{-50}$ & 8.5 & 3.00--3.14 & 99.95 & 132&\\
Ar Ly$\alpha$ & $3265.12\pm 0.27$ & $260^{+38}_{-37}$ & 14.3 & 3.235--3.29 & 38.94 & 50&(2)\\
Ca He$\alpha$ & $3835.26\pm 0.19$ & $186^{+21}_{-20}$ & 15.8 & 3.77--3.855 & 55.85 & 79&\\
Ca Ly$\alpha$ & $4036.97\pm 0.35$ & $202^{+39}_{-33}$ & 13.4 & 3.98--4.10$^\S$ & 94.01 & 95&\\
Fe He$\alpha$ $z$ & $6522.97\pm 0.11$ & $166\pm  5$ & 44.3 & 6.47--6.63 $^\|$& 167.79 & 148&\\
Fe He$\alpha$ $w$ & $6586.13^{+0.06}_{-0.07}$ & $195\pm  3$ & 78.8 & 6.47--6.63$^\|$ & 182.37 & 148&(3)\\
Fe Ly$\alpha$ & $6854.49\pm 0.24$ & $183\pm 11$ & 18.1 & 6.77--6.89 & 143.74 & 113&\\
Ni He$\alpha$ & $7671.73^{+0.60}_{-0.61}$ & $224^{+36}_{-33}$ & 8.0 & 7.55--7.71 & 145.35 & 155&(4)\\
Fe He$\beta$ & $7744.83^{+0.22}_{-0.23}$ & $178^{+11}_{-10}$ & 27.5 & 7.70--7.80 & 82.35 & 94&\\
Fe Ly$\beta$ & $8112.19^{+0.84}_{-0.46}$ & $  0^{+75}_{- 0}$ & 5.9 & 8.05--8.22 & 152.86 & 162&(5)\\
Fe He$\gamma$ & $8152.44\pm 0.50$ & $189\pm 20$ & 12.5 & 8.05--8.22 & 146.75 & 162&(6)\\
\hline
\end{tabular}}\label{tab:width_obs234}
\begin{tabnote}
$^*$ C-statistic and d.o.f. are those in the specified energy band.\\
$^\dagger$ Energy of the most prominent component, unless specified otherwise.\\
$^\ddagger$ Significance was determined by dividing the normalization by its $1\sigma$ error.\\
$^\S$ Energy range from 4.07~keV to 4.09~keV was ignored, to exclude Ar Ly$\gamma$.\\
$^\|$ Gaussians were used for both $z$ and $w$ lines.\\ 
$^{**}$ (1) Line width changed from $1.85_{-0.42}^{+0.41}$~eV to $1.50_{-0.36}^{+0.33}$~eV, by adding Obs 2 data. The parameters may be unreliable. (2) Line width changed from $2.24_{-0.52}^{+0.51}$~eV to $2.88\pm0.42$~eV, by adding Obs 2 data. The parameters may be unreliable. (3) This line is likely to be optically thick and affected by resonance scattering. (4) This energy range is contaminated by Fe satellite lines, and the parameters may be unreliable. (5) This energy range is contaminated by various satellite lines. In addition, the line width changed from $9.0_{-2.6}^{+2.8}$~eV to $0.0_{-0.0}^{+2.1}$~eV by adding Obs 2 data. The parameters may be unreliable. (6) This energy range is contaminated by various satellite lines. The parameters might be affected by them.
\end{tabnote}
\end{table*}

%Obs~2 contains a previously known systematic uncertainty in the energy scale. 

The observed centroid energy of the Fe He$\alpha$ resonance line
of Obs~2 is about 1.8~eV lower than that of Obs~3, and its width
($\sigma$) is about 0.36~eV broader, despite their similar pointing
directions. Obs~2 (and Obs~1) occurred while the SXS dewar was still
coming into thermal equilibrium after launch \citep{fujimoto16}, and
these discrepancies come from the limitations of the method used to
correct the drifting energy scale. The energy scale of the Obs~2 data
was corrected as follows, to align their line centers. First, the
centroid energy of each line of Obs~2 and 3 was determined by fitting
the data separately. Then the energy (PI column) of each photon in the
event file of Obs~2 was recalculated by multiplying a factor $E_{\rm
Obs~3}/E_{\rm Obs~2}$, where $E_{\rm Obs~2}$ and $E_{\rm Obs~3}$ are the
best-fit line center energies of Obs~2 and Obs~3, respectively. The
event files of Obs~2, 3, and 4 were then merged and spectral files were
generated.  Note that the correction factor was determined for each line
and hence, a spectral file was generated for each line separately. Note
also that no additional gain alignment among the detector pixels was
applied. The spectra were fitted in the same manner as described in
section~\ref{sec:spectra}. Note that, for Fe He$\alpha$, the resonance
({\it w}) line and the forbidden ({\it z}) line were manually excluded
from the atomic database and substituded by external Gaussians, to
determine the parameters of these lines. The fitting results are shown
in table~\ref{tab:width_obs234}.

\begin{figure*}
\begin{center}
\includegraphics[width=16.5cm]{./figures/ratio_spectra_171023.ps}
\end{center}
 \caption{(Upper panels) Data and best-fit models of Fe He$\alpha$ {\it w}, Ly$\alpha_1$, and He$\beta_1$. The continuum model and the components other than the main line were subtracted. Solid (red) and dashed (green) lines represent the best-fit Gaussian and Voigtian profiles, respectively. Instrumental broadening with and without thermal broadening are indicated with dotted (blue) and dashed-dotted (black) lines. The horizontal axis is the velocity converted from the observed energy, where the line center is set at the origin. The bin size is 1~eV in the energy space, which corresponds to 45.5~km\,s$^{-1}$, 43.7~km\,s$^{-1}$, and 38.7~km\,s$^{-1}$, respectively. (Lower panels) The ratio spectra of the data to the best-fit Gaussian models, (left) for Fe He$\alpha$ {\it w}, and (right) for Fe Ly$\alpha_1$ and He$\beta_1$ co-added. Note that the line spread function is not deconvolved from the data.}
 \label{fig:stacked_ratio_spectra}
\end{figure*}

In this section, we focus on three brightest and less contaminated Fe transitions, He$\alpha$, Ly$\alpha$, and He$\beta$. They are from the single element and have a common thermal broadening. In addition, their energies are close enough that we can assume no significant difference in the detector line spread functions. Any astronomical velocity deviation components can cause common residuals of the line shapes in velocity space. Figure~\ref{fig:stacked_ratio_spectra} shows the spectra of these lines in velocity space, after subtracting the best-fitting continuum model and the components other than the main line (He$\alpha$ {\it w}, Ly$\alpha_1$, and He$\beta_1$), where the line center energies were set at the origin of the velocity. As we are interested in deviations from Gaussianity, ratios of the data to the best-fit Gaussian models were also shown in figure~\ref{fig:stacked_ratio_spectra}. Ratios of Ly$\alpha_1$ and He$\beta_1$ were co-added. Positive (ratio $>1$) features are seen at around $\pm(400$--$500)$~km\,s$^{-1}$, while there is a negative (ratio $<1$) feature at around $+300$~km\,s$^{-1}$. However, they are not as broad as the detector line spread function (FWHM $\sim230$~km\,s$^{-1}$). Therefore, we do not conclude that these are cluster-related velocity structures.


\begin{table*}
 \tbl{Best-fit widths when Voigt functions were used.}{
  \begin{tabular}{cccccc}
\hline
 & Gaussian width ($\sigma$) & Lorentzian width (FWHM) & C-statistic & d.o.f. & Natural width (FWHM)$^*$\\
 & (km\,s$^{-1}$) & (km\,s$^{-1}$) & & & (km\,s$^{-1}$)\\
\hline
Fe He$\alpha$ {\it w} & $194\pm 3$ & $0.10^{+0.09}_{-0.03}$ & 182.04 & 147 & 13.9\\
Fe Ly$\alpha$ & $113^{+14}_{-13}$ &$172^{+29}_{-17}$ & 137.44 & 112 & 8.2\\
Fe He$\beta$ & $137\pm11$ & $114^{+20}_{-19}$ & 80.39 & 93 & 3.0\\
\hline
\end{tabular}}\label{tab:width_obs234_Voigt}
\begin{tabnote}
$^*$ Calculated using the Einstein $A$ coefficient shown in AtomDB.\\
\end{tabnote}
\end{table*}

We also fitted each line in the same manner as described above, but
using Voigt functions\footnote{For the Voigt function fitting, we used
the patched model that is the same code as implemented in XSPEC
12.9.1l. See also
https://heasarc.gsfc.nasa.gov/xanadu/xspec/issues/issues.html.} instead
of Gaussians, for He$\alpha$ {\it w}, Ly$\alpha_1$, Ly$\alpha_2$,
He$\beta_1$, and He$\beta_2$. The best-fitting shapes after subtracting
the continuum and the components other than the main line are shown with
dashed curves in the upper panels of
figure~\ref{fig:stacked_ratio_spectra}, and the best-fit widths are
summarized in table~\ref{tab:width_obs234_Voigt}. The Lorentzian widths
of Ly$\alpha$ and He$\beta$ were much broader than the natural
width. This may be due to large positive deviations at around
$\pm(400$--$500)$~km\,s$^{-1}$. On the other hand, it was smaller
than the natural width for He$\alpha$.  C-statistic decreased by 
0.3, 6.3, and 2.0 for He$\alpha$, Ly$\alpha$ and He$\beta$,
respectively, when compared with that shown in
table~\ref{tab:width_obs234}. Given these small improvements, we
conclude that it is difficult to distinguish the Voigt and Gaussian line
shapes using the present data.

After integrating the data of the entire SXS FOV ($60~{\rm kpc}\times60~{\rm kpc}$), no clear deviations from Gaussianity were found. This may be because deviations are spatially averaged and smeared out. To investigate the line profile in smaller areas, we extracted spectra from several $2\times2$ pixel ($20\times20~{\rm kpc}$) regions, and analyzed the Fe He$\alpha$ {\it w} profiles similarly. We found no common residuals clearly seen in Obs~2, 3, and 4 when the spectra of the pixels that corresponded to the same or similar sky regions were compared. We also separated the data into two groups, the central region (including the AGN) and the outer region, but obtained similar results. Finally, as independent indicators, the skewness and the kurtosis of the line profiles were calculated, and they were broadly consistent with those of Gaussian. No clear deviation from Gaussianity was found.
