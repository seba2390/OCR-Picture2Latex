\subsection{Ion temperature measurements}
\label{sec:iontemperature}

In the analysis presented in previous sections, the observed line profiles are analyzed assuming that the ions are in thermal equilibrium with electrons and share the same temperature. High-resolution spectra by Hitomi provide us the first opportunity to directly test this assumption for galaxy clusters. As discussed in section~\ref{sec:discussion}, equilibration between electrons and ions takes longer than thermalization of the electron and ion distributions. A difference between the ion and electron temperatures may indicate a departure from thermal equilibrium.

The LOS velocity dispersion due to an isotropic thermal motion of ions is given by $\sigma_{\rm th} = \sqrt{k T_{\rm ion}/m_{\rm ion}}$, where $k$ is the Boltzmann constant, $T_{\rm ion}$ is the ion kinetic temperature, and $m_{\rm ion}$ is the ion mass. The LOS velocity dispersion from random hydrodynamic gas motions including turbulence, $\sigma_{\rm v}$, is assumed common for all the elements. Since only the former depends on $m_{\rm ion}$, one can in principle measure $\sigma_{\rm th}$ (i.e., $T_{\rm ion}$) and $\sigma_{\rm v}$ separately by combining the widths of lines originating from different heavy elements. For example, $kT_{\rm ion} = 4$~keV corresponds to $\sigma_{\rm th}=83, ~98, ~110, ~120$~km~s$^{-1}$ for Fe, Ca, S, and Si, respectively. These thermal velocities tend to be smaller than $\sigma_{\rm v}$ even for the lightest of currently observed elements, making the measurement of $T_{\rm ion}$ challenging. In what follows, we assume that the ions share a single kinetic temperature for simplicity.

\begin{figure*}[t]
\begin{minipage}{0.5\hsize}
 \begin{center}
  \includegraphics[width=8.5cm]{./figures/170930_Tionfit_obs234_all.eps}
 \end{center}
\end{minipage}
\begin{minipage}{0.5\hsize}
 \begin{center}
  \includegraphics[width=7cm]{./figures/170930_Tionfit_obs234_5sigma_10sigma_merged_contour.eps}
 \end{center}
\end{minipage}
\caption{{\it Left:} The total velocity dispersion $\sigma_{\rm v+th}$ of bright lines as a function of the ion mass in the atomic mass unit (amu). For clarity, the data points for the same element are slightly shifted horizontally. Black circles and gray crosses denote the lines detected at more than $10\sigma$ significance and at $5-10 \sigma$ significance, respectively. Solid and dashed lines show the best-fit relation $\sigma_{\rm v+th}=(\sigma_{\rm th}^2 + \sigma_{\rm v}^2)^\frac{1}{2}$ (red solid) and its components $\sigma_{\rm th}$ (green dashed) and $\sigma_{\rm v}$ (blue dashed) for the $>10\sigma$ lines. Dotted lines are the best-fit relation $\sigma_{\rm v+th}$ (red dotted) and its components $\sigma_{\rm th}$ (green dotted) and $\sigma_{\rm v}$ (blue dotted) for the $>5\sigma$ lines. {\it Right:} The 68\% confidence regions of $kT_{\rm ion}$ and $\sigma_{\rm v}$ for two parameters of interest ($\Delta\chi^2=2.3$) with a plus marking the best-fit values. Red solid and green dashed contours represent the results for the $>10\sigma$ and $>5\sigma$ lines, respectively. For reference, the blue horizontal bar indicates the range of the electron temperature measured in T~paper.}
\label{fig:fit_Tion_empirical}
\end{figure*}

The left panel of figure~\ref{fig:fit_Tion_empirical} shows the total velocity dispersion $\sigma_{\rm v+th}$ of lines detected at more than $5\sigma$ significance listed in table~\ref{tab:width_obs234}. Unreliable measurements marked by notes 1--6 in table~\ref{tab:width_obs234} have been excluded. The lines from different elements show nearly consistent velocity dispersions with a weakly-decreasing trend with ion mass. They are fit by $\sigma_{\rm v+th}=(\sigma_{\rm v}^2 + \sigma_{\rm th}^2)^\frac{1}{2}$ varying $T_{\rm ion}$ and $\sigma_{\rm v}$ as free parameters. The best-fit values are $kT_{\rm ion} = 10.2^{+5.0}_{-4.6}$\,keV and $\sigma_{\rm v}=107^{+35}_{-58}$\,km~s$^{-1}$, with $\chi^2=7.104$ for 8 degrees of freedom. If only the most secure measurements at more than $10 \sigma$ significance (black circles in the left panel of figure~\ref{fig:fit_Tion_empirical}) are used, the best-fit values are $kT_{\rm ion} = 7.3^{+5.3}_{-5.0}$\,keV and $\sigma_{\rm v}=129^{+32}_{-45}$\,km~s$^{-1}$, with $\chi^2=2.640$ for 5 degrees of freedom. If we vary just a single parameter $\sigma_{\rm v}$ by setting $\sigma_{\rm th}=0$, we obtain $\sigma_{\rm v}=174.3^{+4.1}_{-4.2}$ km~s$^{-1}$ with $\chi^2=12.20$ for 9 degrees of freedom from the $> 5 \sigma$ lines, and $\sigma_{\rm v}=173.4\pm4.2$ km~s$^{-1}$ with $\chi^2=4.848$ for 6 degrees of freedom from the $> 10 \sigma$ lines.

The red solid and green dashed contours in the right panel of figure~\ref{fig:fit_Tion_empirical} show the 68\% confidence regions of $T_{\rm ion}$ and $\sigma_{\rm v}$ for the $> 10 \sigma$ lines and the $> 5 \sigma$ lines, respectively. As expected, a negative correlation is found between $T_{\rm ion}$ and $\sigma_{\rm v}$. Albeit with large errors, the inferred ion temperature is consistent within the 68\% confidence level with the electron temperature reported in T~paper. The calibrated SXS FWHM has a systematic error of $\sim$0.15~eV (see appendix~\ref{sec:systematic}), which does not alter the results of this subsection. The present errors are dominated by the uncertainties of the widths of the lines in the low energy (2--4~keV) band; higher significance data at lower energies and inclusion of lighter elements will be crucial for improving the measurement.

\begin{figure}[h]
\begin{center}
 \includegraphics[height=8cm,angle=-90]{./figures/calc_cvalue_atomdata_noZcor_wnxb_171005.eps}
\end{center}
\caption{Results of fitting the entire spectrum with a plasma code SPEX. Top and bottom panels show the optimal values of $\sigma_{\rm v}$ and C-statistic, respectively, for a given value of $T_{\rm ion}$.}
\label{fig:fit_Tion_spex}
\end{figure}

For comparison, we also infer the ion temperature by fitting the entire spectrum with a plasma code, SPEX v3.03.00 \citep{kaastra96}. Here we apply a gain correction using equation (A1) of Atomic~paper to match the observed line energies to those implemented in SPEX. We fit the spectrum with models of the collisional ionization equilibrium plasma, the central AGN, and the NXB components. For the central AGN, we adopt the model parameters determined by AGN~paper. We exclude the energy band covering the Fe XXV He$\alpha$ {\it w} line to eliminate the effect of resonance scattering. Figure \ref{fig:fit_Tion_spex} shows the optimal values of C-statistic and $\sigma_{\rm v}$ for a given value of $T_{\rm ion}$.  The best-fit values are $kT_{\rm ion}=6.0^{+4.2}_{-3.7}$\,keV and $\sigma_{\rm  v}=153^{+21}_{-27}$\,km~s$^{-1}$, with the C-statistic value of 4999.86 for 4653 degrees of freedom. Again a negative correlation between $T_{\rm ion}$ and $\sigma_{\rm v}$ is found. These results are consistent with those derived from a set of bright lines shown in figure \ref{fig:fit_Tion_empirical}.

Note that a similar analysis using SPEX is also performed in Atomic~paper. They present the results when the Fe XXV He$\alpha$ {\it w} line is included in the fit. Since this line is likely subject to resonance scattering (RS~paper), the fitted value of $T_{\rm ion}$ depends on how the radiative transfer effect is taken into account. They show that a simple absorption model implemented in SPEX yields the value of $T_{\rm ion}$ in good agreement with $T_{\rm e}$ (see section~7.1 of Atomic~paper for details).
