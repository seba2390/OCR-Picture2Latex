\subsection{Profiles of major emission lines}
\label{sec:spectra}

\begin{figure*}
 \begin{center}
  \includegraphics[width=16cm]{figures/spectra_FeHea_170925.ps}
 \end{center}
 \caption{Fe~He$\alpha$ lines of the full-FOV data of Obs~3+Obs~4 (left) and Obs~1 (right). The LOS velocity dispersion ($\sigma_\mathrm{v}$, {\it w}-line excluded. See also table~\ref{tab:obs34_width}), the bulk velocity calculated with respect to the redshift of NGC~1275 ($v_\mathrm{bulk}$) and the total number of photons in the displayed energy band are shown in each figure. The red curves are the best-fitting models, and the dotted curves are the spectral constituents, i.e., modified APEC or Gaussian. See main text for details. The energy bin size is 1~eV or wider for lower count bins. The resonance line ({\it w}), the intercombination lines ({\it x} and {\it y}), and the forbidden line ({\it z}) are denoted. The letters are as given in \citet{gabriel72}.}
 \label{fig:FeHea}
\end{figure*}

\begin{figure*}
 \begin{center}
  \includegraphics[width=16cm]{figures/spectra_others_170925.ps} 
 \end{center}
 \caption{Same as figure~\ref{fig:FeHea}, but for S~Ly$\alpha$ (upper left), Fe~Ly$\alpha$ (upper right) and Fe~He$\beta$ (lower left) of Obs~3+4. Representative line are denoted in the figures.}
 \label{fig:otherlines}
\end{figure*}

In this section, we show observed line profiles of bright transitions
and demonstrate qualities of these measurements. The data of Obs~2 were not used
in this section and in section~\ref{sec:velocity}, since Obs~2 (and Obs~1)
contains a previously known systematic uncertainty in the energy
scale, and the almost identical pointing direction to that of Obs~2's is covered by Obs~3.
In figure~\ref{fig:FeHea} we show the Fe~He$\alpha$ emission line
complex from Obs~3+Obs~4, and Obs~1.  The panels in
figure~\ref{fig:otherlines} show S~Ly$\alpha$, Fe~Ly$\alpha$ and
Fe~He$\beta$ lines of Obs~3+Obs~4. The figures indicate the best-fitting
LOS velocity dispersions ($\sigma_\mathrm{v}$) and bulk velocities
calculated with respect to the new stellar absorption line redshift
measurement of NGC~1275 ($v_{\rm bulk}\equiv(z-0.017284)c_0-26.4~{\rm
km~s}^{-1}$, where $c_0$ is the speed of light, $z=0.017284$ is the
redshift of NGC~1275, and $-26.4$~km~s$^{-1}$ is the heliocentric
correction. See also appendix~\ref{sec:redshift} for the redshift
measurement). The net photon count is also indicated.

The best-fitting parameters were obtained as follows: We extracted
spectra from the event file (no additional gain correction applied) for
the entire FOVs of Obs~3, Obs~4, and Obs~1, and we combined the spectra
of Obs~3 and Obs~4. The spectral continua were modeled using a wider
energy band of 1.8--9.0~keV using
\verb+bapec+, and the obtained continuum parameters were used in the subsequent fitting for extracting the parameter values associated with spectral lines performed in narrower energy bands displayed in figures~\ref{fig:FeHea} and \ref{fig:otherlines}. In the \verb+bapec+
modelling, Fe He$\alpha$ {\it w} was manually excluded from the
atomic database and substituted by an external Gaussian, to
minimize the effect of resonance scattering \cite[most pronounced for
Fe He$\alpha$ {\it w}, see][hereafter RS~paper]{rspaper}. In the
spectral line modelling, Fe He$\alpha$ {\it w}, Ly$\alpha_1$ and
Ly$\alpha_2$, He$\beta_1$ and He$\beta_2$, and S Ly$\alpha_1$ and
Ly$\alpha_2$ were manually excluded from the atomic database and
substituted by external Gaussians. For an Fe Ly$\alpha$ feature, the
widths of the two Gaussians were linked to each other, while for Fe
He$\beta$ and S Ly$\alpha$ features, the relative centroid energies and
the relative normalizations of each of the two Gaussians were also fixed
to the database values.

\begin{table*}
 \tbl{LOS velocity dispersions of gas motions, obtained from the Fe~He$\alpha$ line of Obs~3+Obs~4 data.}{%
 \begin{tabular}{lllll}
  \hline
 & Unit & Without {\it $z$-correction} & With {\it $z$-correction}$^*$ \\
\hline
$\sigma_{\mathrm v}$ of {\it w} & (km~s$^{-1}$) & $171^{+4}_{-3}$ & $161\pm3$\\
$\sigma_{\mathrm v}$ excluding {\it w} & (km~s$^{-1}$) & $148\pm6$ & $144\pm6$\\
  \hline
 \end{tabular}}\label{tab:obs34_width}
 \begin{tabnote}
 $^*${\it $z$-correction} is an additional gain alignment among the detector pixels. See also the text.
 \end{tabnote}
\end{table*}

We investigated the effects of the Fe~He$\alpha$ resonance line ({\it w} line) and the energy scale correction on the measured $\sigma_\mathrm{v}$. Table~\ref{tab:obs34_width} shows the LOS velocity dispersion ($\sigma_\mathrm{v}$) measured with or without {\it $z$-correction} -- a rescaling of photon energies for individual SXS pixels in order to force the Fe He-alpha lines align, which has been employed in H16 and \citet{hitomi17} to cancel out most pixel-to-pixel calibration uncertainties, but which also removes any true LOS velocity gradients. The value of $\sigma_\mathrm{v}$ obtained with the {\it w} line is higher than that without the {\it w} line, which provides a hint of resonance scattering (see RS~paper for details).
