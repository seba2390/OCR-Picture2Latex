\section{Conclusions}
\label{sec:conclusions}

In this paper, we have presented Hitomi observations of the atmospheric gas motions in the core, $r\lesssim100$~kpc, of the Perseus galaxy cluster. Our findings are summarized as follows.

\begin{enumerate}
 \item We have resolved and measured the line widths of He-like and H-like ions of Si, S, Ar, Ca, and Fe in the hot ICM for the first time.

 \item Using the optically thin emission lines and after correcting for the point spread function of the telescope, we find that the line-of-sight velocity dispersion of the hot gas is mostly low and uniform. The line-of-sight velocity dispersion of the hot gas reaches maxima of approximately 200~km~s$^{-1}$ toward the central AGN and toward the AGN inflated north-western `ghost' bubble. Elsewhere within the observed region, the velocity dispersion appears nearly uniform at $\sigma_{\rm v} \sim 100$~km~s$^{-1}$. The systematic uncertainty affecting the best-fit line-of-sight velocity dispersion values is $\lesssim$20~km~s$^{-1}$ (gain), $\lesssim$3~km~s$^{-1}$ (line spread function) and $\lesssim$5~km~s$^{-1}$ (PSF shape) in most cases.

 \item We detect a large scale bulk velocity gradient with an amplitude of $\sim 100$~km~s$^{-1}$ across the cluster center, consistent with sloshing induced motions.

 \item The mean redshift of the hot atmosphere is consistent with that of the stars of the central galaxy NGC~1275.

 \item The shapes of well-resolved optically thin emission lines are consistent with Gaussian. The lack of evidence for non-Gaussian line shapes indicates that the observed velocity dispersion is dominated by small scale motions. Our results imply that the driving scale of turbulence is mostly smaller than $\sim100$~kpc.

 \item If the observed gas motions are isotropic, the kinetic pressure support in the cluster core is smaller than 10\% of the thermal pressure.

 \item Combining the widths of the lines formed from various elements, we have obtained the first direct constraints on the thermal motions of the ions in the hot ICM. We find no evidence of deviation between the ion temperature and the electron temperature.

\end{enumerate}

Owing to the short lifetime of Hitomi, our results are restricted to the central region of a single galaxy cluster. Future X-ray calorimeter missions, e.g., the X-ray Astronomy Recovery Mission (XARM) and Athena \citep{nandra13}, will be crucial for extending the measurements to larger radii and a larger number of clusters, thereby providing further insights into the dynamics of galaxy clusters.
