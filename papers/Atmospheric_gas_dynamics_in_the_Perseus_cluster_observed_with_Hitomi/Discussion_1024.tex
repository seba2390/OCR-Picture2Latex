\section{Discussion}
\label{sec:discussion}

\subsection{The origin of gas motions}
\label{sec:origin}

The Hitomi SXS observations provided the first direct measurements of the LOS velocities and velocity dispersions of the hot ICM in the core of the Perseus cluster. Using the optically thin emission lines, we find that the LOS velocity dispersion peaks toward the cluster center and around the prominent northwestern `ghost' bubble, reaching $\sigma_{\rm v}\sim200$~km~s$^{-1}$. These velocity dispersion peaks are seen in both PSF-corrected and uncorrected maps. Outside of these peaks, the LOS velocity dispersion appears constant at $\sigma_{\rm v}\sim100$~km~s$^{-1}$. Note that the velocity dispersion peak at the center is seen in the maps derived by both methods, excluding and including the resonance {\it w} line (appendix~\ref{sec:details}). The peak toward the ghost bubble is not seen when the {\it w} line is used for the velocity fits (appendix~\ref{sec:details}), so its existence is less certain.

The maximum velocity of 100~km~s$^{-1}$ determined from line shifts within the investigated area indicates that the velocity of large scale flows is at least $v_{\rm bulk}=100$~km~s$^{-1}$. While some theoretical arguments predict a velocity offset of the order of $\sim100$~km~s$^{-1}$ between the central galaxy and the ICM \citep{inoue14}, the zero point of our observed bulk shear is consistent with the redshift of NCG~1275. We note that as the photons produced within the central $r\sim100$~kpc climb up the gravitational potential well of the cluster, they are also affected by a gravitational redshift of $\sim 20$~km~s$^{-1}$. This shift should be considered in the absolute value of each redshift measurement. The $v_{\rm bulk}$ values are relative values between NGC~1275 and the ICM, and so the gravitational redshift is mostly canceled out. The relative gravitational redshift across the FOV is $\sim 5$~km~s$^{-1}$.

During the process of hierarchical structure formation, turbulent gas motions are driven on Mpc scales by mergers and accretion flows which convert their kinetic energy into turbulence \citep[e.g.][]{bruggen2015}. These turbulent motions then cascade down from the driving scales to dissipative scales, heating the plasma, (re-)accelerating cosmic-rays, and amplifying the magnetic fields \citep[e.g.][]{brunetti2007,miniati2015}. In the Perseus cluster, turbulence is also likely to contribute to powering the radio emission of the minihalo \citep{burns92,sijbring93,walker2017} by re-accelerating the relativistic electrons originating from the AGN and/or hadronic interactions \citep[e.g.][]{gitti2002,ZuHone13}.

Turbulence is also expected to be driven on smaller scales by the AGN, galaxy motions, gas sloshing, and hydrodynamic and magneto-thermal instabilities in the ICM \citep[e.g.][]{churazov2002,Gu13,mendygral2012,ichinohe17,ZuHone13,zuhone17}. The low, relatively uniform velocity dispersion observed in the Perseus core is also consistent with that expected for turbulence induced in the cool core by sloshing \citep{ZuHone13}. Several cold fronts are seen in the Perseus X-ray images \citep{Churazov03,simionescu2012,walker2017}, which reveal a sloshing core. If the observed velocity dispersion is indeed mostly sloshing induced, then an interesting prediction for future observations is that the observed dispersion will abruptly change across the cold fronts, which are mostly located outside the Hitomi FOV.

The observed peaks in $\sigma_{\rm v}$ appear to indicate that gas motions are driven both at the cluster center by the current AGN inflated bubbles and by the buoyantly rising ghost bubbles with diameters of $\sim25$~kpc. The observed peaks in $\sigma_{\rm v}$ could be due to superposed streaming motions around the bubbles and turbulence. This observation appears to contradict models in which gas motions are sourced only at the center (during the initial stages of bubble inflation) or only by structure formation. These results may indicate that both the current AGN inflated bubbles in the cluster center and the buoyantly rising ghost bubbles are driving gas motions in the Perseus cluster.

Part of the observed large scale motions of $v_{\rm bulk}\sim 100$~km~s$^{-1}$ might be due to streaming motions around and in the wakes of buoyantly rising bubbles as well. As already pointed out in H16, to the north of the core, the trend in the LOS velocities of the ICM is consistent with the trend in the velocities of the molecular gas within the northern optical emission line filaments \citep{salome2011}. These trends are consistent with the model where the optical emission line nebulae and the molecular gas result from thermally unstable cooling of low entropy gas uplifted by buoyantly rising bubbles \citep[e.g.][]{hatch2006,mcnamara2016}.

However, most of the bulk motions are likely driven by the gas sloshing in the core of the Perseus cluster \citep{Churazov03,walker2017,zuhone17}. The gas sloshing observed in the innermost cluster core, $r\lesssim100$~kpc, might be due to strong AGN outbursts \citep{Churazov03} or due to a disturbance of the cluster gravitational potential caused by a recent subcluster infall \citep[e.g.][]{markevitch2007} which is likely related to the large-scale sloshing in this system \citep{simionescu2012}. The molecular gas can be advected by the sloshing hot gas, resulting in their similar LOS velocities. The shearing motions associated with gas sloshing are also expected to contribute to the velocity dispersion observed throughout the investigated area.

Given the large density gradient in the core of the Perseus cluster, the effective length along the LOS from which the largest fraction of line flux (and measured line width) arises, $L_{\rm eff}$, is rapidly increasing as a function of radius. The increase of the effective length, $L_{\rm eff}$,  with growing projected distance $r$ implies that larger and larger eddies contribute to the observed line broadening. Therefore, as shown by \citet{zhuravleva2012}, for Kolmogorov-like turbulence driven on scales larger than $\sim 100$~kpc, we would expect to see a radially increasing LOS velocity dispersion. For example, for turbulence driven on scales of 200~kpc, we would expect a factor of 1.7 increase in the measured velocity dispersion over the radial range of 100~kpc (from the core out to $r\sim$100~kpc assuming the density profile of the Perseus cluster). The lack of observed radial increase of $\sigma_{\rm v}$ might indicate that the turbulence in the core of the Perseus cluster is driven primarily on scales smaller than $\sim 100$~kpc. The relative uniformity of the dispersion is also consistent with sloshing-induced turbulence, which is mostly limited to the cool core in the absence of large-scale disturbances such as a major merger \citep[see figures~14--16 in][]{ZuHone13}.

While turbulence on spatial scales $L < L_{\rm eff}$ will increase the observed line widths and the measured $\sigma_{\rm v}$, gas motions on scales $L > L_{\rm eff}$ will shift the line centroids. The superposition of large scale motions over the LOS within our extraction area should therefore lead to non-Gaussian features in the observed line shapes \citep[e.g.][]{Inogamov03}. The lack of evidence for non-Gaussian line shapes in the spectral lines extracted over a spatial scale of $\sim$100~kpc (see section~\ref{sec:nongaussianity}) indicates that the observed velocity dispersion is dominated by small scale motions and corroborates the conclusion that, in the core of the cluster, the driving scale of the turbulence is mostly smaller than $\sim100$~kpc.

From a suite of cosmological cluster simulations by \citet{nelson2014}, and an isolated high-resolution cluster simulation with cooling and AGN feedback physics by \citet{gaspari2012}, \citet{lau2017} generated a set of mock Hitomi SXS spectra to study the distribution and the characteristics of the observed velocities. They concluded that infall of subclusters and mechanical AGN feedback are the key complementary drivers of the observed gas motions. While the gentle, self-regulated mechanical AGN feedback sustains significant velocity dispersions in the inner innermost cool core, the large-scale velocity shear at $\gtrsim 50$~kpc is due to mergers with infalling groups. The comparison with their simulations also suggests that the AGN feedback is ``gentle'', with many small outbursts instead of a few isolated powerful ones \citep[see also][]{Fabian06,Fabian12,mcnamara2012,mcnamara2016}.  Similar conclusions were reached in the simulations by \citet{bourne2017}.

\subsection{Kinetic pressure support}
\label{sec:pressure}

One of the key implications of the gas velocities measured in section \ref{sec:analysis} is that hydrostatic equilibrium holds to better than 10\% near the center of the Perseus cluster. The results presented in figure~\ref{fig:velocity_psfcor} suggest that, if the observed velocity dispersion is due to isotropic turbulence, the inferred range of $\sigma_{\rm v} \sim$100--200~km~s$^{-1}$ corresponds to 2--6\% of the thermal pressure support of the gas with $kT = 4$~keV.

The large scale bulk motion will also contribute to the total kinetic energy. Assuming further that the observed line shifts are due to bulk motions with velocities of $v_{\rm bulk} = 100$~km~s$^{-1}$ with respect to the cluster center, the fraction of the kinetic to thermal energy density is
\begin{eqnarray}
\frac{\epsilon_{\mathrm{kin}}}{\epsilon_{\mathrm{therm}}}
= \frac{\mu m_{\rm p}(3\sigma^{2}_{\rm v}+v^{2}_{\rm bulk})}{3kT}
\sim 0.02 - 0.07,\label{eq:kintotherm}
\end{eqnarray}
for $kT=4$~keV, where $\mu=0.6$ is the mean molecular weight, and $m_{\rm p}$ is the proton mass. The expression can also be rewritten as $\epsilon_{\mathrm{kin}}/\epsilon_{\mathrm{therm}}=(\gamma/3)\mathcal{M}^2$, where the Mach number $\mathcal{M}=v_{\rm 3D~eff}/c_{\rm s} \sim 0.19-0.35$, $v_{\rm 3D~eff}=\sqrt{3\sigma^{2}_{\rm v}+v^{2}_{\rm bulk}}$ is the effective three dimensional velocity, $c_{\mathrm{s}}=\sqrt{\gamma kT/\mu m_{\mathrm{p}}} = 1030 (kT/4~{\rm keV})^{1/2}$~km~s$^{-1}$ is the sound speed, and $\gamma=5/3$ is the adiabatic index. The small amount of the kinetic energy density supports the validity of total cluster mass measurements under the assumption of hydrostatic equilibrium \citep[e.g.,][]{Allen11}, at least in the cores of galaxy clusters.

We note, however, that if the velocity dispersion is mostly sloshing induced, we might be underestimating the kinetic energy density. Sloshing in the Perseus cluster appears to be mostly in the plane of the sky and \citet{ZuHone13} show that such a relative geometry results in a total kinetic energy being a factor (5--6)$\sigma_\mathrm{v}^2$, compared to the factor of 3 in Equation~\ref{eq:kintotherm} for isotropic motions. This would change the upper bound of the kinetic to thermal pressure ratio to 0.11--0.13.

\subsection{Maintaining the balance between cooling and heating}
\label{sec:dissip}

The gas in the core of galaxy clusters appears to be in an approximate global thermal balance, which is likely maintained by several heating and energy transport mechanisms taking place simultaneously. One possible source of heat is the central AGN. Relativistic jets, produced by the central AGN drive weak shocks with Mach numbers of 1.2--1.5 \citep[e.g.][]{forman2005,forman2007,forman2017,nulsen2005,simionescu2009a,million2010,randall2011,randall2015} and inflate bubbles of relativistic plasma in the surrounding X-ray-emitting gas \citep[e.g.][]{Boehringer93,Churazov00,Fabian03,Fabian06,birzan2004,dunn2005,forman2005,forman2007,dunn2006,dunn2008,rafferty2006,McNamara07}. The bubbles appear to be inflated gently, with most of the  energy injected by the AGN going into the enthalpy of bubbles and only $\lesssim20$\% carried by shocks \citep{forman2017,zhuravleva2016,tang2017}. After detaching from the jets, the bubbles rise buoyantly and they often entrain and uplift large quantities of low entropy gas from the innermost regions of their host galaxies \citep{simionescu2008,simionescu2009b,kirkpatrick2009,kirkpatrick2011,Werner10,werner2011,mcnamara2016}. All of this activity is believed to take place in a tight feedback loop, where the hot ICM cools and accretes onto the central AGN, leading to the formation of jets which heat the surrounding gas, lowering the accretion rate, reducing the feedback, until the accretion eventually builds up again \citep[for a review see][]{McNamara07,Fabian12}.

Many questions regarding the energy transport from the bubbles to the ICM remain. Part of the energy might be transported by turbulence generated in situ by bubble-driven gravity waves oscillating within the gas \citep[e.g.][]{churazov01}. While g-modes are efficient at spreading the energy azimuthally, they are not able to transport energy radially \citep[e.g.][]{reynolds2015}. Energy can also be carried by bubble-generated sound waves \citep{Fabian03,Fujita05,Sanders07}, which could propagate fast enough to heat the core \citep{fabian2017}. The energy from bubbles can also be transported to the ICM by cosmic ray streaming and mixing \citep[e.g.][]{loewenstein1991,guo2008,fujita11,pfrommer2013,ruszkowski2017,jacob2017} or by mixing of the bubbles \citep[e.g.][]{hillel2016,hillel2017}.
 
The Hitomi SXS observation of the Perseus cluster allows us to explore the role of the dissipation of gas motions in keeping the ICM from cooling. As discussed in section~\ref{sec:origin}, substantial part of the kinetic energy density in the core of the Perseus cluster could be generated by the AGN, which appears to produce a peak in $\sigma_{\rm v}$ toward the cluster center and possibly around the prominent northwestern ghost bubble. Heating by dissipation of turbulence, induced by buoyantly rising (at a significant fraction of the sound speed) AGN-inflated bubbles, provides an attractive regulating mechanism for balancing the cooling of the ICM through a feedback loop \citep[e.g][]{McNamara07}. The rising bubbles are expected to generate turbulence in their wakes and excite internal waves, which propagate efficiently in azimuthal directions and decay to volume-filling turbulence. Based on the analysis of surface brightness fluctuations measured with Chandra, \citet{Zhuravleva14} showed that the heating rate from the dissipation of gas motions is capable of balancing the radiative cooling at each radius in the Perseus cluster. The direct measurements of the velocity dispersion by the Hitomi SXS are broadly consistent with these previous indirect deductions \citep[see figure~11 in][which compares the Chandra results with the earlier measurements reported by H16]{zhuravleva2017}. Note, however, that the dissipation of observed gas motions is capable of balancing radiative cooling only if (i) these motions dissipate in less than 10\% of the cooling timescale ($\sim$ Gyr) and (ii) they are continuously replenished over the age of the Perseus cluster.

Numerical simulations by \citet{ZuHone2010} showed that gas sloshing can facilitate the heat inflow into the core from the outer, hotter cluster gas via mixing, which can be enough to offset radiative cooling in the bulk of the cool core, except the very center. While the dissipation of turbulence induced by mergers \citep{Fujita04} or galaxy motions \citep{balbus1990,Gu13} could also contribute to heating the ICM, they would be unable to maintain a fine-tuned feedback loop.

\subsection{Thermal equilibrium between electrons and ions}

We performed the first measurement of the ICM ion temperature, based on the thermal broadening of the emission lines. We find the ion temperature to be consistent with the electron temperature, albeit with large uncertainties. Equilibration via Coulomb collisions between the ions and electrons takes place over the timescale given by
\begin{eqnarray}
t_{\rm eq} \sim  6 \times10^6 \,{\rm yr} \left(\frac{n_{\rm e}}{10^{-2}\,{\rm cm}^{-3}}\right)^{-1}\left(\frac{kT}{4 ~{\rm keV}}\right)^{3/2},
\label{eq-tie}
\end{eqnarray}
where $n_{\rm e}$ is the number density of electrons \citep{spitzer65,zeldovich66}. The equilibration time scales for electrons and for the ions are much shorter by factors of about $m_{\rm p}/m_{\rm e} \simeq 1800$ and $\sqrt{m_{\rm p}/m_{\rm e}} \simeq 43$, respectively, where $m_{\rm p}$ is the proton mass and $m_{\rm e}$ is the electron mass. Because the ions in the ICM are almost fully ionised and the rate of Coulomb collisions is proportional to the electric charge squared, their equilibration time scale is governed by that of protons; the ions equilibrate with protons faster than protons among themselves. Therefore, if the ICM has equilibrated via Coulomb collisions, equation (\ref{eq-tie}) gives a lower limit to the time elapsed since the last major heat injection. This timescale is much shorter than any relevant merger or AGN-related timescales, thus we did not expect to find a discrepancy between $T_\mathrm{e}$ and $T_\mathrm{ion}$.

%%%%%%%%%%%%%%%%%%%%%%%%%%%%%%%%%%%%%%%
% EOF
%%%%%%%%%%%%%%%%%%%%%%%%%%%%%%%%%%%%%%%
