\section{Analysis and results}
\label{sec:analysis}

In this section, we present the analysis and the results subject-by-subject. Several setups are commonly adopted in most of the analyses unless otherwise stated. The atmospheric X-ray emission was modeled as the emission from a single-temperature, thermal plasma in collisional ionization equilibrium attenuated by the Galactic absorption (\verb+TBabs*bapec+). The absorbing hydrogen column density was fixed to the value obtained from Leiden/Argentine/Bonn (LAB) survey \citep[$N_{\rm H}=0.138\times10^{22}~\mathrm{cm}^{-2}$;][]{Kalberla05}. \citet{willingale13} pointed out the effect of the molecular hydrogen column density on the total X-ray absorption, and the effect increases the hydrogen column density by $\sim$50\% in the case of Perseus cluster. We however ignored the correction because (i) we do not use the energy below 1.8~keV, where the effect becomes significant, and (ii) the effect is almost only on the continuum parameters, whose effects are second-order and thus negligible in determining the velocity parameters. We ignored the spectral contributions of the cosmic X-ray background (CXB) as they are negligible compared to the emission of the Perseus cluster \citep{kilbourne16b}. We also ignored the contributions from the non-X-ray background because Hitomi SXS has a significant effective area at high energies \citep{okajima17}, which makes them negligible compared to the X-ray emission components.

We adopted the abundance table of proto-solar metal of \citet{Lodders09} in this paper. Unless otherwise stated, the fitting was performed using {\small XSPEC} v12.9.1 \citep{Arnaud96} with AtomDB v3.0.9 \citep{smith01,foster12}.

The spectra were rebinned so that each energy bin contained at least one event. C-statistics were minimized in the spectral analysis. The redistribution matrix files (RMFs) were generated using the \verb+sxsmkrmf+ tool\footnote{https://heasarc.gsfc.nasa.gov/lheasoft/ftools/headas/sxsmkrmf.html} in which we incorporated the electron loss continuum channel into the redistribution \citep[extra-large-size RMF;][]{leutenegger16}\footnote{For the analyses shown in the main text. We instead used large-size RMFs for the analyses presented in appendices for computational efficiency. The changes in the best-fit values due to the RMF difference are typically less than a few \%.}. Point source ARFs (auxiliary response file) were generated in the 1.8--9.0~keV band using the \verb+aharfgen+ tool\footnote{https://heasarc.gsfc.nasa.gov/ftools/caldb/help/aharfgen.html} at source coordinates (RA, Dec)$=$(\timeform{3h19m48s.1}, \timeform{+41D30'42''}) (J2000).

Hereafter in this paper, we distinguish various kinds of line width using the following notations: $\sigma_\mathrm{v+th}$ is the observed line width with only the instrumental broadening subtracted; $\sigma_\mathrm{v}$ is the line width calculated by subtracting both the thermal broadening ($\sigma_\mathrm{th}$) and the instrumental broadening from the observed line width (i.e., LOS velocity dispersion). Unless stated otherwise, $\sigma_\mathrm{th}$ is computed assuming that electrons and ions have the same temperature. The analysis without this assumption is presented in section~\ref{sec:iontemperature}.
