\section{Observations and data reduction}
\label{sec:data}

\begin{figure}
 \begin{center}
  \includegraphics[width=8cm]{figures/pointings_rgb-01.eps}
 \end{center}
\caption{Hitomi SXS pointings of the Perseus cluster performed during the commissioning phase, overlaid on the Chandra 0.5--3.5~keV band relative deviation image \citep[reproduced from][]{Zhuravleva14}. The grids correspond to the 6$\times$6 array of the SXS with a lacking corner for the calibration pixel.}\label{fig:fov}
\end{figure}

\begin{table*}
 \tbl{Summary of the Perseus observations}{%
 \begin{tabular}{llllll}
  \hline
  & ObsID & Observation date & Exposure time (ks) & Pointing direction (RA, Dec) (J2000)\\
  \hline
  Obs~1 & 10040010                     & 2016 February 24 & 48.7  & $(\timeform{3h19m29s.8},  \timeform{+41D29'1''.9})$\\
  Obs~2 & 10040020                     & 2016 February 25 & 97.4  & $(\timeform{3h19m43s.6},  \timeform{+41D31'9''.8})$\\
  Obs~3 & 10040030, 10040040, 10040050 & 2016 March 4     & 146.1 & $(\timeform{3h19m43s.8},  \timeform{+41D31'12''.5})$\\
  Obs~4 & 10040060                     & 2016 March 6     & 45.8  & $(\timeform{3h19m48s.2},  \timeform{+41D30'44''.1})$\\
  \hline
 \end{tabular}}\label{tab:obs}
\end{table*}

The Perseus cluster was observed four times with the SXS during Hitomi's commissioning phase (Obs~1, 2, 3 and 4). A protective gate valve, composed of a $\sim$260~$\mu$m thick beryllium layer, absorbed most X-rays below 2~keV and roughly halved the transmission of X-rays above 2~keV \citep{eckart16}. % \citep{okajima17}.
Figure~\ref{fig:fov} shows the footprint of the four pointings superposed on the Chandra 0.5--3.5~keV band relative deviation image \citep[reproduced from][]{Zhuravleva14}. The observations are summarized in table~\ref{tab:obs}. Obs~1 was pointed $\sim$3~arcmin east of the cluster core. Obs~2 and Obs~3, covering the cluster core and centered on NGC~1275, are the only observations analyzed in H16. Obs~4 was pointed $\sim$0.5~arcmin to south-west of the pointing of Obs~2 and Obs~3.

In order to avoid introducing additional systematic uncertainties into our analysis, we have not applied any additional gain correction adopted in other Hitomi Perseus papers \citep[see e.g.][hereafter Atomic~paper]{atomicpaper} unless otherwise quoted. We started the data reduction from the cleaned event list provided by the pipeline processing version 03.01.005.005 \citep{angelini16} with \verb+HEASOFT+ version 6.21. Detailed description of data screening and additional processing steps are described in \citet[][hereafter T~paper]{tpaper} and elsewhere\footnote{ ``The HITOMI Step-By-Step Analysis Guide version 5; https://heasarc.gsfc.nasa.gov/docs/hitomi/analysis/}.
