\begin{abstract}
Extending the earlier measurements reported in Hitomi collaboration (2016, Nature, 535, 117), we examine the atmospheric gas motions within the central 100~kpc of the Perseus cluster using observations obtained with the Hitomi satellite. After correcting for the point spread function of the telescope and using optically thin emission lines, we find that the line-of-sight velocity dispersion of the hot gas is remarkably low and mostly uniform. The velocity dispersion reaches maxima of approximately 200~km~s$^{-1}$ toward the central active galactic nucleus (AGN) and toward the AGN inflated north-western `ghost' bubble. Elsewhere within the observed region, the velocity dispersion appears constant around 100~km~s$^{-1}$. We also detect a velocity gradient with a 100~km~s$^{-1}$ amplitude across the cluster core, consistent with large-scale sloshing of the core gas. If the observed gas motions are isotropic, the kinetic pressure support is less than 10\% of the thermal pressure support in the cluster core. The well-resolved optically thin emission lines have Gaussian shapes, indicating that the turbulent driving scale is likely below 100~kpc, which is consistent with the size of the AGN jet inflated bubbles. We also report the first measurement of the ion temperature in the intracluster medium, which we find to be consistent with the electron temperature. In addition, we present a new measurement of the redshift to the brightest cluster galaxy NGC~1275.
\end{abstract}
