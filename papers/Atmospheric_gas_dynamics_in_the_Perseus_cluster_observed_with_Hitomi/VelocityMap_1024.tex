\subsection{Velocity maps}
\label{sec:velocity}

\begin{figure*}
 \begin{minipage}{0.495\hsize}
  \centering
  \includegraphics[width=8cm]{figures/benchmark_bulk_ai-01.eps}
 \end{minipage}
 \begin{minipage}{0.495\hsize}
  \centering
  \includegraphics[width=8cm]{figures/benchmark_vel_ai-01.eps}
 \end{minipage}
 \caption{Benchmark velocity maps. {\it Left:} bulk velocity ($v_\mathrm{bulk}$) map with respect to $z=0.017284$ (heliocentric correction of $-26.4~\mathrm{km~s^{-1}}$ applied). {\it Right:} LOS velocity dispersion ($\sigma_\mathrm{v}$) map. The unit of the values is km~s$^{-1}$. Chandra X-ray contours are overlaid. The best-fitting value is overlaid on each region. Only Obs~3 is used and PSF correction is not applied.}\label{fig:comparison}
\end{figure*}

Firstly, we extracted the benchmark velocity maps by objectively dividing the 6~pixel$\times$6~pixel array into 9 subarrays of 2$\times$2~pixels and fitted the spectrum of each region independently, in order to compare the effects of the difference in software and data pipeline versions between H16 and this paper. All model parameters apart from the hydrogen column density were allowed to vary. Only Obs~3 was used for the benchmark maps and the fitting was done using a narrow energy range of 6.4--6.7~keV, excluding the energy band corresponding to the resonance line of Fe He$\alpha$ in the observer-frame (6.575--6.6~keV) to avoid the systematics originating from the possible line broadening due to the resonant scattering effect. Figure~\ref{fig:comparison} left shows the bulk velocity ($v_\mathrm{bulk}$) map with respect to $z=0.017284$ (heliocentric correction of $-26.4~\mathrm{km~s^{-1}}$ applied) and figure~\ref{fig:comparison} right shows the LOS velocity dispersion ($\sigma_\mathrm{v}$) map. We found a similar trend to the H16 results.


\begin{figure*}
 \begin{minipage}{0.495\hsize}
  \centering
  \includegraphics[width=8.0cm]{figures/regions_ai_2-01_rgb-01.eps}
 \end{minipage}
 \begin{minipage}{0.495\hsize}
  \centering
  \includegraphics[width=8.0cm]{figures/regions_sky_0822_ai-01.eps}
 \end{minipage}
 \caption{The regions used for the velocity mapping. {\it Left:} distinct regions defined by discrete pixels are identified by color coding and number and overlaid on the Chandra relative deviation image. {\it Right:} the corresponding regions when PSF is taken into account. The Chandra X-ray contours are overlaid. H$\alpha$ contours \citep{conselice01} are also overlaid in white (left) or red (right). The solid-lined polygons are the regions associated with Obs~1 or Obs~3, and the dashed-lined polygons are the regions associated with Obs~4. See also figure~\ref{fig:fov}.}\label{fig:region}
\end{figure*}

Secondly, we extracted the velocity maps from the regions associated with physically interesting phenomena. Figure~\ref{fig:region} shows the regions used for the velocity mapping. Most of the regions correspond to a specific feature pointed out in the literature \citep[e.g.][]{Churazov00,Fabian06,salome2011}: Reg~0 represents the central AGN and the cluster core; Reg~3 covers the northern filaments; and Reg~4 surrounds the northwestern ghost bubble. We excluded Obs~2 in our velocity mapping to avoid potential systematic uncertainties (see appendix~\ref{sec:gain} for details).

The PSF of the telescope (1.2~arcmin HPD) is rather broad, and thus X-ray photons are scattered out of the FOV and into adjacent regions. Also conversely, photons from outside the detector array's footprint are scattered into the array.

In order to account for the scattering from outside the detector array's footprint, we extended the sky areas for Reg~1 and Reg~2 to a radius of $r=3$~arcmin from the central AGN. We extended Reg~3, 4, and 5 to a radius of 3.5~arcmin from the central AGN. Reg~5 and 6 were likewise extended to a radius of 2.5~arcmin from the center of the FOV of Obs~1. Reg~2 included a part of the region of the $r < 2.5$~arcmin circle and Reg~5 also included a part of the region of the $r < 3.5$~arcmin circle. Sky regions are shown in the right panel of figure~\ref{fig:region}. As the level of PSF blending from outside these regions was found to be less than 1~\%, we ignored them. We assumed a uniform plasma properties within each sky region.

\begin{table*}
\tbl{Ratio of PSF blending effect on each integration region in the 6.4--6.7 keV band in units of percent.}{%
\begin{tabular}{ccccccccc}
\hline
& & \multicolumn{7}{c}{Sky region}  \\
& & Sky~0 & Sky~1 & Sky~2 & Sky~3 & Sky~4 & Sky~5 & Sky~6 \\
\hline
\multirow{12}*{\rotatebox[origin=c]{90}{Integration region}}
& Reg~0 Obs~3 & 62.3 & 10.1 & 13.8 & 7.4  & 6.1  & 0.4  & 0.1 \\ % software version v6
& Reg~0 Obs~4 & 64.2 & 16.6 & 10.2 & 5.4  & 3.2  & 0.3  & 0.1 \\
& Reg~1 Obs~3 & 43.9 & 43.3 & 3.0  & 8.3  & 1.2  & 0.2  & 0.1 \\
& Reg~1 Obs~4 & 22.1 & 67.2 & 4.3  & 5.5  & 0.7  & 0.2  & 0.1 \\
& Reg~2 Obs~3 & 10.2 & 2.8  & 65.5 & 1.5  & 12.0 & 7.6  & 0.5 \\
& Reg~2 Obs~4 & 17.8 & 6.5  & 66.5 & 1.5  & 5.7  & 1.9  & 0.2 \\
& Reg~3 Obs~3 & 12.7 & 6.8  & 2.5  & 63.6 & 13.9 & 0.5  & 0.1 \\
& Reg~3 Obs~4 & 22.7 & 15.7 & 2.9  & 51.3 & 7.0  & 0.3  & 0.1 \\
& Reg~4 Obs~3 & 8.2  & 1.8  & 12.6 & 8.5  & 61.5 & 6.8  & 0.5 \\
& Reg~4 Obs~4 & 17.5 & 2.4  & 16.4 & 12.6 & 48.9 & 2.0  & 0.2 \\
& Reg~5 Obs~1 & 1.3  & 0.9  & 17.5 & 0.4  & 4.0  & 60.8 & 15.0 \\
& Reg~6 Obs~1 & 0.8  & 0.8  & 4.4  & 0.4  & 1.6  & 16.0 & 75.9 \\
 \hline
 \end{tabular}}\label{tab:psf}
\begin{tabnote}
Sky regions correspond to the regions shown in the right panel of figure~\ref{fig:region} and integration regions are associated with the regions indicated in the left panel of figure~\ref{fig:region}. The fractions of photons coming from each sky region to one integration region appear in the same row. The level of PSF blending from outside these regions was found to be less than 1~\% and not listed in the table. For example, Reg~1 Obs~3 is strongly affected by scattered photons from Sky~0, and the contamination from Sky~0 to Reg~5 or Reg~6 is almost zero.
\end{tabnote}
\end{table*}

In order to model all the spectra simultaneously, we estimated the relative flux contributions from all the sky regions (figure~\ref{fig:region} right) to every single integration region (figure~\ref{fig:region} left). We measured the quantity of PSF scattering from inside or outside the corresponding sky using \verb+aharfgen+. For the input, we used the deep Chandra image in the broad band of 1.8--9.0~keV and an image in the 6.4--6.7~keV including the line emission only (see appendix~\ref{sec:details}). We show a matrix of its effect in the 6.4--6.7 keV band in table~\ref{tab:psf}. We also checked its effect in the 1.8--9.0 keV band. The trend in the 1.8--9.0 keV band is consistent with that in the 6.4--6.7 keV band.

In order to determine ICM velocities, we fitted spectra from all regions simultaneously, taking scattering into account (see appendix~\ref{sec:xcm} for technical details). We first obtained the PSF-corrected values of the temperature, Fe abundance and normalization of each region. This fitting was done in the energy range of 1.8--9.0~keV, excluding the narrow energy range of 6.4--6.7~keV, and the AGN contribution to the spectra was included using the model shown in \citet[][hereafter AGN~paper]{agnpaper}, after convolution with the point source ARFs. The velocity width and redshift of each plasma model were fixed to 160~km~s$^{-1}$ and 0.017284 respectively. The obtained C-statistic/d.o.f. (degree of freedom) in the continuum fitting is 63146.77/68003. Detailed description of the measurement of the continuum parameters are shown in AGN~paper and T~paper.

After determining the self-consistent parameter set of the continuum as mentioned above, we again fitted all the spectra simultaneously to obtain the parameters associated with spectral lines. This time, the temperatures and normalizations were fixed to the above obtained values, and the Fe abundance, the LOS velocity dispersion and the redshift were allowed to vary. The fitting was done using a narrow energy range of 6.4--6.7~keV, excluding the energy band corresponding to the resonance line in the observer-frame (6.575--6.6~keV). The obtained C-statistic/d.o.f. in the velocity fitting is 2822.38/2896.

\begin{figure*}
 \begin{minipage}{0.495\hsize}
  \centering
  \includegraphics[width=7.5cm]{figures/psfcor_bulk_ai-01.eps}
 \end{minipage}
 \begin{minipage}{0.495\hsize}
  \centering
  \includegraphics[width=7.5cm]{figures/psfcor_vel_ai-01.eps}
 \end{minipage}
 \caption{{\it Left:} PSF corrected bulk velocity ($v_\mathrm{bulk}$) map with respect to $z=0.017284$ (heliocentric correction applied). {\it Right:} PSF corrected LOS velocity dispersion ($\sigma_\mathrm{v}$) map. The unit of the values is km~s$^{-1}$. The Chandra X-ray contours are overlaid.}\label{fig:velocity_psfcor}
\end{figure*}

\begin{figure*}
 \begin{minipage}{0.495\hsize}
  \centering
  \includegraphics[width=7.5cm]{figures/nocor_bulk_ai-01.eps}
 \end{minipage}
 \begin{minipage}{0.495\hsize}
  \centering
  \includegraphics[width=7.5cm]{figures/nocor_vel_ai-01.eps}
 \end{minipage}
 \caption{Same as figure~\ref{fig:velocity_psfcor}, but PSF correction is not applied.}\label{fig:velocity_nocor}
\end{figure*}

\begin{table*}
  \tbl{Best-fitting bulk velocity ($v_\mathrm{bulk}$) and LOS velocity dispersion ($\sigma_\mathrm{v}$) values of with and without PSF correction.}{%
  \begin{tabular}{lllll}
   \hline
   & \multicolumn{2}{c}{PSF corrected} & \multicolumn{2}{c}{PSF uncorrected}\\
  Region & $v_\mathrm{bulk}$ (km~s$^{-1}$) & $\sigma_\mathrm{v}$ (km~s$^{-1}$) & $v_\mathrm{bulk}$ (km~s$^{-1}$) & $\sigma_\mathrm{v}$ (km~s$^{-1}$)\\
  \hline
  Reg~0 & $75_{-28}^{+26}$  & $189_{-18}^{+19}$ & $43_{-13}^{+12}$  & $163_{-10}^{+10}$  \\
  Reg~1 & $46_{-19}^{+19}$  & $103_{-20}^{+19}$ & $42_{-12}^{+12}$  & $131_{-11}^{+11}$  \\
  Reg~2 & $47_{-14}^{+14}$  & $98_{-17}^{+17}$  & $39_{-11}^{+11}$  & $126_{-12}^{+12}$  \\
  Reg~3 & $-39_{-16}^{+15}$ & $106_{-20}^{+20}$ & $-19_{-11}^{+11}$ & $138_{-12}^{+12}$  \\
  Reg~4 & $-77_{-28}^{+29}$ & $218_{-21}^{+21}$ & $-35_{-14}^{+15}$ & $186_{-12}^{+12}$  \\
  Reg~5 & $-9_{-56}^{+55}$  & $117_{-73}^{+62}$ & $-6_{-26}^{+25}$  & $125_{-28}^{+28}$  \\
  Reg~6 & $-45_{-29}^{+29}$ & $84_{-54}^{+44}$  & $-35_{-22}^{+22}$ & $99_{-32}^{+31}$  \\
     \hline
  \end{tabular}}\label{tab:velocity}
\end{table*}

Figure~\ref{fig:velocity_psfcor} shows the obtained velocity maps with PSF correction. The corresponding velocity maps without PSF correction are shown in figure~\ref{fig:velocity_nocor} for comparison. The best-fitting values are listed in table~\ref{tab:velocity}. The heliocentric correction of $-26.4~\mathrm{km~s^{-1}}$ is applied in the bulk velocity maps.

When producing the PSF-corrected maps, the twelve spectra (Obs~3 and Obs~4 for Reg~0 to Reg~4 and Obs~1 for Reg~5 and Reg~6) were fitted simultaneously with all the cross-terms being incorporated through the matrix shown in table~\ref{tab:psf}. The fitting procedure is complex and deconvolution is often unstable. We thus carefully examined the robustness of the results. These included the check of two parameter confidence surfaces based on C-statistics, i.e., redshift vs LOS velocity dispersion, Fe abundance vs redshift, and Fe abundance vs LOS velocity dispersion for each region, and LOS velocity dispersion vs LOS velocity dispersion and redshift vs redshift for each combination of regions. The redshift, LOS velocity dispersion, and Fe abundance are within 0.0165--0.0180, 0.0--250~km~s$^{-1}$, and 0.35--0.85~solar, respectively. We found no strong correlations among parameters and also confirmed that the true minimum was found in the fitting.

In appendix~\ref{sec:details}, we also desrcibe a different method of deriving the velocties that uses only the {\it w} line (which has been excluded in the fit above). It gives qualitatively similar results with the expected higher values of velocity dispersion. Further detailed investigations of the systematic uncertainties and various checks of the results are presented in appendices~\ref{sec:systematic} and \ref{sec:details}.
