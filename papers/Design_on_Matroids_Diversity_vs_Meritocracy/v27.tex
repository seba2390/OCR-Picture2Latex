\documentclass[12pt]{amsart}

\usepackage{latexsym, amssymb, amsmath, amsthm, natbib}
\usepackage[leqno]{amsmath}
\usepackage{xy}
\xyoption{all}
\usepackage{verbatim}

%\usepackage[demo]{graphicx}
%\usepackage{subcaption}
%\usepackage{graphicx}
\usepackage{caption,subcaption}



\usepackage[margin=1in]{geometry}
%\usepackage[left=1in,top=1in,right=1in,bottom=1in]{geometry}

%\usepackage[T1]{fontenc}
\usepackage{mathtools}

\usepackage{tgpagella}
%\usepackage{mathpazo}

\usepackage{setspace}
\usepackage{enumerate}
\usepackage[multiple]{footmisc}
\usepackage{hyperref}
\usepackage{xcolor}

\pdfminorversion=6

\onehalfspacing
%\doublespace


\usepackage{tikz}
\usetikzlibrary{positioning}

\usepackage{pgfplots}
\pgfplotsset{compat=1.15}
\usepackage{mathrsfs}
\usetikzlibrary{arrows}


\tikzset{
  every node/.style    = {
    text centered,
    % font = \footnotesize,
    line width = .5,
    anchor = center,
  },
  every label/.style   = {
    fill = white, anchor = mid,
  },
  every path/.style   = {
    > = stealth
  },
  point/.style args   = {(#1)#2}{
    rounded corners,
    fill = white,
    minimum height = 20,
    minimum width = 10,
    label = { [name = #1] above:#2 },
  },
  point a/.style args   = {(#1)#2}{
    rounded corners,
    fill = white,
    minimum height = 10,
    minimum width = 20,
    label = { [name = #1] right:#2 },
  },
  point b/.style args   = {(#1)#2}{
    rounded corners,
    fill = white,
    minimum height = 10,
    minimum width = 75,
    label = { [name = #1] right:#2 },
  },
}

\usepackage{epigraph}
\setlength{\epigraphrule}{0pt}



\renewcommand{\thesubfigure}{\roman{subfigure}}



\newtheorem{innercustomthm}{Theorem}
\newenvironment{customthm}[1]
  {\renewcommand\theinnercustomthm{#1}\innercustomthm}
  {\endinnercustomthm}

\usepackage{etoolbox}
\patchcmd{\section}{\scshape}{\bfseries}{}{}
\makeatletter
\renewcommand{\@secnumfont}{\bfseries}
\makeatother

\newcommand{\then}{\Rightarrow}
\newcommand{\ta}{\theta}
\newcommand{\corigin}{c'}
\sloppy

\newtheorem{theorem}{Theorem}
\newtheorem{corollary}{Corollary}
\newtheorem{definition}{Definition}
\newtheorem{lemma}{Lemma}
\newtheorem{proposition}{Proposition}
\newtheorem{claim}{Claim}
\newtheorem{assumption}{Assumption}

\theoremstyle{remark}
\newtheorem{remark}{Remark}
\newtheorem{example}{Example}
\newtheorem{observation}{Observation}

\def\D{\mathcal{D}}
\def\A{\mathcal{A}}
\def\X{\mathcal{X}} \def\x{X} \def\y{Y} \def\z{Z} \def\then{\Rightarrow}
\def\S{\mathcal{S}} \def\W{\mathcal{W}} \def\F{\mathcal{F}} \def\C{\mathcal{C}}
\def\cala{\mathcal{A}} \def\calu{\mathcal{U}}  \def\calx{\mathcal{X}} \def\calt{\mathcal{T}}
\def\cale{\mathcal{E}} \def\cals{\mathcal{S}} \def\cali{\mathcal{I}} \def\calf{\mathcal{F}}
\def\calm{\mathcal{M}} \def\calc{\mathcal{C}}
\def\FX{\mathcal{F}(\mathcal{X})}
\def\t{\tilde t}
\def\d{\tilde d}
\def\c{\tilde c}
\def\xii{\hat \xi}
\def\xij{\bar \xi}

\def\pa{\mathbin{P(a)}}
\def\pw{\mathbin{P(w)}}
\def\pm{\mathbin{P(m)}}
\def\pgen{\mathbin{P}}
\def\ep{\epsilon}
\def\hhx{\hat{\hat{X}}}

\def\lad{the law of aggregate demand}
\def\os{\emptyset}
\def\gr{lexicographic}
\def\con{consistency}
\def\im{initial matching}
\def\oconcave{ordinally concave} %%MBY Use: \oconcave{}

\renewcommand{\P}[1]{\mathbin{P_{#1}}}
\newcommand{\R}[1]{\mathbin{R_{#1}}}
\newcommand{\Z}[1]{\mathbin{Z_{#1}}}
\newcommand{\df}[1]{\textbf{\textit{#1}}}
%\newcommand{\df}[1]{\textit{#1}}
\newcommand{\ip}[1]{\langle #1 \rangle}
\newcommand{\setf}[1]{\left\{ #1 \right\}}
%\newcommand{\X}[1]{\mathcal{X_{#1}}}

\newcommand{\norm}[1]{|| #1 ||}
\newcommand{\abs}[1]{\left| #1 \right|}
\newcommand{\wt}[1]{\widetilde{#1}}

\newcommand\fnsep{\textsuperscript{,}}

\newcommand{\chd}{Ch_d}

\newcommand{\ieh}[1]{{\color{magenta} IEH #1}}
\newcommand{\ih}[1]{{\color{purple} IH #1}}
\newcommand{\fk}[1]{{\color{red} FK: #1 }}
\newcommand{\mby}[1]{{\color{blue} MBY: #1 }}
\newcommand{\ky}[1]{{\color{violet} KY: #1 }}


\begin{document}

\title[Design on Matroids]{Design on Matroids: Diversity vs. Meritocracy$^{\dagger}$}

%[Objective Design Meets Market Design]{Objective Design Meets Market Design: Diversity,
%Meritocracy, and Matroids


\author[Hafalir, Kojima, Yenmez, and Yokote] {Isa E. Hafalir \and Fuhito Kojima  \and M. Bumin Yenmez \and Koji Yokote$^{*}$}



\thanks{\emph{Keywords}: Diversity, meritocracy, college admission, matroids, ordinal concavity.\\
We thank the seminar participants at Brown University, Durham University, Michigan State University, Pennsylvania State University,
Washington University in St. Louis, International Conference on Applied Economic Theory Innovation, Iowa University Market Design Workshop, and SAET Conference.
We are grateful to Ryo Shirakawa and Ryosuke Sato for excellent research assistance. 
Fuhito Kojima is supported by the JSPS KAKENHI Grant-In-Aid 21H04979.
Koji Yokote is supported by the JSPS KAKENHI Grant-In-Aid 22J00145.
Hafalir is affiliated with the UTS Business School, University of Technology Sydney, Sydney, Australia; Kojima is with the Department of Economics, the University of Tokyo, Tokyo, Japan;   Yenmez is with the Department of Economics, Boston College, Chestnut Hill, MA, USA;
Yokote is a JSPS Research Fellow affiliated with the Graduate School of Economics, the University of Tokyo, Tokyo, Japan.
Emails: \texttt{isa.hafalir@uts.edu.au}, \texttt{fuhitokojima1979@gmail.com},
\texttt{bumin.yenmez@bc.edu}, \texttt{koji.yokote@gmail.com}.}





\begin{abstract}
We provide optimal solutions to an institution that has dual goals of diversity
and meritocracy when choosing from a set of applications. For example, in college
admissions, administrators may want to admit a diverse class
in addition to choosing students with the highest qualifications. We provide a class of
choice rules that maximize merit subject to attaining a diversity level.
Using this class, we find all subsets of applications on the
diversity-merit Pareto frontier. In addition, we provide two novel
characterizations of matroids.
\end{abstract}


%Keywords: District Integration, Desegregation, Diversity, Balance, Student Welfare
%JEL: C78, D47, D78,

%\date{\today, First draft: July 15, 2017}

\maketitle

%\tableofcontents



\section{Introduction}
\epigraph{\textit{To see high merit and be unable to raise it to office, to raise it but not to give such promotion precedence, is just destiny.}}{-Confucius}

Meritocratic systems in which goods and political power are given to
people based on qualifications rather than their wealth or social status
have been idealized since ancient times. The Chinese philosopher
Confucius argued that those who govern should do so because of merit, not because of inherited status. The Han dynasty adopted Confucianism and
implemented civil service examinations to select and promote government
officials \citep{dien2001}.
The Greek philosopher Plato, in his book
\emph{The Republic}, stated that the wisest should rule, and hence rulers
shall be philosopher kings. A system based on meritocracy, however, may
increase economic inequality and social and political dysfunction,
the so-called \emph{meritocracy trap} \citep{markovits2019}. To decrease the
inequities that exist between different groups in society, affirmative action
and diversity policies have been implemented worldwide \citep{sowell04}.
Therefore, in practice, it is crucial to find a balance between meritocracy
and diversity.

In this paper, we find optimal subsets of applications to an institution that
is not only interested in choosing applications with the most merit but also having a diverse group. The institution ranks applications according to merit.
For example, applicants may take an exam to determine how qualified they are. American universities rank students using SAT scores and other criteria. Meanwhile, the diversity of
a group is given by an index defined as a function of traits that applicants have.
The type of a student specifies the student's traits and may include information about
gender, race, ethnicity, socioeconomic status, and disability status.

Our focus is on \emph{choice rules} that select a subset of each possible set of applications.
We study a class of choice rules that maximize the merit of the selected group
subject to attaining a diversity level. To do so, we start with an extreme member
of this class that maximizes diversity first and then merit among the sets
that maximize diversity. Even though this rule can be defined in very general environments, it may lack basic desirable properties, which may make its implementation infeasible in practice.
Indeed, some institutions, such as universities, get
thousands of applications every year. For example, in fall 2020, the average number
of applications for the ten colleges in the US that received the
most applications was 84,865.\footnote{See \url{https://tinyurl.com/uv6h3jsh} for
more statistics from the U.S. News \& World Report.} Therefore, the choice rule
must be implementable in a computationally efficient way, and its outcome should not
depend on the order in which applications are evaluated, which is the \emph{path-independence}
property of a choice rule \citep{plott1973path}.
To this end, we define the \emph{diversity choice rule} as follows. In the first step, we find
distributions of applicant types that maximize the diversity index. In the second
step, we choose applications one by one using the merit ranking as long as the set
of chosen applications has a distribution smaller than an optimal distribution found in the first step.\footnote{A distribution $\xi$ is smaller than distribution $\tilde \xi$ if every coordinate of $\tilde \xi$ is greater than or equal to the same coordinate of $\xi$.}
By construction, the diversity choice rule always finds a set of applications that
maximizes diversity. However, because it
is myopic in the second step, it does not necessarily maximize the merit of the
chosen set among sets that maximize diversity. To address this problem, we consider a
restriction on the diversity index under which it lexicographically maximizes diversity
and merit, and is computationally fast.

We provide a novel concept of concavity on functions with discrete domains, called
\emph{ordinal concavity}.\footnote{\label{IP-footnote}After the circulation of our paper, we became aware
of a recent work on mathematical optimization by \cite{chenli2020} who introduce
ordinal concavity while calling it \emph{semi-strict quasi M}$^{\natural}$\emph{-concavity}. While the conditions are equivalent, \cite{chenli2020} and our paper study different problems, and the results are logically independent.} Roughly, ordinal concavity requires that, from two different distributions of types, when one gets closer to each other, either the value of the diversity
index strictly increases on at least one side or the value of the diversity
index remains the same on both sides. In this context, getting closer may either mean
adding or subtracting one in a dimension that we start with or the
existence of a second dimension such that we subtract one from a
dimension and add one in the other one.\footnote{In discrete convex analysis there are two approaches used in this context.
The stronger notion always requires the existence of a second dimension. We use the weaker one
that also allows moving towards each other in the first dimension that one starts with. For example, the stronger definition is used in M-convexity and
the weaker one is used in M$^{\natural}$-convexity (see Section \ref{sec:convexity}).} Ordinal concavity is weaker than the two standard concavity notions used in
the discrete optimization literature: M-concavity and
M$^{\natural}$-concavity.\footnote{See Appendix \ref{app:compare} for the definitions of M-concavity and M$^{\natural}$-concavity.} While these two are cardinal, ordinal concavity is an ordinal notion, as it only depends on comparisons of values that the diversity index takes.


When the diversity index is ordinally concave, the diversity choice rule
maximizes merit among all sets of applications that attain the optimal diversity level,
and its outcome can be constructed in polynomial time (Theorem \ref{thm:diversitychoice}).
To prove the first part, we show that the set of maximal distributions in the set of optimal distributions
identified in the first step is well-behaved: Specifically, it satisfies a notion of discrete convexity called \emph{M-convexity} (Lemma \ref{lem:pareto}).\footnote{See Section \ref{sec:convexity} for the definition of M-convexity.} Furthermore,
the myopic addition of contracts in the second step is equivalent to the outcome of a procedure in the combinatorial optimization literature known as the
\emph{greedy algorithm} on a \emph{matroid} that we construct
(Lemmas \ref{lem:matroid} and \ref{lem:equal}).\footnote{See Section \ref{sec:matroid}
for the definitions of the
greedy algorithm and matroids.}
The computational efficiency proof has two main parts.
In the first part, we establish the \emph{maximizer-cut theorem}, which allows us
to dissect the domain of feasible distributions in the search for an optimal distribution
(Theorem \ref{thm:maximizer-cut}). Using this result, we construct the
\emph{domain-reduction algorithm} that
allows us to find an optimal distribution efficiently. In the second part, we construct
a modified version of the diversity choice rule, which is more computationally
tractable than our original definition, and show that finding an outcome of the diversity choice rule takes quadratic time in the number of applications.

A desirable property of choice rules is path independence. Path independence states that applications can be viewed
in batches in any order without changing the final outcome, an appealing property to
institutions that receive many applications. Furthermore, it
guarantees the existence of a desirable matching in two-sided matching markets.\footnote{See,
for example, \cite{chayen17} who study two-sided matching markets where agents have path-independent choice rules.}
We show that the diversity choice rule is path independent when the diversity index is ordinally concave (Theorem \ref{thm:pi}).
In most matching clearinghouses, the deferred-acceptance algorithm of \cite{gale62} is used to assign applicants to institutions.
This algorithm produces a desirable matching when institution choice rules satisfy path independence, and it
is strategy-proof for applicants when institution choice rules further satisfy the
\emph{law of aggregate demand} \citep{hatmil05}. The law of aggregate demand
requires that the number of applications that are chosen weakly increases when
there are more applications (in the sense of set inclusion) to choose from.
The diversity choice rule does not necessarily satisfy the law of aggregate demand
even when the diversity index is ordinally concave. However, if the diversity
index is \emph{monotone} and ordinally concave, then the diversity choice rule
satisfies the law of aggregate demand (Proposition \ref{prop:structure}).

Next, we consider the class of choice rules that maximize merit subject to achieving
a certain (exogenously given) level of diversity.
We first observe that when the diversity index
is capped at a level, the diversity choice rule for
the modified index maximizes merit subject to attaining the diversity level.
Therefore, every choice rule in this class has the same desirable properties as the
diversity choice rule when the modified indices are ordinally concave. We provide a characterization of
diversity indices such that the modified diversity index for every diversity level is ordinally
concave (Proposition \ref{prop:ordinal-equivalence}). Using this class, we provide the \emph{trace algorithm} that
finds all subsets of applications on the diversity-merit Pareto frontier and show that the trace algorithm is \emph{pseudo polynomial}, which
means that the time complexity is polynomial in the largest integer present in the
input data (Theorem \ref{thm:trace}).\footnote{For this result, we assume that the diversity index takes integer values. Any ordinally
concave diversity index can be replaced with another diversity index that takes integer values and is ordinally concave without changing the
diversity choice rule.} The trace algorithm is useful for an
institution that has the dual goals of maximizing diversity and merit, as
it presents the institution with all alternatives on the diversity-merit
Pareto frontier.


One special case of our model is when the university has a utility function
over sets of contracts.\footnote{This can be modeled as a special case of our
model as follows: All agents have different types and
the diversity index takes distinct values on different distributions. Therefore, the
diversity choice rule is uniquely determined by the first step that maximizes diversity.}
In this particular case, the diversity choice rule maximizes diversity on subsets
of a set of available applications.
An immediate corollary of Theorem \ref{thm:pi} is that when the utility function
over sets of applications satisfies ordinal concavity, the choice rule
constructed by maximizing the objective function satisfies path independence.
This conclusion has been shown under M$^{\natural}$-concavity \citep{eguchi2003generalized},
but the generalization under ordinal concavity has not been established before.
In \cite{hakoyeyo22}, we show that if a choice rule is path independent, then there exists an ordinally concave objective function such that
the choice from any set of contracts is equal to the subset that maximizes the objective function among all subsets.\footnote{Our
setting is more general than the setting of \cite{hakoyeyo22} and, therefore, ordinal concavity is also necessary to get path independence.}
Our results apply to markets where institutions have two
distinct goals that may conflict with each other. We state the model in terms of
the main application of college admissions, where universities admit classes
to maximize the merit of the incoming class as well as its diversity.
Other applications include school choice,
hiring by public institutions or private firms, and auctions
with distributional goals (e.g., procurement auctions and spectrum license auctions).

Our paper is related to the recent literature on market-design problems
with distributional objectives. In practice, distributional objectives are
typically implemented by reserving a number of positions for target groups.
In the market-design literature, reserves were introduced and
analyzed by \cite{hayeyi13}, \cite{ehayeyi14}, and \cite{echyen12}.
Distributional objectives play an important role in matching problems with regional
constraints \citep{kamakoji-basic,kamakoji-concepts,kamakoji-iff}.
Another matching market with distributional constraints is interdistrict school choice
\citep{hafalir2022interdistrict}. Unlike these papers we do not focus on a particular policy but
model it as a function on distributions that satisfies ordinal concavity. In a recent work, \cite{hakoye2022} study the
implementation of diversity policies in a constrained efficient mechanism and introduce pseudo
M$^{\natural}$-concavity.\footnote{Pseudo M$^{\natural}$-concavity and ordinal concavity
are logically independent. Some of our results, but not all, also hold under pseudo M$^{\natural}$-concavity.}
Like us they also represent the distributional policy as a function but their research question
is the existence of constraint efficient mechanisms whereas we focus on the optimal choice of individual
institutions such as universities.

The most closely related paper in terms of
motivation to the current work is \cite{imamura2020}, who introduces axioms
to compare meritocracy and diversity of choice rules and uses
these axioms to characterize choice rules with reserves and quotas. Another related paper is
\cite{kojima-tamura-yokoo}, who study two-sided matching markets with agents
that have M$^{\natural}$-concave utility functions and show the existence of
stable matchings in a variety of matching problems with constraints based on properties
of M$^{\natural}$-concave utility functions. Choice rules with reserves and quotas
can be modeled as a special case of our diversity choice rule by choosing the appropriate
diversity index. 
We also provide two novel characterizations
of matroids (Lemma \ref{lem:matchar} and Proposition \ref{prop:matroid})
that may be helpful in other work. See \cite{oxley} for an introduction
to matroid theory.

%\fk{Would it be useful to discuss discrete convex analysis literature in some manner here?}
%
%\mby{Add: ``Discrete convex analysis is relatively new in economic theory. %\cite{murota:dca:2016} provides an excellent review with some applications in economics. In %addition to the papers that we have discussed above, '' Let's add here some more papers using %discrete convex analysis,. Gul and Pasenderforder? Can, Rikky et al.}

We introduce the model in the next section. In Section \ref{sec:math}, we provide
definitions of mathematical concepts and two novel matroid
characterizations. We study the diversity choice rule in Section
\ref{sec:diversity} and its generalization, which maximizes merit
subject to attaining a given diversity level, and the trace algorithm in Section \ref{sec:frontier}.
Finally, in Section \ref{sec:conclusion}, we conclude the paper.



\section{Model}\label{sec:model}
%In this section, we introduce our model and notation.

\subsection{Agents, Distributions, and Types}
Let $\calc$ denote a finite set of academic \df{schools} (or colleges/divisions) in a university and $\cals$
a finite set of \df{students} applying to the university. Each school represents a
major or program that students can apply to. For example, when students are admitted
as ``undecided'' without specifying a major or program, the set $\calc$ is a singleton.
%Each school $c\in \calc$ has a capacity $q_c\in \mathbb{N}$.

There exist a finite set $\calt$ of \df{student types} and a \df{type function}
$\tau : \cals \rightarrow \calt$
that specifies a type $\tau(s)\in \calt$ for each student $s\in \cals$.
A type specifies diversity-related traits that the university cares about. For example,
it can specify gender, race, ethnicity, sexual orientation, disability status, veteran
status, nationality, or family income.

%We assume that each student $s\in \cals$ applies to one school in the university.\footnote{We make this assumption for
%the ease of exposition. In most universities, either students
%apply without choosing a school or major as ``undecided'' (e.g., \href{https://mitadmissions.org/help/faq/majors/}{Massachusetts Institute of Technology}) or required to apply to at most one major (e.g., \href{https://admissions.northwestern.edu/apply/advice/academic-path.html}{Northwestern University}). \mby{Charles' preliminary research says this is the case for all colleges in top 20 universities in US News ranking.}
%Our results go through without this assumption by breaking the ties in merit scores of two contracts associated with the same student.}

Each application is represented by a \df{contract} specifying a school, a student, and
the terms of admissions that may include financial aid information, and
the set of all contracts is denoted by $\mathcal{X}$.
%Since each student $s\in \cals$ applies to one school, there exists only
%one application $x\in \mathcal{X}$ such that $s(x)=s$.
The university has a \df{merit ranking} $\succ$ of contracts, which is a
strict preference relation (linear order) over $\mathcal{X}$.\footnote{Applications that
are strictly less preferred than having an empty seat for the university are
dropped from $\mathcal{X}$. Thus, without loss of generality, we assume that the university
strictly prefers each application in $\mathcal{X}$ to having an empty seat.} The corresponding
weak preference is denoted by $\succeq$, that is, for each
$x,y\in \mathcal{X}$,
$x \mathrel{\succeq} y$ if $x=y$ or $x \mathrel{\succ} y$.
%For each set of contracts $X\subseteq \mathcal{X}$ and school $c\in \mathcal{C}$,
%$X_c$ denotes the set of contracts in $X$ associated with school $c$,
%i.e., $X_c=\{x\in X|\gamma(x)=c\}$.

Let contracts in set $X=\{x_1,\ldots,x_{|X|}\} \subseteq \mathcal{X}$ and set
$Y=\{y_1,\ldots,y_{|Y|}\} \subseteq \mathcal{X}$
be enumerated such that,
\begin{align*}
\mbox{for each} \; i,j\in \{1,\ldots,|X|\}, \qquad  i < j \; &\implies \; x_i \succ x_j, \; \mbox{ and}\\
\mbox{for each} \; i,j\in \{1,\ldots,|Y|\}, \qquad  i < j \; &\implies \; y_i \succ y_j.
\end{align*}
Then, set $X$ \textbf{merit dominates} set $Y$ if $|X|\geq |Y|$ and,
for each $i\in \{1,\ldots,|Y|\}$,
\[
x_i \mathrel{\succeq} y_i.\footnote{\cite{gale1968} uses this partial
order using weights of elements in a matroid to compare the
outcome of the greedy algorithm and independent sets.}
\]

A \df{distribution} $\xi \in \mathbb Z_+^{|\mathcal{C}|\times |\mathcal{T}|}$ is
a vector such that the entry for school $c\in \mathcal{C}$ and type $t\in \mathcal{T}$
is denoted by $\xi_c^t$.\footnote{$\mathbb Z_+$ is the set of non-negative integers including zero.} The entry $\xi_c^t$ is interpreted as the number of
%contracts between
students of type $t\in \calt$ assigned to school $c\in \calc$ at
$\xi$. For a set of contracts $X\subseteq \mathcal{X}$, $\xi(X) \in \mathbb Z_+^{|\mathcal{C}|\times |\mathcal{T}|}$ denotes
the distribution induced from $X$ so that $\xi_c^t(X)$ denotes the number of
contracts between a student of type $t\in \mathcal{T}$ and school
$c\in \mathcal{C}$ in $X$. For each distribution
$\xi \in \mathbb Z_+^{|\mathcal{C}|\times |\mathcal{T}|}$, $\norm{\xi}$ denotes the sum of coordinates of $\xi$. There may be
feasibility constraints on distributions such as capacity constraints for schools.
The set of \textbf{feasible distributions} is denoted by $\Xi^0 \subseteq \mathbb Z_+^{|\mathcal{C}|\times |\mathcal{T}|}$. We assume that the zero vector is in $\Xi^0$.
For each school $c\in \mathcal{C}$
and type $t\in \mathcal{T}$, let $\chi_{c,t}$ denote the distribution where there is one type-$t$ student at school $c$ and there are no other students.

Given a set of distributions $\Xi$ and a distribution $\xi\in \Xi$, we say
that $\xi$ is \textbf{maximal} in $\Xi$ if there exists no
$\tilde{\xi} \in \Xi\setminus \{\xi\}$ such that $\tilde \xi\geq \xi$.
Therefore, the set of maximal distributions in $\Xi$ is given by
$\{\xi\in \Xi|\nexists \xi'\in \Xi \mbox{ such that } \xi'\geq \xi \mbox{ and } \xi'\neq \xi\}$.

There exists a diversity index $f: \Xi^0 \rightarrow \mathbb{R}_+$.\footnote{$\mathbb{R}_+$
is the set of non-negative real numbers including zero.} The diversity index measures how good
a distribution of students is in terms of a diversity goal. Therefore, if $f(\xi)\geq f(\tilde \xi)$, then it means that distribution $\xi$
is as good as distribution $\tilde \xi$ in terms of the diversity goal. Our analysis only depends on the ordinal content of $f$ and not on the cardinal values
it takes. Therefore, a diversity index $f$ and any strictly increasing transformation of $f$ are equivalent for our purposes.\footnote{A function
$g: \mathbb{R} \rightarrow \mathbb{R}$ is strictly increasing if, for each $x,y\in \mathbb{R}$ such that $x>y$, we have $g(x)>g(y)$. We say that a function
$h: \Xi^0 \rightarrow \mathbb{R}_+$ is a strictly increasing transformation of $f$ if, for each $\xi \in \Xi^0$, $h(\xi)=g(f(\xi))$  where $g$ is a strictly increasing function.}

%% TERMINOLOGY: MONOTONICITY FOR ORDER RELATED NOTION AND INCREASING FOR
%% MORE REAL ANALYSIS TYPE NOTION.


\subsection{Choice Rules}
Given a set of applications, the university must determine which subset
of applications to accept. Accordingly, we assume that the university is endowed
with a choice rule that governs its admissions policies.

\begin{definition}
A \textbf{choice rule} is a function $C: 2^{\mathcal{X}} \rightarrow 2^{\mathcal{X}}$ such that,
for each $X \subseteq \mathcal{X}$,
\[ C(X)\subseteq X \quad \mbox{ and } \quad  \xi(C(X))\in \Xi^0.\]
\end{definition}

A choice rule must be such that the distribution of a chosen set is feasible.
Next, we consider a highly desirable property of choice rules.

\begin{definition}
A choice rule $C$ satisfies \textbf{path independence}, if, for
each $X, X' \subseteq \calx$,
\[C(X' \cup X)=C(C(X') \cup X).\footnote{\cite{plott1973path} introduces
path independence as an axiom of rationality in a model of
social choice. See \cite{chayen17} for an application of path independence in a
matching context.}\]
\end{definition}

Path independence guarantees that applications can be viewed in batches in
any order without changing the final outcome, thereby implying that the university
is applying consistent admissions policies regardless of the sequence or composition
of the batches that are considered during the admissions process. Therefore, it
is a desirable property in college admissions (and other applications). Path
independence is equivalent to the conjunction of the \emph{substitutes} condition and
a mild consistency axiom routinely used in matching theory.\footnote{See the proof of
Theorem \ref{thm:pi} for the definitions of these two notions.}

\begin{definition}
A choice rule $C$ satisfies the \textbf{law of aggregate demand} if, for each $X',X \subseteq \calx$
\[X' \supseteq X \; \implies \; |C(X')|\geq |C(X)|.\footnote{\cite{hatmil05} introduce the law of aggregate demand in a matching market with contracts. \cite{alkan03} calls this property \emph{size monotonicity} in a matching context without contracts.}\]
\end{definition}

The law of aggregate demand states that when a university gets more applications, the number of chosen applications cannot decrease.

In the context of assigning students to schools in a centralized clearinghouse,
path independence guarantees that the most commonly used method, the deferred-acceptance algorithm, works well, e.g., it
produces the student-optimal \emph{stable} matching; and if the law of aggregate demand is also satisfied, it is
\emph{strategy-proof} \citep{hatmil05}.\footnote{In this context, only students are strategic agents.
Therefore, a direct revelation mechanism is strategy-proof if reporting their
true preference ranking over schools is a weakly dominant strategy for each student.}

\section{Mathematical Preliminaries}\label{sec:math}
\subsection{Matroids and the Greedy Rule}\label{sec:matroid}
In this section, we first follow \cite{oxley} to introduce some basic definitions.
Then we provide two novel characterizations of matroids.

A \emph{matroid} is a pair $(\calx,\calf)$ where $\calx$ is a finite set of contracts
and $\calf$ is a collection of subsets of $\calx$ that satisfies the following
three properties.
\begin{description}
  \item[I1] $\emptyset \in \calf$.
  \item[I2] If $X \in \calf$ and $X' \subseteq X$, then $X' \in \calf$.
  \item[I3] If $X_1, X_2 \in \calf$ and $|X_1|<|X_2|$, then there is $x \in X_2 \setminus X_1$
  such that $X_1\cup \{x\} \in \calf$.
\end{description}

Set $\calx$ is called the \emph{ground set} of the matroid. Every set
in $\calf$ is called an \emph{independent set}. An independent set $X$
is called a \emph{base} if no proper superset of $X$ is independent.
\emph{I3} implies that all bases of a matroid have the same cardinality.
In addition, the set of bases $\mathcal{B}$ is characterized by the
following two properties.
\begin{description}
  \item[B1] $\mathcal{B}$ is non-empty.
  \item[B2] If $X_1$ and $X_2$ are in $\mathcal{B}$ and $x_1 \in X_1 \setminus X_2$, then there
  exists an element $x_2$ of $X_2 \setminus X_1$ such that $(X_1\setminus \{x_1\}) \cup \{x_2\} \in \mathcal{B}$.
\end{description}
More precisely, if $(\calx,\calf)$ is a matroid, then the set of its bases satisfies
\emph{B1} and \emph{B2}; moreover, if a collection of subsets $\mathcal{B}$
satisfies \emph{B1} and \emph{B2}, then there exists a matroid of which
$\mathcal{B}$ is the set of bases.
The stronger version of \emph{B2} where the implication is
$(X_1\setminus \{x_1\}) \cup \{x_2\} \in \mathcal{B}$ and
$(X_2\setminus \{x_2\}) \cup \{x_1\} \in \mathcal{B}$ also holds \citep{brualdi1969}. \label{strongerB2}
We next consider a weaker version of \emph{B2} that we call \emph{B2'}.
\begin{description}
  \item[B2'] If $X_1$ and $X_2$ are in $\mathcal{B}$ and $x_1 \in X_1 \setminus X_2$, then there exist an element $x_2$ of $X_2 \setminus X_1$ and $Y\in \mathcal{B}$ such that
  $(X_1\setminus \{x_1\}) \cup \{x_2\} \subseteq Y$.
\end{description}
That is, we weaken the condition \emph{B2} by requiring
$(X_1\setminus \{x_1\}) \cup \{x_2\}$ is only a \textit{subset} of
an element of $\mathcal B$.

In the next lemma, we provide a new characterization for the set of bases of a matroid.

\begin{lemma}\label{lem:matchar}
Let $\mathcal{B}$ be a collection of subsets of $\mathcal{X}$. Then
$\mathcal{B}$ is the collection of bases of a matroid on $\mathcal{X}$
if, and only if, B1 and B2' hold.
\end{lemma}

As already mentioned, it is well known that \emph{B1} and \emph{B2} provide
a characterization for the set of bases. In our proof, we show that
\emph{B1} and \emph{B2'} imply \emph{B2}. Therefore, \emph{B1} and \emph{B2'}
provide another characterization of the set of bases, which is easier to
check than \emph{B1} and \emph{B2} since \emph{B2'} is weaker than \emph{B2}.
We use this characterization in our proofs, and we note that this is
a novel characterization which may be of independent interest and prove
useful elsewhere.

The following is a well-known algorithm, referred to as the greedy
algorithm in the combinatorial-optimization literature.
To define it, we assume that
there exists a weight function on the set of contracts that assigns a distinct real
number to every contract. By changing the set of available contracts,
we get a well-defined choice rule. Therefore, we refer to it as the \emph{greedy rule}.

\medskip
\paragraph{\textbf{Greedy Rule}}
\begin{description}
  \item[Input] Let $X$ be a set of contracts and $\calf$ be a collection of subsets of $\mathcal{X}$.
  \item[Step 1] Set $X_0=\emptyset$ and $k=0$.
  \item[Step 2] If there exist $x \in X \setminus X_k$ and $Y\in \calf$ such that $X_k \cup \{x\} \subseteq Y$,
        then choose such a contract $x_{k+1}$ with the highest non-negative weight,
        let $X_{k+1}=X_k \cup \{x_{k+1}\}$, and go to Step 3.\footnote{A more common definition of the greedy rule requires  $X_k \cup \{x\} \in \calf$ instead of the existence of $Y \in \calf$ with $X_k \cup \{x\} \subseteq Y$. Clearly, that definition is equivalent to the present definition if $(X, \calf)$ satisfies I2, and hence in particular, if it is a matroid.} Otherwise, go to Step 4.
  \item[Step 3] Add 1 to $k$ and go to Step 2.
  \item[Step 4] Return $X_{k+1}$ and stop.
\end{description}

When $(\calx,\calf)$ is a matroid, the greedy rule produces an independent set
that maximizes the total weight among all independent sets that can be
chosen. We next provide a new
characterization of matroids using properties of the greedy rule.

\begin{proposition}\label{prop:matroid}
Let $\mathcal{F}$ be a nonempty collection of subsets of $\mathcal{X}$. The following statements are equivalent.
\begin{enumerate}
\item $(\mathcal{X},\mathcal{F})$ is a matroid.
\item For all weight functions on $\mathcal{X}$, the greedy rule satisfies path independence.
\item For all weight functions on $\mathcal{X}$, the greedy rule satisfies path independence and the law of aggregate demand.
\end{enumerate}
\end{proposition}

If $(\mathcal{X},\mathcal{F})$ is a matroid, then the greedy rule satisfies
path independence \citep{fleiner2001} and the law of aggregate
demand \citep{yokoi2019}. Therefore, (1) implies (3). Furthermore, (3) implies
(2) trivially. In our proof, we show that if the greedy rule satisfies
path independence for all weight functions on $\mathcal{X}$, then
$(\mathcal{X},\mathcal{F})$ is a matroid using our matroid characterization
above (Lemma \ref{lem:matchar}), completing the proof.

\subsection{Convexity for Discrete Sets}\label{sec:convexity}
We use two notions of convexity for discrete sets. See \cite{Murota:SIAM:2003} for intuition and details. The first convexity notion is
M-convexity.\footnote{The letter M in the term M-convex set comes from the
word matroid.}

\begin{definition}
A set of distributions $\Xi$ is \textbf{M-convex} if, $\xi,\tilde{\xi} \in \Xi$ and $\xi_c^t>\tilde{\xi}_c^t$ for some $(c,t) \in \calc \times \calt$,
then there exists $(c',t') \in \calc \times \calt$ with $\xi_{c'}^{t'}<\tilde{\xi}_{c'}^{t'}$ such that
\[\xi-\chi_{c,t}+\chi_{c',t'}\in \Xi \; \mbox{ and } \; \tilde{\xi}+\chi_{c,t}-\chi_{c',t'} \in \Xi.\]
\end{definition}

The second convexity notion is a weakening of M-convexity.

\begin{definition}
A set of distributions $\Xi$ is M$^{\natural}$\textbf{-convex} if, $\xi,\tilde{\xi} \in \Xi$ and $\xi_c^t>\tilde{\xi}_c^t$ for some
$(c,t) \in \calc \times \calt$, then either
\begin{enumerate}[(i)]
\item $\xi-\chi_{c,t} \in \Xi$ and $\tilde{\xi}+\chi_{c,t} \in \Xi$, or
\item there exists $(c',t') \in \calc \times \calt$ with  $\xi_{c'}^{t'}<\tilde{\xi}_{c'}^{t'}$ such that
\[\xi-\chi_{c,t}+\chi_{c',t'}\in \Xi \; \mbox{ and } \; \tilde{\xi}+\chi_{c,t}-\chi_{c',t'} \in \Xi.\]
\end{enumerate}
\end{definition}

In the discrete convex analysis literature, $\chi_{\emptyset}$ is used to denote the distribution with zero coordinates. The notation allows for more compact formulations.
For example, M$^{\natural}$-convexity can be written as: A set of distributions $\Xi$ is M$^{\natural}$-convex if, $\xi,\tilde{\xi} \in \Xi$ and
$\xi_c^t>\tilde{\xi}_c^t$ for some $(c,t) \in \calc \times \calt$, then there exists $(c',t') \in (\calc \times \calt) \cup \{\emptyset\}$
(with $\xi_{c'}^{t'}<\tilde{\xi}_{c'}^{t'}$ whenever $(c',t')\neq \emptyset$) such that
\[\xi-\chi_{c,t}+\chi_{c',t'}\in \Xi \; \mbox{ and } \; \tilde{\xi}+\chi_{c,t}-\chi_{c',t'} \in \Xi.\]
We use the notation $\chi_{\emptyset}$ in the subsequent parts of the paper.

The following lemma shows that a similar relation to the one between independent sets and bases also holds between M$^{\natural}$-convex sets
and M-convex sets.\footnote{Theorem 2.3 in \cite{fujishige2005submodular} proves that the set of maximal elements in an integral g-polymatroid is an integral base polyhedron.
An integral g-polymatroid is a convex hull of an M$^\natural$-convex set and an integral base polyhedron is a convex hull of an M-convex set. One can prove Lemma \ref{lem:mconvex} by using this result. In Appendix \ref{app:proofs}, we provide an independent proof based on \cite{murotashioura2018}.}


\begin{lemma}\label{lem:mconvex}
The set of maximal distributions in an M$^{\natural}$-convex set is M-convex.
\end{lemma}


\section{A Lexicographic Approach to Diversity and Merit}\label{sec:diversity}
In this section, we introduce ordinal concavity and a new choice rule that lexicographically maximizes
diversity first and merit second;\footnote{In Section \ref{sec:frontier}, we generalize this admissions policy so that the
university maximizes merit subject to attaining a diversity level.}
provide a sketch of Theorem \ref{thm:diversitychoice} by making connections to notions of discrete convexity, matroid theory, and the greedy rule;
and establish some further desired properties of this choice rule.

\subsection{Diversity Choice Rule}\label{sec:choice}
In the following choice rule, we first maximize the diversity index among
subsets of any given set of applications. Then we choose applications according to their
merit ranking one by one as long as the chosen set of contracts can be completed
to a subset of applications maximizing diversity.


\medskip
\paragraph{\textbf{Diversity Choice Rule} $\mathbf{C^d}$}

\begin{description}
  \item[Input] Let $X$ be a set of contracts.
  \item[Step 1] $\underset {\xi\in \Xi^0} {\max} \: f(\xi)$ subject to $0 \leq \xi \leq \xi(X)$. Let $\Xi^*(X)$ be the set of distributions that solve this maximization problem. Set $X_0=\emptyset$ and $k=0$.
  \item[Step 2] If there exist $x \in X \setminus X_k$ and $\xi \in \Xi^*(X)$ such that $\xi(X_k \cup \{x\}) \leq \xi$, then choose such a contract
      $x_{k+1}$ of highest merit, let $X_{k+1}=X_k \cup \{x_{k+1}\}$, and go
      to Step 3. Otherwise, go to Step 4.
  \item[Step 3] Add 1 to $k$ and go to Step 2.
  \item[Step 4] Return $X_k$ and stop.
\end{description}

The algorithm ends at a finite index $k$ since the number of contracts is finite.

By construction, the diversity choice rule always produces an outcome that maximizes
diversity among subsets of the set of applications. However, Step 2 of the
diversity choice rule is myopic in choosing contracts, so
it need not produce an outcome that maximizes merit
among the sets that maximize diversity.
To address this problem, we make the following assumption on the diversity index.

\begin{comment}
\begin{definition}\label{def:ordinal}
The diversity index $f: \Xi^0 \rightarrow \mathbb{R}_+$ is \textbf{ordinally concave} if, for every $\xi,\tilde{\xi}\in \Xi^0$
and $(c,t) \in \calc \times \calt$ with  $\xi_c^t>\tilde{\xi}_c^t$, then one of the following holds:
\begin{enumerate}[(i)]
\item $f(\xi-\chi_{c,t})> f(\xi)$, or
\item $f(\tilde{\xi}+\chi_{c,t}) > f(\tilde{\xi})$, or
\item $f(\tilde{\xi}+\chi_{c,t})=f(\tilde{\xi})$ and $f(\xi-\chi_{c,t})=f(\xi)$,
\end{enumerate}
or there exists $(c',t') \in \calc \times \calt$ with $\xi_{c'}^{t'}<\tilde{\xi}_{c'}^{t'}$ such that one of the following holds:
\begin{enumerate}[(i)]
\item[(iv)] $f(\xi-\chi_{c,t}+\chi_{c',t'})> f(\xi)$, or
\item[(v)] $f(\tilde{\xi}+\chi_{c,t}-\chi_{c',t'}) > f(\tilde{\xi})$, or
\item[(vi)] $f(\tilde{\xi}+\chi_{c,t}-\chi_{c',t'})=f(\tilde{\xi})$ and $f(\xi-\chi_{c,t}+\chi_{c',t'})=f(\xi)$.
\end{enumerate}
\end{definition}
\end{comment}

\begin{definition}\label{def:ordinal}
The diversity index $f: \Xi^0 \rightarrow \mathbb{R}_+$ is \textbf{ordinally concave} if, for each $\xi,\tilde{\xi}\in \Xi^0$
and $(c,t) \in \calc \times \calt$ with  $\xi_c^t>\tilde{\xi}_c^t$, there exists $(c',t') \in (\calc \times \calt) \cup \{\emptyset\}$
(with $\xi_{c'}^{t'}<\tilde{\xi}_{c'}^{t'}$ whenever $(c',t')\neq \emptyset$) such that
\begin{enumerate}[(i)]
\item $f(\xi-\chi_{c,t}+\chi_{c',t'})> f(\xi)$, or
\item $f(\tilde{\xi}+\chi_{c,t}-\chi_{c',t'}) > f(\tilde{\xi})$, or
\item $f(\tilde{\xi}+\chi_{c,t}-\chi_{c',t'})=f(\tilde{\xi})$ and $f(\xi-\chi_{c,t}+\chi_{c',t'})=f(\xi)$.
\end{enumerate}
\end{definition}
Each condition in the definition above not only imposes the stated inequality
or equations, but also that the arguments of $f$ are in the domain $\Xi^0$.

To our knowledge, ordinal concavity is new to the economics literature.
In Appendix \ref{app:compare}, we show that ordinal concavity is weaker than
the existing discrete concavity notions.\footnote{As mentioned in footnote \ref{IP-footnote},
we recently became aware of \cite{chenli2020} who introduce \emph{semi-strict quasi M}$^{\natural}$\emph{-concavity}, which is equivalent to ordinal concavity. Meanwhile,
\cite{chenli2020} focus on optimization problems as opposed to economics problems, and their results are logically unrelated to ours.}




\begin{figure}[htb]
    \centering
    \begin{subfigure}{0.4\linewidth}
    \centering
    \begin{tikzpicture}[scale = .8]
  % points
  \coordinate (O)  at (0, 0);
  \coordinate (x0) at (6, 4);
  \coordinate (x1) at (4.5, 5);

  % axes
  \draw [->] (-0.5, 0) -- (6.6, 0);
  \draw [->] (0, -0.5) -- (0, 5.6);

  % points
  \fill (x0) circle (2.5pt);
  \fill (x1) circle (2.5pt);

  % captions
  \draw [dashed] (6, 0) -- (x0) -- (0, 4) node [left] {$f(\xi)$};
  \draw [dashed] (4.5, 0) -- (x1) -- (0, 5) node [left] {$f(\xi - 1)$};

  % scales
  \draw (1.5, .1) -- (1.5, -.1) node       [below] {$\tilde \xi$};
%  \draw (3, .1)   -- (3, -.1)   node       [below] {$\tilde \xi + 1$};
  \draw (4.5, .1) -- (4.5, -.1) node (x1c) [below] {$\xi - 1$};
  \draw (6, .1)   -- (6, -.1)   node (x0c) [below] {$\xi$};

  % arrows
  \draw [->] (x0c) -- (x1c);
\end{tikzpicture}
    \caption{$f(\xi-1)>f(\xi)$.}
    \end{subfigure}
    \hfill
    \begin{subfigure}{0.4\linewidth}
    \centering
    \begin{tikzpicture}[scale = .8]
  % points
  \coordinate (O)   at (0, 0);
  \coordinate (tx0) at (1.5, 4);
  \coordinate (tx1) at (3, 5);

  % axes
  \draw [->] (-0.5, 0) -- (6.6, 0);
  \draw [->] (0, -0.5) -- (0, 5.6);

  % points
  \fill (tx0) circle (2.5pt);
  \fill (tx1) circle (2.5pt);

  % captions
  \draw [dashed] (1.5, 0) -- (tx0) -- (0, 4) node [left] {$f(\tilde \xi)$};
  \draw [dashed] (3, 0)   -- (tx1) -- (0, 5) node [left] {$f(\tilde \xi + 1)$};

  % scales
  \draw (1.5, .1) -- (1.5, -.1) node (tx0c) [below] {$\tilde \xi$};
  \draw (3, .1)   -- (3, -.1)   node (tx1c) [below] {$\tilde \xi + 1$};
  %\draw (4.5, .1) -- (4.5, -.1) node (x1c)  [below] {$\xi - 1$};
  \draw (6, .1)   -- (6, -.1)   node (x0c)  [below] {$\xi$};

  % arrows
  %\draw [->] (x0c)  -- (x1c);
  \draw [->] (tx0c) -- (tx1c);
    \end{tikzpicture}
    \caption{$f(\tilde \xi+1)>f(\tilde \xi)$.}
    \end{subfigure}
    \hfil
    \begin{subfigure}{0.4\linewidth}
    \begin{tikzpicture}[scale=.8]%[> = stealth, line width = .5, scale = .9, font = \scriptsize]
  % points
  \coordinate (O)   at (0, 0);
  \coordinate (x0)  at (6, 5);
  \coordinate (x1)  at (4.5, 5);
  \coordinate (tx0) at (1.5, 4);
  \coordinate (tx1) at (3, 4);

  % axes
  \draw [->] (-0.5, 0) -- (7.6, 0);
  \draw [->] (0, -0.5) -- (0, 6.6);

  % points
  \fill (x0)  circle (2.5pt);
  \fill (x1)  circle (2.5pt);
  \fill (tx0) circle (2.5pt);
  \fill (tx1) circle (2.5pt);

  % captions
  \draw [dashed] (4.5, 0) -- (x1)  -- (0, 5) node [left] {$f(\xi) = f(\xi - 1)$};
  \draw [dashed] (6, 0)   -- (x0)  -- (x1);
  \draw [dashed] (1.5, 0) -- (tx0) -- (0, 4)
    node [left] {$f(\tilde \xi) = f(\tilde \xi + 1)$};
  \draw [dashed] (3, 0)   -- (tx1) -- (tx0);

  % scales
  \draw (1.5, .1) -- (1.5, -.1) node (tx0c) [below] {$\tilde \xi$};
  \draw (3, .1)   -- (3, -.1)   node (tx1c) [below] {$\tilde \xi + 1$};
  \draw (4.5, .1) -- (4.5, -.1) node (x1c)  [below] {$\xi - 1$};
  \draw (6, .1)   -- (6, -.1)   node (x0c)  [below] {$\xi$};

  % arrows
  \draw [->] (x0c)  -- (x1c);
  \draw [->] (tx0c) -- (tx1c);
\end{tikzpicture}
    \caption{$f(\xi-1)=f(\xi)$ and $f(\tilde \xi+1)=f(\tilde \xi)$.}
    \end{subfigure}
\caption{Three possible implications of ordinal concavity for univariate functions.}
\label{fig:univariate}
\end{figure}

To give the intuition for ordinal concavity, let us consider a special
case when there are only one school and one type. Hence, a distribution
specifies how many students are assigned to the university. For
simplicity, take $\Xi^0=\mathbb Z_+$. Consider $\xi, \tilde{\xi}
\in \mathbb Z_+$ such that $\xi>\tilde{\xi}$.
Since $\xi>\tilde{\xi}$ ordinal concavity implies that either
\begin{enumerate}[(i)]
  \item $f(\xi-1)>f(\xi)$, or
  \item $f(\tilde{\xi}+1)>f(\tilde{\xi})$, or
  \item $f(\xi-1)=f(\xi)$ and $f(\tilde{\xi}+1)=f(\tilde{\xi})$.
\end{enumerate}

In words, when we move from $\xi$ and $\tilde{\xi}$ towards each other
by one, either the value of $f$ increases on at least one side or the value of $f$
stays the same on both sides. We illustrate these three possibilities in Figure \ref{fig:univariate}.\footnote{It is
easy to verify that condition (iii) can happen only when $f(\xi)=f(\tilde \xi)$ in the
special case when $|\calc|=|\calt|=1$.}
For example, if $f$ is a concave or strictly increasing (or decreasing)
function on the real line, then its restriction on integers is ordinally concave.
It is also satisfied when $f$ represents a single-peaked preference
relation.


\begin{figure}[htb]
    \centering
    \begin{subfigure}{0.4\linewidth}
    \centering
    \begin{tikzpicture}[scale=.8]
  % points
  \coordinate (O)   at (0, 0);
  \coordinate (x0)  at (6, 2);
  \coordinate (x1)  at (4, 2);
  \coordinate (tx0) at (1, 5);
  \coordinate (tx1) at (3, 5);

  % axes
  \draw [->] (-0.4, 0) -- (7.6, 0) node [below] {$x_c^t$};
  \draw [->] (0, -0.4) -- (0, 6.6) node [left]  {$x_{c'}^{t'}$};

  % labels
  \node (x0n)  [point = {(x0l) $\xi$}]                    at (x0)  {};
  \node (x1n)  [point = {(x1l) $\xi - \chi_{c, t}$}]         at (x1)  {};
  \node (tx0n) [point = {(tx0l) $\tilde \xi$}]            at (tx0) {};
  \node (tx1n) [point = {(tx1l) $\tilde \xi + \chi_{c,t}$}] at (tx1) {};

  % points
  \fill (x0)  circle (2.5pt);
  \fill (x1)  circle (2.5pt);
  \fill (tx0) circle (2.5pt);
  \fill (tx1) circle (2.5pt);

  % arrows
  \draw [->] (x0n)  -- (x1n);
  \draw [->] (tx0n) -- (tx1n);
\end{tikzpicture}
\end{subfigure}
    \hfil
    \begin{subfigure}{0.4\linewidth}
    \centering
\begin{tikzpicture}[scale=.8]
  % points
  \coordinate (O)   at (0, 0);
  \coordinate (x0)  at (6, 1);
  \coordinate (x1)  at (4.5, 2.5);
  \coordinate (tx0) at (1.5, 5.5);
  \coordinate (tx1) at (3, 4);

  % axes
  \draw [->] (-0.4, 0) -- (7.6, 0) node [below] {$x_c^t$};
  \draw [->] (0, -0.4) -- (0, 6.6) node [left]  {$x_{c'}^{t'}$};

  % labels
  \node (x0n)  [point a = {(x0l) $\xi$}]
  at (x0)  {};
  \node (x1n)  [point   = {(x1l) $\xi - \chi_{c, t} + \chi_{c', t'}$}]
  at (x1)  {};
  \node (tx0n) [point   = {(tx0l) $\tilde \xi$}]
  at (tx0) {};
  \node (tx1n) [point b = {(tx1l) $\mbox{ }\tilde \xi + \chi_{c, t} - \chi_{c', t'}$}]
  at (tx1) {};

  % points
  \fill (x0)  circle (2.5pt);
  \fill (x1)  circle (2.5pt);
  \fill (tx0) circle (2.5pt);
  \fill (tx1) circle (2.5pt);

  % arrows
  \draw [->] (x0n)  -- (x1n);
  \draw [->] (tx0n) -- (tx1n);

  % dashed lines
  \draw [dashed] (x0)  -- (4.5, 1) -- (x1);
  \draw [dashed] (tx0) -- (1.5, 4) -- (tx1);
\end{tikzpicture}
\end{subfigure}
\caption{Two possible directions for multivariate functions.}
\label{fig:multivariate}
\end{figure}

When there are more schools and types so that distributions are multidimensional,
moving closer to each other may mean either moving in one direction as in the one-dimensional case above,
or it may mean the existence of another dimension
so that from one distribution, we remove one in one direction and add one in
the other direction and we do the opposite operations on the other distribution.  We show both possible ways in Figure \ref{fig:multivariate}.

Our first result shows that, when $f$ is \oconcave{}, the diversity choice rule
lexicographically maximizes diversity first and merit second in a computationally efficient way.

\begin{theorem}\label{thm:diversitychoice}
Suppose that the diversity index $f$ is \oconcave{}.\footnote{By inspection of the proof,
one can verify that the conclusions of parts (i) and (ii) of the result hold under
a weaker condition than ordinal concavity. More specifically, these conclusions hold
if one of the conditions in the definition of ordinal concavity holds when $f(\xi)=f(\tilde \xi)$.} Then, for each set of contracts $X\subseteq \calx$,
\begin{enumerate}[(i)]
\item $C^d(X)$ maximizes the diversity index $f$ among subsets of $X$,
\item $C^d(X)$ merit dominates each subset of $X$ that maximizes the
diversity index $f$, and
\item
$C^d(X)$ can be calculated in $O(|\mathcal{C}| \times |\mathcal{T}| \times |X|^2)$, assuming
$f$ can be evaluated in a constant time.
\end{enumerate}
\end{theorem}
Next, we provide examples of ordinally concave diversity indices. The first is a simple illustrative example that we use throughout
the paper.


\begin{example}\label{ex:ladfail}
Suppose that there are three students of different types and one school, say $c$.
There is only one contract between each student and the university.
Denote these contracts by $x$, $y$, and $z$. %, and so $\calx=\{x,y,z\}$.
The university has a capacity of two, so $\Xi^0=\{\xi:\norm{\xi} \leq 2\}$
is the set of feasible distributions.

Let the diversity index $f$ be defined as follows:
\[f(\xi(\emptyset))=0, f(\xi(\{x\}))=1, f(\xi(\{y\}))=1, f(\xi(\{z\}))=n,\]
\[f(\xi(\{x,y\}))=1, f(\xi(\{x,z\}))=5, \text{ and } f(\xi(\{y,z\}))=5\]
where $n \geq 5$.\footnote{We consider different values of $n$
in the subsequent sections to illustrate different results.}
To see that $f$ is ordinally concave, we need to consider different cases
depending on the value of $\xi$ in the definition.
Here, we  only consider the first of several cases for illustration,
namely the case with $\xi=\xi(\{x,y\})$, whereas
in Appendix \ref{app:proofs} we provide a full proof.

Let $\xi=\xi(\{x,y\})$. Let $t\in \calt$ be the type of the
student associated with contract $x$ and $t'\in \calt$ be the type of the student
associated with contract $z$.
If $\tilde{\xi}_{c}^{t'}=0$, then $\tilde{\xi}=\xi(\emptyset)$
or $\tilde{\xi}=\xi(\{y\})$. For $\tilde \xi=\xi(\emptyset)$, we have
$f(\tilde \xi + \chi_{c,t})>f(\tilde \xi)$. Therefore, condition (ii) in the definition of ordinal concavity is satisfied. For $\tilde \xi=\xi(\{y\})$, we have $f(\xi-\chi_{c,t})=f(\xi)$ and $f(\tilde \xi+\chi_{c,t})=f(\tilde \xi)$. Therefore, condition (iii) in the definition of ordinal concavity is satisfied.
If $\tilde{\xi}_{c}^{t'}=1$, then $\tilde{\xi}=\xi(\{z\})$ or $\tilde{\xi}=\xi(\{y,z\})$. For both values of $\tilde \xi$,
$f(\xi-\chi_{c,t}+\chi_{c,t'})>f(\xi)$, which means that
condition (i) in the definition of ordinal concavity is satisfied.
\end{example}


In the second example, we consider settings in which a number of seats
are reserved for each student type at each school.

\begin{example}[\emph{Saturated Diversity}]\label{ex:saturated}
For each school $c\in \calc$ and type $t\in \calt$, let $r^t_c\in \mathbb Z_+$ be
the number of reserved seats for type-$t$ students at school $c$. Suppose that
$\Xi^0$ is an M$^{\natural}$-convex set. Then, for
each $\xi\in \Xi^0$,
\[f^s(\xi)=\sum_{(c,t) \in \mathcal{C} \times \mathcal{T}} \min\{ \xi^t_c, r^t_c\},
\]
is an ordinally concave function.
\end{example}

The next example generalizes reservations so that the marginal value of
each type of student at every school is non-increasing.

\begin{example}[\emph{Marginally Decreasing Diversity}]\label{ex:marginal}
For each school $c\in \calc$ and type $t\in \calt$, let $g_{c,t}$ be
a univariate concave function. Suppose that
$\Xi^0$ is an M$^{\natural}$-convex set. Then, for each $\xi\in \Xi^0$,
\[f^m(\xi)=\sum_{(c,t)\in \mathcal{C} \times \mathcal{T}} g_{c,t}(\xi^t_c)
\]
is an ordinally concave function.
\end{example}

We further generalize the example so that diversity also depends
on the number of minority students at the university level.

\begin{example}[\emph{University Diversity}]\label{ex:total}
Let $\mathcal{M} \subseteq \mathcal{T}$ be a set of minority types.
For each school $c\in \calc$ and type $t\in \calt$,
let $g_{c,t}$ be a univariate concave function. Likewise, let $h$ be a univariate concave function. Suppose that
$\Xi^0$ is an M$^{\natural}$-convex set. Then, for each $\xi\in \Xi^0$,
\[f^u(\xi)=h\left(\sum_{(c,t)\in \mathcal{C} \times \mathcal{M}} \xi^t_c\right) +
\sum_{(c,t) \in \mathcal{C} \times \mathcal{T}} g_{c,t}(\xi^t_c)
\]
is an ordinally concave function.
\end{example}

The diversity indices defined in Examples \ref{ex:saturated}-\ref{ex:total} satisfy ordinal concavity (see
Appendix \ref{app:compare}).


\subsection{Sketch of the Proof of Theorem \ref{thm:diversitychoice}}
The first statement in Theorem \ref{thm:diversitychoice} that the diversity
choice rule outcome maximizes diversity among all subsets of the set of
applications follows by construction. Therefore, we illustrate the proofs for
the second and third statements in Theorem \ref{thm:diversitychoice}.
We provide a high-level explanation of our proofs here
and also illustrate each step of the construction in the diversity choice rule using
Example \ref{ex:ladfail}. Section \ref{sec:frontier} has more illustrations.

Fix a set of contracts $X\subseteq \mathcal{X}$. The proof that
$C^d(X)$ maximizes merit among all subsets
of $X$ that maximize diversity has three main steps
and uses discrete convexity notions as well as matroid theory.
\medskip

\noindent
\emph{Step 1:} The set of maximal distributions in $\Xi^*(X)$ is an M-convex set.

First, we study the structure of  $\Xi^*(X)$ that we
find in the diversity choice rule construction. We show
that if the diversity index $f$ is ordinally
concave, then $\Xi^{*}(X)$ satisfies M$^{\natural}$-convexity. Since
the diversity choice rule produces an outcome that is maximal
in $\Xi^{*}(X)$, we focus on maximal distributions in $\Xi^{*}(X)$.
By Lemma \ref{lem:mconvex}, the set of maximal distributions in an
M$^{\natural}$-convex set is M-convex; therefore, the
set of maximal distributions in $\Xi^{*}(X)$ is M-convex (Lemma \ref{lem:pareto}).

Consider Example \ref{ex:ladfail}. Let $n=5$ and $X=\{x,y,z\}$. For the first step,
we maximize $f$ on
$\Xi^0=\{\xi:\norm{\xi} \leq 2\}$ and get $\Xi^*(X)=\{\xi(\{z\}),\xi(\{x,z\}),
\xi(\{y,z\})\}$, which is an M$^{\natural}$-convex set. The set of
maximal distributions in $\Xi^*(X)$ is equal to $\{\xi(\{x,z\}),\xi(\{y,z\})\}$,
which is an M-convex set.



\medskip

\noindent
\emph{Step 2:} Let $\mathcal{F}(X) \equiv \{Y \subseteq X | \xi(Y)
\leq \xi \mbox{ for some } \xi \in \Xi^*(X)\}$. $(X,\mathcal{F}(X))$ is a matroid.

Next, we consider subsets of $X$ that have a distribution less than or
equal to a distribution in $\Xi^*(X)$, and, hence, these sets have
a distribution less than or equal to a maximal distribution in $\Xi^*(X)$.
$\mathcal{F}(X)$ is the collection of such sets.
Depending on the merit ranking, the diversity choice rule can produce
any maximal set in $\mathcal{F}(X)$ because in Step 2 of the diversity choice rule
construction contracts are chosen so
that the outcome has a maximal distribution in $\Xi^*(X)$.
Therefore, the structure of maximal
sets in $\mathcal{F}(X)$ plays a crucial role. We show M-convexity of
the set of maximal distributions in $\Xi^*(X)$ implies that the maximal
sets in $\mathcal{F}(X)$ satisfy the base axioms \emph{B1} and \emph{B2}',
which we provide in Lemma \ref{lem:matchar}, so $(X, \mathcal{F}(X))$ is a matroid
(Lemma \ref{lem:matroid}).

In Example \ref{ex:ladfail}, when $n=5$ and $X=\{x,y,z\}$, the set
of maximal distributions in
$\Xi^*(X)$ is equal to $\{\xi(\{x,z\}),\xi(\{y,z\})\}$. Therefore, the
collection of maximal sets in $\mathcal{F}(X)$ is equal to $\{\{x,z\},\{y,z\}\}$,
which satisfy the base axioms. Hence, $(X, \mathcal{F}(X))$ is a matroid.

\medskip
\noindent
\emph{Step 3:} The greedy rule on $(X,\mathcal{F}(X))$ produces $C^d(X)$.

Finally, we show that the greedy rule on matroid $(X,\mathcal{F}(X))$ produces
$C^d(X)$ (Lemma \ref{lem:equal}). Thus, $C^d(X)$ is a base of the matroid $(X,\mathcal{F}(X))$.
\cite{gale1968} shows that the greedy rule outcome merit dominates any independent set.
Therefore, $C^d(X)$ merit dominates any set in $\mathcal{F}(X)$, which includes
subsets of $X$ that maximize diversity.

In Example \ref{ex:ladfail}, when $n=5$ and $X=\{x,y,z\}$, the greedy rule on
$(X,\mathcal{F}(X))$ may produce $\{x,z\}$ and $\{y,z\}$ depending on the relative
merit ranking of $\{x\}$ and $\{y\}$. Therefore, if $x\succ y$, then the diversity
choice rule produces $\{x,z\}$, and, if $y\succ x$, then the diversity choice
rule produces $\{y,z\}$.

\medskip

The proof of the third statement in Theorem \ref{thm:diversitychoice} works in
two main steps. In the first step, we generalize a technique used in discrete convex
analysis
to our setting to find a distribution that maximizes the diversity index.
Step 1 of the diversity choice rule involves the problem of finding a distribution in
$\Xi^*(X)$, i.e., a maximizer of $f(\xi)$ subject to $\xi\in \Xi^0$ and
$0\leq \xi\leq \xi(X)$. Clearly, checking all distributions is computationally hard
because the size of the domain depends exponentially on the number of colleges and
types (recall $\Xi^0\subseteq \mathbb{Z}^{|\mathcal{C}|\times |\mathcal{T}|}_+$). 
We instead consider the so-called {\it domain-reduction algorithm}.


We illustrate the algorithm in Example \ref{ex:ladfail}. Let $n=5$ and $X=\{x,y,z\}$.
Since $|\mathcal{C}|\times |\mathcal{T}|=3$, we identify $\mathbb{Z}^{|\mathcal{C}|\times |\mathcal{T}|}_+$ with $\mathbb{Z}^3_+$
and assume that $\xi(\{x\})=(1,0,0)$, $\xi(\{y\})=(0,1,0)$, and $\xi(\{z\})=(0,0,1)$.
The algorithm starts from $\xi=(0,0,0)$ and iteratively updates $\xi$ until it reaches a maximizer of $f$.
In every iteration, we identify a direction $d \in \{(1,0,0), (0,1,0), (0,0,1)\}$ in which $f(\xi+d)$ is maximized.
By the definition of $f$,
\begin{align*}
f((0,0,0)+d)=\begin{cases} 1 & \text{ if } d=(1,0,0) \text{ or } (0,1,0), \\
                                   5 & \text{ if } d=(0,0,1).
                \end{cases}
\end{align*}
The maximum function value is attained when $(0,0,0)$ moves toward the direction $d=(0,0,1)$, so we update $\xi=(0,0,0)$ to $\xi+d=(0,0,1)$. Importantly, $d=(0,0,1)$ being a solution to the maximization problem implies that there exists a maximizer $\xi^*$ of $f$ with $\xi^*\geq (0,0,1)$ due to the {\it maximizer-cut theorem} (Theorem \ref{thm:maximizer-cut}) that we establish for ordinally concave functions.\footnote{We verify the maximizer-cut theorem in the current example. The maximizers of $f$ are
\begin{align*}
(0,0,1)(=\xi(\{z\})), \: (1,0,1)(=\xi(\{x,z\})), \: (0,1,1)(=\xi(\{y,z\})),
\end{align*}
showing that there exists a maximizer with the third coordinate being one (each maximizer satisfies the condition).}
In words, we can ``cut off''  distributions that have zero as their third coordinate
 and reduce the set of distributions we search
for from $\{\xi:\xi\geq (0,0,0)\}$ to $\{\xi:\xi\geq (0,0,1)\}$.


%\footnote{The domain-reduction algorithm and the maximizer-cut theorem are orginally established for M-convex functions (see Section 10.1.3 and Theorem 6.77 in \cite{Murota:SIAM:2003}). Our contribution is to generalize them for ordinally concave functions.}
%%MBY: I removed the above footnote because we already mention these in the Appendix when
%we do the proofs.

The algorithm terminates when $\xi$ does not increase in any direction, which is interpreted as $\xi$ locally maximizing diversity.
We prove that  local maximization implies global maximization, i.e., the final distribution $\xi$ is
a global maximizer and, hence, included in $\Xi^*(X)$ (Lemma \ref{lem:outcome-maximizer}).\footnote{This implication is reminiscent of the same property under the standard concavity for univariate continuous functions. In the formal proof, we show that the final distribution $\xi$ is a {\it maximal} distribution in $\Xi^*(X)$ (Lemma \ref{lem:outcome-maximal}).}
At each iteration, the number of directions that we search for is $|\mathcal{C}| \times |\mathcal{T}|$.
Furthermore, since the domain for maximization is restricted to $\{\xi:0\leq \xi\leq \xi(X)\}$ and shrinks in every iteration, the number of iterations is at most $\norm{\xi(X)}$, which is bounded by $|X|$, a linear function of the number of applications. Hence, the domain-reduction algorithm finds a maximizer in $O(|\mathcal{C}| \times |\mathcal{T}| \times |X|)$.


The domain-reduction algorithm finds {\it one} maximizer, but Step 2 of the diversity choice rule searches for {\it all} maximizers.
It turns out that the process of checking all maximizers can be simplified to checking only local distributions around a maximizer
 (Lemma \ref{lem:modified}), which is more computationally tractable.
Building upon this finding, we develop a modified version of the diversity choice rule
and show that the new choice rule produces the same outcome as the original one (Lemma \ref{lem:choice-equivalence}) and can be
calculated in $O(|\mathcal{C}| \times |\mathcal{T}| \times |X|^2)$.

\medskip
\subsection{Path Independence and the Law of Aggregate Demand}
%\fk{BEGIN HERE 11-28-2022}
In this section, we investigate further desirable properties of the
diversity choice rule. We first establish the following result.

\begin{theorem}\label{thm:pi}
Suppose that the diversity index $f$ is \oconcave{}. Then the
diversity choice rule $C^d$ satisfies path independence.
\end{theorem}

Even though the diversity choice rule satisfies path independence
when the diversity index $f$ is ordinally concave, it need not satisfy
the law of aggregate demand. We show this claim simply by providing an
example. Let $C^d$ be the diversity choice rule corresponding to the
diversity index in Example \ref{ex:ladfail} when $n>5$ for a merit ranking
of contracts. Then
\[C^d(\{x,y,z\})=\{z\} \text{ and } C^d(\{x,y\})=\{x,y\}\]
show that $C^d$ does not satisfy the law of aggregate demand because
\[\left|C^d(\{x,y,z\})\right| < \left|C^d(\{x,y\})\right|.\]

To get the law of aggregate demand, we make the following monotonicity assumption
on the diversity index.

\begin{comment}
\begin{definition}
The diversity index $f: \Xi^0 \rightarrow \mathbb{R}_+$ is \textbf{monotone} if, for every $\xi,\tilde{\xi} \in \Xi^0$,
\[\xi\geq \tilde{\xi} \; \implies \; f(\xi) \geq f(\tilde{\xi}). \]
\end{definition}
\end{comment}

\begin{definition}
The diversity index $f: \Xi^0 \rightarrow \mathbb{R}_+$ is \textbf{monotone} if $f(\xi) \geq f(\tilde{\xi})$ for each $\xi,\tilde{\xi} \in \Xi^0$ with $\xi\geq \tilde{\xi}$.
\end{definition}

Monotonicity means that increasing any coordinate of a feasible distribution
weakly increases diversity as long as the new distribution is also feasible. In Example \ref{ex:ladfail},
when $n>5$, monotonicity fails because $f(\xi(\{z\}))>f(\xi(\{x,z\}))$, however,
monotonicity holds when $n=5$. In our other examples in Section \ref{sec:choice},
monotonicity is either satisfied without making any assumptions or satisfied
by making some additional assumptions: In Example \ref{ex:saturated}, the
saturated diversity index $f^s$ satisfies monotonicity. In Example \ref{ex:marginal}, the
marginally decreasing diversity index $f^m$
satisfies monotonicity if, for each school $c\in \calc$ and
type $t\in \calt$, $g_{c,t}$ is increasing. Finally, in Example \ref{ex:total}, university
diversity index $f^u$ satisfies monotonicity if $h$ and, for
each school $c\in \calc$ and type $t\in \calt$, $g_{c,t}$ are increasing.

Assuming the monotonicity of the diversity index, in addition to ordinal concavity,
delivers the law of aggregate demand for the diversity choice rule.\footnote{The same
result holds when the diversity index satisfies M$^{\natural}$-concavity without assuming
monotonicity. The proof is available from the authors.}

\begin{proposition}\label{prop:structure}
Suppose that the diversity index $f$ is \oconcave{} and monotone. Then
the diversity choice rule $C^d$ satisfies the law of aggregate demand.
In particular, for each $X\subseteq \mathcal{X}$ and
$x\in \mathcal{X}\setminus X$, one of the following holds:
\begin{enumerate}[(i)]
  \item $C^d(X\cup\{x\})=C^d(X)$,
  \item $C^d(X\cup\{x\})=C^d(X) \cup \{x\}$, or
  \item $C^d(X \cup\{x\})= (C^d(X) \cup \{x\})\setminus \{y\}$ for some $y\in C^d(X)$.
\end{enumerate}
\end{proposition}
We note that a choice rule satisfies path-independence and the law of aggregate demand if, and only if,
the requirement that one of the three displayed equations holds for
each $X\subseteq \mathcal{X}$ and $x\in \mathcal{X}\setminus X$.


\section{Maximizing Merit Subject to a Diversity Level}\label{sec:frontier}
A university administration may want to maximize merit of an incoming freshman
class subject to attaining a given diversity level instead of lexicographically
maximizing these two objectives. In this section, we introduce a class of choice rules, parameterized by the diversity level, that achieves this goal. Using this class, we provide
an algorithm that produces the diversity-merit Pareto frontier.


\subsection{Maximizing Merit Subject to a Target Diversity Level} \label{sec:choice-target}

Let $\lambda\in \mathbb{R}_+$ be a target diversity level. For a given set of applications $X\subseteq \mathcal{X}$, our goal is to choose $X'\subseteq X$ that maximizes merit
subject to $f(\xi(X'))\geq \min\{f(\xi(C^d(X))),\lambda\}$.
That is, we want the chosen set to have a diversity
of at least $\lambda$ if it is attainable (while achieving the maximum diversity level otherwise).

Key to our analysis is to formulate a new diversity index so that the diversity choice rule developed in the previous section for the new index maximizes
merit subject to achieving the diversity level. Specifically,
consider the following modification of the original diversity index $f$, denoted as $f_\lambda$: for each $\xi\in \Xi^0$,
\[f_{\lambda}(\xi)=\min\{f(\xi),\lambda\}.\]
Therefore, $f_{\lambda}: \Xi^0 \rightarrow \mathbb{R}_+$ is the diversity index
that flattens the top part of the diversity index $f$ by $\lambda$.
For each $X'\subseteq X$, $f(\xi(X'))\geq \min\{f(\xi(C^d(X))),\lambda\}$ is equivalent to $\xi(X')\in \underset {\xi\in \Xi^0} {\arg\max} \: f_\lambda(\xi)$, so our goal is to choose $X'\subseteq X$ that maximizes merit  subject to $\xi(X')$ being an optimum of $f_\lambda$.
%As discussed in Section \ref{sec:choice},
This is exactly what the diversity choice rule does when it is defined using the diversity index $f_{\lambda}$, which we denote by $C^d_{\lambda}$.
For example, if $\lambda\geq f(\xi(C^d(X))$, then $C^d_{\lambda}(X)=C^d(X)$.
If $\lambda=0$, then $C^d_{\lambda}$ maximizes the merit ranking subject to attaining a feasible distribution in $\Xi^0$.

If $f_\lambda$ is ordinally concave, then the desirable properties of $C^d$ established in Theorems \ref{thm:diversitychoice} and \ref{thm:pi}
hold for $C_\lambda^d$ as well. Unfortunately, ordinal concavity of $f$ does not necessarily imply ordinal concavity of $f_\lambda$, as illustrated in the following example.

\begin{example}\label{ex:oc-truncation-fail}
Let $\mathcal{C}=\{c\}$, $\mathcal T=\{t,t'\}$, $\Xi^0=\{0,1\}^2 \subseteq \mathbb{Z}^2_+, \xi(\{x\})=(1,0)$, and $\xi(\{y\})=(0,1)$.
Let $\calx=\{x,y\}$ and the diversity index $f$ be defined as follows:
\[f(\xi(\emptyset))=1, f(\xi(\{x\}))=0, f(\xi(\{y\}))=2, \mbox{ and } f(\xi(\{x,y\}))=1.\]
It is easy to see that $f$ is ordinally concave.
For $\lambda=1$,
\[f_\lambda(\xi(\emptyset))=1, f_\lambda(\xi(\{x\}))=0, f_\lambda(\xi(\{y\}))=1, \mbox{ and } f_\lambda(\xi(\{x,y\}))=1.\]
Consider $\xi(\{x,y\})$, $\xi(\emptyset)$, and $t \in \mathcal{T}$ with $\chi_{c,t}=\xi(\{x\})$. Since
\begin{align*}
&f_\lambda(\xi(\{x,y\}))=1=f_\lambda(\xi(\{y\})) \text{ and } \\
&f_\lambda(\xi(\emptyset))=1>0=f_\lambda(\xi(\{x\})),
\end{align*}
$f_\lambda$ violates ordinal concavity.
\end{example}
To guarantee that $f_\lambda$ is ordinally concave for each $\lambda$, we explore other concavity conditions.

\begin{definition}[\cite{hakoye2022}]\label{def:pseudo}
The diversity index $f: \Xi^0 \rightarrow \mathbb{R}_+$ is \textbf{pseudo M$^\natural$-concave} if, for each
$\xi, \tilde{\xi} \in \Xi^0$ and $(c, t) \in \mathcal{C} \times \mathcal{T}$ with $\xi_c^t>\tilde{\xi}_{c}^t$,
there exists $\left(c^{\prime}, t^{\prime}\right) \in (\mathcal{C} \times \mathcal{T})\cup\{\emptyset\}$ (with $\xi_{c^{\prime}}^{t^{\prime}}<\tilde{\xi}_{c^{\prime}}^{t^{\prime}}$ whenever $(c',t')\neq \emptyset$) such that
$$
\min\{f(\xi),f(\tilde{\xi})\}\leq\min\{f(\xi-\chi_{c,t}+\chi_{c',t'}), f(\tilde{\xi}+\chi_{c,t}-\chi_{c',t'})\}.
$$
\end{definition}
Pseudo M$^\natural$-concavity is similar in spirit to ordinal concavity in the sense that both conditions
require the value of $f$ to increase when $\xi$ and $\tilde \xi$ move toward each other (recall the interpretation of Definition \ref{def:ordinal}). One can check that
the first two statements in Theorem \ref{thm:diversitychoice} also hold under
pseudo M$^\natural$-concavity.

Pseudo M$^\natural$-concavity of $f$ is logically independent of ordinal concavity of $f$,\footnote{The diversity index in Example \ref{ex:oc-truncation-fail} satisfies ordinal concavity but violates pseudo M$^\natural$-concavity. The diversity index in Example \ref{ex:truncation-sufficient-fail} below satisfies pseudo M$^\natural$-concavity but violates ordinal concavity.} but it is related to the ordinal concavity of $f_\lambda$ for each $\lambda\geq 0$.

\begin{proposition}\label{prop:truncation-necessary}
If $f_\lambda$ is ordinally concave for each $\lambda\geq 0$, then $f$ is pseudo M$^\natural$-concave.
\end{proposition}

Unfortunately, the converse of Proposition \ref{prop:truncation-necessary} is false, as illustrated in the following example.

\begin{example}\label{ex:truncation-sufficient-fail}
Let $\mathcal{C}=\{c\}$, $\mathcal{T}=\{t\}$, and $\Xi^0=\{0,1,2\}\subseteq \mathbb{Z}_+$. We identify $\mathbb{Z}^{|\mathcal{C}|\times |\mathcal{T}|}_+$ with $\mathbb{Z}_+$. Define
$f:\Xi^0\rightarrow \mathbb{R}$ as
\begin{align*}
f(0)=0, \: f(1)=0, \: \text{ and } f(2)=1.
\end{align*}
It is easy to see that $f$ satisfies pseudo M$^\natural$-concavity. However, $f_\lambda$ violates ordinal concavity whenever $\lambda \geq 1$ (in which case $f_\lambda=f$).\footnote{Therefore,
this example in fact shows that pseudo M$^\natural$-concavity of $f$ does not imply ordinal concavity of $f$.} To see this point, let $\xi=2$ and $\tilde \xi=0$. Since
\begin{align*}
&1=f_\lambda(\xi)>f_\lambda(\xi-\chi_{c,t})=f_\lambda(1)=0 \text{ and } \\
&0=f_\lambda(\tilde \xi)=f_\lambda(\tilde \xi+\chi_{c,t})=f_\lambda(1)=0,
\end{align*}
$f_\lambda$ violates ordinal concavity.
\end{example}
To guarantee the equivalence to ordinal concavity of $f_\lambda$ for each $\lambda\geq 0$, we strengthen pseudo M$^\natural$-concavity as follows.

\begin{definition}\label{def:semistrict-Mc}
The diversity index $f: \Xi^0 \rightarrow \mathbb{R}_+$ is \textbf{pseudo M$^\natural$-concave$^+$} if, for each $\xi, \tilde{\xi} \in \Xi^0$ and $(c, t) \in \mathcal{C} \times \mathcal{T}$ with $\xi_c^t>\tilde{\xi}_{c}^t$, there exists $\left(c^{\prime}, t^{\prime}\right) \in (\mathcal{C} \times \mathcal{T})\cup\{\emptyset\}$ (with $\xi_{c^{\prime}}^{t^{\prime}}<\tilde{\xi}_{c^{\prime}}^{t^{\prime}}$ whenever $(c',t')\neq \emptyset$) such that
\begin{align*}
\min\{f(\xi),f(\tilde{\xi})\}\leq\min\{f(\xi-\chi_{c,t}+\chi_{c',t'}), f(\tilde{\xi}+\chi_{c,t}-\chi_{c',t'})\}.
%\label{TE-1}
\end{align*}
Moreover,
\begin{enumerate}[(A)]
\item \label{semistrict-a}
If $f(\xi)>f(\xi-\chi_{c,t}+\chi_{c',t'})$ and $f(\tilde{\xi})=f(\tilde{\xi}+\chi_{c,t}-\chi_{c',t'})$ hold, then there exists $\left(c'', t''\right) \in (\mathcal{C} \times \mathcal{T})\cup\{\emptyset\}$ (with $\xi_{c''}^{t''}<\tilde{\xi}_{c''}^{t''}$ whenever $(c'',t'')\neq \emptyset$) such that
\begin{align*}
f(\tilde{\xi})<f(\tilde{\xi}+\chi_{c,t}-\chi_{c'',t''}).
\end{align*}
\item \label{semistrict-b}
If $f(\tilde{\xi})>f(\tilde{\xi}+\chi_{c,t}-\chi_{c',t'})$ and $f(\xi)=f(\xi-\chi_{c,t}+\chi_{c',t'})$ hold, then there exists $\left(c'', t''\right) \in (\mathcal{C} \times \mathcal{T})\cup\{\emptyset\}$ (with $\xi_{c''}^{t''}<\tilde{\xi}_{c''}^{t''}$ whenever $(c'',t'')\neq \emptyset$) such that
\begin{align*}
f(\xi)<f(\xi-\chi_{c,t}+\chi_{c'',t''}).
%\label{TE-2}
\end{align*}
\end{enumerate}
\end{definition}
By the displayed weak inequality in the definition, the if-clause of (\ref{semistrict-a})
is true only if $f(\xi)>f(\tilde \xi)$ and that of (\ref{semistrict-b}) is true only if $f(\tilde \xi)>f(\xi)$.
Hence, the if-clause concerns the case when the higher value of $f$ decreases and the lower value of $f$ remains
the same when $\xi$ and $\tilde \xi$ move towards each other, as in the case of Example \ref{ex:truncation-sufficient-fail}.
In such a case, pseudo M$^\natural$-concavity$^+$ requires that there is another coordinate $(c'',t'')$ for which
the lower value of $f$ {\it strictly} increases as indicated by the displayed strict inequality.

One might find the definition of pseudo-M$^\natural$-concavity$^+$ complicated. In
Appendix \ref{app:semistrict}, we introduce a new condition that implies
pseudo M$^\natural$-concavity$^+$ and is more easily interpretable due to its analogy to the notion of {\it quasi-concavity}, an important assumption on utility functions in the analysis of markets with continuous commodities. In the subsequent analysis, we focus on pseudo M$^\natural$-concavity$^+$ because it allows us to establish an equivalence result and accommodate a canonical example of $f$ in Section \ref{sec:diversity}, as formalized below.


\begin{proposition} \label{prop:ordinal-equivalence}
Function $f_\lambda$ is ordinally concave for each $\lambda\geq 0$ if, and only if, $f$ is pseudo M$^\natural$-concave$^+$.
\end{proposition}

Now we present two diversity indices that are pseudo  M$^\natural$-concave$^+$.

\begin{claim}\label{claim:ex1-pMC}
The diversity index $f$ in Example \ref{ex:ladfail} is pseudo M$^\natural$-concave$^+$.
\end{claim}

\begin{claim} \label{claim:ex2-pMC}
The saturated diversity $f^s$ in Example \ref{ex:saturated} is pseudo M$^\natural$-concave$^+$ if $\Xi^0=\{\xi\in \mathbb{Z}^{|\mathcal{C}|\times |\mathcal{T}|}_+ \mid \sum_{(c,t) \in \mathcal{C}\times \mathcal{T}} \xi_c^t \leq q\}$ for some $q\in \mathbb{Z}_+$.
\end{claim}
Claim \ref{claim:ex2-pMC} implies that the analysis of this section is applicable to the choice rule of a single school with saturated diversity and a capacity constraint.
In Appendix \ref{app:proofs}, we provide proofs of Claims \ref{claim:ex1-pMC} and \ref{claim:ex2-pMC} as well as a counterexample to Claim \ref{claim:ex2-pMC}
when $\Xi^0$ is not given as in the statement. We note that the diversity indices in Examples \ref{ex:marginal} and \ref{ex:total} violate pseudo M$^\natural$-concavity$^+$.

We obtain the following corollary by combining Proposition \ref{prop:ordinal-equivalence} and Theorem \ref{thm:diversitychoice}.
\begin{corollary}\label{corollary:diversitychoice}
Suppose that the diversity index $f$ is pseudo M$^\natural$-concave$^+$. Then, for each $\lambda\geq 0$ and  set of contracts $X\subseteq \calx$,
\begin{enumerate}[(i)]
\item $C^d_\lambda(X)$ maximizes the diversity index $f_\lambda$ among subsets of $X$. In particular, if $\lambda\leq f(\xi(C^d(X)))$, then $C^d_\lambda(X)$ attains diversity level of at least $\lambda$.
\item $C^d_\lambda(X)$ merit dominates each subset $X'$ of $X$ with $f(\xi(X'))\geq \lambda$, and
\item $C^d_\lambda(X)$ can be calculated in $O(|\mathcal{C}| \times |\mathcal{T}| \times |X|^2)$ time, assuming $f$ can be evaluated in a constant time.
\end{enumerate}
\end{corollary}
\begin{proof}
Fix $\lambda$. By Proposition \ref{prop:ordinal-equivalence}, $f_\lambda$ is ordinally concave. Hence, the claims follow from Theorem \ref{thm:diversitychoice}.
\end{proof}
Hence, if $f$ is pseudo M$^\natural$-concave$^+$, then $C^d_{\lambda}$ maximizes merit subject to attaining a diversity level of at least $\lambda$, and its outcome can be constructed in quadratic time in the number of contracts. Under the weaker notion of
pseudo M$^\natural$-concavity, the first two parts of Corollary \ref{corollary:diversitychoice} continue to hold because pseudo
M$^\natural$-concavity of $f$ implies that, for each $\lambda$, $f_{\lambda}$ satisfies pseudo M$^\natural$-concavity.
Recall that the first two statements in Theorem \ref{thm:diversitychoice} continue to
hold under pseudo M$^\natural$-concavity.

\subsection{Diversity-Merit Pareto Frontier}
A university administration may not have a particular target level of diversity in
mind but may want to know the diversity-merit Pareto frontier and choose the incoming class from the Pareto frontier. Therefore, identifying the Pareto frontier is important, especially for institutions that do not have a target diversity level. In
this section, we provide an algorithm to find the diversity-merit Pareto frontier
by using the choice rule developed in Section \ref{sec:choice-target}.

For a given set of applications $X$, we define the \emph{diversity-merit
Pareto frontier of $X$}, $\mathcal{P}(X)$, as follows:
\[\mathcal{P}(X)=\{Y\subseteq X : \not\exists
    Z\subseteq X  \mbox { s.t. } Z\neq Y, Z \mbox{ merit dominates } Y, \mbox { and }
    f(\xi(Z))\geq f(\xi(Y))\}.\]

Throughout this section, we assume that the diversity index
$f$ takes integer values.
We introduce a new algorithm that traces the diversity-merit Pareto frontier. The algorithm takes a set of contracts $X\subseteq \mathcal{X}$ as input and produces a collection of subsets of $X$.



\medskip
\paragraph{\textbf{Trace Algorithm}} %$\mathbf{C^{tr}}$}

\begin{description}
  \item[Input] Let $X$ be a set of contracts.
  \item[Step 1] Set $k=0$, $\lambda_0=0$, and $\mathcal{X}_0=\emptyset$.
  \item[Step 2] Let $\mathcal{X}_{k+1}=\mathcal{X}_k \cup \{C^d_{\lambda_k}(X)\}$. If
   $C^d_{\lambda_k}(X) = C^d(X)$, go to Step 4. Otherwise, set $\lambda_{k+1}=f(C^d_{\lambda_k}(X))+1$ and go to Step 3.
  \item[Step 3] Add 1 to $k$ and go to Step 2.
  \item[Step 4] Return $\mathcal{X}_{k+1}$ and stop.
\end{description}

Since $X$ is finite, the diversity index $f$ can take only a finite number
of values. Therefore, the algorithm ends at some finite $k$ because
$C^d(X)$ maximizes the diversity index among subsets of $X$, and it merit
dominates any subset with a diversity index of $\xi(C^d(X))$ (Theorem \ref{thm:diversitychoice}).

Let $\alpha$ be the maximum value that the diversity index $f$ takes.
The main result of this section is the following.

\begin{theorem}\label{thm:trace}
Suppose that $f$ is pseudo M$^\natural$-concave$^+$.
Then, for each $X\subseteq \mathcal{X}$, the trace algorithm outcome is the diversity-merit  Pareto frontier $\mathcal{P}(X)$.
The time complexity of the algorithm is $O(\alpha \times |\mathcal{C}| \times |\mathcal{T}| \times |X|^2)$, assuming $f$
can be evaluated in a constant time.
\end{theorem}

Hence, the trace algorithm finds all subsets of the set of applications
that generate the diversity-merit Pareto frontier. The computational part states that the trace algorithm is {\it pseudo polynomial} in the sense that the time complexity is polynomial in the largest integer present in the input data describing the matching problem.
We observe that the first part of the result that the trace algorithm finds the diversity-merit Pareto frontier also holds under pseudo M$^\natural$-concavity.

Now, we illustrate the trace algorithm.

\begin{example}\label{ex:frontier}
Consider the setting in Example \ref{ex:ladfail} and suppose that $n\geq 6$.
Suppose that the university is considering
the set of applications $X=\{x,y,z\}$ and the merit ranking of contracts is
$x\succ y \succ z$. Note that the diversity choice rule outcome is
$C^d(X)=\{z\}$.

At the beginning of the algorithm $k=0$,
$\lambda_0=0$, and $\mathcal{X}_0=\emptyset$.
Therefore, we need to calculate $C^d_{\lambda_0}(X)$. For $\lambda_0=0$, $f_{\lambda_0}$ assigns zero to all sets.
Hence, the set of maximal distributions in the
set of maximizers of $f_{\lambda_0}$ is
\[ \{ \xi(\{x,y\}), \xi(\{x,z\}), \xi(\{y,z\})\},\]
and thus, $C^d_{\lambda_0}(X)=\{x,y\}$. Since $C^d_{\lambda_0}(X)\neq C^d(X)=\{z\}$, we set
$\mathcal{X}_1=\mathcal{X}_0 \cup \{C^d_{\lambda_0}(X)\}=\{\{x,y\}\}$ and $\lambda_1=f(\xi(\{x,y\}))+1=2$.

In the second iteration we have $k=1$, $\lambda_1=2$, and $\mathcal{X}_1=\{\{x,y\}\}$.
Hence, we need to find $C^d_{\lambda_1}(X)$. For $\lambda_1=2$, $f_{\lambda_1}$ assigns two to
all sets with a diversity index (with respect to $f$) of at least two. Therefore, the set of maximal
distributions in the set of maximizers for the diversity index $f_{\lambda_1}$ is
\[ \{ \xi(\{x,z\}), \xi(\{y,z\})\},\]
and thus $C^d_{\lambda_1}(X)=\{x,z\}$. Since $C^d_{\lambda_1}(X)\neq C^d(X)=\{z\}$, we set
$\mathcal{X}_2=\mathcal{X}_1 \cup \{C^d_{\lambda_1}(X)\}=\{\{x,y\},\{x,z\}\}$ and $\lambda_2=f(\xi(\{x,z\}))+1=6$.

In the third iteration we have $k=2$, $\lambda_2=6$, and $\mathcal{X}_2=\{\{x,y\},\{x,z\}\}$.
Hence, we need to construct $C^d_{\lambda_2}(X)$. For $\lambda_2=6$, $f_{\lambda_2}$
assigns six to all sets with a diversity index (with respect to $f$) of at least
six. Therefore, the set of maximal distributions in the set of maximizers for the
diversity index $f_{\lambda_2}$ is
\[ \{ \xi(\{z\})\},\]
which implies that we get $\{z\}$, so
$C^d_{\lambda_2}(X)=\{z\}$. Since $C^d_{\lambda_2}(X)=C^d(X)=\{z\}$, we set
$\mathcal{X}_3=\mathcal{X}_2 \cup \{C^d_{\lambda_2}(X)\}=\{\{x,y\},\{x,z\},\{z\}\}$ and return this as the outcome of the trace algorithm.
The outcome generates all the sets in the diversity-merit Pareto
frontier by Theorem \ref{thm:trace}; see Figure \ref{fig:frontier}.

\begin{figure}
  \centering
\definecolor{ffffff}{rgb}{1.,1.,1.}
\begin{tikzpicture}[scale=.9]
\begin{axis}[
x=1.0cm,y=1.0cm,
axis lines=middle,
xmin=-1.5,
xmax=11.0,
ymin=-1.5,
ymax=8.0,
xtick=\empty,
ytick=\empty,]
\draw (0.38,7.72) node[anchor=north west] {$\{x,y\}$};
\draw [dashed] (5.,0.)-- (5.,5.);
\draw [dashed] (5.,5.)-- (0.,5.);
\draw (4.4,5.72) node[anchor=north west] {$\{x,z\}$};
\draw [dashed] (8.,0.)-- (8.,1.);
\draw [dashed] (8.,1.)-- (0.,1.);
\draw (7.55,1.72) node[anchor=north west] {$\{z\}$};
\draw [dashed] (1.,0.)-- (1.,7.);
\draw [dashed] (1.,7.)-- (0.,7.);
%\draw (1.75,-0.1) node[anchor=north west] {2};
\draw (4.75,-0.1) node[anchor=north west] {5};
\draw (7.75,-0.2) node[anchor=north west] {n};
\draw [dashed] (1.,4.)-- (1.,0.);
\draw [dashed] (1.,4.)-- (0.,4.);
\draw [dashed] (5.,3.)-- (0.,3.);
\draw (.75,-0.1) node[anchor=north west] {1};
\draw (4.4,4.70) node[anchor=north west] {$\{y,z\}$};
\draw (0.58,4.70) node[anchor=north west] {$\{x\}$};
\draw (0.58,3.75) node[anchor=north west] {$\{y\}$};
\draw (9.20, -0.1) node[anchor=north west] {diversity};
\draw (-1.3, 8.0) node[anchor=north west] {merit};
\begin{scriptsize}
\draw [fill=black] (5.,5.) circle (2.5pt);
\draw [fill=black] (8.,1.) circle (2.5pt);
\draw [fill=black] (1.,7.) circle (2.5pt);
\draw [fill=ffffff] (5.,4.) circle (2.5pt);
\draw [fill=ffffff] (1.,4.) circle (2.5pt);
\draw [fill=ffffff] (1.,3.) circle (2.5pt);
\end{scriptsize}
\end{axis}
\end{tikzpicture}
\caption{The filled nodes are on the diversity-merit Pareto frontier in Example \ref{ex:frontier} when $n>5$.}
\label{fig:frontier}
\end{figure}

\begin{comment}
\begin{figure}[htb]
    \centering
\definecolor{wrwrwr}{rgb}{0.3803921568627451,0.3803921568627451,0.3803921568627451}
\definecolor{rvwvcq}{rgb}{0.40784313725490196,0.40784313725490196,0.40784313725490196}
\begin{tikzpicture}[scale=.9]
\begin{axis}[
x=1.0cm,y=1.0cm,
axis lines=middle,
xmin=-1.5,
xmax=11.0,
ymin=-1.5,
ymax=8.0,
xtick=\empty,
ytick=\empty,
%xlabel=diversity,
%ylabel=meritocracy,
]
\draw (1.4,7.72) node[anchor=north west] {$\{x,y\}$};
\draw [line width=0.8pt,dash pattern=on 1pt off 1pt,color=wrwrwr] (5.,0.)-- (5.,5.);
\draw [line width=0.8pt,dash pattern=on 1pt off 1pt,color=wrwrwr] (5.,5.)-- (0.,5.);
\draw (4.4,5.72) node[anchor=north west] {$\{x,z\}$};
\draw [line width=0.8pt,dash pattern=on 1pt off 1pt,color=wrwrwr] (8.,0.)-- (8.,2.);
\draw [line width=0.8pt,dash pattern=on 1pt off 1pt,color=wrwrwr] (8.,2.)-- (0.,2.);
\draw (7.55,2.72) node[anchor=north west] {$\{z\}$};
\draw [line width=0.8pt,dash pattern=on 1pt off 1pt,color=wrwrwr] (2.,0.)-- (2.,7.);
\draw [line width=0.8pt,dash pattern=on 1pt off 1pt,color=wrwrwr] (2.,7.)-- (0.,7.);
\draw (1.75,-0.1) node[anchor=north west] {2};
\draw (4.75,-0.1) node[anchor=north west] {5};
\draw (7.75,-0.2) node[anchor=north west] {$n$};
\draw (9.20, -0.1) node[anchor=north west] {diversity};
\draw (-1.3, 8.0) node[anchor=north west] {merit};
\begin{scriptsize}
\draw [fill=rvwvcq] (5.,5.) circle (2.5pt);
\draw [fill=rvwvcq] (8.,2.) circle (2.5pt);
\draw [fill=rvwvcq] (2.,7.) circle (2.5pt);
\end{scriptsize}
\end{axis}
\end{tikzpicture}
\caption{Diversity-meritocracy Pareto frontier in Example \ref{ex:frontier} when $n>5$.}
\label{fig:frontier}
\end{figure}
\end{comment}
\end{example}






\section{Conclusion}\label{sec:conclusion}
When institutions hire workers or admit students, they often
have dual objectives of diversity and meritocracy that may
conflict with each other. In this context, we have identified
a class of institutional choice rules that maximize merit
subject to attaining a diversity level in a computationally efficient
way. We have also introduced the trace algorithm to find the diversity-merit
Pareto frontier. We anticipate that our results
will be useful in markets where there are dual objectives.

We assume that the diversity of a group of agents is measured by an index
satisfying \emph{ordinal concavity}, a notion of discrete concavity that we
introduce. Since ordinal concavity allows \emph{greedy algorithm} to be
effectively used in the contexts of discrete optimizations problems (which are faced
frequently in economics, operations research and computer science), our novel notion
and its desirable properties may prove useful in other applications in the future.

Lastly, our analysis has highlighted an intimate connection between the theories of
discrete convexity and matroids. For instance, we have provided two novel characterizations
of matroids, which are important results by themselves. Moreover, we introduced and analyzed
different concavity notions such as pseudo M$^\natural$-concavity$^+$.
%(such as the notions of pseudo M$^\natural$-concavity$^+$ and semistrict pseudo M$^\natural$-concavity).
We envision that those concavity notions may prove useful in other studies.

\appendix


%\section{Mathematical Appendix}

\section{Concavity Notions for Discrete Functions}\label{app:compare}
There are two notions of concavity for discrete
functions that are commonly used in discrete mathematics.
The first one is M-concavity.

\begin{definition}\label{def:natural}
A function $f$ is \textbf{M-concave} if, for each $\xi,\tilde{\xi}\in \Xi^0$ and $(c,t) \in \calc \times \calt$ with  $\xi_c^t>\tilde{\xi}_c^t$,
there exists $(c',t') \in \calc \times \calt$ with  $\xi_{c'}^{t'}<\tilde{\xi}_{c'}^{t'}$ such that
\[f(\xi-\chi_{c,t}+\chi_{c',t'})+f(\tilde{\xi}+\chi_{c,t}-\chi_{c',t'})
\geq f(\xi)+f(\tilde \xi).\]
\end{definition}

A weaker version of M-concavity is also used.

\begin{definition}\label{def:natural}
A function $f$ is \textbf{M$^{\natural}$-concave} if, for each $\xi,\tilde{\xi}\in \Xi^0$ and $(c,t) \in \calc \times \calt$ with  $\xi_c^t>\tilde{\xi}_c^t$,
there exists $(c',t') \in (\calc \times \calt) \cup \{\emptyset\}$
(with $\xi_{c'}^{t'}<\tilde{\xi}_{c'}^{t'}$ whenever $(c',t')\neq \emptyset$) such that
\[f(\xi-\chi_{c,t}+\chi_{c',t'})+f(\tilde{\xi}+\chi_{c,t}-\chi_{c',t'})
\geq f(\xi)+f(\tilde \xi).\]

\end{definition}

Even though our ordinal concavity is an ordinal concept,
M-concavity and M$^{\natural}$-concavity both depend on the cardinal
values that the diversity index takes. Furthermore, both M$^{\natural}$-concavity
and M-concavity imply ordinal concavity.

\begin{proposition}\label{prop:comparison}
If a function is M$^{\natural}$-concave, then it is ordinally concave. There exists
an ordinally concave function that is not M$^{\natural}$-concave.
\end{proposition}

The diversity indices defined in Examples \ref{ex:saturated}-\ref{ex:total}
satisfy M$^{\natural}$-concavity (see page 140 of \cite{Murota:SIAM:2003}).
Therefore, by Proposition \ref{prop:comparison}, they also satisfy
ordinal concavity.

The following result is immediate from this proposition.

\begin{corollary}
If a function is M-concave, then it is ordinally concave.
\end{corollary}

\begin{comment}
Now we provide an equivalent definition of ordinal concavity.

\begin{proposition}\label{prop:equal}
A function $f$ is ordinally concave if, and only if, the following holds:
For every distributions $\xi,\tilde{\xi}$ with
$(c,t) \in \calc \times \calt$ such that  $\xi_c^t>\tilde{\xi}_c^t$, either
\begin{enumerate}[(i)]
\item $f(\xi) \geq f(\xi-\chi_{c,t})$ implies $f(\tilde{\xi}+\chi_{c,t}) \geq f(\tilde{\xi})$, and
\item $f(\tilde{\xi})\geq f(\tilde{\xi}+\chi_{c,t})$ implies $f(\xi-\chi_{c,t}) \geq f(\xi)$, or
\end{enumerate}

there exists $(c',t')\in \calc \times \calt$ with
$\xi_{c'}^{t'}<\tilde{\xi}_{c'}^{t'}$ such that
\begin{enumerate}[(i)]
\item[(iii)] $f(\xi) \geq f(\xi-\chi_{c,t}+\chi_{c',t'})$ implies $f(\tilde{\xi}+\chi_{c,t}-\chi_{c',t'}) \geq f(\tilde{\xi})$, and
\item[(iv)] $f(\tilde{\xi})\geq f(\tilde{\xi}+\chi_{c,t}-\chi_{c',t'})$ implies $f(\xi-\chi_{c,t}+\chi_{c',t'}) \geq f(\xi)$.
\end{enumerate}
\end{proposition}

\begin{proof}[Proof of Proposition \ref{prop:equal}]
\end{proof}
\end{comment}

\section{Semi-strict Pseudo M$^\natural$-concavity}\label{app:semistrict}
In this appendix, we provide a new definition of concavity, which implies that for each
$\lambda$, $f_{\lambda}$ is ordinally concave. Furthermore, this notion of concavity has a
clear interpretation.


\begin{definition}\label{def:strict-pseudo}
The diversity index $f: \Xi^0 \rightarrow \mathbb{R}_+$ is \textbf{semistrictly pseudo M$^\natural$-concave} if, for each
$\xi, \tilde{\xi} \in \Xi^0$ and $(c, t) \in \mathcal{C} \times \mathcal{T}$ with $\xi_c^t>\tilde{\xi}_{c}^t$, then
there exists $\left(c^{\prime}, t^{\prime}\right) \in (\mathcal{C} \times \mathcal{T})\cup\{\emptyset\}$ (with $\xi_{c^{\prime}}^{t^{\prime}}<\tilde{\xi}_{c^{\prime}}^{t^{\prime}}$ whenever $(c',t')\neq \emptyset$) such that
$$
\min\{f(\xi),f(\tilde{\xi})\}\leq\min\{f(\xi-\chi_{c,t}+\chi_{c',t'}), f(\tilde{\xi}+\chi_{c,t}-\chi_{c',t'})\},
$$
with strict inequality holding whenever $f(\xi)\neq f(\tilde \xi)$ and $\xi-\chi_{c,t}+\chi_{c',t'}\neq \tilde \xi$.
\end{definition}
The difference from pseudo M$^\natural$-concavity is that the increase in the minimum value must be strict if the two function values are different and the two distributions do not coincide with each other as a result of moving toward each other.\footnote{Note that $\xi-\chi_{c,t}+\chi_{c',t'}\neq \tilde \xi$ is equivalent to $\{\xi, \tilde \xi\}\cap \{\xi-\chi_{c,t}+\chi_{c',t'}, \tilde \xi+\chi_{c,t}-\chi_{c',t'}\}=\emptyset$.} One can verify that semistrict pseudo M$^\natural$-concavity implies pseudo M$^\natural$-concavity$^+$.\footnote{The converse of this implication does not hold. Let $\mathcal{C}=\{c\}$ and $\mathcal{T}=\{t,t'\}$; we identify $\mathbb{Z}^{|\mathcal{C}|\times |\mathcal{T}|}_+$ with $\mathbb{Z}^2_+$. Let $f:\Xi^0\rightarrow \mathbb{R}_+$ be such that
$$
\Xi^0=\{(0,0),(0,1),(1,0),(1,1)\}\subseteq \mathbb{Z}^2_+, \: f(0,0)=f(1,0)=0, \:  f(0,1)=f(1,1)=1.
$$
This function satisfies pseudo M$^\natural$-concavity$^+$ but violates semistrict pseudo M$^\natural$-concavity. For $\xi=(1,1)$, $\tilde \xi=(0,0)$, and $(c,t)$ with $\chi_{c,t}=(1,0)$,
$$
f(\xi)=f(\xi-\chi_{c,t})=1, \: f(\tilde \xi)=f(\tilde \xi+\chi_{c,t})=0,
$$
showing that the minimum function value does not strictly increase although $f(\xi)\neq f(\tilde \xi)$ and $\xi-\chi_{c,t}\neq \tilde \xi$.
}


Semistrict pseudo M$^\natural$-concavity can be viewed as a discrete analogue of a variant of {\it quasi concavity},
which has been studied extensively in microeconomic analysis.\footnote{In a market model with continuous commodities, if a preference relation over the commodity space is convex, then any utility function representing the preference relation is quasi concave; see Section 3.C of \cite{masco95}.}
We say that a continuous function $f: \mathbb{R}^{|\mathcal{C}|\times |\mathcal{T}|}\rightarrow \mathbb{R}$ is {\it semistrictly quasi concave},\footnote{Precisely speaking, semistrict quasi concavity is defined for a possibly discontinuous function as follows: for each $\xi, \tilde \xi\in \mathbb{R}^{|\mathcal{C}|\times |\mathcal{T}|}$ and $\lambda \in (0,1)$,
$\min\{f(\xi), f(\tilde \xi)\}<f(\lambda \xi+(1-\lambda) \tilde \xi)$ whenever $f(\xi)\neq f(\tilde \xi)$. If $f$ is continuous, this condition is equivalent to the one in the main text.} if for each $\xi, \tilde \xi\in \mathbb{R}^{|\mathcal{C}|\times |\mathcal{T}|}$ and $\lambda \in (0,1)$,
\begin{align*}
\min\{f(\xi), f(\tilde \xi)\}\leq f(\lambda \xi+(1-\lambda) \tilde \xi),
\end{align*}
with strict inequality holding whenever $f(\xi)\neq f(\tilde \xi)$.
Both semistrict pseudo M$^\natural$-concavity and semistrict quasi concavity state that the minimum function value increases, with the increase being strict whenever the original function values are different.\footnote{There is a subtle difference between continuous and discrete domains. For each $\xi, \tilde \xi\in \mathbb{R}^{|\mathcal{C}|\times |\mathcal{T}|}$ with $\xi\neq \tilde \xi$, it always holds that $\lambda \xi+(1-\lambda) \tilde \xi\neq \tilde \xi$ if $\lambda \in (0,1)$. In a discrete domain, however, it is possible that $\xi-\chi_{c,t}+\chi_{c',t'}=\tilde \xi$. Hence, we add a condition that these two distributions are distinct in the definition of semistrict pseudo M$^\natural$-concavity.}


\section{Main Proofs}\label{app:proofs}
In this section, we include the proofs of our main result.

For each contract $x\in \mathcal{X}$, the
school associated with the contract is denoted by $\gamma(x) \in \mathcal{C}$ and the
student associated with the contract is denoted by $\sigma(x) \in \mathcal{S}$.

\subsection*{Proof of Lemma \ref{lem:matchar}}
%\begin{proof}[Proof of Lemma \ref{lem:matchar}]
The collection of bases of a matroid satisfies \emph{B1} and \emph{B2}. Furthermore,
\emph{B2} implies \emph{B2'}. Therefore, the collection of bases of a matroid
satisfies \emph{B1} and \emph{B2'}. To finish
the proof, we need to show that \emph{B1} and \emph{B2'} imply \emph{B2}.

Suppose, for contradiction, that \emph{B2} does not hold. Then, there exist
$X_1, X_2 \in \mathcal{B}$ and $x_1 \in X_1 \setminus X_2$ such that
for each $x\in X_2 \setminus X_1$ we have
$(X_1\setminus \{x_1\}) \cup \{x\} \not\in \mathcal{B}$.
\emph{B2'} implies that there exist $x_2 \in X_2 \setminus X_1$
and $Y \in \mathcal{B}$ such that
$(X_1\setminus \{x_1\}) \cup \{x_2\} \subseteq Y$. Note that we also have
$(X_1\setminus \{x_1\}) \cup \{x_2\} \not\in \mathcal{B}$ since
$x_2\in X_2 \setminus X_1$ and, therefore, we can take $x=x_2$ in
$(X_1\setminus \{x_1\}) \cup \{x\} \not\in \mathcal{B}$. Furthermore,
since \emph{B2'} implies that there cannot be two sets in $\mathcal{B}$ such
that one is a proper subset of the other, $X_1$ is not a
subset of $Y$. Therefore, $x_1\notin Y$ because otherwise $X_1$ would be
a proper subset of $Y$.

Let $Z=Y \setminus (X_1\setminus \{x_1\})$. Then $Z=Y \setminus X_1$ since $Y$ does
not include $x_1$. Furthermore, $x_2\in Y$ and $x_2\notin X_1$ imply that $x_2\in Z$.

Now let $X_1^{*} = Y$ and $X_2^{*} = X_1$. We have
\begin{enumerate}[(i)]
\item $X_1^{*}, X_2^{*} \in \mathcal B$,
\item $X_1^{*} \setminus X_2^{*} = Y \setminus X_1 = Z$, and
\item $X_2^{*} \setminus X_1^{*} = X_1 \setminus Y = \{x_1\}$.
\end{enumerate}
By \emph{B2'}, since $x_2\in X_1^{*} \setminus X_2^{*}=Z$, there exists
$y \in X_2^{*} \setminus X_1^{*}=\{x_1\}$ such that
$(X_1^{*} \setminus \{x_2\}) \cup \{y\} \subseteq Y'$ for
some $Y'\in \mathcal{B}$. However, $y=x_1$ implies
$(X_1^{*} \setminus \{x_2\}) \cup \{y\}
= (Y \setminus \{x_2\}) \cup \{x_1\} \supseteq X_1$. Since $Y'\supseteq X_1$ and $Y',X_1\in \mathcal{B}$, \emph{B2'} implies that $Y'=X_1$. Hence,
\[X_1=Y' \supseteq (Y \setminus \{x_2\}) \cup \{x_1\},\]
which implies that $Y=(X_1\setminus \{x_1\}) \cup \{x_2\}$ because, by
construction, $Y \supseteq (X_1\setminus \{x_1\}) \cup \{x_2\}$ and $x_1\notin Y$.
This is a contradiction since $(X_1\setminus \{x_1\}) \cup \{x_2\} \notin \mathcal{B}$ and $Y\in \mathcal{B}$. Hence, \emph{B1} and \emph{B2'} imply \emph{B2}.
Therefore, \emph{B1} and \emph{B2'} provide a characterization of the collection
of bases of a matroid.
%\end{proof}
\qed

%==================================================
\bigskip





\subsection*{Proof of Lemma \ref{lem:mconvex}}
%\begin{proof}[Proof of Lemma \ref{lem:mconvex}]\renewcommand{\qedsymbol}{$\blacksquare$}
In their Proposition 3.1, \cite{murotashioura2018} provide an equivalent condition for M$^{\natural}$-convexity.

%We use the following definition of M$^{\natural}$-convexity \citep[Proposition 3.1]{murotashioura2018}.

\begin{lemma}\label{lem:natalt}
A set of distributions $\Xi$ is M$^{\natural}$\textbf{-convex} if and only if, for each $\xi,\tilde{\xi} \in \Xi$,
\begin{enumerate}[(i)]
\item $\norm{\xi}>\norm{\tilde \xi}$ implies that there exists $(c,t) \in \calc \times \calt$ with $\xi_c^t>\tilde{\xi}_c^t$ such that $\xi-\chi_{c,t} \in \Xi$ and $\tilde{\xi}+\chi_{c,t} \in \Xi$, and
\item $\norm{\xi}=\norm{\tilde \xi}$ implies that for each $(c,t) \in \calc \times \calt$ with $\xi_c^t>\tilde{\xi}_c^t$, there exists $(c',t') \in \calc \times \calt$ with  $\xi_{c'}^{t'}<\tilde{\xi}_{c'}^{t'}$ such that
\[\xi-\chi_{c,t}+\chi_{c',t'}\in \Xi \; \mbox{ and } \; \tilde{\xi}+\chi_{c,t}-\chi_{c',t'} \in \Xi.\]
\end{enumerate}
\end{lemma}


Let $\Xi$ be a finite and non-empty M$^{\natural}$-convex set and
$\mathcal{M}$ the set of maximal distributions in $\Xi$. Then
there exists at least one distribution in $\mathcal{M}$.
If there exists exactly one distribution in $\mathcal{M}$, then it is trivially M-convex.
For the rest of the proof, suppose that $\mathcal{M}$ has at least two distributions.

Let $\xi, \tilde \xi \in \mathcal{M}$ be distinct. Without loss of generality assume
that $\norm{\xi} \ge \norm{\tilde \xi}$.

If $\norm{\xi} > \norm{\tilde \xi}$, then, by Lemma \ref{lem:natalt}, there exists
$(c,t) \in \calc \times \calt$ with $\xi_c^t>\tilde{\xi}_c^t$ such that
$\xi-\chi_{c,t} \in \Xi$ and $\tilde{\xi}+\chi_{c,t} \in \Xi$. However, $\tilde{\xi}+\chi_{c,t} \in \Xi$ contradicts the assumption that $\tilde{\xi}$ is
maximal in $\Xi$. Therefore, we must have $\norm{\xi} = \norm{\tilde \xi}$, which
implies that every distribution in $\mathcal{M}$ has the same sum of coordinates.
Furthermore, every distribution in $\Xi$ that has the same sum of coordinates also has
to be maximal.

By Lemma \ref{lem:natalt}, for each
$(c,t) \in \calc \times \calt$ with $\xi_c^t>\tilde{\xi}_c^t$, there exists
$(c',t') \in \calc \times \calt$ with  $\xi_{c'}^{t'}<\tilde{\xi}_{c'}^{t'}$ such that
\[\xi-\chi_{c,t}+\chi_{c',t'}\in \Xi \; \mbox{ and } \; \tilde{\xi}+\chi_{c,t}-\chi_{c',t'} \in \Xi.\]
The equations above imply
that both distributions are also maximal in $\Xi$ because
$\norm{\xi-\chi_{c,t}+\chi_{c',t'}}=\norm{\xi}$ and
$\norm{\tilde{\xi}+\chi_{c,t}-\chi_{c',t'}}=\norm{\tilde \xi}$. Therefore, we
get $\xi-\chi_{c,t}+\chi_{c',t'}\in \mathcal{M}$ and
$\tilde{\xi}+\chi_{c,t}-\chi_{c',t'} \in \mathcal{M}$, which establishes that $\mathcal{M}$
is an M-convex set.
\qed

%==================================================
\bigskip




\subsection*{Proof of Theorem \ref{thm:diversitychoice}}
We first prove parts (i) and (ii) using the following lemmas.

\begin{lemma}\label{lem:pareto}
Suppose that the diversity index $f$ is \oconcave{}.
For each set of contracts $X\subseteq \mathcal{X}$, the set of maximal distributions in $\Xi^*(X)$ is M-convex.
\end{lemma}

\begin{proof}[Proof of Lemma \ref{lem:pareto}]\renewcommand{\qedsymbol}{$\blacksquare$}
Let $\xi,\tilde \xi \in \Xi^*(X)$ be two distinct distributions, $c \in \calc$ a school,
and $t \in \calt$ a type such that $\xi_c^t>\tilde{\xi}_c^t$.
By ordinal concavity, %pseudo $M^{\natural}$-concavity,
either (i)
\begin{center}
$f(\xi-\chi_{c,t})=f(\xi)$ and $f(\tilde{\xi}+\chi_{c,t})=f(\tilde{\xi})$
\end{center}
or (ii) there exist school $c'\in \mathcal{C}$ and type $t'\in \mathcal{T}$ with $\xi_{c'}^{t'}<\tilde{\xi}_{c'}^{t'}$ such that
  \begin{center}
$f(\xi-\chi_{c,t}+\chi_{c',t'})=f(\xi)$ and $f(\tilde{\xi}+\chi_{c,t}-\chi_{c',t'}) = f(\tilde{\xi})$.
  \end{center}
If (i) holds, then $\xi-\chi_{c,t} \in \Xi^*(X)$ and $\tilde{\xi}+\chi_{c,t} \in \Xi^*(X)$.
Otherwise, if (ii) holds, then $\xi-\chi_{c,t}+\chi_{c',t'} \in \Xi^*(X)$ and $\tilde{\xi}+\chi_{c,t}-\chi_{c',t'} \in \Xi^*(X)$. Therefore, $\Xi^*(X)$ is an M$^{\natural}$-convex set.

We finish the proof by using Lemma \ref{lem:mconvex}:
M$^{\natural}$-convexity of $\Xi^*(X)$ implies that the set of maximal distributions
in $\Xi^*(X)$ is M-convex.
\end{proof}

Recall the definition of $\mathcal{F}(X) \equiv \{Y \subseteq X | \xi(Y) \leq \xi \mbox{ for some } \xi \in \Xi^*(X)\}$.

\begin{lemma}\label{lem:matroid}
Suppose that the diversity index $f$ is \oconcave{}.
For each set of contracts $X\subseteq \mathcal{X}$, $(X,\mathcal{F}(X))$ is a matroid.
\end{lemma}

\begin{proof}[Proof of Lemma \ref{lem:matroid}]\renewcommand{\qedsymbol}{$\blacksquare$}
We show that the maximal sets in $\mathcal{F}(X)$ satisfy \emph{B1} and \emph{B2'},
which together with Lemma \ref{lem:matchar} implies that they are the bases of a matroid. Since $\mathcal{F}(X)$ satisfies \emph{I2}, $\mathcal{F}(X)$ is the collection of subsets of the bases, which implies that $(X,\mathcal{F}(X))$ is a matroid (see Theorem 1.2.3 of \cite{oxley}).
%which implies that they are the bases of matroid $(X,\mathcal{F}(X))$ (Lemma \ref{lem:matchar}) because $\mathcal{F}(X)$ satisfies \emph{I2}.
Since $X$ is a finite set, $\Xi^*(X)$ is nonempty. Therefore, \emph{B1} is satisfied.

We now show \emph{B2'}. Let $X_1$ and $X_2$ be two distinct maximal sets in $\mathcal{F}(X)$. Then, by construction, $\xi(X_1)$ and $\xi(X_2)$ are
maximal distributions in $\Xi^*(X)$. %As a result, $f(\xi(X_1))=f(\xi(X_2))$.
%and, by Lemma \ref{lem:mconvex}, $\xi(X_1)$ and $\xi(X_2)$ lie in an M-convex set.
We consider two cases in the rest of the proof.

In the first case, for each school $c\in \calc$ and type $t\in \calt$, $\xi_c^t(X_1)=\xi_c^t(X_2)$.
Since $X_1 \neq X_2$, $|X_1 \setminus X_2|>0$. Then,
for each $x_1 \in X_1 \setminus X_2$,
there exists $x_2 \in X_2 \setminus X_1$ such that
$\gamma(x_1)=\gamma(x_2)$ and $\tau(\sigma(x_1))=\tau(\sigma(x_2))$. Therefore, $\xi((X_1\setminus \{x_1\}) \cup \{x_2\})=\xi(X_1)$ and so $f(\xi((X_1\setminus \{x_1\})\cup \{x_2\})=f(\xi(X_1))$,
which implies that $(X_1\setminus \{x_1\})\cup \{x_2\}\in \mathcal{F}(X)$. Therefore, \emph{B2'} is satisfied.

In the second case, there exist school $c\in \calc$ and type $t\in \calt$ such that
$\xi_c^t(X_1)>\xi_c^t(X_2)$. Since $\xi(X_1),\xi(X_2) \in \Xi^*(X)$ and
the set of maximal distributions in $\Xi^*(X)$ is an M-convex set (Lemma \ref{lem:mconvex}),
there exist school $c'\in \calc$ and type $t'\in \calt$  with $\xi_{c'}^{t'}(X_1)<\xi_{c'}^{t'}(X_2)$ such that
$\xi(X_1)-\chi_{c,t}+\chi_{c',t'} \in \Xi^*(X)$ and
$\xi(X_2)+\chi_{c,t}-\chi_{c',t'} \in \Xi^*(X)$. Since $\xi_c^t(X_1)>\xi_c^t(X_2)$ and $\xi_{c'}^{t'}(X_1)<\xi_{c'}^{t'}(X_2)$, there exist
$x_1 \in X_1 \setminus X_2$ and $x_2 \in X_2 \setminus X_1$ such that
$\gamma(x_1)=c$, $\tau(\sigma(x_1))=t$, $\gamma(x_2)=c'$,
and $\tau(\sigma(x_2))=t'$. Therefore,
\[\xi((X_1\setminus \{x_1\})\cup \{x_2\})=\xi(X_1)-\chi_{c,t}+\chi_{c',t'}\in \Xi^*(X),\]
which implies that $(X_1\setminus \{x_1\})\cup \{x_2\} \in \mathcal{F}(X)$.
Therefore, \emph{B2'} is satisfied.

In both cases, we have shown \emph{B1} and \emph{B2'} and $(X,\mathcal{F}(X))$ is a matroid.
\end{proof}

\begin{lemma}\label{lem:equal}
Suppose that the diversity index $f$ is \oconcave. Then, for each set of contracts
$X\subseteq \mathcal{X}$, the greedy rule on matroid $(X,\mathcal{F}(X))$ produces
$C^d(X)$ when the set of available contracts is $X$.\footnote{To dfine the greedy rule, we set a weight function in such a way that a contract with a higher merit has a higher weight.}
\end{lemma}

\begin{proof}[Proof of Lemma \ref{lem:equal}]\renewcommand{\qedsymbol}{$\blacksquare$}
%By Lemma \ref{lem:matroid}, $(X,\mathcal{F}(X))$ is a matroid.
We show by induction
that $C^d$ and the greedy rule choose the same set of contracts for each index $k$
used in the definitions of both choice rules and terminate at the same index. Let $X_k$ be defined as in the construction of $C^d(X)$ and $X'_k$ be analogously
defined for the greedy rule. For $k=0$, we have $X_k=\emptyset=X'_k$.
By mathematical induction hypothesis, suppose that $X_j=X'_j$ for
each $j=0,\ldots,k$. We now show the hypothesis for $j=k+1$.

By the induction hypothesis, $\{x\in X\setminus X_k| \exists \xi\in \Xi^*(X) \mbox{ s.t. } \xi(X_k \cup \{x\}) \leq \xi\}$ used in the construction of $C^d$ is the same as
$\{x \in X\setminus X'_k|\exists Y \subseteq \mathcal{F}(X)
\mbox{ s.t. } X'_k \cup \{x\} \subseteq Y\}$ used in the greedy rule description.  Therefore, either both algorithms terminate at index $k$ and produce $X_k=X'_k$
or the same contract $x$ is chosen so that $X_{k+1}=X'_{k+1}$. This
finishes the proof of the mathematical induction hypothesis.

Therefore, the greedy rule on matroid $(X,\mathcal{F}(X))$ produces $C^d(X)$.
\end{proof}

Now, we finish the proofs of parts (i) and (ii).
By Lemma \ref{lem:equal}, $C^d(X)$ is a base of the matroid $(X,\mathcal{F}(X))$.
Therefore, by construction of $\mathcal{F}(X)$, $\xi(C^d(X))\in \Xi^*(X)$,
which means that $C^d(X)$ maximizes the diversity index $f$ among subsets of $X$.
Furthermore, by \citep{gale1968}, $C^d(X)$ merit dominates each set in $\mathcal{F}(X)$, which includes all subsets of $X$ that maximizes the diversity index.

We continue with the proof of part (iii). We prove the result in a number of steps.

\paragraph{Step 1: } We prove the so-called {\it maximizer-cut theorem} for ordinally concave functions.\footnote{The maximizer-cut theorem is originally proved for M-convex functions under the name of {\it minimizer-cut theorem}; see Theorem 6.28 of \cite{Murota:SIAM:2003}. Our proof relies on Murota's proof. Roughly speaking, this theorem states that we can ``cut'' non-maximizers from the domain containing a maximizer of $f$.}
\begin{lemma}\label{lem:maximizer-cut-1}
Let $f$ be ordinally concave, $\xi\in \Xi^0$, $(c,t)\in (\mathcal{C}\times \mathcal{T})\cup\{\emptyset\}$, and $(c',t')\in (\mathcal{C}\times \mathcal{T})\cup\{\emptyset\}$ be such that
\begin{align*}
f(\xi-\chi_{c',t'}+\chi_{c,t})=\max_{(\tilde c', \tilde t')\in (\mathcal{C}\times \mathcal{T})\cup \{\emptyset\}}f(\xi-\chi_{\tilde c' \tilde t'}+\chi_{c,t}).
\end{align*}
Then, there exists $\xi^*\in \underset {\xi\in \Xi^0} {\arg\max} \: f(\xi)$ with $(\xi^*)_{c'}^{t'}\leq \xi_{c'}^{t'}-1+(\chi_{c,t})_{c'}^{t'}$.
\end{lemma}
\begin{proof}[Proof of Lemma \ref{lem:maximizer-cut-1}]\renewcommand{\qedsymbol}{$\blacksquare$}
Let $\xi'=\xi-\chi_{c',t'}+\chi_{c,t}$. Suppose, for contradiction, that there does not exist $\xi^*\in \underset {\xi\in \Xi^0} {\arg\max} \: f(\xi)$ with $(\xi^*)_{c'}^{t'}\leq (\xi')_{c'}^{t'}$. Let $\xi^*$ be an element of $\underset {\xi\in \Xi^0} {\arg\max} \: f(\xi)$ that minimizes the $(c',t')$ coordinate. By assumption, we have $(\xi^*)_{c'}^{t'}>(\xi')_{c'}^{t'}$. By ordinal concavity, there exists $(c'', t'') \in (\mathcal{C}\times \mathcal{T})\cup\{\emptyset\}$ (with $(\xi')_{c''}^{t''}>(\xi^*)_{c''}^{t''}$ if $(c'',t'')\neq \emptyset$) such that
\begin{enumerate}
\item $f(\xi^*-\chi_{c',t'}+\chi_{c'',t''})>f(\xi^*)$ or
\item $f(\xi'+\chi_{c',t'}-\chi_{c'',t''})>f(\xi')$ or
\item $f(\xi^*-\chi_{c',t'}+\chi_{c'',t''})=f(\xi^*)$ and $f(\xi'+\chi_{c',t'}-\chi_{c'',t''})=f(\xi')$.
\end{enumerate}
If condition (3) holds, then $\xi^*-\chi_{c',t'}+\chi_{c'',t''}\in \underset {\xi\in \Xi^0} {\arg\max} \: f(\xi)$ and $(\xi^*-\chi_{c',t'}+\chi_{c'',t''})_{c'}^{t'}<(\xi^*)_{c'}^{t'}$, a contradiction to the choice of $\xi^*$. Condition (1) is impossible because $\xi^*\in \underset {\xi\in \Xi^0} {\arg\max} \: f(\xi)$. If condition (2) holds,
\begin{align*}
f(\xi-\chi_{c'',t''}+\chi_{c,t})=f(\xi'+\chi_{c',t'}-\chi_{c'',t''})>f(\xi')=f(\xi-\chi_{c',t'}+\chi_{c,t}),
\end{align*}
a contradiction to the choice of $(c',t')$.
\end{proof}

\begin{lemma} \label{lem:maximizer-cut-2}
Let $f$ be ordinally concave, $\xi \in \Xi^0$ with $\xi\notin \underset {\xi\in \Xi^0} {\arg\max} \: f(\xi)$, and $(c,t), (c',t')\in (\mathcal{C}\times \mathcal{T})\cup\{\emptyset\}$ be such that
\begin{align*}
f(\xi-\chi_{c',t'}+\chi_{c,t})=\max_{(\tilde c', \tilde t')\in (\mathcal{C}\times \mathcal{T})\cup\{\emptyset\}}\max_{(\tilde c, \tilde t)\in (\mathcal{C}\times \mathcal{T})\cup\{\emptyset\}}f(\xi-\chi_{\tilde c', \tilde t'}+\chi_{\tilde c, \tilde t}).
\end{align*}
Then, $(c,t)\neq \emptyset$ or $(c',t')\neq \emptyset$ holds.
\end{lemma}
\begin{proof}[Proof of Lemma \ref{lem:maximizer-cut-2}]\renewcommand{\qedsymbol}{$\blacksquare$}
Suppose, for contradiction, that $(c,t)=(c',t')=\emptyset$, i.e.,
$$
f(\xi)=\max_{(\tilde c', \tilde t')\in (\mathcal{C}\times \mathcal{T})\cup\{\emptyset\}}\max_{(\tilde c, \tilde t)\in (\mathcal{C}\times \mathcal{T})\cup\{\emptyset\}}f(\xi-\chi_{\tilde c', \tilde t'}+\chi_{\tilde c, \tilde t}).
$$
Let $\xi^*$ be an element of $\underset {\xi\in \Xi^0} {\arg\max} \: f(\xi)$ that minimizes $\sum_{(\tilde c, \tilde t)}|(\xi^*)_{\tilde c}^{\tilde t}-\xi_{\tilde c}^{\tilde t}|$. Since $\xi\notin \underset {\xi\in \Xi^0} {\arg\max} \: f(\xi)$, there exists $(c'',t'')\in \mathcal{C}\times \mathcal{T}$ with $(\xi^*)_{c''}^{t''}\neq \xi_{c''}^{t''}$. Suppose that  $(\xi^*)_{c''}^{t''}>\xi_{c''}^{t''}$ (the other case $(\xi^*)_{c''}^{t''}<\xi_{c''}^{t''}$ can be handled analogously). By ordinal concavity, there exists $(c''',t''')\in (\mathcal{C}\times \mathcal{T})\cup\{\emptyset\}$ (with $\xi_{c'''}^{t'''}>(\xi^*)_{c'''}^{t'''}$ if $(c''',t''')\neq \emptyset$) such that
\begin{enumerate}
\item $f(\xi^*-\chi_{c'',t''}+\chi_{c''',t'''})>f(\xi^*)$ or
\item $f(\xi+\chi_{c'',t''}-\chi_{c''',t'''})>f(\xi)$ or
\item $f(\xi^*-\chi_{c'',t''}+\chi_{c''',t'''})=f(\xi^*)$ and $f(\xi+\chi_{c'',t''}-\chi_{c''',t'''})=f(\xi)$.
\end{enumerate}
If condition (3) holds, then $\xi^*-\chi_{c'',t''}+\chi_{c''',t'''}\in \underset {\xi\in \Xi^0} {\arg\max} \: f(\xi)$ and
$$
\sum_{(\tilde c, \tilde t)}|(\xi^*-\chi_{c'',t''}+\chi_{c''',t'''})_{\tilde c}^{\tilde t}-\xi_{\tilde c}^{\tilde t}|<\sum_{(\tilde c, \tilde t)}|(\xi^*)_{\tilde c}^{\tilde t}-\xi_{\tilde c}^{\tilde t}|,
$$
which is a contradiction to the choice of
$\xi^*$. Condition (1) is impossible because $\xi^*\in \underset {\xi\in \Xi^0} {\arg\max} \: f(\xi)$. If condition (2) holds, we obtain a contradiction to the assumption made in the beginning of the proof.
\end{proof}

\begin{theorem}[Maximizer-cut theorem] \label{thm:maximizer-cut}
Let $f$ be ordinally concave, $\xi\in \Xi^0$ with $\xi \not\in \underset {\xi\in \Xi^0} {\arg\max} \: f(\xi)$, and $(c,t), (c',t') \in (\mathcal C \times \mathcal T) \cup \{\emptyset\}$ be such that
$$
f(\xi-\chi_{c',t'}+\chi_{c,t}) = \max_{(\tilde c,\tilde t), (\tilde c',\tilde t') \in (\mathcal C \times \mathcal T) \cup \{\emptyset\}} f(\xi-\chi_{\tilde c',\tilde t'}+\chi_{\tilde c,\tilde t}).
$$
Then, $(c,t) \neq \emptyset$ or $(c',t') \neq \emptyset$ holds and the following statements hold:
\begin{enumerate}[(i)]
\item If $(c,t) \neq \emptyset$ and $(c',t') = \emptyset$, then there exists $\xi^*\in \underset {\xi\in \Xi^0} {\arg\max} \: f(\xi)$ with $(\xi^*)^t_c \ge  \xi^t_c+1$,
\item If $(c,t) = \emptyset$ and $(c',t') \neq \emptyset$, then there exists $\xi^* \in \underset {\xi\in \Xi^0} {\arg\max} \: f(\xi)$ with $(\xi^*)^{t'}_{c'} \le \xi^{t'}_{c'}-1$,
\item If $(c,t) \neq \emptyset$ and $(c',t') \neq \emptyset$, then there exists $\xi^* \in \underset {\xi\in \Xi^0} {\arg\max} \: f(\xi)$ with $(\xi^*)^t_c \ge \xi^t_c+1$ and $(\xi^*)^{t'}_{c'} \le \xi^{t'}_{c'}-1$.
\end{enumerate}
\end{theorem}
\begin{proof}[Proof of Theorem \ref{thm:maximizer-cut}]
Note that $(c,t)\neq \emptyset$ or $(c',t')\neq \emptyset$ follows from Lemma \ref{lem:maximizer-cut-2}.

\smallskip
\noindent
\emph{Proof of (i):}
%\textbf{Proof of (i):}
Let $\xi'=\xi+\chi_{c,t}$. Suppose, for contradiction, that there does not exist $\xi^*\in \underset {\xi\in \Xi^0} {\arg\max} \: f(\xi)$ with $(\xi^*)_{c}^{t}\geq (\xi')_{c}^{t}$.  Let $\xi^*$ be an element of $\underset {\xi\in \Xi^0} {\arg\max} \: f(\xi)$ that maximizes the $(c,t)$ coordinate. By assumption, we have $(\xi^*)_{c}^{t}<(\xi')_{c}^{t}$. By ordinal concavity, there exists $(c'',t'') \in (\mathcal{C}\times \mathcal{T})\cup\{\emptyset\}$ (with $(\xi^*)_{c''}^{t''}>(\xi')_{c''}^{t''}$ if $(c'',t'')\neq \emptyset$) such that
\begin{enumerate}
\item $f(\xi'-\chi_{c,t}+\chi_{c'',t''})>f(\xi')$ or
\item $f(\xi^*+\chi_{c,t}-\chi_{c'',t''})>f(\xi^*)$ or
\item $f(\xi'-\chi_{c,t}+\chi_{c'',t''})=f(\xi')$ and $f(\xi^*+\chi_{c,t}-\chi_{c'',t''})=f(\xi^*)$.
\end{enumerate}
If condition (3) holds, then $\xi^*+\chi_{c,t}-\chi_{c'',t''}\in \underset {\xi\in \Xi^0} {\arg\max} \: f(\xi)$ and $(\xi^*+\chi_{c,t}-\chi_{c'',t''})_{c}^{t}>(\xi^*)_{c}^{t}$, a contradiction to the choice of $\xi^*$. Condition (2) is impossible because $\xi^*\in \underset {\xi\in \Xi^0} {\arg\max} \: f(\xi)$. If condition (1) holds,
$$
f(\xi+\chi_{c'',t''})=f(\xi'-\chi_{c,t}+\chi_{c'',t''})>f(\xi')=f(\xi+\chi_{c,t}),
$$
which is a contradiction to the choices of $(c,t)$ and $(c',t')$.

\smallskip
\noindent
\emph{Proof of (ii):} The proof is similar to that for (i).

\smallskip
\noindent
\emph{Proof of (iii):}  Let $\xi'=\xi-\chi_{c',t'}+\chi_{c,t}$. By Lemma \ref{lem:maximizer-cut-1}, there exists $\xi^*\in \underset {\xi\in \Xi^0} {\arg\max} \: f(\xi)$ such that $(\xi^*)_{c'}^{t'}\leq (\xi')_{c'}^{t'}$; we assume $\xi^*$ maximizes $(\xi^*)_{c}^{t}$ among all such vectors. Suppose, for contradiction, that $(\xi^*)_{c}^{t}\geq (\xi')_{c}^{t}$ is not satisfied, i.e., $(\xi^*)_{c}^{t}<(\xi')_{c}^{t}$. By ordinal concavity, there exists $(c'',t'')\in (\mathcal{C}\times \mathcal{T})\cup\{\emptyset\}$ (with $(\xi^*)_{c''}^{t''}>(\xi')_{c''}^{t''}$ if $(c'',t'')\neq \emptyset$) such that
\begin{enumerate}
\item $f(\xi'-\chi_{c,t}+\chi_{c'',t''})>f(\xi')$ or
\item $f(\xi^*+\chi_{c,t}-\chi_{c'',t''})>f(\xi^*)$ or
\item $f(\xi'-\chi_{c,t}+\chi_{c'',t''})=f(\xi')$ and $f(\xi^*+\chi_{c,t}-\chi_{c'',t''})=f(\xi^*)$.
\end{enumerate}
Suppose that condition (3) holds, which implies $\xi^*+\chi_{c,t}-\chi_{c'',t''}\in \underset {\xi\in \Xi^0} {\arg\max} \: f(\xi)$. By Lemma \ref{lem:maximizer-cut-2}, we have $(c,t)\neq (c',t')$ and hence $(\xi^*+\chi_{c,t}-\chi_{c'',t''})_{c'}^{t'}\leq (\xi^*)_{c'}^{t'}$. Together with $(\xi^*+\chi_{c,t}-\chi_{c'',t''})_c^t>(\xi^*)_c^t$, we obtain a  contradiction to the choice of $\xi^*$. Condition (2) is impossible because $\xi^*\in \underset {\xi\in \Xi^0} {\arg\max} \: f(\xi)$. If condition (1) holds,
\begin{align*}
f(\xi-\chi_{c',t'}+\chi_{c'',t''})=f(\xi'-\chi_{c,t}+\chi_{c'',t''})>f(\xi')=f(\xi-\chi_{c',t'}+\chi_{c,t}),
\end{align*}
which is a contradiction to the choices of $(c,t)$ and $(c',t')$.
\end{proof}
Two remarks on Theorem \ref{thm:maximizer-cut} are in order.
\begin{itemize}
\item Although we assume that $\Xi^0\subseteq \mathbb{Z}^{|\mathcal{C}|\times |\mathcal{T}|}_+$, $\mathbf{0}\in \Xi^0$, and $f(\xi)\geq 0$ for each
$\xi\in \Xi^0$, neither of these assumptions is used in the proof. Hence,
the maximizer-cut theorem holds for ordinally concave functions more generally.
\item Among the three statements (i)-(iii), we use only the first one in the proof below.
\end{itemize}

\paragraph{Step 2: } We develop a variation of the {\it domain-reduction algorithm} that produces a maximizer of the diversity index that is maximal in the set of maximizers.\footnote{The domain-reduction algorithm is originally introduced for M-convex functions; see Section 10.1.3 of \cite{Murota:SIAM:2003}.}
Fix an ordinally concave $f$.
\medskip
\paragraph{\textbf{Domain-reduction algorithm}}
\begin{description}
\item[Input] Let $X$ be a set of contracts.
\item[Step 1] Set $\xi_0=\mathbf{0}$ and $k=0$.
\item[Step 2] Check if
$$
f(\xi_k) \leq \max\{f(\xi_k+\chi_{\tilde c, \tilde t})\mid (\tilde c, \tilde t) \in \mathcal C \times \mathcal T, \xi_k+\chi_{\tilde c, \tilde t}\leq \xi(X)\}.
$$
If this is the case, then choose a maximizer $(c_{k+1}, t_{k+1})$ of the right-hand side, let $\xi_{k+1}=\xi_k+\chi_{c_{k+1}, t_{k+1}}$, and go to Step 3. Otherwise, go to Step 4.
\item[Step 3] Add 1 to $k$ and go to Step 2.
\item[Step 4] Return $\xi_{k}$ and stop.
\end{description}
Let $k^*$ denote the value of $k$ at the end of the algorithm.
For each $k\in \{0, \dots, k^*\}$, let $\Xi^0_k=\{\xi \in \Xi^0\mid \xi_k\leq \xi\leq \xi(X)\}$ and $f_k: \Xi^0_k\rightarrow \mathbb{R}_+$ be defined as $f_k(\xi)=f(\xi)$ for all $\xi\in \Xi^0_k$. One can verify that ordinal concavity of $f$ is inhereted to $f_k$ for each $k$.
We prove that the algorithm produces a maximal distribution in $\underset {\xi\in \Xi^0_0} {\arg\max} \: f_0(\xi)=\Xi^*(X)$ (recall the notation in the definition of the diversity choice rule) by establishing three lemmas.
\begin{lemma}\label{lem:maximum-equal}
For each $k\in \{0, \dots, k^*\},$ $\underset {\xi\in \Xi^0_k} {\max} \: f_k(\xi)=\underset {\xi\in \Xi^0_0} {\max} \: f_0(\xi)$.
\end{lemma}
\begin{proof}[Proof of Lemma \ref{lem:maximum-equal}]\renewcommand{\qedsymbol}{$\blacksquare$}
The proof is by mathematical induction. The claim trivially holds for $k=0$. Suppose that it holds for $k-1$. We show the claim for $k$.

\smallskip
\noindent
\emph{Case 1:}
Suppose that $\xi_{k-1}$ is a maximizer of $f_{k-1}$. Then, $f_{k-1}(\xi_{k-1}+\chi_{c_{k}, t_{k}})\leq f_{k-1}(\xi_{k-1})$. Together with $f(\xi_{k-1}+\chi_{c_{k}, t_{k}})\geq f(\xi_{k-1})$ (which follows from the choice of $(c_{k}, t_{k})$) and $f(\xi)=f_{k-1}(\xi)$ for each $\xi\in \Xi^0_{k-1}$, we obtain $f_{k-1}(\xi_{k-1}+\chi_{c_{k}, t_{k}})=f_{k-1}(\xi_{k-1})$. Substituting $f_{k-1}(\xi_{k-1}+\chi_{c_{k}, t_{k}})=f_k(\xi_k)$, we get $f_k(\xi_k)=f_{k-1}(\xi_{k-1})$. Together with $\Xi^0_{k-1}\supseteq \Xi^0_k$ and the assumption of Case 1, $\xi_k$ is a maximizer of $f_k$ and $\underset {\xi\in \Xi^0_k} {\max} \: f_k(\xi)=\underset {\xi\in \Xi^0_{k-1}} {\max} \: f_{k-1}(\xi)$. This equality and mathematical induction hypothesis give us the desired claim.

\smallskip
\noindent
\emph{Case 2:}
Suppose that $\xi_{k-1}$ is not a maximizer of $f_{k-1}$.
\begin{align*}
f_{k-1}(\xi_{k-1}+\chi_{c_{k}, t_{k}})&=f(\xi_{k-1}+\chi_{c_{k}, t_{k}}) \\
&= \max_{(\tilde c,\tilde t) \in (\mathcal C \times \mathcal T)\cup\{\emptyset\}}  f(\xi_{k-1}+\chi_{\tilde c, \tilde t}) \\
&=\max_{(\tilde c,\tilde t) \in (\mathcal C \times \mathcal T)\cup\{\emptyset\}}  f_{k-1}(\xi_{k-1}+\chi_{\tilde c, \tilde t}) \\
&= \max_{(\tilde c,\tilde t), (\tilde c',\tilde t') \in (\mathcal C \times \mathcal T) \cup \{\emptyset\}} f_{k-1}(\xi_{k-1}-\chi_{\tilde c',\tilde t'}+\chi_{\tilde c,\tilde t}),
\end{align*}
where the second equality follows from the choice of $(c_{k}, t_{k})$ and the last equality follows from the fact that every distribution in $\Xi^0_{k-1}$ is greater than or equal to $\xi_{k-1}$.
By Theorem \ref{thm:maximizer-cut}, there exists a maximizer $\xi^*$ of $f_{k-1}$ such that $\xi^* \geq \xi_{k-1}+\chi_{c_{k}, t_{k}}=\xi_k$, which implies $\xi^*\in \Xi^0_k$. Together with $\Xi^0_{k-1}\supseteq \Xi^0_k$, we obtain $\underset {\xi\in \Xi^0_k} {\max} \: f_k(\xi)=\underset {\xi\in \Xi^0_{k-1}} {\max} \: f_{k-1}(\xi)$. This equality and mathematical induction hypothesis give us the desired claim.
\end{proof}

\begin{lemma}\label{lem:outcome-maximizer}
$\xi_{k^*} \in \underset {\xi\in \Xi^0_0} {\arg\max} \: f_0(\xi)$.
\end{lemma}
\begin{proof}[Proof of Lemma \ref{lem:outcome-maximizer}]\renewcommand{\qedsymbol}{$\blacksquare$}
Suppose, for contradiction, that $\xi_{k^*} \notin \underset {\xi\in \Xi^0_0} {\arg\max} \: f_0(\xi)$.
By Lemma \ref{lem:maximum-equal} and $\Xi^0_{k^*}\subseteq \Xi^0_0$, there exists $\xi^* \in \Xi^0_{k^*}$ such that $\xi^*\in \underset {\xi\in \Xi^0_0} {\arg\max} \: f_0(\xi)$ and $f_{k^*}(\xi_{k^*})<f_{k^*}(\xi^*)$, which implies $f(\xi_{k^*})<f(\xi^*)$.
Assume that  $(\xi^*)^t_c >(\xi_{k^*})^t_c$ (such $c$ and $t$ exist because $\xi^*>\xi_{k^*}$ by the definition of $\Xi^0_{k^*}$).
By ordinal concavity of $f$, there exists $(c',t') \in (\mathcal C \times \mathcal T) \cup \{\emptyset\}$, with $(\xi^*)^{t'}_{c'} <(\xi_{k^*})^{t'}_{c'}$ if $(c',t') \neq \emptyset$, such that one of the inequalities required of  ordinal concavity holds.
However, because $\xi^* > \xi_{k^*}$, it follows that $(c',t')=\emptyset$, so we have
\begin{enumerate}
\item $f(\xi_{k^*}+\chi_{c,t})>f(\xi_{k^*})$ or
\item $f(\xi^*-\chi_{c,t})>f(\xi^*)$ or
\item $f(\xi_{k^*}+\chi_{c,t})=f(\xi_{k^*})$ and $f(\xi^*-\chi_{c,t})=(\xi^*)$.
\end{enumerate}
Condition (2) is impossible because $\xi^* \in \underset {\xi\in \Xi^0_0} {\arg\max} \: f_0(\xi)$. Therefore, condition (1) or (3) holds. In either case, because $\xi_{k^*}+\chi_{c,t}\leq \xi^*\leq \xi(X)$, we have
$$
f(\xi_{k^*}) \leq \max\{f(\xi_{k^*}+\chi_{\tilde c, \tilde t})\mid (\tilde c, \tilde t) \in \mathcal C \times \mathcal T, \xi_{k^*}+\chi_{\tilde c, \tilde t}\leq \xi(X)\}.
$$
We obtain a contradiction to the fact that the algorithm terminates when $k=k^*$.
\end{proof}

\begin{lemma}\label{lem:outcome-maximal}
$\xi_{k^*}$ is a maximal distribution in $\underset {\xi\in \Xi^0_0} {\arg\max} \: f_0(\xi)$.
\end{lemma}
\begin{proof}[Proof of Lemma \ref{lem:outcome-maximal}]\renewcommand{\qedsymbol}{$\blacksquare$}
Suppose, for contradiction, that the statement does not hold. By Lemma \ref{lem:outcome-maximizer}, $\xi_{k^*}\in \underset {\xi\in \Xi^0_0} {\arg\max} \: f_0(\xi)$.
Since it is not a maximal distribution, there exists $\xi^*$ such that $\xi^* \in \underset {\xi\in \Xi^0_0} {\arg\max} \: f_0(\xi)$ and $\xi^*>\xi_{k^*}$.
Assume that  $(\xi^*)^t_c >(\xi_{k^*})^t_c$ (such $c$ and $t$ exist because $\xi^*>\xi_{k^*}$).
By ordinal concavity of $f$, there exists $(c',t') \in (\mathcal C \times \mathcal T) \cup \{\emptyset\}$, with $(\xi^*)^{t'}_{c'} <(\xi_{k^*})^{t'}_{c'}$ if $(c',t') \neq \emptyset$, such that one of the inequalities required of  ordinal concavity holds.
However, because $\xi^* > \xi_{k^*}$, it follows that $(c',t')=\emptyset$, so we have
\begin{enumerate}
\item $f(\xi_{k^*}+\chi_{c,t})>f(\xi_{k^*})$ or
\item $f(\xi^*-\chi_{c,t})>f(\xi^*)$ or
\item $f(\xi_{k^*}+\chi_{c,t})=f(\xi_{k^*})$ and $f(\xi^*-\chi_{c,t})=f(\xi^*)$.
\end{enumerate}
If condition (1) or (2) holds, then together with $\xi_{k^*}+\chi_{c,t}\leq \xi^*\leq \xi(X)$ and $\xi^*-\chi_{c,t}\leq \xi^*\leq \xi(X)$, we obtain a contradiction to $\xi_{k^*}, \xi^* \in \underset {\xi\in \Xi^0_0} {\arg\max} \: f_0(\xi)$. Therefore, condition (3) holds, implying that
$$
f(\xi_{k^*}) \leq \max\{f(\xi_{k^*}+\chi_{\tilde c, \tilde t})\mid (\tilde c, \tilde t) \in \mathcal C \times \mathcal T, \xi_{k^*}+\chi_{\tilde c, \tilde t}\leq \xi(X)\}.
$$
We obtain a contradiction to the fact that the algorithm terminates when $k=k^*$.
\end{proof}

\paragraph{Step 3: } We develop a modified version of the diversity choice rule that produces the same outcome as the original one and is more tractable from a computational viewpoint.
Fix an ordinally concave $f$.

\medskip
\paragraph{\textbf{Modified Diversity Choice Rule}}
\begin{description}
  \item[Input] Let $X$ be a set of contracts. Let $\xi$ be a maximal distribution in $\Xi^*(X)$.
  \item[Step 1]  Set $X_0=\emptyset$, $\xi_0=\xi$, and $k=0$.
  %\item[Step 2] If $X_k \cup R_k=X$, then go to Step 5. Otherwise, find a contract $x_{k+1}\in X\backslash (X_{k}\cup R_{k})$ of maximum merit. Let $(c,t)$ be such that $\xi(\{x_{k+1}\})=\chi_{c,t}$.
  \item[Step 2] Check whether there exists $x\in X\backslash X_k$ that satisfies one of the following conditions:
  \begin{itemize}
  \item[(i)]  $\xi(X_{k}\cup\{x\})\leq \xi_{k}$, or
  \item[(ii)] there exists $(c',t')\in \mathcal{C}\times \mathcal{T}$ such that $\xi_{k}+\chi_{c,t}-\chi_{c',t'}\in \Xi^*(X)$ and $\xi(X_{k}\cup \{x\})\leq \xi_{k}+\chi_{c,t}-\chi_{c',t'}$, where $\chi_{c,t}=\xi(\{x\})$.
  \end{itemize}
  %If (i) $\xi(X_{k}\cup\{x_{k+1}\})\leq \xi_{k}$, or (ii) there exists $(c',t')\in \mathcal{C}\times \mathcal{T}$ such that $\xi_{k}+\chi_{c,t}-\chi_{c',t'}\in \Xi^*(X)$ and $\xi(X_{k}\cup \{x_{k+1}\})\leq \xi_{k}+\chi_{c,t}-\chi_{c',t'}$,
  If there exists such a contract, then choose the one $x_{k+1}$ with the highest merit and let
\begin{align*}
X_{k+1}&=X_{k}\cup \{x_{k+1}\}, \;  \\
\xi_{k+1}&=\begin{cases} \xi_{k} \text{ (if (i) holds), } \\
                              \xi_{k}+\chi_{c,t}-\chi_{c',t'} \text{ (if (ii) holds), }
               \end{cases}
\end{align*}
and go to Step 3. Otherwise, go to Step 4.
  \item[Step 3] Add 1 to $k$ and go to Step 2.
  \item[Step 4] Return $X_{k}$ and stop.
\end{description}
In words, $X_k$ and $R_k$ collect the set of accepted and rejected contracts, respectively.
The process of modifying $\xi_k$ is motivated by the following lemma.

\begin{lemma} \label{lem:modified}
Let $X'\subseteq X$, $x\in X\backslash X'$, and $(c,t)\in \mathcal{C}\times \mathcal{T}$ be such that $\xi(\{x\})=\chi_{c,t}$. Suppose that
there exists a maximal distribution $\xi$ in $\Xi^*(X)$ with $\xi(X')\leq \xi$.
Then, the following implication holds: if there exists a maximal distribution $\xi^*$ in $\Xi^*(X)$ such that $\xi(X'\cup \{x\})\leq \xi^*$, then either (i) $\xi(X'\cup \{x\})\leq \xi$, or (ii) there exists $(c',t')\in \mathcal{C}\times \mathcal{T}$ such that $\xi+\chi_{c,t}-\chi_{c',t'}\in \Xi^*(X)$ and $\xi(X'\cup \{x\})\leq \xi+\chi_{c,t}-\chi_{c',t'} $.
%\begin{align*}
%&\text{ there exists a maximal distribution } \xi^* \text{ in } \Xi^*(X) \text{ such that } \xi(X'\cup \{x\})\leq \xi^* \\
%\Longrightarrow \:&\text{(i) } \xi(X'\cup \{x\})\leq \xi, \text{ or } \text{(ii) there exists } (c',t')\in \mathcal{C}\times \mathcal{T} \text{ such that } \\
%&\xi+\chi_{c,t}-\chi_{c',t'}\in \Xi^*(X)  \text{ and }\xi(X'\cup \{x\})\leq \xi+\chi_{c,t}-\chi_{c',t'}.
%\end{align*}
\end{lemma}
%\ky{To Bumin: I modified the style of the implication. Could you please check if this is OK?}
\begin{proof}[Proof of Lemma \ref{lem:modified}]\renewcommand{\qedsymbol}{$\blacksquare$}
We consider two cases.

\smallskip
\noindent
\emph{Case 1:}
Suppose that $\xi^{t}_{c}(X')<\xi^{t}_{c}$. Then,
\begin{align*}
&\xi^{t}_{c}(X'\cup \{x\})=\xi^{t}_{c}(X')+1\leq \xi^{t}_{c}, \text{ and } \\
&\xi^{\tilde t}_{\tilde c}(X'\cup \{x\})=\xi^{\tilde t}_{\tilde c}(X')\leq \xi^{\tilde t}_{\tilde c} \text{ for all } (\tilde c, \tilde t)\in \mathcal{C}\times \mathcal{T} \text{ with } (\tilde c, \tilde t)\neq (c,t).
\end{align*}
Thus, (i) holds.

\smallskip
\noindent
\emph{Case 2:}
Suppose that $\xi^{t}_{c}(X')=\xi^{t}_{c}$. By the sufficient condition of the implication, there exists a maximal maximizer $\xi^*$ in $\Xi^{*}(X)$ with $\xi(X'\cup \{x\})\leq \xi^*$. Then,
\begin{align*}
\xi^{t}_{c}+1=\xi^{t}_{c}(X')+1=\xi^{t}_{c}(X'\cup \{x\})\leq (\xi^*)^{t}_{c},
\end{align*}
which implies $\xi^{t}_{c}<(\xi^*)^{t}_{c}$.
By Lemma \ref{lem:pareto} (M-convexity of the set of maximal distributions in $\Xi^*(X)$), there exists $(c',t') \in (\mathcal{C}\times \mathcal{T})$ with $\xi^{t'}_{c'}>(\xi^*)^{t'}_{c'}$ such that $\xi+\chi_{c,t}-\chi_{c',t'}$ is a maximal distribution in $\Xi^*(X)$.
It holds that
\begin{align*}
&(\xi+\chi_{c,t}-\chi_{c',t'})^{t'}_{c'}\geq (\xi^*)^{t'}_{c'} \geq \xi^{t'}_{c'}(X'\cup \{x\}), \\
&(\xi+\chi_{c,t}-\chi_{c',t'})^{t}_{c}=\xi^{t}_{c}+1=\xi^{t}_{c}(X')+1=\xi^{t}_{c}(X'\cup \{x\}), \\
&(\xi+\chi_{c,t}-\chi_{c',t'})^{\tilde t}_{\tilde c}=\xi^{\tilde t}_{\tilde c} \geq \xi^{\tilde t}_{\tilde c}(X')=\xi^{\tilde t}_{\tilde c}(X'\cup \{x\}) \\
&\hspace{30mm}\text{ for all } (\tilde c, \tilde t)\in \mathcal{C}\times \mathcal{T} \text{ with } (\tilde c, \tilde t) \neq (c,t) \text{ and } (\tilde c, \tilde t)\neq (c',t').
\end{align*}
Thus, (ii) holds.
\end{proof}

\begin{lemma} \label{lem:choice-equivalence}
The modified diversity choice rule and the (original) diversity choice rule produce the same outcome.
\end{lemma}
\begin{proof}[Proof of Lemma \ref{lem:choice-equivalence}]\renewcommand{\qedsymbol}{$\blacksquare$}
%We show by induction that the two choice rules produce the same set of contracts for each index $k$ used in the definitions of both rules and terminate at the same index.
Let $X_k$ be defined as in the construction of the diversity choice rule and let $X'_k$ and $\xi_k$ be defined as in the construction of the modified diversity choice rule.
We show by induction that $X_k=X'_k$ and $\xi_k$ is a maximal distribution in $\Xi^*(X)$ for each index $k$ used in the definitions of both rules and terminate at the same index.
For $k=0$, we have $X_k=\emptyset=X'_k$ and, by the definition of the modified diversity choice rule, $\xi_0$ is a maximal distribution in $\Xi^*(X)$.
By mathematical induction hypothesis, suppose that $X_k=X'_k$ and $\xi_k$ is a maximal distribution. We now show the hypothesis for $k+1$.

\smallskip
\emph{Case 1:} Suppose that the diversity choice rule does not terminate when the index is $k$.
By the induction hypothesis and the definition of the modified diversity choice rule,
$\xi(X_k)=\xi(X'_k)\leq \xi_k$. By the induction hypothesis, $\xi_k$ is a maximal distribution in $\Xi^*(X)$. Let $x_{k+1}$ be such that $X_{k+1}=X_k\cup\{x_{k+1}\}$. By the definition of the diversity choice rule, $\xi(X_k\cup\{x_{k+1}\})\leq \xi^*$ for some $\xi^*\in \Xi^*(X)$; let us choose $\xi^*$ so that it is maximal. By Lemma \ref{lem:modified}, either (i) $\xi(X_k \cup \{x_{k+1}\})\leq \xi_k$, or (ii) there exists $(c',t')\in \mathcal{C}\times \mathcal{T}$ such that $\xi_k+\chi_{c,t}-\chi_{c',t'}\in \Xi^*(X)$ and $\xi(X_k\cup \{x_{k+1}\})\leq \xi_k+\chi_{c,t}-\chi_{c',t'}$, where $\chi_{c,t}=\xi(\{x_{k+1}\})$. It follows that $x_{k+1}$ satisfies one of the two conditions stated in Step 2 of the modified diversity choice rule. Suppose, for contradiction, that $X'_{k+1}\neq X'_k\cup \{x_{k+1}\}$. Then, by the deifnition of the modified diversity choice rule, there exists $x'\in X\backslash X'_k$ such that $x'$ has a higher merit than $x_{k+1}$ and $\xi(X'_k\cup\{x'\})\leq \xi^{**}$ for some $\xi^{**}\in \Xi^*(X)$. By the induction hypothesis, we have $X'_k=X_k$, which implies $x'\in X\backslash X_k$ and $\xi(X_k\cup\{x'\})\leq \xi^{**}$. Since $x'$ has a higher merit than $x_{k+1}$, we obtain a contradiction to the fact that $x_{k+1}$ is chosen when the index of the diversity choice rule is $k+1$. Therefore, $X'_{k+1}=X'_k\cup \{x_{k+1}\}=X_k\cup\{x_{k+1}\}=X_{k+1}$, where the second equality follows from the induction hypothesis. It remains to show that $\xi_{k+1}$ is a maximal distribution in $\Xi^*(X)$. By Lemma \ref{lem:pareto} (M-convexity of the maximal distributions in $\Xi^*(X)$) and Proposition 4.1 of \cite{Murota:SIAM:2003}, every maximal distribution in $\Xi^*(X)$ has the same sum of coordinates. Since $\xi_k$ is a maximal distribution (which follows from the induction hypothesis) and $\xi_k$ and $\xi_{k+1}$ have the same sum of coordinates (which follows from the definition of the modified diversity choice rule), $\xi_{k+1}$ is a maximal distribution.

\smallskip
\emph{Case 2:} Suppose that the diversity choice rule terminates when the index is $k$. Then, there does not exist $x\in X\backslash X_k$ and $\xi\in \Xi^*(X)$ such that $\xi(X_k\cup\{x\})\leq \xi$. Then, for each $x\in X\backslash X_k=X\backslash X'_k$ (where the equality follows from the induction hypothesis), neither (i) nor (ii) in Step 2 of the modified diversity choice rule holds true. Theorefore, the modified diversity choice rule termines when the index is $k$.




%We focus on two differences between the two rules and show that neither of them influences the outcome.
%
%The first difference lies in deciding whether to accept $x_{k+1}$ or not (i.e., whether to expand $X_k$ to $X_k\cup \{x_{k+1}\}$ or not).
%The modified rule checks only local distributions around $\xi_{k}$ (i.e., $\xi_{k}$ itself or those distributions written as $\xi_k+\chi_{c,t}-\chi_{c',t'}$), whereas the original rule checks all distributions in $\Xi^*(X)$. To show that this difference has no effect on the final outcome, we apply Lemma \ref{lem:modified}. For each round $k$ of the modified rule, we can apply the lemma for the case $X'=X_k$ and $x=x_{k+1}$ because the supposition of the lemma is satisfied for $\xi=\xi_k$ due to the following reasons:
%\begin{itemize}
%\item Every maximal distribution in $\Xi^*(X)$ has the same sum of coordinates by Lemma \ref{lem:pareto} (M-convexity of the maximal distributions in $\Xi^*(X)$) and Proposition 4.1 of \cite{Murota:SIAM:2003}.
%\item Since $\xi_0$ is a maximal distribution in $\Xi^*(X)$ and the sum of coordinates of $\xi_k$ is identical across different rounds, together with the above bullet point, $\xi_k$ is a maximal distribution in $\Xi^*(X)$.
%\item Furthermore, by the definition of Step 3 of the modified rule, $\xi(X_k)\leq \xi_k$.
%\end{itemize}
%
%According to the contrapositive of the implication in Lemma \ref{lem:modified}, if there are no local distributions around $\xi_k$ that are weakly greater than $\xi(X_k\cup\{x_{k+1}\})$, then there are no distributions in $\Xi^*(X)$ that are weakly greater than $\xi(X_k\cup\{x_{k+1}\})$.
%Thus, checking only local distributions around $\xi_k$ (as in Step 3 of the modified rule) tantamounts to checking all distributions in $\Xi^*(X)$ (as in Step 2 of the original rule).
%
%The second difference lies in the set of candidates for $x_{k+1}$.
%In the modified rule, if a contract is rejected (i.e., included in $R_k$), then the contract is never considered as a candidate for $x_{k+1}$ in later rounds. On the other hand, in the original rule, any contract not tentatively accepted (i.e., not included in $X_k$) is considered as a candidate in every round. This difference has no effect on the outcome for the following reason. By Lemma \ref{lem:matroid}, the original diversity choice rule with input $X$ is a greedy rule over a matroid $(X, \mathcal{F}(X))$. Applying Proposition \ref{prop:matroid} for the case $\mathcal{X}=X$ and $\mathcal{F}=\mathcal{F}(X)$, the choice rule satisfies path independence, which implies that ignoring rejected contracts has no effect on the outcome.
\end{proof}

\paragraph{Step 4: }
We derive the time complexity of the diversity choice rule. The first step for calculating the choice rule is to find one maximal distribution in $\Xi^*(X)$. By Lemma \ref{lem:outcome-maximal}, we can use the domain-reduction algorithm. We assume that $f$ can be evaluated in a constant time in what follows.
Step 2 of the algorithm takes $O(|\mathcal{C}\times \mathcal{T}|)$ time.
%which is bounded by $O(|\mathcal{X}|)$ time because $|\mathcal{C}\times \mathcal{T}|\leq |\mathcal{X}|$.
Let $\xi_{k^*}$ denote the outcome of the algorithm. Since the algorithm starts from $\mathbf{0}$ and adds $1$ to some coordinate toward $\xi_{k^*}$ at every round, the number of iterations is $||\xi_{k^*}||$, which is bounded by $||\xi(X)||$ because $\xi_{k^*}\leq \xi(X)$. Since
\begin{align*}
||\xi(X)||=\sum_{(c,t)}\xi_c^t(X)=\sum_{(c,t)}\sum_{x\in X}\xi_c^t(\{x\})=\sum_{x\in X}\sum_{(c,t)}\xi_c^t(\{x\})=|X|,
\end{align*}
the number of iterations is bounded by $O(|X|)$. Thus, finding an outcome of the algorithm takes $O(|\mathcal{C}|\times |\mathcal{T}|\times |X|)$ time.

Given a maximal distribution in $\Xi^*(X)$, we can run the diversity choice rule. By Lemma \ref{lem:choice-equivalence}, it suffices to examine the computational time of the modified rule. Step 2 of the rule takes $O(|C|\times |T|\times |X|)$ time.
%and Step 3 takes $O(|\mathcal{C}| \times |\mathcal{T}|)$ time. Therefore, Steps 2 and 3 take $O(|X|+|\mathcal{C}|\times |\mathcal{T}|)$ time in total.
The number of iterations is equal to $|X|$. Hence, the modified diversity choice rule finds an outcome in $O(|\mathcal{C}|\times |\mathcal{T}|\times |X|^2)$ time.
%, which is the same as the time complexity of finding a maximal distribution in $\Xi^*(X)$.
Together with the time complexity of executing the domain-reduction algorith, we conclude that finding an outcome of the diversity choice rule takes $O(|\mathcal{C}|\times |\mathcal{T}|\times |X|^2)$ time.
\qed

%==================================================
\bigskip




\subsection*{Proof of Theorem \ref{thm:pi}}

%\begin{proof}[Proof of Theorem \ref{thm:pi}]
We need the following properties of choice rules in our proofs.

\begin{definition}%\citep{kelso82}
A choice rule $C$ satisfies the \textbf{irrelevance of rejected contracts} condition, if, for each $X \subseteq \calx$
and $x \in \calx \setminus X$,
\[x\notin C(X\cup \{x\}) \; \implies \;  C(X\cup \{x\})=C(X).\]
\end{definition}

\begin{definition}%\citep{kelso82}
A choice rule $C$ satisfies the \textbf{substitutes} condition, if, for each $X \subseteq \calx$
and $x \in \calx \setminus X$,
\[C(X) \supseteq  C(X\cup \{x\}) \cap X.\]
\end{definition}

\begin{lemma}[\cite{aizmal81}]\label{lem:pi}
A choice rule $C$ is path independent if, and only if, it satisfies the irrelevance of rejected contracts condition
and the substitutes condition.
\end{lemma}

By this lemma path independence is equivalent to the conjunction of the
irrelevance of rejected contracts condition (IRC) and the substitutes condition,
so we show these two properties to prove path independence.

\medskip
\noindent
\emph{Proof of IRC:}
%We first show that the irrelevance of rejected contracts condition holds.
Let $X \subseteq \calx$ and $x \in \calx \setminus X$ such that $x\notin C^d(X\cup \{x\})$. We need to show $C^d(X\cup \{x\})=C^d(X)$.

Let $c=\gamma(x)$, $t=\tau(\sigma(x))$, $\xi_1= \xi(C^d(X))$, and $\xi_2=\xi(C^d(X\cup\{x\}))$.

Since  $x \notin C^d(X\cup \{x\})$, $\xi_2 \leq \xi(X)$. Together with Theorem \ref{thm:diversitychoice} (i), we get $f(\xi_1)=f(\xi_2)$.
Furthermore, $C^d(X\cup\{x\})$ is in $\mathcal{F}(X)$ and
$\mathcal{F}(X\cup \{x\})$. Likewise, $C^d(X)$ is in $\mathcal{F}(X\cup\{x\})$ because
$(\mathcal{X},\mathcal{F}(\mathcal{X}))$ is a matroid (Lemma \ref{lem:matroid}).
Therefore, $C^d(X),C^d(X\cup\{x\}) \in \mathcal{F}(X) \cap \mathcal{F}(X\cup \{x\})$.
By Theorem \ref{thm:diversitychoice} (ii), $C^d(X)$ merit
dominates $C^d(X\cup\{x\})$ and $C^d(X\cup\{x\})$ merit dominates $C^d(X)$.
Therefore, $C^d(X)=C^d(X\cup\{x\})$, which follows from the \emph{antisymmetry} of merit domination that if two sets merit dominate each other they have
to be the same. The antisymmetry of merit domination is straightforward
because if two sets merit dominate each other,
then they have the same number of contracts and, furthermore, because different
contracts have distinct merit rankings, they need to have the
same set of contracts.
\medskip

To finish the proof, we show that $C^d$ satisfies the substitutes condition.

\medskip
\noindent
\emph{Proof of Substitutability:}
%Next we show the substitutes condition.
Let $X \subseteq \calx$ and $x \in \calx \setminus X$. We need
to show $C^d(X)\supseteq C^d(X \cup \{x\}) \cap X$.

Let $c=\gamma(x)$, $t=\tau(\sigma(x))$, $\xi_1= \xi(C^d(X))$, and $\xi_2=\xi(C^d(X\cup\{x\}))$.

If $x\notin C^d(X \cup \{x\})$, then by the irrelevance of rejected contracts condition
we have $C^d(X)=C^d(X\cup\{x\})$. Therefore, $C^d(X)\supseteq C^d(X \cup \{x\}) \cap X= C^d(X)$.

For the rest of the proof suppose that $x\in C^d(X \cup \{x\})$.
We consider several cases depending on the value of $\xi_2$.



\smallskip
\emph{Case 1:} Consider the case $\xi_2 \leq \xi(X)$. Then $f(\xi_1)=f(\xi_2)$.
By construction of $C^d$, $\xi_1$ is maximal in $\Xi^*(X)$.
Likewise, $\xi_2$ is maximal in
$\Xi^*(X\cup\{x\})$. Since $\xi_2 \leq \xi(X)$, we get that $\xi_2$ is also
maximal in $\Xi^*(X)$. By Lemma \ref{lem:pareto}, $\xi_1$ and $\xi_2$ belong
to an M-convex set, so $\norm{\xi_2} = \norm{\xi_1}$.\footnote{Members of an M-convex set
have the same sum of coordinates, see Proposition 4.1 in \citet{Murota:SIAM:2003}.} Therefore,
\[\abs{C^d(X\cup\{x\})\setminus C^d(X)}=\abs{C^d(X)\setminus C^d(X\cup\{x\})}.\]
Since $x\in C^d(X\cup\{x\})\setminus C^d(X)$, we have $|C^d(X\cup\{x\})\setminus C^d(X)|\geq 1$. We show that
$|C^d(X\cup\{x\})\setminus C^d(X)|=1$.

Suppose, for contradiction, that $\abs{C^d(X\cup\{x\})\setminus C^d(X)}\geq 2$. Then,
there exists $x_1 \in X\setminus \{x\}$ such that $x_1 \in C^d(X\cup\{x\})\setminus C^d(X)$. Since
$f(\xi_1)=f(\xi_2)$, $C^d(X\cup\{x\})$ and $C^d(X)$ are bases in $\mathcal{F}(X\cup \{x\})$.
By the stronger version of \emph{B2}, which is stated on page \pageref{strongerB2}, there exists $x_2 \in C^d(X)\setminus C^d(X\cup\{x\})$ such
that $(C^d(X\cup\{x\}) \setminus \{x_1\}) \cup \{x_2\}$ and $(C^d(X) \setminus \{x_2\}) \cup \{x_1\}$
are also bases in $\mathcal{F}(X\cup \{x\})$.
Theorem \ref{thm:diversitychoice} implies that $C^d(X\cup\{x\})$ merit dominates $(C^d(X\cup\{x\})\setminus \{x_1\}) \cup \{x_2\}$, so
$x_1 \mathrel{\succ} x_2$. Furthermore, since
$(C^d(X) \setminus \{x_2\}) \cup \{x_1\}$ is a base in
$\mathcal{F}(X\cup \{x\})$ it must also be a base in $\mathcal{F}(X)$. By Theorem \ref{thm:diversitychoice},
$C^d(X)$ merit dominates $(C^d(X) \setminus \{x_2\}) \cup \{x_1\}$, therefore,
$x_2 \mathrel{\succ} x_1$, which is a
contradiction to $x_1 \mathrel{\succ} x_2$. Therefore, $|C^d(X\cup\{x\})\setminus C^d(X)|=1$ and
$C^d(X \cup\{x\})= (C^d(X) \cup \{x\})\setminus \{y\}$ for some $y\in C^d(X)$. As a result,
$C^d(X)\supseteq C^d(X \cup \{x\}) \cap X = C^d(X) \setminus \{y\}$ for some $y\in C^d(X)$.
This finishes the proof of Case 1.

\smallskip
\emph{Case 2:} Consider the case $\xi_2 \not \leq \xi(X)$. Since $C^d(X\cup \{x\})\subseteq X\cup \{x\}$,
it must be that $(\xi_2)^t_c> \xi^t_c(X)$, so $C^d(X\cup \{x\})$ includes $x$ and all
contracts in $X$ with type-$t$ students and school $c$. Furthermore, $(\xi_2)^t_c=\xi^t_c(X)+1$ and
$(\xi_1)^t_c\leq \xi^t_c(X)$.

\begin{claim}\label{claim:types}
For each school $c'\in \mathcal{C}$ and type $t'\in \mathcal{T}$ such that $(c',t') \neq (c,t)$, we have $(\xi_1)^{t'}_{c'}\geq (\xi_2)^{t'}_{c'}$.
\end{claim}

\begin{proof}[Proof of Claim \ref{claim:types}]\renewcommand{\qedsymbol}{$\blacksquare$}
Suppose, for contradiction, that there exist school $c'\in \calc$ and type $t'\in \calt$ with $(c',t') \neq (c,t)$
such that $(\xi_1)^{t'}_{c'} < (\xi_2)^{t'}_{c'}$. Then, by ordinal concavity,
either (i)
\begin{enumerate}
\item $f(\xi_2-\chi_{c',t'})>f(\xi_2)$ or
\item $f(\xi_1+\chi_{c',t'})>f(\xi_1)$ or
\item $f(\xi_2-\chi_{c',t'})=f(\xi_2)$ and $f(\xi_1+\chi_{c',t'})=f(\xi_1)$
\end{enumerate}
or (ii) there exist school $\hat c\in \calc$ and type $\hat t\in \calt$  with
$(\xi_2)_{\hat c}^{\hat t}<(\xi_1)_{\hat c}^{\hat t}$ such that
\begin{enumerate}
\item $f(\xi_2-\chi_{c',t'}+\chi_{\hat c, \hat t})>f(\xi_2)$ or
\item $f(\xi_1+\chi_{c',t'}-\chi_{\hat c, \hat t})>f(\xi_1)$ or
\item $f(\xi_2-\chi_{c',t'}+\chi_{\hat c, \hat t})=f(\xi_2)$ and $f(\xi_1+\chi_{c',t'}-\chi_{\hat c, \hat t}) = f(\xi_1)$.
\end{enumerate}
Condition (i) cannot hold because under (i)(1) $\xi_2-\chi_{c',t'}\leq \Xi(X\cup \{x\})$ and
$f(\xi_2-\chi_{c',t'})>f(\xi_2)$ give us a contradiction to the result that the outcome of $C^d$
maximizes the diversity index among feasible subsets of $X\cup \{x\}$ (Theorem \ref{thm:diversitychoice}),
because a contract in $(X\cup \{x\}) \setminus C^d(X\cup \{x\})$ with type-$t'$ student and
school $c'$ can be added to $C^d(X\cup \{x\})$ and increase the value of $f$.
Under (i)(2) $\xi_1+\chi_{c',t'} \leq \xi(X)$ and $f(\xi_1+\chi_{c',t'}) > f(\xi_1)$ give us a contradiction to the result that the outcome of $C^d$ maximizes the diversity index among subsets of $X$ (Theorem \ref{thm:diversitychoice}), because a
contract in $X\setminus C^d(X)$ with type-$t'$ student and
school $c'$ can be added to $C^d(X)$ and increase the value of $f$. Under (i)(3),
$\xi_1+\chi_{c',t'} \leq \xi(X)$ and $f(\xi_1+\chi_{c',t'})=f(\xi_1)$ give us a contradiction to the result that the outcome of $C^d$ merit dominates any
feasible subset of $X$ that maximizes diversity
(Theorem \ref{thm:diversitychoice}), because a contract in $X\setminus C^d(X)$ with type-$t'$ student and
school $c'$ can be added to $C^d(X)$ without changing the value of $f$.

Likewise condition (ii) cannot hold because under (ii)(1) $\xi_2-\chi_{c',t'}+\chi_{\hat c, \hat t} \leq \xi(X\cup \{x\})$ and $f(\xi_2-\chi_{c',t'}+\chi_{\hat c, \hat t}) > f(\xi_2)$ give us a contradiction to the result that the outcome of $C^d$ maximizes the diversity index among feasible subsets of $X\cup \{x\}$
(Theorem \ref{thm:diversitychoice}), because a contract in $(X\cup\{x\}) \setminus C^d(X\cup\{x\})$ with
type-$\hat t$ student and school $\hat c$ can be added to $C^d(X\cup\{x\})$ and a contract from
$C^d(X\cup\{x\})$ with type-$t'$ student and school $c'$ can be removed from $C^d(X\cup\{x\})$ to increase the value of $f$.
Under (ii)(2) $\xi_1+\chi_{c',t'}-\chi_{\hat c, \hat t} \leq \xi(X)$ and $f(\xi_1+\chi_{c',t'}-\chi_{\hat c, \hat t}) > f(\xi_1)$ give us a contradiction to the result that the outcome of $C^d$ maximizes the diversity index among feasible subsets of $X$
(Theorem \ref{thm:diversitychoice}), because a contract in $X\setminus C^d(X)$ with type-$t'$ student and
school $c'$ can be added to $C^d(X)$ and a contract from $C^d(X)$ with type-$\hat t$ student and
school $\hat c$ can be removed from $C^d(X)$ to increase the value of $f$.
Under (ii)(3), $f(\xi_2-\chi_{c',t'}+\chi_{\hat c, \hat t}) = f(\xi_2)$ and $\xi_2-\chi_{c',t'}+\chi_{\hat c, \hat t} \leq \xi(X\cup\{x\})$ imply that the lowest merit ranked type-$t'$ student with a contract at school $c'$ in $C^d(X\cup \{x\})\setminus C^d(X)$ has a higher merit ranking than the lowest merit ranked type-$\hat t$ student with a contract at school $\hat c$ in
$C^d(X) \setminus C^d(X\cup \{x\})$. Similarly, $\xi_1+\chi_{c',t'}-\chi_{\hat c, \hat t}\leq \xi(X)$ and $f(\xi_1+\chi_{c',t'}-\chi_{\hat c, \hat t}) = f(\xi_1)$
imply that the lowest merit ranked type-$\hat t$ student with a contract
at school $\hat{c}$ in $C^d(X)\setminus C^d(X\cup \{x\})$ has a higher
merit ranking than the lowest merit type-$t'$ student with a contract at school $c'$ in $C^d(X\cup \{x\}) \setminus C^d(X)$, which is a contradiction since the merit
ranking is strict and $(\hat c, \hat t)\neq (c', t')$.
\end{proof}



\begin{claim}\label{claim:comparetc}
$(\xi_1)^t_c=(\xi_2)^t_c-1$.
\end{claim}


\begin{proof}[Proof of Claim \ref{claim:comparetc}]\renewcommand{\qedsymbol}{$\blacksquare$}
Suppose, for contradiction, that $(\xi_1)^t_c\neq (\xi_2)^t_c-1$. Since $(\xi_1)^t_c\leq \xi^t_c(X)$ and
$(\xi_2)^t_c = \xi^t_c(X)+1$, we get $(\xi_1)^t_c < \xi^t_c(X)=(\xi_2)^t_c-1$.

By ordinal concavity,
either (i)
\begin{enumerate}
\item $f(\xi_2-\chi_{c,t})>f(\xi_2)$, or
\item $f(\xi_1+\chi_{c,t})>f(\xi_1)$, or
\item $f(\xi_2-\chi_{c,t})=f(\xi_2)$ and $f(\xi_1+\chi_{c,t}) = f(\xi_1)$
\end{enumerate}
or (ii) there exist school $c'\in \calc$ and type $t'\in \calt$ with $(\xi_2)_{c'}^{t'}<(\xi_1)_{c'}^{t'}$ such that
\begin{enumerate}
\item $f(\xi_2-\chi_{c,t}+\chi_{c',t'})>f(\xi_2)$
\item $f(\xi_1+\chi_{c,t}-\chi_{c',t'})>f(\xi_1)$ or
\item $f(\xi_2-\chi_{c,t}+\chi_{c',t'})=f(\xi_2)$ and $f(\xi_1+\chi_{c,t}-\chi_{c',t'}) = f(\xi_1)$.
\end{enumerate}
Condition (i) cannot hold because under (i)(1) $\xi_2-\chi_{c,t} \leq \xi(X\cup \{x\})$
and $f(\xi_2-\chi_{c,t})>f(\xi_2)$ give us a contradiction to the result that the outcome of $C^d$
maximizes the diversity index among feasible subsets of $X\cup \{x\}$ (Theorem \ref{thm:diversitychoice}),
because a contract in $(X\cup \{x\}) \setminus C^d(X\cup \{x\})$ with type-$t$ student and
school $c$ can be added to $C^d(X\cup \{x\})$ and increase the value of $f$.
Similarly, under (i)(2) $\xi_1+\chi_{c,t} \leq \xi(X)$ and $f(\xi_1+\chi_{c,t})>f(\xi_1)$
give us a contradiction to the result that the outcome of $C^d$ maximizes the diversity index among feasible subsets of $X$
(Theorem \ref{thm:diversitychoice}), because a contract in $X\setminus C^d(X)$ with type-$t$ student and
school $c$ can be added to $C^d(X)$ to increase the value of $f$. Under (i)(3) $\xi_1+\chi_{c,t} \leq \xi(X)$ and
$f(\xi_1+\chi_{c,t}) = f(\xi_1)$ give us a contradiction to the result that the outcome of $C^d$ merit dominates each feasible
subset of $X$ that maximizes diversity (Theorem \ref{thm:diversitychoice}), because a contract in
$X\setminus C^d(X)$ with type-$t$ student and school $c$ can be added to $C^d(X)$ without changing the value of $f$.
Therefore, condition (ii) must hold.

Under condition (ii)(1) $\xi_2-\chi_{c,t}+\chi_{c',t'} \leq \xi(X\cup \{x\})$
and $f(\xi_2-\chi_{c,t}+\chi_{c',t'})>f(\xi_2)$ give a contradiction to the result that the
outcome of $C^d$ maximizes the diversity index among feasible subsets of $X\cup \{x\}$
(Theorem \ref{thm:diversitychoice}), because a contract in $(X\cup\{x\}) \setminus C^d(X\cup\{x\})$ with
type-$t$ student and school $c$ can be added to $C^d(X\cup\{x\})$ and a contract from
$C^d(X\cup\{x\})$ with type-$t'$ student and school $c'$ can be removed from $C^d(X\cup\{x\})$ to increase the value of $f$.
Likewise, under (ii)(2) $\xi_1+\chi_{c,t}-\chi_{c',t'}\leq \xi(X)$ and $f(\xi_1+\chi_{c,t}-\chi_{c',t'})>f(\xi_1)$
give us a contradiction to the result that the outcome of $C^d$ maximizes the diversity index among feasible subsets of $X$
(Theorem \ref{thm:diversitychoice}), because a contract in $X\setminus C^d(X)$ with type-$t$ student and
school $c$ can be added to $C^d(X)$ and a contract in $C^d(X)$ with type-$t'$ student and
school $c'$ can be removed to increase the value of $f$. Under (ii)(3)
$f(\xi_2-\chi_{c,t}+\chi_{c', t'}) = f(\xi_2)$ and $\xi_2-\chi_{c,t}+\chi_{c', t'} \leq \xi(X\cup\{x\})$ imply that the lowest merit ranked type-$t$ student with a contract
at school $c$ in $C^d(X\cup \{x\})\setminus C^d(X)$ has a higher merit ranking than the lowest merit ranked type-$t'$ student with a contract at school $c'$ in
$C^d(X) \setminus C^d(X\cup \{x\})$. Similarly, $f(\xi_1+\chi_{c,t}-\chi_{c',t'}) = f(\xi_1)$
and $\xi_1+\chi_{c,t}-\chi_{c',t'}\leq \xi(X)$ imply that the lowest merit ranked type-$t'$ student with a contract at school $c'$ in $C^d(X)\setminus C^d(X\cup \{x\})$ has a higher merit ranking than
the lowest merit ranked type-$t'$ student with a contract at school $c'$ in $C^d(X\cup \{x\}) \setminus C^d(X)$, which is a contradiction since the merit ranking is strict and $(c,t)\neq (c', t')$.

Both conditions cannot hold. Therefore, $(\xi_1)^t_c=(\xi_2)^t_c-1$.
\end{proof}

To finish the proof of Case 2, we combine the results that we have established so far:
$(\xi_2)^c_t=\xi^t_c(X\cup \{x\})=\xi^t_c(X)+1$,
$(\xi_1)^c_t=\xi^t_c(X)$, and, for each type $t'\in \calt$ and school $c'\in \calc$
with $(t',c')\neq (t,c)$, $(\xi_1)^{c'}_{t'} \geq (\xi_2)^{c'}_{t'}$.
For a fixed type $t'\in \calt$ and school $c'\in \calc$ and the number of contracts of type-$t'$ students with school $c'$, choice rule $C^d$ chooses contracts with the highest merit ranking.
Therefore, $C^d(X) \supseteq C^d(X \cup \{x\}) \cap X$, which finishes the proof of Case 2. Therefore, $C^d$ satisfies the substitutes condition.
\qed

%==================================================
\bigskip





\subsection*{Proof of Theorem \ref{thm:trace}}
The following result follows from Theorem \ref{thm:diversitychoice}.

\begin{lemma}\label{lem:lambdachoice}
Suppose that $\lambda\in \mathbb{R}_+$ is such that the diversity
index $f_{\lambda}$ is \oconcave{}. Then, for each set of contracts $X\subseteq \calx$,
\begin{enumerate}[(i)]
\item $\min\{f(\xi(C^d_{\lambda}(X))),\lambda\}=\min\{f(\xi(C^d(X))),\lambda\}$ and
\item $C^d_{\lambda}(X)$ merit dominates each $Y\subseteq X$ such that
\[ \min\{f(\xi(C^d_{\lambda}(Y))),\lambda\}=\min\{f(\xi(C^d(X))),\lambda\}.\]
\end{enumerate}
\end{lemma}

Now fix a set of contracts $X\subseteq \mathcal{X}$ and denote the outcome of
the trace algorithm as $C^{tr}(X)$. Using Lemma \ref{lem:lambdachoice}, first,
we show that $C^{tr}(X)\subseteq \mathcal{P}(X)$, and, then,
$\mathcal{P}(X) \subseteq C^{tr}(X)$ to finish the proof.

\begin{claim}\label{claim:pareto}
$C^{tr}(X)\subseteq \mathcal{P}(X)$.
\end{claim}

\begin{proof}[Proof of Claim \ref{claim:pareto}]\renewcommand{\qedsymbol}{$\blacksquare$}
Let $Y\in C^{tr}(X)$. Suppose, for contradiction, that $Y\notin \mathcal{P}(X)$, and, hence, there exists
$Z\subseteq X$ such that $Z\neq Y$, $Z$ merit dominates $Y$, and
$f(\xi(Z))\geq f(\xi(Y))$.

Suppose that $Z$ is chosen at index $k\in \mathbb{N}$ in the
construction of $C^{tr}(X)$. Therefore,
$Y=C^d_{\lambda_k}(X)$. Then, by Lemma \ref{lem:lambdachoice},
$Y=C^d_{\lambda_k}(X)$ merit dominates each subset of $X$ that attains diversity
level of $f(\xi(C^d_{\lambda_k}(X)))=f(\xi(Y))$. Therefore, since
$f(\xi(Z)) \geq f(\xi(Y))$ and $Z\subseteq X$, we get $Y$ merit dominates $Z$.
As noted in the proof of Theorem \ref{thm:pi}, the merit domination is
antisymmetric, which is a contradiction because we have
$Y$ merit dominates $Z$, $Z$ merit dominates $Y$, and $Y\neq Z$.
Therefore, $Y\in \mathcal{P}(X)$. Since $Y$ is any set in $C^{tr}(X)$,
we conclude $C^{tr}(X) \subseteq \mathcal{P}(X)$.
\end{proof}

\begin{claim}\label{claim:trace}
$\mathcal{P}(X) \subseteq C^{tr}(X)$.
\end{claim}

\begin{proof}[Proof of Claim \ref{claim:trace}]\renewcommand{\qedsymbol}{$\blacksquare$}
Let $Y\in \mathcal{P}(X)$. Suppose, for contradiction, that $Y\notin C^{tr}(X)$.
Since $C^d(X)\in C^{tr}(X)$ and $C^{tr}(X)\subseteq \mathcal{P}(X)$, we get that
$C^d(X)\in \mathcal{P}(X)$. Since $Y\notin C^{tr}(X)$, we have $Y\neq C^d(X)$.
By Theorem \ref{thm:diversitychoice}, $f(\xi(C^d(X))\geq f(\xi(Y))$
and $C^d(X)$ merit dominates any subset of $X$ with diversity $f(\xi(C^d(X))$.
Therefore, since $Y \in \mathcal{P}(X)$, we cannot have $f(\xi(C^d(X)) = f(\xi(Y))$,
which implies $f(\xi(C^d(X))) > f(\xi(Y))$.

Since $\lambda_0=0$ and $f(\xi(C^d(X))) > f(\xi(Y))$, there exists an
index $k$ such that $f(\xi(C^d_{\lambda_k}(X))) > f(\xi(Y)) \geq \lambda_k$ where
$\lambda_k$ is defined as in the construction of $C^{tr}(X)$. By Lemma
\ref{lem:lambdachoice}, and because $\min\{f(\xi(C^d_{\lambda}(Y))),\lambda_k\}=\lambda_k=\min\{f(\xi(C^d(X))),\lambda_k\}$,
$C^d_{\lambda_k}(X)$ merit dominates $Y$. This is a contradiction because
$f(\xi(C^d_{\lambda_k}(X))) > f(\xi(Y))$, $C^d_{\lambda_k}(X)$ merit dominates $Y$,
and $Y\in \mathcal{P}(X)$. Hence, we get that $Y\in C^{tr}(X)$. Since $Y$
is an arbitrary set in $\mathcal{P}(X)$, we conclude that $\mathcal{P}(X) \subseteq C^{tr}(X)$.
\end{proof}
Claims \ref{claim:pareto} and \ref{claim:trace} imply that $\mathcal{P}(X) = C^{tr}(X)$.
\qed




\bibliographystyle{aer}
\bibliography{matching}


\newpage
%=============================================

\section*{Online Appendix}
In this appendix, we present the proofs of our auxiliary results.


\subsection*{Proof of Proposition \ref{prop:matroid}}
%\begin{proof}[Proof of Proposition \ref{prop:matroid}]

If $(\mathcal{X},\mathcal{F})$ is a matroid, the greedy rule satisfies
path independence \citep{fleiner2001} and the law of aggregate demand \citep{yokoi2019}. Therefore, (1) implies (3). Furthermore, (3)
implies (2) trivially. To complete the proof, we show that
(2) implies (1).

Suppose that (2) is satisfied. Let $\mathcal B$ denote the
collection of maximal sets in $\mathcal F$.
By assumption, $\mathcal{F}$ is nonempty, which implies that $\mathcal{B}$
is nonempty. Hence, \emph{B1} holds. Before
showing \emph{B2'}, we establish that \emph{I2} is satisfied.

To show \emph{I2}, let $X \in \mathcal{F}$.
Consider a weight function that assigns all contracts in $\mathcal{X}$ a
distinct positive weight. Let $C$ be the greedy rule for such a weight function.
Then, by the greedy rule definition, $C(X)=X$ since $X\in \mathcal{F}$. For any
$X'\subseteq X$, path independence implies that $C(X')=X'$.
In addition, by the greedy rule definition, $C(X')\in \mathcal{F}$,
so we get $X'\in \mathcal{F}$. Therefore, \emph{I2} is satisfied.

Suppose, for contradiction, that \emph{B2'} is not satisfied. Therefore,
there exist $X_1, X_2 \in \mathcal B$ and $x_1 \in X_1 \setminus X_2$
such that for each $x_2 \in X_2 \setminus X_1$, $(X_1 \setminus \{x_1\}) \cup \{x_2\}$ is not included in a feasible set in $\mathcal{F}$.

Consider a weight function that assigns all contracts in $\mathcal{X}$ a
distinct and positive weight so that contracts in $X_1 \setminus \{x_1\}$
have higher weights than contracts in $X_2 \setminus X_1$, and contracts
in $X_2 \setminus X_1$ have higher weights than the weight of $x_1$.
%Suppose that contracts in $\mathcal{X}\setminus \left(X_1 \cup X_2\right)$ have a
%negative weight.
Let $C'$ be the greedy rule for such a weight function.

When $X_1 \cup X_2$ is the set of available contracts for the greedy rule $C'$,
it chooses $X_1 \setminus \{x_1\}$ first because $X_1 \in \mathcal{F}$ and the weights of contracts in $X_1 \setminus \{x_1\}$ are greater
than the weights of other contracts in $X_1 \cup X_2$. Next the greedy rule chooses no $x_2\in X_2 \setminus X_1$ because, by construction, $(X_1 \setminus \{x_1\}) \cup \{x_2\}$ is not included in a feasible set in $\mathcal{F}$. Finally, the greedy rule chooses $x_1$ because $(X_1 \setminus \{x_1\}) \cup \{x_1\}=X_1 \in \calf$. Since $X_1\in \mathcal{B}$, no other contract
can be chosen. Therefore, we get
\[C'(X_1 \cup X_2) = X_1.\]

When $\{x_1\} \cup X_2$ is the set of available contracts for the greedy rule $C'$,
contracts in $X_2$ are chosen first because they have positive weights greater
than the weight of $x_1$ and $X_2 \in \mathcal{F}$. Furthermore, since
$X_2\in \mathcal{B}$, $x_1$ is not chosen and we get
\[C'(\{x_1\} \cup X_2)=X_2.\]

The two displayed equations provide a contradiction to path independence of $C'$: By
Lemma \ref{lem:pi}, path independence implies the substitutes condition. Now, by
the substitutes condition, $x_1\in X_1=C'(X_1\cup X_2)$ implies $x_1\in C'(\{x_1\}\cup X_2)=X_2$, which is a contradiction since $x_1\notin X_2$.

Therefore, we conclude that the maximal sets in $\mathcal{F}$ satisfy \emph{B1} and \emph{B2'},
which together with Lemma \ref{lem:matchar} implies that they are the bases of a matroid. Since $\mathcal{F}$ satisfies \emph{I2}, $\mathcal{F}$ is the collection of subsets of the bases, which implies that $(X,\mathcal{F})$ is a matroid (see Theorem 1.2.3 of \cite{oxley}).
%
%
%\emph{B2'} is satisfied and, hence, $(\calx,\calf)$
%is a matroid by Lemma \ref{lem:matchar}.
%\end{proof}
\qed
%==================================================
\bigskip


\subsection*{Proof of the Statement in Example \ref{ex:ladfail}}
We prove the statement that the diversity index defined in Example \ref{ex:ladfail}
satisfies ordinal concavity. We consider several cases depending on the value of $\xi$ used in the definition of ordinal concavity.

%\medskip
%\noindent
%\emph{Case 1:} $\xi=\xi(\{x,y,z\})$. If $\chi_{(c,t)}=\xi(\{x\})$ or %$\chi_{(c,t)}=\xi(\{y\})$, then $f(\xi-\chi_{(c,t)})>f(\xi)$. Therefore, condition (i)
%in the definition of ordinal concavity is satisfied. Now, let
%$\chi_{(c,t)}=\xi(\{z\})$. Then $\tilde{\xi}_c^t=0$, and for all such $\tilde \xi$
%we have $f(\tilde{\xi}+\xi(\{z\}))>f(\tilde \xi)$. Therefore, condition (ii) in the
%definition of ordinal concavity is satisfied.

\medskip
\noindent
\emph{Case 1:} $\xi=\xi(\{x,y\})$. Let $t\in \calt$ be the type of the student
associated with contract $x$ and $t'\in \calt$ be the type of the student
associated with contract $z$.
If $\tilde{\xi}_{c}^{t'}=0$, then $\tilde{\xi}=\xi(\emptyset)$
or $\tilde{\xi}=\xi(\{y\})$. For $\tilde \xi=\xi(\emptyset)$, we have
$f(\tilde \xi + \xi(\{x\}))>f(\tilde \xi)$. Therefore, condition (ii) in the definition of ordinal concavity is satisfied. For $\tilde \xi=\xi(\{y\})$, we have $f(\xi-\chi_{c,t})=f(\xi)$ and $f(\tilde \xi+\chi_{c,t})=f(\tilde \xi)$. Therefore, condition (iii) in the definition of ordinal concavity is satisfied. However, if $\tilde{\xi}_{c}^{t'}=1$,
then $\tilde{\xi}=\xi(\{z\})$ or $\tilde{\xi}=\xi(\{y,z\})$. For both values of
$\tilde \xi$, $f(\xi-\chi_{c,t}+\chi_{c,t'})>f(\xi)$, which means that
condition (i) in the definition of ordinal concavity is satisfied.

\medskip
\noindent
\emph{Case 2:} $\xi=\xi(\{y,z\})$. If $\chi_{c,t}=\xi(\{y\})$ and $n>5$,
then we have $f(\xi-\chi_{c,t})>f(\xi)$, so condition (i) in the
definition of ordinal concavity is satisfied.

If $\chi_{c,t}=\xi(\{y\})$ and $n=5$, then, for $\tilde \xi = \xi(\emptyset)$, we have
$f(\tilde \xi+\chi_{c,t})>f(\tilde \xi)$, so condition (ii) in the definition
of ordinal concavity is satisfied. For $\tilde \xi = \xi(\{x\})$, we have
$f(\xi-\chi_{c,t})=f(\xi)$ and $f(\tilde \xi+\chi_{c,t})=f(\tilde \xi)$, so
condition (iii) in the definition of ordinal concavity is satisfied
For $\tilde \xi = \xi(\{z\})$, we have
$f(\tilde \xi +\chi_{c,t})=f(\tilde \xi)$ and $f(\xi-\chi_{c,t})=f(\xi)$, so
condition (iii) in the definition of ordinal concavity is satisfied. Finally, if
$\tilde \xi = \xi(\{x,z\})$, let $t'\in \calt$ be such that $\chi_{c,t'}=\xi(\{z\})$.
Then $f(\tilde \xi +\chi_{c,t}-\chi_{c,t'})=f(\tilde \xi)$ and
$f(\xi-\chi_{c,t'}+\chi_{c,t})=f(\xi)$, so condition (iii) in the definition of
ordinal concavity is satisfied.

However, if $\chi_{c,t}=\xi(\{z\})$, then $\tilde{\xi}_c^t=0$,
and for all such $\tilde \xi$ except $\xi(\{x,y\})$, we have
$f(\tilde \xi+\chi_{c,t})>f(\tilde{\xi})$. Therefore, condition (ii)  in the definition of ordinal concavity is satisfied. For $\tilde \xi = \xi(\{x,y\})$, let
$\chi_{c,t'} = \xi(\{x\})$. Then $f(\tilde \xi +\chi_{c,t}-\chi_{c,t'})=f(\tilde \xi)$
and $f(\xi-\chi_{c,t'}+\chi_{c,t})=f(\xi)$, so condition (iii) in the
definition of ordinal concavity is satisfied.

\medskip
\noindent
\emph{Case 3:} $\xi=\xi(\{x,z\})$. Since the diversity index $f$ is symmetric
with respect to $x$ and $y$, this case is analogous to Case 2 above.

\medskip
\noindent
\emph{Case 4:} $\xi=\xi(\{z\})$. In this case, we have $\chi_{c,t}=\xi(\{z\})$ and
$\tilde{\xi}_c^t=0$. For all such $\tilde{\xi}$ except $\xi(\{x,y\})$, we have
$f(\tilde{\xi}+\chi_{c,t})>f(\tilde{\xi})$, so condition (ii) is
satisfied in the definition of ordinal concavity. For $\tilde{\xi}=\xi(\{x,y\})$,
if we let $\chi_{c,t'}=\xi(\{x\})$ where $t'\in \calt$ is the type of student
associated with contract $x$, then $f(\tilde \xi+\chi_{c,t}-\chi_{c,t'}) >
f(\tilde \xi)$, so condition (ii) in the definition of ordinal concavity is
satisfied.

\medskip
\noindent
\emph{Case 5:} $\xi=\xi(\{x\})$. In this case, we have $\chi_{c,t}=\xi(\{x\})$ and
$\tilde{\xi}_c^t=0$. If $\tilde \xi = \xi(\emptyset)$, then
$f(\tilde \xi+\chi_{c,t})>f(\tilde \xi)$.
Therefore, condition (ii) in the definition of ordinal concavity is satisfied.
If $\tilde \xi = \xi(\{y\})$, let $t'\in \calt$ be such that $\chi_{c,t'}=\xi(\{y\})$.
Then $f(\tilde{\xi}+\chi_{c,t}-\chi_{c,t'})=f(\tilde{\xi})$
and $f(\xi-\chi_{c,t}+\chi_{c,t'})=f(\xi)$, so condition (iii) in the definition of
ordinal concavity is satisfied.
If $\tilde{\xi} \in \{\xi(\{z\}),\xi(\{y,z\})\}$, then
let $t''\in \calt$ be such that $\chi_{c,t''}=\xi(\{z\})$.
Then $f(\xi-\chi_{c,t}+\chi_{c,t''})>f(\xi)$,
which means that condition (i) in the definition of ordinal
concavity is satisfied.

\medskip
\noindent
\emph{Case 6:} $\xi=\xi(\{y\})$. Since the diversity index $f$ is symmetric
with respect to $x$ and $y$, this case is analogous to Case 5 above.






\qed


%==================================================
\bigskip


\subsection*{Proof of Proposition \ref{prop:structure}}
%\begin{proof}[Proof of Proposition \ref{prop:structure}]
If $x\notin C^d(X \cup \{x\})$, then by the irrelevance of rejected contracts condition (which follows from Theorem \ref{thm:pi} and Lemma \ref{lem:pi}),
we have $C^d(X)=C^d(X\cup\{x\})$. Theorefore, the law of aggregate demand is satisfied.

Otherwise, if $x\in C^d(X \cup \{x\})$, we consider several cases depending on the
value of $\xi(C^d(X\cup\{x\}))$ as in the proof of Theorem \ref{thm:pi}.

Let $c=\gamma(x)$, $t=\tau(\sigma(x))$, $\xi_1= \xi(C^d(X))$, and $\xi_2=\xi(C^d(X\cup\{x\}))$ for the rest of the proof.

\noindent
\emph{Case 1:} Consider the case $\xi_2 \leq \xi(X)$. In this case, in the proof of Theorem \ref{thm:pi},
we show $C^d(X \cup\{x\})= (C^d(X) \cup \{x\})\setminus \{y\}$ for some $y\in C^d(X)$.


\noindent
\emph{Case 2:} Consider the case $\xi_2 \not \leq \xi(X)$. Since $C^d(X\cup \{x\})\subseteq X\cup \{x\}$,
it must be that $(\xi_2)^t_c> \xi^t_c(X)$, so $C^d(X\cup \{x\})$ includes $x$ and all
contracts in $X_c=\{x\in X|\gamma(x)=c\}$ with type-$t$ students. Furthermore, $(\xi_2)^t_c>(\xi_1)^t_c$ because $(\xi_1)^t_c\leq \xi^t_c(X)$.

\begin{claim}\label{claim:norm}
$\norm{\xi_2}\geq \norm{\xi_1}$.
\end{claim}

\begin{proof}[Proof of Claim \ref{claim:norm}]\renewcommand{\qedsymbol}{$\blacksquare$}
Starting from $y_1=\xi_1$, we construct a finite sequence of distributions $y_1,\ldots,y_k\leq \xi(X\cup \{x\})$
where $k\geq 1$ such that, $f(y_k)=f(\xi_2)$, and, for each $i=2,\ldots,k$,
\begin{enumerate}[(i)]
  \item $\norm{y_i}\geq \norm{y_{i-1}}$,
  \item $f(y_i) \geq f(y_{i-1})$, and
  \item $\norm{y_i-\xi_2} < \norm{y_{i-1}-\xi_2}$.
\end{enumerate}

If $f(\xi_1)=f(\xi_2)$, then we are done because we can let $k=1$ and $y_1=\xi_1$.
Otherwise, if $f(\xi_1)<f(\xi_2)$, then we construct such a sequence inductively
as follows. Suppose that we have $y_1, \ldots, y_i$. If $f(y_i)=f(\xi_2)$,
then we are done. Otherwise, if $f(y_i)\neq f(\xi_2)$,
we must have $f(y_i)<f(\xi_2)$ because $\xi_2\in \Xi^*(X\cup \{x\})$
and $y_i\leq \xi(X\cup \{x\})$. By monotonicity, there exists a pair
$(c',t')\in \calc \times \calt$ such that $(\xi_2)_{c'}^{t'}>(y_i)_{c'}^{t'}$.
Then, by ordinal concavity,
either (i)
\begin{enumerate}
\item $f(\xi_2-\chi_{c',t'})>f(\xi_2)$, or
\item $f(y_i+\chi_{c',t'})>f(y_i)$, or
\item $f(\xi_2-\chi_{c',t'})=f(\xi_2)$ and $f(y_i+\chi_{c',t'})=f(y_i)$
\end{enumerate}
or (ii) there exist school $\hat c\in \calc$ and type $\hat t\in \calt$  with
$(\xi_2)_{\hat c}^{\hat t}<(y_i)_{\hat c}^{\hat t}$ such that
\begin{enumerate}
\item $f(\xi_2-\chi_{c',t'}+\chi_{\hat c, \hat t})>f(\xi_2)$, or
\item $f(y_i+\chi_{c',t'}-\chi_{\hat c, \hat t})>f(y_i)$, or
\item $f(\xi_2-\chi_{c',t'}+\chi_{\hat c, \hat t})=f(\xi_2)$ and $f(y_i+\chi_{c',t'}-\chi_{\hat c, \hat t}) = f(y_i)$.
\end{enumerate}

Suppose that condition (i) holds. We cannot have condition (i)(1) because $\xi_2-\chi_{c',t'}\leq \xi(X\cup\{x\})$ and $f(\xi_2-\chi_{c',t'})>f(\xi_2)$ give us a contradiction because $\xi_2\in \Xi^*(X\cup \{x\})$.
Both condition (i)(2) and (i)(3) imply $f(y_i+\chi_{c',t'})\geq f(y_i)$.
Therefore, we can let $y_{i+1}=y_i+\chi_{c',t'}$
because $\norm{y_{i+1}}=\norm{y_i} +1 > \norm{y_i}$, $f(y_{i+1}) \geq f(y_{i})$, and
$\norm{y_{i+1}-\xi_2}=\norm{y_i-\xi_2}-1 < \norm{y_i-\xi_2}$.

Suppose that condition (ii) holds. We cannot have condition (ii)(1)
because $\xi_2-\chi_{c',t'}+\chi_{\hat c, \hat t} \leq \xi(X\cup\{x\})$ and
$f(\xi_2-\chi_{c',t'}+\chi_{\hat c, \hat t})>f(\xi_2)$ give us a contradiction
because $\xi_2\in \Xi^*(X\cup \{x\})$.
Both conditions (ii)(2) and (ii)(3) imply $f(y_i+\chi_{c',t'}-\chi_{\hat c, \hat t}) \geq f(y_i)$.
Therefore, we can let $y_{i+1}=y_i+\chi_{c',t'}-\chi_{\hat c, \hat t}$
because $\norm{y_{i+1}}=\norm{y_i}$, $f(y_{i+1}) \geq f(y_{i})$, and
$\norm{y_{i+1}-\xi_2}=\norm{y_i-\xi_2}-2< \norm{y_i-\xi_2}$.

Since $\norm{y_i-\xi_2}$ can take only a finite number of values,
there exists $k$ such that $f(y_k)=f(\xi_2)$ or $\norm{y_k-\xi_2}=0$.
If $f(y_k)=f(\xi_2)$, since $\xi_2$ is maximal in $\Xi^*(X\cup \{x\})$
and the set of maximal distributions in $\Xi^*(X\cup \{x\})$ is
M-convex (Lemma \ref{lem:pareto}), we get $\norm{\xi_2} \geq \norm{y_k}$. Otherwise, if
$\norm{y_k-\xi_2}=0$, we get $y_k=\xi_2$ and, hence, $\norm{\xi_2} = \norm{y_k}$. In both cases we
establish $\norm{\xi_2} \geq \norm{y_k}$.

Furthermore, by construction, $\norm{y_k}\geq \norm{y_1}=\norm{\xi_1}$. Together with $\norm{\xi_2} \geq \norm{y_k}$,
this inequality implies $\norm{\xi_2}\geq \norm{\xi_1}$.
\end{proof}

To finish the proof of Case 2, we combine the results that we have:
\begin{itemize}
\item $(\xi_2)_c^t=\xi^t_c(X\cup \{x\})=\xi^t_c(X)+1$,
\item $(\xi_1)_c^t=\xi^t_c(X)$ (Claim \ref{claim:comparetc}),
\item $\norm{\xi_2} \geq \norm{\xi_1}$ (Claim \ref{claim:norm}), and,
\item for each type $t'\in \calt$ and school $c'\in \calc$
with $(t',c')\neq (t,c)$, $(\xi_1)_{c'}^{t'} \geq (\xi_2)_{c'}^{t'}$ (Claim \ref{claim:types}).
\end{itemize}
These lead to two possibilities:
\begin{enumerate}
  \item for each $t'\in \calt$ and $c' \in \calc$ with $(t',c')\neq (t,c)$, $(\xi_1)_{c'}^{t'} = (\xi_2)_{c'}^{t'}$, or
  \item there exist $\hat t \in \calt$ and $\hat c \in \calc$ with $(\hat t, \hat c) \neq (t,c)$ such that
  $(\xi_1)_{\hat c}^{\hat t} = (\xi_2)_{\hat c}^{\hat t}+1$ and for each $t'\in \calt$ and $c'\in \calc$ with $(t',c')\notin
  \{(t,c),(\hat t, \hat c)\}$, $(\xi_1)_{c'}^{t'} = (\xi_2)_{c'}^{t'}$.
\end{enumerate}
For a fixed type $t\in \calt$, school $c\in \calc$, and the number of contracts of type-$t$ students with school $c$, choice rule $C^d$ chooses contracts with the highest merit ranking. Therefore, if condition (1) holds, then $C^d(X\cup \{x\})=C^d(X) \cup \{x\}$. However, if
condition (2) holds, $C^d(X\cup \{x\})=(C^d(X) \cup \{x\})\setminus \{y\}$ for some $y\in C^d(X)$.
%\end{proof}
\qed
%==================================================
\bigskip



\subsection*{Proof of Proposition \ref{prop:truncation-necessary}}
This proposition follows from Proposition \ref{prop:ordinal-equivalence} because pseudo M$^\natural$-concavity is weaker than pseudo M$^\natural$-concavity$^+$. \qed

\subsection*{Proof of Proposition \ref{prop:ordinal-equivalence}}
\begin{subequations}

\medskip
\noindent
\emph{The ``only if'' direction:} Let $\xi, \tilde{\xi} \in \Xi^0$ and $(c, t) \in \mathcal{C} \times \mathcal{T}$ with $\xi_c^t>\tilde{\xi}_{c}^t$.
Our goal is to prove that there exists $\left(c^{\prime}, t^{\prime}\right) \in (\mathcal{C} \times \mathcal{T})\cup\{\emptyset\}$ (with $\xi_{c^{\prime}}^{t^{\prime}}<\tilde{\xi}_{c^{\prime}}^{t^{\prime}}$ whenever $(c',t')\neq \emptyset$) such that
\begin{align}
\min\{f(\xi),f(\tilde{\xi})\}\leq\min\{f(\xi-\chi_{c,t}+\chi_{c',t'}), f(\tilde{\xi}+\chi_{c,t}-\chi_{c',t'})\},
\label{TE-1}
\end{align}
and conditions (A) and (B) are satisfied.

Suppose that $f(\xi)=f(\tilde{\xi})$. Let $\lambda^*$ denote the equal value. By ordinal concavity of $f_{\lambda^*}$, there exists $\left(c^*, t^*\right) \in (\mathcal{C} \times \mathcal{T})\cup\{\emptyset\}$ (with $\xi_{c^*}^{t^*}<\tilde{\xi}_{c^*}^{t^*}$ whenever $(c^*,t^*)\neq \emptyset$) such that
\begin{itemize}
\item[(i$^*$)] $f_{\lambda^*}(\xi-\chi_{c, t}+\chi_{c^{*}, t^{*}})>f_{\lambda^*}(\xi)$, or
\item[(ii$^*$)] $f_{\lambda^*}(\tilde{\xi}+\chi_{c, t}-\chi_{c^{*}, t^{*}})>f_{\lambda^*}(\tilde{\xi})$, or
\item[(iii$^*$)] $f_{\lambda^*}(\tilde{\xi}+\chi_{c, t}-\chi_{c^{*}, t^{*}})=f_{\lambda^*}(\tilde{\xi})$ and $f_{\lambda^*}\left(\xi-\chi_{c, t}+\chi_{c^{*}, t^{*}}\right)=f_{\lambda^*}(\xi)$.
\end{itemize}
By the definition of $f_{\lambda^*}(\cdot)$, neither (i$^*$) nor (ii$^*$) holds. Thus, (iii$^*$) holds, which implies
\begin{align*}
f(\xi-\chi_{c,t}+\chi_{c^*,t^*})\geq \lambda^* \text{ and } f(\tilde{\xi}+\chi_{c,t}-\chi_{c^*,t^*})\geq \lambda^*.
\end{align*}
It follows that (\ref{TE-1}) holds. Note that neither the if-clause of (A) nor that of (B) holds.

In the remaining part, we assume $f(\xi)<f(\tilde{\xi})$ (the other case $f(\xi)>f(\tilde{\xi})$ can be handled analogously).
Under this assumption, for each $(c',t')\in (\mathcal{C}\times \mathcal{T})\cup \{\emptyset\}$ that satisfies (\ref{TE-1}), the if-clause of (A) never holds. Thus, it suffices to prove that (\ref{TE-1}) and condition (B) hold for some $(c',t')\in (\mathcal{C}\times \mathcal{T})\cup \{\emptyset\}$.
%check whether the if-clause of (B) holds or not.


Let $\Phi \subseteq \{(c',t')\in (\mathcal{C}\times \mathcal{T}) \mid \xi_{c^{\prime}}^{t^{\prime}}<\tilde{\xi}_{c^{\prime}}^{t^{\prime}}\}\cup \{\emptyset\}$ be the set of coordinates that satisfy one of the following conditions:
\begin{itemize}
\item[(i)] $f(\xi-\chi_{c, t}+\chi_{c^{\prime}, t^{\prime}})>f(\xi)$, or
\item[(ii)] $f(\tilde{\xi}+\chi_{c, t}-\chi_{c^{\prime}, t^{\prime}})>f(\tilde{\xi})$, or
\item[(iii)] $f(\tilde{\xi}+\chi_{c, t}-\chi_{c^{\prime}, t^{\prime}})=f(\tilde{\xi})$ and $f\left(\xi-\chi_{c, t}+\chi_{c^{\prime}, t^{\prime}}\right)=f(\xi)$.
\end{itemize}
Note that $\Phi\neq \emptyset$ because $f_\lambda=f$ holds for a sufficiently large $\lambda$ and the function satisfies ordinal concavity.

\smallskip
\noindent
\emph{Case 1:}
Suppose there exists $(c',t')\in \Phi$ for which (iii) holds. Then, (\ref{TE-1}) immediately follows (the if-clause of (B) does not hold).

\smallskip
\noindent
\emph{Case 2:}  Suppose that there does not exist $(c',t')\in \Phi$ for which (iii) holds.

\smallskip
\noindent
\emph{Subcase 2-1:} Suppose that there does not exist $(c',t')\in \Phi$ for which (i) holds. In this case, every $(c',t')\in \Phi$ satisfies (ii). Let $\lambda'=f(\tilde{\xi})$. Since $f_{\lambda'}$ satisfies ordinal concavity, there exists $(c'',t'')\in (\mathcal{C}\times \mathcal{T})\cup\{\emptyset\}$ (with $\xi_{c''}^{t''}<\tilde{\xi}_{c''}^{t''}$ whenever $(c'',t'')\neq \emptyset$) such that
\begin{itemize}
\item[(iv)] $f_{\lambda'}(\xi-\chi_{c, t}+\chi_{c'', t''})>f_{\lambda'}(\xi)$, or
\item[(v)] $f_{\lambda'}(\tilde{\xi}+\chi_{c, t}-\chi_{c'', t''})>f_{\lambda'}(\tilde{\xi})$, or
\item[(vi)] $f_{\lambda'}(\tilde{\xi}+\chi_{c, t}-\chi_{c'', t''})=f_{\lambda'}(\tilde{\xi})$ and $f_{\lambda'}\left(\xi-\chi_{c, t}+\chi_{c'', t''}\right)=f_{\lambda'}(\xi)$.
\end{itemize}
By the definition of truncation, (v) never holds. If (iv) holds, then together with $\lambda'=f(\tilde{\xi})>f(\xi)$, we obtain a contradiction to the assumption of Subcase 2-1. Thus, (vi) holds, which establishes (\ref{TE-1}) (the if-clause of (B) does not hold).

\smallskip
\noindent
\emph{Subcase 2-2:} Suppose that there exists $(c',t')\in \Phi$ for which (i) holds. Let $\Phi' \subseteq \Phi$ be the set of coordinates for which (i) holds.

\smallskip
\noindent
\emph{Subcase 2-2-1:}
Suppose that there exists $(c',t')\in \Phi'$ such that
\begin{align}
f(\tilde{\xi}+\chi_{c,t}-\chi_{c',t'})\geq f(\xi) \; (=\min\{f(\xi), f(\tilde{\xi})\}).
\label{TE-3}
\end{align}
Then, (\ref{TE-1}) holds (the if-clause of (B) does not hold).

\smallskip
\noindent
\emph{Subcase 2-2-2:}
Suppose that there does not exist $(c',t')\in \Phi'$ that satisfies (\ref{TE-3}). Let $\lambda''=f(\xi)$. Since $f_{\lambda''}$ satisfies ordinal concavity, there exists $(c''',t''')\in (\mathcal{C}\times \mathcal{T})\cup\{\emptyset\}$ (with $\xi_{c'''}^{t'''}<\tilde{\xi}_{c'''}^{t'''}$ whenever $(c''',t''')\neq \emptyset$) such that
\begin{itemize}
\item[(vii)] $f_{\lambda''}(\xi-\chi_{c, t}+\chi_{c''', t'''})>f_{\lambda''}(\xi)$, or
\item[(viii)] $f_{\lambda''}(\tilde{\xi}+\chi_{c, t}-\chi_{c''', t'''})>f_{\lambda''}(\tilde{\xi})$, or
\item[(ix)] $f_{\lambda''}\left(\xi-\chi_{c, t}+\chi_{c''', t'''}\right)=f_{\lambda''}(\xi)$ and $f_{\lambda''}(\tilde{\xi}+\chi_{c, t}-\chi_{c''', t'''})=f_{\lambda''}(\tilde{\xi})$.
\end{itemize}
By the definition of $f_{\lambda''}(\cdot)$, neither (vii) nor (viii) holds. Thus, (ix) holds.

If $f(\xi)<f(\xi-\chi_{c,t}+\chi_{c''',t'''})$, then $(c''',t''')\in \Phi'$. By the assumption of Subcase 2-2-2, $f(\tilde{\xi}+\chi_{c,t}-\chi_{c''',t'''})<f(\xi)=\lambda''$. Then, $f_{\lambda''}(\tilde{\xi}+\chi_{c, t}-\chi_{c''', t'''})=f(\tilde{\xi}+\chi_{c,t}-\chi_{c''',t'''})<\lambda''= f_{\lambda''}(\tilde{\xi})$, where the last equality follows from $\lambda''=f(\xi)<f(\tilde{\xi})$.  We obtain a contradiction to $f_{\lambda''}(\tilde{\xi}+\chi_{c, t}-\chi_{c''', t'''})=f_{\lambda''}(\tilde{\xi})$ stated in (ix).

It follows that $f(\xi-\chi_{c,t}+\chi_{c''',t'''})\leq f(\xi)=\lambda''$. Together with $f_{\lambda''}\left(\xi-\chi_{c, t}+\chi_{c''', t'''}\right)=f_{\lambda''}(\xi)=\lambda''$ (the former equality follows from (ix)), we have
\begin{align}
f(\xi)=f(\xi-\chi_{c,t}+\chi_{c''',t'''}).
\label{TE-4}
\end{align}
By $f_{\lambda''}(\tilde{\xi}+\chi_{c, t}-\chi_{c''', t'''})=f_{\lambda''}(\tilde{\xi})\geq \lambda''$ (the equality follows from (ix)), we have $f(\tilde{\xi}+\chi_{c,t}-\chi_{c''',t'''})\geq \lambda''=f(\xi)$. This condition and (\ref{TE-4}) imply that (\ref{TE-1}) holds for $(c''',t''')$. Note that the if-clause of (B) holds if $f(\tilde{\xi}+\chi_{c,t}-\chi_{c''',t'''})<f(\tilde{\xi})$.
In this case, by the assumption of Subcase 2-2, there exists a coordinate in $\Phi'$, for which the desired strict inequality in (B) holds.

\medskip
\noindent
\emph{The ``if'' direction:} Let $\xi, \tilde{\xi} \in \Xi^0$ and $(c, t) \in \mathcal{C} \times \mathcal{T}$ with $\xi_c^t>\tilde{\xi}_{c}^t$. By pseudo M$^\natural$-concavity$^+$, there exists $\left(c^{\prime}, t^{\prime}\right) \in (\mathcal{C} \times \mathcal{T})\cup\{\emptyset\}$ (with $\xi_{c^{\prime}}^{t^{\prime}}<\tilde{\xi}_{c^{\prime}}^{t^{\prime}}$ whenever $(c',t')\neq \emptyset$) such that
\begin{align}
\min\{f(\xi),f(\tilde{\xi})\}\leq\min\{f(\xi-\chi_{c,t}+\chi_{c',t'}), f(\tilde{\xi}+\chi_{c,t}-\chi_{c',t'})\},
\label{TE-5}
\end{align}
and conditions (A) and (B) are satisfied. Let $\lambda\geq 0$.

\smallskip
\noindent
\emph{Case 1:} Suppose $f(\xi)=f(\tilde{\xi})$.


\smallskip
\noindent
\emph{Subcase 1-1:} Suppose $\lambda>f(\xi)=f(\tilde{\xi})$. Then, (\ref{TE-5}) implies that one of the following conditions holds:
\begin{itemize}
\item $f(\xi)<f(\xi+\chi_{c,t}-\chi_{c',t'}) \: \bigl(\Longleftrightarrow f_\lambda(\xi)<f_\lambda(\xi+\chi_{c,t}-\chi_{c',t'})\bigr)$, or
\item $f(\tilde{\xi})<f(\tilde{\xi}-\chi_{c,t}+\chi_{c',t'}) \: \bigl(\Longleftrightarrow f_\lambda(\tilde{\xi})<f_\lambda(\tilde{\xi}-\chi_{c,t}+\chi_{c',t'})\bigr)$, or
\item $f(\xi)=f(\xi+\chi_{c,t}-\chi_{c',t'}) \text{ and } f(\tilde{\xi})=f(\tilde{\xi}-\chi_{c,t}+\chi_{c',t'})$
\item[] $\bigl(\Longleftrightarrow f_\lambda(\xi)=f_\lambda(\xi+\chi_{c,t}-\chi_{c',t'}) \text{ and } f_\lambda(\tilde{\xi})=f_\lambda(\tilde{\xi}-\chi_{c,t}+\chi_{c',t'})\bigr)$.
\end{itemize}
Thus, ordinal concavity of $f_\lambda$ holds.

\smallskip
\noindent
\emph{Subcase 1-2:}
Suppose $\lambda\leq f(\xi)=f(\tilde{\xi})$. Then, (\ref{TE-5}) implies $f(\xi-\chi_{c,t}+\chi_{c',t'})\geq f(\xi)$ and $f(\tilde{\xi}+\chi_{c,t}-\chi_{c',t'})\geq f(\tilde{\xi})$, which in turn implies
\begin{align*}
f_\lambda(\xi)=f_\lambda(\xi+\chi_{c,t}-\chi_{c',t'}) \text{ and } f_\lambda(\tilde{\xi})=f_\lambda(\tilde{\xi}-\chi_{c,t}+\chi_{c',t'}).
\end{align*}
Thus, ordinal concavity of $f_\lambda$ holds.

\smallskip
\noindent
\emph{Case 2:}
Suppose $f(\xi)\neq f(\tilde{\xi})$. We assume $f(\xi)<f(\tilde{\xi})$ (the other case $f(\xi)>f(\tilde{\xi})$ can be handled analogously).

\smallskip
\noindent
\emph{Subcase 2-1:}
Suppose $\lambda>f(\tilde{\xi})$. Note that (\ref{TE-5}) implies $f(\xi)\leq f(\xi-\chi_{c,t}+\chi_{c',t'})$.

\smallskip
\noindent
\emph{Subcase 2-1-1:}
Suppose $f(\xi)<f(\xi-\chi_{c,t}+\chi_{c',t'})$. This inequality is equivalent to $f_\lambda(\xi)<f_\lambda(\xi-\chi_{c,t}+\chi_{c',t'})$, showing that ordinal concavity of $f_\lambda$ holds.

\smallskip
\noindent
\emph{Subcase 2-1-2:}
Suppose $f(\xi)=f(\xi-\chi_{c,t}+\chi_{c',t'})$. Equivalently,
\begin{align}
f_\lambda(\xi)=f_\lambda(\xi-\chi_{c,t}+\chi_{c',t'}).
\label{TE-5-5}
\end{align}
\begin{itemize}
\item If $f(\tilde{\xi})<f(\tilde{\xi}+\chi_{c,t}-\chi_{c',t'})$, then equivalently $f_\lambda(\tilde{\xi})<f_\lambda(\tilde{\xi}+\chi_{c,t}-\chi_{c',t'})$, showing that ordinal concavity of $f_\lambda$ holds.
\item If $f(\tilde{\xi})=f(\tilde{\xi}+\chi_{c,t}-\chi_{c',t'})$, then equivalently $f_\lambda(\tilde{\xi})= f_\lambda(\tilde{\xi}+\chi_{c,t}-\chi_{c',t'})$, which together with (\ref{TE-5-5}) implies that ordinal concavity of $f_\lambda$ holds.
\item If $f(\tilde{\xi})>f(\tilde{\xi}+\chi_{c,t}-\chi_{c',t'})$, then the if-clause of (B) holds. Thus, there exists $\left(c'', t''\right) \in (\mathcal{C} \times \mathcal{T})\cup\{\emptyset\}$ (with $\xi_{c''}^{t''}<\tilde{\xi}_{c''}^{t''}$ whenever $(c'',t'')\neq \emptyset$) such that
\begin{align*}
f(\xi)<f(\xi-\chi_{c,t}+\chi_{c'',t''}).
\end{align*}
This inequality is equivalent to $f_\lambda(\xi)<f_\lambda(\xi-\chi_{c,t}+\chi_{c'',t''})$, showing that ordinal concavity of $f_\lambda$ holds.
\end{itemize}

\smallskip
\noindent
\emph{Subcase 2-2:}
Suppose $f(\tilde{\xi})\geq \lambda>f(\xi)$. Note that (\ref{TE-5}) implies $f(\xi)\leq f(\xi-\chi_{c,t}+\chi_{c',t'})$.

\smallskip
\noindent
\emph{Subcase 2-2-1:}
Suppose $f(\xi)<f(\xi-\chi_{c,t}+\chi_{c',t'})$. This inequality is equivalent to $f_\lambda(\xi)<f_\lambda(\xi-\chi_{c,t}+\chi_{c',t'})$, showing that ordinal concavity of $f_\lambda$ holds.

\smallskip
\noindent
\emph{Subcase 2-2-2:}
Suppose $f(\xi)=f(\xi-\chi_{c,t}+\chi_{c',t'})$. Equivalently,
\begin{align}
f_\lambda(\xi)=f_\lambda(\xi-\chi_{c,t}+\chi_{c',t'}).
\label{TE-6}
\end{align}
\begin{itemize}
\item If $f(\tilde{\xi})\leq f(\tilde{\xi}+\chi_{c,t}-\chi_{c',t'})$, then $f_\lambda(\tilde{\xi})= f_\lambda(\tilde{\xi}+\chi_{c,t}-\chi_{c',t'})$, which together with (\ref{TE-6}) implies that ordinal concavity of $f_\lambda$ holds.
\item If $f(\tilde{\xi})>f(\tilde{\xi}+\chi_{c,t}-\chi_{c',t'})$, then the if-clause of (B) holds. Thus, there exists $\left(c'', t''\right) \in (\mathcal{C} \times \mathcal{T})\cup\{\emptyset\}$ (with $\xi_{c''}^{t''}<\tilde{\xi}_{c''}^{t''}$ whenever $(c'',t'')\neq \emptyset$) such that
\begin{align*}
f(\xi)<f(\xi-\chi_{c,t}+\chi_{c'',t''}).
\end{align*}
This inequality is equivalent to $f_\lambda(\xi)<f_\lambda(\xi-\chi_{c,t}+\chi_{c'',t''})$, showing that ordinal concavity of $f_\lambda$ holds.
\end{itemize}

\smallskip
\noindent
\emph{Subcase 2-3:}
Suppose $\lambda\leq f(\xi)$. By (\ref{TE-5}), we have $\lambda\leq f(\xi)\leq f(\xi-\chi_{c,t}+\chi_{c',t'})$ and $\lambda\leq f(\xi)\leq f(\tilde{\xi}-\chi_{c,t}+\chi_{c',t'})$, which implies
\begin{align*}
f_\lambda(\xi)=f_\lambda(\xi-\chi_{c,t}+\chi_{c',t'}) \text{ and } f_\lambda(\tilde{\xi})=f_\lambda(\tilde{\xi}+\chi_{c,t}-\chi_{c',t'}).
\end{align*}
Thus, ordinal concavity of $f_\lambda$ holds.
\qed
\end{subequations}

\bigskip


\subsection*{Proof of Claim \ref{claim:ex1-pMC}}
Let $\xi, \tilde \xi\in \Xi^0$ and $(c,t)$ with $\xi_c^t>\tilde \xi_c^t$. If $\tilde \xi=\xi(\emptyset)$, then
\begin{align*}
f(\tilde \xi)<f(\xi) \text{ and } f(\tilde \xi)<f(\tilde \xi+\chi_{c,t}).
\end{align*}
Together with the fact that $f(\xi(\emptyset))=0$ is the minimum function value, pseudo M$^\natural$-concavity$^+$ is satisfied. In the remaining part, suppose that $\tilde \xi\neq \xi(\emptyset)$. If $||\xi||=||\tilde \xi||=1$ or $\xi\geq \tilde \xi$, then pseudo M$^\natural$-concavity$^+$ trivially holds. In what follows, we consider the remaining three cases.

\smallskip
\noindent
\emph{Case 1:}
Suppose $\{\xi, \tilde \xi\}\subseteq \{\xi(\{x\}), \xi(\{y,z\})\}$.

\smallskip
\noindent
\emph{Subcase 1-1:}
Suppose $\xi=\xi(\{x\})$, which implies $\chi_{c,t}=\xi(\{x\})$. For $(c',t')$ with $\chi_{c',t'}=\xi(\{z\})$,
\begin{align*}
&f(\xi(\{x\})-\chi_{c,t}+\chi_{c',t'})=f(\xi(\{z\}))=n>1=f(\xi(\{x\})), \\
&f(\xi\{y,z\})+\chi_{c,t}-\chi_{c',t'})=f(\xi(\{x,y\}))=1.
\end{align*}
Together with $f(\xi(\{x\}))=1<5=f(\xi(\{y,z\})$, pseudo M$^\natural$-concavity$^+$ holds.

\smallskip
\noindent
\emph{Subcase 1-2:}
Suppose $\xi=\xi(\{y,z\})$ and $\chi_{c,t}=\xi(\{y\})$. For $(c',t')$ with $\chi_{c',t'}=\xi(\{x\})$,
\begin{align*}
&f(\xi(\{y,z\})-\chi_{c,t}+\chi_{c',t'})=f(\xi(\{x,z\}))=5=f(\xi(\{y,z\})),   \\
&f(\xi(\{x\})+\chi_{c,t}-\chi_{c',t'})=f(\xi(\{y\}))=1=f(\xi(\{x\})).
\end{align*}
Hence, pseudo M$^\natural$-concavity$^+$ holds.


\smallskip
\noindent
\emph{Subcase 1-3:}
Suppose $\xi=\xi(\{y,z\})$ and $\chi_{c,t}=\xi(\{z\})$. For $(c',t')$ with $\chi_{c',t'}=\xi(\{x\})$,
\begin{align*}
&f(\xi(\{x\})+\chi_{c,t}-\chi_{c',t'})=f(\xi(\{z\}))=5>1=f(\xi(\{x\})), \\
&f(\xi(\{y,z\})-\chi_{c,t}+\chi_{c',t'})=f(\xi(\{x,y\}))=1.
\end{align*}
Together with $f(\xi(\{x\}))=1<5=f(\xi(\{y,z\})$ pseudo M$^\natural$-concavity$^+$ holds.

\smallskip
\noindent
\emph{Case 2:}
Suppose $\{\xi, \tilde \xi\}\subseteq  \{\xi(\{y\}), \xi(\{x,z\})\}$. Since contracts $x$ and $y$ are symmetric, the proof of this case is similar to that for Case 1.

\smallskip
\noindent
\emph{Case 3:}
Suppose $\{\xi, \tilde \xi\}\subseteq \{\xi(\{z\}), \xi(\{x,y\})\}$.

\smallskip
\noindent
\emph{Subcase 3-1:}
Suppose $\xi=\xi(\{z\})$, which implies $\chi_{c,t}=\xi(\{z\})$. For $(c',t')$ with $\chi_{c',t'}=\xi(\{x\})$,
\begin{align*}
&f(\xi(\{x,y\})+\chi_{c,t}-\chi_{c',t'})=f(\xi(\{y,z\}))=5>1=f(\xi(\{x,y\})), \\
&f(\xi(\{z\})-\chi_{c,t}+\chi_{c',t'})=f(\xi(\{x\}))=1.
\end{align*}
Together with $f(\xi(\{x,y\}))=1<n=f(\xi(\{z\}))$, pseudo M$^\natural$-concavity$^+$ holds.

\smallskip
\noindent
\emph{Subcase 3-2:}
Suppose $\xi=\xi(\{x,y\})$ and $\chi_{c,t}=\xi(\{x\})$. For $(c',t')$ with $\chi_{c',t'}=\xi(\{z\})$,
\begin{align*}
&f(\xi(\{x,y\})-\chi_{c,t}+\chi_{c',t'})=f(\xi(\{y,z\}))=5>f(\xi(\{x,y\})),   \\
&f(\xi(\{z\})+\chi_{c,t}-\chi_{c',t'})=f(\xi(\{x\}))=1.
\end{align*}
Together with $f(\xi(\{x,y\}))=1<n=f(\xi(\{z\}))$, pseudo M$^\natural$-concavity$^+$ holds.

\smallskip
\noindent
\emph{Subcase 3-3:}
Suppose $\xi=\xi(\{x,y\})$ and $\chi_{c,t}=\xi(\{y\})$. Since contracts $x$ and $y$ are symmetric, the proof for this case is similar to that for Subcase 3-2.
\qed

%%%=========================

\bigskip




\subsection*{Proof of Claim \ref{claim:ex2-pMC}}
Suppose that $\Xi^0=\{\xi\in \mathbb{Z}^{|C|\times |T|}_+ \mid \sum_{t \in \mathcal{T}}\xi_c^t\leq q\}$ for some $q\in \mathbb{Z}_+$.
Let $\xi, \tilde{\xi} \in \Xi^0$ and $(c, t) \in \mathcal{C} \times \mathcal{T}$ with $\xi_c^t>\tilde{\xi}_{c}^t$.
We consider three cases. In each case discussed below, the if-clauses of (A) and (B) in the definition of pseudo M$^\natural$-concavity$^+$ do not hold. Therefore, it suffices to show that the weak inequality in the definition holds. Recall that, for each $\xi \in \Xi^0$,
\begin{align*}
f^s(\xi)=\sum_{(c,t) \in \mathcal{C}\times \mathcal{T}} \min\{ \xi^t_c, r^t_c\}.
\end{align*}

\smallskip
\noindent
\emph{Case 1:}
Suppose $f^s(\xi)<f^s(\tilde \xi)$.

\smallskip
\noindent
\emph{Subcase 1-1:}
Suppose $r_c^t\geq \xi_c^t$. Then,
\begin{align*}
r_c^t\geq \min\{ \xi_c^t, r_c^t\}=\xi_c^t>\tilde \xi_c^t=\min\{ \tilde \xi_c^t, r_c^t\}.
\end{align*}
Together with $f^s(\xi)<f^s(\tilde \xi)$, there exists $(\corigin, t')\in \calc\times \calt$ such that
\begin{align*}
\min\{ \xi_{\corigin}^{t'}, r_{\corigin}^{t'}\}<\min\{ \tilde \xi_{\corigin}^{t'}, r_{\corigin}^{t'}\}\leq r_{\corigin}^{t'}.
\end{align*}
By the above two inequalities,
\begin{align*}
f^s(\xi)&=f^s(\xi-\chi_{c,t})+1=f^s(\xi-\chi_{c,t}+\chi_{\corigin,t'}), \text{ and } \\
f^s(\tilde \xi)&\leq f^s(\tilde \xi-\chi_{\corigin, t'})+1=f^s(\tilde \xi+\chi_{c, t}-\chi_{\corigin, t'}).
\end{align*}
It follows that pseudo M$^\natural$-concavity$^+$ is satisfied.

\smallskip
\noindent
\emph{Subcase 1-2:}
Suppose $r_c^t<\xi_c^t$.

\smallskip
\noindent
\emph{Subcase 1-2-1:}
Suppose $\sum_{(\tilde c, \tilde t)\in \mathcal{C}\times \mathcal{T}}\tilde \xi_{c}^{\tilde t}<q$, which implies $\tilde \xi+\chi_{c,t}\in \Xi^0$.
By $r_c^t<\xi_c^t$,
\begin{align*}
f^s(\xi)=f^s(\xi-\chi_{c, t}).
\end{align*}
By the definition of $f^s(\cdot)$,
\begin{align*}
f^s(\tilde \xi)\leq f^s(\tilde \xi+\chi_{c,t}).
\end{align*}
It follows that pseudo M$^\natural$-concavity$^+$ is satisfied.

\smallskip
\noindent
\emph{Subcase 1-2-2:}
Suppose $\sum_{(\tilde c, \tilde t)\in \mathcal{C}\times \mathcal{T}}\tilde \xi_{\tilde c}^{\tilde t}=q$.
By $\xi_c^t>\tilde \xi_c^t$ and $\sum_{(\tilde c,\tilde t)\in \mathcal{C}\times \mathcal{T}}\xi_{\tilde c}^{\tilde t}\leq q=\sum_{(\tilde c, \tilde t)\in \mathcal{C}\times \mathcal{T}}\tilde \xi_{\tilde c}^{\tilde t}$, there exists $(\corigin,t')\in \mathcal{C}\times \mathcal{T}$ such that $\xi_{\corigin}^{t'}<\tilde \xi_{\corigin}^{t'}$. If $r_{\corigin}^{t'}<\tilde \xi_{\corigin}^{t'}$, then together with $r_c^t<\xi_c^t$,
\begin{align*}
&f^s(\xi)=f^s(\xi-\chi_{c,t})\leq f^s(\xi-\chi_{c,t}+\chi_{\corigin,t'}), \text{ and } \\
&f^s(\tilde \xi)=f^s(\tilde \xi-\chi_{\corigin,t'})\leq f^s(\tilde \xi+\chi_{c,t}-\chi_{\corigin,t'}).
\end{align*}
It follows that pseudo M$^\natural$-concavity$^+$ is satisfied. If $r_{\corigin}^{t'}\geq \tilde \xi_{\corigin}^{t'}(>\xi_{\corigin}^{t'})$, then together with $r_c^t<\xi_c^t$,
\begin{align*}
&f^s(\xi)=f^s(\xi-\chi_{c,t})< f^s(\xi-\chi_{c,t}+\chi_{\corigin,t'}), \text{ and } \\
&f^s(\tilde \xi)-1=f^s(\tilde \xi-\chi_{\corigin,t'})\leq f^s(\tilde \xi+\chi_{c,t}-\chi_{\corigin,t'}).
\end{align*}
By the assumption of Case 1, $\min\{f^s(\xi), f^s(\tilde \xi)-1\}=\min\{f^s(\xi), f^s(\tilde \xi)\}$. Together with the above two inequalities, pseudo M$^\natural$-concavity$^+$ is satisfied.

\smallskip
\noindent
\emph{Case 2:}
Suppose $f^s(\xi)=f^s(\tilde \xi)$.\footnote{The proofs for Subcases 2-1 and 2-2-1 are similar to those for Subcases 1-1 and 1-2-1, respectively.}

\smallskip
\noindent
\emph{Subcase 2-1:}
Suppose $r_c^t\geq \xi_c^t$. Then,
\begin{align*}
r_c^t\geq \min\{ \xi_c^t, r_c^t\}=\xi_c^t>\tilde \xi_c^t=\min\{ \tilde \xi_c^t, r_c^t\}.
\end{align*}
Together with $f^s(\xi)=f^s(\tilde \xi)$, there exists $(\corigin,t')\in \calc\times \calt$ such that
\begin{align*}
\min\{ \xi_{\corigin}^{t'}, r_{\corigin}^{t'}\}<\min\{ \tilde \xi_{\corigin}^{t'}, r_{\corigin}^{t'}\}\leq r_{\corigin}^{t'}.
\end{align*}
By the above two inequalities,
\begin{align*}
&f^s(\xi)=f^s(\xi-\chi_{c,t})+1=f^s(\xi-\chi_{c,t}+\chi_{\corigin, t'}), \text{ and } \\
&f^s(\tilde \xi)\leq f^s(\tilde \xi-\chi_{\corigin, t'})+1=f^s(\tilde \xi+\chi_{c,t}-\chi_{\corigin, t'}).
\end{align*}
It follows that pseudo M$^\natural$-concavity$^+$ is satisfied.

\smallskip
\noindent
\emph{Subcase 2-2:}
Suppose $r_c^t<\xi_c^t$.

\smallskip
\noindent
\emph{Subcase 2-2-1:}
Suppose $\sum_{(\tilde c,\tilde t)\in \mathcal{C}\times \mathcal{T}}\tilde \xi_{\tilde c}^{\tilde t}<q$, which implies $\tilde \xi+\chi_{c,t}\in \Xi^0$. By $r_c^t<\xi_c^t$,
\begin{align*}
f^s(\xi)=f^s(\xi-\chi_{c,t}).
\end{align*}
By the definition of $f^s(\cdot)$,
\begin{align*}
f^s(\tilde \xi)\leq f^s(\tilde \xi+\chi_{c,t}).
\end{align*}
It follows that pseudo M$^\natural$-concavity$^+$ is satisfied.

\smallskip
\noindent
\emph{Subcase 2-2-2:}
Suppose $\sum_{(\tilde c,\tilde t)\in \mathcal{C}\times \mathcal{T}}\tilde \xi_{\tilde c}^{\tilde t}=q$. Let $\Phi=\{(\tilde c, \tilde t)\in \mathcal{C}\times \mathcal{T} \mid \xi_{\tilde c}^{\tilde t}<\tilde \xi_{\tilde c}^{\tilde t}\}$. By $\xi_c^t>\tilde \xi_c^t$ and $\sum_{(\tilde c,\tilde t)\in \mathcal{C}\times \mathcal{T}}\xi_{\tilde c}^{\tilde t}\leq q=\sum_{(\tilde c,\tilde t)\in \mathcal{C}\times \mathcal{T}}\tilde \xi_{\tilde c}^{\tilde t}$, we have $\Phi \neq \emptyset$.

Suppose, for contradiction, that $r_{\tilde c}^{\tilde t}\geq \tilde \xi_{\tilde c}^{\tilde t}$ for each $(\tilde c, \tilde t)\in \Phi$.
Then,
\begin{align*}
f^s(\xi)-f^s(\tilde \xi)&=\sum_{(\tilde c, \tilde t)\in \mathcal{C}\times \mathcal{T}}\min\{ \xi_{\tilde c}^{\tilde t}, r_{\tilde c}^{\tilde t}\} -\sum_{(\tilde c, \tilde t)\in \mathcal{C}\times \mathcal{T}}\min\{ \tilde \xi_{\tilde c}^{\tilde t}, r_{\tilde c}^{\tilde t}\} \\
&=\sum_{(\tilde c, \tilde t)\in \Phi}\min\{ \xi_{\tilde c}^{\tilde t}, r_{\corigin}^{\tilde t}\} -\sum_{(\tilde c, \tilde t)\in \Phi}\min\{ \tilde \xi_{\tilde c}^{\tilde t}, r_{\tilde c}^{\tilde t}\} \\
&+\sum_{(\tilde c, \tilde t)\in (\mathcal{C}\times \mathcal{T})\backslash \Phi}\min\{ \xi_{\tilde c}^{\tilde t}, r_{\tilde c}^{\tilde t}\} -\sum_{(\tilde c, \tilde t)\in (\mathcal{C}\times \mathcal{T})\backslash \Phi}\min\{ \tilde \xi_{\tilde c}^{\tilde t}, r_{\tilde c}^{\tilde t}\} \\
&=\sum_{(\tilde c, \tilde t)\in \Phi} \xi_{\tilde c}^{\tilde t} -\sum_{(\tilde c, \tilde t)\in \Phi}\tilde \xi_{\tilde c}^{\tilde t} \\
&+\sum_{(\tilde c, \tilde t)\in (\mathcal{C}\times \mathcal{T})\backslash \Phi}\Bigl(\min\{ \xi_{\tilde c}^{\tilde t}, r_{\tilde c}^{\tilde t}\} -\min\{ \tilde \xi_{\tilde c}^{\tilde t}, r_{\tilde c}^{\tilde t}\}\Bigr) \\
&<\sum_{(\tilde c, \tilde t)\in \Phi} \xi_{\tilde c}^{\tilde t} -\sum_{(\tilde c, \tilde t)\in \Phi}\tilde \xi_{\tilde c}^{\tilde t} \\
&+\sum_{(\tilde c, \tilde t)\in (\mathcal{C}\times \mathcal{T})\backslash \Phi} \Bigl(\xi_{\tilde c}^{\tilde t}-\tilde \xi_{\tilde c}^{\tilde t}\Bigr)  \\
&\leq 0,
\end{align*}
where the third equality follows from the assumption made for contradiction and the definition of $\Phi$, the strict inequality follows from $\min\{ \xi_{\tilde c}^{\tilde t}, r_{\tilde c}^{\tilde t}\} -\min\{ \tilde \xi_{\tilde c}^{\tilde t}, r_{\tilde c}^{\tilde t}\}\leq \xi_{\tilde c}^{\tilde t}-\tilde \xi_{\tilde c}^{\tilde t}$ for
each $(\tilde c, \tilde t)\in (\mathcal{C}\times \mathcal{T})\backslash \Phi$ and $\min\{ \xi_{c}^{t}, r_{c}^{t}\} -\min\{ \tilde \xi_{c}^{t}, r_{c}^{t}\}<\xi_{c}^{t}-\tilde \xi_{c}^{t}$ for
$(c,t)\in (\mathcal{C}\times \mathcal{T})\backslash \Phi$ (where the latter strict inequality follows from $\xi_c^t>\tilde \xi_c^t$ and $r_c^t<\xi_c^t$), and the last inequality follows from
the assumption of Subcase 2-2-2. We obtain a contradiction to the assumption of Case 2.

It follows that there exists $(\corigin,t')\in \Phi$ with $r_{\corigin}^{t'}<\tilde \xi_{\corigin}^{t'}$.
Together with $r_c^t<\xi_c^t$,
\begin{align*}
&f^s(\xi)=f^s(\xi-\chi_{c,t})\leq f^s(\xi-\chi_{c,t}+\chi_{\corigin,t'}), \text{ and } \\
&f^s(\tilde \xi)=f^s(\tilde \xi-\chi_{\corigin,t'})\leq f^s(\tilde \xi+\chi_{c,t}-\chi_{\corigin,t'}).
\end{align*}
It follows that pseudo M$^\natural$-concavity$^+$ is satisfied.

\smallskip
\noindent
\emph{Case 3:}
Suppose $f^s(\xi)>f^s(\tilde \xi)$.

\smallskip
\noindent
\emph{Subcase 3-1:}
Suppose $r_c^t\geq \xi_c^t$.

\smallskip
\noindent
\emph{Subcase 3-1-1:}
Suppose $\sum_{(\tilde c,\tilde t)\in \mathcal{C}\times \mathcal{T}}\tilde \xi_{\tilde c}^{\tilde t}<q$, which implies $\tilde \xi+\chi_{c,t}\in \Xi^0$.
By $r_c^t\geq \xi_c^t$,
\begin{align*}
f^s(\xi)-1=f^s(\xi-\chi_{c,t}).
\end{align*}
By $r_c^t\geq \xi_c^t>\tilde \xi_c^t$,
\begin{align*}
f^s(\tilde \xi)<f^s(\tilde \xi+\chi_{c,t}).
\end{align*}
By the assumption of Case 3, $\min\{f^s(\xi)-1, f^s(\tilde \xi)\}=\min\{f^s(\xi), f^s(\tilde \xi)\}$. Together with the above displayed equality and inequality, pseudo M$^\natural$-concavity$^+$ is satisfied.

\smallskip
\noindent
\emph{Subcase 3-1-2:}
Suppose $\sum_{(\tilde c,\tilde t)\in \mathcal{C}\times \mathcal{T}}\tilde \xi_{\tilde c}^{\tilde t}=q$. By $\xi_c^t>\tilde \xi_c^t$ and $\sum_{(\tilde c,\tilde t)\in \mathcal{C}\times \mathcal{T}}\xi_{\tilde c}^{\tilde t} \leq q=\sum_{(\tilde c, \tilde t)\in \mathcal{C}\times \mathcal{T}}\tilde \xi_{\tilde c}^{\tilde t}$, there exists $(\corigin,t')\in \mathcal{C}\times \mathcal{T}$ such that $\xi_{\corigin}^{t'}<\tilde \xi_{\corigin}^{t'}$. If
$r_{\corigin}^{t'}<\tilde \xi_{\corigin}^{t'}$, then together with $r_c^t\geq \xi_c^t>\tilde \xi_c^t$,
\begin{align*}
&f^s(\xi)-1=f^s(\xi-\chi_{c,t})\leq f^s(\xi-\chi_{c,t}+\chi_{\corigin,t'}), \text{ and } \\
&f^s(\tilde \xi)=f^s(\tilde \xi-\chi_{\corigin,t'})<f^s(\tilde \xi+\chi_{c,t}-\chi_{\corigin,t'}).
\end{align*}
By the assumption of Case 3, $\min\{f^s(\xi)-1, f^s(\tilde \xi)\}=\min\{f^s(\xi), f^s(\tilde \xi)\}$. Together with the above inequalities, pseudo M$^\natural$-concavity$^+$ is satisfied.
If $r_{\corigin}^{t'}\geq \tilde \xi_{\corigin}^{t'}(>\xi_{\corigin}^{t'})$, together with $r_c^t\geq \xi_c^t>\tilde \xi_c^t$,
\begin{align*}
&f^s(\xi)=f^s(\xi-\chi_{c,t})+1=f^s(\xi-\chi_{c,t}+\chi_{\corigin,t'}), \text{ and }\\
&f^s(\tilde \xi)=f^s(\tilde \xi-\chi_{\corigin,t'})+1=f^s(\tilde \xi+\chi_{c,t}-\chi_{\corigin,t'}).
\end{align*}
It follows that pseudo M$^\natural$-concavity$^+$ is satisfied.

\smallskip
\noindent
\emph{Subcase 3-2:}
Suppose $r_c^t<\xi_c^t$.

\smallskip
\noindent
\emph{Subcase 3-2-1:}
Suppose $\sum_{(\tilde c,\tilde t)\in \mathcal{C}\times \mathcal{T}}\tilde \xi_{\tilde c}^{\tilde t}<q$, which implies $\tilde \xi+\chi_{c,t}\in \Xi^0$. By $r_c^t<\xi_c^t$,
\begin{align*}
f^s(\xi)=f^s(\xi-\chi_{c,t}).
\end{align*}
By the definition of $f^s(\cdot)$,
\begin{align*}
f^s(\tilde \xi)\leq f^s(\tilde \xi+\chi_{c,t}).
\end{align*}
It follows that pseudo M$^\natural$-concavity$^+$ is satisfied.

\smallskip
\noindent
\emph{Subcase 3-2-2:}
Suppose $\sum_{(\tilde c,\tilde t)\in \mathcal{C}\times \mathcal{T}}\tilde \xi_{\tilde c}^{\tilde t}=q$. Let $\Phi=\{(\tilde c, \tilde t)\in \mathcal{C}\times \mathcal{T} \mid \xi_{\tilde c}^{\tilde t}<\tilde \xi_{\tilde c}^{\tilde t}\}$. By $\xi_c^t>\tilde \xi_c^t$ and $\sum_{(\tilde c,\tilde t)\in \mathcal{C}\times \mathcal{T}}\xi_{\tilde c}^{\tilde t}\leq q=\sum_{(\tilde c,\tilde t)\in \mathcal{C}\times \mathcal{T}}\tilde \xi_{\tilde c}^{\tilde t}$, we have $\Phi \neq \emptyset$.

Suppose, for contradiction, that $r_{\tilde c}^{\tilde t}\geq \tilde \xi_{\tilde c}^{\tilde t}$ for each
$(\tilde c, \tilde t)\in \Phi$. Then, by following the same line of argument as in Subcase 2-2-2, we obtain $f^s(\xi)<f^s(\tilde \xi)$, which is a contradiction to the assumption of Subcase 3. It follows that there exists $(\corigin,t')\in \Phi$ with $r_{\corigin}^{t'}<\tilde \xi_{\corigin}^{t'}$. Together with $r_c^t<\xi_c^t$,
\begin{align*}
&f^s(\xi)=f^s(\xi-\chi_{c,t})\leq f^s(\xi-\chi_{c,t}+\chi_{\corigin,t'}), \text{ and } \\
&f^s(\tilde \xi)=f^s(\tilde \xi-\chi_{\corigin,t'})\leq f^s(\tilde \xi+\chi_{c,t}-\chi_{\corigin,t'}).
\end{align*}
We conclude that pseudo M$^\natural$-concavity$^+$ is satisfied.
\qed


\medskip
\noindent
\emph{Counterexample to Claim \ref{claim:ex2-pMC} when $\Xi^0$ is not given as in the statement:}
Let $\mathcal{C}=\{c,c'\}$ and $\mathcal{T}=\{t,t'\}$. Suppose that each school's capacity is given by $q_c=2$ and $q_{c'}=1$, i.e.,
\begin{align*}
\Xi^0=\Bigl\{\xi\in \mathbb{Z}^{|\mathcal{C}|\times |\mathcal{T}|}_+ \mid \sum_{t\in T}\xi_c^t\leq 2, \sum_{t\in T}\xi_{c'}^t\leq 1\Bigr\}.
\end{align*}
The number of reserved seats is given by $r_c^t=1$, $r_c^{t'}=1$, $r_{c'}^t=1$, and $r_{c'}^{t'}=0$. Let $\xi, \tilde \xi\in \Xi^0$ be such that
\begin{align*}
&\xi_c^t=2, \: \xi_c^{t'}=0, \: \xi_{c'}^t=1, \: \xi_{c'}^{t'}=0, \\
&\tilde \xi_c^t=1, \: \tilde \xi_c^{t'}=1, \: \tilde \xi_{c'}^t=0, \: \tilde \xi_{c'}^{t'}=0.
\end{align*}
For $(c,t)$, the only candidate of $(c'',t'')\in (\mathcal{C}\times \mathcal{T})\cup\{\emptyset\}$ with $\tilde \xi+\chi_{c,t}-\chi_{c'',t''}\in \Xi^0$ is $(c'',t'')=(c,t')$ (otherwise the capacity constraint for school $c$ is violated at $\tilde \xi+\chi_{c,t}-\chi_{c'',t''}$). Then,
\begin{align*}
&\min\{f^s(\xi), f^s(\tilde \xi)\}=\min\{2,2\}=2, \text{ and } \\
&f^s(\tilde \xi)=2>1=f^s(\tilde \xi+\chi_{c,t}-\chi_{c,t'}).
\end{align*}
It follows that $f^s$ violates pseudo M$^\natural$-concavity, and hence violates pseudo M$^\natural$-concavity$^+$.

%==================================================

\subsection*{Proof of Proposition \ref{prop:comparison}}
%\begin{proof}[Proof of Proposition \ref{prop:comparison}]
First we show that M$^{\natural}$-concavity implies ordinal concavity.

Let $\xi,\tilde{\xi}\in \Xi^0$ and $(c,t) \in \calc \times \calt$ be such that   $\xi_c^t>\tilde{\xi}_c^t$. Then, by M$^{\natural}$-concavity, one of
conditions (i) and (ii) in Definition \ref{def:natural} holds.

Suppose that condition (i) in Definition \ref{def:natural} holds. If
condition (i) or (ii) in Definition \ref{def:ordinal} holds, then ordinal concavity
is satisfied. If conditions (i) and (ii) in Definition \ref{def:ordinal}
do not hold, then we have
\[f(\xi-\chi_{c,t})\leq f(\xi) \mbox{ and } f(\tilde \xi+\chi_{c,t}) \leq f(\tilde \xi).\]
These two inequalities together with condition (i) in Definition \ref{def:natural}
imply that
\[f(\xi-\chi_{c,t})=f(\xi) \mbox{ and } f(\tilde \xi+\chi_{c,t}) = f(\tilde \xi),\]
which is condition (iii) in Definition of \ref{def:ordinal}, so ordinal concavity is satisfied.

Suppose that condition (i) in Definition \ref{def:natural} does not hold. By
M$^{\natural}$-concavity, condition (ii) in Definition \ref{def:natural} holds.
Therefore, there exists $(c',t') \in \calc \times \calt$ with  $\xi_{c'}^{t'}<\tilde{\xi}_{c'}^{t'}$ such that
\[f(\xi-\chi_{c,t}+\chi_{c',t'})+f(\tilde{\xi}+\chi_{c,t}-\chi_{c',t'})
\geq f(\xi)+f(\tilde \xi).\]
If condition (i) or (ii) in Definition \ref{def:ordinal} holds, then ordinal concavity
is satisfied. If conditions (i) and (ii) in Definition \ref{def:ordinal}
do not hold, then we have
\[f(\xi-\chi_{c,t}+\chi_{c',t'})\leq f(\xi) \mbox{ and }
f(\tilde \xi+\chi_{c,t}-\chi_{c',t'}) \leq f(\tilde \xi).\]
These two inequalities together with condition (ii) in Definition \ref{def:natural}
imply that
\[f(\xi-\chi_{c,t}+\chi_{c',t'})=f(\xi) \mbox{ and } f(\tilde \xi+\chi_{c,t}-\chi_{c',t'}) = f(\tilde \xi),\]
which is condition (iii) in Definition of \ref{def:ordinal}, so ordinal concavity is satisfied.

Now we provide a function that satisfies ordinal concavity but not M$^{\natural}$-concavity. Let the diversity index $f$ be defined as $f(0)=0$, $f(1)=3$, and
$f(2)=10$. Since it is strictly increasing it is ordinally concave because
condition (ii) in Definition of \ref{def:ordinal} is satisfied. However,
M$^{\natural}$ concavity fails because for $\xi=2$, $\tilde \xi=0$, and $\chi=1$
we have $f(\xi-\chi)+f(\tilde{\xi}+\chi)=6 < 10=f(\xi)+f(\tilde \xi)$.
\qed
%\end{proof}




\end{document}


