\documentclass[a4paper,11pt]{article}
\usepackage{amsmath,amsthm,amsfonts,amssymb,graphicx,float,wrapfig,calc,microtype,thmtools,underscore}
\usepackage[T1]{fontenc}
\usepackage[english]{babel}
\usepackage[usenames,dvipsnames,svgnames,table]{xcolor}

\usepackage{imakeidx}
\newcommand{\DefNoIndex}[1]{\textcolor{Maroon}{\emph{#1}}}
\newcommand{\defn}[1]{\textcolor{Maroon}{\emph{#1}}\index{#1}}
\newcommand{\hdefn}[2]{\textcolor{Maroon}{\emph{#1-#2}}\index{#2@#1-#2}}
\newcommand{\DefnIndex}[2]{\textcolor{Maroon}{\emph{#1}}\index{#2}}
\newcommand{\word}[1]{#1\index{#1}}
\makeindex

\usepackage[unicode=true]{hyperref}
\hypersetup{
colorlinks,
linkcolor={black},
citecolor={black},
urlcolor={blue!60!black},
pdftitle={Universality in Minor-Closed Graph Classes},
pdfauthor={Huynh -- Mohar -- Samal -- Thomassen -- Wood}}

\usepackage[shortlabels]{enumitem}
\setlist[itemize]{topsep=0ex,itemsep=0ex,parsep=0ex}
\setlist[enumerate]{topsep=0ex,itemsep=0ex,parsep=0ex}

\usepackage[longnamesfirst,numbers,sort&compress]{natbib}
\makeatletter
\def\NAT@spacechar{~}
\makeatother

\usepackage[lmargin=29mm,rmargin=29mm,tmargin=29mm,bmargin=29mm]{geometry}
\renewcommand{\baselinestretch}{1.1}
\setlength{\footnotesep}{\baselinestretch\footnotesep}
\setlength{\parindent}{0cm}
\setlength{\parskip}{1.5ex}
\allowdisplaybreaks

\usepackage[noabbrev,capitalise]{cleveref}
\crefname{lem}{Lemma}{Lemmas}
\crefname{thm}{Theorem}{Theorems}
\crefname{cor}{Corollary}{Corollaries}
\crefname{prop}{Proposition}{Propositions}
\crefname{conj}{Conjecture}{Conjectures}
\crefname{open}{Open Problem}{Open Problems}
\crefname{claim}{Claim}{Claims}
\crefname{enumi}{Item}{Items}
\crefformat{equation}{(#2#1#3)}
\Crefformat{equation}{Equation #2(#1)#3}
\crefformat{enumi}{#2#1#3}
\Crefformat{enumi}{Item (#2#1#3)}

\theoremstyle{plain}
\newtheorem{thm}{Theorem}[section]
\newtheorem{lem}[thm]{Lemma}
\newtheorem{cor}[thm]{Corollary}
\newtheorem{prop}[thm]{Proposition}
\newtheorem{claim}{Claim}[thm]
\theoremstyle{definition}
\newtheorem{conj}[thm]{Conjecture}
\newtheorem{ques}[thm]{Question}

% define \reallywidehat
\usepackage{scalerel,stackengine}
\stackMath
\newcommand\reallywidehat[1]{%
\savestack{\tmpbox}{\stretchto{%
  \scaleto{%
    \scalerel*[\widthof{\ensuremath{#1}}]{\kern.1pt\mathchar"0362\kern.1pt}%
    {\rule{0ex}{\textheight}}%
  }{\textheight}% 
}{2.4ex}}%
\stackon[-6.9pt]{#1}{\tmpbox}%
}

%\newcommand\subiso{\mathrel{\ooalign{\raise0.2ex\hbox{$\subset$}\cr\hidewidth\raise-0.8ex\hbox{\scalebox{0.9}{$\sim$}}\hidewidth\cr}}}

\newcommand{\CartProd}{\mathbin{\square}}
\newcommand{\half}{\ensuremath{\protect\tfrac{1}{2}}}
\newcommand{\T}{\ensuremath{\protect{\mathcal{T}}}}
\renewcommand{\G}{\ensuremath{\protect{\mathcal{G}}}}
\renewcommand{\ge}{\geqslant}
\renewcommand{\le}{\leqslant}
\renewcommand{\geq}{\geqslant}
\renewcommand{\leq}{\leqslant}

\DeclareMathOperator{\tw}{tw}
\DeclareMathOperator{\bw}{bw}
\DeclareMathOperator{\stw}{stw}
\DeclareMathOperator{\pw}{pw}
\DeclareMathOperator{\dist}{dist}
\DeclareMathOperator{\col}{col}
\newcommand{\sreach}{S}

\renewcommand{\thefootnote}{\fnsymbol{footnote}}	
\allowdisplaybreaks

\newcommand{\david}[1]{\textcolor{red}{[David: #1]}}
\newcommand{\tony}[1]{\textcolor{orange}{[Tony: #1]}}
\newcommand{\bojan}[1]{\textcolor{blue}{[Bojan: #1]}}
\newcommand{\robert}[1]{\textcolor{brown}{[Robert: #1]}}
\newcommand{\carsten}[1]{\textcolor{magenta}{[Carsten: #1]}}

% infinite paths
\newcommand{\PP}{P_{\hspace{-0.25ex}\infty}}
\newcommand{\PPP}{C_{\hspace{-0.15ex}\infty}}

\newcommand{\LL}{\mathcal{L}}
\newcommand{\FF}{\mathcal{F}}
\newcommand{\XX}{\mathcal{X}}
\newcommand{\QQ}{\mathcal{Q}}
\newcommand{\CC}{\mathcal{C}}
\newcommand{\YY}{\mathcal{Y}}
\newcommand{\GG}{\mathcal{G}}
\newcommand{\HH}{\mathcal{H}}
\newcommand{\JJ}{\mathcal{J}}
\newcommand{\DD}{\mathcal{D}}
\newcommand{\TT}{\mathcal{T}}
\newcommand{\RR}{\mathcal{R}}
\newcommand{\UU}{\mathcal{U}}
\newcommand{\NN}{\mathbb{N}}
\newcommand{\ZZ}{\mathbb{Z}}
\renewcommand{\SS}{\mathcal{S}}
\newcommand{\PART}{\mathcal{P}}
\newcommand{\GGG}[2]{#1\langle{#2}\rangle}
%\newcommand{\GGGG}[2]{\widetilde{\GGG{#1}{#2}}}

\begin{document}

\author{
Tony Huynh\,\footnotemark[3] \qquad
Bojan Mohar\,\footnotemark[4] \qquad
Robert \v{S}\'amal\,\footnotemark[5] \\
Carsten Thomassen\,\footnotemark[2] \qquad 
David~R.~Wood\,\footnotemark[3]}

\footnotetext[2]{Department of Applied Mathematics and Computer Science, Technical University of Denmark, Lyngby, Denmark (\texttt{ctho@dtu.dk}). Research supported by Independent Research Fund Denmark, 8021-002498 AlgoGraph.}
 
\footnotetext[3]{School of Mathematics, Monash University, Melbourne, Australia  (\texttt{\{tony.huynh2,david.wood\}@monash.edu}). Research supported by the Australian Research Council.}

\footnotetext[4]{Department  of Mathematics, Simon Fraser University, Burnaby, BC, Canada  (\texttt{mohar@sfu.ca}). Research supported in part by the NSERC Discovery Grant R611450 (Canada).}

\footnotetext[5]{Computer Science Institute, Charles University, Prague, Czech Republic  (\texttt{samal@iuuk.mff.cuni.cz}). Partially supported by grant 19-21082S of the Czech Science Foundation. This project has received funding from the European Research Council (ERC) under the European Union’s Horizon 2020 research and innovation programme (grant agreement No 810115). This project has received funding from the European Union’s Horizon 2020 research and innovation programme under the Marie Sk{\l}odowska-Curie grant agreement No 823748.}

\sloppy

\title{\boldmath\bf Universality in Minor-Closed Graph Classes}
\maketitle

\footnotetext{\textbf{MSC Classification}: 05C10 planar graphs, 05C83 graph minors, 05C63 infinite graphs}

\begin{abstract}
Stanis{\l}aw Ulam asked whether there exists a universal countable planar graph (that is, a countable planar graph that contains every countable planar graph as a subgraph). J\'anos Pach~(1981) answered this  question in the negative. We strengthen this result by showing that every countable graph that contains all countable planar graphs must contain (i) an infinite complete graph as a minor, and (ii) a subdivision of the complete graph $K_t$ with multiplicity $t$, for every finite $t$. 

On the other hand, we construct a countable graph that contains all countable planar graphs and has several key properties such as linear colouring numbers, linear expansion, and every finite $n$-vertex subgraph has a balanced separator of size $O(\sqrt{n})$. The graph is $\TT_6\boxtimes P_{\!\infty}$, where $\TT_k$ is the universal treewidth-$k$ countable graph (which we define explicitly), $P_{\!\infty}$ is the 1-way infinite path, and $\boxtimes$ denotes the strong product. 
More generally, for every positive integer $t$ we construct a countable graph that contains every countable $K_t$-minor-free graph and has the above key properties. 

Our final contribution is a construction of a countable graph that contains every countable $K_t$-minor-free graph as an induced subgraph, has linear colouring numbers and linear expansion, and contains no subdivision of the countably infinite complete graph (implying (ii) above is best possible). 
\end{abstract}

\renewcommand{\thefootnote}{\arabic{footnote}}

\newpage
\renewcommand{\baselinestretch}{0}\normalsize
\tableofcontents
\renewcommand{\baselinestretch}{1.1}\normalsize
\newpage

\section{Introduction}
\label{Intro}

A graph\footnote{Graphs in this paper are simple and either finite or countably infinite, unless explicitly stated otherwise.} is \defn{planar} if it has a drawing in the Euclidean plane (or equivalently on the sphere) with no edge-crossings. Planar graphs are of broad importance in graph theory. The 4-Colour Theorem for colouring  maps~\citep{AH89,RSST97} can be restated as every planar graph is 4-colourable. The 4-Colour Conjecture (as it was) motivated the development of much of 20th century graph theory, and its proof was one of the first examples of a computer-based mathematical proof. The Kuratowski--Wagner Theorem~\citep{Kuratowski30,Wagner37} characterises planar graphs as \begin{wrapfigure}{r}{56mm}
\centering
{\vspace*{0.2ex}\includegraphics{Icosahedron}\vspace*{-1.1ex}}
\caption{The icosahedron graph.}
\vspace*{-3ex}
\end{wrapfigure}
those graphs not containing the complete graph $K_5$ or the complete bipartite graph $K_{3,3}$ as a minor. More generally, the Graph Minor Theorem of \citet{RS-GraphMinors} says that any proper minor-closed class of graphs can be characterised by a finite set of excluded minors. Planar graphs are foundational in Robertson and Seymour's Graph Minor Structure Theorem, which shows that graphs in any proper minor-closed class can be constructed using four types of ingredients, where planar graphs are the most basic building block.
 
 
Planar graphs also arise in many areas of mathematics outside of graph theory. The Koebe disc packing theorem \citep{Koebe36}, which says that every planar graph can be represented by touching discs in the plane, is the beginning of a rich theory of conformal mappings~\citep{HeSc93,Stephenson05}. Steinitz's Theorem~\citep{Steinitz22} connects planar graphs and classical geometry: the 1-skeletons of polyhedra are exactly the 3-connected planar graphs. Knot diagrams represent knots as planar graphs equipped with over/under relations~\citep{CG12,KK14}. Embeddings of graphs on surfaces, especially triangulations, are fundamental objects in hyperbolic geometry~\citep{STT-JAMS88,CKP19}. Planar graphs also have applications in topology. For example, \citet{Thomassen89a,Thomassen90} presented a simple proof of  the Jordan Curve Theorem based on planar graphs. Planar graphs also arise in physics. Quantum field theory studies Feynman diagrams, which are graphs describing particle interactions. Planar Feynman diagrams \citep{MM19,BBBD15} lead to notions such as the `planar limit' of quantum field theories. Indeed, there is a beautiful theory of `on-shell diagrams' which says that scattering amplitudes in $N=4$ supersymmetric Yang-Mills theory can be calculated using planar graphs~\citep{GK16}. Planar graphs also arise in computational geometry~\citep{GOT18}, for example as Delaunay triangulations of point sets, which leads to  practical applications such as computer graphics and 3D printing. They are central in graph drawing research \citep{HandbookGraphDrawing}, which underlies the field of network visualisation. 

%Curvature, geometry and spectral properties of planar graphs \citep{Keller11}

The focus of this paper is on infinite graphs, which are ubiquitous mathematical objects. For example, they are important in random walks and percolation~\citep{Grimmett99}, where it is desirable to avoid finite-size boundary effects. They also arise as group diagrams in combinatorial group theory \citep{LyndonSchupp77} and low-dimensional topology~\citep{Gersten83,Stallings83}. Bass--Serre theory analyses the algebraic structure of groups by considering the action of automorphisms on infinite trees~\citep{Serre77,Serre80,Bass93}. Several group-theoretic results can be proven by considering a group action on infinite graphs. For example, there is a simple proof that every subgroup of a free group is free using infinite graphs and covering spaces~\citep{NielsenSchreierTheorem}. Infinite Cayley graphs are central objects in geometric group theory \citep{delaHarpe00}. Infinite planar graphs play a key role in the theory of patterns and tilings~\citep{GS87}. Infinite graphs also arise in nanotechnology  \citep{FHI11} and as models of the world wide web~\citep{Bonato08}. In addition, numerous results for finite graphs have led to extensions for infinite graphs or to interesting open problems. See \citep{Thom83,Komjath11,NashWilliams67,Diestel90} for surveys on infinite graphs. 

%Terry Tao writes, ``In the converse direction, one can view infinite graphs as a discretisation of continuous spaces (and infinite Cayley graphs as a discretisation of homogeneous spaces). Gromov's original proof of his theorem relies on this perspective (or more precisely, the idea that homogeneous spaces can arise as limits of infinite Cayley graphs). So the discrete infinitary theory of infinite graphs form a nice bridge between the discrete finitary world and the continuous infinitary world'' \citep{...}. 

\subsubsection*{Universality for Planar Graphs}

This paper addresses universality questions for planar graphs and other more general classes. A graph $G$ \defn{contains} a graph $H$ if $H$ is isomorphic to some subgraph of $G$. A graph $U$ is \defn{universal} for a graph class $\GG$ if $U\in\GG$ and $U$ contains every graph in $\GG$. 
The starting point for this work is the following question of Ulam: 

%\footnote{Note that the disjoint union of all finite planar graphs is a countable planar graph. So the question of universality of finite planar graphs is not interesting. Similarly, the disjoint union of all countable planar graphs is an uncountable graph that contains all countable planar graphs, which is again not interesting. This says it is important to only consider countable universal graphs.}: 

\emph{Is there a universal graph for the class of countable planar graphs?}

This question was answered in the negative by \citet{Pach81a}, who proved that no countable planar graph contains every countable planar graph. Pach's result suggests the following question, which motivates the present paper. Recall that graphs in this paper are countable (that is, finite or countably infinite), unless explicitly stated otherwise. 

\emph{What is the `simplest' graph that contains every planar graph?}

Of course, the answer depends on one's measure of simplicity. First, it is desirable that a graph that  contains every planar graph has bounded chromatic number. By the 4-Colour Theorem~\citep{AH89,RSST97} and the de~Bruijn--Erd\H{o}s Theorem~\citep{dBE51}, the complete 4-partite graph\footnote{Let $K_n$ be the complete graph with $n$ vertices. Let $K_{m,n}$ be the complete bipartite graph with $m$ vertices in one colour class and $n$ vertices in the other colour class. Let $K_{n_1,\dots,n_k}$ be the complete $k$-partite graph with $n_i$ vertices in the $i$-th colour class.} with each colour class of cardinality $\aleph_0$  contains every planar graph. But this graph is very dense. At the other extreme, one might hope that some graph with bounded genus or excluding some fixed graph as a minor contains every planar graph. Our first theorem (proved in \cref{ForcingMinor}) dashes this hope and strengthens the above-mentioned result of \citet{Pach81a}.

\begin{thm}
\label{InfinteCliqueMinor}
If a (countable) graph $U$ contains every planar graph, then 
\begin{enumerate}[(a)]
\item the infinite complete graph $K_{\aleph_0}$ is a minor of $U$, and
\item $U$ contains a subdivision of $K_t$ for every $t\in\NN$. 
\end{enumerate}
\end{thm}

If a graph $U$ contains every planar graph, then \cref{InfinteCliqueMinor} says it is impossible for $U$ to  exclude a fixed minor or subdivision. However, it is desirable that $U$ satisfies many of the key properties of planar graphs. We contend that these properties include the following:
\begin{enumerate}[(K1)]
\item\label{KeyPropertyBoundedMinDegree} Every finite subgraph of $U$ has bounded minimum degree.
\item\label{KeyPropertySeparators}  Every finite $n$-vertex subgraph of $U$ has a balanced separator of order $O(\sqrt{n})$.
\item\label{KeyPropertyBoundedColouringNumbers} $U$ has bounded $r$-colouring numbers.
\item\label{KeyPropertyNoInfiniteSubdivision} $U$ contains no subdivision of $K_{\aleph_0}$.
\end{enumerate}

We now justify these key properties. \cref{KeyPropertyBoundedMinDegree} is the primary indicator of a sparse graph class, and implies several other desirable properties. In particular, if every finite subgraph of $U$ has minimum degree at most $d\in\NN$, then applying a greedy algorithm, every finite subgraph of $U$ is $(d+1)$-colourable, and in fact is $(d+1)$-list-colourable. 
By the de~Bruijn--Erd\H{o}s Theorem~\citep{dBE51}, $U$ itself is $(d+1)$-colourable, and is $(d+1)$-list-colourable by a similar argument. \cref{KeyPropertySeparators} implies the Lipton--Tarjan separator theorem \citep{LT79}, which is a seminal result about the structure of planar graphs. \cref{KeyPropertyBoundedColouringNumbers} is a robust measure of graph sparsity (defined and discussed in greater depth in \cref{GeneralisedColouringNumbers}). For now, note that \cref{KeyPropertyBoundedColouringNumbers} implies \cref{KeyPropertyBoundedMinDegree}, which in turn implies that $U$ has bounded chromatic number. Moreover, \cref{KeyPropertyBoundedColouringNumbers} implies that numerous other colouring parameters are bounded in $U$, including acyclic chromatic number, game chromatic number, etc. More generally, \cref{KeyPropertyBoundedColouringNumbers} implies that $U$ has bounded expansion, which is a key property in the Graph Sparsity Theory of \citet{Sparsity}. Finally, in light of \cref{InfinteCliqueMinor}(b), \cref{KeyPropertyNoInfiniteSubdivision} is important since planar graphs have no $K_5$ or $K_{3,3}$ subdivision. 

\subsubsection*{Product Constructions}

The second main contribution of this paper is to construct graphs that  contain every planar graph and satisfy \cref{KeyPropertyBoundedMinDegree}--\cref{KeyPropertyBoundedColouringNumbers}. To describe these graphs we need two ingredients. 

The first ingredient is the notion of treewidth, which is a parameter that measures how similar a given graph is to a tree (see \cref{SimplicialDecompositions} for the definition). For example, a connected graph with at least two vertices has treewidth 1 if and only if  it is a tree. Treewidth is of great importance in structural graph theory, especially in Robertson and Seymour's work on graph minors~\citep{RS-GraphMinors}. The Grid-Minor Theorem of \citet{RS-V} shows that treewidth is related to planarity in the sense that a minor-closed graph class $\GG$ has bounded treewidth if and only if some finite planar graph is not in $\GG$. In a sense detailed below, planar graphs are the smallest minor-closed class with a rich structure. Treewidth and planar graphs are also of great importance in algorithmic graph theory. In particular, many NP-complete problems remain NP-complete for finite planar graphs, but are polynomial-time solvable for classes of finite graphs with bounded treewidth (those excluding some planar graph as a minor). This shows that finite planar graphs often lie at the boundary of hard and easy instances of numerous algorithmic problems. 

It is well known that for each $k\in\NN$, there is a graph $\TT_k$ that is universal for the class of treewidth-$k$ graphs. \cref{Treewidth} gives an explicit definition of such a graph that is of independent interest and important for the main results that follow. 

The second ingredient is the notion of a graph product, as illustrated in \cref{ProductExample}. For graphs $G$ and $H$,  the \defn{Cartesian product} $G\CartProd H$ is the graph with vertex-set $V(G)\times V(H)$, where vertices $(v,w)$ and $(x,y)$ are adjacent if $v=x$ and $wy\in E(H)$, or $w=y$ and $vx \in E(G)$. The \defn{direct  product} $G\times H$ is the graph with vertex-set $V(G)\times V(H)$, where vertices $(v,w)$ and $(x,y)$ are adjacent if $vx\in E(G)$ and $wy\in E(H)$. Finally, the \defn{strong product} $G\boxtimes H$ is the union of $G\CartProd H$ and $G\times H$.

\begin{figure}[!ht]
\centering
\includegraphics{ProductExample}
\caption{Cartesian, direct and strong graph products.\label{ProductExample}}
\end{figure}

With these ingredients in hand, we prove the following, where $\PP$ is the 1-way infinite path.

\begin{thm}
\label{InfinitePlanarStructure}
Both $\TT_6\boxtimes\PP$ and $\TT_3\boxtimes\PP\boxtimes K_3$  contain every planar graph and satisfy \cref{KeyPropertyBoundedMinDegree}--\cref{KeyPropertyBoundedColouringNumbers}.
\end{thm}

We actually prove a strengthening of this result in terms of so-called `simple treewidth'. We define a graph $\SS_k$ that is universal for the class of graphs with simple treewidth $k$, and we show that $\TT$ can be replaced by $\SS$ in \cref{InfinitePlanarStructure}. This is noteworthy since $\SS_3$ is planar but $\TT_3$ is not. These results are proved in \cref{PlanarGraphs}. 

We also establish several results analogous to \cref{InfinitePlanarStructure} for more general graph classes. First, consider graphs embeddable on an arbitrary surface. The \defn{join} $A+B$ is the graph obtained from the disjoint union of graphs $A$ and $B$ by adding all edges $uv$ with $u\in V(A)$ and $v \in V(B)$. 
We show that for each $g\in\NN_0$, each of  
$(\SS_6+K_1) \boxtimes \PP \boxtimes K_{\max\{2g,1\}}$, 
$(\SS_3+K_1) \boxtimes \PP \boxtimes K_{\max\{2g,3\}}$ and
$(\SS_6 +K_{2g} )  \boxtimes \PP$ contain every graph of Euler genus $g$, and satisfy \cref{KeyPropertyBoundedMinDegree}--\cref{KeyPropertyBoundedColouringNumbers}. This result is proved in \cref{Genus}. We provide similar results for various non-minor-closed classes, such as graphs that can be drawn on a fixed surface with a bounded number of crossings per edge; see \cref{NonMinorClosed}. Moreover, for each finite graph $X$, we construct a graph that contains every $X$-minor-free graph. This graph is more complicated to describe than the above products, but it still satisfies \cref{KeyPropertyBoundedMinDegree}--\cref{KeyPropertyBoundedColouringNumbers}; see \cref{ExcludedMinors}. 

Graph products also provide a mechanism to construct 4-colourable  sparse graphs that contain every planar graph. To see this, note that $\chi(G\times H)\leq\min\{\chi(G),\chi(H)\}$ for all graphs $G$ and $H$. Also note that if $X$ is a $k$-colourable subgraph of $G$, then $X$ is isomorphic to a subgraph of $G\times K_k$. Together, these observations imply that if $U$ is any graph that contains every planar graph, then $U \times K_4$ is 4-colourable and contains every planar graph. For example,  by \cref{InfinitePlanarStructure}, $(\SS_6\boxtimes\PP) \times K_4$ is a 4-colourable graph that contains every planar graph. Moreover, $U\times K_4$ inherits many of the desirable properties of $U$ with a small change in the constants.

\subsubsection*{Avoiding an Infinite Complete Graph Subdivision}

Our final results are related to property \cref{KeyPropertyNoInfiniteSubdivision}, and are presented in \cref{NoInfiniteSubdivision}. Graphs containing no infinite complete graph subdivision were characterised in \citet{RST-TAMS92}. \cref{InfinteCliqueMinor}(b) says that arbitrarily large (finite) complete graph subdivisions are unavoidable in any graph that contains all planar graphs. On the other hand, we prove that infinite complete graph subdivisions are avoidable. In fact, this result holds for induced subgraphs and in the setting of $K_t$-minor free graphs.

\begin{thm}
\label{Main3} 
For each $t\in\NN$, there is a graph $U_t$ that contains every $K_t$-minor-free graph as an induced subgraph, and satisfies 
\cref{KeyPropertyBoundedMinDegree}, \cref{KeyPropertyBoundedColouringNumbers}
and \cref{KeyPropertyNoInfiniteSubdivision}.
\end{thm}

Since planar graphs contain no $K_5$ minor, the case $t=5$ in \cref{Main3} includes planar graphs. It is open whether $U_t$ satisfies \cref{KeyPropertySeparators}, although $n$-vertex subgraphs of $U_t$ do have balanced separators of size $O(n^{3/4}\log n)$ by a result of \citet{PRS94}; see \cref{Expansion}. 

%\david{ Should we add ``Subgraphs of $U$ with bounded radius have bounded treewidth. This is a key property in approximation algorithms on planar graphs~\citep{Baker94}.'' This property would help distinguish our constructions. Note that it is impossible for $U_t$ to satisfy this property for $t\geq 6$, since there are $K_6$-minor-free graphs with radius 1 and unbounded treewidth.} \tony{I prefer not to have this as a key property, but instead to mention this after \cref{KtMinorFreeNoSubdiv} to distinguish our two constructions.} \david{I am now thinking that we should omit this altogether.}


%%%%%%%%%%%%%%%%%%%%%%%%%%%%%%%%%%%%%%%%%%%%%%%%%%%%%%
\subsection{Related work}
\label{Related}

%\david{\citet{RST-DM91} characterise those graphs not containing a $K_{\kappa}$ minor for each cardinal $\kappa$. Add this to the discussion.}

% \david{Add \citep{MMS09} to the discussion. Abstract. A graph property is any isomorphism closed class of simple graphs. For a simple finite graph $H$, let $\longrightarrow H$ denote the class of all simple countable graphs that admit homomorphisms to $\longrightarrow H$, such classes of graphs are called hom-properties. Given a graph property $P$, a graph $G \in P$ is universal in $P$ if each member of $P$ is isomorphic to an induced subgraph of $G$. In particular, we consider universal graphs in $\longrightarrow H$ and we give a new proof of the existence of a universal graph in $\longrightarrow H$, for any finite graph $H$.}

% \citet{HN19}? \citet{Pach77} 


Before continuing, we briefly survey related work on graph universality, focusing on connections to planar graphs. A graph $U$ is \defn{strongly universal} for a graph class $\GG$ if $U\in\GG$ and every graph in $\GG$ is isomorphic to some induced subgraph of $U$. 

\citet{Ackermann37}, \citet{ER63a} and \citet{Rado64} independently proved that there exists a graph, now called the \defn{Rado graph} or the \defn{random graph} \citep{Cameron01,Cameron97,Cameron84}, that contains every graph as an induced subgraph. \citet{Pach75} went further by showing that there exists a graph $U$ such that every graph $G$ can be isometrically embedded in $U$; that is, the distance between any two vertices in $G$ equals the distance between the corresponding vertices in $U$. See  \citep{Cameron01,Cameron97,Cameron84} for surveys on the Rado graph, and see 
\citep{HK88,KP91,BHM13a,BV14,BH13,Rotman71} for various related results. 

%Rotman71,

%\citet{KP91}: Summary: "A class of graphs has a universal element G0 if every other element of the class is isomorphic to an induced subgraph of G0. In Sections 1–4 we give a survey of some recent developments in the theory of universal graphs in the following areas: (1) graphs universal for isometric embeddings, (2) universal random graphs, (3) universal graphs with forbidden subgraphs, (4) universal graphs with forbidden topological subgraphs. Section 5 is devoted to the problem of deciding how far a class of graphs G is from having a universal element. We introduce a new measure of the complexity of the class G, denoted by cp(G). This is defined to be the minimum cardinal κ such that there exist κ elements in G with the property that any other element of G can be embedded into at least one of them as an induced subgraph. G has a universal element if and only if cp(G)=1. Among other theorems we prove that (i) the complexity of the class of all countable graphs without n≥2 independent edges is finite; (ii) for any cardinal κ, ω1≤κ≤2ω, it is consistent that the complexity of the class of all locally finite countable graphs is equal to κ. In Section 6 we consider some analogous questions for hypergraphs.''

%\citet{Higa04}: We consider embeddings between infinite graphs. In particular, we establish that there is no universal element in the class of countable graphs into which the random graph is not embeddable.

Universal graphs for classes defined by an excluded subgraph have been extensively studied. For example, \citet{Henson71} proved that for each  $t\in\NN$ there is a strongly universal graph for the class of $K_t$-free graphs. Several authors have addressed the question: for which (usually finite) graphs $X$, is there is a strongly universal $X$-free graph \citep{Komjath99,HP84,CS07a,CS16,FK97a,FK97b,CS01,KMP88,KP84,CT07,CSS99}? 

% \citep{CSS99}. Abstract: We apply model theoretic methods to the problem of existence of countable universal graphs with finitely many forbidden connected subgraphs. We show that to a large extent the question reduces to one of local finiteness of an associated “algebraic closure” operator. The main applications are new examples of universal graphs with forbidden subgraphs and simplified treatments of some previously known cases.} 

%\citet{CS07a}; It is shown that if $T$ is a finite tree, then there is a universal countable $T$-free graph iff $T$ is either a path or a path with a pendant edge. 

%\citet{CS16}: Let C be a finite connected graph for which there is a countable universal C-free graph, and whose tree of blocks is a path. Then the blocks of C are complete. This generalizes a result of Füredi and Komjáth, and fits naturally into a set of conjectures regarding the existence of countable C-free graphs, with C an arbitrary finite connected graph.

%\citet{CT07}: We consider the problem of the existence of universal countable C‐free graphs with C a connected finite graph. For C a tree arising by from a path by adjunction of one additional edge we show that a universal countable C‐free graph exists. We determine precisely the 2‐bouquets C (i.e., unions of two complete graphs with jost one point in common) for which a universal countable C‐free graph exists. We lay out some elements of a program for determining all the connected finite graphs C for which a countable universal C‐free graph exists. One element of this program is the Tree Conjecture, which is now proved [2]. Our methods involve a mixture of model theory and combinatorics, with the Δ‐system lemma playing a significant role. 

%\citet{CS01}: The problem of the existence of a universal structure omitting a finite set of forbidden substructures is reducible to the corresponding problem in the category of graphs with a vertex coloring by two colors. It is not known whether this problem reduces further to the category of ordinary graphs. It is also not known whether these problems are decidable.

%\citet{KMP88}:It is shown that various classes of graphs have universal elements. In particular, for each n the class of graphs omitting all paths of length n and the class of graphs omitting all circuits of length at least n possess universal elements in all infinite powers.

%\citet{KP84}: For any cardinals $\alpha,\beta\geq 1$, $K_{\alpha,\beta}$ denotes the complete bipartite graph with $\alpha$ and $\beta$ vertices in its two clases. For any graph $H$ and any infinite cardinal $\gamma$, $\mathcal{G}_\gamma(H)$ denotes the class of all graphs of size $\gamma$ containing no subgraph isomorphic to $H$. For any class $\mathcal{G}$ of graphs, $G_0\in\mathcal{G}$ is a universal graph in this class if every graph from $\mathcal{G}$ is isomorphic to some subgraph of $G_0$. De~Bruijn proved that $\mathcal{G}_\omega(K_{1,\omega})$ does not have a universal graph. Shelah proved that, under ${\rm GCH},\ \mathcal{G}_\kappa (K_{1,\kappa})$ contains a universal element for any uncountable $\kappa$. Hajnal and Pach showed that $\mathcal{G}_\omega(K_{2,2})$ does not have a universal element. The authors prove a theorem telling exactly when $\mathcal{G}_\gamma (K_{\alpha,\beta})$ has a universal graph for finite $\alpha$: Assume GCH and let $1\leq\alpha\leq\beta\leq\gamma$, $\alpha<\omega\leq\gamma$. Then there exists a universal graph in $\mathcal{G}_\gamma(K_{\alpha,\beta})$ if and only if $\gamma>\omega$ or $(\gamma=\omega$, $\alpha=1$, $\beta\leq 3)$. Using $\lozenge$ it is also shown that $\mathcal{G} _{\omega_1}(K_{\omega,\omega_1})$ does not contain a universal graph.

\citet{DHV85} considered universality questions for classes excluding $K_t$ as a minor. For $t\leq 4$, they constructed a strongly universal graph, and for $t\geq 5$, they showed there is no universal graph. Note that this result (for $t\geq 5$) is an immediate consequence of our \cref{InfinteCliqueMinor}(a). Similarly, \citet{Diestel85} considered universality questions for classes excluding $K_{m,n}$ as a subdivision. For $m=2$ and $n=3$, Diestel showed there is a universal graph but no strongly universal graph. In the case $m=n=3$, the proof of \citet{Pach81a} can be extended to show that there is no universal graph. For $2\leq n\leq m$ and $m \geq 4$, Diestel showed that there is no universal graph (thus solving a conjecture of Halin). 

%\citet{DHV85} Abstract. Let $\Gamma$ be a class of countable graphs, and let $\mathcal{F}(\Gamma)$ denote the class of all countable graphs that do not contain any subgraph isomorphic to a member of $\Gamma$. Furthermore, let $T\Gamma$ and $H\Gamma$ denote the class of all subdivisions of graphs in $\Gamma$ and the class of all graphs contracting to a member of $\Gamma$, respectively. As the main result of this paper it is decided which of the classes $\mathcal{F}(TK_n)$ and $\mathcal{F}(HK_n)$ , $n\leq \alpha_0$, contain a universal element. In fact, for $\mathcal{F}(TK_4)=\mathcal{F}(HK_4)$ a strongly universal graph is constructed, whereas for $5\leq n \leq \aleph_0$  the classes $\mathcal{F}(TK_n)$ and $\mathcal{F}(HK_n)$ have no universal elements.

%\citet{Diestel85} Abstract. If a class $\GG$ of countable graphs has a member G* that contains a copy of every $G \in  \GG$ then G* is called universal in $\GG$. If every $G \in \GG$  is isomorphic to an induced subgraph of G* we call G* strongly universal in $\GG$. By determining for which $n, m \in \NN$ the class $\GG(T K^{n,m})$ of all countable graphs with forbidden subdivisions of $K_{n,m}$ has a (strongly) universal element we prove a conjecture of R. Halin and also show that weak and strong universality are not equivalent.\\ MSN Review: If a class $\GG$ of countable graphs has a member $G^*$ that contains an (induced) subgraph isomorphic to any $G\in\GG$, then $G^*$ is said to be (strongly) universal. The author considers classes of graphs defined by forbidden subdivisions of $K_{m,n}$. For $m=2$, $n=3$, the class is shown to have a universal but not strongly universal graph and for $2\leq n\leq m$, $m \geq 4$, it is shown that the class has no universal graph (thus solving a conjecture of Halin).

\citet{BHM13} constructed, for each integer $k$, a strongly universal graph for the class of graphs with the property that every  subgraph is $k$-degenerate (that is, every finite subgraph has minimum degree at most $k$).  Similarly, for every finite graph $H$, \citet{MMS09} constructed a strongly universal graph for the class of graphs that admit a homomorphism to $H$.  By Euler's formula, every finite planar graph has minimum degree at most $5$, and  by the 4-Colour Theorem, every planar graph admits a homomorphism to $K_4$. However, there are 2-degenerate graphs (such as 1-subdivisions of complete graphs) and there are $4$-colourable graphs (such as complete 4-partite graphs) that satisfy none of \cref{KeyPropertySeparators}--\cref{KeyPropertyNoInfiniteSubdivision}. So degeneracy or homomorphisms to a fixed graph $H$ are too broad for our purposes.  

%%%%%%%%%%%%%%%%%%%%%%\subsubsection{Universal Graph for Finite Classes Defined by Excluded Subgraph --- Induced Subgraphs}

Universality questions for finite graphs are also widely studied (usually for induced subgraphs). Results are known for trees \citep{ADK17,CGP78,CG78,GL68,Bodini02}, cycles and paths \citep{AAHKS20}, bounded degree graphs \citep{AC07,AA02,ELO08,AAHKBS17,AN19,Butler09}, graphs with a given number of vertices \citep{Moon65,AKTZ19,Alon17,BT81}, graphs with a given number of vertices and edges \citep{BDS19,CE83}, and sparse  graphs \citep{BCEGS82,BCLR89}. The following question has been the focus of attention for finite planar graphs: what is the least number of vertices in a graph that contains every $n$-vertex planar graph as an induced subgraph~\citep{BGP20,KNR92,GL07,DEJGMM,EJM}? Improving on a long sequence of results, \citet{DEJGMM} recently obtained an optimal answer for this question, by constructing, for each value of $n$, a graph with $n^{1+o(1)}$ vertices that contains every $n$-vertex  planar graph as an  induced subgraph. \citet{EJM} improved this result by constructing a graph with $n^{1+o(1)}$ vertices and edges that contains every $n$-vertex  planar graph as an induced subgraph. \citet{EJM} also constructed a graph with $(1+o(1))n$ vertices and $n^{1+o(1)}$ edges that contains every $n$-vertex planar graph (as a subgraph). The proofs in \cite{BGP20,DEJGMM,EJM} all use \cref{FinitePlanarStructure} below, which is also a key tool in our work. 

%%%%%%%%%%%%%%%\subsubsection{Universality under the Minor Relation}

One can also ask universality questions for the minor relation.  \citet{RST-JCTB94} proved that every $n$-vertex planar graph is a minor of the planar grid $P_{14n} \CartProd P_{14n}$. The question becomes more challenging for infinite planar graphs. For example, \citet{DK99} observed that the planar graph obtained from $K_4$ by joining to each of its four vertices infinitely many new vertices of degree 1 is not a minor of the infinite grid. Nevertheless, \citet{DK99} constructed an infinite planar graph $U$ such that every planar graph is a minor of $U$. 

%\subsubsection{Uncountable Graphs}

Finally, we mention uncountable graphs. \citet{Wagner67} proved that a graph (of any cardinality) is planar if and only if it has at most continuumly many vertices, no $K_5$ or $K_{3,3}$ subdivision, and at most countably many vertices of degree at least 3. See \citep{Kojman98,KP84} for more on universality for uncountable graphs. 

% OTHER PAPERS

%  Cherlin11

% \citep{HP84} Given two graphs G and H, and a natural number n, we write G → (H)2n if in every n-edge-colouring of G, one can find a monochromatic subgraph isomorphic to H.
% Let k be an arbitrary cardinal number, Pω the one-way infinite path and Kp,q the complete bipartite graph with p and q vertices in its classes. We consider the following question: Is it true, that if |G| ≥ k and G → (P ω)2n, then G ⊇ Kp,q?

% \citep{HK88}: If $ X$ is a graph, $ \kappa $ a cardinal, then there is a graph $ Y$ such that if the vertex set of $ Y$ is $ \kappa $-colored, then there exists a monocolored induced copy of $ X$; moreover, if $ X$ does not contain a complete graph on $ \alpha $ vertices, neither does $ Y$. This may not be true, if we exclude noncomplete graphs as subgraphs. It is consistent that there exists a graph $ X$ such that for every graph $ Y$ there is a two-coloring of the edges of $ Y$ such that there is no monocolored induced copy of $ X$. Similarly, a triangle-free $ X$ may exist such that every $ Y$ must contain an infinite complete graph, assuming that coloring $ Y$'s edges with countably many colors a monocolored copy of $ X$ always exists.

%\citep{BV14} Arc Combin.

% \citep{BHM13a}: Rado constructed a (simple) denumerable graph R with the positive integers as vertex set with the following edges: For givenmandnwithm<n,mis adjacent tonifnhas a 1 in them’th position of its binary expan-sion. It is well known thatRis a universal graph in the setIof all countablegraphs (since every graph inIis isomorphic to an induced subgraph ofR).In this paper we describe a general recursive construction which proves theexistence of a countable universal graph for any induced-hereditary property ofcountable general graphs. A general construction of a universal graph for the setof finite graphs in a product of properties of graphs is also presented.The paper is concluded by a comparison between the two types of constructionof universal graphs

% \citep{BHM13}: 
% Rado constructed a (simple) denumerable graph R with the positive integers as vertex set with the following edges: For given m and n with m < n, m is adjacent to n if n has a 1 in the m'th position of its binary expansion. It is well known that R is a universal graph in the set ℑc of all countable graphs (since every graph in ℑc is isomorphic to an induced subgraph of R).
% A brief overview of known universality results for some induced-hereditary subsets of ℑc is provided. We then construct a k-degenerate graph which is universal for the induced-hereditary property of finite k-degenerate graphs. In order to attempt the corresponding problem for the property of countable graphs with colouring number at most k+1, the notion of a property with assignment is introduced and studied. Using this notion, we are able to construct a universal graph in this graph property and investigate its attributes.

% \citep{BH13} The well-known Rado graph R is universal in the set of all countable graphs ℑ, since every countable graph is an induced subgraph of R. We study universality in ℑ and, using R, show the existence of 2 ℵ0 pairwise non-isomorphic graphs which are universal in ℑ and denumerably many other universal graphs in ℑ with prescribed attributes. Then we contrast universality for and universality in induced-hereditary properties of graphs and show that the overwhelming majority of induced-hereditary properties contain no universal graphs. This is made precise by showing that there are 2(2 ℵ0) properties in the lattice IK ≤ of induced-hereditary properties of which only at most 2 ℵ0 contain universal graphs. In a final section we discuss the outlook on future work; in particular the question of characterizing those induced-hereditary properties for which there is a universal graph in the property.

% \citep{Komjath99} A corrected proof is given for the existence of a universal countable {C3, C5, …, C2s+1}-free graph. We also prove that there is a universal countable free graph. There is no universal countable H-free graph if H is the dispoint union of 3 or more complete n-cliques for some n ⩾ 2, plus one vertex, joined to every other point.

%\citep{DMN02} For every pair of finite connected graphs F and H, and every positive integer k, we construct a universal graph U with the following properties: (1) There is a homomorphism π:U→H, but no homomorphism from F to U. (2) For every graph G with maximum degree no more than k having a homomorphism h:G→H, but no homomorphism from F to G, there is a homomorphism α:G→U, such that h=π∘α. In particular, this solves a problem presented in [R. Häggkvist and P. Hell, European J. Combin. 14 (1993), no. 1, 23–27; MR1197472] and [A. Galluccio, P. Hell and J. Nešetřil, Discrete Math. 222 (2000), no. 1-3, 101–109; MR1771392] regarding the chromatic number of a universal graph.


%%%%%%%%%%%%%%%%%%%%%%%%%%%%%%%%%%%%%%%%%%%%%%%%%%%%%%%%%%%%%%%
\section{Preliminaries}
\label{Preliminaries}

This section contains preliminary material, including definitions~(\cref{Definitions}); 
an introduction to extendability~(\cref{Extendability}); 
universality for trees~(\cref{Trees}); and
useful results about simplicial decompositions, chordal graphs and tree-decompositions~(\cref{SimplicialDecompositions}). 
Then we provide more detail about some of the key properties of sparse graph classes that were introduced in \cref{Intro}. In particular, we look at the relationship between separators and treewidth~(\cref{SeparatorsTreewidth}), generalised colouring numbers~(\cref{GeneralisedColouringNumbers}), and bounded expansion~(\cref{Expansion}). 

%%%%%%%%%%%%%%%%%%%%%%%%%%%%%
\subsection{Definitions}
\label{Definitions}

We  use the following notation: $\mathbb{N}:=\{1,2,\dots\}$ and $\mathbb{N}_0:=\mathbb{N}\cup\{0\}$. For $m,n\in\ZZ$ with $m\leq n$, let $[m,n]:=\{m,m+1,\dots,n\}$ and $[n]:=\{1,2,\dots,n\}$. 

We use standard graph-theoretic notation and terminology \citep{Diestel5}.

We consider undirected graphs $G$ with vertex set $V(G)$ and edge set $E(G)\subseteq \binom{V(G)}{2}$ (with no loops or parallel edges). A graph $G$ is \defn{trivial} if $E(G)=\emptyset$. A graph $G$ is \defn{finite} if $V(G)$ is finite, and is \defn{countable} if $V(G)$ is countable. Recall that, for brevity, we assume all graphs are countable. 

The \defn{neighbourhood} of a vertex $v$ in a graph $G$ is $N_G(v):= \{w\in V(G): vw\in E(G)\}$; then $v$ has \defn{degree} $\deg_G(v):= |N_G(v)|$. A graph $G$ is \hdefn{$k$}{regular} if every vertex of $G$ has degree $k$. A graph $G$ is \defn{locally finite} if every vertex in $G$ has finite degree. For $S\subseteq V(G)$, let $N_G(S):=\bigcup_{v\in S}(N_G(v)\setminus S)$.

For a vertex $v$ in a directed graph $G$, the \defn{in-neighbourhood} of $v$ is $N^-_G(v):= \{w\in V(G): wv\in E(G)\}$ and the \defn{out-neighbourhood} of $v$ is $N^+_G(v):= \{w\in V(G): vw\in E(G)\}$. Then $v$ has \defn{in-degree} $\deg^-_G(v):= |N^-_G(v)|$ and \defn{out-degree} $\deg^+_G(v):= |N^+_G(v)|$. 

An \defn{orientation} of a graph $G$ is a directed graph obtained from $G$ by directing each edge from one endpoint to the other. An orientation is \defn{acyclic} if there is no directed cycle. In an acyclically oriented graph, if there is a directed path from a vertex $v$ to a vertex $w$, then $v$ is an \defn{ancestor} of $w$ and $w$ is a \defn{descendant} of $v$.

A graph $H$ is a \defn{subgraph} of a graph $G$ if $V(H)\subseteq V(G)$ and $E(H)\subseteq E(G)$. Two graphs $G_1$ and $G_2$ are \defn{isomorphic} if there is a bijection $f:V(G_1)\to V(G_2)$ such that for all $v,w\in V(G_1)$ we have $vw\in E(G_1)$ if and only if $f(v)f(w)\in E(G_2)$. 

A set $\GG$ of graphs is a \defn{graph class} if for every graph $G\in\GG$, every graph isomorphic to $G$ is also in $\GG$. A graph class $\GG$ is \defn{monotone} if for every $G\in\GG$ every subgraph of $G$ is in $\GG$. A graph class $\GG$ is \defn{hereditary} if for every $G\in\GG$ every induced subgraph of $G$ is in $\GG$.
A graph class $\GG$ is \defn{countable} if $\GG$ has countably many equivalence classes under the isomorphism relation. 

A \defn{colouring} of a graph $G$ is a function that assigns one `colour' to each vertex of $G$, such that adjacent vertices of $G$ are assigned distinct colours. For $k\in\NN$, a \hdefn{$k$}{colouring} is a colouring with at most $k$ colours. The \defn{chromatic number}, $\chi(G)$, of $G$ is the minimum $k\in\NN$ such that $G$ is $k$-colourable. If there is no such $k$, then $G$ has chromatic number $\chi(G)=\infty$. 

A \defn{clique} in a graph $G$ is a set of pairwise adjacent vertices in $G$. The \defn{clique-number}, $\omega(G)$, of $G$ is the maximum $k\in\NN$ such that $G$ has a clique of cardinality $k$. If there is no such $k$, then $G$ has clique number $\omega(G)=\infty$. 

For $d\in\NN$, a graph $G$ is \hdefn{$d$}{degenerate} if every finite subgraph of $G$ has minimum degree at most $d$. The \defn{degeneracy} of $G$ is the minimum $d\in\NN$ such that $G$ is $d$-degenerate. 

A \hdefn{$vw$}{path} in a graph $G$ is a path with endpoints $v$ and $w$. We let $\dist_G(v,w)$ be the length of a shortest $vw$-path in $G$.   If there is no such path, then $\dist_G(u,v)=\infty$.  A $vw$-path $P$ in $G$ is a \defn{geodesic} if the length of $P$ equals $\dist_G(v,w)$. 

\index{infinite path} The \DefNoIndex{1-way infinite path} $\PP$ is the graph with vertex-set $\NN$ and edge set $\{\{n,n+1\}:n\in\NN\}$. The \DefNoIndex{2-way infinite path} $\PPP$ has vertex-set $\ZZ$ and edge set $\{\{n,n+1\}:n\in\ZZ\}$. 

A graph $H$ is a \defn{minor} of a graph $G$ if a graph isomorphic to $H$ can be obtained from a subgraph of $G$ by contracting edges. A graph class $\GG$ is \defn{minor-closed} if for every graph $G\in\GG$, every minor of $G$ is also in $\GG$. A minor-closed class $\GG$ is \defn{proper} if some graph is not in $\GG$. 

A \defn{model} of a graph $H$ in  a graph $G$ is a collection $(X_v)_{v\in V(H)}$ of pairwise disjoint connected subgraphs of $G$ indexed by the vertices of $H$, such that for each  edge $vw\in E(H)$ there is an edge of $G$ between $X_v$ and $X_w$. Each subgraph $X_v$ is called a \defn{branch set} of the model. Note that $H$ is a minor of $G$ if and only if there is a model of $H$ in $G$. 

A \defn{subdivision} of a graph $H$ is any graph obtained from $H$ by repeatedly applying the following operation: delete an edge $vw$, introduce a new vertex $x$, and add new edges $vx$ and $xw$. A graph $G$ \defn{contains an $H$-subdivision} if $G$ contains a subgraph isomorphic to a subdivision of $H$. In this case, $H$ is a minor of $G$. 

Planar graphs form a proper minor-closed class. More generally, for any surface $\Sigma$, the class of graphs embeddable in $\Sigma$ form a proper minor-closed class. The \defn{Euler genus} of the orientable surface with $h$ handles is $2h$. The \defn{Euler genus} of  the non-orientable surface with $c$ cross-caps is $c$. The \defn{Euler genus} of a graph $G$ is the minimum Euler genus of a surface in which $G$ embeds (with no crossings). See~\citep{MoharThom} for background on embeddings of graphs on surfaces.

If $H$ is a subgraph of a graph $G$, then a \defn{chord} of $H$ (with respect to $G$) is an edge $vw\in E(G)\setminus E(H)$ with $v,w\in V(H)$. A graph $G$ is \defn{chordal} if every cycle $C$ of length at least 4 has a \defn{chord}.

A \defn{vertex-partition}, or simply \defn{partition}, of a graph $G$ is a set $\PART$ of non-empty sets of vertices in $G$ such that each vertex of $G$ is in exactly one element of $\PART$. Each element of $\PART$ is called a \defn{part}. A partition $\PART$ of a graph $G$ is \defn{finite} if each part of $\PART$ is finite. The \defn{quotient} of $\PART$ is the graph, denoted by $G/\PART$, with vertex set $\PART$ where distinct parts $A,B\in \PART$ are adjacent in $G/\PART$ if and only if some vertex in $A$ is adjacent in $G$ to some vertex in $B$. A partition of $G$ is \defn{connected} if the subgraph induced by each part is connected. In this case, the quotient is a minor of $G$ (obtained by contracting each part into a single vertex). 

A partition $\PART$ of a graph $G$ is \defn{chordal} if the quotient $G/\PART$ is chordal. 

A set $S$ of vertices in a graph $G$ is \defn{separating} if $G-S$ is disconnected. Sometimes we also say $G[S]$ is separating. A separating set $S$ in $G$ is \defn{minimal} if no proper subset of $S$ is separating. If $S$ is a minimal separating set, then for each component $X$ of $G-S$, each vertex in $S$ has a neighbour in $X$.

%%%%%%%%%%%%%%%%%%%%%%%%%%%%%%%%%%%%%%%%%%%%%%%
\subsection{Extendability}
\label{Extendability}

A graph class $\Gamma$ is \defn{extendable} if the following property holds for every graph $G$: if every finite subgraph of $G$ is in $\Gamma$, then $G$ is in $\Gamma$. (Recall that $G$ is assumed to be countable.) The following result of \citet{Imrich75} provides an example of an extendable class. 

\begin{lem}[\citep{Imrich75}] 
\label{PlanarityExtendable}
The class of planar graphs is extendable. 
\end{lem}

\begin{proof}
If a finite graph $H$ is a subdivision of a graph $G$, then $H$ is a subdivision of a finite subgraph of $G$. That is, if no finite subgraph of $G$ contains a subdivision of $H$, then $H$ is not a subdivision of $G$.  Kuratowski's Theorem says that a (countable) graph $G$ is planar if and only if $G$ contains no $K_5$ or $K_{3,3}$ subdivision. By the above observation, $G$ is planar if and only if no finite subgraph contains a $K_5$ or $K_{3,3}$ subdivision. That is, $G$ is planar if and only if every finite subgraph of $G$ is planar. Hence planarity is extendable.
\end{proof}

A function $f$ is a \defn{graph parameter} if $f(G)\in\NN$ for every graph $G$, and $f(G_1)=f(G_2)$ for all isomorphic graphs $G_1$ and $G_2$. A graph parameter is \defn{extendable} if for every graph $G$ and integer $k\in\NN$, the class $\{G : f(G) \leq k\}$ is extendable. 

The following two lemmas are useful for showing that certain graph parameters are extendable. For example, they each imply that chromatic number is extendable (the de~Bruijn--Erd\H{o}s~Theorem~\citep{dBE51}). 

\begin{lem}[K\"onig's Lemma \citep{Konig27}] 
\label{Konig} 
Let $V_1,V_2,\dots$ be an infinite sequence of disjoint non-empty finite sets. Let $G$ be a graph with vertex set $\bigcup_{n\in\NN} V_n$, such that for all $n\in\NN$ every vertex in $V_{n+1}$ has a neighbour in $V_n$. Then $G$ contains a path $v_1,v_2,\dots$ with $v_n\in V_n$ for all $n\in\NN$. 
\end{lem}

\begin{lem}[Zorn's Lemma; see \citep{Konig27}] \label{Zorn} 
If a partially ordered set $P$ has the property that every chain in $P$ has an upper bound in $P$, then $P$ contains at least one maximal element.
\end{lem}


%%%%%%%%%%%%%%%%%%%%%%%%%%%%%
\subsection{Trees}
\label{Trees}

A \defn{tree} is a connected graph with no cycles. An orientation of a tree is a \hdefn{$1$}{orientation} if every vertex has in-degree at most 1. For every directed edge $vw$ in a 1-oriented tree, $v$ is the \defn{parent} of $w$ and $w$ is a \defn{child} of $v$. If there is a directed path from a vertex $v$ to a vertex $w$ in a 1-oriented tree, then $v$ is an \defn{ancestor} of $w$ and $w$ is a \defn{descendant} of $v$. 
If a vertex $r$ in a 1-oriented tree $T$ has in-degree 0, then every edge of $T$ is oriented away from $r$, implying that $r$ is the only vertex with in-degree 0. In this case, we say $T$ is \defn{rooted} at $r$. In a tree $T$ with a specified root vertex $r$, we implicitly consider the edges of $T$ to be oriented away from $r$. 

\begin{lem}
\label{TreeOrientation}
For every vertex $r$ of a tree $T$ there is a 1-orientation of $T$ rooted at $r$. Moreover, a tree $T$ has an unrooted 1-orientation if and only if $T$ contains a 1-way infinite path.
\end{lem}

\begin{proof}
For the first claim, simply orient every edge of $T$ away from $r$, to obtain a 1-orientation of $T$ rooted at $r$.  

Now suppose that $T$ contains a 1-way infinite path $P$ starting at some vertex $v$. Orient every edge of $P$ towards $v$, and orient every edge of $T-E(P)$ away from $P$. Then every vertex has in-degree exactly 1, and $T$ has an unrooted 1-orientation. 

Finally, suppose that $T$ has an unrooted 1-orientation. So every vertex has a parent. Let $v_1$ be any vertex in $T$. Suppose that $v_1,v_2,\dots,v_i$ is an anti-directed path in $T$ (meaning that each edge is oriented from $v_j$ to $v_{j-1}$). Let $v_{i+1}$ be the parent of $v_i$. So $v_1,\dots,v_i$ are all descendants of $v_{i+1}$. Repeating this step, we obtain a 1-way infinite anti-directed path $v_1,v_2,\dots$. 
\end{proof}


%%%%%%%%%%%%%%%%%%%%%%%%%%%%%
\subsection{Treewidth and Simplicial Decompositions}
\label{SimplicialDecompositions}

This section introduces graph treewidth and its connection to Halin's Simplicial Decomposition Theorem via chordal graphs. 

A \defn{tree-decomposition} of a graph $G$ is a collection $(B_x\subseteq V(G):x\in V(T))$ of subsets of $V(G)$ (called \defn{bags}) indexed by the nodes of a tree $T$, such that:
\begin{enumerate}[label=(\alph*)]
\item for every edge $uv\in E(G)$, some bag $B_x$ contains both $u$ and $v$, and
\item for every vertex $v\in V(G)$, the set $\{x\in V(T):v\in B_x\}$ induces a non-empty subtree of $T$.
\end{enumerate}
We emphasise that the subtree in (b) must be connected. If all the bags are finite and their cardinalities are bounded, then the \defn{width} of a tree-decomposition is the cardinality of the largest bag minus 1. The \defn{treewidth} of a graph $G$, denoted by $\tw(G)$, is the minimum width of a tree-decomposition of $G$. These definitions are due to \citet{RS-II}; see \citep{Diestel90,KT90,KT90a,KT91} for work on tree-decompositions of infinite graphs. Treewidth is recognised as the most important measure of how similar a given graph is to a tree. Indeed, a connected graph with at least two vertices has treewidth 1 if and only if it is a tree. 

A \defn{path-decomposition} is a tree-decomposition in which the underlying tree is a path. We denote a path-decomposition by the corresponding sequence of bags $(B_1,B_2,\dots)$. The \defn{pathwidth} of $G$, denoted by $\pw(G)$, is the minimum width of a path-decomposition of $G$. The \defn{bandwidth} of a graph $G$, denoted by $\bw(G)$, is the minimum $k\in\NN$ for which there is a linear ordering $v_1,v_2,\dots$ of $V(G)$, such that $|i-j|\leq k$ for every edge $v_iv_j\in E(G)$. It is easily seen that $\tw(G)\leq\pw(G)\leq \bw(G)$. 

The general definition of a simplicial decomposition can be found in \cite{Thom83}. Here, we give the definition in the special case of countable graphs containing no $K_{\aleph_0}$. Let $H_1,H_2,\ldots$ be pairwise disjoint (finite or infinite) graphs containing no $K_{\aleph_0}$ and no finite separating complete subgraph. (We consider the empty graph to be complete, so each $H_i$ is connected.)\ Form a sequence of graphs $G_1,G_2, \ldots$ as follows. Let $G_1:=H_1$. Having defined $G_n$, define $G_{n+1}$ by selecting, for some $p\in\NN_0$,  a $K_p$ subgraph in $G_n$ and a $K_p$ subgraph in $H_{n+1}$, and identifying the two copies of $K_p$. Let $G := \bigcup_{n\in\NN} G_n$. If $G$ contains no   $K_{\aleph_0}$, then $H_1,H_2, \ldots$ are said to form a \defn{simplicial decomposition} of $G$, where each $H_i$ is called a \defn{simplicial summand}. \citet{Halin64a} proved the following (where, as always, we assume graphs are countable):

\begin{thm}[\citep{Halin64a}]
\label{HalinsTheorem}
Every graph containing no $K_{\aleph_0}$ has a simplicial decomposition. 
\end{thm}

The next two lemmas show the equivalence between simplicial decompositions and tree-decompositions, and their relationship with chordal graphs. We include the proofs for completeness. 

\begin{lem}
\label{SimpDecTreeDec}
Let $H_1,H_2, \ldots$ be pairwise disjoint graphs, none containing an infinite complete subgraph or a finite separating complete subgraph. Then $H_1,H_2, \ldots$  form a simplicial decomposition of a graph $G$ if and only if there is a tree-decomposition $(B_x:x\in V(T))$ of $G$ for some rooted tree $T$, and a bijection $f:V(T)\to\NN$ such that $f(v)<f(w)$ whenever $v$ is the parent of $w$, and $G[B_x]\cong H_{f(x)}$ for every $x\in V(T)$, and $G[B_x \cap B_y]$ is a finite complete graph for every $xy\in E(T)$.
\end{lem}

\begin{proof}
Suppose $H_1,H_2, \ldots$ form a simplicial decomposition of $G$.
We claim that for each $n\in\NN$, the graph $G_n$ defined above has a tree-decomposition $(B_x:x\in V(T_n))$ for some  $n$-vertex tree $T_n$ rooted at node $r$, and there is a bijection $f:V(T)\to[n]$ such that $f(r)=1$ and $f(v)<f(w)$ whenever $v$ is the parent of $w$, and
$G[B_x]\cong H_{f(x)}$ for every $x\in V(T)$. 
For $n=1$, let $T_1$ be a 1-node tree $T_1$ with vertex set $\{r\}$, let $f:V(T_1)\to[1]$ be a bijection, and let $B_x:=V(H_1)$. Consider $T_1$ to be rooted at $r$. Assume we have the claimed tree-decomposition and bijection for some $n\in\NN$. 
Then $G_{n+1}$ is obtained from $G_n$ by identifying, for some $p\in\NN$, a $K_p$ subgraph in $G_n$ with a $K_p$ subgraph in $H_{n+1}$. The first $K_p$ subgraph is in some bag $B_x$ by \cref{TreeDecompositionClique}. Let $T_{n+1}$ be the tree obtained form $T_n$ by adding one new node $y$ adjacent to $x$, where $f(y):=n+1$ and $B_y:=V(H_{n+1})$. Consider $T_{n+1}$ to be rooted at $r$. We obtain the claimed tree-decomposition and bijection for $n+1$. Let $T:=\bigcup_{n\in\NN} T_n$. Then $(B_x:x\in V(T))$ is the desired tree-decomposition of $G$ and $f$ is the desired bijection.

Conversely, suppose there is a tree-decomposition $(B_x:x\in V(T))$ of $G$ for some rooted tree $T$, and a bijection $f:V(T)\to\NN$ such that 
$f(v)<f(w)$ whenever $v$ is the parent of $w$, and
$G[B_x] \cong H_{f(x)}$ for every $x\in V(T)$, and
$G[B_x \cap B_y]$ is a finite complete graph for every $xy\in E(T)$.
We prove by induction on $n\in\NN$ that $H_1,\dots,H_n$ form a simplicial decomposition of $G_n:=G[ \cup_{x\in V(T),f(x)\in[n]} B_x ]$. This trivially holds for $n=1$. Assume that $H_1,\dots,H_n$ form a simplicial decomposition of $G_n$. Let $y:= f^{-1}(n+1)$. Let $x$ be the parent of $y$ in $T$. So $f(x)<f(y)$, implying $f(x)\in[n]$. By assumption, $G[B_y]\cong H_{n+1}$. Let $S:=B_x \cap B_y$. By assumption, $G[S]\cong K_p$ for some $p\in\NN$. Thus $G_{n+1}$ is obtained from $G_n$ and $H_{n+1}$ by identifying $S$ (in $G_n$) with the image of $S$ under the assumed isomorphism from $G[B_y]$ to $H_{n+1}$. Thus $H_1,\dots,H_n,H_{n+1}$ form a simplicial decomposition of $G_{n+1}$. And $H_1,H_2,\dots$ form a simplicial decomposition of $G$. 
\end{proof}

A \defn{simplicial orientation} of a graph $G$ is an acyclic orientation of $G$ such that the in-neighbourhood of every vertex is a clique.  

\begin{lem}
\label{ChordalCharacterisation}
The following are equivalent for a graph $G$ containing no $K_{\aleph_0}$:
\begin{enumerate}[(a)] 
\item\label{ChordalCharacterisationChordal} $G$ is chordal;
\item\label{ChordalCharacterisationSimpDec} $G$ has a simplicial decomposition in which each simplicial summand is a finite clique;  
\item\label{ChordalCharacterisationTreeDec} $G$ has a tree-decomposition in which each bag is a finite clique; 
\item\label{ChordalCharacterisationSimplicialOrientation} $G$ has a simplicial orientation;
\item\label{ChordalCharacterisationMinimalSep} every minimal separating set in $G$ is a clique. 
\end{enumerate}
\end{lem}

\begin{proof}
\cref{ChordalCharacterisationChordal} $\Longrightarrow$ \cref{ChordalCharacterisationSimpDec}: Assume that $G$ is chordal. By \cref{HalinsTheorem}, there is a simplicial decomposition $H_1,H_2,\dots$ of $G$. Every induced subgraph of a chordal graph is chordal, so each $H_i$ is chordal, implying that every minimal separator in $H_i$ induces a complete subgraph. By assumption, each $H_i$ has no separating complete subgraph. Thus, $H_i$ has no minimal separator, implying $H_i$ is complete. Finally, $H_i$ is finite since $G$ has no $K_{\aleph_0}$.

\cref{ChordalCharacterisationSimpDec}
$\Longleftrightarrow$ \cref{ChordalCharacterisationTreeDec} follows immediately from \cref{SimpDecTreeDec}.

\cref{ChordalCharacterisationTreeDec} $\Longrightarrow$ \cref{ChordalCharacterisationSimplicialOrientation}: Assume that $G$ has a tree-decomposition $(B_x:x\in V(T))$ in which each bag is a clique. Root $T$ at an arbitrary node $r$. For each vertex $v\in V(G)$, let $x_v$ be the node of $T$ closest to $r$ such that $v\in B_{x_v}$. For each node $x\in V(T)$, if $B'_x:=\{v\in V(G):x_v=x\}$, then acyclically orient the clique on $B'_x$. Consider an edge $e=vw$ of $G$. 
If $x_v=x_w$ then we have already oriented $e$. If $x_v$ is closer to $r$ than $x_w$, then orient $e$ from $v$ to $w$. If $x_w$ is closer to $r$ than $x_v$, then orient $e$ from $w$ to $v$. Now $G$ is acyclically oriented, and for every vertex $v\in V(G)$, the in-neighbourhood of $v$ is a subset of $B_{x_v}$ and is therefore a clique.
 
\cref{ChordalCharacterisationSimplicialOrientation} $\Longrightarrow$ \cref{ChordalCharacterisationChordal}: Assume that $G$ has a simplicial orientation. Let $C$ be a cycle in $G$. So $C$ has a vertex $v$ with in-degree 2 in $C$. The neighbours of $v$ in $C$ are adjacent in $G$. So $C$ has a chord (unless $|V(C)|=3$). Thus $G$ is chordal.

\cref{ChordalCharacterisationChordal} $\Longrightarrow$ \cref{ChordalCharacterisationMinimalSep}: Assume that $G$ is chordal. Let $S$ be a minimal separating set in $G$. Let $X$ and $Y$ be distinct components of $G-S$. Suppose that $v$ and $w$ are non-adjacent vertices in $S$. Since $S$ is minimal, 
there is a shortest $vw$-path $P$ in $G$ with every internal vertex in $X$, and
there is a shortest $vw$-path $Q$ in $G$ with every internal vertex in $Y$. 
Thus $P\cup Q$ is a chordless cycle in $G$. This contradiction shows that $S$ is a clique. 

\cref{ChordalCharacterisationMinimalSep} $\Longrightarrow$ 
\cref{ChordalCharacterisationChordal}: Assume that every minimal separating set in $G$ is a clique. Let $C$ be a cycle in $G$ of length at least 4. Let $v,w$ be non-adjacent vertices in $C$. Let $S$ be a minimal separating set in $G$ such that $v$ and $w$ are in distinct components of $G-S$. So $S$ is a clique. Each of the two $vw$-paths in $C$ have a vertex in $S$, implying $C$ has a chord. Hence $G$ is chordal. 
\end{proof}

For a proof of the following elementary and well-known result see \citep{Diestel5}. 

\begin{lem}
\label{TreeDecompositionClique}
For every graph $G$ containing no $K_{\aleph_0}$, for every clique $X$ of $G$, every tree-decomposition of $G$ has a bag containing $X$.
\end{lem}

We emphasise that $G$ must contain no $K_{\aleph_0}$ in \cref{TreeDecompositionClique}. For example, if $G$ is the complete graph with $V(G)=\NN$, and $B_n:=[n]$ for each $n\in\NN$, then $(B_1,B_2,\dots)$ is a path-decomposition of $G$ in which every bag is finite, and no bag contains the clique $V(G)$.

A \defn{simplicial $k$-orientation} of a chordal graph $G$ is an acyclic orientation of $G$ such that the in-neighbourhood of every vertex is a clique of size at most $k$. For brevity, a 
simplicial $k$-orientation is henceforth called a 
\hdefn{$k$}{orientation} (which matches the definition of 1-orientation when $k=1$). A $k$-orientation is \defn{rooted} if each connected component of $G$ has a vertex of in-degree 0. It is easily seen\footnote{Say $G$ is a connected chordal graph, oriented so that the in-neighbourhood of each vertex is a clique. Say $r$ is a vertex of in-degree 0. Let $v$ be any vertex of $G$. Let $P$ be a shortest $rv$-path in $G$ (ignoring the edge orientation). If there is a vertex $x$ in $P$ incident with two incoming edges $yx,zx$ in $P$, then $yz$ is an edge, implying that $P$ is not a shortest $rv$-path. So no vertex in $P$ has two incoming edges in $P$. The edge incident to $r$ in $P$ is outgoing at $r$. So $P$ is oriented from $r$ to $v$. In particular, $v$ has in-degree at least 1. Thus $G$ has at most one vertex of in-degree 0.} that each component of $G$ has at most one vertex of in-degree 0. The next lemma follows from \cref{ChordalCharacterisation,TreeDecompositionClique}. 

\begin{lem}
\label{ChordalCharacterisation2}
The following are equivalent for a graph $G$ and $k\in\NN$:
\begin{itemize}
\item $G$ is chordal with no $K_{k+2}$ subgraph;
\item $G$ is chordal and is $(k+1)$-colourable; 
\item $G$ has a tree-decomposition in which each bag is a clique of size at most $k+1$;
\item $G$ has a $k$-orientation. 
\end{itemize}
\end{lem}

The next lemma (which is well known when $G$ is finite) says that every graph has a `normalised' tree-decomposition. 

\begin{lem}
\label{StandardTreewidth}
Let $G$ be a graph with treewidth at most $k\in\NN$. Then there is a tree $T$ with $V(T)=V(G)$ and there is a tree-decomposition $(B_x:x\in V(T))$ of $G$ with width at most $k$ such that:
\begin{itemize}
\item $T$ is rooted at some vertex $r$ and $|B_r|=1$, 
\item for every edge $vw$ of $G$, $v$ is an ancestor of $w$ or $w$ is an ancestor of $v$ in $T$, and
\item for every non-root node $w\in V(T)$, if $v$ is the parent of $w$ in $T$, then $|B_w\setminus B_v|=1$. 
\end{itemize}
Moreover, $G$ has a $(k+1)$-colouring such that distinct vertices in the same bag are assigned distinct colours. 
\end{lem}

\begin{proof}
Let $(B_x:x\in V(T))$ be an arbitrary tree-decomposition of $G$ with width at most~$k$. Let $x$ be an arbitrary node of $T$. Add a new node $r$ to $T$ only adjacent to $x$, and let $B_r:=\{v\}$ where $v$ is an arbitrary vertex in $B_x$. Consider $T$ to be rooted at $r$. Thus $|B_r|=1$, as desired. Consider each non-root node $y$ in $T$. Let $x$ be the parent node of $y$. Let $t:= |B_y\setminus B_x|$.  If $t=0$ then contract the edge $xy$ into $x$. If $t\geq 2$, then choose one vertex $w\in B_y\setminus B_x$,  subdivide the edge $xy$ introducing a new node $z$ in $T$, and let $B_z := B_y\setminus \{w\}$. Repeating these operations, we may assume that $|B_y\setminus B_x|=1$ for every non-root node $y$ of $T$ with parent $x$. Thus there is a bijection from $V(G)$ to $V(T)$, where each vertex $w$ of $G$ is mapped to the node $x$ in $T$ closest to $r$ with $w\in B_x$. Rename each vertex of $T$ by the corresponding vertex of $G$. We use $r$ to refer to the vertex of $G$ corresponding to the root of $T$. 

For every edge $vw$ of $G$, since $v$ and $w$ appear in a common bag, $v$ is an ancestor of $w$ or $w$ is an ancestor of $v$ in $T$. 

We now $(k+1)$-colour the vertices $w$ of $T$ in order of their distance from $r$ (breaking ties arbitrarily). First, let $c(r):=1$. When we come to colour vertex $w$, if $v$ is the parent of $w$ then $B_w\setminus B_v=\{w\}$, implying all the vertices in $B_w\setminus \{w\}$ are already coloured. Let $c(w)$ be an element of $[k+1]$ not assigned to any vertex in $B_w\setminus \{w\}$. This is possible since $|B_w|\leq k+1$. We now prove that distinct vertices in the same bag are assigned distinct colours. This is true for the root bag since it only contains one vertex. Consider two vertices $v$ and $w$ in the same bag. Let $x$ be the node of $T$ closest to the root, such that $v,w\in B_x$. Then $v=x$ or $w=x$. Without loss of generality, $w=x$, implying $v\in B_w\setminus\{w\}$. By construction, $c(w)\neq c(v)$. Hence distinct vertices in the same bag are assigned distinct colours. 
\end{proof}

%%%%%%%%%%%%%%%%%%%%%%%%%%%%%%%%

\subsection{Balanced Separations}
\label{SeparatorsTreewidth}

A \defn{separation} in a graph $G$ is a pair $(G_1,G_2)$ of subgraphs of $G$ such that $G=G_1\cup G_2$, $E(G_1)\cap E(G_2)=\emptyset$, $V(G_1)\setminus V(G_2)\neq\emptyset$, and $V(G_2)\setminus V(G_1)\neq\emptyset$. The \defn{order} of $(G_1,G_2)$ is $|V(G_1) \cap V(G_2)|$. A \hdefn{$k$}{separation} is a separation of order $k$. A separation $(G_1,G_2)$ is \defn{balanced} if $|V(G_1)\setminus V(G_2)| \leq \frac{2}{3} |V(G)|$ and $|V(G_2)\setminus V(G_1)| \leq \frac{2}{3} |V(G)|$. 

A graph class $\GG$  admits \defn{strongly sublinear separators} if there exists $ c \in\mathbb{R}^+$ and $\beta\in[0,1)$ such that for every graph $G\in\GG$, every subgraph $H$ of $G$ has a balanced separation of order at most $c|V(H)|^\beta$. For example, \citet{LT79} proved that the class of planar graphs admits strongly sublinear separators (with $\beta=\frac12$). More generally, \citet{AST90} proved that every proper minor-closed class admits strongly sublinear separators (again with $\beta=\frac12$). 

\citet[(2.6)]{RS-II} established the following connection between treewidth and balanced separations: Every graph $G$ has a balanced separation of order at most $\tw(G)+1$. \citet{DN19} proved the following converse: If every finite subgraph of a graph $G$ has a balanced separation of order at most $s$, then $\tw(G) \leq 15s$.

\citet{DHJLW21} proved the following strongly sublinear bound on the treewidth of graph products, which is used in \cref{TreewidthPathStructure}.

\begin{lem}[\citep{DHJLW21}]
\label{ProductTreewidth}
Let $G$ be an $n$-vertex subgraph of $H \boxtimes P$ for some graph $H$ and path $P$. Then $\tw(G) \leq 2  \sqrt{ (\tw(H)+1) n } -  1$, 
and $G$ has a balanced separation of order at most $2  \sqrt{ (\tw(H)+1) n }$.
\end{lem}

For $k\in\NN$, a tree-decomposition $(B_x:x\in V(T))$ of a graph $G$ has \defn{adhesion} $k$ if $|B_x\cap B_y|\leq k$ for every edge $xy\in E(T)$. Given a tree-decomposition $(B_x:x\in V(T))$ of a graph $G$, the \defn{torso} of $x$ is the graph obtained from $G[B_x]$ by adding an edge $vw$ whenever $v,w\in B_x\cap B_y$ for some edge $xy\in E(T)$ incident to $x$. If $\Gamma$ is a class of graphs, then a tree-decomposition is said to be \defn{over} $\Gamma$ if every torso is in $\Gamma$. If $\Gamma$ is a graph, then a tree-decomposition is said to be \defn{over $\Gamma$} if every torso is isomorphic to a subgraph of $\Gamma$. For a graph $U$, let $\DD(U)$ be the class of graphs that have a tree-decomposition over $U$. For $k\in\NN$, let $\DD_k(U)$ be the class of graphs that have a tree-decomposition over $U$ with adhesion $k$. 

The following simple lemma will be used in the proof of \cref{MinorUniversal}.

\begin{lem}
\label{TreewidthOverTreeDecomposition}
For every graph class $\Gamma$ and $n\in\NN$, the maximum treewidth of a graph in $\DD(\Gamma)$ with at most $n$ vertices equals the maximum treewidth of a graph in $\Gamma$ with at most $n$ vertices. 
\end{lem}

\begin{proof}
Let $f(n)$ be the maximum treewidth of a graph in $\Gamma$ with at most $n$ vertices. It suffices to prove that if $G$ is a graph in $\DD(\Gamma)$ with at most $n$ vertices, then $\tw(G)\leq f(n)$. Let $(B_x:x\in V(T))$ be a tree-decomposition of $G$ such that each torso is in $\Gamma$. For each node $x\in V(T)$, since $|B_x|\leq n$, the torso of $x$ has a tree-decomposition $(C^x_z:z\in V(T^x))$ of width at most $f(n)$. Let $T^*$ be the tree obtained from the disjoint union of $T^x$ taken over all $x\in V(T)$, where for each edge $xy\in T$, we add an edge between a node $\alpha\in V(T^x)$ and a node $\beta\in V(T^y)$ such that $B_x\cap B_y \subseteq C^x_\alpha\cap C^y_\beta$, which exist since $B_x\cap B_y$ is a clique in the torso of $x$ and in the torso of $y$. We obtain a tree-decomposition of $G$ with width at most $f(n)$. 
\end{proof}

%%%%%%%%%%%%%%%%%%%%%%%%%%%%%%%%%%%%%%%%%%%%%%%%%%%%%
\subsection{Generalised Colouring Numbers}
\label{GeneralisedColouringNumbers}

\citet{KY03} introduced the following definition. For a graph $G$, total order $\preceq$ of $V(G)$, vertex $v\in V(G)$, and  $r\in\NN$, let $\sreach_r(G,\preceq,v)$ be the set of vertices $w\in V(G)$ for which there is a path $v=w_0,w_1,\dots,w_{r'}=w$ of length $r'\in[0,r]$ such that $w\preceq v$ and $v\prec w_i$ for all $i\in[r-1]$. For a graph $G$ and integer $r\in\NN$, the \hdefn{$r$}{colouring number} $\col_r(G)$ is the minimum integer such that there is a total order~$\preceq$ of $V(G)$ with $|\sreach_r(G,\preceq,v)|\leq \col_r(G)$ for every vertex $v$ of $G$. 

%Similarly, let $\wreach_r(G,\preceq,v)$ be the set of vertices $x\in V(G)$ for which there is a path $v=w_0,w_1,\dots,w_{r'}=x$ of length $r'\in[0,r]$ such that $x\preceq v$ and $x\prec w_i$ for all $i\in[r'-1]$. Similarly, the \defn{$r$-weak colouring number} $\wcol_r(G)$ is the minimum integer $k$ such that there is a linear ordering $\preceq$ of~$V(G)$ with $|\wreach_r(G,\preceq,v)|\leq k$ for each vertex $v$ of $G$.

An attractive aspect of generalised colouring numbers is that they interpolate between degeneracy and treewidth \citep{KPRS16}. Indeed, it follows from the definition that $\col_1(G)$ equals the degeneracy of $G$ plus 1, implying $\chi(G)\leq \col_1(G)$. At the other extreme, \citet{GKRSS18} showed that $\col_r(G)\leq \tw(G)+1$ for all $r\in\NN$, and indeed $$\lim_{r\to\infty}\col_r(G)=\tw(G)+1.$$

Generalised colouring numbers provide upper bounds on several graph parameters of interest. For example, a graph colouring is \defn{acyclic} if the union of any two colour classes induces a forest; that is, every cycle is assigned at least three colours. The \defn{acyclic chromatic number} $\chi_\text{a}(G)$ of a graph $G$ is the minimum integer $k$ such that $G$ has an acyclic $k$-colouring. Acyclic colourings are qualitatively different from colourings, since every finite graph with bounded acyclic chromatic number has bounded average degree. \citet{KY03} proved that $\chi_\text{a}(G)\leq \col_2(G)$ for every graph $G$. Other examples include game chromatic number \citep{KT94,KY03}, Ramsey numbers \citep{CS93}, oriented chromatic number \citep{KSZ-JGT97}, arrangeability~\citep{CS93}, etc. 

Generalised colouring numbers are important also because they characterise bounded expansion classes \citep{Zhu09}, they characterise nowhere dense classes \citep{GKRSS18}, and have several algorithmic applications such as the constant-factor approximation algorithm for domination number by \citet{Dvorak13}, and the almost linear-time model-checking algorithm of \citet{GKS17}. 

A graph class $\GG$ has \defn{linear colouring numbers} if there is a constant $c$ such that $\col_r(G)\leq cr$  for every $G\in\GG$ and for every $r\in \NN$. For example, van~den~Heuvel~et~al.~\citep{HOQRS17} proved the following results (always for every $r\in\NN$): Every finite planar graph $G$ satisfies $\col_r(G) \leq 5r+1$. More generally,  every finite graph $G$ with Euler genus $g$ satisfies $\col_r(G) \leq (4g+5)r + 2g+1$. Even more generally, for $t\geq 4$, every finite $K_t$-minor-free graph $G$ satisfies $\col_r(G) \leq \binom{t-1}{2} (2r+1)$. The proofs of these results depend on the following lemma, which we will use in our proof of \cref{KtMinorFreeNoSubdiv}.

\begin{lem}[\citep{HOQRS17}]
\label{PartitionGenColNum}
Let $G$ be a finite graph that has a connected partition $\PART=\{A_1,A_2,\dots,A_n\}$ such that for each part $A_i\in\PART$, there are at most $d$ neighbouring parts $A_j\in N_{G/\PART}(A_i)$ with $j<i$, and $V(A_i)$ is the union of the vertex-sets of $p$ geodesic paths in $G-(A_1\cup\dots\cup A_{i-1})$. Then $\col_r(G) \leq p(d+1)(2r+1)$.
\end{lem}

We need the following rooted variant of $r$-colouring number. For a graph $G$ and $r\in\NN$, the \defn{rooted $r$-colouring number} $\col^*_r(G)$ of $G$ is the minimum integer such that for every ordered clique $C$ of $G$ there is a total order $\preceq$ of $V(G)$ with $C$ at the start, such that $|\sreach_r(G,\preceq,v)| \leq \col^*_r(G)$ for every vertex $v$ of $G$. The next lemma is the motivation for this definition. 

\begin{lem}
\label{GenColourTreeDecomp}
For every graph $U$ and graph $G\in\DD(U)$ and $r\in\NN$, 
$$\col^*_r(G)\leq \col^*_r(U).$$
\end{lem}

\begin{proof}
By assumption, $G$ has a tree-decomposition $(B_x:x\in V(T))$ such that for each node $x\in V(T)$, the torso $G_x$ of $x$ is isomorphic to a subgraph of $U$. Let $C$ be a clique of $G$. 
Root $T$ at a node $s$ such that $C\subseteq B_s$. Such a node exists by the Helly property of trees. For each vertex $v\in V(G)$, let $h(v)$ be the node of $T$ closest to $s$ and with $v\in B_{h(v)}$. Let $\preceq_T$ be any total order on $V(T)$, where $x\prec_T y$ whenever $y$ is a descendant of $x$. Since $G_x\subseteq U$, we have $\col^*_r(G_x)\leq\col^*_r(U)$. 
Let $\preceq_s$ be a total order on $V(G_s)$ with $C$ at the start, and $|\sreach_r(G_x,v,\preceq_s)| \leq \col^*_r(G_x) \leq \col^*_r(U)$ for every vertex $v\in V(G_s)$. We now define a total order $\preceq_x$ on $V(G_x)$ for each non-root node $x\in V(T)$. Consider each non-root node $x$ of $T$ in non-decreasing order of distance from $s$. Let $y$ be the parent of $x$. So we may assume that $\preceq_y$ is already defined. Since $G_x$ is isomorphic to a subgraph of $U$ and $B_x\cap B_y$ is a clique of $G_x$, there is a total order $\preceq_x$ of $V(G_x)$ with $B_x\cap B_y$ at the start of $\preceq_x$ and ordered according to $\preceq_y$, such that $|\sreach_r(G_x,\preceq_x,v)| \leq \col^*_r(G_x) \leq \col^*_r(U)$ for every vertex $v$ of $G_x$. Finally, define a total order $\preceq$ on $V(G)$, where $v \preceq w$ if and only if $h(v) \prec_T h(w)$, or $h(v)=h(w)$ and $v\preceq_{h(v)} w$. Clearly $\preceq$ is a total order on $V(G)$.

Consider $w\in\sreach_r(G,\preceq,v)$ for some $v\in V(G)$. Thus $G$ contains a path $P=(v,w_1,\dots,w_k,w)$ of length at most~$r$ with $w\preceq v\preceq w_i$ for each $i\in[k]$. Let $x:=h(v)$. Since $v\preceq w_i$ we have $x \preceq_T h(w_i)$, and $h(w_i)$ is not an ancestor of $x$. Since $v,w_1,\dots,w_k$ is a path in $G$, we have $h(w_1),\dots,h(w_k)$ are all in the subtree of $T$ rooted at $x$. 
By the definition of tree-decomposition, $w$ and $w_k$ appear in a common bag. Thus $h(w)$ is an ancestor of $h(w_k)$ or vice versa. Since $w\preceq w_k$, we have $h(w) \preceq_T h(w_k)$. That is, $h(w)$ is an ancestor of $h(w_k)$, or $h(w)=h(w_k)$. Thus $w\in B_{h(w_k)}$. Since $w\preceq v$, we have $h(w) \preceq_T x$ and $x$ is not an ancestor of $h(w)$. Hence, $x$ is on the path from $h(w)$ to $h(w_k)$ in $T$. Since $w\in B_{h(w_k)}$, by the definition of tree-decomposition,  $w$ is in every bag in the path from $h(w)$ to $h(w_k)$ in~$T$. In particular, $w$ is in $B_x$. 
This says that both endpoints of $P$ are in $B_x$. If some vertex in $P$ is not in $B_x$, then for some child node $y$ of $x$, $P$ contains a subpath $(w_i,\dots,w_j)$, where $w_i,w_j$ are distinct vertices in $B_x\cap B_y$. Replace the subpath $(w_i,\dots,w_j)$ of $P$ by the edge $w_iw_j$, which is in the torso $G_x$. Do this whenever $P$ contains a vertex not in $B_x$. We obtain a path from $v$ to $w$ of length at most $r$ in $G_x$. Since $G_x$ is ordered by $\preceq_x$ in $\preceq$, we have $w\in \sreach_r(G_x,\preceq_x,v)$. Hence
$\sreach(G,\preceq,v) \subseteq  \sreach_r(G_x,\preceq_x,v)$ and
$|\sreach(G,\preceq,v)| \leq |\sreach_r(G_x,\preceq_x,v)| \leq \col^*_r(G_x) \leq \col^*_r(U)$. Therefore $\col_r(G)\leq \col^*_r(U)$.
\end{proof}

Note that \cref{GenColourTreeDecomp} with $U=K_{\tw(G)+1}$ shows 
that $\col^*_r(G) \leq \tw(G)+1$. This implies that $\col_r(G)\leq \tw(G)+1$, as proved by \citet{GKRSS18}. 

\begin{lem}
\label{MakeRootedColNum}
For every graph $G$ and $r\in\NN$, 
$$\col^*_r(G) \leq \col_r(G)+ \omega(G).$$ 
\end{lem}
 
\begin{proof}
Let $\preceq$ be a total order of $V(G)$ such that $|\sreach_r(G,\preceq,v)| \leq \col_r(G)$ for every vertex $v$ of $G$. Let $C$ be an ordered clique of $G$. Let $\preceq'$ be the total order of $V(G)$ obtained from $\preceq$ by placing $C$ at the start in the given order. For every vertex $v$ of $G$, we have $\sreach_r(G,\preceq',v) \subseteq \sreach_r(G,\preceq,v) \cup C$, which has cardinality at most $\col_r(G)+\omega(G)$.
\end{proof}

We now apply K\"onig's Lemma to show that $\col_r$ is extendable.

\begin{lem}
\label{colExtendable}
$\col_r$ is extendable for every $r\in\NN$.
\end{lem}

\begin{proof}
Let $G$ be a graph, such that for some $c\in\NN$ for every finite subgraph $H$ of $G$, we have $\col_r(H)\leq c$. Since $G$ is countable, we may assume that $V(G)=\{v_1,v_2,\dots\}$. Let $H_n:=G[\{v_1,\dots,v_n\}]$ for $n\in\NN$. For $n\in\NN$, let $V_n$ be the set of total orders $\preceq$ of $V(H_n)$ such that $|\sreach_r(H_n,\preceq,v)|\leq c$ for every $v\in V(H_n)$. By assumption, $V_n\neq\emptyset$. By construction, $V_n$ is finite. Let $Q$ be the graph with vertex set $\bigcup_{n\in\NN} V_n$ where $\preceq_n$ in $V_n$ is adjacent to $\preceq_{n-1}$ in $V_{n-1}$ if $x\preceq_{n-1} y$ if and only if $x\preceq_n y$ for all vertices $x,y\in V(H_{n-1})$. So every vertex in $V_n$ has a neighbour in $V_{n-1}$. By \cref{Konig}, there exist $\preceq_1,\preceq_2,\dots$ where $\preceq_n$ is in $V_n$ for each $n\in \NN$, and $\preceq_n$ is adjacent to $\preceq_{n-1}$ for each $n\geq 2$. Define the relation $\preceq$ on $V(G)$, where $x\preceq y$ whenever $x\preceq_n y$ for some $n\in\NN$. By the definition of adjacency in $Q$, we have $x\preceq y$ if and only if $x\preceq_n y$ for all $n\in \NN$ with $x,y\in V(H_n)$. Call this property $(\star)$. 

We now show that $\preceq$ is a total order of $V(G)$. 
%
Say $x\preceq y$ and $y\preceq z$, where $x=v_i$ and $y=v_j$ and $z=v_k$. %
Let $n:=\max\{i,j,k\}$. So $x,y,z\in V(H_n)$. 
By property $(\star)$, we have $x\preceq_n y$ and $y\preceq_n z$.
By the transitivity of $\preceq_n$, we have $x \preceq_n z$. 
By property $(\star)$, we have $x \preceq z$.
Thus $\preceq$ is transitive. 
%
Say $x\preceq y$ and $y\preceq x$, where $x=v_i$ and $y=v_j$. 
Let $n:=\max\{i,j\}$. So $x,y\in V(H_n)$. 
By property $(\star)$, we have $x\preceq_n y$ and $y\preceq_n y$.
By the antisymmetry of $\preceq_n$, we have $x =y$. 
Hence $\preceq$ is antisymmetric. 
%
Consider vertices $x,y\in V(G)$, where $x=v_i$ and $y=v_j$. 
Let $n:=\max\{i,j\}$. So $x,y\in V(H_n)$. 
By the connexity of $\preceq_n$, we have $x\preceq_n y$ or $y\preceq_n x$.
By property $(\star)$, we have $x\preceq y$ or $y\preceq x$.
Hence $\preceq$ is connex. 
%
Therefore $\preceq$ is a total order of $V(G)$.

It remains to show that $|\sreach_r(G,\preceq,v)| \leq c$ for each vertex $v\in V(G)$. 
Let $N$ be the set of vertices at distance at most $r$ from $v$ in $G$. 
So $\sreach_r(G,\preceq,v) \subseteq N$. 
Let $n:= \max\{ i : v_i \in N\}$. 
Thus $\sreach_r(G,\preceq,v)= \sreach_r(G,\preceq_n,v)$, and 
$|\sreach_r(G,\preceq,v)|= |\sreach_r(G,\preceq_n,v)| \leq c$, as desired. \end{proof}
 
\begin{lem}
\label{GenColourProduct}
For every graph $H$ and path $P$, and for every $r\in\NN$ and $a\in\NN_0$, 
\begin{equation*}
\col_r(( H\boxtimes P) + K_a) \leq (\tw(H)+1)(2r+1) + a.
\end{equation*}
\end{lem}

\begin{proof}
A result of van den Heuvel and Wood~\citep[Lemma~30, arXiv~version]{vdHW18} implies $\col_r(H\boxtimes P) \leq (\tw(H)+1)(2r+1)$ for finite $H$ and $P$. 
By \cref{colExtendable}, the result holds for infinite graphs. The claimed result follows by placing a copy of $K_a$ at the start of the corresponding vertex ordering of $H\boxtimes P$. 
\end{proof}

The next lemma will be used in \cref{ExcludedMinors} to show that our graph that contains every $X$-minor-free graph has linear colouring numbers. 

\begin{lem}
\label{GenColourProductTreeDecomp}
Fix $r\in\NN$ and $a\in\NN_0$. For every graph $H$ and path $P$, and for every graph $G$ in $\DD( (H\boxtimes P)+K_a )$, 
$$\col^*_r(G) \leq  (\tw(H)+1)(2r+3)+2a.$$
\end{lem}

\begin{proof}
Note that $\omega((H\boxtimes P)+K_a) = a + 2\omega(H) \leq a + 2(\tw(H)+1)$. By \cref{MakeRootedColNum,GenColourProduct}, $\col^*_r( (H\boxtimes P)+K_a) \leq (\tw(H)+1)(2r+3)+2a$. The claim then follows from \cref{GenColourTreeDecomp}.
\end{proof}

%%%%%%%%%%%%%
\subsection{Bounded Expansion}
\label{Expansion}

For $r\in \mathbb{N}$, a graph $H$ is an \hdefn{$r$}{shallow minor} of a graph $G$ if there is a model $(X_v)_{v\in V(H)}$ of $H$ in $G$ such that each subgraph $X_v$ has radius at most $r$. \citet{Sparsity} introduced the following definition (for finite graphs $G$). Let 
$$\nabla_r(G):=\sup \left\{\frac{2|E(H)|}{|V(H)|} : \text{$H$ is a finite $r$-shallow minor of $G$ with $V(H)\neq\emptyset$}\right\}.$$
We only consider finite $H$ in this definition, since average degree is not well-defined for infinite graphs.

A graph class $\GG$ has \defn{bounded expansion} with \defn{bounding function} $f$ if $\nabla_r(G)\leq f(r)$ for each $G\in\GG$ and $r\in\mathbb{N}$. We say $\GG$ has \defn{linear expansion} if, for some constant $c$, for all $r\in\NN$, every graph $G\in\GG$ satisfies $\nabla_r(G) \leq cr$. Similarly, $\GG$ has \defn{polynomial expansion} if, for some constant $c$, for all $r\in\NN$, every graph $G\in\GG$ satisfies $\nabla_r(G) \leq cr^c$. For example, when $f(r)$ is a constant, $\GG$ is contained in a proper minor-closed class. As $f(r)$ is allowed to grow with $r$ we obtain larger and larger graph classes.
 
\citet{DN16} noted that a result of \citet{PRS94} implies that graph classes with polynomial expansion admit strongly sublinear separators. \citet{DN16} proved the converse:  A hereditary class of graphs admits strongly sublinear separators if and only if it has  polynomial expansion. See \citep{Dvorak16,Dvorak18,ER18} for more results on this theme. 

%\david{Hakimi proved that every finite graph $G$ has an orientation with in-degree at most $\ceil{\mad(G)/2}$. So the constant factor above is 2, right? Konig's Lemma implies that Hakimi's result generalises to: every (infinite) graph $G$ has an orientation with in-degree at most $\sup\ceil{\mad(H)/2}$, where $H$ ranges over all finite subgraphs of $G$. Is this worth mentioning? See \url{https://11011110.github.io/blog/2019/01/17/orientations-infinite-graphs.html}}

Sergey Norin observed the following connection between colouring numbers and expansion in finite graphs; see~\citep{ER18}. Since every finite $r$-shallow minor of a graph $G$ is a minor of a finite subgraph of $G$, the result for infinite graphs immediately follows. 

\begin{lem}
\label{SergeyCorollary}
For every graph $G$ and $r \in \NN$, 
$$\nabla_r(G) \leq 2 \col_{4r+1}(G).$$
\end{lem}

%%%%%%%%%%%%%%%%%%%%%%%%%%%%%%%%%%%%%%%%%%%%%
\subsection{Planar Triangulations}
\label{Triangulations}

This subsection introduces some elementary notions regarding planar graphs that form a foundation for the results in \cref{ForcingMinor}. A \defn{plane graph} is a graph embedded in the plane, i.e. drawn without crossings. A \defn{near-triangulation} is a 2-connected finite plane graph, in which each face, except possibly one, is bounded by a 3-cycle. The exceptional face is bounded by the \defn{outer cycle}. 

A graph $G$ is \defn{locally Hamiltonian} if for each vertex $v\in V(G)$, the subgraph $G[N_G(v)]$ has a Hamiltonian cycle. By definition, every locally Hamiltonian graph is locally finite. It is easily seen that every connected, locally Hamiltonian graph is 3-connected. 

A \defn{plane triangulation} is a connected locally Hamiltonian plane graph $G$ such that for each vertex $v\in V(G)$, there is a Hamiltonian cycle $C_v$ of $G[N_G(v)]$ such that $v$ is the only vertex of $G$ inside $C_v$ or $v$ is the only vertex of $G$ outside $C_v$. The subgraph of $G$ consisting of $v$, $C_v$, and the edges from $v$ to $C_v$ form a \defn{wheel centred at $v$}. A \defn{planar triangulation} is a planar graph that can be embedded in the plane as a plane triangulation. A planar triangulation $G$ is a \defn{triangulation of} a planar graph $H$ if $H$ is a spanning subgraph of $G$. 


\begin{lem}
\label{LocallyHamiltonianTriangulation}
Every connected locally Hamiltonian plane graph $G$ is a plane triangulation.
\end{lem}

\begin{proof}
For each vertex $v\in V(G)$, let $C_v$ be a Hamiltonian cycle through $N_G(v)$. Let $v\in V(G)$ and let $pq\in E(C_v)$. So $vpq$ is a triangle in $G$. Since $C_v-p-q$ is connected, every vertex in $C_v-p-q$ is inside $vpq$ or every vertex in $C_v-p-q$ is outside $vpq$. In both cases, $vp$ and $vq$ are consecutive in the cyclic ordering of edges incident to $v$ defined by the embedding. Thus the cyclic ordering of $N_G(v)$ defined by $C_v$ coincides with the cyclic ordering of $N_G(v)$ defined by the embedding. 
Without loss of generality, $v$ is inside $C_v$. Suppose for a contradiction that some other vertex $x$ is inside $C_v$. So $x$ is inside the triangle $vpq$ for some $pq\in E(C_v)$. Choose such a vertex $x$ to minimise the distance in $G$ from $x$ to $\{v,p,q\}$. Since $vp$ and $vq$ are consecutive in the cyclic ordering of edges incident to $v$, $v$ has no neighbour inside $vpq$. In particular, $vx\not\in E(G)$. Since $G$ is connected and by the choice of $x$, without loss of generality, $px\in E(G)$. Let $y$ be the neighbour of $p$ inside $vpq$, such that $pv$ and $py$ are consecutive in the cyclic ordering of edges incident to $p$ (which is well-defined since $x$ is a candidate). Since $C_p$ coincides with the cyclic ordering of neighbours of $p$, we have $vy\in E(G)$, which contradicts the fact that $v$ has no neighbour in $vpq$. Hence, $v$ is the only vertex inside $C_v$. And $G$ is a planar triangulation. 
\end{proof}

By definition, a planar triangulation is locally Hamiltonian and is thus 3-connected  (as observed above). This can be strengthened as follows. 

\begin{lem}
\label{Extended3Connected}
Every finite subgraph of a planar triangulation $G$ can be extended to a finite 3-connected subgraph of $G$.
\end{lem}

\begin{proof}
Let $H_0$ be a finite subgraph of $G$. Add shortest paths between the components of $H_0$ to obtain a finite connected subgraph $H_1$ of $G$. Let $H$ be obtained from $H_1$ by adding the wheel in $G$ centred at each vertex in $H_1$. Then $H$ is a finite 3-connected subgraph of $G$.
\end{proof}

This lemma is in sharp contrast to the fact that there exist planar graphs of arbitrarily large (finite) connectivity such that no finite subgraph is 3-connected\footnote{For example, let $G_k$ be a planar graph containing cycles $C_0,C_1, \ldots$ drawn as concentric circles in the plane, such that for $i\in\NN_0$ each vertex in $C_i$ is adjacent to at least $k$ vertices in $C_{i+1}$, and for $i\in\NN$ each vertex in $C_i$ is adjacent to at most one vertex in $C_{i-1}$ and some vertex in $C_i$ is adjacent to no vertex in $C_{i-1}$. Then $G_k$ is $(k+2)$-connected but contains no 3-connected finite subgraph.}.

A locally finite graph $G$ is \hdefn{$k$}{ended} if $k$ is the maximum integer such that $G-S$ has $k$ infinite components for some finite set $S\subseteq V(G)$. 

%Every 1-ended planar triangulation has the following extension property: 
%
% \begin{lem}
% \label{ExtendToNearTriang}
% Every finite subgraph of a 1-ended planar triangulation $G$ can be extended to a finite near-triangulation in $G$. 
% \end{lem}
%
% \begin{proof}
% Let $H_0$ be a finite subgraph of $G$. First extend $H_0$ to a finite 3-connected subgraph $H$ using \cref{Extended3Connected}. Since $G$ is 1-ended, precisely one face of $H$ contains infinitely many vertices of $G$. Add all the other vertices of $G$ (and their incident edges) to $H$ to obtain a finite near-triangulation in $G$.
% \end{proof}


%%%%%%%%%%%%%%%%%%%%%%%%%%%%%%%%%%%%%%%%%%%%%%%%%%%%%%%%%%%%%%%
\section{Forcing a Minor or Subdivision}
\label{ForcingMinor}

This section proves out first main result, \cref{InfinteCliqueMinor}, which says that if a graph $U$  contains every planar graph, then (a) the complete graph $K_{\aleph_0}$ is a minor of $U$, and (b) $U$ contains a subdivision of $K_t$ for every $t\in\NN$. The proof is split across three subsections. In \cref{Limits} we introduce the notion of a `limit', which may be of independent interest, and we prove the `Limit Lemma' (\cref{LimitLemma}), which shows that any graph that contains uncountably many planar graphs of a certain type contains a planar triangulation along with an infinite number of `jump' paths. Then \cref{Routing} presents a number of lemmas about routing paths in graphs obtained from planar triangulations by adding  jumps. All of these results are then combined in \cref{ProofInfinteCliqueMinor}, where the proof of \cref{InfinteCliqueMinor} is completed. In fact, we prove significant strengthenings of both parts of \cref{InfinteCliqueMinor}. Finally, in \cref{ExcludingSubdivision} we show that numerous graph classes do \emph{not} have a universal element, including $k$-apex graphs, chordal graphs containing no $K_{\aleph_0}$, graphs with no $K_{\aleph_0}$ minor (which was proved in \cite{DHV85}), and graphs containing no $K_{\aleph_0}$ subdivision (which was left open in \cite{DHV85}).


%%%%%%%%%%%%%%%%%%%%%%%%%%%%%%
\subsection{Limit Lemma}
\label{Limits}

 Let $G$ be a countably infinite graph. Let $\GG$ be an uncountable set of infinite, locally finite, connected subgraphs of $G$. A \hdefn{$\GG$}{limit} in $G$ is any subgraph of $G$ that can be obtained as follows.  Let $v_0$ be any vertex in $G$ that is contained in uncountably many distinct subgraphs in $\GG$ (which exists since $\GG$ is uncountable and $G$ is countable). Denote this subset of $\GG$ by $\GG_0$. Define an equivalence relation on $\GG_0$, where two graphs in $\GG_0$ are equivalent if they contain the same set of edges incident to $v_0$. By the choice of $v_0$ and since every graph in $\GG$ is connected, this set of edges is nonempty. The number of equivalence classes is countable since every graph in $\GG$ is locally finite. Since $\GG_0$ is uncountable, some equivalence class is uncountable. Call this equivalence class $\GG_1$. Let $v_1,v_2, \ldots$ be the remaining vertices of $G$. We now define a sequence of uncountable families $\GG_0\supseteq \GG_1\supseteq \GG_2\supseteq\dots$. Suppose that $\GG_n$ is defined for some $n\in\NN$. Define an equivalence relation on $\GG_n$, where two graphs in $\GG_n$ are equivalent if they contain the same set of edges incident to $v_n$ (which may be empty). Again, the number of equivalence classes is countable since every graph in $\GG$ is locally finite. Let $\GG_{n+1}$ be an uncountable equivalence class. Since every graph in $\GG$ is connected, $v_n$ is contained in all or none of the graphs in $\GG_{n+1}$. Let $H$ be the graph with vertex set $V(G)$ such that for every $n\ge0$, the set of edges incident with $v_n$ in $H$ is the set of edges incident with $v_n$ in each graph in $\GG_{n+1}$. We call $H$ a \DefNoIndex{$\GG$-limit} in $G$. 

By construction of $\GG$-limits, each finite induced subgraph of $H$ is an induced subgraph of every graph in $\GG_n$ for some sufficiently large $n$. This implies that $H$ inherits many properties of $\GG$ (in the spirit of extendable properties introduced in \cref{Definitions}). For example, by \cref{PlanarityExtendable}, if every graph in $\GG$ is planar, then $H$ is planar. Every degree of a vertex in $H$ is also the degree of some vertex of a graph in $\GG$. However, there is one important property that need not be preserved, namely connectedness. To see this, let $G$ be obtained from the infinite ladder $K_2\CartProd \PPP$ by replacing each edge by two internally disjoint paths of length 2, and let $\GG$ be the set of 2-way infinite paths of $G$. Then the $\GG$-limits in $G$ are precisely those spanning subgraphs that are either a 2-way infinite path plus infinitely many isolated vertices, or the union of two disjoint 2-way infinite paths. Therefore we shall focus on a connected component of a limit. Some components may consist of isolated vertices. It is easy to see that every finite component of a limit graph (if any) is trivial (that is, an isolated vertex). But, the component containing $v_0$ does contain edges and is therefore infinite. 

%Is every non-trivial connected component of a $\mathcal{F}'$-limit isomorphic to a graph in $\mathcal{F'}$? No. e.g. $G$ is doubled 2-way infinite path, and $\FF'$  is the 1-way infinite path, then the limits are the 2-way infinite paths in $G$, which are not in $\FF'$ }


The property of being 1-ended need not be preserved under taking limits. For example, if $G$ is the 2-way infinite path with edges doubled and $\GG$ is the set of all 1-way infinite paths of $G$, then the $\GG$-limits in~$G$ are all 1-way paths (together with isolated vertices) and all 2-way infinite paths, and the latter are 2-ended.

Therefore we introduce the following stronger property: for a function $f:\NN\to\NN$, a plane graph $G$ is \hdefn{$f$}{isoperimetric} if $G$ has at most $f(n)$ vertices inside every cycle of length $n$. A class $\GG$ of plane graphs is \DefNoIndex{$f$-isoperimetric} if every graph in $\GG$ is $f$-isoperimetric. A planar triangulation $G$ is \DefNoIndex{$f$-isoperimetric} if $G$ has an $f$-isoperimetric embedding in the plane. For example, every triangulation of  the infinite grid $\PPP\CartProd \PPP$ is $f$-isoperimetric for some $f$ with $f(n) \in  O(n^2)$; see \citep{Thomassen17}. As another example, it follows from Euler's formula that every plane triangulation with minimum degree at least 7 is $f$-isoperimetric for some $f$ with $f(n) \in  O(n)$. 

%Let $Q$ be a near-triangulation with $n$ vertices on the outerface, and $n'$ internal vertices, each with degree at least 7. Let $Q'$ be the planar triangulation obtained from $Q$ by adding one vertex adjacent to all the vertices on the outerface of $Q$. So $Q'$ has $n+n'+1$ vertices and $3(n+n'+1)-6$ edges. Now $$6(n+n'+1)-12 = 2|E(Q')| = \sum_{v\in V(Q')} \deg(v) \geq n + 3n + 7n'$$ So $2n-6 \geq  n'$. This argument uses that the vertices on the outerface of $Q$ have degree at least 3 in $Q'$. We can push further to show that the average degree of such vertices is more than 3

%\david{Does an infinite planar triangulation have infinite treewidth if and only if it is $f$-isoperimetric for some $f$? Intuition (each sentence needs proof): Say $G$ is a planar triangulation with finite treewidth $k$. Then there is a bag $B$ of the tree-decomposition such that $G-B$ has at least two infinite components. Thus $B$ contains a cycle $C$ with infinitely many vertices inside $C$ and infinitely many vertices outside $C$. So $G$ is $f$-isoperimetric for no $f$. Perhaps the converse can be proved using treewidth--tangle duality. I am not saying this should be included in the paper. Just wondering if it is true or known?}

Note that every $f$-isoperimetric infinite planar triangulation $G$ is 1-ended. To see this, suppose for the sake of contradiction that $G-S$ has at least two infinite components for some finite $S\subseteq V(G)$. As noted earlier, $G$ contains a finite 3-connected subgraph $H$ containing $S$. Since $G-V(H)$ has at least two infinite components, at least two faces of $H$ contain infinitely many vertices of $G$. One of these faces is the interior of a cycle $C$ of length, say $n$, in $H$. But this contradicts that $G$ has at most $f(n)$ vertices inside $C$.

% \begin{lem}[Limit Lemma with Some Fixed Vertices]
% \label{LimitLemma}
% Fix $n\in\NN$ and an arbitrary function $f:\NN\to\NN$. Let $\FF$ be any (possibly finite) class of 4-connected $f$-isoperimetric planar triangulations, such that each $F\in\FF$ has specified vertices $y^F_1,\dots,y^F_n$, and for all $F,F'\in\FF$ we have $y^{F}_iy^{F}_j\in E(F)$ if and only if $y^{F'}_iy^{F'}_j\in E(F')$. Let $G$ be a countable graph and let $x_1,\dots,x_n$ be vertices in $G$, such that there is an uncountable set $\GG$ of subgraphs of $G$, such that for each $G'\in\GG$, there is some $F\in \FF$ and there is an isomorphism from $G'$ to $F$ that maps $x_i$ to  $y^F_i$ for each $i\in[n]$. Let $L$ be a non-trivial connected component of an $\FF$-limit in $G$ with respect to $\GG$. Then $L$ is a 4-connected $f$-isoperimetric planar triangulation with infinitely many pairwise disjoint jumps in $G$. Moreover, every finite induced subgraph of $L$ is an induced subgraph of some graph in $\FF$, and $x_1,\dots,x_n\in V(L)$ and for all  $F\in\FF$ we have $x_ix_j\in E(L)$ if and only if $y^{F}_iy^{F}_j\in E(F)$. 
% \end{lem}

Let $H$ be a subgraph of a graph $G$. A \defn{jump} of $H$ in $G$ is a path $P$ in $G$  such that the endpoints of $P$ are non-adjacent distinct vertices in $H$ and no internal vertex of $P$ is in $H$. (A jump can be a single edge.)

\begin{lem}[Limit Lemma]
\label{LimitLemma}
Fix an arbitrary function $f:\NN\to\NN$. Let $G$ be a countably infinite graph. Let $\GG$ be an uncountable set of subgraphs of $G$, each of which is a 4-connected $f$-isoperimetric planar triangulation. Let $L$ be any non-trivial connected component of some $\GG$-limit in $G$.  Then $L$ is a 4-connected $f$-isoperimetric planar triangulation, and there are infinitely many pairwise disjoint jumps of $L$ in $G$. 
\end{lem}

\begin{proof}
Let $L_0$ be a $\GG$-limit in $G$ defined with respect to a sequence of vertices $v_0,v_1,\dots$ and a sequence of families $\GG_0 \supseteq \GG_1\supseteq \dots$ such that $L$ is a non-trivial connected component of~$L_0$. By construction, every finite induced subgraph of $L$ is an induced subgraph of uncountably many graphs in $\GG$. 
Since every graph in $\GG$ is planar, every subgraph of $L$ is planar, and $L$ is planar by \cref{PlanarityExtendable}. By the construction of a $\GG$-limit, for every vertex $v$ of $L$, for some $G'\in\GG$, the subgraph of $L$ induced by $N_L[v]$ is equal to the subgraph of $G'$ induced by $N_{G'}[v]$. So $L$ is locally Hamiltonian. By \cref{LocallyHamiltonianTriangulation}, $L$ is a planar triangulation. 

Assume that $L$ is drawn in the plane so that the vertex set is unbounded\footnote{We now prove that every infinite planar graph can be drawn in the plane so that the vertex set is unbounded. If the vertex set is bounded, then the graph has an accumulation point $p$. We may think of the graph being drawn on the sphere with $p$ as the north pole. Deleting $p$ from the sphere results in the graph drawn in the plane. A sequence of vertices converging to $p$ on the sphere now becomes a sequence of vertices tending to infinity in the plane. Note that, if $p$ is a vertex or is a point on an edge of the graph, then the graph can be redrawn, vertex by vertex, such that $p$ is never used and such that each new vertex in the new embedding is close to the vertex in the old embedding so that the vertex-accumulation points are the same. We refer to \cite{Mo88} for further results about infinite graphs on surfaces.}. We now show this embedding of $L$ is $f$-isoperimetric. Suppose that $C$ is a cycle of length $k$ in $L$, and $L$ has more than $f(k)$ vertices inside $C$. 
Let $C'$ consist of $C$ along with $f(k)+1$ vertices inside $C$ and also $f(k)+1$ vertices outside $C$ (which exist since $L$ is drawn so that the vertex set is unbounded). By \cref{Extended3Connected}, $C'$ is contained in a finite 3-connected subgraph $C''$ of $L$, which is a subgraph of some triangulation $T$ in $\GG$. Since $C''$ is 3-connected, it has a unique embedding in the plane up to the choice of the outer face. Thus $T$ is a plane triangulation in $\GG$ with more than $f(k)$ vertices inside $C$ (and outside $C$), which contradicts that $\GG$ is $f$-isoperimetric. Hence $L$ is $f$-isoperimetric. Since every graph in $\GG$ is 4-connected, we may assume that $f(3)=0$. Since $L$ is locally Hamiltonian, every separating set with three vertices in $L$ induces a triangle. Since $L$ is $f$-isoperimetric and $f(3)=0$, every triangle in $L$ is a face, implying that $L$ is 4-connected.

It remains to prove that $L$ has infinitely many pairwise disjoint jumps in $G$.  To prove this, it suffices to show that for all pairwise disjoint jumps $J_1, \dots ,J_p$ of $L$ in $G$, there exists $J_{p+1}$ such that $J_1, \dots, J_{p+1}$ are pairwise disjoint jumps of $L$ in $G$. 
%If $p=0$ then let $H$ be any wheel in $L$. Otherwise, by \cref{ExtendToNearTriang} there is a finite near-triangulation $H$ in $L$ containing the end-vertices of $J_1,\dots,J_p$. 
Let $H$ be the subgraph of $L$ induced by the end-vertices of $J_1,\dots,J_p$. 
Let $W$ be the set of internal vertices in $J_1,\dots,J_p$. So $W\subseteq V(G)\setminus V(L)$. 

Define the following finite subgraph $H'$ of $G$ disjoint from $L$. Consider each component $T'$ of $L_0$ that intersects $W$. If $|V(T')|=1$ then add this vertex as an isolated vertex to $H'$. Otherwise, $T'$ is a planar triangulation. Let $C_1(T')$ be a cycle in $T'$ whose interior contains $W\cap V(T')$. Let $C_2(T')$  be a cycle in $T'$ such that $C_1(T')$ is inside $C_2(T')$, and each vertex of $C_2(T')$ is adjacent to a vertex of $C_1(T')$. (Note that ``inside" is well-defined since $T'$ is 1-ended.) The near-triangulation consisting of $C_2(T')$ and all the vertices and edges of $T'$ inside $C_2(T')$ is called the \defn{protecting near-triangulation} of $T'$ (as it, intuitively, ``protects" those vertices of $W$ that are in $T'$). Add each such protecting near-triangulation as a component of $H'$.

Since $H\cup H'$ is finite, $V(H\cup H') \subseteq \{v_1,v_2, \dots ,v_n\}$ for some sufficiently large $n$. Since $\GG_{n+1}$ has uncountably many triangulations, $\GG_{n+1}$ contains at least one triangulation, say $T''$, distinct from $L$. We claim that $T''$ contains the required additional jump $J_{p+1}$.

To prove the claim, first observe that $T''$ contains an edge not in $L$ (since it is easy to prove that a 4-connected triangulation cannot contain a triangulation as a proper subgraph). Since $T''$ is connected and intersects $L$, $T''$ has an edge $uw$ not in $L$ such that $u$ is in $L$. Since $T''$ and $L$ agree on the set of edges in $G$ incident with each vertex in $H$, we have that $u$ is not in $H$. Also, $w$ is not in $H$ because $L$ and $T''$ agree on the edges incident to  vertices in $H$. 

If  $w$ is in $L$, then we may take $J_{p+1}=uw$ (since neither $u$ nor $w$ is in $H$), as claimed. Now assume that $w$ is not in $L$. 

Since $T''$ is 4-connected, by Menger's theorem, $T''-u$ has three paths $P_1,P_2,P_3$ from $w$ to $L$ which are pairwise disjoint except that they all contain $w$. Let $P_4$ be the edge $wu$. Each of $P_1.\dots,P_4$ have exactly one end-vertex in $L$. Observe that $P_1 \cup P_2 \cup P_3 \cup P_4$ does not intersect $H$ by the same argument that $u$ is not in $H$.  In particular, no end-vertex of $J_1, \dots, J_p$ is in $P_1 \cup P_2 \cup P_3 \cup P_4$.

\begin{claim}
\label{OnOuterCycle}
If $xy$ is an edge in $P_1\cup P_2 \cup P_3 \cup P_4$ and $x$ is in some protecting near-triangulation $N$ and $y$ is not in $N$, then $x$ is on the outer-cycle of $N$.
\end{claim}

\begin{proof}
Towards a contradiction suppose $x$ is not on the outer-cycle of $N$.  Then the wheel centred at $x$ in $N$ is also the wheel in $T''$ centred at $x$, by the choice of $n$. In particular, $y$ is also a neighbour of $x$ in $N$, which is a contradiction.   
\end{proof}

Note that if $x$ is an isolated vertex in $H'$, then $x$ is isolated in $L$ and in $T''$ (since $L$ and $T''$ agree on edges incident to vertices in $H'$). Since $P_1,P_2,P_3,P_4$ are in $T''$, no vertex in $P_1\cup P_2 \cup P_3 \cup P_4$ is isolated in $T''$. Thus no vertex in $P_1\cup P_2 \cup P_3 \cup P_4$ is isolated in $H'$. In particular, $w$ is not isolated in $H'$. It is possible that $w$ is in $H'$, in which case $w$ is on the outer cycle of one of the protecting near-triangulations by \cref{OnOuterCycle}.

Let $N$ be a protecting near-triangulation such that $V(N) \cap W \cap V(P_1 \cup P_2 \cup P_3) \neq \emptyset$. For each $i \in [3]$, let $x_i$ be the first vertex of $P_i$ (beginning from $w$) in $N$, and let $y_i$ be the last vertex of $P_i$ that is in $N$.  Note that if $|V(P_i) \cap V(N)|=1$, then $x_i=y_i$, and if $V(P_i) \cap V(N)=\emptyset$, then both $x_i$ and $y_i$ are undefined.  Let $X:=\{x_1, x_2, x_3\}$ and $Y:=\{y_1, y_2, y_3\}$.  Let $C$ be the outer cycle of $N$, $D$ be the outer cycle of $N -C$, and $N':=N[V(C) \cup V(D)]$.   By \cref{OnOuterCycle}, $X \cup Y \subseteq V(C)$.  Moreover, by construction, $W \cap V(N)$ is disjoint from $V(N')$.  Since $N'$ is $3$-connected, there are three vertex-disjoint paths from $X$ to $Y$ in $N'$. Therefore, by rerouting, we may assume that $V(N) \cap W \cap V(P_1 \cup P_2 \cup P_3) = \emptyset$. 

Repeating the above argument for each protecting near-triangulation $N$ such that $V(N) \cap W \cap V(P_1 \cup P_2 \cup P_3) \neq \emptyset$, we may assume that $P_1 \cup P_2 \cup P_3 \cup P_4$ is disjoint from $W$. Since $L$ is 4-connected, it contains no $K_4$. So there are distinct $i,j\in [4]$ such that the end-vertices of $P_i$ and $P_j$ in $L$ are non-adjacent in $L$. Thus, we may take $J_{p+1}=P_i\cup P_j$, which completes the proof.
\end{proof}


%%%%%%%%%%%%%%%%
\subsection{Routing with Jumps}
\label{Routing}

This subsection proves a series of lemmas about routing paths in graphs obtained by adding jumps to planar triangulations. These results are used in the proofs of \cref{InfinteCliqueMinorStrongerStronger,thm:finitesubdivision} below. 

Consider the grid graph $G=P_\ell\CartProd P_m$, where $P_\ell$ is the path $(x_1,\dots,x_\ell)$ and $P_m$ is the path $(y_1,\dots,y_\ell)$. For $i\in[\ell]$, the subpath $C^i:=((x_i,y_1),\dots,(x_i,y_m))$ is called the \DefNoIndex{$i$-th column}\index{column} of $G$. 

\begin{lem} \label{lem:cylinderrouting}
Let $G=P_\ell\CartProd P_m$ for some $\ell,m\in\NN$. Let $a_1, \dots, a_k$ be vertices (ordered from top to bottom) in the first column $C^1$ of $G$, and let $b_1, \dots, b_k$ be vertices (ordered from top to bottom) in the last column $C^m$ of $G$.  If $\ell \geq 2k+1$, then there exist $k$ vertex-disjoint $C^1$--$C^m$ paths $P^1, \dots, P^k$ in $G$ such that the end-vertices of $P^i$ are $a_i$ and $b_i$ for each $i \in [k]$.
\end{lem}

\begin{proof}
We proceed by induction on $k\geq 0$.  The case $k=0$ is vacuous. By symmetry, we may assume that $b_1$ is above $a_1$.  Let $P^1$ be the path starting at $a_1$ which proceeds one unit to the right, up to the row containing $b_1$, and then right until it reaches $b_1$.  For $i \in [2,k]$, let $Q^i$ be the path starting at $a_i$ and proceeding two units to the right to the third column $C^3$ of $G$, and let $a_i'$ be the end-vertex of $Q^i$ on $C^3$. By induction,  there exist $k-1$ vertex-disjoint $C^3$--$C^m$ paths $S^2, \dots, S^k$ in $G-V(P^1)$ such that the end-vertices of $S^i$ are $a_i'$ and $b_i$ for each $i \in [2,k]$. Defining $P^i:=Q^i \cup S^i$ for each $i \in [2,k]$ completes the proof. \end{proof}

The next lemma is a special case of the 2-linkage theorem, independently due to \citet{Thomassen80} and \citet{Seymour80}. For a graph $G$ and $u,v \in V(G)$,  let \defn{$G+uv$} be the graph obtained from $G$ by adding the edge $uv$ if it does not already exist. 

\begin{lem} \label{lem:special2linkage}
Let $G$ be a plane near-triangulation with outer-face $C$, $\{p_1,q_1,p_2,q_2\} \subseteq V(C)$, and $u$ and $v$ be non-adjacent vertices of $G$ such that neither $u$ nor $v$ is on $C$.  If $G$ does not contain a $(\leq 3)$-separation $(G_1, G_2)$ such that $\{p_1, q_1, p_2, q_2\} \subseteq V(G_1)$ and $G_1$ is planar, then there are vertex-disjoint paths $P$ and $Q$ in $G + uv$ such that the end-vertices of $P_1$ are $p_1$ and $p_2$ and the end-vertices of $Q$ are $q_1$ and $q_2$. 
\end{lem}

For a cycle $C$ in a plane graph $G$, let $\Delta(C) \subseteq \mathbb{R}^2$ be the closed disk bounded by $C$.  We say that $C$ is \defn{tight} if there is no cycle $D$ in $G$ such that $\Delta(C) \subseteq \Delta(D)$ and $|V(D)| < |V(C)|$. A cycle $D$ in $G$ \defn{surrounds} a cycle $C$ in $G$ if $V(C) \cap V(D)=\emptyset$ and $\Delta(C) \subseteq \Delta(D)$.

\begin{lem} \label{FindDisjointTightCycle}
For every cycle $C$ of an infinite $f$-isoperimetric plane triangulation $G$, there is a tight cycle that surrounds $C$.
\end{lem}

\begin{proof}
Let $\CC$ be the set of cycles in $G$ with vertex set $V(C)$. (Chords of  $C$ can create cycles in $\CC$ distinct from $C$.)\ Since $G$ is locally finite, $\CC$ is finite. Let $V$ be the set of vertices of $G$ in $\bigcup_{D\in\CC} \Delta(D)$. Since $G$ is $f$-isoperimetric, $V$ is finite. Let $C^1$ be the facial cycle of $G-V$ bounding $V$. So $C^1$ surrounds $C$. If $C^1$ is tight, then $C^1$ satisfies the claim. Otherwise, there is a cycle $C^2$ with $\Delta(C^1)\subseteq \Delta(C^2)$ and $|V(C^2)|<|V(C^1)|$. So $C^2$ surrounds $C$. If $C^2$ is tight, then $C^2$ satisfies the claim. Repeating this argument we obtain a sequence $C^1,C^2,\dots$ of non-tight cycles with $\Delta(C^i) \subseteq \Delta(C^{i+1})$ and $|V(C^{i+1})|<|V(C^i)|$, where each $C^i$ surrounds $C$.  Say $C^i$ has length $\ell_i$ and $n_i$ vertices of $G$ are in $\Delta(C^i)$. Since $|V(C^{i+1})|<|V(C^i)|$, we have $C^i\neq C^{i+1}$ and $n_{i+1}> n_i$. So $\ell_i \leq \ell_1$ (in fact, $\ell_i \leq \ell_1 - i +1$, but we will not use that). Moreover, $n_1 + i -1 \leq n_i \leq f(\ell_i) \leq f( \ell_1)$ (since we may assume that $f$ is a non-decreasing function). With $i=f(\ell_1)+1$, we have $n_1\leq 0$, which is a contradiction.
\end{proof}

\begin{lem} \label{claim:tightcycle}
Let $G$ be a plane graph, let $C$ be a tight cycle in $G$, and let $D$ be a cycle in $G$ such that $\Delta(C) \subseteq \Delta(D)$. Then there are $|V(C)|$ vertex disjoint paths in $G$ from $V(C)$ to $V(D)$.
\end{lem}

\begin{proof}
Suppose that the desired paths do not exist. By Menger's Theorem, there exists $X \subseteq V(G)$ with $|X| < |V(C)|$ such that there is no path in $G-X$ between $V(C)$ and $V(D)$.  Since $G$ is a triangulation, there is a cycle $C'$ between $C$ and $D$ with $V(C')=X$.  But now, $C'$ contradicts that $C$ is tight.  
\end{proof}

Repeatedly applying \cref{claim:tightcycle} we obtain the following.

\begin{lem} \label{claim:cylindricalgrid}
Let $C^1, \dots, C^m$ be tight cycles in a plane graph $G$ such that $|V(C^1)|=\ell$, and $C^j$ surrounds $C^i$ for all $1 \leq i < j \leq m$. Then $G$ contains a subdivision of $C_\ell \CartProd P_m$, where the branch vertices of the $i$-th cycle of $C_\ell \CartProd P_m$ are in $C^i$.
\end{lem}

\begin{lem} \label{claim:switchpaths}
Let $G$ be a $4$-connected $f$-isoperimetric plane triangulation, and let $C$ be a tight cycle in $G$ of length $\ell \geq 2k+4$. Let $x_1, \dots, x_k$ be vertices in clockwise order around $C$.  Then there is a tight cycle $D$ surrounding $C$, and there are vertex-disjoint paths $Q^1, \dots, Q^k$ from $C$ to $D$ such that:
\begin{itemize}
    \item $x_i$ is the unique vertex of $Q^i$ on $C$ for each $i \in [k]$,
    \item if $y_i$ is the other end-vertex of $Q^i$ for each $i\in[k]$, then $y_2, y_1, y_3, \dots, y_k$
    appear in this clockwise order on $D$,
    \item $Q^1 \cup \dots \cup Q^k$ is contained in the subgraph of $G$ between $C$ and $D$ together with one edge $e \in M$.
\end{itemize}
\end{lem}

\begin{proof}
 By \cref{FindDisjointTightCycle}, there are tight cycles $C=C^1, C^2, \dots, C^{2k+1}$ such that $C^j$ surrounds $C^i$ for all $1 \leq i < j \leq 2k+1$; see \cref{SwitchPaths}.  Since $G$ is $f$-isoperimetric, there are only finitely many vertices of $G$ inside $C^{2k+1}$.  By \cref{FindDisjointTightCycle} and since $M$ is infinite, there is a tight cycle $C^{2k+2}$ surrounding $C^{2k+1}$ and there is an edge $vw \in M$ such that $v \in \Delta(C^{2k+2})$, and $\{v,w\} \cap \Delta(C^{2k+1})=\emptyset$. By \cref{FindDisjointTightCycle}, there exists a tight cycle $C^{2k+3}$ surrounding $C^{2k+2}$ such that $w \in \Delta(C^{2k+3})$ and $w \notin V(C^{2k+3})$.  Let $C^{2k+4}$ be a tight cycle surrounding $C^{2k+3}$ with $\Delta(C^{2k+4})$ minimal (under inclusion).

\begin{figure}[!ht]
\centering
\includegraphics[width=\textwidth]{NewSwitch}
\caption{Illustration for the proof of \cref{claim:switchpaths}.\label{SwitchPaths}}
\end{figure}

By \cref{claim:cylindricalgrid}, $G$ contains a subgraph $H$ that is a subdivision of $C_\ell \CartProd P_{2k+4}$, where the branch vertices of the $i$-th cycle of $C_\ell \CartProd P_{2k+4}$ are contained in $C^i$. Choose such an $H$ with $|V(H)|$ minimum.  Let $P^1, \dots, P^\ell$ be the $C^1$--$C^{2k+4}$ paths of the subdivision (ordered clockwise). For $i,j \in [\ell]$, let $G(i,j)$ be the subgraph of $G$ induced by the vertices between $C^{2k+1}$ and $C^{2k+4}$ (inclusive) and between $P^i$ and $P^j$ (inclusive and clockwise from $P^i$ to $P^j$). Since $\ell \geq 2k+4$, there exist $a,b \in [\ell]$ such that $v$ and $w$ are both in $G(a,b)$, and there are at least $k$  branch vertices of $H$ in $V(C^{2k+1}) \setminus V(G(a,b))$. By symmetry, we may assume $1<a<b<\ell$.  For each $i \in [\ell]$, let $s_i$ and $t_i$ be the unique vertices of $P^i$ on $C^{2k+1}$ and $C^{2k+4}$, respectively.  Let $z_1, z_2, \dots, z_k$ be ordered clockwise on $C^{2k+1}$ where $z_1:=s_{a-1}$, $z_2:=s_{b+1}$, and $z_3, \dots, z_k$ are branch vertices of $H$ contained in $V(C^{2k+1}) \setminus V(G(a-1,b+1))$. 
Let $H':=H-(V(C^{2k+2}) \cup V(C^{2k+3}) \cup V(C^{2k+4}))$.  Note that $H'$ is a subdivision of $C_\ell \CartProd P_{2k+1}$, which is a supergraph of $P_\ell \CartProd P_{2k+1}$.  Therefore,  by \cref{lem:cylinderrouting} there exist vertex-disjoint $C^1$--$C^{2k+1}$ paths $R^1, \dots, R^k$ in $H'$ such that the end-vertices of $R^i$ are $x_i$ and $z_i$ for all $i \in [k]$. For $i \in [3,k]$ we use $H$ to extend $R^i$ to a path $Q^i$ with an end-vertex $y_i$ in $C^{2k+4}$.  Let $y_1:=t_{a-1}$ and $y_2:=t_{b+1}$. 

Below we prove there exists vertex-disjoint paths $Q_1$ and $Q_2$ in $G(a-1,b+1) +vw$ such that the end-vertices of $Q_1$ are $z_1$ and $y_2$ and the ends of $Q_2$ are $z_2$ and $y_1$. Use $Q_1$ to extend $R^1$ to a path $Q^1$ with $y_2$ as an end-vertex, and use $Q_2$ to extend $R^2$ to a path $Q^2$ with $y_1$ as an end-vertex, as desired. It remains to prove the existence of the paths $Q_1$ and $Q_2$ in $G(a-1, b+1)+vw$. 

By construction, neither $v$ nor $w$ is on the outer-cycle $O$ of $G(a-1,b+1)$.  We claim that $G(a-1, b+1)+vw$ does not contain a $(\leq 3)$-separation $(G_1, G_2)$, such that $\{s_{a-1}, t_{a-1}, s_{b+1}, t_{b+1}\} \subseteq V(G_1)$ and $G_1$ is planar.  Suppose for the sake of contradiction that $(G_1, G_2)$ is such a separation.  Since $G$ is $4$-connected, $(G_1, G_2)$ is a $2$- or $3$-separation.  First suppose it is a $2$-separation, say $V(G_1) \cap V(G_2)=:\{x,y\}$.  Since $G$ is a plane triangulation, $xy$ is a chord of $O$. Since $\{s_{a-1}, t_{a-1}, s_{b+1}, t_{b+1}\} \subseteq V(G_1)$, we have $\{x,y\} \subseteq V(C^{2k+4}), \{x,y\} \subseteq V(P^{a-1}), \{x,y\} \subseteq V(P^{b+1})$, or $\{x,y\} \subseteq V(C^{2k+1})$.  Since $C^{2k+1}$ is tight, $\{x,y\} \not\subseteq V(C^{2k+1})$.  By the minimality of $|V(H)|$, $\{x,y\} \not \subseteq V(P^{a-1})$ and $\{x,y\} \not \subseteq V(P^{b+1})$.  By the minimality of $\Delta(C^{2k+4})$, $\{x,y\} \not \subseteq V(C^{2k+4})$.
Thus, we may assume that $(G_1, G_2)$ is a $3$-separation, say $V(G_1) \cap V(G_2)=:\{x,y,z\}$.  Since $G$ is $4$-connected, every component of $(G(a-1, b+1)+vw)-\{x,y,z\}$ must contain a vertex of $O$.  It follows that $|\{x,y,z\} \cap V(O)| \geq 2$.  By symmetry, we may assume that $\{x,z\} \subseteq V(O)$.  If $xz$ is a chord of $O$, then $G_1$ cannot contain all of $s_{a-1}, t_{a-1}, s_{b+1}, t_{b+1}$ by the previous case. Thus, $xz$ is not a chord of $O$.  If $y$ is not adjacent to both $x$ and $z$, then $G-\{x,z\}$ is a near-triangulation, and hence $G-\{x,y,z\}$ is connected.   Thus, $xyz$ is a path of $G$.  


Since $\{s_{a-1}, t_{a-1}, s_{b+1}, t_{b+1}\} \subseteq V(G_1)$, me must have $\{x,z\} \subseteq V(C^{2k+4}), \{x,z\} \subseteq V(P^{a-1}), \{x,z\} \subseteq V(P^b)$, or $\{x,z\} \subseteq V(C^{2k+1})$.  
If $\{x,z\} \subseteq V(P^{a-1}), \{x,z\} \subseteq V(P^{b+1})$, or $\{x,z\} \subseteq V(C^{2k+4})$, then $G_1$ is non-planar since it is of the form $G_1'+vw$, where $G_1'$ is a near triangulation and at least one of $v$ or $w$ is not on the outer-face of $G_1'$.  
The remaining case is $\{x,z\} \subseteq V(C^{2k+1})$.  Since $C^{2k+1}$ is tight, there must be a vertex $p $ such that $zp \in E(G)$, and $x,p,z$ are consecutive vertices of $O$.  However, since $\{v,w\} \cap \{x,p,z\}=\emptyset$, $G_1$ is again non-planar. 

Therefore, by \cref{lem:special2linkage}, there are vertex-disjoint paths $Q_1$ and $Q_2$ in $G(a-1,b+1) +vw$ such that the end-vertices of $Q_1$ are $s_{a-1}$ and $t_{b+1}$ and the end-vertices of $Q_2$ are $s_{b+1}$ and $t_{a-1}$, as required.
\end{proof}

The next lemma is a key ingredient in the proof of \cref{InfinteCliqueMinor}(a).

\begin{lem}
\label{StrongerConstructMinor}
Let $G$ be an $f$-isoperimetric 4-connected plane triangulation, for any function $f: \NN \to \NN$. Let $Q$ be a graph obtained from $G$ by adding an infinite matching $M$ of edges not in $G$. Then $K_{\aleph_0}$ is a minor of $Q$. 
\end{lem}

\begin{proof}
Say $(a,b)\leq (c,d)$ if $a<c$, or $a=c$ and $b\leq d$. We claim that for every $i\in\NN$ there exists a 1-way infinite path $P_i$ in $G$, and for every $j\in\NN$ with $i \leq j$, there exists a tight cycle $D_j^i$ in $G$ satisfying the following:
\begin{itemize}
    \item $D_\ell^k$ surrounds $D_j^i$ if $(j,i) \leq (\ell,k)$,
    \item The first vertex of $P_i$ is on $D_i^1$ for each $i\in\NN$,
    \item $|V(P_i) \cap V(D_k^j)|=1$ for all $i,j \leq k$,
    \item For all $i < j$, there exists $\ell\geq\max\{j,k\}$ such that the two vertices of $P_i \cup P_j$ on $D_\ell^k$ are consecutive in the cyclic ordering of the $i$ vertices of $P_1 \cup \dots \cup P_i$ on $D_\ell^k$.
\end{itemize}
The last condition implies that for all $i<j$, there is $P_i$--$P_j$ path $P_{i,j}$ in $G$ such that all these paths are pairwise vertex-disjoint.  Therefore, we obtain a $K_{\aleph_0}$ minor by contracting each $P_i$ to a vertex and contracting all but one edge from each $P_{i,j}$.  So, it suffices to prove the above claim.

Let $D_1^1$ by any tight cycle in $G$. Let the first vertex of $P_1$ be any vertex of $D_1^1$.  Fix $k \geq 2$, and suppose that $D_j^i$ has been defined for all $j < k$ and $i \leq j$, and that the vertices of $P_i$ in $\Delta(D_{k-1}^{k-1})$ have been defined for all $i < k$.  Let $D_k^1$ be any tight cycle such that $D_k^1$ surrounds $D_{k-1}^{k-1}$ and $|V(D_k^1)| \geq 2k+4$.  Note that $D_k^1$ exists by repeatedly applying \cref{FindDisjointTightCycle} and using the fact that $G$ is $f$-isoperimetric. By induction, for all $i<j \leq k-1$, there exist $j \leq \ell$ and $k \leq \ell$ such that the two vertices of $P_i \cup P_j$ on $D_\ell^k$ are consecutive in the cyclic ordering of the $i$ vertices of $P_1 \cup \dots \cup P_i$ on $D_\ell^k$.

By \cref{claim:tightcycle}, we can extend $P_1, \dots, P_{k-1}$ so that each path has one end-vertex in $D_k^1$. Let $X$ be the set of end-vertices of $P_1, \dots, P_{k-1}$ on $D_k^1$. Choose an arbitrary vertex $a$ of $V(D_k^1) \setminus X$ to be the first vertex of $P_k$.  Suppose the cyclic ordering of $X \cup \{a\}$ on $D_k^1$ is $x_1, \dots, x_\ell, a, x_{\ell+1}, \dots, x_k$.  By \cref{claim:switchpaths}, there is a tight cycle $D$ such that $D$ surrounds $D_k^1$. Moreover, we can extend $P_1, \dots, P_k$ so that each path has one end-vertex in $D$, and the cyclic ordering of $X \cup \{a\}$ on $D$ is $x_1, \dots, x_{\ell-1}, a, x_{\ell}, \dots, x_k$.  Define $D_k^2:=D$.  Repeat this argument to define $D_k^3, \dots, D_k^k$.  By construction, for all $i \in [k-1]$, $x_i$ and $a$ are consecutive in the cyclic ordering of $X \cup \{a\}$ on $D_k^j$ for some $j \leq k$, as required. 
\end{proof}

%%%%%%%%%%%%%%%%%%%%%%%%%%%%%%%%%%%%%%%%%%%%%%%%%
\subsection{Proof of \cref{InfinteCliqueMinor}}
\label{ProofInfinteCliqueMinor}

We now combine the lemmas from the previous subsections to prove our first main theorem. 

\begin{thm}
\label{InfinteCliqueMinorStrongerStronger}
Let $U$ be a countable graph containing an uncountable family $\GG$ of subgraphs, each of which is a 4-connected $f$-isoperimetric planar triangulation, for an arbitrary function $f:\NN\to\NN$. Then the complete graph $K_{\aleph_0}$ is a minor of $U$.
\end{thm}

\begin{proof}
Let $L$ be any non-trivial connected component of any $\GG$-limit in $U$. By \cref{LimitLemma}, $L$ is a 4-connected $f$-isoperimetric planar triangulation, and there exists an infinite set $\PART$ of pairwise-disjoint jumps of $L$ in $U$.  Let $Q$ be the graph obtained from $U$ by contracting each path in $\PART$ to an edge, and deleting any remaining vertices not in $L$ and not in a jump in $\PART$. Observe that $Q$ is isomorphic to a graph obtained from $L$ by adding an infinite matching of edges not in $L$. By \cref{StrongerConstructMinor}, $K_{\aleph_0}$ is a minor of $Q$. The result follows since $Q$ is a minor of $U$.
\end{proof}

Let $G:=\PPP\CartProd \PPP$ be the infinite grid. Every triangulation of $G$ is $f$-isoperimetric where $f(n)\in O(n^2)$. There are uncountably many triangulations of $G$, since in each face there are two possible edges to add. Since $G$ has only countably many automorphisms,  any uncountable set of triangulations of $G$  contains an uncountable subset of pairwise non-isomorphic triangulations of $G$. Thus \cref{InfinteCliqueMinorStrongerStronger} implies and strengthens \cref{InfinteCliqueMinor}(a). Indeed, since there are uncountably many 6-regular triangulations of $G$, we obtain the following stronger result\footnote{If $c_x\in\{+,-\}$ for each $x\in\ZZ$, then adding the edge with slope $c_x$ across  every face of the infinite grid with left x-coordinate $x$, gives a 6-regular triangulation of the infinite grid. Since there are uncountably many choices for $(c_x)_{x\in\ZZ}$ there are uncountably many 6-regular triangulations of the infinite grid.}.

\begin{thm}
\label{InfinteCliqueMinorDegree6}
If a graph $U$ contains every 6-regular triangulation of the infinite grid, then $K_{\aleph_0}$ is a minor of $U$.
\end{thm}

For each $t \in \NN$, let \defn{$K_t^t$} be the multigraph on $t$ vertices with $t$ edges between every pair of vertices.  The next theorem implies and strengthens \cref{InfinteCliqueMinor}(b). 

\begin{thm} \label{thm:finitesubdivision}
If a graph $U$ contains every planar graph, then $U$ contains a subdivision of $K_t^t$ for every $t \in \NN$.
\end{thm}

\begin{proof}
Let $m \geq 2(t-1)t^2+4$. Let $H$ be a planar graph obtained from $C_m \CartProd P_\infty$ by adding $t$ new vertices $w_1, \dots, w_t$, such that $w_i$ is adjacent to $t(t-1)$ vertices of the first cycle of $C_m \CartProd P_\infty$, and $N_H(w_i) \cap N_H(w_j)=\emptyset$ for all distinct $i,j \in [t]$.  Let $f: \NN \to \NN$ be such that every triangulation of $H$ is $f$-isoperimetric.  Choose $\ell \in \NN$ so that $\ell \geq 2m$ and $\ell m \geq 2f(m-1)$.  

Let $Y$ be the set of vertices in $H$ contained in the first $\ell$ cycles of $C_m \CartProd P_\infty$ together with $w_1, \dots, w_t$. Let $H^+$ be obtained from $H$ by adding some edges with both end-vertices in $Y$ so that $H^+[Y] $ is a $4$-connected planar triangulation.  Finally, let $\FF$ be the family of all 4-connected triangulations of $H^+$. 

Since $U$ contains all planar graphs, for each $F \in \FF$, there is an injective homomorphism $\phi_F: V(F) \to V(U)$. Since $\FF$ is uncountable, $Y$ is finite, and $V(U)$ is countable, there exists an uncountable subset $\FF_0$ of $\FF$ such that $\phi_F(y)=\phi_{F'}(y)$ for all $F,F' \in \FF_0$ and all $y \in Y$.  Let $\GG$ be the collection of subgraphs $G$ of $U$ such that there exists an $F \in \FF_0$ such that $\phi_F^{-1}$ is an isomorphism from $G$ to $F$.  

Choose $F \in \FF_0$ and let $X:=\phi_F(Y)$ (note that $X$ is independent of the choice of $F$).  Let $L'$ be a $\GG$-limit in $U$, with respect to an enumeration $x_1, x_2, \dots$ of $V(U)$, where $X=\{x_1,\dots,x_{|X|}\}$.  Let $L$ be the component of $L'$ which contains $X$. Thus $L[X]$ is isomorphic to $H^+[Y]$.  Moreover, by \cref{LimitLemma}, $L$ is a $4$-connected $f$-isoperimetric triangulation, and there is an infinite set $\JJ$ of pairwise disjoint jumps of $L$ in $U$.  For $i \in [\ell]$, let $C^\ell$ be the cycle in $L$ corresponding to the $\ell$-th cycle of $C_m \CartProd P_\infty$ in $H^+$.  

\begin{claim}
$C^1$ is a tight cycle of $L$.
\end{claim}

\begin{proof}
Let $D$ be a cycle in $L$ such that $\Delta(C) \subseteq \Delta(D)$. If $V(D) \subseteq X$, then $|V(D)| \geq |V(C)|$ since $L[X]$ is isomorphic to $H^+[Y]$.  Thus, we may assume that $D$ contains a vertex $x \notin X$.  If $D$ contains a vertex of $C^{\lceil \frac{\ell}{2}\rceil}$, then $|V(D)| >  m = |V(C)|$, since $\ell \geq 2m$.  Thus, we may assume that $D$ does not contain a vertex of $C^{\lceil \frac{\ell}{2}\rceil}$.  Therefore, there are at least $m \lceil \frac{\ell}{2}\rceil$ vertices of $L$ inside $D$.  Since $\ell m \geq 2f(m-1)$ and $L$ is $f$-isoperimetric, we have that $|V(D)| \geq m$.
\end{proof}

Let $D(K_t^t)$ be a drawing of $K_t^t$ such that exactly two edges of $K_t^t$ intersect at each crossing, and the number of crossings is finite, say $c$. Let $w_1', \dots, w_t'$ be the vertices of $L$ corresponding to the vertices $w_1, \dots, w_t$ of $H$.  Let $u_1, \dots, u_{(t-1)t^2}$ be the set of neighbours of $\{w_1', \dots, w_t'\}$ on $C^1$.  Recall that $L$ is a 4-connected $f$-isoperimetric plane triangulation, $C^1$ is a tight cycle in $L$, and $|V(C^1)| \geq 2(t-1)t^2+4$. Applying~\cref{claim:switchpaths} $c$ times, there exist tight cycles $D^1:=C^1, D^2, \dots , D^{c+1}$ of $L$, and jumps $J_1, \dots, J_c$ in $\JJ$, and a subdivision $K$  of $K_t^t$ in $L \cup J_1 \cup \dots \cup J_c$ such that 
\begin{itemize}
    \item $w_1', \dots, w_t'$ are the branch vertices of $K$,
    \item $J_1 \cup \dots  \cup J_c \subseteq K$, and
    \item each jump $J_i$ can be drawn in the region of $\mathbb{R}^2$ between $D^i$ and $D^{i+1}$ to obtain a drawing of $K_t^t$ equivalent to $D(K_t^t)$. \qedhere
\end{itemize}
\end{proof}

%%%%%%%%%%%%%%%%%%%%%%%%%%%%%%%%%%%%
\subsection{Excluding Complete Graph Subdivisions}
\label{ExcludingSubdivision}

As mentioned earlier, \citet{DHV85} proved that, for each integer $t\geq 5$, there is no universal graph for the class of graphs having no $K_t$-minor, which is an immediate consequence of our \cref{InfinteCliqueMinor}(a). They also proved that for each integer $t\geq 5$, there is no universal graph for the class of graphs containing no $K_t$-subdivision. We now show that this result can be proved using the Limit Lemma (\cref{LimitLemma}). A graph is \hdefn{$k$}{apex} if it contains a set (called the \defn{apex set}) of at most $k$ vertices whose deletion results in a planar graph. 

\begin{lem}
\label{Apex}
If a graph $G$ contains all $k$-apex graphs, then $G$ contains a subdivision of $K_{k+5}$. 
\end{lem}

\begin{proof} 
Let $\FF$ be an uncountable family of 4-connected $f$-isoperimetric plane triangulations, for any function $f:\NN\to\NN$. For each set $A$ of $k$ vertices in $G$, let $\GG_A$ be the set of subgraphs $H$ of $G$ such that $A$ is a clique in $H$, $A$ is complete to $H-A$, and $H-A$ is isomorphic to a graph in $\FF$. Since $G$ contains all $k$-apex graphs and there are only countably many such sets $A$, $\GG_A$ is uncountable for some set $A$.  Let $\GG_A':=\{H-A : H \in \GG_A\}$.  Let $L$ be any non-trivial component of any  $\GG_A'$-limit in $G$. By \cref{LimitLemma}, $L$ is a 4-connected $f$-isoperimetric planar triangulation, and there are infinitely many pairwise disjoint jumps of $L$ in $G$. Since $A$ is finite, one of these jumps, say $J$, is disjoint from $A$.  Let $x$ and $y$ be the end-vertices of $J$.  Since $L$ is 4-connected, there is a cycle $C$ in $L$ such that $|V(C)| \geq 4$ and $\{x,y\} \subseteq V(C)$.  Let $L_C$ be the subgraph of $L$ contained in the closed disk bounded by $C$.   Note that $L_C \cup J$ contains a subgraph $K$ which is a subdivision of $K_4$. Since $L$ is 4-connected, $K$ can be extended to a subdivision of $K_5$ in $L \cup J$. Together with $A$, we obtain a subdivision of $K_{k+5}$ in $G$.  
\end{proof}

Since no $k$-apex graph contains a subdivision of $K_{k+5}$, \cref{Apex} implies:

\begin{cor}
\label{Apex-universal}
For each $k\in\NN$, there is no universal graph for the class of $k$-apex graphs, and, for each integer $k \geq 5$, there is no universal graph for the class of graphs containing no subdivision of $K_k$.
\end{cor}

The class of graphs containing no subdivision of $K_{\aleph_0}$ is not covered by \cref{Apex-universal}. We now show there is no universal graph for this class, amongst many other examples. 

\begin{lem}
\label{Non-universal}
Let $\GG$ be a non-empty class of graphs such that no graph in $\GG$ contains $K_{\aleph_0}$, and for every graph $G \in \GG$ and for every vertex $v \in V(G)$, the graph $G-v$ contains a subgraph in $\GG$. Let $\FF$ be the family of graphs that contain no subgraph in $\GG$. Then there is no universal graph for $\FF$.
\end{lem}

\begin{proof}
Suppose for the sake of contradiction that $F_0$ is a universal graph for $\FF$. Let $F_0'$ be obtained from $F_0$ by adding a new vertex $v$ adjacent to all vertices in $F_0$. We claim that $F_0'$ is in $\FF$. Suppose 
for the sake of contradiction that $F_0'$ contains a subgraph $G$ isomorphic to a graph in $\GG$. Since $F_0$ does not contain $G$, it must be that $v\in V(G)$. By assumption, $G-v$ contains a graph in $\GG$, and that graph is also a subgraph of $F_0$. This contradiction shows that $F_0'\in \FF$. Since $F_0$ is universal for $\FF$, $F_0$ contains a subgraph $F_1$ isomorphic to $F_0'$. Let $v_1$ be the vertex in $F_1$ corresponding to $v$. Then $F_1-v_1$ contains a subgraph $F_2$ isomorphic to $F_0'$. Let $v_2$ be the vertex in $F_2$ corresponding to $v$. We repeat this argument and obtain a sequence $v,v_1,v_2, \ldots$ of vertices in $F_0$ inducing $K_{\aleph_0}$. Thus $F_0$ contains every graph, which contradicts the assumption that $\GG\neq\emptyset$.
\end{proof}

Numerous graph classes satisfy the assumptions in \cref{Non-universal} for an appropriately defined class $\GG$. In particular, the following graph classes have no universal element:
\begin{itemize}
\item chordal graphs containing no $K_{\aleph_0}$;
\item graphs with no $K_{\aleph_0}$ minor (which was proved in \cite{DHV85});
\item graphs containing no $K_{\aleph_0}$ subdivision (which was left open in \cite{DHV85});
\item graphs containing no subdivision of the $d$-regular tree;
\item graphs containing no tree where all the vertices, except possibly one, have degree $d$; 
\item graphs in which each subgraph has a vertex of finite degree;
\item graphs that do not contain infinitely many pairwise disjoint one-way infinite paths; 
\item graphs that do not contain $d$ pairwise disjoint one-way infinite paths (for any fixed $d\in\NN$).
\end{itemize}

%%%%%%%%%%%%%%%%%%%%%%%%%%%
\section{Universality for Trees and Treewidth}
\label{TreesTreewidth}

This section proves universality results for trees and graphs of given treewidth. While such results are well known, our construction is novel and leads to strengthenings in terms of orientation- and labelling-preserving isomorphisms, which are a key tool used in our constructions for graphs defined by a tree-decomposition (\cref{DefinedByTreeDecomposition}), for $K_t$-minor-free graphs (\cref{ExcludedMinorNoSubdiv}), and for locally finite graphs (\cref{LocallyFiniteGraphs}).

\subsection{Universal Trees}
\label{UniversalTrees}

The following universal graph for (countable) trees is folklore. Let $\T$ be the graph with 
\begin{align*}
V(\T) &:=\{ (x_1,\dots,x_n) : n\in\mathbb{N}_0, x_1,\dots,x_n \in\mathbb{N} \} \text{ and}\\
E(\T) &:= \{ (x_1,\dots,x_n)(x_1,\dots,x_n,x_{n+1}) : n\in\mathbb{N}_0, x_1,\dots,x_n,x_{n+1}\in\mathbb{N} \}.
\end{align*}

\begin{thm}
\label{UniversalTree}
$\T$ is universal for the class of trees.
\end{thm}

\begin{proof}
Consider $\T$ to be rooted at vertex $()$. 
From each vertex $(x_1,\dots,x_n)$ in $\T$ there is a unique path 
$(x_1,\dots,x_n),(x_1,\dots,x_{n-1}),(x_1,\dots,x_{n-1}),\dots,(x_1),()$ to the root. 
Thus $\T$ is a tree. 
Since $V(\T)$ is a countable union of countable sets, $\T$ is countable. 
We now show that every countable tree $X$ is isomorphic to a subtree of $\TT$. Root $X$ at an arbitrary vertex $r$. Let $\ell:V(X)\to\mathbb{N}$ be an injective function, which exists since $X$ is countable. For each non-root vertex $w$ of $X$, if $(r,w_1,w_2,\dots,w_n)$ is the path from $r$ to $w$ in $X$, then  let $\phi(w):=(\ell(w_1),\ell(w_2),\dots,\ell(w_n))$. Finally, let $\phi(r):=()$. Then $\phi$ is an isomorphism from $X$ to a subtree of $\TT$.  
\end{proof}

A \defn{labelling} of a graph $G$ is a function $f:V(G)\cup E(G)\to\NN$.
We now prove a strengthening of \cref{UniversalTree}, which will be used in \cref{DefinedByTreeDecomposition}. 

\begin{lem}
\label{UniversalTreePreserving}
There is a labelling of $\TT$ such that for every 1-orientation of $\TT$, for every rooted tree $T$, and for every labelling of $T$ there is a label-preserving orientation-preserving isomorphism from $T$ to a subtree of $\TT$. 
\end{lem}

\begin{proof}
Since every vertex of $\TT$ has infinite degree, there is a labelling of $\TT$ such that for every vertex $v$ of $\TT$ and for all $\alpha,\beta\in\NN$ there are infinitely many neighbours $w$ of $v$ for which $w$ is labelled $\alpha$ and $vw$ is labelled $\beta$. Fix any 1-orientation of $\TT$. Now, for every vertex $v$ of $\TT$ and for all $\alpha,\beta\in\NN$ there are infinitely many children $w$ of $v$ for which $w$ is labelled $\alpha$ and $vw$ is labelled $\beta$. 

Let $T$ be a tree rooted at $r$, and let $f$ be any labelling of $T$. We now construct a label-preserving orientation-preserving isomorphism $\phi$ from $T$ to a subtree of $\TT$. Let $\phi(r)$ be any vertex in $\TT$ labelled $f(r)$. Consider the vertices of $T$ in order of non-decreasing distance from $r$. Let $v$ be a vertex of $T$ such that $\phi(v)$ is already defined, but $\phi(w)$ is undefined for every child $w$ of $v$. Let $W$ be the set of children of $v$ in $T$. For all $\alpha,\beta\in\NN$ there are infinitely many children $z$ of $\phi(v)$ for which $z$ is labelled $\alpha$ and $vz$ is labelled $\beta$. So $\phi$ can be extended to $W$ so that for each $w\in W$, we have $f(w)$ equals the label assigned to $\phi(w)$ in $\TT$, and $f(vw)$ equals the label assigned to $\phi(vw)$ in $\TT$. Hence $f$ is a label-preserving orientation-preserving isomorphism from $T$ to a subtree of $\TT$.
\end{proof}

As an aside, we show that \cref{UniversalTreePreserving} is not true for (unrooted) 1-oriented trees $T$. Suppose that there is a labelling and 1-orientation of the universal tree $\TT$ such that for every labelled and 1-oriented tree $T$, there is a label-preserving orientation-preserving isomorphism from $T$ to a subtree of $\TT$. In particular, for every labelled anti-directed 1-way infinite path $P$, there is a label-preserving orientation-preserving isomorphism from $P$ to $\TT$. There are uncountably many labelled anti-directed 1-way infinite paths (since at each vertex there are at least two choices of label). However, for each vertex $v$ of $\TT$ there is exactly one anti-directed 1-way infinite path starting at $v$. Hence some labelling of the anti-directed 1-way infinite path does not appear in $\TT$. 

To complete this subsection, we define a universal tree of given maximum degree. Let $\TT^{(d)}$ be the graph with 
\begin{align*}
V(\TT^{(d)}) & := \{ (x_1,\dots,x_n): n\in\NN_0,\, x_1,\dots,x_n\in[d], \,
\forall i\in[n-1] \, x_i\neq x_{i+1} \}\\
E(\TT^{(d)}) & := \{ (x_1,\dots,x_n)(x_1,\dots,x_n,x_{n+1}):\\
& \qquad n\in\NN_0,\, x_1,\dots,x_n,x_{n+1}\in[d], \, \forall i\in[n]\, x_i\neq x_{i+1}\}.
\end{align*}
A proof analogous to that of \cref{UniversalTree} shows that $\TT^{(d)}$ is a tree, and by definition, $\TT^{(d)}$ is $d$-regular. Indeed, up to isomorphism, $\TT^{(d)}$ is the unique $d$-regular tree, sometimes called the Bethe lattice or infinite Cayley tree \citep{Bethe35,Ostilli12}. This tree is used in \cref{SimpleTreewidth} as the basis for the definition of a universal graph of given simple treewidth. 


%%%%%%%%%%%%%%%%%%%%%%%%%%%%%%%%%%%%%%%%%%%%%%%%%%%%%%%%%%%%%%%%
\subsection{Universality for Treewidth}
\label{Treewidth}

Chordal graphs and the theory of simplicial decompositions~\citep{Thom83,Diestel90,Halin84,Halin82,Halin78} can be used to define universal treewidth-$k$ graphs. We take a somewhat different approach that gives a new and explicit definition of a universal treewidth-$k$ graph $\TT_k$. This approach allows us to derive further properties of $\TT_k$, regarding label- and orientation-preserving isomorphisms (\cref{TreewidthUniversalPreserving}), that are essential for results in \cref{ExcludedMinorNoSubdiv}. 

The following definitions lead to a new characterisation of graphs with given treewidth. Let $c$ be a colouring of a 1-oriented tree $T$. Let \defn{$\GGG{T}{c}$} be the graph with vertex set $V(\GGG{T}{c})=V(T)$, where $vw\in E(\GGG{T}{c})$ if and only if $v$ is an ancestor of $w$ in $T$ and $v$ is the only vertex on the $vw$-path in $T$ coloured $c(v)$. 
Where it is clear from the context, we implicitly consider the edges $vw$ of $\GGG{T}{c}$ to be oriented from the ancestor $v$ to the descendent $w$. 

\begin{lem}
\label{FindSpanningTree}
For every 1-oriented tree $T$ and $(k+1)$-colouring $c$ of $T$, the graph $\GGG{T}{c}$ is chordal with no $K_{k+2}$ subgraph, and is simplicially $k$-oriented. Conversely, let $G$ be a connected chordal graph with no $K_{k+2}$ subgraph. Fix a $k$-orientation of $G$. Then $G\subseteq \GGG{T}{c}$ for some 1-oriented spanning tree $T$ of $G$ and for every $(k+1)$-colouring $c$ of $G$.
\end{lem}

\begin{proof}
We first prove that $\GGG{T}{c}$ is chordal with no $K_{k+2}$ subgraph. By construction, each vertex $w$ has in-degree at most $k$ in $\GGG{T}{c}$, and the in-neighbourhood of $v$ in $\GGG{T}{c}$ lies on a single directed path in $T$ that ends at $v$. Consider two directed edges $uw$ and $vw$ in $\GGG{T}{c}$. So both $u$ and $v$ are ancestors of $w$ in $T$, and $c(u)\neq c(v)$. Without loss of generality, $u$ is an ancestor of $v$. Since $uw$ is an edge, $u$ is the only vertex on the $uw$-path in $T$ coloured $c(u)$. In particular, $u$ is the only vertex on the $uv$-path in $T$ coloured $c(u)$. So $uv$ is an edge of $\GGG{T}{c}$. Hence the in-neighbourhood of $w$ in $\GGG{T}{c}$ is a clique, and $\GGG{T}{c}$ is simplicially $k$-oriented. By \cref{ChordalCharacterisation}, $\GGG{T}{c}$ is chordal. Each vertex in $\GGG{T}{c}$ has in-degree at most $k$. Thus $\GGG{T}{c}$ contains no $K_{k+2}$. 

Let $Z$ be the set of oriented edges $vw$ of $G$ for which there is no vertex $x$ such that $vxw$ is an oriented path in $G$. Let $T$ be the spanning subgraph of $G$ with edge-set $Z$ (ignoring the edge orientation). We claim that $T$ is a spanning tree of $G$ with in-degree at most 1 at every vertex. Suppose that some vertex $v$ has in-degree at least 2 in $T$. Let $xv$ and $yv$ be two incoming edges incident to $v$ in $T$. Since the in-neighbourhood of $v$ is a clique in $G$, without loss of generality, $xy$ is an oriented edge in $G$. Thus $xyv$ is an oriented path in $G$, implying that $xv$ is not in $Z$. This contradiction shows that $T$ has in-degree at most 1. Since an acyclically oriented  cycle contains a vertex with in-degree 2, $T$ contains no cycle. We now show that $T$ is connected. For any two vertices $a$ and $b$, since $G$ is connected, there is an $ab$-path in $G$ (ignoring the edge orientation). For each oriented edge $vw$ in $P$, if $vw$ is not in $Z$, then there is a vertex $x$ such that $vxw$ is an oriented path in $G$. Replace $vw$ by $vx$ and $xw$ in $P$. Repeat this operation until every edge of $P$ is in $Z$. We obtain an $ab$-path in $T$ (ignoring the edge orientation). Thus $T$ is connected. Hence $T$ is a spanning tree of $G$ with in-degree at most 1 at each vertex. 

Suppose for the sake of contradiction that for some oriented edge $vw$ of $G$, there is no directed $vw$-path in $T$. Let $vw$ be such an edge that minimises the distance between $v$ and $w$ in $T$ (ignoring edge orientations). Let $P$ be the $vw$-path in $T$ (ignoring orientations). Let $x$ be the least common ancestor of $v$ and $w$ in $T$. So $P$ is oriented from $x$ to $v$ and from $x$ to $w$ (and $x\neq v$ and $x\neq w$). Let $y$ be the neighbour of $w$ on the $vw$-path in $T$. Since the in-neighbourhood of $w$ is a clique, $vy$ or $yv$ is an edge of $G$, which contradicts the choice of $vw$. Thus for every oriented edge $vw$ of $G$, there is a directed $vw$-path in $T$. 

Consider any oriented path $v_1,v_2,\dots,v_p$ in $G$, where $v_1v_p$ is an edge of $G$. Since the in-neighbourhood of $v_p$ is a clique, $v_1v_{p-1}$ is an edge of $G$. Since the in-neighbourhood of $v_{p-1}$ is a clique, $v_1v_{p-2}$ is an edge of $G$. Repeating this argument, $v_1v_i$ is an edge of $G$, for each $i\in[2,p]$. 

Let $c$ be any $(k+1)$-colouring of $G$, which exists since chordal graphs are perfect and by the de~Bruijn--Erd\H{o}s Theorem~\citep{dBE51}. Consider an oriented edge $vw$ of $G$. As shown above there is a directed path $P$ from $v$ to $w$ in $T$, and $v$ is adjacent to every vertex in $P$. Thus $c(v)\neq c(x)$ for every vertex $x$ in $P$. By construction, $vw\in E(\GGG{T}{c})$. Hence $G\subseteq \GGG{T}{c}$.
\end{proof}


\begin{lem}
\label{TreewidthCharacterisation}
The following are equivalent for a graph $G$ and $k\in\NN$:
\begin{enumerate}[(a)]
\item $G$ has treewidth at most $k$, 
\item there is a 1-oriented tree $T$ and a $(k+1)$-colouring $c$ of $T$ such that $V(T)=V(G)$ and $G\subseteq \GGG{T}{c}$, 
\item $G$ is a spanning subgraph of a chordal graph with no $K_{k+2}$ subgraph.
\end{enumerate}
\end{lem}

\begin{proof}
(a) $\Longrightarrow$ (b): 
Let $G$ be a graph with treewidth at most $k$. Apply \cref{StandardTreewidth} to obtain a rooted tree $T$, a tree-decomposition $(B_w:w\in V(T))$ of $G$, and a $(k+1)$-colouring $c$ of $G$. We now show that $G\subseteq \GGG{T}{c}$. By construction, $V(G) = V(\GGG{T}{c})$. Consider an edge $vw\in E(G)$. Then $c(v)\neq c(w)$. Without loss of generality, $w$ is a descendant of $v$. Consider some vertex $x$ on the $vw$-path in $T$. Since $v\in B_v\cap B_w$, it follows that $v$ is in $B_x$, implying $c(v)\neq c(x)$ by \cref{StandardTreewidth}. Hence $vw\in E(\GGG{T}{c})$ by the definition of $E(\GGG{T}{c})$. Therefore $G\subseteq \GGG{T}{c}$. 

(b) $\Longrightarrow$ (c): Assume $T$ is a 1-oriented tree and $c$ is a $(k+1)$-colouring of $T$ such that $V(T)=V(G)$ and $G\subseteq \GGG{T}{c}$. 
By \cref{FindSpanningTree}, $\GGG{T}{c}$ is a chordal graph with no $K_{k+2}$ subgraph containing $G$ as a spanning subgraph, as desired. 

(c) $\Longrightarrow$ (a): Assume $G$ is a spanning subgraph of a chordal graph $G'$ with no $K_{k+2}$ subgraph. By \cref{ChordalCharacterisation}, $G'$ has a tree-decomposition in which each bag is a clique, and therefore of size at most $k+1$. The same tree-decomposition is a tree-decomposition of $G$. So $\tw(G)\leq k$.
\end{proof}

We now define a universal treewidth-$k$ graph. Let $\T$ be the universal tree defined in \cref{UniversalTree} rooted at vertex $()$. For $k\in\mathbb{N}$, let $\TT_k := \GGG{\T}{c}$, where $c$ is a colouring of $\T$ with colour-set $[0,k]$, where every vertex coloured $i\in[0,k]$ has infinitely many children of each colour in $[0,k]\setminus\{i\}$. 

\begin{thm}
\label{TreewidthUniversal}
$\TT_k$ is universal for the class of graphs with treewidth at most $k$.
\end{thm}

\begin{proof}
$\GGG{\T}{c}$ has treewidth at most $k$ by \cref{TreewidthCharacterisation} and since $c$ uses $k+1$ colours. Conversely, let $G$ be a graph with treewidth at most $k$. By \cref{TreewidthCharacterisation}, there is a tree $T$ rooted at some vertex $r$, and there is a $(k+1)$-colouring $\alpha:V(T)\to[0,k]$ such that $V(T)=V(G)$ and $G\subseteq \GGG{T}{\alpha}$. By \cref{UniversalTreePreserving} there is a colour-preserving orientation-preserving isomorphism from $T$ (coloured by $\alpha$) to a subtree of $\TT$ (coloured by $c$). Thus $\GGG{T}{\alpha}$ is isomorphic to a subgraph of $\GGG{\T}{c}$. Hence $G$ is isomorphic to a subgraph of $\GGG{\T}{c}$.
\end{proof}

We now prove two strengthenings of \cref{TreewidthUniversal} that are used in \cref{ExcludedMinorNoSubdiv}. The first is an analogue of \cref{UniversalTreePreserving}. 

\begin{lem}
\label{TreewidthUniversalPreserving}
For each $k\in\NN$, there is a labelling and a rooted $k$-orientation of $\TT_k$ such that for every chordal graph $G$ with no $K_{k+2}$ subgraph, for every labelling of $G$ and for every rooted $k$-orientation of $G$, there is a label-preserving orientation-preserving isomorphism from $G$ to a subgraph of $\TT_k$. 
\end{lem}

\begin{proof}
Recall that $\TT$ is the universal tree rooted at vertex $()$, and that $\TT_k$ is the universal treewidth-$k$ graph with $V(\TT_k)=V(\TT)$. We may orient the edges of $E(\TT_k)$ such that for each oriented edge $vw$, $v$ is an ancestor of $w$ in $\TT$, and each vertex of $\TT_k$ has in-degree at most $k$. By definition, $\TT_k$ has a $(k+1)$-colouring such that each vertex $v$ of $\TT$ has infinitely many children of each colour (distinct from the colour assigned to $v$). Label each vertex of $\TT_k$ such that for each vertex $v$ of $\TT$, for each colour $\alpha$ distinct from the colour assigned to $v$, and for each $\beta\in \NN$, there are infinitely many children of $v$ coloured $\alpha$ and labelled $\beta$. This is possible since $v$ has infinitely many children coloured $\alpha$. Consider each vertex $w$ of $\TT_k$. Let $v_1w,\dots,v_pw$ be the edges of $\TT_k$ incoming to $w$. So $p\leq k$. Let $W$ be the siblings of $w$ assigned the same colour and the same label as $w$. So $W$ is infinite. Since every vertex $x\in W$ has the same colour as $w$ and the same parent as $w$ in $\TT$, by the construction of $\TT_k$, we have that $v_1x,\dots,v_px$ are the edges of $\TT_k$ incoming to $x$. Since $\NN^p$ is countable, there is a labelling of the edges $v_ix$ (where $i\in[p]$ and $x\in W$), such that for each $(\gamma_1,\dots,\gamma_p)\in \NN^p$ there are infinitely many vertices $x\in W$, such that the edge $v_ix$ of $\TT_k$ is labelled $\gamma_i$ for each $i\in[p]$. Call this property $(\star)$. 

Now, let $G$ be a chordal graph with no $K_{k+2}$ subgraph, with a fixed $k$-orientation and labelling $f$. Our goal is to show that there is a label-preserving orientation-preserving isomorphism from $G$ to a subgraph of $\TT_k$. Since $\TT_k$ contains infinitely many pairwise disjoint subgraphs isomorphic to itself, we may assume that $G$ is connected. Let $r$ be the root of $G$. By \cref{FindSpanningTree}, $G$ is an orientation-preserved subgraph of $\GGG{T}{c}$ for some 1-oriented spanning tree $T$ of $G$ and for every $(k+1)$-colouring $c$ of $G$. 
We now define an isomorphism $\phi$ from $G$ to a subgraph of $\TT_k$. We define $\phi(v)$ in order of non-decreasing $\dist_T(r,v)$. Let $\phi(r)$ be any vertex of $\TT_k$ labelled $f(r)$. Let $v$ be a vertex of $T$ for which $\phi(v)$ is already defined, but $\phi$ is defined for no child of $v$. For $\alpha\in[0,k]$ and $\beta\in\NN$, let $W_{\alpha,\beta}$ be the set of children $w$ of $v$ in $T$, such that $c(w)=\alpha$ and $f(w)=\beta$. Consider such a $w\in W_{\alpha,\beta}$. Let $v_1w,\dots,v_pw$ be the edges of $\GGG{T}{c}$ incoming to $w$. So $p\leq k$. By the choice of $v$, we have that $\phi(v_1),\dots,\phi(v_p)$ are already defined. Since every vertex $x\in W_{\alpha,\beta}$ has the same colour as $w$ and the same parent as $w$ in $T$, by the construction of $\GGG{T}{c}$, we have that $v_1x,\dots,v_px$ are the edges of $\GGG{T}{c}$ incoming to $x$. For $(\gamma_1,\dots,\gamma_p)\in \NN^p$, let $W_{\alpha,\beta,\gamma_1,\dots,\gamma_p}$ be the set of vertices $x\in W_{\alpha,\beta}$ such that $f(v_ix)=\gamma_i$ for each $i\in[p]$. 
By property $(\star)$, there are infinitely many children $z$ of $\phi(v)$ in $\TT$, such that $z$ is coloured $\alpha$ and labelled $\beta$, and $\phi(v_i)z$ is labelled $\gamma_i$ for each $i\in[p]$. Injectively map $W_{\alpha,\beta,\gamma_1,\dots,\gamma_p}$ to these vertices $z$ under $\phi$. Hence $\phi$ is a label-preserving orientation-preserving isomorphism from $\GGG{T}{c}$ to a subgraph of $\TT_k$. Since $G$ is an orientation-preserved subgraph of $\GGG{T}{c}$, the result follows. 
\end{proof}

\begin{lem}
\label{Maximal}
Let $G$ be a chordal graph with no $K_{k+2}$ subgraph. Fix a rooted $k$-orientation of $G$. Let $Q$ be a chordal graph containing $G$ as a spanning subgraph, with no $K_{k+2}$ subgraph, with a rooted $k$-orientation that preserves the given $k$-orientation of $G$, and subject to these conditions is edge-maximal. Then for every labelling of $Q$, there is an orientation-preserving label-preserving isomorphism from $Q$ to an induced subgraph of $\TT_k$. \end{lem}

\begin{proof}
By \cref{TreewidthUniversalPreserving}, there is a labelling and rooted $k$-orientation of $\TT_k$ such that for every rooted $k$-orientation of $Q$, there is a label-preserving orientation-preserving isomorphism $\phi$ from $Q$ to a subgraph of $\TT_k$. Let $Q'$ be the graph with vertex-set $V(Q)$, where $vw\in E(Q')$ whenever $\phi(v)\phi(w)\in E(\TT_k)$. Orient $Q'$ by the corresponding orientation of $\TT_k$, which preserves the orientation of $G$, since $Q$ preserves the orientation of $G$ and $\phi$ preserves the orientation of $Q$. Since $Q'$ is isomorphic to an induced subgraph of $\TT_k$ (which is chordal with no $K_{k+2}$ subgraph), $Q'$ is chordal with no $K_{k+2}$ subgraph. By construction, $Q\subseteq Q'$. By the maximality of $Q$, we have $Q=Q'$. So 
$\phi(Q)$ is an induced subgraph of $\TT_k$. 
\end{proof}

%%%%%%%%%%%%%%%%%%%%
\subsection{Graphs Defined by a Tree-Decomposition}
\label{DefinedByTreeDecomposition}

Recall that $\DD(U)$ is the class of graphs that have a tree-decomposition over the graph $U$. We now construct a universal graph for $\DD(U)$.  This result is used to construct a universal graph for graphs of given simple treewidth (\cref{SimpleTreewidth}), as well as to construct a graph that contains every $X$-minor-free graph (\cref{ExcludedMinors}). This is the first place where we see the utility of our results about orientation and label-preserving isomorphisms developed above. 

\begin{lem}
\label{TreeDecompUniversal}
For every graph $U$ containing no $K_{\aleph_0}$ subgraph, there is a universal graph $\widehat{U}$ for $\DD(U)$.
\end{lem}

\begin{proof}
Let $\CC$ be the set of all pairs $((v_1,\dots,v_k),(w_1,\dots,w_k))$, where $\{v_1,\dots,v_k\}$  and $\{w_1,\dots,w_k\}$ are cliques in $U$ for any $k\in\NN$. Since $U$ is countable with no $K_{\aleph_0}$ subgraph, $C$ is countable. Enumerate $\CC=\{C_1,C_2,\dots\}$.  Let $\TT$ be the universal tree 1-oriented and labelled, such that every vertex is labelled 1, and for every vertex $v$ of $\TT$ and every $i\in\NN$ there are infinitely many children $w$ of $v$ with $vw$ labelled $i$. Let $\widehat{U}$ be the graph obtained from $\TT$ as follows. First, for each vertex $v$ of $\TT$, introduce a copy $U_v$ of $U$ in $\widehat{U}$, where $U_v$ and $U_w$ are disjoint for all distinct $v,w\in V(\TT)$. Second, for each edge $vw\in E(\TT)$, if $vw$ is labelled $i\in\NN$ and $C_i=((v_1,\dots,v_k),(w_1,\dots,w_k))$, then for each $j\in[k]$, identify vertex $v_j$ in $U_v$ with vertex $w_j$ in $U_w$. This defines $\widehat{U}$. 

For each vertex $v\in V(\TT)$, let $\widehat{U}_v$ be the subgraph of $\widehat{U}$ corresponding to the copy $U_v$ (after the above vertex identifications). Then $(V(\widehat{U}_v):v\in V(\TT))$ is a tree-decomposition of $\widehat{U}$, where each torso is isomorphic to $U$. Thus $\widehat{U}$ is in $\DD(U)$. 

We now show that $\DD(U)$ contains every graph $G$ in $\DD(U)$. So $G$ has a tree-decomposition $(B_x:x\in V(T))$, such that for each node $x\in V(T)$, there is an isomorphism $\phi_x$ from the torso of $x$ to a subgraph of $U$. Root $T$ at an arbitrary node $r$. Label each vertex of $T$ by 1. Label each oriented edge $xy$ of $T$ as follows. Say $B_x\cap B_y=\{z_1,\dots,z_k\}$, which is a clique in the torsos of $x$ and $y$. For $j\in[k]$, let $v_j$ and $w_j$ be the vertices of $U$ such that $f_x(z_j)=v_j$ and $f_y(z_j)=w_j$. So $\{v_1,\dots,v_k\}$ and $\{w_1,\dots,w_k\}$ are cliques of $U$.  Label $vw$ by $i\in\NN$ so that $C_i= ((v_1,\dots,v_k),(w_1,\dots,w_k))$. By \cref{UniversalTreePreserving}, there is a label-preserving orientation-preserving isomorphism $\phi$ from $T$ to $\TT$. 
By construction, $\bigcup_{v\in V(T)}\widehat{U}_{\phi(v)}$ contains a subgraph isomorphic to $G$.
\end{proof}

\cref{TreeDecompUniversal} can be extended as follows to allow for tree-decompositions with adhesion $k$. The proof is identical to that  \cref{TreeDecompUniversal}, except that in the definition of $\CC$ we only consider cliques with size at most $k$.

\begin{lem}
\label{TreeDecompUniversalAdhesion}
For every graph $U$ and $k\in\NN$, there is a universal graph $\widehat{U}^k$ for the class $\DD_k(U)$.
\end{lem}


%%%%%%%%%%%%%%%%%%%%
\subsection{Simple Treewidth}
\label{SimpleTreewidth}

A tree-decomposition $(B_x:x\in V(T))$ of a graph $G$ is \hdefn{$k$}{simple}, for some $k\in\NN$,  if it has  width  at most $k$, and for every set $S$ of $k$ vertices in $G$, we have $|\{x\in V(T): S\subseteq B_x\}|\leq 2$. The \defn{simple treewidth} of a graph $G$, denoted by $\stw(G)$, is the minimum $k\in\NN$ such that $G$ has a $k$-simple tree-decomposition. Simple treewidth appears in several places in the literature under various guises \citep{KU12,KV12,MJP06,Wulf16}. The following facts are well known: A connected finite graph has simple treewidth 1 if and only if it is a path. A finite graph has simple treewidth at most 2 if and only if it is a outerplanar. A finite graph has simple treewidth at most 3 if and only if it has treewidth 3 and is planar~\citep{KV12}. The edge-maximal finite  graphs with simple treewidth 3 are ubiquitous objects, called  \defn{planar 3-trees} in structural graph theory and graph drawing~\citep{AP-SJADM96,KV12}, called \defn{stacked polytopes} in polytope theory~\citep{Chen16}, and called \defn{Apollonian networks} in enumerative and random graph theory~\citep{FT14}. It is also known and easily proved that $\tw(G) \leq \stw(G)\leq \tw(G)+1$ for every finite graph $G$ (see \citep{KU12,Wulf16}). A similar proof shows this result for infinite graphs. 

Simple treewidth can be characterised in terms of chordal supergraphs. The following lemma for finite graphs is due to \citet{Wulf16}. Let $W_k$ be the graph with vertex set $\{u,v,w,x_1,\dots,x_k\}$, where each of $\{u,x_1,\dots,x_k\}$, $\{v,x_1,\dots,x_k\}$ and $\{w,x_1,\dots,x_k\}$ are cliques. 

\begin{lem}
\label{SimpleTreewidthChordal}
A graph $G$ has simple treewidth at most $k\in\NN$ if and only if $G$ is a subgraph of a chordal graph containing no $K_{k+2}$ or $W_k$ subgraph,
\end{lem}

\begin{proof}
By \cref{TreeDecompositionClique}, in every tree-decomposition of $W_k$ with width $k$, the cliques $\{u,x_1,\dots,x_k\}$, $\{v,x_1,\dots,x_k\}$ and $\{w,x_1,\dots,x_k\}$ are distinct bags. Therefore, $\{x_1, \dots, x_k\}$ is contained in at least three bags, and such a tree-decomposition is not $k$-simple. Thus $\stw(W_k)>k$.

Now consider a graph $G$ with  simple treewidth at most $k$. Starting from a $k$-simple tree-decomposition of $G$, let $G'$ be the graph obtained from $G$ by adding an edge between any two non-adjacent vertices in a common bag. We obtain a $k$-simple tree-decomposition of $G'$, implying $\stw(G')\leq k$. By \cref{ChordalCharacterisation}, $G'$ is chordal, and $W_k\not\subseteq G'$ since $\stw(W_k)>k$. By \cref{TreeDecompositionClique}, $K_{k+2}\not\subseteq G'$. 

We now prove that if $G$ is a subgraph of a chordal graph with no $K_{k+2}$ or $W_k$ subgraph, then $\stw(G)\leq k$. It suffices to prove the result when $G$ is chordal with no $K_{k+2}$ or $W_k$ subgraph. Let $\{X_i:i\in I\}$ be the maximal cliques of $G$. Let $T$ be the graph with vertex-set $I$, where $ij\in E(T)$ if and only if $X_i\cap X_j\neq\emptyset$ and $i\neq j$. Since $G$ is chordal, $T$ is a tree and $(X_i:i\in V(T))$ is a tree-decomposition of $G$. Since $K_{k+2}\not\subseteq G$, the width is at most $k$. If some set $X$ of $k$ vertices appears in distinct bags $X_a,X_b,X_c$, then $X_a\cup X_b\cup X_c$ induce $W_k$. Thus 
$(X_i:i\in V(T))$ is a $k$-simple tree-decomposition of $G$, and $\stw(G)\leq k$. 
\end{proof}

For $k\in\NN$, define the directed graph $\RR_k$ as follows. 
Recall that $\TT^{(k+1)}$ is the universal $(k+1)$-regular tree defined in \cref{UniversalTrees}. Fix an orientation of $\TT^{(k+1)}$ in which each vertex has in-degree 1 and out-degree $k$, which exists by \cref{TreeOrientation}. Let $c:V(\TT^{(k+1)})\to[0,k]$ be a colouring of $V(\TT^{(k+1)})$ such that for each vertex $x$ with out-neighbours $y_1,\dots,y_k$, all of $x,y_1,\dots,y_k$ are assigned different colours. This is possible since, in the graph obtained from $\TT^{(k+1)}$ by adding a $k$-clique on the out-neighbours of each vertex, each biconnected subgraph is $K_{k+1}$ (since by the de~Bruijn--Erd\H{o}s~Theorem~\citep{dBE51} and considering the cut-block-tree, if every biconnected subgraph of a graph $G$ is $c$-colourable, then so is $G$)). Now define $\RR_k:=\GGG{T}{c}$. For example, \cref{UniversalOuterplanar,UniversalPlanar} illustrate $\RR_2$ and $\RR_3$. 

\begin{figure}[!b]
\centering
\includegraphics[width=\textwidth]{UniversalOuterplanar}
\caption{Illustration of the outerplanar graph $\RR_2$ with the 1-oriented 3-regular universal tree highlighted. }
\label{UniversalOuterplanar}
\end{figure}


\begin{figure}[!t]
    \centering
    \includegraphics[width=\textwidth]{UniversalPlanar}
    \caption{Illustration of the planar graph $\RR_3$ with the 1-oriented 4-regular universal tree highlighted. }
    \label{UniversalPlanar}
\end{figure}


\begin{lem}
\label{SimpleTreewidthRk}
$\RR_k$ has simple treewidth $k$.
\end{lem}

\begin{proof}
Let $\TT^{(k+1)}$ be the naturally oriented tree and $c$ be the colouring used in the definition of $\RR_k$. Note that $c$ is a $(k+1)$-colouring of $\RR_k$. For each vertex $x$ of $\TT^{(k+1)}$, let $B_x$ be the closed in-neighbourhood of $x$. 

We now show that $(B_x:x\in V(\TT^{(k+1)}))$ is a $k$-simple tree-decomposition of $\RR_k$. By construction, each edge $vw$ of $\RR_k$ has its endpoints in $B_w$. In fact, $B_w$ is a clique of size $k+1$ in $\RR_k$, and $c(x) \neq c(y)$ for all distinct $x,y \in B_w$. 
Consider the set of bags that contain a given vertex $u$.  Say $u$ is in the bag $B_w$ for some $w\neq u$. Then $uw\in E(\RR_k)$ and $w$ is a descendant of $u$. Let $v$ be any vertex on the $uw$-path in $\TT^{(k+1)}$. Since $uw\in E(\RR_k)$, we have  $uv\in E(\RR_k)$ and $u$ is also in $B_v$. Hence the sets of bags that contain $u$ form a connected subtree of $\TT^{(k+1)}$. Thus $(B_u:u\in V(\TT^{(k+1)}))$ is a tree-decomposition of $\RR_k$ of width at most $k$. 

We now show that this tree-decomposition is $k$-simple. Since every bag is a clique, it suffices to show that every set $S$ of $k$ vertices is contained in at most two bags. Let $C$ be the set of colours assigned to vertices in $S$. Since $S$ is a clique, $|C|=k$. Let $\alpha$ be the element of $[0,k]\setminus C$. 

Every edge in $S$ is between an ancestor and a descendant in $\TT^{(k+1)}$. Let $u=(x_1,\dots,x_n)$ be the vertex in $S$ that is furthest from the root in $\TT^{(k+1)}$. By construction, $S\subseteq B_u$. Consider a vertex $v\neq u$ such that $S\subseteq B_v$. For each vertex $w\in S\setminus\{u\}$, we have $u\not\in B_w$. Thus $v\not\in S$. Since $B_v$ is a clique, $v$ is coloured $\alpha$. Since $S\subseteq B_v$, $v$ is a descendant of $u$ in $\TT^{(k+1)}$. Suppose that $v$ is not a child of $u$ in $T$. Let $w$ be any internal vertex on the $uv$-path in $T$. Then the colour of $w$ is assigned to some vertex in $S$, which implies that $S\not\subseteq B_v$. Thus $v$ is a child of $u$. Hence $v=(x_1,\dots,x_n,\alpha)$, and $B_u$ and $B_v$ are the only bags containing $S$. 

Therefore $(B_u:u\in V(\TT^{(k+1)}))$ is a $k$-simple tree-decomposition of $\RR_k$. Since $\RR_k$ contains $K_{k+1}$, it has simple treewidth $k$. \end{proof}

There are infinite graphs with simple treewidth $k$ that are not subgraphs of $\RR_k$. The first example is the disjoint union of countably many infinite 2-way paths, which has simple treewidth 1, but is not a subgraph of $\RR_1$. Examples for all $k$ are easily constructed. Nevertheless we have the following characterisation of graphs with simple treewidth $k$. 

\begin{lem}
\label{SimpleTreewidthTreeDecomposition} 
For $k\in\NN$, a graph has simple treewidth $k$ if and only if it has a tree-decomposition over $\RR_k$ with adhesion $k-1$.
\end{lem}

\begin{proof}
Let $G$ be a graph with simple treewidth $k$. Let $(B_x:x\in V(T))$ be a $k$-simple tree-decomposition of $G$. We may assume that $B_x$ is a clique in $G$ for each node $x\in V(T)$, and that $B_x\not\subseteq B_y$ and  $B_y\not\subseteq B_x$ for each edge $xy\in E(T)$. 

Say an edge $xy\in E(T)$ is \defn{thick} if $|B_x\cap B_y|=k$ (which implies $|B_x|=|B_y|=k+1$). For each vertex $x\in V(T)$ there are at most $k+1$ possible values for $B_x\cap B_y$ where $xy$ is a thick edge incident to $x$. So if $x$ is incident with $k+2$ thick edges, then $|B_x\cap B_y|=|B_x\cap B_z|$ for distinct $y,z\in N_T(x)$, implying three bags contain $B_x\cap B_y$, which contradicts the $k$-simplicity of the tree-decomposition. Hence each vertex $x\in V(T)$ is incident with at most $k+1$ thick edges. Let $F$ be the forest with $V(F):=V(T)$ consisting only of the thick edges. Hence $F$ has maximum degree at most $k+1$. 

Let $F_1,F_2,\dots$ be the connected components of $F$. Let $G_i := G [ \bigcup\{ B_x : x\in V(F_i)]$ for $i\in\NN$. 
For each vertex $v\in V(G_i)$, initialise $F_{i,v}$ to be the subtree $F_i[ \{ x\in V(F_i): v \in B_x \} ]$. Let $c:V(G)\to[0,k]$ be a $(k+1)$-colouring of $V(G)$ such that vertices in a common bag are assigned distinct colours. We now show that $G_i \subseteq \GGG{F_i}{c}$ (after a small change to $F_i$ in one case). 
Recall that $F_i$ has maximum degree at most $k+1$. 

First suppose that $F_i$ has a vertex $s$ of degree at most $k$. Orient every edge of $F_i$ away from $s$. Now $F_i$ is 1-oriented, and every node of $F_i$ has out-degree at most $k$. Say $B_s=\{v_1,\dots,v_p\}$ where $p\in[k]$. Add new vertices $x_1,\dots,s_{p-1}$ to $F_i$, where  $(s_1,\dots,s_{p-1},s)$ is a directed path, and let $B_{s_i}:=\{v_1,\dots,v_i\}$ for each $i\in[p-1]$. Observe that $B_x$ is still a clique for each node $x\in V(F)$, and vertices in a common bag are still assigned distinct colours under $c$. Consider $F_i$ to be rooted at $s_1$. Now every node of $F_i$ has out-degree at most $k$ and in-degree 1, except for $s_1$ which has in-degree 0. Note that (a) $|B_{s_1}|=1$, and (b) for every non-root node $x$ of $F_i$, if $yx$ is the incoming arc at $x$, then $|B_x\setminus B_y|=1$. In case (a) rename $s_1$ by the element of $B_{s_1}$. In case (b), rename $x$ by the element of $B_x\setminus B_y$. After this renaming, $V(F_i)=V(G_i)$. By the argument in the proof of \cref{TreewidthCharacterisation}(b), $G_i \subseteq \GGG{F_i}{c}$. 

Now assume that $F_i$ is $(k+1)$-regular. 
Assign each edge $xy$ of $F_i$ the unique colour not present on the vertices of $B_x\cap B_y$, which is well-defined since there are exactly $k$ vertices in $B_x\cap B_y$ and they are assigned distinct colours. 
Observe that $F_i$ is now properly $(k+1)$-edge coloured, and every colour is present at each node of $F_i$. 
Let $P=(\ldots,v_{-2},v_{-1},v_0,v_1,v_2,\dots)$ be an infinite 2-way path in $F_i$, where each edge $v_jv_{j+1}$ is coloured $j\bmod{(k+1)}$. Orient each edge $v_jv_{j+1}$ of $P$ from $v_j$ to $v_{j+1}$ and orient each edge of $F_i-E(P)$ away from $P$. Note that each vertex of $F_i$ has in-degree 1 and out-degree $k$. 
For each vertex $v$ of $G_i$ coloured $\alpha\in[0,k]$, by construction, each edge $xy$ of $F_{i,v}$ is not coloured $\alpha$. Since $F_{i,v}$ intersects $P$ in a subpath, $F_{i,v}$ intersects at most $k$ edges in $P$. In particular, some edge of $P$ is not in $F_{i,v}$. 
If $F_{i,v}$ does not intersect $P$, then the vertex of $F_{i,v}$ closest to $P$ has in-degree 0 in $F_{i,v}$. 
If $F_{i,v}$ does intersect $P$, then the vertex $v_a$ in $P\cap F_{i,v}$ with $a$ minimum has in-degree 0 in $F_{i,v}$ since $v_{a-1}v_a$ is the incoming edge at $v_a$ and $v_{a-1}$ is not in $F_{i,v}$
We have shown that $F_{i,v}$ has a vertex $x_v$ of in-degree 0 for each vertex $v\in V(G_i)$. 
Suppose that $x_v=x_w$ for distinct vertices $v,w\in V(G_i)$. Let $x:=x_v$. 
Let $yx$ be the incoming arc at $x$. 
Since $x$ has in-degree 0 in both $F_{i,v}$ and $F_{i,w}$, we have $v,w\in B_x\cap B_y$, which contradicts the fact that $xy$ is thick. 
Hence $x_v\neq x_w$ for distinct vertices $v,w\in V(G_i)$. 
Rename $x_v$ by $v$. Now $V(F_i)=V(G_i)$. 
By the argument in the proof of \cref{TreewidthCharacterisation}(b), $G_i \subseteq \GGG{F_i}{c}$.

Note that up to isomorphism $\TT^{(k+1)}$ is the unique oriented tree with in-degree 1 and out-degree $k$ at every node.  So $F_i$ is a subtree of $\TT^{(k+1)}$. By the definition of $\RR_k$, $G_i$ is isomorphic to a subgraph of $\RR_k$. Let $T'$ be obtained from $T$ by contracting each subtree $F_i$ into a vertex $x_i$. Then $(V(G_i): x_i \in V(T'))$ is a tree-decomposition of $G$. Since each bag $B_x$ is a clique in $G$, the torso of each node $x_i$ is $G_i$ itself. Hence  $(V(G_i): x_i \in V(T'))$ is a tree-decomposition of $G$ over $\RR_k$. 

We now prove the converse. Let $(B_x:x\in V(T))$ be a tree-decomposition of a graph $G$ over $\RR_k$ with adhesion $k-1$. Let $G_x$ be the torso of $x\in V(T)$. So $G_x$ is isomorphic to a subgraph of $\RR_k$. By \cref{SimpleTreewidthRk}, $G_x$ has a $k$-simple tree-decomposition $(B^x_y: y \in V(T^x))$. Initialise $F$ to be the disjoint union of $\{T^x:x\in V(T)\}$, Associate with each node $y$ of $F$ the corresponding bag $B_y:= B^x_y$.  
Then for each edge $x_1x_2\in V(T)$, let $B^{x_1}_{y_1}$ and $B^{x_2}_{y_2}$ 
be bags respectively in the tree-decomposition of $G_{x_1}$ and $G_{x_2}$ with $B_{x_1}\cap B_{x_2} \subseteq B^{x_1}_{y_1} \cap B^{x_2}_{y_2}$, which exist since $B_{x_1}\cap B_{x_2}$ is a clique in $G_{x_1}$ and in $G_{x_2}$ (by the definition of torso). Add an edge in $F$ between $y_1$ and $y_2$. We obtain a tree-decomposition $(B_y:y\in V(F))$ of $G$. Since the tree-decomposition of each $G_x$ is $k$-simple, and $|B_{x_1}\cap B_{x_2}|\leq k-1$ for each edge $x_1x_2$ of $T$, the tree-decomposition  $(B_y:y\in V(F))$ of $G$ is also $k$-simple. Hence $G$ has simple treewidth at most $k$. 
\end{proof}

Let $\SS_k:= \widehat{\RR_k}^{k-1}$ defined in \cref{TreeDecompUniversalAdhesion}. Then  \cref{SimpleTreewidthTreeDecomposition,TreeDecompUniversalAdhesion} imply:

\begin{thm}
\label{SimpleTreewidthUniversal}
For each $k\in\NN$, the graph $\SS_k$ is universal for the class of graphs with simple treewidth $k$. 
\end{thm}

We now focus on graphs with simple treewidth 2 or 3. 

\begin{thm}
$\SS_2$ is universal for the class of outerplanar graphs.
\end{thm}

\begin{proof}
If a graph $G$ has a tree-decomposition of adhesion 1 in which every torso is outerplanar, then $G$ is also outerplanar. If a graph $G$ has a tree-decomposition of adhesion 1 in which every torso has treewidth at most 2, then $G$ also has treewidth at most 2. Now $\SS_2$ has a tree-decomposition of adhesion 1 in which every torso is isomorphic to $\RR_1$ (which is outerplanar). Thus $\SS_2$ is outerplanar. 

We now prove the converse. That is, $\SS_2$ contains every outerplanar graph. By \cref{SimpleTreewidthUniversal}, it suffices to show that every outerplanar graph has simple treewidth 2. Since simple treewidth is extendable (\cref{SimpleTreewidthExtendable} below), it suffices to show that every finite outerplanar graph has simple treewidth 2, which is a folklore and easily proved result \citep{MJP06,Wulf16,KU12}.
\end{proof}

\begin{thm}
$\SS_3$ is universal for the class of planar graphs with treewidth 3.
\end{thm}

\begin{proof}
If a graph $G$ has a tree-decomposition of adhesion 2 in which every torso is planar, then $G$ is also planar. If a graph $G$ has a tree-decomposition of adhesion 2 in which every torso has treewidth at most 3, then $G$ also has treewidth at most 3. Now $\SS_3$ has a tree-decomposition of adhesion 2 in which every torso is isomorphic to $\RR_3$ (which is planar with treewidth 3). Thus $\SS_3$ is planar with treewidth 3. 

We now prove the converse. That is, $\SS_3$ contains every planar graph with treewidth 3. By \cref{SimpleTreewidthUniversal}, it suffices to show that every planar graph with treewidth 3 has simple treewidth 3. Since simple treewidth is extendable (\cref{SimpleTreewidthExtendable} below), it suffices to show that every finite planar graph with treewidth 3 has simple treewidth 3. This was proved by \citet{KU12} (also see \citep{Wulf16}) using the result of \citet{KV12}, who showed that for every finite planar graph $G$ of treewidth 3 there is a 3-tree (an edge-maximal graph of treewidth 3) that is planar and contains $G$ as a spanning subgraph. 
\end{proof}


%%%%%%%%%%%%%%%%%%%%%%%%%%%
\subsection{Treewidth Extendability}
\label{TreewidthExtendability}

Recall that a graph class $\Gamma$ is \defn{extendable} if the following property holds for every graph $G$: if every finite subgraph of $G$ is in $\Gamma$, then $G$ is in $\Gamma$. (Recall that graphs are assumed to be countable.)

\citet{Thomas88} proved the following result. 

\begin{lem}[\citep{Thomas88}]
\label{TreewidthExtendable}
Treewidth is extendable. 
\end{lem}

See \citep{KT91,Thomassen89} for simpler proofs of \cref{TreewidthExtendable}. The proof of \cref{TreewidthExtendable} due to \citet{Thomassen89} generalises as follows. Define a \defn{forbidden pattern} to be a sequence $\FF=(\FF_k:k\in\NN)$ where $\FF_k$ is any class of finite graphs, such that for every $k\in\NN$, every graph in $\FF_{k+1}$ has a subgraph in $\FF_k$. Define the \hdefn{$\FF$}{width} of a graph $G$ to be the minimum $k\in\NN$ such that $G$ is a subgraph of a chordal graph containing no subgraph in $\FF_k$. If there is no such $k$, then $G$ has \defn{infinite} $\FF$-width. Observe that if $\FF_k=\{K_{k+2}\}$ then $\FF$-width equals treewidth by \cref{TreewidthCharacterisation}, and if $\FF_k=\{K_{k+2},W_k\}$ then $\FF$-width equals simple treewidth by \cref{SimpleTreewidthChordal}.

\begin{lem}
\label{FwidthExtendable}
$\FF$-width is extendable for every forbidden pattern $\FF$.
\end{lem}

\begin{proof}
Let $G$ be an infinite graph such that every finite subgraph of $G$ has $\FF$-width at most $k$. Our goal is to show that $G$ has $\FF$-width at most $k$. Let $\GG$ be the class of graphs $G'$ such that $V(G')=V(G)$, $E(G)\subseteq E(G')$ and every finite subgraph of $G'$ has $\FF$-width at most $k$. Thus $G\in\GG$. Consider the partial order defined on $\GG$ by inclusion (of the corresponding edge sets). 

Let $\XX$ be a chain in $\GG$. Let $G'':= \bigcup_{G'\in\XX} G'$. We claim that $G''\in\GG$. Certainly, $V(G'')=V(G)$ and $E(G)\subseteq E(G'')$. If $G''$ contains a finite subgraph $H$ with $\FF$-width greater than $k$, then since $\XX$ is totally ordered by inclusion and $H$ is finite, some graph $G'''$ in $\XX$ would contain $H$, implying that the $\FF$-width of $G'''$ would be greater than $k$, which contradicts the definition of $\GG$. Hence $G''$ is an upper bound on $\XX$ in $\GG$. 

By \cref{Zorn}, $\GG$ contains a maximal element $G_0$; that is,  $E(G')\subseteq E(G_0)$ for every $G'\in \GG$. We claim that $G_0$ is chordal. Suppose on the contrary that $G_0$ has a chordless cycle $C$ of length at least 4. For each pair of non-adjacent vertices $x,y\in V(C)$, let $G_{xy}$ be the graph obtained from $G_0$ by adding the edge $xy$. Since $G_0$ is maximal with respect to inclusion, $G_{xy}\not\in\GG$. Since $V(G_{xy})=V(G)$ and $E(G)\subseteq E(G_{xy})$, it must be that $G_{xy}$ contains a finite subgraph $H_{xy}$ with $\FF$-width greater than $k$. Let $H$ be the union of $H_{xy}-xy$ taken over every pair of non-adjacent vertices $x,y\in V(C)$. Since $|C|$ is finite and each $H_{xy}$ is finite, $H$ is finite. By construction, $H$ is a subgraph of $G_0$, which is in $\GG$. Thus $H$ has $\FF$-width at most $k$. By the definition of $\FF$-width, $H$ is a subgraph of a chordal graph $H'$ containing no element of $\FF_k$. Since $H'$ is chordal, it contains some chord, say $xy$, of $C$. Now $H_{xy}\subseteq H\subseteq H'$. Since $H'$ has $\FF$-width at most $k$, so does $H_{xy}$, which is a contradiction. Hence $G_0$ is chordal. 

Since $G_0\in\GG$, every finite subgraph of $G_0$ has $\FF$-width at most $k$. In particular, no element of $\FF_k$ is a subgraph of $G_0$ (since elements of $\FF_k$ have $\FF$-width greater than $k$). Hence $G_0$ has $\FF$-width at most $k$. Therefore $G$ has $\FF$-width at most $k$ since $G\subseteq G_0$.
\end{proof}

\cref{FwidthExtendable} implies 
\cref{TreewidthExtendable} as well as the following.

\begin{lem}
\label{SimpleTreewidthExtendable}
Simple treewidth is extendable. 
\end{lem}

The following result generalises \cref{TreewidthExtendable}, and is a direct corollary of a result of \citet[Theorem~3.9]{KT90}.
 
\begin{lem}[\citep{KT90}] 
\label{TreeDecompExtendable}
For every hereditary and extendable class of graphs  $\Gamma$, the class $\DD(\Gamma)$ is extendable. 
\end{lem}

For a graph $G$ and $c\in\NN_0$, let $\tw_c(G)$ be the minimum integer $k$ such that $\tw(G-S)\leq k$ for some $S\subseteq V(G)$ with $|S|\leq c$. Of course, $\tw(G)=\tw_0(G)$ and $\tw(G)\leq \tw_c(G)+c$. We need the following generalisation of \cref{TreewidthExtendable}.

\begin{lem}
\label{TreewidthApexExtendable}
$\tw_c$ is extendable for every $c\in\NN_0$.
\end{lem}

\cref{TreewidthApexExtendable} follows from the next lemma by induction on $c$ (with \cref{TreewidthExtendable} in the base case). 

\begin{lem}
Let $\Gamma$ be a monotone and extendable graph class. Let $\Gamma^+$ be the class of graphs $G$ such that $G-v$ is in $\Gamma$ for some vertex $v$ of $G$. Then $\Gamma^+$ is monotone and extendable. 
\end{lem}

\begin{proof}
Since $\Gamma$ is monotone, so too is $\Gamma^+$. It remains to show that $\Gamma^+$ is extendable. 	Let $G$ be a graph such that every finite subgraph of $G$ is in $\Gamma^+$. 	Let $v_1,v_2,\dots$ be an arbitrary ordering of $V(G)$. For $i\in\NN$, initialise $$X_i:= \{ (i,j): j\in[i], G[\{v_1,\dots,v_{j-1},v_{j+1},\dots,v_i\}]  \in \Gamma \}.$$	Add $(i,0)$ to $X_i$ if $G[\{v_1,\dots,v_{i}\}]  \in \Gamma$. Let $X$ be the graph with vertex-set $V(X):= \bigcup_{i\in\NN} X_i$ and all edges of the form $(i,j)(i+1,j)$, $(i-1,0)(i,i)$, or $(i,0)(i+1,0)$. We now show that for $i\geq 2$, each vertex $(i,j)\in X_i$ has a neighbour in $X_{i-1}$. If $j=0$ then $(i-1,0)$ is a neighbour of $(i,j)$ in $X_{i-1}$. 	Now suppose that $j\in[i-1]$. Then $G[\{v_1,\dots,v_{j-1},v_{j+1},\dots,v_i\}]  \in \Gamma$. Since $\Gamma$ is monotone, $G[\{v_1,\dots,v_{j-1},v_{j+1},\dots,v_{i-1}\}] \in \Gamma$, implying that $(i-1,j)$ is a neighbour of $(i,j)$ in $X_{i-1}$. 	Finally, if $j=i$, then $(i-1,0)$ is a neighbour of $(i,j)$ in $X_{i-1}$. 	So every vertex in $X_i$ has a neighbour in $X_{i-1}$. By \cref{Konig}, $X$ contains an infinite path  $(1,x_1),(2,x_2),\dots$ where $(i,x_i)\in X_i$. By construction, for some $i\in \NN$, we have  $x_1=\dots =x_{i-1}=0$ and $i=x_i=x_{i+1}=x_{i+2}=\cdots$. For every finite subgraph $G'$ of $G-v_i$, there is an integer $n$, such that $G'$ is a subgraph of $G[\{v_1,\dots,v_{i-1},v_{i+1},\dots,v_n\}]$, which is in $\Gamma$. Since $\Gamma$ is 	monotone, $G'\in\Gamma$. Since $\Gamma$ is extendable,  $G-v_i\in\Gamma$. Hence $G\in\Gamma^+$ and $\Gamma^+$ is extendable. 
\end{proof}


%%%%%%%%%%%%%%%%%%%%%%%%%%%%%%%%
\section{Treewidth--Path Product Structure}
\label{TreewidthPathStructure}

The primary result of this section describes a graph, the strong product of a bounded treewidth graph and a path, that contains every planar graph, and satisfies several of the key properties mentioned in \cref{Intro}. The result is extended in various ways for graphs embeddable on any fixed surface, for any proper minor-closed class, and for several non-minor-closed graph classes of interest.

%%%%%%%%%%%%%%%%%%%%%%%%%%%%%%%%%%%%%%%%%
\subsection{Layered Partitions}

A \defn{layering} of a graph $G$ is an ordered partition $(V_0,V_1,\dots)$ of $V(G)$ such that for every edge $vw\in E(G)$, if $v\in V_i$ and $w\in V_j$, then $|i-j| \leq 1$. Note that a layering is equivalent to a partition whose quotient is a path. \citet{DJMMUW20} defined the \defn{layered width} of  a partition $\PART$ of a graph $G$  to be the minimum $\ell\in\NN$ such that for some layering $(V_0,V_1,\dots)$ of $G$, each part in $\PART$ has at most $\ell$ vertices in each layer $V_i$. \citet{DJMMUW20} proved the following lemma in the finite case. The straightforward proof also works for infinite graphs.

\begin{lem}[\citep{DJMMUW20}]
\label{MakeProductGeneral}
If a graph $G$ has a partition $\PART$ with layered width $\ell$, then $G$ is isomorphic to a subgraph of $ (G/ \PART) \boxtimes \PP \boxtimes K_\ell$. Conversely, for every graph $H$ and  subgraph $G$ of $H \boxtimes \PP \boxtimes K_{\ell}$, there is a partition $\PART$ of $G$ with layered width at most $\ell$ such that $G/\PART$ is isomorphic to a subgraph of $H$. 
\end{lem}

For $a,c\in\NN_0$ and $k,\ell\in\NN$, let  $\Gamma_{k,c,\ell,a}$ be the class of graphs isomorphic to subgraphs of $( (\TT_k + K_c) \boxtimes \PP \boxtimes K_{\ell}) + K_a$. \cref{TreewidthUniversal,MakeProductGeneral} imply:

\begin{lem}
\label{ProductSubgraphLayeredWidth}
A graph $G$ is in $\Gamma_{k,c,\ell,a}$ if and only if there is a set $A\subseteq V(G)$ of size at most $a$, and a partition $\PART$ of $G-A$ with layered width $\ell$ such that $\tw_c( (G-A)/\PART) \leq k$. 
\end{lem}

The following notation will be helpful to prove that $\Gamma_{k,c,\ell,a}$ is extendable. If $\PART_1$ and $\PART_2$ are partitions of a set $X$, then \DefNoIndex{$\PART_1\subseteq \PART_2$} if for all $A_1\in\PART_1$ and $A_2\in\PART_2$, either $A_1\cap A_2=\emptyset$ or $A_1\subseteq A_2$. If $\PART_1,\PART_2,\dots$ are partitions of a set $X$ and $\PART_1\subseteq \PART_2 \subseteq\dots$, then $\bigcup_{n\in\NN} \PART_n$ is the partition of $X$, where $A$ is in $\bigcup_{n\in\NN} \PART_n$ if $A\in\PART_n$ for some $n\in\NN$, and for all $n\in\NN$ no strict superset of  $A$ is $\PART_n$. 

\begin{lem}
\label{FiniteToInfiniteGamma}
$\Gamma_{k,c,\ell,a}$ is extendable
\end{lem}

\begin{proof} 
Let $G$ be a graph such that every finite subgraph of $G$ is in $\Gamma_{k,c,\ell,a}$. Let $(v_1,v_2,\dots)$ be a vertex-ordering of $G$. Let $G_n := G[ \{v_1,\dots,v_n\}]$ for $n\in\NN$. Let $X_n$ be the set of all triples $(A_n,\LL_n,\PART_n)$ such that $A_n$ is a set of at most $a$ vertices in $G_n$, $\LL_n$ is a layering of $G_n-A_n$, and $\PART_n$ is a partition of $G_n-A_n$ such that $|L\cap P|\leq \ell$ for all $L\in\LL_n$ and $P\in \PART_n$, and $\tw_c((G_n-A_n)/\PART_n)\leq k$. Since $G_n$ is in $\Gamma_{k,c,\ell,a}$, \cref{ProductSubgraphLayeredWidth} implies that $X_n\neq\emptyset$. And $X_n$ is finite since $G_n$ is finite. Let $Q$ be the graph with vertex-set $V(Q):= \bigcup_{n\in\mathbb{N}} X_n$, where each $(A_n,\LL_n,\PART_n)\in X_n$ is adjacent to $(A_n\setminus \{v_n\},\LL_n-v_n,\PART_n-v_n)$, which is in $X_{n-1}$. By \cref{Konig}, there is a path $((A_1,\LL_1,\PART_1),(A_2,\LL_2,\PART_2),\dots)$ in $Q$ with $(A_n,\LL_n,\PART_n)\in X_n$ for all $n\in\NN$.  Then  $A_1\subseteq A_2\subseteq\dots$  and $\LL_1\subseteq \LL_2\subseteq\dots$  and $\PART_1\subseteq \PART_2\subseteq\dots$. Let $A:=\cup_n A_n$ and $\LL:=\cup_n \LL_n$ and $\PART_n:= \bigcup_n \PART_n$. Then $A$ is a set of at most $a$ vertices in $G$, $\LL$ is a layering of $G-A$, and $\PART$ is a partition of $G-A$ such that $|L\cap P|\leq \ell$ for each $L\in\LL$ and $P\in\PART$. For every finite subgraph $G'$ of $(G-A)/\PART$, for some $n\in\NN$, $G'$ is a subgraph of $(G_n-A_n)/\PART_n$, implying that $\tw_c(G') \leq \tw_c((G_n-A_n)/\PART_n) \leq k$. By \cref{TreewidthApexExtendable}, $\tw_c((G-A)/\PART)\leq k$. Hence $G\in \Gamma_{k,c\,\ell,a}$ by \cref{ProductSubgraphLayeredWidth}. Therefore $\Gamma_{k,c\,\ell,a}$ is extendable.  
\end{proof}

\cref{TreewidthUniversal,FiniteToInfiniteGamma} imply:

\begin{lem}
\label{FiniteToInfiniteProduct}
Let $G$ be a graph such that every finite subgraph of $G$ is isomorphic to a subgraph of $( (H+K_c) \boxtimes P)+K_a$ for some graph $H$ with treewidth at most $k$ and for some path $P$. Then $( (\TT_k + K_c) \boxtimes \PP)+K_a$ contains $G$.
\end{lem}

An analogous proof using \cref{SimpleTreewidthUniversal} instead of 
\cref{TreewidthUniversal} and \cref{SimpleTreewidthExtendable} instead of \cref{TreewidthExtendable} gives:

\begin{lem}
\label{FiniteToInfiniteSimple}
Let $G$ be a graph such that every finite subgraph of $G$ is isomorphic to a subgraph of $( (H+K_c) \boxtimes P)+K_a$ for some graph $H$ with simple treewidth at most $k$ and for some path $P$. Then $( (\SS_k+K_c) \boxtimes \PP)+K_a$ contains $G$.
\end{lem}

%%%%%%%%%%%%%%%%%%%%%%%%%%%%
\subsection{Planar Graphs}
\label{PlanarGraphs}

The starting point for the proof of \cref{InfinitePlanarStructure} is the result of \citet{PS18}, who showed that every planar graph has a partition into geodesic paths whose contraction gives a graph with treewidth at most 8. This result was refined by \citet{DJMMUW20} as follows. 

\begin{thm}[\citep{DJMMUW20}]
\label{FinitePlanarStructure}
Every finite planar graph $G$ is isomorphic to a subgraph of $H\boxtimes P$, for some planar graph $H$ with  treewidth at most 8 and for some path $P$.
\end{thm}

\cref{FinitePlanarStructure} has been used to solve several open problems regarding queue layouts~\citep{DJMMUW20}, non-repetitive colourings~\citep{DEJWW20}, centred colourings~\citep{DFMS21}, clustered colourings~\citep{DEMWW}, adjacency labellings (equivalently, strongly universal graphs)~\citep{BGP20,DEJGMM,EJM}, and vertex rankings~\citep{BDJM}.

\citet{UWY} modified the proof of \cref{FinitePlanarStructure} to establish the following.

\begin{thm}[\citep{UWY}]
\label{FinitePlanarStructure6}
Every finite planar graph $G$ is isomorphic to a subgraph of $H\boxtimes P$, for some planar graph $H$ with simple treewidth at most 6 and for some path $P$.
\end{thm}

\cref{FiniteToInfiniteProduct,FinitePlanarStructure6} imply the following theorem introduced in \cref{Intro}. 

\begin{thm}
\label{InfinitePlanarStructure6}
$\SS_6 \boxtimes \PP$ contains every planar graph.
\end{thm}

\citet{DJMMUW20} also proved the following variation on \cref{FinitePlanarStructure}. 

\begin{thm}[\citep{DJMMUW20}]
\label{SimpleFinitePlanarStructure}
Every finite planar graph $G$ is isomorphic to a subgraph of $H\boxtimes P \boxtimes K_3$ for some planar graph $H$ with $\tw(H)\leq 3$ and for some path $P$. 
\end{thm}

In \cref{SimpleFinitePlanarStructure}, since $H$ is planar and $\tw(H)\leq 3$, by the above-mentioned result of \citet{KV12}, we have $\stw(H)\leq 3$. This can also be seen directly from the proof of \cref{SimpleFinitePlanarStructure} and is implicitly mentioned in \citep{DJMMUW20}. Then \cref{FiniteToInfiniteProduct,SimpleFinitePlanarStructure} imply:

\begin{thm}
\label{SimpleInfinitePlanarStructure}
$\SS_3\boxtimes \PP \boxtimes K_3$ contains every planar graph.
\end{thm}

We now show that the graphs described in \cref{InfinitePlanarStructure6,SimpleInfinitePlanarStructure} satisfy properties \cref{KeyPropertyBoundedMinDegree}--\cref{KeyPropertyBoundedColouringNumbers} from \cref{Intro}. It follows from results of \citet{HW21} that every finite subgraph of $\SS_6\boxtimes \PP$ has minimum degree at most 19 and 
every finite subgraph of $\SS_3\boxtimes \PP \boxtimes K_3$ has minimum degree at most 34. Note that $\chi( \SS_6\boxtimes \PP) \leq 14$ and $\chi(\SS_3\boxtimes \PP \boxtimes K_3)\leq 24$ since $\chi(A\boxtimes B) \leq \chi(A)\,\chi(B)$. 
\cref{ProductTreewidth} implies that every $n$-vertex subgraph of  $\SS_6 \boxtimes \PP$ or $\SS_3\boxtimes \PP \boxtimes K_3$ has treewidth $O(\sqrt{n})$ and has a balanced separation of order $O(\sqrt{n})$.  \cref{GenColourProduct} imply that  $\SS_6 \boxtimes \PP$ and $\SS_3\boxtimes \PP \boxtimes K_3$ (since $\tw(\SS_3\boxtimes K_3)\leq 11$) have linear  colouring numbers. Together, this says that $\SS_6 \boxtimes \PP$ and $\SS_3\boxtimes \PP \boxtimes K_3$ satisfy properties \cref{KeyPropertyBoundedMinDegree}--\cref{KeyPropertyBoundedColouringNumbers} from \cref{Intro}.

%%%%%%%%%%%%%%%%%%%%%%%%%%%%%%%%
\subsection{Graphs on Surfaces}
\label{Genus}

\citet{DJMMUW20} proved the following results for finite graphs of bounded Euler genus. A graph $X$ is \defn{apex} if $X-v$ is planar for some vertex $v$. 

\begin{thm}[\citep{DJMMUW20}]
\label{FiniteSurfaceProduct}
Every finite graph $G$ of Euler genus $g$ is isomorphic to a subgraph of:
\begin{enumerate}[label=(\alph*)]
\item $H \boxtimes P \boxtimes K_{\max\{2g,1\}}$ for some apex graph $H$ of treewidth 9 and some path $P$, 
\item $H \boxtimes P \boxtimes K_{\max\{2g,3\}}$ for some apex graph $H$ of treewidth 4 and some path $P$,
\item $(K_{2g} + H ) \boxtimes P$ for some planar graph $H$ of treewidth 8 and some path $P$.
\end{enumerate}
\end{thm}

Since $\stw(G)\leq \stw(G-v)+1$ for every graph $G$ and vertex $v$ of $G$, applying \cref{FinitePlanarStructure6} instead of \cref{FinitePlanarStructure}, the treewidth bounds in \cref{FiniteSurfaceProduct} become:

\begin{thm}
\label{SimpleFiniteSurfaceProduct}
Every finite graph $G$ of Euler genus $g$ is isomorphic to a subgraph of: 
\begin{enumerate}[label=(\alph*)]
\item $ H \boxtimes P \boxtimes K_{\max\{2g,1\}}$ for some apex graph $H$ of simple treewidth 7 and some path $P$, 
\item $H \boxtimes P \boxtimes K_{\max\{2g,3\}}$ for some apex graph $H$ of simple treewidth 4 and some path $P$,
\item $(H + K_{2g} ) \boxtimes P$ for some planar graph $H$ of simple treewidth 6 and some path $P$.
\end{enumerate}
\end{thm}

\cref{FiniteToInfiniteProduct,SimpleFiniteSurfaceProduct} imply:

\begin{thm}
\label{InfiniteSurfaceProduct}
For every $g\in\NN_0$, each of
$(\SS_6+K_1) \boxtimes \PP \boxtimes K_{\max\{2g,1\}}$, 
$(\SS_3+K_1) \boxtimes \PP \boxtimes K_{\max\{2g,3\}}$ and  
$(\SS_6 +K_{2g} )  \boxtimes \PP$ 
contain every graph of Euler genus $g$.
\end{thm}

\citet{DJMMUW20} proved the following product structure theorem for finite apex-minor-free graphs. 

\begin{thm}[\citep{DJMMUW20}] 
\label{ApexMinorFree}
For every finite apex graph $X$, there exists $c\in\NN$ such that every finite  $X$-minor-free graph $G$ is isomorphic to a subgraph of $H\boxtimes P$ for some graph with $H$ with $\tw(H)\leq c$ and for some path $P$.
\end{thm}

\cref{FiniteToInfiniteProduct,ApexMinorFree} imply:

\begin{thm} 
For every finite apex graph $X$ there exists $c\in \NN$ such that $\TT_c \boxtimes \PP$ contains  every  $X$-minor-free graph $G$.
\end{thm}

\citet{DEMWW} showed an analogous result for  bounded degree graphs excluding an arbitrary fixed minor.

\begin{thm}[\citep{DEMWW}]
\label{FiniteMinorFreeDegreeStructure}
For every finite graph $X$ there exists $c\in\NN$ such that for all $\Delta\in\NN$ every finite $X$-minor-free graph with maximum degree $\Delta$ is isomorphic to a subgraph of $H\boxtimes P$ for some graph $H$ with $\tw(H) \leq c\Delta$ and for some path $P$. \end{thm}

\cref{FiniteToInfiniteProduct,FiniteMinorFreeDegreeStructure} imply:

\begin{thm}
\label{MinorFreeDegreeStructure}
For every graph $X$ there exists $c\in\NN$ such that for all $\Delta\in\NN$, the graph $\TT_{c\Delta} \boxtimes \PP$ contains every  $X$-minor-free graph with maximum degree at most $\Delta$.
\end{thm}

\subsection{Beyond Minor-Closed Classes}
\label{NonMinorClosed}

A recent direction pursued by \citet{DMW19b} studies graph product structure theorems for various non-minor-closed graph classes. Here we extend their results for infinite graphs. First consider graphs that can be drawn on a fixed surface with a bounded number of crossings per edge. A graph is \hdefn{$(g,k)$}{planar} if it has a drawing in a surface of Euler genus at most $g$ such that each edge is involved in at most $k$ crossings. Even in the simplest case, there are $(0,1)$-planar graphs that contain arbitrarily large complete graph minors \citep{DEW17}. 

\begin{thm}[\citep{DMW19b}] 
\label{gkPlanarStructure}
Every finite $(g,k)$-planar graph is isomorphic to a subgraph of $H\boxtimes P$, for some graph $H$ of treewidth $O(gk^6)$ and for some path $P$. 
\end{thm}

\cref{FiniteToInfiniteProduct,gkPlanarStructure} imply:

\begin{thm}
\label{gkPlanarInfinite}
For all $g,k\in\NN$ there exists an integer $c\in O(gk^6)$ such that $\TT_c\boxtimes \PP$ contains every $(g,k)$-planar graph.
\end{thm}

Map graphs, which are defined as follows, provide another example of a non-minor-closed classes that has a product structure theorem. Start with a graph $G_0$ embedded in a surface of Euler genus $g$, with each face labelled a `nation' or a `lake', where each vertex of $G_0$ is incident with at most $d$ nations. Let $G$ be the graph whose vertices are the nations of $G_0$, where two vertices are adjacent in $G$ if the corresponding faces in $G_0$ share a vertex. Then $G$ is called a \hdefn{$(g,d)$}{map graph}. A $(0,d)$-map graph is called a (plane) \hdefn{$d$}{map graph}; see \citep{FLS-SODA12,CGP02} for example. The $(g,3)$-map graphs are precisely the graphs of Euler genus at most $g$; see \citep{DEW17}. So $(g,d)$-map graphs generalise graphs embedded in a surface, and we now assume that $d\geq 4$ for the remainder of this section. 

\begin{thm}[\citep{DMW19b}]
\label{MapPartition}
Every finite $(g,d)$-map graph is  isomorphic to  a subgraph of:
\begin{itemize}
\item  $H\boxtimes P\boxtimes K_{O(gd^2)}$, where $H$ is a graph with treewidth at most 14 and $P$ is a path,
\item $H\boxtimes P$, where $H$ is a graph with treewidth $O(gd^2)$ and $P$ is a path.
\end{itemize}
\end{thm}

\cref{FiniteToInfiniteProduct,MapPartition} imply:

\begin{thm}
For all $g,d\in\NN$ there exists $c\in O(gd^2)$ such that $\TT_c\boxtimes \PP$ contains every  $(g,d)$-map graph.
\end{thm}

A \defn{string graph} is the intersection graph of a set of curves in the plane with no three curves meeting at a single point; see  \cite{PachToth-DCG02,FP10,FP14} for example. For $\delta\in\mathbb{N}$, if each curve is in at most $\delta$ intersections with other curves, then the corresponding string graph is called a \hdefn{$\delta$}{string graph}. A \hdefn{$(g,\delta)$}{string} graph is defined analogously for curves on a surface of Euler genus at most $g$.  

\begin{thm}[\citep{DMW19b}] 
\label{StringPartition}
Every finite  $(g,\delta)$-string graph is isomorphic to a subgraph of $H\boxtimes P$, for some graph $H$ of treewidth $O(g\delta^7)$ and some path $P$.
\end{thm}

\cref{FiniteToInfiniteProduct,StringPartition} imply:

\begin{thm}
For all $g,\delta\in\NN$ there exists $c\in O(g\delta^7)$ such that $\TT_c \boxtimes \PP$ contains every  $(g,\delta)$-string graph.
\end{thm}

Finally, we mention the following theorem of \citet{DMW19b}. The \DefnIndex{$k$-th power}{power} of a graph $G$ is the graph $G^k$ with $V(G^k):=V(G)$, where $vw\in E(G^k)$ whenever $\dist_G(v,w)\in[k]$. 

\begin{thm}[\citep{DMW19b}] 
\label{kPower}
For every finite graph $X$ there exists $c\in\NN$ such that for all $k,\Delta\in\NN$ and for every finite $X$-minor-free graph $G$ with maximum degree $\Delta$, the $k$-th power $G^k$ is isomorphic to a subgraph of $H\boxtimes P$, for some graph $H$ of treewidth $k^4(c\Delta)^{k}$ and some path $P$.
\end{thm}

\cref{FiniteToInfiniteProduct,kPower} imply:

\begin{thm}
For every finite graph $X$ there exists $c\in\NN$ such that for all $k,\Delta\in\NN$, for some integer $t\leq k^4(c\Delta)^{k}$, the graph
$\TT_t\boxtimes \PP$ contains the $k$-th power $G^k$ of every  $X$-minor-free graph $G$ with maximum degree $\Delta$.
\end{thm}

%%%%%%%%%%%%%%%%
\subsection{Excluding an Arbitrary Minor}
\label{ExcludedMinors}

\citet{DJMMUW20} proved the following product structure theorem for finite graphs excluding a fixed minor. 

\begin{thm}[\citep{DJMMUW20}] 
\label{MinorProduct}
For every finite graph $X$ there exist $k,a\in\mathbb{N}$ such that every finite $X$-minor-free graph $G$ can be obtained by clique-sums of graphs $G_1,\dots,G_n$ such that for  $i\in[n]$, 
for some graph $H_i$ with treewidth at most $k$ and some path $P_i$, $G_i$ is isomorphic to a subgraph of $(H_i  \boxtimes P_i ) + K_a$. 
\end{thm}

%\david{Add the bound $a\leq t-1$ if $X$ is $t$-apex.}

\begin{thm}
\label{MinorUniversal}
For every finite graph $X$ there is a constant $c$ and there is a graph $U_X$ that contains every $X$-minor-free graph, such that every $n$-vertex subgraph $G$ of $U_X$ has treewidth at most $c \sqrt{n}$ and has a balanced separation of order $c\sqrt{n}$, and $\col_r(G) \leq cr$ for every $r\in\NN$.
\end{thm}

\begin{proof}
\cref{FiniteToInfiniteGamma,TreeDecompExtendable} implies that 
$\DD(\Gamma_{k,c,\ell,a})$ is extendable for all $k,\ell\in\NN$ and $c,a\in\NN_0$. 
\cref{MinorProduct} says that every finite $X$-minor-free graph is in 
$\DD( \Gamma_{k,0,1,a} )$ for some $k,a$ depending only on $X$. 
Thus every $X$-minor-free graph is in $\DD( \Gamma_{k,0,1,a} )$. 
Let $U_X :=\reallywidehat{( \TT_k \boxtimes \PP ) + K_a}$. By  \cref{TreeDecompUniversal}, $U_X$ contains every  $X$-minor-free graph. \cref{ProductTreewidth,TreewidthOverTreeDecomposition} imply that every $n$-vertex subgraph of $U_X$ has treewidth at most $c\sqrt{n}$ and has a balanced separation of order $c\sqrt{n}$. \cref{GenColourProductTreeDecomp,GenColourTreeDecomp} imply that $U_X$ has linear  colouring numbers. \cref{SergeyCorollary} implies that $U_X$ has linear expansion. 
\end{proof}

We finish this section by mentioning one more property of the graph $U_X$ in \cref{MinorUniversal}. \citet{DDOSRSV04} proved that for every finite graph $X$ and integer $k\geq 2$, there is an integer $c$ such that every finite $X$-minor-free graph (a) has a vertex $k$-colouring such that the subgraph induced by any $k-1$ colour classes has treewidth at most $c$, and (b) has an edge $k$-colouring such that the subgraph induced by any $k-1$ colour classes has treewidth at most $c$.  \citet{DDOSRSV04} called these `low treewidth colourings'. \citet{DJMMUW20} showed that this result is a corollary of \cref{MinorProduct}. Their proof also implies that $U_X$ has low treewidth colourings. It is interesting that not only does every $X$-minor-free graph have low treewidth colourings, but there is a single graph that contains all $X$-minor-free graphs that itself admits low treewidth colourings. 

%%%%%%%%%%%%%%%%%%%%%%%%%%%%%%%%%%%%%%%%%%%%%%%%%%%%%
\section{Avoiding an Infinite Complete Graph Subdivision}
\label{NoInfiniteSubdivision}

This section focuses on key property \cref{KeyPropertyNoInfiniteSubdivision}, which says that a graph that contains every planar graph should contain no $K_{\aleph_0}$ subdivision. We start the section by giving several characterisations of graphs that contain no $K_{\aleph_0}$ subdivision (\cref{NoInfiniteSubdivisionCharacterisation}). We then construct a graph that contains every planar graph, but contains no $K_{\aleph_0}$ subdivision, amongst other key properties. This shows that \cref{InfinteCliqueMinor} is best possible in the sense that it cannot be strengthened to conclude that every graph that contains every planar graph must contain a subdivision of $K_{\aleph_0}$. Our construction is, in fact, stronger in two respects: it works for $K_t$-minor-free graphs, and actually gives induced subgraphs. Using the same method, we construct a strongly universal treewidth-$k$ graph, and a graph that contains every locally finite graph as an induced subgraph, but contains no $K_{\aleph_0}$ subdivision (\cref{LocallyFiniteGraphs}). We conclude the section by showing that every graph that contains all planar graphs has a subgraph of infinite edge-connectivity that contains all planar graphs (\cref{InfiniteEdgeConnectivity}). 

%%%%%%%%%%%%%%%%%%%%%%%%%%%%%%%%%%%%%%%%%%%%%%%
\subsection{Characterisations}
\label{NoInfiniteSubdivisionCharacterisation}

\citet{RST-TAMS92} characterised graphs containing no subdivision of $K_{\aleph_0}$ in terms of simplicial decompositions; see parts (a) and (b) in \cref{InfiniteCompleteSubdivision} below. Part \cref{NoSubTreeDec} is an equivalent formulation of \cref{NoSubSimpDec} in terms of tree-decompositions; see \cref{SimpDecTreeDec}. \citet{Diestel94} gave a simple proof showing that \cref{NoSub} implies \cref{NoSubTreeDec}, thus establishing the characterisation of \citet{RST-TAMS92} as a corollary. Diestel's proof uses normal spanning trees; see  \citep{Jung69,BD94,Pitz20} for more on this theme. Parts \cref{NoSubChordal}--\cref{NoSubModel} in \cref{InfiniteCompleteSubdivision} provide a number of equivalent characterisations related to other parts of this paper. In particular, condition \cref{NoSubChordalPart} is used in \cref{ExcludedMinorNoSubdiv} below. Condition \cref{NoSubModel} is interesting since it is an infinite analogue of $r$-shallow minors; see \cref{Expansion}.

\begin{thm}
\label{InfiniteCompleteSubdivision}
The following are equivalent for any countable graph $G$:
\begin{enumerate}[(a)]
\item\label{NoSub} $G$ contains no subdivision of $K_{\aleph_0}$;
\item\label{NoSubSimpDec} $G$ is a spanning subgraph of a chordal graph that contains no $K_{\aleph_0}$ and has a simplicial decomposition in which each simplicial summand is a finite complete graph;
\item\label{NoSubTreeDec} $G$ has a tree-decomposition such that for every infinite set $X\subseteq V(G)$ there exist distinct $v,w\in X$ in no common bag;
\item\label{NoSubChordal} $G$ is a spanning subgraph of a chordal graph with no $K_{\aleph_0}$ subgraph;
\item\label{NoSubChordalPart} $G$ is a spanning subgraph of a graph $G'$ that has a finite chordal partition $\PART$ such that $G'/\PART$ has no $K_{\aleph_0}$ subgraph;
\item\label{NoSubModel} $G$ contains no model of $K_{\aleph_0}$ with branch sets of finite radius.
\end{enumerate}
\end{thm}

As explained above \cref{NoSub}, \cref{NoSubSimpDec} and  \cref{NoSubTreeDec} are equivalent. It is immediate that \cref{NoSubSimpDec} $\Longrightarrow$ \cref{NoSubChordal}. 

\begin{proof}
\cref{NoSubChordal} $\Longrightarrow$ \cref{NoSubChordalPart}:
Assume that $G$ is a spanning subgraph of a chordal graph $G'$ with no $K_{\aleph_0}$ subgraph. Let $\PART$ be the (trivial) chordal partition $\{\{v\}:v\in V(G')\}$ of $G'$. So $G'/\PART\cong G'$, and thus $G'/\PART$ has no $K_{\aleph_0}$ subgraph.
\end{proof}

\begin{proof}
\cref{NoSubChordalPart} $\Longrightarrow$ \cref{NoSub}: Assume that $\PART$ is a finite chordal partition of $G$ such that $G/\PART$ contains no $K_{\aleph_0}$ subgraph. Suppose that $V$ is an infinite set of vertices in $G$ such that no  pair of vertices in $V$ are separated by a finite vertex-cut. For each $v\in V$, let $A_v$ be the part of $\PART$ containing $v$. Since each part of $\PART$ is finite, we may assume (by taking a subset) that $A_v\neq A_w$ for all distinct $v,w\in V$. Suppose that $A_vA_w\notin E(G/\PART)$ for distinct $v,w\in V$. Let $S$ be a minimal subset of $V(G/\PART)$ separating $A_v$ and $A_w$. Since $G/\PART$ is chordal, $S$ is a clique in $G/\PART$, which is finite by assumption. Hence $\bigcup_{X\in S} X$ is a finite subset of $V(G)$ separating $v$ and $w$, which is a contradiction. Thus $A_vA_w\in E(G/\PART)$ for all distinct $v,w\in V$. Therefore $\{A_v:v\in V\}$ induces $K_{\aleph_0}$ in $G/\PART$. This contradiction shows that there is no infinite set $V$ of vertices in $G$ such that each pair of vertices in $V$ are separated by no finite cut. In particular, $G$ contains no $K_{\aleph_0}$ subdivision. 
\end{proof}

\begin{proof}
\cref{NoSubModel} $\Longrightarrow$ \cref{NoSub}: If $G$ contains a subdivision of $K_{\aleph_0}$, then $G$ contains a model of $K_{\aleph_0}$ with branch sets of finite radius. 
\end{proof}

\begin{proof}
\cref{NoSub} $\Longrightarrow$ \cref{NoSubModel}: 
Assume that $G$ contains a model $(X_i)_{i\in\NN}$ of $K_{\aleph_0}$ such that $X_i$ has finite radius for each $i\in\NN$. 
We may assume that $V(G)=\bigcup_{i\in\NN}V(X_i)$, and that each $X_i$ is minimal with the above properties. So each $X_i$ is a tree with finite radius, and every leaf of $X_i$ is adjacent to some vertex in $X_j$ for some $j\neq i$. Say $X_i$ is \defn{clean} if:
\begin{enumerate}[(i)] 
\item for each $j\in\NN\setminus\{i\}$, there is a vertex $v_{i,j}\in X_i$ adjacent to some vertex in $X_j$, 
\item there is a vertex $r_i$ of infinite degree in $X_i$, such that for all distinct $j,k\in\NN$ with $j,k>i$, the $r_iv_{i,j}$-path and the $r_iv_{i,k}$-path in $X_i$ only intersect at $r_i$. 
\end{enumerate}
Suppose that $X_1,\dots,X_{i-1}$ are clean for some $i\in\NN$. 
Since $(X_i)_{i\in\NN}$ is a model of a complete graph, for each $j\in\NN$ with $j<i$, there is a vertex $v_{i,j}\in X_i$ adjacent to some vertex in $X_j$. Since $X_i$ has finite radius, some vertex $r_i\in V(X_i)$ has infinite degree in $G$. By the minimality of $X_i$, for some infinite set $I\subseteq\{i+1,i+2,\dots\}$,  for distinct $j,k\in I$, the $r_iv_{i,j}$-path and the $r_iv_{i,k}$-path in $X_i$ only intersect at $r_i$. Replace $X_i$ by the minimal subtree of $X_i$ containing
the union of the $r_iv_{i,j}$-paths in $X_i$, taken over all $j\in\{1,\dots,i-1\}\cup I$. Replace $(X_i)_{i\in\NN}$ by 
$(X_i)_{i\in\{1,\dots,i\}\cup I}$. 
In this sequence, $X_1,\dots,X_{i}$ are clean. 

Repeat this step to obtain an infinite $K_{\aleph_0}$-model $(X_i)_{i\in\NN}$ in which every $X_i$ is clean. We may assume that $v_{i,j}v_{j,i}$ is an edge of $G$ for all distinct $i,j\in\NN$. 
We now construct a subdivision of $K_{\aleph_0}$ in $G$ rooted at $(r_{i^2})_{i\in\NN}$. 
For $i,j\in\NN$ with $i<j$, let $P_{i,j}$ be the path
from $r_{i^2}$ to $v_{i^2,j^2+i}$ in $X_{i^2}$, 
followed by the edge $v_{i^2,j^2+i}v_{j^2+i,i^2}$,
followed by the path from $v_{j^2+i,i^2}$ to $v_{j^2+i,j^2}$ in $X_{j^2+i}$, 
followed by the edge $v_{j^2+i,j^2}v_{j^2,j^2+i}$,
followed by the path from $v_{j^2,j^2+i}$ to $r_{j^2}$ in $X_{j^2}$. 
Note that the $P_{i,j}$ are internally disjoint
(since $i<j$ and $\ell<k$ and $j<k$ implies $j^2+i<k^2+\ell$). 
So $G$ contains a $K_{\aleph_0}$ subdivision. 
\end{proof}


%%%%%%%%%%%%%%%%%%%%%%%%%%%%%%%%%%%%%%%%%%%%%%%%%%%%%%
\subsection{$K_t$-Minor-Free Graphs}
\label{ExcludedMinorNoSubdiv}

The main result of this section, \cref{KtMinorFreeNoSubdiv}, says that for every  integer $t\geq 3$ there is a graph $U$ that contains every $K_t$-minor-free graph as an induced subgraph, and $U$ satisfies key properties \cref{KeyPropertyBoundedMinDegree}, \cref{KeyPropertyBoundedColouringNumbers}
and \cref{KeyPropertyNoInfiniteSubdivision}. The main tool is that of a chordal partition, which have been used by several authors in the study of $K_t$-minor-free graphs~\cite{ReedSeymour-JCTB98,vdHW18,Andreae86,HOQRS17,SSW19}. Chordal partitions are useful since if $\PART$ is a connected chordal partition of a $K_t$-minor-free graph $G$, then the quotient $G/\PART$ is a minor of $G$, implying $G/\PART$ has no $K_t$ subgraph and $\tw(G/\mathcal{P})\leq t-2$ by \cref{TreewidthCharacterisation}. Thus, using chordal partitions, the structure of $K_t$-minor-free graphs can be described in terms of much simpler graphs of treewidth at most $t-2$. We also use the machinery developed to construct a strongly universal treewidth-$k$ graph~(\cref{StronglyUniversalTreewidth}) and a graph that contains every locally finite graph as an induced subgraph, but contains no $K_{\aleph_0}$ subdivision~(\cref{StronglyUniversalLocallyFinite}).

All these results rely on the following definitions. For $n\in\NN_0$, let $\HH_n=\{H_{n,1},H_{n,2},\dots\}$ be a class of finite graphs. Let $\HH:=\bigcup_{n\in\NN_0}\HH_n$. 
Let $\XX:=\{(m,n):m\in\NN_0,n\in\NN,m<n\}$. 
For $(m,n)\in \XX$, say $(J,A,B)$ is an \hdefn{$(\HH,m,n)$}{triple} if:
\begin{itemize}
    \item $J$ is a graph with $V(H)=A\cup B$ and $A\cap B=\emptyset$;
    \item $J[A]$ is isomorphic to some graph in $\HH_m$; and 
    \item $J[B]$ is isomorphic to some graph in $\HH_n$.
\end{itemize}
If $(J_1,A_1,B_1)$ and $(J_2,A_2,B_2)$ are $(\HH,m,n)$-triples, then write $(J_1,A_1,B_1)\cong (J_2,A_2,B_2)$ if there is an isomorphism $\phi:V(J_1)\to V(J_2)$ such that $\phi(A_1)=A_2$ and $\phi(B_1)=B_2$. 
An \hdefn{$\HH$}{system} is a set 
$\JJ=\{ (J_{m,a,n,b,c},A_{m,a},B_{n,b}) :(m,n)\in\XX,\,a,b,c\in\NN\}$ such that for all $(m,n)\in \XX$ and $a,b,c\in\NN$, 
\begin{itemize}
\item $(J_{m,a,n,b,c},A_{m,a},B_{n,b})$ is an $(\HH,m,n)$-triple;
\item  $J[A_{m,a}]\cong H_{m,a}$ and $J[B_{n,b}]\cong H_{n,b}$;
\item if $c=1$ then there are no edges between $A_{m,a}$ and $B_{n,b}$ in $J_{n,a,m,b,c}$.
 \end{itemize}

For any $\HH$-system $\JJ$ and $k\in\NN$, let \defn{$\JJ^{(k)}$} be the set of all graphs $G$ with the following properties:
\begin{itemize}
\item there is a partition $\PART$ of $G$ and a  tree $T$ with $V(T)=\PART$ rooted at some part $R\in\PART$; 
\item there is a $(k+1)$-colouring $c$ of $T$ such that $G/\PART$ is a spanning subgraph of $Q:=\GGG{T}{c}$ (which implies $\tw(G/\PART)\leq \tw(Q) \leq k$ by \cref{TreewidthCharacterisation}); and
\item there is a labelling of $Q$, such that:
\begin{itemize}
\item for each part $A\in\PART$, 
if $\dist_T(R,A)=m$ and $A$ is labelled $a$, then $G[A]\cong H_{m,a}$; and
\item for every oriented edge $AB\in E(Q)$, 
\begin{itemize}
\item if $\dist_T(R,A)=m$ and $A$ is labelled $a\in\NN$ and $\dist_T(R,B)=n$ and $B$ is labelled $b\in\NN$ and $AB$ is labelled $c\in\NN$, then $$(G[A\cup B],A,B)\cong (J_{m,a,n,b,c},A_{m,a},B_{n,b})$$ 
(which implies that $G[A]\cong H_{m,a}$ and $G[B]\cong H_{n,b}$); and
\item if $AB\in E(Q)\setminus E(G/\PART)$ then $c=1$. 
\end{itemize}
\end{itemize}
\end{itemize}

\begin{lem}
\label{JJJ}
For every class $\HH=\{H_{m,a}:m\in\NN_0,a\in\NN\}$ of finite graphs, for every $\HH$-system $\JJ=\{ (J_{m,a,n,b,c},A_{m,a},B_{n,b}) :(m,n)\in\XX,\,a,b,c\in\NN\}$, and for every $k\in\NN$, 
\begin{itemize}
\item no graph in $\JJ^{(k)}$ contains a subdivision of $K_{\aleph_0}$;
\item $\JJ^{(k)}$ has a strongly universal graph $U_{\JJ,k}$;
\item if, for some $b\in\NN$, every graph in $\HH$ is $b$-colourable, then every graph in $\JJ^{(k)}$ is $b(k+1)$-colourable; and
\item if, for some $d,r\in \NN$, every graph in $\HH$ is $d$-degenerate, and for all $(m,n)\in\XX$ and $a,b,c\in\NN$, in the graph $J_{m,a,n,b,c}$, each vertex in $B_{n,b}$ has at most $r$ neighbours in $A_{m,a}$, then every finite subgraph of any graph in $\JJ^{(k)}$ is $(rk+d)$-degenerate.
\end{itemize}
\end{lem}


\begin{proof}
Let $G\in\JJ^{(k)}$. Let $\PART$ be the partition of $G$ and let $Q$ be the supergraph of $G/\PART$ witnessing that $G\in\JJ^{(k)}$. Let $G'$ be the graph obtained by adding edges to $G$ so that $A\cup B$ is a clique in $G'$ for each edge $AB\in E(Q)$. (This makes sense since $A,B\subseteq V(G)$.)\ Now, $\PART$ is a connected partition of $G'$ with quotient $Q$, so $\PART$ is a chordal partition. Since every graph in $\HH$ is finite, $\PART$ is a finite chordal partition. By \cref{InfiniteCompleteSubdivision}\cref{NoSubChordalPart}, $G'$ and thus $G$ contains no $K_{\aleph_0}$ subdivision. This proves the first claim. 

Let $\TT$ be the universal tree rooted at an arbitrary vertex $r$ (see \cref{UniversalTree}). Let $c$ be a $(k+1)$-colouring of $\TT$, such that each vertex $v$ of $\TT$ has infinitely many children of each colour distinct from the colour assigned to $v$. So $\TT_k=\GGG{\TT}{c}$ is the universal treewidth-$k$ graph. By \cref{TreewidthUniversalPreserving}, there is a $k$-orientation and labelling of $\TT_k$, such that for every chordal graph $Z$ with no $K_{k+2}$ subgraph, for every rooted $k$-orientation of $Z$, and for every labelling of $Z$, there is an orientation-preserving label-preserving isomorphism from $Z$ to a subgraph of $\TT_k$. 

The graph $U=U_{\JJ,k}$ is obtained from $\TT_k$ as follows. For each vertex $v$ of $\TT$ at distance $m$ from $r$ in $\TT$ and labelled $a\in\NN$, replace $v$ by a copy of $H_{m,a}$ with vertex set $U_v\subseteq V(U)$, where $U_v\cap U_w=\emptyset$ for all distinct $v,w\in V(\TT_k)$. Then, for each oriented edge $vw$ of $\TT_k$ labelled $c\in \NN$, if $(m,n):=(\dist_{\TT}(r,v),\dist_{\TT}(r,w))\in\XX$ and $v$ is labelled $a\in\NN$ and $w$ is labelled $b\in\NN$, then add edges between $U_v$ and $U_w$ so that 
\begin{equation}
\label{Iso}
(U[ U_v \cup U_w ], U_v, U_w) \cong (J_{m,a,n,b,c},A_{m,a},B_{n,b}).
\end{equation}
So $\{U_v:v\in V(\TT_k)\}$ is a partition of $U$, where $\TT_k$ takes the role of $Q$ in the above definition of $\JJ^{(k)}$. So $U\in \JJ^{(k)}$. 

We now show that every graph $G$ in $\JJ^{(k)}$ is isomorphic to an induced subgraph of $U$. Let $\PART$, $T$, $R$, $c$ and $Q=\GGG{T}{c}$ be as in the definition of $\JJ^{(k)}$. Thus, there is a rooted $k$-orientation and labelling of $Q$, such that for every oriented edge $AB\in E(Q)$, 
if $(m,n):=(\dist_T(R,A),\dist_T(R,B))\in\XX$ and $A$ is labelled $a\in\NN$ and $B$ is labelled $b\in\NN$ and $AB$ is labelled $c\in\NN$, then 
\begin{equation}
\label{IsoIso}
(G[A\cup B],A,B)\cong (J_{m,a,n,b,c},A_{m,a},B_{n,b}), 
\end{equation}
and if $AB\in E(Q)\setminus E(G/\PART)$ then $c=1$.

Let $Q'$ be a chordal graph containing $Q$ as a spanning subgraph, with a $k$-orientation that preserves the given $k$-orientation of $Q$, and subject to these conditions is edge-maximal. Transfer the labels of the vertices and edges of $Q$ to $Q'$. Label each edge of $Q'-E(Q)$ by 1. By 
\cref{Maximal}, there is an orientation-preserving label-preserving isomorphism $\phi$  from $Q'$ to an induced subgraph of $\TT_k$. Thus, for every part $A\in\PART$ labelled $a\in\NN$, we have $\phi(A)$ is a vertex in $\TT_k$ labelled $a$. Let 
$$U_G:=U\left[\bigcup_{A\in\PART} U_{\phi(A)}\right].$$ 
Since $\phi$ is orientation-preserving and label-preserving, for every oriented edge $AB\in E(Q')$ labelled $c\in\NN$, if $A$ is labelled $a\in\NN$ and $B$ is labelled $b\in\NN$, then $\phi(A)\phi(B)$ is an edge in $\TT_k$ labelled $c$, and  
\begin{equation}
\label{IsoIsoIso}
(G[A\cup B],A,B)\cong 
(J_{m,a,n,b,c},A_{m,a},B_{n,a}) \cong 
(U[ U_{\phi(A)} \cup U_{\phi(B)} ], U_{\phi(A)}, U_{\phi(B)} ). 
\end{equation}
Note that to conclude that $G[A\cup B]\cong J_{m,a,n,b,c}$ in the case that $AB\in E(Q')\setminus E(Q)$, we use the fact that $AB$ is labelled $c=1$, implying that there is no edge between $A_{m,a}$ and $B_{n,b}$ in $J_{m,a,n,b,1}$ (by the definition of $\HH$-system).

For each vertex $v$ of $G$, if $A$ is the part of $\PART$ containing $v$, then let $\psi(v)$ be the vertex that $v$ is mapped to in $U_{\phi(A)}$ under the isomorphism described in \cref{IsoIsoIso}. So $\psi(v)$ is in $U_G$. For every edge $vw\in E(G)$, if $v\in A\in\PART$ and $w\in B\in\PART$, then $A=B$ or $AB\in E(Q)$, implying $\psi(v)\psi(w)\in E(U_G)$ by the isomorphisms in \cref{IsoIsoIso}. So $G$ is isomorphic to a spanning subgraph of $U_G$. We claim that in fact $G$ is isomorphic to $U_G$. Consider an edge $xy$ of $U_G$. So $x\in U_{\phi(A)}$ and $y\in U_{\phi(B)}$ for some $A,B\in V(Q)$. 
Since $\phi(Q')$ is an induced subgraph of $\TT_k$, either $A=B$ or $AB\in E(Q')$. Thus $\psi^{-1}(x)\psi^{-1}(y)$ is an edge of $G$ by the isomorphisms in \cref{IsoIsoIso}. Hence $G$ is isomorphic to $U_G$, which proves the second claim. 

If every graph in $\HH$ is $b$-colourable, then $\chi(G/\PART)\leq\tw(G/\PART)+1\leq k+1$, and a product colouring gives a $b(k+1)$-colouring of $G$. This proves the third  claim. 

For the fourth claim, each graph in $\HH$ has an acyclic orientation with in-degree at most $d$ at each vertex. Say $G\in\JJ^{(k)}$. Let $\PART$ and $Q$ be as in the definition of $\JJ^{(k)}$. So $Q$ has a $k$-orientation. Each $A\in \PART$ has in-degree at most $k$ in $Q$, and the subgraph $G[A]$ (which is isomorphic to a graph in $\HH$) has an acyclic orientation with in-degree at most $d$ at each vertex. For each edge $AB\in E(Q)$, orient each $AB$-edge of $G$ from $A$ to $B$. By assumption, each vertex in $B$ has at most $r$ neighbours in $A$. In total, the in-degree of each vertex in $G$ is $kr+d$. Thus each finite subgraph of $G$ has minimum degree at most $kr+d$. 
\end{proof}

\begin{prop}
\label{StronglyUniversalTreewidth}
For each $k\in\NN$ there is a strongly universal treewidth-$k$ graph.
\end{prop}

\begin{proof}
Let $\HH:=\HH_m:=\{K_1\}$ for each $m\in\NN_0$. 
Let $J'$ be the graph with vertex set $\{v,w\}$ and edge-set $\{vw\}$. 
Let $J''$ be the graph with vertex set $\{v,w\}$ and edge-set $\emptyset$. For $(m,n)\in\XX$ and $a,b,c\in \NN$, let $A_{m,a}:=\{v\}$ and $B_{n,b}:=\{w\}$; 
let $J_{m,a,n,b,c}:=J'$ if $c$ is even, and let $J_{a,b,c}:=J''$ if $c$ is odd. 
So $\JJ=\{ (J_{m,a,n,b,c},A_{m,a},B_{n,b}) :(m,n)\in\XX,a,b,c\in\NN\}$ is an  $\HH$-system. For every graph $G$ in $\JJ^{(k)}$, if $\PART$ is the partition of $G$ in the definition of $\JJ^{(k)}$, then since each part is a singleton, $\tw(G)=\tw(G/\PART)\leq k$.

We now show that every graph $G$ of treewidth $k$ is in $\JJ^{(k)}$. 
By \cref{TreewidthCharacterisation}, $G$ is a spanning subgraph of a chordal graph $Q=\GGG{T}{c}$ with no $K_{k+2}$ subgraph, for some spanning tree $T$ of $Q$ rooted at $R$, and for some $(k+1)$-colouring $c$ of $T$. Let $\PART$ be the partition of $G$ and of $Q$ with one part for each vertex. So $\PART$ is a chordal partition of $Q$. Label every vertex of $G/\PART$ by 1. Fix a $k$-orientation of $Q$. 
Consider an oriented edge $AB$ of $Q$.
Let $m:=\dist_T(R,A)$ and $n:=\dist_T(R,B)$. 
If $AB$ is in $G/\PART$, then label $AB$ by 2, so 
$(G[A\cup B],A,B)\cong(J',\{v\},\{w\})=(J_{m,1,n,1,2},A_{m,1},B_{n,1})$. 
If $AB$ is not in $G/\PART$, then label $AB$ by 1, so 
$(G[A\cup B],A,B)\cong(J'',\{v\},\{w\})=(J_{m,1,n,1,1},A_{m,1},B_{n,1})$.
Since $J'$ has an edge and $J''$ does not, $\PART$ is a partition of $G$, where $T$ and $Q$ satisfy the definition of $\JJ^{(k)}$. Thus $G$ is in $\JJ^{(k)}$.

Hence $\JJ^{(k)}$ is exactly the set of all graphs with treewidth at most $k$. By \cref{JJJ}, $\JJ^{(k)}$ has a strongly universal graph $U_{\JJ,k}$, which therefore is a strongly universal treewidth-$k$ graph. 
\end{proof}

The following definition and lemmas are used in our proof for $K_t$-minor-free graphs below. If $S$ is a set of $k\geq 2$ vertices in a connected graph $G$ and $r\in S$, then a set $X\subseteq V(G)$ is an \hdefn{$S$}{connector} (\DefNoIndex{rooted} at $r$) if there are geodesic paths $P_1,\dots,P_{k-1}$ in $G$ starting at $r$, such that $S\subseteq X=\bigcup_{i=1}^{k-1} V(P_i)$, and $G[X]$ is a connected subgraph with bandwidth at most $k-1$, and every vertex in $G-X$ has at most $2k-2$ neighbours in $X$. An $S$-connector is called an \hdefn{$\ell$}{connector} for any integer $\ell\geq |S|$. 

\begin{lem}[\citep{vdHW18}]
\label{ConnectingSubgraph2}
For any set $S$ of at least two vertices in a connected graph $G$, and
for any vertex $r\in S$, there is an $S$-connector in $G$ rooted at $r$. 
\end{lem}

We now define a specific $\HH$-system to be used in \cref{MinorJJJ}. This system is designed so that \cref{PartitionGenColNum} can be used to show that graphs in 
$\JJ^{(t-2)}$ have bounded generalised colouring numbers. 

Fix $t\in\NN$ with $t\geq 3$. As illustrated in \cref{HnExample}, for $m\in\NN_0$, let $\HH_m$ be the class of all connected finite graphs $H$ with $V(H)\subseteq \NN_0^{m+1} \times [t-2]$, where each $v\in V(H)$ is denoted by $v=(v_0,v_1,\dots,v_m,v_\star) \in \NN_0^{m+1}\times[t-2]$, such that:
\begin{itemize}
\item for all $v,w\in V(H)$ if $v_m=w_m$ and $v_\star=w_\star$, then $v=w$; 
\item there is exactly one vertex $r\in V(H)$ with $r_m=0$; 
\item for every edge $vw\in E(H)$ and for every $i\in[0,m]$, we have $|v_i-w_i|\leq 1$;
\item for each $i\in[t-2]$, the set $\{r\}\cup \{v\in V(H): v_\star = i \}$ induces a path in $H$ (which is a geodesic by the previous rules); and
\item $H$ has bandwidth at most $t-2$. 
\end{itemize}

\begin{figure}[!ht]
\centering
\includegraphics{HnExample}
\caption{A graph in $\HH_0$ with $t=5$. Red edges must be present. Gray edges are possibly present.}
\label{HnExample}
\end{figure}

Note that $\HH_m$ is countable. 
Enumerate $\HH_m=\{H_{m,1},H_{m,2},\dots\}$. 
Let $\HH:=\bigcup_{m\in\NN_0}\HH_m$.  
For $(m,n)\in\XX$, let $\JJ_{m,n}$ be the set of all $(\HH,m,n)$-triples $(J,A,B)$ such that:
\begin{itemize}
    \item $J$ is a graph with $V(J)=A\cup B$ and $A\cap B =\emptyset$;
    \item $J[A]\in\HH_m$ and $J[B]\in\HH_n$;
    \item for all $v\in A$ and $w\in B$, if $vw\in E(J)$ then 
    $|v_i-w_i|\leq 1$ for each $i\in[0,m]$; and
    \item every vertex in $B$ has at most $2t-4$ neighbours in $A$. 
\end{itemize}
Note that $\JJ_{m,n}$ is countable. 
Let $\JJ:=\bigcup_{(m,n)\in\XX}\JJ_{m,n}$. 
Enumerate $\JJ=\{(J_{m,a,n,b,c},A_{m,a},B_{n,b}):(m,n)\in\XX,\,a,b,c\in\NN\}$ so that it is an $\HH$-system. 

The next lemma is inspired by an analogous result for finite graphs by \citet{vdHW18}. 

\begin{lem}
\label{MinorJJJ}
Every $K_t$-minor-free graph $G$ is in $\JJ^{(t-2)}$. 
\end{lem}

\begin{proof}
Since $\TT_k$ contains infinitely many disjoint copies of $\TT_k$ as an induced subgraph, it suffices to assume that $G$ is connected. We may also assume that $G$ is infinite. Let $u_1,u_2,\dots$ be an arbitrary enumeration of $V(G)$. Below we define $(\PART_i,T_i,S_i,c_i)_{i\in\NN}$, where for $i\in\NN$, 
\begin{enumerate}[(P1)] 
 \item $\PART_i$ is a connected chordal partition of $G$ (that is, $G/\PART_i$ is a chordal graph, and $G[A]$ is connected for each $A\in\PART_i$);
 \item $T_i$ is a spanning tree of $G/\PART_i$ and $c_i$ is a $(t-1)$-colouring of $T_i$ such that $G/\PART_i$ is a spanning subgraph of $\GGG{T_i}{c_i}$;
\item  $S_i$ is a subtree of $T_i$ with exactly $i$ nodes, and every node of $T_i-V(S_i)$ is a leaf of $T_i$;
\item $S_1\subseteq S_2\subseteq \dots\subseteq S_i $;
\item for $j\in[i]$ and for each part $A\in\PART_j$, we have $c_j(A)=c_i(A)$;
\item both $S_i$ and $T_i$ are rooted at part $R\in\PART_i$ where $R:=\{u_1\}$;
 \item for each part $B\in V(S_i)$, if $m:=\dist_{S_i}(R,B)$ and $R=X_0,X_1,\dots,X_m=B$ is the $RB$-path in $S_i$, and $B^+$ is the vertex-set of the component of $G-(X_0\cup\dots\cup X_{m-1})$ containing $B$, then 
\begin{itemize}
\item $B$ is a $(t-1)$-connector in $G[B^+]$ rooted at some vertex $r_B$; and 
\item $G[B]$ is isomorphic to some graph in $\HH_m$, where each vertex $v\in B$ is mapped to $(v_0,v_1,\dots,v_m,v_\star)$, where $v_j=\dist_{G[X^+_j]}(r_{X_j},v)$ for each $j\in[0,m]$; and 
\end{itemize}
\item for each directed edge $AB$ of $G/\PART_i$ where $A,B\in V(S_i)$, if $(m,n):=(\dist_{S_i}(R,A),\dist_{S_i}(R,B))\in\XX$, then the triple $(G[A\cup B], A,B)$ is isomorphic to some $(\HH,m,n)$-triple in $\JJ_{m,n}$,
\item for each directed edge $AB$ of $G/\PART_i$ where $A\in V(S_i)$ and $B\in V(T_i)\setminus V(S_i)$, each vertex in $B$ has at most $2t-4$ neighbours in $A$. 
\end{enumerate}

We first define $(\PART_1,T_1,S_1,c_1)$. Let $(B_j)_{j\in J}$ be the vertex-sets of the components of $G-u_1$. Let $\PART_1$ be the partition of $G$ with parts $R:=\{u_1\}$ and $(B_j)_{j\in J}$. So $G/\PART_1$ is a star with $|J|$ leaves. Let $T_1$ be this star. Colour $R$ by 1 and colour each node $B_j$ by 2. So $T_1$ is a spanning tree of $G/\PART_1$ and $c_1$ is a $(t-1)$-colouring of $T_1$ such that $G/\PART_1=T_1=\GGG{T_1}{c_1}$. Let $S_1$ be the tree with one vertex $R$. Consider $T_1$ and $S_1$ to be rooted at $R$. Then $(\PART_1,T_1,S_1,c_1)$ satisfies the above conditions. 

Note that (P1)--(P3) imply that the nodes of $S_i$ and $T_i$ are parts of $\PART_i$. So (P4) implies that for $j\in[i]$ each part of $\PART_j$ that is in $S_j$ is also a part of $\PART_i$ in $S_i$. Note that (P4) also implies that $B^+$ (defined in (P7)) does not depend on $i$.

Assume that $(\PART_1,T_1,S_1,c_1),\dots,(\PART_i,T_i,S_i,c_i)$ are defined. Since $|V(S_i)|=i$ and each part of $\PART_i$ in $S_i$ is finite, some vertex of $G$ is in a part of $\PART_i$ that is a node of $T_i-V(S_i)$. Let $j$ be the minimum integer such that $u_j$ is in a part $A\in \PART_i$ that is a node of $T_i-V(S_i)$. So $A$ is a leaf of $T_i$ by (P3). Let $B_1,\dots,B_s$ be the neighbours of $A$ in $G/\PART_i$. So $\{A,B_1,\dots,B_s\}$ is a clique in $G/\PART_i$, implying  $s\in[0,t-2]$. 
Let $B_s$ be the parent of $A$ in $T_i$. So $B_1,\dots,B_s$ are ancestors of $A$ in $T_i$. Let $m:=\dist_{T_i}(R,B_s)$. Let $R=X_0,X_1,\dots,X_m=B_s$ be the $RB_s$-path in $T_i$. For $\ell\in[0,m]$, let $X^+_\ell$ be the vertex-set of the component of $G-(X_0\cup\dots\cup X_{\ell-1})$ containing $X_\ell$. By (P7), $X_\ell$ is a $(t-1)$-connector in $G[X^+_\ell]$ rooted at some vertex $r_{X_\ell}$. Note that $A\subseteq X^+_\ell$. 

For $p\in[s]$, let $a_p$ be any vertex in $A$ adjacent to some vertex in $B_p$, which exists since $AB_p\in E(G/\PART_i)$. Let $C$ be a $\{u_j,a_1,\dots,a_s\}$-connector in $G[A]$ rooted at $u_j$, which exists by \cref{ConnectingSubgraph2}. Let $P_1,\dots,P_s$ be the corresponding geodesic paths. So $C=\bigcup_{p=1}^s V(P_p)$, and $C$ is a $(t-1)$-connector in $G[A]$. 

Let $\PART_{i+1}$ be the partition of $G$ obtained from $\PART_i$ by replacing $A$ by $C$ and the vertex sets of the components of $G[A]-C$. For $p\in[s]$, since $a_p\in C$, we have $B_iC$ is an edge of $G/\PART_{i+1}$. 
So $\{C,B_1,\dots,B_s\}$ is a clique in $G/\PART_{i+1}$. For each component $X$ of $G[A]-C$, the neighbourhood of $V(X)$ in $G/\PART_{i+1}$ is a subset of $\{C,B_1,\dots,B_s\}$. 
So $G/\PART_{i+1}$ is chordal. By construction, each part of $\PART_{i+1}$ is connected. Thus (P1) is satisfied for $i+1$. 

Let $T_{i+1}$ be the tree obtained from $T$ by renaming the node $A$ by $C$, and by adding, for each component $G[X]$ of $G[A]-C$, the node $X$ and the edge $XC$ to $T_{i+1}$. So $X$ is a leaf in $T_{i+1}$. Let $c_{i+1}$ be the colouring of $T_{i+1}$ obtained from $c_i$ as follows. 
Let $c_{i+1}(C):=c_i(A)$. 
So $C$ has the same in-neighbourhood in $\GGG{T_{i+1}}{c_{i+1}}$ as $A$ in $\GGG{T_i}{c_i}$.
For each component $G[X]$ of $G[A]-C$, the neighbourhood of $X$ in $G/\PART_{i+1}$ is a subset of the clique $\{C,B_1,\dots,B_s\}$ of size at most $t-2$ (since $G$ has no $K_t$ minor). 
Let $c_{i+1}(X)$ be a colour not used by its neighbours in $G/\PART_{i+1}$, which exists since there are $t-1$ colours. 
Since $B_pA$ is an edge of $G/\PART_i$ for each $p\in[s]$, 
if $B_pX$ is an edge of $G/\PART_{i+1}$, then $B_pX$ is an edge of $\GGG{T_{i+1}}{c_{i+1}}$. 
Thus $G/\PART_{i+1}$ is a spanning subgraph of $\GGG{T_{i+1}}{c_{i+1}}$, and (P2) holds for $i+1$. (P5) holds by the definition of $c_{i+1}$. 

Let $S_{i+1}$ be the tree obtained from $S_i$ by adding a new node $C$ adjacent to $B_s$. Thus (P3), (P4) and (P6) are satisfied for $i+1$. 

We now show that (P7) holds for $i+1$. By construction, $C$ is a $(t-1)$-connector in $G[A]$ and $A=C^+$ (where $C^+$ is the vertex-set of the component of $G-(X_0\cup\dots\cup X_m)$ containing $C$). Thus $C$ is a $(t-1)$-connector in $G[C^+]$. For each $v\in C$, let $v_{m+1}:=\dist_{G[A]}(u_j,v)$, and for each $\ell\in[0,m]$, let $v_\ell:=\dist_{G[X^+_\ell]}(r_\ell,v)$. Thus, $|v_\ell-w_\ell|\leq 1$ for each edge $vw$ in $G[C]$ and for each $\ell\in[m+1]$. For each $v\in C$, let $v_\star$ be an integer $p\in[s]$ such that $v_\star$ is in $P_p$. Since $P_p$ is a geodesic, if $v_{m+1}=w_{m+1}$ and $v_\star=w_\star$, then $v=w$. Thus $\{v\in C: v_\star = p \}$ induces a path in $H$, namely $P_p$. This shows that $G[C]$ is isomorphic to a graph in $\HH_{m+1}$, and (P7) holds for $i+1$.

We now show that (P8) holds for $i+1$. Consider $p\in S$. Let $\ell:=\dist_{T_i}(R,B_p)$. Thus $G[B_p]$ is isomorphic to some graph in $\HH_\ell$. By (P9), each vertex in $C$ has at most $2t-4$ neighbours in $B_p$. 
For $v\in B_p$ and $w\in C$ and $j\in[\ell]$ 
we have $v_j=\dist_{G[B^+_p]}(r_{B_p},v)$ (by (P7)) and
$w_j=\dist_{G[B^+_p]}(r_{B_p},w)$ (by the definition of $w_j$); 
so if $vw\in E(G)$ then $|v_j-w_j|\leq 1$. Hence $(G[B_p\cup C],B_p,C)$ is isomorphic to some $(\HH,\ell,m+1)$-triple. This shows that (P8) holds for $i+1$. 

Finally, by \cref{ConnectingSubgraph2}, every vertex in $G[A]-C$ has at most $2t-4$ neighbours in $A$, implying (P9) is satisfied for $i+1$. 

This shows that $(\PART_1,T_1,S_1,c_1),\dots,(\PART_{i+1},T_{i+1},S_{i+1},c_{i+1})$ satisfy (P1)--(P9), which completes the definition of $(\PART_1,T_1,S_1,c_1),(\PART_2,T_2,S_2,c_2),\dots$.

By (P1)--(P4), $S_1,S_2,\dots$ are trees, where each node of $S_i$ is a part of $\PART_i$, and $S_1\subseteq S_2\subseteq \dots$. For $i\in\NN$, let $\PART'_i$ be the subset of $\PART_i$ corresponding to parts in $S_i$. Thus a part of $\PART'_i$ is also a part of $\PART'_\ell$ for all $\ell\geq i$; that is, $\PART'_i\subseteq \PART'_\ell$. Let $\PART:= \bigcup_{i\in\NN}\PART'_i$. We claim that $\PART$ is a partition of $G$. By construction, each vertex of $G$ is in at most one part of $\PART$. Consider vertex $u_j$ of $G$. If $u_j$ is no part of $\PART'_j$, then each of the $j$ nodes of $S_j$ contains a vertex $u_\ell$ with $\ell<j$ (by the choice of connector roots), which is a contradiction. So $u_j$ is in some part of $\PART'_j$. Hence $\PART$ is a partition of $G$. Let $S:=\bigcup_{i\in\NN}S_i$. Then $S$ is a spanning tree of $G/\PART$ rooted at $R=\{u_1\}$. For each part $A\in\PART$, let $c(A):=c_i(A)$ for any $i\in\NN$ for which $A\in\PART'_i$. The choice of $i$ is irrelevant by (P5). Every edge of $G/\PART$ is an edge of $\GGG{S_i}{c_i}$ for some $i\in\NN$, so $G/\PART$ is a spanning subgraph of $\GGG{S}{c}$. For each part $A\in \PART$, if $m:=\dist_S(R,A)$,  then by (P7), $G[A]\cong H_{m,a}$ for some $a\in \NN$; label $A$ by $a$. 
For each oriented edge $AB\in E(G/\PART)$ with $(m,n):=(\dist_S(R,A),\dist_S(R,B))\in\XX$, by (P8) the triple $(G[A\cup B],A,B)$ is isomorphic to some triple $(J_{m,a,n,b,c},A_{m,a},B_{n,b}) \in\JJ_{m,n}$; label $AB$ by $c$. Therefore $G\in\JJ^{(t-2)}$ where $Q:=G/\PART$.
\end{proof}

\begin{thm}
\label{KtMinorFreeNoSubdiv}
For each integer $t\geq 3$, there is a graph $U$ such that:
\begin{itemize}
\item $U$ contains every $K_t$-minor-free graph as an induced subgraph, 
\item $U$ contains no subdivision of $K_{\aleph_0}$,
\item $U$ is $(t-1)^2$-colourable, 
\item $U$ is $(t-2)(2t-3)$-degenerate, 
\item $\col_r(U) \leq (t-2)(t-1)(2r+1)$ for every $r\in\NN$, 
\end{itemize}
\end{thm}

\begin{proof}
Let $\HH$ and $\JJ$ be as defined prior to \cref{MinorJJJ}. By \cref{JJJ}, there is a strongly-universal graph $U:=U_{\JJ,t-2}$ in $\JJ^{(t-2)}$ that contains no subdivision of $K_{\aleph_0}$. By \cref{MinorJJJ}, every $K_t$-minor-free graph $G$ is in $\JJ^{(t-2)}$. So $G$ is isomorphic to an induced subgraph of $U$. 

Since $\chi(H)\leq\bw(H)+1$ for every graph $H$, every graph in $\HH$ is $(t-1)$-colourable. Thus the third part of \cref{JJJ} is applicable with $b=t-1$, implying $U$ is $(t-1)^2$-colourable. By assumption, the fourth part of \cref{JJJ} is applicable with $d=t-2$ (since degeneracy is at most bandwidth) and $k=t-2$ and $r=2t-4$. Thus every finite subgraph of $U$ is $(2t-3)(t-2)$-degenerate.

We now show that $\col_r(U) \leq (t-2)(t-1)(2r+1)$. Let $\PART$ be the $(t-2)$-oriented chordal partition of $U$ from \cref{JJJ}. Let $H$ be a finite subgraph of $U$. Let $\YY$ be the set of parts in $\PART$ that intersect $V(H)$. Let $H^+$ be the subgraph of $U$ induced by the union of all the parts in $\YY$ that intersect $V(H)$. Since $H$ is finite and each part of $\YY$ is finite, $H^+$ is finite. Note that $H^+/\YY$ is an induced subgraph of $U /\PART$. So $\YY$ is a chordal partition of $H^+$, and the $(t-2)$-orientation of $U/\PART$ defines a $(t-2)$-orientation of $H^+/\YY$. Since $\YY$ is finite, there is an ordering $Y_1,\dots,Y_n$ of the parts in $\YY$, such that $i<j$ for every oriented edge $Y_iY_j\in E(H^+/\YY)$. ($Y_1,\dots,Y_n$ is called a `perfect elimination' ordering in the literature on chordal graphs.)\

Consider $Y_i\in\YY$. There are at most $t-2$ neighbouring parts $Y_j\in N_{H^+/\YY}(Y_i)$ with $j<i$ (and they form a clique in $H^+/\YY$, although we will not need this property). Let $H^+_i$ be the connected component of $H^+-(Y_1\cup\dots\cup Y_{i-1})$ that contains $Y_i$. Observe that $H^+_i$ is precisely the subgraph of $H^+$ induced by the union of all parts $Y_j\in\YY$ such that $i=j$ or $Y_j$ is a descendent of $Y_i$ in $H^+/\YY$. By the definition of $\HH$, $V(Y_i)$ is the union of the vertex-sets of $t-2$ geodesic paths $P_1,\dots,P_{t-2}$ in $H^+[Y_i]$. We claim that $P_1,\dots,P_{t-2}$ are geodesics in $H^+_i$. Let $x$ and $y$ be vertices in some $P_j$. Let $P$ be any $xy$-path in $H^+_i$. Say $P$ has length $\ell$. By the definition of $\HH_n$ and $\JJ_{m,n}$, for each edge $vw$ in $P$, we have $|v_i-w_i|\leq 1$. Thus $|x_i-y_i|\leq\ell$, implying $\dist_{P_j}(x,y)\leq\ell$. 
Thus $P_1,\dots,P_{t-2}$ are geodesics in $H^+_i$. 
Hence \cref{PartitionGenColNum} is applicable to $H^+$ with $d=p=t-2$. Thus $\col_r(H) \leq \col_r(H^+) \leq (t-2)(t-1)(2r+1)$. 
By \cref{colExtendable}, $\col_r$ is extendable.
Therefore $\col_r(U) \leq (t-2)(t-1)(2r+1)$. 
\end{proof}


%%%%%%%%%%%%%%%%%%%%%%%%%%%%%%%%%%%%
\subsection{Locally Finite Graphs}
\label{LocallyFiniteGraphs}

We have shown above that there exists a graph that contains every planar graph as an induced subgraph, but contains no subdivision of $K_{\aleph_0}$. Indeed, the result holds for $K_t$-minor-free graphs. We now show the same conclusion holds for all locally finite graphs.

\begin{prop}
\label{StronglyUniversalLocallyFinite}
There is a graph that contains every locally finite graph as an induced subgraph, but contains no subdivision of $K_{\aleph_0}$. 
\end{prop}

\begin{proof}
Let $\HH$ be the class of all finite graphs. Let $\JJ$ be the class of all $(\HH,m,n)$-triples, where $(m,n)\in\XX$. So $\HH$ and $\JJ$ are countable. Enumerate $\HH$ and $\JJ$ so that $\JJ$ is an $\HH$-system. By \cref{JJJ}, $\JJ^{(1)}$ has a strongly universal graph $U_{\JJ,1}$ that contains no subdivision of $K_{\aleph_0}$. We now prove that every locally finite graph $G$ is in $\JJ^{1}$. 

Let $G_1,G_2,\dots$ be the connected components of $G$. For $i\in\NN$, let $r_i$ be any vertex of $G_i$. For $j\in\NN_0$, let $V_{i,j}$ be the set of vertices in $G_i$ at distance $j$ from $r_i$. Note that $V_{i,0}=\{r_i\}$. Fix $i\in\NN$. We prove by induction on $j\in\NN_0$ that $V_{i,j}$ is finite. The base case holds since $|V_{i,0}|=1$. Assume that $V_{i,j}$ is finite. Let $\Delta_{i,j}$ be the maximum degree of vertices in $V_{i,j}$. Thus $|V_{i,j+1}|\leq \Delta_{i,j}\,|V_{i,j}|$, and $V_{i,j+1}$ is finite. 

So $\PART=\{V_{i,j}:i\in\NN,j\in\NN_0\}$ is a partition of $G$. The quotient $G/\PART$ is the disjoint union of infinite 1-way paths (one for each component of $G$). Let $Q$ be the graph obtained from $G/\PART$ by adding an edge between $V_{i,0}$ and $V_{i+1,0}$ for each $i\in\NN$. So $Q$ is a tree, which has a rooted 1-orientation obtained by directing the edges from $V_{i,j}$ to $V_{i,j+1}$, and from $V_{i,0}$ to $V_{i+1,0}$. 

For $i\in\NN$ and $j\in\NN_0$, since $V_{i,j}$ is finite, $G[V_{i,j}]\cong H_a$ for some $a\in\NN$. Label $V_{i,j}$ by $a$. If $V_{i,j}$ is labelled $a$ and $V_{i,j+1}$ is labelled $b$, then there exists $c\in\NN$ such that $(G[V_{i,j}\cup V_{i,j+1}],V_{i,j},V_{i,j+1})\cong(J_{a,b,c},H_a,H_b)\in \JJ$; label the edge $V_{i,j}V_{i,j+1}$ of $Q$ by $c$. For $i\in\NN$, if $V_{i,0}$ is labelled $a$ and $V_{i,j+1}$ is labelled $b$, then label the edge $V_{i,0}V_{i+1,0}$ of $Q$ by 1. Thus $(G[V_{i,0}\cup V_{i+1,0}],V_{i,0},V_{i+1,0})\cong(J_{a,b,1},H_a,H_b)\in \JJ$ since there are no edges between $A_{a,b,1}$ and $B_{a,b,1}$ in $J_{a,b,1}$. 

Together, this shows that $G\in\JJ^{(1)}$, and therefore $G$ is isomorphic to an induced subgraph of $U_{\JJ,1}$. 
\end{proof}

%%%%%%%%%%%%%%%%%%%%%%%%%%%%%%%%%%%%%%%%%%%%%
\subsection{Forcing Infinite Edge-Connectivity}
\label{InfiniteEdgeConnectivity}


As shown above, there is a graph $U$ that contains all planar graphs, but does not contain a subdivision of an infinite clique.  In particular, $U$ does not contain a subgraph of infinite vertex-connectivity. We now show that the analogous statement for edge-connectivity is false.  That is, infinite edge-connectivity is unavoidable in any graph that contains all planar graphs. 

\begin{prop}
\label{Universal edge-connectivity}
If a graph $U$ contains all planar graphs, then $U$ contains a subgraph of infinite edge-connectivity that contains all planar graphs.
\end{prop}

\begin{proof}
We first observe that every finite or countably infinite planar graph $G$ is an induced subgraph of a planar graph $I(G)$ of infinite edge-connectivity. To see this, first extend $G$ to a connected planar graph in which $G$ is an induced subgraph. For each edge $xy$ of $G$, add a new vertex of degree 2 adjacent to $x$ and $y$. Call the resulting graph $G_1$. For each edge $xy$ of $G_1$, add a new vertex of degree 2 adjacent to $x$ and $y$. Call the resulting graph $G_2$. Continue like this defining $G_3,G_4, \ldots$ and put $I(G) := G \cup G_1 \cup G_2 \cup \ldots$. Then $I(G)$ contains $G$ as an induced subgraph and has infinite edge-connectivity. 

Now assume $U$ contains all planar graphs. Let $U'$ be the union of all planar subgraphs of $U$ with infinite edge-connectivity. Then $U'$ has no finite edge-cut. Hence each component of $U'$ has infinite edge-connectivity. By definition, $U'$ contains all planar graphs of infinite edge-connectivity (and hence all planar graphs by the initial observation). Let $U_1',U_2', \ldots$ be the components of $U'$. It remains to be proved that one of these contains all planar graphs. Suppose for the sake of contradiction that $U_i'$ does not contain the planar graph $M_i$ for each $i\in\NN$. Let $M$ be the disjoint union of $M_1, M_2,  \ldots$. Then $I(M)$ is not a subgraph of any $U_i'$ and hence not a subgraph of $U'$, a contradiction which proves the theorem.
\end{proof}

\cref{Universal edge-connectivity} holds more generally for any minor-closed class of graphs containing the series-parallel graphs (that is, the graphs containing no $K_4$ minor). It also holds for the class of outerplanar graphs if, in the proof, we only add vertices of degree 2 adjacent to the end-vertices of edges on the outer-cycle.




%%%%%%%%%%%%%%%%%%%%%%%%%%%%%%%%%%%%%%%%%%%%%%%%%%%
\section{Final Remarks}

We conclude with several remarks and open problems.

\paragraph{The Primary Question:}
What is the simplest graph that contains all planar graphs? We have shown that $\SS_6\boxtimes \PP$ and $\SS_3\boxtimes \PP\boxtimes K_3$ contain all planar graphs. It is a tantalising open problem whether $\SS_3\boxtimes\PP$  contains all planar graphs. Note that for each $\ell\in\NN$, \citet{DJMMUW20} constructed a planar graph (with simple treewidth 3)  that is not a subgraph of $\TT_2 \boxtimes \PP \boxtimes K_{\ell}$. More generally, for all $k,\ell\in\NN$, \citet{DJMMUW20} constructed a graph with simple treewidth $k$  that is not a subgraph of $\TT_{k-1} \boxtimes \PP \boxtimes K_{\ell}$. 

\paragraph{Minimality:} 
Does there exist a graph $U_0$ that contains every planar graph and is a subgraph of every graph that contains every planar graph? Note that $U_0$ must be a common subgraph of $K_{\aleph_0,\aleph_0,\aleph_0,\aleph_0}$, 
$\SS_6 \boxtimes \PP$, $\SS_3 \boxtimes \PP \boxtimes K_3$, and the graph in \cref{KtMinorFreeNoSubdiv} (with $t=5$); it must also contain $K_{\aleph_0}$ as a minor. 

\paragraph{Acyclic Colourings:} 
One way to measure the `quality' of a  graph that contains every planar graph is via the acyclic chromatic number. The result of \citet{KY03} mentioned in \cref{GeneralisedColouringNumbers} along with \cref{GenColourProduct} implies 
 \begin{equation}
 \label{AcyclicColouringProduct}
 \chi_\text{a}( \TT_k \boxtimes P ) 
 \leq 
 \col_2( \TT_k \boxtimes P ) 
 \leq 
 5(k+1).
 \end{equation}
With \cref{InfinitePlanarStructure6} this implies that $\SS_6 \boxtimes \PP$ contains every planar graph and is acyclically $35$-colourable.  What is the minimum $c$ such that there exists an acyclically $c$-colourable graph that contains every planar graph? \citet{Borodin-DM79} proved that every finite planar graph has an acyclic $5$-colouring. An easy application of \cref{Konig} shows that acyclic chromatic number is extendable. Thus every planar graph has an acyclic $5$-colouring. More generally, \cref{InfiniteSurfaceProduct} and \eqref{AcyclicColouringProduct} imply that $(\SS_6+K_{2g})\boxtimes \PP$ contains every graph of Euler genus $g$ and has an acyclic $(10g+35)$-colouring. \citet{AMS96} proved that graphs with Euler genus $g$ are acyclically $O(g^{4/7})$-colourable (and this bound is tight up to a polylogarithmic factor). Is there a graph that contains every graph of Euler genus $g$ and has acyclic chromatic number $o(g)$? The analogous questions for game chromatic number or oriented chromatic number are also of interest.

\paragraph{List Colouring:} Given $d\in\NN$, is there a universal graph for the class of $d$-list-colourable graphs? For $d=5$, any such graph must contain every planar graph.

\paragraph{Bounded Degree Planar Graphs:} If a graph $U$ contains all planar graphs with maximum degree 3, must $U$ have infinite maximum degree? Or is there a graph with bounded maximum degree that contains all planar graphs with maximum degree 3? More generally, is there is a graph with bounded maximum degree that contains all graphs with maximum degree 3 (regardless of planarity)? Of course, the above questions are of interest with 3 replaced by any constant. 

\paragraph{Minor-Closed Classes:} 
\citet{DHV85} first addressed the question of which minor-closed classes have a universal element. \cref{InfinteCliqueMinor}(a) implies that every proper minor-closed class that contains every planar graph has no universal element. It is open whether every proper minor-closed class excluding some finite planar graph has a universal element. The following result is in this direction:

\begin{thm}
The following are equivalent for a minor-closed class $\GG$:
\begin{enumerate}[label=(\alph*)]
\item $\GG$ has finite treewidth,
\item some graph with finite treewidth contains every graph in $\GG$,
\item some finite planar graph is not in $\GG$.
\end{enumerate}
\end{thm}
\begin{proof}
First, (a) implies (b) since for every $k\in\NN$ there is a universal graph for the class of treewidth $k$ graphs (\cref{TreewidthUniversal}). 

To see that (b) implies (c), assume that some graph with treewidth $n\in\NN$ contains every graph in $\GG$. Since treewidth is subgraph-closed, every graph in $\GG$ has treewidth at most $n$. Thus, the $(n+1)\times (n+1)$ grid graph, which has treewidth $n+1$, is not in $\GG$. That is, some finite planar graph is not in $\GG$, as desired. 

Finally, (c) implies (a) by the Grid-Minor Theorem of \citet{RS-V} and since treewidth is extendable (\cref{TreewidthExtendable}). 
\end{proof}

\subsection*{Acknowledgements} Thanks to Kevin Hendrey, Daniel Mathews, and Alex Scott for helpful comments. 

%%%%%%%%%%%%%%%%%%%%%%%%
%%%  Squashing the bibliography 
  \let\oldthebibliography=\thebibliography
  \let\endoldthebibliography=\endthebibliography
  \renewenvironment{thebibliography}[1]{%
    \begin{oldthebibliography}{#1}%
      \setlength{\parskip}{0.2ex}%
      \setlength{\itemsep}{0.2ex}%
  }{\end{oldthebibliography}}

{
\fontsize{10pt}{11pt}
\selectfont
%\bibliographystyle{DavidNatbibStyle}
%\bibliography{myBibliography}
\def\soft#1{\leavevmode\setbox0=\hbox{h}\dimen7=\ht0\advance \dimen7
  by-1ex\relax\if t#1\relax\rlap{\raise.6\dimen7
  \hbox{\kern.3ex\char'47}}#1\relax\else\if T#1\relax
  \rlap{\raise.5\dimen7\hbox{\kern1.3ex\char'47}}#1\relax \else\if
  d#1\relax\rlap{\raise.5\dimen7\hbox{\kern.9ex \char'47}}#1\relax\else\if
  D#1\relax\rlap{\raise.5\dimen7 \hbox{\kern1.4ex\char'47}}#1\relax\else\if
  l#1\relax \rlap{\raise.5\dimen7\hbox{\kern.4ex\char'47}}#1\relax \else\if
  L#1\relax\rlap{\raise.5\dimen7\hbox{\kern.7ex
  \char'47}}#1\relax\else\message{accent \string\soft \space #1 not
  defined!}#1\relax\fi\fi\fi\fi\fi\fi}
\begin{thebibliography}{171}
\providecommand{\natexlab}[1]{#1}
\providecommand{\msn}[1]{MR:\,\href{http://www.ams.org/mathscinet-getitem?mr=MR{#1}}{#1}}
\providecommand{\ZBL}[1]{Zbl:\,\href{https://www.zentralblatt-math.org/zmath/en/search/?q=an:#1}{#1}}
\providecommand{\url}[1]{\texttt{#1}}
\providecommand{\urlprefix}{}
\expandafter\ifx\csname urlstyle\endcsname\relax
  \providecommand{\doi}[1]{doi:\discretionary{}{}{}#1}\else
  \providecommand{\doi}{doi:\discretionary{}{}{}\begingroup
  \urlstyle{rm}\Url}\fi

\bibitem[{Abrahamsen et~al.(2017)Abrahamsen, Alstrup, Holm, Knudsen, and
  St\"{o}ckel}]{AAHKBS17}
\textsc{Mikkel Abrahamsen, Stephen Alstrup, Jacob Holm, Mathias B\ae k~Tejs
  Knudsen, and Morten St\"{o}ckel}.
\newblock \href{https://doi.org/10.4230/LIPIcs.ICALP.2017.128}{Near-optimal
  induced universal graphs for bounded degree graphs}.
\newblock In \emph{44th {I}nternational {C}olloquium on {A}utomata,
  {L}anguages, and {P}rogramming (ICALP)}, vol.~80 of \emph{LIPIcs. Leibniz
  Int. Proc. Inform.}, p. Art. 128. Schloss Dagstuhl. Leibniz-Zent. Inform.,
  2017.

\bibitem[{Abrahamsen et~al.(2020)Abrahamsen, Alstrup, Holm, Knudsen, and
  St\"{o}ckel}]{AAHKS20}
\textsc{Mikkel Abrahamsen, Stephen Alstrup, Jacob Holm, Mathias B\ae k~Tejs
  Knudsen, and Morten St\"{o}ckel}.
\newblock \href{https://doi.org/10.1016/j.dam.2019.10.030}{Near-optimal induced
  universal graphs for cycles and paths}.
\newblock \emph{Discrete Appl. Math.}, 282:1--13, 2020.

\bibitem[{Ackermann(1937)}]{Ackermann37}
\textsc{Wilhelm Ackermann}.
\newblock \href{https://doi.org/10.1007/BF01594179}{Die {W}iderspruchsfreiheit
  der allgemeinen {M}engenlehre}.
\newblock \emph{Math. Ann.}, 114(1):305--315, 1937.

\bibitem[{Alon(2017)}]{Alon17}
\textsc{Noga Alon}.
\newblock \href{https://doi.org/10.1007/s00039-017-0396-9}{Asymptotically
  optimal induced universal graphs}.
\newblock \emph{Geom. Funct. Anal.}, 27(1):1--32, 2017.

\bibitem[{Alon and Asodi(2002)}]{AA02}
\textsc{Noga Alon and Vera Asodi}.
\newblock \href{https://doi.org/10.1016/S0377-0427(01)00455-1}{Sparse universal
  graphs}.
\newblock \emph{J. Comput. Appl. Math.}, 142(1):1--11, 2002.

\bibitem[{Alon and Capalbo(2007)}]{AC07}
\textsc{Noga Alon and Michael Capalbo}.
\newblock \href{https://doi.org/10.1002/rsa.20143}{Sparse universal graphs for
  bounded-degree graphs}.
\newblock \emph{Random Structures Algorithms}, 31(2):123--133, 2007.

\bibitem[{Alon et~al.(1996)Alon, Mohar, and Sanders}]{AMS96}
\textsc{Noga Alon, Bojan Mohar, and Daniel~P. Sanders}.
\newblock \href{https://doi.org/10.1007/BF02762708}{On acyclic colorings of
  graphs on surfaces}.
\newblock \emph{Israel J. Math.}, 94:273--283, 1996.

\bibitem[{Alon and Nenadov(2019)}]{AN19}
\textsc{Noga Alon and Rajko Nenadov}.
\newblock \href{https://doi.org/10.1017/S0305004117000706}{Optimal induced
  universal graphs for bounded-degree graphs}.
\newblock \emph{Math. Proc. Cambridge Philos. Soc.}, 166(1):61--74, 2019.

\bibitem[{Alon et~al.(1990)Alon, Seymour, and Thomas}]{AST90}
\textsc{Noga Alon, Paul Seymour, and Robin Thomas}.
\newblock \href{https://doi.org/10.2307/1990903}{A separator theorem for
  nonplanar graphs}.
\newblock \emph{J. Amer. Math. Soc.}, 3(4):801--808, 1990.

\bibitem[{Alstrup et~al.(2017)Alstrup, Dahlgaard, and Knudsen}]{ADK17}
\textsc{Stephen Alstrup, S{\o}ren Dahlgaard, and Mathias B\ae k~Tejs Knudsen}.
\newblock \href{https://doi.org/10.1145/3088513}{Optimal induced universal
  graphs and adjacency labeling for trees}.
\newblock \emph{J. ACM}, 64(4):Art. 27, 22, 2017.

\bibitem[{Alstrup et~al.(2019)Alstrup, Kaplan, Thorup, and Zwick}]{AKTZ19}
\textsc{Stephen Alstrup, Haim Kaplan, Mikkel Thorup, and Uri Zwick}.
\newblock \href{https://doi.org/10.1137/16M1105967}{Adjacency labeling schemes
  and induced-universal graphs}.
\newblock \emph{SIAM J. Discrete Math.}, 33(1):116--137, 2019.

\bibitem[{Andreae(1986)}]{Andreae86}
\textsc{Thomas Andreae}.
\newblock \href{https://doi.org/10.1016/0095-8956(86)90026-2}{On a pursuit game
  played on graphs for which a minor is excluded}.
\newblock \emph{J. Comb. Theory, Ser. {B}}, 41(1):37--47, 1986.

\bibitem[{Appel and Haken(1989)}]{AH89}
\textsc{Kenneth Appel and Wolfgang Haken}.
\newblock \href{https://doi.org/10.1090/conm/098}{Every planar map is four
  colorable}, vol.~98 of \emph{Contemporary Math.}
\newblock Amer. Math. Soc., 1989.

\bibitem[{Arnborg and Proskurowski(1986)}]{AP-SJADM96}
\textsc{Stefan Arnborg and Andrzej Proskurowski}.
\newblock \href{https://doi.org/10.1137/0607033}{Characterization and
  recognition of partial $3$-trees}.
\newblock \emph{SIAM J. Algebraic Discrete Methods}, 7(2):305--314, 1986.

\bibitem[{Baadsgaard et~al.(2015)Baadsgaard, Jannik Bjerrum-Bohr, Bourjaily,
  and Damgaard}]{BBBD15}
\textsc{Christian Baadsgaard, Niels~Emil Jannik Bjerrum-Bohr, Jacob~L.
  Bourjaily, and Poul~H. Damgaard}.
\newblock \href{https://doi.org/10.1007/JHEP09(2015)136}{Scattering equations
  and {F}eynman diagrams}.
\newblock \emph{J. High Energy Phys.}, p. Article 136, 2015.

\bibitem[{Babai et~al.(1982)Babai, Chung, Erd\H{o}s, Graham, and
  Spencer}]{BCEGS82}
\textsc{L{\'a}szl{\'o} Babai, Fan R.~K. Chung, Paul Erd\H{o}s, Ron~L. Graham,
  and Joel~H. Spencer}.
\newblock On graphs which contain all sparse graphs.
\newblock In \emph{Theory and practice of combinatorics}, vol.~60 of
  \emph{North-Holland Math. Stud.}, pp. 21--26. 1982.

\bibitem[{Bass(1993)}]{Bass93}
\textsc{Hyman Bass}.
\newblock \href{https://doi.org/10.1016/0022-4049(93)90085-8}{Covering theory
  for graphs of groups}.
\newblock \emph{J. Pure Appl. Algebra}, 89(1-2):3--47, 1993.

\bibitem[{Bethe(1935)}]{Bethe35}
\textsc{Hans~A. Bethe}.
\newblock \href{https://doi.org/10.1098/rspa.1935.0122}{Statistical theory of
  superlattices}.
\newblock \emph{Proc. Roy. Soc. Lond. A.}, 150:552--–575, 1935.

\bibitem[{Bhatt et~al.(1989)Bhatt, Chung, Leighton, and Rosenberg}]{BCLR89}
\textsc{Sandeep~N. Bhatt, Fan R.~K. Chung, Frank~Thomson Leighton, and
  Arnold~L. Rosenberg}.
\newblock \href{https://doi.org/10.1137/0402014}{Universal graphs for
  bounded-degree trees and planar graphs}.
\newblock \emph{{SIAM} J. Discrete Math.}, 2(2):145--155, 1989.

\bibitem[{Bodini(2002)}]{Bodini02}
\textsc{Olivier Bodini}.
\newblock \href{https://doi.org/10.1007/3-540-36379-3_3}{On the minimum size of
  a contraction-universal tree}.
\newblock In \emph{Proc. Graph-theoretic Concepts in Computer Science}, vol.
  2573 of \emph{Lecture Notes in Comput. Sci.}, pp. 25--34. Springer, 2002.

\bibitem[{Bollob\'{a}s and Thomason(1981)}]{BT81}
\textsc{B\'{e}la Bollob\'{a}s and Andrew Thomason}.
\newblock \href{https://doi.org/10.1016/S0195-6698(81)80015-7}{Graphs which
  contain all small graphs}.
\newblock \emph{European J. Combin.}, 2(1):13--15, 1981.

\bibitem[{Bonamy et~al.(2020)Bonamy, Gavoille, and Pilipczuk}]{BGP20}
\textsc{Marthe Bonamy, Cyril Gavoille, and Michał Pilipczuk}.
\newblock \href{https://doi.org/10.1137/1.9781611975994.27}{Shorter labeling
  schemes for planar graphs}.
\newblock In \textsc{Shuchi Chawla}, ed., \emph{Proc. ACM-SIAM Symp. on
  Discrete Algorithms (SODA '20)}, pp. 446--462. 2020.
\newblock arXiv:1908.03341.

\bibitem[{Bonato(2008)}]{Bonato08}
\textsc{Anthony Bonato}.
\newblock \href{https://doi.org/10.1090/gsm/089}{A course on the web graph},
  vol.~89 of \emph{Graduate Studies in Mathematics}.
\newblock Amer. Math. Soc., 2008.

\bibitem[{Borodin(1979)}]{Borodin-DM79}
\textsc{Oleg~V. Borodin}.
\newblock \href{https://doi.org/10.1016/0012-365X(79)90077-3}{On acyclic
  colorings of planar graphs}.
\newblock \emph{Discrete Math.}, 25(3):211--236, 1979.

\bibitem[{Bose et~al.(2020)Bose, Dujmovi\'c, Javarsineh, and Morin}]{BDJM}
\textsc{Prosenjit Bose, Vida Dujmovi\'c, Mehrnoosh Javarsineh, and Pat Morin}.
\newblock \href{http://arxiv.org/abs/2007.06455}{Asymptotically optimal vertex
  ranking of planar graphs}.
\newblock arXiv:2007.06455, 2020.

\bibitem[{Brochet and Diestel(1994)}]{BD94}
\textsc{J.-M. Brochet and Reinhard Diestel}.
\newblock \href{https://doi.org/10.2307/2155004}{Normal tree orders for
  infinite graphs}.
\newblock \emph{Trans. Amer. Math. Soc.}, 345(2):871--895, 1994.

\bibitem[{Broere and Heidema(2013)}]{BH13}
\textsc{Izak Broere and Johannes Heidema}.
\newblock \href{https://doi.org/10.7151/dmgt.1671}{Universality for and in
  induced-hereditary graph properties}.
\newblock \emph{Discuss. Math. Graph Theory}, 33(1):33--47, 2013.

\bibitem[{Broere et~al.(2013{\natexlab{a}})Broere, Heidema, and
  Mih\'{o}k}]{BHM13a}
\textsc{Izak Broere, Johannes Heidema, and Peter Mih\'{o}k}.
\newblock \href{https://doi.org/10.2478/s12175-012-0092-z}{Constructing
  universal graphs for induced-hereditary graph properties}.
\newblock \emph{Math. Slovaca}, 63(2):191--200, 2013{\natexlab{a}}.

\bibitem[{Broere et~al.(2013{\natexlab{b}})Broere, Heidema, and
  Mih\'{o}k}]{BHM13}
\textsc{Izak Broere, Johannes Heidema, and Peter Mih\'{o}k}.
\newblock \href{https://doi.org/10.7151/dmgt.1696}{Universality in graph
  properties with degree restrictions}.
\newblock \emph{Discuss. Math. Graph Theory}, 33(3):477--492,
  2013{\natexlab{b}}.

\bibitem[{Broere and Vetr\'{\i}k(2014)}]{BV14}
\textsc{Izak Broere and Tom\'{a}\v{s} Vetr\'{\i}k}.
\newblock Universal graphs for two graph properties.
\newblock \emph{Ars Combin.}, 116:257--262, 2014.

\bibitem[{Buci\'{c} et~al.(2021)Buci\'{c}, Dragani\'{c}, and Sudakov}]{BDS19}
\textsc{Matija Buci\'{c}, Nemanja Dragani\'{c}, and Benny Sudakov}.
\newblock \href{https://doi.org/10.1017/S0963548321000110}{Universal and
  unavoidable graphs}.
\newblock 2021, arXiv:1912.04889.

\bibitem[{Butler(2009)}]{Butler09}
\textsc{Steve Butler}.
\newblock \href{https://doi.org/10.1007/s00373-009-0860-x}{Induced-universal
  graphs for graphs with bounded maximum degree}.
\newblock \emph{Graphs Combin.}, 25(4):461--468, 2009.

\bibitem[{Cameron(1984)}]{Cameron84}
\textsc{Peter~J. Cameron}.
\newblock Aspects of the random graph.
\newblock In \textsc{B\'ela Bollob\'as}, ed., \emph{Graph Theory and
  Combinatorics}, pp. 65--79. Academic Press, 1984.

\bibitem[{Cameron(1997)}]{Cameron97}
\textsc{Peter~J. Cameron}.
\newblock \href{https://doi.org/10.1007/978-3-642-60406-5_32}{The random
  graph}.
\newblock In \emph{The mathematics of {P}aul {E}rd\H{o}s, {II}}, vol.~14 of
  \emph{Algorithms Combin.}, pp. 333--351. Springer, 1997.

\bibitem[{Cameron(2001)}]{Cameron01}
\textsc{Peter~J. Cameron}.
\newblock \href{https://doi.org/10.1007/978-3-0348-8268-2_15}{The random graph
  revisited}.
\newblock In \textsc{C.~Casacuberta, R.~M. Miró-Roig, J.~Verdera, and
  S.~Xambó-Descamps}, eds., \emph{European {C}ongress of {M}athematics}, vol.
  201 of \emph{Progr. Math.}, pp. 267--274. Birkh\"{a}user, 2001.

\bibitem[{Champanerkar et~al.(2019)Champanerkar, Kofman, and Purcell}]{CKP19}
\textsc{Abhijit Champanerkar, Ilya Kofman, and Jessica~S. Purcell}.
\newblock \href{https://doi.org/10.1112/jlms.12195}{Geometry of biperiodic
  alternating links}.
\newblock \emph{J. Lond. Math. Soc. (2)}, 99(3):807--830, 2019.

\bibitem[{Chen and Schelp(1993)}]{CS93}
\textsc{Guantao Chen and Richard~H. Schelp}.
\newblock \href{https://doi.org/10.1006/jctb.1993.1012}{Graphs with linearly
  bounded {R}amsey numbers}.
\newblock \emph{J. Combin. Theory Ser. B}, 57(1):138--149, 1993.

\bibitem[{Chen(2016)}]{Chen16}
\textsc{Hao Chen}.
\newblock \href{https://doi.org/10.1007/s00454-016-9777-3}{Apollonian ball
  packings and stacked polytopes}.
\newblock \emph{Discrete Comput. Geom.}, 55(4):801--826, 2016.

\bibitem[{Chen et~al.(2002)Chen, Grigni, and Papadimitriou}]{CGP02}
\textsc{Zhi-Zhong Chen, Michelangelo Grigni, and Christos~H. Papadimitriou}.
\newblock \href{https://doi.org/10.1145/506147.506148}{Map graphs}.
\newblock \emph{J. ACM}, 49(2):127--138, 2002.

\bibitem[{Cheng and Gao(2012)}]{CG12}
\textsc{Zhiyun Cheng and Hongzhu Gao}.
\newblock \href{https://doi.org/10.1016/S0252-9602(12)60047-2}{Some
  applications of planar graph in knot theory}.
\newblock \emph{Acta Math. Sci. Ser. B (Engl. Ed.)}, 32(2):663--671, 2012.

\bibitem[{Cherlin and Shelah(2007)}]{CS07a}
\textsc{Gregory Cherlin and Saharon Shelah}.
\newblock \href{https://doi.org/10.1016/j.jctb.2006.05.008}{Universal graphs
  with a forbidden subtree}.
\newblock \emph{J. Combin. Theory Ser. B}, 97(3):293--333, 2007.

\bibitem[{Cherlin and Shelah(2016)}]{CS16}
\textsc{Gregory Cherlin and Saharon Shelah}.
\newblock \href{https://doi.org/10.1007/s00493-014-3181-5}{Universal graphs
  with a forbidden subgraph: block path solidity}.
\newblock \emph{Combinatorica}, 36(3):249--264, 2016.

\bibitem[{Cherlin et~al.(1999)Cherlin, Shelah, and Shi}]{CSS99}
\textsc{Gregory Cherlin, Saharon Shelah, and Niandong Shi}.
\newblock \href{https://doi.org/10.1006/aama.1998.0641}{Universal graphs with
  forbidden subgraphs and algebraic closure}.
\newblock \emph{Adv. in Appl. Math.}, 22(4):454--491, 1999.

\bibitem[{Cherlin and Shi(2001)}]{CS01}
\textsc{Gregory Cherlin and Niandong Shi}.
\newblock \href{https://doi.org/10.2307/2695110}{Forbidden subgraphs and
  forbidden substructures}.
\newblock \emph{J. Symbolic Logic}, 66(3):1342--1352, 2001.

\bibitem[{Cherlin and Tallgren(2007)}]{CT07}
\textsc{Gregory Cherlin and Lasse Tallgren}.
\newblock \href{https://doi.org/10.1002/jgt.20245}{Universal graphs with a
  forbidden near-path or 2-bouquet}.
\newblock \emph{J. Graph Theory}, 56(1):41--63, 2007.

\bibitem[{Chung and Erd\H{o}s(1983)}]{CE83}
\textsc{Fan R.~K. Chung and Paul Erd\H{o}s}.
\newblock \href{https://doi.org/10.1007/BF02579290}{On unavoidable graphs}.
\newblock \emph{Combinatorica}, 3(2):167--176, 1983.

\bibitem[{Chung and Graham(1978)}]{CG78}
\textsc{Fan R.~K. Chung and Ron~L. Graham}.
\newblock \href{https://doi.org/10.1016/0095-8956(78)90072-2}{On graphs which
  contain all small trees}.
\newblock \emph{J. Combinatorial Theory Ser. B}, 24(1):14--23, 1978.

\bibitem[{Chung et~al.(1978)Chung, Graham, and Pippenger}]{CGP78}
\textsc{Fan R.~K. Chung, Ron~L. Graham, and Nicholas Pippenger}.
\newblock On graphs which contain all small trees. {II}.
\newblock In \emph{{P}roc. {F}ifth {H}ungarian {C}olloq. on Combinatorics
  ({V}ol. {I}}, vol.~18 of \emph{Colloq. Math. Soc. J\'{a}nos Bolyai}, pp.
  213--223. 1978.

\bibitem[{de~Bruijn and Erd\"{o}s(1951)}]{dBE51}
\textsc{Nicolaas~G. de~Bruijn and Paul Erd\"{o}s}.
\newblock A colour problem for infinite graphs and a problem in the theory of
  relations.
\newblock \emph{Nederl. Akad. Wetensch. Proc. Ser. A. 54 = Indagationes Math.},
  13:369--373, 1951.

\bibitem[{de~la Harpe(2000)}]{delaHarpe00}
\textsc{Pierre de~la Harpe}.
\newblock Topics in geometric group theory.
\newblock University of Chicago Press, 2000.

\bibitem[{DeVos et~al.(2004)DeVos, Ding, Oporowski, Sanders, Reed, Seymour, and
  Vertigan}]{DDOSRSV04}
\textsc{Matt DeVos, Guoli Ding, Bogdan Oporowski, Daniel~P. Sanders, Bruce
  Reed, Paul Seymour, and Dirk Vertigan}.
\newblock \href{https://doi.org/10.1016/j.jctb.2003.09.001}{Excluding any graph
  as a minor allows a low tree-width 2-coloring}.
\newblock \emph{J. Combin. Theory Ser. B}, 91(1):25--41, 2004.

\bibitem[{Diestel(1985)}]{Diestel85}
\textsc{Reinhard Diestel}.
\newblock \href{https://doi.org/10.1016/S0195-6698(85)80008-1}{On universal
  graphs with forbidden topological subgraphs}.
\newblock \emph{European J. Combin.}, 6(2):175--182, 1985.

\bibitem[{Diestel(1990)}]{Diestel90}
\textsc{Reinhard Diestel}.
\newblock \href{https://doi.org/10.1112/blms/24.1.90}{Graph decompositions: a
  study in infinite graph theory}.
\newblock Oxford University Press, 1990.

\bibitem[{Diestel(1994)}]{Diestel94}
\textsc{Reinhard Diestel}.
\newblock \href{https://doi.org/10.1006/jctb.1994.1048}{The depth-first search tree structure of} {$TK_{\aleph_0}$}-free graphs. 
\newblock \emph{J. Combin. Theory Ser. B}, 61(2):260--262, 1994.

\bibitem[{Diestel(2018)}]{Diestel5}
\textsc{Reinhard Diestel}.
\newblock \href{http://diestel-graph-theory.com/}{Graph theory}, vol. 173 of
  \emph{Graduate Texts in Mathematics}.
\newblock Springer, 5th edn., 2018.

\bibitem[{Diestel et~al.(1985)Diestel, Halin, and Vogler}]{DHV85}
\textsc{Reinhard Diestel, Rudolf Halin, and Walter Vogler}.
\newblock \href{https://doi.org/10.1007/BF02579242}{Some remarks on universal
  graphs}.
\newblock \emph{Combinatorica}, 5(4):283--293, 1985.

\bibitem[{Diestel and K\"{u}hn(1999)}]{DK99}
\textsc{Reinhard Diestel and Daniela K\"{u}hn}.
\newblock
  \href{https://doi.org/10.1002/(SICI)1097-0118(199910)32:2<191::AID-JGT8>3.3.CO;2-9}{A
  universal planar graph under the minor relation}.
\newblock \emph{J. Graph Theory}, 32(2):191--206, 1999.

\bibitem[{D\k{e}bski et~al.(2021)D\k{e}bski, Felsner, Micek, and
  Schr\"{o}der}]{DFMS21}
\textsc{Micha{\l} D\k{e}bski, Stefan Felsner, Piotr Micek, and Felix
  Schr\"{o}der}.
\newblock \href{https://doi.org/10.19086/aic.27351}{Improved bounds for
  centered colorings}.
\newblock \emph{Advances in Combinatorics}, 8, 2021.

\bibitem[{Dujmovi\'c et~al.(2017)Dujmovi\'c, Eppstein, and Wood}]{DEW17}
\textsc{Vida Dujmovi\'c, David Eppstein, and David~R. Wood}.
\newblock \href{https://doi.org/10.1137/16M1062879}{Structure of graphs with
  locally restricted crossings}.
\newblock \emph{SIAM J. Discrete Math.}, 31(2):805--824, 2017.

\bibitem[{Dujmovi\'c et~al.(pear)Dujmovi\'c, Esperet, Joret, Gavoille, Micek,
  and Morin}]{DEJGMM}
\textsc{Vida Dujmovi\'c, Louis Esperet, Gwena\"el Joret, Cyril Gavoille, Piotr
  Micek, and Pat Morin}.
\newblock \href{http://arxiv.org/abs/2003.04280}{Adjacency labelling for planar
  graphs (and beyond)}.
\newblock \emph{J. ACM}, to appear.
\newblock arXiv:2003.04280.

\bibitem[{Dujmovi{\'c} et~al.(2020{\natexlab{a}})Dujmovi{\'c}, Esperet, Joret,
  Walczak, and Wood}]{DEJWW20}
\textsc{Vida Dujmovi{\'c}, Louis Esperet, Gwena\"{e}l Joret, Bartosz Walczak,
  and David~R. Wood}.
\newblock \href{https://doi.org/10.19086/aic.12100}{Planar graphs have bounded
  nonrepetitive chromatic number}.
\newblock \emph{Advances in Combinatorics}, 5, 2020{\natexlab{a}}.

\bibitem[{Dujmovi{\'c} et~al.(2021)Dujmovi{\'c}, Esperet, Morin, Walczak, and
  Wood}]{DEMWW}
\textsc{Vida Dujmovi{\'c}, Louis Esperet, Pat Morin, Bartosz Walczak, and
  David~R. Wood}.
\newblock \href{https://doi.org/10.1017/S0963548321000213}{Clustered
  3-colouring graphs of bounded degree}.
\newblock \emph{Combinatorics, Probability \& Computing}, 2021.
\newblock arXiv:2002.11721.

\bibitem[{Dujmovi{\'c} et~al.(2020{\natexlab{b}})Dujmovi{\'c}, Joret, Micek,
  Morin, Ueckerdt, and Wood}]{DJMMUW20}
\textsc{Vida Dujmovi{\'c}, Gwena\"{e}l Joret, Piotr Micek, Pat Morin, Torsten
  Ueckerdt, and David~R. Wood}.
\newblock \href{https://doi.org/10.1145/3385731}{Planar graphs have bounded
  queue-number}.
\newblock \emph{J. ACM}, 67(4):22, 2020{\natexlab{b}}.

\bibitem[{Dujmovi{\'c} et~al.(2019)Dujmovi{\'c}, Morin, and Wood}]{DMW19b}
\textsc{Vida Dujmovi{\'c}, Pat Morin, and David~R. Wood}.
\newblock \href{http://arxiv.org/abs/1907.05168}{Graph product structure for
  non-minor-closed classes}.
\newblock arXiv:1907.05168, 2019.

\bibitem[{Dvo{\v{r}}{\'a}k(2013)}]{Dvorak13}
\textsc{Zden{\v{e}}k Dvo{\v{r}}{\'a}k}.
\newblock \href{https://doi.org/10.1016/j.ejc.2012.12.004}{Constant-factor
  approximation of the domination number in sparse graphs}.
\newblock \emph{European J. Comb.}, 34(5):833--840, 2013.

\bibitem[{Dvo{\v{r}}{\'a}k(2016)}]{Dvorak16}
\textsc{Zden{\v{e}}k Dvo{\v{r}}{\'a}k}.
\newblock \href{https://doi.org/10.1016/j.ejc.2015.09.001}{Sublinear
  separators, fragility and subexponential expansion}.
\newblock \emph{European J. Combin.}, 52(A):103--119, 2016.

\bibitem[{Dvo{\v{r}}{\'a}k et~al.(2021)Dvo{\v{r}}{\'a}k, Huynh, Joret, Liu, and
  Wood}]{DHJLW21}
\textsc{Zden{\v{e}}k Dvo{\v{r}}{\'a}k, Tony Huynh, Gwena\"el Joret, Chun-Hung
  Liu, and David~R. Wood}.
\newblock \href{https://doi.org/10.1007/978-3-030-62497-2_32}{Notes on graph
  product structure theory}.
\newblock In \textsc{David~R. Wood, Jan de~Gier, Cheryl~E. Praeger, and Terence
  Tao}, eds., \emph{2019-20 MATRIX Annals}, pp. 513--533. Springer, 2021.
\newblock arXiv:2001.08860.

\bibitem[{Dvo{\v{r}}{\'a}k and Norin(2016)}]{DN16}
\textsc{Zden{\v{e}}k Dvo{\v{r}}{\'a}k and Sergey Norin}.
\newblock \href{https://doi.org/10.1137/15M1017569}{Strongly sublinear
  separators and polynomial expansion}.
\newblock \emph{SIAM J. Discrete Math.}, 30(2):1095--1101, 2016.

\bibitem[{Dvo\v{r}\'{a}k(2018)}]{Dvorak18}
\textsc{Zden\v{e}k Dvo\v{r}\'{a}k}.
\newblock \href{https://doi.org/10.1016/j.ejc.2018.02.032}{On classes of graphs
  with strongly sublinear separators}.
\newblock \emph{European J. Combin.}, 71:1--11, 2018.

\bibitem[{Dvo\v{r}\'{a}k and Norin(2019)}]{DN19}
\textsc{Zden\v{e}k Dvo\v{r}\'{a}k and Sergey Norin}.
\newblock \href{https://doi.org/10.1016/j.jctb.2018.12.007}{Treewidth of graphs
  with balanced separations}.
\newblock \emph{J. Combin. Theory Ser. B}, 137:137--144, 2019.

\bibitem[{Erd\H{o}s and R\'{e}nyi(1963)}]{ER63a}
\textsc{Paul Erd\H{o}s and Alfred R\'{e}nyi}.
\newblock \href{https://doi.org/10.1007/BF01895716}{Asymmetric graphs}.
\newblock \emph{Acta Math. Acad. Sci. Hungar.}, 14:295--315, 1963.

\bibitem[{Esperet et~al.(2020)Esperet, Joret, and Morin}]{EJM}
\textsc{Louis Esperet, Gwena\"{e}l Joret, and Pat Morin}.
\newblock \href{http://arxiv.org/abs/2010.05779}{Sparse universal graphs for
  planarity}.
\newblock 2020.
\newblock arXiv:2010.05779.

\bibitem[{Esperet et~al.(2008)Esperet, Labourel, and Ochem}]{ELO08}
\textsc{Louis Esperet, Arnaud Labourel, and Pascal Ochem}.
\newblock \href{https://doi.org/10.1016/j.ipl.2008.04.020}{On induced-universal
  graphs for the class of bounded-degree graphs}.
\newblock \emph{Inform. Process. Lett.}, 108(5):255--260, 2008.

\bibitem[{Esperet and Raymond(2018)}]{ER18}
\textsc{Louis Esperet and Jean-Florent Raymond}.
\newblock \href{https://doi.org/10.1016/j.ejc.2017.09.003}{Polynomial expansion
  and sublinear separators}.
\newblock \emph{European J. Combin.}, 69:49--53, 2018.

\bibitem[{Fomin et~al.(2012)Fomin, Lokshtanov, and Saurabh}]{FLS-SODA12}
\textsc{Fedor~V. Fomin, Daniel Lokshtanov, and Saket Saurabh}.
\newblock \href{https://doi.org/10.1137/1.9781611973099.124}{Bidimensionality
  and geometric graphs}.
\newblock In \emph{Proc. 23rd {A}nnual {ACM}-{SIAM} {S}ymp. on {D}iscrete
  {A}lgorithms}, pp. 1563--1575. 2012.

\bibitem[{Fox and Pach(2010)}]{FP10}
\textsc{Jacob Fox and J{\'a}nos Pach}.
\newblock \href{https://doi.org/10.1017/S0963548309990459}{A separator theorem
  for string graphs and its applications}.
\newblock \emph{Combin. Probab. Comput.}, 19(3):371--390, 2010.

\bibitem[{Fox and Pach(2014)}]{FP14}
\textsc{Jacob Fox and J{\'a}nos Pach}.
\newblock \href{https://doi.org/10.1017/S0963548313000412}{Applications of a
  new separator theorem for string graphs}.
\newblock \emph{Combin. Probab. Comput.}, 23(1):66--74, 2014.

\bibitem[{Frieze and Tsourakakis(2014)}]{FT14}
\textsc{Alan Frieze and Charalampos~E. Tsourakakis}.
\newblock \href{https://doi.org/10.1080/15427951.2013.796300}{Some properties
  of random {A}pollonian networks}.
\newblock \emph{Internet Math.}, 10(1-2):162--187, 2014.

\bibitem[{Fu et~al.(2011)Fu, Hashikura, and Imai}]{FHI11}
\textsc{Norie Fu, Akihiro Hashikura, and Hiroshi Imai}.
\newblock \href{https://doi.org/10.1109/ISVD.2011.28}{Proximity and motion
  planning on l{\_}1-embeddable tilings}.
\newblock In \textsc{Fran{\c{c}}ois Anton and Wenping Wang}, eds., \emph{9th
  International Symp. on Voronoi Diagrams in Science and Engineering}, pp.
  150--159. {IEEE} Computer Society, 2011.

\bibitem[{F\"{u}redi and Komj\'{a}th(1997{\natexlab{a}})}]{FK97a}
\textsc{Zolt{\'a}n F\"{u}redi and P\'{e}ter Komj\'{a}th}.
\newblock \href{https://doi.org/10.1007/BF01200905}{Nonexistence of universal
  graphs without some trees}.
\newblock \emph{Combinatorica}, 17(2):163--171, 1997{\natexlab{a}}.

\bibitem[{F\"{u}redi and Komj\'{a}th(1997{\natexlab{b}})}]{FK97b}
\textsc{Zolt\'{a}n F\"{u}redi and P\'{e}ter Komj\'{a}th}.
\newblock
  \href{https://doi.org/10.1002/(SICI)1097-0118(199705)25:1<53::AID-JGT3>3.3.CO;2-U}{On
  the existence of countable universal graphs}.
\newblock \emph{J. Graph Theory}, 25(1):53--58, 1997{\natexlab{b}}.

\bibitem[{Gavoille and Labourel(2007)}]{GL07}
\textsc{Cyril Gavoille and Arnaud Labourel}.
\newblock \href{https://doi.org/10.1007/978-3-540-75520-3\_52}{Shorter implicit
  representation for planar graphs and bounded treewidth graphs}.
\newblock In \textsc{Lars Arge, Michael Hoffmann, and Emo Welzl}, eds.,
  \emph{Proc. 15th Annual European Symp. on Algorithms ({ESA 2007})}, vol. 4698
  of \emph{Lecture Notes in Comput. Sci.}, pp. 582--593. Springer, 2007.

\bibitem[{Gersten(1983)}]{Gersten83}
\textsc{Stephen~M. Gersten}.
\newblock \href{https://doi.org/10.1007/BF02095994}{Intersections of finitely
  generated subgroups of free groups and resolutions of graphs}.
\newblock \emph{Invent. Math.}, 71(3):567--591, 1983.

\bibitem[{Gol'dberg and Livshits(1968)}]{GL68}
\textsc{M.~K. Gol'dberg and \'E.~M. Livshits}.
\newblock \href{https://doi.org/10.1007/BF01116454}{On minimal universal
  trees}.
\newblock \emph{Mathematical notes of the Academy of Sciences of the USSR},
  4:713--717, 1968.

\bibitem[{Goodman et~al.(2018)Goodman, O'Rourke, and T\'{o}th}]{GOT18}
\textsc{Jacob~E. Goodman, Joseph O'Rourke, and Csaba~D. T\'{o}th}, eds.
\newblock Handbook of discrete and computational geometry.
\newblock CRC Press, 2018.

\bibitem[{Grimmett(1999)}]{Grimmett99}
\textsc{Geoffrey Grimmett}.
\newblock \href{https://doi.org/10.1007/978-3-662-03981-6}{Percolation}.
\newblock Springer-Verlag, 1999.

\bibitem[{Grohe et~al.(2018)Grohe, Kreutzer, Rabinovich, Siebertz, and
  Stavropoulos}]{GKRSS18}
\textsc{Martin Grohe, Stephan Kreutzer, Roman Rabinovich, Sebastian Siebertz,
  and Konstantinos Stavropoulos}.
\newblock \href{https://doi.org/10.1137/18M1168753}{Coloring and covering
  nowhere dense graphs}.
\newblock \emph{SIAM J. Discrete Math.}, 32(4):2467--2481, 2018.

\bibitem[{Grohe et~al.(2017)Grohe, Kreutzer, and Siebertz}]{GKS17}
\textsc{Martin Grohe, Stephan Kreutzer, and Sebastian Siebertz}.
\newblock \href{https://doi.org/10.1145/3051095}{Deciding first-order
  properties of nowhere dense graphs}.
\newblock \emph{J. ACM}, 64(3):Art. 17, 2017.

\bibitem[{Gr\"{u}nbaum and Shephard(1987)}]{GS87}
\textsc{Branko Gr\"{u}nbaum and G.~C. Shephard}.
\newblock Tilings and patterns.
\newblock W. H. Freeman, 1987.

\bibitem[{G\"{u}rdo\u{g}an and Kazakov(2016)}]{GK16}
\textsc{\"{O}mer G\"{u}rdo\u{g}an and Vladimir Kazakov}.
\newblock \href{https://doi.org/10.1103/PhysRevLett.117.201602}{New integrable
  4{D} quantum field theories from strongly deformed planar {$\mathcal{N} = 4$}
  supersymmetric {Y}ang-{M}ills theory}.
\newblock \emph{Phys. Rev. Lett.}, 117(20):201602, 2016.

\bibitem[{Hajnal and Komj\'{a}th(1988)}]{HK88}
\textsc{Andr\'as Hajnal and P\'{e}ter Komj\'{a}th}.
\newblock \href{https://doi.org/10.2307/2000770}{Embedding graphs into colored
  graphs}.
\newblock \emph{Trans. Amer. Math. Soc.}, 307(1):395--409, 1988.

\bibitem[{Hajnal and Pach(1984)}]{HP84}
\textsc{Andr\'as Hajnal and J\'anos Pach}.
\newblock
  \href{https://doi.org/10.1016/B978-0-444-86893-0.50028-0}{Monochromatic paths
  in infinite coloured graphs}.
\newblock In \emph{Finite and infinite sets}, vol.~37 of \emph{Colloq. Math.
  Soc. J\'{a}nos Bolyai}, pp. 359--369. North-Holland, 1984.

\bibitem[{Halin(1964)}]{Halin64a}
\textsc{Rudolf Halin}.
\newblock \href{https://doi.org/10.1007/BF01363288}{\"{U}ber simpliziale
  {Z}erf\"{a}llungen beliebiger (endlicher oder unendlicher) {G}raphen}.
\newblock \emph{Math. Ann.}, 156:216--225, 1964.

\bibitem[{Halin(1978)}]{Halin78}
\textsc{Rudolf Halin}.
\newblock \href{https://doi.org/10.1016/S0167-5060(08)70500-4}{Simplicial
  decompositions of infinite graphs}.
\newblock \emph{Ann. Discrete Math.}, 3:93--109, 1978.

\bibitem[{Halin(1982)}]{Halin82}
\textsc{Rudolf Halin}.
\newblock Simplicial decompositions: some new aspects and applications.
\newblock In \emph{Graph theory}, vol.~62 of \emph{North-Holland Math. Stud.},
  pp. 101--110. 1982.

\bibitem[{Halin(1984)}]{Halin84}
\textsc{Rudolf Halin}.
\newblock Simplicial decompositions and triangulated graphs.
\newblock In \emph{Graph theory and combinatorics}, pp. 191--196. Academic
  Press, 1984.

\bibitem[{He and Schramm(1993)}]{HeSc93}
\textsc{Zheng-Xu He and Oded Schramm}.
\newblock \href{https://doi.org/10.2307/2946541}{Fixed points, {K}oebe
  uniformization and circle packings}.
\newblock \emph{Ann. of Math. (2)}, 137(2):369--406, 1993.

\bibitem[{Henson(1971)}]{Henson71}
\textsc{C.~Ward Henson}.
\newblock \href{https://projecteuclid.org/euclid.pjm/1102970260}{A family of
  countable homogeneous graphs}.
\newblock \emph{Pacific J. Math.}, 38:69--83, 1971.

\bibitem[{Hickingbotham and Wood(2021)}]{HW21}
\textsc{Robert Hickingbotham and David~R. Wood}.
\newblock Dense subgraphs and minors in graph products.
\newblock 2021.

\bibitem[{Imrich(1975)}]{Imrich75}
\textsc{Wilfried Imrich}.
\newblock On {W}hitney's theorem on the unique embeddability of {$3$}-connected
  planar graphs.
\newblock In \emph{Recent advances in graph theory ({P}roc. {S}econd
  {C}zechoslovak {S}ympos.)}, pp. 303--306. 1975.

\bibitem[{Jung(1969)}]{Jung69}
\textsc{Heinz~A. Jung}.
\newblock \href{https://doi.org/10.1002/mana.19690410102}{Wurzelb\"{a}ume und
  unendliche {W}ege in {G}raphen}.
\newblock \emph{Math. Nachr.}, 41:1--22, 1969.

\bibitem[{Kannan et~al.(1992)Kannan, Naor, and Rudich}]{KNR92}
\textsc{Sampath Kannan, Moni Naor, and Steven Rudich}.
\newblock \href{https://doi.org/10.1137/0405049}{Implicit representation of
  graphs}.
\newblock \emph{SIAM J. Discrete Math.}, 5(4):596--603, 1992.

\bibitem[{Kierstead and Trotter(1994)}]{KT94}
\textsc{Hal~A. Kierstead and William~T. Trotter}.
\newblock \href{https://doi.org/10.1002/jgt.3190180605}{Planar graph coloring
  with an uncooperative partner}.
\newblock \emph{J. Graph Theory}, 18(6):569--584, 1994.

\bibitem[{Kierstead and Yang(2003)}]{KY03}
\textsc{Hal~A. Kierstead and Daqing Yang}.
\newblock \href{https://doi.org/10.1023/B:ORDE.0000026489.93166.cb}{Orderings
  on graphs and game coloring number}.
\newblock \emph{Order}, 20(3):255--264, 2003.

\bibitem[{Knauer and Ueckerdt(2012)}]{KU12}
\textsc{Kolja Knauer and Torsten Ueckerdt}.
\newblock
  \href{https://kam.mff.cuni.cz/workshops/mcw/work18/mcw2012booklet.pdf}{Simple
  treewidth}.
\newblock In \textsc{Pavel Ryt\'ir}, ed., \emph{Midsummer Combinatorial
  Workshop Prague}. 2012.

\bibitem[{Koebe(1936)}]{Koebe36}
\textsc{Paul Koebe}.
\newblock {K}ontaktprobleme der konformen {A}bbildung.
\newblock \emph{Berichte {\"u}ber die Verhandlungen der S{\"a}chsischen
  Akademie der Wissenschaften zu Leipzig. Math.-Phys. Klasse}, 88:141--164,
  1936.

\bibitem[{Kojman(1998)}]{Kojman98}
\textsc{Menachem Kojman}.
\newblock \href{https://doi.org/10.1112/S0024610798006668}{Representing
  embeddability as set inclusion}.
\newblock \emph{J. London Math. Soc. (2)}, 58(2):257--270, 1998.

\bibitem[{Komj\'{a}th(1999)}]{Komjath99}
\textsc{P\'{e}ter Komj\'{a}th}.
\newblock \href{https://doi.org/10.1016/S0012-365X(98)00339-2}{Some remarks on
  universal graphs}.
\newblock \emph{Discrete Math.}, 199(1-3):259--265, 1999.

\bibitem[{Komj\'{a}th(2011)}]{Komjath11}
\textsc{P\'{e}ter Komj\'{a}th}.
\newblock \href{https://doi.org/10.1016/j.disc.2010.11.004}{The chromatic
  number of infinite graphs---a survey}.
\newblock \emph{Discrete Math.}, 311(15):1448--1450, 2011.

\bibitem[{Komj\'{a}th et~al.(1988)Komj\'{a}th, Mekler, and Pach}]{KMP88}
\textsc{P\'{e}ter Komj\'{a}th, Alan~H. Mekler, and J\'{a}nos Pach}.
\newblock \href{https://doi.org/10.1007/BF02787220}{Some universal graphs}.
\newblock \emph{Israel J. Math.}, 64(2):158--168, 1988.

\bibitem[{Komj\'{a}th and Pach(1984)}]{KP84}
\textsc{P\'{e}ter Komj\'{a}th and J\'{a}nos Pach}.
\newblock \href{https://doi.org/10.1112/S002557930001250X}{Universal graphs
  without large bipartite subgraphs}.
\newblock \emph{Mathematika}, 31(2):282--290, 1984.

\bibitem[{Komj\'{a}th and Pach(1991)}]{KP91}
\textsc{P\'{e}ter Komj\'{a}th and J\'{a}nos Pach}.
\newblock \href{https://doi.org/10.1016/0012-365X(91)90340-8}{Universal
  elements and the complexity of certain classes of infinite graphs}.
\newblock \emph{Discrete Math.}, 95(1-3):255--270, 1991.

\bibitem[{K\"onig(1927)}]{Konig27}
\textsc{D\'enes K\"onig}.
\newblock {\"Uber} eine schlussweise aus dem endlichen ins unendliche.
\newblock \emph{Acta Sci. Math. (Szeged)}, 3(2--3):121--130, 1927.

\bibitem[{Kostochka et~al.(1997)Kostochka, Sopena, and Zhu}]{KSZ-JGT97}
\textsc{Alexandr~V. Kostochka, \'Eric Sopena, and Xuding Zhu}.
\newblock
  \href{https://doi.org/10.1002/(SICI)1097-0118(199704)24:4<331::AID-JGT5>3.0.CO;2-P}{Acyclic
  and oriented chromatic numbers of graphs}.
\newblock \emph{J. Graph Theory}, 24(4):331--340, 1997.

\bibitem[{Kratochv{\'{\i}}l and Vaner(2012)}]{KV12}
\textsc{Jan Kratochv{\'{\i}}l and Michal Vaner}.
\newblock \href{http://arxiv.org/abs/1210.8113}{A note on planar partial
  3-trees}.
\newblock arXiv:1210.8113, 2012.

\bibitem[{Kreutzer et~al.(2016)Kreutzer, Pilipczuk, Rabinovich, and
  Siebertz}]{KPRS16}
\textsc{Stephan Kreutzer, Michal Pilipczuk, Roman Rabinovich, and Sebastian
  Siebertz}.
\newblock \href{https://doi.org/10.4230/LIPIcs.MFCS.2016.85}{The generalised
  colouring numbers on classes of bounded expansion}.
\newblock In \textsc{Piotr Faliszewski, Anca Muscholl, and Rolf Niedermeier},
  eds., \emph{41st International Symp. on Mathematical Foundations of Computer
  Science ({MFCS} 2016)}, vol.~58 of \emph{LIPIcs}, pp. 85:1--85:13. 2016.

\bibitem[{Kriz and Kriz(2014)}]{KK14}
\textsc{Daniel Kriz and Igor Kriz}.
\newblock \href{https://doi.org/10.1016/j.aim.2014.01.006}{A spanning tree
  cohomology theory for links}.
\newblock \emph{Adv. Math.}, 255:414--454, 2014.

\bibitem[{Kuratowski(1930)}]{Kuratowski30}
\textsc{Kazimierz Kuratowski}.
\newblock Sur le probl\'{e}me des courbes gauches en topologie.
\newblock \emph{Fund. Math.}, 16:271--283, 1930.

\bibitem[{K\v{r}\'{\i}\v{z} and Thomas(1990{\natexlab{a}})}]{KT90}
\textsc{Igor K\v{r}\'{\i}\v{z} and Robin Thomas}.
\newblock \href{https://doi.org/10.1016/0012-365X(90)90150-G}{Clique-sums,
  tree-decompositions and compactness}.
\newblock \emph{Discrete Math.}, 81(2):177--185, 1990{\natexlab{a}}.

\bibitem[{K\v{r}\'{\i}\v{z} and Thomas(1990{\natexlab{b}})}]{KT90a}
\textsc{Igor K\v{r}\'{\i}\v{z} and Robin Thomas}.
\newblock \href{https://doi.org/10.1007/BF01787479}{On well-quasi-ordering
  finite structures with labels}.
\newblock \emph{Graphs Combin.}, 6(1):41--49, 1990{\natexlab{b}}.

\bibitem[{K\v{r}\'{\i}\v{z} and Thomas(1991)}]{KT91}
\textsc{Igor K\v{r}\'{\i}\v{z} and Robin Thomas}.
\newblock \href{https://doi.org/10.1016/0095-8956(91)90093-Y}{The {M}enger-like
  property of the tree-width of infinite graphs}.
\newblock \emph{J. Combin. Theory Ser. B}, 52(1):86--91, 1991.

\bibitem[{Lipton and Tarjan(1979)}]{LT79}
\textsc{Richard~J. Lipton and Robert~E. Tarjan}.
\newblock \href{https://doi.org/10.1137/0136016}{A separator theorem for planar
  graphs}.
\newblock \emph{SIAM J. Appl. Math.}, 36(2):177--189, 1979.

\bibitem[{Lyndon and Schupp(1977)}]{LyndonSchupp77}
\textsc{Roger~C. Lyndon and Paul~E. Schupp}.
\newblock Combinatorial group theory.
\newblock Springer-Verlag, 1977.

\bibitem[{Markenzon et~al.(2006)Markenzon, Justel, and Paciornik}]{MJP06}
\textsc{Lilian Markenzon, Claudia~Marcela Justel, and N.~Paciornik}.
\newblock \href{https://doi.org/10.1016/j.dam.2005.05.021}{Subclasses of
  {$k$}-trees: characterization and recognition}.
\newblock \emph{Discrete Appl. Math.}, 154(5):818--825, 2006.

\bibitem[{Mastrolia and Mizera(2019)}]{MM19}
\textsc{Pierpaolo Mastrolia and Sebastian Mizera}.
\newblock \href{https://doi.org/10.1007/jhep02(2019)139}{Feynman integrals and
  intersection theory}.
\newblock \emph{J. High Energy Phys.}, Article 139, 2019.

\bibitem[{Mih\'{o}k et~al.(2009)Mih\'{o}k, Mi\v{s}kuf, and
  Semani\v{s}in}]{MMS09}
\textsc{Peter Mih\'{o}k, Jozef Mi\v{s}kuf, and Gabriel Semani\v{s}in}.
\newblock \href{https://doi.org/10.7151/dmgt.1455}{On universal graphs for
  hom-properties}.
\newblock \emph{Discuss. Math. Graph Theory}, 29(2):401--409, 2009.

\bibitem[{Mohar(1988)}]{Mo88}
\textsc{Bojan Mohar}.
\newblock \href{https://doi.org/10.1016/0095-8956(88)90094-9}{Embeddings of
  infinite graphs}.
\newblock \emph{J. Combin. Theory Ser. B}, 44(1):29--43, 1988.

\bibitem[{Mohar and Thomassen(2001)}]{MoharThom}
\textsc{Bojan Mohar and Carsten Thomassen}.
\newblock Graphs on surfaces.
\newblock Johns Hopkins University Press, 2001.

\bibitem[{Moon(1965)}]{Moon65}
\textsc{John~W. Moon}.
\newblock On minimal {$n$}-universal graphs.
\newblock \emph{Proc. Glasgow Math. Assoc.}, 7:32--33, 1965.

\bibitem[{Nash-Williams(1967)}]{NashWilliams67}
\textsc{Crispin St. J.~A. Nash-Williams}.
\newblock \href{https://doi.org/10.1016/S0021-9800(67)80077-2}{Infinite
  graphs---a survey}.
\newblock \emph{J. Combinatorial Theory}, 3:286--301, 1967.

\bibitem[{Ne{\v{s}}et{\v{r}}il and Ossona~de Mendez(2012)}]{Sparsity}
\textsc{Jaroslav Ne{\v{s}}et{\v{r}}il and Patrice Ossona~de Mendez}.
\newblock \href{https://doi.org/10.1007/978-3-642-27875-4}{Sparsity}, vol.~28
  of \emph{Algorithms and Combinatorics}.
\newblock Springer, 2012.

\bibitem[{Ostilli(2012)}]{Ostilli12}
\textsc{Massimo Ostilli}.
\newblock \href{https://doi.org/10.1016/j.physa.2012.01.038}{Cayley trees and
  {B}ethe lattices: a concise analysis for mathematicians and physicists}.
\newblock \emph{Phys. A}, 391(12):3417--3423, 2012.

\bibitem[{Pach(1975)}]{Pach75}
\textsc{J\'{a}nos Pach}.
\newblock On metric properties of countable graphs.
\newblock \emph{Matematikai Lapok}, 26:305--310, 1975.

\bibitem[{Pach(1981)}]{Pach81a}
\textsc{J\'{a}nos Pach}.
\newblock \href{https://doi.org/10.1016/S0195-6698(81)80043-1}{A problem of
  {U}lam on planar graphs}.
\newblock \emph{European J. Combin.}, 2(4):357--361, 1981.

\bibitem[{Pach and T{\'o}th(2002)}]{PachToth-DCG02}
\textsc{J{\'a}nos Pach and G{\'e}za T{\'o}th}.
\newblock \href{https://doi.org/10.1007/s00454-002-2891-4}{Recognizing string
  graphs is decidable}.
\newblock \emph{Discrete Comput. Geom.}, 28(4):593--606, 2002.

\bibitem[{Pilipczuk and Siebertz(2019)}]{PS18}
\textsc{Micha{\l} Pilipczuk and Sebastian Siebertz}.
\newblock \href{https://doi.org/10.1137/1.9781611975482.91}{Polynomial bounds
  for centered colorings on proper minor-closed graph classes}.
\newblock In \textsc{Timothy~M. Chan}, ed., \emph{Proc. 30th {A}nnual
  {ACM}-{SIAM} {S}ymposium on {D}iscrete {A}lgorithms}, pp. 1501--1520. 2019.
\newblock arXiv:1807.03683.

\bibitem[{Pitz(2020)}]{Pitz20}
\textsc{Max Pitz}.
\newblock \href{https://doi.org/10.1016/j.jctb.2020.07.002}{A unified existence
  theorem for normal spanning trees}.
\newblock \emph{J. Combin. Theory Ser. B}, 145:466--469, 2020.

\bibitem[{Plotkin et~al.(1994)Plotkin, Rao, and Smith}]{PRS94}
\textsc{Serge Plotkin, Satish Rao, and Warren~D. Smith}.
\newblock \href{http://dl.acm.org/citation.cfm?id=314464.314625}{Shallow
  excluded minors and improved graph decompositions}.
\newblock In \textsc{Daniel Sleator}, ed., \emph{Proc. 5th Annual ACM-SIAM
  Symp. on Discrete Algorithms (SODA '94)}, pp. 462--470. ACM, 1994.

\bibitem[{Rado(1964)}]{Rado64}
\textsc{Richard Rado}.
\newblock \href{https://doi.org/10.4064/aa-9-4-331-340}{Universal graphs and
  universal functions}.
\newblock \emph{Acta Arith.}, 9:331--340, 1964.

\bibitem[{Reed and Seymour(1998)}]{ReedSeymour-JCTB98}
\textsc{Bruce~A. Reed and Paul Seymour}.
\newblock \href{https://doi.org/10.1006/jctb.1998.1835}{Fractional colouring
  and {H}adwiger's conjecture}.
\newblock \emph{J. Combin. Theory Ser. B}, 74(2):147--152, 1998.

\bibitem[{Robertson et~al.(1997)Robertson, Sanders, Seymour, and
  Thomas}]{RSST97}
\textsc{Neil Robertson, Daniel~P. Sanders, Paul Seymour, and Robin Thomas}.
\newblock \href{https://doi.org/10.1006/jctb.1997.1750}{The four-colour
  theorem}.
\newblock \emph{J. Combin. Theory Ser. B}, 70(1):2--44, 1997.

\bibitem[{Robertson and Seymour(2010)}]{RS-GraphMinors}
\textsc{Neil Robertson and Paul Seymour}.
\newblock Graph minors {I--XXIII}.
\newblock \emph{J. Combin. Theory Ser. B}, 1983--2010.

\bibitem[{Robertson and Seymour(1986{\natexlab{a}})}]{RS-II}
\textsc{Neil Robertson and Paul Seymour}.
\newblock \href{https://doi.org/10.1016/0196-6774(86)90023-4}{Graph minors.
  {II}. {A}lgorithmic aspects of tree-width}.
\newblock \emph{J. Algorithms}, 7(3):309--322, 1986{\natexlab{a}}.

\bibitem[{Robertson and Seymour(1986{\natexlab{b}})}]{RS-V}
\textsc{Neil Robertson and Paul Seymour}.
\newblock \href{https://doi.org/10.1016/0095-8956(86)90030-4}{Graph minors.
  {V}. {E}xcluding a planar graph}.
\newblock \emph{J. Combin. Theory Ser. B}, 41(1):92--114, 1986{\natexlab{b}}.

\bibitem[{Robertson et~al.(1992)Robertson, Seymour, and Thomas}]{RST-TAMS92}
\textsc{Neil Robertson, Paul Seymour, and Robin Thomas}.
\newblock \href{https://doi.org/10.2307/2154029}{Excluding subdivisions of
  infinite cliques}.
\newblock \emph{Trans. Amer. Math. Soc.}, 332(1):211--223, 1992.

\bibitem[{Robertson et~al.(1994)Robertson, Seymour, and Thomas}]{RST-JCTB94}
\textsc{Neil Robertson, Paul Seymour, and Robin Thomas}.
\newblock \href{https://doi.org/10.1006/jctb.1994.1073}{Quickly excluding a
  planar graph}.
\newblock \emph{J. Combin. Theory Ser. B}, 62(2):323--348, 1994.

\bibitem[{Rotman(1971)}]{Rotman71}
\textsc{Brian Rotman}.
\newblock \href{https://doi.org/10.1112/jlms/s2-4.1.123}{Remarks on some
  theorems of {R}ado on universal graphs}.
\newblock \emph{J. London Math. Soc. (2)}, 4:123--126, 1971.

\bibitem[{Scott et~al.(2019)Scott, Seymour, and Wood}]{SSW19}
\textsc{Alex Scott, Paul Seymour, and David~R. Wood}.
\newblock \href{https://doi.org/10.1002/jgt.22363}{Bad news for chordal
  partitions}.
\newblock \emph{J. Graph Theory}, 90:5--12, 2019.

\bibitem[{Serre(1977)}]{Serre77}
\textsc{Jean-Pierre Serre}.
\newblock Arbres, amalgames, {${\rm SL}_{2}$}.
\newblock Soci\'{e}t\'{e} Math\'{e}matique de France, 1977.

\bibitem[{Serre(1980)}]{Serre80}
\textsc{Jean-Pierre Serre}.
\newblock Trees.
\newblock Springer-Verlag, 1980.

\bibitem[{Seymour(1980)}]{Seymour80}
\textsc{Paul Seymour}.
\newblock \href{https://doi.org/10.1016/0012-365X(80)90158-2}{Disjoint paths in
  graphs}.
\newblock \emph{Discrete Math.}, 29(3):293--309, 1980.

\bibitem[{Sleator et~al.(1988)Sleator, Tarjan, and Thurston}]{STT-JAMS88}
\textsc{Daniel~D. Sleator, Robert~E. Tarjan, and William~P. Thurston}.
\newblock \href{https://doi.org/10.1090/S0894-0347-1988-0928904-4}{Rotation
  distance, triangulations, and hyperbolic geometry}.
\newblock \emph{J. Amer. Math. Soc.}, 1(3):647--681, 1988.

\bibitem[{Stallings(1983)}]{Stallings83}
\textsc{John~R. Stallings}.
\newblock \href{https://doi.org/10.1007/BF02095993}{Topology of finite graphs}.
\newblock \emph{Invent. Math.}, 71(3):551--565, 1983.

\bibitem[{Steinitz(1922)}]{Steinitz22}
\textsc{Ernst Steinitz}.
\newblock Polyeder und {R}aumeinteilungen.
\newblock \emph{Encyclop{\"a}die der Mathematischen Wissenschaften},
  3AB12:1--139, 1922.

\bibitem[{Stephenson(2005)}]{Stephenson05}
\textsc{Kenneth Stephenson}.
\newblock
  \href{https://www.cambridge.org/au/academic/subjects/mathematics/geometry-and-topology/introduction-circle-packing-theory-discrete-analytic-functions?format=HB&isbn=9780521823562}{Introduction
  to circle packing: {T}he theory of discrete analytic functions}.
\newblock Cambridge Univ. Press, 2005.

\bibitem[{Tamassia(2013)}]{HandbookGraphDrawing}
\textsc{Roberto Tamassia}, ed.
\newblock \href{https://doi.org/10.1201/b15385}{Handbook of graph drawing and
  visualization}.
\newblock Chapman and Hall / CRC Press, 2013.

\bibitem[{Thomas(1988)}]{Thomas88}
\textsc{Robin Thomas}.
\newblock \href{https://people.math.gatech.edu/~thomas/PAP/twcpt.pdf}{The
  tree-width compactness theorem for hypergraphs}.
\newblock 1988.

\bibitem[{Thomassen(1980)}]{Thomassen80}
\textsc{Carsten Thomassen}.
\newblock \href{https://doi.org/10.1016/S0195-6698(80)80039-4}{{$2$}-linked
  graphs}.
\newblock \emph{European J. Combin.}, 1(4):371--378, 1980.

\bibitem[{Thomassen(1983)}]{Thom83}
\textsc{Carsten Thomassen}.
\newblock Infinite graphs.
\newblock In \emph{Selected topics in graph theory, 2}, pp. 129--160. Academic
  Press, 1983.

\bibitem[{Thomassen(1989{\natexlab{a}})}]{Thomassen89}
\textsc{Carsten Thomassen}.
\newblock
  \href{https://doi.org/10.1111/j.1749-6632.1989.tb22479.x}{Configurations in
  graphs of large minimum degree, connectivity, or chromatic number}.
\newblock In \emph{{P}roc.\ 3rd {I}nternational {C}onference on Combinatorial
  {M}athematics}, vol. 555 of \emph{Ann. New York Acad. Sci.}, pp. 402--412.
  1989{\natexlab{a}}.

\bibitem[{Thomassen(1989{\natexlab{b}})}]{Thomassen89a}
\textsc{Carsten Thomassen}.
\newblock \href{https://doi.org/10.1007/BF00181436}{The converse of the
  {J}ordan curve theorem and a characterization of planar maps}.
\newblock \emph{Geom. Dedicata}, 32(1):53--57, 1989{\natexlab{b}}.

\bibitem[{Thomassen(1990)}]{Thomassen90}
\textsc{Carsten Thomassen}.
\newblock \href{https://doi.org/10.2307/2324687}{A link between the {J}ordan
  curve theorem and the {K}uratowski planarity criterion}.
\newblock \emph{Amer. Math. Monthly}, 97(3):216--218, 1990.

\bibitem[{Thomassen(2017)}]{Thomassen17}
\textsc{Carsten Thomassen}.
\newblock \href{https://doi.org/10.1016/j.jctb.2016.08.005}{The number of
  colorings of planar graphs with no separating triangles}.
\newblock \emph{J. Combin. Theory Ser. B}, 122:615--633, 2017.

\bibitem[{Ueckerdt et~al.(2021)Ueckerdt, Wood, and Yi}]{UWY}
\textsc{Torsten Ueckerdt, David~R. Wood, and Wendy Yi}.
\newblock \href{http://arxiv.org/abs/2108.00198}{An improved planar graph
  product structure theorem}.
\newblock 2021.
\newblock arXiv:2108.00198.

\bibitem[{van~den Heuvel et~al.(2017)van~den Heuvel, Ossona~de Mendez, Quiroz,
  Rabinovich, and Siebertz}]{HOQRS17}
\textsc{Jan van~den Heuvel, Patrice Ossona~de Mendez, Daniel Quiroz, Roman
  Rabinovich, and Sebastian Siebertz}.
\newblock \href{https://doi.org/10.1016/j.ejc.2017.06.019}{On the generalised
  colouring numbers of graphs that exclude a fixed minor}.
\newblock \emph{European J. Combin.}, 66:129--144, 2017.

\bibitem[{van~den Heuvel and Wood(2018)}]{vdHW18}
\textsc{Jan van~den Heuvel and David~R. Wood}.
\newblock \href{https://doi.org/10.1112/jlms.12127}{Improper colourings
  inspired by {H}adwiger's conjecture}.
\newblock \emph{J. London Math. Soc.}, 98:129--148, 2018.
\newblock arXiv:1704.06536.

\bibitem[{Wagner(1937)}]{Wagner37}
\textsc{Klaus Wagner}.
\newblock \href{https://doi.org/10.1007/BF01594196}{{\"U}ber eine {E}igenschaft
  der ebene {K}omplexe}.
\newblock \emph{Math. Ann.}, 114:570--590, 1937.

\bibitem[{Wagner(1967)}]{Wagner67}
\textsc{Klaus Wagner}.
\newblock
  \href{https://doi.org/10.1016/S0021-9800(67)80103-0}{Fastpl\"{a}ttbare
  {G}raphen}.
\newblock \emph{J. Combinatorial Theory}, 3:326--365, 1967.

\bibitem[{{Wikipedia}(2020)}]{NielsenSchreierTheorem}
\textsc{{Wikipedia}}.
\newblock
  \url{https://en.wikipedia.org/wiki/Nielsen–Schreier_theorem}.
\newblock 2020.

\bibitem[{Wulf(2016)}]{Wulf16}
\textsc{Lasse Wulf}.
\newblock
  \href{https://i11www.iti.kit.edu/_media/teaching/theses/ba-wulf-16.pdf}{Stacked
  treewidth and the {C}olin de {V}erdi\'ere number}.
\newblock 2016.
\newblock Bachelorthesis, Institute of Theoretical Computer Science, Karlsruhe
  Institute of Technology.

\bibitem[{Zhu(2009)}]{Zhu09}
\textsc{Xuding Zhu}.
\newblock \href{https://doi.org/10.1016/j.disc.2008.03.024}{Colouring graphs
  with bounded generalized colouring number}.
\newblock \emph{Discrete Math.}, 309(18):5562--5568, 2009.

\end{thebibliography}
}

\newpage
\printindex
\end{document}