\documentclass[../absorber.tex]{subfiles}
\begin{document}

Particle-in-cell simulations are useful for investigating intense laser-plasma interactions in overdense plasmas, but a truncated plasma boundary can produce an unphysically hot return current.  This return current is present with absorbing, reflecting and thermal particle boundary conditions alike, and it can drastically alter simulation results.  We have devised an absorbing particle boundary condition that stops energetic particles over a defined region of the simulation space.  Stopping these particles over a sufficiently large distance allows the background plasma to generate a suitably cool return current that mimics the results of a semi-infinite, causally separated simulation.

Various different schemes were proposed for statistically selecting, stopping and re-emitting hot particles, with the best results given by the linearly varying absorber described in Sec.~\ref{sec:linear} that calculates the local temperature via Eq.~(\ref{Eq:lin-int}).  The appropriate mean free path of the absorber was explored, showing that an absorber with a mean free path of $\lambda \gtrsim \bigO(10\,c/\omega_p)$ gives proper results for our tests.
% Even a very short stopping region performed well for short times, but a longer absorbing region was required for long-term performance.
As simulation behavior can vary greatly depending on the application, care must be taken to ensure that the absorber parameters used for a particular case appropriately mimic the behavior of a semi-infinite boundary.

\end{document}