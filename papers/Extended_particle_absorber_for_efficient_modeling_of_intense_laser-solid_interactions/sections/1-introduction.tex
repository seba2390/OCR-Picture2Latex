\documentclass[../absorber.tex]{subfiles}
\begin{document}

Particle-in-cell (PIC) simulations have long been used to study the kinetic effects of laser-plasma interactions with overdense plasmas, with applications including the study of novel X-ray light sources,\cite{Watts2002DynamicsSpectra,Tsakiris2006RoutePulses,Park2006High-energyLasers,Ji2014EnergyInteraction,Capdessus2014Regime} generation of mono-energetic ion beams,\cite{Robinson2008RadiationPulses} experiments of collisionless shocks,\cite{Spitkovsky2008OnPlasmas,Martins2009IONSHOCKS,Fiuza2012Weibel-instability-mediatedLasers} transport experiments through warm-dense matter\cite{May2011MechanismInterface} and the fast ignition concept for inertial confinement fusion.\cite{Tabak1994IgnitionLasers,Tonge2009ALasers}  The laser-plasma interactions are often simulated for times on the order of picoseconds (1000s of laser periods) and for distances on the order of hundreds of microns (100s of laser wavelengths and 1000s of collisionless skin depths).  The thickness of the simulated overdense plasma region varies depending on the physical setup as well as the acceleration mechanism being explored.  For targets thin enough that the entire target can be simulated, vacuum regions are often used on either side of the target, consistent with the experimental setup.\cite{Silva2004ProtonInteractions,Sentoku2003AnomalousPlasma,Pukhov2001Three-DimensionalLaser,Lasinski1999Particle-in-cellApplications}  For thicker targets that cannot be simulated in their entirety, the plasma may be extended to the simulation boundary,\cite{Kemp2012InteractionPlasma,Yang1995AbsorptionBremsstrahlung,Pukhov1997LaserTargets,Wilks1993SimulationsInteractions} where an absorbing or thermal particle boundary condition is used.

When the laser-plasma interaction at the front of the target leads to large quantities of energetic electrons, a large flux of particles will in turn be found leaving the rear simulation boundary.  Independent of the particle boundary condition, the exiting stream of energetic particles can be problematic: either sharply removing the current (absorbing boundary condition) or the sudden stopping and accumulation of charge (thermal boundary condition) leads to an electric field buildup at the boundary.  This strong electric field will generate a return current that is carried by a hot, rarefied electron population (nearly symmetric to the incident electrons) instead of the proper cold, dense population.\cite{Bell1997Fast-electronExperiments}  The hot return current can both modify streaming instabilities that arise in the bulk plasma and modify the laser-plasma interactions at the front surface.   To avoid the electron refluxing, the plasma may be elongated such that the rear of the plasma is causally separated from the laser-plasma interaction region for some desired duration.\cite{Levy2013ConservationInteractions,Adam2006DispersionPlasmas}  In this case a small vacuum region is often placed to the right of the plasma to simplify the particle boundary conditions.  However, elongating the plasma introduces extra overhead from simulating the (often very particle-dense) excess material.

In an effort to preserve simulation integrity, while shortening the simulated plasma region, we propose an extended particle boundary condition that sporadically stops particles of certain energies over a defined distance.  In the presence of a low-density, hot particle beam or tail shooting into the plasma, this extended stopping avoids localized charge buildup or current deficiency that occurs when using an absorbing or thermal boundary condition, thus allowing a suitably cool return current to develop in the background plasma over an extended period of time and space.  The initial idea of this extended absorbing boundary condition was briefly described alongside some results in references,\cite{Tonge2009ALasers,Kemp2014LaserplasmaIgnition} but here we provide details for implementation, improvements, potential issues and best practices of such an absorber.
% Maybe add a comment at the end of this paragraph about the energy conservation?

The outline of this paper is as follows.  We first discuss in Sec.~\ref{sec:overdense} the possible issues with truncating a semi-infinite plasma with standard reflecting, absorbing, or thermal bath particle boundary conditions. In Section~\ref{sec:absorber} we present the concept and design of the absorber, along with the parameters that can be specified.  Finally, in Sec.~\ref{sec:results} we present simulation results from the PIC code \textsc{Osiris}, where we test the implementation of the absorber boundary condition on a finite target against a semi-infinite (causally separated) plasma.

\end{document}