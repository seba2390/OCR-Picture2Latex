\documentclass[../absorber.tex]{subfiles}
\begin{document}

In this appendix we describe how we calculate the local proper thermal velocity for use with the damping parameters.  First, we assume a Maxwellian distribution of the form
\begin{equation} \label{Eq:Max}
    f_0(u) = \frac{n_0}{\sqrt{2\pi \Bar{u}_0^2}} \exp \left( \frac{u^2}{2\Bar{u}_0^2} \right)
\end{equation}
for density $n_0$, proper velocity $u \equiv \gamma v$ and thermal velocity $\Bar{u}_0$.  We can approximate the average thermal velocity by summing $|u|$ over all particles in a cell.  For a Maxwellian of the same form as in Eq.~(\ref{Eq:Max}), we have that
\begin{equation} \label{Eq:lin-int}
    \int_{-\infty}^\infty |u| f_0(u) \, du = \sqrt{\frac{2}{\pi}}n_0\Bar{u}_0,
\end{equation}
which can be inverted to find $\Bar{u}_0$.

If instead the plasma is considered to be two distinct species with different thermal velocities and densities [e.g., including a beam with density $n_1$, thermal velocity $\Bar{u}_1$ and corresponding distribution $f_1(u)$], the above integral in Eq.~(\ref{Eq:lin-int}) could yield a distorted thermal velocity.  Another option is to perform the integral using the fourth-root of the thermal velocity, which gives
\begin{equation} \label{Eq:4th-int}
\begin{split}
    \int_{-\infty}^\infty |u|^{1/4} & \left[ f_0(u) + f_1(u) \right] \, du = \\
    & \frac{2^{1/8}\Gamma\left(\frac{5}{8}\right)}{\sqrt{\pi}} \left( n_0 \Bar{u}_0^{1/4} + n_1 \Bar{u}_1^{1/4} \right),
\end{split}
\end{equation}
where $\Gamma$ is the standard gamma function.  Since the densities add linearly but the thermal velocities add as the fourth-root, a high-energy beam should distort the sum less than in Eq.~(\ref{Eq:lin-int}).  Note that Eq.~(\ref{Eq:4th-int}) should be calculated for a single distribution and inverted to solve for $\Bar{u}_0$.  In practice, we found that the fourth-root calculation differed only slightly from the simple average of the thermal velocity due to the extremely low density of the beam.
%??? It would nice to be able to say that in most cases it made little differences but in some cases it did so we use the fourth root option as a default????
% I changed Josh's implementation, and after I did that it really made basically no difference.  I'm going to default to just the regular temperature in the code.
However, the fourth-root calculation may be important for some parameter regimes.

\end{document}