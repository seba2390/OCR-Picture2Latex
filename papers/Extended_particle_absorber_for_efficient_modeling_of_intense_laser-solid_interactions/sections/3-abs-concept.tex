\documentclass[../absorber.tex]{subfiles}
\begin{document}

We desire to model a semi-infinite target with a finite simulation.  We must therefore find a way to remove the energetic beam electrons from the simulation while providing a self-consistent return current.  One way to do this is to stop the relativistic electrons not at a single point (as in the absorbing boundary condition), but over an extended region.  The electrostatic field is then dispersed over a sufficient ``volume'' such that the electric field driving the reflux is spread out and reduced in amplitude. Thus, a larger ``volume'' of bulk plasma is driven backwards without reflecting the high-energy electrons.  In essence, we create an extended boundary condition for the particles.  The plasma can then shield out the field more rapidly such that the potential does not build up over time.

To create this extended boundary condition, we denote some region near the back of the simulation box to be the absorbing region.\footnote
{
 The absorbing region can also be a finite region in the middle of the plasma.  This was done in Ref.~\onlinecite{Tonge2009ALasers} for integrated fast ignition simulations.  In this case the electrons were slowed gradually by a drag force.
}
Within this volume, high-energy particles are selected at random and instantaneously thermalized to a given temperature in a single time step.  We do this rather than slowing down the particles gradually---arguably more physically correct---because it allows us to stop the particles throughout the absorber volume without any knowledge of the global beam characteristics; it also obviates the need to track particles across multiple distributed-memory processes as they decelerate (more algorithmically complex).  The disadvantage of stopping the particles instantaneously is the emission of Bremsstrahlung radiation.\cite{Allen1973AstrophysicalQuantities}  However, this radiation is in most cases both poorly resolved on the simulation grid and can effectively be eliminated by higher-order particle shapes and smoothing.

We define a stopping loop as the process of iterating over all particles in the defined absorbing region, calculating the probability that each particle will be stopped, then stopping a particle (i.e., directly changing its momentum) if a randomly generated number is less than that probability.  Stopping loops are performed at a defined time interval, $t_a$ (can be one or multiple simulation time steps).  At a given position, the probability of particle stopping and corresponding re-emission are controlled by three parameters: an energy threshold, a mean free path, $\lambda$, and a re-emission temperature.  The energy threshold provides a way to distinguish between background particles (energy is below the threshold, nothing happens) and hot electrons (energy is above the threshold, stopping may occur).  The mean free path, particle velocity and time between stopping loops, $t_a$, are used to calculate the probability of stopping.  This should be done such that the mean distance traveled by an ensemble of particles before stopping is given by the mean free path, but that no energetic particles reach the particle boundary.  The re-emission temperature is used to define the Maxwellian distribution from which the particle will be re-emitted in each of the three dimensions.  For both the energy threshold and the re-emission temperature, we allow the user to specify either absolute numbers or to calculate the parameters as multiples of the local plasma temperature near a given particle.  If required, the local plasma temperature is calculated by integrating the absolute value of the proper velocity in each dimension over all particles in a given cell.  See Appendix~\ref{app:temp} for further details on how this is done.  A typical value for the energy threshold is six times the local thermal velocity in any direction, and the re-emission temperature is usually just set to be equal to the local thermal velocity in every direction.  Below we discuss two possible methods to accomplish the desired stopping.

\subsection{Hazard function stopping} \label{sec:hazard}

Let us assume that all the particles we wish to stop are streaming near the speed of light, $c$, in the positive $x$-direction.  Rather than stopping these particles at a single boundary that is one cell thick (like a traditional thermal boundary condition), we aim to stop these particles one at a time over the length of an absorbing region many cells thick.  Thus particles are stopped over time and space.  To facilitate this, we can define a desired normalized distribution that specifies the density of particles in space once they are all stopped.  This is in fact a probability density function, call it $f(s)$.  For example, we could use the function
\begin{equation} \label{Eq:prob}
    f(s) = 2\sin^2(\pi s),
\end{equation}
where $s \equiv x/L_a$ for a particle that has traveled distance $x$ into the absorbing region of length $L_a$.  Defined in this way, all particles will have stopped after a distance of $L_a$.  We can also consider a parameterization in time as $s \equiv t/T_a$ for a particle that takes a time $T_a$ to traverse the absorbing region (i.e., $T_a=L_a/v_x$ for a particle velocity $v_x$ in the direction of the absorbing region).  The associated cumulative distribution function is then
\begin{equation}
    F(s) = \int_0^s f(s')\,ds' = s - \frac{\sin ( 2 \pi s )}{2\pi},
\end{equation}
and the (continuous) probability that a particle will stop in this interval is given by the hazard function $h(s)$, defined as\cite{Evans2000TermsSymbols}
\begin{equation} \label{Eq:hazard}
    h(s) \equiv \frac{f(s)}{1-F(s)} = \frac{4\pi \sin^2 ( \pi s )}{2\pi ( 1-s ) + \sin ( 2 \pi s )}.
\end{equation}
Since we stop particles instantaneously at discrete intervals rather than gradually slowing them over some distance, we assume that a given particle will have been selected to be stopped after at most $N_a=L_a/t_av_x=T_a/t_a$ stopping loops (within each stopping loop only a small number of selected particles are stopped). Thus the probability of stopping for each hot particle located at position $s$ for the hazard function absorber is
\begin{equation} \label{Eq:P-haz}
    P_\mathrm{haz}(s) = \begin{dcases} 
      h(s)/N_a = h(s)t_a/T_a & s < 1 \\
      1 & s \geq 1
   \end{dcases}.
\end{equation}
The probability density function given in Eq.~(\ref{Eq:prob}) is shown along with the probability of stopping for $N_a=100$ in Fig.~\ref{fig:f-and-h}.  Note that the mean free path for this case is $\lambda=L_a/2$, and thus the scheme can also be parameterized by $s=x/2\lambda$.

\begin{figure}
\includegraphics[width=\linewidth]{figures/stopping-probability.jpeg}
\caption{\label{fig:f-and-h} Normalized probability density function (blue) for desired particle distribution, along with the probability of stopping (orange, dashed) for $N_a=100$ stopping loops over the absorbing region. Here the mean free path is $\lambda=0.5\,L_a$.}
\end{figure}

% \begin{figure}
%     \centering

%     \captionsetup{width=0.98\linewidth}
%      \includegraphics[width=\linewidth]{figures/actual-figures/stopping-probability.png}
%     \caption{Normalized probability density function (blue) for desired particle distribution, along with the probability of stopping (orange, dashed) for $N_a=100$ stopping loops over the absorbing region. Here the mean free path is $\lambda=0.5\,L_a$.}
%     \label{fig:f-and-h}
% \end{figure}

Stopping via a hazard function works well when the hot particles are all streaming in the same direction and have large velocity components mainly in one direction (i.e., $|\mathbf{v}| \approx v_x$).
%%% What do you think of this explanation Warren?
However, many simulations employing the absorber may exhibit forward-going particle beams with non-negligible transverse momentum.  Since the hazard function absorber stops particles based only on their longitudinal momentum, the background plasma can become disproportionately and unstably hot in the transverse direction.  If transversely hot particles have very low longitudinal momentum or are located near the front of the absorbing region, they will have a very low probability of stopping and can continue to drift unchecked for long periods of time (especially dangerous for transversely periodic simulations).  In our test simulations the growing transverse currents could eventually form a large reflecting magnetic field near the beginning of the absorbing region, resulting in a reflux of the hot particle beam.  To prevent any unstable transverse momentum growth, we propose the following alternative algorithm that stops hot particles without assuming a specific velocity distribution (e.g., all moving forward near the speed of light).

% However, many simulations employing the absorber may be quasi-1D (very narrow and periodic in the transverse direction).  For these types of simulations, it is likely to have particles with very large transverse velocities streaming toward the boundary.  Stopping particles of this kind 
% %???Not sure I understand this for simulations with plane waves???
% can 
% %??? Is what you were thinking???
% lead to density enhancements which are narrower in the longitudinal direction and not uniform in the transverse direction.
% %???
% %???
% %If the longitudinal length is much larger than then the transverse width of %the density enhancement then this
% These nonuniformities can 
% %???
% induce fields in the stopping region that accelerate particles in the transverse direction---often particles that are drifting backward back into the simulation.  
% %???We should be more clear about why this leads to transverse fields"""
% To avoid such artificial heating, we propose the following alternative algorithm that stops particles independent of an assumed velocity distribution %???
% , e.g., all moving forward near the speed of light
% %???
% .

\subsection{Linearly varying absorber} \label{sec:linear}

A much simpler approach is to stop particles based on their energy alone, independent of their direction of motion or position in the absorbing region.  Let us assume that most of the energetic particles we wish to stop are traveling mainly in the forward $x$-direction at speed $c$ (though some may be travelling backward or have large transverse velocities).  Suppose that, similar to Sec.~\ref{sec:hazard}, our absorbing region begins a distance $L_a$ from the right simulation edge and that we wish to have a mean free path of $\lambda<L_a/2$ (now in any direction) for the particles, such that they are all stopped before the simulation boundary.  An intuitive way to accomplish this is to perform a large number of stopping loops, $N_a \approx 100$, in the time it takes an energetic particle to travel a distance of $x_f$ through the absorbing region; we can naively set the stopping probability to $P = 1/N_a$ for each particle in each loop.  We thus require the minimum time between stopping loops to be $t_a = x_f/N_a c$.
%%% This is me explaining where the 63\% comes from
Such a scheme actually corresponds to a hazard function of $h(x)=1/x_f$ and a probability density function of $f(x)=\frac{1}{x_f} \exp \big( -\frac{x}{x_f} \big)$.  However, after $N_a$ stopping loops (i.e., after particles propagate a maximum distance of $N_a c t_a = x_f$), integrating $\int_0^{x_f} f(x)\,dx$ shows that on average only 63\% of particles will have been stopped using the $P=1/N_a$ probability.  We would then need to use the rest of the absorbing region to stop the remaining hot particles.  We can extend this idea to construct a more effective absorber that meets our needs.

% Hazard function code, stop_dist = half the distance until all particles have stopped, i.e., stop_dist = \lambda = mean free path
% Linear code, stop_dist = distance until you reach half of the absorbing region, i.e., stop_dist = 2\lambda
Based on the above argument, the stopping probability for a particle with velocity magnitude $v=|\mathbf{v}|$ would be $P(v) = vt_a/x_f$.  Note that the velocity used is independent of direction, providing equal stopping for particles traveling rapidly forward, backward, or transversely to the absorbing region.  However, to facilitate the stopping of particles past a distance of $x_f$, we parameterize the absorbing region by two longitudinal positions: (1) a location specifying the start of the absorbing region (defined as $x=0$ here) and (2) a location specifying where the absorber is at its full strength, defined as $x_f$.  Variables such as the mean free path, energy threshold and re-emission temperature are defined at positions (1) and (2); we will refer to these two kinds of parameters as variables of type ``start'' and type ``full'' ($x_f$ refers to the full position).  In front of the start position, the absorber is turned off.  In-between the start and full positions, the stopping parameters (e.g., energy threshold and mean free path) are changed linearly from their start values to their full values.  From the full position to the back of the box, the stopping parameters remain at their full values.  The absorber is designed to start with modest stopping parameters and increase to more stringent stopping in order to avoid any sharp transitions in the simulation physics that may result in wave reflections or other spurious effects.  Employing a linear ramp comes with the added benefit that we can use a similar set of input parameters for wide range of simulations with varying beam characteristics.

To be specific, we define a scale length, $L(x)$, that varies linearly between the start and full scale lengths, $L_s$ and $L_f$, respectively, over the distance $x_f$:
\begin{equation} \label{Eq:length}
    L(x) = \begin{dcases} 
      L_s + 2(L_f-L_s)x & x < x_f \\
      L_f & x \geq x_f
   \end{dcases}.
\end{equation}
The probability of stopping for this linearly varying absorber is then
\begin{equation}
    P_\mathrm{lin}(x,v) = vt_a/L(x).
\end{equation}
We then let $L_s=x_f$ and define $L_f$ such that particles are rapidly stopped, e.g., $L_f=(L_a-x_f)/20$.

% For example, from our previous reasoning we could then define the start mean free path at position (1)---the beginning of the absorbing region---to be $\lambda_s = L_a$.  Then to ensure that all particles are stopped before the end of the absorbing region, we can set position (2) to be halfway through the absorbing region, a distance $L_a$ from the simulation edge.  We desire a new mean free path such that the probability of stopping is large, maybe $P_\mathrm{large} \approx 1/5$.  Thus the full mean free path would be $\lambda_f = L_a P_\mathrm{desired}/P_\mathrm{large} \approx L_a/20$.  Once again, between positions (1) and (2), the mean free path would decrease linearly from $\lambda_s$ to $\lambda_f$.

Using this method, instead of initially specifying the probability density function $f(s)$ and then finding the hazard function $h(s)$, we are specifying $h(s)$ first.  One can solve for $f(s)$ using Eq.~(\ref{Eq:hazard}) and its derivative in $s$, assuming a piecewise form for $h(s)$.  First we write down the hazard function in normalized coordinates as
\begin{equation} \label{Eq:haz_lin}
    h(s) = \begin{dcases} 
      \frac{1}{\lambda_s + 2(\lambda_f-\lambda_s)s} & s < s_f \\
      \frac{1}{\lambda_f} & s \geq s_f
   \end{dcases},
\end{equation}
where similarly to Sec.~\ref{sec:hazard} we have $s\equiv x/L_a$, $\lambda_i = L_i/L_a$ and $s_f=x_f/L_a$.  To solve for the probability density function, we require $f(s)$ to be a continuous piecewise function and to integrate to 1, yielding the result
% \begin{equation} \label{Eq:f_lin}
%     f(s) = \begin{dcases} 
%       x_f \cdot \left( \lambda_s x_f \right)^{\frac{x_f}{\lambda_f-\lambda_s}} \cdot \left( \left[ \lambda_f-\lambda_s \right]x + \lambda_s x_f \right)^{\frac{x_f}{\lambda_s-\lambda_f} - 1} & x < x_f \\
%       \frac{1}{\lambda_f} \cdot \exp{\frac{x_f-x}{\lambda_f}} \cdot \left(\frac{\lambda_f}{\lambda_s}\right)^{\frac{x_f}{\lambda_s-\lambda_f}} & x \geq x_f
%   \end{dcases}.
% \end{equation}
\begin{equation} \label{Eq:f_lin}
f(s) = \left\{ \def\arraystretch{1.4} \begin{array}{@{}l@{\quad}l@{}}
      s_f \cdot \left( \lambda_s s_f \right)^{\frac{s_f}{\lambda_f-\lambda_s}} \cdot & \\
      \quad \left( \left[ \lambda_f-\lambda_s \right]s + \lambda_s s_f \right)^{\frac{s_f}{\lambda_s-\lambda_f} - 1} & \smash{\raisebox{.7\normalbaselineskip}{$s<s_f$}}  \\
      \frac{1}{\lambda_f} \cdot \exp{\frac{s_f-s}{\lambda_f}} \cdot \left(\frac{\lambda_f}{\lambda_s}\right)^{\frac{s_f}{\lambda_s-\lambda_f}} & s\geq s_f \\
    \end{array} \right. .
\end{equation}

\begin{figure}
\includegraphics[width=\linewidth]{figures/stopping-probability-linear.jpeg}
\caption{\label{fig:f-and-h-lin} Normalized probability density function (blue) and probability of stopping (orange, dashed) from Eqs.~(\ref{Eq:haz_lin}) and (\ref{Eq:f_lin}) with $\lambda_s=0.5$, $\lambda_f=0.025$ and $s_f=0.5$. Here we have 200 stopping loops over the entire interval, and the mean free path is $\lambda=0.257\,L_a$.}
\end{figure}

% \begin{figure}
%     \centering

%     \captionsetup{width=0.98\linewidth}
%      \includegraphics[width=\linewidth]{figures/actual-figures/stopping-probability-linear.png}
%     \caption{Normalized probability density function (blue) and probability of stopping (orange, dashed) from Eqs.~(\ref{Eq:haz_lin}) and (\ref{Eq:f_lin}) with $\lambda_s=0.5$, $\lambda_f=0.025$ and $s_f=0.5$. Here we have 200 stopping loops over the entire interval, and the mean free path is $\lambda=0.257\,L_a$.}
%     \label{fig:f-and-h-lin}
% \end{figure}

The probability density function and hazard function for this linearly varying absorber are plotted in Fig.~\ref{fig:f-and-h-lin} for $\lambda_s=0.5$, $\lambda_f=0.025$ and $s_f=0.5$.  
%???I don't understand why the density is largest at the beginning???
In contrast to the previous scheme, many particles are stopped right at the beginning of the absorbing region (small, non-zero probability of stopping but large number of particles), and the mean free path of the electrons is $\lambda=0.257\,L_a$ instead of $0.5\,L_a$ for this case.  The overall length of the absorbing region and individual parameters can be adjusted to give the desired profile, but care should be taken so that very few electrons ever reach the right boundary.  This linearly varying scheme is much more flexible and reliable than the previous scheme.  By stopping based on absolute velocity magnitude, particles are stopped whether traveling forward, backward or transversely to the absorbing region.  Specific results will be shown and discussed in Sec.~\ref{sec:results}.

\subsection{Appropriate mean-free-path length} \label{sec:concept-mfp}

We wish to estimate an appropriate mean free path to use with the absorbing boundary before running the simulation.  To do so we will first make an argument based solely on relevant simulation parameters, then make a second based on the strength of the stopping electric field.

First, typical two-dimensional simulations of this type have square cell sizes of length $0.2\,c/\omega_0$, where $\omega_0$ is the laser frequency, with an accompanying time step of $\Delta t \lesssim 0.141\, \omega_0^{-1}$.  For simulations such as ours the overdense plasma density is $n=10n_c$, where $n_c$ is the critical density.  If a minimum of 50 stopping loops is desired over a mean free path (100 loops over the absorber length), this sets a minimum value of $\lambda \approx 22\,c/\omega_p$.  However, performing the stopping routine at every time step can be computationally expensive; thus stopping every $m$ time steps would increase the minimum mean free path to $\lambda \approx 22m\,c/\omega_p$.

% Example that I use
% 62.832 c/omega_0 = 2 mfp = 198.7 c/omega_p
% mfp = 5 lambda_0 = 15.8 lambda_p
% 50 stopping loops over a mean free path, minimum of 7.05 c/omega_0 = 22.3 c/omega_p = 1.12 lambda_0 = 3.55 lambda_p
% The mean free path should be at least a few skin depths of the full density plasma.  Let's say our full density is $10n_c$.
% Right now we have $dx = 0.2\, c/\omega_0 \approx 0.63\, c/\omega_p$.  A good absorber to use has mean free path of about $100\, c/\omega_p \approx 16\,\lambda_p$.  I should stop every time step and reduce the size of the absorbing region to see what is a good value.

% Now we make an argument about the current density that's necessary to go backward.  The ratio $n_b/n_0 \approx 0.04$, and thus the returning current has $v_{\mathrm{th}} \approx 0.04c$.  The Debye length is then $\lambda_D \approx 0.13\,c/\omega_p$.

%???Not sure about this section. You seem to be trying to figure out how thick to make the absorber. What is more important is the how the absorber spreads out the excess charge so the return current can come from a larger region. Let's discuss. ???
% Now we make an argument about the electric field strength necessary to stop an energetic beam and accelerate the background plasma within a Debye sphere.  To begin, we calculate the energy necessary to stop an energetic beam of electrons (charge $q$, mass $m$ and Lorentz factor $\gamma$) as
% \begin{equation} \label{Eq:e-stop}
% qE_b\lambda = (\gamma-1)m c^2,
% \end{equation}
% where $E_b$ is the electric field that stops the particles over a mean free path $\lambda$.  Next, we argue that over a Debye length, $\lambda_D$, the same electric field generated to stop the electron beam should not accelerate background electrons significantly past the thermal energy.  For a background with thermal velocity $v_\mathrm{th}$, this comes out to
% \begin{equation} \label{Eq:e-thermal}
% qE_\mathrm{th}\lambda_D = \frac{1}{2}mv_\mathrm{th}^2,
% \end{equation}
% where $E_\mathrm{th}$ is the accelerating electric field for the background plasma.  Solving for the electric fields and imposing the condition $E_b \lesssim E_\mathrm{th}$, we obtain
% \begin{equation} \label{Eq:mfp}
% \lambda \gtrsim \frac{2(\gamma-1)c/\omega_p}{v_\mathrm{th}/c}.
% \end{equation}
% For many of our relevant problems, the hot electron beam has temperature $\sim3$~MeV and the background electron population heats up to $v_\mathrm{th} \approx 0.15\,c$.  With those parameters we have $\lambda \approx 80\,c/\omega_p$.

The physics arguments that set the scale of $\lambda$ can be understood as follows. Collisionless plasmas attempt to remain current neutral. If a net current exists at some instant in time and space, then electric fields are induced on the electron-frequency time scale to neutralize the current. However, this process is more complicated at a boundary. Consider an absorbing boundary; as electrons leave, a net charge builds up, generating a repelling electric field. There is no background charge outside the boundary that can move in to cancel this field. As a result the electric field accelerates background electrons just inside the boundary, growing until it can accelerate electrons backward at near the speed of light. A thermal bath boundary condition results in a very similar situation. Once an electron leaves, another is remitted from a specified distribution function. The probability is chosen based on balancing the flux of a thermal plasma. Therefore, if the electrons leaving come from a hot tail, the remitted electrons cannot properly cancel this current.  A large electric field and potential build up, generating a reflux of electrons with relativistic energies.  However, if we instead stop the electrons over a distance along which the plasma can naturally generate a return current, then the physics will most closely resemble reality. The thickness must then be several skin depths wide, i.e., $\lambda \gtrsim \bigO(10\,c/\omega_p)$, as a skin depth sets the scale over which the current from relativistic electrons is neutralized. This is on the same order of magnitude as what was estimated above for numerical reasons.

Both the numerical and physical arguments predict a mean free path of about the same order, and in Sec.~\ref{sec:results} we explore the performance of absorbers of various size.

\subsection{Particle splitting and recombination}

The absorber is required when simulating overdense laser-plasma interactions because there is a large flux of relativistic electrons moving into the plasma. For such situations there is another issue that must be addressed simultaneously. Due to the relatively large charge (they represent many real electrons) of macroparticles in PIC simulations, these hot particles can generate artificially large wakes, causing them to slow down as they propagate through the plasma.\cite{Tonge2009ALasers,May2014}  These wakes can also lead to larger levels of turbulence. To avoid the macroparticle-stopping issue, we also run with the particle splitting algorithm discussed in Ref.~\onlinecite{May2014}.  This involves defining an energy threshold above which energetic electrons will be split into smaller particles to avoid significant energy loss.  Split particles are given a small boost in momentum space (typically 1\%) so that they don't travel along identical trajectories.

However, we anticipate that these numerous small particles will eventually be stopped somewhere in the absorbing region.  This can lead to a load imbalance in which a disproportionately large number of particles are distributed over a small number of processors.  Since energy conservation is already slightly violated by the stopping of fast particles, this recombination is done simply by averaging the momentum of two particles at a time that lie in the same octant of momentum space.  We only look to recombine such particles in the region where particle stopping occurs, and only particles with a charge less than the charge of an unsplit particle are considered for recombination.

\end{document}