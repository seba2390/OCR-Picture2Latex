\documentclass[../absorber.tex]{subfiles}
\begin{document}


The motivation for this work is to efficiently model the interaction of a high-intensity laser with the surface of an overdense plasma. This interaction generates copious amounts of relativistic electrons that propagate forward deep into the target. The forward-going electrons lead to a return current of electrons that then interacts with the laser.  When investigating how a laser is absorbed into relativistic electrons, ideally only the region of interest need be simulated; such a region might be an underdense or vacuum region in front of the target or a location some distance into the target itself.  However, this presumes that the spectrum of electrons---including its currents and heat flux---are the same as if the entire plasma region was simulated.  This may not be the case if the boundary conditions at the edge of the simulation do not properly represent the actual conditions.

The preference of only modeling a small region of a larger plasma is more important for higher target densities, which require smaller time steps and cell sizes (compared to the laser period and wavelength) due to increased plasma frequency and decreased scales of physical interest.  Even in simulations that include a larger transport region for the electrons, it may still not be feasible to simulate the entirety of a physical target---for instance, a millimeter-scale solid-density target. For such simulations to be reliable, it is of course necessary that they not be affected by the choice of the simulation boundary location and the associated boundary conditions. Therefore, it is critical to find a boundary condition that mimics the effects of a quiescent plasma of unbounded depth in the direction opposite from laser incidence.  Such a boundary condition is useful even for targets not fully described by an infinite thickness, in that the physics may be understood at least to some approximation as a superposition of unbounded and finite-thickness effects.

Depending on the laser pulse length and/or the target size, it is sometimes possible to simulate an infinitely deep quiescent plasma by expanding the simulated plasma to distances causally separated from the interaction---or perhaps half that far, so signals (moving near the speed of light) cannot reach the boundary and return.  Such simulations allow us to determine the ``correct'' physics, against which we will compare our results.  However, in practice these simulations are usually impractical.
The required plasma thickness also scales with pulse length, so simulation computer time scales quadratically with the pulse length for one- and two-dimensional periodic cases. For finite-width cases, the transverse size may also need to be extended depending on the pulse length.
% even for a linear target (i.e., one-dimensional or transversely periodic) the computational complexity is quadratic in the simulation duration, with even greater complexity for finite-width interactions
Therefore, a compact target that reproduces the behavior of a larger or infinite target is desirable.

To make clear the need for an appropriate boundary condition, we mention two spurious effects that can occur in simulations of a truncated target. First, we observe that the laser-matter interaction continuously deposits energy into the plasma. Although the details may be complex, we can assume that the energy will somehow diffuse or dissipate deep into the target; if this is not possible (e.g., inhibited by a boundary), the target will heat artificially. As the absorption of the laser has been postulated to be highly dependent on the target temperature in some scenarios,\cite{May2011MechanismInterface} this heating can feed back into the absorption itself and greatly affect the overall simulation.

The second effect of an improper boundary is the modification of the plasma distribution function long before any heating of the bulk electrons through an effect known as electron refluxing.\cite{Sentoku2003HighTarget,Quinn2011RefluxingPulses}  Electrons accelerated by a laser generally make up a super-thermal tail in the distribution function, extending to energies far greater than the background thermal energy.\cite{Wilks1992AbsorptionPulses} These high-energy electrons have low collisionality (even in solid-density targets\cite{Nilson2009BulkInteractions}) and may thus travel nearly ballistically through the material in the absence of strong fields. At a given transverse plane in the plasma, the bulk plasma exhibits a small backward drift to provide a return current. However, at the plasma boundary the return current formation can be more complicated. For finite-thickness targets with a vacuum region on the far side (both in the lab and in simulations), the first electrons to reach the back of the material exit into a vacuum and continue along a ballistic path. However, an electrostatic field can quickly build at the target rear surface; no such field grows inside the target due to the high conductivity of the background plasma.  This decelerating field grows in time, eventually causing energetic electrons to be reflected back into the target.  We refer to the re-injection of electrons from this electrostatic field as electron refluxing.  After this reflection, the electrons again travel ballistically though the target, where they can reach another vacuum boundary region (sides) and go through another reflection; alternatively they can re-enter the laser-plasma interaction region, in which case their large kinetic energy modifies how they interact with the laser, perhaps significantly.

The reflecting electrostatic field responsible for electron refluxing in finite-thickness targets arises due to the adjacent vacuum region, where escaping electrons leave behind a net charge that resides on the surface. We have found that standard PIC particle boundary conditions responding to high-energy particle beams actually exhibit reflectivity similar to a vacuum boundary.  In particular, so-called ``reflecting,'' ``absorbing,'' and ``thermal re-emitting'' boundaries all lead to refluxing particles early in the beam interaction; in each case a strong electric field builds up at the boundary.  A specular reflecting boundary condition---where the sign of the momentum perpendicular to the boundary wall is reversed---clearly leads to refluxing; however, no electrostatic field is developed at the boundary since the boundary itself reflects the particles.

An absorbing particle boundary, in somewhat simplified terms, simply removes exiting particles (and their corresponding current) from the simulation space.  However, electromagnetic PIC codes like \textsc{Osiris} advance the fields forward in time via Faraday's and Ampere's laws while depositing current such that the continuity equation is rigorously satisfied. Therefore, when a particle's current disappears, its charge is in fact frozen at its last location.  In the case of an absorbing boundary, an exiting electron beam will cause a static charge buildup.
The bulk of the plasma near the boundary attempts to shield out the boundary field within a few Debye lengths.  However, the field continues to build as more current crosses the boundary and can eventually become large enough to accelerate background electrons backward at relativistic energies, leading to a hot tail of refluxing electrons.
% The remaining plasma electrons move away from the wall to prevent the field from penetrating into the bulk, but the ions are slow to respond due to their high mass-to-charge ratio. The electrostatic field can thus be renewed quickly enough by the few escaping beam electrons that it persists and causes refluxing.
Vacuum boundaries behave similarly, where ions are slowly driven off the target via what is known as target normal sheath acceleration\cite{Wilks2001EnergeticInteractions,Mora2003PlasmaVacuum} (TNSA); again, electrons reach the rear edge in greater numbers than the ions which are able to escape.

A thermal bath particle boundary---where particles are re-emitted into the simulation space with momentum sampled from a specified thermal distribution---may at the outset seem to remedy the issues caused by reflecting and absorbing boundary conditions. We find, however, that a thermal boundary fails to reduce the artificial refluxing. First of all, the correct bath temperature is somewhat ambiguous, and we observe that an incorrect choice leads to clearly incorrect behavior. A bath that is too hot will artificially heat the background electrons in the target; one that is too cold re-emits the particles with too little thermal velocity to diffuse back into the box, manifesting errors similar to those of an absorbing boundary. A drifting Maxwellian moving back into the box (attempting to maintain current neutrality) could be used in place of the stationary Maxwellian generated by the thermal bath. However, determining the proper drift and thermal velocity of this modified Maxwellian is difficult. Furthermore, if a ``correct'' bath temperature and drift velocity were known from a causally separated simulation, it would be a function of time.  Although there are ways to measure effective temperatures (as discussed in Appendix~\ref{app:temp}), the distribution function is not necessarily well-described by single temperatures; thus it is unclear what one should measure nor how to specify the bath temperature.

%%%WARREN WILL READ CAREFULLY LATER???
% I think I'm just going to take this out.  I don't think it adds much.
% Arguably, the largest issue with a thermal bath boundary, which to some extent it shares with the other boundaries, is that it imposes too strict a limitation on the distribution at the surface. To argue this, we assert that the total plasma distribution should be current neutral: contrary would imply a buildup of charge at some point (specifically, the laser-interaction region), which the plasma conductivity would then short out. Alternatively, without return current the electrons would quickly be stripped off of the plasma surface and the beam would stop being created. Similarly, if the forward electron current of the beam were not almost entirely canceled it would greatly exceed the Alfv\'{e}n current~\cite{Alfven1939OnSpace, Davies2004AlfvenIgnition, Davies2006TheInteractions}, which is the limiting current that can be carried by electrons at any scale and in any geometry. Hence, since the net current through the boundary should be zero and the high-energy beam carries a forward electron current out of the box, the background plasma should at least be a drifting Maxwellian moving back into the box, rather than the stationary Maxwellian generated by a thermal bath.  However, determining the proper drift and thermal velocity of this modified Maxwellian is difficult if not impossible.
% Has anyone tried this?  
%%%???Not yet to my knowledge. I wanted Roman to try this for SRS where there is a hot electron flux???
%Would a drifting Maxwellian solve the problem?  Or is it too hard to figure out what the proper thermal/drift velocities are?

The extended boundary condition we describe in this paper is designed with the goal of self-consistently generating a neutralizing, drifting background distribution that imitates an infinite plasma as closely as possible.

\end{document}