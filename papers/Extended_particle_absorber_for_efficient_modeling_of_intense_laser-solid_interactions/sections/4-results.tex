\documentclass[../absorber.tex]{subfiles}
\begin{document}

To demonstrate the effectiveness of the absorber region, we present results from a variety of 2-dimensional simulations of a laser incident on an overdense plasma.  Simulations were done using \textsc{Osiris},\cite{Fonseca2002} where an absorbing region has been implemented.

\subsection{Simulation setup}
In the simulations, an intense 1-$\mu$m plane-wave laser with normalized amplitude $a_0=3$ and 3~ps in duration (2.9-ps flat envelope with 0.13-ps rise and fall ramps) is incident on uniform plasma with density $n=10n_c$ (where $n_c$ is the critical density).  The exponential ramp has a scale length of 3~$\mu$m and begins at $x=-27.6$~$\mu$m.  The critical density is then located at $x_c=-6.9$~$\mu$m, and we define time $t=0$ to be when the leading edge of the laser pulse would arrive at $x_c$ if traveling at speed $c$. The laser is focused to the critical surface and is launched from the left wall.  The plasma skin depth is $c/\omega_p = 50.3$~nm and $c/\omega_0 = 159.2$~nm.  See Fig.~\ref{fig:laser}(a) for a schematic.

The simulations used periodic boundary conditions in the second dimension ($y$), and the laser was polarized with its electric field in the simulation plane (p-polarized). The simulation dimensions were kept constant in the $y$-direction,  3.2~$\mu$m, and in the $x$-direction were either 923.9 or 1597.8~$\mu$m for truncated and causally separated runs, respectively.  Square cells of size 0.2~$c/\omega_0$ were used, resulting in a simulation domain of 50197$\times$100 cells for the simulation with the largest length in $x$ (29025$\times$100 cells otherwise).  The time step was 0.141~$\omega_0^{-1}$.  The electron (ion) species had 64 (16) particles per cell, and each species used cubic interpolation with an initial temperature of 0.1~keV.  We employed a static load balancing routine\cite{Fonseca2013} at initialization to distribute processing elements in an optimal configuration, and the particle push time was delayed until the laser neared the plasma.

\begin{figure}
\includegraphics[width=\linewidth]{figures/fig-1-schematic.jpeg}
\caption{\label{fig:laser} (a)~Simulation schematic, showing the full box size. The box is truncated at 150~$\mu$m when the absorber is in use. (b)~Laser Poynting flux incident at the plasma critical interface, reflected Poynting flux measured 387~$\mu$m to the left of the critical interface and forward electron energy flux measured over the diagnostic region.  All quantities are synced up in time for better visualization.  Percentages represent integrated energy flux as a fraction of the total incident energy.}
\end{figure}

% \begin{figure}
%     \centering
%     \captionsetup{width=0.98\linewidth}
%      \includegraphics[width=\linewidth]{figures/actual-figures/fig-1-schematic.png}
%     \caption{(a)~Simulation schematic, showing the full box size. The box is truncated at 150~$\mu$m when the absorber is in use. (b)~Laser Poynting flux incident at the plasma critical interface, reflected Poynting flux measured 387~$\mu$m to the left of the critical interface and electron energy flux measured over the diagnostic region.  All quantities are synced up in time for better visualization.  Percentages represent integrated energy flux as a fraction of the total incident energy.}
%     \label{fig:laser}
% \end{figure}

In Fig.~\ref{fig:laser}(b) we show the temporal laser profile, as well as the reflected Poynting flux and
%???how do you define the reflected Poynting flux???
transmitted particle energy flux.  The reflected Poynting flux is calculated by measuring the total Poynting flux 380~$\mu$m before the critical-density interface, then subtracting the known incident laser flux.  Both the Poynting and energy fluxes plotted in Fig.~\ref{fig:laser}(b) are translated in time to line up with the incident laser light.  To diagnose the forward momentum and energy flux deep in the plasma, we choose a diagnostic region 48--64~$\mu$m into the uniform plasma over which we average the particle data in space.  The energy flux is defined as $\int (\gamma-1)m_ec^2 \mathbf{p}/\gamma\,d\mathbf{p}$ for electron mass $m_e$.  In order to avoid particle refluxing from either boundary in the $x$-direction, a 746-$\mu$m vacuum region (computationally inexpensive because of the static load balancing) is placed to the left of the plasma upramp, and the uniform-density plasma is extended to the right a distance of 824~$\mu$m (computationally expensive).  This ensures that any particles reflected from the right boundary region will be causally separated from the diagnostic region for the duration of the simulation (for a time $2\times 760\,\mu$m$/c\approx5$~ps).  The $p_x$-$x$ phasespace is shown in Fig.~\ref{fig:px-x-a} at 3.7~ps after the laser was incident on the critical interface, with the diagnostic region marked by dashed lines.  Note the large size of the plasma required compared to the diagnostic region location, along with the very hot return current reflecting off the right simulation boundary---even though a thermal particle boundary is being used.

\begin{figure}
\includegraphics[width=\linewidth]{figures/fig-2-p1x1-a.jpeg}
\caption{\label{fig:px-x-a} The $p_x$-$x$ phasespace for the causally separated simulation (single run, not averaged).  The dashed lines indicate the diagnostic region, but the plasma has to be much larger in length to be causally separated from the hot return current reflecting off the right boundary.}
\end{figure}

% \begin{figure}
%     \centering
%     \captionsetup{width=0.98\linewidth}
%      \includegraphics[width=\linewidth]{figures/actual-figures/fig-2-p1x1-a.png}
%     \caption{The $p_x$-$x$ phasespace for the causally separated simulation (single run, not averaged).  The dashed lines indicate the diagnostic region, but the plasma has to be much larger in length to be causally separated from the hot return current reflecting off the right boundary.}
%     \label{fig:px-x-a}
% \end{figure}

We ran the simulations until 2~ps after the laser had finished hitting the plasma.  In all cases, hot particles were split into two after reaching a $\gamma$ of 1.4, 1.5, 1.6, 1.7, and 1.8 (i.e., very energetic particles were eventually split into 32 smaller particles); the splitting routine was executed every 10 time steps.  Contact the corresponding author for information about the source code and input files used for these simulations.

The particle acceleration mechanisms in these types of simulations are stochastic; therefore, we expect and indeed do observe large differences in particle statistics due to slightly different simulation configurations.  For example, we performed the causally separated simulation three different times with varied random number seeds and observed a factor of 2--4 variation in particle number in the tail of the momentum distribution over the diagnostic region.  For this reason we performed the simulations presented in this paper three times with different random number seeds.  Unless otherwise noted, visualizations presented here are of data averaged over three different runs; this averaging gives increased confidence that any observed deviations from the causally separated run are due to the particle boundary conditions.


\subsection{Effect of the absorber boundary condition}

To greatly reduce computation time and resources, we desire to shrink the simulation region shown in Fig.~\ref{fig:px-x-a}, but preserve the behavior from the causally separated run.  We truncate the plasma at a distance of 150~$\mu$m (29025 cells in $x$) and vary the length of the absorber, where each absorber is designed to stop all hot particles 5~$\mu$m short of the right boundary.  For all results shown here we use the linearly varying absorber from Sec.~\ref{sec:linear} and calculate the local temperature via Eq.~(\ref{Eq:lin-int}).  We quote the mean free path for each absorber, which as shown in Fig.~\ref{fig:f-and-h-lin} is 26\% of the entire absorber length.  We used an energy threshold of 6 times the local thermal velocity and re-emitted stopped particles at the local temperature.  Stopping was performed every time step for both electrons and ions
%???Why the ions??? Explain that ions had large energy
to give a large number of stopping loops for a fast particle traversing the absorbing region.  Particles are typically stopped every $\sim3$ time steps, but we perform a stopping loop every time step to more accurately assess the different methods.  Though it is much more important to use an absorber for electrons than for ions, we observed a sufficient number of hot ions reaching the thermal boundary to warrant stopping ions as well.  Stopping loops were delayed until hot particles approached the absorber region.  Particle recombination (for electrons) was executed every 5 time steps over the absorbing region; this dramatically reduces the simulation runtime as hot particles that have been split into 32 smaller particles are all stopped over a very short distance.

\begin{figure}
\includegraphics[width=\linewidth]{figures/fig-3-p1x1-all-123.jpeg}
\caption{\label{fig:px-x-1.1} The $p_x$-$x$ phasespace (single runs, not averaged) for the causally separated ($\lambda=\infty$), absorber, and no absorber ($\lambda=0$) simulations 1.1~ps after the incident laser.  A hot reflux of electrons is already shown to be entering the dashed diagnostic region for the truncated run with no absorber.}
\end{figure}

% \begin{figure}
%     \centering
%     \captionsetup{width=0.98\linewidth}
%      \includegraphics[width=\linewidth]{figures/actual-figures/fig-3-p1x1-all-123.png}
%     \caption{The $p_x$-$x$ phasespace (single runs, not averaged) for the causally separated ($\lambda=\infty$), absorber, and no absorber ($\lambda=0$) simulations 1.1~ps after the incident laser.  A hot reflux of electrons is already shown to be entering the dashed diagnostic region for the truncated run with no absorber.}
%     \label{fig:px-x-1.1}
% \end{figure}

The $p_x$-$x$ phasespaces for the causally separated ($\lambda=\infty$, where we are zooming in on a particular region), absorber (with $\lambda=100\,c/\omega_p$) and no absorber/truncated ($\lambda=0$) simulations are shown in Figs.~\ref{fig:px-x-1.1} and \ref{fig:px-x-3.7} at two different times.  After just 1.1~ps, a hot reflux of electrons is visible in the truncated run [see Fig.~\ref{fig:px-x-1.1}(c)] that has already entered the diagnostic region.  These refluxing electrons are seen to completely overwhelm the simulation late in time [see Fig.~\ref{fig:px-x-3.7}(c)], while the simulation with the absorber [see Fig.~\ref{fig:px-x-3.7}(b)] is able to maintain an appropriate return current.  These plots are not averaged over three simulations, so sizeable variations within the pre-plasma are expected for the causally separated run due to differences in random number initialization with a different box size [note that the phasespace in the density upramp and surrounding region are identical in Figs.~\ref{fig:px-x-1.1}(b) and (c)].

\begin{figure}
\includegraphics[width=\linewidth]{figures/fig-3-p1x1-all-210.jpeg}
\caption{\label{fig:px-x-3.7} The $p_x$-$x$ phasespace (single runs, not averaged) for the causally separated ($\lambda=\infty$), absorber, and no absorber ($\lambda=0$) simulations 3.7~ps after the incident laser.  The refluxing electrons for the truncated run have completely altered the particle phasespace; the returning hot electrons cyclically interact with the laser and re-enter the plasma, artificially heating the bulk plasma to a much higher temperature than in the casually separated or absorbing runs.}
\end{figure}

% \begin{figure}
%     \centering
%     \captionsetup{width=0.98\linewidth}
%      \includegraphics[width=\linewidth]{figures/actual-figures/fig-3-p1x1-all-210.png}
%     \caption{The $p_x$-$x$ phasespace (single runs, not averaged) for the causally separated ($\lambda=\infty$), absorber, and no absorber ($\lambda=0$) simulations 3.7~ps after the incident laser.  The refluxing electrons for the truncated run have completely altered the particle phasespace; the returning hot electrons cyclically interact with the laser and re-enter the plasma, artificially heating the bulk plasma to a much higher temperature than in the casually separated or absorbing runs.}
%     \label{fig:px-x-3.7}
% \end{figure}

To better visualize temporal behavior, we plot the electron energy flux in the $x$ direction as a function of time and space for the causally separated, absorber, and no absorber simulations in Fig.~\ref{fig:s1-t}.  For the causally separated simulation, a steady stream of energy flux is observed to the right of the critical-density interface, which is slowly pushed forward in time.  Energetic electrons are also seen to escape to the left as the plasma expands.  This expansion is enhanced after the laser turns off.  When using the absorber with $\lambda=100\,c/\omega_p$, the energy flux looks qualitatively very similar to the causally separated run, except that the energy flux quickly decreases to zero in the absorber region.  In contrast, the truncated simulation ($\lambda=0$) shows that a large fraction of the forward energy flux is reflected from the right boundary (especially visible at 0.8~ps), so much so that it dramatically reduces the overall energy flux as it travels backward.  Once the first reflux arrives back to the laser-plasma interface at around 1.5~ps, the forward energy flux is then permanently altered.  This change in physics, as the hot return current interacts with and is accelerated by the laser, is the primary issue that the absorber is able to eliminate.  Finally, this hot reflux of electrons is also visible in the blue negative energy flux after the laser turns off in the truncated run.

\begin{figure}[htp]
\includegraphics[width=\linewidth]{figures/fig-4-s1-t.jpeg}
\caption{\label{fig:s1-t} Forward particle energy flux as a function of position and time for three different cases.  For the truncated simulation ($\lambda=0$), the forward energy flux can be seen to be neutralized by a refluxing current emitted from the boundary.  The absorber effectively reduces the particle energy flux before the simulation boundary without a reflux current.}
\end{figure}

% \begin{figure}
%     \centering
%     \captionsetup{width=0.98\linewidth}
%      \includegraphics[width=\linewidth]{figures/actual-figures/fig-4-s1-t.png}
%     \caption{Forward particle energy flux as a function of position and time for three different cases.  For the truncated simulation ($\lambda=0$), the forward energy flux can be seen to be neutralized by a refluxing current emitted from the boundary.  The absorber effectively reduces the particle energy flux before the simulation boundary without a reflux current.}
%     \label{fig:s1-t}
% \end{figure}

We also examine energy conservation (fields plus particles) across the simulation region when the absorber boundary condition is in use.  To do this we compute the integral of energy density over a specific domain ($V$) and add the energy flux through the left and right boundaries of that domain ($\partial V$):
\begin{equation} \label{Eq:energy}
    \int_V U\,dV + \oint_{\partial V} \mathbf{S} \cdot d\mathbf{A},
\end{equation}
where $U$ is the energy density [$E^2/8\pi + B^2/8\pi + \sum (\gamma-1)m_e c^2$] and $\mathbf{S}$ is the energy flux [$\mathbf{E}\times \mathbf{B}/4\pi + \sum (\gamma-1)m_ec^2 \mathbf{p}/\gamma$].  We compute a running sum of this value over the simulation time (which should remain at zero) and then divide by the maximum energy present in the simulation box at any given time.  This gives a good measure of the energy conservation of the code, although it is not perfect since we only use data reported every 401 time steps (0.3~ps).  In Fig.~\ref{fig:energy} we plot Eq.~(\ref{Eq:energy}) as a function of time, where the right-hand side of volume $V$ (i.e., the location of $\partial V$ on the right) is given by the $x$ coordinate displayed for an absorber with mean free path $\lambda=100\,c/\omega_p$.  We can see that to the left of the absorber (dashed line), the deviation in the coarsely computed energy conservation is less than 1.4\%.  However, by including the absorber region we see that a large fraction of the energy is steadily removed as energetic particles are stopped.  Once again, it is this extended slowing of the particle beam that allows for an appropriate return current to develop, causing plasma to return back into the main simulation region.

\begin{figure}
\includegraphics[width=\linewidth]{figures/fig-5-energy-conservation-edited.jpeg}
\caption{\label{fig:energy} The scaled deviation in energy conservation [see Eq.~(\ref{Eq:energy})] as a function of time, including all points to the left of a given $x$ value (single run, not averaged).  To the left of the absorbing region, energy is well conserved ($<$1.4\% error), but in the absorbing region energy is steadily removed as particles are stopped.}
\end{figure}

% \begin{figure}
%     \centering
%     \captionsetup{width=0.98\linewidth}
%      \includegraphics[width=\linewidth]{figures/actual-figures/fig-5-energy-conservation-edited.png}
%     \caption{The scaled deviation in energy conservation [see Eq.~(\ref{Eq:energy})] as a function of time, including all points to the left of a given $x$ value (single run, not averaged).  To the left of the absorbing region, energy is well conserved ($<$1.4\% error), but in the absorbing region energy is steadily removed as particles are stopped.}
%     \label{fig:energy}
% \end{figure}

\subsection{Variation of absorber parameters}

As mentioned in Sec.~\ref{sec:absorber} and Appendix~\ref{app:temp}, there are a variety of options for implementing the absorber region. 
%??? We should be clearer???
When determining the energy threshold and re-emission temperature of the stopped particles, we can calculate the background temperature dynamically by weighting the distribution function with the proper velocity to some power,
%or its fourth root 
%to determine the energy threshold and re-emission temperature of stopped particles
or we can simply specify a constant value to use. Using a lower power (such as the fourth root) for the proper velocity will emphasize the bulk over a hot tail; more details are given in Appendix~\ref{app:temp}.  However, using the fourth-root temperature never improved the absorber performance for the simulations shown here, so we calculate the temperature in each cell as given by Eq.~(\ref{Eq:lin-int}).

We can also use the hazard function probability defined in Sec.~\ref{sec:hazard} or the linearly varying probability defined in Sec.~\ref{sec:linear} to stop the particles.  In our tests these two choices produce similar results, but overall the linearly varying absorber maintained the proper response for a longer time.  The main reason for this is that due to the periodicity in $y$, simulations using the hazard-function absorber exhibited a large and increasing transverse temperature in the absorbing region; the hazard-function absorber preferentially stops particles with large forward momentum, allowing energetic particles to stream transversely and for some accelerating/reflecting fields to develop (see last paragraph of Sec.~\ref{sec:hazard}).  For this reason we use the linearly varying absorber in this paper, which stops particles as a function of the magnitude of the velocity and not just the longitudinal component.

% Both the hazard and linearly varying absorber schemes appear to efficiently stop particles.  However, in many of the quasi-1D simulations that we performed, the linearly varying absorber was able to maintain the proper response for longer than the hazard function absorber.  This is because the hazard function relies on particles streaming primarily in one direction.  However, with only 100 cells in the transverse direction, many energetic particles that stream towards the boundary are also very hot in the transverse direction.  Both electrons and ions can develop very large transverse momentum in the absorbing region as they drift backward as part of the return current, and that energy will not be removed by the absorber.  The linearly varying absorber relies on the absolute magnitude of a particle's velocity, and hence any transverse momentum in the absorbing region will be kept small.  For this reason we recommend using the linearly varying absorber with a calculated threshold and re-emission temperature.

We compare a combination of absorbers in Fig.~\ref{fig:variation}, where we show the $p_x$ momentum phasespace for all electrons in the diagnostic region at two different times.  Although all absorber schemes appear to perform equally well early in time, the return current is clearly hotter when constant values of the energy threshold and re-emission temperature are given.  For the static temperature simulation, we set the absorber to stop particles with energy greater than 0.6~keV and to re-emit particles at 0.1~keV (the original plasma temperature); in contrast the dynamic absorber stops particles moving at more than 6 times the locally computed thermal velocity.  Using a static temperature performs poorly because, as seen even in the absorbing region of Fig.~\ref{fig:px-x-3.7}(b), the plasma heats up significantly in response to the energetic electron beam.  Particles stopped and re-emitted at the original temperature are not moving fast enough to provide the necessary return current, and a nonphysical potential develops that accelerates electrons backward with too much energy.  Calculating the local temperature instead allows the absorber to accurately compensate for this dynamic behavior.

Although not shown here, we performed a series of simulations varying the mean free path of the absorber by factors of two between $\lambda=0.1\,c/\omega_p$ and $\lambda=200\,c/\omega_p$.  We observed that if the absorber had a mean free path $\lambda \gtrsim 6\,c/\omega_p$, it was able to closely match the causally separated momentum distribution when averaged over three separate runs.  However, individual simulations with $\lambda \lesssim 20\,c/\omega_p$ exhibited slightly greater variability in comparison to the causally separated data.  In our simulations, an absorber with a mean free path of $6\,c/\omega_p$ performed only $\sim30$ stopping events before nearly all particles were stopped, which was sufficient for a laser 3~ps in duration with $a_0=3$.  However, care must be taken for lasers of longer duration or higher intensity; Fig.~\ref{fig:variation} shows that some absorbers can perform well (a)~initially, but (b)~eventually fail due to the large amount of energetic particles striking the absorber.  Thus $\lambda \gtrsim \bigO(10\,c/\omega_p)$ gives a reasonable estimate of the appropriate mean free path, but the absorber length should be verified for each individual simulation.

% In our simulations the hot electron beam exhibited a temperature of 3~MeV, and the background plasma in the absorbing region had an average thermal velocity of 0.15~$c$.  Equation~(\ref{Eq:mfp}) then gives (ignoring errors from relativistic inaccuracies) a predicted mean free path of $\lambda \approx 80\,c/\omega_p$.  Equation~(\ref{Eq:mfp}) thus gives a reasonable estimate of appropriate mean free path and can be used as a guideline for absorber length---especially for long-time laser-plasma interactions---but a much shorter absorber may be similarly effective depending on the simulation.

% In these simulations, the hot electron beam exhibited a temperature of 3~MeV, and the background plasma in the absorbing region had an average thermal velocity of 0.15~$c$, giving a predicted mean free path (ignoring errors from relativistic inaccuracies) of $\lambda \approx 80\,c/\omega_p$.  In Fig.~\ref{fig:variation} we show the $p_x$ phasespace averaged over electrons in the diagnostic region for a wide range of the mean free path both for (a)~a single set of runs performed and (b)~the sets of three runs averaged together.  For these simulations, when averaged together all absorber lengths appear to perform consistently well, though perhaps greater variance is observed for mean free paths shorter than $\sim100\,c/\omega_p$ in Fig.~\ref{fig:variation}(a).  In addition, our tests using longer lasers showed that absorbers with shorter mean free paths failed at earlier than those with longer mean free paths.  Thus, again, the absorber parameters should be verified for each unique simulation, though Eq.~(\ref{Eq:mfp}) can be used as a guideline for the appropriate mean free path length, especially for long-time simulations.

\begin{figure}
\includegraphics[width=\linewidth]{figures/fig-6-avg-p1.jpeg}
\caption{\label{fig:variation} The $p_x$ phasespace for all electrons in the region 48--64~$\mu$m into the constant-density plasma for various schemes (a)~1.5~ps and (b)~3.7~ps after the laser was incident on the plasma.  Though the performance of all shown absorbers is nearly identical early in time, either using a static temperature threshold and re-emission or using a very short absorber gives improper results later in time.}
\end{figure}

% \begin{figure}
%     \centering
%     \captionsetup{width=0.98\linewidth}
%      \includegraphics[width=\linewidth]{figures/actual-figures/fig-6-avg-p1.png}
%     \caption{The $p_x$ phasespace for all electrons in the region 48--64~$\mu$m into the constant-density plasma for various schemes (a)~1.5~ps and (b)~3.7~ps after the laser was incident on the plasma.  Though the performance of all shown absorbers is nearly identical early in time, either using a static temperature threshold and re-emission or using a very short absorber gives improper results later in time.}
%     \label{fig:variation}
% \end{figure}

% In Sec.~\ref{sec:concept-mfp} we discussed the appropriate mean free path to use for the absorber, given by Eq.~(\ref{Eq:mfp}) as a function of the beam energy and thermal velocity of the background plasma.  We performed various simulations using the linearly varying absorber with a dynamically calculated threshold and re-emission temperature and varied the mean free path of the absorber.  In these simulations, the hot electron beam exhibited a temperature of 3~MeV, and the background plasma in the absorbing region had an average thermal velocity of 0.15~$c$, giving a predicted mean free path (ignoring errors from relativistic inaccuracies) of $\lambda \approx 80\,c/\omega_p$.  In Fig.~\ref{fig:variation} we show the $p_x$ phasespace averaged over electrons in the diagnostic region for a wide range of the mean free path both for (a)~a single set of runs performed and (b)~the sets of three runs averaged together.  For these simulations, when averaged together all absorber lengths appear to perform consistently well, though perhaps greater variance is observed for mean free paths shorter than $\sim100\,c/\omega_p$ in Fig.~\ref{fig:variation}(a).  In addition, our tests using longer lasers showed that absorbers with shorter mean free paths failed at earlier than those with longer mean free paths.  Thus, again, the absorber parameters should be verified for each unique simulation, though Eq.~(\ref{Eq:mfp}) can be used as a guideline for the appropriate mean free path length, especially for long-time simulations.

% We show in Fig.~\ref{fig:p1} the $p_x$ momentum phasespace for all electrons in the diagnostic region (48--64~$\mu$m into the uniform-density plasma) at two different simulation times, with and without the absorber.  A causally separated run is also given to show the ideal response.  We see that if a traditional thermal boundary condition is used instead of the absorbing boundary, a stream of hot electrons reflected from the right simulation boundary comes back through the main plasma body at early times.  These hot electrons eventually arrive at the laser-plasma interface, are again heated, and return as a forward current through the plasma.  This cycle results in an artificially hot plasma due to the improper boundary.  However, good agreement with the causally separated plasma is obtained when the absorbing boundary condition is used.

% Another visualization of this same effect is shown in Fig.~\ref{fig:p1x1}, which shows the $p_x$-$x$ phasespace for a broader range of the simulation space at two separate times.  Without the absorber, a very hot return current is visible in Fig.~\ref{fig:p1x1}(e) at early times, which results in a much hotter plasma overall late in time [Fig.~\ref{fig:p1x1}(f)].  Employing the absorber, however, allows the truncated simulation to maintain the low-temperature return current characteristic of the causally separated simulation.  The beginning of the absorbing region is indicated in Fig.~\ref{fig:p1x1}(c)--(d) by a dashed line, past which hot electrons are seen to gradually be cooled until near the simulation boundary.

% \begin{figure}
%     \centering

%     \captionsetup{width=0.98\linewidth}
%     \begin{subfigure}[b]{0.49\linewidth}
%          \centering
%          \includegraphics[width=\linewidth]{figures/p1x1-0-123.png}
%          \label{fig:p1x1-a}
%      \end{subfigure}
%      \hfill
%     \begin{subfigure}[b]{0.49\linewidth}
%          \centering
%          \includegraphics[width=\linewidth]{figures/p1x1-0-210.png}
%          \label{fig:p1x1-b}
%      \end{subfigure}
%     \begin{subfigure}[b]{0.49\linewidth}
%          \centering
%          \includegraphics[width=\linewidth]{figures/p1x1-20-123.png}
%          \label{fig:p1x1-c}
%      \end{subfigure}
%      \hfill
%     \begin{subfigure}[b]{0.49\linewidth}
%          \centering
%          \includegraphics[width=\linewidth]{figures/p1x1-20-210.png}
%          \label{fig:p1x1-d}
%      \end{subfigure}
%     \begin{subfigure}[b]{0.49\linewidth}
%          \centering
%          \includegraphics[width=\linewidth]{figures/p1x1-00-123.png}
%          \label{fig:p1x1-e}
%      \end{subfigure}
%      \hfill
%     \begin{subfigure}[b]{0.49\linewidth}
%          \centering
%          \includegraphics[width=\linewidth]{figures/p1x1-00-210.png}
%          \label{fig:p1x1-f}
%      \end{subfigure}
%     \caption{The $p_x$-$x$ phasespace for electrons in the broader simulation space at two different times for simulations that are (a)--(b) causally separated (actual right simulation boundary extended to 824~$\mu$m), (c)--(d) with the absorber (dashed line shows start of absorbing region) and (e)--(f) without the absorber.  Note that when truncating the simulation space without the absorber, a hot return current is present at early times, which translates to a much hotter overall plasma late in time.}
%     \label{fig:p1x1}
% \end{figure}

% \begin{figure}
%     \centering

%     \captionsetup{width=0.98\linewidth}
%     \begin{subfigure}[b]{0.98\linewidth}
%          \centering
%          \includegraphics[width=\linewidth]{figures/avg-lin-haz-runs-p1-135.png}
%          \label{fig:lin-haz-p1-a}
%      \end{subfigure}
%      \begin{subfigure}[b]{0.98\linewidth}
%          \centering
%          \includegraphics[width=\linewidth]{figures/avg-lin-haz-runs-p1-210.png}
%          \label{fig:lin-haz-p1-b}
%      \end{subfigure}
%     \caption{The $p_x$ phasespace for all electrons in the region 48--64~$\mu$m into the constant-density plasma (a)~1.47~ps and (b)~3.72~ps after the laser was incident on the plasma for the two different absorber schemes.  As for temperature calculation, performance is seen to be significantly worse when constant values are used.}
%     \label{fig:lin-haz-p1}
% \end{figure}

% \begin{figure}
%     \centering

%     \captionsetup{width=0.98\linewidth}
%     \begin{subfigure}[b]{0.98\linewidth}
%          \centering
%          \includegraphics[width=\linewidth]{figures/not-avg-2-lambda-p1-210.png}
%          \label{fig:lambda-p1-a}
%      \end{subfigure}
%      \begin{subfigure}[b]{0.98\linewidth}
%          \centering
%          \includegraphics[width=\linewidth]{figures/avg-lambda-p1-210.png}
%          \label{fig:lambda-p1-b}
%      \end{subfigure}
%     \caption{The $p_x$ phasespace for all electrons in the region 48--64~$\mu$m into the constant-density plasma 3.72~ps after the laser was incident on the plasma as a function of stopping distance for (a)~one particular set of runs and (b)~all runs averaged together.  Some greater deviations from the causally separated spectrum are observed for mean free paths shorter than $\sim100\,c/\omega_p$, though all runs seem to perform about equally when averaged together.}
%     \label{fig:lambda-p1}
% \end{figure}

\subsection{Best practices}

Here we make a few notes on best practices for performing simulations with the extended absorbing boundary condition.  We found it important to also causally separate the vacuum boundary (where the laser is injected) from the laser-plasma interface.  Even with absorbing particle boundary conditions at this vacuum boundary, most energetic particles that reached the vacuum boundary were immediately reflected back into the simulation space.  This is a combined effect of the laser potential at the wall and the electric field buildup from exiting particles (a nonnegligible number of particles are accelerated toward the laser from the pre-plasma region).  Refluxing from the vacuum boundary leads to a modified distribution at the laser-plasma interaction region, which then artificially inflates the forward electron energy flux in the target.

% In addition, for simulations 
% %of a target with finite size in the transverse direction, 
% with a finite width laser the boundary in the transverse directions also need to 
% %we advise extending out the vacuum boundary to be 
% be large enough causally separate itself from the interaction region.
% %that dimension as well.

For simulations with a finite-width laser, absorber regions can also be placed at the transverse simulation edges to correctly handle the large flux of relativistic electrons expelled transversely from the laser spot.  However, the effectiveness of the absorber relies on having a large number of particles in each cell (for calculating the temperature).  If absorbers are placed at the transverse simulation boundaries, they may overlap with near-vacuum regions in and before the pre-plasma.
%???Not sure I understand what you are trying to say????
Thus for finite-size-target simulations with multiple absorbers, we found it is useful to transition the absorbers positioned along the transverse boundaries to stop and re-emit particles based on a static (rather than dynamically calculated) temperature in those near-vacuum regions.

Finally, the start of the absorber region should be located a reasonable distance away from where accurate plasma measurements are expected.  For example, when comparing Figs.~\ref{fig:px-x-3.7}(a) and \ref{fig:px-x-3.7}(b), the phasespace immediately in front of the absorbing region in (b) does not exactly mimic the causally separated phasespace in (a).  Examining the particle phasespace for irregularities near the absorber region can help determine the appropriate distance at which to measure plasma quantities.

\subsection{Future work}

The implementation described here, though effective, is by no means a comprehensive treatment or unique solution to the reflux problem.  Here we list some ideas that could be used to iterate on our proposed solution.  Particles could be re-emitted from a distribution that is hotter in the return direction than in the forward direction, assisting in establishing the appropriate return current.  Particles could be stopped preferentially based on their direction of motion.  We employed absorbers for both ions and electrons in these simulations, but the ion response and stopping could be explored in greater detail for long-time simulations.  Alternatives that are more computationally expensive could include applying a drag force to energetic particles over the length of the entire absorber or calculating the re-emission temperature from a position located before the absorber region.  Last, it may be possible to develop a thermal bath boundary where particles are re-emitted from a distribution determined from a region somewhere inside the plasma.

\end{document}