\documentclass[%
 aip,
% jmp,
% bmf,
% sd,
% rsi,
 amsmath,amssymb,
%preprint,%
 reprint,%
%author-year,%
%author-numerical,%
% Conference Proceedings
% floatfix,
]{revtex4-2}

\usepackage{graphicx}% Include figure files
\usepackage{dcolumn}% Align table columns on decimal point
\usepackage{bm}% bold math
%\usepackage[mathlines]{lineno}% Enable numbering of text and display math
%\linenumbers\relax % Commence numbering lines
\usepackage[utf8]{inputenc}
\usepackage[T1]{fontenc}
\usepackage{mathptmx}
\usepackage{etoolbox}
% \usepackage{caption}
\usepackage{url}            % simple URL typesetting
\usepackage{subfiles}
\usepackage{array}
\newcolumntype{P}[1]{>{\centering\arraybackslash}m{#1}}
\usepackage{lineno}
\usepackage{microtype}
% \biboptions{sort&compress}
\usepackage{amsmath}
\usepackage{amssymb}
\usepackage{mathtools}
% \usepackage{subcaption}
\bibliographystyle{apsrev4-2}

\DeclareRobustCommand{\bigO}{%
  \text{\usefont{OMS}{cmsy}{m}{n}O}%
}

%% Apr 2021: AIP requests that the corresponding 
%% email to be moved after the affiliations
\makeatletter
\def\@email#1#2{%
 \endgroup
 \patchcmd{\titleblock@produce}
  {\frontmatter@RRAPformat}
  {\frontmatter@RRAPformat{\produce@RRAP{*#1\href{mailto:#2}{#2}}}\frontmatter@RRAPformat}
  {}{}
}%
\makeatother
\begin{document}

\preprint{AIP/123-QED}

\title{Extended particle absorber for efficient modeling of intense laser-solid interactions}
% Force line breaks with \\
\author{Kyle G. Miller}
\email{kylemiller@physics.ucla.edu}
\affiliation{Department of Physics and Astronomy, University of California, Los Angeles, California 90095, USA}%Lines break automatically or can be forced with \\

\author{Joshua May}
% \email{joshmay@physics.ucla.edu}
\affiliation{Department of Physics and Astronomy, University of California, Los Angeles, California 90095, USA}

\author{Frederico Fiuza}
% \email{fiuza@slac.stanford.edu}
\affiliation{SLAC National Accelerator Laboratory, Menlo Park, CA 94025, USA}

\author{Warren B. Mori}
% \email{mori@physics.ucla.edu}
\affiliation{Department of Physics and Astronomy, University of California, Los Angeles, California 90095, USA}

\date{\today}% It is always \today, today,
             %  but any date may be explicitly specified

\begin{abstract}
An extended thermal particle boundary condition is devised to more efficiently and accurately model laser-plasma interactions in overdense plasmas.  Particle-in-cell simulations of such interactions require many particles per cell, and a large region of background plasma is often necessary to correctly mimic a semi-infinite plasma and avoid electron refluxing from a truncated plasma. For long-pulse lasers of many picoseconds, such constraints can become prohibitively expensive.  Here, an extended particle boundary condition (absorber) is designed that instantaneously stops and re-emits energetic particles streaming toward the simulation boundary over a defined region, allowing sufficient time and space for a suitably cool return current to develop in the background plasma.  Tunable parameters of the absorber are explained, and simulations using the absorber with a 3-ps laser are shown to accurately reproduce those of a causally separated boundary while requiring only 20\% the number of particles.
\end{abstract}

\maketitle

\section{Introduction} \label{sec:intro}
\subfile{sections/1-introduction}


% Describe simulations and discuss boundary issue
\section{Boundary issues in overdense plasma simulations} \label{sec:overdense}
\subfile{sections/2-sim-issues}


% Also discuss parameters here
\section{Absorbing boundary region} \label{sec:absorber}
\subfile{sections/3-abs-concept}


\section{\textsc{Osiris} simulation results} \label{sec:results}
\subfile{sections/4-results}


\section{Conclusion}
\subfile{sections/5-conclusion} \label{sec:conclusion}

\begin{acknowledgments}
The authors gratefully acknowledge guidance and feedback from A.~J.~Kemp and S.~C.~Wilks.

This work was performed in part under the auspices of the U.S. Department of Energy by Lawrence Livermore National Laboratory under Contract DE-AC52-07NA27344 and funded by the LLNL LDRD program with tracking code 19-SI-002 under Contract B635445. Additional support was given by DOE grant DE-SC0019010 and NSF grant 1806046.
\end{acknowledgments}

\section*{Data Availability Statement}

The data that support the findings of this study are available from the corresponding author upon reasonable request.

\appendix

\section{Computing local temperature} \label{app:temp}
\subfile{sections/app-temperature}

\nocite{*}
\bibliography{references}% Produces the bibliography via BibTeX.

\end{document}
%
% ****** End of file aipsamp.tex ******