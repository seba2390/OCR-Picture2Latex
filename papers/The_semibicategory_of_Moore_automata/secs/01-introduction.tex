\section{Introduction}\label{sec_intro}
Any arrangement of sets and functions $E \xot d A\times E, E\xto s B$
% \[\vxy{E & \ar[l]_-d A\times E, \quad E \ar[r]^-s & B}\label{moore_ex}\]
can be interpreted as follows:
\begin{itemize}
	\item $E$ as a set of `states', and $A,B$ are sets of `inputs' and `outputs';
	\item the function $d : E\times A \to E$ is a \emph{transition function} (specifying a \textbf{d}ynamics), and the function $s : E\to B$ provides an \emph{output} given an initial state (specifying a final \textbf{s}tate).
\end{itemize}
Altogether, $(d,s)$ form a \emph{Moore automaton} or Moore `machine'.

This definition extends to the case where $E,A,B$ are objects of the same monoidal category $(\clK,\otimes)$ \cite{Ehrig}; the situation is of particular interest when $\clK$ is Cartesian or semiCartesian: for example $\clK=(\TTop,\times)$ (and then $d,s$ are continuous, and in particular we can think of $f$ as an action over $E$ of the free topological monoid $A^*$ on $A$, cf. \cite{reutenauer1979topologie}),\footnote{It is worth to remark that here `topological' can --and should-- be intended broadly as follows: let $\clE$ be a Grothendieck topos; we can consider Moore automata in $\clE$, and then $E$ is an object of $\clE^M$, the category of $\clE$-objects with an action of $M=A^*$, the free monoid on $A$. Cf. also \cite{rogers2021toposes} for a thorough study of `toposes of monoid actions' when $\clE=\Set$.} but also $\clK=(\Cat,\times)$ (and then $f,g$ are functors, cf. \cite{noi:bicategories,guitart1974remarques,guitart1978bimodules}), $\clK=\Par$ (the category of sets and partial functions; in this case, $f,g$ are `partially defined'  functions that can fail on certain inputs), or a category of smooth manifolds \cite{spivak2016steady}.

For fixed input and output $A,B$ the structure of the class $\Mre(A,B)$ of Moore automata can be described as the strict pullback
\[\label{as_pb}\vxy{
	\Mre(A,B) \ar[r]^-{V'}\ar[d]_{U'}\xpb & \clK/B\ar[d]^-{V}\\
	\Alg(A\otimes \firstblank) \ar[r]_-{U} & \clK
	}\]
where $U : \Alg(A\otimes \firstblank) \to \clK$ is the forgetful functor from the category of endofunctor algebras for $A\otimes \firstblank : \clK\to \clK$, or --which is equivalent if $\clK$ has countable colimits-- actions of $A^*$, and $V : \clK/B\to \clK$ is the usual `domain' fibration \cite[p. 30]{CLTT}, \cite{loregian2019categorical} from the slice over $B$. Universal constructions are induced from the underlying category $\clK$ (in particular, all colimits and connected limits are created by $V'$, a result we survey in \cite{noi:completeness}).

As clean as it may seem, this characterisation leaves something out of the story; thinking of objects of $\Mre(A,B)$ as \emph{processes} from $A$ to $B$, it is natural to compose them \emph{sequentially}, as follows:% in the fashion represented graphically as
\[
	\begin{tikzpicture}[oriented WD, bbx=1em, bby=1ex, baseline=(current bounding box.center), scale=0.9]
		\node[bb={1}{1}, font=\tiny] (X1) {$(E,f,g)$};
		\node[bb={1}{1}, font=\tiny, right =2 of X1] (X2) {$(E',f',g')$};
		\node[bb={1}{1}, fit={($(X1.north west)+(-1,3)$) ($(X1.south)+(0,-3)$) ($(X2.east)+(1,0)$)}, bb name = ${\color{gray!70}(E\times E',f'\diamond f,g'\diamond g)}$, dotted, gray!70, font=\scriptsize] (Y) {};
		%
		\draw[ar] (Y_in1') to (X1_in1);
		\draw[ar] (X1_out1) to (X2_in1);
		\draw[ar] (X2_out1) to (Y_out1');
		\draw[label]
		node at ($(Y_in1')!.5!(X1_in1)+(0,7pt)$)  {$A$}
		node at ($(X1_out1)!.5!(X2_in1)+(0,7pt)$)   {$B$}
		node at ($(X2_out1)!.5!(Y_out1')+(0,7pt)$)  {$C$};
	\end{tikzpicture}
\]
Given $(E,f,g)\in\Mre(A,B)$ and $(E',f',g')\in\Mre(B,C)$, we can consider the Moore automaton $(E\times E',f'\diamond f,g'\diamond g)\in\Mre(A,C)$, defined as% follows:
\[
	\begin{cases}
		f'\diamond f : E\times E' \times A \to E\times E' : (e,e',a)\mapsto (f(e,a),f'(e',ge)) \\
		g'\diamond g : E\times E'\to C : (e,e')\mapsto g'(e')
	\end{cases}
	\label{moore_compo}\]
This binary composition law is easily seen to be associative up to isomorphism. Now, do the various $\Mre(A,B)$ in \eqref{as_pb} arrange as the hom-categories of a bicategory $ \Mre $ of Moore automata? This would be the best situation to be in.

However, the composition $\firstblank\diamond\firstblank$ looks irredeemably nonunital, as there is no hope of finding a common section $A\mapsto \id_A$ for the domain and codomain endofunctors %$\sum_{A,B}\Mre(A,B)\to \clK$
 sending $(E,f,g)\in\Mre(A,B)$ to $A$ and $B$ respectively. Indeed, to find such identity arrows, one should fix a triple $(U_A,x,y) : U_A\xot x A\times U_A, U_A \xto y A$ such that, at the very least, the compositions $(E\times U_B,\firstblank,\firstblank)$ and $(U_A\times E,\firstblank,\firstblank)$ are isomorphic to $(E,f,g)$ in $\Mre(A,B)$. So in particular, the carrier $U_A$ must be the terminal object / monoidal unit, which is easily seen to not work.

Confronted with this problem, two approaches are possible: one can forget about the composition law in \eqref{moore_compo}, and find one that  underlies a bicategory structure; or, one can forget about the absence of identity 1-cells, and study the structure exhibited by $\Mre$ `as it is'.

Leaning towards the first approach is not recommendable, as the composition law outlined above arises by restricting an existing bicategory structure. If we consider more general spans like
\[\vxy{E & A\times E \ar[r]^g\ar[l]_f& B,}\label{ex_mealy}\]
arranged in the class $\Mly(A,B)$, all the nice properties expounded above remain true, while at the same time addressing the problem: now that the $g$ is sensitive to the input, the identity 1-cell of an object $A$ is given by the span $1\xot ! A\times 1 \xto{\pi_A} A$,
% \[\vxy{1 & A\times 1\ar[r]^g\ar[l]_f & A}\]
given by the two projections. Moreover, this definition `makes sense', because when $\clK=\Set$ (or for that matter, any Cartesian category) the bicategory $\Mly$ so determined is just the category of \emph{Mealy automata}, which is known to be not just a bicategory, but \emph{the} bicategory \cite{Katis1997,ITA_2002__36_2_181_0,openTransitionSystems21} of pseudofunctors and lax natural transformations from $\bfN$ (the monoid of natural numbers) to $(\clK,\otimes)$: a bicategory of `lax dynamical systems', borrowing the analogy from \cite{Tierney1969}.
Our choice of composition of 1-cells is then constrained by the fact that $\firstblank\diamond\firstblank$ in $\Mly$ enjoys a certain universal property, and that in it composing 1-cells restricts to the subcategory of spans like \eqref{ex_mealy} with $g$ mute in the input variable.

We are thus led to follow the second path, and recognise that $\Mre$ is a \emph{semibicategory}: a structure that is precisely like a bicategory, except that it lacks identity 1-cells.

The purpose of this note is to initiate the study of $\Mre$ \emph{qua} semibicategory, embracing its (lack of) identity and trying to uncover its properties, both regarded as an object of the category of semibicategories, and related the `true' bicategory $\Mly$ where it embeds through a canonical functor $J$ (\autoref{tautj}).

In particular, assuming $\clK$ is Cartesian closed, we focus on the precise sense in which the semibicategory $\Mre$ and the bicategory $\Mly$, linked by $J$, fit into adjoint situations, and we provide a generic way of generating such adjunctions; leveraging on the pullback \eqref{as_pb}, and given a similar characterisation of $\Mly(A,B)$ as a strict pullback in $\Cat$, in \autoref{def_mlymre}.\ref{cf_1}, all adjunctions
\[\xyadj{\clK/B}{\Alg(A\times\firstblank)\cong\clK/B^A}{}{}\]
between slice categories pull back to $\Mre(A,B)$ and $\Mly(A,B)$. Now categorical algebra provides `standard' natural ways to obtain such adjunctions, when $\clK$ is rich enough (e.g., it is locally Cartesian closed): pulling back along a morphism or computing direct/inverse images \cite[A1.5.3]{Johnstone2002}, using the 2-functoriality of comma objects in $\Cat$ \cite{kelly_1989}, or the structure induced by the whole functor $\Mre(A,B)\to \clK/B\times\clK/B^A$ obtained from the pullback span \eqref{as_pb}. Each of these constructions yields a different adjunction, each of which is of some interest when translating 1-cells of $\Mre(A,B)$ and $\Mly(A,B)$ into `machines' that accept an input of type $A$ and produce an output of type $B$. In particular, we provide $J$ with a right adjoint, just to discover a moment later that $J=D_0$ is the $0$th term of a countable sequence of functors $D_n$, each of which is a fully faithful left adjoint $D_n\dashv K_n$ (cf. \eqref{di_enne}): this way, we obtain a rather intrinsic description for a classical equivalence theorem between Moore and Mealy machines \cite[3.1.4,3.1.5]{Shallit}, embedding $\Mre(A,B)$ in $\Mly(A,B)$ as a coreflective subcategory; in \autoref{the_univ_moo}, \autoref{moorif_right} we give a very explicit description of the coreflector as post-composition with a universal Moore cell $\fku X$. Adopting a similar point of view in \autoref{decap_right} we prove that a similar adjunction colocalises $\Mly(A,B)$ onto the subcategory of what we call \emph{soft} Moore machines (\autoref{def_acep}); again, the coreflector in \autoref{decap_right} is post-composition with a universal soft automaton $\fkp X$, that can be characterised (\autoref{constr_pinfty}, \autoref{P_is_acep}) as a certain pullback, or as the image of yet another functor (\autoref{some_adjoints_btwn_Mly}) obtained by a change of base/direct image adjoint situation \eqref{triple_for_precomp_funs}. This situation generalises: it is possible to define a whole hierarchy of subcategories $\aMre[n]$ for $n\ge 1$, each of which is the category of $D_nK_n$-coalgebras, thus arranged in a filtered diagram of coreflective inclusions
\[\vxy{
		\aMre\,\, \ar@{^{(}->}[r]& \,\,\aMre[2]\,\, \ar@{^{(}->}[r]& \,\,\aMre[3]\,\, \ar@{^{(}->}[r]& \,\,\dots\,\, \ar@{^{(}->}[r]& \,\,\Mre
	}\]
analogous in some regards to the `level' filtration of a topos \cite{kelly1989complete,menni2019level}, although the localisations in study here are not essential.

Unsurprisingly, a well-known and `solid' procedure like \cite[3.1.4,3.1.5]{Shallit} is just a universal construction in disguise. However, the interest in such a finding is also pedagogical: if hypothetical readers journeyed automata theory in any direction, armed only with bicategory theory, they would be able to rediscover natural constructions on state machines by just enacting what they know about more familiar universal constructions (composition of 1-cells, bicategorical limits \cite{kelly_1989,Street19,bird1989flexible}, especially of of Kan type,\footnote{Loosely speaking, a `bicategorical limit of a Kan type' is a left or right Kan extension, or a left/right Kan lifting, possibly pointwise or absolute.} \cite{street1981conspectus} the fundamental calculus of a Cartesian closed category, base change adjunctions\dots). The importance of such serendipity is hard to overestimate.
