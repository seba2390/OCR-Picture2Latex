\section{Diagrams and Proofs}
\subsection{Diagrams}
\[\label{diag1}\vxy{
	\clS_0\times\clS_0\times\clS_0 \ar[d]_-{\otimes\times\clS_0} \ar[r]^-{\clS_0\times\otimes} \drtwocell<\omit>{\alpha} & \clS_0\times\clS_0 \ar[d]^-\otimes \\
	\clS_0\times\clS_0 \ar[r]_-\otimes & \clS_0
	}\]
% \[\label{diag2}\vxy[@C=-9mm]{
% 		& X\otimes (Y\otimes (Z \otimes W))) \ar@{->}[rr] \ar@{->}[ld] &  & (X\otimes Y)\otimes (Z \otimes W) \ar@{->}[rd] &  \\
% 		X\otimes ((Y\otimes Z)\otimes W) \ar@{->}[rrd] &  &  &  & ((X\otimes Y)\otimes Z)\otimes W \\
% 		&  & (X\otimes (Y\otimes Z))\otimes W \ar@{->}[rru] &  &
% 	}\]
\def\c{\circ}
\[\label{diag3}\vxy[@C=-9mm]{
		& f\c (g\c (h \c k))) \ar@{->}[rr] \ar@{->}[ld] &  & (f\c g)\c (h \c k) \ar@{->}[rd] &  \\
		f\c ((g\c h)\c k) \ar@{->}[rrd] &  &  &  & ((f\c g)\c h)\c k \\
		&  & (f\c (g\c h))\c k \ar@{->}[rru] &  &
	}\]
	\[\label{the_real_cube}\vxy[@R=3mm@C=3mm]{
		&\Mly(A,B)\ar@<.25em>@{.>}[dl] \ar[rr]\ar[dd]|\hole && A\times\firstblank/B\ar@<.25em>[dl] \ar[dd]\\
		\clM\ar@<.25em>@{.>}[ur]\ar[rr]\ar[dd] && \clK/B\ar[dd] \ar@<.25em>[ur]\\
		& \Alg(A\times\firstblank)\ar@{=}[dl] \ar[rr]|(.525)\hole && \clK\ar@{=}[dl]\\
		\Alg(A\times\firstblank) \ar[rr]&& \clK
	}\kern3em
	\vxy[@R=3mm@C=3mm]{
		&\Mly(A,B)\ar@<.25em>@{.>}[dl] \ar[rr]\ar[dd]|\hole && A\times\firstblank/B\ar@<.25em>[dl] \ar[dd]\\
		\clM\ar@<.25em>@{.>}[ur]\ar[rr]\ar[dd] && \clK/B\ar[dd] \ar@<.25em>[ur]\\
		& \Alg(A\times\firstblank)\ar@<.25em>[dl]\ar@{<-}@<-.25em>[dl] \ar[rr]|(.525)\hole && \clK\ar@<.25em>[dl]\ar@{<-}@<-.25em>[dl]\\
		\Alg(A\times\firstblank) \ar[rr]&& \clK
	}\]
	\[\label{sq_of_J}
    \vxy{
      \Mre(A,B) \ar[r]^{(\firstblank)^\mreExt}\ar[d]_{D_1}& \Mly(A^*,B)\ar[d]^{\Mly(t,B)}\\
      \Mly(A,B) \ar[r]_{(\firstblank)^\mlyExt}& \Mly(A^+,B) \\
    }\qquad\qquad 
	\vxy{
		\Mre(A,B) \ar[r]^{(\firstblank)^\mreExt}\ar@{<-}[d]_{K_1}& \Mly(A^*,B)\ar@{<-}[d]^{(B^t)^*}\dltwocell<\omit>{\theta}\\
		\Mly(A,B) \ar[r]_{(\firstblank)^\mlyExt}& \Mly(A^+,B) \\
	  }\]
\subsection{Proofs}
This is essentially the argument that appears in passing in \cite{Lu_2018}, extended from \emph{strict} semigroupal/monoidal categories to \emph{non-strict} ones.
\begin{construction}\label{constr_unitize}
	Let $\clS$ be a semigroupal category with underlying category $\clS_0$ and equipped with a bifunctor $\_\otimes\_ : \clS_0\times\clS_0 \to \clS_0$
	% \[\vxy{\_\otimes\_ : \clS_0\times\clS_0 \ar[r] & \clS_0}\]
	satisfying the associativity axiom. We define $\clS^\un$ as follows:
	\begin{itemize}
		\item the underlying category $(\clS^\un)_0$ is the coproduct $\clS_0 + \bf 1$, where $\bf 1$ is the terminal category with unique object $\perp$;
		\item the `extended' multiplication functor
		      \[\vxy{\_\otimes^\un\_ : (\clS_0+{\bf 1})\times(\clS_0 +{\bf 1}) \cong (\clS_0\times\clS_0) + (\clS_0\times {\bf 1}) + ({\bf 1}\times\clS_0) + ({\bf 1}\times {\bf 1}) \ar[r]& \clS_0+{\bf 1}}\]
		      is defined piecewise as $S\otimes^\un S' = S\otimes S'$ if $S,S'\in\clS_0$ and $\perp\otimes^\un S' = S'$, $\perp\otimes^\un \perp=\perp$.
		      % \[
		      %   S\otimes^\un S' = S\otimes S' , \qquad \perp\otimes^\un S' = S' , \qquad S\otimes^\un \perp = S , \qquad \perp\otimes^\un \perp=\perp
		      % \]
		      (Notice in particular that in case $\clS$ had a
		      monoidal unit, $\perp$ `replaced' it: we have added $\perp$ as a free unit.) On morphisms we follow a similar strategy; there is only an identity in $\bf 1$, and no morphism $\perp \leftrightarrows S$, so we just have to define $f\otimes^\un g := f\otimes g$ and $\id_\perp \otimes^\un \id_\perp = \id_\perp$, $f\otimes^\un \id_\perp=f$, $\id_\perp\otimes^\un g$ are forced to be $f$ and $g$ respectively, if we want that $\otimes^\un$ sends identities to identities, and that is is bifunctorial. Thus, $\perp\otimes^\un\_$ and $\_\otimes^\un\perp$ are \emph{strictly} the identity functors of the category $(\clS^\un)_0$.
		      % \item The associator has components $\alpha_{XYZ}$ on objects of $\clS_0$, and when either $X,Y$ or $Z$ is $\perp$ we define it to be the appropriate identity morphism of the tensor of the remaining two objects. Naturality of $\alpha$ is ensured by these choices. This also ensures that the pentagon axiom for 4 objects $X,Y,Z,W$ trivially holds either because $X,Y,Z,W$ are all in $\clS_0$ (and thus the pentagon axiom is true in the semigroupal structure of $\clS_0$) or because the pentagon is made of identities.
		      % \item The last bit of structure that we have to assess is the `triangle axiom' for a monoidal structure: the triangle
		      %       \[\vxy{
		      % 	      X \otimes^\un (\perp\otimes^\un Y) \ar@{->}[rr]^{\alpha_{X\perp Y}} \ar@{->}[rd] &  & (X \otimes^\un \perp)\otimes^\un Y \ar@{->}[ld] \\
		      % 	      & X\otimes^\un Y &
		      % 	      }\]
		      %       must commute as it is composed by identities, and reasoning with a similarly straightforward case analysis, we conclude it does.
	\end{itemize}
\end{construction}
\begin{proof}[Proof that the functor in \autoref{unitize_thm} is a left adjoint]\label{proof_of_unitize_thm}
	First of all let's specify the last (straightforward) piece of structure needed to define the unitization.
	\begin{itemize}
		\item The associator has components $\alpha_{XYZ}$ on objects of $\clS_0$, and when either $X,Y$ or $Z$ is $\perp$ we define it to be the appropriate identity morphism of the tensor of the remaining two objects. Naturality of $\alpha$ is ensured by these choices. This also ensures that the pentagon axiom for 4 objects $X,Y,Z,W$ trivially holds either because $X,Y,Z,W$ are all in $\clS_0$ (and thus the pentagon axiom is true in the semigroupal structure of $\clS_0$) or because the pentagon is made of identities.
		\item The last bit of structure that we have to assess is the `triangle axiom' for a monoidal structure: the triangle
		      \[\vxy{
			      X \otimes^\un (\perp\otimes^\un Y) \ar@{->}[rr]^{\alpha_{X\perp Y}} \ar@{->}[rd] &  & (X \otimes^\un \perp)\otimes^\un Y \ar@{->}[ld] \\
			      & X\otimes^\un Y &
			      }\]
		      must commute as it is composed by identities, and reasoning with a similarly straightforward case analysis, we conclude it does.
	\end{itemize}
	We have exhibited a construction for the unitization of a semigroupal category $\clS$; we still have to prove that this is a functor (but this is obvious: each semigroupal functor $H : \clS\to \clS'$ induces a monoidal functor coinciding with $H$ on $\clS_0$ and sending $\perp_\clS$ to $\perp_{\clS'}$), and its universal property. For the latter, we have to check that there is an isomorphism
	\[\SgCat(\clS,\forg \clM)\cong\MonCat(\clS^\un,\clM)\]
	between the category of strong semigroupal functors $\clS\to \forg \clM$ into a monoidal category, and the category of strong monoidal functors $\clS^\un\to\clM$, when both categories are equipped with semigroupal and monoidal natural transformations respectively.

	At the level of the underlying categories, $(\_)^\un$ acts as the `Maybe' functor, and $\clM_0$ is a pointed object in $\Cat$ (by the monoidal unit, $I_\clM : {\bf 1}\to\clM$) so that $H : \clS\to \forg \clM$ induces a unique functor $\hat H : \clS_0 + {\bf 1}\to\clM_0$; this functor is now preserving the tensor product on $\clS_0$ (because $H$ was semigroupal to start with), and it is forced by the universal property of `Maybe' to send $\perp$ into $I_\clM$, so strictly speaking it becomes a \emph{normal} strong monoidal functor (=strictly preserving the identity).
\end{proof}
\begin{proof}[Proof of \autoref{unitize_thm}]\label{proof_of_thm_freebicat}
	Let $\bfS=(\bfS_0,\_\circ\_)$ be a semibicategory as in \autoref{def_semibicat}; define $\free\bfS$ as follows:
	\begin{itemize}
		\item $(\free\bfS)_0$ is the same class of objects of $\bfS$;
		\item for each $X\in\bfS_0$, we apply to the category $\bfS(X,X)$ the construction of \ref{constr_unitize}; thus, each semigroupal category $\bfS(X,X)$ is unitized into $\bfS(X,X)^\un$.
		\item the composition functors are defined as in $\bfS$ when $X,Y,Z$ are such that neither $Y=Z$ or $X=Y$, which means that in such case we take the same
		      \[\_\circ_{XYZ}\_ : \xymatrix{\bfS(Y,Z)\times\bfS(X,Y)\ar[r] & \bfS(X,Z)}\]
		      as composition maps. When $X=Y$ instead we define
		      \[\_\circ_{\un,XXZ}\_ : \xymatrix{
				      \bfS(X,Z)\times(\bfS(X,X)+{\bf 1})\cong
				      \bfS(X,Z)\times \bfS(X,X)+\bfS(Y,Z)\times {\bf 1}
				      \ar[r] & \bfS(X,Z)}\]
		      via the coproduct map of $\_\circ_{XXZ}\_$ and the right unitor of $(\Cat,\times)$. We act similarly to define
		      \[\_\circ_{\un,XYY}\_ : \xymatrix{
			      (\bfS(Y,Y) + {\bf 1})\times\bfS(X,Y)
			      \cong
			      \bfS(Y,Y)\times\bfS(X,Y) + {\bf 1}\times\bfS(X,Y)
			      \ar[r] & \bfS(X,Y)}\]
		      using the coproduct map of $\_\circ_{XYY}\_$ and the left unitor of $(\Cat,\times)$.

		      Clearly, all $\_\circ_{XYZ}\_$ so defined are functors.
		\item The components of the associator $\alpha^\un$ when either $f,g$ or $h$ is the `dummy identity' in a $\bfS(X,X)$ reduce to the identity 2-cells of the remaining two objects, considering how we defined the composition operation; on all other components, $\alpha^\un = \alpha$. Again, these choices ensure the naturality of $\alpha^\un$.
		\item Clearly, any strong semigroupal functor $H:\bfS \to \bfT$ induces a strong monoidal unctor $H^\un : \bfS^\un \to \bfT^\un$, that coincides with $H$ on $\bfS\hookrightarrow \bfS^\un$.
	\end{itemize}
	It remains to show the universal property. A pseudofunctor $H : \bfS \to \bfB$ from a semibicategory to a bicategory must induce a pseudofunctor $\bfS^\un\to\bfT$: at the level of objects nothing has changed; each hom-functor
	\[\vxy{
			H_{XY} : \bfS(X,Y) \ar[r] & \bsB(HX,HY)
		}\]
	with $X\ne Y$ is also unchanged, and since each $\bfB(HX,HX)$ is a monoidal category (hence, in particular, a pointed category) the functor $H$ extends to $H_{XX}^\un : \bfS(X,X)^\un\to\bfB(HX,HX)$; it is easily seen that this extension is to a strong monoidal functor.
\end{proof}
% \begin{proof}[Proof of \autoref{pres_lims}]\label{proof_of_pres_lims}
% 	There is an evident chain of identifications of carriers
% 	\[[A^+,B]\times B\cong [A^+,B]\times [1,B]\cong [1+A^+,B]\cong [A^*,B]\]
% 	which is easily shown to be an isomorphism of Moore machines; fleshing out the same isomorphism in terms of $H$ is far more convoluted (but reasonably enlightening about what exactly the identification is about): let's provide a snippet of pseudocode that shows how the definition can be achieved.
% 	\begin{minted}{haskell}
%     u :: [List A , B] -> [List+ A , B] x Pinfty
%     u f = ( \x -> f x
%           , \x -> case x of
%                     []   -> f []
%                     a:as -> a)
%     --
%     v :: [List+ A , B] x Pinfty -> [List A , B]
%     v f g = \x -> case x of
%       [] -> g []
%       a:as -> f a:as
% 	\end{minted}
% 	Clearly this sets up a pair of inverse bijections.
% \end{proof}
% \begin{proof}[Proof of \autoref{the_adj}]\label{proof_of_the_adj}
% 	In order to start with our analysis, it is useful to record how the functors $L_\fka$ and $R_\fkb$ act on objects: for a Mealy machine $\fke=(E,d,s) : A\mto B$, we have
% 	\begin{itemize}
% 		\item $L_\fkb(\fke)=$ \raisebox{.5em}{\tikz[anchor=center, baseline=(current bounding box.center)]{\Mealy[red!25]{\fke}\gau A\droU B\step{\Moore[blue!25]{\fkb}\dro B}}}
% 		      %= J\fkb\diamond\fke=\fkb\ltimes_{AB}\fke$
% 		      is the Moore machine $\inMo{E\times B}{d''}{s''}$ where% with carrier $E\times B$ and
% 		      \[\label{comp_right_moore}\begin{cases}
% 				      d''=\pi_B\diamond d : A\times E\times B \to E\times B : \lambda aeb.\langle d(a,e) , s(a,e)\rangle \\
% 				      s''=\id_B\diamond s : E\times B \to B : \lambda eb.b
% 			      \end{cases}\]
% 		\item $R_\fka(\fke) =$ \raisebox{.5em}{\tikz[anchor=center, baseline=(current bounding box.center)]{\Moore[blue!25]{\fka}\gau A\droU A\step{\Mealy[red!25]{\fke}\dro B}}}
% 		      %\fke\diamond J\fka = \fke\rtimes_{AB}\fka$
% 		      is the Moore machine $\inMo{A\times E}{d''}{s''}$ where
% 		      \[\label{comp_left_moore}\begin{cases}
% 				      d''=d\diamond\pi_A : A\times A\times E \to A\times E :\lambda aa'e.\langle a , d(a',e)\rangle \\
% 				      s''=s\diamond\id_A : A\times E \to B :\lambda aa'e. s(a,e)
% 			      \end{cases}\]
% 	\end{itemize}
% 	Evidently, \eqref{comp_right_moore} corresponds to the following diagram:
% 	\[\notag\vxy{
% 		E\times B && \ar[ll]_-{\langle d\circ \pi_{A\times E}, s\circ \pi_{A\times E}\rangle} A\times E \times B & E\times B \ar[r]^-{\pi_B} & B
% 		}\]
% 	whose dynamics contains all the information about $\fkx$, since the same output map `moved' to the left leg.

% 	Similarly, \eqref{comp_left_moore} corresponds to the following diagram:
% 	\[\notag\vxy{
% 		A \times E & A\times A\times E\ar[l]_-{A\times d} & A\times E \ar[r]^-s & B
% 		}\]
% 	which \emph{essentially} recovers the Moore machine we started with, apart from the fact that there can't be any isomorphism between $E$, the carrier of $\fke$, and $A\times E$, the carrier of $L_\fka(\fke)$.

% 	Now, let $\fkx=[E,d,s]$ be a soft Moore automaton, $\fky=(F,d',s')$ a generic Mealy automaton. We have to build an identification% pair of adjoint functors
% 	\[
% 		\Mly(A,B)(J\fkx,\fky) \cong \Mre(A,B)(i\fkx,\fkb\ltimes_{AB}\fky).
% 	\]
% 	A morphism $\alpha : E\to F$ in $\Mly(A,B)(J\fkx,\fky)$ fits into a commutative diagram of the following form,
% 	\[\label{dis}\vxy{
% 		E \ar[d]_\alpha & A\times E\ar[l]_-d\ar[d]_{A\times\alpha} \ar[r]^-{\pi_E}& E \ar[r]^-s& B\ar@{=}[d]\\
% 		F & A\times F \ar[rr]_-{s'}\ar[l]^-{d'} & & B
% 		}\]
% 	while a morphism in $\Mre(A,B)(i\fkx,\fkb\ltimes_{AB}\fky)$ consists of a morphism $\beta : E \to F\times A$ such that the following diagrams commute
% 	\[\label{dat}\vxy{
% 		E \ar[d]_\beta && A\times E\ar[d]_{A\times\beta}\ar[ll]_-d & E\ar[d]_\beta\ar[r]^-s & B\ar@{=}[d]\\
% 		F\times B && A\times F\times B\ar[ll]^-{\langle d'\circ \pi_{A\times F},s'\circ \pi_{A\times F}\rangle} & F\times B\ar[r]_-{\pi_B} & B
% 		}\]
% 	(the lower horizontal row has been completed according to \autoref{fatto_conto}, cf. \eqref{comp_right_moore}.) We have to build a bijective correspondence between \eqref{dis} and \eqref{dat}.

% 	Clearly, the universal property of products splits $\beta : E \to F\times A$ in two morphisms $\langle \pi_F\circ\beta, \pi_B\circ\beta\rangle$, the first component of which is a good candidate for an $E\to F$ filling \eqref{dis}, and the second component of which is forced to be equal to $s$.

% 	Hence we are left with two claims to prove (through straightforward equational reasoning):
% 	\begin{enumtag}{la}
% 		\item \label{la_1} $\alpha = \pi_E\circ\beta$ fits in \eqref{dis};
% 		\item \label{la_2} $\beta = \langle \alpha,s\rangle$ fits in \eqref{dat}.
% 	\end{enumtag}
% 	For what concerns \ref{la_1}, we have to follow the chain of equalities
% 	\begin{align*}
% 		\Leuchtstift{\alpha}\circ d & = \pi_F\circ \Leuchtstift{\beta\circ d}                                                                        \\
% 		                            & =\Leuchtstift{\pi_F \circ \langle d'\circ \pi_{A\times F},s'\circ \pi_{A\times F}\rangle} \circ (A\times\beta) \\
% 		                            & =d'\circ \Leuchtstift{\pi_{A\times F}}\circ (A\times\beta)                                                     \\
% 		                            & =d'\circ \Leuchtstift{A\times \pi_F\circ (A\times\beta)}                                                       \\
% 		                            & =d'\circ (A\times \alpha)
% 	\end{align*}
% 	which proves that the left diagram in \eqref{dis} commutes. For the right part of \eqref{dis}, we consider the right part of \eqref{dat} and apply $\pi_B$, to obtain
% 	\[s\circ d = \pi_B\circ \beta\circ d = \pi_B \langle d'\circ \pi_{A\times F},s'\circ \pi_{A\times F}\rangle \circ (A\times \beta)=s'\circ (A\times\alpha)\]
% 	which means that $s'\circ (A\times\alpha) = s\circ d=s\circ \pi_E$, because $[E,d,s]$ was soft.

% 	The converse implication, proving \ref{la_2}, is easier, and it can be carried over completely equationally:
% 	\begin{align*}
% 		\langle \alpha,s\rangle\circ d                                                                        & = \langle \Leuchtstift{\alpha\circ d}, \Leuchtstift{s\circ d}\rangle                                                                                                      \\
% 		                                                                                                      & = \langle d'\circ (A\times \alpha), s\circ \pi_E\rangle                                                                                                                   \\
% 		\langle d'\circ \pi_{A\times F},s'\circ \pi_{A\times F}\rangle \circ (A\times\langle \alpha,s\rangle) & =  \langle d'\circ \Leuchtstift{\pi_{A\times F}\circ (A\times\langle \alpha,s\rangle)},s'\circ \Leuchtstift{\pi_{A\times F}\circ (A\times\langle \alpha,s\rangle)}\rangle \\
% 		                                                                                                      & =\langle d'\circ (A\times\alpha), \Leuchtstift{s'\circ (A\times\alpha)}\rangle                                                                                            \\
% 		                                                                                                      & = \langle d'\circ (A\times \alpha), s\circ \pi_E\rangle.
% 	\end{align*}
% 	From this description it is immediate to argue that the (relative) unit of the relative adjunction has components witnessed by the commutativity of the diagrams
% 	\[\vxy{
% 		E\ar[d]_{\langle 1,s\rangle} && A\times E\ar[d]^{A\times\langle 1,s\rangle}\ar[ll]_-d & E \ar[d]^{\langle 1,s\rangle}\ar[r]^-s & B \ar@{=}[d]\\
% 		E\times B && A\times E\times B\ar[ll]^-{\langle d\circ \pi_{A\times E},s\circ \pi_E\circ \pi_{A\times E}\rangle} & E\times B\ar[r]_-{\pi_B} & B
% 		}\]
% 	(nota that again, the commutativity of the left square is true only by virtue of the assumption that $[E,d,s]$ is soft).
% \end{proof}