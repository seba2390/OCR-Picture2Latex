\section{The semibicategory of Moore machines}
Let's start by recording a precise definition of the categories of Moore and Mealy machines. Fix a monoidal category $(\clK,\otimes)$ in the background, which we will call the `base category'.
\begin{definition}[The hom-categories of $\Mly$ and $\Mre$] \agda{AsPullbacks.agda}\, \label{def_mlymre}
  Fix two objects $A,B\in\clK_0$; $A$ is to be thought of as the `input' object, and $B$ as the `output' object.

  Denote with $\Alg(A\otimes\firstblank)$ the category of endofunctor algebras for the functor $X\mapsto A\otimes X$ (or \emph{acts}, cf. \cite{kilp2000monoids}), and $(A\otimes\firstblank/B)$ the comma category of morphisms $A\otimes X\to B$. Then,
  \begin{enumtag}{cf}
    \item \label{cf_1} the category $\Mly(A,B)$ of Mealy machines with input $A$ and output $B$ is the strict pullback square
    \[\label{mly_pb}\vxy{
        \Mly(A,B) \ar[r]^{U'}\ar[d]_{V'} \xpb & (A\otimes\firstblank/B)\ar[d]^V \\
        \Alg(A\otimes\firstblank) \ar[r]_U & \clK
      }\]
    \item \label{cf_2} the category $\Mre(A,B)$ of  Moore machines with input $A$ and output $B$ is the strict pullback square
    \[\label{mre_pb}\vxy{
        \Mre(A,B) \ar[r]^{U'}\ar[d]_{V'} \xpb & \clK/B\ar[d]^V \\
        \Alg(A\otimes\firstblank) \ar[r]_U & \clK
      }\]
    if $\clK/B$ is the comma category of the identity functor $\id : \clK\to\clK$, and of the constant functor at $B\in\clK_0$.\footnote{It is worth noting three things: the definition of $\Mly(A,B)$ and $\Mre(A,B)$ depends on $\clK$; every lax monoidal functor $F:\clH\to\clK$ induces base changes pseudofunctors $F_* : \Mly_\clH(A,B)\to \Mly_\clK(FA,FB)$ --and similarly for Moore; under suitable assumptions, this can be extended to a 3-functor from (Cartesian monoidal categories, product-preserving functors, Cartesian natural transformations). This is the content of \cite[Remark 2.6]{noi:bicategories}, but it can be deduced in one fell sweep from the characterization in \cite{Katis1997}.}
  \end{enumtag}
\end{definition}
\begin{remark}\agda{Automata.agda}\label{moore_2cells}
  Unwinding the definition, an object of $\Mly(A,B)$ is a span $\inMe Eds : E \xot d A\otimes E \xto s B$,
  % \[\vxy{
  %     \inMe Eds : E \xot d A\otimes E \xto s B
  %   }\label{mmach_eq}\]
  and a morpism $f$ from $\inMe Eds$ to $\inMe F{d'}{s'}$ is an $f : E\to F$ in the base category such that $f\circ d = d'\circ (A\otimes f)$ and $s'\circ (A\otimes f) = s$.
  Similarly, an object of $\Mre(A,B)$ is a span $\inMo Eds : E \xot d A\otimes E, E \xto s B$,
  % \[\vxy{
  %     \inMo Eds : E & A\otimes E,\quad E \ar[r]^-s\ar[l]_-d & B
  %   }\label{momach_eq}\]
  and a morphism  $f$ from $\inMo Eds$ to $\inMo F{d'}{s'}$ is an $f : E\to F$ in the base category such that $f\circ d = d'\circ (A\otimes f)$ and $s'\circ f = s$.
\end{remark}
\begin{corollary}\agda{FMoore/Limits.agda}
  An immediate consequence of this definition (cf. \cite{noi:completeness} for the details) is that when $\clK$ is closed and $\firstblank\otimes\firstblank$ separately commutes with small colimits (e.g., when $\clK$ is closed), each category $\Mly(A,B)$ and $\Mre(A,B)$ is complete and cocomplete:
  \begin{itemize}
    \item the category $\Mre(A,B)$ has a terminal object, the internal hom $[A^*,B]$;
    \item the category $\Mly(A,B)$ has a terminal object, the internal hom $[A^+,B]$.\footnote{Assuming countable coproducts in $\clK$, the free \emph{monoid} $A^*$ on $A$ is the object $\sum_{n\ge 0} A^n$; the free \emph{semigroup} $A^+$ on $A$ is the object $\sum_{\ge 1} A^n$; clearly, if $1$ is the monoidal unit of $\otimes$, $A^*\cong 1+A^+$, and the two objects satisfy `recurrence equations' $A^+\cong A\otimes A^+$ and $A^*\cong 1+A\otimes A^*$.}
  \end{itemize}
\end{corollary}
From now on, we assume $(\clK,\times)$ is a \emph{Cartesian} category (more than often, Cartesian closed). Then, a bicategory $\Mly=\Mly_\clK$ can be defined as follows (cf. \cite{rosebrugh_sabadini_walters_1998} where this is called `$\Circ$', and a more general structure is studied, in case the base category has a non\hyp{}Cartesian monoidal structure --but then a clear analogy with Mealy machines is lost):
\begin{definition}[The bicategory $\Mly$, \cite{rosebrugh_sabadini_walters_1998}] \agda{Mealy/Bicategory.agda} \label{saba_mly}
  In the bicategory $\Mly$
  \begin{itemize}
    \item \emph{0-cells} $A,B,C,\dots$ are the same objects of $\clK$;
    \item \emph{1-cells} $A\mto B$ are the Mealy machines $(E,d,s)$, i.e. the objects of the category $\Mly(A,B)$ in \autoref{mly_pb}, thought as morphisms $\langle s,d\rangle : E\times A \to B\times E$ in $\clK$;
    \item \emph{2-cells} are Mealy machine morphisms as in \autoref{mly_pb};
    \item the composition of 1-cells $\firstblank \diamond \firstblank$ is defined as follows: given 1-cells $\langle s,d\rangle : E\times A\to B\times E$ and $\langle s',d'\rangle : F\times B \to C\times F$ their composition is the 1-cell $\langle s'\diamond s,d'\diamond d\rangle : (F\times E)\times A \to C\times (F\times E)$, obtained as
          \[\label{qui}\vxy[@C=1.5cm]{F\times E \times A \ar[r]^-{F\times \langle s,d\rangle} &
            F\times B \times E \ar[r]^-{\langle s',d'\rangle \times E} &
            C\times F\times E;}\]
    \item the \emph{vertical} composition of 2-cells is the composition of Mealy machine morphisms $f : E \to F$ as in \autoref{mly_pb};%, happening in $\clK$;
    \item the \emph{horizontal} composition of 2-cells is the operation defined thanks to bifunctoriality of $\firstblank\diamond\firstblank : \Mly(B,C)\times \Mly(A,B)\to \Mly(A,C)$;
    \item the associator and the unitors are inherited from the monoidal structure of $\clK$.
  \end{itemize}
\end{definition}
\begin{remark}[The universal property of $\Mly$]
  Regard $\clK$ as a bicategory with a single 0-cell, and let $\Omega \clB$ be the bicategory of pseudofunctors $\bfN\to\clB$, lax natural transformations, and modifications; then (cf. \cite{Katis1997,ITA_2002__36_2_181_0}) there is an equivalence of bicategories $\Mly_\clK\cong \Omega \clK$.
\end{remark}
\begin{remark}[Composition of 1-cells fleshed out] \agda{Set/Automata.agda\#L53}
  The composition of 1-cells in $\Mly$ happens as follows (where we freely employ $\lambda$-notation available in any Cartesian closed category):
  \[
    d_2\diamond d_1 : \lambda efa.\langle d_2(f,s_1(e,a)),d_1(e,a)\rangle \label{d2d1_term}\qquad
    s_2 \diamond s_1 : \lambda efa.s_2(f,s_1(e,a))\notag
  \]
\end{remark}
\begin{definition}[Terminal extensions of machines] \agda{Set/Functors.agda\#L57} \label{termex}\leavevmode
  \begin{itemize}
    \item Every Moore machine $\inMo Eds$ can be extended as $\xymatrix{E & E\times A^* \ar[r]^-{s^\Delta}\ar[l]_-{d^\Delta}& B}$
          % \[\vxy{
          %   E & E\times A^* \ar[r]^-{s^\Delta}\ar[l]_-{d^\Delta}& B
          %   }\]
          where $d^\Delta$ is obtained from the universal property of $A^*$, and $s^\Delta$ is defined inductively using the distributivity of products over coproducts: $s_1 = s$ and for each $n\ge 2$,
          \[s_n := \big(E\times A^n \xto{d\times A^{n-1}} E\times A^{n-1} \to \cdots \xto{d\times A} E\times A\xto{d} E \xto s B\big).\]
    \item Similarly, every Mealy machine $\inMe Eds$ has an extension $\xymatrix{E & E\times A^+ \ar[r]^-{s^\Delta}\ar[l]_-{d^\Delta} & B}$.
          % \[\vxy{
          %   E & E\times A^+ \ar[r]^-{s^\Delta}\ar[l]_-{d^\Delta} & B.
          %   }\]
          % obtained posing $s_1=s$ and
          % \[s_n := E\times A^n \xto{d\times A^{n-1}} E\times A^{n-1} \to \cdots \xto{d\times A} E\times A \xto s B.\]
  \end{itemize}
\end{definition}
\begin{remark}  \label{termex_functors}
  The above procedure defines functors $(\firstblank)^\mreExt : \Mre(A,B)\to \Mly(A^*,B)$ and $(\firstblank)^\mlyExt : \Mly(A,B)\to \Mly(A^+,B)$. We call these the \emph{terminal extensions} of the Moore/Mealy machines at study (due to their link with the terminal morphism in $\Mre(A,B)$ and $\Mly(A,B)$, cf. \cite[Ch. 11]{Ehrig} and \cite{noi:completeness}).
\end{remark}
\begin{definition}[Embedding $\Mre$ in $\Mly$]\agda{Set/Adjoints.agda\#L28} \label{tautj}
  There is a 2-functor $J:\Mre\to\Mly$ from the category of Moore machines to the category of Mealy machines, regarding each Moore machine $(E,d,s)$ as a Mealy machine
  \[\vxy{
    E & E\times A\ar[l]_d\ar[r]^-{\pi_E} & E \ar[r]^s & B
    }\]
\end{definition}
\begin{notation}
  Graphical notation is very useful in bookkeeping the structure of a Mealy or Moore machine and invaluable in simplifying computations: so, let us depict the generic Mealy (red) and Moore (blue) machines with input and output $A,B$ as follows:
  \[
    \begin{tikzpicture}
      \Mealy[red!25]{E}\arrow{A}{B} \step[3]{\Moore{E}\arrow{A}{B}}
    \end{tikzpicture}
  \]
  Instead, when we need to look closer at the structure of a given automaton, we might use the following string-diagrammatic notation: the diagrams
  \[
    \begin{tikzpicture}
      \arTwo[green!25]{d} \gau{E}
      \step[1.5]{\twoAr[yellow!30]{s}\dro{B}}
      \down{\dro{A}}
      \up{\dro{E}}
    \end{tikzpicture}
  \]
  denote the same span of \eqref{moore_2cells} forming a Mealy 1-cell $(E,d,s)$ (and $d$ is oriented right-to-left!), and if the machine lies in the image of the functor $J$ above, the picture simplifies to
  \[
    \begin{tikzpicture}
      \arTwo[green!25]{d}
      \gau{E}\step[1.5]{\Umor[yellow!30]{s}\up{\dro{B}}
        \down\counit}
      \down{\dro{A}}
      \up{\dro{E}}
    \end{tikzpicture}
  \]
\end{notation}
(In such a situation, we say that $s$ is `Moore' or `of Moore type'.) Adopting this notation the composition of $(E,d_1,s_1) : A \mto B$ and $(F,d_2,s_2) : B\mto C$ is written
% veru cursed picture...
\[\label{da_comp}
  \begin{tikzpicture}
    \down{\twoAr[yellow!30]{s_1}\up{\gau{E}}\down{\gau A}\up[2]{\gau{F}}}
    \Uid \step{\twoAr[yellow!30]{s_2}\dro{C}}
    \begin{scope}[xshift=4cm,xscale=.75,yscale=-1,yshift=-1.5cm]
      \down\Did\comult\up[3]\comult
      \step{
        \xScale[2]{\down\Did}
        \Did
        \up[2.5]{\yScale[.5]\braid}
        \up[3]\Uid
      }
      \step[2]{
        \twoAr[yellow!30]{s_1}
        \up[3]{\twoAr[green!25]{d_1}}
      }
      \step[3]{
        \down{\twoAr[green!25]{d_2}}
        \up[2]\Uid
        \down{\dro{F}}
        \up[3]{\dro{E}}
      }
      \gau{E}\down[2]{\gau{F}}\up[3]{\gau A}
    \end{scope}
    \node at (1,-.5) {$s_2\diamond s_1$};
    \node at (5.5,-.5) {$d_2\diamond d_1$};
  \end{tikzpicture}
\]
and in case $s_2$ (resp., $s_1$) is Moore, it suitably simplifies.

From this we deduce at once the following lemma.
\begin{lemma}[Moore overrides Mealy, on both sides]\agda{Set/Automata.agda\#L58} \label{overrides}
  The composition of a Mealy 1-cell $\inMe Eds$ with a Moore 1-cell $\inMo F{d'}{s'}$ is a Moore 1-cell $\inMo {F\times E}{\#}{\#}$. Conversely, the composition of a Moore 1-cell $\inMo Eds$ with a Mealy 1-cell $\inMe F{d'}{s'}$ is a Moore 1-cell $\inMo {F\times E}{\#}{\#}$.

  In algebraic terms, the statement reads as follows: the composition bifunctors given in \eqref{qui}
  % \[\vxy{\firstblank\diamond \firstblank : \Mly(B,C)\times\Mly(A,B) \ar[r] & \Mly(A,C)}\]
  restrict to bifunctors
  \[\vxy[@R=0cm]{
      J\firstblank \diamond\firstblank : \Mre(B,C)\times\Mly(A,B) \ar[r] &\Mre(A,C)\\
      \firstblank\diamond J\firstblank : \Mly(B,C)\times\Mre(A,B) \ar[r] &\Mre(A,C).
    }\]
\end{lemma}
\begin{proof}
  Immediate using graphical reasoning and the definition of composition in \eqref{da_comp}.
\end{proof}
\begin{remark}\agda{Set/Mealyfication.agda} 
  Due to the tautological nature of $J$ `virtually all' ways to compose together a Moore and a Mealy machine agree: %for example, if  $\fkm$ is a Mealy machine, and $\fkn$ a Moore machine, then
  \[\begin{array}{cccc}
      J (\fkm \diamond \fkn) = \fkm \diamond J \fkn &
      J (\fkm \diamond\fkn) = J \fkm \diamond\fkn   &
      \fkm \diamond J\fkn = J \fkm \diamond\fkn     &
      J (\fkm \diamond\fkn) = J \fkm \diamond J\fkn                                        \\
      \footnotesize\fkm\in\Mly,\fkn\in\Mre
                                                    & \footnotesize\fkm\in\Mre,\fkn\in\Mly
                                                    & \footnotesize\fkm,\fkn\in\Mre
                                                    & \footnotesize\fkm,\fkn\in\Mre
    \end{array}\]
  (after the reduction of the involved $\lambda$-terms, we obtain a strict equality.)
\end{remark}
\begin{corollary}[Moorification functors]\label{cor:prepost}
  For any two objects $A,B\in\Mly$ there exist functors
  \[\vxy[@R=0cm]{
      \firstblank\rtimes_{AB} \firstblank : \Mly(A,B)\times\Mre(A,A) \ar[r] &\Mre(A,B)\\
      \firstblank\ltimes_{AB} \firstblank : \Mre(B,B)\times\Mly(A,B) \ar[r] &\Mre(A,B)
    }\]
  Clearly then, given endo-1-cells $\fkm : A\to A$ and $\fkn : B\to B$ in $\Mre$, we can define pre- and post-composition operations as $R_\fkm=\firstblank \rtimes_{AB}\fkm$ and $L_\fkn=\fkn\ltimes_{AB}\firstblank$, which we will use in \autoref{decap_right}, \autoref{moorif_right} to characterize functors converting Mealy into Moore machines.  %\iwi: clarify this
\end{corollary}
\subsection{The semibicategory structure of $\Mre$}
\begin{corollary}[Sometimes you have no identity, and that's ok]
  \label{no_ids_never}
  The composition of a Moore 1-cell $[E,d,s]$ with a Moore 1-cell $[F,d',s']$ is a Moore 1-cell $[F\times E,d'\diamond d,s'\diamond s]$. Hence, if we take as objects the same of $\clK$, as 1-cells the Mealy machines of Moore type $\inMo Eds : A\mto B$, and considering 2-cells in the sense of \autoref{moore_2cells}, we obtain a sub-semibicategory of $\Mly$, the \emph{semibicategory of Moore machines}.
  It is important to note that the composition law of Moore machines is inherently nonunital: given the composition law defined in \autoref{saba_mly} \emph{there can be no other identity 1-cell than} $\inMe 1{\pi_1}{\pi_A} : A\mto A$, and this is not of Moore type.
\end{corollary}
It is worth recording the precise definition of what we denote as a semibicategory; loosely speaking, we are considering a weakening of \cite[Definition 1.1]{CTGDC_2002__43_3_163_0} in which we relax the requirement that we have strictly associative composition; a semibicategory is then the semicategorical\footnote{\emph{Semicategories} were introduced by Mitchell as `categories without identities' and studied to some extent in \cite{Mitchell1972}; \emph{semigroupal categories} are often studied in relation with their cognates the \emph{magmoidal} ones (where a mere binary functor $\firstblank\odot\firstblank : \clM\times\clM\to\clM$ is given, cf. \cite{Davydov2017nuclei}); a semibicategory with a single object, is precisely a (nonstrict) semigroupal category, cf. \cite{Lu_2018} for a proof that semigroupal categories support graphical calculi.} equivalent of a bicategory \cite{10.1007/BFb0074299}, \cite[I.3]{Gray1974}, or again equivalently the multi-object version of semigroupal categories (as defined, e.g., in \cite[3.3]{boyarchenko2007associativity}, \cite{Lu_2018}, although in a strict sense; for the weak version cf. \cite[4.1]{Elgueta2004}, or adapt \cite{CTGDC_2002__43_3_163_0} as we do). Cf. also \cite{KOCK2008,Kock2010}.
\begin{definition} \agda{Categories/SemiBicategory.agda} \label{def_semibicat} 
  A \emph{semibicategory} consists of
  \begin{enumtag}{sc}
    \item a collection of objects or 0-cells $\bfS_0$;
    \item for all objects $X,Y\in\bfS_0$ a \emph{category} $\bfS(X,Y)$;
    \item \label{bicomp} for all objects $X,Y,Z\in\bfS_0$ a \emph{composition} bifunctor $\_\circ\_ : \bfS(Y,Z)\times\bfS(X,Y)\to \bfS(X,Z)$;
    % \[\xymatrix{\_\circ\_ : \bfS(Y,Z)\times\bfS(X,Y)\ar[r] & \bfS(X,Z)}\label{bicomp}\]
    \item for all $f:Z\to W,g:Y\to Z,h:X\to Y$, an \emph{associator} isomorphism $\alpha_{fgh} : f\circ (g\circ h) \cong (f\circ g)\circ h$ (cf. \eqref{diag1}).
    % \[\xymatrix{\alpha_{fgh} : f\circ (g\circ h) \ar[r]^-\cong & (f\circ g)\circ h}\]
  \end{enumtag}
  The associator morphism must satisfy the usual \emph{pentagon equation} for all $f,g,h,k$: the diagram of \eqref{diag3} is commutative.
\end{definition}
Now, a technical result of a general kind allows replacing the semibicategory of Moore machines with a bicategory of `Moore machines', which is identical to $\Mre$ in all respects, apart from the fact that a class of symbols $\id_A$, one for each object $A\in\clK_0$, has been forcefully added in order to play the r\^ole of identity 1-cell.
\begin{theorem}\label{unitize_thm}
  The obvious forgetful 2-functor $\MonCat\to \SgCat$ admits a left adjoint $(\_)^\un : \SgCat \to \MonCat$.
\end{theorem}
\begin{proof}
  \deferredRef{unitize_thm}. Cf. also \autoref{constr_unitize}.
\end{proof}