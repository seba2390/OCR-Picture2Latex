\section{A plethora of adjoints between machines}
The key observation guiding our intuition throughout the section is the following.
\begin{lemma}[Pulling back fibered adjunctions]\agda{PullbackFiberedAdjunctions.agda}\label{fundamental_lemma}
  Given a commutative cube
  \[\label{the_cubo}\vxy[@R=3mm@C=3mm]{
      &\clM'\ar@<.25em>@{.>}[dl] \ar[rr]\ar[dd]|\hole && \clB'\ar@<.25em>[dl] \ar[dd]\\
      \clM\ar@<.25em>@{.>}[ur]\ar[rr]\ar[dd] && \clB\ar[dd] \ar@<.25em>[ur]\\
      & \clA\ar@{=}[dl] \ar[rr]|(.5)\hole && \clK\ar@{=}[dl]\\
      \clA \ar[rr]&& \clK
    }\]
  assume the front and back vertical faces are strict pullbacks in $\Cat$; then any adjunction $F:\clB \leftrightarrows \clB':G$ pulls back to an adjunction $F':\clM\leftrightarrows \clM':G'$.\footnote{A similar result holds, when the bottom horizontal square is replaced by a commutative square of adjunctions in the sense of \cite[IV.7]{McL}; we will invoke the more general version as well: refer for this to the right part of \eqref{the_real_cube}.}
\end{lemma}
\begin{proof}
  Immediate, using the 2-dimensional universal property of the pullbacks involved.
\end{proof}
\begin{corollary}
  Any adjunction $\inlinexyadj{\clK/B}{A\times\firstblank/B}{F}{G}$ fibered over $\clK$ induces an adjunction
  \[\label{pullata}\inlinexyadj{\Mre(A,B)}{\Mly(A,B)}{F}{G}\]
  fibered over $\Alg(A\times\firstblank)$. (When there is no danger of confusion, we will keep denoting as $F\dashv G$ the pulled back adjunction. Refer to the left part of \eqref{the_real_cube} for the diagram we have in mind.)
\end{corollary}
A blanket assumption for all that follows is that $\clK$ is (locally) Cartesian closed. Then there is an evident equivalence of categories $(A\times\firstblank)/B\cong\clK/B^A$; we deduce that:
\begin{lemma}\label{first_of_many}
  There are adjoint triples
  \[\label{the_triple}\vxy[@C=2cm]{
    \Sigma_f \dashv f^* \dashv \Pi_f : \clK/B^A \ar[r]|-{f^*}&\ar@<.5em>[l] \ar@<-.5em>[l] \clK/B\\
    }\]
  fibered over $\clK$, induced by a morphism $f' : B\to B^A$ (or rather by its mate $f : A\times B\to B$) pulls back to an adjoint triple
  fibered over $\Alg(A\times\firstblank)$, between $\Mre(A,B)$ and $\Mly(A,B)$.
\end{lemma}
\begin{example}
  Let $\clK$ be locally Cartesian closed; let $\pi_B : A\times B\to B$ be the projection on the second factor and $\varpi_B : B\to B^A$ its mate; it is well-known that there is a triple of adjoints $\varpi_B\circ\firstblank\dashv \varpi_B^*\dashv \Pi_{\varpi_B}$, acting respectively as composition, pullback, and direct image for $\varpi_B$ \cite[A4.1.2]{Johnstone2002}. As a consequence, we obtain a triple of functors
  \[\label{proj_triple}
    \vxy[@C=1.5cm]{
      \varpi_B\circ\firstblank \dashv\varpi_B^*\dashv \Pi_{\varpi_B} : \Mly(A,B) \ar[r] & \Mre(A,B) \ar@<.5em>[l]\ar@<-.5em>[l]
    }
  \]
\end{example}
This is not the only way to obtain adjoint situations between machines from adjoint situations between comma categories: we can exploit the fact that \autoref{fundamental_lemma} holds in slightly more generality, also for adjunctions that are not fibered over $\clK$; thus,
\begin{itemize}
  \item any natural transformation $\alpha$ between $A\times\firstblank$ and the identity functor of $\clK$ induces, by the 2-dimensional part of the universal property of comma objects, a functor $\alpha^*$ between $\clK/B$ and $(A\times\firstblank)/B\cong \clK/B^A$. But the functor $A\times\firstblank$ is a comonad on any Cartesian category, whence it's naturally copointed by the counit $\epsilon : A\times\firstblank \To \id_\clK$; from this, we get a functor
        \[\label{counitata}\vxy{
            \epsilon^* : \clK/B\ar[r] & (A\times\firstblank)/B\cong\clK/B^A\\
          }\]
        pulling back to a functor $\bar\epsilon^* : \Mre(A,B)\to \Mly(A,B)$.
  \item All morphisms inductively defined by the rule $d^{(0)}:=\pi_E$, $d^{(1)}=d : A\times E\to E$, $d^{(n)} := d \circ (A\times d^{(n-1)})$ as in \autoref{termex} induce functors
        \[\label{di_enne}\vxy[@R=0mm]{
            D_n^* : \clK/B\ar[r] & (A\times\firstblank)/B\cong\clK/B^A : s \ar@{|->}[r] & s \circ d^{(n)}
          }\]
        pulling back thanks to \autoref{fundamental_lemma} to a functor $\bar D_n : \Mre(A,B)\to \Mly(A,B)$.
\end{itemize}
Adjoint situations like \eqref{proj_triple} however provide a nifty characterization for the precomposition functors $\Mly(u,B):\Mly(A',B)\to \Mly(A,B)$ induced by $u : A\to A'$ in the base category. From $u$, using the Cartesian closed structure of $\clK$, we get $B^u : B^{A'}\to B^A$, from which, applying \autoref{fundamental_lemma} to the triple of adjoints
\[\label{triple_for_precomp_funs}\vxy[@C=2cm]{
  B^u\circ\firstblank \dashv (B^u)^* \dashv \Pi_u : \clK/B^A \ar[r]|-{(B^u)^*}&\ar@<.5em>[l] \ar@<-.5em>[l] \clK/B^{A'}\\
  }\]
(same notation of \autoref{first_of_many}) we get a triple of adjoints $\xymatrix{\Mly(A,B) \ar[r]&\ar@<.5em>[l] \ar@<-.5em>[l] \Mly(A',B)}$ and an identification of the leftmost adjoint $B^u\circ\firstblank$ with $\Mly(u,B) : \Mly(A',B)\to \Mly(A,B)$.

As a direct consequence,
\begin{remark}\label{some_adjoints_btwn_Mly}
  Let $\clK$ be locally Cartesian closed; then each functor $\Mly(u,B) : \Mly(A',B)\to \Mly(A,B)$ sits in a triple of adjoints $\Mly(u,B) \dashv (B^u)^* \dashv \Pi_u$.
\end{remark}
\color{black}
\begin{remark}[Adjunctions between hom-categories]\label{some_adjs}
  In case $\clK$ is locally presentable (e.g., $\clK=\Set$, any Grothendieck topos, a docile category of spaces, etc.) AFTs in their classical form \cite[Ch. 5]{Bor2}, \cite[3.3.3---3.3.8]{Bor1} applied to $\epsilon^*, D_n^*$ yield right adjoints $R$ and $K_n$ and thus, by virtue of \autoref{fundamental_lemma} above, we get
  \begin{itemize}
    \item An adjunction $\bar \epsilon^* : \Mre(A,B)\rightleftarrows \Mly(A,B) : \bar R$;
    \item for evey $n\ge 1$ an adjunction $\bar D_n : \Mre(A,B)\rightleftarrows \Mly(A,B)  : K_n$, with $D_n$ fully faithful --and thus $\bar K_n$ is a coreflector.
  \end{itemize}
\end{remark}
Among these, especially the second bears some importance to automata theory. We will analyse its structure, providing a clean explicit construction for $K_0, K_1$ in the rest of the present section; in particular, we will completely characterise the coreflective subcategory of $\bar D_n\bar K_n$-coalgebras. A crucial point of our discussion will be that $\bar K_n$ arises as composition $\fku\ltimes\firstblank$ for a universal 1-cell $\fku$ of Moore type, following the notation of \autoref{cor:prepost}.
\begin{remark}[Biadjunctions between $\Mly$ and $\Mre$]
  The adjunctions in \autoref{some_adjs} constitute the action on hom-categories of \emph{biadjunctions} (cf. \cite{betti1981quasi} for an earlier reference, and the notion of a \emph{left adjoint of a left lax transformation} in \cite[p. 8]{CTGDC_1972__13_3_217_0}. This is a weakening of \cite[Ch. 9]{fiore} that drops the request to have adjoint \emph{equivalences} of hom-categories, replacing them with a family of adjunctions indexed over the objects of the two bicategories) $\bsE\dashv \bsR$, $\{\bsD_n\dashv \bsK_n \mid n\ge 0\}$
  % \[\bsE\dashv \bsR, \qquad \{\bsD_n\dashv \bsK_n \mid n\ge 0\}\]
  between the semibicategory $\Mre$ (or equivalently its bicategorical reflection $\Mre^\un$ in the sense of \autoref{unitize_thm}) and the bicategory $\Mly$ of Mealy machines of \autoref{saba_mly}. Each of these (2-semi) functors is the identity on 0-cells.
\end{remark}
\begin{definition}
  In the notation of \autoref{di_enne}, the functors $D_0$ and $D_1$ are defined respectively as follows: a generic Moore machine $\inMo Eds$ goes to $\inMe Ed{s\pi_E}$ and $\inMe Ed{sd}$, and more in general, $D_n$ is the upper horizontal functor in % right diagonal functor in the cube
  \[\vxy{
    \clK/B \ar[d]_U \ar[r]^-{D_n}& \clK/B^A \ar[d]^{U'}\\
    \clK \ar[r]_-{A^n\times\firstblank}& \clK
    }\]
  adapted from \autoref{fundamental_lemma} (In the lower rows, the functors $X\mapsto A^n\times X$ are clearly left adjoints).
\end{definition}
\begin{definition}[The universal Moore machine]\agda{Set/LimitAutomata.agda\#L65}\label{the_univ_moo}
  Let $X\in\clK_0$ be an object. There exists a Moore machine $\fku X = \inMo X{\pi_X}{\id_X}$ in $\Mre(X,X)$ where $\pi_X : X\times X\to X$ is the projection on the first coordinate. 
  %, and $\id : X \to X$ is the identity.
\end{definition}
The machine $\fku X$ can't be the identity 1-cell in $\Mre(X,X)$ by \autoref{no_ids_never} above, but it is `as near to the identity as a Moore 1-cell can be' to the identity in $\Mly(X,X)$, in a sense made precise by the following result: post-composition with $\fku X$ realizes the Moore machine associated to a Mealy machine.
\begin{theorem}[Moorification is composition with a universal cell]\label{moorif_right}\agda{Set/Adjoints.agda\#L28}
  The functor $K_1$ of \autoref{some_adjs} is naturally isomorphic to the functor $L_{\fku B}=\fku B\ltimes\firstblank : \Mly(A,B)\to \Mre(A,B)$ of \autoref{cor:prepost}.
\end{theorem}
\begin{remark}
  In layman's terms, $\fku B=\inMo{B}{\pi_B}{\id_B}$ is the \emph{universal machine of Moore type}, in the sense that sequential composition with it turns every Mealy machine into an `equivalent' Moore machine.

  This is a clean-cut categorical counterpart for the extremely well-known fact that the two concepts of Moore and Mealy machines are `essentially equivalent' in a sense made precise by \cite[3.1.4, 3.1.5]{Shallit}.
\end{remark}
\begin{construction}\agda{Set/LimitAutomata.agda\#L52}\label{constr_pinfty}
  Let $X\in\clK_0$ be an object. We define a Moore machine $\fkp X=\inMo {P_\infty X}{\delta_X}{\sigma_X}$ in $\Mre(X,X)$ as follows:
  \begin{itemize}
    \item its carrier results from the pullback in the base category $\clK$
          \[\label{pinfty}
            \vxy{
            P_\infty X \ar[r]\ar[d]\xpb & 1 \ar[d]^\h \\
            [X^*,X] \ar[r]& [X^+,X]
            }\]
          where the lower horizontal map results from the obvious inclusion $A^+\hookrightarrow A^*=1+A^+$ and $\h$ picks the element $\lambda(a\kons as).a$.\footnote{In every Cartesian closed category one obtains $\h : X^+\to X$ from the family $\pi_1 : X^n\to X$ of projections on the first coordinate, thanks to the universal property of $X^+=\sum_{n\ge 1} X^n$.}
    \item the map $\delta_X : P_\infty X\times X\to P_\infty X$ is defined by restriction to the dynamics of $[X^*,X]$ as $(f,a)\mapsto f$;
    \item the map $\sigma_X : P_\infty X \to X$ is defined as $f\mapsto f[\,]$.
  \end{itemize}
\end{construction}
\begin{remark}\label{pinfty_notation}
  Informally speaking, the carrier of $\fkp X$ is the subobject $H$ of $[X^*,X]$\footnote{Indeed, the left vertical map is monic; for the sake of clarity we write as if $\clK=\Set$, leaving the straightforward element-free rephrasing to the conscientious readers.}  made of those functions $f : X^*\to X$ that on nonempty lists coincide with the head function: $f[\,]$ is free to assume any value whatsoever, while $f(a\kons as)=a$ for every nonempty list $a\kons as$.

  Given this, $\sigma : f\mapsto f [\,]$ is clearly a bijection. So, it might seem unnecessarily verbose to distinguish $\fku X$ and $\fkp X$; yet, $\sigma$ is \emph{not} a Moore machine homomorphism, as the diagrams of \eqref{moore_2cells} do not commute when filled with $\sigma$.
\end{remark}
A property that is stable under isomorphism, enjoyed by $P_\infty A$ but not $A_\infty$ is the following: the diagram
\[\vxy{
  X\times P_\infty X\ar[r]^{\delta_X} \ar[d]_{\pi_{P_\infty X}}& P_\infty X \ar[d]^s \\
  P_\infty X \ar[r]_s & X
  }\]
commutes for $P_\infty X$ but not for $X$; the same property enforced for $\fku X$ would entail that $\forall(x,x')\in X.x=x'$, a blatantly false statement.
\begin{definition}[Soft machine, \protect{\cite{burroughs1961soft}}]\agda{Set/Soft.agda}\label{def_acep}
  A Moore machine $\inMo Eds$ is called \emph{soft} if the following square commutes:
  \[\label{diag_aceph}\vxy{
      A\times E\ar[r]^-d\ar[d]_{\pi_E} & E\ar[d]^s \\
      E \ar[r]_-s& B
    }\]
  and in simple terms, it can be thought of as a Moore machine whose output map `skips a beat and then starts computing', as its terminal extension $s^\Delta$ (cf. \autoref{termex}) is such that $s^\Delta(a\kons as,e) = s^\Delta(as,e)$.
\end{definition}
Clearly, soft Moore machines $A\mto B$ form a full subcategory $\aMre(A,B)$ of $\Mre(A,B)$; we denote $i : \aMre\to\Mre$ the most obvious locally fully faithful inclusion 2-functor. More is true:
\begin{lemma}[Soft machines cut other heads]\agda{Set/Soft.agda\#L32}\label{cut_head}
  Let $\fkm : A\mto B$ and $\fkn : B\to C$ be a Mealy machine and a Moore machine; if $\fkn$ is soft, then the composition $J\fkn\diamond \fkm$ is also soft.\footnote{Warning: it is \emph{not} true in general that the composition $\fkm\diamond J\fkn$ is soft.} Translated in graphical terms, the statement reads as follows, if we depict a soft Moore machine as \raisebox{.25em}{\tikz{\Moore[cyan!25]{\fkn}\fill ($(.75,.25)+(-.05,.075)$) circle (1pt);}}:
  \[
    \begin{tikzpicture}
      \Mealy{}\step{\Moore[cyan!25]{}\fill ($(.75,.25)+(-.05,.075)$) circle (1pt);}
      \step[3]{\Moore[cyan!25]{}\fill ($(.75,.25)+(-.05,.075)$) circle (1pt);}
    \end{tikzpicture}
  \]
\end{lemma}
% (Warning: it is \emph{not} true in general that the composition $\fkm\diamond J\fkn$ is soft.)
\begin{proof}
  Immediate by inspection, using graphical reasoning (diagram \eqref{diag_aceph} is easy to translate graphically) and the definition of composition in \autoref{da_comp}.
\end{proof}
\begin{remark}\agda{Set/Soft.agda\#L40}\label{P_is_acep}
  The Moore machine $\inMo{P_\infty X}{\delta_X}{\sigma_X}$ in \autoref{constr_pinfty} is soft.
\end{remark}
% \begin{proof}
%   Immediate inspection.
% \end{proof}
\begin{corollary}\agda{Set/Soft.agda\#L43}
  Let $\inMe Eds : A\mto B$ be a generic Mealy machine; then, by virtue of \autoref{overrides}, the composition $\inMo{P_\infty B}{\delta_B}{\sigma_B}\,\ltimes_{AB}\,\inMe Eds$ is a Moore machine, and by virtue of \autoref{cut_head} it is soft. Hence, the functor $L_{\fkp B} : \Mly(A,B) \to \Mre(A,B)$ factors through the obvious full inclusion $i_A : \aMre\to\Mre$.
\end{corollary}
\begin{remark}\label{ace_as_coalg}
  Given that $D_1$ is fully faithful, the unit of the adjunction $D_1\dashv K_1$ is a natural isomorphism; denote $S_1 := D_1K_1$ the comodality (= idempotent comonad) obtained as a consequence. Then the category of $S_1$-coalgebras is isomorphic to the category of soft Moore machines.
\end{remark}
An object of the form $\inMo{P_\infty A}{\delta_A}{\sigma_A}$ is thus a universal soft Moore machine, in the sense specified by the following coreflection theorem:
\begin{theorem}[Decapitation as a right adjoint]\label{decap_right}\agda{Set/Adjoints.agda\#L46}
  The coreflector of the comodality of \autoref{ace_as_coalg} is naturally isomorphic to the postcomposition functor $L_{\fkp B}=\fkp B\ltimes\firstblank$.
\end{theorem}
Similarly, denote $S_n=D_nK_n$ the comodality of the adjunction obtained from \autoref{di_enne}. Then, the category of $S_1$-coalgebras is isomorphic to the full category $\aMre[n]$ of \emph{$n$-soft Moore machines}, if we understand \autoref{def_acep} generalised as follows: the square
\[\label{diag_naceph}\vxy{
  A^n\times E\ar[r]^-{d^{(n)}}\ar[d]_{\pi_E} & E\ar[d]^s \\
  E \ar[r]_-s& B
  }\]
commutes (cf. \eqref{di_enne} for the definition of $d^{(n)}$).

Evidently, $\aMre = \aMre[1]\subseteq \aMre[2]$ and more in general, there is a chain of full inclusions
\[\vxy{
		\aMre\,\, \ar@{^{(}->}[r]& \,\,\aMre[2]\,\, \ar@{^{(}->}[r]& \,\,\aMre[3]\,\, \ar@{^{(}->}[r]& \,\,\dots\,\, \ar@{^{(}->}[r]& \,\,\Mre
	}\]
analogous to the `level' filtration of a topos (cf. \cite{kelly1989complete,Kennett2011,menni2019level,roy1997ball,Street2000}, especially in the case of simplicial sets or geometric shape for higher structures indexed over the natural numbers).

As a final remark, let's record that the results of \autoref{some_adjoints_btwn_Mly} provide an extremely intrinsic characterization for the object $P_\infty X$ of \autoref{constr_pinfty}, and a comparison theorem for the operations $(\firstblank)^\mreExt,(\firstblank)^\mlyExt$ of terminal extension of a Mealy machine and of a Moore machine expounded in \autoref{termex_functors}.
\begin{lemma}\agda{Set/Functors.agda\#L113}
  The left square in \eqref{sq_of_J}
  % \[\label{sq_of_J}
  %   \vxy{
  %     \Mre(A,B) \ar[r]^{(\firstblank)^\mreExt}\ar[d]_{D_1}& \Mly(A^*,B)\ar[d]^{\Mly(t,B)}\\
  %     \Mly(A,B) \ar[r]_{(\firstblank)^\mlyExt}& \Mly(A^+,B) \\
  %   }\]
  commutes (in the notation of \eqref{triple_for_precomp_funs}, $\Mly(t,B)$ is the obvious restriction functor induced by the `toList' map $t : A^+\hookrightarrow A^*$).
\end{lemma}
% \mproof
\begin{proof}
  Immediate inspection.
\end{proof}
\begin{corollary}[Corollary of \autoref{some_adjoints_btwn_Mly}]
  The functor $\Mly(t,B)$ sits in a triple of adjoints $\Mly(t,B) \dashv (B^t)^* \dashv \Pi_t$, where $(B^t)^*$ is given pulling back a morphism $\var{X}{B^{A^+}}$ along $B^t : B^{A^*}\to B^{A^+}$.
\end{corollary}
It follows from an easy inspection that $B^t : [A^*,B]\cong B\times [A^+,B]\to [A^+,B]$ coincides with the projection map, and thus we can consider the right square in \eqref{sq_of_J},
% \[\label{sq_of_ts}
%   \vxy{
%     \Mre(A,B) \ar[r]^{(\firstblank)^\mreExt}\ar@{<-}[d]_{K_1}& \Mly(A^*,B)\ar@{<-}[d]^{(B^t)^*}\dltwocell<\omit>{\theta}\\
%     \Mly(A,B) \ar[r]_{(\firstblank)^\mlyExt}& \Mly(A^+,B) \\
%   }\]
obtained mating the left square, and prove at once that
\begin{proposition}
  The left square of \eqref{sq_of_J} is exact, in the sense of \cite{guitart:1980,Guitart1981,Pavlovi1991} or \cite[1.8.9.(b)]{CLTT}. This is to say, $\theta$ in the right diagram \emph{ibid.} is invertible.
\end{proposition}
% \mproof
% \begin{proof}
%   \todo[inline]{I still don't know if this is true!}
% \end{proof}
Now let $\h : 1\to [A^+,A]$ be the morphism in \eqref{pinfty}: from \autoref{some_adjoints_btwn_Mly} we get a functor $\h^*$ of pullback along $\h$ (with both a left and a right adjoint), and an isomorphism between the Moore machine $\fkp X$ and $(A^t)^*(\h)$.