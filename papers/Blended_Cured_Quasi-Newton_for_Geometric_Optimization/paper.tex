\documentclass[final]{acmsiggraph}


\usepackage[ruled]{algorithm2e} 
\renewcommand{\algorithmcfname}{ALGORITHM}
\usepackage{simplewick}
\usepackage{tikz}
\usepackage{amsfonts}
\usepackage{amsmath}
\usepackage{amsthm}
\usepackage{amssymb}
\usepackage{float}
\usepackage{subfig}
\usepackage{tabularx}
\usepackage{epstopdf}
\usepackage{centernot}
\usepackage[utf8]{inputenc}
\usepackage{cleveref}
\usepackage{stmaryrd}
\usepackage{soul}
\usepackage{wrapfig}
\usepackage{lipsum}
\usepackage{xcolor,colortbl}
\usepackage{amssymb}% http://ctan.org/pkg/amssymb
\usepackage{pifont}% http://ctan.org/pkg/pifont
\usepackage{booktabs}
\usepackage{colortbl}
\usepackage{xfrac}
\usepackage{caption}
\usepackage{wrapfig}
\usepackage{lipsum}
\usepackage{algorithmic}
\usepackage{enumitem}

\usepackage{placeins}

\newcommand{\vr}[1]{\mbox{$\bm{#1}$}}  % vector
\newcommand{\vc}[1]{\mbox{\textbf{{$\mathsf #1$}}}}

\newcommand{\R}[0]{\mathbb{R}}
\newcommand{\Z}[0]{\mathbb{Z}}
\newcommand{\Tr}[0]{\mathrm{Tr}}
\renewcommand{\Re}[0]{\mathrm{Re}}
\newcommand{\SO}[0]{\mathrm{SO}}
\newcommand{\F}[0]{{_{\mathcal{F}}}}

\definecolor{redcolor}{rgb}{0.8,0,0}
\definecolor{bluecolor}{rgb}{0.0,0.1,0.6}
\definecolor{orangecolor}{rgb}{0.9.,0.5.,0.1}
\definecolor{greencolor}{rgb}{0.5,0.7,0.5}
\definecolor{browncolor}{rgb}{0.5,0.2,0.2}
\definecolor{greycolor}{rgb}{0.6,0.6,0.6}

\def\argmin{\mathop{\rm argmin}}
\def\min{\mathop{\rm min}}
\def\notimplies{ \centernot \implies}

\newcommand{\red}[1]{{\color{redcolor}{#1}}}
\newcommand{\orange}[1]{{\color{orangecolor}{#1}}}
\newcommand{\blue}[1]{{\color{bluecolor}{#1}}}

\newcommand{\smallsection}[1]{{\bf \emph{#1}}}
\newcommand{\bfi}[1]{\textit{ \textbf{#1}}}


\title{Blended Cured Quasi-Newton for Geometry Optimization}

\author{
 Yufeng Zhu \\{\footnotesize University of British Columbia \& Adobe Research}
  \and Robert Bridson \\{\footnotesize Autodesk}
  \and Danny M. Kaufman \\{\footnotesize Adobe Research}
}

\pdfauthor{}

\begin{document}

\teaser{
   \includegraphics[width=\textwidth]{figures/Figure_1_teaser_New/Figure_teaser}
 	\caption{\bfi{Twisting.} A stress-test 3D deformation problem. {\bf Left:} we initialize a 1.5M tetrahedra mesh bar with a straight rest shape into a tightly twisted coil, constraining
both ends to stay fixed.  {\bf Right:} minimizing the ISO deformation energy to find a constrained equilibrium with (top to bottom) Projected Newton (PN), Accelerated Quadratic Proxy (AQP) and our BCQN method, we show intermediate shapes at reported wall-clock time (seconds) and iteration counts at those times (BCQN/AQP/PN). BCQN converges at 30 minutes, while AQP and PN continue to optimize.}
 	\label{fig:teaser}
}

\maketitle

\begin{abstract}
Optimizing deformation energies over a mesh, in two or three
dimensions, is a common and critical problem in physical simulation and geometry processing. 
We present three new improvements to the
state of the art: a barrier-aware line-search filter that cures blocked descent steps due to  
element barrier terms and so enables rapid progress;
an energy proxy model that adaptively blends the Sobolev (inverse-Laplacian-processed) gradient and L-BFGS descent to gain the advantages of both,  
while avoiding L-BFGS's current limitations in geometry optimization tasks;
and a characteristic gradient norm providing a
robust and largely mesh- and energy-independent convergence criterion
that avoids wrongful termination when algorithms temporarily slow
their progress. Together these improvements form the basis for
Blended Cured Quasi-Newton (BCQN), a new geometry optimization
algorithm. Over a wide range of problems
over all scales we show that BCQN is generally the fastest and most
robust method available, making some previously intractable problems
practical while offering up to an order of magnitude improvement
in others.
\end{abstract}

\section{Introduction}

\begin{figure}\centering
    \includegraphics[width=\textwidth]{figures/overview}
    \caption{
        System Overview. (a) We use a large distribution of simulations with randomized parameters and appearances to collect data for both the control policy and vision-based pose estimator. (b) The control policy receives observed robot states and rewards from the distributed simulations and learns to map observations to actions using a recurrent neural network and reinforcement learning. (c) The vision based pose estimator renders scenes collected from the distributed simulations and learns to predict the pose of the object from images using a convolutional neural network (CNN), trained separately from the control policy. (d) To transfer to the real world, we predict the object pose from 3 real camera feeds with the CNN, measure the robot fingertip locations using a 3D motion capture system, and give both of these to the control policy to produce an action for the robot.
    }
    \label{fig:overview}
\end{figure}


While dexterous manipulation of objects is a fundamental everyday task for humans,
it is still challenging for autonomous robots.
Modern-day robots are typically designed for specific tasks in constrained settings and are largely unable to utilize complex end-effectors.
In contrast, people are able to perform a wide range of dexterous manipulation tasks in a diverse set of environments, making the human hand a grounded source of inspiration for research into robotic manipulation.

The Shadow Dexterous Hand~\citep{shadow-robot} is an example of a robotic hand designed for human-level dexterity; it has five fingers with a total of \num{24} degrees of freedom.
The hand has been commercially available since 2005; however it still has not seen widespread adoption, which can be attributed to the daunting difficulty of controlling systems of such complexity.
The state-of-the-art in controlling five-fingered hands is severely limited.
Some prior methods have shown promising in-hand manipulation results in simulation but do not attempt to transfer to a real world robot \citep{DBLP:conf/icra/BaiL14, DBLP:conf/sca/MordatchPT12}.
Conversely, due to the difficulty in modeling such complex systems, there has also been work in approaches that only train on a physical robot \citep{falco2018policy, DBLP:conf/humanoids/HoofHN015, DBLP:journals/corr/KumarGTL16, DBLP:conf/icra/KumarTL16}.
However, because physical trials are so slow and costly to run, the learned behaviors are very limited.

In this work, we demonstrate methods to train control policies that perform in-hand manipulation % of a block and an octagonal prism,
and deploy them on a physical robot.
The resulting policy exhibits unprecedented levels of dexterity and naturally discovers grasp types found in humans, such as the tripod, prismatic, and tip pinch grasps, 
and displays contact-rich, dynamic behaviours such as finger gaiting, multi-finger coordination, the controlled use of gravity, and coordinated application of translational and torsional forces to the object.
Our policy can also use vision to sense an object's pose --- an important aspect for robots that should ultimately work outside of a controlled lab setting.

Despite training entirely in a simulator which substantially differs from the real world,
we obtain control policies which perform well on the physical robot.
We attribute our transfer results to (1) extensive randomizations and added effects in the simulated environment alongside calibration, (2) memory augmented control polices which admit the possibility to learn adaptive behaviour and implicit system identification on the fly, and (3) training at large scale with distributed reinforcement learning.
An overview of our approach is depicted in \autoref{fig:overview}.


The paper is structured as follows.
\autoref{sec:setup} gives a system overview, describes the proposed task in more detail, and shows the hardware setup. \autoref{sec:randomizations} describes observations for the control policy, environment randomizations, and additional effects added to the simulator that make transfer possible.
\autoref{sec:train-policy} outlines the control policy training procedure and the distributed RL system.
\autoref{sec:train-vision} describes the vision model architecture and training procedure.
Finally, \autoref{sec:results} describes both qualitative and quantitative results from deploying the control policy and vision model on a physical robot. %, achieving highly


\section{Problem Statement and Overview}

The geometry optimization problem we face is solving
\begin{equation}
    x^* = \argmin_{x\in \R^{dn}} E(x),
\end{equation}
for $n$ vertex locations in $d$-dimensional space stored in vector $x$,
where the energy $E(x)$ is a measure of the deformation, and $x$
is subject to boundary conditions.\footnote{We restrict our attention to
constraining a subset of vertex positions to given values, i.e.\ Dirichlet conditions,
for simplicity.} The energy is expressed as a sum over elements $t$ in a triangulation $T$
(triangles or tetrahedra depending on dimension),
\begin{equation}
\label{eq:obj}
E(x) = \sum_{t \in T} a_t W\big( F_t(x) \big),
\end{equation}
where $a_t>0$ is the area or volume of the rest shape of element $t$, $W$ is an energy
density function taking the deformation gradient as its argument, and $F_t$ computes the
deformation gradient for element $t$.
This problem may be given as is, or may be the result of a discretization of
a continuum problem with linear finite elements for example.


\subsection{Iterative solvers for nonlinear minimization}

Solution methods for the above generally apply an algorithmic strategy of iterated
approximation and stepping~\cite{Bertsekas:2016:NOP}, built
from three primary ingredients: an energy approximation, a
line search, and a termination criteria.\footnote{Alternatively, trust-region
methods are available, though not considered in the current work nor as
popular within the field.} \\

\bfi{Energy Approximation} At the current iterate $x_i$ we form a
local quadratic approximation of the energy, or \emph{proxy}:
\begin{align}
\label{eq:quad_approx}
E_i(x) = E(x_i) +   (x - x_i)^T \nabla E(x_i)  + \tfrac{1}{2}  (x - x_i) ^T H_i (x - x_i)
\end{align} where $H_i$ is a symmetric matrix.
Near the solution, if $H_i$ accurately approximates the Hessian we can achieve
fast convergence optimizing this proxy, but it is also critical that
it be stable --- symmetric positive definite (SPD) --- to ensure the proxy
optimization is well-posed everywhere; we also want $H_i$ to be cheap to solve with,
preferring sparser matrices and ideally not having to refactor at each iteration. \\

\bfi{Line Search} Quadratic models allow us to apply linear solvers
to find stationary points $x_i^* = \argmin_x E_i(x)$ of the local
energy approximation. A step
\begin{align}
\label{eq:descent_step_solve}
p_i = x_i^* - x_i = -H_i ^{-1} \nabla E(x_i) 
\end{align}
towards this stationary point then forms a direction for probable energy descent.
However, quadratic models are only locally accurate for nonlinear energies in general,
thus line-search is used to find an improved length $\alpha_i>0$ along $p_i$ to get a new iterate 
\begin{align}
\label{eq:vanilla_step}
x_{i+1} \leftarrow x_i + \alpha_i p_i,
\end{align}
for adequate decrease in nonlinear energy $E$. Of particular concern for
the geometric problems we face is energies which blow up to infinity for
degenerate (flattened) elements: in a given step, the elements where this
may come close to happening rapidly depart from the proxy, and the step size $\alpha_i$ may
have to be very small indeed, see Figure\ \ref{fig:blocked_line_search}, impeding progress globally. \\

\bfi{Termination} Iteration continues until we are able to stop with
a ``good enough'' solution -- but this requires a precise computational
definition. Typically we monitor some quantity which approaches zero
if and \emph{only if} the iterates are approaching a stationary point.
The standard in unconstrained optimization is to check the norm of the
gradient of the energy, which is zero only at a stationary point and
otherwise positive; however, the raw gradient norm depends on the mesh
size, scaling, and choice of energy, which makes finding an appropriate
tolerance to compare against highly problem-dependent and difficult
to automate. \\



\section{Related Work}

\subsection{Energies and Applications}

A wide range of physical simulation and geometry processing
computations are cast as \emph{variational} tasks to minimize
measures of distortion over domains.

To simulate elastic solids with large deformations,
we typically need to minimize hyper-elastic potentials formed
by integrating strain-energy densities over the body. These
material models date back to Mooney~\shortcite{Mooney:1940:ATO} and
Rivlin~\shortcite{Rivlin:1948:SAO}.  Their Mooney-Rivlin and
Neo-Hookean materials, and many subsequent hyperelastic materials,
e.g.~St.~Venant-Kirchoff, Ogden,  Fung~\cite{Bonet:1998:ASO}, are
constructed from empirical observation and
analysis of deforming real-world materials. Unfortunately, all but
a few of these energy densities are nonconvex. This makes their
minimization highly challenging. Constants in these models are
determined by experiment for scientific computing
applications~\cite{Ogden:1972:LDI}, or alternately are directly set
by users in other cases~\cite{Xu:2015:NMD}, e.g., to meet artistic
needs.

In geometry processing a diverse range of energies have 
been proposed to minimize various mapping distortions, 
generally focused on minimizing either measures of
isometric~\cite{Sorkine:2007:ARA,Chao:2010:ASG,Smith:2015:BPW,Aigerman:2015:Seamless,Liu:2008:ALG}
or
conformal~\cite{Hormann:2000:MIPS,Levy:2002:LSC,Desbrun:2002:IPO,Benchen:2008:CFB,Mullen:2008:SCP,Weber:2012:CEQ}
distortion. While some of these energies do not prohibit
inversion~\cite{Sorkine:2007:ARA,Chao:2010:ASG,Levy:2002:LSC,Desbrun:2002:IPO}
many others have been explicitly constructed with nonconvex terms
that guarantee preservation of local
injectivity~\cite{Hormann:2000:MIPS,Aigerman:2015:Seamless,Smith:2015:BPW}.
Other authors have also added constraints to strictly bound distortion, for example,
but we restrict attention to unconstrained minimization --- but note constrained
optimization often relies on unconstrained algorithms as an inner kernel.

Our goal here is to provide a tool to minimize arbitrary energy
density functions as-is. We take as input energy functions provided
by the user, irrespective of whether these energies are custom-constructed
for geometry tasks, physical energies extracted from experiment,
or energies hand-crafted by artists. Our work focuses on the better
optimization of the important \emph{nonconvex} energies whose
minimization remains the primary challenging bottleneck in many
modern geometry and simulation pipelines.  In the following sections,
to evaluate and compare algorithms, we consider a range of challenging
nonconvex deformation energies currently critical in physical
simulation and geometry processing: Mooney-Rivlin
{\bf(MR)}~\cite{Bower:2009:AMO}, Neo-Hookean
{\bf(NH)}~\cite{Bower:2009:AMO},  Symmetric Dirichlet
{\bf(ISO)}~\cite{Smith:2015:BPW}, Conformal Distortion
{\bf(CONF)}~\cite{Aigerman:2015:Seamless}, and Most-Isometric
Parameterizations {\bf(MIPS)}~\cite{Hormann:2000:MIPS}.

\subsection{Energy Approximations} 

Broadly, existing models for the local energy approximation in (\ref{eq:quad_approx}) fall into four rough categories
that vary in the construction of the \emph{proxy}\footnote{Names and notations for $H_i$ vary across the literature
depending on method and application. For consistency, here, across all
methods we will refer to $H_i$ as the \emph{proxy} matrix --- inclusive of cases where it is the actual Hessian
or direct modification thereof.} matrix $H_i$.
\emph{Newton-type} methods exploit expensive
second-order derivative information;
\emph{first-order} methods use only first derivatives and
apply lightweight fixed proxies;
\emph{quasi-Newton} methods iteratively update proxies to approximate
second derivatives using just differences in gradients;
\emph{Geometric Approximation} methods use
more domain knowledge to directly construct proxies which relate to 
key aspects of the energy, resembling Newton-type methods but
not necessarily taking second derivatives.

\bfi{Newton-type} methods generally can achieve the most rapid convergence
but require the costly assembly, factorization and backsolve of new
linear systems per step.  At each iterate Newton's method
uses the energy Hessian, $\nabla^2 E(x_i)$, to form a proxy matrix.
This works well for convex energies like ARAP\ \cite{Chao:2010:ASG},
but requires modification for nonconvex energies\ \cite{Nocedal:2006:Book}
to ensure that the proxy is at least positive semi-definite (PSD).
Composite Majorization (CM), a tight convex majorizer, was recently
proposed as an analytic PSD approximation of the Hessian\
\cite{Shtengel:2017:GOV}. The CM proxy is efficient to assemble but
is limited to two-dimensional problems and just a trio of energies:
ISO, NH and symmetric ARAP.  More general-purpose solutions include
adding small multiples of the identity, and projection of the Hessian
to the PSD cone but these generally damp convergence too much\
\cite{Liu:2016:TRT,Shtengel:2017:GOV,Nocedal:2006:Book}.  More
effective is the Projected Newton (PN) method that projects per-element
Hessians to the PSD cone prior to assembly\ \cite{Teran:2005:RQF}.
Both CM and PN generally converge rapidly in the nonconvex setting
with CM often outperforming PN in the subset of 2D cases where CM
can be applied\ \cite{Shtengel:2017:GOV}, while PN is more general
purpose for 3D and 2D problems.  Both PN and CM have identical
per-element stencils and so identical proxy structures. Despite low
iteration counts they both scale prohibitively due to per-iteration
cost and storage as we attempt increasingly large optimization
problems.

\bfi{First-order} methods build descent steps by preconditioning the gradient with a fixed proxy matrix. These proxies are generally inexpensive to solve and sparse so that cost and storage remain tractable as we scale to larger systems. However, they often suffer from slower convergence as we lack higher-order information.
%
Direct gradient descent, $H_i \leftarrow Id$, and Jacobi-preconditioned gradient descent, $H_i \leftarrow diag(\nabla^2 E(x_i)\big)$ offer attractive opportunities for parallelization~\cite{Wang:2016:DMF,Fu:2015:CLI} but suffer from especially slow convergence due to poor scaling.
%
The Laplacian matrix, $L$, forms an excellent preconditioner, that both smooths and rescales the gradient~\cite{Neuberger:1985:SDA,Martin:2013:ENL,Kovalsky:2016:AQP}. Unlike the Hessian, the Laplacian is a constant PSD proxy that can be pre-factorized once and backsolved separately per-coordinate. Iterating descent with $H_i \leftarrow L$, is the Sobolev-preconditioned gradient descent (SGD) method. SGD was first introduced, to our knowledge, by Neuberger~\shortcite{Neuberger:1985:SDA}, but has since been rediscovered in graphics as the local-global method for minimizing ARAP~\cite{Sorkine:2007:ARA}. As noted by Kovalsky et al.\ \shortcite{Kovalsky:2016:AQP} Local-global for ARAP is exactly SGD.
More recently Kovalsky et al.~\shortcite{Kovalsky:2016:AQP} developed the highly effective Accelerated Quadratic Proxy (AQP) method by adding a Nesterov-like acceleration~\cite{Nesterov:1983:AMO} step to SGD. This improves AQP's convergence over SGD. However, as this acceleration is applied after line search, steps do not guarantee energy decrease and can even contain collapsed or inverted elements --- preventing further progress. More generally, the Laplacian is
constant and so ignores valuable local curvature information ---
we see this issue in a number of examples in Section~\ref{sec:results}
where AQP stagnates and is unable to converge. Curvature can make
the critical difference to enable progress.

\bfi{Quasi-Newton} methods lie in between these two extremes. They
successively, per descent iterate, update approximations of the
system Hessian using a variety of strategies.  Quasi-Newton methods
employing sequential gradients to updates proxies, i.e.  L-BFGS and
variants, have traditionally been highly successful in scaling up
to large systems~\cite{Bertsekas:2016:NOP}. Their updates can be
performed in a compute and memory efficient manner and
can guarantee the proxy is SPD even where the exact Hessian is not.
While not fully second-order, they achieve superlinear convergence, regaining
much of the advantage of Newton-type methods. L-BFGS convergence
can be improved with the choice of initializer. Initializing with the
diagonal of the Hessian\ \cite{Nocedal:2006:Book}, application-specific
structure\ \cite{Jiang:2004:APL} or even the Laplacian\ \cite{Liu:2016:TRT}
can help. However, so far, for geometry optimization problems,
L-BFGS has consistently and surprisingly failed to perform
competitively~\cite{Kovalsky:2016:AQP,Rabinovich:2016:SLI}
\emph{irrespective} of choice of initializer. Nocedal and Wright point
out that the secant approximation can implicitly create a \emph{dense}
proxy, unlike the sparse true Hessian, directly and incorrectly
coupling distant vertices. This is visible as swelling artifacts
for intermediate iterations in Figure \ref{fig:quadratic_compare}.

\begin{figure}[h!]
\centering
\includegraphics[width=0.9\linewidth]{figures/Figure_2/Figure_2}
\caption{\bfi{Line-search blocking.} Barrier terms in nonconvex energies (here we use ISO) of the form $1/g(\sigma)$ can severely restrict step sizes in line searches even when expensive, high-quality methods such as Newton-type methods are applied. {\bf Left column:} descent-direction vector fields, per vertex, in a descent step generated by BCQN, PN and AQP with potential blocking triangles rendered in red. {\bf Right, bottom rows:} after line-search, close to collapsing elements have restricted the global step size for AQP and PN to effectively block progress. {\bf Right, top row:} BCQN's barrier-aware line-search filtering enables progress with significant descent directions.}
\label{fig:blocked_line_search}
\end{figure}

\bfi{Geometric Approximation} methods specifically for geometry optimization
have also been developed recently: SLIM~\cite{Rabinovich:2016:SLI}
and the AKAP preconditioner~\cite{Claici:2017:IAP}. SLIM extends
the local-global strategy to a wide range of distortion energies
while AKAP applies an approximate Killing Vector Field operator as
the proxy matrix. Both require re-assembly and factorization of
their proxies for each iterate. SLIM and AKAP convergence are
generally well improved over SGD and AQP~\cite{Rabinovich:2016:SLI,Claici:2017:IAP}. However, they do not match the convergence quality
of the second-order, Newton-type methods, CM and
PN~\cite{Shtengel:2017:GOV}. SLIM falls well short of both CM and
PN~\cite{Shtengel:2017:GOV}.  AKAP is more competitive than SLIM
but remains generally slower to converge than PN in our testing,
and is much slower than CM.  At the same time SLIM and AKAP stencils,
and so their fill-in, match CM's and PN's; see
Figure~\ref{fig:sparsity_pattern}. SLIM and AKAP thus require the
same per-iteration compute cost and storage for linear solutions
as PN and CM without the same degree of convergence
benefit~\cite{Shtengel:2017:GOV}.

In summary, for smaller systems Newton-type methods have been, till
now, our likely best choice for geometry optimization, while as we
scale we have inevitably needed to move to first-order methods to
remain tractable, while accepting reduced convergence rates and
even the possibility of nonconvergence altogether.  We develop a
new quasi-Newton method, BCQN, that locally blends gradient information
with the matrix Laplacian at each iterate to regain improved and
robust convergence with efficient per-iterate storage and computation
across scales while avoiding the current pitfalls of L-BFGS methods.


\subsection{Line search}
\label{sec:rel_line_search}

Once we have applied the effort to compute a search direction we would like to maximize its effectiveness by taking as large a step along it as possible. Because the energies we treat are nonlinear, too large a step size will actually make things worse by accidentally increasing energy. A wide range of line-search methods are thus employed that search along the step direction for \emph{sufficient} decrease~\cite{Nocedal:2006:Book}. However, when we seek to minimize nonconvex energies on meshes the situation is even tougher. Most (although not all) popular and important nonlinear energies, both in geometry processing and physics, are composed by the sum of rational fractions of singular values of the deformation gradient $W(F) = W(\sigma) = f(\sigma)/g(\sigma)$ where the denominator $g(\sigma) \rightarrow 0$ as $\sigma_i  \rightarrow 0, \forall i \in [1,d]$. Notice that these $1/g(\sigma)$ barrier functions block element inversion. 
Irrespective of their source, these blocking nonconvex energies rapidly increase energy along any search direction that would collapse elements. To prevent this (and likewise the possibility of getting stuck in an inverted state) search directions are capped to prevent inversion of every element in the mesh. This is codified by Smith and Schaeffer's~\shortcite{Smith:2015:BPW} line-search filter, applied before traditional line search, that computes the maximal step size that guarantees no inversions anywhere. 

Unfortunately, this has some serious consequences for progress. Notice that if even a single element is close to inversion this can amputate the full descent step so much that almost no progress can be made at all; see Figure~\ref{fig:blocked_line_search}. This in many senses seems unfair as we should expect to be able to make progress in other regions where elements may be both far from inversion and yet also far from optimality. 
To address these barrier issues we develop an efficient barrier-aware filter that allows us to avoid blocking contributions from individual elements close to collapse while still taking large steps elsewhere in the mesh, see Figure~\ref{fig:blocked_line_search}, top.

\subsection{Termination}
\label{sec:termination_woes}

Naturally we want to take as few iterates as possible while being sure
that when we stop, we have arrived at an accurate solution according to
some easily specified tolerance. The gold-standard in optimization
is to iterate until the gradient is small $\| \nabla E \| < \epsilon$, for
a specified tolerance $\epsilon>0$. This is robust as $\nabla E$ is zero only at stationary
points, and with a bound on Hessian conditioning near the solution can even provide
an estimate on the distance of $x$ to the solution.

\begin{wrapfigure}{r}{0.5\linewidth}
  \begin{center}
    \includegraphics[width=1\linewidth]{figures/Figures_Term/vertex_scaled_grad}
    \caption{Standard termination measures, e.g.\ the vertex-scaled
    gradient norm above, are inconsistent across mesh, energy and scale changes.}
    \label{fig:term_compare_1}
  \end{center}
\end{wrapfigure}
However, an appropriate
value of $\epsilon$ for a given application is highly depend on the mesh, its
dimensions, degree of refinement, energy, etc.
A common engineering rule of thumb to deal with refinement consistency is to instead divide the
$L2$-norm of $\nabla E$ by the number of mesh vertices.  However, as we see in the
inset figure, this normalization does not help significantly, for
example here across changes in mesh resolution for the 2D swirl test;
see Section~\ref{sec:term_results} for more experiments.

To avoid problem dependence, recent geometry optimization
codes generally either take a fixed (small) number of
iterations~\cite{Rabinovich:2016:SLI} or iterate until an
absolute or relative error in energy $\|E_{i+1} -E_i\|$ and/or
position $\| x_{i+1} - x_i \|$ are small~\cite{Shtengel:2017:GOV,Kovalsky:2016:AQP}.
However, experiments underscore there is not yet any method
which always converges satisfactorily in the same fixed number of
iterations, no matter varying boundary conditions, shape difficulty, mesh
resolution, and choice of energy. Measuring the change in
energy or position, absolutely or in relative terms, unfortunately
cannot distinguish between an algorithm converging and simply
stagnating in its progress far from the solution; again, there is not
yet any method which can provably guarantee any degree of progress at every
iterate before true convergence. 
Figure~\ref{fig:aqp_stop} illustrates, on the swirl example, how the
reference AQP implementation declares convergence well before it reaches
a satsifactory solution, when early on it hits a difficult configuration
where it makes little local progress.

To provide reassuring termination criteria in practice and to enable
fair comparisons of current and future geometry
optimization problems we develop a gradient-based stopping criterion
which remains consistent for optimization problems even as we vary
scale, mesh resolution and energy type. This allows us, and future users,
to set a default convergence tolerance in our solver once and leave it
unchanged, independent of scale, mesh and energy. This likewise
enables us to compare algorithms without the false positives
given by non-converged algorithms that have halted due to lack of progress.

\begin{figure}[h]
\centering
\vspace{16pt}
\includegraphics[width=1\linewidth]{figures/Figure_A_12/aqp_bcqn}
\caption{In the 2D swirl example, BCQN with our reliable termination criterion
(\textbf{right}) only stops once it has actually reached a satsifactory solution.
The reference AQP implementation (\textbf{left}) erroneously declares success
early on when it finds two iterates have barely changed, but this is due only
to hitting a difficult configuration where AQP struggles to make progress.}
\label{fig:aqp_stop}
\end{figure}




\section{Blended Quasi-Newton}
\label{sec:blend}

In this section we construct a new quadratic energy proxy which
effectively blends the Sobolev gradient with L-BFGS-style updates
to capture curvature information, avoiding the troubles previous
quasi-Newton methods have encountered in geometry optimization.
Apart from the aforementioned issue of a dense proxy incorrectly
coupling distant vertices in L-BFGS and SL-BFGS, we also find that
the gradients for non-convex energies with barriers can have highly disparate
scales, causing further trouble for L-BFGS. The much smoother
Sobolev gradient diffuses large entries from highly distorted
elements to the neighborhood, giving a much better scaling.
The Laplacian also provides essentially the correct structure for
the proxy, only directly coupling neighboring elements in the mesh,
and is well-behaved initially when far from the solution, thus we
seek to stay close to the Sobolev gradient, as much as possible, while
still capturing valuable curvature information from gradient history.

The standard (L-)BFGS approach exploits the secant approximation
from the difference in successive gradients, 
$y_i = \nabla E(x_{i+1}) - \nabla E(x_{i})$ compared to the
difference in positions $s_i = x_{i+1}-x_i$,
\begin{equation}
\label{eq:proxy_1}
\begin{aligned}
 \nabla^2 E(x_{i+1}) s_i & \simeq  y_i \\
\Rightarrow \quad \nabla^2 E(x_{i+1})^{-1} y_i & \simeq s_i,
\end{aligned}
\end{equation}
updating the current inverse proxy matrix $D_i$ (approximating
$\nabla^2 E^{-1}$ in some sense) so that $D_{i+1}y_i = s_i$.
The BFGS quasi-Newton update is generically
\begin{equation}
\label{eq:BFGS_update}
\mathrm{QN}_i(z, D) = V_i(z)^T D V_i(z) + \frac{s_i s_i^T}{s_i^Tz},  \> \> V_i(z) = I - \tfrac{z s_i^T}{s_i^Tz}.
\end{equation}
We can understand this as using a projection matrix $V_i$ to annihilate
the old $D$'s action on $z$, then adding a positive semi-definite
symmetric rank-one matrix to
enforce $\mathrm{QN}_i(z,D)z = s_i$. Classic BFGS uses
$D_{i+1} = \mathrm{QN}_i(y_i, D_i)$, whereas L-BFGS uses
\begin{equation}
    D_{i+1} = \mathrm{QN}_i(y_i, \tilde{D}_i),
\end{equation}
where $\tilde{D}_i$ has the oldest $\mathrm{QN}$ update removed,
and crucially represents each $D$ as a product of linear operators,
rather than an explicit full matrix. Only the last $m$ $\{s,y\}$ vector pairs (we
use $m=5$) along with the initial $D_1$ (we use the inverse Laplacian,
storing only its Cholesky factor) are stored; application of $D$ is
then just a few vector dot-products and updates along with backsolves for
the Laplacian.

\begin{figure}[h!]
\centering
\includegraphics[width=0.9\linewidth]{figures/Figure_5/mips_compare}
\caption{A 2D shearing deformation stress
test with MIPS energy, comparing methods by plotting iteration vs.\ energy. Both L-BFGS 
as well as inverse Laplacian initialized (SL-BFGS) have slow convergence as previously
reported -- especially when compared to SGD and AQP which use just
the Laplacian. 
At iteration 240 the visualized deformations show both L-BFGS-based methods suffering from
swelling due to inaccurate coupling of distant elements.
Applying our blending model alone (Blended) is highly
effective, while our full BCQN method gives the best results overall.}
\label{fig:quadratic_compare}
\end{figure}

\subsection{Greedy Laplacian Blending}

Experiments show that far from the solution, the Laplacian is often a much
more effective proxy than the L-BFGS secant version: see AQP/SGD vs.\ L-BFGS in Figure \ref{fig:quadratic_compare}.
In particular, the difference in energies $y$ may introduce spurious coupling or
have badly scaled entries near distorted triangles. In this
case if the energy were based on the Laplacian itself (the 
\emph{Dirichlet} energy), the difference in gradients would be the better
behaved $Ls$. This motivates trying the update with $Ls$ instead of $y$,
\begin{equation}
\label{eq:qn_L}
D_{i+1} = \mathrm{QN}_i(L s_i, \tilde{D}_i),
\end{equation}
which will keep us consistent with Sobolev preconditioning, which is very effective
in initial iterations. However, to achieve the
superlinear convergence L-BFGS offers, near the solution we will want
to come closer to satsifying the secant equation, switching to using $y$ instead.

We can thus imagine a blending strategy, which uses
\begin{equation}
z_i=(1-\beta_i)y_i + \beta_i Ls_i
\end{equation}
in $\mathrm{QN}(z_i, \tilde{D}_i)$, with blending parameter $\beta_i \in [0,1]$.
A greedy strategy might choose $\beta_i$ to scale $Ls_i$ to be as close to $y_i$ as possible,
\begin{equation}
\label{eq:BQCN_proj}
\beta_i = \argmin_{\beta\in[0,1]} \| y_i - \beta L s_i \|^2,
\end{equation}
in other words using the projection of $y_i$ onto $Ls_i$. This comes as close as possible to
satsifying the secant equation with $Ls_i$, then makes up the rest with $y_i$. 
Solving (\ref{eq:BQCN_proj}) gives 
\begin{align}
\label{eq:BCQN_proj2}
\beta_i = \mathrm{proj}_{[0,1]} \left( \frac{{y_i}^T L s_i}{ \|L s_i\|^2 } \right).
\end{align}
Observe that when $Ls$ is roughly aligned with the gradient jump $y$ , but $y$ is as large or larger, $\beta$ grows and Laplacian smoothing increases --- as we might
hope for initially when far from the solution, where the Sobolev gradient is most effective.
When the energy Hessian diverges strongly from from the Laplacian
approximation, perhaps when the cross-terms between coordinates missing from the scalar Laplacian
are important, then $\beta$ will decrease, so that contributions from $y_i$ again grow.
Finally, as the gradient magnitudes decreases close to the solution,
$\beta$ will similarly decay, ideally regaining the
superlinear convergence of L-BFGS near local minima.


\subsection{Blended Quasi-Newton}

With the blending projection (\ref{eq:BCQN_proj2}) in place we experimented with a range of rescalings in hopes of 
further improving efficiency and robustness. After extensive testing we have so far found the following scaling
to offer the best performance:
\begin{align}
\begin{split}
\beta_i =  \mathrm{proj}_{[0,1]} \Big(\frac{ \mathrm{normest}(L) {y_i}^T L s_i}{A(V,T)} \Big), \\
\text{with} \> \> A(V,T) = \Big(\sum_{t \in T} a_t \Big)^{\frac{2 (d - 1)}{d}}.
\end{split}
\end{align}
Here $\mathrm{normest}(L)$ is an efficient estimate of the matrix 2-norm using power iteration,
and $A(V,T)$ is a \emph{constant} normalizing term with appropriate dimensions and so no longer
has the same potential concern for sensitivity in the denominator when $Ls$ is small but $s$ isn't. Both terms are computed just
once before iterations begin and reused throughout. 

As mentioned, we initialize the inverse proxy with $D_1=L^{-1}$,
thus starting with Laplacian preconditioning. With line search
satisfying Wolfe confitions our proxy remains SPD across all
steps~\cite{Nocedal:2006:Book}. Each step jointly updates $D_i$
using the standard two-loop recursion and finds the next descent
direction $s_i = -D_i \nabla E(x_i)$.
Figure~\ref{fig:quadratic_compare} illustrates the gains possible from
blended quasi-Newton compared to both standard L-BFGS and Sobolev gradient algorithms, while then applying our barrier-aware filter, derived in our next section gives best results with our
full BCQN algorithm.


\section{Barrier-Aware Line Search Filtering}
\label{sec:constraints}

As mentioned in Section~\ref{sec:rel_line_search} and shown
in Figure~\ref{fig:blocked_line_search}, the barrier factor $1/g(\sigma)$
in nonconvex energies typically dominates step size in line search.
Even a single element that is brought close to collapse by the
descent direction, $p_i$, can restrict the line search step size
severely.  The computed step size $\alpha_i$ then scales $p_i$
\emph{globally} so that all elements, not just those that are going
to collapse along $p_i$, are prevented from making progress. To avoid
this, a natural strategy suggests itself: when the descent direction would cause
elements to degenerate towards collapse along the full step,
rather than simply truncating line search as in Smith and
Schaefer\ \shortcite{Smith:2015:BPW}, we filter collapsing contributions
from the search direction prior to line search.
We call this strategy \emph{barrier-aware line search filtering}.

\subsection{Curing line search}

\begin{figure}[t!]
\centering
\includegraphics[width=1\linewidth]{figures/Figure_3/Figure_3}
\caption{
\bfi{Direct filtering does not work.} Zeroing out inverting components of descent directions
or gradients makes the search direction inconsistent with the objective and so prevents convergence,
leading to termination at poor solutions (a) and (b).
{\bf Left:} we initialize a 2D shear deformation,
constraining the top of a bar to slide rightwards.
{\bf Middle:} direct filtering of the descent direction (a) and
the gradient (b) allow large descent steps forward unblocked
from the contributions of close-to-collapsed elements. However, this
results in termination at shapes that that do not satisfy optimality of the original minimization.
{\bf Right:} compare to an optimal solution for this problem (c)
obtained with BCQN.
 }
\label{fig:filter_fail}
\end{figure}

Figure~\ref{fig:filter_fail} illustrates how the simplest possible filters,
zeroing out contributions from nearly-inverted elements
in either the search direction (\ref{fig:filter_fail}a)
or the gradient before Laplacian smoothing (\ref{fig:filter_fail}b)
fail. We must be able to make progress in nearly-inverted elements
when the search direction can help, or there is no hope for reaching
the actual solution; simple zeroing fails to converge, which is
no surprise as it in essence is arbitrarily manipulating the
target energy, changing the problem being solved.
We instead want to \emph{augment} the original optimization problem
in a way which doesn't change the solution, but gives us a tool to
safely deal with problem elements so the search direction $p_i$ doesn't
cause them to invert, ideally with a small fixed cost per iteration.

\subsection{One-Sided Barriers in Geometry Optimization}

Element $t \in T$ is inverted at positions $x$ precisely when the orientation function
$a_t(x) = \det(F_t(x))$ is negative. Concatenating over $T$, the global vector-valued function for element
orientations is then
\begin{equation}
a(\cdot) = \big(a_1(\cdot), ..., a_m(\cdot) \big)^T.
\end{equation}
As long as $a(x) > 0$, no element is collapsed or inverted, and the energy remains finite.
Note, however, many energies are also finite for inverted elements $a_t(x)<0$, only blowing up
at collapse $a_t(x)=0$, so technically there may exist local minima where $\nabla E(x^*)=0$
yet some elements are inverted. Generally, practitioners wish to rule these potential solutions
out however, with two implicit but so far informal assumptions of locality: 
the initial guess is not inverted, $a(x_1)>0$, and that the solver follows a path
which never jumps through the barrier to inversion. 

We formalize these requirements in the optimization as
\begin{equation}
\label{eq:hard_constr_E}
\min_x \{E(x) \ : \  a(x) \geq 0 \}.
\end{equation}
Adding the constraint $a(x) \geq  0$ now explicitly restricts our
optimization to noninverting deformations but otherwise leaves the
desired solution unchanged. (See Supplement, Section 1, for proof.)

\subsection{Iterating Away from Collapse}

With problem statement (\ref{eq:hard_constr_E}) in place, we now exploit it in curing the search direction from
collapsing elements. At each iterate $i$, form the projection  
\begin{align}
\label{eq:p_project}
 \min_p \left\{ \| p + D_i \nabla E(x_i) \|_2^2 \> : \> a(x_i) + \nabla a(x_i)^T p \geq 0 \right\} 
\end{align}
of the predicted descent direction $\tilde{p}_i = -D_i \nabla E(x_i)$ onto 
the subset satisfying a linearization of the no-collapse condition.
Satisfying (\ref{eq:p_project}) exactly would ensure that projected
directions would not locally generate collapse and likewise preserve
symmetry~\cite{SKVTG2012}. However, its exact solution is neither
necessary nor efficient. Instead, we construct an approximate
solution to (\ref{eq:p_project}) as a filter that \emph{helps}
avoid collapse, preserves symmetry, and guarantees a low cost for
computation for all descent steps. 

Strict convexity of the projection guarantees that a minimizer $p^*$ of (\ref{eq:p_project}) is given by the
KKT\footnote{Here and in the following $\lambda = (\lambda_1, ..  ,\lambda_m)^T \in R^m$
is a Lagrange multiplier vector and $\vc x \perp \vc y$ is the \emph{complementarity condition}
$y_t z_t = 0,\ \forall t$.} conditions~\cite{Bertsekas:2016:NOP}
\begin{align}
\label{eq:kkt_prog1}
p^*+ D_i \nabla E(x_i) - \nabla a(x_i) \lambda^* = 0, \\
\label{eq:kkt_prog2}
0 \leq \lambda^* \perp a(x_i) + \nabla a(x_i)^T p^* \geq 0.
\end{align}
We simplify with $C_i = \nabla a(x_i)$, $M_i = \nabla a(x_i)^T \nabla a(x_i)$,
and $b_i = a(x_i)$, then form the Schur complement of the above to arrive at an equivalent
Linear Complementarity Problem (LCP)~\cite{Cottle:2009}
\begin{align}
\label{eq:LCP_proj}
\begin{split}
0 \leq \lambda^* \perp M_i \lambda^* + C_i^T  p_i + b_i \geq 0,
\end{split}
\end{align}
and then a damped Jacobi splitting
with $M_i = \omega^{-1}  T_i +  (M_i - \omega^{-1} T_i)$,
diagonal $T_i = \mathrm{diag}(M_i)$ and damping parameter
$\omega \in (0,1)$. This gives us an iterated LCP ranging over
iteration superscripts $j$,
\begin{align}
\label{eq:LCP_proj_split}
\begin{split}
0 \leq \lambda^{j+1} \perp \omega^{-1} T_i \lambda^{j+1} + M_i \lambda^j - \omega^{-1} T_i \lambda^j + C_i^T  p_i  + b_i \geq 0.
\end{split}
\end{align}

\begin{figure}[t!]
\centering
\includegraphics[width=0.9\linewidth]{figures/Figure_4/Figure_4}
\caption{
\bfi{Line search filtering.} {\bf Bottom:} We optimize a
uv-parameterization with the MIPS energy to consider line search
filtering behavior, plotting energy (y-axis) against iteration
counts for a range of methods. Just adding our barrier-aware line
search filtering alone to SGD improves its convergence by
well over an order of magnitude, and almost an order of magnitude
over AQP as well as plain L-BFGS and SL-BFGS. BCQN with blending
and line search filtering improves convergence even further.
{\bf Top:} a comparison of the embeddings and texture-maps
for AQP and SGD with the filter at the $40^\textrm{th}$
iterate.
}
\label{fig:combined_method}
\end{figure}


\subsection{Line Search Filtering}

Each iteration of the splitting (\ref{eq:LCP_proj_split}) simplifies
to the damped projected Jacobi (DPJ) update\footnote{We use the convention $[\cdot]^+ = \max[0, \cdot]$.}
\begin{align}
\label{eq:DPJ}
\lambda^{j+1} \leftarrow \left[\lambda^j - \omega T^{-1}\big(C_i^T (C_i \lambda^j) + c_i\big)\right]^+,
\end{align}
with constant $c_i = C_i^T  p_i + b_i$. Here each of the $m$ entries in $\lambda^{j+1}$
can be updated in parallel (unlike with Gauss-Seidel iteration).
As $M_i$ is PSD this iteration process converges to
(\ref{eq:LCP_proj})~\cite{Cottle:2009} and so to (\ref{eq:p_project}).
We do not seek a tight solution, however, as we just want to be sure the worst blocks
to line search are filtered away. Therefore we initialize with $\lambda^0=0$ to avoid
unnecessary perturbation, use a coarse termination tolerance
for DPJ (see below), and never use more than a maximum of 20 DPJ iterations.

At each DPJ iteration $j$ we check for termination with an LCP
specialized measure, the Fischer-Burmeister
function~\cite{Fischer:1992:ASN}
$\mathrm{FB}(\lambda^j, M_i  \lambda^j  +  c_i)$ evaluated as
\begin{align}
\label{eq:FB}
\mathrm{FB}(a,b) = \sqrt{\sum_{k \in [1,m]}  \left(a_k + b_k - \sqrt{a_k^2 + b_k^2} \right)^2}.
\end{align} 
As we initialize with $\lambda^0 = 0$, when $p_i$ is non-collapsing
$\mathrm{FB} = 0$, and thus no line search filtering iterations
will be applied. Likewise, we stop iterations whenever the $\mathrm{FB}$
measure is roughly satisfied by either a relative error of $<10^{-3}$
or an absolute error $<10^{-6}$.

Filtering thus applies a fixed maximum upper limit on computation
and performs no iterations when not necessary. Upon termination of
DPJ iterations, plugging our final $\lambda$ into (\ref{eq:kkt_prog1})
we obtain our update to form the line search filtered descent
direction
\begin{align}
p^\ell_i = p_i  + C_i \lambda.
\end{align}
As Figure~\ref{fig:blocked_line_search} shows, despite the rough
nature of the filter, it can make a dramatic difference in line search.


\section{Termination Criteria}
\label{sec:term}

Every iterative method for minimizing an objective function $E(x)$
must incorporate stopping criteria: when should an approximate
solution be considered good enough to stop and claim success?
Clearly, in the usual case where the actual minimum value of $E(x)$
is unknown, basing the test on the current value of $E(x_i)$ is
futile. As noted in Section~\ref{sec:termination_woes},
stopping when successive iterates are closer than some tolerance is
vulnerable to false positives (halting far from a solution), as is
using a fixed number of iterations. Although monitoring $\|\nabla E\|$
is robust, each individual problem may need a different tolerance to
define a satisfactory solution even when normalized by number of vertices:
see Figures\ \ref{fig:term_compare_1} and\ \ref{fig:term_compare_2}.
We thus propose a new way to derive and construct an appropriate,
roughly problem-independent, relative scale for a gradient-based
measure for a stopping criterion.

\subsection{Characteristic Gradient Norm}

All energies we consider are summations of per-element energy densities $W(\cdot)$ computed
from the deformation gradient $F_t(x)$ and weights $a_t$, in each element $t$, as per equation (\ref{eq:obj}). 
To simplify the following we can then evaluate energy densities on
the vectorized deformation gradient as $W\big(vec(F_t)\big) =  W(G_t
x)$, where $G_t$ is the linear gradient operator for element $t$.
The full energy gradient is then
\begin{equation}
    \nabla E(x) = \sum_{t \in T} a_t  G_t^T \nabla W(G_t x).
\end{equation}
We wish to generate a ``characteristic'' value we can compare this gradient to meaningfully, with the same
dimensions; we will do this with each component of the above summation separately.

First observe that the deformation gradient, $F_t$, the argument to $W$, is dimensionless and therefore
$\nabla W$ has the same dimensions as $W$, and even as the element Hessian $\nabla^2 W$. For the simplest
quadratic energy densities, this Hessian has the attractive property of being constant; we thus choose
to use the 2-norm of the Hessian, evaluated about the deformation gradient at rest ($F_t=I$), to get
a representative value for $\nabla W$:
\begin{equation}
    \langle W \rangle = \|\nabla^2 W(I)\|_2.
\end{equation}

Second, note that the $i^\textrm{th}$ part of $G_t$ for a triangle
(respectively tetrahedra) $t$ containing vertex $i$ will attain its
maximum value for fields which are constant along the opposing edge
(triangle) and that value will be the reciprocal of the altitude.
Up to a factor of $2$ ($3$), this is the length (area) of the
opposing edge (triangle) divided by the rest area (volume), of the
element, i.e.\ $a_t$. Summing over all incident elements, weighted
by $a_t$, we arrive at a characteristic value for vertex $i$ of
$\ell_i$ equalling the perimeter (surface) area of the one-ring of
vertex $i$. We compute this value for all vertices, giving us the
vector $\ell(V,T) = (\ell_1, ..., \ell_n)^T \in \R^n$, with one
scalar entry per vertex.

The product of our energy and mesh values together form the characteristic value for the norm of the gradient
\begin{equation}
    \langle W \rangle \| \ell(V,T) \|,
\end{equation}
where we take the same vector norm as that with which we evaluate $\|\nabla E(x)\|$; we use the 2-norm in all our
experiments. For all methods we stop iterating when
\begin{equation}
    \|\nabla E(x)\| \leq \epsilon \langle W \rangle \| \ell(V,T) \|,
\end{equation}
given a dimensionless tolerance $\epsilon$ from the user, which is
now essentially mesh- and energy-independent. See Figures\
\ref{fig:term_compare_1}, \ref{fig:ours_yaron} and\ \ref{fig:term_compare_2} as
well as our experimental analysis in Section\ \ref{sec:results} for evaluation.



\section{The BCQN Algorthim}
\label{sec:alg}

\begin{algorithm}[h!]
\label{alg:BCQN}
\caption{Blended Cured Quasi-Newton (BCQN)}

\textbf{Given:} $x_1$, $E$, $\epsilon$  \hspace{10pt} 

\textbf{Initialize and Precompute:}

\hspace{30pt} $s = \epsilon \langle W \rangle \| \ell(V,T) \| $  \hspace{10pt} // Characteristic termination value (\S\ref{sec:term})

\hspace{30pt} $L, \> \> D \leftarrow L^{-1}$ \hspace{10pt} // Initialize blend model  (\S\ref{sec:blend})

\hspace{30pt} $g_1 = \nabla E(x_1), \> \> i = 1$ 
 
$\textbf{while}$ $ \|g_i\| > s$ $\textbf{do}$\hspace{10pt}// Termination criteria  (\S\ref{sec:term})\\

\hspace{10pt} $p \leftarrow -D g_i$ \hspace{10pt}//  Precondition gradient (\S\ref{sec:blend})\\

\hspace{10pt} // Assemble for DPJ iterations (\S\ref{sec:constraints}):

\hspace{20pt} $C \leftarrow \nabla a(x_i)$ 

\hspace{20pt} $M \leftarrow C^T C, \> \> c \leftarrow C^T  p + a(x_i)$

\hspace{20pt} $E \leftarrow \mathrm{diag}(M)^{-1}, \> \> \lambda \leftarrow 0$ 

\hspace{10pt} $\mathit{fb} \leftarrow \textrm{FB}(\lambda, M \lambda  +  c)$ \hspace{10pt}// LCP residual (Equation (\ref{eq:FB}) in \S\ref{sec:constraints})

\hspace{10pt} \textbf{for} $j = 1$ \textbf{to} \text{20} \hspace{10pt}// Line-search preconditioning  (\S\ref{sec:constraints})

\hspace{20pt}\textbf{if} $\mathit{fb} < 10^{-6}$ \textbf{then} \hspace{3pt} \textbf{break} \hspace{5pt} \textbf{end if}

\hspace{20pt}$\mathit{fb} \leftarrow \mathit{fb}_\mathrm{next}$ 

\hspace{20pt} $\lambda \leftarrow [\lambda - \tfrac{1}{2} E \big(C^T (C \lambda) + c\big)]^+$ // Parallel project  (\S\ref{sec:constraints})

\hspace{20pt}$\mathit{fb}_\mathrm{next} \leftarrow \textrm{FB}(\lambda, M  \lambda  +  c)$  

\hspace{20pt}\textbf{if}  $|\mathit{fb} - \mathit{fb}_\mathrm{next}| / \mathit{fb} < 10^{-3}$ \textbf{then} \hspace{3pt} \textbf{break} \hspace{5pt} \textbf{end if}

\hspace{10pt}\textbf{end for}

\hspace{10pt} $p^\ell \leftarrow p + C \lambda$ \hspace{10pt}// Line-search filtered search direction  (\S\ref{sec:constraints})

\hspace{10pt}$\alpha \leftarrow \text{LineSearch}(x_i, p^\ell, E)$ \hspace{10pt} // Line search (\S\ref{sec:blend})

\hspace{10pt}$x_{i+1} = x_i + \alpha p^\ell$  \hspace{10pt} // Descent step (\S\ref{sec:blend})

\hspace{10pt} $g_{i+1} = \nabla E(x_{i+1})$

\hspace{10pt}$D \leftarrow \text{Blend}(D, L, x_{i+1}, x_i, g_{i+1}, g_i)$\hspace{10pt}// BCQN blending update (\S\ref{sec:blend})

\hspace{10pt}$i \leftarrow i+1$

$\textbf{end while}$

\end{algorithm}

%%%%%%%%%%%%%%%%%%%%%%%%%%%%%%%%%%%%%%%%%

\begin{figure}
\centering
\includegraphics[width=1\linewidth]{figures/Figure_A_8/sparsity_pattern}
\caption{\bfi{Sparsity Differences in Proxies.} {\bf Left:} The scalar Laplacian (top) is smaller \emph{and} sparser than the
Hessian and its approximations (bottom) used in CM, PN, SLIM and AKAP. {\bf Right:} This results in a much cheaper factorization and solve for the Laplacian; it
is applied in both BCQN and AQP independently to each coordinate.}
\label{fig:sparsity_pattern}
\end{figure}

Algorithm 1 contains our full BCQN algorithm in pseudocode. The
dominant cost, for both memory and runtime, is the Laplacian solve
embedded in the application of $D$, which again is not stored as a
single matrix, but rather is a linear transformation involving a
few sparse triangular solves with the Laplacian's Cholesky factor
and outer-product updates with a small fixed number of L-BFGS history
vectors. Recall that we separately solve for each coordinate with
a scalar Laplacian, not using a larger vector Laplacian on all
coordinates simultaneously; this also exposes some trivial parallelism.
Apart from the Laplacian, all steps are either linear (dot-products,
vector updates, gradient evaluations, etc.) or typically sublinear
(DJP assembly and iterations, which only operate on the small number
of collapsing triangles, and again are easily parallelized).

As Lipton et al.\ proved \shortcite{lipton:1979:gnd},
the lower bounds for Cholesky factorization on a two-dimensional
mesh problem with $n$ degrees of freedom are $O(n \log n)$ space
and $O(n^{3/2})$ sequential time, and in three-dimensional problems 
where vertex separators are at least $O(n^{2/3})$, their Theorem
10 shows the lower bounds are $O(n^{4/3})$ space and $O(n^2)$
sequential time. On moderate size problems running on current
computers, the cost to transfer memory tends to dominate arithmetic,
so the space bound is more critical until very large problem sizes
are reached.


\subsection{Comparison with Other Algorithms}

The per-iterate performance profile of AQP is most similar to BCQN:
it too is dominated by a Laplacian solve. The only difference is
the extra linear and sublinear work which BCQN does for the quasi-Newton
update and the barrier-aware filtering; even on small problems, this
overhead is usually well under half the time BCQN spends, and as the
next section will show, the improved convergence properties of BCQN
render it faster.

The second-order methods we compare against, PN and CM, as well as the more approximate
proxy methods, SLIM and AKAP, all use a fuller stencil which couples coordinates.
The same asymptotics for Cholesky apply, but whereas AQP and BCQN can solve a scalar $n\times n$ Laplacian
$d$ times (once for each coordinate, independently), these other methods
must solve a single denser $nd\times nd$ matrix, with $d^2$ times more
nonzeros: see Figure~\ref{fig:sparsity_pattern}.
Moreover, the matrix changes at each iteration and must be refactored,
adding substantially to the cost: factorization is significantly slower
than backsolves.



\section{Evaluation}
\label{sec:results}

\subsection{Implementation}

We implemented a common test-harness code to enable the consistent
evaluation of the comparitive performance and convergence behavior
of SGD, PN, CM, AQP, L-BFGS and BCQN across a range of energies and
geometry optimization tasks including parameterization as well as
2D and 3D deformations, where these methods allow. For AQP this extends
the number of energies it can be tested with, while more generally
providing a consistent environment for evaluating all methods. We
hope that this code will also help support the future evaluation
and development of new methods for geometry optimization.

The main body of the test code is in MATLAB to support rapid
prototyping.  All linear system solves are performed with MATLAB's
native calls to SuiteSparse~\cite{Chen:2008:ACS} with additional
computational-heavy modules, primarily common energy, gradient and
iterative LCP evaluations, implemented in C++.  As linear solves
are the bottleneck in all methods covered here, an additional
speed-up to all methods is possible with
Pardiso~\cite{Petra:2014:AAI,Petra:2014:RTS} in place of SuiteSparse;
however, as discussed in Section~\ref{sec:pardiso} this does not change
the relative merits of the methods, and would add an additional external
dependency to the test code. For verification we also confirm that
iterations in the test-harness AQP and CM implementations match the
official AQP~\cite{Kovalsky:2016:AQP} and CM~\cite{Shtengel:2017:GOV}
codes.

All experiments were timed on a four-core Intel
3.50GHz CPU. We have parallelized the damped Jacobi LCP iterations
with Intel TBB; with more cores the overhead reported below for LCP
iterations is expected to diminish rapidly.
For all UV parameterization problems we compute 
initial locally injective embeddings via the initialization code
from Kovalsky et al.~\shortcite{Kovalsky:2016:AQP}. On rare occasions
this code fails to find a locally injective map, so we
then revert to a Tutte embedding as a failsafe using the initialization
code from Rabinovich et al.~\shortcite{Rabinovich:2016:SLI}. To
enforce Dirichlet boundary conditions, i.e.\ positional constraints,
we use a standard subspace projection~\cite{Nocedal:2006:Book}, i.e.\
removing those degrees of freedom from the problem.
When line search is employed we first find a
maximal non-inverting step size with Smith and
Schaefer's method~\shortcite{Smith:2015:BPW}, followed by standard
line search with Armijo and curvature conditions.

\subsection{Termination} 
\label{sec:term_results}

To evaluate termination criteria behavior we first instrumented two
geometry optimization stress-test examples: the \emph{Swirl}
deformation~\cite{Chen:2013:PSI} and the \emph{Hilbert curve} UV
parametrization~\cite{Smith:2015:BPW}. We run both examples to
convergence ($10^{-6}$ using our characteristic gradient) reaching
the final target shapes for each. Within these optimizations we
record the 2-norm of gradient, the vertex-normalized 2-norm of
gradient, the
relative error measure~\cite{Kovalsky:2016:AQP,Shtengel:2017:GOV}
and our characteristic gradient norm for all iterations.

\begin{figure}[t]
\vspace{3mm}
\centering
\includegraphics[width=1\linewidth]{figures/Figures_Term/Swirl_Term_Compare}
\caption{\bfi{Termination criteria comparison.} {\bf Left to right:}
We find key points in the sequential progress of the optimized mesh
in the Swirl optimization (ISO energy) example at regular intervals of $10\times$ decrease
in our characteristic norm. We compare with the relative error measures at these same points.}
\label{fig:swirl_term_compare}
\vspace{3mm}
\end{figure}

Figure~\ref{fig:swirl_term_compare} shows the Swirl mesh
obtained during BCQN iteration at regular intervals of $10\times$
decrease in our characteristic norm. Observe
that they correspond to natural points of progress;
see our supplemental video of the entire optimization sequence
for reference. For comparison we also provide the corresponding relative
error measures, which varies much less steadily.

In Figure~\ref{fig:ours_yaron} we compare
termination criteria more closely for a UV parametrization
problem, the Hilbert curve example. We plot our characteristic gradient norm
(blue) and the relative energy error~\cite{Kovalsky:2016:AQP,Shtengel:2017:GOV}
(orange) as BCQN proceeds. Note
that the characteristic gradient norm provides consistent decrease
corresponding to improved shapes and so provides a
practical measure of improvement. The local error in energy, on the
other hand, varies greatly, making it impossible to judge how much
global progress has been made towards the optimum.

\begin{figure}[h!]
\vspace{3mm}
\centering
\includegraphics[width=1\linewidth]{figures/Figure_A_4/ours_yaron}
\caption{\bfi{Measuring improvement.} 
Solving a UV parametrization of the Hilbert curve with BCQN, we plot our
characteristic gradient norm in \blue{blue} and the relative energy
error in \orange{orange} as the method proceeds, on a logarithmic scale.
Iterates are shown at decreases in the characteristic gradient norm
by factors of 10, illustrating its efficacy as a global measure of progress,
while the relative energy error measures only local changes with little
overall trend.}
\label{fig:ours_yaron}
\vspace{3mm}
\end{figure}

Figure~\ref{fig:term_compare_2} illustrates consistency across
changing tolerance values, mesh resolutions, and scales.
example.  We show the iterates at measures
$10^{-3}$, $10^{-4} $and $10^{-5}$ for both our characteristic gradient
norm and the raw gradient norm, for meshes with varying refinement and
varying dimension (rescaling coordinates by a large factor). Similar
to Figure~\ref{fig:term_compare_1} comparing the vertex-normalized gradient
norm, there are large disparities for the raw gradient norm, but our
characteristic gradient norm is consistent.

\begin{figure}[h!]
\vspace{3mm}
\centering
\includegraphics[width=1\linewidth]{figures/Figures_Term/grad_l2_geo_compare}
\caption{\bfi{Termination criteria comparison across mesh refinement and scale.}
{\bf Left and right}: we show the Swirl optimization
when our characteristic norm (left) and the standard gradient
norm (right) reach $10^{-3}, 10^{-4}$ and $10^{-5}$.
{\bf Top to bottom}: the rows show
optimization with a coarse mesh, a fine mesh, and the
same fine mesh uniformly scaled in dimension by $100\times$.
Note the consistency across mesh resolution and scaling
for our characteristic norm and
the disparity across the standard gradient norm.}
\label{fig:term_compare_2}
\end{figure}

\paragraph{Tolerances} The Swirl and Hilbert curve examples are
both extreme stress tests that require passing through low curvature
regions to transition from unfolding to folding; see e.g.,
Figure~\ref{fig:swirl_term_compare} above and our videos. For these
extreme tests we used a tolerance of $10^{-6}$ for our characteristic
gradient norm to consistently reach the final target shape.
However, for most practical geometry optimization tasks such a tolerance
is excessively precise. In experiments across a wide range of
energies and UV parametrization, 2D and 3D deformation tasks,
including those detailed below, we found that
$\|\nabla E(x)\| \leq 10^{-3} \langle W \rangle \| \ell(V,T) \|$
consistently obtained good-looking solutions with essentially no
further visible (or energy value) improvement possible. We
argue this is a sensible default except in pathological examples.
For all examples discussed here and below, with the exception of the
Swirl and the Hilbert curve tests, we thus use $\epsilon=10^{-3}$ for
testing termination.

\begin{figure*}[h!]
\centering
\includegraphics[width=1\linewidth]{figures/Figure_A_3/iso_2d_scale_table}
\caption{\bfi{UV Parameterization Scaling, Timing and Sparsity.}
Performance statistics and memory use for increasing mesh
sizes up to 23.9M triangles, comparing BCQN with AQP, PN
and CM. For the Gorilla UV parametrization with ISO energy we
repeatedly double the mesh resolution and, for each method, report
number of iterations to convergence (characteristic norm $< 10^{-3}$),
wall-clock time (seconds) to convergence, and the nonzero fill-in
for the linear systems solved by each method. We use \red{\bf*} to
indicate out-of-memory failure for matrix factorization; see
\S\ref{sec:results} for discussion. Also note that stencils for CM
and PN are identical (differing only by actual entries) while AQP
and BCQN both solve with the same smaller scalar Laplacian.}
\label{fig:2d_scale_table}
\end{figure*}


\subsection{Newton-type methods}
\label{sec:newton-type}

While Newton's method, on its own, handles convex energies like ARAP well~\cite{Chao:2010:ASG}
it is insufficient for nonconvex energies: modification of the Hessian is
required~\cite{Shtengel:2017:GOV,Nocedal:2006:Book}. Here we examine
the convergence, performance and scalability of Projected Newton (PN)~\cite{Teran:2005:RQF},
a general-purpose modification for nonconvex energies, and
CM~\cite{Shtengel:2017:GOV}, a more recent convex majorizer currently restricted
to 2D problems and a trio of energies (ISO, Symmetric ARAP and NH), and compare them with 
AQP and BCQN.
For the 2D parameterization problems in Figure~\ref{fig:2d_scale_table}
we can compare all four methods while for the 3D deformation
problems in Figures~\ref{fig:3d_scale_table} and \ref{fig:3d_large_defo}
CM is not applicable.

As we increase the size of the 2D problem by mesh refinement in
Figure~\ref{fig:2d_scale_table}, both CM and PN maintain low and almost constant
iteration counts to converge, with CM enjoying an advantage for larger problems;
in Figure~\ref{fig:3d_scale_table} 

Figures~\ref{fig:2d_scale_table}, \ref{fig:3d_scale_table}, and
\ref{fig:3d_large_defo} examine the scaling behavior of the various
methods under mesh refinement, for 2D parameterization and 3D deformation.
The Newton-type methods PN and CM (when applicable) maintain low
iteration counts that only grow slowly with increasing mesh size;
from the outset BCQN and AQP require more iterations, though the iteration count
also grows slowly for BCQN. Nonetheless, BCQN is the fastest across
all scales in each test as its overall cost per iteration remains much lower.
BCQN iterations require no re-factorizations (which scales poorly, particularly
in 3D, as discussed in Section \ref{sec:alg}) and only solves
smaller and sparser scalar Laplacian problems per coordinate compared
to the larger and denser system of CM and PN. This advantage for BCQN only
increases as problem size grows; indeed, for the largest problems BCQN
succeeded where CM and PN ran out of memory for factorization.

\subsection{A Note on Solving Proxies and Pardiso} 
\label{sec:pardiso}

Recent methods including CM have taken advantage of the efficiencies
and optimizations provided by the Pardiso solver.
While this can improve runtime of the factorization and backsolves
by a constant factor, it cannot change the asymptotic lower bounds on complexity;
the sparse matrix orderings in both SuiteSparse and Pardiso already appear
to achieve the bound on typical mesh problems.
In tests on our computer, across a large range of scales in two and
three dimensions, we found Pardiso was occasionally slower than
SuiteSparse but usually 1.4 to 3 times faster, and at most to 8.1 times faster
(for backsolving with a 3D scalar Laplacian).

Individual iterates of AQP have the same overall efficiency as BCQN (dominated
by the linear solves); switching to Pardiso leaves the relative performance of the two methods
unchanged. While CM and PN are even more dependent on the efficiency of the linear solver,
due to more costly refactorization each step, the same speed-ups possible with Pardiso also apply to
BCQN, so again there is no significant change in relative performance between the
methods.

\begin{figure}[h!]
\vspace{3mm}
\centering
\includegraphics[width=1\linewidth]{figures/Figure_A_2/iso_3d_scale_table}
\caption{\bfi{Three-Dimensional Deformation Scaling, Timing and
Sparsity.} Performance statistics and memory use for increasing
mesh sizes up to 7.8M tetrahedra, comparing BCQN with AQP
and PN. (CM does not extend to 3D.) We initialize a bar with a
straight rest shape to start in a tightly twisted shape, constraining
both ends to stay fixed and then optimize over increasing resolutions.
For each method we report number of iterations to convergence
(characteristic norm $< 10^{-3}$), wall-clock time (seconds) to
convergence, and the nonzero fill-in for the linear system solved
by each method. We use \red{\bf*} to indicate out of memory for the
computation on our test system; see \S\ref{sec:results} for discussion.
}
\label{fig:3d_scale_table}
\end{figure}

\begin{figure}[h!]
\centering
\includegraphics[width=1\linewidth]{figures/Figure_A_7/3d_defo}
\caption{\bfi{Armadillo Deformation Test}. We compare three-dimensional
deformation optimizations of a 1.5M element tetrahedral mesh of the
T-pose armadillo with BCQN and PN. We constrain the armadillo's
feet to rest position, its hands to touch the ground and use the
LBD method to create a locally injective
initialization for the solvers. Here BCQN requires 393 iterations
to converge while PN converges in just 9. However, as BCQN is much
cheaper and more scalable per iterate it takes only 4,148 seconds,
while PN spends 13,447 seconds.}
\label{fig:3d_large_defo}
\end{figure}

\begin{figure}[h!]
\centering
\includegraphics[width=1\linewidth]{figures/Figure_6A/uv_table_1}
\caption{\bfi{UV parameterization.} {\bf Top row:}  3D meshes  for
UV parametrization with ISO, MIPS, and CONF distortion energies.
{\bf Middle two rows:} converged maps and texturing from BCQN on
ISO examples. {\bf Bottom:} for each method / problem pair we report
number of iterations to convergence (characteristic norm $< 10^{-3}$)
and wall-clock time (seconds) to convergence. We use ${\bf \dagger}$
to indicate when AQP does not converge; see \S\ref{sec:1st}.}
\label{fig:uv_table}
\end{figure}

\begin{figure}[h!]
\centering
\includegraphics[width=1\linewidth]{figures/Figure_7A/2d_table_1}
\caption{\bfi{Two-Dimensional Deformation.} {\bf Top:}  initial
conditions and vertex constraints (blue points) for deformation
problems minimizing ISO, MIPS, and NH deformation energies. {\bf
Middle:} converged solutions from BCQN on ISO examples. {\bf Bottom:}
for each method / problem pair we report number of iterations to
convergence (characteristic norm $< 10^{-3}$) and wall-clock time
(seconds) to convergence. We use ${\bf \dagger}$ to indicate when
AQP does not converge; see \S\ref{sec:1st}.}
\label{fig:2d_defo_table}
\end{figure}

\begin{figure}[h!]
\centering
\includegraphics[width=1\linewidth]{figures/Figure_8A/3d_table_1}
\caption{ \bfi{Three-Dimensional Deformation.} {\bf Top:}  initial
conditions for vertex-constrained deformation problems minimizing
ISO and MR deformation energies. {\bf Middle:} converged solutions
satisfying constraints from BCQN on MR examples. {\bf Bottom:} for
each method / problem pair we report number of iterations to
convergence (characteristic norm $< 10^{-3}$) and wall-clock time
(seconds) to convergence. }
\label{fig:3d_defo_table}
\end{figure}

%%%%%%%%%%%%%%%%%%%%%%%%%%%%%%%%%%%%%%%

\subsection{First-order methods}
\label{sec:1st}

Among existing first-order methods for geometry optimization AQP
has so far shown best efficiency~\cite{Kovalsky:2016:AQP} with
improved convergence over SGD as well as standard L-BFGS.
Likewise, as we see in Figures\ \ref{fig:2d_scale_table},
\ref{fig:3d_scale_table}, and \ref{fig:3d_large_defo}, when we scale
to increasingly larger problems AQP will dominate over Newton-type
methods and so potentially offers the promise of reliability across
applications. Finally although small BQCN performs a small fixed
amount of extra work per-iteration in the line-search filter and
quasi-Newton update. Thus in Figures\ \ref{fig:2d_scale_table},
\ref{fig:3d_scale_table}, \ref{fig:uv_table} and \ref{fig:3d_defo_table}.
we compare AQP and BCQN over a range of practical geometry optimization
applications: respectively UV-parameterization, 2D deformation, and
3D deformation with nonconvex energies from geometry processing and
physics.  Throughout we note three key features distinguishing BCQN:

\bfi{Reliability and robustness.} 
AQP will fail to converge in some cases, see e.g. Figure\ \ref{fig:aqp_stop}, while BCQN reliably converges. In our testing AQP fails to converge in over 40\% of our tests with nonconvex energies; see e.g. Figures~\ref{fig:uv_table} and \ref{fig:2d_defo_table}.
This behavior is duplicated in our test-harness code and AQP's reference implementation.

\bfi{Convergence speed.} When AQP is able to converge, BCQN consistently provides faster convergence rates for nonconvex energies. In our experiments convergence rates range up to well over 10X with respect to AQP.

\bfi{Performance.} BCQN is efficient. When AQP is able to converge, BCQN remains fast with up to a well over 7X speedup over AQP on nonconvex energies.

\subsection{Across the Board Comparisons} 

Here we compare the performance and memory usage of BCQN with best-in-class geometry optimization methods across the board: AQP, PN and CM for both 2D parameterization and 3D deformation tasks. Results are summarized in Figures\ \ref{fig:2d_scale_table}, \ref{fig:3d_scale_table} and \ref{fig:3d_large_defo}. Note that CM does not extend to 3D.

In Figures~\ref{fig:2d_scale_table} and \ref{fig:3d_scale_table} we examine the scaling of AQP, PN, CM and BCQN to larger meshes and thus to larger problem sizes in both 2D parametrization (up to 23.9M triangles) and 3D deformation (up to 7.8M tetrahedra). As noted above: from the outset, BCQN requires more iterations than CM and PN; however, BCQN's overall low cost per iteration makes it faster in performance across problem sizes when compared to both CM and PN. We then note that AQP, on the other hand, has slower convergence and so, at smaller sizes it often does not compete with CM and PN. However, once we reach larger mesh problems, e.g. $\sim\geq$ 6M triangles in Figure~\ref{fig:2d_scale_table}, the cost of factorization and backsolve of the denser linear systems of CM and PN becomes significant so that even AQP's slower convergence results in improvement. This is the intended domain for which first-order methods are designed but here too, as we see in Figure~\ref{fig:2d_scale_table}, BCQN continues to outperform both AQP as well as CM and PN across all scales. Please see our supplemental video for visual comparisons of the relative progress of PN, CM, AQP and BCQN.


% \vspace{-0.5em}
\section{Conclusion}
% \vspace{-0.5em}
Recent advances in multimodal single-cell technology have enabled the simultaneous profiling of the transcriptome alongside other cellular modalities, leading to an increase in the availability of multimodal single-cell data. In this paper, we present \method{}, a multimodal transformer model for single-cell surface protein abundance from gene expression measurements. We combined the data with prior biological interaction knowledge from the STRING database into a richly connected heterogeneous graph and leveraged the transformer architectures to learn an accurate mapping between gene expression and surface protein abundance. Remarkably, \method{} achieves superior and more stable performance than other baselines on both 2021 and 2022 NeurIPS single-cell datasets.

\noindent\textbf{Future Work.}
% Our work is an extension of the model we implemented in the NeurIPS 2022 competition. 
Our framework of multimodal transformers with the cross-modality heterogeneous graph goes far beyond the specific downstream task of modality prediction, and there are lots of potentials to be further explored. Our graph contains three types of nodes. While the cell embeddings are used for predictions, the remaining protein embeddings and gene embeddings may be further interpreted for other tasks. The similarities between proteins may show data-specific protein-protein relationships, while the attention matrix of the gene transformer may help to identify marker genes of each cell type. Additionally, we may achieve gene interaction prediction using the attention mechanism.
% under adequate regulations. 
% We expect \method{} to be capable of much more than just modality prediction. Note that currently, we fuse information from different transformers with message-passing GNNs. 
To extend more on transformers, a potential next step is implementing cross-attention cross-modalities. Ideally, all three types of nodes, namely genes, proteins, and cells, would be jointly modeled using a large transformer that includes specific regulations for each modality. 

% insight of protein and gene embedding (diff task)

% all in one transformer

% \noindent\textbf{Limitations and future work}
% Despite the noticeable performance improvement by utilizing transformers with the cross-modality heterogeneous graph, there are still bottlenecks in the current settings. To begin with, we noticed that the performance variations of all methods are consistently higher in the ``CITE'' dataset compared to the ``GEX2ADT'' dataset. We hypothesized that the increased variability in ``CITE'' was due to both less number of training samples (43k vs. 66k cells) and a significantly more number of testing samples used (28k vs. 1k cells). One straightforward solution to alleviate the high variation issue is to include more training samples, which is not always possible given the training data availability. Nevertheless, publicly available single-cell datasets have been accumulated over the past decades and are still being collected on an ever-increasing scale. Taking advantage of these large-scale atlases is the key to a more stable and well-performing model, as some of the intra-cell variations could be common across different datasets. For example, reference-based methods are commonly used to identify the cell identity of a single cell, or cell-type compositions of a mixture of cells. (other examples for pretrained, e.g., scbert)


%\noindent\textbf{Future work.}
% Our work is an extension of the model we implemented in the NeurIPS 2022 competition. Now our framework of multimodal transformers with the cross-modality heterogeneous graph goes far beyond the specific downstream task of modality prediction, and there are lots of potentials to be further explored. Our graph contains three types of nodes. while the cell embeddings are used for predictions, the remaining protein embeddings and gene embeddings may be further interpreted for other tasks. The similarities between proteins may show data-specific protein-protein relationships, while the attention matrix of the gene transformer may help to identify marker genes of each cell type. Additionally, we may achieve gene interaction prediction using the attention mechanism under adequate regulations. We expect \method{} to be capable of much more than just modality prediction. Note that currently, we fuse information from different transformers with message-passing GNNs. To extend more on transformers, a potential next step is implementing cross-attention cross-modalities. Ideally, all three types of nodes, namely genes, proteins, and cells, would be jointly modeled using a large transformer that includes specific regulations for each modality. The self-attention within each modality would reconstruct the prior interaction network, while the cross-attention between modalities would be supervised by the data observations. Then, The attention matrix will provide insights into all the internal interactions and cross-relationships. With the linearized transformer, this idea would be both practical and versatile.

% \begin{acks}
% This research is supported by the National Science Foundation (NSF) and Johnson \& Johnson.
% \end{acks}

\bibliographystyle{acmsiggraph}
\nocite{*}
\bibliography{paper}

\chapter{Supplementary Material}
\label{appendix}

In this appendix, we present supplementary material for the techniques and
experiments presented in the main text.

\section{Baseline Results and Analysis for Informed Sampler}
\label{appendix:chap3}

Here, we give an in-depth
performance analysis of the various samplers and the effect of their
hyperparameters. We choose hyperparameters with the lowest PSRF value
after $10k$ iterations, for each sampler individually. If the
differences between PSRF are not significantly different among
multiple values, we choose the one that has the highest acceptance
rate.

\subsection{Experiment: Estimating Camera Extrinsics}
\label{appendix:chap3:room}

\subsubsection{Parameter Selection}
\paragraph{Metropolis Hastings (\MH)}

Figure~\ref{fig:exp1_MH} shows the median acceptance rates and PSRF
values corresponding to various proposal standard deviations of plain
\MH~sampling. Mixing gets better and the acceptance rate gets worse as
the standard deviation increases. The value $0.3$ is selected standard
deviation for this sampler.

\paragraph{Metropolis Hastings Within Gibbs (\MHWG)}

As mentioned in Section~\ref{sec:room}, the \MHWG~sampler with one-dimensional
updates did not converge for any value of proposal standard deviation.
This problem has high correlation of the camera parameters and is of
multi-modal nature, which this sampler has problems with.

\paragraph{Parallel Tempering (\PT)}

For \PT~sampling, we took the best performing \MH~sampler and used
different temperature chains to improve the mixing of the
sampler. Figure~\ref{fig:exp1_PT} shows the results corresponding to
different combination of temperature levels. The sampler with
temperature levels of $[1,3,27]$ performed best in terms of both
mixing and acceptance rate.

\paragraph{Effect of Mixture Coefficient in Informed Sampling (\MIXLMH)}

Figure~\ref{fig:exp1_alpha} shows the effect of mixture
coefficient ($\alpha$) on the informed sampling
\MIXLMH. Since there is no significant different in PSRF values for
$0 \le \alpha \le 0.7$, we chose $0.7$ due to its high acceptance
rate.


% \end{multicols}

\begin{figure}[h]
\centering
  \subfigure[MH]{%
    \includegraphics[width=.48\textwidth]{figures/supplementary/camPose_MH.pdf} \label{fig:exp1_MH}
  }
  \subfigure[PT]{%
    \includegraphics[width=.48\textwidth]{figures/supplementary/camPose_PT.pdf} \label{fig:exp1_PT}
  }
\\
  \subfigure[INF-MH]{%
    \includegraphics[width=.48\textwidth]{figures/supplementary/camPose_alpha.pdf} \label{fig:exp1_alpha}
  }
  \mycaption{Results of the `Estimating Camera Extrinsics' experiment}{PRSFs and Acceptance rates corresponding to (a) various standard deviations of \MH, (b) various temperature level combinations of \PT sampling and (c) various mixture coefficients of \MIXLMH sampling.}
\end{figure}



\begin{figure}[!t]
\centering
  \subfigure[\MH]{%
    \includegraphics[width=.48\textwidth]{figures/supplementary/occlusionExp_MH.pdf} \label{fig:exp2_MH}
  }
  \subfigure[\BMHWG]{%
    \includegraphics[width=.48\textwidth]{figures/supplementary/occlusionExp_BMHWG.pdf} \label{fig:exp2_BMHWG}
  }
\\
  \subfigure[\MHWG]{%
    \includegraphics[width=.48\textwidth]{figures/supplementary/occlusionExp_MHWG.pdf} \label{fig:exp2_MHWG}
  }
  \subfigure[\PT]{%
    \includegraphics[width=.48\textwidth]{figures/supplementary/occlusionExp_PT.pdf} \label{fig:exp2_PT}
  }
\\
  \subfigure[\INFBMHWG]{%
    \includegraphics[width=.5\textwidth]{figures/supplementary/occlusionExp_alpha.pdf} \label{fig:exp2_alpha}
  }
  \mycaption{Results of the `Occluding Tiles' experiment}{PRSF and
    Acceptance rates corresponding to various standard deviations of
    (a) \MH, (b) \BMHWG, (c) \MHWG, (d) various temperature level
    combinations of \PT~sampling and; (e) various mixture coefficients
    of our informed \INFBMHWG sampling.}
\end{figure}

%\onecolumn\newpage\twocolumn
\subsection{Experiment: Occluding Tiles}
\label{appendix:chap3:tiles}

\subsubsection{Parameter Selection}

\paragraph{Metropolis Hastings (\MH)}

Figure~\ref{fig:exp2_MH} shows the results of
\MH~sampling. Results show the poor convergence for all proposal
standard deviations and rapid decrease of AR with increasing standard
deviation. This is due to the high-dimensional nature of
the problem. We selected a standard deviation of $1.1$.

\paragraph{Blocked Metropolis Hastings Within Gibbs (\BMHWG)}

The results of \BMHWG are shown in Figure~\ref{fig:exp2_BMHWG}. In
this sampler we update only one block of tile variables (of dimension
four) in each sampling step. Results show much better performance
compared to plain \MH. The optimal proposal standard deviation for
this sampler is $0.7$.

\paragraph{Metropolis Hastings Within Gibbs (\MHWG)}

Figure~\ref{fig:exp2_MHWG} shows the result of \MHWG sampling. This
sampler is better than \BMHWG and converges much more quickly. Here
a standard deviation of $0.9$ is found to be best.

\paragraph{Parallel Tempering (\PT)}

Figure~\ref{fig:exp2_PT} shows the results of \PT sampling with various
temperature combinations. Results show no improvement in AR from plain
\MH sampling and again $[1,3,27]$ temperature levels are found to be optimal.

\paragraph{Effect of Mixture Coefficient in Informed Sampling (\INFBMHWG)}

Figure~\ref{fig:exp2_alpha} shows the effect of mixture
coefficient ($\alpha$) on the blocked informed sampling
\INFBMHWG. Since there is no significant different in PSRF values for
$0 \le \alpha \le 0.8$, we chose $0.8$ due to its high acceptance
rate.



\subsection{Experiment: Estimating Body Shape}
\label{appendix:chap3:body}

\subsubsection{Parameter Selection}
\paragraph{Metropolis Hastings (\MH)}

Figure~\ref{fig:exp3_MH} shows the result of \MH~sampling with various
proposal standard deviations. The value of $0.1$ is found to be
best.

\paragraph{Metropolis Hastings Within Gibbs (\MHWG)}

For \MHWG sampling we select $0.3$ proposal standard
deviation. Results are shown in Fig.~\ref{fig:exp3_MHWG}.


\paragraph{Parallel Tempering (\PT)}

As before, results in Fig.~\ref{fig:exp3_PT}, the temperature levels
were selected to be $[1,3,27]$ due its slightly higher AR.

\paragraph{Effect of Mixture Coefficient in Informed Sampling (\MIXLMH)}

Figure~\ref{fig:exp3_alpha} shows the effect of $\alpha$ on PSRF and
AR. Since there is no significant differences in PSRF values for $0 \le
\alpha \le 0.8$, we choose $0.8$.


\begin{figure}[t]
\centering
  \subfigure[\MH]{%
    \includegraphics[width=.48\textwidth]{figures/supplementary/bodyShape_MH.pdf} \label{fig:exp3_MH}
  }
  \subfigure[\MHWG]{%
    \includegraphics[width=.48\textwidth]{figures/supplementary/bodyShape_MHWG.pdf} \label{fig:exp3_MHWG}
  }
\\
  \subfigure[\PT]{%
    \includegraphics[width=.48\textwidth]{figures/supplementary/bodyShape_PT.pdf} \label{fig:exp3_PT}
  }
  \subfigure[\MIXLMH]{%
    \includegraphics[width=.48\textwidth]{figures/supplementary/bodyShape_alpha.pdf} \label{fig:exp3_alpha}
  }
\\
  \mycaption{Results of the `Body Shape Estimation' experiment}{PRSFs and
    Acceptance rates corresponding to various standard deviations of
    (a) \MH, (b) \MHWG; (c) various temperature level combinations
    of \PT sampling and; (d) various mixture coefficients of the
    informed \MIXLMH sampling.}
\end{figure}


\subsection{Results Overview}
Figure~\ref{fig:exp_summary} shows the summary results of the all the three
experimental studies related to informed sampler.
\begin{figure*}[h!]
\centering
  \subfigure[Results for: Estimating Camera Extrinsics]{%
    \includegraphics[width=0.9\textwidth]{figures/supplementary/camPose_ALL.pdf} \label{fig:exp1_all}
  }
  \subfigure[Results for: Occluding Tiles]{%
    \includegraphics[width=0.9\textwidth]{figures/supplementary/occlusionExp_ALL.pdf} \label{fig:exp2_all}
  }
  \subfigure[Results for: Estimating Body Shape]{%
    \includegraphics[width=0.9\textwidth]{figures/supplementary/bodyShape_ALL.pdf} \label{fig:exp3_all}
  }
  \label{fig:exp_summary}
  \mycaption{Summary of the statistics for the three experiments}{Shown are
    for several baseline methods and the informed samplers the
    acceptance rates (left), PSRFs (middle), and RMSE values
    (right). All results are median results over multiple test
    examples.}
\end{figure*}

\subsection{Additional Qualitative Results}

\subsubsection{Occluding Tiles}
In Figure~\ref{fig:exp2_visual_more} more qualitative results of the
occluding tiles experiment are shown. The informed sampling approach
(\INFBMHWG) is better than the best baseline (\MHWG). This still is a
very challenging problem since the parameters for occluded tiles are
flat over a large region. Some of the posterior variance of the
occluded tiles is already captured by the informed sampler.

\begin{figure*}[h!]
\begin{center}
\centerline{\includegraphics[width=0.95\textwidth]{figures/supplementary/occlusionExp_Visual.pdf}}
\mycaption{Additional qualitative results of the occluding tiles experiment}
  {From left to right: (a)
  Given image, (b) Ground truth tiles, (c) OpenCV heuristic and most probable estimates
  from 5000 samples obtained by (d) MHWG sampler (best baseline) and
  (e) our INF-BMHWG sampler. (f) Posterior expectation of the tiles
  boundaries obtained by INF-BMHWG sampling (First 2000 samples are
  discarded as burn-in).}
\label{fig:exp2_visual_more}
\end{center}
\end{figure*}

\subsubsection{Body Shape}
Figure~\ref{fig:exp3_bodyMeshes} shows some more results of 3D mesh
reconstruction using posterior samples obtained by our informed
sampling \MIXLMH.

\begin{figure*}[t]
\begin{center}
\centerline{\includegraphics[width=0.75\textwidth]{figures/supplementary/bodyMeshResults.pdf}}
\mycaption{Qualitative results for the body shape experiment}
  {Shown is the 3D mesh reconstruction results with first 1000 samples obtained
  using the \MIXLMH informed sampling method. (blue indicates small
  values and red indicates high values)}
\label{fig:exp3_bodyMeshes}
\end{center}
\end{figure*}

\clearpage



\section{Additional Results on the Face Problem with CMP}

Figure~\ref{fig:shading-qualitative-multiple-subjects-supp} shows inference results for reflectance maps, normal maps and lights for randomly chosen test images, and Fig.~\ref{fig:shading-qualitative-same-subject-supp} shows reflectance estimation results on multiple images of the same subject produced under different illumination conditions. CMP is able to produce estimates that are closer to the groundtruth across different subjects and illumination conditions.

\begin{figure*}[h]
  \begin{center}
  \centerline{\includegraphics[width=1.0\columnwidth]{figures/face_cmp_visual_results_supp.pdf}}
  \vspace{-1.2cm}
  \end{center}
	\mycaption{A visual comparison of inference results}{(a)~Observed images. (b)~Inferred reflectance maps. \textit{GT} is the photometric stereo groundtruth, \textit{BU} is the Biswas \etal (2009) reflectance estimate and \textit{Forest} is the consensus prediction. (c)~The variance of the inferred reflectance estimate produced by \MTD (normalized across rows).(d)~Visualization of inferred light directions. (e)~Inferred normal maps.}
	\label{fig:shading-qualitative-multiple-subjects-supp}
\end{figure*}


\begin{figure*}[h]
	\centering
	\setlength\fboxsep{0.2mm}
	\setlength\fboxrule{0pt}
	\begin{tikzpicture}

		\matrix at (0, 0) [matrix of nodes, nodes={anchor=east}, column sep=-0.05cm, row sep=-0.2cm]
		{
			\fbox{\includegraphics[width=1cm]{figures/sample_3_4_X.png}} &
			\fbox{\includegraphics[width=1cm]{figures/sample_3_4_GT.png}} &
			\fbox{\includegraphics[width=1cm]{figures/sample_3_4_BISWAS.png}}  &
			\fbox{\includegraphics[width=1cm]{figures/sample_3_4_VMP.png}}  &
			\fbox{\includegraphics[width=1cm]{figures/sample_3_4_FOREST.png}}  &
			\fbox{\includegraphics[width=1cm]{figures/sample_3_4_CMP.png}}  &
			\fbox{\includegraphics[width=1cm]{figures/sample_3_4_CMPVAR.png}}
			 \\

			\fbox{\includegraphics[width=1cm]{figures/sample_3_5_X.png}} &
			\fbox{\includegraphics[width=1cm]{figures/sample_3_5_GT.png}} &
			\fbox{\includegraphics[width=1cm]{figures/sample_3_5_BISWAS.png}}  &
			\fbox{\includegraphics[width=1cm]{figures/sample_3_5_VMP.png}}  &
			\fbox{\includegraphics[width=1cm]{figures/sample_3_5_FOREST.png}}  &
			\fbox{\includegraphics[width=1cm]{figures/sample_3_5_CMP.png}}  &
			\fbox{\includegraphics[width=1cm]{figures/sample_3_5_CMPVAR.png}}
			 \\

			\fbox{\includegraphics[width=1cm]{figures/sample_3_6_X.png}} &
			\fbox{\includegraphics[width=1cm]{figures/sample_3_6_GT.png}} &
			\fbox{\includegraphics[width=1cm]{figures/sample_3_6_BISWAS.png}}  &
			\fbox{\includegraphics[width=1cm]{figures/sample_3_6_VMP.png}}  &
			\fbox{\includegraphics[width=1cm]{figures/sample_3_6_FOREST.png}}  &
			\fbox{\includegraphics[width=1cm]{figures/sample_3_6_CMP.png}}  &
			\fbox{\includegraphics[width=1cm]{figures/sample_3_6_CMPVAR.png}}
			 \\
	     };

       \node at (-3.85, -2.0) {\small Observed};
       \node at (-2.55, -2.0) {\small `GT'};
       \node at (-1.27, -2.0) {\small BU};
       \node at (0.0, -2.0) {\small MP};
       \node at (1.27, -2.0) {\small Forest};
       \node at (2.55, -2.0) {\small \textbf{CMP}};
       \node at (3.85, -2.0) {\small Variance};

	\end{tikzpicture}
	\mycaption{Robustness to varying illumination}{Reflectance estimation on a subject images with varying illumination. Left to right: observed image, photometric stereo estimate (GT)
  which is used as a proxy for groundtruth, bottom-up estimate of \cite{Biswas2009}, VMP result, consensus forest estimate, CMP mean, and CMP variance.}
	\label{fig:shading-qualitative-same-subject-supp}
\end{figure*}

\clearpage

\section{Additional Material for Learning Sparse High Dimensional Filters}
\label{sec:appendix-bnn}

This part of supplementary material contains a more detailed overview of the permutohedral
lattice convolution in Section~\ref{sec:permconv}, more experiments in
Section~\ref{sec:addexps} and additional results with protocols for
the experiments presented in Chapter~\ref{chap:bnn} in Section~\ref{sec:addresults}.

\vspace{-0.2cm}
\subsection{General Permutohedral Convolutions}
\label{sec:permconv}

A core technical contribution of this work is the generalization of the Gaussian permutohedral lattice
convolution proposed in~\cite{adams2010fast} to the full non-separable case with the
ability to perform back-propagation. Although, conceptually, there are minor
differences between Gaussian and general parameterized filters, there are non-trivial practical
differences in terms of the algorithmic implementation. The Gauss filters belong to
the separable class and can thus be decomposed into multiple
sequential one dimensional convolutions. We are interested in the general filter
convolutions, which can not be decomposed. Thus, performing a general permutohedral
convolution at a lattice point requires the computation of the inner product with the
neighboring elements in all the directions in the high-dimensional space.

Here, we give more details of the implementation differences of separable
and non-separable filters. In the following, we will explain the scalar case first.
Recall, that the forward pass of general permutohedral convolution
involves 3 steps: \textit{splatting}, \textit{convolving} and \textit{slicing}.
We follow the same splatting and slicing strategies as in~\cite{adams2010fast}
since these operations do not depend on the filter kernel. The main difference
between our work and the existing implementation of~\cite{adams2010fast} is
the way that the convolution operation is executed. This proceeds by constructing
a \emph{blur neighbor} matrix $K$ that stores for every lattice point all
values of the lattice neighbors that are needed to compute the filter output.

\begin{figure}[t!]
  \centering
    \includegraphics[width=0.6\columnwidth]{figures/supplementary/lattice_construction}
  \mycaption{Illustration of 1D permutohedral lattice construction}
  {A $4\times 4$ $(x,y)$ grid lattice is projected onto the plane defined by the normal
  vector $(1,1)^{\top}$. This grid has $s+1=4$ and $d=2$ $(s+1)^{d}=4^2=16$ elements.
  In the projection, all points of the same color are projected onto the same points in the plane.
  The number of elements of the projected lattice is $t=(s+1)^d-s^d=4^2-3^2=7$, that is
  the $(4\times 4)$ grid minus the size of lattice that is $1$ smaller at each size, in this
  case a $(3\times 3)$ lattice (the upper right $(3\times 3)$ elements).
  }
\label{fig:latticeconstruction}
\end{figure}

The blur neighbor matrix is constructed by traversing through all the populated
lattice points and their neighboring elements.
% For efficiency, we do this matrix construction recursively with shared computations
% since $n^{th}$ neighbourhood elements are $1^{st}$ neighborhood elements of $n-1^{th}$ neighbourhood elements. \pg{do not understand}
This is done recursively to share computations. For any lattice point, the neighbors that are
$n$ hops away are the direct neighbors of the points that are $n-1$ hops away.
The size of a $d$ dimensional spatial filter with width $s+1$ is $(s+1)^{d}$ (\eg, a
$3\times 3$ filter, $s=2$ in $d=2$ has $3^2=9$ elements) and this size grows
exponentially in the number of dimensions $d$. The permutohedral lattice is constructed by
projecting a regular grid onto the plane spanned by the $d$ dimensional normal vector ${(1,\ldots,1)}^{\top}$. See
Fig.~\ref{fig:latticeconstruction} for an illustration of the 1D lattice construction.
Many corners of a grid filter are projected onto the same point, in total $t = {(s+1)}^{d} -
s^{d}$ elements remain in the permutohedral filter with $s$ neighborhood in $d-1$ dimensions.
If the lattice has $m$ populated elements, the
matrix $K$ has size $t\times m$. Note that, since the input signal is typically
sparse, only a few lattice corners are being populated in the \textit{slicing} step.
We use a hash-table to keep track of these points and traverse only through
the populated lattice points for this neighborhood matrix construction.

Once the blur neighbor matrix $K$ is constructed, we can perform the convolution
by the matrix vector multiplication
\begin{equation}
\ell' = BK,
\label{eq:conv}
\end{equation}
where $B$ is the $1 \times t$ filter kernel (whose values we will learn) and $\ell'\in\mathbb{R}^{1\times m}$
is the result of the filtering at the $m$ lattice points. In practice, we found that the
matrix $K$ is sometimes too large to fit into GPU memory and we divided the matrix $K$
into smaller pieces to compute Eq.~\ref{eq:conv} sequentially.

In the general multi-dimensional case, the signal $\ell$ is of $c$ dimensions. Then
the kernel $B$ is of size $c \times t$ and $K$ stores the $c$ dimensional vectors
accordingly. When the input and output points are different, we slice only the
input points and splat only at the output points.


\subsection{Additional Experiments}
\label{sec:addexps}
In this section, we discuss more use-cases for the learned bilateral filters, one
use-case of BNNs and two single filter applications for image and 3D mesh denoising.

\subsubsection{Recognition of subsampled MNIST}\label{sec:app_mnist}

One of the strengths of the proposed filter convolution is that it does not
require the input to lie on a regular grid. The only requirement is to define a distance
between features of the input signal.
We highlight this feature with the following experiment using the
classical MNIST ten class classification problem~\cite{lecun1998mnist}. We sample a
sparse set of $N$ points $(x,y)\in [0,1]\times [0,1]$
uniformly at random in the input image, use their interpolated values
as signal and the \emph{continuous} $(x,y)$ positions as features. This mimics
sub-sampling of a high-dimensional signal. To compare against a spatial convolution,
we interpolate the sparse set of values at the grid positions.

We take a reference implementation of LeNet~\cite{lecun1998gradient} that
is part of the Caffe project~\cite{jia2014caffe} and compare it
against the same architecture but replacing the first convolutional
layer with a bilateral convolution layer (BCL). The filter size
and numbers are adjusted to get a comparable number of parameters
($5\times 5$ for LeNet, $2$-neighborhood for BCL).

The results are shown in Table~\ref{tab:all-results}. We see that training
on the original MNIST data (column Original, LeNet vs. BNN) leads to a slight
decrease in performance of the BNN (99.03\%) compared to LeNet
(99.19\%). The BNN can be trained and evaluated on sparse
signals, and we resample the image as described above for $N=$ 100\%, 60\% and
20\% of the total number of pixels. The methods are also evaluated
on test images that are subsampled in the same way. Note that we can
train and test with different subsampling rates. We introduce an additional
bilinear interpolation layer for the LeNet architecture to train on the same
data. In essence, both models perform a spatial interpolation and thus we
expect them to yield a similar classification accuracy. Once the data is of
higher dimensions, the permutohedral convolution will be faster due to hashing
the sparse input points, as well as less memory demanding in comparison to
naive application of a spatial convolution with interpolated values.

\begin{table}[t]
  \begin{center}
    \footnotesize
    \centering
    \begin{tabular}[t]{lllll}
      \toprule
              &     & \multicolumn{3}{c}{Test Subsampling} \\
       Method  & Original & 100\% & 60\% & 20\%\\
      \midrule
       LeNet &  \textbf{0.9919} & 0.9660 & 0.9348 & \textbf{0.6434} \\
       BNN &  0.9903 & \textbf{0.9844} & \textbf{0.9534} & 0.5767 \\
      \hline
       LeNet 100\% & 0.9856 & 0.9809 & 0.9678 & \textbf{0.7386} \\
       BNN 100\% & \textbf{0.9900} & \textbf{0.9863} & \textbf{0.9699} & 0.6910 \\
      \hline
       LeNet 60\% & 0.9848 & 0.9821 & 0.9740 & 0.8151 \\
       BNN 60\% & \textbf{0.9885} & \textbf{0.9864} & \textbf{0.9771} & \textbf{0.8214}\\
      \hline
       LeNet 20\% & \textbf{0.9763} & \textbf{0.9754} & 0.9695 & 0.8928 \\
       BNN 20\% & 0.9728 & 0.9735 & \textbf{0.9701} & \textbf{0.9042}\\
      \bottomrule
    \end{tabular}
  \end{center}
\vspace{-.2cm}
\caption{Classification accuracy on MNIST. We compare the
    LeNet~\cite{lecun1998gradient} implementation that is part of
    Caffe~\cite{jia2014caffe} to the network with the first layer
    replaced by a bilateral convolution layer (BCL). Both are trained
    on the original image resolution (first two rows). Three more BNN
    and CNN models are trained with randomly subsampled images (100\%,
    60\% and 20\% of the pixels). An additional bilinear interpolation
    layer samples the input signal on a spatial grid for the CNN model.
  }
  \label{tab:all-results}
\vspace{-.5cm}
\end{table}

\subsubsection{Image Denoising}

The main application that inspired the development of the bilateral
filtering operation is image denoising~\cite{aurich1995non}, there
using a single Gaussian kernel. Our development allows to learn this
kernel function from data and we explore how to improve using a \emph{single}
but more general bilateral filter.

We use the Berkeley segmentation dataset
(BSDS500)~\cite{arbelaezi2011bsds500} as a test bed. The color
images in the dataset are converted to gray-scale,
and corrupted with Gaussian noise with a standard deviation of
$\frac {25} {255}$.

We compare the performance of four different filter models on a
denoising task.
The first baseline model (`Spatial' in Table \ref{tab:denoising}, $25$
weights) uses a single spatial filter with a kernel size of
$5$ and predicts the scalar gray-scale value at the center pixel. The next model
(`Gauss Bilateral') applies a bilateral \emph{Gaussian}
filter to the noisy input, using position and intensity features $\f=(x,y,v)^\top$.
The third setup (`Learned Bilateral', $65$ weights)
takes a Gauss kernel as initialization and
fits all filter weights on the train set to minimize the
mean squared error with respect to the clean images.
We run a combination
of spatial and permutohedral convolutions on spatial and bilateral
features (`Spatial + Bilateral (Learned)') to check for a complementary
performance of the two convolutions.

\label{sec:exp:denoising}
\begin{table}[!h]
\begin{center}
  \footnotesize
  \begin{tabular}[t]{lr}
    \toprule
    Method & PSNR \\
    \midrule
    Noisy Input & $20.17$ \\
    Spatial & $26.27$ \\
    Gauss Bilateral & $26.51$ \\
    Learned Bilateral & $26.58$ \\
    Spatial + Bilateral (Learned) & \textbf{$26.65$} \\
    \bottomrule
  \end{tabular}
\end{center}
\vspace{-0.5em}
\caption{PSNR results of a denoising task using the BSDS500
  dataset~\cite{arbelaezi2011bsds500}}
\vspace{-0.5em}
\label{tab:denoising}
\end{table}
\vspace{-0.2em}

The PSNR scores evaluated on full images of the test set are
shown in Table \ref{tab:denoising}. We find that an untrained bilateral
filter already performs better than a trained spatial convolution
($26.27$ to $26.51$). A learned convolution further improve the
performance slightly. We chose this simple one-kernel setup to
validate an advantage of the generalized bilateral filter. A competitive
denoising system would employ RGB color information and also
needs to be properly adjusted in network size. Multi-layer perceptrons
have obtained state-of-the-art denoising results~\cite{burger12cvpr}
and the permutohedral lattice layer can readily be used in such an
architecture, which is intended future work.

\subsection{Additional results}
\label{sec:addresults}

This section contains more qualitative results for the experiments presented in Chapter~\ref{chap:bnn}.

\begin{figure*}[th!]
  \centering
    \includegraphics[width=\columnwidth,trim={5cm 2.5cm 5cm 4.5cm},clip]{figures/supplementary/lattice_viz.pdf}
    \vspace{-0.7cm}
  \mycaption{Visualization of the Permutohedral Lattice}
  {Sample lattice visualizations for different feature spaces. All pixels falling in the same simplex cell are shown with
  the same color. $(x,y)$ features correspond to image pixel positions, and $(r,g,b) \in [0,255]$ correspond
  to the red, green and blue color values.}
\label{fig:latticeviz}
\end{figure*}

\subsubsection{Lattice Visualization}

Figure~\ref{fig:latticeviz} shows sample lattice visualizations for different feature spaces.

\newcolumntype{L}[1]{>{\raggedright\let\newline\\\arraybackslash\hspace{0pt}}b{#1}}
\newcolumntype{C}[1]{>{\centering\let\newline\\\arraybackslash\hspace{0pt}}b{#1}}
\newcolumntype{R}[1]{>{\raggedleft\let\newline\\\arraybackslash\hspace{0pt}}b{#1}}

\subsubsection{Color Upsampling}\label{sec:color_upsampling}
\label{sec:col_upsample_extra}

Some images of the upsampling for the Pascal
VOC12 dataset are shown in Fig.~\ref{fig:Colour_upsample_visuals}. It is
especially the low level image details that are better preserved with
a learned bilateral filter compared to the Gaussian case.

\begin{figure*}[t!]
  \centering
    \subfigure{%
   \raisebox{2.0em}{
    \includegraphics[width=.06\columnwidth]{figures/supplementary/2007_004969.jpg}
   }
  }
  \subfigure{%
    \includegraphics[width=.17\columnwidth]{figures/supplementary/2007_004969_gray.pdf}
  }
  \subfigure{%
    \includegraphics[width=.17\columnwidth]{figures/supplementary/2007_004969_gt.pdf}
  }
  \subfigure{%
    \includegraphics[width=.17\columnwidth]{figures/supplementary/2007_004969_bicubic.pdf}
  }
  \subfigure{%
    \includegraphics[width=.17\columnwidth]{figures/supplementary/2007_004969_gauss.pdf}
  }
  \subfigure{%
    \includegraphics[width=.17\columnwidth]{figures/supplementary/2007_004969_learnt.pdf}
  }\\
    \subfigure{%
   \raisebox{2.0em}{
    \includegraphics[width=.06\columnwidth]{figures/supplementary/2007_003106.jpg}
   }
  }
  \subfigure{%
    \includegraphics[width=.17\columnwidth]{figures/supplementary/2007_003106_gray.pdf}
  }
  \subfigure{%
    \includegraphics[width=.17\columnwidth]{figures/supplementary/2007_003106_gt.pdf}
  }
  \subfigure{%
    \includegraphics[width=.17\columnwidth]{figures/supplementary/2007_003106_bicubic.pdf}
  }
  \subfigure{%
    \includegraphics[width=.17\columnwidth]{figures/supplementary/2007_003106_gauss.pdf}
  }
  \subfigure{%
    \includegraphics[width=.17\columnwidth]{figures/supplementary/2007_003106_learnt.pdf}
  }\\
  \setcounter{subfigure}{0}
  \small{
  \subfigure[Inp.]{%
  \raisebox{2.0em}{
    \includegraphics[width=.06\columnwidth]{figures/supplementary/2007_006837.jpg}
   }
  }
  \subfigure[Guidance]{%
    \includegraphics[width=.17\columnwidth]{figures/supplementary/2007_006837_gray.pdf}
  }
   \subfigure[GT]{%
    \includegraphics[width=.17\columnwidth]{figures/supplementary/2007_006837_gt.pdf}
  }
  \subfigure[Bicubic]{%
    \includegraphics[width=.17\columnwidth]{figures/supplementary/2007_006837_bicubic.pdf}
  }
  \subfigure[Gauss-BF]{%
    \includegraphics[width=.17\columnwidth]{figures/supplementary/2007_006837_gauss.pdf}
  }
  \subfigure[Learned-BF]{%
    \includegraphics[width=.17\columnwidth]{figures/supplementary/2007_006837_learnt.pdf}
  }
  }
  \vspace{-0.5cm}
  \mycaption{Color Upsampling}{Color $8\times$ upsampling results
  using different methods, from left to right, (a)~Low-resolution input color image (Inp.),
  (b)~Gray scale guidance image, (c)~Ground-truth color image; Upsampled color images with
  (d)~Bicubic interpolation, (e) Gauss bilateral upsampling and, (f)~Learned bilateral
  updampgling (best viewed on screen).}

\label{fig:Colour_upsample_visuals}
\end{figure*}

\subsubsection{Depth Upsampling}
\label{sec:depth_upsample_extra}

Figure~\ref{fig:depth_upsample_visuals} presents some more qualitative results comparing bicubic interpolation, Gauss
bilateral and learned bilateral upsampling on NYU depth dataset image~\cite{silberman2012indoor}.

\subsubsection{Character Recognition}\label{sec:app_character}

 Figure~\ref{fig:nnrecognition} shows the schematic of different layers
 of the network architecture for LeNet-7~\cite{lecun1998mnist}
 and DeepCNet(5, 50)~\cite{ciresan2012multi,graham2014spatially}. For the BNN variants, the first layer filters are replaced
 with learned bilateral filters and are learned end-to-end.

\subsubsection{Semantic Segmentation}\label{sec:app_semantic_segmentation}
\label{sec:semantic_bnn_extra}

Some more visual results for semantic segmentation are shown in Figure~\ref{fig:semantic_visuals}.
These include the underlying DeepLab CNN\cite{chen2014semantic} result (DeepLab),
the 2 step mean-field result with Gaussian edge potentials (+2stepMF-GaussCRF)
and also corresponding results with learned edge potentials (+2stepMF-LearnedCRF).
In general, we observe that mean-field in learned CRF leads to slightly dilated
classification regions in comparison to using Gaussian CRF thereby filling-in the
false negative pixels and also correcting some mis-classified regions.

\begin{figure*}[t!]
  \centering
    \subfigure{%
   \raisebox{2.0em}{
    \includegraphics[width=.06\columnwidth]{figures/supplementary/2bicubic}
   }
  }
  \subfigure{%
    \includegraphics[width=.17\columnwidth]{figures/supplementary/2given_image}
  }
  \subfigure{%
    \includegraphics[width=.17\columnwidth]{figures/supplementary/2ground_truth}
  }
  \subfigure{%
    \includegraphics[width=.17\columnwidth]{figures/supplementary/2bicubic}
  }
  \subfigure{%
    \includegraphics[width=.17\columnwidth]{figures/supplementary/2gauss}
  }
  \subfigure{%
    \includegraphics[width=.17\columnwidth]{figures/supplementary/2learnt}
  }\\
    \subfigure{%
   \raisebox{2.0em}{
    \includegraphics[width=.06\columnwidth]{figures/supplementary/32bicubic}
   }
  }
  \subfigure{%
    \includegraphics[width=.17\columnwidth]{figures/supplementary/32given_image}
  }
  \subfigure{%
    \includegraphics[width=.17\columnwidth]{figures/supplementary/32ground_truth}
  }
  \subfigure{%
    \includegraphics[width=.17\columnwidth]{figures/supplementary/32bicubic}
  }
  \subfigure{%
    \includegraphics[width=.17\columnwidth]{figures/supplementary/32gauss}
  }
  \subfigure{%
    \includegraphics[width=.17\columnwidth]{figures/supplementary/32learnt}
  }\\
  \setcounter{subfigure}{0}
  \small{
  \subfigure[Inp.]{%
  \raisebox{2.0em}{
    \includegraphics[width=.06\columnwidth]{figures/supplementary/41bicubic}
   }
  }
  \subfigure[Guidance]{%
    \includegraphics[width=.17\columnwidth]{figures/supplementary/41given_image}
  }
   \subfigure[GT]{%
    \includegraphics[width=.17\columnwidth]{figures/supplementary/41ground_truth}
  }
  \subfigure[Bicubic]{%
    \includegraphics[width=.17\columnwidth]{figures/supplementary/41bicubic}
  }
  \subfigure[Gauss-BF]{%
    \includegraphics[width=.17\columnwidth]{figures/supplementary/41gauss}
  }
  \subfigure[Learned-BF]{%
    \includegraphics[width=.17\columnwidth]{figures/supplementary/41learnt}
  }
  }
  \mycaption{Depth Upsampling}{Depth $8\times$ upsampling results
  using different upsampling strategies, from left to right,
  (a)~Low-resolution input depth image (Inp.),
  (b)~High-resolution guidance image, (c)~Ground-truth depth; Upsampled depth images with
  (d)~Bicubic interpolation, (e) Gauss bilateral upsampling and, (f)~Learned bilateral
  updampgling (best viewed on screen).}

\label{fig:depth_upsample_visuals}
\end{figure*}

\subsubsection{Material Segmentation}\label{sec:app_material_segmentation}
\label{sec:material_bnn_extra}

In Fig.~\ref{fig:material_visuals-app2}, we present visual results comparing 2 step
mean-field inference with Gaussian and learned pairwise CRF potentials. In
general, we observe that the pixels belonging to dominant classes in the
training data are being more accurately classified with learned CRF. This leads to
a significant improvements in overall pixel accuracy. This also results
in a slight decrease of the accuracy from less frequent class pixels thereby
slightly reducing the average class accuracy with learning. We attribute this
to the type of annotation that is available for this dataset, which is not
for the entire image but for some segments in the image. We have very few
images of the infrequent classes to combat this behaviour during training.

\subsubsection{Experiment Protocols}
\label{sec:protocols}

Table~\ref{tbl:parameters} shows experiment protocols of different experiments.

 \begin{figure*}[t!]
  \centering
  \subfigure[LeNet-7]{
    \includegraphics[width=0.7\columnwidth]{figures/supplementary/lenet_cnn_network}
    }\\
    \subfigure[DeepCNet]{
    \includegraphics[width=\columnwidth]{figures/supplementary/deepcnet_cnn_network}
    }
  \mycaption{CNNs for Character Recognition}
  {Schematic of (top) LeNet-7~\cite{lecun1998mnist} and (bottom) DeepCNet(5,50)~\cite{ciresan2012multi,graham2014spatially} architectures used in Assamese
  character recognition experiments.}
\label{fig:nnrecognition}
\end{figure*}

\definecolor{voc_1}{RGB}{0, 0, 0}
\definecolor{voc_2}{RGB}{128, 0, 0}
\definecolor{voc_3}{RGB}{0, 128, 0}
\definecolor{voc_4}{RGB}{128, 128, 0}
\definecolor{voc_5}{RGB}{0, 0, 128}
\definecolor{voc_6}{RGB}{128, 0, 128}
\definecolor{voc_7}{RGB}{0, 128, 128}
\definecolor{voc_8}{RGB}{128, 128, 128}
\definecolor{voc_9}{RGB}{64, 0, 0}
\definecolor{voc_10}{RGB}{192, 0, 0}
\definecolor{voc_11}{RGB}{64, 128, 0}
\definecolor{voc_12}{RGB}{192, 128, 0}
\definecolor{voc_13}{RGB}{64, 0, 128}
\definecolor{voc_14}{RGB}{192, 0, 128}
\definecolor{voc_15}{RGB}{64, 128, 128}
\definecolor{voc_16}{RGB}{192, 128, 128}
\definecolor{voc_17}{RGB}{0, 64, 0}
\definecolor{voc_18}{RGB}{128, 64, 0}
\definecolor{voc_19}{RGB}{0, 192, 0}
\definecolor{voc_20}{RGB}{128, 192, 0}
\definecolor{voc_21}{RGB}{0, 64, 128}
\definecolor{voc_22}{RGB}{128, 64, 128}

\begin{figure*}[t]
  \centering
  \small{
  \fcolorbox{white}{voc_1}{\rule{0pt}{6pt}\rule{6pt}{0pt}} Background~~
  \fcolorbox{white}{voc_2}{\rule{0pt}{6pt}\rule{6pt}{0pt}} Aeroplane~~
  \fcolorbox{white}{voc_3}{\rule{0pt}{6pt}\rule{6pt}{0pt}} Bicycle~~
  \fcolorbox{white}{voc_4}{\rule{0pt}{6pt}\rule{6pt}{0pt}} Bird~~
  \fcolorbox{white}{voc_5}{\rule{0pt}{6pt}\rule{6pt}{0pt}} Boat~~
  \fcolorbox{white}{voc_6}{\rule{0pt}{6pt}\rule{6pt}{0pt}} Bottle~~
  \fcolorbox{white}{voc_7}{\rule{0pt}{6pt}\rule{6pt}{0pt}} Bus~~
  \fcolorbox{white}{voc_8}{\rule{0pt}{6pt}\rule{6pt}{0pt}} Car~~ \\
  \fcolorbox{white}{voc_9}{\rule{0pt}{6pt}\rule{6pt}{0pt}} Cat~~
  \fcolorbox{white}{voc_10}{\rule{0pt}{6pt}\rule{6pt}{0pt}} Chair~~
  \fcolorbox{white}{voc_11}{\rule{0pt}{6pt}\rule{6pt}{0pt}} Cow~~
  \fcolorbox{white}{voc_12}{\rule{0pt}{6pt}\rule{6pt}{0pt}} Dining Table~~
  \fcolorbox{white}{voc_13}{\rule{0pt}{6pt}\rule{6pt}{0pt}} Dog~~
  \fcolorbox{white}{voc_14}{\rule{0pt}{6pt}\rule{6pt}{0pt}} Horse~~
  \fcolorbox{white}{voc_15}{\rule{0pt}{6pt}\rule{6pt}{0pt}} Motorbike~~
  \fcolorbox{white}{voc_16}{\rule{0pt}{6pt}\rule{6pt}{0pt}} Person~~ \\
  \fcolorbox{white}{voc_17}{\rule{0pt}{6pt}\rule{6pt}{0pt}} Potted Plant~~
  \fcolorbox{white}{voc_18}{\rule{0pt}{6pt}\rule{6pt}{0pt}} Sheep~~
  \fcolorbox{white}{voc_19}{\rule{0pt}{6pt}\rule{6pt}{0pt}} Sofa~~
  \fcolorbox{white}{voc_20}{\rule{0pt}{6pt}\rule{6pt}{0pt}} Train~~
  \fcolorbox{white}{voc_21}{\rule{0pt}{6pt}\rule{6pt}{0pt}} TV monitor~~ \\
  }
  \subfigure{%
    \includegraphics[width=.18\columnwidth]{figures/supplementary/2007_001423_given.jpg}
  }
  \subfigure{%
    \includegraphics[width=.18\columnwidth]{figures/supplementary/2007_001423_gt.png}
  }
  \subfigure{%
    \includegraphics[width=.18\columnwidth]{figures/supplementary/2007_001423_cnn.png}
  }
  \subfigure{%
    \includegraphics[width=.18\columnwidth]{figures/supplementary/2007_001423_gauss.png}
  }
  \subfigure{%
    \includegraphics[width=.18\columnwidth]{figures/supplementary/2007_001423_learnt.png}
  }\\
  \subfigure{%
    \includegraphics[width=.18\columnwidth]{figures/supplementary/2007_001430_given.jpg}
  }
  \subfigure{%
    \includegraphics[width=.18\columnwidth]{figures/supplementary/2007_001430_gt.png}
  }
  \subfigure{%
    \includegraphics[width=.18\columnwidth]{figures/supplementary/2007_001430_cnn.png}
  }
  \subfigure{%
    \includegraphics[width=.18\columnwidth]{figures/supplementary/2007_001430_gauss.png}
  }
  \subfigure{%
    \includegraphics[width=.18\columnwidth]{figures/supplementary/2007_001430_learnt.png}
  }\\
    \subfigure{%
    \includegraphics[width=.18\columnwidth]{figures/supplementary/2007_007996_given.jpg}
  }
  \subfigure{%
    \includegraphics[width=.18\columnwidth]{figures/supplementary/2007_007996_gt.png}
  }
  \subfigure{%
    \includegraphics[width=.18\columnwidth]{figures/supplementary/2007_007996_cnn.png}
  }
  \subfigure{%
    \includegraphics[width=.18\columnwidth]{figures/supplementary/2007_007996_gauss.png}
  }
  \subfigure{%
    \includegraphics[width=.18\columnwidth]{figures/supplementary/2007_007996_learnt.png}
  }\\
   \subfigure{%
    \includegraphics[width=.18\columnwidth]{figures/supplementary/2010_002682_given.jpg}
  }
  \subfigure{%
    \includegraphics[width=.18\columnwidth]{figures/supplementary/2010_002682_gt.png}
  }
  \subfigure{%
    \includegraphics[width=.18\columnwidth]{figures/supplementary/2010_002682_cnn.png}
  }
  \subfigure{%
    \includegraphics[width=.18\columnwidth]{figures/supplementary/2010_002682_gauss.png}
  }
  \subfigure{%
    \includegraphics[width=.18\columnwidth]{figures/supplementary/2010_002682_learnt.png}
  }\\
     \subfigure{%
    \includegraphics[width=.18\columnwidth]{figures/supplementary/2010_004789_given.jpg}
  }
  \subfigure{%
    \includegraphics[width=.18\columnwidth]{figures/supplementary/2010_004789_gt.png}
  }
  \subfigure{%
    \includegraphics[width=.18\columnwidth]{figures/supplementary/2010_004789_cnn.png}
  }
  \subfigure{%
    \includegraphics[width=.18\columnwidth]{figures/supplementary/2010_004789_gauss.png}
  }
  \subfigure{%
    \includegraphics[width=.18\columnwidth]{figures/supplementary/2010_004789_learnt.png}
  }\\
       \subfigure{%
    \includegraphics[width=.18\columnwidth]{figures/supplementary/2007_001311_given.jpg}
  }
  \subfigure{%
    \includegraphics[width=.18\columnwidth]{figures/supplementary/2007_001311_gt.png}
  }
  \subfigure{%
    \includegraphics[width=.18\columnwidth]{figures/supplementary/2007_001311_cnn.png}
  }
  \subfigure{%
    \includegraphics[width=.18\columnwidth]{figures/supplementary/2007_001311_gauss.png}
  }
  \subfigure{%
    \includegraphics[width=.18\columnwidth]{figures/supplementary/2007_001311_learnt.png}
  }\\
  \setcounter{subfigure}{0}
  \subfigure[Input]{%
    \includegraphics[width=.18\columnwidth]{figures/supplementary/2010_003531_given.jpg}
  }
  \subfigure[Ground Truth]{%
    \includegraphics[width=.18\columnwidth]{figures/supplementary/2010_003531_gt.png}
  }
  \subfigure[DeepLab]{%
    \includegraphics[width=.18\columnwidth]{figures/supplementary/2010_003531_cnn.png}
  }
  \subfigure[+GaussCRF]{%
    \includegraphics[width=.18\columnwidth]{figures/supplementary/2010_003531_gauss.png}
  }
  \subfigure[+LearnedCRF]{%
    \includegraphics[width=.18\columnwidth]{figures/supplementary/2010_003531_learnt.png}
  }
  \vspace{-0.3cm}
  \mycaption{Semantic Segmentation}{Example results of semantic segmentation.
  (c)~depicts the unary results before application of MF, (d)~after two steps of MF with Gaussian edge CRF potentials, (e)~after
  two steps of MF with learned edge CRF potentials.}
    \label{fig:semantic_visuals}
\end{figure*}


\definecolor{minc_1}{HTML}{771111}
\definecolor{minc_2}{HTML}{CAC690}
\definecolor{minc_3}{HTML}{EEEEEE}
\definecolor{minc_4}{HTML}{7C8FA6}
\definecolor{minc_5}{HTML}{597D31}
\definecolor{minc_6}{HTML}{104410}
\definecolor{minc_7}{HTML}{BB819C}
\definecolor{minc_8}{HTML}{D0CE48}
\definecolor{minc_9}{HTML}{622745}
\definecolor{minc_10}{HTML}{666666}
\definecolor{minc_11}{HTML}{D54A31}
\definecolor{minc_12}{HTML}{101044}
\definecolor{minc_13}{HTML}{444126}
\definecolor{minc_14}{HTML}{75D646}
\definecolor{minc_15}{HTML}{DD4348}
\definecolor{minc_16}{HTML}{5C8577}
\definecolor{minc_17}{HTML}{C78472}
\definecolor{minc_18}{HTML}{75D6D0}
\definecolor{minc_19}{HTML}{5B4586}
\definecolor{minc_20}{HTML}{C04393}
\definecolor{minc_21}{HTML}{D69948}
\definecolor{minc_22}{HTML}{7370D8}
\definecolor{minc_23}{HTML}{7A3622}
\definecolor{minc_24}{HTML}{000000}

\begin{figure*}[t]
  \centering
  \small{
  \fcolorbox{white}{minc_1}{\rule{0pt}{6pt}\rule{6pt}{0pt}} Brick~~
  \fcolorbox{white}{minc_2}{\rule{0pt}{6pt}\rule{6pt}{0pt}} Carpet~~
  \fcolorbox{white}{minc_3}{\rule{0pt}{6pt}\rule{6pt}{0pt}} Ceramic~~
  \fcolorbox{white}{minc_4}{\rule{0pt}{6pt}\rule{6pt}{0pt}} Fabric~~
  \fcolorbox{white}{minc_5}{\rule{0pt}{6pt}\rule{6pt}{0pt}} Foliage~~
  \fcolorbox{white}{minc_6}{\rule{0pt}{6pt}\rule{6pt}{0pt}} Food~~
  \fcolorbox{white}{minc_7}{\rule{0pt}{6pt}\rule{6pt}{0pt}} Glass~~
  \fcolorbox{white}{minc_8}{\rule{0pt}{6pt}\rule{6pt}{0pt}} Hair~~ \\
  \fcolorbox{white}{minc_9}{\rule{0pt}{6pt}\rule{6pt}{0pt}} Leather~~
  \fcolorbox{white}{minc_10}{\rule{0pt}{6pt}\rule{6pt}{0pt}} Metal~~
  \fcolorbox{white}{minc_11}{\rule{0pt}{6pt}\rule{6pt}{0pt}} Mirror~~
  \fcolorbox{white}{minc_12}{\rule{0pt}{6pt}\rule{6pt}{0pt}} Other~~
  \fcolorbox{white}{minc_13}{\rule{0pt}{6pt}\rule{6pt}{0pt}} Painted~~
  \fcolorbox{white}{minc_14}{\rule{0pt}{6pt}\rule{6pt}{0pt}} Paper~~
  \fcolorbox{white}{minc_15}{\rule{0pt}{6pt}\rule{6pt}{0pt}} Plastic~~\\
  \fcolorbox{white}{minc_16}{\rule{0pt}{6pt}\rule{6pt}{0pt}} Polished Stone~~
  \fcolorbox{white}{minc_17}{\rule{0pt}{6pt}\rule{6pt}{0pt}} Skin~~
  \fcolorbox{white}{minc_18}{\rule{0pt}{6pt}\rule{6pt}{0pt}} Sky~~
  \fcolorbox{white}{minc_19}{\rule{0pt}{6pt}\rule{6pt}{0pt}} Stone~~
  \fcolorbox{white}{minc_20}{\rule{0pt}{6pt}\rule{6pt}{0pt}} Tile~~
  \fcolorbox{white}{minc_21}{\rule{0pt}{6pt}\rule{6pt}{0pt}} Wallpaper~~
  \fcolorbox{white}{minc_22}{\rule{0pt}{6pt}\rule{6pt}{0pt}} Water~~
  \fcolorbox{white}{minc_23}{\rule{0pt}{6pt}\rule{6pt}{0pt}} Wood~~ \\
  }
  \subfigure{%
    \includegraphics[width=.18\columnwidth]{figures/supplementary/000010868_given.jpg}
  }
  \subfigure{%
    \includegraphics[width=.18\columnwidth]{figures/supplementary/000010868_gt.png}
  }
  \subfigure{%
    \includegraphics[width=.18\columnwidth]{figures/supplementary/000010868_cnn.png}
  }
  \subfigure{%
    \includegraphics[width=.18\columnwidth]{figures/supplementary/000010868_gauss.png}
  }
  \subfigure{%
    \includegraphics[width=.18\columnwidth]{figures/supplementary/000010868_learnt.png}
  }\\[-2ex]
  \subfigure{%
    \includegraphics[width=.18\columnwidth]{figures/supplementary/000006011_given.jpg}
  }
  \subfigure{%
    \includegraphics[width=.18\columnwidth]{figures/supplementary/000006011_gt.png}
  }
  \subfigure{%
    \includegraphics[width=.18\columnwidth]{figures/supplementary/000006011_cnn.png}
  }
  \subfigure{%
    \includegraphics[width=.18\columnwidth]{figures/supplementary/000006011_gauss.png}
  }
  \subfigure{%
    \includegraphics[width=.18\columnwidth]{figures/supplementary/000006011_learnt.png}
  }\\[-2ex]
    \subfigure{%
    \includegraphics[width=.18\columnwidth]{figures/supplementary/000008553_given.jpg}
  }
  \subfigure{%
    \includegraphics[width=.18\columnwidth]{figures/supplementary/000008553_gt.png}
  }
  \subfigure{%
    \includegraphics[width=.18\columnwidth]{figures/supplementary/000008553_cnn.png}
  }
  \subfigure{%
    \includegraphics[width=.18\columnwidth]{figures/supplementary/000008553_gauss.png}
  }
  \subfigure{%
    \includegraphics[width=.18\columnwidth]{figures/supplementary/000008553_learnt.png}
  }\\[-2ex]
   \subfigure{%
    \includegraphics[width=.18\columnwidth]{figures/supplementary/000009188_given.jpg}
  }
  \subfigure{%
    \includegraphics[width=.18\columnwidth]{figures/supplementary/000009188_gt.png}
  }
  \subfigure{%
    \includegraphics[width=.18\columnwidth]{figures/supplementary/000009188_cnn.png}
  }
  \subfigure{%
    \includegraphics[width=.18\columnwidth]{figures/supplementary/000009188_gauss.png}
  }
  \subfigure{%
    \includegraphics[width=.18\columnwidth]{figures/supplementary/000009188_learnt.png}
  }\\[-2ex]
  \setcounter{subfigure}{0}
  \subfigure[Input]{%
    \includegraphics[width=.18\columnwidth]{figures/supplementary/000023570_given.jpg}
  }
  \subfigure[Ground Truth]{%
    \includegraphics[width=.18\columnwidth]{figures/supplementary/000023570_gt.png}
  }
  \subfigure[DeepLab]{%
    \includegraphics[width=.18\columnwidth]{figures/supplementary/000023570_cnn.png}
  }
  \subfigure[+GaussCRF]{%
    \includegraphics[width=.18\columnwidth]{figures/supplementary/000023570_gauss.png}
  }
  \subfigure[+LearnedCRF]{%
    \includegraphics[width=.18\columnwidth]{figures/supplementary/000023570_learnt.png}
  }
  \mycaption{Material Segmentation}{Example results of material segmentation.
  (c)~depicts the unary results before application of MF, (d)~after two steps of MF with Gaussian edge CRF potentials, (e)~after two steps of MF with learned edge CRF potentials.}
    \label{fig:material_visuals-app2}
\end{figure*}


\begin{table*}[h]
\tiny
  \centering
    \begin{tabular}{L{2.3cm} L{2.25cm} C{1.5cm} C{0.7cm} C{0.6cm} C{0.7cm} C{0.7cm} C{0.7cm} C{1.6cm} C{0.6cm} C{0.6cm} C{0.6cm}}
      \toprule
& & & & & \multicolumn{3}{c}{\textbf{Data Statistics}} & \multicolumn{4}{c}{\textbf{Training Protocol}} \\

\textbf{Experiment} & \textbf{Feature Types} & \textbf{Feature Scales} & \textbf{Filter Size} & \textbf{Filter Nbr.} & \textbf{Train}  & \textbf{Val.} & \textbf{Test} & \textbf{Loss Type} & \textbf{LR} & \textbf{Batch} & \textbf{Epochs} \\
      \midrule
      \multicolumn{2}{c}{\textbf{Single Bilateral Filter Applications}} & & & & & & & & & \\
      \textbf{2$\times$ Color Upsampling} & Position$_{1}$, Intensity (3D) & 0.13, 0.17 & 65 & 2 & 10581 & 1449 & 1456 & MSE & 1e-06 & 200 & 94.5\\
      \textbf{4$\times$ Color Upsampling} & Position$_{1}$, Intensity (3D) & 0.06, 0.17 & 65 & 2 & 10581 & 1449 & 1456 & MSE & 1e-06 & 200 & 94.5\\
      \textbf{8$\times$ Color Upsampling} & Position$_{1}$, Intensity (3D) & 0.03, 0.17 & 65 & 2 & 10581 & 1449 & 1456 & MSE & 1e-06 & 200 & 94.5\\
      \textbf{16$\times$ Color Upsampling} & Position$_{1}$, Intensity (3D) & 0.02, 0.17 & 65 & 2 & 10581 & 1449 & 1456 & MSE & 1e-06 & 200 & 94.5\\
      \textbf{Depth Upsampling} & Position$_{1}$, Color (5D) & 0.05, 0.02 & 665 & 2 & 795 & 100 & 654 & MSE & 1e-07 & 50 & 251.6\\
      \textbf{Mesh Denoising} & Isomap (4D) & 46.00 & 63 & 2 & 1000 & 200 & 500 & MSE & 100 & 10 & 100.0 \\
      \midrule
      \multicolumn{2}{c}{\textbf{DenseCRF Applications}} & & & & & & & & &\\
      \multicolumn{2}{l}{\textbf{Semantic Segmentation}} & & & & & & & & &\\
      \textbf{- 1step MF} & Position$_{1}$, Color (5D); Position$_{1}$ (2D) & 0.01, 0.34; 0.34  & 665; 19  & 2; 2 & 10581 & 1449 & 1456 & Logistic & 0.1 & 5 & 1.4 \\
      \textbf{- 2step MF} & Position$_{1}$, Color (5D); Position$_{1}$ (2D) & 0.01, 0.34; 0.34 & 665; 19 & 2; 2 & 10581 & 1449 & 1456 & Logistic & 0.1 & 5 & 1.4 \\
      \textbf{- \textit{loose} 2step MF} & Position$_{1}$, Color (5D); Position$_{1}$ (2D) & 0.01, 0.34; 0.34 & 665; 19 & 2; 2 &10581 & 1449 & 1456 & Logistic & 0.1 & 5 & +1.9  \\ \\
      \multicolumn{2}{l}{\textbf{Material Segmentation}} & & & & & & & & &\\
      \textbf{- 1step MF} & Position$_{2}$, Lab-Color (5D) & 5.00, 0.05, 0.30  & 665 & 2 & 928 & 150 & 1798 & Weighted Logistic & 1e-04 & 24 & 2.6 \\
      \textbf{- 2step MF} & Position$_{2}$, Lab-Color (5D) & 5.00, 0.05, 0.30 & 665 & 2 & 928 & 150 & 1798 & Weighted Logistic & 1e-04 & 12 & +0.7 \\
      \textbf{- \textit{loose} 2step MF} & Position$_{2}$, Lab-Color (5D) & 5.00, 0.05, 0.30 & 665 & 2 & 928 & 150 & 1798 & Weighted Logistic & 1e-04 & 12 & +0.2\\
      \midrule
      \multicolumn{2}{c}{\textbf{Neural Network Applications}} & & & & & & & & &\\
      \textbf{Tiles: CNN-9$\times$9} & - & - & 81 & 4 & 10000 & 1000 & 1000 & Logistic & 0.01 & 100 & 500.0 \\
      \textbf{Tiles: CNN-13$\times$13} & - & - & 169 & 6 & 10000 & 1000 & 1000 & Logistic & 0.01 & 100 & 500.0 \\
      \textbf{Tiles: CNN-17$\times$17} & - & - & 289 & 8 & 10000 & 1000 & 1000 & Logistic & 0.01 & 100 & 500.0 \\
      \textbf{Tiles: CNN-21$\times$21} & - & - & 441 & 10 & 10000 & 1000 & 1000 & Logistic & 0.01 & 100 & 500.0 \\
      \textbf{Tiles: BNN} & Position$_{1}$, Color (5D) & 0.05, 0.04 & 63 & 1 & 10000 & 1000 & 1000 & Logistic & 0.01 & 100 & 30.0 \\
      \textbf{LeNet} & - & - & 25 & 2 & 5490 & 1098 & 1647 & Logistic & 0.1 & 100 & 182.2 \\
      \textbf{Crop-LeNet} & - & - & 25 & 2 & 5490 & 1098 & 1647 & Logistic & 0.1 & 100 & 182.2 \\
      \textbf{BNN-LeNet} & Position$_{2}$ (2D) & 20.00 & 7 & 1 & 5490 & 1098 & 1647 & Logistic & 0.1 & 100 & 182.2 \\
      \textbf{DeepCNet} & - & - & 9 & 1 & 5490 & 1098 & 1647 & Logistic & 0.1 & 100 & 182.2 \\
      \textbf{Crop-DeepCNet} & - & - & 9 & 1 & 5490 & 1098 & 1647 & Logistic & 0.1 & 100 & 182.2 \\
      \textbf{BNN-DeepCNet} & Position$_{2}$ (2D) & 40.00  & 7 & 1 & 5490 & 1098 & 1647 & Logistic & 0.1 & 100 & 182.2 \\
      \bottomrule
      \\
    \end{tabular}
    \mycaption{Experiment Protocols} {Experiment protocols for the different experiments presented in this work. \textbf{Feature Types}:
    Feature spaces used for the bilateral convolutions. Position$_1$ corresponds to un-normalized pixel positions whereas Position$_2$ corresponds
    to pixel positions normalized to $[0,1]$ with respect to the given image. \textbf{Feature Scales}: Cross-validated scales for the features used.
     \textbf{Filter Size}: Number of elements in the filter that is being learned. \textbf{Filter Nbr.}: Half-width of the filter. \textbf{Train},
     \textbf{Val.} and \textbf{Test} corresponds to the number of train, validation and test images used in the experiment. \textbf{Loss Type}: Type
     of loss used for back-propagation. ``MSE'' corresponds to Euclidean mean squared error loss and ``Logistic'' corresponds to multinomial logistic
     loss. ``Weighted Logistic'' is the class-weighted multinomial logistic loss. We weighted the loss with inverse class probability for material
     segmentation task due to the small availability of training data with class imbalance. \textbf{LR}: Fixed learning rate used in stochastic gradient
     descent. \textbf{Batch}: Number of images used in one parameter update step. \textbf{Epochs}: Number of training epochs. In all the experiments,
     we used fixed momentum of 0.9 and weight decay of 0.0005 for stochastic gradient descent. ```Color Upsampling'' experiments in this Table corresponds
     to those performed on Pascal VOC12 dataset images. For all experiments using Pascal VOC12 images, we use extended
     training segmentation dataset available from~\cite{hariharan2011moredata}, and used standard validation and test splits
     from the main dataset~\cite{voc2012segmentation}.}
  \label{tbl:parameters}
\end{table*}

\clearpage

\section{Parameters and Additional Results for Video Propagation Networks}

In this Section, we present experiment protocols and additional qualitative results for experiments
on video object segmentation, semantic video segmentation and video color
propagation. Table~\ref{tbl:parameters_supp} shows the feature scales and other parameters used in different experiments.
Figures~\ref{fig:video_seg_pos_supp} show some qualitative results on video object segmentation
with some failure cases in Fig.~\ref{fig:video_seg_neg_supp}.
Figure~\ref{fig:semantic_visuals_supp} shows some qualitative results on semantic video segmentation and
Fig.~\ref{fig:color_visuals_supp} shows results on video color propagation.

\newcolumntype{L}[1]{>{\raggedright\let\newline\\\arraybackslash\hspace{0pt}}b{#1}}
\newcolumntype{C}[1]{>{\centering\let\newline\\\arraybackslash\hspace{0pt}}b{#1}}
\newcolumntype{R}[1]{>{\raggedleft\let\newline\\\arraybackslash\hspace{0pt}}b{#1}}

\begin{table*}[h]
\tiny
  \centering
    \begin{tabular}{L{3.0cm} L{2.4cm} L{2.8cm} L{2.8cm} C{0.5cm} C{1.0cm} L{1.2cm}}
      \toprule
\textbf{Experiment} & \textbf{Feature Type} & \textbf{Feature Scale-1, $\Lambda_a$} & \textbf{Feature Scale-2, $\Lambda_b$} & \textbf{$\alpha$} & \textbf{Input Frames} & \textbf{Loss Type} \\
      \midrule
      \textbf{Video Object Segmentation} & ($x,y,Y,Cb,Cr,t$) & (0.02,0.02,0.07,0.4,0.4,0.01) & (0.03,0.03,0.09,0.5,0.5,0.2) & 0.5 & 9 & Logistic\\
      \midrule
      \textbf{Semantic Video Segmentation} & & & & & \\
      \textbf{with CNN1~\cite{yu2015multi}-NoFlow} & ($x,y,R,G,B,t$) & (0.08,0.08,0.2,0.2,0.2,0.04) & (0.11,0.11,0.2,0.2,0.2,0.04) & 0.5 & 3 & Logistic \\
      \textbf{with CNN1~\cite{yu2015multi}-Flow} & ($x+u_x,y+u_y,R,G,B,t$) & (0.11,0.11,0.14,0.14,0.14,0.03) & (0.08,0.08,0.12,0.12,0.12,0.01) & 0.65 & 3 & Logistic\\
      \textbf{with CNN2~\cite{richter2016playing}-Flow} & ($x+u_x,y+u_y,R,G,B,t$) & (0.08,0.08,0.2,0.2,0.2,0.04) & (0.09,0.09,0.25,0.25,0.25,0.03) & 0.5 & 4 & Logistic\\
      \midrule
      \textbf{Video Color Propagation} & ($x,y,I,t$)  & (0.04,0.04,0.2,0.04) & No second kernel & 1 & 4 & MSE\\
      \bottomrule
      \\
    \end{tabular}
    \mycaption{Experiment Protocols} {Experiment protocols for the different experiments presented in this work. \textbf{Feature Types}:
    Feature spaces used for the bilateral convolutions, with position ($x,y$) and color
    ($R,G,B$ or $Y,Cb,Cr$) features $\in [0,255]$. $u_x$, $u_y$ denotes optical flow with respect
    to the present frame and $I$ denotes grayscale intensity.
    \textbf{Feature Scales ($\Lambda_a, \Lambda_b$)}: Cross-validated scales for the features used.
    \textbf{$\alpha$}: Exponential time decay for the input frames.
    \textbf{Input Frames}: Number of input frames for VPN.
    \textbf{Loss Type}: Type
     of loss used for back-propagation. ``MSE'' corresponds to Euclidean mean squared error loss and ``Logistic'' corresponds to multinomial logistic loss.}
  \label{tbl:parameters_supp}
\end{table*}

% \begin{figure}[th!]
% \begin{center}
%   \centerline{\includegraphics[width=\textwidth]{figures/video_seg_visuals_supp_small.pdf}}
%     \mycaption{Video Object Segmentation}
%     {Shown are the different frames in example videos with the corresponding
%     ground truth (GT) masks, predictions from BVS~\cite{marki2016bilateral},
%     OFL~\cite{tsaivideo}, VPN (VPN-Stage2) and VPN-DLab (VPN-DeepLab) models.}
%     \label{fig:video_seg_small_supp}
% \end{center}
% \vspace{-1.0cm}
% \end{figure}

\begin{figure}[th!]
\begin{center}
  \centerline{\includegraphics[width=0.7\textwidth]{figures/video_seg_visuals_supp_positive.pdf}}
    \mycaption{Video Object Segmentation}
    {Shown are the different frames in example videos with the corresponding
    ground truth (GT) masks, predictions from BVS~\cite{marki2016bilateral},
    OFL~\cite{tsaivideo}, VPN (VPN-Stage2) and VPN-DLab (VPN-DeepLab) models.}
    \label{fig:video_seg_pos_supp}
\end{center}
\vspace{-1.0cm}
\end{figure}

\begin{figure}[th!]
\begin{center}
  \centerline{\includegraphics[width=0.7\textwidth]{figures/video_seg_visuals_supp_negative.pdf}}
    \mycaption{Failure Cases for Video Object Segmentation}
    {Shown are the different frames in example videos with the corresponding
    ground truth (GT) masks, predictions from BVS~\cite{marki2016bilateral},
    OFL~\cite{tsaivideo}, VPN (VPN-Stage2) and VPN-DLab (VPN-DeepLab) models.}
    \label{fig:video_seg_neg_supp}
\end{center}
\vspace{-1.0cm}
\end{figure}

\begin{figure}[th!]
\begin{center}
  \centerline{\includegraphics[width=0.9\textwidth]{figures/supp_semantic_visual.pdf}}
    \mycaption{Semantic Video Segmentation}
    {Input video frames and the corresponding ground truth (GT)
    segmentation together with the predictions of CNN~\cite{yu2015multi} and with
    VPN-Flow.}
    \label{fig:semantic_visuals_supp}
\end{center}
\vspace{-0.7cm}
\end{figure}

\begin{figure}[th!]
\begin{center}
  \centerline{\includegraphics[width=\textwidth]{figures/colorization_visuals_supp.pdf}}
  \mycaption{Video Color Propagation}
  {Input grayscale video frames and corresponding ground-truth (GT) color images
  together with color predictions of Levin et al.~\cite{levin2004colorization} and VPN-Stage1 models.}
  \label{fig:color_visuals_supp}
\end{center}
\vspace{-0.7cm}
\end{figure}

\clearpage

\section{Additional Material for Bilateral Inception Networks}
\label{sec:binception-app}

In this section of the Appendix, we first discuss the use of approximate bilateral
filtering in BI modules (Sec.~\ref{sec:lattice}).
Later, we present some qualitative results using different models for the approach presented in
Chapter~\ref{chap:binception} (Sec.~\ref{sec:qualitative-app}).

\subsection{Approximate Bilateral Filtering}
\label{sec:lattice}

The bilateral inception module presented in Chapter~\ref{chap:binception} computes a matrix-vector
product between a Gaussian filter $K$ and a vector of activations $\bz_c$.
Bilateral filtering is an important operation and many algorithmic techniques have been
proposed to speed-up this operation~\cite{paris2006fast,adams2010fast,gastal2011domain}.
In the main paper we opted to implement what can be considered the
brute-force variant of explicitly constructing $K$ and then using BLAS to compute the
matrix-vector product. This resulted in a few millisecond operation.
The explicit way to compute is possible due to the
reduction to super-pixels, e.g., it would not work for DenseCRF variants
that operate on the full image resolution.

Here, we present experiments where we use the fast approximate bilateral filtering
algorithm of~\cite{adams2010fast}, which is also used in Chapter~\ref{chap:bnn}
for learning sparse high dimensional filters. This
choice allows for larger dimensions of matrix-vector multiplication. The reason for choosing
the explicit multiplication in Chapter~\ref{chap:binception} was that it was computationally faster.
For the small sizes of the involved matrices and vectors, the explicit computation is sufficient and we had no
GPU implementation of an approximate technique that matched this runtime. Also it
is conceptually easier and the gradient to the feature transformations ($\Lambda \mathbf{f}$) is
obtained using standard matrix calculus.

\subsubsection{Experiments}

We modified the existing segmentation architectures analogous to those in Chapter~\ref{chap:binception}.
The main difference is that, here, the inception modules use the lattice
approximation~\cite{adams2010fast} to compute the bilateral filtering.
Using the lattice approximation did not allow us to back-propagate through feature transformations ($\Lambda$)
and thus we used hand-specified feature scales as will be explained later.
Specifically, we take CNN architectures from the works
of~\cite{chen2014semantic,zheng2015conditional,bell2015minc} and insert the BI modules between
the spatial FC layers.
We use superpixels from~\cite{DollarICCV13edges}
for all the experiments with the lattice approximation. Experiments are
performed using Caffe neural network framework~\cite{jia2014caffe}.

\begin{table}
  \small
  \centering
  \begin{tabular}{p{5.5cm}>{\raggedright\arraybackslash}p{1.4cm}>{\centering\arraybackslash}p{2.2cm}}
    \toprule
		\textbf{Model} & \emph{IoU} & \emph{Runtime}(ms) \\
    \midrule

    %%%%%%%%%%%% Scores computed by us)%%%%%%%%%%%%
		\deeplablargefov & 68.9 & 145ms\\
    \midrule
    \bi{7}{2}-\bi{8}{10}& \textbf{73.8} & +600 \\
    \midrule
    \deeplablargefovcrf~\cite{chen2014semantic} & 72.7 & +830\\
    \deeplabmsclargefovcrf~\cite{chen2014semantic} & \textbf{73.6} & +880\\
    DeepLab-EdgeNet~\cite{chen2015semantic} & 71.7 & +30\\
    DeepLab-EdgeNet-CRF~\cite{chen2015semantic} & \textbf{73.6} & +860\\
  \bottomrule \\
  \end{tabular}
  \mycaption{Semantic Segmentation using the DeepLab model}
  {IoU scores on the Pascal VOC12 segmentation test dataset
  with different models and our modified inception model.
  Also shown are the corresponding runtimes in milliseconds. Runtimes
  also include superpixel computations (300 ms with Dollar superpixels~\cite{DollarICCV13edges})}
  \label{tab:largefovresults}
\end{table}

\paragraph{Semantic Segmentation}
The experiments in this section use the Pascal VOC12 segmentation dataset~\cite{voc2012segmentation} with 21 object classes and the images have a maximum resolution of 0.25 megapixels.
For all experiments on VOC12, we train using the extended training set of
10581 images collected by~\cite{hariharan2011moredata}.
We modified the \deeplab~network architecture of~\cite{chen2014semantic} and
the CRFasRNN architecture from~\cite{zheng2015conditional} which uses a CNN with
deconvolution layers followed by DenseCRF trained end-to-end.

\paragraph{DeepLab Model}\label{sec:deeplabmodel}
We experimented with the \bi{7}{2}-\bi{8}{10} inception model.
Results using the~\deeplab~model are summarized in Tab.~\ref{tab:largefovresults}.
Although we get similar improvements with inception modules as with the
explicit kernel computation, using lattice approximation is slower.

\begin{table}
  \small
  \centering
  \begin{tabular}{p{6.4cm}>{\raggedright\arraybackslash}p{1.8cm}>{\raggedright\arraybackslash}p{1.8cm}}
    \toprule
    \textbf{Model} & \emph{IoU (Val)} & \emph{IoU (Test)}\\
    \midrule
    %%%%%%%%%%%% Scores computed by us)%%%%%%%%%%%%
    CNN &  67.5 & - \\
    \deconv (CNN+Deconvolutions) & 69.8 & 72.0 \\
    \midrule
    \bi{3}{6}-\bi{4}{6}-\bi{7}{2}-\bi{8}{6}& 71.9 & - \\
    \bi{3}{6}-\bi{4}{6}-\bi{7}{2}-\bi{8}{6}-\gi{6}& 73.6 &  \href{http://host.robots.ox.ac.uk:8080/anonymous/VOTV5E.html}{\textbf{75.2}}\\
    \midrule
    \deconvcrf (CRF-RNN)~\cite{zheng2015conditional} & 73.0 & 74.7\\
    Context-CRF-RNN~\cite{yu2015multi} & ~~ - ~ & \textbf{75.3} \\
    \bottomrule \\
  \end{tabular}
  \mycaption{Semantic Segmentation using the CRFasRNN model}{IoU score corresponding to different models
  on Pascal VOC12 reduced validation / test segmentation dataset. The reduced validation set consists of 346 images
  as used in~\cite{zheng2015conditional} where we adapted the model from.}
  \label{tab:deconvresults-app}
\end{table}

\paragraph{CRFasRNN Model}\label{sec:deepinception}
We add BI modules after score-pool3, score-pool4, \fc{7} and \fc{8} $1\times1$ convolution layers
resulting in the \bi{3}{6}-\bi{4}{6}-\bi{7}{2}-\bi{8}{6}
model and also experimented with another variant where $BI_8$ is followed by another inception
module, G$(6)$, with 6 Gaussian kernels.
Note that here also we discarded both deconvolution and DenseCRF parts of the original model~\cite{zheng2015conditional}
and inserted the BI modules in the base CNN and found similar improvements compared to the inception modules with explicit
kernel computaion. See Tab.~\ref{tab:deconvresults-app} for results on the CRFasRNN model.

\paragraph{Material Segmentation}
Table~\ref{tab:mincresults-app} shows the results on the MINC dataset~\cite{bell2015minc}
obtained by modifying the AlexNet architecture with our inception modules. We observe
similar improvements as with explicit kernel construction.
For this model, we do not provide any learned setup due to very limited segment training
data. The weights to combine outputs in the bilateral inception layer are
found by validation on the validation set.

\begin{table}[t]
  \small
  \centering
  \begin{tabular}{p{3.5cm}>{\centering\arraybackslash}p{4.0cm}}
    \toprule
    \textbf{Model} & Class / Total accuracy\\
    \midrule

    %%%%%%%%%%%% Scores computed by us)%%%%%%%%%%%%
    AlexNet CNN & 55.3 / 58.9 \\
    \midrule
    \bi{7}{2}-\bi{8}{6}& 68.5 / 71.8 \\
    \bi{7}{2}-\bi{8}{6}-G$(6)$& 67.6 / 73.1 \\
    \midrule
    AlexNet-CRF & 65.5 / 71.0 \\
    \bottomrule \\
  \end{tabular}
  \mycaption{Material Segmentation using AlexNet}{Pixel accuracy of different models on
  the MINC material segmentation test dataset~\cite{bell2015minc}.}
  \label{tab:mincresults-app}
\end{table}

\paragraph{Scales of Bilateral Inception Modules}
\label{sec:scales}

Unlike the explicit kernel technique presented in the main text (Chapter~\ref{chap:binception}),
we didn't back-propagate through feature transformation ($\Lambda$)
using the approximate bilateral filter technique.
So, the feature scales are hand-specified and validated, which are as follows.
The optimal scale values for the \bi{7}{2}-\bi{8}{2} model are found by validation for the best performance which are
$\sigma_{xy}$ = (0.1, 0.1) for the spatial (XY) kernel and $\sigma_{rgbxy}$ = (0.1, 0.1, 0.1, 0.01, 0.01) for color and position (RGBXY)  kernel.
Next, as more kernels are added to \bi{8}{2}, we set scales to be $\alpha$*($\sigma_{xy}$, $\sigma_{rgbxy}$).
The value of $\alpha$ is chosen as  1, 0.5, 0.1, 0.05, 0.1, at uniform interval, for the \bi{8}{10} bilateral inception module.


\subsection{Qualitative Results}
\label{sec:qualitative-app}

In this section, we present more qualitative results obtained using the BI module with explicit
kernel computation technique presented in Chapter~\ref{chap:binception}. Results on the Pascal VOC12
dataset~\cite{voc2012segmentation} using the DeepLab-LargeFOV model are shown in Fig.~\ref{fig:semantic_visuals-app},
followed by the results on MINC dataset~\cite{bell2015minc}
in Fig.~\ref{fig:material_visuals-app} and on
Cityscapes dataset~\cite{Cordts2015Cvprw} in Fig.~\ref{fig:street_visuals-app}.


\definecolor{voc_1}{RGB}{0, 0, 0}
\definecolor{voc_2}{RGB}{128, 0, 0}
\definecolor{voc_3}{RGB}{0, 128, 0}
\definecolor{voc_4}{RGB}{128, 128, 0}
\definecolor{voc_5}{RGB}{0, 0, 128}
\definecolor{voc_6}{RGB}{128, 0, 128}
\definecolor{voc_7}{RGB}{0, 128, 128}
\definecolor{voc_8}{RGB}{128, 128, 128}
\definecolor{voc_9}{RGB}{64, 0, 0}
\definecolor{voc_10}{RGB}{192, 0, 0}
\definecolor{voc_11}{RGB}{64, 128, 0}
\definecolor{voc_12}{RGB}{192, 128, 0}
\definecolor{voc_13}{RGB}{64, 0, 128}
\definecolor{voc_14}{RGB}{192, 0, 128}
\definecolor{voc_15}{RGB}{64, 128, 128}
\definecolor{voc_16}{RGB}{192, 128, 128}
\definecolor{voc_17}{RGB}{0, 64, 0}
\definecolor{voc_18}{RGB}{128, 64, 0}
\definecolor{voc_19}{RGB}{0, 192, 0}
\definecolor{voc_20}{RGB}{128, 192, 0}
\definecolor{voc_21}{RGB}{0, 64, 128}
\definecolor{voc_22}{RGB}{128, 64, 128}

\begin{figure*}[!ht]
  \small
  \centering
  \fcolorbox{white}{voc_1}{\rule{0pt}{4pt}\rule{4pt}{0pt}} Background~~
  \fcolorbox{white}{voc_2}{\rule{0pt}{4pt}\rule{4pt}{0pt}} Aeroplane~~
  \fcolorbox{white}{voc_3}{\rule{0pt}{4pt}\rule{4pt}{0pt}} Bicycle~~
  \fcolorbox{white}{voc_4}{\rule{0pt}{4pt}\rule{4pt}{0pt}} Bird~~
  \fcolorbox{white}{voc_5}{\rule{0pt}{4pt}\rule{4pt}{0pt}} Boat~~
  \fcolorbox{white}{voc_6}{\rule{0pt}{4pt}\rule{4pt}{0pt}} Bottle~~
  \fcolorbox{white}{voc_7}{\rule{0pt}{4pt}\rule{4pt}{0pt}} Bus~~
  \fcolorbox{white}{voc_8}{\rule{0pt}{4pt}\rule{4pt}{0pt}} Car~~\\
  \fcolorbox{white}{voc_9}{\rule{0pt}{4pt}\rule{4pt}{0pt}} Cat~~
  \fcolorbox{white}{voc_10}{\rule{0pt}{4pt}\rule{4pt}{0pt}} Chair~~
  \fcolorbox{white}{voc_11}{\rule{0pt}{4pt}\rule{4pt}{0pt}} Cow~~
  \fcolorbox{white}{voc_12}{\rule{0pt}{4pt}\rule{4pt}{0pt}} Dining Table~~
  \fcolorbox{white}{voc_13}{\rule{0pt}{4pt}\rule{4pt}{0pt}} Dog~~
  \fcolorbox{white}{voc_14}{\rule{0pt}{4pt}\rule{4pt}{0pt}} Horse~~
  \fcolorbox{white}{voc_15}{\rule{0pt}{4pt}\rule{4pt}{0pt}} Motorbike~~
  \fcolorbox{white}{voc_16}{\rule{0pt}{4pt}\rule{4pt}{0pt}} Person~~\\
  \fcolorbox{white}{voc_17}{\rule{0pt}{4pt}\rule{4pt}{0pt}} Potted Plant~~
  \fcolorbox{white}{voc_18}{\rule{0pt}{4pt}\rule{4pt}{0pt}} Sheep~~
  \fcolorbox{white}{voc_19}{\rule{0pt}{4pt}\rule{4pt}{0pt}} Sofa~~
  \fcolorbox{white}{voc_20}{\rule{0pt}{4pt}\rule{4pt}{0pt}} Train~~
  \fcolorbox{white}{voc_21}{\rule{0pt}{4pt}\rule{4pt}{0pt}} TV monitor~~\\


  \subfigure{%
    \includegraphics[width=.15\columnwidth]{figures/supplementary/2008_001308_given.png}
  }
  \subfigure{%
    \includegraphics[width=.15\columnwidth]{figures/supplementary/2008_001308_sp.png}
  }
  \subfigure{%
    \includegraphics[width=.15\columnwidth]{figures/supplementary/2008_001308_gt.png}
  }
  \subfigure{%
    \includegraphics[width=.15\columnwidth]{figures/supplementary/2008_001308_cnn.png}
  }
  \subfigure{%
    \includegraphics[width=.15\columnwidth]{figures/supplementary/2008_001308_crf.png}
  }
  \subfigure{%
    \includegraphics[width=.15\columnwidth]{figures/supplementary/2008_001308_ours.png}
  }\\[-2ex]


  \subfigure{%
    \includegraphics[width=.15\columnwidth]{figures/supplementary/2008_001821_given.png}
  }
  \subfigure{%
    \includegraphics[width=.15\columnwidth]{figures/supplementary/2008_001821_sp.png}
  }
  \subfigure{%
    \includegraphics[width=.15\columnwidth]{figures/supplementary/2008_001821_gt.png}
  }
  \subfigure{%
    \includegraphics[width=.15\columnwidth]{figures/supplementary/2008_001821_cnn.png}
  }
  \subfigure{%
    \includegraphics[width=.15\columnwidth]{figures/supplementary/2008_001821_crf.png}
  }
  \subfigure{%
    \includegraphics[width=.15\columnwidth]{figures/supplementary/2008_001821_ours.png}
  }\\[-2ex]



  \subfigure{%
    \includegraphics[width=.15\columnwidth]{figures/supplementary/2008_004612_given.png}
  }
  \subfigure{%
    \includegraphics[width=.15\columnwidth]{figures/supplementary/2008_004612_sp.png}
  }
  \subfigure{%
    \includegraphics[width=.15\columnwidth]{figures/supplementary/2008_004612_gt.png}
  }
  \subfigure{%
    \includegraphics[width=.15\columnwidth]{figures/supplementary/2008_004612_cnn.png}
  }
  \subfigure{%
    \includegraphics[width=.15\columnwidth]{figures/supplementary/2008_004612_crf.png}
  }
  \subfigure{%
    \includegraphics[width=.15\columnwidth]{figures/supplementary/2008_004612_ours.png}
  }\\[-2ex]


  \subfigure{%
    \includegraphics[width=.15\columnwidth]{figures/supplementary/2009_001008_given.png}
  }
  \subfigure{%
    \includegraphics[width=.15\columnwidth]{figures/supplementary/2009_001008_sp.png}
  }
  \subfigure{%
    \includegraphics[width=.15\columnwidth]{figures/supplementary/2009_001008_gt.png}
  }
  \subfigure{%
    \includegraphics[width=.15\columnwidth]{figures/supplementary/2009_001008_cnn.png}
  }
  \subfigure{%
    \includegraphics[width=.15\columnwidth]{figures/supplementary/2009_001008_crf.png}
  }
  \subfigure{%
    \includegraphics[width=.15\columnwidth]{figures/supplementary/2009_001008_ours.png}
  }\\[-2ex]




  \subfigure{%
    \includegraphics[width=.15\columnwidth]{figures/supplementary/2009_004497_given.png}
  }
  \subfigure{%
    \includegraphics[width=.15\columnwidth]{figures/supplementary/2009_004497_sp.png}
  }
  \subfigure{%
    \includegraphics[width=.15\columnwidth]{figures/supplementary/2009_004497_gt.png}
  }
  \subfigure{%
    \includegraphics[width=.15\columnwidth]{figures/supplementary/2009_004497_cnn.png}
  }
  \subfigure{%
    \includegraphics[width=.15\columnwidth]{figures/supplementary/2009_004497_crf.png}
  }
  \subfigure{%
    \includegraphics[width=.15\columnwidth]{figures/supplementary/2009_004497_ours.png}
  }\\[-2ex]



  \setcounter{subfigure}{0}
  \subfigure[\scriptsize Input]{%
    \includegraphics[width=.15\columnwidth]{figures/supplementary/2010_001327_given.png}
  }
  \subfigure[\scriptsize Superpixels]{%
    \includegraphics[width=.15\columnwidth]{figures/supplementary/2010_001327_sp.png}
  }
  \subfigure[\scriptsize GT]{%
    \includegraphics[width=.15\columnwidth]{figures/supplementary/2010_001327_gt.png}
  }
  \subfigure[\scriptsize Deeplab]{%
    \includegraphics[width=.15\columnwidth]{figures/supplementary/2010_001327_cnn.png}
  }
  \subfigure[\scriptsize +DenseCRF]{%
    \includegraphics[width=.15\columnwidth]{figures/supplementary/2010_001327_crf.png}
  }
  \subfigure[\scriptsize Using BI]{%
    \includegraphics[width=.15\columnwidth]{figures/supplementary/2010_001327_ours.png}
  }
  \mycaption{Semantic Segmentation}{Example results of semantic segmentation
  on the Pascal VOC12 dataset.
  (d)~depicts the DeepLab CNN result, (e)~CNN + 10 steps of mean-field inference,
  (f~result obtained with bilateral inception (BI) modules (\bi{6}{2}+\bi{7}{6}) between \fc~layers.}
  \label{fig:semantic_visuals-app}
\end{figure*}


\definecolor{minc_1}{HTML}{771111}
\definecolor{minc_2}{HTML}{CAC690}
\definecolor{minc_3}{HTML}{EEEEEE}
\definecolor{minc_4}{HTML}{7C8FA6}
\definecolor{minc_5}{HTML}{597D31}
\definecolor{minc_6}{HTML}{104410}
\definecolor{minc_7}{HTML}{BB819C}
\definecolor{minc_8}{HTML}{D0CE48}
\definecolor{minc_9}{HTML}{622745}
\definecolor{minc_10}{HTML}{666666}
\definecolor{minc_11}{HTML}{D54A31}
\definecolor{minc_12}{HTML}{101044}
\definecolor{minc_13}{HTML}{444126}
\definecolor{minc_14}{HTML}{75D646}
\definecolor{minc_15}{HTML}{DD4348}
\definecolor{minc_16}{HTML}{5C8577}
\definecolor{minc_17}{HTML}{C78472}
\definecolor{minc_18}{HTML}{75D6D0}
\definecolor{minc_19}{HTML}{5B4586}
\definecolor{minc_20}{HTML}{C04393}
\definecolor{minc_21}{HTML}{D69948}
\definecolor{minc_22}{HTML}{7370D8}
\definecolor{minc_23}{HTML}{7A3622}
\definecolor{minc_24}{HTML}{000000}

\begin{figure*}[!ht]
  \small % scriptsize
  \centering
  \fcolorbox{white}{minc_1}{\rule{0pt}{4pt}\rule{4pt}{0pt}} Brick~~
  \fcolorbox{white}{minc_2}{\rule{0pt}{4pt}\rule{4pt}{0pt}} Carpet~~
  \fcolorbox{white}{minc_3}{\rule{0pt}{4pt}\rule{4pt}{0pt}} Ceramic~~
  \fcolorbox{white}{minc_4}{\rule{0pt}{4pt}\rule{4pt}{0pt}} Fabric~~
  \fcolorbox{white}{minc_5}{\rule{0pt}{4pt}\rule{4pt}{0pt}} Foliage~~
  \fcolorbox{white}{minc_6}{\rule{0pt}{4pt}\rule{4pt}{0pt}} Food~~
  \fcolorbox{white}{minc_7}{\rule{0pt}{4pt}\rule{4pt}{0pt}} Glass~~
  \fcolorbox{white}{minc_8}{\rule{0pt}{4pt}\rule{4pt}{0pt}} Hair~~\\
  \fcolorbox{white}{minc_9}{\rule{0pt}{4pt}\rule{4pt}{0pt}} Leather~~
  \fcolorbox{white}{minc_10}{\rule{0pt}{4pt}\rule{4pt}{0pt}} Metal~~
  \fcolorbox{white}{minc_11}{\rule{0pt}{4pt}\rule{4pt}{0pt}} Mirror~~
  \fcolorbox{white}{minc_12}{\rule{0pt}{4pt}\rule{4pt}{0pt}} Other~~
  \fcolorbox{white}{minc_13}{\rule{0pt}{4pt}\rule{4pt}{0pt}} Painted~~
  \fcolorbox{white}{minc_14}{\rule{0pt}{4pt}\rule{4pt}{0pt}} Paper~~
  \fcolorbox{white}{minc_15}{\rule{0pt}{4pt}\rule{4pt}{0pt}} Plastic~~\\
  \fcolorbox{white}{minc_16}{\rule{0pt}{4pt}\rule{4pt}{0pt}} Polished Stone~~
  \fcolorbox{white}{minc_17}{\rule{0pt}{4pt}\rule{4pt}{0pt}} Skin~~
  \fcolorbox{white}{minc_18}{\rule{0pt}{4pt}\rule{4pt}{0pt}} Sky~~
  \fcolorbox{white}{minc_19}{\rule{0pt}{4pt}\rule{4pt}{0pt}} Stone~~
  \fcolorbox{white}{minc_20}{\rule{0pt}{4pt}\rule{4pt}{0pt}} Tile~~
  \fcolorbox{white}{minc_21}{\rule{0pt}{4pt}\rule{4pt}{0pt}} Wallpaper~~
  \fcolorbox{white}{minc_22}{\rule{0pt}{4pt}\rule{4pt}{0pt}} Water~~
  \fcolorbox{white}{minc_23}{\rule{0pt}{4pt}\rule{4pt}{0pt}} Wood~~\\
  \subfigure{%
    \includegraphics[width=.15\columnwidth]{figures/supplementary/000008468_given.png}
  }
  \subfigure{%
    \includegraphics[width=.15\columnwidth]{figures/supplementary/000008468_sp.png}
  }
  \subfigure{%
    \includegraphics[width=.15\columnwidth]{figures/supplementary/000008468_gt.png}
  }
  \subfigure{%
    \includegraphics[width=.15\columnwidth]{figures/supplementary/000008468_cnn.png}
  }
  \subfigure{%
    \includegraphics[width=.15\columnwidth]{figures/supplementary/000008468_crf.png}
  }
  \subfigure{%
    \includegraphics[width=.15\columnwidth]{figures/supplementary/000008468_ours.png}
  }\\[-2ex]

  \subfigure{%
    \includegraphics[width=.15\columnwidth]{figures/supplementary/000009053_given.png}
  }
  \subfigure{%
    \includegraphics[width=.15\columnwidth]{figures/supplementary/000009053_sp.png}
  }
  \subfigure{%
    \includegraphics[width=.15\columnwidth]{figures/supplementary/000009053_gt.png}
  }
  \subfigure{%
    \includegraphics[width=.15\columnwidth]{figures/supplementary/000009053_cnn.png}
  }
  \subfigure{%
    \includegraphics[width=.15\columnwidth]{figures/supplementary/000009053_crf.png}
  }
  \subfigure{%
    \includegraphics[width=.15\columnwidth]{figures/supplementary/000009053_ours.png}
  }\\[-2ex]




  \subfigure{%
    \includegraphics[width=.15\columnwidth]{figures/supplementary/000014977_given.png}
  }
  \subfigure{%
    \includegraphics[width=.15\columnwidth]{figures/supplementary/000014977_sp.png}
  }
  \subfigure{%
    \includegraphics[width=.15\columnwidth]{figures/supplementary/000014977_gt.png}
  }
  \subfigure{%
    \includegraphics[width=.15\columnwidth]{figures/supplementary/000014977_cnn.png}
  }
  \subfigure{%
    \includegraphics[width=.15\columnwidth]{figures/supplementary/000014977_crf.png}
  }
  \subfigure{%
    \includegraphics[width=.15\columnwidth]{figures/supplementary/000014977_ours.png}
  }\\[-2ex]


  \subfigure{%
    \includegraphics[width=.15\columnwidth]{figures/supplementary/000022922_given.png}
  }
  \subfigure{%
    \includegraphics[width=.15\columnwidth]{figures/supplementary/000022922_sp.png}
  }
  \subfigure{%
    \includegraphics[width=.15\columnwidth]{figures/supplementary/000022922_gt.png}
  }
  \subfigure{%
    \includegraphics[width=.15\columnwidth]{figures/supplementary/000022922_cnn.png}
  }
  \subfigure{%
    \includegraphics[width=.15\columnwidth]{figures/supplementary/000022922_crf.png}
  }
  \subfigure{%
    \includegraphics[width=.15\columnwidth]{figures/supplementary/000022922_ours.png}
  }\\[-2ex]


  \subfigure{%
    \includegraphics[width=.15\columnwidth]{figures/supplementary/000025711_given.png}
  }
  \subfigure{%
    \includegraphics[width=.15\columnwidth]{figures/supplementary/000025711_sp.png}
  }
  \subfigure{%
    \includegraphics[width=.15\columnwidth]{figures/supplementary/000025711_gt.png}
  }
  \subfigure{%
    \includegraphics[width=.15\columnwidth]{figures/supplementary/000025711_cnn.png}
  }
  \subfigure{%
    \includegraphics[width=.15\columnwidth]{figures/supplementary/000025711_crf.png}
  }
  \subfigure{%
    \includegraphics[width=.15\columnwidth]{figures/supplementary/000025711_ours.png}
  }\\[-2ex]


  \subfigure{%
    \includegraphics[width=.15\columnwidth]{figures/supplementary/000034473_given.png}
  }
  \subfigure{%
    \includegraphics[width=.15\columnwidth]{figures/supplementary/000034473_sp.png}
  }
  \subfigure{%
    \includegraphics[width=.15\columnwidth]{figures/supplementary/000034473_gt.png}
  }
  \subfigure{%
    \includegraphics[width=.15\columnwidth]{figures/supplementary/000034473_cnn.png}
  }
  \subfigure{%
    \includegraphics[width=.15\columnwidth]{figures/supplementary/000034473_crf.png}
  }
  \subfigure{%
    \includegraphics[width=.15\columnwidth]{figures/supplementary/000034473_ours.png}
  }\\[-2ex]


  \subfigure{%
    \includegraphics[width=.15\columnwidth]{figures/supplementary/000035463_given.png}
  }
  \subfigure{%
    \includegraphics[width=.15\columnwidth]{figures/supplementary/000035463_sp.png}
  }
  \subfigure{%
    \includegraphics[width=.15\columnwidth]{figures/supplementary/000035463_gt.png}
  }
  \subfigure{%
    \includegraphics[width=.15\columnwidth]{figures/supplementary/000035463_cnn.png}
  }
  \subfigure{%
    \includegraphics[width=.15\columnwidth]{figures/supplementary/000035463_crf.png}
  }
  \subfigure{%
    \includegraphics[width=.15\columnwidth]{figures/supplementary/000035463_ours.png}
  }\\[-2ex]


  \setcounter{subfigure}{0}
  \subfigure[\scriptsize Input]{%
    \includegraphics[width=.15\columnwidth]{figures/supplementary/000035993_given.png}
  }
  \subfigure[\scriptsize Superpixels]{%
    \includegraphics[width=.15\columnwidth]{figures/supplementary/000035993_sp.png}
  }
  \subfigure[\scriptsize GT]{%
    \includegraphics[width=.15\columnwidth]{figures/supplementary/000035993_gt.png}
  }
  \subfigure[\scriptsize AlexNet]{%
    \includegraphics[width=.15\columnwidth]{figures/supplementary/000035993_cnn.png}
  }
  \subfigure[\scriptsize +DenseCRF]{%
    \includegraphics[width=.15\columnwidth]{figures/supplementary/000035993_crf.png}
  }
  \subfigure[\scriptsize Using BI]{%
    \includegraphics[width=.15\columnwidth]{figures/supplementary/000035993_ours.png}
  }
  \mycaption{Material Segmentation}{Example results of material segmentation.
  (d)~depicts the AlexNet CNN result, (e)~CNN + 10 steps of mean-field inference,
  (f)~result obtained with bilateral inception (BI) modules (\bi{7}{2}+\bi{8}{6}) between
  \fc~layers.}
\label{fig:material_visuals-app}
\end{figure*}


\definecolor{city_1}{RGB}{128, 64, 128}
\definecolor{city_2}{RGB}{244, 35, 232}
\definecolor{city_3}{RGB}{70, 70, 70}
\definecolor{city_4}{RGB}{102, 102, 156}
\definecolor{city_5}{RGB}{190, 153, 153}
\definecolor{city_6}{RGB}{153, 153, 153}
\definecolor{city_7}{RGB}{250, 170, 30}
\definecolor{city_8}{RGB}{220, 220, 0}
\definecolor{city_9}{RGB}{107, 142, 35}
\definecolor{city_10}{RGB}{152, 251, 152}
\definecolor{city_11}{RGB}{70, 130, 180}
\definecolor{city_12}{RGB}{220, 20, 60}
\definecolor{city_13}{RGB}{255, 0, 0}
\definecolor{city_14}{RGB}{0, 0, 142}
\definecolor{city_15}{RGB}{0, 0, 70}
\definecolor{city_16}{RGB}{0, 60, 100}
\definecolor{city_17}{RGB}{0, 80, 100}
\definecolor{city_18}{RGB}{0, 0, 230}
\definecolor{city_19}{RGB}{119, 11, 32}
\begin{figure*}[!ht]
  \small % scriptsize
  \centering


  \subfigure{%
    \includegraphics[width=.18\columnwidth]{figures/supplementary/frankfurt00000_016005_given.png}
  }
  \subfigure{%
    \includegraphics[width=.18\columnwidth]{figures/supplementary/frankfurt00000_016005_sp.png}
  }
  \subfigure{%
    \includegraphics[width=.18\columnwidth]{figures/supplementary/frankfurt00000_016005_gt.png}
  }
  \subfigure{%
    \includegraphics[width=.18\columnwidth]{figures/supplementary/frankfurt00000_016005_cnn.png}
  }
  \subfigure{%
    \includegraphics[width=.18\columnwidth]{figures/supplementary/frankfurt00000_016005_ours.png}
  }\\[-2ex]

  \subfigure{%
    \includegraphics[width=.18\columnwidth]{figures/supplementary/frankfurt00000_004617_given.png}
  }
  \subfigure{%
    \includegraphics[width=.18\columnwidth]{figures/supplementary/frankfurt00000_004617_sp.png}
  }
  \subfigure{%
    \includegraphics[width=.18\columnwidth]{figures/supplementary/frankfurt00000_004617_gt.png}
  }
  \subfigure{%
    \includegraphics[width=.18\columnwidth]{figures/supplementary/frankfurt00000_004617_cnn.png}
  }
  \subfigure{%
    \includegraphics[width=.18\columnwidth]{figures/supplementary/frankfurt00000_004617_ours.png}
  }\\[-2ex]

  \subfigure{%
    \includegraphics[width=.18\columnwidth]{figures/supplementary/frankfurt00000_020880_given.png}
  }
  \subfigure{%
    \includegraphics[width=.18\columnwidth]{figures/supplementary/frankfurt00000_020880_sp.png}
  }
  \subfigure{%
    \includegraphics[width=.18\columnwidth]{figures/supplementary/frankfurt00000_020880_gt.png}
  }
  \subfigure{%
    \includegraphics[width=.18\columnwidth]{figures/supplementary/frankfurt00000_020880_cnn.png}
  }
  \subfigure{%
    \includegraphics[width=.18\columnwidth]{figures/supplementary/frankfurt00000_020880_ours.png}
  }\\[-2ex]



  \subfigure{%
    \includegraphics[width=.18\columnwidth]{figures/supplementary/frankfurt00001_007285_given.png}
  }
  \subfigure{%
    \includegraphics[width=.18\columnwidth]{figures/supplementary/frankfurt00001_007285_sp.png}
  }
  \subfigure{%
    \includegraphics[width=.18\columnwidth]{figures/supplementary/frankfurt00001_007285_gt.png}
  }
  \subfigure{%
    \includegraphics[width=.18\columnwidth]{figures/supplementary/frankfurt00001_007285_cnn.png}
  }
  \subfigure{%
    \includegraphics[width=.18\columnwidth]{figures/supplementary/frankfurt00001_007285_ours.png}
  }\\[-2ex]


  \subfigure{%
    \includegraphics[width=.18\columnwidth]{figures/supplementary/frankfurt00001_059789_given.png}
  }
  \subfigure{%
    \includegraphics[width=.18\columnwidth]{figures/supplementary/frankfurt00001_059789_sp.png}
  }
  \subfigure{%
    \includegraphics[width=.18\columnwidth]{figures/supplementary/frankfurt00001_059789_gt.png}
  }
  \subfigure{%
    \includegraphics[width=.18\columnwidth]{figures/supplementary/frankfurt00001_059789_cnn.png}
  }
  \subfigure{%
    \includegraphics[width=.18\columnwidth]{figures/supplementary/frankfurt00001_059789_ours.png}
  }\\[-2ex]


  \subfigure{%
    \includegraphics[width=.18\columnwidth]{figures/supplementary/frankfurt00001_068208_given.png}
  }
  \subfigure{%
    \includegraphics[width=.18\columnwidth]{figures/supplementary/frankfurt00001_068208_sp.png}
  }
  \subfigure{%
    \includegraphics[width=.18\columnwidth]{figures/supplementary/frankfurt00001_068208_gt.png}
  }
  \subfigure{%
    \includegraphics[width=.18\columnwidth]{figures/supplementary/frankfurt00001_068208_cnn.png}
  }
  \subfigure{%
    \includegraphics[width=.18\columnwidth]{figures/supplementary/frankfurt00001_068208_ours.png}
  }\\[-2ex]

  \subfigure{%
    \includegraphics[width=.18\columnwidth]{figures/supplementary/frankfurt00001_082466_given.png}
  }
  \subfigure{%
    \includegraphics[width=.18\columnwidth]{figures/supplementary/frankfurt00001_082466_sp.png}
  }
  \subfigure{%
    \includegraphics[width=.18\columnwidth]{figures/supplementary/frankfurt00001_082466_gt.png}
  }
  \subfigure{%
    \includegraphics[width=.18\columnwidth]{figures/supplementary/frankfurt00001_082466_cnn.png}
  }
  \subfigure{%
    \includegraphics[width=.18\columnwidth]{figures/supplementary/frankfurt00001_082466_ours.png}
  }\\[-2ex]

  \subfigure{%
    \includegraphics[width=.18\columnwidth]{figures/supplementary/lindau00033_000019_given.png}
  }
  \subfigure{%
    \includegraphics[width=.18\columnwidth]{figures/supplementary/lindau00033_000019_sp.png}
  }
  \subfigure{%
    \includegraphics[width=.18\columnwidth]{figures/supplementary/lindau00033_000019_gt.png}
  }
  \subfigure{%
    \includegraphics[width=.18\columnwidth]{figures/supplementary/lindau00033_000019_cnn.png}
  }
  \subfigure{%
    \includegraphics[width=.18\columnwidth]{figures/supplementary/lindau00033_000019_ours.png}
  }\\[-2ex]

  \subfigure{%
    \includegraphics[width=.18\columnwidth]{figures/supplementary/lindau00052_000019_given.png}
  }
  \subfigure{%
    \includegraphics[width=.18\columnwidth]{figures/supplementary/lindau00052_000019_sp.png}
  }
  \subfigure{%
    \includegraphics[width=.18\columnwidth]{figures/supplementary/lindau00052_000019_gt.png}
  }
  \subfigure{%
    \includegraphics[width=.18\columnwidth]{figures/supplementary/lindau00052_000019_cnn.png}
  }
  \subfigure{%
    \includegraphics[width=.18\columnwidth]{figures/supplementary/lindau00052_000019_ours.png}
  }\\[-2ex]




  \subfigure{%
    \includegraphics[width=.18\columnwidth]{figures/supplementary/lindau00027_000019_given.png}
  }
  \subfigure{%
    \includegraphics[width=.18\columnwidth]{figures/supplementary/lindau00027_000019_sp.png}
  }
  \subfigure{%
    \includegraphics[width=.18\columnwidth]{figures/supplementary/lindau00027_000019_gt.png}
  }
  \subfigure{%
    \includegraphics[width=.18\columnwidth]{figures/supplementary/lindau00027_000019_cnn.png}
  }
  \subfigure{%
    \includegraphics[width=.18\columnwidth]{figures/supplementary/lindau00027_000019_ours.png}
  }\\[-2ex]



  \setcounter{subfigure}{0}
  \subfigure[\scriptsize Input]{%
    \includegraphics[width=.18\columnwidth]{figures/supplementary/lindau00029_000019_given.png}
  }
  \subfigure[\scriptsize Superpixels]{%
    \includegraphics[width=.18\columnwidth]{figures/supplementary/lindau00029_000019_sp.png}
  }
  \subfigure[\scriptsize GT]{%
    \includegraphics[width=.18\columnwidth]{figures/supplementary/lindau00029_000019_gt.png}
  }
  \subfigure[\scriptsize Deeplab]{%
    \includegraphics[width=.18\columnwidth]{figures/supplementary/lindau00029_000019_cnn.png}
  }
  \subfigure[\scriptsize Using BI]{%
    \includegraphics[width=.18\columnwidth]{figures/supplementary/lindau00029_000019_ours.png}
  }%\\[-2ex]

  \mycaption{Street Scene Segmentation}{Example results of street scene segmentation.
  (d)~depicts the DeepLab results, (e)~result obtained by adding bilateral inception (BI) modules (\bi{6}{2}+\bi{7}{6}) between \fc~layers.}
\label{fig:street_visuals-app}
\end{figure*}


\end{document}
