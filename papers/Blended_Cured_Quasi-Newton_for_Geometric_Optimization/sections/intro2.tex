\section{Introduction}
Many fundamental \emph{physical} and \emph{geometric} modeling tasks
reduce to minimizing nonlinear measures of deformation over meshes.
Simulating elastic bodies, parametrization, deformation, shape
interpolation, deformable inverse kinematics, and animation all
require robust, efficient, and easily automated \emph{geometry
optimization}.
By \emph{robust} we mean the algorithm should solve
every reasonable problem to any accuracy given commensurate time,
and only reports success when the accuracy has truly been achieved.
By \emph{efficient} we mean rapid convergence in wall-clock time,
even if that may mean more (but cheaper) iterations.
By \emph{automated} we mean the user needn't adjust algorithm
parameters or tolerances at all to get good results
when going between different problems. With these three attributes,
a geometry optimization algorithm can be reliably used in production
software.

We propose a new algorithm, Blended Cured Quasi-Newton (BCQN),
with three core contributions based on analysis of where prior methods
faced difficulties:
\begin{itemize}
\item an \textbf{adaptively blended quadratic energy proxy} for the
deformation energy combining the Sobolev gradient and a quasi-Newton
secant approximation, allowing a low cost per iterate with second-order
acceleration but avoiding secant artifacts where the Laplacian is
more robust;
\item a \textbf{barrier-aware filter} 
on search directions, that gains larger step sizes and so improved convergence progress in line search for the common case of iterates where individual elements degenerate towards collapse.
\item a \textbf{characteristic gradient norm convergence criterion},
which is immune to terminating prematurely due to algorithm stagnation,
and is consistent across mesh sizes, scales, and choice of energy
so per-problem adjustment is unnecessary.
\end{itemize}
Over a wide range of test cases we show that BCQN makes the solution
of some previously intractable problems practical, offers up to an
order of magnitude speed-up in other cases, and in all cases
investigated so far either improves on or closely matches the
performance of the best-in-class optimizers available. We claim
BCQN achieves our goals for production software.

