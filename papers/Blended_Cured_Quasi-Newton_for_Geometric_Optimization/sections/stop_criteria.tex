\section{Termination Criteria}
\label{sec:term}

Every iterative method for minimizing an objective function $E(x)$
must incorporate stopping criteria: when should an approximate
solution be considered good enough to stop and claim success?
Clearly, in the usual case where the actual minimum value of $E(x)$
is unknown, basing the test on the current value of $E(x_i)$ is
futile. As noted in Section~\ref{sec:termination_woes},
stopping when successive iterates are closer than some tolerance is
vulnerable to false positives (halting far from a solution), as is
using a fixed number of iterations. Although monitoring $\|\nabla E\|$
is robust, each individual problem may need a different tolerance to
define a satisfactory solution even when normalized by number of vertices:
see Figures\ \ref{fig:term_compare_1} and\ \ref{fig:term_compare_2}.
We thus propose a new way to derive and construct an appropriate,
roughly problem-independent, relative scale for a gradient-based
measure for a stopping criterion.

\subsection{Characteristic Gradient Norm}

All energies we consider are summations of per-element energy densities $W(\cdot)$ computed
from the deformation gradient $F_t(x)$ and weights $a_t$, in each element $t$, as per equation (\ref{eq:obj}). 
To simplify the following we can then evaluate energy densities on
the vectorized deformation gradient as $W\big(vec(F_t)\big) =  W(G_t
x)$, where $G_t$ is the linear gradient operator for element $t$.
The full energy gradient is then
\begin{equation}
    \nabla E(x) = \sum_{t \in T} a_t  G_t^T \nabla W(G_t x).
\end{equation}
We wish to generate a ``characteristic'' value we can compare this gradient to meaningfully, with the same
dimensions; we will do this with each component of the above summation separately.

First observe that the deformation gradient, $F_t$, the argument to $W$, is dimensionless and therefore
$\nabla W$ has the same dimensions as $W$, and even as the element Hessian $\nabla^2 W$. For the simplest
quadratic energy densities, this Hessian has the attractive property of being constant; we thus choose
to use the 2-norm of the Hessian, evaluated about the deformation gradient at rest ($F_t=I$), to get
a representative value for $\nabla W$:
\begin{equation}
    \langle W \rangle = \|\nabla^2 W(I)\|_2.
\end{equation}

Second, note that the $i^\textrm{th}$ part of $G_t$ for a triangle
(respectively tetrahedra) $t$ containing vertex $i$ will attain its
maximum value for fields which are constant along the opposing edge
(triangle) and that value will be the reciprocal of the altitude.
Up to a factor of $2$ ($3$), this is the length (area) of the
opposing edge (triangle) divided by the rest area (volume), of the
element, i.e.\ $a_t$. Summing over all incident elements, weighted
by $a_t$, we arrive at a characteristic value for vertex $i$ of
$\ell_i$ equalling the perimeter (surface) area of the one-ring of
vertex $i$. We compute this value for all vertices, giving us the
vector $\ell(V,T) = (\ell_1, ..., \ell_n)^T \in \R^n$, with one
scalar entry per vertex.

The product of our energy and mesh values together form the characteristic value for the norm of the gradient
\begin{equation}
    \langle W \rangle \| \ell(V,T) \|,
\end{equation}
where we take the same vector norm as that with which we evaluate $\|\nabla E(x)\|$; we use the 2-norm in all our
experiments. For all methods we stop iterating when
\begin{equation}
    \|\nabla E(x)\| \leq \epsilon \langle W \rangle \| \ell(V,T) \|,
\end{equation}
given a dimensionless tolerance $\epsilon$ from the user, which is
now essentially mesh- and energy-independent. See Figures\
\ref{fig:term_compare_1}, \ref{fig:ours_yaron} and\ \ref{fig:term_compare_2} as
well as our experimental analysis in Section\ \ref{sec:results} for evaluation.

