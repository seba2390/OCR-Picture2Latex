\section{Conclusion}

In this work we have taken new steps to both advance the state of
the art for optimizing challenging nonconvex deformation energies
and to better evaluate new and improved methods as they are
subsequently developed. Looking forward these minimization tasks
are likely to remain fundamental bottlenecks in practical codes and
so advancement here is critical. Our three primary contributions
together form the BCQN algorithm which pushes current limits in
deformation optimization forward in terms of speed, reliability, and
automatibility.
At the same time looking ahead we also expect that each contribution
individually should lead to even more improvements in the near
future.

\subsection{Limitations and Future Work}

While our focus is on recent challenging
nonconvex energies not addressed by the popular
local-global framework, similar to AQP we have observed
significant speedup for convex energies as well.  Currently
in comparing AQP and BCQN on the same set of 2D and 3D tasks with
the convex ARAP energy we observe a generally modest improvement
in convergence, up to a little over $4\times$, which is generally balanced
by the small additional overhead of BCQN iterations. Note for energies
like ARAP there is no barrier, hence no need for our line search filtering,
but other opportunities for improvement may be abailable in future research.

While our current blending model works well in
our extensive testing, it is empirically constructed; it is in no
sense proven optimal. We believe that further analysis, better
understanding and additional improvements in quasi-Newton blending
are all exciting and promising avenues of future investigation.

Finally, we note that while we have focused here on optimizing
deformation energies defined on meshes, there is a wide range
of critical optimization problems that take similar general structure:
minimizing separable nonlinear energies on graphs. Further extensions
are thus exciting directions of ongoing investigation.