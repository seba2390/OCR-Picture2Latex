\documentclass[12pt,a4paper]{article}
 
\pdfoutput=1 
\usepackage{graphicx}
\usepackage{jcappub}
\usepackage{amsmath,amsthm,latexsym,amssymb,amsfonts,epsfig}
\usepackage{hyperref}
\usepackage{caption}
\usepackage{subcaption}
\usepackage{psfrag}
\usepackage[cp1250]{inputenc}
\usepackage[usenames,dvipsnames]{xcolor}
\usepackage{float}
\usepackage{tabularx}
\newcommand{\IB}[1]{{\color{red}IB: #1}}
\newcommand{\red}{\textcolor{red}} % effective\textcolor{red}
\newcommand{\blue}{\textcolor{blue}} % effective\textcolor{blue}
\newcommand{\green}{\textcolor{OliveGreen}}% effective\textcolor{green}
\newcommand{\orange}{\textcolor{orange}}
\newcommand{\be}{\begin{equation}}
\newcommand{\ee}{\end{equation}}
\newcommand{\bea}{\begin{eqnarray}}
\newcommand{\eea}{\end{eqnarray}}
\numberwithin{equation}{section}
\usepackage{physics}
\numberwithin{equation}{section}



\author{}
\begin{document}
\title{}
\maketitle
\subsection{Corrected version 2}
We observe that equations of motion for the background field is of the form \cite{Cai:2012va}
\begin{equation}
    \ddot{\phi} \mathcal{P} + \mathcal{D} \dot{\phi} + V,_{\phi} = 0
    \label{eq:eom}
\end{equation}
where
\begin{equation}
  \mathcal{P} = (1-g) + 6 \gamma H \dot{\phi} + 3 \beta \dot{\phi}^2 + \frac{3 \gamma^2}{2} \dot{\phi}^4
  \label{eq:p}
\end{equation}
and
\begin{equation}
\begin{split}
    \mathcal{D} & = 3 (1-g) H +(9 \gamma H^2-\frac{1}{2} g,_{\phi}) \dot{\phi} + 3 \beta H \dot{\phi}^2 \\
    &\;\;\;\;\;\;-\frac{3}{2} (1-g) \gamma \dot{\phi}^3 -\frac{9 \gamma^2 H \dot{\phi}^2}{2} -\frac{3 \beta \gamma \dot{\phi}^4 }{2}
    \end{split}
    \label{eq:d}
\end{equation}
Near the bounce the first two terms are dominant over $V,_{\phi}$ and 
\begin{equation}
    \frac{\mathcal{D}}{\mathcal{P}} \simeq -\frac{\frac{1}{2} g,_{\phi} \dot{\phi}}{  {1-g + 3 \beta \dot{\phi}^2  }}. 
    \label{eq:dbp}
\end{equation}
We arrive at this approximation from the numerical results we obtain for various parameters we consider.
\begin{figure}[H]
    \centering
    \includegraphics[width=0.45\textwidth]{dp.pdf} 
    \hspace{0.5cm}
     \includegraphics[width=0.45\textwidth]{dp approx.pdf} 
     \\
      \includegraphics[width=0.45\textwidth]{error.pdf} 
     \caption{Left,  right panels show $\frac{\mathcal{D}}{\mathcal{P}}$ and the approximate expression in equation \eqref{eq:dbp} near the bounce for one of the parameters.We have verified that the percentage error is significantly small for the rest of the parameters we have considered } 
 \label{fig:dp approx}
\end{figure}    

Upon examining various results of numerical calculations we conclude that the approximation in equation \eqref{eq:dphi} is insufficient for all the parameters we consider. In order to find a new expression for $\dot{\phi}$, we consider an initial solution to the equation of motion given by
\begin{equation}
    \dot{\phi} \approx \dot{\phi}_B e^{\frac{-t^2}{T^2}} 
    \label{eq:skgau}
\end{equation}
where $\phi_B = \sqrt{\frac{2(g_0-1)}{3 \beta}}$
and solve the equation perturbatively around the bounce.
Approximate equations of motion near the bounce are given by 
\begin{equation}
    \ddot{\phi} \approx \dot{\phi} \frac{\mathcal{D}}{\mathcal{P}} \approx \dot{\phi} \frac{\frac{1}{2} g,_{\phi} \dot{\phi}}{  {1-g + 3 \beta \dot{\phi}^2  }}.
    \label{eq:aeom}
\end{equation}
Let $\phi_0 =  \dot{\phi}_B e^{\frac{-t^2}{T^2}} $, this solution agrees with numerical calculations within a small range of error for the parameters chosen in. Now upon substituting this back in the equation
\eqref{aeom} and taylor expanding the right hand side of the equation we obtain
\begin{equation}
    \ddot{\phi} = c1 t + c2 t^2 +..
\end{equation}
where 
\begin{equation}
     c_1 = \frac{4 b^{2-\frac{1}{1+b}}(-1+g_0)g_0}{\left(1+b+2g_0+2b g_0-6b^{\frac{1}{1+b}}g_0 \right) \beta p}
\end{equation}
and 
\begin{equation}
    c_2 = \frac{4(-1+b)b^{2-\frac{1}{1+b}}g_0\frac{1}{p}^{\frac{3}{2}}\left(\frac{-1+g_0}{\beta}\rigt)^{\frac{3}{2}}}{\sqrt{3}\left(1+b+2g_0+2b g_0-6b^{\frac{1}{1+b}}g_0 \right)}
\end{equation}
 Upon solving the differential equation the term $c1$ can be identified as
\begin{equation}
   c_1 = \frac{2}{T^2} 
\end{equation}
Since for the parameters selected in . higher order terms are small, Near the bounce we are left with only the linear term. For these parameters $c1=7.9$ and $c2=3.25$, thus we get back the result $T\approx0.5$ as stated in the paper. We compare our approximations to numerical results in figure $\ref{fig:comparison}$
\begin{figure}[H]
    \centering
    \includegraphics[width=0.9\textwidth]{compare.pdf} 
     \caption{ $\phi'(t)$ as a function of t  near the bounce for the chosen parameters. } 
 \label{fig:comparison}
\end{figure} 
In  order for the higher order term to be small
\begin{equation}
    \frac{c_3}{c_2} T < 1 
\end{equation}
Largest term in the expansion for $\frac{c_3}{c_2} T$ is
\begin{equation}
    \sqrt{\frac{2}{3}} \frac{\sqrt{-1+4g_0}}{3 g_0\sqrt{b_g}}
\end{equation}
Hence the expansion remain valid approximately as long as $b_g>\frac{8}{27}$. Considering that the second term can not be ignored for some parameters, we modify equation \eqref{eq:dphi} as
\begin{equation}
    dot{\phi}(t) = \phi_b e^{-\frac{t}{T}^2+ \sigma t^3} 
    \label{eq:newphi}
\end{equation}
where $\sigma = \frac{c2}{3}$. By putting back the expression in equation \eqref{eq:newphi} in equation \eqref{eq:aeom} one can obtain further higher order terms.  contribution from the first  Integral within the WKB approximation is 
\begin{equation}
     \cosh(\int_{\tau_{B}^-}^{\tau_{B}^+} \omega(\tau'') d\tau'') 
     \label{eq:amp}
\end{equation}
where, assuming $\upsilon$ is small
\begin{equation}
    \omega^2 =   \frac{2}{T^2}+\frac{4 t^2}{T^4}-6 t \sigma -\frac{12 t^3 \sigma }{T^2}+ 9t^4 \sigma^2.
\end{equation}
Upon substituting the parameters from . we obtain amplification of order $10^{10}$ for the power spectrum, as expected. Note that duration of the bounce is different given this form of $\phi$ and the new $\tau_{B-}$ is obtained by solving the equation 
\begin{equation}
    -\frac{t}{T}^2+ \sigma t^3 = 1. 
    \label{eq:tb}
\end{equation}
Let this new $tau_{B-} = c T$. Upon minimizing the integral $\eqref{eq:amp}$ we see that
\begin{equation}
    \sigma_{min} = \frac{cT+2 c^3 T-\sqrt{c^2 T^2+2c^4T^2}}{3 c^4 T^4}
\end{equation}
Substituting this value in equation \eqref{eq:tb} $c\approx0.84$ and $\sigma \approx \frac{0.4}{T^3}$. In order for this to be satisfied, when $b_g=2$, $T\approx0.2$ and $g_0 \approx 1.8 $. Numerically integrating \eqref{eq:amp} we obtain that $r_{min} \approx (\frac{1}{134})^2$. 
\end{document}