\documentclass[12pt,a4paper]{article}
\pdfoutput=1 
\usepackage{graphicx}
\usepackage{jcappub}
\usepackage{amsmath,amsthm,latexsym,amssymb,amsfonts,epsfig}
\usepackage{hyperref}
\usepackage{caption}
\usepackage{subcaption}
\usepackage{psfrag}
\usepackage[cp1250]{inputenc}
\usepackage[usenames,dvipsnames]{xcolor}
\usepackage{float}
\usepackage{tabularx}
\newcommand{\IB}[1]{{\color{red}IB: #1}}
\newcommand{\red}{\textcolor{red}} % effective\textcolor{red}
\newcommand{\blue}{\textcolor{blue}} % effective\textcolor{blue}
\newcommand{\green}{\textcolor{OliveGreen}}% effective\textcolor{green}
\newcommand{\orange}{\textcolor{orange}}
\newcommand{\be}{\begin{equation}}
\newcommand{\ee}{\end{equation}}
\newcommand{\bea}{\begin{eqnarray}}
\newcommand{\eea}{\end{eqnarray}}
\numberwithin{equation}{section}
\usepackage{physics}
\numberwithin{equation}{section}


\title{Requiem to "Proof of Inflation" or Sourced Fluctuations in a Non-Singular Bounce}

\author[a]{Ido Ben-Dayan,}
\author[a]{Udaykrishna Thattarampilly}
\affiliation{Physics Department, Ariel University, Ariel 40700, Israel}

\emailAdd{ido.bendayan@gmail.com}
\emailAdd{uday7adat@gmail.com}

\abstract{Popular wisdom suggests that measuring the tensor to scalar ratio $r$ on CMB scales is a proof of Inflation since alternatives predict $r$ which is many orders of magnitude below the sensitivity of future experiments. A bouncing universe with sourced fluctuations allows for a nearly scale invariant spectra of both scalar and tensor perturbations challenging this point of view. Past works have analyzed the model until the bounce, under the assumption that the bounce will not change the final predictions. In this work we discard this assumption. We explicitly follow the evolution of the universe and fluctuations across the bounce until reheating. The evolution is stable and the existence of the sourced fluctuations does not destroy the bounce. The bounce enhances the scalar spectrum while leaving the tensor spectrum unchanged. The scalar spectrum matches current observations and predicts a tensor-to-scalar ratio of $r\lesssim 6\times 10^{-5}$, which is similar to predictions of small field models of Inflation.}  %(which may be observed in upcoming CMB experiments. Hence, a measurement of $r$ will no longer be a "proof of inflation".)}
 
\keywords{}

\begin{document}
Since we are only interested in retarded Green's functions, we may assume $\tau'<\tau$. Further, Green's functions are split in to to regimes, namely
\begin{equation}
    \begin{split}
 &= G_{0k}(\tau,\tau') \qquad \qquad -\tau'  < \tau < -\tau_{B}^{-} \\
 &=  G_{1k}(\tau,\tau')  \qquad \qquad -\tau'  < -\tau_{B}^{-} < -\tau\\
 &=  G_{2k}(\tau,\tau')   \qquad \qquad  -\tau_{B}^{-} < -\tau' < -\tau .
\end{split}
\end{equation}
$G_{1k}(\tau,\tau')$ and $G_{2k}(\tau,\tau')$ are calculated in appendix \ref{app:greens} and are respectively
\begin{equation}
     G_{1k}^T(\tau,\tau') \simeq \frac{1}{k} \left( \cosh(\omega(\tau-\tau_B^-)) \sin(-k\tau')  +  \frac{k}{\omega} \sinh(\omega(\tau-\tau_B^-))   \cos(k(\tau-\tau')) \right) \Theta(\tau-\tau')  .
     \label{eq:greenfstb}
\end{equation}
\begin{equation}
\begin{split}
     G_{2k}^T(\tau,\tau') &\simeq \frac{-1}{k} \cosh(\omega(\tau-\tau_{B-})) \\
     &\;\;\;\;\;\;\left(\cosh(\omega(\tau'-\tau_{B-})) \sin(-k\tau') +\frac{k}{\omega} \cos(-k\tau') \sinh(\omega(\tau'-\tau_{B-})) \right) \\
     & +\;\frac{1}{k}\sinh(\omega(\tau-\tau_{B-}))\\
     &\;\;\;\;\;\left(\frac{k}{\omega}\cos(-k\tau') \cosh(\omega(\tau'-\tau_{B-})) - \frac{k^2}{\omega^2}\sin(-k\tau') \sinh(\omega(\tau'-\tau_{B-})) \right).
     \label{eq:greenfstb2}
\end{split}
\end{equation}
\begin{split}
v_{k,s} = \int_{-\infty}^{\tau_B^-} d\tau' G_{1k}(\tau,\tau') J_k (\tau',\Vec{k}) + \int_{\tau_B^-}^{\tau_B^+} d\tau' G_{2k}(\tau,\tau') J_k (\tau',\Vec{k})  \, ,
\label{eq:vi} 
\end{split}
\end{equation}
Near the bounce, assuming that the bounce is short
\begin{equation}
    G_{1k}(\tau,\tau')= \cosh(\omega(\tau-\tau_B^-)) G_{0k}(\tau,\tau')
    \label{eq:green1}.
\end{equation}
 The first integral in \eqref{eq:vi} is the same sourced perturbations from ekpyrosis amplified by a factor of $ \cosh(\omega(\tau-\tau_B^-))$ which is approximately $1$ for short bounce and small $\omega$. Contribution from the second integral is relatively small given that bounce is short and $\frac{k}{\omega}<<1$, which is true for super horizon scales. Under these conditions 
\begin{equation}
     G_{2k}(\tau,\tau') \approx \Theta(\tau-\tau')  \tau_B^-  \cosh(\omega(\tau'-\tau_B^-)).
     \label{eq:green 2}
\end{equation}
Only significant change in spectrum caused by this change in Green's functions is in the time integral. 
Time integral for tensor perturbation is hence given by(where $\mathcal{I}^T_{ekp}$ is time integral for sourced perturbations in ekpyrosis)
\begin{equation}
\mathcal{I}^T = \mathcal{I}^T_{ekp} \cosh(\omega(\tau-\tau_B^-)) + \int^{\tau_B^+}_{\tau_B^-} d\tau' \frac{G^T_{2k}(\tau,\tau')}{a(\tau)a(\tau')} I(\tau')^{-2} \, .
\label{eq:timei}
\end{equation}
The second integral can be approximated near the bounce as 
\begin{equation}
\mathcal{I}^T \approx \int d\tau'   \tau_B^-  \cosh(\omega(\tau'-\tau_B^-))  \approx 1 \, ,
\label{eq:timei}
\end{equation}
which is very small compared to $\mathcal{I}^T_{ekp}\simeq (k\tau_{B}^-)^{-2}$ for $n=-2$ \cite{r1}.
Thus
\begin{equation}
\mathcal{I}^T \simeq \mathcal{I}^T_{ekp} \cosh(\omega(\tau-\tau_B^-)) \simeq \mathcal{I}^T_{ekp}
\label{eq:timei}
\end{equation}
Only significant change in spectrum caused by this change in Green's functions is in the time integral. 
Time integral for tensor perturbation is hence given by(where $\mathcal{I}^T_{ekp}$ is time integral for sourced perturbations in ekpyrosis)
\begin{equation}
\mathcal{I}^T = \mathcal{I}^T_{ekp} \cosh(\omega(\tau-\tau_B^-)) + \int^{\tau_B^+}_{\tau_B^-} d\tau' \frac{G^T_{2k}(\tau,\tau')}{a(\tau)a(\tau')} I(\tau')^{-2} \, .
\label{eq:timei}
\end{equation}
The second integral can be approximated near the bounce as 
\begin{equation}
\mathcal{I}^T \approx \int d\tau'   \tau_B^-  \cosh(\omega(\tau'-\tau_B^-))  \approx 1 \, ,
\label{eq:timei}
\end{equation}
which is very small compared to $\mathcal{I}^T_{ekp}\simeq (k\tau_{B}^-)^{-2}$ for $n=-2$ \cite{r1}.
Thus
\begin{equation}
\mathcal{I}^T \simeq \mathcal{I}^T_{ekp} \cosh(\omega(\tau-\tau_B^-)) \simeq \mathcal{I}^T_{ekp}
\label{eq:timei}
\end{equation}
\section{Derivation of Green's functions}
\label{app:greens}
Tensor Greens functions during the ekpyrotic phase is given by(given $\tau<\tau_B^- $ and $\tau'<\tau_B^- $)
    \begin{equation}
\begin{split}
    G^T_{0k}(\tau,\tau') &= i \Theta(\tau-\tau')  \frac{\pi}{4} \sqrt{\tau \tau'} %\times \\ & \, 
    \left[H^{(1)}_{\frac{1}{2}+b}(-k\tau) H^{(2)}_{\frac{1}{2}+b}(-k\tau')-H^{(1)}_{\frac{1}{2}+b}(-k\tau')H^{(2)}_{\frac{1}{2}+b}(-k\tau)\right].\cr
    \end{split}
    \label{eq:geensfsten}
\end{equation}
Let us consider the case where $\tau>\tau_B^-$ and  $\tau'<\tau_B^-$, Greens functions in this regime are given by
\begin{equation}
    G^T_{1k}(\tau,\tau') = d_1(\tau') e^{\omega \tau} + d_2(\tau') e^{-\omega \tau}.
\end{equation}
We use the fact that the Greens function and its derivative should be continuous at $\tau_B^-$ to derive
\IB{check:}
\begin{equation}
    \begin{split}
        d_1(\tau') = \frac{e^{-\omega \tau_B^-}}{2\omega} \left( \omega  G^T_{0k}(\tau_B^-,\tau') +  \dfrac{d}{d\tau}   G^T_{0k}(\tau_B^-,\tau')  \right)
    \end{split}
\end{equation}
and 
\begin{equation}
    \begin{split}
        d_2(\tau') = \frac{e^{\omega \tau_B^-}}{2\omega} \left( \omega G^T_{0k}(\tau_B^-,\tau') -  \dfrac{d}{d\tau}   G^T_{0k}(\tau_B^-,\tau')  \right)
    \end{split}
\end{equation}
Combining we get
\begin{equation}
     G^T_{1k}(\tau,\tau') = \left( G^T_{0k}(\tau_B^-,\tau') \cosh(\omega(\tau-\tau_B^-))  + \frac{\sinh(\omega(\tau-\tau_B^-))}{\omega}  \dfrac{d}{d\tau}G^T_{0k}(\tau_B^-,\tau') \right)\Theta(\tau-\tau')
\end{equation}
notice that $\tau>\tau'$ hence Heaviside function is simply 1.In the super horizon limit
\begin{equation}
     G_{0k}^T(\tau,\tau') \simeq \Theta(\tau-\tau')   \frac{\sin(-k\tau')}{k} .
     \label{eq:greenfst}
\end{equation}
For $b<<1$ and $k\tau<<1$
\begin{equation}
\begin{split}
    \dfrac{d}{d\tau}G^T_{0k}(\tau_B^-,\tau') \simeq \cos(k(\tau-\tau'))
    \end{split}
\end{equation}
and 
\begin{equation}
     G_{1k}^T(\tau,\tau') \simeq \frac{1}{k} \left( \cosh(\omega(\tau-\tau_B^-)) \sin(-k\tau')  +  \frac{k}{\omega} \sinh(\omega(\tau-\tau_B^-))   \cos(k(\tau-\tau')) \right) \Theta(\tau-\tau')  .
     \label{eq:greenfstbc}
\end{equation}
Similarly let us consider the case where $\tau>\tau_B^-$ and  $\tau'>\tau_B^-$, Greens functions in this regime are given by
\begin{equation}
    G^T_{2k}(\tau,\tau') = c_1(\tau) e^{\omega \tau'} + c_2(\tau) e^{-\omega \tau'}.
\end{equation}
We again use contiuity of Green's function and derivative at $\tau'=\tau_B^-$ to arrive at the results
\begin{equation}
    \begin{split}
        c_1(\tau) =  \frac{e^{-\omega \tau_B^-}}{\omega} \left( \omega  G^T_{1k}(\tau,\tau_B^-) + \dfrac{d}{d\tau'}  G^T_{1k}(\tau,\tau_B^-)   \right)
    \end{split}
\end{equation}
and 
\begin{equation}
    \begin{split}
        c_2(\tau) = \frac{e^{\omega \tau_B^-}}{\omega} \left( \omega G^T_{1k}(\tau,\tau_B^-) - \dfrac{d}{d\tau'}  G^T_{1k}(\tau,\tau_B^-)   \right).
    \end{split}
\end{equation}
Using the super horizon solutions $ G^T_{1k}$ that we derived earlier we can assess the super horizon solutions for $ G^T_{2k}$ 
\begin{equation}
\begin{split}
     G_{2k}^T(\tau,\tau') &\simeq \frac{-1}{k} \cosh(\omega(\tau-\tau_{B-})) \\
     &\;\;\;\;\;\;\left(\cosh(\omega(\tau'-\tau_{B-})) \sin(-k\tau') +\frac{k}{\omega} \cos(-k\tau') \sinh(\omega(\tau'-\tau_{B-})) \right) \\
     & +\;\frac{1}{k}\sinh(\omega(\tau-\tau_{B-}))\\
     &\;\;\;\;\;\left(\frac{k}{\omega}\cos(-k\tau') \cosh(\omega(\tau'-\tau_{B-})) - \frac{k^2}{\omega^2}\sin(-k\tau') \sinh(\omega(\tau'-\tau_{B-})) \right)
     \label{eq:greenfstbca}
\end{split}
\end{equation}
The momentum integrals are same as those calculated in \cite{r1,r4}. Hence to calculate the change in power spectrum across the bounce it is enough to calculate the time integrals. Since $v_T$ and $v_S$ are the same up to a source boundary conditions and procedure for finding Greens functions remain the same largely with the difference being that $\omega$ is a function of time and is not a small constant.
In the regime $\tau>\tau_B^-$ and  $\tau'<\tau_B^-$, Greens functions are
\begin{equation}
    G^S_{1k}(\tau,\tau') = d_1(\tau') e^{\int^{\tau} \omega d\tau''} + d_2(\tau') e^{-\int^{\tau} \omega d\tau''},
\end{equation}
when $\tau>\tau_B^-$ and  $\tau'>\tau_B^-$, Greens functions are given by
\begin{equation}
    G^S_{2k}(\tau,\tau') = d_1(\tau) e^{\int \omega d\tau'} + d_2(\tau) e^{-\int \omega d\tau'}.
\end{equation}
Using WKB approximation Greens functions for scalar perturbations are obtained by replacing $\omega$ with 
with $\int \omega d\tau$ in \eqref{eq:green 2} and \eqref{eq:green1}.
Thus final scalar Greens functions are approximately  of the form 
\begin{equation}
     G_{1k}^S(\tau,\tau') \approx \Theta(\tau-\tau')  G_{0k}(\tau,\tau') \cosh(\int_{\tau_{B}^-}^{\tau} \omega(\tau'') d\tau'') 
     \label{eq:greenssc1}
\end{equation}
and
\begin{equation}
     G_{2k}^S(\tau,\tau') \approx \Theta(\tau-\tau')   \tau_B^-  \cosh(\int_{\tau_{B}^-}^{\tau'} \omega(\tau'') d\tau'').
     \label{eq:greenssc2}
\end{equation}
The only part of the sourced fluctuations modified by this change compared to \cite{Artymowski:2020pci} is the time integral. Solution to equation \eqref{eq:final} is hence given by
\begin{equation}
\zeta = \left( \frac{1}{z(\tau_B^-)} \int^{\tau_B^-}_{-\infty} d\tau' z(\tau') \frac{G_{1k}^S(\tau,\tau')}{a(\tau)a(\tau')} I(\tau')^{-2} + \frac{1}{z(\tau_B^+)} \int^{\tau_B^+}_{\tau_B^-} d\tau' z(\tau') \frac{G_{2k}^S(\tau,\tau')}{a(\tau)a(\tau')} I(\tau')^{-2} \right) \times \mathcal{K} \, ,
\label{eq:timei}
\end{equation}
where $\mathcal{K}$ is the momentum integral. 
The first time integral  in equation \eqref{eq:timei} is 
\begin{equation}
\begin{split}
    \mathcal{I}^S_1 &=  \int^{\tau_B^-}_{-\infty} d\tau' z(\tau') \frac{G_{1k}^S(\tau,\tau')}{a(\tau)a(\tau')} I(\tau')^{-2} \\
   & =\int^{\tau_B^-}_{-\infty} d\tau' z(\tau')  I(\tau')^{-2}  G_{0k}(\tau,\tau') \cosh(\int_{\tau_{B}^-}^{\tau} \omega(\tau'') d\tau'')\\
   &= \cosh(\int_{\tau_{B}^-}^{\tau} \omega(\tau'') d\tau'') I^S_{ekp}
    \end{split}.
    \label{eq:timei1}
\end{equation}
%Which near the bounce and for super horizon scales can be approximated as 
%\begin{equation}
%\begin{split}
%\mathcal{I}^S & \approx I^S_{v_{k,ekp}} \cosh(\int_{\tau_{B}^-
%}^{\tau{B^+}}  \omega(\tau'') d\tau'') \frac{1}{(1+\omega)^2}  \\
%&+\int d\tau' z(\tau') \frac{1}{(1+\omega)^2}  \tau_B^-  \cosh(\int_{\tau_{B}^-}^{\tau'} \omega(\tau'') %\end{split}
%label{eq:timei}
%\end{equation}
%$\zeta = \frac{v}{z}$ hence
Using eqautions \eqref{eq:green 2} and \eqref{eq:timei1} in equation \eqref{eq:timei} and assuming $a(\tau) \simeq 1$ for the short duration of bounce 
\begin{equation}
\begin{split}
\zeta &\propto  \zeta_{k,ekp} \cosh(\int_{\tau_{B}^-}^{\tau_B^+} \omega(\tau'') d\tau'') \\
&+ \frac{1}{z(\tau_{B+})} \int_{\tau_{B}^-}^{\tau_B^+} d\tau' z(\tau')   \tau_B^-  \cosh(\int_{\tau_{B}^-}^{\tau'} \omega(\tau'') d\tau'') \times \mathcal{K}  ,
\end{split}
\label{eq:zeta}
\end{equation}
The second term in \eqref{eq:zeta} is smaller than the first by an order of magnitude for short bounce. Thus in general for a short bounce amplification across the bounce for sourced perturbations is 
\begin{equation}
F_{\zeta} \simeq \cosh(\int_{\tau_{B}^-}^{\tau_B^+} \omega(\tau'') d\tau'') .
\label{eq:zeta1}
\end{equation}
\end{document}