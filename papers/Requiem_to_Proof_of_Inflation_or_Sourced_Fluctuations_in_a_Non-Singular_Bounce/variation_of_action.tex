\documentclass[12pt,a4paper,draft]{article}
\usepackage[utf8]{inputenc}
\usepackage[english]{babel}
\usepackage{amsmath}
\usepackage{amsfonts}
\usepackage{amssymb}
\usepackage{makeidx}
\usepackage{graphicx}
\usepackage{lmodern}
\usepackage{physics}
\usepackage[left=2cm,right=2cm,top=2cm,bottom=2cm]{geometry}
%\newcommand{\blue}{\textcolor{blue}}



\author{}
\begin{document}
\title{}
\maketitle
\begin{equation}
\mathcal{L} = K(\phi,X)+G(\phi,X)\Box \phi
\end{equation}
The energy momentum tensor is given by (As the Lagrangian is of second order)
\begin{equation}
    T_{\mu \nu} = \pdv{\mathcal{L}}{(\grad_{\mu}\grad_{\alpha}\phi)} \grad_{\alpha}\grad_{\nu}\phi + \pdv{\mathcal{L}}{(\grad_{\mu} \phi)} \grad_{\nu} \phi- g_{\mu \nu} \mathcal{L}.
\end{equation}
Upon integrating the action by parts and removing the boundary term 
\begin{equation}
\begin{split}
    S &= \int \sqrt{-g} \left( K(\phi,X)+G(\phi,X)\Box \phi\right) d^4x = \int \sqrt{-g} \left(  K(\phi,X)- \grad_{\mu} G(\phi,X) \grad^{\mu} \phi \right) d^4x \\
    &= \int \sqrt{-g} \left( K(\phi,X)- G,_{\phi} \grad_{\mu} \phi \grad^{\mu} \phi - G,_{X} \grad_{\mu} X \grad^{\mu} \phi \right).
    \end{split}
    \label{eq:action}
\end{equation}
From this new Lagrangian one can derive the energy momentum tensor, In order to  do this notice that
\begin{equation}
    \begin{split}
    \pdv{\mathcal{L}}{(\grad_{\mu} \phi)} \grad_{\nu} \phi &= \pdv{L}{X}  \grad_{\mu} \phi \grad_{\nu} \phi \\
    &= (K,_X- 2 G,_{\phi}+G,_X \Box \phi) \grad_{\mu} \phi \grad_{\nu} \phi
    \end{split}
\end{equation}
Here we have integrated by parts, the term 
\begin{equation}
    -\int   G,_{X} \grad_{\mu} X \grad^{\mu} \phi = \int   G,_{X} X \Box \phi.
\end{equation}
The only term in the Lagrangian containing second order terms is $G,_{X} \grad_{\mu} X \grad^{\mu} \phi$. Expanding this term in terms of gradient of $\phi$ we obtain
\begin{equation}
    G,_{X} \grad_{\mu} X = G,_{X} (\grad_{\rho}\grad_{\mu} \phi+\grad_{\mu}\grad_{\rho} \phi  ) \grad^{\nu} \phi
\end{equation}
\begin{equation}
    \begin{split}
         \pdv{\mathcal{L}}{(\grad_{\mu}\grad_{\rho}\phi)} \grad_{\rho}\grad_{\nu}\phi &=\pdv{\left( G,_{X} \grad_{\alpha} X  \grad^{\alpha}\phi\right)}{(\grad_{\mu}\grad_{\rho}\phi)} \grad_{\rho}\grad_{\nu}\phi
         &=  G,_{X} (\grad_{\nu} X \grad_{\mu} \phi+\grad_{\mu}X \grad_{\nu} \phi )
    \end{split}
\end{equation}
Hence the total energy momentum tensor is 
\begin{equation}
\begin{split}
    T_{\mu \nu} &= g^{\mu \nu}\left(-K(\phi,X)+2 X G,_{\phi} + G,_{X} \grad_{\mu} X \grad^{\mu} \phi \right) \\
    & + (K,_X- 2 G,_{\phi}+G,_X \Box \phi) \grad_{\mu} \phi \grad_{\nu} \phi \\
    &- G,_{X} (\grad_{\nu} X \grad_{\mu} \phi+\grad_{\mu}X \grad_{\nu} \phi )
    \end{split}
    \label{eq:emt}
\end{equation}
\section{From the variation of action}
The energy momentum tensor can also be expressed as 
\begin{equation}
    T_{\mu \nu} = \frac{2}{\sqrt{-g}} \fdv{S_{\phi}}{g^{\mu \nu}}
\end{equation}
where $S_{\phi}$ is the matter part of the action. We know that 
\begin{equation}
    \fdv{\sqrt{-g}}{g_{\mu \nu}} = -\frac{1}{2} \sqrt{-g} \, g_{\mu \nu}. 
\end{equation}
Varying the determinant term in the action \eqref{eq:action} gives us the first 3 terms in equation \eqref{eq:emt}. \\
\begin{equation}
   \frac{2}{\sqrt{-g}} \fdv{\sqrt{-g}}{g_{\mu \nu}}\mathcal{L} = -\, g_{\mu \nu} \mathcal{L} = g^{\mu \nu}\left(-K(\phi,X)+2 X G,_{\phi} + G,_{X} \grad_{\mu} X \grad^{\mu} \phi \right) . 
\end{equation}
This gives the first three terms of the energy momentum tensor as we have seen from the previous calculations. 
Notice that
\begin{equation}
    \fdv{K}{g^{\mu \nu}} = K,_X \fdv{X}{g^{\mu \nu}}
\end{equation}
and
\begin{equation}
\begin{split}
    \fdv{X}{g^{\mu \nu}} &= \frac{1}{2} \fdv{ g^{\alpha \beta}}{g^{\mu \nu}}  \grad_{\alpha} \phi \grad_{\beta} \phi \\
    &= \frac{1}{4} \left( \delta^{\alpha}_{\mu} \delta^{\beta}_{\nu}+ \delta^{\alpha}_{\nu} \delta^{\beta}_{\mu} \right) \grad_{\alpha} \phi \grad_{\beta} \phi \\
    &=\frac{1}{4} \left( \grad_{\mu} \phi \grad_{\nu} \phi+\grad_{\nu} \phi \grad_{\mu} \phi \right) \\
    &= \frac{1}{2} \grad_{\mu} \phi \grad_{\nu} \phi
    \end{split}
\end{equation}
Hence the variation of K w.r to the metric gives us the term 
\begin{equation}
    \frac{2}{\sqrt{-g}} \sqrt{-g} K,_X \frac{1}{2} \grad_{\mu} \phi \grad_{\nu} \phi = K,_X  \grad_{\mu} \phi \grad_{\nu} \phi.
\end{equation}
Now similarly
\begin{equation}
\begin{split}
    \fdv{(G \Box \phi)}{g^{\mu \nu}} &= \frac{1}{2} G,_{X}\Box \phi \grad_{\mu} \phi \grad_{\nu} \phi + other\, terms
    \end{split}
\end{equation}
In order to find terms that does not contain the variation of G, we first perform an integration by parts and rewrite the term  $G \Box \phi$ as $-\grad_{\mu} G \grad^{\mu} \phi$ after ignoring the boundary terms. Upon varying this term we get(we do not consider any variations of G since they are already taken in to account in the previous term)
\begin{equation}
\begin{split}
  \fdv{  \left(-\grad_{\mu} G \grad^{\mu} \phi\right)}{g^{\mu \nu}} & = \fdv{ g^{\alpha \beta}}{g^{\mu \nu}} \grad_{\alpha} G \grad_{\beta} \phi 
  = \frac{1}{2} \left( \delta^{\alpha}_{\mu} \delta^{\beta}_{\nu}+ \delta^{\alpha}_{\nu} \delta^{\beta}_{\mu} \right) \grad_{\alpha} G \grad_{\beta} \phi \\
  &=  -\frac{1}{2} \left( \delta^{\alpha}_{\mu} \delta^{\beta}_{\nu}+ \delta^{\alpha}_{\nu} \delta^{\beta}_{\mu} \right) \left( G,_{\phi} \grad_{\alpha} \phi \grad_{\beta} \phi + G,_X  \grad_{\alpha} X \grad_{\beta} \phi\right)\\
  &= - G,_{\phi} \grad_{\mu} \phi \grad_{\nu} \phi-\frac{1}{2} G,_X \left(  \grad_{\mu} X \grad_{\nu} \phi -\grad_{\nu} X \grad_{\mu} \phi \right).
    \end{split}
\end{equation}
Now
\begin{equation}
\begin{split}
    \frac{2}{\sqrt{-g}} \sqrt{-g} \fdv{(G \Box \phi)}{g^{\mu \nu}} &= - 2 G,_{\phi} \grad_{\mu} \phi \grad_{\nu} \phi\\
    &- G,_X \left(  \grad_{\mu} X \grad_{\nu} \phi -\grad_{\nu} X \grad_{\mu} \phi \right)+ G,_{X}\Box \phi \grad_{\mu} \phi \grad_{\nu} \phi
    \end{split}
\end{equation} 
Collecting all the terms we will recover the tensor in \eqref{eq:emt}.
\end{document}