\usepackage{amssymb}
\usepackage{hyperref}
\usepackage{tikz}


%% For creating new environments that affect the body of the text inside
\usepackage{environ}


%% Allow nice importing of other LaTeX files (especially for pictures).
\usepackage{import}


%% Fix a few strange hyphenations
\hyphenation{ho-meo-morph-ism}
\hyphenation{ho-meo-morph-isms}


%% Define the theorem styles and numbering
\theoremstyle{plain}
\newtheorem{thm}{Theorem}%[section]
\newtheorem{theorem}[thm]{Theorem}
\newtheorem*{theorem*}{Theorem}
\newtheorem{cor}[thm]{Corollary}
\newtheorem{corollary}[thm]{Corollary}
\newtheorem{lemma}[thm]{Lemma}
\newtheorem*{lemma*}{Lemma}
\newtheorem{conj}[thm]{Conjecture}
\newtheorem{conjecture}[thm]{Conjecture}
\newtheorem*{conjecture*}{Conjecture}
\newtheorem{prop}[thm]{Proposition}
\newtheorem{proposition}[thm]{Proposition}
\newtheorem{claim}{Claim}%[thm]
\newtheorem*{fact*}{Fact}
\newtheorem*{question*}{Question}

\newtheorem{main}{Theorem}
\renewcommand{\themain}{\Alph{main}}
\newtheorem{maintheorem}[main]{Theorem}
\renewcommand{\themaintheorem}{\Alph{main}}
\newtheorem{maincorollary}[main]{Corollary}
\renewcommand{\themaincorollary}{\Alph{main}}
\newtheorem{mainproposition}[main]{Proposition}
\renewcommand{\themainproposition}{\Alph{main}}
\newtheorem{mainlemma}[main]{Lemma}
\renewcommand{\themainlemma}{\Alph{main}}

\theoremstyle{remark}
\newtheorem*{rem}{Remark}
\newtheorem*{remark}{Remark}
\newtheorem*{note}{Note}

\theoremstyle{definition}
\newtheorem{definition}[thm]{Definition}
\newtheorem*{convention}{Convention}
\newtheorem{alg}[thm]{Algorithm}
\newtheorem{example}[thm]{Example}
\newtheorem*{standing hypothesis}{Standing Hypothesis}


%% Redefine equations to have "symbolic" tags.
\renewcommand{\theequation}{\fnsymbol{equation}}

%\numberwithin{equation}{thm}


%% Define referencing commands
\newcommand{\thmref}[1]{Theorem~\ref{#1}}
\newcommand{\thmrefstar}[1]{Theorem~\ref*{#1}}
\newcommand{\corref}[1]{Corollary~\ref{#1}}
\newcommand{\correfstar}[1]{Corollary~\ref*{#1}}
\newcommand{\lemref}[1]{Lemma~\ref{#1}}
\newcommand{\propref}[1]{Proposition~\ref{#1}}
\newcommand{\proprefstar}[1]{Proposition~\ref*{#1}}
\newcommand{\defref}[1]{Definition~\ref{#1}}
\newcommand{\exampleref}[1]{Example~\ref{#1}}
\newcommand{\exerref}[1]{Exercise~\ref{#1}}
\newcommand{\itemref}[1]{\textnormal{(\ref{#1})}}
\newcommand{\itemrefstar}[1]{(\ref*{#1})}
\newcommand{\figref}[1]{Figure~\ref{#1}}
\newcommand{\secref}[1]{Section~\ref{#1}}
\newcommand{\claimref}[1]{Claim~\ref{#1}}


%% Set the labeling style
%\renewcommand{\labelenumi}{(\roman{enumi})}


%% Standardize in-text definition formatting.
\newcommand{\defn}[1]{\emph{#1}}


%% Allow placing picture files in another directory.
\graphicspath{{pictures/}}

%% A macro for placing figures of the same width
\newcommand{\pic}[3]
{
\begin{figure}[ht]
\center
\includegraphics[width=3.5in]{#1}
\caption{#2}
\label{#3}
\end{figure}
}


%% Create shortcut commands for various fonts and common symbols
\newcommand{\N}{\mathbb{N}}
\newcommand{\Z}{\mathbb{Z}}
\newcommand{\Q}{\mathbb{Q}}
\newcommand{\R}{\mathbb{R}}
\newcommand{\C}{\mathbb{C}}
\newcommand{\F}{\mathbb{F}}
\renewcommand{\H}{\mathbb{H}}
%\newcommand{\E}{\mathbb{E}}

\newcommand{\into}{\hookrightarrow}
\renewcommand{\setminus}{\smallsetminus}

%% Declare custom math operators
\DeclareMathOperator{\tr}{tr}
\DeclareMathOperator{\diag}{diag}
\DeclareMathOperator*{\argmin}{argmin}
\DeclareMathOperator*{\argmax}{argmax}
\DeclareMathOperator{\Span}{Span}
\DeclareMathOperator{\rank}{rank}

\DeclareMathOperator{\supp}{supp}
\DeclareMathOperator{\radius}{radius}
\DeclareMathOperator{\diam}{diam}
%\DeclareMathOperator{\s}{s}
\DeclareMathOperator{\id}{id}
\DeclareMathOperator{\Isom}{Isom}

\DeclareMathOperator{\susp}{\Sigma}

%% For sets and systems
\newcommand{\br}[1]{\left\langle #1 \right\rangle}
\newcommand{\paren}[1]{\left(#1\right)}
\newcommand{\sq}[1]{\left[#1\right]}
\newcommand{\set}[1]{\left\{#1\right\}}
\newcommand{\setp}[2]{\left\{#1 \mid #2\right\}}
\newcommand{\abs}[1]{\left| #1 \right|}
\newcommand{\norm}[1]{\left\| #1 \right\|}
\newcommand{\gpgen}[1]{\left < #1 \right >}


%%%%%%%%%%%%%%%%%%%%%%%%%%%%%%%%%%%%%%%%%%%%%%%%%%


\newcommand{\res}[1]{\vert_{#1}}
\newcommand{\mae}[1]{\mbox{#1-a.e.}}

\DeclareMathOperator{\Lk}{Lk}
\newcommand{\link}[2]{\Lk (#1)}

\newcommand{\cl}[1]{\overline{#1}}
\newcommand{\bd}{\partial}
\newcommand{\double}[1]{#1 \times #1}
\newcommand{\dbX}{\double{\bd X}}
\newcommand{\Leb}{\lambda}

\newcommand{\GE}{\mathcal{G}^E}
\newcommand{\Reg}{\mathcal R}
\newcommand{\RE}{\Reg^E}
\newcommand{\Specialset}{\mathcal S}
\newcommand{\Zerowidth}{\mathcal{Z}}
\newcommand{\ZE}{\Zerowidth^E}
\newcommand{\GER}{\GE \times \R}
\newcommand{\RER}{\RE \times \R}
\newcommand{\lmod}{\backslash}
\newcommand{\rmod}{/}
\newcommand{\modgp}[2]{#2 \lmod #1}%{#1 \rmod #2}
\newcommand{\modG}[1]{\modgp{#1}{\Gamma}}

\DeclareMathOperator{\Dualset}{\mathcal D}
\newcommand{\D}{\Dualset}
\DeclareMathOperator{\QE}{\mathcal Q_{\RE}}
\DeclareMathOperator{\QSX}{\mathcal Q_{SX}}

\DeclareMathOperator{\Busemanninclusion}{\iota}
\DeclareMathOperator{\Borelsection}{\iota}
\DeclareMathOperator{\pr}{pr}%{\pi}

\DeclareMathOperator{\cratio}{B}
\DeclareMathOperator{\emap}{E}

\DeclareMathOperator{\flip}{flip}
\DeclareMathOperator{\shadow}{\mathcal O}

\newcommand{\sa}{\mathfrak M}
\newcommand{\bsa}{\mathfrak B}
\newcommand{\Gf}{\mathfrak F}
\newcommand{\Gfone}{\Gf_1}
\newcommand{\varnu}{\widehat \nu}

\newcommand{\W}{\mathcal W}
\DeclareMathOperator{\A}{\mathcal A}
\DeclareMathOperator{\walls}{walls}
\DeclareMathOperator{\Centers}{Centers}
%\DeclareMathOperator{\radius}{radius}

%\DeclareMathOperator{\width}{width}

\newcommand{\nub}[2]{\underbrace{ \vphantom{\Big[} #1 \vphantom{\Big]} }_{\mbox{\normalsize $#2$}}}

\newcommand{\w}{\omega}
\newcommand{\VV}[1]{\ensuremath{\check{\mathrm{#1}}}}
\newcommand{\G}{\Gamma}
\newcommand{\BG}{\beta \Gamma}
\newcommand{\Hd}{\mathcal{H}d}
\newcommand{\Hangle}{\Hd}%{\mathcal{H}\angle}
\newcommand{\g}{\gamma}

\newcommand{\s}{\subseteq}
\newcommand{\nperp}{\perp\mkern-18mu/}

%% We want something for dummy veriables, but we want to make it easy to change in the future.
\newcommand{\x}{-}

\newcommand{\roundspheres}{\mathfrak K}
\newcommand{\closedsets}{\mathcal C (\bd X)}
\newcommand{\invariantsets}{\closedsets^\G}%{\mathfrak I}
\newcommand{\invariantsetsnaught}{\closedsets^{\G_0}}%{\mathfrak I}
\newcommand{\invariantsetsG}{\closedsets^G}%{\mathfrak I}

\newcommand{\CI}{\invariantsets}
\newcommand{\CInaught}{\invariantsetsnaught}
\newcommand{\CIG}{\invariantsetsG}
\DeclareMathOperator{\dT}{d_T}
\renewcommand{\epsilon}{\varepsilon}
