%\documentclass[preprint,12pt]{elsarticle}
%\if0
\usepackage{amssymb}
\usepackage{mathtools}
%\usepackage[dvipdfmx]{graphicx}
\usepackage{cite}
\usepackage{graphicx}
\usepackage{bm}
\usepackage{here}
\usepackage[subrefformat=parens]{subcaption}
\fi
%\usepackage{amssymb}
\usepackage{amsmath}
\usepackage[dvipdfmx]{}
\usepackage[dvipdfmx]{color}
%\usepackage{cite}
%\usepackage{upgreek}
\usepackage{url}
%\usepackage[dvipdfmx]{hyperref}
%\usepackage{pxjahyper}
%\usepackage {hyperref}
\usepackage{graphicx}
\usepackage{bm}
\usepackage{here}
\usepackage{caption}
\usepackage[subrefformat=parens]{subcaption}
\captionsetup{compatibility=false}

%% The amsthm package provides extended theorem environments
%% \usepackage{amsthm}

%% The lineno packages adds line numbers. Start line numbering with
%% \begin{linenumbers}, end it with \end{linenumbers}. Or switch it on
%% for the whole article with \linenumbers after \end{frontmatter}.
%% \usepackage{lineno}

%% natbib.sty is loaded by default. However, natbib options can be
%% provided with \biboptions{...} command. Following options are
%% valid:

%%   round  -  round parentheses are used (default)
%%   square -  square brackets are used   [option]
%%   curly  -  curly braces are used      {option}
%%   angle  -  angle brackets are used    <option>
%%   semicolon  -  multiple citations separated by semi-colon
%%   colon  - same as semicolon, an earlier confusion
%%   comma  -  separated by comma
%%   numbers-  selects numerical citations
%%   super  -  numerical citations as superscripts
%%   sort   -  sorts multiple citations according to order in ref. list
%%   sort&compress   -  like sort, but also compresses numerical citations
%%   compress - compresses without sorting
%%
%% \biboptions{comma,round}

% \biboptions{}

%% This list environment is used for the references in the
%% Program Summary
%%
\newcounter{bla}
\newenvironment{refnummer}{%
\list{[\arabic{bla}]}%
{\usecounter{bla}%
 \setlength{\itemindent}{0pt}%
 \setlength{\topsep}{0pt}%
 \setlength{\itemsep}{0pt}%
 \setlength{\labelsep}{2pt}%
 \setlength{\listparindent}{0pt}%
 \settowidth{\labelwidth}{[9]}%
 \setlength{\leftmargin}{\labelwidth}%
 \addtolength{\leftmargin}{\labelsep}%
 \setlength{\rightmargin}{0pt}}}
 {\endlist}
\begin{document}

\section{Files included}

The logical coordinates in the Lagrangian code are identified by the mesh number of $( i, j, k )$. One Lagrange mesh is shown in Fig. \ref{Lmesh}. The discretization method in Ref. \cite{Schulz} is employed in the Lagrangian fluid code. 
We use the spatial coordinate of $ {\bm R} = (x(i, j, k), y(i, j, k), z(i, j, k))$. The vector $ {\bar {\bm R}} $ is nomal to $ {\bm R}$. 
\begin{figure}[H]
	\centering
	\includegraphics[height=8cm]{images/Lagrange_mesh1.eps}
	\caption{Lagrangian Mesh.}\label{Lmesh}
\end{figure}

The definition points of the discretized physical quantities in the Lagrange and Euler codes are presented in Figs. \ref {definition_La} and \ref{definition_Eu}, respectively. The subscripts $i$, $j$ and $k$ correspond to the positions in space, and the subscript $n$ corresponds to time $n\times dt$. The displacements in the $i$, $j$ and $l$ directions are defined as follows: 
	\begin{eqnarray*}
		&&\begin{cases}
			dx^n_{i+\frac{1}{2},j,k} \equiv dxi_{i+\frac{1}{2},j,k}=x^n_{i+1,j,k}-x^n_{i,j,k}\\
			dy^n_{i+\frac{1}{2},j,k} \equiv dyi_{i+\frac{1}{2},j,k}=y^n_{i+1,j,k}-y^n_{i,j,k}\\
			dz^n_{i+\frac{1}{2},j,k} \equiv dzi_{i+\frac{1}{2},j,k}=z^n_{i+1,j,k}-z^n_{i,j,k}
		\end{cases}\\
		&&\begin{cases}
			dx^n_{i,j+\frac{1}{2},k} \equiv dxj_{i,j+\frac{1}{2},k}=x^n_{i,j+1,k}-x^n_{i,j,k}\\
			dy^n_{i,j+\frac{1}{2},k} \equiv dyj_{i,j+\frac{1}{2},k}=y^n_{i,j+1,k}-y^n_{i,j,k}\\
			dz^n_{i,j+\frac{1}{2},k} \equiv dzj_{i,j+\frac{1}{2},k}=z^n_{i,j+1,k}-z^n_{i,j,k}
		\end{cases}\\
		&&\begin{cases}
			dx^n_{i,j,k+\frac{1}{2}} \equiv dxk_{i,j,k+\frac{1}{2}}=x^n_{i,j,k+1}-x^n_{i,j,k}\\
			dy^n_{i,j,k+\frac{1}{2}} \equiv dyk_{i,j,k+\frac{1}{2}}=y^n_{i,j,k+1}-y^n_{i,j,k}\\
			dz^n_{i,j,k+\frac{1}{2}} \equiv dzk_{i,j,k+\frac{1}{2}}=z^n_{i,j,k+1}-z^n_{i,j,k}
		\end{cases}\\
	\end{eqnarray*}	

\begin{figure}[H]
	\centering
	\includegraphics[height=8cm]{images/Lag_mesh_de.eps}
	\caption{Definition points of discretized physical quantities in the Lagrangian code.}\label{definition_La}
	\end{figure}
	
\begin{figure}[H]
	\centering
	\includegraphics[height=8cm]{images/Eul_mesh_de.eps}
	\caption{Definition points of discretized physical quantities in the Eulerian code.}\label{definition_Eu}
	\end{figure}

In the Lagrange code, the derivatives in $ x $, $ y $ and $ z $ are as follows:
\begin{equation}
\begin{split}
\frac{\partial }{\partial x}
&=\frac{\partial}{\partial i}\frac{\partial i}{\partial x}+\frac{\partial}{\partial j}\frac{\partial j}{\partial x}+\frac{\partial}{\partial k}\frac{\partial k}{\partial x}\\
&=\frac{1}{J}\left[(\frac{\partial y}{\partial j}\frac{\partial z}{\partial k}-\frac{\partial y}{\partial k}\frac{\partial z}{\partial j})\frac{\partial}{\partial i}+(\frac{\partial y}{\partial k}\frac{\partial z}{\partial i}-\frac{\partial y}{\partial i}\frac{\partial z}{\partial k})\frac{\partial}{\partial j}+(\frac{\partial y}{\partial i}\frac{\partial z}{\partial j}-\frac{\partial y}{\partial j}\frac{\partial z}{\partial i})\frac{\partial}{\partial k}\right]\\
& \equiv \frac{1}{J}\left[Dix\frac{\partial}{\partial i}+Djx\frac{\partial}{\partial j}+Dkx\frac{\partial}{\partial k}\right]
\end{split}
\end{equation}
\begin{equation}
\begin{split}
\frac{\partial }{\partial y}&
=\frac{\partial}{\partial i}\frac{\partial i}{\partial y}+\frac{\partial}{\partial j}\frac{\partial j}{\partial y}+\frac{\partial}{\partial k}\frac{\partial k}{\partial y}\\
&=\frac{1}{J}\left[(\frac{\partial z}{\partial j}\frac{\partial x}{\partial k}-\frac{\partial z}{\partial k}\frac{\partial x}{\partial j})\frac{\partial}{\partial i}+(\frac{\partial z}{\partial k}\frac{\partial x}{\partial i}-\frac{\partial z}{\partial i}\frac{\partial x}{\partial k})\frac{\partial}{\partial j}+(\frac{\partial z}{\partial i}\frac{\partial x}{\partial j}-\frac{\partial z}{\partial j}\frac{\partial x}{\partial i})\frac{\partial}{\partial k}\right]\\
& \equiv \frac{1}{J}\left[Diy\frac{\partial}{\partial i}+Djy\frac{\partial}{\partial j}+Dky\frac{\partial}{\partial k}\right]
\end{split}
\end{equation}
\begin{equation}
\begin{split}
\frac{\partial }{\partial z}&
=\frac{\partial}{\partial i}\frac{\partial i}{\partial z}+\frac{\partial}{\partial j}\frac{\partial j}{\partial z}+\frac{\partial}{\partial k}\frac{\partial k}{\partial z}\\
&=\frac{1}{J}\left[(\frac{\partial x}{\partial j}\frac{\partial y}{\partial k}-\frac{\partial x}{\partial k}\frac{\partial y}{\partial j})\frac{\partial}{\partial i}+(\frac{\partial x}{\partial k}\frac{\partial y}{\partial i}-\frac{\partial x}{\partial i}\frac{\partial y}{\partial k})\frac{\partial}{\partial j}+(\frac{\partial x}{\partial i}\frac{\partial y}{\partial j}-\frac{\partial x}{\partial j}\frac{\partial y}{\partial i})\frac{\partial}{\partial k}\right]\\
& \equiv \frac{1}{J}\left[Diz\frac{\partial}{\partial i}+Djz\frac{\partial}{\partial j}+Dkz\frac{\partial}{\partial k}\right]
\end{split}
\end{equation}
The generalized formula is as follows: 
\begin{equation}\label{ijktrans}
\frac{\partial}{\partial \bm r}=\frac{1}{J}\left[Di\bm r\frac{\partial}{\partial i}+Dj\bm r\frac{\partial}{\partial j}+Dk\bm r\frac{\partial}{\partial k}\right]
\end{equation}
\begin{equation}
	Di\bm r \equiv \left(
	\begin{array}{c}
	Dix\\
	Diy\\
	Diz
	\end{array}
	\right), 
	Dj\bm r \equiv \left(
	\begin{array}{c}
	Djx\\
	Djy\\
	Djz
	\end{array}
	\right),  
	Dk\bm r \equiv \left(
	\begin{array}{c}
	Dkx\\
	Dky\\
	Dkz
	\end{array}
	\right)
\end{equation}

%\end{document}