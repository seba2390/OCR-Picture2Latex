%%%%%	\pdfoutput=1
\documentclass[preprint,12pt]{elsarticle}
\if0
\usepackage{amssymb}
\usepackage{mathtools}
%\usepackage[dvipdfmx]{graphicx}
\usepackage{cite}
\usepackage{graphicx}
\usepackage{bm}
\usepackage{here}
\usepackage[subrefformat=parens]{subcaption}
\fi
%\usepackage{amssymb}
\usepackage{amsmath}
\usepackage[dvipdfmx]{}
\usepackage[dvipdfmx]{color}
%\usepackage{cite}
%\usepackage{upgreek}
\usepackage{url}
%\usepackage[dvipdfmx]{hyperref}
%\usepackage{pxjahyper}
%\usepackage {hyperref}
\usepackage{graphicx}
\usepackage{bm}
\usepackage{here}
\usepackage{caption}
\usepackage[subrefformat=parens]{subcaption}
\captionsetup{compatibility=false}

%% The amsthm package provides extended theorem environments
%% \usepackage{amsthm}

%% The lineno packages adds line numbers. Start line numbering with
%% \begin{linenumbers}, end it with \end{linenumbers}. Or switch it on
%% for the whole article with \linenumbers after \end{frontmatter}.
%% \usepackage{lineno}

%% natbib.sty is loaded by default. However, natbib options can be
%% provided with \biboptions{...} command. Following options are
%% valid:

%%   round  -  round parentheses are used (default)
%%   square -  square brackets are used   [option]
%%   curly  -  curly braces are used      {option}
%%   angle  -  angle brackets are used    <option>
%%   semicolon  -  multiple citations separated by semi-colon
%%   colon  - same as semicolon, an earlier confusion
%%   comma  -  separated by comma
%%   numbers-  selects numerical citations
%%   super  -  numerical citations as superscripts
%%   sort   -  sorts multiple citations according to order in ref. list
%%   sort&compress   -  like sort, but also compresses numerical citations
%%   compress - compresses without sorting
%%
%% \biboptions{comma,round}

% \biboptions{}

%% This list environment is used for the references in the
%% Program Summary
%%
\newcounter{bla}
\newenvironment{refnummer}{%
\list{[\arabic{bla}]}%
{\usecounter{bla}%
 \setlength{\itemindent}{0pt}%
 \setlength{\topsep}{0pt}%
 \setlength{\itemsep}{0pt}%
 \setlength{\labelsep}{2pt}%
 \setlength{\listparindent}{0pt}%
 \settowidth{\labelwidth}{[9]}%
 \setlength{\leftmargin}{\labelwidth}%
 \addtolength{\leftmargin}{\labelsep}%
 \setlength{\rightmargin}{0pt}}}
 {\endlist}
\begin{document}
\renewcommand{\include}[1]{}
\renewcommand\documentclass[2][]{}
\setcounter{tocdepth}{2}
%\tableofcontents
\thispagestyle{empty}
%\setcounter{page}{0}
%\input{CPCNVAtemplate.tex}
%%%%%\journal{Computer Physics Communications}


\begin{frontmatter}

%% Title, authors and addresses

%% use the tnoteref command within \title for footnotes;
%% use the tnotetext command for the associated footnote;
%% use the fnref command within \author or \address for footnotes;
%% use the fntext command for the associated footnote;
%% use the corref command within \author for corresponding author footnotes;
%% use the cortext command for the associated footnote;
%% use the ead command for the email address,
%% and the form \ead[url] for the home page:
%%
%% \title{Title\tnoteref{label1}}
%% \tnotetext[label1]{}
%% \author{Name\corref{cor1}\fnref{label2}}
%% \ead{email address}
%% \ead[url]{home page}
%% \fntext[label2]{}
%% \cortext[cor1]{}
%% \address{Address\fnref{label3}}
%% \fntext[label3]{}

\title{Code O-SUKI-N 3D: Upgraded Direct-Drive Fuel Target 3D Implosion Code in Heavy Ion Inertial Fusion}

%% use optional labels to link authors explicitly to addresses:
%% \author[label1,label2]{<author name>}
%% \address[label1]{<address>}
%% \address[label2]{<address>}


\author[a]{H. Nakamura}
\author[a]{K. Uchibori}
\author[a]{S. Kawata\corref{author}}
\author[b]{T. Karino}
\author[a]{R. Sato}
\author[c]{A. I. Ogoyski}


\cortext[author] {Corresponding author.\\\textit{E-mail address:} kwt@cc.utsunomiya-u.ac.jp, s.kwta.g@gmail.com}
\address[a]{Graduate School of Engineering, Utsunomiya University, Utsunomiya 321-8585, Japan}
\address[b]{Collaborative Laboratories for Advanced Decommissioning Science, Japan Atomic Energy Agency, Fukushima 970-8026, Japan}
\address[c]{Department of Physics, Varna Technical University, Varna 9010, Bulgaria}

\begin{abstract}
%% Text of abstract

The Code O-SUKI-N 3D is an upgraded version of the 2D Code O-SUKI (Comput. Phys. Commun. 240, 83 (2019)). Code O-SUKI-N 3D is an integrated 3-dimensional (3D) simulation program system for fuel implosion, ignition and burning of a direct-drive nuclear-fusion pellet in heavy ion beam (HIB) inertial confinement fusion (HIF).The Code O-SUKI-N 3D consists of the three programs of Lagrangian fluid implosion program, data conversion program, and Euler fluid implosion, ignition and burning program. The Code O-SUKI-N 3D can also couple with the HIB illumination and energy deposition program of OK3 (Comput. Phys. Commun. 181, 1332 (2010)). The spherical target implosion 3D behavior is computed by the 3D Lagrangian fluid code until the time just before the void closure of the fuel implosion. After that, all the data by the Lagrangian implosion code are converted to the data for the 3D Eulerian code. In the 3D Euler code, the DT fuel compression at the stagnation, ignition and burning are computed. The Code O-SUKI-N 3D simulation system provides a capability to compute and to study the HIF target implosion dynamics. 

\end{abstract}

\begin{keyword}
%% keywords here, in the form: keyword \sep keyword
Implosion; Heavy ion beam; Inertial confinement fusion; Direct-drive fuel pellet implosion; Ignition; Burning.

\end{keyword}

\end{frontmatter}

%%
%% Start line numbering here if you want
%%
% \linenumbers

% Computer program descriptions should contain the following
% PROGRAM SUMMARY.

\noindent
{\bf Program summary}
  %Delete as appropriate.

\begin{small}
\noindent
{\em Program Title:}  O-SUKI-N 3D                                        \\
{\em Licensing provisions: CC BY NC 3.0 }                                   \\
{\em Programming language:} C++                                 \\
%{\em Supplementary material:}                                 \\
  % Fill in if necessary, otherwise leave out.
{\em Computer:} Workstation (Xeon, 2 GHz or higher recommended)\\
{\em RAM:} 120GBytes minimum\\
{\em Operating system:} UNIX, Linux (For example: CentOS 6.4, Ubuntu 18.04.1 LTS)\\
{\em Journal reference of previous version: Code O-SUKI: 2D version}                  \\
 %{\em Does the new version supersede the previous version?: No}   \\
% {\em Reasons for the new version:}\\
% {\em Summary of revisions:}*\\
{\em Nature of problem:}     
Nuclear fusion energy would be energy source for society. In this paper we focus on heavy ion beam (HIB) inertial confinement fusion (HIF). A spherical mm-radius deuterium (D) - tritium (T) fuel pellet is irradiated by HIBs to be compressed to about a thousand times of the solid density. The DT fuel temperature reaches $\sim$5-10KeV for the ignition to release the fusion energy. The typical HIBs total input energy is several MJ, and the HIBs pulse length is about a few tens of ns. The O-SUKI-N 3D code system provides an integrated tool to simulate the HIF DT fuel pellet implosion, ignition and burning in 3 dimensions (3D). The O-SUKI-N 3D code system is an upgraded version of the Code O-SUKI (Comput. Phys. Commun. 240, 83 (2019)) which is a 2D implosion simulation system in HIF. The DT fuel is compressed to the high density, and so the DT fuel spatial deformation may be serious at the DT fuel stagnation. Therefore, the O-SUKI and O-SUKI-N 3D systems employ a Lagrangian fluid code first to simulate the DT fuel implosion phase until just before the stagnation. Then all the simulation data from the Lagrangian code are converted to them for the Euler fluid code, in which the DT fuel ignition and burning are simulated. \\
  %Describe the nature of the problem here. \\
{\em Solution method:}     
In the two fluid codes (Lagrangian and Euler fluid codes) in the O-SUKI-N 3D system the three-temperature fluid model (J. Appl. Phys. 60, 898 (1986)) is employed to simulate the pellet dynamics in HIF. 
\\
  %Describe the method solution here.
{\em Additional comments including Restrictions and Unusual features:} The Lagrange code is weak against the spatial mesh deformation from nature of its numerical algorithm. When short-wavelength perturbations are imposed near the poles of the spherical target, the spatial meshes might crash and the computation run may stop. \\
  %Provide any additional comments here.
   \\

%\begin{thebibliography}{0}
%\bibitem{1}OK1          % This list should only contain those items referenced in the                 
%\bibitem{2}OK2         % Program Summary section.   
%\bibitem{3}OK3         % Type references in text as [1], [2], etc.
	                            % This list is different from the bibliography at the end of 
                               % the Long Write-Up.
%\end{thebibliography}
%* Items marked with an asterisk are only required for new versions
%of programs previously published in the CPC Program Library.\\
\end{small}




\section{Introduction}\label{int}

Parallelism and concurrency \cite{CM} are the core concepts within computer science. There are mainly two camps in capturing concurrency: the interleaving concurrency and the true concurrency.

The representative of interleaving concurrency is bisimulation/weak bisimulation equivalences. CCS (A Calculus of Communicating Systems) \cite{CCS} \cite{CC} is a calculus based on bisimulation semantics model. CCS has good semantic properties based on the interleaving bisimulation. These properties include monoid laws, static laws, new expansion law for strongly interleaving bisimulation, $\tau$ laws for weakly interleaving bisimulation, and full congruences for strongly and weakly interleaving bisimulations, and also unique solution for recursion.

The other camp of concurrency is true concurrency. The researches on true concurrency are still active. Firstly, there are several truly concurrent bisimulations, the representatives are: pomset bisimulation, step bisimulation, history-preserving (hp-) bisimulation, and especially hereditary history-preserving (hhp-) bisimulation \cite{HHP1} \cite{HHP2}. These truly concurrent bisimulations are studied in different structures \cite{ES1} \cite{ES2} \cite{CM}: Petri nets, event structures, domains, and also a uniform form called TSI (Transition System with Independence) \cite{SFL}. There are also several logics based on different truly concurrent bisimulation equivalences, for example, SFL (Separation Fixpoint Logic) and TFL (Trace Fixpoint Logic) \cite{SFL} are extensions on true concurrency of mu-calculi \cite{MUC} on bisimulation equivalence, and also a logic with reverse modalities \cite {RL1} \cite{RL2} based on the so-called reverse bisimulations with a reverse flavor. Recently, a uniform logic for true concurrency \cite{LTC1} \cite{LTC2} was represented, which used a logical framework to unify several truly concurrent bisimulations, including pomset bisimulation, step bisimulation, hp-bisimulation and hhp-bisimulation.

There are simple comparisons between HM logic and bisimulation, as the uniform logic \cite{LTC1} \cite{LTC2} and truly concurrent bisimulations; the algebraic laws \cite{ALNC}, ACP \cite{ACP} and bisimulation, as the algebraic laws APTC \cite{ATC} and truly concurrent bisimulations; CCS and bisimulation, as truly concurrent bisimulations  and \emph{what}, which is still missing.

In this paper, we design a calculus for true concurrency (CTC) following the way paved by CCS for bisimulation equivalence. This paper is organized as follows. In section \ref{bac}, we introduce some preliminaries, including a brief introduction to CCS, and also preliminaries on true concurrency. We introduce the syntax and operational semantics of CTC in section \ref{sos}, its properties for strongly truly concurrent bisimulations in section \ref{stcb}, its properties for weakly truly concurrent bisimulations in section \ref{wtcb}. In section \ref{app}, we show the applications of CTC by an example called alternating-bit protocol. Finally, in section \ref{con}, we conclude this paper. 
%\documentclass[preprint,12pt]{elsarticle}
%\if0
\usepackage{amssymb}
\usepackage{mathtools}
%\usepackage[dvipdfmx]{graphicx}
\usepackage{cite}
\usepackage{graphicx}
\usepackage{bm}
\usepackage{here}
\usepackage[subrefformat=parens]{subcaption}
\fi
%\usepackage{amssymb}
\usepackage{amsmath}
\usepackage[dvipdfmx]{}
\usepackage[dvipdfmx]{color}
%\usepackage{cite}
%\usepackage{upgreek}
\usepackage{url}
%\usepackage[dvipdfmx]{hyperref}
%\usepackage{pxjahyper}
%\usepackage {hyperref}
\usepackage{graphicx}
\usepackage{bm}
\usepackage{here}
\usepackage{caption}
\usepackage[subrefformat=parens]{subcaption}
\captionsetup{compatibility=false}

%% The amsthm package provides extended theorem environments
%% \usepackage{amsthm}

%% The lineno packages adds line numbers. Start line numbering with
%% \begin{linenumbers}, end it with \end{linenumbers}. Or switch it on
%% for the whole article with \linenumbers after \end{frontmatter}.
%% \usepackage{lineno}

%% natbib.sty is loaded by default. However, natbib options can be
%% provided with \biboptions{...} command. Following options are
%% valid:

%%   round  -  round parentheses are used (default)
%%   square -  square brackets are used   [option]
%%   curly  -  curly braces are used      {option}
%%   angle  -  angle brackets are used    <option>
%%   semicolon  -  multiple citations separated by semi-colon
%%   colon  - same as semicolon, an earlier confusion
%%   comma  -  separated by comma
%%   numbers-  selects numerical citations
%%   super  -  numerical citations as superscripts
%%   sort   -  sorts multiple citations according to order in ref. list
%%   sort&compress   -  like sort, but also compresses numerical citations
%%   compress - compresses without sorting
%%
%% \biboptions{comma,round}

% \biboptions{}

%% This list environment is used for the references in the
%% Program Summary
%%
\newcounter{bla}
\newenvironment{refnummer}{%
\list{[\arabic{bla}]}%
{\usecounter{bla}%
 \setlength{\itemindent}{0pt}%
 \setlength{\topsep}{0pt}%
 \setlength{\itemsep}{0pt}%
 \setlength{\labelsep}{2pt}%
 \setlength{\listparindent}{0pt}%
 \settowidth{\labelwidth}{[9]}%
 \setlength{\leftmargin}{\labelwidth}%
 \addtolength{\leftmargin}{\labelsep}%
 \setlength{\rightmargin}{0pt}}}
 {\endlist}
\begin{document}

\section{O-SUKI-N 3D code algorithm description}
\par

\subsection{O-SUKI-N 3D code structure}
     The O-SUKI-N 3D code system consists of three parts: The Lagrangian fluid code \cite{Schulz}, the data conversion code from the Lagrangian code to the Euler code, and Euler code. The fluid model is the three-temperature model in Ref. \cite{Tahir}. The Lagrangian fluid code, the data conversion code and the Euler code are described below in detail. 
     
     In the Lagrangian fluid code the spatial meshes move together with the fluid motion \cite{Schulz}. However, the Lagrange meshes can not follow the fluid large deformation. On the other hand, the Euler meshes are fixed to the space, and the fluid moves through the meshes. Therefore, just before the void closure time, that is, the stagnation phase, the Lagrangian code is used to simulate the DT fuel implosion. After the void closure time, the Euler code is employed to simulate the DT fuel further compression, ignition and burning. Between the Lagrangian code and the Euler code the data should be converted by the data conversion code. 

	All the simulation process is performed in its integrated way by using the script of "CodeO-SUKI-N-fusion-start.sh". The processes executed by this shell script are as follows: \\
1. Make the stack size infinite.\\
2. Remove all output data file and make the new output files.\\
3. Change the permission of shell scripts to executable. \\
4. Compile the main function of the Lagrangian code and execute it.\\
5. If any problems do not appear during the calculation of the Lagrangian code, compile the main function of the data conversion code and execute it.\\
6. If there is no problem during the data conversion, compile the main function of the Euler code and execute it.\\
     

\subsection{Steps in Lagrangian code}\par
     The Lagrangian code has the following steps: 

\begin{enumerate}
\item Initialize the variables and calculation of total input energy. \par
\item Calculation of time step size.\par
\item Calculation of coordinates.\par
\item Solve equation of motion. \par
\item Solve density by equation of continuity.\par
\item Calculation of artificial viscosity.\par
\item Transfer the data to the OK3. \par
\item Calculation of energy deposition distribution in code OK3. For details of the OK3, see the refs.\cite{ogoyski1,ogoyski2,ogoyski3}. \par
\item Solve energy equations\par
\item Calculation of heat conduction\par
\item Calculation of temperature relaxation among three temperatures.\par
\item Solve equation of state\par
\item Save the results.\par
\item End the Lagrangian calculation right before the void closure.\par
\item Transfer the data to converting code. \par
\end {enumerate}


\subsection{Data Conversion code from Lagrangian fluid code to Euler fluid code}

\begin {enumerate}
\item Read variables saved in Lagrangian code.\par
\item Generate the Eulerian mesh.\par
\item Calculate the interpolation of the physical quantity to them on the Eulerian mesh.\par
\item Write the converted data to the Eulerian code.\par
\end {enumerate}


\subsection{Steps in Eulerian code}

\begin {enumerate}
\item Read the mesh number from the converted data and define the each matrices.\par
\item Initialize the variables.\par
\item Calculation of time step size.\par
\item Solve equation of motion. \par
\item Track the material boundaries of DT, Al and Pb.\par
\item Linearly interpolate the boundary lines and transcribe them on the Eulerian code. \par
\item Discriminate the materials by using the transferred boundary line. \par
\item Solve density by equation of continuity.\par
\item Calculate artificial viscosity.\par
\item Solve energy equations\par
\item Calculation of fusion reaction.\par
\item Calculation of heat conduction\par
\item Calculation of temperature relaxation among three temperatures.\par
\item Solve equation of state.\par
\item Save the results.\par
\item End.
\end{enumerate}

%\include{end}
\section{Syntax and Operational Semantics}\label{sos}

We assume an infinite set $\mathcal{N}$ of (action or event) names, and use $a,b,c,\cdots$ to range over $\mathcal{N}$. We denote by $\overline{\mathcal{N}}$ the set of co-names and let $\overline{a},\overline{b},\overline{c},\cdots$ range over $\overline{\mathcal{N}}$. Then we set $\mathcal{L}=\mathcal{N}\cup\overline{\mathcal{N}}$ as the set of labels, and use $l,\overline{l}$ to range over $\mathcal{L}$. We extend complementation to $\mathcal{L}$ such that $\overline{\overline{a}}=a$. Let $\tau$ denote the silent step (internal action or event) and define $Act=\mathcal{L}\cup\{\tau\}$ to be the set of actions, $\alpha,\beta$ range over $Act$. And $K,L$ are used to stand for subsets of $\mathcal{L}$ and $\overline{L}$ is used for the set of complements of labels in $L$. A relabelling function $f$ is a function from $\mathcal{L}$ to $\mathcal{L}$ such that $f(\overline{l})=\overline{f(l)}$. By defining $f(\tau)=\tau$, we extend $f$ to $Act$.

Further, we introduce a set $\mathcal{X}$ of process variables, and a set $\mathcal{K}$ of process constants, and let $X,Y,\cdots$ range over $\mathcal{X}$, and $A,B,\cdots$ range over $\mathcal{K}$, $\widetilde{X}$ is a tuple of distinct process variables, and also $E,F,\cdots$ range over the recursive expressions. We write $\mathcal{P}$ for the set of processes. Sometimes, we use $I,J$ to stand for an indexing set, and we write $E_i:i\in I$ for a family of expressions indexed by $I$. $Id_D$ is the identity function or relation over set $D$.

For each process constant schema $A$, a defining equation of the form

$$A\overset{\text{def}}{=}P$$

is assumed, where $P$ is a process.

\subsection{Syntax}

We use the Prefix $.$ to model the causality relation $\leq$ in true concurrency, the Summation $+$ to model the conflict relation $\sharp$ in true concurrency, and the Composition $\parallel$ to explicitly model concurrent relation in true concurrency. And we follow the conventions of process algebra.

\begin{definition}[Syntax]\label{syntax}
Truly concurrent processes are defined inductively by the following formation rules:

\begin{enumerate}
  \item $A\in\mathcal{P}$;
  \item $\textbf{nil}\in\mathcal{P}$;
  \item if $P\in\mathcal{P}$, then the Prefix $\alpha.P\in\mathcal{P}$, for $\alpha\in Act$;
  \item if $P,Q\in\mathcal{P}$, then the Summation $P+Q\in\mathcal{P}$;
  \item if $P,Q\in\mathcal{P}$, then the Composition $P\parallel Q\in\mathcal{P}$;
  \item if $P\in\mathcal{P}$, then the Prefix $(\alpha_1\parallel\cdots\parallel\alpha_n).P\in\mathcal{P}\quad(n\in I)$, for $\alpha_,\cdots,\alpha_n\in Act$;
  \item if $P\in\mathcal{P}$, then the Restriction $P\setminus L\in\mathcal{P}$ with $L\in\mathcal{L}$;
  \item if $P\in\mathcal{P}$, then the Relabelling $P[f]\in\mathcal{P}$.
\end{enumerate}

The standard BNF grammar of syntax of CTC can be summarized as follows:

$$P::=A\quad|\quad\textbf{nil}\quad|\quad\alpha.P\quad|\quad P+P\quad |\quad P\parallel P\quad |\quad (\alpha_1\parallel\cdots\parallel\alpha_n).P \quad|\quad P\setminus L\quad |\quad P[f].$$
\end{definition}

\subsection{Operational Semantics}

The operational semantics is defined by LTSs (labelled transition systems), and it is detailed by the following definition.

\begin{definition}[Semantics]\label{semantics}
The operational semantics of CTC corresponding to the syntax in Definition \ref{syntax} is defined by a series of transition rules, named $\textbf{Act}$, $\textbf{Sum}$, $\textbf{Com}$, $\textbf{Res}$, $\textbf{Rel}$ and $\textbf{Con}$ indicate that the rules are associated respectively with Prefix, Summation, Composition, Restriction, Relabelling and Constants in Definition \ref{syntax}. They are shown in Table \ref{TRForCTC}.

\begin{center}
    \begin{table}
        \[\textbf{Act}_1\quad \frac{}{\alpha.P\xrightarrow{\alpha}P}\]

        \[\textbf{Sum}_1\quad \frac{P\xrightarrow{\alpha}P'}{P+Q\xrightarrow{\alpha}P'}\]

        \[\textbf{Com}_1\quad \frac{P\xrightarrow{\alpha}P'\quad Q\nrightarrow}{P\parallel Q\xrightarrow{\alpha}P'\parallel Q}\]

        \[\textbf{Com}_2\quad \frac{Q\xrightarrow{\alpha}Q'\quad P\nrightarrow}{P\parallel Q\xrightarrow{\alpha}P\parallel Q'}\]

        \[\textbf{Com}_3\quad \frac{P\xrightarrow{\alpha}P'\quad Q\xrightarrow{\beta}Q'}{P\parallel Q\xrightarrow{\{\alpha,\beta\}}P'\parallel Q'}\quad (\beta\neq\overline{\alpha})\]

        \[\textbf{Com}_4\quad \frac{P\xrightarrow{l}P'\quad Q\xrightarrow{\overline{l}}Q'}{P\parallel Q\xrightarrow{\tau}P'\parallel Q'}\]

        \[\textbf{Act}_2\quad \frac{}{(\alpha_1\parallel\cdots\parallel\alpha_n).P\xrightarrow{\{\alpha_1,\cdots,\alpha_n\}}P}\quad (\alpha_i\neq\overline{\alpha_j}\quad i,j\in\{1,\cdots,n\})\]

        \[\textbf{Sum}_2\quad \frac{P\xrightarrow{\{\alpha_1,\cdots,\alpha_n\}}P'}{P+Q\xrightarrow{\{\alpha_1,\cdots,\alpha_n\}}P'}\]

        \[\textbf{Res}_1\quad \frac{P\xrightarrow{\alpha}P'}{P\setminus L\xrightarrow{\alpha}P'\setminus L}\quad (\alpha,\overline{\alpha}\notin L)\]

        \[\textbf{Res}_2\quad \frac{P\xrightarrow{\{\alpha_1,\cdots,\alpha_n\}}P'}{P\setminus L\xrightarrow{\{\alpha_1,\cdots,\alpha_n\}}P'\setminus L}\quad (\alpha_1,\overline{\alpha_1},\cdots,\alpha_n,\overline{\alpha_n}\notin L)\]

        \[\textbf{Rel}_1\quad \frac{P\xrightarrow{\alpha}P'}{P[f]\xrightarrow{f(\alpha)}P'[f]}\]

        \[\textbf{Rel}_2\quad \frac{P\xrightarrow{\{\alpha_1,\cdots,\alpha_n\}}P'}{P[f]\xrightarrow{\{f(\alpha_1),\cdots,f(\alpha_n)\}}P'[f]}\]

        \[\textbf{Con}_1\quad\frac{P\xrightarrow{\alpha}P'}{A\xrightarrow{\alpha}P'}\quad (A\overset{\text{def}}{=}P)\]

        \[\textbf{Con}_2\quad\frac{P\xrightarrow{\{\alpha_1,\cdots,\alpha_n\}}P'}{A\xrightarrow{\{\alpha_1,\cdots,\alpha_n\}}P'}\quad (A\overset{\text{def}}{=}P)\]

        \caption{Transition rules of CTC}
        \label{TRForCTC}
    \end{table}
\end{center}
\end{definition}

\subsection{Properties of Transitions}

\begin{definition}[Sorts]\label{sorts}
Given the sorts $\mathcal{L}(A)$ and $\mathcal{L}(X)$ of constants and variables, we define $\mathcal{L}(P)$ inductively as follows.

\begin{enumerate}
  \item $\mathcal{L}(l.P)=\{l\}\cup\mathcal{L}(P)$;
  \item $\mathcal{L}((l_1\parallel \cdots\parallel l_n).P)=\{l_1,\cdots,l_n\}\cup\mathcal{L}(P)$;
  \item $\mathcal{L}(\tau.P)=\mathcal{L}(P)$;
  \item $\mathcal{L}(P+Q)=\mathcal{L}(P)\cup\mathcal{L}(Q)$;
  \item $\mathcal{L}(P\parallel Q)=\mathcal{L}(P)\cup\mathcal{L}(Q)$;
  \item $\mathcal{L}(P\setminus L)=\mathcal{L}(P)-(L\cup\overline{L})$;
  \item $\mathcal{L}(P[f])=\{f(l):l\in\mathcal{L}(P)\}$;
  \item for $A\overset{\text{def}}{=}P$, $\mathcal{L}(P)\subseteq\mathcal{L}(A)$.
\end{enumerate}
\end{definition}

Now, we present some properties of the transition rules defined in Table \ref{TRForCTC}.

\begin{proposition}
If $P\xrightarrow{\alpha}P'$, then
\begin{enumerate}
  \item $\alpha\in\mathcal{L}(P)\cup\{\tau\}$;
  \item $\mathcal{L}(P')\subseteq\mathcal{L}(P)$.
\end{enumerate}

If $P\xrightarrow{\{\alpha_1,\cdots,\alpha_n\}}P'$, then
\begin{enumerate}
  \item $\alpha_1,\cdots,\alpha_n\in\mathcal{L}(P)\cup\{\tau\}$;
  \item $\mathcal{L}(P')\subseteq\mathcal{L}(P)$.
\end{enumerate}
\end{proposition}

\begin{proof}
By induction on the inference of $P\xrightarrow{\alpha}P'$ and $P\xrightarrow{\{\alpha_1,\cdots,\alpha_n\}}P'$, there are fourteen cases corresponding to the transition rules named $\textbf{Act}_{1,2}$, $\textbf{Sum}_{1,2}$, $\textbf{Com}_{1,2,3,4}$, $\textbf{Res}_{1,2}$, $\textbf{Rel}_{1,2}$ and $\textbf{Con}_{1,2}$ in Table \ref{TRForCTC}, we just prove the one case $\textbf{Act}_1$ and $\textbf{Act}_2$, and omit the others.

Case $\textbf{Act}_1$: by $\textbf{Act}_1$, with $P\equiv\alpha.P'$. Then by Definition \ref{sorts}, we have (1) $\mathcal{L}(P)=\{\alpha\}\cup\mathcal{L}(P')$ if $\alpha\neq\tau$; (2) $\mathcal{L}(P)=\mathcal{L}(P')$ if $\alpha=\tau$. So, $\alpha\in\mathcal{L}(P)\cup\{\tau\}$, and $\mathcal{L}(P')\subseteq\mathcal{L}(P)$, as desired.

Case $\textbf{Act}_2$: by $\textbf{Act}_2$, with $P\equiv(\alpha_1\parallel\cdots\parallel\alpha_n).P'$. Then by Definition \ref{sorts}, we have (1) $\mathcal{L}(P)=\{\alpha_1,\cdots,\alpha_n\}\cup\mathcal{L}(P')$ if $\alpha_i\neq\tau$ for $i\leq n$; (2) $\mathcal{L}(P)=\mathcal{L}(P')$ if $\alpha_1,\cdots,\alpha_n=\tau$. So, $\alpha_1,\cdots,\alpha_n\in\mathcal{L}(P)\cup\{\tau\}$, and $\mathcal{L}(P')\subseteq\mathcal{L}(P)$, as desired.
\end{proof} 
%\documentclass[preprint,12pt]{elsarticle}
%\if0
\usepackage{amssymb}
\usepackage{mathtools}
%\usepackage[dvipdfmx]{graphicx}
\usepackage{cite}
\usepackage{graphicx}
\usepackage{bm}
\usepackage{here}
\usepackage[subrefformat=parens]{subcaption}
\fi
%\usepackage{amssymb}
\usepackage{amsmath}
\usepackage[dvipdfmx]{}
\usepackage[dvipdfmx]{color}
%\usepackage{cite}
%\usepackage{upgreek}
\usepackage{url}
%\usepackage[dvipdfmx]{hyperref}
%\usepackage{pxjahyper}
%\usepackage {hyperref}
\usepackage{graphicx}
\usepackage{bm}
\usepackage{here}
\usepackage{caption}
\usepackage[subrefformat=parens]{subcaption}
\captionsetup{compatibility=false}

%% The amsthm package provides extended theorem environments
%% \usepackage{amsthm}

%% The lineno packages adds line numbers. Start line numbering with
%% \begin{linenumbers}, end it with \end{linenumbers}. Or switch it on
%% for the whole article with \linenumbers after \end{frontmatter}.
%% \usepackage{lineno}

%% natbib.sty is loaded by default. However, natbib options can be
%% provided with \biboptions{...} command. Following options are
%% valid:

%%   round  -  round parentheses are used (default)
%%   square -  square brackets are used   [option]
%%   curly  -  curly braces are used      {option}
%%   angle  -  angle brackets are used    <option>
%%   semicolon  -  multiple citations separated by semi-colon
%%   colon  - same as semicolon, an earlier confusion
%%   comma  -  separated by comma
%%   numbers-  selects numerical citations
%%   super  -  numerical citations as superscripts
%%   sort   -  sorts multiple citations according to order in ref. list
%%   sort&compress   -  like sort, but also compresses numerical citations
%%   compress - compresses without sorting
%%
%% \biboptions{comma,round}

% \biboptions{}

%% This list environment is used for the references in the
%% Program Summary
%%
\newcounter{bla}
\newenvironment{refnummer}{%
\list{[\arabic{bla}]}%
{\usecounter{bla}%
 \setlength{\itemindent}{0pt}%
 \setlength{\topsep}{0pt}%
 \setlength{\itemsep}{0pt}%
 \setlength{\labelsep}{2pt}%
 \setlength{\listparindent}{0pt}%
 \settowidth{\labelwidth}{[9]}%
 \setlength{\leftmargin}{\labelwidth}%
 \addtolength{\leftmargin}{\labelsep}%
 \setlength{\rightmargin}{0pt}}}
 {\endlist}
\begin{document}

%--- Lagrangian and OK code

\subsection{Lagrangian code and OK3}
\begin{enumerate}
\item {\bf BC\_LC.cpp}\\
The boundary conditions are included in the procedure.
%ここから追記
\item {BeamMaking.cpp}\\
The function calculates the total input energy.
%ここまで追記
\item {\bf CONSTANT.h}\\
The file contains the definition of constant values and normalization factors.
\item {\bf Derf.c}\\
The file contains the error function in the double precision.
\item {\bf HIFScheme.h}\\
The file contains 1, 2, 3, 6, 12, 20, 32, 60 and 120-beam irradiation schemes. (see also Refs. \cite{ogoyski1, ogoyski2, ogoyski3}.)
%The file contains all direction coordinates of the irradiation schemes used in our simulations. The beam directions are defined in a spherical coordinate system linked to the reactor chamber center.
\item {\bf IMOK.cpp}\\
%\if0 The file contains a procedure to pass the data such  to OK3 and generate the spherical coordinates.  It also contains a function to pass the three-dimensional deposited energy distribution on a fuel target calculated in OK3 into a two-dimension Lagrangian target.\fi
%The file contains a procedure to transfer the data such as the target temperature and others to OK3. After the deposited energy distribution in OK3 is calculated, it is passed to the Lagrangian code.以下に修正
The file contains a procedure to transfer the Lagrange mesh date. The file sets the initial target surface numerically.

\item {\bf InitMesh\_LC.cpp}\\
The file initializes the Lagrangian coordinates and determines the number of the target layer. The number of the layers can be selected from 1 to 5 layers. The user must set the mesh number of each layer in this file.
\item {\bf InputOK3.h}\\
The input data file contains the target parameters, the HIB parameters. %, target mesh parameters and also the reactor chamber parameters. 削除
\item {\bf Insulation.cpp}\\
The file contains a procedure to calculate the adiabat $\alpha$ to evaluate the fuel preheating \cite{CPC-O-SUKI, ICFBook}. 
\item {\bf Lagrange\_set.cpp}
This function performs auxiliary calculations for spatial differentiation and thermal conductivity calculations in the Lagrange code. 
\item {\bf Legendre.cpp}\\
%The procedure performs the mode analyses based on the Legendre function in order to find the implosion non-uniformity. The analysis results are also output in this procedure. 以下に修正
The procedure performs the mode analyses based on the spherical harmonics in order to find the implosion non-uniformity. The analysis results are also output in this procedure. 
\item {\bf Lr\_LC.cpp}\\
A procedure to calculate the Rosseland mean free path (see Ref. \cite {Zeldovich}). 
\item {\bf MS.cpp}\\
A function to solve matrix by the Gauss elimination method. 
%ここから追記
This function is optimized specifically for Langnge calculations.  
%ここまで追記
\item {\bf MS\_TDMA.cpp}\\
A function to solve matrix by TDMA (TriDiagonal-Matrix Algorithm).
\item {\bf OK3code.cpp}\\
The file is the main routine of the HIBs illumination code of OK3 and contains the following procedures\cite{ogoyski1,ogoyski2,ogoyski3}. The details for each procedure relating to the HIBs illumination code OK3 are found in Ref. \cite{CPC-O-SUKI, ogoyski1, ogoyski2, ogoyski3}. The relating procedures are listed here: Irradiation(), InitEdp1(), Focus(), fDis(), Divider(), kBunch(), PointC(), PointF(), PointAlpha(), BeamCenterRot(), BeamletRot( ), Rotation() and StoppingPower1.cpp. The procedure of StoppingPower1.cpp contains a function Stop1. This function serves a heart of the OK1 code \cite{ogoyski1} and describes the energy deposition model. It calculates the stopping power from the projectile ions into the solid target. The one-ion stopping power is considered to be a sum of the deposition energy in the target nuclei, the target bound and free electrons and the target ions\cite{mehlhorn}. 

\item {\bf PelletSurface.h}\\
The file sets the initial target surface numerically.
\item {\bf RMS.cpp}\\
The procedure in this file calculates the root-mean-square (RMS) deviation in target non uniformity.
\item {\bf ResultIMP.cpp}\\
This file contains a procedure to calculate the implosion velocity.
\item {\bf SLC.cpp}\\
%This file contains a procedure that outputs the time history of each physical quantities obtained by cutting out one of the theta directions. 以下に修正
This file contains the procedure to output the time history of each physical quantity obtained by cutting one each in $\theta$ and $\phi$ directions. The positions of $\theta$ and $\phi$ are changed in "input\_LC.h". 

\item {\bf Acceleration.cpp}\\
A procedure for calculating the target acceleration. 

\item {\bf artv\_LC.cpp}\\
This file contains a procedure calculate the artificial viscosity. When dealing with shock waves propagating in a compressive fluid at a supersonic speed in fluid dynamics simulations, it is impossible to employ sufficient number of multiple meshes to describe the real shock front structure, because its thickness is very thin. As a method, we introduce the following artificial viscosity devised by Von Neumann and Richtmyer\cite{artv}.\\

     The three-dimensional artificial viscosity is written:
 
\begin{equation}
		q_A=\rho c^2_1\left|\displaystyle\frac{\partial }{\partial i}\left(\frac{\partial u}{\partial i}\right)^A_{\_}\right|\left(\frac{\partial u}{\partial i}\right)^A_{\_}\\
		\label{artv_LC1}
\end{equation}
\begin{equation}
		q_B=\rho c^2_1\left|\displaystyle\frac{\partial }{\partial j}\left(\frac{\partial u}{\partial j}\right)^B_{\_}\right|\left(\frac{\partial u}{\partial j}\right)^B_{\_}\\
		\label{artv_LC2}
\end{equation}
\begin{equation}
		q_C=\rho c^2_1\left|\displaystyle\frac{\partial }{\partial k}\left(\frac{\partial u}{\partial k}\right)^C_{\_}\right|\left(\frac{\partial u}{\partial k}\right)^C_{\_}
		\label{artv_LC3}
	\end{equation}
	
	\begin{eqnarray*}
		&&\left(\frac{\partial u}{\partial i}\right)^A_{\_}=\min{\left[\left(\frac{\partial u}{\partial i}\right)^A,0\right]}\nonumber\\
		&&\left(\frac{\partial u}{\partial j}\right)^B_{\_}=\min{\left[\left(\frac{\partial u}{\partial j}\right)^B,0\right]}\nonumber\\
		&&\left(\frac{\partial u}{\partial k}\right)^C_{\_}=\min{\left[\left(\frac{\partial u}{\partial k}\right)^C,0\right]}\nonumber\\
		&&\left(\frac{\partial u}{\partial i}\right)^A=\frac{\bar{\bm R}_i\cdot{\bm u}_i}{\left|\bar{\bm R}_i\right|}\nonumber\\
		&&\left(\frac{\partial u}{\partial j}\right)^B=\frac{\bar{\bm R}_j\cdot{\bm u}_j}{\left|\bar{\bm R}_j\right|}\nonumber\\
		&&\left(\frac{\partial u}{\partial k}\right)^C=\frac{\bar{\bm R}_k\cdot{\bm u}_k}{\left|\bar{\bm R}_k\right|}\nonumber
	\end{eqnarray*}

Here $\bar{\bm R_i}, \bar{\bm R_j}$ and $\bar{\bm R_k}$ are the normal vectors to the $i, j, k$ directions, respectively. $q_A$, $q_B$ and $q_C$ are the artificial viscosities in the directions of $\bar{\bm R_i}$, $\bar{\bm R_j}$ and $\bar{\bm R_k}$, respectively. Equations (\ref{artv_LC1}), (\ref{artv_LC2}) and (\ref{artv_LC3}) are discretized as follows: 
	\begin{eqnarray}
\begin{split}
		&{q_A}^{n}_{i+\frac{1}{2},j+\frac{1}{2},k+\frac{1}{2}}=\\
&\left({q_{A1}}^{n}_{i+\frac{1}{2},j+\frac{1}{2},k+\frac{1}{2}}+{q_{A2}}^{n}_{i+\frac{1}{2},j+\frac{1}{2},k+\frac{1}{2}}+{q_{A3}}^{n}_{i+\frac{1}{2},j+\frac{1}{2},k+\frac{1}{2}}+{q_{A4}}^{n}_{i+\frac{1}{2},j+\frac{1}{2},k+\frac{1}{2}}\right)
\end{split}\\
\begin{split}
		&{q_B}^{n}_{i+\frac{1}{2},j+\frac{1}{2},k+\frac{1}{2}}=\\
&\left({q_{B1}}^{n}_{i+\frac{1}{2},j+\frac{1}{2},k+\frac{1}{2}}+{q_{B2}}^{n}_{i+\frac{1}{2},j+\frac{1}{2},k+\frac{1}{2}}+{q_{B3}}^{n}_{i+\frac{1}{2},j+\frac{1}{2},k+\frac{1}{2}}+{q_{A4}}^{n}_{i+\frac{1}{2},j+\frac{1}{2},k+\frac{1}{2}}\right)
\end{split}\\
\begin{split}
		&{q_C}^{n}_{i+\frac{1}{2},j+\frac{1}{2},k+\frac{1}{2}}=\\
&\left({q_{C1}}^{n}_{i+\frac{1}{2},j+\frac{1}{2},k+\frac{1}{2}}+{q_{C2}}^{n}_{i+\frac{1}{2},j+\frac{1}{2},k+\frac{1}{2}}+{q_{C3}}^{n}_{i+\frac{1}{2},j+\frac{1}{2},k+\frac{1}{2}}+{q_{A4}}^{n}_{i+\frac{1}{2},j+\frac{1}{2},k+\frac{1}{2}}\right)
\end{split}
	\end{eqnarray}
	
	Here, the expression appeared are summarized below: 
	\begin{eqnarray}
		{q_{A1}}^{n}_{i+\frac{1}{2},j+\frac{1}{2},k+\frac{1}{2}}=\rho^{n}_{i+\frac{1}{2},j+\frac{1}{2},k+\frac{1}{2}}c_1^2ddiV^{A1}_{i+\frac{1}{2},j+\frac{1}{2},k+\frac{1}{2}}diV^{A1}_{i+\frac{1}{2},j+\frac{1}{2},k+\frac{1}{2}}\\
		{q_{A2}}^{n}_{i+\frac{1}{2},j+\frac{1}{2},k+\frac{1}{2}}=\rho^{n}_{i+\frac{1}{2},j+\frac{1}{2},k+\frac{1}{2}}c_1^2ddiV^{A2}_{i+\frac{1}{2},j+\frac{1}{2},k+\frac{1}{2}}diV^{A2}_{i+\frac{1}{2},j+\frac{1}{2},k+\frac{1}{2}}\\
		{q_{A3}}^{n}_{i+\frac{1}{2},j+\frac{1}{2},k+\frac{1}{2}}=\rho^{n}_{i+\frac{1}{2},j+\frac{1}{2},k+\frac{1}{2}}c_1^2ddiV^{A3}_{i+\frac{1}{2},j+\frac{1}{2},k+\frac{1}{2}}diV^{A3}_{i+\frac{1}{2},j+\frac{1}{2},k+\frac{1}{2}}\\
		{q_{A4}}^{n}_{i+\frac{1}{2},j+\frac{1}{2},k+\frac{1}{2}}=\rho^{n}_{i+\frac{1}{2},j+\frac{1}{2},k+\frac{1}{2}}c_1^2ddiV^{A4}_{i+\frac{1}{2},j+\frac{1}{2},k+\frac{1}{2}}diV^{A4}_{i+\frac{1}{2},j+\frac{1}{2},k+\frac{1}{2}}\\
		{q_{B1}}^{n}_{i+\frac{1}{2},j+\frac{1}{2},k+\frac{1}{2}}=\rho^{n}_{i+\frac{1}{2},j+\frac{1}{2},k+\frac{1}{2}}c_1^2ddjV^{B1}_{i+\frac{1}{2},j+\frac{1}{2},k+\frac{1}{2}}djV^{B1}_{i+\frac{1}{2},j+\frac{1}{2},k+\frac{1}{2}}\\
		{q_{B2}}^{n}_{i+\frac{1}{2},j+\frac{1}{2},k+\frac{1}{2}}=\rho^{n}_{i+\frac{1}{2},j+\frac{1}{2},k+\frac{1}{2}}c_1^2ddjV^{B2}_{i+\frac{1}{2},j+\frac{1}{2},k+\frac{1}{2}}djV^{B2}_{i+\frac{1}{2},j+\frac{1}{2},k+\frac{1}{2}}\\
		{q_{B3}}^{n}_{i+\frac{1}{2},j+\frac{1}{2},k+\frac{1}{2}}=\rho^{n}_{i+\frac{1}{2},j+\frac{1}{2},k+\frac{1}{2}}c_1^2ddjV^{B3}_{i+\frac{1}{2},j+\frac{1}{2},k+\frac{1}{2}}djV^{B3}_{i+\frac{1}{2},j+\frac{1}{2},k+\frac{1}{2}}\\
		{q_{B4}}^{n}_{i+\frac{1}{2},j+\frac{1}{2},k+\frac{1}{2}}=\rho^{n}_{i+\frac{1}{2},j+\frac{1}{2},k+\frac{1}{2}}c_1^2ddjV^{B4}_{i+\frac{1}{2},j+\frac{1}{2},k+\frac{1}{2}}djV^{B4}_{i+\frac{1}{2},j+\frac{1}{2},k+\frac{1}{2}}\\
		{q_{C1}}^{n}_{i+\frac{1}{2},j+\frac{1}{2},k+\frac{1}{2}}=\rho^{n}_{i+\frac{1}{2},j+\frac{1}{2},k+\frac{1}{2}}c_1^2ddkV^{C1}_{i+\frac{1}{2},j+\frac{1}{2},k+\frac{1}{2}}dkV^{C1}_{i+\frac{1}{2},j+\frac{1}{2},k+\frac{1}{2}}\\
		{q_{C2}}^{n}_{i+\frac{1}{2},j+\frac{1}{2},k+\frac{1}{2}}=\rho^{n}_{i+\frac{1}{2},j+\frac{1}{2},k+\frac{1}{2}}c_1^2ddkV^{C2}_{i+\frac{1}{2},j+\frac{1}{2},k+\frac{1}{2}}dkV^{C2}_{i+\frac{1}{2},j+\frac{1}{2},k+\frac{1}{2}}\\
		{q_{C3}}^{n}_{i+\frac{1}{2},j+\frac{1}{2},k+\frac{1}{2}}=\rho^{n}_{i+\frac{1}{2},j+\frac{1}{2},k+\frac{1}{2}}c_1^2ddkV^{C3}_{i+\frac{1}{2},j+\frac{1}{2},k+\frac{1}{2}}dkV^{C3}_{i+\frac{1}{2},j+\frac{1}{2},k+\frac{1}{2}}\\
		{q_{C4}}^{n}_{i+\frac{1}{2},j+\frac{1}{2},k+\frac{1}{2}}=\rho^{n}_{i+\frac{1}{2},j+\frac{1}{2},k+\frac{1}{2}}c_1^2ddkV^{C4}_{i+\frac{1}{2},j+\frac{1}{2},k+\frac{1}{2}}dkV^{C4}_{i+\frac{1}{2},j+\frac{1}{2},k+\frac{1}{2}}
	\end{eqnarray}
\small
	\begin{eqnarray*}
		&&diV^{A1}_{i+\frac{1}{2},j+\frac{1}{2},k+\frac{1}{2}}=\min{\left[\frac{\bar{\bm Ri}^{n+\frac{1}{2}}_{i+\frac{1}{2},j+\frac{1}{2},k+\frac{1}{2}}\cdot{\frac{\partial \bm u}{\partial i}}^{n+\frac{1}{2}}_{i+\frac{1}{2},j,k}}{\left|\bar{\bm Ri}^n_{i+\frac{1}{2},j+\frac{1}{2},k+\frac{1}{2}}\right|},0\right]}\\
		&&diV^{A2}_{i+\frac{1}{2},j+\frac{1}{2},k+\frac{1}{2}}=\min{\left[\frac{\bar{\bm Ri}^{n+\frac{1}{2}}_{i+\frac{1}{2},j+\frac{1}{2},k+\frac{1}{2}}\cdot{\frac{\partial \bm u}{\partial i}}^{n+\frac{1}{2}}_{i+\frac{1}{2},j+1,k}}{\left|\bar{\bm Ri}^n_{i+\frac{1}{2},j+\frac{1}{2},k+\frac{1}{2}}\right|},0\right]}\\
		&&diV^{A3}_{i+\frac{1}{2},j+\frac{1}{2},k+\frac{1}{2}}=\min{\left[\frac{\bar{\bm Ri}^{n+\frac{1}{2}}_{i+\frac{1}{2},j+\frac{1}{2},k+\frac{1}{2}}\cdot{\frac{\partial \bm u}{\partial i}}^{n+\frac{1}{2}}_{i+\frac{1}{2},j+1,k+1}}{\left|\bar{\bm Ri}^n_{i+\frac{1}{2},j+\frac{1}{2},k+\frac{1}{2}}\right|},0\right]}\\
		&&diV^{A4}_{i+\frac{1}{2},j+\frac{1}{2},k+\frac{1}{2}}=\min{\left[\frac{\bar{\bm Ri}^{n+\frac{1}{2}}_{i+\frac{1}{2},j+\frac{1}{2},k+\frac{1}{2}}\cdot{\frac{\partial \bm u}{\partial i}}^{n+\frac{1}{2}}_{i+\frac{1}{2},j,k+1}}{\left|\bar{\bm Ri}^n_{i+\frac{1}{2},j+\frac{1}{2},k+\frac{1}{2}}\right|},0\right]}\\
		&&djV^{B1}_{i+\frac{1}{2},j+\frac{1}{2},k+\frac{1}{2}}=\min{\left[\frac{\bar{\bm Rj}^{n+\frac{1}{2}}_{i+\frac{1}{2},j+\frac{1}{2},k+\frac{1}{2}}\cdot{\frac{\partial \bm u}{\partial j}}^{n+\frac{1}{2}}_{i,j+\frac{1}{2},k}}{\left|\bar{\bm Rj}^n_{i+\frac{1}{2},j+\frac{1}{2},k+\frac{1}{2}}\right|},0\right]}\\
		&&djV^{B2}_{i+\frac{1}{2},j+\frac{1}{2},k+\frac{1}{2}}=\min{\left[\frac{\bar{\bm Rj}^{n+\frac{1}{2}}_{i+\frac{1}{2},j+\frac{1}{2},k+\frac{1}{2}}\cdot{\frac{\partial \bm u}{\partial j}}^{n+\frac{1}{2}}_{i+1,j+\frac{1}{2},k}}{\left|\bar{\bm Rj}^n_{i+\frac{1}{2},j+\frac{1}{2},k+\frac{1}{2}}\right|},0\right]}\\
		&&djV^{B3}_{i+\frac{1}{2},j+\frac{1}{2},k+\frac{1}{2}}=\min{\left[\frac{\bar{\bm Rj}^{n+\frac{1}{2}}_{i+\frac{1}{2},j+\frac{1}{2},k+\frac{1}{2}}\cdot{\frac{\partial \bm u}{\partial j}}^{n+\frac{1}{2}}_{i+1,j+\frac{1}{2},k+1}}{\left|\bar{\bm Rj}^n_{i+\frac{1}{2},j+\frac{1}{2},k+\frac{1}{2}}\right|},0\right]}\\
		&&djV^{B4}_{i+\frac{1}{2},j+\frac{1}{2},k+\frac{1}{2}}=\min{\left[\frac{\bar{\bm Rj}^{n+\frac{1}{2}}_{i+\frac{1}{2},j+\frac{1}{2},k+\frac{1}{2}}\cdot{\frac{\partial \bm u}{\partial j}}^{n+\frac{1}{2}}_{i,j+\frac{1}{2},k+1}}{\left|\bar{\bm Rj}^n_{i+\frac{1}{2},j+\frac{1}{2},k+\frac{1}{2}}\right|},0\right]}\\
		&&dkV^{C1}_{i+\frac{1}{2},j+\frac{1}{2},k+\frac{1}{2}}=\min{\left[\frac{\bar{\bm Rk}^{n+\frac{1}{2}}_{i+\frac{1}{2},j+\frac{1}{2},k+\frac{1}{2}}\cdot{\frac{\partial \bm u}{\partial k}}^{n+\frac{1}{2}}_{i,j,k+\frac{1}{2}}}{\left|\bar{\bm Rk}^n_{i+\frac{1}{2},j+\frac{1}{2},k+\frac{1}{2}}\right|},0\right]}\\
		&&dkV^{C2}_{i+\frac{1}{2},j+\frac{1}{2},k+\frac{1}{2}}=\min{\left[\frac{\bar{\bm Rk}^{n+\frac{1}{2}}_{i+\frac{1}{2},j+\frac{1}{2},k+\frac{1}{2}}\cdot{\frac{\partial \bm u}{\partial k}}^{n+\frac{1}{2}}_{i+1,j,k+\frac{1}{2}}}{\left|\bar{\bm Rk}^n_{i+\frac{1}{2},j+\frac{1}{2},k+\frac{1}{2}}\right|},0\right]}
	\end{eqnarray*}
	\begin{eqnarray*}	
		&&dkV^{C3}_{i+\frac{1}{2},j+\frac{1}{2},k+\frac{1}{2}}=\min{\left[\frac{\bar{\bm Rk}^{n+\frac{1}{2}}_{i+\frac{1}{2},j+\frac{1}{2},k+\frac{1}{2}}\cdot{\frac{\partial \bm u}{\partial k}}^{n+\frac{1}{2}}_{i+1,j+1,k+\frac{1}{2}}}{\left|\bar{\bm Rk}^n_{i+\frac{1}{2},j+\frac{1}{2},k+\frac{1}{2}}\right|},0\right]}\\
		&&dkV^{C4}_{i+\frac{1}{2},j+\frac{1}{2},k+\frac{1}{2}}=\min{\left[\frac{\bar{\bm Rk}^{n+\frac{1}{2}}_{i+\frac{1}{2},j+\frac{1}{2},k+\frac{1}{2}}\cdot{\frac{\partial \bm u}{\partial k}}^{n+\frac{1}{2}}_{i,j+1,k+\frac{1}{2}}}{\left|\bar{\bm Rk}^n_{i+\frac{1}{2},j+\frac{1}{2},k+\frac{1}{2}}\right|},0\right]}\\
		&&ddiV^{A*}_{i,j+\frac{1}{2},k+\frac{1}{2}}=\left|diV^{A*}_{i+\frac{1}{2},j+\frac{1}{2},k+\frac{1}{2}}-diV^{A*}_{i-\frac{1}{2},j+\frac{1}{2},k+\frac{1}{2}}\right|\\
		&&ddiV^{A*}_{i+\frac{1}{2},j+\frac{1}{2},k+\frac{1}{2}}=\frac{1}{2}\left(ddiV^{A*}_{i,j+\frac{1}{2},k+\frac{1}{2}}+ddiV^{A*}_{i+1,j+\frac{1}{2},k+\frac{1}{2}}\right)\\
		&&ddjV^{B*}_{i+\frac{1}{2},j,k+\frac{1}{2}}=\left|djV^{B*}_{i+\frac{1}{2},j+\frac{1}{2},k+\frac{1}{2}}-djV^{B*}_{i+\frac{1}{2},j-\frac{1}{2},k+\frac{1}{2}}\right|\\
		&&ddjV^{B*}_{i+\frac{1}{2},j+\frac{1}{2},k+\frac{1}{2}}=\frac{1}{2}\left(ddjV^{B*}_{i+\frac{1}{2},j,k+\frac{1}{2}}+ddjV^{B*}_{i+\frac{1}{2},j+1,k+\frac{1}{2}}\right)\\
		&&ddkV^{C*}_{i+\frac{1}{2},j+\frac{1}{2},k}=\left|dkV^{C*}_{i+\frac{1}{2},j+\frac{1}{2},k+\frac{1}{2}}-dkV^{C*}_{i+\frac{1}{2},j+\frac{1}{2},k-\frac{1}{2}}\right|\\
		&&ddkV^{C*}_{i+\frac{1}{2},j+\frac{1}{2},k+\frac{1}{2}}=\frac{1}{2}\left(ddkV^{C*}_{i+\frac{1}{2},j+\frac{1}{2},k}+ddkV^{C*}_{i+\frac{1}{2},+\frac{1}{2}j,k+1}\right)\\
	\end{eqnarray*}
\normalsize
For the normal vectors to the $ i, j$ and $k$ directions, the outer products are used at  each side as shown in Fig. \ref{fig:qAbm}, and the averaged values are obtained:  
\small
\begin{eqnarray*}\label{vector}
	&&\bar{\bm Ri}^n_{i,j+\frac{1}{2},k+\frac{1}{2}}=\frac{\bar{\bm R_{A1}}^n_{i,j+\frac{1}{2},k+\frac{1}{2}}+\bar{\bm R_{A2}}^n_{i,j+\frac{1}{2},k+\frac{1}{2}}+\bar{\bm R_{A3}}^n_{i,j+\frac{1}{2},k+\frac{1}{2}}+\bar{\bm R_{A4}}^n_{i,j+\frac{1}{2},k+\frac{1}{2}}}{4}\\	
	&&\bar{\bm Rj}^n_{i+\frac{1}{2},j,k+\frac{1}{2}}=\frac{\bar{\bm R_{B1}}^n_{i+\frac{1}{2},j,k+\frac{1}{2}}+\bar{\bm R_{B2}}^n_{i+\frac{1}{2},j,k+\frac{1}{2}}+\bar{\bm R_{B3}}^n_{i+\frac{1}{2},j,k+\frac{1}{2}}+\bar{\bm R_{B4}}^n_{i+\frac{1}{2},j,k+\frac{1}{2}}}{4}\\
	&&\bar{\bm Rk}^n_{i+\frac{1}{2},j+\frac{1}{2},k}=\frac{\bar{\bm R_{C1}}^n_{i+\frac{1}{2},j+\frac{1}{2},k}+\bar{\bm R_{C2}}^n_{i+\frac{1}{2},j+\frac{1}{2},k}+\bar{\bm R_{C3}}^n_{i+\frac{1}{2},j+\frac{1}{2},k}+\bar{\bm R_{C4}}^n_{i+\frac{1}{2},j+\frac{1}{2},k}}{4}
\end{eqnarray*}
\normalsize
\begin{figure}[H]
	\centering
	\includegraphics[width=10cm]{images/avrtAbm.eps}
	\caption{The nomal vector to the $i$ direction}\label{fig:qAbm}
\end{figure}


% \begin{figure}[H]
%	\centering
%	\includegraphics[width=15cm]{images/avrtBbm.png}
%	\caption{The nomal vector of $j$ derection}\label{fig:qBbm}
%\end{figure}
%\begin{figure}[H]
%	\centering
%	\includegraphics[width=15cm]{images/avrtCbm.png}
%	\caption{The nomal vector of $k$ derection}\label{fig:qCbm}
%\end{figure}
%}


%\item {\bf check.cpp}\\
%This file contains a function that checks whether the argument is divergent.???
\item {\bf coc\_LC.cpp}\\
The file calculates the Lagrangian mesh dynamics. The Lagrangian meshes move together with the fluid motion. The new position coordinates for each mesh point are renewed at $n+1$. 

%\item {\bf cotc\_define.h}\\
%Define variables for calculating the heat conduction. 未使用につき削除

%\item {\bf cotc\_e.cpp, cotc\_r.cpp}\\ 以下に修正
\item {\bf cotc3D\_e.cpp, cotc3D\_r.cpp}\\
For calculation of the heat conduction, the following basic equation is used\cite{Christiansen}.
\begin{eqnarray}
		&&C_{V_k}\frac{DT}{Dt}=\frac{1}{\rho}{\bm \nabla}\cdot(\kappa_k{\bm \nabla}T_k)\ \ \ \ \ \ \ (k=e,r)\\
		%&&\ \ \ \ \kappa_i=4.3\times10^{-12}T^{5/2}_{i}(\log{\Lambda})m^{-1/2}Z^{-4}\ \ [{\rm W/mK}]\nonumber\\
		&&\ \ \ \ \kappa_e=1.83\times10^{-10}T^{5/2}_{e}(\log{\Lambda})^{-1}Z^{-1}\ \ [{\rm W/mK}]\nonumber\\
		&&\ \ \ \ \kappa_r=\frac{16}{3}\sigma L_RT^3_r\ \ [{\rm W/mK}]\nonumber
	\end{eqnarray}
	
\begin{align*}
\kappa&: Heat conductivity\\
T_k&: Ion,\ electron,\ radiation\ temperature[\rm K]\\
\log{\Lambda}&: Coulomb\ logarithm\\
m&: Mass\\
Z&: Ionization\ degree\\
\sigma&: Stefan-Boltzmann\ constant\\
L_R&: \rm {Rosseland\ mean\ free\ path}\\
\end{align*}

Here,  If $\bm S$ is the area vector of each surface of the mesh, the basic equation is transformed as follows: 
	\begin{eqnarray}\label{trans2}
		C_v\frac{DT}{Dt}&=&\frac{1}{\rho}{\bm \nabla}\cdot(\kappa{\bm \nabla}T)\nonumber\\
		\iiint \rho C_v\frac{DT}{Dt}dV&=&\iiint{\bm \nabla}\cdot(\kappa{\bm \nabla}T)dV\nonumber\\
		MC_v\frac{DT}{Dt}&=&\sum_i \left(\kappa{\bm \nabla} T\right)\cdot \bm{S_i}\nonumber\\
		\frac{DT}{Dt}&=&\frac{1}{MC_v}\sum_i 	\bm{S_i} \cdot \left[\frac{\kappa}{J}Di\bm r\frac{\partial T}{\partial i}+\frac{\kappa}{J}Dj\bm r\frac{\partial T}{\partial j}+\frac{\kappa}{J}Dk\bm r\frac{\partial T}{\partial k}\right]\nonumber\\
	\end{eqnarray}
	
Equation (\ref{trans2}) is discretized as follows:
\small
	\begin{eqnarray}\label{trans3}
		\frac{T^{n+1}_{i+\frac{1}{2},j+\frac{1}{2},k+\frac{1}{2}}-T^{n}_{i+\frac{1}{2},j+\frac{1}{2},k+\frac{1}{2}}}{Dt^{n+\frac{1}{2}}}&=&\frac{1}{MC_v}\nonumber\\
		&\times&\left[A_2\Bigl(T^{n+1}_{i+\frac{3}{2},j+\frac{1}{2},k+\frac{1}{2}}-T^{n+1}_{i+\frac{1}{2},j+\frac{1}{2},k+\frac{1}{2}}\Bigr)\right.\nonumber\\
		&-&A_1\Bigl(T^{n+1}_{i+\frac{1}{2},j+\frac{1}{2},k+\frac{1}{2}}-T^{n+1}_{i-\frac{1}{2},j+\frac{1}{2},k+\frac{1}{2}}\Bigr)\nonumber\\
		&+&B_2\Bigl(T^{n+1}_{i+\frac{1}{2},j+\frac{3}{2},k+\frac{1}{2}}-T^{n+1}_{i+\frac{1}{2},j+\frac{1}{2},k+\frac{1}{2}}\Bigr)\nonumber\\
		&-&B_1\Bigl(T^{n+1}_{i+\frac{1}{2},j+\frac{1}{2},k+\frac{1}{2}}-T^{n+1}_{i+\frac{1}{2},j-\frac{1}{2},k+\frac{1}{2}}\Bigr)\nonumber\\
		&+&C_2\Bigl(T^{n+1}_{i+\frac{1}{2},j+\frac{1}{2},k+\frac{3}{2}}-T^{n+1}_{i+\frac{1}{2},j+\frac{1}{2},k+\frac{1}{2}}\Bigr)\nonumber\\
		&-&\left.C_1\Bigl(T^{n+1}_{i+\frac{1}{2},j+\frac{1}{2},k+\frac{1}{2}}-T^{n+1}_{i+\frac{1}{2},j-\frac{1}{2},k-\frac{1}{2}}\Bigr)\right]\nonumber\\
	\end{eqnarray}
\normalsize
Here, the coefficients in Eq. (\ref{trans3}) are listed: 
\small
\begin{eqnarray*}
&&A_2=\Bigl|{Di\bm r}^n_{i+1,j+\frac{1}{2},k+\frac{1}{2}}\Bigr|{\Bigl|\bm{Si}^n_{i+1,j+\frac{1}{2},k+\frac{1}{2}}\Bigr|}\frac{\kappa^n_{i+1,j+\frac{1}{2},k+\frac{1}{2}}}{J^n_{i+1,j+\frac{1}{2},k+\frac{1}{2}}}\\
&&A_1=\Bigl|{Di\bm r}^n_{i,j+\frac{1}{2},k+\frac{1}{2}}\Bigr|{\Bigl|\bm{Si}^n_{i,j+\frac{1}{2},k+\frac{1}{2}}\Bigr|}\frac{\kappa^n_{i,j+\frac{1}{2},k+\frac{1}{2}}}{J^n_{i+1,j+\frac{1}{2},k+\frac{1}{2}}}\\
&&B_2=\Bigl|{Dj\bm r}^n_{i+\frac{1}{2},j+1,k+\frac{1}{2}}\Bigr|{\Bigl|\bm{Sj}^n_{i+\frac{1}{2},j+1,k+\frac{1}{2}}\Bigr|}\frac{\kappa^n_{i+\frac{1}{2},j+1,k+\frac{1}{2}}}{J^n_{i+\frac{1}{2},j+1,k+\frac{1}{2}}}\\
&&B_1=\Bigl|{Dj\bm r}^n_{i+\frac{1}{2},j,k+\frac{1}{2}}\Bigr|{\Bigl|\bm{Sj}^n_{i+\frac{1}{2},j,k+\frac{1}{2}}\Bigr|}\frac{\kappa^n_{i+\frac{1}{2},j,k+\frac{1}{2}}}{J^n_{i+\frac{1}{2},j,k+\frac{1}{2}}}\\
&&C_2=\Bigl|{Dk\bm r}^n_{i+\frac{1}{2},j+\frac{1}{2},k+1}\Bigr|{\Bigl|\bm{Sk}^n_{i+\frac{1}{2},j+\frac{1}{2},k+1}\Bigr|}\frac{\kappa^n_{i+\frac{1}{2},j+\frac{1}{2},k+1}}{J^n_{i+\frac{1}{2},j+\frac{1}{2},k+1}}\\
&&C_1=\Bigl|{Dk\bm r}^n_{i+\frac{1}{2},j+\frac{1}{2},k}\Bigr|{\Bigl|\bm{Sk}^n_{i+\frac{1}{2},j+\frac{1}{2},k}\Bigr|}\frac{\kappa^n_{i+\frac{1}{2},j+\frac{1}{2},k}}{J^n_{i+\frac{1}{2},j+\frac{1}{2},k}}\\
\end{eqnarray*}
\normalsize
	
	
\item {\bf define\_LC.h}\\
It contains the procedure declarations for the Lagrangian code and Ok3. 

\item {\bf dif\_LC.cpp}\\
The following Lagrangian equation of motion is used.
\begin{equation}\label{dme}
		\rho\frac{D\bm u}{Dt}=-{\bm\nabla}(P+q)
	\end{equation}
Equation (\ref {dme}) is expressed as follows: 
	\begin{equation}
\begin{split}\label{dmeu}
\frac{D\bm u}{Dt}\bigg|^n_{i,j,k}&
=-\frac{1}{\rho}\frac{\partial}{\partial \bm r} (P+q)\\&
=-\frac{1}{M^n_{i,j,k}}\left[Di\bm r\frac{\partial}{\partial i}(P+q_A)+Di\bm r\frac{\partial}{\partial j}(P+q_B)+Di\bm r\frac{\partial}{\partial k}(P+q_C)\right]^n_{i,j,k}\\&
\end{split}
\end{equation}
Equation (\ref{dmeu}) is discretized as follows: 

\small
	\begin{eqnarray}
		\bm u^{n+\frac{1}{2}}_{i,j,k}=\bm u^{n-\frac{1}{2}}_{i,j,k} - \frac{Dt^n}{M^n_{i,j,k}}\left[Di\bm r\frac{\partial}{\partial i}(P+q_A)+Di\bm r\frac{\partial}{\partial j}(P+q_B)+Di\bm r\frac{\partial}{\partial k}(P+q_C)\right]^n_{i,j,k}
		\label{dmeu2}
	\end{eqnarray}
\normalsize

Here, each term is shown: 
\begin{equation}
\begin{split}
Di\bm r\frac{\partial}{\partial i}(P+q_A)\bigg|^n_{i,j,k}&=
\frac{1}{4}\left[Di\bm r\frac{\partial}{\partial i}(P+q_{A1})\right]_{i,j+\frac{1}{2},k+\frac{1}{2}}\\&
+\frac{1}{4}\left[Di\bm r\frac{\partial}{\partial i}(P+q_{A2})\right]_{i,j-\frac{1}{2},k+\frac{1}{2}}\\&
+\frac{1}{4}\left[Di\bm r\frac{\partial}{\partial i}(P+q_{A3})\right]_{i,j-\frac{1}{2},k-\frac{1}{2}}\\&
+\frac{1}{4}\left[Di\bm r\frac{\partial}{\partial i}(P+q_{A4})\right]_{i,j+\frac{1}{2},k-\frac{1}{2}}\\
\end{split}
\end{equation}

\begin{equation}
\begin{split}
Dj\bm r\frac{\partial}{\partial j}(P+q_B)\bigg|^n_{i,j,k}&=
\frac{1}{4}\left[Dj\bm r\frac{\partial}{\partial j}(P+q_{B1})\right]_{i+\frac{1}{2},j,k+\frac{1}{2}}\\&
+\frac{1}{4}\left[Dj\bm r\frac{\partial}{\partial j}(P+q_{B2})\right]_{i-\frac{1}{2},j,k+\frac{1}{2}}\\&
+\frac{1}{4}\left[Dj\bm r\frac{\partial}{\partial j}(P+q_{B3})\right]_{i-\frac{1}{2},j,k-\frac{1}{2}}\\&
+\frac{1}{4}\left[Dj\bm r\frac{\partial}{\partial j}(P+q_{B4})\right]_{i+\frac{1}{2},j,k-\frac{1}{2}}\\
\end{split}
\end{equation}

\begin{equation}
\begin{split}
Dk\bm r\frac{\partial}{\partial k}(P+q_C)\bigg|^n_{i,j,k}&=
\frac{1}{4}\left[Dk\bm r\frac{\partial}{\partial k}(P+q_{C1})\right]_{i+\frac{1}{2},j+\frac{1}{2},k}\\&
+\frac{1}{4}\left[Dk\bm r\frac{\partial}{\partial k}(P+q_{C2})\right]_{i-\frac{1}{2},j+\frac{1}{2},k}\\&
+\frac{1}{4}\left[Dk\bm r\frac{\partial}{\partial k}(P+q_{C3})\right]_{i-\frac{1}{2},j-\frac{1}{2},k}\\&
+\frac{1}{4}\left[Dk\bm r\frac{\partial}{\partial k}(P+q_{C4})\right]_{i+\frac{1}{2},j-\frac{1}{2},k}\\
\end{split}
\end{equation}
	
\item {\bf dt\_LC.cpp}\\
This procedure calculates and controls the time step to satisfy the numerical stability condition. The time step $ \Delta t $ in the calculation must satisfy the following conditions.
	\begin{equation}
	\label{eq:Courant}
		\Delta t=\frac{\Delta r}{C_{S}+V_{max}}
	\end{equation}
The time step for the Lagrangian method $ Dt ^ {n + \frac {1}{2}} $ is represented by the following expression.
	\begin{equation}
		Dt^{n+\frac{1}{2}}=\alpha\displaystyle\frac{dr_{min}}{C_S+V_{max}}
	\end{equation}
		\begin{tabbing}
		12345678901234567890123\=789\=12345\=\kill\\
		\>\>$\alpha$\>: Numerical coefficient constant $(\alpha\leq1)$\\
		\>\>$dr_{min}$\>: the minimum grid spacing\\
		\>\>$C_S$\>: Sound speed\\
		\>\>$V_{max}$\>: the maximum flow speed\\
	\end{tabbing}
	
\item {\bf eoenergy\_LC.cpp}\\
The file contains a procedure for calculation of the energy equation. The following Lagrangian energy equation is used except for the heat conductions terms.
	\begin{eqnarray}
	\label{EoEkiso}
	\begin{cases}
		\frac{DT_i}{Dt}=-\frac{k_B}{C_{V_i}}\left[B_{T_i}\frac{D\rho}{Dt}+\frac{p_i+q}{M}\frac{DJ}{Dt}\right]\\
		\frac{DT_e}{Dt}=-\frac{k_B}{C_{V_e}}\left[B_{T_e}\frac{D\rho}{Dt}+\frac{p_e}{M}\frac{DJ}{Dt}\right]\\
		\frac{DT_r}{Dt}=-\frac{k_B}{C_{V_r}}\left[B_{T_r}\frac{D\rho}{Dt}+\frac{p_r}{M}\frac{DJ}{Dt}\right]
	\end{cases}
	\end{eqnarray}
Equation (\ref{EoEkiso}) is discretized as follows: 
	\small \begin{eqnarray}
		T^{n+1}_{i+\frac{1}{2},j+\frac{1}{2},k+\frac{1}{2}}&=&T^n_{i+\frac{1}{2},j+\frac{1}{2},k+\frac{1}{2}}\nonumber\\
		&-&\frac{1}{{C_V}^{n+\frac{1}{2}}_{i+\frac{1}{2},j+\frac{1}{2}k+\frac{1}{2}}}\Bigg[{B_T}^{n+1}_{i+\frac{1}{2},j+\frac{1}{2},k+\frac{1}{2}}(\rho^{n+1}_{i+\frac{1}{2},j+\frac{1}{2}k+\frac{1}{2}}-\rho^n_{i+\frac{1}{2},j+\frac{1}{2},k+\frac{1}{2}})\nonumber\\
		&+&\frac{P^{n+\frac{1}{2}}_{i+\frac{1}{2},j+\frac{1}{2},k+\frac{1}{2}}+q^{n+\frac{1}{2}}_{i+\frac{1}{2},j+\frac{1}{2},k+\frac{1}{2}}}{M_{i+\frac{1}{2},j+\frac{1}{2},k+\frac{1}{2}}}(J^{n+1}_{i+\frac{1}{2},j+\frac{1}{2}k+\frac{1}{2}}-J^n_{i+\frac{1}{2},j+\frac{1}{2}k+\frac{1}{2}})\Bigg]\nonumber
	\end{eqnarray}
	
\normalsize	
\item {\bf eos.cpp}\\
The file contains the procedures to calculate the equation of state. The equation of state for ions is the ideal one. For the equation of state for electrons and the ionization, we use the equation of state based on the Thomas-Fermi model shown in Ref. \cite{Bell}. Users can select the Thomas-Fermi model or the ideal equation of state in the header file of "input\_LC.h". For the equation of state for the radiation, we use the equilibrium blackbody equations \cite{Zeldovich}. 
	
\item {\bf init\_LC.h}\\
It contains the initial conditions such as the initial target temperature and so on. 

\item {\bf init\_matrix\_LC.cpp}
The file get the matrix.


\item {\bf input\_LC.h}\\
%The input data for Lagrangian code contains the target layer thickness, the input beam pulse, the beam radius and so on. 以下に修正
The input data for Lagrangian code contains radius, $\theta$ and $\phi$ direction mush number, each layers mesh number,HIB number, beam pulse parameters, fuel target structure,  output date step, etc. 

\item {\bf jacobian\_LC.cpp}\\
 The volume of each mesh is calculated. The Jacobian $J$ is expressed by the following formula. 
\begin{eqnarray}\label{jaco}
\begin{split}
	J
	&=\frac{\partial (x,y,z)}{\partial (i,j,k)}\\
	&=\left[
	\begin{array}{rrr}
	\frac{\partial x}{\partial i} & \frac{\partial x}{\partial j} & \frac{\partial x}{\partial k} \\
	\frac{\partial y}{\partial i} & \frac{\partial y}{\partial j} & \frac{\partial y}{\partial k} \\
	\frac{\partial z}{\partial i} & \frac{\partial z}{\partial j} & \frac{\partial z}{\partial k} 
	\end{array}
	\right]\\
	&=\frac{\partial x}{\partial i}\frac{\partial y}{\partial j}\frac{\partial z}{\partial k}+\frac{\partial x}{\partial j}\frac{\partial y}{\partial k}\frac{\partial z}{\partial i}+\frac{\partial x}{\partial k}\frac{\partial y}{\partial i}\frac{\partial z}{\partial j}
	-\frac{\partial x}{\partial k}\frac{\partial y}{\partial j}\frac{\partial z}{\partial i}-\frac{\partial x}{\partial j}\frac{\partial y}{\partial i}\frac{\partial z}{\partial k}-\frac{\partial x}{\partial i}\frac{\partial y}{\partial k}\frac{\partial z}{\partial j}
\end{split}
\end{eqnarray}
	From Eq. (\ref {jaco}), the Jacobian is expressed as follows: 
	\begin{eqnarray}\label{Ja_R}
	J^n_{k+\frac{1}{2},l+\frac{1}{2},m+\frac{1}{2}}&=&
\left(\frac{\partial x}{\partial i}\right)^n_{i+\frac{1}{2},j+\frac{1}{2},k+\frac{1}{2}}
\left(\frac{\partial y}{\partial j}\right)^n_{i+\frac{1}{2},j+\frac{1}{2},k+\frac{1}{2}}
\left(\frac{\partial z}{\partial k}\right)^n_{i+\frac{1}{2},j+\frac{1}{2},k+\frac{1}{2}}\nonumber\\
&+&\left(\frac{\partial x}{\partial j}\right)^n_{i+\frac{1}{2},j+\frac{1}{2},k+\frac{1}{2}}
\left(\frac{\partial y}{\partial k}\right)^n_{i+\frac{1}{2},j+\frac{1}{2},k+\frac{1}{2}}
\left(\frac{\partial z}{\partial i}\right)^n_{i+\frac{1}{2},j+\frac{1}{2},k+\frac{1}{2}}\nonumber\\
&+&\left(\frac{\partial x}{\partial k}\right)^n_{i+\frac{1}{2},j+\frac{1}{2},k+\frac{1}{2}}
\left(\frac{\partial y}{\partial i}\right)^n_{i+\frac{1}{2},j+\frac{1}{2},k+\frac{1}{2}}
\left(\frac{\partial z}{\partial j}\right)^n_{i+\frac{1}{2},j+\frac{1}{2},k+\frac{1}{2}}\nonumber\\
&-&\left(\frac{\partial x}{\partial k}\right)^n_{i+\frac{1}{2},j+\frac{1}{2},k+\frac{1}{2}}
\left(\frac{\partial y}{\partial j}\right)^n_{i+\frac{1}{2},j+\frac{1}{2},k+\frac{1}{2}}
\left(\frac{\partial z}{\partial i}\right)^n_{i+\frac{1}{2},j+\frac{1}{2},k+\frac{1}{2}}\nonumber\\
&-&\left(\frac{\partial x}{\partial j}\right)^n_{i+\frac{1}{2},j+\frac{1}{2},k+\frac{1}{2}}
\left(\frac{\partial y}{\partial i}\right)^n_{i+\frac{1}{2},j+\frac{1}{2},k+\frac{1}{2}}
\left(\frac{\partial z}{\partial k}\right)^n_{i+\frac{1}{2},j+\frac{1}{2},k+\frac{1}{2}}\nonumber\\
&-&\left(\frac{\partial x}{\partial i}\right)^n_{i+\frac{1}{2},j+\frac{1}{2},k+\frac{1}{2}}
\left(\frac{\partial y}{\partial k}\right)^n_{i+\frac{1}{2},j+\frac{1}{2},k+\frac{1}{2}}
\left(\frac{\partial z}{\partial j}\right)^n_{i+\frac{1}{2},j+\frac{1}{2},k+\frac{1}{2}}\nonumber
\end{eqnarray}
	Here, each term is obtained: 
\begin{eqnarray}
\begin{cases}
&(\frac{\partial x_{i,j,k}}{\partial i})^n_{i+\frac{1}{2},j+\frac{1}{2},k+\frac{1}{2}}=\frac{\Delta_i x^n_{i+\frac{1}{2},j+1,k}+\Delta_i x^n_{i+\frac{1}{2},j,k+1}+\Delta_i x^n_{i+\frac{1}{2},j+1,k+1}+\Delta_i x^n_{i+\frac{1}{2},j,k}}{4}\\
&(\frac{\partial x_{i,j,k}}{\partial j})^n_{i+\frac{1}{2},j+\frac{1}{2},k+\frac{1}{2}}=\frac{\Delta_j x^n_{i+1,j+\frac{1}{2},k}+\Delta_j x^n_{i,j+\frac{1}{2},k+1}+\Delta_j x^n_{i+1,j+\frac{1}{2},k+1}+\Delta_j x^n_{i,j+\frac{1}{2},k}}{4}\\
&(\frac{\partial x_{i,j,k}}{\partial k})^n_{i+\frac{1}{2},j+\frac{1}{2},k+\frac{1}{2}}=\frac{\Delta_k x^n_{i+1,j,k+\frac{1}{2}}+\Delta_k x^n_{i,j+1,k+\frac{1}{2}}+\Delta_k x^n_{i+1,j+1,k+\frac{1}{2}}+\Delta_k x^n_{i,j,k+\frac{1}{2}}}{4}\\
&(\frac{\partial y_{i,j,k}}{\partial i})^n_{i+\frac{1}{2},j+\frac{1}{2},k+\frac{1}{2}}=\frac{\Delta_i y^n_{i+\frac{1}{2},j+1,k}+\Delta_i y^n_{i+\frac{1}{2},j,k+1}+\Delta_i y^n_{i+\frac{1}{2},j+1,k+1}+\Delta_i y^n_{i+\frac{1}{2},j,k}}{4}\\
&(\frac{\partial y_{i,j,k}}{\partial j})^n_{i+\frac{1}{2},j+\frac{1}{2},k+\frac{1}{2}}=\frac{\Delta_j y^n_{i+1,j+\frac{1}{2},k}+\Delta_j y^n_{i,j+\frac{1}{2},k+1}+\Delta_j y^n_{i+1,j+\frac{1}{2},k+1}+\Delta_j y^n_{i,j+\frac{1}{2},k}}{4}\\
&(\frac{\partial y_{i,j,k}}{\partial k})^n_{i+\frac{1}{2},j+\frac{1}{2},k+\frac{1}{2}}=\frac{\Delta_k y^n_{i+1,j,k+\frac{1}{2}}+\Delta_k y^n_{i,j+1,k+\frac{1}{2}}+\Delta_k y^n_{i+1,j+1,k+\frac{1}{2}}+\Delta_k y^n_{i,j,k+\frac{1}{2}}}{4}\\
&(\frac{\partial z_{i,j,k}}{\partial i})^n_{i+\frac{1}{2},j+\frac{1}{2},k+\frac{1}{2}}=\frac{\Delta_i z^n_{i+\frac{1}{2},j+1,k}+\Delta_i z^n_{i+\frac{1}{2},j,k+1}+\Delta_i z^n_{i+\frac{1}{2},j+1,k+1}+\Delta_k z^n_{k+\frac{1}{2},l,m}}{4}\\
&(\frac{\partial z_{i,j,k}}{\partial j})^n_{i+\frac{1}{2},j+\frac{1}{2},k+\frac{1}{2}}=\frac{\Delta_j z^n_{i+1,l+\frac{1}{2},k}+\Delta_j z^n_{i,j+\frac{1}{2},k+1}+\Delta_j z^n_{i+1,j+\frac{1}{2},k+1}+\Delta_j z^n_{i,j+\frac{1}{2},k}}{4}\\
&(\frac{\partial z_{i,j,k}}{\partial k})^n_{i+\frac{1}{2},j+\frac{1}{2},k+\frac{1}{2}}=\frac{\Delta_k z^n_{i+1,j,k+\frac{1}{2}}+\Delta_k z^n_{i,j+1,k+\frac{1}{2}}+\Delta_k z^n_{i+1,j+1,k+\frac{1}{2}+}\Delta_k z^n_{i,j,k+\frac{1}{2}}}{4}
\end{cases}
\end{eqnarray}

\item {\bf main\_LC.cpp}\\
%The main procedure of the Lagrangian fluid code. 以下に修正
The main procedure of the Lagrangian fluid code. If you want to artificially add non-uniformity in the $\theta$ and $\phi$ directions without using the OK3 code, change it here. 
\item {\bf outputRMS.cpp}\\
It contains a procedure to output the results for the RMS non-uniformity.
\item {\bf output\_LC.cpp}\\
The result data are stored by this procedure. The time interval of data output is 0.1 ns in the Lagrangian code. The user can adjust the output step in "input\_LC.h". The physical quantity (for example, velocity) defined at the grid points of the mesh is output to outputS1. The physical quantity defined at the center of the mesh (for example, temperature, density) is output to outputS2. 
\item {\bf output\_to\_EulerCode.cpp}\\
This file contains a procedure for outputting the data used in Euler code. After the beam irradiation is completed, the file is output every 0.1ns. 
\item {\bf relax.cpp}\\
The following equation is used as the basic equation for the temperature relaxation\cite{Tahir}. 
	\begin{eqnarray}
		\begin{cases}
			C_{V_i}\frac{dT_i}{dt}=-K_{ie}\\
			C_{V_e}\frac{dT_e}{dt}=K_{ie}-K_{re}\\
			C_{V_r}\frac{dT_r}{dt}=K_{re}
		\end{cases}
	\end{eqnarray}

Here, $K_{ie}$ is the energy exchange rate between the ions and the electrons,  and $K_{re}$ the energy exchange rate between the radiation and the electrons.
\begin{eqnarray}
	\begin{cases}
		K_{ie}=C_{V_i}\omega_{ie}(T_i-T_e)\\
		K_{re}=C_{V_r}\omega_{re}(T_e-T_r)
	\end{cases}
	\end{eqnarray}
	$ \omega_{ie}$ and $ \omega_{re} $ are the collision frequencies between the ions and the electrons and between the radiation and the electrons, respectively. They are obtained by the following formulae: The Compton effect  between the radiation and the electrons is included. Each expression and the solution method are found in Refs. \cite{Tahir, CPC-O-SUKI}. 

%\item {\bf relax\_define.h}\\
I%t contains the procedure declarations for the temperature relaxation. 未使用につき削除

\end{enumerate}
%\include{end}

%\documentclass[preprint,12pt]{elsarticle}
%\if0
\usepackage{amssymb}
\usepackage{mathtools}
%\usepackage[dvipdfmx]{graphicx}
\usepackage{cite}
\usepackage{graphicx}
\usepackage{bm}
\usepackage{here}
\usepackage[subrefformat=parens]{subcaption}
\fi
%\usepackage{amssymb}
\usepackage{amsmath}
\usepackage[dvipdfmx]{}
\usepackage[dvipdfmx]{color}
%\usepackage{cite}
%\usepackage{upgreek}
\usepackage{url}
%\usepackage[dvipdfmx]{hyperref}
%\usepackage{pxjahyper}
%\usepackage {hyperref}
\usepackage{graphicx}
\usepackage{bm}
\usepackage{here}
\usepackage{caption}
\usepackage[subrefformat=parens]{subcaption}
\captionsetup{compatibility=false}

%% The amsthm package provides extended theorem environments
%% \usepackage{amsthm}

%% The lineno packages adds line numbers. Start line numbering with
%% \begin{linenumbers}, end it with \end{linenumbers}. Or switch it on
%% for the whole article with \linenumbers after \end{frontmatter}.
%% \usepackage{lineno}

%% natbib.sty is loaded by default. However, natbib options can be
%% provided with \biboptions{...} command. Following options are
%% valid:

%%   round  -  round parentheses are used (default)
%%   square -  square brackets are used   [option]
%%   curly  -  curly braces are used      {option}
%%   angle  -  angle brackets are used    <option>
%%   semicolon  -  multiple citations separated by semi-colon
%%   colon  - same as semicolon, an earlier confusion
%%   comma  -  separated by comma
%%   numbers-  selects numerical citations
%%   super  -  numerical citations as superscripts
%%   sort   -  sorts multiple citations according to order in ref. list
%%   sort&compress   -  like sort, but also compresses numerical citations
%%   compress - compresses without sorting
%%
%% \biboptions{comma,round}

% \biboptions{}

%% This list environment is used for the references in the
%% Program Summary
%%
\newcounter{bla}
\newenvironment{refnummer}{%
\list{[\arabic{bla}]}%
{\usecounter{bla}%
 \setlength{\itemindent}{0pt}%
 \setlength{\topsep}{0pt}%
 \setlength{\itemsep}{0pt}%
 \setlength{\labelsep}{2pt}%
 \setlength{\listparindent}{0pt}%
 \settowidth{\labelwidth}{[9]}%
 \setlength{\leftmargin}{\labelwidth}%
 \addtolength{\leftmargin}{\labelsep}%
 \setlength{\rightmargin}{0pt}}}
 {\endlist}
\begin{document}

%----- Conversion Code
\subsection{Conversion code}
The Euler meshes are constructed based on the size of the small Lagrange mesh in the conversion code. The 3D conversion process is performed after setting the upper limit of the Euler total mesh number. In order to meet the computer resource limitation, this prescription is employed in the O-SUKI-N 3D code. 

\begin{enumerate}
\item{\bf boundary\_set.cpp}
The function makes the boundary point data between DT and Al layer. 
\item{\bf check\_quantities.cpp}
The function outputs the data of the transformed Euler mesh as a text (csv) file. 
\item{\bf define\_convert.h}
Define the variables necessary for the conversion code.

\item{\bf GenerateEulerMesh.cpp}
The procedure determines the number of the Euler meshes and to secure the necessary memory, just before the data conversion.
\item{\bf Interpolation.cpp}
The function interpolates the data on the Lagrangian mesh to those on the Euler meshes. Figure \ref{interpolation_speed} shows the interpolation method from the Lagrange data to the Euler data. The "MeshSearch.cpp" provides the relation between the Lagrangian mesh location and the Euler mesh location. The following interpolation equation is used to obtain each physical quantity on the Euler meshes. For example, here $\bm u$ shows a velocity. 

\begin{eqnarray}\label{eq:interpolation}
	\bm u(P)&=&\frac{1}{sumR}\nonumber\\		
		&\times&\left[\Bigl(\frac{1}{r_{i,j,k}}\Bigr)^2\bm u_{i,j,k}+\Bigl(\frac{1}{r_{i+1,j,k}}\Bigr)^2\bm u_{i+1,j,k}+\Bigl(\frac{1}{r_{i+1,j+1,k}}\Bigr)^2\bm u_{i+1,j+1,k}\right.\nonumber\\
		&+&\Bigl(\frac{1}{r_{i,j+1,k}}\Bigr)^2\bm u_{i,j+1,k}+\Bigl(\frac{1}{r_{i,j,k+1}}\Bigr)^2\bm u_{i,j,k+1}+\Bigl(\frac{1}{r_{i+1,j,k+1}}\Bigr)^2\bm u_{i+1,j,k}\nonumber\\
		&+&\left.\Bigl(\frac{1}{r_{i+1,j+1,k}}\Bigr)^2\bm u_{i+1,j+1,k}+\Bigl(\frac{1}{r_{i,j+1,k+1}}\Bigr)^2\bm u_{i,j+1,k+1}\right]
	\end{eqnarray}

\begin{eqnarray}
	sumR&=&\left(\frac{1}{r_{i,j,k}}\right)^2+\left(\frac{1}{r_{i+1,j,k}}\right)^2+\left(\frac{1}{r_{i+1,j+1,k}}\right)^2+\left(\frac{1}{r_{i,j+1,k}}\right)^2\nonumber\\
	&+&\left(\frac{1}{r_{i,j,k+1}}\right)^2+\left(\frac{1}{r_{i+1,j,k+1}}\right)^2+\left(\frac{1}{r_{i+1,j+1,k+1}}\right)^2+\left(\frac{1}{r_{i,j+1,k+1}}\right)^2\nonumber
\end{eqnarray}

\begin{figure}[H]
	\centering
	\includegraphics[width=12cm]{images/speed_in.eps}
	\caption{Interpolation of velocity}\label{interpolation_speed}
\end{figure}

Usually the Euler mesh size is small compared with the size of the Lagrange mesh. For the interpolation of physical quantities other than velocity, the physical quantity of the corresponding Lagrange mesh acquired by the "MeshSearch.cpp" is interpolated by the 0th order method. On the other hand, if the corresponding Lagrange mesh is smaller than the Euler mesh, it is done in the same way shown in Fig. \ref{interpolation_speed} and Eq. (\ref{eq:interpolation}) for the example velocity interpolation. This is the special treatment in 3D to optimize the required memory size. 

\item{\bf main\_convert.cpp}
This is the main procedure of the conversion code. The procedure selects the output Lagrangian data transferred to the Euler code among the Lagrangian data sets obtained in the Lagrangian code. The Lagrangian meshes are deformed along with the fluid motion. The Lagrangian code stops, before no mesh is crushed. The conversion range is the all DT layer and a part of the Aluminum region. The volume of the Al region is 2.5 times larger than the thickness of the DT layer. The required number of the Euler meshes is calculated. The Lagrange data sets are examined from the data set from the time of 2ns earlier than the last output data set. The function selects the conversion date, which has the smallest number of the Euler mesh required. If the number of Euler meshes exceeds the number of Euler meshes set in the ''input\_LC.h'', the Euler mesh total number is forced to set to the upper limit defined beforehand. 



\item{\bf MeshSearch.cpp}
This procedure examines the location of each Euler mesh among the Lagrangian meshes. The MeshSearch function divides a Lagrange mesh into 12 triangular tetrahedra as shown in Fig. \ref{meshcut}, and examines if the definition point of an Euler mesh is contained in the specific Lagrange mesh. 

\begin{figure}[H]
	\centering
	\includegraphics[width=10cm]{images/meshcut.eps}
	\caption{One Lagrange mesh and 12 triangular tetrahedra}\label{meshcut}
\end{figure}

In Fig. \ref{meshsearch}, $\vec P$ represents a coordinate vector of a specific Euler mesh and $\vec R$ represents a coordinate vector of the Lagrangian mesh. The $point1$, $point2$, $point3$, $point4$ and $\vec V_P$ are the positions specified in Fig. \ref{meshsearch}, and vectors composed of the position vectors are as follows: 
\begin{eqnarray*}
	\begin{cases}
		{\vec V}_{11}={\vec R}_{point2}-{\vec R}_{point1}\\
		{\vec V}_{12}={\vec R}_{point3}-{\vec R}_{point1}\\
		{\vec V}_{P1}={\vec P}-{\vec R}_{point1}
	\end{cases}
	\begin{cases}
		{\vec V}_{21}={\vec R}_{point3}-{\vec R}_{point1}\\
		{\vec V}_{22}={\vec R}_{point4}-{\vec R}_{point1}\\
		{\vec V}_{P2}={\vec P}-{\vec R}_{point1}
	\end{cases}\\
	\begin{cases}
		{\vec V}_{31}={\vec R}_{point4}-{\vec R}_{point1}\\
		{\vec V}_{32}={\vec R}_{point1}-{\vec R}_{point1}\\
		{\vec V}_{P3}={\vec P}-{\vec R}_{point1}
	\end{cases}
\begin{cases}
		{\vec V}_{41}={\vec R}_{point4}-{\vec R}_{point2}\\
		{\vec V}_{42}={\vec R}_{point3}-{\vec R}_{point2}\\
		{\vec V}_{P4}={\vec P}-{\vec R}_{point2}
	\end{cases}
\end{eqnarray*}

\begin{figure}[H]
	\centering
	\includegraphics[width=10cm]{images/Con_meshP.eps}
	\caption{An Euler mesh point $P$ in a tetrahedron of the Lagrange mesh. }\label{meshsearch}
\end{figure}

If the point $P$ is in the triangular pyramid, the following conditions are met. 

\begin{eqnarray}
		\begin{cases}
			\left({\vec V}_{11}\times{\vec V}_{12}\right)\cdot{\vec V}_{P1}>0\\
			\left({\vec V}_{21}\times{\vec V}_{22}\right)\cdot{\vec V}_{P2}>0\\
			\left({\vec V}_{31}\times{\vec V}_{32}\right)\cdot{\vec V}_{P3}>0\\
			\left({\vec V}_{41}\times{\vec V}_{42}\right)\cdot{\vec V}_{P4}>0
		\end{cases}
\end{eqnarray}

\item{\bf output.cpp}
In this procedure the converted data is output. 
\item{\bf read\_variable.cpp}
This procedure reads the file data output by the Lagrange code, after the Lagrangian data set selection. 
\end{enumerate}

%\include{end}
%\documentclass[preprint,12pt]{elsarticle}
%\if0
\usepackage{amssymb}
\usepackage{mathtools}
%\usepackage[dvipdfmx]{graphicx}
\usepackage{cite}
\usepackage{graphicx}
\usepackage{bm}
\usepackage{here}
\usepackage[subrefformat=parens]{subcaption}
\fi
%\usepackage{amssymb}
\usepackage{amsmath}
\usepackage[dvipdfmx]{}
\usepackage[dvipdfmx]{color}
%\usepackage{cite}
%\usepackage{upgreek}
\usepackage{url}
%\usepackage[dvipdfmx]{hyperref}
%\usepackage{pxjahyper}
%\usepackage {hyperref}
\usepackage{graphicx}
\usepackage{bm}
\usepackage{here}
\usepackage{caption}
\usepackage[subrefformat=parens]{subcaption}
\captionsetup{compatibility=false}

%% The amsthm package provides extended theorem environments
%% \usepackage{amsthm}

%% The lineno packages adds line numbers. Start line numbering with
%% \begin{linenumbers}, end it with \end{linenumbers}. Or switch it on
%% for the whole article with \linenumbers after \end{frontmatter}.
%% \usepackage{lineno}

%% natbib.sty is loaded by default. However, natbib options can be
%% provided with \biboptions{...} command. Following options are
%% valid:

%%   round  -  round parentheses are used (default)
%%   square -  square brackets are used   [option]
%%   curly  -  curly braces are used      {option}
%%   angle  -  angle brackets are used    <option>
%%   semicolon  -  multiple citations separated by semi-colon
%%   colon  - same as semicolon, an earlier confusion
%%   comma  -  separated by comma
%%   numbers-  selects numerical citations
%%   super  -  numerical citations as superscripts
%%   sort   -  sorts multiple citations according to order in ref. list
%%   sort&compress   -  like sort, but also compresses numerical citations
%%   compress - compresses without sorting
%%
%% \biboptions{comma,round}

% \biboptions{}

%% This list environment is used for the references in the
%% Program Summary
%%
\newcounter{bla}
\newenvironment{refnummer}{%
\list{[\arabic{bla}]}%
{\usecounter{bla}%
 \setlength{\itemindent}{0pt}%
 \setlength{\topsep}{0pt}%
 \setlength{\itemsep}{0pt}%
 \setlength{\labelsep}{2pt}%
 \setlength{\listparindent}{0pt}%
 \settowidth{\labelwidth}{[9]}%
 \setlength{\leftmargin}{\labelwidth}%
 \addtolength{\leftmargin}{\labelsep}%
 \setlength{\rightmargin}{0pt}}}
 {\endlist}
\begin{document}

%----- Euler Code
\subsection{Eulerian code}
\begin{enumerate}
\item {\bf BoundaryTracking.cpp}\\
It is a function to track the material boundary surfaces. Each boundary point is specified by the coordinates of the three variables: $( BoundaryMesh\_i$,  $BoundaryMesh\_j$, $BoundaryMesh\_k )$. The function interpolates the velocities $u$, $v$ and $w$ at the coordinates by the volume interpolation, and tracks the position of each boundary point. In Fig. \ref {BP} dotted lines represent the material boundaries. When the boundary point exists at the position shown in Fig. \ref {BMV}, the boundary point velocity ($u_b, \ v_b, \ w_b$) is calculated by the volume interpolation method and is obtained by the following equations:
	\begin{eqnarray}
		u_b&=&V_{u1}u_{i+1,j+1,k+1}+V_{u2}u_{i,j+1,k+1}+V_{u3}u_{i,j,k+1}+V_{u4}u_{i+1,j,k+1} \\ \nonumber 
		&+&V_{u5}u_{i+1,j+1,k}+V_{u6}u_{i,j+1,k}+V_{u7}u_{i,j,k}+V_{u8}u_{i+1,j,k}\\
		v_b&=&V_{v1}v_{i+1,j+1,k+1}+V_{v2}v_{i,j+1,k+1}+V_{v3}v_{i,j,k+1}+V_{v4}v_{i+1,j,k+1} \\ \nonumber
		&+&V_{v5}v_{i+1,j+1,k}+V_{v6}v_{i,j+1,k}+V_{v7}v_{i,j,k}+V_{v8}v_{i+1,j,k}\\
		w_b&=&V_{w1}w_{i+1,j+1,k+1}+V_{w2}w_{i,j+1,k+1}+V_{w3}w_{i,j,k+1}\\ \nonumber
	 &+&V_{w4}w_{i+1,j,k+1} +V_{w5}w_{i+1,j+1,k}+V_{w6}w_{i,j+1,k}+V_{w7}w_{i,j,k}\\ \nonumber
	 &+&V_{w8}w_{i+1,j,k}
	\end{eqnarray}

	\begin{figure}[H]
		\centering
		\includegraphics[height=7cm]{images/Boundarypoint.eps}
		\caption{Material boundary points.}\label{BP}
	\end{figure}
	\begin{figure}[H]
		\centering
		\includegraphics[width=10cm]{images/BoundMeshVeloc.eps}
		\caption{Velocity interpolation by the volume interpolation.}\label{BMV}
	\end{figure}
	
%\item {\bf DIFINITS.H}\\
\item {\bf GenerateMatrix.cpp}\\
The mesh total numbers of $(im,\ jm,\ km)$ are loaded from the converted data in GenerateMatrix(). all the variables required  in the Euler code are defined based on the number $(im,\ jm,\ km)$. 

%\item {\bf InitBoundary.cpp}\\
\if0
\item {\bf Legendre.cpp}\\
The procedure performs the mode analyses based on the Legendre function in order to find the implosion non-uniformity. The analysis results are also output in this procedure. 
\fi
\item {\bf MS\_TDMA.cpp}\\
A function to solve matrixes by TDMA (TriDiagonal-Matrix Algorithm).
\item {\bf MaterialRecognation.cpp}\\
A function to discriminate each material by the material boundary lines. 
\item {\bf PaintMaterial.cpp\\}
The material is specified between the two material boundary lines in the procedure. 
\item {\bf RMFP\_ECSH.cpp}\\
A procedure to calculate the Rosseland mean free path (see Ref. \cite{Zeldovich}). 
\item {\bf ScanLine.cpp}\\
A procedure that specifies the material on each Euler mesh. 
\item {\bf artv\_ECSH.cpp}\\
This file contains a procedure to calculate the artificial viscosity. The three-dimensional artificial viscosity is written as follows: 
\begin{eqnarray}
q_x = \rho C^2_Q \left( \frac{\partial u}{\partial i} \right)^2 + \rho C_L C_s |\frac{\partial u}{\partial i}| \\
q_y = \rho C^2_Q \left( \frac{\partial v}{\partial j} \right)^2 + \rho C_L C_s |\frac{\partial v}{\partial j}| \\
q_z = \rho C^2_Q \left( \frac{\partial w}{\partial k} \right)^2 + \rho C_L C_s |\frac{\partial w}{\partial k}| 
\end{eqnarray}
Here, the discretized artificial viscosities are shown below: 
%$ u $ is the direction of $ r $ direction, $ v $ is the speed of $ z $ direction. 
\small
\begin{eqnarray}
{q_x}^n_{i+\frac{1}{2},j+\frac{1}{2},k+\frac{1}{2}} &=& \rho^n_{i+\frac{1}{2},j+\frac{1}{2},k+\frac{1}{2}} C_Q^2 (u^n_{i+1,j+\frac{1}{2},k+\frac{1}{2}} - u^n_{i,j+\frac{1}{2},k+\frac{1}{2}})^2 \\\nonumber
&& + \rho^n_{i+\frac{1}{2},j+\frac{1}{2},k+\frac{1}{2}} C_L C_s \left| u^n_{i+1,j+\frac{1}{2},k+\frac{1}{2}}-u^n_{i,j+\frac{1}{2},k+\frac{1}{2}} \right| \\
{q_y}^n_{i+\frac{1}{2},j+\frac{1}{2},k+\frac{1}{2}} &=& \rho^n_{i+\frac{1}{2},j+\frac{1}{2},k+\frac{1}{2}} C_Q^2 (u^n_{i+\frac{1}{2},j+1,k+\frac{1}{2}} - u^n_{i+\frac{1}{2},j,k+\frac{1}{2}})^2 \\ \nonumber
&& + \rho^n_{i+\frac{1}{2},j+\frac{1}{2},k+\frac{1}{2}} C_L C_s \left| u^n_{i+\frac{1}{2},j+1,k+\frac{1}{2}}-u^n_{i+\frac{1}{2},j,k+\frac{1}{2}} \right| \\
{q_z}^n_{i+\frac{1}{2},j+\frac{1}{2},k+\frac{1}{2}} &=& \rho^n_{i+\frac{1}{2},j+\frac{1}{2},k+\frac{1}{2}} C_Q^2 (u^n_{i+\frac{1}{2},j+\frac{1}{2},k+1} - u^n_{i+\frac{1}{2},j+\frac{1}{2},k})^2 \\ \nonumber
&& + \rho^n_{i+\frac{1}{2},j+\frac{1}{2},k+\frac{1}{2}} C_L C_s \left| u^n_{i+\frac{1}{2},j+\frac{1}{2},k+1}-u^n_{i+\frac{1}{2},j+\frac{1}{2},k} \right| \\
\end{eqnarray}
\normalsize

\item {\bf define\_ECSH.h}\\
It contains the constant values, the normalization factors and the procedure declarations required. 
\item {\bf dif\_ECSH.cpp}\\
The following equations of motion are used.
\begin{eqnarray}
\label{dmeEu}
\frac{\partial \tilde{u}}{\partial \tilde{t}} &=& -  \left( \tilde{u} \frac{\partial \tilde{u}}{\partial \tilde{x}} + \tilde{v} \frac{\partial \tilde{u}}{\partial \tilde{y}} +\tilde{w} \frac{\partial \tilde{u}}{\partial \tilde{z}} \right) -  \frac{1}{\tilde{\rho}}  \frac{\partial \left(\tilde{p} + \tilde{q}\right)}{\partial \tilde{x}}   \\
\frac{\partial \tilde{v}}{\partial \tilde{t}} &=& -  \left( \tilde{u} \frac{\partial \tilde{v}}{\partial \tilde{x}} + \tilde{v} \frac{\partial \tilde{v}}{\partial \tilde{y}} +\tilde{w} \frac{\partial \tilde{v}}{\partial \tilde{z}} \right) -  \frac{1}{\tilde{\rho}} \frac{\partial \left(\tilde{p} + \tilde{q}\right)}{\partial \tilde{y}}   \\
\frac{\partial \tilde{w}}{\partial \tilde{t}} &=& -  \left( \tilde{u} \frac{\partial \tilde{w}}{\partial \tilde{x}} + \tilde{v} \frac{\partial \tilde{w}}{\partial \tilde{y}} +\tilde{w} \frac{\partial \tilde{w}}{\partial \tilde{z}} \right) -  \frac{1}{\tilde{\rho}} \ \frac{\partial \left(\tilde{p} +  \tilde{q}\right)}{\partial \tilde{z}}   
\end{eqnarray}

Equations (\ref{dmeEu}) are discretized as follows: 

\begin{eqnarray}
u^{n+\frac{1}{2}}_{i,j+\frac{1}{2},k+\frac{1}{2}} &=& u^{n-\frac{1}{2}}_{i,j+\frac{1}{2},k+\frac{1}{2}} - Dt^n \Biggl[ \left\{ \left(u \frac{\partial u}{\partial x} \right)^{n-\frac{1}{2}}_{i,j+\frac{1}{2},k+\frac{1}{2}} + \left(v \frac{\partial u}{\partial y} \right)^{n-\frac{1}{2}}_{i,j+\frac{1}{2},k+\frac{1}{2}} \right. \\  \nonumber
&& \left.  + \left(w \frac{\partial u}{\partial z} \right)^{n-\frac{1}{2}}_{i,j+\frac{1}{2},k+\frac{1}{2}} \right\} + \left\{ \frac{1}{\rho} \frac{\partial (p+q)}{\partial x} \right\}^{n-\frac{1}{2}}_{i,j+\frac{1}{2},k+\frac{1}{2}} \Biggr] 
\end{eqnarray}
Here, 
\begin{eqnarray*}
&& u \frac{\partial u}{\partial x} \bigg|^{n-\frac{1}{2}}_{i,j+\frac{1}{2},k+\frac{1}{2}} \\
&& \ \ \ \ \ \  = \left\{ 
\begin{array}{ll}
u^{n-\frac{1}{2}}_{i,j+\frac{1}{2},k+\frac{1}{2}} \frac{u^{n-\frac{1}{2}}_{i,j+\frac{1}{2},k+\frac{1}{2}} - u^{n-\frac{1}{2}}_{i-1,j+\frac{1}{2},k+\frac{1}{2}}}{Dx^{n-\frac{1}{2}}_{i-\frac{1}{2},j+\frac{1}{2},k+\frac{1}{2}}} &\left( u^{n-\frac{1}{2}}_{i,j+\frac{1}{2},k+\frac{1}{2}} \geq 0 \right) \\
	u^{n-\frac{1}{2}}_{i,j+\frac{1}{2},k+\frac{1}{2}} \frac{u^{n-\frac{1}{2}}_{i+1,j+\frac{1}{2},k+\frac{1}{2}} - u^{n-\frac{1}{2}}_{i+1,j+\frac{1}{2},k+\frac{1}{2}}}{Dx^{n-\frac{1}{2}}_{i-\frac{1}{2},j+\frac{1}{2},k+\frac{1}{2}}} &\left( u^{n-\frac{1}{2}}_{i,j+\frac{1}{2},k+\frac{1}{2}} <0 \right)
\end{array} 
\right. \\
&& v \frac{\partial u}{\partial y} \bigg|^{n-\frac{1}{2}}_{i,j+\frac{1}{2},k+\frac{1}{2}} \\
&& \ \ \ \ \ \  = \left\{ 
\begin{array}{ll}
	v^{n-\frac{1}{2}}_{i,j+\frac{1}{2},k+\frac{1}{2}} \frac{u^{n-\frac{1}{2}}_{i,j+\frac{1}{2},k+\frac{1}{2}} - u^{n-\frac{1}{2}}_{i,j-\frac{1}{2},k+\frac{1}{2}}}{Dy^{n-\frac{1}{2}}_{i-\frac{1}{2},j+\frac{1}{2},k+\frac{1}{2}}} &\left( v^{n-\frac{1}{2}}_{i,j+\frac{1}{2},k+\frac{1}{2}} \geq 0 \right) \\
	v^{n-\frac{1}{2}}_{i,j+\frac{1}{2},k+\frac{1}{2}} \frac{u^{n-\frac{1}{2}}_{i,j+\frac{3}{2},k+\frac{1}{2}} + u^{n-\frac{1}{2}}_{i,j+\frac{1}{2},k+\frac{1}{2}}}{Dy^{n-\frac{1}{2}}_{i-\frac{1}{2},j+\frac{1}{2},k+\frac{1}{2}}} &\left( v^{n-\frac{1}{2}}_{i,j+\frac{1}{2},k+\frac{1}{2}} < 0 \right) 
\end{array} 
\right. \\
&&  w \frac{\partial u}{\partial z} \bigg|^{n-\frac{1}{2}}_{i,j+\frac{1}{2},k+\frac{1}{2}} \\
&& \ \ \ \ \ \  = \left\{ 
\begin{array}{ll}
	w^{n-\frac{1}{2}}_{i,j+\frac{1}{2},k+\frac{1}{2}} \frac{u^{n-\frac{1}{2}}_{i,j+\frac{1}{2},k+\frac{1}{2}} - u^{n-\frac{1}{2}}_{i,j+\frac{1}{2},k-\frac{1}{2}}}{Dz^{n-\frac{1}{2}}_{i-\frac{1}{2},j+\frac{1}{2},k+\frac{1}{2}}} &\left( w^{n-\frac{1}{2}}_{i,j+\frac{1}{2},k+\frac{1}{2}} \geq 0 \right) \\
	w^{n-\frac{1}{2}}_{i,j+\frac{1}{2},k+\frac{1}{2}} \frac{u^{n-\frac{1}{2}}_{i,j+\frac{1}{2},k+\frac{3}{2}} - u^{n-\frac{1}{2}}_{i,j+\frac{1}{2},k+\frac{1}{2}}}{Dz^{n^\frac{1}{2}}_{i-\frac{1}{2},j+\frac{1}{2},k+\frac{1}{2}}} &\left( w^{n-\frac{1}{2}}_{i,j+\frac{1}{2},k+\frac{1}{2}} < 0 \right) 
\end{array} 
\right. 
\end{eqnarray*}

\begin{eqnarray*}
 &&\frac{1}{\rho} \frac{\partial(p+q)}{\partial x}\bigg|^{n-\frac{1}{2}}_{i,j+\frac{1}{2},k+\frac{1}{2}} = \frac{2}{\rho^{n-\frac{1}{2}}_{i+\frac{1}{2},j+\frac{1}{2},k+\frac{1}{2}} + \rho^{n-\frac{1}{2}}_{i-\frac{1}{2},j+\frac{1}{2},k+\frac{1}{2}}}  \\
&& \left\{ \frac{p^{n-\frac{1}{2}}_{i+\frac{1}{2},j+\frac{1}{2},k+\frac{1}{2}} + q^{n-\frac{1}{2}}_{i+\frac{1}{2},j+\frac{1}{2},k+\frac{1}{2}} - \left(p^{n-\frac{1}{2}}_{i-\frac{1}{2},j+\frac{1}{2},k+\frac{1}{2}} + q^{n-\frac{1}{2}}_{i-\frac{1}{2},j+\frac{1}{2},k+\frac{1}{2}} \right) }{Dx^{n-\frac{1}{2}}_{i,j+\frac{1}{2},k+\frac{1}{2}}} \right\}
\end{eqnarray*}


\item {\bf eod\_ECSH.cpp}\\
The following continuity equation is used.
\begin{equation}
\label{eodE}
\frac{\partial \rho}{\partial t} = - \rho \left( \frac{\partial u}{\partial x} +\frac{\partial v}{\partial y} + \frac{\partial w}{\partial z} \right) - \left( u \frac{\partial \rho}{\partial x} + v \frac{\partial \rho}{\partial y} + w \frac{\partial \rho}{\partial z}\right)
\end{equation}
Equation (\ref{eodE}) is discretized as follows: 
\begin{eqnarray}
\rho^{n+1}_{i+\frac{1}{2},j+\frac{1}{2},k+\frac{1}{2}} &=& \rho^{n}_{i+\frac{1}{2},j+\frac{1}{2},k+\frac{1}{2}} - Dt^n \Biggl[ \rho^n_{i+\frac{1}{2}, j+\frac{1}{2}.k+\frac{1}{2}} \left\{ \frac{\partial u}{\partial x} \bigg|^n_{i+\frac{1}{2},j+\frac{1}{2},k+\frac{1}{2}} \right. \\ \nonumber
&& \left. + \frac{\partial v}{\partial y} \bigg|^n_{i+\frac{1}{2},j+\frac{1}{2},k+\frac{1}{2}}+ \frac{\partial w}{\partial z} \bigg|^n_{i+\frac{1}{2},j+\frac{1}{2},k+\frac{1}{2}} \right\} + \left\{  u \frac{\partial \rho}{\partial x} \bigg|^n_{i+\frac{1}{2},j+\frac{1}{2},k+\frac{1}{2}} \right. \\ \nonumber
&& \left. + v \frac{\partial \rho}{\partial y} \bigg|^n_{i+\frac{1}{2},j+\frac{1}{2},k+\frac{1}{2}} + w \frac{\partial \rho}{\partial z} \bigg|^n_{i+\frac{1}{2},j+\frac{1}{2},k+\frac{1}{2}} \right\} \Biggr] 
\end{eqnarray}
\begin{eqnarray*}
&&  \frac{\partial u}{\partial x} \bigg|^n_{i+\frac{1}{2},j+\frac{1}{2},k+\frac{1}{2}} = \frac{ u^n_{i+1,j+\frac{1}{2},k+\frac{1}{2}} - u^n_{i,j+\frac{1}{2},k+\frac{1}{2}}}{D x^n_{i+\frac{1}{2},j+\frac{1}{2},k+\frac{1}{2}}} \\
&& \frac{\partial v}{\partial y} \bigg|^n_{i+\frac{1}{2},j+\frac{1}{2},k+\frac{1}{2}} = \frac{ v^n_{i+\frac{1}{2},j+1,k+\frac{1}{2}} - v^n_{i+\frac{1}{2},j,k+\frac{1}{2}}}{D y^n_{i+\frac{1}{2},j+\frac{1}{2},k+\frac{1}{2}}} \\
&& \frac{\partial w}{\partial z} \bigg|^n_{i+\frac{1}{2},j+\frac{1}{2},k+\frac{1}{2}} = \frac{ w^n_{i+\frac{1}{2},j+\frac{1}{2},k+1} - w^n_{i+\frac{1}{2},j+\frac{1}{2},k}}{D z^n_{i+\frac{1}{2},j+\frac{1}{2},k+\frac{1}{2}}} \\
&& u \frac{\partial \rho}{\partial x} \bigg|^n_{i+\frac{1}{2},j+\frac{1}{2},k+\frac{1}{2}} \\
&& \ \ \ \  = \left\{ 
\begin{array}{ll}
	u^n_{i+\frac{1}{2},j+\frac{1}{2},k+\frac{1}{2}} \frac{\rho^n_{i+\frac{1}{2},j+\frac{1}{2},k+\frac{1}{2}} - \rho^n_{i-\frac{1}{2},j+\frac{1}{2},k+\frac{1}{2}}}{Dx^n_{i,j+\frac{1}{2},k+\frac{1}{2}}} &\left( u^n_{i+\frac{1}{2},j+\frac{1}{2},k+\frac{1}{2}} \geq 0 \right) \\
	u^n_{i+\frac{1}{2},j+\frac{1}{2},k+\frac{1}{2}} \frac{\rho^n_{i+\frac{3}{2},j+\frac{1}{2},k+\frac{1}{2}} - \rho^n_{i+\frac{1}{2},j+\frac{1}{2},k+\frac{1}{2}}}{Dx^n_{i+1, j+\frac{1}{2},k+\frac{1}{2}}} &\left( u^n_{i+\frac{1}{2},j+\frac{1}{2},k+\frac{1}{2}} <0 \right)
\end{array} 
\right. \\
&& v \frac{\partial \rho}{\partial y} \bigg|^n_{i+\frac{1}{2},j+\frac{1}{2},k+\frac{1}{2}} \\
&& \ \ \ \  = \left\{ 
\begin{array}{ll}
	v^n_{i+\frac{1}{2},j+\frac{1}{2},k+\frac{1}{2}} \frac{\rho^n_{i+\frac{1}{2},j+\frac{1}{2},k+\frac{1}{2}} - \rho^n_{i+\frac{1}{2},j-\frac{1}{2},k+\frac{1}{2}}}{Dy^n_{i+\frac{1}{2},j,k+\frac{1}{2}}} &\left( v^n_{i+\frac{1}{2},j+\frac{1}{2},k+\frac{1}{2}} \geq 0 \right) \\
	v^n_{i+\frac{1}{2},j+\frac{1}{2},k+\frac{1}{2}} \frac{\rho^n_{i+\frac{3}{2},j+\frac{1}{2},k+\frac{1}{2}} - \rho^n_{i+\frac{1}{2},j+\frac{1}{2},k+\frac{1}{2}}}{Dy^n_{i+\frac{1}{2}, j+1,k+\frac{1}{2}}} &\left( v^n_{i+\frac{1}{2},j+\frac{1}{2},k+\frac{1}{2}} <0 \right)
\end{array} 
\right. \\
&& w \frac{\partial \rho}{\partial z} \bigg|^n_{i+\frac{1}{2},j+\frac{1}{2},k+\frac{1}{2}} \\
&& \ \ \ \  = \left\{ 
\begin{array}{ll}
	w^n_{i+\frac{1}{2},j+\frac{1}{2},k+\frac{1}{2}} \frac{\rho^n_{i+\frac{1}{2},j+\frac{1}{2},k+\frac{1}{2}} - \rho^n_{i+\frac{1}{2},j+\frac{1}{2},k-\frac{1}{2}}}{Dz^n_{i+\frac{1}{2},j+\frac{1}{2},k}} &\left( w^n_{i+\frac{1}{2},j+\frac{1}{2},k+\frac{1}{2}} \geq 0 \right) \\
	w^n_{i+\frac{1}{2},j+\frac{1}{2},k+\frac{1}{2}} \frac{\rho^n_{i+\frac{1}{2},j+\frac{1}{2},k+\frac{3}{2}} - \rho^n_{i+\frac{1}{2},j+\frac{1}{2},k+\frac{1}{2}}}{Dz^n_{i+\frac{1}{2},j+\frac{1}{2},k+1}} &\left( w^n_{i+\frac{1}{2},j+\frac{1}{2},k+\frac{1}{2}} < 0 \right) \\
\end{array} 
\right.
\end{eqnarray*}
\item{\bf eoenergy\_ECSH}\\
The following basic energy equations are used.
\begin{eqnarray}
	\label{eq:EuEoE}
\frac{\partial T_i}{\partial t} &=& - \left( u \cdot \nabla \right) T_i -\frac{k_B}{C_{V_i}} \Biggl[ \left( \rho B_{T_i} + \frac{p_i +q}{\rho} \right) \left( \nabla \cdot u \right)  \Biggr] \\
\frac{\partial T_e}{\partial t} &=& - \left( u \cdot \nabla \right) T_e -\frac{k_B}{C_{V_e}} \Biggl[ \left( \rho B_{T_e} + \frac{p_e}{\rho} \right) \left( \nabla \cdot u \right)  \Biggr] \\
\frac{\partial T_r}{\partial t} &=& - \left( u \cdot \nabla \right) T_r -\frac{k_B}{C_{V_r}} \Biggl[ \left( \rho B_{T_r} + \frac{p_r}{\rho} \right) \left( \nabla \cdot u \right)  \Biggr]
\end{eqnarray}
Here, $B_{T_i} = 0$ in HIF. The discretized energy equation for the ion temperature, for example, becomes as follows: 
\begin{eqnarray}
{T_i}^{n+1}_{i+\frac{1}{2},j+\frac{1}{2},k+\frac{1}{2}} &=& {T_i}^n_{i+\frac{1}{2},j+\frac{1}{2},k+\frac{1}{2}} - Dt^n \Biggl[ \left\{  u \frac{\partial T_i}{\partial x} \bigg|^n_{i+\frac{1}{2},j+\frac{1}{2},k+\frac{1}{2}} \right. \\ \nonumber
&& \left. + v \frac{\partial T_i}{\partial y} \bigg|^n_{i+\frac{1}{2},j+\frac{1}{2},k+\frac{1}{2}} + w \frac{\partial T_i}{\partial z} \bigg|^n_{i+\frac{1}{2},j+\frac{1}{2},k+\frac{1}{2}} \right\} \\ \nonumber
&& + \frac{1}{{C_{V_i}}^n_{i+\frac{1}{2},j+\frac{1}{2},k+\frac{1}{2}}} \Biggl[ \frac{{p_i}^n_{i+\frac{1}{2},j+\frac{1}{2},k+\frac{1}{2}} + q^n_{i+\frac{1}{2},j+\frac{1}{2},k+\frac{1}{2}}}{\rho^n_{i+\frac{1}{2},j+\frac{1}{2}+k\frac{1}{2}}}  \\ \nonumber
&& \left\{ \frac{\partial u}{\partial x} \bigg|^n_{i+\frac{1}{2},j+\frac{1}{2},k+\frac{1}{2}} + \frac{\partial v}{\partial y} \bigg|^n_{i+\frac{1}{2},j+\frac{1}{2},k+\frac{1}{2}}  + \frac{\partial w}{\partial z} \bigg|^n_{i+\frac{1}{2},j+\frac{1}{2},k+\frac{1}{2}} \right\} \Biggr] \Biggr]
\end{eqnarray}
\begin{eqnarray*}
&& u \frac{\partial {T_i}}{\partial x} \bigg|^n_{i+\frac{1}{2},j+\frac{1}{2},k+\frac{1}{2}} \\
&& \ \ \ \  = \left\{ 
\begin{array}{ll}
	u^n_{i+\frac{1}{2},j+\frac{1}{2},k+\frac{1}{2}} \frac{{T_i}^n_{i+\frac{1}{2},j+\frac{1}{2},k+\frac{1}{2}} -  {T_i}^n_{i-\frac{1}{2},j+\frac{1}{2},k+\frac{1}{2}}}{Dx^n_{i,j+\frac{1}{2},k+\frac{1}{2}}} &\left( u^n_{i+\frac{1}{2},j+\frac{1}{2},k+\frac{1}{2}} \geq 0 \right) \\
	u^n_{i+\frac{1}{2},j+\frac{1}{2},k+\frac{1}{2}} \frac{{T_i}^n_{i+\frac{3}{2},j+\frac{1}{2},k+\frac{1}{2}} - {T_i}^n_{i+\frac{1}{2},j+\frac{1}{2},k+\frac{1}{2}}}{Dx^n_{i+1, j+\frac{1}{2},k+\frac{1}{2}}} &\left( u^n_{i+\frac{1}{2},j+\frac{1}{2},k+\frac{1}{2}} <0 \right)
\end{array} 
\right. \\
&& v \frac{\partial {T_i}}{\partial y} \bigg|^n_{i+\frac{1}{2},j+\frac{1}{2},k+\frac{1}{2}} \\
&& \ \ \ \  = \left\{ 
\begin{array}{ll}
	v^n_{i+\frac{1}{2},j+\frac{1}{2},k+\frac{1}{2}} \frac{{T_i}^n_{i+\frac{1}{2},j+\frac{1}{2},k+\frac{1}{2}} - {T_i}^n_{i+\frac{1}{2},j-\frac{1}{2},k+\frac{1}{2}}}{Dy^n_{i\frac{1}{2},j,k+\frac{1}{2}}} &\left( v^n_{i+\frac{1}{2},j+\frac{1}{2},k+\frac{1}{2}} \geq 0 \right) \\
	v^n_{i+\frac{1}{2},j+\frac{1}{2},k+\frac{1}{2}} \frac{{T_i}^n_{i+\frac{1}{2},j+\frac{3}{2},k+\frac{1}{2}} - {T_i}^n_{i+\frac{1}{2},j+\frac{1}{2},k+\frac{1}{2}}}{Dy^n_{i+\frac{1}{2}, j+1,k+\frac{1}{2}}} &\left( v^n_{i+\frac{1}{2},j+\frac{1}{2},k+\frac{1}{2}} <0 \right)
\end{array} 
\right. \\
&& w \frac{\partial {T_i}}{\partial z} \bigg|^n_{i+\frac{1}{2},j+\frac{1}{2},k+\frac{1}{2}} \\
&& \ \ \ \  = \left\{ 
\begin{array}{ll}
	w^n_{i+\frac{1}{2},j+\frac{1}{2},k+\frac{1}{2}} \frac{{T_i}^n_{i+\frac{1}{2},j+\frac{1}{2},k+\frac{1}{2}} - {T_i}^n_{i+\frac{1}{2},j+\frac{1}{2},k-\frac{1}{2}}}{Dz^n_{i\frac{1}{2},j+\frac{1}{2},k}} &\left( w^n_{i+\frac{1}{2},j+\frac{1}{2},k+\frac{1}{2}} \geq 0 \right) \\
	w^n_{i+\frac{1}{2},j+\frac{1}{2},k+\frac{1}{2}} \frac{{T_i}^n_{i+\frac{1}{2},j+\frac{1}{2},k+\frac{3}{2}} - {T_i}^n_{i+\frac{1}{2},j+\frac{1}{2},k+\frac{1}{2}}}{Dz^n_{i+\frac{1}{2}, j+\frac{1}{2},k+1}} &\left( w^n_{i+\frac{1}{2},j+\frac{1}{2},k+\frac{1}{2}} <0 \right)
\end{array} 
\right. \\
&& \frac{\partial u}{\partial x} \bigg|^n_{i+\frac{1}{2},j+\frac{1}{2},k+\frac{1}{2}} = \frac{u^n_{i+1,j+\frac{1}{2},k+\frac{1}{2}} - u^n_{i,j+\frac{1}{2},k+\frac{1}{2}}}{Dx^n_{i,j+\frac{1}{2},k+\frac{1}{2}}} \\
&& \frac{\partial v}{\partial y} \bigg|^n_{i+\frac{1}{2},j+\frac{1}{2},k+\frac{1}{2}} = \frac{v^n_{i+\frac{1}{2},j+1,k+\frac{1}{2}} - v^n_{i+\frac{1}{2},j,k+\frac{1}{2}}}{Dy^n_{i+\frac{1}{2},j,k+\frac{1}{2}}} \\
&& \frac{\partial w}{\partial z} \bigg|^n_{i+\frac{1}{2},j+\frac{1}{2},k+\frac{1}{2}} = \frac{w^n_{i+\frac{1}{2},j+\frac{1}{2},k+1} - w^n_{i+\frac{1}{2},j+\frac{1}{2},k}}{Dz^n_{i+\frac{1}{2},j+\frac{1}{2},k}}
\end{eqnarray*}
	
\item{\bf eos\_ECSH.cpp}\\
The same equation is used as the equation of state in the Lagrangian code.
\item{\bf fusion.cpp}\\
The fusion reactions are calculated in this procedure. The details are shown in Ref. \cite{CPC-O-SUKI}. The fusion reaction formulae for deuterium and tritium are shown below.
	\begin{eqnarray}
	\begin{split}
		\rm D+\rm D&\xrightarrow[50\%]{}\rm T(1.01{\rm MeV})+\rm p(3.02{\rm MeV})\\
		&\xrightarrow[50\%]{}{\rm He}^3(0.82{\rm MeV})+\rm n(2.45{\rm MeV})\\
		\rm D+\rm T&\rightarrow{\rm He}^4(3.5{\rm MeV})+\rm n(14.1{\rm MeV})
		\end{split}
		\label{NucEq}
	\end{eqnarray}
	D decreases due to the DD and DT reactions from the expression (\ref {NucEq}). The number density $n_{\rm D} $ change  is given bellow: 	
	\begin{eqnarray}
	\label{eq:ReacD}
		\frac{\partial n_{\rm D}}{\partial t}&=&-N_{\rm DD}-N_{\rm DT}\nonumber\\
							 &=&-\frac{1}{2}\langle\sigma v\rangle_{\rm DD}n_{\rm D}n_{\rm D}-\langle\sigma v\rangle_{\rm DT}n_{\rm D}n_{\rm T}
	\end{eqnarray}

Considering the diffusion term of $ \alpha $ particles and the term of $ \alpha $ particle absorption, $ n_\alpha $ is described as follows: 
		\begin{equation}
	\label{eq:alphaReac}
		\frac{\partial n_\alpha}{\partial t}=+\langle\sigma v\rangle_{\rm DT}n_{\rm D}n_{\rm T}-\bm\nabla\cdot\bm F-\omega_\alpha n_\alpha
	\end{equation}
	The discretized $\alpha$ particle reaction is written as: 
	\begin{equation}
		{n_\alpha}^{n+1}_{i+\frac{1}{2},j+\frac{1}{2},k+\frac{1}{2}}={n_\alpha}^{n}_{i+\frac{1}{2},j+\frac{1}{2},k+\frac{1}{2}}+\ \Delta t{n_{\rm D}}^{n}_{i+\frac{1}{2},j+\frac{1}{2},k+\frac{1}{2}}{n_{\rm T}}^{n}_{i+\frac{1}{2},j+\frac{1}{2},k+\frac{1}{2}}\langle\sigma v{\rangle_{\rm DT}}^{n}_{i+\frac{1}{2},j+\frac{1}{2},k+\frac{1}{2}}.
	\end{equation}
The D-D and the D-T reaction rates are shown in Refs. (\cite {CPC-O-SUKI, NRLpf}).	
The flux of the $\alpha$ particle is shown below $\mbox{\boldmath $F$}$.
\begin{eqnarray}
\mbox{\boldmath $F$}  = - D_\alpha \mbox{\boldmath $\nabla$} n_\alpha
\end{eqnarray}
Here $D_\alpha$ is the diffusion coefficient and is expressed by the following equation.
\begin{eqnarray}
\label{eq:alpha_kakusan}
D_\alpha = \frac{ \frac{1}{3} v_\alpha \lambda_\alpha}{1+ \frac{4}{3} \lambda_\alpha \frac{|\nabla n_\alpha|}{n_\alpha}}
\end{eqnarray}
Here $v_\alpha$ is the speed of $\alpha$ particle and $\lambda_\alpha$ the mean free path of $\alpha$. The second term of the denominator in Eq. (\ref{eq:alpha_kakusan}) expresses the flux limiting effect, which limits the excess flux by the steep gradient of the $\alpha$ density. The flux $\mbox{\boldmath $F$}$ of the $\alpha$ particles in the $x$, $y$ and $z$ directions are expressed by the following equations:  
\begin{eqnarray}
F_x = - \frac{ \frac{1}{3} n_\alpha v_\alpha \lambda_\alpha }{n_\alpha + \frac{4}{3} \lambda_\alpha \left| \frac{\partial n_\alpha}{\partial x} \right| } \frac{\partial n_\alpha}{\partial x} \\
F_y = - \frac{ \frac{1}{3} n_\alpha v_\alpha \lambda_\alpha }{n_\alpha + \frac{4}{3} \lambda_\alpha \left| \frac{\partial n_\alpha}{\partial y} \right| } \frac{\partial n_\alpha}{\partial y} \\
F_z = - \frac{ \frac{1}{3} n_\alpha v_\alpha \lambda_\alpha }{n_\alpha + \frac{4}{3} \lambda_\alpha \left| \frac{\partial n_\alpha}{\partial z} \right| } \frac{\partial n_\alpha}{\partial z} 
\end{eqnarray}
	
The energy increases by the $ \alpha $ particle energy deposition are shown below: 
	\begin{eqnarray}
	\label{NucEnIon}
		\Delta T_i=\frac{E_\alpha n_\alpha f_i}{\rho C_{v_i}}\\
	\label{NucEnEle}
		\Delta T_e=\frac{E_\alpha n_\alpha f_e}{\rho C_{v_e}}
	\end{eqnarray}
Here $f$ represents the distribution factor of the $\alpha $ particle energy among ions and electrons \cite{Fraley}. 

\begin{eqnarray}
f_i = \frac{1}{1+\frac{32}{T_e(KeV)}}, \hspace{1cm} f_e= 1-f_i
\end{eqnarray}

The discretized energy increases for ions and electrons are described as follows.

	\begin{eqnarray}
		{T_i}^{n+1}_{i+\frac{1}{2},j+\frac{1}{2},k+\frac{1}{2}}={T_i}^{n}_{i+\frac{1}{2},j+\frac{1}{2},k+\frac{1}{2}}+\Delta t\frac{E_\alpha {n_\alpha}^n_{i+\frac{1}{2},j+\frac{1}{2},k+\frac{1}{2}}{f_i}^n_{i+\frac{1}{2},j+\frac{1}{2},k+\frac{1}{2}}}{\rho^n_{i+\frac{1}{2},j+\frac{1}{2},k+\frac{1}{2}}{C_{v_i}}^n_{i+\frac{1}{2},j+\frac{1}{2},k+\frac{1}{2}}}\\
		{T_e}^{n+1}_{i+\frac{1}{2},j+\frac{1}{2},k+\frac{1}{2}}={T_e}^{n}_{i+\frac{1}{2},j+\frac{1}{2},k+\frac{1}{2}}+\Delta t\frac{E_\alpha {n_\alpha}^n_{i+\frac{1}{2},j+\frac{1}{2},k+\frac{1}{2}}{f_e}^n_{i+\frac{1}{2},j+\frac{1}{2},k+\frac{1}{2}}}{\rho^n_{i+\frac{1}{2},j+\frac{1}{2},k+\frac{1}{2}}{C_{v_e}}^n_{i+\frac{1}{2},j+\frac{1}{2},k+\frac{1}{2}}}
	\end{eqnarray} 
\item{\bf init\_ECSH.cpp}\\
The file initializes the Eulerian code.
\item{\bf load\_convert.cpp}\\
A procedure to read the converted data.
\item{\bf main\_ECSH.cpp}\\
The main function of the Eulerian code.
\item {\bf output\_ECSH.cpp}\\
The results are stored in this procedure.

\end{enumerate}
%\include{end}
%\documentclass[preprint,12pt]{elsarticle}
%\if0
\usepackage{amssymb}
\usepackage{mathtools}
%\usepackage[dvipdfmx]{graphicx}
\usepackage{cite}
\usepackage{graphicx}
\usepackage{bm}
\usepackage{here}
\usepackage[subrefformat=parens]{subcaption}
\fi
%\usepackage{amssymb}
\usepackage{amsmath}
\usepackage[dvipdfmx]{}
\usepackage[dvipdfmx]{color}
%\usepackage{cite}
%\usepackage{upgreek}
\usepackage{url}
%\usepackage[dvipdfmx]{hyperref}
%\usepackage{pxjahyper}
%\usepackage {hyperref}
\usepackage{graphicx}
\usepackage{bm}
\usepackage{here}
\usepackage{caption}
\usepackage[subrefformat=parens]{subcaption}
\captionsetup{compatibility=false}

%% The amsthm package provides extended theorem environments
%% \usepackage{amsthm}

%% The lineno packages adds line numbers. Start line numbering with
%% \begin{linenumbers}, end it with \end{linenumbers}. Or switch it on
%% for the whole article with \linenumbers after \end{frontmatter}.
%% \usepackage{lineno}

%% natbib.sty is loaded by default. However, natbib options can be
%% provided with \biboptions{...} command. Following options are
%% valid:

%%   round  -  round parentheses are used (default)
%%   square -  square brackets are used   [option]
%%   curly  -  curly braces are used      {option}
%%   angle  -  angle brackets are used    <option>
%%   semicolon  -  multiple citations separated by semi-colon
%%   colon  - same as semicolon, an earlier confusion
%%   comma  -  separated by comma
%%   numbers-  selects numerical citations
%%   super  -  numerical citations as superscripts
%%   sort   -  sorts multiple citations according to order in ref. list
%%   sort&compress   -  like sort, but also compresses numerical citations
%%   compress - compresses without sorting
%%
%% \biboptions{comma,round}

% \biboptions{}

%% This list environment is used for the references in the
%% Program Summary
%%
\newcounter{bla}
\newenvironment{refnummer}{%
\list{[\arabic{bla}]}%
{\usecounter{bla}%
 \setlength{\itemindent}{0pt}%
 \setlength{\topsep}{0pt}%
 \setlength{\itemsep}{0pt}%
 \setlength{\labelsep}{2pt}%
 \setlength{\listparindent}{0pt}%
 \settowidth{\labelwidth}{[9]}%
 \setlength{\leftmargin}{\labelwidth}%
 \addtolength{\leftmargin}{\labelsep}%
 \setlength{\rightmargin}{0pt}}}
 {\endlist}
\begin{document}

\section{Shell script files for setup and postprocessing}
%
A Shell file is prepared for the integrated run throughout from the Lagrange, conversion and Euler codes. However, each code can be also run manually one by one.
%
After finishing all the simulation process, users may need to visualize the simulation data. Some of the data computed are visualized by the following shell scripts. All shell files require gnuplot 4.6 or later.

%
\subsection{Calculation set up shell}
\begin{enumerate}
\item{\bf setup\_fusion.h}
The shell file remove the calculation output date and makes the output file.
%\item{\bf remove\_all.sh}
%The shell file remove the calculation output date.
\end{enumerate}
%

\subsection{Visualization for the Lagrange code data}
All the visualized data images are stored in the "pic\_La" directory. 
\begin{enumerate}
\item{\bf adiabat.sh}\\
The visualized graph for the time history of the adiabat $\alpha$ calculated in "Insulation.cpp" in the Lagrangian code. 

\item{\bf Animation\_Ti\_MODE.sh}\\
The shell file visualizes the mode analysis results of the ion temperature calculated by "Legendre.cpp" in the Lagrangian code. 

\item{\bf ImplosionVelocity.sh}\\
The shell plots the time histories of the implosion speed averaged over the azimuthal direction for the DT inner surface, the DT outer surface and the averaged DT speed. 
\item{\bf RMSoutput.sh}\\
The shell file plots the time histories of the root-mean-square (RMS) for the ion temperature and the mass density in the DT layer and Al layer. The RMS data is calculated by "RMS.cpp" in the Lagrangian code. 
\item{\bf SLC\_t\_r.sh}\\
The shell file outputs the images of the $r-t$ diagrams representing the time history of the Lagrangian meshes at $\theta =$30, 60, 120 and 150 degrees and at $\phi =$10, 100, 190 and 280 degrees. To execute the shell file, users need to specify the boundary mesh number of each material in the Lagrangian code. 
\end{enumerate}

\subsection{Visualization for the Euler code data}
All visualized data files are stored in the "pic\_Eu" directory. 
\begin{enumerate}
\item{\bf Animation\_atomic\_XY.sh}\\
The shell file visualizes the distributions of the atomic number on the XY plane for each output data in the Euler code. 
\item{\bf Animation\_atomic\_YZ.sh}\\
The shell file visualizes the distributions of the atomic number on the YZ plane for each output data in the Euler code. 
\item{\bf Animation\_rho\_XY.sh}\\
The shell file visualizes the distributions of the mass density on the XY plane for each output data in the Euler code.  
\item{\bf Animation\_rho\_YZ.sh}\\
The shell file visualizes the distributions of the mass density on the YZ plane for each output data in the Euler code.  
\if0
\item{\bf Animation\_Ti\_MODE.sh} \\
The shell file visualizes the mode analysis results for the ion temperature distribution calculated in the "Legendre.cpp" in Euler code. 
\fi
\item{\bf Animation\_Ti\_XY.sh}\\
The shell file visualizes the distributions of the ion temperature on the XY plane for each output data in the Euler code. 
\item{\bf Animation\_Ti\_YZ.sh}\\
The shell file visualizes the distributions of the ion temperature on the YZ plane for each output data in the Euler code. 
\item{\bf Fusiongain.sh}\\
The shell file plots the history of the fusion energy gain. 
\item{\bf rhoR.sh}\\
The shell file plots the history of the $\rho R$. 
\end{enumerate}
%\include{end}
In this section, we evaluate our method on two large-scale multimodal video benchmarks. The results show that our method outperforms representative baseline methods and achieves the state-of-the-art performance on both benchmarks. 



\subsection{Datasets and Setups}\label{sec:dataset_setups}
We evaluate our method on two large-scale multimodal video benchmarks: NTU RGB+D~\cite{ntu_rgbd} (classification) and PKU-MMD~\cite{pku_mmd} (detection). These datasets are selected for the following reasons. (1) They are (one of the) largest RGB-D video benchmarks in each category. (2) The privileged information transfer is reasonable because the domains of the two datasets are similar. (3) They contain abundant modalities, which are required for graph distillation. 

We use NTU RGB+D as our dataset in the source domain, and PKU-MMD in the target domain. In our experiments, unless stated otherwise, we apply graph distillation whenever applicable. Specifically, the visual encoders of all modalities are jointly trained on NTU RGB+D by graph distillation. On PKU-MMD, after initializing the visual encoder with the pre-trained weights obtained from NTU RGB+D, we also learn all available modalities by graph distillation on the target domain. By default, only a single modality is used at test time.

\noindent\textbf{NTU RGB+D~\cite{ntu_rgbd}.} 
It contains 56,880 videos from 60 action classes. Each video has exactly one action class and comes with four modalities: RGB, depth, 3D joints, and infrared. The training and testing sets have 40,320 and 16,560 videos, respectively. All results are reported with cross-subject evaluation.

\noindent\textbf{PKU-MMD~\cite{pku_mmd}.} 
It contains 1,076 long videos from 51 action classes. Each video contains approximately 20 action instances of various lengths and consists of four modalities: RGB, depth, 3D joints, and infrared. All results are evaluated based on the Average Precision (mAP) at different temporal Intersection over Union (tIoU) thresholds between the predicted and the ground truth intervals.

\noindent\textbf{Modalities.} We use a total of six modalities in our experiments: RGB, depth (D), optical flow (F), and three skeleton features (S) named Joint-Joint Distances (JJD), Joint-Joint Vector (JJV), and Joint-Line Distances (JLD)~\cite{ding2017investigation,10-stream}, respectively. The RGB and depth videos are provided in the datasets. The optical flow is calculated on the RGB videos using the dual TV-L1 method~\cite{zach2007duality}. The three spatial skeleton features are extracted from 3D joints using the method in \cite{ding2017investigation} and \cite{10-stream}. Note that we select a subset of the ten skeleton features in~\cite{ding2017investigation,10-stream} to ensure the simplicity and reproducibility of our method, and our approach can potentially perform better with the complete set of features.

\noindent\textbf{Baselines.}
In addition to comparing with the state-of-the-art, we implement three representative baselines that could be used to leverage multimodal privileged information: \textit{multi-task learning}~\cite{caruana1998multitask}, \textit{knowledge distillation}~\cite{distillation_hinton}, and \textit{cross-modal distillation}~\cite{distillation_gupta}. For the multi-task model, we predict the raw pixels of the other modalities from the representation of a single modality, and use the $L_2$ distance as the multi-task loss. For the distillation methods, the imitation loss is calculated as the high-temperature cross-entropy loss on the soft logits~\cite{distillation_hinton}, and $L_2$ loss on both representations and soft logits in cross-modal distillation~\cite{distillation_gupta}. These distillation methods originally only support two modalities, and therefore we average the pairwise losses to get the final loss.



\begin{table}[t]
\centering
\scriptsize
\caption{Comparison with state-of-the-art on NTU RGB+D. Our models are trained on all modalities and tested on the single modality specified in the table. The available modalities are RGB, depth (D), optical flow (F), and skeleton (S).}
\label{ntu_state_of_the_art}
\begin{tabular}{lc@{\hskip 0.1in}c@{\hskip 0.8in}l@{\hskip 0.4in}c@{\hskip 0.1in}c}
\toprule
Method & Test Modality & mAP & Method & Test Modality & mAP  \\
\midrule
Shahroudy~\cite{shahroudy2017deep} & RGB+D & 0.749 & Ours & RGB & \textbf{0.895} \\
Liu~\cite{liu2017viewpoint} & RGB+D & 0.775 & Ours  & D & 0.875 \\
Liu~\cite{skeleton_visualization} & S & 0.800 & Ours  & F & 0.857 \\
Ding~\cite{ding2017investigation} & S & 0.823 & Ours  & S & 0.837 \\
Li~\cite{10-stream} & S & 0.829 &&& \\
\bottomrule
\end{tabular}
\end{table}

\begin{table}[t]
\centering
\scriptsize
\caption{Comparison of action detection methods on PKU-MMD with state-of-the-art models. Our models are trained with graph distillation using all privileged modalities
and tested on the modalities specified in the table. ``Transfer'' refers to pre-training on NTU RGB+D on action classification. The available modalities are RGB, depth (D), optical flow (F), and skeleton (S).}
\label{pku_state_of_the_art}
\begin{tabular}{l@{\hskip 0.1in}c@{\hskip 0.1in}c@{\hskip 0.1in}c@{\hskip 0.1in}c}
\toprule
\multicolumn{2}{c}{} & \multicolumn{3}{c}{mAP @ tIoU thresholds ($\theta$)} \\
\cmidrule(r){3-5}
Method & Test Modality & 0.1 & 0.3 & 0.5 \\ 
\midrule
Deep RGB (DR) \cite{pku_mmd} & RGB & 0.507 & 0.323 & 0.147 \\
Qin and Shelton \cite{pku_result_qin} & RGB & 0.650 & 0.510 & 0.294 \\
Deep Optical Flow (DOF) \cite{pku_mmd} & F & 0.626 & 0.402 & 0.168 \\
Raw Skeleton (RS) \cite{pku_mmd} & S & 0.479 & 0.325 & 0.130 \\
Convolution Skeleton (CS) \cite{pku_mmd} & S & 0.493 & 0.318 & 0.121 \\
Wang and Wang \cite{pku_result_wang_workshop} & S & 0.842 & - & 0.743 \\
RS+DR+DOF \cite{pku_mmd} & RGB+F+S & 0.647 & 0.476 & 0.199 \\
CS+DR+DOF \cite{pku_mmd} & RGB+F+S & 0.649 & 0.471 & 0.199 \\
\midrule
Ours (w/o $|$ w/ transfer) & RGB & 0.824 $|$ 0.880 & 0.813 $|$ 0.868 & 0.743 $|$ 0.801 \\
Ours (w/o $|$ w/ transfer) & D   & 0.823 $|$ 0.872 & 0.817 $|$ 0.860 & 0.752 $|$ 0.792 \\
Ours (w/o $|$ w/ transfer) & F   & 0.790 $|$ 0.826 & 0.783 $|$ 0.814 & 0.708 $|$ 0.747 \\
Ours (w/o $|$ w/ transfer) & S   & 0.836 $|$ 0.857 & 0.823 $|$ 0.846 & 0.764 $|$ 0.784 \\
Ours (w/ transfer) & RGB+D+F+S & \bf{0.903} & \bf{0.895} & \bf{0.833} \\
\bottomrule
\end{tabular}
\end{table}



\noindent\textbf{Implementation Details.} 
For action classification, we train the visual encoder from scratch for 200 epochs using SGD with momentum with learning rate $10^{-2}$ and decay to $10^{-1}$ at epoch 125 and 175. $\lambda_1$ and $\lambda_2$ are set to $10,5$ respectively in Eq.~\eqref{eq:message_ab}. At test time we sample 5 clips for inference. For action detection, the visual and sequence encoder are trained for 400 epochs. The visual encoder is trained using SGD with momentum with learning rate $10^{-3}$, and the sequence encoder is trained with the Adam optimizer~\cite{kingma2015adam} with learning rate $10^{-3}$. The activity threshold $\gamma$ is set to $0.4$. For both tasks, we down-sample the frame rates of the datasets by a factor of 3. The clip length and detection window $T_c$ and $T_w$ are both set to 10. For the graph distillation, $\alpha$ is set to 10 in Eq.~\eqref{eq:graph_learning_softmax}. The output dimensions of the visual and sequence encoder are both set to 512. Since it is nontrivial to jointly train on multiple modalities from scratch, we employ curriculum learning~\cite{bengio2009curriculum} to train the distillation graph. To do so, we first fix the distillation graph as an identity matrix (uniform graph) in the first 200 epochs. In the second stage, we compute the constant vector $\mathbf{c}$ in Eq.~\eqref{eq:message_graph_final} according to the cross-validation results, and then learn the graph in an end-to-end manner.



\subsection{Comparison with State-of-the-Art}\label{sec:exp_soa}



\begin{figure}[t]
\begin{center}
\includegraphics[width=\linewidth]{predictions}
\end{center}
\caption{\textbf{A comparison of the prediction results on PKU-MMD.} (a) Both models make correct predictions. (b) The model without distillation in the source makes errors. Our model learns motion and skeleton information from the privileged modalities in the source domain, which helps the prediction for classes such as ``hand waving'' and ``falling''. (c) Both models make reasonable errors.}
\label{fig:detection}
\end{figure}





\noindent\textbf{Action Classification.} Table~\ref{ntu_state_of_the_art} shows the comparison of action classification with state-of-the-art models on NTU RGB+D dataset. Our graph distillation models are trained and tested on the same dataset in the source domain. NTU RGB+D is a very challenging dataset and has been recently studied in numerous studies~\cite{10-stream,liu2017viewpoint,skeleton_visualization,luo2017unsupervised,shahroudy2017deep}. Nevertheless, as we see, our model achieves the state-of-the-art results on NTU RGB+D. It yields a 4.5\% improvement, over the previous best result, using the depth video and a remarkable 6.6\% using the RGB video. After inspecting the results, we found the improvement mainly attributes to the learned graph capturing complementary information across multiple modalities. Fig.~\ref{fig:graph} shows example distillation graphs learned on NTU RGB+D. The results show that our method, without transfer learning, is effective for action classification in the source domain.


\noindent\textbf{Action Detection.} Table~\ref{pku_state_of_the_art} compares our method on PKU-MMD with previous work. Our model outperforms existing methods across all modalities. The results substantiate that our method can effectively leverage the privileged knowledge from multiple modalities. Fig.~\ref{fig:detection} illustrates detection results on the depth modality with and without the proposed distillation.


\subsection{Ablation Studies on Limited Training Data}\label{sec:ablation}
Section~\ref{sec:exp_soa} has shown that our method achieves the state-of-the-art results on two public benchmarks. However, in practice, the training data are often limited in size. To systematically evaluate our method on limited training data, as proposed in the introduction, we construct mini-NTU RGB+D and mini-PKU-MMD by randomly sub-sampling 5\% of the training data from their full datasets and use them for training. For evaluation, we test the model on the full test set.



\begin{table}[t]
\centering
\scriptsize
\caption{The comparison with (a) baseline methods using Privileged Information (PIs) on mini-NTU RGB+D, (b) distillation graphs on mini-NTU RGB+D and mini-PKU-MMD. Empty graph trains each modality independently. Uniform graph uses a uniform weight in distillation. Prior graph is built according to the cross-validation accuracy of each modality. Learned graph is learned by our method. ``D'' refers to the depth modality.}
\subtable[\label{ntu_baselines}Baseline methods using PIs.]
{
  \renewcommand{\arraystretch}{1.1}
  \begin{tabular}{lcc}
  \toprule
  Method & mAP / RGB \\
  \midrule
  Empty graph & 0.464 \\
  Multi-task \cite{caruana1998multitask}  & 0.456 \\
  Cross-distillation \cite{distillation_gupta}  & 0.503 \\
  Knowledge distillation \cite{distillation_hinton}  & 0.524 \\
  Learned graph & \bf{0.619} \\
  \bottomrule
  \end{tabular}
}
\subtable[\label{different_graphs}Different distillation graphs.]{
  \begin{tabular}{l@{\hskip 0.1in}c@{\hskip 0.2in}c}
  \toprule
  \multicolumn{1}{c}{} & mini-NTU & mini-PKU \\
  \cmidrule(r){2-3}
  Graph & {\tiny mAP / RGB} & {\tiny mAP @ 0.5 / D} \\
  \midrule
  Empty graph & 0.464 & 0.501 \\
  Uniform graph & 0.537 & 0.513 \\
  Prior graph & 0.571 & 0.515 \\
  Learned graph & \bf{0.619} & \bf{0.559}\\
  \bottomrule
  \end{tabular}
}
\end{table}

\begin{table}[t]
\centering
\scriptsize
\caption{The mAP comparison on mini-PKU-MMD at different tIoU threshold $\theta$. The depth modality is chosen for testing. ``src'', ``trg'', and ``PI'' stand for source, target, and privileged information, respectively.}
\label{pku_mmd_baselines}
\begin{tabular}{c@{\hskip 0.2in}l@{\hskip 0.4in}c@{\hskip 0.4in}c@{\hskip 0.4in}c}
\toprule
\multicolumn{2}{c}{} & \multicolumn{3}{c}{mAP @ tIoU thresholds ($\theta$)} \\
\cmidrule(r){3-5}
 & Method & $0.1$ & $0.3$ & $0.5$ \\
\midrule
1&trg only & 0.248 & 0.235 & 0.200 \\
2&src + trg & 0.583 & 0.567 & 0.501 \\
3&src w/ PIs + trg & 0.625 & 0.610 & 0.533 \\
4&src + trg w/ PIs & 0.626 & 0.615 & 0.559 \\
5&src w/ PIs + trg w/ PIs & 0.642 & 0.629 & 0.562 \\
\midrule
6&src w/ PIs + trg & 0.625 & 0.610 & 0.533  \\
7&src w/ PIs + trg w/ 1 PI & 0.632 & 0.615 & 0.549 \\
8&src w/ PIs + trg w/ 2 PIs & 0.636 & 0.624 & 0.557 \\
9&src w/ PIs + trg w/ all PIs & 0.642 & 0.629 & 0.562 \\
\bottomrule
\end{tabular}
\end{table}



\noindent\textbf{Comparison with Baseline Methods.} Table~\ref{ntu_baselines} shows the comparison with the baseline models that uses privileged information (see Section~\ref{sec:dataset_setups}). The fact that our method outperforms the representative baseline methods validates the efficacy of the graph distillation method.

\noindent\textbf{Efficacy of Distillation Graph.} Table \ref{different_graphs} compares the performance of predefined and learned distillation graphs. The proposed learned graph is compared with an empty graph (no distillation), a uniform graph of equal weights, and a prior graph computed using the cross-validation accuracy of each modality. Results show that the learned graph structure with modality-specific prior and example-specific information obtains the best results on both datasets.



\begin{figure}[t]
% \mpage{0.48}{\small{(a) Falling}}\hfill
% \mpage{0.48}{\small{(b) Brushing teeth}}\hfill
\begin{center}
\includegraphics[width=0.8\linewidth]{graph}
\end{center}
\caption{\textbf{The visualization of graph distillation on NTU RGB+D.} The numbers indicate the ranks of the distillation weights, with 1 being the largest and 5 being the smallest. (a) Class ``falling'': Our graph assigns more weight to optical flow because optical flow captures the motion information. (b) Class ``brushing teeth'': In this case, motion is negligible, and our graph assigns the smallest weight to it. Instead, it assigns the largest weight to skeleton data.}
\label{fig:graph}
\end{figure}



\noindent\textbf{Efficacy of Privileged Information.} Table~\ref{pku_mmd_baselines} compares our distillation and transfer under different training settings. The input at test time is a single depth modality. By comparing row 2 and 3 in Table~\ref{pku_mmd_baselines}, we see that when transferring the visual encoder to the target domain, the one pre-trained with privileged information in the source domain performs better than its counterpart. As discussed in Section~\ref{sec:collective}, graph distillation can also be applied to the target domain. By comparing row 3 and 5 (or row 2 and 4) of Table~\ref{pku_mmd_baselines}, we see that performance gain is achieved by applying the graph distillation in the target domain. The results show that our graph distillation can capture useful information from multiple modalities in both the source and target domain.

\noindent\textbf{Efficacy of Having More Modalities.} The last three rows of Table \ref{pku_mmd_baselines} show that performance gain is achieved by increasing the number of modalities used as the privileged information. Note that the test modality is depth, the first privileged modality is RGB, and the second privileged modality is the skeleton feature JJD. The results also suggest that these modalities provide each other complementary information during the graph distillation.



\subsection{Graph Distillation on UCF-101}

In this section, we consider graph edge distillation, a special case of graph distillation on UCF-101~\cite{soomro2012ucf101} in which only two modalities (RGB and optical flow) are available. Table~\ref{ucf101} shows the action classification results on UCF-101 using the two-stream architecture proposed in~\cite{two_stream_simonyan}. The optical flow modality performs significantly better than RGB when training from scratch. This is consistent with previous findings that dense optical flow is able to achieve very good performance in spite of limited training data \cite{two_stream_simonyan}. To testify our method, we train a model on the RGB modality from scratch with distillation. Our distilled model performs much better than the model directly trained from scratch. Note that our distilled model outperforms the fine-tuning model that uses pretrained weights on ImageNet.

\begin{table}[ht]
\scriptsize
\centering
\caption{Action classification results on UCF101. For graph distillation model, we distill knowledge from the optical flow stream to the RGB stream.}
\label{ucf101}
\begin{tabular}{l@{\hskip 0.1in}c@{\hskip 0.2in}c@{\hskip 0.1in}c}
\toprule
Method & Test Modality & mAP & Diff. \\
\midrule
From scratch   & Flow & 0.803 & - \\
From scratch   & RGB & 0.484 & +0.000 \\
ImageNet pretrained & RGB & 0.728 & +0.244 \\
Graph distillation & RGB & \textbf{0.757} & \textbf{+0.273} \\
\bottomrule
\end{tabular}
\end{table}

%\documentclass[preprint,12pt]{elsarticle}
%\if0
\usepackage{amssymb}
\usepackage{mathtools}
%\usepackage[dvipdfmx]{graphicx}
\usepackage{cite}
\usepackage{graphicx}
\usepackage{bm}
\usepackage{here}
\usepackage[subrefformat=parens]{subcaption}
\fi
%\usepackage{amssymb}
\usepackage{amsmath}
\usepackage[dvipdfmx]{}
\usepackage[dvipdfmx]{color}
%\usepackage{cite}
%\usepackage{upgreek}
\usepackage{url}
%\usepackage[dvipdfmx]{hyperref}
%\usepackage{pxjahyper}
%\usepackage {hyperref}
\usepackage{graphicx}
\usepackage{bm}
\usepackage{here}
\usepackage{caption}
\usepackage[subrefformat=parens]{subcaption}
\captionsetup{compatibility=false}

%% The amsthm package provides extended theorem environments
%% \usepackage{amsthm}

%% The lineno packages adds line numbers. Start line numbering with
%% \begin{linenumbers}, end it with \end{linenumbers}. Or switch it on
%% for the whole article with \linenumbers after \end{frontmatter}.
%% \usepackage{lineno}

%% natbib.sty is loaded by default. However, natbib options can be
%% provided with \biboptions{...} command. Following options are
%% valid:

%%   round  -  round parentheses are used (default)
%%   square -  square brackets are used   [option]
%%   curly  -  curly braces are used      {option}
%%   angle  -  angle brackets are used    <option>
%%   semicolon  -  multiple citations separated by semi-colon
%%   colon  - same as semicolon, an earlier confusion
%%   comma  -  separated by comma
%%   numbers-  selects numerical citations
%%   super  -  numerical citations as superscripts
%%   sort   -  sorts multiple citations according to order in ref. list
%%   sort&compress   -  like sort, but also compresses numerical citations
%%   compress - compresses without sorting
%%
%% \biboptions{comma,round}

% \biboptions{}

%% This list environment is used for the references in the
%% Program Summary
%%
\newcounter{bla}
\newenvironment{refnummer}{%
\list{[\arabic{bla}]}%
{\usecounter{bla}%
 \setlength{\itemindent}{0pt}%
 \setlength{\topsep}{0pt}%
 \setlength{\itemsep}{0pt}%
 \setlength{\labelsep}{2pt}%
 \setlength{\listparindent}{0pt}%
 \settowidth{\labelwidth}{[9]}%
 \setlength{\leftmargin}{\labelwidth}%
 \addtolength{\leftmargin}{\labelsep}%
 \setlength{\rightmargin}{0pt}}}
 {\endlist}
\begin{document}

\section{Testing the program O-SUKI-N 3D}
The several tests are shown below to present the target fuel implosion dynamics. In the example cases, the HIBs and the target fuel have the following common parameters, which are the same values employed in Ref. \cite{CPC-O-SUKI}: the beam radius at the entrance of a reactor chamber $R_{en}$ = 35 mm, the beam particle density distribution is in the Gaussian profile and all projectile Pb ions have 8 GeV. The target is a multilayered pellet, in which the pellet outer radius is 4 mm, a Pb layer thickness is 0.029 mm, the Al thickness is 0.460 mm, and the DT thickness is 0.083 mm; the Pb, Al and DT layers have the radial mesh numbers of 4, 46 and 30 in these example cases, respectively, and the total mesh number in the theta direction is 90. The input beam pulse is shown in Fig. 12 in Ref. \cite{CPC-O-SUKI}. The beam radius is 3.8mm on the target surface. However, $R_b$ = 3.8mm changes at $\tau_{wb}$ to 3.7mm for the wobbling beam irradiation. Here $\tau_{wb}$ is the rotational period of the beam axis. The rotational frequency is 424MHz ($rotaionnumber$ = 11). 



%% INPUT PULSE  
%\begin{figure}[H]
%		\centering
%		\includegraphics[width=10cm]{images/pulse.eps}
%		\caption{An example for the input beam pulse.}\label{pulse}
%\end{figure}



First the 3D Langrange code was run without the OK3 illumination code. This is the case for $OK\_Switch=10$, and we added the artificial non-uniformity $Y_3^2$ (the spherical harmonics) with the amplitude of $30.0\%$. In Fig. \ref{NoOK3_23_Ti} the ion temperature distribution is shown at $t$=35ns, and in Fig. \ref{NoOK3_23_rho} the mass density distribution is presented at $t$=35ns. The target shape is largely distorted due to the non-uniformity of the HIBs deposition energy distribution.  


%% LAGRANGE CASE WITHOUT OK3
\begin{figure}[H]
		\centering
		\includegraphics[width=10cm]{images/NoOK3_Non23_30_35ns_Ti.eps}
		\caption{Ion temperature in the 3D Lagrange code without OK3 code at $t$=35ns. The non-uniformity distiribution is $Y_3^2$ with the amplitude of $30\%$.}\label{NoOK3_23_Ti}
\end{figure}
\begin{figure}[H]
		\centering
		\includegraphics[width=10cm]{images/NoOK3_Non23_30_35ns_rho.eps}
		\caption{Mass density in the 3D Lagrange code without OK3 code at $t$=35ns. The non-uniformity distriution is $Y_3^2$ with the amplitude of $30\%$.}\label{NoOK3_23_rho}
\end{figure}


We also performed run-through simulation tests. In the example cases, the OK3 code was coupled with the run-through simulations. The implosion data were obtained by the Lagrange code, and the data just before the void closure time were transferred to the Euler code through the data Conversion code. Two cases are computed for the target fuel implosion dynamics with the spiral wobbling or without the oscillating HIBs. These examples are the run-through simulations with the OK3 illumination code ($OK\_Switch = 1$). The input beam pulse, employed in the run-through tests, is shown in Fig. \ref {Beam}. This beam input energy is 5.4MJ. We show the $r-t$ diagram for the case without the HIBs wobbling in Fig. \ref{rt}. The Lagrange-code test results stored in the output directory are visualized in Figs. \ref {fusion_Ti} for the target ion temperature ($T_i$) distributions at $t$ = 0.0, 15.0, 20.0 and 22.5 ns for the case with the HIBs wobbling behavior.  The RMS non-uniformity results are shown in Figs. \ref{fusion_RMS} (a) for DT layer's Ion temperature($T_i$), (b) for DT layer's Mass density($\rho$), (c) for Al layer's Ion temperature($T_i$) and (d) for Al layer's Mass density($\rho$). 
%
When the HIBs have the wobbling motion during the implosion with the wobbling frequency of 424MHz, the radius acceleration distributions are shown in Figs. \ref{Vr_tp} (a) in the $\theta$ direction and (b) in the $\phi$ direction at $t=6.25t_w=11.2ns$ (solid lines) and at $t=6.75t_w=12.2ns$ (dotted lines). Here $t_w$ shows the one rotation time. Figures \ref{Vr_tp} present that the non-uniformity phase of the implosion acceleration is controlled externally by the HIBs wobbling behavior \cite{CPC-O-SUKI, RSato2}.  
%

\begin{figure}[H]
		\centering
		\includegraphics[width=7.5cm]{images/Beam.eps}
		\caption{Input beam pulse shape used in the example run-through tests.}\label{Beam}
\end{figure}
\begin{figure}[H]
		\centering
		\includegraphics[width=8cm]{images/YesWob_SLC.eps}
		\caption{The $r-t$ diagram for the implosion with the HIBs wobbling illumination. The black line area shows the Pb layer, the blue line area Al and the red line area is DT.}\label{rt}
\end{figure}
\begin{figure}[H]
		\centering
		\includegraphics[width=6.5cm]{images/YesWob_Ti_0ns.eps}
		\includegraphics[width=6.5cm]{images/YesWob_Ti_15ns.eps}\\
		\includegraphics[width=6.5cm]{images/YesWob_Ti_20ns.eps}
		\includegraphics[width=6.5cm]{images/YesWob_Ti_225ns.eps}\\
		\caption{Ion temperature distributions in the example run-through test with the HIBs wobbling illumination at (a) $t$=0.0ns, (b) 15.0ns, (c) 20.0ns and (d) 22.5ns.}\label{fusion_Ti}
\end{figure}
\begin{figure}[H]
		\centering
		\includegraphics[width=6.5cm]{images/FusionRMS_DTTi.eps}
		\includegraphics[width=6.5cm]{images/FusionRMS_DTrho.eps}\\
		\includegraphics[width=6.5cm]{images/FusionRMS_AlTi.eps}
		\includegraphics[width=6.5cm]{images/FusionRMS_Alrho.eps}\\
		\caption{RMS non-uniformity histories of (a) the DT ion temperature, (b) the DT mass density, (c) the Al ion temperature and (d) the Al mass density for the cases with the wobbling HIBs (solid lines) and without the wobbling HIBs (dotted lines).}\label{fusion_RMS}
\end{figure}
%
\begin{figure}[H]
		\centering
		\includegraphics[width=6.5cm]{images/theta-Vr.eps}
		\includegraphics[width=6.5cm]{images/phi-Vr.eps}\\
		\caption{Radial acceleration distributions in (a) $\theta$ and (b) $\phi$. The solid lines show the acceleration ditributions at $t=6.25t_w=11.3ns$, and the dotted lines at $t=6.75t_w=12.2ns$.}\label{Vr_tp}
\end{figure}
%

After the Lagrange code computation, the implosion data are converted and transferred to the Euler code. Figures \ref{Ti_EuWobblIgnited} show the ion temperature distributions by the Euler code. Figures \ref{Ti_EuWobblIgnited} show that the DT fuel is ignited and the gain obtained is about 17.5 in this example case. For a realistic HIF reactor design, the implosion parameters should be further optimized to obtain a sufficient gain, which should be larger than 30$\sim$40 in HIF \cite{CPC-O-SUKI, Kawata1, Kawata2, RSato2}. 

\begin{figure}[H]
		\centering
		\includegraphics[width=13cm]{images/EuWobblIgnited.eps}
		\caption{Ion temperature distributions (a) at $t=$24.88ns, (b) at 28.44ns and at 29.21ns.}\label{Ti_EuWobblIgnited}
\end{figure}

\if0
In Fig. \ref{NoOK3_03_Ti}, a non-uniform energy deposition of the HIBs illumination is introduced based on the spherical harmonics $Y_3^0$ with the amplitude of $3.0\%$ in the 3D Lagrange code. The implosion data was obtained by the Lagrange code, and the data just before the void closure time were transferred to the Euler code through the data Conversion code.  Figure \ref{Ti_Eu_Y03} shows the ion temperature distributions  by the Euler code at (a) at $t$=36.36ns, (b) 36.57ns, (c) 41.32ns and (d) 42.41ns. In this example case the DT fuel is not yet ignited due to the insufficient ion temperature. 

\begin{figure}[H]
		\centering
		\includegraphics[width=8.5cm]{images/NoOK3_Non03_03_35ns_Ti.eps}
		\caption{Ion temperature in the 3D Lagrange code without OK3 code at $t$=35ns. The non-uniformity distriution is $Y_3^0$ with the amplitude of $3\%$.}\label{NoOK3_03_Ti}
\end{figure}


%% TIME VS ION TEMPERATURE Euler Y03
\begin{figure}[H]
		\centering
		\includegraphics[width=6.5cm]{images/ion_Eu_Y03_36_36ns.eps}
		\includegraphics[width=6.5cm]{images/ion_Eu_Y03_36_57ns.eps} \\
		\includegraphics[width=6.5cm]{images/ion_Eu_Y03_41_32ns.eps}
		\includegraphics[width=6.5cm]{images/ion_Eu_Y03_42_41ns.eps} \\
		\caption{Ion temperature distributions under a non-uniform energy deposition based on the spherical harmonics $Y_0^3$ by the Euler code,  (a) at $t$=36.36ns, (b) 36.57ns, (c) 41.32ns and (d) 42.41ns.}\label{Ti_Eu_Y03}
\end{figure}
\fi


In order to check the accuracy of the 3D Euler code, we also performed the Euler code tests, using the initial conditions of the 2D Euler code. The initial conditions in the Euler code are the output of the Lagrangian code.  To this end, the 2D Euler initial conditions were converted into 3D. Therefore, the physical values are uniform in the $\phi$ direction. The Lagrangian test 2D results for the target ion temperature ($T_i$) and the mass density ($\rho$) distribution at $t$ = 29 ns are shown in Figs. 14 and 15 in Ref. \cite{CPC-O-SUKI} for the cases with and without the wobbling HIBs.  The 2D Eulerian test results for the fusion energy gain is shown in Fig. 16 in Ref. \cite{CPC-O-SUKI}.  In Fig. \ref{Ti_Eu_3d} we show the ion temperature distributions by the 3D Euler code. The wobbling HIBs are not used in this simulation. In this case the fuel is ignited at $t \sim $30.1ns. The histories of the fusion gain $G$ of the 2D code and the 3D code are shown in Fig. \ref{FusionGain_Eu}. The fusion gain was 52.5 by the 2D code and 57.6 by the 3D code. In addition, we also did another test for the wobbling HIBs (see Figs. 15 and 16 in Ref. \cite{CPC-O-SUKI}), and the fusion gain was 76.1 in 2D \cite{CPC-O-SUKI} and 67.4 in 3D. The results would confirm that the 3D Euler code reproduces the 2D results successfully for the ignition time and the fusion gain obtained. 


%% TIME VS ION TEMPERATURE Euler
\begin{figure}[H]
		\centering
		\includegraphics[width=6.5cm]{images/ion_Eu_30_42ns.eps}
		\includegraphics[width=6.5cm]{images/ion_Eu_30_53ns.eps} \\
		\includegraphics[width=6.5cm]{images/ion_Eu_32_35ns.eps}
		\includegraphics[width=6.5cm]{images/ion_Eu_32_58ns.eps} \\
		\caption{Ion temperature distributions by the 3D Euler code without the HIBs wobbling at (a) $t$=30.42ns, (b) 30.53ns, (c) 32.35ns and (d) 32.58ns}\label{Ti_Eu_3d}
\end{figure}


%% ENERGY GAIN Euler
\begin{figure}[H]
		\centering
		\includegraphics[width=11cm]{images/FusionGain_Eu.eps}
		\caption{Fusion energy gain curves for the cases with 3D code (a solid line) and with 2D code (a dotted line).}\label{FusionGain_Eu}
\end{figure}

We also simulated the double-cone ignition scheme\cite{Double-cone} using a 3D Euler code. The double-cone ignition scheme was proposed by Prof. Jie Zhang \cite{Double-cone}, and the two compressed DT clouds are created by the gold cones. The two DT spherical clouds collide each other like the impact fusion \cite{Winterberg}. In this example case, the compressed DT maximum density of the DT fuel is set to be $1.0\times 10^5$[kg/m$^3$] with the Gaussian spatial distribution. The DT ignition will be attained by an additional heating, which is not taken into consideration in this example. The ion, electron and radiation temperatures are 10[eV] initially in the Euler code. The radius of the fuel is 92[$\mu$m] and the mass was $0.1$[mg]. We set the colliding speed $w$ of the two DT fuel clouds to $3.0\times10^5$ [m/s]. The ion temperature distributions are shown in Fig. \ref{Double_cone_Ti}.


%% DOUBLE-CONE
\begin{figure}[H]
		\centering
		\includegraphics[width=6.5cm]{images/double_cone_0ns.eps}
		\includegraphics[width=6.5cm]{images/double_cone_15_06ns.eps} \\
		\includegraphics[width=6.5cm]{images/double_cone_29_80ns.eps}
		\includegraphics[width=6.5cm]{images/double_cone_46_78ns.eps} \\
		\caption{Ion temperature distributions for the Double-cone ignition scheme \cite{Double-cone} at (a) $t$=0.0ns, (b) 15.06ns, (c) 29.80ns and (d) 46.78ns.}\label{Double_cone_Ti}
\end{figure}

	
%\include{end}

\section{Conclusions}
We have developed and presented the O-SUKI-N 3D code, which is useful to simulate 3D spherical DT fuel target implosion in HIF. The O-SUKI-N code is an upgraded implosion simulation system from the 2D O-SUKI code, and consists of the Lagrangian fluid code, the data conversion from the Lagrangian code data to the Euler code data, and the Euler code. Near the void closure phase of the DT fuel implosion, the DT fuel spatial deformation is serious. At the stagnation phase the DT fuel is compressed to about a thousand times of the solid density. The O-SUKI-N 3D code would provide a useful tool for the integrated DT fuel target implosion simulation in HIF.

\section*{Declaration of Competing Interest}
     The authors declare that they have no known competing financial interests or personal relationships that could have appeared to influence the work reported in this paper. 
     
\section*{CRediT author statement}
Hiroki Nakamura: Software for Euler code and Conversion code, Validation, Visualization; Ken Uchibori: Software for Lagrange code and Conversion code, Validation, Visualization, Writing draft; Shigeo Kawata: Basic idea, Conceptualization, Methodology, Investigation, Supervision, writing paper; Takahiro Karino: Methodology, Supervision; Ryo Sato: Methodology, Validation; Alexander I. Ogoyski: Software for OK3 code, Validation. 

\section*{Acknowledgments}
	The work was partly supported by JSPS, Japan-U. S.  Exchange Program, MEXT, CORE (Center for Optical Research and Education, Utsunomiya University), Shanghai Jiao Tong University and ILE/Osaka University. The work was also partly done under the collaborations with Xi'an Jia Tong University, Inst. of Modern Physics, Lanzhou, Inst. of Physics, Beijing, Fudan university, Shanghai, Renmin University of China, Beijing, and ELI-Beamlines, Prague. 

\bibliographystyle{elsarticle-num}
\bibliography{<your-bib-database>}

%% Authors are advised to submit their bibtex database files. They are
%% requested to list a bibtex style file in the manuscript if they do
%% not want to use elsarticle-num.bst.

%% References without bibTeX database:

 \begin{thebibliography}{99}

%% \bibitem must have the following form:
%%   \bibitem{key}...
%%
\bibitem{CPC-O-SUKI}
R. Sato, S. Kawata, T. Karino, K. Uchibori, T. Iinuma, H. Katoh and A.I. Ogoyski, Comput. Phys. Commun. 240 (2019) 83-100. 
\bibitem{ICFBook}
S. Atzeni and J. Meyer-ter-Vehn, The Physics of Inertial Fusion, Oxford University Press, 2009. 

\bibitem {kwtANDniu}
 S. Kawata and K. Niu, J. Phys. Soc. Jpn. 53 (1984) 3416-3426. 
 
 
 \bibitem{Kawata1}
S. Kawata, T. Karino and A. I. Ogoyski, Matter and Radiation at Extremes 1(2) (2016)  89-113. 

 \bibitem{Kawata2}
S. Kawata, Advances in Physics x 6(1) (2021)  1873860. 


 
\bibitem{ogoyski1}
A. I. Ogoyski, T. Someya and S. Kawata, Comput. Phys. Commun. 157 (2004) 160-172.
 \bibitem{ogoyski2}
A. I. Ogoyski, S. Kawata and T. Someya, Compt. Phys. Commun. 161 (2004) 143-150.
 \bibitem{ogoyski3}
A. I. Ogoyski, S. Kawata, P. H. Popov, Compt. Phys. Commun. 181 (2010) 1332-1333. 

\bibitem{Schulz}
W. D. Schulz, ”Two-Dimensional Lagrangian Hydrodynamic Difference Equations”, University of California Lawrence Radiation Laboratory Livermore, California, UCRL-6776, 1963. 

\bibitem{Tahir}
N. A. Tahir, K. A. Long, E. W. Laing, J. Appl. Phys. 60 (1986) 898. 

\bibitem{Zeldovich}
Ya. B. Zel'dovich, Yu. P. Raizer, Physics of Shock Waves and High-Temperature Hydrodynamic Phenomena, Dover Books on Physics, New York, 2002. 

\bibitem {mehlhorn}
T.A. Mehlhorn, J. Appl. Phys. 52 (1981) 6522-6532. 

\bibitem{artv}
 J. Von Neumann and R. D. Richtmyer, J. Appl. Phys. 21 (1950) 232-237. 
 
 \bibitem{Christiansen}
J. P. Christianen, D. E. T. F. Ashby, and K. V. Roberts, Computer Physics Communications 7 (1974) 271-287. 
 
\bibitem{Bell}
A. R. Bell, Rutherford Laboratory Report, RL-80-091, 1981. 

\bibitem{NRLpf}
A. S. Richardson, 2019 NRL Plasma Formulary, (2019). 

\bibitem{Fraley}
G. S. Fraley, E. J. Linnebur, R. J. Mason, R. L. Morse, Phys. Fluids, 17 (1974) 474-489. 

\bibitem{RSato2}
R. Sato, S. Kawata, T. Karino, K. Uchibori and A. I. Ogoysk, Scientific Reports, 6659 (2019) https://doi.org/10.1038/s41598-019-43221-7. 

\bibitem{Double-cone}
J. Zhang, W. M. Wang, X. H. Yang, D. Wu, Y. Y. Ma, J. L. Jiao, Z. Zhang, F. Y. Wu, X. H. Yuan, Y. T. Li and J. Q. Zhu, Rhilosophical Tran. Royal Soc. A (2020) https://doi.org/10.1098/rsta.2020.0015. 

\bibitem{Winterberg}
F. Winterberg, Z. Naturforschg. 19a (1964) 231-239. 

%%%%%%%%%%%%
\if0 
 \bibitem{IGHoffman}
I. Hofmann, Matter and Radiation at Extremes, 3(1) (2018) 1-11.  
\bibitem {bohne}
D. B\"{o}hne, I. Hofmann, G. Kessler, G.L. Kulcinski, J. Meyer-ter-Vehn, et al., Nucl. Eng. Des. 73 (2) (1982) 195-200. 

\bibitem {ymk}
T. Yamaki, et al., HIBLIC-1, Conceptual Design of a Heavy Ion Fusion Reactor, Research Information Center, Institute for Plasma Physics, Nagoya University, Report IPPJ-663, 1985.

\bibitem {moir}
R.W. Moir, R.L. Bieri, X.M. Chen, T.J. Dolan, M.A. Hoffman, et al., Fusion Technol. 25 (1994) 5-25.

\bibitem {ziegler}
 J. F. Ziegler, J. P. Biersack, U. Littmark, The Stopping and Range of Ions in matter, volume 1, Pergamon, New York, 1985.

\bibitem {CM}
D.A. Callahan-Miller, M. Tabak, Nucl. Fusion 39 (1999) 883-892.
\bibitem {arnold}
R.C. Arnold, E. Colton, S. Fenster, M. Foss, G. Magelssen, et al., Nucl. Inst. Meth. 199 (1982) 557-561.
\bibitem {piriz}
A.R. Piriz, A.R.N.A. Tahir, D.H.H. Hoffmann, M. Temporal, Phys. Rev. E 67 (017501) (2003) 1-3.
\bibitem {qin}
H. Qin, R.C. Davidson, B.G. Logan, Phys. Rev. Lett. 104 (2010) 254801.
\bibitem {KSrad}
S. Kawata, T. Sato, T. Teramoto, E. Bandoh, Y. Masubichi, et al., Laser Part. Beams 11 (1993) 757-768.
\bibitem {KSdynamic}
S. Kawata, T. Sato, T. Teramoto, E. Bandoh, Y. Masubichi, et al., Phys. Plasmas 19 (2012) 024503.
\bibitem {KKrobust}
S. Kawata, T. Karino, Phys. Plasmas 22 (2015) 042106.
\bibitem {bodner}
S.E. Bodner, Phys. Rev. Lett. 33 (1974) 761-764.
\bibitem {takabe}
H. Takabe, K. Mima, L. Montierth, R.L. Morse, Phys. Fluids 28 (1985) 3676-3682. 

\bibitem{Sasaki}
 J. Sasaki, T. Nakamura, Y. Uchida, T. Someya, K. Shimizu, M. Shitamura, T. Teramoto, A. I. Blagoev, S. Kawata, Jpn. J. Appl. Phys. 40(1) (2001) 968-971.

\bibitem {Miyazawa}
K. Miyazawa, A.I. Ogoyski, S. Kawata, T. Someya, T. Kikuchi, Phys. Plasmas 12 (2005) 122702-122711.
\fi

 \end{thebibliography}


\input{end}