%\documentclass[preprint,12pt]{elsarticle}
%\if0
\usepackage{amssymb}
\usepackage{mathtools}
%\usepackage[dvipdfmx]{graphicx}
\usepackage{cite}
\usepackage{graphicx}
\usepackage{bm}
\usepackage{here}
\usepackage[subrefformat=parens]{subcaption}
\fi
%\usepackage{amssymb}
\usepackage{amsmath}
\usepackage[dvipdfmx]{}
\usepackage[dvipdfmx]{color}
%\usepackage{cite}
%\usepackage{upgreek}
\usepackage{url}
%\usepackage[dvipdfmx]{hyperref}
%\usepackage{pxjahyper}
%\usepackage {hyperref}
\usepackage{graphicx}
\usepackage{bm}
\usepackage{here}
\usepackage{caption}
\usepackage[subrefformat=parens]{subcaption}
\captionsetup{compatibility=false}

%% The amsthm package provides extended theorem environments
%% \usepackage{amsthm}

%% The lineno packages adds line numbers. Start line numbering with
%% \begin{linenumbers}, end it with \end{linenumbers}. Or switch it on
%% for the whole article with \linenumbers after \end{frontmatter}.
%% \usepackage{lineno}

%% natbib.sty is loaded by default. However, natbib options can be
%% provided with \biboptions{...} command. Following options are
%% valid:

%%   round  -  round parentheses are used (default)
%%   square -  square brackets are used   [option]
%%   curly  -  curly braces are used      {option}
%%   angle  -  angle brackets are used    <option>
%%   semicolon  -  multiple citations separated by semi-colon
%%   colon  - same as semicolon, an earlier confusion
%%   comma  -  separated by comma
%%   numbers-  selects numerical citations
%%   super  -  numerical citations as superscripts
%%   sort   -  sorts multiple citations according to order in ref. list
%%   sort&compress   -  like sort, but also compresses numerical citations
%%   compress - compresses without sorting
%%
%% \biboptions{comma,round}

% \biboptions{}

%% This list environment is used for the references in the
%% Program Summary
%%
\newcounter{bla}
\newenvironment{refnummer}{%
\list{[\arabic{bla}]}%
{\usecounter{bla}%
 \setlength{\itemindent}{0pt}%
 \setlength{\topsep}{0pt}%
 \setlength{\itemsep}{0pt}%
 \setlength{\labelsep}{2pt}%
 \setlength{\listparindent}{0pt}%
 \settowidth{\labelwidth}{[9]}%
 \setlength{\leftmargin}{\labelwidth}%
 \addtolength{\leftmargin}{\labelsep}%
 \setlength{\rightmargin}{0pt}}}
 {\endlist}
\begin{document}
\section{Introduction}\label{sec:1}

Code O-SUKI-N 3D (3 dimension) is an upgraded 3D version of our Code O-SUKI \cite{CPC-O-SUKI}, and it provides a capability to simulate a deuterium (D) - tritium (T) fuel target implosion, ignition and burning in 3D in heavy ion beam (HIB) inertial confinement fusion (ICF). 

In ICF, DT fuel target implosion, ignition and burning are essentially important to release a sufficient fusion energy output. In ICF a few mg DT in a fuel pellet is compressed to about a thousand times the solid density by an input driver energy, for example, lasers or heavy ion beams (HIBs) or pulse power. In addition, the ion temperature of the compressed DT must reach $\sim$5-10 KeV \cite{ICFBook}. In order to compress the DT fuel stably to the high density, the implosion non-uniformity should be less than a few percent\cite {kwtANDniu} The key issues of the fuel implosion in ICF include how to realize the uniform implosion. The O-SUKI-N 3D code system provides an integrated computer simulation tool to study the DT fuel implosion, ignition and burning in heavy ion inertial confinement fusion (HIF) \cite{Kawata1, Kawata2}. 
         
The DT fuel implosion is simulated until just before the void closure time by the Lagrangian code, which can couple with the OK3 code to include the time-dependent HIBs energy deposition profile in the target energy absorber layer. For example, the detail HIBs illumination on a HIF DT target can be computed by a computer code of OK3 \cite{ogoyski1, ogoyski2, ogoyski3}. The Lagrange code data are converted to the data imported to the Euler code. The Euler code is robust against the target fuel deformation. The DT fuel ignition and burning are simulated further by the 3D Euler fluid code. The O-SUKI-N 3D code system simulates the 3D HIF target implosion dynamics, and would contribute to release the fusion energy stably for society. 

%\include{end}