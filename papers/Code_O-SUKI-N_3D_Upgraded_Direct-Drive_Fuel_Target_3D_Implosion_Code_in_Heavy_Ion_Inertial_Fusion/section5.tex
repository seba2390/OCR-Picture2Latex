%\documentclass[preprint,12pt]{elsarticle}
%\if0
\usepackage{amssymb}
\usepackage{mathtools}
%\usepackage[dvipdfmx]{graphicx}
\usepackage{cite}
\usepackage{graphicx}
\usepackage{bm}
\usepackage{here}
\usepackage[subrefformat=parens]{subcaption}
\fi
%\usepackage{amssymb}
\usepackage{amsmath}
\usepackage[dvipdfmx]{}
\usepackage[dvipdfmx]{color}
%\usepackage{cite}
%\usepackage{upgreek}
\usepackage{url}
%\usepackage[dvipdfmx]{hyperref}
%\usepackage{pxjahyper}
%\usepackage {hyperref}
\usepackage{graphicx}
\usepackage{bm}
\usepackage{here}
\usepackage{caption}
\usepackage[subrefformat=parens]{subcaption}
\captionsetup{compatibility=false}

%% The amsthm package provides extended theorem environments
%% \usepackage{amsthm}

%% The lineno packages adds line numbers. Start line numbering with
%% \begin{linenumbers}, end it with \end{linenumbers}. Or switch it on
%% for the whole article with \linenumbers after \end{frontmatter}.
%% \usepackage{lineno}

%% natbib.sty is loaded by default. However, natbib options can be
%% provided with \biboptions{...} command. Following options are
%% valid:

%%   round  -  round parentheses are used (default)
%%   square -  square brackets are used   [option]
%%   curly  -  curly braces are used      {option}
%%   angle  -  angle brackets are used    <option>
%%   semicolon  -  multiple citations separated by semi-colon
%%   colon  - same as semicolon, an earlier confusion
%%   comma  -  separated by comma
%%   numbers-  selects numerical citations
%%   super  -  numerical citations as superscripts
%%   sort   -  sorts multiple citations according to order in ref. list
%%   sort&compress   -  like sort, but also compresses numerical citations
%%   compress - compresses without sorting
%%
%% \biboptions{comma,round}

% \biboptions{}

%% This list environment is used for the references in the
%% Program Summary
%%
\newcounter{bla}
\newenvironment{refnummer}{%
\list{[\arabic{bla}]}%
{\usecounter{bla}%
 \setlength{\itemindent}{0pt}%
 \setlength{\topsep}{0pt}%
 \setlength{\itemsep}{0pt}%
 \setlength{\labelsep}{2pt}%
 \setlength{\listparindent}{0pt}%
 \settowidth{\labelwidth}{[9]}%
 \setlength{\leftmargin}{\labelwidth}%
 \addtolength{\leftmargin}{\labelsep}%
 \setlength{\rightmargin}{0pt}}}
 {\endlist}
\begin{document}

\section{Instructions for the user}\par
Before running the O-SUKI code, the user must set the target pellet and HIB parameters accordingly as follows:   \\
\begin{small}%enumerate}
%
{(a)\em OK3 code calculation type:} In 3D O-SUKI-N code, one can select the OK3 illumination code calculation type. The $OK\_Swich = 1$ is the full calculation with OK3. The $OK\_Swich = 5$ is the 1D uniform energy distribution type, and the HIB's energy distribution changes only in the radius direction. The $OK\_Swich = 10$ is the 1D energy distribution with the illumination non-uniformity in the $\theta$ and $\phi$ directions. One can add artificially non-uniformity in the $\theta$ and $\phi$ directions in ''main\_LC.cpp''\\
%
{(b)\em Projectile ion type:} Five projectile ion types are included in OK3—Pb, U, Cs, C and p. Users can choose one of them or add other species expanding the arrays aZb and aAb in "Input\_LC.h".\\
{(c)\em Ion beam parameters:} The user can specify the HIB radii on the target surface changing the parameter $tdbrc$ in "input\_LC.h". The design of the beam input pulse is also done in the same file. The pulse rise start time, rise time, and beam power are set by variables $t\_beamj$, $del\_t\_beamj$, and $Powerj (j=1\sim5)$, respectively. Users should also input the total input beam energy into $input\_energy$ in the "define\_ECSH.h" manually, when the users want to run the Euler code independetly. As the parameter value of the wobbling beam, the maximum radius of the beam axis trajectory in the rotation and the oscillating frequency should be specified. Users can set the desirable values for the maximum beam trajectory radius $rRot$ in the "InputOK3.h" and the rotational number $rotationnumber$ in the "input\_LC.h".\\
{(d)\em The beam irradiation position:} The file HIFScheme.h contains 1, 2, 3, 6, 12, 20, 32, 60 and 120-beam irradiation schemes. Users can choose one of them or add other HIB irradiation schemes supplementing the file.  \\
{(e)\em The reactor chamber:} Users can specify the chamber radius by changing the parameter of $Rch$. The parameter $dz$ fixes the pellet displacement from the reactor chamber center in the Cartesian PS coordinates (see Fig. \ref{misalign}). In OK3 the target alignment errors of $dx, dy$ and $dz$ can be specified. One can change this setting in the "input\_LC.h".\\
{(f)\em The target pellet structure and mesh number:} The parameter values of target are set in "input\_LC.h" and "init\_LC.h". In "input\_LC.h", users can change the boundary radius of each layer, the total mesh number and the mesh number for each layer. The present O-SUKI-N 3D includes an example DT-Al-Pb structure target defined by target layer-thickness parameters: $Rin, Rbc1, Rbc2$ and $Rout$. Users can add other target materials by expanding the arrays of $aZt0, aZtm, aAt, aUi, aro$ and $SC$ in "InputOK3.h". \\
{(g)\em The maximal Euler mesh number:} The maximal Euler mesh number is set in ''input\_LC.h''. The upper limit of the mesh number should be defined depending on the resource limitation of the workstation used.


\begin{figure}[H]
\centering
\includegraphics[width=5cm]{images/misalign2.eps}
\caption{Schematic diagram for the target misalignment}\label{misalign}
\end{figure}

If users want to employ a new substance for target structure, usera also need to add the solid density and the atomic mass in "CONSTANT.h". Users can control the Lagrangian radial mesh number for each layer by changing the value $MWC$ in "Input\_LC.h". When $MWC$ = 0, the radial mesh width ($dR1 = dR2 = \cdots$) in all layers becomes equal. When the MWC is large, the radial mesh number for each layer ($num\_k1 = num\_k2 = \cdots$) becomes close to the same number.\\
\end{small}%enumerate}
\par
Users can run "CodeO-SUKI-N-fusion-start.sh" to start running the O-SUKI-N 3D code simulations. When the shell script is executed, the Lagrange fluid code, the data conversion code and the Euler fluid code are sequentially activated. The results of the Lagrangian simulation are saved in the "output" directory, and the results of the Eulerian simulation are saved in "output\_euler". 
%\end{document}