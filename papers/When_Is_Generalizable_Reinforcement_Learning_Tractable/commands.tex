\usepackage{amsmath,amssymb,graphicx,xcolor,mathtools,nicefrac}
\usepackage[shortlabels]{enumitem}
\newtheorem{theorem}{Theorem}
\newtheorem{lemma}{Lemma}
\newtheorem{example}[theorem]{Example}
\newtheorem{corollary}{Corollary}
\newtheorem{proposition}{Proposition}
\newtheorem{remark}{Remark}
\newtheorem{claim}[theorem]{Claim}
\newtheorem{assumption}{Assumption}
\newtheorem{condition}{Condition}
\newtheorem{property}{Property}


\usepackage{thmtools,thm-restate}

%\newtheorem*{theorem-non}{Theorem}
%\newtheorem*{claim}[theorem-non]{Claim}

\newenvironment{proof}{{\bf Proof.}}{$\Box$}
\usepackage{hyperref}


\usepackage{amsmath,amssymb}
\usepackage{verbatim,float,url}
\usepackage{graphicx,subfigure,psfrag}
\usepackage[numbers]{natbib}
\usepackage{xcolor}
\usepackage{microtype}
\usepackage{xparse}
\usepackage{xspace}
\usepackage{tikz,textcomp,thmtools,nameref,cleveref}
\usetikzlibrary{positioning}
\usepackage{tikz-qtree}

\usepackage{algorithm, algorithmic}


%\usepackage{algpseudocode,algorithmicx,algorithm}

\usepackage[font=small]{caption}



%%%%%% MACROS %%%%%%
\newcommand{\red}[1]{\textcolor{red}{#1}}
\newcommand{\green}[1]{\textcolor{green}{#1}}
\newcommand{\blue}[1]{\textcolor{blue}{#1}}


\newcommand{\dmcomment}[1]{{\bf{{\blue{{DM --- #1}}}}}}
\newcommand{\ylcomment}[1]{{\bf{{\green{{YL --- #1}}}}}}
\newcommand{\prcomment}[1]{{\bf{{\red{{PR --- #1}}}}}}


%\newcommand{\kkcomment}[1]{{\bf{{\green{{KK --- #1}}}}}}
%\newcommand{\kbcomment}[1]{{\bf{{\violet{{KB --- #1}}}}}}
%\newcommand{\orange}[1]{\textcolor{orange}{#1}}
%\newcommand{\dmcomment}[1]{{\bf{{\orange{{#1}}}}}}
%\newcommand{\dmstrike}[1]{{\bf{{\orange{\sout{{#1}}}}}}}



%\crefname{assumption}{assumption}{assumptions}
%\Crefname{assumption}{Assumption}{Assumptions}


\let\hat\widehat
\let\tilde\widetilde


\newcommand*\circled[1]{\tikz[baseline=(char.base)]{
            \node[shape=circle,draw,inner sep=2pt] (char) {#1};}}
            
            
            
            
\newcommand{\strprox}{Strong Proximity}
\newcommand{\weakprox}{Weak Proximity}

\DeclareMathOperator*{\argmax}{argmax}
\DeclareMathOperator*{\argmin}{argmin}
\DeclareMathOperator*{\arginf}{arginf}



\DeclareMathOperator{\Dir}{Dir}
\DeclareMathOperator{\Cat}{Categorical}

\DeclareMathOperator{\sgn}{sgn}
\DeclareMathOperator{\E}{\mathbb{E}}
\DeclareMathOperator{\F}{\mathcal{F}}
\DeclareMathOperator{\R}{\mathbb{R}}
\DeclareMathOperator{\Prob}{\mathbb{P}}
\DeclareMathOperator{\Like}{\mathcal{L}}
\DeclareMathOperator{\thetahat}{\widehat{\theta}}

\DeclareMathOperator{\StateSet}{\mathcal{S}}
\DeclareMathOperator{\ActSet}{\mathcal{A}}
\DeclareMathOperator{\Trans}{\mathcal{T}}
\DeclareMathOperator{\RandPol}{\pi_{\text{rand}}}
\DeclareMathOperator{\TvDist}{\mathbb{TV}}

\DeclareMathOperator{\rewvar}{\xi_r}
\DeclareMathOperator{\transvar}{\xi_{tr}}


\newcommand{\qone}{q_{\mathcal{D}}}
\newcommand{\qtwo}{q_2}

\DeclareMathOperator{\MdpFam}{\mathcal{M}}
\DeclareMathOperator{\FamDist}{\mathcal{D}}
\DeclareMathOperator{\OptPolFam}{\pi^*}

%\DeclareMathOperator{\oracle}{\ensuremath{\widehat{V}}}
%\DeclareMathOperator{\Rew}{R}
\newcommand{\Rew}{R}
\newcommand{\oracle}{\widehat{V}}

\DeclarePairedDelimiter\floor{\lfloor}{\rfloor}

\DeclareMathOperator{\Mtest}{M_{test}}

\newcommand{\V}[3]{%
V^{{#1%
}}_{{#2%
}}({#3})
}%

\newcommand{\approxV}[2]{%
\widehat{V}^{{#1%
}}_{{#2%
}}}

\DeclareMathOperator{\polsub}{\alpha}
\DeclareMathOperator{\solvesub}{\beta}

\newcommand{\mutinf}[2]{I \left( {#1} ; {#2} \right)}

\DeclarePairedDelimiterX{\infdivx}[2]{(}{)}{
  #1\;\delimsize|\delimsize|\;#2
}
\newcommand{\kld}[2]{\ensuremath{\mathbb{D_{KL}}\infdivx{#1}{#2}}\xspace}




