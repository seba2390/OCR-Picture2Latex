\subsection{Denial-of-service attack}\label{sec:denialofservice}

Patterson \cite{patterson2013vulnerability} presents ways to launch DoS attacks   on GPUs. To launch a DoS attack, the GPU  should be assigned a long-running task, which makes it unavailable for other system-tasks such as drawing the desktop. Since a GPU task cannot be pre-empted, the system becomes unresponsive. DoS attacks can be launched using graphics APIs, eg., DirectX,  OpenGL and WebGL. Using WebGL, they show a DoS attack   on the drawing function ({\tt gl.draw}), vertex and fragment shaders. To attack the drawing function, instead of calling it multiple times, it should be called only once and given a large amount of workload, e.g., many complex shapes. This is because, after each invocation of the drawing function, the control returns to CPU, which would foil the DoS attack. Further, the number of shapes should be small enough to fit in the GPU memory and avoid a GPU-crash, and large enough to make GPU unresponsive while rendering them. Similarly, to launch DoS attack on vertex and fragment shaders, an infinite loop can be written in their codes. Table \ref{Table:DoSAttacks}  shows the results of these attacks on various operating systems. 



\begin{table}[htbp]
\centering
\caption{Results of DoS attacks on different GPUs and operating systems \cite{patterson2013vulnerability}}
%\begin{tabular}{ |L||L|L|L| } \hline
\begin{tabular}{ |l|p{4cm}|p{5.5cm}|p{4cm}| } \hline
  & \multicolumn{1}{c}{Nvidia GPU} &  \multicolumn{1}{|c}{ATI GPU} &  \multicolumn{1}{|c|}{Intel GPU} \\
 \hline

 \multicolumn{4}{|c|}{\textbf{Results of flooding the  {\tt gl.draw} function}} \\\hline

 %\hline
 Windows XP &  Total system freeze & System freeze, then GPU recovery message & Not tested\\
 \hline
 Windows 7 & System freeze, then graphics driver reset & System freeze, then graphics driver reset. Occasional total system freeze. & System freeze, then graphics driver reset.\\
 \hline
 Mac OS X & Total system freeze & Total system freeze & Not tested\\
 \hline
  Red Hat Linux & \multicolumn{3}{G|}{System freeze, then graphics driver reset} \\\hline
\multicolumn{4}{|c|}{\textbf{Results of the attack on vertex/fragment shader}} \\\hline
 Windows XP &  Total system freeze & System freeze, then GPU recovery message & Not tested\\
  \hline  
Windows 7 & \multicolumn{3}{G|}{System freeze, then graphics driver reset} \\
    \hline
 Mac OS X & Total system freeze & Total system freeze & Not tested\\
 \hline
 Red Hat Linux & \multicolumn{3}{c|}{Vertex shader: total system freeze. Fragment shader:  hang, then graphics driver reset}  \\
 \hline

\end{tabular}
\label{Table:DoSAttacks}
\end{table}


As for countermeasures, they note that some OSes use timers to ascertain  when the GPU stops responding and then reset the driver to reclaim GPU. However, they observe that the implementations of these drivers are not perfect and may lead to system crash, e.g., after five timeouts, Windows system crashed completely. Linux OS could not detect the attack on vertex shader and Mac OS X was found to have no protection against this attack.  They also recommend some mitigation strategies for future GPU systems. First, by performing static analysis, the runtime can be estimated and if it exceeds a threshold, the kernel launch can be prohibited. Second, by allowing simultaneous execution of multiple tasks on GPUs, some resources can be ensured for OS processes, even if other task stops responding. Finally, resetting should be performed in time to avoid complete system crash.
 

  

