\section{Background and Motivation}\label{sec:background}
\subsection{Terms and concepts}\label{sec:terms}

We now introduce some concepts which will be used throughout this paper.  We refer the reader to previous works  for a detailed background on  
GPU architecture \cite{mittal2016SurveyGPURF,mittal2014surveyGPUcache,mittal2017design} and AES encryption algorithm \cite{kadam2018rcoal}. 
  
 

\textbf{{\tt cudaContext}}: It is a data type which stores device configuration during runtime such as error codes, loaded modules containing device code and allocated device memory. These parameters are useful for controlling the device. 
  
\textbf{{\tt cl\_mem:}} In OpenCL execution model, a memory object is a handle to a portion of  global memory. {\tt cl\_mem} is an object type used for describing the memory objects \cite{opencl_clmem}.


\textbf{Shader:} A shader refers to a program which processes a particular stage of a graphics engine for calculating rendering effects. For example, a fragment shader is a shader stage which processes fragments into multiple colors and a depth value. The vertex shader is another stage which processes individual vertices. 

\textbf{Memory access coalescing:} It refers to merging of memory requests of different threads of a single warp into as few cache lines as possible. 

\textbf{Soft and hard booting:} In soft (or warm)  rebooting, the system is restarted without turning off the power supply to the system. In hard (or cold) rebooting, the power to the system is first turned-off (i.e., shut-down) and then turned-on which results in rebooting the system.

\textbf{SMM:} SMM is a special operating mode in x86 CPUs which is used for performing tasks related to system-control, e.g., hardware management \cite{kim2016demand}. SMM can provision a memory portion which can be accessed only by SMM and not hypervisor or OS kernel. Further, in SMM, only a trusted program, namely ``system management interrupt'' (SMI) can run. All other tasks, including any malicious task, are suspended and they cannot interrupt the SMI program. Note that a hypervisor is a firmware which launches and runs virtual machines.

\textbf{DMA:} DMA allows certain devices to directly access main memory without requiring the intervention of CPU. 
  
\textbf{Buffer overflow:} Buffer overflow refers to writing data outside the boundary of the buffer to the adjacent locations. It can lead to program crashes, data corruption, and security breaches. For example, stack overflow by a thread can impact execution of other threads by overwriting other memory spaces.  

\textbf{Canary:} In memory management, a canary is the memory location which does not store useful data, and is generally placed just before the return address of a function. An adversary generally attempts to cause a buffer overflow to overwrite the return address for redirecting the execution to a malicious code. However, use of a canary ensures that the buffer overflow also overwrites the canary value. Then, before using the return address, the canary can be  checked. If it has been changed, overwriting of the return address can be easily detected. 

\textbf{Covert and side-channel:} A channel refers to a medium through which sensitive data is leaked. If the channel is hidden, it is termed as ``covert channel''. A covert channel is created intentionally and is not otherwise meant for communication. The adversary tries to conceal its existence from the victim.   A ``side-channel'' is created incidentally, where the adversary gets sensitive information from the characteristics of the system's operation \cite{wang2006covert}. For example,  if the timing/power values of the encryption algorithm depend on the key, then, based on the timing/power measurements, an adversary can guess the key used in encryption.  In a side-channel, there is no communication, but only leakage of sensitive information through the side-channel. 

 

\textbf{Denial-of-service attack:} In a DoS attack, the adversary tries to make the device unavailable by temporarily or permanently hampering its services. This may be done by overloading the system with useless requests which prohibits handling of genuine requests from a benign user. Due to this, a DoS attack can be detected, which is different from other attacks, such as side-channel attack, where the system is generally not harmed and hence, no evidence of the attack remains.
 

 

 
  
