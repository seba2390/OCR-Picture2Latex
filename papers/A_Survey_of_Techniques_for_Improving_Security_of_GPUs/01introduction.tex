\section{Introduction}\label{sec:introduction}

The computing industry is currently at an interesting inflection point. GPUs, which were originally used for a narrow range of graphics (e.g., video-gaming) applications, are now spreading their wings to a broad spectrum of compute-intensive and mission-critical applications, most notably, cryptography, finance, health, space and defense. After passing the initial `rounds'   of scrutiny on the metrics of performance and energy \cite{mittal2015gpupowermgmtsurvey,mittal2015cpugpusurvey},  it is time that GPUs face and pass the test on the metric of security, which is especially crucial in mission-critical applications.

In fact, a recent incident has strongly highlighted the need and even urgency of  improving GPU security. A malicious person hid a bitcoin miner in ESEA (a video game service) software \cite{eseaBitCoin}. This miner used the GPUs in users' machines for mining bitcoin without their knowledge. The miner overheated and harmed the machines by overloading the GPUs. Thus, the malicious person earned cryptocurrency at the expense of the users' resources. This incident shows that the community can no longer afford to ignore GPU security considerations. 

While a large part of the conventional wisdom gained from CPU security research is also applicable to GPU security, both \textit{attacking} and \textit{securing} GPU present challenges of their own. Many attacks exploit the correlation between an event and its impact such as the change in latency, power consumption or number of memory accesses. However, due to its massively-parallel architecture and undocumented management policies, isolating individual events and their impact is generally not feasible. These very reasons along with the fast-evolving architecture of GPUs, also make it difficult to design effective security solutions for them \cite{zhu2017understanding}. These factors demand a careful study of GPU security. Several recent works reveal security vulnerabilities of GPUs along with their countermeasures. 


  
\textbf{Contributions:} In this paper, we present a survey of techniques for analyzing and improving the security of GPUs. Figure \ref{fig:overview} presents an overview of the paper. Section \ref{sec:background} first presents a brief background on important concepts and terms. Then, it highlights the need for improving GPU security and tradeoffs involved in protecting them.
After this, it presents a classification of research works and also summarizes their key ideas. In security jargon, attacks are divided into two types: ``passive attacks'' which leak system-information but do not the change the system, and ``active attacks'' which change the data or operation of the system. We review passive attacks in Sections \ref{sec:dataleakage} and \ref{sec:sidechannel} and active attacks in Section \ref{sec:malwareOverflowDoS}. Specifically, Section \ref{sec:dataleakage} reviews attacks for leaking sensitive information. Section \ref{sec:sidechannel} discusses side and covert-channel attacks. Section \ref{sec:malwareOverflowDoS} discusses GPU malware, buffer-overflow and DoS\footnote{The following acronyms are frequently used in this paper: advanced encryption standard (AES),  address space layout randomization (ASLR),  bandwidth (BW), covert channel (CC),  compute unified device architecture (CUDA), direct memory access (DMA), denial of service (DoS),   error correcting code (ECC),    functional unit (FU), global/local/shared memory (GlM/LoM/ShM), input output memory management unit (IOMMU), input output (I/O),   memory management unit (MMU),  peripheral component interconnect (PCI), parallel thread execution (PTX),  single instruction multiple thread (SIMT),  streaming multiprocessor (SM),  system management mode (SMM), shared virtual memory (SVM), virtual machine (VM).}  attacks. In these sections, we discuss a work under single category only, even though many of the works span across the boundaries. 

\begin{figure} [h]
\centering
\includegraphics[scale=0.40]{Overview-crop.pdf}
\caption{Organization of the paper }\label{fig:overview}
\end{figure}


\textbf{Scope:} For the sake of a clear presentation, we limit the scope of this paper as follows. We include only those techniques which have been proposed and evaluated in the context of GPUs, even though some of the security techniques proposed in the context of CPUs may also be applicable for GPUs. We do not include works which use GPU merely for accelerating a cryptography algorithm, but those that seek to leak the encryption key. We include works that attack GPU or run attack-code/malware on GPU. We believe that this survey will underscore the need of designing GPUs with security as the first principle rather than retrofitting for it.
 
 
 
