%\documentclass[linenumbers]{aastex62}
\documentclass{aastex62}
%% The default is a single spaced, 10 point font, single spaced article.
%% There are 5 other style options available via an optional argument. They
%% can be envoked like this:
%%
%% \documentclass[argument]{aastex62}
%% 
%% where the layout options are:
%%
%%  twocolumn   : two text columns, 10 point font, single spaced article.
%%                This is the most compact and represent the final published
%%                derived PDF copy of the accepted manuscript from the publisher
%%  manuscript  : one text column, 12 point font, double spaced article.
%%  preprint    : one text column, 12 point font, single spaced article.  
%%  preprint2   : two text columns, 12 point font, single spaced article.
%%  modern      : a stylish, single text column, 12 point font, article with
%% 		  wider left and right margins. This uses the Daniel
%% 		  Foreman-Mackey and David Hogg design.
%%  RNAAS       : Preferred style for Research Notes which are by design 
%%                lacking an abstract and brief. DO NOT use \begin{abstract}
%%                and \end{abstract} with this style.
%%
%% Note that you can submit to the AAS Journals in any of these 6 styles.
%%
%% There are other optional arguments one can envoke to allow other stylistic
%% actions. The available options are:
%%
%%  astrosymb    : Loads Astrosymb font and define \astrocommands. 
%%  tighten      : Makes baselineskip slightly smaller, only works with 
%%                 the twocolumn substyle.
%%  times        : uses times font instead of the default
%%  linenumbers  : turn on lineno package.
%%  trackchanges : required to see the revision mark up and print its output
%%  longauthor   : Do not use the more compressed footnote style (default) for 
%%                 the author/collaboration/affiliations. Instead print all
%%                 affiliation information after each name. Creates a much
%%                 long author list but may be desirable for short author papers
%%
%% these can be used in any combination, e.g.
%%
%% \documentclass[twocolumn,linenumbers,trackchanges]{aastex62}
%%
%% AASTeX v6.* now includes \hyperref support. While we have built in specific
%% defaults into the classfile you can manually override them with the
%% \hypersetup command. For example,
%%
%%\hypersetup{linkcolor=red,citecolor=green,filecolor=cyan,urlcolor=magenta}
%%
%% will change the color of the internal links to red, the links to the
%% bibliography to green, the file links to cyan, and the external links to
%% magenta. Additional information on \hyperref options can be found here:
%% https://www.tug.org/applications/hyperref/manual.html#x1-40003
%%
%% If you want to create your own macros, you can do so
%% using \newcommand. Your macros should appear before
%% the \begin{document} command.
%%
\usepackage{amsmath}
\usepackage{amssymb}
\usepackage{graphicx}
\usepackage{epstopdf}
\usepackage{bbold}
\usepackage{mathtools}
\usepackage[caption=false]{subfig}
\usepackage{xcolor}
\usepackage{hyperref}

\newcommand{\vdag}{(v)^\dagger}
\newcommand\aastex{AAS\TeX}
\newcommand\latex{La\TeX}
\newcommand{\ikt}{{\it Kepler}}
\newcommand{\ik}{{\it Kepler~}}

\newcommand{\kepler}{{\it Kepler}}

\newcommand{\gaia}{{\it Gaia~}}
\newcommand{\gaiat}{{\it Gaia}}

\newcommand{\blue}[1]{\textcolor{blue}{#1}}
\newcommand{\js}[1]{\textcolor{blue}{#1}}
\newcommand{\jsc}[1]{\textcolor{blue}{[Jason: #1]}}
\newcommand{\jsk}[1]{\textcolor{blue}{\sout{#1}}}

\newcommand{\teff}{\ensuremath{T_{\rm eff}}}
\newcommand{\logg}{\ensuremath{\log{g}}}
\newcommand{\feh}{[Fe/H]}
\newcommand{\rstar}{\ensuremath{R_\star}}
\newcommand{\mstar}{\ensuremath{M_\star}}
\newcommand{\rsun}{R\ensuremath{_\odot}}
\newcommand{\msun}{M\ensuremath{_\odot}}
\newcommand{\sigpm}{\ensuremath{\pm\sigma}}
\newcommand{\sigp}{\ensuremath{+\sigma}}
\newcommand{\sigm}{\ensuremath{-\sigma}}
\newcommand{\Ag}{\ensuremath{{\rm A}_{\rm g}}}
\newcommand{\Ab}{\ensuremath{{\rm A}_{\rm B}}}
\newcommand{\rprs}{\ensuremath{R_{\rm p}/R_\star}}
\newcommand{\rhostar}{\ensuremath{\rho_\star}}
\newcommand{\rhoc}{\ensuremath{\rho_{\star {\rm c}}}}
\newcommand{\rpl}{\ensuremath{R_{\rm p}}}
\newcommand{\mpl}{\ensuremath{M_{\rm p}}}
\newcommand{\lpl}{\ensuremath{L_{\rm p}}}
\newcommand{\rhopl}{\ensuremath{\rho_{\rm p}}}
\newcommand{\loggpl}{\ensuremath{\logg_{\rm P}}}
\newcommand{\teq}{\ensuremath{T_{\rm eq}}}
\newcommand{\esinw}{\ensuremath{e sin\omega}}
\newcommand{\ecosw}{\ensuremath{e cos\omega}}
\newcommand{\drho}{\ensuremath{d\rho_{c,i-j}}}
\newcommand{\adrs}{\ensuremath{a/R_\star}}

\newcommand{\rev}[1]{\bf{#1}}
%\newcommand{}[1]{{\bf{#1}}}%\newcommand{}[1]{{\color{blue}{#1}}}
%insert text as {inserted text} to highlight new text for the revision, then simply change the command to: \newcommand{}[1]{{{#1}}}  

%Journal Definitions 
\def\pnas{PNAS}

%% Reintroduced the \received and \accepted commands from AASTeX v5.2
\received{2022 May 10}
\revised{2023 September 30}
\accepted{2023 October 2}%\today}
%% Command to document which AAS Journal the manuscript was submitted to.
%% Adds "Submitted to " the arguement.
\submitjournal{\it The Planetary Science Journal}


%% The following command can be used to set the latex table counters.  It
%% is needed in this document because it uses a mix of latex tabular and
%% AASTeX deluxetables.  In general it should not be needed.
%\setcounter{table}{1}

%%%%%%%%%%%%%%%%%%%%%%%%%%%%%%%%%%%%%%%%%%%%%%%%%%%%%%%%%%%%%%%%%%%%%%%%%%%%%%%%
%%
%% The following section outlines numerous optional output that
%% can be displayed in the front matter or as running meta-data.
%%
%% If you wish, you may supply running head information, although
%% this information may be modified by the editorial offices.
\shorttitle{Orbital Periods and Other Properties of \ik Planet Candidates}
\shortauthors{Lissauer et al.}
%%
%% You can add a light gray and diagonal water-mark to the first page 
%% with this command:
% \watermark{text}
%% where "text", e.g. DRAFT, is the text to appear.  If the text is 
%% long you can control the water-mark size with:
%  \setwatermarkfontsize{dimension}
%% where dimension is any recognized LaTeX dimension, e.g. pt, in, etc.
%%
%%%%%%%%%%%%%%%%%%%%%%%%%%%%%%%%%%%%%%%%%%%%%%%%%%%%%%%%%%%%%%%%%%%%%%%%%%%%%%%%

%% This is the end of the preamble.  Indicate the beginning of the
%% manuscript itself with \begin{document}.

\begin{document}

\title{Updated Catalog of \ik Planet Candidates: Focus on Accuracy and Orbital Periods }

%% LaTeX will automatically break titles if they run longer than
%% one line. However, you may use \\ to force a line break if
%% you desire. In v6.2 you can include a footnote in the title.


\correspondingauthor{Jack J. Lissauer}
\email{ Jack.Lissauer@nasa.gov }

\author[0000-0001-6513-1659]{Jack J. Lissauer}
\affiliation{Space Science \& Astrobiology Division\\
MS 245-3\\
NASA Ames Research Center \\
 Moffett Field, CA 94035, USA}

\author[0000-0002-5904-1865]{Jason F. Rowe}
\affil{Department of Physics and Astronomy\\
Bishops University \\
2600 Rue College \\
Sherbrooke, QC J1M 1Z7,
Canada}

\author[0000-0002-6227-7510]{Daniel Jontof-Hutter}
\affiliation{Department of Physics\\
University of the Pacific\\ 
601 Pacific Avenue\\
Stockton, CA 95211, USA}

\author[0000-0003-3750-0183]{Daniel C. Fabrycky}
\affiliation{Department of Astronomy and Astrophysics\\
University of Chicago\\
5640 South Ellis Avenue\\
Chicago, IL 60637, USA}

\author[0000-0001-6545-639X]{Eric B. Ford}
\affiliation{Department of Astronomy and Astrophysics\\
525 Davey Laboratory\\
The Pennsylvania State University\\ 
University Park, PA 16802, USA}
\affiliation{Center for Exoplanets \& Habitable Worlds\\
525 Davey Laboratory\\
The Pennsylvania State University\\ 
University Park, PA 16802, USA}
\affiliation{Department for Astrostatistics\\
525 Davey Laboratory\\
The Pennsylvania State University\\ 
University Park, PA 16802, USA}
\affiliation{Institute for Computational \& Data Sciences\\
The Pennsylvania State University\\ 
University Park, PA 16802, USA}

\author[0000-0003-1080-9770]{Darin Ragozzine}
\affiliation{Department of Physics and Astronomy  \\
Brigham Young University\\ 
Provo, UT 84602, USA}

\author[0000-0003-2202-3847]{Jason H. Steffen}
\affiliation{Department of Physics and Astronomy\\
University of Nevada Las Vegas\\
4505 South Maryland Parkway\\
Las Vegas, NV 89154, USA}

%\author{Tsevi Mazeh}
%\affiliation{School of Physics and Astronomy\\
%Raymond and Beverly Sackler Faculty of Exact Sciences\\
%Tel Aviv University\\ 
%Tel Aviv 69978, Israel}

\author[0000-0002-7217-446X]{Kadri M. Nizam}
\affiliation{Department of Physics\\
University of the Pacific\\ 
601 Pacific Avenue\\
Stockton, CA 95211, USA}
\affiliation{Department of Astronomy and Astrophysics\\
525 Davey Laboratory\\
The Pennsylvania State University\\ 
University Park, PA 16802, USA}


\begin{abstract}
%
We present a new catalog of \ik planet candidates that prioritizes accuracy of planetary dispositions and properties over uniformity. This catalog contains 4376 transiting planet candidates, including 1791 residing within 709 multi-planet systems, and provides the best parameters available for a large sample of \ik planet candidates. We also provide a second set of stellar and planetary properties for transiting candidates that are uniformly-derived for use in occurrence rates studies. Estimates of orbital periods have been improved, but as in  previous catalogs, our tabulated values for period uncertainties do not fully account for transit timing variations (TTVs). We show that many planets are likely to have TTVs with long periodicities caused by various processes, including orbital precession, and that such TTVs imply that ephemerides of \ik planets are not as accurate on multi-decadal timescales as predicted by the small formal errors (typically 1 part in 10$^6$ and rarely $>10^{-5}$) in the planets' measured mean orbital periods during the \ik epoch. %{We identify multiplanet candidates that have periods that are too close to each other to remain stable, which we use to estimate the number of systems that are distributed between two blended stars as $\sim~2.3\%$ across \ikt's multiplanet systems, and we use an error of a factor of two in the estimated period of a planet candidate that was identified by stability considerations to estimate the number of orbital periods that are aliases of the true periods to be $\sim~0.39\%$; other evidence suggests that a somewhat larger fraction of planets with estimated orbital periods of $\lesssim 1$~day actually complete two or more orbits within this amount of time.} 
Analysis of normalized transit durations implies that eccentricities of planets are anti-correlated with the number of companion transiting planets. Our primary catalog lists all known \ik planet candidates that orbit and transit only one star; for completeness, we also provide an abbreviated listing of the properties of the %12 circumbinary planets discovered by \ik as well as the 
two dozen non-transiting planets that have been identified around stars that host transiting planets discovered by \ikt. 

\end{abstract}

\keywords{{\it Unified Astronomy Thesaurus concepts:} Exoplanets (498); Exoplanet catalogs (488); Transit photometry (1709); Exoplanet dynamics (490); Planetary theory (1258)}

\section{Introduction}\label{sec:intro}

NASA's \ik spacecraft monitored a single star field for four years during its prime mission, with a duty cycle of almost 90\%. The principal objective of the \ik Mission was to take a statistical census of planets having orbital periods of up to $\sim$~1 year. The \ik Project released eight catalogs of planet candidates found during the mission \citep{Borucki:2011a,Borucki:2011b,Batalha:2013,Burke:2014,Rowe:2015,Mullally:2015,Coughlin:2016,Thompson:2018}. Each new catalog used more sophisticated methods, and aside from the last one, each used more \ik data and listed more planet candidates than its predecessor. The Project's latter catalogs employed successively more automated procedures.%, eventually prioritizing objectivity and reproducibility over accuracy in order to provide planet candidate lists most suitable for studies of planetary occurrence rates.  

The primary goal of the \ik Project's final catalog of planetary candidates \citep{Thompson:2018}, often referred to as DR25 (which is an abbreviation of Data Release 25), was to produce a listing of planet candidates (PCs) found and vetted in a well-defined and reproducible manner for the exoplanet community to use as input for studies of planet occurrence rates (e.g., \citealt{Hsu:2019}).  As such, the data were processed in a highly automated manner, with uniformity and reproducibility prioritized over using all available information to identify and classify each individual potential planet signature.  {Previously found planet candidates that were not identified by the final search for transit-like patterns do not appear in the \cite{Thompson:2018} catalog. Furthermore, their vetting of individual candidates did not include the hands-on treatment (e.g., examination of lightcurves and centroid shifts by eye) that was a feature of the first six \ik planet catalogs.}

The neglect of hands-on vetting from previous catalogs ended up excluding dozens of \ik planets, (some of which are well-known and were verified with high confidence) from \cite{Thompson:2018}'s list of Kepler Objects of Interest (KOIs; the integer number refers to the target star, whereas the digits after the decimal point refer to the putative planet), and led to classifying others as false positives (FPs) rather than as viable planet candidates. One particularly relevant example is that the \cite{Thompson:2018} catalog is more biased against planets exhibiting significant transit timing variations (TTVs) than are previous catalogs.  This limitation leads to a bias against planet pairs with near-resonant orbits, {as foreseen by \cite{Garcia:2011}. An unpublished analysis of strategies for detecting planets with large TTVs by A.~Moorhead \& one of us (EBF) found that most such planets would naturally be detected by applying standard transit search algorithms to $\sim$one-year-long subsets of the data, since large near-resonant TTVs typically accumulate over multi-year timescales.  Therefore,} the use of candidates from multiple searches conducted with different amounts of data reduces the bias against planets with substantial TTVs.  {This improvement in sensitivity to planets with large TTVs (i.e., TTVs comparable to or exceeding the transit duration) is most significant for planets that would, in the absence of TTVs, likely be detected using only 13 or 16 months of data (the durations searched for the second and third \ik PC catalog releases).  }%Planets with large TTVs that would only exceed the detection threshold by combining $\ge~2$ years of transits could easily be missed by .  

Our new catalog is more analogous to the \ik Project's final cumulative catalog, DR25supp, which was not presented in a refereed publication\footnote{Documentation for DR25supp is available at: https://exoplanetarchive.ipac.caltech.edu/docs/PurposeOfKOITable.html\#q1-q17\_sup\_dr25.}, than to the DR25 catalog. Both the DR25supp catalog and our Table \ref{tab:planetcatalog}  differ from the DR25 catalog in two key respects: the use of manual vetting and including KOIs from multiple sources, including previous \ik project catalogs. For DR25supp, the cumulative DR24 catalog was combined with the DR25 catalog and the \ik False Positive Working Group re-dispositioned all KOIs whose dispositions were disputed  in the last six project catalogs (listed as a planet candidate in at least one catalog but listed as a false positive in at least one other); apart from dispositions, the most recent properties were listed. We began with the  DR25supp catalog, added other KOIs from various sources, and manually vetted select KOIs, as described in Section \ref{sec:selection}. 

The \ik mission defined a threshold crossing event (TCE) as a periodic signature with multiple event statistic (MES) at least 7.1, where the MES is effectively the signal-to-noise ratio of the putative planetary transits in the folded lightcurve, as measured by the \ik pipeline. The threshold of MES $\geq 7.1$ was chosen to keep the expected number of KOIs resulting from white noise small (see \S\ref{sec:selection} for details). The \ik pipeline looked for planet candidates in the lightcurve of an individual target star one by one, beginning with the signal possessing the largest MES and then performing the search again on a lightcurve from which data at the times of the transits of this KOI were removed. The process was repeated until no additional signal with MES $\geq 7.1$ was found. Because data were removed from the lightcurve prior to searching for additional candidates, the search for multis (systems with multiple transiting candidates) is less complete than that for singles (systems with only a single transiting candidate). This bias against multis is analyzed quantitatively by \cite{Zink:2019}. 

Various groups have published lists of additional \ik planet candidates; see \S\ref{sec:selection} for details.
Together with those appearing in the official \ik planet candidate tabulations, the total number of \ik planet candidates listed in one or more catalogs is $\sim$~5000, although several hundred of these have subsequently been re-classified as FPs. More than 2700 of these planet candidates have been verified (either confirmed using radial velocity data or via TTVs, or statistically validated as having a high likelihood of being true planets) and assigned official \ik planet designations such as Kepler-11 g \citep{Lissauer:2011a}. 
 
 We assembled our catalog using data from the final \ik project planet candidate catalog \citep{Thompson:2018}, previous planet candidate catalogs produced by the \ik project, and planet candidate lists from other groups (\S\ref{sec:selection}). We incorporated the improved stellar properties derived using distance measurements by ESA's {\it Gaia} spacecraft \citep{Gaia:2018,Berger:2020a} and, where available, spectroscopic measurements taken with the Keck I telescope \citep{Petigura:2017,Fulton:2018}.  Figure \ref{fig:perrad} shows the distribution of orbital period vs.~radius for our \ik planetary candidates based on the work reported herein.
 
Our goals in compiling a new catalog of \ik planet candidates are to provide a comprehensive listing of KOIs with significantly more accurate vetting and to give improved estimates of planet properties. As described in more detail in Section \ref{sec:catalog}, our new listing relies on more homogeneous and robust techniques to compute planetary parameters, removing previous biases such as the dependence on orbital period estimates of planets exhibiting TTVs with the amount of data analyzed when they were first announced. We also list more planet properties and use more robust techniques to compute the values and uncertainties of estimated planetary characteristics.

{The primary advantages of using our new catalog are: We present the most complete listing of \ik planet candidates to date, based on the \ik project's catalogs, community efforts, and our own analysis. We have provided initial dispositions for new KOIs in our sample. We have also revisited dispositions for those KOIs that were dispositioned as FP in DR25supp despite being listed as a PC in DR24 or DR25, or having a Kepler number according to NASA Exoplanet Science Institute (NExScI).   We provide additional disposition cuts based on S/N, mass measured from radial velocity variations, and derived planetary radius, $R_p$. 

The  parameter values listed in our catalog (Table \ref{tab:planetcatalog}) are more accurate than those in previous catalogs, with significant effort to systematically and uniformly improve transit models and calculations of posteriors for model parameters, including corrections for bias in impact parameter and the mean-stellar density computed from the photometric model, \rhoc~\citep{Gilbert:2022}.  Orbital periods have been revisited and in some cases recomputed to address the complications of transit timing variations by providing the best fit constant period to transit times observed by \ikt.  We investigated multiplanet systems with suspiciously close-period planet candidates and corrected period-aliasing.   Significant improvements in stellar parameters from ground based followup and the parallax survey from \gaia have been incorporated \citep{Fulton:2018,Berger:2020a}.  This has allowed for planetary parameters to be derived in a more uniform manner than in other cumulative catalogs.  We include a concise description of all planetary characteristics listed (see \S\ref{sec:unified}) to allow for the community to maximize the combined knowledge of exoplanets in the \ik field-of-view.
}   
 

%-- bring in DR25 Supp, USP, machine learning and other searches (Bryson, Caceres)  - list all major sources either here or line 254 (6 lines up at present) - list can be split, as full list will appear in sec 2 

%-- review dispositions for all candidates.  new from community, assessment of PC from photometry ; from DR24+DR25 + DR25supp review candidates when change from PC -> FP and recover when justified.
%+ our own S/N and size cutoffs.  excessively close periods and periods extremely close to 2/1

%-clean up periods for candidates with very strong TTVs and uniform approach to period reporting not dependent on when PC first identified

For ease of reading, we often refer to planet candidates simply as ``planets''. Planets that are the sole transiting candidate of their host star are referred to as ``singles'', whereas the term  ``multis'' is used for both systems with more than one transiting planets and individual planets in such systems.  

We present our catalog of \ik planet candidates in Section \ref{sec:catalog}. In Section \ref{sec:singles}, we {characterize the sample of planet candidates, }compare the ensemble of planet candidates in multis to those in singles, compare planets in two-planet systems with those in higher-multiplicity multis, and quantify the reliability of the sample of multis as representing true planetary systems.  
%{  We have used our new catalog to investigate the uniformity of transit-modelled stellar density and impact parameters, which has allowed us to perform a study of 
%the distributions of eccentricities of subsets of the \ik exoplanet candidates (\S\ref{sec:ecc}). } 
{We investigate the distribution orbital eccentricity for various subsets of \ik planet candidates (\S\ref{sec:ecc}), improving on previous studies thanks to 
the enhanced accuracy of the stellar densities and impact parameters in our catalog.   We find significant changes in the eccentricity distribution as a function of the inferred size of the planet candidates (\S\ref{sec:eccsize}) and the number of \ik planet candidates detected around a given host star (\S\ref{sec:eccmultiplicity}). We also find significant differences between planet candidates with orbital periods less than 6 days and those with longer orbital periods (\S\ref{sec:eccperiod}).}
We  {consider various factors that can lead to} orbital period variations of \ik planets   (Section \ref{sec:periodtheory}). {We show that planets on eccentric orbits have variations in the times between successive transits on time scales much longer than the 4-year duration of the \ik mission  (\S\ref{sec:precession}). Section \ref{sec:individual} analyses long-term variations in mean orbital periods of planets  within  several \ik systems  showing significant TTVs that have been solved for dynamically.}  We conclude the main text by summarizing our principal results in Section  \ref{sec:conclusions}.

We select which objects to include in our catalog of planet candidates and list their properties to maximize accuracy on an object-by-object basis.  Therefore our selection criteria and various planetary properties are not homogeneous, and \emph{our planet candidate list is not appropriate for use as input for planetary occurrence rate calculations}.  Nonetheless, some aspects of our derivation of planetary properties provide estimates that are more accurate and at least as uniform as those found in previous studies. Therefore, we also present a second set of planetary properties using a uniformly-derived set of stellar parameters in Table \ref{tab:planetcatalog} and outline a process for utilizing some of the information tabulated therein for studies of occurrence rates in Appendix \ref{sec:App_ORtables}. Our primary planet candidate catalog (Table \ref{tab:planetcatalog}) is restricted to transiting planets orbiting just one star.  An abbreviated catalog of non-transiting planets found (using TTVs and/or radial velocity measurements) around stars with transiting \ik planets %and of transiting \ik circumbinary planets are 
is provided in Appendix \ref{sec:App_nontransiting}. 

\smallskip

\section{Planet Catalog} \label{sec:catalog}

In this section, we introduce our catalog of \ik planet candidates and describe the calculation of properties of the planets, as well as the sources used for characterizing their host stars.  Figure \ref{fig:perrad} displays the  radius vs.~period distribution of the planets in our \ik catalog and highlights the abundance of multiplanet systems discovered.  Multiplanet systems provide a special opportunity to study the potentially rich dynamical history of exoplanet formation and evolution.  Our construction and review of the \ik exoplanet catalog focuses on orbital periods and the prevalence of multi-planet discoveries.

We discuss stellar parameters in \S\ref{sec:star}, transit models in \S\ref{sec:transitmodel}, candidate selection and catalog unification in \S\ref{sec:selection} and the calculation and interpretation of orbital periods in \S\ref{sec:periods}. Our planet catalog is presented in \S\ref{sec:unified}.  Weaknesses of this catalog and its previous incarnations are discussed, including the impact of transit timing variations, relationships between the historical \ik catalogs, and the non-uniformity of candidate selection and biases of some derived properties. Section \ref{sec:2433} focuses in on the KOI-2433 system, which now has seven planet candidates. 

\subsection{Input Stellar Properties}\label{sec:star}

In preparing this catalog (Table \ref{tab:planetcatalog}) and throughout our study, we take stellar properties for the hosts of more than 99.7\% of the planet candidates from one of three sources.  When available, we select parameters from the latest catalog provided by California \ik Survey (CKS){\footnote{Stellar parameters for KOI-1792 were chosen from \cite{Berger:2020a} despite the availability of CKS parameters for that target for reasons specified in \S\ref{sec:selection}.}}, which lists properties of \ik planet hosts that have both spectral measurements from the Keck I telescope and well-determined \gaia properties, especially distances \citep{Fulton:2018}.  This list includes $\sim 60$\% of the hosts of multis as well as $\sim 60$\% of the single planet hosts brighter than \ik magnitude $Kp = 14.2$, but fewer than 6\% of the fainter hosts of singles. For stars with \gaia parallaxes/distances that were not included in the CKS sample, we use properties from  \cite{Berger:2020a}, which includes $\sim 95\%$ of the \ik targets; this list accounts for most of the remaining planet hosts.  For stars absent from both catalogs, we use the stellar parameters listed in \ik DR25 \citep{Thompson:2018}.  

Parameters for KOI-3206 were obtained from the \gaia online archive. We adopted custom parameters, as described in the following paragraph, for stellar hosts in two binary star systems.  No useful data were found for KOIs 2324, 4713, 5226, 5718, so we have adopted solar parameters with large uncertainties ($R_\star$ = $1\pm1$~\rsun, $M_\star = 1\pm1$~\msun, \logg = $4.5\pm4.5$) for these planet-hosting stars. 

%There are data on the following: teff_val teff_percentile_lower teff_percentile_upper
%radius_val, radius_percentile_lower, radius_precentile_upper

%KOI-3206 (useful since Rstar > Rsun)
%Teff: 5263.4, 5070.0, 5449.83 R: 2.2199218 2.0706396 2.392514

%KOI-4713 (not as good, just a poor constraint on Teff)
%Teff: 5248.25 4425.0 6876.0 ()
% This ranked list of three source catalogs provides us with stellar properties for almost all of the planets listed, and they are the best available properties of the vast majority of \ik planet host stars. 

Transit depths for both KOI-119 (Kepler-108) and KOI-284 (Kepler-132) suffer substantial dilution due to stellar companions.  For the case of KOI-119, we adopt the nominal dilution of 69.9\% as reported in \cite{Mills:2017a} for their (preferred) mutually inclined solution.  Observations indicate the KOI-284 system consists of 2 nearly identical stars with a total of 4 known transiting exoplanets.  From orbital stability considerations, the orbital periods of KOI-284.02 (6.41 days) and KOI-284.03 (6.17 days) are inconsistent with these planets orbiting the same star.  Thus, KOI-284 represents the special case of a {\it split multi}; see \S\ref{sec:falsemultis} for more details on this system and other \ik split multis.  For computing the planetary parameters presented in this paper, we adopted a dilution of 50\% for the KOI-284 transit models, which implies that half of the light in the photometric aperture is due to the companion star.
 
\subsection{Transit Models}\label{sec:transitmodel}

The calculation of transit models and the preparation of data products is worth reviewing in the context of potential biases and providing motivation for future improvements of the \ik catalog and its legacy value.  For all transit models, we use Presearch Data Conditioning (PDC) lightcurves \citep{Stumpe:2014} as reported in DR25.  {Observations with a quality flag set for any of bit 1,2,3,4,6,7,9,13,15,16,17 were rejected from our analysis for the reasons described in Table 2.3 of the {\it Kepler Archive Manual} \citep{Thompson:2016}.}  PDC lightcurves were prepared for transit analysis by detrending using a second-order polynomial filter with running window of 5~days.  The width of the running window was always truncated to avoid gaps larger than 10 long cadence (30-minute) observations of valid \ik photometry.  Thus, the filter does not cross large gaps in \ik photometry that arise from monthly data downloads or quarterly spacecraft rotations, and it avoids problems with significant jumps in the reported photometric flux from thermal settling of the spacecraft after attitude adjustments that are not fully captured by PDC \citep{vanCleve:2016}.  Observations within 1 transit duration of the mid-transit time for each observed transit were excluded from the polynomial fit.  Thus, the photometric baseline of in-transit data was interpolated based on out-of-transit observations only.  Outliers in the detrended data were identified and removed; outliers are defined herein as single long-cadence photometric measurements more than five standard deviations away from the mean after removal of a best fit transit model.

%Best fit models were calculated using a Levenberg-Marquardt (LM) non-linear, least-squares method with numerical derivatives \citep{More:1980}

The DR25 transit models and updated models for this paper use the same software (\texttt{TRANSITFIT5} transit modeling software; \citealt{Rowe:2015,Rowe:2016}) and techniques for parameter estimation and the calculation of posteriors.  Additional details can be found in \cite{Rowe:2014, Rowe:2015, Thompson:2018}.  Briefly, a multi-planet transit model was calculated for each lightcurve and used to isolate the transits for each individual planet in the system by subtracting the model with the depth set to zero for the planet of interest.  This lightcurve was then used to fit each KOI separately with a photometric transit model using the analytic quadratic limb-darkening model from \cite{Mandel:2002}. Limb-darkening coefficients are based on the tables of \citet{Claret:2011} and were fixed to values used in the DR25 KOI catalog.  The photometric model parameterization uses the mean stellar density (\rhostar), fixed quadratic limb-darkening coefficients, photometric zero point, the mid-transit time ($T_0$), orbital period ($P$), impact parameter ($b$) and scaled planet radius (\rprs). Eccentricity was fixed at zero for these models; thus, \rhostar~is replaced by \rhoc.  The adoption of mean stellar density as a fitted parameter assumes the mass of the host star is much larger than the combined mass of the transiting planet and any other planet(s) orbiting closer to the star, whether transiting or not.  Errors from TTVs for specific systems (see additional discussion below) were corrected by adjusting the observation times based on a linear interpolation of measured center of transit times (TTs) to create an aligned ephemeris.  {We calculated the center of transit times for each observed transit by fitting two transit durations of \ik photometric data centered on the predicted or pre-calculated time of each observed transit seeded with the best fit transit model, and only allowing the mid-transit time to vary.}  Biases in TTs can be introduced from overlapping transits as the multi-planet models used for lightcurve preparation do not simultaneously fit transit parameters and center-of-transit times.

\begin{figure}[!hbt]
\includegraphics[scale=1.0,angle=0]{rvsp_line17c.pdf}
\caption{Orbital period vs.~radius of \ikt's planetary candidates.  Those planets that are the only candidate for their given star are represented by black dots, those in two-planet systems as dark blue circles, members of three-planet systems as green triangles, those in four-planet systems as light blue squares, in systems of five PCs as yellow five-pointed stars, with six PCs as orange six-pointed stars, the seven PCs associated with KOI-2433 as pink seven-pointed stars, and the eight planets orbiting KOI-351 (Kepler-90) as red eight-pointed stars. The legend lists the number of stars hosting each multiplicity. The planetary candidates are listed in Table \ref{tab:planetcatalog} (First letter of disposition = ``P''). % and also obey the cuts S/N~$>8$ and $b<1$. 
Non-transiting planets (listed in Table \ref{tab:nontransiting}) and circumbinary planets are not included. Planet candidates only observed to transit once (mono-transits) are not plotted because their orbital periods are highly uncertain (see item \#4 in the list presented in Section \ref{sec:unified}), but they are accounted for in the multiplicity designation of their companion planets and the total numbers of systems of each multiplicity given in the lower right of the figure. KOI-846.01, with a radius $R_p=30.043$ R$_\oplus$ (see \S\ref{sec:selection}), falls outside the plotting window. All planets to the right of the dotted gray vertical line at $P = 730$ days (as well as some of the planets with shorter periods) transited only twice during the \ik mission, and therefore were not detected by the standard \ik pipeline, which required a minimum of three transits for a detection.  It is immediately apparent that there is a paucity of giant planets in multi-planet systems, especially giants with short orbital periods $P<15$ days.  The upward slope in the lower envelope of the plotted points is caused by the low S/N of small transiting planets with long orbital periods, for which few transits occurred during the time intervals that \ik observed. Adapted from a previous figure generously provided by Rebekah Dawson. }
\label{fig:perrad}
\end{figure}

We adopted Markov Chains calculated for DR25, apart from KOI PCs with large impact parameters ($b > 1$) and a few other KOIs that required model updates.  The DR25 Markov Chain Monte Carlo (MCMC) sampler assumed a non-informative prior for the impact parameter, which works well for non-grazing transits.  However, when $b > 1$, the minimum value of scaled planetary radius, \rprs, required for a planetary transit grows linearly with $b$.   Uniform sampling would result in a bias towards very large impact parameters, which are unphysical for planetary transits of stars.  To sample correctly, a prior was introduced to disfavor large impact parameters by de-weighting the model likelihood by $b^2$ when $b > 1-$\rprs, i.e., multiplying the prior by $b^{-2}$ for such regions of parameter space and not allowing $b>10$.  We computed new MCMC models for all KOI PCs with $b > 1$  and used these models to calculate the values presented in Table \ref{tab:planetcatalog}.  Some KOIs in our cumulative catalog required updated best fit models, and new MCMC runs to allow for sufficient sampling of low $b$ parameter space.  New models are presented for the following KOIs: 1681.02, 1681.03, 1681.04, 2092.03, 2398.01, 2474.01, 2578.01, 2604.01, 2695.01, 2775.01, 2919.01, 2933.01, 3013.01, 3130.01, 3384.01, 3572.01, 3853.01, 4007.01, 4034.02, 4035.01, 4056.01, 4345.01, 4498.01, 4528.01, 4625.01, 4632.01, 4670.01, 4743.01, 4778.01, 4782.02, 4838.02, 4886.01, 4890.01, 5804.01, 5831.01, 6103.02, 6941.01, 7368.01 as well as for all of the new KOIs listed in Section \ref{sec:selection}.

The development of the \ik catalog introduced a few quirks that biased some of the reported best fit parameters. As the \ik mission progressed, transit modelling techniques were improved and the methodology of reporting parameters changed, such as the choice of reporting either maximum likelihood vs.~mode or median from Markov Chains without much consistency between iterations.  A more insidious consequence from model evolution were biases that resulted from how subsequent-generation models were initialized based on the ancestral adaptation of previous models as a reference starting point.  Prior to the \cite{Mullally:2015} catalog, each update introduced new models that incorporated new photometry, resulting in longer time-series observations that ideally should produce more accurate measurements of periods, transit-depths and overall better fidelity.  However, there are two identifiable deficiencies from this approach: a bias towards extreme values of impact parameter due to the nature of \ik observations and excessive dependence on the average period measured up to the time that the KOI was initially announced when appreciable TTVs occur.

In general, impact parameters are not well measured for \ik planets.  This is due to the low S/N of a majority of \ik discoveries and the long (30-minute) cadence, which is comparable to the ingress and egress durations of the typical observed transit.  Combined with a potentially simplistic limb-darkening model, it is common to see the posterior distribution of $b$ from the transit model skew towards 0 or 1.  From a probabilistic view, the data for most \ik transits are insufficient to confidently distinguish a model with $b=0$ from models with $b \sim 0.5$.  The evolution of the \ik models has meant that over time, a large number of models would always be initialized near the boundary of allowable values of $b$.  Using least-square methods such as Levenberg-Marquardt (e.g., \citealt{More:1980}), which explores the local gradient of the model parameter value, can result in many models remaining near $b=0$ when initialized there.  The solution to the problem is to recognize that distributions of model parameters are more robust when sampling from estimates of posteriors based on methods such as MCMC.   For this paper, we explicitly note how all model parameters are reported in \S\ref{sec:unified}. Additional improvements may result from improved modelling of limb-darkening in future studies.

Figure \ref{fig:b_hist} compares the impact distribution of PCs presented in Table \ref{tab:planetcatalog} (orange) against that in DR25 (\citealt{Thompson:2018}; green). The DR25 catalog shows a pronounced excess of model fits with $b\sim0$ and $b>1$. Our efforts to reinitialize fits and include priors de-weighting large impact parameters produces a distribution of impact parameters more consistent with an isotropic inclination distribution \citep{Kipping:2016}, although there are still excess populations near $b = 0$ and for $b \gtrsim 1$.  

\begin{figure}[!hbt]
\includegraphics[width=0.8\textwidth,angle=0]{bhist_20230907.pdf}
\caption{{Comparison of the impact parameter ($b$) distributions for planet candidates from this work (orange) to the values listed for planet candidates in DR25 (green).  Only planetary candidates with $b<1.2$ are shown; PCs with $b>1.2$ represent $0.5\%$ of our sample and $1.4\%$ of the DR25 PCs. Restricting the samples to those KOIs that are classified PCs in both catalogs does not substantially alter either of the distributions. %The number of KOI PCs with $b>1.2$ is 58 for DR25 and 23 from our sample.23/4376= 0.005256; 58/4034 = 0.0143778
}
}\label{fig:b_hist}
\end{figure}

As recognized by \cite{Newton:1687}, the orbital period of a planet is, in general, not constant.  The orbital periods observed by \ik are the mean values that were observed during the 4-years of the primary mission, at least to a good approximation.  In the case of systems with large TTVs that were discovered early in the mission, the evolution of the models can result in the period reported in catalogs published by the \ik mission being only valid for the duration of data used for the initial discovery and characterization.  This occurred because the transit models assumed a non-interacting Keplerian orbit.  TTVs were handled by explicitly measuring the transit time of each individual event, then re-sampling the time-stamps with linear interpolation based on the TT of each event to be aligned.  Resampling used the observed-minus-calculated (O-C) values with the calculated transit time based on the reported mean period from the best fit model.  As the transit models evolved with each new data release, the transit times and O-C values were measured based on the transit model from the previous catalog. The new TTs were incorporated and the model updated.  Since the O-C values were fixed when the transit model parameters were updated, any significant change in the orbital period was captured in the O-C, not the reported transit model period.  Our solution to this problem is to calculate the mean orbital period, {\it as observed by \ikt}, directly from a straight-line fit to the measured times of each transit.  An O-C diagram (O-C vs.~time) should have no significant slope.  If a single planet demonstrated clear TTVs, then TTVs were typically calculated and included for all planets in the system.  See \S\ref{sec:unified} for an explicit description of how each model parameter is reported, \S\ref{sec:periods} for extended discussion of orbital periods during the \ik epoch, and \S\ref{sec:periodtheory} for an analysis of period variations over longer time scales.

Transit timing variations are commonly observed when two or more planets interact dynamically \citep{Agol:2005,Holman:2005}.  Other potential causes of TTVs include stellar binarity and astrophysical effects such as activity and star spots. (The latter two processes do not cause variations in acutal times of transit, but their observational signals can mimic TTVs.) The process by which TTVs have been accounted for in previous catalogs was inhomogeneous and overall {\it ad hoc}.  The primary criteria for selecting the solution with TTVs included was either the visual identification of TTVs from examination of O-C diagrams or for the inclusion with specific KOIs for the detailed study of individual systems.  For example, KOI-8298.01 is reported to use TTVs in the model but has a period less than 0.2 days.  The model with TTVs shows a significantly deeper transit, hence inclusion of TTVs.  However, this may indicate that either KOI-8298.01 is not a transiting planet (e.g., a manifestation of stellar variability) or that the apparent times of transit are significantly affected by spot crossings.  Thus, the TTV flag is not a definitive indication of whether or not TTVs are present and should not be used as evidence for the validity of a KOI being a true planet.  
%We do not attempt to rectify this situation, but instead our goal is to report which transit models have had TTVs included, how TTVs are included in the models and to pull tabulations from the published TTV catalogs of \cite{Holczer:2016} and \cite{Kane:2019}. 
The first digit of the TTV flag merely reports which KOIs have transit models that include TTVs and the second and third digits pull tabulations from the published TTV catalogs of \cite{Holczer:2016} and \cite{Kane:2019}, respectively. These external catalogs provide excellent assessments of TTVs for most KOIs.  

The S/N of the sum of all observed transits was calculated via
\begin{equation}\label{eq:S/N}
    {\rm S/N} = \sqrt{\sum^n_{i=1}\left ( \frac{m_i - 1}{\sigma}  \right )^2 },
\end{equation}
where $m_i$ is the calculated flux from the fitted transit model at each observation $i$ with a total of $n$ observations, and $\sigma$ is the standard deviation {based on out-of-transit observations using detrended PDC photometry with outliers removed}. The model is scaled to have the out of transit flux equal to unity.  Figure \ref{fig:perrad_snrmap} shows the complex dependence of the S/N of the population of exoplanet transit signatures on planetary radius and orbital period.  The trends in S/N are further discussed in \S\ref{sec:unified}.

\begin{figure}[!hbt]
\includegraphics[width=0.49\textwidth,angle=0]{RpvsPerwSNRmap.pdf}
\includegraphics[width=0.49\textwidth,angle=0]{Kepler_avgSNRperTransit.pdf}
\caption{Median signal to noise ratio for \ik planet candidates as a function of location in the period-radius plane. Each square represents a factor of $\approx 1.1$ in radius and $\approx 1.2$ in period. The left panel shows the total S/N, and the right panel shows the average (mean) S/N per transit.  The average S/N was calculated by dividing the total S/N by the square root of the number of observed transits for each planet candidate. Note the differences between the color scales of the two panels. The number of planets represented in each colored square ranges from 1 to 32. 
}\label{fig:perrad_snrmap}
\end{figure}

Figure \ref{fig:mean_stellar_density} compares the mean stellar density based on input stellar parameters to the mean stellar density calculated from our circular orbit transit models. Colors denote the source of the adopted stellar parameters. The left panel shows the complete sample.  The right panel shows only PCs with good S/N, non-grazing transits and well measured \rhoc.   The distribution is clearly skewed such that the mean stellar density estimated by fitting the lightcurve and assuming  circular orbits is generally larger than that from stellar parameters tables.  This trend is expected because detection bias favors planets that transit near the periastron of their orbits.  A simple application of Kepler's 2$^{\rm nd}$ law dictates that for a given impact parameter, transits observed near periastron are shorter than predicted from a circular orbit, and impact parameters should be roughly uniformly distributed for $b<1$. The distribution is also biased by impact parameter and dilution (see \S6.2 of \citealt{Rowe:2015b}).

\begin{figure}[!hbt]
\includegraphics[width=0.49\textwidth,angle=0]{stellar_density.pdf}
\includegraphics[width=0.49\textwidth,angle=0]{stellar_density_snr10_rhoe20.pdf}
\caption{Comparison of the mean stellar density from our circular orbit transit models (\rhoc) to the mean stellar density from our adopted stellar parameters (\rhostar). The red diagonal line is for \rhostar = \rhoc, and the two parallel darker red lines show a bias of 10\% in \rhostar.  The colored markers note the source of stellar parameter: blue = DR25 \citep{Thompson:2018}, green = \cite{Berger:2020a} and orange = CKS \citep{Fulton:2018}.  The left panel shows all planetary candidates from our sample with 0.1 $<$ \rhostar $<$ 10 g/cm$^3$.  The right panel is restricted to planetary candidates that have S/N $>$ 10, $b$ $<$ 0.9, and  uncertainty (average of $\sigma_+(\rhoc)$ and $\sigma_-(\rhoc)$) of less than 20\%.  Uncertainties are half-widths of the 68.27\% credible interval ($\pm 1 \sigma$). 
}\label{fig:mean_stellar_density}
\end{figure}

A visual examination of the right panel in Fig.~\ref{fig:mean_stellar_density} appears to shows a bias in \rhostar~between the CKS \citep{Fulton:2018} and \cite{Berger:2020a} samples. It is important to note that our stellar parameters give preference to CKS, then \cite{Berger:2020a}, and use those of DR25 only for targets not appearing in either of the preferred catalogs. The CKS sample was skewed towards the inclusion of multiplanet systems.  As shown in \S\ref{sec:ecc}, planets in compact multiplanet systems tend to have more circular orbits (Figs.~\ref{fig:EccMulti} and \ref{fig:EccSpacing}). Thus, the observed bias is a selection effect.  Figure \ref{fig:EccStellarSource} shows that the measured transit duration distributions do not depend on the choice of stellar parameters catalog in any systematic manner.  {The vertical blue error bars tend to be large because of the high uncertainties in DR25 stellar parameters.  These stars were not characterized in \gaia DR2 or CKS, which suggests that they may have strong stellar blends or other observational challenges. This also explains the larger scatter of the blue points relative to the middle diagonal line.}


\subsection{Planet Candidate Selection} \label{sec:selection}

We pulled planet candidates from a variety of sources, including 9564 KOIs from the cumulative  DR25 supplement catalog\footnote{Retrieved from the NASA Exoplanet Archive on {2022/10/27}.}, ultra-short period (USP) planets from \cite{Sanchis-Ojeda:2014}, long period ($P \gtrsim 1$~year) and transit candidates {only observed to transit once, which we refer to as mono-transits}, from \cite{{Kawahara:2019}}, {the autoregressive planet search from \cite{Caceres:2019}}, PCs from the machine learning search by \cite{Shallue:2018} that sought additional planet candidates around targets already having 2 or more PCs {and low-S/N candidates found by revisiting marginal TCEs \citep{Bryson:2021}.}  The inclusion of new catalogs and discoveries yields new KOIs:  KOI-1843.03, 8298-8303 are ultra-short period planet candidates (including the 3 shortest-period PCs in our catalog) from \cite{Sanchis-Ojeda:2014}; KOI-8304~--~8335, 1108.04, 4307.02, 408.06, 2525.02, 847.02, 671.05, 3349.02, 693.03, 7194.02, 1870.02 are long-period planet candidates; KOI-500.06, 351.08, 691.03, 354.03, 191.05, 1165.03, 2248.05, 542.03, 1589.06, 2193.03, 1240.03, 1992.04, 1276.03, 416.05, 1889.03, 4772.04, 2433.08, 597.04 are from \cite{Shallue:2018}\footnote{We classified 2061.03 from the machine learning search as an FP because we recognized that it resulted from a poor masking of 2061.01, due to large TTVs. We did this by using Quasiperiodic Automated Transit Search (QATS, \citealt{Carter:2013}), determining an approximate ephemeris $T_n [BJD] = 2454949.27 + n \times 14.097 + 0.12 \times \cos( 2\pi/P_{\rm ttv} (t-2455154) )$, where $P_{\rm ttv}=1160$~days.}, and  {KOI-4246.03, 4302.02, 8336.01, 8337.01 and 8338.01 are from \cite{Bryson:2021}}. 

Finally, KOI-8339~--{~8394, the majority of which we have dispositioned as false positives/false alarms because their S/N $< 7.1$, are from \cite{Caceres:2019}. There are 30 potential candidates from Table 5 of \cite{Caceres:2019} not included as we were unable to compute a best fit model based on the their predicted ephemeris. Table 5 of \cite{Caceres:2019} also lists 11 identifications around targets with pre-existing KOIs, including 8302.01 from \cite{Sanchis-Ojeda:2014}, that are already in our table, some of which are discussed below because we corrected their periods to the values given in \cite{Caceres:2019}.}   

We provide new transit models and parameter posteriors for these new KOIs using the same models and methodology presented in \S\ref{sec:transitmodel}.  We corrected the period of KOI-1353.03 (Kepler-289d) to match the reported value in \cite{Schmitt:2014} rather than that in the DR25 catalog because the latter was the result of an aliasing problem. {One of our new long-period PCs, KOI-3349.02, was previously only observed to transit once (\citealt{Kawahara:2019} and references therein), but we located a second, nearly identical, transit in the SAP (Simple Aperature Photometry) lightcurve, showing it is a duo-transit (only two transits observed) planet with $P = 805$~days.}

%KOIs with KeplerName + FP from nexsci
%
% These are 'certified FPs' whatever that means 
%
%3032.01 - FP -- odd-even, remains as FP
%3138.02 - FP -- confirmed, revert to PC
%3184.02 - FP -- has 1.500 period ratio to .01 but phase is diff.  -> PC
%245.04 - FP -- never validated is a FA, remains as FP
%125.01 - FP -- no evidence of binarity in LC, revert to PC
%126.01 - FP -- famous binary, remains as FP
%1416.01 - FP -- can see secondary, remains as FP
%1450.01 - FP -- Can see secondary + phase curve, remains as FP
%129.01 - FP -- revert to PC
%631.01 - FP -- revert to PC
%
% Because the 'certified FP' list has no publication it's really hard
% to figure out why something was listed as FP.  It's possible that 
% there are some RVs floating around, but we don't have that info.
%Jessie Christiansen 2023 - 12:07 PM, Jun 21 (PDT): There are 4 validated Kepler planets with published refutations, and six with dispositions subsequently set to FP by the FPWG.
%Confirmed large planet KOIs (with 'R' flag)
%1102.01, 1203.03, 2768.02 had large 'b' need to see how they changed
%855.01, 846.01, 1792.01 all have large radius but otherwise fine.  
%
%Confirmed low SNR planets
%2034.02, 4024.01, 6233.01 look fine. 
%7368.01 -- looks poor, but won't change disposition 


{We revisited the dispositions of KOIs that have Kepler numbers but were classified as FPs in DR25supp.  Since there is no peer reviewed source for the reasoning leading to the dispositions, we do not know whether or not additional observations beyond \ik photometry and centroids were used.  Based on our assessment of TCERT data validation reports downloaded from the NASA Exoplanet Archive, we reverted the following KOIs from FP disposition to PC: KOI-125.01, 129.01 (but see below), 631.01, 3138.02 and 3184.02.  DR25supp listed these KOIs as showing evidence of stellar eclipses; however, we found no evidence of a secondary eclipse in the photometry lightcurve, either directly or through measurement of an odd-even effect.

We confirmed that the following KOIs with Kepler numbers are indeed false positives: KOI-3032.01, 126.01 (see \citealt{Carter:2011}), 1416.01 and 1450.01.  Their photometric lightcures show clear evidence of secondary eclipses that are indicative of stellar companions.  Despite an analysis of putative TTVs by \cite{Hadden:2014}, Kepler-37e (KOI-245.04) is a false alarm that was never validated \citep{Barclay:2013}. 

Table 7 of \cite{Carmichael:2019} lists five KOIs, all of which are listed as PCs in DR25supp and three of which have been given Kepler numbers, as having measured masses above the planet/brown dwarf dividing line  (see also \citealt{Carmichael:2023}). We have reclassified as false positives all five of these KOIs: KOI-423.01 = Kepler-39~b, KOI-189.01 = Kepler-486~b, KOI-205.01 = Kepler-492~b,  KOI-415.01 and KOI-607.01. Note that Kepler-39~b has a mass of $\sim 20$~M$_{\rm Jupiter}$, whereas the others have masses $\gtrsim 40$~M$_{\rm Jupiter}$.
%KOI-607b has a mass of M b = 95.1 ± 3.4 M J, a radius of R b = 1.089 ± 0.089 R J.

\cite{Santerne:2016} analyze radial velocity measurements of more than 100 KOIs that were listed as giant planet candidates in one or more of the \ik project's catalogs. They present convincing evidence that the following five KOIs, which we would have otherwise classified as planet candidates, are produced by eclipsing binary stars: 129.01, 969.01, 1465.01, 1784.01 and 3787.01. Furthermore, the following three KOIs, which we would have been classified as planet candidates had they not failed our upper size limits, are also produced by eclipsing binary stars: 
3411.01, 3811.01 and 5745.01. Table \ref{tab:planetcatalog} dispositions the thirteen KOIs listed in this and the previous paragraph with the letter `M' to distinguish them from other false positives.}

%M's would have been P or R but RV mass determinations in one of these papers show it isn't a planet \cite{Santerne:2016}        
%%129.01:mass of 0.11 M_sun (Santerne+ 2016) - should be M - based upon just 2 RVs, so not convincing, but note that this one had been F and was rescued based in part on having a Kepler #
%KOI-531.01, dispositioned P, is claimed clearly not a planet by same ref, but reasoning deserves consideration.  Keep as P.
%969.01 - P, This candi- date is not a transiting planet but an equal-mass EB; nexsci F - should be M - two observations, SB2, RV variations anti-correlated between the two stars.  
%1465.01 - P, EB should be M (5 RVs, seems clean and convincing)
%1546.01 P,  problems, very likely EB, but complicated, see the paper - leave as P, but note that there is some evidence that it is an EB
%1784.01 P, triple star (eclipsing one could be BD, but not planet) - should be M - SB2, 2 observations with RVs not anti-correlated, indicating complex system, probably triple star
%3411.01, R, primary-only EB; should be M, 2 observations, SB2
%3685.01 is correctly listed as F; .02 suffers from severe dilution (not to deal with here, but perhaps in next catalog)
%3787.01, P, low-mass EB - should be M, 2 observations, SB2
%3811.01, R, secondary-only EB - should be M, 2 observations, SB2
%5384.01, P, probably an EB, but uncertain so ?? keep as P
%5745.01, R, secondary-only EB - make M, 2 observations, SB2

%-- there are 10 KOIs with Kepler (number) designations and F dispositions on NexSci: 125.01 ffpp, 126.01 fffp (note this is a well-studied triple star system, Carter+2012, so definitely F), 129.01 pfpp, 245.04 ffnn, 631.01 pfpp, 1416.01 ffpp, 1450.01 fffp, 3032.01 ffpp, 3138.02 fffn (the first 8 on this list have Kepler \# listed in our table), Kepler \# not in our table, 3184.02 fffn, Kepler \# not in our table, period is 1.50046 times as long as PC .01, which may indicate that the same transits are being seen or that it is real. If real, planets are likely to be locked in resonance. (probably worth comparing list of Kepler numbers in our table vs. that currently on NexSci or whatever your preferred database is.

{We undertook a photometric analysis of new candidates presented in Table 5 of \cite{Caceres:2019} to assign dispositions based solely on \ik photometry.  Photometry was processed in a manner similar to DR25 \citep{Thompson:2018}, by detrending the data with a second-order Savitzky-Golay filter.  Data in the transit window as predicted from the reported transit duration, period and center-of-transit time were excluded from polynomial fits.  This means that we did not complete an exhaustive test against false alarms due to in-phase periodicity (e.g., {\it depth-test} as described in \citealt{Coughlin:2016} for uniqueness).  We attempted to compute best fit transit models through $\chi^2$ minimization.  For 33 of the 86 proposed new candidates, the model either failed to converge or returned a fit consistent with a flat line.  In these cases, we then ran a BLS ({box} least squares; see \citealt{Kovacs:2002}) search restricted to $\pm 0.1$ days around the reported period{, which allowed us to recover 3 of them. As we could not find evidence of the remaining 30 proposed PCs, we do not include them in our KOI table.}  The ephemerides reported in \cite{Caceres:2019} are only accurate to one \ik long-cadence ($\sim$~30 minutes), which combined with uncertainties in the reported period and barycentric drift may result in poor recovery with our methods.  If the BLS search found a candidate event with $P$ within 0.01 days of the reported event, we adopted the period and center-of-transit times from our localized search.  Thus, there is risk that we have not recovered all events as previously reported.  With best fit models we ran our MCMC algorithm to compute posteriors.  All new candidates from this activity that had a S/N as determined by our transit models to be less than 7.1 have been flagged as false positives in  Table \ref{tab:planetcatalog}.  The S/N reported in Table 5 of \cite{Caceres:2019} is a detection statistic that appears to not be strongly related to the folded transit S/N that we report (see Equation \ref{eq:S/N} above).  We did not assess the photometric centroids for this sample; however, we noted that there is a substantial mismatch when comparing \rhoc~from our transit models to \rhostar~from stellar properties catalogs for KOI-8345.01 and KOI-8366.01, and for this reason we have flagged these two KOIs as false positives.} %may add discussion of the mean S/N for this sample
%add discussion of candidates in potentially new multis

{We investigated several cases in which the \cite{Caceres:2019} table listed signals around existing KOIs with periods that are a small integer multiple or fraction of those listed in DR25supp. This investigation led us to revise the orbital periods of two PCs downward by a factor of two, KOI-6262.01 from $P = 0.673$ days to 0.3365 days and KOI-4777.01 (which was independently identified with the correct period by \citealt{Canas:2022}) from $P = 0.824$ days to  0.412 days. Additionally, we revised the periods of FPs (both of which were previously identified as having centroid offsets) KOI-4305.01 (from 0.935 to 0.234 days) and KOI-4872.01 (from 1.035 to 0.207 day). Reported detections for KOI-6749, KOI-6984, KOI-2431, KOI-4788, KOI-2642 are the secondaries; these targets are already FPs.} 

KOIs have been vetted into new categories using several criteria.  Nonetheless, the vast majority of KOIs are dispositioned as one of the two categories: Planetary Candidates (PC) and False Positives (FP, which here, as in previous catalogs, includes false alarms).  False positives are defined as transit-like astrophysical events that are not produced by a planet.  Eclipsing binaries are the primary source of false positives.  False alarms are spurious detections caused by features in the target star’s lightcurve that are not transit-like.  False alarms can be caused by stellar variability and/or instrumental systematics.  We provide additional vetting criteria based on S/N and planetary radius. A description of all vetting flags is presented in Section \ref{sec:unified}. 

Initial FP and PC classification is adopted from the DR25 and DR25supp catalogs. DR25supp is represented by the cumulative catalog from the NASA Exoplanet Archive retrieved on {2022/10/27}. If vetting classification is not available in either DR25 or DR25supp, we use DR24 \citep{Coughlin:2016}, which includes KOIs from previous searches that were not seen in the TCE search conducted for DR24.

Consistent with previous \ik catalogs, generally we require the total transit S/N~$\geq 7.1$ for a KOI to be considered a PC; KOIs below this threshold are considered false alarms. {We allowed four exceptions to this rule, KOI-2022.02 = Kepler-349~c, KOI-2034.02 = Kepler-1065~c, KOI-4024.01 = Kepler-1541~b and KOI-7368.01 = Kepler-1974~b, all of which have been both validated as planets and passed our visual inspection---which revealed marginal evidence of a visible transit event in their lightcurves.

The  S/N~$\geq 7.1$ } criterion was based on limiting the number of false alarms considered to be PCs to 1 in the 90 day -- 2 year period range per $10^5$ stars searched the \ik sample in the presence of white noise on 6-hour timescales.  However, much of the noise in \ik lightcurves is correlated, so multiple false alarms become problematic for S/N in the range of $10 \gtrsim$~S/N~$\geq 7.1$, as has been verified through injection tests and reliability studies \citep{Thompson:2018,Hsu:2019}.  Moreover, low S/N KOIs do not have sufficient signals to allow for precise centroiding and other vetting procedures used to distinguish astrophysical signals such as eclipsing binaries (EBs) from transiting planets.  Therefore, higher S/N cuts are needed to obtain the purer (higher confidence) samples of PCs that are required for some studies, including most of our analyses of the characteristics of the population of \ikt's planet candidates. 

Some KOIs have all transit model parameter uncertainties listed as zero.  This indicates that the MCMC computations did not converge.  This can happen when the S/N of transit event is low or the event is non-transit-like in shape.  While we did not consider MCMC convergence when assigning PC or FP (P or F) dispositions, there is a high probability that these KOIs are indeed false alarms.  {There are 220 KOIs flagged as `P' or `S' without MCMC computed posteriors, 24 are members of candidate multiplanet systems and a total of 26 are PCs, 14 of which have a S/N $>$ 10.}

There is increasing degeneracy when distinguishing between ultra-cool stars, brown dwarfs and planets for transiting objects with radii approaching the size of Jupiter that lack mass information.  We have flagged KOIs with an estimated planetary radii $R_p > 21.947$~R$_\oplus \approx 2$ R$_{\rm Jupiter}$, as well as KOIs with period $P>20$ days and a $1 \sigma$ lower limit for the planetary radii $R_p$ that exceeds 13.17~R$_\oplus \approx 1.2$~R$_{\rm Jupiter}$, as likely FPs, but as these boundaries are not precisely defined, KOIs that pass all criteria for being classified as PC apart from size are given the disposition ``R''. {%Exceptions to this general rule were made for three KOIs that have Kepler numbers but radii using our nominally-preferred stellar parameters that exceeded our limit on planetary size and passed our visual inspection of the \ik lightcurves which we dispositioned as `P': KOI-846.01 = Kepler-699~b, KOI-855.01 = Kepler-706~b and 1792.01 = Kepler-953~b. The radius of the star KOI-1792 estimated by \cite{Fulton:2018} is xx times as large as that estimated by \cite{berger:2020a}; the former leads to a radius estimate for KOI-1792.01 of $R_p = xx$ R$_\oplus$, so we used the \cite{berger:2020a} parameters for this star. The other two cases are both evolved stars, whose radii are difficult to estimate, so we suspect that the stellar radii have been overestimated, but in these cases, we only have one \gaiat-constrained radius estimate (from \citealt{berger:2020a}), so we kept the probably-overestimated nominal values of $R_p$.}  
The following KOIs have Kepler numbers but radii that exceeded our limit on planetary size using nominal stellar parameters: KOI-846.01, 855.01 and 1792.01.  There is no strong evidence from \ik photometry or in the literature that contradicts their status as a confirmed planet.  Thus, no change in their disposition as PC is warranted based on our study. The radius of the star KOI-1792 estimated by \cite{Fulton:2018} is almost three times as large as that estimated by \cite{Berger:2020a}; the former leads to a radius estimate for KOI-1792.01 of $R_p = 31.1$~R$_\oplus$, so we used the \cite{Berger:2020a} parameters for this star. The other two cases both appear to be evolved stars (KOI-846 being slightly evolved, whereas KOI-854 is a giant), whose radii are difficult to estimate, so we suspect that the stellar radii have been overestimated, but in these cases, we only have one \gaiat-constrained radius estimate (from \citealt{Berger:2020a}), so we retained the probably-overestimated nominal values of $R_\star$ and $R_p$.}

{We give new KOIs dispositions based on an assessment of the photometric transit.  All new KOIs are assigned a PC status unless they failed to meet our S/N, radius cuts or visual inspection.  Visual inspection includes determination whether observed transit duration is in very rough agreement with the stellar parameters and the phased lightcurve has a transit-like shape.  We did not attempt to measure photometric centroids, thus it is possible that many of the new short-period KOIs from the \cite{Caceres:2019} sample (8339~--~8394) may be background binary blends.  As most of the orbital periods from this sample are relatively short, there is reasonable expectation that the modelled mean stellar density should match stellar tables because the orbits should be nearly circular.  For example, KOI-8386.01 is likely a background binary due to the significant order-of-magnitude mismatch in the transit-duration and stellar classification, but we nonetheless dispositioned it as a PC as we have not conducted an analysis of photometric crowding or photometric centroid shift during the transit event. None of the KOIs from the \cite{Caceres:2019} sample failed {the radius cuts, so all were dispositioned either ``S'' (36 cases), ``P'' (18 cases) or ``F'' (2 cases that were discarded based upon visual inspection of the lightcurve); only 5 of the PCs have S/N $> 12$, and only 1 has S/N $> 14$.}} 

We visually re-examined KOIs that were classified as PCs in either DR24 or DR25 but subsequently classified as FPs in DR25supp. %but not vice-versa
Based on our analysis, we reverted the following KOIs back to PCs: 82.06, 198.01, 1693.01, 1796.01, 1902.01, 2306.01, 2307.01. These KOIs do not exhibit measurable centroid shifts from examination of DR25 or DR25supp vetting reports and show visually identifiable transits that can be modeled with MCMC that meet convergence criteria.  %198 has 2 RVs suggesting an EB Santerne 2016, but we don't consider just 2 to be conclusive

\subsection{Orbital Periods During the \ik Epoch}\label{sec:periods}

The orbital periods listed in Table \ref{tab:planetcatalog} are estimated by fitting the \ik lightcurve assuming a Keplerian, non-interacting orbit and using a limb-darkened transit model. As such, we are estimating something akin to the planet's mean period over the interval of \ik observations. 

{Transit times are the measure of when mid-transit occurs.  We define the mid-transit as when the project distance between the center of the star and planet is minimized.  For a circular orbit, this is equivalent to when the transit model is deepest. We measure transit times (TT) and then report transit timing variations relative to the orbital period and reference time.  The latter has typically been the first or mid-observation transit. The transit model is used as a template to measure the center-of-transit time for each transit, and the transit model orbital period from the initial fit is used to calculate the difference between the observed transit and the calculated transit (O-C).}   If TTVs are present, then the transit model is updated by {\it deTTVing} the lightcurve, whereby the time stamps from \ik observations are adjusted to have all transits aligned.  Time stamp adjustments use a linear interpolation based on the O-C transit times such that the new time stamps have an effective O-C of zero.  A new transit model is then fit to the data with updated time stamps. 

If TTVs are not properly accounted for, the ingress and egress of sequential transits become misaligned, leading to errors in the reported characteristics of the transit, especially the transit's duration, depth and the impact parameter.  Since the scaled planetary radius ($R_p/R_\star$) and transit model mean stellar density (\rhoc) are correlated with the impact parameter ($b$) due to geometry and stellar limb-darkening, those parameters also have increased errors.  The models based on deTTVed lightcurves better estimate the transit shape, {often leading to substantially improved estimates of the above mentioned parameters, especially} for $b$ and $R_p/R_\star$. However, this process {previously led to errors in the reported period that were not accounted for in estimates of period uncertainties.}

TTVs are calculated independent of the fitted transit model, and when possible initial TTVs were adopted from previous catalogs.  Examples where the period reported in previous catalogs are significantly different from the observed mean period (averaged over the time interval during which transits were observed by \ikt) include KOI-142.01 (Kepler-88~b) and KOI-377.01 and .02 (Kepler-9~b and c).  Early KOI catalogs used only a few Quarters of observations, and the mean period over that time frame differed significantly from estimates using the entire \ik observational baseline.  The model parameters for each previous catalog were used as a seed for the updated model.  Thus, the period from a previous catalog was used as the seed for the best fit in the subsequent catalog in which it appears.  If additional TTVs (O-C) were measured, the mean-period from a previous catalog was held fixed before the new model fit was made.  In some cases, the previously reported period did not represent the true mean period in the presence of strong TTVs. 

For the catalog presented herein, the mean observed period from \ik observations is of interest as it is the best approximation to the long-term mean period that can be straightforwardly estimated for all KOIs apart from those that only transited once (mono-transits). The value of $P$ is calculated by applying a correction to the best fit period, $P_{bf}$, from the transit models. For each KOI with four or more measured transit times, a straight line model is fit to the Observed-Calculated (O-C) vs.~Observed (O) values using standard least-squares minimization weighted by measurement uncertainties in O-C values.  The slope, $m$, from the fit gives the correction, with
\begin{equation}
P = P_{bf} (1 + m).
\end{equation}
We report $P$ and propagate the uncertainty in $m$ to compute the orbital period uncertainty thereof for all KOIs in Table \ref{tab:planetcatalog}.  The measured O-C values are based on $P_{bf}$, thus any non-zero slope indicates that $P_{bf}$ does not represent the mean period.  This correction specifically addresses the issue of incorrect periods for planets in systems such as Kepler-9 and Kepler-88.  For non-TTV planets, there is good agreement between $P_{bf}$ and $P$.  {If a planet has just 2 or 3 observed transits, we adopt $P$ from $P_{bf}$.}  To avoid additional inhomogeneities in our catalog, only $P$ is reported.

The incorporation of TTVs in orbital period calculations, as described above, yields {more robust estimates of orbital periods of planets with observed TTVs than provided by previous catalogs.  For most KOIs, no  TTVs  are detected, and the revised period is statistically similar to the reported period from DR25.  However, there are extreme examples, such as Kepler-9~b and c and Kepler-88~b, where the period changes significantly ($\sim 1$ hour) relative to the DR25 catalog.  Overall, the distribution of fractional changes in period ratio is non-Gaussian, with substantially larger numbers in the tails of the distribution. }  

However, we account for TTVs only to the extent that they average out over the time in which \ik observed the planets to transit.  Typically, this cancellation is incomplete for planets with observed TTVs. The periods and uncertainties quoted in Table \ref{tab:planetcatalog} thus do not account for TTVs that have periodicities that are long compared to the \ik observations, and do not fully account for TTVs with periodicities comparable to the amount of time between \ikt's first and last observations of transits of a particular planet. Some types of TTVs are of very small amplitude over four years (and thus unlikely to have been detected), but grow to become substantially larger on time scales of decades to centuries, affecting the long-term mean orbital period.  These differences in orbital period are important for understanding three-body resonances and producing ephemerides that are accurate far into the future.  We discuss these issues in more detail in Section \ref{sec:periodtheory}.

We searched the lightcurves of the 10 duo-transit PCs with $P > 730$~days to determine whether or not there are \ik data that can exclude the possibility that additional transit(s) were missed because they occurred during a data gap(s) and the actual period of the planet is either half or one-third of the reported value. Data points with SAP\_QUALITY equal to 16 or greater were excluded from the examined lightcurves. The period of KOI-375.01 (= Kepler-1704~b) could, indeed, be half of the reported value of 988.9 days, but \ik data exclude the possibility of any of the other 9 PCs having periods equal to  half or one-third of the values listed in Table \ref{tab:planetcatalog}.

\subsection{Unified Planet Candidate Catalog} \label{sec:unified}

Our catalog of \ik planet candidates is presented in Table \ref{tab:planetcatalog}.  We list, from left to right, catalog numbers of the target star and planet (columns 1~--~3), fundamental transit model parameters (4~--~19), properties derived from the transit model and \ik photometry (columns 20~--~38), parameters that depend on the transit model and stellar parameters (39~--~44), stellar parameters (45~--~62)\footnote{Columns 14~--~16 list stellar parameters derived from the transit model. Data in column 45 are taken from DR25.  Data in columns 46~--~61 are taken from the source specified in column 62.}, and vetting dispositions (63).  Data in columns 64~--~85 list the same properties as provided in columns 39~--~44 and 46~--~61 with stellar parameters taken from \cite{Berger:2020a}; these columns are left blank if properties of the target star are not given by \cite{Berger:2020a}.

Table \ref{tab:astro_constants} presents our adopted values for astrophysical constants.  The mean radius of the Sun (R$_{\odot}$) and Earth (R$_{\oplus}$) are used to calculate the absolute radius of planets (\rpl) and incident flux, $S$, normalized to that of the Earth.  The gravitational constant (G) is used to calculate transit durations, the scaled semimajor axis (\adrs) from the mean stellar density (\rhoc) and orbital inclination ($i$).  The astronomical unit (AU) and solar effective temperature \teff$_{,\odot}$ are used in the calculation of the flux incident upon Earth, S$_\oplus$.  

For many properties, we report the location of the maximum posterior density and give additional information about the uncertainty.  For some properties, these uncertainties are typically well-described by a symmetric distribution, so we report the half-width of the 68.27\% credible interval, assuming a symmetric distribution. For other properties, the posterior distribution is often significantly asymmetric, so we report separate uncertainties in the positive and negative directions based on the 68.27\% credible interval that minimizes the width of the marginal distribution for that parameter. The units for all uncertainties are the same as for the quantities themselves.  

Overall we have updated transit models for more than 1000 KOI systems. Included among these are 121 KOIs that previously had unreasonably large impact parameters.  We have updated the dispositions of 11 KOIs from FP to PC and 6 KOIs from PC to FP.  We flagged 361 cases with very weak S/N transit signals as likely false-alarms, flagged 100 KOIs with large planet radii that may be due to stellar binaries and flagged 13 KOIs that previously had a PC flag that now have a mass measurement inconsistent with the exoplanet hypothesis.

\begin{longrotatetable}
\begin{table}[!hbt]
    \scriptsize
    \centering
    \begin{tabular}{l c c c c c c c c}
\hline
KOI          & KIC         &   Period [d]           & T0 [MJD]          & \rprs       & $b$            & \rhoc [g/cm$^3$]      & u$_1$    & u$_2$ \\
 Disp        & Kepler-Name &   \rpl [R$_\oplus$]       & $d_{\rm transit}$ [ppm]      & $T_{\rm dur}$ [h]     & $T_{1.5}$ [h]      & \#TT$_{\rm obs}$                 & \#TT      & TTVflag \\
             & \adrs      &   $i$ [deg]              & $S$ [S$_\oplus$]        & S/N          & MES          & $\Delta$S/N$_{ttv}$          & S/N$_{wTTV}$  & S/N$_{woTTV}$ \\
             & kepmag      &   \rhostar [g/cm$^3$] & \teff [K]        & \rstar [\rsun] & \mstar [\msun] & \logg                 & \feh   & sparflag \\
\hline
1.01 & 11446443 & 2.47061338 $\pm$ 0.00000002 & 787.064865 $\pm$ 0.000009 & 0.123865 $\pm$ $^{ 0.000066 } _{ 0.000062 }$ & 0.8179 $\pm$ $^{ 0.0006 } _{ 0.0006 }$ & 1.83652 $\pm$ $^{ 0.00623 } _{ 0.00849 }$ & 0.3860 & 0.2724 \\
PPPP & Kepler-1 b & 14.102 $\pm$ $^{ 0.392 } _{ 0.30100 }$ & 14230.5 $\pm$ 4.6 & 1.743025 $\pm$ 0.001158 & 1.226357 $\pm$ 0.001031 & 431 & 431 & 000 \\
& 8.4001 $\pm$ $^{ 0.0095 } _{ 0.01300 }$ & 84.4130 $\pm$ $^{ 0.0096 } _{ 0.01400 }$ & 873.86 $\pm$ $^{ 55.88 } _{ 39.91 }$ & 4360.4 & 6468.0 & -0.19 & 7070.7 & 7323.0 \\
& 11.338 & 1.21942 $\pm$ $^{ 0.06243 } _{ 0.06146 }$ & 5815 $\pm$ 66 & 1.04 $\pm$ $^{ 0.02 } _{ 0.02 }$ & 0.99 $\pm$ $^{ 0.03 } _{ 0.03 }$ & 4.390 $\pm$ $^{ 0.078 } _{ 0.07800 }$ & 0.013 $\pm$ 0.041 & 2 \\
[.2cm]
2.01 & 10666592 & 2.20473545 $\pm$ 0.00000005 & 788.535435 $\pm$ 0.000017 & 0.075182 $\pm$ $^{ 0.000010 } _{ 0.000008 }$ & 0.2242 $\pm$ $^{ 0.1585 } _{ 0.2163 }$ & 0.40810 $\pm$ $^{ 0.00024 } _{ 0.00030 }$ & 0.3043 & 0.3129 \\
PPPP & Kepler-2 b & 16.180 $\pm$ $^{ 0.389 } _{ 0.357 }$ & 6668.3 $\pm$ 1.4 & 3.874104 $\pm$ 0.000522 & 3.599230 $\pm$ 0.000472 & 602 & 602 & 006 \\
& 4.7156 $\pm$ $^{ 0.0009 } _{ 0.00120 }$ & 89.9596 $\pm$ $^{ 0.0323 } _{ 0.23440 }$ & 4145.89 $\pm$ $^{ 182.66 } _{ 182.66 }$ & 5952.0 & 3862.2 & 13.12 & 20695.8 & 20905.0 \\
& 10.463 & 0.27881 $\pm$ $^{ 0.00877 } _{ 0.00691 }$ & 6447 $\pm$ 64 & 1.97 $\pm$ $^{ 0.04 } _{ 0.04 }$ & 1.51 $\pm$ $^{ 0.03 } _{ 0.02 }$ & 4.150 $\pm$ $^{ 0.110 } _{ 0.11 }$ & 0.192 $\pm$ 0.042 & 2 \\
[.2cm]
3.01 & 10748390 & 4.88780321 $\pm$ 0.00000024 & 786.095757 $\pm$ 0.000081 & 0.057856 $\pm$ $^{ 0.000451 } _{ 0.000123 }$ & 0.0539 $\pm$ $^{ 0.1855 } _{ 0.0410 }$ & 3.69279 $\pm$ $^{ 0.03943 } _{ 0.28590 }$ & 0.6403 & 0.1021 \\
PPPP & Kepler-3 b & 4.914 $\pm$ $^{ 0.080 } _{ 0.08800 }$ & 4315.6 $\pm$ 6.5 & 2.363853 $\pm$ 0.005403 & 2.232651 $\pm$ 0.002984 & 184 & 184 & 007 \\
& 16.6988 $\pm$ $^{ 0.0811 } _{ 0.40530 }$ & 89.7905 $\pm$ $^{ 0.1773 } _{ 0.62850 }$ & 85.27 $\pm$ $^{ 4.84 } _{ 4.44 }$ & 862.1 & 2034.6 & 0.02 & 4445.0 & 4449.0 \\
& 9.174 & 2.30908 $\pm$ $^{ 0.14896 } _{ 0.09863 }$ & 4541 $\pm$ 54 & 0.77 $\pm$ $^{ 0.01 } _{ 0.01 }$ & 0.77 $\pm$ $^{ 0.03 } _{ 0.02 }$ & 4.542 $\pm$ $^{ 0.023 } _{ 0.01400 }$ & 0.410 $\pm$ 0.084 & 1 \\
[.2cm]
4.01 & 3861595 & 3.84937138 $\pm$ 0.00000116 & 787.262929 $\pm$ 0.000525 & 0.039604 $\pm$ $^{ 0.000723 } _{ 0.001286 }$ & 0.9147 $\pm$ $^{ 0.0161 } _{ 0.0346 }$ & 0.22168 $\pm$ $^{ 0.12742 } _{ 0.06371 }$ & 0.3150 & 0.3029 \\
PPPP & Kepler-1658 b & 13.073 $\pm$ $^{ 0.563 } _{ 0.56300 }$ & 1297.6 $\pm$ 14.4 & 2.638645 $\pm$ 0.091166 & 2.107915 $\pm$ 0.040755 & 279 & 279 & 001 \\
& 5.5613 $\pm$ $^{ 0.8861 } _{ 0.36920 }$ & 80.7504 $\pm$ $^{ 1.4116 } _{ 0.94100 }$ & 5158.05 $\pm$ $^{ 587.69 } _{ 452.07 }$ & 132.7 & 235.6 & -0.27 & 6624.3 & 6706.0 \\
& 11.432 & 0.08682 $\pm$ $^{ 0.00798 } _{ 0.00818 }$ & 6776 $\pm$ 139 & 3.05 $\pm$ $^{ 0.09 } _{ 0.08 }$ & 1.78 $\pm$ $^{ 0.06 } _{ 0.14 }$ & 3.716 $\pm$ $^{ 0.026 } _{ 0.03700 }$ & -0.076 $\pm$ 0.082 & 1 \\
[.2cm]
5.01 & 8554498 & 4.78032745 $\pm$ 0.00000087 & 787.803497 $\pm$ 0.000159 & 0.036536 $\pm$ $^{ 0.000218 } _{ 0.000187 }$ & 0.9517 $\pm$ $^{ 0.0017 } _{ 0.0020 }$ & 0.34568 $\pm$ $^{ 0.01766 } _{ 0.01494 }$ & 0.3746 & 0.28 \\
PPPP & ----- & 8.699 $\pm$ $^{ 0.496 } _{ 0.41900 }$ & 959.3 $\pm$ 4.2 & 2.019873 $\pm$ 0.015371 & 1.382730 $\pm$ 0.018526 & 282 & 282 & 002 \\
& 7.4764 $\pm$ $^{ 0.1228 } _{ 0.11340 }$ & 82.6917 $\pm$ $^{ 0.1306 } _{ 0.13060 }$ & 1386.71 $\pm$ $^{ 142.44 } _{ 129.49 }$ & 380.8 & 360.2 & -0.50 & 5139.5 & 5446.0 \\
& 11.665 & 0.18222 $\pm$ $^{ 0.02956 } _{ 0.02649 }$ & 5936 $\pm$ 109 & 2.19 $\pm$ $^{ 0.11 } _{ 0.10 }$ & 1.37 $\pm$ $^{ 0.13 } _{ 0.10 }$ & 3.894 $\pm$ $^{ 0.049 } _{ 0.05400 }$ & 0.235 $\pm$ 0.158 & 1 \\
[.2cm]
5.02 & 8554498 & 7.05201465 $\pm$ 0.00010277 & 785.646252 $\pm$ 0.030691 & 0.003162 $\pm$ $^{ 0.000379 } _{ 0.000506 }$ & 0.7272 $\pm$ $^{ 0.0001 } _{ 0.6674 }$ & 0.34677 $\pm$ $^{ 1.71561 } _{ 0.34312 }$ & 0.3746 & 0.28 \\
FFNN & ----- & 0.739 $\pm$ $^{ 0.130 } _{ 0.11100 }$ & 10.8 $\pm$ 2.6 & 4.343656 $\pm$ 1.115429 & 4.272998 $\pm$ 1.116031 & 191 & 187 & 000 \\
& 11.5844 $\pm$ $^{ 4.3004 } _{ 4.73040 }$ & 89.4505 $\pm$ $^{ 0.5491 } _{ 3.56930 }$ & 826.72 $\pm$ $^{ 81.43 } _{ 81.43 }$ & 6.2 & 0.0 & -9.27 & 6127.5 & 6329.0 \\
& 11.665  & 0.18222 $\pm$ $^{ 0.02956 } _{ 0.02649 }$ & 5936 $\pm$ 109 & 2.19 $\pm$ $^{ 0.11 } _{ 0.10 }$ & 1.37 $\pm$ $^{ 0.13 } _{ 0.10 }$ & 3.894 $\pm$ $^{ 0.049 } _{ 0.05400 }$ & 0.235 $\pm$ 0.158 & 1 \\
[.2cm]
6.01 & 3248033 & 1.33410696 $\pm$ 0.00000319 & 788.451898 $\pm$ 0.001062 & 0.010121 $\pm$ $^{ 0.000348 } _{ 0.000174 }$ & 0.1530 $\pm$ $^{ 0.3525 } _{ 0.1443 }$ & 1.23481 $\pm$ $^{ 0.11145 } _{ 0.41795 }$ & 0.3306 & 0.3007 \\
FFFF & ----- & 1.443 $\pm$ $^{ 0.066 } _{ 0.05100 }$ & 120.0 $\pm$ 3.7 & 2.106019 $\pm$ 0.035052 & 2.081369 $\pm$ 0.033207 & 804 & 803 & 100 \\
& 4.8806 $\pm$ $^{ 0.1619 } _{ 0.62050 }$ & 89.0266 $\pm$ $^{ 0.9732 } _{ 5.83920 }$ & 3811.78 $\pm$ $^{ 191.84 } _{ 177.08 }$ & 43.0 & 21.5 & -8.89 & 16014.0 & 16685.3 \\
& 12.161 & 0.79984 $\pm$ $^{ 0.04527 } _{ 0.04859 }$ & 6344 $\pm$ 66 & 1.30 $\pm$ $^{ 0.03 } _{ 0.03 }$ & 1.20 $\pm$ $^{ 0.04 } _{ 0.02 }$ & 4.323 $\pm$ $^{ 0.060 } _{ 0.06 }$ & 0.041 $\pm$ 0.042 & 2 \\
[.2cm]
7.01 & 11853905 & 3.21366892 $\pm$ 0.00000100 & 786.114673 $\pm$ 0.000297 & 0.024505 $\pm$ $^{ 0.000138 } _{ 0.000078 }$ & 0.0210 $\pm$ $^{ 0.2160 } _{ 0.0161 }$ & 0.46369 $\pm$ $^{ 0.00795 } _{ 0.03497 }$ & 0.4025 & 0.2645 \\
PPPP & Kepler-4 b & 4.050 $\pm$ $^{ 0.122 } _{ 0.11300 }$ & 727.6 $\pm$ 2.7 & 3.986002 $\pm$ 0.010266 & 3.887977 $\pm$ 0.008271 & 338 & 338 & 005 \\
& 6.3260 $\pm$ $^{ 0.0402 } _{ 0.16090 }$ & 89.4047 $\pm$ $^{ 0.5456 } _{ 1.63690 }$ & 1191.33 $\pm$ $^{ 61.85 } _{ 73.09 }$ & 346.8 & 294.5 & -0.44 & 10568.0 & 10881.0 \\
& 12.211 & 0.47490 $\pm$ $^{ 0.02319 } _{ 0.03213 }$ & 5833 $\pm$ 64 & 1.51 $\pm$ $^{ 0.04 } _{ 0.04 }$ & 1.15 $\pm$ $^{ 0.05 } _{ 0.05 }$ & 4.117 $\pm$ $^{ 0.106 } _{ 0.10600 }$ & 0.171 $\pm$ 0.042 & 2 \\
[.2cm]
8.01 & 5903312 & 1.16015301 $\pm$ 0.00000161 & 787.921021 $\pm$ 0.000720 & 0.012011 $\pm$ $^{ 0.000350 } _{ 0.000275 }$ & 0.7483 $\pm$ $^{ 0.0001 } _{ 0.5935 }$ & 4.27560 $\pm$ $^{ 0.32135 } _{ 1.46902 }$ & 0.3792 & 0.2764 \\
FFFF & ----- & 1.230 $\pm$ $^{ 0.052 } _{ 0.04400 }$ & 169.6 $\pm$ 4.5 & 1.324100 $\pm$ 0.020189 & 1.305910 $\pm$ 0.018947 & 1153 & 1152 & 100 \\
& 6.7340 $\pm$ $^{ 0.1781 } _{ 0.89060 }$ & 89.3221 $\pm$ $^{ 0.6774 } _{ 4.33550 }$ & 2024.98 $\pm$ $^{ 93.44 } _{ 110.43 }$ & 48.4 & 36.6 & -8.99 & 15328.0 & 16106.1 \\
& 12.450 & 1.73088 $\pm$ $^{ 0.04162 } _{ 0.07127 }$ & 5883 $\pm$ 66 & 0.94 $\pm$ $^{ 0.02 } _{ 0.02 }$ & 1.00 $\pm$ $^{ 0.02 } _{ 0.03 }$ & 4.460 $\pm$ $^{ 0.044 } _{ 0.04400 }$ & -0.063 $\pm$ 0.042 & 2 \\
    \end{tabular}
    \caption{Abbreviated Catalog of planet candidates. The positions of the KOI Disposition and Kepler-Name columns have been moved relative to the electronic table to facilitate viewing the examples presented in the typeset manuscript. The tabulated information is described in the numbered list in Section \ref{sec:unified}, where the numbers refer to the column numbers in the electronic table. (Alternative values of stellar and planetary parameters listed in Columns 63-84 are not presented in this abbreviated sample.)}
    \label{tab:planetcatalog}
\end{table}
\end{longrotatetable}




\begin{table}[!hbt]
    \centering
    \begin{tabular}{c c c}
    \hline
        Name & Value & units \\
    \hline
    R$_{\odot}$     & 6.957$\times$10$^8$     & m \\
    R$_{\oplus}$    & 6.371$\times$10$^6$     & m \\
    G               & 6.674$\times$10$^{-11}$ & m$^3$ kg$^{-1}$ s$^{-2}$ \\
    au              & 1.4956$\times$10$^{11}$ & m \\
    \teff$_{,\odot}$ & 5778                    & K \\
    \hline
    \end{tabular}
    \caption{Adopted astrophysical constants for derived planetary parameters.  The mean radius of the Sun (R$_{\odot}$), Earth (R$_{\oplus}$), the gravitational constant (G), astronomical unit (au), and solar effective temperature (\teff$_{,\odot}$) are used to calculate absolute planetary radii, orbital inclinations and incident flux.}
    \label{tab:astro_constants}
\end{table}


We determine the following transit parameters and their uncertainties by fitting transit models of \citet{Mandel:2002} to \ik lightcurves assuming circular orbits (and adjusted for TTVs if TTVflag = 1): mean stellar density, \rhoc; orbital period, $P$; impact parameter, $b$; scaled radius, \rprs.  Limb-darkening coefficients are held fixed. The transit model is used calculate the depth of transit, $d_{\rm transit}$, and the signal to noise ratio, S/N.  The fitted transit model parameters, together with equations (3) and (2) from \citet{Seager:2003}, are used to calculate transit duration measured from $1^{\rm st}$ to $4^{\rm th}$ contact, $T_{\rm dur}$, and an alternate measure of transit duration (see items 24 -- 27 in the numbered list below for details).
Planetary inclinations are calculated using the formula: 
\begin{equation}
i = \arccos{\left(b\frac{R_\star}{a}\right)}.
\end{equation}

\begin{enumerate}
    \item \textbf{KIC}: The \ik input catalog number of the target star.
    \item \textbf{KOI}: The \ik object of interest number, with the integer portion referring to the target star (source of light) and the decimal portion referring to the particular signal, i.e., the putative planet's transits. 
    \item \textbf{Kepler ID}: The number and letter assigned to verified planets. Blank if the KOI has not been assigned a \ik planet number/letter.  Kepler IDs are do not automatically qualify a KOI to be dispositioned a PC (see Section \ref{sec:selection}) and are supplied for cross-identification purposes only. 
    \item \textbf{Period}, $P$ [days]: Mean orbital period (the period that gives the best fit to observed TTs). For candidates with only one transit observed, the period is estimated based on transit duration and impact parameter, assuming a circular orbit, and given as a negative value to distinguish it from periods computed for multiple transit objects. Note that mono-transit planet candidates (those observed to transit just once) are not included in any of our analyses, figures or tabulations that require knowledge of the orbital period, but are accounted for in assessing multiplicity of planetary systems in all cases.
    \item \textbf{$\sigma(P)$}: Uncertainty (half-width of 68.27\% confidence interval for the average period during the epoch of \ik observations) of $P$.  See discussion in Section \ref{sec:periods}. Based on the uncertainty in fitted slope of the measured TTs. As discussed in the second and third paragraphs of Section \ref{sec:periodtheory}, the actual mean orbital periods of \ik planets over time scales much longer than the four years of \ik observations can differ from the values given in Table \ref{tab:planetcatalog} by many times as much as the listed uncertainties. The reported periods and uncertainties for mono-transit candidates assume circular orbits and are considered unreliable.  
   % \item \textbf{TTV period}: I think that this is an orbital period with TTVs handled in some way, in which case we probably should denote it Period$_TTV$??  [JR: This is the period measured from the slope of the O-C TTV measurements.]I don't understand why it's useful to include both P and TTVP - it seems to me that the latter is simply better, and should be the only orbital period included.  if you disagree, insert text explaining why. That said, it may be useful to have a flag stating when you fit the slope to correct the period.  If you don't fit the slope iff TTVflag=1, then I suggest that TTVflag=1 should mean TTVs used for transit model, TTVflag=2 should mean TTV slope used for fitting period, TTVflag=3 means both.
    %\item \textbf{TTV period$_e$}: [JR: uncertainty in the slope]
    \item \textbf{Epoch}, T0 [BJD~--~2454900]:  Time, calculated using the constant period reported in Column \#4, at which the center of the planetary disk is closest to the center of the stellar disk for the last transit that occurred prior to halfway between the start of the first Quarter and end of the final Quarter that \ik observed the target star in question, whether or not said transit was actually observed by \ikt. This is the mode value from MCMC, calculated using the SciPy kernel density estimator \texttt{stats.gaussian\_kde} with default settings \citep{Virtanen:2020}.  
    BJD $\equiv$ Julian Date viewed from the barycenter of the Solar System. 
    \item \textbf{$\sigma({\rm T0})$}: Uncertainty in the epoch.
    \item \textbf{Planet/star radius ratio}, $R_p/R_\star$: This is the mode value from MCMC of the ratio of the planet's radius to the stellar radius.
    \item \textbf{$\sigma_+(R_p/R_\star)$}: Upwards (to higher values) uncertainty of the ratio of planet's radius to stellar radius.
    \item\textbf{$\sigma_-(R_p/R_\star)$}: Downwards uncertainty of the ratio of planet's radius to stellar radius.
   \item \textbf{Impact parameter}, $b$:  We report the best fit value. See Section \ref{sec:transitmodel}.
   \item \textbf{$\sigma_+(b)$}: Upwards uncertainty of $b$.
   \item \textbf{$\sigma_-(b)$}: Downwards uncertainty of $b$.
   \item \textbf{Stellar density from lightcurve}, \rhoc~[g/cm$^3$]: The mean-stellar density computed from the photometric model.  A \emph{circular orbit} model has been assumed, and a separate fit is done for each planet in a multi. This is the mode value from MCMC.
   \item \textbf{$\sigma_+(\rhoc)$}: Upwards uncertainty of \rhoc.
   \item \textbf{$\sigma_-(\rhoc)$}: Downwards uncertainty of \rhoc.
   \item \textbf{$u_1$}: First quadratic limb-darkening parameter.
   \item \textbf{$u_2$}: Second quadratic limb-darkening parameter.
   \item \textbf{TTV flag}: A three digit number preceded by `T' represents the results of three separate searches for TTVs. The first digit is 1 if TTVs have been included in our transit model, 0 otherwise. The second digit refers to results listed in \citet{Holczer:2016}'s catalog, with 2 signifying sinusoidal TTVs, 1 polynomial TTVs, 0 means no TTVs found, and - means not investigated. The third digit is  the overall rating from \citet{Kane:2019} catalog, with 9 signifying the strongest TTVs, 8 strong TTVs, 7 weak and/or noisy TTVs, and 6 and below no TTVs of interest; we use `-' for KOIs not rated by \citet{Kane:2019}.
   \item \textbf{\# transits}: Total number of transits observed by \ikt.
   \item \textbf{\# TTs}: Number of transits for which the transit time has been measured.  This number is always $\leq$ the number in the previous column.
    \item \textbf{$d_{\rm transit}$} [ppm]: Depth of transit. Specifically, the depth of the transit model evaluated at the mid-transit time assuming a circular orbit. We report the mode of the MCMC distribution.  
   \item \textbf{$\sigma(d_{\rm transit})$}: Uncertainty of $d_{\rm transit}$.
   \item \textbf{Transit duration}, $T_{\rm dur}$ [hr]:  Transit duration, measured from first contact to fourth contact ($T_{\rm dur}\equiv$ T$_{1,4}$), which is the standard measurement for exoplanet transit duration. Based on transit model parameters ($\rhoc$, $P$, $R_p/R_\star$, $b$), using equation (3) of \citet{Seager:2003}.  The mode of the MCMC distribution is reported.  
   \item \textbf{$\sigma(T_{\rm dur})$}: Uncertainty of $T_{\rm dur}$.
   \item \textbf{Alternate measure of transit duration}, $T_{1.5,3.5}$ [hr]: Mode of (T$_{1,4}$ + T$_{2,3})/2$. T$_{2,3}$ uses equation (2) from \citet{Seager:2003}.  If $b> 1-(R_p/R_\star)$ then T$_{2,3}$ is set to zero. Note $T_{\rm dur} > T_{1.5,3.5} \ge 0.5~T_{\rm dur}$.%We report the mode from the MCMC distribution of $T_{1.5}$.  
   %Alternate calculation of transit duration, measured from 2nd contact to 3rd contact (hr), for planets with $b< 1-(R_p/R_\star)$.  Based on transit model parameters (rhostar, per, r/R*, b).  uses formula (2) from \citet{Seager:2003}.
   \item \textbf{$\sigma(T_{1.5,3.5})$}: Uncertainty of $T_{1.5,3.5}$.
    \item \textbf{Signal to noise ratio}, S/N: The ratio of the signal, S, which is the integral of the transit model over all transits, to the noise, N, estimated as the standard deviation of the photometric lightcurve out of transit.  Calculated assuming a constant period if  TTVflag=0, and incorporating TTVs if the first digit of the TTVflag~=~1. S/N~$\geq$ 7.1 is a generally necessary (but not sufficient) for a KOI to be dispositioned as a PC in our catalog. {Four exceptions to this general requirement are made for KOIs with Kepler numbers whose lightcurves also passed our visual inspection (see Section \ref{sec:selection}).}
%We estimate the S/N of the observed transit by estimating the noise in the photometric lightcurve from the standard deviation ($\sigma$) of the detected lightcurve with out of transit observations compared to the transit model,
%\begin{equation}
%{\rm S/N} = \sqrt{\sum_{i=1}^{n} \left(\frac{Tm_i - 1}{\sigma} \right)^2},
%\end{equation}
%where $Tm_i$ is the value of the transit model for each observation, $i$.
   \item \textbf{MES}: Multiple event statistic, as reported in DR25.  KOIs that were not listed in DR25 have `0.0' listed in this column. 
   \item \textbf{Improvement in S/N when transit model allows for TTVs}: S/NwTTV~--~S/NwoTTV: Difference in S/N calculated without TTVs and with TTVs (e.g., \citealt{Ofir:2018}). Positive for most KOIs; typically large when the preferred fit uses TTVs (since less out-of-transit data obfuscates the signal) and small when it does not. 
   \item \textbf{$\chi^2_{\rm wttv}$}: Chi-squared calculated with TTVs based on the best fit transit model.  If the first digit of TTVflag = 0, then photometric uncertainties have been scaled to have $\chi^2_{\rm wottv}$ = DOF (degrees of freedom).  If the first digit of TTVflag = 1, then photometric uncertainties have been scaled to have $\chi^2_{\rm wttv}$ = DOF. 
   \item \textbf{$\chi^2_{\rm wottv}$}: Chi-squared calculated without TTVs.  The values are normalized as for $\chi^2_{\rm wttv}$. 
   %\item \textbf{$\delta\chi^2_{\rm inj}$}: Predicted difference between $\chi^2_{\rm wttv}$ and $\chi^2_{\rm wottv}$ based on injection tests. 
   \item \textbf{Scaled semimajor axis, $a/R_\star$}: Scaled semimajor axis.
   \item \textbf{$\sigma_+(a/R_\star)$}: Upward uncertainty of the scaled semimajor axis.
   \item \textbf{$\sigma_-(a/R_\star)$}: Downward uncertainty of the scaled semimajor axis.
   \item \textbf{Inclination, $i [^\circ]$}: Inclination of the planet's orbit relative to the plane of the sky; an edge-on orbit has $i = 90^\circ$. 
   \item \textbf{$\sigma_+(i)$}: Upwards uncertainty of the inclination. Note that $i+\sigma_+(i) \le 90^\circ$.
   \item \textbf{$\sigma_-(i)$}: Downwards uncertainty of the inclination.
   \item \textbf{Planet radius, $R_p$} [R$_\oplus$]:  Radius of the planet.
   \item \textbf{$\sigma_+(R_p)$}: Upwards uncertainty of $R_p$.
   \item \textbf{$\sigma_-(R_p)$}: Downwards uncertainty of $R_p$.
   \item \textbf{(Bolometric) Incident flux, $S$ [S$_\oplus]$}: Amount of flux intercepted by the planet relative to that intercepted by the Earth.
   \item \textbf{$\sigma_+(S)$}: Upward uncertainty of incident flux.
   \item \textbf{$\sigma_-(S)$}: Downward uncertainty of incident flux.
     \item \textbf{Kepmag}: Target star magnitude in the \ik passband.
   \item \textbf{Stellar density}, $\rhostar$ [g/cm$^3$]: The mean stellar density from the stellar parameter tables. (Not the estimate derived from the transit model, which is given in Column 14).  
   \item \textbf{$\sigma_+(\rhostar)$}: Upwards uncertainty of \rhostar.
   \item \textbf{$\sigma_-(\rhostar)$}: Downwards uncertainty of \rhostar.
   \item \textbf{Stellar temperature}, $\teff$ [K]: The target star's effective temperature, taken from the stellar properties catalog.
   \item \textbf{$\sigma(\teff)$}: Uncertainty of \teff.
   \item \textbf{Stellar radius}, $\rstar$ [R$_\odot$]: Radius of the star as given in the stellar properties catalog.
   \item \textbf{$\sigma_+(\rstar)$}:  Upwards uncertainty of \rstar.
   \item \textbf{$\sigma_-(\rstar)$}:  Downwards uncertainty of \rstar.
   \item \textbf{Stellar mass}, $\mstar$ [M$_\odot$]: Mass of the star. %as given in the stellar properties catalog.
   \item \textbf{$\sigma_+(\mstar)$}:  Upwards uncertainty of \mstar.
   \item \textbf{$\sigma_-(\mstar)$}:  Downwards uncertainty of \mstar.
     \item \textbf{Stellar surface gravity}, $\logg$ [cgs]: Surface gravity of the star as given in the stellar properties catalog. For stellar parameters taken from DR25 and \citet{Berger:2020a}, \logg\ is from isochrone models.  For stellar parameters taken from \citet{Fulton:2018}, \logg\ is based on spectroscopy. 
   \item \textbf{$\sigma_+(\logg)$}:  Upwards uncertainty of \logg.
   \item \textbf{$\sigma_-(\logg)$}:  Downwards uncertainty of \logg.
   \item \textbf{Metallicity}, [dex]:  Metallicity of the target star.
   \item \textbf{$\sigma$}(Metallicity):  Uncertainty of the metallicity of the target star. 
   \item \textbf{Stellar model source}:  0 = solar parameters, 1 = DR25, 2 = \citet{Berger:2020a}, 3 = \citet{Fulton:2018}, 4 = special (see Section \ref{sec:star}). {Aside from KOI-1792 (for reasons discussed in Section \ref{sec:selection}) and two binary star systems for which special parameters are used, }\citet{Fulton:2018} values are used where available, and the \citet{Berger:2020a} values are always preferred over those listed in DR25.
% \item \textbf{RUWE}: Renormalized Unit Weight Error in \gaia astrometry, with values taken from \gaia DR3. Values above 1.2 suggest that the source is a multiple star system.
 \item \textbf{Dispositions}: This four letter code gives the dispensations (status flags) from this work, DR25supp, DR25, and DR24 in reverse chronological order, with F = False positive (or False alarm), P = Planet/Candidate, N = not included in specified catalog,  S = rejected by us because S/N $<7.1^($\footnote{Note, the rounding done to produce the S/N numbers displayed in Column 31 of this table is done \emph{after} assignment of `S' in the flags. }$^)$, M = Jupiter-sized objects for which mass measured via RV clearly exceeds the 13 M$_{\rm Jupiter}$ limit for classification as a planet but otherwise would have been dispositioned `P' or `R' and R = Radius too large (see Section \ref{sec:selection}), but meets all other criteria. `N' is never applicable to the first column (our catalog provides dispositions for all KOIs listed) and `S', `M' and `R' are used exclusively for our dispositions. {The seven validated planets (with Kepler numbers listed on Nexsci) that failed either the radius cut or the S/N cut but passed visual inspection of the lightcurve were dispositioned `P' (see Section \ref{sec:selection} for details).}
\end{enumerate}

64~--~85. Values of the same properties reported in columns 39~--~44 and 46~--~61 with stellar parameters taken from \cite{Berger:2020a}. All zeros if host star not characterized by \cite{Berger:2020a}.\\

\bigskip
We treat KOIs with dispositions beginning with `S' and `M' as false positives throughout our study; KOIs whose dispositions beginning with `R' are treated as false positives for most purposes, but are included together with planet candidates in our study of the distribution of planetary radii (\S\ref{sec:sizes}).  The total number of planet candidates is 4376, including 35 mono-transits; additionally, there are 100 KOIs, including 2 mono-transits, that we vetted as `R'. There are 709 multiple transiting planet systems, which account for 1791 candidates, including {7 mono-transits}.%numbers from  \texttt{koiprops\_20211209.csv}, as well as z KOIs that we vetted as `R'.

Figure \ref{fig:perrad} displays the sizes and radii of almost all of the planet candidates in our catalog, and indicates the multiplicity of the planetary systems in which they are observed.  The bulk of multi-planet candidates have radii $R_p \lesssim 4$~R$_\oplus$ and orbital periods $2 \lesssim P \lesssim 100$ days.  Figure \ref{fig:perrad_multimap} confirms these general observations and demonstrates the paucity of additional transiting planets in systems hosting transiting hot jupiters (e.g., \citealt{Steffen:2012}) and planets with periods longer than $\sim 100$ days. The period-radius valley separating super-Earths from sub-Neptunes slopes downward from $\sim 2$~R$_\oplus$ at $P=2$~days to $\sim 1.5$~R$_\oplus$ at $P=40$~days (left panel of Figure \ref{fig:rpvsperwrstar}). %Although there is a trend for typical planetary multiplicity to increase as the radius of the planet decreases, there is only a small drop in the fraction of multis as a function of $R_p$ across the period-radius valley near 1.7~R$_\oplus$ (Fig.~\ref{fig:radius_multis_singlesKDE}).  

\begin{figure}[!hbt]
\begin{center}
\includegraphics[width=0.7\textwidth,angle=0]{RpvsPerwMPlfracmap.pdf}
\caption{The observed multiplicity of transiting planets from the \ik sample as a function of period and radius, with individual squares representing a factor of $10^{3/8} \approx 2.371$ in $P$ and $12.5^{1/6} \approx 1.523$ in $R_p$.  Black dots represent individual \ik planets and black stars represent the terrestrial planets in our Solar System.  The colored squares give the multiplicity fraction for each area that contains at least 4 planets.  A planet is considered to be part of a multiplanet system if more than one transiting planet candidate is seen in the photometric lightcurve.  If a square were to contain four planets, three of which were from multi-planet systems,  then the multi-planet fraction would be equal to 0.75.  The multiplicity fraction shows a paucity in multiplanets observed for hot jupiters, ultra-short periods ($P<1$ day) and long period planets ($P > 100$ days), with an observed increase in multiplicity as the planet radius decreases in the well-populated population with periods from 1~--~40 days.  
}\label{fig:perrad_multimap}
\end{center}
\end{figure}

Figure \ref{fig:perrad_snrmap} shows the distribution of transit S/N for planet candidates.  The left panel shows the total S/N calculated using Equation \ref{eq:S/N}, and the right panel shows the average S/N per transit.  The total S/N has a floor of 7.1 as discussed in \S\ref{sec:selection} and marks the boundary where the smallest planets can be found for a specified range in orbital period before the population is dominated by false alarms.  The rate of false alarms drastically increases for periods close to 1-year because of the {\it rolling band noise}, which is video cross-talk between detectors that produces a slowly drifting band of noise {and static {\it star-like} artifacts}. Rolling-band is most prevalent on detectors 22, 26, 44 and 58 and creates a mismatch in the sky estimate leading to a semi-periodic instrumental systematic that can resemble a photometric signature of a transit-like signal (see Table 13 from \citealt{vanCleve:2016}).  The 372 day heliocentric orbit of \ik and quarterly change of the spacecraft roll results in rolling band only being present once or twice per year for any given target star.  Significant effort was invested to identify and eliminate false-alarms with periods centered on 372 days without being overly aggressive against low S/N Earth-like transits \citep{Thompson:2018}. Nonetheless, visual examination of the left panel of our Figure \ref{fig:rpvsperwrstar} and Figure {7} of \citet{Thompson:2018} shows an overall increase in planet candidates centered on 372 days.  As noted by \cite{Burke:2019}, extra care and analysis is needed when assessing the statistical properties of planets in this regime, and this analysis would strongly benefit from a directed study of noise properties and follow-up observations from facilities such as {\it HST} and {\it Plato}. {Moreover, low-S/N long period PCs can also be produced by a few systematic dips in a lightcurve that line up to produce a signal that looks transit-like. Such chance alignments are common for TCEs that appear to transit just 3 or 4 times, but their frequency declines with 5 or more transits \citep{Mullally:2018}.} 


\begin{figure}[!hbt]
\includegraphics[width=0.49\textwidth,angle=0]{RpvsPerwrstar.pdf}
\includegraphics[width=0.49\textwidth,angle=0]{RpvsS0wrstar.pdf}
\caption{The \ik planet population showing radius ($R_p$) vs.~period ($P$) on the left and radius vs.~incident flux (S$_{\oplus}$) on the right.  The points have been colored by the radius of the host star ($R_\star$).  Solar System planets are marked with black stars, and the conservative habitable zone for an Earth-like planet around a Sun-like star \citep{Kopparapu:2014} is shown by the hashed green lines. 
}\label{fig:rpvsperwrstar}
\end{figure}

 {We computed minimum periods of each of the 7 mono-transit PCs in multis by examining their target's lightcurve to determine the shortest possible period for which no additional transits would have occurred at times when \ik obtained good data (i.e., was observing and photometric noise was not excessive). These lower bounds, as well as the (crude) estimates of the orbital periods using the duration and shape of the lightcurve together with the stellar properties and an assumed circular orbit (the absolute value of the negative  $P$ listed for these planets in Table~\ref{tab:planetcatalog}) and the periods of longest-period observed companion planet, are given in Table~\ref{tab:minp}. %There are 8 single-transit PCs in multi-planet systems, 435.02 (in a system with 5 other transiting PCs), 671.05 (4 other PCs), 693.03 (2), 1108.04 (3), 1870.02 (1), 2525.02 (1), 3349.02 (1), and 4307.02 (1); 
 None of these mono-transiting planets in multis have minimum periods significantly shorter than the period estimates from lightcurve analysis, which would be the case if their transverse orbital velocity at the time of transit was larger than that of a planet on a circular orbit with the minimum calculated period. Note that all of the multi-transiting PCs in these seven systems, apart from KOI-2525.01 and 4307.01, have been verified as planets and given \ik numbers.}
%\textcolor{red}{Fit straight line to TTVs and remove variations. Then take fourier transform, identify most significant peak, fit that frequency, remove that variation and repeat until no significant peaks remain. [JR: the plan is to only do a straight line, fourier decomp is expensive and not tested for corner cases]}


\begin{table}[]
    \centering
    \begin{tabular}{|c|c|c|c|c|}
        \multicolumn{5}{c}{}\\
        \hline
        KOI & $P$ [minimum] & $P$ [estimate] & system & $P$ [neighbor]\\
         & (days) & (days) & multiplicity & (days)\\
        \hline
        $435.02$  & $528.5$ & $934$ & 6 & $62.30$\\
        $671.05$  & $691.5$ & $4865$ & 5 & $16.26$\\
        $693.03$  & $588.0$ & $719$ & 3 & $28.78$\\
        $1108.04$ & $507.0$ & $1289$ & 4 & $18.93$\\
        $1870.02$ & $490.0$ & $550$ & 2 & $7.96$\\
        $2525.02$ & $418.5$ & $562$ & 2 & $57.29$\\
        %$3349.02$ & $455.5$ & $455$ & 2 & $82.17$ \\
        $4307.02$ & $483.0$ & $993$ & 2 & $160.85$ \\
        \hline
        \end{tabular} 
 \caption{ {The minimum orbital period of each of the mono-transit planet candidates in multi-planet systems based on \ik lightcurve coverage (the smallest value, stepping by 0.5 days, in which the data could not rule out a second transit) is listed together with the orbital period estimated from the transit duration and shape and the stellar properties, system multiplicity and the period of the transiting planet orbiting immediately interior to it.  The uncertainties of the estimated periods are poorly-quantified and likely to be substantial. }}
    \label{tab:minp}
\end{table}



Figure \ref{fig:gallery} provides a compact sketch of the architectures of all 709 multi-planet systems discovered by \ikt.  Systems are grouped into panels by number of planets detected and sorted within each panel by the orbital period of the innermost planet. Planetary radii and the presence of detected TTVs are also indicated for planets in systems with 4 or more PCs.  Note that all 81 systems with 4 or more planets have at least one planet with $P<13.5$ days, and only two such systems lack planets with $P<10.4$~days. The smallest period ratio of planet candidates plausibly orbiting the same star is 1.1167, between the two small $\sim 0.75$~R$_\oplus$ PCs KOI-3444.04 ($P = 14.15$~days) and  KOI-3444.01 ($P = 12.67$~days); no other pair of \ik PCs orbiting the same star has a period ratio smaller than 1.15.

\begin{figure}[!hbt]
\includegraphics[scale=1.2]{galleryttv7b.pdf}
\caption{ These plots show the orbital periods of each of the planets in every \ik multi-planet system.  The three panels show systems of differing multiplicity. Within each panel, all symbols along a given vertical line represent planets belonging to the same star, and the systems are ordered horizontally according to the  orbital period of the innermost planet.  The top panel presents systems with two (red) or three (blue) detected transiting planets%($N_{\rm tra} = 2-3$)
, the middle panel systems with four transiting planets, and the bottom panel systems with at least five transiting planets. The top panel does not plot KOI-1843.03 due to its short orbital period, $0.177$~day, which is $<40\%$ as long as that of any other \ik PC in a multi; however, its two companion planets are represented by blue points touching the vertical axis. In the middle and bottom panels, the symbol size is proportional to planetary radius, although in the middle panel planets larger than 5 R$_\oplus$ are shown as if $R_p = 5$ R$_\oplus$, and the systems are labeled by KOI number at the bottom, with red numbers being used for systems with %$N_{\rm tra} = 
6 -- 8 transiting planets. Colors of planets in the lower two panels denote the TTV flag given by \cite{Kane:2019} -- light yellow is 8 or 9 (moderate or strong TTVs), dark green is 7 (probable TTVs), and black is no TTVs (note, however, that recently-identified PCs were not examined by \citealt{Kane:2019}). The periods of mono-transit objects are constrained to be greater than the plotted values by not showing a second transit in the recorded data; see Table~\ref{tab:minp}. %numbers use data taken from \texttt{koiprops\_20211209.csv}
}
\label{fig:gallery} 
\end{figure}


\subsection{KOI-2433: A Candidate Seven-Planet System}\label{sec:2433}

Figures \ref{fig:perrad} and \ref{fig:gallery} show that there are now two \ik systems with more than six transiting planet candidates. One of these is the familiar 8-planet KOI-351 (Kepler-90) system (e.g., \citealt{Lissauer:2014, Shallue:2018}). With the addition of KOI-2433.08 from \cite{Shallue:2018}'s list, a second \ik target now has eight KOIs, which we elaborate upon in the following paragraphs. 
According to Table \ref{tab:planetcatalog}, the 10 and 15 day planets, which were validated by \cite{Rowe:2014} as Kepler-385~b \& c, both have S/N $\sim$ 28. The 56 day planet has an S/N = 16.4 and was validated as Kepler-385~d by \cite{Armstrong:2021}. The S/N of the other five KOIs range from 11.1~--~14.3, above the S/N~$> 10$ required as one of many tests for planet validation of PCs in multis by \cite{Rowe:2014}. 
 
The 0.6 day KOI 2433.05 has disposition FFFF\footnote{See the item \#63 in the numbered list in \S\ref{sec:unified} for an explanation of the dispositions given in the last column of Table \ref{tab:planetcatalog}.}. Its lightcurve clearly exhibits secondary events indicative of an eclipsing binary. Furthermore, pixel-level data show the periodic dimming to be spatially offset from the target star. Thus, we do not consider this ultra-short period KOI further.

The {three validated planets plus the 28 day} KOI have dispositions PPPP. The disposition of the 6 day KOI is PPNP; its S/N~=~12.4. The 86 day KOI has PPFN; its S/N~=~11.2. The 3.4 day KOI-2433.08 has PNNN; its S/N~=~11.1.

The outer three planet candidates have neighboring pairs with period ratios nominally placing them just wide of first-order MMRs ({as does the pair of validated planets orbiting interior to this threesome}), increasing the likelihood of them being real planets \citep{Lissauer:2014}. Indeed, it is possible that the validated pair of planets are locked within a two-body resonance and that the three outer planets librate within a three-body resonance. Period ratios between the planets all exceed 1.5, which is sufficient for system stability provided the planets have masses typical for their sizes and small eccentricities, which are the norm for \ik planets in systems of high multiplicity (Fig.~\ref{fig:EccMulti}).  In sum, we consider KOI-2433 as having seven strong planet candidates. Nonetheless, validation of all of these candidates to well above 99\% probability of representing true planets is beyond the scope of this paper.

\smallskip

\section{Characteristics of the Planet Population: Multis vs.~Singles} \label{sec:singles}

\ik found far more multiple planet candidate systems (multis) than would be the case if candidates were randomly 
distributed among target stars \citep{Lissauer:2011b, Latham:2011}. \cite{Lissauer:2012} presented a statistical analysis that combined the large numbers of multis observed by \ik that were listed in \citealt{Borucki:2011b} (as modified by \citealt{Lissauer:2011b}) together with the assumption that false positives are nearly randomly distributed among \ik targets to demonstrate that the fidelity of \ik multiple planet candidates is far higher than that for singles.  \cite{Lissauer:2014} expanded upon the statistical analysis of \cite{Lissauer:2012} and developed techniques that \cite{Rowe:2014} used to validate more than 700 of \ikt's multiple planet candidates as true planets. 

The distribution of stellar host magnitudes for singles and multis are shown in Figure \ref{fig:magnitudes}. We compare the S/N distributions of \ik singles and multis in \S\ref{sec:snr} and estimate the fraction of apparent \ik multis in Table \ref{tab:planetcatalog} that do not represent planets orbiting the same star in \S\ref{sec:falsemultis}.

\begin{figure}
    \centering
    \includegraphics[width=0.7\textwidth]{snr_hist_kepler_mag_counts.pdf}
    \caption{Histogram of \ik planet hosting star magnitudes, with hosts of a single PC in blue and stellar hosts of multis in red.  All PCs that were observed by \ik are represented in this plot.}% Data taken from \texttt{koiprops\_20230830v2.csv}}
    \label{fig:magnitudes}
\end{figure}

Many groups, dating back to the aforementioned \citet{Lissauer:2011b} and \citet{Latham:2011}, have compared characteristics of the distributions of single planet candidates in the \ik sample with those of individual planets in multis. We compare size distributions (in \S\ref{sec:sizes})   and orbital period distributions (in \S\ref{sec:periodDist}) for ensembles of planets in singles to those within multis and between planets in two-planet systems to those in systems of higher multiplicity. The distributions of normalized transit durations and eccentricities of subsets of planets segregated by multiplicity, orbital period, size and orbital spacing are presented and analyzed in \S\ref{sec:ecc}. 

Up through Subsection \ref{sec:periodDist}, all PCs are accounted for in the determination of the multiplicity of a planetary system. For our analysis of planetary size distributions (\S\ref{sec:sizes}), we treat KOIs dispositioned as ``R'' as if they were PCs, but do not count PCs with high impact parameter or  S/N~$< 12$. In \S\ref{sec:periodDist}, we omit PCs with S/N~$< 12$ as well as those with only one observed transit. In \S\ref{sec:ecc}, we employ somewhat different restrictions described in the introduction to that subsection.


\subsection{Signal to Noise Distributions {\& Reliability of the Sample}}\label{sec:snr}

Figure \ref{fig:snr_comparison} compares the histograms of singles vs.~multis as functions of S/N. Note that there are similar numbers of multis and singles over a wide range in S/N, but singles dominate for both the smallest values of S/N and the largest ones. Examining the distributions more quantitatively, \ik found almost 50\% more planet candidates in singles than in multis, but Figure \ref{fig:snr_comparison} shows that the numbers of singles and multis are nearly equal across the S/N range 25~--~180.  Below S/N~=~12 and above S/N~=~300, well over twice as many singles as multis have been identified; intermediate ratios are found in transition regions (see Figure \ref{fig:snr_comparison}).  The predominance of singles at high S/N is primarily accounted for by the paucity of large planets, especially hot Jupiters, in multis \citep{Latham:2011, Steffen:2012}. 

\begin{figure}
    \centering
    \includegraphics[width=0.7\textwidth]{snr_hist_counts_log.pdf}
    \caption{Number of singles (blue) and multis (red) as functions of S/N. The dotted red curve is restricted to the lowest S/N planet in each multi-planet system. All PCs that were observed by \ik are represented. See Fig.~\ref{fig:perrad_snrmap} for a plot of how typical S/N varies with planetary radius and orbital period.  }% Data taken from \texttt{koiprops\_20230830v2.csv}}
    \label{fig:snr_comparison}
\end{figure}

The excess of singles at low S/N is probably caused by a combination of the following effects, {the first two of which are related to \ik multis being a highly reliable subsample of the \ik planet candidate population \citep{Latham:2011, Lissauer:2011b, Lissauer:2012, Lissauer:2014, Rowe:2014}}: (1) A larger fraction of singles being false alarms (this wasn't the situation in 2014 because of the more aggressive search for PCs in multis \citep{Rowe:2014}, but probably is the case now because of the more automated procedure used to find and vet KOIs in recent catalogs); (2) Planet candidates with low S/N cannot be tested as rigorously as can PCs with high S/N, so a larger fraction of EBs and other FPs are included in the PC list, since they are more common among targets with a single transit-like pattern than around targets with more than one such pattern \citep{Latham:2011, Lissauer:2011b, Lissauer:2012, Lissauer:2014, Rowe:2014}; (3) When multiple planets transit, the detectability of planets other than the highest S/N candidate by the \ik pipeline is reduced significantly \citep{Zink:2019}, especially if the highest S/N planet has low S/N, so it is less likely that other transiting planets in the system have been detected; (4) TTVs, which are detected in more than twice as large a fraction of multis compared to singles (Table \ref{tab:planetcatalog} and \S\ref{sec:periodDist}), reduce MES, but if they are accounted for in our model fits, they don't reduce S/N; and, perhaps, (5) Real differences between the populations, such as longer period planets having lower S/N, all other factors being equal, combined with neighboring companion planets typically being within a factor of $\sim 2 - 3$ in period and planets less likely to transit at long periods unless typical relative inclinations of planetary orbits decrease significantly as periods increase.  Determining what fraction of the difference is caused by (3) is important both to a comparison between multis and singles and to understanding the fidelity of the sample of single planet candidates, but it is beyond the scope of this work. 

%From Zink et al: Planet multiplicity introduces detection biases above and beyond those to which single transit systems are subject. When faced with a system of multiple transiting planets, the Kepler pipeline will typically find the largest MES (Multiple Event Statistic; comparable to SNR) signal, fit the transit function, and then discard the corresponding data points. The width of discarded data is 3 times the transit duration, with 1.5 times removed on each side of the transit centre. Very few TTVs (Transit-Timing Variations) are large enough to escape this window. Such deletion is necessary to avoid confusion when looking for additional planets, but introduces data gaps into the lightcurve as noted by Schmitt, Jenkins \& Fischer (2017).
%Differences at high S/N:
% Max among all singles: 5952.0; 2nd is 4360.4; 37 exceed 1000; 36 of which are > 10R_e; 33 have P<10 days
% Max among all multis: 1244.3; 2 exceed 1000, both > 10 R_e; both have P>10 days; 
% 2: 1244.3
% 3: 577.1
% 4: 1120.6
% 5: 304.5
% 6: 242.7
% 7: nan
% 8: 355.0

% Top 5 snr for multis
% KOI     SNR
% 3681    1244.3
% 94      1120.6
% 191      812.6
% 129      602.9
% 377      577.1

% planets with Period < 0, because only a single transit.
% Multis (7):{435, 671, 693, 1108, 1870, 2525, 3349}
% 435.02 has 5 siblings, all verified, 671.05 has 4 siblings, all verified, 693.03 has 2 siblings, all verified, 1108.04 has 3 siblings, all verified, 1870.02 has 1 sibling, verified, 2525.02 has 1 sibling, NOT verified and 3349.02 has 1 sibling, verified
% Singles (27):{8320, 8321, 8323, 8325, 8330, 8331, 8332, 1421, 8333, 8334, 1174, 7194, 1208, 99, 8304, 8305, 8306, 8307, 8310, 8311, 8313, 8314, 8315, 8316, 8317, 8318, 8319}
\smallskip
\subsection{Split Multis \& Orbital Period Aliases}\label{sec:falsemultis}

Searching through photometric time series for transiting planets may yield false positives, a term that conventionally means a real astrophysical signal within the same detector pixels, but caused by something other than a planetary transit.   For the purpose of computing occurrence rates, a subclass of false positives is signals resulting from transits of real planets  that are \emph{not} hosted by the intended target star, such as the planets observed to transit KOI-119 (\S\ref{sec:star}). Such blending causes the planetary radius to be significantly under-estimated (as the target star is invariably the one providing the most photons absorbed by the pixels), and the stellar host type and other parameters can be incorrect, to the detriment of statistical occurrence efforts. These real planets, which we classify as PCs, should therefore be termed false positives for the purposes of computing planetary occurrence rates. {(The number of \ik planet candidates identified as likely to orbit stars other than the \ik target star is quite small, and all prime suspects that we know of are discussed within the first half of this subsection.)}

In the case of candidate multi-planet systems, it could be that each periodic signal is due to a real planet, but these planets are not orbiting the same star. Recognizing the planets' reality, we do not use the word ``false'' here, but instead call these systems ``split multis". Although they are real planets, split multis can be a source of contamination for dynamical studies. Given that the orbital periods of planet candidates span a very large range, random planets around different stars will not necessarily appear strange when (mis)interpreted as orbiting the same star. 
In some fraction of cases, however, we would notice their periods are too close together to be stable if interpreted as being around the same star. 
% EBF Added:
Given the steep dependence of occurrence rates on planet size, split multis are expected to be most common for binary stars of similar luminosities.  
Therefore, when studying multiple planet systems, it can be advantageous to restrict one's attention to a subset of \ik targets that have been filtered to minimize contamination from binary stars with similar luminosities \citep[e.g.,][]{Hsu:2019,He:2020}.

Here we discuss a few potential split multis.  
KOI-284, first listed in the catalog of \cite{Borucki:2011b} and introduced here in \S\ref{sec:star}, includes the planet candidates KOI-284.02 (with orbital period is $P = 6.415$~days) and KOI-284.03 (with $P = 6.178$~days), both of which appear to be larger than Earth.  \cite{Lissauer:2011b} noted that if both of these planet candidates represented planets orbiting the same stellar host, then for any reasonable densities their proximal orbits would lead to dynamical instability on a short time scale. Thus, they do not represent planets in the same planetary system; this system is the prototypical \ik example of a split multi. Further investigation revealed that the target ``star'' is actually a binary system with nearly identical components \citep{Lissauer:2012}, and both of these PCs were subsequently validated as true planets, one orbiting each star, by \cite{Lissauer:2014}.  These two planetary systems are now known collectively as Kepler-132. There are two additional validated planets, with orbital periods of 18 and 110 days.  However, it is not yet known which member of the stellar binary any of the planets orbit, apart from the two 6 day period planets needing to orbit different stars.

Two planet candidates apparent in the lightcurve of KOI-2248, denoted KOI-2248.01 ($P = 2.818$~days) and KOI-2248.04 ($P = 2.646$~days), were first listed in the \cite{Batalha:2013} catalog, and they were highlighted as a split multi by \cite{Fabrycky:2014}. Neither of these KOIs has ever been dispositioned as a false positive in any KOI catalog, including our own (Table \ref{tab:planetcatalog}). Two other planet candidates in the system were listed in DR25supp, KOI-2248.02 ($P = 9.49$~days) and KOI-2248.03 ($P = 0.762$~days). \cite{Shallue:2018} listed a ``new'' candidate associated with the same \ik target and $P = 4.745$~days that we have included in our catalog and dubbed KOI-2248.05 (\S\ref{sec:selection}).  None of the KOIs in the system has yet been verified as a {\it bona fide} planet. One member of the pair of planet candidates with similar periods, KOI-2248.04, has S/N = 8.4 in our catalog, which is quite peculiar given that it was identified so early in the \ik mission, calling its veracity into question.  The other member of that nominally-unstable pair, KOI-2248.01, has fairly low DR25 disposition score of 0.895, but we find that it has a respectable S/N = 16.5.  The 9.5-day signal, KOI-2248.02, has an unacceptably low S/N = 4.5, so we classify it as a false alarm, probably an alias of KOI-2248.05, which has a period almost exactly half as long.

Thus, KOI-2248 hosts four planet candidates, two of which (both somewhat suspect for other reasons) could not represent planets orbiting the same star, as a system containing both of these putative planets would be dynamically unstable. High-resolution imaging of the target star KOI-2248 has been carried out by  \cite{Furlan:2017}, whose Table 8 (accessed by Vizier, 16 July 2018)  lists KOI-2248 with two stellar near neighbors on the sky plane detected by WFC3 (Hubble Space Telescope): one (denoted B) that is 0.169 magnitude fainter in the F555W filter band (a nearly-equal brightness companion) with a separation of just 0.148 arcsec, so it could well host one of the $P \sim 2.7$~day PCs.  The other stellar sky-plane neighbor (denoted C) is 3.872 arcsec away and 5.318 magnitudes fainter, so it is not likely to be the host of either planet. The only survey using WFC3 listed in Table 1 of \cite{Furlan:2017} is that by \cite{Gilliland:2015} and \cite{Cartier:2015}, but neither of these latter papers addresses KOI-2248, so we suspect that the results were posted to the KFOP (\ik Follow-up Observing Program) website, but not published in the refereed literature.

Although Table \ref{tab:planetcatalog} does not list any proximate-period split multis other than those in KOI-284 and KOI-2248 as discussed above, an early draft version of this table listed the planet candidate KOI-521.02 with an orbital period $P = 10.82$~days, which would place it too close to the Neptune-size PC KOI-521.01 ($P = 10.16$~days) for dynamical stability. We therefore re-examined the \ik lightcurve of this target and found that alternate transits of KOI-521.02 had been missed, so the actual period of this PC is only half of what is listed in previous catalogs, $P = 5.41$~days. The Q1-Q16 KOI catalog was the first to list  KOI-521.02, and dispositioned it as an FP \citep{Mullally:2015}.  In the subsequent catalogs DR24, DR25 and DR25supp, it was listed as a PC, albeit one with weak S/N (initially 8.7 and later 7.3; with the corrected period it has a more respectable S/N = 9.7).  The estimated orbital periods of both KOI-521.01 and 521.02 changed by less than 1 part in $10^4$ between previous catalogs, and were similarly close to these estimates in early drafts of our catalog.  The TCE searches for DR24 and DR25 both identified KOI-521.02 with the correct 5.41 day period. However, because this TCE had a period that differed by a factor of two from a previously-cataloged KOI on the same target and the orbital phases matched, it was assigned the previous period in both of those PC catalogs. 

A systematic search for period aliasing was done in \S5.4 of the \cite{Batalha:2013} catalog paper. However, it is unlikely that the error in the initial period estimate for KOI-521.02, which was not identified as a KOI until a later catalog search, would have been recognized had we not given it special scrutiny because it appeared to be part of a split multi. This example suggests that some nontrivial fraction of low S/N PCs in our catalog could have listed periods that err by a factor of two. Several other PCs have had their estimated periods changed by a factor of roughly two subsequent to their first appearance in an official \ik KOI catalog, most notably KOI-730.03, which was initially listed with a period of half its true value \citep{Borucki:2011b} that would have placed it within a 1:1 (co-orbital) resonance with KOI-730.02, but the additional scrutiny that this putative pair of Trojan planets received quickly led to its estimated period being corrected by \cite{Lissauer:2011b}.  By the end of the mission, the DR24 TCE table, KOI-730.03 was found with its correct period, with a large MES of 17. So in the calculation below, we do not count it as an alias corrected by dynamical considerations. 

Note that PCs with estimated periods (nearly) identical to or a factor of two different from the period of another KOI of the same target may suffer from the discarding of data surrounding the transit of the first candidate \citep{Schmitt:2017}, a type of aliasing not an issue when periods are not commensurate.  This effect contributed in the case of KOI-2248.02, but not for KOI-730.03, whose transit phase differed substantially from that of its commensurate sibling.  Both KOI-730.03 and KOI-521.02 had their periods adjusted only after receiving additional attention due to apparent co-orbital planets or unstable systems. We do not know of other PCs that were corrected for dynamical reasons.

Motivated by these examples, we next estimate the number of period aliases that likely remain in the sample of multis. The number of PCs {with measured periods in multis that have at least one other PC with measured period is 1781. We took each of the pairs of planets in the same target, 1610 in total, multiplied the lower orbital period by 2, and counted how often that change would make the pair unstable. Following \cite{Lissauer:2011b} and \cite{Fabrycky:2014}, planetary masses were taken as  $(R_p/$R$_\oplus)^\alpha$M$_\oplus$, where $\alpha=3$ for $R_p<$~R$_\oplus$ (sub-Earths), $\alpha=2.06$ for R$_\oplus<R_p<$~R$_{\rm Saturn}$, and finally, M$_{\rm Saturn}$ for $R_p>$~R$_{\rm Saturn}$. Pairs of planets were deemed unstable if their difference in semimajor axis was less than $2\sqrt{3}$ their mutual Hill separation \citep{Gladman:1993}. No higher-order accounting of stability (for triples, for instance) was performed for this calculation. Stability requirements for systems of $\geq 3$ planets are more restrictive, but cannot be expressed in such simple terms (see, e.g., \citealt{Petit:2020} and \citealt{Lissauer:2021}).

% These 1780 mock systems, each of which has one alias relative to the actual planet candidates, hosted 259 unstable pairs. Since only 1 out of 1616 real pairs of planets were found (via instability) to be an alias, then the probability that a planet candidate has an aliased period is (1/1616)/(259/1780) $\approx 0.43$\% . Another 
% Among planets in multis, there are roughly $0.0035\times1773-1 \approx 7$ yet-undiscovered aliased planets, though since this is based on only $1$ such detected system, it should be considered an order of magnitude statement. 

Of these 1610 mock pairs, 281 ($17.5$\%) were unstable. Since only 1 out of 1610 real pairs of planets were found (via instability) to be an alias, then the estimated rate for a planet candidate to have an aliased period is (1/281) $\approx 0.36$\%.  
This rate is small enough that we can safely neglect the rate of systems becoming unstable due to multiple PCs being affected by  aliases in the same system.   
% Since some systems contain more than two planet candidates, an alternative method to estimate the rate of planet candidates with an orbit period involves  multiplying one planet's period by 0.5 and 2.0 and leaving the orbital periods of other planets in the system unchanged.  This leads to $N_{pl}-1$ mock systems per $N_{pl}$ multiple PC system.
% With this approach, we find $1773\times2=3546$ mock systems, of which 518 have an unstable pair. The estimated rate of planet candidates with an aliased period is (1/1773)/(518/3546) $\approx 0.39$\, nearly identical to our first estimate.
%
Among planets with multiple transits in multi-transiting systems, we expect roughly $0.0036\times1781 \approx 6$ aliased planets, $5$ of which are yet-undiscovered, though since this is based on only $1$ such detected system, it should be considered an order of magnitude statement. If the systems with just a single transiting planet with a quoted period also have this aliasing problem at the same rate, then $0.0036\times2550 \approx 9$} are also aliased.  Aliasing may occur at a different rate among singles, however, as the signal to noise distribution differs (Fig.~\ref{fig:snr_comparison}), and the dataset when searching for additional transits needs to be censored in certain areas. 

%Using a similar approach, we update the estimated rate of split multis from \cite{Fabrycky:2014}. 
%We chose pairs of ($P$, $M_p/M_\star$) values of all the planet candidates in multis, and determine that 85,826/1,569,106~$\approx 5.5$\% are unstable. Having two detected unstable pairs (via the split multi channel) out of 1605 PC pairs, we estimate that $\sim 36$ of our pairs may actually be split multis. Thus, split multis are likely a larger contaminant for the study of planetary system architectures than are period aliases, even though only $\sim2.3$\% of sampled pairs in multis are expected to include a planet orbiting a star other than the target star. 

Using a similar approach, we update the estimated rate of split multis from \cite{Fabrycky:2014}. 
We choose pairs of ($P$, $M_p/M_\star$) values of {all of the planet candidates, and determine that 453,281/9,419,770~$\approx 4.8$\% are unstable\footnote{{\cite{Fabrycky:2014} restricted their choice to pairs of planet candidates in multis. Applying that prescription to our data set results in $88,758/1,590,436 \approx 5.6$\% being unstable, a slightly larger percentage since the period distribution of PCs overall is broader than that of PCs in multis (Fig.~\ref{fig:period_multiplicities_split}). This yields an estimate of $\sim 36$ split multis. We prefer using all PCs because split multis can include planets that are single and/or those that are in multis.}}. Having two detected unstable pairs (via the split multi channel) out of 1610 PC pairs, we estimate that $2/0.048\sim 42$ of our pairs may actually be split multis. Thus, split multis are likely a larger contaminant for the study of planetary system architectures than are period aliases, even though only $\sim2.6$\% of sampled pairs in multis are expected to be planets orbiting different stars.} 

The above estimates suggest that, among planets in multis, there are $\sim 7$ times as many split multis as period aliases. A factor of 2 comes from twice as many split multis having been identified.  A slightly larger factor is because period aliases are more likely to result in instability since it's fairly common for a pair of planets in the same system to have a period ratio near two (Fig.~\ref{fig:gallery}; see also \citealt{Fabrycky:2014}), but the overall distribution of periods is quite broad (Fig.~\ref{fig:period_multiplicities_split}), so few pairs of randomly-selected planets have a period ratio near unity.

The estimates of the number of period aliases calculated in this section do not account for the aliases found for planets with periods $\lesssim 1 $~day analogous to those noted in Section \ref{sec:selection}, which are caused by \ik automated pipeline searches for TCEs being limited to signals with periods $> 0.5$ day. Aliases among ultra-short period planets are very unlikely to be found via the techniques discussed in this section because only one planetary system (Kepler-42) is known to have more than a single planet with $P < 2.25$~days (e.g., \citealt{Steffen:2013b,Lissauer:2023}).

\subsection{Size Distribution} \label{sec:sizes}

A total of 100 KOIs, none of which are in multis and 73 of which have $P > 10$~days, are dispositioned as ``R'' in Table \ref{tab:planetcatalog},
%, including  in multis and  singles with $P > 10$~days.
indicating that they were rejected as planet candidates solely due to their estimated size.  The placement of our upper bound on the estimated radius of a body that we classify as a planet candidate (\S\ref{sec:selection}) is somewhat arbitrary, so we consider KOIs that are rejected based on size alone together with planet candidates (KOIs vetted as ``P'') for all studies presented in this subsection.  

Figure \ref{fig:cdfimpactparambgt1} contrasts the fractional cumulative distributions of impact parameters of small and large planets in singles and multis. Small planets are clearly underrepresented for $b$ very close to unity because near-grazing transits of small planets are difficult to detect, and they are quite rare for larger impact parameters because $R_p/R_\star \gtrsim b-1$ is a requirement for a transit to be observable. Therefore, when comparing radius distributions, we only consider planets with estimated impact parameters $b + \sigma_+(b) <$ 0.95. (Because we report planet sizes using the modes of the distributions, we do not need to adopt the stricter $b < 0.8$ cutoff used by \citealt{Petigura:2020}.) We also exclude from our analysis in this subsection those candidates with S/N $< 12$ because the population of low S/N candidates has more FPs and typically higher fractional uncertainties on estimates of $R_p$; this cut removes the 4 PCs around targets for which solar parameters were assumed, for which planetary radii are especially poorly estimated. Planets that are excluded from the counts by these cuts are nevertheless included when determining the multiplicity of the system in which companion planets that meet these criteria reside.

%Kadri TODO: Check numbers in caption for number of planets b > 1
\begin{figure}
    \centering
    \includegraphics[width=0.7\textwidth]{radius_w_mono_cdf_impact_param_over_radii_subsets.pdf}
    \caption{The cumulative impact parameter distributions of various subsets of planet candidates, as well as KOIs rejected solely because they are too large (dispositioned as R in Table \ref{tab:planetcatalog}), that were observed by \ikt. Red curves represent planet candidates in multis, blue curves single planet candidates; small planets are shown by solid lines and large planets by dashed  lines. There are 6 small ($R_p < 5$~R$_\oplus$) and 54 large singles that have $b > 1$. In multis, there are 2 small and 1 large planets  that have $b > 1$.  PCs with S/N~$< 12$ are not counted for the CDFs, but they are considered when determining the multiplicity of the systems in which the counted planets reside.}% Data taken from \texttt{koiprops\_20230830v2.csv}.}
    \label{fig:cdfimpactparambgt1}
\end{figure}

Figures \ref{fig:radius_multis_singles} and \ref{fig:large_radius_multiplicities_split} compare the size distributions of ensembles of planets in systems of different multiplicity.  In all cases, we lump together systems with three or more planets to have adequate numbers of planet candidates for statistically robust results. 

Giant planets are more common among \ik singles than among \ik multis  \citep{Latham:2011}. Nonetheless, when the cumulative size distributions for singles and planets in multis are normalized to unity at $R_p = 5$~R$_\oplus$ (Figure \ref{fig:radius_multis_singles}), the curves for planets up to this size are very similar{. Over the the size range $R_p < 5$~R$_\oplus$,   differences between the distributions are of  marginal statistical significance, and for $R_p < 2.5$~R$_\oplus$ the distributions of singles and planets in multis are consistent with being drawn from the same population.}%, providing no strong evidence for different size distributions of small and medium-sized planets between multis and singles. However, the differential size distributions shown in Figure \ref{fig:radius_multis_singlesKDE} suggest that planets below the radius valley may make up a slightly larger fraction of the population in multis than in singles. The numbers are dominated by planets smaller than 3~R$_\oplus$, so significant differences may exist for planets 3~R$_\oplus < R_p < 5$~R$_\oplus$. Caution is advised when interpreting these differences as

There are, however, various biases in the discovery of planets in multis as opposed to singles that could allow the actual distributions of transiting planets with $R_p < 2.5~$R$_\oplus$ to have some size dependence. Those tilting the size distribution of small planets towards singles include: detecting  a transiting planet in a \ik lightcurve reduced the amount of data used by the \ik pipeline to search for additional planets and thereby lowered the efficiency of detecting any other transiting planets associated with the same target star, reducing the probability of finding more planets, especially small ones \citep{Zink:2019}; geometric factors imply that a larger fraction of planets of a given orbital period transit large stars than small stars, and small planets are difficult to detect around large stars. By contrast, less photometrically noisy, brighter and/or smaller stars make all (but especially small) planets easier to detect, yielding a bias towards detecting multiple small planets around the best target stars.  Searching for additional planet candidates has at times been more aggressive for targets with at least one candidate already identified, and some searches for transiting planet signatures such as that of \citet{Shallue:2018} have focused exclusively on lightcurves of targets already known to possess planet candidates.

% [singles]: P-disp: 21, R-disp: 112
% [multis]: P-disp: 1, R-disp: 9

%Kadri TODO: Check numbers in caption for number of planets b > 1
\begin{figure}
    \centering
    % \includegraphics[width=0.75\textwidth]{Current_Figures/Section_3/singles_and_multis_radii_cdf_5.pdf}
    %\includegraphics[width=0.45\textwidth]{Current_Figures/Section_3/singles_and_multis_radii_cdf_4,5.pdf}
    \includegraphics[width=0.48\textwidth]{radius_w_mono_cdf_singles_multis_log.pdf}
    \includegraphics[width=0.48\textwidth]{radius_w_mono_cdf_m2_m3_plus_log.pdf}
    \caption{Normalized cumulative distribution function (CDF) of planetary radii of planets for specified system multiplicity. {The panel on the left compares singles with planets in multis, whereas the panel on the right compares planets in two-planet systems with those in systems of higher multiplicity.} We normalized the CDFs to unity at $R_p = 5$~R$_{\oplus}$. Candidates with S/N $< 12$, or $b + \sigma_+(b) \geq$ 0.95 are excluded from this radius distribution, although planets are considered to be in multis even if all of their companions fail to meet one or both of these cuts. These plots include KOIs rejected because they are too large (dispositioned as R in Table \ref{tab:planetcatalog}). Only the portion of the distributions with  $R_p < 30$~R$_{\oplus}$ are shown, although larger bodies that satisfy our criteria are included in the computation of the numbers given within brackets. No candidates in multis satisfying our upper limit on impact parameter have $R_p > 30$~R$_{\oplus}$, whereas 1 singles with status P and 12 with status R are larger than this value and not shown. (The green triangle with $P\approx150$ days in Fig.~\ref{fig:perrad} represents KOI-1426.03, which has V-shaped transits and $b>1$.)%Normalized cumulative distribution function (CDF) of planetary radii of planets that have observed transiting companions (red) and those that do not (blue). We normalized the CDFs to unity at $R_p = 5$~R$_{\oplus}$. Candidates with S/N $< 12$, or $b + \sigma_+(b) \geq$ 0.95, or for which just one transit has been observed, are excluded from this radius distribution, although planets are considered to be in multis even if all of their companions fail to meet one or both of these cuts. This plot includes KOIs rejected because they are too large (dispositioned as R in Table \ref{tab:planetcatalog}). Only the portion of the distribution with  $R_p < 16$~R$_{\oplus}$ is shown, although larger bodies that satisfy our criteria are included in the computation of the numbers given within brackets. Among candidates larger than this, 11 singles have status P, 36 have status R, 0 multis have status P and R. }% Data taken from \texttt{koiprops\_20230830v2.csv}}
    \label{fig:radius_multis_singles}}
\end{figure}%

Plotting a similar comparison between the size distributions of planets in systems with two planet candidates vs.~those in higher-multiplicity systems (the panel on the right in Figure \ref{fig:radius_multis_singles}) shows a slight excess of super-Earth size planets relative to \hbox{(sub-)Neptunes} in systems with three or more PCs, with the distribution of two-planet systems lying between those for single planets and those for high multiplicity systems. While the differences between singles and multis over the entire range in radii are highly significant, both the Kolmogorov-Smirnov test and the Anderson-Darling test {show marginally significant differences between 2's and 3+'s. These marginal differences persist when restricting to the range $R_p < 5$~R$_\oplus$, but such differences are no longer evident if restricting to $R_p < 2.5$~R$_\oplus$, even though the majority of planets in multis are smaller than 2.5~R$_\oplus$.}

% A top hat KDE to show distribution of planet radii.
% logscale on x-axis from 0.2 to 20 Rearth.
% singles in blue, multis in red, (all in black?)
% begin with width of 0.02 decades? 


The divergence between the curves in the left panel of Fig.~\ref{fig:radius_multis_singles} comes in gradually as $R_p$ increases. There is no sign of divergence below  4~R$_\oplus$, but it is plainly there above 5~R$_\oplus$.  Nonetheless, it is clear that multis are deficient relative to singles for planets that are larger than Neptune. Note that because \ik has detected far more small planets than large ones, errors in estimates of $R_p$ could well mean that a small fraction of $\sim 3$ R$_\oplus$ planets with overestimated sizes contribute a substantial fraction of the population of apparently Neptune-size objects, and the actual transition between the multis-rich population of ``small'' planets that are $\lesssim 1$\% H/He by mass and the multis-poor population of gas-rich planets may occur closer to 4~R$_\oplus$. Indeed, although this transition {\it appears to} occur at a somewhat larger radius than that of the radius cliff, which is the sharp reduction in the overall occurrence rate of \ik planets observed near 3~R$_\oplus$ \citep{Kite:2019, Hsu:2019} that manifests as a reduction in slope in all of the curves in Fig.~\ref{fig:radius_multis_singles}, the break in the ratio of number of multis to that of singles could be coincident with the radius cliff.

Figure \ref{fig:large_radius_multiplicities_split}
compares cumulative size distributions of large planets (and KOIs that we dispositioned as R because they failed our upper radius cutoff)  in multis vs.~singles and in two-planet systems vs.~higher multiplicity systems. The size distributions of singles and multis are essentially indistinguishable  over the range 4.5~--~10~R$_\oplus$, but the number of planets in multis  in this range % 72 multis, 138 singles
is only a little more than half the number in singles, whereas similar numbers are present for smaller planets.  Very few planets in multis have $R_p > 12~$R$_\oplus \approx$~1.1~R$_{\rm Jupiter}$, but plenty of singles have radii larger than 12~R$_\oplus$ (e.g., \citealt{Santerne:2016}).  The divergence of the curve for singles with $P > 10$ days from that for all singles shows that almost half of the members of the plotted population with  $R_p > 12$~R$_\oplus$ orbit within the period range of inflated hot Jupiters.  Most of the excess at longer periods (as well as some at short periods) is probably caused by false positives, which could be nearby (on the sky plane) eclipsing binaries or transits of the target star by ultracool dwarfs that are too faint to show an occultation (sometimes referred to as a secondary eclipse) deeper than can be explained by heating of the dayside by radiation from the primary star, or which travel on sufficiently eccentric and inclined orbits that no such occultation occurs. This was our motivation for classifying KOIs that otherwise would have been considered PCs that have $P>20$~days and $R_p > 1.2$~R$_{\rm Jupiter} \approx 13$~R$_\oplus$, i.e., significantly, albeit not substantially, larger than this boundary, with the disposition ``R''. Further investigation of KOIs vetted ``R'' is a topic worthy of investigation by observers interested in small stars within eclipsing binary systems, but is beyond the scope of this work.


\begin{figure}
    \centering
    % \includegraphics[width=0.7\textwidth]{img/June2019/CDF_Radius_SvsM_6_to_20.eps}
    % \includegraphics[width=0.7\textwidth]{Current_Figures/Section_3/singles_radii_cut_6_subset_gt_ten_days.pdf}
    % \includegraphics[width=0.45\textwidth]{Current_Figures/Section_3/singles_radii_cut_2,5_subset_gt_ten_days.pdf}
    \includegraphics[width=0.48\textwidth]{radius_w_mono_cdf_long_period_singles_gt_3REarth_linear.pdf}
    \includegraphics[width=0.48\textwidth]{radius_w_mono_cdf_long_period_singles_gt_4REarth_linear.pdf}\\
    \includegraphics[width=0.48\textwidth]{radius_w_mono_cdf_long_period_singles_gt_4_5REarth_linear.pdf}
    \includegraphics[width=0.48\textwidth]{radius_w_mono_cdf_long_period_singles_gt_5REarth_linear.pdf}
    \caption{Cumulative distribution function of planetary radii comparing ``large'' planets in multiplanet systems with similarly-sized planets in singles, as well as to planets in singles with the additional constraint $P >$ 10 days; all curves are normalized to unity at 10 R$_{\oplus}$. The four panels, upper left, upper right, lower left and lower right, consider planets larger than 3, 4, 4.5 and 5 R$_\oplus$, respectively. The numbers in square brackets represent planets (plus KOIs rejected solely because they are too large and therefore dispositioned as ``R''  in Table \ref{tab:planetcatalog}) larger than the minimum value represented in the particular panel (no upper size cutoff). This plot uses the same criteria for inclusion as used in Fig.~\ref{fig:radius_multis_singles}.}% We exclude candidates with S/N $<$ 12 or $b + \sigma_+(b) \geq$ 0.95, without changing the multiplicity.  Data taken from \texttt{koiprops\_20230830v2.csv}  }
 % older version included KS test statistic.
    \label{fig:large_radius_multiplicities_split}
\end{figure}

Overall, there appears to be an abundant population of planets with sizes less than 3~R$_\oplus$ (in agreement with \citealt{Kite:2019,Hsu:2019}), 44\% of which are in multis.  \ik found a much less abundant population of giant planets; 29\% within the size range 5~R$_\oplus < R_p < 10$~R$_\oplus$ reside in multis, whereas the population of larger objects is dominated by singles that includes HJs and FPs.  Boundaries are smoothed over by a combination of radius errors and true fuzziness.  Note that the size range from 5~--~10~R$_\oplus$ includes a huge diversity of planets from super-puffs (currently only known in systems with multiple transiting planets because no cool very low-mass/low-density \ik transiting planets have RV mass measurements) to exo-jupiters that are more enriched in heavy elements than the prototype present in our Solar System.  However, all planets in this size range are clearly gas-rich, with H/He abundances by mass of the same order as or larger than that of astrophysical metals. 

%The radius valley near 1.7~R$_\oplus$ and radius cliff around 3~R$_\oplus$ are barely discernible on radius CDF plots, so we include a differential plot showing the relative numbers of planets with $R_p < 5$~R$_\oplus$ in Figure \ref{fig:radius_multis_singlesKDE}.
%{\cite{Berger:2020b} found that the radius distribution of planets around the radius valley differed between young and old planets, and between planets subjected high and low incident fluxes. In Figure~\ref{fig:radius_multis_singlesKDE} we compare the radius distributions of singles and multis.} Multis appear to host a larger ratio of super-earth size planets to sub-neptunes than do singles, although the null hypothesis that the observed distributions are drawn from the same sample cannot be rejected with 95\% confidence according the the Anderson-Darling test, but differences are marginally significant at this level according to the Kolmogorov-Smirnov (KS) test. Also, the radius valley and radius cliff lie at similar locations for singles and planets in multis.

%\begin{figure}
 %   \centering
    % \includegraphics[width=0.75\textwidth]{Current_Figures/Section_3/singles_and_multis_radii_cdf_5.pdf}
    %\includegraphics[width=0.45\textwidth]{Current_Figures/Section_3/singles_and_multis_radii_cdf_4,5.pdf}
%    \includegraphics[width=0.75\textwidth]{Current_Figures/Section_3/singles_and_multis_radii_kde.pdf}
%    \caption{Normalized number of planets as functions of planetary radii for planets that have observed transiting companions (red) and those that do not (blue). Gaussian kernel density estimators (KDEs) of bandwidth 0.1~R$_\oplus$ were averaged over to compute these curves. We normalized the CDFs to unity at $R_p = 5$~R$_{\oplus}$, and the same requirements were used for inclusion as for Fig.~\ref{fig:radius_multis_singles}. }% Data taken from \texttt{koiprops\_20220127.csv}}
%    \label{fig:radius_multis_singlesKDE}
%\end{figure}

\subsection{Period Distribution} \label{sec:periodDist}

We now compare the distributions of the orbital periods of \ik single planet systems, that of planets in two-planet systems, and planets in systems of higher multiplicity.  As in  \S\ref{sec:sizes}, we consider only planets with S/N $> 12$, but nonetheless count objects that are classified as planet candidates yet are rejected from these samples due to S/N below this threshold in determining the multiplicity of a planetary system.  However, here we do not include mono-transit PCs and KOIs with dispositions of R in our sample,  and apart from the portion of our analysis wherein we compare the period distributions of different ranges of planetary sizes, we do not impose any restriction on estimated impact parameters.

Figure \ref{fig:period_multiplicities_split} displays the orbital period distributions of single \ik PCs, that of planets in two-planet systems, and PCs in systems with three or more transiting planets. The vast majority of planets within multis, $\sim 90\%$, have orbital periods between 2 and 100 days, whereas 78\% of singles are in that same period range ($10\%$ of singles have periods less than 2 days and $\sim 12\%$ have periods longer than 100 days).  The difference is even larger restricting to the shortest periods ($P<1.6$ days), at which only $\sim 20\%$ of the \ik PCs have known transiting siblings. A KS test shows highly-significant differences between the distributions of singles and multis ($p$-value $\approx 10^{-6}$) and one comparing two-planet systems with higher multiplicity shows significance with $p$-value $\approx 0.27$. %koiprops20230505 
 These results reinforce the findings of \cite{Lissauer:2014} and  \cite{Rowe:2014}. Geometric factors reduce the probability of longer-period planetary siblings of a transiting planet also to be transiting when viewed from the Solar System. The tendency of hot Jupiters to lack nearby companions (Fig.~\ref{fig:perrad_multimap}) contributes to the smallness of the fraction of short-period planets residing in multis; larger period ratios for neighboring planets with short periods (perhaps due to tidal decay of the orbits of very short-period planets), and tides driving very short-period planets to the host star's equatorial plane also likely contribute. See Appendix B of  \cite{Lissauer:2014} for a more comprehensive discussion of possible additional causes of the paucity of very short period planets in multis. %Figure \ref{fig:perdist} shows a zoom-in of the same curves as in Fig.~\ref{fig:period_multiplicities_split} for $P < 40$ days plus additional curves for all PCs, all multis, and inner, middle and outer planets within multis.

%Kadri TODO: Update KS-Test values
% Include Anderson-Darling test

% 25%, 10%, 5%, 2.5%, 1%, 0.5%, 0.1%.
% M2 vs M3+:  Anderson_ksampResult(statistic=0.7534446680252689, critical_values=array([0.325, 1.226, 1.961, 2.718, 3.752, 4.592, 6.546]), significance_level=0.160882252056311)

\begin{figure}
\centering
%\includegraphics[width=0.75\textwidth]{Current_Figures/Section_4/cdf_period_lin_log.pdf}
\includegraphics[width=0.48\textwidth]{period_cdf_singles_vs_multi_subsets_linear.pdf}
    \includegraphics[width=0.48\textwidth]{period_cdf_singles_vs_multi_subsets_log.pdf}
\caption{The value of the vertical coordinate gives the fraction of planets within systems of specified multiplicity  having orbital periods less than the value of the horizontal coordinate. %The numbers listed within the square brackets to the curves in the lower right are for the entire population analyzed, even though the diagram on the left with linear horizontal axis scale only plots the portion of the sample with orbital period $P < 80$~days. 
Although multis are deficient in planets relative to singles at both the shortest and longest orbital periods, the ensembles of planets in multis and those in singles both have median periods of $\sim$10 days. This plot uses the same criteria for inclusion as stated in the first paragraph of \S\ref{sec:sizes}. %We performed a KS-test comparing the distribution of periods in systems with two transiting planets against systems with higher multiplicity. Singles vs Multis:  KstestResult(statistic=0.11757286007806318, pvalue=4.842792833414933e-13)planets in two-planet systems vs.~planets in higher multiplicity systems:  KstestResult(statistic=0.04402684050545795, pvalue=0.34334536095450197). It looks like there are more long-period ($P>50$ days) 2's than 3+'s. 
}%Data taken from \texttt{koiprops\_20230830v2.csv}}
\label{fig:period_multiplicities_split}
\end{figure}

The normalized distribution of orbital periods of planets with exactly one companion PC is very similar to that of planets in higher multiplicity systems for periods up to $\sim 20$~days (Fig.~\ref{fig:period_multiplicities_split}). The  fraction of planets with two or more companions having periods in the range 20~days $< P < 80$~days is larger than that of planets possessing one transiting sibling, with a compensating deficit of PCs with multiple companions for $P > 80$~days, but despite this visual divergence, the overall differences between the two period distributions are not statistically significant (neither the KS test nor the AD test allows us to reject the hypothesis that the curves were drawn from the same distribution at the 95\% confidence level). Nonetheless, since the 2-planet systems are intermediate between single planets and high-multiplicity systems, there may well be real differences that are obscured by the small number statistics. \cite{He:2021} compute conditional occurrence rates of an additional putative planet as a function of both the period and radius of the detected and putative planets.

Figure \ref{fig:cdf_log3binradiisplit} shows the period distributions of various subsets of \ik PCs that have been grouped by planet size and system multiplicity.  Of the eight curves in Fig.~\ref{fig:cdf_log3binradiisplit}, seven conform to the following trends: For a given size range, singles have a broader period distribution (more planets at both very short and very long periods) than do multis, with the cumulative fractions crossing ``near'' 10 days.  For all size ranges considered in multis and for non-giant singles, larger planets tend to have longer periods.  The exceptional curve is for giant planets in singles, a larger fraction of which are detected at short periods than is the case for mid-sized planets in singles.  The tendency to observe smaller transiting planets at shorter periods is consistent with observational biases.  Mid-sized planets with $P \lesssim 2$~days are quite uncommon (the hot Neptune desert), consistent with H/He envelopes being stripped from close-in planets with $M_p \lesssim 20$~M$_\oplus$. 

\begin{figure}[!hbt]
    \centering
   % \includegraphics[width=0.45\textwidth]{Current_Figures/Section_3/three_subset_radius_period_(1,8-2,5).pdf}
    %\includegraphics[width=0.45\textwidth]{Current_Figures/Section_3/three_subset_radius_period_(1,8-4).pdf}\\
    \includegraphics[width=0.75\textwidth]{radius_cdf_population_period_with_radius_subsets_log.pdf}
    \caption{Cumulative period distribution of \ik planet candidates, considering the singles (blue curves) and multis (red curves) independently and separating the planets by radius into 4 bins ($R_p < 1.8$~R$_\oplus, \, 1.8$~R$_\oplus \leq R_p < 5$~R$_\oplus,\,5$~R$_\oplus \leq R_p < 10$~R$_\oplus$,  10~R$_\oplus \leq R_p$). Only planets with $b + \sigma_+(b) < 0.95$, S/N~$> 12$ and more than one transit observed are included, as radii of planets with grazing transits are not well-constrained and the false positive rate is relatively high among PCs with low S/N.}% Data taken from \texttt{koiprops\_20230830v2.csv}}
    \label{fig:cdf_log3binradiisplit}
\end{figure}

Figure \ref{fig:period_TTVs} shows the distributions of orbital periods of planets with and without Transit Timing Variations, with the planets also being categorized as being a single or a member of a multi.  Here, we consider planets to have TTVs if our lightcurve fitting prefers solutions with TTVs, and/or they are listed as having TTVs in the \cite{Holczer:2016} catalog, and/or they are classified as having strong or moderate TTVs by \cite{Kane:2019} (8 or 9 overall rating)\footnote{The criteria for being counted as having TTVs in Fig.~\ref{fig:period_TTVs} differ from those used for coloring planets to denote TTVs in the bottom two panels of Fig.~\ref{fig:gallery}. Here, we do not count PCs based on a \emph{tentative} TTV signature identified by \cite{Kane:2019}, but we include PCs with TTVs identified by \cite{Holczer:2016} as well as TTVs being used in our fits (which is the case for all PCs in a multi if we detect TTVs in any of the PCs associated with that target star). The latter criterion includes PCs in multis that don't show TTVs themselves, but we prefer including a small number of planets in multis without TTVs in the distribution of planets with TTVs to the alternative of not testing for TTVs in PCs added to the KOI table in recent years and small number statistics.}.  Note that only 7\% of single PCs with S/N~$> 12$ have TTVs that meet our criteria, whereas 15--20\% of such planets in multis have TTVs\footnote{ Our decision to use of TTVs in fitting is made on a system by system basis. The lower limit quoted here for multis only counts planets with Holczer or Kane TTVs plus one for each \emph{system} for which we used TTVs in the fits that does not have any planets with Holczer or Kane TTVs, whereas the upper limit includes all planets in systems for which our fits used TTVs.}. This result is consistent with \cite{2014Xie}, who found TTV rates that grow with transit multiplicity. For both singles and multis with $P < 200$~days%koiprops_20230505.csv
, we find that TTVs are more likely to be observed in planets having longer periods, with the period differences being larger in singles than in multis.  For self-similar (scale-invariant) planetary system architectures, both TTV amplitudes and near-resonant superperiods increase linearly with orbital period, so the dependence of the fraction of planets with TTVs on orbital period may be explained in whole or in part by observational selection effects (small TTV amplitudes for the shortest period planets and observational baseline small compared to TTV superperiods for long period planets).

\begin{figure}
\centering
\includegraphics[width=0.48\textwidth]{period_cdf_with_ttv_flag_linear.pdf}
\includegraphics[width=0.48\textwidth]{period_cdf_with_ttv_flag_log.pdf}
\caption{The normalized cumulative distributions of orbital periods of single planets with TTVs, singles without TTVs, multis with TTVs and multis lacking TTVs.  Here we consider planets to have TTVs if the first digit in their TTV flag in Table \ref{tab:planetcatalog} is 1 (which somewhat overestimates the number of planets in multis with TTVs because TTVs are used for fitting all planets in a system if any planet in the system is found to exhibit TTVs), the second digit is 1 or 2, and/or the third digit is 8 or 9.  Only PCs with at least three TTs measured and S/N~$> 12$ are counted for these distributions. The plot on the left uses a linear scale in period and is truncated at 80 days, whereas the one on the right has a logarithmic period scale and extends to the longest-period \ik planets displaying at least 3 transits. }%Data taken from \texttt{koiprops\_20230830v2.csv}} there are two KOIs w/ P<0.2 days, 8298 and 8300, both of which are PCs that weren't in previous catalogs. the depth of 8298 is much larger when TTVs are included
\label{fig:period_TTVs}
\end{figure}

We next examine the ``cumulative'' fraction of planets with \emph{inner} companions as a function of planetary orbital period.  The green curves in Figure \ref{fig:neighbors} show ${\cal F}_{\rm inner}(\textrm{Period})$, the fraction of transiting planets with periods less than the value specified on the horizontal axis that have at least one transiting sibling on an interior orbit. This fraction can be calculated from the formula:

\begin{equation}
{\cal F}_{\rm inner}(\textrm{Period}) \equiv \dfrac{N_{\rm inner}^{P<\textrm{Period}}}{N^{P<\textrm{Period}}} ~~,
\label{eqn:innercompanionfraction}
\end{equation}

\noindent where $N^{P<\textrm{Period}}$ gives the total number of PCs with $P < \textrm{Period}$ and $N_{\rm inner}^{P<\textrm{Period}}$ represents the number of such planets having one or more transiting companions with period shorter than its own period.  The resulting distribution (in green) increases at any orbital period where a planet has an inner sibling, and (once its value exceeds zero) decreases where a planet lacking inner companions is added to the count, since the denominator accounts for all PCs, including singles and planets within multis with no inner neighbors. 

\begin{figure}[!hbt]
\centering
\includegraphics[width=0.75\textwidth]{fraction_of_transiting_companions.pdf}
\caption{Cumulative fraction of candidates with inner (outer) companions as function of orbital period, as specified by Equation \ref{eqn:innercompanionfraction} (Equation \ref{eqn:outercompanionfraction}). The solid green curve shows the cumulative fraction of PCs up to the plotted orbital period that have inner planetary companions (shorter-period PCs associated with the same target star). The solid orange curve marks the cumulative fraction of planets with $P$ \emph{larger than} the specified period that have even longer-period transiting companions. The dashed curves show the analogous curves for large planets (PCs with $R_p > 5$~R$_\oplus$) having inner or outer planetary companions, with PCs of all sizes except mono-transits counting as companion planets. Again only PCs with S/N~$> 12$ are included for these distributions, although this requirement isn't enforced for companions.}%Data taken from \texttt{koiprops\_20230830v2.csv}}
\label{fig:neighbors}
\end{figure}

Similarly, we examine the fractions of planets with \emph{outer} companions. The orange curves in Fig.~\ref{fig:neighbors} show ${\cal F}_{\rm outer}(\textrm{Period})$, the fraction of transiting planets with at least one transiting sibling with $P>\textrm{Period}$, which is given by: 

%\begin{subequations}

\begin{equation}
{\cal F}_{\rm outer}(\textrm{Period}) \equiv \dfrac{N_{\rm outer}^{P>\textrm{Period}}}{N^{P>\textrm{Period}}} ~~. 
\label{eqn:outercompanionfraction}
\end{equation}

% FIXME: JASON START HERE

\noindent The terms on the right hand side of Equation (\ref{eqn:outercompanionfraction}) are defined analogously to those in Equation (\ref{eqn:innercompanionfraction}), but note that in this case, planets are added to the distribution starting from the longest period and moving to the shortest.  To zeroth order, the solid orange and green curves appear to be mirror images of one another, albeit a bit stretched out towards longer periods. The symmetry and ``reflection'' near 10 days and somewhat broader shape at long periods all mimic trends visible in the right panel of Fig.~\ref{fig:period_multiplicities_split}.

The solid curves in Fig.~\ref{fig:neighbors} can be compared to subsamples (dashed  curves) that show the fraction of large planets ($R_p > 5$~R$_\oplus$) that have transiting siblings, where siblings of any size are included.  It is well-known that giant transiting planets, especially hot jupiters, rarely have companions that also transit \citep{Latham:2011, Steffen:2012}. The dashed curves in Fig.~\ref{fig:neighbors} quantify the differences, showing (among other things) that large planets are only one-third as likely to have outer companions and two-thirds as likely to have inner companions as are \ik planets overall (note that no effort has been made to attempt to correct these numbers for possible detection biases).  Although large planets overall are substantially less likely to have outer companions, large and small planets with $P \gtrsim 30$ days are equally likely to have one or more outer companions. 

%For each planet with an inner companion of period $P_{i}<P$, we added 1/N(P), where N(P) refers to the number of planets overall with shorter period than P.  We compare these curves to subsamples (dashed  curves) that include large planets only. This curve begins to increase at larger orbital period and highlights that hot jupiters are significantly less likely to have inner companions compared to smaller planets. Complementing these curves in orange, for each planet with an outer companion of period $P' > P$, we added 1/N(P'), where N(P') refers to the number of planets overall with longer period than P. The curve decreases at any orbital period where a planet has an outer neighbor, and increases where any orbital period does not include an outer neighbor, since the denominator includes all planets, including singles and planets within multis with no outer neighbors. The subsample of large planets with outer companions (dashed orange curve) does not plateau at the same value as the large planets with inner companions showing that warms jupiters are more likely to have inner companions than hot jupiters are to have outer companions.


%\begin{figure}[!hbt]
%\centering
%\includegraphics[width=0.75\textwidth]{Current_Figures/Section_4/cdf_period_subsets.pdf}
%\caption{x-axis: P; y-axis: fraction of planet with period close to x value with at least one companion with $5<P<15$ days. For each PC, the 12 planets with closest periods below that of the planet in question, the planet itself and the 12 planets with closest periods above that of the planet in question are tested on this yes/no question and the fraction yes is plotted at the x-value; different colors could be used for different period ranges or to signify multiple planets in the specified period range (which was chosen because it is multi-rich).}
%\label{fig:}
%\end{figure}

%  src.analysis.radius 
%  --------------------
% Total singles: 2559
% singles: b in [0.9, 1) for Rp < 1.8: 102      
% singles: b in [0.9, 1) for 1.8 <= Rp < 2.5:68 https://www.overleaf.com/project/5f2da7e4871a0200014f4a11
% singles: b in 0.9, 1) for 2.5 <= Rp: 117   

% singles: b in [1, 1e+25) for Rp < 1.8: 2      
% singles: b in [1, 1e+25) for 1.8 <= Rp < 2.5:0
% singles: b in 1, 1e+25) for 2.5 <= Rp: 37     

% Total multis: 1778
% multis: b in [0.9, 1) for Rp < 1.8: 67       
% multis: b in [0.9, 1) for 1.8 <= Rp < 2.5:57 
% multis: b in 0.9, 1) for 2.5 <= Rp: 41     

% multis: b in [1, 1e+25) for Rp < 1.8: 4      
% multis: b in [1, 1e+25) for 1.8 <= Rp < 2.5:1
% multis: b in 1, 1e+25) for 2.5 <= Rp: 2   

% Of the 728 singles larger than 8R$_{\oplus}$, 354 have b $>$ 0.95, 402 have b $>$ 0.9. Of the 60 members of multiplanet systems larger than, 8R$_{\oplus}$, 23 have b $>$ 0.95, 25 have b $>$ 0.9. We performed a KS test on the CDFs of the large singles and multis, and found a $p$-value $\approx$ 10$^{-4}$.


\subsection{Eccentricity \& Transit Duration Distributions}\label{sec:ecc}

{The improved estimates of stellar, planetary and transit parameters in our new catalog (Table \ref{tab:planetcatalog}) enable us to} characterize the eccentricity distribution of various subsets of the \ik planet candidates using the distribution of period-normalized transit durations.
{As accurate period estimates are required for this analysis, mono-transits (planets candidates with only a single detected transit) were excluded from the analysis in this section.  Our analysis presented below shows that there is not evidence for changes in the eccentricity distribution of \ik planet candidates as a function of the host star effective temperatures over the range from 4000~K to 6200~K.  Similarly, we see no dependence on period for the eccentricity distribution for planets with orbital periods greater than 6 days.  In contrast, we show that there are significant differences in the eccentricity distribution as function of planet size and the number of planets detected by \ik in a given system.  We find marginal evidence suggestive of a trend for planets with a known companion within a factor of 2.04 in orbital period to have a smaller eccentricity than more widely spaced planets.  Below, we describe the specific comparisons and statistical tests performed to support these findings.}

In order to compare the eccentricity distributions of various subsets of \ik planet candidates, we begin by measuring the transit duration for each PC. 
We calculate the posterior mode of the average of the full transit duration, {$T_{1,4}$}, and the duration of the ``flat bottom'' portion of the transit, {$T_{2,3}$},  normalized by the analogous predicted transit duration for the measured orbital period and host star properties, assuming a circular orbit and central transit ($b=0$). Using the notation introduced in Section \ref{sec:catalog}, the normalized transit duration, $\tau$, is given by:

\begin{equation}
    %\tau \equiv \frac{t_{1.5}}{t_{1.5}(e = b = 0,P,\rho_\star)} = \frac{t_{1,4} + t_{2,3}}{{ t}_{1,4}(e = b = 0,P,\rho_\star) + {t}_{2,3}(e = b = 0,P,\rho_\star)}  .
    \tau \equiv \frac{T_{1.5}}{T_{1.5}(e = b = 0,P,\rho_\star)} = \frac{T_{1,4} + T_{2,3}}{{ T}_{1,4}(e = b = 0,P,\rho_\star) + {T}_{2,3}(e = b = 0,P,\rho_\star)}  .
    \label{eqn:tdur}
\end{equation}  

\noindent For planet candidates with {$b \ge 1-R_p/R_\star$, $T_{2,3}=0$}, and we exclude these cases from our analysis due to difficulty in precisely constraining physical parameters for grazing and near-grazing transits. The particular definition of the normalized transit duration given in Equation (\ref{eqn:tdur}) was chosen for robustness in measurement as well as its independence (to first-order) of the value of $R_p/R_\star$.  {By defining $\tau$ relative to the duration for $b=0$ (rather than the best estimate of $b$), we avoid a dependence on $b$, which is advantageous because estimates of the value of impact parameters can have significant measurement uncertainties (\S\ref{sec:transitmodel} and Table \ref{tab:planetcatalog}).}

Planets with $\tau > 1$  (when accounting for measurement uncertainties) must transit while the planet-star separation exceeds the semimajor axis, and there can be an interesting constraint on the pericenter direction $\omega$ (e.g., \citealt{DawsonJohnson:2012}).  
In most cases, $\tau$ is comparable to or less than unity and there is only a minimal constraint on the marginal distribution of $\omega$, since the transit duration could be shortened due to either the planet being near pericenter or the impact parameter $b \neq 0$ (or both).  

Star-planet orbital planes are isotropically distributed and thus randomly oriented relative to \ikt's line of sight, hence the \emph{intrinsic} distributions of the impact parameters (within the range $b \lesssim 1$) and the pericenter angles (of all planets, not just transiting ones) are nearly uniform.  
Therefore, the distribution of $\tau$ provides a useful probe of the eccentricity distribution of a population of transiting planets \citep{Ford:2008}.  
In practice, the distribution of observed impact parameters is affected by detection biases, since shorter transits that occur for larger $b$ provide less time in transit to accumulate signal.  
Fortunately, this is a relatively modest effect for most planets, since the changes in transit duration, and hence transit signal, are typically small, and most detected planets have a S/N much greater than required for detection.

Multiple studies have begun to characterize the eccentricity distribution based on the observed period-normalized transit duration ($\tau$) distribution.  \citet{Moorhead:2011}, \cite{Kane:2012}, \cite{Plavchan:2014} and \cite{Xie:2016} focused on using the $\tau$ distribution to characterize the eccentricity distribution (as opposed to modeling the several observed properties at once,  as in, e.g., \cite{Mulders:2018,He:2019,Sandford:2019,He:2020,Zhu:2018,MacDonald:2020,Yang:2020}).  
The strength of conclusions from these early studies was limited due to the uncertainty in the host star density.  While random measurement errors can be easily incorporated into the analysis \citep{Ford:2008}, the potential for systematic error is more concerning.  For example, \citet{Moorhead:2011} showed a trend for the $\tau$ distribution to broaden with increasing host star temperature, which could be attributed to either the eccentricity distribution changing with host star temperature or the errors in host star densities increasing for stars that have had time to evolve far from the zero age main sequence.  Such concerns helped to motivate follow-up campaigns to characterize host star properties using high-resolution spectroscopy \citep{Fulton:2018} and more detailed stellar modeling \citep{Berger:2020a}, both of which  incorporate improved distance measurements from \gaiat.

To assess the effects of potential systematic errors in the host star density on the eccentricity distribution, we focus our analysis on planets whose host star properties are available from either \citet{Fulton:2018} or \citet{Berger:2020a}, both of which represent dramatic improvements of stellar properties from those derived using photometric data (e.g., the KIC).  (Asteroseimic densities are expected to be even more accurate, but are available for only a substantially smaller subset of planet candidates \citep{vanEylen:2015,vanEylen:2019}.)  
While the stellar properties estimated using high-resolution spectroscopy \citep{Fulton:2018} are likely more accurate than those with lower resolution spectroscopy, they are not available for a substantial fraction of the stars in the full sample (e.g., most faint stars with a single known transiting planets, stars whose planets were identified after the CKS survey).  However, the \cite{Fulton:2018} properties are available for a substantial majority of the hosts of systems with multiple transiting planets.

Figure \ref{fig:EccStellarSource} compares the cumulative distribution of normalized transit durations ($\tau$ values) based on using the  host star properties from \citet{Fulton:2018} (CKS) and those from \citet{Berger:2020a}.  The orange and green curves are for an overlap sample for which both sets of stellar parameters are available.  The wide blue curve shows the results using the \citet{Berger:2020a} properties for the full sample.  (The results for all transiting planet candidates for which CKS parameters are available is not plotted because it is visually indistinguishable from the overlap sample using stellar properties from CKS.)  ~The distributions of $\tau$ values for planets using host star properties from \citet{Fulton:2018} show small, but statistically significant, differences from those using the stellar properties of \citet{Berger:2020a}.  Most of this difference arises due to the fact that the CKS sample is significantly less complete than the \citet{Berger:2020a} sample, particularly for targets with a single transiting planet candidate, and the distributions using the overlap sample are statistically indistinguishable.  

\begin{figure}
    \centering
    \includegraphics[scale=.3]{EccStellarSource.pdf}
    \caption{Cumulative distribution of $\tau$, the normalized transit duration (Equation \ref{eqn:tdur}), based on stellar properties from either \citet{Berger:2020a} or \citet[][CKS]{Fulton:2018}.  
    The thick blue curve is the sample used for our primary analysis of transit durations.  
    The orange and green curves are for a smaller sample for which both Berger and CKS stellar parameters are available.  
    The period-normalized transit duration distributions based on the different stellar properties do not differ significantly if we restrict the comparison to the same sample. The difference between the full and overlap samples is primarily due to the CKS sample favoring stellar hosts of multi-planet systems (see Fig.~\ref{fig:EccMulti}) and bright target stars, which implicitly affects both the stellar properties and \ikt's sensitivity to planets around those stars.}
    \label{fig:EccStellarSource}
\end{figure}

In order to minimize the risk of systematic biases, we perform nearly all of our subsequent analyses of the $\tau$ distribution based on stellar properties from \citet{Berger:2020a}, even when parameters from \citet{Fulton:2018} are available. The single exception is for comparing the distributions between multiple planet systems of differing multiplicity, where we make comparisons using each set of stellar parameters (since the CKS sample has much better completeness for hosts of multi-planet systems than for single planet hosts).   We also applied cuts so that our subsequent analysis only includes PCs with a measured orbital period, transit S/N $\geq 12$ (so other transit parameters are well measured), an impact parameter $b < 1- R_p/R_\star$, host star temperature between 4000~K and 6600~K, planet radii estimated to be smaller than 12.5 R$_\oplus$, and reported uncertainty in the stellar density less than 25\%.   

Having defined a sample of 2762 planets for which systematic biases should be minimal, we performed several checks. We verified that there were not significant differences in the $\tau$ distribution if we increased the thresholds for the minimum transit S/N to be included in our primary analysis.  Similarly, we confirmed that using a fixed maximum impact parameter of 0.95 or greater did not affect our conclusions.  

Motivated by \citet{Moorhead:2011}, we divided the sample into four bins based on host star temperature (Figure \ref{fig:EccTeff}) and performed 4-sample Anderson-Darling (AD) test of the null hypothesis that each subsample was drawn from the same distribution.  
(The AD test is usually more powerful than the more common Kolmogorov-Smirnov (KS) test.  The KS test is most sensitive to a shift of the distribution, but significantly less sensitive to differences in the shape of the wings.)  The AD $p$-value was $<3\times10^{-5}$.   The highest temperature bin is clearly the most discrepant from the other three.  If we exclude this bin,  the $p$-value from a 3-sample AD test is 0.016, much less extreme, but still low enough to reject, at the level comparable to $2.4 \sigma$, the null hypothesis that the $\tau$ distribution is the same for the three remaining subsets of host stars with $4000 \le T_{\rm eff} \le 6200$~K.  

\citet{Moorhead:2011} also saw significant differences between planets orbiting stars with temperature above and below the Kraft break ($\approx$~6200~K).
{(Main sequence stars with effective temperature greater than the Kraft break have negligible convective envelopes, resulting in dramatically reduced tidal dissipation in the star compared to cooler main sequence stars.)}
However, uncertainty in stellar parameters led them to focus on stars with $T_{\rm eff} \le 5100$~K, rather than risk confusing changes in the $\tau$ distribution due to the eccentricity distribution with those caused by measurement uncertainties and selection effects. 
Thanks to improved stellar parameters made possible by \gaia distance measurements, we no longer find differences in the $\tau$ distribution of stars cooler than the Kraft break and see a more significant difference between the $\tau$ distribution of planets with stars on either side of the Kraft break.  

These differences are unlikely to be due to uncertainties in stellar properties for stars with $6200 \le T_{\rm eff} \le 6600$~K, since imposing strict cuts on uncertainty in stellar densities has a minimal effect on this subset.
The observed differences could be partially due to the reduced efficiency of tidal dissipation in hotter stars with no significant convective envelope.  
However, we caution that the differences do not go away, even if we focus only on planets with periods greater than 10 days, for which tidal effects during the main sequence would be small.

%The AD test does not make use of the information that the three remaining subsamples are ordered.  Therefore, we visually inspected the cumulative distribution functions (CDFs) of $\tau$ (see Fig.\ \ref{fig:EccTeff}) and found no pattern in the shape of the $\tau$ distribution as the temperature increases (as would be expected if there were a systematic error in stellar temperatures were responsible for the differences).

\begin{figure}
    \centering
    \includegraphics[scale=.3]{EccTeff.pdf}
    \caption{Cumulative distribution of $\tau$, the normalized transit duration, for four subsets of planets based on the effective temperature of their host star reported in \citet{Berger:2020a}.  The subsets are: 4000~--~5200~K (black, dotted), 5200~--~5600~K (blue, solid), 5600~--~6200~K (green, dash-dot), and 6200~--~6600~K (red, dashes).  A 4-sample AD test strongly rejects the null hypothesis that the $\tau$ distributions are drawn from a single population.  Excluding the 6200~--~6600~K bin,  a 3-sample AD test still nominally rejects the null hypothesis that the remaining bins are drawn from a common distribution, but with weak significance.}
    \label{fig:EccTeff}
\end{figure}

Therefore, subsequent analyses of the $\tau$ distribution are focused on planets with host stars with $4000 \le T_{\rm eff} \le 6200$~K,
so as to minimize the risk of systematic bias (e.g., due to small planets being more readily detectable around bright stars). These restrictions leave us with 2485 planets. For the analyses below, we only consider those PCs that passed all of these cuts in determining system multiplicity. However, we do consider PCs that were rejected by these cuts in determining the smallest period ratio to a neighboring planet in \S\ref{sec:eccspacing}. 


%
\subsubsection{Fitting the Eccentricity Distribution}
%
We attempted to fit multiple simple analytic models to the observed $\tau$ distribution.  
For each analytic model, we generate simulated populations of planets assuming a uniform distribution of $\omega\sim U\left[0,2\pi\right)$ and $b\sim U\left[0, b_{\max}\right)$, {where $b_{\max} =  1 - R_p/R_{\star}$ is based on the value reported in Table \ref{tab:planetcatalog} for each planet.}  
We sample the planet-star radius ratio, orbital periods and stellar densities from those of the observed sample and add measurement noise based on the reported uncertainties (parameterized as mixture of two half-Gaussians).  
We weight each simulated planet by its geometric transit probability, which accounts for the dependence on the $e$ and $\omega$ drawn for each planet.  {\citep[However, this does not account for how differences in transit duration affect the detection probability conditional on the planet transiting, as done in][]{He:2019,He:2020}.} 
We find that a Rayleigh distribution of eccentricities that is  truncated to be less than one, which is best fit with Rayleigh parameter of 0.053, is not sufficient to reproduce the observed $\tau$ distribution (see Fig.~\ref{fig:EccModel}, red dashed curve).  
Using a small Rayleigh parameter under-predicts the number of extreme $\tau$ values, while using a large Rayleigh parameter results in too broad a distribution.  
This led us to consider a continuous mixture of Rayleigh distributions, where the weights for each Rayleigh parameter are proportional to a Rayleigh distribution (with Rayleigh parameter 0.043), i.e., a Rayleigh of Rayleighs, as in \S6.1.2 of \cite{Lissauer:2011b}.  
While this results in a slight improvement in the fit for long-duration transits, it did not significantly improve the fit for $\tau$ in the range of 0.8~--~1.0 (see the left panel Fig.~\ref{fig:EccModel}, green dot-dashed curve). %
%This led us to consider a mixture of two Rayleigh distributions, parameterized by two Rayleigh parameters, $\sigma_{\rm lo}$ and $\sigma_{\rm hi}$, and $f_{\rm hi}$, the fraction of planets assigned to the high eccentricity population.  
%Indeed, we find that such a model can reproduce the observed $\tau$ well, e.g., $\sigma_{\rm lo}=0.002$ and $\sigma_{\rm hi}=0.2$, and $f_{\rm hi}=0.1$, as in Fig.\ \ref{fig:EccModel}.

\begin{figure}
    \centering
    \includegraphics[scale=.15]{EccModel.pdf}
    \includegraphics[scale=.15]{EccDistModels.pdf}
    \caption{Left:  Cumulative distributions of $\tau$, the normalized transit duration, for our sample of \ik planets (dotted black curve) and two simulated populations: The eccentricity is drawn from a single Rayleigh distribution with Rayleigh parameter $\sigma = 0.053$ (red dashed curve), and the eccentricity distribution is drawn from a continuous mixture of Rayleigh distributions with weights given by a Rayleigh distribution with $\sigma=0.043$ (green dot-dashed curve).  Neither model reproduces both the rapid rise of the $\tau$ distribution due to low eccentricity planets and the tail of planets with $\tau$ larger than unity seen in the \ik sample.  Right:  The eccentricity distributions implied by either the best fit Rayleigh distribution (green) or Rayleigh of Rayleigh distributions (orange) described above.  While the two models make very similar predictions for the distribution of normalized transit durations, they have substantially different implications for the tail of the eccentricity distribution.}
    \label{fig:EccModel}
\end{figure}

More detailed modeling of the joint {probability density function (PDF)} of multiplicity, inclination and eccentricity distributions suggests that the joint {PDF} is not simply the product of the distribution for each quantity individually \citep{He:2019,He:2020,Yang:2020,Millholland:2017}.
Therefore, we do not attempt to perform a detailed characterization of the uncertainties in such a model.

{While the transit duration distribution can provide a powerful constraint on parameters given an assumed eccentricity distribution, it has much less statistical power for comparing different functional forms for the eccentricity distribution.
This is illustrated by the substantial differences between the two histograms in the right panel of Fig.~\ref{fig:EccModel}, despite the very similar predictions of the two models for the transit durations (long dashed and dot-dashed curves in left panel).}
%similarities in the theoretical (long-dashed) curves in the panel on the right demonstrate the limits in deducing a unique eccentricity distribution from the distribution of normalized transit durations.

\subsubsection{Transit Duration vs.~Multiplicity}\label{sec:eccmultiplicity}
%
Long-term orbital stability and planet formation models suggest that the eccentricity and mutual inclination distributions of planets depend on the multiplicity of their host planetary system \citep{Pu:2015,Gratia:2021,Bartram:2021}.
Indeed, previous studies of the observed transit duration ratio distribution find evidence that the mutual inclination distribution decreases with multiplicity of the inner planetary system \citep{He:2020,Yang:2020}. 
Therefore, in Figure \ref{fig:EccMulti} we compare the $\tau$ distributions for: ({\it i}) systems with a single known transiting planet (``singles''), 
%systems with two or more known transiting planets (``multiples’’), 
({\it ii}) systems with two known transiting planets (``doubles''),  ({\it iii}) systems with three known transiting planets (``triples'') and ({\it iv}) systems with four or more known transiting planets (``high multiplicity'').  

%TODO: Please explain if/how the transit duration vs multiplicity finding goes above and beyond what would be expected from observational biases; i.e. we only see doubles and higher multis in progressively lower mutual inclination systems (=> concentration of b and therefore tau) and presumably more highly eccentric planets wouldn't survive as well with increasing multiplicity.


\begin{figure}
    \centering
    \includegraphics[scale=.3]{EccMulti4.pdf}
    \caption{Comparison of the $\tau$ distribution of planets with $R_p\le 5$~R$_{\oplus}$ in systems with a single known transiting planet (``singles''; solid curve), two known transiting planets (``doubles''; dotted curves) and three planet systems (``triples''; dot-dashed curves) or systems with four or more known transiting planets (``4+'' systems; dashed curves). Light blue curves are based on stellar properties from CKS \citep{Fulton:2018} instead of the \cite{Berger:2020a} catalog (shown in black).  Since CKS parameters are not available for all stars, the planet samples used to compute the blue curves  differ somewhat from those used for the the black curves. No blue curve is shown for singles because CKS only analyzed a small fraction of the fainter stars hosting just one planet candidate. In contrast, the CKS survey devoted extra effort to survey hosts of multi-planet systems \citep{Petigura:2017}, so the differences in the samples are smaller and more random.  Both the KS and AD tests strongly reject the null hypothesis that the distributions for planets in singles and doubles are drawn from the same population.  Similarly,  we reject the null hypothesis that any pair of the $\tau$ distributions for doubles, triples and higher multiplicity systems are drawn from the same distribution.  %The trend for the distribution to get more highly concentrated going from singles to doubles to high multiplicity systems lends further credence to the significance of a difference in the eccentricity distribution between doubles and higher multiplicity systems. 
    }
    \label{fig:EccMulti}
\end{figure}

Both the KS or AD tests strongly reject the null hypotheses that the $\tau$ distributions of planets smaller than $5 R_{\oplus}$ are the same when comparing 
% singles and multiples or 
singles and doubles ($p_{KS}\approx 3.3\times10^{-4}$, $p_{AD}\approx 1.5 \times 10^{-5}$).  Comparing all multiples to singles, both the  KS and AD tests very strongly reject the null hypothesis that the $\tau$ distributions are the same ($p<10^{-14}$), confirming the result that \cite{Xie:2016} obtained for a smaller sample of \ik planets whose host stars had been characterized using spectra obtained by LAMOST. 
When comparing the $\tau$ distributions of planets in doubles to those of planets in higher multiplicity systems, the $p$-values for the KS and AD tests are $\approx 1.3\times10^{-5}$ and $\approx 1.4\times10^{-6}$, respectively.  (If using the CKS stellar properties, then the $p$-value for the KS test decreases to $\approx 2.4\times10^{-6}$, but the $p$-value for the AD test does not change significantly.) The significance is strengthened by the clear pattern of the $\tau$ distribution becoming more highly concentrated near unity as one moves from singles to doubles to higher multiplicity systems. 
When comparing the $\tau$ distributions of planets in triples to those of planets in systems with $\ge 4$ detected planets (using CKS stellar parameters), the $p$-values for the KS and AD tests are $\approx 0.01$ and $\approx 0.009$, respectively.  

Note that the observed transit duration distributions for multiple planet systems are subject to complex observational biases, due to a combination of transit signal-to-noise and geometric transit probabilities that can be correlated within a planetary system.  Properly accounting for these biases requires a full forward model that accounts for the joint distribution of planet sizes, orbital periods, eccentricities and inclinations. For example, \cite{He:2019,He:2020} %(e.g., He et al. 2019, 2020).  \citet{Zhu:2018,He:2020,Yang:2020} 
%Zhu et al. (2018), He et al. (2020) and Yang et al. (2020) 
find that the mutual inclination distribution is narrower for systems with more detected transiting planets.  However, a narrow distribution of mutual inclinations does not imply a narrow distribution of impact parameters.  Even small mutual inclinations can cause significant $\Delta b$ between planets (by construction, since \ik detects planets with $b\lesssim1$).  Further, the $\Delta b$ can have either a positive or negative sign.  Indeed, we have verified that the distribution of maximum likelihood estimate of $b$'s is not more narrowly peaked for widely-spaced multiple planet systems than for closely spaced multi-planet systems (nor that of systems with only a single detected planet).  Therefore, we conclude that the difference in normalized transit duration distribution shown in Fig. 20 is primarily due to differences in the eccentricity distributions between systems with one, two or more detected transiting planets.  The strength of these differences is consistent with the change in eccentricity distribution as a function of number of detected transiting planets resulting primarily from the constraints imposed by long-term orbital stability \citep{He:2020}. %(He et al. 2020).  

\subsubsection{Transit Duration vs.~Orbital Period}\label{sec:eccperiod}
%
We next compare the $\tau$ distribution for planets as a function of orbital period. Dividing the sample at 11.8~days (near the median orbital period of our sample), the $p$-values for the KS and AD tests are $\approx~0.092$ and 0.045, respectively.  
Figure \ref{fig:EccPeriod} partitions the sample into five subsets with boundaries at 2, 6, 12 and 24 days, for which the 5-sample AD test $p$-value is 0.15.  
The biggest difference across subsets is that two shortest period subsets have the sharpest $\tau$ distributions (implying lower eccentricities).
If we combine planets with period in 0~--~2~days and 2~--~6~days, then the $p$-value for a 4-sample AD test is 0.08.  
If we perform a 2-sample AD test for planets with period in 0~--~6~days and planets with period in 6~--~1200, then the $p$-value for an AD test is 0.012. 
It is tempting to attribute the differences in transit duration differences as primarily due to the increased efficiency of orbital circularization for small orbital periods.
However, we caution that the size distribution of the planets with $P < 6$ days is weighted towards significantly smaller values than the size distribution of planets with larger periods, since  \ik has greater sensitivity for detecting planets at shorter orbital periods.  

%While the difference in the remaining subsets are small, it is notable that the longest period bin has the largest tails to large values of $\tau$ (implying a larger fraction of planets with high eccentricity).

\begin{figure}
    \centering
   \includegraphics[scale=.3]{EccPeriod.pdf}
    \caption{Comparison of the cumulative $\tau$ distribution for planets divided into five subsets based on their orbital period.  The curves correspond to  $P < 2$~days (solid light blue), $2 < P < 6$~days (dashed royal blue), $6 < P < 12$~days (dotted black), $12 < P < 24$~days (single dot-dashed green curve)), and $24 < P < 1200$~days (double dot-dashed red).  A 5-sample AD test fails to reject the null hypothesis that the $\tau$ distribution for all subsets is drawn from a common distribution.  However, if we compare planets with $P<6$~days to planets with $24 < P < 1200$~days, then a 2-sample AD test rejects the null hypothesis that the $\tau$ distribution for all subsets is drawn from a common distribution with a $p$-value of 0.01.  
    This is consistent with expectations if tidal effects are effective at circularizing a significant fraction of planets with $P<6$~days.  }
    \label{fig:EccPeriod}
\end{figure}


\subsubsection{Transit Duration vs.~Planet Size}\label{sec:eccsize}
%
Splitting the cumulative $\tau$ distributions for planets by planet radius  at 2.16~R$_\oplus$ (near the median planet size of our sample), the $p$-value for the KS test is 0.15, and the $p$-value for the AD test is 0.07. Thus, neither of these tests finds statistically-significant differences between the two samples.

For the left panel of Figure \ref{fig:EccRadius}, we  divide the population into subsets based on theoretically-motivated size bins with boundaries 0.5, 1.0, 1.6, 1.8, 2.7, 5.0, 9.0 and 12.5~R$_\oplus$. Visual inspection shows similarity between the distributions of the two smallest size bins; the {three} middle size bins also appear to have similar size distributions to one another, as do the distributions for two largest size bins. Furthermore, statistical tests do not find any significant differences among the distributions within any of these three subsets of planet size bins. 


\begin{figure}
\centering
%\includegraphics[width=0.75\textwidth]{Current_Figures/Section_4/cdf_period_lin_log.pdf}
\includegraphics[width=0.48\textwidth]{EccRadius7.pdf}
    \includegraphics[width=0.48\textwidth] {EccRadius3.pdf}
    \caption{Comparison of the normalized cumulative $\tau$ distributions for planets grouped by size. Left panel: Subsets are:  0.5~R$_\oplus < R_p \le 1.0$~R$_\oplus$ (dotted light blue),
    1.0~R$_\oplus < R_p \le 1.6$~R$_\oplus$ (solid dark blue), 
    1.6~R$_\oplus < R_p \le 1.8$~R$_\oplus$ (dotted light green), 
    1.8~R$_\oplus < R_p \le 2.7$~R$_\oplus$ (dashed dark green), 
    2.7~R$_\oplus < R_p \le 5.0$~R$_\oplus$ (dotted light red), 5~R$_\oplus < R_p \le 9$~R$_\oplus$ (dash-dotted dark red), 9~R$_\oplus < R_p \le 12.5$~R$_\oplus$ (dash-double-dotted orange).  A 7-sample AD test strongly rejects the null hypothesis that each subset is drawn from a common distribution ($p$-value $\approx~4\times 10^{-11}$).  The two smallest radius curves are among the three subsets with the most highly concentrated $\tau$ distribution, whereas the two largest radius curves are the subsets with the largest tails.  Both of these would be expected if larger (and thus more massive) planets have typically experienced more significant dynamical excitation (e.g., planet-planet scattering followed by secular evolution).   Right panel: Subsets are: 0.5~R$_\oplus < R_p \le 1.6$~R$_\oplus$ (solid blue), 1.6~R$_\oplus < R_p \le 5$~R$_\oplus$ (dotted green), and 5~R$_\oplus < R_p \le 12.5$~R$_\oplus$ (dashed red).  A 3-sample AD test strongly rejects the null hypothesis that each subset is drawn from a common distribution ($p$-value $\sim 10^{-11}$). }
    \label{fig:EccRadius}
\end{figure}

Thus, we combine planets into three size bins,  0.5~--~1.6~R$_\oplus$, 1.6~--~5~R$_\oplus$, and 5~--~12.5~R$_\oplus$, in the panel on the right of Fig.~\ref{fig:EccRadius}. The hypothesis that all three of these bins have the same underlying distributions can be strongly rejected, with $p$-value $\sim10^{-11}$ (using 3-sampled AD test), and
comparing any pair of these three results in a $p$-value of $<10^{-3}$ (KS tests) or $<10^{-4}$ (AD tests), showing highly significant evidence for differences in the eccentricity distributions between these three broad size bins. 
Further supporting this interpretation, planets with $R_p\le 1.6$~R$_\oplus$ have a particularly concentrated $\tau$ distribution (i.e., small eccentricities), while that for planets with $R_p > 5$~R$_\oplus$ has a relatively large tail (i.e., larger eccentricities), with planets 1.6~R$_\oplus < R_p \le 5$~R$_\oplus$ having an intermediate distribution of  $\tau$ values.

These differences are consistent with the observation that planet sizes (and likely masses) are correlated within a planetary system \citep{Ciardi:2013,Weiss:2018,Millholland:2017,He:2020}, together with the theoretical idea that more massive planets excite larger eccentricities in neighboring planets via planet-planet scattering and/or secular perturbations \citep{FordRasio:2008,Johansen:2012,Pu:2015,Laskar:2000}.  However, they run counter to (although do not necessarily conflict with) the expectation that within a given planetary system, equipartition of angular momentum deficit would typically lead to more massive planets having smaller eccentricities \citep{Lissauer:1995}. To contrast the eccentricities of large and small planets within the same system, we looked at all systems with two or more planets with $6 < P < 1200$ days (the lower limit being chosen to minimize the effects of tidal damping, see Fig.~\ref{fig:EccPeriod}) that meet our criteria for analysis in this section and found that in 136 out of 273 cases (50\%) the largest such planet has a larger value of $\tau$ than the smallest planet.  This implies that there is not a strong preference for planets  within a given inner planetary system to have different eccentricity distributions (after one removes planets potentially affected by tidal circularization).
%
These conclusions are not affected by restricting the sample to planets with orbital periods greater than 8 days (rather than 6 days).

%\bigskip
%\blue{
%Plan for section:\\
%JL:  Will cleanup text and add obvious reference \\
%EF:  Then second round of clean and referencse \\
%EF: I think we're done here.
%}

\subsubsection{Transit Duration vs.~Spacing Between Orbits}\label{sec:eccspacing}

We divided planets in multiple planet systems (excluding the split multis KOI-284 and KOI-2248; see \S\ref{sec:falsemultis}) into two or four nearly equal-size subsets based upon their period ratios with their nearest detected neighbor in $\log P$.  
The period ratio boundaries are the first three quartiles of the period ratio distribution, 1.65, 2.06 and 2.94. % and 49.13.  % 100% quantile=max
Very closely spaced planet pairs typically need to have small eccentricities to avoid close encounters (with potential for exceptions related to resonances).  
As expected, the period-normalized transit duration distribution is more sharply peaked for planets with a nearby neighbor.  However, there was not a statistically significant difference in the $\tau$ distribution between these subsets of planets based on performing a 2 or 4 sample AD test using our nominal sample of planets.  
However, if, as shown in Figure \ref{fig:EccSpacing}, we exclude planets with orbital periods less than 8 days (intentionally larger than 6 days to be confident that we exclude all planets which are likely to be significantly affected by tidal circularization), then {the 2-sample} AD tests rejects the null hypothesis that the observed normalized transit duration distributions for the different subsets of planets (based on period ratio to nearest detected neighbor) are drawn from the same underlying distribution with a $p$-value of 0.037.  
% A 4-sample AD test results in a $p$-value of 0.054.

\begin{figure}
    \centering
    \includegraphics[scale=.3]{EccSpacing.pdf}
    \caption{Comparison of cumulative $\tau$ distributions for planets in multi-planet systems with orbital periods greater than 8 days.  The two curves show results for subsets chosen based on the orbital period ratio.  Here, the closely-spaced subset of planets is defined as those having a detected companion planet (which could have $P<8$ days) with period within a factor of 2.06, a value selected so that the each curve represents nearly the same number of planets.   The widely-spaced subset is the complement of the closely-spaced planets.  A 2-sample AD test rejects the null hypothesis that the $\tau$ distributions for the closely and widely spaced planets are drawn from a common distribution ($p$-value $=0.037$).    }
    \label{fig:EccSpacing}
\end{figure}


We also compared the duration distributions of planets in multiple planet systems near first-order MMRs with those of other planets in multiple planet systems and did not find statistically-significant differences. 
However, we note that the number of planets near resonances is small (214), so nontrivial differences in the distributions may become evident when larger samples become available for study (e.g., combining \ikt, {\it K2}, {\it TESS} and {\it PLATO} data).  

%Close v. far from nearest known neighbor (far could be more e because spacing or i correlated w/e)  
%EF: Is this still on to do list?
%JL: Divide period ratios into 4 equal size subsets. Compare 3rd to 4th to see if difference 
%EF: Not statistically significant. By eye you might guess there's a difference between the closer half and the wider half, but even that is not statistically significant according to AD.  However, the test isn't quite right because one planet's duration can appear in two period ratios (inner and outer planet pairs)  
%EF: text added\\

%Near res v others w/Prat$<2.5$  
%EF: Is this still on to do list?\\
%JL: Yes.  I suggest |zeta|$<0.2$, where zeta1 or zeta2 from Arch 1 is used as appropriate and the choice of 0.2 is a compromise between equalizing sample sizes and not wanting to go very far from resonance, motivated by Fig. 5 of Arch 2\\
%EF: See EccReson.pdf.   Differences not statistically significant.  Also tried increasing to 0.25 or 0.3.  Also tried using only zeta1.  Samples are small (162,630). Increasing to 0.3 makes it (260,532).  Want me to add text for it to the paper?
%}

%may want to restrict the e study to planets w/ $b<.8$ to get better e estimates from durations. EF: Looks like we don't need to be nearly so paranoid. 
%Just use standard t14 durations, or average of t14 and t23? The latter eliminates dependence on Rp/R*. EF: The latter}

\newpage
\section{Long-term Average Planetary Orbital Periods}\label{sec:periodtheory}

The fractional uncertainties quoted for the periods of the vast majority of planets listed in Table \ref{tab:planetcatalog} are $<10^{-5}$, with $\sim 10^{-6}$ being typical (corresponding to 2 minutes per 4 years). These values represent the formal uncertainties in the best fit, constant-period ephemeris computed using the measured midpoints of transits and adjusted for the motion of the spacecraft relative to the rest frame of the barycenter of our Solar System (\S\ref{sec:periods}).  However, as discussed below, the actual mean orbital periods of the planets can differ from the values given in Table \ref{tab:planetcatalog} by many times the listed uncertainties. For studies of the architectures and dynamics of planetary systems, the mean orbital period, $\Bar{P}$, is generally far more important than the mean time interval between transits measured directly from \ik data, $P$. 

Typical radial velocities of \ik target stars relative to the barycenter of the Solar System are of order $10^{-4}$ times the speed of light, so the actual periods of the planets in the rest frame of their planetary system differ from the measured orbital periods by that fractional amount due time dilation.  This small error in tabulated orbital periods is not important to understanding the dynamics of an exoplanetary system because relativistic effects within these systems are small and the periods of all planets in a given system are altered by the same factor, so period ratios remain unchanged.  Moreover, the radial velocities of these planetary systems do not vary substantially, so ephemerides are also not significantly affected\footnote{If the planet's host star is a member of a binary star system, then its radial velocity relative to the Solar System can vary by a nontrivial amount. Consider a binary of two 1 M$_\odot$ stars on a circular orbit with semimajor axis of 100 AU that is viewed edge-on. The orbital period is 707 years, and when the stars are near one of the eclipses, their relative radial velocity observed from our Solar System is changing at a rate of $\sim~18$ m/s/yr $\approx 6.1~\times 10^{-8} c/$yr, where $c$ represents the speed of light in vacuo. Thus, ignoring other factors, the ratio of periods of planets around one of these stars to those around the other should be changing by $\sim 6\times 10^{-8}$/year, which for a precision of 1 part in $10^6$ would be detectable from two sets of observations taken a few decades apart.}.

In contrast to time dilation caused by uniform stellar motion relative to our Solar System,  TTVs may produce errors in estimated planetary orbital periods that must be accounted for in some dynamical investigations and also when making ephemeris predictions. %DR: While we're discussing small effects, we can include other non-Keplerian effects like k2star, k2planet, GR. Of course, unknown planets are potentially a huge effect, especially if you consider planets past 1 AU. 
Periodic TTVs with timescales that are short compared to the interval of \ik observations largely average out and do not produce significant errors in our estimates of orbital periods.  However, when the TTV timescale is long compared to the \ik observations, the period estimated from a hypothetical future set of transit time observations of similar quality and duration to the \ik data could differ significantly.  For example, using the integrations of planets in the Kepler-80 (KOI-500) system performed by \citet{MacDonald:2016}, we see that the observed orbital period changes by as much as a few~$\times 10^{-4}$ days -- several times larger than the reported uncertainties -- when averaged over eight years of data instead of four. The cause of this discrepancy is that the \ik observations cover less than half of a TTV cycle (Fig.~\ref{fig:Kep80}), and therefore the observed orbital period is not the same as the long-term mean period. Note that in this case, some of the planets' periods were  observed near the turning points in their evolution, far from the mean, as should often be the case.  This is compounded by other subtle issues like uncertain apsidal/nodal precession and differences between the measured anomalistic period and the true orbital period. 

\cite{Holczer:2016} produced a catalog of \ik planets displaying periodic TTVs with timescales that are comparable to or shorter than the interval of \ik observations.  Their Table 6 lists estimated long-term mean orbital periods for 199 planet candidates based upon fitting a period plus a sinusoidal modulation to observed transit times.  
Figure \ref{fig:HolczerPeriods} compares our estimates of orbital periods to those in their Table 6. Inspection of these plots shows that the differences between \citet{Holczer:2016}'s sinusoidal-fit estimates of planetary periods and the ones that we list in Table \ref{tab:planetcatalog} are positively correlated with the uncertainties listed in our table.  In most cases the difference between the estimates given in the two papers is less than our tabulated uncertainties, although the difference greatly exceeds our error estimates for some planets. The uncertainties given in \citet{Holczer:2016}'s Table 6 are typically much smaller than ours; however, since the period uncertainties that they obtained using constant-period fits (their Table 2) are even smaller for most planets, we caution the reader against overinterpreting their quoted uncertainties.  

\begin{figure}[!hbt]
 \includegraphics [height = 2.6 in]{HolczerPeriods1.pdf}
 \hspace{-0.6 in}
 \includegraphics [height = 2.6 in]{HolczerPeriods2.pdf}
  \hspace{-0.6 in}
 \includegraphics [height = 2.6 in]{HolczerPeriods3.pdf}
  \hspace{-0.6 in}
 \includegraphics [height = 2.6 in]{HolczerPeriods4.pdf}
 \caption{Comparison between the orbital periods listed in (our) Table \ref{tab:planetcatalog} ($P$, denoted $P_{\rm const}$ here) and those presented in Table  6 of \citet{Holczer:2016}, $P_{\rm sin}$.  The horizontal coordinate represents the fractional uncertainty given in Table \ref{tab:planetcatalog}, the length of the bars shows \citet{Holczer:2016}'s fractional uncertainty, and the vertical coordinate represents the fractional differences between our periods and those of \citet{Holczer:2016}. Open circles represent planets in multis; black filled circles represent single planets. The plot on the upper left shows the largest scale, and successive plots (upper right, lower left and lower right) zoom in by a factor of five relative to the previous plot. The green diagonal lines divide planets whose period estimates differ between the two tables by more than their uncertainties in our table (above and to the left) from those whose period differences are less than our uncertainties (below and to the right). One planet, the single KOI-1209.01, which has an orbital period of about nine months, lies outside the range of the plots, with fractional period difference of $3.9 \times 10^{-3}$, far larger than either the fractional uncertainty in Table \ref{tab:planetcatalog} ($3.6 \times 10^{-6}$) or that in  \citet{Holczer:2016}'s table ($ 5.2\times 10^{-6}$); its location is pointed to by the arrow at the upper left of the largest-scale plot. Colored points indicate planets in the dynamically ``solved'' systems of interest explored in this section:
 Kepler-11 (orange), Kepler-36 (cyan), Kepler-60 (purple), Kepler-80 (blue), Kepler-223 (magenta) and Kepler-419 (olive). % KOI-738/Kep29 is not in Holczer corrected periods table.
 \label{fig:HolczerPeriods}} 
 \end{figure}

 \subsection{Apse Precession and TTVs}\label{sec:precession}

Precession causes the time interval between successive periapse passages to differ from the time interval required to travel 360$^\circ$ in azimuth (which is the reason that anomalistic periods differ from orbital periods).  This non-Keplerian behavior leads to eccentric planets spending a little less time at some radial distances between successive transits, resulting in variations in time intervals between successive transits that average out only over timescales much longer than that of \ik observations \citep{Agol:2005}. 


The eccentricity of a planet near resonance can be viewed as a superposition of its free and forced eccentricity vectors in the ($e \sin \omega$, $e \cos \omega$) plane.  The precession of the forced eccentricity is relatively rapid, and accounts for some of the TTVs that are found among \ik planets; its effects are partially accounted for in period estimates.  The precession of the free eccentricity is much slower.  It is another source of TTVs, which we refer to as secular TTVs.  However, if $ \Delta \varpi T_{\rm obs}/P \ll 1$, i.e., precession is $\ll 2\pi$ during the \ikt's prime mission, then the secular TTVs won't be recognized  and thus can't be removed from/accounted for in calculations of the planet's period. 
%This hypothesis is consistent with the observation that $\widetilde{f}_{\textrm{3-body}}$ for KOI-730 are much higher than for 500, since 730's planetary eccentricities are substantially larger.  
Secular precession of the planets' free eccentricities is unrelated to resonances and generally has a period much longer than the baseline of the \ik observations, so it is not accounted for in our estimates of mean periods and uncertainties thereof (nor was it in previous \ik planet catalogs). 
Consider a planet with eccentricity $e \ll 1$ whose longitude of periapse precesses by $\Delta \varpi \ll 1$ radians per orbit and which transits near periapse.  During the interval of time between successive periapses passages, the planet moves through $2\pi+\Delta \varpi$ radians.  As the planet completes one radial oscillation during this interval, its average angular velocity is equal to the long-term average value.  However, from Kepler's second law, we know that the planet is sweeping out angle at a rate $1+e$ times as fast near periapse.  Thus, the (measured) time interval between successive transits, $P$, is related to the mean orbital period $\Bar{P}$ and the difference between the longitude of periapse and that of the transit mid-point by:

%\begin{equation}\label{eqn:periods}
%\frac{P}{\Bar{P}} = 1 + \frac{e\Delta \varpi}{2\pi}.
%\end{equation}
%\noindent For transits centered $\varpi$ from periapse }, the generalization of Eq.~(\ref{eqn:periods}) is:

\begin{equation}\label{eqn:periodsII}
\frac{P}{\Bar{P}} = 1 - \frac{e\Delta \varpi}{2\pi}\cos{\varpi}.
\end{equation}

\noindent As the cosine term in Equation (\ref{eqn:periodsII}) integrates to zero, this effect averages out over one precession period of the eccentricity%, and has a `coherence time' $(2\pi)^{-1}$ as long as the precession period
. \emph{Nonetheless, the  long-term mean orbital period can differ significantly from the mean period between successive transits during the \ik era even for planets having TTVs that are too small to be observable during the epoch of \ik observations.}  

As planets in \ik multis typically have eccentricities of one to several percent \citep{Xie:2016, Jontof:2016}, $\Bar{P} - P \ll P$, but can nonetheless be much larger than the formal uncertainties in measured orbital periods. The differences between these periods is so small that it can be ignored for our comparisons of the period distributions of various subsets of the \ik sample (\S\ref{sec:periodDist}) and for most studies of period ratios of two planets. However, the (ill-quantified) errors in our estimates of $\Bar{P}$ are important for studies of the distribution of three-body resonances (Lissauer et al., in preparation).    


\subsection{Case Studies of Select Dynamically-Solved Planetary Systems}\label{sec:individual}

% Get Holczer tables electronically here. Choose Table 6 for periods updated with TTV fits.
%https://vizier.u-strasbg.fr/viz-bin/VizieR?-source=J/ApJS/225/9

Transit timing variations have been used for detailed dynamical analyses of a small fraction of the \ik multi-planet systems.  Most of the publications presenting these studies list osculating orbital periods at an epoch near the midpoint of the \ik data.  For several landmark systems, we integrated numerous (in most cases 101) samples of system parameters from the MCMC chains deduced via photodynamical or TTV analyses of the \ik data to compute estimated transit times during and after the \ik epoch.  Table \ref{tab:periods} compares orbital periods (in most cases at an epoch near the midpoint of \ik observations) from the dynamical solutions to long-term average periods of these planets that we computed by integrating these dynamically-solved systems, our standard estimates of the orbital periods obtained via a best constant-period fit to observed transit times (Table \ref{tab:planetcatalog}), and  (where available) to the average  orbital periods estimated in the sinusoidal fits of \citet{Holczer:2016}. The values of ``$P$ at epoch'' in Table \ref{tab:periods} for planets of Kepler-29 and Kepler-60 represent osculating orbital periods at the epochs listed in the subsections below, following our dynamical fits to the transit times reported by \citet{Rowe:2015b}.

\begin{table}[]
    \centering
    \begin{tabular}{|c|c||c|c|c|c|}
        \multicolumn{6}{c}{\textbf{COMPARISON OF ESTIMATES OF ORBITAL PERIODS OF SELECT PLANETS}}\\
        \hline
        KOI & Kepler- & $P$ [Table \ref{tab:planetcatalog}] & $P$ [\citealt{Holczer:2016}] & $P$ at epoch & $\Bar{P}$\\
        \hline
        157.06 & 11b & 10.304031~$\pm$ 0.000026 & -- & $10.30260\pm0.00027$ & $10.30391\pm0.00004$\\
        157.01 & 11c & 13.024917~$\pm$~0.000018 & 13.0249115 $\pm$ 0.0000014 & $13.02555\pm0.00018$ & $13.02507\pm0.00004$\\
         157.02 & 11d & 22.687141~$\pm$~0.000037 & -- & $22.68546\pm0.00037$&$22.68708\pm0.00003$ \\
         157.03 & 11e & 31.995517 $\pm$ 0.000067 & 31.9954254 $\pm$ 0.0000009 &  $31.99834\pm0.00052$ & $31.99555\pm0.00004$  \\
        157.04 & 11f & 46.68587 $\pm$ 0.00027 & 46.6857474 $\pm$ 0.0000036 &  $46.6933\pm0.0018$ & $46.6855\pm0.0005$ \\
         157.05 & 11g & 118.37857 $\pm$ 0.00025 & -- &  $118.38089\pm0.00057 $ & $118.3782\pm0.0005$  \\
         277.02 & 36b & 13.84899 $\pm$ 0.00034 & 13.848692 $\pm$ 0.000006 & $13.849194\pm0.00004$ & 
         $13.848063\pm0.0002$
         \\
         277.01 & 36c & 16.231949 $\pm$ 0.00026 & 16.232080 $\pm$ 0.000001 & $16.231774\pm0.00002$ &
         $16.232628\pm0.0002$
         \\
         500.05 & 80f & 0.9867860 $\pm$ 0.0000013 & -- & $ 0.9867873 \pm 0.0000044 $  & $ 0.9867862  \pm  0.0000012 $ \\
         500.03 & 80d & 3.0721523 $\pm$ 0.0000058 & -- &  $ 3.07253 \pm 0.00029 $ & $ 3.0721293  \pm  0.0000086 $ \\
         500.04 & 80e & 4.6453934 $\pm$ 0.0000084 & -- &  $ 4.64474 \pm 0.00022 $ & $ 4.645410  \pm  0.000014 $  \\
         500.01 & 80b & 7.0535287 $\pm$ 0.0000094 & 7.0535152 $\pm$ 0.0000013&  $ 7.05357 \pm 0.00036 $ & $ 7.053570 \pm 0.000025 $   \\
         500.02 & 80c & 9.521646 $\pm$ 0.000014 & 9.5216221 $\pm$ 0.0000013 &  $ 9.52330 \pm 0.00030 $ & $ 9.521525  \pm  0.000047 $  \\
         500.06 & 80g & 14.64538 $\pm$ 0.00011 & -- &  $ 14.6503 \pm 0.0018 $ & $ 14.6457  \pm  0.0013 $  \\
         
        730.04 & 223b &  7.384456 $\pm$ 0.000072 & -- &  7.38449 $\pm$ 0.00022& $ 7.3845\pm0.0002$ \\
        730.02 & 223c & 9.848211 $\pm$ 0.000082 & -- &  $9.84564\pm0.00052$ & $9.84934\pm0.00014$  \\
        730.01 & 223d  & 14.78701 $\pm$ 0.00018 & 14.7869296 $\pm$ 0.0000095 & $14.78869\pm0.00029$  & $14.7841\pm0.0002$ \\
        730.03 & 223e & 19.72434 $\pm$ 0.00033 & 19.725722 $\pm$ 0.000018 &  $19.72567\pm0.00055$ & $19.7256\pm0.0007$  \\
        738.01 & 29b  & 10.339236 $\pm$ 0.000056 & -- & 10.33842 $\pm$ 0.00029  & $10.336927\pm 0.000025$  \\
        738.02 & 29c & 13.28712 $\pm$ 0.00011 & -- & 13.28841 $\pm$ 0.00053 & $13.290961\pm 0.000037$ \\
        1474.01 & 419b & 69.7262 $\pm$ 0.0012 & 69.7281819 $\pm$ 0.0000004 &  $69.7960\pm$ 0.0042 &    $69.7869 \pm 0.0454 $  \\ 
        2086.01 & 60b & 7.132950 $\pm$ 0.000041 & -- & 7.13335 $\pm$ 0.00013 & 7.1325157 $\pm$ 0.000025 \\ 
        2086.02 & 60c & 8.918977 $\pm$ 0.000041 & 8.91867 $\pm$ 0.00020 &  8.91866 $\pm$ 0.00018 & 8.919029 $\pm$ 0.000004 \\
        2086.03 & 60d & 11.89825 $\pm$ 0.00010 & -- &  11.89810 $\pm$ 0.00020 & 11.899566 $\pm$ 0.000070 \\
        \hline
    \end{tabular} 
 \caption{Orbital periods, in days, of selected well-studied planets estimated using various different methods. From left to right, the columns give the values presented in Table \ref{tab:planetcatalog} of this work, Table 6 of \cite{Holczer:2016}, period at epoch (typically near the mid-time of \ik observations) from dynamical fits to TTs, and long-term (averaged over the same intervals used for ordering the samples to select representative systems shown in Figures \ref{fig:Kep11}~--~\ref{fig:Kep60Periods};  $10^4$ years for Kepler-419~b, $10^6$ days~$\approx 2738$ years for Kepler-11's planets, 100 years for Kepler-80, 1000 years for planets in other systems) average periods computed by integrating the systems using samples of the initial conditions at epoch obtained from dynamical studies.}
    \label{tab:periods}
\end{table}



Period variations for one or more of the planets in each of these dynamically active multi-planet systems are shown in Figs.~\ref{fig:Kep11}~--~\ref{fig:Kep60Periods}.  Each panel within these figures shows three samples, which we selected by ordering the solutions by the  long-term average period of the first KOI found in the system (KOI number ending in .01), then selecting  the median member of the list (usually the sample with the 51$^{\rm st}$ longest value, in black), one with that planet's period roughly one standard deviation shorter than the mean (17$^{\rm th}$ sample, in red), and one with this planet's mean period one standard deviation longer than the mean (85$^{\rm th}$ sample, in blue).

All of the systems that we analyze in this subsection have strong interactions among all or most of the known planets, leading to substantial TTVs.  Three of these systems, Kepler-11, Kepler-36 and Kepler-419, don't have any planets known to be librating within orbital resonances.  One system, Kepler-29, has two planets that are locked in a 9:7 (second-order) MMR.  The other three systems considered below, Kepler-60, Kepler-80 and Kepler-223, each have three or more planets locked into a resonant chain that includes (zeroth-order) three-body resonances and probably first-order two-body mean-motion resonances (in some cases, it isn't known whether or not the two-body resonance variables are librating).%  The publications presenting these studies generally list osculating orbital periods at an epoch near the midpoint of the \ik data. 

To estimate long-term periods of all known transiting planets in the abovementioned seven \ik planetary systems, we ran simulations with an Embedded Runge-Kutta Prince-Dormand integrator (\citealt{PD:1981}; gsl\_odeiv2\_step\_rk8pd within the GNU Scientific Library, \citealt{Gough:2009}). From the transit times (more precisely, the times near inferior conjunction when the distance between the center of the planet and that of the star projected onto the plane of the sky reaches a minimum, as we typically do not have sufficient information on the impact parameter to know whether or not a transit actually occurs) simulated over the specified interval (typically 1000 years) beginning with the start of \ik observations, we determine the best fitting linear ephemeris. The transit periods thus derived are given in the final column of Table~\ref{tab:periods}.   Due to numerical dissipation, our non-symplectic code causes the orbital periods to decrease slightly; in a run for 1 million days with only the 10-day planet Kepler-11~b, $P_b$ decreased $-1.16\times10^{-6}$~day.  This effect on simulations with all 6 planets of Kepler-11, $P_b$ averages a loss of $-5.8\times10^{-7}$~day, which is a systematic bias, but is dwarfed by the statistical uncertainty of $\sim4\times10^{-5}$~day for that system. (We would advocate using a symplectic algorithm for integrations longer than those reported here.)

We present results of our modeling in the rightmost columns of Table \ref{tab:periods} and in Figs.~\ref{fig:Kep11}~--~\ref{fig:Kep60Periods}. Note the differences in mean period and period at epoch of the various solutions that fit the observed data, as well as differences of periods averaged over four years when taken starting at differing times for three representative solutions for each of the planets.

\subsubsection{Kepler-11 = KOI-157}

Six transiting planets are known to orbit Kepler-11, all larger and more massive than the Earth but less massive than Uranus. Each of the five inner planets is located near but not in a first-order mean-motion resonance with one or two of its neighbors: The inner pair, b and c, are slightly wide of the 5:4 MMR; the third and fifth planets, d and f, are just wide of the 2:1 resonance, whereas the fourth and fifth planets, e and f, are orbit a bit closer to one another than the 3:2 resonance. The outer planet, g, orbits well exterior to the inner five. Values of the osculating orbital periods for Kepler-11's planets, taken from \cite{Bedell:2017}, are at epoch $T_{\rm BJD} = 2455700$.

Figure \ref{fig:Kep11} shows orbital period evolution of all six planets as calculated for three of the 101 simulations, selected by rank according to the mean period of Kepler-11 c (KOI-157.01) over the $10^6$ day interval as discussed above.   Note that four-year running averages of the periods of the dynamically-active five inner planets vary substantially more than the uncertainties in the measured mean periods during the \ik epoch, especially for the two shortest-period planets.

\begin{figure}[!hbt]
%\vspace{-0.5 in}
 \includegraphics [width =  2.2in, angle =0]{Period_Kep11b_20y.pdf}
  \includegraphics [width =  2.2in, angle =0]{Period_Kep11b_200y.pdf}
   \includegraphics [width =  2.2in, angle =0]{Period_Kep11b_2000y.pdf}
\newline
%\vspace{-0.5 in}
 \includegraphics  [width =  2.2in, angle =0]{Period_Kep11c_20y.pdf}
  \includegraphics   [width =  2.2in, angle =0]{Period_Kep11c_200y.pdf}
    \includegraphics [width =  2.2in, angle =0]{Period_Kep11c_2000y.pdf}
     \newline
% \vspace{-0.5 in}
  \includegraphics [width =  2.2in, angle =0]{Period_Kep11d_20y.pdf}
    \includegraphics [width =  2.2in, angle =0]{Period_Kep11d_200y.pdf}
 \includegraphics [width =  2.2in, angle =0]{Period_Kep11d_2000y.pdf}
      \newline
%   \vspace{-0.5 in}
 \includegraphics [width =  2.2in, angle =0]{Period_Kep11e_20y.pdf}
   \includegraphics [width =  2.2in, angle =0]{Period_Kep11e_200y.pdf}
      \includegraphics [width =  2.2in, angle =0]{Period_Kep11e_2000y.pdf}
\newline
% \vspace{-0.5 in}
  \includegraphics [width = 2.2in, angle =0]{Period_Kep11f_20y.pdf}
    \includegraphics [width = 2.2in, angle =0]{Period_Kep11f_200y.pdf}
        \includegraphics [width = 2.2in, angle =0]{Period_Kep11f_2000y.pdf}
      \newline
%   \vspace{-0.5 in}    
 \includegraphics [width = 2.2in, angle =0]{Period_Kep11g_20y.pdf}
   \includegraphics [width = 2.2in, angle =0]{Period_Kep11g_200y.pdf}  
      \includegraphics [width = 2.2in, angle =0]{Period_Kep11g_2000y.pdf}  
 \caption{Transit-to-transit and 4-year average periods for each of the planets known to orbit Kepler-11, with panels ordered from top to bottom by increasing orbital period. The small dots in the panels on the left show transit-to-transit orbital periods (the length of the time interval between the midpoint of one transit and the midpoint of the subsequent transit) from three samples of the 101 solutions used to compute the period estimates and uncertainties listed in Table \ref{tab:periods}. Specifically, the 101 solutions are ordered by increasing mean long-term ($10^6$~days) orbital period of KOI-157.01 = Kepler-11~c, and blue represents the 17$^{\rm th}$ sample (one standard deviation below the median), black the 51$^{\rm st}$ sample (the median) and red the solution that is 85$^{\rm th}$ (one standard deviation above the median) on this list. The more brightly colored points with a larger symbol size in the left panels and all points in the middle and right panels represent the running average of 4-year segments centered on the given time.  Time is measured from the beginning of \ik science operations. The green $\times$'s near the left of each panel represent our fits to the \ik transit times assuming constant period (Table \ref{tab:planetcatalog}). Blue points are plotted first, then black and finally red points on top. {The upper five panels in the right column have been thinned to show only every ninth, seventh, fourth, third and second transit-to-transit interval to limit the size of the manuscript file. }\label{fig:Kep11}  } 
\end{figure}


\subsubsection{Kepler-36 = KOI-277}

Kepler-36 has two planets, each more than four times as massive as Earth, on %very proximate 
orbits moderately close to the 7:6 mean motion resonance.  The planets are not in a low-order mean-motion resonance, but their proximal orbits lead to strong dynamical interactions.  The outer planet, which is a bit less than twice as massive as the inner one, nonetheless has 15 times the volume.  
Some planetary parameters allowed by short-term fits to the \ik data in the system discovery paper by  \cite{Carter:2012} were subsequently eliminated by imposing the requirement for long-term stability  \citep{Deck:2012}. 

We performed a photodynamical analysis of the data using the PhotoDynamical Multiplanet Model (PhoDyMM, Ragozzine et al., in prep.) similar to other analyses in the past \citep[e.g.,][]{MacDonald:2021}. Each of the 101 samples from the posterior distribution was then integrated using REBOUND's WHFast integrator \citep{Rein:2015} with a timestep of 2\% of the inner planet's orbit for $10^6$ inner planet orbits. Inspection of the final orbital states for all 101 systems showed that they were still on stable orbits very similar to those of the current system. While our integrations are substantially shorter than those of \cite{Carter:2012}, they are still much longer than the typical Lyapunov time of 10 years \citep{Deck:2012}.  Therefore, it seems likely that the change in stability properties results from having a more accurate set of initial conditions that were derived from the full \ik lightcurve, which included an additional year of Short Cadence data. % ($\sim$10$^7$ days instead of $\sim$10$^9$ days). I get 10^6 vs 10^10, but 10^11 w/o tides isn't physical.

In analogy with Fig.~\ref{fig:Kep11}, Figure \ref{fig:Kep36} shows orbital period evolution for three of the 101 simulations, selected as described in the caption. Note that the orbital periods of both planets averaged over 4 year intervals vary by almost 10 times as much as the uncertainties in their orbital periods measured during the \ik epoch.
 
\begin{figure}[!hbt]
%  \vspace{-0.5 in}
\includegraphics [width = 2.4 in, angle =0]{Period_Kep36b_10y.pdf}
  \includegraphics [width = 2.4 in, angle =0]{Period_Kep36b_100y.pdf}
  \newline
% \vspace{-0.5 in}
 \includegraphics [width = 2.4 in, angle =0]{Period_Kep36c_10y.pdf}
   \includegraphics [width = 2.4 in, angle =0]{Period_Kep36c_100y.pdf}
 \caption{Transit-to-transit and 4-year average periods for Kepler-36 b and Kepler-36 c. The dots in panels on the left show transit-to-transit orbital periods from three samples following photodynamical fits to the lightcurve. The solid curves in the panels on the left and the curves on the right show the average of 4-year segments centered on the given time. Black represents the sample with the median long-term (1000 years) average period of KOI-277.01 (= Kepler-36~c), blue the sample with the 17$^{\rm th}$ lowest average period and red the one that is   85$^{\rm th}$ on this list. Time is measured from the beginning of \ik science operations. The green $\times$ near the left side of all panels represent our fits to the measured transit times assuming constant period (i.e., the orbital period and uncertainty of that planet listed in Table \ref{tab:planetcatalog}).
 \label{fig:Kep36} } 
\end{figure} 
 
\subsubsection{Kepler-80 = KOI-500}
Kepler-80 harbors six known transiting planets. 
The innermost is a USP planet that is (at least on the time scales relevant here) dynamically detached from its siblings. The middle four planets, with periods of 3.1, 4.6, 7.1, and 9.5 days, were studied extensively by \cite{MacDonald:2016}. Two years later, a sixth transiting planet, with $P = 14.6$~days,
 was discovered by \cite{Shallue:2018}.
 
 All six planets were included in a full photodynamical analysis by \citet{MacDonald:2021}. This photodynamical analysis was performed using the PhotoDynamical Multiplanet Model (PhoDyMM, Ragozzine et al., in prep.), with all six masses allowed to float, but all the longitudes of ascending node were fixed at 0$^{\circ}$. The middle four planets form a resonance chain, with each neighboring pair having  period ratio $\sim 1 - 3$\% larger than either 3/2 or 4/3 and each neighboring threesome librating within a three-body resonance. The orbital period of outer planet suggests that it, too, is a member of the resonance chain. 
However, transits of the outer planet are much shallower, and only 14\% of
the fits to the \ik data by \citet{MacDonald:2021} find it to be
 librating within a resonance.
 
%Because the four middle planets orbiting Kepler-80 are constrained by \ik data to be locked in a resonance chain and the outer planet must be at least very close to resonance with the chain, we consider it likely, albeit far from certain, that the outer planet is also a member of the resonance chain. Therefore, we have done two analyses of the system, the first in which we consider a random set of 101 samples from the fits to the data produced by PhoDyMM, and the other in which we only consider the ** of these solutions that have the five outer planets librating within resonance.

%For our first solution
As for Kepler-36,
101 samples were taken from the posterior distribution from PhoDyMM. Each of these mock systems was integrated for 100 years (nearly $4\times10^4$ orbits of the innermost planet). The $9^{th}$ through $14^{th}$ numerical rows of
Table \ref{tab:periods} list the mean and dispersion of the periods of each planet in this sample at epoch $T_{\rm epoch}$ = 800.0 days and averaged over 100 years. Figure \ref{fig:Kep80} shows period variations of each of the planets derived from our integrations that used three samples from the posteriors calculated by Ragozzine et al.~(in prep.). The four-year averaged periods of the five planets in the resonance chain have fractional variations of $\sim 10^{-5} - 10^{-4}$.

%Our second solution, presented in the $15^{th}$ through $21^{st}$ numberical rows of Table \ref{tab:periods} and Figure \ref{fig:Kep80stable}, only considered the ** of these 101 samples wherein the outer planet participated in the resonance chain.


\begin{figure}[!hbt]
 % \vspace{-1.0 in}
 \includegraphics [width = 2.1 in, angle =0]{Period_Kep80f_10y.pdf}
  \includegraphics [width = 2.1 in, angle =0]{Period_Kep80f_100y.pdf}
  
%  \vspace{-.9 in}
   \includegraphics [width = 2.1 in, angle =0]{Period_Kep80d_10y.pdf}
  \includegraphics [width = 2.1 in, angle =0]{Period_Kep80d_100y.pdf}
 
%  \vspace{-.9 in}
   \includegraphics [width = 2.1 in, angle =0]{Period_Kep80e_10y.pdf}
  \includegraphics [width = 2.1 in, angle =0]{Period_Kep80e_100y.pdf}
 
%  \vspace{-.9 in}
   \includegraphics [width = 2.1 in, angle =0]{Period_Kep80b_10y.pdf}
  \includegraphics [width = 2.1 in, angle =0]{Period_Kep80b_100y.pdf}
 
%  \vspace{-.9 in}
  \includegraphics [width = 2.1 in, angle =0]{Period_Kep80c_10y.pdf}
   \includegraphics [width = 2.1 in, angle =0]{Period_Kep80c_100y.pdf}

%  \vspace{-.9 in}
   \includegraphics [width = 2.1 in, angle =0]{Period_Kep80g_10y.pdf}
  \includegraphics [width = 2.1 in, angle =0]{Period_Kep80g_100y.pdf}
 \caption{Transit-to-transit and 4-year average periods for each of the planets known to orbit Kepler-80, ordered by increasing orbital period. The dots in panels on the left show transit-to-transit orbital periods from three samples of the solution posteriors, as in Fig.~\ref{fig:Kep36}. Time is measured from the beginning of \ik science operations. The green $\times$ at the midpoint of \ik observations represent our fits to the TTs assuming constant period (Table \ref{tab:planetcatalog}).  {The upper right panel has been thinned to show every fourth mean transit-to-transit interval.}  \label{fig:Kep80}  } 
\end{figure}

\begin{figure}[!hbt]

 %\caption{Transit-to-transit and 4-year average periods for each of the planets known to orbit Kepler-80, ordered by increasing orbital period. Only solutions in which the outer planet was librating in the resonance chain were included in this analysis. The dots in panels on the left show transit-to-transit orbital periods from three samples of the ....Time is measured from the beginning of \ik science operations. The green $\times$ at the beginning represent our fits to the lightcurve assuming constant period (Table \ref{tab:planetcatalog}).   \label{fig:Kep80stable}}
\end{figure}

\subsubsection{Kepler-223 = KOI-730}

Kepler-223 is a system of four planets in mean-motion resonance, where the three-body angles are consistent with libration, and the TTVs have a few-hour amplitude with a timescale of a few years \citep{Mills:2016}.  The planets have higher eccentricities than typical for closely-spaced planetary systems, and the period ratios of the planets are extremely  close to ratios of small integers, with departures of order 0.1\% or less \citep{Lissauer:2011b,Mills:2016}. 

To understand how the observed transit timing variations manifest on a timescale beyond the \ik time series, we draw samples from the posterior of the photodynamic fits to the data.  \cite{Mills:2016} produced a sample of solutions that assure the two Laplace angles,  one relating the inner three planets and the other relating the outer three planets, librate with small amplitude over a 100 year span, and called it the C3 posterior.  Values of their osculating orbital periods for Kepler-223's planets are given at $T_{\rm epoch}$ = 800.0 days.  %FIX ME: %One of these systems\footnote{internal numbering: run 194} we use as an example of the possible period variation on timescales much longer than the \ik data; see Figure~\ref{fig:Kep223Periods}. We computed the TTV ($O-C$) during the \ik epoch  and verified that it matches \ik observations (Figure 1 of \citealt{Mills:2016}), then extended this computation for 100 years.  We plot table 1's period and error for comparison with the first 4-year average value. The variations on 4-year segments have a standard deviation of $4.4\times10^{-4}$~d, $1.2\times10^{-3}$~d, $1.1\times10^{-3}$~d, and $2.3\times10^{-3}$~d, for planets b-e respectively, factors of 5 -- 14 larger than the formal error bars in Table 1. 

\cite{Mills:2016} ran $N$-body simulations of 300 systems from the C3 posterior for $10^7$~years, with outputs every $10^4$~years, and found all of them to be stable.  We inspected the osculating semimajor axes over that time.  In many cases, the oscillations in semimajor axis on 100 year timescale are small, but then grow by a factor of several or even an order of magnitude, and simultaneously lose the libration of the three-body angles, indicating that these solutions were not in secular steady-state.

We found that only 12 systems out of the 300 in the C3 posterior seemed to keep the same semimajor axis behavior from the first 100 years for all $10^7$ years. %Applying the Copernican principle, 
If the other trajectories were taken as models of the observed system, then the system must have been observed at a special time, which violates the Copernican Principle and thus is highly unlikely.  
Therefore, we consider the 12 continually-librating solutions likely represent better models of the system, and we restrict our attention to them for our analysis of long-term orbital period variations of this system.  

\begin{figure}[!hbt]

 % \vspace{-.5 in}
 \includegraphics [width = 2.38 in, angle =0]{Period_Kep223b_20y.pdf}
  \includegraphics [width = 2.38 in, angle =0]{Period_Kep223b_200y.pdf}
\newline

  %\vspace{-.5 in}
 \includegraphics [width = 2.38 in, angle =0]{Period_Kep223c_20y.pdf}
   \includegraphics [width = 2.38 in, angle =0]{Period_Kep223c_200y.pdf}
 \newline

 % \vspace{-.5 in}
  \includegraphics [width = 2.38 in, angle =0]{Period_Kep223d_20y.pdf}
    \includegraphics [width = 2.38 in, angle =0]{Period_Kep223d_200y.pdf}
   \newline

  %\vspace{-.5 in}
\includegraphics [width = 2.38 in, angle =0]{Period_Kep223e_20y.pdf}
   \includegraphics [width = 2.38 in, angle =0]{Period_Kep223e_200y.pdf}
\caption{Transit-to-transit and 4-year average periods for each of the planets known to orbit Kepler-223, ordered by increasing orbital period. The dots in panels on the left show transit-to-transit orbital periods from three samples of the C3 posterior from the photodynamical models of \cite{Mills:2016} for which the three-body resonant arguments of the inner and outer threesomes of planets remain in libration for $10^7$ years. The solid curves show the average of 4-year segments centered on the given time. Black represents the sample with the median long-term (1000 years) average period of KOI-730.01 among the 12 samples that remained in libration (since this number is even, we broke the degeneracy by choosing the one whose period was closer to the average of the ensemble); the sample with the second shortest average period of this planet is shown in blue, and red represents the one with the second longest average period.  Time is measured from the beginning of \ik science operations. The green $\times$ at the beginning represent our fits to the measured TTs assuming constant period (Table \ref{tab:planetcatalog}).   \label{fig:Kep223}  } 
\end{figure}


The long-term (1000 year) mean periods of each of Kepler-223's four planets and the standard deviations thereof are listed in Table \ref{tab:periods}.  Figure \ref{fig:Kep223} shows the period variations of three of these possible realizations, with black representing the system with median mean period of KOI-730.01 (we broke the degeneracy caused by the even number of systems in the sample by selecting the one with period closer to the mean of the 12 samples), and blue and red showing the sample systems for which KOI-730.01 has the second shortest and second longest mean values, respectively. 


On shorter timescales, due to orbital fluctuations, the period ratios cross back and forth across the nominal period ratios of 4:3 and 3:2 -- this is true both when computing the ratios with either osculating periods \citep[Extended Data Fig. 5]{Mills:2016} or modelled transit periods. The four-year average periods vary by a few parts in $10^{-4}$, and the period ratio for the inner pair of planets drops slightly below 4/3 for some portions of the simulated interval in each of the 12 samples. In contrast, the ratio of periods of the middle pair of planets stay slightly above these small integer ratios.

%Kepler-223 is a special case, a system of 4 planets in mean-motion resonance, where the three-body angles are consistent with libration, and the TTVs have a few-hour amplitude with a timescale of a few years \citep{Mills:2016}.  

%To understand how these transit timing variations manifest on a timescale beyond the \ik timespan, we draw samples from the posterior of the photodynamic fits to the data. \cite{Mills:2016} have produced a sample of solutions that assure the two Laplace angles --- the one relating the inner three planets and the one relating the outer three planets --- librate with small amplitude over a 100 year span, and call it the C3 posterior. One of these systems\footnote{internal numbering: run 194} we use as an example of the possible period variation on timescales much longer than the \ik data; see Figure~\ref{fig:Kep223Periods}. We computed the TTV ($O-C$) during the \ik epoch  and verified that it matches \ik observations (Figure 1 of \citealt{Mills:2016}), then extended this computation for 100 years.  We plot table 1's period and error for comparison with the first 4-year average value. The variations on 4-year segments have a standard deviation of $4.4\times10^{-4}$~d, $1.2\times10^{-3}$~d, $1.1\times10^{-3}$~d, and $2.3\times10^{-3}$~d, for planets b-e respectively, factors of 5 -- 14 larger than the formal error bars in Table 1. 

%\cite{Mills:2016} ran $N$-body simulations of 300 systems from the C3 posterior for $10^7$~years, with outputs every $10^4$~years, and found all of them to be stable.  We inspected the osculating semimajor axes over that time.  In many cases, the oscillations in semimajor axis on 100 year timescale is small, and then grows by a factor of several or even an order of magnitude for the rest of the time-span, suggesting these solutions were not in secular steady-state. For instance, the case we highlight in Figure~\ref{fig:Kep223}, the ranges of periods visited do not stay in the ranges plotted, but grow by about a factor of 2 -- 3 over the next $10^3$ years. Many other systems had period variations that grew even more, and concurrently lose the libration of the three-body angles. This observation suggests that if these trajectories are taken as models of the observed system, the system must have been observed at a special time, which is highly unlikely.  We found 12 systems out of the 300 seemed to keep the same semimajor axis behavior from the first 100 years for all $10^7$ years. By the above argument, these may represent better models of the system, and they would have slightly less variation in period than illustrated in Figure \ref{fig:Kep223}. Therefore it is these, run over $10^3$~yr, that we use to determine the long-term average period listed in the final column of Table~\ref{tab:periods}.

\subsubsection{Kepler-29 = KOI-738}
We took 101 samples from the TTV posteriors of Kepler-29 from \cite{Jontof:2021} and simulated transit times over 10$^6$ days. The system contains two sub-neptunes on proximate, dynamically-interacting orbits with orbital periods close to the 9:7 mean motion resonance \citep{Migaszewski:2017}. 

% Our average period for each planet follows fits to the observed transit times, 

\begin{figure}[!hbt]
%  \vspace{-1.5 in}
\includegraphics [width = 2.4 in, angle =0]{Period_Kep29b_20y.pdf}
  \includegraphics [width = 2.4 in, angle =0]{Period_Kep29b_200y.pdf}

 % \vspace{-.5 in}
  \includegraphics [width = 2.4 in, angle =0]{Period_Kep29c_20y.pdf}
   \includegraphics [width = 2.4 in, angle =0]{Period_Kep29c_200y.pdf}
 \caption{Transit-to-transit and 4-year average periods for both of the planets known to orbit Kepler-29, ordered by increasing orbital period. The dots in panels on the left show transit-to-transit orbital periods from three samples following \cite{Jontof:2021}'s dynamical fits to the long cadence transit times of \cite{Rowe:2015b}. The solid curves show the average of 4-year segments centered on the given time. Black represents the sample with the median long-term (1000 years) average period of KOI-738.01; blue the sample with the 17$^{\rm th}$ lowest average period and red the one that is   85$^{\rm th}$ on this list. Time is measured from the beginning of \ik science operations. The green $\times$ near the left side of all panels represent our fits to the TTs assuming constant period (Table \ref{tab:planetcatalog}).  
 \label{fig:Kep29} } 
\end{figure}

In Figure~\ref{fig:Kep29}, we see that \ik observed Kepler-29 when the transit-to-transit periods of Kepler-29 b and c were near their extrema. Four-year mean periods vary by a few parts in $10^{-4}$, with the periods of the two planets being highly anti-correlated.  The ratio of orbital periods over the \ik baseline is 1.2851, slightly (about 1 part in 2000) \emph{less than} that of the 9:7 commensurability (1.285714).  The average period ratio over 1000 years of the samples shown in Figure~\ref{fig:Kep29} are 1.285812 (17$^{\rm th}$), 1.285754 (51$^{\rm st}$), 1.285729 (85$^{\rm th}$) respectively, all just above commensurability. Indeed, much of the variation averages out on decadal time scales, but four year average period ratios oscillate about 9/7 and the long-term average period ratio in this system is several standard deviations away from the value  implied by the orbital periods for the planets listed in our catalog.  

\cite{Holczer:2016} found strong TTVs for both planets in KOI-738, but each planet's TTs were fit by a polynomial rather than a sine wave, so mean orbital periods were not estimated.  This fit was likely selected because the orbital periods of the planets oscillate with a periodicity of about ten years, and their TTVs during the four years of the \ik mission resemble parabolas; see the panels on the left in Fig.~\ref{fig:Kep29}.


\subsubsection{Kepler-419 = KOI-1474}

Two planets are known to orbit Kepler-419: a transiting inner planet (Table \ref{tab:planetcatalog}) on a quite eccentric orbit ($e \approx 0.8$) and a  non-transiting super-jupiter that has a period almost ten times as long but nonetheless induces large TTVs because of its high mass and the large eccentricity of the transiting planet's orbit.  The system is well characterized by both RV and TTV analysis (see Table \ref{tab:nontransiting} for the properties of the non-transiting planet). The transiting planet, Kepler-419~b, displays a quintessential ``photoeccentric effect'' \citep{Dawson:2012}, wherein a lower bound on the eccentricity of a planet can be estimated from the shape and duration of the transit lightcurve. The period ratio of these two planets is $\sim$~9.657. 

Table \ref{tab:periods} lists the value of the osculating orbital period for Kepler-419~b at $T_{\rm BJD} = 2454958$ from \cite{Almenara:2018}.  We took the 4044 posterior samples of the system parameters from the combined RV/photodynamical fits of \cite{Almenara:2018} and ran simulations for 10,000 years. For each of these samples, we computed a long-term orbital period by taking a linear fit to simulated transit times. We tabulate the average and standard deviation of these 4044 long-term periods. The transit-to-transit interval of the inner planet and its four year running average for three of the samples are plotted in Figure~\ref{fig:Kep419Periods}.  Note that the plots in this figure cover longer timescales than those for other planets considered in this subsection because the longer orbital periods of the planets lead to larger characteristic time intervals over which major variations are observed.  

%Values of the osculating orbital periods for Kepler-419~b at $T_{\rm BJD} = 2454959.3$ from \cite{Almenara:2018}.  We took samples of the system parameters from the combined RV/photodynamical fits of \cite{Almenara:2018} and plot the transit-to-transit interval of the inner planet and its four year running average in Figure~\ref{fig:Kep419Periods}.  Note that the plots in this figure cover longer timescales than those for other planets considered in this subsection because the longer orbital periods of the planets lead to larger characteristic time intervals over which major variations are observed.  

%For Kepler-419, we use 101 posterior samples of Mstar, a from Almenara+. We ran simulations for 10,000 years to calculate an average and standard deviation orbital period over the long term following a linear fit to simulated transit times. For the mean and standard deviation of the period at epoch, we took a linear fit to simulated transit times over the baseline of 22 Kepler transits (22 orbits) in Almenara+. The results are in Table~\ref{tab:periods} (Why does this differ to the Holczer period? My guess is the TTV signal in Fig 8 of Alamenar+ shows that a cosine is a poor model for the TTV signal.) 

\begin{figure}[!hbt]
 \includegraphics [width = 2.2 in]{Period_Kep419b_100y.pdf}
  \includegraphics [width = 2.2 in]{Period_Kep419b_1ky.pdf}
  \includegraphics [width = 2.2 in]{Period_Kep419b_10ky.pdf}
 \caption{Transit-to-transit and 4-year average periods for Kepler-419~b, the only known \emph{transiting} planet in this system. We integrated 4044 samples of system parameters from \citet{Almenara:2018} for  $10^4$ years, then sorted them by averaged orbital period in Kepler-419~b, and display results from the 642$^{\rm nd}$ (blue), 2023$^{\rm rd}$ (black) and 3403$^{\rm rd}$ (red) members of the resultant list.  Left panel: the small points mark transit-to-transit period, while the larger points mark a 4-year running average period over 100 years. The right two panels include only the 4-year running averages over intervals of 1000 yr and $10^4$ yr, respectively. {The panel on the right has been thinned to show every fourth mean transit-to-transit interval.} 
 \label{fig:Kep419Periods}} 
\end{figure}

At the present epoch, Kepler-419~b is close to periapse when its transits. The dips in orbital period near 2700 and 7300 years occur when the planet passes through apoapse near the time when it transits. This illustrates the variations predicted from Equation (\ref{eqn:periodsII}) with $\varpi \ll 1$.

We examined the intervals between times when the center of the non-transiting planet Kepler-419~c was closest to the sky-projected location of its host star Kepler-419 and closer to the Solar System than is Kepler-419. These intervals varied with the same periodicity and opposite phase as the corresponding variations of the averaged transit-to-transit period of Kepler-419~b. However, unlike its transiting companion (right panel of Fig.~\ref{fig:Kep419Periods}), this planet's variations are nearly sinusoidal. These behaviors are consistent with \cite{Almenara:2018}'s findings that the  periapse locations of the two planets oscillate about anti-alignment with a small amplitude and the orbital eccentricity of Kepler-419~c is small ($e < 0.2$) whereas Kepler-419~b has $e \approx 0.8$. 

\subsubsection{Kepler-60 = KOI-2086}

\begin{figure}[!hbt]
 \includegraphics [height = 0.9 in]{Period_Kep60b_20y.pdf}
  \includegraphics [height = 0.9 in]{Period_Kep60b_200y.pdf}
  \includegraphics [height = 0.9 in]{Period_Kep60b_1000y.pdf}
  \newline
   \includegraphics [height = 0.9 in]{Period_Kep60c_20y.pdf}
  \includegraphics [height = 0.9 in]{Period_Kep60c_200y.pdf}
  \includegraphics [height = 0.9 in]{Period_Kep60c_1000y.pdf}
  \newline
   \includegraphics [height = 0.9 in]{Period_Kep60d_20y.pdf}
  \includegraphics [height = 0.9 in]{Period_Kep60d_200y.pdf}
  \includegraphics [height = 0.9 in]{Period_Kep60d_1000y.pdf} 
 \caption{Transit-to-transit and 4-year average periods for each of the planets known to orbit Kepler-60, ordered by increasing orbital period. The dots in panels on the left show transit-to-transit orbital periods from three samples (of the 101 considered) following dynamical fits to the long cadence transit times of \cite{Rowe:2015b}. The solid curves represent the average of 4-year segments centered on the given time. Black represents the sample with the median long-term (1000 years) average period of KOI-2086.01; blue the 17$^{\rm th}$ sample in the  list ranked by  average period and red the  85$^{\rm th}$ member of the list. Time is measured from the beginning of \ik science operations. The green $\times$ at the beginning represent our fits to the lightcurve assuming constant period (Table \ref{tab:planetcatalog}).   {The top panel on the right  has been thinned to show every fourth mean transit-to-transit interval, whereas the middle and bottom right panels show every third mean transit-to-transit interval.} 
 \label{fig:Kep60Periods}} 
\end{figure}

The Kepler-60 system contains three confirmed planets with periods between 7 and 12 days, with both neighboring pairs orbiting near first-order MMRs. The lightcurve also reveals an unverified planet candidate with an orbital period of 336 days that we do not consider in our analysis of the inner threesome.

We took 101 samples from the TTV posteriors of Kepler-60 from \cite{Jontof:2021}. From simulations of these 101 samples, we estimated uncertainties on the period ratios of Kepler-60 c/b (5:4) and Kepler d/c (4:3) given the $17^{\rm th}$, $51^{\rm st}$ and $85^{\rm th}$ period ratio after averaging period ratios over 1000-year simulations and sorting.  We found the inner pair have a period ratio of 1.250473 $\pm~0.000015 $ and the outer pair have period ratio 1.334176 $\pm~0.000027$. Averaged over four years, none of the neighboring planet pairs dropped below the ratio of small integers signifying their resonance in any of the three samples plotted in Figure \ref{fig:Kep60Periods}.


\pagebreak
%\clearpage

\section{Conclusions} \label{sec:conclusions}

We have assembled in Table \ref{tab:planetcatalog} a catalog of \ik planet candidates that prioritizes completeness %of multiple planet systems 
and makes use of additional information to improve accuracy whenever practical rather than providing a sample that has been defined and analyzed homogeneously, as done for the final PC catalog produced by the \ik Project \citep{Thompson:2018}. We have also listed an alternative set of planetary properties  (\S\ref{sec:unified}) for most planet candidates that inputs the more uniformly-derived set of stellar properties from \cite{Berger:2020a}.  \cite{Berger:2020a}'s results are available for $\sim 95$\% of \ik target stars that host one or more PCs {(and also for the vast majority of other \ik targets)}, and selecting the values listed in the appropriate columns in Table \ref{tab:planetcatalog} therefore yields measurements of planetary properties that are well-suited for studies of occurrence rates (see Appendix \ref{sec:App_ORtables} for details). {Figure \ref{fig:perrad} displays the planet candidates on the orbital period-planetary radius plane, showing the multiplicity of the system in which each PC resides. Figure \ref{fig:gallery} shows the periods of the planets in each of the multi-planet systems included in our catalog.}

Table \ref{tab:planetcatalog} presents an extensive set of stellar and planetary properties for each of almost 9700 KOIs (\ik Objects of Interest), almost half of which are considered viable planet candidates. Section \ref{sec:unified} provides a column-by-column list of types of data presented in Table \ref{tab:planetcatalog}, and more details on the derivation of many key planetary and stellar properties are provided elsewhere in \S\ref{sec:catalog}. 
A less comprehensive listing of the properties of %circumbinary transiting planets and 
non-transiting planetary companions to transiting \ik planets is provided in Table %s \ref{tab:cbp} and
\ref{tab:nontransiting}; see Appendix \ref{sec:App_nontransiting}.%, respectively. 

{Table \ref{tab:planetcatalog} is superior to previous cumulative catalogs of \ik planet candidates in that it provides a more complete listing of KOIs, more accurate and diverse dispositions of KOIs (for details see item \#64 in the list of tabulated properties provided in \S\ref{sec:unified}), and more accurate stellar and derived planetary properties. {Because we utilize information from previous \ik planet candidate catalogs, community studies and our own analyses, our assessments of dispositions should be at least as reliable as those of any previous \ik PC catalog for the portion of the sample listed in both catalogs.} The most substantial improvements in planetary properties are for orbital periods of planets exhibiting TTVs (\S\S\ref{sec:periods}, \ref{sec:Pconclusions}), as well as transit models and calculated radii of planets with grazing transits (\S\ref{sec:transitmodel}) and/or substantially-revised estimates of host star size.

Figure \ref{fig:transitingplanets} illustrates all planet candidates in our catalog as they transit their stars. This image is the successor to diagrams released as part of press packages for some of the official \ik project catalogs of planet candidates; we show it here to emphasize the fact that with substantial improvements in estimates of stellar radii and planetary impact parameters, the current version now has substantial scientific content.}

\begin{figure}
    \centering
    \includegraphics[scale=.175]{Keplerplot_small_cbar.png}
    \caption{{Visualization of all planet candidates listed in Table \ref{tab:planetcatalog} transiting in front of their host stars.  The radii of all of the planets and stars are shown to the same scale, and the vertical distance of each planet from the center of its star shows the impact parameter of the transit. Systems are ordered by stellar size, and the color of each stellar disk represents the star's $T_{\rm eff}$, with the color scale shown below. The single star located between the top two rows on the right shows Jupiter and the Earth transiting the solar disk (for scale). The 12 host stars that have $T_{\rm} < 3500$~K and the 2 with $T_{\rm} > 10000$~K are represented by colors shown at the extrema of the scale bar. The version of this image included within the pdf manuscript is compressed to reduce the size of the file; a full-resolution version that clearly displays each of the transiting planets upon the disk of its host star is provided within the electronic version of the manuscript.}}%using koiprops20230830v2
    \label{fig:transitingplanets}
\end{figure}



\subsection{Estimating Orbital Periods}  \label{sec:Pconclusions}

We have made special efforts, both in data analysis (\S\ref{sec:periods}) and theoretically (\S\ref{sec:periodtheory}), to improve the accuracy of planetary orbital periods to aid in ephemeris predictions and dynamical studies. The values of $P$ listed in Table \ref{tab:planetcatalog} are generally of equal or higher accuracy than those in previous tabulations, with estimates for many of the planets exhibiting TTVs significantly improved.
Fractional uncertainties quoted for orbital periods of the majority of planet candidates listed in recent \ik catalogs, including those presented herein, are $<10^{-5}$, with values $\sim 10^{-6}$ (corresponding to 2 minutes per 4 years) being typical.  These small uncertainties suggest {(in some cases misleadingly)} that ephemeris predictions for most \ik planets are robust for decades to come.  However, the tabulated periods and quoted uncertainties are for the mean times between midpoints of successive transits during the time interval in which transits were observed, and do not reflect possible long-timescale TTVs (Section \ref{sec:periodtheory}). 

Transit timing variations (TTVs) produce errors in estimates of some planets' orbital periods that need to be accounted for in certain dynamical investigations and ephemeris predictions.
Periodic sinusoidal TTVs with timescales short compared to the interval of \ik observations largely average out and do not produce significant errors in estimates of orbital periods.  TTVs with timescales comparable to the four year interval of \ik observations have been fit for dozens of \ik planet candidates to estimate long-term average orbital periods by \cite{Holczer:2016}, and more detailed dynamical models have been used to estimate long-term average periods of a small number of well-studied planets, including the seven systems presented herein (\S\ref{sec:individual}, Figures \ref{fig:Kep11}~--~\ref{fig:Kep60Periods}). 
{These figures show that in many cases the four-year average transit-to-transit orbital period deviates from the long-term average orbital period by a factor many times as large as the formal uncertainty of the four-year average.} Most \ik planets that show large TTVs are near mean-motion orbital resonances with other planets. The largest effect for planets moderately close to two-body resonances is due to rotation of the forced eccentricity vector by resonant perturbations; the timescale of this precession for most \ik planets is short compared to  the four years of \ik observations, so the variations tend to average out.

Libration of planets locked in resonances typically occurs on timescales longer than the \ik baseline, but most \ik planets do not appear to be resonantly-locked. The more general but smaller (during the era of \ik observations) effect is caused by secular precession of the planets' free eccentricities, which usually takes much longer than the four-year baseline of the \ik observations to complete a revolution, so it is not accounted for in estimates of mean periods or uncertainties.  This precession causes a discrepancy between \ik era mean orbital period and long-term mean orbital period to exist even for planets having TTVs that are too small to be observable during the epoch of \ik observations, and the magnitude of this discrepancy increases with the eccentricity of the planet's orbit. Thus, some systems/planets that do not show TTVs, for which an observer might assume that the linear ephemeris is reliable, could have deviations on longer timescales, although for most planets without clear TTVs the \ik predictions are likely to be very good. The main concern is systems where the variations on 4-year timescales are too small to be detected, but where the amplitude of the decadal timescale TTVs is substantial. {Because of the very complicated and poorly-quantified selection biases of the \ik sample, as well as small number statistics applying to some classes of dynamical configurations, we do not attempt to quantify the numbers of systems that exhibit the types of behavior seen within these systems.}
 Nonetheless, it should thus be kept in mind that the actual uncertainties in future TTs have broad (albeit low) tails that are not captured by tabulated uncertainties and are growing in length and height with time until additional transits are observed.

{We identify multiplanet candidates that have periods that are too close to each other to remain stable, which we use 
to estimate the percentage of apparent multiplanet systems wherein the planets are distributed  between two blended stars as $\sim~2.6\%$. Similarly, we use an error of a factor of two in the estimated period of a planet candidate that we identified by stability considerations to estimate the number of orbital periods that are aliases of the true periods to be $\sim~0.36\%$ (Section \ref{sec:falsemultis}). Other evidence suggests that a somewhat larger fraction of planets with estimated orbital periods of $\lesssim 1$~day actually complete two or more complete orbits within the period listed in catalogs (Section \ref{sec:selection}).} 

\subsection{Correlations Between Planetary Properties and Systems Multiplicity}  \label{sec:Mconclusions}

We find that the vast majority of PCs with low S/N are candidate single planets rather than being in multiplanet systems (multis).  In contrast, for those with moderate S/N, there are similar numbers of PCs in multis and in singles (Fig.~\ref{fig:snr_comparison}). Since the fraction of actual planetary detections that are in multis probably is similar for PCs with low S/N and those with moderate S/N, we suspect that a substantial fraction of these low S/N single planet candidates are false positives. % because.%, for higher S/N candidates, the ratio of singles to multis is closer to unity while for lower S/N the singles significantly outnumber the multis.
%This suggests that a substantial fraction of PCs with low S/N do not, in fact, represent transiting planets, i.e., are false positives. 
% - REMOVED BECAUSE BOTH REFERENCED PAPERS REQUIRED S/N>16. This may thus reduce the excess of \ik singles relative to expectations from the distribution of numbers of planetary systems with higher multiplicity \citep{Lissauer:2011b}, often referred to as the \ik dichotomy \citep{Johansen:2012}. However, planets in multis are almost three times as likely to show TTVs as singles (\S\ref{sec:periodDist}; see also \citealt{Holczer:2016}), supporting the hypothesis of more than one population of planetary systems.}

An early \ik result was that the fraction of large planets among systems with multiple transiting planets is smaller than among lone transiting planets \citep{Latham:2011}. A few years later it was pointed out that  multis are more concentrated to the period range $1.6-100$ days than are singles \citep{Lissauer:2014,Rowe:2014}.   

The size distributions of singles and multis are quite similar over the range $\sim$~0.5~--~3~R$_\oplus$ (Fig.~\ref{fig:radius_multis_singles}), although there is a hint of a larger fraction of planets in multis below the radius valley at $
\sim 1.7$~R$_\oplus$. The size distributions of singles and multis are also indistinguishable over the range $\sim$~5~--~10~R$_\oplus$  (Fig.~\ref{fig:large_radius_multiplicities_split}), although the fraction of planets with sizes 5~--~10 R$_\oplus$ that are found in multis is only about two-thirds that among small planets; planets in the 5~--~10 R$_\oplus$ size interval range from low-mass super-puffs with tens of percent H/He by mass to cool giant planets hundreds or even thousands of times as massive as the Earth{; mature brown dwarfs and the smallest main sequence stars also fall within this size range}.  The transition between this multis/singles abundance ratio is gradual in the range 3~--~5 R$_\oplus$, which may imply a fuzzy boundary or simply be the result of errors in estimated planetary sizes (primarily small planets having sizes overestimated, since there are far more small planets than large ones). Placing the boundary between ``small'' and ``large'' planets near the size of Neptune is consistent with the results of the contemporaneous study of \cite{Ghezzi:2021}, who investigated correlations between stellar metallicity and maximum radius of observed transiting planets. Above 10 R$_\oplus$, the number of multis drops off steeply relative to singles, with few planets of radius $R_p > 12$~R$_\oplus$ found in multi-transiting system. The concentration of the largest planets in singles is partly due to inflated hot jupiters rarely having close companions, but the singles also appear to have inflated jupiters at longer periods~--~perhaps on eccentric orbits that bring them close.  Alternatively, it may be that most PCs significantly larger than Jupiter with $P>10$~days are FPs caused by eclipsing binary stars. Both the size distribution and the period distribution of planets in two-planet systems are intermediate between the distributions of single planets and those in systems with more than two transiting planets. 

\subsection{Planetary Eccentricities}
We analyze the distributions of normalized transit durations (Eq.~\ref{eqn:tdur}) to confirm the previous result that single transiting planets are more likely to have high eccentricity than are planets in multiply transiting systems. We extend this result by demonstrating that planets in systems with two transiting planets are typically more eccentric than those in systems with three transiting planets, and the orbits of PCs in systems with four or more transiting planets tend to be even less elongated (Fig.~\ref{fig:EccMulti}). 

Planets with orbital periods $P > 6$~days are typically more eccentric than  short-period transiting planets.  In contrast, we find no other clear trends in the eccentricity distribution with orbital period (Fig.~\ref{fig:EccPeriod}). 

Transiting planets in the rocky size range ($R_p <$~1.6~R$_\oplus$) have lower average $e$ than do sub-neptunes and neptunes, which in turn are typically less eccentric than planets with $R_p >$~5~R$_\oplus$ (Fig.~\ref{fig:EccRadius}). However, no such trend exists within the population of planets in systems with both large and small transiting planets.


\subsection{Epilogue}
It has been more than a decade since \ik ceased its collection of data from its prime field of view. Nevertheless, the list of \ik planet candidate remains the largest and most homogeneous collection of exoplanets known. Our new catalog contributes to the understanding of \ik planet candidates, especially with our focus on theinformation-rich systems with multiple transiting planets. Improvements in estimates of orbital periods and a better understanding of processes that alter apparent orbital periods on a variety of time scales advance our understanding of planetary dynamics and improve ephemerides for prediction of future transits. More accurate impact parameters, identification of correlations with multiplicity, and identifying trends with eccentricities also provide new avenues for research into the formation and evolution of planetary systems.  


%---------Acknowledgements----------
\begin{acknowledgements}

\section*{Acknowledgements}
We thank Travis Berger for informative discussions regarding the merits and shortcomings of various catalogs of \ik stellar properties and for providing tabulations of calculated stellar parameters in advance of publication.
Steve Bryson provided us with valuable information regarding the properties of the DR25Supp catalog.% and \gaia RUWE values. 
Jose Manuel Almenara and Rosemary Mardling kindly provided data from their analysis of the Kepler-419 system that we used as input for our calculations of the long-term average period of Kepler-419~b.
Tony Dobrovolskis, Kevin Zahnle and two anonymous referees made constructive  comments on earlier versions of this manuscript. 

Primary support for authors in the USA was provided from NASA's Astrophysics Data Analysis Program 16-ADAP16-0034. 
J.F.R. acknowledges Canada Research Chair program and NSERC Discovery, which also covered funding for the publication of this article.
E.B.F. acknowledges NASA \ik Participating Scientist Program Cycle II grant NNX14AN76G, and NASA Origins of Solar Systems grant \#NNX14AI76G.  
The Center for Exoplanets and Habitable Worlds is supported by the Pennsylvania State University and the Eberly College of Science.  Compute Canada and Calcul Quebec computing resources supported the analysis of \ik observations.  


\end{acknowledgements}

\vspace{5mm}
\facilities{\ik space telescope}

\smallskip

\appendix

\section{Tabulated Properties Useful for Planetary Occurrence Studies}\label{sec:App_ORtables}

As noted in Section \ref{sec:catalog}, our primary stellar and planetary properties catalog (first 63 columns of Table \ref{tab:planetcatalog}) prioritized accuracy over uniformity. In contrast, replacing the values given in columns 39~--~44 and 46~--~61 of Table \ref{tab:planetcatalog} by those given in columns 64~--~85 provides an analogous listing using the stellar parameters from \cite{Berger:2020a} for all KOIs whose stellar parameters are listed in \cite{Berger:2020a}'s tabulation. \cite{Berger:2020a} provides stellar properties of $> 90$\% of \ikt's targets, unbiased by whether or not they host planet candidates. Stellar properties from the \cite{Fulton:2018} catalog incorporated information from spectra taken by the Keck 1 telescope, which are only available for a tiny fraction of \ik target stars, most of which are KOIs and either host multiple PCs or are brighter than $K_p=14.2$. Most of the planetary properties listed in the Table \ref{tab:planetcatalog} that were derived using the stellar parameters in \cite{Berger:2020a} or are independent of stellar parameters were derived in a uniform manner that makes them suitable for use in planetary occurrence rate studies.  We provide specific recommendations for such studies in this Appendix.

When performing occurrence rate studies, we recommend that
researchers only include planetary systems associated with target stars that pass a uniform set of selection criteria that do not contain an
implicit dependence on presence of KOIs. We also recommend using those stellar and planetary properties that are listed in columns 64~--~85 of Table \ref{tab:planetcatalog}, which are based upon tabulations of \cite{Berger:2020a}, rather than the heterogeneous listing presented in columns  39~--~44 and 46~--~61. Additionally, our best-available dispositions of KOIs, given by the first letter of the 63$^{\rm rd}$ column in this table, were derived by heterogeneous methods, thus are not appropriate {to adopt without modification. Specifically, no KOIs other than those found and classified as planet candidates by a homogeneous and well-characterized process such as that used for DR25 should be counted. Nonetheless, it may be of use to include our dispositions of DR25 PCs in assessing the reliability of the sample, i.e., in rejecting some KOIs that were classified as PCs in DR25}. Analogously,
dispositions from the DR25 supplemental catalog should not be
used to add candidates that were not listed as such in DR25, as their selections, like our own choices, were not identified by a fully automated and reproducible process. 
The \ik DR25 planet catalog  \citep{Thompson:2018} is the premier catalog derived from a uniform and
systematic analysis of \ik lightcurves, and the dispositions from DR25 are given by the third letter of our the four-letter code given in the 63$^{\rm rd}$ column in Table \ref{tab:planetcatalog}.  

\cite{Hsu:2019}
performed an analysis of planet occurrence rates that makes
use of stellar properties from \gaia DR2.  The \cite{Hsu:2019}
target star criteria is just one example of a set of selection criteria
that do not have an implicit dependence on whether KOIs were
identified for a given target.  Future studies may wish to make use of
other large surveys (e.g., LAMOST, \gaia DR3 and beyond) that provide stellar information
for most of the \ik planet search targets.  When updating stellar
parameters, care must be exercised to update derived quantiles
self-consistently.  For example, the measured transit epoch, depth,
and duration do not depend on the stellar properties, but the inferred
planet size, semimajor axis, incident flux, and orbital inclination
would need to be updated to be consistent with the alternative set of
stellar properties.

While we recommend that the selection of planet candidates be based on
DR25 data products due to their automated detection and vetting process,
the planetary parameters from DR25 can be improved upon while
still maintaining a nearly homogeneous analysis, e.g., by using those values listed in the abovementioned columns of our Table \ref{tab:planetcatalog}.  In particular, the
DR25 planet properties table was based on a ``best fit" and did not
make use of MCMC simulations to characterize the uncertainties in
planetary properties.  While MCMC posterior samples were provided for
all DR25 planet candidates, these are not the basis for the catalog
values.  Therefore, statistical analyses can likely be improved upon
by updating the planet parameters with information from the MCMC
chains.   The MCMC posterior samples provided herein feature
important improvements that are advantageous for occurrence rates
studies.  First, our MCMC posterior samples correct a
bug that caused a biased distribution of impact parameters in the
previously released MCMC posteriors.  Additionally, our
results are derived from simultaneously fitting all the identified
planets in system, rather than by iteratively fitting one planet at a
time and masking out observations near transits of previous planets.
Therefore, we expect that the precision and accuracy of our MCMC
posterior samples represent an improvement on those originally
provided with \ik DR25.  Occurrence rate studies may thus
choose to update measured parameters (e.g., transit depths, durations,
impact parameters) with the results from this study.


\section{Supplemental Catalog of Non-transiting Planets}\label{sec:App_nontransiting}

The planet catalog presented in Table \ref{tab:planetcatalog} does not include circumbinary planets (CBPs) found by \ikt, nor does it list photometrically-identified non-transiting planets, nor non-transiting planets found around stars known to also host transiting planets found by \ikt.  For completeness, we provide references to lists of the first two classes of PCs and a tabulation of non-transiting companions to transiting \ik planets in this Appendix.  

Circumbinary transiting planets are searched for and analyzed quite differently from \ik planets that orbit around just one star (whether or not said star has more distant stellar companions).  Table 1 of \cite{Martin:2022} summarizes the properties of all 12 confirmed \ik CBPs. \cite{Welsh:2019}  identifies one additional candidate transiting \ik CBP.  Only one \ik multi-planet CBP system is known.

Phase variation photometry has been used to identify non-transiting hot jupiter candidates around stars that do not have transiting planet candidates.  A few of these candidates have been confirmed via radial velocity observations. See \cite{Lillo-Box:2021} and references therein for lists of these objects. 

Table \ref{tab:nontransiting} lists non-transiting \ik planets found from TTVs and RVs. Only planets with at least moderately well-constrained orbital periods are included;  planet candidates with poorly-constrained periods (from multiple possible TTV solutions or just a lower bound from RV data) are omitted.  All of the listed planets are in multiple planet (although not necessarily multi-transiting) systems, since the detection of one or more transiting planet(s) motivated further study.
Note that KOI 1442.10 is quite massive, and may be above the giant planet/brown dwarf boundary.

\begin{table}[!hbt]
%\small%footnotesize
%\fontsize{10pt}{20pt}
%\fontsize{5pt}{10pt}
    \centering
    \begin{tabular}{|c|c||c|c||c|c|}
        \multicolumn{6}{c}{\textbf{LIST OF NON-TRANSITING \ik PLANETS}}\\
        \hline
        Source & Method & KOI & Kepler- &  $P$ (days) &$M_p$ or $M_p \sin{i}$ (M$_\oplus$)  \\
        \hline
Y18 & RV  & 3 & 3~c & $3407^{+360}_{-190}$ & $507^{+30}_{-27}$ \\
%M14  & RV & 69 & -- & $>1460$    &  $>954$  \\ %93~c
B16  & RV  & 70 & 20~g &  $34.940^{+0.038}_{-0.035}$ &  $19.96^{+3.08}_{-3.61}$ \\ 
M17 & TTV+RV & 84 & 19~c & $28.731^{+0.012}_{-0.005}$ & 13.1 $\pm$ 2.7 \\
M17 & TTV+RV & 84 & 19~d & $62.95^{+0.04}_{-0.30}$ & $22.5^{+1.2}_{-5.6}$\\
M19 & RV & 85 & 65~e & $258.8^{+1.5}_{-1.3}$ & $260^{+200}_{-50}$\\
M14  & RV  & 104 & 94~c  &  820.3 $\pm$ 3  &  3126 $\pm$ 200   \\ 
W20 & TTV+RV & 142 & 88~c & $22.2649 \pm 0.0007$ & $214 \pm 5$\\
W20 & RV & 142 & 88~d & $1403 \pm 14$ & $965 \pm 44$\\
M14  & RV  & 148 & 48~e &  982 $\pm$ 8 &  657 $\pm$ 25   \\ 
E14 & RV & 214 & 424~c & 223 & 2215 \\
%N14 & TTV & 227 & -- & $\sim11.8$ or $\sim26.6$ or $\sim35.6$ &  \\
M19  & RV  & 244 & 25~d &  $122.4^{+0.8}_{-0.7}$ &  72 $\pm$ 10   \\ 
B23  & RV  & 246 & 68~d &  $632.62 \pm 1.03$ &  238 $\pm$ 5  \\ 
B23  & RV  & 246 & 68~e &  $3455^{+348}_{-169}$ &  86 $\pm$ 10  \\ 
B23  & RV  & 273 & 454~c &  $524.19 \pm 0.20$ &  1433 $\pm$ 38  \\ 
B23  & RV  & 273 & 454~d &  $4073^{+399}_{-186}$ &  $734 ^{+86}_{-51}$  \\ 
%M14  & RV  & 292 & 97~c &  $>789$ &  $>344$   \\ 
%N14 & TTV & 319 & -- & $\sim80$ or $\sim120$ & \\
%M16  & TTV & 372 & 539~c & $>1000$ & $760 \pm 380$ \\
F19 & TTV & 448 & 159~d & $88.73^{+0.60}_{-0.05}$ & $121^{+5}_{-4}$\\
S17 & TTV & 872 & 46~c & $57.325^{+0.116}_{-0.098}$ &$115 \pm 5$\\
Fr19 & TTV & 880  & 82~f & 75.732 $\pm$ 0.012 & 20.9 $\pm$ 1.0  \\
N14 & TTV & 884 & 247~e & 60.05 $\pm$ 0.1 & $850 \pm 200$\\
O16  & RV  & 1241 & 56~d & $1002 \pm 5$ & $1784 \pm 120$\\
Q15 & RV & 1299 & 432~c & $406.2^{+3.9}_{-2.5}$ & $772^{+70}_{-76}$\\
M14  & RV  & 1442 & 407~c &  3000 $\pm$ 500 &  4000 $\pm$ 2000 \\ 
A18 & RV+TTV & 1474 & 419~c &  $673.35 \pm 0.84$ & 2432 $\pm$ 86\\
S19 & TTV & 1781 & 411~e & $31.509728 \pm 0.000085$ & $10.8 \pm 1.1$\\
 \hline
    \end{tabular}
    \caption{References (sources of the parameters reported herein, not necessarily the discovery paper): Y18 = \cite{Yee:2018}, B16 = \cite{Buchhave:2016}, M17 = \cite{Malavolta:2017}, M19 = \cite{Mills:2019}, M14 = \cite{Marcy:2014}, W20 = \citet{Weiss:2020}, %N13 = \cite{Nesvorny:2013}, 
    E14 = \cite{Endl:2014}, B23 = \cite{Bonomo:2023}, %G16 = \cite{Gettel:2016}, %M16 = Mancini, 
    F19 = \cite{Fox:2019}; substantially longer period and larger mass solutions for this planet are consistent with the data, but all of them both provide poorer fits to the TTV data and are less likely a priori based on planetary demographics, S17 = \cite{Saad:2017}, Fr19 = \cite{Freudenthal:2019}, N14 = \cite{Nesvorny:2014}%; all of these planets have multiple solutions, although the periods for the two solutions for the planet orbiting Kepler-247 differ by only a small amount and the quoted uncertainties encompass the 1$\sigma$ uncertainties of both solutions
    , O16 = \cite{Otor:2016}, Q15 = \cite{Quinn:2015},  A18 = \cite{Almenara:2018}, S19 = \cite{Sun:2019}. Only the integer portions of KOI numbers are given; the \ik project's protocol was to use decimals beginning with .20 for non-transiting planet candidates, but the \cite{Marcy:2014} study used .10 and most other sources do not use any decimal numerical designators appended to KOI numbers for non-transiting planets.  Letters in most of the \ik names are those assigned by the authors, even if they didn't use \ik numbers; the 60 day period planet orbiting Kepler-247 is designated `e' despite it being referred to as KOI-884~c in the discovery paper because three transiting planets were announced prior to the publication of said paper. Planet masses are given for discoveries using TTVs; $M_p \sin{i}$ is listed for  RV  detections. %$^\dagger$
}
    \label{tab:nontransiting}
\end{table}  


The orbital periods of most non-transiting planets found by radial velocity measurements are not known to high precision, and the ability to detect non-transiting planets from TTVs strongly depends on period ratios, leading to a biased sample. Also, radii of non-transiting \ik planets have not been measured. Thus, we don't  use any of the planets listed in Table \ref{tab:nontransiting} %and  \ref{tab:cbp} 
for our statistical studies, even when computing the multiplicity of the systems hosting their sibling transiting planets, nor are they represented in any of the figures within this article.




\begin{thebibliography}{}

\bibitem[Agol et al.(2005)]{Agol:2005} Agol, E., Steffen, J.,  Sari, R., \&  Clarkson, W.\ 2005, \mnras, 359, 567

\bibitem[Almenara et al.(2018)]{Almenara:2018}
Almenara, J.~M.,  D{\'\i}az, R.~F.,  H{\'e}brard, G., et al.\ 2018, \aap, 615, 90

\bibitem[Armstrong et al.(2021)]{Armstrong:2021} Armstrong, D.~J. Gamper, J., \& Damoulas, T.\ 2021, \mnras, 504, 5327

\bibitem[Barclay et al.(2013)]{Barclay:2013} Barclay, T., Rowe, J.~F. Lissauer, J.~J.\ 2013, Nature, 494, 452

\bibitem[Bartram et al.(2021)]{Bartram:2021} Bartram, P., Wittig, A., Lissauer, J.~J., Gavino, S. \& Urrutxua, H.~2021.   \mnras, 506, 6181

%\bibitem[Batalha et al.(2011)]{Batalha:2011} Batalha, N.~M., et al.\ 2011, \apj, 729, 27 

\bibitem[Batalha et al.(2013)]{Batalha:2013} Batalha, N.~M., Rowe, J.~F.,  Bryson, S.~T., et al.\ 2013, \apjs, 204, 24 

%\bibitem[Beaug{\'e} \& Nesvorn{\'y}(2013)]{Beauge:2013} Beaug{\'e}, C.~\& Nesvorn{\'y}, D.~2013, \apj, 763, 12

%\bibitem[Bellman(1957)]{Bellman:1957}
%Bellman, R. 1957, Dynamic Programming, Princeton University Press, Princeton, N. J.

%\bibitem[Bellman(1961)]{Bellman:1961}
%Bellman, R. 1961, 
%Commun. ACM, 4, 284
\bibitem[Bedell et al.(2017)]{Bedell:2017} Bedell, M., Bean, J.~L., Mel{\'e}ndez, J., et al.~2017, \apj, 839, 94

\bibitem[Berger et al.(2020)]{Berger:2020a} Berger, T.~A., Huber, D., van Saders, J.~L., Gaidos, E., Tayar, J., \& Kraus, A.~L.~2020, \aj, 159, 280

%\bibitem[Berger et al.(2020b)]{Berger:2020b} Berger, T.~A., Huber, D., Gaidos, E., van Saders, J.~L., Gaidos, E., \& Weiss, L.~M.~2020, \aj, 160, 108

\bibitem[Bonomo et al.(2023)]{Bonomo:2023}
Bonomo, A.~S., Dumusque, X., Massa, A., et al.~2023, \aap, 677, 33

\bibitem[Borucki et al.(2011a)]{Borucki:2011a}
Borucki, W. J., Koch, D., Basri, G., et al.~2011a,
\apj, 728, 117

\bibitem[Borucki et al.(2011b)]{Borucki:2011b}
Borucki, W.~J., Koch, D.~G., Basri, G., et al.~2011b,
\apj, 736, 19

%\bibitem[Borucki et al.(2013)]{Borucki:2013} Borucki, W. J., et al.~2013, Science 340, 587 

%\bibitem[Borucki et al.(2012)]{Borucki:2012}
%Borucki, W. J., et al.~2012,
%\apj, 736, 19

%\bibitem[Brown et al.(2011)]{Brown:2011} Brown, T. M., Latham, D. W., Everett, M. R., \& Esquerdo, G. A.~2011, \aj, 142, 112

%\bibitem[Bryson et al.(2013)]{Bryson:2013} Bryson, S., et al.~2013, PASP 125, 889

\bibitem[Bryson et al.(2017)]{Bryson:2017} Bryson, S. et al.~2017, Kepler Science Document KSCI-19093-003, id. 12. Edited by Michael R. Haas and Natalie M. Batalha

\bibitem[Bryson et al.(2021)]{Bryson:2021} Bryson, S., Flynn, K.,  Hanna, H.,  Green, T.,  Coughlin, J.~L.  \&  Kunimoto, M.~2021, PASP 133, 104401

\bibitem[Buchhave et al.(2016)]{Buchhave:2016} Buchhave, L. A., Dressing, C.~D., Dumusque, X., et al.~2016, \aj, 152, 160

\bibitem[Burke et al.(2014)]{Burke:2014}
Burke, C.~J., Bryson, S.~T., Mullally, F., et al.~2014,  \apjs, 210, 19

\bibitem[Burke et al.(2019)]{Burke:2019}
Burke, C.~J., Mullally,~F., Thompson, S.~E., Coughlin, J.~L. \& Rowe, J.~F.~2019, \aj, 157, 4

\bibitem[Caceres et al.(2019)]{Caceres:2019} Caceres, G.~A., Feigelson, E.~D.,  Jogesh B.~G.,  Bahamonde, N.,  Christen, A., Bertin, K.,  Meza, C., \& Cur{\'e}, M.~2019, \aj, 158, 58

\bibitem[Ca{\~n}as et al.(2022)]{Canas:2022} Ca{\~n}as, C.~I., Mahadevan, S., Cochran, W.~D., et al.~2022, \aj, 163, 3

\bibitem[Carmichael(2023)]{Carmichael:2023}
Carmichael, T.~W.~2023, \mnras, 519, 5177

\bibitem[Carmichael et al.(2019)]{Carmichael:2019}
Carmichael, T.~W., Latham, D.~W., \& Vanderburg, A.~M.~2019, \aj, 158, 38

\bibitem[Carter et al.(2011)]{Carter:2011} Carter, J.~A., Fabrycky, D.~C., Ragozzine, D., et al.~2011, Science, 331, 562

\bibitem[Carter et al.(2012)]{Carter:2012} Carter, J.~A., Agol, E., Chaplin, W.~J., et al.~2012, Science, 337, 566

\bibitem[Carter \& Agol(2013)]{Carter:2013} Carter, J.~A., Agol, E.~2013, \apj, 765, 132

\bibitem[Cartier  et al.(2015)]{Cartier:2015} Cart1er, K.~M.~S., Gilliland, R.~L.,  Wright, J.~T., \& Ciardi, D.~R.~2015, \apj, 804, 97

%\bibitem[Chabrier et al.(2000)]{Chabrier:2000} Chabrier, G., Baraffe,  I., Allard, F., Hauschildt, P.\ 2000. \apj, 542, 464

\bibitem[Christiansen et al.(2015)]{Christiansen:2015} Christiansen, J.~L., Clarke, B.~D., Burke, C.~J., et al.~2015, \apj, 810, 95

\bibitem[Ciardi et al.(2013)]{Ciardi:2013} Ciardi, D.~R., Fabrycky, D.~C., Ford, E.~B., et al.~2013, \apj, 763, 41 

\bibitem[Claret \& Bloemen(2011)]{Claret:2011} Claret, A. \& Bloemen, S.~2011, \aap, 529, 75

%\bibitem[Coughlin et al.(2013)]{Coughlin:2013} Coughlin, J., et al.~2013, in preparation

\bibitem[Coughlin et al.(2016)]{Coughlin:2016}  Coughlin, J. L., Mullally, F., Thompson, S. E., et al.~2016, \apjs, 224, 12

\bibitem[Dai et al. (2018)]{Dai:2018} Dai, F., Masuda, K., \& Winn, J.~2018, 864, 38 

\bibitem[Dawson \& Johnson(2012)]{DawsonJohnson:2012} Dawson, R.~I. \& Johnson, J.~A.\ 2012, \apj, 756, 122 %doi:10.1088/0004-637X/756/2/122

\bibitem[Dawson et al.(2012)]{Dawson:2012} Dawson, R.~I.,  Johnson, J.~A., Morton, T.~D.~2012,  \apj, 761, 163

\bibitem[Deck et al.(2012)]{Deck:2012} Deck, K.~M., Holman, M.~J., Agol, E., Carter, J.~A., Lissauer, J.~J., Ragozzine, D., \& Winn, J.~N.~2012,  \apj, 755, 21

%Desert et al.(2012)]{Desert:2012} Desert, J.-M. et al.~2012, in preparation

%\bibitem[Dotter et al.(2008)]{Dotter:2008} Dotter, A., Chaboyer, B., Jevremovic, D., Kostov, V., Baron, E., \& Ferguson, W.~2008, \apjs, 178, 89

%\bibitem[Doyle et al.(2011)]{Doyle:2011} Doyle, L. R., et al.~2011, Science, 333, 1602
\bibitem[Endl et al.(2014)]{Endl:2014}
Endl, M.  Caldwell, D.~A.,  Barclay, T., et al.~2014, \apj, 795, 151

\bibitem[Fabrycky et al.(2014)]{Fabrycky:2014} Fabrycky, D.~C., Lissauer, J.~J., Ragozzine, D.,~et al.~2014, \apj, 790, 146 

\bibitem[Ford(2005)]{Ford:2005} Ford, E.~B.~2005, \aj, 129, 1706

\bibitem[Ford et al.(2008)]{Ford:2008} Ford, E.~B., Quinn, S.~N., \& Veras, D.\ 2008, \apj, 678, 1407 %doi:10.1086/587046

\bibitem[Ford \& Rasio(2008)]{FordRasio:2008} Ford, E.~B. \& Rasio, F.~A.\ 2008, \apj, 686, 621 %doi:10.1086/590926

\bibitem[Fox \& Wiegert(2019)]{Fox:2019} Fox, C., \& Wiegert, P.~2019, \mnras, 482, 639 

\bibitem[Freudenthal et al.(2019)]{Freudenthal:2019} Freudenthal, J., von Essen, C., Ofir, A. et al.,~2019, \aap, 628, 108

\bibitem[Fulton \& Petigura(2018)]{Fulton:2018} Fulton, B.~J. \& Petigura, E.~A.~2018, \aj, 156, 264

\bibitem[Furlan et al. (2017)]{Furlan:2017}
Furlan, E., Ciardi, D.~R., Everett, M.~E., et al.~ 2017, \aj, 153, 71

%\bibitem[Fressin et al.(2011)]{Fressin:2011} Fressin, F., et al.\ 2011, \apjs, 197, 5

%\bibitem[Fressin et al.(2012)]{Fressin:2012} Fressin, F., et al.\ 2012, Nature, 482, 195
 
\bibitem[Fressin et al.(2013)]{Fressin:2013} Fressin, F., Torres, G., Charbonneau, D., Bryson, S.~T., et al.\ 2013, \apj, 766, 81

\bibitem[Gaia Collaboration et al.(2018)]{Gaia:2018}
Gaia Collaboration, Brown, A. G. A., Vallenari, A., et al.~2018,
\aap, 616, A1%, doi: 10.1051/0004-6361/201833051749, 15

\bibitem[Garc{\'\i}a-Melendo \& L{\'o}pez-Morales (2011)]{Garcia:2011} Garc{\'\i}a-Melendo, E., L{\'o}pez-Morales, M.~2011, \mnras, 417, L16

\bibitem[Ghezzi et al.(2021)]{Ghezzi:2021} Ghezzi, L., Martinez, C.~F., Wilson, R.~F.,  Cunha, K., Smith, V.~V. \& Majewski, S.~R.~2021, \apj, 920, 19

\bibitem[Gilliland  et al.(2015)]{Gilliland:2015} Gilliland, R.~L., Cart1er, K.~M.~S., Adams, E.~R.,  Ciardi, D.~R., Kalas, P., \&  Wright, J.~T., ~2015, \aj, 149, 24

%\bibitem[Gettel et al.(2016)]{Gettel:2016} Gettel, S., Charbonneau, D., Dressing, C.~D. et al.~2016, \apj, 816, 95 

\bibitem[Gilbert  et al.(2022)]{Gilbert:2022} Gilbert, G.~J. MacDougall, M.~G., \& Petigura, E.~A.~2022, \aj, 164, 92

\bibitem[Gladman (1993)]{Gladman:1993} Gladman, B., 1993, \icarus, 106, 247

\bibitem[Gough (2009)]{Gough:2009} Gough, B.~2009, GNU Scientific Library Reference Manual 3rd edition, Network Theory Ltd.

\bibitem[{Gratia \& Lissauer(2021)}]{Gratia:2021} Gratia, P.~\& Lissauer, J.~J.~2021,  \icarus,  358, 114038

\bibitem[Gregory(2011)]{Gregory:2011} Gregory, P.~C.~2011, \mnras, 410, 94

\bibitem[Hadden \& Lithwick(2014)]{Hadden:2014} Hadden, S., \& Lithwick, Y.~2014, \apj, 787, 80

\bibitem[He et al.(2019)]{He:2019} He, M.~Y., Ford, E.~B., \& Ragozzine, D.\ 2019, \mnras, 490, 4575 %doi:10.1093/mnras/stz2869

\bibitem[He et al.(2020)]{He:2020} He, M.~Y., Ford, E.~B., Ragozzine, D., et al.~2020, \aj, 160, 276 %doi:10.3847/1538-3881/abba18

\bibitem[He et al.(2021)]{He:2021} He, M.~Y., Ford, E.~B., Ragozzine, D. \& Ashby, K.,~2019, \aj, 158, 109


\bibitem[Hsu et al.(2019)]{Hsu:2019} Hsu, D.~C., Ford, E.~B. \& Ragozzine, D.,~2021, \aj, 162, 216

\bibitem[Holczer et al.(2016)]{Holczer:2016} Holczer, T., Mazeh, T., Nachmani, G., et al.\ 2016, \apjs, 225, 9

\bibitem[Holman \& Murray(2005)]{Holman:2005} Holman, M. J., \& Murray, N. W.~2005, Science, 307, 1288 

%\bibitem[Holman et al.(2010)]{Holman:2010} Holman, M. J., et al.~2010, Science, 330, 51

%\bibitem[Horch  et al.(2011)]{Horch:2011} Horch, E. P., van Altena, W. F., Howell, S. B., Sherry, W. H., \& Ciardi, D. R.~2011, AJ, 141, 181

%\bibitem[Howard et al.(2013)]{Howard:2013} Howard, A. W., et al.~2013, in prep

%\bibitem[Howell et al.(2011)]{Howell:2011} Howell, S. B., Everett, M. E., Sherry, W., Horch, E. \& Ciardi, D. M.~2011,\aj, 142, 19

%\bibitem[Jackson et al.(2010)]{Jackson:2010} Jackson, B., Scargle, J.,  Cusanza, C., Barnes, D.,
%Kanygin, D., Sarmiento, R., Subramaniam, S.,
%\& Chuang, T.-W.~2010, Optimal Partitioning of Data in Higher Dimensions,
%2010 Conference on Intelligent Data Understanding, 
%\verb+https://c3.nasa.gov/dashlink/static/media/publication/Paper_8_.pdf+


\bibitem[Johansen et al.(2012)]{Johansen:2012} Johansen, A., Davies, M.~B., Church, R.~P., et al.\ 2012, \apj, 758, 39 %doi:10.1088/0004-637X/758/1/39

\bibitem[Jontof-Hutter et al.(2016)]{Jontof:2016} Jontof-Hutter, D., Ford, E.~B., Rowe, J.~F., Lissauer, J.~J., Fabrycky, D.~C.~et al.~2016,  \apj, 820, 39

\bibitem[Jontof-Hutter et al.(2021)]{Jontof:2021} Jontof-Hutter, D., Wolfgang, A., Ford, E.~B., Lissauer, J.~J., Fabrycky, D.~C.~\& Rowe, J.~F.,~2021,  \aj, 161, 246 

\bibitem[Kane et al.(2019)]{Kane:2019} Kane, M., Ragozzine, D., Flowers, X., Holczer, T., Mazeh, T. \& Relles, H. M.~2013,  \aj, 157, 171

\bibitem[Kawahara \& Masuda (2019)]{Kawahara:2019} Kawahara, H. \& Masuda, K.~2019, \aj, 157, 218

\bibitem[Kipping \& Sandford (2016)]{Kipping:2016} Kipping, D.~M., Sandford, E.~2017, \mnras, 463, 1323

\bibitem[Kopparapu et al.(2014)]{Kopparapu:2014} Kopparapu, R.~K., Ramirez, R.~M., SchottelKotte, J., Kasting, J.~F., Domagal-Goldman, S. \& Eymet, V., ApJL, 787, L29

\bibitem[Latham et al.(2011)]{Latham:2011} Latham, D.~W., et al.\ 2011, \apjl, 732, 24

\bibitem[Kane et al.(2012)]{Kane:2012} Kane, S.~R., Ciardi, D.~R., Gelino, D.~M., et al.\ 2012, \mnras, 425, 757. %doi:10.1111/j.1365-2966.2012.21627.x

\bibitem[Kite et al.(2019)]{Kite:2019} Kite, E.~S., Fegley, B., Jr., Schaefer, L. \& Ford, E.~B.
\apjl, 887, L33

\bibitem[Kov{\'a}cs et al.(2002)]{Kovacs:2002} Kov{\'a}cs, G.,  {Zucker}, S. \& {Mazeh}, T.~2002 \aap, 391, 369

\bibitem[Laskar(2000)]{Laskar:2000} Laskar, J.\ 2000, \prl, 84, 3240 %doi:10.1103/PhysRevLett.84.3240

\bibitem[Lillo-Box et al.(2021)]{Lillo-Box:2021} Lillo-Box, J., Millholland, S.~\& Laughlin, G.~2021, \aap, 654, 9 

\bibitem[Lissauer(1995)]{Lissauer:1995}
Lissauer, J.~J., 1995,
\icarus, 114, 217

\bibitem[Lissauer \& Gavino(2021)]{Lissauer:2021} Lissauer, J.~J.~\& Gavino, S.~2021, \icarus, 364, 114470

\bibitem[Lissauer et al.(2011a)]{Lissauer:2011a}
Lissauer, J.~J., Fabrycky, D.~C., Ford, E.~B., et al.~2011a,
Nature, 470, 53

\bibitem[Lissauer et al.(2011b)]{Lissauer:2011b}
Lissauer, J.~J., Ragozzine, D., Fabrycky, D.~C., et al.~2011b, 
\apjs, 197, 8 

\bibitem[Lissauer et al.(2012)]{Lissauer:2012}
Lissauer, J.~J., Marcy, G.~W., Rowe, J.~F., et al.~2012,
\apj, 750, 112 

\bibitem[Lissauer et al.(2013)]{Lissauer:2013}
Lissauer, J.~J., Jontof-Hutter, D., Rowe, J.~F., et al.~2013,
\apj, 770, 131

\bibitem[Lissauer et al.(2014)]{Lissauer:2014}
Lissauer, J.~J., Marcy, G.~W., Bryson, S.~T., et al.~2014,
\apj, 784, 44

\bibitem[Lissauer et al.(2023)]{Lissauer:2023}
Lissauer, J.~J., Batalha, N.~M. \& Borucki, W.~J.~2023.
 In Protostars \& Planets VII, Astronomical Society of the Pacific Conference Series, 534, 
Eds.: Inutsuka, S. Aikawa, Y., Muto, T., Tomida, K. and Tamura, M., 839

%\bibitem[Lithwick et al.(2012)]{Lithwick:2012} Lithwick, Y., Xie, J. \& Wu, Y.~2012, \apj, 761, 122

%\bibitem[Lynden-Bell \& Pringle(1974)] {Lynden-Bell:1974} Lynden-Bell, D., \& Pringle, J.~E.~1974, \mnras, 168, 603

\bibitem[MacDonald et al.(2016)]{MacDonald:2016} MacDonald, M.~G., Ragozzine, D., Fabrycky, D.~C., et al.~2016, \aj, 152, 105

\bibitem[MacDonald et al.(2020)]{MacDonald:2020} MacDonald, M.~G., Dawson, R.~I., Morrison, S.~J., et al.\ 2020, \apj, 891, 20. %doi:10.3847/1538-4357/ab6f04

\bibitem[MacDonald et al.(2021)]{MacDonald:2021} MacDonald, M.~G., Shakespeare, C.~J., Ragozzine, D., et al.~2021, \aj, 162, 114

\bibitem[Malavolta et al.(2017)]{Malavolta:2017} Malavolta, L., Borsato, L., Granata, V., et al.~2017, \aj, 153, 224

\bibitem[Mandel \& Agol(2002)]{Mandel:2002} Mandel, K. \& Agol, E.~2002, \apjl, 580, 171

\bibitem[Marcy et al.(2014)]{Marcy:2014} Marcy, G., Isaacson, H., Howard, A.~W., et al.~2014, \apjs, 210, 20

\bibitem[Martin \& Fitzmaurice(2022)]{Martin:2022} Martin, D.~V. \& Fitzmaurice, E.~2022, \mnras, 512, 602

\bibitem[Migaszewski et al.(2017)]{Migaszewski:2017} Migaszewski, C., Go{\'z}dziewski, K., \&  Panichi, F.~2017, \mnras, 465, 2366

\bibitem[Millholland et al.(2017)]{Millholland:2017} Millholland, S., Wang, S., \& Laughlin, G.\ 2017, \apjl, 849, L33 %doi:10.3847/2041-8213/aa9714

\bibitem[Mills \& Fabrycky(2017)]{Mills:2017a} Mills, S.~M., \& Fabrycky, D.~C.~2017, \aj, 153, 45

%\bibitem[Mills \& Fabrycky(2017)]{Mills:2017b} Mills, S.~M., \& Fabrycky, D.~C.~2017b, \apjl, 838, L11

\bibitem[Mills et al.(2016)]{Mills:2016} Mills, S.~M., Fabrycky, D.~C., Migaszewski, C., Ford, E.~B., Petigura, E., Isaacson, H.~2016, Nature,  533,  509

\bibitem[Mills et al.(2019)]{Mills:2019} Mills, S.~M., Howard, A.~W., Weiss, L.~M., et al.~2019, \aj, 157, 145

\bibitem[Moorhead et al.(2011)]{Moorhead:2011} Moorhead, A.~V., Ford, E.~B., Morehead, R.~C., et al.~2011, \apjs, 197, 1

\bibitem[More et al.(1980)]{More:1980} More, J.~J., Garbow, B.~S., Hillstrom, K.~E.~1980, User Guide for MINPACK-1: Argonne National Laboratory Report ANL-80-74

%\bibitem[{Morton}(2012)]{Morton:2012}
%{Morton}, T.~D.~2012. \apj, 761, 6

%\bibitem[Morton \& Johnson(2011)]{Morton:2011}
%Morton, T. D., \& Johnson, J. A.~2011, \apj, 738, 170

%\bibitem[Muirhead et al.(2012)]{Muirhead:2012a} Muirhead, P.~S., et al.~2012, \apj, 747, 144

%\bibitem[Munk et al.(2012)]{Muirhead:2012b} Muirhead, P.~S., Hamren, K., Schlawin, E., Rojas-Ayala, B., Covey, K.~R., \& Lloyd, J.~P.~2012, \apj, 750, L37

\bibitem[Mulders et al.(2018)]{Mulders:2018} Mulders, G.~D., Pascucci, I., Apai, D., et al.\ 2018, \aj, 156, 24 %doi:10.3847/1538-3881/aac5ea

\bibitem[Mullally et al.(2015)]{Mullally:2015} Mullally, F., Coughlin, J.~L., Thompson, S.~E., et al.~2015, \apjs, 217, 31

\bibitem[Mullally et al.(2018)]{Mullally:2018} Mullally, F., Thompson, S.~E., Coughlin, J.~L., Burke, C.~J. \& Rowe, J.~F.~2015, \aj, 155, 210

\bibitem[Newton(1687)]{Newton:1687} Newton, I., 1687, Philosophiae Naturalis Principia Mathematica, Streater, London

%Murray and Dermott (1999)]{Murray:1999} Murray, C.~D. and Dermott, S.F., Solar System Dynamics, 1999, Cambridge University Press, Cambridge, UK.

\bibitem[Nesvorn{\'y} et al.(2013)]{Nesvorny:2013} Nesvorn{\'y}, D., Kipping, D., Terrell, D., Hartman, J., Bakos, G.~{\'A}, Buchhave, L.~A.~2013, \apj, 777, 3

\bibitem[Nesvorn{\'y} et al.(2014)]{Nesvorny:2014} Nesvorn{\'y}, D., Kipping, D., Terrell, D., Feroz, F.~2014, \apj, 790, 31

%\bibitem[Orosz et al.(2019)]{Orosz:2019} Orosz, J. A., et al.~2019, \apj, 157, 174

\bibitem[Ofir et al.(2018)]{Ofir:2018} Ofir, A., Xie, J.-W., Jiang, C.-F., Sari, R., Aharonson, O.~2018, \apjs, 234, 9

\bibitem[Otor et al.(2016)]{Otor:2016} Otor, O.~J.,  Montet, B.~T.,  Johnson, J.~A., et al. ~2016, \aj, 152, 165

%\bibitem[Papaloizou (2015)]{Papaloizou:2015} Papaloizou, J. C. B. ~2015, IJAsB, 14, 291

%\bibitem[Papaloizou (2016)]{Papaloizou:2016} Papaloizou, J. C. B. ~2016, CeMDA, 126 157

\bibitem[Petigura (2020)]{Petigura:2020} Petigura, E.~A.~2020, \aj, 160, 89

\bibitem[Petigura et al.(2017)]{Petigura:2017} Petigura, E.~A., Howard, A.~W., Marcy, G,~W., et al. ~2017, \aj, 154, 107

\bibitem[Petit et al.(2020)]{Petit:2020} Petit, A.~C., Pichierri, G., Davies, M.~B. \& Johansen, A.~2020, \aap, 641, 176

\bibitem[Petrovich et al. (2019)]{Petrovich:2019} Petrovich, C., Deibert, E., \& Wu, Y.~2019, \aj, 157, 180
       
\bibitem[Plavchan et al.(2014)]{Plavchan:2014} Plavchan, P., Bilinski, C., \& Currie, T.\ 2014, \pasp, 126, 34 %doi:10.1086/674819

\bibitem[Prince \& Dormand (1981)]{PD:1981} Prince, P.J., Dormand, J.R. 1981, J. Comput. Appl. Math. 7, 67

\bibitem[{{Pu} \& {Wu}(2015)}]{Pu:2015}
{Pu}, B., \& {Wu}, Y.~2015, \apj, 807, 44

%\bibitem[Quillen(2011)]{Quillen:2011} Quillen, A.~C.~2011, \mnras, 418, 1043 

\bibitem[Quinn et al.(2015)]{Quinn:2015}
Quinn, S.~N., White, T.~R., Latham, D.~W., et al.~2015, \apj, 803, 49
%\bibitem[Ragozzine \& Holman(2010)]{Ragozzine:2010} Ragozzine, s.D., \& Holman, M.~J.\ 2010,  arXiv:1006.3727 

%\bibitem[Robin et al.(2003)]{Robin:2003} Robin, A.~C.., Reyl{\'e}, C., Derri{\`e}re, S. \& Picaud, S.~2003, \aap, 409, 523

\bibitem[Rein \& Tamayo(2015)]{Rein:2015} Rein, H., \& Tamayo, D.~2015, \mnras, 452, 376

\bibitem[Rowe  \& Thompson(2015)]{Rowe:2015b} Rowe, J.~F. \& Thompson, S.~E.~2015, arXiv:1504.00707

%\bibitem[Rivera \& {Lissauer}(2000)]{Rivera:2000} Rivera, E.~J.  \& Lissauer, J.~J.~2000, \apj, 530, 454

\bibitem[Rowe et al.(2014)]{Rowe:2014}
Rowe, J.~F., Bryson, S.~T., Marcy, G.~W., et al.~2014, \apj,  784, 45

\bibitem[Rowe et al.(2015)]{Rowe:2015}
Rowe, J.~F., Coughlin, J.~L., Antoci, V., et al.~2015, \apjs,  217, 16

\bibitem[Rowe (2016)]{Rowe:2016}
Rowe, J.F.~2016, Kepler: Kepler Transit Model CodebaseRelease., v.1.0, Zenodo, doi:10.5281/zenodo.60297

\bibitem[Saad-Olivera et al.(2017)]{Saad:2017} Saad-Olivera, X., Nesvorn{\'y}, D., Kipping, D.~M., \& Roig, F.~2017, \aj, 153, 198

%\bibitem[Saad-Olivera et al.(2019)]{Saad:2019} Saad-Olivera, X., Costa de Souza, A., Roig, F., \& Nesvorný, D.~2019, \mnras, 482, 4965

\bibitem[Sanchis-Ojeda et al.(2014)]{Sanchis-Ojeda:2014} {Sanchis-Ojeda}, R., Rappaport, S.,  Winn, J.~N.,  Kotson, M.~C.,  Levine, A., \&  El Mellah, I.~2014, \apj, 787, 47

\bibitem[Sandford et al.(2019)]{Sandford:2019} Sandford, E., Kipping, D., \& Collins, M.\ 2019, \mnras, 489, 3162 %doi:10.1093/mnras/stz2350

\bibitem[Santerne et al.(2016)]{Santerne:2016} Santerne, A., Moutou, C., Tsantaki, M., et al.~2016, \aap, 587, 64
%cargle et al.(2013)]{Scargle:2013}
%Scargle, J., Norris, J., Jackson, B. and Chiang, J.~2013
%Studies in Astronomical Time Series Analysis. VI. Bayesian Block Representations
%\apj, 764, 167

\bibitem[Schmitt et al.(2014)]{Schmitt:2014} {Schmitt}, J.~R., Agol, E.,  Deck, K.~M., et al.~2014, \apj, 795, 167

\bibitem[Schmitt et al.(2017)]{Schmitt:2017} {Schmitt}, J.~R., Jenkins, J.~M., Fischer, D.~A.~2017, \aj, 153, 180
 
\bibitem[Seager \& Mall{\'e}n-Ornelas (2003)]{Seager:2003} Seager, S., \& Mall{\'e}n-Ornelas, G.~2003, \apj, 585, 1038

\bibitem[Shallue \& Vanderburg (2018)]{Shallue:2018} Shallue, C.~J., \& Vanderburg, A.~2018, \aj, 155, 94

\bibitem[Spalding \& Batygin (2016)]{Spalding:2016} Spalding, C., \& Batygin, K.~2016, \apj, 830, 5

\bibitem[Steffen et al.(2012)]{Steffen:2012}
{Steffen}, J.~H., Ragozzine, D., Fabrycky, D.~C., et al.~2012, PNAS, 109, 7982

%\bibitem[Steffen et al.(2013)]{Steffen:2013a}
%{Steffen}, J. H., et al.~2013, MNRAS, 428, 1077

\bibitem[Steffen \& Farr(2013)]{Steffen:2013b}
{Steffen}, J. H. \& {Farr}, W.~M.~2013,  \apjl, 774, L12

%\bibitem[Steffen \& Huang(2015)]{Steffen:2015}
%{Steffen}, J.~H. \& {Huang}, J.~A.~2015,  \mnras, 448, 1956

\bibitem[Steffen \& Hwang (2016)]{Steffen:2016}
{Steffen}, J.~H. \& {Hwang}, J.~L.~2016,  \pnas, 113, 43, 12023

\bibitem[Stumpe et al.(2014)]{Stumpe:2014}
{Stumpe}, M.~C., Smith, J.~C., Van Cleve, J.~E., et al.~2014, \pasp, 126, 100

\bibitem[Sun et al.(2019)]{Sun:2019}
Sun, L., Ioannidis P., Gu, S., Schmitt, J. H.~M.~M.,  Wang, X., \& 
Kouwenhoven, M.~B.~N.~ 2019, \aap, 624, 15


\bibitem[Thompson et al.(2016)]{Thompson:2016}
Thompson, S.~E., Fraquelli, D,, Van Cleve, J.~E. and Caldwell, D.~A.~2016,
{\it Kepler Archive Manual}, Kepler Science Document KDMC-10008-006, id. 9. Edited by F.~Abney, D.~Sanderfer, M.~R. Haas, and S.~B. Howell 

\bibitem[Thompson et al.(2018)]{Thompson:2018}
Thompson, S.~E., Coughlin, J.~L.,  Hoffman, K., et al.~2018,
\apjs, 235, 38 

%\bibitem[Torres et al.(2011)]{Torres:2011} Torres, G., et al.~2011, \apj, 727, 24

%\bibitem[Valenti \& Fischer(2005)]{Valenti:2005} Valenti, J. A. \& Fischer D. A.~2005, \apjs, 159, 141

\bibitem[Vanderburg et al.(2020)]{Vanderburg:2020}  Vanderburg, A., Rowden, P., Bryson, S., et al.~2020,  \apj, 893, 27

\bibitem[Van Cleve \& Caldwell(2016)]{vanCleve:2016} Van Cleve, J.~E. \& Caldwell, D.~A.~2016, {\it Kepler Science Document} KSCI-19033-002

\bibitem[Van Eylen \& Albrecht(2015)]{vanEylen:2015} Van Eylen, V. \& Albrecht, S.\ 2015, \apj, 808, 126 %doi:10.1088/0004-637X/808/2/126

\bibitem[Van Eylen et al.(2019)]{vanEylen:2019} Van Eylen, V., Albrecht, S., Huang, X., et al.\ 2019, \aj, 157, 61 %doi:10.3847/1538-3881/aaf22f

\bibitem[Virtanen et al.(2020)]{Virtanen:2020} Virtanen, P., Gommers, R., Oliphant, T.~E. et al.~2020, Nature Methods, 17(3), 261%-272

%\bibitem[Walther(2010)]{Walther:2010}
%Walther, G.~2010,
%Annals of Statistics, 38, 1010

\bibitem[Weiss et al.(2018)]{Weiss:2018} Weiss, L.~M., Marcy, G.~W., Petigura, E.~A., et al.\ 2018, \aj, 155, 48 %doi:10.3847/1538-3881/aa9ff6


\bibitem[Weiss et al.(2020)]{Weiss:2020}
Weiss, L.~M.,  Fabrycky, D.~C., Agol, E.,  Mills, S.~M., et al.~2020,  \aj, 159, 242
%Wold, Herman O. A. 1938,
%A Study in the Analysis of Stationary Time Series,
%Almqvist \& Wiksells: Uppsala

\bibitem[Welsh \& Orosz(2018)]{Welsh:2018} Welsh, W.~F., \& Orosz, J.~A.~2018, In: Deeg H., Belmonte J. (eds) Handbook of Exoplanets. Springer, p. 2749%-2768

\bibitem[Welsh(2019)]{Welsh:2019} Welsh, W.~F.~2019, \baas, 51, 6, 402.03

\bibitem[{Xie et al.(2016)}]{Xie:2016}
Xie, J.-W., Dong, S., Zhu, Z., et al.~2016, \pnas, 113, 11431

\bibitem[{Xie et al.(2014)}]{2014Xie}
Xie, J.-W., Wu, Y., Lithwick, Y.~2014, \apj, 789, 165

\bibitem[Yang et al.(2020)]{Yang:2020} Yang, J.-Y., Xie, J.-W., \& Zhou, J.-L.\ 2020, \aj, 159, 164. %doi:10.3847/1538-3881/ab7373


\bibitem[{Yee et al.(2018)}]{Yee:2018}
Yee, S.~W.,  Petigura, E.~A.,  Fulton, B.~J., et al.~2018, \aj, 155, 255

%\bibitem[{Zawadzki et a. (2022)}]{Zawadzki:2022}
%Zawadzki, B.,  Carrera, D., \& Ford, E.B.~2022, arXiv:2202.05342 


\bibitem[Zhu et al.(2018)]{Zhu:2018} Zhu, W., Petrovich, C., Wu, Y., et al.\ 2018, \apj, 860, 101 %doi:10.3847/1538-4357/aac6d5

\bibitem[Zink et al.(2019)]{Zink:2019} Zink, J.~K., Christiansen, J.~L., Hansen, B.~M.~S.,~2019, \mnras, 483, 4479



\end{thebibliography}
\end{document}
%Note: comment out previous line to compile Boneyard.

\section{Boneyard}


\input{ecc_discussion}
%\subsection{Eccentricities} 


\subsection{Items associated with Catalog}

In preparing this catalog and throughout our study, we take stellar parameters from - hope it's this easy, or at least simple enough that it isn't a subsubsection, but if not, need to explain.

Should we simply go with Berger, Huber et al.?  I suspect that we can't do this, as they seem to only have the ~95\% \ik targets that have good GAIA distances.  JR suggests that we use this single catalog, because the missing stars are typically bad ones unlikely to have PCs.  BH18 state "Of the 197,104 stars present in the KSPC, we
identify 195,713 Gaia DR2 source matches. Some of
these stars have poorly determined parallaxes/distances
($\sigma plx > 0.2$), low effective temperatures ($Teff < 3000$ K), plx extremely low logg ($< 0.1$), and/or missing J, H, and K photometry and/or errors. Excluding these stars reduces our final sample to 186,813 \ik stars." Some of these cuts indeed seem only to exclude problem targets, but the Teff cut does remove some real planets, so JL suspects that we'll need at least 2 star catalogs to incorporate Gaia distances where available and have properties for all PCs.
Another question is whether the Fulton \& Petigura catalog is better for GAIA + CKS stars, or more likely GAIA + CKS but no astroseismology.  The following ranking looks reasonable in terms of priority from my perspective, but way more cumbersome than I'd like:

1) asteroseimology-based catalog (I don't think any are available that use GAIA, unfortunately) - note, if none includes GAIA constraints, this might not be much better than (2)

*2) FP18 for GAIA + CKS - this really is better than BH18 according to Travis, as BH18 made efforts for unformity

*3) Berger et al. for GAIA

4) is CKS included in star catalog used w/DR25?  if so, is it best to use DR25 as the one source for non-GAIA because they incorporated all of the info available at that time? Otherwise should use CKS17

*5) DR25(cumulative)

WARNING: BH18 LACKs STELLAR MASSES, as does the NexSci Table - is that equivalent to the DR25 table?  FP18 INCLUDES MASSES, HOWEVER IT DOESN'T GIVE UNCERTAINTIES FOR EITHER M OR R
IN MOST CASES WE ONLY CARE ABOUT R*, IN STABILITY CALCULATIONS WE NEED M*, NOT SURE IF WE EVER NEED UNCERTAINTIES.

Stellar classes according to BH18: ~ ~ MS: 123,500; ~ SG: 44,500; ~ RG: 23,300. ~ ~ cool MS binaries: 4000 (probably included in MS number as well). The substantial fraction of sub-giants among \ik targets as classified by GAIA got me thinking about a few things: 
general statistics of planets orbiting sub-giants vs.~dwarfs, let's assume that subgiants have 2x R*, same T, same intrinsic noise as MS

targets: if no bias from KIC at given magnitude, then ratio is 8 $t_{sg}/t_{MS}$, where the t's are the amounts of time a star spends at this evolutionary state.  There probably is some bias against sg, but also there is a significant contribution from more massive stars that were hotter and disqualified when on MS, no?

b will be min-chisq value
Per,epoch,r/R*,tdepth,Tdur(t14),t23,radius are all modes from KDE.
KDE has sample width that is 1/100 of range from MCMC samples. 
MCMC has 5-sigma outliers removed

tranets/target: 2x for large enough planets 

depth: 1/4
duration: 2x (same b)
$1/4 < MES < 1/2$; perhaps MES ~ 1/sqrt8
for given planet sizes and S/N(need brighter *), more multis because $>$ solid angle, but fewer multis at larger planet sizes, so may balance out.



Darin looked more closely at the objects with periods ratios very near 1 and found: 
KOI-1101 as discussed previously. During the call, we felt like the most likely scenario was a confusion when comparing between catalogs when making the cumulative catalog and that these are the same object (same period, phase, etc.)
KOI-1545 looks like a F-M binary or a Hot Jupiter (nominal radius, assuming correct stellar parameters, is 12 Earth radii) and a secondary eclipse. The DV Summary albedo estimate is 3+/-1, so that favors F-M binary; this comes from a 200 ppm depth secondary eclipse even though the period is ~6 days. 
KOI-3300 is a emphemeris match to our friend KIC 7821010 (and its secondary eclipse)
KOI-3338 is a blended eclipsing binary. Depth changes that are mostly attributable to dilution (possibly variable in nature). 
KOI-3729 is a eclipsing binary with depth changes attributable to dilution/contamination.
note: the last 3 on the list all have both objects listed as FPs on NexSci, so probably won't make it to even our least picky catalog of PCs

Notes from EBF: \\
1. For all but the most inclusive catalog, I suggest that we cut stars identified as giants.  \\
2. For all but the most restrictive catalog, I suggest that we try to include cool stars, even if it takes a supplemental stellar input source.\\
4.  Since star mass enters only weakly into nearly all calculations, I think we could adopt a simple ZAMS mass-radius relationship \\

EBF:    If anyone is going to do tests where they draw planets, swap the order, ask if they're still detectable and the look at distributions of results (e.g., period ratios, size ratios, duration ratios), then please talk to EBF about how to make use of posterior samples before going far down that road.
(When accounting for detection bias against pairs, if you want to ask "would we detect this pair if reversed", that depends on having knowledge of the impact parameter.  In the past, these were badly biased.  I think someone (Thompson?  J Rowe?) claimed that they had calculated posterior distributions for impact parameters that didn't show evidence of bias.  Assuming that's correct, we should perform swaps drawing from the posterior distributions for b's rather than based on point estimates or assuming a Gaussian distribution.  )

Do we want to account for limb darkening by having a correction factor on depth that depends either on impact parameter or normalized duration?

Check debiased size ratios for small period ratios vs. large period ratios.

For size ratios, do we need to limit our study to DR25 to remove search biases?


Nov 21 catalog includes everything listed as PC in DR24 or DR25 or DR25supplement (which came out of Bryson FP group)

JR: Need to look at Giant planets (R>20Rjup).. first pass is to check source of radii - from mode/median of posterior or from chi-sq min? This was done for a sample of multis.  Has not been completed for the single planet population [may be a substantial amount of work.].  

 
On 11/15/18 11:09 AM, Jason Rowe wrote:
 Here are the problem cases:
KOI-82, six KOIs:
82.06 - Good candidate.  Was labeled as FP due to centroids, but the star is saturated.  Examination of DR25 DV report shows no obvious centroid shift when comparing in and out of transit observations 
KOI-408 five KOIs
 408.05  - has been a FP for a while.  Only appears transit like when you add up all the transits (Period=93.8 days).  The host star is quite active.  Nothing that definitely says this is a false alarm, but should be considered unreliable.  
 KOI-806 - five KOIs
 806.04
 806.05 are both troublesome.  Further examination shows that these are almost certainly false alarms.  JR needs to double check the TTVs.. if the TTVs are large then that would certainly contribute to the poor looking LCs.
KOI-1278 - five KOIs
 1278.05 - low S/N likely FA.  The MCMC calculations are not localized for period or TO.  This strongly suggests that this KOI is a false alarm. 
 KOI-2433 - seven KOs
2433.05 - been a FP for a while - this is a blend.  Shows secondary eclipse + centroid offset.
2433.07 - looks fine - good candidate.
 KOI-4355 - five KOIs
All the planets in this system are Crap with a capital C.  

Nomenclature for Unofficial Planet Candidates

The \ik project, and later the NASA Exoplanet Science Institute, assigned KOI (\ik Object of Interest) numbers to all planet candidates and various other objects.  For transiting planet candidates, KOI numbers take the form of an integer designating the target star followed by a decimal point and a two-digit number, ranging from 01 -- 07, identifying the possible transit signature on that target, e.g., KOI 0351.07 represents the seventh transit pattern identified around the star designated KOI-351.  With the completion of the \ik project, no additional KOI numbers are being assigned.  The largest KOI number that has been assigned is 8297.01, which represents a pattern designated a False Positive (FP); the largest KOI number for an active planet candidate in 8286.01, and the largest multi-planet candidate system is KOI-7685.  

We denote planet candidates in our catalog that lack official KOI numbers using a prescription that clearly identifies them as being distinct from the original list.  New candidates around \ik targets that are not included as official KOIs are denoted by a number that begins with 9 and has four digits prior to the decimal point, e.g., 9001.01.  % New candidates around \ik targets that are have KOI numbers for other transit candidates are denoted using the official integer KOI designation of the target followed by the number 9 and an integer indicating the transit pattern; e.g., 0351.98 indicates the eighth possible transit signature found around KOI-351 (despite it being the first one not in the official catalog).

% The thick gray curve shows all planet candidates.  The blue curve represents single planet candidates and the red curves represent planet candidates in multis. The dashed green curve shows planets in two-candidate system and the dashed orange curve shows planets in systems with three or more candidates.  The dotted blue curve shows innermost planets in multis, the dotted red curve outer planets in multis and the dotted purple curve shows planets in multis with neighbors on both sides.

% \begin{figure}
%     \centering
%     \subfloat[][Cumulative distributions without S/N cut and demotion]{
%     \includegraphics[width=0.48\textwidth]{June2020/CDF_Subset_Breakdown_Log.pdf}}
%     \hspace{1mm}
%     \subfloat[][Cumulative distribution with S/N cut and demotion]{
%     \includegraphics[width=0.48\textwidth]{June2020/CDF_Subset_Breakdown_SNR_Cut_15_Demote_Log.pdf}}
    
%     \caption{Normalized CDFs of orbital periods for all candidates (a) and candidates with S/N $>$ 12 where systems have their multiplicity demoted if low S/N candidates are excluded (b). We compare single and multis, as well as the innermost, outermost, and middle candidates of multis with an odd number of candidates. Data taken from \texttt{koiprops\_20200612.csv}
% % A previous version include KS statistic on these CDFs.
%     \label{fig:perdistlog}}
% \end{figure}

% The following is commented out but is consolidated into one graph with two panels beginning at line 884 above.
%\begin{figure}
%    \centering
%    \includegraphics[width=0.7\textwidth]{Jan2020/Figure13b.pdf}
%    \caption{Normalized cumulative period distributions equal to Figure %\ref{fig:perdist} in log scale. We performed a Kolmogorov-Smirnov (KS) test on the %singles and multis and found a $p$-value of order $10^{-12}$ indicating it is highly %unlikely that the two samples are drawn from the same population. The main %difference is that the CDF of singles ramps much more quickly for orbital periods %less than 4 days ($D_\text{max} = 0.xx6, \, p-\text{val} = xx \times 10^{-xx}$, %@Period = 4 days). Note: the curve for the singles not only crosses that for all %multis, but also innermost and outermost, which shows the profound paucity of %multis at very short or at long periods. Data taken from %\texttt{koiprops\_20191202.csv}}
%    \label{fig:perdistlog}
%\end{figure}
%\begin{figure}
%    \centering
%    \includegraphics[width=0.7\textwidth]{Jan2020/Figure13c.pdf}
%    \caption{Similar to Fig~\ref{fig:perdistlog} but only with candidates with S/N$\geq$ 15 with multis maintaining their status. Generated using \texttt{koiprops\_20191202.csv}.}
%    \label{fig:perdistlog_highS/N}
%\end{figure}
%\begin{figure}
%    \centering
 %   \includegraphics[width=0.7\textwidth]{Jan2020/Figure13d.pdf}
 %   \caption{Similar to Fig~\ref{fig:perdistlog_highS/N} but multis are now be demoted. Generated using \texttt{koiprops\_20191202.csv}.}
%    \label{fig:perdistlog_highS/N_demote}
%\end{figure}

% \includegraphics[width=0.7\textwidth]{img/NewPlots/CDF_Log3BinRadiiSplit.eps}

% The version below can go to boneyard.
%\begin{figure}
%    \centering
%    \includegraphics[width=0.9\linewidth]{img/NewPlots/CDF_Log3BinRadiiSplit_lt_1b.eps}
%    \caption{Similar to Figure \ref{fig:cdf_log3binradiisplit}, but planets with impact parameter $b \geq 1$ removed from the dataset. Total removed planet: 303 singles, 26 multis - similar fractions of large singles and large multis removed by cut}
%    \label{fig:cdf_log3binradiisplit_impact_lt_1}
%\end{figure}

\subsection{Items associated with Multis vs.~Singles} 

\subsection{Host Star Properties}\label{sec:hostsmultisvsingles}

\blue{plot normalized cumulative distribution of stellar host sizes for multis and singles.  multis could prefer larger hosts because wider spread in i allowed and larger stars tend to be brighter because of target selection, or could prefer smaller stars, because they allow detection of smaller planets and multis tend to have smaller planets.  note if we find that distributions of small planets are similar between singles and multis, then this could be normalized out by eliminating giants from the sample}

There are several reasons why higher multiplicity could correspond with smaller planets: 
dynamical packing

MES - more planets can be found in stars that are quieter, for which smaller planets are easier to find - to debias this, compare 2 highest MES planets 

\begin{figure}[!hbt]
    \centering
    \includegraphics[height=2.8in]{Current_Figures/Section_3/main_sequence_teff_6100_split_period.pdf}
    % \includegraphics[height=2.8in]{Current_Figures/Section_3/giants_teff_6100_split_period.pdf}
    \caption{Normalized cumulative distribution function of period with data split into sub-samples of main sequence stars (log$g$ $>$ 4.1) with effective temperature above vs.~below $6100$~K. Multis are represented by red curves whereas singles are shown in blue. Data taken from \texttt{koiprops\_20211110.csv}.\\
    2 sample KS-Test of hot multi vs.~hot single stars return: ($p$-value: $0.0001081394873246877$, $D$:$0.200276150063986$) while cool multi vs.~cool single stars return: ($p$-value: $1.062481069535881e-05$, $D$:$0.08746075286534706$).
    }
    \label{fig:cdf_perod_split_teff}
\end{figure}
% KS test of cool vs.~hot:
%     \begin{itemize}
%         \item[-] Singles: Null hypothesis accepted with results: $D_\text{max} = 0.0520, \, p-\text{val} = 0.1079$
%         \item[-] Multis: Null hypothesis rejected with results: $D_\text{max} = 0.1079, \, p-\text{val} = 0.0183$
%         \item[-] Combined singles and multis: Null hypothesis rejected with results: $D_\text{max} = 0.0624, \, p-\text{val} = 0.0060$
%     \end{itemize}

Both multis and singles have a larger fraction of planets at short periods around cool stars, but a closer look shows different patterns. For singles, the two distributions are similar up to $\sim 70$ days, with larger stars having more planets at longer periods. Multis have very few planet with $P < 1.6$ days around cool stars, but this paucity extends up to 4 days around hot stars; most of this deficit is made up for by 10 days, and all by 20 days.  Beyond 20 days, the cumulative distributions for hot and cool stars lie on top of one another.  [Points in this paragraph get repeated below, so some distilling is needed to this and the next few paragraphs.]

We see from Figure~\ref{fig:cdf_perod_split_teff} that, taking the singles and the planets in multis separately, a period distribution shifted to larger values is correlated with larger host star temperature.  To understand the origin of this feature, we first consider what effect geometric detection bias has.  The hotter stars are bigger and more massive, and $R_\star \propto M_\star^{0.8-1.0}$ has a steeper dependence on stellar mass than does $a \propto M_\star^{1/3}$ (at fixed $P$), so the geometric transit probability $\sim R_\star/a$ is larger, but that acts on all planets proportionally.  However, $R_\star/a$ is also sets the characteristic mutual inclination (in radians) that two planets have before their mutual transit probability declines.  Hence, provided planetary inclination statistics are the same across all stellar types, planets orbiting hot stars can stay in the multis sample to longer orbital period than those around cool stars.  This consideration may be sufficient to explain the multi curves (red) in figure~\ref{fig:cdf_perod_split_teff}, pending detailed study using forward models. 

In part, the dependence of singles on host star temperature also is helped by this explanation, as intrinsic multis of a given mutual inclination would drop into the single-transiting category at a shorter period for cool stars than hot stars. However, comparing within temperature bins, singles have a broader distribution of orbital periods than do planets in multis. The preponderance of singles around cool stars for log$_{10}P < 0.2$ ($P \lesssim 1.5$~days) cannot be due to this effect alone. The mutual inclinations between such short-period planets and their longer-period companions is observed to be larger than pairs of longer-period planets \citep{Dai:2018}. That architectural feature would explain the singleness of this short-period set, but would not explain its strong dependence on host star temperature. 

Another reason for the paucity of nearby short-period planets in multis could be that planets in such orbits may not survive long, and thus have reached these orbital zones via gradual migration over a significant fractional orbital distance or high-$e$ migration.  Either of these scenarios would preclude nearby companions.

\cite{Steffen:2016} compared the period \emph{vs.}~radius distribution of planets in multis and singles, and discovered the additional set of planets at short orbital period to be Earth-sized planets (as well as noticed that this population tended to be around cooler stars, the point we are discussing here). We thus considered whether it lies at the S/N boundary, and could only be detected around the small stars. 

To find a definitive answer to the cause of these trends, we recommend a forward-modelling approach, which takes into account (a) the intrinsic mutual inclinations including the excess for short-period inner planets \citep{Dai:2018}, (b) the size-period correlation \citep{Steffen:2016}, in addition to the standard ingredients: (c) the brightness, noise, and temperature distribution of the target stars.

If it \emph{is} astrophysical rather than due to biases, this stellar temperature dependence could speak to the origin of this hot earth population. \cite{Steffen:2016} suggested that it could rule out tidal consumption of the analogues around hot stars, as tidal dissipation is weaker for hot stars lacking convective envelopes (Winn et al. 2010, Dawson 2014, and references therein), and thus orbital decay of planets is less likely.  We would also add that the irradiation is much smaller, so it is not likely that these are remnant cores of very hot Jupiters \citep{Valsecchi:2015}, as hotter stars would more likely evaporate their nearby planets down to the core. It would also seem to cast into doubt the idea that the planets appear single because the $J_2$ of a mildly inclined star causes the hot Earth to precess away from the plane of outer planets \citep{Spalding:2016}.  Such an effect would also favor formation of this single population around the faster-spinning hot stars (Kraft 1967). ** but young cool low-mass stars rotate rapidly and are closer in luminosity to higher-mass stars**  Since \cite{Steffen:2016}, a promising additional mechanism for forming close-in, misaligned planets has been theorized: secular chaos followed by tidal migration \citep{Petrovich:2019}.  It does not have an explicit stellar temperature dependence, though. Determining why the coolest stars have more very hot super-Earths thus seems to be a new puzzle in need of theoretical work. 
\begin{figure}
    \centering
    % \includegraphics[width=0.7\textwidth]{img/NewPlots/CDF_ImpactParameter_gt_1.eps}
    \includegraphics[width=0.7\textwidth]{Current_Figures/Section_4/Figure5.pdf}
    \caption{TO BE REMOVED: The cumulative impact parameter distributions of various subsets of planet candidates observed by \ikt. Red curves represent planet candidates in multis, blue curves singles; small planets are shown by solid lines and large planets by dashed lines. All small planets have $b < 1$; 18 large singles and 6 large planets in multis have $b \geq 1$. Data taken from \texttt{koiprops\_20200130.csv} Note that this is the analog of the previous figure but using an older catalog}
    % \caption{The cumulative impact parameter distributions of various subsets of planet candidates observed by \ikt, normalized by the number of each class of planets with $b \leq 1$. Red curves represent planet candidates in multis,  blue curves single planet candidates; small planets are shown by solid lines and large planets by dashed lines. All small planets have $b<1.2$; xx large singles and yy large planets in multis have $b>1.2$ Data taken from \texttt{koiprops\_20190116.csv}. One expects that transiting planets should be uniformly distributed in impact parameter up to $b=1$ and drop off as a function of size at higher impact parameter, but the difficulty in detecting planets with shorter transits means that the slope of the curves should be a decreasing function of $b$ with the first derivative becoming increasing negative as b increases, and this dropoff to occur at smaller b for smaller planets.  However, the observed flattening in the curves at $b \lesssim0.1$ is far larger than can be explained by detection efficiency, and must be caused by errors that don't balance out because $b\geq 0$  for low values of b and faster fo KS-Test for Singles:=\{$p$-val $\approx 10^{-21}$, $D_\text{max}$ = 0.251556\}, Multis:=\{$p$-val = 0.9886, $D_\textrm{max} = 0.070632$\}. Upturn in curve for large singles at 0.9 implies that sizes are not reliable for $b>0.9$. (Add curve with disposition beginning with 'R', rejected because size and period make it likely a stellar companion)}
    \label{fig:cdfimpactparambgt1_old}
\end{figure}


\begin{table}[]
    \centering
    \begin{tabular}{|c|c|c|c|}
         \multicolumn{4}{c}{\textbf{SINGLES}}\\
         \hline
         Type & Total & $0.9 < b < 1$ & $b \geq 1$ \\
         \hline
         $R_p < 1.8$~R$_\oplus$  & 1040 & 102 & 2\\
         1.8R$_\oplus \leq R_p < 2.5$~R$_\oplus$ & 591 & 68 & 0\\
         2.5R$_\oplus \leq R_p$  & 928 & 117 & 37\\
         \hline
         \multicolumn{4}{c}{\textbf{MULTIS}}\\
         \hline
         Type & Total & $0.9 < b < 1$ & $b \geq 1$ \\
         \hline
         $R_p < 1.8$~R$_\oplus$  & 831 & 67 & 4\\
         1.8R$_\oplus \leq R_p < 2.5$~R$_\oplus$ & 453 & 57 & 1\\
         2.5R$_\oplus \leq R_p$  & 494 & 41 & 2\\
         \hline
    \end{tabular}
    \caption{Statistics for Figure \ref{fig:cdf_log3binradiisplit}. The third and fourth columns indicate how many candidates with high impact parameters are not included in the CDFs. Data taken from \texttt{koiprops\_20210503.csv}}
    \label{tab:SizeStats}
\end{table}

\begin{table}[]
    \centering
    \begin{tabular}{|c|c|c|c|}
         \multicolumn{4}{c}{\textbf{SINGLES}}\\
         \hline
         Type & Total & $0.9 < b < 1$ & $b \geq 1$ \\
         \hline
         $R_p < 1.8$~R$_\oplus$  & 1040 & 102 & 2\\
         1.8R$_\oplus \leq R_p < 4$~R$_\oplus$ & 1100 & 124 & 5\\
         4R$_\oplus \leq R_p$  & 419 & 61 & 32\\
         \hline
         \multicolumn{4}{c}{\textbf{MULTIS}}\\
         \hline
         Type & Total & $0.9 < b < 1$ & $b \geq 1$ \\
         \hline
         $R_p < 1.8$~R$_\oplus$  & 831 & 67 & 4\\
         1.8R$_\oplus \leq R_p < 4$~R$_\oplus$ & 823 & 87 & 1\\
         4R$_\oplus \leq R_p$  & 124 & 11 & 2\\
         \hline
    \end{tabular}
    \caption{Statistics for Figure \ref{fig:cdf_log3binradiisplit}. The third and fourth columns indicate how many candidates with high impact parameters are not included in the CDFs. Data taken from \texttt{koiprops\_20210503.csv}}
    \label{tab:SizeStats}
\end{table}

\begin{table}[]
    \centering
    \begin{tabular}{|c|c|c|c|}
         \multicolumn{4}{c}{\textbf{SINGLES}}\\
         \hline
         Type & Total & $0.9 < b < 1$ & $b \geq 1$ \\
         \hline
         $R_p < 1.8$~R$_\oplus$  & 1040 & 102 & 2\\
         1.8R$_\oplus \leq R_p < 5$~R$_\oplus$ & 1166 & 134 & 8\\
         5R$_\oplus \leq R_p$  & 353 & 51 & 29\\
         \hline
         \multicolumn{4}{c}{\textbf{MULTIS}}\\
         \hline
         Type & Total & $0.9 < b < 1$ & $b \geq 1$ \\
         \hline
         $R_p < 1.8$~R$_\oplus$  & 831 & 67 & 4\\
         1.8R$_\oplus \leq R_p < 5$~R$_\oplus$ & 856 & 89 & 2\\
         5R$_\oplus \leq R_p$  & 91 & 9 & 1\\
         \hline
    \end{tabular}
    \caption{Statistics for Figure \ref{fig:cdf_log3binradiisplit}. The third and fourth columns indicate how many candidates with high impact parameters are not included in the CDFs. Data taken from \texttt{koiprops\_20210503.csv}}
    \label{tab:SizeStats}
\end{table}


\begin{figure}
    \centering
    \includegraphics[width=0.7\textwidth]{2020-11/0.2-snr/snr_comparison_loglog.pdf}
    \caption{Number of singles (blue) and multis (red) as functions of S/N; the dashed red curve is for the lowest S/N planet in each multi. Each bin spans a factor of 1.** in S/N. are set to be equal in size in logspace, with $\log_{10}(\textrm{bin width}) \approx 0.04$ using \texttt{numpy.logspace(0.9, 3.2, 80)}. KOI-3681.01 has the largest S/N of any planet in a multi, 1244. Data taken from \texttt{koiprops\_20201106.csv}}
    \label{fig:snr_comparison_loglog}
\end{figure}

\begin{figure}
    \centering
    \includegraphics[width=0.7\textwidth]{2020-11/0.2-snr/snr_comparison_rp_cut.pdf}
    \caption{Number of singles (blue) and multis (red) as functions of S/N. Planets are filtered such that $R < 4R_\oplus$ and $2 \leq P < 50$. Maximum S/N in multis is 289.6 from KOI 246.01. The bins are set to be equal in size in logspace, with $\log_{10}(\textrm{bin width}) \approx 0.04$ using \texttt{numpy.logspace(0.9, 3.2, 80)}. Data taken from \texttt{koiprops\_20201106.csv}}
    \label{fig:snr_comparison_loglog}
\end{figure}

\begin{figure}
    \centering
    \includegraphics[width=0.7\textwidth]{2020-11/0.2-snr/SNR_stellar_param_cut.pdf}
    \caption{Number of singles (blue) and multis (red) as functions of S/N. Planets belonging to stars with the properties $t_\textrm{eff} < 4500$ \textbf{and} $\logg \geq 4.2$ are removed. Data taken from \texttt{koiprops\_20201106.csv}}
    \label{fig:snr_comparison_stellar_cut}
\end{figure}

\begin{figure}
    \centering
    % \includegraphics[width=0.7\textwidth]{img/September2019/S/N_check.eps}
    \includegraphics[width=0.7\textwidth]{Jan2020/SNR_0_40.pdf}
    \caption{Number of singles (blue) and multis (red) as functions of S/N for S/N~$<40$. Note that numbers generally increase as S/N drops until around 9, when numbers, especially of multis, drop. The drop is steeper below 5, but numbers are too small to get good stats on multis vs.~singles. Data taken from \texttt{koiprops\_20200130.csv} (Counting just the weakest candidate in each multi reduces the count as expected at high S/N (by factor $\sim$3, but reduces the count of low S/N detections by much less.)}
    \label{fig:S/N}
\end{figure}

\begin{figure}
    \centering
    % \includegraphics[width=0.7\textwidth]{img/September2019/S/N_40_500.eps}
    \includegraphics[width=0.7\textwidth]{Jan2020/SNR_40_500.pdf}
    \caption{Number of singles (blue) and multis (red) as functions of S/N for $40\leq$~S/N~$<500$. Binwidth = 10. Singles dominate at low and high S/N, but between 16 -- 120, multis are almost as common. Data taken from \texttt{koiprops\_20200130.csv}}
    \label{fig:S/N_high}
\end{figure}
\begin{figure}
\centering

\includegraphics[width=0.4\textwidth]{img/NewPlots/CDF_MultiSplit.eps}

\caption{Distribution  of the fraction of planets within specified multiplicity systems having orbital periods less than the specified value.  Periods of planets within multis are plotted together as well as separated into the number of transiting planets in the system. Systems with four and higher were combined to obtain a sample. We performed another KS-test of periods in systems with 2 against systems with 4+. The null hypothesis was accepted, i.e., we cannot say they are drawn from different distributions. KS-Test for 2v$4_+$:=\{$p$-val = 0.7053, $D_\textrm{max} = 0.042684$\}, 2v3:=\{$p$-val = 0.7198, $D_\textrm{max} = 0.038065$\}. Despite these high $p$-values, it looks like there are more long-period 2's than 3+'s- compute $p$-values 2v$3_+$ both for all periods and for just $P>20$ days. Data taken from \texttt{koiprops\_20190116.csv}}
\label{fig:period_multiplicities_split_old}
\end{figure}

%Figure \ref{fig:radius_multis_singles}  compares the cumulative size distributions of planets in singles and multis, with the curves normalized to unity at $R_p = 5$R$_\oplus$ to emphasize the comparison of the distributions of small and midsized planets.  As,  \ref{fig:large_radius_multiplicities_split} contrast the size distributions of planets in systems of varying multiplicity.  There are far more large planets in singles than in multis, but the three curves showing normalized CDFs of planets up to various sizes in systems with multiplicities ${\cal M}_2$, ${\cal M}_3$ and ${\cal M}_{4+}$ lie very close to one another.  While the curves for the ${\cal M}_{4+}$'s generally lie above those for the ${\cal M}_{2}$'s, suggesting an even greater concentration of small planets in systems of high multiplicity than in ${\cal M}_2$'s, the KS test does not reject the hypothesis that the curves were drawn from the same distribution.  Moreover, the curves for the ${\cal M}_3$'s cross both the curves for the ${\cal M}_2$'s and those for the ${\cal M}_{4+}$'s.  

\begin{figure}
    \centering
    \includegraphics[width=0.7\textwidth]{img/June2019/S/N_P_lt_2.eps}
    \caption{very few multis with $P<2$ [JL: the differences between this and the next 3 figures don't seem interesting - just more low S/N at long periods - should we just drop them and keep previous figure?]}
    \label{fig:S/N_p_lt_2}
\end{figure}

\begin{figure}
    \centering
    \includegraphics[width=0.7\textwidth]{img/June2019/S/N_P_gt_2_lt_10.eps}
    \caption{excess of low S/N singles, periods 2--10 days}
    \label{fig:S/N_p_gt_2_lt_10}
\end{figure}

\begin{figure}
    \centering
    \includegraphics[width=0.7\textwidth]{img/June2019/S/N_P_gt_10.eps}
    \caption{Caption}
    \label{fig:S/N_p_gt_10}
\end{figure}

\begin{figure}
    \centering
    \includegraphics[width=0.7\textwidth]{img/June2019/S/N_P_gt_100.eps}
    \caption{Caption}
    \label{fig:S/N_p_gt_100}
\end{figure}

% nnz(single.Period > 100 & single.S/N< 20) == 416
% nnz(multi.Period > 100 & multi.S/N< 20) == 48
% nnz(single.Period > 100 & single.S/N> 20) == 177
% nnz(multi.Period > 100 & multi.S/N> 20) == 69
\begin{figure}
\includegraphics[scale=0.9,angle=270]{single_multis.png}
\includegraphics[scale=0.5]{img/cdf_multi_single_out_in_normalized.eps}
\caption{The normalized cumulative period distributions of various subsets of planet candidates observed by \ikt. The thick gray curve shows all planet candidates.  The red curves represents single planet candidates and the black curves planet candidates in multis.The dashed green curve shows planets in two candidate system and the dashed orange curve shows planets in systems with three or more candidates.  The dotted blue curve shows innermost planets in multis, the dotted red curve outer planets in multis and the dotted purple curve shows planets in multis with neighbors on both sides.   The normalization includes all orbital periods, but the plot on the left only shows the cumulative distributions up to 40 days and the plot on the right, which is logarithmic in period, displays only periods between 30 and 300 days.  Each curve has been normalized by the total number of objects of its class that have been detected. To remove spurious detections of noise, we impose only count planet candidates with S/N~$>$~10, although PCs with lower S/N are still considered when determining the multiplicity of a system. For the purpose of producing this plot, a planet is considered to have TTVs if the three digit number in the TTV flag (column 19 of Table \ref{tab:planetcatalog}) is $\geq 7$. singles and multis with TTVs
Note the strong differences between the four curves for periodicities : 
(slope of the red curve in Figure \ref{fig:perdist}) monotonically decreases for periods above 0.5 days.  There are relatively few planet candidates with $P < 1$ day; the period distribution is relatively flat (the green curve has approximately constant slope) in the period range  $1 < P < 5$ days and then gradually flattens out (fewer transiting planet candidates per linear range in period) for larger periods. 
Make a normalized CDF of planet periods of Kepler planets.  
a. going outward from the inner most planet.
b. going inward from the outer most planet.
Do both graphs on a log scale. 
Include the following groups.
1. singles (in red)
2. multis (in black)
3. multis, inner of pair, outer of pair, dotted and dashed, both dark grey.
(This plot will update and extend Fig 1 of Lissauer 2014 Validation of Multis.) Note that both the Innermost and Outermost curves cross the Singles curve.}
% \label{fig:perdist}
\end{figure}

% \begin{figure}
%     \centering
%     \includegraphics[width=0.7\textwidth]{img/June2019/CDF_Radius_SvsM.eps}
%     \caption{Similar to Figure \ref{fig:radius_multiplicities_split}, but multis are combined into a single curve. Data taken from \texttt{koiprops\_20190610.csv} KS test for $R<4$; $p$-value=.223, KS test for $3<R<5$: KS test for $5<R$, normalized:}
%     \label{fig:large_x_radius_multiplicities_split}
% \end{figure}

%boneyare has {fig:period_multiplicities_split_old}
% \begin{figure}
%     \centering
%     \includegraphics[width=0.7\textwidth]{img/NewPlots/CDF_MultiSplit.eps}
%     \caption{Distribution  of the fraction of planets within specified multiplicity systems having orbital periods less than the specified value.  Periods of planets within multis are plotted together as well as separated into the number of transiting planets in the system. Systems with four and higher were combined to obtain a sample. We performed another KS-test of periods in systems with 2 against systems with 4+. The null hypothesis was accepted, i.e. we cannot say they are drawn from different distributions. KS-Test for 2v$4_+$:=\{$p$-val = 0.7053, $D_\textrm{max} = 0.042684$\}, 2v3:=\{$p$-val = 0.7198, $D_\textrm{max} = 0.038065$\}. Despite these high $p$-values, it looks like there are more long-period 2's than 3+'s- compute $p$-values 2v$3_+$ both for all periods and for just $P>20$ days. Data taken from \texttt{koiprops\_20190116.csv}}
%     \label{fig:period_multiplicities_split}
% \end{figure}

% \begin{figure}
%     \centering
%     \includegraphics[width=0.7\textwidth]{img/NewPlots/CDF_LogMultiSplit.eps}
%     \caption{Similar to Figure \ref{fig:period_multiplicities_split}, but in semilog-x. Data taken from \texttt{koiprops\_20190116.csv}}
%     \label{fig:log_multiplicities_split}
% \end{figure}
\begin{figure}
    \centering
    % \includegraphics[width=0.75\textwidth]{Current_Figures/Section_3/singles_and_multis_radii_cdf_5.pdf}
    %\includegraphics[width=0.45\textwidth]{Current_Figures/Section_3/singles_and_multis_radii_cdf_4,5.pdf}
    \includegraphics[width=0.75\textwidth]{Current_Figures/Section_3/singles_and_multis_radii_cdf_5.pdf}
    \caption{Normalized cumulative distribution function (CDF) of planetary radii of planets that have observed transiting companions (red) and those that do not (blue). We normalized the CDFs to unity at $R_p = 5$~R$_{\oplus}$. Candidates with S/N $< 12$, or $b + \sigma_+(b) \geq$ 0.95, or for which just one transit has been observed, are excluded from this radius distribution, although planets are considered to be in multis even if all of their companions fail to meet one or both of these cuts. This plot includes KOIs rejected because they are too large (dispositioned as R in Table \ref{tab:planetcatalog}). 
    Only the portion of the distribution with  $R_p < 16$~R$_{\oplus}$ is shown, although larger bodies that satisfy our criteria are included in the computation of the numbers given within brackets. Among candidates larger than this, 10 singles have status P, 36 have status R, 0 multis have status P, and 1 multi, which has just one PC companion, has status R. }% Data taken from \texttt{koiprops\_20220127.csv}}
    \label{fig:radius_multis_singles}
\end{figure}

Plotting a similar comparison between the size distributions of planets in systems with two planet candidates vs.~those in higher-multiplicity systems (Figure \ref{fig:radius_multiplicities_split_2v3}) shows a slight excess of planets of super-Earth size planets relative to \hbox{(sub-)neptunes} in systems with three or more PCs, with the distribution of two-planet systems lying between those for single planets and those for high multiplicity systems. But while the differences between singles and multis are highly significant, neither the Kolmogorov-Smirnov test nor the Anderson-Darling test show the difference between 2's and 3+'s to be of statistically significant at the 5\% level.  That said, both tests show significance at the 10\% level and the sign of the difference is consistent with the trend from singles to multis of fewer large planets in systems with larger multiplicity.

% A top hat KDE to show distribution of planet radii.
% logscale on x-axis from 0.2 to 20 Rearth.
% singles in blue, multis in red, (all in black?)
% begin with width of 0.02 decades? 

\begin{figure}
    \centering
    % \includegraphics[width=0.75\textwidth]{Current_Figures/Section_3/multis_subset_radii_cdf_5.pdf}
    \includegraphics[width=0.45\textwidth]{Current_Figures/Section_3/multis_subset_radii_cdf_2,5.pdf}
    \includegraphics[width=0.45\textwidth]{Current_Figures/Section_3/multis_subset_radii_cdf_5.pdf}
        \caption{Comparison of cumulative distribution functions of planet radii between PCs in systems with two candidates and those in systems of higher multiplicity. The  criteria for inclusion of KOIs are the same as in Fig.~\ref{fig:radius_multis_singles}. In the panel on the right, we normalized the CDFs at 5~R$_{\oplus}$ as in Fig.~\ref{fig:radius_multis_singles}, but the panel on the left normalizes the curves at half this radius to emphasize the larger number of planets in two-planet systems that are similar in size to Neptune. The numbers shown within the plot include the one object in a 2-planet system that is not plotted because its estimated size exceeds 16~R$_\oplus$; no planet candidate (nor KOI dispostioned as ``R'') larger than 12~R$_\oplus$ is present in a system with more than two PCs.} %Candidates where S/N $< 12$ or $b + \sigma_+(b) \geq$ 0.95 are excluded from this radius distribution, although they are accounted for in determining the system multiplicity. and there are no objects vetted `R' in systems with two or more PCs. ( This plot includes KOIs rejected because they are too large (dispositioned as R in Table \ref{tab:planetcatalog}). This plot uses  The difference between the two curves is marginally significant (KstestResult( pvalue=0.097, Anderson_ksample significance=0.074)). Data taken from \texttt{koiprops\_20211209.csv}}
    \label{fig:radius_multiplicities_split_2v3}
\end{figure}




\begin{figure}
    \centering
    \includegraphics[width=0.7\textwidth]{img/June2019/CDF_RadiusLogMultiSplit.eps}
    \caption{Similar to Figure \ref{fig:radius_multiplicities_split}, but in semilog-x. Data taken from \texttt{koiprops\_20190610.csv}}
    \label{fig:log_radius_multiplicities_split}
\end{figure}

\begin{figure}
    \centering
    \includegraphics[width=0.7\textwidth]{img/June2019/size_distribution_large_multis_vs_singles.eps}
    \caption{Relative size distribution of singles and multis restricted to $P>10$ days, $b<0.9$  and S/N~$<7.1$ for planets in the $5-20 R_\oplus$ size range.  Note the large number of very large singles.  If a similar plot is made with the size range  restricted to $6-11 R_\oplus$, then the curves look identical.  Not sure what to display here - top down or bottom up starting at 5 or 6, normalized around 11, but do we want the period cut? Note that especially if we place the lower limit at 6, there are hardly any large multis with $P<10$ days, so perhaps a figure with 3 curves, with longer period singles represented by a dotted blue curve?}
    \label{fig:size_distribution_large_multis_singles}
\end{figure}

% Of the 221 singles with $R_p \geq R_\oplus$, 32 are $b > 0.95$ and 55 are $b > 0.9$. Of the 36 multis with $R_p \geq 8R_\oplus$, 4 are $b > 0.95$ and 6 are $b > 0.9$
\begin{figure}
    \centering
    \subfloat[][koi20190913]{
    \includegraphics[width=0.48\textwidth]{img/September2019/CDF_Log3BinRadiiSplit.eps}}
    \hspace{1mm}
    \subfloat[][koi20191202]{
    \includegraphics[width=0.48\textwidth]{Jan2020/Figure16_Rp8.pdf}}
    \caption{Plot of cumulative period distribution,  separating the planets by radius into 3 bins ($R_p < 1.8R_\oplus, \, 1.8R_\oplus < R_p < 8R_\oplus,\,8R_\oplus < R_p$). Only planets with $b<0.9$ are included. Data in right plot taken from \texttt{koiprops\_20191202.csv}; note this plot states 5 Rearth, but actually uses 8 Rearth}
    \label{fig:cdf_log3binradiisplit8}
\end{figure}


\begin{figure}[!hbt]
\centering
\includegraphics[width=0.75\textwidth]{Current_Figures/Section_3/period_cdf.pdf}
\caption{The  cumulative period distributions of various subsets of planet candidates observed by \ikt,  normalized by the total number of objects of its class that have been detected. The normalization includes all orbital periods.  To remove spurious detections, only planet candidates with  of S/N~$> 12$ are considered. }% Data taken from \texttt{koiprops\_20211209.csv}}
\label{fig:perdist}
\end{figure}


\begin{figure}[!hbt]
\includegraphics[scale=1.0,angle=0]{rvsp_new_line2.eps}
\caption{newer version.}
\label{fig:perradold}
\end{figure}

\begin{figure}[!hbt]
\includegraphics[scale=1.0,angle=0]{rvsp_new_line.eps}%Rp_vs_P.eps}
\caption{Old P vs. R plot}
\label{fig:perrad2018}
\end{figure}

% \begin{figure}[!hbt]
% \centering
% \begin{minipage}{0.47\textwidth}
%     \centering
%     \includegraphics[width = \textwidth]{img/Normal/HIST_Period_Inner_All_Pairs.eps}
% \end{minipage}
% \begin{minipage}{0.47\textwidth}
%     \centering
%     \includegraphics[width=\textwidth]{img/Normal/HIST_Period_Outer_All_Pairs.eps}
% \end{minipage}
% \caption{The period distribution of inner planets (left) and outer planets (right) among all adjacent pairs in multis. (The orbital periods of intermediate planets are counted in both histograms.) 
% \label{fig:periods}} 
% \end{figure}

%\begin{figure}[!hbt]
%      \centering
%      \includegraphics[width=0.7\textwidth]{img/Normal/CDF_Derivative_Period_Ratios_Multi.eps}
%      \caption{Slope of CDF (Period Ratio) with various window sizes. Axes labels omitted for brevity but are exactly the same to that of Figure \ref{fig:CDF_Derivative_PRatio}.}
%      \label{fig:CDF_Derivative_PRatio_Subplot}
%  \end{figure}
%  
\begin{figure}
    \centering
    \includegraphics[width=0.7\textwidth]{img/NewPlots/CDF_RadiiSplit.eps}
    \caption{Plot cumulative number versus period with data split into subsets of smaller/larger than 1.8$R_\oplus$. Small singles ramp up in CDF earlier than large singles, perhaps due to atmospheric mass loss close to the star? Data taken from \texttt{koiprops\_20181121.csv}}
    % \caption{Plot of data, split into subsets of smaller/larger than the median radius. To find median radius, singles that have impact parameter more than 0.95 are first excluded. The data are then split by the median of all remaining planets. Median radius found to be $2.13 R_\oplus$. Small singles ramp up in CDF earlier than large singles, perhaps due to atmospheric mass loss close to the star?}
    \label{fig:cdf_radiisplit}
\end{figure}

\begin{figure}
    \centering
    \includegraphics[width=0.7\textwidth]{img/NewPlots/CN_RadiiSplit.eps}
    \caption{Alternative to Figure \ref{fig:cdf_radiisplit} with Cumulative Number on Y-axis. Data taken from \texttt{koiprops\_20181121.csv}}
    \label{fig:cn_radiisplit}
\end{figure}

\begin{figure}
    \centering
    \includegraphics[width=0.7\textwidth]{img/NewPlots/CN_LogRadiiSplit.eps}
    \caption{Alternative to Figure \ref{fig:cdf_radiisplit} with Cumulative Number on Y-axis and log scale on X-axis. Data taken from \texttt{koiprops\_20181121.csv}.}
    \label{fig:cn_logradiisplit}
\end{figure}
\section{items associated with Accurate Orbital Periods}
\subsubsection{Theoretical Model: Eccentric Planet with a Distant Massive Companion}
We measured that change in orbital period for an eccentric low-mass inner planet ($\mu = 3.0\times 10^{-6}$, $e = 0.2$) orbiting a sun-like star at 20 days, perturbed by a co-planar massive planet ($\mu = 1.0\times 10^{-3}$) on a circular orbit at 81.1 days, far from any mean motion resonance. Figure~\ref{fig:Case1} shows the precession of the longitude of pericenter of the inner planet alongside its transit-to-transit orbital period variations smoothed over a moving four-year interval. ***The right panels are quite similar to 419 b.  Perhaps it's better to show that planet at more times - I'm thinking of observations either right before or during a dip, where predictions could be far off not that far into the future.

\begin{figure}
    \centering
   \includegraphics[scale=.3]{Current_Figures/TransitDurations/EccRadius7.pdf}
    \caption{Comparison of cumulative $\tau$ distribution for planets divided based on their size.  Subsets are:  0.5~R$_\oplus < R_p \le 1.0$~R$_\oplus$ (dotted light blue),
    1.0~R$_\oplus < R_p \le 1.6$~R$_\oplus$ (solid dark blue), 
    1.6~R$_\oplus < R_p \le 1.8$~R$_\oplus$ (dotted light green), 
    1.8~R$_\oplus < R_p \le 2.7$~R$_\oplus$ (dashed dark green), 
    2.7~R$_\oplus < R_p \le 5.0$~R$_\oplus$ (dotted light red), 5~R$_\oplus < R_p \le 9$~R$_\oplus$ (dash-dotted dark red), 9~R$_\oplus < R_p \le 12.5$~R$_\oplus$ (dash-dot-dotted orange).  A 7-sample AD test strongly rejects the null hypothesis that each subset is drawn from a common distribution ($p$-value $\approx~3\times ~10^{-11}$).  The two smallest radius curves are among the three subsets with the most highly concentrated $\tau$ distribution, whereas the two largest radius curves are the subsets with the largest tails.  Both of these would be expected if larger (and thus more massive) planets have typically experienced more significant dynamical excitation (e.g., planet-planet scattering followed by secular evolution).    }
    \label{fig:EccRadius}
\end{figure}


\begin{figure}
\centering
%\includegraphics[width=0.75\textwidth]{Current_Figures/Section_4/cdf_period_lin_log.pdf}
\includegraphics[width=0.45\textwidth]{Current_Figures/TransitDurations/EccRadius7.pdf}
    \includegraphics[width=0.45\textwidth] {Current_Figures/TransitDurations/EccRadius3.pdf}
    \caption{Comparison of the normalized cumulative $\tau$ distribution for planets grouped by size. Left panel: Subsets are:  0.5~R$_\oplus < R_p \le 1.0$~R$_\oplus$ (dotted light blue),
    1.0~R$_\oplus < R_p \le 1.6$~R$_\oplus$ (solid dark blue), 
    1.6~R$_\oplus < R_p \le 1.8$~R$_\oplus$ (dotted light green), 
    1.8~R$_\oplus < R_p \le 2.7$~R$_\oplus$ (dashed dark green), 
    2.7~R$_\oplus < R_p \le 5.0$~R$_\oplus$ (dotted light red), 5~R$_\oplus < R_p \le 9$~R$_\oplus$ (dash-dotted dark red), 9~R$_\oplus < R_p \le 12.5$~R$_\oplus$ (dash-dot-dotted orange).  A 7-sample AD test strongly rejects the null hypothesis that each subset is drawn from a common distribution ($p$-value $\approx~3\times ~10^{-11}$).  The two smallest radius curves are among the three subsets with the most highly concentrated $\tau$ distribution, whereas the two largest radius curves are the subsets with the largest tails.  Both of these would be expected if larger (and thus more massive) planets have typically experienced more significant dynamical excitation (e.g., planet-planet scattering followed by secular evolution).   Right panel: Subsets are: $R_p < 0.5~$R$_\oplus < R_p \le 1.6$~R$_\oplus$ (solid blue), 1.6~R$_\oplus < R_p \le 5$~R$_\oplus$ (dotted green), 5~R$_\oplus < R_p \le 12.5$~R$_\oplus$ (dashed red).  A 3-sample AD test strongly rejects the null hypothesis that each subset is drawn from a common distribution ($p$-value $< 10^{-11}$). }
    \label{fig:EccRadius}
\end{figure}

Thus, we combine planets into three size bins,  0.5~--~1.6~R$_\oplus$, 1.6~--~5~R$_\oplus$, and 5~--~12.5~R$_\oplus$, in  Fig.~\ref{fig:EccRadius3}. The hypothesis that all three of these bins have the same underlying distributions can be strongly rejected, with $p$-value $< 10^{-11}$ (using 3-sampled AD test), and
comparing any pair of these three results in a $p$-value of $\approx~10^{-3}$ (KS tests) or $\approx~10^{-4}$ (AD tests), showing highly significant evidence for differences in the eccentricity distribution between these three broad size bins. 
Further supporting this interpretation, planets with $R_p\le~1.6$~R$_\oplus$ have a particularly concentrated $\tau$ distribution (i.e., small eccentricities), while that for planets with $R_p > 5$~R$_\oplus$ has a relatively large tail (i.e., larger eccentricities), with planets 1.6~R$_\oplus < R_p \le~5$~R$_\oplus$ having an intermediate distribution of  $\tau$ values.

\begin{figure}
    \centering
   \includegraphics[scale=.3]{Current_Figures/TransitDurations/EccRadius3.pdf}
    \caption{Comparison of cumulative $\tau$ distribution for planets grouped by size.  Subsets are: $R_p < 0.5~$R$_\oplus < R_p \le 1.6$~R$_\oplus$ (solid blue), 1.6~R$_\oplus < R_p \le 5$~R$_\oplus$ (dotted green), 5~R$_\oplus < R_p \le 12.5$~R$_\oplus$ (dashed red).  A 3-sample AD test strongly rejects the null hypothesis that each subset is drawn from a common distribution ($p$-value $< 10^{-11}$).    }
    \label{fig:EccRadius3}
\end{figure}

\begin{figure}[!hbt]
 \includegraphics [height = 1.2 in]{Case1_varpi.eps}
 \includegraphics [height = 1.2 in]{Case1_period.eps}
  \includegraphics [height = 1.2 in]{Case2_varpi.eps}
 \includegraphics [height = 1.2 in]{Case2_period.eps}
 \caption{
 \label{fig:Case1} Precession of the longitude of pericenter and associated transit-to-transit period variations in a theoretical systems composed of a low-mass inner eccentric planet orbiting at 20 days perturbed by a coplanar massive neighbor on a circular orbit. at 81.1 days. Left panels: $e_{\rm inner}$ = 0.2, $P_{\rm outer}$ = 81.1 days. Right panels: $e_{\rm inner}$ = 0.8, $P_{\rm outer}$ = 106.0 days.} 
\end{figure}

%lead to differences between the periods that we calculate and  actual mean planetary orbital periods that can be significant both for studies of planetary system dynamics and for ephemeris predictions.  Most  \ik planets that show large TTVs are close to mean-motion orbital resonances with other planets.  observed for result from variations in planetary he other effect is that precession of eccentric orbits (which is caused primarily by perturbations from other planets) causes measured periods (mean time between transits) to differ from actual orbital periods, i.e., mean time to move through 360$^\circ$ in longitude (\S\ref{sec:periods}). See \cite{lithwick:2012} for a discussion.  The largest effect for planets near 2-body resonances are consequences of the rotation of the forced eccentricity caused by resonant perturbations (this is why most planets/moons locked in 2-body resonances have period ratios slightly larger than values estimated from the resonant coefficients).  However, the timescale for this precession is generally shorter than or comparable to the duration of the \ik observations, and the variations thus tend to average out. This is correct to first-order, but is that enough??  The more general but usually smaller effect is caused by s


\end{document}

\begin{comment} % comment out old graphs
\begin{figure}[!hbt]
 \includegraphics [height = 2.2 in]{Kep29b_01.eps}
 \includegraphics [height = 2.2 in]{Kep29c_01.eps}
  \includegraphics [height = 2.2 in]{Kep29b_02.eps}
 \includegraphics [height = 2.2 in]{Kep29c_02.eps}
  \includegraphics [height = 2.2 in]{Kep29b_03.eps}
 \includegraphics [height = 2.2 in]{Kep29c_03.eps}
  \includegraphics [height = 2.2 in]{Kep29b_04.eps}
 \includegraphics [height = 2.2 in]{Kep29c_04.eps}
 \caption{Projected average transit-to-transit orbital periods of planets transiting Kepler-29. The blue triangle marks the polynomial fit of Holczer+ 2016, and the red point with its error bar marks our fit to the lightcurve assuming a constant period. The black points mark the period between simulated transits from a single sample of the posteriors following the TTV models against the Kepler data only of Vissapragada+ 2020, smoothed and centered over four year intervals. In the lower panels, simulated transit periods from a second sample from the TTV model are shown in grey.
 \label{fig:Kep29Periods}} 
\end{figure}
\end{comment}


\begin{comment} % comment out old graphs
\begin{figure}[!hbt]
 \includegraphics [height = 2.2 in]{Kep419b_01.eps}
  \includegraphics [height = 2.2 in]{Kep419b_02.eps}
  \includegraphics [height = 2.2 in]{Kep419b_03.eps}
   \includegraphics [height = 2.2 in]{Kep419b_04.eps}
 \caption{Projected average transit-to-transit orbital periods of Kepler-419 b. ** remove The blue triangle The red point with its error bar marks our fit to the lightcurve assuming a constant period. The black points mark the period between simulated transits from a single sample of the posteriors following the photodynamical models of Almenara+, smoothed and centered over a moving interval. Top left: 10 years, with a 4-year centered smoothing interval. Top right: 100 years, with an additional sample model in green. Bottom left: after 1000 years, slight differences between the models become noticeable. Bottom right: the period is smoothed over a 40-year interval.
 \label{fig:Kep419Periods}} 
\end{figure}
\end{comment}

\begin{table}[!hbt]
    \centering
    \begin{tabular}{c c c c c c c c c c c}
\hline
KOI          & KIC         &   & Period [d]      & T0 [MJD]          & \rprs       & b            & \rhostar [g/cm$^3$]   & u$_1$ & u$_2$ & TTVflag \\
 Disposition & Kepler-Name &   & $\pm \sigma$ & $\pm \sigma$ & $\pm \sigma$ & $\pm \sigma$ & $\pm \sigma$ &       &       &         \\
\hline
1.01 & 11446443 &            & 2.47061338 & 787.064865 & 0.123865 & 0.8179 & 1.83652 & 0.3860 & 0.2724 & 000 \\
PPPP & Kepler-1 b &            & $\pm$ 0.00000002 & $\pm$ 0.000009 & $\pm$ $^{ 0.000066 } _{ 0.000062 }$ & $\pm$ $^{ 0.0006 } _{ 0.0006 }$ & $\pm$ $^{ 0.00623 } _{ 0.00849 }$ & & & \\
2.01 & 10666592 &            & 2.20473545 & 788.535431 & 0.075182 & 0.2242 & 0.40810 & 0.3043 & 0.3129 & 006 \\
PPPP & Kepler-2 b &            & $\pm$ 0.00000005 & $\pm$ 0.000017 & $\pm$ $^{ 0.000010 } _{ 0.000008 }$ & $\pm$ $^{ 0.1585 } _{ 0.2163 }$ & $\pm$ $^{ 0.00024 } _{ 0.00030 }$ & & & \\ 
3.01 & 10748390 &            & 4.88780321 & 786.095763 & 0.057856 & 0.0539 & 3.69279 & 0.6403 & 0.1021 & 007 \\
PPPP & Kepler-3 b &            & $\pm$ 0.00000024 & $\pm$ 0.000082 & $\pm$ $^{ 0.000451 } _{ 0.000123 }$ & $\pm$ $^{ 0.1855 } _{ 0.0410 }$ & $\pm$ $^{ 0.03943 } _{ 0.28590 }$ & & & \\
4.01 & 3861595 &            & 3.84937138 & 787.262929 & 0.039553 & 0.9147 & 0.22263 & 0.3150 & 0.3029 & 001 \\
PPPP & Kepler-1658 b &            & $\pm$ 0.00000116 & $\pm$ 0.000525 & $\pm$ $^{ 0.000738 } _{ 0.001311 }$ & $\pm$ $^{ 0.0160 } _{ 0.0349 }$ & $\pm$ $^{ 0.12884 } _{ 0.06442 }$ & & & \\
5.01 & 8554498 &            & 4.78032742 & 787.803506 & 0.036539 & 0.9517 & 0.34592 & 0.3746 & 0.28 & 002 \\
PPPP &  &            & $\pm$ 0.00000087 & $\pm$ 0.000159 & $\pm$ $^{ 0.000217 } _{ 0.000186 }$ & $\pm$ $^{ 0.0016 } _{ 0.0019 }$ & $\pm$ $^{ 0.01705 } _{ 0.01583 }$ & & & \\
5.02 & 8554498 &            & 7.05201465 & 785.658528 & 0.003162 & 0.7272 & 0.34677 & 0.3746 & 0.28 & 000 \\
FFNN &  &            & $\pm$ 0.00010277 & $\pm$ 0.030691 & $\pm$ $^{ 0.000379 } _{ 0.000506 }$ & $\pm$ $^{ 0.0001 } _{ 0.6674 }$ & $\pm$ $^{ 1.71561 } _{ 0.34312 }$ & & & \\
6.01 & 3248033 &            & 1.33410702 & 788.451930 & 0.010121 & 0.1530 & 1.23669 & 0.3306 & 0.3007 & 100 \\
FFFF &  &            & $\pm$ 0.00000319 & $\pm$ 0.001062 & $\pm$ $^{ 0.000348 } _{ 0.000174 }$ & $\pm$ $^{ 0.3525 } _{ 0.1443 }$ & $\pm$ $^{ 0.11925 } _{ 0.39750 }$ & & & \\
7.01 & 11853905 &            & 3.21366892 & 786.114786 & 0.024505 & 0.0210 & 0.46369 & 0.4025 & 0.2645 & 005 \\
PPPP & Kepler-4 b &            & $\pm$ 0.00000100 & $\pm$ 0.000295 & $\pm$ $^{ 0.000138 } _{ 0.000086 }$ & $\pm$ $^{ 0.2160 } _{ 0.0161 }$ & $\pm$ $^{ 0.00795 } _{ 0.03497 }$ & & & \\
8.01 & 5903312 &            & 1.16015300 & 787.921015 & 0.012011 & 0.7483 & 4.27560 & 0.3792 & 0.2764 & 100 \\
FFFF &  &            & $\pm$ 0.00000161 & $\pm$ 0.000720 & $\pm$ $^{ 0.000375 } _{ 0.000250 }$ & $\pm$ $^{ 0.0001 } _{ 0.5935 }$ & $\pm$ $^{ 0.32135 } _{ 1.46902 }$ & & & \\
9.01 & 11553706 &            & 3.71980916 & 785.991412 & 0.308409 & 1.2787 & 0.05733 & 0.3104 & 0.3052 & 000 \\
FFFF &  &            & $\pm$ 0.00000073 & $\pm$ 0.000224 & $\pm$ $^{ 0.209958 } _{ 0.040637 }$ & $\pm$ $^{ 0.1865 } _{ 0.0775 }$ & $\pm$ $^{ 0.00672 } _{ 0.00153 }$ & & & \\
    \end{tabular}
    \caption{Abbreviated Catalog of planet candidates. Columns and rows have been interchanged relative to the electronic table to facilitate viewing the examples presented in the typeset manuscript. The tabulated information is described in the numbered list in Section \ref{sec:unified}. }
    \label{tab:planetcatalog}
\end{table}


\begin{longrotatetable}
\begin{table}[!hbt]
    \scriptsize
    \centering
    \begin{tabular}{l c c c c c c c c}
\hline
KOI          & KIC         &   Period [d]           & T0 [MJD]          & \rprs       & $b$            & \rhoc [g/cm$^3$]      & u$_1$    & u$_2$ \\
 Disp        & Kepler-Name &   \rpl [R$_\oplus$]       & $d_{\rm transit}$ [ppm]      & $T_{\rm dur}$ [h]     & $t_{1.5,3.5}$ [h]      & nTT$_{\rm obs}$                 & nTT      & TTVflag \\
             & \adrs      &   $i$ [deg]              & $S$ [S$_\oplus$]        & S/N          & MES          & $\Delta$S/N$_{ttv}$          & S/N$_{wTTV}$  & S/N$_{woTTV}$ \\
             & Kepmag      &   \rhostar [g/cm$^3$] & \teff [K]        & \rstar [\rsun] & \mstar [\msun] & \logg                 & \feh   & sparflag \\
\hline
1.01 & 11446443 & 2.47061338 $\pm$ 0.00000002 & 787.064865 $\pm$ 0.000009 & 0.123869 $\pm$ $^{ 0.000062 } _{ 0.000066 }$ & 0.8179 $\pm$ $^{ 0.0006 } _{ 0.0006 }$ & 1.83652 $\pm$ $^{ 0.00623 } _{ 0.00849 }$ & 0.3860 & 0.2724 \\
PPPP & Kepler-1 b & 14.189 $\pm$ $^{ 0.338 } _{ 0.28600 }$ & 14230.3 $\pm$ 4.5 & 1.7432 $\pm$ 0.0000 & 1.2264 $\pm$ 0.0010 & 431 & 431 & 000 \\
& 8.4001 $\pm$ $^{ 0.0086 } _{ 0.01380 }$ & 84.4130 $\pm$ $^{ 0.0096 } _{ 0.01400 }$ & 837.13 $\pm$ $^{ 87.96 } _{ 69.11000 }$ & 4360.4 & 6468.0 & -0.2 & 7070.7 & 7323.0 \\
& 11.338 & 0.83466 $\pm$ $^{ 0.09385 } _{ 0.08717 }$ & 5730 $\pm$ 102 & 1.05 $\pm$ $^{ 0.02 } _{ 0.02000 }$ & 0.98 $\pm$ $^{ 0.07 } _{ 0.07000 }$ & 4.383 $\pm$ $^{ 0.040 } _{ 0.04100 }$ & 0.019 $\pm$ 0.144 & 1 \\
[.2cm]
2.01 & 10666592 & 2.20473545 $\pm$ 0.00000005 & 788.535435 $\pm$ 0.000017 & 0.075182 $\pm$ $^{ 0.000010 } _{ 0.000008 }$ & 0.2242 $\pm$ $^{ 0.1585 } _{ 0.2163 }$ & 0.40810 $\pm$ $^{ 0.00024 } _{ 0.00030 }$ & 0.3043 & 0.3129 \\
PPPP & Kepler-2 b & 16.346 $\pm$ $^{ 0.392 } _{ 0.33200 }$ & 6668.1 $\pm$ 1.4 & 3.8886 $\pm$ 0.0000 & 3.5992 $\pm$ 0.0005 & 602 & 602 & 006 \\
& 4.7156 $\pm$ $^{ 0.0009 } _{ 0.00120 }$ & 89.9515 $\pm$ $^{ 0.0404 } _{ 0.22630 }$ & 4247.32 $\pm$ $^{ 366.95 } _{ 440.34000 }$ & 5952.0 & 3862.2 & 13.1 & 20695.8 & 20905.0 \\
& 10.463 & 0.18862 $\pm$ $^{ 0.01464 } _{ 0.01457 }$ & 6436 $\pm$ 132 & 1.99 $\pm$ $^{ 0.04 } _{ 0.04000 }$ & 1.51 $\pm$ $^{ 0.06 } _{ 0.07000 }$ & 4.015 $\pm$ $^{ 0.026 } _{ 0.02900 }$ & 0.221 $\pm$ 0.143 & 1 \\
[.2cm]
3.01 & 10748390 & 4.88780321 $\pm$ 0.00000024 & 786.095758 $\pm$ 0.000082 & 0.057836 $\pm$ $^{ 0.000471 } _{ 0.000102 }$ & 0.0539 $\pm$ $^{ 0.1855 } _{ 0.0410 }$ & 3.68960 $\pm$ $^{ 0.04912 } _{ 0.26525 }$ & 0.6403 & 0.1021 \\
PPPP & Kepler-3 b & 4.919 $\pm$ $^{ 0.078 } _{ 0.09200 }$ & 4315.6 $\pm$ 6.5 & 2.3626 $\pm$ 0.0000 & 2.2327 $\pm$ 0.0030 & 184 & 184 & 007 \\
& 16.7102 $\pm$ $^{ 0.0646 } _{ 0.43630 }$ & 89.7421 $\pm$ $^{ 0.2256 } _{ 0.58020 }$ & 85.73 $\pm$ $^{ 4.45 } _{ 4.86000 }$ & 862.1 & 2034.6 & 0.0 & 4445.0 & 4449.0 \\
& 9.174 & 1.63997 $\pm$ $^{ 0.10580 } _{ 0.07005 }$ & 4541 $\pm$ 54 & 0.77 $\pm$ $^{ 0.01 } _{ 0.01000 }$ & 0.77 $\pm$ $^{ 0.03 } _{ 0.02000 }$ & 4.542 $\pm$ $^{ 0.023 } _{ 0.01400 }$ & 0.410 $\pm$ 0.084 & 1 \\
[.2cm]
4.01 & 3861595 & 3.84937138 $\pm$ 0.00000116 & 787.262929 $\pm$ 0.000525 & 0.039553 $\pm$ $^{ 0.000820 } _{ 0.001229 }$ & 0.9147 $\pm$ $^{ 0.0161 } _{ 0.0346 }$ & 0.22263 $\pm$ $^{ 0.12884 } _{ 0.06442 }$ & 0.3150 & 0.3029 \\
PPPP & Kepler-1658 b & 13.047 $\pm$ $^{ 0.582 } _{ 0.53400 }$ & 1297.6 $\pm$ 14.4 & 2.6386 $\pm$ 0.0912 & 2.1079 $\pm$ 0.0408 & 279 & 279 & 001 \\
& 5.5660 $\pm$ $^{ 0.8912 } _{ 0.37130 }$ & 80.7511 $\pm$ $^{ 1.4120 } _{ 0.94140 }$ & 5152.10 $\pm$ $^{ 578.54 } _{ 433.90000 }$ & 132.7 & 235.6 & 0.1 & 6624.8 & 6706.0 \\
& 11.432 & 0.06166 $\pm$ $^{ 0.00567 } _{ 0.00581 }$ & 6776 $\pm$ 139 & 3.05 $\pm$ $^{ 0.09 } _{ 0.08000 }$ & 1.78 $\pm$ $^{ 0.06 } _{ 0.14000 }$ & 3.716 $\pm$ $^{ 0.026 } _{ 0.03700 }$ & -0.076 $\pm$ 0.082 & 1 \\
[.2cm]
5.01 & 8554498 & 4.78032746 $\pm$ 0.00000087 & 787.803509 $\pm$ 0.000159 & 0.036539 $\pm$ $^{ 0.000217 } _{ 0.000186 }$ & 0.9517 $\pm$ $^{ 0.0016 } _{ 0.0020 }$ & 0.34592 $\pm$ $^{ 0.01705 } _{ 0.01583 }$ & 0.3746 & 0.28 \\
PPPP & ----- & 8.719 $\pm$ $^{ 0.462 } _{ 0.42400 }$ & 961.3 $\pm$ 4.3 & 2.0200 $\pm$ 0.0153 & 1.3841 $\pm$ 0.0185 & 282 & 282 & 002 \\
& 7.4791 $\pm$ $^{ 0.1200 } _{ 0.11140 }$ & 82.6954 $\pm$ $^{ 0.1287 } _{ 0.12870 }$ & 1399.31 $\pm$ $^{ 127.92 } _{ 151.18000 }$ & 380.8 & 360.2 & -0.5 & 5139.5 & 5446.0 \\
& 11.665 & 0.12941 $\pm$ $^{ 0.02099 } _{ 0.01881 }$ & 5936 $\pm$ 109 & 2.19 $\pm$ $^{ 0.11 } _{ 0.10000 }$ & 1.37 $\pm$ $^{ 0.13 } _{ 0.10000 }$ & 3.894 $\pm$ $^{ 0.049 } _{ 0.05400 }$ & 0.235 $\pm$ 0.158 & 1 \\
[.2cm]
5.02 & 8554498 & 7.05201465 $\pm$ 0.00010277 & 785.646252 $\pm$ 0.030691 & 0.003162 $\pm$ $^{ 0.000379 } _{ 0.000506 }$ & 0.7272 $\pm$ $^{ 0.0001 } _{ 0.6674 }$ & 0.34677 $\pm$ $^{ 1.71561 } _{ 0.34312 }$ & 0.3746 & 0.28 \\
FFNN & ----- & 0.744 $\pm$ $^{ 0.105 } _{ 0.12300 }$ & 10.8 $\pm$ 2.6 & 4.3437 $\pm$ 1.1154 & 4.2730 $\pm$ 1.1160 & 191 & 187 & 000 \\
& 11.5844 $\pm$ $^{ 4.3004 } _{ 4.73040 }$ & 89.4505 $\pm$ $^{ 0.5491 } _{ 3.56930 }$ & 820.72 $\pm$ $^{ 93.22 } _{ 71.71000 }$ & 6.2 & 0.0 & -9.3 & 6127.5 & 6329.0 \\
& 11.665 & 0.12941 $\pm$ $^{ 0.02099 } _{ 0.01881 }$ & 5936 $\pm$ 109 & 2.19 $\pm$ $^{ 0.11 } _{ 0.10000 }$ & 1.37 $\pm$ $^{ 0.13 } _{ 0.10000 }$ & 3.894 $\pm$ $^{ 0.049 } _{ 0.05400 }$ & 0.235 $\pm$ 0.158 & 1 \\
[.2cm]
6.01 & 3248033 & 1.33410702 $\pm$ 0.00000319 & 788.451930 $\pm$ 0.001062 & 0.010143 $\pm$ $^{ 0.000327 } _{ 0.000196 }$ & 0.1530 $\pm$ $^{ 0.3525 } _{ 0.1443 }$ & 1.23481 $\pm$ $^{ 0.11145 } _{ 0.41795 }$ & 0.3306 & 0.3007 \\
FFFF & ----- & 1.433 $\pm$ $^{ 0.065 } _{ 0.04500 }$ & 120.0 $\pm$ 3.5 & 2.1060 $\pm$ 0.0351 & 2.0814 $\pm$ 0.0332 & 804 & 803 & 100 \\
& 4.8806 $\pm$ $^{ 0.1619 } _{ 0.62050 }$ & 89.0266 $\pm$ $^{ 0.9732 } _{ 5.83920 }$ & 3786.26 $\pm$ $^{ 347.51 } _{ 318.55000 }$ & 43.0 & 21.5 & -8.9 & 16014.0 & 16685.3 \\
& 12.161 & 0.57373 $\pm$ $^{ 0.04644 } _{ 0.04901 }$ & 6385 $\pm$ 118 & 1.29 $\pm$ $^{ 0.03 } _{ 0.03000 }$ & 1.25 $\pm$ $^{ 0.05 } _{ 0.06000 }$ & 4.310 $\pm$ $^{ 0.026 } _{ 0.03000 }$ & 0.034 $\pm$ 0.133 & 1 \\
[.2cm]
7.01 & 11853905 & 3.21366892 $\pm$ 0.00000100 & 786.114673 $\pm$ 0.000297 & 0.024496 $\pm$ $^{ 0.000147 } _{ 0.000069 }$ & 0.0210 $\pm$ $^{ 0.2160 } _{ 0.0161 }$ & 0.46369 $\pm$ $^{ 0.00795 } _{ 0.03497 }$ & 0.4025 & 0.2645 \\
PPPP & Kepler-4 b & 4.036 $\pm$ $^{ 0.105 } _{ 0.11400 }$ & 727.8 $\pm$ 2.7 & 3.9823 $\pm$ 0.0000 & 3.8880 $\pm$ 0.0083 & 338 & 338 & 005 \\
& 6.3260 $\pm$ $^{ 0.0402 } _{ 0.16090 }$ & 89.4047 $\pm$ $^{ 0.5456 } _{ 1.63690 }$ & 1230.75 $\pm$ $^{ 122.20 } _{ 103.40000 }$ & 346.8 & 294.5 & -0.4 & 10567.9 & 10881.0 \\
& 12.211 & 0.35330 $\pm$ $^{ 0.03626 } _{ 0.03402 }$ & 5968 $\pm$ 110 & 1.50 $\pm$ $^{ 0.04 } _{ 0.04000 }$ & 1.22 $\pm$ $^{ 0.06 } _{ 0.08000 }$ & 4.165 $\pm$ $^{ 0.034 } _{ 0.03700 }$ & 0.153 $\pm$ 0.147 & 1 \\
[.2cm]
8.01 & 5903312 & 1.16015301 $\pm$ 0.00000161 & 787.921021 $\pm$ 0.000720 & 0.011986 $\pm$ $^{ 0.000400 } _{ 0.000225 }$ & 0.7483 $\pm$ $^{ 0.0001 } _{ 0.5935 }$ & 4.29576 $\pm$ $^{ 0.32316 } _{ 1.43112 }$ & 0.3792 & 0.2764 \\
FFFF & ----- & 1.231 $\pm$ $^{ 0.050 } _{ 0.03800 }$ & 169.6 $\pm$ 4.5 & 1.3252 $\pm$ 0.0205 & 1.3058 $\pm$ 0.0184 & 1153 & 1152 & 100 \\
& 6.7084 $\pm$ $^{ 0.2145 } _{ 0.85780 }$ & 89.4576 $\pm$ $^{ 0.5419 } _{ 4.47100 }$ & 2133.38 $\pm$ $^{ 142.91 } _{ 181.89000 }$ & 48.4 & 36.6 & -8.9 & 15328.0 & 16106.1 \\
& 12.450 & 1.18181 $\pm$ $^{ 0.07005 } _{ 0.08532 }$ & 5930 $\pm$ 96 & 0.94 $\pm$ $^{ 0.02 } _{ 0.02000 }$ & 0.98 $\pm$ $^{ 0.03 } _{ 0.05000 }$ & 4.485 $\pm$ $^{ 0.018 } _{ 0.02800 }$ & -0.179 $\pm$ 0.113 & 1 \\
[.2cm]
9.01 & 11553706 & 3.71980916 $\pm$ 0.00000073 & 785.991429 $\pm$ 0.000224 & 0.308409 $\pm$ $^{ 0.209958 } _{ 0.040637 }$ & 1.2787 $\pm$ $^{ 0.1865 } _{ 0.0706 }$ & 0.05733 $\pm$ $^{ 0.00672 } _{ 0.00153 }$ & 0.3104 & 0.3052 \\
FFFF & ----- & 36.000 $\pm$ $^{ 24.328 } _{ 5.03300 }$ & 2957.7 $\pm$ 10.6 & 3.5270 $\pm$ 0.0110 & 1.7635 $\pm$ 0.0055 & 368 & 367 & 000 \\
& 3.4731 $\pm$ $^{ 0.1254 } _{ 0.03270 }$ & 68.2544 $\pm$ $^{ 1.2292 } _{ 2.29450 }$ & 618.07 $\pm$ $^{ 52.66 } _{ 57.92000 }$ & 373.9 & 579.9 & -0.7 & 10541.9 & 10850.0 \\
& 13.123 & 0.86767 $\pm$ $^{ 0.08153 } _{ 0.08911 }$ & 6107 $\pm$ 112 & 1.05 $\pm$ $^{ 0.02 } _{ 0.02000 }$ & 1.02 $\pm$ $^{ 0.06 } _{ 0.07000 }$ & 4.400 $\pm$ $^{ 0.033 } _{ 0.03900 }$ & -0.218 $\pm$ 0.133 & 1 \\
[.2cm]
    \end{tabular}
    \caption{Catalog of planet candidates using stellar properties from \cite{Berger:2020a}. Only KOIs with stellar properties provided in \cite{Berger:2020a}'s catalog are listed here.  See Table \ref{tab:planetcatalog} for  further details and displayed examples of the data accessible in the full online tabulation.}
    \label{tab:planetcatalogB}
\end{table}
\end{longrotatetable}

\begin{table}[!hbt]
\scriptsize
    \centering
    \begin{tabular}{|c|c|c||c|c|c|c|c||c|c|c|}
     \multicolumn{11}{c}{\textbf{\ik CIRCUMBINARY PLANETS}}\\
     \multicolumn{11}{c}{}\\
     \hline
         KIC & KOI & Kepler &  $P_{\rm bin}$~(d) & $M_{\star A}$~(M$_\odot$) & $R_{\star A}$~(R$_\odot$)& $M_{\star B}$~(M$_\odot$) &  $R_{\star B}$~(R$_\odot$) & $P$~(d) & $M_p$~(M$_\oplus$)  & $R_p$~(R$_\oplus$)   \\
        \hline
12644769 & 1611 & 16b   & 41  & $0.6897\pm 0.0035$ & $0.6489\pm 0.0013$ &  $0.2026\pm 0.0007$ & $0.2263\pm 0.0007$ & 229  & 100   & 8       \\ 
8572936 & 2459 & 34b & 27.8 & $1.0479\pm 0.0033$ & $1.1618\pm 0.0031$ & $1.0208\pm 0.0022$ & $1.0927\pm 0.0032$ & 289   & 70   &  8         \\ 
9837578 & 2937 & 35b   & 20.7   & $0.8876 \pm 0.0053$   & $1.0284 \pm 0.0029$   & $0.8094 \pm 0.0041$  & $0.7861 \pm 0.0022$   & 131.4   & 40 & 8       \\ 
6762829 & 1740 & 38b   & 18.8   &  $0.949 \pm 0.059$  & $1.757 \pm 0.034$   & $0.249 \pm 0.010$   & $0.2724 \pm 0.0053$   & 106 &   $< 122$    &  $4.35 \pm 0.11$     \\ 
10020424 & 3154 &  47b  &  7.448 & $0.957 \pm 0.014$   & $0.936 \pm 0.005$   & $0.342 \pm 0.003$   &  $0.338 \pm 0.002$      & 49.5   & $< 26$   &  $3.05 \pm 0.04$      \\ 
10020424 & 3154 &  47d  &  7.448 & $0.957 \pm 0.014$   & $0.936 \pm 0.005$   & $0.342 \pm 0.003$   &  $0.338 \pm 0.002$        &   187 &   7 &  19$^{+24}_{-12}$      
\\ 10020424 & 3154 &  47c  &  7.448 & $0.957 \pm 0.014$   & $0.936 \pm 0.005$   & $0.342 \pm 0.003$   &  $0.338 \pm 0.002$        &   303 &  3  &    $4.65  \pm  0.08$    \\ 
4862625 &    &  64b  &  20  &  $1.528  \pm  0.087$  & $1.734  \pm  0.044$   &   $0.408  \pm  0.024$  &  $0.378  \pm  0.023$  &  137  & $< 169$ & $6.18 \pm 0.17$       \\ 
12351927 &    & 413b   &  10.1  & $0.820 \pm 0.015$   & $0.766 \pm 0.009$   &  $0.542  \pm  0.008$  & $ 0.484  \pm  0.024$  & 66 &  $67 \pm 22$  &    $4.35 \pm 0.10$    \\ 
 9632895 & 1451 &  453b  & $27.3$   & $0.944  \pm  0.010$   & $0.833  \pm  0.011$  &      $0.1951 \pm 0.0020$  &   $0.2150\pm0.0014$ &  240  &   $< 16$ & 6.2     \\ 
  5473556 & 2939 & 1647b & 11 & $1.2207 \pm 0.0012$ & $1.7903 \pm 0.0055$ & $0.9678 \pm 0.0039$ & $0.9663 \pm 0.0057$ & 1108 & 480 & 11 \\ 
6504534 & 3152 &  1661b  & 28.16  &  $0.841 \pm 0.022$  & $0.762 \pm 0.010$    &  $0.262 \pm 0.005$ & $0.276 \pm 0.006$    & $175$   & $17 \pm 12$   &     $3.87 \pm 0.02$   \\ 
10753734 & & & 19.4 & & & & & 260 & & 6 \\
 \hline
    \end{tabular}
    \caption{Data from \cite{Welsh:2018}, except 3154 from \cite{Orosz:2019}, 3152 from Socia, Q., Welsh, W.;  Orosz, J. et al. 2020 and 10753734 from Welsh, W
2019.}
    \label{tab:cbp}
\end{table}  


% End of file 
