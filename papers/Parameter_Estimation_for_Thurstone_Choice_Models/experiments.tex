In this section, we present our experimental results using both simulations and real-world data. Our first goal is to provide experimental validation of the claim that the mean squared error can depend on the cardinality of comparison sets in different ways for different Thurstone choice models, which is suggested by our theory. Our second goal is to evaluate Fiedler value for different weighted-adjacency matrices observed in real-world data, which demonstrates that it can assume a wide range of values depending on the application scenario. 

\subsection{$\mse$ versus Cardinality of Comparison Sets}

We consider the following simulation experiment. We fix the number of items $n$ and the number of observations $m$. We then run experiments for different values of the cardinality of comparison sets $k$. For each given value of parameter $k$, we generate comparison sets as independent uniform random sets of cardinality $k$ from the set of all items. We then draw choices according to a Thurstone choice model \GT\ for the value of parameter vector $\theta^\star = \vec{0}$. For every fixed value of $k$, we run $100$ repetitions to estimate the mean squared error. We do this for the distribution of noise according to a double-exponential distribution (Bradley-Terry model) and according to a uniform distribution, both with unit variance.

Figure~\ref{fig:synt} shows the results for the case of $n = 10$ and $m = 100$. The results clearly demonstrate that the mean squared error exhibits qualitatively different decay with the cardinality of comparison sets for the two Thurstone choice models under consideration. Our theoretical results in Section~\ref{sec:kary} suggest that the mean squared error should decrease with the cardinality of comparison sets as $1/(1-1/k)$ for the double-exponential distribution, and as $1/k^2$ for the uniform distribution of noise. Observe that the latter two terms decrease with $k$ to a strictly positive value and to zero value, respectively. The empirical results in Figure~\ref{fig:synt} confirm these claims. 

\begin{figure}[t]
\centering
\vspace*{-1.85cm}
\includegraphics[width=0.49\textwidth]{MSEdoubleexp.pdf}
\includegraphics[width=0.49\textwidth]{MSEunif.pdf}
\vspace*{-2cm}
\caption{Mean squared error for two different Thurstone choice models \GT: (left) double-exponential distribution of noise, and (right) uniform distribution of noise. The vertical bars denote 95\% confidence intervals. The results demonstrate two qualitatively different relations between the mean squared error and the cardinality of comparison sets, which confirm the theory.}
\label{fig:synt}
\end{figure}

\subsection{Fiedler Values of Weighted-Adjacency Matrices}

We found that Fiedler value of a weighted-adjacency matrix plays a key role in upper bounds on the mean squared error of parameter estimator in Section~\ref{sec:pairs} and Section~\ref{sec:kary}. Here we evaluate Fiedler value for different weighted-adjacency matrices of different schedules of comparisons. Throughout this section, we use the definition of a weighted-adjacency matrix in (\ref{equ:weightmatrix}) with the weight function $w(k) = 1/k^2$. Our first two examples are representative of schedules in sport competitions, which are typically carefully designed by sport associations and exhibit a large degree of regularity. Our second two examples are representative of comparisons that are induced by user choices in the context of online services, which exhibit much more irregularity. 

\paragraph{Sport competitions} We consider the fixtures of games for the season 2014-2015 for (i) football Barclays premier league and (ii) basketball NBA league. In the Barclays premier league, there are 20 teams, each team plays a home and an away game with each other team; thus there are 380 games in total. In the NBA league, there are 30 teams, 1,230 regular games, and 81 playoff games.\footnote{The NBA league consists of two conferences, each with three divisions, and the fixture of games has to obey constraints on the number of games played between teams from different divisions.} We evaluate Fiedler value of weighted-adjacency matrices defined for first $m$ matches of each season; see Figure~\ref{fig:premier-nba}.

%The NBA league consists of two conferences, namely Eastern and Western conference; each conference consists of three divisions; each division consists of five teams. The schedule of games is determined subject to the following constraints: each team has to play 4 games against each other team in the same division; 4 games against 6 out-of-division teams from the same conference; 3 games against the remaining 4 teams in the same conference; 2 games against each team of the other conference. The playoff round is an elimination tournament played by the top 8 teams from each conference; the winner between a pair of teams is determined by the best-of-seven rule, i.e., as soon one of the two teams wins four games, requiring at least 4 and at most 7 games.

For the Barclays premier league dataset, at the end of the season, the Fiedler value of the weighted-adjacency matrix is of value $n/[2(n-1)] \approx 1/2$. The schedule of matches is such that at the middle of the season, each team played against each other team exactly once, at which point the Fiedler value is $n/[4(n-1)] \approx 1/4$. The Fiedler value is of a strictly positive value after the first round of matches. For most part of the season, its value is near to $1/4$ and it grows to the highest value of approximately $1/2$ in the last round of the matches.

For the NBA league dataset, at the end of the season, the Fiedler value of the weighted-adjacency matrix is approximately $0.375$. It grows more slowly with the number of games played than for the Barclays premier league; this is intuitive as the schedule of games is more irregular, with each team not playing against each other team the same number of times.

\begin{figure}[t]
\centering
\vspace*{-1.85cm}
\includegraphics[width=0.49\textwidth]{FootballFixtures.pdf}\hspace*{-0.2cm}
\includegraphics[width=0.49\textwidth]{NBAFixtures.pdf}
\vspace*{-2cm}
\caption{Fiedler value of the weighted-adjacency matrices for the game fixtures of two sports in the season 2014-2015: (left) football Barclays premier league, and (right) basketball NBA league.}
\label{fig:premier-nba}
\end{figure}

\paragraph{Crowdsourcing contests} We consider participation of users in contests of two competition-based online labour platforms: (i) online platform for software development TopCoder and (ii) online platform for various kinds of tasks Tackcn. We refer to coders in TopCoder and workers in Taskcn as users. We consider contests of different categories observed in year 2012. In both these systems, the participation in contests is according to choices made by users. 

For each set of tasks of given category, we conduct the following analysis. We consider a conditioned dataset that consists only of a set of top-$n$ users with respect to the number of contests they participated in given year, and of all contests attended by at least two users from this set. We then evaluate Fiedler value of the weighted-adjacency matrix for parameter $n$ ranging from $2$ to the smaller of $100$ or the total number of users. Our analysis reveals that the Fiedler value tends to decrease with $n$. This indicates that the larger the number of users included, the less connected the weighted-adjacency matrix is. See the top plots in Figure~\ref{fig:TopCoder}. 

We also evaluated the smallest number of contests from the beginning of the year that is needed for the Fiedler value of the weighted-adjacency matrix to assume a strictly positive value. See the bottom plots in Figure~\ref{fig:TopCoder}. We observe that this threshold number of contests tends to increase with the number of top users considered. There are instances for which this threshold substantially increases for some number of the top users. This, again, indicates that the algebraic connectivity of the weighted-adjacency matrices tends to decrease with the number of top users considered.

\begin{figure}[t]
\centering
\vspace*{-1.8cm}
\includegraphics[width=0.49\textwidth]{Top_lambda2_TopCoder_pCat0.pdf}\hspace*{-0.2cm}
\includegraphics[width=0.49\textwidth]{Top_lambda2_Taskcn_pCat2.pdf}\hspace*{-0.2cm}
\\\vspace*{-4cm}
\includegraphics[width=0.49\textwidth]{Top_mzero_TopCoder_pCat0.pdf}\hspace*{-0.2cm}
\includegraphics[width=0.49\textwidth]{Top_mzero_Taskcn_pCat2.pdf}\hspace*{-0.2cm}
\vspace*{-1.5cm}
\caption{(Left) Topcoder data restricted to top-$n$ coders and (Right) same as left but for Taskcn, for Design and Website task categories, respectively. The top plots show the Fiedler value and the bottom plots show the minimum number of contests to observe a strictly positive Fiedler value.}
\label{fig:TopCoder}
\end{figure}


%\begin{figure*}[t]
%\centering
%\vspace*{-1cm}
%\includegraphics[width=0.2\textwidth]{Top_lambda2_TopCoder_pCat0.pdf}\hspace*{-0.2cm}
%\includegraphics[width=0.2\textwidth]{Top_lambda2_TopCoder_pCat1.pdf}\hspace*{-0.2cm}
%\includegraphics[width=0.2\textwidth]{Top_lambda2_TopCoder_pCat2.pdf}\hspace*{-0.2cm}
%\includegraphics[width=0.2\textwidth]{Top_lambda2_TopCoder_pCat3.pdf}\hspace*{-0.2cm}
%\includegraphics[width=0.2\textwidth]{Top_lambda2_TopCoder_pCat4.pdf}\\\vspace*{-2cm}
%\includegraphics[width=0.2\textwidth]{Top_mzero_TopCoder_pCat0.pdf}\hspace*{-0.2cm}
%\includegraphics[width=0.2\textwidth]{Top_mzero_TopCoder_pCat1.pdf}\hspace*{-0.2cm}
%\includegraphics[width=0.2\textwidth]{Top_mzero_TopCoder_pCat2.pdf}\hspace*{-0.2cm}
%\includegraphics[width=0.2\textwidth]{Top_mzero_TopCoder_pCat3.pdf}\hspace*{-0.2cm}
%\includegraphics[width=0.2\textwidth]{Top_mzero_TopCoder_pCat4.pdf}
%\vspace*{-1cm}
%\caption{Topcoder data restricted to top-$n$ coders: (top) Fiedler value, (bottom) minimum number of contests for strictly positive Fiedler value.}
%\label{fig:TopCoder}
%\end{figure*}


%\begin{figure*}[t]
%\centering
%\vspace*{-3cm}
%%\includegraphics[width=0.25\textwidth]{lambda_Tackcn_Top10_pCat1.pdf}
%\includegraphics[width=0.45\textwidth]{lambda_Tackcn_Top10_pCat2.pdf}
%%\includegraphics[width=0.25\textwidth]{lambda_Tackcn_Top10_pCat3.pdf}
%\includegraphics[width=0.45\textwidth]{lambda_Tackcn_Top10_pCat4.pdf}
%\vspace*{-2.5cm}
%\caption{Taskcn.}
%\label{fig:taskcn}
%\end{figure*}

