We consider the estimation accuracy of individual strength parameters of a Thurstone choice model when each input observation consists of a choice of one item from a set of two or more items (so called top-1 lists). This model accommodates the well-known choice models such as the Luce choice model for comparison sets of two or more items and the Bradley-Terry model for pair comparisons.   

We provide a tight characterization of the mean squared error of the maximum likelihood parameter estimator. We also provide similar characterizations for parameter estimators defined by a rank-breaking method, which amounts to deducing one or more pair comparisons from a comparison of two or more items, assuming independence of these pair comparisons, and maximizing a likelihood function derived under these assumptions. We also consider a related binary classification problem where each individual parameter takes value from a set of two possible values and the goal is to correctly classify all items within a prescribed classification error. 

The results of this paper shed light on how the parameter estimation accuracy depends on given Thurstone choice model and the structure of comparison sets. In particular, we found that for unbiased input comparison sets of a given cardinality, when in expectation each comparison set of given cardinality occurs the same number of times, for a broad class of Thurstone choice models, the mean squared error decreases with the cardinality of comparison sets, but only marginally according to a diminishing returns relation. On the other hand, we found that there exist Thurstone choice models for which the mean squared error of the maximum likelihood parameter estimator can decrease much faster with the cardinality of comparison sets. 

We report empirical evaluation of some claims and key parameters revealed by theory using both synthetic and real-world input data from some popular sport competitions and online labor platforms.