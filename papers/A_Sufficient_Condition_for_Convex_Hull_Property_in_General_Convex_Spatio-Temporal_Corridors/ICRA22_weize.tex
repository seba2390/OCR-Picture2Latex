\documentclass[letterpaper, 10 pt, conference]{ieeeconf}
\IEEEoverridecommandlockouts
% The preceding line is only needed to identify funding in the first footnote. If that is unneeded, please comment it out.
\usepackage{cite}
\usepackage{amsmath,amssymb,amsfonts}
%\usepackage{amsthm}
\usepackage{algorithmic}
\usepackage{graphicx}
\usepackage{textcomp}
\usepackage{xcolor}
%\newtheorem{theorem}{Theorem}
%\newtheorem{corollary}{Corollary}[theorem]
%\newtheorem{lemma}[theorem]{Lemma}

\newtheorem{theorem}{Theorem}[section]
\newtheorem{corollary}{Corollary}[theorem]
\newtheorem{lemma}[theorem]{Lemma}
%\newtheorem*{remark}{Remark}

\def\BibTeX{{\rm B\kern-.05em{\sc i\kern-.025em b}\kern-.08em
    T\kern-.1667em\lower.7ex\hbox{E}\kern-.125emX}}

\makeatletter
\DeclareRobustCommand{\textsupsub}[2]{{%
  \m@th\ensuremath{%
    ^{\mbox{\fontsize\sf@size\z@#1}}%
    _{\mbox{\fontsize\sf@size\z@#2}}%
  }%
}}
\makeatother

%\long\def\/*#1*/{} %this is a macro for block comment

\title{{\LARGE \textbf{A Sufficient Condition for Convex Hull Property in General Convex Spatio-Temporal Corridors}}}

\author{Weize Zhang$^{1*}$, Peyman Yadmellat$^{1}$, and Zhiwei Gao$^{2}$
\thanks{$^{1}$Noah’s Ark Lab., Huawei Technologies Canada, Markham, Ontario,
Canada L3R 5A4.}
\thanks{$^{2}$Noah’s Ark Lab., Beijing Huawei Digital Technologies Co. Ltd.,
Beijing, China.}
\thanks{$^{*}$Correspondence: \texttt{\small {weize.zhang@huawei.com}}}
}

\begin{document}

\maketitle
\thispagestyle{empty}
\pagestyle{empty}


\begin{abstract}
\label{sec:abstract}

%% 1. what is the problem 
Scientific applications that run on leadership computing facilities often face the challenge 
of being unable to fit leading science cases onto accelerator devices due to memory constraints 
(memory-bound applications).
%
% 2. what is your solution 
In this work, the authors studied one such US Department of Energy mission-critical condensed matter 
physics application, Dynamical Cluster Approximation (DCA++), and this paper discusses how device memory-bound challenges were successfully reduced  by proposing an effective 
``all-to-all'' communication method---a ring communication algorithm. 
%
This implementation takes advantage of acceleration on GPUs and remote direct memory access (RDMA) for fast data exchange between GPUs. 
%
\\Additionally, the ring algorithm was optimized with sub-ring communicators
and multi-threaded support to further reduce communication overhead and 
expose more concurrency, respectively.
%
% 3. What's the cherry-picked evaluation result you want to mention
The computation and communication were also analyzed 
by using the Autonomic Performance Environment for Exascale 
(APEX) profiling tool,  and this paper further discusses the 
performance trade-off for the ring algorithm implementation. 
%
The memory analysis on the ring algorithm shows that the allocation size for the authors' most 
memory-intensive data structure per GPU is now reduced to $1/p$ of the original size, where $p$ is the number of GPUs in the ring communicator.
%
The communication analysis suggests that 
the distributed Quantum Monte Carlo execution time grows linearly as sub-ring size increases, and the cost of messages passing through the network interface connector could be a limiting factor.


%
% \todoRed{Ronnie: Next sentence needs rewrite, too much information about Green's function that no one knows in the abstract; recommend generalizing.} \emph {However, DCA++ is currently facing memory-bound challenge as 
% a larger device array $G_t$ is limited by device memory size, where
% $G_t$ is a two-particle Green's function that allows condensed matter
% scientists to explore larger and more complex (higher fidelity)
% physics cases.}

\end{abstract}

\keywords{DCA++, Quantum Monte Carlo, GPU Remote Direct Memory Access, memory-bound issue, exascale machines}

%\begin{IEEEkeywords}
%motion planning, convex corridor, time dimension, convex hull property, smoothness
%\end{IEEEkeywords}
\section{Introduction}  \label{sec:introduction}

\newcommand\inexpIntro[3]{#1?(#2,#3).}
\newcommand\rinexpIntro[3]{*#1?(#2,#3).}
\newcommand\outexpIntro[3]{#1!(#2,#3).}
\newcommand\outatomIntro[3]{#1!(#2,#3)}

We propose a fully automated method for proving termination of \(\pi\)-calculus processes.
Although there have been a lot of studies on termination analysis for the \(\pi\)-calculus
and related calculi~\cite{Deng06IC,Demangeon07,SangiorgiTermination,KobayashiHybrid,Yoshida04IC,DBLP:journals/jlp/DemangeonHS10,Venet98SAS}, most of them have been rather theoretical,
and there have been surprisingly little efforts in developing  fully automated termination
verification methods and tools based on them. To our knowledge,
Kobayashi's \typical{}~\cite{TyPiCal,KobayashiHybrid} is the only exception that
can prove termination of \(\pi\)-calculus processes (extended with natural numbers)
fully automatically, but its termination analysis is quite limited (see Section~\ref{sec:relatedwork}).

Our method is based on a reduction to termination analysis for sequential programs:
we translate a \(\pi\)-calculus process \(P\) to a sequential program \(S_P\), so that
if \(S_P\) is terminating, so is \(P\). The reduction allows us to use
powerful, mature methods and tools
for termination analysis of sequential programs~\cite{heizmann2016ultimate,freqterm,DBLP:conf/lics/PodelskiR04,Kuwahara2014Termination,DBLP:journals/cacm/CookPR11}.

The idea of the translation is to convert a chain of communications on replicated input
channels to a chain of recursive function calls of the target sequential program.
Let us consider the following Fibonacci process:
\begin{align*}
    & \rinexpIntro{\fib}{n}{r}
        \ifexp{n<2}{ \soutatom{r}{1} \\ &\quad}
                   { \nuexp{s_1} \nuexp{s_2} (\outatomIntro{\fib}{n-1}{s_1} \PAR \outatomIntro{\fib}{n-2}{s_2} \PAR \sinexp{s_1}{x}\sinexp{s_2}{y}\soutatom{r}{x+y}) \\}
    & \PAR \outatomIntro{\fib}{m}{r}
\end{align*}
Here, the process
$\rinexpIntro{\fib}{n}{r} \ldots$ is a function server that computes the \(n\)-th Fibonacci number
in parallel and returns the result to \(r\),
and $\outatom{\fib}{m}{r}$ sends a request for computing the \(m\)-th Fibonacci number;
those who are not familiar with the syntax of the \(\pi\)-calculus may wish to consult
Section~\ref{sec:targetlanguage} first.
To prove that the process above is terminating for any integer \(m\),
it suffices to show that there is no infinite chain of communications on $\fib$:
\[
    \fib(m,r) \to \fib(m_1,r_1) \to \fib(m_2,r_2) \to \cdots.
\]
We convert the process above to the following program:\footnote{The actual translation
  given later is a little more complex.}
\begin{verbatim}
 let rec fib(n) = if n<2 then () else (fib(n-1) [] fib(n-2)) in
 fib(m)
\end{verbatim}
Here, \texttt{[]} represents the non-deterministic choice.
Note that, although the calculation of Fibonacci numbers is not preserved,
for each chain of communications on \texttt{fib}, there is a corresponding
sequence of recursive calls:
\[
\mathtt{fib}(m) \to \mathtt{fib}(m_1) \to \mathtt{fib}(m_2) \to \cdots.
\]
Thus, the termination of the sequential program above implies the termination of
the original process.
As shown in the example above, (i) each communication on a replicated input channel
is converted to a function call, (ii) each communication on a non-replicated input
channel is just removed (or, in the actual translation, replaced by a call of
a trivial function defined by \(f(\seq{x})=(\,)\)), and (iii) parallel composition
is replaced by a non-deterministic choice.
We formalize the translation outlined above and prove its correctness.

The basic translation sketched above sometimes loses too much information.
For example, consider the following process:
\begin{align*}
    & \rinexpIntro{\pre}{n}{r} \soutatom{r}{n-1} \\
    & \PAR \rinexpIntro{f}{n}{r} \ifexp{n<0}{ \soutatom{r}{1} }
                                       { \nuexp{s} (\outatomIntro{\pre}{n}{s} \PAR \sinexp{s}{x}\outatomIntro{f}{x}{r}) } \\
    & \PAR \outatomIntro{f}{m}{r}
\end{align*}
The translation sketched above would yield:
\begin{verbatim}
  let pred(n) = n-1 in
  let rec f(n) = if n<0 then () else (pred(n) [] f(*)) in
  f(m)
\end{verbatim}
Here, \texttt{*} represents a non-deterministic integer: since we have removed
the input $\sinatom{s}{x}$, we do not have information about the value of \( x \).
As a result, the sequential program above is non-terminating, although the original
process is terminating.
To remedy this problem, we also refine the basic translation above by using a refinement
type system for the \(\pi\)-calculus. Using the refinement type system,
we can infer that the value of \(x\) in the original process is less than \(n\),
so that we can refine the definition of \texttt{f} to:
\begin{verbatim}
 let rec f(n) = ... else (pred(n) [] let x=* in assume(x<n);f(x))
\end{verbatim}
The target program is now terminating, from which
we can deduce that the original process is also terminating.
We have implemented an automated tool based on the refined translation above.

The contributions of this paper are summarized as follows.
\begin{itemize}
\item The formalization of the basic translation from the \(\pi\)-calculus
  (extended with integers) to sequential programs, and a proof of its correctness.
\item The formalization of a refined translation based on a refinement type system.
\item An implementation of the refined translation, including automated refinement type
  inference based on CHC solving, and experiments to evaluate the effectiveness of
  our method.
\end{itemize}

The rest of this paper is structured as follows.
Section~\ref{sec:targetlanguage} introduces the source and target languages
of our translation.
Section~\ref{sec:approach} 
formalizes the basic translation, and proves its correctness.
Section~\ref{sec:refinement} refines the basic translation by using a refinement type system.
Section~\ref{sec:implementation} reports an implementation and experiments.
Section~\ref{sec:relatedwork} discusses related work,
and Section~\ref{sec:conclusion} concludes the paper.

%\section{Related Work}
% \vspace{-0.5em}
In this section, we review related works of our proposed framework, including GNNs, transformers, and other in single-cell analysis.
% deep learning methods 

% \vspace{-1.5em}
\subsection{Deep Learning on Multimodal Integration} 
% \vspace{-0.5em}
There is a growing number of deep learning-based methods for multimodal single-cell analysis in the community. For instance,
scMDC~\cite{lin2022clustering} is an end-to-end autoencoder-based model with one encoder and two decoders. The encoder takes the concatenation of two modalities as an input and then reconstructs two modalities separately via two individual decoders.
% where the learned latent embedding would be used for clustering analysis.
DCCA~\cite{zuo2021deep} learns a coordinated but distinct representation for each omics data by mutually supervising each other on the basis of semantic similarity across embeddings, and then reconstructs back to the original dimension as output via a decoder for each omics data.
Cross-modal Autoencoders~\cite{yang2021multi} utilize multiple autoencoders to map different modalities onto the same latent space, and incorporate prior knowledge through the use of adversarial loss and paired anchor loss in the training process. 
BABEL~\cite{wu2021babel} consists of two neural-network-based encoders and two decoders for translation between gene expression and chromatin accessibility. Both Cross-modal Autoencoders and BABEL focus on multimodal translation by adding interoperability constraints to train multiple encoders and decoders. 
Another approach, scMM~\cite{minoura2021mixture}, captures nonlinear latent structures with variational autoencoders. It exploits a mixture-of-expert framework with a deep generative model and attains end-to-end learning by modeling raw counts of each modality. While these models have made significant advancements in multimodal integration, most of them are based on autoencoders and tend to overlook the underlying biological interactions of molecules and cells.
% However, most models for multimodal integration are based on autoencoders, which overlook the underlying biological interactions of molecules and cells. 

% \vspace{-1.9em}
\subsection{GNNs and Transformers in Single-Cell} % Analysis} 
% \vspace{-0.5em}
To capture the biological interactions of molecules and cells, there has been an increasing number of GNNs and transformer frameworks published in the field of single-cell analysis. 
One benefit of transformers applied in single-cell data is to capture long-range dependency in a global view. Another benefit is to interpret biological phenomena via the attention mechanism in transformers.
From GNNs' perspective, graphs are natural to represent all kinds of data in single-cell data, like gene-to-gene graphs, cell-to-cell graphs, and cell-to-gene graphs. Another benefit of GNNs is to easily add prior knowledge into graphs, like pathways between genes or overlaps between genes and peaks. % domain knowledge or
For example, scGNN~\cite{wang2021scgnn} models cell-cell interaction by incorporating GNN with multi-modal autoencoders. Specifically, scGNN builds a cell graph by capturing cell-type-specific regulatory signals and utilizes a Left-Truncated Mixture Gaussian model for scRNA-Seq data. GLUE~\cite{cao2022multi} pre-trains modality-specific variational autoencoders to get cell embeddings and then encodes a knowledge-based graph with GNNs. The next step involves performing an adversarial multimodal alignment of the cells through an iterative optimization process. In addition, ScMoGNN~\cite{wen2022graph} models the cell similarity and feature similarity by building a cell-feature graph and extracts information from data with a graph encoder. ScMoGNN takes advantage of gene pathway data as prior knowledge to enhance the graph and denoise the data. Moreover, scBERT~\cite{yang2022scbert} follows the pre-training and fine-tuning paradigm of bidirectional encoder representations from transformers (BERT) for cell annotation of scRNA-seq data. The process of annotation involves extracting high-level patterns of cell types from the reference dataset. Different from these approaches which focus on single-modality data,  we are the first to introduce transformers and GNNs to single-cell multimodal prediction. 
% \documentclass[ex_article]{subfiles}
% \begin{document}
\section{Problem Formulation}\label{sec:problem}
%\subsection{Linear system}
We consider the linear time-invariant (LTI) system
\begin{align}
  \begin{aligned}
    \dot x(t) = Ax(t)+Bu(t), \quad
    y(t)     = C x(t),\quad
    x(0) \sim \mathcal D,
  \end{aligned}\label{eq:system}
\end{align}
where $x(t) \in \R^n$ is state, $u(t)\in \R^m$ is input,
$y(t)\in \R^p$ is output, $A\in \R^{n\times n}$,
$B\in \R^{n\times m}$, and $C \in \R^{p\times n}$ are constant matrices, and $\mathcal D$ is a probability distribution over $\R^n$.
In this paper, we assume that $B$ and $C$ are not zero matrices,
and
$(A, B, C, \mathcal {D})$
is unknown unlike the situation in \cite{fatkhullin2021optimizing}.
%\subsection{LQR problem with structured constraints} \label{Sec2-B}
The infinite-horizon continuous-time LQR problem is formulated as
\begin{align}
  \minimize  & E_{x(0)\sim \mathcal{D}}\qty[\int_0^\infty \qty(y^\top(t) Q y(t) + u^\top(t) R u(t))dt ] \label{eq:objectivefunction} \\
  \subjectto & ~\eqref{eq:system}
\end{align}
with constant positive definite matrices $Q \in \R^{p\times p}$ and $R\in \R^{m\times m}$.
The expectation is taken with respect to the initial state $x(0) \sim \mathcal{D}$.
For the static output feedback $u(t) = -Ky(t)$ with $K\in \R^{m\times p}$ to system~\eqref{eq:system},
the objective function~\eqref{eq:objectivefunction} becomes
$
  f(K) := E_{x(0)\sim \mathcal{D}}\qty[\tilde{f}(K;x(0))],
$
where 
\begin{align}
  \tilde f(K;v) & := \int_0^\infty \qty[v^* e^{A_K^\top t}C^\top(Q+ K^\top RK)Ce^{A_K t}v]dt\label{eq:cost}
\end{align}
for $v\in \C^n$.
Then, the closed-loop is given by
\begin{align}
    \dot x(t)  = A_Kx(t),\quad
    y(t)          = Cx(t), \label{eq:closedloop}
\end{align}
where 
\begin{align}
  A_K := A-BKC. \label{eq:AK}
\end{align}



In this paper, we consider the constraints $K\in \Omega$,
where $\Omega \subset \R^{m\times p}$ is a closed convex set
that specifies the structural information of feedback gains.
This is because a structured policy is often used in practical situations.
For example,
%\subsubsection{Decentralized control}
\begin{itemize}
    \item Decentralized control: In decentralized control, some components of $K$ need to be $0$~\cite{jovanovic2016controller}.
This implies that $\Omega$ should be a certain linear subspace of $\R^{m\times p}$.

\item Linear port-Hamiltonian system: For a linear port-Hamiltonian system~\cite{Jacob2012linear}, 
if the feedback gain is positive semi-definite, the closed loop system is also a port-Hamiltonian system and passive. To ensure passivity, $\Omega$ should be defined as the set of positive semi-definite matrices, which is closed and convex.

\end{itemize}
%\subsubsection{Linear port-Hamiltonian system}



%See Section~\ref{sec:examples}.


By using Bellman lemma~\cite{bellman1957notes}, the problem~\eqref{eq:objectivefunction} with structured constraints can be formulated as
\begin{align}
  \begin{aligned}
    \minimize_K & f(K) = \tr(X\Sigma) \\
    \subjectto  & K \in \Omega\,\, \text{and}\,\, A_K\text{ is \textit{Hurwitz}},
  \end{aligned}\label{eq:problem}
\end{align}
where
$
  \Sigma  := E[x(0)x^\top(0)]
$
and $X$ is the solution to
\begin{align}
  A_K^\top X + XA_K + C^\top \qty(K^\top RK + Q)C = 0.
\end{align}
It is difficult to solve \eqref{eq:problem}, since  $f(K)$ is non-convex and saddle points may exist~\cite{fatkhullin2021optimizing}.
Moreover,
 the feasible set may have exponentilally many disconnected components~\cite{feng2019exponential}. 
Although an iterative method was proposed in \cite{zhu2015adaptive} to obtain a suboptimal static output feedback gain in the model free setting,
it cannot be applied directly to problem \eqref{eq:problem} due to the constraint $K\in \Omega$.

% In this paper, we suppose that $(A, B, C, \mathcal {D})$ in \eqref{eq:system}
% is unknown unlike the situation in \cite{fatkhullin2021optimizing}.
% Thus, we develop a model free algorithm in Section \ref{sec:model-free} for solving the problem \eqref{eq:problem}.


To develop a model free algorithm with theoretical guarantees for solving problem \eqref{eq:problem}, we impose the following throughout this paper:
\begin{assumption}\label{assume:sigma-hurwitz}
  \indent
  \begin{enumerate}
    \item $\Sigma \succ 0$.
    \item The pair $(A, C)$ is observable.
    \item There exists $K_0 \in \Omega$
          such that $A_{K_0}$ is \textit{Hurwitz}
          and $K_0$ is known.
  \end{enumerate}
\end{assumption}


Since $A_{K_0}$ is \textit{Hurwitz},
there exist positive definite matrices $G, H$ and a skew-adjoint matrix $J$
such that
$
  A_{K_0} = (J-G)H.
$
The proof is found in~\cite{prajna2002lmi}.
Let $H = L^\top L$ be the Cholesky decomposition.
Using the coordinate transformation $x'(t) = Lx(t)$, the closed-loop system~\eqref{eq:closedloop} becomes
\begin{align}
  \dot{x'}(t)  = A'_{K_0}x'(t), \quad 
  y(t)         = C'x'(t),
\end{align}
where $A'_{K_0}  = LJL^\top-LGL^\top, C' = CL^{-1}$.
Since $LGL^\top\succ 0$ and $LJL^\top = -(LJL^\top)^\top$, we have
  $A'_{K_0}+{A'_{K_0}}^\top = -2LGL^\top \prec 0$.
In the following, we assume system \eqref{eq:closedloop} after the above coordinate transformation, because we consider a static output feedback that is invariant by the coordinate transformation. That is, without loss of generality, we can assume
$A_{K_0}+A_{K_0}^\top \prec 0$.

Under Assumption \ref{assume:sigma-hurwitz},
 $f(K)$ of \eqref{eq:problem} is defined only on the set $S$ of stabilizing controllers, which is defined as
\begin{align}
  S = \{K\in \R^{m\times p}\mid A_K\text{ is \textit{Hurwitz}}\}. \label{eq:S}
\end{align}
If $K\notin S$, there exists an eigenvalue $\mu$ of $A$ such that ${\rm Re}(\mu) \geq 0$ and $f(K)$ goes to infinity.

\begin{remark} \label{remark1}
The objective function of
 problem \eqref{eq:objectivefunction}
is not a
standard LQR cost
%is given by
% \begin{align}
%     \int_0^\infty \qty(x^\top(t) Q x(t) + u^\top(t) R u(t))dt,
% \end{align}
% where $Q \in \R^{n\times n}$ and $R\in \R^{m\times m}$ are positive definite matrices. However, in the model free and output feedback setting, this cost cannot be calculated in practice, because the information of the state $x(t)$ is not available. Therefore, we consider the problem \eqref{eq:objectivefunction} following
as in some previous researches~\cite{modares2016optimal, rizvi2018output}. While similar convergence properties to the standard LQR cost can be obtained for our formulation in the model based setting if $(A, C)$ is observable~\cite{fatkhullin2021optimizing}, more detailed studies of the objective function properties are necessary for model-free version of the convergence analysis.  
\end{remark}



%\subsection{Examples}\label{sec:examples}


% \end{document}
\section{Influence in Completely Bounded Block-multilinear Forms}
\label{sec:proof}
\newcommand{\blocks}{\mathrm{blocks}}
\newcommand{\lt}{\mathrm{left}}
\newcommand{\rt}{\mathrm{right}}



In this section we prove the non-commutative root-influence inequality (\thmref{thm:bh-intro}),  the special case of the Aaronson-Ambainis conjecture given in \thmref{thm:aa}, and also briefly mention how the simulation result in \corref{cor:sim} follows from \thmref{thm:aa} and the results in \cite{AA14}. We first need some preliminaries from free probability theory. 



\subsection{Low-degree Polynomials of Haar Random Unitaries}

As discussed in the proof overview, we require bounds on the operator norm (as well as normalized trace) of low-degree polynomials of random unitaries and these follow from known results in free probability theory. Here we explain these connections and also prove some auxillary lemmas needed for the proof of \thmref{thm:bh-intro} and \thmref{thm:aa}. 



Let $z_{\ui}$ denote the non-commutative monomial $z_{i_1} z_{i_2} \cdots z_{i_d}$ for a $d$-tuple $\ui  = (i_1, \ldots, i_d) \in [t]^d$ and let $p(z_1, \ldots, z_t)$ be a non-commutative polynomial in the variables $z_1, \ldots, z_t$. We are interested in computing the operator norm $\|\cdot\|_{\op}$ and the normalized trace  $\tr_N$ of the polynomial $p(z_1, \ldots, z_t)$ (or its higher moments) when substituting $N \times N$ Haar random unitaries for the variables $z_i$.

As explained previously, the theory of free probability gives us tools that allow us to compute  the above in the limit $N \to \infty$. In particular, Voiculescu \cite{V98} showed that the  (normalized) trace of polynomials in Haar random unitaries and their conjugates converges to the trace of the same polynomial evaluated on certain infinite-dimensional operators called \emph{Haar unitaries} that satisfy a non-commutative notion of independence called \emph{free independence}. This was strengthened by Collins and Male \cite{CM11} who showed that such convergence also holds for the operator norm. A short primer on free probability is given in \appref{sec:free}, but for now one can think of $\CA$ as a self-adjoint algebra of bounded linear operators on a Hilbert space and $\phi$ as a trace functional for such operators in the statement given below.


\begin{theorem}[\cite{V98, CM11}] \label{thm:voiculescu}
    Let $p(z_1, \ldots, z_{2t})$ be a non-commutative polynomial in $\BR\langle z_1, \ldots, z_{2t}\rangle$. If $U_1, \ldots, U_t$ are $N \times N$ Haar random unitaries, then almost surely,
    \begin{align*}
     \ \tr_N[p(U_1, \ldots, U_t, U^*_1, \ldots, U_t^*)] &~\xrightarrow[N \to \infty]{}~ \phi[p(u_1, \ldots, u_t, u^*_1, \ldots, u^*_t)],\\
    \  \|p(U_1, \ldots, U_t, U^*_1, \ldots, U_t^*)\|_{\op} &~\xrightarrow[N \to \infty]{}~ \| p(u_1, \ldots, u_t, u^*_1, \ldots, u^*_t)\|,
    \end{align*}
    where $u_1, \ldots, u_t$ are free Haar unitaries in a $C^*$-probability space $(\CA, \phi)$ and $\|\cdot\|$ is the norm for the underlying $C^*$-algebra.
\end{theorem}




Using the above result it suffices to consider free Haar unitaries in a $C^*$-probability space to compute the operator norm and trace of polynomials of random unitaries. For a non-commutative polynomial $p(z_1, \ldots, z_t) = \sum_{|\ui|\le d} c_{\ui}z_{\ui}$, denoting by $\|p\|_2 =  \left(\sum_{|\ui| \le d} |c_{\ui}|^2\right)^{1/2}$, one can show the following easily using techniques from free probability. 

\begin{lemma} \label{thm:trace}
    Let $p(z_1, \ldots, z_t) = \sum_{|\ui|\le d} c_{\ui}z_{\ui} $ be a non-commutative degree-$d$ polynomial in $\R\langle z_1, \ldots, z_t\rangle$ and $u_1, \ldots, u_t$ be free Haar unitaries in a $C^*$-probability space $(\CA, \phi)$. Then, 
     \[ \phi[p(u_1, \ldots, u_t) (p(u_1, \ldots, u_t))^*] =  \|p\|_2^2.\]
\end{lemma}

The above implies that $\tr_N[p(U_1, \ldots, U_t) (p(U_1, \ldots, U_t))^*]$ converges to $\|p\|_2^2$ almost surely as $N \to \infty$. We shall defer the proof of \lref{thm:trace} to \appref{sec:app}, but to aid our intuition we note here that since the  $U_i$'s are independent $N \times N$ Haar random unitaries, the expected value

\[ \BE\left[\tr_N[p(U_1, \ldots, U_t) (p(U_1, \ldots, U_t))^*\right] = \|p\|_2^2,\] 
{and from concentration of measure, it is natural to expect that it converges to the above value}. 


Similarly, to compute the operator norm of $p(U_1, \ldots, U_t)$ for Haar random unitaries one can instead study the norm of the polynomial evaluated on free Haar unitaries. Such bounds are easier to prove using the trace method since free independence imposes strong restrictions on the non-commutative moments. For instance, if $U_1$ and $U_2$ are independent $N \times N$ Haar random matrices, then $\BE[\tr_N(U_1U_2U^*_1U_2^*)]$ is non-zero (albeit quite small), while the corresponding trace evaluated on free Haar unitaries $u_1$ and $u_2$ is zero, that is $\phi(u_1u_2u^*_1u_2^*) = 0$. Thus, computing the trace $\phi[p(u_1,\ldots, u_t, u^*_1, \ldots, u_t^*)]$ reduces to handling the combinatorics of the patterns of $u_i$'s and $u_i^*$'s. 

In particular, we will rely on the following result that follows from the work of Kemp and Speicher \cite{KS05}  who consider the operator norm of homogeneous polynomials evaluated on free $R$-diagonal operators, a class that includes free Haar unitaries as well. We also remark that a bound where the right-hand side below is worse by a multiplicative $O(d^{1/2})$ factor also follows from the work of Haagerup\footnote{We note that Haagerup considered the more general case of polynomials in both $u_i$'s and $u^*_i$'s.}\cite{H78} who proved it in another context, predating even the introduction of free probability theory. 


\begin{theorem}[\cite{KS05}]
\label{thm:kemp-speicher}
    Let $p(z_1, \ldots, z_t) = \sum_{|\ui| = d} c_{\ui}z_{\ui} $ be a homogeneous non-commutative degree-$d$ polynomial in $\R\langle z_1, \ldots, z_t\rangle$ and $u_1, \ldots, u_t$ be free Haar unitaries in a $C^*$-probability space. Then, 
    \[ 
    \|p(u_1, \ldots, u_t)\| \le \sqrt{e(d+1)} \cdot \|p\|_2,
    \]
    where the left-hand side denotes the norm in the underlying $C^*$-algebra. 
\end{theorem}

For completeness, we  introduce the necessary free probability background and some combinatorial details in \appref{sec:app}, and we present the fairly short proof of \thmref{thm:kemp-speicher} (from \cite{KS05}) there in a self-contained way. We shall need to extend the above bound to non-homogeneous polynomials. Let $p(z_1, \ldots, z_t) = \sum_{|\ui| \le d} c_{\ui}z_{\ui}$ and  let $p_k(z_1, \ldots, z_t) = \sum_{|\ui| = k} c_{\ui}z_{\ui}$ denote the degree-$k$ homogeneous part of $p$. Writing $p_k = p_k(u_1, \ldots, u_t)$ for $0 \le k  \le d$ and $p = p(u_1, \ldots, u_t)$, it follows from the triangle inequality,  \thmref{thm:kemp-speicher}, and Cauchy-Schwarz, that
    \begin{align*}
        \ \|p\| &\le \sum_{k=0}^d \|p_k\| 
        \le 
        \sum_{k=0}^d\sqrt{e(k+1)}\|p_k\|_2
        \le
       \sqrt{e}\left(\sum_{k=0}^d (k+1)\right)^{1/2} \left(\sum_{k=0}^d  \|p_k\|^2_2\right)^{1/2} \leq \sqrt{e}(d+1)  \cdot\|p\|_2.
    \end{align*}
Thus, we essentially get the same bound as in the homogeneous case, at the expense of an additional $O(d^{1/2})$ factor.



Collecting all the above we have the following as a direct consequence:

\begin{theorem} \label{thm:op-norm}
    Let $p(z_1, \ldots, z_t) = \sum_{|\ui|\le d} c_{\ui}z_{\ui} $ be a non-commutative degree-$d$ polynomial in $\R\langle z_1, \ldots, z_t\rangle$ and $U_1, \ldots, U_t$ be independent $N \times N$ Haar random unitaries. Then, as $N \to \infty$, the following holds almost surely, 
    \[ \tr_N[p(U_1, \ldots, U_t) (p(U_1, \ldots, U_t))^*] =  \|p\|_2^2,\]
    and
    \[ \|p(U_1, \ldots, U_t)\|_{\op} \le \sqrt{e}(d+1)  \cdot \|p\|_2,\]
    Moreover, the factor $(d+1)$ in the operator norm bound can be improved to $\sqrt{d+1}$ if the polynomial is homogeneous.
\end{theorem}

Based on the above theorem, we prove the following key lemma which captures the polar decomposition strategy mentioned in the earlier proof overview (\secref{sec:bh}). This will serve as the key ingredient in the proof of \thmref{thm:aa} and \thmref{thm:bh-intro}. 

\begin{lemma}\label{lem:polar}
    Let $p$ be a non-commutative degree-$d$ polynomial in $\R\langle y_1, \ldots, y_m, z_1, \ldots, z_t\rangle$ given by
    \[ p(y_1, \ldots, y_m, z_1, \ldots, z_t) = \sum_{i=1}^m y_i q_i(z_1, \ldots, z_t) + q_0(z_1, \ldots, z_t).\]
    Then, for every $\delta > 0$, there exist an integer $N$ and $N \times N$ unitaries $V_1,\ldots, V_m, W_1, \ldots, W_t$ such that 
    \[ \|p(V_1, \ldots, V_m, W_1, \ldots, W_t)\|_{\op} \ge \frac{1}{\sqrt{e}(d+1)} \sum_{i=1}^m \|q_i\|_2 - \delta.\]
    Moreover, the factor in front can be improved to $(e(d+1))^{-1/2}$ if $p$ is homogeneous. 
\end{lemma}

\begin{proof}[Proof of \lref{lem:polar}]
     For an arbitrary integer $N$, let us pick independent $N \times N$ Haar random unitaries $W_1, \ldots, W_t$ which we substitute for the variables $z_1,\ldots,z_t$, respectively, and let $M_i = q_i(W_1, \ldots, W_t)$ be the corresponding random matrices. Then, for any tuple of matrices $V_1, \ldots, V_m$ that we could substitute for the variables $y_1, \ldots, y_m$, we have that 
    \[ 
    p(V_1, \ldots, V_m, W_1, \ldots, W_t) = \sum_{i=1}^m V_i M_i + M_0.
    \] 
     \thmref{thm:op-norm} and union bound imply that as $N \to \infty$, with probability $1$ all the following events simultaneously hold: 
    \begin{itemize}
        \item $\|M_i\|_{\op} \le \sqrt{e}(d+1) \cdot \|q_i\|_2$ for each $i$,
        \item $\tr_N(M^*_iM_i) = \|q_i\|_2^2$ for each $i$, where $\tr_N(M)$ is the normalized trace.
    \end{itemize}
   To show that the operator norm must be large, let us fix a sufficiently large $N$ and a choice of $N\times N$ unitaries $W_1, \ldots, W_t$ such that $M_i$ satisfies $\|M_i\|_{\op} \le \sqrt{e}(d+1) \cdot \|q_i\|_2 + \epsilon$ and $\tr_N(M^*_iM_i) \ge \|q_i\|_2^2 - \epsilon$ for each $0\le i\le m$, where $\epsilon$ can be made arbitrarily small by increasing $N$. For $0 \leq i \leq m$, let $M_i = U_i P_i$ be the left polar decomposition of $M_i$, where $U_i$ is a unitary matrix and $P_i$ is a positive semidefinite matrix.
   
   We select the tuple of unitary matrices $V_1, \ldots, V_m$ that we substitute for the variables $y_1, \ldots, y_m$ to be $V_i = U_0U^*_i$ for $i \in [m]$. With this we have that $\|p(V_1, \ldots, V_m, W_1, \ldots, W_t)\|_{\op}$ is at least
    \begin{align*}
         \Big\|M_0 + \sum_{i=1}^m V_iM_i\Big\|_{\op} & = \Big\|U_0 P_0 + \sum_{i=1}^m U_0 U_i^* U_iP_i \Big\|_{\op} \\
        \ & =  \Big\|U_0 P_0 + \sum_{i=1}^m U_0 P_i\Big\|_{\op}  = \Big\| P_0 + \sum_{i=1}^m  P_i\Big\|_{\op}\ge \tr_N\Big(P_0 + \sum_{i=1}^m P_i\Big) \ge \tr_N\Big(\sum_{i=1}^m P_i\Big),
    \end{align*}
    where the last equality follows since the operator norm is unitarily invariant and the last two inequalities follow from the positive semidefiniteness of the $P_i$'s.

    For every positive semidefinite matrix $P$, we have that $\tr_N(P) \ge {\tr_N(P^2)}/{\|P\|_{\op}}$. 
  
    Hence,
     \[ \|p(V_1, \ldots, V_m, W_1, \ldots, W_t)\|_{\op} \ge \sum_{i=1}^m \frac{\tr_N(P_i^2)}{\|P_i\|_{\op}}.\]
     By our choice of $M_i$, we have that $\tr_N(P_i^2) = \tr_N(M_i^* M_i) \ge \|q_i\|_2^2 - \eps$ and $\|P_i\|_{\op} = \|M_i\|_{\op} \le \sqrt{e}(d+1)\|q_i\|_2 + \eps$. Since $\eps$ can be made arbitrarily small by increasing $N$, it follows that 
      \[ \|p(V_1, \ldots, V_m, W_1, \ldots, W_t)\|_{\op} \ge \frac1{\sqrt{e}(d+1)} \sum_{i=1}^m \|q_i\|_2 - \delta ,\]
     for large enough $N$. The improved bound for the homogeneous case follows directly by plugging the bound of \thmref{thm:op-norm} into the above proof.
\end{proof}





\subsection{Non-commutative root-influence inequality}
\label{sec:bh-proof}


For clarity in the proofs below, we remind our  convention that all tuples or blocks are denoted with boldface fonts (e.g. $\BU_1$ or $\BA$), while a single element is denoted without boldface (e.g. $U_1(i)$ or $A_i$ or $A$). Before proceeding with the proof, we restate the statement for convenience.

\bh*





\begin{proof}[Proof of \thmref{thm:bh-intro}] 
Since $f$ is homogeneous, we can write
   \begin{align*}
    f(\x_1,\ldots, \x_d) &= \sum_{i_1, \ldots, i_d \in [n]} \hf_{i_1, \ldots, i_d} ~x_1(i_1)x_2(i_2)\cdots x_d({i_d}) \\
    \ & = \sum_{i=1}^n  x_1(i) \underbrace{\left(\sum_{i_2,\ldots, i_d \in [n]} \hf_{i_1, \ldots, i_d} ~x_2(i_2)\cdots x_d({i_d})\right)}_{\textstyle := f_i(\x_2,\ldots, \x_d)}.
\end{align*}
 In this case, it follows from \eqref{eqn:inf-tensor} that for each $i \in [n]$, we have 
 \begin{equation}\label{eqn:var}
     \ \Var[f_i] = \|f_i\|^2_2 = \inf_{1,i}(f) \text{ and }  \Var[f] = \sum_{i=1}^n \inf_{1,i}(f).
 \end{equation}

  Let us denote the corresponding non-commutative block-multilinear polynomials by $f(\BU_1, \ldots, \BU_d)$ and $f_i(\BU_2, \ldots,\BU_d)$ where $\BU_b = (U_b(1), \ldots, U_b(n))$ denotes the $b^\text{th}$ block of non-commutative variables. To show a lower bound on $\cbnorm{f}$ it suffices to exhibit a collection of square matrices $\{U_b(i)\}_{b\in [d], i \in [n]}$ with operator norm at most~1, such that $\|f(\BU_1, \ldots, \BU_d)\|_{\op}$ is large. 
  
%  

Applying \lref{lem:polar} for the homogeneous case (with $p = f$, $q_i=f_i$ for $i \in [n]$, and $q_0=0)$, it follows that for every $\delta > 0$ there exists an integer $N$ and a choice of tuples of $N \times N$ unitaries $\BU_1, \ldots, \BU_d$ such that  
      \[ \cbnorm{f} \ge \|f(\BU_1, \ldots, \BU_d)\|_{\op} \ge \frac1{\sqrt{e(d+1)}} \sum_{i\in [n]} \|f_i\|_2  -\delta \stackrel{\eqref{eqn:var}}{\ge}  \frac{1}{\sqrt{e(d+1)}} \left(\sum_{i=1}^n \sqrt{\Inf_{1,i}(f)} \right) -\delta.\]
Taking $\delta \to 0$, we get the statement of the lemma. The proof for the inequality when $b=d$ is the last block follows similarly by using the right polar decomposition.
\end{proof}

\subsection{Aaronson-Ambainis Conjecture for non-homogeneous forms}

In this section, we prove \thmref{thm:aa}, which requires handling non-homogeneous forms. The proof will be similar to the proof of \thmref{thm:bh-intro} but we will need to be careful about certain details. 

\begin{proof}[Proof of \thmref{thm:aa}]
Any block-multilinear polynomial $f(x_1, \ldots, x_d)$ can be written as 
\begin{align*}
    f(\x_1,\ldots, \x_d) &= \BE f + \sum_{b\in [d]} f_b(\x_b, \x_{b+1}, \ldots, \x_d),
\end{align*}
where $f_b$ consists of all monomials of $f$ that start with a variable in the $b^\text{th}$ block $\x_b$. Note that $f_b$ depends only on the variables in blocks $\x_b, \x_{b+1},\ldots, \x_d$. Moreover, it follows from \eqref{eqn:inf-tensor} that 
 \begin{equation}\label{eqn:var-general}
     \ \Var[f] = \sum_{b \in [d]} \|f_b\|_2^2 = \sum_{b \in [d]} \Var[f_b],
 \end{equation}
so there exists a block $\beta \in [d]$ such that $\Var[f_{\beta}] \ge \frac{1}{d}\Var[f]$. 

Since $f_{\beta}$ contributes a lot to the variance, it is natural to try to find an influential variable in the block $\x_{\beta}$. Towards this end,  we pull out the variables $x_{\beta}(i)$ and write
\begin{align*}
    f_{\beta}(\x_{\beta},\ldots, \x_d) &= \sum_{i\in [n]} x_{\beta}(i) f_{\beta,i}(\x_{\beta+1}, \ldots, \x_d),
\end{align*}
for block-multilinear polynomials $f_{\beta,i}(\x_{\beta+1}, \ldots, \x_d)$. Note that some of the $f_{\beta,i}$'s could be identically zero, so let us define $S$ to be the set of those $i$ such that $f_{\beta,i}$ is non-zero. We note that
\begin{align} \label{eqn:part-inf}
  \|f_{\beta,i}\|_2^2  =  \Inf_{\beta,i}(f_{\beta}) \le \Inf_{\beta,i}(f)  
\end{align}
which implies that
\begin{align}\label{eqn:var-main}
    \frac{1}{d} \Var[f] \le \Var[f_{\beta}] = \sum_{i \in S}\|f_{\beta,i}\|_2^2 = \sum_{i \in S} \Inf_{\beta,i}(f_{\beta}).
\end{align}
\begin{sloppypar}
Denote the corresponding non-commutative block-multilinear polynomials by $f(\BU_1, \ldots, \BU_d)$,  $f_b(\BU_{b}, \ldots,\BU_d)$, and $f_{\beta}(\BU_{\beta+1}, \ldots,\BU_d)$ where $\BU_b = (U_b(1), \ldots, U_b(n))$ denotes the $b^\text{th}$ block of non-commutative variables. To show a lower bound on $\cbnorm{f}$ it suffices to exhibit a collection of square matrices $\{U_b(i)\}_{b\in [d], i \in [n]}$ with operator norm at most~1 such that $\|f(\BU_1, \ldots, \BU_d)\|_{\op}$ is large.
\end{sloppypar}
  
 We set the matrices in blocks $\BU_1, \ldots, \BU_{\beta-1}$ to be zero (that is, the all-zero matrix $\BZ$). Note that with this choice all polynomials $f_b(\U_b, \ldots, \U_d)$ where $b < \beta$ vanish and the non-commutative polynomial becomes 
 \[ f(\BZ, \ldots, \BZ, \BU_{\beta}, \BU_{\beta+1}, \ldots, \BU_d) = \sum_{i\in S} U_{\beta}(i) f_{\beta,i}(\BU_{\beta+1}, \ldots, \BU_d) + \sum_{b=\beta+1}^d f_b(\BU_b, \BU_{b+1}, \ldots, \BU_d) + \Ef,\]
  which is a non-commutative polynomial of the form considered in \lref{lem:polar} (with $m = |S|$, $q_i = f_{\beta,i}$ and $q_0 = \sum_{b=\beta+1}^d f_b + \Ef$). Thus, by \lref{lem:polar} for every small $\delta>0$ there exists an integer $N$ and a choice of $N \times N$ matrices for the blocks $\BU_{\beta},\ldots, \BU_d$ such that 
        \begin{align*}
             \ \cbnorm{f} & \ge \|f(\BZ, \ldots, \BZ, \BU_{\beta}, \BU_{\beta+1}, \ldots, \BU_d)\|_{\op} & \\
             \  & \ge \frac1{\sqrt{e}(d+1)} \sum_{i\in S} \|f_{\beta,i}\|_2 -\delta  \stackrel{\eqref{eqn:part-inf}}{=}  \frac{1}{\sqrt{e}(d+1)} \left(\sum_{i \in S} \sqrt{\Inf_{\beta,i}(f_{\beta})} \right) -\delta & \\
             \ &\stackrel{\eqref{eqn:var-main}}{\ge}  \frac{1}{\sqrt{e}(d+1)} \left( \frac{\sum_{i \in S} \Inf_{\beta,i}(f_{\beta})}{\sqrt{\maxinf(f)}} \right) -\delta  \stackrel{\eqref{eqn:part-inf}}{\ge}  \frac{1}{\sqrt{e}(d+1)^{2}} \left( \frac{\Var[f]}{ \sqrt{\maxinf(f)}} \right) -\delta
        \end{align*}
        Taking $\delta \to 0$ and using the assumption that $\|f\|_{\cb} \le 1$, we obtain the statement of the theorem:
     \[
     1\geq \cbnorm{f} \ge \frac{1}{\sqrt{e}(d+1)^{2}} \cdot \frac{\Var[f]}{\sqrt{\maxinf(f)}} \implies \maxinf(f) \ge  \frac{(\Var[f])^2}{e(d+1)^4}. \qedhere
     \]
\end{proof}
 

     
     

\subsection{Approximating completely bounded forms with decision trees}



In this section, we briefly mention how to obtain \corref{cor:sim}.
Aaronson and Ambainis \cite[Theorem 3.3]{AA14} showed that querying the most influential variable reduces the variance of the function~$f$, and if that influence is lower bounded by a polynomial in $\Var[f]/d$, then after $\poly(d)$ queries (the exact quantitative dependence can be read off from their proof), the variance of the function becomes small enough so that it can be approximated almost-everywhere by its expectation.  Since the family of degree-$d$ block-multilinear forms with completely bounded norm at most one is closed under restrictions, one can apply \thmref{thm:aa} repeatedly. This gives us \corref{cor:sim}.
\section{Search Space Coverage} \label{SearchSpaceCoverage}
We have proven the safety of the proposed method of choosing control points. Now we need to analyze the advantage of the proposed method, i.e. more search space coverage. From theorem \ref{theorem:baseTheorem} we known that $\sum_{i=0}^{n}f(\frac{i}{n})B\textsupsub{$n$}{$i$}(t)$ ($f(t)$ is the concave upper bound function) is the highest possible Bezier curve. It suffices to analyze the difference between the highest possible Bezier curve and $f(t)$. We are going to prove that the difference is $O(\frac{1}{n^{2}})$ if $f(t)$ is twice differentiable.

\begin{lemma}
For two twice differentiable functions $f_{1}(t)$ and $f_{2}(t)$ defined in $\left[t_{1}, t_{2}\right]$, if $\exists t_{0} \in \left[t_{1}, t_{2}\right]$ such that $f_{1}(t_{0}) = f_{2}(t_{0})$ and $f^{\prime}_{1}(t_{0}) = f^{\prime}_{2}(t_{0})$ , we have $f_{1}(t) - f_{2}(t)$ = $O((t_{2}-t_{1})^2)$.
\label{lemma:simpleTangentO2Diff}
\end{lemma}

\begin{proof}
The proof is trivial by applying Taylor's formula on $f_{2}(t) - f_{1}(t)$.
\end{proof}

\begin{lemma}
If $g_{1}(t)$ is a twice differentiable function defined in $\left[t_{1}, t_{2}\right]$, and $g_{2}(t)$ is the linear function passing $(t_{1}, g_{1}(t_{1}))$ and $(t_{2}, g_{1}(t_{2}))$, then $g_{1}(t) - g_{2}(t)$ $=$ $O((t_{1} - t_{2})^{2})$  
\label{lemma:2ndOrderDiff}
\end{lemma}

\begin{proof}
Note $h(t) = g_{1}(t) - g_{2}(t)$, we have obviously $h(t_{1}) = h(t_{2})=0$ and $h(t)$ is twice differentiable. $\forall t \in \left[t_{1}, t_{2}\right]$, from Taylor's formula we have
{
\setlength\abovedisplayskip{1pt}
\setlength\belowdisplayskip{1pt}
\begin{equation}\begin{split}
0=h(t_{1}) = h(t) + h^{(1)}(t)(t_{1} - t) + O((t_{2}-t_{1})^{2})
\label{eq:Taylor1}
\end{split}\end{equation}
}
{
\setlength\abovedisplayskip{1pt}
\setlength\belowdisplayskip{1pt}
\begin{equation}\begin{split}
0=h(t_{2}) = h(t) + h^{(1)}(t)(t_{2} - t) + O((t_{2}-t_{1})^{2})
\label{eq:Taylor2}
\end{split}\end{equation}
}
Calculating the difference of (\ref{eq:Taylor1}) and (\ref{eq:Taylor2}) yields
{
\setlength\abovedisplayskip{1pt}
\setlength\belowdisplayskip{1pt}
\begin{equation}\begin{split}
0=h^{(1)}(t)(t_{1}-t_{2})+O((t_{2}-t_{1})^{2})
\label{eq:Taylor3}
\end{split}\end{equation}
}
From (\ref{eq:Taylor3}) we know that $h^{(1)}(t)(t_{1}-t_{2})$ = $O((t_{2}-t_{1})^{2})$, so $h^{(1)}(t)(t_{1} - t)$ = $O((t_{2}-t_{1})^{2})$ $\forall t \in \left[t_{1}, t_{2}\right]$. From (\ref{eq:Taylor1}) we know that $h(t)$ = $-h^{(1)}(t)(t_{1} - t)$ + $O((t_{2}-t_{1})^{2})$ = $O((t_{2}-t_{1})^{2})$. This concludes the proof.
\end{proof}

\begin{corollary}
Suppose $s_{i}$ is a concave (convex) series, the Bezier curve $C(t)=\sum_{i=0}^{n}s_{i}B\textsupsub{$n$}{$i$}(t)$ is inferior than the first and the last segment of the CPETS, and the difference is $O(\frac{1}{n^{2}})$.
\label{corollary:tangentFistLastPointDiff}
\end{corollary}

\begin{proof}
The proof is trivial by applying corollary \ref{corollary:tangentFistLastPoint} and lemma \ref{lemma:simpleTangentO2Diff}.
\end{proof}

\begin{corollary}
If the concave upper bound function $f(t)$ is twice differentiable, the difference between $f(t)$ and the CPETS is $O(\frac{1}{n^{2}})$, where $n$ is the number of control points.
\label{corollary:diffUpperBoundCPETS}
\end{corollary}

\begin{proof}
The proof is obvious by applying lemma \ref{lemma:2ndOrderDiff} on interval $\left[\frac{i}{n}, \frac{i+1}{n}\right]$ ($i=0,\;1,\;...\;,\;n-1$).
\end{proof}

With corollary \ref{corollary:diffUpperBoundCPETS}, it suffices to prove that the difference between the highest possible Bezier curve $\sum_{i=0}^{n}f(\frac{i}{n})B\textsupsub{$n$}{$i$}(t)$ and the CPETS is $O(\frac{1}{n^{2}})$.

\begin{lemma}
$\forall n > 1$, if the upper bound function $f(t)$ defined in $\left[0,1\right]$ is concave and twice differentiable, for series $s_{i}=f(\frac{i}{n})$, $\forall j \in \left[0, n-1\right]$ and 
$\forall t \in \left[\frac{j}{n}, \frac{j+1}{n}\right]$, note $C(t)=\sum_{i=0}^{n}s_{i}B\textsupsub{$n$}{$i$}(t)$, we have $C(t)$ $-$ $(s_{j}+n(s_{j+1}-s{j})(t-\frac{j}{n}))$=$O(\frac{1}{n^{2}})$, i.e. the difference between the Bezier curve and CPETS is $O(\frac{1}{n^{2}})$.
\label{lemma:mainLemmaDiff}
\end{lemma}

\begin{proof}
We will prove it by recurrence, similar to the proof of lemma \ref{lemma:mainLemma}. For $n \le 2$, it is easy to verify that lemma \ref{lemma:mainLemmaDiff} holds using lemma \ref{lemma:simpleTangentO2Diff}. 

Suppose lemma \ref{lemma:mainLemmaDiff} holds for $n=k$ $(k > 2)$. When $n=k+1$, we only need to show that lemma \ref{lemma:mainLemmaDiff} holds for intervals $\left[\frac{1}{k+1}, \frac{2}{k+1}\right]$, $\left[\frac{2}{k+1}, \frac{3}{k+1}\right]$, ... , 
$\left[\frac{k-1}{k+1}, \frac{k}{k+1}\right]$, because for the first and the last interval, lemma \ref{lemma:mainLemmaDiff} holds due to corollary \ref{corollary:tangentFistLastPointDiff}.

With the assumption of recurrence, $\forall j \in \left[1, k-1\right]$ and 
$\forall t \in \left[\frac{j}{k}, \frac{j+1}{k}\right]$, using the recurrence definition of Bezier curve (\ref{eq:BezierRecurDefinition}), we have:
{
\setlength\abovedisplayskip{1pt}
\setlength\belowdisplayskip{1pt}
\begin{equation}\begin{split}
C(t) &= D(t) + O(\frac{1}{{k}^{2}}) \\
 &= D(t) + O(\frac{1}{(k+1)^{2}})
\label{eq:BezierRecurIniqualityDiff}
\end{split}\end{equation}
}
where $D(t)$ is defined in (\ref{eq:BezierRecurIniqualityRightSide}). One can use lemma \ref{lemma:simpleTangentO2Diff} to verify that 
{
\setlength\abovedisplayskip{1pt}
\setlength\belowdisplayskip{1pt}
\begin{equation}\begin{split}
D(t) = s_{j}+(k+1)(s_{j+1}-s_{j})(t - \frac{j}{k}) + O(\frac{1}{(k+1)^{2}})
\label{eq:shiftedInequalityLinearDiff}
\end{split}\end{equation}
}
$\forall t \in \left[\frac{j}{k+1}, \frac{j+1}{k+1}\right]$, note $\Delta t = \frac{j}{k} - \frac{j}{k+1} > 0$, one can verify that $t+\Delta t \in \left[\frac{j}{k}, \frac{j+1}{k}\right]$, and $\Delta t = O(\frac{1}{(k+1)^{2}})$. Thus $C(t)$ = $C(t+\Delta t) - C^{\prime}(t+\Delta t)\Delta t + o(\Delta t)$ = $C(t+\Delta t) + O(\frac{1}{(k+1)^{2}})$. From (\ref{eq:BezierRecurIniqualityDiff}), (\ref{eq:BezierRecurIniqualityRightSide}) and (\ref{eq:shiftedInequalityLinearDiff}), we have
{
\setlength\abovedisplayskip{1pt}
\setlength\belowdisplayskip{1pt}
\begin{equation}\begin{split}
C(t) &= C(t+\Delta t) + O(\frac{1}{(k+1)^{2}}) \\
&= D(t+\Delta t) + O(\frac{1}{(k+1)^{2}}) \\
&= s_{j}+(k+1)(s_{j+1}-s_{j})(t + \Delta t - \frac{j}{k}) \\
&+ O(\frac{1}{(k+1)^{2}}) \\
&= s_{j}+(k+1)(s_{j+1}-s_{j})(t - (\frac{j}{k} - \frac{j}{k} + \frac{j}{k+1})) \\
&+ O(\frac{1}{(k+1)^{2}}) \\
&= s_{j}+(k+1)(s_{j+1}-s_{j})(t - \frac{j}{k+1}) + O(\frac{1}{(k+1)^{2}})
\label{eq:shiftedInequalityDiff}
\end{split}\end{equation}
}
This concludes the proof for $n=k+1$. Lemma \ref{lemma:mainLemmaDiff} is thus proven.
\end{proof}

Lemma \ref{lemma:mainLemmaDiff} and corollary \ref{corollary:diffUpperBoundCPETS} show that the difference between the upper bound function and the highest possible Bezier curve is $O(\frac{1}{n^{2}})$, where $n$ is the number of control points.

\section{Simulation results} \label{Simulation}

\begin{figure}[tbp]
\begin{center}
\includegraphics[width=9cm]{comparison_of_performance.png}
\end{center}
\vspace{-0.2 in}
\caption{Comparison of performance. A) The shapes of two types of corridors. B) Planning result in general convex corridal. C) Planning result in trapezoidal corridor. D) Comparison of acceleration from two planning results.}
\label{Fig:comparisonOfPerformance}
\vspace{-0.2 in}
\end{figure}

In this section, the motion planning results based on the general convex corridors are compared with trapezoidal corridors-based planning. The same set of parameters (weights, time horizon) were used for both cases to make the results comparable. The scenario includes an ego vehicle and a decelerating front vehicle, similar to the scenario depicted in Fig. \ref{Fig:comparison_of_shapes} A1. The corresponding corridors are shown in Fig. \ref{Fig:comparisonOfPerformance} A. A desirable planning should consider the smoothness and the closeness between the planned trajectory and the reference trajectory. It is also reasonable to consider a higher weight for the trajectory points in the near future compared to the later-time points. Therefore, the cost function is defined as (\ref{eq:costFunction}):
{
\setlength\abovedisplayskip{1pt}
\setlength\belowdisplayskip{1pt}
\begin{equation}\begin{split}
&Cost=\\
&w_{00} \int_{0}^{T_{s}} (S(t)-S_{r}(t))^2 \,dx+w_{01} \int_{T_{s}}^{T_{l}} (S(t)-S_{r}(t))^2 \,dx \\
&+w_{10} \int_{0}^{T_{s}} (\ddot S(t)- \ddot S_{r}(t))^2 \,dx+w_{11} \int_{T_{s}}^{T_{l}} (\ddot S(t)- \ddot S_{r}(t))^2 \,dx \\ 
&+w_{20} \int_{0}^{T_{s}} (\dddot S(t))^2 \,dx+w_{21} \int_{T_{s}}^{T_{l}} (\dddot S(t))^2 \,dx, \\
\label{eq:costFunction}
\end{split}\end{equation}
}
where $w_{00}=5.0$, $w_{01}=2.5$, $w_{10}=10.0$, $w_{11}=3.0$, $w_{20}=25.0$, $w_{21}=10.0$, $T_{l}=20.0$, $T_{s}=6.0$. The terms $T_{l}$ and $T_{s}$ mean that the planning has an extra emphasis on the near future points. The initial speed of the planning is 5.5 $m/s$. $S_{r}(t)$ is a reference trajectory, which is generated by a uniformly accelerated motion with an acceleration of -0.05 $m/s^{2}$. Note that we do not require the reference trajectory to be safe here. A reference trajectory with high acceleration could be unsafe, but it may result in a planned trajectory with a higher acceleration. 

The motion planning problem is then converted into a QP (quadratic programming) problem, whose details are presented in~\cite{ding2019safe} \cite{li2021speed}. OSQP 0.5.1 is adopted as the solver to this problem. As shown in Fig.~\ref{Fig:comparisonOfPerformance}, the general convex corridor results in a larger search space, leading to much smaller deceleration compared to that of the trapezoidal corridor (-0.52 $m/s^{2}$ $vs$ -1.0 $m/s^{2}$) and subsequently less harsh brakes and improved smoothness.

To evaluate the performance of this algorithm in a real-time, multi-frame condition, a cut-in scenario is set up in our simulator (developed based on ROS). The simulation scenario is based on an real-world scenario encountered in road test. The ego vehicle is cruising under a speed limit of 60 $km/h$ (16.77 $m/s$\footnote{Due to the performance of the speed controller, the ego vehicle's actual speed is around 16 $m/s$.}), while another vehicle cuts in at $t=1.0$ s, at a distance of 22 $m$ and at a speed of 7 $m/s$. The cut-in vehicle decelerates at 0.5 $m/s^2$ for 3 seconds, then continues moving at a constant speed. There is an additional 5-meter safety margin behind the cut-in vehicle. The close-loop longitudinal dynamics of the ego vehicle is simulated by a first-order dynamics with a time constant of 0.3 $s$. The corridors are generated with the algorithm presented in~\cite{xin2021enable}~and~\cite{li2021speed}, respectively for general convex shape and trapezoidal shape corridors.

\begin{figure}[tbp]
\begin{center}
\includegraphics[width=9cm]{comparison_in_simulator.png}
\end{center}
\vspace{-0.2 in}
\caption{Comparison of performance in simulator. A) The cut-in scenario from top view. B) The lowest speed achieved with general convex corridor. C) The lowest speed achieved with trapezoidal corridor. D) Comparison of ego vehicle's speed from two planning results. E) Comparison of ego vehicle's acceleration from two planning results.}
\label{Fig:comparisonInSimulator}
\vspace{-0.2 in}
\end{figure}

The comparison is shown in Fig.~\ref{Fig:comparisonInSimulator}. The lowest deceleration is -4.46 $m/s^2$ using general convex corridors, compared to -5.23 $m/s^2$ of using trapezoidal corridors. In the case of using general convex corridors, the lowest speed is registered at 3.41 $m/s$, compared to that of trapezoidal corridors at 2.42 $m/s$. Our proposed approach decelerates earlier, which helps in reducing the peak value of deceleration and the length of the deceleration period as an indication of an improved overall comfort.
In this paper, 2D and 3D CNN models were used to generate pelvic sCTs from T1-weighted MR images. Our sCT generation methods were fully automated, requiring no deformable registration or manual segmentation of bone tissues. As shown in Figure~\ref{fig3}, the 2D and 3D CNN models generated high quality sCTs. MAE curves shown in Figure~\ref{fig4} indicated that both models could precisely estimate soft-tissue HU values but had difficulty in reproducing air and high-density bone tissues. 

The MAEs within the body contour across all patients were 40.5 $\pm$ 5.4 HU and 37.6 $\pm$ 5.1 HU for the 2D and 3D models, respectively. The time required for generating a pelvic sCT using our CNN models was about 5.5 s. Our MAE results are comparable to previous studies. Kim $et \ al.$\cite{RN41} presented a voxel-based weighted summation method that produced an MAE of 74.3 $\pm$ 3.9 HU. However, manual contouring of bone tissues required for this method can be tedious and time-consuming. An MAE of 40.5 $\pm$ 8.2 HU was achieved by Dowling $et \ al.$\cite{RN11} using an average MRI-CT atlas from 38 patients. Andreasen $et \ al.$\cite{RN42} reported an MAE of 54 $\pm$ 8 HU using an atlas-based method with pattern recognition, and its prediction time was about 20.8 min. Another random forest model proposed by Andreasen $et \ al.$\cite{RN43} generated sCTs with an MAE of 58 $pm$ 9 HU. A hybrid method suggested by Siversson $et \ al.$ \cite{RN45} obtained an MAE of 36.5 $\pm$ 4.1 HU when ignoring errors introduced by gas cavities. This hybrid method was implemented in the cloud-based commercial software MriPlanner (Spectronic Medical AB, Helsingborg, Sweden), which required 50 to 80 min to generate a sCT.\cite{RN45} The patch-based 3D context-aware generative adversarial network presented by Nie $et \ al.$\cite{RN26} achieved an MAE of 39.0 $\pm$ 4.6 HU. 

Our CNN models reproduced low-density bone as shown in Figure ~\ref{fig4}. The bone-region DSCs were 0.81 $\pm$ 0.04 and 0.82 $\pm$ 0.04 from the 2D and 3D models, respectively. These results are comparable to reported DSC results of 0.79 $\pm$ 0.12\cite{RN10} and 0.91$\pm$0.03{\cite{RN11}}, where the authors compared bone contours manually drawn on the sCT and CT.

It was feasible to train the proposed 3D model with 16 image volumes from scratch. Results of the Wilcoxon signed-rank tests shown in Table~\ref{tab1} demonstrated a statistically significant improvement in overall MAE, bone DSC, and bone precision of the 3D model compared to the 2D model. However, as shown in Figure~\ref{fig4}, the 2D model seemed to perform better in estimating the high-density bone HU values. It should be noted that smaller overall MAEs do not guarantee improved sCT dose calculation and patient positioning performance. While the models performed well, we will continue to acquire more patient data to potentially improve model accuracy and further test model differences.

As this was a retrospective study, the MR image voxel sizes were not matched, resulting in different voxel intensities between images. This may have affected the sCT generation accuracy although we applied intensity normalization. A potential study could examine how voxel size variations affects sCT estimation. 

The proposed 3D model can be implemented on a 12 GB GPU to process volumetric images with dimensions of 256 $\times$ 256 $\times$ 30. More GPU memory would be required to process higher resolution 3D images. Considering the limited access to multi-GPU systems, a 3D architecture with fewer convolutional layers could be considered to deal with higher resolutions. However, the performance could be affected by the reduced parameters and smaller receptive fields of the less complex model. Another approach would be to extract 30-slice sub-volumes from CT and MR images for training the 3D model. The sCT could then be generated by averaging 30-slice sCT sub-volumes produced by the model. 

A number of techniques could be investigated for improving model performance.  Nie $et \ al.$\cite{RN26} showed that introducing an additional adversarial discriminator improved overall sCT quality. The same approach could be adapted in our proposed 2D and 3D CNN models.  Non-rigid deformation\cite{RN44} could also be applied to both CT and MR images in the process of the on-the-fly data augmentation to produce more training pairs. Multiple MR images acquired with different sequences could be fed into models to provide more information for distinguishing different tissues. Multi-GPU systems with more memory would enable the exploration of larger batch sizes for training CNN models, which could reduce variances in gradient estimation and accelerate the training. 



\section{Discussion and Conclusions}



Our method based on stabilizing forward and backward pass, resulted in improved accuracy over the baseline and it was able to predict optimal dampening, sharpness and tail-fatness before training. 
Our findings are coherent with the line of research that has established that stabilizing gradients and representations at initialization results in better performance \cite{glorot2010understanding, orthogonal_initialization, he2015delving, roberts2022principles, defazio2022scaling, bengio1994learning, hochreiter1997long, hochreiter2001gradient, arjovsky2016unitary, pascanu2013difficulty}. Moreover it gives an initial reply to the question raised by
\cite{surrogate2019, zenke2021remarkable}, which asked  for a theoretical justification of initialization and SG choice for Spiking Neural Networks. With a similar intention, \cite{rossbroich2022fluctuation} proposed an approach that guarantees sparsity of activity at initialization to pick the weights distribution at initialization, resulting in improved accuracy. Our method differs from theirs in that it starts from a principle of stability to derive constraints, instead of a principle of sparsity. It differs also in that we use it to define the SG shape at initialization, not only the weights distribution, and we can show mathematically how weights initialization is intertwined to the SG shape choice. Our results suggest that a tedious hyper-parameter grid-search can be often avoided by making use of sound and established principles of learning.

One of the conditions was designed to hit the most sensitive part of an SG, its center, which resulted in a low sparsity requirement at initialization. This is very uncommon in the Neuromorphic literature, since sparsity brings large energy gains \cite{henderson2020towards,blouw2019benchmarking, 9395703,taulsnn, rossbroich2022fluctuation}.
However, the energy gains of SNNs also come from their binary activity. A matrix-vector multiplication, with a $\mathbb{R}^{m\times n}$ matrix, has an energy cost of $mnE_{MAC}$ for a real vector, and of $mn\rho E_{AC}$ for a binary vector, where $\rho$ is the Bernouilli probability of the binary vector, and in our case the neuron firing rate, and $E_{AC}, E_{MAC}$ are the energies of an accumulate and a multiply-accumulate operation \cite{yin2021accurate, hunger2005floating}. Since MAC are more costly than AC, 31 times on a $45$nm complementary metal–oxide–semiconductor \cite{yin2021accurate, horowitz20141}, we have energy savings with any $\rho$, e.g., when all neurons fire ($\rho=1$) and when they fire half of the time steps ($\rho=1/2$). This gain does not depend on the simulation speed, since it compares a spiking and an analogue computation, at the same computation speed.
Typically requiring more sparsity through a sparsity encouraging loss term, leads to a measurable decrease in performance \cite{zenke2021remarkable, rossbroich2022fluctuation}. However we observed that it is actually possible to achieve higher performance with higher sparsity, by starting with a strong firing rate at initialization, since their synergy acts as a regularization mechanism. This was possible also because the sparsity encouraging loss term was introduced gradually, and because its contribution was kept comparable to the task loss towards the end of training.

We observed that the more complex the task is and the more complex the network to train is, the more drastic is the difference in performance of different SG shapes. It is known that learning is possible with a wide variety of SG shapes \cite{zenke2021remarkable} and the community has not yet settled for one shape or one method to reliably choose which SG to use in each case \cite{surrogate2019}. We showed how to apply a well known stability principle to the forward and backward pass of the simplest Spiking Neural Network, the LIF, as a starting point, but we think that the principles of good Neuromorphic initialization can be further elaborated, in order to tackle more complex tasks and networks.




%%%%%%%%%%%%%%%%%%%%%%%%%%%%%%%%%%%%%%%%%%%%%%%%%%%%%%%%%%%%%%%%%%%%%%%%%%%%%%%
\bibliographystyle{IEEEtran}
\bibliography{references}

\end{document}