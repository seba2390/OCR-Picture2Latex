\documentclass[letterpaper, 10 pt, conference]{ieeeconf}
\IEEEoverridecommandlockouts
% The preceding line is only needed to identify funding in the first footnote. If that is unneeded, please comment it out.
\usepackage{cite}
\usepackage{amsmath,amssymb,amsfonts}
%\usepackage{amsthm}
\usepackage{algorithmic}
\usepackage{graphicx}
\usepackage{textcomp}
\usepackage{xcolor}
%\newtheorem{theorem}{Theorem}
%\newtheorem{corollary}{Corollary}[theorem]
%\newtheorem{lemma}[theorem]{Lemma}

\newtheorem{theorem}{Theorem}[section]
\newtheorem{corollary}{Corollary}[theorem]
\newtheorem{lemma}[theorem]{Lemma}
%\newtheorem*{remark}{Remark}

\def\BibTeX{{\rm B\kern-.05em{\sc i\kern-.025em b}\kern-.08em
    T\kern-.1667em\lower.7ex\hbox{E}\kern-.125emX}}

\makeatletter
\DeclareRobustCommand{\textsupsub}[2]{{%
  \m@th\ensuremath{%
    ^{\mbox{\fontsize\sf@size\z@#1}}%
    _{\mbox{\fontsize\sf@size\z@#2}}%
  }%
}}
\makeatother

%\long\def\/*#1*/{} %this is a macro for block comment

\title{{\LARGE \textbf{A Sufficient Condition for Convex Hull Property in General Convex Spatio-Temporal Corridors}}}

\author{Weize Zhang$^{1*}$, Peyman Yadmellat$^{1}$, and Zhiwei Gao$^{2}$
\thanks{$^{1}$Noah’s Ark Lab., Huawei Technologies Canada, Markham, Ontario,
Canada L3R 5A4.}
\thanks{$^{2}$Noah’s Ark Lab., Beijing Huawei Digital Technologies Co. Ltd.,
Beijing, China.}
\thanks{$^{*}$Correspondence: \texttt{\small {weize.zhang@huawei.com}}}
}

\begin{document}

\maketitle
\thispagestyle{empty}
\pagestyle{empty}


  In this paper, we explore the connection between secret key agreement and secure omniscience within the setting of the multiterminal source model with a wiretapper who has side information. While the secret key agreement problem considers the generation of a maximum-rate secret key through public discussion, the secure omniscience problem is concerned with communication protocols for omniscience that minimize the rate of information leakage to the wiretapper. The starting point of our work is a lower bound on the minimum leakage rate for omniscience, $\rl$, in terms of the wiretap secret key capacity, $\wskc$. Our interest is in identifying broad classes of sources for which this lower bound is met with equality, in which case we say that there is a duality between secure omniscience and secret key agreement. We show that this duality holds in the case of certain finite linear source (FLS) models, such as two-terminal FLS models and pairwise independent network models on trees with a linear wiretapper. Duality also holds for any FLS model in which $\wskc$ is achieved by a perfect linear secret key agreement scheme. We conjecture that the duality in fact holds unconditionally for any FLS model. On the negative side, we give an example of a (non-FLS) source model for which duality does not hold if we limit ourselves to communication-for-omniscience protocols with at most two (interactive) communications.  We also address the secure function computation problem and explore the connection between the minimum leakage rate for computing a function and the wiretap secret key capacity.
  
%   Finally, we demonstrate the usefulness of our lower bound on $\rl$ by using it to derive equivalent conditions for the positivity of $\wskc$ in the multiterminal model. This extends a recent result of Gohari, G\"{u}nl\"{u} and Kramer (2020) obtained for the two-user setting.
  
   
%   In this paper, we study the problem of secret key generation through an omniscience achieving communication that minimizes the 
%   leakage rate to a wiretapper who has side information in the setting of multiterminal source model.  We explore this problem by deriving a lower bound on the wiretap secret key capacity $\wskc$ in terms of the minimum leakage rate for omniscience, $\rl$. 
%   %The former quantity is defined to be the maximum secret key rate achievable, and the latter one is defined as the minimum possible leakage rate about the source through an omniscience scheme to a wiretapper. 
%   The main focus of our work is the characterization of the sources for which the lower bound holds with equality \textemdash it is referred to as a duality between secure omniscience and wiretap secret key agreement. For general source models, we show that duality need not hold if we limit to the communication protocols with at most two (interactive) communications. In the case when there is no restriction on the number of communications, whether the duality holds or not is still unknown. However, we resolve this question affirmatively for two-user finite linear sources (FLS) and pairwise independent networks (PIN) defined on trees, a subclass of FLS. Moreover, for these sources, we give a single-letter expression for $\wskc$. Furthermore, in the direction of proving the conjecture that duality holds for all FLS, we show that if $\wskc$ is achieved by a \emph{perfect} secret key agreement scheme for FLS then the duality must hold. All these results mount up the evidence in favor of the conjecture on FLS. Moreover, we demonstrate the usefulness of our lower bound on $\wskc$ in terms of $\rl$ by deriving some equivalent conditions on the positivity of secret key capacity for multiterminal source model. Our result indeed extends the work of Gohari, G\"{u}nl\"{u} and Kramer in two-user case.
%\begin{IEEEkeywords}
%motion planning, convex corridor, time dimension, convex hull property, smoothness
%\end{IEEEkeywords}
% \leavevmode
% \\
% \\
% \\
% \\
% \\
\section{Introduction}
\label{introduction}

AutoML is the process by which machine learning models are built automatically for a new dataset. Given a dataset, AutoML systems perform a search over valid data transformations and learners, along with hyper-parameter optimization for each learner~\cite{VolcanoML}. Choosing the transformations and learners over which to search is our focus.
A significant number of systems mine from prior runs of pipelines over a set of datasets to choose transformers and learners that are effective with different types of datasets (e.g. \cite{NEURIPS2018_b59a51a3}, \cite{10.14778/3415478.3415542}, \cite{autosklearn}). Thus, they build a database by actually running different pipelines with a diverse set of datasets to estimate the accuracy of potential pipelines. Hence, they can be used to effectively reduce the search space. A new dataset, based on a set of features (meta-features) is then matched to this database to find the most plausible candidates for both learner selection and hyper-parameter tuning. This process of choosing starting points in the search space is called meta-learning for the cold start problem.  

Other meta-learning approaches include mining existing data science code and their associated datasets to learn from human expertise. The AL~\cite{al} system mined existing Kaggle notebooks using dynamic analysis, i.e., actually running the scripts, and showed that such a system has promise.  However, this meta-learning approach does not scale because it is onerous to execute a large number of pipeline scripts on datasets, preprocessing datasets is never trivial, and older scripts cease to run at all as software evolves. It is not surprising that AL therefore performed dynamic analysis on just nine datasets.

Our system, {\sysname}, provides a scalable meta-learning approach to leverage human expertise, using static analysis to mine pipelines from large repositories of scripts. Static analysis has the advantage of scaling to thousands or millions of scripts \cite{graph4code} easily, but lacks the performance data gathered by dynamic analysis. The {\sysname} meta-learning approach guides the learning process by a scalable dataset similarity search, based on dataset embeddings, to find the most similar datasets and the semantics of ML pipelines applied on them.  Many existing systems, such as Auto-Sklearn \cite{autosklearn} and AL \cite{al}, compute a set of meta-features for each dataset. We developed a deep neural network model to generate embeddings at the granularity of a dataset, e.g., a table or CSV file, to capture similarity at the level of an entire dataset rather than relying on a set of meta-features.
 
Because we use static analysis to capture the semantics of the meta-learning process, we have no mechanism to choose the \textbf{best} pipeline from many seen pipelines, unlike the dynamic execution case where one can rely on runtime to choose the best performing pipeline.  Observing that pipelines are basically workflow graphs, we use graph generator neural models to succinctly capture the statically-observed pipelines for a single dataset. In {\sysname}, we formulate learner selection as a graph generation problem to predict optimized pipelines based on pipelines seen in actual notebooks.

%. This formulation enables {\sysname} for effective pruning of the AutoML search space to predict optimized pipelines based on pipelines seen in actual notebooks.}
%We note that increasingly, state-of-the-art performance in AutoML systems is being generated by more complex pipelines such as Directed Acyclic Graphs (DAGs) \cite{piper} rather than the linear pipelines used in earlier systems.  
 
{\sysname} does learner and transformation selection, and hence is a component of an AutoML systems. To evaluate this component, we integrated it into two existing AutoML systems, FLAML \cite{flaml} and Auto-Sklearn \cite{autosklearn}.  
% We evaluate each system with and without {\sysname}.  
We chose FLAML because it does not yet have any meta-learning component for the cold start problem and instead allows user selection of learners and transformers. The authors of FLAML explicitly pointed to the fact that FLAML might benefit from a meta-learning component and pointed to it as a possibility for future work. For FLAML, if mining historical pipelines provides an advantage, we should improve its performance. We also picked Auto-Sklearn as it does have a learner selection component based on meta-features, as described earlier~\cite{autosklearn2}. For Auto-Sklearn, we should at least match performance if our static mining of pipelines can match their extensive database. For context, we also compared {\sysname} with the recent VolcanoML~\cite{VolcanoML}, which provides an efficient decomposition and execution strategy for the AutoML search space. In contrast, {\sysname} prunes the search space using our meta-learning model to perform hyperparameter optimization only for the most promising candidates. 

The contributions of this paper are the following:
\begin{itemize}
    \item Section ~\ref{sec:mining} defines a scalable meta-learning approach based on representation learning of mined ML pipeline semantics and datasets for over 100 datasets and ~11K Python scripts.  
    \newline
    \item Sections~\ref{sec:kgpipGen} formulates AutoML pipeline generation as a graph generation problem. {\sysname} predicts efficiently an optimized ML pipeline for an unseen dataset based on our meta-learning model.  To the best of our knowledge, {\sysname} is the first approach to formulate  AutoML pipeline generation in such a way.
    \newline
    \item Section~\ref{sec:eval} presents a comprehensive evaluation using a large collection of 121 datasets from major AutoML benchmarks and Kaggle. Our experimental results show that {\sysname} outperforms all existing AutoML systems and achieves state-of-the-art results on the majority of these datasets. {\sysname} significantly improves the performance of both FLAML and Auto-Sklearn in classification and regression tasks. We also outperformed AL in 75 out of 77 datasets and VolcanoML in 75  out of 121 datasets, including 44 datasets used only by VolcanoML~\cite{VolcanoML}.  On average, {\sysname} achieves scores that are statistically better than the means of all other systems. 
\end{itemize}


%This approach does not need to apply cleaning or transformation methods to handle different variances among datasets. Moreover, we do not need to deal with complex analysis, such as dynamic code analysis. Thus, our approach proved to be scalable, as discussed in Sections~\ref{sec:mining}.
%The following section provides a brief survey on test compliance tracking, radio-based sensing/verification and co-presence authentication.

\subsection{Test Compliance Tracking}

The OSA test compliance tracking works as follows: Prior to the test, the doctor secures a tracker to the identified patient. Once attached, the tracker continuously observes and records the patient's vital signs until the test is finished. During the follow-up visit, the compliance data is compared with the screening data recorded by the OSA test equipment. If there were significant gaps, during which the compliance tracker fails to observe the patient's vital signs or significant inconsistency between the compliance data and the OSA screening data, the test would be nullified. The initial assessment ensures the patient never removes the tracker, and the subsequent data analysis verifies the identification information between the patient wearing the tracker and subject undergoing the test. 

Existing industry solutions utilize skin-contacting sensors to execute test compliance tracking \cite{noauthor_httpspatentsgooglecompatentus8679012b1en_nodate, noauthor_httpwwwsleepreviewmagcom201806tech-fraudulent-sleep-data_nodate}. These approaches are reliable and have minimal chance of false identifications. However, the patient may feel discomfort due to the attached sensors, which hampers the confidence of the test results. Several relevant experiments achieve identity verification by remotely monitoring and verifying the subject's physiological or behavioral traits. Common methods use imaging or audio sensors to perform facial-based or voice-based recognition \cite{hutchison_robust_2005, li_sound-based_2010,abushariah_voice_2012} or  authenticate the subject through behavioral characteristics, such as tactile dynamics and gait patterns\cite{collins_silhouette-based_2002,connor_biometric_2018}. Though essential, these methods are not suitable in addressing our problem due to the low-light and low-sound conditions and the subject's state of consciousness during such sleep studies.

\subsection{Radio-based Identity Verification}
Radio-based identity verification is one of among a few promising directions in addressing the aforementioned issues. It obtains unique physiological traits from radio signals reflected from the subject and does not require the subject's active involvement or ambient conditions unsuitable for the sleeping study. Several research works employ Doppler radar measurement of cardiopulmonary motion at decimeter or mmWave band for continuous user authentication \cite{ChenDopplerSignaturesRadar2008,MolchanovTargetClassificationUsing2011,RahmanNoncontactDopplerRadar2016,VanDorpFeaturebasedHumanMotion2008}. Others utilize the channel measurement protocols inherent to off-the-shelf WiFi devices to extract unique features for individual identification \cite{AbdelnasserUbiBreatheUbiquitousNoninvasive2015, LiuTrackingVitalSigns2015, ZhangWifiidHumanIdentification2016}. Recent development further incorporates advanced signal processing algorithms and machine learning techniques to improve the identification accuracy and reliability \cite{LiuContinuousUserVerificationb,IslamIdentityAuthenticationOSA2020}.

Despite many advances, two fundamental problems associated with our setting remain unsolved. First, %although the identity verification process operates on non-volitional features and require little involvement from the target users, 
the challenges to establish the connection between the radio signature and the subject's identity are largely omitted in existing work. In other words, the initial enrollment, during which the system captures the subject's physiological measurements to be compared with the traits extracted from radio signals, is contingent upon the assurance of the user's identity, which must be verified in a more reliable method. Second, the results in existing work are mostly obtained through single-subject experiments under controlled settings \cite{rahman_doppler_2018,lin_cardiac_2017,shi_contactless_2018,islam_identity_2019}. The challenges to apply radio-based approaches in complex environments subject to disruptive events and multiple targets, e.g., scenarios mostly encountered for in-home sleep arrangements, remain to be addressed.

\subsection{Co-Presence Authentication}
Another group of authentication methods applicable to test compliance tracking is co-presence verification, through which authenticator-certified devices perceive roughly the same ambient context via their on-board sensors. Context-based co-presence verification has been a long-standing challenge in security research. In \cite{ScannellProximitybasedAuthenticationMobile2009, NarayananLocationPrivacyPrivate2011, ZhengLocationBasedHandshake2017}, fluctuations in the radio signal have been used in verifying the immediate proximity between unmet/unassociated users/devices. In \cite{SchurmannSecureCommunicationBased2011, MiettinenContextBasedZeroInteractionPairing2014, HanYouFeelWhat2018, PutzAcousticIntegrityCodes2020}, mutually-observed ambient context--such as sound, luminosity, and the correlation between different sensory modalities-- have been exploited to secure the trust between legitimate parties. Since the context information are usually noisy and differ between observers, co-presence verification predominately incorporates error-tolerant algorithms to match close-to but not identical data. Common techniques include distance-bounding protocols \cite{SingeleeLocationVerificationUsing2005}, fuzzy extractor \cite{NarayananLocationPrivacyPrivate2011, ZhengLocationBasedHandshake2017}, and commitment schemes \cite{MiettinenContextBasedZeroInteractionPairing2014}, and machine learning classifiers \cite{WuLearningDevicePairing2018}. 

There exist two antithetical issues regarding context-based co-presence verification, however. On one hand, momentary snapshots of the ambient context contain little entropy to be robust against forgery or brute-force attacks. In our case, the intrinsic entropy of a person's short-term breathing pattern can be as low as 3 bits (based on our empirical analysis),  which is significantly lower than the entropy level required for security credentials. On the other hand, long-term observations of the ambient context may increase the randomness of the shared experience, but require strict data synchronization and processing techniques to extract usable fingerprints that serve as proof of sustained co-presence among devices.
\section{Problem Formulation} \label{ProblemFormulation}

\begin{figure}[tbp]
\begin{center}
\includegraphics[width=8cm]{probFormulation.png}
\end{center}
\vspace{-0.2 in}
\caption{A) The range of the i-th control point that guarantees convex hull property. B) The problem scaled back to [0, 1]. It suffices to consider the concave upper bound function $f(t)$ and the control points picked along the upper bound with equal time spacing. The poly-line of the control polygon formed with these control points is defined as CPETS (in red poly-line).}
\label{Fig:probFormulation}
\vspace{-0.2 in}
\end{figure}

Considering a convex spatio-temporal corridor defined in $\left[t_{1}, t_{2}\right]$. The upper bound function $f_{ub}(t)$ is concave, while the lower bound function $f_{lb}(t)$ is convex. We pick $n+1$ control points in the corridor to form a scaled Bezier curve ~\cite{ding2019safe} of degree $n$ which is defined in $\left[t_{1}, t_{2}\right]$. The vertical axis label $Y$ is a generic spacial label, which could represent $X$, $Y$, $S$, $L$, or any spacial variable, as shown in Fig. \ref{Fig:probFormulation} A. The vertical coordinate of the i-th control point is noted as $s_{i}$. We want to find a sufficient condition to make sure the Bezier curve lies in the corridor.

\begin{theorem}
If the series $s_{i}$ (i=0,\;1,\;...,\;n) satisfies
\begin{equation}\begin{split}
s_{i} \in \left[f_{lb}(t_{1}+\frac{i}{n}(t_{2}-t_{1})), f_{ub}(t_{1}+\frac{i}{n}(t_{2}-t_{1}))\right]
\label{eq.0}
\end{split}\end{equation}
the scaled Bezier curve generated from control points lies in the corridor.
\label{theorem:baseTheorem}
\end{theorem}

Theorem \ref{theorem:baseTheorem} shows that it suffices to pick the control points with equal time spacing, and the range of each control point is the value of the upper bound/lower bound functions at the corresponding time. 

To simplify the proof of theorem \ref{theorem:baseTheorem}, without loss of generality, we can scale the time interval back to $\left[0, 1\right]$, consider the upper bound function $f(t)$ only, and pick the control points along the upper bound function with equal time spacing (which will generate the highest possible Bezier curve). The vertical coordinate of the i-th control point thus satisfies $s_{i}=f(\frac{i}{n})$. The poly-line of the control polygon formed with these control points is defined as CPETS (control polygon of equal time spacing), as shown in the red poly-line in Fig. \ref{Fig:probFormulation} B. Due to the property of concave functions, we have $\text{CPETS} \leq f(t)$. This yields theorem \ref{theorem:equivalentTheorem}, which is equivalent to theorem \ref{theorem:baseTheorem}.

\begin{theorem}
$\forall t\in\left[0, 1\right]$, the Bezier curve $C(t)=\sum_{i=0}^{n}s_{i}B\textsupsub{$n$}{$i$}(t)$ satisfies $C(t)$$\leq$CPETS$\leq$$f(t)$, where $f(t)$ is a concave function, $s_{i}=f(\frac{i}{n})$, and 
$B\textsupsub{$n$}{$i$}(t)$ is the Bernstein polynomial which is defined as $B\textsupsub{$n$}{$i$}(t)=C\textsupsub{$i$}{$n$}t^{i}(1-t)^{n-i}$.
\label{theorem:equivalentTheorem}
\end{theorem}

We will decompose the proof of theorem \ref{theorem:equivalentTheorem} into several lemmas. In this paper, all terms share the same definition as they are defined in theorem~\ref{theorem:equivalentTheorem}, unless stated otherwise.


\subsection{Proofs of \Cref{sec:ergodicity-hmc}}


% \begin{proof}
%\end{proof}

\subsubsection{Proof of \Cref{theo:irred_harris} }
\label{sec:proof-crefth-harris_0}
We first prove  \eqref{theo:irred_harris_a}.  Under the assumption that $\F$ is twice continuously
  differentiable, it follows by a straightforward induction, that for
  all $h >0$ and $q \in \rset^d$, $p \mapsto
  \Phiverletq[h][k](q,p)$, defined by  \eqref{eq:def_Phiverletq}, and $p \mapsto \gperthmc[k](q,p)$, defined by \eqref{eq:def_gperthmc}, are
  continuously differentiable and for all $(q,p) \in \rset^d \times
  \rset^d$,
\begin{equation}
  \Jac_{p,\gperthmc[T]}(q,p) =  \sum_{i=1}^{T-1}(T-i)\defEns{\nabla^2 \F \circ \Phiverletq[h][i](q,p)} \Jac_{p,\Phiverletq[h][i]}(q,p) \eqsp,
\end{equation}
where for all $q \in \rset^d$, $\Jac_{p,\gperthmc[k]}(q,p)$ ($\Jac_{p,\Phiverletq[i][h]}(q,p)$ respectively) is the Jacobian of the function $\tilde{p} \mapsto
\gperthmc[k](q,\tilde{p})$ ($\tilde{p} \mapsto
\Phiverletq[i][h](q,\tilde{p})$ respectively) at $p \in \rset^{d}$.


%  Under \Cref{assum:regOne}, $\sup_{x \in \rset^d} \normLigne{\nabla^2 \F(x)}
% \leq \constzero$ and by
% \Cref{lem:bound_first_iterate_leapfrog_a},
%  $ \sup_{(q,p) \in \rset^d \times \rset^d} \normLigne{\nabla_p \Phiverletq[h][i](q,p)} \leq (1+h \vartheta_1(h))^i$ for any $i \in \nsets$.
% Therefore for any $h >0$, $T \in \nsets$, setting $S = hT$ and using that $\tilde{h} \mapsto \vartheta_1(\tilde{h})$ is nondecreasing and greater than $1$ on $\rset_+^*$ and for any $u,s \geq 0$, $u \geq 1$, $(1+s/u)^{u-1} \leq \log(s+1) \rme^s$, we get that
Under \Cref{assum:regOne}, $\sup_{x \in \rset^d} \normLigne{\nabla^2 \F(x)}
 \leq \constzero$, therefore by \Cref{lem:inverse_1}, we have that for any $T \in \nsets$ and $h >0$,
\begin{equation}
  \label{eq:inverse_1}
 \sup_{(\q,\p) \in \rset^d \times \rset^d} \norm{\Jac_{p,\gperthmc[T]}(q,p)}
 \leq  T (\{1 + h \constzero^{1/2} \vartheta_1(h \constzero^{1/2})\}^T  -1) /h  \eqsp.
\end{equation}
%$\sup_{p \in \rset^d } \nabla_p \gperthmc[k](q,p) \leq C$.
% \begin{equation}
% \label{lem:inverse_1}
% \sup_{p \in \rset^d } \nabla_p \gperthmc[k](q,p) \leq C \eqsp.
% \end{equation}
% Then for all $q, p_1,p_2 \in \rset^d$,
% \begin{equation}
% \label{lem:inverse_1}
% \norm{\gperthmc[k](q,p_1) -\gperthmc[k](q,p_2)} \leq C \norm{p_1 - p_2} \eqsp.
% \end{equation}
For any $q \in \rset^d$, $T\in \nsets$ and $h >0$, define $\phia_{q,T,h}(p)$  for all  $p \in \rset^d$ by
\begin{equation}
  \phia_{q,T,h} (p) = p-(h/T) \gperthmc[T](q,p) \eqsp.
\end{equation}
It is a well known fact (see for example
\cite[Exercise 3.26]{duistermaat:kolk:2004}) that if
\begin{equation}
  \label{eq:inverse_1_2}
  \sup_{(q,p) \in \rset^d \times \rset^d} (h/T)\norm{ \Jac_{p,\gperthmc[T]}(q,p)} < 1 \eqsp,
\end{equation}
then for any $q \in \rset^d$, $\phia_{q,T,h}$ is a
diffeomorphism and  therefore by \eqref{eq:qk}, the same conclusion holds
for $p \mapsto \Phiverletq[h][T](q,p)$. Using \eqref{eq:inverse_1}, if $T \in \nsets$ and $h > 0$  satisfies \eqref{eq:condition-h,T-harris},
then the condition \eqref{eq:inverse_1_2} is verified and as a result \eqref{theo:irred_harris_a}.

Denoting for any $q \in \rset^d$ by $\Phiverletqi[h][T](q,\cdot) : \rset^d \to \rset$ the
continuously differentiable inverse of $p \mapsto
\Phiverletq[h][T](q,p)$ and using a change of variable with $\Phiverletqi[h][T](q,\cdot)$ in \eqref{eq:def_kernel_hmc} concludes the proof of \eqref{eq:def_kernel_hmc_false_density}.

We now show that $\Tker_{h,T}$ satisfies the condition which implies that $\Pkerhmc[h][T]$ is a \Tkernel. We first establish some estimates on the function $(q,p) \mapsto \Phiverletqi[h][T](q,p)$. By
\eqref{eq:inverse_1_2} and \eqref{eq:qk}, for any $q,p,v \in \rset^d$, there exists $\varepsilon \in \ooint{0,1}$ such that $  \normLigne{\Phiverletq[h][T](q,p)-\Phiverletq[h][T](q,v)} \geq (hT) \normLigne{\phi_{q,T,h}(p)-\phi_{q,T,h}(v)} \geq (hT) (1-\varepsilon)\norm{p-v}$ which implies that that there exists $C \geq 0$ satisfying
\begin{equation}
  \label{eq:regularity_phinverse1}
  \begin{aligned}
    \norm{\Phiverletqi[h][T](q,p)-\Phiverletqi[h][T](q,v)} &\leq (1-\varepsilon)^{-1} \norm{v-p}\eqsp, \\
    \norm{  \Phiverletqi[h][T](q,p)} &\leq C\defEns{\norm{\p} + \norm{\Phiverletq[h][T](q,0)}} \eqsp.
  \end{aligned}
\end{equation}
In addition, for $q,x,p \in \rset^d$, we have setting $\tilde{q} = \Phiverletqi[h][T](q,p)$ that
\begin{align}
  \nonumber
  \normLigne{\Phiverletqi[h][T](q,p) - \Phiverletqi[h][T](x,p)} &= \normLigne{\tilde{q} - \Phiverletqi[h][T](x, \Phiverletq[h][T](q,\tilde{q}))} \\
  \nonumber
                                                                &= \normLigne{\Phiverletqi[h][T](x, \Phiverletq[h][T](x,\tilde{q})) - \Phiverletqi[h][T](x, \Phiverletq[h][T](q,\tilde{q}))} \eqsp,
\end{align}
which implies by \eqref{eq:regularity_phinverse1} and \Cref{lem:bound_first_iterate_leapfrog_a}
that there exists $C \geq 0$ satisfying
\begin{equation}
  \label{eq:regularity_phinverse2}
  \norm{\Phiverletqi[h][T](q,p) - \Phiverletqi[h][T](x,p)} \leq C \norm{q-x} \eqsp.
\end{equation}


We now can prove that $\Tker_{h,T}$ is the continuous component of $\Pkerhmc[h][T]$. First by \eqref{eq:def_tker}, for all $\eventB \in \borelSet(\rset^d)$,
\begin{equation}
\label{eq:minoration_pseudo_density_P}
    \Tker_{h,T}(q, \eventB) \geq (2 \uppi)^{-d/2} \Leb(\eventB)
 \times \inf_{\bar{q} \in \eventB} \defEns{ \balphaacc(q,\bar{q}) \rme^{-\norm{\Phiverletqi_q(\bar{q})}^2/2}\detj_{\Phiverletqi[h][T](q,\cdot)}(\bar{q})} \eqsp,
\end{equation}
with the convention $0 \times \plusinfty = 0$ and
\begin{equation}
%  \label{eq:7}
  \balphaacc(q,\bar{q}) =  \alphaacc\defEns{(q,\Phiverletqi[h][T](q,\bar{q})),\Phiverlet[h][T](q,\Phiverletqi[h][T](q,\bar{q}))}\eqsp. 
\end{equation}
Since the function $  (q,p) \mapsto (\Phiverletq[h][T](q,p),\Phiverletqi[h][T](q,p), \detj_{\Phiverletqi[h][T](q,\cdot)}(p)) $
is continuous on $\rset^d\times \rset^d$ by \Cref{lem:bound_first_iterate_leapfrog_a}, \eqref{eq:regularity_phinverse1} and \eqref{eq:regularity_phinverse2}, and for any $q,p \in \rset^d$, $\Jac_{\Phiverletq[h][T](\q,\cdot)}(\Phiverletqi[h][T](q,p))
\Jac_{\Phiverletqi[h][t](q,\cdot)}(\p) = \operatorname{I}_n$, we get that  $\Tker_{h,T}(q,\eventB) >0$ for all $q \in \rset^d$ and all compact set $\eventB$ satisfying $\Leb(\eventB) > 0$. Therefore, using that the Lebesgue measure is regular which implies that for any $\msa \in \mcb(\rset^d)$ with $\Leb(\msa) >0$, there exists a compact set $\msb \subset\msa$, $\Leb(\msb)>0$, we can conclude that $\Pkerhmc[h][T]$ is irreducible with respect to the Lebesgue measure. In addition, we get  $\Tker_{h,T}(q,\rset^d) >0$, and therefore we obtain that $\Pkerhmc[h][T]$ is aperiodic.  Similarly we get that any compact set is $(1,\Leb)$-small.

It remains to show that for any $\eventB \in\mcb(\rset^d)$, $q \mapsto \Tker_{h,T}(q,\eventB)$ is lower semi-continuous which is a straightforward consequence of Fatou's Lemma and that for any $p \in \rset^d$, $q \mapsto (\Phiverlet[h][T](q,p), \Phiverletqi[h][T](q,p),\detj_{\Phiverletqi[h][T](q,\cdot)}(p))$ is continuous.

% We now show that all the compact sets are $(1,\Leb)$-small. Let $\eventB \subset \rset^d$ be compact.  Using
% \eqref{eq:inverse_1_2} there exists $C \geq 0$ such that for all
% $q,p,v \in \rset^d$, $ \norm{p-v} \leq C \normLigne{
%   \Phiverletq[h][T](q,p)- \Phiverletq[h][T](q,v)}$. It follows
% that for all $p \in \rset^d$, $\sup_{q \in \eventB} \normLigne{
%   \Phiverletqi[h][T](q,p)} \leq C\defEnsLigne{\norm{\p} + \sup_{q \in
%     \eventB} \normLigne{\Phiverletq[h][T](q,0)}}$. Using this upper
% bound and $\Jac_{\Phiverletq(\q,\cdot)}(\Phiverletqi_q(\p))
% \Jac_{\Phiverletqi_q}(\tilde{\p}) = \operatorname{I}_n$ in
% \eqref{eq:minoration_pseudo_density_P}, where $\operatorname{I}_n$ is
% the identity matrix, we deduce that there exists $\varepsilon >0$ such
% that for all $\eventA\in \borelSet(\rset^d)$, $\eventA \subset \eventB$,
% \begin{equation}
%   \inf_{q \in \eventB} \Pkerhmc[h][T](q, \eventA)  \geq \varepsilon \Leb(\eventA) \eqsp,
% \end{equation}
% and therefore $\eventB$ is small for $\Pkerhmc[h][T]$.
% \begin{equation}
% \norm{  \Phiverletqi_q(p_1)-  \Phiverletqi_q(p_2)} \leq C \norm{p_1-p_2} \eqsp,
% \end{equation}
% This result, \eqref{eq:def_acc_ratio}, \Cref{lem:bound_first_iterate_leapfrog} and \eqref{eq:minoration_pseudo_density_P} imply that $
% \Pkerhmc[h][T]$ is irreducible with respect to the Lebesgue measure
% and aperiodic.
% and any ball on $\rset^d$ is small.

% A straightforward adaptation of the proof of \cite[Corollary
% 2]{tierney:1994} shows that $ \Pkerhmc[h][T]$ is Harris recurrent, see \Cref{propo:harris_rec} in \Cref{sec:harr-recurr-metr}. The desired conclusion then follows from \cite[Theorem 13.0.1]{meyn:tweedie:2009}.
 % \Cref{theo:irred_harris} implies
% that for all $T \geq 0$, there exists $\hirr>0$ such that for all $h \in \ocintLigne{0,\hirr}$ and all $\q \in \rset^d$
%   \begin{equation}
% \lim_{n \to \plusinfty}    \tvnorm{\delta_\q \Pkerhmc[h][T]^n - \pi} = 0 \eqsp.
%   \end{equation}


% By \cite[Theorem 17.1.4, Theorem
% 17.1.7]{meyn:tweedie:2009}, it suffices the to prove that for all
% bounded harmonic function $\harmonic : \rset^d \to \rset$ satisfying
% \begin{equation}
%   \label{eq:def_harm}
%   \Pkerhmc[h][T]\harmonic = \harmonic \eqsp,
% \end{equation}
% %$\Pkerhmc[h][T]\harmonic = \harmonic$,
% are constant. First since $\Pkerhmc[h][T]$ is irreducible with respect
% to the Lebesgue measure and aperiodic, by \cite[Theorem
% 14.0.1]{meyn:tweedie:2009} for $\Leb$-almost all $q$ we get $\lim_{n
%   \to \plusinfty} \Pkerhmc[h][T]^n \harmonic(q) = \pi(\harmonic)$ and therefore by
% \eqref{eq:def_harm} $\harmonic(q) = \pi(\harmonic)$. Therefore we get that for all $q \in \rset^d$ by \eqref{eq:def_kernel_hmc_false_density},
% \begin{multline}
%    \Pkerhmc[h][T]^n \harmonic(q) = \pi(\harmonic)  \int_{\rset^d}  \alphaacc\defEns{(q,\tilde{p}),\Phiverlet[h][T](q,\tilde{p})} \rme^{-\norm{\tilde{p}}^2/2} \rmd \tilde{p} \\
% +   \harmonic(x) \int_{\rset^d}  \parentheseDeux{1-\alphaacc\defEns{(q,\tilde{p}),\Phiverlet[h][T](q,\tilde{p})}} \rme^{-\norm{\tilde{p}}^2/2} \rmd \tilde{p} \eqsp.
% \end{multline}
% Combining this result with \eqref{eq:def_harm}, we get for all $q \in \rset^d$
% \begin{equation}
% (\harmonic(q)-\pi(\harmonic)) \int_{\rset^d} \alphaacc\defEns{(q,\tilde{p}),\Phiverlet[h][T](q,\tilde{p})} \rme^{-\norm{\tilde{p}}^2/2} \rmd \tilde{p} = 0\eqsp.
% \end{equation}
% It follows from \Cref{lem:bound_first_iterate_leapfrog} and \eqref{eq:def_acc_ratio} that for all $q \in \rset^d$, $\harmonic(q) = \pi(\harmonic)$
% which concludes the proof.
% =======
% The proof of \ref{theo:irred_harris_b} using a change of variable with $\Phiverletqi[h][T](q,\cdot)$.

% We now show that $\Tker_{h,T}$ satisfies the condition which implies that $\Pkerhmc[h][T]$ is a \Tkernel.
% First, for all $\eventB \in \borelSet(\rset^d)$,
% \begin{equation}
% \label{eq:minoration_pseudo_density_P}
% \Tker_{h,T}(q, \eventB) \geq (2 \uppi)^{-d/2} \Leb(\eventB)
%  \times \inf_{\bar{q} \in \eventB} \defEns{\alphaacc\defEns{(q,\Phiverletqi[h][T](q,\bar{q})),\Phiverlet[h][T](q,\Phiverletqi[h][T](q,\bar{q}))} \rme^{-\norm{\Phiverletqi[h][T](q,\bar{q})}^2/2} \detj_{\Phiverletqi[h][T]}(q,\bar{q})} \eqsp,
% \end{equation}
% with the convention $0 \times \plusinfty = 0$. Since $\Phiverletqi[h][T](q,\cdot)$
% is a diffeomorphism on $\rset^d$, we get that  $
% \Tker_{h,T}(q,\eventB) >0$ for all $q \in \rset^d$ and all compact set $\eventB$ satisfying $\Leb(\eventB) > 0$. Since the Lebesgue measure is regular, this implies that $\Pkerhmc[h][T]$ is irreducible with respect to the Lebesgue measure and aperiodic.

% By Fatou's Lemma, for any $\eventB \in\mcb(\rset^d)$, $q \mapsto \Tker_{h,T}(q,\eventB)$ is lower semi-continuous.
% We now show that all the compact sets are small. Let $\eventB \subset \rset^d$ be compact.  Using
% \eqref{eq:inverse_1_2} there exists $C \geq 0$ such that for all
% $q,p,v \in \rset^d$, $ \norm{p-v} \leq C \normLigne{
%   \Phiverletq[h][T](q,p)- \Phiverletq[h][T](q,v)}$. It follows
% that for all $p \in \rset^d$, $\sup_{q \in \eventB} \normLigne{
%   \Phiverletqi[h][T](q,\p)} \leq C\defEnsLigne{\norm{\p} + \sup_{q \in
%     \eventB} \normLigne{\Phiverletq[h][T](q,0)}}$. Using this upper
% bound and $\Jac_{\Phiverletq(\q,\cdot)}(\Phiverletqi[h][T](q,\p))
% \Jac_{\Phiverletqi[h][T]}(q,\tilde{\p}) = \operatorname{I}_n$ in
% \eqref{eq:minoration_pseudo_density_P}, where $\operatorname{I}_n$ is
% the identity matrix, we deduce that there exists $\varepsilon >0$ such
% that for all $\eventA\in \borelSet(\rset^d)$, $\eventA \subset \eventB$,
% \begin{equation}
%   \inf_{q \in \eventB} \Pkerhmc[h][T](q, \eventA)  \geq \varepsilon \Leb(\eventA) \eqsp,
% \end{equation}
% and therefore $\eventB$ is small for $\Pkerhmc[h][T]$.
% >>>>>>> f8207bad5c0353bdfe37210ffc64a715e92e53ed

Finally, the last statements of \ref{theo:irred_harris_c} follows from \Cref{propo:harris_rec} in \Cref{sec:harr-recurr-metr} which implies that  $ \Pkerhmc[h][T]$ is Harris recurrent and  \cite[Theorem 13.0.1]{meyn:tweedie:2009} which implies  \eqref{eq:harris-theorem}.

\subsubsection{Proof of \Cref{theo:irred_D}}
\label{sec:proof-crefth_irred_D}
We use \Cref{coro:irred}. Indeed $\Pkerhmc[h][T]$ is
of form \eqref{eq:def_pkerb} and it is straightforward to check that it
satisfies \Cref{assumG:phi} (note that \Cref{lem:bound_first_iterate_leapfrog_a}
shows that $\Phiverlet[h][T]$ is a Lipshitz function on $\rset^{2d}$).

We now check that $\Pkerhmc[h][T]$ satisfies \Cref{assumG:irred_b}($\rassG,0,\MassG$) for all $\rassG,\MassG \in
\rset_+^*$ using \Cref{le:degree_application}.  By \eqref{eq:qk}, for all $T \in \nsets$, $h >0$, $q,p \in \rset^d$,
\begin{equation}
  \label{eq:phiverlet_gqth}
  \Phiverletq[h][T](q,p) = T
h p + g_{q,T,h}(p)
\end{equation}
where $g_{q,T,h}(p) = q - (Th^2/2) \nabla \F(q) -
h^2 \gperthmc[T](q,p)$ where $\gperthmc[T]$ is defined by \eqref{eq:def_gperthmc}. \Cref{lem:inverse_1} shows that for any $T \in \nsets$ and $h >0$, it holds that
\begin{equation}
    \label{eq:2:theo:irred_D}
\sup_{p,v,q \in  \rset^d} \frac{\norm{g_{q,T,h}(p)-g_{q,T,h}(v)}}{\norm{p - v}} \leq T h [\{1 + h \constzero^{1/2} \vartheta_1(h \constzero^{1/2} )\}^T-1] \eqsp,
\end{equation}
which implies that the condition
\Cref{le:degree_application}-\ref{propo:irred_b_item_i} is satisfied. To check that
condition  \Cref{le:degree_application}-\ref{propo:irred_b_item_ii} holds, we consider separately the two cases: $\beta <1$ and $\beta =1$.

\begin{enumerate}[label=$\bullet$, wide, labelwidth=!, labelindent=0pt]
\item Consider first the case $\beta <1$. By \Cref{assum:regOne}-\ref{assum:regOne_b},
for any $T \in \nsets$ and $h >0$, we get
\begin{equation}
\norm{\gperthmc[T](\q,\p)} \leq  T \sum_{i=1}^{T-1} \norm{\nabla \F \circ \Phiverletq[h][i](\q,\p)} \leq
\constzeroT T \sum_{i=1}^{T-1} \defEns{ 1 + \norm{\Phiverletq[h][i](\q,\p)}^{\expozero}}
 \eqsp.
\end{equation}
Hence, by \Cref{lem:bound_first_iterate_leapfrog_b}-\ref{lem:bound_first_iterate_leapfrog_1}
there exists $C \geq 0$ such that for all $R\in \rset_+^*$ and
$q,p \in \rset^d$, $\norm{q} \leq R$,
\begin{equation}
\label{eq:3:theo:irred_D}
\norm{g_{q,T,h}(p)} \leq C \defEns{1+R^{\beta} +\norm{p}^{\expozero}} \eqsp,
\end{equation}
which implies that condition \ref{propo:irred_b_item_ii} of \Cref{le:degree_application} holds for any $T \in \nsets$ and $h >0$.

\item Consider now the case $\beta =1$.  For any $T \in \nsets$, $h >0$,  $q,p \in \rset^d$ we get using \Cref{assum:regOne}-\ref{assum:regOne_a}
\begin{align}
  \norm{g_{q,T,h}(p)} &\leq \norm{q} + Th^2 \constzero  \norm{q} /2 + Th^2 \norm{\nabla U(0)} /2\\
  & \qquad \qquad +h^2 \norm{\gperthmc[T](q,p) - \gperthmc[T](q,0)} + h^2 \norm{ \gperthmc[T](q,0)} \eqsp.
\end{align}
Therefore using \Cref{lem:inverse_1}, for any $q,p \in \rset^d$, $\norm{q} \leq R$ for $R \geq 0$, for any $T \in \nsets$ and $h >0$ satisfying \eqref{eq:condition-h,T-harris}, there exists $C \geq 0$ such that
\begin{equation}
  \norm{g_{q,T,h}(p)} \leq C + h T  [ \{1+ h \constzero^{1/2} \vartheta_1(h\constzero^{1/2})\}^T-1]  \norm{p} \eqsp,
\end{equation}
showing that condition \ref{propo:irred_b_item_ii} of \Cref{le:degree_application} is satisfied.
\end{enumerate}

Therefore,  \Cref{le:degree_application} can be applied and for any $T \in \nsets$ and $h >0$ if $\beta <1$ and for any $h > 0$ and $T \in \nsets$ satisfying \eqref{eq:condition-h,T-harris} if $\beta =1$, $\Pkerhmc[h][T]$ satisfies \Cref{assumG:irred_b}($\rassG,0,\MassG$) for all $\rassG,\MassG \in
\rset_+^*$.  \Cref{coro:irred} concludes the proof of \ref{theo:irred_D_a} and \ref{theo:irred_D_b}.
The last statement then follows from   \cite[Theorem 14.0.1]{meyn:tweedie:2009}.

% Using this result and \Cref{theo:irred}, we get that for all $\rassG,\MassG  \in \rset_+^*$ there exists $\varepsilon >0$ such that
% for all $\q \in \ball{0}{\rassG}$ and $\eventA \in \borelSet(\rset^d)$,
% \begin{equation}
%   \Pkerhmc[h][T](q, \eventA) \geq \varepsilon \Leb(\eventA \cap \ball{0}{M}) \eqsp.
% \end{equation}
% \Cref{coro:irred} Combining this result and \eqref{eq:1:theo:irred_D} concludes the proof of \ref{theo:irred_D_a} and \ref{theo:irred_D_b}.

% The proof is a consequence of \Cref{lem:bound_first_iterate_leapfrog},
% \Cref{le:degree_application} and \Cref{theo:irred}.  \alain{give some
%   details}

%%% Local Variables:
%%% mode: latex
%%% TeX-master: "main"
%%% End:

\section{Search Space Coverage} \label{SearchSpaceCoverage}
We have proven the safety of the proposed method of choosing control points. Now we need to analyze the advantage of the proposed method, i.e. more search space coverage. From theorem \ref{theorem:baseTheorem} we known that $\sum_{i=0}^{n}f(\frac{i}{n})B\textsupsub{$n$}{$i$}(t)$ ($f(t)$ is the concave upper bound function) is the highest possible Bezier curve. It suffices to analyze the difference between the highest possible Bezier curve and $f(t)$. We are going to prove that the difference is $O(\frac{1}{n^{2}})$ if $f(t)$ is twice differentiable.

\begin{lemma}
For two twice differentiable functions $f_{1}(t)$ and $f_{2}(t)$ defined in $\left[t_{1}, t_{2}\right]$, if $\exists t_{0} \in \left[t_{1}, t_{2}\right]$ such that $f_{1}(t_{0}) = f_{2}(t_{0})$ and $f^{\prime}_{1}(t_{0}) = f^{\prime}_{2}(t_{0})$ , we have $f_{1}(t) - f_{2}(t)$ = $O((t_{2}-t_{1})^2)$.
\label{lemma:simpleTangentO2Diff}
\end{lemma}

\begin{proof}
The proof is trivial by applying Taylor's formula on $f_{2}(t) - f_{1}(t)$.
\end{proof}

\begin{lemma}
If $g_{1}(t)$ is a twice differentiable function defined in $\left[t_{1}, t_{2}\right]$, and $g_{2}(t)$ is the linear function passing $(t_{1}, g_{1}(t_{1}))$ and $(t_{2}, g_{1}(t_{2}))$, then $g_{1}(t) - g_{2}(t)$ $=$ $O((t_{1} - t_{2})^{2})$  
\label{lemma:2ndOrderDiff}
\end{lemma}

\begin{proof}
Note $h(t) = g_{1}(t) - g_{2}(t)$, we have obviously $h(t_{1}) = h(t_{2})=0$ and $h(t)$ is twice differentiable. $\forall t \in \left[t_{1}, t_{2}\right]$, from Taylor's formula we have
{
\setlength\abovedisplayskip{1pt}
\setlength\belowdisplayskip{1pt}
\begin{equation}\begin{split}
0=h(t_{1}) = h(t) + h^{(1)}(t)(t_{1} - t) + O((t_{2}-t_{1})^{2})
\label{eq:Taylor1}
\end{split}\end{equation}
}
{
\setlength\abovedisplayskip{1pt}
\setlength\belowdisplayskip{1pt}
\begin{equation}\begin{split}
0=h(t_{2}) = h(t) + h^{(1)}(t)(t_{2} - t) + O((t_{2}-t_{1})^{2})
\label{eq:Taylor2}
\end{split}\end{equation}
}
Calculating the difference of (\ref{eq:Taylor1}) and (\ref{eq:Taylor2}) yields
{
\setlength\abovedisplayskip{1pt}
\setlength\belowdisplayskip{1pt}
\begin{equation}\begin{split}
0=h^{(1)}(t)(t_{1}-t_{2})+O((t_{2}-t_{1})^{2})
\label{eq:Taylor3}
\end{split}\end{equation}
}
From (\ref{eq:Taylor3}) we know that $h^{(1)}(t)(t_{1}-t_{2})$ = $O((t_{2}-t_{1})^{2})$, so $h^{(1)}(t)(t_{1} - t)$ = $O((t_{2}-t_{1})^{2})$ $\forall t \in \left[t_{1}, t_{2}\right]$. From (\ref{eq:Taylor1}) we know that $h(t)$ = $-h^{(1)}(t)(t_{1} - t)$ + $O((t_{2}-t_{1})^{2})$ = $O((t_{2}-t_{1})^{2})$. This concludes the proof.
\end{proof}

\begin{corollary}
Suppose $s_{i}$ is a concave (convex) series, the Bezier curve $C(t)=\sum_{i=0}^{n}s_{i}B\textsupsub{$n$}{$i$}(t)$ is inferior than the first and the last segment of the CPETS, and the difference is $O(\frac{1}{n^{2}})$.
\label{corollary:tangentFistLastPointDiff}
\end{corollary}

\begin{proof}
The proof is trivial by applying corollary \ref{corollary:tangentFistLastPoint} and lemma \ref{lemma:simpleTangentO2Diff}.
\end{proof}

\begin{corollary}
If the concave upper bound function $f(t)$ is twice differentiable, the difference between $f(t)$ and the CPETS is $O(\frac{1}{n^{2}})$, where $n$ is the number of control points.
\label{corollary:diffUpperBoundCPETS}
\end{corollary}

\begin{proof}
The proof is obvious by applying lemma \ref{lemma:2ndOrderDiff} on interval $\left[\frac{i}{n}, \frac{i+1}{n}\right]$ ($i=0,\;1,\;...\;,\;n-1$).
\end{proof}

With corollary \ref{corollary:diffUpperBoundCPETS}, it suffices to prove that the difference between the highest possible Bezier curve $\sum_{i=0}^{n}f(\frac{i}{n})B\textsupsub{$n$}{$i$}(t)$ and the CPETS is $O(\frac{1}{n^{2}})$.

\begin{lemma}
$\forall n > 1$, if the upper bound function $f(t)$ defined in $\left[0,1\right]$ is concave and twice differentiable, for series $s_{i}=f(\frac{i}{n})$, $\forall j \in \left[0, n-1\right]$ and 
$\forall t \in \left[\frac{j}{n}, \frac{j+1}{n}\right]$, note $C(t)=\sum_{i=0}^{n}s_{i}B\textsupsub{$n$}{$i$}(t)$, we have $C(t)$ $-$ $(s_{j}+n(s_{j+1}-s{j})(t-\frac{j}{n}))$=$O(\frac{1}{n^{2}})$, i.e. the difference between the Bezier curve and CPETS is $O(\frac{1}{n^{2}})$.
\label{lemma:mainLemmaDiff}
\end{lemma}

\begin{proof}
We will prove it by recurrence, similar to the proof of lemma \ref{lemma:mainLemma}. For $n \le 2$, it is easy to verify that lemma \ref{lemma:mainLemmaDiff} holds using lemma \ref{lemma:simpleTangentO2Diff}. 

Suppose lemma \ref{lemma:mainLemmaDiff} holds for $n=k$ $(k > 2)$. When $n=k+1$, we only need to show that lemma \ref{lemma:mainLemmaDiff} holds for intervals $\left[\frac{1}{k+1}, \frac{2}{k+1}\right]$, $\left[\frac{2}{k+1}, \frac{3}{k+1}\right]$, ... , 
$\left[\frac{k-1}{k+1}, \frac{k}{k+1}\right]$, because for the first and the last interval, lemma \ref{lemma:mainLemmaDiff} holds due to corollary \ref{corollary:tangentFistLastPointDiff}.

With the assumption of recurrence, $\forall j \in \left[1, k-1\right]$ and 
$\forall t \in \left[\frac{j}{k}, \frac{j+1}{k}\right]$, using the recurrence definition of Bezier curve (\ref{eq:BezierRecurDefinition}), we have:
{
\setlength\abovedisplayskip{1pt}
\setlength\belowdisplayskip{1pt}
\begin{equation}\begin{split}
C(t) &= D(t) + O(\frac{1}{{k}^{2}}) \\
 &= D(t) + O(\frac{1}{(k+1)^{2}})
\label{eq:BezierRecurIniqualityDiff}
\end{split}\end{equation}
}
where $D(t)$ is defined in (\ref{eq:BezierRecurIniqualityRightSide}). One can use lemma \ref{lemma:simpleTangentO2Diff} to verify that 
{
\setlength\abovedisplayskip{1pt}
\setlength\belowdisplayskip{1pt}
\begin{equation}\begin{split}
D(t) = s_{j}+(k+1)(s_{j+1}-s_{j})(t - \frac{j}{k}) + O(\frac{1}{(k+1)^{2}})
\label{eq:shiftedInequalityLinearDiff}
\end{split}\end{equation}
}
$\forall t \in \left[\frac{j}{k+1}, \frac{j+1}{k+1}\right]$, note $\Delta t = \frac{j}{k} - \frac{j}{k+1} > 0$, one can verify that $t+\Delta t \in \left[\frac{j}{k}, \frac{j+1}{k}\right]$, and $\Delta t = O(\frac{1}{(k+1)^{2}})$. Thus $C(t)$ = $C(t+\Delta t) - C^{\prime}(t+\Delta t)\Delta t + o(\Delta t)$ = $C(t+\Delta t) + O(\frac{1}{(k+1)^{2}})$. From (\ref{eq:BezierRecurIniqualityDiff}), (\ref{eq:BezierRecurIniqualityRightSide}) and (\ref{eq:shiftedInequalityLinearDiff}), we have
{
\setlength\abovedisplayskip{1pt}
\setlength\belowdisplayskip{1pt}
\begin{equation}\begin{split}
C(t) &= C(t+\Delta t) + O(\frac{1}{(k+1)^{2}}) \\
&= D(t+\Delta t) + O(\frac{1}{(k+1)^{2}}) \\
&= s_{j}+(k+1)(s_{j+1}-s_{j})(t + \Delta t - \frac{j}{k}) \\
&+ O(\frac{1}{(k+1)^{2}}) \\
&= s_{j}+(k+1)(s_{j+1}-s_{j})(t - (\frac{j}{k} - \frac{j}{k} + \frac{j}{k+1})) \\
&+ O(\frac{1}{(k+1)^{2}}) \\
&= s_{j}+(k+1)(s_{j+1}-s_{j})(t - \frac{j}{k+1}) + O(\frac{1}{(k+1)^{2}})
\label{eq:shiftedInequalityDiff}
\end{split}\end{equation}
}
This concludes the proof for $n=k+1$. Lemma \ref{lemma:mainLemmaDiff} is thus proven.
\end{proof}

Lemma \ref{lemma:mainLemmaDiff} and corollary \ref{corollary:diffUpperBoundCPETS} show that the difference between the upper bound function and the highest possible Bezier curve is $O(\frac{1}{n^{2}})$, where $n$ is the number of control points.

\section{Simulation results} \label{Simulation}

\begin{figure}[tbp]
\begin{center}
\includegraphics[width=9cm]{comparison_of_performance.png}
\end{center}
\vspace{-0.2 in}
\caption{Comparison of performance. A) The shapes of two types of corridors. B) Planning result in general convex corridal. C) Planning result in trapezoidal corridor. D) Comparison of acceleration from two planning results.}
\label{Fig:comparisonOfPerformance}
\vspace{-0.2 in}
\end{figure}

In this section, the motion planning results based on the general convex corridors are compared with trapezoidal corridors-based planning. The same set of parameters (weights, time horizon) were used for both cases to make the results comparable. The scenario includes an ego vehicle and a decelerating front vehicle, similar to the scenario depicted in Fig. \ref{Fig:comparison_of_shapes} A1. The corresponding corridors are shown in Fig. \ref{Fig:comparisonOfPerformance} A. A desirable planning should consider the smoothness and the closeness between the planned trajectory and the reference trajectory. It is also reasonable to consider a higher weight for the trajectory points in the near future compared to the later-time points. Therefore, the cost function is defined as (\ref{eq:costFunction}):
{
\setlength\abovedisplayskip{1pt}
\setlength\belowdisplayskip{1pt}
\begin{equation}\begin{split}
&Cost=\\
&w_{00} \int_{0}^{T_{s}} (S(t)-S_{r}(t))^2 \,dx+w_{01} \int_{T_{s}}^{T_{l}} (S(t)-S_{r}(t))^2 \,dx \\
&+w_{10} \int_{0}^{T_{s}} (\ddot S(t)- \ddot S_{r}(t))^2 \,dx+w_{11} \int_{T_{s}}^{T_{l}} (\ddot S(t)- \ddot S_{r}(t))^2 \,dx \\ 
&+w_{20} \int_{0}^{T_{s}} (\dddot S(t))^2 \,dx+w_{21} \int_{T_{s}}^{T_{l}} (\dddot S(t))^2 \,dx, \\
\label{eq:costFunction}
\end{split}\end{equation}
}
where $w_{00}=5.0$, $w_{01}=2.5$, $w_{10}=10.0$, $w_{11}=3.0$, $w_{20}=25.0$, $w_{21}=10.0$, $T_{l}=20.0$, $T_{s}=6.0$. The terms $T_{l}$ and $T_{s}$ mean that the planning has an extra emphasis on the near future points. The initial speed of the planning is 5.5 $m/s$. $S_{r}(t)$ is a reference trajectory, which is generated by a uniformly accelerated motion with an acceleration of -0.05 $m/s^{2}$. Note that we do not require the reference trajectory to be safe here. A reference trajectory with high acceleration could be unsafe, but it may result in a planned trajectory with a higher acceleration. 

The motion planning problem is then converted into a QP (quadratic programming) problem, whose details are presented in~\cite{ding2019safe} \cite{li2021speed}. OSQP 0.5.1 is adopted as the solver to this problem. As shown in Fig.~\ref{Fig:comparisonOfPerformance}, the general convex corridor results in a larger search space, leading to much smaller deceleration compared to that of the trapezoidal corridor (-0.52 $m/s^{2}$ $vs$ -1.0 $m/s^{2}$) and subsequently less harsh brakes and improved smoothness.

To evaluate the performance of this algorithm in a real-time, multi-frame condition, a cut-in scenario is set up in our simulator (developed based on ROS). The simulation scenario is based on an real-world scenario encountered in road test. The ego vehicle is cruising under a speed limit of 60 $km/h$ (16.77 $m/s$\footnote{Due to the performance of the speed controller, the ego vehicle's actual speed is around 16 $m/s$.}), while another vehicle cuts in at $t=1.0$ s, at a distance of 22 $m$ and at a speed of 7 $m/s$. The cut-in vehicle decelerates at 0.5 $m/s^2$ for 3 seconds, then continues moving at a constant speed. There is an additional 5-meter safety margin behind the cut-in vehicle. The close-loop longitudinal dynamics of the ego vehicle is simulated by a first-order dynamics with a time constant of 0.3 $s$. The corridors are generated with the algorithm presented in~\cite{xin2021enable}~and~\cite{li2021speed}, respectively for general convex shape and trapezoidal shape corridors.

\begin{figure}[tbp]
\begin{center}
\includegraphics[width=9cm]{comparison_in_simulator.png}
\end{center}
\vspace{-0.2 in}
\caption{Comparison of performance in simulator. A) The cut-in scenario from top view. B) The lowest speed achieved with general convex corridor. C) The lowest speed achieved with trapezoidal corridor. D) Comparison of ego vehicle's speed from two planning results. E) Comparison of ego vehicle's acceleration from two planning results.}
\label{Fig:comparisonInSimulator}
\vspace{-0.2 in}
\end{figure}

The comparison is shown in Fig.~\ref{Fig:comparisonInSimulator}. The lowest deceleration is -4.46 $m/s^2$ using general convex corridors, compared to -5.23 $m/s^2$ of using trapezoidal corridors. In the case of using general convex corridors, the lowest speed is registered at 3.41 $m/s$, compared to that of trapezoidal corridors at 2.42 $m/s$. Our proposed approach decelerates earlier, which helps in reducing the peak value of deceleration and the length of the deceleration period as an indication of an improved overall comfort.
\mySection{Related Works and Discussion}{}
\label{chap3:sec:discussion}

In this section we briefly discuss the similarities and differences of the model presented in this chapter, comparing it with some related work presented earlier (Chapter \ref{chap1:artifact-centric-bpm}). We will mention a few related studies and discuss directly; a more formal comparative study using qualitative and quantitative metrics should be the subject of future work.

Hull et al. \citeyearpar{hull2009facilitating} provide an interoperation framework in which, data are hosted on central infrastructures named \textit{artifact-centric hubs}. As in the work presented in this chapter, they propose mechanisms (including user views) for controlling access to these data. Compared to choreography-like approach as the one presented in this chapter, their settings has the advantage of providing a conceptual rendezvous point to exchange status information. The same purpose can be replicated in this chapter's approach by introducing a new type of agent called "\textit{monitor}", which will serve as a rendezvous point; the behaviour of the agents will therefore have to be slightly adapted to take into account the monitor and to preserve as much as possible the autonomy of agents.

Lohmann and Wolf \citeyearpar{lohmann2010artifact} abandon the concept of having a single artifact hub \cite{hull2009facilitating} and they introduce the idea of having several agents which operate on artifacts. Some of those artifacts are mobile; thus, the authors provide a systematic approach for modelling artifact location and its impact on the accessibility of actions using a Petri net. Even though we also manipulate mobile artifacts, we do not model artifact location; rather, our agents are equipped with capabilities that allow them to manipulate the artifacts appropriately (taking into account their location). Moreover, our approach considers that artifacts can not be remotely accessed, this increases the autonomy of agents.

The process design approach presented in this chapter, has some conceptual similarities with the concept of \textit{proclets} proposed by Wil M. P. van der Aalst et al. \citeyearpar{van2001proclets, van2009workflow}: they both split the process when designing it. In the model presented in this chapter, the process is split into execution scenarios and its specification consists in the diagramming of each of them. Proclets \cite{van2001proclets, van2009workflow} uses the concept of \textit{proclet-class} to model different levels of granularity and cardinality of processes. Additionally, proclets act like agents and are autonomous enough to decide how to interact with each other.

The model presented in this chapter uses an attributed grammar as its mathematical foundation. This is also the case of the AWGAG model by Badouel et al. \citeyearpar{badouel14, badouel2015active}. However, their model puts stress on modelling process data and users as first class citizens and it is designed for Adaptive Case Management.

To summarise, the proposed approach in this chapter allows the modelling and decentralized execution of administrative processes using autonomous agents. In it, process management is very simply done in two steps. The designer only needs to focus on modelling the artifacts in the form of task trees and the rest is easily deduced. Moreover, we propose a simple but powerful mechanism for securing data based on the notion of accreditation; this mechanism is perfectly composed with that of artifacts. The main strengths of our model are therefore : 
\begin{itemize}
	\item The simplicity of its syntax (process specification language), which moreover (well helped by the accreditation model), is suitable for administrative processes;
	\item The simplicity of its execution model; the latter is very close to the blockchain's execution model \cite{hull2017blockchain, mendling2018blockchains}. On condition of a formal study, the latter could possess the same qualities (fault tolerance, distributivity, security, peer autonomy, etc.) that emanate from the blockchain;
	\item Its formal character, which makes it verifiable using appropriate mathematical tools;
	\item The conformity of its execution model with the agent paradigm and service technology.
\end{itemize}
In view of all these benefits, we can say that the objectives set for this thesis have indeed been achieved. However, the proposed model is perfectible. For example, it can be modified to permit agents to respond incrementally to incoming requests as soon as any prefix of the extension of a bud is produced. This makes it possible to avoid the situation observed on figure \ref{chap3:fig:execution-figure-4} where the associated editor is informed of the evolution of the subtree resulting from $C$ only when this one is closed. All the criticisms we can make of the proposed model in particular, and of this thesis in general, have been introduced in the general conclusion (page \pageref{chap5:general-conclusion}) of this manuscript.




\section{Conclusions}
\label{sec:conclusions}

In this paper, we apply shared-workload techniques at the \sql level for
improving the throughput of \qaasl systems without incurring in additional
query execution costs. Our approach is based on query rewriting for grouping
multiple queries together into a single query to be executed in one go. This
results in a significant reduction of the aggregated data access done by the
shared execution compared to executing queries independently.

%execution times and costs of the shared scan operator when
%varying query selectivity and predicate evaluation. We observed that for
%\athena, although the cost only depends on the amount of data read, it is
%conditioned to its ability to use its statistics about the data. In some cases
%a wrong query execution plan leads to higher query execution costs, which the
%end-user has to pay. 

%\bigquery's minimum query execution cost is determined by
%the input size of a query.  However, the query cost can increase depending not
%just in the amount of computation it requires, but in the mix of resources the
%query requires.  

We presented a cost and runtime evaluation of the shared operator driving data access costs. 
Our experimental study using the TPC-H benchmark confirmed the benefits of our
query rewrite approach. Using a shared execution approach reduces significantly
the execution costs. For \athena, we are able to make it 107x cheaper and for
\bigquery, 16x cheaper taking into account Query 10 which we cannot execute,
but 128x if it is not taken into account. Moreover, when having queries that do
not share their entire execution plan, i.e., using a single global plan, we
demonstrated that it is possible to improve throughput and obtain a 10x cost
reduction in \bigquery.

%We followed the TPC systems pricing guideline for
%computing how expensive is to have a TPC-H workload working on the evaluated
%\qaasl systems. The result is that even though we are able to reduce overall
%costs a TPC-H workload in 15x for \bigquery (128x excluding query 10 which we
%could not optimize) and in 107x for \athena, the overall price is at least 10x
%more expensive than the cheapest system price published by the TPC.

There are multiple ways to extend our work. The first is
to implement a full \sql-to-\sql translation layer to encapsulate the proposed
per-operator rewrites.  Another one is to incorporate the initial work on
building a cost-based optimizer for shared execution
\cite{Giannikis:2014:SWO:2732279.2732280} as an external component for \qaasl
systems.  Moreover, incorporating different lines of work (e.g., adding
provenance computation \cite{GA09} capabilities) also based on query
rewriting is part of our future work to enhance our system.


%%%%%%%%%%%%%%%%%%%%%%%%%%%%%%%%%%%%%%%%%%%%%%%%%%%%%%%%%%%%%%%%%%%%%%%%%%%%%%%
\bibliographystyle{IEEEtran}
\bibliography{references}

\end{document}