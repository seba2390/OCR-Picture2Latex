\section{Introduction}
% Autonomous vehicles will revolutionize how people travel in the future \cite{pettersson2015setting}. 

Motion planning is one of the main components of an autonomous driving system with an essential role in generating feasible, safe trajectories. Semantic safe corridors and Bezier curves are commonly employed in optimization-based motion planning to guarantee safety, thanks to the convex hull property of Bezier curves. The idea is to generate a spatial safety corridor based on the surrounding occupancies, and use the corridor as a constraint in trajectory planning. The trajectories are then generated in $SLT$ (Frenet) frame~\cite{fan2018baidu} or $XYT$ (Cartesian) frame~\cite{sun2018fast}, aiming to optimize the overall swiftness, smoothness, and smartness \cite{fleury1995primitives}, 
under the constraints of continuity, vehicle dynamics limitation, traffic rules, and above all, safety \cite{lu2020adaptive}.  

One reason for using optimization-based approaches is that convex corridors
guarantee the convexity of the problem, allowing for solving the motion planning problem as a quadratic programming problem \cite{xu2014motion}, which are well-studied and can be implemented using off-the-shelf solvers~\cite{qian2016optimal}. Among different parametric curves, piece-wise Bezier curves are more popular as the convex hull property of Bezier curves\footnote{The Bezier curve defined by $n+1$ control points lies in the convex hull of the given control points, which is also called the control polygon.} \cite{aziz1990bezier} guarantees the curve segments to be confined within the corresponding 
safety corridor~\cite{gao2018online}. As a result, the safety of the generated trajectories are guaranteed. 

\begin{figure}[tbp]
\begin{center}
\includegraphics[width=8cm]{sampleProb.png}
\end{center}
\vspace{-0.2 in}
\caption{A) A scenario where the front vehicle decelerates, and the side vehicle pushes the ego vehicle to the right. B) The corresponding $ST/LT$ graphs. C) Motion planning in $ST/LT$ plane using rectangular spario-temporal corridors.}
\label{Fig:sampleProb}
\vspace{-0.2 in}
\end{figure}

The spatial corridors can be extended to spatio-temporal corridors to account for dynamic objects and predicted occupations in upcoming environments. When corridors consist of spatial dimension with no time-dependency (\textit{e.g.} defined in $XY$ plane or in $SL$ plane), the convex hull property holds, which allows for using general convex corridors for maximum search space coverage as utilized in~\cite{liu2018convex}.
However, when there is a time-dependency, \textit{i.e.} corridors are defined in spatio-temporal frames (\textit{e.g.} $ST$ or $LT$ planes), the convex hull property does not generally hold, as shown in Fig. \ref{Fig:ConvexHullNotHold}, imposing a limitation on the shape of corridors. This is stemmed from the fact that the spatial axis and the temporal axis are not equivalent. 

The convex hull property problem is often addressed by restricting the shape of corridors to be rectangular~\cite{ding2019safe}, where the convex hull property is ensured, resulting in a large uncovered search space and conservative or unsmoothed trajectories. The uncovered search space may lead to optimization failure in crowded scenarios, where no trajectories can be found within the corridors. As an extension, multiple rectangular corridors can be chained together to reduce the uncovered search space as illustrated in Fig.~\ref{Fig:sampleProb}, at the expense of significantly increasing the dimension of the optimization vector. This approach tends to be computationally intensive as it requires a sequential optimization steps to optimize a trajectory segment for each corridor and ensuring continuity from one segment to the next. Another method is to use a sampling based approach that supports general convex corridor shapes \cite{moghadam2020autonomous}. However the sampling-based approaches generally results in sub-optimal solutions, and there is no guarantee to find an optimal trajectory. A recent approach is to employ trapezoidal corridors \cite{li2021speed} with a sufficient condition for convex hull property to hold. The trapezoidal shape is effective in reducing the number of corridors, specifically in dynamic scenarios when there are objects with constant speed. While this is an improvement compared to the previous methods, the uncovered space can still be significant in certain scenarios. Fig.~\ref{Fig:comparison_of_shapes} A1 shows a case when there is a decelerating front vehicle, and Fig.~\ref{Fig:comparison_of_shapes} A2 shows a cut-in vehicle that with accelerating and decelerating speed profile. By comparing Fig.~\ref{Fig:comparison_of_shapes} B1/C1 and Fig.~\ref{Fig:comparison_of_shapes} B2/C2, one can see that in both cases, trapezoidal corridors have larger uncovered search space than that of general convex corridors, which may lead to unnecessary harsh brakes in motion planning.
%Even though the conventional Bezier curve is defined in $\left[0, 1\right]$, it can be scaled to fit into any time interval \cite{gao2018online}.
%A sample motion planning problem using rectangular spatio-temporal corridors is illustrated in Fig.~\ref{Fig:sampleProb}. As can be seen from the figure, a sequence of corridors is chained to increase the search space coverage.


%Recent work has considered trapezoidal corridors to increase the search space covered by a corridor and provided a sufficient condition to ensure the convex hull property. However, the uncovered search area can still be limited compared to general convex-shaped corridors, specifically when the moving objects have a non-monotonic, frequently changing speed profile.
%In this paper, we focus on the usage of different types of corridors. 
%Various corridor shapes have been studied in the literature, including rectangles~\cite{ding2019safe}, trapezes~\cite{li2021speed}, and general convex shapes~\cite{xin2021enable}. Naturally, more complicated shapes cover more search space and yield better optimization results. 
%The generation of corridors is well-studied and is out of the scope of this paper. 
%When corridors consist of spatial dimension with no time-dependency (\textit{e.g.} defined in $XY$ plane or in $SL$ plane), the convex hull property holds, which allows for using general convex corridors for maximum search space coverage as utilized in~\cite{liu2018convex}.
%However, when there is a time-dependency, \textit{i.e.} corridors are defined in spatio-temporal frame (\textit{e.g.} $ST$ or $LT$ planes), the convex hull property does not generally hold, as shown in Fig. \ref{Fig:ConvexHullNotHold}. The convex hull property in the time-dependent cases depends on the shape of the corridor, which precludes using arbitrary shaped corridors. 

\begin{figure}[tbp]
\begin{center}
\includegraphics[width=8cm]{convex_hull_not_hold.png}
\end{center}
\vspace{-0.2 in}
\caption{Convex hull property does not hold for spatio-temporal corridors of general convex shapes. A) The corridor is in $SL$ plane. Picking three control points along the boundary of the corridor, the generated
Bezier curve lies in the corridor. B) The corridor is in $LT$ plane. Picking three control points along the boundary of the corridor, the generated
Bezier curve is partially beyond the corridor. }
\label{Fig:ConvexHullNotHold}
\vspace{-0.2 in}
\end{figure}

The convex hull property in the time-dependent cases depends on the shape of the corridor, which precludes using arbitrary shaped corridors. Naturally, more complicated shapes cover more search space and yield better optimization results. An ideal case is to use general convex-shaped corridors. As the main contribution of this paper, we prove a sufficient condition to guarantee the convex hull property for general convex corridors in spatio-temporal frames. The use of general convex corridors can shrink the uncovered search space to $O(\frac{1}{n^2})$, compared to $O(1)$ of trapezoidal corridors.

This paper is organized as follows: Section \ref{ProblemFormulation} converts the main theorem into an equivalent theorem. Section \ref{Proof} decomposes the theorem into several lemmas and proves them by recurrence. Section \ref{SearchSpaceCoverage} shows that the uncovered search space is $O(\frac{1}{n^2})$ under continuity assumptions. Section \ref{Simulation} compares the proposed approach with the state of the art approach quantitatively under simulated scenarios. Section \ref{Discussion} provides deeper analysis to the proposed approach. Section \ref{CONCLUSIONS} presents our conclusions.

\begin{figure}[tbp]
\begin{center}
\includegraphics[width=8.5cm]{comparison_of_shapes.png}
\end{center}
\vspace{-0.2 in}
\caption{The advantage of general convex corridors is that they cover more search space. A1) $ST$ graph of a decelerating front vehicle. A2) $ST$ graph of a cut-in vehicle that switches between accelerating and decelerating. B1) The coverage of 1 trapezoidal corridor. B2) the coverage of 4 trapezoidal corridors. C1) The coverage of 1 general convex corridor. C2) The coverage of 4 general convex corridors.}
\label{Fig:comparison_of_shapes}
\vspace{-0.2 in}
\end{figure}