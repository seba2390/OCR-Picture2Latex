\section{Simulation results} \label{Simulation}

\begin{figure}[tbp]
\begin{center}
\includegraphics[width=9cm]{comparison_of_performance.png}
\end{center}
\vspace{-0.2 in}
\caption{Comparison of performance. A) The shapes of two types of corridors. B) Planning result in general convex corridal. C) Planning result in trapezoidal corridor. D) Comparison of acceleration from two planning results.}
\label{Fig:comparisonOfPerformance}
\vspace{-0.2 in}
\end{figure}

In this section, the motion planning results based on the general convex corridors are compared with trapezoidal corridors-based planning. The same set of parameters (weights, time horizon) were used for both cases to make the results comparable. The scenario includes an ego vehicle and a decelerating front vehicle, similar to the scenario depicted in Fig. \ref{Fig:comparison_of_shapes} A1. The corresponding corridors are shown in Fig. \ref{Fig:comparisonOfPerformance} A. A desirable planning should consider the smoothness and the closeness between the planned trajectory and the reference trajectory. It is also reasonable to consider a higher weight for the trajectory points in the near future compared to the later-time points. Therefore, the cost function is defined as (\ref{eq:costFunction}):
{
\setlength\abovedisplayskip{1pt}
\setlength\belowdisplayskip{1pt}
\begin{equation}\begin{split}
&Cost=\\
&w_{00} \int_{0}^{T_{s}} (S(t)-S_{r}(t))^2 \,dx+w_{01} \int_{T_{s}}^{T_{l}} (S(t)-S_{r}(t))^2 \,dx \\
&+w_{10} \int_{0}^{T_{s}} (\ddot S(t)- \ddot S_{r}(t))^2 \,dx+w_{11} \int_{T_{s}}^{T_{l}} (\ddot S(t)- \ddot S_{r}(t))^2 \,dx \\ 
&+w_{20} \int_{0}^{T_{s}} (\dddot S(t))^2 \,dx+w_{21} \int_{T_{s}}^{T_{l}} (\dddot S(t))^2 \,dx, \\
\label{eq:costFunction}
\end{split}\end{equation}
}
where $w_{00}=5.0$, $w_{01}=2.5$, $w_{10}=10.0$, $w_{11}=3.0$, $w_{20}=25.0$, $w_{21}=10.0$, $T_{l}=20.0$, $T_{s}=6.0$. The terms $T_{l}$ and $T_{s}$ mean that the planning has an extra emphasis on the near future points. The initial speed of the planning is 5.5 $m/s$. $S_{r}(t)$ is a reference trajectory, which is generated by a uniformly accelerated motion with an acceleration of -0.05 $m/s^{2}$. Note that we do not require the reference trajectory to be safe here. A reference trajectory with high acceleration could be unsafe, but it may result in a planned trajectory with a higher acceleration. 

The motion planning problem is then converted into a QP (quadratic programming) problem, whose details are presented in~\cite{ding2019safe} \cite{li2021speed}. OSQP 0.5.1 is adopted as the solver to this problem. As shown in Fig.~\ref{Fig:comparisonOfPerformance}, the general convex corridor results in a larger search space, leading to much smaller deceleration compared to that of the trapezoidal corridor (-0.52 $m/s^{2}$ $vs$ -1.0 $m/s^{2}$) and subsequently less harsh brakes and improved smoothness.

To evaluate the performance of this algorithm in a real-time, multi-frame condition, a cut-in scenario is set up in our simulator (developed based on ROS). The simulation scenario is based on an real-world scenario encountered in road test. The ego vehicle is cruising under a speed limit of 60 $km/h$ (16.77 $m/s$\footnote{Due to the performance of the speed controller, the ego vehicle's actual speed is around 16 $m/s$.}), while another vehicle cuts in at $t=1.0$ s, at a distance of 22 $m$ and at a speed of 7 $m/s$. The cut-in vehicle decelerates at 0.5 $m/s^2$ for 3 seconds, then continues moving at a constant speed. There is an additional 5-meter safety margin behind the cut-in vehicle. The close-loop longitudinal dynamics of the ego vehicle is simulated by a first-order dynamics with a time constant of 0.3 $s$. The corridors are generated with the algorithm presented in~\cite{xin2021enable}~and~\cite{li2021speed}, respectively for general convex shape and trapezoidal shape corridors.

\begin{figure}[tbp]
\begin{center}
\includegraphics[width=9cm]{comparison_in_simulator.png}
\end{center}
\vspace{-0.2 in}
\caption{Comparison of performance in simulator. A) The cut-in scenario from top view. B) The lowest speed achieved with general convex corridor. C) The lowest speed achieved with trapezoidal corridor. D) Comparison of ego vehicle's speed from two planning results. E) Comparison of ego vehicle's acceleration from two planning results.}
\label{Fig:comparisonInSimulator}
\vspace{-0.2 in}
\end{figure}

The comparison is shown in Fig.~\ref{Fig:comparisonInSimulator}. The lowest deceleration is -4.46 $m/s^2$ using general convex corridors, compared to -5.23 $m/s^2$ of using trapezoidal corridors. In the case of using general convex corridors, the lowest speed is registered at 3.41 $m/s$, compared to that of trapezoidal corridors at 2.42 $m/s$. Our proposed approach decelerates earlier, which helps in reducing the peak value of deceleration and the length of the deceleration period as an indication of an improved overall comfort.