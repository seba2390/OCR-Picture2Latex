\section{Search Space Coverage} \label{SearchSpaceCoverage}
We have proven the safety of the proposed method of choosing control points. Now we need to analyze the advantage of the proposed method, i.e. more search space coverage. From theorem \ref{theorem:baseTheorem} we known that $\sum_{i=0}^{n}f(\frac{i}{n})B\textsupsub{$n$}{$i$}(t)$ ($f(t)$ is the concave upper bound function) is the highest possible Bezier curve. It suffices to analyze the difference between the highest possible Bezier curve and $f(t)$. We are going to prove that the difference is $O(\frac{1}{n^{2}})$ if $f(t)$ is twice differentiable.

\begin{lemma}
For two twice differentiable functions $f_{1}(t)$ and $f_{2}(t)$ defined in $\left[t_{1}, t_{2}\right]$, if $\exists t_{0} \in \left[t_{1}, t_{2}\right]$ such that $f_{1}(t_{0}) = f_{2}(t_{0})$ and $f^{\prime}_{1}(t_{0}) = f^{\prime}_{2}(t_{0})$ , we have $f_{1}(t) - f_{2}(t)$ = $O((t_{2}-t_{1})^2)$.
\label{lemma:simpleTangentO2Diff}
\end{lemma}

\begin{proof}
The proof is trivial by applying Taylor's formula on $f_{2}(t) - f_{1}(t)$.
\end{proof}

\begin{lemma}
If $g_{1}(t)$ is a twice differentiable function defined in $\left[t_{1}, t_{2}\right]$, and $g_{2}(t)$ is the linear function passing $(t_{1}, g_{1}(t_{1}))$ and $(t_{2}, g_{1}(t_{2}))$, then $g_{1}(t) - g_{2}(t)$ $=$ $O((t_{1} - t_{2})^{2})$  
\label{lemma:2ndOrderDiff}
\end{lemma}

\begin{proof}
Note $h(t) = g_{1}(t) - g_{2}(t)$, we have obviously $h(t_{1}) = h(t_{2})=0$ and $h(t)$ is twice differentiable. $\forall t \in \left[t_{1}, t_{2}\right]$, from Taylor's formula we have
{
\setlength\abovedisplayskip{1pt}
\setlength\belowdisplayskip{1pt}
\begin{equation}\begin{split}
0=h(t_{1}) = h(t) + h^{(1)}(t)(t_{1} - t) + O((t_{2}-t_{1})^{2})
\label{eq:Taylor1}
\end{split}\end{equation}
}
{
\setlength\abovedisplayskip{1pt}
\setlength\belowdisplayskip{1pt}
\begin{equation}\begin{split}
0=h(t_{2}) = h(t) + h^{(1)}(t)(t_{2} - t) + O((t_{2}-t_{1})^{2})
\label{eq:Taylor2}
\end{split}\end{equation}
}
Calculating the difference of (\ref{eq:Taylor1}) and (\ref{eq:Taylor2}) yields
{
\setlength\abovedisplayskip{1pt}
\setlength\belowdisplayskip{1pt}
\begin{equation}\begin{split}
0=h^{(1)}(t)(t_{1}-t_{2})+O((t_{2}-t_{1})^{2})
\label{eq:Taylor3}
\end{split}\end{equation}
}
From (\ref{eq:Taylor3}) we know that $h^{(1)}(t)(t_{1}-t_{2})$ = $O((t_{2}-t_{1})^{2})$, so $h^{(1)}(t)(t_{1} - t)$ = $O((t_{2}-t_{1})^{2})$ $\forall t \in \left[t_{1}, t_{2}\right]$. From (\ref{eq:Taylor1}) we know that $h(t)$ = $-h^{(1)}(t)(t_{1} - t)$ + $O((t_{2}-t_{1})^{2})$ = $O((t_{2}-t_{1})^{2})$. This concludes the proof.
\end{proof}

\begin{corollary}
Suppose $s_{i}$ is a concave (convex) series, the Bezier curve $C(t)=\sum_{i=0}^{n}s_{i}B\textsupsub{$n$}{$i$}(t)$ is inferior than the first and the last segment of the CPETS, and the difference is $O(\frac{1}{n^{2}})$.
\label{corollary:tangentFistLastPointDiff}
\end{corollary}

\begin{proof}
The proof is trivial by applying corollary \ref{corollary:tangentFistLastPoint} and lemma \ref{lemma:simpleTangentO2Diff}.
\end{proof}

\begin{corollary}
If the concave upper bound function $f(t)$ is twice differentiable, the difference between $f(t)$ and the CPETS is $O(\frac{1}{n^{2}})$, where $n$ is the number of control points.
\label{corollary:diffUpperBoundCPETS}
\end{corollary}

\begin{proof}
The proof is obvious by applying lemma \ref{lemma:2ndOrderDiff} on interval $\left[\frac{i}{n}, \frac{i+1}{n}\right]$ ($i=0,\;1,\;...\;,\;n-1$).
\end{proof}

With corollary \ref{corollary:diffUpperBoundCPETS}, it suffices to prove that the difference between the highest possible Bezier curve $\sum_{i=0}^{n}f(\frac{i}{n})B\textsupsub{$n$}{$i$}(t)$ and the CPETS is $O(\frac{1}{n^{2}})$.

\begin{lemma}
$\forall n > 1$, if the upper bound function $f(t)$ defined in $\left[0,1\right]$ is concave and twice differentiable, for series $s_{i}=f(\frac{i}{n})$, $\forall j \in \left[0, n-1\right]$ and 
$\forall t \in \left[\frac{j}{n}, \frac{j+1}{n}\right]$, note $C(t)=\sum_{i=0}^{n}s_{i}B\textsupsub{$n$}{$i$}(t)$, we have $C(t)$ $-$ $(s_{j}+n(s_{j+1}-s{j})(t-\frac{j}{n}))$=$O(\frac{1}{n^{2}})$, i.e. the difference between the Bezier curve and CPETS is $O(\frac{1}{n^{2}})$.
\label{lemma:mainLemmaDiff}
\end{lemma}

\begin{proof}
We will prove it by recurrence, similar to the proof of lemma \ref{lemma:mainLemma}. For $n \le 2$, it is easy to verify that lemma \ref{lemma:mainLemmaDiff} holds using lemma \ref{lemma:simpleTangentO2Diff}. 

Suppose lemma \ref{lemma:mainLemmaDiff} holds for $n=k$ $(k > 2)$. When $n=k+1$, we only need to show that lemma \ref{lemma:mainLemmaDiff} holds for intervals $\left[\frac{1}{k+1}, \frac{2}{k+1}\right]$, $\left[\frac{2}{k+1}, \frac{3}{k+1}\right]$, ... , 
$\left[\frac{k-1}{k+1}, \frac{k}{k+1}\right]$, because for the first and the last interval, lemma \ref{lemma:mainLemmaDiff} holds due to corollary \ref{corollary:tangentFistLastPointDiff}.

With the assumption of recurrence, $\forall j \in \left[1, k-1\right]$ and 
$\forall t \in \left[\frac{j}{k}, \frac{j+1}{k}\right]$, using the recurrence definition of Bezier curve (\ref{eq:BezierRecurDefinition}), we have:
{
\setlength\abovedisplayskip{1pt}
\setlength\belowdisplayskip{1pt}
\begin{equation}\begin{split}
C(t) &= D(t) + O(\frac{1}{{k}^{2}}) \\
 &= D(t) + O(\frac{1}{(k+1)^{2}})
\label{eq:BezierRecurIniqualityDiff}
\end{split}\end{equation}
}
where $D(t)$ is defined in (\ref{eq:BezierRecurIniqualityRightSide}). One can use lemma \ref{lemma:simpleTangentO2Diff} to verify that 
{
\setlength\abovedisplayskip{1pt}
\setlength\belowdisplayskip{1pt}
\begin{equation}\begin{split}
D(t) = s_{j}+(k+1)(s_{j+1}-s_{j})(t - \frac{j}{k}) + O(\frac{1}{(k+1)^{2}})
\label{eq:shiftedInequalityLinearDiff}
\end{split}\end{equation}
}
$\forall t \in \left[\frac{j}{k+1}, \frac{j+1}{k+1}\right]$, note $\Delta t = \frac{j}{k} - \frac{j}{k+1} > 0$, one can verify that $t+\Delta t \in \left[\frac{j}{k}, \frac{j+1}{k}\right]$, and $\Delta t = O(\frac{1}{(k+1)^{2}})$. Thus $C(t)$ = $C(t+\Delta t) - C^{\prime}(t+\Delta t)\Delta t + o(\Delta t)$ = $C(t+\Delta t) + O(\frac{1}{(k+1)^{2}})$. From (\ref{eq:BezierRecurIniqualityDiff}), (\ref{eq:BezierRecurIniqualityRightSide}) and (\ref{eq:shiftedInequalityLinearDiff}), we have
{
\setlength\abovedisplayskip{1pt}
\setlength\belowdisplayskip{1pt}
\begin{equation}\begin{split}
C(t) &= C(t+\Delta t) + O(\frac{1}{(k+1)^{2}}) \\
&= D(t+\Delta t) + O(\frac{1}{(k+1)^{2}}) \\
&= s_{j}+(k+1)(s_{j+1}-s_{j})(t + \Delta t - \frac{j}{k}) \\
&+ O(\frac{1}{(k+1)^{2}}) \\
&= s_{j}+(k+1)(s_{j+1}-s_{j})(t - (\frac{j}{k} - \frac{j}{k} + \frac{j}{k+1})) \\
&+ O(\frac{1}{(k+1)^{2}}) \\
&= s_{j}+(k+1)(s_{j+1}-s_{j})(t - \frac{j}{k+1}) + O(\frac{1}{(k+1)^{2}})
\label{eq:shiftedInequalityDiff}
\end{split}\end{equation}
}
This concludes the proof for $n=k+1$. Lemma \ref{lemma:mainLemmaDiff} is thus proven.
\end{proof}

Lemma \ref{lemma:mainLemmaDiff} and corollary \ref{corollary:diffUpperBoundCPETS} show that the difference between the upper bound function and the highest possible Bezier curve is $O(\frac{1}{n^{2}})$, where $n$ is the number of control points.
