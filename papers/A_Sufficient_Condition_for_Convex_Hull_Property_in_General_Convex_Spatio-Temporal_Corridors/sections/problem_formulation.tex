\section{Problem Formulation} \label{ProblemFormulation}

\begin{figure}[tbp]
\begin{center}
\includegraphics[width=8cm]{probFormulation.png}
\end{center}
\vspace{-0.2 in}
\caption{A) The range of the i-th control point that guarantees convex hull property. B) The problem scaled back to [0, 1]. It suffices to consider the concave upper bound function $f(t)$ and the control points picked along the upper bound with equal time spacing. The poly-line of the control polygon formed with these control points is defined as CPETS (in red poly-line).}
\label{Fig:probFormulation}
\vspace{-0.2 in}
\end{figure}

Considering a convex spatio-temporal corridor defined in $\left[t_{1}, t_{2}\right]$. The upper bound function $f_{ub}(t)$ is concave, while the lower bound function $f_{lb}(t)$ is convex. We pick $n+1$ control points in the corridor to form a scaled Bezier curve ~\cite{ding2019safe} of degree $n$ which is defined in $\left[t_{1}, t_{2}\right]$. The vertical axis label $Y$ is a generic spacial label, which could represent $X$, $Y$, $S$, $L$, or any spacial variable, as shown in Fig. \ref{Fig:probFormulation} A. The vertical coordinate of the i-th control point is noted as $s_{i}$. We want to find a sufficient condition to make sure the Bezier curve lies in the corridor.

\begin{theorem}
If the series $s_{i}$ (i=0,\;1,\;...,\;n) satisfies
\begin{equation}\begin{split}
s_{i} \in \left[f_{lb}(t_{1}+\frac{i}{n}(t_{2}-t_{1})), f_{ub}(t_{1}+\frac{i}{n}(t_{2}-t_{1}))\right]
\label{eq.0}
\end{split}\end{equation}
the scaled Bezier curve generated from control points lies in the corridor.
\label{theorem:baseTheorem}
\end{theorem}

Theorem \ref{theorem:baseTheorem} shows that it suffices to pick the control points with equal time spacing, and the range of each control point is the value of the upper bound/lower bound functions at the corresponding time. 

To simplify the proof of theorem \ref{theorem:baseTheorem}, without loss of generality, we can scale the time interval back to $\left[0, 1\right]$, consider the upper bound function $f(t)$ only, and pick the control points along the upper bound function with equal time spacing (which will generate the highest possible Bezier curve). The vertical coordinate of the i-th control point thus satisfies $s_{i}=f(\frac{i}{n})$. The poly-line of the control polygon formed with these control points is defined as CPETS (control polygon of equal time spacing), as shown in the red poly-line in Fig. \ref{Fig:probFormulation} B. Due to the property of concave functions, we have $\text{CPETS} \leq f(t)$. This yields theorem \ref{theorem:equivalentTheorem}, which is equivalent to theorem \ref{theorem:baseTheorem}.

\begin{theorem}
$\forall t\in\left[0, 1\right]$, the Bezier curve $C(t)=\sum_{i=0}^{n}s_{i}B\textsupsub{$n$}{$i$}(t)$ satisfies $C(t)$$\leq$CPETS$\leq$$f(t)$, where $f(t)$ is a concave function, $s_{i}=f(\frac{i}{n})$, and 
$B\textsupsub{$n$}{$i$}(t)$ is the Bernstein polynomial which is defined as $B\textsupsub{$n$}{$i$}(t)=C\textsupsub{$i$}{$n$}t^{i}(1-t)^{n-i}$.
\label{theorem:equivalentTheorem}
\end{theorem}

We will decompose the proof of theorem \ref{theorem:equivalentTheorem} into several lemmas. In this paper, all terms share the same definition as they are defined in theorem~\ref{theorem:equivalentTheorem}, unless stated otherwise.

