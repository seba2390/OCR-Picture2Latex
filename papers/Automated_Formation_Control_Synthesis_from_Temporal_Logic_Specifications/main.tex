\documentclass[letterpaper, 10 pt, conference]{ieeeconf}

\IEEEoverridecommandlockouts
% \overrideIEEEmargins
% See the \addtolength command later in the file to balance the column lengths
% on the last page of the document
\usepackage{color}
\PassOptionsToPackage{hyphens}{url}\usepackage{hyperref}
\usepackage{graphicx}
\usepackage{graphics} % for pdf, bitmapped graphics files
% \usepackage{mathptmx} % assumes new font selection scheme installed
% \usepackage{times} % assumes new font selection scheme installed
\usepackage{amsmath} % assumes amsmath package installed
\usepackage{amssymb}  % assumes amsmath package installed
\usepackage{amsthm}
\usepackage{algorithm}
\usepackage{algorithmic}
% \usepackage{program}
\usepackage{mathrsfs}
\usepackage{subcaption}
\usepackage{tikz}
\usetikzlibrary {arrows.meta}

\usepackage{balance}

% \usepackage[backref]{hyperref}
\newtheorem{theorem}{Theorem}
\newtheorem{remark}{Remark}
\newtheorem{definition}{Definition}
\newtheorem{problem}{Main Problem}
\newtheorem{subproblem}{Sub-Problem}
\newtheorem{lemma}{Lemma}
\setlength{\belowcaptionskip}{-0.1cm}
\allowdisplaybreaks[4]

% \newcommand{\qed}{\hfill$\square$}
\setlength{\abovedisplayskip}{1mm}
\setlength{\belowdisplayskip}{1mm}

\title{\LARGE \bf Automated Formation Control Synthesis from \\ Temporal Logic Specifications}


\author{Shuhao Qi$^{\dagger}$, Zengjie Zhang$^{\dagger}$, Sofie Haesaert, Zhiyong Sun% <-this % stops a space
\thanks{$\dagger$ The authors contributed to this paper equally.}
\thanks{This work was supported by the European project SymAware under the grant No. 101070802, and by the European project COVER under the grant No. 101086228.}
\thanks{S. Qi, Z. Zhang, S. Haesaert, Z. Sun are with the Department of Electrical Engineering, Eindhoven University of Technology, The Netherlands.
        {\tt\small \{s.qi, z.zhang3, s.haesaert, z.sun\}@tue.nl}}%}
}

\usepackage{soul} 
\usepackage{amsmath}
\usepackage{amssymb}
\usepackage{xspace}
\usepackage{xifthen}


% Units
\newcommand{\unit}[1]{\ensuremath{\mathrm{\,#1}}\xspace}
\newcommand{\Gyr}{\unit{Gyr}}
\newcommand{\eV}{\unit{eV}}
\newcommand{\keV}{\unit{keV}}
\newcommand{\MeV}{\unit{MeV}}
\newcommand{\GeV}{\unit{GeV}}
\newcommand{\TeV}{\unit{TeV}}
\newcommand{\MB}{\unit{MB}}
\newcommand{\GB}{\unit{GB}}
\newcommand{\TB}{\unit{TB}}
\newcommand{\degree}{\ensuremath{{}^{\circ}}\xspace}
\newcommand{\mas}{\unit{mas}}
\newcommand{\amin}{\unit{arcmin}}
\newcommand{\asec}{\unit{arcsec}}
\newcommand{\angstrom}{\unit{\AA}}
\newcommand{\um}{\unit{$\mu$m}}
\newcommand{\cm}{\unit{cm}}
\newcommand{\km}{\unit{km}}
\newcommand{\kms}{\km \second^{-1}}
\newcommand{\pc}{\unit{pc}}
\newcommand{\kpc}{\unit{kpc}}
\newcommand{\second}{\unit{s}}
\newcommand{\us}{\unit{$\mu$s}}
\newcommand{\photons}{\unit{ph}}
\newcommand{\photon}{\unit{ph}}
\newcommand{\sr}{\unit{sr}}
\newcommand{\Msolar}{\unit{M_\odot}}
\newcommand{\Msun}{\unit{M_\odot}}
\newcommand{\Mstar}{\unit{M_{*}}}
\newcommand{\Lsolar}{\unit{L_\odot}}
\newcommand{\Lsun}{\unit{L_\odot}}
\newcommand{\Lstar}{\unit{L_{*}}}
\newcommand{\Lum}{\ensuremath{ L }\xspace}
\newcommand{\Dsun}{\unit{D_\odot}}
\newcommand{\cmcubes}{\ensuremath{\cm^{3}\second^{-1}}\xspace}
\newcommand{\magn}{\unit{mag}}
%ADW: This is dangerous...
%\renewcommand{\mag}{\magn} 
\newcommand{\mmag}{\unit{mmag}}
\newcommand{\e}{\unit{e^{-}}}
\newcommand{\rms}{\unit{rms}}
\newcommand{\pix}{\unit{pix}}
\newcommand{\rmspix}{\unit{rms/pix}}
\newcommand{\ermspix}{\e \rmspix}
\newcommand{\feh}{{\rm [Fe/H]}}

\newcommand{\teff}{T_{\rm eff}}
%\newcommand{\mas}{\unit{mas}}
\newcommand{\yr}{\unit{yr}}
\newcommand{\masyr}{\unit{\mas \yr^{-1}}}
\begin{document}

\maketitle
\thispagestyle{empty}
\pagestyle{empty}


%%%%%%%%%%%%%%%%%%%%%%%%%%%%%%%%%%%%%%%%%%%%%%%%%%%%%%%%%%%%%%%%%%%%%%%%%%%%%%%%
\begin{abstract}
In many practical scenarios, multi-robot systems are envisioned to support humans in executing complicated tasks within structured environments, such as search-and-rescue tasks. We propose a framework for a multi-robot swarm to fulfill complex tasks represented by temporal logic specifications. Given temporal logic specifications on the swarm formation and navigation, we develop a controller with runtime safety and convergence guarantees that drive the swarm to formally satisfy the specification. In addition, the synthesized controller will autonomously switch formations as necessary and react to uncontrollable events from the environment. The efficacy of the proposed framework is validated with a simulation study on the navigation of multiple quadrotor robots.
\end{abstract}


%%%%%%%%%%%%%%%%%%%%%%%%%%%%%%%%%%%%%%%%%%%%%%%%%%%%%%%%%%%%%%%%%%%%%%%%%%%%%%%%
\section{INTRODUCTION}
\label{sec:intro}

Multi-robot system control with complex tasks is challenging due to the high dimensions and the substantial amount of constraints~\cite{chen2022}. Taking the search-and-rescue task as an example, the robots have to sequentially achieve a series of subtasks, including locating the survivors, navigating to the rescue spots, and transporting the survivors to a designated safe zone. In this type of complex task, robot swarms should autonomously change formations to cooperatively execute sensing and communication operations or to traverse narrow spaces. Such practical tasks involve complex navigational requirements imposed by the sequential subtasks in combination with formation control requirements on the robots. While the former navigational requirements have been successfully solved by first specifying them with Linear-time Temporal Logic (LTL) and using related tools~\cite{calin2013ijrr, 2009tro}, there is not much work that explicitly incorporates the formation requirements. This paper studies the control synthesis of multi-robot systems for complex tasks represented by LTL specifications with automated formations.
% it could be good to mention that specifying the formation in LTL is also new.

For constrained control of multi-robot systems, problems such as obstacle avoidance and reachability in complex environments can be solved with recently developed control methods that also give runtime guarantees. These methods include control barrier functions (CBF) \cite{ames2017} and finite-time control Lyapunov function (CLF) ~\cite{FxT2022}. CBFs can ensure the strict satisfaction of state-dependent constraints for dynamic systems by imposing the set invariance property~\cite{ames2017}. In addition, fixed-time CLF~\cite{FxT2022}  can ensure that the system converges to a given equilibrium point or a set within user-defined time. Also, in practice, it is convenient to incorporate CBF and CLF constraints in optimization-based controllers, like quadratic programming (QP), which can balance safety and convergence without massive computation~\cite{Ames2017tac, FxT2022}. 
In this sense, CBF-CLF QP has been employed to ensure the safety and convergence of multi-robot systems~\cite{wang2017safety, tan2021distributed} and the satisfaction of the spatio-temporal constraints of temporal logic specifications~\cite{lindemann2018control, Sri2021}.
 

LTL specifications have been used to enable the collective behavior of multi-robot systems~\cite{calin2013ijrr, sun2022multi,calin2007tro}.  The allocation numbers of robots or systems to specific regions in space and time can be represented by counting temporal logic~\cite{tro2020CTL} and graph temporal logic~\cite{djeumou2020rss}. Alternatively, when each agent in a system has been individually assigned a distinct temporal logic specification, local cooperation of agents has been developed in \cite{lindemann2019control} where the least violating control is used in case of conflicting specifications. In contrast, in \cite{calin2007tro}, the collective behavior of multi-robot systems is achieved, where the mean and variance features of the robot swarm are employed to control arbitrarily large swarm systems to satisfy temporal logic specifications. In some practical problems, multi-robot systems are expected to not only exhibit simple collective behaviors but also to achieve specific formations to traverse special terrain and to execute sensing and communication tasks. In contrast to \cite{calin2007tro}, we aim at the explicit specification for automated formations in the navigation control of a robot swarm. 


More precisely in this paper, we propose a formal framework for a multi-robot swarm system to solve a complicated autonomous navigation task with automated formations. A linear temporal logic (LTL) formula is used to specify this complicated navigation task and a symbolic model is abstracted to characterize the behavior of the swarm system. Based on this model, a symbolic controller is synthesized to generate the waypoints and desired formations subject to the LTL specification incorporating the influence of the environmental signals. Then, a QP-based control refinement method with CBFs and fixed-time CLFs is developed to ensure the satisfaction of the LTL specification with runtime guarantees. In such a way, the developed control method enables the swarm system to succeed in the navigation task with automated formations. Such a framework holds vast potential for real-world robotic swarm applications. The main contributions of this paper include 1) a framework for autonomous navigation and formation switching of multi-robot systems under LTL specifications, including finite abstractions, symbolic control synthesis, and control refinement; 2) an efficient QP-based control refinement with runtime guarantees on the task specification (autonomous navigation with automated formations) and collision avoidance.



% Firstly, the robot swarm is required to reach a specific goal point in the environment while passing through a series of waypoints. Secondly, the swarm should automatically change its formation according to the terrain restrictions of the environment. Thirdly, all robots in the swarm should strictly avoid collisions with other robots and obstacles. We use an LTL formula defined on multiple atomic propositions to specify the desired waypoints and formations of the swarm. Then, we formulate the transition of the waypoints and formations of the system as a DFTS and synthesize a symbolic control strategy by solving a game over the product of the DFTS and the automaton transformed from the LTL specification. Meanwhile, we use a group of CBFs and fixed-time CLFs to prescribe the runtime safety and the fixed-time convergence of the robot swarm. A QP problem is solved to generate the control inputs for individual robots by refining the symbolic control strategy. The refinement of the controller ensures the satisfaction of the predefined LTL specifications. Our main contribution is the proposed framework itself using formal methods and symbolic control strategies to solve practical complicated multi-robot tasks, especially the realization of the correspondence between the abstract transition model and the concrete multi-robot swarm system and how it helps achieve the objectives of the swarm navigation task. The framework and its realization are promising to be extended to other types of complex multi-robot coordinated tasks.

The rest of the paper is organized as follows. Sec.~\ref{sec:fps} gives the preliminary knowledge and %formulates
the main problem. %In Sec.~\ref{sec:frame}, we propose
The main results are given in Sec.~\ref{sec:frame}.% and  %to this problem, namely the synthesis of the symbolic control strategy and the control refinement by solving a QP problem. In
In Sec.~\ref{sec:cas}, we validate our framework and solutions with a simulation case on robot swarm navigation. Finally, Sec.~\ref{sec:con} concludes the paper.


%%%%%%%%%%%%%%%%%%%%%%%%%%%%%%%%%%%%%%%%%%%%%%%%%%%%%%%%%%%%%%%%%%%%%%%%%%%%%%%%
\section{Preliminaries and Problem Statement}
\label{sec:fps}
%
%{\color{orange}
%In this section, we first formulate the dynamic models of multi-robot systems and the associated definitions of formation control. Furthermore, linear temporal logic formulas are introduced and used to represent high-level specifications of formation control, and the general problem in the research is finally stated.}

\subsection{Multi-Robot System}
\label{sec:mrs}
Consider a multi-robot swarm system with $r $ %\in \mathbb{I}^+$
robots, where each robot is described by  the following dynamic equation,
\begin{equation}
\mathcal{R}_i: \dot{x}_i(t) = g(x_i(t), u_i(t)), \quad i=1,\ldots,r,
\label{eq:sys_pb}
\end{equation}
where $x_i(t)\!\in\!\mathbb{X}\!\subset\!\mathbb{R}^n, u_i(t)\!\in\!\mathbb{U}\!\subset\!\mathbb{R}^m$ are the state and the control input of the $i$-th robot at time $t\!\in\!\mathbb{R}_{\geq 0}$, respectively, and $g\!:\!\mathbb{R}^n\!\times\!\mathbb{R}^m\!\rightarrow\!\mathbb{R}^n$ is a smooth function that describes the dynamic model of the robot. For brevity, we use vectors $x(t)=[\,x_1^T(t), x_2^T(t), \cdots, x_r^T(t)\,]^T\!\in\!\mathbb{X}^r\!\subset\!\mathbb{R}^{rn} $ and $u(t)=[\,u_1^T(t), u_2^T(t), \cdots, u_r^T(t)\,]^T\!\in\!\mathbb{U}^r\!\subset\!\mathbb{R}^{rm} $ to denote the state and control input of all robots in the system. Especially, the initial system state is denoted by $x(0) = [\,x_1^T(0), x_2^T(0), \cdots, x_r^T(0)\,]^T$.

In this paper, the interaction among the robots is described by an undirected and completed graph, in the sense that all robots in the swarm are fully connected by local communication.
%The interaction between the robots is described by an undirected graph $\mathcal{G} = \{ \mathcal{V}, \mathcal{E} \}$, where $\mathcal{V} = \{1,2,\cdots,r\}$ denotes the set of the robots and $\mathcal{E} \subseteq \mathcal{V} \times\mathcal{V}$ is the set of graph edges that describe robot interactions, where $(i,j) \in \mathcal{E}$ with  $i \neq j$ if robots $i$ and $j$ have interaction.
We define $x_c = \frac{1}{r} \sum_{i=1}^r x_i$ as the \textit{centroid}, or the geometric center of the swarm and $x_{ij} = x_{i}-x_{j}$ as the relative displacement between two robots $i \neq j$. A certain \textit{formation} of the swarm system is described by all the relative displacements, i.e., $f = \{f_{ij}\}_{\frac{r(r-1)}{2}}$, where $f_{ij} \in \mathbb{R}^n$, for $i,j\in\{1,2,\cdots,r\}$, $i\neq j$, denotes the desired displacement between two robots. We define $\mathbb{F}$ as the formation space that contains all possible formations $f$ of the swarm system. Similarly, we define $\mathbb W \subset \mathbb{R}^n$ as the workspace of the swarm that contains all possible positions $w$ of the centroid $x_c$.

% {\color{red}
% In a practical scenario, the environment of multi-robot systems would be structured, indicating the environment is decomposed into different spaces with different functionalities. In such environments, multi-robot systems normally are required or have to change formations to fulfill the tasks. Without the loss of generality, such a structured environment can be represented by a set of closed sets, i.e., $\mathcal{E}=\{\mathcal{C}_i\}_{i=1}^{N_c}$ where $\mathcal{C}_i=\{p \in \mathbb{W} \subset \mathbb{R}^n: h_i(p) \leq 0\}$ is the sub-level set of a continuous function, where $\mathbb{W}$ denotes the workspace of a robot swarm explicitly mapped from $\mathbb X$.
% }
\subsection{LTL Specification for Navigation Task}
\label{sec:ltl}

We first introduce Linear Temporal Logic (LTL)~\cite{modelchecking2008, calin2017book} as follows. \\
%, a formal language used to specify the navigation task with automated formations for the swarm system.
\noindent{\textbf{Syntax.}} The syntax of LTL is recursively defined as,
\begin{equation}
    \psi ::= \top \mid p \mid \neg \psi \mid \psi_1 \wedge \psi_2 \mid \bigcirc \psi \mid \psi_1 \mathsf{U} \psi_2,
\end{equation}
where  $\psi_1$, $\psi_2$ and $\psi$ are LTL formulas, $p \in \AP$ is an atomic proposition, $\neg$ is the negation operator, $\wedge$ is the conjunction operator that connects two LTL formulas, and $\bigcirc$ and $\mathsf{U}$ represent the \textit{next} and \textit{until} temporal operators, respectively. Based on these essential operators, other logical and temporal operators, namely \textit{disjunction} $\vee$, \textit{implication} $\rightarrow$, \textit{eventually} $\lozenge$, and \textit{always} $\square$ can be defined as, $\psi_1 \vee \psi_2:= \lnot \!\left(\psi_1 \wedge \psi_2 \right)$, $\psi_1 \rightarrow \psi_2$ $:= \lnot \psi_1 \vee \psi_2$, $\lozenge \psi :=\top \mathsf{U} \psi$, and $\square \psi :=\neg \lozenge \neg \psi$.
\smallskip

\noindent{\textbf{Semantics.}} Consider a set of atomic propositions $\AP = \left\{ p_1, \dots, p_N \right\}$ which defines an alphabet $2^{\AP}$, where each letter $\omega \in 2^{\AP}$ contains the set of atomic propositions that are true. An infinite string of letters is a word $\pmb{\omega} = \omega_0 \omega_{1} \omega_{2} \dots$, where $\omega_i \in 2^{\AP}$, $i \in \mathbb{N}_{\geq 0}$, with a suffix $\pmb{\omega}_k = \omega_k \omega_{k+1} \omega_{k+2} \dots$, $k \in \mathbb{N}_{\geq 0}$. For a given word $\pmb{\omega}$, basic semantics of LTL are given as
$\pmb\omega_k \models p$, if $p \in \omega_k$; $\pmb\omega_k \models \lnot p$, if $p \notin \omega_k$;
$\pmb\omega_k \models \psi_1 \wedge \psi_2$, if $\pmb \omega_k \models \psi_1$ and $\pmb \omega_k \models \psi_2$;
$\pmb\omega_k \models \bigcirc \psi$, if $\pmb\omega_{k+1} \models \psi$;
$\pmb\omega_k \models \psi_1 \mathsf{U} \psi_2$, if $\exists$ $i \in \mathbb{N}$ such that $\pmb\omega_{k+i} \models \psi_2$, and $\pmb\omega_{k+j} \models \psi_1$ holds $\forall \, 0\leq j < i$.
\smallskip

\noindent{\textbf{Reactive LTL formulas for robot swarm.}}
In this paper, we use an LTL formula to specify whether the swarm system achieves the navigation task with automated formation. Motivated by~\cite{calin2007tro}, we select to use the centroid position and formation as essential features to respectively characterize the navigation and interaction behaviors of the swarm system. Therefore, we introduce atomic propositions for the swarm centroid in the workspace $\AP_w$ based on a labeling function $\mathcal{L}_w\!:\!\mathbb W\!\rightarrow\!2^{\AP_w}$ and those for the formation $\AP_f$ with a labeling function $\mathcal{L}_f\!:\!\mathbb F \!\rightarrow\!2^{\AP_f}$. Moreover, to specify how the swarm system should react to external signals from the environment, we use a finite set $\mathcal{E}$ to denote all possible states of the environmental signal and the set of atomic propositions $\AP_e$ to describe the property of the signal. Their correspondence is described by a labeling mapping $\mathcal{L}_e\!:\!\mathcal{E}\!\rightarrow\!2^{\AP_e}$. In this sense, the overall set of atomic propositions of the swarm system for the navigation task is $\AP\!:=\!\AP_e\cup\AP_w\cup\AP_f$. Then, we can use an LTL formula $\psi$ defined on the output word of the swarm system $\pmb{\omega} = \omega_0 \omega_{1} \omega_{2} \dots$, where $\omega_k\!\in\!2^{\mathsf{AP}}$, $k\!\in\!\mathbb{N}$, to specify the navigation task. We say that the robot achieves the navigation task if its output word satisfies the specification, i.e., $\pmb{\omega}\vDash \psi$. In this sense, the overall labeling mapping of the swarm system is the combination of all mappings $\mathcal{L}_w$, $\mathcal{L}_f$, and $\mathcal{L}_e$, i.e., $\mathcal{L}: \mathbb{W} \times \mathbb{F} \times \mathcal{E} \rightarrow 2^{\AP}$. The subset LTL formulas known as Generalized Reactive(1) (GR(1))~\cite{BLOEM2012911}, is specifically well fitted to deal with both external and internal variables. GR(1) formulas are of the form  $\psi:= \psi_e \rightarrow \psi_s$, where $\psi_e$ constrains the allowed behavior of the uncontrolled propositions  $\mathsf{AP}_e$ and $\psi_s$ constrains the desired behaviors of the system with the controlled propositions $\mathsf{AP}_w \cup\mathsf{AP}_f$.
% To this end, we consider a special class of LTL that can represent assume-guarantee specifications. GR(1) formulas are of the form  $\psi:= \psi_e \rightarrow \psi_s$, where $\psi_e$ depicts the behaviors of the uncontrolled environment and $\psi_s$ represents the desired behaviors of the system. Note that in this context  $\psi_e$ and $\psi_s$ are defined in a structured way as defined in~\cite{BLOEM2012911}}.



% The synthesis problem is a dynamic game between the uncontrollable environment and the to-be-synthesized system. Therefore, the synthesized strategy for GR(1) specifications should make the system satisfy $\psi_s$ by reasonably reacting to all the behaviors of environments depicted by $\psi_e$.


\subsection{Problem Statement}\label{sec:probstate}
% \red{Change this pb and remove all notions of way points etc and take as a starting point the definition of AP in Eq.~\eqref{eq:specs}.}

Given the multi-robot system in \eqref{sec:mrs} and the task specification $\psi$ defined in \eqref{sec:ltl}, the navigation task studied in this paper can be formulated as a synthesis problem for the specification $\psi$. Before we give the formal problem statement, we explain how automated formation is addressed for the navigation task. The main objective of the navigation task is that the centroid of the swarm system should ultimately reach the navigation goal along a feasible path in the environment (\textit{navigation}). Meanwhile, the environmental terrain requires that certain spots must be passed by certain formations. Therefore, the swarm system should also automatically switch to the feasible formations when passing the corresponding spots (\textit{automated formation}). Also, the swarm system should always react to the environmental signals and perform the correct response (\textit{reaction}). Besides, all robots in the swarm should avoid collisions with each other and the obstacles in the environment (\textit{collision avoidance}). The formal problem statement is given as follows.

%, which specifies the desired navigation and formation task with $\AP_w$ and $\AP_f$ respectively, and the desired reactive behaviors to external signal depicted with $\AP_e$. The main objective of the study is to design a controller that generates control inputs for each robot in Eq.~\eqref{eq:sys_pb} to drive the swarm to satisfy centroid reaches the waypoints with the corresponding formations as specified by $\psi$. Specifically, a robot swarm is expected to fulfill navigation tasks and formation control as expected and respond to external signals instantly. To achieve this, two implicit requirements should be satisfied: 1) The swarm should autonomously change proper formations to adapt to a structured environment $\mathcal{E}$; 2) The runtime safety (free collision) for robot swarm should be ensured. The main problem is formally formulated as follows,

% \begin{problem}[Navigation with automated formations]
% \label{pb:overall}
% % \red{Rewrite the pb based on the previous spec.}
% Given a multi-robot swarm system $\mathcal{R} = \{\mathcal{R}_1, \mathcal{R}_2, \cdots, \mathcal{R}_N\}$ defined on a closed domain $\X \subset \mathbb{R}^{r\times n}$ as in Eq.~\eqref{eq:sys_pb} and a GR(1) formula $\psi:= \psi_e \rightarrow \psi_s$ defined over $\AP$ in Eq.~\eqref{eq:specs}, find a  provably-correct reactive control policy {\color{red}$\pi:\mathbb{X}^n \rightarrow \mathbb{U}^n$} that navigates the robot swarm $\mathcal{R}$ to satisfy the $\psi$ in a structured environment $\mathcal{E}$ by autonomously changing formation and avoiding obstacles. \qed
% \end{problem}

%{\color{blue}
\begin{problem}[Navigation with Automated Formation]
\label{pb:overall}
% \red{Rewrite the pb based on the previous spec.}
For a multi-robot swarm system $\mathcal{R} = \{\mathcal{R}_1, \mathcal{R}_2, \cdots, \mathcal{R}_r\}$ as defined in Eq.~\eqref{eq:sys_pb} with a state space $\mathbb{X}^r$, a workspace $\mathbb{W}$, a formation space $\mathbb{F}$, an environmental signal space $\mathcal{E}$, a labeling mapping $\mathcal{L}:\mathbb{W}\times\mathbb{F}\times\mathcal{E} \rightarrow \mathsf{AP}$, and a GR(1) formula $\psi:= \psi_e \rightarrow \psi_s$ defined on the alphabet $2^{\AP}$ specifying the requirements on navigation, automated formation, reaction, and collision avoidance, find a provably-correct reactive control policy to ensure that the output word $\pmb{\omega}$ satisfies the specification $\psi$, i.e, $\pmb{\omega} \vDash \psi$. \qed
\end{problem}
%}



% The solution to this problem is given in Sec.~\ref{sec:frame}.
% The two sub-problems of Problem \ref{pb:overall} prescribe the requirements for task specifications and runtime safety, respectively.


%%%%%%%%%%%%%%%%%%%%%%%%%%%%%%%%%%%%%%%%%%%%%%%%%%%%%%%%%%%%%%%%%%%%%%%%%%%%%%%%
% \section{Preliminary}\label{sec:pre}
% \red{This section will be deleted and briefly introduced in the QP controller.}
% In this section, we introduce the preliminary knowledge about control barrier functions (CBFs) and fixed-time control Lyapunov functions (FxT-CLFs) that will be used in the design of swarm controllers.

% \subsection{Fixed-Time Control Lyapunov Function}

% Stability is a property that guarantees the system is driven to an equilibrium (or a set of equilibria). In contrast to asymptotic stability (AS) which pertains to convergence as time goes to infinity, finite-time stability is a concept that guarantees the convergence of solutions in finite time. Fixed-time stability (FxTS) is an even stronger notion than finite-time stability, where the time of convergence does not depend upon the initial conditions.
% Consider the control affine system as follows,
% \begin{equation}\label{eq:caffi}
% \dot{x}(t) = f(x(t)) + g(x(t))u(t),
% \end{equation}
% where $x(t) \in \mathbb{X}^n \subset \mathbb{R}^n$ and $u(t) \in \mathbb{U}^n \subset \mathbb{R}^m$ are, respectively, the low-level physical state and input of the system at time $t \in \mathbb{R}_{\geq 0}$, $f: \mathbb{R}^n \rightarrow \mathbb{R}^n$ is a smooth vector field, and $g: \mathbb{R}^n \rightarrow \mathbb{R}^{n \times m}$ is the smooth gain matrix.
%  Then, the convergence of the system trajectories to a compact set within a fixed time can be encoded using a class of fixed-time (FxT) CLFs~\cite{FxT2022}.
% % {\color{red}Please check whether this definition in the current form is reasonable.}
% \begin{definition}
% (Fixed-Time Control Lyapunov Function, FxT CLF):
% For a control affine system defined as \eqref{eq:caffi}, a continuously differentiable function $V: \mathbb{R}^n \rightarrow \mathbb{R}$ is referred to as a fixed-time (FxT) CLF if $V(x)$ is positive definite w.r.t $\mathbb{X} \in \mathbb{X}^n$ and the following condition holds for all $x \in \mathbb{X}^n \backslash \mathbb{X}$,
% \begin{equation}\label{eq:cbf}
% \inf _{u \in \mathbb{U}}\left\{L_f V(x)+L_g V(x) u\right\} \leq-\alpha_1 V(x)^{\gamma_1}-\alpha_2 V(x)^{\gamma_2},
% \end{equation}
% where $\alpha_1, \alpha_2\!>\!0$, $\displaystyle \gamma_1\!=\!1\!+\!\frac{1}{\mu}$, $\displaystyle \gamma_2\!=\!1\!-\!\frac{1}{\mu}$, $\mu \!>\! 1$. All control inputs $u(t)$ that render an FxT-CLF ensure  that the system state converges to $x \in \mathbb{X}$ within a finite time $T$ from any initial state $x_0 \!\in\! \mathbb{X}^n$, where $\displaystyle T \!\leq\! \frac{\mu \pi}{2 \sqrt{\alpha_1 \alpha_2}} \!\leq\! T_{u d}$, $T_{u d} \!\in\! \mathbb{R}^+$. \qed
% \end{definition}


% \subsection{Control Barrier Function}

% The principle of safety demands the avoidance  of dangerous occurrences, both in the present and future time. It is common to define the forward invariance of safe sets as safety.
% \begin{definition}[Forward Invariance]
% A set $\mathcal{C} \subset \mathbb{X}^n$ is forward invariant if $\forall \, x_0 \in \mathcal{C}$, $x(t) \in \mathcal{C}$ for $x(0) = x_0$ and all $t>0$. The system $\dot{x}=f(x)$ is safe with respect to the set $\mathcal{C}$ if the set $\mathcal{C}$ is forward invariant.  \qed
% \end{definition}
% In this work, we use the following notion of zeroing Control Barrier Functions (CBFs), introduced in \cite{ames2017}, to ensure forward invariance of the safe set $\mathcal{C}$.
% \begin{definition} [(Zeroing) Control Barrier Function, CBF]
% Let $\mathcal{C} \!\subset\! D \!\subset\! \mathbb{R}^n$ be the super-level set of $a$ continuously differentiable function $h: D \rightarrow \mathbb{R}$. Then $h$ is a (zero) control barrier function if there exists an extended class $\mathcal{K}_{\infty}$ function $\alpha$ such that for the control affine system~\eqref{eq:caffi},
% \begin{equation}
% \sup_{u \in U}\left[L_f h(x)+L_g h(x) u\right] \geq-\alpha(h(x)),~\forall \, x \in D,
% \end{equation}
% where extended class $\mathcal{K}_{\infty}$ function is a strictly increasing function $\alpha:\mathbb{R} \rightarrow \mathbb{R}$ with $\alpha(0) = 0$.
% \end{definition}
%  Furthermore, the condition of control inputs for the given CBF $h$ is derived to ensure forward invariance,
% \begin{equation}
% K_{\mathrm{cbf}}(x)=\left\{u \in U| L_f h(x)\!+\!L_g h(x) u\!+\!\alpha(h(x)) \geq 0\right\}.
% \end{equation}




%%%%%%%%%%%%%%%%%%%%%%%%%%%%%%%%%%%%%%%%%%%%%%%%%%%%%%%%%%%%%%%%%%%%%%%%%%%%%%%%
\section{Framework and Controller Design}
\label{sec:frame}
%In this section, we first decompose the {main problem} Problem \ref{pb:overall}  into smaller problems. Subsequently, the symbolic controller is introduced in detail. Finally, the QP-based control refinement is designed to provide a provably correct solution to satisfy LTL specifications.

% \subsection{Solution Based on A Symbolic Model}
\subsection{The Control Design Framework }
% {\color{red}
% The challenges posed by navigation with automated formations problem for robot swarms stated in Prob.~\ref{pb:overall} present significant difficulties for both existing automata-based methods and optimization-based methods. More specifically, attempting to establish an equivalent abstraction for a multi-robot system, particularly one characterized by nonlinear dynamics, leads to the well-known problem of computational explosion. By contrast, optimization-based methods cannot easily handle complex tasks like Prob.~\ref{pb:overall} which involves formation assignment and reactions for external signals.

% In light of these challenges, we propose to hierarchically decompose Prob.~\ref{pb:overall} into a high-level sub-problem in discrete state space and a low-level sub-problem in continuous state space. Inspired by ``existential abstraction" as introduced in~\cite{2013iros}, we select to abstract a discrete-state transition system via coarse abstraction rather than equivalent abstraction, which can avoid computational difficulty. For each transition in the abstract system, there must be a controller to drive the dynamic system of the robot swarm defined in Eq.~\ref{eq:sys_pb} to fulfill the transition. Consequently, a symbolic strategy designed to ensure that the abstract system satisfies a TL specification can also guide the original dynamic system to meet the same TL specifications. Therefore, we can solve Prob.~\ref{pb:overall} by solving the control synthesis problem for an abstract system. Given this, the decomposed sub-problems will be formulated as follows.
% }
To capture behaviors of the dynamic system in Eq.~\ref{eq:sys_pb}, an abstract system can be defined in the following formulation,
\begin{definition}\label{def:fts}
(Deterministic Finite Transition System, DFTS):
A deterministic finite transition system is a tuple $\mathcal{T}\!=\! (S, s_0,  A, \delta$, $\AP$, $\mathcal{L})$, where $S$ and $A$ are finite sets of states and actions, $s_0$ is the initial state, $\delta\!:\! S \!\times\! A \!\rightarrow\! S$ is a transition function that prescribes the state transition under a certain input, $\AP$ is a finite set of atomic propositions, and $\mathcal{L}\!:\! S \!\rightarrow\! 2^{\AP}$ is a labeling function. \qed
\end{definition}

	Given a sequence of actions $\mathbf{a} \!=\! \ac_0 \ac_1 \ac_2 \cdots$ with $\ac_i \!\in\! \A$, % generated by ${\color{green}\right }$,
	a DFTS $\mathcal{T}$ initiated at $s_0 \!\in\! S$ generates a trajectory or a run $\pmb{s} \!=\! s_0 s_1 s_2, \ldots$, where $s_{i+1}\!=\!\delta(s_i,\ac_i)$, $i \!\in\! \mathbb{N}$. Accordingly, the output word of the DFTS $\pmb \omega = \omega_0 \omega_1 \omega_2 \cdots$ is uniquely defined for a given initial state $s_0 \in S$, where $\omega_i \in 2^{\AP}$, $i \in \mathbb{N}$.
Let a history $h_k = s_0 a_0 s_1 a_1  s_2 a_2...s_k$ be given at time $k$ with $h_k\!\in\! H$, we can then define a control strategy as a map $\pia$ from the set of histories to the action set, i.e., $\pia: H \!\rightarrow\! \A$.
%\red{For a DFTS $\mathcal{T}$,
A control strategy in the form  $\pia\!:\! S \!\rightarrow\! \A$ is a Markov strategy as it only depends on current state of $\mathcal{T}$.
%[With this type of strategy you cannot solve a LTL spec!] }

\smallskip

In the next subsection, we will show that we can exactly develop a symbolic model as the abstraction of the swarm system in Eq.~\eqref{eq:sys_pb} using a DFTS with the same atomic proposition sets $ \mathsf{AP}_w$ and $\mathsf{AP}_f$. %$2^{\mathsf{AP}}$.
More specifically, let us denote two finite sets $\mathcal{W}\!:=\!\{w_1,w_2,\ldots\}\!\subset\!\mathbb W$ and $\mathcal{F}\!:=\!\{f_1,f_2,\ldots\}\!\subset\!\mathbb F$ with waypoints in the workspace $\mathbb W$ and the formation space $\mathbb F$ of the swarm system. %, i.e., any $w\in\mathcal{W}$ or $f\in\mathcal{F}$ corresponds to a point in $\mathbb W$ or $\mathbb F$.
The state set of the DFTS $S\!:=\mathcal{W}\!\times\!\mathcal{F}$ is an abstraction of the feature space of the swarm system $\mathbb{W}\times \mathbb{F}$. As a consequence, the task specification $\psi$ defined for the swarm system can now be specified for the abstract model with labeling map ${\mathcal{L}}\!:\!\mathcal{W}\!\times\!\mathcal{F} \!\times\!\mathcal{E} \!\rightarrow\!2^{\AP}$. Consider the history signal $h_k^{\mathcal{E}}\!:=\!s_0 e_0 a_0 s_1 e_1 a_1  s_2e_1 a_2...s_k\!\in\! H^{\mathcal{E}}$ extended with the uncontrolled variables $e_k\in \mathcal E$. Then, we can decompose the main problem into two sub-problems with one being the control synthesis for the abstract model $\mathcal{T}$ and the other as control refinement to guarantee the soundness of the output words $\pmb{\omega}$ for the concrete swarm system.


\begin{subproblem}[Symbolic control synthesis]
\label{spb:strategy}
Consider a DFTS $\mathcal{T}$ defined in Def.~\ref{def:fts} as the abstraction of the swarm system given in Eq.~\eqref{eq:sys_pb}. Synthesize a symbolic control strategy, $\pia: H^\mathcal{E} \rightarrow A$, such that for any feasible environmental signal from $\mathcal{E}$ and for the initial state $s_0$, the LTL specification $\psi$ is satisfied.
\end{subproblem}


% The robot swarm should also satisfy the runtime safety requirements, i.e., collision avoidance with other robots and obstacles. We use $d_O \in \mathbb{R}^+$ to represent the minimal distance between two interacted robots. Based on this, we define a safety set $\mathcal{D} = \{x | h_{\mathcal{D}}(x) \geq 0 \}$, where $\mathcal{D}(x) = \|x \|^2 - d_O^2$, where $d_O \in \mathbb{R}^+$ is a tolerance threshold. Collision avoidance with other robots requires that $x_{ij} \in \mathcal{D}$ for all $(i,j) \in \mathcal{E}$. Also, we use the set $\mathcal{O} \!=\! \{x \in \mathbb{R}^n| h_{\mathcal{O}}(x) \!\geq\! 0 \}$ to represent the maximal ellipsoid collision-free region for the robot swarm, where $h_{\mathcal{O}}(x)\!=\!  1\!-\!(x\!-\!\eta)^T \!P(x\!-\!\eta)$, where $\eta \!\in\! \mathbb{R}^n$ and $P \in \mathbb{R}^{n \times n}$ are constant parameters determined by the environment. Collision avoidance with obstacles requires $x_c \in \mathcal{O}$.


Given the control strategy $\pia$ for the abstract system $\mathcal{T}$, we should also make sure there is a refined control policy for the swarm system that drives the multi-robot system to satisfy the navigation specification $\psi$ without collisions inside and outside the swarm.

% \red{ Given the control strategy ${\color{green}\pia}$ for the abstract system $\mathcal{T}$, we should also make sure there is a refined control policy for the swarm system that drives the multi-robot system to strictly execute $\pmb{s}$ generated by ${\color{green}\pi}$ in $\X$. Continuous states set, centroid position set and formation set corresponding to each $s\!\in\!S$ can be obtained via the mapping functions respectively: $X_s\!=\!\mathcal{M}(s)$, $W_s\!=\!\mathcal{W}(s)$, $F_s\!=\!\mathcal{F}(s)$.}


% {\color{red}
% \begin{subproblem}[Control refinement]
% \label{spb:control}

% For each transition $s \rightarrow s'$ in \pmb{s} generated by ${\color{green}\pi}$, design a controller $u: \X\times\mathbb{W}\times \mathbb{F} \rightarrow \U$, such that the robot swarm achieves $x(T_{ud}) \in \mathcal{M}(s')$, i.e., $x_c(T_{ud}) \!\in\!\mathcal{W}(s')$ and $\{x_{ij}(T_{ud})\}_{\frac{r(r-1)}{2}}\!\in\!\mathcal{F}(s')$ within a fixed time $T_{ud}$ for any initial conditions $x(0) \in \mathcal{M}(s)$, and each robot should stay in the set $\mathcal{W}(s)\!\cup\!\mathcal{W}(s')$ without any collisions.
% \end{subproblem}
% }


\begin{subproblem}[Control refinement]
\label{spb:control}
For the swarm system given in Eq.~\eqref{eq:sys_pb} and its abstract model, a DFTS $\mathcal{T}$ defined in Def.~\ref{def:fts}, with a symbolic control strategy $\pia$ solved in \textit{Sub-Problem~\ref{spb:strategy}}, design a controller that maps the symbolic state action pairs to a continous control map %$u: \X\times\mathbb{W}\times \mathbb{F} \rightarrow \U$
 $ \pi:\X^r\times S \times A\rightarrow \mathbb U^r$,
such that the output word of the swarm system $\pmb{\omega}$ assigned by the labeling mapping $\mathcal{L}$ satisfies the LTL specification defined in Sec.~\ref{sec:ltl}, i.e., $\pmb{\omega} \models \psi$.
\qed
\end{subproblem}

% Such decomposition will fully combine the advantages of automata-based methods and optimization-based methods.


\subsection{Symbolic Control Synthesis}
\label{sec:scs}
This subsection discusses the synthesis of the symbolic control strategy in \textit{Sub-Problem~\ref{spb:strategy}}. The synthesis process is performed in three steps:
 
\subsubsection{Step 1: Determining Waypoint Sets}
The first step is to determine the sets of waypoints in the workspace and the formation, namely $\mathcal{W}$ and $\mathcal{F}$ which are important to construct the abstraction model $\mathcal{T}$, as introduced in Sec.~\ref{sec:frame}.
A waypoint is a desired position $w\!\in\!\mathcal{W}$ for the swarm centroid to reach or a desired formation $f\!\in\!\mathcal{F}$ for the swarm to achieve.
In this work, the waypoints are selected by fully incorporating the physical property of the swarm, the practical conditions of the environment, and the specific requirements of the tasks.  For example, there might exist some narrow spaces where the swarm can only pass with a certain formation. Also,  the number of the waypoints is selected as possibly small while maintaining a dense distribution to make a balance between the scale of the abstraction model and the feasibility of the navigation task.
%the partitioned continuous spaces $\mathbb{W}$ and $\mathbb{F}$,  based on waypoints in the finite sets $\mathcal{W}$ and $\mathcal{F}$, should be determined such that all continuous variables $w\in\mathbb{W}$ and $f\in\mathbb{F}$ in the same partition should correspond to the same letter as their waypoints in  $W\in\mathcal{W}$ and $F\in\mathcal{F}$, respectively, based on the labeling map $\mathcal{L}$. %Improper partitioning may lead to infeasibility of \textit{Sub-Problem~\ref{spb:control}}.


 
\subsubsection{Step 2: Realization of the DFTS}
After determining the finite sets of waypoints $\mathcal{W}$ and $\mathcal{F}$, the next step is to use them to construct a DFTS $\mathcal{T}\!=\!\{S, s_0, A, \delta, \AP, \mathcal{L}\}$ to realize an abstract model for the swarm system. Directly, the state set is $S\!=\!\mathcal{W}\!\times\!\mathcal{F}$. The action set $A$ is also a finite symbolic set that contains all possible actions for state transitions. The atomic proposition set $\mathsf{AP}$ and labeling mapping $\mathcal{L}$ are the same as Sec.~\ref{sec:ltl}.
Thus, the most critical part of this step is to determine the transition function $\delta$ which describes to which waypoint the system transits given the current waypoint. We first assume a full transition and then eliminate the infeasible transitions by looking into the environmental conditions and the dynamic model of the swarm system.
%
%The action set $A$ is also a finite symbolic set that prescribes all possible actions for certain states which are determined by the adjacency relation among different waypoints and formations, i.e., which waypoints and formations can be reached after the current ones. For a certain waypoint $\tilde{w} \in \tilde{W}$ and $\tilde{f} \in \tilde{F}$, the transition $\delta$ prescribes the next feasible waypoint and formation under a certain action $\ac \in A$.
%
%There are several principles to realize the transition $\delta$. Firstly, the adjacency relation among the waypoints and the formation space provides the largest sets of possible transitions. Secondly, infeasible transitions subject to environmental restrictions should be eliminated. 
For example, we eliminate a transition from one waypoint to another if it forces the swarm to pass an area with an impractical formation.
%is not realizable if the current environmental feature does not guarantee sufficient space to tolerate the transient stage for the next formation. 
Moreover, the transition should also incorporate the feasibility of the control refinement process described by \textit{Sub-Problem~\ref{spb:control}}, which will be discussed in the next subsection.

% To find a valid state set $S= \mathcal{W} \times \mathcal{F}$, each state $s_i \in S$ representing $W_i$ and $F_i$ should ensure that robot swarm can achieve $F_i$ when $x_c \in W_i$, otherwise the state will be deleted from $S$.

% We use a DFTS $\mathcal{T} = \{S, A, \delta, \AP, \mathcal{L}\}$ introduced in Definition~\ref{def:fts} to represent the transition of the waypoints and formations of the robot swarm system, where $S = \tilde{W} \times \tilde{F}$ is the finite state space of $\mathcal{T}$, where $\tilde{W}$ and $\tilde{F}$ are finite symbolic sets that abstract the waypoint set $W$ and the formation set $F$, respectively, where $|\tilde{W}| = |W|$, $|\tilde{F}| = |F|$, $A$ is a finite set of actions of $\mathcal{T}$, $\delta: A \times \tilde{W} \times \tilde{F} \rightarrow \tilde{W} \times \tilde{F}$ prescribes the transition of the waypoints and the formations, $\AP = \AP_f \times \AP_w$ is a complex atomic proposition set, where $\AP_f$ and $\AP_w$ are finite sets of atomic propositions related to the abstract waypoint set $\tilde{W}$ and the abstract formation set $\tilde{F}$, respectively, and $\mathcal{L} = \{\mathcal{L}_w, \mathcal{L}_f\}$, where $\mathcal{L}_w: \tilde{W} \rightarrow 2^{\AP_w}$ and $\mathcal{L}_f: \tilde{F} \rightarrow 2^{\AP_f}$ are labeling mappings. For any sequence of actions $\mathbf{a} = a_0 a_1a_2 \cdots$, where $a_i \in A$, $i \in \mathbb{N}_{\geq 0}$, $\mathcal{T}$ generates two state trajectories $\mathbf{w} = w_0 w_1w_2\cdots$ and $\mathbf{f} = f_0 f_1f_2 \cdots$, where $w_i \in W$, $f_i \in F$, $i \in \mathbb{N}_{\geq 0}$. Based on the labeling mappings $\mathcal{L}_w$ and $\mathcal{L}_f$, the changes of the waypoints and formations can now be translated to a complex word $\pmb{\omega}= (\omega^w_0, \omega^f_0)(\omega^w_1, \omega^f_1)(\omega^w_2, \omega^f_2)\cdots$ on which we assign a LTL formula $\psi$ that specifies the swarm navigation task, where $\omega_i^w \in 2^{\AP_w}$, $\omega_j^f \in 2^{\AP_f}$, $i,j \in \mathbb{N}_{\geq 0}$. Then, we say that the waypoints and formations of the robot swarm satisfy the navigation task specification if $\pmb{\omega} \models \psi$.

% We do not promise a universal and general approach to realizing a DFTS for swarm control. Instead, the realization pretty much depends on specific scenarios.


\subsubsection{Step 3: Synthesis of  Control Strategy}
Given the abstract, symbolic model, we can use an off-the-shelf tool to solve the control synthesis problem. More precisely, we use the Omega solver \cite{filippidis2016symbolic}
in Tulip~\cite{wongpiromsarn2011tulip,tulip2016}. Internally, it constructs an enumerated transducer that ensures the satisfaction of the GR(1) formula for any admissible behavior of the uncontrolled variables $e_k$. For the overall synthesis procedure can be referred to~\cite{BLOEM2012911, tulip2016}. Note that if the environment variable takes a value other than the imposed assumptions, no guarantees can be provided on the system behavior. 

 
\subsection{Control Refinement with Runtime Guarantees}
\label{sec:qp}

In this subsection, we design a QP-based control refinement to solve \textit{Sub-Problem~\ref{spb:control}}. Let the symbolic state $ s=(w,f)\in S$ and the symbolic action $a\in A$ be given for which the next symbolic state should be $s_+ = \delta(s,a)$. The refined control inputs $u(t)$ should control the robot swarm such that the desired waypoints $(w,f)$ defined by the symbolic control are reached. Additionally, different constraints are added concerning the runtime safety requirements.

\subsubsection{Computing Robot Control Inputs Via Solving QP}


Without losing generality, we consider a swarm with single integrator models, i.e., $\dot{x}_i(t) \!=\! u_i(t)$  for $i\!\in\!\{1,2,\cdots,r\}$. For \textit{Sub-Problem~\ref{spb:control}}, we need to solve the following QP problem formulated in Eq.~\eqref{eq: swarm_qp}, for all $i \!\neq\! j$, $i,j \!\in\! \{1,2,\cdots,r\}$,
\begin{subequations}
\vspace{0.1cm}
\label{eq: swarm_qp}
\begin{align}
\min _{z}  \textstyle  z^{\mathrm{T}} H z  +Q^{\mathrm{T}} z & \\
\text { s.}\text{t. } \quad \left\| u_i \right\| \leq\, u_{\max}&, \label{subeq: limit}\\
\begin{split}
\frac{\partial h_{\mathcal{W}}(x_c, w)}{\partial x_c}  u_c  \!\leq&\,   \delta_1 h_{\mathcal{W}}(x_c,w)\\
&\textstyle - \alpha_1\! \max^{\gamma_1} \{0, h_{\mathcal{W}}(x_c, w)\}\\
&\textstyle -\alpha_2\! \max^{\gamma_2} \{0, h_{\mathcal{W}}(x_c, w)\},
 \label{subeq: center}
\end{split} \\
\begin{split}
\frac{\partial h_{\mathcal{F}}(x_{ij}, f_{ij})}{\partial x_{ij}}  u_{ij} \!\leq&\, \delta_1 h_{\mathcal{F}}(x_{ij}, f_{ij}) \\
&\textstyle -\alpha_1 \!\max^{\gamma_1} \{0, h_{\mathcal{F}}(x_{ij}, f_{ij})\} \\
&\textstyle -\alpha_2 \!\max^{\gamma_2} \{0, h_{\mathcal{F}}(x_{ij}, f_{ij})\},
\label{subeq: form}
\end{split} \\
%&\tilde{h}_F(\tilde{x}_{ij}) = \max \{0, h_{\mathcal{F}}(x_{ij}, f_{ij})\},, \\
 \frac{\partial h_{\mathcal{D}}(x_{ij})}{\partial x_{ij}} u_{ij} \geq & -\delta_2 h_{\mathcal{D}}(x_{ij}), \label{subeq: collision}\\
\frac{\partial h_{\mathcal{O}}(x_i)}{\partial x_i} u_i \geq & -\delta_2 h_{\mathcal{O}}(x_i),\label{subeq: obs}
%& \tilde{h}_G(\tilde{x}_c) = \max \{0, h_{\mathcal{W}}(\tilde{x}_c)\},~\tilde{x}_c = x_c - p_k,~ \eqref{eq:center}.
\end{align}
\end{subequations}
where $z = [u^T, \delta^T]^T\!\in\!\mathbb{R}^{2 \times r + 2}$ are decision variables, in which $\delta\!=\![\,\delta_1,\,\delta_2\,] \in \mathbb{R}^2$ are slack variables, $H$ is a diagonal matrix with positive constant elements, $Q\!=\![\,\mathbf{0}_{2\times r}, w_{\delta_1}, 0\,]$ where $w_{\delta_1}\!\in\!\mathbb{R}^+$ is a penalizing scalar of the slack variable $\delta_1$, $u_{\max}\!\in\!\mathbb{R}^+$ defines the control limit of the system, and $u_{ij}\!=\!u_{i}-u_{j}$ for any $i,j\in\{1,2,\cdots,r\}$, denote the control difference between two robots $i\neq j$. Sublevel sets of $h_{\mathcal{W}}(x, w) \!=\! \|x-w \|^2  \!-\! d_G^2$ and
$h_{\mathcal{F}}(x_{ij}, f_{ij})\!=\!\|x\!-\!f_{ij} \|^2\!-\!d_F^2$ represent desired centroid position $w\!\in\!\mathbb W$ and swarm formation $f\!\in\!\mathbb F$ for the robot swarm respectively, where $d_G, d_F \in \mathbb{R}^+$ are tolerance thresholds. Superlevel sets of $h_{\mathcal{D}}(x)\!=\!\|x \|^2 - d_O^2$ and $h_{\mathcal{O}}(x)\!=\!1\!-\!(x\!-\!\eta)^T \!P(x\!-\!\eta)$ denote collision-free sets with other robots and local obstacle-free sets respectively, where $d_O\!\in\!\mathbb{R}^+$ is the minimal distance among agents and $\eta\!\in\!\mathbb{R}^n$ and $P\!\in \mathbb{R}^{n\!\times\!n}$ are constant parameters determined by the environment. Constant parameters $\alpha_1$, $\alpha_2$, $\gamma_1$, $\gamma_1$ are chosen as $\alpha_1\!=\!\alpha_2\!=\!\frac{\mu \pi}{2 T_{ud}}$, $\gamma_1\!=\!1\!+\!\frac{1}{\mu}$, $\gamma_2\!=\!1\!-\!\frac{1}{\mu}$ with user-defined constants $\mu>1$ and $T_{ud}$.

% For each waypoint $w \in W$, we define a set $\mathcal{W}(w)$ $=\{x \!\in\! \mathbb{R}^n | h_{\mathcal{W}}(x, w) \!\leq\! 0 \}$ to represent a circled region around $w$, where $h_{\mathcal{W}}(x, w) \!=\! \|x-w \|^2  \!-\! d_G^2$, where $d_G \!\in\! \mathbb{R}^+$ is a tolerance threshold.

% For any $F' \in F$, we define a set $\mathcal{F}(f_{ij}) = \{x \in \mathbb{R}^n | h_{\mathcal{F}}(x, f_{ij}) \leq 0 \}$, where $f_{ij} \in F'$ and $(i,j) \in \mathcal{E}$, $h_{\mathcal{F}}(x, f_{ij})=$ $\|x - f_{ij} \|^2$ $- d_F^2$, where $d_F \in \mathbb{R}^+$ is a tolerance threshold.

% we define a safety set $h_{\mathcal{D}} = \{x | h_{\mathcal{D}}(x) \geq 0 \}$, where $\mathcal{D}(x) = \|x \|^2 - d_O^2$, where $d_O \in \mathbb{R}^+$ is a tolerance threshold.

% Also, we use the set $\mathcal{O} \!=\! \{x \in \mathbb{R}^n| h_{\mathcal{O}}(x) \!\geq\! 0 \}$ to represent the maximal ellipsoid collision-free region for the robot swarm, where $h_{\mathcal{O}}(x)\!=\!  1\!-\!(x\!-\!\eta)^T \!P(x\!-\!\eta)$, where $\eta \!\in\! \mathbb{R}^n$ and $P \in \mathbb{R}^{n \times n}$ are constant parameters determined by the environment.

To achieve the desired waypoints $w$ and $f$, fixed-time CLF ensures the convergence to a given set within user-defined time $T_{ud}$, and the concrete formulation can be found in~\cite{FxT2022}. Constraints (\ref{subeq: center}) and (\ref{subeq: form}) are formulated in the form of fixed-time CLF, which respectively drive the swarm system to reach a waypoint $w \!\in\! \mathbb{W}$ and achieve a formation $f\!=\!\{f_{ij}\}_{\frac{r(r-1)}{2}}$ that corresponds to the chosen next  abstract state $(w, f)$, with $w\!\in\!\mathcal{W}$, $f\!\in\!\mathcal{F}$. As for runtime safety, CBF is an effective tool to ensure the set invariance of a dynamic system in a given set, which is often used to ensure system safety of collision avoidance. In the form of control inputs condition for CBF~\cite{ames2017}, constraints (\ref{subeq: obs}) and (\ref{subeq: collision}) attempt to keep the robots within the safety sets $\mathcal{D}$ and $\mathcal{O}$ to avoid collisions with each other and with the obstacles in the environment. This corresponds to the \textit{collision avoidance} objective in Sec.~\ref{sec:probstate}. Note that the \textit{reactivity} objective has been achieved already in the symbolic control in Sec.~\ref{sec:scs}.

\subsubsection{Specification Satisfaction Analysis}
If the QP in Eq.~\eqref{eq: swarm_qp} has a feasible solution $u(t)$ and the slack variable $\delta_1$ remains negative, the obtained control inputs $u(t)$ guarantee that the swarm reaches the given waypoint $(w,f)$ within a maximum time $T_{ud}$~\cite{FxT2022}. This indicates that, for any waypoints and desired formations that satisfy the specification $\psi$, the swarm can track them within a finite predefined timing bound. Although the swarm may traverse other positions and formations between two waypoints, it ultimately reaches the next waypoint $(w,f)$ within a strict timing.
%Note that the system migrates over the partitions of the workspace $\mathbb{W}$ defined by the waypoints and switches its formations only during the state transition of the abstraction model $\mathcal{T}$. Although it is possible that the swarm system temporally traverses other abstract states during the transient stage, the duration of the transient stage is strictly bounded. 
This means that, with the tolerance of the transient states of the swarm with a strict timing bound, the behavior of the swarm satisfies the given task specification $\psi$. Moreover, the constraints~\eqref{subeq: collision} and~\eqref{subeq: obs} guarantee that the system never traverses to dangerous regions. Therefore, we can claim that \textit{Sub-Problem~\ref{spb:control}} is solved if the QP in Eq.~\eqref{eq: swarm_qp} is feasible.



\begin{figure*}[htbp]
     \centering
     \vspace{0.2cm}
     \begin{subfigure}[b]{0.18\textwidth}
         \centering
         \includegraphics[width=\textwidth]{p1.eps}
         \caption{``battery'': True}
         \label{fig:f11}
     \end{subfigure}
     \hfill
     \begin{subfigure}[b]{0.18\textwidth}
         \centering
         \includegraphics[width=\textwidth]{p2.eps}
         \caption{``battery'': True}
         \label{fig:f12}
     \end{subfigure}
     \hfill
     \begin{subfigure}[b]{0.18\textwidth}
         \centering
         \includegraphics[width=\textwidth]{p3.eps}
         \caption{``battery'': True}
         \label{fig:f13}
     \end{subfigure}
     \hfill
     \begin{subfigure}[b]{0.18\textwidth}
         \centering
         \includegraphics[width=\textwidth]{p4.eps}
         \caption{``battery'': False}
         \label{fig:f14}
     \end{subfigure}
     \hfill
     \begin{subfigure}[b]{0.18\textwidth}
         \centering
         \includegraphics[width=\textwidth]{p5.eps}
         \caption{``battery'': True}
         \label{fig:f15}
     \end{subfigure}
        \caption{The planar view of the environment and the robot trajectories in a simulation run, as time changes (left to right).}
        \label{fig:f1}
        \vspace{-0.1cm}
\end{figure*}


If the QP in Eq.~\eqref{eq: swarm_qp} is infeasible, it might be caused by the constraints encoding the impractical waypoints or formations. In this case, the infeasibility can be resolved by pruning the transition function $\delta$ of the abstraction model $\mathcal{T}$ as developed in Sec.~\ref{sec:scs}. Specifically, the waypoints leading to the infeasible QP are recognized as infeasible transitions in the abstraction model $\mathcal{T}$ and are eliminated from $\delta$. This is iteratively performed until all transitions in a synthesized strategy ensure the feasibility of the QP.

%can be satisfied which implies each robot can avoid the obstacle $\mathcal{O}$ and other robots. Additionally, if the slack variable $\delta_1$ remains negative~\cite{FxT2022}, the fixed-time convergence encoded by constraints~\eqref{subeq: center} and \eqref{subeq: form} can be guaranteed such that $x_c(T_{ud})\!\in\!\mathcal{W}$ and $\{x_{ij}(T_{ud})\}_{\frac{r(r-1)}{2}}\!\in\!\mathcal{F}$ within a user-defined time $T_{ud}$. Such properties would be significant to TL specifications with time bounds. Therefore, a feasible solution with negative $\delta_1$ can ensure the runtime safety and fixed-time convergence of the transitions in the abstract model.
%However, an undesired \textit{deadlock} phenomenon may occur when $u_i = 0$ hold for all agents due to $h_{\mathcal{O}}(x_i) = 0$ and $h_{\mathcal{D}}(x_{ij}) = 0$~\cite{wang2017safety}. Practically, the deadlock is mainly caused by narrow spaces and odd-shaped obstacles. In our framework, we can prescribe the task specifications to make the swarm avoid going through narrow corridors or stay away from odd-shaped obstacles.




% Based on arguments in~\cite{FxT2022}, larger values of $T_{ud}$ and $u_{\max}$ lead to a larger set of initial robot conditions allowing the swarm system to reach the given waypoint $w$ and formation $F'$ within time $T_{ud}$.


% Note that the \textit{deadlock} phenomenon may occur when $u_i = 0$, $h_{\mathcal{O}}(x_i) = 0$, $u_{ij}=0$, $h_{\mathcal{D}}(x_{ij}) =0$ hold for all $i \in \mathcal{V}$ and $(i,j)\in \mathcal{E}$~\cite{wang2017safety}. How to avoid deadlock has always been a challenging problem and remains an open question.

% In~\cite{FxT2022}, it is argued that larger values of $T_{ud}$ and $u_{\max}$ lead to a larger fixed-time domain of attraction, which means that a larger set of initial robot conditions allow the swarm system to reach the given waypoint $w$ and formation $F'$ within time $T_{ud}$. We can infer that there exist $\overline{T}_{ud} \!\in\! \mathbb{R}^+$ and $\overline{u}_{\max} \!\in\! \mathbb{R}^+$, such that for any $x_i(0) \!\in\! \X$, $i \!\in\! \mathcal{V}$, $x_c$ and $x_{ij}$ reach $\mathcal{W}(w)$ and $\mathcal{F}(f_{ij})$ within $T_{ud}$ for all $f_{ij} \!\in\! F' \in F$, $(i,j)\!\in\! \mathcal{E}$, if $T_{ud} > \overline{T}_{ud}$ and $v_{\max} > \overline{v}_{\max}$.

% The convergence property of the closed-loop system evaluates whether the swarm reaches the given waypoint $w \in W$ and formation $F' \!\in\! F$ sufficiently soon. From a theoretical perspective, it evaluates whether $x_c$, $x_{ij}$ enter the bounded set $\mathcal{W}(w)$, $\mathcal{F}(f_{ij})$ for all $f_{ij} \!\in\! F'$, $(i,j) \!\in\! \mathcal{E}$, within the given fixed time $T_{ud}$, from any initial robot positions $x_i(0) \!\in\! \X$, $i \!\in\! \mathcal{V}$. Convergence is an important guarantee for the satisfaction of the task specification $\psi$ and is prescribed by the feasibility of the constraints \eqref{subeq: center} and \eqref{subeq: form}. Involving slack variables $\delta_1 \!\in\! \mathbb{R}^+$ can effectively relax the constraints and improve their feasibility. Nevertheless, the input limitations \eqref{subeq: limit} also affect the feasibility of these constraints.



% Runtime safety describes whether the robots keep collision-free for all time. From a theoretical perspective, given $x_i(0) \in \mathcal{O}$ and $x_i(0) - x_j(0) \in \mathcal{D}$, whether $x_i(t) \in \mathcal{O}$ and $x_i(t) - x_j(t) \in \mathcal{D}$ hold for all $\mathbb{R}_{\geq 0}$. Runtime safety is ensured by the feasibility of constraints \eqref{subeq: collision} and \eqref{subeq: obs}. Similar to the constraints concerned with convergence, the feasibility of these constraints can also be improved by the slack variable $\delta_2 \in \mathbb{R}^+$. However, when the robots reach the boundaries of sets $\mathcal{D}$ and $\mathcal{O}$, i.e., there exist $i \in \mathcal{V}$, such that $h_{\mathcal{O}}(x_i) = 0$, or there exist $(i,j) \in \mathcal{E}$, such that $h_{\mathcal{D}}(x_{ij}) =0$, $\delta_2$ will lose its functionality in constraint \eqref{subeq: collision} or \eqref{subeq: obs} and render hard constraints. Nevertheless, these constraints are still feasible in this case. At least, there exists a trivial solution $u_i = 0$ or $u_{ij}=0$. Note that the \textit{deadlock} phenomenon may occur when $u_i = 0$, $h_{\mathcal{O}}(x_i) = 0$, $u_{ij}=0$, $h_{\mathcal{D}}(x_{ij}) =0$ hold for all $i \in \mathcal{V}$ and $(i,j)\in \mathcal{E}$~\cite{wang2017safety}. How to avoid deadlock has always been a challenging problem and remains an open question. It is commonly recognized that deadlock is due to the over-conservativeness of the runtime safety constraints, which provokes extreme actions of the robots. For example, the swarm is asked to squeeze into a narrow corridor or it is stuck in a dead corner of an odd-shaped obstacle. In our framework, we can prescribe the task specifications at the high-level such that the swarm can avoid going through narrow corridors or stay away from odd-shaped obstacles. In this sense, we can greatly reduce the likelihood of the deadlock phenomenon.




% \red{
% \begin{theorem}
% Given ...
% If a feasible high-level strategy exists for the abstract model $\mathcal{T}$, the robot swarm can be driven by the QP controller to satisfy $\psi$.

% \end{theorem}
% }








% \addtolength{\textheight}{-3cm}
%%%%%%%%%%%%%%%%%%%%%%%%%%%%%%%%%%%%%%%%%%%%%%%%%%%%%%%%%%%%%%%%%%%%%%%%%%%%%%%%
\section{Case Study}\label{sec:cas}

In this section, we use a swarm navigation control case in simulation to validate the efficacy of our proposed framework and solution. Consider a homogeneous swarm system that contains $r=3$ quadrotor robots moving in a two-dimensional planar environment $\X \subset \mathbb{R}^2$, where $\X$ is a $5\,$m $\times5\,$m square area. The dynamic model of the $i$-th robot of the swarm system is given as the following single integrator $\dot{x}_i(t) = u_i(t)$, $i=1,2,3$, where $x_i(t), u_i(t) \in \mathbb{R}^2$ are the position and control input of robot $i$ at time $t \in \mathbb{R}_{\geq 0}$. In this setting, the workspace containing the possible positions of the centroid is also $\mathbb{X}$, i.e., $\mathbb{W}:=\mathbb{X}$.

The terrain of the environment is illustrated in Fig.~\ref{fig:f1}. The environment is split into a $5\times 5$ grid which generates 25 even square blocks, each with a size of $1\,$m $\times 1\,$m. The yellow block is the starting point of the robot swarm. The blue block is the navigation goal that the swarm needs to reach ultimately. The red blocks are the obstacles that the robots should avoid. All waypoints in $\mathcal{W}$ are assigned as the center of the accessible square partitions. Also, we determine an abstract formation set including three different formations $\mathcal{F}=\{f_1, f_2, f_3\}$ by partitioning $\mathbb F$, where $f_1$, $f_2$, $f_3$ represent a horizontal formation (as shown in Fig.~\ref{fig:f12} and Fig.~\ref{fig:f14}), a vertical formation (as shown in Fig.~\ref{fig:f11} and Fig.~\ref{fig:f15}), and a triangle shape formations (Fig.~\ref{fig:f13}), respectively. The definitions of $\mathcal{W}$ and $\mathcal{F}$ can be found in our online document on~\cite{ourCode}.

The DFTS $\mathcal{T}\!=\!\{S, s_0, A, \delta, \AP, \mathcal{L}\}$ as the abstract model of the quadrotor swarm is realized as follows. The state space $S \!=\! \mathcal{W} \!\times\! \mathcal{F}$, where $\mathcal{W}$, $\mathcal{F}$ are realized as finite sets with $25$ and $3$ elements, respectively. The transition relation $\delta$ is represented as a matrix and can also be found in our Github repository~\cite{ourCode}. The main principles we use to construct the transition relation are as follows.
\begin{itemize}
\item When the swarm passes by a narrow corridor, it should switch to the thinnest formation to fit its direction.
\item The swarm should not enter an obstacle (red) region.
\end{itemize}
The atomic proposition set is $\AP = \AP_w \cup \AP_f \cup \AP_e$, where $\AP_w \!=\! \{$freespace, home, goal, obstacle$\}$, $\AP_f\!=\!\{$horizon, vertical, triangle$\}$, and $\AP_e=\{$battery$\}$. The label mapping is $\mathcal{L} \!=\! \{\mathcal{L}_w, \mathcal{L}_f,\mathcal{L}_e\}$. Mapping $\mathcal{L}_w$ labels the yellow block in Fig.~\ref{fig:f1} as ``home'', the blue block as ``goal'', the red blocks as ``obstacle'', and all other blocks as ``freespace''. Mapping $\mathcal{L}_f$ is defined such that $\mathcal{L}_f(f_1) =$ ``horizon'', $\mathcal{L}_f(f_2)=$ ``vertical'', and $\mathcal{L}_f(f_3) =$ ``triangle''. Then, mapping $\mathcal{L}_e$ gives ``battery'' if all the batteries of the robots are charged. The abstract model is visualized in Fig.~\ref{fig:strategy}, where the $x$-$y$ planes along the formation axis show the planar view of the environment for different formations. Thus, Fig.~\ref{fig:strategy} clearly shows the transition between the $25 \times 3$ states of the DFTS.


\begin{figure}[htpb]
    \centering
    \includegraphics[width=0.38\textwidth]{strategy.eps}
    \caption{The visualization of the states of the abstract model. The three planes distributed along the $z$-axis are the environment with waypoints corresponding to three formations. Each small square block is an abstract state. The blue line denotes the transition of the abstract states in a simulation run. The passing waypoints are marked as red dots.}
    \label{fig:strategy}
    \vspace{-0.1cm}
\end{figure}


The task is interpreted in the following natural language.
\begin{enumerate}
\item The swarm should infinitely visit ``goal" in ``triangle" formation, as long as ``battery'' is true.
\item All robots should avoid entering regions with obstacles.
\item The swarm should go back ``home'' to recharge once ``battery'' becomes false.
\end{enumerate}
It can be specified as an LTL formula $\psi \!=\! \psi_e \!\rightarrow\! \psi_s$, where $\psi_e := \square (\neg \mathrm{battery} \wedge  \mathrm{home} \rightarrow \bigcirc \mathrm{battery)} \wedge \square (\neg \mathrm{battery} \wedge  \neg \mathrm{home} \rightarrow \bigcirc \neg \mathrm{battery})$ and $\psi_s := \square \neg \mathrm{obstacle}
\wedge \square \lozenge (\mathrm{goal} \wedge \mathrm{triangle})
\wedge \square \lozenge \mathrm{battery}$.
The runtime safety requirements are formulated as bounded sets defined in Sec.~\ref{sec:qp} and encoded in the QP problem \eqref{eq: swarm_qp} for which the parameters can be found in our Github repository~\cite{ourCode}. We give two important parameters $T_{ud}\!=\!4\,$s and $u_{\max}\!=\!5\,$m/s.
The synthesis of the symbolic control strategy is solved using an off-the-shelf LTL toolbox, TuLiP~\cite{tulip2016}.
The QP problem is solved using the CasADi library~\cite{Andersson2019} with the ipopt solver on a commercial laptop with CPU i7-10750H.



%We apply the synthesized control strategy and the solution of the QP to the DFTS and the robots, respectively
The trajectories of the robots in a simulation run are shown in Fig.~\ref{fig:f1}. The swarm starts at ``home'' with the ``vertical'' formation. After leaving ``home'', the swarm goes along the horizontal corridor in the ``horizon'' formation, as shown in Fig.~\ref{fig:f11}, since the narrow space does not allow other formations. In Fig.~\ref{fig:f12}, the robot turns right and switches to the ``horizon'' formation to pass the short horizontal corridor. After reaching the ``goal'' in the open space, it switches to the ``triangle'' formation as specified by $\psi$, as shown in Fig.~\ref{fig:f13}. When any robots have low power (``battery'' signal is false) as shown in Fig.~\ref{fig:f14}, the swarm goes back to ``home'' to recharge. Once it gets charged at ``home'' and the ``battery'' signal is true again, the robot resumes its previous task to navigate itself to the ``goal'' again, as shown in Fig.~\ref{fig:f15}. From Fig.~\ref{fig:f1}, we can see that the controlled swarm ensures an obstacle-free trajectory when approaching the desired task. Also, proper formations are automatically switched to traverse narrow areas. The resulting behavior of the robot swarm completely satisfies the LTL specifications. This is also reflected by Fig.~\ref{fig:strategy} which visualizes the trajectory of the robot swarm in the abstract space. %Following the trajectory, we can infer similar conclusions to the above arguments that the task specification is satisfied. Therefore, we do not elaborate detailed interpretations for all abstract states.
A video demonstration of this use case is accessible at {\small\url{https://www.youtube.com/watch?v=r1aecBOeDq0}}.
% {\small }.

Fig.~\ref{fig:formation} indicates the satisfaction of runtime safety requirements, where the robot trajectories given a new waypoint and a desired formation are shown. In this case, a robot swarm from the initial position $O$ is required to achieve a ``triangle'' formation and its center simultaneously reaches the green rounded region within $T_{ud} \!=\! 4\,$s. During this period, all robots should avoid collision with the red rounded obstacle. In Fig.~\ref{fig:formation}, the trajectories of the robots are drawn as solid lines and the formation of the swarm is in dotted lines. It is shown that the robots successfully avoid the obstacle and finally reach the waypoint at a tolerable range. The ultimate formation is ``triangle'' and the reaching time is within $T_{ud}$. This study shows that the robot controller solved from the QP problem strictly ensures not only the runtime safety requirements but also the fixed-time convergence condition.

\begin{figure}[htbp]
\noindent
\hspace*{\fill}
\begin{tikzpicture}[scale=1,font=\scriptsize]
\node[anchor=south west] (eva) at (0cm, 0cm){\includegraphics[width=0.3\textwidth]{formation.eps}};
\node[circle,inner sep=0pt,minimum size=1.2mm,fill] (O) at (0.39cm,0.54cm) {};
\node[circle,inner sep=0pt,minimum size=1.2mm,fill,anchor=south east] (w) at (4.4cm,4.4cm) {};
\node[anchor=north east] () at (O) {$O$};
\node[] () at (2cm, 2.8cm) {Obstacle};
\node[anchor=south east] () at (w) {Waypoint};


\end{tikzpicture}
\hspace{\fill}
\caption{The satisfaction of the runtime safety requirements.}
\label{fig:formation}
\end{figure}

%%%%%%%%%%%%%%%%%%%%%%%%%%%%%%%%%%%%%%%%%%%%%%%%%%%%%%%%%%%%%%%%%%%%%%%%%%%%%%%%
\section{CONCLUSION}\label{sec:con}
In this paper, we develop a formal-method-based framework for multi-robot swarm systems to design a reactive controller for the autonomous navigation task with automated formations. Under the synthesized symbolic controller for the abstract model, the QP-based control refinement approach can ensure that the behavior of the swarm system satisfies the LTL specification with runtime guarantees. In the future, the framework will be expected to extend to more robotic applications with more complicated specifications.

% In the proposed multi-robot swarm solution,  the definition of the waypoints and formations can be generalized to incorporate other types of goals, and therefore the proposed framework can be extended to more complicated robotic tasks. We remark that the robot trajectories generated by the symbolic controller show some `jerky' behaviors. Future work will also investigate the generation of smooth robot trajectories.

% \red{Thanks to the fixed-time convergence enabled by QP controller, the proposed method can be extended to timed TL, like metric temporal logic.}

\balance

%%%%%%%%%%%%%%%%%%%%%%%%%%%%%%%%%%%%%%%%%%%%%%%%%%%%%%%%%%%%%%%%%%%%%%%%%%%%%%%%
\bibliographystyle{ieeetr}
\bibliography{ref}

%\bibliographystyle{IEEEtran}
%\bibliography{ref}
\end{document}
