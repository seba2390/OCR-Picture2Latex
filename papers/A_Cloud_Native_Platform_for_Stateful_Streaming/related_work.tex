\section{Related work}

Cloud native applications are defined by their leverage of cloud orchestrators
such as Kubernetes~\cite{kube} or Apache Mesos~\cite{mesos} and their
microservice architecture. Derived from the ``Twelve-Factor App''
guidebook~\cite{12factor}, the microservice architecture is an evolution of the
traditional monolithic and service-oriented architectures, common to enterprise
applications. It favors small, stateless, siloed components with clean
interfaces to maximize horizontal scalability and concurrent development
while minimizing downtimes. The most commonly
published works about cloud-native transformation of legacy workloads cover
stateless applications~\cite{8629101,8457916,8457847,10.1145/3104028,10.1145/3241403.3241440,8004166,KRATZKE20171,7584353,7436659}.

Facebook developed Turbine, which is a cloud native platform for managing their
streaming applications on their Tupperware container
platform~\cite{turbine-2020}.  The main features are a scalable task scheduler,
auto-scaler and consistent and reliable update mechanism. Turbine running on
Tupperware is similar to cloud native Streams running on Kubernetes, with the
exception that Kubernetes handles much more of the platform responsibilities. In
the area of relational databases, {Amazon}~\cite{10.1145/3035918.3056101} and
{Alibaba}~\cite{10.14778/3352063.3352141} undertook the redesign of existing
databases to better fit their respective cloud infrastructure.

For stateful applications, the \emph{lift-and-shift}~\cite{liftshift} approach
is more common than a complete redesign of the supporting platform, often
accompanied with a shim operator that exposes some of the application's concepts
to the cloud platform through the application's native client
interface~\cite{FlinkOperator,BanzaiKafka,StrimziKafka}.

