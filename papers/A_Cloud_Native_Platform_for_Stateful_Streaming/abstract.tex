\begin{abstract}

We present the architecture of a cloud native version of IBM Streams, with
Kubernetes as our target platform. Streams is a general purpose streaming system
with its own platform for managing applications and the compute clusters that
execute those applications. Cloud native Streams replaces that platform with
Kubernetes. By using Kubernetes as its platform, Streams is able to offload job
management, life cycle tracking, address translation, fault tolerance and
scheduling. This offloading is possible because we define custom resources that
natively integrate into Kubernetes, allowing Streams to use Kubernetes' eventing
system as its own. We use four design patterns to implement our system:
controllers, conductors, coordinators and causal chains.
% SPACE_CUT_START
%Controllers control a single resource; conductors monitor the events of multiple
%resources to transition a state machine; coordinators enable other actors to
%request changes to a controlled resource; and causal chains are formed by their
%deterministic interactions.
% SPACE_CUT_END
Composing controllers, conductors and coordinators allows us to build
deterministic state machines out of an asynchronous distributed system. The
resulting implementation eliminates 75\% of the original platform code. Our
experimental results show that the performance of Kubernetes is an adequate
replacement in most cases, but it has problems with oversubscription, networking
latency, garbage collection and pod recovery.

\end{abstract}
