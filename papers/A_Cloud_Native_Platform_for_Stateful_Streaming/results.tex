\section{Results}
\label{sec:results}

\subsection{Experimental results}

Raw performance was not a motivation (\S~\ref{sec:motivation}) or design goal
(\S~\ref{sec:arch}). However, if cloud native Streams' platform performance was
significantly worse than legacy Streams', then it would not be an acceptable
replacement. The primary goal of our experiments is to demonstrate that cloud
native Streams has acceptable comparable performance, and the secondary goal is
to identify aspects of Kubernetes which can be improved.

We ran our experiments on a 14 node cluster using Streams v4.3.1.0 as legacy,
Kubernetes v1.14 and Docker v18.09.6. Each node has two 4-core Intel Xeon X5570
processors at 2.93 GHz with hyperthreading enabled and 48GB of memory.  One
node is dedicated to management, leaving 13 nodes and 104 physical cores (208
logical cores) for applications.

Unless otherwise stated, our test application has a source operator which
continuously generates tuples and feeds into an $n$-way parallel region. Each
channel in the parallel region has a pipeline of $n$ operators, and all
channels eventually converge into a sink operator. We fuse each operator into
its own PE. We vary $n$ in our experiments, which means that the number of
operators and PEs grows with $n^2$. As described earlier, each PE is a separate
process and runs in its own pod. Different experiments need a different number
of pre- and post-processing operators before and after the parallel region.

\begin{figure}
\centering
\begin{subfigure}{1.0\linewidth}
  \centering
  \includegraphics[width=\linewidth]{figures/submitted-eps-converted-to}
  \caption{Time taken for jobs to reach the \code{Submitted} state.}
  \label{fig:submit-cancel-submitted}
\end{subfigure}
\begin{subfigure}{1.0\linewidth}
  \centering
  \includegraphics[width=\linewidth]{figures/healthy-eps-converted-to}
  \caption{Time taken for jobs to reach full health.}
  \label{fig:submit-cancel-healthy}
\end{subfigure}
\begin{subfigure}{1.0\linewidth}
  \centering
  \includegraphics[width=\linewidth]{figures/cancel-eps-converted-to}
  \caption{Time taken for jobs to fully terminate.}
  \label{fig:submit-cancel-terminate}
\end{subfigure}
\caption{Job life cycle times.}
\label{fig:submit-cancel}
\end{figure}

\paragraph{Job life cycle:}
\label{sec:results:job-submission-termination}

The three job life cycle phases exposed to users are submission, full health
and full termination. The submission time is how long it takes for the
platform to create all of the PEs and job resources. Such jobs are still
initializing and are not yet processing data. Only when all PEs are running and
have established all connections is the application processing data and
considered fully healthy. Finally, after the user cancels the job, the platform
considers the job fully terminated after all PEs and associated resources are
gone.

Figure~\ref{fig:submit-cancel} shows how long it takes to reach each phase of
the job life cycle for both cloud native and legacy Streams, with each data
point representing the average of 10 runs.
Figure~\ref{fig:submit-cancel-submitted} shows that cloud native Streams is
consistently faster to reach the \code{Submitted} state. Reaching the
\code{Submitted} state only requires resource creation; for legacy Streams, that
means registering all resources in ZooKeeper, and for cloud native Streams that
means all resources are stored in etcd. However, legacy Streams also
computes PE schedules: it rejects jobs for which it cannot find a valid
schedule. In cloud native Streams, Kubernetes schedules PE’s pods asynchronously.

The time it takes for cloud native Streams to reach full health,
Figure~\ref{fig:submit-cancel-healthy}, is dependent on whether the
cluster is oversubscribed. Each PE is a process in a separate pod, and the
experiments scale to 1027 PEs. But the cluster will be fully subscribed by at least
208 PEs; there are more processes than cores. Before the cluster is
fully subscribed, cloud native Streams performs competitively with legacy
Streams. Both versions suffer as the cluster becomes more oversubscribed, but
cloud native Streams eventually takes twice as long.

The job termination experiments, Figure~\ref{fig:submit-cancel-terminate}, have
results for two different approaches for cloud native Streams: manual resource
deletion and relying on Kubernetes' garbage collector. In the case of manual
deletion, the job controller actively cleans up by telling Kubernetes to delete
resources in bulk by their label. Bulk deletion minimizes the number of API calls
and therefore reduces the strain on Kubernetes' API server. We originally relied
on Kubernetes' resource garbage collector to automatically reclaim resources
owned by deleted jobs. Kubernetes' garbage collector, however, does not scale
well as the number of resources grows.

\begin{figure}[t]
    \centering
    \includegraphics[width=1.0\linewidth]{figures/throughput-eps-converted-to}
    \caption{PE-PE communication throughput.}
    \label{fig:throughput}
\end{figure}

\paragraph{PE-to-PE communication throughput:} The experiments in
Figure~\ref{fig:throughput} use an application designed to test PE-to-PE
throughput. It consists of two PEs, pinned to two fixed nodes in our
testbed, while varying the size of the tuple payload from 1 byte to 4 MB.
Transmissions run for 5 minutes, with throughput measurements every 10 seconds.

Cloud native Streams achieves significantly lower throughput than legacy when
the tuple size is smaller than 4 KB. This performance degradation is due to the
deeper networking stack used within Kubernetes. Because of its networking
architecture, a single packet sent from one container is required to cross
various virtual interfaces and packet filters before reaching another container.
Comparatively, a packet sent between two PEs within the legacy Streams platform
is directly sent to the default interface for the target route. This increased
complexity has the most pronounced effect with payloads less than 8 KB.  With
larger payloads, the increased networking cost is mostly amortized.

\begin{figure}[t]
    \centering
    \includegraphics[width=1.0\linewidth]{figures/pr-width-change-eps-converted-to}
    \caption{Parallel region width change time.}
    \label{fig:pr-width-change}
\end{figure}

\begin{figure}[t]
    \centering
    \includegraphics[width=\linewidth]{figures/pe-recovery.pdf}
    \caption{Application PE failure recovery time.}
    \label{fig:pe-recovery}
\end{figure}

\begin{figure}[t]
    \centering
    \includegraphics[width=1.0\linewidth]{figures/cr-pe-recovery.pdf}
    \caption{CR App. PE Failure Recovery.}
    \label{fig:cr-pe-recovery}
\end{figure}

\paragraph{Parallel region width change:}

Figure~\ref{fig:pr-width-change} has two sets of experiments: increasing and
decreasing parallel region width. The $x$-axis is the starting width of the
parallel region, and we measure how long it takes an application currently at
full health to either double or halve that parallel region. In general, cloud
native Streams benefits from the concurrency enabled by the design from
\S~\ref{sec:parupdates}. Starting new PEs and terminating the old PEs
happens concurrently, handled by Kubernetes' pod scheduling. In legacy Streams,
these phases need to happen sequentially. Cloud native Streams also has less
communication with its metadata backing store (etcd) than legacy Streams
(ZooKeeper). Starting at a width of 12, the cluster is oversubscribed (the
number of PEs grows with the square of the width). At that point, when doubling
the width, we see the same behavior regarding starting new PEs in an oversubscribed
cluster as in Figure~\ref{fig:submit-cancel-healthy}.

\paragraph{PE failure recovery:}
\label{sec:results:pe-recovery}

We test PE recovery time in Figure~\ref{fig:pe-recovery}. The $x$-axis is the
number of PEs in the application, and each dot represents how long it took the
application to return to full health after we killed a particular PE. This
process times how long it takes for the platform to detect that the PE is gone,
start a replacement, and wait for the replacement PE to re-establish all
connections. The clustered times are due to different PEs being in similar
places in the application topology. In cloud native Streams, the delay is in
re-establishing the connections; the PEs are restarted quickly as that is almost
entirely handled through Kubernetes' pod management.

\paragraph{Consistent region PE failure recovery:}

The experiments in Figure~\ref{fig:cr-pe-recovery} are the same as in the PE
failure recovery with the addition that the operators are also in a consistent
region. This addition means that their recovery is managed by the consistent
region protocol (\S~\ref{sec:consistentregion}) which requires more
communication and coordination than just restarting the PEs.  The outlier
latencies tend to be the PEs which have more connections.

\paragraph{Discussion:}

One benefit of a bespoke platform is specialization. Our experimental results
show that there is currently a cost for some actions when using Kubernetes as
a generic platform. Improving these parts of Kubernetes will improve
its ability to handle workloads such as Streams.

\emph{Oversubscription:} Cloud native Streams behaves poorly compared to legacy
Streams when the cluster is oversubscribed.  We have identified two potential
culprits.  First, the DNS propagation in Kubernetes seems to be slower than the
name resolution mechanism in legacy Streams. This latency is likely caused by
the extra complexity of pod networking. Second, many more subsystems are
involved in cloud native Streams than legacy Streams when creating new PEs:
where \code{fork()} is enough for the latter, the former calls upon the Docker
daemon and various Linux kernel facilities such as \code{cgroups} to start new
pods.

\emph{Networking latency:} The increased latency is the biggest pain point as
IBM Streams was designed and engineered as a low latency, high throughput
streaming solution. This is especially true as the most common tuple size used
by Streams customers is around 500 bytes, within the size range where the
latency degradation reaches 50\%. A solution is to use two different planes for
control and data: Kubernetes networking for the Streams control plane, while a
separate network for the Streams data plane. Such a separation can be achieved
through user-space networking, either through a bespoke user-space TCP/IP
network stack integrated into Streams' runtime or through a Kubernetes plugin
supporting user-space networking, such as Microsoft FreeFlow\cite{freeflow}.

\emph{Garbage collector:} When handled completely by the Kubernetes garbage
collector, our resource deletion time experienced significant latency even with
a modest number of resources on an undersubscribed cluster. The garbage
collector could likely be tuned to reduce the deletion time, but such tuning
introduces the danger of overfitting it to one specific workload at the
detriment of others. Garbage collector plugins similar to scheduler plugins
could solve this problem.

\emph{PE recovery:} We initially suspected the container runtime added latency,
but further investigation conducted by stressing container creation and
deletion did not show any behavior that would explain the increasing recovery
latency past 100 PEs. Another intuition concerns the networking address
allocation for PEs: when recovering a failed PE, the legacy Streams platform
will respawn the process on the same host. By doing so, the name resolution
stays stable and other PEs communicating with the respawned process will be
able to reconnect to it immediately. However, on the Kubernetes platform, PEs
may not end up with the same container IP address, even when allocated on the
same host. Therefore, all PEs communicating with the respawned process first
need to get a fresh name translation record, which is dependent on how fast the
Kubernetes DNS subsystem propagate changes.  Validation requires more
investigation. However, some workloads may benefit from stable IP addresses
for pods. Such stable addressing could be implemented by either updating an
existing network plugin or implementing a network plugin specific to the
workload.

\subsection{Lines of Code}

Rearchitecting a legacy product to be cloud native should offload significant
responsibility to the cloud platform. This process should significantly
reduce the lines of code in the implementation. Table~\ref{tab:sloc} shows
that reduction.

\begin{table}[h]
    \centering
    \footnotesize
    \begin{tabular}{|l|r|r|}
        \hline
                             & \textbf{legacy}   & \textbf{cloud native}  \\
        \hline
        \textbf{SPL}         &           429,406 &                415,809 \\
        \textbf{platform}    &           569,933 &                148,375 \\
        \textbf{install}     &            14,785 &                      0 \\
        \hline
        \hline
        \textbf{total}       &         1,014,124 &                564,184 \\
        \hline
    \end{tabular}
    \caption{Physical lines of source code across Streams versions.}
    \label{tab:sloc}
\end{table}

We use scc~\cite{scc} to count code. It counts physical source lines of code,
which is defined as lines of code which are not comments or blank. The
languages included in our count are C++, Java, Perl, XML Schema and YAML.  The
SPL compiler is primarily C++, with some of the user-exposed code generation
features in Perl.  The SPL runtime is split between C++ and Java. The legacy
platform is about 80\% Java, 20\% C++. The install is primarily Java.  We do
not include code related to the build process or system tests.

Cloud native Streams is about half the size of the legacy version, and the
platform is about a quarter the size of the old platform. The implementation of
the architecture presented in this paper is about 26,000 lines of code, which
means that we reused about 122,000 lines of platform code. Most of that code is
the job submission pipeline (\S~\ref{sec:jobsubmission}). The cloud native
version does not have an installer: users apply the YAMLs for the CRDs and make
the Docker images available which contain the instance operator and the
application runtime.

