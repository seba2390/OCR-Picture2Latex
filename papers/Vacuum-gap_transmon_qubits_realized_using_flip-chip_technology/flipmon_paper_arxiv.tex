\documentclass[letterpaper,twocolumn,prl,aps,superscriptaddress,floatfix]{revtex4}
\usepackage{mathptmx}
\renewcommand{\ttdefault}{mathptmx}
% \usepackage[latin9]{inputenc}
\setcounter{secnumdepth}{3}
\usepackage{amsmath}
\usepackage{amssymb}
\usepackage{graphicx}
\usepackage{esint}
\usepackage{float}
\usepackage{url}
\usepackage{mhchem}
% \usepackage{subcaption}
% \usepackage{bm}% bold math
% \usepackage{physics}



\makeatletter

%%%%%%%%%%%%%%%%%%%%%%%%%%%%%% LyX specific LaTeX commands.
\pdfpageheight\paperheight
\pdfpagewidth\paperwidth

%% Because html converters don't know tabularnewline
\providecommand{\tabularnewline}{\\}

%%%%%%%%%%%%%%%%%%%%%%%%%%%%%% Textclass specific LaTeX commands.
\@ifundefined{textcolor}{}
{%
 \definecolor{BLACK}{gray}{0}
 \definecolor{WHITE}{gray}{1}
 \definecolor{RED}{rgb}{1,0,0}
 \definecolor{GREEN}{rgb}{0,1,0}
 \definecolor{BLUE}{rgb}{0,0,1}
 \definecolor{CYAN}{cmyk}{1,0,0,0}
 \definecolor{MAGENTA}{cmyk}{0,1,0,0}
 \definecolor{YELLOW}{cmyk}{0,0,1,0}
}

%%%%%%%%%%%%%%%%%%%%%%%%%%%%%% User specified LaTeX commands.
%\documentclass [prl,aps,letterpaper,preprint,amsmath,amssymb,floatfix,superscriptaddress] {revtex4}
% makes Times Roman font for text AND math
% define macros
%\renewcommand\thesection{\Alph{section}}
%\newcommand{\onlinecite}[1]{\hspace{-1 ex} \nocite{#1}\citenum{#1}}

%%%%%%%%%%%%%%%%%%%%%%%%%%%%%%%%%%%%%%%%%%%%%%%%%%%%%%%%%%%%%%%%%%%%%%%%%%%%%%%%%%%%%%%%%%%%%%%%%%%%%%%%%%%%%%%%%%%%%%%%%%%%%%%%%%%%%%%%%%%%%%%%%%%%%%%%%%%%%%%%%%%%%%%%%%%%%%%%%%%%%%%%%%%%%%%%%%%%%%%%%%%%%%%
\usepackage{xcolor}\usepackage{soul}

\setcounter{MaxMatrixCols}{10}
%TCIDATA{OutputFilter=Latex.dll}
%TCIDATA{Version=5.50.0.2953}
%TCIDATA{<META NAME="SaveForMode" CONTENT="1">}
%TCIDATA{BibliographyScheme=BibTeX}
%TCIDATA{LastRevised=Wednesday, July 06, 2016 07:06:51}
%TCIDATA{<META NAME="GraphicsSave" CONTENT="32">}
%TCIDATA{Language=American English}

\newcommand{\dg}{$^\circ$ }
\newcommand{\dgc}{$^\circ\mathrm{C}$}
% \newcommand{\bra}[1]{\ensuremath{\left\langle#1\right|}}
% \newcommand{\ket}[1]{\ensuremath{\left|#1\right\rangle}}
\definecolor{blue}{rgb}{0,0,1}
\definecolor{red}{rgb}{1,0,0}
\definecolor{green}{rgb}{0,1,0}
\newcommand{\red}[1]{\textcolor{red}{ #1}}
\newcommand{\Blue}[1]{\textcolor{blue}{ #1}}
\newcommand{\Green}[1]{\textcolor{green}{ #1}}
%% Macros for Scientific Word 2.5 documents saved with the LaTeX filter.
%Copyright (C) 1994-95 TCI Software Research, Inc.
\typeout{TCILATEX Macros for Scientific Word 2.5 <22 Dec 95>.}
\typeout{NOTICE:  This macro file is NOT proprietary and may be 
freely copied and distributed.}
%
\makeatletter
%
%%%%%%%%%%%%%%%%%%%%%%
% macros for time
\newcount\@hour\newcount\@minute\chardef\@x10\chardef\@xv60
\def\tcitime{
\def\@time{%
  \@minute\time\@hour\@minute\divide\@hour\@xv
  \ifnum\@hour<\@x 0\fi\the\@hour:%
  \multiply\@hour\@xv\advance\@minute-\@hour
  \ifnum\@minute<\@x 0\fi\the\@minute
  }}%

%%%%%%%%%%%%%%%%%%%%%%
% macro for hyperref
\@ifundefined{hyperref}{\def\hyperref#1#2#3#4{#2\ref{#4}#3}}{}

% macro for external program call
\@ifundefined{qExtProgCall}{\def\qExtProgCall#1#2#3#4#5#6{\relax}}{}
%%%%%%%%%%%%%%%%%%%%%%
%
% macros for graphics
%
\def\FILENAME#1{#1}%
%
\def\QCTOpt[#1]#2{%
  \def\QCTOptB{#1}
  \def\QCTOptA{#2}
}
\def\QCTNOpt#1{%
  \def\QCTOptA{#1}
  \let\QCTOptB\empty
}
\def\Qct{%
  \@ifnextchar[{%
    \QCTOpt}{\QCTNOpt}
}
\def\QCBOpt[#1]#2{%
  \def\QCBOptB{#1}
  \def\QCBOptA{#2}
}
\def\QCBNOpt#1{%
  \def\QCBOptA{#1}
  \let\QCBOptB\empty
}
\def\Qcb{%
  \@ifnextchar[{%
    \QCBOpt}{\QCBNOpt}
}
\def\PrepCapArgs{%
  \ifx\QCBOptA\empty
    \ifx\QCTOptA\empty
      {}%
    \else
      \ifx\QCTOptB\empty
        {\QCTOptA}%
      \else
        [\QCTOptB]{\QCTOptA}%
      \fi
    \fi
  \else
    \ifx\QCBOptA\empty
      {}%
    \else
      \ifx\QCBOptB\empty
        {\QCBOptA}%
      \else
        [\QCBOptB]{\QCBOptA}%
      \fi
    \fi
  \fi
}
\newcount\GRAPHICSTYPE
%\GRAPHICSTYPE 0 is for TurboTeX
%\GRAPHICSTYPE 1 is for DVIWindo (PostScript)
%%%(removed)%\GRAPHICSTYPE 2 is for psfig (PostScript)
\GRAPHICSTYPE=\z@
\def\GRAPHICSPS#1{%
 \ifcase\GRAPHICSTYPE%\GRAPHICSTYPE=0
   \special{ps: #1}%
 \or%\GRAPHICSTYPE=1
   \special{language "PS", include "#1"}%
%%%\or%\GRAPHICSTYPE=2
%%%  #1%
 \fi
}%
%
\def\GRAPHICSHP#1{\special{include #1}}%
%
% \graffile{ body }                                  %#1
%          { contentswidth (scalar)  }               %#2
%          { contentsheight (scalar) }               %#3
%          { vertical shift when in-line (scalar) }  %#4
\def\graffile#1#2#3#4{%
%%% \ifnum\GRAPHICSTYPE=\tw@
%%%  %Following if using psfig
%%%  \@ifundefined{psfig}{\input psfig.tex}{}%
%%%  \psfig{file=#1, height=#3, width=#2}%
%%% \else
  %Following for all others
  % JCS - added BOXTHEFRAME, see below
    \leavevmode
    \raise -#4 \BOXTHEFRAME{%
        \hbox to #2{\raise #3\hbox to #2{\null #1\hfil}}}%
}%
%
% A box for drafts
\def\draftbox#1#2#3#4{%
 \leavevmode\raise -#4 \hbox{%
  \frame{\rlap{\protect\tiny #1}\hbox to #2%
   {\vrule height#3 width\z@ depth\z@\hfil}%
  }%
 }%
}%
%
\newcount\draft
\draft=\z@
\let\nographics=\draft
\newif\ifwasdraft
\wasdraftfalse

%  \GRAPHIC{ body }                                  %#1
%          { draft name }                            %#2
%          { contentswidth (scalar)  }               %#3
%          { contentsheight (scalar) }               %#4
%          { vertical shift when in-line (scalar) }  %#5
\def\GRAPHIC#1#2#3#4#5{%
 \ifnum\draft=\@ne\draftbox{#2}{#3}{#4}{#5}%
  \else\graffile{#1}{#3}{#4}{#5}%
  \fi
 }%
%
\def\addtoLaTeXparams#1{%
    \edef\LaTeXparams{\LaTeXparams #1}}%
%
% JCS -  added a switch BoxFrame that can 
% be set by including X in the frame params.
% If set a box is drawn around the frame.

\newif\ifBoxFrame \BoxFramefalse
\newif\ifOverFrame \OverFramefalse
\newif\ifUnderFrame \UnderFramefalse

\def\BOXTHEFRAME#1{%
   \hbox{%
      \ifBoxFrame
         \frame{#1}%
      \else
         {#1}%
      \fi
   }%
}


\def\doFRAMEparams#1{\BoxFramefalse\OverFramefalse\UnderFramefalse\readFRAMEparams#1\end}%
\def\readFRAMEparams#1{%
 \ifx#1\end%
  \let\next=\relax
  \else
  \ifx#1i\dispkind=\z@\fi
  \ifx#1d\dispkind=\@ne\fi
  \ifx#1f\dispkind=\tw@\fi
  \ifx#1t\addtoLaTeXparams{t}\fi
  \ifx#1b\addtoLaTeXparams{b}\fi
  \ifx#1p\addtoLaTeXparams{p}\fi
  \ifx#1h\addtoLaTeXparams{h}\fi
  \ifx#1X\BoxFrametrue\fi
  \ifx#1O\OverFrametrue\fi
  \ifx#1U\UnderFrametrue\fi
  \ifx#1w
    \ifnum\draft=1\wasdrafttrue\else\wasdraftfalse\fi
    \draft=\@ne
  \fi
  \let\next=\readFRAMEparams
  \fi
 \next
 }%
%
%Macro for In-line graphics object
%   \IFRAME{ contentswidth (scalar)  }               %#1
%          { contentsheight (scalar) }               %#2
%          { vertical shift when in-line (scalar) }  %#3
%          { draft name }                            %#4
%          { body }                                  %#5
%          { caption}                                %#6


\def\IFRAME#1#2#3#4#5#6{%
      \bgroup
      \let\QCTOptA\empty
      \let\QCTOptB\empty
      \let\QCBOptA\empty
      \let\QCBOptB\empty
      #6%
      \parindent=0pt%
      \leftskip=0pt
      \rightskip=0pt
      \setbox0 = \hbox{\QCBOptA}%
      \@tempdima = #1\relax
      \ifOverFrame
          % Do this later
          \typeout{This is not implemented yet}%
          \show\HELP
      \else
         \ifdim\wd0>\@tempdima
            \advance\@tempdima by \@tempdima
            \ifdim\wd0 >\@tempdima
               \textwidth=\@tempdima
               \setbox1 =\vbox{%
                  \noindent\hbox to \@tempdima{\hfill\GRAPHIC{#5}{#4}{#1}{#2}{#3}\hfill}\\%
                  \noindent\hbox to \@tempdima{\parbox[b]{\@tempdima}{\QCBOptA}}%
               }%
               \wd1=\@tempdima
            \else
               \textwidth=\wd0
               \setbox1 =\vbox{%
                 \noindent\hbox to \wd0{\hfill\GRAPHIC{#5}{#4}{#1}{#2}{#3}\hfill}\\%
                 \noindent\hbox{\QCBOptA}%
               }%
               \wd1=\wd0
            \fi
         \else
            %\show\BBB
            \ifdim\wd0>0pt
              \hsize=\@tempdima
              \setbox1 =\vbox{%
                \unskip\GRAPHIC{#5}{#4}{#1}{#2}{0pt}%
                \break
                \unskip\hbox to \@tempdima{\hfill \QCBOptA\hfill}%
              }%
              \wd1=\@tempdima
           \else
              \hsize=\@tempdima
              \setbox1 =\vbox{%
                \unskip\GRAPHIC{#5}{#4}{#1}{#2}{0pt}%
              }%
              \wd1=\@tempdima
           \fi
         \fi
         \@tempdimb=\ht1
         \advance\@tempdimb by \dp1
         \advance\@tempdimb by -#2%
         \advance\@tempdimb by #3%
         \leavevmode
         \raise -\@tempdimb \hbox{\box1}%
      \fi
      \egroup%
}%
%
%Macro for Display graphics object
%   \DFRAME{ contentswidth (scalar)  }               %#1
%          { contentsheight (scalar) }               %#2
%          { draft label }                           %#3
%          { name }                                  %#4
%          { caption}                                %#5
\def\DFRAME#1#2#3#4#5{%
 \begin{center}
     \let\QCTOptA\empty
     \let\QCTOptB\empty
     \let\QCBOptA\empty
     \let\QCBOptB\empty
     \ifOverFrame 
        #5\QCTOptA\par
     \fi
     \GRAPHIC{#4}{#3}{#1}{#2}{\z@}
     \ifUnderFrame 
        \nobreak\par #5\QCBOptA
     \fi
 \end{center}%
 }%
%
%Macro for Floating graphic object
%   \FFRAME{ framedata f|i tbph x F|T }              %#1
%          { contentswidth (scalar)  }               %#2
%          { contentsheight (scalar) }               %#3
%          { caption }                               %#4
%          { label }                                 %#5
%          { draft name }                            %#6
%          { body }                                  %#7
\def\FFRAME#1#2#3#4#5#6#7{%
 \begin{figure}[#1]%
  \let\QCTOptA\empty
  \let\QCTOptB\empty
  \let\QCBOptA\empty
  \let\QCBOptB\empty
  \ifOverFrame
    #4
    \ifx\QCTOptA\empty
    \else
      \ifx\QCTOptB\empty
        \caption{\QCTOptA}%
      \else
        \caption[\QCTOptB]{\QCTOptA}%
      \fi
    \fi
    \ifUnderFrame\else
      \label{#5}%
    \fi
  \else
    \UnderFrametrue%
  \fi
  \begin{center}\GRAPHIC{#7}{#6}{#2}{#3}{\z@}\end{center}%
  \ifUnderFrame
    #4
    \ifx\QCBOptA\empty
      \caption{}%
    \else
      \ifx\QCBOptB\empty
        \caption{\QCBOptA}%
      \else
        \caption[\QCBOptB]{\QCBOptA}%
      \fi
    \fi
    \label{#5}%
  \fi
  \end{figure}%
 }%
%
%
%    \FRAME{ framedata f|i tbph x F|T }              %#1
%          { contentswidth (scalar)  }               %#2
%          { contentsheight (scalar) }               %#3
%          { vertical shift when in-line (scalar) }  %#4
%          { caption }                               %#5
%          { label }                                 %#6
%          { name }                                  %#7
%          { body }                                  %#8
%
%    framedata is a string which can contain the following
%    characters: idftbphxFT
%    Their meaning is as follows:
%             i, d or f : in-line, display, or floating
%             t,b,p,h   : LaTeX floating placement options
%             x         : fit contents box to contents
%             F or T    : Figure or Table. 
%                         Later this can expand
%                         to a more general float class.
%
%
\newcount\dispkind%

\def\makeactives{
  \catcode`\"=\active
  \catcode`\;=\active
  \catcode`\:=\active
  \catcode`\'=\active
  \catcode`\~=\active
}
\bgroup
   \makeactives
   \gdef\activesoff{%
      \def"{\string"}
      \def;{\string;}
      \def:{\string:}
      \def'{\string'}
      \def~{\string~}
      %\bbl@deactivate{"}%
      %\bbl@deactivate{;}%
      %\bbl@deactivate{:}%
      %\bbl@deactivate{'}%
    }
\egroup

\def\FRAME#1#2#3#4#5#6#7#8{%
 \bgroup
 \@ifundefined{bbl@deactivate}{}{\activesoff}
 \ifnum\draft=\@ne
   \wasdrafttrue
 \else
   \wasdraftfalse%
 \fi
 \def\LaTeXparams{}%
 \dispkind=\z@
 \def\LaTeXparams{}%
 \doFRAMEparams{#1}%
 \ifnum\dispkind=\z@\IFRAME{#2}{#3}{#4}{#7}{#8}{#5}\else
  \ifnum\dispkind=\@ne\DFRAME{#2}{#3}{#7}{#8}{#5}\else
   \ifnum\dispkind=\tw@
    \edef\@tempa{\noexpand\FFRAME{\LaTeXparams}}%
    \@tempa{#2}{#3}{#5}{#6}{#7}{#8}%
    \fi
   \fi
  \fi
  \ifwasdraft\draft=1\else\draft=0\fi{}%
  \egroup
 }%
%
% This macro added to let SW gobble a parameter that
% should not be passed on and expanded. 

\def\TEXUX#1{"texux"}

%
% Macros for text attributes:
%
\def\BF#1{{\bf {#1}}}%
\def\NEG#1{\leavevmode\hbox{\rlap{\thinspace/}{$#1$}}}%
%
%%%%%%%%%%%%%%%%%%%%%%%%%%%%%%%%%%%%%%%%%%%%%%%%%%%%%%%%%%%%%%%%%%%%%%%%
%
%
% macros for user - defined functions
\def\func#1{\mathop{\rm #1}}%
\def\limfunc#1{\mathop{\rm #1}}%

%
% miscellaneous 
%\long\def\QQQ#1#2{}%
\long\def\QQQ#1#2{%
     \long\expandafter\def\csname#1\endcsname{#2}}%
%\def\QTP#1{}% JCS - this was changed becuase style editor will define QTP
\@ifundefined{QTP}{\def\QTP#1{}}{}
\@ifundefined{QEXCLUDE}{\def\QEXCLUDE#1{}}{}
%\@ifundefined{Qcb}{\def\Qcb#1{#1}}{}
%\@ifundefined{Qct}{\def\Qct#1{#1}}{}
\@ifundefined{Qlb}{\def\Qlb#1{#1}}{}
\@ifundefined{Qlt}{\def\Qlt#1{#1}}{}
\def\QWE{}%
\long\def\QQA#1#2{}%
%\def\QTR#1#2{{\em #2}}% Always \em!!!
%\def\QTR#1#2{\mbox{\begin{#1}#2\end{#1}}}%cb%%%
\def\QTR#1#2{{\csname#1\endcsname #2}}%(gp) Is this the best?
\long\def\TeXButton#1#2{#2}%
\long\def\QSubDoc#1#2{#2}%
\def\EXPAND#1[#2]#3{}%
\def\NOEXPAND#1[#2]#3{}%
\def\PROTECTED{}%
\def\LaTeXparent#1{}%
\def\ChildStyles#1{}%
\def\ChildDefaults#1{}%
\def\QTagDef#1#2#3{}%
%
% Macros for style editor docs
\@ifundefined{StyleEditBeginDoc}{\def\StyleEditBeginDoc{\relax}}{}
%
% Macros for footnotes
\def\QQfnmark#1{\footnotemark}
\def\QQfntext#1#2{\addtocounter{footnote}{#1}\footnotetext{#2}}
%
% Macros for indexing.
\def\MAKEINDEX{\makeatletter\input gnuindex.sty\makeatother\makeindex}%	
\@ifundefined{INDEX}{\def\INDEX#1#2{}{}}{}%
\@ifundefined{SUBINDEX}{\def\SUBINDEX#1#2#3{}{}{}}{}%
\@ifundefined{initial}%  
   {\def\initial#1{\bigbreak{\raggedright\large\bf #1}\kern 2\p@\penalty3000}}%
   {}%
\@ifundefined{entry}{\def\entry#1#2{\item {#1}, #2}}{}%
\@ifundefined{primary}{\def\primary#1{\item {#1}}}{}%
\@ifundefined{secondary}{\def\secondary#1#2{\subitem {#1}, #2}}{}%
%
%
\@ifundefined{ZZZ}{}{\MAKEINDEX\makeatletter}%
%
% Attempts to avoid problems with other styles
\@ifundefined{abstract}{%
 \def\abstract{%
  \if@twocolumn
   \section*{Abstract (Not appropriate in this style!)}%
   \else \small 
   \begin{center}{\bf Abstract\vspace{-.5em}\vspace{\z@}}\end{center}%
   \quotation 
   \fi
  }%
 }{%
 }%
\@ifundefined{endabstract}{\def\endabstract
  {\if@twocolumn\else\endquotation\fi}}{}%
\@ifundefined{maketitle}{\def\maketitle#1{}}{}%
\@ifundefined{affiliation}{\def\affiliation#1{}}{}%
\@ifundefined{proof}{\def\proof{\noindent{\bfseries Proof. }}}{}%
\@ifundefined{endproof}{\def\endproof{\mbox{\ \rule{.1in}{.1in}}}}{}%
\@ifundefined{newfield}{\def\newfield#1#2{}}{}%
\@ifundefined{chapter}{\def\chapter#1{\par(Chapter head:)#1\par }%
 \newcount\c@chapter}{}%
\@ifundefined{part}{\def\part#1{\par(Part head:)#1\par }}{}%
\@ifundefined{section}{\def\section#1{\par(Section head:)#1\par }}{}%
\@ifundefined{subsection}{\def\subsection#1%
 {\par(Subsection head:)#1\par }}{}%
\@ifundefined{subsubsection}{\def\subsubsection#1%
 {\par(Subsubsection head:)#1\par }}{}%
\@ifundefined{paragraph}{\def\paragraph#1%
 {\par(Subsubsubsection head:)#1\par }}{}%
\@ifundefined{subparagraph}{\def\subparagraph#1%
 {\par(Subsubsubsubsection head:)#1\par }}{}%
%%%%%%%%%%%%%%%%%%%%%%%%%%%%%%%%%%%%%%%%%%%%%%%%%%%%%%%%%%%%%%%%%%%%%%%%
% These symbols are not recognized by LaTeX
\@ifundefined{therefore}{\def\therefore{}}{}%
\@ifundefined{backepsilon}{\def\backepsilon{}}{}%
\@ifundefined{yen}{\def\yen{\hbox{\rm\rlap=Y}}}{}%
\@ifundefined{registered}{%
   \def\registered{\relax\ifmmode{}\r@gistered
                    \else$\m@th\r@gistered$\fi}%
 \def\r@gistered{^{\ooalign
  {\hfil\raise.07ex\hbox{$\scriptstyle\rm\text{R}$}\hfil\crcr
  \mathhexbox20D}}}}{}%
\@ifundefined{Eth}{\def\Eth{}}{}%
\@ifundefined{eth}{\def\eth{}}{}%
\@ifundefined{Thorn}{\def\Thorn{}}{}%
\@ifundefined{thorn}{\def\thorn{}}{}%
% A macro to allow any symbol that requires math to appear in text
\def\TEXTsymbol#1{\mbox{$#1$}}%
\@ifundefined{degree}{\def\degree{{}^{\circ}}}{}%
%
% macros for T3TeX files
\newdimen\theight
\def\Column{%
 \vadjust{\setbox\z@=\hbox{\scriptsize\quad\quad tcol}%
  \theight=\ht\z@\advance\theight by \dp\z@\advance\theight by \lineskip
  \kern -\theight \vbox to \theight{%
   \rightline{\rlap{\box\z@}}%
   \vss
   }%
  }%
 }%
%
\def\qed{%
 \ifhmode\unskip\nobreak\fi\ifmmode\ifinner\else\hskip5\p@\fi\fi
 \hbox{\hskip5\p@\vrule width4\p@ height6\p@ depth1.5\p@\hskip\p@}%
 }%
%
\def\cents{\hbox{\rm\rlap/c}}%
\def\miss{\hbox{\vrule height2\p@ width 2\p@ depth\z@}}%
%\def\miss{\hbox{.}}%        %another possibility 
%
\def\vvert{\Vert}%           %always translated to \left| or \right|
%
\def\tcol#1{{\baselineskip=6\p@ \vcenter{#1}} \Column}  %
%
\def\dB{\hbox{{}}}%                 %dummy entry in column 
\def\mB#1{\hbox{$#1$}}%             %column entry
\def\nB#1{\hbox{#1}}%               %column entry (not math)
%
%\newcount\notenumber
%\def\clearnotenumber{\notenumber=0}
%\def\note{\global\advance\notenumber by 1
% \footnote{$^{\the\notenumber}$}}
%\def\note{\global\advance\notenumber by 1
\def\note{$^{\dag}}%
%
%

\def\newfmtname{LaTeX2e}
\def\chkcompat{%
   \if@compatibility
   \else
     \usepackage{latexsym}
   \fi
}

\ifx\fmtname\newfmtname
  \DeclareOldFontCommand{\rm}{\normalfont\rmfamily}{\mathrm}
  \DeclareOldFontCommand{\sf}{\normalfont\sffamily}{\mathsf}
  \DeclareOldFontCommand{\tt}{\normalfont\ttfamily}{\mathtt}
  \DeclareOldFontCommand{\bf}{\normalfont\bfseries}{\mathbf}
  \DeclareOldFontCommand{\it}{\normalfont\itshape}{\mathit}
  \DeclareOldFontCommand{\sl}{\normalfont\slshape}{\@nomath\sl}
  \DeclareOldFontCommand{\sc}{\normalfont\scshape}{\@nomath\sc}
  \chkcompat
\fi

%
% Greek bold macros
% Redefine all of the math symbols 
% which might be bolded	 - there are 
% probably others to add to this list

\def\alpha{\Greekmath 010B }%
\def\beta{\Greekmath 010C }%
\def\gamma{\Greekmath 010D }%
\def\delta{\Greekmath 010E }%
\def\epsilon{\Greekmath 010F }%
\def\zeta{\Greekmath 0110 }%
\def\eta{\Greekmath 0111 }%
\def\theta{\Greekmath 0112 }%
\def\iota{\Greekmath 0113 }%
\def\kappa{\Greekmath 0114 }%
\def\lambda{\Greekmath 0115 }%
\def\mu{\Greekmath 0116 }%
\def\nu{\Greekmath 0117 }%
\def\xi{\Greekmath 0118 }%
\def\pi{\Greekmath 0119 }%
\def\rho{\Greekmath 011A }%
\def\sigma{\Greekmath 011B }%
\def\tau{\Greekmath 011C }%
\def\upsilon{\Greekmath 011D }%
\def\phi{\Greekmath 011E }%
\def\chi{\Greekmath 011F }%
\def\psi{\Greekmath 0120 }%
\def\omega{\Greekmath 0121 }%
\def\varepsilon{\Greekmath 0122 }%
\def\vartheta{\Greekmath 0123 }%
\def\varpi{\Greekmath 0124 }%
\def\varrho{\Greekmath 0125 }%
\def\varsigma{\Greekmath 0126 }%
\def\varphi{\Greekmath 0127 }%

\def\nabla{\Greekmath 0272 }
\def\FindBoldGroup{%
   {\setbox0=\hbox{$\mathbf{x\global\edef\theboldgroup{\the\mathgroup}}$}}%
}

\def\Greekmath#1#2#3#4{%
    \if@compatibility
        \ifnum\mathgroup=\symbold
           \mathchoice{\mbox{\boldmath$\displaystyle\mathchar"#1#2#3#4$}}%
                      {\mbox{\boldmath$\textstyle\mathchar"#1#2#3#4$}}%
                      {\mbox{\boldmath$\scriptstyle\mathchar"#1#2#3#4$}}%
                      {\mbox{\boldmath$\scriptscriptstyle\mathchar"#1#2#3#4$}}%
        \else
           \mathchar"#1#2#3#4% 
        \fi 
    \else 
        \FindBoldGroup
        \ifnum\mathgroup=\theboldgroup % For 2e
           \mathchoice{\mbox{\boldmath$\displaystyle\mathchar"#1#2#3#4$}}%
                      {\mbox{\boldmath$\textstyle\mathchar"#1#2#3#4$}}%
                      {\mbox{\boldmath$\scriptstyle\mathchar"#1#2#3#4$}}%
                      {\mbox{\boldmath$\scriptscriptstyle\mathchar"#1#2#3#4$}}%
        \else
           \mathchar"#1#2#3#4% 
        \fi     	    
	  \fi}

\newif\ifGreekBold  \GreekBoldfalse
\let\SAVEPBF=\pbf
\def\pbf{\GreekBoldtrue\SAVEPBF}%
%

\@ifundefined{theorem}{\newtheorem{theorem}{Theorem}}{}
\@ifundefined{lemma}{\newtheorem{lemma}[theorem]{Lemma}}{}
\@ifundefined{corollary}{\newtheorem{corollary}[theorem]{Corollary}}{}
\@ifundefined{conjecture}{\newtheorem{conjecture}[theorem]{Conjecture}}{}
\@ifundefined{proposition}{\newtheorem{proposition}[theorem]{Proposition}}{}
\@ifundefined{axiom}{\newtheorem{axiom}{Axiom}}{}
\@ifundefined{remark}{\newtheorem{remark}{Remark}}{}
\@ifundefined{example}{\newtheorem{example}{Example}}{}
\@ifundefined{exercise}{\newtheorem{exercise}{Exercise}}{}
\@ifundefined{definition}{\newtheorem{definition}{Definition}}{}


\@ifundefined{mathletters}{%
  %\def\theequation{\arabic{equation}}
  \newcounter{equationnumber}  
  \def\mathletters{%
     \addtocounter{equation}{1}
     \edef\@currentlabel{\theequation}%
     \setcounter{equationnumber}{\c@equation}
     \setcounter{equation}{0}%
     \edef\theequation{\@currentlabel\noexpand\alph{equation}}%
  }
  \def\endmathletters{%
     \setcounter{equation}{\value{equationnumber}}%
  }
}{}

%Logos
\@ifundefined{BibTeX}{%
    \def\BibTeX{{\rm B\kern-.05em{\sc i\kern-.025em b}\kern-.08em
                 T\kern-.1667em\lower.7ex\hbox{E}\kern-.125emX}}}{}%
\@ifundefined{AmS}%
    {\def\AmS{{\protect\usefont{OMS}{cmsy}{m}{n}%
                A\kern-.1667em\lower.5ex\hbox{M}\kern-.125emS}}}{}%
\@ifundefined{AmSTeX}{\def\AmSTeX{\protect\AmS-\protect\TeX\@}}{}%
%

%%%%%%%%%%%%%%%%%%%%%%%%%%%%%%%%%%%%%%%%%%%%%%%%%%%%%%%%%%%%%%%%%%%%%%%
% NOTE: The rest of this file is read only if amstex has not been
% loaded.  This section is used to define amstex constructs in the
% event they have not been defined.
%
%
\ifx\ds@amstex\relax
   \message{amstex already loaded}\makeatother\endinput% 2.09 compatability
\else
   \@ifpackageloaded{amstex}%
      {\message{amstex already loaded}\makeatother\endinput}
      {}
   \@ifpackageloaded{amsgen}%
      {\message{amsgen already loaded}\makeatother\endinput}
      {}
\fi
%%%%%%%%%%%%%%%%%%%%%%%%%%%%%%%%%%%%%%%%%%%%%%%%%%%%%%%%%%%%%%%%%%%%%%%%
%%
%
%
%  Macros to define some AMS LaTeX constructs when 
%  AMS LaTeX has not been loaded
% 
% These macros are copied from the AMS-TeX package for doing
% multiple integrals.
%
\let\DOTSI\relax
\def\RIfM@{\relax\ifmmode}%
\def\FN@{\futurelet\next}%
\newcount\intno@
\def\iint{\DOTSI\intno@\tw@\FN@\ints@}%
\def\iiint{\DOTSI\intno@\thr@@\FN@\ints@}%
\def\iiiint{\DOTSI\intno@4 \FN@\ints@}%
\def\idotsint{\DOTSI\intno@\z@\FN@\ints@}%
\def\ints@{\findlimits@\ints@@}%
\newif\iflimtoken@
\newif\iflimits@
\def\findlimits@{\limtoken@true\ifx\next\limits\limits@true
 \else\ifx\next\nolimits\limits@false\else
 \limtoken@false\ifx\ilimits@\nolimits\limits@false\else
 \ifinner\limits@false\else\limits@true\fi\fi\fi\fi}%
\def\multint@{\int\ifnum\intno@=\z@\intdots@                          %1
 \else\intkern@\fi                                                    %2
 \ifnum\intno@>\tw@\int\intkern@\fi                                   %3
 \ifnum\intno@>\thr@@\int\intkern@\fi                                 %4
 \int}%                                                               %5
\def\multintlimits@{\intop\ifnum\intno@=\z@\intdots@\else\intkern@\fi
 \ifnum\intno@>\tw@\intop\intkern@\fi
 \ifnum\intno@>\thr@@\intop\intkern@\fi\intop}%
\def\intic@{%
    \mathchoice{\hskip.5em}{\hskip.4em}{\hskip.4em}{\hskip.4em}}%
\def\negintic@{\mathchoice
 {\hskip-.5em}{\hskip-.4em}{\hskip-.4em}{\hskip-.4em}}%
\def\ints@@{\iflimtoken@                                              %1
 \def\ints@@@{\iflimits@\negintic@
   \mathop{\intic@\multintlimits@}\limits                             %2
  \else\multint@\nolimits\fi                                          %3
  \eat@}%                                                             %4
 \else                                                                %5
 \def\ints@@@{\iflimits@\negintic@
  \mathop{\intic@\multintlimits@}\limits\else
  \multint@\nolimits\fi}\fi\ints@@@}%
\def\intkern@{\mathchoice{\!\!\!}{\!\!}{\!\!}{\!\!}}%
\def\plaincdots@{\mathinner{\cdotp\cdotp\cdotp}}%
\def\intdots@{\mathchoice{\plaincdots@}%
 {{\cdotp}\mkern1.5mu{\cdotp}\mkern1.5mu{\cdotp}}%
 {{\cdotp}\mkern1mu{\cdotp}\mkern1mu{\cdotp}}%
 {{\cdotp}\mkern1mu{\cdotp}\mkern1mu{\cdotp}}}%
%
%
%  These macros are for doing the AMS \text{} construct
%
\def\RIfM@{\relax\protect\ifmmode}
\def\text{\RIfM@\expandafter\text@\else\expandafter\mbox\fi}
\let\nfss@text\text
\def\text@#1{\mathchoice
   {\textdef@\displaystyle\f@size{#1}}%
   {\textdef@\textstyle\tf@size{\firstchoice@false #1}}%
   {\textdef@\textstyle\sf@size{\firstchoice@false #1}}%
   {\textdef@\textstyle \ssf@size{\firstchoice@false #1}}%
   \glb@settings}

\def\textdef@#1#2#3{\hbox{{%
                    \everymath{#1}%
                    \let\f@size#2\selectfont
                    #3}}}
\newif\iffirstchoice@
\firstchoice@true
%
%    Old Scheme for \text
%
%\def\rmfam{\z@}%
%\newif\iffirstchoice@
%\firstchoice@true
%\def\textfonti{\the\textfont\@ne}%
%\def\textfontii{\the\textfont\tw@}%
%\def\text{\RIfM@\expandafter\text@\else\expandafter\text@@\fi}%
%\def\text@@#1{\leavevmode\hbox{#1}}%
%\def\text@#1{\mathchoice
% {\hbox{\everymath{\displaystyle}\def\textfonti{\the\textfont\@ne}%
%  \def\textfontii{\the\textfont\tw@}\textdef@@ T#1}}%
% {\hbox{\firstchoice@false
%  \everymath{\textstyle}\def\textfonti{\the\textfont\@ne}%
%  \def\textfontii{\the\textfont\tw@}\textdef@@ T#1}}%
% {\hbox{\firstchoice@false
%  \everymath{\scriptstyle}\def\textfonti{\the\scriptfont\@ne}%
%  \def\textfontii{\the\scriptfont\tw@}\textdef@@ S\rm#1}}%
% {\hbox{\firstchoice@false
%  \everymath{\scriptscriptstyle}\def\textfonti
%  {\the\scriptscriptfont\@ne}%
%  \def\textfontii{\the\scriptscriptfont\tw@}\textdef@@ s\rm#1}}}%
%\def\textdef@@#1{\textdef@#1\rm\textdef@#1\bf\textdef@#1\sl
%    \textdef@#1\it}%
%\def\DN@{\def\next@}%
%\def\eat@#1{}%
%\def\textdef@#1#2{%
% \DN@{\csname\expandafter\eat@\string#2fam\endcsname}%
% \if S#1\edef#2{\the\scriptfont\next@\relax}%
% \else\if s#1\edef#2{\the\scriptscriptfont\next@\relax}%
% \else\edef#2{\the\textfont\next@\relax}\fi\fi}%
%
%
%These are the AMS constructs for multiline limits.
%
\def\Let@{\relax\iffalse{\fi\let\\=\cr\iffalse}\fi}%
\def\vspace@{\def\vspace##1{\crcr\noalign{\vskip##1\relax}}}%
\def\multilimits@{\bgroup\vspace@\Let@
 \baselineskip\fontdimen10 \scriptfont\tw@
 \advance\baselineskip\fontdimen12 \scriptfont\tw@
 \lineskip\thr@@\fontdimen8 \scriptfont\thr@@
 \lineskiplimit\lineskip
 \vbox\bgroup\ialign\bgroup\hfil$\m@th\scriptstyle{##}$\hfil\crcr}%
\def\Sb{_\multilimits@}%
\def\endSb{\crcr\egroup\egroup\egroup}%
\def\Sp{^\multilimits@}%
\let\endSp\endSb
%
%
%These are AMS constructs for horizontal arrows
%
\newdimen\ex@
\ex@.2326ex
\def\rightarrowfill@#1{$#1\m@th\mathord-\mkern-6mu\cleaders
 \hbox{$#1\mkern-2mu\mathord-\mkern-2mu$}\hfill
 \mkern-6mu\mathord\rightarrow$}%
\def\leftarrowfill@#1{$#1\m@th\mathord\leftarrow\mkern-6mu\cleaders
 \hbox{$#1\mkern-2mu\mathord-\mkern-2mu$}\hfill\mkern-6mu\mathord-$}%
\def\leftrightarrowfill@#1{$#1\m@th\mathord\leftarrow
\mkern-6mu\cleaders
 \hbox{$#1\mkern-2mu\mathord-\mkern-2mu$}\hfill
 \mkern-6mu\mathord\rightarrow$}%
\def\overrightarrow{\mathpalette\overrightarrow@}%
\def\overrightarrow@#1#2{\vbox{\ialign{##\crcr\rightarrowfill@#1\crcr
 \noalign{\kern-\ex@\nointerlineskip}$\m@th\hfil#1#2\hfil$\crcr}}}%
\let\overarrow\overrightarrow
\def\overleftarrow{\mathpalette\overleftarrow@}%
\def\overleftarrow@#1#2{\vbox{\ialign{##\crcr\leftarrowfill@#1\crcr
 \noalign{\kern-\ex@\nointerlineskip}$\m@th\hfil#1#2\hfil$\crcr}}}%
\def\overleftrightarrow{\mathpalette\overleftrightarrow@}%
\def\overleftrightarrow@#1#2{\vbox{\ialign{##\crcr
   \leftrightarrowfill@#1\crcr
 \noalign{\kern-\ex@\nointerlineskip}$\m@th\hfil#1#2\hfil$\crcr}}}%
\def\underrightarrow{\mathpalette\underrightarrow@}%
\def\underrightarrow@#1#2{\vtop{\ialign{##\crcr$\m@th\hfil#1#2\hfil
  $\crcr\noalign{\nointerlineskip}\rightarrowfill@#1\crcr}}}%
\let\underarrow\underrightarrow
\def\underleftarrow{\mathpalette\underleftarrow@}%
\def\underleftarrow@#1#2{\vtop{\ialign{##\crcr$\m@th\hfil#1#2\hfil
  $\crcr\noalign{\nointerlineskip}\leftarrowfill@#1\crcr}}}%
\def\underleftrightarrow{\mathpalette\underleftrightarrow@}%
\def\underleftrightarrow@#1#2{\vtop{\ialign{##\crcr$\m@th
  \hfil#1#2\hfil$\crcr
 \noalign{\nointerlineskip}\leftrightarrowfill@#1\crcr}}}%
%%%%%%%%%%%%%%%%%%%%%

% 94.0815 by Jon:

\def\qopnamewl@#1{\mathop{\operator@font#1}\nlimits@}
\let\nlimits@\displaylimits
\def\setboxz@h{\setbox\z@\hbox}


\def\varlim@#1#2{\mathop{\vtop{\ialign{##\crcr
 \hfil$#1\m@th\operator@font lim$\hfil\crcr
 \noalign{\nointerlineskip}#2#1\crcr
 \noalign{\nointerlineskip\kern-\ex@}\crcr}}}}

 \def\rightarrowfill@#1{\m@th\setboxz@h{$#1-$}\ht\z@\z@
  $#1\copy\z@\mkern-6mu\cleaders
  \hbox{$#1\mkern-2mu\box\z@\mkern-2mu$}\hfill
  \mkern-6mu\mathord\rightarrow$}
\def\leftarrowfill@#1{\m@th\setboxz@h{$#1-$}\ht\z@\z@
  $#1\mathord\leftarrow\mkern-6mu\cleaders
  \hbox{$#1\mkern-2mu\copy\z@\mkern-2mu$}\hfill
  \mkern-6mu\box\z@$}


\def\projlim{\qopnamewl@{proj\,lim}}
\def\injlim{\qopnamewl@{inj\,lim}}
\def\varinjlim{\mathpalette\varlim@\rightarrowfill@}
\def\varprojlim{\mathpalette\varlim@\leftarrowfill@}
\def\varliminf{\mathpalette\varliminf@{}}
\def\varliminf@#1{\mathop{\underline{\vrule\@depth.2\ex@\@width\z@
   \hbox{$#1\m@th\operator@font lim$}}}}
\def\varlimsup{\mathpalette\varlimsup@{}}
\def\varlimsup@#1{\mathop{\overline
  {\hbox{$#1\m@th\operator@font lim$}}}}

%
%%%%%%%%%%%%%%%%%%%%%%%%%%%%%%%%%%%%%%%%%%%%%%%%%%%%%%%%%%%%%%%%%%%%%
%
\def\tfrac#1#2{{\textstyle {#1 \over #2}}}%
\def\dfrac#1#2{{\displaystyle {#1 \over #2}}}%
\def\binom#1#2{{#1 \choose #2}}%
\def\tbinom#1#2{{\textstyle {#1 \choose #2}}}%
\def\dbinom#1#2{{\displaystyle {#1 \choose #2}}}%
\def\QATOP#1#2{{#1 \atop #2}}%
\def\QTATOP#1#2{{\textstyle {#1 \atop #2}}}%
\def\QDATOP#1#2{{\displaystyle {#1 \atop #2}}}%
\def\QABOVE#1#2#3{{#2 \above#1 #3}}%
\def\QTABOVE#1#2#3{{\textstyle {#2 \above#1 #3}}}%
\def\QDABOVE#1#2#3{{\displaystyle {#2 \above#1 #3}}}%
\def\QOVERD#1#2#3#4{{#3 \overwithdelims#1#2 #4}}%
\def\QTOVERD#1#2#3#4{{\textstyle {#3 \overwithdelims#1#2 #4}}}%
\def\QDOVERD#1#2#3#4{{\displaystyle {#3 \overwithdelims#1#2 #4}}}%
\def\QATOPD#1#2#3#4{{#3 \atopwithdelims#1#2 #4}}%
\def\QTATOPD#1#2#3#4{{\textstyle {#3 \atopwithdelims#1#2 #4}}}%
\def\QDATOPD#1#2#3#4{{\displaystyle {#3 \atopwithdelims#1#2 #4}}}%
\def\QABOVED#1#2#3#4#5{{#4 \abovewithdelims#1#2#3 #5}}%
\def\QTABOVED#1#2#3#4#5{{\textstyle 
   {#4 \abovewithdelims#1#2#3 #5}}}%
\def\QDABOVED#1#2#3#4#5{{\displaystyle 
   {#4 \abovewithdelims#1#2#3 #5}}}%
%
% Macros for text size operators:

%JCS - added braces and \mathop around \displaystyle\int, etc.
%
\def\tint{\mathop{\textstyle \int}}%
\def\tiint{\mathop{\textstyle \iint }}%
\def\tiiint{\mathop{\textstyle \iiint }}%
\def\tiiiint{\mathop{\textstyle \iiiint }}%
\def\tidotsint{\mathop{\textstyle \idotsint }}%
\def\toint{\mathop{\textstyle \oint}}%
\def\tsum{\mathop{\textstyle \sum }}%
\def\tprod{\mathop{\textstyle \prod }}%
\def\tbigcap{\mathop{\textstyle \bigcap }}%
\def\tbigwedge{\mathop{\textstyle \bigwedge }}%
\def\tbigoplus{\mathop{\textstyle \bigoplus }}%
\def\tbigodot{\mathop{\textstyle \bigodot }}%
\def\tbigsqcup{\mathop{\textstyle \bigsqcup }}%
\def\tcoprod{\mathop{\textstyle \coprod }}%
\def\tbigcup{\mathop{\textstyle \bigcup }}%
\def\tbigvee{\mathop{\textstyle \bigvee }}%
\def\tbigotimes{\mathop{\textstyle \bigotimes }}%
\def\tbiguplus{\mathop{\textstyle \biguplus }}%
%
%
%Macros for display size operators:
%

\def\dint{\mathop{\displaystyle \int}}%
\def\diint{\mathop{\displaystyle \iint }}%
\def\diiint{\mathop{\displaystyle \iiint }}%
\def\diiiint{\mathop{\displaystyle \iiiint }}%
\def\didotsint{\mathop{\displaystyle \idotsint }}%
\def\doint{\mathop{\displaystyle \oint}}%
\def\dsum{\mathop{\displaystyle \sum }}%
\def\dprod{\mathop{\displaystyle \prod }}%
\def\dbigcap{\mathop{\displaystyle \bigcap }}%
\def\dbigwedge{\mathop{\displaystyle \bigwedge }}%
\def\dbigoplus{\mathop{\displaystyle \bigoplus }}%
\def\dbigodot{\mathop{\displaystyle \bigodot }}%
\def\dbigsqcup{\mathop{\displaystyle \bigsqcup }}%
\def\dcoprod{\mathop{\displaystyle \coprod }}%
\def\dbigcup{\mathop{\displaystyle \bigcup }}%
\def\dbigvee{\mathop{\displaystyle \bigvee }}%
\def\dbigotimes{\mathop{\displaystyle \bigotimes }}%
\def\dbiguplus{\mathop{\displaystyle \biguplus }}%
%
%Companion to stackrel
\def\stackunder#1#2{\mathrel{\mathop{#2}\limits_{#1}}}%
%
%
% These are AMS environments that will be defined to
% be verbatims if amstex has not actually been 
% loaded
%
%
\begingroup \catcode `|=0 \catcode `[= 1
\catcode`]=2 \catcode `\{=12 \catcode `\}=12
\catcode`\\=12 
|gdef|@alignverbatim#1\end{align}[#1|end[align]]
|gdef|@salignverbatim#1\end{align*}[#1|end[align*]]

|gdef|@alignatverbatim#1\end{alignat}[#1|end[alignat]]
|gdef|@salignatverbatim#1\end{alignat*}[#1|end[alignat*]]

|gdef|@xalignatverbatim#1\end{xalignat}[#1|end[xalignat]]
|gdef|@sxalignatverbatim#1\end{xalignat*}[#1|end[xalignat*]]

|gdef|@gatherverbatim#1\end{gather}[#1|end[gather]]
|gdef|@sgatherverbatim#1\end{gather*}[#1|end[gather*]]

|gdef|@gatherverbatim#1\end{gather}[#1|end[gather]]
|gdef|@sgatherverbatim#1\end{gather*}[#1|end[gather*]]


|gdef|@multilineverbatim#1\end{multiline}[#1|end[multiline]]
|gdef|@smultilineverbatim#1\end{multiline*}[#1|end[multiline*]]

|gdef|@arraxverbatim#1\end{arrax}[#1|end[arrax]]
|gdef|@sarraxverbatim#1\end{arrax*}[#1|end[arrax*]]

|gdef|@tabulaxverbatim#1\end{tabulax}[#1|end[tabulax]]
|gdef|@stabulaxverbatim#1\end{tabulax*}[#1|end[tabulax*]]


|endgroup
  

  
\def\align{\@verbatim \frenchspacing\@vobeyspaces \@alignverbatim
You are using the "align" environment in a style in which it is not defined.}
\let\endalign=\endtrivlist
 
\@namedef{align*}{\@verbatim\@salignverbatim
You are using the "align*" environment in a style in which it is not defined.}
\expandafter\let\csname endalign*\endcsname =\endtrivlist




\def\alignat{\@verbatim \frenchspacing\@vobeyspaces \@alignatverbatim
You are using the "alignat" environment in a style in which it is not defined.}
\let\endalignat=\endtrivlist
 
\@namedef{alignat*}{\@verbatim\@salignatverbatim
You are using the "alignat*" environment in a style in which it is not defined.}
\expandafter\let\csname endalignat*\endcsname =\endtrivlist




\def\xalignat{\@verbatim \frenchspacing\@vobeyspaces \@xalignatverbatim
You are using the "xalignat" environment in a style in which it is not defined.}
\let\endxalignat=\endtrivlist
 
\@namedef{xalignat*}{\@verbatim\@sxalignatverbatim
You are using the "xalignat*" environment in a style in which it is not defined.}
\expandafter\let\csname endxalignat*\endcsname =\endtrivlist




\def\gather{\@verbatim \frenchspacing\@vobeyspaces \@gatherverbatim
You are using the "gather" environment in a style in which it is not defined.}
\let\endgather=\endtrivlist
 
\@namedef{gather*}{\@verbatim\@sgatherverbatim
You are using the "gather*" environment in a style in which it is not defined.}
\expandafter\let\csname endgather*\endcsname =\endtrivlist


\def\multiline{\@verbatim \frenchspacing\@vobeyspaces \@multilineverbatim
You are using the "multiline" environment in a style in which it is not defined.}
\let\endmultiline=\endtrivlist
 
\@namedef{multiline*}{\@verbatim\@smultilineverbatim
You are using the "multiline*" environment in a style in which it is not defined.}
\expandafter\let\csname endmultiline*\endcsname =\endtrivlist


\def\arrax{\@verbatim \frenchspacing\@vobeyspaces \@arraxverbatim
You are using a type of "array" construct that is only allowed in AmS-LaTeX.}
\let\endarrax=\endtrivlist

\def\tabulax{\@verbatim \frenchspacing\@vobeyspaces \@tabulaxverbatim
You are using a type of "tabular" construct that is only allowed in AmS-LaTeX.}
\let\endtabulax=\endtrivlist

 
\@namedef{arrax*}{\@verbatim\@sarraxverbatim
You are using a type of "array*" construct that is only allowed in AmS-LaTeX.}
\expandafter\let\csname endarrax*\endcsname =\endtrivlist

\@namedef{tabulax*}{\@verbatim\@stabulaxverbatim
You are using a type of "tabular*" construct that is only allowed in AmS-LaTeX.}
\expandafter\let\csname endtabulax*\endcsname =\endtrivlist

% macro to simulate ams tag construct


% This macro is a fix to eqnarray
\def\@@eqncr{\let\@tempa\relax
    \ifcase\@eqcnt \def\@tempa{& & &}\or \def\@tempa{& &}%
      \else \def\@tempa{&}\fi
     \@tempa
     \if@eqnsw
        \iftag@
           \@taggnum
        \else
           \@eqnnum\stepcounter{equation}%
        \fi
     \fi
     \global\tag@false
     \global\@eqnswtrue
     \global\@eqcnt\z@\cr}


% This macro is a fix to the equation environment
 \def\endequation{%
     \ifmmode\ifinner % FLEQN hack
      \iftag@
        \addtocounter{equation}{-1} % undo the increment made in the begin part
        $\hfil
           \displaywidth\linewidth\@taggnum\egroup \endtrivlist
        \global\tag@false
        \global\@ignoretrue   
      \else
        $\hfil
           \displaywidth\linewidth\@eqnnum\egroup \endtrivlist
        \global\tag@false
        \global\@ignoretrue 
      \fi
     \else   
      \iftag@
        \addtocounter{equation}{-1} % undo the increment made in the begin part
        \eqno \hbox{\@taggnum}
        \global\tag@false%
        $$\global\@ignoretrue
      \else
        \eqno \hbox{\@eqnnum}% $$ BRACE MATCHING HACK
        $$\global\@ignoretrue
      \fi
     \fi\fi
 } 

 \newif\iftag@ \tag@false
 
 \def\tag{\@ifnextchar*{\@tagstar}{\@tag}}
 \def\@tag#1{%
     \global\tag@true
     \global\def\@taggnum{(#1)}}
 \def\@tagstar*#1{%
     \global\tag@true
     \global\def\@taggnum{#1}%  
}

% Do not add anything to the end of this file.  
% The last section of the file is loaded only if 
% amstex has not been.



\makeatother
\endinput

\usepackage[unicode=true,pdfusetitle,
 bookmarks=true,bookmarksnumbered=false,bookmarksopen=false,
 breaklinks=false,pdfborder={0 0 0},pdfborderstyle={},backref=false,colorlinks=true]
 {hyperref}
\hypersetup{
 colorlinks,linkcolor=red,citecolor=blue}
\makeatother

\begin{document}

% \preprint{APS/123-QED}

% \title{Flipmon: Towards scalable superconducting qubits}% Force line breaks with \\
%\thanks{A footnote to the article title}%

\title{Vacuum-gap transmon qubits realized using  flip-chip technology}


\author{Xuegang Li}
\thanks{These authors contributed equally.}
\affiliation{%
	Beijing Academy of Quantum Information Sciences,  Beijing 100193, China
}%
\author{Yingshan Zhang}
\thanks{These authors contributed equally.}
\affiliation{%
	Beijing Academy of Quantum Information Sciences, Beijing 100193, China
}%
\author{Chuhong Yang}
\thanks{These authors contributed equally.}
\affiliation{%
	Beijing Academy of Quantum Information Sciences,  Beijing 100193, China
}%
\author{Zhiyuan Li}
\affiliation{%
	Beijing Academy of Quantum Information Sciences,  Beijing 100193, China
}%
\author{Junhua Wang}
\affiliation{%
	Beijing Academy of Quantum Information Sciences,  Beijing 100193, China
}%
\author{Tang Su}
\affiliation{%
	Beijing Academy of Quantum Information Sciences,  Beijing 100193, China
}%
\author{Mo Chen}
\affiliation{%
	Beijing Academy of Quantum Information Sciences,  Beijing 100193, China
}%
\author{Yongchao Li}
\affiliation{%
	Beijing Academy of Quantum Information Sciences,  Beijing 100193, China
}%
\author{Chengyao Li}
\affiliation{%
	Beijing Academy of Quantum Information Sciences,  Beijing 100193, China
}%
\author{Zhenyu Mi}
\affiliation{%
	Beijing Academy of Quantum Information Sciences,  Beijing 100193, China
}%
\author{Xuehui Liang}
\affiliation{%
	Beijing Academy of Quantum Information Sciences,  Beijing 100193, China
}%
\author{Chenlu Wang}
\affiliation{%
	Beijing Academy of Quantum Information Sciences,  Beijing 100193, China
}%
\author{Zhen Yang}
\affiliation{%
	Beijing Academy of Quantum Information Sciences,  Beijing 100193, China
}%
\author{Yulong Feng}
\affiliation{%
	Beijing Academy of Quantum Information Sciences,  Beijing 100193, China
}%
\author{Kehuan Linghu}
\affiliation{%
	Beijing Academy of Quantum Information Sciences,  Beijing 100193, China
}%
\author{Huikai Xu}
\affiliation{%
	Beijing Academy of Quantum Information Sciences,  Beijing 100193, China
}%
\author{Jiaxiu Han}
\affiliation{%
	Beijing Academy of Quantum Information Sciences,  Beijing 100193, China
}%
\author{Weiyang Liu}
\affiliation{%
	Beijing Academy of Quantum Information Sciences,  Beijing 100193, China
}%
\author{Peng Zhao}
\affiliation{%
	Beijing Academy of Quantum Information Sciences,  Beijing 100193, China
}%
\author{Teng Ma}
\affiliation{%
	Beijing Academy of Quantum Information Sciences,  Beijing 100193, China
}%
\author{Ruixia Wang}
\affiliation{%
	Beijing Academy of Quantum Information Sciences,  Beijing 100193, China
}%
\author{Jingning Zhang}
\affiliation{%
	Beijing Academy of Quantum Information Sciences,  Beijing 100193, China
}%
\author{Yu Song}
\affiliation{%
	Beijing Academy of Quantum Information Sciences,  Beijing 100193, China
}%
\author{Pei Liu}
\affiliation{State Key Laboratory of Low Dimensional Quantum Physics, Department of Physics, Tsinghua University, Beijing 100084, China}
\author{Ziting Wang}
\affiliation{Beijing National Laboratory for Condensed Matter Physics,
	Institute of Physics, Chinese Academy of Sciences, Beijing 100190, China}
\author{Zhaohua Yang}
\affiliation{Beijing National Laboratory for Condensed Matter Physics,
	Institute of Physics, Chinese Academy of Sciences, Beijing 100190, China}
\author{Guangming Xue}
 \email{xuegm@baqis.ac.cn}
\affiliation{%
	Beijing Academy of Quantum Information Sciences,  Beijing 100193, China
}%
\author{Yirong Jin}
 \email{jinyr@baqis.ac.cn}
\affiliation{%
	Beijing Academy of Quantum Information Sciences,  Beijing 100193, China
}%
\author{Haifeng Yu}
 \email{hfyu@baqis.ac.cn}
\affiliation{%
	Beijing Academy of Quantum Information Sciences,  Beijing 100193, China
}%

%\author{Second Author}%
% \email{Second.Author@institution.edu}


\date{\today}% It is always \today, today,
             %  but any date may be explicitly specified

\begin{abstract}
Significant progress has been made in building large-scale superconducting quantum processors based on flip-chip technology. In this work, we use the flip-chip technology to realize a modified transmon qubit, donated as the "flipmon", whose large shunt capacitor is replaced by a vacuum-gap parallel plate capacitor. To further reduce the qubit footprint, we place one of the qubit pads and a single Josephson junction on the bottom chip and the other pad on the top chip which is galvanically connected with the single Josephson junction through an indium bump. The electric field participation ratio can arrive at nearly 53\% in air when the vacuum-gap is about 5 $\mathrm{\mu m}$, and thus potentially leading to a lower dielectric loss. The coherence times of the flipmons are measured in the range of 30-60\,$\mathrm{\mu}$s, which are comparable with that of traditional transmons with similar fabrication processes. The electric field simulation indicates that the metal-air interface's participation ratio increases significantly and may dominate the qubit's decoherence. This suggests that more careful surface treatment needs to be considered. No evidence shows that the indium bumps inside the flipmons cause significant decoherence. With well-designed geometry and good surface treatment, the coherence of the flipmons can be further improved.


%Furthermore, the electric field participation ratio of the indium bump surface is several orders of magnitude smaller than that of the other interfaces. A more elaborately geometry design can minimize the interfaces participation ratio further in the future.




%Scalable superconducting qubit architectures should have small footprint, flexible connectivity, well addressability, and tunability. While it may be a challenge in maintaining good coherence properties, which is substantial for implementing practical quantum computations or simulations. Here we introduce a new type of transmon, denoted as flipmon, which can potentially meet the requirements above. The flipmon includes a vacuum-gapped parallel plate shunt capacitor formed by applying a flip-chip interconnection process. Simulation of the electric field distribution indicates that flipmon has a higher vacuum energy participation ratio than typical planar transmon/Xmon. As a result, the dielectric loss can be suppressed. Tantalum was used for the capacitor pads, and an indium bump was used to connect two pads to construct a shunt capacitor. The critical current of the bumps was measured to over 10\, mA with high yield, and the distance between the pads could be well controlled through the stop bumps. The coherence properties of the flipmons were measured in the range of 30--60\,$\mathrm{\mu}$s, which is comparable to the standard 2D transmons/Xmons, but with a much smaller footprint. No evidence was found that the indium bump inside the flipmon causes significant decoherence. A two-dimensional qubit array based on flipmons was also proposed. Our results show that superconducting quantum circuits with higher density and good coherence are well achievable if the interfaces can be further improved.
\end{abstract}

%\keywords{flipmon, flip-chip bonding, metal-air interface}%Use showkeys class option if keyword
                              %display desired
\maketitle


%\tableofcontents

\section{\label{sec:intro}introduction}
Superconducting quantum processors have reached the scale of over 50 qubits with high gate fidelity, good addressability~\cite{Google2019,Gongeabg7812}. One of the short-term goals for the quantum computing hardware community is to further increase the scale to the order of 1000 qubits, which may implement some practical applications, such as quantum computational chemistry~\cite{McArdle2020}, combinatorial optimization problem~\cite{harrigan2021quantum}. In the challenge of achieving this goal, qubits with compact design, flexible connectivity, and good coherence are crucial issues. Transmon is one of the most promising candidates to meet these requirements~\cite{Koch2007Charge}. Traditional planar transmon designs face wiring problem, which limits their connectivity. In order to solve this problem, 3D or semi-3D interconnection technologies are intensively explored, including air-bridge~\cite{chen2014fabrication,dunsworth2018method}, flip-chip ~\cite{Foxen_2017,rosenberg20173d,satzinger2019simple}, through-silicon via (TSV)~\cite{yost2020solid,mallek2021fabrication}, pass through holes~\cite{Gongeabg7812}.
Currently, flip-chip technology is the most popular one due to its simple fabrication process and process stability. Besides, flip-chip technology provides another probability for the design of superconducting quantum processors (such as resonators~\cite{satzinger2019simple,kelly2020low}, capacitive couplers~\cite{gold2021entanglement}, and so on).

%A major difficulty for all these interconnection technologies is that, unlike in the semiconductor industry, where insulating mediums are commonly used as the isolation layer for different circuit layers, superconducting quantum circuits require an extremely high-quality factor for the elements in it, including qubits, resonators, couplers, etc. Furthermore, the fabrication of the interconnects must be compatible with the other processes for fabricating quantum processors, especially the processes for junction fabrication. Flip-chip technology with superconducting Indium bumps is convinced to be a robust approach for the 3D interconnection of superconducting quantum circuits. With the help of flip-chip technology, one can layout the circuit elements more flexibly.

\begin{figure*}[ht]
% 	\includegraphics[width=.9\textwidth]{Flipmon_fig1_s3.pdf}
	\includegraphics{Flipmon_fig1_s3.pdf}
	\caption{\label{fig:design} Schematics of (a) the traditional transmon, (b) modified transmon with flip-chip, and (c) flipmon. The metal film on the bottom chip (the chip) and the top chip (the carrier) are magenta and yellow, respectively. The single Josephson junction (red) is deposited between two floated capacitor pads on the bottom substrate (blue). The substrate of the chip is not drawn for concision. The metal pad on the chip is either capacitively (b) or galvanically (c) connected with one of the metal pads of the shunt capacitor on the carrier. The vacuum gap between the chip and the carrier is about 5 $\mu m$ in the experiment. (d) The simulated distribution of electrical field in flipmon. Top: side view at the y position = 0 $\mu m$. Bottom: top view at the z position = 5 $\mu m$. The white rectangle in the side view shows that there is no electric field distribution inside the indium bump. (e) Magnified optical micrograph of the geometric structure on the carrier of flipmon. (f) The SEM image of a "T-shaped" junction.}
\end{figure*}

A large shunt capacitor is the key component of the transmon, which can suppress the charging noise and maintain a sufficient anharmonicity. In addition to the traditional transmon with a large coplanar shunt capacitor, a new transmon design (mergemon) which merges the shunt capacitor into the Josephson junction, has been demonstrated in Ref.~\cite{zhao2020merged,mamin2021merged}.  Such a design can achieve a much smaller qubit footprint, whereas a large junction may increase the probability of two-level systems (TLS) in the junction area.
In this work, we propose a new design of transmon by using the flip-chip technology, which we denoted as "flipmon". The central idea of this design is to use a vacuum-gap parallel plate capacitor as the shunt capacitor of transmon. Although vacuum-gap capacitors have been demonstrated in Ref.~\cite{bosman2017multi,cicak2009vacuum}, as far as we know, it is the first time that the vacuum-gap capacitor is used in a transmon design by using the flip-chip technology. This design not only reduces the footprint of the qubit but also has flexible connectivity. The flipmons also increase the electric field participation ratio in the air to nearly 53\%. In contrast, the traditional transmon design has an electric field participation ratio of only about 10\% in the air~\cite{wang2015surface}. The higher electric field participation ratio in the air may decrease the dielectric loss~\cite{cicak2009vacuum}.

We first tested the superconductivity of the flip-chip bumps whose superconducting critical current was over $10~\mathrm{mA}$ with almost 100\% yield. The heights of indium bumps were measured using the scanning electron microscopic (SEM) with deviations less than 0.5 $\mu m$. In order to demonstrate the performance of the flipmons, we measured three flipmon samples with the averaged coherence times in a range of 30-60~$\mathrm{\mu s}$. The detailed simulation of the electric field distribution of the flipmon indicates that the metal-air (MA) interface may be a dominant source of decoherence. We believe that the indium bumps inside the flipmons have little impact on their energy relaxation, as discussed in Section~\ref{sec:meas}. In general, flipmon may be a promising candidate for future high-density large-scale superconducting quantum processors.

%We also designed a transmon under same fabrication process without the indium bumps inside the shunt capacitor with as same coherence properties as flipmon.



%We also measured the traditional transmon with coherence in range of $30-60\mathrm{\mu s}$.

%In our experiment, batches of flipmon qubits with different designs were measured, with averaged coherence times from $30-60\mathrm{\mu s}$, which are comparable to those of typical transmon designs. The superconductivity of the flip-chip bumps has been fully tested with minimum superconducting critical current over $10\mathrm{mA}$ and almost 100\% yield. It is evident that the indium bump inside the flipmon does not induce significant loss to the qubit. Still, the coherence of the flipmons was not as good as the transmons with similar fabrication processes. We simulated the electric field distribution in the capacitor components of the flipmon and gave a full discussion of the source of decoherence. We focused on the interface losses. By simulation of the participation ratio of different interfaces, including metal-metal (MM), metal-air (MA), metal-substrate (MS), and substrate-air (SA), we conclude that MA interface loss may be the main source of decoherence in flipmons. As a result, with more careful surface treatment strategies, we believe that the MA loss can be further suppressed and the coherence of flipmons can potentially be improved to the order of a millisecond. In addition to the other advantages of flip-chip technology, flipmon may be a promising candidate design for future high-density large-scale superconducting quantum processors. A two-dimensional qubit array was designed, demonstrating such a potential.
%As noisy intermediate-scale quantum era is approaching \cite{preskill2018quantum}, one of the short-term goals for the quantum computing community is to increase the scale of a quantum processor to the order of a thousand qubits, without sacrificing coherence and addressability. The footprint and coherence can be good at the same time in compact designs like merged element transmon that merges qubit junction capacitor with shunt capacitor \cite{zhao2020merged,mamin2021merged}. Still, restrictions on decoherence that are set by two-level systems in qubit capacitor need to be lifted through annealing or other approaches. However, planar transmon designs lack the capability to wire a large two-dimensional array of qubits to surrounding pads for control and readout. Air-bridges \cite{chen2014fabrication,dunsworth2018method} can solve the wiring problem partly, allowing one wire to fly over another. This structure will eventually be superseded by 3D approaches, which allow extracting the wirings into additional layers, and using vias for interconnection between layers. However, unlike in semiconductor industry, the common techniques of adding insulating medium to ensure isolation between multiple circuit layers is not practical for quantum chips, because such lossy medium introduces a large amount of two-level systems (TLSs) that can extract energy from the qubit and kill the excitation rapidly. Through-silicon via (TSV) technology is a good candidate for scalable architecture but quite difficult to fabricate together with a high-coherence quantum chip. Great progress has been made \cite{yost2020solid,mallek2021fabrication}, but for most groups it is still a next-generation technology. Meanwhile, we can use relatively mature flip-chip bonding to fold large structures in order to increase the density of area-consuming elements and earn flexibility in wiring. The metallization are still planar but now we have two layers instead. \red{Isolation or galvanic connection between the two layers on separate chips are engineered by air gap or superconductive indium bumps, respectively}. The additional layer makes wiring easier. The most area-consuming elements in a superconducting quantum device are capacitors and resonators. \red{Coplanar waveguide resonators can be folded in half in flip-chip geometry \cite{kelly2020low}.} Another solution is to shrink the gap between metal electrodes of a capacitor or between signal line and ground of a resonator to save some area.  Naturally, we can make use of the small gap between flipped chips to construct parallel plate capacitors that have small footprint as well as relatively low loss. It has been proven that replacing lossy dielectrics with vacuum gaps can lower the total loss tangent of capacitors \cite{cicak2009vacuum}. When compared to a planar interdigital capacitor, the footprint of such a vacuum-gap capacitor is smaller, although it showed no advantage in loss, presumably because of the loss from native oxides on the inner surfaces of the parallel plates \cite{cicak2010low}. Vacuum-gap capacitor with a diameter of less than 100\,$\mathrm{\mu}$m and gap of hundreds of nanometers has been applied to transmon, through the release of a sacrificial layer \cite{bosman2017multi}. However, the focus of that work was on ultrastrong coupling rather than scalability, thus coherence time was not quite optimized.



%So far, in all transmon designs with flip-chip interconnects, planar capacitors take up the most of qubit shunt capacitance \cite{rosenberg20173d,kelly2020low}, similar to typical planar transmons. Why not take advantage of vacuum-gap capacitors? An engineering problem is the difficulty in controlling the distance between the bonded chips which can alter the capacitance value. More importantly, there is a problem with decoherence. Dielectric loss is a key factor that limits the relaxation time of superconducting qubits. Smaller footprint causes increase in dielectric loss of all interfaces \cite{gambetta2016investigating,niepce2020geometric}. The relaxation time $T_1$ of a planar transmon is typically tens of microseconds \cite{kjaergaard2020superconducting}. Traditionally, shunt capacitors are made in aluminium or niobium by evaporation or sputtering. When \ce{TiN} is used instead, $T_1$ and Ramsey dephasing time $T_\mathrm{2}^*$ of $\sim 60\,\mathrm{\mu}$s has been demonstrated even with interdigital capacitor rather than large pads \cite{chang2013improved}. Recently, a dramatic improvement has been achieved. $T_1$ of 0.36\,ms \cite{place2021new} and 0.5\,ms \cite{yu2021tatransmon} has been measured on 2D transmon by resorting to tantalum as the electrodes. Their results manifested that the interfaces between tantalum and air or sapphire substrate are good. During flip-chip bonding, additional decoherence may originate from the surface of indium bumps, lossy intermetallic metallization and interfaces. It is beneficial to pick tantalum as the base metal because there is no need for intermetallic metallization as diffusion barrier. These facts indicate that tantalum may bring about better interfaces that enable low-loss flip-chip-bonded vacuum-gap capacitors.

%In this work, we proposed and realized a new type of qubit, flipmon, that uses vacuum-gap tantalum capacitor as the shunt capacitor of transmon. Actually, we choose tantalum for all planar structures except for qubit junction, which needs double-angle evaporation of aluminium as superconducting islands and \ce{AlOx} as tunnelling layer. During process development, we first characterize the indium bumps on tantalum through direct-current measurements. We fabricated and characterized typical transmon without flip-chip bonding and two types of flipmon, namely flipmon w/o an indium bump inside its shunt capacitor. The flipmon samples we measured showed four times less coherence time than typical transmon, predictable from decreased footprint and increased participation ratio. However, it is worth noting that bumps showed no eminent effects on qubit decoherence and the flip-chip geometry did not bring about evidently larger loss tangent of interfaces. It can be foreseen that with more careful surface treatment strategies, we can further decrease dielectric loss from interfaces and get a even better performance while scaling to higher-density. A two-dimensional array was also designed, demonstrating the potential of flipmon.



\section{\label{sec:design}Flipmon design}
A traditional planar transmon with a floating shunted capacitor is shown in Fig.~\ref{fig:design} (a). Generally, larger capacitor pads lead to a smaller participation ratio of some lossy interfaces, such as the MA, metal-substrate (MS), and substrate-air (SA) interface. As a result, coherence of the qubit can be improved. When scaling up the quantum circuits with transmons, some difficulties, including wiring, crosstalk, etc., appear. Flip-chip technology is a feasible measure to alleviate this problem by bonding two chips together, where indium bumps are commonly used because of their ease of fabrication, soft material properties, and good superconductivity. Here after, for convenience of description, we call the top chip as "chip", and the bottom one as "carrier".


%In a superconducting quantum circuit, indium bumps can act as an interconnection for various circuit elements, including ground planes, control and readout lines, and even transmission line resonators. Until now, no one tried to put it into a qubit, which is the key component of a quantum circuit. Furthermore, by introducing indium bumps into a qubit, it is possible to use a vaccum-gap capacitor as the shunt capacitor and thus can suppress the dielectric loss.

%There are two challenges to achieve this. First, a qubit need to keep its quantum coherence to achieve accurate control and readout. A new material with new interfaces introduced in it may introduce significant decoherence. Second, qubits' parameters must be well controlled in order to meet the requirements for scaling up. bumps inside qubits may cause their parameters hard to control, especially their capacitance.

%However, by introducing indium bumps into a quantum circuit, it is possible to use vaccum-gap capacitor as the shunt capacitor for qubits.
%A vacuum-gap capacitor have very uniform field distribution, and zero dielectric loss, when compare to the commonly used co-planar capacitor.


% Such parallel plate capacitor also has other benefits, for example, it can decrease the footprint of a quantum processor, and it's parameter is easy to design.

%To explore the possibility of utilizing the vacuum-gap capacitor in transmon qubits, in addition, including the indium bumps into them, we designed two different qubits with the flip-chip scheme.
We designed two different qubit schemes using the flip-chip technology to explore the possibility of utilizing the vacuum-gap capacitor in a transmon. The first design is shown in Fig.~\ref{fig:design} (b). It is similar to the traditional planar transmon, except for an additional pad arranged above the capacitor pads on the carrier. The metallizations on the carrier and the chip are represented in magenta and yellow, respectively. The substrate of the chip is not shown for clarity. The second design, denoted as "flipmon", is shown in Fig.~\ref{fig:design} (c). It utilizes the flip-chip technology to place one of the capacitor pads on the chip and the other on the carrier. A single Josephson junction (marked as the red square) formed on the carrier is connected to the chip through an indium bump. This geometry can achieve a smaller qubit footprint.

\begin{table}[t]
\begin{center}
\begin{tabular}{ |c|c| c| c|c| }
    \hline
	 &${\rm Sub_t}$&${\rm SM_t}$&${\rm SA_t}$&${\rm MA_t}$\\
	 \hline
	 $p_i$&0.105&1.31e-5&1.12e-5&3.32e-5\\
	 \hline
	 \hline
	 & Vacuum gap&Indium pump&& \\
	 \hline
	 $p_i$&0.532&1.03e-11&&\\
	 \hline
	 \hline
	 &${\rm Sub_b}$&${\rm SM_b}$&${\rm SA_b}$& ${\rm MA_b}$\\
	 \hline
     $p_i$& 0.363&3.86e-5&1.20e-4&2.07e-5\\
     \hline

\end{tabular}
\end{center}
\label{tab:EPR}
\caption{Energy participation ratio of all components of the flipmon geometry. The subscript ${\rm t}$ in the first row is representing the top chip (the chip). The subscript ${\rm b}$ in the last row is representing the bottom chip (the carrier). The energy participation ratio of the indium pump surface is almost zero and thus leads nearly no impact on the qubit energy relaxation. Over half of the energy is distributed in the vacuum gap and thus can reduce the dielectric loss.}
\end{table}


One of the challenges of the flipmon is whether the indium pump brings additional loss. Fortunately, through the finite element simulation of the electric field distribution, shown in Fig.~\ref{fig:design} (d), we see that there is little electric field around the indium pump. Most of the electric field is concentrated between the parallel plate capacitor, and almost no electric field is distributed on the edge of the pads. As a result, we can ignore the corner effect.  In order to further determine the energy participation ratio of each component, including the dielectric layers and the interface layers, we calculated the fraction of the electric field energy stored in each component relative to the total charge energy,
\begin{equation*}
% 	p_{\mathrm{tot}}=\sum_{i}p_i \tan\delta_i, i \in \{ \mathrm{SA,SM,MA,MM} \}.
p_i\approx \int_{V_i} \frac{\epsilon_i \overrightarrow{E_i} \cdot \overrightarrow{E_i^*}}{U_{tot}} ,
\end{equation*}\label{Participation ratio}where $\epsilon_i$ is the dielectric constant of each component and $U_{tot}$ is the energy integral of all objects. The thickness of the interface layer is typical $\sim 3\,nm$, which is challenging to simulate~\cite{wang2015surface}. In addition, we assume a perfect metal film with zero thickness in the simulation and then calculate the corresponding interface electric field according to the boundary condition~\cite{gambetta2016investigating},
\begin{center}
\begin{equation*}
\begin{aligned}
\overrightarrow{E_s}_{\|}=\overrightarrow{E_j}_{\|},\\
\epsilon_s\overrightarrow{E_s}_{\perp}=\epsilon_j\overrightarrow{E_j}_{\perp}
\end{aligned}
\end{equation*}
\end{center}
where $E_s$ is the electric field of each interface component, and $E_j$ is the electric field of the corresponding adjacent dielectric component. For simulation, we set the dielectric constant of all metal oxides to $10$, the thickness of all interface layers to $3\,nm$~\cite{wenner2011surface}, and the vacuum gap between two capacitor pads to $5\,\mathrm{\mu m}$. Then, we obtained the energy participation ratio of all the components, shown in Table I. We can see that the participation ratio of the MA interface is significantly higher than that of the traditional planar transmon.%, and maybe that is why the coherence of flipmon is lower than the traditional planar transmon. We believe the participation ratios of all interfaces can be optimized by appropriately modifying the geometry of the flipmon (such as expanding the scale of flipmon, or enlarging the vacuum gap between carrier chip and flipped chip) in the future.



A full circuit arranged with four flipmons is shown in Fig.~\ref{fig:sample}. For each flipmon, a control line and a readout resonator are coupled to it capacitively. All the readout resonators then coupled to a Purcell filter. The Purcell filter is designed as a $\lambda/2-$type transmission line resonator with asymmetry coupling to the input and output lines through inter-digital capacitors. the coupling capacitor to the output side is much larger than that to the input side. Such a design guarantees that almost all the photons carrying signal goes into the output amplification chain and be measured. To eliminate cross-effects among the qubits, the flipmons are separated far enough without any direct coupling. Furthermore, we use the fixed-frequency qubits in all designs to avoid the influence of the noise from magnetic flux fluctuations. Indium bumps (small circles shown in Fig.~\ref{fig:design}) outside the flipmons are used to connect the ground planes of the chip and the carrier together, so as to suppress the cross-talk of the qubits, similar to the function of air-bridges.

%Furthermore, to avoid being affected by the noise from magnetic flux fluctuations, we use a single Josephson junction to realize a fixed qubit frequency.

%For convenience, we denoted them as "flipmons", which are shown in Fig.~\ref{fig:design}. Here after, we call the design schemes of Fig.~\ref{fig:design} (b, c) as flipmon A and flipmon B, respectively, and we denote the faced-down chip as "chip", and the faced-up one as "carrier". In flipmon A design, we only added a third pad on top of a typical transmon structure. The added pad can attract electric field and redistribute the capacitance of the transmon. As a result, flipmon A only introduced a new component of vacuum-gap capacitor. In flipmon B design, a indium bump was introduced to connect the two capacitor pads through a junction. Fig.~\ref{fig:design} (d, e) show the simulated electric field distribution and a microscopic photo of flipmon B, respectively, and Fig.~\ref{fig:design} (f) shows the scanning electron microscopic (SEM) image of the junction. The gap between the chip and the carrier was designed to $5\,\mathrm{\mu m}$, and the designed qubit parameters are shown in \red{table X}.





%A flip-chip-bonded device consists of two chips facing each other. For convenience, we name the face down one as chip, and the other one facing up as carrier. The key components of a transmon are Josephson junctions and shunt capacitors.
%The electrodes of the capacitor can be isolated islands in a floating transmon \cite{corcoles2015demonstration,braumuller2016concentric,versluis2017scalable}, or one of them can be connected galvanically to ground, as in grounded transmon \cite{barends2014superconducting}. Qubits with a single Jospehson junction is consistently better in $T_\mathrm{2}^*$ than those with two or more junctions forming a loop because of resilience to flux noise.
%In our design, we chose the floating scheme with a single junction.
% \cite{corcoles2015demonstration,braumuller2016concentric,versluis2017scalable,barends2014superconducting}.
%The shunting pads of our traditional transmon on the carrier are similar to Ref.~[\onlinecite{place2021new}].
% Large pads with large distance in between are applied to reduce dielectric loss, but also lead to large footprint.

%\Blue{By definition, for a flipmon, most of the electrical field are in the vacuum gap between chip and carrier. However, the Josephson junction is placed on the carrier, so are two electrodes of its shunt capacitor, just like a typical transmon (Fig.~\ref{fig:design}(a)). Therefore, we need a third electrode on the chip to reroute the electrical field lines. There are two choices.} One is to let the third electrode couple capacitively to the electrodes on the carrier. We denote it as type A (Fig.~\ref{fig:design}(b)). The other is to galvanically connect the third electrode to one of the electrodes on the carrier through an indium bump. This structure is called type B. Both types of flipmon were designed, but we mainly focus on type B here and leave the discussion of type A to Appendix \ref{sec:more-flipmon}, since its larger footprint actually brings no eminent gain in loss. Fig.~\ref{fig:design}(c) and (e) are a type B flipmon. On the carrier, a circular pad and a ring-shaped pad are concentric, and the qubit junction is in the gap between them. The circular pad is connected to another circular pad on the chip through a bump. The indium bump is in the center of both circular pads where electrical energy is lowest. Therefore, the surface oxidation on the sidewall of the indium bump makes less contribution to the overall dielectric loss than if we place the bump elsewhere.

%As the distance between chip and carrier can alter the capacitance matrix and thus the Hamiltonian of the system, we fix it to 5\,$\mathrm{\mu}$m as the design value and later control the flip-chip process to achieve it. Flipmon structure increases the flexibility as the third electrode on the chip can serve as a connector to other qubits or couplers. Qubit junctions and corresponding electrodes on the carrier make sure that bias lines, flux lines, transmission lines and readout resonators can all be placed on the carrier.

\begin{figure}[t]
% 	\includegraphics[width=.45\textwidth]{Flipmon_fig2_s1.pdf}
	\includegraphics{Flipmon_fig2_s1.pdf}
	\caption{\label{fig:fab1} Characterization of indium bump on tantalum. (a) The R-T curve of a test chip containing \ce{Ta} electrodes and 126 indium bumps with a diameter of 30\,$\mathrm{\mu}$m and height of $\sim 5\, \mu m$. The superconducting transitions at 4.26~K and 3.44~K are due to the tantalum in $\alpha$ phase, and the indium, respectively. (b) The I-V curve of the same chip shows that the critical current of all zero-resistance indium bumps exceeds 10~mA. This is much larger than the critical current of the Josephson junction. The inset is the optical micrograph of a test chip.}
\end{figure}


% A typical transmon chip has 5 qubits in a row, each with a separate readout resonator and sharing a transmission line for readout and drive pulses. While a flipmon chip has 4 qubits in a row, with resonators and transmission lines similar to typical transmon, but also sharing a Purcell filter to reduce radiation loss. The current coherence is too low to be limited by Purcell effect\cite{purcell1995spontaneous}, so the additional filter does not make a difference.

%We believe the loss of our qubits design mainly comes from interfaces.
% Microwave loss can be ignored in the air and commonly-used substrates like silicon or sapphire, with loss tangent of $\sim10^{-6}-10^{-7}$\cite{o2008microwave,melville2020comparison}, much smaller than interfaces.
%There are four interfaces that matter: substrate-metal (SM), substrate-air (SA), metal-air (MA) and metal-metal (MM).  \red{Surface treatment before and after metallization can reduce the loss tangent of interfaces, but the total loss tangent of the qubit is
%\begin{equation*}
%	p_{\mathrm{tot}}=\sum_{i}p_i \tan\delta_i, i \in \{ \mathrm{SA,SM,MA,MM} \}.
%\end{equation*}}
%Therefore, it not only depends on loss tangent $\tan\delta_i$ of each interface, but also participation ratio $p_i$. The participation ratio of an interface is the percentage of electrical energy stored in it. Participation ratios can be simulated by adding dielectric layers to explicitly represent the interfaces and calculating the integral of electric energy over their surfaces \cite{wang2015surface,gambetta2016investigating}, or they can be calculated analytically \cite{murray2018analytical,martinis2021optimal}. With different designs, loss tangent of interfaces can also be fitted \cite{woods2019determining}. We resorted to a more qualitative approach. Energy density was calculated in energy-participation ratio simulation \cite{minev2020energy} during the extraction of system Hamiltonian, treating metal planes as perfect electrical conductor boundaries with no dielectric layers. The obtained distribution of electrical field could approximate the distribution on a real scenario. We then calculated participation ratio according to Eqs.~(1)--(4) from Ref.~[\onlinecite{gambetta2016investigating}].

% The distance between the capacitor pads of a flipmon is much smaller than that of a typical floating transmon. As a result, the area of the pads can be much reduced. In our design, the pads' area is $800\times 600\,\mathrm{\mu m}^2 $ for a typical transmon, approximately five times as that of the type B flipmon, which is $300\times 300\,\mathrm{\mu m}^2 $. Smaller footprint means more concentrated electric field, thus the participation ratio of the MA interface becomes larger. Furthermore, there is a more profound effect from the change of structure that promotes vacuum-gap capacitor. In particular, the electric field of a typical transmon in the MA interface only consists a small part of total electric energy. On the other hand,

%Fig.~\ref{fig:design}(d) illustrates that flipmon has a large part of electric energy in MA interface. As it turns out, in Type B flipmon, $53.2\%$ of field energy is in the air, and participation ratios are $p_{\mathrm{MS,chip}}=1.31\times 10^{-5},p_{\mathrm{MA,chip}}=3.32\times 10^{-5},p_{\mathrm{SA,chip}}=1.12\times 10^{-5},p_{\mathrm{MS,carrier}}=3.86\times 10^{-5},p_{\mathrm{MA,carrier}}=2.08\times 10^{-5},p_{\mathrm{SA,carrier}}=1.20\times 10^{-4}$. The value of $p_{\mathrm{MM}}$ cannot be extracted with the current method. Flip-chip process also contributes to new channels of loss. For a typical transmon, the only MM interface is tantalum-aluminium between shunt capacitor electrodes and qubit junction. But for flipmon, there are also interfaces on both ends of an indium bump between itself and the electrodes in qubit area or ground planes otherwise. With no diffusion and thus no intermetallic metallization, we only need to pay attention to tantalum-indium interfaces. Furthermore, there is a new MA interface on the surface of indium bumps. These extra MM and MA interfaces may have an influence on qubit coherence as well. The fresh surface of metal films can be oxidized or absorb contaminants when exposed to air. Tantalum has pure oxide \ce{Ta2O5} that grows slowly. Also, the interfaces between tantalum and substrate or other metals can be of high quality if carefully treated \cite{place2021new,yu2021tatransmon}. As the rest of the work will imply, thanks to the outstanding properties of tantalum, the increase in participation ratio is in part compensated by the decrease in loss tangent.
\begin{figure}[ht]
% 	\includegraphics[width=.45\textwidth]{Flipmon_fig3_s2.pdf}
	\includegraphics{Flipmon_fig3_s2.pdf}
	\caption{\label{fig:gap} The vacuum gap between the chip and the carrier is determined by the height of the indium bumps. An SEM image of one of the samples is shown in the inset. The average vacuum gap of the nine samples is about $5~\mu m$ with deviations less than $0.5~\mu m$. This stable flip-chip process is attributed to the stop bumps.}
\end{figure}


\section{\label{sec:fab}Flip-chip interconnection and Device Fabrication}
%We have mentioned the two challenges in flipmon fabrications in the last section.
In order to obtain high-quality flip-chip capacitors, first of all, we must get high-quality indium bump flip-chip interconnects with good superconductivity and mechanical properties. We use tantalum instead of aluminum as the superconducting film. First, a significant improvement of coherence time has been achieved. $T_1$ of 0.36\,ms \cite{place2021new} and 0.5\,ms \cite{yu2021tatransmon} has been measured on traditional transmon by resorting to tantalum as the electrodes. Second, aluminum can form a lossy intermetallic layer with indium, while tantalum is a hard metal, and it cannot form a stable phase of TaIn alloy at low temperature\cite{atomly}. As a result, we did not add any under bump metallization (UBM) layer during the indium bump fabrication process. Indium bumps were directly grown on the tantalum film at speed exceeding 2 nm/sec through a thermal evaporation system. Before evaporation, the surface of the films was cleaned by an ion milling process to increase adhesion~\cite{schulte2012characterization,kim2008effect} and to ensure that indium and tantalum formed a superconducting connection. Approximately 5 $\mathrm{\mu m}$'s indium bumps with a diameter of 20 to 50 $\mathrm{\mu m}$ were grown on both the chip and the carrier. After being diced into smaller units, the chip and the carrier were then aligned and bonded together by a flip-chip bonder with a bonding force ranging from 5-10~kN. Before the bonding process, an important plasma cleaning process was performed to remove the oxides on the surface of the bumps.

In order to test the superconducting performance and yield rate of the indium bump connection, we first fabricated a series of test devices, which were designed as two braided wires, and the indium bumps acted as cross-connections~\cite{Foxen_2017}, shown in Fig.~\ref{fig:fab1} (b) insertion. Several pads were distributed at different points of the weaving to locate any connection failures. Tantalum films were first grown on two wafers by ultra-high vacuum DC magnetron sputtering, with the thickness of which was around 120 nm. Then the weaves, which were overlapped complementary arrays of bars, were patterned on the chip and the carrier by ultraviolet (UV) laser direct writing lithography (DWL). The Pads were also patterned for the current and voltage characteristics (I-V) measurement and arranged around the bar arrays on the carrier chip. After developing, the film were etched by reactive ion etching (RIE) with $\mathrm{SF_6}$. Then a total of 400 circular indium bumps were patterned on both sides of the overlap area by the DWL. After developing again, the indium was deposited by thermal evaporation. After being immersed in an acetone bath for several hours, the photo-resist was stripped and an array of bumps appeared.

The measured resistance-temperature (R-T) curves and the I-V characteristics of a test device in a physical property measurement system (PPMS) are shown in Fig.~\ref{fig:fab1} (a, b). Two obvious transitions are found in the R-T curves, corresponding to the superconducting transitions of tantalum ($\sim~4.3$~K) in $\alpha$ phase and indium ($\sim~3.4$~K), respectively. The I-V curve shows that the critical current of all bumps exceeds 10~mA. We do not reach the critical current due to the limitation of the current source. No failures are found in all batches of our test devices, which indicates that the yield of the indium bump fabrication process is almost 100\%.


The height of the indium bump directly determines the shunted capacitance of the flipmons. To precisely control the height of the indium bumps, we add some large indium bumps on the carrier as the stop bumps, as shown in Fig.~\ref{fig:measure} as large rectangle blocks. The stop bumps are fabricated simultaneously with the connection indium bumps. When the chip and the carrier are bonded together, the connecting bumps will first contact face-to-face and deform under the pressure until the stop bumps finally contact with the surface of the chip. Since the bonding force is constant, the pressure on each bump will drop sharply. The bumps deformation is almost ceased, and therefore the gap distance between two chips is determined by the height of  the stop bumps. The stop bumps can also help balance the force distribution on different chip areas so that the gap distance across the entire device is quite uniform. Fig.~\ref{fig:gap} shows the measured gap distances distribution of all nine samples and one of the relevant SEM images is shown in the inset. The gap distance can be well controlled in $5\pm 0.4\mathrm{\mu m}$. Since the capacitance of a parallel plate capacitor is inversely proportional to the gap distance, the distance fluctuation only causes less than 3\%'s variation of the qubit charging energy $E_c$. Then we extend the above fabrication process to our flipmon samples except for some extra steps to improve the quality. More detailed fabrication recipe of flipmon is shown in~Appendix.\ref{sec:fabrication}.

%The flipmon qubits were fabricated with the same processes and recipes discribed in Ref. \red{X}, except that the flip-chip processes described above are followed after the fabrication of junctions. Tantalum was used as the base superconductor, and "cross type" junctions were formed by standard double-angle evaporation and oxidization processes.
%\red{The fabrication recipe of a typical transmon sample is described in Ref.\ \cite{yu2021tatransmon}.} Typical transmon samples are on a different wafer from flipmon carrier that did not go through indium patterning.  After growing tantalum circuitry and junctions, the two wafers for flipmon carrier and chip were diced and each carrier is flip-chip bonded to its chip separately.

%Before we make the flipmon, it is necessary to first evaluate the flip-chip interconnects. We fabricated a series of test devices to study the galvanic properties of the flip-chip indium-indium connections. The test patterns, similar to Ref.~[\onlinecite{foxen2017qubit}], is shown in the inset of Fig.~\ref{fig:fab1}(b). They were composed of overlapped complementary arrays of bars forming two weaved wires, with indium bumps acting as interconnections. Several pads were distributed along the wire for transport measurement. Tantalum films were first sputtered on both sapphire substrates. The weaved wires were patterned on them by ultraviolet (UV) laser direct writer (LDW), followed by inductively coupled etching (ICP) with \ce{SF6} and \ce{CHF3} as etching gas. Photo-resist was stripped in N-Methylpyrrolidone (NMP) bath at 50\textcelsius{} for three hours. Bump patterns were then defined by another step of LDW and the bumps were formed by thermal evaporation of indium film of about 8\,$\mathrm{\mu}$m thick. An in-situ argon ion beam etching (IBE) is necessary for the removal of surface oxidation before the deposition of indium. We used acetone for the lift-off of indium. Right before the flip-chip bonding process, both the chip and the carrier were treated for five minutes in a microwave plasma oven with reducing gas atmosphere. Finally, they were aligned and bonded together using a flip-chip bonder, with a bonding force of 5-10 kilograms.

%We measured the test chips in a Physical Property Measurement System (PPMS) with varied temperatures down to 2 Kelvins while monitoring the resistance of the bumped wire between a pair of pads by 4-probe method using repeated 500-\,$\mathrm{\mu}$A current pulse. Fig.~\ref{fig:fab1}(a) shows the resistance-temperature curve of a segment containing \ce{Ta} electrodes and 126 In bumps with diameter of 30\,$\mathrm{\mu}$m. Two superconducting transitions at 4.26\,K and 3.44\,K can be identified, corresponding to tantalum in the $\alpha$ phase and indium, respectively.

%Surface treatments, including those of tantalum and indium, are vital to tantalum-indium and indium-indium interfaces that can provide reliable adhesion \cite{schulte2012characterization,kim2008effect} in addition to holding adequate superconducting currents without heating the device. For the above sample, we swept the current and probe the voltage at 2\,K. As shown in Fig.~\ref{fig:fab1}(b), the voltage on the 126 bumps remained almost zero for current up to 10\,mA, which was the largest accessible current of the equipment. This is a solid proof of critical current exceeding 10\,mA and the robustness of our flip-chip process. Such current-bearing ability is sufficient since the largest direct current needed on a superconducting quantum chip is for flux tuning and a whole modulation period can usually be seen within 2\,mA bias.



%The above fabrication process was then extended to our flipmon samples except for some extra steps to improve the quality. Special attention was paid to the removal of residual resist, which can otherwise be a primary source of dielectric loss \cite{niepce2020geometric}. Prior to the deposition of tantalum, the sapphire wafer was immersed in several inorganic solutions and annealed to keep the surface clean and flat. Oxygen plasma ashing was performed both before and after ICP etching, for five and ten minutes respectively, in order to remove residual resist after development and the resist-like chemicals deposited during etching. The "T-shape" \cite{kelly2015fault} Josephson junctions (Fig.~\ref{fig:design}(f)) were patterned by electron beam lithography (EBL) on the carrier wafer. PMMA A4/ LOR 10B bilayer resist constructed the undercut and Dolan bridge structure. After two-minute oxygen plasma ashing, in-situ IBE was performed for the removal of surface oxide of tantalum and then the junction was formed by double-angle evaporation of aluminium. Indium bumps with a height of 5\,$\mathrm{\mu}$m were deposited to fit our designed chip distance. After lift-off of indium in acetone, samples were immersed in NMP bath at 80\textcelsius{}, which may be beneficial to the removal of residual resist around the junction region. A UV ozone treatment at 80\textcelsius{} for ten minutes helps to thoroughly remove residual resist insoluble in previous process. The reducing gas treatment was extended to ten minutes in order to restore fresh surfaces of tantalum and indium. Flip-chip bonding, wire bonding and the transfer to fridge were performed as fast as possible to restrict reoxidation of metal surfaces.

\section{\label{sec:meas} Measurement results}

In order to determine the stability of the fabrication process and the performance of the qubits, we have measured three flipmon samples with the same design. Fig.~\ref{fig:sample} shows an optical micrographic photo of a sample. Sample $\#1$ is wire-bonded in a multi-port sample holder. Sample $\#2$ and $\#3$ are wire-bonded in a two-port sample holder without using the XY control lines.


%The frequency of readout resonators are around 6.9 GHz, and qubit frequency is about 4.5-5 GHz. The coherence properties and anharmonicities of three flipmon samples are summarized in Fig.~\ref{fig:measure}. Sample $\#1$ showed a smaller coherence time owing to the use of inferior sample holder. Detailed measurement results of both types of flipmon and typical transmon can be found in Appendix \ref{sec:more-flipmon}.



\begin{figure}[ht]
% 	\includegraphics[width=.45\textwidth]{Flipmon_fig4_s1.pdf}
	\includegraphics{Flipmon_fig4_s1.pdf}
	\caption{\label{fig:measure} The average of qubit characteristic times $T_1$, $T_\mathrm{2}^*$, $T_\mathrm{2E}$ and anharmonicity $\eta$ from three flipmon samples. Error bars are the fluctuations between different qubits in the same sample.}
\end{figure}



Each sample is well-packaged in a $\mu$-metal shield. We add adequate attenuation of 78~dB to the input transmission line to fully reduce the input thermal noise. The frequencies of the readout resonators are around 6.9~GHz, and the qubit frequencies are in the range of 4.5-5~GHz. The coherence characteristics of the three flipmon samples were measured, including energy relaxation time, decoherence time, and spin-echo time. The results are shown in Fig.~\ref{fig:measure}. Sample \#1 showed a lower coherence time, which may due to the poor packaging. Appendix \ref{sec:more-flipmon} provides detailed qubit characteristic measurement results of three qubit designs in Fig.~\ref{fig:design} (a), (b), (c).

The average energy relaxation time of flipmon sample \#2 and sample \#3 is in the range of 30-60 us. The sample designed as Fig.~\ref{fig:design} (b) are fabricated with the same recipe as flipmon and has almost the same energy relaxation time as flipmon, shown in Table~\ref{tab:barbell} and Table~\ref{tab:pplate}. This means that the indium bumps have little impact on the qubit relaxation time. However, in addition to the additional flip-chip process, the traditional transmons have a higher energy relaxation time in the range of 120-140 us using the same fabrication recipe as the flipmon. We believe the additional flip-chip process may not increase the loss tangent of the MS and SA interfaces. Therefore, the MA interfaces are most likely the dominant resource of flipmon energy relaxation due to the larger electric field participation ratio and the induced extra loss of the MA interface from the flip-chip process.

%So the MA interfaces are most likely the dominant resource of qubit energy relaxation. On one hand, the flipmons have larger electric field participation ratio than traditional transmon. On the other hand, the extra flip-chip process may increase the the loss tangent of MA interface.

The sufficient anharmonicity $\eta$ is necessary to maintain transmon as a two-level state. Under the perturbation approximation, the anharmonicity is equal to the charging energy $E_c$ determined by the large shunt capacitor. Thus, we can measure the anharmonicity to infer the charging energy. We first pulse the qubit into the first excited state with a $\pi$ pulse and then scan the frequency between the first excited state and the second excited state. Then, we calculated the anharmonicity of three flipmon samples as shown in Fig.~\ref{fig:measure}. The almost same anharmonicity indicates that the vacuum gap distance determined by the indium bump height can be well-controlled due to the use of stop bumps.

%This fact is corroborated by the distribution and SEM images of the distance between carrier and chip in Fig.~\ref{fig:gap}.


% \begin{figure}
% 	\includegraphics[width=.45\textwidth]{flipmon_2d_schematic.eps}% Here is how to import EPS art
% 	\caption{\label{fig:2d} Schematics of a 2D square lattice of qubits with tunable couplers. Wires in black and red are on the carrier and the chip, respectively.}
% \end{figure}

\section{\label{sec:conclu}Conclusion}
In conclusion, we demonstrate a new design of transmon, denoted as flipmon, with the help of flip-chip technology. Good coherence properties that comparable to that of traditional transmons are obtained. The results show that our flip-chip bumps are of high quality and have little impact on the qubit energy relaxation. We have also developed a technology that uses the stop indium bumps to precisely control the gap distance between the chip and the carrier. The simulation of the electric field and the experimental results indicate the limitation of the energy relaxation time of the flipmons comes mainly from the MA interfaces. As a result, through the use of more careful surface treatment strategies~\cite{nersisyan2019manufacturing,tsioutsios2020free,mergenthaler2021ultrahigh}, the coherence of the flipmons can be further improved. Flipmons are naturally compatible with flip-chip technology, which is promising for the 3D wiring of qubit arrays. They are also compact and can provide flexible connectivity. We believe that flipmons and related quantum circuit architectures may be a good candidate for scaling quantum processors to the order of 1000 qubits or more.

%the surface quality, as a new design of qubit that inherits both the flexibility in wiring from flip-chip technology and competitive coherence time compared to commonly-used planar transmon from base metal of tantalum. Our work emphasized the significance of the combination of design, material, and fabrication process in developing scalable superconducting qubit. Similarly, with improved interfaces, especially MA, such long coherence time can also be expected in flipmon. For example, we can process interfaces more thoroughly \cite{nersisyan2019manufacturing} or use silicon shadow masks instead of photoresist \cite{tsioutsios2020free}, revise etching recipe to control the sidewall profile of etched metal, and use ultrahigh vacuum packaging strategies \cite{mergenthaler2021ultrahigh} to prevent contamination when exposed to air. Apart from tantalum, \ce{TiN} also has good MA interface \cite{melville2020comparison} and can be another candidate as the base metal of flipmon. Other kinds of metal nitride can also be explored. Moreover, it can be envisioned that if we keep the whole process in an environment with no oxygen or nitrogen like some foundries, there will be neither SA nor MA interfaces at all. Thus only the optimization of SM matters. Approaches like molecular beam epitaxy (MBE) can be used for this purpose.

%The footprint of flipmon can shrink further if we pull the chip and the carrier closer together, as long as the coherence of qubit does not deteriorate severely due to increased participation ratio of interfaces. For example, if their distance is 1\,$\mathrm{\mu}$m instead of 5\,$\mathrm{\mu}$m, the size of a qubit can be approximately 5 times smaller than the current value.

%The flexibility of flipmon not only allows wiring in two layers, but also coupling among several chips bonded to the same carrier  \cite{gold2021entanglement}. As soon as through-silicon vias are available, we can assembly superconducting quantum computing chips as superconducting multichip modules \cite{das2018cryogenic}, similar to playing with Legos. This work was merely a small step towards scalable superconducting qubits, and the potentials of flipmon are far from been fully explored yet. We are confident that flipmon can bring merits to the goal of scaling up to a thousand of qubits.
\paragraph{}
\
\begin{acknowledgments}
This work was supported by the NSF of Beijing (Grant No. Z190012), the NSFC of China (Grants No. 11890704, No. 12004042,No. 11905100), National Key Research and Development Program of China (Grant No. 2016YFA0301800), and the Key-Area Research and Development Program of Guang Dong Province (Grant No. 2018B030326001).
\end{acknowledgments}

%\appendix
\appendix
% \section{Sample images}
% Fig.~\ref{fig:sample} shows a microscopic image taken for the carrier of a type B sample before flip-chip bonding.
% \begin{figure*}
% 	\includegraphics[width=.5\textwidth]{flipmon_sample_s3.png}
% 	%	\includegraphics[width=.9\textwidth]{FlipmonSample_fig1_v1}
% 	\caption{\label{fig:sample} Top view of a carrier containing 4 type B qubits (green) with readout resonators (purple), 3 type B qubits for 4-probe resistance measurement, Purcell filter (red), transmission line (yellow) and lines for qubit drive (cyan). Small rounds are indium bumps connecting carrier and chip, while large rectangular bumps are position-stopping ones.}
% \end{figure*}

% \section{DC measurement results of flip-chip test samples with \ce{Nb} as base metal}
% Apart from tantalum, we also fabricated and measured test chips with niobium as base metal. Under bump metallization is omitted as niobium, just like tantalum, does not need a barrier to avoid the diffusion of indium. Critical temperatures of niobium and indium and resistance of arrays of bumps with current till 10\,mA are shown in Fig.~\ref{fig:nb}. Samples with different bump diameters show similar behaviours, proving the robustness of flip-chip process on \ce{Nb}.
% \begin{figure}
% 	\includegraphics[width=.45\textwidth]{TongDuan_Nb}
% 	\caption{\label{fig:nb}Characterization of indium bump on niobium.(a) Resistance-temperature curve of a segment containing Nb electrodes and 126 In bumps with diameter of 50\,$\mathrm{\mu}$m. Two superconducting transitions at 9.20\,K and 3.22\,K can be identified, corresponding to niobium ($\mathrm{T_{c}}=9.25$\,K) and indium ($\mathrm{T_{c}}=3.40$\,K), respectively.  (b) I-V curve of similar segments with bump diameters of 20--50\,$\mathrm{\mu}$m, all showing cricital current above 10\,mA.}
% \end{figure}

% \section{Control of flip-chip gap through position-stopping bumps}
% \label{sec:sem}
% The gap between carrier and chip is difficult to control. Position-stopping bumps are large in size and placed only on the carrier, so that the bonder can no longer compress any bump when chip touches position-stopping bumps. Thus the gap equals the thickness of bumps on the carrier, which is well-controlled by the weight of evaporated indium. Fig.~\ref{fig:sem} are some SEM images of bonded samples, all the gaps are quite close to the target value of 5\,$\mathrm{\mu}$m.
% \begin{figure}
% 	\includegraphics[width=.45\textwidth]{WaferDistance}
% 	\caption{\label{fig:sem}SEM images of the side view of a few flip-chip bonded samples. The gap between carrier and chip are around 5\,$\mathrm{\mu}$m, as intended.}
% \end{figure}

\section{More Details of the samples}
\label{sec:sample image}
Fig.~\ref{fig:sample} shows an optical micrograph taken for the carrier chip of a flipmon sample before flip-chip bonding.
\begin{figure*}
	\includegraphics[width=.7\textwidth]{flipmon_sample_s3.png}
	%	\includegraphics[width=.9\textwidth]{FlipmonSample_fig1_v1}
	\caption{\label{fig:sample} Top view of a carrier containing four flipmons (green) with individual readout resonators (purple), three test flipmons for Josephson junction resistance measurement, one Purcell filter (red) embedded in one transmission line (yellow) and four XY control lines(cyan). Small circles are indium bumps used for connecting the carrier and the chip, while large rectangular bumps are the stop bumps.}
\end{figure*}
\label{sec:more-flipmon}
The measurement results of a traditional transmon sample are listed in Table~\ref{tab:transmon}. The qubit parameters of the sample designed as Fig.~\ref{fig:design} (b) are listed in Table~\ref{tab:pplate}. The parameters of each flipmon in Fig.~\ref{fig:measure} are listed in Table~\ref{tab:barbell}.
\begin{table*}[b]%The best place to locate the table environment is directly after its first reference in text
	\caption{\label{tab:transmon}%
		Measured parameters of a traditional transmon sample.
	}
	\begin{ruledtabular}
		\begin{tabular}{lccccc}
			\textrm{ }&
			\textrm{Qubit1}&
			\textrm{Qubit2}&
			\textrm{Qubit3}&
			\textrm{Qubit4}&
			\textrm{Qubit5}\\
			\colrule
			Readout resonator frequency $f_\mathrm{r}$\,(GHz) & 7.101 & 7.133 & 7.166 & 7.198 & 7.226\\
			Transmon qubit frequency $f_\mathrm{q}$\,(GHz) & 4.450 & 4.559 & 4.412 & 4.418 & 4.502\\
			Anharmonicity $\eta/2\pi$\,(MHz) & --- & --- & 266 & --- & --- \\
			Dispersive shift $\chi/2\pi$\,(MHz) & 0.64 &0.71 & 0.65 & 0.63 & 0.65\\
			Relaxation time $T_1$\,($\mathrm{\mu}$s) & 158.3 & 109.2 & 136.3 &120.9 &131.2\\
			Ramsey dephasing time $T_\mathrm{2}^*$\,($\mathrm{\mu}$s) & 75.5 & 62.3 &42.0 & 52.1 &65.2\\
			Spin echo dephasing time $T_\mathrm{2E}$\,($\mathrm{\mu}$s) & 194.2 & 131.3 & 176.0 &173.8 &168.9\\			
		\end{tabular}
	\end{ruledtabular}
\end{table*}


%For type A flipmon, there are two capacitors in parallel, so the area taken by its shunt capacitor, $500\times 500\,\mathrm{\mu m}^2 $, is approximately twice as type B. The ratio is a bit larger because the distance between the two electrodes on the carrier are larger for type A. We fabricated type A and type B in separate chips on the same wafers. Presumably, its coherence should be a little longer than type B because of larger . As a matter of fact, as Table~\ref{tab:pplate} shows, type A had the same level of coherence time as type B, indicating that its larger footprint did not bring about good coherence time. The coherence time vary across different number of qubits in an array, different sample boxes and different fridges, but all in the range of tens of microseconds. We can conclude that the fabrication process of flipmon is robust and flipmon coherence is comparable to contemporary transmon qubits.
\begin{table*}[b]%The best place to locate the table environment is directly after its first reference in text
	\caption{\label{tab:pplate}%
		Measured qubit parameters of the sample designed as Fig.~\ref{fig:design} (b).
	}
	\begin{ruledtabular}
		\begin{tabular}{lcccc}
			\textrm{ }&
			\textrm{\#1Q1}&
			\textrm{\#1Q2}&
			\textrm{\#2Q1}&
			\textrm{\#3Q1}\\
			\colrule
			$f_\mathrm{r}$\,(GHz) & 6.625 & 6.725 & 6.917 & 7.190 \\
			$f_\mathrm{q}$\,(GHz) & 5.228 & 5.128 & 4.973 & 5.161 \\
			$\eta/2\pi$\,(MHz) & --- & 237 & 242 & ---\\
			$T_1$\,($\mathrm{\mu}$s) & 36.5 & 56.9 & 27.1 & 49.8 \\
			$T_\mathrm{2}^*$\,($\mathrm{\mu}$s) & 32.0 & 65.2 &29.9& 75.0 \\
			$T_\mathrm{2E}$\,($\mathrm{\mu}$s) & 73.0 & 95.1 & 30.2 & --- \\			
		\end{tabular}
	\end{ruledtabular}
\end{table*}

\begin{table*}[b]%The best place to locate the table environment is directly after its first reference in text
	\caption{\label{tab:barbell}%
		Measured parameters of three flipmon samples. \#$i$Q$j$ is the $j$-th qubit on sample \#$i$.
	}
	\begin{ruledtabular}
		\begin{tabular}{lcccccccccccc}
			\textrm{ }&
			\textrm{\#1Q1}&\textrm{\#1Q2}&\textrm{\#2Q1}&
			\textrm{\#2Q2}&\textrm{\#2Q3}&\textrm{\#2Q4}&
			\textrm{\#3Q1}&\textrm{\#3Q2}&\textrm{\#3Q3}&
			\textrm{\#3Q4}&\textrm{\#3Q5}&\textrm{\#3Q6}\\
			\colrule
			$f_\mathrm{r}$\,(GHz) & 6.856 & 6.965 & 6.66 & 6.769 & 6.867 & 6.953 & 6.647 & 6.694 & 6.74 & 6.815 & 6.869 & 6.917\\
			$f_\mathrm{q}$\,(GHz) & 4.8 & 4.853 & 4.642 & 4.823 & 4.72 & 4.608 & 4.965 & 4.732 & 4.986 & 4.978 & 4.857 & 4.951\\
			$\eta/2\pi$\,(MHz) & 222.1 & 219.9 & 247 & 241.3 & 231.4 & 223 & 226.6 & 228.6 & 224.2 & 226.4 & 225 & 226.1\\
			$\chi/2\pi$\,(MHz) & --- & 0.32 & 0.47 & 0.44 & 0.41 & 0.3 & 0.5 & 0.4 & 0.8 & 0.6 & 0.6 & 0.4\\
			$T_1$\,($\mathrm{\mu}$s) & 27.7 & 31.3 & 43.7 & 39 & 53 & 35 & 33.8 & 32.6 & 47.5 & 30.1 & 44.4 & 42.7\\
			$T_\mathrm{2}^*$\,($\mathrm{\mu}$s) & 19.5 & 26.1 & 48.7 & 32.6 & 33.9 & 42.7 & 56.4 & 32.4 & 32.6 & 15.4 & 36.9 & 38.9\\
			$T_\mathrm{2E}$\,($\mathrm{\mu}$s) & 36.8 & 42.5 & 53.3 & 43.5 & 54.8 & 46.4 & 67.6 & 37.4 & 45.7 & 35.5 & 46 & 44.8\\			
		\end{tabular}
	\end{ruledtabular}
\end{table*}

\section{Details about the flipmon fabrication process}
\label{sec:fabrication}
In the fabrication process of flipmon, we need to keep in mind to improve the quality of the metal film. Special attention is paid to the removal of residual resist, which can otherwise be a primary source of dielectric loss \cite{niepce2020geometric}. Before the deposition of tantalum, the sapphire wafer was immersed in several inorganic solutions and annealed to keep the surface clean and flat. Then the Ta film with 120 nm was prepared by the dc magnetron sputtering. Next, all elements, except for the Josephson junctions, were patterned on the chip and the carrier by ultraviolet (UV) laser direct writing lithography (DWL). The inductively coupled plasma (ICP) etching with \ce{SF6} as etching gas was used to etch the tantalum. Notice, the oxygen plasma ashing was performed both before and after the ICP etching, for five and ten minutes, respectively, to remove the residual resist after the development and the resist-like chemicals deposited during etching.

The "T-shape" Josephson junctions (Fig.~\ref{fig:design}(f)) were patterned by electron beam lithography (EBL) on the carrier. The undercut structure and the Dolan bridge structure was constructed with PMMA A4/ LOR 10B bilayer resist. After two-minute oxygen plasma ashing, in-situ argon ion beam etching (IBE) was performed to remove the surface oxides of tantalum. Then the Josephson junction was formed by double-angle evaporation of aluminum. Next, the second DWL was used to pattern the indium bumps with a height of 5\,$\mathrm{\mu}$m to fit the designed vacuum gap distance. After the development again, we used the thermal evaporation to deposit the indium. Then the lift-off process was performed in the acetone, and followed by the immersion in N-Methylpyrrolidone (NMP) bath at 80$^{\circ}$C, which may be beneficial to the removal of residual resist around the junction region. A UV ozone treatment at 80$^{\circ}$C for ten minutes was helpful to remove the residual resist insoluble in the previous process thoroughly. The reducing gas treatment was extended to ten minutes to restore fresh surfaces of tantalum and indium. After quickly flip-chip bonding and wire bonding, we transferred the samples to the fridge as fast as possible to reduce the re-oxidation of metal surfaces.


\section{layout of flipmon 2D array}
In order to demonstrate scalability, flipmon qubits are arranged in a two-dimensional square lattice, as shown in Fig.~\ref{fig:layout}. There is a $3\times3$ qubits array of flipmon with 12 tunable couplers. In the layout, qubits are a grounded version of flipmon, with the ring-shaped pad removed to avoid cutting up the ground plane. Tunable couplers are also the flipmon with rectangular pads, forming asymmetric floating couplers \cite{sete2021floating}. Readout resonators, transmission lines, control lines for direct current bias, and microwave driving are all on the carrier substrate. Without discontinuity caused by indium bumps on these lines, they may have better impedance matches and less loss or crosstalk. An array of indium bumps on both sides of coplanar lines balances the electric potential and suppresses spurious modes. Large rectangular bumps are stop bumps. Circular bumps on the capacitive shunt of qubits and couplers ensure that wiring between them can be put on the chip, separated in space with coplanar lines on the carrier. This may serve as an example of the flexibility in the wiring with the flipmon structure.

\label{sec:layout}
\begin{figure}
	\includegraphics[width=.45\textwidth]{flipmon_2d_schematic.png}% Here is how to import EPS art
		\caption{\label{fig:layout}Layout of a 2D square lattice of qubits including tunable couplers with the design of flipmon. The layer in magenta and plum color are metal to be etched on the carrier and the chip, respectively. The layer in blue is for indium bumps.}
\end{figure}


% The \nocite command causes all entries in a bibliography to be printed out
% whether or not they are actually referenced in the text. This is appropriate
% for the sample file to show the different styles of references, but authors
% most likely will not want to use it.
%\nocite{*}
% \bibliographystyle{Zou}
%\ibliography{bib_flipmon}% Produces the bibliography via BibTeX.
%merlin.mbs apsrev4-1.bst 2010-07-25 4.21a (PWD, AO, DPC) hacked
%Control: key (0)
%Control: author (72) initials jnrlst
%Control: editor formatted (1) identically to author
%Control: production of article title (0) allowed
%Control: page (0) single
%Control: year (1) truncated
%Control: production of eprint (-1) disabled
\providecommand{\noopsort}[1]{}\providecommand{\singleletter}[1]{#1}%
\begin{thebibliography}{31}%
\makeatletter
\providecommand \@ifxundefined [1]{%
 \@ifx{#1\undefined}
}%
\providecommand \@ifnum [1]{%
 \ifnum #1\expandafter \@firstoftwo
 \else \expandafter \@secondoftwo
 \fi
}%
\providecommand \@ifx [1]{%
 \ifx #1\expandafter \@firstoftwo
 \else \expandafter \@secondoftwo
 \fi
}%
\providecommand \natexlab [1]{#1}%
\providecommand \enquote  [1]{``#1''}%
\providecommand \bibnamefont  [1]{#1}%
\providecommand \bibfnamefont [1]{#1}%
\providecommand \citenamefont [1]{#1}%
\providecommand \href@noop [0]{\@secondoftwo}%
\providecommand \href [0]{\begingroup \@sanitize@url \@href}%
\providecommand \@href[1]{\@@startlink{#1}\@@href}%
\providecommand \@@href[1]{\endgroup#1\@@endlink}%
\providecommand \@sanitize@url [0]{\catcode `\\12\catcode `\$12\catcode
  `\&12\catcode `\#12\catcode `\^12\catcode `\_12\catcode `\%12\relax}%
\providecommand \@@startlink[1]{}%
\providecommand \@@endlink[0]{}%
\providecommand \url  [0]{\begingroup\@sanitize@url \@url }%
\providecommand \@url [1]{\endgroup\@href {#1}{\urlprefix }}%
\providecommand \urlprefix  [0]{URL }%
\providecommand \Eprint [0]{\href }%
\providecommand \doibase [0]{http://dx.doi.org/}%
\providecommand \selectlanguage [0]{\@gobble}%
\providecommand \bibinfo  [0]{\@secondoftwo}%
\providecommand \bibfield  [0]{\@secondoftwo}%
\providecommand \translation [1]{[#1]}%
\providecommand \BibitemOpen [0]{}%
\providecommand \bibitemStop [0]{}%
\providecommand \bibitemNoStop [0]{.\EOS\space}%
\providecommand \EOS [0]{\spacefactor3000\relax}%
\providecommand \BibitemShut  [1]{\csname bibitem#1\endcsname}%
\let\auto@bib@innerbib\@empty
%</preamble>
\bibitem [{\citenamefont {Arute}\ \emph {et~al.}(2019)\citenamefont {Arute},
  \citenamefont {Arys}, \citenamefont {Babbush}, \citenamefont {Bacon},\ and\
  \citenamefont {et~al.}}]{Google2019}%
  \BibitemOpen
  \bibfield  {author} {\bibinfo {author} {\bibfnamefont {F.}~\bibnamefont
  {Arute}}, \bibinfo {author} {\bibfnamefont {K.}~\bibnamefont {Arys}},
  \bibinfo {author} {\bibfnamefont {R.}~\bibnamefont {Babbush}}, \bibinfo
  {author} {\bibfnamefont {D.}~\bibnamefont {Bacon}}, \ and\ \bibinfo {author}
  {\bibfnamefont {J.~C.~B.}\ \bibnamefont {et~al.}},\ }\bibfield  {title}
  {\enquote {\bibinfo {title} {Quantum supremacy using a programmable
  superconducting processor},}\ }\href
  {https://www.nature.com/articles/s41586-019-1666-5} {\bibfield  {journal}
  {\bibinfo  {journal} {Nature}\ }\textbf {\bibinfo {volume} {574}},\ \bibinfo
  {pages} {505} (\bibinfo {year} {2019})}\BibitemShut {NoStop}%
\bibitem [{\citenamefont {Gong}\ \emph {et~al.}(2021)\citenamefont {Gong},
  \citenamefont {Wang}, \citenamefont {Zha}, \citenamefont {Chen},
  \citenamefont {Huang}, \citenamefont {Wu}, \citenamefont {Zhu}, \citenamefont
  {Zhao}, \citenamefont {Li}, \citenamefont {Guo}, \citenamefont {Qian},
  \citenamefont {Ye}, \citenamefont {Chen}, \citenamefont {Ying}, \citenamefont
  {Yu}, \citenamefont {Fan}, \citenamefont {Wu}, \citenamefont {Su},
  \citenamefont {Deng}, \citenamefont {Rong}, \citenamefont {Zhang},
  \citenamefont {Cao}, \citenamefont {Lin}, \citenamefont {Xu}, \citenamefont
  {Sun}, \citenamefont {Guo}, \citenamefont {Li}, \citenamefont {Liang},
  \citenamefont {Bastidas}, \citenamefont {Nemoto}, \citenamefont {Munro},
  \citenamefont {Huo}, \citenamefont {Lu}, \citenamefont {Peng}, \citenamefont
  {Zhu},\ and\ \citenamefont {Pan}}]{Gongeabg7812}%
  \BibitemOpen
  \bibfield  {author} {\bibinfo {author} {\bibfnamefont {M.}~\bibnamefont
  {Gong}}, \bibinfo {author} {\bibfnamefont {S.}~\bibnamefont {Wang}}, \bibinfo
  {author} {\bibfnamefont {C.}~\bibnamefont {Zha}}, \bibinfo {author}
  {\bibfnamefont {M.-C.}\ \bibnamefont {Chen}}, \bibinfo {author}
  {\bibfnamefont {H.-L.}\ \bibnamefont {Huang}}, \bibinfo {author}
  {\bibfnamefont {Y.}~\bibnamefont {Wu}}, \bibinfo {author} {\bibfnamefont
  {Q.}~\bibnamefont {Zhu}}, \bibinfo {author} {\bibfnamefont {Y.}~\bibnamefont
  {Zhao}}, \bibinfo {author} {\bibfnamefont {S.}~\bibnamefont {Li}}, \bibinfo
  {author} {\bibfnamefont {S.}~\bibnamefont {Guo}}, \bibinfo {author}
  {\bibfnamefont {H.}~\bibnamefont {Qian}}, \bibinfo {author} {\bibfnamefont
  {Y.}~\bibnamefont {Ye}}, \bibinfo {author} {\bibfnamefont {F.}~\bibnamefont
  {Chen}}, \bibinfo {author} {\bibfnamefont {C.}~\bibnamefont {Ying}}, \bibinfo
  {author} {\bibfnamefont {J.}~\bibnamefont {Yu}}, \bibinfo {author}
  {\bibfnamefont {D.}~\bibnamefont {Fan}}, \bibinfo {author} {\bibfnamefont
  {D.}~\bibnamefont {Wu}}, \bibinfo {author} {\bibfnamefont {H.}~\bibnamefont
  {Su}}, \bibinfo {author} {\bibfnamefont {H.}~\bibnamefont {Deng}}, \bibinfo
  {author} {\bibfnamefont {H.}~\bibnamefont {Rong}}, \bibinfo {author}
  {\bibfnamefont {K.}~\bibnamefont {Zhang}}, \bibinfo {author} {\bibfnamefont
  {S.}~\bibnamefont {Cao}}, \bibinfo {author} {\bibfnamefont {J.}~\bibnamefont
  {Lin}}, \bibinfo {author} {\bibfnamefont {Y.}~\bibnamefont {Xu}}, \bibinfo
  {author} {\bibfnamefont {L.}~\bibnamefont {Sun}}, \bibinfo {author}
  {\bibfnamefont {C.}~\bibnamefont {Guo}}, \bibinfo {author} {\bibfnamefont
  {N.}~\bibnamefont {Li}}, \bibinfo {author} {\bibfnamefont {F.}~\bibnamefont
  {Liang}}, \bibinfo {author} {\bibfnamefont {V.~M.}\ \bibnamefont {Bastidas}},
  \bibinfo {author} {\bibfnamefont {K.}~\bibnamefont {Nemoto}}, \bibinfo
  {author} {\bibfnamefont {W.~J.}\ \bibnamefont {Munro}}, \bibinfo {author}
  {\bibfnamefont {Y.-H.}\ \bibnamefont {Huo}}, \bibinfo {author} {\bibfnamefont
  {C.-Y.}\ \bibnamefont {Lu}}, \bibinfo {author} {\bibfnamefont {C.-Z.}\
  \bibnamefont {Peng}}, \bibinfo {author} {\bibfnamefont {X.}~\bibnamefont
  {Zhu}}, \ and\ \bibinfo {author} {\bibfnamefont {J.-W.}\ \bibnamefont
  {Pan}},\ }\bibfield  {title} {\enquote {\bibinfo {title} {Quantum walks on a
  programmable two-dimensional 62-qubit superconducting processor},}\ }\href
  {\doibase 10.1126/science.abg7812} {\bibfield  {journal} {\bibinfo  {journal}
  {Science}\ } (\bibinfo {year} {2021}),\ 10.1126/science.abg7812}\BibitemShut
  {NoStop}%
\bibitem [{\citenamefont {McArdle}\ \emph {et~al.}(2020)\citenamefont
  {McArdle}, \citenamefont {Endo}, \citenamefont {Aspuru-Guzik}, \citenamefont
  {Benjamin},\ and\ \citenamefont {Yuan}}]{McArdle2020}%
  \BibitemOpen
  \bibfield  {author} {\bibinfo {author} {\bibfnamefont {S.}~\bibnamefont
  {McArdle}}, \bibinfo {author} {\bibfnamefont {S.}~\bibnamefont {Endo}},
  \bibinfo {author} {\bibfnamefont {A.}~\bibnamefont {Aspuru-Guzik}}, \bibinfo
  {author} {\bibfnamefont {S.~C.}\ \bibnamefont {Benjamin}}, \ and\ \bibinfo
  {author} {\bibfnamefont {X.}~\bibnamefont {Yuan}},\ }\bibfield  {title}
  {\enquote {\bibinfo {title} {Quantum computational chemistry},}\ }\href
  {\doibase 10.1103/RevModPhys.92.015003} {\bibfield  {journal} {\bibinfo
  {journal} {Rev. Mod. Phys.}\ }\textbf {\bibinfo {volume} {92}},\ \bibinfo
  {pages} {015003} (\bibinfo {year} {2020})}\BibitemShut {NoStop}%
\bibitem [{\citenamefont {Harrigan}\ \emph {et~al.}(2021)\citenamefont
  {Harrigan}, \citenamefont {Sung}, \citenamefont {Neeley}, \citenamefont
  {Satzinger}, \citenamefont {Arute}, \citenamefont {Arya}, \citenamefont
  {Atalaya}, \citenamefont {Bardin}, \citenamefont {Barends}, \citenamefont
  {Boixo} \emph {et~al.}}]{harrigan2021quantum}%
  \BibitemOpen
  \bibfield  {author} {\bibinfo {author} {\bibfnamefont {M.~P.}\ \bibnamefont
  {Harrigan}}, \bibinfo {author} {\bibfnamefont {K.~J.}\ \bibnamefont {Sung}},
  \bibinfo {author} {\bibfnamefont {M.}~\bibnamefont {Neeley}}, \bibinfo
  {author} {\bibfnamefont {K.~J.}\ \bibnamefont {Satzinger}}, \bibinfo {author}
  {\bibfnamefont {F.}~\bibnamefont {Arute}}, \bibinfo {author} {\bibfnamefont
  {K.}~\bibnamefont {Arya}}, \bibinfo {author} {\bibfnamefont {J.}~\bibnamefont
  {Atalaya}}, \bibinfo {author} {\bibfnamefont {J.~C.}\ \bibnamefont {Bardin}},
  \bibinfo {author} {\bibfnamefont {R.}~\bibnamefont {Barends}}, \bibinfo
  {author} {\bibfnamefont {S.}~\bibnamefont {Boixo}},  \emph {et~al.},\
  }\bibfield  {title} {\enquote {\bibinfo {title} {Quantum approximate
  optimization of non-planar graph problems on a planar superconducting
  processor},}\ }\href {\doibase https://doi.org/10.1038/s41567-020-01105-y}
  {\bibfield  {journal} {\bibinfo  {journal} {Nature Physics}\ }\textbf
  {\bibinfo {volume} {17}},\ \bibinfo {pages} {332} (\bibinfo {year}
  {2021})}\BibitemShut {NoStop}%
\bibitem [{\citenamefont {Koch}\ \emph {et~al.}(2007)\citenamefont {Koch},
  \citenamefont {Yu}, \citenamefont {Gambetta}, \citenamefont {Houck},
  \citenamefont {Schuster}, \citenamefont {Majer}, \citenamefont {Blais},
  \citenamefont {Devoret}, \citenamefont {Girvin},\ and\ \citenamefont
  {Schoelkopf}}]{Koch2007Charge}%
  \BibitemOpen
  \bibfield  {author} {\bibinfo {author} {\bibfnamefont {J.}~\bibnamefont
  {Koch}}, \bibinfo {author} {\bibfnamefont {T.~M.}\ \bibnamefont {Yu}},
  \bibinfo {author} {\bibfnamefont {J.}~\bibnamefont {Gambetta}}, \bibinfo
  {author} {\bibfnamefont {A.~A.}\ \bibnamefont {Houck}}, \bibinfo {author}
  {\bibfnamefont {D.~I.}\ \bibnamefont {Schuster}}, \bibinfo {author}
  {\bibfnamefont {J.}~\bibnamefont {Majer}}, \bibinfo {author} {\bibfnamefont
  {A.}~\bibnamefont {Blais}}, \bibinfo {author} {\bibfnamefont {M.~H.}\
  \bibnamefont {Devoret}}, \bibinfo {author} {\bibfnamefont {S.~M.}\
  \bibnamefont {Girvin}}, \ and\ \bibinfo {author} {\bibfnamefont {R.~J.}\
  \bibnamefont {Schoelkopf}},\ }\bibfield  {title} {\enquote {\bibinfo {title}
  {Charge-insensitive qubit design derived from the cooper pair box},}\ }\href
  {\doibase 10.1103/PhysRevA.76.042319} {\bibfield  {journal} {\bibinfo
  {journal} {Phys. Rev. A}\ }\textbf {\bibinfo {volume} {76}},\ \bibinfo
  {pages} {042319} (\bibinfo {year} {2007})}\BibitemShut {NoStop}%
\bibitem [{\citenamefont {Chen}\ \emph {et~al.}(2014)\citenamefont {Chen},
  \citenamefont {Megrant}, \citenamefont {Kelly}, \citenamefont {Barends},
  \citenamefont {Bochmann}, \citenamefont {Chen}, \citenamefont {Chiaro},
  \citenamefont {Dunsworth}, \citenamefont {Jeffrey}, \citenamefont {Mutus}
  \emph {et~al.}}]{chen2014fabrication}%
  \BibitemOpen
  \bibfield  {author} {\bibinfo {author} {\bibfnamefont {Z.}~\bibnamefont
  {Chen}}, \bibinfo {author} {\bibfnamefont {A.}~\bibnamefont {Megrant}},
  \bibinfo {author} {\bibfnamefont {J.}~\bibnamefont {Kelly}}, \bibinfo
  {author} {\bibfnamefont {R.}~\bibnamefont {Barends}}, \bibinfo {author}
  {\bibfnamefont {J.}~\bibnamefont {Bochmann}}, \bibinfo {author}
  {\bibfnamefont {Y.}~\bibnamefont {Chen}}, \bibinfo {author} {\bibfnamefont
  {B.}~\bibnamefont {Chiaro}}, \bibinfo {author} {\bibfnamefont
  {A.}~\bibnamefont {Dunsworth}}, \bibinfo {author} {\bibfnamefont
  {E.}~\bibnamefont {Jeffrey}}, \bibinfo {author} {\bibfnamefont
  {J.}~\bibnamefont {Mutus}},  \emph {et~al.},\ }\bibfield  {title} {\enquote
  {\bibinfo {title} {Fabrication and characterization of aluminum airbridges
  for superconducting microwave circuits},}\ }\href {\doibase
  https://doi.org/10.1063/1.4863745} {\bibfield  {journal} {\bibinfo  {journal}
  {Applied Physics Letters}\ }\textbf {\bibinfo {volume} {104}},\ \bibinfo
  {pages} {052602} (\bibinfo {year} {2014})}\BibitemShut {NoStop}%
\bibitem [{\citenamefont {Dunsworth}\ \emph {et~al.}(2018)\citenamefont
  {Dunsworth}, \citenamefont {Barends}, \citenamefont {Chen}, \citenamefont
  {Chen}, \citenamefont {Chiaro}, \citenamefont {Fowler}, \citenamefont
  {Foxen}, \citenamefont {Jeffrey}, \citenamefont {Kelly}, \citenamefont
  {Klimov} \emph {et~al.}}]{dunsworth2018method}%
  \BibitemOpen
  \bibfield  {author} {\bibinfo {author} {\bibfnamefont {A.}~\bibnamefont
  {Dunsworth}}, \bibinfo {author} {\bibfnamefont {R.}~\bibnamefont {Barends}},
  \bibinfo {author} {\bibfnamefont {Y.}~\bibnamefont {Chen}}, \bibinfo {author}
  {\bibfnamefont {Z.}~\bibnamefont {Chen}}, \bibinfo {author} {\bibfnamefont
  {B.}~\bibnamefont {Chiaro}}, \bibinfo {author} {\bibfnamefont
  {A.}~\bibnamefont {Fowler}}, \bibinfo {author} {\bibfnamefont
  {B.}~\bibnamefont {Foxen}}, \bibinfo {author} {\bibfnamefont
  {E.}~\bibnamefont {Jeffrey}}, \bibinfo {author} {\bibfnamefont
  {J.}~\bibnamefont {Kelly}}, \bibinfo {author} {\bibfnamefont
  {P.}~\bibnamefont {Klimov}},  \emph {et~al.},\ }\bibfield  {title} {\enquote
  {\bibinfo {title} {A method for building low loss multi-layer wiring for
  superconducting microwave devices},}\ }\href {\doibase
  https://doi.org/10.1063/1.5014033} {\bibfield  {journal} {\bibinfo  {journal}
  {Applied Physics Letters}\ }\textbf {\bibinfo {volume} {112}},\ \bibinfo
  {pages} {063502} (\bibinfo {year} {2018})}\BibitemShut {NoStop}%
\bibitem [{\citenamefont {Foxen}\ \emph {et~al.}(2017)\citenamefont {Foxen},
  \citenamefont {Mutus}, \citenamefont {Lucero}, \citenamefont {Graff},
  \citenamefont {Megrant}, \citenamefont {Chen}, \citenamefont {Quintana},
  \citenamefont {Burkett}, \citenamefont {Kelly}, \citenamefont {Jeffrey},
  \citenamefont {Yang}, \citenamefont {Yu}, \citenamefont {Arya}, \citenamefont
  {Barends}, \citenamefont {Chen}, \citenamefont {Chiaro}, \citenamefont
  {Dunsworth}, \citenamefont {Fowler}, \citenamefont {Gidney}, \citenamefont
  {Giustina}, \citenamefont {Huang}, \citenamefont {Klimov}, \citenamefont
  {Neeley}, \citenamefont {Neill}, \citenamefont {Roushan}, \citenamefont
  {Sank}, \citenamefont {Vainsencher}, \citenamefont {Wenner}, \citenamefont
  {White},\ and\ \citenamefont {Martinis}}]{Foxen_2017}%
  \BibitemOpen
  \bibfield  {author} {\bibinfo {author} {\bibfnamefont {B.}~\bibnamefont
  {Foxen}}, \bibinfo {author} {\bibfnamefont {J.~Y.}\ \bibnamefont {Mutus}},
  \bibinfo {author} {\bibfnamefont {E.}~\bibnamefont {Lucero}}, \bibinfo
  {author} {\bibfnamefont {R.}~\bibnamefont {Graff}}, \bibinfo {author}
  {\bibfnamefont {A.}~\bibnamefont {Megrant}}, \bibinfo {author} {\bibfnamefont
  {Y.}~\bibnamefont {Chen}}, \bibinfo {author} {\bibfnamefont {C.}~\bibnamefont
  {Quintana}}, \bibinfo {author} {\bibfnamefont {B.}~\bibnamefont {Burkett}},
  \bibinfo {author} {\bibfnamefont {J.}~\bibnamefont {Kelly}}, \bibinfo
  {author} {\bibfnamefont {E.}~\bibnamefont {Jeffrey}}, \bibinfo {author}
  {\bibfnamefont {Y.}~\bibnamefont {Yang}}, \bibinfo {author} {\bibfnamefont
  {A.}~\bibnamefont {Yu}}, \bibinfo {author} {\bibfnamefont {K.}~\bibnamefont
  {Arya}}, \bibinfo {author} {\bibfnamefont {R.}~\bibnamefont {Barends}},
  \bibinfo {author} {\bibfnamefont {Z.}~\bibnamefont {Chen}}, \bibinfo {author}
  {\bibfnamefont {B.}~\bibnamefont {Chiaro}}, \bibinfo {author} {\bibfnamefont
  {A.}~\bibnamefont {Dunsworth}}, \bibinfo {author} {\bibfnamefont
  {A.}~\bibnamefont {Fowler}}, \bibinfo {author} {\bibfnamefont
  {C.}~\bibnamefont {Gidney}}, \bibinfo {author} {\bibfnamefont
  {M.}~\bibnamefont {Giustina}}, \bibinfo {author} {\bibfnamefont
  {T.}~\bibnamefont {Huang}}, \bibinfo {author} {\bibfnamefont
  {P.}~\bibnamefont {Klimov}}, \bibinfo {author} {\bibfnamefont
  {M.}~\bibnamefont {Neeley}}, \bibinfo {author} {\bibfnamefont
  {C.}~\bibnamefont {Neill}}, \bibinfo {author} {\bibfnamefont
  {P.}~\bibnamefont {Roushan}}, \bibinfo {author} {\bibfnamefont
  {D.}~\bibnamefont {Sank}}, \bibinfo {author} {\bibfnamefont {A.}~\bibnamefont
  {Vainsencher}}, \bibinfo {author} {\bibfnamefont {J.}~\bibnamefont {Wenner}},
  \bibinfo {author} {\bibfnamefont {T.~C.}\ \bibnamefont {White}}, \ and\
  \bibinfo {author} {\bibfnamefont {J.~M.}\ \bibnamefont {Martinis}},\
  }\bibfield  {title} {\enquote {\bibinfo {title} {Qubit compatible
  superconducting interconnects},}\ }\href {\doibase 10.1088/2058-9565/aa94fc}
  {\bibfield  {journal} {\bibinfo  {journal} {Quantum Science and Technology}\
  }\textbf {\bibinfo {volume} {3}},\ \bibinfo {pages} {014005} (\bibinfo {year}
  {2017})}\BibitemShut {NoStop}%
\bibitem [{\citenamefont {Rosenberg}\ \emph {et~al.}(2017)\citenamefont
  {Rosenberg}, \citenamefont {Kim}, \citenamefont {Das}, \citenamefont {Yost},
  \citenamefont {Gustavsson}, \citenamefont {Hover}, \citenamefont {Krantz},
  \citenamefont {Melville}, \citenamefont {Racz}, \citenamefont {Samach} \emph
  {et~al.}}]{rosenberg20173d}%
  \BibitemOpen
  \bibfield  {author} {\bibinfo {author} {\bibfnamefont {D.}~\bibnamefont
  {Rosenberg}}, \bibinfo {author} {\bibfnamefont {D.}~\bibnamefont {Kim}},
  \bibinfo {author} {\bibfnamefont {R.}~\bibnamefont {Das}}, \bibinfo {author}
  {\bibfnamefont {D.}~\bibnamefont {Yost}}, \bibinfo {author} {\bibfnamefont
  {S.}~\bibnamefont {Gustavsson}}, \bibinfo {author} {\bibfnamefont
  {D.}~\bibnamefont {Hover}}, \bibinfo {author} {\bibfnamefont
  {P.}~\bibnamefont {Krantz}}, \bibinfo {author} {\bibfnamefont
  {A.}~\bibnamefont {Melville}}, \bibinfo {author} {\bibfnamefont
  {L.}~\bibnamefont {Racz}}, \bibinfo {author} {\bibfnamefont {G.}~\bibnamefont
  {Samach}},  \emph {et~al.},\ }\bibfield  {title} {\enquote {\bibinfo {title}
  {{3D} integrated superconducting qubits},}\ }\href {\doibase
  10.1038/s41534-017-0044-0} {\bibfield  {journal} {\bibinfo  {journal} {npj
  quantum information}\ }\textbf {\bibinfo {volume} {3}},\ \bibinfo {pages} {1}
  (\bibinfo {year} {2017})}\BibitemShut {NoStop}%
\bibitem [{\citenamefont {Satzinger}\ \emph {et~al.}(2019)\citenamefont
  {Satzinger}, \citenamefont {Conner}, \citenamefont {Bienfait}, \citenamefont
  {Chang}, \citenamefont {Chou}, \citenamefont {Cleland}, \citenamefont
  {Dumur}, \citenamefont {Grebel}, \citenamefont {Peairs}, \citenamefont
  {Povey} \emph {et~al.}}]{satzinger2019simple}%
  \BibitemOpen
  \bibfield  {author} {\bibinfo {author} {\bibfnamefont {K.}~\bibnamefont
  {Satzinger}}, \bibinfo {author} {\bibfnamefont {C.}~\bibnamefont {Conner}},
  \bibinfo {author} {\bibfnamefont {A.}~\bibnamefont {Bienfait}}, \bibinfo
  {author} {\bibfnamefont {H.-S.}\ \bibnamefont {Chang}}, \bibinfo {author}
  {\bibfnamefont {M.-H.}\ \bibnamefont {Chou}}, \bibinfo {author}
  {\bibfnamefont {A.}~\bibnamefont {Cleland}}, \bibinfo {author} {\bibfnamefont
  {{\'E}.}~\bibnamefont {Dumur}}, \bibinfo {author} {\bibfnamefont
  {J.}~\bibnamefont {Grebel}}, \bibinfo {author} {\bibfnamefont
  {G.}~\bibnamefont {Peairs}}, \bibinfo {author} {\bibfnamefont
  {R.}~\bibnamefont {Povey}},  \emph {et~al.},\ }\bibfield  {title} {\enquote
  {\bibinfo {title} {Simple non-galvanic flip-chip integration method for
  hybrid quantum systems},}\ }\href {\doibase
  https://doi.org/10.1063/1.5089888} {\bibfield  {journal} {\bibinfo  {journal}
  {Applied Physics Letters}\ }\textbf {\bibinfo {volume} {114}},\ \bibinfo
  {pages} {173501} (\bibinfo {year} {2019})}\BibitemShut {NoStop}%
\bibitem [{\citenamefont {Yost}\ \emph {et~al.}(2020)\citenamefont {Yost},
  \citenamefont {Schwartz}, \citenamefont {Mallek}, \citenamefont {Rosenberg},
  \citenamefont {Stull}, \citenamefont {Yoder}, \citenamefont {Calusine},
  \citenamefont {Cook}, \citenamefont {Das}, \citenamefont {Day} \emph
  {et~al.}}]{yost2020solid}%
  \BibitemOpen
  \bibfield  {author} {\bibinfo {author} {\bibfnamefont {D.-R.~W.}\
  \bibnamefont {Yost}}, \bibinfo {author} {\bibfnamefont {M.~E.}\ \bibnamefont
  {Schwartz}}, \bibinfo {author} {\bibfnamefont {J.}~\bibnamefont {Mallek}},
  \bibinfo {author} {\bibfnamefont {D.}~\bibnamefont {Rosenberg}}, \bibinfo
  {author} {\bibfnamefont {C.}~\bibnamefont {Stull}}, \bibinfo {author}
  {\bibfnamefont {J.~L.}\ \bibnamefont {Yoder}}, \bibinfo {author}
  {\bibfnamefont {G.}~\bibnamefont {Calusine}}, \bibinfo {author}
  {\bibfnamefont {M.}~\bibnamefont {Cook}}, \bibinfo {author} {\bibfnamefont
  {R.}~\bibnamefont {Das}}, \bibinfo {author} {\bibfnamefont {A.~L.}\
  \bibnamefont {Day}},  \emph {et~al.},\ }\bibfield  {title} {\enquote
  {\bibinfo {title} {Solid-state qubits integrated with superconducting
  through-silicon vias},}\ }\href {\doibase 10.1038/s41534-020-00289-8}
  {\bibfield  {journal} {\bibinfo  {journal} {npj Quantum Information}\
  }\textbf {\bibinfo {volume} {6}},\ \bibinfo {pages} {1} (\bibinfo {year}
  {2020})}\BibitemShut {NoStop}%
\bibitem [{\citenamefont {Mallek}\ \emph {et~al.}(2021)\citenamefont {Mallek},
  \citenamefont {Yost}, \citenamefont {Rosenberg}, \citenamefont {Yoder},
  \citenamefont {Calusine}, \citenamefont {Cook}, \citenamefont {Das},
  \citenamefont {Day}, \citenamefont {Golden}, \citenamefont {Kim} \emph
  {et~al.}}]{mallek2021fabrication}%
  \BibitemOpen
  \bibfield  {author} {\bibinfo {author} {\bibfnamefont {J.~L.}\ \bibnamefont
  {Mallek}}, \bibinfo {author} {\bibfnamefont {D.-R.~W.}\ \bibnamefont {Yost}},
  \bibinfo {author} {\bibfnamefont {D.}~\bibnamefont {Rosenberg}}, \bibinfo
  {author} {\bibfnamefont {J.~L.}\ \bibnamefont {Yoder}}, \bibinfo {author}
  {\bibfnamefont {G.}~\bibnamefont {Calusine}}, \bibinfo {author}
  {\bibfnamefont {M.}~\bibnamefont {Cook}}, \bibinfo {author} {\bibfnamefont
  {R.}~\bibnamefont {Das}}, \bibinfo {author} {\bibfnamefont {A.}~\bibnamefont
  {Day}}, \bibinfo {author} {\bibfnamefont {E.}~\bibnamefont {Golden}},
  \bibinfo {author} {\bibfnamefont {D.~K.}\ \bibnamefont {Kim}},  \emph
  {et~al.},\ }\bibfield  {title} {\enquote {\bibinfo {title} {Fabrication of
  superconducting through-silicon vias},}\ }\href
  {https://arxiv.org/abs/2103.08536} {\bibfield  {journal} {\bibinfo  {journal}
  {arXiv:2103.08536}\ } (\bibinfo {year} {2021})}\BibitemShut {NoStop}%
\bibitem [{\citenamefont {Kelly}\ and\ \citenamefont
  {Jeffrey}(2020)}]{kelly2020low}%
  \BibitemOpen
  \bibfield  {author} {\bibinfo {author} {\bibfnamefont {J.~S.}\ \bibnamefont
  {Kelly}}\ and\ \bibinfo {author} {\bibfnamefont {E.}~\bibnamefont
  {Jeffrey}},\ }\href@noop {} {\enquote {\bibinfo {title} {Low footprint
  resonator in flip chip geometry},}\ } (\bibinfo {year} {2020}),\ \bibinfo
  {note} {{US} Patent App. 16/753,431}\BibitemShut {NoStop}%
\bibitem [{\citenamefont {Gold}\ \emph {et~al.}(2021)\citenamefont {Gold},
  \citenamefont {Paquette}, \citenamefont {Stockklauser}, \citenamefont
  {Reagor}, \citenamefont {Alam}, \citenamefont {Bestwick}, \citenamefont
  {Didier}, \citenamefont {Nersisyan}, \citenamefont {Oruc}, \citenamefont
  {Razavi} \emph {et~al.}}]{gold2021entanglement}%
  \BibitemOpen
  \bibfield  {author} {\bibinfo {author} {\bibfnamefont {A.}~\bibnamefont
  {Gold}}, \bibinfo {author} {\bibfnamefont {J.}~\bibnamefont {Paquette}},
  \bibinfo {author} {\bibfnamefont {A.}~\bibnamefont {Stockklauser}}, \bibinfo
  {author} {\bibfnamefont {M.~J.}\ \bibnamefont {Reagor}}, \bibinfo {author}
  {\bibfnamefont {M.~S.}\ \bibnamefont {Alam}}, \bibinfo {author}
  {\bibfnamefont {A.}~\bibnamefont {Bestwick}}, \bibinfo {author}
  {\bibfnamefont {N.}~\bibnamefont {Didier}}, \bibinfo {author} {\bibfnamefont
  {A.}~\bibnamefont {Nersisyan}}, \bibinfo {author} {\bibfnamefont
  {F.}~\bibnamefont {Oruc}}, \bibinfo {author} {\bibfnamefont {A.}~\bibnamefont
  {Razavi}},  \emph {et~al.},\ }\bibfield  {title} {\enquote {\bibinfo {title}
  {Entanglement across separate silicon dies in a modular superconducting qubit
  device},}\ }\href {https://arxiv.org/abs/2102.13293} {\bibfield  {journal}
  {\bibinfo  {journal} {arXiv:2102.13293}\ } (\bibinfo {year}
  {2021})}\BibitemShut {NoStop}%
\bibitem [{\citenamefont {Zhao}\ \emph {et~al.}(2020)\citenamefont {Zhao},
  \citenamefont {Park}, \citenamefont {Zhao}, \citenamefont {Bal},
  \citenamefont {McRae}, \citenamefont {Long},\ and\ \citenamefont
  {Pappas}}]{zhao2020merged}%
  \BibitemOpen
  \bibfield  {author} {\bibinfo {author} {\bibfnamefont {R.}~\bibnamefont
  {Zhao}}, \bibinfo {author} {\bibfnamefont {S.}~\bibnamefont {Park}}, \bibinfo
  {author} {\bibfnamefont {T.}~\bibnamefont {Zhao}}, \bibinfo {author}
  {\bibfnamefont {M.}~\bibnamefont {Bal}}, \bibinfo {author} {\bibfnamefont
  {C.}~\bibnamefont {McRae}}, \bibinfo {author} {\bibfnamefont
  {J.}~\bibnamefont {Long}}, \ and\ \bibinfo {author} {\bibfnamefont
  {D.}~\bibnamefont {Pappas}},\ }\bibfield  {title} {\enquote {\bibinfo {title}
  {Merged-element transmon},}\ }\href {\doibase
  https://doi.org/10.1103/PhysRevApplied.14.064006} {\bibfield  {journal}
  {\bibinfo  {journal} {Physical Review Applied}\ }\textbf {\bibinfo {volume}
  {14}},\ \bibinfo {pages} {064006} (\bibinfo {year} {2020})}\BibitemShut
  {NoStop}%
\bibitem [{\citenamefont {Mamin}\ \emph {et~al.}(2021)\citenamefont {Mamin},
  \citenamefont {Huang}, \citenamefont {Carnevale}, \citenamefont {Rettner},
  \citenamefont {Arellano}, \citenamefont {Sherwood}, \citenamefont {Kurter},
  \citenamefont {Trimm}, \citenamefont {Sandberg}, \citenamefont {Shelby} \emph
  {et~al.}}]{mamin2021merged}%
  \BibitemOpen
  \bibfield  {author} {\bibinfo {author} {\bibfnamefont {H.}~\bibnamefont
  {Mamin}}, \bibinfo {author} {\bibfnamefont {E.}~\bibnamefont {Huang}},
  \bibinfo {author} {\bibfnamefont {S.}~\bibnamefont {Carnevale}}, \bibinfo
  {author} {\bibfnamefont {C.}~\bibnamefont {Rettner}}, \bibinfo {author}
  {\bibfnamefont {N.}~\bibnamefont {Arellano}}, \bibinfo {author}
  {\bibfnamefont {M.}~\bibnamefont {Sherwood}}, \bibinfo {author}
  {\bibfnamefont {C.}~\bibnamefont {Kurter}}, \bibinfo {author} {\bibfnamefont
  {B.}~\bibnamefont {Trimm}}, \bibinfo {author} {\bibfnamefont
  {M.}~\bibnamefont {Sandberg}}, \bibinfo {author} {\bibfnamefont
  {R.}~\bibnamefont {Shelby}},  \emph {et~al.},\ }\bibfield  {title} {\enquote
  {\bibinfo {title} {Merged-element transmons: Design and qubit performance},}\
  }\href {https://arxiv.org/abs/2103.09163} {\bibfield  {journal} {\bibinfo
  {journal} {arXiv:2103.09163}\ } (\bibinfo {year} {2021})}\BibitemShut
  {NoStop}%
\bibitem [{\citenamefont {Bosman}\ \emph {et~al.}(2017)\citenamefont {Bosman},
  \citenamefont {Gely}, \citenamefont {Singh}, \citenamefont {Bruno},
  \citenamefont {Bothner},\ and\ \citenamefont {Steele}}]{bosman2017multi}%
  \BibitemOpen
  \bibfield  {author} {\bibinfo {author} {\bibfnamefont {S.~J.}\ \bibnamefont
  {Bosman}}, \bibinfo {author} {\bibfnamefont {M.~F.}\ \bibnamefont {Gely}},
  \bibinfo {author} {\bibfnamefont {V.}~\bibnamefont {Singh}}, \bibinfo
  {author} {\bibfnamefont {A.}~\bibnamefont {Bruno}}, \bibinfo {author}
  {\bibfnamefont {D.}~\bibnamefont {Bothner}}, \ and\ \bibinfo {author}
  {\bibfnamefont {G.~A.}\ \bibnamefont {Steele}},\ }\bibfield  {title}
  {\enquote {\bibinfo {title} {Multi-mode ultra-strong coupling in circuit
  quantum electrodynamics},}\ }\href {\doibase 10.1038/s41534-017-0046-y}
  {\bibfield  {journal} {\bibinfo  {journal} {npj Quantum Information}\
  }\textbf {\bibinfo {volume} {3}},\ \bibinfo {pages} {1} (\bibinfo {year}
  {2017})}\BibitemShut {NoStop}%
\bibitem [{\citenamefont {Cicak}\ \emph {et~al.}(2009)\citenamefont {Cicak},
  \citenamefont {Allman}, \citenamefont {Strong}, \citenamefont {Osborn},\ and\
  \citenamefont {Simmonds}}]{cicak2009vacuum}%
  \BibitemOpen
  \bibfield  {author} {\bibinfo {author} {\bibfnamefont {K.}~\bibnamefont
  {Cicak}}, \bibinfo {author} {\bibfnamefont {M.~S.}\ \bibnamefont {Allman}},
  \bibinfo {author} {\bibfnamefont {J.~A.}\ \bibnamefont {Strong}}, \bibinfo
  {author} {\bibfnamefont {K.~D.}\ \bibnamefont {Osborn}}, \ and\ \bibinfo
  {author} {\bibfnamefont {R.~W.}\ \bibnamefont {Simmonds}},\ }\bibfield
  {title} {\enquote {\bibinfo {title} {Vacuum-gap capacitors for low-loss
  superconducting resonant circuits},}\ }\href {\doibase
  10.1109/TASC.2009.2019665} {\bibfield  {journal} {\bibinfo  {journal} {IEEE
  transactions on applied superconductivity}\ }\textbf {\bibinfo {volume}
  {19}},\ \bibinfo {pages} {948} (\bibinfo {year} {2009})}\BibitemShut
  {NoStop}%
\bibitem [{\citenamefont {Wang}\ \emph {et~al.}(2015)\citenamefont {Wang},
  \citenamefont {Axline}, \citenamefont {Gao}, \citenamefont {Brecht},
  \citenamefont {Chu}, \citenamefont {Frunzio}, \citenamefont {Devoret},\ and\
  \citenamefont {Schoelkopf}}]{wang2015surface}%
  \BibitemOpen
  \bibfield  {author} {\bibinfo {author} {\bibfnamefont {C.}~\bibnamefont
  {Wang}}, \bibinfo {author} {\bibfnamefont {C.}~\bibnamefont {Axline}},
  \bibinfo {author} {\bibfnamefont {Y.~Y.}\ \bibnamefont {Gao}}, \bibinfo
  {author} {\bibfnamefont {T.}~\bibnamefont {Brecht}}, \bibinfo {author}
  {\bibfnamefont {Y.}~\bibnamefont {Chu}}, \bibinfo {author} {\bibfnamefont
  {L.}~\bibnamefont {Frunzio}}, \bibinfo {author} {\bibfnamefont
  {M.}~\bibnamefont {Devoret}}, \ and\ \bibinfo {author} {\bibfnamefont
  {R.~J.}\ \bibnamefont {Schoelkopf}},\ }\bibfield  {title} {\enquote {\bibinfo
  {title} {Surface participation and dielectric loss in superconducting
  qubits},}\ }\href {\doibase 10.1063/1.4934486} {\bibfield  {journal}
  {\bibinfo  {journal} {Applied Physics Letters}\ }\textbf {\bibinfo {volume}
  {107}},\ \bibinfo {pages} {162601} (\bibinfo {year} {2015})}\BibitemShut
  {NoStop}%
\bibitem [{\citenamefont {Gambetta}\ \emph {et~al.}(2016)\citenamefont
  {Gambetta}, \citenamefont {Murray}, \citenamefont {Fung}, \citenamefont
  {McClure}, \citenamefont {Dial}, \citenamefont {Shanks}, \citenamefont
  {Sleight},\ and\ \citenamefont {Steffen}}]{gambetta2016investigating}%
  \BibitemOpen
  \bibfield  {author} {\bibinfo {author} {\bibfnamefont {J.~M.}\ \bibnamefont
  {Gambetta}}, \bibinfo {author} {\bibfnamefont {C.~E.}\ \bibnamefont
  {Murray}}, \bibinfo {author} {\bibfnamefont {Y.-K.-K.}\ \bibnamefont {Fung}},
  \bibinfo {author} {\bibfnamefont {D.~T.}\ \bibnamefont {McClure}}, \bibinfo
  {author} {\bibfnamefont {O.}~\bibnamefont {Dial}}, \bibinfo {author}
  {\bibfnamefont {W.}~\bibnamefont {Shanks}}, \bibinfo {author} {\bibfnamefont
  {J.~W.}\ \bibnamefont {Sleight}}, \ and\ \bibinfo {author} {\bibfnamefont
  {M.}~\bibnamefont {Steffen}},\ }\bibfield  {title} {\enquote {\bibinfo
  {title} {Investigating surface loss effects in superconducting transmon
  qubits},}\ }\href {\doibase 10.1109/TASC.2016.2629670} {\bibfield  {journal}
  {\bibinfo  {journal} {IEEE Transactions on Applied Superconductivity}\
  }\textbf {\bibinfo {volume} {27}},\ \bibinfo {pages} {1} (\bibinfo {year}
  {2016})}\BibitemShut {NoStop}%
\bibitem [{\citenamefont {Wenner}\ \emph {et~al.}(2011)\citenamefont {Wenner},
  \citenamefont {Barends}, \citenamefont {Bialczak}, \citenamefont {Chen},
  \citenamefont {Kelly}, \citenamefont {Lucero}, \citenamefont {Mariantoni},
  \citenamefont {Megrant}, \citenamefont {O¡¯Malley}, \citenamefont {Sank}
  \emph {et~al.}}]{wenner2011surface}%
  \BibitemOpen
  \bibfield  {author} {\bibinfo {author} {\bibfnamefont {J.}~\bibnamefont
  {Wenner}}, \bibinfo {author} {\bibfnamefont {R.}~\bibnamefont {Barends}},
  \bibinfo {author} {\bibfnamefont {R.}~\bibnamefont {Bialczak}}, \bibinfo
  {author} {\bibfnamefont {Y.}~\bibnamefont {Chen}}, \bibinfo {author}
  {\bibfnamefont {J.}~\bibnamefont {Kelly}}, \bibinfo {author} {\bibfnamefont
  {E.}~\bibnamefont {Lucero}}, \bibinfo {author} {\bibfnamefont
  {M.}~\bibnamefont {Mariantoni}}, \bibinfo {author} {\bibfnamefont
  {A.}~\bibnamefont {Megrant}}, \bibinfo {author} {\bibfnamefont
  {P.}~\bibnamefont {O¡¯Malley}}, \bibinfo {author} {\bibfnamefont
  {D.}~\bibnamefont {Sank}},  \emph {et~al.},\ }\bibfield  {title} {\enquote
  {\bibinfo {title} {Surface loss simulations of superconducting coplanar
  waveguide resonators},}\ }\href {\doibase https://doi.org/10.1063/1.3637047}
  {\bibfield  {journal} {\bibinfo  {journal} {Applied Physics Letters}\
  }\textbf {\bibinfo {volume} {99}},\ \bibinfo {pages} {113513} (\bibinfo
  {year} {2011})}\BibitemShut {NoStop}%
\bibitem [{\citenamefont {Place}\ \emph {et~al.}(2021)\citenamefont {Place},
  \citenamefont {Rodgers}, \citenamefont {Mundada}, \citenamefont {Smitham},
  \citenamefont {Fitzpatrick}, \citenamefont {Leng}, \citenamefont {Premkumar},
  \citenamefont {Bryon}, \citenamefont {Vrajitoarea}, \citenamefont {Sussman}
  \emph {et~al.}}]{place2021new}%
  \BibitemOpen
  \bibfield  {author} {\bibinfo {author} {\bibfnamefont {A.~P.}\ \bibnamefont
  {Place}}, \bibinfo {author} {\bibfnamefont {L.~V.}\ \bibnamefont {Rodgers}},
  \bibinfo {author} {\bibfnamefont {P.}~\bibnamefont {Mundada}}, \bibinfo
  {author} {\bibfnamefont {B.~M.}\ \bibnamefont {Smitham}}, \bibinfo {author}
  {\bibfnamefont {M.}~\bibnamefont {Fitzpatrick}}, \bibinfo {author}
  {\bibfnamefont {Z.}~\bibnamefont {Leng}}, \bibinfo {author} {\bibfnamefont
  {A.}~\bibnamefont {Premkumar}}, \bibinfo {author} {\bibfnamefont
  {J.}~\bibnamefont {Bryon}}, \bibinfo {author} {\bibfnamefont
  {A.}~\bibnamefont {Vrajitoarea}}, \bibinfo {author} {\bibfnamefont
  {S.}~\bibnamefont {Sussman}},  \emph {et~al.},\ }\bibfield  {title} {\enquote
  {\bibinfo {title} {New material platform for superconducting transmon qubits
  with coherence times exceeding 0.3 milliseconds},}\ }\href
  {https://www.nature.com/articles/s41467-021-22030-5} {\bibfield  {journal}
  {\bibinfo  {journal} {Nature communications}\ }\textbf {\bibinfo {volume}
  {12}},\ \bibinfo {pages} {1779} (\bibinfo {year} {2021})}\BibitemShut
  {NoStop}%
\bibitem [{\citenamefont {Wang}\ \emph {et~al.}(2021)\citenamefont {Wang},
  \citenamefont {Li}, \citenamefont {Xu}, \citenamefont {Li}, \citenamefont
  {Wang}, \citenamefont {Yang}, \citenamefont {Mi}, \citenamefont {Liang},
  \citenamefont {Su}, \citenamefont {Yang}, \citenamefont {Wang}, \citenamefont
  {Wang}, \citenamefont {Li}, \citenamefont {Chen}, \citenamefont {Li},
  \citenamefont {Linghu}, \citenamefont {Han}, \citenamefont {Zhang},
  \citenamefont {Feng}, \citenamefont {Song}, \citenamefont {Ma}, \citenamefont
  {Jingning~Zhang}, \citenamefont {Zhao}, \citenamefont {Liu}, \citenamefont
  {Xue}, \citenamefont {Jin},\ and\ \citenamefont {Yu}}]{yu2021tatransmon}%
  \BibitemOpen
  \bibfield  {author} {\bibinfo {author} {\bibfnamefont {C.}~\bibnamefont
  {Wang}}, \bibinfo {author} {\bibfnamefont {X.}~\bibnamefont {Li}}, \bibinfo
  {author} {\bibfnamefont {H.}~\bibnamefont {Xu}}, \bibinfo {author}
  {\bibfnamefont {Z.}~\bibnamefont {Li}}, \bibinfo {author} {\bibfnamefont
  {J.}~\bibnamefont {Wang}}, \bibinfo {author} {\bibfnamefont {Z.}~\bibnamefont
  {Yang}}, \bibinfo {author} {\bibfnamefont {Z.}~\bibnamefont {Mi}}, \bibinfo
  {author} {\bibfnamefont {X.}~\bibnamefont {Liang}}, \bibinfo {author}
  {\bibfnamefont {T.}~\bibnamefont {Su}}, \bibinfo {author} {\bibfnamefont
  {C.}~\bibnamefont {Yang}}, \bibinfo {author} {\bibfnamefont {G.}~\bibnamefont
  {Wang}}, \bibinfo {author} {\bibfnamefont {W.}~\bibnamefont {Wang}}, \bibinfo
  {author} {\bibfnamefont {Y.}~\bibnamefont {Li}}, \bibinfo {author}
  {\bibfnamefont {M.}~\bibnamefont {Chen}}, \bibinfo {author} {\bibfnamefont
  {C.}~\bibnamefont {Li}}, \bibinfo {author} {\bibfnamefont {K.}~\bibnamefont
  {Linghu}}, \bibinfo {author} {\bibfnamefont {J.}~\bibnamefont {Han}},
  \bibinfo {author} {\bibfnamefont {Y.}~\bibnamefont {Zhang}}, \bibinfo
  {author} {\bibfnamefont {Y.}~\bibnamefont {Feng}}, \bibinfo {author}
  {\bibfnamefont {Y.}~\bibnamefont {Song}}, \bibinfo {author} {\bibfnamefont
  {T.}~\bibnamefont {Ma}}, \bibinfo {author} {\bibfnamefont {R.~W.}\
  \bibnamefont {Jingning~Zhang}}, \bibinfo {author} {\bibfnamefont
  {P.}~\bibnamefont {Zhao}}, \bibinfo {author} {\bibfnamefont {W.}~\bibnamefont
  {Liu}}, \bibinfo {author} {\bibfnamefont {G.}~\bibnamefont {Xue}}, \bibinfo
  {author} {\bibfnamefont {Y.}~\bibnamefont {Jin}}, \ and\ \bibinfo {author}
  {\bibfnamefont {H.}~\bibnamefont {Yu}},\ }\bibfield  {title} {\enquote
  {\bibinfo {title} {Transmon qubit with relaxation time exceeding 0.5
  milliseconds},}\ }\href {https://arxiv.org/abs/2105.09890} {\bibfield
  {journal} {\bibinfo  {journal} {arXiv:2105.09890}\ } (\bibinfo {year}
  {2021})}\BibitemShut {NoStop}%
\bibitem [{\citenamefont {Miao}\ and\ \citenamefont {Sheng}(2020)}]{atomly}%
  \BibitemOpen
  \bibfield  {author} {\bibinfo {author} {\bibfnamefont {L.}~\bibnamefont
  {Miao}}\ and\ \bibinfo {author} {\bibfnamefont {M.}~\bibnamefont {Sheng}},\
  }\href {https://atomly.net/} {\enquote {\bibinfo {title} {Atomly material
  science database},}\ } (\bibinfo {year} {2020})\BibitemShut {NoStop}%
\bibitem [{\citenamefont {Schulte}\ \emph {et~al.}(2012)\citenamefont
  {Schulte}, \citenamefont {Cooper}, \citenamefont {Phillips},\ and\
  \citenamefont {Shinde}}]{schulte2012characterization}%
  \BibitemOpen
  \bibfield  {author} {\bibinfo {author} {\bibfnamefont {E.~F.}\ \bibnamefont
  {Schulte}}, \bibinfo {author} {\bibfnamefont {K.~A.}\ \bibnamefont {Cooper}},
  \bibinfo {author} {\bibfnamefont {M.}~\bibnamefont {Phillips}}, \ and\
  \bibinfo {author} {\bibfnamefont {S.~L.}\ \bibnamefont {Shinde}},\ }\bibfield
   {title} {\enquote {\bibinfo {title} {Characterization of a novel fluxless
  surface preparation process for die interconnect bonding},}\ }in\ \href
  {\doibase 10.1109/ECTC.2012.6248801} {\emph {\bibinfo {booktitle} {2012 IEEE
  62nd Electronic Components and Technology Conference}}}\ (\bibinfo
  {organization} {IEEE},\ \bibinfo {year} {2012})\ pp.\ \bibinfo {pages}
  {26--30}\BibitemShut {NoStop}%
\bibitem [{\citenamefont {Kim}\ \emph {et~al.}(2008)\citenamefont {Kim},
  \citenamefont {Schoeller}, \citenamefont {Cho},\ and\ \citenamefont
  {Park}}]{kim2008effect}%
  \BibitemOpen
  \bibfield  {author} {\bibinfo {author} {\bibfnamefont {J.}~\bibnamefont
  {Kim}}, \bibinfo {author} {\bibfnamefont {H.}~\bibnamefont {Schoeller}},
  \bibinfo {author} {\bibfnamefont {J.}~\bibnamefont {Cho}}, \ and\ \bibinfo
  {author} {\bibfnamefont {S.}~\bibnamefont {Park}},\ }\bibfield  {title}
  {\enquote {\bibinfo {title} {Effect of oxidation on indium solderability},}\
  }\href {\doibase https://doi.org/10.1007/s11664-007-0346-7} {\bibfield
  {journal} {\bibinfo  {journal} {Journal of electronic materials}\ }\textbf
  {\bibinfo {volume} {37}},\ \bibinfo {pages} {483} (\bibinfo {year}
  {2008})}\BibitemShut {NoStop}%
\bibitem [{\citenamefont {Nersisyan}\ \emph {et~al.}(2019)\citenamefont
  {Nersisyan}, \citenamefont {Poletto}, \citenamefont {Alidoust}, \citenamefont
  {Manenti}, \citenamefont {Renzas}, \citenamefont {Bui}, \citenamefont {Vu},
  \citenamefont {Whyland}, \citenamefont {Mohan}, \citenamefont {Sete} \emph
  {et~al.}}]{nersisyan2019manufacturing}%
  \BibitemOpen
  \bibfield  {author} {\bibinfo {author} {\bibfnamefont {A.}~\bibnamefont
  {Nersisyan}}, \bibinfo {author} {\bibfnamefont {S.}~\bibnamefont {Poletto}},
  \bibinfo {author} {\bibfnamefont {N.}~\bibnamefont {Alidoust}}, \bibinfo
  {author} {\bibfnamefont {R.}~\bibnamefont {Manenti}}, \bibinfo {author}
  {\bibfnamefont {R.}~\bibnamefont {Renzas}}, \bibinfo {author} {\bibfnamefont
  {C.-V.}\ \bibnamefont {Bui}}, \bibinfo {author} {\bibfnamefont
  {K.}~\bibnamefont {Vu}}, \bibinfo {author} {\bibfnamefont {T.}~\bibnamefont
  {Whyland}}, \bibinfo {author} {\bibfnamefont {Y.}~\bibnamefont {Mohan}},
  \bibinfo {author} {\bibfnamefont {E.~A.}\ \bibnamefont {Sete}},  \emph
  {et~al.},\ }\bibfield  {title} {\enquote {\bibinfo {title} {Manufacturing low
  dissipation superconducting quantum processors},}\ }in\ \href {\doibase
  10.1109/IEDM19573.2019.8993458} {\emph {\bibinfo {booktitle} {2019 IEEE
  International Electron Devices Meeting (IEDM)}}}\ (\bibinfo {organization}
  {IEEE},\ \bibinfo {year} {2019})\ pp.\ \bibinfo {pages} {31--1}\BibitemShut
  {NoStop}%
\bibitem [{\citenamefont {Tsioutsios}\ \emph {et~al.}(2020)\citenamefont
  {Tsioutsios}, \citenamefont {Serniak}, \citenamefont {Diamond}, \citenamefont
  {Sivak}, \citenamefont {Wang}, \citenamefont {Shankar}, \citenamefont
  {Frunzio}, \citenamefont {Schoelkopf},\ and\ \citenamefont
  {Devoret}}]{tsioutsios2020free}%
  \BibitemOpen
  \bibfield  {author} {\bibinfo {author} {\bibfnamefont {I.}~\bibnamefont
  {Tsioutsios}}, \bibinfo {author} {\bibfnamefont {K.}~\bibnamefont {Serniak}},
  \bibinfo {author} {\bibfnamefont {S.}~\bibnamefont {Diamond}}, \bibinfo
  {author} {\bibfnamefont {V.}~\bibnamefont {Sivak}}, \bibinfo {author}
  {\bibfnamefont {Z.}~\bibnamefont {Wang}}, \bibinfo {author} {\bibfnamefont
  {S.}~\bibnamefont {Shankar}}, \bibinfo {author} {\bibfnamefont
  {L.}~\bibnamefont {Frunzio}}, \bibinfo {author} {\bibfnamefont
  {R.}~\bibnamefont {Schoelkopf}}, \ and\ \bibinfo {author} {\bibfnamefont
  {M.}~\bibnamefont {Devoret}},\ }\bibfield  {title} {\enquote {\bibinfo
  {title} {Free-standing silicon shadow masks for transmon qubit
  fabrication},}\ }\href {\doibase 10.1063/1.5138953} {\bibfield  {journal}
  {\bibinfo  {journal} {AIP Advances}\ }\textbf {\bibinfo {volume} {10}},\
  \bibinfo {pages} {065120} (\bibinfo {year} {2020})}\BibitemShut {NoStop}%
\bibitem [{\citenamefont {Mergenthaler}\ \emph {et~al.}(2021)\citenamefont
  {Mergenthaler}, \citenamefont {Paredes}, \citenamefont {M{\"u}ller},
  \citenamefont {M{\"u}ller}, \citenamefont {Filipp}, \citenamefont {Sandberg},
  \citenamefont {Hertzberg}, \citenamefont {Adiga}, \citenamefont {Brink},\
  and\ \citenamefont {Fuhrer}}]{mergenthaler2021ultrahigh}%
  \BibitemOpen
  \bibfield  {author} {\bibinfo {author} {\bibfnamefont {M.}~\bibnamefont
  {Mergenthaler}}, \bibinfo {author} {\bibfnamefont {S.}~\bibnamefont
  {Paredes}}, \bibinfo {author} {\bibfnamefont {P.}~\bibnamefont {M{\"u}ller}},
  \bibinfo {author} {\bibfnamefont {C.}~\bibnamefont {M{\"u}ller}}, \bibinfo
  {author} {\bibfnamefont {S.}~\bibnamefont {Filipp}}, \bibinfo {author}
  {\bibfnamefont {M.}~\bibnamefont {Sandberg}}, \bibinfo {author}
  {\bibfnamefont {J.}~\bibnamefont {Hertzberg}}, \bibinfo {author}
  {\bibfnamefont {V.}~\bibnamefont {Adiga}}, \bibinfo {author} {\bibfnamefont
  {M.}~\bibnamefont {Brink}}, \ and\ \bibinfo {author} {\bibfnamefont
  {A.}~\bibnamefont {Fuhrer}},\ }\bibfield  {title} {\enquote {\bibinfo {title}
  {Ultrahigh vacuum packaging and surface cleaning for quantum devices},}\
  }\href {\doibase 10.1063/5.0034574} {\bibfield  {journal} {\bibinfo
  {journal} {Review of Scientific Instruments}\ }\textbf {\bibinfo {volume}
  {92}},\ \bibinfo {pages} {025121} (\bibinfo {year} {2021})}\BibitemShut
  {NoStop}%
\bibitem [{\citenamefont {Niepce}\ \emph {et~al.}(2020)\citenamefont {Niepce},
  \citenamefont {Burnett}, \citenamefont {Latorre},\ and\ \citenamefont
  {Bylander}}]{niepce2020geometric}%
  \BibitemOpen
  \bibfield  {author} {\bibinfo {author} {\bibfnamefont {D.}~\bibnamefont
  {Niepce}}, \bibinfo {author} {\bibfnamefont {J.~J.}\ \bibnamefont {Burnett}},
  \bibinfo {author} {\bibfnamefont {M.~G.}\ \bibnamefont {Latorre}}, \ and\
  \bibinfo {author} {\bibfnamefont {J.}~\bibnamefont {Bylander}},\ }\bibfield
  {title} {\enquote {\bibinfo {title} {Geometric scaling of two-level-system
  loss in superconducting resonators},}\ }\href {\doibase
  10.1088/1361-6668/ab6179} {\bibfield  {journal} {\bibinfo  {journal}
  {Superconductor Science and Technology}\ }\textbf {\bibinfo {volume} {33}},\
  \bibinfo {pages} {025013} (\bibinfo {year} {2020})}\BibitemShut {NoStop}%
\bibitem [{\citenamefont {Sete}\ \emph {et~al.}(2021)\citenamefont {Sete},
  \citenamefont {Chen}, \citenamefont {Manenti}, \citenamefont {Kulshreshtha},\
  and\ \citenamefont {Poletto}}]{sete2021floating}%
  \BibitemOpen
  \bibfield  {author} {\bibinfo {author} {\bibfnamefont {E.~A.}\ \bibnamefont
  {Sete}}, \bibinfo {author} {\bibfnamefont {A.~Q.}\ \bibnamefont {Chen}},
  \bibinfo {author} {\bibfnamefont {R.}~\bibnamefont {Manenti}}, \bibinfo
  {author} {\bibfnamefont {S.}~\bibnamefont {Kulshreshtha}}, \ and\ \bibinfo
  {author} {\bibfnamefont {S.}~\bibnamefont {Poletto}},\ }\bibfield  {title}
  {\enquote {\bibinfo {title} {Floating tunable coupler for scalable quantum
  computing architectures},}\ }\href {https://arxiv.org/abs/2103.07030}
  {\bibfield  {journal} {\bibinfo  {journal} {arXiv:2103.07030}\ } (\bibinfo
  {year} {2021})}\BibitemShut {NoStop}%
\end{thebibliography}%

\end{document}
%

