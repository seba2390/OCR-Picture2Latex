
%% bare_jrnl_compsoc.tex
%% V1.4b
%% 2015/08/26
%% by Michael Shell
%% See:
%% http://www.michaelshell.org/
%% for current contact information.
%%
%% This is a skeleton file demonstrating the use of IEEEtran.cls
%% (requires IEEEtran.cls version 1.8b or later) with an IEEE
%% Computer Society journal paper.
%%
%% Support sites:
%% http://www.michaelshell.org/tex/ieeetran/
%% http://www.ctan.org/pkg/ieeetran
%% and
%% http://www.ieee.org/

%%*************************************************************************
%% Legal Notice:
%% This code is offered as-is without any warranty either expressed or
%% implied; without even the implied warranty of MERCHANTABILITY or
%% FITNESS FOR A PARTICULAR PURPOSE! 
%% User assumes all risk.
%% In no event shall the IEEE or any contributor to this code be liable for
%% any damages or losses, including, but not limited to, incidental,
%% consequential, or any other damages, resulting from the use or misuse
%% of any information contained here.
%%
%% All comments are the opinions of their respective authors and are not
%% necessarily endorsed by the IEEE.
%%
%% This work is distributed under the LaTeX Project Public License (LPPL)
%% ( http://www.latex-project.org/ ) version 1.3, and may be freely used,
%% distributed and modified. A copy of the LPPL, version 1.3, is included
%% in the base LaTeX documentation of all distributions of LaTeX released
%% 2003/12/01 or later.
%% Retain all contribution notices and credits.
%% ** Modified files should be clearly indicated as such, including  **
%% ** renaming them and changing author support contact information. **
%%*************************************************************************


% *** Authors should verify (and, if needed, correct) their LaTeX system  ***
% *** with the testflow diagnostic prior to trusting their LaTeX platform ***
% *** with production work. The IEEE's font choices and paper sizes can   ***
% *** trigger bugs that do not appear when using other class files.       ***                          ***
% The testflow support page is at:
% http://www.michaelshell.org/tex/testflow/


\documentclass[10pt,journal,compsoc]{IEEEtran}
%
% If IEEEtran.cls has not been installed into the LaTeX system files,
% manually specify the path to it like:
% \documentclass[10pt,journal,compsoc]{../sty/IEEEtran}

\usepackage{amssymb}


% *** CITATION PACKAGES ***
%
\ifCLASSOPTIONcompsoc
  % IEEE Computer Society needs nocompress option
  % requires cite.sty v4.0 or later (November 2003)
  \usepackage[nocompress]{cite}
\else
  % normal IEEE
  \usepackage{cite}
\fi

% *** GRAPHICS RELATED PACKAGES ***
%
\ifCLASSINFOpdf
  \usepackage[pdftex]{graphicx}
  % declare the path(s) where your graphic files are
  % \graphicspath{{../pdf/}{../jpeg/}}
  % and their extensions so you won't have to specify these with
  % every instance of \includegraphics
  % \DeclareGraphicsExtensions{.pdf,.jpeg,.png}
\else
  % or other class option (dvipsone, dvipdf, if not using dvips). graphicx
  % will default to the driver specified in the system graphics.cfg if no
  % driver is specified.
  % \usepackage[dvips]{graphicx}
  % declare the path(s) where your graphic files are
  % \graphicspath{{../eps/}}
  % and their extensions so you won't have to specify these with
  % every instance of \includegraphics
  % \DeclareGraphicsExtensions{.eps}
\fi


% *** SPECIALIZED LIST PACKAGES ***
%
\usepackage{algorithm}
\usepackage{algpseudocode}
% algorithmic.sty was written by Peter Williams and Rogerio Brito.
% This package provides an algorithmic environment fo describing algorithms.
% You can use the algorithmic environment in-text or within a figure
% environment to provide for a floating algorithm. Do NOT use the algorithm
% floating environment provided by algorithm.sty (by the same authors) or
% algorithm2e.sty (by Christophe Fiorio) as the IEEE does not use dedicated
% algorithm float types and packages that provide these will not provide
% correct IEEE style captions. The latest version and documentation of
% algorithmic.sty can be obtained at:
% http://www.ctan.org/pkg/algorithms
% Also of interest may be the (relatively newer and more customizable)
% algorithmicx.sty package by Szasz Janos:
% http://www.ctan.org/pkg/algorithmicx




% *** ALIGNMENT PACKAGES ***
%
\usepackage{array}
% Frank Mittelbach's and David Carlisle's array.sty patches and improves
% the standard LaTeX2e array and tabular environments to provide better
% appearance and additional user controls. As the default LaTeX2e table
% generation code is lacking to the point of almost being broken with
% respect to the quality of the end results, all users are strongly
% advised to use an enhanced (at the very least that provided by array.sty)
% set of table tools. array.sty is already installed on most systems. The
% latest version and documentation can be obtained at:
% http://www.ctan.org/pkg/array


% IEEEtran contains the IEEEeqnarray family of commands that can be used to
% generate multiline equations as well as matrices, tables, etc., of high
% quality.




% *** SUBFIGURE PACKAGES ***
\ifCLASSOPTIONcompsoc
 \usepackage[caption=false,font=footnotesize,labelfont=sf,textfont=sf]{subfig}
\else
 \usepackage[caption=false,font=footnotesize]{subfig}
\fi
% subfig.sty, written by Steven Douglas Cochran, is the modern replacement
% for subfigure.sty, the latter of which is no longer maintained and is
% incompatible with some LaTeX packages including fixltx2e. However,
% subfig.sty requires and automatically loads Axel Sommerfeldt's caption.sty
% which will override IEEEtran.cls' handling of captions and this will result
% in non-IEEE style figure/table captions. To prevent this problem, be sure
% and invoke subfig.sty's "caption=false" package option (available since
% subfig.sty version 1.3, 2005/06/28) as this is will preserve IEEEtran.cls
% handling of captions.
% Note that the Computer Society format requires a sans serif font rather
% than the serif font used in traditional IEEE formatting and thus the need
% to invoke different subfig.sty package options depending on whether
% compsoc mode has been enabled.
%
% The latest version and documentation of subfig.sty can be obtained at:
% http://www.ctan.org/pkg/subfig




% Do not attempt to use stfloats with fixltx2e as they are incompatible.
% Instead, use Morten Hogholm'a dblfloatfix which combines the features
% of both fixltx2e and stfloats:
%
\usepackage{dblfloatfix}
% The latest version can be found at:
% http://www.ctan.org/pkg/dblfloatfix




\ifCLASSOPTIONcaptionsoff
 \usepackage[nomarkers]{endfloat}
\let\MYoriglatexcaption\caption
\renewcommand{\caption}[2][\relax]{\MYoriglatexcaption[#2]{#2}}
\fi
% endfloat.sty was written by James Darrell McCauley, Jeff Goldberg and 
% Axel Sommerfeldt. This package may be useful when used in conjunction with 
% IEEEtran.cls'  captionsoff option. Some IEEE journals/societies require that
% submissions have lists of figures/tables at the end of the paper and that
% figures/tables without any captions are placed on a page by themselves at
% the end of the document. If needed, the draftcls IEEEtran class option or
% \CLASSINPUTbaselinestretch interface can be used to increase the line
% spacing as well. Be sure and use the nomarkers option of endfloat to
% prevent endfloat from "marking" where the figures would have been placed
% in the text. The two hack lines of code above are a slight modification of
% that suggested by in the endfloat docs (section 8.4.1) to ensure that
% the full captions always appear in the list of figures/tables - even if
% the user used the short optional argument of \caption[]{}.
% IEEE papers do not typically make use of \caption[]'s optional argument,
% so this should not be an issue. A similar trick can be used to disable
% captions of packages such as subfig.sty that lack options to turn off
% the subcaptions:
% For subfig.sty:
% \let\MYorigsubfloat\subfloat
% \renewcommand{\subfloat}[2][\relax]{\MYorigsubfloat[]{#2}}
% However, the above trick will not work if both optional arguments of
% the \subfloat command are used. Furthermore, there needs to be a
% description of each subfigure *somewhere* and endfloat does not add
% subfigure captions to its list of figures. Thus, the best approach is to
% avoid the use of subfigure captions (many IEEE journals avoid them anyway)
% and instead reference/explain all the subfigures within the main caption.
% The latest version of endfloat.sty and its documentation can obtained at:
% http://www.ctan.org/pkg/endfloat
%
% The IEEEtran \ifCLASSOPTIONcaptionsoff conditional can also be used
% later in the document, say, to conditionally put the References on a 
% page by themselves.




% *** PDF, URL AND HYPERLINK PACKAGES ***
%
\usepackage{url}
% url.sty was written by Donald Arseneau. It provides better support for
% handling and breaking URLs. url.sty is already installed on most LaTeX
% systems. The latest version and documentation can be obtained at:
% http://www.ctan.org/pkg/url
% Basically, \url{my_url_here}.


\usepackage{amsmath,amsthm}
\usepackage{float,xspace}
\usepackage{thm-restate}
\usepackage{balance}
% \usepackage{algorithm} 
% \usepackage{algpseudocode}
\usepackage{cite,enumerate}
\usepackage{footnote}
\usepackage{float}

\usepackage{pifont}
\newcommand{\cmark}{\ding{51}}%
\newcommand{\xmark}{\ding{55}}%
\usepackage{pbox}
\usepackage{cellspace}
\usepackage[flushleft]{threeparttable}
\usepackage[table]{xcolor}

% \usepackage{flushend}

\usepackage[keeplastbox]{flushend}
\usepackage{placeins}

\usepackage{hyperref}
% \usepackage{color}
\usepackage{framed}
\newcommand{\MSE}{\mathrm{MSE}}
 %\usepackage[lite]{mtpro2}
\newcommand{\E}{\mathbb{E}}
\newcommand{\NN}{\mathrm{NN}}
\newcommand{\ANN}{\mathrm{ANN}}
%\newcommand{\H}{\mathrm{H}}
\newcommand{\IP}{\mathrm{IP}}
\newcommand{\BCS}{\mathrm{BCS}}
\newcommand{\CBE}{\mathrm{CBE}}
\newcommand{\DOPH}{\mathrm{DOPH}}
\newcommand{\RCS}{\mathrm{RCS}}
%\newcommand{\K}{\mathrm{K}}
%\newcommand{\M}{\mathrm{L}}
%\newcommand{\JS}{\mathrm{JS}}
\newcommand{\K}{\mathrm{K}}
%\newcommand{L}{\mathrm{L}}
\newcommand{\JS}{\mathrm{JS}}
\newcommand{\tb}{\mathrm{\psi}}
\usepackage{bbm}
\usepackage{verbatim}
\newcommand{\IPS}[1]{\langle #1 \rangle}

%%%ORINGINAL Vectors %%%%
\newcommand{\uo}{\mathbf{u}}
\newcommand{\vo}{\mathbf{v}}
%\newcommand{\d1}{\mathbbmtt{d}}
% \linespread
\newcommand\numberthis{\addtocounter{equation}{1}\tag{\theequation}}
%%% Binary Vectors post simhash %%%%
\newcommand{\us}{\mathbf{u}_s}
\newcommand{\vs}{\mathbf{v}_s}

%%% Binary Vectors post Oddsketch/RIP Sketch %%%%
\newcommand{\uc}{\mathbf{u^{(s)}}}
\newcommand{\vc}{\mathbf{v^{(s)}}}
\newcommand{\dc}{{d^{(s)}}}

% \newcommand{\S}{\mathcal{S}
\newcommand{\pivot}{\mathrm{PivotHash}}
\newcommand{\mask}{\mathrm{MaskHash}}

\newcommand{\tu}{\mathrm{{\psi}_u}}
\newcommand{\tm}{\mathrm{\Psi}}
\newcommand{\dH}{\mathrm{d_H}}
\newcommand{\F}{\mathrm{F_1}}
\newcommand{\N}{\mathrm{N}}
\newcommand{\Ham}{\mathrm{Ham}}
\newcommand{\Sim}{\mathrm{Sim}}
\newcommand{\D}{\mathcal{D}}
\newcommand{\Var}{\mathrm{Var}}
\newcommand{\Cov}{\mathrm{Cov}}
\newcommand{\sign}{\mathrm{sign}}
\newcommand{\poly}{\mathrm{poly}}
\newcommand{\dis}{\mathrm{d}}
\newcommand{\dist}{\mathrm{dist}}
\newcommand{\Dist}{\mathrm{D}}
\newcommand{\binsketch}{\mathrm{BinSketch}}
\newcommand{\odd}{\mathrm{OddSketch}}
\newcommand{\simhash}{\mathrm{SimHash}}
\newcommand{\minhash}{\mathrm{MinHash}}
\newcommand{\RMSE}{\mathrm{RMSE}}
\newcommand{\simsketch}{\mathrm{Simsketch}}
%\newcommand{\G}{\mathcal{G}}
\newcommand{\R}{\mathbb{R}}
\renewcommand{\O}{\tilde{O}}

\newcommand{\Cos}{\mathrm{Cos}}
 \newtheorem{thm}{Theorem}
 \newtheorem{thm1}{Theorem}
% \newtheorem{theorem}{Theorem}
\newtheorem{lem}[thm]{Lemma}
 \newtheorem{prop}[thm]{Proposition}
%\newtheorem{def}[thm1]{Definition}
\newtheorem{defi}[thm]{Definition}
 %\newtheorem{remark}[theorem]{Remark}
%\newtheorem{note}[theorem]{Note}
\newtheorem{obs}[thm]{Observation}
\newtheorem{rem}[thm]{Remark}
\newtheorem{cor}[thm]{Corollary}

\newcommand{\fsketch}{{\tt FSketch}\xspace}
\newcommand{\minfsketch}{{\tt Median-FSketch}\xspace}

\newcommand{\bl}[1]{{#1}}

\usepackage[utf8]{inputenc} % allow utf-8 input
\usepackage{hyperref}       % hyperlinks
\usepackage{url}            % simple URL typesetting
\usepackage{booktabs}       % professional-quality tables
\usepackage{nicefrac}       % compact symbols for 1/2, etc.
\usepackage{paralist}
\usepackage{todonotes}


% *** Do not adjust lengths that control margins, column widths, etc. ***
% *** Do not use packages that alter fonts (such as pslatex).         ***
% There should be no need to do such things with IEEEtran.cls V1.6 and later.
% (Unless specifically asked to do so by the journal or conference you plan
% to submit to, of course. )

\begin{document}
%
% paper title
% Titles are generally capitalized except for words such as a, an, and, as,
% at, but, by, for, in, nor, of, on, or, the, to and up, which are usually
% not capitalized unless they are the first or last word of the title.
% Linebreaks \\ can be used within to get better formatting as desired.
% Do not put math or special symbols in the title.
\title{Dimensionality Reduction for Categorical Data}
%
%
% author names and IEEE memberships
% note positions of commas and nonbreaking spaces ( ~ ) LaTeX will not break
% a structure at a ~ so this keeps an author's name from being broken across
% two lines.
% use \thanks{} to gain access to the first footnote area
% a separate \thanks must be used for each paragraph as LaTeX2e's \thanks
% was not built to handle multiple paragraphs
%
%
%\IEEEcompsocitemizethanks is a special \thanks that produces the bulleted
% lists the Computer Society journals use for "first footnote" author
% affiliations. Use \IEEEcompsocthanksitem which works much like \item
% for each affiliation group. When not in compsoc mode,
% \IEEEcompsocitemizethanks becomes like \thanks and
% \IEEEcompsocthanksitem becomes a line break with idention. This
% facilitates dual compilation, although admittedly the differences in the
% desired content of \author between the different types of papers makes a
% one-size-fits-all approach a daunting prospect. For instance, compsoc 
% journal papers have the author affiliations above the "Manuscript
% received ..."  text while in non-compsoc journals this is reversed. Sigh.

\author{Debajyoti~Bera,
        Rameshwar~Pratap,
        and~Bhisham~Dev~Verma% <-this % stops a space
\IEEEcompsocitemizethanks{\IEEEcompsocthanksitem D. Bera is with the Department
of Computer Science and Engineering, Indraprastha Institute of Information Technology (IIIT-Delhi), New Delhi, India, 110020.\protect\\
% note need leading \protect in front of \\ to get a newline within \thanks as
% \\ is fragile and will error, could use \hfil\break instead.
E-mail: see http://www.michaelshell.org/contact.html
\IEEEcompsocthanksitem R. Pratap and B. D. Verma are with the Indian Institute of Technology, Mandi, Himachal Pradesh, India.\protect\\
\IEEEcompsocthanksitem Emails: dbera@iiitd.ac.in, rameshwar@iitmandi.ac.in and d18039@students.iitmandi.ac.in.}% <-this % stops an unwanted space
\thanks{Manuscript accepted for publication by IEEE Transactions on Knowledge and Data Engineering. Copyright 1969, IEEE.}}

% note the % following the last \IEEEmembership and also \thanks - 
% these prevent an unwanted space from occurring between the last author name
% and the end of the author line. i.e., if you had this:
% 
% \author{....lastname \thanks{...} \thanks{...} }
%                     ^------------^------------^----Do not want these spaces!
%
% a space would be appended to the last name and could cause every name on that
% line to be shifted left slightly. This is one of those "LaTeX things". For
% instance, "\textbf{A} \textbf{B}" will typeset as "A B" not "AB". To get
% "AB" then you have to do: "\textbf{A}\textbf{B}"
% \thanks is no different in this regard, so shield the last } of each \thanks
% that ends a line with a % and do not let a space in before the next \thanks.
% Spaces after \IEEEmembership other than the last one are OK (and needed) as
% you are supposed to have spaces between the names. For what it is worth,
% this is a minor point as most people would not even notice if the said evil
% space somehow managed to creep in.



% The paper headers
\markboth{Journal of \LaTeX\ Class Files,~Vol.~14, No.~8, August~2015}%
{Shell \MakeLowercase{\textit{et al.}}: Bare Demo of IEEEtran.cls for Computer Society Journals}
% The only time the second header will appear is for the odd numbered pages
% after the title page when using the twoside option.
% 
% *** Note that you probably will NOT want to include the author's ***
% *** name in the headers of peer review papers.                   ***
% You can use \ifCLASSOPTIONpeerreview for conditional compilation here if
% you desire.



% The publisher's ID mark at the bottom of the page is less important with
% Computer Society journal papers as those publications place the marks
% outside of the main text columns and, therefore, unlike regular IEEE
% journals, the available text space is not reduced by their presence.
% If you want to put a publisher's ID mark on the page you can do it like
% this:
%\IEEEpubid{0000--0000/00\$00.00~\copyright~2015 IEEE}
% or like this to get the Computer Society new two part style.
%\IEEEpubid{\makebox[\columnwidth]{\hfill 0000--0000/00/\$00.00~\copyright~2015 IEEE}%
%\hspace{\columnsep}\makebox[\columnwidth]{Published by the IEEE Computer Society\hfill}}
% Remember, if you use this you must call \IEEEpubidadjcol in the second
% column for its text to clear the IEEEpubid mark (Computer Society jorunal
% papers don't need this extra clearance.)



% use for special paper notices
%\IEEEspecialpapernotice{(Invited Paper)}



% for Computer Society papers, we must declare the abstract and index terms
% PRIOR to the title within the \IEEEtitleabstractindextext IEEEtran
% command as these need to go into the title area created by \maketitle.
% As a general rule, do not put math, special symbols or citations
% in the abstract or keywords.
\IEEEtitleabstractindextext{%
\begin{abstract}
Categorical attributes are those that can take a discrete set of values, e.g., colours. This work is about compressing vectors over categorical attributes to low-dimension discrete vectors. The current hash-based methods compressing vectors over categorical attributes to low-dimension discrete vectors do not provide any guarantee on the Hamming distances between the compressed representations. Here we present \fsketch to create sketches for sparse categorical data and an estimator to estimate the pairwise Hamming distances among the uncompressed data only from their sketches. We claim that these sketches can be used in the usual data mining tasks in place of the original data without compromising the quality of the task. For that, we ensure that the sketches also are categorical, sparse, and the Hamming distance estimates are reasonably precise. Both the sketch construction and the Hamming distance estimation algorithms require just a single-pass; furthermore, changes to a data point can be incorporated into its sketch in an efficient manner. The compressibility depends upon how sparse the data is and is independent of the original dimension -- making our algorithm attractive for many real-life scenarios. Our claims are backed by rigorous theoretical analysis of the properties of \fsketch and supplemented by extensive comparative evaluations with related algorithms on some real-world datasets. We show that \fsketch is significantly faster, and the accuracy obtained by using its sketches are among the top for the standard unsupervised tasks of $\mathrm{RMSE}$, clustering and similarity search.

\end{abstract}

% Note that keywords are not normally used for peerreview papers.
\begin{IEEEkeywords}
Dimensionality Reduction, Sketching, Feature Hashing, Clustering,  Classification, Similarity Search.
\end{IEEEkeywords}}


% make the title area
\maketitle


% To allow for easy dual compilation without having to reenter the
% abstract/keywords data, the \IEEEtitleabstractindextext text will
% not be used in maketitle, but will appear (i.e., to be "transported")
% here as \IEEEdisplaynontitleabstractindextext when the compsoc 
% or transmag modes are not selected <OR> if conference mode is selected 
% - because all conference papers position the abstract like regular
% papers do.
\IEEEdisplaynontitleabstractindextext
% \IEEEdisplaynontitleabstractindextext has no effect when using
% compsoc or transmag under a non-conference mode.



% For peer review papers, you can put extra information on the cover
% page as needed:
% \ifCLASSOPTIONpeerreview
% \begin{center} \bfseries EDICS Category: 3-BBND \end{center}
% \fi
%
% For peerreview papers, this IEEEtran command inserts a page break and
% creates the second title. It will be ignored for other modes.
\IEEEpeerreviewmaketitle

\section{Introduction}

% Why’s rec sys important

The proliferation of online content in recent years has created an overwhelming amount of information for users to navigate. 
To address this issue, recommender systems are employed  in various industries to help users find relevant items from a vast selection, including products, images, videos, and music. 
% , making it easier for them to find what they want. 
By providing personalized recommendations, businesses and organizations can better serve their users and keep them engaged with the platform. 
Therefore, recommender systems are vital for businesses as they drive growth by boosting engagement, sales, and revenue.

\begin{figure}[!ht]
  \centering
  \includegraphics[width=0.85\linewidth]{figures/Homefeed_web_square.png}
  % \caption{Illustration of generating Pinterest Homefeed page using a 3-stage RS}
  \caption{Pinterest Homefeed Page}
  \label{fig:hf}
  \Description{Pinterest Homefeed page}
\end{figure}

% Background of PINS Homefeed ranking. Previous PINS work (PinSage, Pinnerformer)
As one of the largest content sharing and social media platforms, Pinterest hosts billions of pins with rich contextual and visual information, and brings inspiration to over 400 million users. 
Upon visiting Pinterest, users are immediately presented with the Homefeed page as shown in Figure~\ref{fig:hf}, which serves as the primary source of inspiration and accounts for the majority of overall user engagement on the platform.
The Homefeed page is powered by a 3-stage recommender system that retrieves, ranks, and blends content based on user interests and activities. 
At the retrieval stage, we filter billions of pins created on Pinterest to thousands, based on a variety of factors such as user interests, followed boards, etc. Then we use a pointwise ranking model to rank candidate pins by predicting their personalized relevance to users. 
Finally, the ranked result is adjusted using a blending layer to meet business requirements.


% Existing work for realtime ranking
Realtime recommendation is crucial because it provides a quick and up-to-date recommendation to users, improving their overall experience and satisfaction. 
The integration of realtime data, such as recent user actions, results in more accurate recommendations and increases the probability of users discovering relevant items~\cite{alibaba_seq_tfmr, pi2020search}.

Longer user action sequences result in improved user representation and hence better recommendation performance. 
However, using long sequences in ranking poses challenges to infrastructure, as they require significant computational resources and can result in increased latency.
To address this challenge, some approaches have utilized hashing and nearest neighbor search in long user sequences~\cite{pi2020search}.
% Research in this area typically focuses on reducing the complexity of the system\cite{chatzopoulos2016readme}, handling high-velocity data streams\cite{6542425}, and dealing with dynamic changes in user behavior~\cite{grbovic2018real, 10.1145/2699670}. 
Other work encodes users' past actions over an extended time frame to a user embedding~\cite{pinnerformer} to represent long-term user interests. User embedding features are often generated as \textit{batch} features (e.g. generated daily), which are cost-effective to serve across multiple applications with low latency. 
The limitation of existing sequential recommendation is that they either only use realtime user actions, or only use a batch user representation learned from long-term user action history.


We introduce a novel realtime-batch hybrid ranking approach that combines both \textit{realtime} user action signals and \textit{batch} user representations. 
To capture the realtime actions of users, we present TransAct - a new transformer-based module designed to encode recent user action sequences and comprehend users' immediate preferences.
For user actions that occur over an extended period of time, we transform them into a batch user representation~\cite{pinnerformer}.

By combining the expressive power of TransAct with batch user embeddings, the hybrid ranking model offers users realtime feedback on their recent actions, while also accounting for their long-term interests. The realtime component and batch component complement each other for recommendation accuracy. This leads to an overall improvement in the user experience on the Homefeed page.
% Limitation of previous sequential recommendation/ novelty
% Sequential recommendation systems use a user's action history as input and apply recommendation algorithms to suggest appropriate items. Recent work~\cite{donkers2017sequential, hidasi2015session, tan2016improved, zhou2019deep, tang2018personalized, tuan20173d} has been using deep learning techniques, such as recurrent neural networks (RNNs)~\cite{rnn} and convolutional neural networks (CNNs)~\cite{cnn}, to process users' action history. Some studies~\cite{DIN, zhang2019next, alibaba_seq_tfmr, li2020time, SASRec, sun2019bert4rec} have also adopted the attention mechanism~\cite{tfmr} to model user action sequence features. 
% In this work, we also employ the self-attention mechanism for realtime user sequence modeling for its superior capability in encoding sequential inputs. 
% % One limitation of the existing sequential recommendation is that they
% To the best of our knowledge, this is the first method that builds a realtime-batch hybrid ranking model which uses users' action sequences in both realtime and batch settings. 
% We provide an in-depth analysis of the challenges posed by utilizing real-time user sequence features, such as the potential decrease in recommendation diversity and engagement decay.

The major contributions of this paper are summarized as follows:
\begin{itemize}
 \item We describe Pinnability, the architecture of Pinterest's Homefeed production ranking system. The Homefeed personalized recommendation product accounts for the majority of the overall user engagement on Pinterest. 

 \item  We propose TransAct, a transformer-based realtime user action sequential model that effectively captures users' short-term interests from their recent actions. We demonstrate that combining TransAct with daily-generated user representations~\cite{pinnerformer} to a hybrid model leads to the best performance in Pinnability. This design choice is justified through a comprehensive ablation study. Our code implementation is publicly available\footnote{Our code is available on Github: \url{https://github.com/pinterest/transformer_user_action}}.

\item We describe the serving optimization implemented in Pinnability to make feasible the computational complexity increase of 65 times when introducing TransAct to the Pinnability model. Specifically, optimizations are done to enable GPU serving of our prior CPU-based model.
\item We describe online A/B experiments on a real-world recommendation system using TransAct. We demonstrate some practical issues in the online environment, such as recommendation diversity drop and engagement decay, and propose solutions to address these issues.  
% \item Our model has been deployed as the Homefeed ranking model of Pinterest, one of the largest content sharing and social media platforms. As a result, it boosts the Homefeed repin\footnote{A "repin" on Pinterest refers to the action of saving an existing pin to another board by a user.} volume by 11\%.

\end{itemize}

% Structure of the paper
% \td{check}
The remainder of this paper is organized as follows: Related
work is reviewed in Section~\ref{sec:related_work}. Section~\ref{sec:method} describes the design of TransAct and the details of bringing it to production. Experiment results are reported in Section~\ref{sec:exp}. We discuss some findings beyond experiments in Section~\ref{sec:discussion}. Finally, we conclude our work in Section~\ref{sec:conclusion}.
% Learnings









% \begin{itemize}
% \item Hybrid setup
% \item Ablation study: Using both positive and negative actions, Early fusion with candidate pin
% \item Diversity: time window mask
% \item Frequent retraining
% \item is good for cold-start users (non-core)
% \end{itemize}


% 1. Hybrid setup
% w. No extensive ablation study on the model architecture Model: early fusion, random time window.
% Did not discuss practical challenge: latency, GPU serving
% Ali’s Search based user seq does have practical challenge
% id feature only (but we don’t have img_sig feature)	
% 3. Diversity


% Most existing recommender systems use a wide and deep architecture, where the inputs are user features, item features and context features. The model usually learns the relevance score between the user and item over a given context. The traditional way to get user or item features are mostly through feature engineering. And then the model use the hand-picked features to predict the relevance between user and items. Raw features are usually very limited in the amount of information that they can represent. More advanced techniques include learning pre-trained item or user embedding, where it can use a reasonable amount of space to represent richer information. Some successful pre-trained embeddings are Pinnerformer\cite{pinnerformer}, itemsage\cite{itemsage}, etc. However, one of the major shortcomings of pre-trained embedding is that they are usually expensive to infer in realtime due to their high computational cost and infra complexity. 
%% !TEX root = ../arxiv.tex

Unsupervised domain adaptation (UDA) is a variant of semi-supervised learning \cite{blum1998combining}, where the available unlabelled data comes from a different distribution than the annotated dataset \cite{Ben-DavidBCP06}.
A case in point is to exploit synthetic data, where annotation is more accessible compared to the costly labelling of real-world images \cite{RichterVRK16,RosSMVL16}.
Along with some success in addressing UDA for semantic segmentation \cite{TsaiHSS0C18,VuJBCP19,0001S20,ZouYKW18}, the developed methods are growing increasingly sophisticated and often combine style transfer networks, adversarial training or network ensembles \cite{KimB20a,LiYV19,TsaiSSC19,Yang_2020_ECCV}.
This increase in model complexity impedes reproducibility, potentially slowing further progress.

In this work, we propose a UDA framework reaching state-of-the-art segmentation accuracy (measured by the Intersection-over-Union, IoU) without incurring substantial training efforts.
Toward this goal, we adopt a simple semi-supervised approach, \emph{self-training} \cite{ChenWB11,lee2013pseudo,ZouYKW18}, used in recent works only in conjunction with adversarial training or network ensembles \cite{ChoiKK19,KimB20a,Mei_2020_ECCV,Wang_2020_ECCV,0001S20,Zheng_2020_IJCV,ZhengY20}.
By contrast, we use self-training \emph{standalone}.
Compared to previous self-training methods \cite{ChenLCCCZAS20,Li_2020_ECCV,subhani2020learning,ZouYKW18,ZouYLKW19}, our approach also sidesteps the inconvenience of multiple training rounds, as they often require expert intervention between consecutive rounds.
We train our model using co-evolving pseudo labels end-to-end without such need.

\begin{figure}[t]%
    \centering
    \def\svgwidth{\linewidth}
    \input{figures/preview/bars.pdf_tex}
    \caption{\textbf{Results preview.} Unlike much recent work that combines multiple training paradigms, such as adversarial training and style transfer, our approach retains the modest single-round training complexity of self-training, yet improves the state of the art for adapting semantic segmentation by a significant margin.}
    \label{fig:preview}
\end{figure}

Our method leverages the ubiquitous \emph{data augmentation} techniques from fully supervised learning \cite{deeplabv3plus2018,ZhaoSQWJ17}: photometric jitter, flipping and multi-scale cropping.
We enforce \emph{consistency} of the semantic maps produced by the model across these image perturbations.
The following assumption formalises the key premise:

\myparagraph{Assumption 1.}
Let $f: \mathcal{I} \rightarrow \mathcal{M}$ represent a pixelwise mapping from images $\mathcal{I}$ to semantic output $\mathcal{M}$.
Denote $\rho_{\bm{\epsilon}}: \mathcal{I} \rightarrow \mathcal{I}$ a photometric image transform and, similarly, $\tau_{\bm{\epsilon}'}: \mathcal{I} \rightarrow \mathcal{I}$ a spatial similarity transformation, where $\bm{\epsilon},\bm{\epsilon}'\sim p(\cdot)$ are control variables following some pre-defined density (\eg, $p \equiv \mathcal{N}(0, 1)$).
Then, for any image $I \in \mathcal{I}$, $f$ is \emph{invariant} under $\rho_{\bm{\epsilon}}$ and \emph{equivariant} under $\tau_{\bm{\epsilon}'}$, \ie~$f(\rho_{\bm{\epsilon}}(I)) = f(I)$ and $f(\tau_{\bm{\epsilon}'}(I)) = \tau_{\bm{\epsilon}'}(f(I))$.

\smallskip
\noindent Next, we introduce a training framework using a \emph{momentum network} -- a slowly advancing copy of the original model.
The momentum network provides stable, yet recent targets for model updates, as opposed to the fixed supervision in model distillation \cite{Chen0G18,Zheng_2020_IJCV,ZhengY20}.
We also re-visit the problem of long-tail recognition in the context of generating pseudo labels for self-supervision.
In particular, we maintain an \emph{exponentially moving class prior} used to discount the confidence thresholds for those classes with few samples and increase their relative contribution to the training loss.
Our framework is simple to train, adds moderate computational overhead compared to a fully supervised setup, yet sets a new state of the art on established benchmarks (\cf \cref{fig:preview}).

%\section{Background and Motivation}

\subsection{IBM Streams}

IBM Streams is a general-purpose, distributed stream processing system. It
allows users to develop, deploy and manage long-running streaming applications
which require high-throughput and low-latency online processing.

The IBM Streams platform grew out of the research work on the Stream Processing
Core~\cite{spc-2006}.  While the platform has changed significantly since then,
that work established the general architecture that Streams still follows today:
job, resource and graph topology management in centralized services; processing
elements (PEs) which contain user code, distributed across all hosts,
communicating over typed input and output ports; brokers publish-subscribe
communication between jobs; and host controllers on each host which
launch PEs on behalf of the platform.

The modern Streams platform approaches general-purpose cluster management, as
shown in Figure~\ref{fig:streams_v4_v6}. The responsibilities of the platform
services include all job and PE life cycle management; domain name resolution
between the PEs; all metrics collection and reporting; host and resource
management; authentication and authorization; and all log collection. The
platform relies on ZooKeeper~\cite{zookeeper} for consistent, durable metadata
storage which it uses for fault tolerance.

Developers write Streams applications in SPL~\cite{spl-2017} which is a
programming language that presents streams, operators and tuples as
abstractions. Operators continuously consume and produce tuples over streams.
SPL allows programmers to write custom logic in their operators, and to invoke
operators from existing toolkits. Compiled SPL applications become archives that
contain: shared libraries for the operators; graph topology metadata which tells
both the platform and the SPL runtime how to connect those operators; and
external dependencies. At runtime, PEs contain one or more operators. Operators
inside of the same PE communicate through function calls or queues. Operators
that run in different PEs communicate over TCP connections that the PEs
establish at startup. PEs learn what operators they contain, and how to connect
to operators in other PEs, at startup from the graph topology metadata provided
by the platform.

We use ``legacy Streams'' to refer to the IBM Streams version 4 family. The
version 5 family is for Kubernetes, but is not cloud native. It uses the
lift-and-shift approach and creates a platform-within-a-platform: it deploys a
containerized version of the legacy Streams platform within Kubernetes.

\subsection{Kubernetes}

Borg~\cite{borg-2015} is a cluster management platform used internally at Google
to schedule, maintain and monitor the applications their internal infrastructure
and external applications depend on. Kubernetes~\cite{kube} is the open-source
successor to Borg that is an industry standard cloud orchestration platform.

From a user's perspective, Kubernetes abstracts running a distributed
application on a cluster of machines. Users package their applications into
containers and deploy those containers to Kubernetes, which runs those
containers in \emph{pods}. Kubernetes handles all life cycle management of pods,
including scheduling, restarting and migration in case of failures.

Internally, Kubernetes tracks all entities as \emph{objects}~\cite{kubeobjects}.
All objects have a name and a specification that describes its desired state.
Kubernetes stores objects in etcd~\cite{etcd}, making them persistent,
highly-available and reliably accessible across the cluster. Objects are exposed
to users through \emph{resources}. All resources can have
\emph{controllers}~\cite{kubecontrollers}, which react to changes in resources.
For example, when a user changes the number of replicas in a
\code{ReplicaSet}, it is the \code{ReplicaSet} controller which makes sure the
desired number of pods are running. Users can extend Kubernetes through
\emph{custom resource definitions} (CRDs)~\cite{kubecrd}. CRDs can contain
arbitrary content, and controllers for a CRD can take any kind of action.

Architecturally, a Kubernetes cluster consists of nodes. Each node runs a
\emph{kubelet} which receives pod creation requests and makes sure that the
requisite containers are running on that node. Nodes also run a
\emph{kube-proxy} which maintains the network rules for that node on behalf of
the pods. The \emph{kube-api-server} is the central point of contact: it
receives API requests, stores objects in etcd, asks the scheduler to schedule
pods, and talks to the kubelets and kube-proxies on each node. Finally,
\emph{namespaces} logically partition the cluster. Objects which should not know
about each other live in separate namespaces, which allows them to share the
same physical infrastructure without interference.

\subsection{Motivation}
\label{sec:motivation}

Systems like Kubernetes are commonly called ``container orchestration''
platforms. We find that characterization reductive to the point of being
misleading; no one would describe operating systems as ``binary executable
orchestration.'' We adopt the idea from Verma et al.~\cite{borg-2015} that
systems like Kubernetes are ``the kernel of a distributed system.'' Through CRDs
and their controllers, Kubernetes provides state-as-a-service in a distributed
system. Architectures like the one we propose are the result of taking that view 
seriously.

The Streams legacy platform has obvious parallels to the Kubernetes
architecture, and that is not a coincidence: they solve similar problems.
Both are designed to abstract running arbitrary user-code across a distributed
system.  We suspect that Streams is not unique, and that there are many
non-trivial platforms which have to provide similar levels of cluster
management.  The benefits to being cloud native and offloading the platform
to an existing cloud management system are: 
\begin{itemize}
    \item Significantly less platform code.
    \item Better scheduling and resource management, as all services on the cluster are 
        scheduled by one platform.
    \item Easier service integration.
    \item Standardized management, logging and metrics.
\end{itemize}
The rest of this paper presents the design of replacing the legacy Streams 
platform with Kubernetes itself.



\section{Case Studies}
\label{sec:case_studies}
In this section, we present a case study of Facebook posts from an Australian public page.
The page shifts between early 2020 (\emph{2019-2020 Australian bushfire season}) and late 2020 (\emph{COVID-19 crises}) from being a moderate-right group for discussion around climate change to a far-right extremist group for conspiracy theories.


\begin{figure*}[!tbp]
	\begin{subfigure}{0.21\textwidth}
		\includegraphics[width=\textwidth]{images/facebook1.png}
		\caption{}
		\label{subfig:first-posting}
		\includegraphics[width=0.9\textwidth]{images/facebook3.jpg}
		\caption{}
		\label{subfig:comment-post-1}
	\end{subfigure}
    \begin{subfigure}{0.28\textwidth}
		\includegraphics[width=\textwidth]{images/facebook2.jpg}
		\caption{}
		\label{subfig:second-posting}
	\end{subfigure}
    \begin{subfigure}{0.23\textwidth}
		\includegraphics[width=\textwidth]{images/facebook4.jpg}
		\caption{}
		\label{subfig:comment-post-2a}
	\end{subfigure}
    \begin{subfigure}{0.23\textwidth}
		\includegraphics[width=\textwidth]{images/facebook5.jpg}
		\caption{}
		\label{subfig:comment-post-2b}
	\end{subfigure}
	\caption{
		Examples of postings and comment threads from a public Facebook page from two periods of time early 2020 (a) and late 2020 (b)-(e), which show a shift from climate change debates to extremist and far-right messaging.
	}
	\label{fig:facebook}
\end{figure*}

We focus on a sample of 2 postings and commenting threads from one Australian Facebook page we classified as ``far-right'' based on the content on the page. 
We have anonymized the users in \Cref{fig:facebook} to avoid re-identification.
The first posting and comment thread (see \Cref{subfig:first-posting}) was collected on Jan 10, 2020, and responded to the Australian bushfire crisis that began in late 2019 and was still ongoing in January 2020. It contains an ambivalent text-based provocation that references disputes in the community regarding the validity of climate change and climate science. 

The second posting and comment thread (see \Cref{subfig:second-posting}) was collected from the same page in September 2020, months after the bushfire crisis had abated.
At that time, a new crisis was energizing and connecting the far-right groups in our dataset --- i.e., the COVID-19 pandemic and the government interventions to curb the spread of the virus. 
The post is different in style compared to the first.
It is image-based instead of text-based and highly emotive, with a photo collage bringing together images of prison inmates with iron masks on their faces (top row) juxtaposed to people wearing face masks during COVID-19 (bottom row). 
The image references the public health orders issued during Melbourne's second lockdown and suggests that being ordered to wear masks is an infringement of citizen rights and freedoms, similar to dehumanizing restraints used on prisoners.

To analyze reactions to the posts, two researchers used a deductive analytical approach to separately code and to analyze the commenting threads --- see \Cref{subfig:comment-post-1} for comments of the first posting, and \Cref{subfig:comment-post-2a,subfig:comment-post-2b} for comments on the second posting. 
Conversations were also inductively coded for emerging themes. 
During the analysis, we observed qualitative differences in the types of content users posted, interactions between commenters, tone and language of debate, linked media shared in the commenting thread, and the opinions expressed.
The rest of this section further details these differences.
To ensure this was not a random occurrence, we tested the exemplar threads against field notes collected on the group during the entire study.
We also used Facebook's search function within pages to find a sample of posts from the same period and which dealt with similar topics. 
After this analysis, we can confidently say that key changes occurred in the group between the bushfire crisis and COVID-19, that we detail next.

\subsubsection*{Exemplar 1 --- climate change skepticism.}
To explore this transformation in more depth, we analyzed comments scraped on the first posting --- \cref{sub@subfig:comment-post-1} shows a small sample of these comments.
The language used was similar to comments that we observed on numerous far-right nationalist pages at the time of the bushfires.
These comments are usually text-based, employing emojis to denote emotions, and sometimes being mocking or provocative in tone. 
Noteworthy for this commenting thread is the 50/50 split in the number of members posting in favor of action on climate change (on one side) and those who posted anti-Greens and anti-climate change science posts and memes (on the other side).
The two sides aligned strongly with political partisanship --- either with Liberal/National coalition (climate change deniers) or Labor/Green (climate change believers) parties. 
This is rather unusual for pages classified as far-right. 

We observed trolling practices between the climate change deniers and believers, which often descend into \emph{flame wars} --- i.e., online ``firefights that take place between disembodied combatants on electronic bulletin boards''~\citep{bukatman1994flame}.
The result is a boosted engagement on the post but also the frustration and confusion of community members and lurkers who came to the discussions to become informed or debate rationally on key differences between the two positions.
They often even become targeted, victimized, and baited by trolls on both sides of the partisan divide. 
The opinions expressed by deniers in commenting sections range from skepticism regarding climate change science to plain denial.
Deniers also regard a range of targets as embroiled in a climate change conspiracy to deceive the public, such as The Greens and their environmental policy, in some cases the government, the United Nations, and climate change celebrities like David Attenborough and Greta Thunberg. 
These figures are blamed for either exaggerating risks of climate change or creating a climate change hoax to increase the influence of the UN on domestic governments or to increase domestic governments' social control over citizens. 

Both coders noted that flame wars between these opposing personas contained very few links to external media. 
Where links were added, they often seemed disconnected from the rest of the conversation and were from users whose profiles suggested they believed in more radical conspiracy theories.
One such example is ``geo-engineering'' (see \cref{sub@subfig:comment-post-1}).
Its adherents believe that solar geo-engineering programs designed to combat climate change are secretly used by a global elite to depopulate the world through sterilization or to control and weaponize the weather.

Nonetheless, apart from the random comments that hijack the thread, redirecting users to external ``alternative'' news sites and Twitter, and the trolls who seem to delight in victimizing unsuspecting victims, the discussion was pretty healthy.
There are many questions, rational inquiries, and debates between users of different political persuasion and views on climate change.
This, however, changes in the span of only a couple of months.

\subsubsection*{Exemplar 2 --- posting and commenting thread.}
We observe a shift in the comment section of the post collected during the second wave of the COVID pandemic (\Cref{sub@subfig:second-posting}) --- which coincided with government laws mandating the public to wear masks and stay at home in Victoria, Australia.
There emerges much more extreme far-right content that converges with anti-vaccination opinions and content.
We also note a much higher prevalence of conspiracy theories often implicating racialized targets.
This is exemplified in the comments on the second post (\Cref{sub@subfig:comment-post-2a,sub@subfig:comment-post-2b}) where Islamophobia and antisemitism are confidently asserted alongside anti-mask rhetoric.
These comments consider face masks similar to the religious head coverings worn by some Muslim women, which users describe as ``oppressive'' and ``silencing''. 
In this way, anti-maskers cast women as a distinct, sympathetic marginalized demographic.
However, this is enacted alongside the racialization and demonization of Islam as an oppressive religion. 

Given the extreme racialization of anti-mask rhetoric, some commenters contest these positions, arguing that the page is becoming less an anti-Scott Morrison page (Australia's Prime Minister at the time) and changing into a page that harbors ``far-right dickheads''.
This questioning is actively challenged by far-right commenters and conspiracy theorists on the page, who regarded pro-mask users and the Scott Morrison government as ``puppets'' being manipulated by higher forces (see \Cref{sub@subfig:comment-post-2b}). 

This indicates a significant change on the page's membership towards the extreme-right, who employs more extreme forms of racialized imagery, with more extreme opinion being shared.
Conspiracy theorists become more active and vocal, and they consistently challenge the opinions of both center conservative and left-leaning users. 
This is evident in the final two comments in \Cref{subfig:comment-post-2b}, which reflect QAnon style conspiracy theories and language.
Public health orders to wear masks are being connected to a conspiracy that all of these decisions are directed by a secret network of global Jewish elites, who manipulate the pandemic to increase their power and control. 
This rhetoric intersects with the contemporary ``QAnon'' conspiracy theory, which evolved from the ``Pizzagate'' conspiracy theory.
They also heavily draw on well-established antisemitic blood libel conspiracy theories, which foster beliefs that a powerful global elite is controlling the decisions of organizations such as WHO and are responsible for the vaccine rollout and public health orders related to the pandemic.
The QAnon conspiracy is also influenced by Bill Gates' Microchips conspiracy theory, i.e., the theory that the WHO and the Bill Gates Foundation global vaccine programs are used to inject tracking microchips into people.

These conspiracy theories have, since COVID-19, connected formerly separate communities and discourses, uniting existing anti-vaxxer communities, older demographics who are mistrustful of technology, far-right communities suspicious of global and national left-wing agendas, communities protesting against 5G mobile networks (for fear that they will brainwash, control, or harm people), as well as generating its own followers out of those anxious during the 2020 onset of the COVID-19 pandemic.
We detect and describe some of these opinion dynamics in the next section.

\section{Experiments}\label{sec:experiments}

We performed our experiments on a machine having {Intel(R) Xeon(R) CPU E5-2650 v3 @ 2.30GHz, 94 GB RAM, and running a Ubuntu 64-bits OS.}
%\textcolor{red}{@Deb can we cook up some story on the dataset we used for experiments.. something like existing categorical datasets for experimentation as they are low-dimensional.. also BrainCell I was not able to find what to mention..}

%\subsection{Evaluation criteria}

We first study the effect of the internal parameters of our proposed solution on its performance. We start with the effect of the prime number $p$; then we compare \fsketch with the appropriate baselines for several unsupervised data-analytic tasks (see Table~\ref{tab:baseline_characteristic}) and objectively establish these advantages of \fsketch over the others.

% \minfsketch requires three parameters --- arity $k$, dimension $d$, and a prime number $p$. We first empirically evaluate the effect of $p$ and $k$ on its performance. Then we study the effect of $d$ on the behavior of arity-1 \minfsketch ({\it i.e.}, \fsketch) when used for data analytic tasks like $\RMSE$, clustering, similarity search and classification. For all these tasks, we objectively establish these advantages of \fsketch over the others.
\begin{enumerate}[(a)]
    \item Significant speed-up in the dimensionality reduction time,
    \item considerable savings in the time for the end-tasks (e.g., clustering) which now runs on the low-dimensional sketches,%perform these tasks on our sketches with much smaller dimensions.
    \item but with comparable accuracy of the end-tasks (e.g., clustering).
\end{enumerate}

Several baselines threw {\it out-of-memory} errors or did not stop on certain datasets. We discuss the errors separately in Section~\ref{sec:appendix_oom_error} in Appendix.

\subsection{Dataset description}
The efficacy of our solution is best described for high-dimensional datasets. Publicly available categorical datasets being mostly low-dimensional, we treated several integer-valued freely available real-world datasets as categorical. Our empirical evaluation was done on the following seven such datasets with dimensions between $5000$ and $1.3$ million, and sparsity from $0.07\%$ to 30\%.%   We mention their description as follows: 
\begin{itemize}
    \item \texttt{Gisette Data Set}~\cite{UCI,Gisette}: %This dataset consists of handwritten digits, and have been used for classification problem where the aim is to separate the highly confusible digits '4' and '9'.
	This dataset consists of integer feature vectors corresponding to images of handwritten digits and was constructed from the MNIST data. Each image, of $28 \times 28$ pixels, has been pre-processed (to retain the pixels necessary to disambiguate the digit $4$ from $9$) and then projected onto a higher-dimensional feature space represented to construct a 5000-dimension integer vector. %The digits have been size-normalized and centered in a fixed-size image of dimension 28x28. From the images pixels were sampled at random so that it contains the information necessary to disambiguate the digit $4$ from $9$. Further higher order features were created to project the problem into a higher dimensional feature space.  
   
    % \textcolor{red}{This dataset consits of bag of word respresentation of tagged web pages and categries are the frequency of highly frequent word.}
    \item \texttt{BoW (Bag-of-words)}~\cite{UCI,DeliciousMIL}: We consider the following five corpus -- NIPS full papers, KOS blog entries, Enron Emails, NYTimes news articles, and tagged web pages from the social bookmarking site \texttt{delicious.com.} These datasets are ``BoW"(Bag-of-words) representations of the corresponding text corpora. In all these datasets, the attribute takes integer values which we consider as categories.
    %\textcolor{red}{For Bag-of-words datasets the categories are the frequency of highly frequent word in the dataset.}
     %\item \texttt{DeliciousMIL}~\cite{UCI,DeliciousMIL}: This dataset consists of a subset of tagged web pages from the social bookmarking site \texttt{delicious.com.} This dataset consists of a bag of word representation of tagged web pages and we consider categories as the frequency of the corresponding words.
    \item \texttt{1.3 Million Brain Cell Dataset}~\cite{genomics20171}: This dataset contains the result of a single cell RNA-sequencing \texttt{(scRNA-seq)} of $1.3$ million cells captured and sequenced from an \texttt{E18.5} mouse brain~\footnote{\url{https://support.10xgenomics.com/single-cell-gene-expression/datasets/1.3.0/1M_neurons}}. Each gene represents a data point and for every gene, the dataset stores the read-count of that gene corresponding to each cell -- these read-counts form our features.
\end{itemize}
%We chose the last dataset due to its very high dimension and the earlier ones due to their popularity in dimensionality-reduction experiments. We observed in our initial experiments that many baselines suffer from \textit{``out-of-memory” (OOM)} and \textit{``did not stop” (DNS)} even after running them for sufficiently long time. Hence we decided to conduct our experiments on a random sample of $2000$ points from each dataset.
%We summarise the   dimensionality, the number of categories, and the sparsity of these datasets in Table~\ref{tab:datasets}.

We chose the last dataset due to its very high dimension and the earlier ones due to their popularity in dimensionality-reduction experiments. \bl{We consider all the data points for KOS, Enron, Gisette, DeliciousMIL, a $10,000$ sized sample for NYTimes, and a $2000$ sized samples for BrainCell}. We summarise the dimensionality, the number of categories, and the sparsity of these datasets in the Table~\ref{tab:datasets}.



%{
% Note that if we include all points in of our datasets, then many baselines give \textit{``out-of-memory” (OOM)} and \textit{``did not stop” (DNS)} even after running them for sufficiently long time. Therefore, we decided to include a random sample of $2000$ points in our experiments, to get the  experimental results for most of them. 
%  We consider a uniform sample of size $2000$ for each of the experiment. 
%{\color{red} Why 2000 data points -- several baselines gave DNS, OOM errors on entire dataset so we use 2000 samples. Explain the datasets -- what are categories in them?}
\begin{table}
%\footnotesize
\centering
  \caption{ Datasets}
  %\vspace{-2mm}
  \label{tab:datasets}
%   \addtolength\tabcolsep{-4pt}
  \resizebox{0.48\textwidth}{!}{%
  \noindent\begin{tabular}{lcccc}
  % \toprule
  \hline
     Datasets & Categories & Dimension & Sparsity & \bl{No. of points} \\
  %  \midrule
  \hline
    \textrm{Gisette}~\cite{Gisette,UCI} & $999$ & $5000$ & $1480$&$\bl{13500}$\\
    \textrm{Enron Emails}~\cite{UCI} & $150$ & $28102$ & $2021$&$\bl{39861}$\\
    \textrm{DeliciousMIL}~\cite{DeliciousMIL,UCI} & $58$ & $8519$ & $200$&$\bl{12234}$\\
    \raggedright\textrm{NYTimes articles}~\cite{UCI} & $114$ & $102660$ & $ 871$&$\bl{10000}$\\
    \bl{\textrm{NIPS full papers}~\cite{UCI} }& \bl{$132$} & \bl{$12419$} & \bl{$914$} & \bl{$1500$}\\
    \raggedright\textrm {KOS blog entries}~\cite{UCI} & $42$ & $6906$ & $457$&$\bl{3430}$\\
    \textrm {Million Brain Cells from E18 Mice}~\cite{genomics20171} & $2036$ & $1306127$ & $1051$&$\bl{2000}$\\
  % PubMed abstracts:  & $10000$ & $141043$ & $ $\\
  \hline
%  \bottomrule
\end{tabular}%
}
%   \addtolength\tabcolsep{4pt}
\end{table}
 
%We performed our experiments on a 10-core 32 GB Windows 10 server.
%\noindent\textbf{Datasets and Baseline methods:}
%\footnote{We defer our discussion on the reproducibility details of the experiments presented in this section to appendix (Section~\ref{sec:Reproducibility}).
%We chose six datasets with dimensions between 5000 to $1.3$ million, $0.08\%$to 30\% sparsity and 42 to 2036 categories and sampled the initial $2000$  data points for our experiments.


\begin{table}[t]
\centering
  \caption{ 13 baselines}
  %\vspace{-2mm}
  \label{tab:baselines_methods}
\begin{tabular}[t]{rrl}
\hline
%    & { 13 baseline algorithms}\\
% \hline
1.&  {\rm SSD} &\textrm{Sketching via Stable Distribution}~\cite{SSD}\\
2.&    {\rm OHE} & \textrm{One Hot Encoding+BinSketch~\cite{ICDM}}\\
3.&    {\rm FH} & \textrm{Feature Hashing}~\cite{WeinbergerDLSA09}\\
4.&    {\rm SH} & \textrm{Signed-random projection/SimHash}~\cite{simhash}\\
5.&      {\rm KT} & \textrm{Kendall rank correlation coefficient}~\cite{kendall1938measure}\\
6.&    {\rm LSA} & \textrm{Latent Semantic Analysis}~\cite{LSI}\\
7.&    {\rm LDA} & \textrm{Latent Dirichlet Allocation}~\cite{LDA}\\
8.&    {\rm MCA} & \textrm{Multiple Correspondence Analysis}~\cite{MCA}\\
9.&    {\rm NNMF} & \textrm{Non-neg. Matrix Factorization}~\cite{NNMF}\\
10.&    {\rm PCA} & Vanilla \textrm{Principal component analysis}\\
11.&    \bl{ {\rm VAE}} &  \bl{\textrm{Variational autoencoder}}~\cite{Kingma2014}\\
12.&    \bl{ {\rm CATPCA}} &  \bl{\textrm{Categorical PCA}}~\cite{Sulc2015DimensionalityRO}\\
13.&    \bl{ {\rm HCA}} &  \bl{\textrm{Hierarchical Cluster Analysis}}~\cite{Sulc2015DimensionalityRO}\\
    % $\mathrm{\chi^2}$ & $\chi^2$-\textrm{feature selection}~\cite{chi_square}\\
    % {\rm MI} & \textrm{Mutual information feature selection}~\cite{MI}\\
    
\hline
\end{tabular}
    %\vspace*{-5mm}
\end{table}



% \hspace*{-1cm}
% \setlength{\tabrowsep}{6pt}
\begin{table*}[!t]
\centering
\caption{{ Summarisation of the baselines. } }\label{tab:baseline_characteristic}
%\vspace{-2mm}
% \scalebox{1}{
\resizebox{\textwidth}{!}{%
  %\begin{tabular}{ c|c|c|c|c|c|c|c|c|c|c|r}%\label{tab:baseline_characteristic}
  % IEEE does not like inter-column vertical lines
  \begin{threeparttable}
  \begin{tabular}{Slcccccccccccccc}
    \toprule
    Characteristics&FSketch&FH&SH&SSD&OHE&KT&NNMF&MCA&LDA&LSA&PCA&{VAE}&{CATPCA}&{HCA}\\
    \midrule
\pbox{5em}{Output\\ discrete\\ sketch}&\cmark&\cmark&\cmark&\xmark&\cmark&\cmark&\xmark&\xmark&\xmark&\xmark&\xmark&\xmark &\xmark&\cmark\\[2em]
%\hline
\pbox{5em}{Output\\ real-valued\\ sketch}&\xmark&\xmark&\xmark&\cmark&\xmark&\xmark&\cmark&\cmark&\cmark&\cmark&\cmark&\cmark&\cmark&\xmark\\[2em]
%\hline
\pbox{5em}{Approximating\\ distance\\ measure}&{Hamming}&\pbox{4em}{ Dot\\product}& {Cosine}& {Hamming}&{Hamming}& \texttt{NA} &\texttt{NA} &\texttt{NA} &\texttt{NA} & \texttt{NA}& \texttt{NA} & \texttt{NA} &\texttt{NA}&\texttt{NA}\\[2em]
%&\texttt{distance}&\texttt{ product}& \texttt{Similarity}&  & & & & & & & \\
%\hline
\pbox{5em}{Require \\labelled data}&\xmark&\xmark&\xmark&\xmark&\xmark&\xmark&\xmark&\xmark&\xmark&\xmark&\xmark&\xmark&\xmark&\xmark\\[2em]
%\hline
\pbox{5em}{Dependency\\ on the size\\ of sample \tnote{*}} &\xmark&\xmark&\xmark&\xmark&\xmark&\xmark&\xmark &\cmark&\xmark&\cmark &\cmark&\xmark&\cmark&\xmark\\[2em]
%\hline 
\pbox{5em}{End tasks\\ comparison}&All&All&All&All&All&All& \pbox{5em}{Clustering,\\ Similarity Search}
&\pbox{5em}{Clustering,\\ Similarity Search}
&\pbox{5em}{Clustering,\\ Similarity Search}
&\pbox{5em}{Clustering,\\ Similarity Search}
&\pbox{5em}{Clustering,\\ Similarity Search}
&\pbox{5em}{Clustering,\\ Similarity Search}
&\pbox{5em}{Clustering,\\ Similarity Search}&\pbox{5em}{Clustering,\\ Similarity Search}\\
\bottomrule
 \end{tabular}
 \begin{tablenotes}
 \item[*] The size of the maximum possible reduced dimension is the minimum of the number of data points and the dimension.
 \end{tablenotes}
 \end{threeparttable}
}
\end{table*}

\subsection{Baselines}
Recall that \fsketch (hence \minfsketch) {\em compresses
categorical vectors to shorter categorical vectors in an unsupervised manner that	``preserves'' Hamming distances}. %We use the following baselines for comparison with \fsketch.   

Our first baseline is based on one-hot-encoding (OHE) which is one of the most common methods to convert categorical data to a numeric vector and can approximate pairwise Hamming distance (refer to Appendix~\ref{appendix:OHE+BS}). Since OHE actually increases the dimension to very high levels (e.g., the dimension of the binary vectors obtained by encoding the NYTimes dataset is $11,703,240$), the best way to use it is by further compressing the one-hot encoded vectors. For empirical evaluation we applied BinSketch~\cite{ICDM} which is the state-of-the-art binary-to-binary dimensionality reduction technique that preserves Hamming distance. We refer to the entire process of OHE followed by BinSketch  simply by OHE in the rest of this section.

%We use \textit{one-hot-encoding}+ BinSketch~\cite{ICDM} (abbreviated as OHE) to compress high-dimensional categorical vectors into low-dimensional binary vectors that approximate the corresponding original pairwise Hamming distance.  We first use \textit{one-hot-encoding} to convert categorical features into binary vectors which exactly estimate the corresponding pairwise Hamming distance. Then we can apply BinSketch on top of binary vectors obtained from \textit{one-hot-encoding} to compress them into low-dimensional binary vectors which  estimate the pairwise Hamming distance between binary vectors, and as a  consequence the corresponding pairwise Hamming distance on original categorical vectors. A major limitation of OHE is that binary vectors obtained from {one-hot-encoding} can be  very high-dimensional  as each categorical feature is converted into a $c$-dimensional binary vector, where $c$ is possible number of categories of that feature. {\color{red} To emphasize this, for BrainCell and NYTimes datasets,  the dimension of binary vectors obtained after one-hot-encoding becomes $2,659,274,572$ and , respectively.} This may lead to slow running time and computation challenges.

To the best of our knowledge, there is no sketching algorithm other than OHE that compresses high-dimensional categorical vectors to low-dimensional categorical (or integer) vectors that preserves the original pairwise Hamming distances. Hence, we chose as baseline state-of-the-art and popularly employed algorithms that either preserve Hamming distance or output discrete-valued sketches (preserving some other similarity measure).  
We list them in Table~\ref{tab:baselines_methods} and  tabulate their characteristic in Table~\ref{tab:baseline_characteristic}. Their implementation details are discussed in Appendix~\ref{sec:Reproducibility_baseline}.


%  We use Sketching via Stable Distribution (SSD)~\cite{SSD} that compresses high-dimensional real valued vectors to low-dimensional real valued vectors which closely approximates the pairwise Hamming distance. 


\begin{figure*}
\centering
% \includegraphics[scale = 0.22]{images/different_p/RMSE.png}
\includegraphics[width=\linewidth]{images/RMSE_different_p_new.pdf}
%\vspace{-4mm}
\caption{{Comparison of $\RMSE$ measure obtained from \texttt{FSketch} algorithm on various choices of  $p$. \bl{Values of $c$ for NIPS, Enron, NYTimes, and GISETTE are 132, 150, 114, and 999, respectively.}
}}
\label{fig:varying_p}
\end{figure*}
 
 
%  However, 
% we include sketching algorithms as baselines that output discrete valued sketches which  preserve other similarity measures. We use Feature Hashing (FH)~\cite{WeinbergerDLSA09} and signed-random-projection (SRP)  (alternatively known as SimHash (SH))~\cite{simhash}, which compress real-valued vectors and approximate inner product and cosine similarity in their respective sketches. 
% \textcolor{red}{We use Skeching via Stable Distribution (SSD)~\cite{SSD}~ a hamming distance estimation technique which uses stable distribution to estimate hamming distance and One Hot Encoding (OHE) scheme to estimate the hamming distance by compressing the one hot encoded samples using BinSketch\cite{ICDM}.
% }
 We include Kendall rank correlation coefficient (KT)~\cite{kendall1938measure} -- a feature selection algorithm which generates discrete valued sketches. Note that if we apply Feature Hashing (FH), SimHash (SH), and KT naively on  categorical datasets, we get discrete valued sketches on which Hamming distance can be computed.% We compute the Hamming distance of the sketch and report it as the original pairwise Hamming distance.
% \textcolor{red}{On applying SSD on categorical dataset corresponding to each sample we get a real value for each sample, for two  sample the difference of obtained value through SSD gives the estimate of original hamming distance between these two sample. On applying OHE on categorical data set firstly we get binary conversion of categrical data using one hot encoding and on the top of that OHE uses well establish binary sketching technique BinSketch~\cite{ICDM} to get compressed binary sketch and to estimate original hamming distance from the compressed binary sketch.}
We also include a few other well known dimensionality reduction methods such as Principal component analysis (PCA), Non-negative Matrix Factorisation (NNMF)~\cite{NNMF}, Latent Dirichlet Allocation (LDA)~\cite{LDA}, Latent Semantic Analysis (LSA)~\cite{LSI}, \bl{Variational Autoencoder (VAE) \cite{Kingma2014}, Categorical PCA (CATPCA)\cite{Sulc2015DimensionalityRO}, Hierarchical Cluster Analysis (HCA) \cite{Sulc2015DimensionalityRO}}   all of which output real-valued sketches.
% \textcolor{red}{We use the sketches obtained from these  sketching methods for two ends tasks clustering and similarity search. }
% % We also include $\chi^2$-{feature selection}, and   {Mutual Information based feature selection} algorithm  for comparison. However, we use them only in the classification tasks for comparison as they   require labeled data for feature selection.









%  Recall that \fsketch (hence \minfsketch) {\em compresses
% categorical vectors to shorter categorical vectors in an unsupervised manner that	``preserves'' Hamming
% distances} and we did not find any out-of-the-box method that does the same. So, for comparison we identified the state-of-the-art methods that either output integer-valued
% sketches (allowing Hamming distance calculation) or simply perform
% dimensionality reduction of categorical vectors that can be used for clustering, classification and similarity search. We summarise the  baseline methods used for comparison in
% Table~\ref{tab:baselines_methods}  
% and  summarise the characteristic of all the baseline algorithms in Table~\ref{tab:baseline_characteristic}. We  give implementation details of the baseline algorithms in Appendix~\ref{sec:Reproducibility_baseline}.


%%%%%%%%%% BELOW IS COMMENTED %%%%%%%%%%%%%%
\begin{comment}
\begin{enumerate}
 \item  We used our own implements for  {Feature Hashing (FH)}~\cite{WeinbergerDLSA09} and   {SimHash (SH)}\cite{simhash} algorithms. We give its link via the following repository here \footnote{\url{https://github.com/Anonymus135/F-Sketch}}.
 \item for  {Kendall rank correlation coefficient}\cite{kendall1938measure} we use \texttt{pandas data frame implementation}\footnote{\url{https://pandas.pydata.org/pandas-docs/stable/reference/api/pandas.DataFrame.corr.html}}. 
 \item for {Latent Semantic Analysis (LSA)}~\cite{LSI}, {Latent Dirichlet Allocation (LDA)}\cite{LDA}, {Non-negative Matrix Factorisation (NNMF)}\cite{NNMF}, and  vanilla {Principal component analysis (PCA)}, we use the implementation available \texttt{sklearn.decomposition} library \footnote{\url{https://scikit-learn.org/stable/modules/classes.html\#module-sklearn.decomposition}}.
 \item for {Multiple Correspondence Analysis (MCA)}~\cite{MCA}, we use the following implementation\footnote{\url{https://pypi.org/project/mca/}}.
 \item finally for $\chi^2$-{feature selection} \footnote{\url{https://scikit-learn.org/stable/modules/generated/sklearn.feature_selection.chi2.html\#}}\cite{chi_square} and  {Mutual Information based feature selection}~\cite{MI}\footnote{\url{https://scikit-learn.org/stable/modules/generated/sklearn.feature_selection.mutual_info_classif.html}} we use the implementation available at \\ \texttt{sklearn.feature\_selection} library.
 \end{enumerate}
We include $\chi^2$-{feature selection}, and   {Mutual Information
based feature selection} algorithm  only in the classification tasks for
comparison as they   require labelled data for feature selection. PCA, MCA and
LSA can reduce the data dimension up to the minimum of the number of data points
and original data dimension.
%%%%%%%%%% BELOW IS COMMENTED %%%%%%%%%%%%%%
\end{comment}

%(detailed descriptions about hardware, datasets andbaselines are given in Append~\ref{sec:Reproducibility}).

 








%\begin{comment}
%\noindent\textbf{Candidate algorithms:} We compare and contrast the performance of our algorithm with the following baselines. 
%\begin{inparaenum}
% [(a)]\item  \texttt{Feature Hashing}~\cite{WeinbergerDLSA09}, \item \texttt{Signed-random projection} or \texttt{Simhash}\cite{simhash}, \item \texttt{Kendall rank correlation coefficient}\cite{kendall1938measure}, \item \texttt{Latent Semantic Analysis (LSA)}\cite{LSI}, \item \texttt{Latent Dirichlet Allocation (LDA)}\cite{LDA}, \item \texttt{Multiple Correspondence Analysis (MCA)}\cite{MCA}, \item \texttt{Non-negative Matrix Factorization (NNMF)}\cite{NNMF}, \item vanilla \texttt{Principal component analysis (PCA)}, \item $\chi^2$-\texttt{feature selection}\cite{chi_square}, and \item \texttt{Mutual Information based feature selection}\cite{MI}. 
%\end{inparaenum}
%We defer the details used in the implementation of the  baselines to the appendix under the reproducibility section. {\color{red} We can simply refer to the above table. For all algorithms and datasets add shortnames that are referred to in the plots.}
%\end{comment}
%\subsection{Choice of Parameters}


\begin{figure*}
\centering
% \includegraphics[scale = 0.22]{images/different_p/RMSE.png}
\includegraphics[width=\linewidth]{images/Space_efficiency_new.pdf}
%\vspace{-4mm}
    \caption{{Space overhead of uncompressed vectors stored as a list of non-zero entries and their positions. $Y$-axis represents the ratio of the space used by uncompressed vector to that obtained from \fsketch.
%    Comparison of space saving  obtained from \texttt{FSketch} algorithm on various choices of   $p$. Y-axis  corresponds to the ratio of the space required by a full-dimensional dataset with the space required by the sketch obtained by \fsketch.
}}
\label{fig:varying_p_space}
\end{figure*}

\begin{figure*}
\centering
\includegraphics[width=\linewidth]{images/new_integer_enron_box_plot.pdf}
%\vspace{-2mm}
    \caption{{Comparison of avg.\ error in estimating Hamming distance of a pair of points from the Enron dataset.}}%\textit{“Hamming error:= actual Hamming distance - estimated Hamming distance from the sketch”} on various reduced dimension, over  $100$ iterations for a random pair of points from Enron dataset.}}

\label{fig:box_plot_hamming_error}
\end{figure*}


\subsection{Choice of $p$}
We discussed in Section~\ref{sec:analysis} that a larger value of $p$ (a prime number) leads to a tighter estimation of Hamming distance but degrades sketch sparsity, which negatively affects performance at multiple fronts, and moreover, demands more space to store a sketch. We conducted an experiment to study this trade-off, where we ran our proposal with different values of $p$, and computed the corresponding $\RMSE$ values. The $\RMSE$ is defined as the square-root of the average error, among all pairs of data points, between their actual Hamming distances and the corresponding estimate obtained via \texttt{FSketch}. Note that a lower $\RMSE$ indicates that the sketch correctly estimates the underlying pairwise Hamming distance.  We also note the corresponding \textit{space overhead} which is defined as the ratio of the space used by uncompressed vector and its sketch obtained from~\texttt{FSketch}. We consider storing a data point in a typical sparse vector format -- a list of non-zero entries and their positions (see Table~\ref{tab:space-example}). We summarise our results in Figures~\ref{fig:varying_p} and  ~\ref{fig:varying_p_space}, respectively. We observe that  a large value of $p$ leads to a lower $\RMSE$ (in Figure~\ref{fig:varying_p}), however simultaneously it leads to a smaller space compression (Figure~\ref{fig:varying_p_space}). \bl{
As a heuristic, we decided to set $p$ as the next prime after $c$ as shown in this table.}\\
{
 \footnotesize
\noindent\begin{tabular}{lc@{\hskip 2em}lc@{\hskip 2em}lc}
    \toprule
     Brain cell & $2039$ & NYTimes & $127$ & Enron & $151$ \\
     KOS & $43$ & Delicious & $59$ & Gisette & $1009$\\
     \bl{NIPS} &\bl{$137$}&&&\\ 
     \bottomrule
\end{tabular}
}

\bl{That said, the experiments reveal that, at least for the datasets in the above experiments, setting $p$ to be at least $c/4$ may be practically sufficient, since there does not appear to be much advantage in using a larger $p$.}
%both the space saving and the RMSE do not appear to differ substantially once $p$ is sufficiently large -- we observe a rule of thumb that for the datasets in these experiments $p$ should be chosen to be at least $c/4$. 
%As a special case, $p=2$ corresponds to a binary sketch and requires the least space but suffers in terms of $\RMSE$. Setting $p$ to be $1/3$-rd of the number of categories ($c$) appears to be an optimal choice for the datasets at hand. %Even though there does not appear to be any real advantage in using a larger $p$. 


% We discussed in Section~\ref{sec:analysis} that a larger value of $p$ leads to a tighter estimation of Hamming distance but degrades sketch sparsity, which negatively affects performance at multiple fronts, and requires more space to store a sketch. We conducted two experiments to study this trade-off, using $k=1$. In Figure~\ref{fig:varying_p} we show how a large value of $p$ leads to a lower $\RMSE$ ($\RMSE$ experiments are explained in Subsection~\ref{subsubsec:rmse}) and in Figure~\ref{fig:varying_p_space} we show that the same leads to a better space compression with respect to uncompressed vectors that are stored in a typical sparse vector format (see Table~\ref{tab:space-example}).

% $p=2$ corresponds to a binary sketch and requires the least space but suffers in terms of $\RMSE$. Setting $p$ to be $1/3$-rd of the number of categories ($c$) appears to be an optimal choice for the datasets at hand. %Even though there does not appear to be any real advantage in using a larger $p$, 
% For all our experiments we decided to set $p$ as the next prime after $c$.


%From Theorem~\ref{thm:main}, for two data points $x,y \in \{0,1,\ldots,c-1\}^n$, the estimation of their hamming distance from their sketches $x’,y’ \in \{0,1,\ldots,p-1\}^d$, depends on the choice of the prime number $p$. Further, the space needed for storing the  sketches $x’,y’$ is $O(d\log p)$. Clearly, higher values of $p$ requires more space for storing the sketch. We empirically give a trade-off between the choice of $p$ and the quality of estimate of the hamming distance between $x$ and $y$ via $\RMSE$. To do so, we run the experiments mentioned in Subsection~\ref{subsubsec:rmse} for \texttt{FSketch} by running it for  different values of $p$, and observing the corresponding $\RMSE$. We summarise our results in Figure~\ref{fig:varying_p}.  When $p=2$, the output sketch is binary, which requires lesser space. However, in this case, $\RMSE$ value is higher. As we increase the value of $p$ the $\RMSE$ values start decreasing and converging toward zero. 
%
%We further give a trade off on choice of $p$ with the space required to store the sketch. In sparse representation full dimensional data requires $\sigma \log n$ to store the non-zero indices, and  $\sigma \log c$ to store the corresponding values. Thus, in total it requires $\sigma \log cn$ space.  Further, the reduced dimensional data obtained from \texttt{FSketch} requires $\sigma_s \log d$, and  $\sigma \log p$ to store the corresponding values, and in total requires $\sigma_s \log pd$ space, where $\sigma_s$ is the sparsity of the sketch. We empirically compare these two quantities by computing the ratio of these two,  and summaries our results for different choices of $p$ in Figure~\ref{fig:varying_p_space}.
%\textcolor{red}{space saving w.r.t. p}




\begin{figure*}
{
\centering
% \includegraphics[scale = 0.22]{images/different_p/RMSE.png}
\includegraphics[width=\linewidth]{images/New_Reduction_time_revision1.pdf}
    \caption{Comparison among the baselines on the dimensionality reduction time. See Appendix~\ref{appendix:section:extended_exp} for results on the other datasets which show a similar trend and Section~\ref{sec:appendix_oom_error} for the errors encountered by some baselines.}
%\textcolor{red}{Note that in this plot several baselines don’t have measures corresponding to all values for the reduced dimensions. This is due to the following: several baselines such as  MCA, PCA, LSA can give dimensionality reduction upto maximum of number of data points and dimension. For others we got either \textit{out-of-memory (OOM)} or \textit{did-not-stop (DNS)} beyond the respective reduced dimensions.}
%}}
\label{fig:reduction_time}
}
\end{figure*}


% \begin{table*}
% \centering
% %\resizebox{\columnwidth}{!}{
%     \caption{{Speedup of \texttt{FSketch} \textit{w.r.t.} baselines on the reduced dimension $1000$. 
%     % We obtain similar pattern on other reduced dimension as well. 
%     \texttt{OOM} indicates ``out-of-memory error'' and {\tt DNS} indicates ``did not stop'' after a sufficiently long time.
% } }\label{tab:speed_up_dim_time}
% %\vspace{-2mm}
% \scalebox{1}{
%   \begin{tabular}{ l|c|c|c|c|c|c|c|c|c|r}%\label{tab:speed_up_dim_time}
%     \toprule
%     Dataset&OHE&KT&NNMF&MCA&LDA&LSA&PCA&SSD&SH&FH\\
%     \midrule
% NYTimes&$5774 \times$ &\texttt{OOM}&$3647\times$&\texttt{OOM}&$200\times$&$30\times$&$10\times$& $158 \times$ & $1.8\times$&$0.84\times$\\
% Enron&$2259 \times$ &\texttt{DNS}&$5736\times$&$161\times$&$80\times$&$17\times$&$6.4\times$& $99.21 \times$ & $1.24\times$&$0.81\times$\\
% KOS&$404\times$  &$5133\times$&$1463\times$&$13\times$&$97\times$&$6.3\times$&$1.6\times$& $ 19.32 \times$  & $0.44\times$&$0.85\times$\\
% DeliciousMIL&$160 \times$ &$17434\times$&$1665\times$&$21\times$&$151\times$&$9\times$&$3\times$& $9.56 \times$  & $0.56\times$&$0.97\times$\\
% Gisette&$480 \times$  &$1183\times$&$630.2\times$&$4\times$&$245\times$&$3.4\times$&$0.52\times$& $15.4 \times$ & $0.21\times$&$1.03\times$\\
% Brain Cell& \texttt{OOM} &\texttt{OOM}& {\tt DNS}  &\texttt{OOM} &$364\times$ & $87\times$ & $68\times$ & $436 \times$ &  $6.1\times$ & $0.92\times$ \\
% \bottomrule
%  \end{tabular}
% }
%     %\vspace*{-5mm}
% \end{table*}




\begin{table*}
{
\centering
%\resizebox{\columnwidth}{!}{
    \caption{{Speedup of \texttt{FSketch} \textit{w.r.t.} baselines on the reduced dimension $1000$. 
    % We obtain similar pattern on other reduced dimension as well. 
    \texttt{OOM} indicates ``out-of-memory error'' and {\tt DNS} indicates ``did not stop'' after a sufficiently long time.
} }\label{tab:speed_up_dim_time}
%\vspace{-2mm}
\scalebox{0.99}{%
  \begin{tabular}{ lccccccccccccr}%\label{tab:speed_up_dim_time}
    \toprule 
    Dataset  &   OHE          &       KT       &      NNMF       &    MCA        &     LDA     &     LSA       &   PCA         &   VAE         &     SSD            &     SH       &    FH        & CATPCA & HCA   \\
    \midrule
NYTimes      &$ \texttt{OOM} $&$ \texttt{OOM}  $&$ 6149\times $&$  \texttt{OOM} $&$ 189\times $&$  11.5\times $&$ 88.14\times $&$ 4340 \times $&$ 164.9\times $&$ 1.2\times   $&$ 0.99\times$     &${\tt DNS}$&${\tt DNS}$ \\
Enron        &$ \texttt{OOM} $&$ \texttt{DNS}  $&$2624\times$ &$  \texttt{OOM} $&$ 122\times $&$ 15.5 \times $&$\texttt{OOM} $&$ \texttt{DNS}$&$ 25.5\times   $&$ 1.25\times  $&$ 0.87\times$     &${\tt DNS}$&$1268.2\times$\\ 
KOS          &$ 629\times    $&$ 14455\times   $&$ 1754\times    $&$   20.41\times $&$ 128\times $&$ 6.40\times  $&$ 9.5\times $&$  1145\times $&$ 14.62\times   $&$ 0.79\times  $&$ 0.98\times$  &${\tt DNS}$&$81.24\times$\\
DeliciousMIL &$ 1332\times   $&$ 14036\times   $&$ 1753\times    $&$   40.39\times $&$ 136\times $&$ 6.6\times   $&$ 18.1\times $&$  1557\times $&$ 29.2 \times $  &$ 0.61\times  $&$ 0.90\times$ &${\tt DNS}$&$117.6\times$\\
Gisette      &$ 399\times    $&$ 1347\times    $&$ 459\times  $&$   5.7\times   $&$ 269\times $&$ 5.4\times   $&$ 4.2\times   $&$  285\times  $&$  8.1\times    $&$ 0.69\times  $&$ 0.98\times$   &${\tt DNS}$&$16.78\times$\\ 
% Brain Cell   &$ \texttt{OOM} $&$ \texttt{OOM}  $&$ {\tt DNS}     $&$ \texttt{OOM}  $&$ 173\times $&$ 79.37\times $&$ 62.63\times $&$ 855.65\times$&$ 2114.78\times $&$ 5.01\times  $&$ 0.89\times$ &&\\
NIPS         &$ 378\times    $&$ 15863\times    $&$ 1599\times  $&$  26.6\times   $&$ 302\times $&$ 6.4\times   $&$ 3.17\times   $&$  451\times  $&$  29.9\times    $&$ 0.47\times  $&$1.20\times$ &${\tt DNS}$&$58.49\times$\\ 
Brain Cell     &$ \texttt{OOM} $&$\texttt{OOM} $&$ {\tt DNS}  $&$ \texttt{OOM} $&$ 322\times $&$ 79.38\times  $&$ 62.7\times $&$1198\times$    &$ 443 \times$ &  $5\times$ & $0.89\times$  &${\tt DNS}$&${\tt DNS}$\\
\bottomrule
 \end{tabular}
 }
}
    %\vspace*{-5mm}
\end{table*}



\begin{figure*}
\centering
% \includegraphics[scale = 0.22]{images/different_p/RMSE.png}
\includegraphics[width=\linewidth]{images/Updated_RMSE_new.pdf}
%\vspace{-2mm}
\caption{{Comparison on $\RMSE$ among baselines. A lower value is an indication of better performance.  See Appendix~\ref{appendix:section:extended_exp} for results on the other datasets which show a similar trend. %$\mathrm{KT}$ couldn’t finish in 10 hours (\textcolor{red}{@Bhisham pls mention the correct time.}) on NYTimes and Enron datasets. 
}}
\label{fig:rmse}
\end{figure*}



\subsection{Variance of \fsketch}
In Section~\ref{subsec:analysis} we explained that the bias of our estimator is upper bounded with a high likelihood. However, there remains the question of its variance. To decide the worthiness of our method, we compared the variance of the estimates of the Hamming distance obtained from \fsketch and from the other randomised sketching algorithms with integer-valued sketches (KT was not included as it is a deterministic algorithm, and hence, has zero variance).

% $\hat{h}$ from Definition~\ref{defn:estimator} on the sketches obtained from \fsketch and from other randomized sketching algorithms with integer-valued sketches.
Figure~\ref{fig:box_plot_hamming_error} shows the Hamming error (estimation error) for a randomly chosen pair of points from the Enron dataset, averaged over 100 iterations. 
We make two observations.

First is that the estimate using \texttt{FSketch} is closer to the actual Hamming distance even at a smaller
reduced dimension; in fact, as the reduced dimension is increased, the variance
becomes smaller and the Hamming error converges to zero. Secondly, \fsketch causes a smaller error compared to the other baselines. On the other hand, {feature hashing} highly underestimates the actual Hamming distance, but has low variance, and tends to have negligible Hamming error with an increase of the reduced dimension. The behaviour of {SimHash} is counter-intuitive as on lower reduced dimensions it closely estimates the actual Hamming distances, but on larger dimensions it starts to highly underestimate the actual Hamming distances. This creates an ambiguity on the choice of a dimension for generating a low-dimensional sketch of a dataset.%, as it is not clear which reduce dimension should be appropriate for closely estimating the actual Hamming distance. 
 Similar to \fsketch, the sketches produced by SSD, though real-valued, allow estimation of pairwise Hamming distances. However the estimation error increases with the reduced dimension. Lastly, OHE seems to be highly underestimating pairwise Hamming distances. 

%We observed that the estimates obtained using \fsketch is closer to the actual Hamming distance even at a smaller reduced dimension and much smaller compared to those of other baselines. We defer the detailed explanation to Appendix~\ref{subsec:variance_details}. 
% Observing the really low variance of $\hat{h}$, we conjecture that the gain in performance from \minfsketch may not outweigh its additional space and time complexity. We decided to continue with \fsketch for performance evaluation in data analytic tasks.


%%%%%% BELOW IS COMMENTED %%%%%%%%%%%%%
% \begin{comment}
% \subsection{Comparing variance of \fsketch with other compression
% algorithms}\label{subsec:variance_details}

% \begin{figure}[!ht]
% \centering
% \includegraphics[width=\linewidth]{images/integer_enron_box_plot.pdf}
% %\vspace{-2mm}
%     \caption{{Comparison of avg.\ error in estimating Hamming distance of a pair of points from the Enron dataset.}}%\textit{“Hamming error:= actual Hamming distance - estimated Hamming distance from the sketch”} on various reduced dimension, over  $100$ iterations for a random pair of points from Enron dataset.}}
% \end{figure}

% %\noindent\textbf{Insights:} 
% First note that KT was not included as it is a deterministic algorithm, and
% hence, has zero variance. We make two important observations. First, is that the estimate using \texttt{FSketch} is closer to the actual Hamming distance even at a smaller
% reduced dimension; in fact, as the reduced dimension is increased, the variance
% becomes smaller and the Hamming error converges to zero. Secondly, \fsketch causes a smaller error compared to the other baselines. Whereas the {Feature hashing} highly underestimates the actual Hamming distance, with low variance, and tends to have negligible Hamming error with an increase of the reduced dimension. The behavior of {SimHash} is counter-intuitive as at lower reduced dimension it closely estimates the actual Hamming distance, whereas with an increase of the reduced dimension starts highly underestimating the actual Hamming distance. This leaves the ambiguity on the choice of {SimHash} for finding a low-dimensional sketch of Categorical dataset, as it is not clear which reduce dimension should be appropriate for closely estimating the actual Hamming distance. 
% \end{comment}
 %%%%%% ABOVE IS COMMENTED %%%%%%%%%%%%%



%\noindent\textbf{Insights:} 
%We make two important observations. First, is that the estimate using \texttt{FSketch} is closer to the actual Hamming distance even at a smaller reduced dimension; in fact, as the reduced dimension is increased, the variance becomes smaller and the Hamming error converges to zero. Secondly, the error from sketches obtained from the other two algorithms are worse, even though their variance is lower at really small dimensions. Contrary to our expectation, the error of SimHash becomes worse with increasing dimension. Feature hashing behaves as expected but \fsketch shows a better performance even for dimension as low as 100.
%\textcolor{red}{On lower dimensions, SimHash underestimates the $\RMSE$ error whereas on higher dimensions it starts overestimating the error. Thus, it becomes difficult to find an appropriate reduced dimension where SimHash correctly estimates the actual Hamming distance.}
%from the estimate used by baseline algorithms. To do so, we first take a random sample of a pair of points from the   {Enron dataset}, we compute their actual Hamming distance. We then reduce the dimension of data to various reduced  dimensions using the baseline algorithms, and estimate the Hamming distance from the sketch. We repeat this process  {$100$ times} and plot the \textit{“Hamming error:= actual Hamming distance-estimated Hamming distance”} using box plot, and summarise it in Figure~\ref{fig:box_plot_hamming_error}.  We didn’t include the discrete hashing algorithm Kendall correlation coefficient as it is deterministic. 


\subsection{Speedup in dimensionality reduction}\label{subsubsec:dim_red_time}
 We compress the datasets to several dimensions using \fsketch and the baselines and report their running times in Figure~\ref{fig:reduction_time}. We notice that \fsketch has a comparable speed \textit{w.r.t.} Feature hashing and SimHash, and is significantly faster than the other baselines. However, both feature hashing and SimHash are not able to accurately estimate the Hamming distance between data points and hence perform poorly on $\RMSE$ measure (Subsection~\ref{subsubsec:rmse}) and the other tasks. Many baselines such as OHE, KT, NNMF, MCA, \bl{CATPCA, HCA} give \textit{``out-of-memory”} (OOM) error, and also didn’t stop (DNS) even after running for a sufficiently long time ($\sim~10$ hrs) on high dimensional datasets such as Brain Cell and NYTimes. On other moderate dimensional datasets such as Enron and KOS, our speedup \textit{w.r.t.} these baselines are of the order of a few thousand. We report the numerical speedups that we observed in Table~\ref{tab:speed_up_dim_time}.




% We compress the datasets to several dimensions using \fsketch and the baselines algorithms and report the running times in Figure~\ref{fig:reduction_time}. We notice that \fsketch has a comparable speed \textit{w.r.t.} Feature hashing and SimHash, and is significantly faster than the other baselines. However, both Feature Hashing and SimHash are not able to accurately estimate the Hamming distance between data points and hence perform poorly on $\RMSE$ measure (Subsection~\ref{subsubsec:rmse}) and the other tasks (Subsection~\ref{subsec:end_task_comparision}). We summarise the speedups that we saw in Table~\ref{tab:speed_up_dim_time}.

% \begin{table*}
% \centering
% %\resizebox{\columnwidth}{!}{
%     \caption{{Speedup of \texttt{FSketch} \textit{w.r.t.} baselines on the reduced dimension $1000$. 
%     % We obtain similar pattern on other reduced dimension as well. 
%     \texttt{OOM} indicates ``out-of-memory error'' and {\tt DNS} indicates ``did not stop'' after a sufficiently long time.
% } }\label{tab:speed_up_dim_time}
% %\vspace{-2mm}
% \scalebox{1}{
%   \begin{tabular}{ l|c|c|c|c|c|c|c|c|c|r}%\label{tab:speed_up_dim_time}
%     \toprule
%     Dataset&MI&$\chi^2$&KT&NNMF&MCA&LDA&LSA&PCA&SH&FH\\
%     \midrule
% NYTimes&\texttt{NA}&\texttt{NA}&\texttt{OOM}&$3647\times$&\texttt{OOM}&$200\times$&$30\times$&$10\times$&$1.8\times$&$0.84\times$\\
% Enron&\texttt{NA}&\texttt{NA}&\texttt{DNS}&$5736\times$&$161\times$&$80\times$&$17\times$&$6.4\times$&$1.24\times$&$0.81\times$\\
% KOS&\texttt{NA}&\texttt{NA}&$5133\times$&$1463\times$&$13\times$&$97\times$&$6.3\times$&$1.6\times$&$0.44\times$&$0.85\times$\\
% DeliciousMIL&$103\times$&$0.4\times$&$17434\times$&$1665\times$&$21\times$&$151\times$&$9\times$&$3\times$&$0.56\times$&$0.97\times$\\
% Gisette&$18.3\times$&$0.054\times$&$1183\times$&$630.2\times$&$4\times$&$245\times$&$3.4\times$&$0.52\times$&$0.21\times$&$1.03\times$\\
%       Brain Cell&\texttt{NA} & \texttt{NA} & \texttt{OOM}& {\tt DNS}  &\texttt{OMM} &$364\times$ & $87\times$ & $68\times$ & $6.1\times$ & $0.92\times$ \\
% \bottomrule
%  \end{tabular}
% }
%     %\vspace*{-5mm}
% \end{table*}




% \begin{comment}
% {Note: \\
% Memory Error: Kindell Tau gives memory error for NYTimes and Enron data set, MCA also gives memory error for NYTimes\\
% For NMF as the value of reduced dimension increase the reduction time increase with high rate so we are able to take reduced dimension only up to 3000 (for 3000 the reduction time 71869 seconds)} --- \textcolor{red}{@Bhisham on which dataset??}\\ 
% PCA, MCA and LSA can reduce dimension only upto 2000 as we are using 2000 samples in our experiments
% \end{comment}
 


\begin{figure*}
{
\centering
% \includegraphics[scale = 0.22]{images/different_p/RMSE.png}
    \includegraphics[width=\linewidth]{images/Purity_index_revision1.pdf}
\caption{{Comparing the quality of clusters on the compressed datasets.  See Appendix~\ref{appendix:section:extended_exp} for results on the other datasets which show a similar trend.
}}
\label{fig:purity}
}
\end{figure*}
 
 \begin{figure*}
 {
\centering
% \includegraphics[scale = 0.22]{images/different_p/RMSE.png}
\includegraphics[width=\linewidth]{images/TopK_revision1.pdf}
\caption{{Comparing the performance of the similarity search task (estimating top-$k$ similar points with $k=100$) achieved on the reduced dimensional data obtained from various baselines.  See Appendix~\ref{appendix:section:extended_exp} for results on the other datasets which show a similar trend.
}}
\label{fig:similarity_search}
}
\end{figure*}

\subsection{Performance on root-mean-squared-error (RMSE)}\label{subsubsec:rmse}
How good are the sketches for estimating Hamming distances between the uncompressed points in practice? To answer this, we compare \fsketch with integer-valued sketching algorithms, namely,  {feature hashing}, {SimHash}, {Kendall correlation coefficient} and OHE+BinSketch.   Note that {feature hashing} and {SimHash} are known to approximate inner product and cosine similarity, respectively. However, we consider them in our comparison nonetheless as they output discrete sketches and Hamming distance can be computed on their sketch. We also include SSD for comparison which outputs real-valued sketches and  estimates original pairwise Hamming distance. 
For each of the methods we compute its $\RMSE$ as the square-root of the average error, among all pairs of data points, between their actual Hamming distances and their corresponding estimates (for \fsketch the estimate was obtained using Definition~\ref{defn:estimator}). Figure~\ref{fig:rmse} compares these values of $\RMSE$ for different dimensions; note that a lower $\RMSE$ is an indication of better performance. It is immediately clear that the $\RMSE$ of \texttt{FSketch} is the lowest among all; furthermore, it falls to zero rapidly with increase in reduced  dimension.  This demonstrates that our proposal \texttt{FSketch} estimates the underlying  pairwise Hamming distance better than the others.

% How good are the sketches for estimating Hamming distances between the uncompressed points in practice? To answer this, we compare \fsketch with integer-valued sketching algorithms, namely,  {Feature Hashing}, {SimHash}, and {Kendall correlation coefficient}. {Feature Hashing} and {SimHash} are known to approximate inner product and cosine similarity, respectively. However, we consider them nonetheless as Hamming distance can be computed on the discrete sketches that they output.
% For each of the methods we compute its $\RMSE$ as the square-root of the average error, among all pairs of data points, between their actual Hamming distances and the corresponding estimate obtained using Definition~\ref{defn:estimator}. Figure~\ref{fig:rmse} compares these values of $\RMSE$ for different dimensions. It is immediately clear that the $\RMSE$ of \texttt{FSketch} is the lowest among all; furthermore, it falls to zero rapidly with decreasing dimension --- this demonstrates the high fidelity of our Hamming distance estimator.

%indicates that \fsketch is able to estimate the Hamming distances quite accurately in practice.
% Whereas $\RMSE$ of SimHash is the worst and increases with the increase of reduced dimension. For Feature Hashing, and Kendall   coefficient $\RMSE$ score is high, however, it starts decreasing with the increase of reduced dimension. 
%In this experiment,  the aim is to show that the sketch obtained from the  baselines methods shows how closely they are able to estimate the original similarity between data points. To do so, for a pair of data points, we estimate their hamming distance using various baseline methods.  We compute the square of the difference between the estimated hamming distance and their actual hamming  distance. We repeat this for all pairs of data points,  add all such numbers obtained from squared difference,  compute their mean, and then compute the square root.  We report this as $\RMSE$. A lower value of $\RMSE$ is an indication of better performance.  We repeat this for several values of $d$ (reduced dimension) and report the corresponding values of $\RMSE$. In this experiment, we include the following baselines for comparison -- {Feature Hashing}, {SimHash}, {Kendall correlation coefficient}. {Feature Hashing} and {SimHash} are known to approximate Inner product and Cosine similarity, respectively. However, we include them in the comparison nevertheless, as they output discrete sketches and that Hamming distance can be defined on them in order to compute the $\RMSE$.

%\noindent\textbf{Insights:} $\RMSE$ of \texttt{FSketch} is the lowest among the baseline, and it is simultaneously close to zero,  which implies that it accurately measures the actual distance. Whereas $\RMSE$ of SimHash is the worst and increases with the increase of reduced dimension. For Feature Hashing, and Kendall   coefficient $\RMSE$ score is high, however, it starts decreasing with the increase of reduced dimension. 
%{In each insights need to mention that our  quality of estimate is comparable wr.t to baselines while we obtained significant speedup in running the experiments corresponding to the full dimensional data. }

% \subsection{Performance on  end tasks}\label{subsec:end_task_comparision} We compare the performance of \fsketch with baselines on the task of clustering and similarity search. We discuss them one-by-one as follows:

% \begin{table}[t]
    
% \caption{{%Speedup obtained on several end tasks while running them the sketch of dimension $1000$ obtained \textit{via} \texttt{FSketch} \textit{w.r.t.} running time on the full dimensional dataset.
% Speedup from running tasks on 1000-dimensional sketches instead of the full
%     dimensional dataset.
%     %For \textit{e.g.} speedup obtained in clustering= (Time is taken by clustering on full-dimensional data)/(Time is taken by clustering $1000$ dimension sketch obtained \textit{via} \texttt{FSketch}).
%     } }\label{tab:speed_up_task_time}
% %\vspace{-3mm}
%  \centering
% \addtolength\tabcolsep{-4pt}
%   \begin{tabular}{lccccr}
%     \toprule
%     Task&NYTimes&Enron&KOS&Gisette&DeliciousMIL\\
%     \midrule
% Clustering   &$117.25\times$&$16.99\times$&$2.83\times$&$2.82\times$&$4.124\times$\\
% Similarity Search&$106.98\times$ &$33.72\times$&$7.17\times$&$5.07\times$&$13.52\times$\\
% % Classification&\texttt{NA}&\texttt{NA}&\texttt{NA}&$1.252\times$&$2.62\times$\\
%  \bottomrule
%   \end{tabular}
%     %\vspace*{-3mm}
%   \addtolength\tabcolsep{4pt}
% \end{table}



\begin{table}[t]
{
\caption{{%Speedup obtained on several end tasks while running them the sketch of dimension $1000$ obtained \textit{via} \texttt{FSketch} \textit{w.r.t.} running time on the full dimensional dataset.
Speedup from running tasks on 1000-dimensional sketches instead of the full
    dimensional dataset. We got a DNS error while running clustering on the uncompressed BrainCell dataset. 
%For \textit{e.g.} speedup obtained in clustering= (Time is taken by clustering on full-dimensional data)/(Time is taken by clustering $1000$ dimension sketch obtained \textit{via} \texttt{FSketch}).
    } }\label{tab:speed_up_task_time}
%\vspace{-3mm}
 \centering
\resizebox{0.48\textwidth}{!}{%
\addtolength\tabcolsep{-4pt}
  \begin{tabular}{lccccccc}
    \toprule
    Task           & Brain cell      & NYTimes        &   Enron            &  NIPS         &  KOS          &   Gisette     &    DeliciousMIL\\
    \midrule
Clustering         &$NA   $&$ 139.64\times  $&$  21.15\times    $&$ 10.6\times  $&$ 3.93\times  $&$ 4.35\times  $&$  5.84\times  $\\
% Clustering         &$ 206.5\times   $&$ 139.64\times  $&$  21.15\times    $&$ 10.6\times  $&$ 3.93\times  $&$ 4.35\times  $&$  5.84\times  $\\
Similarity Search  &$    1231.6\times  $&$  118.12\times $&$ 48.15\times  $&$ 15.1\times  $&$ 10.56\times  $&$ 8.34\times  $&$ 17.76\times $\\
% Classification&\texttt{NA}&\texttt{NA}&\texttt{NA}&$1.252\times$&$2.62\times$\\
 \bottomrule
  \end{tabular}
    %\vspace*{-3mm}
  \addtolength\tabcolsep{4pt}
  }
  
  }
\end{table}

% \subsubsection{Clustering}
\subsection{Performance on clustering}
We compare the performance of \fsketch with baselines on the task of clustering and similarity search, and present the results for the first task in this section. The objective of the clustering experiment 
%First we present the results for the clustering experiment whose objective 
was to test if the data points in the reduced dimension maintain the original clustering structure. If they do, then it will be immensely helpful for those techniques that use a clustering, e.g., spam filtering. We used the {\it purity index} to measure the quality of $k$-mode and $k$-means clusters on the reduced datasets obtained through the compression algorithms; the ground truth was obtained using $k$-mode on the uncompressed data (for more details refer to Appendix~\ref{appendix:clustering}).

%The results are reported in Figure~\ref{fig:purity} where we readily observe that the quality obtained using \texttt{FSketch} is among the top, if not the best.
%We obtain similar results in our experiments for classification and similarity search. All the details are available in Appendix~\ref{sec:appendix_end_task}. 

%In the clustering experiment we used the {\it purity index} to compare 
%In clustering experiment,  the aim is to check if after dimensionality reduction the data points maintain the original clustering structure.  To do so we first evaluate the ground truth clustering on the original dataset using the classical $k$-mode algorithm~\cite{kmode} for several values of $k$. We then perform dimensionality reduction using baseline methods on various values of reduced dimension. We then compare the clustering quality on reduced dimension with the ground truth clustering results \textit{via} \textit{purity index}. 


We summarise our findings on quality in Figure~\ref{fig:purity}. \bl{ The compressed versions of the NIPS, Enron, and KOS  datasets that were obtained from \fsketch yielded the best purity index as compared to those obtained from the other baselines; for the other datasets the compressed versions from \fsketch are among the top. Even though it appears that KT offers comparable performance on the KOS, DeliciousMIL, and Gisette datasets \textit{w.r.t.}  \texttt{FSketch}, the downside of using KT is that its 
% a) it requires labeled datasets for compression whereas \texttt{FSketch} is completely unsupervised, and b) the
compression time is much higher than that of \texttt{FSketch} (see Table~\ref{tab:speed_up_dim_time}) on those datasets, and moreover it gives OOM/DNS error on the remaining datasets. Performance of FH also remains in the top few. However, its performance degrades on the NIPS dataset.}


We tabulate the speedup of clustering of \fsketch-compressed data over uncompressed data in Table~\ref{tab:speed_up_task_time} where we observe significant speedup in the clustering time, e.g.,  \bl{$139\times$} when run on a $1000$ dimensional \fsketch.

Recall that the dimensionality reduction time of our proposal is among the fastest among all the baselines which further reduces the total time to perform clustering by speeding up the dimensionality reduction phase. % Note that this speedup also  propagates in the total clustering  time -- sum of dimensionality reduction time and clustering time.
Thus the overall observation is that \fsketch appears to be the most suitable method for clustering among the current alternatives, especially, for high-dimensional datasets on which clustering would take a long time.

% where we observe significant
% % \paragraph{Insights}
% %We obtain significant 
% speedup in the clustering time along with very high purity index as compared to the baselines. For example, we observed a $117.25\times$ speedup while running it on a $1000$ dimensional \fsketch {\it vs.} running it on the full dimensional data (see Table~\ref{tab:speed_up_task_time}).  The purity index of the clustering results achieved on the compressed data obtained from \texttt{FSketch} is best for Enron and DeliciousMIL datasets and is among the top for the other datasets. Note that KT offers comparable performance on KOS, DeliciousMIL, and Gisette datasets \textit{w.r.t.}  \texttt{FSketch}. However the downside of using KT is that
% its 
% % a) it requires labeled datasets for compression whereas \texttt{FSketch} is completely unsupervised, and b) the
% compression time of KT is much higher than  \texttt{FSketch} (see Table~\ref{tab:speed_up_dim_time}) on KOS, DeliciousMIL, and Gisette datasets, and moreover it gives OOM/DNS error on the remaining datasets. 

% Further the dimensionality reduction time of our proposal is the fastest among all the baselines. Note that this speedup also  propagates in the total clustering  time -- sum of dimensionality reduction time and clustering time. 

% \paragraph{Insights:} The purity index of the clustering results  achieved on the reduced dimension data obtained from \texttt{FSketch} is among the top, if not the best \textit{w.r.t.} the clustering results obtained on the sketch obtained from the other baselines.  
% \textcolor{red}{add insight ..why KT is better..KT is supervised feature selection while ours is unsupervised and fast too}
%We obtain significant speedup in clustering time while running it on the low-dimensional sketch. For \textit{e.g.} we obtain $117.25\times$ speedup in clustering time while running it on $1000$ dimensional sketch obtained from \texttt{FSketch} \textit{w.r.t.} running it on full dimensional data. We summarise such speedup on other tasks as well and put them in Table~\ref{tab:speed_up_task_time}.




%\subsubsection{\texttt{Similarity Search:}}
\subsection{Performance on similarity search}
We take up another unsupervised task -- that of similarity search. The objective here is to show that after dimensionality reduction the similarities of points with respect to some query points are maintained. To do so, we randomly split the dataset in two parts $5\%$ and $95\%$  -- the smaller partition is referred to as the \textit{query partition} and each point of this partition is called a query vector; we call the larger partition as \textit{training partition}. For each query vector, we find top-$k$ similar points in the training partition. We then perform dimensionality reduction using all the methods (for various values of reduced dimensions). Next, we process the compressed dataset where, for each query point, we compute the  top-$k$ similar points in the corresponding low-dimensional version of the training points, by maintaining the same split. For each query point, we compute the accuracy of baselines by taking the Jaccard ratio between the set of top-$k$ similar points obtained in full dimensional data with  the top-$k$ similar points obtained in reduced dimensional dataset. We repeat this for all the points in the querying partition, compute the average, and report this as accuracy.%  This is repeated for all the baselines and for different values of reduced dimension. 

%{
%\paragraph{Insights} 
We summarise our findings in Figure~\ref{fig:similarity_search}. Note that PCA, MCA and LSA can reduce the data dimension up to the minimum of the number of data points and the original data dimension. Therefore their reduced dimension is at most $2000$ \bl{ for Brain cell dataset}.

The top few methods appear to be feature hashing (FH), Kendall-Tau (KT), \bl{HCA} along with \fsketch. However, KT give \textit{OOM} and \textit{DNS} on the Brain cell, NYTimes and Enron datasets, and \bl{HCA give DNS error on BrainCell and NYTimes datasets}.  Ffurther, their dimensionality reduction time are much worse than \texttt{FSketch} (see Table~\ref{tab:speed_up_dim_time}). 

% FH and \fsketch appear neck to neck for similarity search despite the fact that there is no known theoretical understanding of FH for Hamming distance --- in fact, it was included in the baselines as a heuristic because it offers discrete-valued sketches on which Hamming distance can be calculated. Nevertheless, FH and \fsketch perform equally well for similarity search and require similar times for dimensionality reduction making both of them the best approaches for this task. We want to bring to notice at this point that FH was not a consistent top-performer for clustering.

\bl{\fsketch outperforms FH on the BrainCell and the Enron datasets; however, on the remaining datasets, both of them 
 appear neck to neck for similarity search despite the fact that there is no known theoretical understanding of FH for Hamming distance --- in fact, it was included in the baselines as a heuristic because it offers discrete-valued sketches on which Hamming distance can be calculated. 
%  Nevertheless, FH and \fsketch perform equally well for similarity search and require similar times for dimensionality reduction making both of them the best approaches for this task. 
 Here want to point out that FH was not a consistent top-performer for clustering and similarity search.} 


The two other methods that are designed for Hamming distance, namely SSD and OHE, perform significantly  worse than FSketch; in fact, the accuracy of OHE lies almost to the bottom on all the four datasets.

% \textcolor{red}{Performance of SSD and OHE is highly worse than FSketch and FH. For all four datasets the top-k accuracy of OHE lies almost in the bottom.}
%Performance of FH is slightly worse than \texttt{FSketch}  and its dimensionality reduction time is comparable to \texttt{FSketch} on all the datasets. Note that FH is known to approximate Inner products for real-valued sketches. To the best of our knowledge, it is not known that the sketch obtained from FH can approximate Hamming distance for categorical data. We included it in the comparison as a heuristic because it offers discrete-valued sketches and Hamming distance can be defined on it. Again the performance of KT was among the top few on KOS and DeliciousMIL datasets, however its compression time is much higher than \fsketch (see Table~\ref{tab:speed_up_dim_time}). 

%To summarize, the accuracy of the similarity search task achieved on the sketch obtained from \texttt{FSketch} significantly outperforms \textit{w.r.t.} the corresponding results obtained on the sketches obtained from the other baselines. This indicates the \texttt{FSketch} accurately estimates the similarity between data points from the low-dimension.
% \paragraph{Insights:} The accuracy of the similarity search task achieved on the sketch obtained from \texttt{FSketch} significantly outperforms \textit{w.r.t.} the corresponding results obtained on the sketches obtained from the other baselines. This indicates the \texttt{FSketch} accurately estimates the similarity between data points from the low-dimension.  

% \textcolor{red}{elaborate more on insight FH is comparable with FSketch}


\bl{We also summarise the speedup of \fsketch-compressed data over uncompressed data, on similarity search task,  in Table~\ref{tab:speed_up_task_time}. We observe a significant speedup --  \textit{e.g.} \bl{$1231.6\times$} speedup on the BrainCell dataset when run on a   $1000$ dimensional \fsketch. }

To summarise, \fsketch is one of the best approaches towards similarity search for high-dimensional datasets and the best if we also require theoretical guarantees or applicability towards other data analytic tasks.




% \begin{comment}{
% \subsubsection{\texttt{Classification}:} In this task the aim is to check if after dimensionality reduction the data points maintain the classification boundary.  We first randomly split the datasets into training $(70\%)$ and test partition $(30\%)$. We train the classifier on the training partition and measure its performance on test performance. We perform dimensionality reduction using baseline methods on various values of reduced dimension. We then train  the classifier on the low-dimensional version of the same training partition and evaluate its performance on the low-dimensional version of the same test partition.   We then compare the classification quality on reduced dimension with the classification result on full dimensional dataset.  In this experiment, we include all the baselines for comparison. 

% \begin{figure*}[t]
% \centering
% % \includegraphics[scale = 0.22]{images/different_p/RMSE.png}
% \includegraphics[width=\linewidth]{images/New_Classification_Logistic.pdf}
% %\vspace{-2mm}
% \caption{{Accuracy of the classifier trained on the reduced dimensional data obtained from various baselines. Black dotted line corresponds to the classification accuracy on the original dimension. \textbf{logistic regression}}}
% \label{fig:classification2}
% \end{figure*}

% \begin{figure*}[t]
% \centering
% % \includegraphics[scale = 0.22]{images/different_p/RMSE.png}
% \includegraphics[width=\linewidth]{images/New_classification_knn.pdf}
% %\vspace{-2mm}
% \caption{{Accuracy of the classifier trained on the reduced dimensional data obtained from various baselines. Black dotted line corresponds to the classification accuracy on the original dimension. \textbf{KNN}}}
% \label{fig:classification2}
% \end{figure*}

 

% % \paragraph{Insights:} Here again, the classification accuracy on the reduced dimension data obtained from \texttt{FSketch}  is comparable with other baselines, and the classification result on the full dimensional dataset. Note that \texttt{FSketch} doesn’t account for the label information of the data points, and performs dimensionality reduction in completely unsupervised fashion. However, other feature selection based methods such as $\chi^2$ and Mutual Information feature selection take into account label information, and their correlation with features. 

% {
% \paragraph{Insights:} Here again, the classification accuracy on the reduced dimension data obtained from \texttt{FSketch}  is comparable with other baselines, and the classification result on the full dimensional dataset on Gisette dataset, whereas on DeliciousMIL dataset it is within $10\%$ \textit{w.r.t.} other baselines. Note that \texttt{FSketch} doesn’t account for the label information of the data points, and performs dimensionality reduction in completely unsupervised fashion. However, other feature selection based methods such as $\chi^2$ and Mutual Information feature selection take into account label information, and their correlation with features. Moreover, as mentioned earlier, our main advantage remains the significant speedup in  dimensionality reduction time \textit{w.r.t.} most of the baselines.  }

% {
% Apart from the high-quality output in clustering, similarity search, and classification on the compressed data obtained via \texttt{FSketch}, we simultaneously observe significant speedup while running these tasks on the sketches as compared to running them on the original vectors. For example: on NYTimes, the speed-up obtained on clustering and similarity search on the sketches are $117\times$ and $107\times$, respectively, as compared to running them on the full-dimensional datasets.  We summarise the speedups obtained during the evaluation on end tasks in Table~\ref{tab:speed_up_task_time}. We discuss the details required for reproducibility of the empirical evaluation on end tasks in Appendix~\ref{sec:Reproducibility_end_taks}.
% }




% % Apart from the high-quality  output in clustering, similarity search and classification, we also observe significant speedup in clustering of the low-dimensional compressed vectors { w.r.t.} clustering of the uncompressed vectors.
% % We summarise the speedups obtained during the evaluation on end tasks  in Table~\ref{tab:speed_up_task_time}. We discuss the  details required for reproducibility of the empirical evaluation on end task in Appendix~\ref{sec:Reproducibility_end_taks}.

% \textcolor{red}{explain insight on classification ..and add one more dataset}
% \end{comment}

 



% \vspace{-0.5em}
\section{Conclusion}
% \vspace{-0.5em}
Recent advances in multimodal single-cell technology have enabled the simultaneous profiling of the transcriptome alongside other cellular modalities, leading to an increase in the availability of multimodal single-cell data. In this paper, we present \method{}, a multimodal transformer model for single-cell surface protein abundance from gene expression measurements. We combined the data with prior biological interaction knowledge from the STRING database into a richly connected heterogeneous graph and leveraged the transformer architectures to learn an accurate mapping between gene expression and surface protein abundance. Remarkably, \method{} achieves superior and more stable performance than other baselines on both 2021 and 2022 NeurIPS single-cell datasets.

\noindent\textbf{Future Work.}
% Our work is an extension of the model we implemented in the NeurIPS 2022 competition. 
Our framework of multimodal transformers with the cross-modality heterogeneous graph goes far beyond the specific downstream task of modality prediction, and there are lots of potentials to be further explored. Our graph contains three types of nodes. While the cell embeddings are used for predictions, the remaining protein embeddings and gene embeddings may be further interpreted for other tasks. The similarities between proteins may show data-specific protein-protein relationships, while the attention matrix of the gene transformer may help to identify marker genes of each cell type. Additionally, we may achieve gene interaction prediction using the attention mechanism.
% under adequate regulations. 
% We expect \method{} to be capable of much more than just modality prediction. Note that currently, we fuse information from different transformers with message-passing GNNs. 
To extend more on transformers, a potential next step is implementing cross-attention cross-modalities. Ideally, all three types of nodes, namely genes, proteins, and cells, would be jointly modeled using a large transformer that includes specific regulations for each modality. 

% insight of protein and gene embedding (diff task)

% all in one transformer

% \noindent\textbf{Limitations and future work}
% Despite the noticeable performance improvement by utilizing transformers with the cross-modality heterogeneous graph, there are still bottlenecks in the current settings. To begin with, we noticed that the performance variations of all methods are consistently higher in the ``CITE'' dataset compared to the ``GEX2ADT'' dataset. We hypothesized that the increased variability in ``CITE'' was due to both less number of training samples (43k vs. 66k cells) and a significantly more number of testing samples used (28k vs. 1k cells). One straightforward solution to alleviate the high variation issue is to include more training samples, which is not always possible given the training data availability. Nevertheless, publicly available single-cell datasets have been accumulated over the past decades and are still being collected on an ever-increasing scale. Taking advantage of these large-scale atlases is the key to a more stable and well-performing model, as some of the intra-cell variations could be common across different datasets. For example, reference-based methods are commonly used to identify the cell identity of a single cell, or cell-type compositions of a mixture of cells. (other examples for pretrained, e.g., scbert)


%\noindent\textbf{Future work.}
% Our work is an extension of the model we implemented in the NeurIPS 2022 competition. Now our framework of multimodal transformers with the cross-modality heterogeneous graph goes far beyond the specific downstream task of modality prediction, and there are lots of potentials to be further explored. Our graph contains three types of nodes. while the cell embeddings are used for predictions, the remaining protein embeddings and gene embeddings may be further interpreted for other tasks. The similarities between proteins may show data-specific protein-protein relationships, while the attention matrix of the gene transformer may help to identify marker genes of each cell type. Additionally, we may achieve gene interaction prediction using the attention mechanism under adequate regulations. We expect \method{} to be capable of much more than just modality prediction. Note that currently, we fuse information from different transformers with message-passing GNNs. To extend more on transformers, a potential next step is implementing cross-attention cross-modalities. Ideally, all three types of nodes, namely genes, proteins, and cells, would be jointly modeled using a large transformer that includes specific regulations for each modality. The self-attention within each modality would reconstruct the prior interaction network, while the cross-attention between modalities would be supervised by the data observations. Then, The attention matrix will provide insights into all the internal interactions and cross-relationships. With the linearized transformer, this idea would be both practical and versatile.

% \begin{acks}
% This research is supported by the National Science Foundation (NSF) and Johnson \& Johnson.
% \end{acks}

% trigger a \newpage just before the given reference
% number - used to balance the columns on the last page
% adjust value as needed - may need to be readjusted if
% the document is modified later
%\IEEEtriggeratref{8}
% The "triggered" command can be changed if desired:
%\IEEEtriggercmd{\enlargethispage{-5in}}

% references section


\bibliographystyle{IEEEtran}
\pdfoutput=1
\documentclass{article}
\usepackage[final]{pdfpages}
\begin{document}
\includepdf[pages=1-9]{CVPR18VOlearner.pdf}
\includepdf[pages=1-last]{supp.pdf}
\end{document}
%% argument is your BibTeX string definitions and bibliography database(s)
%\bibliography{reference}


% biography section
% 
% If you have an EPS/PDF photo (graphicx package needed) extra braces are
% needed around the contents of the optional argument to biography to prevent
% the LaTeX parser from getting confused when it sees the complicated
% \includegraphics command within an optional argument. (You could create
% your own custom macro containing the \includegraphics command to make things
% simpler here.)
%\begin{IEEEbiography}[{\includegraphics[width=1in,height=1.25in,clip,keepaspectratio]{mshell}}]{Michael Shell}
% or if you just want to reserve a space for a photo:

% \begin{IEEEbiography}{Michael Shell}
% Biography text here.
% \end{IEEEbiography}

% % if you will not have a photo at all:
% \begin{IEEEbiographynophoto}{John Doe}
% Biography text here.
% \end{IEEEbiographynophoto}

% % insert where needed to balance the two columns on the last page with
% % biographies
% %\newpage

% \begin{IEEEbiographynophoto}{Jane Doe}
% Biography text here.
% \end{IEEEbiographynophoto}

% You can push biographies down or up by placing
% a \vfill before or after them. The appropriate
% use of \vfill depends on what kind of text is
% on the last page and whether or not the columns
% are being equalized.

%\vfill

% Can be used to pull up biographies so that the bottom of the last one
% is flush with the other column.
%\enlargethispage{-5in}

\begin{IEEEbiography}[{\includegraphics[width=1in,height=1.25in,clip,keepaspectratio]{dbera-web.jpg}}]{Debajyoti~Bera}
received his B.Tech. in Computer Science and Engineering in 2002 at Indian Institute of Technology (IIT), Kanpur, India and his Ph.D. degree in Computer Science from Boston University, Massachusetts, USA in 2010. Since 2010 he is an assistant professor at Indraprastha Institute of Information Technology, (IIIT-Delhi), New Delhi, India.
His research interests include quantum computing, randomized algorithms, and engineering algorithms for networks, data mining, and information security.
\end{IEEEbiography}
\vspace{-4mm}

\begin{IEEEbiography}[{\includegraphics[width=1in,height=1.25in,clip,keepaspectratio]{Rameshwar.JPG}}]{Rameshwar Pratap}
 has earned Ph.D in Theoretical Computer Science in 2014  from Chennai Mathematical Institute (CMI). Earlier, he completed Masters in Computer Application (MCA) from  Jawaharlal Nehru University and   BSc in Mathematics, Physics, and Computer Science from University of Allahabad. Post Ph.D he has worked  TCS Innovation Labs (New Delhi, India),  and Wipro AI-Research (Bangalore, India). Since 2019 he is working as an assistant professor at School of Computing and Electrical Engineering (SCEE), IIT Mandi. His research interests include algorithms for dimensionality reduction,  robust sampling, and algorithmic fairness. %Broadly, he is interested in the areas of Algorithms in Machine Learning with a specific focus on areas such as dimensionality reduction, scalable similarity search, robust sampling, and algorithmic fairness. 
 \end{IEEEbiography}
 
 \begin{IEEEbiography}[{\includegraphics[width=1in,height=1.25in,clip,keepaspectratio]{Bhisham.JPG}}]{Bhisham Dev Verma}
 is pursuing Ph.D from IIT Mandi. He has done his Masters in Applied Mathematics from IIT Mandi and BSc in Mathematics, Physics, and Chemistry from Himachal Pradesh University. His research interest includes data mining, algorithms for dimension reduction, optimization and machine learning.
 \end{IEEEbiography}


\newpage
\clearpage
\appendices

\section{Analysis of one-hot encoding + binary compression}\label{appendix:OHE+BS}

Let $x$ and $y$ be two $n$-dimensional categorical vectors with sparsity at most $\sigma$; $c$ will denote the maximum number of values any attribute can take. Let $x'$ and $y'$ be the one-hot encodings of $x$ and $y$, respectively. Further, let $x''$ and $y''$ denote the compression of $x'$ and $y'$, respectively, using BinSketch~\cite{ICDM} which is the state-of-the-art dimensionality reduction for binary vector using Hamming distance.

Observe that the sparsity of $x'$ is same as that of $x$ and a similar claim holds for $y'$ and $y$. However, $HD(x',y')$ does not hold a monotonic relationship with $HD(x,y)$. %Suppose $n_1$ denotes the number of features of $x$ that are missing in $y$, $n_2$ denotes the number of features of $y$ that are missing in $x$, $n_3$ denotes the number of features that are present, but different, in $x$ and $y$, and $n_4$ denotes $n-n_1-n_2-n_3$.
It is easy to show that $HD(x,y) \le HD(x',y') \le 2HD(x,y)$. Therefore, 
\begin{equation}
    |HD(x,y) - HD(x',y')| \le HD(x,y) \le 2\sigma. \label{eqn:2}
\end{equation}
    
We need the following lemma that was used to analyse BinSketch~\cite[Lemma~12,Appendix~A]{ICDM}.
\begin{lem}
Suppose we compress two $n'$-dimensional binary vectors $x'$ and $y'$ with sparsity at most $\sigma$ to $g$-dimensional binary sketches, denotes $x''$ and $y''$ respectively, by following an algorithm proposed in the BinSketch work. If $g$ is set to $\sigma\sqrt{\tfrac{\sigma}{2}\ln \frac{6}{\delta}}$ for any $\delta \in (0,1)$, then the following holds with probability at least $1-\delta$.
$$|HD(x',y') - HD(x'',y'')| \le 6 \sqrt{\tfrac{\sigma}{2} \ln \tfrac{6}{\delta}}.$$
\end{lem}

Combining the above inequality with that in Equation~\ref{eqn:2} gives us
$$|HD(x,y)-HD(x'',y'')| \le 2\sigma + 6\sqrt{\tfrac{\sigma}{2} \ln \tfrac{6}{\delta}} \le 2\sigma\sqrt{\ln \tfrac{2}{\delta}}$$
if we set the reduced dimension to $\sigma\sqrt{\tfrac{\sigma}{2}\ln \frac{6}{\delta}}$.

This bound is worse compared to that of \fsketch where we able to prove an accuracy of $\Theta(\sqrt{\sigma\ln \tfrac{2}{\delta}})$ using reduced dimension value of $4\sigma$ (see Lemma~\ref{lem:hconcentrationlemma}).

\section{Proofs from Section~\ref{subsec:analysis}}\label{appendix:subsec:analysis}

%\expectationlemma*
%
%\begin{proof} Fix a mapping $\rho$ and then define $F_j(x)$ as the vector $[x_{i_1}, x_{i_2}, \ldots ~:~ i_k \in \{1, \ldots n\}]$ of values of $x$ that are mapped to $j$ in $\phi(x)$ in the increasing order of their coordinates, i.e., $\rho(i_k)=j$ and $i_1 < \ldots i_k < i_{k+1}$. Since $\rho$ is fixed, $F_j(y)$ is also a vector of the same length. The key observation is that if $F_j(x)=F_j(y)$ then $\phi_j(x)=\phi_j(y)$; the converse is not always true.
%
%It is given that $x$ and $y$ differ at $h$ coordinates. Therefore, $F_j(x)\not= F_j(y)$ iff any of those coordinates are mapped to $j$ by $\rho$. Thus, 
%\begin{align*}
%    \Pr_\rho[F_j(x)=F_j(y)] = (1-\tfrac{1}{d})^h. \numberthis \label{eq:prob_pi}
%\end{align*}
%
%
%Next we analyse the chance of $\phi_j(x)=\phi_j(y)$ when $F_j(x)\not=F_j(y)$. Note that $\phi_j(x)=(x_{i_1} \cdot r_{i_1} + x_{i_2} \cdot r_{i_2} + \ldots) \mod{p}$ (and a similar expression exists for $y$), where $r_i$s are randomly chosen during initialization (they are fixed for $x$ and $y$). Using a similar analysis as that in the Frievald's algorithm~\cite[Ch~1(Verifying matrix multiplication)]{upfal},
%\begin{align*}
%\Pr_{\rho,R}[\phi_j(x)=\phi_j(y) ~|~ F_j(x)\not=F_j(y)] = \tfrac{1}{p}. \numberthis \label{eq:prob_pi_r}   
%\end{align*}
%
%Due to Equations~\ref{eq:prob_pi}, \ref{eq:prob_pi_r}, we have
%\begin{align*}
%& \Pr_{\rho,R} [\phi_j(x)\not=\phi_j(y)] \\
%= & \Pr_{\rho,R} [\phi_j(x)\not=\phi_j(y) ~|~ F_j(x)\not=F_j(y)] \cdot \Pr_{\rho,R} [F_j(x)\not=F_j(y)] \\
%+ & \Pr_{\rho,R} [\phi_j(x)\not=\phi_j(y) ~|~ F_j(x)=F_j(y)] \cdot \Pr_{\rho,R} [F_j(x)=F_j(y)]\\
%= & (1-\tfrac{1}{p})(1-(1-\tfrac{1}{d})^h).
%\end{align*}
%The lemma follows from the linearity of expectation.
%\end{proof}



\concentrationlemma*

\begin{proof}
Fix any $R$ and $x,y$; the rest of the proof applies to any $R$, and therefore, holds for a random $R$ as well. Define a vector $z \in \{0,\pm 1,\ldots,\pm c\}^n$ in which $z_i=(x_i - y_i)$; the number of non-zero entries of $z$ are at most $2\sigma$ since the number of non-zero entries of $x$ and $y$ are at most $\sigma$. Let $J_0$ be the set of coordinates from $\{1, \ldots, n\}$ at which $z$ is 0, and let $J_1$ be the set of the rest of the coordinates; from above, $J_1 \le 2\sigma$.% We will prove the lemma for $d=\tfrac{2}{\epsilon}\sqrt{\sigma \ln \tfrac{2}{\delta}}$ which automatically proves it for larger values of $d$.

Define the event $E_j$ as ``$[\phi_j(x)\not=\phi_j(y)]$''. Note that $f$ can be written as a sum of indicator random variables, $\sum_j I(E_j)$, and we would like to prove that $f$ is almost always close to $f^*=\E[f]$.

Observe that $\phi_j(x)=\phi_j(y)$ iff $\sum_{i \in \rho^{-1}(j)} z_i \cdot r_i = 0 \mod{p}$ iff $\sum_{i \in \rho^{-1}(j) \cap J_1} z_i \cdot r_i = 0 \mod{p}$. In other words, $\rho(i)$ could be set to anything for $i \in J_0$ without any effect on the event $E_j$; hence, we will assume that the mapping $\rho$ is defined as a random mapping only for $i \in J_1$, and further for the ease of analysis, we will denote them as $\rho(i_1), \rho(i_2), \ldots, \rho(i_{2\sigma})$ (if $|J_1| < 2\sigma$ then move a few coordinates from $J_0$ to $J_1$ without any loss of correctness).

To prove the concentration bound we will employ martingales. Consider the sequence of these random variables $\rho'=\rho(i_1), \rho(i_2), \ldots, \rho(i_{2\sigma})$ -- these are independent. Define a function $g(\rho')$ of these random variables as a sum of indicator random variables as stated below (note that $R$ and $\rho(i)$, for $i \in J_0$, are fixed at this point)
\begin{align*}
    & g(\rho(i_1), \rho(i_2), \ldots \rho(i_{2\sigma}))\\
    & = \sum_j I\left( \sum_{i \in \rho^{-1}(j) \cap J_1} z_i \cdot r_i \not= 0 \mod{p} \right) \\
    & = \sum_j I(E_j) = f
\end{align*}
 

Now consider an arbitrary $t\in \{1, \ldots, 2\sigma\}$ and let $q=\rho(i_t)$; observe that $z_{i_t}$ influences only $E_q$. Choose an arbitrary value $q' \in \{1, \ldots, d\}$ that is different from $q$. Observe that, if $\rho$ is modified only by setting $\rho(i_t)=q'$ then we claim that ``bounded difference holds''.

\begin{prop}
$|~ g(\rho(i_1), \ldots, \rho(i_{t-1}), q, \ldots, \rho(i_{2\sigma})) - g(\rho(i_1), \ldots, \rho(i_{t-1}), q', \ldots, \rho(i_{2\sigma})) ~| \le 2$.
\end{prop}
The proposition holds since the only effects of the change of $\rho(i_t)$ from $q$ to $q'$ are seen in $E_q$ and $E_{q'}$ (earlier $E_q$ depended upon $z_{i_t}$ that now changes to $E_{q'}$ being depended upon $z_{i_t}$). Since $g()$ obeys bounded difference, therefore, we can apply McDiarmid's inequality~\cite[Ch~17]{upfal},~\cite{mcdiarmid_1989}.
% McDiarmid's inequality~\cite[Ch 17]{upfal},~\cite{mcdiarmid_1989} states that
\begin{thm}[McDiarmid's inequality\label{thm:McDiarmid}]
Consider independent random variables $X_1,\ldots , X_m \in \mathcal{X}$, and a mapping $ f:\mathcal{X}^m \rightarrow \R$ which   for all $i$ and for all $x_1,\ldots x_m, {x_i}'$ satisfies the property:
$|f(x_1,\ldots,x_i,\ldots,x_m)-f(x_1,\ldots,{x_i}',\ldots,x_m)|\leq c_i$, where $x_1,\ldots x_m, {x_i}'$ are possible values for the input variables of the function $f$. Then,
% \[
% \Pr\big[ \E[f(X_1,\ldots , X_m) - f(X_1,\ldots , X_m)  \geq \varepsilon \big]\leq \exp \left(\frac{-2\varepsilon^2}{\sum_{i=1}^m c_i^2} \right),
% \]
    \begin{align*}
	\Pr\Big[ \big| \E[f(X_1,\ldots , X_m) - f(X_1,\ldots , X_m)]\big|  \geq \varepsilon \Big] \\
	\leq 2\exp \left(\frac{-2\varepsilon^2}{\sum_{i=1}^m c_i^2} \right).
    \end{align*}
\end{thm}

% ~\footnote{McDiarmid's inequality~\cite[Ch 17]{upfal},~\cite{mcdiarmid_1989} states that
% \begin{thm}
% Consider independent random variables $X_1,\ldots , X_m \in \mathcal{X}$, and a mapping $ f:\mathcal{X}^m \rightarrow \R$ which   for all $i$ and for all $x_1,\ldots x_m, {x_i}'$ satisfies the property:
% $|f(x_1,\ldots,x_i,\ldots,x_m)-f(x_1,\ldots,{x_i}',\ldots,x_m)|\leq c_i$, where $x_1,\ldots x_m, {x_i}'$ are possible values for the input variables of the function $f$. Then,
% % \[
% % \Pr\big[ \E[f(X_1,\ldots , X_m) - f(X_1,\ldots , X_m)  \geq \varepsilon \big]\leq \exp \left(\frac{-2\varepsilon^2}{\sum_{i=1}^m c_i^2} \right),
% % \]
% \[
% \Pr\Big[ \big| \E[f(X_1,\ldots , X_m) - f(X_1,\ldots , X_m)\big|  \geq \varepsilon \Big]\leq 2\exp \left(\frac{-2\varepsilon^2}{\sum_{i=1}^m c_i^2} \right).
% \]
% \end{thm}
% }
This inequality implies that, for every $x,y,R$,
$$\Pr_{\rho} \Big[ \big| \E[f] - f \big| \ge \alpha \Big] \le 2\exp\left( -\frac{2\alpha^2}{(2\sigma)2^2} \right) = \exp \left( - \frac{\alpha^2}{4\sigma} \right).$$
Hence, the lemma is proved.
\end{proof}

\upperboundf*

\begin{proof}
    Since $f^*=dP(1-D^h)=dP - dPD^h$, if $f \ge dP$ then $|f-f^*| \ge dPD^h$. 
    \begin{align*}
	\Pr[f \ge dP] & \le \Pr[|f-f^*| \ge dPD^h]\\
		      & \le 2 \exp(-\tfrac{d^2 P^2 D^{2h}}{4\sigma}) \tag{using Lemma~\ref{lem:concentration}}\\
		      & = 2 \exp(-\tfrac{P^2}{4\sigma} d^2(1-\tfrac{1}{d})^{2h})\\
		      & \le 2 \exp(-\tfrac{P^2}{4\sigma} (d-h)^2) \tag{$\because~(1-\tfrac{1}{d})^h \ge 1-\tfrac{h}{d}$}\\
		      & \le 2 \exp(-P^2\sigma) \tag{$\because$ $\tfrac{(d-h)^2}{4\sigma} \ge \sigma$}
    \end{align*}
    Here we have used the fact that $h \le 2\sigma$ which, along with the setting $d=4\sigma$, implies that $(d-h) \ge 2\sigma$.
\end{proof}

\hconcentrationlemma*

\begin{proof}
    Denote $|\hat{h}-h|$ by $\Delta h$ and let $\alpha=\sqrt{d \ln
    \tfrac{2}{\delta}}$. We will prove that $\Delta h < \tfrac{32}{P}\sqrt{\sigma \ln \tfrac{2}{\delta}}$ for the case $|f-f^*| \le \alpha$ which, by
    Lemma~\ref{lem:concentration}, happens
    with probability at least $(1-2\exp{(-\tfrac{\alpha^2}{4\sigma})})=1-\delta$.

    First we make a few technical observations all of which are based on
    standard inequalities of binomial series and logarithmic functions.
    It will be helpful to remember that $D=1-1/d \in (0,1)$.
    \begin{obs}\label{obs:primep}
	For reasonable values of $\sigma$, and
	reasonable values of $\delta$, almost all
	primes satisfy the bound $P \ge \tfrac{4}{\sqrt{\sigma}}\sqrt{\ln
	\tfrac{2}{\delta}}$. We will assume this inequality to hold without loss
	of generality~\footnote{If the reader is wondering why we are not
	proving this fact, it may be observed that this relationship does not
	hold for small values of $\sigma$, e.g., $\sigma=16, \delta=0.01$.}.
    \end{obs}
    For example, $p=2$ is sufficient for $\sigma \approx 1000$ and $\delta
    \approx 0.001$ (remember that $P=1-\tfrac{1}{p}$). Furthermore, observe that
    $P$ is an increasing function of $p$, and the right hand side is a
    decreasing function of $\sigma$, increasing with decreasing delta but at an
    extremely slow logarithmic rate.
    
    \begin{obs}\label{obs:dp-alpha}
	$\tfrac{dP}{\alpha} > 4$ can be assumed without loss of generality. This holds since the left hand side is
	$\tfrac{dP}{\sqrt{d}\sqrt{ln(2/\delta)}} =
	\tfrac{P\sqrt{d}}{\sqrt{\ln(2/\delta)}} \ge
	\tfrac{4\sqrt{d}}{\sqrt{\sigma}}$ (by Observation~\ref{obs:primep})
	which is at least $4$.
    \end{obs}

    \begin{obs}\label{obs:f_less_than_dP}
	Based on the above assumptions, $f < dP$.
    \end{obs}
    \begin{proof}[Proof of Observation] We will prove that $\sqrt{d \ln \tfrac{2}{\delta}} < dPD^h$. Since $|f-f^*| \le \sqrt{d \ln \tfrac{2}{\delta}}$ and $f^*=dP(1-D^h)$, it follows that $f \le f^* + \sqrt{d \ln \tfrac{2}{\delta}} < dP$.

    \begin{align*}
	\sqrt{d}PD^h & = \frac{dPD^h}{\sqrt{d}} \ge \frac{P}{\sqrt{d}}d(1-\tfrac{1}{d})^h \ge \frac{P}{\sqrt{d}}d(1-\tfrac{h}{d})\\
		     & = \frac{P}{\sqrt{d}}(d-h) \ge \frac{P}{\sqrt{d}}\frac{d}{2} \tag{$\because$ $h \le 2\sigma$, $d-h \ge 2\sigma = \tfrac{d}{2}$}\\
		     & = P \sqrt{\sigma} \ge 4\sqrt{\ln \tfrac{2}{\delta}} \tag{Observation~\ref{obs:primep}}
    \end{align*}
    which proves the claim stated at the beginning of the proof.
    \end{proof}

    Based on this observation, $\hat{h}$ is calculated as $\ln\left( 1 - \tfrac{f}{dP} \right)/\ln D$ (see Definition~\ref{defn:estimator}). 
    Thus, we get $D^{\hat{h}} = 1 - \tfrac{f}{dP}$.
    Further, from Equation~\ref{eqn:fstar} we get $D^h = 1-\tfrac{f^*}{dP}$.
    

    \begin{obs}\label{obs:appendix-1} $D^h \ge D^{2\sigma} \ge \tfrac{9}{16}$. This is since $h \le 2\sigma$ and 
	$D^\sigma = (1-\tfrac{1}{d})^\sigma \ge 1 - \tfrac{\sigma}{d} =
	\tfrac{3}{4}$.
    \end{obs}

    \begin{restatable}{obs}{obsappendix}\label{obs:appendix-2}
	$D^{\hat{h}} > \tfrac{5}{16}$.
    \end{restatable}
	This is not so straight forward as
	Observation~\ref{obs:appendix-1} since $\hat{h}$ is calculated using a
	formula and is not guaranteed, ab initio, to be upper bounded by
	$2\sigma$.
	\begin{proof}[Proof of Observation]
	    We will prove that $\tfrac{f}{dP} < \tfrac{11}{16}$ which will imply that $D^{\hat{h}} = 1 - \tfrac{f}{dP} > \tfrac{5}{16}$.
	   
	    For the proof of the lemma we have considered the case that $f \le f^* + \alpha$. 
	    Therefore, $\tfrac{f}{dP} \le \tfrac{f^*}{dP} + \tfrac{\alpha}{dP}$.
	    Substituting the value of $f^*=dP(1-D^h)$ from Equation~\ref{eqn:fstar} and
	    using Observation~\ref{obs:appendix-1} we get the bound $\tfrac{f}{dP} \le
	    \tfrac{7}{16} + \tfrac{\alpha}{dP}$. We can further simplify the bound using Observation~\ref{obs:dp-alpha}:
    $$\tfrac{f}{dP} \le \tfrac{7}{16} + \tfrac{\alpha}{dP} \le \tfrac{7}{16} +
    \tfrac{1}{4} < \tfrac{11}{16}, \mbox{ validating the observation.}$$

	\end{proof}

    Now we get into the main proof which proceeds by considering two possible
    cases.
    
    {\bf (Case $\hat{h} \ge h$, i.e., $\Delta h=\hat{h}-h$:)} 
    We start with the identity $D^h - D^{\hat{h}} = \tfrac{f-f^*}{dP}$.

    Notice that the RHS is bounded from the above by $\tfrac{\alpha}{dP}$ and
    the LHS can bounded from the below as
    $$D^h - D^{\hat{h}} = D^h(1-D^{\Delta h}) > \tfrac{9}{16}(1-D^{\Delta h})$$
    where we have used Observation~\ref{obs:appendix-1}. Combining these facts
    we get $\tfrac{\alpha}{dP} > \tfrac{9}{16}(1-D^{\Delta h})$.

    {\bf (Case $h \ge \hat{h}$, i.e., $\Delta h = h - \hat{h}$:)} In a similar
    manner, we start with the identity $D^{\hat{h}} - D^h = \tfrac{f^* -
    f}{dP}$ in which the RHS we bound again from the above by
    $\tfrac{\alpha}{dP}$ and the LHS is treated similarly (but now using
    Observation~\ref{obs:appendix-2}).
    $$D^{\hat{h}} - D^h = D^{\hat{h}}(1-D^{\Delta h}) > \tfrac{5}{16}
    (1-D^{\Delta h})$$ and then, $\tfrac{\alpha}{dP} >
    \tfrac{5}{16}(1-D^{\Delta h})$.

    So in both the cases we show that $\tfrac{\alpha}{dP} >
    \tfrac{5}{16}(1-D^{\Delta h})$. Our desired bound on $\Delta h$ can now be
    obtained.
    \begin{align*}
	\Delta h \ln D & \ge \ln\left( 1 - \tfrac{16}{5}\tfrac{\alpha}{dP} \right)
			 \ge -\tfrac{16\alpha}{5dP}/(1-\tfrac{16\alpha}{5dP}) =
	-\tfrac{16\alpha}{5dP-16\alpha}\\
	& \mbox{ (using the inequality $\ln(1+x) \ge
	\tfrac{x}{1+x}$ for $x > -1$)} \\
	\therefore \Delta h & \le \frac{1}{\ln\tfrac{1}{D}} \frac{16\alpha}{5dP -
	16\alpha} \le \frac{16\alpha d}{5dP - 16\alpha}\\
	& \mbox{ (it is easy to show that $\ln \tfrac{1}{D} =
	\ln\tfrac{1}{1-1/d} \ge 1/d$)}\\
	& = \frac{\tfrac{16}{5}d}{\tfrac{dP}{\alpha} - \tfrac{16}{5}} \\
	& < \frac{\tfrac{16}{5}d}{\tfrac{dP}{5\alpha}} 
	\mbox{ (using Observation~\ref{obs:dp-alpha}, $\tfrac{dP}{\alpha}-\tfrac{16}{5} >\tfrac{dP}{5\alpha}$)}\\
	& = \frac{16\alpha}{P} = \frac{16}{P}\sqrt{d \ln \tfrac{2}{\delta}} =
	\frac{32}{P}\sqrt{\sigma \ln \tfrac{2}{\delta}}
    \end{align*}
\end{proof}


\hconcentrationlemmatight*

\begin{proof}[Proof of (a) $f < dP$ with high probability]
    Following the steps of the proof of Lemma~\ref{lem:upper-bound-f},
    \begin{align*}
	\Pr[f \ge dP] & \le 2\exp(-\tfrac{d^2 P^2 D^{2h}}{4\sigma}) \\
		      & \le 2\exp(-P^2\tfrac{(d-h)^2}{4\sigma})
    \end{align*}
    Let $L$ denote $\sqrt{\ln\tfrac{2}{\delta}}$; note that $L > 1$. Now, $d=16L\sqrt{\sigma}$ and $h \le \sqrt{\sigma}$. So, $d-h \ge 15L\sqrt{\sigma} > 15\sqrt{\sigma}$ and, therefore, $\tfrac{(d-h)^2}{\sigma} > 225$. Using this bound in the equation above, we can upper bound the right-hand side as $2\exp(-225(1-\tfrac{1}{p})^2/4)$ which is a decreasing function of $p$, the lowest (for $p=2$) being $2\exp(-225/4*4) \approx 10^{-6}$.
\end{proof}

\begin{proof}[Proof of (b) a better estimator of $h$]
    The proof is almost exactly same as that of
    Lemma~\ref{lem:hconcentrationlemma}, with only a few differences. We set
    $\alpha=d/8$ where $d=16\sqrt{\sigma \ln \tfrac{2}{\delta}}$. Incidentally,
    the value of $\alpha$ remains the same in terms of $\sigma$ ( $\alpha=\sqrt{4\sigma
    \ln \tfrac{2}{\delta}}$). Thus, the
    probability of error remains same as before;
    $$2 \exp{(-\tfrac{d^2}{64 \cdot 4 \sigma})} = \delta.$$

    Observation~\ref{obs:primep} is true without any doubt.
    $\tfrac{dP}{\alpha} = 8P$ which is greater than 4 for any prime number; so
    Observation~\ref{obs:dp-alpha} is true in this scenario.

    Observation~\ref{obs:f_less_than_dP} requires a new proof. Following the steps of the above proof of Observation~\ref{obs:f_less_than_dP}, it suffices to prove that $dPD^h > \tfrac{d}{8}$.
    \begin{align*}
	PD^h & = P(1-\tfrac{1}{d})^h \ge P(1-\tfrac{h}{d}) \\
	     & =P(\tfrac{d-h}{d}) \ge P\tfrac{15L\sqrt{\sigma}}{16L\sqrt{\sigma}} = P\tfrac{15}{16} > \tfrac{1}{2}\tfrac{15}{16} > \tfrac{1}{8}
    \end{align*}

    Observation~\ref{obs:appendix-1} is now tighter since $D^h \ge D^{\sqrt{\sigma}}
    = (1-\tfrac{1}{d})^{\sqrt{\sigma}} \ge 1-\tfrac{\sqrt{\sigma}}{d} = 1 -
    \tfrac{1}{16\sqrt{\ln 2/\delta}} \ge \tfrac{3}{4}$ for reasonable values of
    $\delta$.
    Similarly Observation~\ref{obs:appendix-2} is also tighter (it relies on only the above
    observations) since $\tfrac{f^*}{dP} = 1 - D^h \le 1-\tfrac{3}{4}$ and $\tfrac{\alpha}{dP} < \tfrac{1}{4}$; we get $D^{\hat{h}} > \tfrac{1}{2}$.

    Following similar steps as above, for the case $\hat{h} \ge h$, we get $\tfrac{\alpha}{dP} > \tfrac{3}{4}(1-D^{\Delta h})$ and for the case $\hat{h} < h$, we get $\tfrac{\alpha}{dP} > \tfrac{1}{2}(1-D^{\Delta h})$ leading to the common condition that $\tfrac{\alpha}{dP} > \tfrac{1}{2}(1-D^{\Delta h})$.

    The final thing to calculate is the bound on $\Delta h$.
    \begin{align*}
	\Delta h \ln D & \ge \ln\left( 1 - \tfrac{2\alpha}{dP} \right)
			 \ge -\tfrac{2\alpha}{dP}/(1-\tfrac{2\alpha}{dP}) =
	-\tfrac{2\alpha}{dP-2\alpha}\\
	& \mbox{ (using the inequality $\ln(1+x) \ge
	\tfrac{x}{1+x}$ for $x > -1$)} \\
	\therefore \Delta h & \le \frac{1}{\ln\tfrac{1}{D}} \frac{2\alpha}{dP -
	2\alpha} \le \frac{2\alpha d}{dP - 2\alpha}\\
	& \mbox{ (it is easy to show that $\ln \tfrac{1}{D} =
	\ln\tfrac{1}{1-1/d} \ge 1/d$)}\\
	& = \frac{2d}{\tfrac{dP}{\alpha} - 2} \\
	& < \frac{2d}{\tfrac{dP}{2\alpha}} 
	\mbox{ (using Observation~\ref{obs:dp-alpha}, $\tfrac{dP}{\alpha}-2 >\tfrac{dP}{2\alpha}$)}\\
	& = \frac{4\alpha}{P} = \frac{4}{P}\sqrt{4\sigma \ln \tfrac{2}{\delta}} =
	\frac{8}{P}\sqrt{\sigma \ln \tfrac{2}{\delta}}
    \end{align*}
\end{proof}

\section{Complexity analysis of \fsketch}\label{subsec:appendix-complexity}
There
are two major operations with respect to \fsketch --- construction of sketches
and estimation of Hamming distance from two sketches. We will discuss their time and space requirements. There are efficient representations of sparse data vectors, but for the sake of simplicity we assume full-size arrays to store vectors; similarly we assume simple dictionaries for storing the interval variables $\rho,R$ by \fsketch. While it may be possible to reduce the number of random bits by employing $k$-wise independent bits and mappings, we left it out of the scope of this work and for future exploration.
\begin{enumerate}
    \item{Construction:} Sketches are constructed by the \fsketch algorithm which does a linear pass over the input vector, maps every non-zero attribute to some entry of the sketch vector and then updates that corresponding entry. The time to process one data vector becomes $\Theta(n) + O(\sigma \cdot poly(\log p))$ which is $O(n)$ for constant $p$. The interval variables, $\rho,R,p$, require space $\Theta(n \log d)$, $\Theta(n \log p)$ and $\Theta(\log p)$, respectively, which is almost $O(n)$ if $\sigma \ll n$. Furthermore, $\rho$ and $R$, that can consume bulk of this space, can be freed once the sketch construction phase is over. A sketch itself consumes $\Theta(d \log p)$ space.
    \item{Estimation:} There is no additional space requirement for estimating the Hamming distance of a pair of points from their sketches. The estimator scans both the sketches and computes their Hamming distance; finally it computes an estimate by using Definition~\ref{defn:estimator}. The running time is $O(d \log p)$.
\end{enumerate}



\section{Proofs from Section~\ref{subsec:complexity}}\label{appendix:subsec:complexity}

\sparsitylemma*

\begin{proof}
The lemma can be proved by treating it as a balls-and-bins problem. Imagine
    throwing $\sigma$ balls (treat them as the non-zero attributes of $x$) into
    $d$ bins (treat them as the sketch cells) independently and uniformly at
    random. If the $j$th-bin remains empty then $\phi_j(x)$ must be zero (the
    converse is not true). Therefore, the expected number of non-zero cells in the
    sketch is upper bounded by the expected number of empty bins, which can be
    easily shown to be $d[1-(1-\tfrac{1}{d})^\sigma]$. Using the stated value of
    $d$, this expression can further be upper bounded.
    $$d[1-(1-\tfrac{1}{d})^\sigma] \le d[1-(1-\tfrac{\sigma}{d})] =
    \tfrac{d}{4}$$
    Furthermore, let $NZ$ denote the number of non-zero entries in $\phi(x)$. We
    derived above $\E[NZ] \le \tfrac{d}{4}$. Markov inequality can help in upper
    bounding the probability that $\phi(x)$ contains many non-zero entries.
    $$\Pr[NZ \ge \tfrac{d}{2}] \le \E[NZ]/\tfrac{d}{2} \le \tfrac{1}{2}$$
\end{proof}


\chapter{Supplementary Material}
\label{appendix}

In this appendix, we present supplementary material for the techniques and
experiments presented in the main text.

\section{Baseline Results and Analysis for Informed Sampler}
\label{appendix:chap3}

Here, we give an in-depth
performance analysis of the various samplers and the effect of their
hyperparameters. We choose hyperparameters with the lowest PSRF value
after $10k$ iterations, for each sampler individually. If the
differences between PSRF are not significantly different among
multiple values, we choose the one that has the highest acceptance
rate.

\subsection{Experiment: Estimating Camera Extrinsics}
\label{appendix:chap3:room}

\subsubsection{Parameter Selection}
\paragraph{Metropolis Hastings (\MH)}

Figure~\ref{fig:exp1_MH} shows the median acceptance rates and PSRF
values corresponding to various proposal standard deviations of plain
\MH~sampling. Mixing gets better and the acceptance rate gets worse as
the standard deviation increases. The value $0.3$ is selected standard
deviation for this sampler.

\paragraph{Metropolis Hastings Within Gibbs (\MHWG)}

As mentioned in Section~\ref{sec:room}, the \MHWG~sampler with one-dimensional
updates did not converge for any value of proposal standard deviation.
This problem has high correlation of the camera parameters and is of
multi-modal nature, which this sampler has problems with.

\paragraph{Parallel Tempering (\PT)}

For \PT~sampling, we took the best performing \MH~sampler and used
different temperature chains to improve the mixing of the
sampler. Figure~\ref{fig:exp1_PT} shows the results corresponding to
different combination of temperature levels. The sampler with
temperature levels of $[1,3,27]$ performed best in terms of both
mixing and acceptance rate.

\paragraph{Effect of Mixture Coefficient in Informed Sampling (\MIXLMH)}

Figure~\ref{fig:exp1_alpha} shows the effect of mixture
coefficient ($\alpha$) on the informed sampling
\MIXLMH. Since there is no significant different in PSRF values for
$0 \le \alpha \le 0.7$, we chose $0.7$ due to its high acceptance
rate.


% \end{multicols}

\begin{figure}[h]
\centering
  \subfigure[MH]{%
    \includegraphics[width=.48\textwidth]{figures/supplementary/camPose_MH.pdf} \label{fig:exp1_MH}
  }
  \subfigure[PT]{%
    \includegraphics[width=.48\textwidth]{figures/supplementary/camPose_PT.pdf} \label{fig:exp1_PT}
  }
\\
  \subfigure[INF-MH]{%
    \includegraphics[width=.48\textwidth]{figures/supplementary/camPose_alpha.pdf} \label{fig:exp1_alpha}
  }
  \mycaption{Results of the `Estimating Camera Extrinsics' experiment}{PRSFs and Acceptance rates corresponding to (a) various standard deviations of \MH, (b) various temperature level combinations of \PT sampling and (c) various mixture coefficients of \MIXLMH sampling.}
\end{figure}



\begin{figure}[!t]
\centering
  \subfigure[\MH]{%
    \includegraphics[width=.48\textwidth]{figures/supplementary/occlusionExp_MH.pdf} \label{fig:exp2_MH}
  }
  \subfigure[\BMHWG]{%
    \includegraphics[width=.48\textwidth]{figures/supplementary/occlusionExp_BMHWG.pdf} \label{fig:exp2_BMHWG}
  }
\\
  \subfigure[\MHWG]{%
    \includegraphics[width=.48\textwidth]{figures/supplementary/occlusionExp_MHWG.pdf} \label{fig:exp2_MHWG}
  }
  \subfigure[\PT]{%
    \includegraphics[width=.48\textwidth]{figures/supplementary/occlusionExp_PT.pdf} \label{fig:exp2_PT}
  }
\\
  \subfigure[\INFBMHWG]{%
    \includegraphics[width=.5\textwidth]{figures/supplementary/occlusionExp_alpha.pdf} \label{fig:exp2_alpha}
  }
  \mycaption{Results of the `Occluding Tiles' experiment}{PRSF and
    Acceptance rates corresponding to various standard deviations of
    (a) \MH, (b) \BMHWG, (c) \MHWG, (d) various temperature level
    combinations of \PT~sampling and; (e) various mixture coefficients
    of our informed \INFBMHWG sampling.}
\end{figure}

%\onecolumn\newpage\twocolumn
\subsection{Experiment: Occluding Tiles}
\label{appendix:chap3:tiles}

\subsubsection{Parameter Selection}

\paragraph{Metropolis Hastings (\MH)}

Figure~\ref{fig:exp2_MH} shows the results of
\MH~sampling. Results show the poor convergence for all proposal
standard deviations and rapid decrease of AR with increasing standard
deviation. This is due to the high-dimensional nature of
the problem. We selected a standard deviation of $1.1$.

\paragraph{Blocked Metropolis Hastings Within Gibbs (\BMHWG)}

The results of \BMHWG are shown in Figure~\ref{fig:exp2_BMHWG}. In
this sampler we update only one block of tile variables (of dimension
four) in each sampling step. Results show much better performance
compared to plain \MH. The optimal proposal standard deviation for
this sampler is $0.7$.

\paragraph{Metropolis Hastings Within Gibbs (\MHWG)}

Figure~\ref{fig:exp2_MHWG} shows the result of \MHWG sampling. This
sampler is better than \BMHWG and converges much more quickly. Here
a standard deviation of $0.9$ is found to be best.

\paragraph{Parallel Tempering (\PT)}

Figure~\ref{fig:exp2_PT} shows the results of \PT sampling with various
temperature combinations. Results show no improvement in AR from plain
\MH sampling and again $[1,3,27]$ temperature levels are found to be optimal.

\paragraph{Effect of Mixture Coefficient in Informed Sampling (\INFBMHWG)}

Figure~\ref{fig:exp2_alpha} shows the effect of mixture
coefficient ($\alpha$) on the blocked informed sampling
\INFBMHWG. Since there is no significant different in PSRF values for
$0 \le \alpha \le 0.8$, we chose $0.8$ due to its high acceptance
rate.



\subsection{Experiment: Estimating Body Shape}
\label{appendix:chap3:body}

\subsubsection{Parameter Selection}
\paragraph{Metropolis Hastings (\MH)}

Figure~\ref{fig:exp3_MH} shows the result of \MH~sampling with various
proposal standard deviations. The value of $0.1$ is found to be
best.

\paragraph{Metropolis Hastings Within Gibbs (\MHWG)}

For \MHWG sampling we select $0.3$ proposal standard
deviation. Results are shown in Fig.~\ref{fig:exp3_MHWG}.


\paragraph{Parallel Tempering (\PT)}

As before, results in Fig.~\ref{fig:exp3_PT}, the temperature levels
were selected to be $[1,3,27]$ due its slightly higher AR.

\paragraph{Effect of Mixture Coefficient in Informed Sampling (\MIXLMH)}

Figure~\ref{fig:exp3_alpha} shows the effect of $\alpha$ on PSRF and
AR. Since there is no significant differences in PSRF values for $0 \le
\alpha \le 0.8$, we choose $0.8$.


\begin{figure}[t]
\centering
  \subfigure[\MH]{%
    \includegraphics[width=.48\textwidth]{figures/supplementary/bodyShape_MH.pdf} \label{fig:exp3_MH}
  }
  \subfigure[\MHWG]{%
    \includegraphics[width=.48\textwidth]{figures/supplementary/bodyShape_MHWG.pdf} \label{fig:exp3_MHWG}
  }
\\
  \subfigure[\PT]{%
    \includegraphics[width=.48\textwidth]{figures/supplementary/bodyShape_PT.pdf} \label{fig:exp3_PT}
  }
  \subfigure[\MIXLMH]{%
    \includegraphics[width=.48\textwidth]{figures/supplementary/bodyShape_alpha.pdf} \label{fig:exp3_alpha}
  }
\\
  \mycaption{Results of the `Body Shape Estimation' experiment}{PRSFs and
    Acceptance rates corresponding to various standard deviations of
    (a) \MH, (b) \MHWG; (c) various temperature level combinations
    of \PT sampling and; (d) various mixture coefficients of the
    informed \MIXLMH sampling.}
\end{figure}


\subsection{Results Overview}
Figure~\ref{fig:exp_summary} shows the summary results of the all the three
experimental studies related to informed sampler.
\begin{figure*}[h!]
\centering
  \subfigure[Results for: Estimating Camera Extrinsics]{%
    \includegraphics[width=0.9\textwidth]{figures/supplementary/camPose_ALL.pdf} \label{fig:exp1_all}
  }
  \subfigure[Results for: Occluding Tiles]{%
    \includegraphics[width=0.9\textwidth]{figures/supplementary/occlusionExp_ALL.pdf} \label{fig:exp2_all}
  }
  \subfigure[Results for: Estimating Body Shape]{%
    \includegraphics[width=0.9\textwidth]{figures/supplementary/bodyShape_ALL.pdf} \label{fig:exp3_all}
  }
  \label{fig:exp_summary}
  \mycaption{Summary of the statistics for the three experiments}{Shown are
    for several baseline methods and the informed samplers the
    acceptance rates (left), PSRFs (middle), and RMSE values
    (right). All results are median results over multiple test
    examples.}
\end{figure*}

\subsection{Additional Qualitative Results}

\subsubsection{Occluding Tiles}
In Figure~\ref{fig:exp2_visual_more} more qualitative results of the
occluding tiles experiment are shown. The informed sampling approach
(\INFBMHWG) is better than the best baseline (\MHWG). This still is a
very challenging problem since the parameters for occluded tiles are
flat over a large region. Some of the posterior variance of the
occluded tiles is already captured by the informed sampler.

\begin{figure*}[h!]
\begin{center}
\centerline{\includegraphics[width=0.95\textwidth]{figures/supplementary/occlusionExp_Visual.pdf}}
\mycaption{Additional qualitative results of the occluding tiles experiment}
  {From left to right: (a)
  Given image, (b) Ground truth tiles, (c) OpenCV heuristic and most probable estimates
  from 5000 samples obtained by (d) MHWG sampler (best baseline) and
  (e) our INF-BMHWG sampler. (f) Posterior expectation of the tiles
  boundaries obtained by INF-BMHWG sampling (First 2000 samples are
  discarded as burn-in).}
\label{fig:exp2_visual_more}
\end{center}
\end{figure*}

\subsubsection{Body Shape}
Figure~\ref{fig:exp3_bodyMeshes} shows some more results of 3D mesh
reconstruction using posterior samples obtained by our informed
sampling \MIXLMH.

\begin{figure*}[t]
\begin{center}
\centerline{\includegraphics[width=0.75\textwidth]{figures/supplementary/bodyMeshResults.pdf}}
\mycaption{Qualitative results for the body shape experiment}
  {Shown is the 3D mesh reconstruction results with first 1000 samples obtained
  using the \MIXLMH informed sampling method. (blue indicates small
  values and red indicates high values)}
\label{fig:exp3_bodyMeshes}
\end{center}
\end{figure*}

\clearpage



\section{Additional Results on the Face Problem with CMP}

Figure~\ref{fig:shading-qualitative-multiple-subjects-supp} shows inference results for reflectance maps, normal maps and lights for randomly chosen test images, and Fig.~\ref{fig:shading-qualitative-same-subject-supp} shows reflectance estimation results on multiple images of the same subject produced under different illumination conditions. CMP is able to produce estimates that are closer to the groundtruth across different subjects and illumination conditions.

\begin{figure*}[h]
  \begin{center}
  \centerline{\includegraphics[width=1.0\columnwidth]{figures/face_cmp_visual_results_supp.pdf}}
  \vspace{-1.2cm}
  \end{center}
	\mycaption{A visual comparison of inference results}{(a)~Observed images. (b)~Inferred reflectance maps. \textit{GT} is the photometric stereo groundtruth, \textit{BU} is the Biswas \etal (2009) reflectance estimate and \textit{Forest} is the consensus prediction. (c)~The variance of the inferred reflectance estimate produced by \MTD (normalized across rows).(d)~Visualization of inferred light directions. (e)~Inferred normal maps.}
	\label{fig:shading-qualitative-multiple-subjects-supp}
\end{figure*}


\begin{figure*}[h]
	\centering
	\setlength\fboxsep{0.2mm}
	\setlength\fboxrule{0pt}
	\begin{tikzpicture}

		\matrix at (0, 0) [matrix of nodes, nodes={anchor=east}, column sep=-0.05cm, row sep=-0.2cm]
		{
			\fbox{\includegraphics[width=1cm]{figures/sample_3_4_X.png}} &
			\fbox{\includegraphics[width=1cm]{figures/sample_3_4_GT.png}} &
			\fbox{\includegraphics[width=1cm]{figures/sample_3_4_BISWAS.png}}  &
			\fbox{\includegraphics[width=1cm]{figures/sample_3_4_VMP.png}}  &
			\fbox{\includegraphics[width=1cm]{figures/sample_3_4_FOREST.png}}  &
			\fbox{\includegraphics[width=1cm]{figures/sample_3_4_CMP.png}}  &
			\fbox{\includegraphics[width=1cm]{figures/sample_3_4_CMPVAR.png}}
			 \\

			\fbox{\includegraphics[width=1cm]{figures/sample_3_5_X.png}} &
			\fbox{\includegraphics[width=1cm]{figures/sample_3_5_GT.png}} &
			\fbox{\includegraphics[width=1cm]{figures/sample_3_5_BISWAS.png}}  &
			\fbox{\includegraphics[width=1cm]{figures/sample_3_5_VMP.png}}  &
			\fbox{\includegraphics[width=1cm]{figures/sample_3_5_FOREST.png}}  &
			\fbox{\includegraphics[width=1cm]{figures/sample_3_5_CMP.png}}  &
			\fbox{\includegraphics[width=1cm]{figures/sample_3_5_CMPVAR.png}}
			 \\

			\fbox{\includegraphics[width=1cm]{figures/sample_3_6_X.png}} &
			\fbox{\includegraphics[width=1cm]{figures/sample_3_6_GT.png}} &
			\fbox{\includegraphics[width=1cm]{figures/sample_3_6_BISWAS.png}}  &
			\fbox{\includegraphics[width=1cm]{figures/sample_3_6_VMP.png}}  &
			\fbox{\includegraphics[width=1cm]{figures/sample_3_6_FOREST.png}}  &
			\fbox{\includegraphics[width=1cm]{figures/sample_3_6_CMP.png}}  &
			\fbox{\includegraphics[width=1cm]{figures/sample_3_6_CMPVAR.png}}
			 \\
	     };

       \node at (-3.85, -2.0) {\small Observed};
       \node at (-2.55, -2.0) {\small `GT'};
       \node at (-1.27, -2.0) {\small BU};
       \node at (0.0, -2.0) {\small MP};
       \node at (1.27, -2.0) {\small Forest};
       \node at (2.55, -2.0) {\small \textbf{CMP}};
       \node at (3.85, -2.0) {\small Variance};

	\end{tikzpicture}
	\mycaption{Robustness to varying illumination}{Reflectance estimation on a subject images with varying illumination. Left to right: observed image, photometric stereo estimate (GT)
  which is used as a proxy for groundtruth, bottom-up estimate of \cite{Biswas2009}, VMP result, consensus forest estimate, CMP mean, and CMP variance.}
	\label{fig:shading-qualitative-same-subject-supp}
\end{figure*}

\clearpage

\section{Additional Material for Learning Sparse High Dimensional Filters}
\label{sec:appendix-bnn}

This part of supplementary material contains a more detailed overview of the permutohedral
lattice convolution in Section~\ref{sec:permconv}, more experiments in
Section~\ref{sec:addexps} and additional results with protocols for
the experiments presented in Chapter~\ref{chap:bnn} in Section~\ref{sec:addresults}.

\vspace{-0.2cm}
\subsection{General Permutohedral Convolutions}
\label{sec:permconv}

A core technical contribution of this work is the generalization of the Gaussian permutohedral lattice
convolution proposed in~\cite{adams2010fast} to the full non-separable case with the
ability to perform back-propagation. Although, conceptually, there are minor
differences between Gaussian and general parameterized filters, there are non-trivial practical
differences in terms of the algorithmic implementation. The Gauss filters belong to
the separable class and can thus be decomposed into multiple
sequential one dimensional convolutions. We are interested in the general filter
convolutions, which can not be decomposed. Thus, performing a general permutohedral
convolution at a lattice point requires the computation of the inner product with the
neighboring elements in all the directions in the high-dimensional space.

Here, we give more details of the implementation differences of separable
and non-separable filters. In the following, we will explain the scalar case first.
Recall, that the forward pass of general permutohedral convolution
involves 3 steps: \textit{splatting}, \textit{convolving} and \textit{slicing}.
We follow the same splatting and slicing strategies as in~\cite{adams2010fast}
since these operations do not depend on the filter kernel. The main difference
between our work and the existing implementation of~\cite{adams2010fast} is
the way that the convolution operation is executed. This proceeds by constructing
a \emph{blur neighbor} matrix $K$ that stores for every lattice point all
values of the lattice neighbors that are needed to compute the filter output.

\begin{figure}[t!]
  \centering
    \includegraphics[width=0.6\columnwidth]{figures/supplementary/lattice_construction}
  \mycaption{Illustration of 1D permutohedral lattice construction}
  {A $4\times 4$ $(x,y)$ grid lattice is projected onto the plane defined by the normal
  vector $(1,1)^{\top}$. This grid has $s+1=4$ and $d=2$ $(s+1)^{d}=4^2=16$ elements.
  In the projection, all points of the same color are projected onto the same points in the plane.
  The number of elements of the projected lattice is $t=(s+1)^d-s^d=4^2-3^2=7$, that is
  the $(4\times 4)$ grid minus the size of lattice that is $1$ smaller at each size, in this
  case a $(3\times 3)$ lattice (the upper right $(3\times 3)$ elements).
  }
\label{fig:latticeconstruction}
\end{figure}

The blur neighbor matrix is constructed by traversing through all the populated
lattice points and their neighboring elements.
% For efficiency, we do this matrix construction recursively with shared computations
% since $n^{th}$ neighbourhood elements are $1^{st}$ neighborhood elements of $n-1^{th}$ neighbourhood elements. \pg{do not understand}
This is done recursively to share computations. For any lattice point, the neighbors that are
$n$ hops away are the direct neighbors of the points that are $n-1$ hops away.
The size of a $d$ dimensional spatial filter with width $s+1$ is $(s+1)^{d}$ (\eg, a
$3\times 3$ filter, $s=2$ in $d=2$ has $3^2=9$ elements) and this size grows
exponentially in the number of dimensions $d$. The permutohedral lattice is constructed by
projecting a regular grid onto the plane spanned by the $d$ dimensional normal vector ${(1,\ldots,1)}^{\top}$. See
Fig.~\ref{fig:latticeconstruction} for an illustration of the 1D lattice construction.
Many corners of a grid filter are projected onto the same point, in total $t = {(s+1)}^{d} -
s^{d}$ elements remain in the permutohedral filter with $s$ neighborhood in $d-1$ dimensions.
If the lattice has $m$ populated elements, the
matrix $K$ has size $t\times m$. Note that, since the input signal is typically
sparse, only a few lattice corners are being populated in the \textit{slicing} step.
We use a hash-table to keep track of these points and traverse only through
the populated lattice points for this neighborhood matrix construction.

Once the blur neighbor matrix $K$ is constructed, we can perform the convolution
by the matrix vector multiplication
\begin{equation}
\ell' = BK,
\label{eq:conv}
\end{equation}
where $B$ is the $1 \times t$ filter kernel (whose values we will learn) and $\ell'\in\mathbb{R}^{1\times m}$
is the result of the filtering at the $m$ lattice points. In practice, we found that the
matrix $K$ is sometimes too large to fit into GPU memory and we divided the matrix $K$
into smaller pieces to compute Eq.~\ref{eq:conv} sequentially.

In the general multi-dimensional case, the signal $\ell$ is of $c$ dimensions. Then
the kernel $B$ is of size $c \times t$ and $K$ stores the $c$ dimensional vectors
accordingly. When the input and output points are different, we slice only the
input points and splat only at the output points.


\subsection{Additional Experiments}
\label{sec:addexps}
In this section, we discuss more use-cases for the learned bilateral filters, one
use-case of BNNs and two single filter applications for image and 3D mesh denoising.

\subsubsection{Recognition of subsampled MNIST}\label{sec:app_mnist}

One of the strengths of the proposed filter convolution is that it does not
require the input to lie on a regular grid. The only requirement is to define a distance
between features of the input signal.
We highlight this feature with the following experiment using the
classical MNIST ten class classification problem~\cite{lecun1998mnist}. We sample a
sparse set of $N$ points $(x,y)\in [0,1]\times [0,1]$
uniformly at random in the input image, use their interpolated values
as signal and the \emph{continuous} $(x,y)$ positions as features. This mimics
sub-sampling of a high-dimensional signal. To compare against a spatial convolution,
we interpolate the sparse set of values at the grid positions.

We take a reference implementation of LeNet~\cite{lecun1998gradient} that
is part of the Caffe project~\cite{jia2014caffe} and compare it
against the same architecture but replacing the first convolutional
layer with a bilateral convolution layer (BCL). The filter size
and numbers are adjusted to get a comparable number of parameters
($5\times 5$ for LeNet, $2$-neighborhood for BCL).

The results are shown in Table~\ref{tab:all-results}. We see that training
on the original MNIST data (column Original, LeNet vs. BNN) leads to a slight
decrease in performance of the BNN (99.03\%) compared to LeNet
(99.19\%). The BNN can be trained and evaluated on sparse
signals, and we resample the image as described above for $N=$ 100\%, 60\% and
20\% of the total number of pixels. The methods are also evaluated
on test images that are subsampled in the same way. Note that we can
train and test with different subsampling rates. We introduce an additional
bilinear interpolation layer for the LeNet architecture to train on the same
data. In essence, both models perform a spatial interpolation and thus we
expect them to yield a similar classification accuracy. Once the data is of
higher dimensions, the permutohedral convolution will be faster due to hashing
the sparse input points, as well as less memory demanding in comparison to
naive application of a spatial convolution with interpolated values.

\begin{table}[t]
  \begin{center}
    \footnotesize
    \centering
    \begin{tabular}[t]{lllll}
      \toprule
              &     & \multicolumn{3}{c}{Test Subsampling} \\
       Method  & Original & 100\% & 60\% & 20\%\\
      \midrule
       LeNet &  \textbf{0.9919} & 0.9660 & 0.9348 & \textbf{0.6434} \\
       BNN &  0.9903 & \textbf{0.9844} & \textbf{0.9534} & 0.5767 \\
      \hline
       LeNet 100\% & 0.9856 & 0.9809 & 0.9678 & \textbf{0.7386} \\
       BNN 100\% & \textbf{0.9900} & \textbf{0.9863} & \textbf{0.9699} & 0.6910 \\
      \hline
       LeNet 60\% & 0.9848 & 0.9821 & 0.9740 & 0.8151 \\
       BNN 60\% & \textbf{0.9885} & \textbf{0.9864} & \textbf{0.9771} & \textbf{0.8214}\\
      \hline
       LeNet 20\% & \textbf{0.9763} & \textbf{0.9754} & 0.9695 & 0.8928 \\
       BNN 20\% & 0.9728 & 0.9735 & \textbf{0.9701} & \textbf{0.9042}\\
      \bottomrule
    \end{tabular}
  \end{center}
\vspace{-.2cm}
\caption{Classification accuracy on MNIST. We compare the
    LeNet~\cite{lecun1998gradient} implementation that is part of
    Caffe~\cite{jia2014caffe} to the network with the first layer
    replaced by a bilateral convolution layer (BCL). Both are trained
    on the original image resolution (first two rows). Three more BNN
    and CNN models are trained with randomly subsampled images (100\%,
    60\% and 20\% of the pixels). An additional bilinear interpolation
    layer samples the input signal on a spatial grid for the CNN model.
  }
  \label{tab:all-results}
\vspace{-.5cm}
\end{table}

\subsubsection{Image Denoising}

The main application that inspired the development of the bilateral
filtering operation is image denoising~\cite{aurich1995non}, there
using a single Gaussian kernel. Our development allows to learn this
kernel function from data and we explore how to improve using a \emph{single}
but more general bilateral filter.

We use the Berkeley segmentation dataset
(BSDS500)~\cite{arbelaezi2011bsds500} as a test bed. The color
images in the dataset are converted to gray-scale,
and corrupted with Gaussian noise with a standard deviation of
$\frac {25} {255}$.

We compare the performance of four different filter models on a
denoising task.
The first baseline model (`Spatial' in Table \ref{tab:denoising}, $25$
weights) uses a single spatial filter with a kernel size of
$5$ and predicts the scalar gray-scale value at the center pixel. The next model
(`Gauss Bilateral') applies a bilateral \emph{Gaussian}
filter to the noisy input, using position and intensity features $\f=(x,y,v)^\top$.
The third setup (`Learned Bilateral', $65$ weights)
takes a Gauss kernel as initialization and
fits all filter weights on the train set to minimize the
mean squared error with respect to the clean images.
We run a combination
of spatial and permutohedral convolutions on spatial and bilateral
features (`Spatial + Bilateral (Learned)') to check for a complementary
performance of the two convolutions.

\label{sec:exp:denoising}
\begin{table}[!h]
\begin{center}
  \footnotesize
  \begin{tabular}[t]{lr}
    \toprule
    Method & PSNR \\
    \midrule
    Noisy Input & $20.17$ \\
    Spatial & $26.27$ \\
    Gauss Bilateral & $26.51$ \\
    Learned Bilateral & $26.58$ \\
    Spatial + Bilateral (Learned) & \textbf{$26.65$} \\
    \bottomrule
  \end{tabular}
\end{center}
\vspace{-0.5em}
\caption{PSNR results of a denoising task using the BSDS500
  dataset~\cite{arbelaezi2011bsds500}}
\vspace{-0.5em}
\label{tab:denoising}
\end{table}
\vspace{-0.2em}

The PSNR scores evaluated on full images of the test set are
shown in Table \ref{tab:denoising}. We find that an untrained bilateral
filter already performs better than a trained spatial convolution
($26.27$ to $26.51$). A learned convolution further improve the
performance slightly. We chose this simple one-kernel setup to
validate an advantage of the generalized bilateral filter. A competitive
denoising system would employ RGB color information and also
needs to be properly adjusted in network size. Multi-layer perceptrons
have obtained state-of-the-art denoising results~\cite{burger12cvpr}
and the permutohedral lattice layer can readily be used in such an
architecture, which is intended future work.

\subsection{Additional results}
\label{sec:addresults}

This section contains more qualitative results for the experiments presented in Chapter~\ref{chap:bnn}.

\begin{figure*}[th!]
  \centering
    \includegraphics[width=\columnwidth,trim={5cm 2.5cm 5cm 4.5cm},clip]{figures/supplementary/lattice_viz.pdf}
    \vspace{-0.7cm}
  \mycaption{Visualization of the Permutohedral Lattice}
  {Sample lattice visualizations for different feature spaces. All pixels falling in the same simplex cell are shown with
  the same color. $(x,y)$ features correspond to image pixel positions, and $(r,g,b) \in [0,255]$ correspond
  to the red, green and blue color values.}
\label{fig:latticeviz}
\end{figure*}

\subsubsection{Lattice Visualization}

Figure~\ref{fig:latticeviz} shows sample lattice visualizations for different feature spaces.

\newcolumntype{L}[1]{>{\raggedright\let\newline\\\arraybackslash\hspace{0pt}}b{#1}}
\newcolumntype{C}[1]{>{\centering\let\newline\\\arraybackslash\hspace{0pt}}b{#1}}
\newcolumntype{R}[1]{>{\raggedleft\let\newline\\\arraybackslash\hspace{0pt}}b{#1}}

\subsubsection{Color Upsampling}\label{sec:color_upsampling}
\label{sec:col_upsample_extra}

Some images of the upsampling for the Pascal
VOC12 dataset are shown in Fig.~\ref{fig:Colour_upsample_visuals}. It is
especially the low level image details that are better preserved with
a learned bilateral filter compared to the Gaussian case.

\begin{figure*}[t!]
  \centering
    \subfigure{%
   \raisebox{2.0em}{
    \includegraphics[width=.06\columnwidth]{figures/supplementary/2007_004969.jpg}
   }
  }
  \subfigure{%
    \includegraphics[width=.17\columnwidth]{figures/supplementary/2007_004969_gray.pdf}
  }
  \subfigure{%
    \includegraphics[width=.17\columnwidth]{figures/supplementary/2007_004969_gt.pdf}
  }
  \subfigure{%
    \includegraphics[width=.17\columnwidth]{figures/supplementary/2007_004969_bicubic.pdf}
  }
  \subfigure{%
    \includegraphics[width=.17\columnwidth]{figures/supplementary/2007_004969_gauss.pdf}
  }
  \subfigure{%
    \includegraphics[width=.17\columnwidth]{figures/supplementary/2007_004969_learnt.pdf}
  }\\
    \subfigure{%
   \raisebox{2.0em}{
    \includegraphics[width=.06\columnwidth]{figures/supplementary/2007_003106.jpg}
   }
  }
  \subfigure{%
    \includegraphics[width=.17\columnwidth]{figures/supplementary/2007_003106_gray.pdf}
  }
  \subfigure{%
    \includegraphics[width=.17\columnwidth]{figures/supplementary/2007_003106_gt.pdf}
  }
  \subfigure{%
    \includegraphics[width=.17\columnwidth]{figures/supplementary/2007_003106_bicubic.pdf}
  }
  \subfigure{%
    \includegraphics[width=.17\columnwidth]{figures/supplementary/2007_003106_gauss.pdf}
  }
  \subfigure{%
    \includegraphics[width=.17\columnwidth]{figures/supplementary/2007_003106_learnt.pdf}
  }\\
  \setcounter{subfigure}{0}
  \small{
  \subfigure[Inp.]{%
  \raisebox{2.0em}{
    \includegraphics[width=.06\columnwidth]{figures/supplementary/2007_006837.jpg}
   }
  }
  \subfigure[Guidance]{%
    \includegraphics[width=.17\columnwidth]{figures/supplementary/2007_006837_gray.pdf}
  }
   \subfigure[GT]{%
    \includegraphics[width=.17\columnwidth]{figures/supplementary/2007_006837_gt.pdf}
  }
  \subfigure[Bicubic]{%
    \includegraphics[width=.17\columnwidth]{figures/supplementary/2007_006837_bicubic.pdf}
  }
  \subfigure[Gauss-BF]{%
    \includegraphics[width=.17\columnwidth]{figures/supplementary/2007_006837_gauss.pdf}
  }
  \subfigure[Learned-BF]{%
    \includegraphics[width=.17\columnwidth]{figures/supplementary/2007_006837_learnt.pdf}
  }
  }
  \vspace{-0.5cm}
  \mycaption{Color Upsampling}{Color $8\times$ upsampling results
  using different methods, from left to right, (a)~Low-resolution input color image (Inp.),
  (b)~Gray scale guidance image, (c)~Ground-truth color image; Upsampled color images with
  (d)~Bicubic interpolation, (e) Gauss bilateral upsampling and, (f)~Learned bilateral
  updampgling (best viewed on screen).}

\label{fig:Colour_upsample_visuals}
\end{figure*}

\subsubsection{Depth Upsampling}
\label{sec:depth_upsample_extra}

Figure~\ref{fig:depth_upsample_visuals} presents some more qualitative results comparing bicubic interpolation, Gauss
bilateral and learned bilateral upsampling on NYU depth dataset image~\cite{silberman2012indoor}.

\subsubsection{Character Recognition}\label{sec:app_character}

 Figure~\ref{fig:nnrecognition} shows the schematic of different layers
 of the network architecture for LeNet-7~\cite{lecun1998mnist}
 and DeepCNet(5, 50)~\cite{ciresan2012multi,graham2014spatially}. For the BNN variants, the first layer filters are replaced
 with learned bilateral filters and are learned end-to-end.

\subsubsection{Semantic Segmentation}\label{sec:app_semantic_segmentation}
\label{sec:semantic_bnn_extra}

Some more visual results for semantic segmentation are shown in Figure~\ref{fig:semantic_visuals}.
These include the underlying DeepLab CNN\cite{chen2014semantic} result (DeepLab),
the 2 step mean-field result with Gaussian edge potentials (+2stepMF-GaussCRF)
and also corresponding results with learned edge potentials (+2stepMF-LearnedCRF).
In general, we observe that mean-field in learned CRF leads to slightly dilated
classification regions in comparison to using Gaussian CRF thereby filling-in the
false negative pixels and also correcting some mis-classified regions.

\begin{figure*}[t!]
  \centering
    \subfigure{%
   \raisebox{2.0em}{
    \includegraphics[width=.06\columnwidth]{figures/supplementary/2bicubic}
   }
  }
  \subfigure{%
    \includegraphics[width=.17\columnwidth]{figures/supplementary/2given_image}
  }
  \subfigure{%
    \includegraphics[width=.17\columnwidth]{figures/supplementary/2ground_truth}
  }
  \subfigure{%
    \includegraphics[width=.17\columnwidth]{figures/supplementary/2bicubic}
  }
  \subfigure{%
    \includegraphics[width=.17\columnwidth]{figures/supplementary/2gauss}
  }
  \subfigure{%
    \includegraphics[width=.17\columnwidth]{figures/supplementary/2learnt}
  }\\
    \subfigure{%
   \raisebox{2.0em}{
    \includegraphics[width=.06\columnwidth]{figures/supplementary/32bicubic}
   }
  }
  \subfigure{%
    \includegraphics[width=.17\columnwidth]{figures/supplementary/32given_image}
  }
  \subfigure{%
    \includegraphics[width=.17\columnwidth]{figures/supplementary/32ground_truth}
  }
  \subfigure{%
    \includegraphics[width=.17\columnwidth]{figures/supplementary/32bicubic}
  }
  \subfigure{%
    \includegraphics[width=.17\columnwidth]{figures/supplementary/32gauss}
  }
  \subfigure{%
    \includegraphics[width=.17\columnwidth]{figures/supplementary/32learnt}
  }\\
  \setcounter{subfigure}{0}
  \small{
  \subfigure[Inp.]{%
  \raisebox{2.0em}{
    \includegraphics[width=.06\columnwidth]{figures/supplementary/41bicubic}
   }
  }
  \subfigure[Guidance]{%
    \includegraphics[width=.17\columnwidth]{figures/supplementary/41given_image}
  }
   \subfigure[GT]{%
    \includegraphics[width=.17\columnwidth]{figures/supplementary/41ground_truth}
  }
  \subfigure[Bicubic]{%
    \includegraphics[width=.17\columnwidth]{figures/supplementary/41bicubic}
  }
  \subfigure[Gauss-BF]{%
    \includegraphics[width=.17\columnwidth]{figures/supplementary/41gauss}
  }
  \subfigure[Learned-BF]{%
    \includegraphics[width=.17\columnwidth]{figures/supplementary/41learnt}
  }
  }
  \mycaption{Depth Upsampling}{Depth $8\times$ upsampling results
  using different upsampling strategies, from left to right,
  (a)~Low-resolution input depth image (Inp.),
  (b)~High-resolution guidance image, (c)~Ground-truth depth; Upsampled depth images with
  (d)~Bicubic interpolation, (e) Gauss bilateral upsampling and, (f)~Learned bilateral
  updampgling (best viewed on screen).}

\label{fig:depth_upsample_visuals}
\end{figure*}

\subsubsection{Material Segmentation}\label{sec:app_material_segmentation}
\label{sec:material_bnn_extra}

In Fig.~\ref{fig:material_visuals-app2}, we present visual results comparing 2 step
mean-field inference with Gaussian and learned pairwise CRF potentials. In
general, we observe that the pixels belonging to dominant classes in the
training data are being more accurately classified with learned CRF. This leads to
a significant improvements in overall pixel accuracy. This also results
in a slight decrease of the accuracy from less frequent class pixels thereby
slightly reducing the average class accuracy with learning. We attribute this
to the type of annotation that is available for this dataset, which is not
for the entire image but for some segments in the image. We have very few
images of the infrequent classes to combat this behaviour during training.

\subsubsection{Experiment Protocols}
\label{sec:protocols}

Table~\ref{tbl:parameters} shows experiment protocols of different experiments.

 \begin{figure*}[t!]
  \centering
  \subfigure[LeNet-7]{
    \includegraphics[width=0.7\columnwidth]{figures/supplementary/lenet_cnn_network}
    }\\
    \subfigure[DeepCNet]{
    \includegraphics[width=\columnwidth]{figures/supplementary/deepcnet_cnn_network}
    }
  \mycaption{CNNs for Character Recognition}
  {Schematic of (top) LeNet-7~\cite{lecun1998mnist} and (bottom) DeepCNet(5,50)~\cite{ciresan2012multi,graham2014spatially} architectures used in Assamese
  character recognition experiments.}
\label{fig:nnrecognition}
\end{figure*}

\definecolor{voc_1}{RGB}{0, 0, 0}
\definecolor{voc_2}{RGB}{128, 0, 0}
\definecolor{voc_3}{RGB}{0, 128, 0}
\definecolor{voc_4}{RGB}{128, 128, 0}
\definecolor{voc_5}{RGB}{0, 0, 128}
\definecolor{voc_6}{RGB}{128, 0, 128}
\definecolor{voc_7}{RGB}{0, 128, 128}
\definecolor{voc_8}{RGB}{128, 128, 128}
\definecolor{voc_9}{RGB}{64, 0, 0}
\definecolor{voc_10}{RGB}{192, 0, 0}
\definecolor{voc_11}{RGB}{64, 128, 0}
\definecolor{voc_12}{RGB}{192, 128, 0}
\definecolor{voc_13}{RGB}{64, 0, 128}
\definecolor{voc_14}{RGB}{192, 0, 128}
\definecolor{voc_15}{RGB}{64, 128, 128}
\definecolor{voc_16}{RGB}{192, 128, 128}
\definecolor{voc_17}{RGB}{0, 64, 0}
\definecolor{voc_18}{RGB}{128, 64, 0}
\definecolor{voc_19}{RGB}{0, 192, 0}
\definecolor{voc_20}{RGB}{128, 192, 0}
\definecolor{voc_21}{RGB}{0, 64, 128}
\definecolor{voc_22}{RGB}{128, 64, 128}

\begin{figure*}[t]
  \centering
  \small{
  \fcolorbox{white}{voc_1}{\rule{0pt}{6pt}\rule{6pt}{0pt}} Background~~
  \fcolorbox{white}{voc_2}{\rule{0pt}{6pt}\rule{6pt}{0pt}} Aeroplane~~
  \fcolorbox{white}{voc_3}{\rule{0pt}{6pt}\rule{6pt}{0pt}} Bicycle~~
  \fcolorbox{white}{voc_4}{\rule{0pt}{6pt}\rule{6pt}{0pt}} Bird~~
  \fcolorbox{white}{voc_5}{\rule{0pt}{6pt}\rule{6pt}{0pt}} Boat~~
  \fcolorbox{white}{voc_6}{\rule{0pt}{6pt}\rule{6pt}{0pt}} Bottle~~
  \fcolorbox{white}{voc_7}{\rule{0pt}{6pt}\rule{6pt}{0pt}} Bus~~
  \fcolorbox{white}{voc_8}{\rule{0pt}{6pt}\rule{6pt}{0pt}} Car~~ \\
  \fcolorbox{white}{voc_9}{\rule{0pt}{6pt}\rule{6pt}{0pt}} Cat~~
  \fcolorbox{white}{voc_10}{\rule{0pt}{6pt}\rule{6pt}{0pt}} Chair~~
  \fcolorbox{white}{voc_11}{\rule{0pt}{6pt}\rule{6pt}{0pt}} Cow~~
  \fcolorbox{white}{voc_12}{\rule{0pt}{6pt}\rule{6pt}{0pt}} Dining Table~~
  \fcolorbox{white}{voc_13}{\rule{0pt}{6pt}\rule{6pt}{0pt}} Dog~~
  \fcolorbox{white}{voc_14}{\rule{0pt}{6pt}\rule{6pt}{0pt}} Horse~~
  \fcolorbox{white}{voc_15}{\rule{0pt}{6pt}\rule{6pt}{0pt}} Motorbike~~
  \fcolorbox{white}{voc_16}{\rule{0pt}{6pt}\rule{6pt}{0pt}} Person~~ \\
  \fcolorbox{white}{voc_17}{\rule{0pt}{6pt}\rule{6pt}{0pt}} Potted Plant~~
  \fcolorbox{white}{voc_18}{\rule{0pt}{6pt}\rule{6pt}{0pt}} Sheep~~
  \fcolorbox{white}{voc_19}{\rule{0pt}{6pt}\rule{6pt}{0pt}} Sofa~~
  \fcolorbox{white}{voc_20}{\rule{0pt}{6pt}\rule{6pt}{0pt}} Train~~
  \fcolorbox{white}{voc_21}{\rule{0pt}{6pt}\rule{6pt}{0pt}} TV monitor~~ \\
  }
  \subfigure{%
    \includegraphics[width=.18\columnwidth]{figures/supplementary/2007_001423_given.jpg}
  }
  \subfigure{%
    \includegraphics[width=.18\columnwidth]{figures/supplementary/2007_001423_gt.png}
  }
  \subfigure{%
    \includegraphics[width=.18\columnwidth]{figures/supplementary/2007_001423_cnn.png}
  }
  \subfigure{%
    \includegraphics[width=.18\columnwidth]{figures/supplementary/2007_001423_gauss.png}
  }
  \subfigure{%
    \includegraphics[width=.18\columnwidth]{figures/supplementary/2007_001423_learnt.png}
  }\\
  \subfigure{%
    \includegraphics[width=.18\columnwidth]{figures/supplementary/2007_001430_given.jpg}
  }
  \subfigure{%
    \includegraphics[width=.18\columnwidth]{figures/supplementary/2007_001430_gt.png}
  }
  \subfigure{%
    \includegraphics[width=.18\columnwidth]{figures/supplementary/2007_001430_cnn.png}
  }
  \subfigure{%
    \includegraphics[width=.18\columnwidth]{figures/supplementary/2007_001430_gauss.png}
  }
  \subfigure{%
    \includegraphics[width=.18\columnwidth]{figures/supplementary/2007_001430_learnt.png}
  }\\
    \subfigure{%
    \includegraphics[width=.18\columnwidth]{figures/supplementary/2007_007996_given.jpg}
  }
  \subfigure{%
    \includegraphics[width=.18\columnwidth]{figures/supplementary/2007_007996_gt.png}
  }
  \subfigure{%
    \includegraphics[width=.18\columnwidth]{figures/supplementary/2007_007996_cnn.png}
  }
  \subfigure{%
    \includegraphics[width=.18\columnwidth]{figures/supplementary/2007_007996_gauss.png}
  }
  \subfigure{%
    \includegraphics[width=.18\columnwidth]{figures/supplementary/2007_007996_learnt.png}
  }\\
   \subfigure{%
    \includegraphics[width=.18\columnwidth]{figures/supplementary/2010_002682_given.jpg}
  }
  \subfigure{%
    \includegraphics[width=.18\columnwidth]{figures/supplementary/2010_002682_gt.png}
  }
  \subfigure{%
    \includegraphics[width=.18\columnwidth]{figures/supplementary/2010_002682_cnn.png}
  }
  \subfigure{%
    \includegraphics[width=.18\columnwidth]{figures/supplementary/2010_002682_gauss.png}
  }
  \subfigure{%
    \includegraphics[width=.18\columnwidth]{figures/supplementary/2010_002682_learnt.png}
  }\\
     \subfigure{%
    \includegraphics[width=.18\columnwidth]{figures/supplementary/2010_004789_given.jpg}
  }
  \subfigure{%
    \includegraphics[width=.18\columnwidth]{figures/supplementary/2010_004789_gt.png}
  }
  \subfigure{%
    \includegraphics[width=.18\columnwidth]{figures/supplementary/2010_004789_cnn.png}
  }
  \subfigure{%
    \includegraphics[width=.18\columnwidth]{figures/supplementary/2010_004789_gauss.png}
  }
  \subfigure{%
    \includegraphics[width=.18\columnwidth]{figures/supplementary/2010_004789_learnt.png}
  }\\
       \subfigure{%
    \includegraphics[width=.18\columnwidth]{figures/supplementary/2007_001311_given.jpg}
  }
  \subfigure{%
    \includegraphics[width=.18\columnwidth]{figures/supplementary/2007_001311_gt.png}
  }
  \subfigure{%
    \includegraphics[width=.18\columnwidth]{figures/supplementary/2007_001311_cnn.png}
  }
  \subfigure{%
    \includegraphics[width=.18\columnwidth]{figures/supplementary/2007_001311_gauss.png}
  }
  \subfigure{%
    \includegraphics[width=.18\columnwidth]{figures/supplementary/2007_001311_learnt.png}
  }\\
  \setcounter{subfigure}{0}
  \subfigure[Input]{%
    \includegraphics[width=.18\columnwidth]{figures/supplementary/2010_003531_given.jpg}
  }
  \subfigure[Ground Truth]{%
    \includegraphics[width=.18\columnwidth]{figures/supplementary/2010_003531_gt.png}
  }
  \subfigure[DeepLab]{%
    \includegraphics[width=.18\columnwidth]{figures/supplementary/2010_003531_cnn.png}
  }
  \subfigure[+GaussCRF]{%
    \includegraphics[width=.18\columnwidth]{figures/supplementary/2010_003531_gauss.png}
  }
  \subfigure[+LearnedCRF]{%
    \includegraphics[width=.18\columnwidth]{figures/supplementary/2010_003531_learnt.png}
  }
  \vspace{-0.3cm}
  \mycaption{Semantic Segmentation}{Example results of semantic segmentation.
  (c)~depicts the unary results before application of MF, (d)~after two steps of MF with Gaussian edge CRF potentials, (e)~after
  two steps of MF with learned edge CRF potentials.}
    \label{fig:semantic_visuals}
\end{figure*}


\definecolor{minc_1}{HTML}{771111}
\definecolor{minc_2}{HTML}{CAC690}
\definecolor{minc_3}{HTML}{EEEEEE}
\definecolor{minc_4}{HTML}{7C8FA6}
\definecolor{minc_5}{HTML}{597D31}
\definecolor{minc_6}{HTML}{104410}
\definecolor{minc_7}{HTML}{BB819C}
\definecolor{minc_8}{HTML}{D0CE48}
\definecolor{minc_9}{HTML}{622745}
\definecolor{minc_10}{HTML}{666666}
\definecolor{minc_11}{HTML}{D54A31}
\definecolor{minc_12}{HTML}{101044}
\definecolor{minc_13}{HTML}{444126}
\definecolor{minc_14}{HTML}{75D646}
\definecolor{minc_15}{HTML}{DD4348}
\definecolor{minc_16}{HTML}{5C8577}
\definecolor{minc_17}{HTML}{C78472}
\definecolor{minc_18}{HTML}{75D6D0}
\definecolor{minc_19}{HTML}{5B4586}
\definecolor{minc_20}{HTML}{C04393}
\definecolor{minc_21}{HTML}{D69948}
\definecolor{minc_22}{HTML}{7370D8}
\definecolor{minc_23}{HTML}{7A3622}
\definecolor{minc_24}{HTML}{000000}

\begin{figure*}[t]
  \centering
  \small{
  \fcolorbox{white}{minc_1}{\rule{0pt}{6pt}\rule{6pt}{0pt}} Brick~~
  \fcolorbox{white}{minc_2}{\rule{0pt}{6pt}\rule{6pt}{0pt}} Carpet~~
  \fcolorbox{white}{minc_3}{\rule{0pt}{6pt}\rule{6pt}{0pt}} Ceramic~~
  \fcolorbox{white}{minc_4}{\rule{0pt}{6pt}\rule{6pt}{0pt}} Fabric~~
  \fcolorbox{white}{minc_5}{\rule{0pt}{6pt}\rule{6pt}{0pt}} Foliage~~
  \fcolorbox{white}{minc_6}{\rule{0pt}{6pt}\rule{6pt}{0pt}} Food~~
  \fcolorbox{white}{minc_7}{\rule{0pt}{6pt}\rule{6pt}{0pt}} Glass~~
  \fcolorbox{white}{minc_8}{\rule{0pt}{6pt}\rule{6pt}{0pt}} Hair~~ \\
  \fcolorbox{white}{minc_9}{\rule{0pt}{6pt}\rule{6pt}{0pt}} Leather~~
  \fcolorbox{white}{minc_10}{\rule{0pt}{6pt}\rule{6pt}{0pt}} Metal~~
  \fcolorbox{white}{minc_11}{\rule{0pt}{6pt}\rule{6pt}{0pt}} Mirror~~
  \fcolorbox{white}{minc_12}{\rule{0pt}{6pt}\rule{6pt}{0pt}} Other~~
  \fcolorbox{white}{minc_13}{\rule{0pt}{6pt}\rule{6pt}{0pt}} Painted~~
  \fcolorbox{white}{minc_14}{\rule{0pt}{6pt}\rule{6pt}{0pt}} Paper~~
  \fcolorbox{white}{minc_15}{\rule{0pt}{6pt}\rule{6pt}{0pt}} Plastic~~\\
  \fcolorbox{white}{minc_16}{\rule{0pt}{6pt}\rule{6pt}{0pt}} Polished Stone~~
  \fcolorbox{white}{minc_17}{\rule{0pt}{6pt}\rule{6pt}{0pt}} Skin~~
  \fcolorbox{white}{minc_18}{\rule{0pt}{6pt}\rule{6pt}{0pt}} Sky~~
  \fcolorbox{white}{minc_19}{\rule{0pt}{6pt}\rule{6pt}{0pt}} Stone~~
  \fcolorbox{white}{minc_20}{\rule{0pt}{6pt}\rule{6pt}{0pt}} Tile~~
  \fcolorbox{white}{minc_21}{\rule{0pt}{6pt}\rule{6pt}{0pt}} Wallpaper~~
  \fcolorbox{white}{minc_22}{\rule{0pt}{6pt}\rule{6pt}{0pt}} Water~~
  \fcolorbox{white}{minc_23}{\rule{0pt}{6pt}\rule{6pt}{0pt}} Wood~~ \\
  }
  \subfigure{%
    \includegraphics[width=.18\columnwidth]{figures/supplementary/000010868_given.jpg}
  }
  \subfigure{%
    \includegraphics[width=.18\columnwidth]{figures/supplementary/000010868_gt.png}
  }
  \subfigure{%
    \includegraphics[width=.18\columnwidth]{figures/supplementary/000010868_cnn.png}
  }
  \subfigure{%
    \includegraphics[width=.18\columnwidth]{figures/supplementary/000010868_gauss.png}
  }
  \subfigure{%
    \includegraphics[width=.18\columnwidth]{figures/supplementary/000010868_learnt.png}
  }\\[-2ex]
  \subfigure{%
    \includegraphics[width=.18\columnwidth]{figures/supplementary/000006011_given.jpg}
  }
  \subfigure{%
    \includegraphics[width=.18\columnwidth]{figures/supplementary/000006011_gt.png}
  }
  \subfigure{%
    \includegraphics[width=.18\columnwidth]{figures/supplementary/000006011_cnn.png}
  }
  \subfigure{%
    \includegraphics[width=.18\columnwidth]{figures/supplementary/000006011_gauss.png}
  }
  \subfigure{%
    \includegraphics[width=.18\columnwidth]{figures/supplementary/000006011_learnt.png}
  }\\[-2ex]
    \subfigure{%
    \includegraphics[width=.18\columnwidth]{figures/supplementary/000008553_given.jpg}
  }
  \subfigure{%
    \includegraphics[width=.18\columnwidth]{figures/supplementary/000008553_gt.png}
  }
  \subfigure{%
    \includegraphics[width=.18\columnwidth]{figures/supplementary/000008553_cnn.png}
  }
  \subfigure{%
    \includegraphics[width=.18\columnwidth]{figures/supplementary/000008553_gauss.png}
  }
  \subfigure{%
    \includegraphics[width=.18\columnwidth]{figures/supplementary/000008553_learnt.png}
  }\\[-2ex]
   \subfigure{%
    \includegraphics[width=.18\columnwidth]{figures/supplementary/000009188_given.jpg}
  }
  \subfigure{%
    \includegraphics[width=.18\columnwidth]{figures/supplementary/000009188_gt.png}
  }
  \subfigure{%
    \includegraphics[width=.18\columnwidth]{figures/supplementary/000009188_cnn.png}
  }
  \subfigure{%
    \includegraphics[width=.18\columnwidth]{figures/supplementary/000009188_gauss.png}
  }
  \subfigure{%
    \includegraphics[width=.18\columnwidth]{figures/supplementary/000009188_learnt.png}
  }\\[-2ex]
  \setcounter{subfigure}{0}
  \subfigure[Input]{%
    \includegraphics[width=.18\columnwidth]{figures/supplementary/000023570_given.jpg}
  }
  \subfigure[Ground Truth]{%
    \includegraphics[width=.18\columnwidth]{figures/supplementary/000023570_gt.png}
  }
  \subfigure[DeepLab]{%
    \includegraphics[width=.18\columnwidth]{figures/supplementary/000023570_cnn.png}
  }
  \subfigure[+GaussCRF]{%
    \includegraphics[width=.18\columnwidth]{figures/supplementary/000023570_gauss.png}
  }
  \subfigure[+LearnedCRF]{%
    \includegraphics[width=.18\columnwidth]{figures/supplementary/000023570_learnt.png}
  }
  \mycaption{Material Segmentation}{Example results of material segmentation.
  (c)~depicts the unary results before application of MF, (d)~after two steps of MF with Gaussian edge CRF potentials, (e)~after two steps of MF with learned edge CRF potentials.}
    \label{fig:material_visuals-app2}
\end{figure*}


\begin{table*}[h]
\tiny
  \centering
    \begin{tabular}{L{2.3cm} L{2.25cm} C{1.5cm} C{0.7cm} C{0.6cm} C{0.7cm} C{0.7cm} C{0.7cm} C{1.6cm} C{0.6cm} C{0.6cm} C{0.6cm}}
      \toprule
& & & & & \multicolumn{3}{c}{\textbf{Data Statistics}} & \multicolumn{4}{c}{\textbf{Training Protocol}} \\

\textbf{Experiment} & \textbf{Feature Types} & \textbf{Feature Scales} & \textbf{Filter Size} & \textbf{Filter Nbr.} & \textbf{Train}  & \textbf{Val.} & \textbf{Test} & \textbf{Loss Type} & \textbf{LR} & \textbf{Batch} & \textbf{Epochs} \\
      \midrule
      \multicolumn{2}{c}{\textbf{Single Bilateral Filter Applications}} & & & & & & & & & \\
      \textbf{2$\times$ Color Upsampling} & Position$_{1}$, Intensity (3D) & 0.13, 0.17 & 65 & 2 & 10581 & 1449 & 1456 & MSE & 1e-06 & 200 & 94.5\\
      \textbf{4$\times$ Color Upsampling} & Position$_{1}$, Intensity (3D) & 0.06, 0.17 & 65 & 2 & 10581 & 1449 & 1456 & MSE & 1e-06 & 200 & 94.5\\
      \textbf{8$\times$ Color Upsampling} & Position$_{1}$, Intensity (3D) & 0.03, 0.17 & 65 & 2 & 10581 & 1449 & 1456 & MSE & 1e-06 & 200 & 94.5\\
      \textbf{16$\times$ Color Upsampling} & Position$_{1}$, Intensity (3D) & 0.02, 0.17 & 65 & 2 & 10581 & 1449 & 1456 & MSE & 1e-06 & 200 & 94.5\\
      \textbf{Depth Upsampling} & Position$_{1}$, Color (5D) & 0.05, 0.02 & 665 & 2 & 795 & 100 & 654 & MSE & 1e-07 & 50 & 251.6\\
      \textbf{Mesh Denoising} & Isomap (4D) & 46.00 & 63 & 2 & 1000 & 200 & 500 & MSE & 100 & 10 & 100.0 \\
      \midrule
      \multicolumn{2}{c}{\textbf{DenseCRF Applications}} & & & & & & & & &\\
      \multicolumn{2}{l}{\textbf{Semantic Segmentation}} & & & & & & & & &\\
      \textbf{- 1step MF} & Position$_{1}$, Color (5D); Position$_{1}$ (2D) & 0.01, 0.34; 0.34  & 665; 19  & 2; 2 & 10581 & 1449 & 1456 & Logistic & 0.1 & 5 & 1.4 \\
      \textbf{- 2step MF} & Position$_{1}$, Color (5D); Position$_{1}$ (2D) & 0.01, 0.34; 0.34 & 665; 19 & 2; 2 & 10581 & 1449 & 1456 & Logistic & 0.1 & 5 & 1.4 \\
      \textbf{- \textit{loose} 2step MF} & Position$_{1}$, Color (5D); Position$_{1}$ (2D) & 0.01, 0.34; 0.34 & 665; 19 & 2; 2 &10581 & 1449 & 1456 & Logistic & 0.1 & 5 & +1.9  \\ \\
      \multicolumn{2}{l}{\textbf{Material Segmentation}} & & & & & & & & &\\
      \textbf{- 1step MF} & Position$_{2}$, Lab-Color (5D) & 5.00, 0.05, 0.30  & 665 & 2 & 928 & 150 & 1798 & Weighted Logistic & 1e-04 & 24 & 2.6 \\
      \textbf{- 2step MF} & Position$_{2}$, Lab-Color (5D) & 5.00, 0.05, 0.30 & 665 & 2 & 928 & 150 & 1798 & Weighted Logistic & 1e-04 & 12 & +0.7 \\
      \textbf{- \textit{loose} 2step MF} & Position$_{2}$, Lab-Color (5D) & 5.00, 0.05, 0.30 & 665 & 2 & 928 & 150 & 1798 & Weighted Logistic & 1e-04 & 12 & +0.2\\
      \midrule
      \multicolumn{2}{c}{\textbf{Neural Network Applications}} & & & & & & & & &\\
      \textbf{Tiles: CNN-9$\times$9} & - & - & 81 & 4 & 10000 & 1000 & 1000 & Logistic & 0.01 & 100 & 500.0 \\
      \textbf{Tiles: CNN-13$\times$13} & - & - & 169 & 6 & 10000 & 1000 & 1000 & Logistic & 0.01 & 100 & 500.0 \\
      \textbf{Tiles: CNN-17$\times$17} & - & - & 289 & 8 & 10000 & 1000 & 1000 & Logistic & 0.01 & 100 & 500.0 \\
      \textbf{Tiles: CNN-21$\times$21} & - & - & 441 & 10 & 10000 & 1000 & 1000 & Logistic & 0.01 & 100 & 500.0 \\
      \textbf{Tiles: BNN} & Position$_{1}$, Color (5D) & 0.05, 0.04 & 63 & 1 & 10000 & 1000 & 1000 & Logistic & 0.01 & 100 & 30.0 \\
      \textbf{LeNet} & - & - & 25 & 2 & 5490 & 1098 & 1647 & Logistic & 0.1 & 100 & 182.2 \\
      \textbf{Crop-LeNet} & - & - & 25 & 2 & 5490 & 1098 & 1647 & Logistic & 0.1 & 100 & 182.2 \\
      \textbf{BNN-LeNet} & Position$_{2}$ (2D) & 20.00 & 7 & 1 & 5490 & 1098 & 1647 & Logistic & 0.1 & 100 & 182.2 \\
      \textbf{DeepCNet} & - & - & 9 & 1 & 5490 & 1098 & 1647 & Logistic & 0.1 & 100 & 182.2 \\
      \textbf{Crop-DeepCNet} & - & - & 9 & 1 & 5490 & 1098 & 1647 & Logistic & 0.1 & 100 & 182.2 \\
      \textbf{BNN-DeepCNet} & Position$_{2}$ (2D) & 40.00  & 7 & 1 & 5490 & 1098 & 1647 & Logistic & 0.1 & 100 & 182.2 \\
      \bottomrule
      \\
    \end{tabular}
    \mycaption{Experiment Protocols} {Experiment protocols for the different experiments presented in this work. \textbf{Feature Types}:
    Feature spaces used for the bilateral convolutions. Position$_1$ corresponds to un-normalized pixel positions whereas Position$_2$ corresponds
    to pixel positions normalized to $[0,1]$ with respect to the given image. \textbf{Feature Scales}: Cross-validated scales for the features used.
     \textbf{Filter Size}: Number of elements in the filter that is being learned. \textbf{Filter Nbr.}: Half-width of the filter. \textbf{Train},
     \textbf{Val.} and \textbf{Test} corresponds to the number of train, validation and test images used in the experiment. \textbf{Loss Type}: Type
     of loss used for back-propagation. ``MSE'' corresponds to Euclidean mean squared error loss and ``Logistic'' corresponds to multinomial logistic
     loss. ``Weighted Logistic'' is the class-weighted multinomial logistic loss. We weighted the loss with inverse class probability for material
     segmentation task due to the small availability of training data with class imbalance. \textbf{LR}: Fixed learning rate used in stochastic gradient
     descent. \textbf{Batch}: Number of images used in one parameter update step. \textbf{Epochs}: Number of training epochs. In all the experiments,
     we used fixed momentum of 0.9 and weight decay of 0.0005 for stochastic gradient descent. ```Color Upsampling'' experiments in this Table corresponds
     to those performed on Pascal VOC12 dataset images. For all experiments using Pascal VOC12 images, we use extended
     training segmentation dataset available from~\cite{hariharan2011moredata}, and used standard validation and test splits
     from the main dataset~\cite{voc2012segmentation}.}
  \label{tbl:parameters}
\end{table*}

\clearpage

\section{Parameters and Additional Results for Video Propagation Networks}

In this Section, we present experiment protocols and additional qualitative results for experiments
on video object segmentation, semantic video segmentation and video color
propagation. Table~\ref{tbl:parameters_supp} shows the feature scales and other parameters used in different experiments.
Figures~\ref{fig:video_seg_pos_supp} show some qualitative results on video object segmentation
with some failure cases in Fig.~\ref{fig:video_seg_neg_supp}.
Figure~\ref{fig:semantic_visuals_supp} shows some qualitative results on semantic video segmentation and
Fig.~\ref{fig:color_visuals_supp} shows results on video color propagation.

\newcolumntype{L}[1]{>{\raggedright\let\newline\\\arraybackslash\hspace{0pt}}b{#1}}
\newcolumntype{C}[1]{>{\centering\let\newline\\\arraybackslash\hspace{0pt}}b{#1}}
\newcolumntype{R}[1]{>{\raggedleft\let\newline\\\arraybackslash\hspace{0pt}}b{#1}}

\begin{table*}[h]
\tiny
  \centering
    \begin{tabular}{L{3.0cm} L{2.4cm} L{2.8cm} L{2.8cm} C{0.5cm} C{1.0cm} L{1.2cm}}
      \toprule
\textbf{Experiment} & \textbf{Feature Type} & \textbf{Feature Scale-1, $\Lambda_a$} & \textbf{Feature Scale-2, $\Lambda_b$} & \textbf{$\alpha$} & \textbf{Input Frames} & \textbf{Loss Type} \\
      \midrule
      \textbf{Video Object Segmentation} & ($x,y,Y,Cb,Cr,t$) & (0.02,0.02,0.07,0.4,0.4,0.01) & (0.03,0.03,0.09,0.5,0.5,0.2) & 0.5 & 9 & Logistic\\
      \midrule
      \textbf{Semantic Video Segmentation} & & & & & \\
      \textbf{with CNN1~\cite{yu2015multi}-NoFlow} & ($x,y,R,G,B,t$) & (0.08,0.08,0.2,0.2,0.2,0.04) & (0.11,0.11,0.2,0.2,0.2,0.04) & 0.5 & 3 & Logistic \\
      \textbf{with CNN1~\cite{yu2015multi}-Flow} & ($x+u_x,y+u_y,R,G,B,t$) & (0.11,0.11,0.14,0.14,0.14,0.03) & (0.08,0.08,0.12,0.12,0.12,0.01) & 0.65 & 3 & Logistic\\
      \textbf{with CNN2~\cite{richter2016playing}-Flow} & ($x+u_x,y+u_y,R,G,B,t$) & (0.08,0.08,0.2,0.2,0.2,0.04) & (0.09,0.09,0.25,0.25,0.25,0.03) & 0.5 & 4 & Logistic\\
      \midrule
      \textbf{Video Color Propagation} & ($x,y,I,t$)  & (0.04,0.04,0.2,0.04) & No second kernel & 1 & 4 & MSE\\
      \bottomrule
      \\
    \end{tabular}
    \mycaption{Experiment Protocols} {Experiment protocols for the different experiments presented in this work. \textbf{Feature Types}:
    Feature spaces used for the bilateral convolutions, with position ($x,y$) and color
    ($R,G,B$ or $Y,Cb,Cr$) features $\in [0,255]$. $u_x$, $u_y$ denotes optical flow with respect
    to the present frame and $I$ denotes grayscale intensity.
    \textbf{Feature Scales ($\Lambda_a, \Lambda_b$)}: Cross-validated scales for the features used.
    \textbf{$\alpha$}: Exponential time decay for the input frames.
    \textbf{Input Frames}: Number of input frames for VPN.
    \textbf{Loss Type}: Type
     of loss used for back-propagation. ``MSE'' corresponds to Euclidean mean squared error loss and ``Logistic'' corresponds to multinomial logistic loss.}
  \label{tbl:parameters_supp}
\end{table*}

% \begin{figure}[th!]
% \begin{center}
%   \centerline{\includegraphics[width=\textwidth]{figures/video_seg_visuals_supp_small.pdf}}
%     \mycaption{Video Object Segmentation}
%     {Shown are the different frames in example videos with the corresponding
%     ground truth (GT) masks, predictions from BVS~\cite{marki2016bilateral},
%     OFL~\cite{tsaivideo}, VPN (VPN-Stage2) and VPN-DLab (VPN-DeepLab) models.}
%     \label{fig:video_seg_small_supp}
% \end{center}
% \vspace{-1.0cm}
% \end{figure}

\begin{figure}[th!]
\begin{center}
  \centerline{\includegraphics[width=0.7\textwidth]{figures/video_seg_visuals_supp_positive.pdf}}
    \mycaption{Video Object Segmentation}
    {Shown are the different frames in example videos with the corresponding
    ground truth (GT) masks, predictions from BVS~\cite{marki2016bilateral},
    OFL~\cite{tsaivideo}, VPN (VPN-Stage2) and VPN-DLab (VPN-DeepLab) models.}
    \label{fig:video_seg_pos_supp}
\end{center}
\vspace{-1.0cm}
\end{figure}

\begin{figure}[th!]
\begin{center}
  \centerline{\includegraphics[width=0.7\textwidth]{figures/video_seg_visuals_supp_negative.pdf}}
    \mycaption{Failure Cases for Video Object Segmentation}
    {Shown are the different frames in example videos with the corresponding
    ground truth (GT) masks, predictions from BVS~\cite{marki2016bilateral},
    OFL~\cite{tsaivideo}, VPN (VPN-Stage2) and VPN-DLab (VPN-DeepLab) models.}
    \label{fig:video_seg_neg_supp}
\end{center}
\vspace{-1.0cm}
\end{figure}

\begin{figure}[th!]
\begin{center}
  \centerline{\includegraphics[width=0.9\textwidth]{figures/supp_semantic_visual.pdf}}
    \mycaption{Semantic Video Segmentation}
    {Input video frames and the corresponding ground truth (GT)
    segmentation together with the predictions of CNN~\cite{yu2015multi} and with
    VPN-Flow.}
    \label{fig:semantic_visuals_supp}
\end{center}
\vspace{-0.7cm}
\end{figure}

\begin{figure}[th!]
\begin{center}
  \centerline{\includegraphics[width=\textwidth]{figures/colorization_visuals_supp.pdf}}
  \mycaption{Video Color Propagation}
  {Input grayscale video frames and corresponding ground-truth (GT) color images
  together with color predictions of Levin et al.~\cite{levin2004colorization} and VPN-Stage1 models.}
  \label{fig:color_visuals_supp}
\end{center}
\vspace{-0.7cm}
\end{figure}

\clearpage

\section{Additional Material for Bilateral Inception Networks}
\label{sec:binception-app}

In this section of the Appendix, we first discuss the use of approximate bilateral
filtering in BI modules (Sec.~\ref{sec:lattice}).
Later, we present some qualitative results using different models for the approach presented in
Chapter~\ref{chap:binception} (Sec.~\ref{sec:qualitative-app}).

\subsection{Approximate Bilateral Filtering}
\label{sec:lattice}

The bilateral inception module presented in Chapter~\ref{chap:binception} computes a matrix-vector
product between a Gaussian filter $K$ and a vector of activations $\bz_c$.
Bilateral filtering is an important operation and many algorithmic techniques have been
proposed to speed-up this operation~\cite{paris2006fast,adams2010fast,gastal2011domain}.
In the main paper we opted to implement what can be considered the
brute-force variant of explicitly constructing $K$ and then using BLAS to compute the
matrix-vector product. This resulted in a few millisecond operation.
The explicit way to compute is possible due to the
reduction to super-pixels, e.g., it would not work for DenseCRF variants
that operate on the full image resolution.

Here, we present experiments where we use the fast approximate bilateral filtering
algorithm of~\cite{adams2010fast}, which is also used in Chapter~\ref{chap:bnn}
for learning sparse high dimensional filters. This
choice allows for larger dimensions of matrix-vector multiplication. The reason for choosing
the explicit multiplication in Chapter~\ref{chap:binception} was that it was computationally faster.
For the small sizes of the involved matrices and vectors, the explicit computation is sufficient and we had no
GPU implementation of an approximate technique that matched this runtime. Also it
is conceptually easier and the gradient to the feature transformations ($\Lambda \mathbf{f}$) is
obtained using standard matrix calculus.

\subsubsection{Experiments}

We modified the existing segmentation architectures analogous to those in Chapter~\ref{chap:binception}.
The main difference is that, here, the inception modules use the lattice
approximation~\cite{adams2010fast} to compute the bilateral filtering.
Using the lattice approximation did not allow us to back-propagate through feature transformations ($\Lambda$)
and thus we used hand-specified feature scales as will be explained later.
Specifically, we take CNN architectures from the works
of~\cite{chen2014semantic,zheng2015conditional,bell2015minc} and insert the BI modules between
the spatial FC layers.
We use superpixels from~\cite{DollarICCV13edges}
for all the experiments with the lattice approximation. Experiments are
performed using Caffe neural network framework~\cite{jia2014caffe}.

\begin{table}
  \small
  \centering
  \begin{tabular}{p{5.5cm}>{\raggedright\arraybackslash}p{1.4cm}>{\centering\arraybackslash}p{2.2cm}}
    \toprule
		\textbf{Model} & \emph{IoU} & \emph{Runtime}(ms) \\
    \midrule

    %%%%%%%%%%%% Scores computed by us)%%%%%%%%%%%%
		\deeplablargefov & 68.9 & 145ms\\
    \midrule
    \bi{7}{2}-\bi{8}{10}& \textbf{73.8} & +600 \\
    \midrule
    \deeplablargefovcrf~\cite{chen2014semantic} & 72.7 & +830\\
    \deeplabmsclargefovcrf~\cite{chen2014semantic} & \textbf{73.6} & +880\\
    DeepLab-EdgeNet~\cite{chen2015semantic} & 71.7 & +30\\
    DeepLab-EdgeNet-CRF~\cite{chen2015semantic} & \textbf{73.6} & +860\\
  \bottomrule \\
  \end{tabular}
  \mycaption{Semantic Segmentation using the DeepLab model}
  {IoU scores on the Pascal VOC12 segmentation test dataset
  with different models and our modified inception model.
  Also shown are the corresponding runtimes in milliseconds. Runtimes
  also include superpixel computations (300 ms with Dollar superpixels~\cite{DollarICCV13edges})}
  \label{tab:largefovresults}
\end{table}

\paragraph{Semantic Segmentation}
The experiments in this section use the Pascal VOC12 segmentation dataset~\cite{voc2012segmentation} with 21 object classes and the images have a maximum resolution of 0.25 megapixels.
For all experiments on VOC12, we train using the extended training set of
10581 images collected by~\cite{hariharan2011moredata}.
We modified the \deeplab~network architecture of~\cite{chen2014semantic} and
the CRFasRNN architecture from~\cite{zheng2015conditional} which uses a CNN with
deconvolution layers followed by DenseCRF trained end-to-end.

\paragraph{DeepLab Model}\label{sec:deeplabmodel}
We experimented with the \bi{7}{2}-\bi{8}{10} inception model.
Results using the~\deeplab~model are summarized in Tab.~\ref{tab:largefovresults}.
Although we get similar improvements with inception modules as with the
explicit kernel computation, using lattice approximation is slower.

\begin{table}
  \small
  \centering
  \begin{tabular}{p{6.4cm}>{\raggedright\arraybackslash}p{1.8cm}>{\raggedright\arraybackslash}p{1.8cm}}
    \toprule
    \textbf{Model} & \emph{IoU (Val)} & \emph{IoU (Test)}\\
    \midrule
    %%%%%%%%%%%% Scores computed by us)%%%%%%%%%%%%
    CNN &  67.5 & - \\
    \deconv (CNN+Deconvolutions) & 69.8 & 72.0 \\
    \midrule
    \bi{3}{6}-\bi{4}{6}-\bi{7}{2}-\bi{8}{6}& 71.9 & - \\
    \bi{3}{6}-\bi{4}{6}-\bi{7}{2}-\bi{8}{6}-\gi{6}& 73.6 &  \href{http://host.robots.ox.ac.uk:8080/anonymous/VOTV5E.html}{\textbf{75.2}}\\
    \midrule
    \deconvcrf (CRF-RNN)~\cite{zheng2015conditional} & 73.0 & 74.7\\
    Context-CRF-RNN~\cite{yu2015multi} & ~~ - ~ & \textbf{75.3} \\
    \bottomrule \\
  \end{tabular}
  \mycaption{Semantic Segmentation using the CRFasRNN model}{IoU score corresponding to different models
  on Pascal VOC12 reduced validation / test segmentation dataset. The reduced validation set consists of 346 images
  as used in~\cite{zheng2015conditional} where we adapted the model from.}
  \label{tab:deconvresults-app}
\end{table}

\paragraph{CRFasRNN Model}\label{sec:deepinception}
We add BI modules after score-pool3, score-pool4, \fc{7} and \fc{8} $1\times1$ convolution layers
resulting in the \bi{3}{6}-\bi{4}{6}-\bi{7}{2}-\bi{8}{6}
model and also experimented with another variant where $BI_8$ is followed by another inception
module, G$(6)$, with 6 Gaussian kernels.
Note that here also we discarded both deconvolution and DenseCRF parts of the original model~\cite{zheng2015conditional}
and inserted the BI modules in the base CNN and found similar improvements compared to the inception modules with explicit
kernel computaion. See Tab.~\ref{tab:deconvresults-app} for results on the CRFasRNN model.

\paragraph{Material Segmentation}
Table~\ref{tab:mincresults-app} shows the results on the MINC dataset~\cite{bell2015minc}
obtained by modifying the AlexNet architecture with our inception modules. We observe
similar improvements as with explicit kernel construction.
For this model, we do not provide any learned setup due to very limited segment training
data. The weights to combine outputs in the bilateral inception layer are
found by validation on the validation set.

\begin{table}[t]
  \small
  \centering
  \begin{tabular}{p{3.5cm}>{\centering\arraybackslash}p{4.0cm}}
    \toprule
    \textbf{Model} & Class / Total accuracy\\
    \midrule

    %%%%%%%%%%%% Scores computed by us)%%%%%%%%%%%%
    AlexNet CNN & 55.3 / 58.9 \\
    \midrule
    \bi{7}{2}-\bi{8}{6}& 68.5 / 71.8 \\
    \bi{7}{2}-\bi{8}{6}-G$(6)$& 67.6 / 73.1 \\
    \midrule
    AlexNet-CRF & 65.5 / 71.0 \\
    \bottomrule \\
  \end{tabular}
  \mycaption{Material Segmentation using AlexNet}{Pixel accuracy of different models on
  the MINC material segmentation test dataset~\cite{bell2015minc}.}
  \label{tab:mincresults-app}
\end{table}

\paragraph{Scales of Bilateral Inception Modules}
\label{sec:scales}

Unlike the explicit kernel technique presented in the main text (Chapter~\ref{chap:binception}),
we didn't back-propagate through feature transformation ($\Lambda$)
using the approximate bilateral filter technique.
So, the feature scales are hand-specified and validated, which are as follows.
The optimal scale values for the \bi{7}{2}-\bi{8}{2} model are found by validation for the best performance which are
$\sigma_{xy}$ = (0.1, 0.1) for the spatial (XY) kernel and $\sigma_{rgbxy}$ = (0.1, 0.1, 0.1, 0.01, 0.01) for color and position (RGBXY)  kernel.
Next, as more kernels are added to \bi{8}{2}, we set scales to be $\alpha$*($\sigma_{xy}$, $\sigma_{rgbxy}$).
The value of $\alpha$ is chosen as  1, 0.5, 0.1, 0.05, 0.1, at uniform interval, for the \bi{8}{10} bilateral inception module.


\subsection{Qualitative Results}
\label{sec:qualitative-app}

In this section, we present more qualitative results obtained using the BI module with explicit
kernel computation technique presented in Chapter~\ref{chap:binception}. Results on the Pascal VOC12
dataset~\cite{voc2012segmentation} using the DeepLab-LargeFOV model are shown in Fig.~\ref{fig:semantic_visuals-app},
followed by the results on MINC dataset~\cite{bell2015minc}
in Fig.~\ref{fig:material_visuals-app} and on
Cityscapes dataset~\cite{Cordts2015Cvprw} in Fig.~\ref{fig:street_visuals-app}.


\definecolor{voc_1}{RGB}{0, 0, 0}
\definecolor{voc_2}{RGB}{128, 0, 0}
\definecolor{voc_3}{RGB}{0, 128, 0}
\definecolor{voc_4}{RGB}{128, 128, 0}
\definecolor{voc_5}{RGB}{0, 0, 128}
\definecolor{voc_6}{RGB}{128, 0, 128}
\definecolor{voc_7}{RGB}{0, 128, 128}
\definecolor{voc_8}{RGB}{128, 128, 128}
\definecolor{voc_9}{RGB}{64, 0, 0}
\definecolor{voc_10}{RGB}{192, 0, 0}
\definecolor{voc_11}{RGB}{64, 128, 0}
\definecolor{voc_12}{RGB}{192, 128, 0}
\definecolor{voc_13}{RGB}{64, 0, 128}
\definecolor{voc_14}{RGB}{192, 0, 128}
\definecolor{voc_15}{RGB}{64, 128, 128}
\definecolor{voc_16}{RGB}{192, 128, 128}
\definecolor{voc_17}{RGB}{0, 64, 0}
\definecolor{voc_18}{RGB}{128, 64, 0}
\definecolor{voc_19}{RGB}{0, 192, 0}
\definecolor{voc_20}{RGB}{128, 192, 0}
\definecolor{voc_21}{RGB}{0, 64, 128}
\definecolor{voc_22}{RGB}{128, 64, 128}

\begin{figure*}[!ht]
  \small
  \centering
  \fcolorbox{white}{voc_1}{\rule{0pt}{4pt}\rule{4pt}{0pt}} Background~~
  \fcolorbox{white}{voc_2}{\rule{0pt}{4pt}\rule{4pt}{0pt}} Aeroplane~~
  \fcolorbox{white}{voc_3}{\rule{0pt}{4pt}\rule{4pt}{0pt}} Bicycle~~
  \fcolorbox{white}{voc_4}{\rule{0pt}{4pt}\rule{4pt}{0pt}} Bird~~
  \fcolorbox{white}{voc_5}{\rule{0pt}{4pt}\rule{4pt}{0pt}} Boat~~
  \fcolorbox{white}{voc_6}{\rule{0pt}{4pt}\rule{4pt}{0pt}} Bottle~~
  \fcolorbox{white}{voc_7}{\rule{0pt}{4pt}\rule{4pt}{0pt}} Bus~~
  \fcolorbox{white}{voc_8}{\rule{0pt}{4pt}\rule{4pt}{0pt}} Car~~\\
  \fcolorbox{white}{voc_9}{\rule{0pt}{4pt}\rule{4pt}{0pt}} Cat~~
  \fcolorbox{white}{voc_10}{\rule{0pt}{4pt}\rule{4pt}{0pt}} Chair~~
  \fcolorbox{white}{voc_11}{\rule{0pt}{4pt}\rule{4pt}{0pt}} Cow~~
  \fcolorbox{white}{voc_12}{\rule{0pt}{4pt}\rule{4pt}{0pt}} Dining Table~~
  \fcolorbox{white}{voc_13}{\rule{0pt}{4pt}\rule{4pt}{0pt}} Dog~~
  \fcolorbox{white}{voc_14}{\rule{0pt}{4pt}\rule{4pt}{0pt}} Horse~~
  \fcolorbox{white}{voc_15}{\rule{0pt}{4pt}\rule{4pt}{0pt}} Motorbike~~
  \fcolorbox{white}{voc_16}{\rule{0pt}{4pt}\rule{4pt}{0pt}} Person~~\\
  \fcolorbox{white}{voc_17}{\rule{0pt}{4pt}\rule{4pt}{0pt}} Potted Plant~~
  \fcolorbox{white}{voc_18}{\rule{0pt}{4pt}\rule{4pt}{0pt}} Sheep~~
  \fcolorbox{white}{voc_19}{\rule{0pt}{4pt}\rule{4pt}{0pt}} Sofa~~
  \fcolorbox{white}{voc_20}{\rule{0pt}{4pt}\rule{4pt}{0pt}} Train~~
  \fcolorbox{white}{voc_21}{\rule{0pt}{4pt}\rule{4pt}{0pt}} TV monitor~~\\


  \subfigure{%
    \includegraphics[width=.15\columnwidth]{figures/supplementary/2008_001308_given.png}
  }
  \subfigure{%
    \includegraphics[width=.15\columnwidth]{figures/supplementary/2008_001308_sp.png}
  }
  \subfigure{%
    \includegraphics[width=.15\columnwidth]{figures/supplementary/2008_001308_gt.png}
  }
  \subfigure{%
    \includegraphics[width=.15\columnwidth]{figures/supplementary/2008_001308_cnn.png}
  }
  \subfigure{%
    \includegraphics[width=.15\columnwidth]{figures/supplementary/2008_001308_crf.png}
  }
  \subfigure{%
    \includegraphics[width=.15\columnwidth]{figures/supplementary/2008_001308_ours.png}
  }\\[-2ex]


  \subfigure{%
    \includegraphics[width=.15\columnwidth]{figures/supplementary/2008_001821_given.png}
  }
  \subfigure{%
    \includegraphics[width=.15\columnwidth]{figures/supplementary/2008_001821_sp.png}
  }
  \subfigure{%
    \includegraphics[width=.15\columnwidth]{figures/supplementary/2008_001821_gt.png}
  }
  \subfigure{%
    \includegraphics[width=.15\columnwidth]{figures/supplementary/2008_001821_cnn.png}
  }
  \subfigure{%
    \includegraphics[width=.15\columnwidth]{figures/supplementary/2008_001821_crf.png}
  }
  \subfigure{%
    \includegraphics[width=.15\columnwidth]{figures/supplementary/2008_001821_ours.png}
  }\\[-2ex]



  \subfigure{%
    \includegraphics[width=.15\columnwidth]{figures/supplementary/2008_004612_given.png}
  }
  \subfigure{%
    \includegraphics[width=.15\columnwidth]{figures/supplementary/2008_004612_sp.png}
  }
  \subfigure{%
    \includegraphics[width=.15\columnwidth]{figures/supplementary/2008_004612_gt.png}
  }
  \subfigure{%
    \includegraphics[width=.15\columnwidth]{figures/supplementary/2008_004612_cnn.png}
  }
  \subfigure{%
    \includegraphics[width=.15\columnwidth]{figures/supplementary/2008_004612_crf.png}
  }
  \subfigure{%
    \includegraphics[width=.15\columnwidth]{figures/supplementary/2008_004612_ours.png}
  }\\[-2ex]


  \subfigure{%
    \includegraphics[width=.15\columnwidth]{figures/supplementary/2009_001008_given.png}
  }
  \subfigure{%
    \includegraphics[width=.15\columnwidth]{figures/supplementary/2009_001008_sp.png}
  }
  \subfigure{%
    \includegraphics[width=.15\columnwidth]{figures/supplementary/2009_001008_gt.png}
  }
  \subfigure{%
    \includegraphics[width=.15\columnwidth]{figures/supplementary/2009_001008_cnn.png}
  }
  \subfigure{%
    \includegraphics[width=.15\columnwidth]{figures/supplementary/2009_001008_crf.png}
  }
  \subfigure{%
    \includegraphics[width=.15\columnwidth]{figures/supplementary/2009_001008_ours.png}
  }\\[-2ex]




  \subfigure{%
    \includegraphics[width=.15\columnwidth]{figures/supplementary/2009_004497_given.png}
  }
  \subfigure{%
    \includegraphics[width=.15\columnwidth]{figures/supplementary/2009_004497_sp.png}
  }
  \subfigure{%
    \includegraphics[width=.15\columnwidth]{figures/supplementary/2009_004497_gt.png}
  }
  \subfigure{%
    \includegraphics[width=.15\columnwidth]{figures/supplementary/2009_004497_cnn.png}
  }
  \subfigure{%
    \includegraphics[width=.15\columnwidth]{figures/supplementary/2009_004497_crf.png}
  }
  \subfigure{%
    \includegraphics[width=.15\columnwidth]{figures/supplementary/2009_004497_ours.png}
  }\\[-2ex]



  \setcounter{subfigure}{0}
  \subfigure[\scriptsize Input]{%
    \includegraphics[width=.15\columnwidth]{figures/supplementary/2010_001327_given.png}
  }
  \subfigure[\scriptsize Superpixels]{%
    \includegraphics[width=.15\columnwidth]{figures/supplementary/2010_001327_sp.png}
  }
  \subfigure[\scriptsize GT]{%
    \includegraphics[width=.15\columnwidth]{figures/supplementary/2010_001327_gt.png}
  }
  \subfigure[\scriptsize Deeplab]{%
    \includegraphics[width=.15\columnwidth]{figures/supplementary/2010_001327_cnn.png}
  }
  \subfigure[\scriptsize +DenseCRF]{%
    \includegraphics[width=.15\columnwidth]{figures/supplementary/2010_001327_crf.png}
  }
  \subfigure[\scriptsize Using BI]{%
    \includegraphics[width=.15\columnwidth]{figures/supplementary/2010_001327_ours.png}
  }
  \mycaption{Semantic Segmentation}{Example results of semantic segmentation
  on the Pascal VOC12 dataset.
  (d)~depicts the DeepLab CNN result, (e)~CNN + 10 steps of mean-field inference,
  (f~result obtained with bilateral inception (BI) modules (\bi{6}{2}+\bi{7}{6}) between \fc~layers.}
  \label{fig:semantic_visuals-app}
\end{figure*}


\definecolor{minc_1}{HTML}{771111}
\definecolor{minc_2}{HTML}{CAC690}
\definecolor{minc_3}{HTML}{EEEEEE}
\definecolor{minc_4}{HTML}{7C8FA6}
\definecolor{minc_5}{HTML}{597D31}
\definecolor{minc_6}{HTML}{104410}
\definecolor{minc_7}{HTML}{BB819C}
\definecolor{minc_8}{HTML}{D0CE48}
\definecolor{minc_9}{HTML}{622745}
\definecolor{minc_10}{HTML}{666666}
\definecolor{minc_11}{HTML}{D54A31}
\definecolor{minc_12}{HTML}{101044}
\definecolor{minc_13}{HTML}{444126}
\definecolor{minc_14}{HTML}{75D646}
\definecolor{minc_15}{HTML}{DD4348}
\definecolor{minc_16}{HTML}{5C8577}
\definecolor{minc_17}{HTML}{C78472}
\definecolor{minc_18}{HTML}{75D6D0}
\definecolor{minc_19}{HTML}{5B4586}
\definecolor{minc_20}{HTML}{C04393}
\definecolor{minc_21}{HTML}{D69948}
\definecolor{minc_22}{HTML}{7370D8}
\definecolor{minc_23}{HTML}{7A3622}
\definecolor{minc_24}{HTML}{000000}

\begin{figure*}[!ht]
  \small % scriptsize
  \centering
  \fcolorbox{white}{minc_1}{\rule{0pt}{4pt}\rule{4pt}{0pt}} Brick~~
  \fcolorbox{white}{minc_2}{\rule{0pt}{4pt}\rule{4pt}{0pt}} Carpet~~
  \fcolorbox{white}{minc_3}{\rule{0pt}{4pt}\rule{4pt}{0pt}} Ceramic~~
  \fcolorbox{white}{minc_4}{\rule{0pt}{4pt}\rule{4pt}{0pt}} Fabric~~
  \fcolorbox{white}{minc_5}{\rule{0pt}{4pt}\rule{4pt}{0pt}} Foliage~~
  \fcolorbox{white}{minc_6}{\rule{0pt}{4pt}\rule{4pt}{0pt}} Food~~
  \fcolorbox{white}{minc_7}{\rule{0pt}{4pt}\rule{4pt}{0pt}} Glass~~
  \fcolorbox{white}{minc_8}{\rule{0pt}{4pt}\rule{4pt}{0pt}} Hair~~\\
  \fcolorbox{white}{minc_9}{\rule{0pt}{4pt}\rule{4pt}{0pt}} Leather~~
  \fcolorbox{white}{minc_10}{\rule{0pt}{4pt}\rule{4pt}{0pt}} Metal~~
  \fcolorbox{white}{minc_11}{\rule{0pt}{4pt}\rule{4pt}{0pt}} Mirror~~
  \fcolorbox{white}{minc_12}{\rule{0pt}{4pt}\rule{4pt}{0pt}} Other~~
  \fcolorbox{white}{minc_13}{\rule{0pt}{4pt}\rule{4pt}{0pt}} Painted~~
  \fcolorbox{white}{minc_14}{\rule{0pt}{4pt}\rule{4pt}{0pt}} Paper~~
  \fcolorbox{white}{minc_15}{\rule{0pt}{4pt}\rule{4pt}{0pt}} Plastic~~\\
  \fcolorbox{white}{minc_16}{\rule{0pt}{4pt}\rule{4pt}{0pt}} Polished Stone~~
  \fcolorbox{white}{minc_17}{\rule{0pt}{4pt}\rule{4pt}{0pt}} Skin~~
  \fcolorbox{white}{minc_18}{\rule{0pt}{4pt}\rule{4pt}{0pt}} Sky~~
  \fcolorbox{white}{minc_19}{\rule{0pt}{4pt}\rule{4pt}{0pt}} Stone~~
  \fcolorbox{white}{minc_20}{\rule{0pt}{4pt}\rule{4pt}{0pt}} Tile~~
  \fcolorbox{white}{minc_21}{\rule{0pt}{4pt}\rule{4pt}{0pt}} Wallpaper~~
  \fcolorbox{white}{minc_22}{\rule{0pt}{4pt}\rule{4pt}{0pt}} Water~~
  \fcolorbox{white}{minc_23}{\rule{0pt}{4pt}\rule{4pt}{0pt}} Wood~~\\
  \subfigure{%
    \includegraphics[width=.15\columnwidth]{figures/supplementary/000008468_given.png}
  }
  \subfigure{%
    \includegraphics[width=.15\columnwidth]{figures/supplementary/000008468_sp.png}
  }
  \subfigure{%
    \includegraphics[width=.15\columnwidth]{figures/supplementary/000008468_gt.png}
  }
  \subfigure{%
    \includegraphics[width=.15\columnwidth]{figures/supplementary/000008468_cnn.png}
  }
  \subfigure{%
    \includegraphics[width=.15\columnwidth]{figures/supplementary/000008468_crf.png}
  }
  \subfigure{%
    \includegraphics[width=.15\columnwidth]{figures/supplementary/000008468_ours.png}
  }\\[-2ex]

  \subfigure{%
    \includegraphics[width=.15\columnwidth]{figures/supplementary/000009053_given.png}
  }
  \subfigure{%
    \includegraphics[width=.15\columnwidth]{figures/supplementary/000009053_sp.png}
  }
  \subfigure{%
    \includegraphics[width=.15\columnwidth]{figures/supplementary/000009053_gt.png}
  }
  \subfigure{%
    \includegraphics[width=.15\columnwidth]{figures/supplementary/000009053_cnn.png}
  }
  \subfigure{%
    \includegraphics[width=.15\columnwidth]{figures/supplementary/000009053_crf.png}
  }
  \subfigure{%
    \includegraphics[width=.15\columnwidth]{figures/supplementary/000009053_ours.png}
  }\\[-2ex]




  \subfigure{%
    \includegraphics[width=.15\columnwidth]{figures/supplementary/000014977_given.png}
  }
  \subfigure{%
    \includegraphics[width=.15\columnwidth]{figures/supplementary/000014977_sp.png}
  }
  \subfigure{%
    \includegraphics[width=.15\columnwidth]{figures/supplementary/000014977_gt.png}
  }
  \subfigure{%
    \includegraphics[width=.15\columnwidth]{figures/supplementary/000014977_cnn.png}
  }
  \subfigure{%
    \includegraphics[width=.15\columnwidth]{figures/supplementary/000014977_crf.png}
  }
  \subfigure{%
    \includegraphics[width=.15\columnwidth]{figures/supplementary/000014977_ours.png}
  }\\[-2ex]


  \subfigure{%
    \includegraphics[width=.15\columnwidth]{figures/supplementary/000022922_given.png}
  }
  \subfigure{%
    \includegraphics[width=.15\columnwidth]{figures/supplementary/000022922_sp.png}
  }
  \subfigure{%
    \includegraphics[width=.15\columnwidth]{figures/supplementary/000022922_gt.png}
  }
  \subfigure{%
    \includegraphics[width=.15\columnwidth]{figures/supplementary/000022922_cnn.png}
  }
  \subfigure{%
    \includegraphics[width=.15\columnwidth]{figures/supplementary/000022922_crf.png}
  }
  \subfigure{%
    \includegraphics[width=.15\columnwidth]{figures/supplementary/000022922_ours.png}
  }\\[-2ex]


  \subfigure{%
    \includegraphics[width=.15\columnwidth]{figures/supplementary/000025711_given.png}
  }
  \subfigure{%
    \includegraphics[width=.15\columnwidth]{figures/supplementary/000025711_sp.png}
  }
  \subfigure{%
    \includegraphics[width=.15\columnwidth]{figures/supplementary/000025711_gt.png}
  }
  \subfigure{%
    \includegraphics[width=.15\columnwidth]{figures/supplementary/000025711_cnn.png}
  }
  \subfigure{%
    \includegraphics[width=.15\columnwidth]{figures/supplementary/000025711_crf.png}
  }
  \subfigure{%
    \includegraphics[width=.15\columnwidth]{figures/supplementary/000025711_ours.png}
  }\\[-2ex]


  \subfigure{%
    \includegraphics[width=.15\columnwidth]{figures/supplementary/000034473_given.png}
  }
  \subfigure{%
    \includegraphics[width=.15\columnwidth]{figures/supplementary/000034473_sp.png}
  }
  \subfigure{%
    \includegraphics[width=.15\columnwidth]{figures/supplementary/000034473_gt.png}
  }
  \subfigure{%
    \includegraphics[width=.15\columnwidth]{figures/supplementary/000034473_cnn.png}
  }
  \subfigure{%
    \includegraphics[width=.15\columnwidth]{figures/supplementary/000034473_crf.png}
  }
  \subfigure{%
    \includegraphics[width=.15\columnwidth]{figures/supplementary/000034473_ours.png}
  }\\[-2ex]


  \subfigure{%
    \includegraphics[width=.15\columnwidth]{figures/supplementary/000035463_given.png}
  }
  \subfigure{%
    \includegraphics[width=.15\columnwidth]{figures/supplementary/000035463_sp.png}
  }
  \subfigure{%
    \includegraphics[width=.15\columnwidth]{figures/supplementary/000035463_gt.png}
  }
  \subfigure{%
    \includegraphics[width=.15\columnwidth]{figures/supplementary/000035463_cnn.png}
  }
  \subfigure{%
    \includegraphics[width=.15\columnwidth]{figures/supplementary/000035463_crf.png}
  }
  \subfigure{%
    \includegraphics[width=.15\columnwidth]{figures/supplementary/000035463_ours.png}
  }\\[-2ex]


  \setcounter{subfigure}{0}
  \subfigure[\scriptsize Input]{%
    \includegraphics[width=.15\columnwidth]{figures/supplementary/000035993_given.png}
  }
  \subfigure[\scriptsize Superpixels]{%
    \includegraphics[width=.15\columnwidth]{figures/supplementary/000035993_sp.png}
  }
  \subfigure[\scriptsize GT]{%
    \includegraphics[width=.15\columnwidth]{figures/supplementary/000035993_gt.png}
  }
  \subfigure[\scriptsize AlexNet]{%
    \includegraphics[width=.15\columnwidth]{figures/supplementary/000035993_cnn.png}
  }
  \subfigure[\scriptsize +DenseCRF]{%
    \includegraphics[width=.15\columnwidth]{figures/supplementary/000035993_crf.png}
  }
  \subfigure[\scriptsize Using BI]{%
    \includegraphics[width=.15\columnwidth]{figures/supplementary/000035993_ours.png}
  }
  \mycaption{Material Segmentation}{Example results of material segmentation.
  (d)~depicts the AlexNet CNN result, (e)~CNN + 10 steps of mean-field inference,
  (f)~result obtained with bilateral inception (BI) modules (\bi{7}{2}+\bi{8}{6}) between
  \fc~layers.}
\label{fig:material_visuals-app}
\end{figure*}


\definecolor{city_1}{RGB}{128, 64, 128}
\definecolor{city_2}{RGB}{244, 35, 232}
\definecolor{city_3}{RGB}{70, 70, 70}
\definecolor{city_4}{RGB}{102, 102, 156}
\definecolor{city_5}{RGB}{190, 153, 153}
\definecolor{city_6}{RGB}{153, 153, 153}
\definecolor{city_7}{RGB}{250, 170, 30}
\definecolor{city_8}{RGB}{220, 220, 0}
\definecolor{city_9}{RGB}{107, 142, 35}
\definecolor{city_10}{RGB}{152, 251, 152}
\definecolor{city_11}{RGB}{70, 130, 180}
\definecolor{city_12}{RGB}{220, 20, 60}
\definecolor{city_13}{RGB}{255, 0, 0}
\definecolor{city_14}{RGB}{0, 0, 142}
\definecolor{city_15}{RGB}{0, 0, 70}
\definecolor{city_16}{RGB}{0, 60, 100}
\definecolor{city_17}{RGB}{0, 80, 100}
\definecolor{city_18}{RGB}{0, 0, 230}
\definecolor{city_19}{RGB}{119, 11, 32}
\begin{figure*}[!ht]
  \small % scriptsize
  \centering


  \subfigure{%
    \includegraphics[width=.18\columnwidth]{figures/supplementary/frankfurt00000_016005_given.png}
  }
  \subfigure{%
    \includegraphics[width=.18\columnwidth]{figures/supplementary/frankfurt00000_016005_sp.png}
  }
  \subfigure{%
    \includegraphics[width=.18\columnwidth]{figures/supplementary/frankfurt00000_016005_gt.png}
  }
  \subfigure{%
    \includegraphics[width=.18\columnwidth]{figures/supplementary/frankfurt00000_016005_cnn.png}
  }
  \subfigure{%
    \includegraphics[width=.18\columnwidth]{figures/supplementary/frankfurt00000_016005_ours.png}
  }\\[-2ex]

  \subfigure{%
    \includegraphics[width=.18\columnwidth]{figures/supplementary/frankfurt00000_004617_given.png}
  }
  \subfigure{%
    \includegraphics[width=.18\columnwidth]{figures/supplementary/frankfurt00000_004617_sp.png}
  }
  \subfigure{%
    \includegraphics[width=.18\columnwidth]{figures/supplementary/frankfurt00000_004617_gt.png}
  }
  \subfigure{%
    \includegraphics[width=.18\columnwidth]{figures/supplementary/frankfurt00000_004617_cnn.png}
  }
  \subfigure{%
    \includegraphics[width=.18\columnwidth]{figures/supplementary/frankfurt00000_004617_ours.png}
  }\\[-2ex]

  \subfigure{%
    \includegraphics[width=.18\columnwidth]{figures/supplementary/frankfurt00000_020880_given.png}
  }
  \subfigure{%
    \includegraphics[width=.18\columnwidth]{figures/supplementary/frankfurt00000_020880_sp.png}
  }
  \subfigure{%
    \includegraphics[width=.18\columnwidth]{figures/supplementary/frankfurt00000_020880_gt.png}
  }
  \subfigure{%
    \includegraphics[width=.18\columnwidth]{figures/supplementary/frankfurt00000_020880_cnn.png}
  }
  \subfigure{%
    \includegraphics[width=.18\columnwidth]{figures/supplementary/frankfurt00000_020880_ours.png}
  }\\[-2ex]



  \subfigure{%
    \includegraphics[width=.18\columnwidth]{figures/supplementary/frankfurt00001_007285_given.png}
  }
  \subfigure{%
    \includegraphics[width=.18\columnwidth]{figures/supplementary/frankfurt00001_007285_sp.png}
  }
  \subfigure{%
    \includegraphics[width=.18\columnwidth]{figures/supplementary/frankfurt00001_007285_gt.png}
  }
  \subfigure{%
    \includegraphics[width=.18\columnwidth]{figures/supplementary/frankfurt00001_007285_cnn.png}
  }
  \subfigure{%
    \includegraphics[width=.18\columnwidth]{figures/supplementary/frankfurt00001_007285_ours.png}
  }\\[-2ex]


  \subfigure{%
    \includegraphics[width=.18\columnwidth]{figures/supplementary/frankfurt00001_059789_given.png}
  }
  \subfigure{%
    \includegraphics[width=.18\columnwidth]{figures/supplementary/frankfurt00001_059789_sp.png}
  }
  \subfigure{%
    \includegraphics[width=.18\columnwidth]{figures/supplementary/frankfurt00001_059789_gt.png}
  }
  \subfigure{%
    \includegraphics[width=.18\columnwidth]{figures/supplementary/frankfurt00001_059789_cnn.png}
  }
  \subfigure{%
    \includegraphics[width=.18\columnwidth]{figures/supplementary/frankfurt00001_059789_ours.png}
  }\\[-2ex]


  \subfigure{%
    \includegraphics[width=.18\columnwidth]{figures/supplementary/frankfurt00001_068208_given.png}
  }
  \subfigure{%
    \includegraphics[width=.18\columnwidth]{figures/supplementary/frankfurt00001_068208_sp.png}
  }
  \subfigure{%
    \includegraphics[width=.18\columnwidth]{figures/supplementary/frankfurt00001_068208_gt.png}
  }
  \subfigure{%
    \includegraphics[width=.18\columnwidth]{figures/supplementary/frankfurt00001_068208_cnn.png}
  }
  \subfigure{%
    \includegraphics[width=.18\columnwidth]{figures/supplementary/frankfurt00001_068208_ours.png}
  }\\[-2ex]

  \subfigure{%
    \includegraphics[width=.18\columnwidth]{figures/supplementary/frankfurt00001_082466_given.png}
  }
  \subfigure{%
    \includegraphics[width=.18\columnwidth]{figures/supplementary/frankfurt00001_082466_sp.png}
  }
  \subfigure{%
    \includegraphics[width=.18\columnwidth]{figures/supplementary/frankfurt00001_082466_gt.png}
  }
  \subfigure{%
    \includegraphics[width=.18\columnwidth]{figures/supplementary/frankfurt00001_082466_cnn.png}
  }
  \subfigure{%
    \includegraphics[width=.18\columnwidth]{figures/supplementary/frankfurt00001_082466_ours.png}
  }\\[-2ex]

  \subfigure{%
    \includegraphics[width=.18\columnwidth]{figures/supplementary/lindau00033_000019_given.png}
  }
  \subfigure{%
    \includegraphics[width=.18\columnwidth]{figures/supplementary/lindau00033_000019_sp.png}
  }
  \subfigure{%
    \includegraphics[width=.18\columnwidth]{figures/supplementary/lindau00033_000019_gt.png}
  }
  \subfigure{%
    \includegraphics[width=.18\columnwidth]{figures/supplementary/lindau00033_000019_cnn.png}
  }
  \subfigure{%
    \includegraphics[width=.18\columnwidth]{figures/supplementary/lindau00033_000019_ours.png}
  }\\[-2ex]

  \subfigure{%
    \includegraphics[width=.18\columnwidth]{figures/supplementary/lindau00052_000019_given.png}
  }
  \subfigure{%
    \includegraphics[width=.18\columnwidth]{figures/supplementary/lindau00052_000019_sp.png}
  }
  \subfigure{%
    \includegraphics[width=.18\columnwidth]{figures/supplementary/lindau00052_000019_gt.png}
  }
  \subfigure{%
    \includegraphics[width=.18\columnwidth]{figures/supplementary/lindau00052_000019_cnn.png}
  }
  \subfigure{%
    \includegraphics[width=.18\columnwidth]{figures/supplementary/lindau00052_000019_ours.png}
  }\\[-2ex]




  \subfigure{%
    \includegraphics[width=.18\columnwidth]{figures/supplementary/lindau00027_000019_given.png}
  }
  \subfigure{%
    \includegraphics[width=.18\columnwidth]{figures/supplementary/lindau00027_000019_sp.png}
  }
  \subfigure{%
    \includegraphics[width=.18\columnwidth]{figures/supplementary/lindau00027_000019_gt.png}
  }
  \subfigure{%
    \includegraphics[width=.18\columnwidth]{figures/supplementary/lindau00027_000019_cnn.png}
  }
  \subfigure{%
    \includegraphics[width=.18\columnwidth]{figures/supplementary/lindau00027_000019_ours.png}
  }\\[-2ex]



  \setcounter{subfigure}{0}
  \subfigure[\scriptsize Input]{%
    \includegraphics[width=.18\columnwidth]{figures/supplementary/lindau00029_000019_given.png}
  }
  \subfigure[\scriptsize Superpixels]{%
    \includegraphics[width=.18\columnwidth]{figures/supplementary/lindau00029_000019_sp.png}
  }
  \subfigure[\scriptsize GT]{%
    \includegraphics[width=.18\columnwidth]{figures/supplementary/lindau00029_000019_gt.png}
  }
  \subfigure[\scriptsize Deeplab]{%
    \includegraphics[width=.18\columnwidth]{figures/supplementary/lindau00029_000019_cnn.png}
  }
  \subfigure[\scriptsize Using BI]{%
    \includegraphics[width=.18\columnwidth]{figures/supplementary/lindau00029_000019_ours.png}
  }%\\[-2ex]

  \mycaption{Street Scene Segmentation}{Example results of street scene segmentation.
  (d)~depicts the DeepLab results, (e)~result obtained by adding bilateral inception (BI) modules (\bi{6}{2}+\bi{7}{6}) between \fc~layers.}
\label{fig:street_visuals-app}
\end{figure*}

\section{\minfsketch: Combining multiple \fsketch}\label{subsec:minfsketch}
%We observed that our \fsketch-based estimator always overestimates the actual Hamming distance.

We proved in Lemma~\ref{lem:hconcentrationlemma} that our estimate $\hat{h}$ is within an additive error of $h$. 
A standard approach to improve the accuracy in such situations is to obtain several independent estimates and then compute a suitable statistic of the estimates. We were faced with a choice of mean, median and minimum of the estimates of which we decided to choose median after extensive empirical evaluation (see Section~{\ref{subsec:box_plot_median}}) and obtaining theoretical justification (explained in Section~\ref{appendix:subsec:minfsketch}). We first explain our algorithms in the next subsection.
%\begin{proof}
%Define the function $g(f) = \ln(1-\tfrac{f}{dP})/\ln D$; observe that
%$\hat{h}=g(f)$ and $h=g(f^*)$. $g(f)$ is convex within the domain $f\in [0,dP)$.
%Even though $f$ is allowed to range from 0 to $d$, for the sake of simplicity
%    assume that $f$ is less than $dP$ (this assumption is justified since
%    $f^*=dP - dPD^h \ll dP$~\ref{lem:expectation} and $f \le f^* +
%    \epsilon^2 d^2$~\ref{cor:eps-d}). Since $g(f)$ is convex in
%this domain, so Jensen's inequality can be applied to get the next observation
%that shows that $\hat{h}$ is a biased estimator of $h$. So, $\E[\hat{h}] = \E[g(f)] \ge g(\E[f]) = g(f^*) = h$.
%\end{proof}

\begin{figure}
    \centering
    \includegraphics[width=0.9\linewidth]{images/catsketch.pdf}
    \caption{\minfsketch for categorical data --- sketch of each data point is a 2-dimensional array whose each row is an \fsketch. The $i$-th rows corresponding to all the data points use the same values of $\rho,R$.}% The prime number $p$ can be kept same throughout.}
    \label{fig:catsketch-schema}
    %\vspace*{-2mm}
\end{figure}

%We have also given a bound above on the difference between $\hat{h}$ and $h$.
%To reduce this noise, we create \minfsketch which is simply a collection of multiple independent instances of \fsketch.

\subsection{Algorithms for generating a sketch and estimaing Hamming distance}
Let $k,d$ be some suitably chosen integer parameters. An arity-$k$ dimension-$d$ \minfsketch for a categorical data, say $x$, is an array of $k$ sketches: $\Phi(x) = \langle \phi^1(x), \phi^2(x), \ldots \phi^k(x) \rangle$; the $i$-th entry of $\Phi(x)$ is a $d$-dimensional \fsketch. See Figure~\ref{fig:catsketch-schema} for an illustration. Note that the internal parameters $\rho,R,p$ required to run \fsketch to obtain the $i$-entry are same across all data points; the parameters corresponding to different $i$ are, however, chosen independently ($p$ can be the same).

Our algorithm for Hamming distance estimation is inspired from the Count-Median sketch~\cite{CORMODE200558} and Count sketch~\cite{CHARIKAR20043}. It estimates the Hamming distances between the pairs of ``rows'' from $\Phi(x)$ and $\Phi(y)$ and returns the median of the estimated distances. This procedure is followed in Algorithm~\ref{algo:fsketchestimate}.%; it is possible that for two \fsketch vectors, their observed Hamming distance $f \ge dP$ and in that case, the algorithm computes the corresponding $\hat{h}$ as as the worst possible value of $2\sigma$.


\newcommand{\hxy}{\widehat{h}}
%\newcommand{\hxy}{\widehat{h_{x,y}}}

    \begin{algorithm}
	\noindent\hspace*{\algorithmicindent} \textbf{Input:} $\Phi(x)=\langle \phi^1(x), \phi^2(x), \ldots \phi^k(x) \rangle$, $\Phi(y)=\langle \phi^1(y), \phi^2(y), \ldots \phi^k(y) \rangle$
	\begin{algorithmic}[1]
	    \For{$i=1 \ldots k$}
            \State Compute $f=$ Hamming distance between $\phi^i(x)$ and $\phi^i(y)$
            \State If $f < dP$, $\hxy^i=\ln\left(1-\frac{f}{dP}\right)/\ln D$
            \State Else $\hxy^i=2\sigma$
        \EndFor
        \State \Return $\hat{h} = \min\{ \hxy^1, \hxy^2, \ldots \hxy^k\}$
	\end{algorithmic}
	\caption{Estimate Hamming distance between $x$ and $y$ from their \minfsketch\label{algo:fsketchestimate}}
    \end{algorithm}
    

\begin{figure*}[t]
\centering
\includegraphics[width=\linewidth]{images/Median.pdf}
\caption{{Box plot for the median, mean, and minimum of the \texttt{FSketch}’s estimate obtain \textit{via} it’s from its $10$ repetitions, then each experiment is repeated $10$ times for computing   the variance of these statistics.  The black dotted line corresponds to the actual Hamming distance. }}
\label{fig:box_plot_median_mean_min}
\end{figure*}
    
    
\subsection{Theoretical justification}\label{appendix:subsec:minfsketch}

%\overestimator*
%
%\begin{proof}
%Define the function $g(f) = \ln(1-\tfrac{f}{dP})/\ln D$; observe that
%$\hat{h}=g(f)$ and $h=g(f^*)$. $g(f)$ is convex within the domain $f\in [0,dP)$.
%Even though $f$ is allowed to range from 0 to $d$, for the sake of simplicity
%    assume that $f$ is less than $dP$ (this assumption is justified since
%    $f^*=dP - dPD^h \ll dP$~\ref{lem:expectation} and $f \le f^* +
%    \epsilon^2 d^2$~\ref{cor:eps-d}). Since $g(f)$ is convex in
%this domain, so Jensen's inequality can be applied to get the next observation
%that shows that $\hat{h}$ is a biased estimator of $h$. So, $\E[\hat{h}] = \E[g(f)] \ge g(\E[f]) = g(f^*) = h$.
%\end{proof}

We now give a proof that our \minfsketch estimator offers a better
approximation. Recall that $\sigma$ indicates the maximum number of non-zero
attributes in any data vector, and is often much small compared to the their
dimension, $n$. Surprisingly, our results are independent of $n$.
\begin{lem}\label{lem:medianlemma}
    Let $h^m$ denote the median of the estimates of Hamming distances obtained
    from $t$ independent
    \fsketch vectors of dimension $4\sigma$ and let $h$ denote the actual Hamming distance. Then,
    $$\Pr\big[|h^m - h| \ge 18\sqrt{\sigma}\big] \le \delta$$
    for any desired $\delta \in (0,1)$ if we use $t \ge 48\ln \tfrac{1}{\delta}$.
\end{lem}

\begin{proof}
    We start by using Lemma~\ref{lem:hconcentrationlemma} with
    $p=3$ and error ($\delta$ in the lemma statement) = $\tfrac{1}{4}$. Let
    $\hat{h}^i$ denote the $k$-th estimate. From the lemma we get that
    $$\Pr\big[ |\hat{h}^i - h| \ge 18\sqrt{\sigma} \big] \le \tfrac{1}{4}$$

    Define indicator random variables $W_1 \ldots W_t$ as $W_i=1$ iff
    $|\hat{h}^i - h| \ge 18\sqrt{\sigma}$. We immediately have $\Pr[W_i] \le
    \tfrac{1}{4}$. 
    Notice that $W_i=1$ can also be interpreted to indicate the event
    $h-18\sqrt{\sigma} \le \hat{h}^i \le h + 18 \sqrt{\sigma}$.
    Now, $h^m$ is the median of $\{\hat{h}^1, \hat{h}^2, \ldots
    \hat{h}^t\}$, and so, $h^m$ falls outside the range $[h-18\sqrt{\sigma},
    h+18\sqrt{\sigma}]$ only if more than half of the estimates fall outside this
    range., i.e., if $\sum_{i=1}^t W_i > t/2$. Since $\E[\sum_i W_i] \le t/4$, the probability of this event is
    easily bounded by $\exp{(-(\tfrac{1}{2}^2
    \tfrac{t}{4}/3))} = e^{-t/48} \le \delta$ using Chernoff's bound.
\end{proof}


\subsection{Choice of statistics in \minfsketch}\label{subsec:box_plot_median}
We conducted an experiment to decide whether to take median, mean or minimum of $k$ \fsketch estimates in the \minfsketch algorithm. We randomly sampled a pair of points and estimated the Hamming distance from its low-dimensional representation obtained from  \texttt{FSketch}. We repeated this $10$ times over different random mappings and computed the median, mean, and minimum of those $10$ different estimates. We further repeat this experiment $10$ times and generate a box-plot of the readings which is presented in Figure~\ref{fig:box_plot_median_mean_min}. We observe that median has the lowest variance and also closely estimates the actual Hamming distance between the pair of points. 




\section{Dimensionality reduction algorithms}
\begin{table*}[h]
    \caption{A tabular summary of popular dimensionality reduction algorithms. Linear dimensionality reduction algorithms are those whose features in reduced dimension are linear combinations of the input features, and the others are known as non-linear algorithms. Supervised dimensionality reduction methods are those that require labelled datasets for dimensionality reduction. \label{table:dim-red}
}
  \begin{center}
        %\begin{tabular}{ |c | c | c |c | c | c | c|}
        \begin{tabular} { | p {0.55cm} | p {2 cm} | p {4 cm} | p {2 cm} | p {2.5 cm} | p {2 cm} | p {2 cm} |}
        
        \hline
        S. No.
&Data type of input vectors &Objective/ Properties &Data type of sketch vectors &Result &Supervised or Unsupervised&Type of dimensionality reduction\\
\hline
1 &Real-valued vectors &Approximating pairwise euclidean distance, inner product
&Real-valued vectors&JL-lemma~\cite{JL83}&Unsupervised&Linear \\
\hline
2&Real-valued vectors&
Approximating pairwise euclidean distance, inner product
&Real-valued vectors&Feature Hashing~\cite{WeinbergerDLSA09}&Unsupervised&Linear \\
\hline
3&Real-valued vectors&Approximating pairwise cosine or angular similarity &Binary vectors&SimHash~\cite{simhash}&Unsupervised&Non-Linear\\
\hline
4&Real-valued vectors&Approximating pairwise $\ell_p$ norm for $p \in (0, 2]$&Real-valued vectors&$p$-stable random projection (SSD)~\cite{Indyk06}
&Unsupervised &Linear\\
\hline
5&Sets&Approximating pairwise Jaccard similarity &Integer valued vectors
&MinHash~\cite{BroderCFM98}&Unsupervised&Non-linear\\
\hline
6&Sparse binary vectors
&Approximating pairwise Hamming distance, Inner product, Jaccard and Cosine similarity 
&Binary vectors&BinSketch~\cite{ICDM}&Unsupervised&Non-linear\\
\hline
7&Real-valued vectors&Minimize  the variance in low dimension 
&Real-valued vectors&Principal Component Analysis (PCA)&Unsupervised
&Linear\\
\hline
8&Real-valued vectors (labelled input)&Maximizes class separability in the reduced dimensional space&Real-valued vectors&Linear Discriminant Analysis~\cite{FLDA} &Supervised&Linear\\
\hline
9&Real-valued vectors&
Embedding high-dimensional data for visualization in a low-dimensional space of two or three dimensions&
Real-valued vectors &$t$-SNE~\cite{vanDerMaaten2008}
&Unsupervised&Non-linear\\
\hline
10&Real-valued vectors &Minimize the reconstruction error 
&Real-valued vectors&Auto-encoder~\cite{10.5555/2987189.2987190}
&Unsupervised&Non-linear\\
\hline
11&Real-valued vectors
&Extracting nonlinear structures in low-dimension via Kernel function
&Real-valued vectors&Kernel-PCA~\cite{NIPS1998_226d1f15}&Unsupervised&Non-linear\\
\hline
 12 &Real-valued vectors &
Factorize input matrix into two small size non-negative matrices &Real-valued vectors &Non-negative matrix factorization (NNMF)~\cite{NNMF}
 &Unsupervised &Linear\\
\hline
13 &Real-valued
vectors &Compute a \textit{quasi-isometric}  low-dimensional embedding &Real-valued
vectors&Isomap~\cite{tenenbaum_global_2000}
&Unsupervised&Non-linear\\
\hline
14&
Real-valued
vectors
&Preserves the \textit{topological structure} of the data
&Real-valued
vectors
&Self-organizing map
~\cite{kohonen-self-organized-formation-1982}
&Unsupervised&Non-linear\\
\hline
\end{tabular}
    \end{center}
    \end{table*}




% that's all folks
\end{document}


