%%%%%%%%%%%%%%%%%%%%%%%%%%%%%%%%%%%%%%%%%%%%
\documentclass[aps,prl,showpacs,floatfix,twocolumn,superscriptaddress,longbibliography]{revtex4-1}

\usepackage[pdftex]{graphicx}
\usepackage{color}
\usepackage{hyperref}
\usepackage{verbatim}
\usepackage{soul}
\usepackage{glossaries}
\usepackage{upgreek}
\usepackage{todonotes}
%\usepackage{sidecap}

\def\mathbi#1{\ensuremath{\textbf{\em #1}}}
\def\Q{\mathbi{Q}}
\def\QCO{\ensuremath{\mathbi{Q}_{\mathrm{CO}}}}
\def\QSO{\ensuremath{\mathbi{Q}_{\mathrm{SO}}}}
\def\TCO{\ensuremath{T_{\mathrm{CO}}}}
\def\TSO{\ensuremath{T_{\mathrm{SO}}}}
\def\TC{\ensuremath{T_{\mathrm{cycle}}}}
\def\um{\ensuremath{\upmu\text{m}}}
\def\LSNO{\ensuremath{\mathrm{La}_{2-x}\mathrm{Sr}_x\mathrm{NiO}_{4+\delta}}}

\newacronym{RIXS}{RIXS}{Resonant inelastic X-ray scattering}
\newacronym{XAS}{XAS}{X-ray absorption spectrum}
\newacronym{XPCS}{XPCS}{X-ray photon correlation spectroscopy}
\newacronym{LSNO}{LSNO}{La$_{2-x}$Sr$_{x}$NiO$_{4+\delta}$}
\newacronym{LSCO}{LSCO}{La$_{2-x}$Sr$_{x}$CuO$_{4}$}
\newacronym{LBCO}{LBCO}{La$_{2-x}$Ba$_{x}$CuO$_{4}$}
\newacronym{CO}{CO}{charge order}
\newacronym{SO}{SO}{spin order}
\newacronym{HWHM}{HWHM}{half-width at half-maximum}
\newacronym{CSX}{CSX}{Coherent Soft X-Ray}
\newacronym{HTT}{HTT}{high temperature tetragonal}
\newacronym{LTO}{LTO}{low temperature orthorhombic}
\newacronym{EPC}{EPC}{electron-phonon coupling}

\begin{document}

\title{Charge Condensation and Lattice Coupling Drives Stripe Formation in Nickelates}

\author{Y. Shen}\email[]{yshen@bnl.gov}
\affiliation{Condensed Matter Physics and Materials Science Department, Brookhaven National Laboratory, Upton, New York 11973, USA}

\author{G. Fabbris}
\affiliation{Condensed Matter Physics and Materials Science Department, Brookhaven National Laboratory, Upton, New York 11973, USA}
\affiliation{Advanced Photon Source, Argonne National Laboratory, Lemont, Illinois 60439, USA}

\author{H. Miao}
\affiliation{Condensed Matter Physics and Materials Science Department, Brookhaven National Laboratory, Upton, New York 11973, USA}
\affiliation{Material Science and Technology Division, Oak Ridge National Laboratory, Oak Ridge, Tennessee 37830, USA}

\author{Y. Cao}
\affiliation{Condensed Matter Physics and Materials Science Department, Brookhaven National Laboratory, Upton, New York 11973, USA}
\affiliation{Materials Science Division, Argonne National Laboratory, Lemont, Illinois 60439, USA}

\author{D. Meyers}
\affiliation{Condensed Matter Physics and Materials Science Department, Brookhaven National Laboratory, Upton, New York 11973, USA}
\affiliation{Department of Physics, Oklahoma State University, Stillwater, Oklahoma 74078, USA}

\author{D. G. Mazzone}
\affiliation{Condensed Matter Physics and Materials Science Department, Brookhaven National Laboratory, Upton, New York 11973, USA}
\affiliation{Laboratory for Neutron Scattering and Imaging, Paul Scherrer Institut, CH-5232 Villigen, Switzerland}

\author{T. Assefa}
\affiliation{Condensed Matter Physics and Materials Science Department, Brookhaven National Laboratory, Upton, New York 11973, USA}

\author{X. M. Chen}
\affiliation{Condensed Matter Physics and Materials Science Department, Brookhaven National Laboratory, Upton, New York 11973, USA}

\author{K. Kisslinger}
\affiliation{Center for Functional Nanomaterials, Brookhaven National Laboratory, Upton, New York 11973, USA}

\author{D. Prabhakaran}
\author{A. T. Boothroyd}
\affiliation{Department of Physics, University of Oxford, Clarendon Laboratory, Oxford, OX1 3PU, United Kingdom}

\author{J. M. Tranquada}
\affiliation{Condensed Matter Physics and Materials Science Department, Brookhaven National Laboratory, Upton, New York 11973, USA}

\author{W. Hu}
\author{A. M. Barbour}
\author{S. B. Wilkins}
\author{C. Mazzoli}
\affiliation{National Synchrotron Light Source II, Brookhaven National Laboratory, Upton, New York 11973, USA}

\author{I. K. Robinson}
\author{M. P. M. Dean}\email[]{mdean@bnl.gov}
\affiliation{Condensed Matter Physics and Materials Science Department, Brookhaven National Laboratory, Upton, New York 11973, USA}

\date{\today}

\begin{abstract}
Revealing the predominant driving force behind symmetry breaking in correlated materials is sometimes a formidable task due to the intertwined nature of different degrees of freedom. This is the case for La$_{2-x}$Sr$_{x}$NiO$_{4+\delta}$ in which coupled incommensurate charge and spin stripes form at low temperatures. Here, we use resonant X-ray photon correlation spectroscopy to study the temporal stability and domain memory of the charge and spin stripes in La$_{2-x}$Sr$_{x}$NiO$_{4+\delta}$. Although spin stripes are more spatially correlated, charge stripes maintain a better temporal stability against temperature change. More intriguingly, charge order shows robust domain memory with thermal cycling up to 250~K, far above the ordering temperature. These results demonstrate the pinning of charge stripes to the lattice and that charge condensation is the predominant factor in the formation of stripe orders in nickelates.
\end{abstract}

\maketitle

Emergent phenomena in strongly correlated materials arise due to multifarious interactions among charge, spin and lattice degrees of freedom. Such complexity hampers the ability to understand their remarkable states and realize new functionalities \cite{Chumak2015spintronics}. Identifying dominant interaction is, however, challenging as different interactions act simultaneously and can yield complex ground states with more than one form of order \cite{Fradkin2015}. A representative phenomenon of this type is the electronic stripes that appear in various strongly correlated materials \cite{Mori1998LMO, Lee2006review, Ulbrich2012review, Comin2016review}. These effects have been considered extensively in cuprate high-temperature superconductors, which host charge and sometimes spin stripe order, typically with a simple factor-of-two relationship between the charge and spin incommensurabilities  \cite{Tranquada1995cuprate, Mook2000YBCO, Hoffman2002Bi2212}. Nickelates also host both superconductivity and stripe order \cite{Tranquada1994LNO, Tranquada1996LSNO, Li2019Nisuperconductor}, but no system has yet been shown to simultaneously host both orders. The existence of stripe order in La$_4$Ni$_3$O$_8$, which appears rather similar to superconducting Nd$_{1-x}$Sr$_x$NiO$_2$ \cite{Zhang2016LaNiO438, Zhang2017Ni438, Zhang2019LaNiO438, Lin2020strong}, does, however, support the likely proximity of stripe order and superconductivity. While static stripe order appears to suppress bulk 3D superconductivity, some researchers have suggested that stripe fluctuations may act to promote superconductivity \cite{Emery1997spin, Kivelson1998electronic, Agterberg2020PDW_review}. Therefore, understanding the driving forces behind charge and spin stripe formation and dynamics in strongly correlated materials has attracted considerable attention and may be crucial to understanding unconventional superconductivity. Stripe formation has been studied in the past through detailed measurements of stripe transition temperatures and correlation lengths \cite{Chen1993LSNO, Cheong1994LSNO, Lee1997LSNO, Yoshizawa2000LSNO, Lee2001LSNO, Kajimoto2001LSNO, Ghazi2004LSNO, Freeman2004LSNO, Raczkowski2006LSNO} and associated Landau model analysis \cite{Wochner1998LSNO, Zachar1998Landau}. The problem has also been addressed via model Hamiltonian analysis that suggested that lattice coupling might be crucial to stabilize stripes \cite{Zaanen1994freezing, Hotta2004LSNO}. The implementation of resonant \gls{XPCS} at modern low-emittance synchrotron sources opens new routes to directly probe stripe formation and dynamics \cite{Chen2016LBCO, Thampy2017LBCO, Chen2019LBCO, Ricci2019LSNO}. 

Herein, we report the first resonant \gls{XPCS} experiment to simultaneously probe \gls{CO}, \gls{SO} and lattice coupling in a stripe-ordered material, focusing on the prototypical material \gls{LSNO} with $x=0.225$ $\delta=0.07$. Although \gls{SO} is more correlated and stable at 70~K, \gls{CO} is more robust in temporal stability against temperature changes, which we attribute to \gls{EPC}. This is further supported by our discovery that the \gls{CO} domains are effectively pinned to the lattice and the corresponding speckle patterns remain highly reproducible with thermal cycling up to 250~K, well above the transition temperature {\TCO}. \gls{SO}, however, is not directly coupled to the lattice and loses its domain memory once the sample is warmed across the magnetic transition temperature {\TSO}. These results imply that charge condensation, and its coupling to the lattice and disorder, is the driving force behind stripe ordering.

\begin{figure}[t]
\includegraphics{figure1.pdf}
\caption{Experimental configuration and \acrfull{CO} and \acrfull{SO} superlattice peaks. (a) The instrumental setup for the measurements at CSX. The X-ray beam is set to the Ni $L_3$-edge energy and tuned in order to maximize the strength of the \gls{CO} and \gls{SO} intensity \cite{supp}. It then propagates through the pinhole and is scattered by the \acrfull{LSNO} sample onto the detector. For the domain memory study, a 0.5 {\um} thick Pt mask was deposited on the sample \cite{supp}. (b) An optical micrograph of the Pt mask on the (110) surface of \gls{LSNO} single crystal. (c) Temperature dependence of the correlation lengths along [$H$, $H$, 0] and [0, 0, $L$] directions. The correlation length is defined as $\xi = d/\textrm{HWHM}$ where $\textrm{HWHM}$ stands for half-width at half-maximum in reciprocal lattice units and $d$ is the unit cell size in the appropriate direction \cite{incommensurability}. (d) Temperature dependence of the peak heights evaluated from fitting of the \gls{CO} and \gls{SO} superlattice peaks, which are normalized according to their values at 60~K. The signals were fitted with a three-dimensional Lorentzian function. (e) Incommensurability defined by the peak position of \gls{CO} {\Q} vector as a function of temperature. The shaded areas indicate the onset temperature range for \gls{CO} and \gls{SO}.}
\end{figure}

X-ray measurements were carried out at the \gls{CSX} 23-ID-1 beamline at the National Syncrotron Light Source II with X-ray energy tuned to the Ni $L_3$-edge (Fig.~1a). The \gls{LSNO} single crystal was synthesized by the floating-zone method with a Sr concentration of $x = 0.225$ \cite{Prabhakaran2002growth}. As shown later, the \gls{CO} incommensurability is $\epsilon \approx 0.27$, larger than $x$, which is likely related to oxygen doping since $\delta = 0.07$ \cite{Freeman2006magnetization}. The sample's surface normal was close to the [$H$, $H$, 0] direction. Thus, we made ($H$, $H$, $L$) the scattering plane and focused on peaks with ${\QCO} = (\epsilon, \epsilon, 1)$ and ${\QSO} = (1/2-\epsilon/2, 1/2-\epsilon/2, 0)$ \cite{supp}. The reciprocal lattice units (r.l.u.) is defined in terms of {\Q} = ($H$, $K$, $L$) = $(2\pi/a, 2\pi/b, 2\pi/c)$ within the space group I4/mmm and $a = b = 3.84$~{\AA}, $c = 12.65$~{\AA}. For the domain memory measurements, we used a 0.5~{\um} thick Pt mask, which had been deposited on the sample in order to reproducibly illuminate the same sample volume independent of possible thermal drifts in the sample position (Fig.~1b) \cite{supp}.

We start by characterizing the superlattice peaks corresponding to \gls{CO} and \gls{SO} at different temperatures using standard resonant X-ray diffraction. With decreasing temperature, the peak heights first increase substantially through the transition temperatures along with enhanced correlation lengths for both \gls{CO} and \gls{SO} (Fig.~1c, d). Below $\sim$70~K, the peak heights drop and the spatial correlations are relaxed, consistent with previous reports \cite{Hatton2002LSNO, Ghazi2004LSNO, Schlappa2009LSNO}. The reason for this is not uniquely determined, but it may be connected to a spin reorientation at lower temperature \cite{Freeman2004LSNO} or the influence of spin exchange interactions \cite{Ghazi2004LSNO}. Throughout the temperature range, the correlation lengths along [$H$, $H$, 0] direction are much larger than those along [0, 0, $L$] and \gls{SO} possesses a larger correlation length than \gls{CO} (Fig.~1c). Due to the critical fluctuations and short-range correlations near the phase transitions, the onset temperatures, {\TCO} and {\TSO}, are not uniquely defined. We estimate them both to occur between 96 and 114~K. Regarding the incommensurability, the inter-site Coulomb repulsion tends to stabilize $\epsilon$ equal to the hole concentration \cite{Sachan1995LSNO}, while the commensurability effect optimizes stripe formation at $x = 1/3$. The actual incommensurability is a compromise of these two factors \cite{Yoshizawa2000LSNO}. With increasing temperature, thermal fluctuations are expected to start to outcompete Coulomb repulsion \cite{Hatton2002LSNO, Ishizaka2004LSNO, Miao2019LBCO}, driving the incommensurability closer to 1/3 at higher temperature (Fig.~1e).

\begin{figure}[t]
\includegraphics{figure2.pdf}
\caption{Speckle patterns of \gls{CO} and \gls{SO}. (a),(b) Representative detector images around the \gls{CO} and \gls{SO} superlattice peaks measured with a 10~{\um} pinhole. The white pixels arise from the beamstop or detector errors and are omitted from the data. (c),(d) Line cuts through the horizontal red dashed lines in (a) and (b). The envelope of the peak is estimated by smoothing and fitting processes that are shown as red and orange lines, respectively. The black dashed lines are uniform fluorescent background evaluated from fittings.}
\end{figure}

To elucidate the temporal stabilities of \gls{CO} and \gls{SO}, we employ \gls{XPCS} to study the domain distribution and its fluctuations. In \gls{XPCS}, the coherent photons scattered by different domains interfere with each other, leading to a complex ``speckle'' pattern modulated by the usual diffraction lineshape \cite{Brauer1995XPCS, Shpyrko2014XPCS, Chen2016LBCO, Thampy2017LBCO, Ricci2019LSNO, Lee2021Dimensionality}. Figure 2a, b shows the representative speckles of the \gls{CO} and \gls{SO} superlattice peaks at 70~K. The shape of the peak envelope is determined by the spatial correlations and instrument geometry. In particular, the horizontal width of the \gls{SO} peak is mainly determined by the correlations along the [-1, 1, 0] direction while the vertical width is dominated by $c$ axis correlations, elongating the envelope vertically. For the \gls{CO} peak, the vertical width has less contribution from $c$ axis correlations so that the envelope appears more isotropic. Meanwhile, the distribution of the underlying stripe domains is encoded in the positions of the speckles \cite{Chen2019LBCO}, and the shape of the speckles is determined by the Fourier transform of the beam footprint projected onto the detector. The non-zero $L$ component of the \gls{CO} peak makes the footprint of the beam more anisotropic. To show the speckle modulation more clearly, we present in Fig.~2c, d the line cuts through the red dashed lines in Fig.~2a, b. The peak envelope is estimated by two independent methods: smoothing with the Savitzky-Golay filter and fitting with a squared Lorentzian function. The sharp speckle modulation observed here indicates that the fluctuations for \gls{CO} and \gls{SO} are slower than the time windows of the measurements, which is 1~s at 70~K \cite{supp}. Otherwise the contrast of the interference patterns will be significantly reduced \cite{Chen2016LBCO}.

\begin{figure}[t]
\includegraphics{figure3.pdf}
\caption{Temporal stability of \gls{CO} and \gls{SO}. (a),(b) Time dependence of the intermediate scattering functions at different temperatures. The solid lines are guides to the eye. (c),(d) The scattering functions after certain time delays.}
\end{figure}

\begin{figure*}[t]
\includegraphics{figure4.pdf}
\caption{Domain memory in \gls{CO} but not \gls{SO}. (a),(b) Representative speckle images before and after thermal cyclings which are indicated by the curved arrows. The open circles stand for the cycling temperatures, {\TC}. For each measurement, we collected images at 70~K, changed the temperature to {\TC} and waited for 10 minutes. Then the sample was cooled back to 70~K and equilibrated for 30 minutes before collecting another image. For both the heating and cooling processes, the temperature ramping rate was fixed to 4~K/min. The white bar in the first speckle image indicates $10^{-3}$ \AA{}$^{-1}$. (c) Temperature dependence of the normalized speckle cross-correlation function, $\xi_{\mathrm{CC}}$. The solid and dashed lines are guides to the eye. The shaded area indicates the range of \gls{CO} and \gls{SO} transition temperatures.}
\end{figure*}

In order to quantify the fluctuation timescale, we measure the time dependence of the speckle patterns and calculate the normalized one-time correlation function \cite{Chen2016LBCO}
\begin{align}
g_2 (\tau) = \frac{\langle I(t)I(t+\tau)\rangle}{\langle I(t)\rangle^2} = 1 + \beta|F(\tau)|^2,
\end{align}
where $I$ represents the total intensity including background, $\tau$ is the lag time and $\langle\ldots\rangle$ stands for the time and ensemble average. The time-dependent evolution can be extracted from the intermediate scattering function, $|F(\tau)|$, which describes the correlation of the speckle patterns separated by a certain time delay. In a statically ordered system, $|F(\tau)|$ will remain unchanged while speckle dynamics causes it to drop as a function of time delay. Distinct from \gls{LBCO}, in which the \gls{CO} is static over a timescale of at least two hours \cite{Chen2016LBCO, Thampy2017LBCO}, $|F(\tau)|$ in \gls{LSNO} decays after several minutes for both \gls{CO} and \gls{SO}, indicating charge and spin dynamics (Fig.~3). Moreover, we find that \gls{CO} and \gls{SO} are both most stable around 70~K when they have longest correlation lengths, but \gls{SO} is more stable than \gls{CO} at 70~K. Although stripes involve a co-modulation of both charge and spin \cite{Zachar1998Landau}, we observe that these have different thermal evolution. As temperature is driven away from 70~K, the temporal stability for \gls{SO} decreases faster, indicating that \gls{SO} is less stable against temperature changes. A qualitatively, but not quantitatively similar trend in \gls{SO} was reported recently in Ref.~\cite{Ricci2019LSNO}. The longer timescales observed here may reflect sample discrimination in strontium and oxygen compositions or improved coherent flux and stability at \gls{CSX} compared to the Advanced Light Source.

From simple energetic considerations, if an order is less temporally stable and has shorter correlation lengths one would expect it to be more fragile to thermal disturbance. The unexpected robustness of \gls{CO} against temperature changes indicates that \gls{CO} is coupled to other degrees of freedom which constrain the \gls{CO} domains during/after the charge condensation (Fig.~3). Such hypotheses can be examined more deeply in term of domain pinning memory effects. Since the speckle positions are primarily determined by the positions of the ordering domains, the comparison of speckle patterns collected at 70~K before and after cycling the sample temperature to {\TC} can evaluate whether the domain distributions are reproduced \cite{Chen2019LBCO}. The usage of Pt mask further ensures that the illuminated sample volume is fixed throughout the thermal cycling (Fig.~1b). It turns out that the speckle patterns of \gls{CO} are rather similar with {\TC} up to 250~K, well above {\TCO} (Fig.~4a). The \gls{SO} speckles, however, change their positions once {\TC} crosses {\TSO} ($\sim 100$~K) (Fig.~4b). This effect can be quantified by calculating the normalized cross-correlation function $\xi_{\mathrm{CC}}$ which describes the similarity between two speckle patterns \cite{Chen2019LBCO, supp}. $\xi_{\mathrm{CC}}$ approaches zero when the two speckle images are different while two identical images will give $\xi_{\mathrm{CC}}$ of one. Correspondingly, we calculate $\xi_{\mathrm{CC}}$ for both \gls{CO} and \gls{SO} speckle patterns with different {\TC} (Fig.~4c). The results again show that \gls{CO} domain distributions are essentially unchanged after thermal cycling to a temperature far above {\TCO} while \gls{SO} speckle pattern loses reproducibility after the system is driven into the disordered state.

% Discussion Section I: Structural disorder

The domain memory effect of \gls{CO} is caused by coupling to the host lattice. Local potentials arising from structural disorder induced, for example, by Sr doping, structural domain boundaries or octahedral tilts, provide nucleation centers for the \gls{CO} domains and effectively pin the domains during stripe condensation. Since the average lattice structure of \gls{LSNO} has translational symmetry over a lengthscale smaller than \gls{CO} wavelength it cannot, in self, pin the \gls{CO} domains into reproducible locations. In charge-ordered cuprate \gls{LBCO}, the speckle pattern of \gls{CO} domains loses memory after the sample is heated across the transition temperature from the \gls{LTO} phase into the \gls{HTT} phase \cite{Chen2019LBCO}. Thus, it is expected that the pinning landscape for \gls{CO} in \gls{LBCO} is constrained by twin boundaries created by the \gls{LTO} structural distortion. In \gls{LSNO}, the lattice remains in the \gls{HTT} phase and no long-range \gls{LTO} distortion is observed \cite{Hucker2004PhaseDiagram}. However, short-range stripe-related distortions have been reported to persist up to high temperatures \cite{Simon2013LSNO}. It is possible that either these distortions, or local defects due to Sr-related doping disorder, determine the pinning landscape of \gls{LSNO} in a similar manner.

% Discussion Section II: Electron-phonon coupling

The pinning effect of \gls{CO} to the structural disorder also evinces the relevance of \gls{EPC} in nickelates, which has been illustrated by the discovery of phonon anomalies and nematic behaviors in \gls{LSNO} \cite{Pashkevich2000stripe, Tranquada2002LSNO, Merritt2020phonon, Zhong2017LSNO}. It has been argued theoretically that without \gls{EPC} \gls{CO} will remain dynamic and not order \cite{Hotta2004LSNO}. For structure-driven \gls*{CO}, phonons soften to zero energy and drag the valence charge along with it to form spatial modulations. Here, however, phonons are softened by a maximum of 20\% \cite{Tranquada2002LSNO}, and charge stripes are formed to reduce Coulomb interactions. \gls*{EPC} helps pin pre-formed charge stripes according to the lattice symmetry, promoting the static \gls{CO}. The presence of \gls{EPC} further couples the \gls{CO} domains to structural disorder, which strengthens the \gls{CO} against thermal fluctuations. Consequently, when \gls{CO} and \gls{SO} lose correlations progressively upon heating or cooling away from 70~K, the fluctuations of \gls{CO} speckles increase more slowly (Fig.~3).

% Discussion Section III: Spin order
\gls{SO} behaves in a different way. During the formation of \gls{SO}, the spins can align either parallel or antiparallel to their quantization axis. This would disrupt the reproducibility of \gls{SO} speckles after thermal cycling across {\TSO} even if the domain walls are in the same place (Fig.~4). Moreover, the rotational degree of freedom provides an additional fluctuation channel to the ordered spins, facilitating the loss of \gls{SO} stability when driven away from 70~K (Fig.~3). This is in line with the observation of spin reorientation in \gls{LSNO} at low temperatures \cite{Lee2001LSNO, Freeman2004LSNO, Freeman2006magnetization}.

% Discussion Section IV: Nickelates and cuprates

The robustness of \gls{CO} stability and its pinning to the lattice demonstrate that the stripe order in \gls{LSNO} is charge driven. This directly verifies prior theoretical predictions based on Landau theory of coupled charge and spin order parameters \cite{Zachar1998Landau} and may reflect that stripe-order is charge-driven in general. Our approach will be extendable to other materials and even to other degrees of freedom such as orbital order, bringing a powerful means to disentangle the formation mechanisms of intertwined ground states.

\begin{acknowledgements}
This material is based upon work supported by the U.S. Department of Energy (DOE), Office of Basic Energy Sciences. Work at Brookhaven National Laboratory was supported by the U.S. DOE, Office of Science, Office of Basic Energy Sciences, under Contract No.~DE-SC0012704. The work at Argonne National Laboratory was supported by the U.S. Department of Energy, Office of Basic Energy Sciences, under Contract No. DE-AC0206CH11357.  D.~G.~M.\ acknowledges funding from the Swiss National Science Foundation, Fellowship No.\ P2EZP2\_175092. This research used resources at the. 23-ID-1 beamline of the National Synchrotron Light Source, a U.S. DOE Office of Science User Facility operated for the DOE Office of Science by Brookhaven National Laboratory under Contract No. DE-AC02-98CH10886.
\end{acknowledgements}

\bibliography{refs}
\end{document}

