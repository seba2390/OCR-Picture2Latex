\section{Correlation Confidence Intervals}
\label{sec:ci}
Although the strength of the relationship between $\mathcal{X}$ and $\mathcal{Z}$ on one dataset is quantified by the correlation levels $\rsys$ and $\rsum$, each $r$ is only a point estimate of the true correlation of the metrics, denoted $\rho$, on inputs and systems distributed similarly to those in $\mathcal{D}$ and in $\mathcal{S}$.
Although we cannot directly calculate $\rho$, it is possible to estimate it through a CI.

\subsection{The Fisher Transformation}
\label{sec:fisher}
The standard method for calculating a CI for a correlation is the Fisher transformation \citep{Fisher92}.
The transformation maps a correlation coefficient to a normal distribution, calculates the CI on the normal curve, and applies the reverse transformation to obtain the upper and lower bounds:
\begin{align*}
    z_r &= \textrm{arctanh}(r) \\
    r_u, r_\ell &= \textrm{tanh}\left(z_r \pm  z_{\alpha/2} \cdot c\; / \sqrt{n-b}\right)
\end{align*}
where $r$ is the correlation coefficient, $n$ is the number of observations, $z_{\alpha/2}$ is the critical value of a normal distribution, and $b$ and $c$ are constants.\footnote{
$b=3, 3, 4$ and $c=1, \sqrt{1+r^2/2}, \sqrt{.437}$ for Pearson, Spearman, and Kendall, respectively \citep{BonettWr00}.
}

Applying the Fisher transformation to calculate CIs for $\rhosys$ and $\rhosum$ is potentially problematic.
First, it assumes that the input variables are normally distributed \citep{BonettWr00}.
The metrics' scores and human annotations on the datasets that we experiment with are, in general, not normally distributed (see Appendix~\ref{appendix:normality}).
Thus, this assumption is violated, and we expect this is the case for other summarization datasets as well.
Second, it is not clear whether the transformation should be applied to the summary-level correlation since its final value is an average of correlations, which is not strictly a correlation.\footnote{
Correlation coefficients cannot be averaged because they are not additive in the arithmetic sense, however it is standard practice in summarization.
}

\subsection{Bootstrapping}
\label{sec:ci_bootstrapping}
A popular nonparametric method of calculating a CI is bootstrapping \citep{EfronTi93}.
Bootstrapping is a procedure that estimates the distribution of a test statistic by repeatedly sampling with replacement from the original dataset and calculating the test statistic on each sample.
Unlike the Fisher transformation, bootstrapping is a very flexible procedure that does not assume the data is normally distributed nor that the test statistic is a correlation, making it appropriate for summarization.

However, it is not clear how to perform bootstrap sampling for correlation levels.
Consider a more standard bootstrapped CI calculation for the mean accuracy of a question-answering model on a dataset with $k$ instances.
Since the mean accuracy is a function of the $k$ individual correct/incorrect labels, each bootstrap sample can be constructed by sampling with replacement from the original $k$ instances $k$ times.
In contrast, the correlation levels are functions of the matrices $X$ and $Z$, so each bootstrap sample should also be a pair of matrices of the same size that are sampled from the original data.

There are at least three potential methods for sampling the matrices:
\begin{enumerate}
    \item \textsc{Boot-Systems:} Randomly sample with replacement $N$ systems from $\mathcal{S}$, then select the sampled system scores for all of the inputs.

    \item \textsc{Boot-Inputs:} Randomly sample with replacement $M$ inputs from $\mathcal{D}$, then select all of the system scores for the sampled inputs.
    
    \item \textsc{Boot-Both:} Randomly sample with replacement $M$ inputs from $\mathcal{D}$ and $N$ systems from $\mathcal{S}$, then select the sampled system scores for the sampled inputs.
\end{enumerate}
Once the samples are taken, the corresponding values from $X$ and $Z$ are selected to create the sampled matrices.
An illustration of each method is shown in Figure~\ref{fig:sampling}.

\section{Sample-Based PCA}
\label{sec:sampling}

To tackle workload-aware DR, we demonstrate how sample-based PCA can bridge the performance , but that the number of samples required varies per dataset.
Finally, we show how dynamically increasing the sampling rate can help identify how much to sample a given dataset, providing a foundation for workload-aware DR.

\begin{figure}
\includegraphics[width=\linewidth]{figs/progressive.pdf}
\caption[]{ Improvement in representation size for  $TLB = 0.80$ across three datasets. Higher sampling rates improve quality until reaching a state equivalent to running PCA over the full dataset ("convergence")}
\label{fig:progressive}
\end{figure}

\begin{comment}
\subsection{PCA Speed vs. Quality}

While improved quality provides faster repeated query execution, the cost of DR via PCA dominates this speedup, encouraging the use of faster, lower-quality alternatives~\cite{keogh-study}. 

To briefly quantify this trade-off, we augment a widely-cited time series similarity search DR study from VLDB 2008~\cite{keogh-study} by evaluating PCA---which the authors did not benchmark due to it being ``untenable for large data sets." 
We compare PCA via SVD to baseline techniques based on runtime and DR performance with respect to $TLB$ over the largest datasets from~\cite{keogh-study}. 
We use their two fastest methods as our baselines as they show the remainder exhibited ``very little difference'': Fast Fourier Transform (FFT) and Piecewise Aggregate Approximation (PAA).

\minihead{TLB Performance Comparison}
We compute the minimum dimensionality achieved by each technique subject to a $TLB$ constraint. 
On average, PCA provides bases that are $2.3\times$ (up to $3.9\times$) and $3.7\times$ (up to $26\times$)  smaller than PAA and FFT for $TLB = 0.75$, and $2.9\times$ (up to $8.3\times$) and $1.8\times$ (up to $5.1\times$) smaller for $TLB = 0.99$.
While the margin between PCA and alternatives is dataset-dependent, PCA almost always preserves $TLB$ with a lower dimensional representation.

%\section{Additional End-to-End Plots}
%\input{endendplots}

\minihead{Runtime Performance Comparison} 
PCA implemented via out-of-the-box SVD is on average over \red{$26\times$ (up to $56\times$)} slower than PAA and over \red{$4.6\times$ (up to $9.7\times$)} times slower than FFT when computing the smallest $TLB$-preserving basis.
%This substantiates the observation that classic PCA is incredibly slow compared to alternatives~\cite{keogh-study}. 

\end{comment}

\subsection{Incremental, Progressive Sampling}
To bridge this performance-runtime gap, we turn to data sampling. 
Many real-world \red{datasets} are intrinsically low-dimensional, as evidenced by their rapid falloff in their eigenvalue spectrum.
A data sample thus captures much of the dataset's ``interesting'' behavior, so fitting models over data samples generalize well. 
We verify this by varying the target $TLB$ and examining the minimum number of uniformly selected samples required to obtain a $TLB$-preserving transform with output dimension $k$ equal to input dimension $\dvar$.

On average, a sample of under $0.64\%$ $(\text{up to } 5.5\%)$ of the input is sufficient for $TLB = 0.75$, and under $4.2\%$ $(\text{up to } 38.6\%)$ is sufficient for $TLB=0.99$.  
If this sample rate is known, we obtain up to \red{$91\times$ speedup} compared to a na\"ive implementation of PCA via SVD---with no algorithmic improvement. 

However, this benefit is dataset-dependent, and unknown a priori.
We thus turn to progressive sampling (gradually increasing the sample size) to identify how large a sample suffices.
Figure~\ref{fig:progressive} shows how the dimensionality required to attain a given $TLB$ changes when we vary dataset and proportion of data sampled.
Increasing the number of samples provides lower dimensional transformations for the same quality.
However, this decrease in dimension plateaus as the number of samples increases.
Thus, while progressive sampling would allow us to tune the amount of time spent on DR, we must determine when the downstream value of decreased dimension is overpowered by the cost of additional DR---that is, whether to sample to convergence (evaluated in \S\ref{subsec:lesion}) or terminate early (e.g., at $0.3$ proportion of data sampled for SmallKitchenAppliances). 







Each sampling method makes its own assumptions about the degrees of freedom in the sampling process that results in different interpretations of the corresponding CIs. \textsc{Boot-Inputs} assumes that there is only uncertainty on the inputs while the systems are held constant.
CIs derived from this sampling technique would express a range of values for the true correlation $\rho$ between $\mathcal{X}$ and $\mathcal{Z}$ for the \emph{specific} set of systems $\mathcal{S}$ and inputs from the same distribution as those in $\mathcal{D}$.
The opposite assumption is made for \textsc{Boot-Systems} (uncertainty in systems, inputs are fixed).
\textsc{Boot-Both}, which can be viewed as sampling systems followed by sampling inputs, assumes uncertainty on both the systems and the inputs.
Therefore the corresponding CI estimates $\rho$ for systems and inputs distributed the same as those in $\mathcal{S}$ and $\mathcal{D}$.

Algorithm~\ref{alg:ci} contains the pseudocode for calculating a CI via bootstrapping using the \textsc{Boot-Both} sampling method.
In \S\ref{sec:ci_simulations} we experimentally evaluate the Fisher transformation and the three bootstrap sampling methods, then analyze the CIs of several different metrics in \S\ref{sec:ci_experiments}.

\begin{algorithm}[t]
{
\small
\caption{Bootstrap Confidence Interval}
\label{alg:ci}
\hspace*{\algorithmicindent} \textbf{Input:} $X, Z \in \mathbb{R}^{N\times M}$, $k \in \mathbb{N}, \alpha \in [0, 1]$ \\
\hspace*{\algorithmicindent} \textbf{Output:} $(1-\alpha)\times 100\%$-confidence interval
\begin{algorithmic}[1]
\State samples $\gets$ an empty list
\For{$k$ iterations}
    \State $S$ $\gets$ samp. $\{1,\dots, N\}$ w/ repl. $N$ times
    \State $D$ $\gets$ samp. $\{1,\dots, M\}$ w/ repl. $M$ times
    \State $X_s, Z_s \gets$ empty $N \times M$ matrices
    \For{$(i, j) \in \{1, \dots, N\} \times \{1, \dots M\}$}
        \State $X_s[i, j] \gets X[S[i], D[j]]$
        \State $Z_s[i, j] \gets Z[S[i], D[j]]$
    \EndFor
    \State Append $r(X_s, Z_s)$ to samples
\EndFor
\State $\ell, u \gets (\alpha/2)\times 100$ and $(1-\alpha/2)\times 100$ percentiles of samples
\State \Return $\ell, u$
\end{algorithmic}
}
\end{algorithm}
