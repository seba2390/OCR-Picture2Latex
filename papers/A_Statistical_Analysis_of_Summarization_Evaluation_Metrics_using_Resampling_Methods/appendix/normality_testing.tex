\begin{table}
    \centering
    \begin{adjustbox}{width=\columnwidth}
    \begin{tabular}{ccccccccc}
        \toprule
        \multirow{2}{*}{\textbf{Metric}} & \multicolumn{2}{c}{\textbf{TAC'08}} & & \multicolumn{2}{c}{\textbf{Fabbri et al.}} & & \multicolumn{2}{c}{\textbf{Bhandari et al.}} \\
        \cmidrule{2-3} \cmidrule{5-6} \cmidrule{8-9}
        & $r_\textsc{Sum}$ & $r_\textsc{Sys}$ & & $r_\textsc{Sum}$ & $r_\textsc{Sys}$ & & $r_\textsc{Sum}$ & $r_\textsc{Sys}$ \\
        \midrule
Resp/Rel/Pyr & 100.0 & 0.00 &   & 32.0 & 0.52 &   & 75.0 & 0.84\\
AutoSummENG & 18.8 & 0.26 &   & 33.0 & 0.01 &   & 28.0 & 0.55\\
MeMoG & 37.5 & 0.53 &   & 33.0 & 0.01 &   & 28.0 & 0.55\\
NPowER & 29.2 & 0.36 &   & 33.0 & 0.01 &   & 28.0 & 0.55\\
BERTScore & 35.4 & 0.00 &   & 26.0 & 0.15 &   & 28.0 & 0.18\\
BEwTE & 22.9 & 0.06 &   & 37.0 & 0.00 &   & 33.0 & 0.68\\
METEOR & 27.1 & 0.15 &   & 27.0 & 0.00 &   & 30.0 & 0.61\\
MoverScore & 47.9 & 0.25 &   & 35.0 & 0.00 &   & 31.0 & 0.50\\
QAEval-F$_1$ & 58.3 & 0.00 &   & 40.0 & 0.01 &   & 45.0 & 0.21\\
ROUGE-1 & 33.3 & 0.06 &   & 32.0 & 0.00 &   & 30.0 & 0.91\\
ROUGE-2 & 31.2 & 0.71 &   & 34.0 & 0.00 &   & 61.0 & 0.62\\
ROUGE-L & 25.0 & 0.13 &   & 26.0 & 0.13 &   & 37.0 & 0.12\\
ROUGE-SU4 & 29.2 & 0.44 &   & 32.0 & 0.00 &   & 44.0 & 0.84\\
S3 & 20.8 & 0.32 &   & 26.0 & 0.00 &   & 47.0 & 0.66\\
  \bottomrule
    \end{tabular}
    \end{adjustbox} 
    \caption{For $\rsys$ the $p$-value of the Shapiro-Wilk test.
    For $\rsum$, the percent of the per-input document tests which had a significant result at $\alpha = 0.05$.
    A significant $p$-value means $H_0$ (the data is distributed normally) is rejected.
    For $\rsum$, the larger the percentage the more the data appears to be not normally distributed.}
    \label{tab:normality}
\end{table}

\section{Normality Testing}
\label{appendix:normality}
% Some statistical methods for calculating the CIs or running hypothesis tests for correlation coefficients assume the input variables are normally distributed.
% To understand if this assumption holds for the summarization data, we ran the Shapiro-Wilk test for normality \citep{ShapiroWi65}, which was reported to have the highest power out of several alternatives \citep{RazaliWa11}.

To understand if the normality assumption holds for summarization data we ran the Shapiro-Wilk test for normality \citep{ShapiroWi65}, which was reported to have the highest power out of several alternatives \citep{RazaliWa11,DBSR18,DPSR20}.
The results of the tests for the ground-truth responsiveness scores and automatic metrics are in Table~\ref{tab:normality}.
Most of the $p$-values are significant, i.e., applying a statistical test which assumes normality is incorrect in general.

