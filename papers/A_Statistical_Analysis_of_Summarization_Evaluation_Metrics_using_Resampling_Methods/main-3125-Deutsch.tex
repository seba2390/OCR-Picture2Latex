% File tacl2018v2.tex
% Sep 20, 2018

% The English content of this file was modified from various *ACL instructions
% by Lillian Lee and Kristina Toutanova
%
% LaTeXery is mostly all adapted from acl2018.sty.

\documentclass[11pt,a4paper]{article}
\usepackage{times,latexsym}
\usepackage{url}
\usepackage[T1]{fontenc}

%% Package options:
%% Short version: "hyperref" and "submission" are the defaults.
%% More verbose version:
%% Most compact command to produce a submission version with hyperref enabled
%%    \usepackage[]{tacl2018v2}
%% Most compact command to produce a "camera-ready" version
%%    \usepackage[acceptedWithA]{tacl2018v2}
%% Most compact command to produce a double-spaced copy-editor's version
%%    \usepackage[acceptedWithA,copyedit]{tacl2018v2}
%
%% If you need to disable hyperref in any of the above settings (see Section
%% "LaTeX files") in the TACL instructions), add ",nohyperref" in the square
%% brackets. (The comma is a delimiter in case there are multiple options specified.)

\usepackage[acceptedWithA]{tacl2018v2}

\usepackage{amsmath}
\usepackage{amsthm}
\usepackage{booktabs}
\usepackage{multicol}
\usepackage{adjustbox}
\usepackage{subcaption}
\usepackage{multirow}
\usepackage{algpseudocode}
\usepackage{algorithm}
\usepackage{amssymb}
\usepackage{makecell}
\usepackage[multiple]{footmisc}
\usepackage{resizegather}
\usepackage{microtype}
\usepackage[font=small]{caption}  % allowed to be 10pt (I think \small makes it go 1pt lower than the base, which is 11pt).

\usepackage{float}


%%%% Material in this block is specific to generating TACL instructions
\usepackage{xspace,mfirstuc,tabulary}
\newcommand{\dateOfLastUpdate}{Sept. 20, 2018}
\newcommand{\styleFileVersion}{tacl2018v2}

\newcommand{\ex}[1]{{\sf #1}}

\newif\iftaclinstructions
\taclinstructionsfalse % AUTHORS: do NOT set this to true
\iftaclinstructions
\renewcommand{\confidential}{}
\renewcommand{\anonsubtext}{(No author info supplied here, for consistency with
TACL-submission anonymization requirements)}
\newcommand{\instr}
\fi

%
\iftaclpubformat % this "if" is set by the choice of options
\newcommand{\taclpaper}{final version\xspace}
\newcommand{\taclpapers}{final versions\xspace}
\newcommand{\Taclpaper}{Final version\xspace}
\newcommand{\Taclpapers}{Final versions\xspace}
\newcommand{\TaclPapers}{Final Versions\xspace}
\else
\newcommand{\taclpaper}{submission\xspace}
\newcommand{\taclpapers}{{\taclpaper}s\xspace}
\newcommand{\Taclpaper}{Submission\xspace}
\newcommand{\Taclpapers}{{\Taclpaper}s\xspace}
\newcommand{\TaclPapers}{Submissions\xspace}
\fi

%%%% End TACL-instructions-specific macro block
%%%%

\title{ \vspace*{-0.5in}
{{\small \hfill TACL'21}\\
\vspace*{.25in}} A Statistical Analysis of Summarization Evaluation Metrics Using Resampling Methods}

% Author information does not appear in the pdf unless the "acceptedWithA" option is given
% See tacl2018v2.sty for other ways to format author information
\author{
    Daniel Deutsch, Rotem Dror, and Dan Roth \\
    Department of Computer and Information Science \\
    University of Pennsylvania \\
    \texttt{\{ddeutsch,rtmdrr,danroth\}@seas.upenn.edu} \\
}

\date{}

\newif\ifcomments
% Uncomment line below to keep comments; comment line below to make them regular text
\commentstrue
\ifcomments
    \providecommand\dd[1]{\textcolor{blue}{[DD: #1]}}
    \providecommand\dr[1]{\textcolor{olive}{[DR: #1]}}
    \providecommand\rd[1]{\textcolor{purple}{[RD: #1]}}
    \providecommand\todo[1]{\textcolor{red}{[TODO: #1]}}
\else
    \providecommand{\dd}[1]{}
    \providecommand{\dr}[1]{}
    \providecommand{\rd}[1]{}
    \providecommand{\todo}[1]{}
\fi

\newcommand{\rsum}{r_\textsc{Sum}}
\newcommand{\rsys}{r_\textsc{Sys}}
\newcommand{\rglo}{r_\textsc{Glo}}

\newcommand{\rhosum}{\rho_\textsc{Sum}}
\newcommand{\rhosys}{\rho_\textsc{Sys}}
\newcommand{\rhoglo}{\rho_\textsc{Glo}}

\begin{document}

\maketitle

\begin{abstract}
The quality of a summarization evaluation metric is quantified by calculating the correlation between its scores and human annotations across a large number of summaries. Currently, it is unclear how precise these correlation estimates are, nor whether differences between two metrics' correlations reflect a true difference or if it is due to mere chance. In this work, we address these two problems by proposing methods for calculating confidence intervals and running hypothesis tests for correlations using two resampling methods, bootstrapping and permutation. After evaluating which of the proposed methods is most appropriate for summarization through two simulation experiments, we analyze the results of applying these methods to several different automatic evaluation metrics across three sets of human annotations. We find that the confidence intervals are rather wide, demonstrating high uncertainty in the reliability of automatic metrics. Further, although many metrics fail to show statistical improvements over ROUGE, two recent works, QA\-Eval and BERTScore, do in some evaluation settings.\footnote{
Our code is available at \url{https://github.com/CogComp/stat-analysis-experiments}.
}
\end{abstract}
3D human pose estimation has ubiquitous applications in sport analysis, human-computer interaction, and fitness and dance teaching. While there has been remarkable progress in 3D pose estimation from a monocular image or video~\cite{hmrKanazawa17, Moon_2020_ECCV_I2L-MeshNet, kolotouros2019spin, kocabas2019vibe, xiang2019monocular}, inevitable challenges such as the depth ambiguity and the self-occlusion are still unsolved. 



\begin{figure}
     \centering
     \begin{subfigure}[h]{0.23\textwidth}
         \centering
         \includegraphics[width=\textwidth]{figures/cover/image-comp.jpg}
         \caption*{Input image}
     \end{subfigure}
     \begin{subfigure}[h]{0.23\textwidth}
         \centering
         \includegraphics[width=\textwidth]{figures/cover/smplify-comp.jpg}
         \caption*{SMPLify-X~\cite{SMPL-X:2019}}
     \end{subfigure}
     \vspace*{0.2cm}
     \begin{subfigure}[h]{0.45\textwidth}
         \centering
         \includegraphics[width=0.98\linewidth, trim=25 50 25 50]{figures/cover/scene_cover_green-comp.png}
         \caption*{3D visualization of our (left) and SMPLify-X (right) results}
     \end{subfigure}
     \vspace*{-0.2cm}
     \caption{While the state-of-the-art single-view 3D pose estimator~\cite{SMPL-X:2019} yields a small reprojection error, the recovered 3D poses may be erroneous due to the depth ambiguity. We make use of the mirror in the image to resolve the ambiguity and reconstruct more accurate human pose as well as the mirror geometry.}
     \vspace*{-0.5cm}
    \label{fig:demo1}
\end{figure}



In many scenes like dancing rooms and gyms, people are often in front of a mirror. In this case, we are able to see the person and his/her mirror image simultaneously. The mirror image actually provides an additional virtual view of the person, which can resolve the single-view depth ambiguity if the mirror is properly placed. Moreover, unseen part of the person can also be observed from the mirror image, so that the occlusion problem can be alleviated. 


In this paper, we investigate the feasibility of leveraging such mirror images to improve the accuracy of 3D human pose estimation. We develop an optimization-based framework with mirror symmetry constraints that are applicable without knowing the mirror geometry and camera parameters. We also provide a method to utilize the properties of vanishing points to recover the mirror normal along with the camera parameters, so that an additional mirror normal constraint can be imposed to further improve the human pose estimation accuracy. The effectiveness of our framework is validated on a new dataset for this new task with 3D pose ground-truth provided by a multi-view camera system. 


An important application of the proposed approach is to generate pseudo ground-truth annotations to train existing 3D pose estimators. To this end, we collect a large-scale set of Internet images that contain people and mirrors and generate 3D pose annotations with the proposed optimization method. The dataset is named Mirrored-Human.  
Compared with existing 3D human pose datasets~\cite{h36m_pami,mono-3dhp2017,vonMarcard2018} that are captured with very few subjects and background scenes, Mirrored-Human has a significantly larger diversity in human poses, appearances and backgrounds, as shown in Fig.~\ref{fig:dataset}. The experiments show that, by combining Mirrored-Human with existing datasets as training data, both accuracy and generalizability of existing 3D pose estimation methods can be significantly improved for both single-person and multi-person cases.   

In summary, we make the following contributions:
\begin{itemize}
    \item We introduce a new task of reconstructing  human pose from a single image in which we can see the person and the person's mirror image. 
    \item We develop a novel optimization-based framework with mirror symmetry constraints to solve this new task, as well as a method to recover mirror geometry from a single image.
    \item We collect a large-scale dataset named Mirrored-Human from the Internet, provide our reconstructed 3D poses as pseudo ground-truth, and show that training on this new dataset can improve the performance of existing 3D human pose estimators. 
\end{itemize}









We assume familiarity with terminologies such as Boolean variable, literal, clause, CNF formula, the SAT problem, and assignment flips in SLS solvers and refer the reader to \egcite~\cite{ST13SATProblem}.
We furthermore trust that the reader has basic knowledge of the proof system Resolution \cite{Blake37Canoncial,Robinson65Machine-oriented}.
\introduceterm{Stochastic local search} (SLS) solvers operate on complete assignments for a formula~$F$.
These solvers are started with a randomly generated complete initial assignment~$\alpha_0$.
If~$\alpha_0$ satisfies~$F$, a solution is found.
Otherwise, the SLS solver tries to find a solution by performing a random walk over the set of complete assignments for the underlying formula.
A formula~$F$ \introduceterm{logically implies} a clause~$C$ if every complete truth assignment which satisfies~$F$ also satisfies~$C$, for which we write $F \vDash C$. If $L$ is a set of clauses we write $F \vDash L$ if $F \vDash C$ for all $C \in L$.







\begin{definition}
	\label{def:RestartsUseful}
	Let $X$ be a random variable 
	for
	the runtime of an 
	SLS 
	algorithm~$\mathcal{A}$ on some input.
	For $t > 0$, the algorithm~$\mathcal{A}_t$
	is obtained by restarting~$\mathcal{A}$ after time~$t$ if no solution was found. %
	Restarts are \introduceterm{useful} if there is a $t > 0$ %
	such that
	\begin{align*}
		\expectation{X_t} < \expectation{X},
	\end{align*}
	where $X_t$ models the runtime of~$\mathcal{A}_t$. 
\end{definition}




\begin{definition}[\cite{norman1994continuous}]
	\label{def:cdf_quantile}
	Let $X$ be a real-valued random variable.
	\begin{itemize}
	\item
		Its \introduceterm{cumulative distribution function}
		(cdf) %
		is the function $F \colon \R \to [0,1]$ with 
			\[
			F(t) := \prob{X \leq t}. %
			\]
	\item
		Its \introduceterm{quantile function} $Q \colon (0,1) \to \R$ is given by $Q(p) := \inf \setdescr{t \in \R}{F(t) \geq p}$.
	\item
		A non-negative, integrable function~$f$ such that 
		$F(t) = \integralLow[u]{-\infty}{t}{f(u)}$
		is 
		called 
		\introduceterm{probability density function} (pdf) of~$X$.
	\end{itemize}
\end{definition}







\begin{definition}[\cite{Wicksell17OnLogarithmicCorrelation}]
	An absolutely continuous, positive random variable~$X$ is \introduceterm{(three-parameter) lognormally distributed} with parameters 
	$\sigma^2 > 0$, $\gamma > 0$, and $\mu \in \R$, 
	if $\log(X - \gamma)$ is normally distributed with mean $\mu$ and variance $\sigma^2$. In the following, we refer to $\sigma$ as the \introduceterm{shape}, $\mu$ as the \introduceterm{scale}, and $\gamma$ as the \introduceterm{location parameter}.
\end{definition}


\begin{definition}
	Let $X_1, \dots, X_n$ be independent, identically distributed real-valued random variables
	with realizations $x_i$ of $X_i$.
	Then the \introduceterm{empirical cumulative distribution function} (\introduceterm{ecdf}) of the sample $(x_1, \dots, x_n)$ is defined as
		\[
		\hat{F}_n(t) := \frac{1}{n} \sum_{i=1}^n \Indikator{\set{x_i \leq t}}, \quad t \in \R,
		\]	
	where $\Indikator{A}$ is the indicator of event $A$.
\end{definition}












\section{Correlation Confidence Intervals}
\label{sec:ci}
Although the strength of the relationship between $\mathcal{X}$ and $\mathcal{Z}$ on one dataset is quantified by the correlation levels $\rsys$ and $\rsum$, each $r$ is only a point estimate of the true correlation of the metrics, denoted $\rho$, on inputs and systems distributed similarly to those in $\mathcal{D}$ and in $\mathcal{S}$.
Although we cannot directly calculate $\rho$, it is possible to estimate it through a CI.

\subsection{The Fisher Transformation}
\label{sec:fisher}
The standard method for calculating a CI for a correlation is the Fisher transformation \citep{Fisher92}.
The transformation maps a correlation coefficient to a normal distribution, calculates the CI on the normal curve, and applies the reverse transformation to obtain the upper and lower bounds:
\begin{align*}
    z_r &= \textrm{arctanh}(r) \\
    r_u, r_\ell &= \textrm{tanh}\left(z_r \pm  z_{\alpha/2} \cdot c\; / \sqrt{n-b}\right)
\end{align*}
where $r$ is the correlation coefficient, $n$ is the number of observations, $z_{\alpha/2}$ is the critical value of a normal distribution, and $b$ and $c$ are constants.\footnote{
$b=3, 3, 4$ and $c=1, \sqrt{1+r^2/2}, \sqrt{.437}$ for Pearson, Spearman, and Kendall, respectively \citep{BonettWr00}.
}

Applying the Fisher transformation to calculate CIs for $\rhosys$ and $\rhosum$ is potentially problematic.
First, it assumes that the input variables are normally distributed \citep{BonettWr00}.
The metrics' scores and human annotations on the datasets that we experiment with are, in general, not normally distributed (see Appendix~\ref{appendix:normality}).
Thus, this assumption is violated, and we expect this is the case for other summarization datasets as well.
Second, it is not clear whether the transformation should be applied to the summary-level correlation since its final value is an average of correlations, which is not strictly a correlation.\footnote{
Correlation coefficients cannot be averaged because they are not additive in the arithmetic sense, however it is standard practice in summarization.
}

\subsection{Bootstrapping}
\label{sec:ci_bootstrapping}
A popular nonparametric method of calculating a CI is bootstrapping \citep{EfronTi93}.
Bootstrapping is a procedure that estimates the distribution of a test statistic by repeatedly sampling with replacement from the original dataset and calculating the test statistic on each sample.
Unlike the Fisher transformation, bootstrapping is a very flexible procedure that does not assume the data is normally distributed nor that the test statistic is a correlation, making it appropriate for summarization.

However, it is not clear how to perform bootstrap sampling for correlation levels.
Consider a more standard bootstrapped CI calculation for the mean accuracy of a question-answering model on a dataset with $k$ instances.
Since the mean accuracy is a function of the $k$ individual correct/incorrect labels, each bootstrap sample can be constructed by sampling with replacement from the original $k$ instances $k$ times.
In contrast, the correlation levels are functions of the matrices $X$ and $Z$, so each bootstrap sample should also be a pair of matrices of the same size that are sampled from the original data.

There are at least three potential methods for sampling the matrices:
\begin{enumerate}
    \item \textsc{Boot-Systems:} Randomly sample with replacement $N$ systems from $\mathcal{S}$, then select the sampled system scores for all of the inputs.

    \item \textsc{Boot-Inputs:} Randomly sample with replacement $M$ inputs from $\mathcal{D}$, then select all of the system scores for the sampled inputs.
    
    \item \textsc{Boot-Both:} Randomly sample with replacement $M$ inputs from $\mathcal{D}$ and $N$ systems from $\mathcal{S}$, then select the sampled system scores for the sampled inputs.
\end{enumerate}
Once the samples are taken, the corresponding values from $X$ and $Z$ are selected to create the sampled matrices.
An illustration of each method is shown in Figure~\ref{fig:sampling}.

%\documentclass[a4paper,11pt]{article}
\documentclass[conference, a4paper, 10pt]{IEEEtran}

\usepackage{times}
\usepackage[latin1]{inputenc}
\usepackage[T1]{fontenc}
\usepackage[english]{babel}
\usepackage{graphicx}
\usepackage{amsmath, amsfonts, dsfont,amssymb, graphicx, array, tabularx, booktabs}
\usepackage{amsthm}
\usepackage{epsf,epsfig}
%\usepackage{ulem}
\usepackage{setspace}
\usepackage{subfigure}
\usepackage[linesnumbered,ruled,vlined]{algorithm2e}
\newcommand\mycommfont[1]{\footnotesize\ttfamily\textcolor{black}{#1}}
\SetCommentSty{mycommfont}

\usepackage{multirow}
\usepackage{url}
\usepackage{color}
\usepackage[nolist,printonlyused]{acronym}      % Acronym
\usepackage{comment}
\usepackage{cite}
\usepackage{bm}
\usepackage{tikz}
\usetikzlibrary{decorations.pathreplacing}

% \usepackage[linesnumbered,ruled,vlined]{algorithm2e}
% \SetKwInOut{Initialization}{Initialization}
% \usepackage{algorithmic,float}
% \usepackage{subfig}
% \usepackage{etoolbox}
\SetKwInOut{Parameter}{Parameter}

\usepackage{hyperref}
\usepackage{mathtools}

\usepackage{blindtext}
%\usepackage{showframe}
%\renewcommand*\ShowFrameColor{\color{red}}
\renewcommand{\arraystretch}{1.2}

\usepackage{stfloats}

\newcommand{\gf}[1]{\textcolor{cyan}{{#1}}}
\newcommand{\mt}[1]{\textcolor{red}{{#1}}}
\newcommand{\dg}[1]{\textcolor{magenta}{{#1}}}

\newtheorem{thm}{Theorem} %[section]
%\newtheorem{cor}[thm]{Corollary}
%\newtheorem{lem}[thm]{Lemma}
%\newtheorem{prop}[thm]{Proposition}
%\newtheorem{res}[thm]{Result}

\newtheorem{cor}{Corollary}
\newtheorem{lem}{Lemma}
\newtheorem{prop}{Proposition}
\newtheorem{res}{Result}

\newtheorem{definition}{Definition}
%\newtheorem{algorithm}{Algorithm}
\newtheorem{remark}{Remark}
\newtheorem{observ}{Observation}
\newtheorem{assump}{Assumption}
% My macros
%\newcommand{\maxi}{\ensuremath{\mbox{maximize}}}
%\newcommand{\mini}{\ensuremath{\mbox{minimize}}}
\newcommand{\maxi}{\mathop{\displaystyle \mbox{maximize}}}
\newcommand{\mini}{\mathop{\displaystyle \mbox{minimize}}}
\newcommand{\mx}[1]{\mathbf{#1}}
% \newcommand{\mx}[1]{\mathbf{#1}}
\newcommand{\bs}[1]{\boldsymbol{#1}}
\providecommand{\keywords}[1]{\textbf{\textit{Index terms---}} #1}

\addtolength{\abovecaptionskip}{-3mm}
\addtolength{\belowcaptionskip}{-3mm}
\addtolength{\floatsep}{-4mm}
\addtolength{\textheight}{+1mm}
\addtolength{\textwidth}{+1mm}

\definecolor{amber}{rgb}{1.0, 0.49, 0.0}
\definecolor{ao}{rgb}{0.0, 0.5, 0.0}

\def\REV#1{\textcolor{black}{#1}}
\def\R2#1{\textcolor{black}{#1}}
\def\R3#1{\textcolor{black}{#1}}
\def\REVI#1{\textcolor{blue}{#1}}
\def\REVG#1{\textcolor{ao}{#1}}

\bibliographystyle{IEEEtran}


\begin{document}

\title{Sampling}
\singlespacing

\begin{comment}
\title{MU-MIMO Receiver Design and Performance Analysis in Time-Varying Rayleigh Fading}% Environment}
\title{On the Achievable SINR in MU-MIMO Systems Operating in Time-Varying Rayleigh Fading}
\title{Performance Analysis of MU-SIMO Systems Operating in Continuous Time-Varying Fading Channels}
\title{On Pilot Spacing and Power Control in MU-MIMO Systems with Continuous Time-Varying Rayleigh Fading Channels}
\title{Optimizing Pilot Spacing in MU-MIMO Systems with Continuous Time-Varying Fast Fading Channels}
\title{Optimizing Pilot Spacing in MU-MIMO Systems Operating in the Presence of Channel Aging}
\end{comment}
\title{Optimizing Pilot Spacing in MU-MIMO Systems Operating Over Aging Channels}

\singlespacing
\author{
Sebastian Fodor$^\flat$, G\'{a}bor Fodor$^{\star\dag}$, Do\u{g}a G\"{u}rg\"{u}no\u{g}lu$^{\dag}$,
Mikl\'{o}s Telek$^{\ddag\sharp}$ \\
\small $^\flat$Stockholm University, Stockholm, Sweden. E-mail: \texttt{sebbifodor@fastmail.com}\\
\small $^\star$Ericsson Research, Stockholm, Sweden. E-mail: \texttt{Gabor.Fodor@ericsson.com}\\
\small $^\dag$KTH Royal Institute of Technology, Stockholm, Sweden. E-mail: \texttt{gaborf|dogag@kth.se}\\
% \small $^\diamond$KTH Royal Institute of Technology, Stockholm, Sweden. E-mail: \texttt{dogag@kth.se}\\
\small $^\ddag$Budapest University of Technology and Economics, Budapest, Hungary. E-mail: \texttt{telek@hit.bme.hu}\\
\small $^\sharp$MTA-BME Information Systems Research Group, Budapest, Hungary. E-mail: \texttt{telek@hit.bme.hu}
}
\maketitle
\pagestyle{plain}
% \thispagestyle{empty}

% !TEX root = main.tex

\acrodef{APK}{Android Application Package}


\begin{abstract}
In the uplink of multiuser multiple input multiple output (MU-MIMO) systems operating over aging channels, pilot spacing is crucial for acquiring channel state information and achieving high signal-to-interference-plus-noise ratio (SINR). Somewhat surprisingly, very few works examine the impact of pilot spacing on the correlation structure of subsequent channel estimates and the resulting quality of channel state information considering channel aging. In this paper, we consider a fast-fading environment characterized by its exponentially decaying autocorrelation function, and model pilot spacing as a sampling problem to capture the inherent trade-off between the quality of channel state information and the number of symbols available for information carrying data symbols. We first establish a quasi-closed form for the achievable asymptotic deterministic equivalent SINR when the channel estimation algorithm utilizes multiple pilot signals. Next, we establish upper bounds on the achievable SINR and spectral efficiency, as a function of pilot spacing, which helps to find the optimum pilot spacing within a limited search space. Our key insight is that to maximize the achievable SINR and the spectral efficiency of MU-MIMO systems, proper pilot spacing must be applied to control the impact of the aging channel and to tune the trade-off between pilot and data symbols.
\end{abstract}
\keywords{autoregressive processes, channel estimation, estimation theory, multiple input multiple output, receiver design}

\section{Introduction}
In wireless communications, pilot symbol-assisted channel estimation and prediction are used to achieve
reliable coherent reception, and thereby to provide a variety of high quality services in a spectrum efficient
manner. In most practical systems, the transmitter and receiver nodes acquire and predict channel state information
by employing predefined pilot sequences during the training phase, after which information symbols can be
appropriately modulated and precoded at the transmitter and estimated at the receiver.
Since the elapsed time between pilot transmissions and the transmit power level of pilot symbols have a
large impact on the quality of channel estimation, a large number of papers investigated the optimal
spacing and power control of pilot signals in both single and multiple antenna systems
\cite{Yan:01, Zhang:07B, Abeida:10, Hijazi:10, GH:12,Truong:13, Kong:2015, Chiu:15, Kashyap:17, Kim:20, Yuan:20, Fodor:21}.

Specifically in the uplink of \ac{MU-MIMO} systems, several papers proposed pilot-based channel estimation and
receiver algorithms assuming that the complex vector channel undergoes block fading, meaning that the channel is
constant
%during the coherence time of the channel
between two subsequent channel estimation instances
\cite{Couillet:2012, Wen:2013, Hoydis:13, Mallik:18}.
In the block fading
model, the evolution of the channel is memoryless in the sense that each channel realization is drawn independently
of previous channel instances from some characteristic distribution. While the block fading model is useful for
obtaining analytical expressions for the achievable \ac{SINR} and capacity \cite{Hoydis:13, Hanlen:2012}, it fails
to capture the correlation between subsequent channel realizations and the aging of the channel between estimation
instances \cite{Truong:13, Kong:2015, Yuan:20, Fodor:21}.

\begin{comment}
The wireless channels in the uplink of \ac{MU-MIMO} systems can often be advantageously modelled as
\ac{AR} processes, because \ac{AR} channel models capture the time-varying (aging) nature of the channels and
facilitate channel estimation and prediction \cite{Yan:01, Zhang:07B, Lehmann:08, Abeida:10, Hijazi:10, GH:12,Truong:13,
Kong:2015, Chiu:15, Kashyap:17, Kim:20, Yuan:20, Fodor:21}.
\end{comment}

Due to the importance of capturing the evolution of the wireless channel in time, several papers developed
time-varying channel models, as an alternative to block fading models, whose states are advantageously estimated
and predicted by means of suitably spaced pilot signals. In particular,
a large number of related works assume that the wireless channel can be represented as an \ac{AR} process whose
states are estimated and predicted using Kalman filters, which exploit the correlation between subsequent
channel realizations \cite{Abeida:10, Hijazi:10, Truong:13, Kim:20, Fodor:21}.
These papers assume that the coefficients of the related \ac{AR} process are known, and the
current and future states of the process (and thereby of the wireless channel) can be well estimated.
Other important related works concentrate on estimating the coefficients of \ac{AR} processes based on
suitable pilot-based observations and measurements \cite{Mahmoudi:11, Xia:15, Esfandiari:20}.
In our recent work \cite{Fodor:21}, it was shown that when an \ac{AR} process is a good model of the
wireless channel and the \ac{AR} coefficients are well estimated, not only the channel estimation can
exploit the memoryful property of the channel, but also a new \ac{MU-MIMO} receiver can be
designed, which minimizes the \ac{MSE} of the received data symbols by exploiting the correlation
between subsequent channel states.
It is important to realize that the above references build on discrete time \ac{AR} models, in which the state transition matrix is an input of the model and can be estimated by some suitable system identification technique, such as the one
proposed in \cite{Esfandiari:20}.
However, these papers do not ask the question of how often
the channel state of a continuous time channel should be observed by suitably spaced pilot signals to realize a certain state transition matrix in the \ac{AR} model of the channel.

Specifically, a key characteristic of a continuous time Rayleigh fading environment is that the autocorrelation
function of the associated stochastic process is a zeroth-order Bessel function, which must be properly modelled \cite{Zheng:03, Wang:07}.
This requirement is problematic when developing discrete-time \ac{AR} models, % for wireless channels,
since it is well-known
that Rayleigh fading cannot be perfectly modelled with any finite order \ac{AR} process
(since the autocorrelation function of discrete time \ac{AR} processes does not follow a Bessel function),
although the statistics of \ac{AR} process can approximate those of Rayleigh fading \cite{McGuire:05,Zheng:05}.

Recognizing the importance of modeling fast fading, including Rayleigh fading, channels with proper autocorrelation function as a basis for
pilot spacing optimization, papers \cite{Savazzi:09, Savazzi:09B} use a continuous time process as a representation of
the wireless channel, and address the problem of pilot spacing as a sampling problem. %of this continuous time process.
According to this approach, pilot placement can be considered as a sampling problem of the fading variations,
and the quality of the channel estimate is determined by the density and accuracy of channel sampling \cite{Savazzi:09B}.
However, these papers consider \ac{SISO} systems, do not deal with the problem of pilot and data power control, and
are not applicable to \ac{MU-MIMO} systems employing a \ac{MMSE} receiver, which was proposed in, for example, \cite{Fodor:21}.
On the other hand, paper \cite{Truong:13} analyzes the impact of channel aging on the performance of MIMO systems, without investigating the interplay between pilot spacing and the resulting state transition matrix of the \ac{AR} model of the fast fading channel.
The most important related works, their assumptions and key performance metrics are listed and compared with those
of the current paper in Table \ref{tab:tab1}.

\begin{comment}
--- Original table with more references ------
\begin{table*}[t]
	\centering
	\caption{Overview of Related Literature}
	\vspace{2mm}
	\label{tab:tab1}
	\footnotesize
	\begin{tabular}{
			|p{0.12\textwidth}|
			>{\centering}p{0.12\textwidth}|
			>{\centering}p{0.13\textwidth}|
			>{\centering}p{0.1\textwidth}|
			>{\centering}p{0.12\textwidth}|
			>{\centering}p{0.12\textwidth}|
			p{0.14\textwidth}|}
		% \begin{tabularx}{\columnwidth}{|X|X|X|X|X|X|}
		\hline
		\hline
		\textbf{~~Reference} & \textbf{Block fading vs. Aging channel} & \textbf{Is \ac{AR} modeling used?}
		& \textbf{Channel est.} & \textbf{SISO or MIMO receiver structure} & \textbf{Key performance indicators} & \textbf{~~Comment}  \\
		\hline
		\hline
		Hoydis et al., \cite{Hoydis:13} & block fading channel & not applicable & MMSE based on a single observation
		& regularized MMSE receiver that takes into account the estimated channel of each user & average SINR, spectral efficiency & both UL and DL are considered \\
		\hline
		Truong et al., \cite{Truong:13} & channel aging between pilots & discrete time AR approximating Bessel & MMSE based on perfectly known AR params and channel prediction
		& max. ration combiner (MRC) receiver (not AR-aware) & average SINR, achievable rate & both UL and DL are considered \\
		\hline
        Zhang et al., \cite{Zhang:07B} & channel aging between pilots & discrete time AR approximating Bessel & adaptive est. of AR params & SISO joint channel and data est.
        & BER & joint AR(2) parameter estimation and demodulation for SISO systems \\
        \hline
		Savazzi and Spagnolini \cite{Savazzi:09} & channel aging between pilots & discrete time AR to model channel evolution over multiple channel estimation instances  & interpolation based on multiple observations & SISO
		& average SINR and BER & power control is out of scope \\
		\hline
		Savazzi and Spagnolini \cite{Savazzi:09B} & channel aging between pilots & discrete time AR to model channel evolution over multiple channel estimation instances & interpolation based on multiple observations & SISO
		& average SINR and BER & pilot/data power control is out of scope \\
		\hline
		Mallik et al., \cite{Mallik:18} & block fading channel & not applicable & MMSE channel estimation & SIMO with MRC
		& average SINR, symbol error probability & pilot/data power control is out of scope \\
		\hline
		Akin and Gursoy \cite{Akin:07} & channel aging between pilots & discrete time first order AR (Gauss-Markov) process & MMSE channel  estimation
		& SISO & achievable rate and bit energy $E_b/N_0$ & optimal power distribution and training period for SISO are derived \\
		\hline
		Chiu and Wu \cite{Chiu:15} & channel aging between plots & discrete time AR model approximating a Rayleigh fading & Kalman filter assisted channel estimation and tracking
		& receiver structure is out-of-scope & MSE of channel estimation, achievable data rate, capacity lower bound & pilot/data power control is out of scope \\
		\hline
		Fodor et al., \cite{Fodor:21} & no channel aging between pilots; correlation between channel realizations & discrete time AR(1) model & Kalman filter assisted channel estimation & AR(1)-aware MIMO MMSE receiver
		& MSE of the received data symbols & optimum pilot power control for AR(1) channels is determined \\
		\hline
		Present paper & channel aging between pilots & AR(1) to model channel aging between pilots & MMSE interpolation based on multiple subsequent observations
		& MU-MIMO with MMSE receiver & average SINR & both pilot spacing and pilot/data power control are considered \\
		\hline
	\end{tabular}
\end{table*}
\end{comment}
% Current table with less references and modified text
\begin{table*}[h!]
	\centering
	\caption{Overview of Related Literature}
	\vspace{2mm}
	\label{tab:tab1}
	\footnotesize
	\begin{tabular}{
			|p{0.08\textwidth}|
			>{\centering}p{0.12\textwidth}|
			>{\centering}p{0.13\textwidth}|
			>{\centering}p{0.1\textwidth}|
			>{\centering}p{0.12\textwidth}|
			>{\centering}p{0.12\textwidth}|
			p{0.14\textwidth}|}
		% \begin{tabularx}{\columnwidth}{|X|X|X|X|X|X|}
		\hline
		\textbf{Reference} & \textbf{Block fading vs. Aging channel} & \textbf{Is \ac{AR} modeling used?}
		& \textbf{Channel est.} & \textbf{SISO or MIMO receiver} & \textbf{Key performance indicators} & \textbf{~~Comment}  \\
		\hline
		\begin{comment}
		\hline
		Hoydis et al., \cite{Hoydis:13} & block fading channel & not applicable & MMSE based on a single observation
		& regularized MMSE receiver that takes into account the estimated channel of each user & average SINR, spectral efficiency & both UL and DL are considered \\
		\end{comment}
		
		Truong et al., \cite{Truong:13} & channel aging between pilots & discrete time AR approximating a Bessel func. & MMSE based on known AR params %and channel prediction
		& max. ratio combiner (MRC) receiver (not AR-aware) & average SINR, achievable rate & both UL and DL are considered \\
		\hline
        Zhang et al., \cite{Zhang:07B} & channel aging between pilots & discrete time AR approximating a Bessel f. & adaptive est. of AR params & SISO joint channel and data est.
        & BER & %joint
        AR(2) parameter estimation and demodulation %for SISO systems
        \\
        \hline
		Savazzi and Spagnolini \cite{Savazzi:09} & channel aging between pilots & %discrete time
		AR %to model
		channel evolution over %multiple channel
		estimation instances  & interpolation based on multiple observations & SISO
		& average SINR and BER & power control is out of scope \\
		\hline
		\begin{comment}
		Savazzi and Spagnolini \cite{Savazzi:09B} & channel aging between pilots & discrete time AR to model channel evolution over multiple channel estimation instances & interpolation based on multiple observations & SISO
		& average SINR and BER & pilot/data power control is out of scope \\
		\hline
		\end{comment}
		Mallik et al., \cite{Mallik:18} & block fading channel & not applicable & MMSE channel estimation & SIMO with MRC
		& average SINR, symbol error probability & pilot/data power control is out of scope \\
		\hline
		Akin and Gursoy \cite{Akin:07} & channel aging between pilots & discrete time first order AR (Gauss-Markov) process & MMSE channel  estimation
		& SISO & achievable rate and bit energy $E_b/N_0$ & optimal power distribution and training period for SISO are derived \\
		\hline
		Chiu and Wu \cite{Chiu:15} & channel aging between plots & discrete time AR model approximating a Rayleigh fading & Kalman filter assisted channel estimation %and tracking
		& receiver structure is out-of-scope & MSE of channel est., %achievable
		data rate, capacity %lower bound
		& pilot/data power control is out of scope \\
		\hline
		Fodor et al., \cite{Fodor:21} & no aging between pilots; correlated
		pilot intervals% between channel realizations
		& discrete time AR(1) model & Kalman assisted ch. est. & AR(1)-aware MIMO MMSE receiver
		& MSE of the received data symbols & optimum pilot power control for AR(1) channels \\
		\hline
		Present paper & channel aging between pilots & AR(1) to model channel aging between pilots & MMSE interpolation by multiple %subsequent
		observations
		& MU-MIMO with MMSE receiver & average (det. equivalent) SINR & both pilot spacing and pilot/data power control are considered \\
		\hline
	\end{tabular}
\end{table*}

In this paper, we are interested in determining the average \ac{SINR} in the uplink of \ac{MU-MIMO} systems
operating in fast fading as a function of pilot spacing, pilot/data power allocation,
number of antennas and spatially multiplexed users. Specifically, we ask the following two important questions,
which are not answered by previous works:

\begin{itemize}
	\item
	What is the average \ac{SINR} in a closed or quasi-closed form in the uplink of \ac{MU-MIMO} systems
	in fast fading in the presence of antenna correlation?
	How does the average \ac{SINR} depend on pilot spacing and pilot/data power control?
	\item
	What is the optimum pilot spacing and pilot/data power allocation as a function of
	the number of antennas and the Doppler frequency associated with the continuous time fast fading channel?
\end{itemize}

In the light of the above discussion and questions, the main contributions of the present paper
are as follows:

\begin{itemize}
	\item
	Proposition \ref{Prop:SINR} derives the asymptotic deterministic equivalent \ac{SINR} of any user
	in a \ac{MU-MIMO} system
	in every data slot for fast fading channels that can be characterized by an associated \ac{AR} process;
	\item
	Theorem \ref{thm:upperbound} and Proposition \ref{UpperB1} establish an upper bound on the achievable \ac{SINR} as a function of
	pilot spacing, which is instrumental for determining the optimum pilot spacing.
	\item
	Proposition \ref{UpperB2}, building on Proposition \ref{UpperB1}, provides and upper bound on the average achievable spectral efficiency, which is instrumental in limiting the search space for the optimal frame size as a function of the Doppler frequency.
\end{itemize}
In addition, we believe that the engineering insights drawn from the numerical studies are
useful when designing pilot spacing, for example in the form of determining the number of reference signals in an uplink frame structure, for \ac{MU-MIMO} systems.

Specifically, to answer the above questions, we proceed as follows. In the next section, we present our system model,
which admits correlated wireless channels between any of the single-antenna mobile terminal and the receive
antennas of the \ac{BS}.
Next, Section \ref{Sec:ChannelE} focuses on channel estimation, which is based on subsequent pilot-based
measurements and an \ac{MMSE}-interpolation for the channel states in between estimation instances.
Section \ref{Sec:SINR} proposes an algorithm to determine the average \ac{SINR}.
Section \ref{Sec:PilotSpacing}  studies the impact of pilot spacing and power control on the achievable \ac{SINR} and the \ac{SE} of all users in the system.
That section investigates the impact of pilot spacing %and power control
on the achievable \ac{SINR}
and establishes an upper bound on this \ac{SINR}. We show that this upper bound is monotonically
decreasing as the function of pilot spacing. This property is very useful, because it enables to
limit the search space of the possible pilot spacings when looking for the optimum pilot spacing in Section \ref{Sec:Alg}.
That section also considers the special case when the channel coefficients associated with the different
receive antennas are uncorrelated and identically distributed. It turns out that in this special
case a simplified \ac{SINR} expression can be derived.
Section \ref{Sec:Num} presents numerical results and discusses engineering insights.
Finally, Section \ref{Sec:Conc} draws conclusions.

\section{System Model}

\begin{figure}
\centering
\begin{tikzpicture}[scale = 1]
  \def\x{0.65}
  \def\y{0.07}

  \draw[black, thick] (0,0) rectangle (6,3.5);

  \draw (\x,0) -- (\x,3.5);
  \draw (2*\x,0) -- (2*\x,3.5);
  \draw (3*\x,0) -- (3*\x,3.5);
  \draw (6-\x,0) -- (6-\x,3.5);
  \draw (6-2*\x,0) -- (6-2*\x,3.5);

  \draw (0, \x) -- (6, \x);
  \draw (0, 3.5-\x) -- (6, 3.5-\x);
  \draw (0, 3.5-2*\x) -- (6, 3.5-2*\x);

  \node at (\x/2, \x/2) {\footnotesize Pilot};
  \node at (\x/2, 3.5-\x/2) {\footnotesize Pilot};
  \node at (\x/2, 3.5-3*\x/2) {\footnotesize Pilot};

  \node at (3*\x/2, \x/2) {\footnotesize Data};
  \node at (3*\x/2, 3.5-\x/2) {\footnotesize Data};
  \node at (3*\x/2, 3.5-3*\x/2) {\footnotesize Data};

  \node at (5*\x/2, \x/2) {\footnotesize Data};
  \node at (5*\x/2, 3.5-\x/2) {\footnotesize Data};
  \node at (5*\x/2, 3.5-3*\x/2) {\footnotesize Data};

  \node at (6-3*\x/2, \x/2) {\footnotesize Data};
  \node at (6-3*\x/2, 3.5-\x/2) {\footnotesize Data};
  \node at (6-3*\x/2, 3.5-3*\x/2) {\footnotesize Data};

  \node at (6-\x/2, \x/2) {\footnotesize Pilot};
  \node at (6-\x/2, 3.5-\x/2) {\footnotesize Pilot};
  \node at (6-\x/2, 3.5-3*\x/2) {\footnotesize Pilot};

  \draw [->] (3.5*\x, 1.5*\x) -- (0.8*\x, 0.8*\x);
  \draw [->] (3.5*\x, 1.7*\x) -- (0.8*\x, 3.5-1.8*\x);
  \node at (5*\x, 1.6*\x) {\footnotesize Pilot Symbols};

  \draw [->] (3.5*\x, 2.6*\x) -- (2.8*\x, 0.8*\x);
  \draw [->] (3.5*\x, 2.8*\x) -- (2.8*\x, 3.5-1.8*\x);
  \node at (5*\x, 2.7*\x) {\footnotesize Data Symbols};

  \filldraw [black] (\x/2,3.5-2.3*\x) circle (0.7pt);
  \filldraw [black] (\x/2,3.5-2.6*\x) circle (0.7pt);
  \filldraw [black] (\x/2,3.5-2.9*\x) circle (0.7pt);

  \filldraw [black] (3.3*\x,3.5-\x/2) circle (0.7pt);
  \filldraw [black] (3.6*\x,3.5-\x/2) circle (0.7pt);
  \filldraw [black] (3.9*\x,3.5-\x/2) circle (0.7pt);

  \draw [decorate,decoration={brace,amplitude = 10pt}] (6-\x,-0.1) -- (\x,-0.1)
  node [black,midway,yshift=-25pt] {$\Delta$: number of data slots};
  \draw [decorate,decoration={brace,amplitude = 5pt}] (\x,-0.1) -- (0,-0.1)
  node [black,midway,yshift=-13pt] {$T$: symbol duration};


  \draw [decorate,decoration={brace,amplitude = 10pt}] (-0.1, 0) -- (-0.1,3.5)
  node [black,midway,xshift=-18pt] {\rotatebox{270}{$F$ frequencies}};

  \draw [->, thick] (0,3.7) -- (6,3.7);
  \node at (6.15, 3.7) {$t$};

  \draw plot [smooth, tension=1] coordinates { (0, 6) (2,4.2) (4,5) (6,4)};
  \draw [dashed, gray] plot [smooth, tension=1] coordinates { (0,5.4) (2,4.9) (4,4.7) (6,4.5)};

  \filldraw (\x/2-\y,5.3-\y) rectangle (\x/2+\y, 5.3+\y);
  \node at (\x/2, 5) {$\hat{\mx{h}}(0)$};
  \filldraw (6-\x/2-\y,4.52-\y) rectangle (6-\x/2+\y, 4.52+\y);
  \node at (6.2-\x/2, 5.1) {$\hat{\mx{h}}((\Delta\!+\!1)T)$};

  \draw [->] (4.1,5.6) -- (4,5.1);
  \node at (4.1, 5.9) {$\mx{h}(t)$};

  \draw [->] (2.2,5.5) -- (2.3,5);
  \node at (2.2, 5.8) {$\hat{\mx{h}}(t)$};
\end{tikzpicture}
\caption{Pilot (P) and data (D) symbols in the time-frequency domains of the system in the $(0,(\Delta+1)T)$ interval. The solid line above the
time-frequency resource grid represents the continuous time complex channel $\mx{h}(t)$, while the dashed line represents
the \ac{MMSE} channel estimate $\hat{\mx{h}}(t)$.
Notice that in each time slot of length $T$ all symbols are either pilot or
data symbols.}
\label{fig:Model}
\end{figure}

\begin{table}[t]
\caption{System Parameters}
\vspace{2mm}
\label{tab:notation}
\footnotesize
\begin{tabularx}{\columnwidth}{|X|X|}
\hline
\hline
\textbf{Notation} & \textbf{Meaning} \\
\hline
\hline
% Network layout
$K$ & Number of \ac{MU-MIMO} users \\
\hline
$N_r$ & Number of antennas at the BS \\
\hline
$F$ & Number of frequency channels used for pilot and data transmission within one slot \\
\hline
$\Delta$ & number of data slots in a data-pilot cycle \\
\hline
$\tau_p=F, \tau_d=\Delta F$ & Number of pilot/data symbols within a coherent set of subcarriers  \\
\hline
$\mx{s}\in \mathds{C}^{\tau_p \times 1}$ & Sequence of pilot symbols\\
\hline
$x$ & Data symbol \\
\hline
%$P_p, P, P_{\text{tot}}$ & Pilot power per symbol, data power per symbol, and total power budget  \\
$P_p, P$ & Pilot power per symbol, data power per symbol\\
\hline
$\mx{Y}^p(t) \in \mathds{C}^{N_r \times \tau_p}, y(t) \in \mathds{C}^{N_r}$ & Received pilot and data signal at time $t$, respectively  \\
\hline
% $\mx{N}, \mx{n}_d(t)$ & Additive white Gaussian noise at the received pilot and data signal, respectively \\
% \hline
$\alpha$ & Large scale fading between the mobile station and the base station \\
\hline
$\mx{C} \in \mathds{C}^{N_r \times N_r}$ & Stationary covariance matrix of the fast fading channel \\
\hline
$\mx{h}(t), \hat{\mx{h}}(t) \in \mathds{C}^{N_r}$ & Fast fading channel and estimated channel \\
% $\sigma_p^2 \mx{I}_{N_r}, \sigma_d^2 \mx{I}_{N_r}, \mx{C} \in \mathds{C}^{N_r \times N_r}$
% & Covariance of $\mx{N}$, $\mx{n}_d$, $\mx{h}(t)$, respectively \\
% \hline
\hline
$\bs{\varepsilon}(t) \in \mathds{C}^{N_r}, \bs{\Sigma} \in \mathds{C}^{N_r \times N_r}$
& Channel estimation error and its covariance matrix\\
\hline
%$\mx{G}, \mx{G}^\text{naive}, \mx{G}^\star$ & MU-MIMO receivers: generic, naive, and optimal, respectively. \\
$\mx{G}^\star$ & Optimal MU-MIMO receiver. \\
\hline
$f_D$ & Maximum Doppler frequency \\
\hline
$T$ & Slot duration \\
%\hline
%$\bs{\Lambda}_p$ & AWGN on the received pilot symbols at the \ac{BS} \\
\hline
\end{tabularx}
\end{table}

\subsection{Uplink Pilot Signal Model}
By extending the single antenna channel model of \cite{Savazzi:09},
each transmitting \ac{MS} uses a single time slot to send $F$ pilot symbols,
followed by $\Delta$ time slots, each of which containing $F$ data symbols according to Figure \ref{fig:Model}.
Each symbol is transmitted within a coherent time slot of duration $T$.
Thus, the total frame duration is $(1+\Delta) T$, such that each frame consists of 1 pilot
and $\Delta$ data time slots, which we will index with $i=1 \ldots \Delta$.
User-$k$ transmits each of the $F$ pilot symbols with transmit power $P_{p,k}$, and each data
symbol in slot-$i$ with transmit power $P_k(i),~k=1 \ldots K$.
To simplify notation, in the sequel we tag User-1, and will
drop index $k=1$ when referring to the tagged user.

Assuming that the coherence bandwidth accommodates at least $F$ pilot symbols,
this system allows to create $F$ orthogonal pilot sequences.
To facilitate spatial multiplexing
and \ac{CSIR} acquisition at the \ac{BS}, the \acp{MS} use orthogonal complex
sequences, such as shifted Zadoff-Chu sequences of length $\tau_p=F$, which we denote as:
\begin{align}
\mathbf{s} &\triangleq \left[s_1,...,s_{\tau_p}\right]^T \in \mathds{C}^{{\tau_p \times 1}},
\end{align}
whose elements satisfy %are scaled appropriately according to
$|s_i|^2 = 1$. %, for $i=1,..,\tau_p$ \cite{Sesia:11}.
% To enable spatial multiplexing, the length of the pilot sequences
% $\tau_p$ is chosen such that a maximum of $K$ users can be served simultaneously, implying that
% $\tau_p \geq K$ holds.
Under this assumption, the system can spatially multiplex $K\leq F$ \acp{MS}.
Focusing on the received pilot signal from the tagged user at the \ac{BS},
the received pilot signal takes the form of \cite{Fodor:21}:
\begin{align}
\mathbf{Y}^p(t)
&=
\alpha \sqrt{P_{p}}\mathbf{h}(t) \mathbf{s}^T +\mathbf{N}(t) ~~ \in \mathds{C}^{N_r \times \tau_p},
\label{eqn:received_training_seq}
\end{align}
where %we assume that
$\mathbf{h}(t) \in \mathds{C}^{N_r \times 1} \sim \mathcal{CN}(\mathbf{0},\mathbf{C})$, that is,
$\mathbf{h}(t)$ is a %circular symmetric
complex normal distributed column vector
with mean vector $\mathbf{0}$ and covariance matrix $\mx{C} \in \mathds{C}^{N_r \times N_r}$. % $\in \mathds{C}^{N_r \times N_r}$. %(of size $N_r$), %at all $t$ time instance,
Furthermore, $\alpha$ denotes
% propagation loss
large scale fading, $P_p$ denotes the pilot power of the tagged user,
and $\mathbf{N}(t)\in \mathds{C}^{N_r \times \tau_p}$
is the %spatially and temporally
\ac{AWGN} with element-wise variance $\sigma_p^2$.
It will be convenient to introduce $\mathbf{\tilde Y}^p(t)$ by stacking the columns of  $\mathbf{Y}^p(t)$ as:
\begin{align}
\mathbf{\tilde Y}^p(t)=\textbf{vec}\big(\mathbf{Y}^p(t)\big)=\alpha\sqrt{P_p} \mathbf{S} \mathbf{h}(t) +\mathbf{\tilde N}(t) ~\in \mathds{C}^{\tau_p N_r \times 1},
\label{eqn:received_training_seq2}
\end{align}
where $\textbf{vec}$ is the column stacking vector operator,
$\mathbf{\tilde Y}^p(t)$, $\mathbf{\tilde N}(t) \in \mathds{C}^{\tau_p N_r \times 1}$
and
$\mathbf{S} \triangleq \mathbf{s}\otimes \mathbf{I}_{N_r} \in \mathds{C}^{\tau_p N_r \times N_r}$
is such that $\mathbf{S}^H\mathbf{S}=\tau_p\mathbf{I}_{N_r}$,
where $\mathbf{I}_{N_r}$ is the identity matrix of size $N_r$.
\vspace{-2mm}
\subsection{Channel Model}
\vspace{-1mm}
% Consider a single block of $\Delta$ data symbols sent at time instances $iT$ for $i = 1\ldots \Delta$
% and a pilot symbol sent at time $0$ and subsequently at $(\Delta+1)T$.
In \eqref{eqn:received_training_seq},
the channel $\mx{h}(t)$ evolves continuously according to a multivariate complex stochastic process
with stationary covariance matrix $\mx{C}$.
That is, for symbol duration $T$, the channel ($\mx{h}(t)$) evolves according to the following \ac{AR} process:
\begin{align}
\label{eq:AR1}
\mx{h}(t+T) &= \mx{A} \mx{h}(t) + \bs{\vartheta}(t),
\end{align}
where the transition matrix of the \ac{AR} process is denoted by $\mx{A}$.
This \ac{AR} model has been commonly used to approximate Rayleigh fading channels in e.g. \cite{Baddour05}.
Equation \eqref{eq:AR1} implies that the autocorrelation function of the channel process is:
\begin{align}
\mathds{E}\left(\mx{h}(t)\mx{h}^H(t+iT)\right) &= \mx{C}\left(\mx{A}^H\right)^i, \quad \forall i.
\end{align}
%satisfies the following equation:
Consequently, the autocorrelation function of
the fast fading channel ($\mx{h}(t)$) is modelled as:
\begin{align}
\label{eq:autocorr}
\mx{R}(i) \triangleq
\mathds{E}\left(\mx{h}(t)\mx{h}^H(t+iT)\right) &= \mx{C} e^{\mx{Q}^H i T },
\end{align}
where matrix $\mx{Q}$ describes the correlation decay, such that:
\begin{align}
e^{\mx{Q}T} = \mx{A}.
\end{align}

Similarly, for user $k$,
\begin{align}
\label{eq:autocorrk}
\mx{R}_k(i) \triangleq
\mathds{E}\left(\mx{h}_k(t)\mx{h}_k^H(t+iT)\right) &= \mx{C}_k e^{\mx{Q}_k^H i T },
\end{align}

%We denote by $R(i) \triangleq cJ_0(2 \pi f_D i T)$ the correlation between the $i$-th element of
%the channel vectors that are $i \cdot T$ apart from each other in time.
% For the pilot channels we
In each pilot slot,
the \ac{BS} utilizes \ac{MMSE} channel estimation to obtain the channel estimate of each user, as it will be detailed in Section \ref{Sec:ChannelE}.
Without loss of generality, to simplify the notation, hereafter we assume that the time unit is $T$ and $iT=i$.

\begin{comment}
\begin{align}
\label{eq:LS}
\hat{\mx{h}}(t) &= \mx{h}(t) + \boldsymbol{\varepsilon}(t), \\
\boldsymbol{\varepsilon}(t) & \sim
%\mathcal{CN}\left(0,\frac{\boldsymbol{\Lambda}_p}{\alpha^2 F P_p} \right) \triangleq
\mathcal{CN}\left(0,\mx{\Sigma}\right)
\text{~with~} \mx{\Sigma} = \frac{\boldsymbol{\Lambda}_p}{\alpha^2 F P_p}
\end{align}
where $\mathcal{CN}\left(0,\mx{\Sigma}\right)$ is the $N_r$ dimensional complex normal distribution with mean $0$ and covariance matrix $\mx{\Sigma}$,  $\boldsymbol{\Lambda}_p$ is the covariance matrix of the \ac{AWGN} on the pilot symbols at the receiver
and $\alpha$ is the path loss between the \ac{MS} and the \ac{BS} \cite{Fodor:21}, which are assumed to be identical for all $F$ frequencies.
\end{comment}


\subsection{Data Signal Model}
When spatially multiplexing $K$ \ac{MU-MIMO} users,
the received data signal at the \ac{BS} at time $t$ is \cite{Fodor:21}:

\begin{align}
\mathbf{y}(t)
&=
\underbrace{\mathbf{\alpha} \mathbf{h}(t) \sqrt{P} x(t)}_{\text{tagged user}}
+ \underbrace{\sum_{k=2}^K \mathbf{\alpha}_{k} \mathbf{h}_k(t) \sqrt{P_{k}} x_{k}(t)}_{\text{co-scheduled MU-MIMO users}}
% \underbrace{\sum_{i \neq 1}^L \sum_{k=1}^K \mathbf{g}_{1,i,k} \sqrt{P_{i,k}} x_{i,k}}_{\text{Other cells}}
+\mathbf{n}_d(t),
%~~\in \mathds{C}^{N_r \times 1},
\label{eq:mumimo2}
\end{align}
\noindent where $\mathbf{y}(t)\in \mathds{C}^{N_r \times 1}$;
% and $\mathbf{\alpha}_{k} \mathbf{h}_k(t) \in \mathds{C}^{N_r \times 1}$
% denotes the channel vector,
and $x_k(t)$ denotes the transmitted data symbol of User-$k$
at time $t$ with transmit power $P_k$.
Furthermore, $\mathbf{n}_d(t)~\sim \mathcal{CN}\left(\mx{0},\sigma_d^2 \mathbf{I}_{N_r}\right)$
is the \ac{AWGN} at the receiver.
\vspace{-2mm}
\section{Channel Estimation}
\label{Sec:ChannelE}
In this section, we are interested in calculating the \ac{MMSE} estimation of the
channel in each slot $i$, based on received pilot signals, as a function of the frame
size corresponding to pilot spacing (see $\Delta$ in Figure \ref{fig:Model}).
Note that estimating the channel at the receiver can be based on
multiple received pilot signals both before and after the actual data slot $i$.
While using pilot signals that are received before data slot $i$ requires to
store the samples of the received pilot, using pilot signals that arrive after
data slot $i$ necessarily induces some delay in estimating the transmitted data symbol.
In the numerical section, we will refer to specific channel estimation strategies
as, for example, "1 before, 1 after" or "2 before, 1 after" depending on the number
of utilized pilot signals received prior to or following data slot $i$
for \ac{CSIR} acquisition.
In the sequel we use the specific case of "2 before, 1 after" to illustrate the
operation of the \ac{MMSE} channel estimation scheme,
that is when the receiver uses the pilot signals
$\mathbf{\tilde Y}^p(-\Delta-1)$, $\mathbf{\tilde Y}^p(0)$, and $\mathbf{\tilde Y}^p(\Delta+1)$
for \ac{CSIR} acquisition.
We are also interested in determining the distribution of the
resulting channel estimation error, whose covariance matrix, denoted by
$\mx{Z}(\Delta,i)$, will play an important role in subsequently determining
the deterministic equivalent of the \ac{SINR}.
\vspace{-2mm}
\subsection{\ac{MMSE} Channel Estimation and Channel Estimation Error}
As illustrated in Figure \ref{fig:Model},
in each data slot $i$, the \ac{BS} utilizes the \ac{MMSE} estimates of the channel
obtained in the neighboring pilot slots, for example at $(-\Delta-1)$, $0$ and $(\Delta+1)$, %according to \eqref{eqn:received_training_seq2}.
using the respective received pilot signals according to \eqref{eqn:received_training_seq2}, that is
$\mathbf{\tilde Y}^p\big((-\Delta-1)\big)$, $\mathbf{\tilde Y}^p(0)$ and $\mathbf{\tilde Y}^p\big((\Delta+1)\big)$, using the following lemma.

\begin{lem}
\label{lem:mmsechannel}	
The \textup{MMSE} channel estimator approximates the autoregressive fast fading channel in time slot $i$
based on the received pilots at $(-\Delta-1)$, $0$ and $(\Delta+1)$ as
\begin{align}
\label{eq:hmmse}
&\mathbf{\hat h}_{\textup{MMSE}}(\Delta,i)=
\mx{H^\star}(\Delta,i) \mathbf{\hat Y}^p(\Delta),
\end{align}
where
\begin{align*}
\mx{H^\star}(\Delta, i) = \frac{1}{\alpha \sqrt{P_p} \tau_p} \mx{E}(\Delta,i). \big(\mx{M}(\Delta)+\mx{\Sigma}_3\big)^{-1}.(\mx{s}^H\otimes \mx{I}_{3N_r}),
\end{align*}

$\mathbf{\hat Y}^p(\Delta) \triangleq \begin{bmatrix}
\mathbf{\tilde Y}^p((-\Delta-1)) \\
\mathbf{\tilde Y}^p(0) \\
\mathbf{\tilde Y}^p((\Delta+1))
\end{bmatrix}$~~and~~$\mx{\Sigma}_3\triangleq \frac{\sigma_p^2}{\alpha^2P_p \tau_p} \mx{I}_{3N_r}$,
\begin{align}
\label{eq:E}
\mx{E}(\Delta,i) &\triangleq
\begin{bmatrix}
\mx{R}(\Delta \!+\! 1 \!+\! i) & \mx{R}(i) &
\mx{R}(\Delta \!+\! 1 \!-\! i)
\end{bmatrix},
\\
\label{eq:M}
\mx{M}(\Delta) &\triangleq
\begin{bmatrix}
\mx{C} & \mx{R}(\Delta \!+\! 1) & \mx{R}(2\Delta \!+\! 2)  \\
\mx{R}^H(\Delta \!+\! 1) & \mx{C} & \mx{R}(\Delta \!+\! 1) \\
\mx{R}^H(2\Delta \!+\! 2) & \mx{R}^H(\Delta \!+\! 1) & \mx{C}\\
\end{bmatrix}.
\end{align}

\end{lem}

\begin{proof}
The \ac{MMSE} channel estimator aims at minimizing the \ac{MSE} between the channel estimate
$\mathbf{\hat h}_{\textrm{MMSE}}(\Delta, i) = \mathbf{H}^\star(\Delta, i) \mathbf{\hat Y}^p(\Delta)$ and the channel $\mathbf{h}(i)$, that is
\begin{align}
\mathbf{H^\star}(\Delta, i)= \text{arg} \min_{\mathbf{H}} \mathds{E}_{\mathbf{h},\mathbf{n}}\{ ||\mathbf{H} \mathbf{\hat Y}^p(\Delta) - \mathbf{h}(i)||^2 \}.
\end{align}
The solution of this quadratic optimization problem is
$\mathbf{H^\star}(\Delta, i)= \mx{a}(\Delta, i)^H \mx{F}^{-1}(\Delta)$
with
\begin{align}
\label{eq:Fandb}
\mx{F}(\Delta) &\triangleq \mathds{E}_{\mathbf{h},\mathbf{n}}\left(\mx{\hat Y}^p(\Delta) \left(\mx{\hat Y}^p(\Delta)\right)^H\right), \\
\mx{a}(\Delta, i) &\triangleq \mathds{E}_{\mathbf{h},\mathbf{n}}\left( \mx{\hat Y}^p(\Delta) \mx{h}^{H}(i)\right).
\end{align}
Let
\begin{align}
\mx{\bar h}(\Delta) \triangleq \begin{bmatrix}
\mx{h}((-\Delta-1)) \\
\mx{h}(0) \\
\mx{h}((\Delta+1))
\end{bmatrix} \nonumber
\end{align}
and
\begin{align}
\mx{\tilde{\bar{N}}}(\Delta) \triangleq \begin{bmatrix}
\mx{\tilde{N}}((-\Delta-1)) \\
\mx{\tilde{N}}(0) \\
\mx{\tilde{N}}((\Delta+1))
\end{bmatrix}.
\end{align}
Using $\mx{\bar h}(\Delta)$, we have
\begin{align}
\begin{bmatrix}
\mx{S}\mx{h}((-\Delta-1)) \\
\mx{S}\mx{h}(0) \\
\mx{S}\mx{h}((\Delta+1))
\end{bmatrix}=
\begin{bmatrix}
\mx{S}  & \mx{0} & \mx{0} \\
\mx{0} & \mx{S}  & \mx{0} \\
\mx{0} & \mx{0} & \mx{S}
\end{bmatrix}\mx{\bar h}(\Delta)
\nonumber \\
~~~~~~~~= (\mx{I}_3\otimes\mx{S}) \mx{\bar h}(\Delta)=  (\mx{s}\otimes \mx{I}_{3N_r}) \mx{\bar h}(\Delta).
\end{align}
Since $\mx{\bar h}(\Delta)$ and $\mx{\tilde{\bar{N}}}(\Delta)$ are independent and
\begin{align}
\mathds{E}_{\mathbf{h},\mathbf{n}}(\mx{\bar h}(\Delta)\mx{\bar h}^H(\Delta))&=\mx{M}(\Delta),\\
\mathds{E}_{\mathbf{h},\mathbf{n}}(\mx{h}(i)\mx{\bar h}^H(\Delta))&=\mx{E}(\Delta,i).
\end{align}
Therefore, for $\mx{F}(\Delta)$ and $\mx{a}(\Delta,i)$, we have
\begin{align*}
\mx{F}(\Delta) &=
%\mathds{E}_{\mathbf{h},\mathbf{n}}\left(\mx{\tilde{\bar{Y}}}^p\mbox{$\mx{\tilde{\bar{Y}}}^p$}^H\right)\\&=
\mathds{E}_{\mathbf{h},\mathbf{n}}
\left(\left(\alpha\sqrt{P_p} (\mx{I}_3\otimes\mx{S}) \mx{\bar{h}}(\Delta) +\mx{\tilde{\bar{N}}}(\Delta)\right)\right.\nonumber \\
&~~~~~~~~~~~~\cdot \left.\left(\alpha\sqrt{P_p} (\mx{I}_3\otimes\mx{S}) \mx{\bar{h}}(\Delta) +\mx{\tilde{\bar{N}}}(\Delta)\right)^H\right)\\
&=\alpha^2P_p(\mx{I}_3\otimes\mx{S})\mx{M}(\Delta)\left(\mx{I}_3\otimes\mx{S}^H\right)+ \sigma_p^2 \mathbf{I}_{3N_r\tau_p},\\
\mx{a}(\Delta, i) &= \mathds{E}_{\mathbf{h},\mathbf{n}}\left( \mx{\hat Y}(\Delta) \mx{h}^{H}(i)\right) \\
&=\mathds{E}_{\mathbf{h},\mathbf{n}}
\left(\left(\alpha\sqrt{P_p} (\mx{I}_3\otimes\mx{S}) \mx{\bar{h}}(\Delta) +\mx{\tilde{\bar{N}}}(\Delta)\right)
\mx{h}^{H}(i)\right)\\
&= \alpha\sqrt{P_p} ~(\mx{I}_3\otimes\mx{S})~\mx{E}(\Delta,i)^T,
\end{align*}
which yields Lemma \ref{lem:mmsechannel}.
\end{proof}

The \ac{MMSE} estimate of the channel is then expressed as:
\begin{align}
\nonumber
&\mathbf{\hat h}_{\textrm{MMSE}}(\Delta, i)=\mathbf{H^\star}(\Delta, i) \mathbf{\hat Y}^p(\Delta) ~~~~ \\
&~~~~=\mathbf{H^\star}(\Delta, i)
\left(\alpha\sqrt{P_p} (\mx{I}_3\otimes\mx{S}) \mx{\bar{h}}(\Delta) +\mx{\tilde{\bar{N}}}(\Delta)\right)
\nonumber
\\
&~~~~=
\frac{1}{\alpha\sqrt{P_p} \tau_p}
\mx{E}(\Delta, i) \left(   \mx{M}(\Delta) + \mx{\Sigma}_{3} \right)^{-1}
\nonumber
\\&~~~~~~~~~~.
\left(\alpha\sqrt{P_p} \tau_p \mx{\bar{h}}(\Delta) + \left(\mx{I}_3 \otimes \mx{S}^H\right) \mx{\tilde{\bar{N}}}(\Delta)\right).
\label{eq:MMSEt}
\end{align}

Next, we are interested in deriving the distribution of the estimated channel and
the channel estimation error, since these will be important for understanding the
impact of pilot spacing on the achievable \ac{SINR} and spectral efficiency of
the \ac{MU-MIMO} system. To this end, the following two corollaries of
Lemma \ref{lem:mmsechannel} and \eqref{eq:MMSEt} will be important in the sequel.
\begin{cor}
\label{cor:rmmse}	
The estimated channel $\mathbf{\hat h}_{\textup{MMSE}}(\Delta, i)$ is a circular symmetric complex normal distributed vector
$\mathbf{\hat h}_{\textup{MMSE}}(\Delta, i) \sim %\mathcal{CN}(\mathbf{0},\mathbf{R}_{\textup{MMSE}}(\Delta, i))$,
\mathcal{CN}\Big(\mathbf{0},\mathbf{\hat \Phi}_{\textup{MMSE}}(\Delta, i)\Big)$,
with
\begin{align}
\label{eq:rmmse}
\mathbf{\hat \Phi}_{\textup{MMSE}}(\Delta, i) \triangleq
& \mathds{E}_{\mathbf{h},\mathbf{n}} \{\mathbf{\hat h}_{\textup{MMSE}}(\Delta, i) \mathbf{\hat h}_{\textup{MMSE}}^H(\Delta, i)\}
 \nonumber \\
=& \mx{E}(\Delta, i) \big(\mx{M}(\Delta)+\mx{\Sigma}_3\big)^{-1} \mx{E}(\Delta, i)^H.
\end{align}

\end{cor}

\begin{proof}
Equation \eqref{eq:rmmse} follows directly from \eqref{eq:MMSEt}.	
\end{proof}

An immediate consequence of Corollary \ref{cor:rmmse}
is the following corollary regarding the covariance of the channel estimation
error, as a function of pilot spacing.
\begin{cor}
\label{Cor:ChEstError}
The channel estimation error in slot $i$, $\mx{\hat{h}}_{\textup{MMSE}}(\Delta, i)-\mx{h}(\Delta, i)$,
is complex normal distributed with zero mean vector and
covariance matrix given by:
\begin{align}
\label{eq:Z}
\mx{Z}(\Delta, i) \triangleq  \mx{C} -
\mx{E}(\Delta, i) \big(\mx{M}(\Delta)+\mx{\Sigma}_3\big)^{-1}
\mx{E}(\Delta, i)^H.
\end{align}
\end{cor}

In the following section we will calculate the \ac{SINR} of the received data symbols.
For simplicity of notation, we use $\mx{\hat h}_{\textup{MMSE}}(\Delta, i)=\mx{\hat h}(\Delta, i)$, and  introduce
$$\mx{b}(\Delta, i) \triangleq \alpha\sqrt{P(i)}\mx{\hat{h}}(\Delta, i)$$
with covariance matrix
\begin{align}
\label{eq:Phi}
\bs{\Phi}(\Delta, i) & \triangleq
\mathds{E}\left( \mx{b}(\Delta, i)\mx{b}^H(\Delta, i) \right)  \nonumber \\
&=\mathds{E}\left( \Big(\alpha\sqrt{P(i)} \mx{\hat{h}}(\Delta, i) \Big)\Big(\alpha\sqrt{P(i)} \mx{\hat{h}}(\Delta, i) \Big)^H  \right)  \nonumber \\
&=\alpha^2P(i)(\mx{C}-\mx{Z}(\Delta, i)).
\end{align}
% \dg{For consistency, shouldn't we use $P_i$ instead of $P(i)$?}
\subsection{Summary}

This section derived the \ac{MMSE} channel estimator (Lemma \ref{lem:mmsechannel})
that uses the received pilot signals
both before and after a given data slot $i$ and depends on the frame size $\Delta$ (pilot spacing).
As important corollaries of the channel estimation scheme, we established the distribution
of both the estimated channel (Corollary \ref{cor:rmmse})
and the associated channel estimation error in each data
slot $i$ (Corollary \ref{Cor:ChEstError}), as functions of both the employed pilot spacing and pilot power. These
results serve as a starting point for deriving the achievable \ac{SINR} and spectral
efficiency.

\section{\ac{SINR} Calculation}
\label{Sec:SINR}

\subsection{Instantaneous \ac{SINR}}
We start with recalling an important lemma from \cite{Fodor:22}, which calculates the instantaneous \ac{SINR}
in an \ac{AR} fast fading environment
when the \ac{BS} uses the \ac{MMSE} estimation of the fading channel, and employs the optimal linear
receiver:
\begin{align}
\label{eq:Gstar2}
\mx{G}^\star(\Delta, i) &=
% \textup{arg} \min_{\mx{G}} \textup{MSE}\left(\mx{G},\mx{\hat H}(t), \mx{\hat H}(t-1)\right)
\mx{b}^H(\Delta,i) \mx{J}^{-1}(\Delta,i),
\end{align}
where $\mx{J}(\Delta, i) \in \mathds{C}^{N_r \times N_r}$
is defined as
\begin{align}
%\label{eq:A3}
\nonumber
\mx{J}(\Delta, i)
&\triangleq
\sum_{k=1}^K  \mx{b}_k(\Delta, i) \mx{b}_k^H(\Delta, i) + \bs{\beta}(\Delta, i),
\end{align}
where
\begin{align}
\label{eq:beta}
\bs{\beta}(\Delta, i) \triangleq \sum_{k=1}^K \alpha_k^2 P_k  \mx{Z}_k(\Delta, i) + \sigma^2_d \mx{I}_{N_r}.
\end{align}
%and $\mx{I}$ is the identity matrix of size $N_r$.

When using the above receiver, which minimizes the \ac{MSE} of the received data symbols in the
presence of channel estimation errors, the following result from \cite{Fodor:22} will be useful in the sequel:
\begin{lem}[See \cite{Fodor:22}, Lemma 3]
Assume that
the receiver employs \textup{\ac{MMSE}} symbol estimation,
that is it employs the optimal linear receiver $\mx{G}^\star(\Delta, i)$ given in \eqref{eq:Gstar2}. %and the instantaneous channel estimates,
Then the instantaneous \textup{\ac{SINR}} of the estimated data symbols of the tagged user,
$\gamma(\Delta, i)$ %\Big(\mathbf{G}^\star, \hat{\mathbf{H}}(t),\hat{\mathbf{H}}(t-1)\Big)$
is given as:
% the instantaneous \textup{\ac{SINR}} of the data symbols, can be expressed as:
%
\begin{equation}
\label{eq:lemma2Eq}
\gamma(\Delta, i) %\Big(\mathbf{G}^\star(t), \hat{\mathbf{H}}(t),\hat{\mathbf{H}}(t-1)\Big)
=
\mx{b}^H(\Delta, i) \mathbf{\bar{J}}^{-1}(\Delta, i) \mx{b}(\Delta, i),
\end{equation}
where
\vspace{-1mm}
\begin{equation}
\mathbf{\bar{J}}(\Delta, i) \triangleq \mathbf{J}(\Delta, i)- \mx{b}(\Delta, i)\mx{b}^H(\Delta, i).
\end{equation}
\end{lem}
\begin{comment}
Using the \ac{MMSE} estimate $\bar{\mx{h}}(i) = \mx{E}(i) \mx{\hat{\bar{h}}}(\Delta)$ of the channel $\mx{h}(i)$,
we can calculate the \ac{SINR} of the estimated data symbol sent at time $i$ when the \ac{BS} employs an optimal linear receiver as provided in \cite{Fodor:22}.
Lemma 2 in \cite{Fodor:22} shows that when $K$ \acp{MS} enumerated $k = 1 \ldots K$ transmit data signals
through channels $\mx{h}_k$ with \ac{MMSE} estimates
$\bar{\mx{h}}_k$, estimation error covariance $\mx{Z}_k$, path losses $\alpha_k$,
and employing transmit powers $P_k$, the \ac{SINR} of the estimated data symbol of user $k$, $\hat x = \mx{G}\mx{y}$:

\begin{align}
    \gamma_k = \alpha_k^2 P_k \bar{\mx{h}}_k^H \left( \sum_{l \neq k}^K \alpha_l^2 P_l \bar{\mx{h}}_l\bar{\mx{h}}_l^H
    + \sum_{l = 1}^K \alpha_l^2 P_l \mx{Z}_l + \sigma_d^2 \mx{I} \right)^{-1}\bar{\mx{h}}_k,
\end{align}

where $\sigma_d^2$ is the variance of the \ac{AWGN} on the data signal.
Applying this formula on the previous section, where the data symbol of User-$k$, in data time slot $i$ is transmitted through the channel $\mx{h}_k(iT)$,
with \ac{MMSE} estimate $\bar{\mx{h}}_k(iT) = \mx{E}_k(i) \bs{\zeta}_k(\Delta)$ and channel covariance matrix $\mx{Z}(i)$
the \ac{SINR} of the estimated data symbol sent by User-$k$ is given by
\end{comment}
For the \ac{AR} fading case considered in this paper,
based on the definitions of $\mx{b}(\Delta, i)$, $\mx{J}(\Delta, i)$ and $\mx{\bar{J}}(\Delta, i)$,
the instantaneous \ac{SINR} of the tagged user is then expressed as:
\begin{align}
\gamma(i) &  =  \mx{b}^H(\Delta, i) \mathbf{\bar{J}}^{-1}(\Delta, i) \mx{b}(\Delta, i)
  \nonumber\\
 &=
\textup{tr}\left( \mx{b}(\Delta, i)\mx{b}^H(\Delta, i) \mathbf{\bar{J}}^{-1}(\Delta, i) \right).
\end{align}

\subsection{Slot-by-Slot Deterministic Equivalent of the \ac{SINR} as a Function of Pilot Spacing $\Delta$}
We can now prove the following important proposition that gives the asymptotic deterministic equivalent
of the instantaneous \ac{SINR} in data slot $i$, $\bar{\gamma}(\Delta, i)$, when the number of antennas $N_r$ approaches infinity.
This asymptotic equivalent \ac{SINR} gives a good approximation of averaging the instantaneous \ac{SINR} of the tagged user \cite{Wagner:2012, Hoydis:13, Fodor:22}.

\begin{prop}
\label{Prop:SINR}
% According to this theorem, for User-$1$ the \ac{SINR} is given by
The asymptotic deterministic equivalent \ac{SINR} of the tagged user in data slot $i$ can be calculated as:
\begin{align}
\label{eq:hoydis1}
\bar{\gamma}(\Delta, i) &= \textup{tr}\Big( \bs{\Phi}(\Delta, i)\mx{T}(\Delta, i)  \Big),
\end{align}
where $\mx{T}(\Delta, i)$ is defined as: %given by
\begin{align}
\label{eq:hoydis2}
\mx{T}(\Delta, i) \triangleq \left( \sum_{m = 2}^{K} \frac{\bs{\Phi}_m(\Delta, i)}{1 + \delta_m(\Delta, i)} + \bs{\beta}(\Delta, i) \right)^{-1},
\end{align}
and $\delta_m(\Delta, i)$ are the solutions of the following system of $K$ equations
\begin{align}
\label{eq:hoydis3}
\delta_m(\Delta, i) &= \textup{tr}\left(\bs{\Phi}_m(\Delta, i)
\left( \sum_{l = 2}^{K} \frac{\bs{\Phi}_l(\Delta, i)}{1 + \delta_l(\Delta, i)} + \bs{\beta}(\Delta, i)  \right)^{-1} \right)
\end{align}
for $\forall m = 1, \ldots, K$.
\end{prop}
The above system of $K$ equations gives the deterministic equivalent of the \ac{SINR} of the tagged user,
and a different set of $K$ equations must be used for each user.
\begin{proof}
The $\mx{b}_k(\Delta, i)$ vectors are independent for $k = 1 \ldots K$,
and the covariance matrix of $\mx{b}_k(\Delta, i)$ is $\bs{\Phi}_k(\Delta, i)$
(c.f. \eqref{eq:Phi}).
We can then express the expected value of the \ac{SINR} of the tagged user as follows:
\begin{align}
\label{eq:gammaB}
& \bar{\gamma}(\Delta, i) \triangleq \mathds{E}\Big(\gamma(\Delta, i)\Big)   \\
&= \mathds{E}\left( \textup{tr}\left( \bs{\Phi}(\Delta, i)  \left( \sum_{l = 2}^K  \mx{b}_l(\Delta, i) \mx{b}_l^H(\Delta, i)  +\bs{\beta}(\Delta, i) \right)^{\!\!-1}  \right) \right).
\nonumber
\end{align}
%We proceed
The proposition is established
by invoking Theorem \ref{thm:upperbound} in \cite{Wagner:2012}, which is applicable in multiuser systems and
% Theorem \ref{thm:upperbound} in \cite{Wagner:2012}
gives the value of the deterministic equivalent of $\bar{\gamma}(\Delta, i)$
implicitly using a system of $K$ equations and noticing that
$\bar{\gamma}(\Delta, i) = \delta_1(\Delta, i)$, since $\delta_1(\Delta, i)=\textup{tr}\big( \bs{\Phi}(\Delta, i) \mx{T}(\Delta, i)\big)$ according to \eqref{eq:hoydis3}.
\end{proof}

\subsection{Summary}
This section established the instantaneous slot-by-slot \ac{SINR} of a tagged user ($\bar \gamma(i)$) of a \ac{MU-MIMO} system operating over %continuous time
a fast fading channels modelled as \ac{AR} processes,
by applying our previous result obtained for discrete-time \ac{AR} channels
reported in \cite{Fodor:21}.
Next, we invoked Theorem \ref{thm:upperbound} in \cite{Wagner:2012}, to establish the deterministic
equivalent \ac{SINR} for each slot, as a function of the frame size (pilot spacing) $\Delta$, see Proposition \ref{Prop:SINR}.
These results serve as a basis for formulating the pilot spacing optimization problem over the frame size and pilot power as optimization variables.

\section{Pilot Spacing and Power Control}
\label{Sec:PilotSpacing}
In this section, we study the impact of pilot spacing and power control on the achievable \ac{SINR}
and the \ac{SE} of all users in the system.
The asymptotic \ac{SE} of the $i$-th data symbol of user $k$ is
\begin{align}
\label{eq:SE}
\text{SE}_k(\Delta, i) \triangleq \log\Big(1 + \bar{\gamma}_k(\Delta, i)\Big),
\end{align}
where $\bar{\gamma}_k(\Delta, i)$ denotes the average \ac{SINR} of  user $k$ when sending the $i$-th data symbol, and when $\Delta$ data symbols are sent between every pair of pilot symbols.
Consequently, the %information rate
average \ac{SE}
of  user $k$ over the $(\Delta+1)$ slot long frame is
\begin{align}
\frac{\sum_{i=1}^{\Delta}\text{SE}_k(\Delta, i)}{\Delta + 1},
\end{align}
which can be optimized over $\Delta$.
More importantly, %the sum information rate
the aggregate average \ac{SE}
of the \ac{MU-MIMO} system for the $K$ users can be expressed as:
\begin{align}
\text{SE}(\Delta) = \frac{\sum_{k=1}^K \sum_{i=1}^{\Delta}\text{SE}_k(\Delta,i)}{\Delta + 1}.
\label{eq:multiopt}
\end{align}


\subsection{An Upper Bound of the Deterministic Equivalent \ac{SINR} and the \ac{SE}}

Let us assume that $\mx{Q}_k=q_k \mx{I}_{N_r}$, that is the
channel vector $\mx{h}_k(t)$ consists of independent \ac{AR} processes in the spatial domain, implying that:
\begin{align}
\label{eq:r1}
\mx{R}_k(i) \triangleq
\mathds{E}\left(\mx{h}_k(t)\mx{h}_k^H(t+i)\right) &= \mx{C}_k e^{q_k^* i},
\end{align}
where $q_k$ is a scalar, $q_k^*$ denotes complex conjugation, and let $\bar{q}_k\triangleq Re(q_k)<0$.

Note that the exponential approximation of the autocorrelation function of the fast fading process expressed
in \eqref{eq:r1} is related to the Doppler frequency of Rayleigh fading through:
\begin{align}
\label{eq:RayleighApprox}
\underbrace{\mx{C}J_0(2 \pi f_D i)}_{\text{True autocorrelation of Rayleigh fading}} &\approx \mx{R}(i),
\end{align}
where $J_0(.)$ is the zeroth order Bessel function \cite{Wang:03}.
Based on the exponential approximation of this Rayleigh fading process in \eqref{eq:r1}, the Doppler frequency of the approximate model is obtained from $2 \pi f_D i = Re(q_k^* i)$, i.e. $f_D = 2 \pi/\bar{q}_k$.

To optimize \eqref{eq:multiopt}, we first find an upper bound of $\text{SE}_k(\Delta,i)$
via an upper bound of $\bar{\gamma}_k(\Delta, i)$.
To simplify the notation, the following discussion refers to the tagged user, and later we utilize that the same relations hold for all users.
We introduce the following upper bound of $\bar{\gamma}(\Delta, i)$:
\begin{align}
\label{eq:upper}
\bar{\gamma}^{(u)}(\Delta, i) \!&\triangleq \textup{tr}\!\left(\! \bs{\Phi}^{(u)}(\Delta, i)  \left(\sum_{l = 1}^K \!\alpha_l^2 P_l \mx{Z}^{(u)}_l(\Delta, i) \!+\! \sigma_d^2 \mx{I_{N_r}}\right)^{\!\!\!-1}  \!\right)\!,
\end{align}
where $\mx{Z}^{(u)}(\Delta, i)$ and $\bs{\Phi}^{(u)}(\Delta, i)$ are given by
\begin{align}
\label{eq:zu}
\mx{Z}^{(u)}(\Delta, i) &\triangleq \mx{C} - \rho(\Delta, i) \mx{C} \left(\eta  \mx{C} + \bs{\Sigma} \right)^{-1} \mx{C},  \\
\label{eq:phiu}
\bs{\Phi}^{(u)}(\Delta, i) &\triangleq \alpha^2 P \rho(\Delta, i) \mx{C} \left( \eta \mx{C} + \bs{\Sigma} \right)^{-1} \mx{C},
\end{align}
with $\eta$ being a constant, $\mx{\Sigma}\triangleq \frac{\sigma_p^2}{\alpha^2P_p \tau_p} \mx{I}_{N_r}$ and
\begin{align}
\label{eq:rho}
%\mx{\Sigma} &\triangleq \frac{\sigma_p^2}{\alpha^2P_p %\tau_p} \mx{I}_{N_r} \\
\rho(\Delta, i) &\triangleq e^{2\bar{q} (\Delta + 1 + i)} + e^{2\bar{q}  i} + e^{2\bar{q} (\Delta + 1 - i)}.
\end{align}

\begin{thm}
\label{thm:upperbound}
If $\bar{q}<0$ and
\begin{align}
\label{eq:etacond}
0<\eta<\frac{1}{2}\left(2+a^2-a\sqrt{8+a^2}\right),
\end{align}
with $a\triangleq e^{2\bar{q}}$
then  $\bar{\gamma}(\Delta, i)\leq \bar{\gamma}^{(u)}(\Delta, i)$.
\end{thm}
%\overset{(a)}{=}
\begin{proof}
We prove the theorem based on the following inequalities
\begin{align}
\bar{\gamma}(\Delta, i)
&\overset{(a)}{\leq}
\textup{tr}\left( \bs{\Phi}(\Delta, i) \bs{\beta}(\Delta, i)^{-1} \right) \nonumber \\
&\overset{(b)}{\leq}
   \textup{tr}\left( \bs{\Phi}^{(u)}(\Delta, i) \bs{\beta}(\Delta, i)^{-1} \right)
\overset{(c)}{\leq}
   \bar{\gamma}^{(u)}(\Delta, i),  \label{eq:gammabound}
\end{align}
which are proved in consecutive lemmas.
\end{proof}

%The following properties of positive semi-definite matrices are essential to prove the theorem.
\begin{lem}
\label{lem:AB}
Let $\mx{A}$, $\mx{B}$ and $\mx{C}$ be positive definite matrices and $\mx{D}$ be any matrix, such that $\mx{A} \preceq \mx{B}$ (i.e. $\mx{B}-\mx{A}$ is a positive semidefinite matrix), then
\begin{align}
\label{eq:ABinv}
\mx{A}^{-1} &\succeq \mx{B}^{-1},
\\
\label{eq:AB3}
\textup{tr}\left( \mx{D}^H \mx{A} \mx{D} \right) &\leq \textup{tr}\left(\mx{D}^H \mx{B} \mx{D} \right) \\
\label{eq:AB1}
\textup{tr}\left( \mx{A} \mx{C} \right) &\leq \textup{tr}\left(\mx{B} \mx{C} \right)
\\
\label{eq:AB2}
\textup{tr}\left( \mx{C} \mx{A}^{-1} \right) &\geq \textup{tr}\left( \mx{C} \mx{B}^{-1} \right).
\end{align}
\end{lem}
\begin{proof}
$\mx{A}^{-1} \succeq \mx{B}^{-1}$ is given in \cite[p. 495, Corollary 7.7.4(a)]{Horn2013}.
\eqref{eq:AB3} follows from the fact that $\mx{D}^H (\mx{B}-\mx{A}) \mx{D}$ is a positive semidefinite matrix since $\mx{B}-\mx{A}$ is a positive semidefinite matrix and
for any $\mx{x}$
\begin{align}
\mx{x}^H \mx{D}^H (\mx{B}-\mx{A}) \mx{D} \mx{x} = \mx{y}^H (\mx{B}-\mx{A}) \mx{y} \geq 0
\end{align}
where $\mx{y} \triangleq \mx{D} \mx{x}$.
Let $\mx{C}=\mx{D}^H \mx{D}$ be the Cholesky decomposition of $\mx{C}$ then
\eqref{eq:AB1} and \eqref{eq:AB2} follows from \eqref{eq:AB3}, by utilizing the cyclic property of the trace operator.
\end{proof}

\begin{lem}
\label{lem:beta}
For $\bar{q}<0$ and $\eta$ satisfying \eqref{eq:etacond},
the following relation holds
\begin{align}
&\mx{E}(\Delta, i) \big(\mx{M}(\Delta,i)+\mx{\Sigma}_3\big)^{-1} \mx{E}(\Delta, i)^H
\nonumber \\&
~~~~~~~~~~~~~~\preceq \rho(\Delta, i) \mx{C} \left( \eta \mx{C} \!+\! \bs{\Sigma} \right)^{-1} \mx{C}
\label{eq:mrel}
\end{align}
\end{lem}
\begin{proof}
The proof is in Appendix \ref{Sec:L4}.
\end{proof}

Having prepared with Lemma \ref{lem:AB} and Lemma \ref{lem:beta}, we can prove the (a), (b) and (c) inequalities in \eqref{eq:gammabound} by Lemma \ref{lem:det} ((a) part) and Lemma \ref{lem:ineq} ((b) and (c) parts) as follows.
\begin{lem}
\label{lem:det}
The deterministic equivalent \ac{SINR} of the tagged user satisfies
\begin{align}
& \bar{\gamma}(\Delta, i) \leq   \textup{tr}\left( \bs{\Phi}(\Delta, i) \bs{\beta}(\Delta, i)^{-1} \right).
\nonumber
\end{align}
\end{lem}
\begin{proof}
The proof is in Appendix \ref{Sec:L5}.
\end{proof}

%We can now state the following lemma, which completes the proof of inequality \eqref{eq:gammabound} by proving the (b) and (c) inequalities of \eqref{eq:gammabound}.
\begin{lem}
\label{lem:ineq}
When the conditions of Theorem \ref{thm:upperbound} hold, we have
\begin{align}
 \textup{tr}\left( \bs{\Phi}(\Delta, i) \bs{\beta}(\Delta, i)^{-1} \right) &\leq
\textup{tr}\left( \bs{\Phi}^{(u)}(\Delta, i) \bs{\beta}(\Delta, i)^{-1} \right)
\label{eq:gammabound1}
\\
  \textup{tr}\left( \bs{\Phi}^{(u)}(\Delta, i) \bs{\beta}(\Delta, i)^{-1} \right)
&\leq \bar{\gamma}^{(u)}(\Delta, i).
\label{eq:gammabound2}
\end{align}
\end{lem}

\begin{proof}
When the conditions of Theorem \ref{thm:upperbound} hold,
Lemma \ref{lem:beta} implies that
 $\bs{\Phi}(\Delta, i)\preceq \bs{\Phi}^{(u)}(\Delta, i)$
 and
$\mx{Z}(\Delta, i) \succeq \mx{Z}^{(u)}(\Delta, i)$.
Using the first relation and %Lemma \ref{lem:AB}
the Lemma \ref{lem:AB}
gives \eqref{eq:gammabound1},
while using the second relation and Lemma \ref{lem:AB} gives \eqref{eq:gammabound2}.
\end{proof}

%\gf{\it GF16: Maybe introducing $\bs{\beta}^{(u)}(\Delta,i) \succeq \bs{\beta}(\Delta,i)$ would simplify the presentation of the proof.}

\subsection{Useful Properties of the Upper Bounds on the Deterministic Equivalent \ac{SINR} and Overall System Spectral Efficiency}

Theorem \ref{thm:upperbound} is useful, because it establishes an upper bound,
denoted by $\bar{\gamma}^{(u)}(\Delta, i)$,
of the deterministic equivalent of the \ac{SINR},
$\bar{\gamma}(\Delta, i)$.

To use the $\bar{\gamma}^{(u)}(\Delta, i)$ upper bound for limiting the search space for an optimal $\bar{\gamma}(\Delta, i)$ in Section \ref{Sec:Alg}, we need the following properties of the upper bound.

%Clearly, the deterministic equivalent \ac{SINR} is not necessarily a monotonic function of $\Delta$. However, we can now prove that the upper bound established by Theorem \ref{thm:upperbound} is monotonically decreasing in $\rho$, and tends to zero as $\rho \rightarrow 0$, which properties will turn out to imply some useful properties of the overall spectral efficiency, as we will see later in Proposition \ref{UpperB2}.


\begin{prop}
\label{UpperB1}
The $\bar{\gamma}^{(u)}(\Delta, i)$ upper bound has the following properties:
$\partial \bar{\gamma}^{(u)}(\Delta, i) / \partial \rho(\Delta, i) \geq 0$ and $\rho(\Delta, i) \rightarrow 0 \Rightarrow \bar{\gamma}^{(u)}(\Delta, i) \rightarrow 0$.
\end{prop}
\begin{proof}
The proof is in Appendix \ref{Sec:P2}.
\end{proof}

Similarly, the SINR of user $k$ satisfies the inequality
$\bar{\gamma}_k(\Delta, i) \leq \bar{\gamma}_k^{(u)}(\Delta, i)$
where $\bar{\gamma}_k^{(u)}(\Delta, i)$ is defined in a similar way as $\bar{\gamma}_1^{(u)}(\Delta, i)$.
The $\bar{\gamma}_k^{(u)}(\Delta, i)$ upper bound is such that
$\partial \bar{\gamma}_k^{(u)}(\Delta, i) / \partial \rho_k(\Delta, i) \geq 0$ and $\rho_k(\Delta, i) \rightarrow 0 \Rightarrow \bar{\gamma}_k^{(u)}(\Delta, i) \rightarrow 0$.

Since our most important performance measure is the overall \ac{SE}, we are interested in establishing a corresponding upper bound on the overall \ac{SE} of the system.
% Having established an upper bound of the \ac{SINR},
To this end,
we introduce the related upper bound on the \ac{SE} of user $k$:
\begin{align}
\label{eq:SEku}
	\textup{SE}_k^{(u)}(\Delta) \triangleq \frac{ \sum_{i=1}^{\Delta}\log\big(1 + \bar{\gamma}^{(u)}_k(\Delta, i) \big) }{\Delta}.
\end{align}
and bound the aggregate average \ac{SE} of the \ac{MU-MIMO} system (c.f. \eqref{eq:multiopt}).
Notice that the denominator in $\textup{SE}_k^{(u)}$ is $\Delta$ while the denominator in $\textup{SE}_k$ is $\Delta + 1$.
This will be necessary for the monotonicity property in Proposition \ref{UpperB2}.
\begin{prop}
\label{UpperB2}
	\begin{align}
	\label{eq:SEupper}
	\textup{SE}^{(u)}(\Delta) \triangleq \sum_{k=1}^K \textup{SE}_k^{(u)}(\Delta)
	\geq \textup{SE}(\Delta),
	\end{align}
	and $\textup{SE}^{(u)}(\Delta)$ decreases with $\Delta$ and approaches $0$ when $\Delta$ approaches infinity.
\end{prop}

\begin{proof}
The proof is in Appendix \ref{Sec:P3}.
\end{proof}

\subsection{Summary}
This section first established an upper bound on the deterministic equivalent \ac{SINR} in Theorem \ref{thm:upperbound}. Next, Proposition \ref{UpperB1} and Proposition \ref{UpperB2} have stated some useful properties of this upper bound and a corresponding upper bound on the overall system spectral efficiency. Specifically, Proposition \ref{UpperB2} suggests that the upper bound on the spectral efficiency of the system is monotonically decreasing in $\Delta$ and tends to zero as $\Delta$ approaches infinity. As we will see in the next section, this property can be exploited to limit the search space for finding the optimal $\Delta$.

\section{A Heuristic Algorithm to Find the Optimum Pilot Power and Frame Size (Pilot Spacing)}
\label{Sec:Alg}
\subsection{A Heuristic Algorithm for Finding the Optimal $\Delta$}
In this section we build on the property of the system-wide spectral efficiency, as stated by Proposition \ref{UpperB2}, to develop a heuristic algorithm to find the optimal $\Delta$. While we cannot prove a convexity or non-convexity property of $\text{SE}(\Delta)$, we can utilize the fact that
$\text{SE}(\Delta) \leq \text{SE}^{(u)}(\Delta)$ as follows.
As Algorithm \ref{alg:Dynamic_Game}
 scans through the possible values of $\Delta$, it checks if the current best $\Delta$ (that is $\Delta_{\textup{opt}}$) is one less than the currently examined $\Delta$ (Line 17).
As it will be exemplified in Figure \ref{fig:6} in the numerical section, the key is to notice that the SE upper bound
determines the search space of the possible $\Delta$ values, where the associated SE can possibly exceed the currently found highest SE. Specifically, the search space can be limited to (Line 18):
\begin{align}
\Delta_{\textup{max}} &= \textup{SE}^{{(u)}^{-1}}(\textup{SE} _{\textup{$\Delta$}}),
\end{align}
where $\textup{SE}^{{(u)}^{-1}}$ denotes the inverse function of $\textup{SE}^{(u)}(.)$ and
$\textup{SE}_\Delta \triangleq \textup{SE}(\Delta)$ as calculated in \eqref{eq:multiopt}.

{\small
\begin{algorithm}[t!]
  %\algsetup{linenosize=\small}
  %\scriptsize
\DontPrintSemicolon
\caption{Optimum frame size algorithm using an SE upper bound}
\label{alg:Dynamic_Game}
%\SetAlgoLined
\KwIn{%\ac{SE} improvement threshold $\epsilon$,
$\mx{Q}$, %$T$,
$\mx{C}$, $\mx{\Sigma}$, $\alpha^2$, $P_{\text{tot}}$ %, $\Delta_\text{incr}$
}
% Initial data power $P_{\lambda,\kappa}^\star(\mathbf{0})$\\
% $i=0$\\
%$\text{SE}_\text{prev}=0$,
$\textup{SE}_{1}=\textup{SE}(1)$ using \eqref{eq:multiopt}, $\Delta_\text{max}={\textup{SE}^{(u)}}^{-1}(\textup{SE}_{1})$ \\
$\Delta=1$, $\Delta_{\textup{opt}} = \Delta_\text{max}$,
$\textup{SE}_{\textup{opt}}=\textup{SE}(\Delta_{\textup{opt}})$ using \eqref{eq:multiopt} \\
\While{$\Delta < \Delta_\textup{max}$\hspace{2mm}}{
\For{$k = 1 \ldots K$}{
\For{$i = 1 \ldots \Delta$\hspace{2mm}}{
Calculate $\mx{R}_k(i), \mx{R}_k(\Delta+1),$ \\
~~~$\mx{R}_k(\Delta+1 \pm i),\mx{R}_k(2\Delta+2)$ using \eqref{eq:autocorrk}\\
Calculate $\mx{E}_k(\Delta,i)$ using \eqref{eq:E}\\
Calculate $\mx{Z}_k(\Delta,i)$ using \eqref{eq:Z}\\
Calculate $\mx{\Phi}_k(\Delta,i)$ using \eqref{eq:Phi}\\
Calculate $\bs{\beta}_k(\Delta,i)$ using \eqref{eq:beta}\\
Calculate $\bar \gamma_k(\Delta,i)$ using
\eqref{eq:hoydis1}\\
%Proposition \ref{Prop:SINR}\\
Calculate $\text{SE}_k(\Delta,i)$ using \eqref{eq:SE}\\}}
$\textup{SE}_\Delta=\textup{SE}(\Delta)$ using \eqref{eq:multiopt} \\
%$\textup{SE}_\textup{change}=\textup{SE}(\Delta)-\textup{SE}_\textup{prev}$\\
%$\textup{SE}_\textup{prev}=\textup{SE}(\Delta)$\\
\If{$\textup{SE}_\Delta > \textup{SE}_{\textup{opt}}$}{
    $\Delta_{\textup{opt}} = \Delta$, $\textup{SE}_{\textup{opt}}=\textup{SE}_\Delta$
    }
	\If{$\Delta_{\textup{opt}} = \Delta-1 $}%{$|\textup{SE}_\textup{change}|<\epsilon$\textup{~OR~}$\textup{SE}_\textup{change}<0$}
    {
    $\Delta_\text{max}={\textup{SE}^{(u)}}^{-1}(\textup{SE}_{\Delta})$
    }
$\Delta=\Delta+1$
}
\KwOut{$\Delta_\textup{opt}$}
\medskip
\end{algorithm}
}

\subsection{The Case of Independent and Identical Channel Coefficients}
\label{Sec:IID}

In the special case where the elements of the vector $\mx{h}(i)$ are independent stochastically identical stochastic processes, the covariance matrices become real multiples of the identity matrix
$\mx{C} \triangleq c \mx{I}_{N_r}$, $\bs{\Sigma} = s \mx{I}_{N_r}$, $\mx{R}(i) = r(i) \mx{I}_{N_r}$, $\mx{Z}(i)= z(i) \mx{I}_{N_r}$, $\bs{\Phi}(i) = \phi(i)\mx{I}_{N_r}$,
$\bs{\beta}(i) = \beta(i) \mx{I}_{N_r}$,
further more $\mx{E}(i) = \mx{e}(i) \otimes \mx{I}_{N_r}$,
with:
\begin{align}
s &\triangleq \frac{\sigma_p^2}{\alpha^2P_p \tau_p},\\
\label{eq:Rdiag}
r(i) &\triangleq ce^{q^* i},
\end{align}
%
\begin{align}
\label{eq:Ediag}
 \mx{e}(i) &\triangleq
\begin{bmatrix}r(\Delta+1+i)&r(i)&r(\Delta+1-i)\end{bmatrix}
\nonumber\\
&~~\cdot\begin{bmatrix}
c+s & r(\Delta+1) &r(2\Delta+2) \\
r^H(\Delta+1) &c+s &r(\Delta+1) \\
r^H(2\Delta+2) & r^H(\Delta+1) & c+s\\
\end{bmatrix}^{-1}
\end{align}
%
\begin{align}
\label{eq:Zdiag}
 z(i) &\triangleq
\left(c - \mx{e}(i)  \begin{bmatrix}
r^H(\Delta+1+i) \\
r^H(i) \\
r^H(\Delta+1-i) \\
\end{bmatrix}
\right),
\end{align}
%
\begin{align}
\label{eq:Phidiag}
 \phi(i)  &\triangleq \alpha^2P(i)(c-z(i)),
\end{align}
%
\begin{align}
\label{eq:Betadiag}
\beta(i) &\triangleq  \left(\sum_{k=1}^K \alpha_k^2 P_k  z_k(i) + \sigma^2_d\right).
\end{align}

In this special case, calculating the deterministic equivalent of the \ac{SINR} by Proposition \ref{Prop:SINR} simplifies to solving a set of scalar equations as stated in the following corollary.

\begin{cor}
\label{Cor:DetEq}
In this special case, the deterministic equivalent of the \textup{SINR} in slot $i$,
$\bar{\gamma}(i)$, can be obtained as the solution of the scalar equation
\begin{align}
\beta(i)  &=
\frac{N_r \phi(i)}{\bar{\gamma}(i)} - \sum_{k = 2}^K \frac{\phi_k(i)}{1 + \frac{\bar{\gamma}(i)\phi_k(i)}{\phi(i)}}.
\label{eq:diag_hoydis1}
\end{align}
\end{cor}

\begin{proof}
Since the matrices $\bs{\Phi}_k(i)$ and $\mx{Z}_k(i)$ are constant multiple of identity matrices, \eqref{eq:hoydis3} can then be rewritten as
\begin{align}
\delta_k(i) &= N_r \phi_k(i) \left( \sum_{l = 2}^{K} \frac{\phi_l(i)}{1 + \delta_l(i)} +
\beta(i) \right)^{-1}
\label{eq:diag_hoydis}
\end{align}
for $k = 1, \ldots, K$.
Using $\bar{\gamma}(i) = \delta_1(i)$ and comparing \eqref{eq:diag_hoydis} for different values of $k$ we get
\begin{align}
\delta_k(i) = \frac{ \phi_k(i)  }{  \phi_1(i)  } \delta_1(i) = \frac{ \phi_k(i)  }{  \phi_1(i) } \bar{\gamma}(i).
\label{eq:deltarealtion}
\end{align}
Substituting the rightmost expression of \eqref{eq:deltarealtion} into \eqref{eq:diag_hoydis} with $k = 1$ and rearranging gives the corollary.
\end{proof}

Notice that calculations inside the inner for loop of Algorithm \ref{alg:Dynamic_Game}, that is the
calculations in Lines 6-13 can be substituted by equations
\eqref{eq:Rdiag}, \eqref{eq:Ediag}, \eqref{eq:Zdiag}, \eqref{eq:Phidiag} and \eqref{eq:Betadiag}.

\section{Numerical Results}
\label{Sec:Num}

\begin{table}[ht]
\caption{System Parameters}
\vspace{1mm}
\label{tab:params}
\footnotesize
\begin{tabularx}{\columnwidth}{|X|X|}
		\hline
		\hline
		\textbf{Parameter}                     & \textbf{Value} \\
		\hline
		\hline
        % Network layout
        % \ac{AR} state transition matrix, $\mx{A}=a\mx{I}_{N_r}$ & $a=0, 0.1, \dots 0.95$ \\ \hline
		Number of receive antennas at the \ac{BS} antennas  & $N_r=10, 100$  \\ \hline
		Path loss of the tagged MS               & $\alpha=90$ dB \\ \hline
        Frame size                              & $\Delta=2 \dots 50$ \\ \hline
		Pilot and data power levels             & $P_p=50...125$ mW; $P=125$ mW \\ \hline
        MIMO receivers                         & MMSE receiver given by \eqref{eq:Gstar2} \\ \hline
        Channel estimation                      & MMSE channel estimation given by Lemma \ref{lem:mmsechannel} \\ \hline
        Maximum Doppler frequency               & $f_D=50, 500, 1500$ Hz \\ \hline
        Slot duration ($T$)                     & $32\mu$s \\ \hline
        Number of users                         & $K=2$ \\ \hline
		\hline
\end{tabularx}
\end{table}

In this section, we consider a single cell of a \ac{MU-MIMO} ($K=2$) system with $N_r=10$ and $N_r=100$ receive antennas,
in which the wireless channel between the served \ac{MS} and the \ac{BS} is Rayleigh fading according to \eqref{eq:RayleighApprox}, which we approximate with \eqref{eq:r1}.

The \ac{MU-MIMO} case with greater number of users ($K>2$) gives similar results
albeit with somewhat lower \ac{SINR} values from the point of view of the tagged user.
The \ac{BS} estimates the state of the wireless channel based on
the properly (i.e. $\Delta \times T$) spaced the pilot signals using \ac{MMSE} channel estimation and interpolation according to Lemma \ref{lem:mmsechannel}, and uses \ac{MMSE}
symbol estimation employing the optimal linear receiver $\mx{G}^\star(i T)$ in each slot as given in \eqref{eq:Gstar2}.
Specifically, except for the results shown in Figure \ref{fig:7},
in each time slot $i=1 \dots \Delta$, the \ac{BS} uses one pilot signal transmitted by the \ac{MS} at the
beginning of the frame at time instance $i=0$ and one pilot sent at the beginning of the next frame at time
instance $i=\Delta+1$. We refer to these two pilot signals as sent "before" and "after" time slot $i$.
In practice, the \ac{BS} can store the received data symbols until it receives the pilot signal in slot
$i=\Delta+1$ before using an \ac{MMSE} interpolation of the channel states between $i=0$ and $i=\Delta+1$.
Furthermore,
%except for the results reported in Figure \ref{fig:8},
we will assume that the \ac{BS} estimates
perfectly the autocorrelation function of the channel, including the associated maximum Doppler frequency
and, consequently, the characterizing zeroth order Bessel function.
\begin{comment}
Since many previous works used an \ac{AR}(1) approximation of the fast fading channel, Figure \ref{fig:8} examines
the case, in which the \ac{BS} assumes that the channel is AR(1) (although the actual channel is characterized by
a Doppler-dependent Bessel autocorrelation function) or when the \ac{BS} underestimates or overestimates the
actual Doppler frequency.
\end{comment}
The most important system parameters are listed in Table \ref{tab:params}.
Here we assume that the slot duration ($T$) corresponds to a symbol duration in
5G \ac{OFDM} systems using 122 MHz clock frequency, which can be used up to 20 GHz
carrier frequencies \cite{Zaidi:16}.
Note that the numerical results presented below are obtained by using the %closed form
results on the deterministic equivalent of the \ac{SINR} and the corresponding average spectral efficiency.

\begin{figure}[t]
\begin{center}
%\includegraphics[width=0.85\hsize]{cont1eps}
%\includegraphics[width=\hsize]{figura_new}
\includegraphics[width=\hsize]{ExpFig1}
\caption{
Spectral efficiency as a function of frame size ($\Delta$) with maximum Doppler frequency
$f_D=50, 500, 1500$ Hz with $N_r=10$ (lower three curves) and $N_r=100$ (upper three curves).
At higher maximum Doppler frequency, the optimum frame size is smaller than at low Doppler
frequency.
}
\label{fig:1}
\end{center}
\end{figure}

Figure \ref{fig:1} shows the achieved spectral efficiency averaged over the data slots $i=1\dots\Delta$,
that is averaged over the data slots of a frame of size $\Delta+1$.
Short frames imply that the pilot overhead
is relatively large, which results in poor spectral efficiency.
On the other hand, too large frames (that is when $\Delta$
is too large) make the channel estimation quality in the "middle" time slots poor, since for these time slots both
available channel estimates $\mx{\hat h}(0)$ and $\mx{\hat h}(\Delta+1)$ convey little useful information, especially
at high Doppler frequencies when the channel ages rapidly.
Indeed, as seen in Figure \ref{fig:1}, the frame size
has a large impact on the achievable spectral efficiency, suggesting that the optimum frame size depends critically on
the Doppler frequency.
As we can see, the spectral efficiency as a function of the frame size is in general neither monotone nor concave,
and is hence hard to optimize.

\begin{figure}[t]
\begin{center}
%\includegraphics[width=0.85\hsize]{cont1eps}
%\includegraphics[width=\hsize]{figura_new}
\includegraphics[width=\hsize]{ExpFig2V2}
\caption{
Spectral efficiency for each data slot $i=1 \dots \Delta$ when the frame size is kept fixed ($\Delta=50$).
At high maximum Doppler frequency, the spectral efficiency is low at the "middle" slot,
while at low maximum Doppler the spectral efficiency reaches its maximum at the middle slots.
}
\label{fig:2}
\end{center}
\end{figure}

Figure \ref{fig:2} shows the spectral efficiency for
each data slot $i=1 \dots \Delta$ within a frame of size $\Delta=50$.
At lower Doppler frequencies, that is when the channel fades relatively slowly,
the channel state information acquisition in the middle slots benefits from using the estimates at $i=0$ and $i=\Delta+1$,
and making an \ac{MMSE} interpolation of the channel coefficients as proposed in Lemma \ref{lem:mmsechannel}.	
However, at a high Doppler frequency, the channel state in the middle data slots are weakly
correlated with the channel estimates $\mx{\hat h}(0)$ and $\mx{\hat h}(\Delta+1)$,
which makes the \ac{MMSE} channel estimation error in Corollary \ref{cor:rmmse} large.
This insight suggests that in such cases, the optimum frame size
is much less than when the Doppler frequency is low.

\begin{figure}[t]
\begin{center}
%\includegraphics[width=0.85\hsize]{cont1eps}
%\includegraphics[width=\hsize]{figura_new}
\includegraphics[width=\hsize]{ExpFig3}
\caption{
Spectral efficiency as a function of the pilot/data power ratio and the frame size
for maximum Doppler frequency $f_D=500$ Hz and $f_D=1500$ Hz when $N_r=10$.
In both cases, the spectral efficiency depends heavily on the employed pilot power
and pilot spacing (frame size).
}
\label{fig:3}
\end{center}
\end{figure}

The average spectral efficiency as a function of the pilot/data power ratio and the
frame size is shown in Figure \ref{fig:3}. This figure clearly shows that setting the
proper frame size and tuning the pilot/data power ratio are both important to maximize
the average spectral efficiency of the system. The optimal frame size and power
configuration are different for different Doppler frequencies, which in turn emphasizes
the importance of accurate Doppler frequency estimates.
%(which will be discussed further
%in conjunction with Figure \ref{fig:8}).

\begin{figure}[t]
\begin{center}
%\includegraphics[width=0.85\hsize]{cont1eps}
%\includegraphics[width=\hsize]{figura_new}
\includegraphics[width=\hsize]{ExpFig4}
\caption{
Optimal frame size as a function of the maximum Doppler frequency for different values of
the employed pilot power ($P_p=50$ mW and $P_p=125$ mW) when the BS is equipped with $N_r=10$
and $N_r=100$ receive antennas.
}
\label{fig:4}
\end{center}
\end{figure}

The optimal frame size as a function of the maximum Doppler frequency is shown in Figure \ref{fig:4}.
The optimal frame size decreases rapidly, as the Doppler effect increases. As this figure shows,
a much larger frame size is optimal when the number of antennas is high and the \ac{MS} uses high
pilot power to achieve a high pilot \ac{SNR}.

\begin{figure}[t]
\begin{center}
%\includegraphics[width=0.85\hsize]{cont1eps}
%\includegraphics[width=\hsize]{figura_new}
\includegraphics[width=\hsize]{ExpFig5}
\caption{
Optimal spectral efficiency as a function of the maximum Doppler frequency, that is the spectral
efficiency when using the optimal frame size as shown in Figure \ref{fig:4}.
}
\label{fig:5}
\end{center}
\end{figure}

Figure \ref{fig:5} shows the achieved spectral efficiency when the frame size is set optimally,
as a function of the maximum Doppler frequency. At $f_D=500$ Hz, for example, when the optimal
frame size is 8 (see also Figures \ref{fig:1} and \ref{fig:4}), the achieved spectral efficiency
when using $N_r=10$ antennas is a bit below 1 bps/Hz. We can see that setting the optimal frame
size is indeed important, because it helps to make the achievable spectral efficiency quite
robust with respect to even a significant increase in the Doppler frequency.

\begin{figure}[t]
\begin{center}
%\includegraphics[width=0.85\hsize]{cont1eps}
%\includegraphics[width=\hsize]{figura_new}
\includegraphics[width=\hsize]{ExpFig6}
\caption{
Upper bounding the achievable spectral efficiency as a function of the frame size ($\Delta$)
at $f_D=500$ Hz and $f_D=1500$ Hz. Note that the upper bound is monotonically decreasing, which
helps to limit the search space for the optimum frame size.
}
\label{fig:6}
\end{center}
\end{figure}

Figure \ref{fig:6} illustrates the upper bounds on spectral efficiency as a function of the
frame size for different Doppler frequencies. Recall from Figure \ref{fig:1} that the spectral
efficiency of the system is a non-concave function of the frame size. Therefore, limiting the
possible frame sizes that can optimize spectral efficiency is useful, which can be achieved
by the upper bounds shown in the figure. Since the upper bound is monotonically decreasing,
finding a point of the spectral efficiency curve (see the curve marked with $f_D=500$ Hz
and its upper bounding curve) with a negative derivative helps to find the range of possible
frame sizes that maximize spectral efficiency. For $f_D=500$ Hz, as illustrated in the figure,
larger frame sizes than $\Delta=41$ would lead to a lower upper bound
than the spectral efficiency achieved at $\Delta=7$.
Therefore, when searching for the optimal $\Delta$, once we found that the spectral efficiency
at $\Delta=8$ is less than at $\Delta=7$ (negative derivative), the search space is limited to $(7,41)$.

\begin{figure}[t]
\begin{center}
%\includegraphics[width=0.85\hsize]{cont1eps}
%\includegraphics[width=\hsize]{figura_new}
\includegraphics[width=\hsize]{ExpFig7}
\caption{
Spectral efficiency in each time slot for $f_D=500$ Hz and $f_D=1500$ Hz, when using
1 or 2 pilot symbols preceding that time slot and 0 or 1 pilot symbols after that
time slot for channel estimation. Three combinations of these channel estimation schemes
are denoted as "2b, 1a", "1b, 1a" and "2b", where "b" refers to utilizing the pilot symbols
sent before and "a" refers to utilizing the pilot symbol sent after time slot $i$.
}
\label{fig:7}
\end{center}
\end{figure}

Figure \ref{fig:7} compares the average spectral efficiency when the system uses different
number of pilot signals to estimate the channel state for each data slot within the frame.
Specifically, three schemes are compared:
\begin{itemize}
\item
2 before, 1 after (2b, 1a): Three channel estimates using the pilot signals at the beginning
of the current frame and the preceding frame and at the end of the current frame are used
to interpolate the channel state at every data slot in the current frame.
\item
1 before, 1 after (1b, 1a): The two neighboring pilot signals (that is in the beginning and at the end
of the current frame) are used.
\item
2 before (2b): The pilot signals at the beginning of the current and preceding frames are used.
This scheme has an advantage over the previous schemes in that decoding the received data symbols
is possible "on the fly" without having to await the upcoming pilot signal at the end of the
current frame.
\end{itemize}

Notice that the "1b, 1a" scheme outperforms the "2b" scheme, because the channel estimation
instances are closer to the data transmission instance in time. Furthermore, the "2b, 1a"
scheme further improves the \ac{SE} performance, although this improvement over the
"1b, 1a" scheme is marginal. More importantly, we can observe that the optimal pilot
spacing is similar in these three schemes, but depends heavily on the Doppler frequency.

\begin{figure}[t]
\begin{center}
%\includegraphics[width=0.85\hsize]{cont1eps}
%\includegraphics[width=\hsize]{figura_new}
\includegraphics[width=\hsize]{ExpFig8V4}
\caption{
Spectral efficiency as a function of the frame size $\Delta$ when the receiver under or overestimates the actual Doppler frequency of the channel ($f_D=200$ Hz and $f_D=500$ Hz). Overestimating the actual Doppler frequency causes significant spectral efficiency degradation for most frame sizes.
}
\label{fig:8}
\end{center}
\end{figure}

Finally, Figure \ref{fig:8} examines the negative impact of Doppler frequency estimation
errors when
%assuming an AR(1) model at the receiver instead of the actual Rayleigh fading
%process.
the Doppler frequency of the channel is under or overestimated.
The figure shows the spectral efficiency as a function of the frame size for the cases when
$f_D=200$ Hz and $f_D=500$ Hz. For both cases, the Doppler frequency is either correctly estimated or overestimated (to $5f_D$) or underestimated (to $0.2 f_D$).
On the one hand, this figure clearly illustrates the performance degradation
in terms of average spectral efficiency when the receiver underestimates or overestimates
the maximum Doppler frequency.
%or when it assumes a discrete-time AR(1) process instead of
%taking correctly into account that the channel is a continuous time Rayleigh process.
On the other hand, when using the optimal frame size, the spectral efficiency performance
of these schemes are rather similar in most cases.

\section{Conclusions}
\label{Sec:Conc}
This paper investigated the fundamental trade-off between using resources in the time domain for pilot signals and data signals in the uplink of \ac{MU-MIMO} systems operating over fast fading wireless channels that age between subsequent pilot signals. While previous works indicated that when the autocorrelation coefficient between subsequent channel realization instances in discrete time is high, both the channel estimation and the \ac{MU-MIMO} receiver can take advantage of the memoryful property of the channel in the time domain. However, previous works do not answer the question how often the channel should be observed and estimated such that the subsequent channel samples are sufficiently correlated while
taking into account that pilot signals do not carry information bearing symbols and degrade the overall spectral efficiency.
To find the optimal pilot spacing, we first established the deterministic equivalent of the achievable \ac{SINR} and the associated overall spectral efficiency of the \ac{MU-MIMO} system. We then used some useful properties of an upper bound
of this spectral efficiency, which allowed us to limit the search space for the optimal pilot spacing ($\Delta$).
The numerical results indicate that the optimal pilot spacing is sensitive to the Doppler frequency of the channel and that proper pilot spacing has a significant impact on the achievable spectral efficiency.

\appendices
\section{Proof of Lemma \ref{lem:beta}}
\label{Sec:L4}
\begin{proof} % Proof of Lemma \ref{lem:beta}
Notice that
\begin{align}
\label{eq:MDelta}
\mx{M}(\Delta,i) &=
\underbrace{\begin{bmatrix}
1 & e^{2qi} & e^{4qi}  \\
{(e^{2qi})}^* & 1 & e^{2qi} \\
{(e^{4qi})}^* &  {(e^{2qi})}^* & 1\\
\end{bmatrix}}_{\triangleq \mx{M}_3(\Delta,i)}
 \otimes \mx{C},
\end{align}
%
and the eigenvalues of $\mx{M}_3(\Delta,i)$ are:
\begin{align*}
\lambda_1(i) &= 1 - a^{2i}; \\
\lambda_2(i) &= \frac{1}{2}\left(2+a^{2i}-a^i\sqrt{8+a^{2i}}\right);\\
\lambda_3(i) &= \frac{1}{2}\left(2+a^{2i}+a^i\sqrt{8+a^{2i}}\right),
\end{align*}
where $0<a<1$.
For all $i\geq1$ the smallest eigenvalue is $\lambda_2(i)$, which monotone increases with $i$. That is, $\min_{i\geq 1,j\in\{1,2,3\}} \lambda_j(i) = \lambda_2(1)$.

Let
\begin{align}
\mx{M}^{(u)}(\Delta) &\triangleq
\underbrace{\begin{bmatrix}
\eta & 0 & 0  \\
0 & \eta & 0  \\
0 &  0 & \eta \\
\end{bmatrix}}_{\triangleq \mx{M}^{(u)}_3(\Delta)}
 \otimes \mx{C}.
\end{align}

When $\eta<\lambda_2(1)$ according to \eqref{eq:etacond}, we have
\begin{align}
\label{lem3:13}
\mx{M}_3^{(u)}(\Delta) \preceq \mx{M}_3(\Delta).
\end{align}
Utilizing that
the spectrum of a Kronecker product $\sigma(\mx{A}\otimes \mx{B})$ is \cite{Horn:91}
\begin{align}
    \sigma(\mx{A}\otimes\mx{B}) &=
    \{\,
    \mu_A\mu_B \mid \mu_A \in \sigma(\mx{A}), \mu_B \in
    \sigma(\mx{B}) \,\},
\end{align}
for $\forall i\geq 1$, we further have
\begin{align}
\label{lem3:1}
\mx{M}^{(u)}(\Delta) \preceq \mx{M}(\Delta,i),
\end{align}
which implies
%\begin{align}
%\mx{M}^{(u)}(\Delta)+\mx{\Sigma}_3 \preceq  \mx{M}(\Delta,i)+\mx{\Sigma}_3,
%\end{align}
%and
\begin{align}\label{eq:Mu}
\big(\mx{M}^{(u)}(\Delta)+\mx{\Sigma}_3\big)^{-1} \succeq  \big(\mx{M}(\Delta,i)+\mx{\Sigma}_3\big)^{-1},
\end{align}
according to \eqref{eq:ABinv}.
The statement of the lemma % \eqref{eq:mrel},
comes from \eqref{eq:Mu} using \eqref{eq:AB3},
$\mx{M}^{(u)}(\Delta)=\eta \mx{I}_{3} \otimes \mx{C}$,
and noting that
\begin{align}
\nonumber
&\mx{E}(\Delta, i) \big(\mx{M}^{(u)}(\Delta)+\mx{\Sigma}_3\big)^{-1} \mx{E}(\Delta, i)^H
\\\nonumber &= \mx{E}(\Delta, i) \begin{bmatrix}
\eta \mx{C} + \bs{\Sigma} & 0 & 0  \\
0 & \eta \mx{C} + \bs{\Sigma} & 0  \\
0 &  0 & \eta \mx{C} + \bs{\Sigma} \\
\end{bmatrix}^{-1} \mx{E}(\Delta, i)^H
\\\nonumber &=
\mx{R}(\Delta \!+\! 1 \!+\! i)
(\eta \mx{C} + \bs{\Sigma})^{-1}
\mx{R}(\Delta \!+\! 1 \!+\! i)^H
\\\nonumber &~~~+
\mx{R}(i)(\eta \mx{C} + \bs{\Sigma})^{-1}
\mx{R}(i)^H
\\\nonumber &~~~+
\mx{R}(\Delta \!+\! 1 \!-\! i)
(\eta \mx{C} + \bs{\Sigma})^{-1}
\mx{R}(\Delta \!+\! 1 \!-\! i)^H
\\&= \rho(\Delta, i) \mx{C} \left( \eta \mx{C} + \bs{\Sigma} \right)^{-1} \mx{C},
\end{align}
where $\mx{R}(i)$ and $\rho(\Delta, i)$ are defined in \eqref{eq:r1} and \eqref{eq:rho}.
\end{proof}
%----

\section{Proof of Lemma \ref{lem:det}}
\label{Sec:L5}
\begin{proof}
\begin{align*}
& \bar{\gamma}(\Delta, i)
\\&= \mathds{E}\left( \textup{tr}\left( \bs{\Phi}(\Delta, i)  \left( \sum_{k=2}^K  \mx{b}_l(\Delta, i) \mx{b}_l^H(\Delta, i)  +\bs{\beta}(\Delta, i) \right)^{-1}  \right) \right)
\end{align*}
\begin{align*}
&=  \int\limits_{\mx{v}_2\in\mathbb{R}^{N_r}} \!\!\!\ldots\!\!\!  \int\limits_{\mx{v}_K\in\mathbb{R}^{N_r}}  \prod_{k=2}^K Pr(\mx{b}_l(\Delta, i)=\mx{v}_l)
\\& ~~~~~~\cdot \textup{tr}\left( \bs{\Phi}(\Delta, i)  \left( \sum_{k=2}^K  \mx{v}_l \mx{v}_l^H  +\bs{\beta}(\Delta, i) \right)^{-1} \right)
 d\mx{v}_K \ldots d\mx{v}_2
\end{align*}
\begin{align*}
& \leq \int\limits_{\mx{v}_2\in\mathbb{R}^{N_r}} \int\limits_{\mx{v}_K\in\mathbb{R}^{N_r}}  \prod_{k=2}^K Pr(\mx{b}_l(\Delta, i)=\mx{v}_l)
\\& ~~~~~~~~~~~~~~~~~~\cdot
\textup{tr}\left( \bs{\Phi}(\Delta, i)   \bs{\beta}(\Delta, i)^{-1} \right) d\mx{v}_K \ldots d\mx{v}_2
\\&= \textup{tr}\left( \bs{\Phi}(\Delta, i) \bs{\beta}(\Delta, i)^{-1} \right),
\nonumber
\end{align*}
where we used that $\sum_{l = 2}^K  \mx{v}_l \mx{v}_l^H$ is a positive definite matrix,
$\sum_{l = 2}^K  \mx{v}_l \mx{v}_l^H +\bs{\beta}(\Delta, i) \succeq \bs{\beta}(\Delta, i)$
and Lemma \ref{lem:AB}.
\end{proof}

\section{Proof of Proposition \ref{UpperB1}}
\label{Sec:P2}
\begin{proof}
To prove monotonicity in $\rho$ first notice that
\begin{align*}
\rho(\Delta_1, i_1) > \rho(\Delta_2, i_2) \Rightarrow \mx{Z}^{(u)}(\Delta_1,i_1) \preceq  \mx{Z}^{(u)}(\Delta_2,i_2), \\
\rho(\Delta_1, i_1) > \rho(\Delta_2, i_2) \Rightarrow \bs{\Phi}^{(u)}(\Delta_1,i_1) \succeq  \bs{\Phi}^{(u)}(\Delta_2,i_2).
\end{align*}
and so
\begin{gather*}
\rho(\Delta_1, i_1) > \rho(\Delta_2, i_2)
\\
\Downarrow
\\
\begin{align*}
&\bs{\Phi}^{(u)}(\Delta_1,i_1) \left( \sum_{l = 1}^K \alpha_l^2 P_l \mx{Z}^{(u)}_l(\Delta_1, i_1) + \sigma_d^2 \mx{I}_{N_r} \right)^{-1}  \\
&~~\succeq \bs{\Phi}^{(u)}(\Delta_2,i_2) \left( \sum_{l = 1}^K \alpha_l^2 P_l \mx{Z}^{(u)}_l(\Delta_2, i_2) + \sigma_d^2 \mx{I}_{N_r} \right)^{-1}
\end{align*}
\\
\Downarrow
\\
\begin{align*}
&\textup{tr} \left(\bs{\Phi}^{(u)}(\Delta_1,i_1) \left( \sum_{l = 1}^K \alpha_l^2 P_l \mx{Z}^{(u)}_l(\Delta_1, i_1) + \sigma_d^2 \mx{I}_{N_r} \right)^{-1}\right)  \\
&\geq \textup{tr} \! \left( \!\bs{\Phi}^{(u)}(\Delta_2,i_2) \left( \sum_{l = 1}^K \alpha_l^2 P_l \mx{Z}^{(u)}_l(\Delta_2, i_2) + \sigma_d^2 \mx{I}_{N_r}\! \right)^{\!\!-1}\right)
\end{align*}
\\
\Downarrow
\\
\bar{\gamma}^{(u)}(\Delta_1,i_1) \geq \bar{\gamma}^{(u)}(\Delta_2,i_2).
\end{gather*}
Finally to prove convergence to 0 notice that
\begin{align*}
    \rho(\Delta, i) \rightarrow 0 &\Rightarrow \mx{Z}^{(u)}(\Delta_1,i_1) \rightarrow \mx{C}, \\
    \rho(\Delta, i) \rightarrow 0 &\Rightarrow \bs{\Phi}^{(u)}(\Delta_1,i_1) \rightarrow \mx{0}.
\end{align*}
And so, when $ \rho(\Delta, i) \rightarrow 0$ we have
\begin{align*}
    \bar{\gamma}^{(u)}&(\Delta,i) = \\
    &\textup{tr} \left(\bs{\Phi}^{(u)}(\Delta,i) \left( \sum_{l = 1}^K \alpha_l^2 P_l \mx{Z}^{(u)}_l(\Delta, i) + \sigma_d^2 \mx{I}_{N_r} \right)^{-1}\right)
\end{align*}
\begin{align*}
   & \stackrel{\rho(\Delta, i) \rightarrow 0}{\rightarrow}\textup{tr} \left(\mx{0} \left( \sum_{l = 1}^K \alpha_l^2 P_l \mx{C} + \sigma_d^2 \mx{I}_{N_r} \right)^{-1}\right) = 0.
\end{align*}
\end{proof}

\section{Proof of Proposition \ref{UpperB2}}
\label{Sec:P3}
\begin{proof}
From Theorem \ref{thm:upperbound} and \eqref{eq:SEku} the inequality follows.
For monotonicity, notice that $\rho_k(\Delta + 1, i) < \rho_k(\Delta, i )$ and
$\rho_k(\Delta + 1, i + 1) < \rho_k(\Delta, i )$.
Since by Proposition \ref{UpperB1} the upper bound of the \ac{SINR} is increasing with $\rho_k$ we have
\begin{align}
    \bar{\gamma}_k^{(u)}(\Delta + 1,i) &\leq \bar{\gamma}_k^{(u)}(\Delta,i) \nonumber \\
    \bar{\gamma}_k^{(u)}(\Delta + 1,i + 1) &\leq \bar{\gamma}_k^{(u)}(\Delta,i),
\end{align}
from which it follows that
\begin{align}
    \log\big(1 + \bar{\gamma}_k^{(u)}(\Delta + 1,i)\big) &\leq \log\big(1 + \bar{\gamma}_k^{(u)}(\Delta,i)\big)
    \label{eq:delta_step1}\\
    \log\big(1 + \bar{\gamma}_k^{(u)}(\Delta + 1,i + 1)\big) &\leq \log\big(1 + \bar{\gamma}^{(u)}(\Delta,i)\big).
    \label{eq:delta_step2}
\end{align}
Let $\ell = \arg\min_i \bar{\gamma}_k^{(u)}(\Delta + 1,i)$,
we then have
\begin{align}
 \nonumber &  \frac{1}{\Delta+1}  \times \sum_{i=1}^{\Delta+1} \log\big(1 + \bar{\gamma}_k^{(u)}(\Delta + 1,i)\big)
 \\ \nonumber
 & \leq  \frac{1}{\Delta} \times \left( \sum_{i=1}^{\ell-1} \log\big(1 + \bar{\gamma}_k^{(u)}(\Delta + 1,i)\big) \right. \\
    & ~~~~~~~~~~~~~~~~~~~~ + \left. \sum_{i=\ell+1}^{\Delta+1} \log\big(1 + \bar{\gamma}_k^{(u)}(\Delta + 1,i)\big) \right),
\end{align}
since on the right hand side we are removing the smallest term before calculating the mean.
Invoking \eqref{eq:delta_step1} and \eqref{eq:delta_step2} on the first and second sum, respectively, it follows that
\begin{align}
\nonumber
&   \frac{1}{\Delta + 1} \times \sum_{i=1}^{\Delta+1} \log\big(1 + \bar{\gamma}_k^{(u)}(\Delta + 1,i) \big)
\\ \nonumber
&  \leq  \frac{1}{\Delta + 1} \times \left( \sum_{i=1}^{\ell-1} \log\big(1 + \bar{\gamma}_k^{(u)}(\Delta,i)\big) \right. \\ \nonumber
&  ~~~~~~~~~~~~~~~~~~~~~+\left. \sum_{i=\ell}^{\Delta} \log\big(1 + \bar{\gamma}_k^{(u)}(\Delta,i)\big) \right)
\\
&=  \frac{1}{\Delta } \times \sum_{i=1}^{\Delta} \log\big(1 + \bar{\gamma}_k^{(u)}(\Delta,i)\big).
\end{align}
From which it follows that
\begin{align}
& \textup{SE}_k^{(u)}(\Delta+1) = \frac{\sum_{i=1}^{\Delta+1} \log\big(1 + \bar{\gamma}_k^{(u)}(\Delta + 1,i)\big)}{\Delta + 1}
\nonumber  \\
&~~~~~ \leq  \frac{\sum_{i=1}^{\Delta} \log\big(1 + \bar{\gamma}_k^{(u)}(\Delta ,i)\big)}{\Delta} = \textup{SE}_k^{(u)}(\Delta),
\label{eq:SEu}
\end{align}
that is $\textup{SE}_k^{(u)}(\Delta)$ is decreasing in $\Delta$.

To prove convergence to zero, recall from Proposition \ref{UpperB1} that
$\partial \bar{\gamma}_k^{(u)}(\Delta, i) / \partial \rho_k(\Delta, i) \geq 0$ and
\begin{align}
\nonumber
\rho_k(\Delta, i) \rightarrow 0 & \Rightarrow \bar{\gamma}_k^{(u)}(\Delta, i) \rightarrow 0
\end{align}
\begin{align}
&     \Rightarrow \log\big(1 + \bar{\gamma}_k^{(u)}(\Delta, i)\big) \rightarrow 0,
\label{eq:conv1}
\end{align}
where
\begin{align}
\nonumber
\rho_k(\Delta, i) = e^{2\bar{q}_k (\Delta + 1 + i)} + e^{2\bar{q}_k  i} + e^{2\bar{q}_k (\Delta + 1 - i)}.
\end{align}
We show that for any $\varepsilon > 0$, there is some $M$ such that
\begin{align}
 \textup{SE}^{(u)}(M) < \varepsilon.
\end{align}
Due to $\bar{q}_k < 0$, we have $\rho_k(\Delta, i) < \rho_k(1,1)$,
which implies
\begin{align}
    \log\big(1 + \bar{\gamma}_k^{(u)}(\Delta, i)\big) < \log\big(1 + \bar{\gamma}_k^{(u)}(1, 1)\big),
\end{align}
for all $\Delta$ and $i$.
Let $A \triangleq \log\big(1 + \bar{\gamma}_k^{(u)}(1, 1)\big)$
and
$N$ such that $N\varepsilon - 2A > 0$, and set
\begin{align}
\label{eq:epsilon}
    \epsilon \triangleq \frac{N\varepsilon - 2A}{N - 2}.
\end{align}

Since $\bar{q}_k < 0$, we have
\begin{align*}
\rho_k(\Delta, i) &< 3 \max(e^{2\bar{q}_k  (\Delta+1+i)},e^{2\bar{q}_k  i},e^{2\bar{q}_k  (\Delta+1-i)})
\\&= 3 e^{2\bar{q}_k  \min(\Delta+1+i,i,\Delta+1-i)},
\end{align*}
and it follows that for $\frac \Delta N \leq i \leq \frac{(N-1)\Delta}{N}$
\begin{align}
\rho_k(\Delta, i) < 3e^{2\bar{q}_k \frac \Delta N}.
\end{align}
Notice that by equation \eqref{eq:conv1} we can choose some large $M$,
such that
\begin{align}
&\frac M N \leq i \leq \frac{(N-1)M}{N}
    \Rightarrow
     \log(1 + \bar{\gamma}_k^{(u)}(M, i)) < \epsilon.
\end{align}
We can now show that when $M=\Delta$, then
$\textup{SE}_k^{(u)}(\Delta) < \varepsilon$.
To this end,
we split up the sum in the numerator of \eqref{eq:SEu}, that is
$\sum_{i=1}^{\Delta} \log(1 + \bar{\gamma}^{(u)}(\Delta ,i))$, into three terms,
and bound the first and third terms using the general upper bound $A$,
and the middle term by $\epsilon$:
\begin{align}
\nonumber
\textup{SE}_k^{(u)}(\Delta) &= \frac{\sum_{i=1}^{\Delta} \log(1 + \bar{\gamma}^{(u)}(\Delta ,i))}{\Delta}  \\ \nonumber
&=
\frac{\sum_{i=1}^{\Delta / N} \log(1 + \bar{\gamma}^{(u)}(\Delta ,i))}{\Delta} \\ \nonumber
&~~~~+
\frac{\sum_{i=\Delta / N + 1}^{(N - 1)\Delta / N} \log(1 + \bar{\gamma}^{(u)}(\Delta ,i))}{\Delta}
\\ \nonumber
&~~~~+
\frac{\sum_{i=(N-1)\Delta / N + 1}^{\Delta} \log(1 + \bar{\gamma}^{(u)}(\Delta ,i))}{\Delta}
\\ \nonumber
%\end{align}
%\begin{align}
&<
\frac{(\Delta / N)A}{\Delta} + \frac{((N-2)\Delta / N)\epsilon}{\Delta} +
\frac{(\Delta / N)A}{\Delta}  \\ &=
\frac{2A + (N-2)\epsilon}{N} = \varepsilon,
\end{align}
where the last equation is due to the definition of $\epsilon$ in \eqref{eq:epsilon}, which completes the proof.
\end{proof}
\bibliography{sampling}
\end{document}


Each sampling method makes its own assumptions about the degrees of freedom in the sampling process that results in different interpretations of the corresponding CIs. \textsc{Boot-Inputs} assumes that there is only uncertainty on the inputs while the systems are held constant.
CIs derived from this sampling technique would express a range of values for the true correlation $\rho$ between $\mathcal{X}$ and $\mathcal{Z}$ for the \emph{specific} set of systems $\mathcal{S}$ and inputs from the same distribution as those in $\mathcal{D}$.
The opposite assumption is made for \textsc{Boot-Systems} (uncertainty in systems, inputs are fixed).
\textsc{Boot-Both}, which can be viewed as sampling systems followed by sampling inputs, assumes uncertainty on both the systems and the inputs.
Therefore the corresponding CI estimates $\rho$ for systems and inputs distributed the same as those in $\mathcal{S}$ and $\mathcal{D}$.

Algorithm~\ref{alg:ci} contains the pseudocode for calculating a CI via bootstrapping using the \textsc{Boot-Both} sampling method.
In \S\ref{sec:ci_simulations} we experimentally evaluate the Fisher transformation and the three bootstrap sampling methods, then analyze the CIs of several different metrics in \S\ref{sec:ci_experiments}.

\begin{algorithm}[t]
{
\small
\caption{Bootstrap Confidence Interval}
\label{alg:ci}
\hspace*{\algorithmicindent} \textbf{Input:} $X, Z \in \mathbb{R}^{N\times M}$, $k \in \mathbb{N}, \alpha \in [0, 1]$ \\
\hspace*{\algorithmicindent} \textbf{Output:} $(1-\alpha)\times 100\%$-confidence interval
\begin{algorithmic}[1]
\State samples $\gets$ an empty list
\For{$k$ iterations}
    \State $S$ $\gets$ samp. $\{1,\dots, N\}$ w/ repl. $N$ times
    \State $D$ $\gets$ samp. $\{1,\dots, M\}$ w/ repl. $M$ times
    \State $X_s, Z_s \gets$ empty $N \times M$ matrices
    \For{$(i, j) \in \{1, \dots, N\} \times \{1, \dots M\}$}
        \State $X_s[i, j] \gets X[S[i], D[j]]$
        \State $Z_s[i, j] \gets Z[S[i], D[j]]$
    \EndFor
    \State Append $r(X_s, Z_s)$ to samples
\EndFor
\State $\ell, u \gets (\alpha/2)\times 100$ and $(1-\alpha/2)\times 100$ percentiles of samples
\State \Return $\ell, u$
\end{algorithmic}
}
\end{algorithm}

\section{Significance Testing}
\label{sec:hypo}
Although CIs express the strength of the correlation between two metrics, they do not directly express whether one metric $\mathcal{X}$ correlates to another $\mathcal{Z}$ better than $\mathcal{Y}$ does due to their shared dependence on $\mathcal{Z}$.
This statistical analysis is performed by hypothesis testing.
The specific one-tailed hypothesis test we are interested in is:
\begin{align*}
    H_0 &: \rho(\mathcal{X}, \mathcal{Z}) - \rho(\mathcal{Y}, \mathcal{Z}) \leq 0 \\
    H_1 &: \rho(\mathcal{X}, \mathcal{Z}) - \rho(\mathcal{Y}, \mathcal{Z}) > 0
\end{align*}

\subsection{Williams' Test}
\label{sec:williams}
One method for hypothesis testing the difference between two correlations with a dependent variable that is used frequently to compare machine translation metrics is Williams' test \citep{Williams59}.
It uses the pairwise correlations between $X$, $Y$, and $Z$ to calculate a $t$-statistic and a corresponding $p$-value.\footnote{
The full equation is omitted for space.
See \citet{GrahamBa14} for details.
} Williams' test is frequently used to compare machine translation metrics' performances at the system-level \citep[among others]{MWFMB20}.


However, the test faces the same issues as the Fisher transformation: It assumes the input variables are normally distributed \citep{DunnCl71}, and it is not clear whether the test should be applied at the summary-level.

\subsection{Permutation Tests}
\label{sec:hypo_permutation}
Bootstrapping can be used to calculate a $p$-value in the form of a paired bootstrap test in which the sampling methods described in \S\ref{sec:ci_bootstrapping} can be used to resample new matrices from $X$, $Y$, and $Z$ in parallel (details omitted for space).
However, an alternative and closely related nonparametric hypothesis test is the permutation test \citep{Noreen89}.
Permutation tests tend to be used more frequently than paired bootstrap tests for hypothesis testing because they directly test whether any observed difference between two values is due to random chance.
In contrast, paired bootstrap tests indirectly reason about this difference by estimating the variance of the test statistic.



\begin{figure*}
    \centering
    \includegraphics[width=\textwidth]{figures/permutations/permutation-figure.pdf}
    \caption{An illustration of the three permutation methods which swap system scores, document scores, or scores for individual summaries between $X$ and $Y$.
    }
    \label{fig:permutations}
\end{figure*}

Similarly to bootstrapping, a permutation test applied to two paired samples estimates the distribution of the test statistic under $H_0$ by calculating its value on new resampled datasets.
In contrast to bootstrapping, the resampled datasets are constructed by randomly permuting which sample each observation in a pair belongs to (i.e., resampling without replacement).
This relies on assuming the pair is exchangeable under $H_0$, which means $H_0$ is true for either sample assignment for the pair.
Then, the $p$-value is calculated as the proportion of times the test statistic across all possible permutations is greater than the observed value.
A significant $p$-value implies the observed test statistic is very unlikely to occur if $H_0$ were true, resulting in its rejection.
In practice, calculating the distribution of $H_0$ across all possible permutations is intractable, so it is instead estimated on a large number of randomly sampled permutations.\footnote{
    This is known as an approximate randomization test.
}

\begin{algorithm}[t]
{\small
\caption{Permutation Hypothesis Test}
\label{alg:permutation}
\hspace*{\algorithmicindent} \textbf{Input:} $X, Y, Z \in \mathbb{R}^{N\times M}$, $k \in \mathbb{N}, \alpha \in [0, 1]$
\hspace*{\algorithmicindent} \textbf{Output:} $p$-value
\begin{algorithmic}[1]
\State Standardize $X$ and $Y$
\State c $\gets$ 0
\State $\delta \gets r(X, Z) - r(Y, Z)$
\For{$k$ iterations}
    \State $X_s, Y_s \gets$ empty $N \times M$ matrices
    \For{$(i, j) \in \{1, \dots, N\} \times \{1, \dots, M\}$}
        \If{random Boolean is true} \Comment{swap}
            \State $X_s[i, j] \gets Y[i, j]$
            \State $Y_s[i, j] \gets X[i, j]$
        \Else \Comment{do not swap}
            \State $X_s[i, j] \gets X[i, j]$
            \State $Y_s[i, j] \gets Y[i, j]$
        \EndIf
    \EndFor
    \State $\delta_s \gets r(X_s, Z) - r(Y_s, Z)$
    \If{$\delta_s > \delta$}
        \State $c \gets c + 1$
    \EndIf
\EndFor
\State \Return $c / k$
\end{algorithmic}
}
\end{algorithm}


For example, a permutation test applied to testing the difference between two QA models' mean accuracies on the same dataset would sample a permutation by swapping the models' outputs for the same input.
Under $H_0$, the models' mean accuracies are equal, so randomly exchanging the outputs is not expected to change their means.
In the case of evaluation metrics, each permutation sample can be taken by randomly swapping the scores in $X$ and $Y$.
There are at least three ways of doing so:
\begin{enumerate}
    \item \textsc{Perm-Systems}: For each system, swap its scores for all inputs with probability 0.5.
    \item \textsc{Perm-Inputs}: For each input, swap its scores for all systems with probability 0.5.
    \item \textsc{Perm-Both}: For each summary, swap its scores with probability 0.5.
\end{enumerate}
To account for differences in scale, we standardize $X$ and $Y$ before performing the permutation.
Fig.~\ref{fig:permutations} contains an illustration of each method, and the pseudocode for a permutation test using the \textsc{Perm-Both} method is provided in Alg.~\ref{alg:permutation}.

Similarly to the bootstrap sampling methods, each of the permutation methods makes assumptions about the system and input document underlying distribution.
This results in different interpretations of how the tests' conclusions will generalize.
Since \textsc{Perm-Systems} randomly assigns system scores for all documents in $\mathcal{D}$ to either sample, we only expect the test's conclusion to generalize to a system distributed similarly to those in $\mathcal{S}$ evaluated on the \emph{specific} set of documents $\mathcal{D}$.
The opposite is true for \textsc{Perm-Inputs}.
The results for \textsc{Perm-Both} (which can be viewed as first swapping systems followed by swapping inputs) are expected to generalize for both systems and documents distributed similarly to those in $\mathcal{S}$ and $\mathcal{D}$.


In \S\ref{sec:power} we run a simulation to compare the different hypothesis testing approaches, then analyze the results of hypothesis tests applied to summarization metrics in \S\ref{sec:hypo_experiments}.




\section{Simulation Experiments}
\label{sec:simulations}
We run two sets of simulation experiments in order to determine which CI (\S\ref{sec:ci_simulations}) and hypothesis test (\S\ref{sec:power}) methods are most appropriate for summarization metrics.

The datasets used in the simulations are the multi-document summarization dataset TAC'08 \citep{DangOw08} and two subsets of the single-document summarization CNN/DM dataset \citep{NZSGX16} annotated by \citet{FKMSR21} and \citet{BGALN20}.
These datasets have $N=58/16/25$ summarization models and $M=48/100/100$ inputs, respectively.
The summaries were assigned overall responsiveness, relevance, or Lightweight Pyramid \citep{SGGRPBAD19} scores, respectively, by human annotators. The scores of the automatic metrics are correlated to these human annotations.

\subsection{Confidence Interval Simulation}
\label{sec:ci_simulations}
In practice, evaluation metrics are almost always used to score summaries produced by systems $\mathcal{S}'$ on inputs $\mathcal{D}'$ which are disjoint (or nearly disjoint) from and assumed to be distributed similarly to the data that was used to calculate the CI, $\mathcal{S}$ and $\mathcal{D}$.
It is still desirable to use the CI as an estimate of the correlation of a metric on $\mathcal{S}'$ and $\mathcal{D}'$, however this scenario violates assumptions made by some of the bootstraping sampling methods (e.g., \textsc{Boot-Systems} assumes that $\mathcal{D}$ is fixed).
This simulation aims to demonstrate the effect of violating these assumptions on the accuracy of the CIs.

\paragraph{Setup.}
The simulation works as follows.
The systems $\mathcal{S}$ and inputs $\mathcal{D}$ are each randomly partitioned into two equally sized disjoint sets $\mathcal{S}_A$, $\mathcal{S}_B$, $\mathcal{D}_A$, and $\mathcal{D}_B$.
Then the submatrices $X_A$, $Z_A$, $X_B$, and $Z_B$ are selected from $X$ and $Z$ based on the system and input partitions.
Matrices $X_A$ and $Z_A$ are used to calculate a 95\% CI using one of the methods described in \S\ref{sec:ci}, and then it is checked whether sample correlation $r(X_B, Z_B)$ is contained by the CI.
The entire procedure is repeated 1000 times, and the proportion of times the CI contains the sample correlation is calculated.

It is expected that a CI which generalizes well to the held-out data should contain the sample correlation 95\% of the time under the assumption that the data in $A$ and $B$ is distributed similarly.
The larger the difference from 95\%, the worse the CI is at estimating the correlation on the held-out data.

\begin{table}[t]
    \centering
    \begin{adjustbox}{width=\columnwidth}
    \begin{tabular}{ccccccccc}
        \toprule
        \multirow{2}{*}[-0.2em]{\makecell{\bf CI \\ \bf Method}} & \multicolumn{2}{c}{\bf TAC'08} & & \multicolumn{2}{c}{\bf Fabbri et al.} & & \multicolumn{2}{c}{\bf Bhandari et al.} \\
        \cmidrule{2-3} \cmidrule{5-6} \cmidrule{8-9}
         & $\rhosys$ & $\rhosum$ & & $\rhosys$ & $\rhosum$ & & $\rhosys$ & $\rhosum$ \\ 
        \midrule
% Fisher & 0.90 & 1.00 & & 0.78 & 1.00 & & 0.90 & 1.00 \\
Fisher & 0.72 & 1.00 & & 0.87 & 1.00 & & 0.85 & 1.00 \\
% \midrule
% \multicolumn{1}{l}{\small \emph{Bootstrapping}} & & & & & & & & \\
\textsc{Boot-Systems} & 0.76 & 0.72 & & 0.81 & 0.73 & & 0.80 & 0.72 \\
\textsc{Boot-Inputs} & 0.58 & 0.70 & & 0.70 & 0.73 & & 0.68 & 0.62 \\
\textsc{Boot-Both} & \bf 0.82 & \bf 0.92 & & \bf 0.98 & \bf 0.93 & & \bf 0.94 & \bf 0.88 \\
        \bottomrule
    \end{tabular}
    \end{adjustbox}
    \caption{The proportion of times the 95\% confidence interval for the true correlations $\rho$ of QAEval-F$_1$ calculated using Pearson contains the sample correlation of a held-out set of systems and inputs for the different methods of calculating confidence intervals.
    Values in bold are closest to 0.95 (and less than 1.0) and significantly different under a one-tailed difference of proportions $z$-test at $\alpha = 0.05$.
    }
    \label{tab:ci_simulation}
\end{table}

The results of the simulation calculated on TAC'08 and CNN/DM using both the Fisher transformation and the different bootstrap sampling methods to CIs for QAEval-F$_1$ \citep{DeutschBeRo20} are shown in Table~\ref{tab:ci_simulation}.\footnote{
    The Fisher transformation was directly applied to the averaged summary-level correlation.
}

\paragraph{\textsc{Boot-Both} generalizes the best.}
Among the bootstrap methods, \textsc{Boot-Both} produces CIs that come closest to the ideal 95\% rate.
Any deviations from this number reflect that the assumption that all of the inputs and systems are distributed similarly is not true, but overall violating this assumption does not have a major impact.

The other bootstrap methods, which sample only systems or inputs, captures the correlation on the held-out data far less than 95\% of the time.
For instance, the CIs for $\rhosys$ on \citet{BGALN20} only successfully estimate the held-out correlation on 80\% and 68\% of trials.
This means that a 95\% CI calculated using \textsc{Boot-Inputs} is actually only a 68\% CI on the held-out data.
This pattern is the same across the different correlation levels and datasets.
The lower values for only sampling inputs indicates that more variance comes from the systems rather than the inputs.

\paragraph{Fisher analysis.}
The Fisher transformation at the system-level creates CIs that generalize worse than \textsc{Boot-Both}.
The summary-level CI captures the held-out sample correlation 100\% of the time, implying that the CI width is too large to be useful.
We believe this is due to the fact that as the absolute value of $r(X, Z)$ decreases, the width of the Fisher CI increases.
Summary-level correlations are lower than system-level correlations (see \S\ref{sec:ci_experiments}), and therefore Fisher results in a worse CI estimate at the summary-level.

\paragraph{Conclusion.}
This experiment presents strong evidence that violating the assumptions that either the systems/inputs are fixed or that the data is normally distributed does result in worse CIs.
Hence, the \textsc{Boot-Both} method provides the most accurate CIs for scenarios in which summarization metrics are frequently used.

\subsection{Power Analysis}
\label{sec:power}
The power of a hypothesis test is the probability of accepting the alternative hypothesis given that it is actually true (equal to $1.0$ -- the type-II error rate).
It is desirable to have as high of a power as possible in order to avoid missing a significant difference between metrics.
This simulation estimates the power of each of the hypothesis tests.

\paragraph{Setup.}
Measuring power requires a scenario in which it is known that $\rho$ is greater for one metric than another (i.e., $H_1$ is true).
Since this is not known to be true for any pair of proposed evaluation metrics, we artificially create such a scenario by adding randomness to the calculation of ROUGE-1.\footnote{
    We use the recall variant of ROUGE for experiments on TAC'08 and \citet{BGALN20} and the F$_1$ variant on \citet{FKMSR21} throughout the paper.
}
We define $\mathcal{R}_k$ to be ROUGE-1 calculated using a random $k\%$ of the candidate summary's tokens.
We assume that since $\mathcal{R}_k$ only evaluates a summary with $k\%$ of its tokens, it is quite likely that it is a worse metric than standard ROUGE-1 for $k < 100$.

To estimate the power, we score summaries with ROUGE-1 and $\mathcal{R}_k$ for different $k$ values and count how frequently each hypothesis test rejects $H_0$ in favor of identifying ROUGE-1 as a superior metric.
This trial is repeated 1000 times, and the proportion of significant results is the estimate of the power.

Since the various hypothesis tests make different assumptions about whether the systems and inputs are fixed or variable, it is not necessarily fair to directly compare their powers.
Because the assumptions of \textsc{Boot-Both} and \textsc{Perm-Both} most closely align with the typical use case of summarization, we compare their powers.
We additionally include Williams' test because it is frequently used for machine translation metrics and it produces interesting results, discussed below.

\begin{figure}
    \centering
    \includegraphics[width=\columnwidth]{figures/power/pearson.pdf}
    \caption{The system- and summary-level Pearson estimates of the power of the \textsc{Boot-Both}, \textsc{Perm-Both}, and Williams hypothesis test methods calculated on the annotations from \citet{FKMSR21}.
    The power for \textsc{Boot-Both} and Williams at the system-level is $\approx 0$ for all values.}
    \label{fig:power}
\end{figure}

\paragraph{\textsc{Perm-Both} has the highest power.}
Fig.~\ref{fig:power} plots the power curves for various values of $k$ on the CNN/DM annotations by \citet{FKMSR21}.
We find that \textsc{Perm-Both} has the highest power among the three tests for all values of $k$.
As $k$ approaches $100\%$, the difference between ROUGE-1 and $\mathcal{R}_k$ becomes smaller and harder to detect, thus the power for all methods approaches 0.

\textsc{Boot-Both} has lower power than \textsc{Perm-Both} both at the summary-level and system-level, in which it is near 0.
This result is consistent with permutation tests being more useful for hypothesis testing than their bootstrapping counterparts.
We believe the power differences in both levels are due to the variance of the two correlation levels.
As we observe in \S\ref{sec:ci_experiments}, the system-level CIs have significantly larger variance than at the summary-level, making it harder for the paired bootstrap to reject the system-level $H_0$.

\begin{figure*}
    \centering
    \includegraphics[width=1.0\textwidth]{figures/confidence-intervals/kendall.pdf}
    \caption{The 95\% confidence intervals for $\rhosum$ (blue) and $\rhosys$ (orange) calculated using Kendall's correlation coefficient on TAC'08 (left) and CNN/DM summaries (middle, \citet{FKMSR21}; right, \citet{BGALN20}) are rather large, reflecting the uncertainty about how well these metrics agree with human judgments of summary quality.}
    \label{fig:ci}
\end{figure*}

\paragraph{Williams' test has low power.}
Interestingly, the power of Williams' test for all $k$ is $\approx 0$, implying the test never rejects $H_0$ in this simulation.
This is surprising because Williams' test is frequently used to compare machine translation metrics at the system-level and does find differences between metrics.
We believe this is due to the strength of the correlations of ROUGE-1 to the ground-truth judgments as follows.

The $p$-value calculated by Williams is a function of the pairwise correlations of $X$, $Y$, and $Z$ and the number of observations.
The closer both $r(X, Z)$ and $r(Y, Z)$ are to 0, the higher the $p$-value.
The correlation of ROUGE-1 in this simulation is around 0.6 and 0.3 at the system- and summary-levels.
In contrast, the system-level correlations for the metrics submitted to the Workshop on Machine Translation (WMT) 2019's metrics shared task for de-en are on average 0.9 \citep{MWBG19}.
Among the 231 possible pairwise metric comparisons in WMT'19 for de-en, Williams' test yields 81 significant results.
If the correlations are shifted to have an average value of 0.6, only 3 significant results are found.
Thus we conclude that Williams' test's power is worse for detecting differences between lower correlation values.

Because this simulation is performed with summarization metrics on a real summarization dataset, we believe it is faithful enough to a realistic scenario to conclude that Williams' test does indeed have low power when applied to summarization metrics.
However, we do not expect Williams' test to have 0 power when used to detect differences between machine translation metrics.

\paragraph{Conclusion.}
Since \textsc{Perm-Both} has the best statistical power at both the system- and summary-levels, we recommend it for hypothesis testing the difference between summarization metrics.
\section{Summarization Analysis}
We run two experiments that calculate CIs (\S\ref{sec:ci_experiments}) and run hypothesis tests (\S\ref{sec:hypo_experiments}) for many different summarization metrics on the TAC'08 and CNN/DM datasets (\S\ref{sec:simulations}).
Each experiment also includes an analysis which discusses the implications of the results for the summarization community.

The metrics used for experimentation are the following:
AutoSummENG \citep{GKVS08},
BERTScore \citep{ZKWWA20},
BEwT-E \citep{TratzHo08},
METEOR \citep{DenkowskiLa14},
MeMoG \citep{GiannakopoulosKa10},
MoverScore \citep{ZPLGME19},
NPowER \citep{GiannakopoulosKa13},
QAEval \citep{DeutschBeRo20},
ROUGE \citep{Lin04},
and S$^3$ \citep{PeyrardBoGu17}.
We use the metrics' implementations in the SacreROUGE library \citep{DeutschRo20}.

\subsection{Confidence Intervals}
\label{sec:ci_experiments}
Fig.~\ref{fig:ci} shows the 95\% CIs calculated via \textsc{Boot-Both} for $\rhosum$ and $\rhosys$ for each metric calculated using Kendall's $\tau$.
Since ROUGE is the most commonly used metric, the following discussion will mostly focus on its results, however the conclusions largely apply to other metrics as well.

\paragraph{Confidence intervals are large.}
The most apparent observation is that the CIs are rather large, especially for $\rhosys$.
The ROUGE-2 $\rhosys$ CIs are $[.49, .74]$ for TAC'08 and $[-.09, .84]$ on CNN/DM using the annotations from \citet{FKMSR21}.
The wide range of values demonstrates that there is a large amount of uncertainty around how precise the correlations reported in the literature truly are.

The size of the CIs has serious implications for how trustable existing automatic evaluations are.
Since Kendall's $\tau$ is a function of the number of pairs of systems in which the automatic metric and ground-truth agree on their rankings, the metrics' CIs can be translated to upper- and lower-bounds on the number of incorrect rankings.
Specifically, ROUGE-2's system-level CI on \citet{FKMSR21} implies it incorrectly ranks systems with respect to humans 9-54\% of the time.
This means that potentially more than half of the time ROUGE ranks one summarization model higher than another on CNN/DM, it is wrong according to humans, a rather surprising result.
However, it is consistent with similar findings by \citet{RCDN13}, who estimated the same result to be around 37\% for top-performing systems on TAC 2008-2011.

We suspect that the true ranking accuracy of ROUGE (as well as the other metrics) is not likely to be at the extremes of the confidence interval due to the distribution of the bootstrapping samples shown in Fig.~\ref{fig:ci}.
However, this experiment highlights the uncertainty around how well automatic metrics replicate human annotations of summary quality.
An improved ROUGE score does not necessarily mean a model produces better summaries.
Likewise, not improving ROUGE should not disqualify a model from further consideration.
Consequently, researchers should rely less heavily on automatic metrics for determining the quality of summarization models than they currently do.
Instead, the community needs to develop more robust evaluation methodologies, whether it be task-specific downstream evaluations or faster and cheaper human evaluation. 




\paragraph{Comparing CNN/DM annotations.}
The CIs calculated on the annotations by \citet{BGALN20} are in general higher and more narrow than on \citet{FKMSR21}.
We believe this is due to the method of selecting the summaries to be annotated for each of the datasets.
\citet{BGALN20} selected summaries based on a stratified sample of automatic metric scores, whereas \citet{FKMSR21} selected summaries uniformly at random.
Therefore, the summaries in \citet{BGALN20} are likely easier to score (due to a mix of high- and low-quality summaries) and are less representative of the real data distribution than those in \citet{FKMSR21}.


\begin{figure*}
    \centering
    \includegraphics[width=1.0\textwidth]{figures/hypothesis-testing/pearson-row.pdf}
    \caption{The results of running the \textsc{Perm-Both} hypothesis test to find a significant difference between metrics' Pearson correlations.
    A blue square means the test returned a significant $p$-value at $\alpha = 0.05$, indicating the row metric has a higher correlation than the column metric.
    An orange outline means the result remained significant after applying the Bonferroni correction.
    }
    \label{fig:hypo}
\end{figure*}




\subsection{Hypothesis Testing}
\label{sec:hypo_experiments}
Although nearly all of the CIs for the metrics are overlapping, this does not necessarily mean that no metric is statistically better than another since the differences between two metrics' correlations could be significant.

In Fig.~\ref{fig:hypo}, we report the $p$-values for testing $H_0: \rho(\mathcal{X}, \mathcal{Z}) - \rho(\mathcal{Y}, \mathcal{Z}) \leq 0$ using the \textsc{Perm-Both} permutation test at the system- and summary-levels on TAC'08 and CNN/DM for all possible metric combinations (see \citet{AKSR20} for a discussion about how to interpret $p$-values).
The Bonferroni correction \citep[which lowers the significance level for rejecting each individual null hypothesis such that the probability of making one or more type-I errors is bounded by $\alpha$;][]{Bonferroni36,DBBR17} was applied to test suites grouped by the $\mathcal{X}$ metric at $\alpha = 0.05$.\footnote{
    A version of the results when the correction is applied to $p$-values grouped by the dataset and correlation level pair is included in Appendix~\ref{sec:bonferroni_full}.
}
A significant result means that we conclude that $\rho(\mathcal{X}, \mathcal{Z}) > \rho(\mathcal{Y}, \mathcal{Z})$.

The metrics which are identified as being statistically superior to others at the system-level on TAC'08 and CNN/DM using the annotations from \citet{FKMSR21} are QAEval and BERTScore.
Although they are statistically indistinguishable from each other, QA\-Eval does improve over more metrics than BERTScore does on TAC'08.
At the summary-level, BERTScore has significantly better results than all other metrics.
Overall, none of the other metrics consistently outperform all variants of ROUGE.
Results using either the Spearman or Kendall correlation coefficients are largely consistent with Fig.~\ref{fig:hypo}, although QA\-Eval no longer improves over some metrics, such as ROUGE-2, at the system-level on TAC'08.

The results on the CNN/DM annotations provided by \citet{BGALN20} are less clear.
The ROUGE variants appear to perform well, a conclusion also reached by \citet{BGALN20}.
The hypothesis tests also find that S3 is statistically better than most other metrics.
S3 scores systems using a learned combination of features which includes ROUGE scores, likely explaining this result.
Similarly to the CI experiment, the results on the annotations provided by \citet{BGALN20} and \citet{FKMSR21} are rather different, potentially due to differences in how the datasets were sampled.
\citet{FKMSR21} uniformly sampled summaries to annotate, whereas \citet{BGALN20} sampled them based on their approximate quality scores, so we believe the dataset of \citet{FKMSR21} is more likely to reflect the real data distribution.


\section{Limitations}
The large widths of the CIs in \S\ref{sec:ci_experiments} and the lack of some statistically significant differences between metrics in \S\ref{sec:hypo_experiments} are directly tied to the size of the datasets that were used in our analyses.
However, to the best of our knowledge, the datasets we used are some of the largest available with annotations of summary quality.
Therefore, the results presented here are our best efforts at accurately measuring the metrics' performances with the data available.
If we had access to larger datasets with more summaries labeled across more systems, we suspect that the scores of the human annotators and automatic metrics would stabilize to the point where the CI widths would narrow and it would be easier to find significant differences between metrics.

Although it is desirable to have larger datasets, collecting them is difficult because obtaining human annotations of summary quality is expensive and prone to noise.
Some studies report having difficulty obtaining high-quality judgments from crowdworkers \citep{GillickLi10,FKMSR21}, whereas others have been successful using the crowdsourced Lightweight Pyramid Score \citep{SGGRPBAD19}, which was used in \citet{BGALN20}.

Then, it is unclear how well our experiments' conclusions will generalize to other datasets with different properties, such as documents coming from different domains or different length summaries.
The experiments in \citet{BGALN20} show that metric performance depends on which dataset you use to evaluate, whether it be TAC or CNN/DM, which is supported by our results.
However, our experiments also show variability in performance within the same dataset when using different quality annotations (see the differences in results between \citet{FKMSR21} and \cite{BGALN20}).
Clearly, more research needs to be done to understand how much of these changes in performance is due to differences in the properties of the input documents and summaries versus how the summaries were annotated.
\section{Related Work}
\paragraph{Summarization}
CIs and hypothesis testing were applied for summarization evaluation metrics over the years in a relatively inconsistent manner -- if at all.
To the best of our knowledge, the only instances of calculating CIs for summarization metrics is at the system-level using a bootstrapping procedure equivalent to \textsc{Boot-Systems} \citep{RankelCoSc12,DavisCoSc12}.
Some works do perform hypothesis testing, but it is not clear which statistical test was run \citep{TratzHo08, GKVS08}.
Others report whether or not the correlation itself is significantly different from 0 \citep{Lin04}, which does not quantify the strength of the correlation nor allow for comparisons.
Some studies apply Williams' test to compare summarization metrics.
For instance, \citet{Graham15} use it to compare BLEU \citep{PRWZ02} and several variants of ROUGE, and \citet{BGALN20} compares several different metrics at the system-level.
However, our experiments demonstrated in \S\ref{sec:power} that Williams' test has lower power than the suggested methods due to the lower correlation values.

As an alternative to comparing metrics' correlations, \citet{OCDN12} argue for comparison based on the number of system pairs in which both human judgments and metrics agree on statistically significant differences between the systems, a metric also used in the TAC shared-task for summarization metrics \citep[][\emph{i.a.}]{DangOw09}.
This can be viewed similarly to Kendall's $\tau$ in which only statistically significant differences between systems are counted as concordant.
However, the differences in discriminative power across metrics was not statistically tested itself.

More broadly in evaluating summarization systems, \citet{RCSO11} argue for comparing the performance of summarization models via paired $t$-tests or Wilcoxon signed-rank tests \citep{Wilcoxon92}.
They demonstrate these tests have more power than the equivalent unpaired test when used to separate human and model summarizers.

\paragraph{Machine Translation}
The summarization and machine translation (MT) communities face the same problem of developing and evaluating automatic metrics to evaluate the outputs of models.
Since 2008, the Workshop on Machine Translation (WMT) has run a shared-task for developing evaluation metrics \citep[among others]{MWFMB20}.
Although the methodology has changed over the years, they have converged on comparing metrics' system-level correlations using Williams' test \citep{GrahamBa14}.
Since Williams' test assumes the input data is normally distributed and our experiments show it has low power for summarization, we do not recommend it for comparing summarization metrics.
However, human annotations for MT are standardized to be normally distributed, and the metrics have higher correlations to human judgments, thus Williams' test will probably have higher power when applied to MT metrics.
Nevertheless, the methods proposed in this work can be directly applied to MT metrics as well.





\section{Conclusion}
In this work, we proposed several different methods for estimating CIs and hypothesis testing for summarization evaluation metrics using resampling methods.
Our simulation experiments demonstrate that assuming variability in both the systems and input documents leads to the best generalization for CIs and that permutation-based hypothesis testing has the highest statistical power.
Experiments on several different evaluation metrics across three datasets demonstrate high uncertainty in how well metrics correlate to human judgments and that QA\-Eval and BERTScore do achieve higher correlations than ROUGE in some settings.

\section*{Acknowledgments}
The authors would like to thank Lyle Ungar, Daniel Khashabi, Eyal Ben David, and the anonymous reviewers for their valuable feedback on our work.

This work was partly supported by a a Focused Award from Google, by contracts FA8750-19-2-1004 and FA8750-19-2-0201 with the US Defense Advanced Research Projects Agency (DARPA), and by the Office of the Director of National Intelligence (ODNI), Intelligence Advanced Research Projects Activity (IARPA), via IARPA Contract No. 2019-19051600006 under the BETTER Program.
The views and conclusions contained herein are those of the authors and should not be interpreted as necessarily representing the official policies, either expressed or implied, of ODNI, IARPA, DARPA, the Department of Defense, or the U.S. Government. The U.S. Government is authorized to reproduce and distribute reprints for governmental purposes notwithstanding any copyright annotation therein.



\bibliography{new_ccg,new_cited,ccg,cited}
\bibliographystyle{acl_natbib}

% \clearpage
\appendix
\begin{table}
    \centering
    \begin{adjustbox}{width=\columnwidth}
    \begin{tabular}{ccccccccc}
        \toprule
        \multirow{2}{*}{\textbf{Metric}} & \multicolumn{2}{c}{\textbf{TAC'08}} & & \multicolumn{2}{c}{\textbf{Fabbri et al.}} & & \multicolumn{2}{c}{\textbf{Bhandari et al.}} \\
        \cmidrule{2-3} \cmidrule{5-6} \cmidrule{8-9}
        & $r_\textsc{Sum}$ & $r_\textsc{Sys}$ & & $r_\textsc{Sum}$ & $r_\textsc{Sys}$ & & $r_\textsc{Sum}$ & $r_\textsc{Sys}$ \\
        \midrule
Resp/Rel/Pyr & 100.0 & 0.00 &   & 32.0 & 0.52 &   & 75.0 & 0.84\\
AutoSummENG & 18.8 & 0.26 &   & 33.0 & 0.01 &   & 28.0 & 0.55\\
MeMoG & 37.5 & 0.53 &   & 33.0 & 0.01 &   & 28.0 & 0.55\\
NPowER & 29.2 & 0.36 &   & 33.0 & 0.01 &   & 28.0 & 0.55\\
BERTScore & 35.4 & 0.00 &   & 26.0 & 0.15 &   & 28.0 & 0.18\\
BEwTE & 22.9 & 0.06 &   & 37.0 & 0.00 &   & 33.0 & 0.68\\
METEOR & 27.1 & 0.15 &   & 27.0 & 0.00 &   & 30.0 & 0.61\\
MoverScore & 47.9 & 0.25 &   & 35.0 & 0.00 &   & 31.0 & 0.50\\
QAEval-F$_1$ & 58.3 & 0.00 &   & 40.0 & 0.01 &   & 45.0 & 0.21\\
ROUGE-1 & 33.3 & 0.06 &   & 32.0 & 0.00 &   & 30.0 & 0.91\\
ROUGE-2 & 31.2 & 0.71 &   & 34.0 & 0.00 &   & 61.0 & 0.62\\
ROUGE-L & 25.0 & 0.13 &   & 26.0 & 0.13 &   & 37.0 & 0.12\\
ROUGE-SU4 & 29.2 & 0.44 &   & 32.0 & 0.00 &   & 44.0 & 0.84\\
S3 & 20.8 & 0.32 &   & 26.0 & 0.00 &   & 47.0 & 0.66\\
  \bottomrule
    \end{tabular}
    \end{adjustbox} 
    \caption{For $\rsys$ the $p$-value of the Shapiro-Wilk test.
    For $\rsum$, the percent of the per-input document tests which had a significant result at $\alpha = 0.05$.
    A significant $p$-value means $H_0$ (the data is distributed normally) is rejected.
    For $\rsum$, the larger the percentage the more the data appears to be not normally distributed.}
    \label{tab:normality}
\end{table}

\section{Normality Testing}
\label{appendix:normality}
% Some statistical methods for calculating the CIs or running hypothesis tests for correlation coefficients assume the input variables are normally distributed.
% To understand if this assumption holds for the summarization data, we ran the Shapiro-Wilk test for normality \citep{ShapiroWi65}, which was reported to have the highest power out of several alternatives \citep{RazaliWa11}.

To understand if the normality assumption holds for summarization data we ran the Shapiro-Wilk test for normality \citep{ShapiroWi65}, which was reported to have the highest power out of several alternatives \citep{RazaliWa11,DBSR18,DPSR20}.
The results of the tests for the ground-truth responsiveness scores and automatic metrics are in Table~\ref{tab:normality}.
Most of the $p$-values are significant, i.e., applying a statistical test which assumes normality is incorrect in general.


\section{Extended Bonferroni Correction}
\label{sec:bonferroni_full}
Fig.~\ref{fig:hypo_full_table} contains the results from the pairwise hypothesis tests (\S\ref{sec:hypo_experiments}) when then Bonferroni correction is applied to set of $p$-values grouped by the dataset and correlation level pair instead of each dataset, correlation level, and metric shown in Fig.~\ref{fig:hypo}.
The results are overall very similar with only a handful of results now becoming not significant.

\begin{figure*}[h!]
    \centering
    \includegraphics[width=1.0\textwidth]{figures/hypothesis-testing/pearson-table.pdf}
    \caption{The results of running the \textsc{Perm-Both} hypothesis test to find a significant difference between metrics' Pearson correlations with the Bonferroni correction applied per dataset and correlation level pair instead of per metric (as in Fig.~\ref{fig:hypo}).
    A blue square means the test returned a significant $p$-value at $\alpha = 0.05$, indicating the row metric has a higher correlation than the column metric.
    An orange outline means the result remained significant after applying the Bonferroni correction.
    }
    \label{fig:hypo_full_table}
\end{figure*}


\end{document}


