\begin{abstract}
The quality of a summarization evaluation metric is quantified by calculating the correlation between its scores and human annotations across a large number of summaries. Currently, it is unclear how precise these correlation estimates are, nor whether differences between two metrics' correlations reflect a true difference or if it is due to mere chance. In this work, we address these two problems by proposing methods for calculating confidence intervals and running hypothesis tests for correlations using two resampling methods, bootstrapping and permutation. After evaluating which of the proposed methods is most appropriate for summarization through two simulation experiments, we analyze the results of applying these methods to several different automatic evaluation metrics across three sets of human annotations. We find that the confidence intervals are rather wide, demonstrating high uncertainty in the reliability of automatic metrics. Further, although many metrics fail to show statistical improvements over ROUGE, two recent works, QA\-Eval and BERTScore, do in some evaluation settings.\footnote{
Our code is available at \url{https://github.com/CogComp/stat-analysis-experiments}.
}
\end{abstract}