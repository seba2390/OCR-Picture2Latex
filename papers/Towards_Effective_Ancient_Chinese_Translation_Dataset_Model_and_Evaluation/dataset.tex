\section{Erya Dataset}
%加引用
%把guwenbert作为bib的引用
We construct an open-source collection, named \textbf{Erya dataset}, consisting of ancient monolingual corpus and ancient-modern parallel corpus. Our Erya dataset currently stands as the most extensive ancient Chinese resource.

% through the process of corpus aggregation and data pre-procession. Also, we propose a classification method, which will be used in later experiments.

%重点放在平行语料上
\subsection{Data Collection and Cleaning}                       
In order to construct our raw corpus, we first crawl data from the Internet~\cite{wyw5156} and collect open source data~\cite{TP-toolbox-web2,TP-toolbox-web3, TP-toolbox-web1,TP-toolbox-web4,TP-toolbox-web5,TP-toolbox-web6}. Most of the ancient Chinese texts are written throughout the dynasties of ancient China spanning from 1000 BC to AD 1600. Then, we carry out a cleaning process to eliminate noise from the raw corpus and ensure consistent formatting as follows:
\begin{itemize}[leftmargin=*]
\itemsep0em 
%下面加一些参考文献
\item \textbf{Noise Filtering}
We design three rules to filter out noises: (1) delete tokens other than Chinese characters (\eg Arabic numerals and English words); (2) simplify traditional Chinese characters; and (3) unify non-text characters such as punctuation (\eg converting \begin{CJK*}{UTF8}{gbsn}「\end{CJK*} to \begin{CJK*}{UTF8}{gbsn}“\end{CJK*}).

\item \textbf{De-duplication}
Since our raw corpus is collected from different sources, \ie the contents may overlap with each other, we further leverage MinHash algorithm for efficient de-duplication. We compare text pieces from various sources and only keep one copy if the similarity between the two pieces is below 0.5.
% Since our raw corpus is collected from different sources, \ie the contents may overlap with each other, we further apply de-duplication algorithms, including MinHash and longest common subsequence. 

%minhash

% Due to the enormous size of monolingual data, and most of the data is in the form of articles, we adopt the MinHash algorithm for more efficient deduplication.
% The text pieces from different sources are pairwise compared for similarity by MinHash algorithm, and one copy will be retained only when the similarity between two documents is less than 0.5.


% Besides the MinHash algorithm, we also propose a more fine-grained duplicate removal method based on longest common subsequence. Denote two ancient Chinese sentences from the corpus as $X, Y$, the length of their longest common subsequence as $l$. Then $l$ is computed via dynamic programming, thus the similarity between $X$ and $Y$ is $\frac{l}{max(len(X), len(Y))}$. When this similarity between two sentences is greater than 0.5, we then output them to a separate file, so that we can later compare them manually.

\item \textbf{Automatic Punctuation}
Since some collected monolingual texts lack punctuation, we employ the guwen-punc~\cite{TP-toolbox-web} toolkit to add punctuation.


%Considering the ancient Chinese corpus may lack punctuation, we employ the guwen-punc~\cite{TP-toolbox-web} toolkit to add punctuation.

\end{itemize}



\subsection{Data Classification}%%
After data processing to construct a cleaned Erya dataset for ancient Chinese, we further propose a method to classify it based on textual characteristics. We incorporate the traditional ``Four-Branch Classification''~\cite{sibu} and a chronological classification~\cite{RN02, four} to consider the characteristics of different dynasties, grammar, sentence structure, and background knowledge. The details of the whole classification system are as follows:
% We apply the traditional ``Four-Branch Classification''~\cite{张凌霄2002四部分类法} to better accommodate the characteristics of our datasets. 
% On the basis of this, considering that in different dynasties, the grammar, sentence structure, and background knowledge are not exactly the same, we also propose a chronological classification~\cite{程湘清1991汉语史断代专书研究方法论} for historical texts. The details of whole classification system are as follows:

%Based on the traditional ``Four-Branch Classification" system~\cite{张凌霄2002四部分类法} which can basically adapt to the knowledge structure of ancient Chinese, 

%, in order to examine the model's ability to transfer across different historical periods.
%. However, the criteria for the ``Four Categories" classification system are influenced by many non-knowledge factors. For example, the development of the ``Canonical" category was actually influenced by the knowledge value judgments of the ruling class, and it includes multiple genres. Therefore, in order to accurately classify the corpus, and evaluate different capabilities of model on texts of different categories, we improved the ``Four Categories" system.


\begin{itemize}
    \item \textbf{History}: It includes historical literature that document dynastic changes, significant events and notable figures.
    \begin{itemize}
        \item \textbf{Old Chinese}: Pre-Qin to Han Dynasty (before AD3)
        \item \textbf{Middle Chinese}: Three kingdoms to Song Dynasty (AD4 to AD12)
        \item \textbf{Early Modern Chinese}: Yuan to Qing Dynasty (AD13 to AD19)
    \end{itemize}
    \item \textbf{Article}: It contains various literary works, including poetry, prose, philosophy works and literary criticism. 
    \item \textbf{Novel}: It mixes ancient and quasi-modern styles, thus having its own unique textual genres and linguistic features.
\end{itemize}

% %经典三线表
% \begin{table}[H]
% \caption{\textbf{Classification of ancient Chinese}}
% \centering
% \begin{tabular}{cc}%四个c代表有四列且内容居中
% \toprule%第一道横线
% \textbf{Category}&\textbf{Description}\\
% \midrule%第二道横线 
% \multirow{2}{*}{History}&It contains historical accounts, geographical and official literature\\%Data1跨两行,自动表格宽度
% &on administrative rituals.\\
% \midrule%第二道横线 
% \multirow{2}{*}{Article}&It contains various literary works, including poetry, prose,\\%Data1跨两行,自动表格宽度
% &and literary criticism.\\
% \midrule%第二道横线 
% \multirow{2}{*}{Novel}&Mostly a mix of ancient and modern styles, \\%Data1跨两行,自动表格宽度
% &thus having its own unique style and linguistic features.\\
% \bottomrule%第四道横线
% \end{tabular}
% \end{table}


%经典三线表
% \begin{table}[H]
% \caption{\textbf{Chronological Classification of history}}
% \centering
% \begin{tabular}{cc}%四个c代表有四列且内容居中
% \toprule%第一道横线
% \textbf{Category}&\textbf{Description}\\
% \midrule%第二道横线 
% \textbf{Pre-historic times}&Pre-Qin to Han Dynasty (before 3 A.D.)\\%Data1跨两行,自动表格宽度
% \midrule%第二道横线 
% \textbf{Medieval period}&Western Jin to Northern Song (4 A.D. to 12 A.D.)\\
% \midrule%第二道横线 
% \textbf{Contemporary period}&Southern Song to Qing Dynasty\\
% &(13 A.D. to 19 A.D.)\\
% \bottomrule%第四道横线
% \end{tabular}
% \end{table}

\subsection{Statistics}

In summary, our Erya dataset consists of a total of 88,808,928 ancient Chinese sentences and 1,941,396,399 characters with an average sentence length of 21.9. The parallel corpus within it comprises a total of 2,087,804 sentences and 84,769,383 characters. The average sentence length of ancient and modern sentences in the parallel part are 17.3 and 23.3 characters respectively. We present a comparison between existing ancient Chinese resources in Table~\ref{tab:dataset}. We can see that our Erya dataset stands out as the most abundant resource among both monolingual and parallel data at present.

% As for the details of our dataset (Erya-dataset), the corpus of ancient Chinese texts consists of a total of 88,808,928 sentences and 1,941,396,399 characters. Its average sentence length is 21.86 characters, and it occupies 5.26GB. The parallel corpus within it comprises a total of 2,087,804 sentences and 84,769,383 characters. The source texts have an average sentence length of 17.3 characters, while the target texts have an average sentence length of 23.31 characters. The parallel corpus occupies 248.9MB.


%经典三线表
% \begin{table}[htbp]
% \small
% \caption{\textbf{Statistics of Erya-dataset}}
% \centering
% \begin{tabular}{c|cccc}%四个c代表有四列且内容居中
% \toprule%第二道横线 
% &\textbf{\#Sentences}&\textbf{\#Characters}&\textbf{Avg Sentence Length}&\textbf{Size} \\%Data1跨两行,自动表格宽度
% \midrule%第三道横线 
% \textbf{Monolingual}&88,808,928&1,941,396,399&21.86&5.26GB\\
% \midrule%第三道横线 
% \textbf{Source}&2,087,804&36,110,268&17.30&102.9MB\\
% \midrule%第三道横线 
% \textbf{Target}&2,087,804&48,659,115&23.31&146MB\\
% \bottomrule%第四道横线
% \end{tabular}
% \end{table}

\begin{table}[htbp]
\small
\caption{\textbf{Comparison between existing ancient Chinese resources and Erya dataset. Mono. and Para. denote the number of characters in monolingual corpus and parallel corpus (including source and target), respectively.}}
\label{tab:dataset}
\centering
\resizebox{1.0\linewidth}{!}{
\begin{tabular}{c|ccccccc|c}%四个c代表有四列且内容居中
\toprule%第二道横线 
\textbf{Datasets}&Time~\cite{DBLP:conf/acl-lchange/ChangSYD21}&Guwen~\cite{DBLP:conf/nlpcc/YangCC21}&Anchi~\cite{DBLP:journals/corr/abs-2009-11473}&DBU~\cite{TP-toolbox-web1}&Daizhige~\cite{TP-toolbox-web2}&THUCC~\cite{TP-toolbox-web3}&NLD~\cite{DBLP:journals/talip/LiuYQL20}&\textbf{Erya dataset}\\%Data1跨两行,自动表格宽度
\midrule%第三道横线 
% \textbf{Type}&mono&mono&mono&mono&mono&-&-&-&mono\\
% \textbf{\#Sent}&269.4K&-&-&785.2K&-&-&-&-&88809K\\
\textbf{Mono.}&4M&1,743M&395M&89M&1,743M&-&-&\textbf{1,941M}\\
\midrule%第三道横线 
% \textbf{Type}&para&para&para&para&-&para&para&para&para\\
\textbf{Para.}&1.6M&2.5M&31.5M&56.5M&-&13.4M&37.9M&\textbf{84.8M}\\
% \textbf{\#Char}&463K, 975K&-&-&22.9M, 33.6M&-&5.5M, 7.9M&-&15.2M, 22.6M&36.1M, 48.7M\\
\bottomrule%第四道横线
\end{tabular}
}
\end{table}

The monolingual ancient texts can be utilized to learn the general knowledge in ancient Chinese, while the parallel data can bridge the linguistic gap between ancient and modern Chinese. Furthermore, considering the scarcity of a benchmark for ancient Chinese translation, we design \textbf{Erya benchmark}, a subset of the parallel data based on our classification criteria. We have taken into account the diverse textual characteristics of ancient Chinese literature and linguistic evolution throughout the ages. The detailed statistics are listed in Table~\ref{tab:stat}.



% Our Erya dataset includes both monolingual and parallel corpora. We use the parallel part to pre-train and fine-tune our model, and we expect to utilize the monolingual part in the future for further pre-training specifically focuses on ancient Chinese. Considering the diverse textual characteristics of ancient Chinese literature and linguistic evolution throughout ages, we build a benchmark based on our classification criteria to facilitate the research on ancient Chinese translation. Details are shown in Table~\ref{tab:stat}.
% %, we introduce the Erya benchmark as shown in Table~\ref{tab:stat}, which 
% %comprises five books and spans a diverse range of literary genres and temporal epochs. 
% The benchmark is aimed at evaluating translation quality and domain transfer capability of model across a range of literary genres and temporal epochs.  


\begin{table}[htbp]
\caption{\textbf{Statistics of Erya benchmark. \#ASL is the average sentence length.}}
\small
\label{tab:stat}
\centering
\resizebox{0.8\linewidth}{!}{
\begin{tabular}{c|ccc|c|c}%四个c代表有四列且内容居中
\toprule%第二道横线 
&&\textbf{History}&&\multirow{2}{*}{\textbf{Article}}&\multirow{2}{*}{\textbf{Novel}} \\%Data1跨两行,自动表格宽度
&\textbf{Old}&\textbf{Middle}&\textbf{Early Modern}& \\
\midrule%第三道横线 
\multirow{2}{*}{\textbf{Title}}&\textit{Book of Han}&\textit{New Tang History}&\textit{Ming History}&\textit{Xu Xiake's Travels}&\textit{Taiping Guangji}\\
&(\begin{CJK*}{UTF8}{gbsn}汉书\end{CJK*})&(\begin{CJK*}{UTF8}{gbsn}新唐书\end{CJK*})&(\begin{CJK*}{UTF8}{gbsn}明史\end{CJK*})&(\begin{CJK*}{UTF8}{gbsn}徐霞客游记\end{CJK*})&(\begin{CJK*}{UTF8}{gbsn}太平广记\end{CJK*})\\
\midrule%第三道横线 
\textbf{\#Train}&18,646&9,396&66,730&16,649&45,162 \\
\midrule%第三道横线 
\textbf{\#Valid}&2,331&1,174&8,341&2,081&5,645 \\
\midrule%第三道横线 
\textbf{\#Test}&2,331&1,175&8,342&2,082&5,646 \\
% \midrule%第三道横线 
% \textbf{\#Total Sentence}&23308&11745&83413&20812&56453 \\
\midrule%第三道横线 
\textbf{\#ASL}&21.2&20.5&21.5&25.1&20.0 \\
% \midrule%第三道横线 
% \textbf{\#Size}&3.45MB&1.62MB&12.1MB&3.58MB&8.11MB \\
% \midrule%第三道横线 
% \textbf{Out of domain}&\textit{Shiji}(\begin{CJK*}{UTF8}{gbsn}史记\end{CJK*})&\textit{Book of Chen}(\begin{CJK*}{UTF8}{gbsn}陈书\end{CJK*})&\textit{History of Liao}(\begin{CJK*}{UTF8}{gbsn}辽史\end{CJK*})&\textit{Commentary on the Water Classic}(\begin{CJK*}{UTF8}{gbsn}水经注)\end{CJK*}&- \\
% \midrule%第三道横线 
% \textbf{\#Out of domain}&17701&7096&9278&11630&- \\
\bottomrule%第四道横线
\end{tabular}
}
\end{table}



% \caption{\textbf{Comparison between existing ancient and modern Chinese datasets and Erya-dataset}}
% \centering
% \begin{tabular}{c|ccc}%四个c代表有四列且内容居中
% \toprule%第二道横线 
% \textbf{Dataset Source}&\textbf{Type}&\textbf{\#Sentences}&\textbf{\#Characters} \\%Data1跨两行,自动表格宽度
% \midrule%第三道横线
% \multirow{2}{*}{Time-aware~\cite{DBLP:conf/acl-lchange/ChangSYD21}}&parallel&33.5K&643K, 975K\\
% &monolingual&269.4K&4M\\
% \midrule%第三道横线
% \multirow{2}{*}{Guwen-UNILM~\cite{DBLP:conf/nlpcc/YangCC21}}&parallel&68K&-\\
% &monolingual&-&1743M\\
% \midrule%第三道横线
% \multirow{2}{*}{AnchiBERT~\cite{DBLP:journals/corr/abs-2009-11473}}&parallel&1200K&-\\
% &monolingual&-&395M\\
% \midrule%第三道横线
% \multirow{2}{*}{DBU\footnote{\url{github.com/NiuTrans/Classical-Modern}}}&parallel&972.5K&22.9M, 33.6M\\
% &monolingual&785.2K&88.8M\\
% \midrule%第三道横线
% A new large dataset~\cite{DBLP:journals/talip/LiuYQL20}&parallel&1246K&15.2M, 22.6M\\
% \midrule%第三道横线
% Limited aligned Corpora~\cite{DBLP:journals/corr/abs-1803-01557}&parallel&57.4K&-\\
% \midrule%第三道横线
% THUCC\footnote{\url{github.com/THUNLP-MT/THUCC}}&parallel&293.3K&5.5M, 7.9M\\
% \midrule%第三道横线
% Daizhige\footnote{\url{github.com/up2hub/daizhige}}&monolingual&-&1743M\\
% \midrule%第三道横线
% \multirow{2}{*}{Erya-dataset}&parallel&2088K&36.1M, 48.7M\\
% &monolingual&88809K&1941M\\
% \bottomrule%第四道横线
% \end{tabular}
% \end{table}

%经典三线表
% \begin{table}[H]
% \caption{\textbf{Statistics of ERYA-large and ERYA-base}}
% \centering
% \begin{tabular}{c|ccc|cc}%四个c代表有四列且内容居中
% \toprule%第二道横线 
% &&ERYA-large&&\multicolumn{2}{c} \textbf{ERYA-base} \\%Data1跨两行,自动表格宽度
% &\textbf{Monolingual}&\textbf{Source}&\textbf{Target}&\textbf{Source}&\textbf{Target} \\%Data1跨两行,自动表格宽度
% \midrule%第三道横线 
% \textbf{\#Total Sentences}&88808928&1849896&1849896&237908&237908 \\
% \midrule%第三道横线 
% \textbf{\#Characters}&1941396399&31018187&41311925&5092081&7347190 \\
% \midrule%第三道横线 
% \textbf{\#Avg Sentence Length}&21.86&16.78&22.33&21.4&30.88 \\
% \midrule%第三道横线 
% \textbf{Size}&5.26GB&87.7MB&130MB&15.2MB&16MB \\
% \bottomrule%第四道横线
% \end{tabular}
% \end{table}


% To be specific, in order to better evaluate the performance of our model in different task domains, we also performed a more refined classification for the ERYA based on textual characteristics described in the section 3.2, then constructed a dataset for testing named ERYA-test, which is a subset of ERYA. The more refined statistics for ERYA-test is as follows:

% \begin{table}[H]
% \caption{\textbf{Statistics and Classification of ERYA-test}}
% \centering
% \begin{tabular}{c|ccc|c|c}%四个c代表有四列且内容居中
% \toprule%第二道横线 
% &&\textbf{History}&&\textbf{Article}&\textbf{Novel} \\%Data1跨两行,自动表格宽度
% &\textbf{Pre-historic}&\textbf{Medieval}&\textbf{Contemporary}& \\
% \midrule%第三道横线 
% \textbf{Title}&\begin{CJK*}{UTF8}{gbsn}汉书\end{CJK*}&\begin{CJK*}{UTF8}{gbsn}新唐书\end{CJK*}&\begin{CJK*}{UTF8}{gbsn}明史\end{CJK*}&\begin{CJK*}{UTF8}{gbsn}徐霞客游记\end{CJK*}&\begin{CJK*}{UTF8}{gbsn}太平广记\end{CJK*} \\
% \midrule%第三道横线 
% \textbf{\#Train}&18646&9396&66730&16649&45162 \\
% \midrule%第三道横线 
% \textbf{\#Valid}&2331&1174&8341&2081&5645 \\
% \midrule%第三道横线 
% \textbf{\#Test}&2331&1175&8342&2082&5646 \\
% \midrule%第三道横线 
% \textbf{\#Total Sentence}&23308&11745&83413&20812&56453 \\
% \midrule%第三道横线 
% \textbf{\#Avg Sentence}&\multirow{2}{*}{21.21}&\multirow{2}{*}{20.5}&\multirow{2}{*}{21.54}&\multirow{2}{*}{25.12}&\multirow{2}{*}{20.02} \\
% \textbf{Length}&&&& \\
% \midrule%第三道横线 
% \textbf{\#Size}&3.45MB&1.62MB&12.1MB&3.58MB&8.11MB \\
% \midrule%第三道横线 
% \textbf{Out of domain}&\begin{CJK*}{UTF8}{gbsn}史记\end{CJK*}&\begin{CJK*}{UTF8}{gbsn}陈书\end{CJK*}&\begin{CJK*}{UTF8}{gbsn}辽史\end{CJK*}&\begin{CJK*}{UTF8}{gbsn}水经注\end{CJK*}&- \\
% \midrule%第三道横线 
% \textbf{\#Out of domain}&17701&7096&9278&11630&- \\
% \bottomrule%第四道横线
% \end{tabular}
% \end{table}
