% This is samplepaper.tex, a sample chapter demonstrating the
% LLNCS macro package for Springer Computer Science proceedings;
% Version 2.20 of 2017/10/04
%
\documentclass[runningheads]{llncs}
%
\usepackage{soul, color, xcolor}
\usepackage{graphicx}
\usepackage{CJKutf8}
\usepackage{CJK}
\usepackage{graphicx}
\usepackage{float}%提供float浮动环境
\usepackage{hyperref}
\hypersetup{hypertex=true,
            colorlinks=true,
            linkcolor=blue,
            anchorcolor=blue,
            citecolor=blue}
\usepackage{booktabs}%提供命令\toprule、\midrule、\bottomrule
\usepackage{multirow}%提供跨列命令\multicolumn{}{}{}
% Used for displaying a sample figure. If possible, figure files should
% be included in EPS format.
%
% If you use the hyperref package, please uncomment the following line
% to display URLs in blue roman font according to Springer's eBook style:
\renewcommand\UrlFont{\color{blue}\rmfamily}
\usepackage{biblatex}
%\usepackage{changes}
\usepackage{bm}
\usepackage{amssymb}
\usepackage{xspace}
\usepackage{enumitem}
\usepackage{amsmath}
\usepackage{bbding}

% \usepackage[backend=biber,style=gb7714-2015,seconds=true,gbnamefmt=lowercase,gbpub=false]{biblatex}

\newcommand{\ie}{\textit{i.e.,}\xspace}
\newcommand{\aka}{\textit{a.k.a.,}\xspace}
\newcommand{\eg}{\textit{e.g.,}\xspace}
\newcommand{\wrt}{\textit{w.r.t.}\xspace}
\newcommand{\wo}{\textit{w/o}\xspace}
\newcommand{\w}{\textit{w/}\xspace}
\newcommand{\etc}{\textit{etc}}
\newcommand{\ignore}[1]{}

\renewcommand{\labelitemi}{\textbullet}
\renewcommand{\labelitemii}{-}

\addbibresource{ref.bib}

\begin{document}
%
%\title{Erya: A Suite for Translation between Ancient Chinese and Modern Chinese}
\title{Towards Effective Ancient Chinese Translation: Dataset, Model, and Evaluation}
%
\titlerunning{Towards Effective Ancient Chinese Translation}
% If the paper title is too long for the running head, you can set
% an abbreviated paper title here
%
\author{Geyang Guo$^1$ \and
Jiarong Yang$^1$ \and
Fengyuan Lu$^1$ \and
Jiaxin Qin$^1$ \and
\\Tianyi Tang$^1$ \and
Wayne Xin Zhao$^{1,2}$\thanks{Corresponding author.}}
%
\authorrunning{G. Guo et al.}
% First names are abbreviated in the running head.
% If there are more than two authors, 'et al.' is used.
%
\institute{$^1$Gaoling School of Artificial Intelligence, Renmin University of China\\
$^2$Beijing Key Laboratory of Big Data Management and Analysis Methods\\
\email{\{guogeyang,yangjiarong001,lufengyuan,2020201476\}@ruc.edu.cn,\\steventianyitang@outlook.com,batmanfly@gmail.com}}
%
\maketitle              % typeset the header of the contribution
%
  In this paper, we explore the connection between secret key agreement and secure omniscience within the setting of the multiterminal source model with a wiretapper who has side information. While the secret key agreement problem considers the generation of a maximum-rate secret key through public discussion, the secure omniscience problem is concerned with communication protocols for omniscience that minimize the rate of information leakage to the wiretapper. The starting point of our work is a lower bound on the minimum leakage rate for omniscience, $\rl$, in terms of the wiretap secret key capacity, $\wskc$. Our interest is in identifying broad classes of sources for which this lower bound is met with equality, in which case we say that there is a duality between secure omniscience and secret key agreement. We show that this duality holds in the case of certain finite linear source (FLS) models, such as two-terminal FLS models and pairwise independent network models on trees with a linear wiretapper. Duality also holds for any FLS model in which $\wskc$ is achieved by a perfect linear secret key agreement scheme. We conjecture that the duality in fact holds unconditionally for any FLS model. On the negative side, we give an example of a (non-FLS) source model for which duality does not hold if we limit ourselves to communication-for-omniscience protocols with at most two (interactive) communications.  We also address the secure function computation problem and explore the connection between the minimum leakage rate for computing a function and the wiretap secret key capacity.
  
%   Finally, we demonstrate the usefulness of our lower bound on $\rl$ by using it to derive equivalent conditions for the positivity of $\wskc$ in the multiterminal model. This extends a recent result of Gohari, G\"{u}nl\"{u} and Kramer (2020) obtained for the two-user setting.
  
   
%   In this paper, we study the problem of secret key generation through an omniscience achieving communication that minimizes the 
%   leakage rate to a wiretapper who has side information in the setting of multiterminal source model.  We explore this problem by deriving a lower bound on the wiretap secret key capacity $\wskc$ in terms of the minimum leakage rate for omniscience, $\rl$. 
%   %The former quantity is defined to be the maximum secret key rate achievable, and the latter one is defined as the minimum possible leakage rate about the source through an omniscience scheme to a wiretapper. 
%   The main focus of our work is the characterization of the sources for which the lower bound holds with equality \textemdash it is referred to as a duality between secure omniscience and wiretap secret key agreement. For general source models, we show that duality need not hold if we limit to the communication protocols with at most two (interactive) communications. In the case when there is no restriction on the number of communications, whether the duality holds or not is still unknown. However, we resolve this question affirmatively for two-user finite linear sources (FLS) and pairwise independent networks (PIN) defined on trees, a subclass of FLS. Moreover, for these sources, we give a single-letter expression for $\wskc$. Furthermore, in the direction of proving the conjecture that duality holds for all FLS, we show that if $\wskc$ is achieved by a \emph{perfect} secret key agreement scheme for FLS then the duality must hold. All these results mount up the evidence in favor of the conjecture on FLS. Moreover, we demonstrate the usefulness of our lower bound on $\wskc$ in terms of $\rl$ by deriving some equivalent conditions on the positivity of secret key capacity for multiterminal source model. Our result indeed extends the work of Gohari, G\"{u}nl\"{u} and Kramer in two-user case.
%
%
%

% !TEX root = ../arxiv.tex

Unsupervised domain adaptation (UDA) is a variant of semi-supervised learning \cite{blum1998combining}, where the available unlabelled data comes from a different distribution than the annotated dataset \cite{Ben-DavidBCP06}.
A case in point is to exploit synthetic data, where annotation is more accessible compared to the costly labelling of real-world images \cite{RichterVRK16,RosSMVL16}.
Along with some success in addressing UDA for semantic segmentation \cite{TsaiHSS0C18,VuJBCP19,0001S20,ZouYKW18}, the developed methods are growing increasingly sophisticated and often combine style transfer networks, adversarial training or network ensembles \cite{KimB20a,LiYV19,TsaiSSC19,Yang_2020_ECCV}.
This increase in model complexity impedes reproducibility, potentially slowing further progress.

In this work, we propose a UDA framework reaching state-of-the-art segmentation accuracy (measured by the Intersection-over-Union, IoU) without incurring substantial training efforts.
Toward this goal, we adopt a simple semi-supervised approach, \emph{self-training} \cite{ChenWB11,lee2013pseudo,ZouYKW18}, used in recent works only in conjunction with adversarial training or network ensembles \cite{ChoiKK19,KimB20a,Mei_2020_ECCV,Wang_2020_ECCV,0001S20,Zheng_2020_IJCV,ZhengY20}.
By contrast, we use self-training \emph{standalone}.
Compared to previous self-training methods \cite{ChenLCCCZAS20,Li_2020_ECCV,subhani2020learning,ZouYKW18,ZouYLKW19}, our approach also sidesteps the inconvenience of multiple training rounds, as they often require expert intervention between consecutive rounds.
We train our model using co-evolving pseudo labels end-to-end without such need.

\begin{figure}[t]%
    \centering
    \def\svgwidth{\linewidth}
    \input{figures/preview/bars.pdf_tex}
    \caption{\textbf{Results preview.} Unlike much recent work that combines multiple training paradigms, such as adversarial training and style transfer, our approach retains the modest single-round training complexity of self-training, yet improves the state of the art for adapting semantic segmentation by a significant margin.}
    \label{fig:preview}
\end{figure}

Our method leverages the ubiquitous \emph{data augmentation} techniques from fully supervised learning \cite{deeplabv3plus2018,ZhaoSQWJ17}: photometric jitter, flipping and multi-scale cropping.
We enforce \emph{consistency} of the semantic maps produced by the model across these image perturbations.
The following assumption formalises the key premise:

\myparagraph{Assumption 1.}
Let $f: \mathcal{I} \rightarrow \mathcal{M}$ represent a pixelwise mapping from images $\mathcal{I}$ to semantic output $\mathcal{M}$.
Denote $\rho_{\bm{\epsilon}}: \mathcal{I} \rightarrow \mathcal{I}$ a photometric image transform and, similarly, $\tau_{\bm{\epsilon}'}: \mathcal{I} \rightarrow \mathcal{I}$ a spatial similarity transformation, where $\bm{\epsilon},\bm{\epsilon}'\sim p(\cdot)$ are control variables following some pre-defined density (\eg, $p \equiv \mathcal{N}(0, 1)$).
Then, for any image $I \in \mathcal{I}$, $f$ is \emph{invariant} under $\rho_{\bm{\epsilon}}$ and \emph{equivariant} under $\tau_{\bm{\epsilon}'}$, \ie~$f(\rho_{\bm{\epsilon}}(I)) = f(I)$ and $f(\tau_{\bm{\epsilon}'}(I)) = \tau_{\bm{\epsilon}'}(f(I))$.

\smallskip
\noindent Next, we introduce a training framework using a \emph{momentum network} -- a slowly advancing copy of the original model.
The momentum network provides stable, yet recent targets for model updates, as opposed to the fixed supervision in model distillation \cite{Chen0G18,Zheng_2020_IJCV,ZhengY20}.
We also re-visit the problem of long-tail recognition in the context of generating pseudo labels for self-supervision.
In particular, we maintain an \emph{exponentially moving class prior} used to discount the confidence thresholds for those classes with few samples and increase their relative contribution to the training loss.
Our framework is simple to train, adds moderate computational overhead compared to a fully supervised setup, yet sets a new state of the art on established benchmarks (\cf \cref{fig:preview}).


\section{Related Work}
\label{sec:related_work}
We now provide a brief overview of related work in the areas of language grounding and transfer for reinforcement learning.
%There has been work on learning to make optimal local decisions for structured prediction problems~\cite{daume2006searn}.
%
%\newcite{branavan2010reading} looked at a similar task of building a partial model of the environment while following instructions. The differences with our work are (1) the text in their case is instructions, while we only have text describing the environment, and (2) their environment is deterministic, hence the transition function can be learned more easily. 
%
%TODO - model-based RL, value iteration, predictron.


\subsection{Grounding Language in Interactive Environments}
In recent years, there has been increasing interest in systems that can utilize textual knowledge to learn control policies. Such applications include interpreting help documentation~\fullcite{branavan2010reading}, instruction following~\fullcite{vogel2010learning,kollar2010toward,artzi2013weakly,matuszek2013learning,Andreas15Instructions} and learning to play computer games~\fullcite{branavan2011nonlinear,branavan2012learning,narasimhan2015language,he2016deep}. In all these applications, the models are trained and tested on the same domain.

Our work represents two departures from prior work on grounding. First, rather than optimizing control performance for a single domain,
we are interested in the multi-domain transfer scenario, where language 
descriptions drive generalization. Second, prior work used text in the form of strategy advice to directly learn the policy. Since the policies are typically optimized for a specific task, they may be harder to transfer across domains. Instead, we utilize text to bootstrap the induction of the environment dynamics, moving beyond task-specific strategies. 

%Previous work has explored the use of text manuals in game playing, %ranging from constructing useful features by mining patterns in %text~\cite{eisenstein2009reading}, learning a semantic interpreter %with access to limited gameplay examples~\cite{goldwasser2014learning} %to learning through reinforcement from in-game %rewards~\cite{branavan2011learning}. These efforts have demonstrated %the usefulness of exploiting domain knowledge encoded in text to learn %effective policies. However, these methods use the text to infer %directly the best strategy to perform a task. In contrast, our goal is %to learn mappings from the text to the dynamics of an environment and %separate out the learning of the strategy/motives. 

Another related line of work consists of systems that learn spatial and topographical maps of the environment for robot navigation using natural language descriptions~\fullcite{walter2013learning,hemachandra2014learning}. These approaches use text mainly containing appearance and positional information, and integrate it with other semantic sources (such as appearance models) to obtain more accurate maps. In contrast, our work uses language describing the dynamics of the environment, such as entity movements and interactions, which 
is complementary to static positional information received through state observations. Further, our goal is to help an agent learn policies that generalize over different stochastic domains, while their works consider a single domain.

%karthik: I don't see the direct relevance
%Another line of work explores using textual interactive %environments~\cite{narasimhan2015language,he2016deep} to ground %language understanding into actions taken by the system in the %environment. In these tasks, understanding of language is crucial, %without which a system would not be able to take reasonable actions. %Our motivation is different -- we take an existing set of tasks and %domains which are amenable to learning through reinforcement, and %demonstrate how to utilize textual knowledge to learn faster and more %optimal policies in both multitask and transfer setups.

\subsection{Transfer in Reinforcement Learning}
Transferring policies across domains is a challenging problem in reinforcement learning~\fullcite{konidaris2006framework,taylor2009transfer}. The main hurdle lies in learning a good mapping between the state and action spaces of different domains to enable effective transfer. Most previous approaches have either explored skill transfer~\fullcite{konidaris2007building,konidaris2012transfer} or value function/policy transfer~\fullcite{liu2006value,taylor2007transfer,taylor2007cross}. There have also been attempts at model-based transfer for RL~\fullcite{taylor2008transferring,nguyen2012transferring,gavsic2013pomdp,wang2015learning,joshi2018cross} but these methods either rely on hand-coded inter-task mappings for state and actions spaces or require significant interactions in the target task to learn an effective mapping. Our approach doesn't use any explicit mappings and can learn to predict the dynamics of a target task using its descriptions.

% Work by \newcite{konidaris2006autonomous} look at knowledge transfer by learning a mapping from sensory signals to reward functions.

A closely related line of work concerns transfer methods for deep reinforcement learning. \citeA{parisotto2016actor}  train a deep network to mimic pre-trained experts on source tasks using policy distillation. The learned parameters are then used to initialize a network on a target task to perform transfer. Rusu et al.~\citeyear{rusu2016progressive} facilitate transfer by freezing parameters learned on source tasks and adding a new set of parameters for every new target task, while using both sets to learn the new policy. Work by Rajendran et al.~\citeyear{rajendran20172t} uses attention networks to selectively transfer from a set of expert policies to a new task. \textcolor{black}{Barreto et al.~\citeyear{barreto2017successor} use features based on successor representations~\fullcite{dayan1993improving} for transfer across tasks in the same domain. Kansky~et~al.~\citeyear{kansky2017schema} learn a generative model of causal physics in order to help zero-shot transfer learning.} Our approach is orthogonal to all these directions since we use text to bootstrap transfer, and can potentially be combined with these methods to achieve more effective transfer. 

\textcolor{black}{There has also been prior work on zero-shot policy generalization for tasks with input goal specifications. \fullciteA{schaul2015universal} learn a universal value function approximator that can generalize across both states and goals. \fullcite{andreas2016modular} use policy sketches, which are annotated sequences of subgoals, in order to learn a hierarchical policy that can generalize to new goals. \fullciteA{oh2017zero} investigate zero-shot transfer for instruction following tasks, aiming to generalize to unseen instructions in the same domain. The main departure of our work compared to these is in the use of environment descriptions for generalization across domains rather than generalizing across text instructions.}

Perhaps closest in spirit to our hypothesis is the recent work by~\fullcite{harrison2017guiding}. Their approach makes use of paired instances of text descriptions and state-action information from human gameplay to learn a machine translation model. This model is incorporated into a policy shaping algorithm to better guide agent exploration. Although the motivation of language-guided control policies is similar to ours, their work considers transfer across tasks in a single domain, and requires human demonstrations to learn a policy.

\textcolor{black}{
\subsection{Using Task Features for Transfer}
The idea of using task features/dictionaries for zero-shot generalization has been explored previously in the context of image classification. \fullciteA{kodirov2015unsupervised} learn a joint feature embedding space between domains and also induce the corresponding projections onto this space from different class labels. 
\fullciteA{kolouri2018joint} learn a joint dictionary across visual features and class attributes using sparse coding techniques. \fullciteA{romera2015embarrassingly} model the relationship between input features, task attributes and classes as a linear model to achieve efficient yet simple zero-shot transfer for classification. \fullciteA{socher2013zero} learn a joint semantic representation space for images and associated words to perform zero-shot transfer.}

\textcolor{black}{
Task descriptors have also been explored in zero-shot generalization for control policies. \fullciteA{sinapov2015learning} use task meta-data as features to learn a mapping between pairs of tasks. This mapping is then used to select the most relevant source task to transfer a policy from. \fullciteA{isele2016using} build on the ELLA framework~\fullcite{ruvolo2013ella,ammar2014online}, and their key idea is to maintain two shared linear bases across tasks -- one for the policy ($L$) and the other for task descriptors ($D$). Once these bases are learned on a set of source tasks, they can be used to predict policy parameters for a new task given its corresponding descriptor. 
% The training scheme is similar to Actor-mimic scheme~\cite{parisotto2016actor} -- for each task, an expert policy is trained separately and then distilled into policy parameters dependent on the shared basis $L$. 
In these lines of work, the task features were either manually engineered or directly taken from the underlying system parameters defining the dynamics of the environment. In contrast, our framework only requires access to crowd-sourced textual descriptions, alleviating the need for expert domain knowledge.}





% A major difference in our work is that we utilize natural language descriptions of different environments to bootstrap transfer, requiring less exploration in the new task.

% using a policy distillation~\cite{parisotto2016actor,rusu2016progressive,yin2017knowledge} or selective attention over expert networks learnt in the source tasks~\cite{rajendran20172t}. Though these approaches provide some benefits, they still suffer from the requirement of efficiently exploring the new environment to learn how to transfer their existing policies. In contrast, we utilize natural language descriptions of different environments to bootstrap transfer, leading to more focused exploration in the target task. 


% Describe amn in detail






\section{Dataset}
\label{sec:dataset}
%\sarah{add statistics about distribution of merge patterns}
%\alexey{I added some numbers in the section 4 (around line 270). Detailed numbers are in Appendix. We can move it up here if needed...}
%To create a dataset for self-supervised pretraining, we clone all non-fork repositories with more than 20 stars in GitHub that have C, C++, C\#, Python, Java, JavaScript, TypeScript, PHP, Go, and Ruby as their top language. The resulting dataset comprises over 64 million source code files. 
%\chris{why do we list languages here that we don't ever evaluate on?  A reviewer will find this confusing and ask about it.  We found that language specific models work better than multi-lingual models, right?}

The finetuning dataset is mined from over 100,000 open source software repositories in multiple programming languages with merge conflicts. It contains commits from git histories with exactly two parents, which resulted in a merge conflict.  We replay \texttt{git merge} on the two parents to see if it generates any conflicts. Otherwise, we ignore the merge from our dataset. We use the approach introduced by~\citet{Dinella2021} to extract resolution regions---however, we do not restrict ourselves to conflicts with less than 30 lines only.  Lastly, we extract token-level conflicts and conflict resolution classification labels (introduced in Section \ref{formulation}) from line-level conflicts and resolutions. Tab.~\ref{tab:fintuning_dataset} provides a summary of the finetuning dataset.

\begin{table}[htb]
\centering
\caption{Number of merge conflicts in the dataset.}
\begin{tabular}{llllllllllll} \toprule
\textbf{Programming language} & \textbf{Development set}  & \textbf{Test set} \\ \midrule
C\# & 27874 & 6969 \\ 
JavaScript & 66573 & 16644\\ 
TypeScript & 22422 & 5606\\ 
Java & 103065 & 25767 \\ 
\bottomrule
\end{tabular}
\label{tab:fintuning_dataset}
\end{table}
The finetuning dataset is split into development and test sets in the proportion 80/20 at random at the file-level. The development set is further split into training and validation sets in 80/20 proportion at the merge conflict level.    


\section{Framework}
\label{sec:framework}


\graphicspath{{./}{./images_rss_2015/}{./images_hri_2016/}}


%Under this:
% Shared autonomy assumes $\transition(\stateenv' \given \stateenv, \actionuser, \actionrobot) = \transition(\stateenv' \given \stateenv, 0, \actionrobot)$ - that is, the user does not directly affect the state.
% collab doesn't really care about the user state in cost 


We present our framework for minimizing a cost function for shared autonomy with an unknown user goal. We assume the user's goal is fixed, and they take actions to achieve that goal without considering autonomous assistance. These actions are used to predict the user's goal based on how optimal the action is for each goal (\cref{sec:framework_prediction}). Our system uses this distribution to minimize the expected cost-to-go  (\cref{sec:framework_unknown_goal}). As solving for the optimal action is infeasible, we use hindsight optimization to approximate a solution (\cref{sec:framework_hindsight}). For reference, see \cref{table:variable_definitions} in \cref{sec:variable_definitions} for variable definitions.

\subsection{Cost minimization with a known goal}
\label{sec:framework_known_goal}

We first formulate the problem for a known user goal, which we will use in our solution with an unknown goal. We model this problem as a Markov Decision Process (MDP). 

Formally, let $\stateenv \in \Stateenv$ be the environment state (e.g. human and robot pose). Let $\actionuser \in \Actionuser$ be the user actions, and $\actionrobot \in \Actionrobot$ the robot actions. Both agents can affect the environment state - if the user takes action $\actionuser$ and the robot takes action $\actionrobot$ while in state $\stateenv$, the environment stochastically transitions to a new state $\stateenv'$ through $\transitionallargs$. 

We assume the user has an intended goal $\goal \in \Goal$ which does not change during execution. We augment the environment state with this goal, defined by $\state = \left(\stateenv, \goal\right) \in \Stateenv \times \Goal$. We overload our transition function to model the transition in environment state without changing the goal, $\transition( (\stateenv', g) \given (\stateenv, g), \actionuser, \actionrobot) = \transitionallargs$.

%In our scenario, the user wants to move the robot to one goal in a discrete set of goals $\goal \in \Goal$.
We assume access to a user policy for each goal $\policyuser(\actionuser \given \state) = \policyusergoal(\actionuser \given \stateenv) = p(\actionuser \given \stateenv, \goal)$. We model this policy using the maximum entropy inverse optimal control (MaxEnt IOC) framework of~\citet{ziebart_2008}, where the policy corresponds to stochastically optimizing a cost function $\costuser(\state, \actionuser) = \costusergoal(\stateenv, \actionuser)$. We assume the user selects actions based only on $\state$, the current environment state and their intended goal, and does not model any actions that the robot might take. Details are in \cref{sec:framework_prediction}.

The robot selects actions to minimize a cost function dependent on the user goal and action $\costrobot(\state, \actionuser, \actionrobot) = \costrobotgoal(\stateenv, \actionuser, \actionrobot)$. At each time step, we assume the user first selects an action, which the robot observes before selecting $\actionrobot$. The robot selects actions based on the state and user inputs through a policy $\policyrobot(\actionrobot \given \state, \actionuser) = p(\actionrobot \given \state, \actionuser)$. We define the value function for a robot policy $\vrobot^{\policyrobot}$ as the expected cost-to-go from a particular state, assuming some user policy $\policyuser$:
\begin{align*}
  \vrobot^{\policyrobot}(\state) &= \expctarg{\sumtime \costrobot(\state_t, \actionuser_t, \actionrobot_t) \given \state_0 = \state}\\
  \actionuser_t &\sim \policyuser(\cdot \given \state_t)\\
  \actionrobot_t &\sim \policyrobot(\cdot \given \state_t, \actionuser_t)\\
  \state_{t+1} &\sim \transition(\cdot \given \state_t, \actionuser_t, \actionrobot_t)
\end{align*}
%\vrobot^{\policyrobot}(\state) &= \expctover{\policyuser, \policyrobot, \transition}{\sumtime \costrobot(\state_t, \actionuser_t, \actionrobot_t) \given \state_0 = \state}\\

The optimal value function $\vopt$ is the cost-to-go for the best robot policy:
\begin{align*}
  \vopt(\state) &= \min_{\policyrobot} \vrobot^{\policyrobot}(\state)
\end{align*}

The action-value function $\qopt$ computes the immediate cost of taking action $\actionrobot$ after observing $\actionuser$, and following the optimal policy thereafter:
\begin{align*}
  \qopt(\stateactions) &= \costrobot(\stateactions) + \expctarg{\vopt(\state')}
\end{align*}
%\qopt(\stateactions) &= \costrobot(\stateactions) + \expctover{\state'}{\vopt(\state')}
Where $\state' \sim \transition(\cdot \given \stateactions)$. The optimal robot action is given by $\argmin_\actionrobot \qopt(\stateactions)$.

In order to make explicit the dependence on the user goal, we often write these quantities as:
\begin{align*}
  \vgoal(\stateenv) &= \vopt(\state)\\
  \qgoal(\stateenvactions) &= \qopt(\stateactions)
\end{align*}

Computing the optimal policy and corresponding action-value function is a common objective in reinforcement learning. We assume access to this function in our framework, and describe our particular implementation in the experiments.

%The action-value function $\qrobot$ is defined as the immediate cost of taking action $\actionrobot$ after observing $\actionuser$, plus the cost of following $\policyrobot$ thereafter:
%\begin{align*}
%  \qrobot^{\policyrobot}(\stateactions) &= \costrobot(\stateactions) + \expctover{\state'}{\vrobot^{\policyrobot}(\state')}
%\end{align*}
%
%We define the optimal value and action-value functions as the cost-to-go for the best robot policy:
%\begin{align*}
%  \vopt(\state) &= \min_{\policyrobot} \vrobot^{\policyrobot}(\state)\\
%  \qopt(\stateactions) &= \costrobot(\stateactions) + \expctover{\state'}{\vopt(\state)}
%\end{align*}


\subsection{Cost Minimization with an unknown goal}
\label{sec:framework_unknown_goal}

We formulate the problem of minimizing a cost function with an unknown user goal as a Partially Observable Markov Decision Process (POMDP). A POMDP maps a distribution over states, known as the \emph{belief} $\belief$, to actions. We assume that all uncertainty is over the user's goal, and the environment state is known. This subclass of POMDPs, where uncertainty is constant, has been studied as a Hidden Goal MDP~\citep{fern_2010}, and as a POMDP-lite~\citep{chen_2016}.

In this framework, we infer a distribution of the user's goal by observing the user actions $\actionuser$. Similar to the known-goal setting (\cref{sec:framework_known_goal}), we define the value function of a belief as:
\begin{align*}
  \vrobot^{\policyrobot}(\belief) &= \expctarg{\sumtime \costrobot(\state_t, \actionuser_t, \actionrobot_t)  \given \belief_0 = \belief} \\
  \state_t &\sim \belief_t\\
  \actionuser_t &\sim \policyuser(\cdot \given \state_t)\\
  \actionrobot_t &\sim \policyrobot(\cdot \given \state_t, \actionuser_t)\\
  \belief_{t+1} &\sim \transitionbelief(\cdot \given \belief_t, \actionuser_t, \actionrobot_t)
\end{align*}
%\vrobot^{\policyrobot}(\belief) &= \expctover{\belief, \policyuser, \policyrobot, \transition}{\sumtime \costrobot(\state_t, \actionuser_t, \actionrobot_t)  \given \belief_0 = \belief} \\
Where the belief transition $\transitionbelief$ corresponds to transitioning the known environment state $\stateenv$ according to $\transition$, and updating our belief over the user's goal as described in $\cref{sec:framework_prediction}$. We can define quantities similar to above over beliefs:
\begin{align}
  \vopt(\belief) &= \min_{\policyrobot} \vrobot^{\policyrobot}(\belief) \label{eq:v_belief}\\
  \qopt(\beliefactions) &= \expctarg{\costrobot(\belief, \actionuser, \actionrobot) + \expctover{\belief'}{\vopt(\belief')}} \nonumber
\end{align}
%\qopt(\beliefactions) &= \expctover{\belief}{\costrobot(\belief, \actionuser, \actionrobot) + \expctover{\belief'}{\vopt(\belief')}}%\\&= \expctover{\belief}{\costrobot(\belief, \actionuser, \actionrobot)} + \min_\policyrobot \expctover{\belief}{\expctover{\belief'}{\vopt(\belief')}}


%\begin{align*}
%  \qrobot^{\policyrobot}(\beliefactions) &= \expctover{\belief}{\costrobot(\beliefactions) + \expctover{\belief'}{\vrobot^{\policyrobot}(\belief')}}
%\end{align*}




%As the robot does not know the user's goal a priori, we infer this goal from the user actions $\actionuser$ based on our models $\policyuser(\state)$.

%We can model the robots objective of minimizing a cost function with uncertainty using a Partially Observable Markov Decision Process (POMDP) with uncertainty over the user's goal. A POMDP maps a distribution over states, known as the \emph{belief} $\belief$, to actions. We assume that all uncertainty is over the user's goal, and the environment state is known, as in a Hidden Goal MDP~\citep{fern_2010}. Note that allowing the cost to depend on the observation $\actionuser$ is non-standard, but important for shared autonomy, as prior works suggest that users prefer maintaining control authority~\citep{kim_2012}. Our shared autonomy POMDP is defined by the tuple $\left(\Staterobgoal, \Actionrobot, \transition, \costrobot, \Actionuser, \policyuser, \right)$. The optimal solution to this POMDP minimizes the expected accumulated cost $\costrobot$. As this is intractable to compute, we utilize Hindsight Optimization to select actions, described in \cref{sec:framework_hindsight}.





%%We model the robot as a deterministic dynamical system with transition function $\transition: \Stateenv \times \Actionrobot \rightarrow \Stateenv$. %  where applying action $\actionrobot$ in state $\stateenv$ results in state $\stateenv'$.
%%The user supplies continuous inputs $\actionuser \in \Actionuser$ via an interface (e.g. joystick, mouse). These user inputs map to robot actions through a known deterministic function $\userinputtoaction: \Actionuser \rightarrow \Actionrobot$, corresponding to the effect of \emph{direct teleoperation}.
%
%In our scenario, the user wants to move the robot to one goal in a discrete set of goals $\goal \in \Goal$. We assume access to a stochastic user policy for each goal $\policyusergoal(\stateenv) = p(\actionuser | \stateenv, \goal)$, usually learned from user demonstrations. %Here, the user assumes inputs get mapped directly to actions through $\userinputtoaction$ - thus, they assume direct teleoperation.
%In our system, we model this policy using the maximum entropy inverse optimal control (MaxEnt IOC) framework of~\citet{ziebart_2008}, which assumes the user is approximately optimizing some cost function for their intended goal $g$, $\costusergoal: \Stateenv \times \Actionuser \rightarrow \mathcal{R}$. This model corresponds to a goal specific Markov Decision Process (MDP), defined by the tuple $\left(\Stateenv, \Actionuser, \transition, \costusergoal\right)$. We discuss details in \cref{sec:framework_prediction}. 
%
%Unlike the user, our system does not know the intended goal. We model this with a Partially Observable Markov Decision Process (POMDP) with uncertainty over the user's goal. A POMDP maps a distribution over states, known as the \emph{belief} $\belief$, to actions. Define the system state $\staterobgoal \in \Staterobgoal$ as the robot state augmented by a goal, $\staterobgoal = (\stateenv, \goal)$ and $\Staterobgoal = \Stateenv \times \Goal$. In a slight abuse of notation, we overload our transition function such that $\transition: \Staterobgoal \times \Actionrobot \rightarrow \Staterobgoal$, which corresponds to transitioning the robot state as above, but keeping the goal the same.
%
%In our POMDP, we assume the robot state is known, and all uncertainty is over the user's goal. Observations in our POMDP correspond to user inputs $\actionuser \in \Actionuser$. Given a sequence of user inputs, we infer a distribution over system states (equivalently a distribution over goals) using an observation model $\pomdpohm$. This corresponds to computing $\policyusergoal(\stateenv)$ for each goal, and applying Bayes' rule. We provide details in \cref{sec:framework_prediction}.
%
%The system uses cost function $\costrobot: \Staterobgoal \times \Actionrobot \times \Actionuser \rightarrow \mathcal{R}$, corresponding to the cost of taking robot action $\actionrobot$ when in system state $\staterobgoal$ and the user has input $\actionuser$. Note that allowing the cost to depend on the observation $\actionuser$ is non-standard, but important for shared autonomy, as prior works suggest that users prefer maintaining control authority~\citep{kim_2012}. This formulation enables us to penalize robot actions which deviate from $\userinputtoaction(\actionuser)$. Our shared autonomy POMDP is defined by the tuple $\left(\Staterobgoal, \Actionrobot, \transition, \costrobot, \Actionuser, \pomdpohm \right)$. The optimal solution to this POMDP minimizes the expected accumulated cost $\costrobot$. As this is intractable to compute, we utilize Hindsight Optimization to select actions, described in %\cref{sec:hindsight}.





\subsection{Hindsight Optimization}
\label{sec:framework_hindsight}

Computing the optimal solution for a POMDP with continuous states and actions is generally intractable. Instead, we approximate this quantity through \emph{Hindsight Optimization}~\citep{chong_2000,yoon_2008}, or QMDP~\citep{littman_1995}. This approximation estimates the value function by switching the order of the min and expectation in \cref{eq:v_belief}:
\begin{align*}
  \vhs(\belief) &= \expctover{\belief}{\min_{\policyrobot} \vrobot^{\policyrobot}(\state)}\\
  &= \expctover{\goal}{\vgoal(\stateenv)}\\
  \qhs(\beliefactions) &= \expctover{\belief}{\costrobot(\stateactions) + \expctover{\state'}{\vhs(\state')}}\\
  &= \expctover{\goal}{\qgoal(\stateenvactions)}
\end{align*}

Where we explicitly take the expectation over $\goal \in \Goal$, as we assume that is the only uncertain part of the state.

Conceptually, this approximation corresponds to assuming that all uncertainty will be resolved at the next timestep, and computing the optimal cost-to-go. As this is the best case scenario for our uncertainty, this is a lower bound of the cost-to-go, $\vhs(\belief) \leq \vopt(\belief)$. Hindsight optimization has demonstrated effectiveness in other domains~\citep{yoon_2007, yoon_2008}. However, as it assumes uncertainty will be resolved, it never explicitly gathers information~\citep{littman_1995}, and thus performs poorly when this is necessary.

We believe this method is suitable for shared autonomy for many reasons. Conceptually, we assume the user provides inputs at all times, and therefore we gain information without explicit information gathering. Works in other domains with similar properties have shown that this approximation performs comparably to methods that consider explicit information gathering~\citep{koval_2014}. Computationally, computing $\qhs$ can be done with continuous state and action spaces, enabling fast reaction to user inputs. 
%say that this is a lower bound on cost-to-go?

%Let $\qpomdp(\belief, \actionrobot, \actionuser)$ be the action-value function of the POMDP, estimating the cost-to-go of taking action $\actionrobot$ when in belief $\belief$ with user input $\actionuser$, and acting optimally thereafter. In our setting, uncertainty is only over goals, $\belief(\staterobgoal) = \belief(\goal) = p(\goal | \trajtot)$.

%Let $\qgoal(\staterobot, \actionrobot, \actionuser)$ correspond to the action-value for goal $\goal$, estimating the cost-to-go of taking action $\actionrobot$ when in state $\staterobot$ with user input $\actionuser$, and acting optimally for goal $\goal$ thereafter. The QMDP approximation is~\citep{littman_1995}:
%\begin{align*}
%  \qpomdp(\belief, \actionrobot, \actionuser) &= \sum_{\goal} \belief(\goal) \qgoal(\staterobot, \actionrobot, \actionuser)
%\end{align*}

Computing $\qgoal$ for shared autonomy requires utilizing the user policy $\policyusergoal$, which can make computation difficult. This can be alleviated with the following approximations:
\subsubsection*{Stochastic user with robot}
Estimate $\actionuser$ using $\policyusergoal$ at each time step, e.g. by sampling, and utilize the full cost function $\costrobotgoal(\stateenvactions)$ and transition function $\transitionallargs$ to compute $\qgoal$. This would be the standard QMDP approach for our POMDP.

\subsubsection*{Deterministic user with robot}
Estimate $\actionuser$ as the most likely $\actionuser$ from $\policyusergoal$ at each time step, and utilize the full cost function $\costrobotgoal(\stateenvactions)$ and transition function $\transitionallargs$ to compute $\qgoal$. This uses our policy predictor, as above, but does so deterministically, and is thus more computationally efficient.

\subsubsection*{Robot takes over}
Assume the user will stop supplying inputs, and the robot will complete the task. This enables us to use the cost function $\costrobotgoal(\stateenv, 0, \actionrobot)$ and transition function $\transition(\stateenv' \given \stateenv, 0, \actionrobot)$ to compute $\qgoal$. For many cost functions, we can analytically compute this value, e.g. cost of always moving towards the goal at some velocity. An additional benefit of this method is that it makes no assumptions about the user policy $\policyusergoal$, making it more robust to modelling errors. We use this method in our experiments.

Finally, as we often cannot calculate $\argmax_{\actionrobot} \qhs(\beliefactions)$ directly, we use a first-order approximation, which leads to us to following the gradient of $\qhs(\beliefactions)$.
%In cases were an action exists to assist for all goals, this approximation will take that action. When there aren't any such actions, the output will look similar to a blending between the user control and our assistance strategy, solving for the parameters of blending based on the cost functions. This sort of blending has been shown to be effective in the past~\citep{dragan_2013_assistive}. See \figref{fig:teledata}.


%add something about 1st order approximation for continuous systems?

%Maybe more specifics for our system? 
%-First order approx for qmdp
%-we optimize directly for user's value function
%---actually, we aren't fully solving the POMDP assuming user is optimal

\subsection{User Prediction}
\label{sec:framework_prediction}

In order to infer the user's goal, we rely on a model $\policyusergoal$ to provide the distribution of user actions at state $\stateenv$ for user goal $\goal$. In principle, we could use any generative predictor for this model, e.g.~\citep{koppula_2013, wang_2013_intentioninference}. We choose to use maximum entropy inverse optimal control (MaxEnt IOC)~\citep{ziebart_2008}, as it explicitly models a user cost function $\costusergoal$. We optimize this directly by defining $\costrobotgoal$ as a function of $\costusergoal$.

In this work, we assume the user does not model robot actions. We use this assumption to define an MDP with states $\stateenv \in \Stateenv$ and user actions $\actionuser \in \Actionuser$ as before, transition $\transitionuser(\stateenv' \given \stateenv, \actionuser) = \transition(\stateenv' \given \stateenv, \actionuser, 0)$, and cost $\costusergoal(\stateenv, \actionuser)$. MaxEnt IOC computes a stochastically optimal policy for this MDP.

The distribution of actions at a single state are computed based on how optimal that action is for minimizing cost over a horizon $T$. Define a sequence of environment states and user inputs as $\traj = \left\{ \stateenv_0, \actionuser_0, \cdots, \stateenv_T, \actionuser_T \right\}$. Note that sequences are not required to be trajectories, in that $\stateenv_{t+1}$ is not necessarily the result of applying $\actionuser_t$ in state $\stateenv_t$. Define the cost of a sequence as the sum of costs of all state-input pairs, $\costgoaluser(\traj) = \sum_{t} \costgoaluser(\stateenv_t, \actionuser_t)$. Let $\trajtot$ be a sequence from time $0$ to $t$, and $\trajat{\stateenv}$ a sequence of from time $t$ to $T$, starting at $\stateenv$.

\citet{ziebart_thesis} shows that minimizing the worst-case predictive loss results in a model where the probability of a sequence decreases exponentially with cost, $p(\traj \given \goal) \propto \exp(-\costgoaluser(\traj))$. Importantly, one can efficiently learn a cost function consistent with this model from demonstrations~\citep{ziebart_2008}.

Computationally, the difficulty in computing $p(\traj \given \goal)$ lies in the normalizing constant $\int_{\traj} \exp(-\costgoaluser(\traj))$, known as the partition function. Evaluating this explicitly would require enumerating all sequences and calculating their cost. However, as the cost of a sequence is the sum of costs of all state-action pairs, dynamic programming can be utilized to compute this through soft-minimum value iteration when the state is discrete~\citep{ziebart_2009,ziebart_2012}:
\begin{align*}
  \qgoalsoftt{t}(\stateenv, \actionuser) &= \costgoaluser(\stateenv, \actionuser) + \expctarg{\vgoalsoftt{t+1}(\stateenv')}\\
  \vgoalsoftt{t}(\stateenv) &= \softmin_{\actionuser} \qgoalsoftt{t}(\stateenv, \actionuser)
\end{align*}
Where $\softmin_{x} f(x) = - \log \int_{x} \exp(-f(x)) dx$ and $\stateenv' \sim \transitionuser(\cdot \given \stateenv, \actionuser)$.

The log partition function is given by the soft value function, $\vgoalsoftt{t}(\stateenv) = - \log \int_{\trajat{\stateenv}} \exp\left(-\costgoaluser(\trajat{\stateenv})\right)$, where the integral is over all sequences starting at $\stateenv$ and time $t$. Furthermore, the probability of a single input at a given environment state is given by $\policyuser_t(\actionuser \given \stateenv, \goal) = \exp(\vgoalsoftt{t}(\stateenv) -\qgoalsoftt{t}(\stateenv, \actionuser))$~\citep{ziebart_2009}.

%make more clear that while our user policy doesn't consider robot assistance, it still affects this positive feedback thing
Many works derive a simplification that enables them to only look at the start and current states, ignoring the inputs in between~\citep{ziebart_2012, dragan_2013_assistive}. Key to this assumption is that $\traj$ corresponds to a trajectory, where applying action $\actionuser_t$ at $\stateenv_t$ results in $\stateenv_{t+1}$. However, if the system is providing assistance, this may not be the case. In particular, if the assistance strategy believes the user's goal is $\goal$, the assistance strategy will select actions to minimize $\costusergoal$. Applying these simplifications will result positive feedback, where the robot makes itself more confident about goals it already believes are likely. In order to avoid this, we ensure that the prediction comes from user inputs only, and not robot actions:
\begin{align*}
  p(\traj \given \goal) &= \prod_t \policyuser_t(\actionuser_{t} \given \stateenv_t, \goal)
\end{align*}
%Where the user applied input $\actionuser_t$ at state $\state_t$.
To compute the probability of a goal given the partial sequence up to $t$, we apply Bayes' rule:
\begin{align*}
  p(\goal \given \trajtot) &= \frac{p(\trajtot \given \goal) p(\goal) }{\sum_{\goal'} p(\trajtot \given \goal') p(\goal')}
\end{align*}
This corresponds to our POMDP observation model, used to transition our belief over goals through $\transitionbelief$.


\subsubsection{Continuous state and action approximation}
Soft-minimum value iteration is able to find the exact partition function when states and actions are discrete. However, it is computationally intractable to apply in continuous state and action spaces. Instead, we follow \citet{dragan_2013_assistive} and use a second order approximation about the optimal trajectory. They show that, assuming a constant Hessian, we can replace the difficult to compute soft-min functions $\vgoalsoft$ and $\qgoalsoft$ with the min value and action-value functions $\vgoaluser$ and $\qgoaluser$:
\begin{align*}
  \policyuser_t(\actionuser \given \stateenv, \goal) &= \exp(\vgoaluser(\stateenv) -\qgoaluser(\stateenv, \actionuser))
\end{align*}
Recent works have explored extensions of the MaxEnt IOC model for continuous spaces~\citep{boularias_2011, levine_2012, finn_2016}. We leave experiments using these methods for learning and prediction as future work.


\subsection{Multi-Target MDP}
\label{sec:framework_multitarget}

There are often multiple ways to achieve a goal. We refer to each of these ways as a \emph{target}. For a single goal (e.g. object to grasp), let the set of targets (e.g. grasp poses) be $\target \in \Target$. We assume each target has a cost function $\costtarg$, from which we compute the corresponding value and action-value functions $\vtarg$ and $\qtarg$, and soft-value functions $\vtargsoft$ and $\qtargsoft$. We derive the quantities for goals, $\vgoal, \qgoal, \vgoalsoft, \qgoalsoft$, as functions of these target functions.

We state the theorems below, and provide proofs in the appendix (\cref{sec:mingoal_thms}).

\subsubsection{Multi-Target Assistance}
\label{sec:framework_multigarget_assistance}
We assign the cost of a state-action pair to be the cost for the target with the minimum cost-to-go after this state:
\begin{align}
  \costgoal(\stateenvactions) &= \costtargstar(\stateenvactions) \quad \target* = \argmin_\target \vtarg(\stateenv') \label{eq:goal_target_cost}
\end{align}
Where $\stateenv'$ is the environment state after actions $\actionuser$ and $\actionrobot$ are applied at state $\stateenv$. For the following theorem, we require that our user policy be deterministic, which we already assume in our approximations when computing robot actions in \cref{sec:framework_hindsight}.
\begin{restatable}{theorem}{valfundecompose}
\label{thm:mingoal_assist}
Let $\vtarg$ be the value function for target $\target$. Define the cost for the goal as in \cref{eq:goal_target_cost}. For an MDP with deterministic transitions, and a deterministic user policy $\policyuser$, the value and action-value functions $\vgoal$ and $\qgoal$ can be computed as:
\begin{align*}
  \qgoal(\stateenvactions) &= \qtargstar(\stateenvactions) \qquad \target^* = \argmin_\target \vtarg(\stateenv') \\
  \vgoal(\stateenv) &= \min_\target \vtarg(\stateenv)
\end{align*}
\end{restatable}

\subsubsection{Multi-Target Prediction}
\label{sec:framework_multigarget_prediction}
Here, we don't assign the goal cost to be the cost of a single target $\costtarg$, but instead use a distribution over targets.%based on the cost-to-go.
\begin{restatable}{theorem}{softvalfundecompose}
  \label{thm:mingoal_pred}
  Define the probability of a trajectory and target as $p(\traj, \target) \propto \exp(-\costtarg(\traj))$. Let $\vtargsoft$ and $\qtargsoft$ be the soft-value functions for target $\target$. For an MDP with deterministic transitions, the soft value functions for goal $\goal$, $\vgoalsoft$ and $\qgoalsoft$, can be computed as:
\begin{align*}
  \vgoalsoft(\stateenv) &= \softmin_\target \vtargsoft(\stateenv)\\
  \qgoalsoft(\stateenv, \actionuser) &= \softmin_\target \qtargsoft(\stateenv, \actionuser)
\end{align*}
\end{restatable}

%
%Marginalizing out g:
%\begin{align*}
%  p(a_t | s) &= \sum_g p(a_t, g | s)\\
%  &= \frac{ \sum_g \exp(-Q_g^{t}(s_t, a_t))} {\sum_{g'}\exp(-V_{g'}^{t}(s_{t}))}
%\end{align*}
%
%We can also write this out as:
%\begin{align*}
%  \exp\left( \log\left( p(a_t | s) \right) \right)&= \exp\left( \log\left(  \frac{ \sum_g \exp(-Q_g^{t}(s_t, a_t))} {\sum_{g'}\exp(-V_{g'}^{t}(s_{t}))}\right) \right)\\
%  &= \exp\left( \log\left(  \sum_g \exp(-Q_g^{t}(s_t, a_t)) \right) - \log\left(\sum_{g'}\exp(-V_{g'}^{t}(s_{t})) \right) \right)\\
%  &= \exp\left( \softmin_g V_{g}^{t}(s_{t}) - \softmin_g Q_g^{t}(s_t, a_t)\right)
%\end{align*}
%

\begin{figure}[t]
\centering
 \begin{subfigure}{0.24\textwidth}
   \centering 
   \includegraphics[width=0.97\textwidth, trim=440 250 500 210, clip=true]{rss_multigoal_1.png}
  \caption{}
 \label{fig:multigoal_1}
 \end{subfigure}
 \begin{subfigure}{0.24\textwidth}
   \centering 
   \includegraphics[width=0.97\textwidth, trim=440 250 500 210, clip=true]{rss_multigoal_2.png}
  \caption{}
 \label{fig:multigoal_2}
 \end{subfigure}
 \begin{subfigure}{0.24\textwidth}
   \centering 
   \includegraphics[width=0.97\textwidth, trim=440 250 500 210, clip=true]{rss_multigoal_3_arb.png}
  \caption{}
 \label{fig:multigoal_3_arb}
 \end{subfigure}
 \begin{subfigure}{0.24\textwidth}
   \centering 
   \includegraphics[width=0.97\textwidth, trim=440 250 500 210, clip=true]{rss_multigoal_3_pred.png}
  \caption{}
 \label{fig:multigoal_3_pred}
 \end{subfigure}
 \caption{Value function for a goal (grasp the ball) decomposed into value functions of targets (grasp poses). (\subref{fig:multigoal_1}, \subref{fig:multigoal_2}) Two targets and their corresponding value function $\vtarg$. In this example, there are 16 targets for the goal. (\subref{fig:multigoal_3_arb}) The value function of a goal $\vgoal$ used for assistance, corresponding to the minimum of all 16 target value functions (\subref{fig:multigoal_3_pred}) The soft-min value function $\vgoalsoft$ used for prediction, corresponding to the soft-min of all 16 target value functions.}
 \label{fig:multigoal}
\end{figure}





\section{Experimental Evaluation}
\label{sec:experiment}
To demonstrate the viability of our modeling methodology, we show experimentally how through the deliberate combination and configuration of parallel FREEs, full control over 2DOF spacial forces can be achieved by using only the minimum combination of three FREEs.
To this end, we carefully chose the fiber angle $\Gamma$ of each of these actuators to achieve a well-balanced force zonotope (Fig.~\ref{fig:rigDiagram}).
We combined a contracting and counterclockwise twisting FREE with a fiber angle of $\Gamma = 48^\circ$, a contracting and clockwise twisting FREE with $\Gamma = -48^\circ$, and an extending FREE with $\Gamma = -85^\circ$.
All three FREEs were designed with a nominal radius of $R$ = \unit[5]{mm} and a length of $L$ = \unit[100]{mm}.
%
\begin{figure}
    \centering
    \includegraphics[width=0.75\linewidth]{figures/rigDiagram_wlabels10.pdf}
    \caption{In the experimental evaluation, we employed a parallel combination of three FREEs (top) to yield forces along and moments about the $z$-axis of an end effector.
    The FREEs were carefully selected to yield a well-balanced force zonotope (bottom) to gain full control authority over $F^{\hat{z}_e}$ and $M^{\hat{z}_e}$.
    To this end, we used one extending FREE, and two contracting FREEs which generate antagonistic moments about the end effector $z$-axis.}
    \label{fig:rigDiagram}
\end{figure}


\subsection{Experimental Setup}
To measure the forces generated by this actuator combination under a varying state $\vec{x}$ and pressure input $\vec{p}$, we developed a custom built test platform (Fig.~\ref{fig:rig}). 
%
\begin{figure}
    \centering
    \includegraphics[width=0.9\linewidth]{figures/photos/rig_labeled.pdf}
    \caption{\revcomment{1.3}{This experimental platform is used to generate a targeted displacement (extension and twist) of the end effector and to measure the forces and torques created by a parallel combination of three FREEs. A linear actuator and servomotor impose an extension and a twist, respectively, while the net force and moment generated by the FREEs is measured with a force load cell and moment load cell mounted in series.}}
    \label{fig:rig}
\end{figure}
%
In the test platform, a linear actuator (ServoCity HDA 6-50) and a rotational servomotor (Hitec HS-645mg) were used to impose a 2-dimensional displacement on the end effector. 
A force load cell (LoadStar  RAS1-25lb) and a moment load cell (LoadStar RST1-6Nm) measured the end-effector forces $F^{\hat{z_e}}$ and moments $M^{\hat{z_e}}$, respectively.
During the experiments, the pressures inside the FREEs were varied using pneumatic pressure regulators (Enfield TR-010-g10-s). 

The FREE attachment points (measured from the end effector origin) were measured to be:
\begin{align}
    \vec{d}_1 &= \bmx 0.013 & 0 & 0 \emx^T  \text{m}\\
    \vec{d}_2 &= \bmx -0.006 & 0.011 & 0 \emx^T  \text{m}\\
    \vec{d}_3 &= \bmx -0.006 & -0.011 & 0 \emx^T \text{m}
%    \vec{d}_i &= \bmx 0 & 0 & 0 \emx^T , && \text{for } i = 1,2,3
\end{align}
All three FREEs were oriented parallel to the end effector $z$-axis:
\begin{align}
    \hat{a}_i &= \bmx 0 & 0 & 1 \emx^T, \hspace{20pt} \text{for } i = 1,2,3
\end{align}
Based on this geometry, the transformation matrices $\bar{\mathcal{D}}_i$ were given by:
\begin{align}
    \bar{\mathcal{D}}_1 &= \bmx 0 & 0 & 1 & 0 & -0.013 & 0 \\ 0 & 0 & 0 & 0 & 0 & 1 \emx^T  \\
    \bar{\mathcal{D}}_2 &= \bmx 0 & 0 & 1 & 0.011 & 0.006 & 0 \\ 0 & 0 & 0 & 0 & 0 & 1 \emx^T  \\
    \bar{\mathcal{D}}_3 &= \bmx 0 & 0 & 1 & -0.011 & 0.006 & 0 \\ 0 & 0 & 0 & 0 & 0 & 1 \emx^T 
%    \bar{\mathcal{D}}_i &= \bmx 0 & 0 & 1 & 0 & 0 & 0 \\ 0 & 0 & 0 & 0 & 0 & 1 \emx^T , && \text{for } i = 1,2,3
\end{align}
These matrices were used in equation \eqref{eq:zeta} to yield the state-dependent fluid Jacobian $\bar{J}_x$ and to compute the resulting force zontopes.
%while using measured values of $\vec{\zeta}^{\,\text{meas}} (\vec{q}, \vec{P})$ and $\vec{\zeta}^{\,\text{meas}} (\vec{q}, 0)$ in \eqref{eq:fiberIso} yields the empirical measurements of the active force.



\subsection{Isolating the Active Force}
To compare our model force predictions (which focus only on the active forces induced by the fibers)
to those measured empirically on a physical system, we had to remove the elastic force components attributed to the elastomer. 
Under the assumption that the elastomer force is merely a function of the displacement $\vec{x}$ and independent of pressure $\vec{p}$ \cite{bruder2017model}, this force component can be approximated by the measured force at a pressure of $\vec{p}=0$. 
That is: 
\begin{align}
    \vec{f}_{\text{elast}} (\vec{x}) = \vec{f}_{\text{\,meas}} (\vec{x}, 0)
\end{align}
With this, the active generalized forces were measured indirectly by subtracting off the force generated at zero pressure:
\begin{align}
    \vec{f} (\vec{x}, \vec{p})  &= \vec{f}_{\text{meas}} (\vec{x}, \vec{p}) - \vec{f}_{\text{meas}} (\vec{x}, 0)     \label{eq:fiberIso}
\end{align}


%To validate our parallel force model, we compare its force predictions, $\vec{\zeta}_{\text{pred}}$, to those measured empirically on a physical system, $\vec{\zeta}_\text{meas}$. 
%From \eqref{eq:Z} and \eqref{eq:zeta}, the force at the end effector is given by:
%\begin{align}
%    \vec{\zeta}(\vec{q}, \vec{P}) &= \sum_{i=1}^n \bar{\mathcal{D}}_i \left( {\bar{J}_V}_i^T(\vec{q_i}) P_i + \vec{Z}_i^{\text{elast}} (\vec{q_i}) \right) \\
%    &= \underbrace{\sum_{i=1}^n \bar{\mathcal{D}}_i {\bar{J}_V}_i^T(\vec{q_i}) P_i}_{\vec{\zeta}^{\,\text{fiber}} (\vec{q}, \vec{P})} + \underbrace{\sum_{i=1}^n \bar{\mathcal{D}}_i \vec{Z}_i^{\text{elast}} (\vec{q_i})}_{\vec{\zeta}^{\text{elast}} (\vec{q})}   \label{eq:zetaSplit}
%     &= \vec{\zeta}^{\,\text{fiber}} (\vec{q}, \vec{P}) + \vec{\zeta}^{\text{elast}} (\vec{q})
%\end{align}
%\Dan{These will need to reflect changes made to previous section.}
%The model presented in this paper does not specify the elastomer forces, $\vec{\zeta}^{\text{elast}}$, therefore we only validate its predictions %of the fiber forces, $\vec{\zeta}^{\,\text{fiber}}$. 
%We isolate the fiber forces by noting that $\vec{\zeta}^{\text{elast}} (\vec{q}) = \vec{\zeta}(\vec{q}, 0)$ and rearranging \eqref{eq:zetaSplit}
%\begin{align}
%    \vec{\zeta}^{\,\text{fiber}} (\vec{q}, \vec{P})  &= \vec{\zeta}(\vec{q}, \vec{P}) - \vec{\zeta}(\vec{q}, 0)     \label{eq:fiberIso}
%%    \vec{\zeta}^{\,\text{fiber}}_{\text{emp}} (\vec{q}, \vec{P})  &= \vec{\zeta}_{\text{emp}}(\vec{q}, \vec{P}) - %\vec{\zeta}_{\text{emp}}(\vec{q}, 0)
%\end{align}
%Thus we measure the fiber forces indirectly by subtracting off the forces generated at zero pressure.  


\subsection{Experimental Protocol}
The force and moment generated by the parallel combination of FREEs about the end effector $z$-axis  was measured in four different geometric configurations: neutral, extended, twisted, and simultaneously extended and twisted (see Table \ref{table:RMSE} for the exact deformation amounts). 
At each of these configurations, the forces were measured at all pressure combinations in the set
\begin{align}
    \mathcal{P} &= \left\{ \bmx \alpha_1 & \alpha_2 & \alpha_3 \emx^T p^{\text{max}} \, : \, \alpha_i = \left\{ 0, \frac{1}{4}, \frac{1}{2}, \frac{3}{4}, 1 \right\} \right\}
\end{align}
with $p^{\text{max}}$ = \unit[103.4]{kPa}. 
\revcomment{3.2}{The experiment was performed twice using two different sets of FREEs to observe how fabrication variability might affect performance. The results from Trial 1 are displayed in Fig.~\ref{fig:results} and the error for both trials is given in Table \ref{table:RMSE}.}



\subsection{Results}

\begin{figure*}[ht]
\centering

\def\picScale{0.08}    % define variable for scaling all pictures evenly
\def\plotScale{0.2}    % define variable for scaling all plots evenly
\def\colWidth{0.22\linewidth}

\begin{tikzpicture} %[every node/.style={draw=black}]
% \draw[help lines] (0,0) grid (4,2);
\matrix [row sep=0cm, column sep=0cm, style={align=center}] (my matrix) at (0,0) %(2,1)
{
& \node (q1) {(a) $\Delta l = 0, \Delta \phi = 0$}; & \node (q2) {(b) $\Delta l = 5\text{mm}, \Delta \phi = 0$}; & \node (q3) {(c) $\Delta l = 0, \Delta \phi = 20^\circ$}; & \node (q4) {(d) $\Delta l = 5\text{mm}, \Delta \phi = 20^\circ$};

\\

&
\node[style={anchor=center}] {\includegraphics[width=\colWidth]{figures/photos/s0w0pic_colored.pdf}}; %\fill[blue] (0,0) circle (2pt);
&
\node[style={anchor=center}] {\includegraphics[width=\colWidth]{figures/photos/s5w0pic_colored.pdf}}; %\fill[blue] (0,0) circle (2pt);
&
\node[style={anchor=center}] {\includegraphics[width=\colWidth]{figures/photos/s0w20pic_colored.pdf}}; %\fill[blue] (0,0) circle (2pt);
&
\node[style={anchor=center}] {\includegraphics[width=\colWidth]{figures/photos/s5w20pic_colored.pdf}}; %\fill[blue] (0,0) circle (2pt);

\\

\node[rotate=90] (ylabel) {Moment, $M^{\hat{z}_e}$ (N-m)};
&
\node[style={anchor=center}] {\includegraphics[width=\colWidth]{figures/plots3/s0w0.pdf}}; %\fill[blue] (0,0) circle (2pt);
&
\node[style={anchor=center}] {\includegraphics[width=\colWidth]{figures/plots3/s5w0.pdf}}; %\fill[blue] (0,0) circle (2pt);
&
\node[style={anchor=center}] {\includegraphics[width=\colWidth]{figures/plots3/s0w20.pdf}}; %\fill[blue] (0,0) circle (2pt);
&
\node[style={anchor=center}] {\includegraphics[width=\colWidth]{figures/plots3/s5w20.pdf}}; %\fill[blue] (0,0) circle (2pt);

\\

& \node (xlabel1) {Force, $F^{\hat{z}_e}$ (N)}; & \node (xlabel2) {Force, $F^{\hat{z}_e}$ (N)}; & \node (xlabel3) {Force, $F^{\hat{z}_e}$ (N)}; & \node (xlabel4) {Force, $F^{\hat{z}_e}$ (N)};

\\
};
\end{tikzpicture}

\caption{For four different deformed configurations (top row), we compare the predicted and the measured forces for the parallel combination of three FREEs (bottom row). 
\revcomment{2.6}{Data points and predictions corresponding to the same input pressures are connected by a thin line, and the convex hull of the measured data points is outlined in black.}
The Trial 1 data is overlaid on top of the theoretical force zonotopes (grey areas) for each of the four configurations.
Identical colors indicate correspondence between a FREE and its resulting force/torque direction.}
\label{fig:results}
\end{figure*}






% & \node (a) {(a)}; & \node (b) {(b)}; & \node (c) {(c)}; & \node (d) {(d)};


For comparison, the measured forces are superimposed over the force zonotope generated by our model in Fig.~\ref{fig:results}a-~\ref{fig:results}d.
To quantify the accuracy of the model, we defined the error at the $j^{th}$ evaluation point as the difference between the modeled and measured forces
\begin{align}
%    \vec{e}_j &= \left( {\vec{\zeta}_{\,\text{mod}}} - {\vec{\zeta}_{\,\text{emp}}} \right)_j
%    e_j &= \left( F/M_{\,\text{mod}} - F/M_{\,\text{emp}} \right)_j
    e^F_j &= \left( F^{\hat{z}_e}_{\text{pred}, j} - F^{\hat{z}_e}_{\text{meas}, j} \right) \\
    e^M_j &= \left( M^{\hat{z}_e}_{\text{pred}, j} - M^{\hat{z}_e}_{\text{meas}, j} \right)
\end{align}
and evaluated the error across all $N = 125$ trials of a given end effector configuration.
% using the following metrics:
% \begin{align}
%     \text{RMSE} &= \sqrt{ \frac{\sum_{j=1}^{N} e_j^2}{N} } \\
%     \text{Max Error} &= \max \{ \left| e_j \right| : j = 1, ... , N \}
% \end{align}
As shown in Table \ref{table:RMSE}, the root-mean-square error (RMSE) is less than \unit[1.5]{N} (\unit[${8 \times 10^{-3}}$]{Nm}), and the maximum error is less than \unit[3]{N}  (\unit[${19 \times 10^{-3}}$]{Nm}) across all trials and configurations.

\begin{table}[H]
\centering
\caption{Root-mean-square error and maximum error}
\begin{tabular}{| c | c || c | c | c | c|}
    \hline
     & \rule{0pt}{2ex} \textbf{Disp.} & \multicolumn{2}{c |}{\textbf{RMSE}} & \multicolumn{2}{c |}{\textbf{Max Error}} \\ 
     \cline{2-6}
     & \rule{0pt}{2ex} (mm, $^\circ$) & F (N) & M (Nm) & F (N) & M (Nm) \\
     \hline
     \multirow{4}{*}{\rotatebox[origin=c]{90}{\textbf{Trial 1}}}
     & 0, 0 & 1.13 & $3.8 \times 10^{-3}$ & 2.96 & $7.8 \times 10^{-3}$ \\
     & 5, 0 & 0.74 & $3.2 \times 10^{-3}$ & 2.31 & $7.4 \times 10^{-3}$ \\
     & 0, 20 & 1.47 & $6.3 \times 10^{-3}$ & 2.52 & $15.6 \times 10^{-3}$\\
     & 5, 20 & 1.18 & $4.6 \times 10^{-3}$ & 2.85 & $12.4 \times 10^{-3}$ \\  
     \hline
     \multirow{4}{*}{\rotatebox[origin=c]{90}{\textbf{Trial 2}}}
     & 0, 0 & 0.93 & $6.0 \times 10^{-3}$ & 1.90 & $13.3 \times 10^{-3}$ \\
     & 5, 0 & 1.00 & $7.7 \times 10^{-3}$ & 2.97 & $19.0 \times 10^{-3}$ \\
     & 0, 20 & 0.77 & $6.9 \times 10^{-3}$ & 2.89 & $15.7 \times 10^{-3}$\\
     & 5, 20 & 0.95 & $5.3 \times 10^{-3}$ & 2.22 & $13.3 \times 10^{-3}$ \\  
     \hline
\end{tabular}
\label{table:RMSE}
\end{table}

\begin{figure}
    \centering
    \includegraphics[width=\linewidth]{figures/photos/buckling.pdf}
    \caption{At high fluid pressure the FREE with fiber angle of $-85^\circ$ started to buckle.  This effect was less pronounced when the system was extended along the $z$-axis.}
    \label{fig:buckling}
\end{figure}

%Experimental precision was limited by unmodeled material defects in the FREEs, as well as sensor inaccuracy. While the commercial force and moment sensors used have a quoted accuracy of 0.02\% for the force sensor and 0.2\% for the moment sensor (LoadStar Sensors, 2015), a drifting of up to 0.5 N away from zero was noticed on the force sensor during testing.

It should be noted, that throughout the experiments, the FREE with a fiber angle of $-85^\circ$ exhibited noticeable buckling behavior at pressures above $\approx$ \unit[50]{kPa} (Fig.~\ref{fig:buckling}). 
This behavior was more pronounced during testing in the non-extended configurations (Fig.~\ref{fig:results}a~and~\ref{fig:results}c). 
The buckling might explain the noticeable leftward offset of many of the points in Fig.~\ref{fig:results}a and Fig.~\ref{fig:results}c, since it is reasonable to assume that buckling reduces the efficacy of of the FREE to exert force in the direction normal to the force sensor. 

\begin{figure}
    \centering
    \includegraphics[width=\linewidth]{figures/zntp_vs_x4.pdf}
    \caption{A visualization of how the \emph{force zonotope} of the parallel combination of three FREEs (see Fig.~\ref{fig:rig}) changes as a function of the end effector state $x$. One can observe that the change in the zonotope ultimately limits the work-space of such a system.  In particular the zonotope will collapse for compressions of more than \unit[-10]{mm}.  For \revcomment{2.5}{scale and comparison, the convex hulls of the measured points from Fig.~\ref{fig:results}} are superimposed over their corresponding zonotope at the configurations that were evaluated experimentally.}
    % \marginnote{\#2.5}
    \label{fig:zntp_vs_x}
\end{figure}

\section{Conclusion}
In this paper, we introduce Erya for ancient Chinese translation consisting of Erya dataset, model, and benchmark.
Erya dataset is currently the largest ancient Chinese corpora collection including both monolingual corpus and ancient-modern parallel data. 
We further propose a multi-task learning combining DAS and DMLM 
 to train Erya model. 
Finally, we conduct comprehensive evaluation using Erya benchmark. Extensive experiments have validated the superior capability of Erya model under both zero-shot and fine-tuning settings. 
% We will release all the above-mentioned resources to facilitate the research on ancient Chinese.

%we devise a new word alignment method called disyllabic word alignment, with which we propose a multi-task training method combining DAS and DMLM.

\section*{Acknowledgments}
This work was partially supported by National Natural Science Foundation of China under Grant No. 62222215, Beijing Natural Science Foundation under Grant No. 4222027, and Beijing Outstanding Young Scientist Program under Grant No. BJJWZYJH012019100020098. Xin Zhao is the corresponding author. Special thanks to Manman Wang for the advice on ancient Chinese.


\printbibliography

\end{document}
