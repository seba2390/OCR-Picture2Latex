\section{\model: Methodology}
\label{sec:method}
\begin{figure*}[tp]
\hspace{-0.1in}
\begin{minipage}{1\textwidth}
\includegraphics[width=7.1in]{figs/short_frame.pdf}
\end{minipage} 
% \setlength{\abovecaptionskip}{0.2cm}
% \setlength{\belowcaptionskip}{0.2cm}
\vspace{-0.1in}
\caption{(a) Visualized loss landscape comparison; (b) \model~model framework (best view in color); (c) Fourier Serialized gradient estimation in forward and bachward propagation. }
\label{fig:model}
% \vspace{-0.1in}
\end{figure*}


\subsection{Overview}
We formally introduce our \model~ model.
Notice that since the effect of feature dispersion module propagates along with convolutional hashing, we then introduce these modules in the following order:
(1) \textit{latent feature dispersion} (\cref{sec:fd}) aims to disperse the embedded features into wider embedding structures to hedge the inevitable information loss in hashing;
(2) \textit{adaptive graph convolutional hashing} (\cref{sec:hashing}) provides an effective encoding approach to significantly improve the hashed feature expressivity whilst maintaining the matching efficiency in the hamming space;
(3) \textit{Fourier serialized gradient estimation} (\cref{sec:ge}) introduces the Fourier Series decomposition for $\sign(\cdot)$ in the frequency domain to provide more accurate gradient approximation.
Based on the learned hash codes, \model~ develops efficient online matching with the Hamming distance measurement (\cref{sec:score}).
Our model illustration is attached in Figure~\ref{fig:model}(b).



% \begin{figure*}[tp]
% \hspace{-0.3cm}
% \begin{minipage}{1\textwidth}
% \includegraphics[width=7.1in]{figs/short_frame}
% \end{minipage} 
% \setlength{\abovecaptionskip}{0.2cm}
% \setlength{\belowcaptionskip}{0.2cm}
% \caption{Visualized loss landscape comparison and \model~model framework (best view in color).}
% \label{fig:model}
% \end{figure*}

\subsection{Latent Feature Dispersion}
\label{sec:fd}

To tackle the feature erosion issue, we seek to disperse the embedded features as one effective strategy to hedge the inevitable information loss caused by numerical binarization.
From the perspective of singular value decomposition (SVD), singular values and corresponding singular vectors reconstruct the original matrix;
normally, large singular values can be interpreted to associate with \textit{major feature structures} of the matrix~\cite{wei2018grassmann}. 
Since we want to avoid condensing and gathering informative features in (relatively small) embedding sub-structures, it is natural to bridge the target by working on these singular values.
Hence, based on this intuition, we aim to \textit{normalize singular values for equalizing their respective contributions in constituting latent features}.
To achieve this, Power Normalization~\cite{koniusz2016higher,zhang2022spectral} is one of the solutions that tackle related problems such as feature imbalance~\cite{koniusz2018deeper}.
Inspired by the recent approximation attempt~\cite{yu2020toward}, we now introduce a lightweight feature dispersion technique in graph convolution as follows.


Concretely, let $\emb{I}$ denote the identity matrix, we start from generating a \textit{standard normal random vector} $\emb{p}^{(0)}$$\sim$$\mathcal{N}(\emb{0}, \emb{I})$ where $\emb{p}^{(0)}$ $\in$ $\mathbb{R}^{c}$.
Based on the embedding matrix to conduct feature dispersion, e.g., let $\emb{V}= \emb{V}^{(0)}$, we compute the desired \textbf{dispersing vector} $\emb{p}^{(k)}$ by iteratively performing $\emb{p}^{(k)} = \emb{V}^\mathsf{T}\emb{V}\emb{p}^{(k-1)}$.
The iteration for generating dispersing vectors is independent of the graph convolution iterations\footnote{\scriptsize In our work, we set $K \leq L$ mainly to enable the associated complexity of dispersing vector generation is upper bounded by the graph convolution complexity.}.
We have the projection matrix $\emb{P}$ of $\emb{p}^{(K)}$ via:
\begin{sequation}
\label{eq:projection}
\emb{P} = \frac{\emb{p}^{(K)}\emb{p}^{(K)^\mathsf{T}}}{||\emb{p}^{(K)}||_2^2}.
\end{sequation}%
Then we have the feature-dispersed representation matrix with the hyper-parameter $\epsilon$ $\in$ $(0,1)$ as follows:
\begin{sequation}
\label{eq:disperse}
\widetilde{\emb{V}} = \emb{V}(\emb{I} - \epsilon \emb{P}).
\end{sequation}%
Consequently, integrating the dispersed matrix $\widetilde{\emb{V}}$, we have the \textbf{feature-dispersed} graph convolution as:
\begin{sequation}
\label{eq:fdconv}
\widetilde{\emb{V}}^{(l+1)} = (\emb{D}^{-\frac{1}{2}} \emb{A} \emb{D}^{-\frac{1}{2}} )\widetilde{\emb{V}}^{(l)},  \text{ where } \widetilde{\emb{V}}^{(0)} = \emb{V}^{(0)}(\emb{I} - \epsilon \emb{P}).
\end{sequation}%
Note that we explicitly conduct this feature dispersion operation one time only at the initial step, i.e., {\small $\widetilde{\emb{V}}^{(0)}$}, and, more importantly, such feature dispersion can be diffused via the multi-layer graph convolutions from $0$ to $L$.
Compare to the unprocessed embedding counterpart, e.g., {\small${\emb{V}}^{(l)}$}, embedding matrix {\small$\widetilde{\emb{V}}^{(l)}$} at each layer presents a dispersed feature structure with a \textit{more balanced distribution of singular values in expection}. 
We formally explain this as follows: 
\vspace{-0.05in}
\begin{thm}[\textbf{Feature Dispersion}]
\label{tm:svd}
Let ${\emb{V}}^{(l)} = \emb{U}_1\emb{\Sigma}\emb{U}_2^\mathsf{T}$, where $\emb{U}_1$ and $\emb{U}_2$ are unitary matrices and descending singular value matrix $\emb{\Sigma} = \diag(\sigma_1, \sigma_2, \cdots, \sigma_c)$.  
Then $\mathbb{E}({\small\widetilde{\emb{V}}^{(l)}}) = \emb{U}_1\emb{\Sigma}\emb{\Sigma}_{\mu}\emb{U}_2^\mathsf{T}$ where $\emb{\Sigma}_{\mu} = \diag(\mu_1, \mu_2, \cdots, \mu_c)_{0<\mu_{1 \cdots c}<1}$ is in ascending order.
\end{thm}

% \begin{figure}[tp]
% \hspace{-0.5cm}
% \begin{minipage}{0.5\textwidth}
% \includegraphics[width=3.5in]{figs/fd}
% \end{minipage} 
% \setlength{\abovecaptionskip}{0.2cm}
% \setlength{\belowcaptionskip}{0.2cm}
% \caption{An example matrix spectrum on Dianping dataset (singular values are in the descending order).}
% \label{fig:svd}
% \end{figure}
\vspace{-0.05in}
Intuitively, given the same orthonormal bases, compared to {\footnotesize$\emb{V}^{(l)}$}, it is harder in expectation to reconstruct {\footnotesize$\widetilde{\emb{V}}^{(l)}$} with informative features being dispersed out in larger matrix sub-structures.
This eventually provides the functionality to hedge the information loss in numerical binarization. 
% We attach the theorem proof in~\cref{sec:discuss} and evaluate the module effectiveness later in~\cref{sec:ablation}.
We attach the theorem proof in Appendix C and evaluate the module effectiveness later in~\cref{sec:ablation}.



\subsection{Adaptive Graph Convolutional Hashing}
\label{sec:hashing}
One feasible solution for increasing expressivity and smoothing loss landscapes is to include the \textit{relaxation strategy}.
Hence, apart from the topology-aware embedding binarization with $\sign(\cdot)$:
\begin{equation}
\setlength\abovedisplayskip{2pt}
\setlength\belowdisplayskip{2pt}
\label{eq:hashing}
\emb{Q}_{x}^{(l)} = \sign(\widetilde{\emb{V}}_{x}^{(l)}),
\end{equation}
our model \model~additionally computes a layer-wise positive rescaling factor for each node, e.g., $\alpha_x^{(l)} \in \mathbb{R}^+$, such that $\widetilde{\emb{V}}^{(l)}_x \approx$ $\alpha_x^{(l)} \emb{Q}^{(l)}_x$.
In this work, we introduce a simple but effective approach to directly calculate the rescaling factors as follows: 
\begin{equation}
\setlength\abovedisplayskip{2pt}
\setlength\belowdisplayskip{2pt}
\label{eq:rescale}
\alpha_x^{(l)} = \frac{1}{d} ||\widetilde{\emb{{V}}}_x^{(l)}||_1.
\end{equation}
Instead of setting these factors as learnable, such deterministic computation substantially prunes the parameter search space while attaining the adaptive approximation functionality for different graph nodes. 
We demonstrate this in~\cref{sec:ablation} of experiments. 



After $L$ iterations of feature propagation and hashing, we obtain the table of \textbf{adaptive hash codes} $\mathcal{Q} = \{\emb{\alpha}, \emb{Q}\}$, where $\emb{\alpha} \in \mathbb{R}^{(|\mathcal{V}_1|+|\mathcal{V}_2|)\times (L+1)}$ and $\emb{Q} \in \mathbb{R}^{(|\mathcal{V}_1|+|\mathcal{V}_2|)\times d}\}$.
For each node $x$, its corresponding hash codes are indexed and assembled:
\begin{equation}
\setlength\abovedisplayskip{2pt}
\setlength\belowdisplayskip{2pt}
\emb{\alpha}_x = \alpha_x^{(0)} || \alpha_x^{(1)} || \cdots || \alpha_x^{(L)}, \text{  and  } \emb{{Q}}_x = \emb{Q}_x^{(0)} || \emb{Q}_x^{(1)} || \cdots || \emb{Q}_x^{(L)}.
\end{equation}
Intuitively, the hash code table $\mathcal{Q}$ represents the bipartite structural information that is propagated back and forth at different iteration steps $l$, i.e., from $0$ to the maximum step $L$.
It not only tracks the intermediate knowledge hashed for all graph nodes, but also maintains the value approximation to their original continuous embeddings, e.g., {\footnotesize $\widetilde{\emb{{V}}}_x^{(l)}$}.
% In addition, with the slightly more space cost (complexity analysis in~\cref{sec:discuss}), such detached hash encoding approach still supports the bitwise operations (~\cref{sec:score}) for accelerating inference and matching. 
In addition, with the slightly more space cost (complexity analysis in Appendix C, such detached hash encoding approach still supports the bitwise operations (~\cref{sec:score}) for accelerating inference and matching. 





\subsection{Fourier Serialized Gradient Estimation}
\label{sec:ge}

To provide the accordant gradient estimation for hash function $\sign(\cdot)$, we approximate it by introducing its Fourier Series decomposition in the frequency domain. 
Specifically, $\sign(\cdot)$ can be viewed as a special case of the periodical Square Wave Function $t(x)$ within the length $2H$, i.e., $\sign(\phi) = t(\phi)$, $|\phi| < H$.  
Since $t(x)$ can be decomposed in Fourier Series, we shall have: 
\begin{equation}
\setlength\abovedisplayskip{2pt}
\setlength\belowdisplayskip{2pt}
\sign(\phi) = \frac{4}{\pi}\sum_{i=1,3,5,\cdots}^{+\infty}\frac{1}{i}\sin(\frac{\pi i\phi}{H}), {\rm \ \ where \ \ } |\phi| < H.
\end{equation}


% \begin{figure}[tp]
% % \hspace{-0.6cm}
% \begin{minipage}{0.5\textwidth}
% \includegraphics[width=3.3in]{figs/gradient}
% \end{minipage} 
% % \setlength{\abovecaptionskip}{0.2cm}
% % \setlength{\belowcaptionskip}{0.2cm}
% \vspace{-0.1in}
% \caption{Gradient estimation of $\sign(\cdot)$ with Fourier Series.}
% \label{fig:gradient}
% \end{figure}

Fourier Series decomposition of $\sign(\cdot)$ with infinite terms is a lossless transformation~\cite{rust2013convergence}.
Thus, as shown in Figure~\ref{fig:model}(c), we can set the finite expanding term $n$ to obtain its approximation version as follows: 
\begin{equation}
\setlength\abovedisplayskip{2pt}
\setlength\belowdisplayskip{2pt}
{\sign(\phi)} \doteq \frac{4}{\pi}\sum_{i=1,3,5,\cdots}^{n}\frac{1}{i}\sin(\frac{\pi i\phi}{H}).  \\
\end{equation}
The corresponding derivatives can be derived accordingly as:
\begin{equation}
\setlength\abovedisplayskip{2pt}
\setlength\belowdisplayskip{2pt}
\label{eq:gradient}
\frac{\partial{{\sign(\phi)}}}{\partial \phi}   \doteq \frac{4}{H} \sum_{i=1,3,5,\cdots}^{n} \cos(\frac{\pi i\phi}{H}). 
\end{equation}


Different from other gradient estimators such as tanh-alike~\cite{gong2019differentiable,qin2020forward} and SignSwish~\cite{darabi2018bnn}, approximating $\sign(\cdot)$ function with its Fourier Series will not corrupt the main direction of factual gradients in model optimization~\cite{xu2021learning}.
This is beneficial to bridge a coordinated transformation from the continuous values to its corresponding binarization for node representations, which significantly retains the discriminability of binarized representations and produces better retrieval accuracy accordingly.
We present this performance comparison in~\cref{sec:fs_exp} of experiments. 
To summarize, as shown in Equation~(\ref{eq:formal_grad}), to learn and optimize the binarized embeddings for graph nodes, we apply the strict $\sign(\cdot)$ function for forward propagation and estimate the gradients $\frac{\partial\sign(\phi)}{\partial \phi}$ for backward propagation.
\begin{equation}
\setlength\abovedisplayskip{2pt}
\setlength\belowdisplayskip{2pt}
\label{eq:formal_grad}
\left\{ 
\begin{aligned}
& \boldsymbol{Q}^{(l)} = \sign(\phi),  &\text{Forward propagation.} \\
& \frac{\partial \boldsymbol{Q}^{(l)}}{\partial \phi} \doteq \frac{4}{H} \sum_{i=1,3,5,\cdots}^{n} \cos(\frac{\pi i\phi}{H}). & \text{Backward propagation.}
\end{aligned}
\right.
\end{equation}

\subsection{Score Prediction and Model Optimization}
\label{sec:score}


\subsubsection{\textbf{Matching score prediction.}}
\label{sec:score_computation}
Given two nodes $x \in \mathcal{V}_1$ and $y \in \mathcal{V}_2$, one natural manner to implement the score function is \textit{inner-product}, mainly for its simplicity as:
\begin{equation}
\setlength\abovedisplayskip{2pt}
\setlength\belowdisplayskip{2pt}
\label{eq:inner_score}
\widehat{\emb{Y}}_{x,y} =  (\alpha_x\emb{Q}_x)^\mathsf{T} \cdot (\alpha_y\emb{Q}_y).
\end{equation}
However, the inner product in Equation~(\ref{eq:inner_score}) is still conducted in the (continuous) Euclidean space with \textit{full-precision arithmetics}.
To bridge the connection between the inner product and Hamming distance measurement, we introduce Theorem~\ref{tm:equal} as follows:

\begin{thm}[\textbf{Hamming Distance Matching}]
\label{tm:equal}
Given two hash codes, we have $(\alpha_x\emb{Q}_x)^\mathsf{T} \cdot (\alpha_y\emb{Q}_y)$ $=$ $\alpha_x\alpha_y$ $(d - 2D_{H}(\emb{Q}_x, \emb{Q}_y))$.
\end{thm}

$D_H(\cdot, \cdot)$ denotes the Hamming distance between two inputs.
Based on Theorem~\ref{tm:equal}, we transform the score computation to the Hamming distance matching. 
By doing so, we can reduce most number of the floating-point operations (\#FLOPs) in the original score computation formulation (Equation~(\ref{eq:inner_score})) to efficient hamming distance matching.
% This can develop substantial computation acceleration that is analyzed in Appendix~\ref{sec:discuss}.
This can develop substantial computation acceleration that is analyzed in Appendix C.



\subsubsection{\textbf{Multi-loss Objective Function.}}
Our objective function consists of two components, i.e., graph reconstruction loss $\mathcal{L}_{rec}$ and BPR loss $\mathcal{L}_{bpr}$. 
Generally, these two loss functions harness the regularization effect to each other.
The intuition of such design is: 
\begin{itemize}[leftmargin=*]
\item $\mathcal{L}_{rec}$ reconstructs the observed bipartite graph topology;
\item $\mathcal{L}_{bpr}$ ranks the matching scores computed from the hash codes. 
\end{itemize}
Concretely, we implement $\mathcal{L}_{rec}$ with Cross-entropy loss:  
\begin{equation}
\label{eq:rec}
\setlength\abovedisplayskip{2pt}
\setlength\belowdisplayskip{2pt}
\resizebox{1\linewidth}{!}{$
\displaystyle
\mathcal{L}_{rec} = \sum_{x \in \mathcal{V}_1} \Big(\sum_{y\in \mathcal{N}(x)} \ln\sigma\Big(({\emb{V}}^{(0)}_x)^\mathsf{T} \cdot {\emb{V}}^{(0)}_y\Big) + \sum_{y' \notin \mathcal{N}(x)}\ln\Big(1-\sigma\big(({\emb{V}}^{(0)}_x)^\mathsf{T} \cdot {\emb{V}}^{(0)}_{y'}\big)\Big)\Big),
$}
\end{equation}
where $\sigma$ is the activation function, e.g., Sigmoid.
$\mathcal{L}_{rec}$ bases on the initial continuous embeddings before the graph convolution, e.g., {\small${\emb{V}}^{(0)}_x$}, providing the most fundamental information for topology reconstruction.
As for $\mathcal{L}_{bpr}$, we employ \textit{Bayesian Personalized Ranking} (BPR) loss as:
\begin{equation}
\setlength\abovedisplayskip{2pt}
\setlength\belowdisplayskip{2pt}
\label{eq:hd-bpr}
\mathcal{L}_{bpr} = -\sum_{x \in \mathcal{V}_1} \sum_{\tiny y\in \mathcal{N}(x) \atop y'\notin \mathcal{N}(x)} \ln \sigma(\widehat{\emb{Y}}_{x,y} - \widehat{\emb{Y}}_{x,y'}).
\end{equation}
$\mathcal{L}_{bpr}$ encourages the predicted score of an observed edge to be higher than its unobserved counterparts~\cite{lightgcn}.
Let $\Theta$ denote the set of trainable embeddings regularized by the parameter $\lambda_2$ to avoid over-fitting.
our final objective function is finally defined as:
\begin{equation}
\label{eq:L}
\setlength\abovedisplayskip{2pt}
\setlength\belowdisplayskip{2pt}
\mathcal{L} = \mathcal{L}_{rec} + \lambda_1\mathcal{L}_{bpr} + \lambda_2 ||\Theta||_2^2.
\end{equation} 

% So far, we have introduced all technical parts of \model~and attached the pseudocodes in Appendix~\ref{app:notation_and_code}.
% We present all the theorem proofs and complexity analyses in Appendix~\ref{sec:discuss}.

So far, we have introduced all technical parts of \model~and attached the pseudocodes in Appendix B.
We present all the theorem proofs and complexity analyses in Appendix C.





