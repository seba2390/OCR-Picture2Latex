\appendix

\section{Loss Landscape Visualization}
\label{sec:visualization}
We simulate the optimization trajectories of learnable embeddings and visually compare the loss landscapes of non-hashing and hashing versions in Figure~\ref{fig:model}(a).
Specifically, we manually assign perturbations~\cite{nahshan2021loss, bai2020binarybert} to the embeddings on MovieLens dataset as: {\footnotesize$\emb{V}_x^{(l)} = \emb{V}_x^{(l)} \pm p \cdot$ $\overline{|{\emb{V}_x^{(l)}|}}$ $\cdot \emb{1}^{(l)}$}.
where {\footnotesize$\overline{|{\emb{V}_x^{(l)}|}}$} represents the absolute mean of entries in {\footnotesize$\emb{V}_x^{(l)}$} and perturbation magnitudes $p$ are from $\{0.01, \cdots, 0.50\}$. $\emb{1}$ is an all-one vector. 
For pairs of perturbed node embeddings, we plot their loss distribution accordingly.
As we can observe, the non-hashing version produces a flat loss surface, showing the local convexity.
On the contrary, the hashing counterpart has a bumping and complex loss landscape.




\section{Notation Table and \model~Pseudo-codes}
\label{app:notation_and_code}
We use bold uppercase and calligraphy characters for matrices and sets. The non-bolded denote graph nodes or scalars. 
Key notations and Pseudocodes are explained in Table~\ref{tab:notation} and Algorithm~\ref{alg:model}.

\begin{table}[t]
% \setlength{\abovecaptionskip}{0cm}
% \setlength{\belowcaptionskip}{0cm}
\caption {Notations and meanings.}
\vspace{-0.15in}
\label{tab:notation}
  \footnotesize
  \begin{tabular}{c|l} 
     \hline
          {\bf Notation} & {\bf Meaning}\\
     \hline\hline
          {\notsotiny $\mathcal{G},\mathcal{V}_1$, $\mathcal{V}_2$, $\mathcal{E}$} & Bipartite graph with sets of nodes and edges.\\
    \hline
        {$c$, $d$}  & Convolution dimension and hash code dimension.\\
    \hline
        {$\emb{V}_x^{(l)}$}  & {Node $x$'s embedding at iteration $l$.} \\
    \hline
         \tabincell{l}{$\emb{Y}$}   & \tabincell{l}{{Edge transactions where $\emb{Y}_{x,y}=1$ indicates the interaction} \\ existence between nodes $x$ and $y$, and otherwise $\emb{Y}_{x,y}=0$.} \\
    \hline
        {$\emb{A}$}, $\emb{D}$    & {Adjacency matrix and associated diagonal degree matrix.}  \\
    \hline
        $\emb{p}^{(k)}$, $\emb{P}$ & Dispersing vector at iteration $k$ and the projection matrix.\\
    \hline
        $\widetilde{V}$ & Feature-dispersed embedding matrix. \\
    \hline
        {$\widehat{\emb{Y}}$} & {Estimated matching scores.} \\
    \hline
        {$\emb{Q}_x^{(l)}$}   &  {Hash code segment of node $x$ at iteration $l$.} \\
    \hline
        $\alpha^{(l)}$  & $x$'s rescaling factor computed at the $l$-th convolution.\\
    \hline
        {$\emb{Q}_x$}  &   {Target hash codes of node $x$.} \\
    \hline
        {$L$}, {$K$}  &  {Numbers of convolutional hashing and dispersion generation.}\\
    \hline
        $\mathcal{L}_{rec}$, $\mathcal{L}_{bpr}$, $\mathcal{L}$ & {Two loss terms of final objective function $\mathcal{L}$.} \\
    \hline
      {$\eta$, $H$, $n$, $\lambda_1$, $\lambda_2$}  & hyper-parameters.\\
    \hline
  \end{tabular}
\end{table}


\begin{algorithm}[t]
\small
\caption{\model~algorithm.}
\label{alg:model}
\LinesNumbered  
\While{\rm{model not converge}}{
	\For{$k = 0, \cdots, K-1$}{
		$\emb{p}^{(k+1)}$ $\gets$ $(\emb{V}^{(0)})^\mathsf{T}\emb{V}^{(0)}\emb{p}^{(k)}$; \\
	}
	$\emb{P} \gets$ obtain the projection matrix \Comment*[r]{Eq.(\ref{eq:projection})} 
	$\widetilde{\emb{V}}^{(0)} \gets$ obtain the feature-dispersed embeddings \Comment*[r]{Eq.(\ref{eq:disperse})}
    \For{$l = 0, \cdots, L-1$}{
          $\widetilde{\emb{V}}^{(l+1)} \gets(\emb{D}^{-\frac{1}{2}} \emb{A} \emb{D}^{-\frac{1}{2}} )\widetilde{\emb{V}}^{(l)}$ \Comment*[r]{Eq.(\ref{eq:fdconv})}
          $\emb{{Q}}^{(l+1)} \gets \sign\big(\widetilde{\emb{V}}^{(l+1)})$ \Comment*[r]{Eq.(\ref{eq:hashing})}
          $\emb{\alpha} \gets$ calculate the rescaling factors \Comment*[r]{Eq.(\ref{eq:rescale})}
        }
      \For{$x \in \mathcal{V}_1, y \in \mathcal{N}(x)$}{
      $\widehat{\emb{Y}}_{x,y} \gets$ $\alpha_x\alpha_y$ $(d - 2D_{H}(\emb{Q}_x, \emb{Q}_y))$ \Comment*[r]{Eq.(\ref{eq:inner_score})\&Thm.\ref{tm:equal}}
     }
      $\mathcal{L} \gets$ compute loss and optimize the model \Comment*[r]{Eq's.(\ref{eq:rec}-\ref{eq:L})} 
      
}
\textbf{Function} \tt{Gradient\_estimator}($\mathcal{L}$): \\
$\frac{\partial \mathcal{L}}{\partial \emb{V}} \gets \frac{\partial \mathcal{L}}{\partial \emb{Q}} \cdot \frac{4}{H} \sum_{i=1,3,5,\cdots}^{n} \cos(\frac{\pi i \emb{V}}{H})$ \Comment*[r]{Eq.(\ref{eq:gradient})}
\end{algorithm}


% \vspace{-0.20in}
%\subsection{Analysis:}


\textbf{Use of Multiple Projection Heads:} The use of different projection heads for each view on OpenImages classification gives us a boost of $1.1$ mAP on Obj-Obj+Dilate crop. Pre-training on COCO and finetuning on VOC dataset for object-detection task gives a boost of $0.4$ mAP. Hence using multiple projection heads results in a consistent improvement. 

\textbf{Varying Dilation Parameter:} Table 3 (appendix) shows the effect of varying the dilation parameter. A sweet spot exists at a moderate dilation value of $\delta=0.1$ for COCO object detection. 

% \textbf{Computational Cost:} BING adds negligible time to the pre-training. Generating object proposals takes ~29 mins for the full OHMS dataset (one-time cost) and ~16 mins for COCO. Instead of pre-generating, adding the BING operator to the data loader pipeline has a trivial overhead (+$0.1\%$). %As an example, the wall-clock time taken for 1 epoch of training is 1'46'' for the Dense-CL baseline and 1'45'' for our method.
% \textbf{}



%Between two views, we measure the number of common pixels; and then measure the fraction of these common pixels that overlap with a ground truth bounding box (object). We find that this fraction for COCO is $99\%$ for object-scene crops and $92.1\%$ for the scene-scene crop. In the case of OpenImages-Subset, the numbers are, respectively, $99.1\%$ and $87.3\%$. This is another way of seeing that OpenImages-Subset can benefit more from object-scene crops, borne out by the numbers in Tables \ref{tab:ssl_comparison_classification} and \ref{tab:coco_detection}. 


% \as{Shlok: could you please make this description a little better and clear?}
% We find the overlapping pixels between two crops ($C_{int} = C_1 \cap C_2$). Next we calculate intersection of $C_{int}$  with the most overlapping ground truth object ($O$) and calculate the score $\frac{C_{int} \cap O}{C_{int}}$ for each image and average it. 
% To do this, we calculate the \% intersection of the most overlapping ground truth object with the inter
% Next we try to find the probability of an actual ground truth object co-occuring in between two crops. We find  object-overlap between both Scene-Object crops and Scene-Scene crops. To do this we firstly calculate the overlapping region between two crops. Overlapping region is the area of overlap between two crops before the resize operation. Then for all the ground truth objects present in the original image  we find the object with maximum overlap in the overlapping region. Intuitively for a object to have high overlap, the object should be present in both the crops. 

% Similarly instead of taking an crop with maximum overlap we calculate average of all the crops that are present in the image. We find this average probability to be 65.12 \% for Object-Scene crop and 73.47 \% for Scene-Scene crop. 
% This is consistent with the findings of the InfoMin \cite{tian2020contrastive} that there is a tradeoff between how much information views can share.  

% Similarly in the case of OpenImages we can see from Fig \ref{fig:radius_openimages} that as we increase the radius of the object-object crops the performance firstly increases and then decreases, suggesting there is a sweet point on mutual information on OpenImages dataset as well.
% \\

% \textbf{Performance on 5 classes per image images?}











\section{Experiment Setup Details}
\label{app:exp}
{\textbf{Datasets.}}
We evaluate our model on the following six six datasets:
\begin{enumerate}[leftmargin=*]
\item \textbf{MovieLens}\footnote{\url{https://grouplens.org/datasets/movielens/1m/}} is a widely adopted benchmark between \textit{users} and \textit{movies}. Similar to the setting in~\cite{hashgnn,he2016fast}, if the user $x$ has rated item $y$, we set the edge $\emb{Y}_{x,y} = 1$, otherwise $\emb{Y}_{x,y} = 0$. 
\item \textbf{Gowalla}\footnote{\url{https://github.com/gusye1234/LightGCN-PyTorch/tree/master/data/gowalla}}~\cite{ngcf,hashgnn,lightgcn,dgcf} is the dataset~\cite{liang2016modeling} between \textit{customers} and \textit{their check-in locations} collected from Gowalla. 
\item \textbf{Pinterest}\footnote{\url{https://sites.google.com/site/xueatalphabeta/dataset-1/pinterest_iccv}} is an open dataset for image recommendation between \textit{users} and \textit{images}.
Edges represent the pins over images initiated by users. 
\item \textbf{Yelp2018}\footnote{\url{https://github.com/gusye1234/LightGCN-PyTorch/tree/master/data/yelp2018}} is from Yelp Challenge 2018 Edition, bipartitely modeling between \textit{users} and \textit{local businesses}.
\item \textbf{AMZ-Book}\footnote{\url{https://github.com/gusye1234/LightGCN-PyTorch/tree/master/data/amazon-book}} is the bipartite graph between \textit{readers} and \textit{books}, organized from the book collection of Amazon-review~\cite{he2016ups}.  
\item \textbf{Dianping}\footnote{\url{https://www.dianping.com/}} is a commercial dataset between \textit{users} and \textit{local businesses} recording their diverse interactions, e.g., clicking, saving, and purchasing. 
\end{enumerate}
% \vspace{0.05in}

\textbf{Evaluation metrics.}
To evaluate the model performance of Hamming space retrieval over bipartite graphs, we directly deploy our model \model~ in the basic user-item recommendation scenarios.
Specifically, given a query node, we apply the hash codes to match Top-N answers for the query with the closest Hamming distances, and thus adopt two widely-used evaluation protocols Recall@N and NDCG@N to measure the ranking capability.

\vspace{0.05in}

{\textbf{Implementations.}}
We implement our models under Python 3.6 and PyTorch 1.14.0 on a Linux machine with 1 Nvidia GeForce RTX 3090 GPU, 4 Intel Core i7-8700 CPUs, 32 GB of RAM with 3.20GHz.
For all the baselines, we follow the official hyper-parameter settings.
We apply a grid search if lacking recommended model settings.
The dimension is searched in \{$32, 64, 128, 256, 512$\}. 
The learning rate $\eta$ is tuned within \{$10^{-3}, 5\times10^{-3}, 10^{-2}, 5\times10^{-2}$\} and the coefficient $\lambda$ is tuned among \{$10^{-5}, 10^{-4}, 10^{-3}$\}. 
We initialize and optimize all models with default normal initializer and Adam optimizer~\cite{adam}. 

\vspace{0.05in}

\textbf{Baselines.} All baselines studied in this paper are introduced as:
\label{app:baselines}
\begin{enumerate}[leftmargin=*]
\item \textbf{LSH}~\cite{lsh} is a classical hashing method. LSH is proposed to approximate the similarity search for massive high-dimensional data and we introduce it for Top-N object search by following the adaptation in~\cite{hashgnn}. 

\item \textbf{HashNet}~\cite{hashnet} is a representative deep hashing method that is originally proposed for multimedia retrieval tasks.
Similar to~\cite{hashgnn}, we adapt it for graph data by modifying it with the general graph convolutional network.

\item \textbf{CIGAR}~\cite{kang2019candidate} is a state-of-the-art neural-network-based framework for fast Top-N candidate generation in recommendation. 
CIGAR can be further followed by a full-precision re-ranking algorithm. And we only use its hashing part for fair comparison.

\item \textbf{Hash\_Gumbel} is a variance of \model~with Gumbel-softmax for hash encoding and gradient estimation~\cite{gumbel1,gumbel2}.
Specifically, we first expand each embedding bit to a size-two one-hot encoding. 
Then it utilizes the Gumbel-softmax trick to replace $\sign(\cdot)$ as relaxation for binary hash code generation. 

\item \textbf{HashGNN}~\cite{hashgnn} is the state-of-the-art learning to hash based method with GCN framework. 
We use HashGNN$_{h}$ to denote the vanilla version with \textit{hard encoding} proposed in~\cite{hashgnn}, where each element of user-item embeddings is strictly binarized. 
We use HashGNN$_{s}$ to denote its proposed approximated version.

\item \textbf{NeurCF}~\cite{neurcf} is one representative deep neural network model for collaborative filtering in recommendation. 
% NeurCF models latent features to capture nonlinear feature interactions between users and items.

\item \textbf{NGCF}~\cite{ngcf} is one of the representative graph-based recommender models with collaborative filtering methodology. 

\item \textbf{DGCF}~\cite{dgcf} is a state-of-the-art graph-based model that learns disentangled user intents for better recommendation. 

\item \textbf{LightGCN}~\cite{lightgcn} is another latest state-of-the-art GCN-based recommender model that has been widely evaluated. 

\end{enumerate}
