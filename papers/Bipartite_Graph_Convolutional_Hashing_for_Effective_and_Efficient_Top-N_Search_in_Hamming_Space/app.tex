\appendix

\section{Loss Landscape Visualization}
\label{sec:visualization}
We simulate the optimization trajectories of learnable embeddings and visually compare the loss landscapes of non-hashing and hashing versions in Figure~\ref{fig:model}(a).
Specifically, we manually assign perturbations~\cite{nahshan2021loss, bai2020binarybert} to the embeddings on MovieLens dataset as: {\footnotesize$\emb{V}_x^{(l)} = \emb{V}_x^{(l)} \pm p \cdot$ $\overline{|{\emb{V}_x^{(l)}|}}$ $\cdot \emb{1}^{(l)}$}.
where {\footnotesize$\overline{|{\emb{V}_x^{(l)}|}}$} represents the absolute mean of entries in {\footnotesize$\emb{V}_x^{(l)}$} and perturbation magnitudes $p$ are from $\{0.01, \cdots, 0.50\}$. $\emb{1}$ is an all-one vector. 
For pairs of perturbed node embeddings, we plot their loss distribution accordingly.
As we can observe, the non-hashing version produces a flat loss surface, showing the local convexity.
On the contrary, the hashing counterpart has a bumping and complex loss landscape.




\section{Notation Table and \model~Pseudo-codes}
\label{app:notation_and_code}
We use bold uppercase and calligraphy characters for matrices and sets. The non-bolded denote graph nodes or scalars. 
Key notations and Pseudocodes are explained in Table~\ref{tab:notation} and Algorithm~\ref{alg:model}.

\begin{table}[t]
% \setlength{\abovecaptionskip}{0cm}
% \setlength{\belowcaptionskip}{0cm}
\caption {Notations and meanings.}
\vspace{-0.15in}
\label{tab:notation}
  \footnotesize
  \begin{tabular}{c|l} 
     \hline
          {\bf Notation} & {\bf Meaning}\\
     \hline\hline
          {\notsotiny $\mathcal{G},\mathcal{V}_1$, $\mathcal{V}_2$, $\mathcal{E}$} & Bipartite graph with sets of nodes and edges.\\
    \hline
        {$c$, $d$}  & Convolution dimension and hash code dimension.\\
    \hline
        {$\emb{V}_x^{(l)}$}  & {Node $x$'s embedding at iteration $l$.} \\
    \hline
         \tabincell{l}{$\emb{Y}$}   & \tabincell{l}{{Edge transactions where $\emb{Y}_{x,y}=1$ indicates the interaction} \\ existence between nodes $x$ and $y$, and otherwise $\emb{Y}_{x,y}=0$.} \\
    \hline
        {$\emb{A}$}, $\emb{D}$    & {Adjacency matrix and associated diagonal degree matrix.}  \\
    \hline
        $\emb{p}^{(k)}$, $\emb{P}$ & Dispersing vector at iteration $k$ and the projection matrix.\\
    \hline
        $\widetilde{V}$ & Feature-dispersed embedding matrix. \\
    \hline
        {$\widehat{\emb{Y}}$} & {Estimated matching scores.} \\
    \hline
        {$\emb{Q}_x^{(l)}$}   &  {Hash code segment of node $x$ at iteration $l$.} \\
    \hline
        $\alpha^{(l)}$  & $x$'s rescaling factor computed at the $l$-th convolution.\\
    \hline
        {$\emb{Q}_x$}  &   {Target hash codes of node $x$.} \\
    \hline
        {$L$}, {$K$}  &  {Numbers of convolutional hashing and dispersion generation.}\\
    \hline
        $\mathcal{L}_{rec}$, $\mathcal{L}_{bpr}$, $\mathcal{L}$ & {Two loss terms of final objective function $\mathcal{L}$.} \\
    \hline
      {$\eta$, $H$, $n$, $\lambda_1$, $\lambda_2$}  & hyper-parameters.\\
    \hline
  \end{tabular}
\end{table}


\begin{algorithm}[t]
\small
\caption{\model~algorithm.}
\label{alg:model}
\LinesNumbered  
\While{\rm{model not converge}}{
	\For{$k = 0, \cdots, K-1$}{
		$\emb{p}^{(k+1)}$ $\gets$ $(\emb{V}^{(0)})^\mathsf{T}\emb{V}^{(0)}\emb{p}^{(k)}$; \\
	}
	$\emb{P} \gets$ obtain the projection matrix \Comment*[r]{Eq.(\ref{eq:projection})} 
	$\widetilde{\emb{V}}^{(0)} \gets$ obtain the feature-dispersed embeddings \Comment*[r]{Eq.(\ref{eq:disperse})}
    \For{$l = 0, \cdots, L-1$}{
          $\widetilde{\emb{V}}^{(l+1)} \gets(\emb{D}^{-\frac{1}{2}} \emb{A} \emb{D}^{-\frac{1}{2}} )\widetilde{\emb{V}}^{(l)}$ \Comment*[r]{Eq.(\ref{eq:fdconv})}
          $\emb{{Q}}^{(l+1)} \gets \sign\big(\widetilde{\emb{V}}^{(l+1)})$ \Comment*[r]{Eq.(\ref{eq:hashing})}
          $\emb{\alpha} \gets$ calculate the rescaling factors \Comment*[r]{Eq.(\ref{eq:rescale})}
        }
      \For{$x \in \mathcal{V}_1, y \in \mathcal{N}(x)$}{
      $\widehat{\emb{Y}}_{x,y} \gets$ $\alpha_x\alpha_y$ $(d - 2D_{H}(\emb{Q}_x, \emb{Q}_y))$ \Comment*[r]{Eq.(\ref{eq:inner_score})\&Thm.\ref{tm:equal}}
     }
      $\mathcal{L} \gets$ compute loss and optimize the model \Comment*[r]{Eq's.(\ref{eq:rec}-\ref{eq:L})} 
      
}
\textbf{Function} \tt{Gradient\_estimator}($\mathcal{L}$): \\
$\frac{\partial \mathcal{L}}{\partial \emb{V}} \gets \frac{\partial \mathcal{L}}{\partial \emb{Q}} \cdot \frac{4}{H} \sum_{i=1,3,5,\cdots}^{n} \cos(\frac{\pi i \emb{V}}{H})$ \Comment*[r]{Eq.(\ref{eq:gradient})}
\end{algorithm}



\section{Case Studies}
\label{sec:case_studies}
In this section, we present a case study of Facebook posts from an Australian public page.
The page shifts between early 2020 (\emph{2019-2020 Australian bushfire season}) and late 2020 (\emph{COVID-19 crises}) from being a moderate-right group for discussion around climate change to a far-right extremist group for conspiracy theories.


\begin{figure*}[!tbp]
	\begin{subfigure}{0.21\textwidth}
		\includegraphics[width=\textwidth]{images/facebook1.png}
		\caption{}
		\label{subfig:first-posting}
		\includegraphics[width=0.9\textwidth]{images/facebook3.jpg}
		\caption{}
		\label{subfig:comment-post-1}
	\end{subfigure}
    \begin{subfigure}{0.28\textwidth}
		\includegraphics[width=\textwidth]{images/facebook2.jpg}
		\caption{}
		\label{subfig:second-posting}
	\end{subfigure}
    \begin{subfigure}{0.23\textwidth}
		\includegraphics[width=\textwidth]{images/facebook4.jpg}
		\caption{}
		\label{subfig:comment-post-2a}
	\end{subfigure}
    \begin{subfigure}{0.23\textwidth}
		\includegraphics[width=\textwidth]{images/facebook5.jpg}
		\caption{}
		\label{subfig:comment-post-2b}
	\end{subfigure}
	\caption{
		Examples of postings and comment threads from a public Facebook page from two periods of time early 2020 (a) and late 2020 (b)-(e), which show a shift from climate change debates to extremist and far-right messaging.
	}
	\label{fig:facebook}
\end{figure*}

We focus on a sample of 2 postings and commenting threads from one Australian Facebook page we classified as ``far-right'' based on the content on the page. 
We have anonymized the users in \Cref{fig:facebook} to avoid re-identification.
The first posting and comment thread (see \Cref{subfig:first-posting}) was collected on Jan 10, 2020, and responded to the Australian bushfire crisis that began in late 2019 and was still ongoing in January 2020. It contains an ambivalent text-based provocation that references disputes in the community regarding the validity of climate change and climate science. 

The second posting and comment thread (see \Cref{subfig:second-posting}) was collected from the same page in September 2020, months after the bushfire crisis had abated.
At that time, a new crisis was energizing and connecting the far-right groups in our dataset --- i.e., the COVID-19 pandemic and the government interventions to curb the spread of the virus. 
The post is different in style compared to the first.
It is image-based instead of text-based and highly emotive, with a photo collage bringing together images of prison inmates with iron masks on their faces (top row) juxtaposed to people wearing face masks during COVID-19 (bottom row). 
The image references the public health orders issued during Melbourne's second lockdown and suggests that being ordered to wear masks is an infringement of citizen rights and freedoms, similar to dehumanizing restraints used on prisoners.

To analyze reactions to the posts, two researchers used a deductive analytical approach to separately code and to analyze the commenting threads --- see \Cref{subfig:comment-post-1} for comments of the first posting, and \Cref{subfig:comment-post-2a,subfig:comment-post-2b} for comments on the second posting. 
Conversations were also inductively coded for emerging themes. 
During the analysis, we observed qualitative differences in the types of content users posted, interactions between commenters, tone and language of debate, linked media shared in the commenting thread, and the opinions expressed.
The rest of this section further details these differences.
To ensure this was not a random occurrence, we tested the exemplar threads against field notes collected on the group during the entire study.
We also used Facebook's search function within pages to find a sample of posts from the same period and which dealt with similar topics. 
After this analysis, we can confidently say that key changes occurred in the group between the bushfire crisis and COVID-19, that we detail next.

\subsubsection*{Exemplar 1 --- climate change skepticism.}
To explore this transformation in more depth, we analyzed comments scraped on the first posting --- \cref{sub@subfig:comment-post-1} shows a small sample of these comments.
The language used was similar to comments that we observed on numerous far-right nationalist pages at the time of the bushfires.
These comments are usually text-based, employing emojis to denote emotions, and sometimes being mocking or provocative in tone. 
Noteworthy for this commenting thread is the 50/50 split in the number of members posting in favor of action on climate change (on one side) and those who posted anti-Greens and anti-climate change science posts and memes (on the other side).
The two sides aligned strongly with political partisanship --- either with Liberal/National coalition (climate change deniers) or Labor/Green (climate change believers) parties. 
This is rather unusual for pages classified as far-right. 

We observed trolling practices between the climate change deniers and believers, which often descend into \emph{flame wars} --- i.e., online ``firefights that take place between disembodied combatants on electronic bulletin boards''~\citep{bukatman1994flame}.
The result is a boosted engagement on the post but also the frustration and confusion of community members and lurkers who came to the discussions to become informed or debate rationally on key differences between the two positions.
They often even become targeted, victimized, and baited by trolls on both sides of the partisan divide. 
The opinions expressed by deniers in commenting sections range from skepticism regarding climate change science to plain denial.
Deniers also regard a range of targets as embroiled in a climate change conspiracy to deceive the public, such as The Greens and their environmental policy, in some cases the government, the United Nations, and climate change celebrities like David Attenborough and Greta Thunberg. 
These figures are blamed for either exaggerating risks of climate change or creating a climate change hoax to increase the influence of the UN on domestic governments or to increase domestic governments' social control over citizens. 

Both coders noted that flame wars between these opposing personas contained very few links to external media. 
Where links were added, they often seemed disconnected from the rest of the conversation and were from users whose profiles suggested they believed in more radical conspiracy theories.
One such example is ``geo-engineering'' (see \cref{sub@subfig:comment-post-1}).
Its adherents believe that solar geo-engineering programs designed to combat climate change are secretly used by a global elite to depopulate the world through sterilization or to control and weaponize the weather.

Nonetheless, apart from the random comments that hijack the thread, redirecting users to external ``alternative'' news sites and Twitter, and the trolls who seem to delight in victimizing unsuspecting victims, the discussion was pretty healthy.
There are many questions, rational inquiries, and debates between users of different political persuasion and views on climate change.
This, however, changes in the span of only a couple of months.

\subsubsection*{Exemplar 2 --- posting and commenting thread.}
We observe a shift in the comment section of the post collected during the second wave of the COVID pandemic (\Cref{sub@subfig:second-posting}) --- which coincided with government laws mandating the public to wear masks and stay at home in Victoria, Australia.
There emerges much more extreme far-right content that converges with anti-vaccination opinions and content.
We also note a much higher prevalence of conspiracy theories often implicating racialized targets.
This is exemplified in the comments on the second post (\Cref{sub@subfig:comment-post-2a,sub@subfig:comment-post-2b}) where Islamophobia and antisemitism are confidently asserted alongside anti-mask rhetoric.
These comments consider face masks similar to the religious head coverings worn by some Muslim women, which users describe as ``oppressive'' and ``silencing''. 
In this way, anti-maskers cast women as a distinct, sympathetic marginalized demographic.
However, this is enacted alongside the racialization and demonization of Islam as an oppressive religion. 

Given the extreme racialization of anti-mask rhetoric, some commenters contest these positions, arguing that the page is becoming less an anti-Scott Morrison page (Australia's Prime Minister at the time) and changing into a page that harbors ``far-right dickheads''.
This questioning is actively challenged by far-right commenters and conspiracy theorists on the page, who regarded pro-mask users and the Scott Morrison government as ``puppets'' being manipulated by higher forces (see \Cref{sub@subfig:comment-post-2b}). 

This indicates a significant change on the page's membership towards the extreme-right, who employs more extreme forms of racialized imagery, with more extreme opinion being shared.
Conspiracy theorists become more active and vocal, and they consistently challenge the opinions of both center conservative and left-leaning users. 
This is evident in the final two comments in \Cref{subfig:comment-post-2b}, which reflect QAnon style conspiracy theories and language.
Public health orders to wear masks are being connected to a conspiracy that all of these decisions are directed by a secret network of global Jewish elites, who manipulate the pandemic to increase their power and control. 
This rhetoric intersects with the contemporary ``QAnon'' conspiracy theory, which evolved from the ``Pizzagate'' conspiracy theory.
They also heavily draw on well-established antisemitic blood libel conspiracy theories, which foster beliefs that a powerful global elite is controlling the decisions of organizations such as WHO and are responsible for the vaccine rollout and public health orders related to the pandemic.
The QAnon conspiracy is also influenced by Bill Gates' Microchips conspiracy theory, i.e., the theory that the WHO and the Bill Gates Foundation global vaccine programs are used to inject tracking microchips into people.

These conspiracy theories have, since COVID-19, connected formerly separate communities and discourses, uniting existing anti-vaxxer communities, older demographics who are mistrustful of technology, far-right communities suspicious of global and national left-wing agendas, communities protesting against 5G mobile networks (for fear that they will brainwash, control, or harm people), as well as generating its own followers out of those anxious during the 2020 onset of the COVID-19 pandemic.
We detect and describe some of these opinion dynamics in the next section.



\section{Experiment Setup Details}
\label{app:exp}
{\textbf{Datasets.}}
We evaluate our model on the following six six datasets:
\begin{enumerate}[leftmargin=*]
\item \textbf{MovieLens}\footnote{\url{https://grouplens.org/datasets/movielens/1m/}} is a widely adopted benchmark between \textit{users} and \textit{movies}. Similar to the setting in~\cite{hashgnn,he2016fast}, if the user $x$ has rated item $y$, we set the edge $\emb{Y}_{x,y} = 1$, otherwise $\emb{Y}_{x,y} = 0$. 
\item \textbf{Gowalla}\footnote{\url{https://github.com/gusye1234/LightGCN-PyTorch/tree/master/data/gowalla}}~\cite{ngcf,hashgnn,lightgcn,dgcf} is the dataset~\cite{liang2016modeling} between \textit{customers} and \textit{their check-in locations} collected from Gowalla. 
\item \textbf{Pinterest}\footnote{\url{https://sites.google.com/site/xueatalphabeta/dataset-1/pinterest_iccv}} is an open dataset for image recommendation between \textit{users} and \textit{images}.
Edges represent the pins over images initiated by users. 
\item \textbf{Yelp2018}\footnote{\url{https://github.com/gusye1234/LightGCN-PyTorch/tree/master/data/yelp2018}} is from Yelp Challenge 2018 Edition, bipartitely modeling between \textit{users} and \textit{local businesses}.
\item \textbf{AMZ-Book}\footnote{\url{https://github.com/gusye1234/LightGCN-PyTorch/tree/master/data/amazon-book}} is the bipartite graph between \textit{readers} and \textit{books}, organized from the book collection of Amazon-review~\cite{he2016ups}.  
\item \textbf{Dianping}\footnote{\url{https://www.dianping.com/}} is a commercial dataset between \textit{users} and \textit{local businesses} recording their diverse interactions, e.g., clicking, saving, and purchasing. 
\end{enumerate}
% \vspace{0.05in}

\textbf{Evaluation metrics.}
To evaluate the model performance of Hamming space retrieval over bipartite graphs, we directly deploy our model \model~ in the basic user-item recommendation scenarios.
Specifically, given a query node, we apply the hash codes to match Top-N answers for the query with the closest Hamming distances, and thus adopt two widely-used evaluation protocols Recall@N and NDCG@N to measure the ranking capability.

\vspace{0.05in}

{\textbf{Implementations.}}
We implement our models under Python 3.6 and PyTorch 1.14.0 on a Linux machine with 1 Nvidia GeForce RTX 3090 GPU, 4 Intel Core i7-8700 CPUs, 32 GB of RAM with 3.20GHz.
For all the baselines, we follow the official hyper-parameter settings.
We apply a grid search if lacking recommended model settings.
The dimension is searched in \{$32, 64, 128, 256, 512$\}. 
The learning rate $\eta$ is tuned within \{$10^{-3}, 5\times10^{-3}, 10^{-2}, 5\times10^{-2}$\} and the coefficient $\lambda$ is tuned among \{$10^{-5}, 10^{-4}, 10^{-3}$\}. 
We initialize and optimize all models with default normal initializer and Adam optimizer~\cite{adam}. 

\vspace{0.05in}

\textbf{Baselines.} All baselines studied in this paper are introduced as:
\label{app:baselines}
\begin{enumerate}[leftmargin=*]
\item \textbf{LSH}~\cite{lsh} is a classical hashing method. LSH is proposed to approximate the similarity search for massive high-dimensional data and we introduce it for Top-N object search by following the adaptation in~\cite{hashgnn}. 

\item \textbf{HashNet}~\cite{hashnet} is a representative deep hashing method that is originally proposed for multimedia retrieval tasks.
Similar to~\cite{hashgnn}, we adapt it for graph data by modifying it with the general graph convolutional network.

\item \textbf{CIGAR}~\cite{kang2019candidate} is a state-of-the-art neural-network-based framework for fast Top-N candidate generation in recommendation. 
CIGAR can be further followed by a full-precision re-ranking algorithm. And we only use its hashing part for fair comparison.

\item \textbf{Hash\_Gumbel} is a variance of \model~with Gumbel-softmax for hash encoding and gradient estimation~\cite{gumbel1,gumbel2}.
Specifically, we first expand each embedding bit to a size-two one-hot encoding. 
Then it utilizes the Gumbel-softmax trick to replace $\sign(\cdot)$ as relaxation for binary hash code generation. 

\item \textbf{HashGNN}~\cite{hashgnn} is the state-of-the-art learning to hash based method with GCN framework. 
We use HashGNN$_{h}$ to denote the vanilla version with \textit{hard encoding} proposed in~\cite{hashgnn}, where each element of user-item embeddings is strictly binarized. 
We use HashGNN$_{s}$ to denote its proposed approximated version.

\item \textbf{NeurCF}~\cite{neurcf} is one representative deep neural network model for collaborative filtering in recommendation. 
% NeurCF models latent features to capture nonlinear feature interactions between users and items.

\item \textbf{NGCF}~\cite{ngcf} is one of the representative graph-based recommender models with collaborative filtering methodology. 

\item \textbf{DGCF}~\cite{dgcf} is a state-of-the-art graph-based model that learns disentangled user intents for better recommendation. 

\item \textbf{LightGCN}~\cite{lightgcn} is another latest state-of-the-art GCN-based recommender model that has been widely evaluated. 

\end{enumerate}
