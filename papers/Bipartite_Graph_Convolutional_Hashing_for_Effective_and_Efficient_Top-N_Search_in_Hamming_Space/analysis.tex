\section{Theoretical Proofs and Analyses}
\label{sec:discuss}
\addtocounter{thm}{-2}

\begin{thm}[\textbf{Feature Dispersion}]
Let ${\emb{V}}^{(l)} = \emb{U}_1\emb{\Sigma}\emb{U}_2^\mathsf{T}$, where $\emb{U}_1$ and $\emb{U}_2$ are unitary matrices and descending singular value matrix $\emb{\Sigma} = \diag(\sigma_1, \sigma_2, \cdots, \sigma_c)$.  
Then $\mathbb{E}({\small\widetilde{\emb{V}}^{(l)}}) = \emb{U}_1\emb{\Sigma}\emb{\Sigma}_{\mu}\emb{U}_2^\mathsf{T}$ where $\emb{\Sigma}_{\mu} = \diag(\mu_1, \mu_2, \cdots, \mu_c)_{0<\mu_{1 \cdots c}<1}$ is in ascending order.
\end{thm}


\begin{proof}
We focus on comparing ({\small$\widetilde{\emb{V}}^{(0)}$, ${\emb{V}}^{(0)}$}), which can be easily popularized to any convolution layer, i.e., ({\small$\widetilde{\emb{V}}^{(l)}$, ${\emb{V}}^{(l)}$}). 
Conducting SVD decomposition on {\small${\emb{V}}^{(0)}$}, we have ${\emb{V}}^{(0)} = \emb{U}_1\emb{\Sigma}\emb{U}_2^\mathsf{T}$, where {\small$\emb{U}_1$} and {\small$\emb{U}_2$} are unitary matrices of singular vectors.
Then following {\small$\emb{p}^{(k)} = {\emb{V}^{(0)}}^\mathsf{T}\emb{V}^{(0)}\emb{p}^{(k-1)}$}, we shall have {\small $\emb{p}^{(k)} = ({\emb{V}^{(0)}}^\mathsf{T}\emb{V}^{(0)})^k\emb{p}^{(0)}$}.
Replacing {\small${\emb{V}}^{(0)}$} with its SVD decomposition, we get the following equation:
\begin{sequation}
\emb{p}^{(k)} = (\emb{U}_2\emb{\Sigma}^{2k}\emb{U}_2^\mathsf{T})\emb{p}^{(0)}.
\end{sequation}%
Then we transform the projection matrix in Equation~(\ref{eq:projection}) as follows:
\begin{sequation}
\begin{aligned}
\emb{P} = \frac{\emb{p}^{(k)} {\emb{p}^{(k)}}^\mathsf{T}}{{\emb{p}^{(k)}}^\mathsf{T} \emb{p}^{(k)}} & = \frac{(\emb{U}_2\emb{\Sigma}^{2k}\emb{U}_2^\mathsf{T})\emb{p}^{(0)} {\emb{p}^{(0)}}^\mathsf{T} (\emb{U}_2\emb{\Sigma}^{2k}\emb{U}_2^\mathsf{T})}
{{\emb{p}^{(0)}}^\mathsf{T} (\emb{U}_2\emb{\Sigma}^{2k}\emb{U}_2^\mathsf{T}) (\emb{U}_2\emb{\Sigma}^{2k}\emb{U}_2^\mathsf{T})\emb{p}^{(0)} } \\
 & = \emb{U}_2\emb{\Sigma}^{2k} \frac{\emb{U}_2^\mathsf{T}\emb{p}^{(0)} {\emb{p}^{(0)}}^\mathsf{T} \emb{U}_2} 
 {{\emb{p}^{(0)}}^\mathsf{T} \emb{U}_2\emb{\Sigma}^{4k}\emb{U}_2^\mathsf{T}\emb{p}^{(0)} }\emb{\Sigma}^{2k}\emb{U}_2^\mathsf{T}.
\end{aligned}
\end{sequation}%
Let $\emb{t} = \emb{U}_2^\mathsf{T}\emb{p}^{(0)}$, we can further simplify the above equation to:
\begin{sequation}
\emb{P} = \emb{U}_2 \emb{\Sigma}^{2k} \frac{ \emb{t} \emb{t}^\mathsf{T}}{\emb{t}^\mathsf{T}\emb{\Sigma}^{4k} \emb{t} }\emb{\Sigma}^{2k} \emb{U}_2^\mathsf{T}, 
 \text{ where scalar }  \emb{t}^\mathsf{T}\emb{\Sigma}^{4k} \emb{t} = \sum_{j=1}^c t_j^2 \sigma_j^{4k}.
\end{sequation}%
% Here {\small $\emb{t}^\mathsf{T}\emb{\Sigma}^{4k} \emb{t}$} is a scalar as:
% \begin{sequation}
% \emb{t}^\mathsf{T}\emb{\Sigma}^{4k} \emb{t} = \sum_{j=1}^c t_j^2 \sigma_j^{4k}.
% \end{sequation}%
Recalling that {\small $\widetilde{\emb{V}}^{(0)} = \emb{V}^{(0)}(\emb{I} - \epsilon \emb{P})$}, {\small $\mathbb{E}(\widetilde{\emb{V}}^{(0)}) = \emb{V}^{(0)} - \epsilon\cdot\mathbb{E}(\emb{V}^{(0)}\emb{P})$}.
Then we focus on the term {\small $\mathbb{E}(\emb{V}^{(0)}\emb{P})$} as follows:
\begin{sequation}
\mathbb{E}(\emb{V}^{(0)}\emb{P}) = \frac{1}{\emb{t}^\mathsf{T}\emb{\Sigma}^{4k} \emb{t}} \emb{U}_1 \emb{\Sigma}^{2k+1} \cdot \mathbb{E}(\emb{t} \emb{t}^\mathsf{T}) \cdot  \emb{\Sigma}^{2k} \emb{U}_2^\mathsf{T} 
\end{sequation}%
Since {\small$\emb{p}^{(0)}$$\sim$$\mathcal{N}(\emb{0}, \emb{I})$} and {\small $\emb{U}_2$} is a unitary matrix, thus {\small $\emb{t}$$\sim$$\mathcal{N}(\emb{0}, \emb{I})$}. 
This indices that each element of {\small $\emb{t}$}, e.g., {\small $t_j$ $\in$ $\emb{t}$}, is \textit{i.i.d.} random variable. Thus, {\small $\mathbb{E}({t}_j \cdot t_{k})=0$} for {\small$j\neq k$} and {\small $\mathbb{E}(\emb{t}\emb{t}^\mathsf{T})$} is a diagonal matrix, i.e., {\small $\mathbb{E}(\emb{t}\emb{t}^\mathsf{T})=\diag(t_1^2, t_2^2, \cdots, t_c^2)$}.
We then have:
\begin{sequation}
\mathbb{E}(\emb{V}^{(0)}\emb{P}) = \emb{U}_1 \cdot \diag \big(\frac{\sigma_1 t_1^2 \sigma_1^{4k}}{\sum_{j=1}^c t_j^2 \sigma_j^{4k}},  \cdots,  \frac{\sigma_c  t_c^2\sigma_c^{4k}}{\sum_{j=1}^c t_j^2 \sigma_j^{4k}}\big) \cdot \emb{U}_2^\mathsf{T}.
\end{sequation}%
Therefore, 
\begin{sequation}
\mathbb{E}(\widetilde{\emb{V}}^{(0)}) = \emb{U}_1 \cdot \diag \big( \sigma_1 - \epsilon \frac{\sigma_1 t_1^2 \sigma_1^{4k}}{\sum_{j=1}^c t_j^2 \sigma_j^{4k}},  \cdots,  \sigma_c - \epsilon \frac{\sigma_c  t_c^2\sigma_c^{4k}}{\sum_{j=1}^c t_j^2 \sigma_j^{4k}} \big)  \cdot \emb{U}_2^\mathsf{T}.
\end{sequation}%
Let {\small $\mu_k = 1 - \epsilon \frac{t_k^2\sigma_k^{4k}}{\sum_{j=1}^c t_j^2 \sigma_j^{4k}}$}, with {\small $\epsilon$ $\in$ $(0,1)$}, obviously, {\small $0<\mu_k<1$}. 
Furthermore, {\small $\forall k_1$ $\geq$ $k_2$}, we have:
\begin{sequation}
\label{eq:increase}
\resizebox{1\linewidth}{!}{$
\displaystyle
\mu_{k_1} - \mu_{k_2} = \epsilon \mathbb{E}(\frac{t_{k_1}^2\sigma_{k_1}^{4k}}{\sum_{j=1}^c t_j^2 \sigma_j^{4k}} - \frac{t_{k_2}^2\sigma_{k_2}^{4k}}{\sum_{j=1}^c t_j^2 \sigma_j^{4k}}) \geq \epsilon\sigma_{k_1}^{4k} \cdot \mathbb{E} (\frac{t_{k_1}^2 - t_{k_2}^2}{\sum_{j=1}^c t_j^2 \sigma_j^{4k}}) =0,
$}
\end{sequation}%
as {\small $\sigma_{k_2}^{4k} \geq \sigma_{k_1}^{4k}$}, and $t_{k_1}$ and $t_{k_2}$ are \textit{i.i.d.} random variables with same normal distribution.
Equation~(\ref{eq:increase}) proves that {\small $\mu_k$} is \textit{monotone non-decreasing} in expectation, which completes the proof.
\end{proof}





\begin{thm}[\textbf{Hamming Distance Matching}]
Given two hash codes, we have $(\alpha_x\emb{Q}_u)^\mathsf{T} \cdot (\alpha_y\emb{Q}_y)$ $=$ $\alpha_x\alpha_y$ $(d - 2D_{H}(\emb{Q}_x, \emb{Q}_y))$.
\end{thm} 

\begin{proof}
\begin{equation}
\begin{aligned}
\setlength\abovedisplayskip{2pt}
\setlength\belowdisplayskip{2pt}
\emb{Q}_x^\mathsf{T} \cdot \emb{Q}_y &= \big|\{ i|(\emb{Q}_{x})_i = (\emb{Q}_{y})_i = 1\}\big| +  \big|\{ i|(\emb{Q}_{x})_i = (\emb{Q}_{y})_i = -1\}\big| \\ 
& -  \big|\{ i|(\emb{Q}_{x})_i \neq (\emb{Q}_{y})_i\}\big|\\
& = d - 2 \cdot \big|\{ i|(\emb{Q}_{x})_i \neq (\emb{Q}_{y})_i\}\big|  = \underline{d - 2D_{H}(\emb{Q}_x, \emb{Q}_y))},\\
\end{aligned}
\end{equation}%
which completes the proof.
\end{proof}


\textbf{Training time complexity.}
As shown in Table~\ref{tab:train_time}, $|\mathcal{E}|$, $B$, $s$, and $n$ are the edge number, batch size, numbers of train iterations and Fourier Series decomposition terms.
(1) The time complexity of the graph normalization, i.e., $\emb{D}^{-\frac{1}{2}} \emb{A} \emb{D}^{-\frac{1}{2}}$, is $O(2|\mathcal{E}|)$.
(2) Before the graph convolution, we first conduct the feature dispersion only for the initial node embeddings, e.g., $\emb{V}_x^{(0)}$, which takes $O(\frac{2csK|\mathcal{E}|^2}{B})$ complexity.
In our experiment, hyper-parameter $K \leq 3$.
(3) In graph convolution, the time complexity is $O(\frac{2csL|\mathcal{E}|^2}{B})$, where $L \leq 4$ is a common setting~\cite{lightgcn,ngcf,kipf2016semi,graphsage} to avoid \textit{over-smoothing}~\cite{li2019deepgcns}.
(4) As for the loss computation, \model~takes $O\big(2sc|\mathcal{E}|\big)$ to compute $\mathcal{L}_{rec}$ and $O\big(2sd|\mathcal{E}|\big)$ for $\mathcal{L}_{bpr}$.
(5) Lastly, \model~takes $O(snd|\mathcal{E}|)$ to estimate the gradients for the $d$-dimension hash codes.
Thus, thee total complexity in total is quadratic to the graph edge numbers, i.e., $|\mathcal{E}|$, which is common in GCN frameworks.
 % and actually acceptable for large bipartite graphs, which may dispel the concerns of the realistic training cost.


 \begin{table}[t]
\setlength{\abovecaptionskip}{0.2cm}
\setlength{\belowcaptionskip}{0.2cm}
\centering
\notsotiny
\caption{Traing time complexity.}
\vspace{-0.05in}
\label{tab:train_time}
\setlength{\tabcolsep}{2mm}{
  \begin{tabular}{c | c | c | c | c }
\toprule
    {Graph Norm.}  & {Feat. Disp.}  & {Conv. \& Hash.}   & { Loss Comp.}  &{Grad. Est.} \\
\midrule[0.1pt]
    { $O(2|\mathcal{E}|)$} & {$O(\frac{2csK|\mathcal{E}|^2}{B})$} & {$O(\frac{2csL|\mathcal{E}|^2}{B})$} & { $O\big(2s(c$+$d)|\mathcal{E}|\big)$} & { $O(snd|\mathcal{E}|)$} \\
\bottomrule
\end{tabular}}
\end{table}

\begin{table}[t]
\setlength{\abovecaptionskip}{0.2cm}
\setlength{\belowcaptionskip}{0.2cm}
\centering
\footnotesize
\caption{Complexity of space and computation.}
\vspace{-0.05in}
\label{tab:prediction}
\setlength{\tabcolsep}{2.2mm}{
\begin{tabular}{c | c | c c}
\toprule
  ~          & {\footnotesize Space cost}   &  {\footnotesize \#FLOP} & {\footnotesize \#BOP}       \\
\midrule
\midrule
 {\scriptsize float32-based}      & {\scriptsize $32|\mathcal{V}_1 \cup \mathcal{V}_2|d$}       &   {\scriptsize $O\big(|\mathcal{V}_1| \cdot |\mathcal{V}_2| d\big)$}      &   {-}         \\
\midrule[0.1pt]
{\scriptsize \model}       & {\scriptsize $|\mathcal{V}_1 \cup \mathcal{V}_2| (d+32(L+1))$}    & {\scriptsize $O\big(4|\mathcal{V}_1| \cdot |\mathcal{V}_2| \big)$}            & {\scriptsize $O\big(|\mathcal{V}_1| \cdot |\mathcal{V}_2| d\big)$}    \\
\bottomrule
\end{tabular}}
\end{table}


% \vspace{0.05in}

% \subsubsection{\textbf{Hash Codes Space Cost.}}
\textbf{Hash codes space cost.}
As shown in Table~\ref{tab:prediction}, the total space cost of hash codes is {\small $O(|\mathcal{V}_1 \cup \mathcal{V}_2| (d+32(L+1)))$} bits, supposing that we use float32 for those rescaling factors in $L+1$ iterations.
Compared to the continuous embedding size, i.e., $32|\mathcal{V}_1 \cup \mathcal{V}_2|d$ bits, the theoretical size reduction ratio thus is:
\begin{equation}
\setlength\abovedisplayskip{2pt}
\setlength\belowdisplayskip{2pt}
\label{eq:space}
ratio = \frac{32|\mathcal{V}_1 \cup \mathcal{V}_2|d}{|\mathcal{V}_1 \cup \mathcal{V}_2| (d+32(L+1))} = \frac{32d}{d+32(L+1)}.
\end{equation}
As we just explained, stacking too many iteration layers will incurring performance detriment~\cite{li2019deepgcns}. Hence, if $L\leq4$ and $d=1024$, \model~can achieve considerable size compression. 

\vspace{0.05in}

{\textbf{Online matching.}}
The original score formulation in Equation~(\ref{eq:inner_score}) contains $d$ floating-point operations (\#FLOPs).
As shown in Table~\ref{tab:prediction}, using Hamming distance matching can convert the most of floating-point arithmetics to binary operations (\#BOPs), with slightly more \#FLOPs for scalar computations, i.e., $4\ll d$.




