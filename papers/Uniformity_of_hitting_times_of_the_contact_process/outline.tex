% !TEX root = main.tex

\section{Proof of Proposition~\ref{propEssTightness}}

%RECALL PROBLEM
%BIG IDEA: LEVEL-CROSSING UNDER UNCONDITIONAL MEASURE
The challenge of Proposition~\ref{propEssTightness} is to establish bounds on the defects $\sigma(x) - \t(x)$ for sites $x$ far away from the origin.
The essence of the proof of Proposition~\ref{propEssTightness} is a carefully devised level-crossing argument leveraging the renewal structure of standard hitting times under the unconditioned measure $\P$. To achieve this goal, we introduce variants of the times $\{u_n, v_n\}_{n\ge1}$ from Section~\ref{essHitSec} that are tailor-made for the purpose of our proof.

%TAKE LARGE BALL AROUND x
%IF AT LEAST ONE SHELL GENERATES SURVIVING INFECTION => USE GM
More precisely, after fixing $L$ as in the statement of the proposition, we consider the ball 
$$B(x, R^2) = \{y \in \Zd:\, |y - x| \le R^2 \}$$
of radius $R^2$ around $x$, where $R$ is large and depends on $L$, but not on $x$. We pay special attention to the times at which the  infection hits each of the  $R$ shells 
$$\partial B(x,iR) = \{y \in \Zd: |y-x| = iR \},\quad 1\le i \le R. $$
If for some $i$ the space-time point where the infection hits the $i$th shell for the first time gives rise to a surviving contact process, then, by a renewal argument, it suffices to bound the essential hitting time of a point at distance at most $R^2$ from the origin. In particular, for this it is sufficient to invoke earlier location-dependent upper bounds.

%RENEWAL => PROBABILITY THAT ALL INFECTIONS DIE OUT DECAYS EXPONENTIALLY
The alternative is that upon hitting each of the $R$ shells, the infection started from the hitting point dies out. In the super-critical regime, a further renewal argument shows that this is an event of probability decaying geometrically in the number of shells. 


%RENEWAL REQUIRES RAPID DEATH OF NON-SURVIVING INFECTIONS
We clearly need a due amount of care in order to make the renewal argument rigorous. In particular, it is important to know that the infections starting from the hitting points die out before the next shell is reached.

Due to the regenerative arguments sketched above, we need the location-dependent bounds from~\cite[Theorem 15]{GaretMarch14} not only if the infection is started from the origin, but from an arbitrary set $A \subseteq \Zd$ with $o \in A$. More precisely, we now put $u^A_0(x) = v^A_0(x) = 0$ and
\[u^A_{k+1}(x)=\inf\{t\ge v^A_k(x):x\in\xi^A_t\},\]
whereas the recursion for $v^A_{k+1}$ is unchanged, i.e., 
\[v^A_{k+1}(x)=\sup\{s\ge0:\; (x,u^A_{k+1}(x)) \rightsquigarrow \Zd \times \{s\} \}. \]
Finally, we set $\sigma^A(x)=u_{K^A(x)}(x) $ where 
\[K^A(x)=\min\{n\ge0:v^A_n(x)=\infty\text{ or }u^A_n(x)=\infty\}.\]


%The proof is based on the shape theorem and the notion of essential hitting times.

\begin{lemma}
	\label{cor20}
There exist $c$, $c' > 0$ such that for all $L > 0$, $x \in \Zd$ and $A \subseteq \Zd$ with $o \in A$,
\begin{align}\label{eqcor20}
\oP\left(\sigma^A(x) > L\right) \leq \exp\{-cL + c'|x|\}.
\end{align}
\end{lemma}
%For $A = \{0\}$, this follows immediately from \cite[Corollary 20]{GaretMarch14}, the novelty being uniformity in the initial condition $A$. 
%In view of \cite[Corollary 20]{GaretMarch14}, it is plausible that $C$ may be chosen as a linear function of $\|x\|$; however, we do not need this to prove the (much stronger) bound of Proposition \ref{propEssTightness}. 
\begin{proof}
The result follows from Chebyshev's inequality and the claim that there exist $\beta, \gamma > 0$ such that, for all $x \in \Zd$ and all $A \subseteq \Zd$ with $o \in A$,
$$\oE(\exp\{\beta \sigma^A(x)\}) \leq \exp\{\gamma |x|\}. $$
	For the case $A = \{o\}$, this is \cite[Theorem 15]{GaretMarch14}. The statement with general $A \subseteq \Zd$ with $o \in A$ is proved in exactly the same way, so we do not go over the details.
\end{proof}


\begin{proof}[Proof of Proposition~\ref{propEssTightness}]
  Fix $L > 0$, which we may assume to be large throughout the proof without loss of generality. Define $R = \lfloor L^{1/4}\rfloor$. Let us first fix $x \in \Zd$ with $|x| \leq R^2$. Then, by Lemma~\ref{cor20},
\begin{equation}\oP(\sigma(x) - t(x) > L) \leq \oP(\sigma(x) > L) \stackrel{\eqref{eqcor20}}{\leq} \exp\left\{-cL + c'|x|\right\} < \exp\left\{-cL/2\right\}.\label{eq:zeroth_ing}\end{equation}

Now we fix $x$ with $|x| > R^2$ and aim to show that
\begin{equation}\label{eqDestBd}\P(\sigma(x) < \infty,\;\sigma(x) - t(x) > L) < C_0\exp\{-L^{\gamma_0}\}\end{equation}
for some $C_0,\gamma_0 > 0$; the desired bound then follows since $\oP(\sigma(x) < \infty) = 1$. As suggested in the outline of proof, we define the shell radius $r_i = R(R-i+1)$ and let
$$U_i = \inf\{t: \xi_t^o \cap B(x,r_i) \neq \varnothing\},\qquad i = 1,\ldots, R.$$
denote the first time that the infection hits the $i$th shell. On the event $\{U_i < \infty\}$, let $Z_i \in \partial B(x,r_i)$ denote the unique point of $\xi^o_{U_i} \cap B(x, r_i)$. Also on $\{U_i < \infty\}$, let
$$V_i = \inf\{s \geq U_i:\;(Z_i,U_i) \not\rightsquigarrow \Zd \times \{s\}\}$$
	denote the time at which  the infection starting from $(Z_i, U_i)$ dies out.

	Then there are three alternative scenarios under which the event $\{\sigma(x) < \infty,\;\sigma(x) - t(x) > L\}$ can occur: 
	\begin{enumerate}
		\item there exists at least one shell $\partial B(x, r_i)$ such that the infection generated by the first hitting point survives forever, i.e., $V_i = \infty$,
		\item for each $i \in \{1,\ldots, n\}$ the infection generated by the hitting point $(Z_i, U_i)$ dies out before the global infection reaches the $(i+1)$th shell, 
		\item one of these non-surviving infections lives longer than it takes the global infection to reach the next shell.
	\end{enumerate}
	More precisely,
\begin{equation}\label{eq:3terms}\begin{split}
\P(\sigma(x) < \infty,\; \sigma(x) - t(x) > L) &\leq \sum_{n=1}^R\P(U_n < \infty,\; V_n = \infty,\; \sigma(x) - t(x) > L) \\&+
 \P(U_1 < V_1 < U_2 < V_2 <\cdots < U_R < V_R < \infty)\\&
	+ \sum_{n=1}^{R-1} \P(U_{n+1} < V_{n} <\infty)
\end{split}\end{equation}
	We separately bound the three terms on the right-hand side. By repeated use of the Markov property, we have 
	\begin{equation}\P(U_1 < V_1 < U_2 < V_2 <\cdots < U_R < V_R < \infty)\leq(1-\rho)^R, \label{eq:first_ing}\end{equation} 
		where we recall that $\rho = \P_\lambda(o \rightsquigarrow \infty)$  denotes the survival probability. Next, for any $n$,
\begin{align}
&\P(U_n < \infty,\; V_n = \infty,\;\sigma(x) - t(x) > L) \leq \P(U_n < \infty,\;V_n = \infty,\;\sigma(x) - U_n > L) \nonumber\\
&=\sum_{y \in \partial B(x,r_n)}\; \sum_{\substack{A \subseteq \Zd,\\A\text{ finite}}} \P(U_n < \infty,\; Z_n = y,\; (y, U_n) \rightsquigarrow \infty,\; \xi^o_{U_n} = A,\; \sigma(x) - U_n > L)\nonumber\\
&= \sum_{y \in \partial B(x,r_n)}\; \sum_{\substack{A \subseteq \Zd,\\A \text{ finite}}} \P(U_n < \infty,\; Z_n = y,\;  \xi^o_{U_n} = A) \cdot \rho\cdot \oP(\sigma^{A-y}(x-y) > L)\nonumber\\
&\leq \rho \cdot \P(U_n < \infty) \cdot \exp\{-cL/2\},\label{eq:second_ing}
\end{align}
	the last inequality is obtained as in \eqref{eq:zeroth_ing} using $|x - y| \le R^2$. We now turn to the term inside the third sum on the right-hand side of \eqref{eq:3terms}. Since the contact process is in the super-critical regime, the survival time of a non-surviving infection started at a single site has exponential tails, cf.~\cite[Proposition 5]{GaretMarch12}, so that the Markov property yields
\begin{equation}\begin{split}\P(U_{n+1} < V_n < \infty) &\leq \P(V_n - U_n \in (\sqrt{R}, \infty)) + \P(U_{n+1} - U_n < 2\sqrt{R})\\
&\leq \exp\{-c\sqrt{R} \} + \P(\partial B(x,r_i) \times [0,2\sqrt{R}] \rightsquigarrow B(x,r_{i+1}) \times[0,2\sqrt{R}]).\end{split}\label{eq:third_ing}
\end{equation}
Since the propagation speed of the infection is at most linear outside an event of exponentially decaying probability ~\cite[Proposition 5]{GaretMarch12}, the second term on the right-hand side is less than
\begin{align}\nonumber&\#\partial B(x,r_i) \cdot 2d\lambda \cdot \int_0^{2\sqrt{R}} \max_{y \in \partial B(x,r_i)} \P\Big((y,s) \rightsquigarrow B(x,r_{i+1})\times [s,2\sqrt{R}] \Big)\; \mathrm{d}s
\\\nonumber&\leq \#\partial B(x,r_i) \cdot 2d\lambda \cdot 2\sqrt{R} \cdot \P\Big(\bigcup_{s \leq 2\sqrt{R}}\xi^y_s \nsubseteq B(y, R/2)\Big) \\&\leq \#\partial B(x,r_i) \cdot 2d\lambda \cdot 2\sqrt{R} \cdot \exp\{-c\sqrt{R}\}.\label{eq:third_ing0}
\end{align}
Combining the bounds \eqref{eq:3terms}, \eqref{eq:first_ing}, \eqref{eq:second_ing}, \eqref{eq:third_ing} and \eqref{eq:third_ing0} implies  \eqref{eqDestBd} and finishes the proof.
\end{proof}
