\documentclass[11pt,a4paper,reqno]{amsart}
\usepackage{amsmath,amssymb,amsfonts}
\usepackage{mathtools}
%\usepackage{amsthm}
% \usepackage[pdfpagelabels=true,linktocpage]{hyperref}
\usepackage[usenames,dvipsnames]{color}
\usepackage{xcolor}
\usepackage{xspace}
\usepackage{microtype}
\usepackage{todonotes}
\usepackage{colonequals}
%\usepackage{wasysym}
 \usepackage{booktabs}



\usepackage{multirow}
\usepackage{multicol}

\usepackage{paralist}

\usepackage{subfigure}
\usepackage{url}
\usepackage{appendix}
\usepackage{graphicx}
%\usepackage{float}
%\usepackage[pdfpagelabels=true]{hyperref}
\usepackage{algorithm}
\usepackage{listings}
%\usepackage{newalg}
\usepackage{nicefrac}
\usepackage{tikz} 
\usepackage{pgfplots}
%\usepgfplotslibrary{external}
%\pgfplotsset{compat=1.8}
%\tikzexternalize[prefix=TikzPictures/]
%\usepackage{marvosym}
%\usepackage{wrapfig}




\lstset{
	basicstyle=\ttfamily,
    keywordstyle=\bfseries,
    showstringspaces=false,
    numbers=left,
    numberstyle=\tiny,
    morekeywords={}
}

\usetikzlibrary{arrows,decorations.pathmorphing,positioning,fit,trees,shapes,shadows,automata,calc} 
%\usetikzlibrary{patterns,arrows,arrows.meta,calc,shapes,shadows,decorations.pathmorphing,decorations.pathreplacing,automata,shapes.multipart,positioning,shapes.geometric,fit,circuits,trees,shapes.gates.logic.US,fit,automata,snakes,shapes.geometric}

\pagestyle{plain}

\tikzset{outline/.style args={#1}{%
  draw=#1,thick,fill=#1!50}}

\tikzset{
  dot hidden/.style={},
  line hidden/.style={},
  dice hidden/.style={},
  dot color/.style={dot hidden/.append style={color=#1}},
  dot color/.default=black,
  line color/.style={line hidden/.append style={color=#1}},
  line color/.default=black,
  dice color/.style={dice hidden/.append style={color=#1,fill}},
  dice color/.default=white
}\def\dotsize{0.1}
\newcommand{\drawdie}[2][]{%
\begin{tikzpicture}[x=1em,y=1em,#1]
  \draw 	[thick, rounded corners=0.5,line hidden,dice hidden] (0,0) rectangle (1,1);
  \ifodd#2
    \fill[dot hidden] (0.5,0.5) circle (\dotsize);
  \fi
  \ifnum#2>1
  \fill[dot hidden] (0.25,0.25) circle (\dotsize);
  \fill[dot hidden] (0.75,0.75) circle (\dotsize);
  \ifnum#2>3
    \fill[dot hidden] (0.25,0.75) circle (\dotsize);
    \fill[dot hidden] (0.75,0.25) circle (\dotsize);
    \ifnum#2>5
      \fill[dot hidden] (0.75,0.5) circle (\dotsize);
      \fill[dot hidden] (0.25,0.5) circle (\dotsize);
    \fi
  \fi
\fi
\end{tikzpicture}
}



\makeatletter
\let\tmp=\phi \let\phi=\varphi \let\varphi=\tmp
\let\tmp=\epsilon \let\epsilon=\varepsilon \let\varepsilon=\tmp

\let\emptyset=\varnothing
\def\O{\Ocal}
\def\o{\scalebox{0.65}{\Ocal}}
\def\aseq#1{\overset{#1}{=}}
\def\i{{\bm\imath}}
\def\j{{\bm\jmath}}
\def\stab{\operatorname{\mathrm{Stab}}}
\def\emp{\operatorname{\qbf}}
\def\convconend{\mathrm{\Kbf}_{k,l}}
\def\dist{\operatorname{{\bm\delta}}}
\def\dista{\operatorname{\hat{\bm\delta}}}
\def\distaa{\operatorname{\hatt{\bm\delta}}}
\def\d{\operatorname{\drm}}
\def\kd{\operatorname{\mathrm{kd}}}
\def\ad{\operatorname{\mathbf{ad}}}
\def\ikd{\operatorname{\mathbf{k}}}
\def\aikd{\operatorname{{\bm\kappa}}}\let\akd=\aikd
\def\aep{\operatorname{\mathbf{aep}}}
\def\Ent{\operatorname{\mathcal{E}}\!\!{nt}}\let\ent=\Ent
\def\MI{\operatorname{\mathcal{I}}}
\def\Aut{\operatorname{\mathrm{Aut}}}
\def\IO{\operatorname{\mathrm{IO}}}\let\io=\IO
\def\Ext{\operatorname{\mathrm{Ext}}}
\def\Hom{\operatorname{\mathrm{Hom}}}
\def\Interior{\operatorname{\mathrm{Interior}}}
\def\closure{\operatorname{\mathrm{Closure}}}
\def\supp{\operatorname{\mathrm{supp}}}
\def\defect{\operatorname{\mathrm{Defect}}}
\def\aeq{\sim\raisebox{-0.4mm}{$\mkern-8mu{\scriptscriptstyle{}a}\mkern3mu$}}
\def\prob{\operatorname{\mathbf{Prob}}}\let\Prob=\prob
\def\probhom{\prob_{\mathbf{h}}}\let\Probhom=\probhom
\def\probas{\prob^{\infty}}
\def\probhomas{\probhom^{\infty}}
\def\Set{\operatorname{\mathbf{Set}}}
\def\Id{\operatorname{\mathrm{Id}}}
\def\lin{\operatorname{\Lsf}}
\def\qlin{\operatorname{\Qsf\Lsf}}
\def\ac{\operatorname{\Afrak\Cfrak}}
\def\dia{{\bm\Diamond}}
\def\emptyconfig{\mkern3mu\raisebox{0.2ex}{\ensuremath{\bm/}}\mkern-13mu\Ocal}
\def\emptycat{\bm{\emptyset}}
\def\res{\Gammabf^{\circ}}
\def\sres{\bar\Gammabf}
\def\c#1{(\!#1\!)}
\def\<#1>{\left\langle#1\right\rangle}
\def\size#1{[\mkern-5mu[#1]\mkern-5mu]}
\def\sep{\,||\,}
\def\mix{\operatorname{\Mcal\mkern-2mu\mathit{ix}}}
%\def\rel{|\mkern-5.1mu\llcorner}
\def\rel{|\mkern-2.5mu\raisebox{-0.57ex}{\rule{0.3em}{0.15ex}}}
\def\linseq#1{\overset{\to}{#1}}
\let\trop=\top
\let\bar=\overline
\newlength{\hattheight}
\def\Xcalbf{\bm\Xcal}
\def\hatt#1{\hat{\hat{#1}}}%
%      \settoheight{\hattheight}{\ensuremath{\hat{#1}}}%
%      \addtolength{\hattheight}{-0.3ex}%
%       \hat{\vphantom{\rule{1pt}{\hattheight}}%
%      \hat{\vphantom{\rule{1pt}{\hattheight}}%
%      \smash{\hat{#1}}}%
% }
\newcommand{\into}[1][]{\stackrel{#1}{\hookrightarrow}}
\renewcommand{\to}[1][]{\stackrel{#1}{\rightarrow}}
\newcommand{\too}[1][]{\stackrel{#1}{\longrightarrow}}
\newcommand{\ot}[1][]{\stackrel{#1}{\leftarrow}}
\newcommand{\oot}[1][]{\stackrel{#1}{\longleftarrow}}
%\newcommand{\into}[1][]{\stackrel{#1}{\hookrightarrow}}
\def\indep{\,\makebox[0cm][l]{$\bot$}\mkern2mu\bot\,}
\newcommand{\un}[2][5mu]{\underline{#2\mkern-#1}\mkern#1} 

\newcommand{\Deltan}[1][n]{\Delta^{\!\!\!^{(\!#1\!)}}\!}

\def\bluepar#1\par{%
    \par\textcolor{blue}{#1}\par}
\def\redpar#1\par{%
    \par\textcolor{red}{\hspace{-1.87\parindent}$\bullet$ #1}\par}
\def\greenpar#1\par{%
    \par\textcolor{green}{#1}\par}
\def\jimpar#1\par{%
	\par\textcolor{purple}{\hspace{-1.87\parindent}$\blacksquare$  #1}\par}

\def\skippar#1\par{\par\relax}

\def\blue#1{{\color{blue}#1}}
\def\red#1{{\color{red}#1}}
\def\green#1{{\color{green}#1}}

\def\theenumi{\roman{enumi}}


\makeatother

\newtheorem{theorem}{Theorem}[section]
%[section] means numbering within sections
\newtheorem{corollary}[theorem]{Corollary}
\newtheorem{approximation}[theorem]{Approximation}
\newtheorem{result}[theorem]{Result}
%[theorem] means sharing numbering system with theorem-environment
\newtheorem{lemma}[theorem]{Lemma}
\newtheorem{proposition}[theorem]{Proposition}
\theoremstyle{definition}
\newtheorem{definition}[theorem]{Definition}
\newtheorem{note}[theorem]{Note}
\newtheorem{remark}[theorem]{Remark}
\newtheorem{conjecture}[theorem]{Conjecture}
\newtheorem{example}[theorem]{Example}
\newtheorem{counterexample}[theorem]{Counterexample}
\newtheorem{setup}[theorem]{Setup}




\keywords{Contact process, coexistence, hitting times}
\subjclass[2010]{60K35; 82C22}


\begin{document}

\author{Markus Heydenreich}
%\address[Markus Heydenreich, Christian Hirsch]{Mathematisches Institut, Ludwig-Maximilians-Universit\"at M\"unchen, Theresienstra\ss e 39, 80333 M\"unchen, Germany}
%\email{m.heydenreich@lmu.de, christian.hirsch@lmu.de}
\author{Christian Hirsch}
\address[Markus Heydenreich, Christian Hirsch]{Mathematisches Institut, Ludwig-Maximilians-Universit\"at M\"unchen, Theresienstra\ss e 39, 80333 M\"unchen, Germany}
\email{m.heydenreich@lmu.de,christian.hirsch@lmu.de}
\author{Daniel Valesin}
\address[Daniel Valesin]{Johann Bernoulli Institute, University of Groningen, Nijenborgh 9, 9747AG Groningen, The Netherlands}
\email{D.Rodrigues.Valesin@rug.nl}



\thanks{The work of CH was funded by LMU Munich's Institutional Strategy LMUexcellent within the framework of the German Excellence Initiative.}



\title{Uniformity of hitting times of the contact process}

\date{\today}

\begin{abstract}
For the supercritical contact process on the hyper-cubic lattice started from a single infection at the origin and conditioned on survival, we establish two uniformity results for the hitting times $t(x)$, defined for each site $x$ as the first time at which it becomes infected. First, the family of random variables $(t(x)-t(y))/|x-y|$, indexed by $x \neq y$ in $\mathbb{Z}^d$, is stochastically tight. Second, for each $\varepsilon >0$ there exists $x$ such that, for infinitely many integers $n$, $t(nx) < t((n+1)x)$ with probability larger than $1-\varepsilon$.  A key ingredient in our proofs is a tightness result concerning the essential hitting times of the supercritical contact process introduced by Garet and Marchand (Ann.\ Appl.\ Probab., 2012).\end{abstract}

\maketitle


	Reinforcement learning has achieved great success in areas such as Game-playing \citep{silver2018general,vinyals2019grandmaster}, robotics \cite{kober2013reinforcement}, large language models \citep{ouyang2022training}, etc.
However, due to safety concerns or physical limitations, in some real-world reinforcement learning problems, we must consider additional constraints that may influence the optimal policy and the learning process \citep{garcia2015comprehensive}.
% For example, a robotic arm must not take actions that may cause harm to itself or the environments.
A standard framework to handle such cases is the constrained Markov Decision Process (CMDP) \citep{altman1999constrained}.
Within the CMDP framework, the agent has to maximize
the expected cumulative reward while
obeying a finite number of constraints, which are usually in the form of expected cumulative cost criteria.

However, we are sometimes concerned with the problem with a continuum of constraints.
For example,
the constraints we meet might be time-evolving or subject to uncertain parameters, which
cannot be formulated as an ordinary CMDP
(see Examples \ref{Example_Time_Evolving} and  \ref{Example_Uncertain}).
In this paper we would study a generalized CMDP  
to address the above problem.  Because the constraints are not only infinite-number but also lie
in a continuous set,
the generalization is not trivial. Fortunately, we find that we can borrow the idea behind semi-infinite programming (SIP) \citep{remez1934determination, hettich1993semi} to deal with the semi-infinite constraints.
Accordingly, we propose \emph{semi-infinitely constrained Markov decision processes} (SICMDPs)
as a novel complement to the ordinary CMDP framework.
%More specifically,  an SICMDP model %, we consider 
%contains a continuum of constraints whereas an ordinary CMDP contains a finite number of constraints. 

%This generalization is natural but not trivial. However, we can brows the idea  
%The idea is quite natural and can be backtracked
%to the practice of extending linear programming to linear semi-infinite programming (LSIP) %\cite{remez1934determination, GobernaLSIO1998}.
%In addition, 
%As a complementary approach to the ordinary CMDP framework, 
%SICMDP can be used to model these problems  which cannot be described by a finite number of constraints
%that are not covered by .
%For example,
%the restrictions we consider can be time-evolving or subject to uncertain parameters
%, thus
%cannot be described by a finite number of constraints but a continuum of constraints 
%(see Examples \ref{Example_Time_Evolving} and  \ref{Example_Uncertain}).

We also present two reinforcement learning algorithms to solve SICMDPs called SI-CRL and SI-CPO, respectively.
SI-CRL is a model-based reinforcement learning algorithm designed for tabular cases, and SI-CPO is a policy optimization algorithm for non-tabular cases.
% and analyze its performance both theoretically and empirically.
The main challenge is that we need to deal with a continuum of constraints, thus reinforcement learning algorithms for ordinary CMDPs do not work anymore.
In SI-CRL, we tackle this difficulty by first transforming the reinforcement learning problem to an equivalent LSIP problem, which can then be solved using methods in the LSIP literature like the dual exchange methods \citep{Hu1990,reemtsen1998numerical}.
In SI-CPO, we resort to the idea of cooperative stochastic approximation developed in \cite{lan2020algorithms, wei2020comirror}.
As far as we know, we are the first to introduce tools from semi-infinitely programming (SIP) into the reinforcement learning community for solving constrained reinforcement learning problems.

% To the best of our knowledge, we are the first to apply tools from semi-infinitely programming (SIP) to solve reinforcement learning problems.
Furthermore, we give theoretical analysis for both SI-CRL and SI-CPO.
We decompose the error of SI-CRL into two parts: the statistical error from approximating the true SICMDP with an offline dataset and the optimization error due to the fact that the solution of the LSIP problem obtained by the dual exchange method is inexact.
On the optimization side, we show that the iteration complexity of SI-CRL is $O\left(\left\{\mathrm{diam}(Y)L\sqrt{|\gS|^2|\gA|m}/\left[(1-\gamma)\epsilon\right]\right\}^m\right)$.
On the statistical side, we show that the sample complexity of SI-CRL is $\widetilde O\left(\frac{|S|^2|A|^2}{\epsilon^2(1-\gamma)^3}\right)$ if the offline dataset is generated by a generative model, and $\widetilde O\left(\frac{|S||A|}{\nu_{\min} \epsilon^2(1-\gamma)^3}\right)$ if the dataset is generated by a probability measure $\nu$ as considered in \cite{chen2019information}.
Here $\widetilde O$ means that all logarithm terms are discarded.
For SI-CPO, things become a little more complicated because other than the statistical error and the optimization error, we also need to consider the function approximation error, which comes from imperfect policy parametrizations.
It is shown if the function approximation error can be controlled to $O(\epsilon)$ order, the iteration complexity of SI-CPO is $\widetilde{O}\left(\frac{1}{\epsilon^2(1-\gamma)^6}\right)$ and the sample complexity of SI-CPO is $\widetilde{O}(\frac{1}{\epsilon^4(1-\gamma)^{10}})$.
Here our iteration complexity bound is equivalent to a typical $\widetilde O(1/\sqrt{T})$ global convergence rate.

We perform a set of numerical experiments to illustrate the SICMDP model and validate our proposed algorithms.
Specifically, we examine two numerical examples, namely the discharge of sewage and ship route planning.
Through the discharge of sewage example, we show the advantage of the SICMDP framework over the CMDP baseline obtained by naive discretization in modeling realistic sequential decision-making problems.
Moreover, we demonstrate the effectiveness of the SI-CRL and SI-CPO algorithms in such tabular environments. 
In the ship route planning example, we illustrate the benefits of the SICMDP framework and the ability of the SI-CPO algorithm to address complex continuous control tasks involving continuous state spaces with modern deep reinforcement learning techniques.

% In summary, our contributions are listed as follows.
% First, we present the SICMDP model, which can be viewed as a generalization of the ordinary CMDP model.
% Second, we propose an algorithm to perform reinforcement learning for SICMDPs, which is called SI-CRL, and we believe that we are the first to apply tools from SIP
% to solve reinforcement learning problems.
% Third, we give a theoretical analysis of SI-CRL and identify both its sample complexity and iteration complexity.
% In addition, we perform numerical experiments to illustrate the SICMDP model and validate the SI-CRL algorithm.
% \{This paragraph can be removed!!! \}





	%!TEX ROOT = ../../centralized_vs_distributed.tex

%\subsection{Paper outline}\label{sec:outline}

\begin{table}
	\centering
	\caption{Theoretical tools (italic) and technical results (roman).}
	\label{tab:results}
	\footnotesize
	\begin{tabular}{|c|c|c|c|}
		\hline
		 & \textbf{Model} & \textbf{Stability} & \textbf{Variance} \\
		\hline
		\multirow{5}{*}{\shortstack{\textbf{Cont.} \\\textbf{time} \\\textbf{(CT)}}} & 
										\makecell{Single int. \\ \eqref{eq:cont-time-single-int-model},\eqref{eq:prop-control}} & 
										\makecell{\textit{Scalar SDDEs~\cite{KuchlerLangevinEqs}} \\ 
											Closed form~\eqref{eq:cont-time-single-int-variance-condition}} &
										\makecell{\textit{Scalar SDDEs~\cite{KuchlerLangevinEqs}} \\
											Closed form~\eqref{eq:cont-time-single-int-steady-state-variance}} \\
		\cline{2-4}
									& \makecell{Double int.\\ \eqref{eq:cont-time-double-int-model}--\eqref{eq:control-input-PD}} & 
									\makecell{\textit{Exponential} \\ \textit{polynomials~\cite{BAPTISTINI1997259}} \\ 
										\textit{SDDEs~\cite{datko1978procedure,wangBoundedness}} \\
										{Implicit}~\eqref{eq:cont-time-double-int-stability-condition}} & 
									\makecell{\textit{SDDEs~\cite{wangBoundedness}, time-}\\
										\textit{scale separation~\cite{khalil2002nonlinear}}\\
										{Integral form}~\eqref{eq:2nd-order-cont-ss-variance}\\
										Approximated~\eqref{eq:x-dynamics-1st-order-approximation}} \\
		\hline
		\multirow{4}{*}{\shortstack{\textbf{Disc.} \\ \textbf{time} \\\textbf{(DT)}}} &  
										\makecell{Single int. \\ \eqref{eq:disc-time-single-int-model},\eqref{eq:prop-control}} & 
										\makecell{\textit{Root locus~\cite{Westphal2001}}\\ 
											Closed form~\eqref{eq:disc-time-single-int-stability-condition}} &
										\makecell{\textit{Moment matching w/} \\
											\textit{Yule-Walker eqs.~\cite{yuleWalkerEqs}} \\
											{Recursive}~\eqref{eq:disc-time-single-int-moment-matching-eqs},\eqref{eq:disc-time-single-int-variance-explicit}} \\
		\cline{2-4}
									&  \makecell{Double int. \\ \eqref{eq:disc-time-double-int-model}} & 
									\makecell{\textit{Jury criterion~\cite{Jury}} \\ 
										{Closed form}~\eqref{eq:disc-time-double-int-characteristic-polinomial}} &
									\makecell{\textit{Moment matching w/} \\
										\textit{Yule-Walker eqs.~\cite{yuleWalkerEqs}} \\
										{Closed form}~\eqref{eq:disc-time-double-int-moment-matching-eqs}} \\
		\hline
	\end{tabular}
\end{table}

\myParagraph{\titlecap{Paper outline}}
In~\autoref{sec:setup} we describe models for communication and controller architecture
and formulate the minimum-variance control design problem.
\revision{While we first utilize ring topology to provide analytical insight
we also demonstrate that our framework can be extended to general undirected topologies; see~\autoref{sec:generic-topology}.}\linebreak
\revision{In Sections~\ref{sec:cont-time}--\ref{sec:cont-time-single-int-control-design}, we %solve the problem for continuous-time agent dynamics,
lay the ground for our main result.
In~\autoref{sec:cont-time}, 
we derive conditions for mean-square stability 
and compute the steady-state variance of continuous-time stochastically forced systems
using Stochastic Delay Differential Equations (SDDEs). 
In~\autoref{sec:cont-time-single-int-control-design},
we prove that the control design problem is convex} \review{and in 
\revision{\autoref{sec:numerical-results} we present} our main results:
by numerically computing the optimal controller gains,
we show that the closed-loop performance is optimized by sparse architectures.
Furthermore, 
we derive analytical expression~\eqref{eq:trade-off} for continuous-time single-integrator dynamics 
which demonstrates that the minimizer is in general nontrivial.}
To address wireless communication,
we study discrete-time systems in~\autoref{sec:disc-time}
and show that the fundamental behavior of the system does not change.
\autoref{tab:results} summarizes our technical results and the theoretical tools used throughout the paper.
\revision{Apart from classical control techniques such as the Jury stability criterion,
we also leverage more unconventional tools from mathematical literature,
such as exponential polynomials~\cite{BAPTISTINI1997259}.}
Concluding remarks are given in~\autoref{sec:conclusion}.
	\section{Influence in Completely Bounded Block-multilinear Forms}
\label{sec:proof}
\newcommand{\blocks}{\mathrm{blocks}}
\newcommand{\lt}{\mathrm{left}}
\newcommand{\rt}{\mathrm{right}}



In this section we prove the non-commutative root-influence inequality (\thmref{thm:bh-intro}),  the special case of the Aaronson-Ambainis conjecture given in \thmref{thm:aa}, and also briefly mention how the simulation result in \corref{cor:sim} follows from \thmref{thm:aa} and the results in \cite{AA14}. We first need some preliminaries from free probability theory. 



\subsection{Low-degree Polynomials of Haar Random Unitaries}

As discussed in the proof overview, we require bounds on the operator norm (as well as normalized trace) of low-degree polynomials of random unitaries and these follow from known results in free probability theory. Here we explain these connections and also prove some auxillary lemmas needed for the proof of \thmref{thm:bh-intro} and \thmref{thm:aa}. 



Let $z_{\ui}$ denote the non-commutative monomial $z_{i_1} z_{i_2} \cdots z_{i_d}$ for a $d$-tuple $\ui  = (i_1, \ldots, i_d) \in [t]^d$ and let $p(z_1, \ldots, z_t)$ be a non-commutative polynomial in the variables $z_1, \ldots, z_t$. We are interested in computing the operator norm $\|\cdot\|_{\op}$ and the normalized trace  $\tr_N$ of the polynomial $p(z_1, \ldots, z_t)$ (or its higher moments) when substituting $N \times N$ Haar random unitaries for the variables $z_i$.

As explained previously, the theory of free probability gives us tools that allow us to compute  the above in the limit $N \to \infty$. In particular, Voiculescu \cite{V98} showed that the  (normalized) trace of polynomials in Haar random unitaries and their conjugates converges to the trace of the same polynomial evaluated on certain infinite-dimensional operators called \emph{Haar unitaries} that satisfy a non-commutative notion of independence called \emph{free independence}. This was strengthened by Collins and Male \cite{CM11} who showed that such convergence also holds for the operator norm. A short primer on free probability is given in \appref{sec:free}, but for now one can think of $\CA$ as a self-adjoint algebra of bounded linear operators on a Hilbert space and $\phi$ as a trace functional for such operators in the statement given below.


\begin{theorem}[\cite{V98, CM11}] \label{thm:voiculescu}
    Let $p(z_1, \ldots, z_{2t})$ be a non-commutative polynomial in $\BR\langle z_1, \ldots, z_{2t}\rangle$. If $U_1, \ldots, U_t$ are $N \times N$ Haar random unitaries, then almost surely,
    \begin{align*}
     \ \tr_N[p(U_1, \ldots, U_t, U^*_1, \ldots, U_t^*)] &~\xrightarrow[N \to \infty]{}~ \phi[p(u_1, \ldots, u_t, u^*_1, \ldots, u^*_t)],\\
    \  \|p(U_1, \ldots, U_t, U^*_1, \ldots, U_t^*)\|_{\op} &~\xrightarrow[N \to \infty]{}~ \| p(u_1, \ldots, u_t, u^*_1, \ldots, u^*_t)\|,
    \end{align*}
    where $u_1, \ldots, u_t$ are free Haar unitaries in a $C^*$-probability space $(\CA, \phi)$ and $\|\cdot\|$ is the norm for the underlying $C^*$-algebra.
\end{theorem}




Using the above result it suffices to consider free Haar unitaries in a $C^*$-probability space to compute the operator norm and trace of polynomials of random unitaries. For a non-commutative polynomial $p(z_1, \ldots, z_t) = \sum_{|\ui|\le d} c_{\ui}z_{\ui}$, denoting by $\|p\|_2 =  \left(\sum_{|\ui| \le d} |c_{\ui}|^2\right)^{1/2}$, one can show the following easily using techniques from free probability. 

\begin{lemma} \label{thm:trace}
    Let $p(z_1, \ldots, z_t) = \sum_{|\ui|\le d} c_{\ui}z_{\ui} $ be a non-commutative degree-$d$ polynomial in $\R\langle z_1, \ldots, z_t\rangle$ and $u_1, \ldots, u_t$ be free Haar unitaries in a $C^*$-probability space $(\CA, \phi)$. Then, 
     \[ \phi[p(u_1, \ldots, u_t) (p(u_1, \ldots, u_t))^*] =  \|p\|_2^2.\]
\end{lemma}

The above implies that $\tr_N[p(U_1, \ldots, U_t) (p(U_1, \ldots, U_t))^*]$ converges to $\|p\|_2^2$ almost surely as $N \to \infty$. We shall defer the proof of \lref{thm:trace} to \appref{sec:app}, but to aid our intuition we note here that since the  $U_i$'s are independent $N \times N$ Haar random unitaries, the expected value

\[ \BE\left[\tr_N[p(U_1, \ldots, U_t) (p(U_1, \ldots, U_t))^*\right] = \|p\|_2^2,\] 
{and from concentration of measure, it is natural to expect that it converges to the above value}. 


Similarly, to compute the operator norm of $p(U_1, \ldots, U_t)$ for Haar random unitaries one can instead study the norm of the polynomial evaluated on free Haar unitaries. Such bounds are easier to prove using the trace method since free independence imposes strong restrictions on the non-commutative moments. For instance, if $U_1$ and $U_2$ are independent $N \times N$ Haar random matrices, then $\BE[\tr_N(U_1U_2U^*_1U_2^*)]$ is non-zero (albeit quite small), while the corresponding trace evaluated on free Haar unitaries $u_1$ and $u_2$ is zero, that is $\phi(u_1u_2u^*_1u_2^*) = 0$. Thus, computing the trace $\phi[p(u_1,\ldots, u_t, u^*_1, \ldots, u_t^*)]$ reduces to handling the combinatorics of the patterns of $u_i$'s and $u_i^*$'s. 

In particular, we will rely on the following result that follows from the work of Kemp and Speicher \cite{KS05}  who consider the operator norm of homogeneous polynomials evaluated on free $R$-diagonal operators, a class that includes free Haar unitaries as well. We also remark that a bound where the right-hand side below is worse by a multiplicative $O(d^{1/2})$ factor also follows from the work of Haagerup\footnote{We note that Haagerup considered the more general case of polynomials in both $u_i$'s and $u^*_i$'s.}\cite{H78} who proved it in another context, predating even the introduction of free probability theory. 


\begin{theorem}[\cite{KS05}]
\label{thm:kemp-speicher}
    Let $p(z_1, \ldots, z_t) = \sum_{|\ui| = d} c_{\ui}z_{\ui} $ be a homogeneous non-commutative degree-$d$ polynomial in $\R\langle z_1, \ldots, z_t\rangle$ and $u_1, \ldots, u_t$ be free Haar unitaries in a $C^*$-probability space. Then, 
    \[ 
    \|p(u_1, \ldots, u_t)\| \le \sqrt{e(d+1)} \cdot \|p\|_2,
    \]
    where the left-hand side denotes the norm in the underlying $C^*$-algebra. 
\end{theorem}

For completeness, we  introduce the necessary free probability background and some combinatorial details in \appref{sec:app}, and we present the fairly short proof of \thmref{thm:kemp-speicher} (from \cite{KS05}) there in a self-contained way. We shall need to extend the above bound to non-homogeneous polynomials. Let $p(z_1, \ldots, z_t) = \sum_{|\ui| \le d} c_{\ui}z_{\ui}$ and  let $p_k(z_1, \ldots, z_t) = \sum_{|\ui| = k} c_{\ui}z_{\ui}$ denote the degree-$k$ homogeneous part of $p$. Writing $p_k = p_k(u_1, \ldots, u_t)$ for $0 \le k  \le d$ and $p = p(u_1, \ldots, u_t)$, it follows from the triangle inequality,  \thmref{thm:kemp-speicher}, and Cauchy-Schwarz, that
    \begin{align*}
        \ \|p\| &\le \sum_{k=0}^d \|p_k\| 
        \le 
        \sum_{k=0}^d\sqrt{e(k+1)}\|p_k\|_2
        \le
       \sqrt{e}\left(\sum_{k=0}^d (k+1)\right)^{1/2} \left(\sum_{k=0}^d  \|p_k\|^2_2\right)^{1/2} \leq \sqrt{e}(d+1)  \cdot\|p\|_2.
    \end{align*}
Thus, we essentially get the same bound as in the homogeneous case, at the expense of an additional $O(d^{1/2})$ factor.



Collecting all the above we have the following as a direct consequence:

\begin{theorem} \label{thm:op-norm}
    Let $p(z_1, \ldots, z_t) = \sum_{|\ui|\le d} c_{\ui}z_{\ui} $ be a non-commutative degree-$d$ polynomial in $\R\langle z_1, \ldots, z_t\rangle$ and $U_1, \ldots, U_t$ be independent $N \times N$ Haar random unitaries. Then, as $N \to \infty$, the following holds almost surely, 
    \[ \tr_N[p(U_1, \ldots, U_t) (p(U_1, \ldots, U_t))^*] =  \|p\|_2^2,\]
    and
    \[ \|p(U_1, \ldots, U_t)\|_{\op} \le \sqrt{e}(d+1)  \cdot \|p\|_2,\]
    Moreover, the factor $(d+1)$ in the operator norm bound can be improved to $\sqrt{d+1}$ if the polynomial is homogeneous.
\end{theorem}

Based on the above theorem, we prove the following key lemma which captures the polar decomposition strategy mentioned in the earlier proof overview (\secref{sec:bh}). This will serve as the key ingredient in the proof of \thmref{thm:aa} and \thmref{thm:bh-intro}. 

\begin{lemma}\label{lem:polar}
    Let $p$ be a non-commutative degree-$d$ polynomial in $\R\langle y_1, \ldots, y_m, z_1, \ldots, z_t\rangle$ given by
    \[ p(y_1, \ldots, y_m, z_1, \ldots, z_t) = \sum_{i=1}^m y_i q_i(z_1, \ldots, z_t) + q_0(z_1, \ldots, z_t).\]
    Then, for every $\delta > 0$, there exist an integer $N$ and $N \times N$ unitaries $V_1,\ldots, V_m, W_1, \ldots, W_t$ such that 
    \[ \|p(V_1, \ldots, V_m, W_1, \ldots, W_t)\|_{\op} \ge \frac{1}{\sqrt{e}(d+1)} \sum_{i=1}^m \|q_i\|_2 - \delta.\]
    Moreover, the factor in front can be improved to $(e(d+1))^{-1/2}$ if $p$ is homogeneous. 
\end{lemma}

\begin{proof}[Proof of \lref{lem:polar}]
     For an arbitrary integer $N$, let us pick independent $N \times N$ Haar random unitaries $W_1, \ldots, W_t$ which we substitute for the variables $z_1,\ldots,z_t$, respectively, and let $M_i = q_i(W_1, \ldots, W_t)$ be the corresponding random matrices. Then, for any tuple of matrices $V_1, \ldots, V_m$ that we could substitute for the variables $y_1, \ldots, y_m$, we have that 
    \[ 
    p(V_1, \ldots, V_m, W_1, \ldots, W_t) = \sum_{i=1}^m V_i M_i + M_0.
    \] 
     \thmref{thm:op-norm} and union bound imply that as $N \to \infty$, with probability $1$ all the following events simultaneously hold: 
    \begin{itemize}
        \item $\|M_i\|_{\op} \le \sqrt{e}(d+1) \cdot \|q_i\|_2$ for each $i$,
        \item $\tr_N(M^*_iM_i) = \|q_i\|_2^2$ for each $i$, where $\tr_N(M)$ is the normalized trace.
    \end{itemize}
   To show that the operator norm must be large, let us fix a sufficiently large $N$ and a choice of $N\times N$ unitaries $W_1, \ldots, W_t$ such that $M_i$ satisfies $\|M_i\|_{\op} \le \sqrt{e}(d+1) \cdot \|q_i\|_2 + \epsilon$ and $\tr_N(M^*_iM_i) \ge \|q_i\|_2^2 - \epsilon$ for each $0\le i\le m$, where $\epsilon$ can be made arbitrarily small by increasing $N$. For $0 \leq i \leq m$, let $M_i = U_i P_i$ be the left polar decomposition of $M_i$, where $U_i$ is a unitary matrix and $P_i$ is a positive semidefinite matrix.
   
   We select the tuple of unitary matrices $V_1, \ldots, V_m$ that we substitute for the variables $y_1, \ldots, y_m$ to be $V_i = U_0U^*_i$ for $i \in [m]$. With this we have that $\|p(V_1, \ldots, V_m, W_1, \ldots, W_t)\|_{\op}$ is at least
    \begin{align*}
         \Big\|M_0 + \sum_{i=1}^m V_iM_i\Big\|_{\op} & = \Big\|U_0 P_0 + \sum_{i=1}^m U_0 U_i^* U_iP_i \Big\|_{\op} \\
        \ & =  \Big\|U_0 P_0 + \sum_{i=1}^m U_0 P_i\Big\|_{\op}  = \Big\| P_0 + \sum_{i=1}^m  P_i\Big\|_{\op}\ge \tr_N\Big(P_0 + \sum_{i=1}^m P_i\Big) \ge \tr_N\Big(\sum_{i=1}^m P_i\Big),
    \end{align*}
    where the last equality follows since the operator norm is unitarily invariant and the last two inequalities follow from the positive semidefiniteness of the $P_i$'s.

    For every positive semidefinite matrix $P$, we have that $\tr_N(P) \ge {\tr_N(P^2)}/{\|P\|_{\op}}$. 
  
    Hence,
     \[ \|p(V_1, \ldots, V_m, W_1, \ldots, W_t)\|_{\op} \ge \sum_{i=1}^m \frac{\tr_N(P_i^2)}{\|P_i\|_{\op}}.\]
     By our choice of $M_i$, we have that $\tr_N(P_i^2) = \tr_N(M_i^* M_i) \ge \|q_i\|_2^2 - \eps$ and $\|P_i\|_{\op} = \|M_i\|_{\op} \le \sqrt{e}(d+1)\|q_i\|_2 + \eps$. Since $\eps$ can be made arbitrarily small by increasing $N$, it follows that 
      \[ \|p(V_1, \ldots, V_m, W_1, \ldots, W_t)\|_{\op} \ge \frac1{\sqrt{e}(d+1)} \sum_{i=1}^m \|q_i\|_2 - \delta ,\]
     for large enough $N$. The improved bound for the homogeneous case follows directly by plugging the bound of \thmref{thm:op-norm} into the above proof.
\end{proof}





\subsection{Non-commutative root-influence inequality}
\label{sec:bh-proof}


For clarity in the proofs below, we remind our  convention that all tuples or blocks are denoted with boldface fonts (e.g. $\BU_1$ or $\BA$), while a single element is denoted without boldface (e.g. $U_1(i)$ or $A_i$ or $A$). Before proceeding with the proof, we restate the statement for convenience.

\bh*





\begin{proof}[Proof of \thmref{thm:bh-intro}] 
Since $f$ is homogeneous, we can write
   \begin{align*}
    f(\x_1,\ldots, \x_d) &= \sum_{i_1, \ldots, i_d \in [n]} \hf_{i_1, \ldots, i_d} ~x_1(i_1)x_2(i_2)\cdots x_d({i_d}) \\
    \ & = \sum_{i=1}^n  x_1(i) \underbrace{\left(\sum_{i_2,\ldots, i_d \in [n]} \hf_{i_1, \ldots, i_d} ~x_2(i_2)\cdots x_d({i_d})\right)}_{\textstyle := f_i(\x_2,\ldots, \x_d)}.
\end{align*}
 In this case, it follows from \eqref{eqn:inf-tensor} that for each $i \in [n]$, we have 
 \begin{equation}\label{eqn:var}
     \ \Var[f_i] = \|f_i\|^2_2 = \inf_{1,i}(f) \text{ and }  \Var[f] = \sum_{i=1}^n \inf_{1,i}(f).
 \end{equation}

  Let us denote the corresponding non-commutative block-multilinear polynomials by $f(\BU_1, \ldots, \BU_d)$ and $f_i(\BU_2, \ldots,\BU_d)$ where $\BU_b = (U_b(1), \ldots, U_b(n))$ denotes the $b^\text{th}$ block of non-commutative variables. To show a lower bound on $\cbnorm{f}$ it suffices to exhibit a collection of square matrices $\{U_b(i)\}_{b\in [d], i \in [n]}$ with operator norm at most~1, such that $\|f(\BU_1, \ldots, \BU_d)\|_{\op}$ is large. 
  
%  

Applying \lref{lem:polar} for the homogeneous case (with $p = f$, $q_i=f_i$ for $i \in [n]$, and $q_0=0)$, it follows that for every $\delta > 0$ there exists an integer $N$ and a choice of tuples of $N \times N$ unitaries $\BU_1, \ldots, \BU_d$ such that  
      \[ \cbnorm{f} \ge \|f(\BU_1, \ldots, \BU_d)\|_{\op} \ge \frac1{\sqrt{e(d+1)}} \sum_{i\in [n]} \|f_i\|_2  -\delta \stackrel{\eqref{eqn:var}}{\ge}  \frac{1}{\sqrt{e(d+1)}} \left(\sum_{i=1}^n \sqrt{\Inf_{1,i}(f)} \right) -\delta.\]
Taking $\delta \to 0$, we get the statement of the lemma. The proof for the inequality when $b=d$ is the last block follows similarly by using the right polar decomposition.
\end{proof}

\subsection{Aaronson-Ambainis Conjecture for non-homogeneous forms}

In this section, we prove \thmref{thm:aa}, which requires handling non-homogeneous forms. The proof will be similar to the proof of \thmref{thm:bh-intro} but we will need to be careful about certain details. 

\begin{proof}[Proof of \thmref{thm:aa}]
Any block-multilinear polynomial $f(x_1, \ldots, x_d)$ can be written as 
\begin{align*}
    f(\x_1,\ldots, \x_d) &= \BE f + \sum_{b\in [d]} f_b(\x_b, \x_{b+1}, \ldots, \x_d),
\end{align*}
where $f_b$ consists of all monomials of $f$ that start with a variable in the $b^\text{th}$ block $\x_b$. Note that $f_b$ depends only on the variables in blocks $\x_b, \x_{b+1},\ldots, \x_d$. Moreover, it follows from \eqref{eqn:inf-tensor} that 
 \begin{equation}\label{eqn:var-general}
     \ \Var[f] = \sum_{b \in [d]} \|f_b\|_2^2 = \sum_{b \in [d]} \Var[f_b],
 \end{equation}
so there exists a block $\beta \in [d]$ such that $\Var[f_{\beta}] \ge \frac{1}{d}\Var[f]$. 

Since $f_{\beta}$ contributes a lot to the variance, it is natural to try to find an influential variable in the block $\x_{\beta}$. Towards this end,  we pull out the variables $x_{\beta}(i)$ and write
\begin{align*}
    f_{\beta}(\x_{\beta},\ldots, \x_d) &= \sum_{i\in [n]} x_{\beta}(i) f_{\beta,i}(\x_{\beta+1}, \ldots, \x_d),
\end{align*}
for block-multilinear polynomials $f_{\beta,i}(\x_{\beta+1}, \ldots, \x_d)$. Note that some of the $f_{\beta,i}$'s could be identically zero, so let us define $S$ to be the set of those $i$ such that $f_{\beta,i}$ is non-zero. We note that
\begin{align} \label{eqn:part-inf}
  \|f_{\beta,i}\|_2^2  =  \Inf_{\beta,i}(f_{\beta}) \le \Inf_{\beta,i}(f)  
\end{align}
which implies that
\begin{align}\label{eqn:var-main}
    \frac{1}{d} \Var[f] \le \Var[f_{\beta}] = \sum_{i \in S}\|f_{\beta,i}\|_2^2 = \sum_{i \in S} \Inf_{\beta,i}(f_{\beta}).
\end{align}
\begin{sloppypar}
Denote the corresponding non-commutative block-multilinear polynomials by $f(\BU_1, \ldots, \BU_d)$,  $f_b(\BU_{b}, \ldots,\BU_d)$, and $f_{\beta}(\BU_{\beta+1}, \ldots,\BU_d)$ where $\BU_b = (U_b(1), \ldots, U_b(n))$ denotes the $b^\text{th}$ block of non-commutative variables. To show a lower bound on $\cbnorm{f}$ it suffices to exhibit a collection of square matrices $\{U_b(i)\}_{b\in [d], i \in [n]}$ with operator norm at most~1 such that $\|f(\BU_1, \ldots, \BU_d)\|_{\op}$ is large.
\end{sloppypar}
  
 We set the matrices in blocks $\BU_1, \ldots, \BU_{\beta-1}$ to be zero (that is, the all-zero matrix $\BZ$). Note that with this choice all polynomials $f_b(\U_b, \ldots, \U_d)$ where $b < \beta$ vanish and the non-commutative polynomial becomes 
 \[ f(\BZ, \ldots, \BZ, \BU_{\beta}, \BU_{\beta+1}, \ldots, \BU_d) = \sum_{i\in S} U_{\beta}(i) f_{\beta,i}(\BU_{\beta+1}, \ldots, \BU_d) + \sum_{b=\beta+1}^d f_b(\BU_b, \BU_{b+1}, \ldots, \BU_d) + \Ef,\]
  which is a non-commutative polynomial of the form considered in \lref{lem:polar} (with $m = |S|$, $q_i = f_{\beta,i}$ and $q_0 = \sum_{b=\beta+1}^d f_b + \Ef$). Thus, by \lref{lem:polar} for every small $\delta>0$ there exists an integer $N$ and a choice of $N \times N$ matrices for the blocks $\BU_{\beta},\ldots, \BU_d$ such that 
        \begin{align*}
             \ \cbnorm{f} & \ge \|f(\BZ, \ldots, \BZ, \BU_{\beta}, \BU_{\beta+1}, \ldots, \BU_d)\|_{\op} & \\
             \  & \ge \frac1{\sqrt{e}(d+1)} \sum_{i\in S} \|f_{\beta,i}\|_2 -\delta  \stackrel{\eqref{eqn:part-inf}}{=}  \frac{1}{\sqrt{e}(d+1)} \left(\sum_{i \in S} \sqrt{\Inf_{\beta,i}(f_{\beta})} \right) -\delta & \\
             \ &\stackrel{\eqref{eqn:var-main}}{\ge}  \frac{1}{\sqrt{e}(d+1)} \left( \frac{\sum_{i \in S} \Inf_{\beta,i}(f_{\beta})}{\sqrt{\maxinf(f)}} \right) -\delta  \stackrel{\eqref{eqn:part-inf}}{\ge}  \frac{1}{\sqrt{e}(d+1)^{2}} \left( \frac{\Var[f]}{ \sqrt{\maxinf(f)}} \right) -\delta
        \end{align*}
        Taking $\delta \to 0$ and using the assumption that $\|f\|_{\cb} \le 1$, we obtain the statement of the theorem:
     \[
     1\geq \cbnorm{f} \ge \frac{1}{\sqrt{e}(d+1)^{2}} \cdot \frac{\Var[f]}{\sqrt{\maxinf(f)}} \implies \maxinf(f) \ge  \frac{(\Var[f])^2}{e(d+1)^4}. \qedhere
     \]
\end{proof}
 

     
     

\subsection{Approximating completely bounded forms with decision trees}



In this section, we briefly mention how to obtain \corref{cor:sim}.
Aaronson and Ambainis \cite[Theorem 3.3]{AA14} showed that querying the most influential variable reduces the variance of the function~$f$, and if that influence is lower bounded by a polynomial in $\Var[f]/d$, then after $\poly(d)$ queries (the exact quantitative dependence can be read off from their proof), the variance of the function becomes small enough so that it can be approximated almost-everywhere by its expectation.  Since the family of degree-$d$ block-multilinear forms with completely bounded norm at most one is closed under restrictions, one can apply \thmref{thm:aa} repeatedly. This gives us \corref{cor:sim}.

%	\section*{Acknowledgements}
%	None!

\bibliography{refs}
\bibliographystyle{abbrv}


\end{document}
