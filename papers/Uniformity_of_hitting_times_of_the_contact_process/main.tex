\documentclass[11pt,a4paper,reqno]{amsart}
% ====================================================================
% Packages
% ====================================================================
\usepackage[utf8]{inputenc} % allow utf-8 input
\usepackage[T1]{fontenc}    % use 8-bit T1 fonts
\usepackage{hyperref}       % hyperlinks
\usepackage{url}    % simple URL typesetting
\usepackage{booktabs}       % professional-quality tables
\usepackage{amsfonts}       % blackboard math symbols
\usepackage{nicefrac}       % compact symbols for 1/2, etc.
\usepackage{microtype}      % microtypography
\usepackage{lineno}
\usepackage{natbib}
\usepackage{pifont}
\usepackage{fleqn}
\usepackage{amsmath,amssymb}
\usepackage{graphicx}
\usepackage{multicol,graphicx,xcolor}
\usepackage{tikz}
\usepackage{verbatim}
\usepackage{bm}
\usetikzlibrary{shapes.misc, positioning}
\usetikzlibrary{matrix, calc, positioning, arrows,shapes,backgrounds, arrows.meta, quotes}
\usepackage{gensymb,stmaryrd,mathtools,textcomp,xspace,grffile}
\usetikzlibrary{shapes.geometric,fit,matrix,positioning,shapes.multipart}
\usetikzlibrary{decorations.markings}
\usetikzlibrary{shadows,decorations.pathreplacing,fadings}
\pgfdeclarelayer{background}
\pgfsetlayers{background,main}
\usetikzlibrary{topaths, calc,3d}
\usetikzlibrary{fit}
\usepackage{color}

\usepackage{bm}		% Bold maths symbols, including upright Greek
\usepackage{pdflscape}	% Landscape pages
\usepackage{ae,aecompl}
%\usepackage{newtxtext}
%\usepackage{newtxmath}
\usepackage[section]{placeins}
\usepackage{float}
\usepackage[normalem]{ulem}
\usepackage[toc,page]{appendix}
\usepackage{longtable}

\usepackage[normalem]{ulem}

\usepackage{tabularx}
    \newcolumntype{L}{>{\raggedright\arraybackslash}X}

\usepackage{caption}
\usepackage{subcaption}

\newcommand{\marginnote}[1]{\marginpar{\framebox{\framebox{#1}}}}
% Comment macro: Usage \comment{Author}{Comment}
\newcommand{\comment}[2]{[\marginnote{#1}\textit{#1}:\ \textit{#2}]}

% \newcommand{\bra}[1]{#1^H}
% \newcommand{\ket}[1]{#1}
% \newcommand{\innerp}[2]{#1^H#2}
% \newcommand{\outerp}[2]{#1#2^H}
% \newcommand{\matrixel}[3]{#1^H#2#3}
% \newcommand{\proj}[1]{\outerp{#1}{#1}}
% \newcommand{\abs}[1]{\vert#1\vert}
% 
% 
% %norms
% \newcommand{\norm}[2]{\left\Vert#1\right\Vert_{#2}}
% 
% %a matrix or vector
% \newcommand{\mat}[2]{\lefto[\begin{array}{#1}#2\end{array}\right]}
% 
% %mathbf etc.
% \newcommand{\mbf}{\mathbf}
% \newcommand{\mbb}{\mathbb}
% \newcommand{\mcl}{\mathcal}
 \newcommand{\trm}{\textrm}
% 
% 
% %Theorems, etc.
% %\theoremstyle{definition}
% \theoremstyle{plain}
%
% 
% %Expectation
% \newcommand{\expect}[1]{\langle#1\rangle}
% 
% %Trace
% \DeclareMathOperator{\tr}{tr}
% \DeclareMathOperator{\Tr}{Tr}
% 
% %Miscellaneous stuff
% \DeclareMathOperator{\supp}{supp}
% \DeclareMathOperator{\Span}{span}
% 
% %identity
% \DeclareMathOperator{\id}{\mathbf{1}}
% 
% %BS addition
% \DeclareMathOperator{\bst}{\stackrel{\land}{+}}
% \DeclareMathOperator{\bs}{\widehat{+}}
% 
% %ladder operators
% \newcommand{\up}{a^\dagger}
% \newcommand{\down}{a}
% 
% %vec operator
% %\DeclareMathOperator{\vect}{vec}
% 
% %equations
% \def\ba#1\ea{\begin{align*}#1\end{align*}}
% %with numeration
% \def\ban#1\ean{\begin{align}#1\end{align}}
% % the same, centered
% \def\bac#1\eac{\vspace{\abovedisplayskip}{\par\centering$\begin{aligned}#1\end{aligned}$\par}\addvspace{\belowdisplayskip}}

%%%%% parentheses denoting arguments, we want an opening object. Use \lefto in
%%%%% these cases.
% \newcommand{\lefto}{\mathopen{}\left}
% 
% %real and imaginary part
% \renewcommand{\Re}{\mathfrak{Re}}
% \renewcommand{\Im}{\mathfrak{Im}}
% 
% % in case you need to break equations over several lines
 \newcommand{\dummyrel}[1]{\mathrel{\hphantom{#1}}\strut\mskip-\medmuskip}
%%%%%%%%%%%%%%%%%%%%%
%:- Theorems
 \newtheorem{thm}{Theorem}
 \newtheorem{cor}[thm]{Corollary}   % Turned off theorem numbering
 \newtheorem{prop}{Proposition}
 \newtheorem{defi}{Definition}
 \newtheorem{lem}[thm]{Lemma}
 \newtheorem{exmpl}{Example}
% \newtheorem{st}{Statement}
% \newtheorem{conj}{Conjecture}

%%%%%
%% robust recovery from sparse noise
\safemath{\dictab}{[\,\dicta\,\,\dictb\,]}


\safemath{\ysig}{\bmy}
\safemath{\ysighat}{\hat{\ysig}}
\safemath{\ysigdim}{M}
%
\safemath{\xsig}{\bmx}
\safemath{\xsigdim}{N}
\safemath{\nx}{n_x}
%
\safemath{\zsig}{\bmz}
\safemath{\zsigdim}{\ysigdim}
%
\safemath{\rsig}{\bmr}
%
\safemath{\Adict}{\bA}
\safemath{\Adicttilde}{\widetilde{\Adict}}
\safemath{\Adictdim}{\outputdim\times\xsigdim}
\safemath{\avec}{\bma}
\safemath{\avectilde}{\tilde{\avec}}
%
\safemath{\Bdict}{\bB}
\safemath{\Bdicttilde}{\widetilde{\Bdict}}
%
\safemath{\Cdict}{\bC}
\safemath{\cvec}{\bmc}
%
\safemath{\Ddict}{\bD}
\safemath{\Ddictdim}{\ysigdim\times\xsigdim}
\safemath{\dvec}{\bmd}
\safemath{\Ddicttilde}{\widetilde{\bD}}
%
\safemath{\Bonb}{\bB}
\safemath{\bvec}{\bmb}
\safemath{\Bonbdim}{\ysigdim\times\ysigdim}
%
\safemath{\noise}{\bmn}
\safemath{\noisedim}{\ysigim}
%
\safemath{\err}{\bme}
\safemath{\errdim}{\ysigdim}
\safemath{\errset}{\setE}
\safemath{\nerr}{n_e}
%
\safemath{\delop}{\bP_\errset}
\safemath{\delopc}{\bP_{{\errset}^c}}

%




%%
% Complex i and j 
\safemath{\cplxi}{\imath}
\safemath{\cplxj}{\jmath}
% Comb signal
%\safemath{\comb}{\matI\matI\matI}
\newcommand{\comb}[1]{\vecdelta_{#1}}
%:- Definition dictionary
\safemath{\dict}{\matD}
\safemath{\inputdim}{N}		% number of columns of dictionary D
\safemath{\outputdim}{M}		%number of rows of dictionary D
\safemath{\sparsity}{S}	%sparsity
\safemath{\inputdimA}{{N_a}}	%total number of elements in dictionary A
\safemath{\inputdimB}{{N_b}}	%total number of elements in dictionary B
\safemath{\elemA}{{n_a}}	%number of elements chosen from dictionary A
\safemath{\elemB}{{n_b}}	%number of elements chosen from dictionary B
\safemath{\resA}{\matR_a}	%restriction map to elements of dictionary A
\safemath{\resB}{\matR_b}	%restriction map to elements of dictionary B
\safemath{\subD}{\matS} %subdictionary
\safemath{\subA}{\matS_a} %subdictionary part of A
\safemath{\subB}{\matS_b} %subdictionary part of B
\safemath{\dicta}{\matA} 	% first subdictionary
\safemath{\dictb}{\matB} 	% second subdictionary
\safemath{\hollowS}{H}
\safemath{\hollowA}{H_a}
\safemath{\hollowB}{H_b}
\safemath{\cross}{Z}
\safemath{\coh}{\mu_d}			% coherence dictionary
\safemath{\coha}{\mu_a}			% coherence first subdictionary
\safemath{\cohb}{\mu_b}			% coherence second subdictionary
\safemath{\mubs}{\nu}	%block sub-coherence
\safemath{\cohm}{\mu_m} %mutual coherence
\safemath{\dictset}{\setD}	% set of dictionaries
\safemath{\dictsetp}{\dictset(\coh,\coha,\cohb)}	% set of dictionaries parametrized
\safemath{\dictsetgen}{\dictset_\text{gen}}
\safemath{\dictsetgenp}{\dictsetgen(\coh)}
\safemath{\dictsetonb}{\dictset_\text{onb}}
\safemath{\dictsetonbp}{\dictsetonb(\coh)}

\safemath{\leftside}{U}
\safemath{\rightsideA}{R_a}
\safemath{\rightsideB}{R_b}

\safemath{\indexS}{\setI_S} %set of indices participating in sub-dictionary S


\safemath{\na}{n_a}			% cardinality of set of linearly independent columns of first dictionary
\safemath{\nb}{n_b}			% cardinality of set of linearly independent columns of second dictionary
\safemath{\coeffa}{p_i}	%coefficients for columns of A
\safemath{\coeffb}{q_j}	%coefficients for columns of B
\safemath{\seta}{\setP}		% set of linearly independent columns of A
\safemath{\setb}{\setQ}     % set of linearly independent columns of B
\safemath{\setw}{\setW}	%set of n largest elements of w
\safemath{\setz}{\setZ}	%set of L-n largest elements of z
\safemath{\cola}{\veca}		% generic element of the dictionary A
\safemath{\colb}{\vecb}		% generic element of the dictionary B
\safemath{\cold}{\vecd}		% generic element of the dictionary D
\safemath{\inputvec}{\vecx} 	%coefficient vector (input)
\safemath{\error}{\vece}	%error vector
\safemath{\noiseout}{\vecz} 	%noisy output vector
\safemath{\inputvecel}{x}
\safemath{\inputveca}{\vecx_a}
\safemath{\inputvecb}{\vecx_b}
\safemath{\outputvec}{\vecy}	%output of Dictionary
\safemath{\lambdamin}{\lambda_{\mathrm{min}}}
%:- Math operators
\DeclareMathOperator{\spark}{spark}
\newcommand{\pos}[1]{\lefto[#1\right]^+}
\newcommand{\normtwo}[1]{\vecnorm{#1}_2}
\newcommand{\normone}[1]{\vecnorm{#1}_1}
\newcommand{\normzero}[1]{\vecnorm{#1}_0}
\newcommand{\norminf}[1]{\vecnorm{#1}_\infty}
\newcommand{\norminftilde}[1]{\vecnorm{#1}_{\widetilde\infty}}
\newcommand{\normfro}[1]{\vecnorm{#1}_F}
%\newcommand{\spectralnorm}[1]{\vecnorm{#1}_{2,2}}
\newcommand{\spectralnorm}[1]{\vecnorm{#1}}
\safemath{\elltwo}{\ell_2}
\safemath{\ellone}{\ell_1}
\safemath{\ellzero}{\ell_0}
\safemath{\ellinf}{\ell_\infty}
\safemath{\ellinftilde}{\ell_{\widetilde\infty}}
\safemath{\licard}{Z(\coh,\coha,\cohb)}
\safemath{\xsol}{\hat{x}}
\safemath{\xbord}{x_b}		%Solution at the border
\safemath{\xstat}{x_s}		%Solution stationary in l0 prob
\safemath{\xstatLone}{\tilde{x}_s}
\safemath{\order}{\mathcal{O}} %order notation (big O)
\safemath{\scales}{\Theta} %scales as
%
\safemath{\ones}{\mathbf{1}} %all ones matrix
\safemath{\zeroes}{\mathbf{0}} %all zeroes matrix
%
\safemath{\thlone}{\kappa(\coh,\cohb)} %treshold l1 problem
\safemath{\constoneA}{\delta} %constant in l1 theorem to save space
\safemath{\constoneB}{\epsilon} %constant in l1 theorem to save space
\safemath{\nlarge}{L}				   %num large elements
\safemath{\sumlarge}{S_\nlarge}
\DeclareMathOperator{\kernel}{kern}	   % kernel of a matrix
\safemath{\maxlarger}{P_\nlarge}	   % maximum in Gribonval and Nielsen
\safemath{\Pzero}{\textrm{P0}}	
\safemath{\Pone}{\textrm{P1}}
\safemath{\vecfir}{\vecw}			 % \vecv element of the kernel of the dictionary, \vecv=[\vecfir \vecsec]
\safemath{\vecsec}{\vecz}
\safemath{\elvecfir}{w}              % element of vecfir
\safemath{\elvecsec}{z}				 % element of vecsec
\safemath{\nlargefir}{n}
\safemath{\normout}{\gamma}
\safemath{\auxfun}{h}
\safemath{\supp}{\textrm{supp}}%support

\safemath{\indexa}{\ell}
\safemath{\indexb}{r}
\safemath{\indexc}{i}
\safemath{\indexd}{j}


\safemath{\project}{P}%projector
\newtheorem{theorem}{Theorem}[section]
%[section] means numbering within sections
\newtheorem{corollary}[theorem]{Corollary}
\newtheorem{approximation}[theorem]{Approximation}
\newtheorem{result}[theorem]{Result}
%[theorem] means sharing numbering system with theorem-environment
\newtheorem{lemma}[theorem]{Lemma}
\newtheorem{proposition}[theorem]{Proposition}
\theoremstyle{definition}
\newtheorem{definition}[theorem]{Definition}
\newtheorem{note}[theorem]{Note}
\newtheorem{remark}[theorem]{Remark}
\newtheorem{conjecture}[theorem]{Conjecture}
\newtheorem{example}[theorem]{Example}
\newtheorem{counterexample}[theorem]{Counterexample}
\newtheorem{setup}[theorem]{Setup}




\keywords{Contact process, coexistence, hitting times}
\subjclass[2010]{60K35; 82C22}


\begin{document}

\author{Markus Heydenreich}
%\address[Markus Heydenreich, Christian Hirsch]{Mathematisches Institut, Ludwig-Maximilians-Universit\"at M\"unchen, Theresienstra\ss e 39, 80333 M\"unchen, Germany}
%\email{m.heydenreich@lmu.de, christian.hirsch@lmu.de}
\author{Christian Hirsch}
\address[Markus Heydenreich, Christian Hirsch]{Mathematisches Institut, Ludwig-Maximilians-Universit\"at M\"unchen, Theresienstra\ss e 39, 80333 M\"unchen, Germany}
\email{m.heydenreich@lmu.de,christian.hirsch@lmu.de}
\author{Daniel Valesin}
\address[Daniel Valesin]{Johann Bernoulli Institute, University of Groningen, Nijenborgh 9, 9747AG Groningen, The Netherlands}
\email{D.Rodrigues.Valesin@rug.nl}



\thanks{The work of CH was funded by LMU Munich's Institutional Strategy LMUexcellent within the framework of the German Excellence Initiative.}



\title{Uniformity of hitting times of the contact process}

\date{\today}

\begin{abstract}
For the supercritical contact process on the hyper-cubic lattice started from a single infection at the origin and conditioned on survival, we establish two uniformity results for the hitting times $t(x)$, defined for each site $x$ as the first time at which it becomes infected. First, the family of random variables $(t(x)-t(y))/|x-y|$, indexed by $x \neq y$ in $\mathbb{Z}^d$, is stochastically tight. Second, for each $\varepsilon >0$ there exists $x$ such that, for infinitely many integers $n$, $t(nx) < t((n+1)x)$ with probability larger than $1-\varepsilon$.  A key ingredient in our proofs is a tightness result concerning the essential hitting times of the supercritical contact process introduced by Garet and Marchand (Ann.\ Appl.\ Probab., 2012).\end{abstract}

\maketitle


	% !TEX root = ../arxiv.tex

Unsupervised domain adaptation (UDA) is a variant of semi-supervised learning \cite{blum1998combining}, where the available unlabelled data comes from a different distribution than the annotated dataset \cite{Ben-DavidBCP06}.
A case in point is to exploit synthetic data, where annotation is more accessible compared to the costly labelling of real-world images \cite{RichterVRK16,RosSMVL16}.
Along with some success in addressing UDA for semantic segmentation \cite{TsaiHSS0C18,VuJBCP19,0001S20,ZouYKW18}, the developed methods are growing increasingly sophisticated and often combine style transfer networks, adversarial training or network ensembles \cite{KimB20a,LiYV19,TsaiSSC19,Yang_2020_ECCV}.
This increase in model complexity impedes reproducibility, potentially slowing further progress.

In this work, we propose a UDA framework reaching state-of-the-art segmentation accuracy (measured by the Intersection-over-Union, IoU) without incurring substantial training efforts.
Toward this goal, we adopt a simple semi-supervised approach, \emph{self-training} \cite{ChenWB11,lee2013pseudo,ZouYKW18}, used in recent works only in conjunction with adversarial training or network ensembles \cite{ChoiKK19,KimB20a,Mei_2020_ECCV,Wang_2020_ECCV,0001S20,Zheng_2020_IJCV,ZhengY20}.
By contrast, we use self-training \emph{standalone}.
Compared to previous self-training methods \cite{ChenLCCCZAS20,Li_2020_ECCV,subhani2020learning,ZouYKW18,ZouYLKW19}, our approach also sidesteps the inconvenience of multiple training rounds, as they often require expert intervention between consecutive rounds.
We train our model using co-evolving pseudo labels end-to-end without such need.

\begin{figure}[t]%
    \centering
    \def\svgwidth{\linewidth}
    \input{figures/preview/bars.pdf_tex}
    \caption{\textbf{Results preview.} Unlike much recent work that combines multiple training paradigms, such as adversarial training and style transfer, our approach retains the modest single-round training complexity of self-training, yet improves the state of the art for adapting semantic segmentation by a significant margin.}
    \label{fig:preview}
\end{figure}

Our method leverages the ubiquitous \emph{data augmentation} techniques from fully supervised learning \cite{deeplabv3plus2018,ZhaoSQWJ17}: photometric jitter, flipping and multi-scale cropping.
We enforce \emph{consistency} of the semantic maps produced by the model across these image perturbations.
The following assumption formalises the key premise:

\myparagraph{Assumption 1.}
Let $f: \mathcal{I} \rightarrow \mathcal{M}$ represent a pixelwise mapping from images $\mathcal{I}$ to semantic output $\mathcal{M}$.
Denote $\rho_{\bm{\epsilon}}: \mathcal{I} \rightarrow \mathcal{I}$ a photometric image transform and, similarly, $\tau_{\bm{\epsilon}'}: \mathcal{I} \rightarrow \mathcal{I}$ a spatial similarity transformation, where $\bm{\epsilon},\bm{\epsilon}'\sim p(\cdot)$ are control variables following some pre-defined density (\eg, $p \equiv \mathcal{N}(0, 1)$).
Then, for any image $I \in \mathcal{I}$, $f$ is \emph{invariant} under $\rho_{\bm{\epsilon}}$ and \emph{equivariant} under $\tau_{\bm{\epsilon}'}$, \ie~$f(\rho_{\bm{\epsilon}}(I)) = f(I)$ and $f(\tau_{\bm{\epsilon}'}(I)) = \tau_{\bm{\epsilon}'}(f(I))$.

\smallskip
\noindent Next, we introduce a training framework using a \emph{momentum network} -- a slowly advancing copy of the original model.
The momentum network provides stable, yet recent targets for model updates, as opposed to the fixed supervision in model distillation \cite{Chen0G18,Zheng_2020_IJCV,ZhengY20}.
We also re-visit the problem of long-tail recognition in the context of generating pseudo labels for self-supervision.
In particular, we maintain an \emph{exponentially moving class prior} used to discount the confidence thresholds for those classes with few samples and increase their relative contribution to the training loss.
Our framework is simple to train, adds moderate computational overhead compared to a fully supervised setup, yet sets a new state of the art on established benchmarks (\cf \cref{fig:preview}).

	% !TEX root = main.tex

\section{Proof of Proposition~\ref{propEssTightness}}

%RECALL PROBLEM
%BIG IDEA: LEVEL-CROSSING UNDER UNCONDITIONAL MEASURE
The challenge of Proposition~\ref{propEssTightness} is to establish bounds on the defects $\sigma(x) - \t(x)$ for sites $x$ far away from the origin.
The essence of the proof of Proposition~\ref{propEssTightness} is a carefully devised level-crossing argument leveraging the renewal structure of standard hitting times under the unconditioned measure $\P$. To achieve this goal, we introduce variants of the times $\{u_n, v_n\}_{n\ge1}$ from Section~\ref{essHitSec} that are tailor-made for the purpose of our proof.

%TAKE LARGE BALL AROUND x
%IF AT LEAST ONE SHELL GENERATES SURVIVING INFECTION => USE GM
More precisely, after fixing $L$ as in the statement of the proposition, we consider the ball 
$$B(x, R^2) = \{y \in \Zd:\, |y - x| \le R^2 \}$$
of radius $R^2$ around $x$, where $R$ is large and depends on $L$, but not on $x$. We pay special attention to the times at which the  infection hits each of the  $R$ shells 
$$\partial B(x,iR) = \{y \in \Zd: |y-x| = iR \},\quad 1\le i \le R. $$
If for some $i$ the space-time point where the infection hits the $i$th shell for the first time gives rise to a surviving contact process, then, by a renewal argument, it suffices to bound the essential hitting time of a point at distance at most $R^2$ from the origin. In particular, for this it is sufficient to invoke earlier location-dependent upper bounds.

%RENEWAL => PROBABILITY THAT ALL INFECTIONS DIE OUT DECAYS EXPONENTIALLY
The alternative is that upon hitting each of the $R$ shells, the infection started from the hitting point dies out. In the super-critical regime, a further renewal argument shows that this is an event of probability decaying geometrically in the number of shells. 


%RENEWAL REQUIRES RAPID DEATH OF NON-SURVIVING INFECTIONS
We clearly need a due amount of care in order to make the renewal argument rigorous. In particular, it is important to know that the infections starting from the hitting points die out before the next shell is reached.

Due to the regenerative arguments sketched above, we need the location-dependent bounds from~\cite[Theorem 15]{GaretMarch14} not only if the infection is started from the origin, but from an arbitrary set $A \subseteq \Zd$ with $o \in A$. More precisely, we now put $u^A_0(x) = v^A_0(x) = 0$ and
\[u^A_{k+1}(x)=\inf\{t\ge v^A_k(x):x\in\xi^A_t\},\]
whereas the recursion for $v^A_{k+1}$ is unchanged, i.e., 
\[v^A_{k+1}(x)=\sup\{s\ge0:\; (x,u^A_{k+1}(x)) \rightsquigarrow \Zd \times \{s\} \}. \]
Finally, we set $\sigma^A(x)=u_{K^A(x)}(x) $ where 
\[K^A(x)=\min\{n\ge0:v^A_n(x)=\infty\text{ or }u^A_n(x)=\infty\}.\]


%The proof is based on the shape theorem and the notion of essential hitting times.

\begin{lemma}
	\label{cor20}
There exist $c$, $c' > 0$ such that for all $L > 0$, $x \in \Zd$ and $A \subseteq \Zd$ with $o \in A$,
\begin{align}\label{eqcor20}
\oP\left(\sigma^A(x) > L\right) \leq \exp\{-cL + c'|x|\}.
\end{align}
\end{lemma}
%For $A = \{0\}$, this follows immediately from \cite[Corollary 20]{GaretMarch14}, the novelty being uniformity in the initial condition $A$. 
%In view of \cite[Corollary 20]{GaretMarch14}, it is plausible that $C$ may be chosen as a linear function of $\|x\|$; however, we do not need this to prove the (much stronger) bound of Proposition \ref{propEssTightness}. 
\begin{proof}
The result follows from Chebyshev's inequality and the claim that there exist $\beta, \gamma > 0$ such that, for all $x \in \Zd$ and all $A \subseteq \Zd$ with $o \in A$,
$$\oE(\exp\{\beta \sigma^A(x)\}) \leq \exp\{\gamma |x|\}. $$
	For the case $A = \{o\}$, this is \cite[Theorem 15]{GaretMarch14}. The statement with general $A \subseteq \Zd$ with $o \in A$ is proved in exactly the same way, so we do not go over the details.
\end{proof}


\begin{proof}[Proof of Proposition~\ref{propEssTightness}]
  Fix $L > 0$, which we may assume to be large throughout the proof without loss of generality. Define $R = \lfloor L^{1/4}\rfloor$. Let us first fix $x \in \Zd$ with $|x| \leq R^2$. Then, by Lemma~\ref{cor20},
\begin{equation}\oP(\sigma(x) - t(x) > L) \leq \oP(\sigma(x) > L) \stackrel{\eqref{eqcor20}}{\leq} \exp\left\{-cL + c'|x|\right\} < \exp\left\{-cL/2\right\}.\label{eq:zeroth_ing}\end{equation}

Now we fix $x$ with $|x| > R^2$ and aim to show that
\begin{equation}\label{eqDestBd}\P(\sigma(x) < \infty,\;\sigma(x) - t(x) > L) < C_0\exp\{-L^{\gamma_0}\}\end{equation}
for some $C_0,\gamma_0 > 0$; the desired bound then follows since $\oP(\sigma(x) < \infty) = 1$. As suggested in the outline of proof, we define the shell radius $r_i = R(R-i+1)$ and let
$$U_i = \inf\{t: \xi_t^o \cap B(x,r_i) \neq \varnothing\},\qquad i = 1,\ldots, R.$$
denote the first time that the infection hits the $i$th shell. On the event $\{U_i < \infty\}$, let $Z_i \in \partial B(x,r_i)$ denote the unique point of $\xi^o_{U_i} \cap B(x, r_i)$. Also on $\{U_i < \infty\}$, let
$$V_i = \inf\{s \geq U_i:\;(Z_i,U_i) \not\rightsquigarrow \Zd \times \{s\}\}$$
	denote the time at which  the infection starting from $(Z_i, U_i)$ dies out.

	Then there are three alternative scenarios under which the event $\{\sigma(x) < \infty,\;\sigma(x) - t(x) > L\}$ can occur: 
	\begin{enumerate}
		\item there exists at least one shell $\partial B(x, r_i)$ such that the infection generated by the first hitting point survives forever, i.e., $V_i = \infty$,
		\item for each $i \in \{1,\ldots, n\}$ the infection generated by the hitting point $(Z_i, U_i)$ dies out before the global infection reaches the $(i+1)$th shell, 
		\item one of these non-surviving infections lives longer than it takes the global infection to reach the next shell.
	\end{enumerate}
	More precisely,
\begin{equation}\label{eq:3terms}\begin{split}
\P(\sigma(x) < \infty,\; \sigma(x) - t(x) > L) &\leq \sum_{n=1}^R\P(U_n < \infty,\; V_n = \infty,\; \sigma(x) - t(x) > L) \\&+
 \P(U_1 < V_1 < U_2 < V_2 <\cdots < U_R < V_R < \infty)\\&
	+ \sum_{n=1}^{R-1} \P(U_{n+1} < V_{n} <\infty)
\end{split}\end{equation}
	We separately bound the three terms on the right-hand side. By repeated use of the Markov property, we have 
	\begin{equation}\P(U_1 < V_1 < U_2 < V_2 <\cdots < U_R < V_R < \infty)\leq(1-\rho)^R, \label{eq:first_ing}\end{equation} 
		where we recall that $\rho = \P_\lambda(o \rightsquigarrow \infty)$  denotes the survival probability. Next, for any $n$,
\begin{align}
&\P(U_n < \infty,\; V_n = \infty,\;\sigma(x) - t(x) > L) \leq \P(U_n < \infty,\;V_n = \infty,\;\sigma(x) - U_n > L) \nonumber\\
&=\sum_{y \in \partial B(x,r_n)}\; \sum_{\substack{A \subseteq \Zd,\\A\text{ finite}}} \P(U_n < \infty,\; Z_n = y,\; (y, U_n) \rightsquigarrow \infty,\; \xi^o_{U_n} = A,\; \sigma(x) - U_n > L)\nonumber\\
&= \sum_{y \in \partial B(x,r_n)}\; \sum_{\substack{A \subseteq \Zd,\\A \text{ finite}}} \P(U_n < \infty,\; Z_n = y,\;  \xi^o_{U_n} = A) \cdot \rho\cdot \oP(\sigma^{A-y}(x-y) > L)\nonumber\\
&\leq \rho \cdot \P(U_n < \infty) \cdot \exp\{-cL/2\},\label{eq:second_ing}
\end{align}
	the last inequality is obtained as in \eqref{eq:zeroth_ing} using $|x - y| \le R^2$. We now turn to the term inside the third sum on the right-hand side of \eqref{eq:3terms}. Since the contact process is in the super-critical regime, the survival time of a non-surviving infection started at a single site has exponential tails, cf.~\cite[Proposition 5]{GaretMarch12}, so that the Markov property yields
\begin{equation}\begin{split}\P(U_{n+1} < V_n < \infty) &\leq \P(V_n - U_n \in (\sqrt{R}, \infty)) + \P(U_{n+1} - U_n < 2\sqrt{R})\\
&\leq \exp\{-c\sqrt{R} \} + \P(\partial B(x,r_i) \times [0,2\sqrt{R}] \rightsquigarrow B(x,r_{i+1}) \times[0,2\sqrt{R}]).\end{split}\label{eq:third_ing}
\end{equation}
Since the propagation speed of the infection is at most linear outside an event of exponentially decaying probability ~\cite[Proposition 5]{GaretMarch12}, the second term on the right-hand side is less than
\begin{align}\nonumber&\#\partial B(x,r_i) \cdot 2d\lambda \cdot \int_0^{2\sqrt{R}} \max_{y \in \partial B(x,r_i)} \P\Big((y,s) \rightsquigarrow B(x,r_{i+1})\times [s,2\sqrt{R}] \Big)\; \mathrm{d}s
\\\nonumber&\leq \#\partial B(x,r_i) \cdot 2d\lambda \cdot 2\sqrt{R} \cdot \P\Big(\bigcup_{s \leq 2\sqrt{R}}\xi^y_s \nsubseteq B(y, R/2)\Big) \\&\leq \#\partial B(x,r_i) \cdot 2d\lambda \cdot 2\sqrt{R} \cdot \exp\{-c\sqrt{R}\}.\label{eq:third_ing0}
\end{align}
Combining the bounds \eqref{eq:3terms}, \eqref{eq:first_ing}, \eqref{eq:second_ing}, \eqref{eq:third_ing} and \eqref{eq:third_ing0} implies  \eqref{eqDestBd} and finishes the proof.
\end{proof}

	\subsection{Proofs of \Cref{sec:ergodicity-hmc}}


% \begin{proof}
%\end{proof}

\subsubsection{Proof of \Cref{theo:irred_harris} }
\label{sec:proof-crefth-harris_0}
We first prove  \eqref{theo:irred_harris_a}.  Under the assumption that $\F$ is twice continuously
  differentiable, it follows by a straightforward induction, that for
  all $h >0$ and $q \in \rset^d$, $p \mapsto
  \Phiverletq[h][k](q,p)$, defined by  \eqref{eq:def_Phiverletq}, and $p \mapsto \gperthmc[k](q,p)$, defined by \eqref{eq:def_gperthmc}, are
  continuously differentiable and for all $(q,p) \in \rset^d \times
  \rset^d$,
\begin{equation}
  \Jac_{p,\gperthmc[T]}(q,p) =  \sum_{i=1}^{T-1}(T-i)\defEns{\nabla^2 \F \circ \Phiverletq[h][i](q,p)} \Jac_{p,\Phiverletq[h][i]}(q,p) \eqsp,
\end{equation}
where for all $q \in \rset^d$, $\Jac_{p,\gperthmc[k]}(q,p)$ ($\Jac_{p,\Phiverletq[i][h]}(q,p)$ respectively) is the Jacobian of the function $\tilde{p} \mapsto
\gperthmc[k](q,\tilde{p})$ ($\tilde{p} \mapsto
\Phiverletq[i][h](q,\tilde{p})$ respectively) at $p \in \rset^{d}$.


%  Under \Cref{assum:regOne}, $\sup_{x \in \rset^d} \normLigne{\nabla^2 \F(x)}
% \leq \constzero$ and by
% \Cref{lem:bound_first_iterate_leapfrog_a},
%  $ \sup_{(q,p) \in \rset^d \times \rset^d} \normLigne{\nabla_p \Phiverletq[h][i](q,p)} \leq (1+h \vartheta_1(h))^i$ for any $i \in \nsets$.
% Therefore for any $h >0$, $T \in \nsets$, setting $S = hT$ and using that $\tilde{h} \mapsto \vartheta_1(\tilde{h})$ is nondecreasing and greater than $1$ on $\rset_+^*$ and for any $u,s \geq 0$, $u \geq 1$, $(1+s/u)^{u-1} \leq \log(s+1) \rme^s$, we get that
Under \Cref{assum:regOne}, $\sup_{x \in \rset^d} \normLigne{\nabla^2 \F(x)}
 \leq \constzero$, therefore by \Cref{lem:inverse_1}, we have that for any $T \in \nsets$ and $h >0$,
\begin{equation}
  \label{eq:inverse_1}
 \sup_{(\q,\p) \in \rset^d \times \rset^d} \norm{\Jac_{p,\gperthmc[T]}(q,p)}
 \leq  T (\{1 + h \constzero^{1/2} \vartheta_1(h \constzero^{1/2})\}^T  -1) /h  \eqsp.
\end{equation}
%$\sup_{p \in \rset^d } \nabla_p \gperthmc[k](q,p) \leq C$.
% \begin{equation}
% \label{lem:inverse_1}
% \sup_{p \in \rset^d } \nabla_p \gperthmc[k](q,p) \leq C \eqsp.
% \end{equation}
% Then for all $q, p_1,p_2 \in \rset^d$,
% \begin{equation}
% \label{lem:inverse_1}
% \norm{\gperthmc[k](q,p_1) -\gperthmc[k](q,p_2)} \leq C \norm{p_1 - p_2} \eqsp.
% \end{equation}
For any $q \in \rset^d$, $T\in \nsets$ and $h >0$, define $\phia_{q,T,h}(p)$  for all  $p \in \rset^d$ by
\begin{equation}
  \phia_{q,T,h} (p) = p-(h/T) \gperthmc[T](q,p) \eqsp.
\end{equation}
It is a well known fact (see for example
\cite[Exercise 3.26]{duistermaat:kolk:2004}) that if
\begin{equation}
  \label{eq:inverse_1_2}
  \sup_{(q,p) \in \rset^d \times \rset^d} (h/T)\norm{ \Jac_{p,\gperthmc[T]}(q,p)} < 1 \eqsp,
\end{equation}
then for any $q \in \rset^d$, $\phia_{q,T,h}$ is a
diffeomorphism and  therefore by \eqref{eq:qk}, the same conclusion holds
for $p \mapsto \Phiverletq[h][T](q,p)$. Using \eqref{eq:inverse_1}, if $T \in \nsets$ and $h > 0$  satisfies \eqref{eq:condition-h,T-harris},
then the condition \eqref{eq:inverse_1_2} is verified and as a result \eqref{theo:irred_harris_a}.

Denoting for any $q \in \rset^d$ by $\Phiverletqi[h][T](q,\cdot) : \rset^d \to \rset$ the
continuously differentiable inverse of $p \mapsto
\Phiverletq[h][T](q,p)$ and using a change of variable with $\Phiverletqi[h][T](q,\cdot)$ in \eqref{eq:def_kernel_hmc} concludes the proof of \eqref{eq:def_kernel_hmc_false_density}.

We now show that $\Tker_{h,T}$ satisfies the condition which implies that $\Pkerhmc[h][T]$ is a \Tkernel. We first establish some estimates on the function $(q,p) \mapsto \Phiverletqi[h][T](q,p)$. By
\eqref{eq:inverse_1_2} and \eqref{eq:qk}, for any $q,p,v \in \rset^d$, there exists $\varepsilon \in \ooint{0,1}$ such that $  \normLigne{\Phiverletq[h][T](q,p)-\Phiverletq[h][T](q,v)} \geq (hT) \normLigne{\phi_{q,T,h}(p)-\phi_{q,T,h}(v)} \geq (hT) (1-\varepsilon)\norm{p-v}$ which implies that that there exists $C \geq 0$ satisfying
\begin{equation}
  \label{eq:regularity_phinverse1}
  \begin{aligned}
    \norm{\Phiverletqi[h][T](q,p)-\Phiverletqi[h][T](q,v)} &\leq (1-\varepsilon)^{-1} \norm{v-p}\eqsp, \\
    \norm{  \Phiverletqi[h][T](q,p)} &\leq C\defEns{\norm{\p} + \norm{\Phiverletq[h][T](q,0)}} \eqsp.
  \end{aligned}
\end{equation}
In addition, for $q,x,p \in \rset^d$, we have setting $\tilde{q} = \Phiverletqi[h][T](q,p)$ that
\begin{align}
  \nonumber
  \normLigne{\Phiverletqi[h][T](q,p) - \Phiverletqi[h][T](x,p)} &= \normLigne{\tilde{q} - \Phiverletqi[h][T](x, \Phiverletq[h][T](q,\tilde{q}))} \\
  \nonumber
                                                                &= \normLigne{\Phiverletqi[h][T](x, \Phiverletq[h][T](x,\tilde{q})) - \Phiverletqi[h][T](x, \Phiverletq[h][T](q,\tilde{q}))} \eqsp,
\end{align}
which implies by \eqref{eq:regularity_phinverse1} and \Cref{lem:bound_first_iterate_leapfrog_a}
that there exists $C \geq 0$ satisfying
\begin{equation}
  \label{eq:regularity_phinverse2}
  \norm{\Phiverletqi[h][T](q,p) - \Phiverletqi[h][T](x,p)} \leq C \norm{q-x} \eqsp.
\end{equation}


We now can prove that $\Tker_{h,T}$ is the continuous component of $\Pkerhmc[h][T]$. First by \eqref{eq:def_tker}, for all $\eventB \in \borelSet(\rset^d)$,
\begin{equation}
\label{eq:minoration_pseudo_density_P}
    \Tker_{h,T}(q, \eventB) \geq (2 \uppi)^{-d/2} \Leb(\eventB)
 \times \inf_{\bar{q} \in \eventB} \defEns{ \balphaacc(q,\bar{q}) \rme^{-\norm{\Phiverletqi_q(\bar{q})}^2/2}\detj_{\Phiverletqi[h][T](q,\cdot)}(\bar{q})} \eqsp,
\end{equation}
with the convention $0 \times \plusinfty = 0$ and
\begin{equation}
%  \label{eq:7}
  \balphaacc(q,\bar{q}) =  \alphaacc\defEns{(q,\Phiverletqi[h][T](q,\bar{q})),\Phiverlet[h][T](q,\Phiverletqi[h][T](q,\bar{q}))}\eqsp. 
\end{equation}
Since the function $  (q,p) \mapsto (\Phiverletq[h][T](q,p),\Phiverletqi[h][T](q,p), \detj_{\Phiverletqi[h][T](q,\cdot)}(p)) $
is continuous on $\rset^d\times \rset^d$ by \Cref{lem:bound_first_iterate_leapfrog_a}, \eqref{eq:regularity_phinverse1} and \eqref{eq:regularity_phinverse2}, and for any $q,p \in \rset^d$, $\Jac_{\Phiverletq[h][T](\q,\cdot)}(\Phiverletqi[h][T](q,p))
\Jac_{\Phiverletqi[h][t](q,\cdot)}(\p) = \operatorname{I}_n$, we get that  $\Tker_{h,T}(q,\eventB) >0$ for all $q \in \rset^d$ and all compact set $\eventB$ satisfying $\Leb(\eventB) > 0$. Therefore, using that the Lebesgue measure is regular which implies that for any $\msa \in \mcb(\rset^d)$ with $\Leb(\msa) >0$, there exists a compact set $\msb \subset\msa$, $\Leb(\msb)>0$, we can conclude that $\Pkerhmc[h][T]$ is irreducible with respect to the Lebesgue measure. In addition, we get  $\Tker_{h,T}(q,\rset^d) >0$, and therefore we obtain that $\Pkerhmc[h][T]$ is aperiodic.  Similarly we get that any compact set is $(1,\Leb)$-small.

It remains to show that for any $\eventB \in\mcb(\rset^d)$, $q \mapsto \Tker_{h,T}(q,\eventB)$ is lower semi-continuous which is a straightforward consequence of Fatou's Lemma and that for any $p \in \rset^d$, $q \mapsto (\Phiverlet[h][T](q,p), \Phiverletqi[h][T](q,p),\detj_{\Phiverletqi[h][T](q,\cdot)}(p))$ is continuous.

% We now show that all the compact sets are $(1,\Leb)$-small. Let $\eventB \subset \rset^d$ be compact.  Using
% \eqref{eq:inverse_1_2} there exists $C \geq 0$ such that for all
% $q,p,v \in \rset^d$, $ \norm{p-v} \leq C \normLigne{
%   \Phiverletq[h][T](q,p)- \Phiverletq[h][T](q,v)}$. It follows
% that for all $p \in \rset^d$, $\sup_{q \in \eventB} \normLigne{
%   \Phiverletqi[h][T](q,p)} \leq C\defEnsLigne{\norm{\p} + \sup_{q \in
%     \eventB} \normLigne{\Phiverletq[h][T](q,0)}}$. Using this upper
% bound and $\Jac_{\Phiverletq(\q,\cdot)}(\Phiverletqi_q(\p))
% \Jac_{\Phiverletqi_q}(\tilde{\p}) = \operatorname{I}_n$ in
% \eqref{eq:minoration_pseudo_density_P}, where $\operatorname{I}_n$ is
% the identity matrix, we deduce that there exists $\varepsilon >0$ such
% that for all $\eventA\in \borelSet(\rset^d)$, $\eventA \subset \eventB$,
% \begin{equation}
%   \inf_{q \in \eventB} \Pkerhmc[h][T](q, \eventA)  \geq \varepsilon \Leb(\eventA) \eqsp,
% \end{equation}
% and therefore $\eventB$ is small for $\Pkerhmc[h][T]$.
% \begin{equation}
% \norm{  \Phiverletqi_q(p_1)-  \Phiverletqi_q(p_2)} \leq C \norm{p_1-p_2} \eqsp,
% \end{equation}
% This result, \eqref{eq:def_acc_ratio}, \Cref{lem:bound_first_iterate_leapfrog} and \eqref{eq:minoration_pseudo_density_P} imply that $
% \Pkerhmc[h][T]$ is irreducible with respect to the Lebesgue measure
% and aperiodic.
% and any ball on $\rset^d$ is small.

% A straightforward adaptation of the proof of \cite[Corollary
% 2]{tierney:1994} shows that $ \Pkerhmc[h][T]$ is Harris recurrent, see \Cref{propo:harris_rec} in \Cref{sec:harr-recurr-metr}. The desired conclusion then follows from \cite[Theorem 13.0.1]{meyn:tweedie:2009}.
 % \Cref{theo:irred_harris} implies
% that for all $T \geq 0$, there exists $\hirr>0$ such that for all $h \in \ocintLigne{0,\hirr}$ and all $\q \in \rset^d$
%   \begin{equation}
% \lim_{n \to \plusinfty}    \tvnorm{\delta_\q \Pkerhmc[h][T]^n - \pi} = 0 \eqsp.
%   \end{equation}


% By \cite[Theorem 17.1.4, Theorem
% 17.1.7]{meyn:tweedie:2009}, it suffices the to prove that for all
% bounded harmonic function $\harmonic : \rset^d \to \rset$ satisfying
% \begin{equation}
%   \label{eq:def_harm}
%   \Pkerhmc[h][T]\harmonic = \harmonic \eqsp,
% \end{equation}
% %$\Pkerhmc[h][T]\harmonic = \harmonic$,
% are constant. First since $\Pkerhmc[h][T]$ is irreducible with respect
% to the Lebesgue measure and aperiodic, by \cite[Theorem
% 14.0.1]{meyn:tweedie:2009} for $\Leb$-almost all $q$ we get $\lim_{n
%   \to \plusinfty} \Pkerhmc[h][T]^n \harmonic(q) = \pi(\harmonic)$ and therefore by
% \eqref{eq:def_harm} $\harmonic(q) = \pi(\harmonic)$. Therefore we get that for all $q \in \rset^d$ by \eqref{eq:def_kernel_hmc_false_density},
% \begin{multline}
%    \Pkerhmc[h][T]^n \harmonic(q) = \pi(\harmonic)  \int_{\rset^d}  \alphaacc\defEns{(q,\tilde{p}),\Phiverlet[h][T](q,\tilde{p})} \rme^{-\norm{\tilde{p}}^2/2} \rmd \tilde{p} \\
% +   \harmonic(x) \int_{\rset^d}  \parentheseDeux{1-\alphaacc\defEns{(q,\tilde{p}),\Phiverlet[h][T](q,\tilde{p})}} \rme^{-\norm{\tilde{p}}^2/2} \rmd \tilde{p} \eqsp.
% \end{multline}
% Combining this result with \eqref{eq:def_harm}, we get for all $q \in \rset^d$
% \begin{equation}
% (\harmonic(q)-\pi(\harmonic)) \int_{\rset^d} \alphaacc\defEns{(q,\tilde{p}),\Phiverlet[h][T](q,\tilde{p})} \rme^{-\norm{\tilde{p}}^2/2} \rmd \tilde{p} = 0\eqsp.
% \end{equation}
% It follows from \Cref{lem:bound_first_iterate_leapfrog} and \eqref{eq:def_acc_ratio} that for all $q \in \rset^d$, $\harmonic(q) = \pi(\harmonic)$
% which concludes the proof.
% =======
% The proof of \ref{theo:irred_harris_b} using a change of variable with $\Phiverletqi[h][T](q,\cdot)$.

% We now show that $\Tker_{h,T}$ satisfies the condition which implies that $\Pkerhmc[h][T]$ is a \Tkernel.
% First, for all $\eventB \in \borelSet(\rset^d)$,
% \begin{equation}
% \label{eq:minoration_pseudo_density_P}
% \Tker_{h,T}(q, \eventB) \geq (2 \uppi)^{-d/2} \Leb(\eventB)
%  \times \inf_{\bar{q} \in \eventB} \defEns{\alphaacc\defEns{(q,\Phiverletqi[h][T](q,\bar{q})),\Phiverlet[h][T](q,\Phiverletqi[h][T](q,\bar{q}))} \rme^{-\norm{\Phiverletqi[h][T](q,\bar{q})}^2/2} \detj_{\Phiverletqi[h][T]}(q,\bar{q})} \eqsp,
% \end{equation}
% with the convention $0 \times \plusinfty = 0$. Since $\Phiverletqi[h][T](q,\cdot)$
% is a diffeomorphism on $\rset^d$, we get that  $
% \Tker_{h,T}(q,\eventB) >0$ for all $q \in \rset^d$ and all compact set $\eventB$ satisfying $\Leb(\eventB) > 0$. Since the Lebesgue measure is regular, this implies that $\Pkerhmc[h][T]$ is irreducible with respect to the Lebesgue measure and aperiodic.

% By Fatou's Lemma, for any $\eventB \in\mcb(\rset^d)$, $q \mapsto \Tker_{h,T}(q,\eventB)$ is lower semi-continuous.
% We now show that all the compact sets are small. Let $\eventB \subset \rset^d$ be compact.  Using
% \eqref{eq:inverse_1_2} there exists $C \geq 0$ such that for all
% $q,p,v \in \rset^d$, $ \norm{p-v} \leq C \normLigne{
%   \Phiverletq[h][T](q,p)- \Phiverletq[h][T](q,v)}$. It follows
% that for all $p \in \rset^d$, $\sup_{q \in \eventB} \normLigne{
%   \Phiverletqi[h][T](q,\p)} \leq C\defEnsLigne{\norm{\p} + \sup_{q \in
%     \eventB} \normLigne{\Phiverletq[h][T](q,0)}}$. Using this upper
% bound and $\Jac_{\Phiverletq(\q,\cdot)}(\Phiverletqi[h][T](q,\p))
% \Jac_{\Phiverletqi[h][T]}(q,\tilde{\p}) = \operatorname{I}_n$ in
% \eqref{eq:minoration_pseudo_density_P}, where $\operatorname{I}_n$ is
% the identity matrix, we deduce that there exists $\varepsilon >0$ such
% that for all $\eventA\in \borelSet(\rset^d)$, $\eventA \subset \eventB$,
% \begin{equation}
%   \inf_{q \in \eventB} \Pkerhmc[h][T](q, \eventA)  \geq \varepsilon \Leb(\eventA) \eqsp,
% \end{equation}
% and therefore $\eventB$ is small for $\Pkerhmc[h][T]$.
% >>>>>>> f8207bad5c0353bdfe37210ffc64a715e92e53ed

Finally, the last statements of \ref{theo:irred_harris_c} follows from \Cref{propo:harris_rec} in \Cref{sec:harr-recurr-metr} which implies that  $ \Pkerhmc[h][T]$ is Harris recurrent and  \cite[Theorem 13.0.1]{meyn:tweedie:2009} which implies  \eqref{eq:harris-theorem}.

\subsubsection{Proof of \Cref{theo:irred_D}}
\label{sec:proof-crefth_irred_D}
We use \Cref{coro:irred}. Indeed $\Pkerhmc[h][T]$ is
of form \eqref{eq:def_pkerb} and it is straightforward to check that it
satisfies \Cref{assumG:phi} (note that \Cref{lem:bound_first_iterate_leapfrog_a}
shows that $\Phiverlet[h][T]$ is a Lipshitz function on $\rset^{2d}$).

We now check that $\Pkerhmc[h][T]$ satisfies \Cref{assumG:irred_b}($\rassG,0,\MassG$) for all $\rassG,\MassG \in
\rset_+^*$ using \Cref{le:degree_application}.  By \eqref{eq:qk}, for all $T \in \nsets$, $h >0$, $q,p \in \rset^d$,
\begin{equation}
  \label{eq:phiverlet_gqth}
  \Phiverletq[h][T](q,p) = T
h p + g_{q,T,h}(p)
\end{equation}
where $g_{q,T,h}(p) = q - (Th^2/2) \nabla \F(q) -
h^2 \gperthmc[T](q,p)$ where $\gperthmc[T]$ is defined by \eqref{eq:def_gperthmc}. \Cref{lem:inverse_1} shows that for any $T \in \nsets$ and $h >0$, it holds that
\begin{equation}
    \label{eq:2:theo:irred_D}
\sup_{p,v,q \in  \rset^d} \frac{\norm{g_{q,T,h}(p)-g_{q,T,h}(v)}}{\norm{p - v}} \leq T h [\{1 + h \constzero^{1/2} \vartheta_1(h \constzero^{1/2} )\}^T-1] \eqsp,
\end{equation}
which implies that the condition
\Cref{le:degree_application}-\ref{propo:irred_b_item_i} is satisfied. To check that
condition  \Cref{le:degree_application}-\ref{propo:irred_b_item_ii} holds, we consider separately the two cases: $\beta <1$ and $\beta =1$.

\begin{enumerate}[label=$\bullet$, wide, labelwidth=!, labelindent=0pt]
\item Consider first the case $\beta <1$. By \Cref{assum:regOne}-\ref{assum:regOne_b},
for any $T \in \nsets$ and $h >0$, we get
\begin{equation}
\norm{\gperthmc[T](\q,\p)} \leq  T \sum_{i=1}^{T-1} \norm{\nabla \F \circ \Phiverletq[h][i](\q,\p)} \leq
\constzeroT T \sum_{i=1}^{T-1} \defEns{ 1 + \norm{\Phiverletq[h][i](\q,\p)}^{\expozero}}
 \eqsp.
\end{equation}
Hence, by \Cref{lem:bound_first_iterate_leapfrog_b}-\ref{lem:bound_first_iterate_leapfrog_1}
there exists $C \geq 0$ such that for all $R\in \rset_+^*$ and
$q,p \in \rset^d$, $\norm{q} \leq R$,
\begin{equation}
\label{eq:3:theo:irred_D}
\norm{g_{q,T,h}(p)} \leq C \defEns{1+R^{\beta} +\norm{p}^{\expozero}} \eqsp,
\end{equation}
which implies that condition \ref{propo:irred_b_item_ii} of \Cref{le:degree_application} holds for any $T \in \nsets$ and $h >0$.

\item Consider now the case $\beta =1$.  For any $T \in \nsets$, $h >0$,  $q,p \in \rset^d$ we get using \Cref{assum:regOne}-\ref{assum:regOne_a}
\begin{align}
  \norm{g_{q,T,h}(p)} &\leq \norm{q} + Th^2 \constzero  \norm{q} /2 + Th^2 \norm{\nabla U(0)} /2\\
  & \qquad \qquad +h^2 \norm{\gperthmc[T](q,p) - \gperthmc[T](q,0)} + h^2 \norm{ \gperthmc[T](q,0)} \eqsp.
\end{align}
Therefore using \Cref{lem:inverse_1}, for any $q,p \in \rset^d$, $\norm{q} \leq R$ for $R \geq 0$, for any $T \in \nsets$ and $h >0$ satisfying \eqref{eq:condition-h,T-harris}, there exists $C \geq 0$ such that
\begin{equation}
  \norm{g_{q,T,h}(p)} \leq C + h T  [ \{1+ h \constzero^{1/2} \vartheta_1(h\constzero^{1/2})\}^T-1]  \norm{p} \eqsp,
\end{equation}
showing that condition \ref{propo:irred_b_item_ii} of \Cref{le:degree_application} is satisfied.
\end{enumerate}

Therefore,  \Cref{le:degree_application} can be applied and for any $T \in \nsets$ and $h >0$ if $\beta <1$ and for any $h > 0$ and $T \in \nsets$ satisfying \eqref{eq:condition-h,T-harris} if $\beta =1$, $\Pkerhmc[h][T]$ satisfies \Cref{assumG:irred_b}($\rassG,0,\MassG$) for all $\rassG,\MassG \in
\rset_+^*$.  \Cref{coro:irred} concludes the proof of \ref{theo:irred_D_a} and \ref{theo:irred_D_b}.
The last statement then follows from   \cite[Theorem 14.0.1]{meyn:tweedie:2009}.

% Using this result and \Cref{theo:irred}, we get that for all $\rassG,\MassG  \in \rset_+^*$ there exists $\varepsilon >0$ such that
% for all $\q \in \ball{0}{\rassG}$ and $\eventA \in \borelSet(\rset^d)$,
% \begin{equation}
%   \Pkerhmc[h][T](q, \eventA) \geq \varepsilon \Leb(\eventA \cap \ball{0}{M}) \eqsp.
% \end{equation}
% \Cref{coro:irred} Combining this result and \eqref{eq:1:theo:irred_D} concludes the proof of \ref{theo:irred_D_a} and \ref{theo:irred_D_b}.

% The proof is a consequence of \Cref{lem:bound_first_iterate_leapfrog},
% \Cref{le:degree_application} and \Cref{theo:irred}.  \alain{give some
%   details}

%%% Local Variables:
%%% mode: latex
%%% TeX-master: "main"
%%% End:


%	\section*{Acknowledgements}
%	None!

\bibliography{refs}
\bibliographystyle{abbrv}


\end{document}
