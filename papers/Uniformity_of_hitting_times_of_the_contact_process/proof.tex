\section{Proof of Theorems~\ref{mainThm} and~\ref{thmTightness}}
\label{thmSec}

%%%%%%%%%%%%%%%%%%%%%%%%%%%%%%%%%%%%%%%%%%%%%%%%%%%%%%%%%%%%%%%%%%%%%%%%%%%
%%%%%%%%%%%%%%%%%%%%%%%%%%%%%%%%%%%%%%%%%%%%%%%%%%%%%%%%%%%%%%%%%%%%%%%%%%%
\subsection{Proof of Theorem~\ref{mainThm}}
%%%%%%%%%%%%%%%%%%%%%%%%%%%%%%%%%%%%%%%%%%%%%%%%%%%%%%%%%%%%%%%%%%%%%%%%%%%
%%%%%%%%%%%%%%%%%%%%%%%%%%%%%%%%%%%%%%%%%%%%%%%%%%%%%%%%%%%%%%%%%%%%%%%%%%%

Before getting into the details, we provide a rough outline of the proof. To prove Theorem~\ref{mainThm}, we consider $\oE[\t(kx) - \t((k-1)x)]$ as discrete derivatives of the expected hitting time 
$\oE\t(n x)$, in the sense that 
$$\oE\t(n x)  = \sum_{k = 1}^n \oE[\t(k x) - \t((k-1)  x)].$$
The shape theorem for the supercritical contact process shows that the left-hand side is of the order $n\mu(x)$. Hence, a large number of small derivatives $\oE[\t((n+1) x) - \t(n x)]$ would need to be compensated by a substantial number of atypically large derivatives. However, leveraging the concept of essential hitting times allows us to prove the following sub-additivity bound that is sufficiently strong to exclude excessive growth. 



\begin{lemma}
	\label{derivBoundLem}
	Let $p > 0$ and $x, y \in \Zd$ be arbitrary. Then, 
	$$\oE\left[|t(x) - \sigma(y)|^p\cdot \mathds{1}\{\t(x) \ge \s(y) \} \right] \leq \oP(\t(x) \ge \s(y)) \cdot \oE \left[t(x-y)^p \right] $$
\end{lemma}

Before proving Lemma~\ref{derivBoundLem}, we explain how it implies Theorem~\ref{mainThm}. For this, we need an $L_p$-version of the shape theorem~\cite[Theorem 3]{GaretMarch12}.
\begin{lemma}
	\label{shapeThm}
	Let $p > 0$ be arbitrary. Then, $$\lim_{|x| \to \infty}\frac{\oE\left[t(x)^p \right]}{\mu(x)^p} = 1.$$
\end{lemma}
\begin{proof}
By Proposition~\ref{propEssTightness} it suffices to prove the claim with $\t$ replaced by $\s$. The shape theorem~\cite[Theorem 3]{GaretMarch12} gives almost sure convergence of $\sigma(x)/\mu(x)$ as $|x| \to \infty$. By~\cite[Theorem 15]{GaretMarch14}, $\{\sigma(x)/\mu(x)\}_{x \in \Zd \setminus \{o\}}$ is bounded in $L_p$ for every $p$; hence, for every $p$, $\{\sigma(x)^p/\mu(x)^p\}_{x \in \Zd \setminus \{o\}}$ is uniformly integrable. This concludes the proof.
\end{proof}

\begin{proof}[Proof of Theorem~\ref{mainThm}]
	First, for any $n \ge 1$ and $x\in\Zd$, 
	$$\oE[\t(nx)] = \sum_{k = 1}^n \oE[\t(kx) - \t((k-1)x)] \le \sum_{k = 1}^n \oE[\t(kx) - \t((k-1)x)] \one \{\oE[\t(kx)] \ge \oE[\t((k-1)x)]\}.$$
	Hence, introducing 
	$$I_{n,x} = \{k \in \{1,\ldots, n \}:\, \oE[\t(kx)] \ge \oE[\t((k-1)x)]\}$$
	as the index set of correctly ordered expected hitting times, we arrive at 
		$$\oE[\t(nx)] \le \sum_{k \in I_{n,x}} \oE[\t(kx) - \s((k-1)x)] + \sum_{k \in I_{n,x}}\oE[\s((k-1)x) - \t((k-1)x)]$$
	By Lemma~\ref{derivBoundLem}, the first sum is bounded above by 
		$$\oE[\t(x)]\sum_{k \in I_{n,x}}\oP(\s((k-1) x) \le \t(k x)) .$$
		Additionally, by Proposition~\ref{propEssTightness}, the second sum is bounded from above by a universal constant $C_1$. Hence, by Lemma~\ref{shapeThm}, 
		\begin{align*}
			\mu(x) &= \lim_{n \to \infty}n^{-1} \oE \t(nx) \\ 
			&\le C_1 + \oE[\t(x)] \liminf_{n \to \infty}\frac{\sum_{k \in I_{n,x}}\oP(\s((k-1) x) \le \t(k x)) }{n}\\
			&\le C_1 + \oE[\t(x)] \liminf_{n \to \infty}\frac{\sum_{k \in I_{n,x}}\oP(\t((k-1) x) \le \t(k x)) }{n}.
		\end{align*}
Now, dividing by $\mu(x)$ and taking the limit $|x| \to \infty$, the two statements of the theorem respectively follow from the trivial bounds
\begin{align*}& \sum_{k \in I_{n,x}}\oP(\t((k-1) x) \le \t(k x)) \leq \sum_{k=1}^n\oP(\t((k-1) x) \le \t(k x)), \\& \sum_{k \in I_{n,x}}\oP(\t((k-1) x) \le \t(k x)) \leq \#I_{n,x}.
\end{align*}
\end{proof}

It remains to prove Lemma~\ref{derivBoundLem}.

\begin{proof}[Proof of Lemma~\ref{derivBoundLem}]
	Let 
	$$\t(y, x) = \inf\{s \geq \s(y):\;(y, \s(y)) \rightsquigarrow (x, s)\}$$
	denote the first time that the infection from the space-time point $(y, \s(y))$ reaches $x$. By definition of $\s(y)$, this hitting time is almost surely finite under $\oP$. Moreover, $\t(y, x)$ is measurable with respect to the Harris construction \emph{after} time $\s(y)$, whereas $\{\t(x) \ge \s(y)\}$ is measurable with respect to the Harris construction \emph{before} time $\s(y)$. Hence, the renewal-type property of $\s(y)$ as in \cite[Lemma 8]{GaretMarch12} gives  
	\begin{align*}
	\oE\left[|t(x) - \sigma(y)|^p\cdot \mathds{1}\{t(x) \geq \sigma(y)\}\right] &\le \oE\left[t(y,x)^p \cdot \mathds{1}\{t(x) \geq \sigma(y)\} \right]\\& = \oP\left(t(x) \geq \sigma(y) \right) \cdot \oE\left[t(x-y)^p\right],
	\end{align*}
which proves the claim.
	%Similarly,
	%\begin{align*}
%		\oE[(\t(x) - \s(y)) \one\{\t(x) \ge \s(y)\}] \le \oE[\t(y, x) \one\{\t(x) \ge \s(y)\}] \le \oE[\t(x - y)] \oP(\t(x) \ge \s(y)),
%	\end{align*}
%	as asserted.

\end{proof}


%%%%%%%%%%%%%%%%%%%%%%%%%%%%%%%%%%%%%%%%%%%%%%%%%%%%%%%%%%%%%%%%%%%%%%%%%%%
%%%%%%%%%%%%%%%%%%%%%%%%%%%%%%%%%%%%%%%%%%%%%%%%%%%%%%%%%%%%%%%%%%%%%%%%%%%
\subsection{Proof of Theorem~\ref{thmTightness}}
%%%%%%%%%%%%%%%%%%%%%%%%%%%%%%%%%%%%%%%%%%%%%%%%%%%%%%%%%%%%%%%%%%%%%%%%%%%
%%%%%%%%%%%%%%%%%%%%%%%%%%%%%%%%%%%%%%%%%%%%%%%%%%%%%%%%%%%%%%%%%%%%%%%%%%%

In order to prove Theorem~\ref{thmTightness}, we note that $\t(x) - \t(y)$ can be large only if either $\t(x) - \s(y)$ or $\s(y) - \t(y)$ is large. In particular, with Lemma~\ref{derivBoundLem} and Proposition~\ref{propEssTightness} at our disposal, a quick algebraic manipulation gives us Theorem~\ref{thmTightness}. 


\begin{proof}[Proof of Theorem~\ref{thmTightness}]
By the inequality
$$\oE\left[\left|\frac{t(x)-t(y)}{|x-y|} \right|^p \right] \leq 1 +  \oE\left[\left|\frac{t(x)-t(y)}{|x-y|} \right|^{p'} \right]$$
for $0 < p < p'$, it suffices to prove the statement for $p \geq 1$.	
	
	By Minkowski inequality,
\begin{align*}
&\left(\oE\left[|t(x) - t(y)|^p\right]\right)^{1/p}\\&\quad  \leq \left(\oE \left[(t(x) - t(y))^p \cdot \mathds{1}\{t(x) \ge t(y)\}\right]\right)^{1/p} + \left(\oE\left[(t(y) - t(x))^p \cdot \mathds{1}\{t(y) \ge t(x) \} \right]\right)^{1/p},
\end{align*}
	so that we have reduced the task to establishing an upper bound for the first summand. By distinguishing further between the events $\{\t(x) \ge \s(y)\}$ and $\{\t(y) \le \t(x) \le \s(y)\}$, and applying Lemma~\ref{derivBoundLem}, we arrive at 
	\begin{align*}
	&\left(\oE\left[|t(x) - t(y)|^p\cdot \mathds{1}\{t(x) \ge t(y) \}\right]\right)^{1/p}\\
	&\leq \left(\oE\left[|t(x) - \sigma(y)|^p\cdot \mathds{1}\{t(x) \ge t(y) \}\right]\right)^{1/p} + \left(\oE\left[| \sigma(y)-t(y)|^p\right]\right)^{1/p}\\
		&\leq \left(\oE\left[|t(x) - \sigma(y)|^p\cdot \mathds{1}\{t(x) \ge \sigma(y) \}\right]\right)^{1/p} + 2\left(\oE\left[| \sigma(y)-t(y)|^p\right]\right)^{1/p}\\
	& \leq \left(\oE\left[t(x-y)^p \right] \right)^{1/p} + 2\left(\oE\left[(\sigma(y) - t(y))^p \right] \right)^{1/p}.
	\end{align*}
	%\begin{align*}
	%	\lVert (\t(x) - \t(y)) \one\{\t(x) \ge \t(y)\}\rVert_p &\le \lVert(\t(x) - \s(y)) \one\{\t(x) \ge \s(y)\}\rVert_p + \lVert\s(y) - \t(y)\rVert_p \\
	%	&\le \lVert\t(x-y) \rVert_p + \lVert\s(y) - \t(y)\rVert_p.
	%\end{align*}
	An application of Lemma~\ref{shapeThm} and Proposition~\ref{propEssTightness} then concludes the proof.
\end{proof}

%

