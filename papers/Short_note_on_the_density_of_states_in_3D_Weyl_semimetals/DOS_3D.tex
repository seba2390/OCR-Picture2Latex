\documentclass[aps]{revtex4}
%\documentclass[twocolumn,aps,showpacs]{revtex4}
%\documentclass[aps,showpacs]{revtex4}
%\documentclass[english]{article}
\usepackage{graphicx,psfrag}
%\usepackage{graphicx,amsmath}
\textwidth=16cm % for prints
%\textwidth=10cm % for power point
\textheight=23cm
\oddsidemargin=0.cm
\topmargin=-1.3cm
%\textwidth=16cm
%\textheight=23cm
%\oddsidemargin=0.cm
%\topmargin=-1.3cm
\def\no{\noindent}
\def\bc{\begin{center}}
\def\ec{\end{center}}
\def\vs{\vskip0.5cm}
\def\beq{\begin{equation}}
\def\eeq{\end{equation}}
\def\cl{\centerline}
\def\d{\downarrow}
\def\u{\uparrow}
\def\doub{\downarrow\uparrow}
\def\bj{{\bf j}}
\def\bl{{\bf k}}
\def\br{{\bf r}}
\def\bq{{\bf q}}
\def\bk{{\bf k}}
\def\bareta{{\bar\eta}}

\begin{document}

\title{Short note on the density of states in 3D Weyl semimetals
}

\author{K. Ziegler and A. Sinner}
\affiliation{
Institut f\"ur Physik, Universit\"at Augsburg,
D-86135 Augsburg, Germany}
\date{\today}

\begin{abstract}
The average density of states in a disordered three-dimensional Weyl system is discussed in the
case of a continuous distribution of random scattering. Our results clearly indicate that
the average density of states does not vanish, reflecting the absence of a critical point
for a metal-insulator transition. This calculation supports recent suggestions of an avoided
quantum critical point in the disordered three-dimensional Weyl semimetal. However, the
effective density of states can be very small such that the saddle-approximation with
a vanishing density of states might be valid for practical cases.
\end{abstract}
%\pacs{05.60.Gg, 66.30.Fq, 05.40.-a}

\maketitle


\section{Introduction}

The existence of a metal-insulator transition in disordered three-dimensional (3D) Weyl semimetals 
has been debated in the recent literature 
\cite{huse13,brouwer14,gurarie14,pixley16,pixley16a,pixley16b,ziegler16,pixley17,sbierski17,sinner17,1805}. 
It is closely
related to the question, whether or not the average density of states (DOS) at the spectral 
node vanishes below some critical disorder strength. The self-consistent Born approximation 
provides such a critical value with a vanishing DOS for weak disorder. It has been argued 
that rare regions of the random distribution may lead to a non-vanishing average DOS, though \cite{huse13}. 
This was supported by recent
numerical studies based on the T-matrix approach, which gives an exponentially small
DOS \cite{pixley17} but was questioned in a recent study based on an instanton solution \cite{1805}.
In this short note we show that, depending on the type and strength,
a continuous distribution of disorder can create a substantial average DOS at the 
spectral node in 3D Weyl systems. This requires at least two impurities to create a resonant
state between these impurities. A single impurity or a single instanton does not
contribute to the spectral weight at the Weyl node, though, in accordance with the arguments in 
Ref. \cite{1805}. 
This supports the picture of an avoided quantum critical point in the presence of a 
distribution of impurities, as advocated in Ref. \cite{pixley17}.


\section{Model}

The 3D Weyl Hamiltonian for electrons with momentum ${\vec p}$ is expanded 
in terms of Pauli matrices $\tau_j$ ($j=0,1,2,3$; $\tau_0$ is the $2\times2$ unit matrix) 
as $H=H_0-U\tau_0$, where %\cite{koshino14}
\beq
H_0=v_F{\vec\tau}\cdot{\vec p}
%-U\tau_0 
\ \ \ {\rm with}\ \ {\vec\tau}=(\tau_1,\tau_2,\tau_3)
\ .
\label{ham00}
\eeq
$v_F$ is the Fermi velocity and
$U$ is a disorder term, represented by a random potential with mean $\langle U\rangle=E_F$ (Fermi energy)
and variance $g$. 
%$m$ is a gap term which will be set to zero after the discussion of the symmetries.
The average Hamiltonian $\langle H\rangle=H_0-E_F\tau_0$ 
generates a spherical Fermi surface with radius $|E_F|$, and
with electrons (holes) for $E_F>0$ ($E_F<0$). Physical quantities are expressed in such units that $v_F\hbar=1$.

The DC limit $\omega\to0$ of the conductivity of 3D Weyl fermions depends only on the scattering rate $\eta$ 
and the Fermi energy $E_F$
\cite{ziegler16}:
\beq
\sigma(\eta,E_F)=2\frac{e^2}{h}\eta^2\int_0^\lambda\frac{(\eta^2+k^2)^2+E_F^2(2\eta^2+2k^2/3+E_F^2)}
{[(\eta^2-E_F^2+k^2)^2+4\eta^2E_F^2]^2} \frac{k^2dk}{2\pi^2}
\label{cond2}
\eeq
with momentum cut-off $\lambda$. 
At the node ($E_F=0$) the DC conductivity in Eq. (\ref{cond2}) is reduced to the expression
\beq
\sigma =2\frac{e^2}{h}\eta^2\int_0^\lambda \frac{k^2}{(\eta^2+k^2)^2}\frac{dk}{2\pi^2}
=\frac{e^2\eta}{2\pi^2h}\left[\arctan(1/\zeta)-\frac{\zeta}{1+\zeta^2}\right] \ \ \ 
(\zeta=\eta/\lambda)
\ ,
\label{cond000}
\eeq
which becomes for $\lambda\gg\eta$
\beq
\sigma\sim \frac{ e^2}{4\pi h}\eta
\ .
\label{cond001}
\eeq
The last result was also derived by Fradkin some time ago \cite{fradkin86a}. 
In contrast to the 2D case, where $\sigma=e^2/\pi h$, the 3D case gives a linearly 
increasing behavior with respect to the scattering rate. 

The results in (\ref{cond2}) -- (\ref{cond001}) clearly indicate that a metal-insulator transition 
in disordered 3D Weyl systems is directly linked to the scattering rate $\eta$. The latter describes 
the broadening of the poles of the one-particle Green's function and is proportional to 
the average DOS
\beq
\rho_{\br}(E_F)=\lim_{\epsilon\to 0}\frac{1}{\pi}Im\left[{\bar G}_{\br\br}(-i\epsilon))\right]
\ , \ \ \ 
{\bar G}(-i\epsilon)
=\langle(H_0-U\tau_0-i\epsilon)^{-1}\rangle
\ ,
\label{av_dos}
\eeq
where is ${\bar G}_{\br\br}$ is the diagonal element of ${\bar G}$
with respect to space coordinates.
The self-consistent Born approximation \cite{fradkin86a,ziegler16} at the node $E_F=0$ reads 
\beq
\eta=\eta I
\\ \ \ {\rm with}\ \  
I=\gamma\left[
\lambda-\eta\arctan(\lambda/\eta)\right] %,\ \ \ \beta=\beta % \sqrt{\eta^2+\Delta^2/4} 
\eeq
for the effective disorder strength $\gamma=g/2\pi^2$.
There are two solutions, namely $\eta=0$ and a solution with $\eta\ne0$,
which exists only for
sufficiently large $\gamma$. Moreover, $\eta$ vanishes continuously as we reduce $\gamma$. 
For $\eta\sim 0$ we obtain the linear behavior
\beq
\eta\sim\frac{2\lambda}{\pi}(\gamma\lambda -1)
\ ,
\label{scba2}
\eeq
where $\gamma_c=1/\lambda$ appears as a critical point with $\eta=0$ for $\gamma\le \gamma_c$
and $\eta>0$ for $\gamma>\gamma_c$.

\begin{figure*}[t]
\includegraphics[width=8cm]{c-plane.eps}
%\includegraphics[width=9cm]{cond2.eps}
\caption{
%(Color online) 
Poles of the one-particle Green's function and the Cauchy-Lorentz distribution. 
The contour of the $U_\br$--integration encloses only one pole of the Cauchy-Lorentz 
distribution but not the other poles.  
}
\label{fig:c-plane}
\end{figure*}


\section{Average density of states}


\subsection{Few impurities: Lippmann-Schwinger equation}

At the node $E_F=0$ the pure DOS $\rho_{0;\br}(E_F=0)$ vanishes. However,
a few impurities have already a significant effect on the local DOS: % at the node $E_F=0$:
Assuming an impurity potential $U_N$ on $N$ sites, we use the identity (lattice version of the 
Lippmann-Schwinger equation) 
%\cite{ziegler85}
\beq
(G_0^{-1}-U_N)^{-1}=G_0 + G_0({\bf 1} - U_N P_NG_0P_N)^{-1}_NU_N G_0
\ ,
\label{LSE}
\eeq
where $P_N$ is the projector on the impurity sites and $(...)^{-1}_N$ is the inverse 
on the impurity sites. Although $\rho_{0;\br}(E_F=0)$ vanishes,
the second term on the right-hand side of Eq. (\ref{LSE}) can contribute with the poles of 
$({\bf 1} - U_N P_NG_0P_N)^{-1}_N$
to the DOS. These poles are ``rare events'' and require a fine-tuning of the impurity potential, 
whereas the generic case of a general $U_N$ would still have a vanishing DOS. In a realistic situation
the number of impurities is macroscopic with a non-zero density in the infinite system. Then
the identity (\ref{LSE}) cannot be used for practical calculations and we have to average over 
many impurity realizations.
This leads to the average Green's function of Eq. (\ref{av_dos}), which will be calculated subsequently.


\subsection{One vs. two impurities}

The Green's function $G_0$ of the system without impurities reads 
\beq
G_{0,\br}(-i\epsilon)=\frac{1}{|{\cal B}|}\int_{\cal B}\frac{e^{i{\vec k}\cdot\br}}{\epsilon^2+k^2}
\left(i\epsilon\tau_0+{\vec k}\cdot {\vec \tau}\right)d^3k
\equiv i\epsilon\gamma_0\tau_0+{\vec \gamma}\cdot{\vec \tau}
\ ,
\label{1pgf}
\eeq
where ${\cal B}$ is the Brillouin zone of the underlying lattice and 
\[
\gamma_0=\frac{1}{|{\cal B}|}\int_{\cal B}\frac{e^{i{\vec k}\cdot\br}}{\epsilon^2+k^2}d^3k
\ ,\ \ \ 
\gamma_j=\frac{1}{|{\cal B}|}\int_{\cal B}\frac{e^{i{\vec k}\cdot\br}k_j}{\epsilon^2+k^2}d^3k
\ \ \ (j=1,2,3)
\ .
%\label{coeff2}
\]
Then the diagonal element $G_{0,0} =i\epsilon \gamma\tau_0$ vanishes with $\epsilon\sim0$. 
This implies that for a single impurity there is no bound 
state at finite impurity strength $U_\br$, since in the impurity term of the Lippmann-Schwinger
equation (\ref{LSE}) the $2\times2$ matrix
\beq
({\bf 1} - U_\br P_\br G_0P_\br)^{-1}=\frac{1}{1-i\epsilon \gamma_0 U_\br}\tau_0
%G_{0,\br}\frac{U_r}{1-i\epsilon \gamma_0 U_r}G_{0,\br}
\label{2x2}
\eeq
has a pole at $U_\br\sim \infty$.
The latter reflects the statement that a potential well in 3D Weyl semimetals does never generate 
spectral density at zero energy \cite{1805}.
For two impurities, though, there is a resonant inter-site bound state between the impurities, 
since $G_{0,\br-\br'}$ ($\br'\ne \br$) does not vanish for $\epsilon\to 0$ but decays with a 
power law for $|\br-\br'|$ due to the Pauli matrix coefficients $\gamma_j$ in Eq. (\ref{1pgf}):
\beq
({\bf 1} - U P_{\{\br,\br'\}}G_0P_{\{\br,\br'\}})^{-1}
=\pmatrix{
1-i\epsilon\gamma_0U_\br & 0 & -U_\br\gamma_3 & -U_\br(\gamma_1-i\gamma_2) \cr
0 & 1-i\epsilon\gamma_0U_\br & -U_\br(\gamma_1+i\gamma_2) & U_\br\gamma_3 \cr
-U_{\br'}\gamma_3 & -U_{\br'}(\gamma_1-i\gamma_2) & 1-i\epsilon\gamma_0U_{\br'} & 0 \cr
-U_{\br'}(\gamma_1+i\gamma_2) & U_{\br'}\gamma_3 & 0 & 1-i\epsilon\gamma_0U_{\br'} \cr
}^{-1}
\ .
\eeq
The degenerate eigenvalues of this matrix 
\beq
\frac{1}{1-i\epsilon\gamma_0(U_{\br}+U_{\br'})/2\pm \sqrt{U_{\br}U_{\br'}
(\gamma_1^2+\gamma_2^2+\gamma_3^2)-\epsilon^2\gamma_0^2(U_{\br}-U_{\br'})^2/4}}
%\to_{\epsilon\to0} \frac{1}{1\pm \sqrt{U_{\br}U_{\br'}(\gamma_1^2+\gamma_2^2+\gamma_3^2)}}
\eeq
have poles for finite $U_{\br}$, $U_{\br'}$.
Thus, the corresponding bound states contribute with a non-vanishing density of states. 
In the remainder of the paper this result will be generalized to multiple impurities with corresponding 
resonant bound states. 


\subsection{Distribution with simple poles}
\label{sect:CLD}

From here on we consider a continuous distribution of the disorder potential $U$ with
$\prod_\br P(U_\br)dU_\br$ and average one-particle Green's function
\beq
{\bar G}(-i\epsilon)=\int (H_0-U\tau_0-i\epsilon)^{-1}\prod_\br P(U_\br)d U_\br
\ .
\eeq
For $\epsilon>0$ the one-particle Green's function $(H_0-U\tau_0-i\epsilon)^{-1}$ has poles 
for $U_\br$ on the upper complex half-plane.
Assuming that the distribution density $P(U_\br)$ has isolated poles in the lower complex half-plane,
the Cauchy integration can be applied by closing the integration along the real axis in the
lower complex half-plane,
as depicted in Fig. \ref{fig:c-plane}. The simplest realization is the Cauchy-Lorentz distribution
\beq
P_{CL}(U_\br)=\frac{1}{\pi}\frac{\eta}{(U_\br-E_F)^2+\eta^2}
\ ,
\eeq
which gives
\beq
{\bar G}(-i\epsilon)
=(H_0-(E_F+i\epsilon+i\eta)\tau_0)^{-1}
\ .
\eeq
The average DOS then reads
\beq
\rho_{\br}(E_F)=\frac{\eta}{\pi}
[(H_0-E_F\tau_0)^2+\eta^2\tau_0]^{-1}_{\br\br}
\ .
\eeq
The Cauchy-Lorentz distribution has an infinite second moment (i.e., $g$ is infinite). 
A distributions with a finite second moment can be created from the differential of the Cauchy-Lorentz
distribution with respect to $\eta$. 
Many distributions, like the popular Gaussian distribution
\beq
P_G(U_\br)=\frac{1}{\sqrt{\pi g}}e^{-(U_\br-E_F)^2/g}
\ ,
\eeq
do not have a simple pole structure, though. Then another approach can be applied to show that 
there is a non-vanishing average DOS.

\begin{figure*}[t]
\includegraphics[width=4cm]{cubes_a.eps}
%\includegraphics[width=9cm]{cond2.eps}
\caption{
%(Color online) 
Dividing the system into cubes $\{ S\}$ of size $|S|$ with boundary $\partial S$.
}
\label{fig:cubes}
\end{figure*}


\subsection{Distribution without simple poles}
\label{sect:lower_bound}

Now we only assume that the distribution of $U_\br$ is continuous. Then the path
of integration can also be deformed away from the poles of the Green's function
to obtain a similar result as in the case of simple poles. The calculation would be 
more complex, though. Therefore, we use a different approach, whose main idea is to
divide the system into cubes $\{ S\}$ of finite identical size (cf. Fig. \ref{fig:cubes}).
Then we estimate (i) the average DOS inside an isolated cube and (ii) the 
contribution of the boundary $\partial S$ between the cubes. This approach was used for a periodic
lattice \cite{ledermann44}, for a random tight-binding model with symmetric Hamiltonian \cite{wegner81}
and for two-dimensional Dirac fermions with random mass \cite{ziegler87}. Later it was applied to 
S-wave superconductor with random order parameter \cite{ziegler88}, and to D-wave superconductor with 
random chemical potential \cite{ziegler98}.

For the average local DOS
\beq
{\bar \rho}_\br=\int\rho_\br(U) \prod_{\br}P(U_\br)dU_\br
\eeq
we obtain from the estimation procedure with steps (i) and (ii) the inequality 
(cf. Supplemented Materials)
\beq
%\sum_{\br\in S}{\bar\rho}_\br\ge {\bar P}_S[|S|-|\partial S|] 
\sum_{\br\in S}{\bar\rho}_\br\ge
\inf_{\{-a\le U_\br'\le a\}}\left[\int_{-v}^v
\sum_{\br \in S}\rho_{S,\br}(U'+E)dE \inf_{-v\le w\le v}\prod_{\br\in S}P(U_\br'+w)\right]
-{\bar P}_S|\partial S|
\ ,
\label{bound}
\eeq
where $|S|$ ($|\partial S|$) is the number of sites of $S$ ($\partial S$) and
\[
{\bar P}_S=\inf_{\{-a\le U_\br'\le a\},-v\le w\le v}\prod_{\br\in S}P(U_\br'+w)
\ .
\]
${\bar P}_S|\partial S|$ is the contribution of the boundary of a cube
and the integral is the integrated DOS on a cube $S$. The boundary term is substracted
because we have removed the boundary. In other words, the left-hand side of (\ref{bound}) is
the average DOS on the entire lattice, the right-hand side is the average DOS on the isolated 
cube $S$.

The value of the lower bound requires an adjustment of the still undetermined parameters 
$a$ and $v$. The integrated DOS $\int_{-v}^v\sum_{\br \in S}\rho_{S,\br}(U'+E)dE$  on $S$
is the number of eigenvalues on the interval $[-v,v]$ of the $S$--projected Hamiltonian 
$H_0-U'$. The projected Hamiltonian is an $|S|\times |S|$ Hermitean matrix with finite elements, 
whose eigenvalues are also finite. Thus, for a fixed $a$ we can choose a sufficiently large $v$
such that all eigenvalues of the projected Hamiltonian are inside the interval $[-v,v]$.
In this case the integrated DOS is $|S|$ and we get from (\ref{bound}) the inequality
\beq
\sum_{\br\in S}{\bar\rho}_\br\ge {\bar P}_S[|S|-|\partial S|] 
\ .
\label{bound2}
\eeq 
$S$ can always be chosen such that the size of the cube $|S|$ is larger than the size
$|\partial S|$ of its boundary. Then the right-hand side of (\ref{bound}) is strictly positive.
$v$ should not be too large, though, in order to avoid that ${\bar P}_S$ becomes
too small, assuming that a typical $P(U_\br)$ decays for large values.
The actual value of $P_S$ depends on the distribution and can be exponentially small.

The average DOS of the entire lattice is estimated by the sum over all cubes, normalized by 
its number $N$. Since all cubes have the same lower bound, this sum is bounded by the right-hand 
side of (\ref{bound2}). This indicates that our estimation works only for a macroscopic number of 
impurities, the case of a single impurity (\ref{2x2}) would always give a lower bound zero.


\section{Conclusion}

There is a crucial difference in terms of the average DOS: For a discrete distribution
the average DOS is non-zero only if the disorder potential is ``resonant'' with the
pure Green's function $G_0$, according to the second term in Eq. (\ref{LSE}). In particular,
a single impurity fails to create spectral weight at the Weyl node.
On the other hand, for a dense distribution of impurities, represented by a continuous random potential, 
there is always a non-vanishing average DOS due to inter-impurity bound states, provided that the values 
of $U_\br$ cover the entire spectrum of $H_0$. 
  
The existence of a critical disorder strength $\gamma_c$, as indicated by the self-consistent
approximation in Eq. (\ref{scba2}), contradicts the existence of a lower non-zero bound of 
the average DOS in Sect. \ref{sect:lower_bound}. Therefore, the self-consistent calculation
is not sufficiently accurate to describe the transport properties of the 3D Weyl semimetal
properly. Since the lower bound of the average DOS is only a qualitative, although rigorous, estimation, 
still a reliable approximation is necessary to obtain an approximative value for the
average DOS. The exact result obtained for the Cauchy-Lorentz distribution in Sect.
\ref{sect:CLD} gives only a hint, because this distribution is not generic. A possible
option is a $N^{-\alpha}$ expansion with non-integer $\alpha$ \cite{ziegler83}.  
 
%A similar situation appears in the case of a 2D semimetal with a small random Dirac mass.
%A self-consistent calculation gives a metal-insulator transition as a function of the average gap
%\cite{ziegler97}. ``Rare events'', on the other hand, can always create states inside the gap.

\vskip0.3cm

Acknowledgment: This work was supported by a grant of the Julian Schwinger Foundation.

\begin{thebibliography}{99}

\bibitem{huse13}
R. Nandkishore, D. A. Huse, and S. Sondhi,  Phys. Rev. B {\bf 89}, 245110 (2014).
%arXiv:1307.3252 (2013).

\bibitem{brouwer14}
B. Sbierski, G. Pohl, E.J. Bergholtz and P.W. Brouwer, Phys. Rev. Lett. {\bf 113}, 026602 (2014).

\bibitem{gurarie14}
S.V. Syzranov, V. Gurarie, L. Radzihovsky, Phys. Rev. Lett. {\bf 114}, 166601 (2015).
%arXiv:1402.3737
%Phys. Rev. B {\bf 91}, 035133 (2015). %arXiv:1411.4635.

\bibitem{pixley16}
J.H. Pixley, D.A. Huse, and S. Das Sarma, Phys. Rev. X {\bf 6}, 021042 (2016).

\bibitem{pixley16a}
J.H. Pixley, P. Goswami, and S. Das Sarma, Phys. Rev. B {\bf 93}, 085103 (2016).

\bibitem{pixley16b}
J.H. Pixley, D.A. Huse, and S. Das Sarma, Phys. Rev. B {\bf 94}, 121107(R) (2016).

\bibitem{ziegler16}
K. Ziegler, Eur. Phys. J. B {\bf 89}, 268 (2016).

\bibitem{pixley17}
J.H. Pixley, Yang-Zhi Chou, P. Goswami, D.A. Huse, R. Nandkishore, L. Radzihovsky, S. Das Sarma,
Phys. Rev. B {\bf 95}, 235101 (2017).
%arXiv:1701.00783.

\bibitem{sbierski17}
B. Sbierski, K.A. Madsen, P.W. Brouwer, and Ch. Karrasch, Phys. Rev. B {\bf 96}, 064203 (2017).

\bibitem{sinner17}
A. Sinner and K. Ziegler, Phys. Rev. B {\bf 96}, 165140 (2017). 

\bibitem{1805}
M. Buchhold, S. Diehl and A. Altland, arxiv:1805.00018.

\bibitem{fradkin86a}
E. Fradkin, Phys. Rev. B {\bf 33}, 3263 (1986).

\bibitem{ledermann44}
W. Ledermann, Proc. R. Soc. London {\bf 182}, 362 (1944).

\bibitem{wegner81}
F. Wegner, Z. Physik B - Condensed Matter {\bf 44}, 9 (1981).

\bibitem{ziegler87}
K. Ziegler, Nucl. Phys. B {\bf 285} [FS19], 606 (1987).

\bibitem{ziegler88}
K. Ziegler, Commun. Math. Phys. {\bf 120}, 177 (1988).

\bibitem{ziegler98}
K. Ziegler, M.H. Hettler, P.J. Hirschfeld, Phys. Rev. B {\bf 57}, 10825 (1998).

%\bibitem{simon00}
%S. Villain-Guillot, G. Jug and K. Ziegler, Ann. Phys. {\bf 9}, 27 (2000).

\bibitem{ziegler83}
K. Ziegler, Phys. Lett. {\bf 99}A, 19 (1983).

%\bibitem{ziegler97}
%K. Ziegler, Phys. Rev. {\bf 55}, 10661 (1997).

\end{thebibliography}


\end{document}

