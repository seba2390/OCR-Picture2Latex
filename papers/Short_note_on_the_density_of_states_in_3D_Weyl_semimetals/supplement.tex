\documentclass[aps]{revtex4}
%\documentclass[twocolumn,aps,showpacs]{revtex4}
%\documentclass[aps,showpacs]{revtex4}
%\documentclass[english]{article}
\usepackage{graphicx,psfrag}
%\usepackage{graphicx,amsmath}
\textwidth=16cm % for prints
%\textwidth=10cm % for power point
\textheight=23cm
\oddsidemargin=0.cm
\topmargin=-1.3cm
%\textwidth=16cm
%\textheight=23cm
%\oddsidemargin=0.cm
%\topmargin=-1.3cm
\def\no{\noindent}
\def\bc{\begin{center}}
\def\ec{\end{center}}
\def\vs{\vskip0.5cm}
\def\beq{\begin{equation}}
\def\eeq{\end{equation}}
\def\cl{\centerline}
\def\d{\downarrow}
\def\u{\uparrow}
\def\doub{\downarrow\uparrow}
\def\bj{{\bf j}}
\def\bl{{\bf k}}
\def\br{{\bf r}}
\def\bq{{\bf q}}
\def\bk{{\bf k}}
\def\bareta{{\bar\eta}}

\begin{document}

\title{{\it Supplementary Material}:\\
Short note on the density of states in 3D Weyl semimetals
}

\author{K. Ziegler and A. Sinner}
\affiliation{
Institut f\"ur Physik, Universit\"at Augsburg,
D-86135 Augsburg, Germany}
\date{\today}

\maketitle

\section{Lower bound of the average density of states}
\label{sect:lower_bound}

The main idea of calculating the DOS of an infinite system is to divide the large system into 
smaller (finite) cubes, calculate the DOS of these smaller
cubes and estimate the contribution of the boundaries between them. This concept was 
developed for a periodic lattice by Ledermann \cite{ledermann44}. Later it was extended 
to estimate the average DOS of a tight-binding model with random potential by Wegner \cite{wegner81}, 
using the relation between
the DOS and the integrated DOS. The calculational advantage of using a finite cube is its discrete
spectrum. Then the corresponding DOS is a sum of Dirac Delta functions (or poles of the corresponding
Green's function), which can be studied, for instance, by averaging with respect to a continuous 
disorder distribution. Then the Dirac Delta functions contribute to the average with their 
spectral weights.

This concept can be generalized by introducing a generating function for the local DOS,  
which is the phase of a unimodular function \cite{ziegler87,ziegler88,ziegler98}. The phase has special 
properties under the change of the matrix elements of the underlying tight-binding Hamiltonian, 
which leads to a flexible method for estimating the average DOS. 
%It was applied to 
%two-dimensional Dirac fermions with random mass \cite{ziegler87}, S-wave superconductor with 
%random order parameter \cite{ziegler88}, and D-wave superconductor with random chemical potential
%\cite{ziegler98}.

For a diagonal matrix $U$ and a short-range tight-binding matrix $H_0$ with lattice sites $\{\br\}$
there is a generating function 
\beq
F_\Lambda=i\log\left[\frac{\det(H_0-U-i\epsilon)}{\det(H_0-U+i\epsilon)}\right]
\ \ \ {\rm with}\ 
\rho_\br=\frac{1}{2\pi}\frac{\partial F_\Lambda}{\partial U_\br}
\ \ \ (\epsilon>0)
\label{def_F}
\eeq
for the local density of states $\rho_\br$ on the lattice $\Lambda$. 
The specific form of $H_0$ is not important for the following discussion, as long as it
is short ranged. The latter is crucial because it allows us to obtain a sufficiently
thin surface to disconnected cubes on the lattice $\Lambda$. Whether $H_0$ is a symmetric
tight-binding Hamiltonian or a discrete Dirac operator with spinor states does not affect
the validity of the approach.

$F_\Lambda$ has some remarkable properties: It is real, since the argument of the logarithm 
is unimodular, and it is an increasing function for any $U_\br$, since the DOS is non-negative.
$F_\Lambda$ is bounded for the shift of a single variable $U_\br'\to U_\br$ as
\beq
0\le F_\Lambda(U_\br)-F_\Lambda(U_\br')\le 2\pi\ \ (U_\br > U_\br')
\label{diff1}
\eeq
and for $n$ shifted variables 
$U_{\br_1}',U_{\br_2}',...,U_{\br_n}'\to U_{\br_1},U_{\br_2},...,U_{\br_n}$ it is bounded as
\beq
0\le F_\Lambda(U_{\br_1},U_{\br_2},...,U_{\br_n})-F_\Lambda(U_{\br_1}',U_{\br_2}',...,U_{\br_n}')
\le 2\pi n\ \ (U_{\br_j} > U_{\br_j}')
\ .
\label{diff2}
\eeq
$F_\Lambda$ is additive on the lattice in the limit $U_\br\to\infty$ on $\partial S$: 
\beq
\lim_{U_\br\to\infty,\ \br\in \partial S}F_\Lambda= F_S+F_{S''}
\label{add1}
\eeq
for a cube $S$ with the boundary $\partial S$ and the complement $S''$ outside $S\cup\partial S$.
$F_S$ ($F_{S''}$) is the function $F_\Lambda$ with $H_0-U\pm i\epsilon$ projected onto the subspace 
$S$ ($S''$).
The combination of (\ref{diff2}) and (\ref{add1}) implies
\beq
F_S+F_{S''}-2\pi |\partial S|\le F_\Lambda\le F_S+F_{S''} 
\ ,
\label{ineq1}
\eeq
where $|\partial S|$ is the number of lattice sites in $\partial S$.

Before we discuss the lower bound of the average DOS, an interpretation of the generating
function $F_\Lambda$ might be useful. According to its definition in Eq. (\ref{def_F})
$F_\Lambda/2$ is the phase of the determinant of $H_0-U-i\epsilon$. If we increase a single
$U_\br$ this phase also increases, as indicated by (\ref{diff1}). Since this happens for
the increase of any of elements of $U$, the phase changes add up and the shift
$U_{\br_1}',U_{\br_2}',...,U_{\br_n}'\to U_{\br_1},U_{\br_2},...,U_{\br_n}$ provides a winding
number of the determinant. The change of all elements of $U$ by a constant $E$ leads to
the integrated DOS on the interval $[0,E]$:
\beq
N(0,E)=\int_0^E\sum_\br\rho_\br(U+y)dy
\ ,
\eeq
which is the number of eigenvalues of $H_0-U$ on the interval $[0,E]$. In other words, the
winding number of a global change of $U$ is equal to the number of eigenstates on the interval
of the change.  

Now we return to the average DOS
\beq
\sum_{\br\in S}{\bar \rho}_\br
%{\bar \rho}_\br 
:=\sum_{\br\in S}\int\rho_\br(U) \prod_{\br\in \Lambda}P(U_\br)dU_\br
=\frac{1}{2\pi}\int\sum_{\br\in S}\frac{\partial F_\Lambda}{\partial U_\br} 
\prod_{\br\in \Lambda}P(U_\br)dU_\br
\ .
\label{ados}
\eeq
The distribution density on $S$ can be written as an integral transform
\beq
\prod_{\br\in S}P(U_\br)
%=\int_{-v}^v \prod_{\br\in S}P'(U_\br-u)du
=P'(U\Pi_S)\int_{-v}^v \prod_{\br\in S}P(U_\br-u)du
\ ,
\label{int-tr}
\eeq
where $\Pi_S$ is the projector onto $S$.
This gives us for the right-hand side of (\ref{ados})
\[
\frac{1}{2\pi}\int\sum_{\br\in S}\frac{\partial F_\Lambda(U)}{\partial U_\br} 
\int_{-v}^v\prod_{\br\in S}P(U_\br-u)duP'(U\Pi_S)\prod_{\br\in S}dU_\br \prod_{\br\notin S}P(U_\br)dU_\br
\]
and with the new integration variable $U_\br'=U_\br-u$ we have
\[
=\frac{1}{2\pi}\int\int_{-v}^v\sum_{\br\in S}\frac{\partial F_\Lambda(U'+u\Pi_S)}{\partial U_\br}
P'((U'+u)\Pi_S)du 
\prod_{\br\in S}P(U_\br')dU_\br'\prod_{\br\notin S}P(U_\br)dU_\br
\]
\beq
\ge \frac{1}{2\pi}\int\inf_{-v\le w\le v}P'((U'+w)\Pi_S)
\int_{-v}^v\sum_{\br\in S}\frac{\partial F_\Lambda(U'+u\Pi_S)}{\partial U_\br}du 
\prod_{\br\in S}P(U_\br')dU_\br'\prod_{\br\notin S}P(U_\br)dU_\br
\eeq
With the relation
\[
\int_{-v}^v\sum_{\br\in S}\frac{\partial F_\Lambda(U'+u\Pi_S)}{\partial U_\br}du
=\int_{-v}^v\frac{\partial F_\Lambda(U'+u\Pi_S)}{\partial u}du
=F_\Lambda(U'+v\Pi_S)-F_\Lambda(U'-v\Pi_S)
\]
we obtain
\beq
\sum_{\br\in S}{\bar \rho}_\br
\ge\frac{1}{2\pi}\int [F_\Lambda(U'+v\Pi_S)-F_\Lambda(U'-v\Pi_S)]
\inf_{-v\le w\le v}P'((U'+w)\Pi_S)\prod_{\br\in S}P(U_\br')dU_\br'\prod_{\br\notin S}P(U_\br)dU_\br
\ .
\eeq
Now we apply the inequalities (\ref{ineq1}) and get a lower bound
\beq
\sum_{\br\in S}{\bar \rho}_\br
\ge\frac{1}{2\pi}\int [F_S(U'+v)-F_S(U'-v)-2\pi|\partial S|]\inf_{-v\le w\le v}P'((U'+w)\Pi_S)
\prod_{\br\in S}P(U_\br')dU_\br'
\ ,
\eeq
where the integration outside of $S$ has been performed, since the integrand
does not depend on $U_\br$ for $\br\notin S$. Next, we estimate the integral as
\[
\frac{1}{2\pi}\int [F_S(U'+v)-F_S(U'-v)-2\pi|\partial S|]\inf_{-v\le w\le v}P'((U'+w)\Pi_S)
\prod_{\br\in S}P(U_\br')dU_\br'
\]
\beq
\ge 
\inf_{-a\le U_\br'\le a,\ \br\in S}[F_S(U'+v)-F_S(U'-v)-2\pi|\partial S|]
\frac{1}{2\pi}\inf_{-v\le w\le v}P'((U'+w)\Pi_S) 
\ .
\eeq
$F_S(U'+v)-F_S(U'-v)$ is the integrated DOS on the cube $S$
\[
\int_{-v}^v\sum_{\br \in S}\rho_{S,\br}(U'+E)dE
\ ,
\]
which counts the number of eigenstates of the $|S|\times|S|$--matrix
$\Pi_S(H_0-U')\Pi_S$ on the interval $[-v,v]$. Finally, from Eq. (\ref{int-tr}) we get
\beq
P'((U'+w)\Pi_S)=\frac{\prod_{\br\in S}P(U_\br'+w)}{\int_{-v}^v\prod_{\br\in S}P(U_\br'+w-u)du}
\ge \prod_{\br\in S}P(U_\br'+w)
\ ,
\label{fin_in}
\eeq
which gives inequality (19) of the Letter.


\section{Derivation of properties (\ref{diff2}) and (\ref{add1})}
\label{sect:properties}

As discussed above, we obtain a lower bound of the average DOS essentially through properties  
(\ref{diff2}) and (\ref{add1}) of the generating function $F_\Lambda$.
These properties were discussed previously in Refs. \cite{ziegler87}--\cite{ziegler98}
but for a consistent notation we summarize them in the following. 


\subsection{Property (\ref{diff2})}

The inequality (\ref{diff1}) for one shifted variable is directly related to the Lippmann-Schwinger 
equation for a single impurity via
\beq
F_\Lambda(U_\br')-F_\Lambda(U_\br'') = 2\pi\int_{U_\br''}^{U_\br'} \rho_\br(U_\br)dU_\br
=-i\int_{U_\br''}^{U_\br'}
\left[\frac{1}{1/\gamma_\br^*-U_\br}-\frac{1}{1/\gamma_\br-U_\br}\right]dU_\br\le 2\pi
\ ,
\label{diff1a}
\eeq
where $\gamma_\br=G_{0,\br\br}$ is the spatial diagonal element of the Green's function. The integral 
is also non-negative because the imaginary part of $\gamma_\br$ is positive for $\epsilon>0$.
A special case is that of Weyl particles because of $\gamma_\br\sim 0$. Then the above expression is
always zero for finite $U_\br$, $U_\br'$ because the pole of the integrand is at infinity, as mention in
the Letter.

Then we apply two times (\ref{diff1a}) to obtain for two shifted variables 
\[
0\le F_\Lambda(U_{\br_1},U_{\br_2})-F_\Lambda(U_{\br_1}',U_{\br_2}')
= F_\Lambda(U_{\br_1},U_{\br_2})-F_\Lambda(U_{\br_1}',U_{\br_2})
+F_\Lambda(U_{\br_1}',U_{\br_2})-F_\Lambda(U_{\br_1}',U_{\br_2}')
\le 4\pi
\ .
\]
This procedure can be repeated $n$ times for $n$ shifted variables to give (\ref{diff2}). 
The result is justified by complete induction: Suppose (\ref{diff2}) is correct. Then we get for $n+1$
\[
F_\Lambda(U_{\br_1},U_{\br_2},...,U_{\br_{n+1}})-F_\Lambda(U_{\br_1}',U_{\br_2}',...,U_{\br_{n+1}}')
=F_\Lambda(U_{\br_1},U_{\br_2},...,U_{\br_{n+1}})-F_\Lambda(U_{\br_1}',U_{\br_2}',...,U_{\br_n}',U_{\br_{n+1}})
\]
\[
+F_\Lambda(U_{\br_1}',U_{\br_2}',...,U_{\br_n}',U_{\br_{n+1}})
-F_\Lambda(U_{\br_1}',U_{\br_2}',...,U_{\br_n}',U_{\br_{n+1}}')
\le 2\pi n +2\pi=2\pi(n+1)
\ .
\]


\subsection{Property (\ref{add1})}

The relation (\ref{add1}) can be derived by splitting $U=U_{S\cup S''}+U_{\partial S}$ with
the projectors $\Pi_S$, $\Pi_{S''}$ and $\Pi_{\partial S}$ onto $S$, $S''$ and $\partial S$,
respectively:
\[
U_{S\cup S''}=\Pi_SU\Pi_S+\Pi_{S''}U\Pi_{S''}-\Pi_S-\Pi_{S''}
\ ,\ \ 
U_{\partial S}=\Pi_{\partial S}U\Pi_{\partial S}+\Pi_S+\Pi_{S''}
\ .
\]
Then we obtain the following equations
\[
\frac{\det(H_0-U-i\epsilon)}{\det(H_0-U+i\epsilon)}
=\frac{\det(H_0-U_{S\cup S''}-U_{\partial S}-i\epsilon)}
{\det(H_0-U_{S\cup S''}-U_{\partial S}+i\epsilon)}
\]
\beq
=\frac{\det\{U_{\partial S}^{1/2}[
U_{\partial S}^{-1/2}(H_0-U_{S\cup S''}-i\epsilon)U_{\partial S}^{-1/2}-{\bf 1}]U_{\partial S}^{1/2}\}}
{\det\{U_{\partial S}^{1/2}[
U_{\partial S}^{-1/2}(H_0-U_{S\cup S''}+i\epsilon)U_{\partial S}^{-1/2}-{\bf 1}]U_{\partial S}^{1/2}\}}
=\frac{\det[
U_{\partial S}^{-1/2}(H_0-U_{S\cup S''}-i\epsilon)U_{\partial S}^{-1/2}-{\bf 1}]}
{\det[
U_{\partial S}^{-1/2}(H_0-U_{S\cup S''}+i\epsilon)U_{\partial S}^{-1/2}-{\bf 1}]}
\ .
\eeq
The limit $U_\br\to\infty$ on $\partial S$ gives
\beq
\lim_{U_\br\to\infty,\ \br\in \partial S}U_{\partial S}^{-1/2}=\Pi_S+\Pi_{S''}
\ .
\eeq
Since $\partial S$ separates two regions $S$ and $S''$ on the lattice
with $\Pi_SH_0\Pi_{S''}=0$, this leads to
\[
\lim_{U_\br\to\infty,\ \br\in \partial S} U_{\partial S}^{-1/2}(H_0-U_{S\cup S''}\pm i\epsilon)U_{\partial S}^{-1/2}
=(\Pi_S+\Pi_{S''})(H_0-U_{S\cup S''}\pm i\epsilon)(\Pi_S+\Pi_{S''})
\] 
and, because of $\Pi_SH_0\Pi_{S''}=0$, we get
\beq
=\Pi_{S}(H_0-U\pm i\epsilon)\Pi_{S}+\Pi_{S''}(H_0-U\pm i\epsilon)\Pi_{S''}
\ .
\eeq
For the ratio of determinants we have
\beq
\lim_{U_\br\to\infty,\ \br\in \partial S}
\frac{\det(H_0-U-i\epsilon)}{\det(H_0-U+i\epsilon)}
=\frac{\det_S(H_0-U-i\epsilon)}{\det_S(H_0-U+i\epsilon)}
\frac{\det_{S''}(H_0-U-i\epsilon)}{\det_{S''}(H_0-U+i\epsilon)}
\ ,
\eeq
where the index of the determinants refers to the projection of the matrix.
Inserting this result into $F_\Lambda$ gives Eq. (\ref{add1}).


\begin{thebibliography}{99}

\bibitem{ledermann44}
W. Ledermann, Proc. R. Soc. London {\bf 182}, 362 (1944).

\bibitem{wegner81}
F. Wegner, Z. Physik B - Condensed Matter {\bf 44}, 9 (1981).

\bibitem{ziegler87}
K. Ziegler, Nucl. Phys. B {\bf 285} [FS19], 606 (1987).

\bibitem{ziegler88}
K. Ziegler, Commun. Math. Phys. {\bf 120}, 177 (1988).

\bibitem{ziegler98}
K. Ziegler, M.H. Hettler, P.J. Hirschfeld, Phys. Rev. B {\bf 57}, 10825 (1998).

\end{thebibliography}

\end{document}
