
\documentclass[12pt]{article}
\let\OLDthebibliography\thebibliography
\renewcommand\thebibliography[1]{
    \OLDthebibliography{#1}
    \setlength{\parskip}{0pt}
    \setlength{\itemsep}{0pt plus 0.3ex}
}

\pagestyle{plain}
\usepackage{multirow}
\usepackage[margin=0.9in]{geometry}
\usepackage[utf8]{inputenc}
\usepackage{booktabs}
\usepackage{caption}
\usepackage{subcaption}
\usepackage{adjustbox}
\usepackage{lipsum}
\usepackage[parfill]{parskip}
\usepackage{pgfplots}
\usepackage[labelfont=bf]{caption}


\title{\large \textbf{Integrated Optimization of Predictive and Prescriptive Tasks}\thanks{\href{https://www.abstractsonline.com/pp8/\#!/6818/presentation/11528}{\color{red}{This work was presented in INFORMS Annual Meeting, Seattle, Washington, October 20-23, 2019.}}}}

\author{Mehmet Kolcu\\ 
{\small Industrial and Systems Engineering, Wayne State University, Detroit, MI 48201}\\
{\small mehmet.kolcu@wayne.edu}\\
\and 
Alper E. Murat\\ 
{\small Industrial and Systems Engineering, Wayne State University, Detroit, MI 48201}\\
{\small amurat@wayne.edu}\\
}
\usepackage{graphicx}
\graphicspath{ {./images/} }
\usepackage{amsmath}
\usepackage[bookmarks]{hyperref}
\usetikzlibrary{positioning,fit,arrows.meta,backgrounds}
\usepackage[linesnumbered,ruled,vlined]{algorithm2e}
\DeclareMathOperator*{\argmin}{argmin}
\usepackage{setspace} 
\captionsetup[subfigure]{font=scriptsize,labelfont=scriptsize}
\usepackage{amsfonts}
\pgfplotsset{compat=1.16} 
\tikzset{
		module/.style={%
		draw, rounded corners,
        minimum width=#1,
        minimum height=7mm,
        font=\sffamily
        },
    module/.default=2cm,
    >=LaTeX
}
\begin{document}
\maketitle
\doublespacing
\begin{abstract}
\noindent In traditional machine learning techniques, the degree of closeness between true and predicted values generally measures the quality of predictions.  However, these learning algorithms do not consider prescription problems where the predicted values will be used as input to decision problems. In this paper, we efficiently leverage feature variables, and we propose a new framework directly integrating predictive tasks under prescriptive tasks in order to prescribe consistent decisions. We train the parameters of predictive algorithm within a prescription problem via bilevel optimization techniques.  We present the structure of our method and demonstrate its performance using synthetic data compared to classical methods like point-estimate-based, stochastic optimization and recently developed machine learning based optimization methods. In addition, we control generalization error using different penalty approaches, and optimize the integration over validation data set.\\
% In addition, the similarities and differences between our bilevel framework and other proposed methodologies will be explained in a mathematical manner.\\
\noindent \textit{\textbf{Keywords}}: Prescriptive, Predictive, Regression, Progressive Hedging Algorithm, Machine Learning, Bilevel Optimization. \\
\end{abstract}

\section{Introduction}
Today we are living in a world called the information age. The exponential growth of data availability, ease of accessibility in computational power, and more efficient optimization techniques have paved the way for massive developments in the field of predictive analytic. Particularly when organizations have realized the benefits of predictive methods in improving their efficiency and gaining an advantage over their competitors, these predictive techniques become more powerful. There is a mutual relationship between predictive and prescriptive analytics. We cannot deny the role of optimization techniques while obtaining predictive models because most of the predictive models are trained over the minimization of a loss or maximization of a gain function. On the other hand, prescriptive models containing uncertainty need the estimated inputs from predictive analytics to handle uncertainty in optimization parameters. Although these statistical (machine) learning methods have been provided well-aimed predictions for uncertain parameters in many different fields of science, scenario-based stochastic optimization introduced by \cite{RePEc:inm:ormnsc:v:1:y:1955:i:3-4:p:197-206} and similar works in \cite{10.5555/2031490}, \cite{SHAPIRO2003353}, \cite{doi:10.1137/S1052623499363220}, and \cite{Shapiro2005} are widely preferred techniques in order to tackle uncertainty in decision problems. One of the reasons why these stochastic or robust optimization methods or similar solutions provided by \cite{BEN:09} and \cite{doi:10.1137/080734510} do perform better than point estimate-based optimization is that training process of statistical learning methods does not take into account optimal actions because traditional learning algorithms measure prediction quality based on the degree of closeness between true and predicted values. This gap between prescriptive and predictive analytics leads point estimate-based decisions to a failure in the prescription phase.

There are different variables and parameters in prediction and prescription tasks. Prediction tasks mainly have components such as feature data, response data, and the parameters connecting these two via a function. Prescription problems generally have cost vector parameters in the objective function, right-hand-side and left-hand-side parameters in the constraints, and decision variables. Our interest in this paper is to fit a predictive function with responses and feature data, and feed prescriptive model parameters by this fitted function.

For the sake of uniformity, we use the same notation the rest of this paper. We indicated feature data with $X$, responses with $Y$, parameter of predictive algorithm with $\beta$, and decision variables with $Z$. Responses ($Y$) are the bridge parameter connecting predictive and prescriptive tasks.

In order to visualize our motivation, imagine a newsvendor problem with a cost function, $Cost=C_h(Z-Y)^++C_b(Y-Z)^+$, containing holding cost ($C_h$) and backordering cost ($C_b$) with demand parameter ($Y$). In the company of feature data, a predictive regression model can be built for future demand as $\hat{Y}=\Psi(\Tilde{X},\hat{\beta}^*)$ where $\hat{\beta}^*=\argmin\limits_{\beta \in \mathbb{R}^{p}} (\Tilde{Y}-\Psi(\Tilde{X},\beta))^T(\Tilde{Y}-\Psi(\Tilde{X},\beta))$ in order to minimize cost function.
However, the training criteria of this predictive regression model will be based on closeness between true and predicted responses via a loss function, the predictive regression model will not capture the effect of holding and backordering costs in the cost function, and future predictions will most probably fail in prescription stage. In Figure \ref{plot:graph1}\subref{plot:graph11}, one predictive regression and two different prescriptive regression models (considering holding and backordering costs) are presented based on two different scenario. When backordering cost is greater than holding cost, using prescriptive regression 1 gives a lower cost on the average since it keeps predictions higher in order to avoid shortage cost as seen in Figure \ref{plot:graph1}\subref{plot:graph12}. Similarly, when holding cost is greater than backordering cost, using prescriptive regression 2 gives a lower cost on the average since it keeps predictions lower in order to avoid holding cost as seen in Figure \ref{plot:graph1}\subref{plot:graph13}. The question is how to obtain such a predictive model directly caring the characteristics of prescriptive model.\\
In this paper, our purpose is to build a framework for predictive regression model caring the characteristics of prescriptive model and providing best decisions. We will call our framework as "Integrated Predictive and Prescriptive Optimization" (IPPO). While IPPO is seeking the most accurate predictive regression model, the desired predictive regression model will tackle uncertainty by providing best actions in prescription stage.
\begin{figure}
\makebox[\linewidth][c]{%
\begin{subfigure}[b]{.42\textwidth}
\centering
\includegraphics[width=.95\textwidth]{PredictiveRegvsPrescriptiveReg1}
\caption{Regression Models}
\label{plot:graph11}
\end{subfigure} \hspace{-5mm}%
\begin{subfigure}[b]{.42\textwidth}
\centering
\includegraphics[width=.95\textwidth]{PredictiveRegvsPrescriptiveReg2}
\caption{Predictive Reg. vs Prescriptive Reg 1 }
\label{plot:graph12}
\end{subfigure} \hspace{-5mm}%
\begin{subfigure}[b]{.42\textwidth}
\centering
\includegraphics[width=.95\textwidth]{PredictiveRegvsPrescriptiveReg3}
\caption{Predictive Reg. vs Prescriptive Reg 2 }
\label{plot:graph13}
\end{subfigure}%
}\\
\caption{Comparison of Predictive and Prescriptive Regression Models Based on Cost Function}
\label{plot:graph1}
\end{figure}
\section{Related Work}
There have been recent developments in supervised machine learning algorithms (classification and regression tasks), but no matter what kind of learning algorithm is employed, the common and final task is generally to train these algorithms in order to find the closest predictions to true values. 
Latest works in the literature show that scientists are realizing that training machine learning models solely based on a prediction criterion like  mean squared error, mean absolute error, or log-likelihood loss is not enough when these noisy predictions will be parameters of prescription tasks. Bengio \cite{Bengio1997UsingAF} is one of the first works considering this issue and building an integrated framework for both prediction and prescription tasks, he emphasized the importance of evaluation criteria of predictive tasks. His neural network model was not designed to minimize prediction error, but instead to  maximize a financial task where those noisy predictions are used as input. 
Another integrated framework developed by Kao et al.\cite{NIPS2009_3686} trains parameters of regression model based on an unconstrained optimization problem with quadratic cost function. It is a hybrid algorithm between ordinary least square and empirical optimization. Tulabandhula and Rudin \cite{Tulabandhula2011MachineLW} minimizes a weighted combination of prediction error and operational cost, but operational cost must be assessed based on true parameters. Bertsimas and Kallus \cite{doi:10.1287/mnsc.2018.3253} add a new dimension to this field by introducing the conditional stochastic optimization term. This simple but efficient idea leverages non-parametric machine learning methods in order to assign weights into train data points for a given test data point, then calculates optimal decisions over stochastic optimization based on calculated weights. However, this methodology is using machine learning tools outside of prescription problem. 
A different approach developed by Ban and Rudin \cite{Ban2019TheBD} considers the decision variables as a function of auxiliary data, but this method may fail if the connection between optimal decisions and parameters in a constrained optimization problem, or may end up with indefeasible decisions. Oroojlooyjadid et al.\cite{Oroojlooyjadid2016ApplyingDL} and Zhang and Gao \cite{Zhang2017AssessingTP} built an extension of \cite{Ban2019TheBD} since both papers approach to the solution from the same perspective, but they solve the problem via neural network to capture the non-linearity between auxiliary data and decisions. 
One more neural network based integrated task is proposed by Donti et al. \cite{NIPS2017_7132}, and primarily they focus on quadratic stochastic optimization problems since they are tuning neural network parameters by differentiating the optimization solution to a stochastic programming problem.
One of the latest works in integrating predictive and prescriptive tasks is developed by Elmachtoub and Grigas  \cite{Elmachtoub2017SmartT} which aims to find parameter of linear regression model inside of decision problem via a new loss function. This method minimizes the difference between objective value provided by true parameters and objective value where decisions provided by predicted parameters are assessed.

\section{Integrated Predictive and Prescriptive Optimization}
Given a decision problem (DP) with parameter uncertainty, our goal is to estimate the predictive relationship between responses (uncertain parameters of the decision problem) and a set of input features such that the prescriptive modeling of the decision problem using the predicted responses results in the best decisions. We consider decision problems that can be formulated as an optimization model. Further, the statistical relationship between uncertain parameters (Y) and input features (X) can be approximated through a parametric regression model, i.e., $Y=\Psi(X,\beta)+\epsilon$, where $\beta$ is a vector of $k$ parameters and $\epsilon$ is an error term and $\Psi(\cdot)$ is some function describing the relationship between Y and X. Without loss of generality, given the estimates $\hat{Y}=(\hat{y}_o,\hat{y}_c)$ of uncertain parameters $Y=(y_o,y_c)$, we define the deterministic decision problem (DP) as follows:
\begin{subequations}
\begin{flalign}
\min\limits_{{z \in Z}} \hspace{2 mm} & f(z;\hat{y}_o) \label{eq:eq1-4}\\
\textrm{s.t.} \hspace{2 mm} & g_i(z;\hat{y}_c)\leq 0 \hspace{20 mm} \forall i \in I \hspace{5 mm} c_1\\
& h_j(z;\hat{y}_c)= 0 \hspace{20 mm} \forall j \in J \hspace{3.5 mm} c_2
\end{flalign}
\end{subequations}
In the above formulation, $z$ denote the decision variables, and $\hat{Y}=(\hat{y}_o,\hat{y}_c)$ denote the estimates of the uncertain parameters in the objective and constraints set, respectively. Let ($\hat{z}$) denote the optimal solution of DP given $\hat{Y}=(\hat{y}_o,\hat{y}_c)$, i.e. $\hat{z}=\argmin\{f(z;\hat{y}_o)|z \in Z$ satisfying $c_1$ and $c_2$\}. The uncertain parameters in DP are estimated through a parametric regression model. Let’s assume given the historical data $D=\{(\Tilde{X}^n,\Tilde{Y}^n)\}^N_{n=1}=(\Tilde{X},\Tilde{Y})$ for the response Y and explanatory features X, the prediction problem (PP) using parametric regression is expressed as:
\begin{flalign}
\hat{\beta}^* & =\argmin\limits_{\beta \in \mathbb{R}^{p}} (\Tilde{Y}-\Psi(\Tilde{X},\beta))^T(\Tilde{Y}-\Psi(\Tilde{X},\beta)) \label{eq:eq1-5}
\end{flalign}
In classical approach, the predictive (PP) and prescriptive (DP) tasks which are often treated independently and often in a sequence, i.e., first predict $\hat{Y}=\Psi(\Tilde{X},\hat{\beta}^*)$ (using PP) and then prescribe (using DP). This process is illustrated in Figure \ref{plot:graph2_1}.\\
\begin{figure}[ht]
\centering
\includegraphics[width=8.45cm, height=3.79cm]{independentmethodology}
% \includegraphics[width=12.5cm, height=5.5cm]{bilevel_methodology}
\caption{Independent Framework}
\label{plot:graph2_1}
\end{figure}
\newpage
We herein develop an integrated framework joining the predictive (PP) and prescriptive (DP). Three modules of the integration framework are as follows:
\begin{enumerate}
\item Given a set of independent features, a predictive regression model generating responses which are input to the optimization model as part of the input parameter set (Module 1),
\item An optimization model prescribing decisions based on input parameters (Module 2),
\item Another optimization model evaluating the quality of prescribed decisions with respect to ground truth in the response space and updating the parameters of the predictive model (Module 3).
\end{enumerate}
Figure \ref{plot:graph2_2} illustrates these three modules. The sequential predictive and prescriptive tasks (modules 1 and 2) are concurrently optimized through the module 3. While module 1 is a prediction model, modules 2 and 3 are decision optimization problems with their respective decisions influencing one another. The embedding structure of modules 1 and 2 within module 3 is similar to those of bilevel optimization problems.\\
Hence, we model the integration framework as a nested optimization model. In the next section we model the integrated prediction and prescription problem as a bilevel optimization model.

% Figure \ref{plot:graph2_2} represents the integration of these three steps. Although Figure \ref{plot:graph2_2} depicts predicted and prescriptive tasks in a sequential manner, our purpose is to build a mathematical framework that optimize these two tasks in an integrated way. Step 2 and Step 3 require two separate optimization model, but decisions taking place in both steps influence each other. At this point, we investigate structure of bilevel optimization models since bilevel models have a similar structure of our intended model.
\begin{figure}[ht]
\centering
\includegraphics[width=13.63cm, height=6.11cm]{bilevelmethodology1}
\caption{Integrated Framework}
\label{plot:graph2_2}
\end{figure}

\subsection{Bilevel Models and Solving Techniques}
Bilevel problems are nested optimization problems where an upper-level optimization problem is constrained by a lower-level optimization problem \cite{Bard1998PracticalBO}. A common application of the bilevel problems is a static leader-follower game in economics \cite{Stackelberg}, where the upper level decision maker (leader) has complete knowledge of the lower level problem (follower). Decision variables of the upper level serve as parameters of the lower level. 

We model the integrated framework for the prediction and prescription tasks as a bilevel optimization problem. The upper-level problem jointly determines the parameters of the regression model ($\beta$) and prescription decisions. At lower level, we make decisions with the help of predicted parameters, and we evaluate these actions with respect to true parameters by fixing those "here-and-now" decisions ($Z_f^l$) at upper level problem in constraint \ref{eq:eq1-3}, so that we integrate all proposed steps in one framework formulated as in (\ref{eq:eq1-1})-(\ref{eq:eq1-2}).
% \begin{subequations}
% \begin{flalign}
% \min\limits_{{Z_f^u,Z_s^u,\beta}} \hspace{2 mm} & F(Z_f^u,Z_s^u) \label{eq:eq1-1}\\
% \textrm{s.t.} \hspace{2 mm} & G_i(Z_f^u,Z_s^u;Y)\leq 0 \hspace{49 mm} \forall i \in I\\
% & H_j(Z_f^u,Z_s^u;Y)=0 \hspace{49 mm} \forall j \in J\\
% & Z_f^u=Z_f^l \label{eq:eq1-3}\\
% & \beta \quad Free,\quad Z_f^u,Z_s^u \geq 0\\
% &\min\limits_{Z_f^l,Z_s^l}  \hspace{3 mm} F(Z_f^l,Z_s^l)\\
% &\hspace{2.5 mm} \textrm{s.t.} \hspace{3.5 mm} G_i(Z_f^l,Z_s^l;\hat{Y}=\psi(\Tilde{X},\beta))\leq 0 \hspace{19 mm} \forall i \in I\\
% &\hspace{12 mm} H_j(Z_f^l,Z_s^l;\hat{Y}=\psi(\Tilde{X},\beta))= 0 \hspace{18 mm} \forall j \in J\\
% &\hspace{12 mm} Z_f^l,Z_s^l\geq 0 \label{eq:eq1-2}
% \end{flalign}
% \end{subequations}
\begin{subequations}
\begin{flalign}
\min\limits_{{Z_f^u,Z_s^u,\beta}} \hspace{2 mm} & F(Z_f^u,Z_s^u) \label{eq:eq1-1}\\
\textrm{s.t.} \hspace{2 mm} & G_i(Z_f^u,Z_s^u;Y)\leq 0 \hspace{49 mm} \forall i \in I\\
& H_j(Z_f^u,Z_s^u;Y)=0 \hspace{49 mm} \forall j \in J\\
& Z_f^u=Z_f^l \label{eq:eq1-3}\\
&\min\limits_{Z_f^l,Z_s^l}  \hspace{3 mm} F(Z_f^l,Z_s^l)\\
&\hspace{2.5 mm} \textrm{s.t.} \hspace{3.5 mm} G_i(Z_f^l,Z_s^l;\hat{Y}=\psi(\Tilde{X},\beta))\leq 0 \hspace{19 mm} \forall i \in I\\
&\hspace{12 mm} H_j(Z_f^l,Z_s^l;\hat{Y}=\psi(\Tilde{X},\beta))= 0 \hspace{18 mm} \forall j \in J \label{eq:eq1-2}
\end{flalign}
\end{subequations}

% Bilevel problems are nested optimization problems where an optimization problem is constrained by another one. Independent decision makers have conflicting objective functions, and they influence decision of each other. Decision variables of each level act like parameters in other level. Thus, the bilevel problems can be interpreted as a static Stackelberg game \cite{Stackelberg}. The leader take an action first by selecting a vector $z^u \in Z^U$ to optimize upper level objective value $F (z^u,z^l)$, then follower take its own action by selecting a vector $z^l \in Z^L$ to optimize lower level objective value $f (z^u,z^l)$ based on leader's decision, so follower affects the leader's objective value. In literature, bilevel optimization models are generally formulated as in (\ref{eq:eq6})-(\ref{eq:eq7});\\
% \begin{subequations}
% \begin{flalign}
% \min\limits_{z^u \in Z^U} \hspace{2 mm} & F(z^u,z^l) \label{eq:eq6}\\
% \textrm{s.t.} \hspace{3 mm} & G_i(z^u,z^l)\leq 0 \hspace{22 mm} \forall i \in I\\
% & H_j(z^u,z^l)= 0 \hspace{21.6 mm} \forall j \in J\\
% &\min\limits_{z^l \in Z^L} \hspace{2 mm} f(z^u,z^l)\\
% &\hspace{3 mm} \textrm{s.t.} \hspace{4 mm} g_m(z^u,z^l)\leq 0 \hspace{9.4 mm} \forall m \in M\\
% &\hspace{12.5 mm} h_n(z^u,z^l)= 0 \hspace{9.8 mm} \forall n \in N \label{eq:eq7}
% \end{flalign}
% \end{subequations}
The most popular solution technique for bilevel optimization problems is to transform bilevel problem into single level problem by replacing objective function of lower level problem with Karush–Kuhn–Tucker (KKT) conditions \cite{Bard1998PracticalBO}. The KKT conditions appear as dual and complementarity constraints, that is why KKT conditions require convexity, so this approach is limited to convex lower level problems. Complementarity constraints (\ref{eq:eq2-2} and \ref{eq:eq2-3})  convert the problem nonlinear models; thus, these constraints are replaced with logic constraints by defining new binary variables and sufficiently enough an M parameter. With this final touch, bilevel model turns into a mixed-integer problem, and it can be solved by traditional solvers as in form (\ref{eq:eq8})-(\ref{eq:eq9}).\\
\vspace{-10mm}
\begin{subequations}
\begin{flalign}
\min\limits_{\substack{Z_f^u,Z_s^u,Z_f^l,Z_s^l; \\ \beta,\pi_m,\mu_n}} \hspace{2 mm} & F(Z_f^u,Z_s^u) \label{eq:eq8}\\
\textrm{s.t.} \hspace{8 mm} & G_i(Z_f^u,Z_s^u;Y)\leq 0 \hspace{22 mm} \forall i \in I\\
& H_j(Z_f^u,Z_s^u;Y)= 0 \hspace{21.5 mm} \forall j \in J\\
& Z_f^u=Z_f^l \hspace{39 mm} \label{eq:eq2-1}\\
& G_i(Z_f^l,Z_s^l;\hat{Y})\leq 0 \hspace{23 mm} \forall i \in I\\
& H_j(Z_f^l,Z_s^l;\hat{Y})= 0 \hspace{22 mm} \forall j \in J\\
& \nabla_{Z_f^l,Z_s^l} \hspace{1 mm} L(Z_f^l,Z_s^l,\pi_i,\mu_j)=0\\
& G_i(Z_f^l,Z_s^l;\hat{Y})\pi_i =0 \hspace{20 mm} \forall i \in I \label{eq:eq2-2}\\
& H_j(Z_f^l,Z_s^l;\hat{Y})\mu_j =0 \hspace{19 mm} \forall j \in J \label{eq:eq2-3}\\
& \pi_i \geq 0,\quad \mu_j \geq 0 \hspace{26.2 mm} \forall i \in I,\quad \forall j \in J \label{eq:eq9}\\
where \hspace{5 mm}& \nonumber \\
& L(Z_f^l,Z_s^l,\pi_i,\mu_j)=F(Z_f^l,Z_s^l)+\sum\limits_{i=1}^{I} \pi_i G_i(Z_f^l,Z_s^l;\hat{Y})+\sum\limits_{j=1}^{J} \mu_j H_j(Z_f^l,Z_s^l;\hat{Y}) \nonumber\\
& \hat{Y}=\psi(\Tilde{X},\beta) \nonumber
\end{flalign}
\end{subequations}
\newpage
\subsection{Controlling Generalization Error in IPPO}
\label{sec:ControllingGeneralizationErrorinIPPO}
Our model is developed based on finding best decision in train data set. In order to ensure quality of prediction model in external data set, we propose three different ways to control generalization error within this framework by regularizing predictive model parameters. The first method is to rewrite objective function as weighted average of predictive error and prescriptive error as shown in formulation (\ref{eq:eq3-1})-(\ref{eq:eq3-2}) where $0 \leq \lambda_1 \leq 1$. When $\lambda_1=1$, we solve pure bilevel optimization model without generalization error, and when $\lambda_1=0$, we ignore the prescription part, and optimize directly predictive algorithm solely, and that leads us to point estimate-based prescriptions.
\begin{subequations}
\begin{flalign}
\min\limits_{{Z_f^u,Z_s^u,\beta}} \hspace{2 mm} & \lambda_1F(Z_f^u,Z_s^u)+(1-\lambda_1) L(Y,\hat{Y}=\psi(\Tilde{X},\beta)) \label{eq:eq3-1}\\
\textrm{s.t.} \hspace{2 mm} & G_i(Z_f^u,Z_s^u;Y)\leq 0 \hspace{49 mm} \forall i \in I\\
& H_j(Z_f^u,Z_s^u;Y)=0 \hspace{49 mm} \forall j \in J\\
& Z_f^u=Z_f^l\\
&\min\limits_{Z_f^l,Z_s^l}  \hspace{3 mm} F(Z_f^l,Z_s^l)\\
&\hspace{2.5 mm} \textrm{s.t.} \hspace{3.5 mm} G_i(Z_f^l,Z_s^l;\hat{Y}=\psi(\Tilde{X},\beta))\leq 0 \hspace{19 mm} \forall i \in I\\
&\hspace{12 mm} H_j(Z_f^l,Z_s^l;\hat{Y}=\psi(\Tilde{X},\beta))= 0 \hspace{18 mm} \forall j \in J \label{eq:eq3-2}
\end{flalign}
\end{subequations}
The second method uses predictive error term again, but in a way that it can be restricted by a constraint. However this restriction cannot be less than the loss value which is provided by predictive model solely because constraining loss less than the optimal ($L^*$) makes optimization model indefeasible, this method is shown in formulation (\ref{eq:eq4-1})-(\ref{eq:eq4-2}) with a restriction parameter $\lambda_2 \geq 1$.\\
\begin{subequations}
\begin{flalign}
\min\limits_{{Z_f^u,Z_s^u,\beta}} \hspace{2 mm} & F(Z_f^u,Z_s^u) \label{eq:eq4-1}\\
\textrm{s.t.} \hspace{2 mm} & G_i(Z_f^u,Z_s^u;Y)\leq 0 \hspace{49 mm} \forall i \in I\\
& H_j(Z_f^u,Z_s^u;Y)=0 \hspace{49 mm} \forall j \in J\\
& L(Y,\hat{Y}=\psi(\Tilde{X},\beta)) \leq \lambda_2L^*\\
& Z_f^u=Z_f^l\\
&\min\limits_{Z_f^l,Z_s^l}  \hspace{3 mm} F(Z_f^l,Z_s^l)\\
&\hspace{2.5 mm} \textrm{s.t.} \hspace{3.5 mm} G_i(Z_f^l,Z_s^l;\hat{Y}=\psi(\Tilde{X},\beta))\leq 0 \hspace{19 mm} \forall i \in I\\
&\hspace{12 mm} H_j(Z_f^l,Z_s^l;\hat{Y}=\psi(\Tilde{X},\beta))= 0 \hspace{18 mm} \forall j \in J \label{eq:eq4-2}
\end{flalign}
\end{subequations}
The last method is to shrink predictive model parameters by penalizing with a penalty coefficient $\lambda_3 \geq 0$ as \cite{Tibshirani94regressionshrinkage} introduced, this model is formulated in (\ref{eq:eq5-1})-(\ref{eq:eq5-2}).\\
\begin{subequations}
\begin{flalign}
\min\limits_{{Z_f^u,Z_s^u,\beta}} \hspace{2 mm} & F(Z_f^u,Z_s^u)+\lambda_3 \beta^T \beta \label{eq:eq5-1}\\
\textrm{s.t.} \hspace{2 mm} & G_i(Z_f^u,Z_s^u;Y)\leq 0 \hspace{49 mm} \forall i \in I\\
& H_j(Z_f^u,Z_s^u;Y)=0 \hspace{49 mm} \forall j \in J\\
& Z_f^u=Z_f^l\\
&\min\limits_{Z_f^l,Z_s^l}  \hspace{3 mm} F(Z_f^l,Z_s^l)\\
&\hspace{2.5 mm} \textrm{s.t.} \hspace{3.5 mm} G_i(Z_f^l,Z_s^l;\hat{Y}=\psi(\Tilde{X},\beta))\leq 0 \hspace{19 mm} \forall i \in I\\
&\hspace{12 mm} H_j(Z_f^l,Z_s^l;\hat{Y}=\psi(\Tilde{X},\beta))= 0 \hspace{18 mm} \forall j \in J \label{eq:eq5-2}
\end{flalign}
\end{subequations}

\subsection{Proposed Decomposition Method for IPPO}
Bilevel optimization problems are NP-hard, and it is not easy to solve. However, our proposed predictive and prescriptive integrated methodology has a unique feature. All the defined variables belong to its own scenario except the regression parameters. Regression parameters are common for all scenarios. After converting bilevel to a single level problem by applying KKT conditions, our model becomes a two-stage mixed-integer program whose first stage variables are regression parameters. Here, we create copies of regression parameters across all scenarios and make the problem fully scenario-based decomposable, but we need to include a non-anticipativity or implementability constraint to ensure all regression parameters are equal each other for all scenarios. Progressive hedging algorithm (PHA) proposed by \cite{RockWets91} can be used as decomposition techniques for our two stage mixed-integer problem. In our framework, we will provide best candidate solution as initial regression parameters for depicted Figure \ref{plot:graph2_2}, and PHA solves all scenario problems independently, then we wıll update regression parameters ıteratıvely. These steps repeat until a convergence is satisfied.
\begin{subequations}
\begin{flalign}
\min\limits_{z^f \in Z^F,z^s \in Z^S} \hspace{2 mm} & c^fz^f+\sum\limits_{i=1}^{I}c_i^sz_i^s \label{eq:eq10}\\
\textrm{s.t.} \hspace{9 mm} & Az^f \geq b \label{eq:eq11}\\
& T_iz^f+W_iz_i^s \geq r_i \hspace{21.6 mm} \forall i \in I \label{eq:eq12}
\end{flalign}
\end{subequations}
For better understanding, let’s consider the formulation in (\ref{eq:eq10})-(\ref{eq:eq12}), $z^f$ indicates first stage variable, and $z^s_i$ indicates second stage variable for $i^{th}$ scenario.
\begin{subequations}
\begin{flalign}
\min\limits_{\substack{z^f \in Z^F,z^s \in Z^S;\\z^d \in Z^D}} \hspace{2 mm} & \sum\limits_{i=1}^{I}(c^fz^d_i+c_i^sz_i^s) \label{eq:eq13}\\
\textrm{s.t.} \hspace{9 mm} & Az_i^d \geq b \hspace{37 mm} \forall i \in I \label{eq:eq14}\\
& T_iz^d_i+W_iz_i^s \geq r_i \hspace{21.6 mm} \forall i \in I \label{eq:eq15}\\
& z_i^d-z^f=0 \hspace{31 mm} \forall i \in I \label{eq:eq16}
\end{flalign}
\end{subequations}
In the formulation  (\ref{eq:eq13})-(\ref{eq:eq16}), first stage variable is duplicated, and $z^d_i$ variables are created for each scenario, but they are linked via non-anticipativity constraint \ref{eq:eq16}. By relaxing constraint \ref{eq:eq16}, all scenarios can be easily solved in a parallel way. 

PHA iterates and converges to a common solution taking into account all the scenarios belonging to the original problem. We show the details and steps for basic PHA in Algorithm \ref{algo:algo1}. Let $\rho>0$ be penalty factor, $\delta$ be stopping criteria, and $W$ be dual prices for non-anticipativity constraint \ref{eq:eq16}.\\
\begin{algorithm}
\textbf{Initialization} \linebreak
\linebreak
$k=0$ \linebreak
\linebreak
$z^{d,k}_i=\argmin\limits_{z^d \in Z^D,z^s \in Z^S}(c^fz^d_i+c_i^sz_i^s)$ \quad s.t. (\ref{eq:eq14})-(\ref{eq:eq15})\hspace{5 mm} $\forall i \in I$ \linebreak
\linebreak
$\overline{z}^k=\frac{\sum\limits_{i=1}^{I}z^{d,k}_i}{I}$ \linebreak
\linebreak
$w_i^k=\rho(z^{d,k}_i-\overline{z}^k)$  \hspace{5 mm} $\forall i \in I$ \linebreak
\\
\textbf{Iteration Update}\linebreak
\linebreak
$k=k+1$\linebreak
\\
\textbf{Decomposition}\linebreak
\linebreak
$z^{d,k}_i=\argmin\limits_{z^d \in Z^D,z^s \in Z^S}(c^fz^d_i+c_i^sz_i^s+w_i^{k-1}z^d_i+\frac{\rho}{2}(z^d_i-\overline{z}^{k-1})^2$ \quad s.t. (\ref{eq:eq14})-(\ref{eq:eq15}) \hspace{5 mm} $\forall i \in I$ \linebreak
\linebreak
$\overline{z}^k=\frac{\sum\limits_{i=1}^{I}z^{d,k}_i}{I}$ \linebreak
\linebreak
$w_i^k=w_i^{k-1}+\rho(z^{d,k}_i-\overline{z}^k)$  \hspace{5 mm} $\forall i \in I$ \linebreak
\\
\textbf{Convergence Check}\linebreak
\linebreak
If all scenario solutions $z^{d,k}_i$ are equal with at most $\delta$ deviation, stop. Else, go to step 2.\linebreak
\\
\caption{The Progressive Hedging Algorithm}
\label{algo:algo1}
\end{algorithm}

\section{Experimental Setup}
In this part, we discuss why and how we select predictive and prescriptive model, then we will introduce parameters and variables of these two tasks. Next, we explain data creation process step by step, and we will show formulation of integrated predictive and prescriptive task. 

\subsection{Prescriptive and Predictive Model Selection}
We validate the performance of integrated predictive and prescriptive methodology and compare it with various well-known and recently developed methods. We perform numerical experiments on two different prescriptive models. The first one is well-known newsvendor problem used by \cite{Ban2019TheBD}, but we extend it from single product to multi-product newsvendor problem ($d_l=12$ products), and we use different costs for each scenario (production , holding, and backordering) instead of fix costs in order to increase the complexity of problem. 
Extensive form of classical newsvendor problem with multi-product is expressed as formulated in (\ref{eq:eq2})-(\ref{eq:eq17}).
\begin{subequations}
\begin{flalign}
\min\limits_{Q,U,O} \hspace{5 mm} & \frac{1}{J}\sum\limits_{j=1}^{J}\left[ \frac{1}{N}\sum\limits_{n=1}^{N}(c_{n,j}Q_j+b_{n,j}U_{n,j}+h_{n,j}O_{n,j})\right] \label{eq:eq2}\\
\textrm{s.t.} \hspace{5 mm} & U_{n,j} \geq Y_{n,j}-Q_j \hspace{30 mm} \forall j \in J, \forall n \in N\\
&O_{n,j} \geq Q_j-Y_{n,j} \hspace{30 mm} \forall j \in J, \forall n \in N\\
&Q_{j}, U_{n,j},O_{n,j}\geq 0 \hspace{29 mm} \forall j \in J, \forall n \in N \label{eq:eq17}
\end{flalign}
\end{subequations}
\begin{tabular}{ll}
\multicolumn{2}{l}{\textit{\textbf{Decision Variables for Newsvendor Problem}}} \\
$Q_j$&Amount of regular order done in advance for product $j$\\
$U_{n,j}$&Amount of shortage for product $j$ in scenario $n$\\
$O_{n,j}$&Amount of surplus for product $j$ in scenario $n$\\
\multicolumn{2}{l}{} \\
\multicolumn{2}{l}{\textit{\textbf{Parameters  for Newsvendor Problem}}} \\
$c_{n,j}$&Cost of order for product $j$ in scenario $n$\\
$b_{n,j}$&Cost of backordering for product $j$ in scenario $n$\\
$h_{n,j}$&Cost of hold for product $j$ in scenario $n$\\
$Y_{n,j}$&Amount of demand for product $j$ in scenario $n$\\
\end{tabular}\\

The second prescriptive model is two-stage shipment planning problem leveraged by \cite{doi:10.1287/mnsc.2018.3253} where there is a network between $d_w=4$ warehouses and $d_l=12$ locations. The goal is to produce and hold a product at a cost in warehouses to satisfy the future demand of locations. Then the product is shipped, when needed, from warehouses to locations with transportation cost. In case current total supply in warehouses does not satisfy the demand of locations, the last-minute production takes place with a higher cost. The extensive form of two-stage shipment problem is formulated as follow in (\ref{eq:eq18})-(\ref{eq:eq19}).

\begin{subequations}
\begin{flalign}
\min\limits_{Z,T,S} \hspace{5 mm} & \sum\limits_{i=1}^{I}P_1Z_i+\frac{1}{N}\sum\limits_{n=1}^{N}\left[\sum\limits_{i=1}^{I}P_2T_{n,i}+\sum\limits_{i=1}^{I}\sum\limits_{j=1}^{J}C_{n,i,j}S_{n,i,j}\right] \label{eq:eq18}\\
\textrm{s.t.} \hspace{5 mm} & \sum\limits_{i=1}^{I}S_{n,i,j}\geq Y_{n,j} \hspace{31 mm} \forall j \in J,\forall n \in N\\
&\sum\limits_{j=1}^{J}S_{n,i,j}\leq Z_{i}+T_{n,i} \hspace{22 mm} \forall i \in I,\forall n \in N\\
&Z_{i},\quad T_{n,i},\quad S_{n,i,j}\geq 0 \hspace{21 mm} \forall i \in I,\forall j \in J,\forall n \in N \label{eq:eq19}
\end{flalign}
\end{subequations}
\begin{tabular}{ll}
\multicolumn{2}{l}{\textit{\textbf{Decision Variables}}} \\
$Z_{i}$&Amount of production done in advance at warehouse $i$\\
$T_{n,i}$&Amount of production done last minute at warehouse $i$ in scenario $n$\\
$S_{n,i,j}$&Amount of shipment from warehouse $i$ to location $j$ in scenario $n$\\
\multicolumn{2}{l}{} \\
\multicolumn{2}{l}{\textit{\textbf{Parameters}}} \\
$P_1$&Cost of production done in advance at warehouse\\
$P_2$&Cost of production done last minute at warehouses\\
$C_{n,i,j}$&Cost of shipment from warehouse $i$ to location $j$ in scenario $n$\\
$Y_{n,j}$&Amount of demand at location $j$ in scenario $n$\\
\end{tabular}\\

As for predictive model, since we embed the predictive model inside of prescriptive model, we choose linear regression model as in \ref{eq:eq20} to maintain the linearity of prescriptive model. However, other predictive methodologies still can be applied to capture nonlinearity outside of this integrated framework as preprocess, and dimensionality can be reduced between feature variables and responses especially in high dimensional data as built in deep learning, then these converted feature variables can be embedded inside of prescriptive model via linear regression again as described above.
\begin{flalign}
\hat{Y}_{n,j} & =\beta_{j,0}X_{n,0}+\sum\limits_{f=1}^{F}\beta_{j,f}X_{n,f} \quad \forall j \in J \quad \forall n \in N \quad where \quad X_{n,0}=1 \label{eq:eq20}
\end{flalign}

\subsection{Data Generation}
\label{subsec:DataGeneration}
In both experiments, we randomly generate feature variables of predictive model based on a $d_x=3$ dimensional multivariate normal distribution with size of $n=2000$ observations, $X \in \mathbb{R}^{n \times d_x}$, i.e., $X\sim N(\mu, \Sigma)$, where $\mu=[0,0,0]$ and $\Sigma=[[1,0.5,-0.5],[0.5,1,-0.5],[-0.5,0.5,1]]$. Then, we choose the true parameters of our predictive model, linear regression in our case, as $\beta \in \mathbb{R}^{d_l \times d_x}$ matrix for slopes and intercepts. Next, we calculated response according to the model $Y=\beta_0+\beta X^T+\varepsilon$, where $\varepsilon$ is independently generated noise term and follows normal distribution, i.e., $\varepsilon \sim N(0, \sigma)$. Here the standard deviation of added noise controls the correlation between feature values and responses (Responses represent demand is in both prescriptive problems). To see the behavior of our method and other methods, we have employed 10 different noise standard deviations, thus we create 10 different feature and response pairs with different correlations. we measure these correlations based on R-Square value of a linear regression model. As for shipment cost, we randomly simulate its matrix as from warehouse $i$ to location $j$ based on uniform distribution $C_{n,i,j} \sim U(0,30)$ for each scenario. In newsvendor problem. we create order, backordering, and holding costs again based on uniform distribution $c_{n,j} \sim U(0,300)$, $b_{n,j} \sim U(0,3000)$, and $h_{n,j} \sim U(0,150)$ for each scenario and product, respectively. Out of created $n=2000$ observations, we randomly choose train, validation, and test sets from $n=2000$ observations with size of $70, 15, 15$, respectively. This splitting process is repeated by $30$ times, and all results are reported based on the average cost of these $30$ replications in both problems.
\subsection{IPPO Formulations}
We modify newsvendor and two-stage shipment problems here for integration process. First, we introduce $\beta$ variable to make predictions for demand via linear regression. Then we also introduce counterpart decision variables of original variables in newsvendor and shipment models because these counterpart decision variables will be made based on predicted demands. In our lower level, we make decisions based on output of predictive algorithm, linear regression, and submit these decisions to the upper level, so that we can evaluate the quality of these decisions based on true parameters in \ref{eq:eq24} and \ref{eq:eq33}.\\
\begin{subequations}
\begin{flalign}
\min\limits_{Q^U,U^U,O^U,\beta} & \hspace{3 mm} \frac{1}{J}\sum\limits_{j=1}^{J}\left[\frac{1}{N}\sum\limits_{n=1}^{N}(c_{n,j}Q_{n,j}^{U}+b_{n,j}U_{n,j}^U+h_{n,j}O_{n,j}^U)\right] \label{eq:eq21}\\
\textrm{s.t.} \hspace{5 mm} & U_{n,j}^U \geq Y_{n,j}-Q_{n,j}^U \hspace{47.1 mm} \forall j \in J, \forall n \in N \label{eq:eq22}\\
& O_{n,j}^U \geq Q_{n,j}^U-Y_{n,j} \hspace{47 mm} \forall j \in J, \forall n \in N \label{eq:eq23}\\
& Q^U_{n,j}= Q^L_{n,j}   \hspace{58.3 mm} \forall j \in J,\forall n \in N \label{eq:eq24}\\
& Q_{n,j}^U, U_{n,j}^U,O_{n,j}^U\geq 0, \quad \beta \quad free \hspace{25.1 mm} \forall j \in J, \forall n \in N \label{eq:eq25}\\
&\min\limits_{Q^L,U^L,O^L}  \frac{1}{J}\sum\limits_{j=1}^{J}\left[ \frac{1}{N}\sum\limits_{n=1}^{N}(c_{n,j}Q_{n,j}^{L}+b_{n,j}U_{n,j}^L+h_{n,j}O_{n,j}^L)\right] \label{eq:eq26}\\
&\hspace{6 mm} \textrm{s.t.} \hspace{4 mm} U_{n,j}^L \geq \hat{Y}_{n,j}-Q_{n,j}^L \hspace{32 mm} \forall j \in J, \forall n \in N \label{eq:eq27}\\
&\hspace{16 mm} O_{n,j}^L \geq Q_{n,j}^L-\hat{Y}_{n,j} \hspace{31.5 mm} \forall j \in J, \forall n \in N \label{eq:eq28}\\
&\hspace{16 mm} Q_{n,j}^L, U_{n,j}^L,O_{n,j}^L\geq 0 \hspace{31 mm} \forall j \in J,\forall n \in N \label{eq:eq29}
\end{flalign}
\end{subequations}
We modify newsvendor and two-stage shipment problems here for integration process. First, we introduce $\beta$ variable to make predictions for demand via linear regression. Then we also introduce counterpart decision variables of original variables in newsvendor and shipment models because these counterpart decision variables will be made based on predicted demands. In our lower level, we make decisions based on output of predictive algorithm, linear regression, and submit these decisions to the upper level, so that we can evaluate the quality of these decisions based on true parameters. Constraints \ref{eq:eq24} and \ref{eq:eq33} undertake this task.\\
Detailed formulation of integrated newsvendor problem and  two stage shipment problem is provided below in (\ref{eq:eq21})-(\ref{eq:eq29}) and (\ref{eq:eq30})-(\ref{eq:eq38}), respectively. If controlling generalization error is needed, one of the recommendations formulated in \autoref{sec:ControllingGeneralizationErrorinIPPO} can be included in these models.\\
In both formulations, there will be a trade-off between lower and upper problems, such that lower problem minimizes its own cost based on $\beta$ variables provided by upper level problem, and upper level problem minimizes its own objective value based on prescriptions provided by lower level problem. To be able to solve bilevel model, we need to add optimality conditions for lower level problem based on KKT conditions. We introduce dual variables for lower level constraints, and write these conditions, but KKT conditions bring nonlinearity because of complementary slackness, so new binary variables can be defined, and SOS constraints and big M method can be used. After introducing KKT conditions, predictive task integrated two stage shipment problem becomes a mix integer problem with single level, and this formulation is given in (\ref{eq:eq8})-(\ref{eq:eq9}).
\begin{subequations}
\begin{flalign}
\min\limits_{\beta,Z^U,T^U,S^U} & \frac{1}{N}\sum\limits_{n=1}^{N}(\sum\limits_{i=1}^{I}P_1Z^U_{n,i}+\sum\limits_{i=1}^{I}P_2T^U_{n,i}+\sum\limits_{i=1}^{I}\sum\limits_{j=1}^{J}C_{n,i,j}S^U_{n,i,j}) \label{eq:eq30}\\
\textrm{s.t.} \hspace{5 mm} & \sum\limits_{i=1}^{I}S^U_{n,i,j}\geq Y_{n,j} \hspace{60 mm} \forall n \in N,\forall j \in J \label{eq:eq31}\\
& \sum\limits_{j=1}^{J}S^U_{n,i,j}\leq Z^U_{n,i}+T^U_{n,i} \hspace{48.5 mm} \forall n \in N,\forall i \in I  \label{eq:eq32}\\
& Z^U_{n,i}= Z^L_{n,i}   \hspace{69.45 mm} \forall n \in N,\forall i \in I \label{eq:eq33}\\
& Z^U_{n,i}, \quad T^U_{n,i}, \quad S^U_{n,i,j}\geq 0,\quad \beta \quad free  \hspace{26 mm} \forall n \in N,\forall i \in I, \forall j \in J \label{eq:e34}\\
&\min\limits_{Z^L,T^L,S^L}  \frac{1}{N}\sum\limits_{n=1}^{N}(\sum\limits_{i=1}^{I}P_1Z^L_{n,i}+\sum\limits_{i=1}^{I}P_2T^L_{n,i}+\sum\limits_{i=1}^{I}\sum\limits_{j=1}^{J}C_{n,i,j}S^L_{n,i,j}) \label{eq:eq35}\\
&\hspace{6 mm} \textrm{s.t.} \hspace{4 mm} \sum\limits_{i=1}^{I}S^L_{n,i,j}\geq \hat{Y}_{n,j}  \hspace{44 mm} \forall n \in N,\forall j \in J \label{eq:eq36}\\
&\hspace{16 mm} \sum\limits_{j=1}^{J}S^L_{n,i,j}\leq Z^L_{n,i}+T^L_{n,i} \hspace{32.5 mm} \forall n \in N,\forall i \in I \label{eq:eq37}\\
&\hspace{16 mm} Z^L_{n,i},\quad T^L_{n,i},\quad S^L_{n,i,j}\geq 0 \hspace{31 mm} \forall n \in N,\forall i \in I, \forall j \in J \label{eq:eq38}
\end{flalign}
\end{subequations}   

\subsection{Convergence to Stochastic Optimization}
The Equation in \ref{eq:eq20} defines the linear regression where the output is a weighted combination of inputs plus an intercept. Linear regressions are generally trained based on mean squared deviations. In our proposed integrated model, this Equation in \ref{eq:eq20} produces predictions for each scenario, and lower level objective and constraints prescribes decisions based on these predictions as seen in (\ref{eq:eq26})-(\ref{eq:eq29}) and (\ref{eq:eq35})-(\ref{eq:eq38}) for each scenario again. If level of correlation between $X$ and $Y$ goes to zero, slopes of predictive model in  \ref{eq:eq20} or the feature variable contributions goes to zero, and predictions will be all equal each other thanks to intercepts. The same prediction for all scenarios will prescribe the same decisions across all scenarios as seen in forwarded decisions from lower level to upper level in (\ref{eq:eq24}) and (\ref{eq:eq33}). This converts our integrated methodology to a single level problem and becomes scenario formulation of a two-stage stochastic problem as shown in (\ref{eq:eq13})-(\ref{eq:eq16}).

\section{Computational Results}
This section discusses the performance of our integrated methodology and other methods under different circumstances. All results for these experiments are obtained from Gurobi python API \cite{gurobi}. We compared results of various well-known methods like Point-Estimate-Based Optimization, Stochastic Optimization, and recent methods like Conditional Stochastic Optimization (kNN), and The Feature-Based Optimization by \cite{doi:10.1287/mnsc.2018.3253}, \cite{Ban2019TheBD}, respectively. We investigate the behaviors of these methods under different correlations between features $X$ and responses $Y$, and evaluate performance of validation data set to see if generalization error is needed.

\subsection{Newsvendor Problem}
First and common feature of all methods, as we see in Figure \ref{plot:graph3}, solution quality improves when we increase the correlation between features (side info) and responses, and this is expected because the more information is provided, the better results are obtained. However, improvement rates are different in each method. Second, kNN from \cite{doi:10.1287/mnsc.2018.3253} gives better result compared to stochastic optimization (k neighbors value is optimized over validation data set). The reason why it gives better solution is because it leverages specific neighbors in train data, instead of minimizing expected cost over all train data, eliminates irrelevant scenarios, search optimal solution by minimizing expected cost around neighbors.
\begin{figure}
\makebox[\linewidth][c]{%
\begin{subfigure}[b]{.6\textwidth}
\centering
\includegraphics[width=.95\textwidth]{NewsvendorTrain}
\caption{Train Data Set}
\end{subfigure} \hspace{-5mm}%
\begin{subfigure}[b]{.6\textwidth}
\centering
\includegraphics[width=.95\textwidth]{NewsvendorTest}
\caption{Test Data Set}
\end{subfigure}%
}\\
\caption{Comparison of Different Methods for Newsvendor Problem}
\label{plot:graph3}
\end{figure}


% Please add the following required packages to your document preamble:
% \usepackage{multirow}
\begin{table}
\resizebox{\textwidth}{!}{\begin{tabular}{|c|c|c|c|c|c|c|c|c|c|c|c|}
\hline
\multirow{2}{*}{\textbf{\begin{tabular}[c]{@{}c@{}}Correlation \\ between X and Y\end{tabular}}} & \multicolumn{4}{c|}{\textbf{True Objective Values}}           & \multicolumn{4}{c|}{\textbf{IPPO}}                            & \multirow{2}{*}{\textbf{Performance}} & \multirow{2}{*}{\textbf{\begin{tabular}[c]{@{}c@{}}Optimal Regularization \\ Parameter\end{tabular}}} & \multirow{2}{*}{\textbf{\begin{tabular}[c]{@{}c@{}}Optimal Neighbors\\ for  kNN\end{tabular}}} \\ \cline{2-9}
                                                                                                 & \textbf{Mean} & \textbf{Max} & \textbf{Min} & \textbf{S.Dev.} & \textbf{Mean} & \textbf{Max} & \textbf{Min} & \textbf{S.Dev.} &                                       &                                                                                                       &                                                                                                \\ \hline
\%7                                                                                                & 873.5         & 924.8        & 815.0        & 27.2            & 2016.3        & 2118.5       & 1897.9       & 58.3            & \%7                                     & 1 & 41                                                                                                    \\ \hline
\%13                                                                                               & 853.7         & 892.9        & 813.6        & 21.0            & 1484.9        & 1549.1       & 1410.3       & 35.0            & \%10                                    & 1& 41                                                                                                     \\ \hline
\%20                                                                                               & 852.7         & 889.5        & 814.5        & 19.6            & 1312.5        & 1369.6       & 1252.7       & 28.3            & \%11                                    & 1   & 31                                                                                                  \\ \hline
\%26                                                                                               & 852.5         & 889.2        & 815.3        & 19.0            & 1216.8        & 1270.1       & 1164.7       & 25.0            & \%13                                    & 1 & 28                                                                                         \\ \hline
\%34                                                                                               & 852.5         & 889.4        & 816.1        & 18.6            & 1148.9        & 1199.4       & 1102.0       & 22.9            & \%15                                    & 1    & 28                                                                                                 \\ \hline
\%43                                                                                               & 852.5         & 889.6        & 816.8        & 18.3            & 1093.3        & 1141.6       & 1050.7       & 21.4            & \%17                                    & 1    & 22                                                                                                 \\ \hline
\%53                                                                                               & 852.5         & 889.8        & 817.3        & 18.2            & 1050.1        & 1096.5       & 1010.9       & 20.4            & \%20                                    & 1    & 16                                                                                                 \\ \hline
\%64                                                                                               & 852.5         & 890.0        & 817.8        & 18.1            & 1006.9        & 1051.5       & 971.0        & 19.5            & \%24                                    & 1    & 16                                                                                                 \\ \hline
\%76                                                                                               & 852.5         & 890.2        & 818.3        & 18.0            & 966.7         & 1009.7       & 933.4        & 18.8            & \%31                                    & 1       & 10                                                                                              \\ \hline
\%92                                                                                               & 852.5         & 890.4        & 818.9        & 17.9            & 914.2         & 955.0        & 881.2        & 18.2            & \%53                                    & 1      & 5                                                                                               \\ \hline
\end{tabular}}
\caption{Newsvendor Problem Train Data Set Statistical Values} 
\label{tab:table1}
\end{table} 
% Please add the following required packages to your document preamble:
% \usepackage{multirow}
\begin{table}
\resizebox{\textwidth}{!}{\begin{tabular}{|c|c|c|c|c|c|c|c|c|c|c|c|}
\hline
\multirow{2}{*}{\textbf{\begin{tabular}[c]{@{}c@{}}Correlation \\ between X and Y\end{tabular}}} & \multicolumn{4}{c|}{\textbf{True Objective Values}}           & \multicolumn{4}{c|}{\textbf{IPPO}}                            & \multirow{2}{*}{\textbf{Performance}} & \multirow{2}{*}{\textbf{\begin{tabular}[c]{@{}c@{}}Optimal Regularization \\ Parameter\end{tabular}}} & \multirow{2}{*}{\textbf{\begin{tabular}[c]{@{}c@{}}Optimal Neighbors\\ for  kNN\end{tabular}}} \\ \cline{2-9}
                                                                                                 & \textbf{Mean} & \textbf{Max} & \textbf{Min} & \textbf{S.Dev.} & \textbf{Mean} & \textbf{Max} & \textbf{Min} & \textbf{S.Dev.} &                                       &                                                                                                       &                                                                                                \\ \hline
\%7                                                                                                & 873.0         & 972.6        & 688.7        & 59.2            & 2229.5        & 2546.0       & 1987.2       & 140.0           & \%7                                     & 1 & 41                                                                                                    \\ \hline
\%13                                                                                               & 851.3         & 929.5        & 711.3        & 44.0            & 1596.6        & 1774.5       & 1431.6       & 80.2            & \%9                                     & 1 &41                                                                                                    \\ \hline
\%20                                                                                               & 849.1         & 924.9        & 724.4        & 40.5            & 1391.9        & 1527.4       & 1248.6       & 63.4            & \%10                                    & 1    &31                                                                                                 \\ \hline
\%26                                                                                               & 848.2         & 922.4        & 731.7        & 39.3            & 1278.3        & 1391.2       & 1148.2       & 55.0            & \%12                                    & 1 &28                                                                                                    \\ \hline
\%34                                                                                               & 847.7         & 920.9        & 737.2        & 38.7            & 1197.6        & 1293.9       & 1075.7       & 50.0            & \%14                                    & 1    &28                                                                                            \\ \hline
\%43                                                                                               & 847.3         & 921.2        & 741.8        & 38.4            & 1131.6        & 1214.2       & 1016.3       & 46.2            & \%16                                    & 1 &22                                                                                                \\ \hline
\%53                                                                                               & 847.0         & 925.1        & 745.5        & 38.3            & 1080.3        & 1154.2       & 970.4        & 43.7            & \%18                                    & 1    &16                                                                                           \\ \hline
\%64                                                                                               & 846.7         & 929.1        & 749.1        & 38.4            & 1029.0        & 1103.7       & 925.0        & 41.8            & \%22                                    & 1    &16                                                                                               \\ \hline
\%76                                                                                               & 846.4         & 932.7        & 752.4        & 38.6            & 981.3         & 1061.2       & 882.6        & 40.5            & \%28                                    & 1       &10                                                                                              \\ \hline
\%92                                                                                               & 846.1         & 937.5        & 756.8        & 38.9            & 919.0         & 1007.0       & 827.2        & 39.5            & \%47                                    & 1          &5                                                                                        \\ \hline
\end{tabular}}
\caption{Newsvendor Problem Test Data Set Statistical Values} 
\label{tab:table2}
\end{table} 



\begin{figure}

\makebox[\linewidth][c]{%
\begin{subfigure}[b]{.42\textwidth}
\centering
\includegraphics[width=.95\textwidth]{NewsvendorTrainValidation3_80}
\caption{R-Square=\%7}
\end{subfigure} \hspace{-5mm}%
\begin{subfigure}[b]{.42\textwidth}
\centering
\includegraphics[width=.95\textwidth]{NewsvendorTrainValidation2_05}
\caption{R-Square=\%13}
\end{subfigure} \hspace{-5mm}%
\begin{subfigure}[b]{.42\textwidth}
\centering
\includegraphics[width=.95\textwidth]{NewsvendorTrainValidation1_18}
\caption{R-Square=\%26}
\end{subfigure}%
}\\

\makebox[\linewidth][c]{%
\begin{subfigure}[b]{.42\textwidth}
\centering
\includegraphics[width=.95\textwidth]{NewsvendorTrainValidation0_96}
\caption{R-Square=\%34}
\end{subfigure} \hspace{-5mm}%
\begin{subfigure}[b]{.42\textwidth}
\centering
\includegraphics[width=.95\textwidth]{NewsvendorTrainValidation0_78}
\caption{R-Square=\%43}
\end{subfigure} \hspace{-5mm}%
\begin{subfigure}[b]{.42\textwidth}
\centering
\includegraphics[width=.95\textwidth]{NewsvendorTrainValidation0_64}
\caption{R-Square=\%53}
\end{subfigure}%
}\\


\makebox[\linewidth][c]{%
\begin{subfigure}[b]{.42\textwidth}
\centering
\includegraphics[width=.95\textwidth]{NewsvendorTrainValidation0_50}
\caption{R-Square=\%64}
\end{subfigure} \hspace{-5mm}%
\begin{subfigure}[b]{.42\textwidth}
\centering
\includegraphics[width=.95\textwidth]{NewsvendorTrainValidation0_37}
\caption{R-Square=\%76}
\end{subfigure} \hspace{-5mm}%
\begin{subfigure}[b]{.42\textwidth}
\centering
\includegraphics[width=.95\textwidth]{NewsvendorTrainValidation0_20}
\caption{R-Square=\%92}
\end{subfigure}%
}\\
\caption{Newsvendor Problem Train and Validation Performance Over Regularization Parameter $\lambda_1$ For Different R-Square Values Between X and Y}
\label{plot:graph4}
\end{figure}

We observe that value of k neighbors increases when correlation level between $X$ and $Y$ decreases. Since this correlation decreases, kNN tends to minimize expected cost over more train data as we reported in Table \ref{tab:table1} and \ref{tab:table2}. We did not include result of \cite{Elmachtoub2017SmartT} in Figure \ref{plot:graph3} and \ref{plot:graph5} because \cite{Elmachtoub2017SmartT} build their method when there is an uncertainty in cost vector of objective function.\\
Since IPPO methodology fully leverages side information and evaluates decisions with respect to true parameters inside of integrated framework, it gives the best solution.
In newsvendor problem, we did not observe over-fitting issue, and there is no need to control generalization error. As we report in Figure \ref{plot:graph4}, regardless of correlation level between $X$ and $Y$, validation and train performance improve constantly until $\lambda_1=1$ where no generalization is needed. That is why IPPO with and without generalization error in Figure \ref{plot:graph3} overlaps.

\subsection{Two-Stage Shipment Problem}
We observe similar behaviors here as in newsvendor problem. Main difference in Two-Stage shipment problem is IPPO overfits in especially low level $X$ and $Y$ correlations since IPPO methodology fully leverages side information and searches best solution for train data set. As we see the effect of regularization coefficient ($\lambda_1$) in Figure \ref{plot:graph6}, train performance constantly improves, but validation performance become worse after some point. In figure \ref{plot:graph5}, we report train and test performance with and without controlling generalization error. Although train performance is better than other methods without controlling generalization error, a better solution is achieved via optimizing regularization coefficient ($\lambda_1$). Some detailed statistical measures are given in Table \ref{tab:table3} for train set and in Table \ref{tab:table4} for test set. Performance column is calculated in Equation \ref{eq:eq39} based on second best method, which is The Feature-Based Optimization. These results are based on 30 replications, mean column is presented in Figure \ref{plot:graph5}. \\[-30pt]

\begin{flalign}
Performance =\left[\frac{The \hspace{1 mm} Feature \hspace{1 mm} Based-True \hspace{1 mm} Objective}{IPPO-True \hspace{1 mm} Objective}-1\right]100 \label{eq:eq39}
\end{flalign}


\begin{figure}
\makebox[\linewidth][c]{%
\begin{subfigure}[b]{.6\textwidth}
\centering
\includegraphics[width=.95\textwidth]{ShipmentTrain}
\caption{Train Data Set}
\end{subfigure} \hspace{-5mm}%
\begin{subfigure}[b]{.6\textwidth}
\centering
\includegraphics[width=.95\textwidth]{ShipmentTest}
\caption{Test Data Set}
\end{subfigure}%
}\\
\caption{Comparison of Different Methods for Shipment Problem}
\label{plot:graph5}
\end{figure}

% Please add the following required packages to your document preamble:
% \usepackage{multirow}
\begin{table}
\resizebox{\textwidth}{!}{\begin{tabular}{|c|c|c|c|c|c|c|c|c|c|c|c|}
\hline
\multirow{2}{*}{\textbf{\begin{tabular}[c]{@{}c@{}}Correlation \\ between X and Y\end{tabular}}} & \multicolumn{4}{c|}{\textbf{True Objective Values}}           & \multicolumn{4}{c|}{\textbf{IPPO}}                            & \multirow{2}{*}{\textbf{Performance}} & \multirow{2}{*}{\textbf{\begin{tabular}[c]{@{}c@{}}Optimal Regularization \\ Parameter\end{tabular}}} & \multirow{2}{*}{\textbf{\begin{tabular}[c]{@{}c@{}}Optimal Neighbors\\ for  kNN\end{tabular}}} \\ \cline{2-9}
                                                                                                 & \textbf{Mean} & \textbf{Max} & \textbf{Min} & \textbf{S.Dev.} & \textbf{Mean} & \textbf{Max} & \textbf{Min} & \textbf{S.Dev.} &                                       &                                                                                                       &                                                                                                \\ \hline
\%7                                                                                                & 792.5         & 855.7        & 738.0        & 22.2            & 925.0         & 988.4        & 864.1        & 28.4            & \%11                                                                                          & 0.49 &50                                                                                                 \\ \hline
\%13                                                                                               & 773.3         & 820.2        & 728.2        & 18.6            & 847.2         & 889.9        & 798.1        & 21.2            & \%32                                                                                          & 0.36    &26                                                                                          \\ \hline
\%20                                                                                               & 771.7         & 812.5        & 730.8        & 17.5            & 824.9         & 862.0        & 781.6        & 18.7            & \%56                                                                                          & 0.38   &28                                                                                          \\ \hline
\%26                                                                                               & 771.3         & 808.7        & 732.7        & 17.0            & 813.5         & 847.9        & 773.0        & 17.9            & \%78                                                                                          & 0.32      &28                                                                                    \\ \hline
\%34                                                                                               & 771.1         & 806.1        & 734.0        & 16.7            & 805.4         & 838.0        & 766.8        & 17.3            & \%105                                                                                         & 0.28  &17                                                                                                \\ \hline
\%43                                                                                               & 771.0         & 804.0        & 735.0        & 16.5            & 798.9         & 830.0        & 761.6        & 16.9            & \%138                                                                                         & 0.24     &15                                                                                             \\ \hline
\%53                                                                                               & 770.8         & 802.4        & 735.8        & 16.4            & 793.6         & 823.7        & 757.7        & 16.7            & \%179                                                                                         & 0.22        &15                                                                                          \\ \hline
\%64                                                                                               & 770.7         & 801.0        & 736.7        & 16.3            & 788.6         & 818.7        & 753.7        & 16.5            & \%242                                                                                         & 0.17           &15                                                                                    \\ \hline
\%76                                                                                               & 770.6         & 801.7        & 737.4        & 16.2            & 783.8         & 814.9        & 750.1        & 16.3            & \%347                                                                                         & 0.13              &15                                                                                    \\ \hline
\%92                                                                                               & 770.5         & 802.7        & 737.2        & 16.1            & 777.6         & 809.8        & 743.6        & 16.1            & \%695                                                                                         & 0.08                 &13                                                                                 \\ \hline
\end{tabular}}
\caption{Shipment Problem Train Data Set Statistical Values between True Objective and IPPO Objective} 
\label{tab:table3}
\end{table} 
% Please add the following required packages to your document preamble:
% \usepackage{multirow}
\begin{table}
\resizebox{\textwidth}{!}{\begin{tabular}{|c|c|c|c|c|c|c|c|c|c|c|c|}
\hline
\multirow{2}{*}{\textbf{\begin{tabular}[c]{@{}c@{}}Correlation \\ between X and Y\end{tabular}}} & \multicolumn{4}{c|}{\textbf{True Objective Values}}           & \multicolumn{4}{c|}{\textbf{IPPO}}                            & \multirow{2}{*}{\textbf{Performance}} & \multirow{2}{*}{\textbf{\begin{tabular}[c]{@{}c@{}}Optimal Regularization \\ Parameter\end{tabular}}} & \multirow{2}{*}{\textbf{\begin{tabular}[c]{@{}c@{}}Optimal Neighbors\\ for  kNN\end{tabular}}} \\ \cline{2-9}
                                                                                                 & \textbf{Mean} & \textbf{Max} & \textbf{Min} & \textbf{S.Dev.} & \textbf{Mean} & \textbf{Max} & \textbf{Min} & \textbf{S.Dev.} &                                       &                                                                                                       &                                                                                                \\ \hline
\%7                                                                                                & 792.6         & 903.0        & 691.6        & 51.3            & 944.6         & 1089.4       & 821.4        & 67.6            & \%15                                    & 0.49 &50                                                                                                 \\ \hline
\%13                                                                                               & 773.8         & 857.3        & 690.3        & 40.0            & 859.6         & 932.8        & 766.6        & 46.8            & \%34                                    & 0.36    &26                                                                                              \\ \hline
\%20                                                                                               & 772.6         & 845.3        & 695.1        & 36.3            & 834.6         & 901.8        & 751.0        & 40.0            & \%57                                    & 0.38       &28                                                                                           \\ \hline
\%26                                                                                               & 772.4         & 838.6        & 697.9        & 34.5            & 821.6         & 886.0        & 742.4        & 37.2            & \%80                                    & 0.32          &28                                                                                        \\ \hline
\%34                                                                                               & 772.4         & 834.8        & 700.3        & 33.4            & 812.4         & 874.8        & 736.5        & 35.4            & \%105                                   & 0.28             &17                                                                                     \\ \hline
\%43                                                                                               & 772.3         & 833.2        & 702.3        & 32.5            & 804.8         & 865.6        & 731.1        & 34.0            & \%138                                   & 0.24                &15                                                                                  \\ \hline
\%53                                                                                               & 772.3         & 831.9        & 703.6        & 31.9            & 798.9         & 858.5        & 728.0        & 33.0            & \%178                                   & 0.22                   &15                                                                               \\ \hline
\%64                                                                                               & 772.3         & 830.6        & 704.1        & 31.4            & 793.2         & 851.4        & 724.3        & 32.1            & \%238                                   & 0.17                      &15                                                                            \\ \hline
\%76                                                                                               & 772.2         & 829.4        & 704.6        & 30.9            & 787.7         & 844.8        & 720.9        & 31.4            & \%343                                   & 0.13                         &15                                                                         \\ \hline
\%92                                                                                               & 772.2         & 827.8        & 705.3        & 30.4            & 780.6         & 836.1        & 714.4        & 30.6            & \%691                                   & 0.08                        &13                                                                          \\ \hline\end{tabular}}
\caption{Shipment Problem Test Data Set Statistical Values between True Objective and IPPO Objective} 
\label{tab:table4}
\end{table} 


\begin{figure}

\makebox[\linewidth][c]{%
\begin{subfigure}[b]{.42\textwidth}
\centering
\includegraphics[width=.95\textwidth]{ShipmentTrainValidation3_80}
\caption{R-Square=\%7}
\end{subfigure} \hspace{-5mm}%
\begin{subfigure}[b]{.42\textwidth}
\centering
\includegraphics[width=.95\textwidth]{ShipmentTrainValidation2_05}
\caption{R-Square=\%13}
\end{subfigure} \hspace{-5mm}%
\begin{subfigure}[b]{.42\textwidth}
\centering
\includegraphics[width=.95\textwidth]{ShipmentTrainValidation1_18}
\caption{R-Square=\%26}
\end{subfigure}%
}\\

\makebox[\linewidth][c]{%
\begin{subfigure}[b]{.42\textwidth}
\centering
\includegraphics[width=.95\textwidth]{ShipmentTrainValidation0_96}
\caption{R-Square=\%34}
\end{subfigure} \hspace{-5mm}%
\begin{subfigure}[b]{.42\textwidth}
\centering
\includegraphics[width=.95\textwidth]{ShipmentTrainValidation0_78}
\caption{R-Square=\%43}
\end{subfigure} \hspace{-5mm}%
\begin{subfigure}[b]{.42\textwidth}
\centering
\includegraphics[width=.95\textwidth]{ShipmentTrainValidation0_64}
\caption{R-Square=\%53}
\end{subfigure}%
}\\
\makebox[\linewidth][c]{%
\begin{subfigure}[b]{.42\textwidth}
\centering
\includegraphics[width=.95\textwidth]{ShipmentTrainValidation0_50}
\caption{R-Square=\%64}
\end{subfigure} \hspace{-5mm}%
\begin{subfigure}[b]{.42\textwidth}
\centering
\includegraphics[width=.95\textwidth]{ShipmentTrainValidation0_37}
\caption{R-Square=\%76}
\end{subfigure} \hspace{-5mm}%
\begin{subfigure}[b]{.42\textwidth}
\centering
\includegraphics[width=.95\textwidth]{ShipmentTrainValidation0_20}
\caption{R-Square=\%92}
\end{subfigure}%
}\\
\caption{Shipment Problem Train and Validation Performance Over Regularization Parameter $\lambda_1$ For Different R-Square Values Between X and Y}
\label{plot:graph6}

\end{figure}


\section{Conclusion}
In this work, we provide an integrated framework fully leveraging feature data to predict responds in order to take best actions in prescriptive task. This methodology can be employed in any prescriptive task whose parameters uncertain with feature data as long as prescriptive model is convex because KKT optimality conditions requires duality, hence limited to convexity. To be able optimize predictive and prescriptive tasks at the same time, predictive task must be in a linear form, so this limits the complexity of prediction task. However, alternative methods can be used to capture nonlinearity outside of proposed framework. Our purpose is to train predictive model based on not predictive error, but also based on prescriptive error which is the assessment of decision variables provided by predicted responses with respect to true responses. Bilevel models are NP-hard problems, and it is not easy to solve, but our framework is a special case, where all decision variables are based on scenarios except the parameters of predictive model. Thus this formulation can be easily decomposed by decoupling the parameters of predictive model, and solved by PHA step by step. Also, we demonstrate the behavior of our and other frameworks under different correlation of feature and response data. Finally, we provide our results and compare them to traditional and recently introduced methods, and we perform well with and even without controlling generalization error.
.\\



\clearpage
%merlin.mbs apsrmp4-1.bst 2010-07-25 4.21a (PWD, AO, DPC) hacked
%Control: key (0)
%Control: author (3) reversed first dotless
%Control: editor formatted (0) differently from author
%Control: production of article title (0) allowed
%Control: page (1) range
%Control: year (0) verbatim
%Control: production of eprint (0) enabled
\begin{thebibliography}{312}%
\makeatletter
\providecommand \@ifxundefined [1]{%
 \@ifx{#1\undefined}
}%
\providecommand \@ifnum [1]{%
 \ifnum #1\expandafter \@firstoftwo
 \else \expandafter \@secondoftwo
 \fi
}%
\providecommand \@ifx [1]{%
 \ifx #1\expandafter \@firstoftwo
 \else \expandafter \@secondoftwo
 \fi
}%
\providecommand \natexlab [1]{#1}%
\providecommand \enquote  [1]{``#1''}%
\providecommand \bibnamefont  [1]{#1}%
\providecommand \bibfnamefont [1]{#1}%
\providecommand \citenamefont [1]{#1}%
\providecommand \href@noop [0]{\@secondoftwo}%
\providecommand \href [0]{\begingroup \@sanitize@url \@href}%
\providecommand \@href[1]{\@@startlink{#1}\@@href}%
\providecommand \@@href[1]{\endgroup#1\@@endlink}%
\providecommand \@sanitize@url [0]{\catcode `\\12\catcode `\$12\catcode
  `\&12\catcode `\#12\catcode `\^12\catcode `\_12\catcode `\%12\relax}%
\providecommand \@@startlink[1]{}%
\providecommand \@@endlink[0]{}%
\providecommand \url  [0]{\begingroup\@sanitize@url \@url }%
\providecommand \@url [1]{\endgroup\@href {#1}{\urlprefix }}%
\providecommand \urlprefix  [0]{URL }%
\providecommand \Eprint [0]{\href }%
\providecommand \doibase [0]{http://dx.doi.org/}%
\providecommand \selectlanguage [0]{\@gobble}%
\providecommand \bibinfo  [0]{\@secondoftwo}%
\providecommand \bibfield  [0]{\@secondoftwo}%
\providecommand \translation [1]{[#1]}%
\providecommand \BibitemOpen [0]{}%
\providecommand \bibitemStop [0]{}%
\providecommand \bibitemNoStop [0]{.\EOS\space}%
\providecommand \EOS [0]{\spacefactor3000\relax}%
\providecommand \BibitemShut  [1]{\csname bibitem#1\endcsname}%
\let\auto@bib@innerbib\@empty
%</preamble>
\bibitem [{\citenamefont {Aaronson}\ \emph {et~al.}(2019)\citenamefont
  {Aaronson}, \citenamefont {Cojocaru}, \citenamefont {Gheorghiu},\ and\
  \citenamefont {Kashefi}}]{ACGK19}%
  \BibitemOpen
  \bibfield  {author} {\bibinfo {author} {\bibnamefont {Aaronson},
  \bibfnamefont {Scott}}, \bibinfo {author} {\bibfnamefont {Alexandru}\
  \bibnamefont {Cojocaru}}, \bibinfo {author} {\bibfnamefont {Alexandru}\
  \bibnamefont {Gheorghiu}}, \ and\ \bibinfo {author} {\bibfnamefont {Elham}\
  \bibnamefont {Kashefi}}} (\bibinfo {year} {2019}),\ \bibfield  {title}
  {\enquote {\bibinfo {title} {Complexity-theoretic limitations on blind
  delegated quantum computation},}\ }in\ \href {\doibase
  10.4230/LIPIcs.ICALP.2019.6} {\emph {\bibinfo {booktitle} {46th International
  Colloquium on Automata, Languages, and Programming (ICALP 2019)}}},\ \bibinfo
  {series} {LIPIcs}, Vol.\ \bibinfo {volume} {132},\ \bibinfo {editor} {edited
  by\ \bibinfo {editor} {\bibfnamefont {Christel}\ \bibnamefont {Baier}},
  \bibinfo {editor} {\bibfnamefont {Ioannis}\ \bibnamefont {Chatzigiannakis}},
  \bibinfo {editor} {\bibfnamefont {Paola}\ \bibnamefont {Flocchini}}, \ and\
  \bibinfo {editor} {\bibfnamefont {Stefano}\ \bibnamefont {Leonardi}}}\
  (\bibinfo  {publisher} {Schloss Dagstuhl})\ pp.\ \bibinfo {pages}
  {6:1--6:13},\ \Eprint {http://arxiv.org/abs/arXiv:1704.08482}
  {arXiv:1704.08482} \BibitemShut {NoStop}%
\bibitem [{\citenamefont {Ac\'{\i}n}\ \emph {et~al.}(2007)\citenamefont
  {Ac\'{\i}n}, \citenamefont {Brunner}, \citenamefont {Gisin}, \citenamefont
  {Massar}, \citenamefont {Pironio},\ and\ \citenamefont {Scarani}}]{ABGMPS07}%
  \BibitemOpen
  \bibfield  {author} {\bibinfo {author} {\bibnamefont {Ac\'{\i}n},
  \bibfnamefont {Antonio}}, \bibinfo {author} {\bibfnamefont {Nicolas}\
  \bibnamefont {Brunner}}, \bibinfo {author} {\bibfnamefont {Nicolas}\
  \bibnamefont {Gisin}}, \bibinfo {author} {\bibfnamefont {Serge}\ \bibnamefont
  {Massar}}, \bibinfo {author} {\bibfnamefont {Stefano}\ \bibnamefont
  {Pironio}}, \ and\ \bibinfo {author} {\bibfnamefont {Valerio}\ \bibnamefont
  {Scarani}}} (\bibinfo {year} {2007}),\ \bibfield  {title} {\enquote {\bibinfo
  {title} {Device-independent security of quantum cryptography against
  collective attacks},}\ }\href {\doibase 10.1103/PhysRevLett.98.230501}
  {\bibfield  {journal} {\bibinfo  {journal} {Phys. Rev. Lett.}\ }\textbf
  {\bibinfo {volume} {98}},\ \bibinfo {pages} {230501}}\BibitemShut {NoStop}%
\bibitem [{\citenamefont {Ac\'{\i}n}\ \emph {et~al.}(2012)\citenamefont
  {Ac\'{\i}n}, \citenamefont {Massar},\ and\ \citenamefont {Pironio}}]{AMP12}%
  \BibitemOpen
  \bibfield  {author} {\bibinfo {author} {\bibnamefont {Ac\'{\i}n},
  \bibfnamefont {Antonio}}, \bibinfo {author} {\bibfnamefont {Serge}\
  \bibnamefont {Massar}}, \ and\ \bibinfo {author} {\bibfnamefont {Stefano}\
  \bibnamefont {Pironio}}} (\bibinfo {year} {2012}),\ \bibfield  {title}
  {\enquote {\bibinfo {title} {Randomness versus nonlocality and
  entanglement},}\ }\href {\doibase 10.1103/PhysRevLett.108.100402} {\bibfield
  {journal} {\bibinfo  {journal} {Phys. Rev. Lett.}\ }\textbf {\bibinfo
  {volume} {108}},\ \bibinfo {pages} {100402}},\ \Eprint
  {http://arxiv.org/abs/arXiv:1107.2754} {arXiv:1107.2754} \BibitemShut
  {NoStop}%
\bibitem [{\citenamefont {Aggarwal}\ \emph {et~al.}(2019)\citenamefont
  {Aggarwal}, \citenamefont {Chung}, \citenamefont {Lin},\ and\ \citenamefont
  {Vidick}}]{ACLV19}%
  \BibitemOpen
  \bibfield  {author} {\bibinfo {author} {\bibnamefont {Aggarwal},
  \bibfnamefont {Divesh}}, \bibinfo {author} {\bibfnamefont {Kai-Min}\
  \bibnamefont {Chung}}, \bibinfo {author} {\bibfnamefont {Han-Hsuan}\
  \bibnamefont {Lin}}, \ and\ \bibinfo {author} {\bibfnamefont {Thomas}\
  \bibnamefont {Vidick}}} (\bibinfo {year} {2019}),\ \bibfield  {title}
  {\enquote {\bibinfo {title} {A quantum-proof non-malleable extractor},}\ }in\
  \href {\doibase 10.1007/978-3-030-17656-3_16} {\emph {\bibinfo {booktitle}
  {Advances in Cryptology -- EUROCRYPT 2019}}},\ \bibinfo {editor} {edited by\
  \bibinfo {editor} {\bibfnamefont {Yuval}\ \bibnamefont {Ishai}}\ and\
  \bibinfo {editor} {\bibfnamefont {Vincent}\ \bibnamefont {Rijmen}}}\
  (\bibinfo  {publisher} {Springer})\ pp.\ \bibinfo {pages} {442--469},\
  \Eprint {http://arxiv.org/abs/arXiv:1710.00557} {arXiv:1710.00557}
  \BibitemShut {NoStop}%
\bibitem [{\citenamefont {Aharonov}\ \emph {et~al.}(2010)\citenamefont
  {Aharonov}, \citenamefont {{Ben-Or}},\ and\ \citenamefont {Eban}}]{ABE10}%
  \BibitemOpen
  \bibfield  {author} {\bibinfo {author} {\bibnamefont {Aharonov},
  \bibfnamefont {Dorit}}, \bibinfo {author} {\bibfnamefont {Michael}\
  \bibnamefont {{Ben-Or}}}, \ and\ \bibinfo {author} {\bibfnamefont {Elad}\
  \bibnamefont {Eban}}} (\bibinfo {year} {2010}),\ \bibfield  {title} {\enquote
  {\bibinfo {title} {Interactive proofs for quantum computations},}\ }in\
  \href@noop {} {\emph {\bibinfo {booktitle} {Proceedings of Innovations in
  Computer Science, ICS 2010}}}\ (\bibinfo  {publisher} {Tsinghua University
  Press})\ pp.\ \bibinfo {pages} {453--469},\ \Eprint
  {http://arxiv.org/abs/arXiv:0810.5375} {arXiv:0810.5375} \BibitemShut
  {NoStop}%
\bibitem [{\citenamefont {Ahlswede}\ and\ \citenamefont
  {Csisz\'ar}(1993)}]{AC93}%
  \BibitemOpen
  \bibfield  {author} {\bibinfo {author} {\bibnamefont {Ahlswede},
  \bibfnamefont {Rudolph}}, \ and\ \bibinfo {author} {\bibfnamefont {Imre}\
  \bibnamefont {Csisz\'ar}}} (\bibinfo {year} {1993}),\ \bibfield  {title}
  {\enquote {\bibinfo {title} {Common randomness in information theory and
  cryptography---{Part I}: Secret sharing},}\ }\href {\doibase
  10.1109/18.243431} {\bibfield  {journal} {\bibinfo  {journal} {IEEE Trans.
  Inf. Theory}\ }\textbf {\bibinfo {volume} {39}}~(\bibinfo {number} {4}),\
  \bibinfo {pages} {1121--1132}}\BibitemShut {NoStop}%
\bibitem [{\citenamefont {Alagic}\ \emph {et~al.}(2016)\citenamefont {Alagic},
  \citenamefont {Broadbent}, \citenamefont {Fefferman}, \citenamefont
  {Gagliardoni}, \citenamefont {Schaffner},\ and\ \citenamefont
  {St.~Jules}}]{ABFGSSJ16}%
  \BibitemOpen
  \bibfield  {author} {\bibinfo {author} {\bibnamefont {Alagic}, \bibfnamefont
  {Gorjan}}, \bibinfo {author} {\bibfnamefont {Anne}\ \bibnamefont
  {Broadbent}}, \bibinfo {author} {\bibfnamefont {Bill}\ \bibnamefont
  {Fefferman}}, \bibinfo {author} {\bibfnamefont {Tommaso}\ \bibnamefont
  {Gagliardoni}}, \bibinfo {author} {\bibfnamefont {Christian}\ \bibnamefont
  {Schaffner}}, \ and\ \bibinfo {author} {\bibfnamefont {Michael}\ \bibnamefont
  {St.~Jules}}} (\bibinfo {year} {2016}),\ \bibfield  {title} {\enquote
  {\bibinfo {title} {Computational security of quantum encryption},}\ }in\
  \href {\doibase 10.1007/978-3-319-49175-2_3} {\emph {\bibinfo {booktitle}
  {Proceedings of the 9th International Conference on Information Theoretic
  Security, ICITS 2016}}},\ \bibinfo {editor} {edited by\ \bibinfo {editor}
  {\bibfnamefont {Anderson~C.A.}\ \bibnamefont {Nascimento}}\ and\ \bibinfo
  {editor} {\bibfnamefont {Paulo}\ \bibnamefont {Barreto}}}\ (\bibinfo
  {publisher} {Springer})\ pp.\ \bibinfo {pages} {47--71},\ \Eprint
  {http://arxiv.org/abs/arXiv:1602.01441} {arXiv:1602.01441} \BibitemShut
  {NoStop}%
\bibitem [{\citenamefont {Alagic}\ \emph {et~al.}(2018)\citenamefont {Alagic},
  \citenamefont {Gagliardoni},\ and\ \citenamefont {Majenz}}]{AGM18}%
  \BibitemOpen
  \bibfield  {author} {\bibinfo {author} {\bibnamefont {Alagic}, \bibfnamefont
  {Gorjan}}, \bibinfo {author} {\bibfnamefont {Tommaso}\ \bibnamefont
  {Gagliardoni}}, \ and\ \bibinfo {author} {\bibfnamefont {Christian}\
  \bibnamefont {Majenz}}} (\bibinfo {year} {2018}),\ \bibfield  {title}
  {\enquote {\bibinfo {title} {Unforgeable quantum encryption},}\ }in\ \href
  {\doibase 10.1007/978-3-319-78372-7_16} {\emph {\bibinfo {booktitle}
  {Advances in Cryptology -- {EUROCRYPT} 2018, Proceedings, Part {III}}}},\
  \bibinfo {series} {LNCS}, Vol.\ \bibinfo {volume} {10822},\ \bibinfo {editor}
  {edited by\ \bibinfo {editor} {\bibfnamefont {Jesper~B.}\ \bibnamefont
  {Nielsen}}\ and\ \bibinfo {editor} {\bibfnamefont {Vincent}\ \bibnamefont
  {Rijmen}}}\ (\bibinfo  {publisher} {Springer})\ pp.\ \bibinfo {pages}
  {489--519},\ \Eprint {http://arxiv.org/abs/arXiv:1709.06539}
  {arXiv:1709.06539} \BibitemShut {NoStop}%
\bibitem [{\citenamefont {Alagic}\ and\ \citenamefont {Majenz}(2017)}]{AM17}%
  \BibitemOpen
  \bibfield  {author} {\bibinfo {author} {\bibnamefont {Alagic}, \bibfnamefont
  {Gorjan}}, \ and\ \bibinfo {author} {\bibfnamefont {Christian}\ \bibnamefont
  {Majenz}}} (\bibinfo {year} {2017}),\ \bibfield  {title} {\enquote {\bibinfo
  {title} {Quantum non-malleability and authentication},}\ }in\ \href {\doibase
  10.1007/978-3-319-63715-0_11} {\emph {\bibinfo {booktitle} {Advances in
  Cryptology -- CRYPTO 2017, Proceedings, Part II}}},\ \bibinfo {series}
  {LNCS}, Vol.\ \bibinfo {volume} {10402},\ \bibinfo {editor} {edited by\
  \bibinfo {editor} {\bibfnamefont {Jonathan}\ \bibnamefont {Katz}}\ and\
  \bibinfo {editor} {\bibfnamefont {Hovav}\ \bibnamefont {Shacham}}}\ (\bibinfo
   {publisher} {Springer})\ pp.\ \bibinfo {pages} {310--341},\ \Eprint
  {http://arxiv.org/abs/arXiv:1610.04214} {arXiv:1610.04214} \BibitemShut
  {NoStop}%
\bibitem [{\citenamefont {Alicki}\ and\ \citenamefont {Fannes}(2004)}]{AF04}%
  \BibitemOpen
  \bibfield  {author} {\bibinfo {author} {\bibnamefont {Alicki}, \bibfnamefont
  {Robert}}, \ and\ \bibinfo {author} {\bibfnamefont {Mark}\ \bibnamefont
  {Fannes}}} (\bibinfo {year} {2004}),\ \bibfield  {title} {\enquote {\bibinfo
  {title} {Continuity of quantum conditional information},}\ }\href {\doibase
  10.1088/0305-4470/37/5/L01} {\bibfield  {journal} {\bibinfo  {journal} {J.
  Phys. A}\ }\textbf {\bibinfo {volume} {37}},\ \bibinfo {pages}
  {L55--L57}}\BibitemShut {NoStop}%
\bibitem [{\citenamefont {Alon}\ \emph {et~al.}(2020)\citenamefont {Alon},
  \citenamefont {Chung}, \citenamefont {Chung}, \citenamefont {Huang},
  \citenamefont {Lee},\ and\ \citenamefont {Shen}}]{ACCHLS21}%
  \BibitemOpen
  \bibfield  {author} {\bibinfo {author} {\bibnamefont {Alon}, \bibfnamefont
  {Bar}}, \bibinfo {author} {\bibfnamefont {Hao}\ \bibnamefont {Chung}},
  \bibinfo {author} {\bibfnamefont {Kai-Min}\ \bibnamefont {Chung}}, \bibinfo
  {author} {\bibfnamefont {Mi-Ying}\ \bibnamefont {Huang}}, \bibinfo {author}
  {\bibfnamefont {Yi}~\bibnamefont {Lee}}, \ and\ \bibinfo {author}
  {\bibfnamefont {Yu-Ching}\ \bibnamefont {Shen}}} (\bibinfo {year} {2020}),\
  \href@noop {} {\enquote {\bibinfo {title} {Round efficient secure multiparty
  quantum computation with identifiable abort},}\ }\bibinfo {howpublished} {to
  appear at CRYPTO 2021},\ \bibinfo {note} {e-Print
  \href{http://eprint.iacr.org/2020/1464}{IACR 2020/1464}}\BibitemShut
  {NoStop}%
\bibitem [{\citenamefont {Ambainis}\ \emph {et~al.}(2009)\citenamefont
  {Ambainis}, \citenamefont {Bouda},\ and\ \citenamefont {Winter}}]{ABW09}%
  \BibitemOpen
  \bibfield  {author} {\bibinfo {author} {\bibnamefont {Ambainis},
  \bibfnamefont {Andris}}, \bibinfo {author} {\bibfnamefont {Jan}\ \bibnamefont
  {Bouda}}, \ and\ \bibinfo {author} {\bibfnamefont {Andreas}\ \bibnamefont
  {Winter}}} (\bibinfo {year} {2009}),\ \bibfield  {title} {\enquote {\bibinfo
  {title} {Non-malleable encryption of quantum information},}\ }\href {\doibase
  10.1063/1.3094756} {\bibfield  {journal} {\bibinfo  {journal} {J. Math.
  Phys.}\ }\textbf {\bibinfo {volume} {50}}~(\bibinfo {number} {4}),\ \bibinfo
  {pages} {042106}},\ \Eprint {http://arxiv.org/abs/arXiv:0808.0353}
  {arXiv:0808.0353} \BibitemShut {NoStop}%
\bibitem [{\citenamefont {Ambainis}\ \emph {et~al.}(2000)\citenamefont
  {Ambainis}, \citenamefont {Mosca}, \citenamefont {Tapp},\ and\ \citenamefont
  {de~Wolf}}]{AMTW00}%
  \BibitemOpen
  \bibfield  {author} {\bibinfo {author} {\bibnamefont {Ambainis},
  \bibfnamefont {Andris}}, \bibinfo {author} {\bibfnamefont {Michele}\
  \bibnamefont {Mosca}}, \bibinfo {author} {\bibfnamefont {Alain}\ \bibnamefont
  {Tapp}}, \ and\ \bibinfo {author} {\bibfnamefont {Ronald}\ \bibnamefont
  {de~Wolf}}} (\bibinfo {year} {2000}),\ \bibfield  {title} {\enquote {\bibinfo
  {title} {Private quantum channels},}\ }in\ \href@noop {} {\emph {\bibinfo
  {booktitle} {Proceedings of the 41st Symposium on Foundations of Computer
  Science, FOCS~'00}}}\ (\bibinfo  {publisher} {IEEE})\ p.\ \bibinfo {pages}
  {547},\ \Eprint {http://arxiv.org/abs/arXiv:quant-ph/0003101}
  {arXiv:quant-ph/0003101} \BibitemShut {NoStop}%
\bibitem [{\citenamefont {Ambainis}\ and\ \citenamefont {Smith}(2004)}]{AS04}%
  \BibitemOpen
  \bibfield  {author} {\bibinfo {author} {\bibnamefont {Ambainis},
  \bibfnamefont {Andris}}, \ and\ \bibinfo {author} {\bibfnamefont {Adam}\
  \bibnamefont {Smith}}} (\bibinfo {year} {2004}),\ \bibfield  {title}
  {\enquote {\bibinfo {title} {Small pseudo-random families of matrices:
  Derandomizing approximate quantum encryption},}\ }in\ \href {\doibase
  10.1007/978-3-540-27821-4_23} {\emph {\bibinfo {booktitle} {Proceedings of
  the 8th International Workshop on Randomization and Computation, RANDOM
  2004}}}\ (\bibinfo  {publisher} {Springer})\ pp.\ \bibinfo {pages}
  {249--260},\ \Eprint {http://arxiv.org/abs/arXiv:quant-ph/0404075}
  {arXiv:quant-ph/0404075} \BibitemShut {NoStop}%
\bibitem [{\citenamefont {Arnon-Friedman}(2018)}]{ArnonThesis}%
  \BibitemOpen
  \bibfield  {author} {\bibinfo {author} {\bibnamefont {Arnon-Friedman},
  \bibfnamefont {Rotem}}} (\bibinfo {year} {2018}),\ \emph {\bibinfo {title}
  {Reductions to IID in Device-independent Quantum Information Processing}},\
  \href@noop {} {Ph.D. thesis}\ (\bibinfo  {school} {Swiss Federal Institute of
  Technology (ETH) Zurich}),\ \Eprint {http://arxiv.org/abs/arXiv:1812.10922}
  {arXiv:1812.10922} \BibitemShut {NoStop}%
\bibitem [{\citenamefont {Arnon-Friedman}\ \emph {et~al.}(2018)\citenamefont
  {Arnon-Friedman}, \citenamefont {Dupuis}, \citenamefont {Fawzi},
  \citenamefont {Renner},\ and\ \citenamefont {Vidick}}]{ADFRV18}%
  \BibitemOpen
  \bibfield  {author} {\bibinfo {author} {\bibnamefont {Arnon-Friedman},
  \bibfnamefont {Rotem}}, \bibinfo {author} {\bibfnamefont {Fr{\'e}d{\'e}ric}\
  \bibnamefont {Dupuis}}, \bibinfo {author} {\bibfnamefont {Omar}\ \bibnamefont
  {Fawzi}}, \bibinfo {author} {\bibfnamefont {Renato}\ \bibnamefont {Renner}},
  \ and\ \bibinfo {author} {\bibfnamefont {Thomas}\ \bibnamefont {Vidick}}}
  (\bibinfo {year} {2018}),\ \bibfield  {title} {\enquote {\bibinfo {title}
  {Practical device-independent quantum cryptography via entropy
  accumulation},}\ }\href@noop {} {\bibfield  {journal} {\bibinfo  {journal}
  {Nat. Commun.}\ }\textbf {\bibinfo {volume} {9}}~(\bibinfo {number} {1}),\
  \bibinfo {pages} {1--11}}\BibitemShut {NoStop}%
\bibitem [{\citenamefont {Arnon-Friedman}\ \emph {et~al.}(2016)\citenamefont
  {Arnon-Friedman}, \citenamefont {Portmann},\ and\ \citenamefont
  {Scholz}}]{AFPS16}%
  \BibitemOpen
  \bibfield  {author} {\bibinfo {author} {\bibnamefont {Arnon-Friedman},
  \bibfnamefont {Rotem}}, \bibinfo {author} {\bibfnamefont {Christopher}\
  \bibnamefont {Portmann}}, \ and\ \bibinfo {author} {\bibfnamefont
  {Volkher~B.}\ \bibnamefont {Scholz}}} (\bibinfo {year} {2016}),\ \bibfield
  {title} {\enquote {\bibinfo {title} {Quantum-proof multi-source randomness
  extractors in the {Markov} model},}\ }in\ \href {\doibase
  10.4230/LIPIcs.TQC.2016.2} {\emph {\bibinfo {booktitle} {11th Conference on
  the Theory of Quantum Computation, Communication and Cryptography (TQC
  2016)}}},\ \bibinfo {series} {LIPIcs}, Vol.~\bibinfo {volume} {61}\ (\bibinfo
   {publisher} {Schloss Dagstuhl})\ pp.\ \bibinfo {pages} {2:1--2:34},\ \Eprint
  {http://arxiv.org/abs/arXiv:1510.06743} {arXiv:1510.06743} \BibitemShut
  {NoStop}%
\bibitem [{\citenamefont {Arnon-Friedman}\ \emph {et~al.}(2019)\citenamefont
  {Arnon-Friedman}, \citenamefont {Renner},\ and\ \citenamefont
  {Vidick}}]{AFRV19}%
  \BibitemOpen
  \bibfield  {author} {\bibinfo {author} {\bibnamefont {Arnon-Friedman},
  \bibfnamefont {Rotem}}, \bibinfo {author} {\bibfnamefont {Renato}\
  \bibnamefont {Renner}}, \ and\ \bibinfo {author} {\bibfnamefont {Thomas}\
  \bibnamefont {Vidick}}} (\bibinfo {year} {2019}),\ \bibfield  {title}
  {\enquote {\bibinfo {title} {Simple and tight device-independent security
  proofs},}\ }\href {\doibase 10.1137/18M1174726} {\bibfield  {journal}
  {\bibinfo  {journal} {SIAM J. Comput.}\ }\textbf {\bibinfo {volume}
  {48}}~(\bibinfo {number} {1}),\ \bibinfo {pages} {181--225}},\ \Eprint
  {http://arxiv.org/abs/arXiv:1607.01797} {arXiv:1607.01797} \BibitemShut
  {NoStop}%
\bibitem [{\citenamefont {Aspect}\ \emph {et~al.}(1982)\citenamefont {Aspect},
  \citenamefont {Dalibard},\ and\ \citenamefont {Roger}}]{Aspect82}%
  \BibitemOpen
  \bibfield  {author} {\bibinfo {author} {\bibnamefont {Aspect}, \bibfnamefont
  {Alain}}, \bibinfo {author} {\bibfnamefont {Jean}\ \bibnamefont {Dalibard}},
  \ and\ \bibinfo {author} {\bibfnamefont {G\'erard}\ \bibnamefont {Roger}}}
  (\bibinfo {year} {1982}),\ \bibfield  {title} {\enquote {\bibinfo {title}
  {Experimental test of {Bell}'s inequalities using time-varying analyzers},}\
  }\href {\doibase 10.1103/PhysRevLett.49.1804} {\bibfield  {journal} {\bibinfo
   {journal} {Phys. Rev. Lett.}\ }\textbf {\bibinfo {volume} {49}},\ \bibinfo
  {pages} {1804--1807}}\BibitemShut {NoStop}%
\bibitem [{\citenamefont {Aspect}\ \emph {et~al.}(1981)\citenamefont {Aspect},
  \citenamefont {Grangier},\ and\ \citenamefont {Roger}}]{Aspect81}%
  \BibitemOpen
  \bibfield  {author} {\bibinfo {author} {\bibnamefont {Aspect}, \bibfnamefont
  {Alain}}, \bibinfo {author} {\bibfnamefont {Philippe}\ \bibnamefont
  {Grangier}}, \ and\ \bibinfo {author} {\bibfnamefont {G\'erard}\ \bibnamefont
  {Roger}}} (\bibinfo {year} {1981}),\ \bibfield  {title} {\enquote {\bibinfo
  {title} {Experimental tests of realistic local theories via {Bell}'s
  theorem},}\ }\href {\doibase 10.1103/PhysRevLett.47.460} {\bibfield
  {journal} {\bibinfo  {journal} {Phys. Rev. Lett.}\ }\textbf {\bibinfo
  {volume} {47}},\ \bibinfo {pages} {460--463}}\BibitemShut {NoStop}%
\bibitem [{\citenamefont {Backes}\ \emph {et~al.}(2004)\citenamefont {Backes},
  \citenamefont {Pfitzmann},\ and\ \citenamefont {Waidner}}]{BPW04}%
  \BibitemOpen
  \bibfield  {author} {\bibinfo {author} {\bibnamefont {Backes}, \bibfnamefont
  {Michael}}, \bibinfo {author} {\bibfnamefont {Birgit}\ \bibnamefont
  {Pfitzmann}}, \ and\ \bibinfo {author} {\bibfnamefont {Michael}\ \bibnamefont
  {Waidner}}} (\bibinfo {year} {2004}),\ \bibfield  {title} {\enquote {\bibinfo
  {title} {A general composition theorem for secure reactive systems},}\ }in\
  \href {\doibase 10.1007/978-3-540-24638-1_19} {\emph {\bibinfo {booktitle}
  {Theory of Cryptography, Proceedings of TCC 2004}}},\ \bibinfo {series}
  {LNCS}, Vol.\ \bibinfo {volume} {2951}\ (\bibinfo  {publisher} {Springer})\
  pp.\ \bibinfo {pages} {336--354}\BibitemShut {NoStop}%
\bibitem [{\citenamefont {Backes}\ \emph {et~al.}(2007)\citenamefont {Backes},
  \citenamefont {Pfitzmann},\ and\ \citenamefont {Waidner}}]{BPW07}%
  \BibitemOpen
  \bibfield  {author} {\bibinfo {author} {\bibnamefont {Backes}, \bibfnamefont
  {Michael}}, \bibinfo {author} {\bibfnamefont {Birgit}\ \bibnamefont
  {Pfitzmann}}, \ and\ \bibinfo {author} {\bibfnamefont {Michael}\ \bibnamefont
  {Waidner}}} (\bibinfo {year} {2007}),\ \bibfield  {title} {\enquote {\bibinfo
  {title} {The reactive simulatability ({RSIM}) framework for asynchronous
  systems},}\ }\href {\doibase 10.1016/j.ic.2007.05.002} {\bibfield  {journal}
  {\bibinfo  {journal} {Inform. and Comput.}\ }\textbf {\bibinfo {volume}
  {205}}~(\bibinfo {number} {12}),\ \bibinfo {pages} {1685--1720}},\ \bibinfo
  {note} {extended version of~\textcite{PW01}, e-Print
  \href{http://eprint.iacr.org/2004/082}{IACR 2004/082}}\BibitemShut {NoStop}%
\bibitem [{\citenamefont {Badertscher}\ \emph {et~al.}(2020)\citenamefont
  {Badertscher}, \citenamefont {Cojocaru}, \citenamefont {Colisson},
  \citenamefont {Kashefi}, \citenamefont {Leichtle}, \citenamefont {Mantri},\
  and\ \citenamefont {Wallden}}]{BCCKLMW20}%
  \BibitemOpen
  \bibfield  {author} {\bibinfo {author} {\bibnamefont {Badertscher},
  \bibfnamefont {Christian}}, \bibinfo {author} {\bibfnamefont {Alexandru}\
  \bibnamefont {Cojocaru}}, \bibinfo {author} {\bibfnamefont {L{\'e}o}\
  \bibnamefont {Colisson}}, \bibinfo {author} {\bibfnamefont {Elham}\
  \bibnamefont {Kashefi}}, \bibinfo {author} {\bibfnamefont {Dominik}\
  \bibnamefont {Leichtle}}, \bibinfo {author} {\bibfnamefont {Atul}\
  \bibnamefont {Mantri}}, \ and\ \bibinfo {author} {\bibfnamefont {Petros}\
  \bibnamefont {Wallden}}} (\bibinfo {year} {2020}),\ \bibfield  {title}
  {\enquote {\bibinfo {title} {Security limitations of classical-client
  delegated quantum computing},}\ }in\ \href {\doibase
  10.1007/978-3-030-64834-3_23} {\emph {\bibinfo {booktitle} {Advances in
  Cryptology -- ASIACRYPT 2020, Proceedings, Part {II}}}},\ \bibinfo {series}
  {LNCS}, Vol.\ \bibinfo {volume} {12492},\ \bibinfo {editor} {edited by\
  \bibinfo {editor} {\bibfnamefont {Shiho}\ \bibnamefont {Moriai}}\ and\
  \bibinfo {editor} {\bibfnamefont {Huaxiong}\ \bibnamefont {Wang}}}\ (\bibinfo
   {publisher} {Springer},\ \bibinfo {address} {Cham})\ pp.\ \bibinfo {pages}
  {667--696},\ \Eprint {http://arxiv.org/abs/arXiv:2007.01668}
  {arXiv:2007.01668} \BibitemShut {NoStop}%
\bibitem [{\citenamefont {Banfi}\ \emph {et~al.}(2019)\citenamefont {Banfi},
  \citenamefont {Maurer}, \citenamefont {Portmann},\ and\ \citenamefont
  {Zhu}}]{BMPZ19}%
  \BibitemOpen
  \bibfield  {author} {\bibinfo {author} {\bibnamefont {Banfi}, \bibfnamefont
  {Fabio}}, \bibinfo {author} {\bibfnamefont {Ueli}\ \bibnamefont {Maurer}},
  \bibinfo {author} {\bibfnamefont {Christopher}\ \bibnamefont {Portmann}}, \
  and\ \bibinfo {author} {\bibfnamefont {Jiamin}\ \bibnamefont {Zhu}}}
  (\bibinfo {year} {2019}),\ \bibfield  {title} {\enquote {\bibinfo {title}
  {Composable and finite computational security of quantum message
  transmission},}\ }in\ \href {\doibase 10.1007/978-3-030-36030-6_12} {\emph
  {\bibinfo {booktitle} {Theory of Cryptography, Proceedings of {TCC} 2019,
  Part {I}}}},\ \bibinfo {series} {LNCS}, Vol.\ \bibinfo {volume} {11891}\
  (\bibinfo  {publisher} {Springer})\ pp.\ \bibinfo {pages} {282--311},\
  \Eprint {http://arxiv.org/abs/arXiv:1908.03436} {arXiv:1908.03436}
  \BibitemShut {NoStop}%
\bibitem [{\citenamefont {Barnum}\ \emph {et~al.}(2002)\citenamefont {Barnum},
  \citenamefont {Cr{\'e}peau}, \citenamefont {Gottesman}, \citenamefont
  {Smith},\ and\ \citenamefont {Tapp}}]{BCGST02}%
  \BibitemOpen
  \bibfield  {author} {\bibinfo {author} {\bibnamefont {Barnum}, \bibfnamefont
  {Howard}}, \bibinfo {author} {\bibfnamefont {Claude}\ \bibnamefont
  {Cr{\'e}peau}}, \bibinfo {author} {\bibfnamefont {Daniel}\ \bibnamefont
  {Gottesman}}, \bibinfo {author} {\bibfnamefont {Adam}\ \bibnamefont {Smith}},
  \ and\ \bibinfo {author} {\bibfnamefont {Alain}\ \bibnamefont {Tapp}}}
  (\bibinfo {year} {2002}),\ \bibfield  {title} {\enquote {\bibinfo {title}
  {Authentication of quantum messages},}\ }in\ \href {\doibase
  10.1109/SFCS.2002.1181969} {\emph {\bibinfo {booktitle} {Proceedings of the
  43rd Symposium on Foundations of Computer Science, FOCS~'02}}}\ (\bibinfo
  {publisher} {IEEE})\ pp.\ \bibinfo {pages} {449--458},\ \Eprint
  {http://arxiv.org/abs/arXiv:quant-ph/0205128} {arXiv:quant-ph/0205128}
  \BibitemShut {NoStop}%
\bibitem [{\citenamefont {Barrett}\ \emph {et~al.}(2013)\citenamefont
  {Barrett}, \citenamefont {Colbeck},\ and\ \citenamefont {Kent}}]{BCK13}%
  \BibitemOpen
  \bibfield  {author} {\bibinfo {author} {\bibnamefont {Barrett}, \bibfnamefont
  {Jonathan}}, \bibinfo {author} {\bibfnamefont {Roger}\ \bibnamefont
  {Colbeck}}, \ and\ \bibinfo {author} {\bibfnamefont {Adrian}\ \bibnamefont
  {Kent}}} (\bibinfo {year} {2013}),\ \bibfield  {title} {\enquote {\bibinfo
  {title} {Memory attacks on device-independent quantum cryptography},}\ }\href
  {\doibase 10.1103/PhysRevLett.110.010503} {\bibfield  {journal} {\bibinfo
  {journal} {Phys. Rev. Lett.}\ }\textbf {\bibinfo {volume} {110}},\ \bibinfo
  {pages} {010503}},\ \Eprint {http://arxiv.org/abs/arXiv:1201.4407}
  {arXiv:1201.4407} \BibitemShut {NoStop}%
\bibitem [{\citenamefont {Barrett}\ \emph {et~al.}(2005)\citenamefont
  {Barrett}, \citenamefont {Hardy},\ and\ \citenamefont {Kent}}]{BHK05}%
  \BibitemOpen
  \bibfield  {author} {\bibinfo {author} {\bibnamefont {Barrett}, \bibfnamefont
  {Jonathan}}, \bibinfo {author} {\bibfnamefont {Lucien}\ \bibnamefont
  {Hardy}}, \ and\ \bibinfo {author} {\bibfnamefont {Adrian}\ \bibnamefont
  {Kent}}} (\bibinfo {year} {2005}),\ \bibfield  {title} {\enquote {\bibinfo
  {title} {No signaling and quantum key distribution},}\ }\href {\doibase
  10.1103/PhysRevLett.95.010503} {\bibfield  {journal} {\bibinfo  {journal}
  {Phys. Rev. Lett.}\ }\textbf {\bibinfo {volume} {95}}~(\bibinfo {number}
  {1}),\ \bibinfo {pages} {1--4}}\BibitemShut {NoStop}%
\bibitem [{\citenamefont {Baumgratz}\ \emph {et~al.}(2014)\citenamefont
  {Baumgratz}, \citenamefont {Cramer},\ and\ \citenamefont {Plenio}}]{BCP14}%
  \BibitemOpen
  \bibfield  {author} {\bibinfo {author} {\bibnamefont {Baumgratz},
  \bibfnamefont {Tillmann}}, \bibinfo {author} {\bibfnamefont {Marcus}\
  \bibnamefont {Cramer}}, \ and\ \bibinfo {author} {\bibfnamefont {Martin~B.}\
  \bibnamefont {Plenio}}} (\bibinfo {year} {2014}),\ \bibfield  {title}
  {\enquote {\bibinfo {title} {Quantifying coherence},}\ }\href {\doibase
  10.1103/PhysRevLett.113.140401} {\bibfield  {journal} {\bibinfo  {journal}
  {Phys. Rev. Lett.}\ }\textbf {\bibinfo {volume} {113}},\ \bibinfo {pages}
  {140401}},\ \Eprint {http://arxiv.org/abs/arxiv:1311.0275} {arxiv:1311.0275}
  \BibitemShut {NoStop}%
\bibitem [{\citenamefont {Beaver}(1992)}]{Bea92}%
  \BibitemOpen
  \bibfield  {author} {\bibinfo {author} {\bibnamefont {Beaver}, \bibfnamefont
  {Donald}}} (\bibinfo {year} {1992}),\ \bibfield  {title} {\enquote {\bibinfo
  {title} {Foundations of secure interactive computing},}\ }in\ \href {\doibase
  10.1007/3-540-46766-1_31} {\emph {\bibinfo {booktitle} {Advances in
  Cryptology -- CRYPTO~'91}}},\ \bibinfo {series} {LNCS}, Vol.\ \bibinfo
  {volume} {576}\ (\bibinfo  {publisher} {Springer})\ pp.\ \bibinfo {pages}
  {377--391}\BibitemShut {NoStop}%
\bibitem [{\citenamefont {Bell}(1964)}]{Bell64}%
  \BibitemOpen
  \bibfield  {author} {\bibinfo {author} {\bibnamefont {Bell}, \bibfnamefont
  {John~Stewart}}} (\bibinfo {year} {1964}),\ \bibfield  {title} {\enquote
  {\bibinfo {title} {On the {E}instein-{P}odolsky-{R}osen paradox},}\
  }\href@noop {} {\bibfield  {journal} {\bibinfo  {journal} {Physics}\ }\textbf
  {\bibinfo {volume} {1}}~(\bibinfo {number} {3}),\ \bibinfo {pages}
  {195--200}}\BibitemShut {NoStop}%
\bibitem [{\citenamefont {Bell}(1966)}]{Bell66}%
  \BibitemOpen
  \bibfield  {author} {\bibinfo {author} {\bibnamefont {Bell}, \bibfnamefont
  {John~Stewart}}} (\bibinfo {year} {1966}),\ \bibfield  {title} {\enquote
  {\bibinfo {title} {On the problem of hidden variables in quantum
  mechanics},}\ }\href {\doibase 10.1103/RevModPhys.38.447} {\bibfield
  {journal} {\bibinfo  {journal} {Rev. Mod. Phys.}\ }\textbf {\bibinfo {volume}
  {38}},\ \bibinfo {pages} {447--452}}\BibitemShut {NoStop}%
\bibitem [{\citenamefont {Bell}\ and\ \citenamefont {Aspect}(2004)}]{BellFree}%
  \BibitemOpen
  \bibfield  {author} {\bibinfo {author} {\bibnamefont {Bell}, \bibfnamefont
  {John~Stewart}}, \ and\ \bibinfo {author} {\bibfnamefont {Alain}\
  \bibnamefont {Aspect}}} (\bibinfo {year} {2004}),\ \enquote {\bibinfo {title}
  {Free variables and local causality},}\ in\ \href {\doibase
  10.1017/CBO9780511815676.014} {\emph {\bibinfo {booktitle} {Speakable and
  Unspeakable in Quantum Mechanics: Collected Papers on Quantum Philosophy}}},\
  Chap.~\bibinfo {chapter} {12},\ \bibinfo {edition} {2nd}\ ed.\ (\bibinfo
  {publisher} {Cambridge University Press})\ pp.\ \bibinfo {pages}
  {100--104}\BibitemShut {NoStop}%
\bibitem [{\citenamefont {Bellare}\ \emph {et~al.}(1997)\citenamefont
  {Bellare}, \citenamefont {Desai}, \citenamefont {Jokipii},\ and\
  \citenamefont {Rogaway}}]{BDJR97}%
  \BibitemOpen
  \bibfield  {author} {\bibinfo {author} {\bibnamefont {Bellare}, \bibfnamefont
  {Mihir}}, \bibinfo {author} {\bibfnamefont {Anand}\ \bibnamefont {Desai}},
  \bibinfo {author} {\bibfnamefont {Eron}\ \bibnamefont {Jokipii}}, \ and\
  \bibinfo {author} {\bibfnamefont {Phillip}\ \bibnamefont {Rogaway}}}
  (\bibinfo {year} {1997}),\ \bibfield  {title} {\enquote {\bibinfo {title} {A
  concrete security treatment of symmetric encryption},}\ }in\ \href {\doibase
  10.1109/SFCS.1997.646128} {\emph {\bibinfo {booktitle} {Proceedings of the
  38th Annual Symposium on Foundations of Computer Science, FOCS~'97}}}\
  (\bibinfo  {publisher} {IEEE})\ pp.\ \bibinfo {pages} {394--403}\BibitemShut
  {NoStop}%
\bibitem [{\citenamefont {Bellare}\ \emph {et~al.}(1998)\citenamefont
  {Bellare}, \citenamefont {Desai}, \citenamefont {Pointcheval},\ and\
  \citenamefont {Rogaway}}]{BDPR98}%
  \BibitemOpen
  \bibfield  {author} {\bibinfo {author} {\bibnamefont {Bellare}, \bibfnamefont
  {Mihir}}, \bibinfo {author} {\bibfnamefont {Anand}\ \bibnamefont {Desai}},
  \bibinfo {author} {\bibfnamefont {David}\ \bibnamefont {Pointcheval}}, \ and\
  \bibinfo {author} {\bibfnamefont {Phillip}\ \bibnamefont {Rogaway}}}
  (\bibinfo {year} {1998}),\ \bibfield  {title} {\enquote {\bibinfo {title}
  {Relations among notions of security for public-key encryption schemes},}\
  }in\ \href {\doibase 10.1007/BFb0055718} {\emph {\bibinfo {booktitle}
  {Advances in Cryptology -- CRYPTO~'98}}}\ (\bibinfo  {publisher} {Springer})\
  pp.\ \bibinfo {pages} {26--45}\BibitemShut {NoStop}%
\bibitem [{\citenamefont {Bellare}\ and\ \citenamefont {Rogaway}(2006)}]{BR06}%
  \BibitemOpen
  \bibfield  {author} {\bibinfo {author} {\bibnamefont {Bellare}, \bibfnamefont
  {Mihir}}, \ and\ \bibinfo {author} {\bibfnamefont {Phillip}\ \bibnamefont
  {Rogaway}}} (\bibinfo {year} {2006}),\ \bibfield  {title} {\enquote {\bibinfo
  {title} {The security of triple encryption and a framework for code-based
  game-playing proofs},}\ }in\ \href {\doibase 10.1007/11761679_25} {\emph
  {\bibinfo {booktitle} {Advances in Cryptology -- EUROCRYPT 2006}}},\ \bibinfo
  {series} {LNCS}, Vol.\ \bibinfo {volume} {4004},\ \bibinfo {editor} {edited
  by\ \bibinfo {editor} {\bibfnamefont {Serge}\ \bibnamefont {Vaudenay}}}\
  (\bibinfo  {publisher} {Springer})\ pp.\ \bibinfo {pages} {409--426},\
  \bibinfo {note} {e-Print \href{http://eprint.iacr.org/2004/331}{IACR
  2004/331}}\BibitemShut {NoStop}%
\bibitem [{\citenamefont {{Ben-Aroya}}\ and\ \citenamefont
  {{Ta-Shma}}(2012)}]{BT12}%
  \BibitemOpen
  \bibfield  {author} {\bibinfo {author} {\bibnamefont {{Ben-Aroya}},
  \bibfnamefont {Avraham}}, \ and\ \bibinfo {author} {\bibfnamefont {Amnon}\
  \bibnamefont {{Ta-Shma}}}} (\bibinfo {year} {2012}),\ \bibfield  {title}
  {\enquote {\bibinfo {title} {Better short-seed quantum-proof extractors},}\
  }\href {\doibase 10.1016/j.tcs.2011.11.036} {\bibfield  {journal} {\bibinfo
  {journal} {Theoretical Computer Science}\ }\textbf {\bibinfo {volume}
  {419}},\ \bibinfo {pages} {17--25}},\ \Eprint
  {http://arxiv.org/abs/arXiv:1004.3737} {arXiv:1004.3737} \BibitemShut
  {NoStop}%
\bibitem [{\citenamefont {{Ben-Or}}\ \emph {et~al.}(2006)\citenamefont
  {{Ben-Or}}, \citenamefont {Cr\'epeau}, \citenamefont {Gottesman},
  \citenamefont {Hassidim},\ and\ \citenamefont {Smith}}]{BCGHS06}%
  \BibitemOpen
  \bibfield  {author} {\bibinfo {author} {\bibnamefont {{Ben-Or}},
  \bibfnamefont {Michael}}, \bibinfo {author} {\bibfnamefont {Claude}\
  \bibnamefont {Cr\'epeau}}, \bibinfo {author} {\bibfnamefont {Daniel}\
  \bibnamefont {Gottesman}}, \bibinfo {author} {\bibfnamefont {Avinatan}\
  \bibnamefont {Hassidim}}, \ and\ \bibinfo {author} {\bibfnamefont {Adam}\
  \bibnamefont {Smith}}} (\bibinfo {year} {2006}),\ \bibfield  {title}
  {\enquote {\bibinfo {title} {Secure multiparty quantum computation with
  (only) a strict honest majority},}\ }in\ \href {\doibase
  10.1109/FOCS.2006.68} {\emph {\bibinfo {booktitle} {Proceedings of the 47th
  Symposium on Foundations of Computer Science, FOCS~'06}}},\ pp.\ \bibinfo
  {pages} {249--260},\ \Eprint {http://arxiv.org/abs/arXiv:0801.1544}
  {arXiv:0801.1544} \BibitemShut {NoStop}%
\bibitem [{\citenamefont {{Ben-Or}}\ \emph {et~al.}(2005)\citenamefont
  {{Ben-Or}}, \citenamefont {Horodecki}, \citenamefont {Leung}, \citenamefont
  {Mayers},\ and\ \citenamefont {Oppenheim}}]{BHLMO05}%
  \BibitemOpen
  \bibfield  {author} {\bibinfo {author} {\bibnamefont {{Ben-Or}},
  \bibfnamefont {Michael}}, \bibinfo {author} {\bibfnamefont {Micha\l{}}\
  \bibnamefont {Horodecki}}, \bibinfo {author} {\bibfnamefont {Debbie}\
  \bibnamefont {Leung}}, \bibinfo {author} {\bibfnamefont {Dominic}\
  \bibnamefont {Mayers}}, \ and\ \bibinfo {author} {\bibfnamefont {Jonathan}\
  \bibnamefont {Oppenheim}}} (\bibinfo {year} {2005}),\ \bibfield  {title}
  {\enquote {\bibinfo {title} {The universal composable security of quantum key
  distribution},}\ }in\ \href {\doibase 10.1007/978-3-540-30576-7_21} {\emph
  {\bibinfo {booktitle} {Theory of Cryptography, Proceedings of TCC 2005}}},\
  \bibinfo {series} {LNCS}, Vol.\ \bibinfo {volume} {3378}\ (\bibinfo
  {publisher} {Springer})\ pp.\ \bibinfo {pages} {386--406},\ \Eprint
  {http://arxiv.org/abs/arXiv:quant-ph/0409078} {arXiv:quant-ph/0409078}
  \BibitemShut {NoStop}%
\bibitem [{\citenamefont {{Ben-Or}}\ and\ \citenamefont {Mayers}(2004)}]{BM04}%
  \BibitemOpen
  \bibfield  {author} {\bibinfo {author} {\bibnamefont {{Ben-Or}},
  \bibfnamefont {Michael}}, \ and\ \bibinfo {author} {\bibfnamefont {Dominic}\
  \bibnamefont {Mayers}}} (\bibinfo {year} {2004}),\ \href@noop {} {\enquote
  {\bibinfo {title} {General security definition and composability for quantum
  \& classical protocols},}\ }\bibinfo {howpublished} {e-Print},\ \Eprint
  {http://arxiv.org/abs/arXiv:quant-ph/0409062} {arXiv:quant-ph/0409062}
  \BibitemShut {NoStop}%
\bibitem [{\citenamefont {Bennett}\ \emph
  {et~al.}(1996{\natexlab{a}})\citenamefont {Bennett}, \citenamefont
  {Bernstein}, \citenamefont {Popescu},\ and\ \citenamefont
  {Schumacher}}]{Bennett96}%
  \BibitemOpen
  \bibfield  {author} {\bibinfo {author} {\bibnamefont {Bennett}, \bibfnamefont
  {Charles~H}}, \bibinfo {author} {\bibfnamefont {Herbert~J.}\ \bibnamefont
  {Bernstein}}, \bibinfo {author} {\bibfnamefont {Sandu}\ \bibnamefont
  {Popescu}}, \ and\ \bibinfo {author} {\bibfnamefont {Benjamin}\ \bibnamefont
  {Schumacher}}} (\bibinfo {year} {1996}{\natexlab{a}}),\ \bibfield  {title}
  {\enquote {\bibinfo {title} {Concentrating partial entanglement by local
  operations},}\ }\href {\doibase 10.1103/PhysRevA.53.2046} {\bibfield
  {journal} {\bibinfo  {journal} {Phys. Rev. A}\ }\textbf {\bibinfo {volume}
  {53}},\ \bibinfo {pages} {2046--2052}}\BibitemShut {NoStop}%
\bibitem [{\citenamefont {Bennett}\ \emph
  {et~al.}(1992{\natexlab{a}})\citenamefont {Bennett}, \citenamefont
  {Bessette}, \citenamefont {Brassard}, \citenamefont {Salvail},\ and\
  \citenamefont {Smolin}}]{BBBSS92}%
  \BibitemOpen
  \bibfield  {author} {\bibinfo {author} {\bibnamefont {Bennett}, \bibfnamefont
  {Charles~H}}, \bibinfo {author} {\bibfnamefont {Fran{\c{c}}ois}\ \bibnamefont
  {Bessette}}, \bibinfo {author} {\bibfnamefont {Gilles}\ \bibnamefont
  {Brassard}}, \bibinfo {author} {\bibfnamefont {Louis}\ \bibnamefont
  {Salvail}}, \ and\ \bibinfo {author} {\bibfnamefont {John}\ \bibnamefont
  {Smolin}}} (\bibinfo {year} {1992}{\natexlab{a}}),\ \bibfield  {title}
  {\enquote {\bibinfo {title} {Experimental quantum cryptography},}\ }\href
  {\doibase 10.1007/BF00191318} {\bibfield  {journal} {\bibinfo  {journal} {J.
  Crypt.}\ }\textbf {\bibinfo {volume} {5}}~(\bibinfo {number} {1}),\ \bibinfo
  {pages} {3--28}}\BibitemShut {NoStop}%
\bibitem [{\citenamefont {Bennett}\ and\ \citenamefont
  {Brassard}(1984)}]{BB84}%
  \BibitemOpen
  \bibfield  {author} {\bibinfo {author} {\bibnamefont {Bennett}, \bibfnamefont
  {Charles~H}}, \ and\ \bibinfo {author} {\bibfnamefont {Gilles}\ \bibnamefont
  {Brassard}}} (\bibinfo {year} {1984}),\ \bibfield  {title} {\enquote
  {\bibinfo {title} {Quantum cryptography: Public key distribution and coin
  tossing},}\ }in\ \href@noop {} {\emph {\bibinfo {booktitle} {Proceedings of
  IEEE International Conference on Computers, Systems, and Signal
  Processing}}},\ pp.\ \bibinfo {pages} {175--179}\BibitemShut {NoStop}%
\bibitem [{\citenamefont {Bennett}\ \emph {et~al.}(1995)\citenamefont
  {Bennett}, \citenamefont {Brassard}, \citenamefont {Cr{\'e}peau},\ and\
  \citenamefont {Maurer}}]{BBCM95}%
  \BibitemOpen
  \bibfield  {author} {\bibinfo {author} {\bibnamefont {Bennett}, \bibfnamefont
  {Charles~H}}, \bibinfo {author} {\bibfnamefont {Gilles}\ \bibnamefont
  {Brassard}}, \bibinfo {author} {\bibfnamefont {Claude}\ \bibnamefont
  {Cr{\'e}peau}}, \ and\ \bibinfo {author} {\bibfnamefont {Ueli}\ \bibnamefont
  {Maurer}}} (\bibinfo {year} {1995}),\ \bibfield  {title} {\enquote {\bibinfo
  {title} {Generalized privacy amplification},}\ }\href {\doibase
  10.1109/18.476316} {\bibfield  {journal} {\bibinfo  {journal} {IEEE Trans.
  Inf. Theory}\ }\textbf {\bibinfo {volume} {41}}~(\bibinfo {number} {6}),\
  \bibinfo {pages} {1915--1923}}\BibitemShut {NoStop}%
\bibitem [{\citenamefont {Bennett}\ \emph
  {et~al.}(1992{\natexlab{b}})\citenamefont {Bennett}, \citenamefont
  {Brassard}, \citenamefont {Cr{\'{e}}peau},\ and\ \citenamefont
  {Skubiszewska}}]{BBCS92}%
  \BibitemOpen
  \bibfield  {author} {\bibinfo {author} {\bibnamefont {Bennett}, \bibfnamefont
  {Charles~H}}, \bibinfo {author} {\bibfnamefont {Gilles}\ \bibnamefont
  {Brassard}}, \bibinfo {author} {\bibfnamefont {Claude}\ \bibnamefont
  {Cr{\'{e}}peau}}, \ and\ \bibinfo {author} {\bibfnamefont
  {Marie{-}H{\'{e}}l{\`{e}}ne}\ \bibnamefont {Skubiszewska}}} (\bibinfo {year}
  {1992}{\natexlab{b}}),\ \bibfield  {title} {\enquote {\bibinfo {title}
  {Practical quantum oblivious transfer},}\ }in\ \href {\doibase
  10.1007/3-540-46766-1_29} {\emph {\bibinfo {booktitle} {Advances in
  Cryptology -- CRYPTO~'91}}},\ \bibinfo {series} {LNCS}, Vol.\ \bibinfo
  {volume} {576}\ (\bibinfo  {publisher} {Springer})\ pp.\ \bibinfo {pages}
  {351--366}\BibitemShut {NoStop}%
\bibitem [{\citenamefont {Bennett}\ \emph
  {et~al.}(1992{\natexlab{c}})\citenamefont {Bennett}, \citenamefont
  {Brassard},\ and\ \citenamefont {Mermin}}]{BBM92}%
  \BibitemOpen
  \bibfield  {author} {\bibinfo {author} {\bibnamefont {Bennett}, \bibfnamefont
  {Charles~H}}, \bibinfo {author} {\bibfnamefont {Gilles}\ \bibnamefont
  {Brassard}}, \ and\ \bibinfo {author} {\bibfnamefont {N.~David}\ \bibnamefont
  {Mermin}}} (\bibinfo {year} {1992}{\natexlab{c}}),\ \bibfield  {title}
  {\enquote {\bibinfo {title} {Quantum cryptography without {Bell}'s
  theorem},}\ }\href {\doibase 10.1103/PhysRevLett.68.557} {\bibfield
  {journal} {\bibinfo  {journal} {Phys. Rev. Lett.}\ }\textbf {\bibinfo
  {volume} {68}},\ \bibinfo {pages} {557--559}}\BibitemShut {NoStop}%
\bibitem [{\citenamefont {Bennett}\ \emph
  {et~al.}(1996{\natexlab{b}})\citenamefont {Bennett}, \citenamefont
  {Brassard}, \citenamefont {Popescu}, \citenamefont {Schumacher},
  \citenamefont {Smolin},\ and\ \citenamefont {Wootters}}]{Benett96b}%
  \BibitemOpen
  \bibfield  {author} {\bibinfo {author} {\bibnamefont {Bennett}, \bibfnamefont
  {Charles~H}}, \bibinfo {author} {\bibfnamefont {Gilles}\ \bibnamefont
  {Brassard}}, \bibinfo {author} {\bibfnamefont {Sandu}\ \bibnamefont
  {Popescu}}, \bibinfo {author} {\bibfnamefont {Benjamin}\ \bibnamefont
  {Schumacher}}, \bibinfo {author} {\bibfnamefont {John~A.}\ \bibnamefont
  {Smolin}}, \ and\ \bibinfo {author} {\bibfnamefont {William~K.}\ \bibnamefont
  {Wootters}}} (\bibinfo {year} {1996}{\natexlab{b}}),\ \bibfield  {title}
  {\enquote {\bibinfo {title} {Purification of noisy entanglement and faithful
  teleportation via noisy channels},}\ }\href {\doibase
  10.1103/PhysRevLett.76.722} {\bibfield  {journal} {\bibinfo  {journal} {Phys.
  Rev. Lett.}\ }\textbf {\bibinfo {volume} {76}},\ \bibinfo {pages}
  {722--725}}\BibitemShut {NoStop}%
\bibitem [{\citenamefont {Bennett}\ \emph {et~al.}(1988)\citenamefont
  {Bennett}, \citenamefont {Brassard},\ and\ \citenamefont {Robert}}]{BBR88}%
  \BibitemOpen
  \bibfield  {author} {\bibinfo {author} {\bibnamefont {Bennett}, \bibfnamefont
  {Charles~H}}, \bibinfo {author} {\bibfnamefont {Gilles}\ \bibnamefont
  {Brassard}}, \ and\ \bibinfo {author} {\bibfnamefont {Jean-Marc}\
  \bibnamefont {Robert}}} (\bibinfo {year} {1988}),\ \bibfield  {title}
  {\enquote {\bibinfo {title} {Privacy amplification by public discussion},}\
  }\href {\doibase 10.1137/0217014} {\bibfield  {journal} {\bibinfo  {journal}
  {SIAM J. Comput.}\ }\textbf {\bibinfo {volume} {17}}~(\bibinfo {number}
  {2}),\ \bibinfo {pages} {210--229}}\BibitemShut {NoStop}%
\bibitem [{\citenamefont {Berta}\ \emph {et~al.}(2010)\citenamefont {Berta},
  \citenamefont {Christandl}, \citenamefont {Colbeck}, \citenamefont {Renes},\
  and\ \citenamefont {Renner}}]{Berta10}%
  \BibitemOpen
  \bibfield  {author} {\bibinfo {author} {\bibnamefont {Berta}, \bibfnamefont
  {Mario}}, \bibinfo {author} {\bibfnamefont {Matthias}\ \bibnamefont
  {Christandl}}, \bibinfo {author} {\bibfnamefont {Roger}\ \bibnamefont
  {Colbeck}}, \bibinfo {author} {\bibfnamefont {Joseph~M.}\ \bibnamefont
  {Renes}}, \ and\ \bibinfo {author} {\bibfnamefont {Renato}\ \bibnamefont
  {Renner}}} (\bibinfo {year} {2010}),\ \bibfield  {title} {\enquote {\bibinfo
  {title} {The uncertainty principle in the presence of quantum memory},}\
  }\href {\doibase 10.1038/nphys1734} {\bibfield  {journal} {\bibinfo
  {journal} {Nat. Phys.}\ }\textbf {\bibinfo {volume} {6}}~(\bibinfo {number}
  {9}),\ \bibinfo {pages} {659--662}}\BibitemShut {NoStop}%
\bibitem [{\citenamefont {Berta}\ \emph {et~al.}(2017)\citenamefont {Berta},
  \citenamefont {Fawzi},\ and\ \citenamefont {Scholz}}]{BFS17}%
  \BibitemOpen
  \bibfield  {author} {\bibinfo {author} {\bibnamefont {Berta}, \bibfnamefont
  {Mario}}, \bibinfo {author} {\bibfnamefont {Omar}\ \bibnamefont {Fawzi}}, \
  and\ \bibinfo {author} {\bibfnamefont {Volkher~B.}\ \bibnamefont {Scholz}}}
  (\bibinfo {year} {2017}),\ \bibfield  {title} {\enquote {\bibinfo {title}
  {Quantum-proof randomness extractors via operator space theory},}\ }\href
  {\doibase 10.1109/TIT.2016.2627531} {\bibfield  {journal} {\bibinfo
  {journal} {IEEE Trans. Inf. Theory}\ }\textbf {\bibinfo {volume}
  {63}}~(\bibinfo {number} {4}),\ \bibinfo {pages} {2480--2503}},\ \Eprint
  {http://arxiv.org/abs/arxiv:1409.3563} {arxiv:1409.3563} \BibitemShut
  {NoStop}%
\bibitem [{\citenamefont {Biham}\ \emph {et~al.}(2000)\citenamefont {Biham},
  \citenamefont {Boyer}, \citenamefont {Boykin}, \citenamefont {Mor},\ and\
  \citenamefont {Roychowdhury}}]{BBBMR00}%
  \BibitemOpen
  \bibfield  {author} {\bibinfo {author} {\bibnamefont {Biham}, \bibfnamefont
  {Eli}}, \bibinfo {author} {\bibfnamefont {Michel}\ \bibnamefont {Boyer}},
  \bibinfo {author} {\bibfnamefont {P.~Oscar}\ \bibnamefont {Boykin}}, \bibinfo
  {author} {\bibfnamefont {Tal}\ \bibnamefont {Mor}}, \ and\ \bibinfo {author}
  {\bibfnamefont {Vwani}\ \bibnamefont {Roychowdhury}}} (\bibinfo {year}
  {2000}),\ \bibfield  {title} {\enquote {\bibinfo {title} {A proof of the
  security of quantum key distribution (extended abstract)},}\ }in\ \href
  {\doibase 10.1145/335305.335406} {\emph {\bibinfo {booktitle} {Proceedings of
  the 32nd Symposium on Theory of Computing, STOC~'00}}}\ (\bibinfo
  {publisher} {ACM})\ pp.\ \bibinfo {pages} {715--724},\ \Eprint
  {http://arxiv.org/abs/arXiv:quant-ph/9912053} {arXiv:quant-ph/9912053}
  \BibitemShut {NoStop}%
\bibitem [{\citenamefont {Biham}\ \emph {et~al.}(2006)\citenamefont {Biham},
  \citenamefont {Boyer}, \citenamefont {Boykin}, \citenamefont {Mor},\ and\
  \citenamefont {Roychowdhury}}]{BBBMR06}%
  \BibitemOpen
  \bibfield  {author} {\bibinfo {author} {\bibnamefont {Biham}, \bibfnamefont
  {Eli}}, \bibinfo {author} {\bibfnamefont {Michel}\ \bibnamefont {Boyer}},
  \bibinfo {author} {\bibfnamefont {P.~Oscar}\ \bibnamefont {Boykin}}, \bibinfo
  {author} {\bibfnamefont {Tal}\ \bibnamefont {Mor}}, \ and\ \bibinfo {author}
  {\bibfnamefont {Vwani}\ \bibnamefont {Roychowdhury}}} (\bibinfo {year}
  {2006}),\ \bibfield  {title} {\enquote {\bibinfo {title} {A proof of the
  security of quantum key distribution},}\ }\href {\doibase
  10.1007/s00145-005-0011-3} {\bibfield  {journal} {\bibinfo  {journal} {J.
  Crypt.}\ }\textbf {\bibinfo {volume} {19}}~(\bibinfo {number} {4}),\ \bibinfo
  {pages} {381--439}},\ \bibinfo {note} {full version of \textcite{BBBMR00}},\
  \Eprint {http://arxiv.org/abs/arXiv:quant-ph/0511175}
  {arXiv:quant-ph/0511175} \BibitemShut {NoStop}%
\bibitem [{\citenamefont {Biham}\ \emph {et~al.}(2002)\citenamefont {Biham},
  \citenamefont {Boyer}, \citenamefont {Brassard}, \citenamefont {van~de
  Graaf},\ and\ \citenamefont {Mor}}]{BBBvdGM02}%
  \BibitemOpen
  \bibfield  {author} {\bibinfo {author} {\bibnamefont {Biham}, \bibfnamefont
  {Eli}}, \bibinfo {author} {\bibfnamefont {Michel}\ \bibnamefont {Boyer}},
  \bibinfo {author} {\bibfnamefont {Gilles}\ \bibnamefont {Brassard}}, \bibinfo
  {author} {\bibfnamefont {Jeroen}\ \bibnamefont {van~de Graaf}}, \ and\
  \bibinfo {author} {\bibfnamefont {Tal}\ \bibnamefont {Mor}}} (\bibinfo {year}
  {2002}),\ \bibfield  {title} {\enquote {\bibinfo {title} {Security of quantum
  key distribution against all collective attacks},}\ }\href {\doibase
  10.1007/s00453-002-0973-6} {\bibfield  {journal} {\bibinfo  {journal}
  {Algorithmica}\ }\textbf {\bibinfo {volume} {34}}~(\bibinfo {number} {4}),\
  \bibinfo {pages} {372--388}},\ \Eprint
  {http://arxiv.org/abs/quant-ph/9801022} {quant-ph/9801022} \BibitemShut
  {NoStop}%
\bibitem [{\citenamefont {Biham}\ and\ \citenamefont {Mor}(1997)}]{BM97b}%
  \BibitemOpen
  \bibfield  {author} {\bibinfo {author} {\bibnamefont {Biham}, \bibfnamefont
  {Eli}}, \ and\ \bibinfo {author} {\bibfnamefont {Tal}\ \bibnamefont {Mor}}}
  (\bibinfo {year} {1997}),\ \bibfield  {title} {\enquote {\bibinfo {title}
  {Security of quantum cryptography against collective attacks},}\ }\href
  {\doibase 10.1103/PhysRevLett.78.2256} {\bibfield  {journal} {\bibinfo
  {journal} {Phys. Rev. Lett.}\ }\textbf {\bibinfo {volume} {78}},\ \bibinfo
  {pages} {2256--2259}},\ \Eprint {http://arxiv.org/abs/arXiv:quant-ph/9605007}
  {arXiv:quant-ph/9605007} \BibitemShut {NoStop}%
\bibitem [{\citenamefont {Blum}(1983)}]{Blu83}%
  \BibitemOpen
  \bibfield  {author} {\bibinfo {author} {\bibnamefont {Blum}, \bibfnamefont
  {Manuel}}} (\bibinfo {year} {1983}),\ \bibfield  {title} {\enquote {\bibinfo
  {title} {Coin flipping by telephone a protocol for solving impossible
  problems},}\ }\href {\doibase 10.1145/1008908.1008911} {\bibfield  {journal}
  {\bibinfo  {journal} {ACM SIGACT News}\ }\textbf {\bibinfo {volume}
  {15}}~(\bibinfo {number} {1}),\ \bibinfo {pages} {23--27}}\BibitemShut
  {NoStop}%
\bibitem [{\citenamefont {Boileau}\ \emph {et~al.}(2005)\citenamefont
  {Boileau}, \citenamefont {Tamaki}, \citenamefont {Batuwantudawe},
  \citenamefont {Laflamme},\ and\ \citenamefont {Renes}}]{Boileau}%
  \BibitemOpen
  \bibfield  {author} {\bibinfo {author} {\bibnamefont {Boileau}, \bibfnamefont
  {J-C}}, \bibinfo {author} {\bibfnamefont {Kiyoshi}\ \bibnamefont {Tamaki}},
  \bibinfo {author} {\bibfnamefont {Jamie}\ \bibnamefont {Batuwantudawe}},
  \bibinfo {author} {\bibfnamefont {Raymond}\ \bibnamefont {Laflamme}}, \ and\
  \bibinfo {author} {\bibfnamefont {Joseph~M.}\ \bibnamefont {Renes}}}
  (\bibinfo {year} {2005}),\ \bibfield  {title} {\enquote {\bibinfo {title}
  {Unconditional security of a three state quantum key distribution
  protocol},}\ }\href {\doibase 10.1103/PhysRevLett.94.040503} {\bibfield
  {journal} {\bibinfo  {journal} {Phys. Rev. Lett.}\ }\textbf {\bibinfo
  {volume} {94}},\ \bibinfo {pages} {040503}}\BibitemShut {NoStop}%
\bibitem [{\citenamefont {Born}(1926)}]{Born26}%
  \BibitemOpen
  \bibfield  {author} {\bibinfo {author} {\bibnamefont {Born}, \bibfnamefont
  {Max}}} (\bibinfo {year} {1926}),\ \bibfield  {title} {\enquote {\bibinfo
  {title} {Zur {Q}uantenmechanik der {S}to{\ss}vorg{\"a}nge},}\ }\href@noop {}
  {\bibfield  {journal} {\bibinfo  {journal} {Zeitschrift f{\"u}r Physik}\
  }\textbf {\bibinfo {volume} {37}}~(\bibinfo {number} {12}),\ \bibinfo {pages}
  {863--867}}\BibitemShut {NoStop}%
\bibitem [{\citenamefont {Boykin}\ and\ \citenamefont
  {Roychowdhury}(2003)}]{BR03}%
  \BibitemOpen
  \bibfield  {author} {\bibinfo {author} {\bibnamefont {Boykin}, \bibfnamefont
  {P~Oscar}}, \ and\ \bibinfo {author} {\bibfnamefont {Vwani}\ \bibnamefont
  {Roychowdhury}}} (\bibinfo {year} {2003}),\ \bibfield  {title} {\enquote
  {\bibinfo {title} {Optimal encryption of quantum bits},}\ }\href {\doibase
  10.1103/PhysRevA.67.042317} {\bibfield  {journal} {\bibinfo  {journal} {Phys.
  Rev. A}\ }\textbf {\bibinfo {volume} {67}},\ \bibinfo {pages} {042317}},\
  \Eprint {http://arxiv.org/abs/arXiv:quant-ph/0003059}
  {arXiv:quant-ph/0003059} \BibitemShut {NoStop}%
\bibitem [{\citenamefont {Bozzio}\ \emph {et~al.}(2019)\citenamefont {Bozzio},
  \citenamefont {Diamanti},\ and\ \citenamefont {Grosshans}}]{BDG19}%
  \BibitemOpen
  \bibfield  {author} {\bibinfo {author} {\bibnamefont {Bozzio}, \bibfnamefont
  {Mathieu}}, \bibinfo {author} {\bibfnamefont {Eleni}\ \bibnamefont
  {Diamanti}}, \ and\ \bibinfo {author} {\bibfnamefont {Fr\'ed\'eric}\
  \bibnamefont {Grosshans}}} (\bibinfo {year} {2019}),\ \bibfield  {title}
  {\enquote {\bibinfo {title} {Semi-device-independent quantum money with
  coherent states},}\ }\href {\doibase 10.1103/PhysRevA.99.022336} {\bibfield
  {journal} {\bibinfo  {journal} {Phys. Rev. A}\ }\textbf {\bibinfo {volume}
  {99}},\ \bibinfo {pages} {022336}},\ \Eprint
  {http://arxiv.org/abs/arXiv:1812.09256} {arXiv:1812.09256} \BibitemShut
  {NoStop}%
\bibitem [{\citenamefont {Branciard}\ \emph {et~al.}(2012)\citenamefont
  {Branciard}, \citenamefont {Cavalcanti}, \citenamefont {Walborn},
  \citenamefont {Scarani},\ and\ \citenamefont {Wiseman}}]{BCWSW12}%
  \BibitemOpen
  \bibfield  {author} {\bibinfo {author} {\bibnamefont {Branciard},
  \bibfnamefont {Cyril}}, \bibinfo {author} {\bibfnamefont {Eric~G.}\
  \bibnamefont {Cavalcanti}}, \bibinfo {author} {\bibfnamefont {Stephen~P.}\
  \bibnamefont {Walborn}}, \bibinfo {author} {\bibfnamefont {Valerio}\
  \bibnamefont {Scarani}}, \ and\ \bibinfo {author} {\bibfnamefont {Howard~M.}\
  \bibnamefont {Wiseman}}} (\bibinfo {year} {2012}),\ \bibfield  {title}
  {\enquote {\bibinfo {title} {One-sided device-independent quantum key
  distribution: Security, feasibility, and the connection with steering},}\
  }\href {\doibase 10.1103/PhysRevA.85.010301} {\bibfield  {journal} {\bibinfo
  {journal} {Phys. Rev. A}\ }\textbf {\bibinfo {volume} {85}},\ \bibinfo
  {pages} {010301}}\BibitemShut {NoStop}%
\bibitem [{\citenamefont {Brand{\~a}o}\ \emph {et~al.}(2016)\citenamefont
  {Brand{\~a}o}, \citenamefont {Ramanathan}, \citenamefont {Grudka},
  \citenamefont {Horodecki}, \citenamefont {Horodecki}, \citenamefont
  {Horodecki}, \citenamefont {Szarek},\ and\ \citenamefont
  {Wojew{\'o}dka}}]{BRGHHHSW16}%
  \BibitemOpen
  \bibfield  {author} {\bibinfo {author} {\bibnamefont {Brand{\~a}o},
  \bibfnamefont {Fernando G S~L}}, \bibinfo {author} {\bibfnamefont
  {Ravishankar}\ \bibnamefont {Ramanathan}}, \bibinfo {author} {\bibfnamefont
  {Andrzej}\ \bibnamefont {Grudka}}, \bibinfo {author} {\bibfnamefont {Karol}\
  \bibnamefont {Horodecki}}, \bibinfo {author} {\bibfnamefont {Micha\l{}}\
  \bibnamefont {Horodecki}}, \bibinfo {author} {\bibfnamefont {Pawe\l{}}\
  \bibnamefont {Horodecki}}, \bibinfo {author} {\bibfnamefont {Tomasz}\
  \bibnamefont {Szarek}}, \ and\ \bibinfo {author} {\bibfnamefont {Hanna}\
  \bibnamefont {Wojew{\'o}dka}}} (\bibinfo {year} {2016}),\ \bibfield  {title}
  {\enquote {\bibinfo {title} {Realistic noise-tolerant randomness
  amplification using finite number of devices},}\ }\href {\doibase
  10.1038/ncomms11345} {\bibfield  {journal} {\bibinfo  {journal} {Nat.
  Commun.}\ }\textbf {\bibinfo {volume} {7}},\ \bibinfo {pages} {11345}},\
  \Eprint {http://arxiv.org/abs/arXiv:1310.4544} {arXiv:1310.4544} \BibitemShut
  {NoStop}%
\bibitem [{\citenamefont {Brassard}\ \emph {et~al.}(1998)\citenamefont
  {Brassard}, \citenamefont {Cr\'epeau}, \citenamefont {Mayers},\ and\
  \citenamefont {Salvail}}]{BCMS98}%
  \BibitemOpen
  \bibfield  {author} {\bibinfo {author} {\bibnamefont {Brassard},
  \bibfnamefont {Gilles}}, \bibinfo {author} {\bibfnamefont {Claude}\
  \bibnamefont {Cr\'epeau}}, \bibinfo {author} {\bibfnamefont {Dominic}\
  \bibnamefont {Mayers}}, \ and\ \bibinfo {author} {\bibfnamefont {Louis}\
  \bibnamefont {Salvail}}} (\bibinfo {year} {1998}),\ \href@noop {} {\enquote
  {\bibinfo {title} {Defeating classical bit commitments with a quantum
  computer},}\ }\bibinfo {howpublished} {e-print},\ \Eprint
  {http://arxiv.org/abs/arXiv:quant-ph/9806031} {arXiv:quant-ph/9806031}
  \BibitemShut {NoStop}%
\bibitem [{\citenamefont {Brassard}\ \emph {et~al.}(2000)\citenamefont
  {Brassard}, \citenamefont {L\"utkenhaus}, \citenamefont {Mor},\ and\
  \citenamefont {Sanders}}]{Brassardetal2000}%
  \BibitemOpen
  \bibfield  {author} {\bibinfo {author} {\bibnamefont {Brassard},
  \bibfnamefont {Gilles}}, \bibinfo {author} {\bibfnamefont {Norbert}\
  \bibnamefont {L\"utkenhaus}}, \bibinfo {author} {\bibfnamefont {Tal}\
  \bibnamefont {Mor}}, \ and\ \bibinfo {author} {\bibfnamefont {Barry~C.}\
  \bibnamefont {Sanders}}} (\bibinfo {year} {2000}),\ \bibfield  {title}
  {\enquote {\bibinfo {title} {Limitations on practical quantum
  cryptography},}\ }\href {\doibase 10.1103/PhysRevLett.85.1330} {\bibfield
  {journal} {\bibinfo  {journal} {Phys. Rev. Lett.}\ }\textbf {\bibinfo
  {volume} {85}},\ \bibinfo {pages} {1330--1333}}\BibitemShut {NoStop}%
\bibitem [{\citenamefont {Braunstein}\ and\ \citenamefont
  {Pirandola}(2012)}]{BP12}%
  \BibitemOpen
  \bibfield  {author} {\bibinfo {author} {\bibnamefont {Braunstein},
  \bibfnamefont {Samuel~L}}, \ and\ \bibinfo {author} {\bibfnamefont {Stefano}\
  \bibnamefont {Pirandola}}} (\bibinfo {year} {2012}),\ \bibfield  {title}
  {\enquote {\bibinfo {title} {Side-channel-free quantum key distribution},}\
  }\href {\doibase 10.1103/PhysRevLett.108.130502} {\bibfield  {journal}
  {\bibinfo  {journal} {Phys. Rev. Lett.}\ }\textbf {\bibinfo {volume} {108}},\
  \bibinfo {pages} {130502}}\BibitemShut {NoStop}%
\bibitem [{\citenamefont {Broadbent}\ \emph {et~al.}(2009)\citenamefont
  {Broadbent}, \citenamefont {Fitzsimons},\ and\ \citenamefont
  {Kashefi}}]{BFK09}%
  \BibitemOpen
  \bibfield  {author} {\bibinfo {author} {\bibnamefont {Broadbent},
  \bibfnamefont {Anne}}, \bibinfo {author} {\bibfnamefont {Joseph}\
  \bibnamefont {Fitzsimons}}, \ and\ \bibinfo {author} {\bibfnamefont {Elham}\
  \bibnamefont {Kashefi}}} (\bibinfo {year} {2009}),\ \bibfield  {title}
  {\enquote {\bibinfo {title} {Universal blind quantum computation},}\ }in\
  \href {\doibase 10.1109/FOCS.2009.36} {\emph {\bibinfo {booktitle}
  {Proceedings of the 50th Symposium on Foundations of Computer Science,
  FOCS~'09}}}\ (\bibinfo  {publisher} {IEEE Computer Society})\ pp.\ \bibinfo
  {pages} {517--526},\ \Eprint {http://arxiv.org/abs/arXiv:0807.4154}
  {arXiv:0807.4154} \BibitemShut {NoStop}%
\bibitem [{\citenamefont {Broadbent}\ \emph {et~al.}(2013)\citenamefont
  {Broadbent}, \citenamefont {Gutoski},\ and\ \citenamefont {Stebila}}]{BGS13}%
  \BibitemOpen
  \bibfield  {author} {\bibinfo {author} {\bibnamefont {Broadbent},
  \bibfnamefont {Anne}}, \bibinfo {author} {\bibfnamefont {Gus}\ \bibnamefont
  {Gutoski}}, \ and\ \bibinfo {author} {\bibfnamefont {Douglas}\ \bibnamefont
  {Stebila}}} (\bibinfo {year} {2013}),\ \bibfield  {title} {\enquote {\bibinfo
  {title} {Quantum one-time programs},}\ }in\ \href {\doibase
  10.1007/978-3-642-40084-1_20} {\emph {\bibinfo {booktitle} {Advances in
  Cryptology -- CRYPTO 2013}}},\ \bibinfo {series} {LNCS}, Vol.\ \bibinfo
  {volume} {8043}\ (\bibinfo  {publisher} {Springer})\ pp.\ \bibinfo {pages}
  {344--360},\ \Eprint {http://arxiv.org/abs/arXiv:1211.1080} {arXiv:1211.1080}
  \BibitemShut {NoStop}%
\bibitem [{\citenamefont {Broadbent}\ and\ \citenamefont
  {Jeffery}(2015)}]{BJ15}%
  \BibitemOpen
  \bibfield  {author} {\bibinfo {author} {\bibnamefont {Broadbent},
  \bibfnamefont {Anne}}, \ and\ \bibinfo {author} {\bibfnamefont {Stacey}\
  \bibnamefont {Jeffery}}} (\bibinfo {year} {2015}),\ \bibfield  {title}
  {\enquote {\bibinfo {title} {Quantum homomorphic encryption for circuits of
  low t-gate complexity},}\ }in\ \href {\doibase 10.1007/978-3-662-48000-7_30}
  {\emph {\bibinfo {booktitle} {Advances in Cryptology -- CRYPTO 2015}}},\
  \bibinfo {editor} {edited by\ \bibinfo {editor} {\bibfnamefont {Rosario}\
  \bibnamefont {Gennaro}}\ and\ \bibinfo {editor} {\bibfnamefont {Matthew}\
  \bibnamefont {Robshaw}}}\ (\bibinfo  {publisher} {Springer})\ pp.\ \bibinfo
  {pages} {609--629},\ \Eprint {http://arxiv.org/abs/arXiv:1412.8766}
  {arXiv:1412.8766} \BibitemShut {NoStop}%
\bibitem [{\citenamefont {Broadbent}\ and\ \citenamefont
  {Schaffner}(2016)}]{BS16}%
  \BibitemOpen
  \bibfield  {author} {\bibinfo {author} {\bibnamefont {Broadbent},
  \bibfnamefont {Anne}}, \ and\ \bibinfo {author} {\bibfnamefont {Christian}\
  \bibnamefont {Schaffner}}} (\bibinfo {year} {2016}),\ \bibfield  {title}
  {\enquote {\bibinfo {title} {Quantum cryptography beyond quantum key
  distribution},}\ }\href {\doibase 10.1007/s10623-015-0157-4} {\bibfield
  {journal} {\bibinfo  {journal} {Des. Codes Cryptogr.}\ }\textbf {\bibinfo
  {volume} {78}}~(\bibinfo {number} {1}),\ \bibinfo {pages} {351--382}},\
  \Eprint {http://arxiv.org/abs/arXiv:1510.06120} {arXiv:1510.06120}
  \BibitemShut {NoStop}%
\bibitem [{\citenamefont {Broadbent}\ and\ \citenamefont
  {Wainewright}(2016)}]{BW16}%
  \BibitemOpen
  \bibfield  {author} {\bibinfo {author} {\bibnamefont {Broadbent},
  \bibfnamefont {Anne}}, \ and\ \bibinfo {author} {\bibfnamefont {Evelyn}\
  \bibnamefont {Wainewright}}} (\bibinfo {year} {2016}),\ \bibfield  {title}
  {\enquote {\bibinfo {title} {Efficient simulation for quantum message
  authentication},}\ }in\ \href {\doibase 10.1007/978-3-319-49175-2_4} {\emph
  {\bibinfo {booktitle} {Proceedings of the 9th International Conference on
  Information Theoretic Security, ICITS 2016}}}\ (\bibinfo  {publisher}
  {Springer})\ pp.\ \bibinfo {pages} {72--91},\ \Eprint
  {http://arxiv.org/abs/arXiv:1607.03075} {arXiv:1607.03075} \BibitemShut
  {NoStop}%
\bibitem [{\citenamefont {Brunner}\ \emph {et~al.}(2014)\citenamefont
  {Brunner}, \citenamefont {Cavalcanti}, \citenamefont {Pironio}, \citenamefont
  {Scarani},\ and\ \citenamefont {Wehner}}]{BCPSW14}%
  \BibitemOpen
  \bibfield  {author} {\bibinfo {author} {\bibnamefont {Brunner}, \bibfnamefont
  {Nicolas}}, \bibinfo {author} {\bibfnamefont {Daniel}\ \bibnamefont
  {Cavalcanti}}, \bibinfo {author} {\bibfnamefont {Stefano}\ \bibnamefont
  {Pironio}}, \bibinfo {author} {\bibfnamefont {Valerio}\ \bibnamefont
  {Scarani}}, \ and\ \bibinfo {author} {\bibfnamefont {Stephanie}\ \bibnamefont
  {Wehner}}} (\bibinfo {year} {2014}),\ \bibfield  {title} {\enquote {\bibinfo
  {title} {Bell nonlocality},}\ }\href {\doibase 10.1103/RevModPhys.86.419}
  {\bibfield  {journal} {\bibinfo  {journal} {Rev. Mod. Phys.}\ }\textbf
  {\bibinfo {volume} {86}},\ \bibinfo {pages} {419--478}},\ \Eprint
  {http://arxiv.org/abs/arXiv:1303.2849} {arXiv:1303.2849} \BibitemShut
  {NoStop}%
\bibitem [{\citenamefont {Buhrman}\ \emph {et~al.}(2014)\citenamefont
  {Buhrman}, \citenamefont {Chandran}, \citenamefont {Fehr}, \citenamefont
  {Gelles}, \citenamefont {Goyal}, \citenamefont {Ostrovsky},\ and\
  \citenamefont {Schaffner}}]{BCFGGOS14}%
  \BibitemOpen
  \bibfield  {author} {\bibinfo {author} {\bibnamefont {Buhrman}, \bibfnamefont
  {Harry}}, \bibinfo {author} {\bibfnamefont {Nishanth}\ \bibnamefont
  {Chandran}}, \bibinfo {author} {\bibfnamefont {Serge}\ \bibnamefont {Fehr}},
  \bibinfo {author} {\bibfnamefont {Ran}\ \bibnamefont {Gelles}}, \bibinfo
  {author} {\bibfnamefont {Vipul}\ \bibnamefont {Goyal}}, \bibinfo {author}
  {\bibfnamefont {Rafail}\ \bibnamefont {Ostrovsky}}, \ and\ \bibinfo {author}
  {\bibfnamefont {Christian}\ \bibnamefont {Schaffner}}} (\bibinfo {year}
  {2014}),\ \bibfield  {title} {\enquote {\bibinfo {title} {Position-based
  quantum cryptography: Impossibility and constructions},}\ }\href {\doibase
  10.1137/130913687} {\bibfield  {journal} {\bibinfo  {journal} {SIAM J.
  Comput.}\ }\textbf {\bibinfo {volume} {43}}~(\bibinfo {number} {1}),\
  \bibinfo {pages} {150--178}},\ \bibinfo {note} {a preliminary version
  appeared at CRYPTO 2011},\ \Eprint {http://arxiv.org/abs/arXiv:1009.2490}
  {arXiv:1009.2490} \BibitemShut {NoStop}%
\bibitem [{\citenamefont {Calderbank}\ and\ \citenamefont {Shor}(1996)}]{CS96}%
  \BibitemOpen
  \bibfield  {author} {\bibinfo {author} {\bibnamefont {Calderbank},
  \bibfnamefont {A~R}}, \ and\ \bibinfo {author} {\bibfnamefont {Peter~W.}\
  \bibnamefont {Shor}}} (\bibinfo {year} {1996}),\ \bibfield  {title} {\enquote
  {\bibinfo {title} {Good quantum error-correcting codes exist},}\ }\href
  {\doibase 10.1103/PhysRevA.54.1098} {\bibfield  {journal} {\bibinfo
  {journal} {Phys. Rev. A}\ }\textbf {\bibinfo {volume} {54}},\ \bibinfo
  {pages} {1098--1105}}\BibitemShut {NoStop}%
\bibitem [{\citenamefont {Canetti}(2000)}]{Can00}%
  \BibitemOpen
  \bibfield  {author} {\bibinfo {author} {\bibnamefont {Canetti}, \bibfnamefont
  {Ran}}} (\bibinfo {year} {2000}),\ \bibfield  {title} {\enquote {\bibinfo
  {title} {Security and composition of multiparty cryptographic protocols},}\
  }\href {\doibase 10.1007/s001459910006} {\bibfield  {journal} {\bibinfo
  {journal} {J. Crypt.}\ }\textbf {\bibinfo {volume} {13}}~(\bibinfo {number}
  {1}),\ \bibinfo {pages} {143--202}},\ \bibinfo {note} {e-Print
  \href{http://eprint.iacr.org/1998/018}{IACR 1998/018}}\BibitemShut {NoStop}%
\bibitem [{\citenamefont {Canetti}(2001)}]{Can01}%
  \BibitemOpen
  \bibfield  {author} {\bibinfo {author} {\bibnamefont {Canetti}, \bibfnamefont
  {Ran}}} (\bibinfo {year} {2001}),\ \bibfield  {title} {\enquote {\bibinfo
  {title} {Universally composable security: A new paradigm for cryptographic
  protocols},}\ }in\ \href {\doibase 10.1109/SFCS.2001.959888} {\emph {\bibinfo
  {booktitle} {Proceedings of the 42nd Symposium on Foundations of Computer
  Science, FOCS~'01}}}\ (\bibinfo  {publisher} {IEEE})\ pp.\ \bibinfo {pages}
  {136--145}\BibitemShut {NoStop}%
\bibitem [{\citenamefont {Canetti}(2020)}]{Can20}%
  \BibitemOpen
  \bibfield  {author} {\bibinfo {author} {\bibnamefont {Canetti}, \bibfnamefont
  {Ran}}} (\bibinfo {year} {2020}),\ \href@noop {} {\enquote {\bibinfo {title}
  {Universally composable security: A new paradigm for cryptographic
  protocols},}\ }\bibinfo {howpublished} {e-Print
  \href{http://eprint.iacr.org/2000/067}{IACR 2000/067}},\ \bibinfo {note}
  {updated version of~\textcite{Can01}}\BibitemShut {NoStop}%
\bibitem [{\citenamefont {Canetti}\ \emph
  {et~al.}(2006{\natexlab{a}})\citenamefont {Canetti}, \citenamefont {Cheung},
  \citenamefont {Kaynar}, \citenamefont {Liskov}, \citenamefont {Lynch},
  \citenamefont {Pereira},\ and\ \citenamefont {Segala}}]{CCKLLPS06a}%
  \BibitemOpen
  \bibfield  {author} {\bibinfo {author} {\bibnamefont {Canetti}, \bibfnamefont
  {Ran}}, \bibinfo {author} {\bibfnamefont {Ling}\ \bibnamefont {Cheung}},
  \bibinfo {author} {\bibfnamefont {Dilsun~Kirli}\ \bibnamefont {Kaynar}},
  \bibinfo {author} {\bibfnamefont {Moses}\ \bibnamefont {Liskov}}, \bibinfo
  {author} {\bibfnamefont {Nancy~A.}\ \bibnamefont {Lynch}}, \bibinfo {author}
  {\bibfnamefont {Olivier}\ \bibnamefont {Pereira}}, \ and\ \bibinfo {author}
  {\bibfnamefont {Roberto}\ \bibnamefont {Segala}}} (\bibinfo {year}
  {2006}{\natexlab{a}}),\ \bibfield  {title} {\enquote {\bibinfo {title}
  {Task-structured probabilistic {I/O} automata},}\ }in\ \href {\doibase
  10.1109/WODES.2006.1678432} {\emph {\bibinfo {booktitle} {Proceedings of the
  8th International Workshop on Discrete Event Systems, {WODES} 2006}}}\
  (\bibinfo  {publisher} {IEEE})\ pp.\ \bibinfo {pages} {207--214},\ \bibinfo
  {note} {extended version available at
  \url{http://theory.csail.mit.edu/~lcheung/papers/task-PIOA-TR.pdf}}\BibitemShut
  {NoStop}%
\bibitem [{\citenamefont {Canetti}\ \emph
  {et~al.}(2006{\natexlab{b}})\citenamefont {Canetti}, \citenamefont {Cheung},
  \citenamefont {Kaynar}, \citenamefont {Liskov}, \citenamefont {Lynch},
  \citenamefont {Pereira},\ and\ \citenamefont {Segala}}]{CCKLLPS06b}%
  \BibitemOpen
  \bibfield  {author} {\bibinfo {author} {\bibnamefont {Canetti}, \bibfnamefont
  {Ran}}, \bibinfo {author} {\bibfnamefont {Ling}\ \bibnamefont {Cheung}},
  \bibinfo {author} {\bibfnamefont {Dilsun~Kirli}\ \bibnamefont {Kaynar}},
  \bibinfo {author} {\bibfnamefont {Moses}\ \bibnamefont {Liskov}}, \bibinfo
  {author} {\bibfnamefont {Nancy~A.}\ \bibnamefont {Lynch}}, \bibinfo {author}
  {\bibfnamefont {Olivier}\ \bibnamefont {Pereira}}, \ and\ \bibinfo {author}
  {\bibfnamefont {Roberto}\ \bibnamefont {Segala}}} (\bibinfo {year}
  {2006}{\natexlab{b}}),\ \bibfield  {title} {\enquote {\bibinfo {title}
  {Time-bounded task-{PIOAs}: {A} framework for analyzing security
  protocols},}\ }in\ \href {\doibase 10.1007/11864219_17} {\emph {\bibinfo
  {booktitle} {Proceedings of the 20th International Symposium on Distributed
  Computing, {DISC} 2006}}},\ pp.\ \bibinfo {pages} {238--253}\BibitemShut
  {NoStop}%
\bibitem [{\citenamefont {Canetti}\ \emph {et~al.}(2007)\citenamefont
  {Canetti}, \citenamefont {Dodis}, \citenamefont {Pass},\ and\ \citenamefont
  {Walfish}}]{CDPW07}%
  \BibitemOpen
  \bibfield  {author} {\bibinfo {author} {\bibnamefont {Canetti}, \bibfnamefont
  {Ran}}, \bibinfo {author} {\bibfnamefont {Yevgeniy}\ \bibnamefont {Dodis}},
  \bibinfo {author} {\bibfnamefont {Rafael}\ \bibnamefont {Pass}}, \ and\
  \bibinfo {author} {\bibfnamefont {Shabsi}\ \bibnamefont {Walfish}}} (\bibinfo
  {year} {2007}),\ \bibfield  {title} {\enquote {\bibinfo {title} {Universally
  composable security with global setup},}\ }in\ \href {\doibase
  10.1007/978-3-540-70936-7_4} {\emph {\bibinfo {booktitle} {Theory of
  Cryptography, Proceedings of TCC 2007}}},\ \bibinfo {series} {LNCS}, Vol.\
  \bibinfo {volume} {4392}\ (\bibinfo  {publisher} {Springer})\ pp.\ \bibinfo
  {pages} {61--85},\ \bibinfo {note} {e-Print
  \href{http://eprint.iacr.org/2006/432}{IACR 2006/432}}\BibitemShut {NoStop}%
\bibitem [{\citenamefont {Canetti}\ and\ \citenamefont
  {Fischlin}(2001)}]{CF01}%
  \BibitemOpen
  \bibfield  {author} {\bibinfo {author} {\bibnamefont {Canetti}, \bibfnamefont
  {Ran}}, \ and\ \bibinfo {author} {\bibfnamefont {Marc}\ \bibnamefont
  {Fischlin}}} (\bibinfo {year} {2001}),\ \bibfield  {title} {\enquote
  {\bibinfo {title} {Universally composable commitments},}\ }in\ \href
  {\doibase 10.1007/3-540-44647-8_2} {\emph {\bibinfo {booktitle} {Advances in
  Cryptology --- CRYPTO 2001}}},\ \bibinfo {editor} {edited by\ \bibinfo
  {editor} {\bibfnamefont {Joe}\ \bibnamefont {Kilian}}}\ (\bibinfo
  {publisher} {Springer})\ pp.\ \bibinfo {pages} {19--40},\ \bibinfo {note}
  {e-Print \href{http://eprint.iacr.org/2001/055}{IACR 2001/055}}\BibitemShut
  {NoStop}%
\bibitem [{\citenamefont {Canetti}\ \emph {et~al.}(2003)\citenamefont
  {Canetti}, \citenamefont {Krawczyk},\ and\ \citenamefont {Nielsen}}]{CKN03}%
  \BibitemOpen
  \bibfield  {author} {\bibinfo {author} {\bibnamefont {Canetti}, \bibfnamefont
  {Ran}}, \bibinfo {author} {\bibfnamefont {Hugo}\ \bibnamefont {Krawczyk}}, \
  and\ \bibinfo {author} {\bibfnamefont {Jesper~B.}\ \bibnamefont {Nielsen}}}
  (\bibinfo {year} {2003}),\ \bibfield  {title} {\enquote {\bibinfo {title}
  {Relaxing chosen-ciphertext security},}\ }in\ \href {\doibase
  10.1007/978-3-540-45146-4_33} {\emph {\bibinfo {booktitle} {Advances in
  Cryptology -- CRYPTO 2003}}},\ \bibinfo {editor} {edited by\ \bibinfo
  {editor} {\bibfnamefont {Dan}\ \bibnamefont {Boneh}}}\ (\bibinfo  {publisher}
  {Springer})\ pp.\ \bibinfo {pages} {565--582}\BibitemShut {NoStop}%
\bibitem [{\citenamefont {Canetti}\ \emph {et~al.}(2002)\citenamefont
  {Canetti}, \citenamefont {Lindell}, \citenamefont {Ostrovsky},\ and\
  \citenamefont {Sahai}}]{CLOS02}%
  \BibitemOpen
  \bibfield  {author} {\bibinfo {author} {\bibnamefont {Canetti}, \bibfnamefont
  {Ran}}, \bibinfo {author} {\bibfnamefont {Yehuda}\ \bibnamefont {Lindell}},
  \bibinfo {author} {\bibfnamefont {Rafail}\ \bibnamefont {Ostrovsky}}, \ and\
  \bibinfo {author} {\bibfnamefont {Amit}\ \bibnamefont {Sahai}}} (\bibinfo
  {year} {2002}),\ \bibfield  {title} {\enquote {\bibinfo {title} {Universally
  composable two-party and multi-party secure computation},}\ }in\ \href
  {\doibase 10.1145/509907.509980} {\emph {\bibinfo {booktitle} {Proceedings of
  the 34th Symposium on Theory of Computing, STOC~'02}}}\ (\bibinfo
  {publisher} {ACM})\ p.\ \bibinfo {pages} {494–503},\ \bibinfo {note}
  {e-Print \href{http://eprint.iacr.org/2002/140}{IACR 2002/140}}\BibitemShut
  {NoStop}%
\bibitem [{\citenamefont {Carter}\ and\ \citenamefont {Wegman}(1979)}]{CW79}%
  \BibitemOpen
  \bibfield  {author} {\bibinfo {author} {\bibnamefont {Carter}, \bibfnamefont
  {Larry}}, \ and\ \bibinfo {author} {\bibfnamefont {Mark~N.}\ \bibnamefont
  {Wegman}}} (\bibinfo {year} {1979}),\ \bibfield  {title} {\enquote {\bibinfo
  {title} {Universal classes of hash functions},}\ }\href {\doibase
  10.1016/0022-0000(79)90044-8} {\bibfield  {journal} {\bibinfo  {journal} {J.
  Comput. Syst. Sci.}\ }\textbf {\bibinfo {volume} {18}}~(\bibinfo {number}
  {2}),\ \bibinfo {pages} {143--154}}\BibitemShut {NoStop}%
\bibitem [{\citenamefont {Chandran}\ \emph {et~al.}(2009)\citenamefont
  {Chandran}, \citenamefont {Goyal}, \citenamefont {Moriarty},\ and\
  \citenamefont {Ostrovsky}}]{CGMO09}%
  \BibitemOpen
  \bibfield  {author} {\bibinfo {author} {\bibnamefont {Chandran},
  \bibfnamefont {Nishanth}}, \bibinfo {author} {\bibfnamefont {Vipul}\
  \bibnamefont {Goyal}}, \bibinfo {author} {\bibfnamefont {Ryan}\ \bibnamefont
  {Moriarty}}, \ and\ \bibinfo {author} {\bibfnamefont {Rafail}\ \bibnamefont
  {Ostrovsky}}} (\bibinfo {year} {2009}),\ \bibfield  {title} {\enquote
  {\bibinfo {title} {Position based cryptography},}\ }in\ \href {\doibase
  10.1007/978-3-642-03356-8_23} {\emph {\bibinfo {booktitle} {Advances in
  Cryptology -- CRYPTO 2009}}},\ \bibinfo {editor} {edited by\ \bibinfo
  {editor} {\bibfnamefont {Shai}\ \bibnamefont {Halevi}}}\ (\bibinfo
  {publisher} {Springer})\ pp.\ \bibinfo {pages} {391--407}\BibitemShut
  {NoStop}%
\bibitem [{\citenamefont {Chen}\ \emph {et~al.}(2017)\citenamefont {Chen},
  \citenamefont {Chung}, \citenamefont {Lai}, \citenamefont {Vadhan},\ and\
  \citenamefont {Wu}}]{CCLVW17}%
  \BibitemOpen
  \bibfield  {author} {\bibinfo {author} {\bibnamefont {Chen}, \bibfnamefont
  {Yi-Hsiu}}, \bibinfo {author} {\bibfnamefont {Kai-Min}\ \bibnamefont
  {Chung}}, \bibinfo {author} {\bibfnamefont {Ching-Yi}\ \bibnamefont {Lai}},
  \bibinfo {author} {\bibfnamefont {Salil~P.}\ \bibnamefont {Vadhan}}, \ and\
  \bibinfo {author} {\bibfnamefont {Xiaodi}\ \bibnamefont {Wu}}} (\bibinfo
  {year} {2017}),\ \href@noop {} {\enquote {\bibinfo {title} {Computational
  notions of quantum min-entropy},}\ }\bibinfo {howpublished} {e-print},\
  \Eprint {http://arxiv.org/abs/arXiv:1704.07309} {arXiv:1704.07309}
  \BibitemShut {NoStop}%
\bibitem [{\citenamefont {Childs}(2005)}]{Chi05}%
  \BibitemOpen
  \bibfield  {author} {\bibinfo {author} {\bibnamefont {Childs}, \bibfnamefont
  {Andrew~M}}} (\bibinfo {year} {2005}),\ \bibfield  {title} {\enquote
  {\bibinfo {title} {Secure assisted quantum computation},}\ }\href@noop {}
  {\bibfield  {journal} {\bibinfo  {journal} {Quantum Inf. Comput.}\ }\textbf
  {\bibinfo {volume} {5}}~(\bibinfo {number} {6}),\ \bibinfo {pages}
  {456--466}},\ \Eprint {http://arxiv.org/abs/arXiv:quant-ph/0111046}
  {arXiv:quant-ph/0111046} \BibitemShut {NoStop}%
\bibitem [{\citenamefont {Chiribella}\ \emph {et~al.}(2009)\citenamefont
  {Chiribella}, \citenamefont {D'Ariano},\ and\ \citenamefont
  {Perinotti}}]{CDP09}%
  \BibitemOpen
  \bibfield  {author} {\bibinfo {author} {\bibnamefont {Chiribella},
  \bibfnamefont {Giulio}}, \bibinfo {author} {\bibfnamefont {Giacomo~Mauro}\
  \bibnamefont {D'Ariano}}, \ and\ \bibinfo {author} {\bibfnamefont {Paolo}\
  \bibnamefont {Perinotti}}} (\bibinfo {year} {2009}),\ \bibfield  {title}
  {\enquote {\bibinfo {title} {Theoretical framework for quantum networks},}\
  }\href {\doibase 10.1103/PhysRevA.80.022339} {\bibfield  {journal} {\bibinfo
  {journal} {Phys. Rev. A}\ }\textbf {\bibinfo {volume} {80}},\ \bibinfo
  {pages} {022339}},\ \Eprint {http://arxiv.org/abs/arXiv:0904.4483}
  {arXiv:0904.4483} \BibitemShut {NoStop}%
\bibitem [{\citenamefont {Chitambar}\ and\ \citenamefont {Gour}(2019)}]{CG19}%
  \BibitemOpen
  \bibfield  {author} {\bibinfo {author} {\bibnamefont {Chitambar},
  \bibfnamefont {Eric}}, \ and\ \bibinfo {author} {\bibfnamefont {Gilad}\
  \bibnamefont {Gour}}} (\bibinfo {year} {2019}),\ \bibfield  {title} {\enquote
  {\bibinfo {title} {Quantum resource theories},}\ }\href {\doibase
  10.1103/RevModPhys.91.025001} {\bibfield  {journal} {\bibinfo  {journal}
  {Rev. Mod. Phys.}\ }\textbf {\bibinfo {volume} {91}},\ \bibinfo {pages}
  {025001}}\BibitemShut {NoStop}%
\bibitem [{\citenamefont {Christandl}\ \emph {et~al.}(2007)\citenamefont
  {Christandl}, \citenamefont {Ekert}, \citenamefont {Horodecki}, \citenamefont
  {Horodecki}, \citenamefont {Oppenheim},\ and\ \citenamefont
  {Renner}}]{christandl2007unifying}%
  \BibitemOpen
  \bibfield  {author} {\bibinfo {author} {\bibnamefont {Christandl},
  \bibfnamefont {Matthias}}, \bibinfo {author} {\bibfnamefont {Artur}\
  \bibnamefont {Ekert}}, \bibinfo {author} {\bibfnamefont {Micha{\l}}\
  \bibnamefont {Horodecki}}, \bibinfo {author} {\bibfnamefont {Pawe{\l}}\
  \bibnamefont {Horodecki}}, \bibinfo {author} {\bibfnamefont {Jonathan}\
  \bibnamefont {Oppenheim}}, \ and\ \bibinfo {author} {\bibfnamefont {Renato}\
  \bibnamefont {Renner}}} (\bibinfo {year} {2007}),\ \bibfield  {title}
  {\enquote {\bibinfo {title} {Unifying classical and quantum key
  distillation},}\ }in\ \href {\doibase 10.1007/978-3-540-70936-7_25} {\emph
  {\bibinfo {booktitle} {Theory of Cryptography Conference, Proceedings of
  {TCC} 2007}}},\ \bibinfo {series} {LNCS}, Vol.\ \bibinfo {volume} {4392},\
  \bibinfo {editor} {edited by\ \bibinfo {editor} {\bibfnamefont {Salil~P.}\
  \bibnamefont {Vadhan}}}\ (\bibinfo  {publisher} {Springer})\ pp.\ \bibinfo
  {pages} {456--478},\ \Eprint {http://arxiv.org/abs/arXiv:quant-ph/0608199}
  {arXiv:quant-ph/0608199} \BibitemShut {NoStop}%
\bibitem [{\citenamefont {Christandl}\ \emph {et~al.}(2009)\citenamefont
  {Christandl}, \citenamefont {K\"onig},\ and\ \citenamefont {Renner}}]{CKR09}%
  \BibitemOpen
  \bibfield  {author} {\bibinfo {author} {\bibnamefont {Christandl},
  \bibfnamefont {Matthias}}, \bibinfo {author} {\bibfnamefont {Robert}\
  \bibnamefont {K\"onig}}, \ and\ \bibinfo {author} {\bibfnamefont {Renato}\
  \bibnamefont {Renner}}} (\bibinfo {year} {2009}),\ \bibfield  {title}
  {\enquote {\bibinfo {title} {Postselection technique for quantum channels
  with applications to quantum cryptography},}\ }\href {\doibase
  10.1103/PhysRevLett.102.020504} {\bibfield  {journal} {\bibinfo  {journal}
  {Phys. Rev. Lett.}\ }\textbf {\bibinfo {volume} {102}},\ \bibinfo {pages}
  {020504}},\ \Eprint {http://arxiv.org/abs/arXiv:0809.3019} {arXiv:0809.3019}
  \BibitemShut {NoStop}%
\bibitem [{\citenamefont {Christandl}\ \emph {et~al.}(2004)\citenamefont
  {Christandl}, \citenamefont {Renner},\ and\ \citenamefont {Ekert}}]{CRE04}%
  \BibitemOpen
  \bibfield  {author} {\bibinfo {author} {\bibnamefont {Christandl},
  \bibfnamefont {Matthias}}, \bibinfo {author} {\bibfnamefont {Renato}\
  \bibnamefont {Renner}}, \ and\ \bibinfo {author} {\bibfnamefont {Artur}\
  \bibnamefont {Ekert}}} (\bibinfo {year} {2004}),\ \href@noop {} {\enquote
  {\bibinfo {title} {A generic security proof for quantum key distribution},}\
  }\bibinfo {howpublished} {e-Print},\ \Eprint
  {http://arxiv.org/abs/arXiv:quant-ph/0402131} {arXiv:quant-ph/0402131}
  \BibitemShut {NoStop}%
\bibitem [{\citenamefont {Christensen}\ \emph {et~al.}(2013)\citenamefont
  {Christensen}, \citenamefont {McCusker}, \citenamefont {Altepeter},
  \citenamefont {Calkins}, \citenamefont {Gerrits}, \citenamefont {Lita},
  \citenamefont {Miller}, \citenamefont {Shalm}, \citenamefont {Zhang},
  \citenamefont {Nam}, \citenamefont {Brunner}, \citenamefont {Lim},
  \citenamefont {Gisin},\ and\ \citenamefont {Kwiat}}]{Christensen}%
  \BibitemOpen
  \bibfield  {author} {\bibinfo {author} {\bibnamefont {Christensen},
  \bibfnamefont {Bradley~G}}, \bibinfo {author} {\bibfnamefont {Kevin~T.}\
  \bibnamefont {McCusker}}, \bibinfo {author} {\bibfnamefont {J.~B.}\
  \bibnamefont {Altepeter}}, \bibinfo {author} {\bibfnamefont {Brice}\
  \bibnamefont {Calkins}}, \bibinfo {author} {\bibfnamefont {Thomas}\
  \bibnamefont {Gerrits}}, \bibinfo {author} {\bibfnamefont {Adriana~E.}\
  \bibnamefont {Lita}}, \bibinfo {author} {\bibfnamefont {Aaron}\ \bibnamefont
  {Miller}}, \bibinfo {author} {\bibfnamefont {L.~K.}\ \bibnamefont {Shalm}},
  \bibinfo {author} {\bibfnamefont {Y.}~\bibnamefont {Zhang}}, \bibinfo
  {author} {\bibfnamefont {S.~W.}\ \bibnamefont {Nam}}, \bibinfo {author}
  {\bibfnamefont {Nicolas}\ \bibnamefont {Brunner}}, \bibinfo {author}
  {\bibfnamefont {Charles Ci~Wen}\ \bibnamefont {Lim}}, \bibinfo {author}
  {\bibfnamefont {Nicolas}\ \bibnamefont {Gisin}}, \ and\ \bibinfo {author}
  {\bibfnamefont {Paul~G.}\ \bibnamefont {Kwiat}}} (\bibinfo {year} {2013}),\
  \bibfield  {title} {\enquote {\bibinfo {title} {Detection-loophole-free test
  of quantum nonlocality, and applications},}\ }\href {\doibase
  10.1103/PhysRevLett.111.130406} {\bibfield  {journal} {\bibinfo  {journal}
  {Phys. Rev. Lett.}\ }\textbf {\bibinfo {volume} {111}},\ \bibinfo {pages}
  {130406}}\BibitemShut {NoStop}%
\bibitem [{\citenamefont {Chung}\ \emph
  {et~al.}(2014{\natexlab{a}})\citenamefont {Chung}, \citenamefont {Li},\ and\
  \citenamefont {Wu}}]{CLW14}%
  \BibitemOpen
  \bibfield  {author} {\bibinfo {author} {\bibnamefont {Chung}, \bibfnamefont
  {Kai-Min}}, \bibinfo {author} {\bibfnamefont {Xin}\ \bibnamefont {Li}}, \
  and\ \bibinfo {author} {\bibfnamefont {Xiaodi}\ \bibnamefont {Wu}}} (\bibinfo
  {year} {2014}{\natexlab{a}}),\ \href@noop {} {\enquote {\bibinfo {title}
  {Multi-source randomness extractors against quantum side information, and
  their applications},}\ }\bibinfo {howpublished} {e-Print},\ \Eprint
  {http://arxiv.org/abs/arXiv:1411.2315} {arXiv:1411.2315} \BibitemShut
  {NoStop}%
\bibitem [{\citenamefont {Chung}\ \emph
  {et~al.}(2014{\natexlab{b}})\citenamefont {Chung}, \citenamefont {Shi},\ and\
  \citenamefont {Wu}}]{CSW14}%
  \BibitemOpen
  \bibfield  {author} {\bibinfo {author} {\bibnamefont {Chung}, \bibfnamefont
  {Kai-Min}}, \bibinfo {author} {\bibfnamefont {Yaoyun}\ \bibnamefont {Shi}}, \
  and\ \bibinfo {author} {\bibfnamefont {Xiaodi}\ \bibnamefont {Wu}}} (\bibinfo
  {year} {2014}{\natexlab{b}}),\ \href@noop {} {\enquote {\bibinfo {title}
  {Physical randomness extractors: Generating random numbers with minimal
  assumptions},}\ }\bibinfo {howpublished} {e-Print},\ \Eprint
  {http://arxiv.org/abs/arXiv:1402.4797} {arXiv:1402.4797} \BibitemShut
  {NoStop}%
\bibitem [{\citenamefont {Clauser}\ \emph {et~al.}(1969)\citenamefont
  {Clauser}, \citenamefont {Horne}, \citenamefont {Shimony},\ and\
  \citenamefont {Holt}}]{CHSH69}%
  \BibitemOpen
  \bibfield  {author} {\bibinfo {author} {\bibnamefont {Clauser}, \bibfnamefont
  {John}}, \bibinfo {author} {\bibfnamefont {Michael}\ \bibnamefont {Horne}},
  \bibinfo {author} {\bibfnamefont {Abner}\ \bibnamefont {Shimony}}, \ and\
  \bibinfo {author} {\bibfnamefont {Richard}\ \bibnamefont {Holt}}} (\bibinfo
  {year} {1969}),\ \bibfield  {title} {\enquote {\bibinfo {title} {Proposed
  experiment to test local hidden-variable theories},}\ }\href {\doibase
  10.1103/PhysRevLett.23.880} {\bibfield  {journal} {\bibinfo  {journal} {Phys.
  Rev. Lett.}\ }\textbf {\bibinfo {volume} {23}}~(\bibinfo {number} {15}),\
  \bibinfo {pages} {880--884}}\BibitemShut {NoStop}%
\bibitem [{\citenamefont {Coffman}\ \emph {et~al.}(2000)\citenamefont
  {Coffman}, \citenamefont {Kundu},\ and\ \citenamefont
  {Wootters}}]{Coffman00}%
  \BibitemOpen
  \bibfield  {author} {\bibinfo {author} {\bibnamefont {Coffman}, \bibfnamefont
  {Valerie}}, \bibinfo {author} {\bibfnamefont {Joydip}\ \bibnamefont {Kundu}},
  \ and\ \bibinfo {author} {\bibfnamefont {William~K.}\ \bibnamefont
  {Wootters}}} (\bibinfo {year} {2000}),\ \bibfield  {title} {\enquote
  {\bibinfo {title} {Distributed entanglement},}\ }\href {\doibase
  10.1103/PhysRevA.61.052306} {\bibfield  {journal} {\bibinfo  {journal} {Phys.
  Rev. A}\ }\textbf {\bibinfo {volume} {61}},\ \bibinfo {pages}
  {052306}}\BibitemShut {NoStop}%
\bibitem [{\citenamefont {Colbeck}(2006)}]{Col06}%
  \BibitemOpen
  \bibfield  {author} {\bibinfo {author} {\bibnamefont {Colbeck}, \bibfnamefont
  {Roger}}} (\bibinfo {year} {2006}),\ \emph {\bibinfo {title} {Quantum And
  Relativistic Protocols For Secure Multi-Party Computation}},\ \href@noop {}
  {Ph.D. thesis}\ (\bibinfo  {school} {University of Cambridge}),\ \Eprint
  {http://arxiv.org/abs/arXiv:0911.3814} {arXiv:0911.3814} \BibitemShut
  {NoStop}%
\bibitem [{\citenamefont {Colbeck}\ and\ \citenamefont {Renner}(2011)}]{CR11}%
  \BibitemOpen
  \bibfield  {author} {\bibinfo {author} {\bibnamefont {Colbeck}, \bibfnamefont
  {Roger}}, \ and\ \bibinfo {author} {\bibfnamefont {Renato}\ \bibnamefont
  {Renner}}} (\bibinfo {year} {2011}),\ \bibfield  {title} {\enquote {\bibinfo
  {title} {No extension of quantum theory can have improved predictive
  power},}\ }\href {\doibase 10.1038/ncomms1416} {\bibfield  {journal}
  {\bibinfo  {journal} {Nat. Commun.}\ }\textbf {\bibinfo {volume} {2}},\
  \bibinfo {pages} {411}},\ \Eprint {http://arxiv.org/abs/arXiv:1005.5173}
  {arXiv:1005.5173} \BibitemShut {NoStop}%
\bibitem [{\citenamefont {Colbeck}\ and\ \citenamefont {Renner}(2012)}]{CR12}%
  \BibitemOpen
  \bibfield  {author} {\bibinfo {author} {\bibnamefont {Colbeck}, \bibfnamefont
  {Roger}}, \ and\ \bibinfo {author} {\bibfnamefont {Renato}\ \bibnamefont
  {Renner}}} (\bibinfo {year} {2012}),\ \bibfield  {title} {\enquote {\bibinfo
  {title} {Free randomness can be amplified},}\ }\href {\doibase
  10.1038/nphys2300} {\bibfield  {journal} {\bibinfo  {journal} {Nat. Phys.}\
  }\textbf {\bibinfo {volume} {8}}~(\bibinfo {number} {6}),\ \bibinfo {pages}
  {450--454}},\ \Eprint {http://arxiv.org/abs/arXiv:1105.3195}
  {arXiv:1105.3195} \BibitemShut {NoStop}%
\bibitem [{\citenamefont {Coles}\ \emph {et~al.}(2017)\citenamefont {Coles},
  \citenamefont {Berta}, \citenamefont {Tomamichel},\ and\ \citenamefont
  {Wehner}}]{Coles}%
  \BibitemOpen
  \bibfield  {author} {\bibinfo {author} {\bibnamefont {Coles}, \bibfnamefont
  {Patrick~J}}, \bibinfo {author} {\bibfnamefont {Mario}\ \bibnamefont
  {Berta}}, \bibinfo {author} {\bibfnamefont {Marco}\ \bibnamefont
  {Tomamichel}}, \ and\ \bibinfo {author} {\bibfnamefont {Stephanie}\
  \bibnamefont {Wehner}}} (\bibinfo {year} {2017}),\ \bibfield  {title}
  {\enquote {\bibinfo {title} {Entropic uncertainty relations and their
  applications},}\ }\href {\doibase 10.1103/RevModPhys.89.015002} {\bibfield
  {journal} {\bibinfo  {journal} {Rev. Mod. Phys.}\ }\textbf {\bibinfo {volume}
  {89}},\ \bibinfo {pages} {015002}}\BibitemShut {NoStop}%
\bibitem [{\citenamefont {Conway}\ and\ \citenamefont
  {Kochen}(2006)}]{Conway2006}%
  \BibitemOpen
  \bibfield  {author} {\bibinfo {author} {\bibnamefont {Conway}, \bibfnamefont
  {John}}, \ and\ \bibinfo {author} {\bibfnamefont {Simon}\ \bibnamefont
  {Kochen}}} (\bibinfo {year} {2006}),\ \bibfield  {title} {\enquote {\bibinfo
  {title} {The free will theorem},}\ }\href {\doibase
  10.1007/s10701-006-9068-6} {\bibfield  {journal} {\bibinfo  {journal} {Found.
  Phys.}\ }\textbf {\bibinfo {volume} {36}}~(\bibinfo {number} {10}),\ \bibinfo
  {pages} {1441--1473}}\BibitemShut {NoStop}%
\bibitem [{\citenamefont {Coretti}\ \emph {et~al.}(2013)\citenamefont
  {Coretti}, \citenamefont {Maurer},\ and\ \citenamefont {Tackmann}}]{CMT13}%
  \BibitemOpen
  \bibfield  {author} {\bibinfo {author} {\bibnamefont {Coretti}, \bibfnamefont
  {Sandro}}, \bibinfo {author} {\bibfnamefont {Ueli}\ \bibnamefont {Maurer}}, \
  and\ \bibinfo {author} {\bibfnamefont {Bj\"orn}\ \bibnamefont {Tackmann}}}
  (\bibinfo {year} {2013}),\ \bibfield  {title} {\enquote {\bibinfo {title}
  {Constructing confidential channels from authenticated channels---public-key
  encryption revisited},}\ }in\ \href {\doibase 10.1007/978-3-642-42033-7_8}
  {\emph {\bibinfo {booktitle} {Advances in Cryptology -- ASIACRYPT 2013}}},\
  \bibinfo {series} {LNCS}, Vol.\ \bibinfo {volume} {8269}\ (\bibinfo
  {publisher} {Springer})\ pp.\ \bibinfo {pages} {134--153},\ \bibinfo {note}
  {e-Print \href{http://eprint.iacr.org/2013/719}{IACR 2013/719}}\BibitemShut
  {NoStop}%
\bibitem [{\citenamefont {Cover}\ and\ \citenamefont {Thomas}(2012)}]{CT12}%
  \BibitemOpen
  \bibfield  {author} {\bibinfo {author} {\bibnamefont {Cover}, \bibfnamefont
  {Thomas~M}}, \ and\ \bibinfo {author} {\bibfnamefont {Joy~A.}\ \bibnamefont
  {Thomas}}} (\bibinfo {year} {2012}),\ \href@noop {} {\emph {\bibinfo {title}
  {Elements of information theory}}}\ (\bibinfo  {publisher} {John Wiley \&
  Sons})\BibitemShut {NoStop}%
\bibitem [{\citenamefont {Cramer}\ \emph {et~al.}(2015)\citenamefont {Cramer},
  \citenamefont {Damg{\aa}rd},\ and\ \citenamefont {Nielsen}}]{CDN15}%
  \BibitemOpen
  \bibfield  {author} {\bibinfo {author} {\bibnamefont {Cramer}, \bibfnamefont
  {Ronald}}, \bibinfo {author} {\bibfnamefont {Ivan~B.}\ \bibnamefont
  {Damg{\aa}rd}}, \ and\ \bibinfo {author} {\bibfnamefont {Jesper~B.}\
  \bibnamefont {Nielsen}}} (\bibinfo {year} {2015}),\ \href {\doibase
  10.1017/CBO9781107337756} {\emph {\bibinfo {title} {Secure Multiparty
  Computation and Secret Sharing}}}\ (\bibinfo  {publisher} {Cambridge
  University Press})\BibitemShut {NoStop}%
\bibitem [{\citenamefont {Cr\'{e}peau}\ \emph {et~al.}(2002)\citenamefont
  {Cr\'{e}peau}, \citenamefont {Gottesman},\ and\ \citenamefont
  {Smith}}]{CGS02}%
  \BibitemOpen
  \bibfield  {author} {\bibinfo {author} {\bibnamefont {Cr\'{e}peau},
  \bibfnamefont {Claude}}, \bibinfo {author} {\bibfnamefont {Daniel}\
  \bibnamefont {Gottesman}}, \ and\ \bibinfo {author} {\bibfnamefont {Adam}\
  \bibnamefont {Smith}}} (\bibinfo {year} {2002}),\ \bibfield  {title}
  {\enquote {\bibinfo {title} {Secure multi-party quantum computation},}\ }in\
  \href {\doibase 10.1145/509907.510000} {\emph {\bibinfo {booktitle}
  {Proceedings of the 34th Symposium on Theory of Computing, STOC~'02}}}\
  (\bibinfo  {publisher} {ACM})\ pp.\ \bibinfo {pages} {643--652},\ \Eprint
  {http://arxiv.org/abs/arXiv:quant-ph/0206138} {arXiv:quant-ph/0206138}
  \BibitemShut {NoStop}%
\bibitem [{\citenamefont {Cr\'e{}peau}\ and\ \citenamefont
  {Kilian}(1988)}]{CK88}%
  \BibitemOpen
  \bibfield  {author} {\bibinfo {author} {\bibnamefont {Cr\'e{}peau},
  \bibfnamefont {Claude}}, \ and\ \bibinfo {author} {\bibfnamefont {Joe}\
  \bibnamefont {Kilian}}} (\bibinfo {year} {1988}),\ \bibfield  {title}
  {\enquote {\bibinfo {title} {Achieving oblivious transfer using weakened
  security assumptions},}\ }in\ \href {\doibase 10.1109/SFCS.1988.21920} {\emph
  {\bibinfo {booktitle} {Proceedings of the 29th Symposium on Foundations of
  Computer Science, FOCS~'88}}},\ pp.\ \bibinfo {pages} {42--52}\BibitemShut
  {NoStop}%
\bibitem [{\citenamefont {Curty}\ \emph {et~al.}(2014)\citenamefont {Curty},
  \citenamefont {Xu}, \citenamefont {Cui}, \citenamefont {Lim}, \citenamefont
  {Tamaki},\ and\ \citenamefont {Lo}}]{CXCLTL14}%
  \BibitemOpen
  \bibfield  {author} {\bibinfo {author} {\bibnamefont {Curty}, \bibfnamefont
  {Marcos}}, \bibinfo {author} {\bibfnamefont {Feihu}\ \bibnamefont {Xu}},
  \bibinfo {author} {\bibfnamefont {Wei}\ \bibnamefont {Cui}}, \bibinfo
  {author} {\bibfnamefont {Charles Ci~Wen}\ \bibnamefont {Lim}}, \bibinfo
  {author} {\bibfnamefont {Kiyoshi}\ \bibnamefont {Tamaki}}, \ and\ \bibinfo
  {author} {\bibfnamefont {Hoi-Kwong}\ \bibnamefont {Lo}}} (\bibinfo {year}
  {2014}),\ \bibfield  {title} {\enquote {\bibinfo {title} {Finite-key analysis
  for measurement-device-independent quantum key distribution},}\ }\href
  {\doibase 10.1038/ncomms4732} {\bibfield  {journal} {\bibinfo  {journal}
  {Nat. Commun.}\ }\textbf {\bibinfo {volume} {5}},\ \bibinfo {pages} {3732}},\
  \Eprint {http://arxiv.org/abs/arXiv:1307.1081} {arXiv:1307.1081} \BibitemShut
  {NoStop}%
\bibitem [{\citenamefont {Damg{\aa}rd}\ \emph {et~al.}(2007)\citenamefont
  {Damg{\aa}rd}, \citenamefont {Fehr}, \citenamefont {Salvail},\ and\
  \citenamefont {Schaffner}}]{DFSS07}%
  \BibitemOpen
  \bibfield  {author} {\bibinfo {author} {\bibnamefont {Damg{\aa}rd},
  \bibfnamefont {Ivan~B}}, \bibinfo {author} {\bibfnamefont {Serge}\
  \bibnamefont {Fehr}}, \bibinfo {author} {\bibfnamefont {Louis}\ \bibnamefont
  {Salvail}}, \ and\ \bibinfo {author} {\bibfnamefont {Christian}\ \bibnamefont
  {Schaffner}}} (\bibinfo {year} {2007}),\ \bibfield  {title} {\enquote
  {\bibinfo {title} {Secure identification and {QKD} in the
  bounded-quantum-storage model},}\ }in\ \href {\doibase
  10.1007/978-3-540-74143-5_19} {\emph {\bibinfo {booktitle} {Advances in
  Cryptology -- CRYPTO 2007}}},\ \bibinfo {editor} {edited by\ \bibinfo
  {editor} {\bibfnamefont {Alfred}\ \bibnamefont {Menezes}}}\ (\bibinfo
  {publisher} {Springer})\ pp.\ \bibinfo {pages} {342--359}\BibitemShut
  {NoStop}%
\bibitem [{\citenamefont {Damg{\aa}rd}\ \emph {et~al.}(2008)\citenamefont
  {Damg{\aa}rd}, \citenamefont {Fehr}, \citenamefont {Salvail},\ and\
  \citenamefont {Schaffner}}]{DFSS08}%
  \BibitemOpen
  \bibfield  {author} {\bibinfo {author} {\bibnamefont {Damg{\aa}rd},
  \bibfnamefont {Ivan~B}}, \bibinfo {author} {\bibfnamefont {Serge}\
  \bibnamefont {Fehr}}, \bibinfo {author} {\bibfnamefont {Louis}\ \bibnamefont
  {Salvail}}, \ and\ \bibinfo {author} {\bibfnamefont {Christian}\ \bibnamefont
  {Schaffner}}} (\bibinfo {year} {2008}),\ \bibfield  {title} {\enquote
  {\bibinfo {title} {Cryptography in the bounded-quantum-storage model},}\
  }\href {\doibase 10.1137/060651343} {\bibfield  {journal} {\bibinfo
  {journal} {SIAM J. Comput.}\ }\textbf {\bibinfo {volume} {37}}~(\bibinfo
  {number} {6}),\ \bibinfo {pages} {1865--1890}},\ \bibinfo {note} {a
  preliminary version appeared at FOCS '05},\ \Eprint
  {http://arxiv.org/abs/arXiv:quant-ph/0508222} {arXiv:quant-ph/0508222}
  \BibitemShut {NoStop}%
\bibitem [{\citenamefont {De}\ \emph {et~al.}(2012)\citenamefont {De},
  \citenamefont {Portmann}, \citenamefont {Vidick},\ and\ \citenamefont
  {Renner}}]{DPVR12}%
  \BibitemOpen
  \bibfield  {author} {\bibinfo {author} {\bibnamefont {De}, \bibfnamefont
  {Anindya}}, \bibinfo {author} {\bibfnamefont {Christopher}\ \bibnamefont
  {Portmann}}, \bibinfo {author} {\bibfnamefont {Thomas}\ \bibnamefont
  {Vidick}}, \ and\ \bibinfo {author} {\bibfnamefont {Renato}\ \bibnamefont
  {Renner}}} (\bibinfo {year} {2012}),\ \bibfield  {title} {\enquote {\bibinfo
  {title} {Trevisan's extractor in the presence of quantum side information},}\
  }\href {\doibase 10.1137/100813683} {\bibfield  {journal} {\bibinfo
  {journal} {SIAM J. Comput.}\ }\textbf {\bibinfo {volume} {41}}~(\bibinfo
  {number} {4}),\ \bibinfo {pages} {915--940}},\ \Eprint
  {http://arxiv.org/abs/arXiv:0912.5514} {arXiv:0912.5514} \BibitemShut
  {NoStop}%
\bibitem [{\citenamefont {Demay}\ and\ \citenamefont {Maurer}(2013)}]{DM13}%
  \BibitemOpen
  \bibfield  {author} {\bibinfo {author} {\bibnamefont {Demay}, \bibfnamefont
  {Gregory}}, \ and\ \bibinfo {author} {\bibfnamefont {Ueli}\ \bibnamefont
  {Maurer}}} (\bibinfo {year} {2013}),\ \bibfield  {title} {\enquote {\bibinfo
  {title} {Unfair coin tossing},}\ }in\ \href {\doibase
  10.1109/ISIT.2013.6620488} {\emph {\bibinfo {booktitle} {Proceedings of the
  2013 IEEE International Symposium on Information Theory, ISIT 2013}}}\
  (\bibinfo  {publisher} {IEEE})\ pp.\ \bibinfo {pages}
  {1556--1560}\BibitemShut {NoStop}%
\bibitem [{\citenamefont {Devetak}\ and\ \citenamefont {Winter}(2005)}]{DW05}%
  \BibitemOpen
  \bibfield  {author} {\bibinfo {author} {\bibnamefont {Devetak}, \bibfnamefont
  {Igor}}, \ and\ \bibinfo {author} {\bibfnamefont {Andreas}\ \bibnamefont
  {Winter}}} (\bibinfo {year} {2005}),\ \bibfield  {title} {\enquote {\bibinfo
  {title} {Distillation of secret key and entanglement from quantum states},}\
  }\href {\doibase 10.1098/rspa.2004.1372} {\bibfield  {journal} {\bibinfo
  {journal} {Proc. R. Soc. London, Ser. A}\ }\textbf {\bibinfo {volume}
  {461}}~(\bibinfo {number} {2053}),\ \bibinfo {pages} {207--235}},\ \Eprint
  {http://arxiv.org/abs/arXiv:quant-ph/0306078} {arXiv:quant-ph/0306078}
  \BibitemShut {NoStop}%
\bibitem [{\citenamefont {Dickinson}\ and\ \citenamefont {Nayak}(2006)}]{DN06}%
  \BibitemOpen
  \bibfield  {author} {\bibinfo {author} {\bibnamefont {Dickinson},
  \bibfnamefont {Paul}}, \ and\ \bibinfo {author} {\bibfnamefont {Ashwin}\
  \bibnamefont {Nayak}}} (\bibinfo {year} {2006}),\ \bibfield  {title}
  {\enquote {\bibinfo {title} {Approximate randomization of quantum states with
  fewer bits of key},}\ }in\ \href {\doibase 10.1063/1.2400876} {\emph
  {\bibinfo {booktitle} {AIP Conference Proceedings}}},\ Vol.\ \bibinfo
  {volume} {864},\ pp.\ \bibinfo {pages} {18--36},\ \Eprint
  {http://arxiv.org/abs/arXiv:quant-ph/0611033} {arXiv:quant-ph/0611033}
  \BibitemShut {NoStop}%
\bibitem [{\citenamefont {DiVincenzo}\ \emph {et~al.}(2004)\citenamefont
  {DiVincenzo}, \citenamefont {Horodecki}, \citenamefont {Leung}, \citenamefont
  {Smolin},\ and\ \citenamefont {Terhal}}]{DHLST04}%
  \BibitemOpen
  \bibfield  {author} {\bibinfo {author} {\bibnamefont {DiVincenzo},
  \bibfnamefont {David}}, \bibinfo {author} {\bibfnamefont {Micha\l{}}\
  \bibnamefont {Horodecki}}, \bibinfo {author} {\bibfnamefont {Debbie}\
  \bibnamefont {Leung}}, \bibinfo {author} {\bibfnamefont {John}\ \bibnamefont
  {Smolin}}, \ and\ \bibinfo {author} {\bibfnamefont {Barbara}\ \bibnamefont
  {Terhal}}} (\bibinfo {year} {2004}),\ \bibfield  {title} {\enquote {\bibinfo
  {title} {Locking classical correlation in quantum states},}\ }\href@noop {}
  {\bibfield  {journal} {\bibinfo  {journal} {Phys. Rev. Lett.}\ }\textbf
  {\bibinfo {volume} {92}},\ \bibinfo {pages} {067902}},\ \Eprint
  {http://arxiv.org/abs/arXiv:quant-ph/0303088} {arXiv:quant-ph/0303088}
  \BibitemShut {NoStop}%
\bibitem [{\citenamefont {Dodis}\ and\ \citenamefont {Wichs}(2009)}]{DW09}%
  \BibitemOpen
  \bibfield  {author} {\bibinfo {author} {\bibnamefont {Dodis}, \bibfnamefont
  {Yevgeniy}}, \ and\ \bibinfo {author} {\bibfnamefont {Daniel}\ \bibnamefont
  {Wichs}}} (\bibinfo {year} {2009}),\ \bibfield  {title} {\enquote {\bibinfo
  {title} {Non-malleable extractors and symmetric key cryptography from weak
  secrets},}\ }in\ \href {\doibase 10.1145/1536414.1536496} {\emph {\bibinfo
  {booktitle} {Proceedings of the 41st Symposium on Theory of Computing,
  STOC~'09}}}\ (\bibinfo  {publisher} {ACM})\ pp.\ \bibinfo {pages}
  {601--610},\ \bibinfo {note} {e-Print
  \href{http://eprint.iacr.org/2008/503}{IACR 2008/503}}\BibitemShut {NoStop}%
\bibitem [{\citenamefont {Dulek}\ \emph {et~al.}(2020)\citenamefont {Dulek},
  \citenamefont {Grilo}, \citenamefont {Jeffery}, \citenamefont {Majenz},\ and\
  \citenamefont {Schaffner}}]{DGJMS20}%
  \BibitemOpen
  \bibfield  {author} {\bibinfo {author} {\bibnamefont {Dulek}, \bibfnamefont
  {Yfke}}, \bibinfo {author} {\bibfnamefont {Alex~B.}\ \bibnamefont {Grilo}},
  \bibinfo {author} {\bibfnamefont {Stacey}\ \bibnamefont {Jeffery}}, \bibinfo
  {author} {\bibfnamefont {Christian}\ \bibnamefont {Majenz}}, \ and\ \bibinfo
  {author} {\bibfnamefont {Christian}\ \bibnamefont {Schaffner}}} (\bibinfo
  {year} {2020}),\ \bibfield  {title} {\enquote {\bibinfo {title} {Secure
  multi-party quantum computation with a dishonest majority},}\ }in\ \href
  {\doibase 10.1007/978-3-030-45727-3_25} {\emph {\bibinfo {booktitle}
  {Advances in Cryptology -- EUROCRYPT 2020}}},\ \bibinfo {editor} {edited by\
  \bibinfo {editor} {\bibfnamefont {Anne}\ \bibnamefont {Canteaut}}\ and\
  \bibinfo {editor} {\bibfnamefont {Yuval}\ \bibnamefont {Ishai}}}\ (\bibinfo
  {publisher} {Springer})\ pp.\ \bibinfo {pages} {729--758},\ \Eprint
  {http://arxiv.org/abs/arXiv:1909.13770} {arXiv:1909.13770} \BibitemShut
  {NoStop}%
\bibitem [{\citenamefont {Dunjko}\ \emph {et~al.}(2014)\citenamefont {Dunjko},
  \citenamefont {Fitzsimons}, \citenamefont {Portmann},\ and\ \citenamefont
  {Renner}}]{DFPR14}%
  \BibitemOpen
  \bibfield  {author} {\bibinfo {author} {\bibnamefont {Dunjko}, \bibfnamefont
  {Vedran}}, \bibinfo {author} {\bibfnamefont {Joseph}\ \bibnamefont
  {Fitzsimons}}, \bibinfo {author} {\bibfnamefont {Christopher}\ \bibnamefont
  {Portmann}}, \ and\ \bibinfo {author} {\bibfnamefont {Renato}\ \bibnamefont
  {Renner}}} (\bibinfo {year} {2014}),\ \bibfield  {title} {\enquote {\bibinfo
  {title} {Composable security of delegated quantum computation},}\ }in\ \href
  {\doibase 10.1007/978-3-662-45608-8_22} {\emph {\bibinfo {booktitle}
  {Advances in Cryptology -- ASIACRYPT 2014, Proceedings, Part II}}},\ \bibinfo
  {series} {LNCS}, Vol.\ \bibinfo {volume} {8874}\ (\bibinfo  {publisher}
  {Springer})\ pp.\ \bibinfo {pages} {406--425},\ \Eprint
  {http://arxiv.org/abs/arXiv:1301.3662} {arXiv:1301.3662} \BibitemShut
  {NoStop}%
\bibitem [{\citenamefont {Dunjko}\ and\ \citenamefont {Kashefi}(2016)}]{DK16}%
  \BibitemOpen
  \bibfield  {author} {\bibinfo {author} {\bibnamefont {Dunjko}, \bibfnamefont
  {Vedran}}, \ and\ \bibinfo {author} {\bibfnamefont {Elham}\ \bibnamefont
  {Kashefi}}} (\bibinfo {year} {2016}),\ \href@noop {} {\enquote {\bibinfo
  {title} {Blind quantum computing with two almost identical states},}\
  }\bibinfo {howpublished} {e-Print},\ \Eprint
  {http://arxiv.org/abs/arXiv:1604.01586} {arXiv:1604.01586} \BibitemShut
  {NoStop}%
\bibitem [{\citenamefont {Dupuis}\ \emph {et~al.}(2020)\citenamefont {Dupuis},
  \citenamefont {Fawzi},\ and\ \citenamefont {Renner}}]{DFR20}%
  \BibitemOpen
  \bibfield  {author} {\bibinfo {author} {\bibnamefont {Dupuis}, \bibfnamefont
  {Fr{\'e}d{\'e}ric}}, \bibinfo {author} {\bibfnamefont {Omar}\ \bibnamefont
  {Fawzi}}, \ and\ \bibinfo {author} {\bibfnamefont {Renato}\ \bibnamefont
  {Renner}}} (\bibinfo {year} {2020}),\ \bibfield  {title} {\enquote {\bibinfo
  {title} {Entropy accumulation},}\ }\href {\doibase
  10.1007/s00220-020-03839-5} {\bibfield  {journal} {\bibinfo  {journal}
  {Commun. Math. Phys.}\ }\textbf {\bibinfo {volume} {379}}~(\bibinfo {number}
  {3}),\ \bibinfo {pages} {867--913}},\ \Eprint
  {http://arxiv.org/abs/arXiv:1607.01796} {arXiv:1607.01796} \BibitemShut
  {NoStop}%
\bibitem [{\citenamefont {Dupuis}\ \emph {et~al.}(2012)\citenamefont {Dupuis},
  \citenamefont {Nielsen},\ and\ \citenamefont {Salvail}}]{DNS12}%
  \BibitemOpen
  \bibfield  {author} {\bibinfo {author} {\bibnamefont {Dupuis}, \bibfnamefont
  {Fr{\'e}d{\'e}ric}}, \bibinfo {author} {\bibfnamefont {Jesper~B.}\
  \bibnamefont {Nielsen}}, \ and\ \bibinfo {author} {\bibfnamefont {Louis}\
  \bibnamefont {Salvail}}} (\bibinfo {year} {2012}),\ \bibfield  {title}
  {\enquote {\bibinfo {title} {Actively secure two-party evaluation of any
  quantum operation},}\ }in\ \href {\doibase 10.1007/978-3-642-32009-5_46}
  {\emph {\bibinfo {booktitle} {Advances in Cryptology -- CRYPTO 2012}}},\
  \bibinfo {series} {LNCS}, Vol.\ \bibinfo {volume} {7417},\ \bibinfo {editor}
  {edited by\ \bibinfo {editor} {\bibfnamefont {Reihaneh}\ \bibnamefont
  {Safavi-Naini}}\ and\ \bibinfo {editor} {\bibfnamefont {Ran}\ \bibnamefont
  {Canetti}}}\ (\bibinfo  {publisher} {Springer})\ pp.\ \bibinfo {pages}
  {794--811},\ \bibinfo {note} {e-Print
  \href{http://eprint.iacr.org/2012/304}{IACR 2012/304}}\BibitemShut {NoStop}%
\bibitem [{\citenamefont {Einstein}\ \emph {et~al.}(1935)\citenamefont
  {Einstein}, \citenamefont {Podolsky},\ and\ \citenamefont {Rosen}}]{EPR35}%
  \BibitemOpen
  \bibfield  {author} {\bibinfo {author} {\bibnamefont {Einstein},
  \bibfnamefont {Albert}}, \bibinfo {author} {\bibfnamefont {Boris}\
  \bibnamefont {Podolsky}}, \ and\ \bibinfo {author} {\bibfnamefont {Nathan}\
  \bibnamefont {Rosen}}} (\bibinfo {year} {1935}),\ \bibfield  {title}
  {\enquote {\bibinfo {title} {Can quantum-mechanical description of physical
  reality be considered complete?}}\ }\href {\doibase 10.1103/PhysRev.47.777}
  {\bibfield  {journal} {\bibinfo  {journal} {Phys. Rev.}\ }\textbf {\bibinfo
  {volume} {47}},\ \bibinfo {pages} {777--780}}\BibitemShut {NoStop}%
\bibitem [{\citenamefont {Ekert}(1991)}]{Eke91}%
  \BibitemOpen
  \bibfield  {author} {\bibinfo {author} {\bibnamefont {Ekert}, \bibfnamefont
  {Artur}}} (\bibinfo {year} {1991}),\ \bibfield  {title} {\enquote {\bibinfo
  {title} {Quantum cryptography based on {Bell}'s theorem},}\ }\href {\doibase
  10.1103/PhysRevLett.67.661} {\bibfield  {journal} {\bibinfo  {journal} {Phys.
  Rev. Lett.}\ }\textbf {\bibinfo {volume} {67}},\ \bibinfo {pages}
  {661--663}}\BibitemShut {NoStop}%
\bibitem [{\citenamefont {Ekert}\ and\ \citenamefont {Renner}(2014)}]{ER14}%
  \BibitemOpen
  \bibfield  {author} {\bibinfo {author} {\bibnamefont {Ekert}, \bibfnamefont
  {Artur}}, \ and\ \bibinfo {author} {\bibfnamefont {Renato}\ \bibnamefont
  {Renner}}} (\bibinfo {year} {2014}),\ \bibfield  {title} {\enquote {\bibinfo
  {title} {The ultimate physical limits of privacy},}\ }\href {\doibase
  10.1038/nature13132} {\bibfield  {journal} {\bibinfo  {journal} {Nature}\
  }\textbf {\bibinfo {volume} {507}}~(\bibinfo {number} {7493}),\ \bibinfo
  {pages} {443--447}},\ \bibinfo {note} {perspectives}\BibitemShut {NoStop}%
\bibitem [{\citenamefont {Elkouss}\ \emph {et~al.}(2009)\citenamefont
  {Elkouss}, \citenamefont {Leverrier}, \citenamefont {Alleaume},\ and\
  \citenamefont {Boutros}}]{ELAB09}%
  \BibitemOpen
  \bibfield  {author} {\bibinfo {author} {\bibnamefont {Elkouss}, \bibfnamefont
  {David}}, \bibinfo {author} {\bibfnamefont {Anthony}\ \bibnamefont
  {Leverrier}}, \bibinfo {author} {\bibfnamefont {Romain}\ \bibnamefont
  {Alleaume}}, \ and\ \bibinfo {author} {\bibfnamefont {Joseph~J.}\
  \bibnamefont {Boutros}}} (\bibinfo {year} {2009}),\ \bibfield  {title}
  {\enquote {\bibinfo {title} {Efficient reconciliation protocol for
  discrete-variable quantum key distribution},}\ }in\ \href {\doibase
  10.1109/ISIT.2009.5205475} {\emph {\bibinfo {booktitle} {Proceedings of the
  2009 IEEE International Symposium on Information Theory, ISIT 2009}}}\
  (\bibinfo  {publisher} {IEEE})\ pp.\ \bibinfo {pages}
  {1879--1883}\BibitemShut {NoStop}%
\bibitem [{\citenamefont {Elkouss}\ \emph {et~al.}(2011)\citenamefont
  {Elkouss}, \citenamefont {Martinez-Mateo},\ and\ \citenamefont
  {Martin}}]{EMM11}%
  \BibitemOpen
  \bibfield  {author} {\bibinfo {author} {\bibnamefont {Elkouss}, \bibfnamefont
  {David}}, \bibinfo {author} {\bibfnamefont {Jesus}\ \bibnamefont
  {Martinez-Mateo}}, \ and\ \bibinfo {author} {\bibfnamefont {Vicente}\
  \bibnamefont {Martin}}} (\bibinfo {year} {2011}),\ \bibfield  {title}
  {\enquote {\bibinfo {title} {Information reconciliation for quantum key
  distribution},}\ }\href@noop {} {\bibfield  {journal} {\bibinfo  {journal}
  {Quantum Inf. Comput.}\ }\textbf {\bibinfo {volume} {11}}~(\bibinfo {number}
  {3}),\ \bibinfo {pages} {226--238}}\BibitemShut {NoStop}%
\bibitem [{\citenamefont {Fehr}\ and\ \citenamefont {Schaffner}(2008)}]{FS08}%
  \BibitemOpen
  \bibfield  {author} {\bibinfo {author} {\bibnamefont {Fehr}, \bibfnamefont
  {Serge}}, \ and\ \bibinfo {author} {\bibfnamefont {Christian}\ \bibnamefont
  {Schaffner}}} (\bibinfo {year} {2008}),\ \bibfield  {title} {\enquote
  {\bibinfo {title} {Randomness extraction via $\delta$-biased masking in the
  presence of a quantum attacker},}\ }in\ \href {\doibase
  10.1007/978-3-540-78524-8_26} {\emph {\bibinfo {booktitle} {Theory of
  Cryptography, Proceedings of TCC 2008}}},\ \bibinfo {series} {LNCS}, Vol.\
  \bibinfo {volume} {4948}\ (\bibinfo  {publisher} {Springer})\ pp.\ \bibinfo
  {pages} {465--481},\ \Eprint {http://arxiv.org/abs/arXiv:0706.2606}
  {arXiv:0706.2606} \BibitemShut {NoStop}%
\bibitem [{\citenamefont {Fitzsimons}\ and\ \citenamefont
  {Kashefi}(2017)}]{FK17}%
  \BibitemOpen
  \bibfield  {author} {\bibinfo {author} {\bibnamefont {Fitzsimons},
  \bibfnamefont {Joseph~F}}, \ and\ \bibinfo {author} {\bibfnamefont {Elham}\
  \bibnamefont {Kashefi}}} (\bibinfo {year} {2017}),\ \bibfield  {title}
  {\enquote {\bibinfo {title} {Unconditionally verifiable blind computation},}\
  }\href {\doibase 10.1103/PhysRevA.96.012303} {\bibfield  {journal} {\bibinfo
  {journal} {Phys. Rev. A}\ }\textbf {\bibinfo {volume} {96}},\ \bibinfo
  {pages} {012303}},\ \Eprint {http://arxiv.org/abs/arXiv:1203.5217}
  {arXiv:1203.5217} \BibitemShut {NoStop}%
\bibitem [{\citenamefont {Freedman}\ and\ \citenamefont
  {Clauser}(1972)}]{FreedmanClauser}%
  \BibitemOpen
  \bibfield  {author} {\bibinfo {author} {\bibnamefont {Freedman},
  \bibfnamefont {Stuart~J}}, \ and\ \bibinfo {author} {\bibfnamefont {John~F.}\
  \bibnamefont {Clauser}}} (\bibinfo {year} {1972}),\ \bibfield  {title}
  {\enquote {\bibinfo {title} {Experimental test of local hidden-variable
  theories},}\ }\href {\doibase 10.1103/PhysRevLett.28.938} {\bibfield
  {journal} {\bibinfo  {journal} {Phys. Rev. Lett.}\ }\textbf {\bibinfo
  {volume} {28}},\ \bibinfo {pages} {938--941}}\BibitemShut {NoStop}%
\bibitem [{\citenamefont {Fuchs}(1998)}]{Fuchs98}%
  \BibitemOpen
  \bibfield  {author} {\bibinfo {author} {\bibnamefont {Fuchs}, \bibfnamefont
  {Christopher~A}}} (\bibinfo {year} {1998}),\ \bibfield  {title} {\enquote
  {\bibinfo {title} {Information gain vs.\ state disturbance in quantum
  theory},}\ }\href@noop {} {\bibfield  {journal} {\bibinfo  {journal}
  {Fortschritte der Physik: Progress of Physics}\ }\textbf {\bibinfo {volume}
  {46}}~(\bibinfo {number} {4-5}),\ \bibinfo {pages} {535--565}},\ \Eprint
  {http://arxiv.org/abs/arXiv:quant-ph/9611010} {arXiv:quant-ph/9611010}
  \BibitemShut {NoStop}%
\bibitem [{\citenamefont {Fuchs}\ \emph {et~al.}(1997)\citenamefont {Fuchs},
  \citenamefont {Gisin}, \citenamefont {Griffiths}, \citenamefont {Niu},\ and\
  \citenamefont {Peres}}]{Fuchsetal1997}%
  \BibitemOpen
  \bibfield  {author} {\bibinfo {author} {\bibnamefont {Fuchs}, \bibfnamefont
  {Christopher~A}}, \bibinfo {author} {\bibfnamefont {Nicolas}\ \bibnamefont
  {Gisin}}, \bibinfo {author} {\bibfnamefont {Robert~B.}\ \bibnamefont
  {Griffiths}}, \bibinfo {author} {\bibfnamefont {Chi-Sheng}\ \bibnamefont
  {Niu}}, \ and\ \bibinfo {author} {\bibfnamefont {Asher}\ \bibnamefont
  {Peres}}} (\bibinfo {year} {1997}),\ \bibfield  {title} {\enquote {\bibinfo
  {title} {Optimal eavesdropping in quantum cryptography. i. information bound
  and optimal strategy},}\ }\href {\doibase 10.1103/PhysRevA.56.1163}
  {\bibfield  {journal} {\bibinfo  {journal} {Phys. Rev. A}\ }\textbf {\bibinfo
  {volume} {56}},\ \bibinfo {pages} {1163--1172}}\BibitemShut {NoStop}%
\bibitem [{\citenamefont {Fuchs}\ and\ \citenamefont {Van
  De~Graaf}(1999)}]{FuchsvanGraaf}%
  \BibitemOpen
  \bibfield  {author} {\bibinfo {author} {\bibnamefont {Fuchs}, \bibfnamefont
  {Christopher~A}}, \ and\ \bibinfo {author} {\bibfnamefont {Jeroen}\
  \bibnamefont {Van De~Graaf}}} (\bibinfo {year} {1999}),\ \bibfield  {title}
  {\enquote {\bibinfo {title} {Cryptographic distinguishability measures for
  quantum-mechanical states},}\ }\href@noop {} {\bibfield  {journal} {\bibinfo
  {journal} {IEEE Trans. Inf. Theory}\ }\textbf {\bibinfo {volume}
  {45}}~(\bibinfo {number} {4}),\ \bibinfo {pages} {1216--1227}}\BibitemShut
  {NoStop}%
\bibitem [{\citenamefont {Fung}\ \emph {et~al.}(2007)\citenamefont {Fung},
  \citenamefont {Qi}, \citenamefont {Tamaki},\ and\ \citenamefont
  {Lo}}]{FQTL07}%
  \BibitemOpen
  \bibfield  {author} {\bibinfo {author} {\bibnamefont {Fung}, \bibfnamefont
  {Chi-Hang~Fred}}, \bibinfo {author} {\bibfnamefont {Bing}\ \bibnamefont
  {Qi}}, \bibinfo {author} {\bibfnamefont {Kiyoshi}\ \bibnamefont {Tamaki}}, \
  and\ \bibinfo {author} {\bibfnamefont {Hoi-Kwong}\ \bibnamefont {Lo}}}
  (\bibinfo {year} {2007}),\ \bibfield  {title} {\enquote {\bibinfo {title}
  {Phase-remapping attack in practical quantum-key-distribution systems},}\
  }\href {\doibase 10.1103/PhysRevA.75.032314} {\bibfield  {journal} {\bibinfo
  {journal} {Phys. Rev. A}\ }\textbf {\bibinfo {volume} {75}}~(\bibinfo
  {number} {3}),\ \bibinfo {pages} {032314}},\ \Eprint
  {http://arxiv.org/abs/arXiv:quant-ph/0601115} {arXiv:quant-ph/0601115}
  \BibitemShut {NoStop}%
\bibitem [{\citenamefont {Garg}\ \emph {et~al.}(2017)\citenamefont {Garg},
  \citenamefont {Yuen},\ and\ \citenamefont {Zhandry}}]{GYZ17}%
  \BibitemOpen
  \bibfield  {author} {\bibinfo {author} {\bibnamefont {Garg}, \bibfnamefont
  {Sumegha}}, \bibinfo {author} {\bibfnamefont {Henry}\ \bibnamefont {Yuen}}, \
  and\ \bibinfo {author} {\bibfnamefont {Mark}\ \bibnamefont {Zhandry}}}
  (\bibinfo {year} {2017}),\ \bibfield  {title} {\enquote {\bibinfo {title}
  {New security notions and feasibility results for authentication of quantum
  data},}\ }in\ \href {\doibase 10.1007/978-3-319-63715-0_12} {\emph {\bibinfo
  {booktitle} {Advances in Cryptology -- CRYPTO 2017}}},\ \bibinfo {series}
  {LNCS}, Vol.\ \bibinfo {volume} {10402},\ \bibinfo {editor} {edited by\
  \bibinfo {editor} {\bibfnamefont {Jonathan}\ \bibnamefont {Katz}}\ and\
  \bibinfo {editor} {\bibfnamefont {Hovav}\ \bibnamefont {Shacham}}}\ (\bibinfo
   {publisher} {Springer})\ pp.\ \bibinfo {pages} {342--371},\ \Eprint
  {http://arxiv.org/abs/arXiv:1607.07759} {arXiv:1607.07759} \BibitemShut
  {NoStop}%
\bibitem [{\citenamefont {Gavinsky}\ \emph {et~al.}(2007)\citenamefont
  {Gavinsky}, \citenamefont {Kempe}, \citenamefont {Kerenidis}, \citenamefont
  {Raz},\ and\ \citenamefont {de~Wolf}}]{GKKRD07}%
  \BibitemOpen
  \bibfield  {author} {\bibinfo {author} {\bibnamefont {Gavinsky},
  \bibfnamefont {Dmitry}}, \bibinfo {author} {\bibfnamefont {Julia}\
  \bibnamefont {Kempe}}, \bibinfo {author} {\bibfnamefont {Iordanis}\
  \bibnamefont {Kerenidis}}, \bibinfo {author} {\bibfnamefont {Ran}\
  \bibnamefont {Raz}}, \ and\ \bibinfo {author} {\bibfnamefont {Ronald}\
  \bibnamefont {de~Wolf}}} (\bibinfo {year} {2007}),\ \bibfield  {title}
  {\enquote {\bibinfo {title} {Exponential separations for one-way quantum
  communication complexity, with applications to cryptography},}\ }in\ \href
  {\doibase 10.1145/1250790.1250866} {\emph {\bibinfo {booktitle} {Proceedings
  of the 39th Symposium on Theory of Computing, STOC~'07}}}\ (\bibinfo
  {publisher} {ACM})\ pp.\ \bibinfo {pages} {516--525},\ \Eprint
  {http://arxiv.org/abs/arXiv:quant-ph/0611209} {arXiv:quant-ph/0611209}
  \BibitemShut {NoStop}%
\bibitem [{\citenamefont {Gerhardt}\ \emph {et~al.}(2011)\citenamefont
  {Gerhardt}, \citenamefont {Liu}, \citenamefont {Lamas-Linares}, \citenamefont
  {Skaar}, \citenamefont {Kurtsiefer},\ and\ \citenamefont
  {Makarov}}]{GLLSKM11}%
  \BibitemOpen
  \bibfield  {author} {\bibinfo {author} {\bibnamefont {Gerhardt},
  \bibfnamefont {Ilja}}, \bibinfo {author} {\bibfnamefont {Qin}\ \bibnamefont
  {Liu}}, \bibinfo {author} {\bibfnamefont {Ant{\'\i}a}\ \bibnamefont
  {Lamas-Linares}}, \bibinfo {author} {\bibfnamefont {Johannes}\ \bibnamefont
  {Skaar}}, \bibinfo {author} {\bibfnamefont {Christian}\ \bibnamefont
  {Kurtsiefer}}, \ and\ \bibinfo {author} {\bibfnamefont {Vadim}\ \bibnamefont
  {Makarov}}} (\bibinfo {year} {2011}),\ \bibfield  {title} {\enquote {\bibinfo
  {title} {Full-field implementation of a perfect eavesdropper on a quantum
  cryptography system},}\ }\href {\doibase 10.1038/ncomms1348} {\bibfield
  {journal} {\bibinfo  {journal} {Nat. Commun.}\ }\textbf {\bibinfo {volume}
  {2}},\ \bibinfo {pages} {349}},\ \Eprint
  {http://arxiv.org/abs/arXiv:1011.0105} {arXiv:1011.0105} \BibitemShut
  {NoStop}%
\bibitem [{\citenamefont {Gheorghiu}\ and\ \citenamefont
  {Vidick}(2019)}]{GV19}%
  \BibitemOpen
  \bibfield  {author} {\bibinfo {author} {\bibnamefont {Gheorghiu},
  \bibfnamefont {Alexandru}}, \ and\ \bibinfo {author} {\bibfnamefont {Thomas}\
  \bibnamefont {Vidick}}} (\bibinfo {year} {2019}),\ \bibfield  {title}
  {\enquote {\bibinfo {title} {Computationally-secure and composable remote
  state preparation},}\ }in\ \href {\doibase 10.1109/FOCS.2019.00066} {\emph
  {\bibinfo {booktitle} {Proceedings of the 60th Symposium on Foundations of
  Computer Science, FOCS~'19}}},\ pp.\ \bibinfo {pages} {1024--1033},\ \Eprint
  {http://arxiv.org/abs/arXiv:1904.06320} {arXiv:1904.06320} \BibitemShut
  {NoStop}%
\bibitem [{\citenamefont {Gisin}\ \emph {et~al.}(2006)\citenamefont {Gisin},
  \citenamefont {Fasel}, \citenamefont {Kraus}, \citenamefont {Zbinden},\ and\
  \citenamefont {Ribordy}}]{GisinFaselKraus2006}%
  \BibitemOpen
  \bibfield  {author} {\bibinfo {author} {\bibnamefont {Gisin}, \bibfnamefont
  {Nicolas}}, \bibinfo {author} {\bibfnamefont {Sylvain}\ \bibnamefont
  {Fasel}}, \bibinfo {author} {\bibfnamefont {Barbara}\ \bibnamefont {Kraus}},
  \bibinfo {author} {\bibfnamefont {Hugo}\ \bibnamefont {Zbinden}}, \ and\
  \bibinfo {author} {\bibfnamefont {Gr\'egoire}\ \bibnamefont {Ribordy}}}
  (\bibinfo {year} {2006}),\ \bibfield  {title} {\enquote {\bibinfo {title}
  {Trojan-horse attacks on quantum-key-distribution systems},}\ }\href
  {\doibase 10.1103/PhysRevA.73.022320} {\bibfield  {journal} {\bibinfo
  {journal} {Phys. Rev. A}\ }\textbf {\bibinfo {volume} {73}},\ \bibinfo
  {pages} {022320}},\ \Eprint {http://arxiv.org/abs/arXiv:quant-ph/0507063}
  {arXiv:quant-ph/0507063} \BibitemShut {NoStop}%
\bibitem [{\citenamefont {Giustina}\ \emph {et~al.}(2013)\citenamefont
  {Giustina}, \citenamefont {Mech}, \citenamefont {Ramelow}, \citenamefont
  {Wittmann}, \citenamefont {Kofler}, \citenamefont {Beyer}, \citenamefont
  {Lita}, \citenamefont {Calkins}, \citenamefont {Gerrits}, \citenamefont
  {Nam}, \citenamefont {Ursin},\ and\ \citenamefont {Zeilinger}}]{Giustina13}%
  \BibitemOpen
  \bibfield  {author} {\bibinfo {author} {\bibnamefont {Giustina},
  \bibfnamefont {Marissa}}, \bibinfo {author} {\bibfnamefont {Alexandra}\
  \bibnamefont {Mech}}, \bibinfo {author} {\bibfnamefont {Sven}\ \bibnamefont
  {Ramelow}}, \bibinfo {author} {\bibfnamefont {Bernhard}\ \bibnamefont
  {Wittmann}}, \bibinfo {author} {\bibfnamefont {Johannes}\ \bibnamefont
  {Kofler}}, \bibinfo {author} {\bibfnamefont {J{\"o}rn}\ \bibnamefont
  {Beyer}}, \bibinfo {author} {\bibfnamefont {Adriana}\ \bibnamefont {Lita}},
  \bibinfo {author} {\bibfnamefont {Brice}\ \bibnamefont {Calkins}}, \bibinfo
  {author} {\bibfnamefont {Thomas}\ \bibnamefont {Gerrits}}, \bibinfo {author}
  {\bibfnamefont {Sae~Woo}\ \bibnamefont {Nam}}, \bibinfo {author}
  {\bibfnamefont {Rupert}\ \bibnamefont {Ursin}}, \ and\ \bibinfo {author}
  {\bibfnamefont {Anton}\ \bibnamefont {Zeilinger}}} (\bibinfo {year} {2013}),\
  \bibfield  {title} {\enquote {\bibinfo {title} {Bell violation using
  entangled photons without the fair-sampling assumption},}\ }\href {\doibase
  10.1038/nature12012} {\bibfield  {journal} {\bibinfo  {journal} {Nature}\
  }\textbf {\bibinfo {volume} {497}}~(\bibinfo {number} {7448}),\ \bibinfo
  {pages} {227--230}}\BibitemShut {NoStop}%
\bibitem [{\citenamefont {Giustina}\ \emph {et~al.}(2015)\citenamefont
  {Giustina}, \citenamefont {Versteegh}, \citenamefont {Wengerowsky},
  \citenamefont {Handsteiner}, \citenamefont {Hochrainer}, \citenamefont
  {Phelan}, \citenamefont {Steinlechner}, \citenamefont {Kofler}, \citenamefont
  {Larsson}, \citenamefont {Abell\'an}, \citenamefont {Amaya}, \citenamefont
  {Pruneri}, \citenamefont {Mitchell}, \citenamefont {Beyer}, \citenamefont
  {Gerrits}, \citenamefont {Lita}, \citenamefont {Shalm}, \citenamefont {Nam},
  \citenamefont {Scheidl}, \citenamefont {Ursin}, \citenamefont {Wittmann},\
  and\ \citenamefont {Zeilinger}}]{Giustina15}%
  \BibitemOpen
  \bibfield  {author} {\bibinfo {author} {\bibnamefont {Giustina},
  \bibfnamefont {Marissa}}, \bibinfo {author} {\bibfnamefont {Marijn A.~M.}\
  \bibnamefont {Versteegh}}, \bibinfo {author} {\bibfnamefont {S\"oren}\
  \bibnamefont {Wengerowsky}}, \bibinfo {author} {\bibfnamefont {Johannes}\
  \bibnamefont {Handsteiner}}, \bibinfo {author} {\bibfnamefont {Armin}\
  \bibnamefont {Hochrainer}}, \bibinfo {author} {\bibfnamefont {Kevin}\
  \bibnamefont {Phelan}}, \bibinfo {author} {\bibfnamefont {Fabian}\
  \bibnamefont {Steinlechner}}, \bibinfo {author} {\bibfnamefont {Johannes}\
  \bibnamefont {Kofler}}, \bibinfo {author} {\bibfnamefont {Jan-\AA{}ke}\
  \bibnamefont {Larsson}}, \bibinfo {author} {\bibfnamefont {Carlos}\
  \bibnamefont {Abell\'an}}, \bibinfo {author} {\bibfnamefont {Waldimar}\
  \bibnamefont {Amaya}}, \bibinfo {author} {\bibfnamefont {Valerio}\
  \bibnamefont {Pruneri}}, \bibinfo {author} {\bibfnamefont {Morgan~W.}\
  \bibnamefont {Mitchell}}, \bibinfo {author} {\bibfnamefont {J\"orn}\
  \bibnamefont {Beyer}}, \bibinfo {author} {\bibfnamefont {Thomas}\
  \bibnamefont {Gerrits}}, \bibinfo {author} {\bibfnamefont {Adriana~E.}\
  \bibnamefont {Lita}}, \bibinfo {author} {\bibfnamefont {Lynden~K.}\
  \bibnamefont {Shalm}}, \bibinfo {author} {\bibfnamefont {Sae~Woo}\
  \bibnamefont {Nam}}, \bibinfo {author} {\bibfnamefont {Thomas}\ \bibnamefont
  {Scheidl}}, \bibinfo {author} {\bibfnamefont {Rupert}\ \bibnamefont {Ursin}},
  \bibinfo {author} {\bibfnamefont {Bernhard}\ \bibnamefont {Wittmann}}, \ and\
  \bibinfo {author} {\bibfnamefont {Anton}\ \bibnamefont {Zeilinger}}}
  (\bibinfo {year} {2015}),\ \bibfield  {title} {\enquote {\bibinfo {title}
  {Significant-loophole-free test of {Bell}'s theorem with entangled
  photons},}\ }\href {\doibase 10.1103/PhysRevLett.115.250401} {\bibfield
  {journal} {\bibinfo  {journal} {Phys. Rev. Lett.}\ }\textbf {\bibinfo
  {volume} {115}},\ \bibinfo {pages} {250401}}\BibitemShut {NoStop}%
\bibitem [{\citenamefont {Goldreich}(2004)}]{Gol04}%
  \BibitemOpen
  \bibfield  {author} {\bibinfo {author} {\bibnamefont {Goldreich},
  \bibfnamefont {Oded}}} (\bibinfo {year} {2004}),\ \href@noop {} {\emph
  {\bibinfo {title} {Foundations of Cryptography: Volume 2, Basic
  Applications}}}\ (\bibinfo  {publisher} {Cambridge University Press},\
  \bibinfo {address} {New York, NY, USA})\BibitemShut {NoStop}%
\bibitem [{\citenamefont {Goldreich}\ \emph {et~al.}(1987)\citenamefont
  {Goldreich}, \citenamefont {Micali},\ and\ \citenamefont
  {Wigderson}}]{GMW87}%
  \BibitemOpen
  \bibfield  {author} {\bibinfo {author} {\bibnamefont {Goldreich},
  \bibfnamefont {Oded}}, \bibinfo {author} {\bibfnamefont {Silvia}\
  \bibnamefont {Micali}}, \ and\ \bibinfo {author} {\bibfnamefont {Avi}\
  \bibnamefont {Wigderson}}} (\bibinfo {year} {1987}),\ \bibfield  {title}
  {\enquote {\bibinfo {title} {How to play any mental game},}\ }in\ \href
  {\doibase 10.1145/28395.28420} {\emph {\bibinfo {booktitle} {Proceedings of
  the 19th Symposium on Theory of Computing, STOC~'87}}}\ (\bibinfo
  {publisher} {ACM})\ pp.\ \bibinfo {pages} {218--–229}\BibitemShut {NoStop}%
\bibitem [{\citenamefont {Goldreich}\ \emph {et~al.}(1986)\citenamefont
  {Goldreich}, \citenamefont {Micali},\ and\ \citenamefont
  {Wigderson}}]{GMW86}%
  \BibitemOpen
  \bibfield  {author} {\bibinfo {author} {\bibnamefont {Goldreich},
  \bibfnamefont {Oded}}, \bibinfo {author} {\bibfnamefont {Silvio}\
  \bibnamefont {Micali}}, \ and\ \bibinfo {author} {\bibfnamefont {Avi}\
  \bibnamefont {Wigderson}}} (\bibinfo {year} {1986}),\ \bibfield  {title}
  {\enquote {\bibinfo {title} {Proofs that yield nothing but their validity and
  a methodology of cryptographic protocol design},}\ }in\ \href {\doibase
  10.1109/SFCS.1986.47} {\emph {\bibinfo {booktitle} {Proceedings of the 27th
  Symposium on Foundations of Computer Science, FOCS~'86}}}\ (\bibinfo
  {publisher} {IEEE})\ pp.\ \bibinfo {pages} {174--187}\BibitemShut {NoStop}%
\bibitem [{\citenamefont {Gottesman}\ and\ \citenamefont {Lo}(2003)}]{GL03}%
  \BibitemOpen
  \bibfield  {author} {\bibinfo {author} {\bibnamefont {Gottesman},
  \bibfnamefont {Daniel}}, \ and\ \bibinfo {author} {\bibfnamefont {Hoi-Kwong}\
  \bibnamefont {Lo}}} (\bibinfo {year} {2003}),\ \bibfield  {title} {\enquote
  {\bibinfo {title} {Proof of security of quantum key distribution with two-way
  classical communications},}\ }\href {\doibase 10.1109/TIT.2002.807289}
  {\bibfield  {journal} {\bibinfo  {journal} {IEEE Trans. Inf. Theory}\
  }\textbf {\bibinfo {volume} {49}}~(\bibinfo {number} {2}),\ \bibinfo {pages}
  {457--475}},\ \Eprint {http://arxiv.org/abs/arXiv:quant-ph/0105121}
  {arXiv:quant-ph/0105121} \BibitemShut {NoStop}%
\bibitem [{\citenamefont {Gottesman}\ \emph {et~al.}(2004)\citenamefont
  {Gottesman}, \citenamefont {Lo}, \citenamefont {L\"{u}tkenhaus},\ and\
  \citenamefont {Preskill}}]{GLLP04}%
  \BibitemOpen
  \bibfield  {author} {\bibinfo {author} {\bibnamefont {Gottesman},
  \bibfnamefont {Daniel}}, \bibinfo {author} {\bibfnamefont {Hoi-Kwong}\
  \bibnamefont {Lo}}, \bibinfo {author} {\bibfnamefont {Norbert}\ \bibnamefont
  {L\"{u}tkenhaus}}, \ and\ \bibinfo {author} {\bibfnamefont {John}\
  \bibnamefont {Preskill}}} (\bibinfo {year} {2004}),\ \bibfield  {title}
  {\enquote {\bibinfo {title} {Security of quantum key distribution with
  imperfect devices},}\ }\href@noop {} {\bibfield  {journal} {\bibinfo
  {journal} {Quantum Inf. Comput.}\ }\textbf {\bibinfo {volume} {4}}~(\bibinfo
  {number} {5}),\ \bibinfo {pages} {325--360}},\ \Eprint
  {http://arxiv.org/abs/arXiv:quant-ph/0212066} {arXiv:quant-ph/0212066}
  \BibitemShut {NoStop}%
\bibitem [{\citenamefont {Goyal}\ \emph {et~al.}(2010)\citenamefont {Goyal},
  \citenamefont {Ishai}, \citenamefont {Sahai}, \citenamefont {Venkatesan},\
  and\ \citenamefont {Wadia}}]{GISVW10}%
  \BibitemOpen
  \bibfield  {author} {\bibinfo {author} {\bibnamefont {Goyal}, \bibfnamefont
  {Vipul}}, \bibinfo {author} {\bibfnamefont {Yuval}\ \bibnamefont {Ishai}},
  \bibinfo {author} {\bibfnamefont {Amit}\ \bibnamefont {Sahai}}, \bibinfo
  {author} {\bibfnamefont {Ramarathnam}\ \bibnamefont {Venkatesan}}, \ and\
  \bibinfo {author} {\bibfnamefont {Akshay}\ \bibnamefont {Wadia}}} (\bibinfo
  {year} {2010}),\ \bibfield  {title} {\enquote {\bibinfo {title} {Founding
  cryptography on tamper-proof hardware tokens},}\ }in\ \href {\doibase
  10.1007/978-3-642-11799-2_19} {\emph {\bibinfo {booktitle} {Theory of
  Cryptography, Proceedings of TCC 2010}}},\ \bibinfo {series} {LNCS}, Vol.\
  \bibinfo {volume} {5978}\ (\bibinfo  {publisher} {Springer})\ pp.\ \bibinfo
  {pages} {308--326},\ \bibinfo {note} {e-Print
  \href{http://eprint.iacr.org/2010/153}{IACR 2010/153}}\BibitemShut {NoStop}%
\bibitem [{\citenamefont {Gutoski}(2012)}]{Gut12}%
  \BibitemOpen
  \bibfield  {author} {\bibinfo {author} {\bibnamefont {Gutoski}, \bibfnamefont
  {Gus}}} (\bibinfo {year} {2012}),\ \bibfield  {title} {\enquote {\bibinfo
  {title} {On a measure of distance for quantum strategies},}\ }\href {\doibase
  10.1063/1.3693621} {\bibfield  {journal} {\bibinfo  {journal} {J. Math.
  Phys.}\ }\textbf {\bibinfo {volume} {53}}~(\bibinfo {number} {3}),\ \bibinfo
  {pages} {032202}},\ \Eprint {http://arxiv.org/abs/arXiv:1008.4636}
  {arXiv:1008.4636} \BibitemShut {NoStop}%
\bibitem [{\citenamefont {Gutoski}\ and\ \citenamefont {Watrous}(2007)}]{GW07}%
  \BibitemOpen
  \bibfield  {author} {\bibinfo {author} {\bibnamefont {Gutoski}, \bibfnamefont
  {Gus}}, \ and\ \bibinfo {author} {\bibfnamefont {John}\ \bibnamefont
  {Watrous}}} (\bibinfo {year} {2007}),\ \bibfield  {title} {\enquote {\bibinfo
  {title} {Toward a general theory of quantum games},}\ }in\ \href {\doibase
  10.1145/1250790.1250873} {\emph {\bibinfo {booktitle} {Proceedings of the
  39th Symposium on Theory of Computing, STOC~'07}}}\ (\bibinfo  {publisher}
  {ACM})\ pp.\ \bibinfo {pages} {565--574},\ \Eprint
  {http://arxiv.org/abs/arXiv:quant-ph/0611234} {arXiv:quant-ph/0611234}
  \BibitemShut {NoStop}%
\bibitem [{\citenamefont {Hardy}(2005)}]{Har05}%
  \BibitemOpen
  \bibfield  {author} {\bibinfo {author} {\bibnamefont {Hardy}, \bibfnamefont
  {Lucien}}} (\bibinfo {year} {2005}),\ \href@noop {} {\enquote {\bibinfo
  {title} {Probability theories with dynamic causal structure: A new framework
  for quantum gravity},}\ }\bibinfo {howpublished} {e-Print},\ \Eprint
  {http://arxiv.org/abs/arXiv:gr-qc/0509120} {arXiv:gr-qc/0509120} \BibitemShut
  {NoStop}%
\bibitem [{\citenamefont {Hardy}(2007)}]{Har07}%
  \BibitemOpen
  \bibfield  {author} {\bibinfo {author} {\bibnamefont {Hardy}, \bibfnamefont
  {Lucien}}} (\bibinfo {year} {2007}),\ \bibfield  {title} {\enquote {\bibinfo
  {title} {Towards quantum gravity: a framework for probabilistic theories with
  non-fixed causal structure},}\ }\href {\doibase 10.1088/1751-8113/40/12/S12}
  {\bibfield  {journal} {\bibinfo  {journal} {J. Phys. A}\ }\textbf {\bibinfo
  {volume} {40}}~(\bibinfo {number} {12}),\ \bibinfo {pages} {3081}},\ \Eprint
  {http://arxiv.org/abs/arXiv:gr-qc/0608043} {arXiv:gr-qc/0608043} \BibitemShut
  {NoStop}%
\bibitem [{\citenamefont {Hardy}(2011)}]{Har11}%
  \BibitemOpen
  \bibfield  {author} {\bibinfo {author} {\bibnamefont {Hardy}, \bibfnamefont
  {Lucien}}} (\bibinfo {year} {2011}),\ \href@noop {} {\enquote {\bibinfo
  {title} {Reformulating and reconstructing quantum theory},}\ }\bibinfo
  {howpublished} {e-print},\ \Eprint {http://arxiv.org/abs/arXiv:1104.2066}
  {arXiv:1104.2066} \BibitemShut {NoStop}%
\bibitem [{\citenamefont {Hardy}(2012)}]{Har12}%
  \BibitemOpen
  \bibfield  {author} {\bibinfo {author} {\bibnamefont {Hardy}, \bibfnamefont
  {Lucien}}} (\bibinfo {year} {2012}),\ \bibfield  {title} {\enquote {\bibinfo
  {title} {The operator tensor formulation of quantum theory},}\ }\href
  {\doibase 10.1098/rsta.2011.0326} {\bibfield  {journal} {\bibinfo  {journal}
  {Philos. Trans. R. Soc. London, Ser. A}\ }\textbf {\bibinfo {volume}
  {370}}~(\bibinfo {number} {1971}),\ \bibinfo {pages} {3385--3417}},\ \Eprint
  {http://arxiv.org/abs/arXiv:1201.4390} {arXiv:1201.4390} \BibitemShut
  {NoStop}%
\bibitem [{\citenamefont {Hardy}(2015)}]{Har15}%
  \BibitemOpen
  \bibfield  {author} {\bibinfo {author} {\bibnamefont {Hardy}, \bibfnamefont
  {Lucien}}} (\bibinfo {year} {2015}),\ \bibfield  {title} {\enquote {\bibinfo
  {title} {Quantum theory with bold operator tensors},}\ }\href {\doibase
  10.1098/rsta.2014.0239} {\bibfield  {journal} {\bibinfo  {journal} {Philos.
  Trans. R. Soc. London, Ser. A}\ }\textbf {\bibinfo {volume} {373}}~(\bibinfo
  {number} {2047}),\ 10.1098/rsta.2014.0239}\BibitemShut {NoStop}%
\bibitem [{\citenamefont {Hayashi}\ and\ \citenamefont
  {Tsurumaru}(2012)}]{HT12}%
  \BibitemOpen
  \bibfield  {author} {\bibinfo {author} {\bibnamefont {Hayashi}, \bibfnamefont
  {Masahito}}, \ and\ \bibinfo {author} {\bibfnamefont {Toyohiro}\ \bibnamefont
  {Tsurumaru}}} (\bibinfo {year} {2012}),\ \bibfield  {title} {\enquote
  {\bibinfo {title} {Concise and tight security analysis of the
  {Bennett}{\textendash}{Brassard} 1984 protocol with finite key lengths},}\
  }\href {\doibase 10.1088/1367-2630/14/9/093014} {\bibfield  {journal}
  {\bibinfo  {journal} {New J. Phys.}\ }\textbf {\bibinfo {volume}
  {14}}~(\bibinfo {number} {9}),\ \bibinfo {pages} {093014}},\ \Eprint
  {http://arxiv.org/abs/arXiv:1107.0589} {arXiv:1107.0589} \BibitemShut
  {NoStop}%
\bibitem [{\citenamefont {Hayden}\ \emph {et~al.}(2011)\citenamefont {Hayden},
  \citenamefont {Leung},\ and\ \citenamefont {Mayers}}]{HLM11}%
  \BibitemOpen
  \bibfield  {author} {\bibinfo {author} {\bibnamefont {Hayden}, \bibfnamefont
  {Patrick}}, \bibinfo {author} {\bibfnamefont {Debbie}\ \bibnamefont {Leung}},
  \ and\ \bibinfo {author} {\bibfnamefont {Dominic}\ \bibnamefont {Mayers}}}
  (\bibinfo {year} {2011}),\ \href@noop {} {\enquote {\bibinfo {title} {The
  universal composable security of quantum message authentication with key
  recycling},}\ }\bibinfo {howpublished} {presented at QCrypt 2011, e-Print},\
  \Eprint {http://arxiv.org/abs/arXiv:1610.09434} {arXiv:1610.09434}
  \BibitemShut {NoStop}%
\bibitem [{\citenamefont {Hayden}\ \emph {et~al.}(2004)\citenamefont {Hayden},
  \citenamefont {Leung}, \citenamefont {Shor},\ and\ \citenamefont
  {Winter}}]{HLSW04}%
  \BibitemOpen
  \bibfield  {author} {\bibinfo {author} {\bibnamefont {Hayden}, \bibfnamefont
  {Patrick}}, \bibinfo {author} {\bibfnamefont {Debbie}\ \bibnamefont {Leung}},
  \bibinfo {author} {\bibfnamefont {Peter~W.}\ \bibnamefont {Shor}}, \ and\
  \bibinfo {author} {\bibfnamefont {Andreas}\ \bibnamefont {Winter}}} (\bibinfo
  {year} {2004}),\ \bibfield  {title} {\enquote {\bibinfo {title} {Randomizing
  quantum states: Constructions and applications},}\ }\href {\doibase
  10.1007/s00220-004-1087-6} {\bibfield  {journal} {\bibinfo  {journal}
  {Commun. Math. Phys.}\ }\textbf {\bibinfo {volume} {250}},\ \bibinfo {pages}
  {371--391}},\ \Eprint {http://arxiv.org/abs/arXiv:quant-ph/0307104v3}
  {arXiv:quant-ph/0307104v3} \BibitemShut {NoStop}%
\bibitem [{\citenamefont {Helstrom}(1976)}]{Hel76}%
  \BibitemOpen
  \bibfield  {author} {\bibinfo {author} {\bibnamefont {Helstrom},
  \bibfnamefont {Carl~W}}} (\bibinfo {year} {1976}),\ \href@noop {} {\emph
  {\bibinfo {title} {Quantum Detection and Estimation Theory}}},\ \bibinfo
  {series} {Mathematics in science and engineering}, Vol.\ \bibinfo {volume}
  {123}\ (\bibinfo  {publisher} {Academic Press})\BibitemShut {NoStop}%
\bibitem [{\citenamefont {Hensen}\ \emph {et~al.}(2015)\citenamefont {Hensen},
  \citenamefont {Bernien}, \citenamefont {Dr{\'e}au}, \citenamefont {Reiserer},
  \citenamefont {Kalb}, \citenamefont {Blok}, \citenamefont {Ruitenberg},
  \citenamefont {Vermeulen}, \citenamefont {Schouten}, \citenamefont
  {Abell{\'a}n}, \citenamefont {Amaya}, \citenamefont {Pruneri}, \citenamefont
  {Mitchell}, \citenamefont {Markham}, \citenamefont {Twitchen}, \citenamefont
  {Elkouss}, \citenamefont {Wehner}, \citenamefont {Taminiau},\ and\
  \citenamefont {Hanson}}]{Hensen}%
  \BibitemOpen
  \bibfield  {author} {\bibinfo {author} {\bibnamefont {Hensen}, \bibfnamefont
  {B}}, \bibinfo {author} {\bibfnamefont {H.}~\bibnamefont {Bernien}}, \bibinfo
  {author} {\bibfnamefont {A.~E.}\ \bibnamefont {Dr{\'e}au}}, \bibinfo {author}
  {\bibfnamefont {A.}~\bibnamefont {Reiserer}}, \bibinfo {author}
  {\bibfnamefont {N.}~\bibnamefont {Kalb}}, \bibinfo {author} {\bibfnamefont
  {M.~S.}\ \bibnamefont {Blok}}, \bibinfo {author} {\bibfnamefont
  {J.}~\bibnamefont {Ruitenberg}}, \bibinfo {author} {\bibfnamefont {R.~F.~L.}\
  \bibnamefont {Vermeulen}}, \bibinfo {author} {\bibfnamefont {R.~N.}\
  \bibnamefont {Schouten}}, \bibinfo {author} {\bibfnamefont {C.}~\bibnamefont
  {Abell{\'a}n}}, \bibinfo {author} {\bibfnamefont {W.}~\bibnamefont {Amaya}},
  \bibinfo {author} {\bibfnamefont {V.}~\bibnamefont {Pruneri}}, \bibinfo
  {author} {\bibfnamefont {M.~W.}\ \bibnamefont {Mitchell}}, \bibinfo {author}
  {\bibfnamefont {M.}~\bibnamefont {Markham}}, \bibinfo {author} {\bibfnamefont
  {D.~J.}\ \bibnamefont {Twitchen}}, \bibinfo {author} {\bibfnamefont
  {D.}~\bibnamefont {Elkouss}}, \bibinfo {author} {\bibfnamefont
  {S.}~\bibnamefont {Wehner}}, \bibinfo {author} {\bibfnamefont {T.~H.}\
  \bibnamefont {Taminiau}}, \ and\ \bibinfo {author} {\bibfnamefont
  {R.}~\bibnamefont {Hanson}}} (\bibinfo {year} {2015}),\ \bibfield  {title}
  {\enquote {\bibinfo {title} {Loophole-free bell inequality violation using
  electron spins separated by 1.3 kilometres},}\ }\href {\doibase
  10.1038/nature15759} {\bibfield  {journal} {\bibinfo  {journal} {Nature}\
  }\textbf {\bibinfo {volume} {526}}~(\bibinfo {number} {7575}),\ \bibinfo
  {pages} {682--686}}\BibitemShut {NoStop}%
\bibitem [{\citenamefont {Hofheinz}\ \emph {et~al.}(2006)\citenamefont
  {Hofheinz}, \citenamefont {M\"{u}ller-Quade},\ and\ \citenamefont
  {Unruh}}]{HMU06}%
  \BibitemOpen
  \bibfield  {author} {\bibinfo {author} {\bibnamefont {Hofheinz},
  \bibfnamefont {Dennis}}, \bibinfo {author} {\bibfnamefont {J\"{o}rn}\
  \bibnamefont {M\"{u}ller-Quade}}, \ and\ \bibinfo {author} {\bibfnamefont
  {Dominique}\ \bibnamefont {Unruh}}} (\bibinfo {year} {2006}),\ \bibfield
  {title} {\enquote {\bibinfo {title} {On the (im)possibility of extending coin
  toss},}\ }in\ \href@noop {} {\emph {\bibinfo {booktitle} {Advances in
  Cryptology -- EUROCRYPT 2006}}},\ \bibinfo {series} {LNCS}, Vol.\ \bibinfo
  {volume} {4004}\ (\bibinfo  {publisher} {Springer})\ pp.\ \bibinfo {pages}
  {504--521},\ \bibinfo {note} {e-Print
  \href{http://eprint.iacr.org/2006/177}{IACR 2006/177}}\BibitemShut {NoStop}%
\bibitem [{\citenamefont {Hofheinz}\ and\ \citenamefont {Shoup}(2013)}]{HS13}%
  \BibitemOpen
  \bibfield  {author} {\bibinfo {author} {\bibnamefont {Hofheinz},
  \bibfnamefont {Dennis}}, \ and\ \bibinfo {author} {\bibfnamefont {Victor}\
  \bibnamefont {Shoup}}} (\bibinfo {year} {2013}),\ \bibfield  {title}
  {\enquote {\bibinfo {title} {{GNUC}: A new universal composability
  framework},}\ }\href {\doibase 10.1007/s00145-013-9160-y} {\bibfield
  {journal} {\bibinfo  {journal} {J. Crypt.}\ ,\ \bibinfo {pages}
  {1--86}}}\bibinfo {note} {E-Print \href{http://eprint.iacr.org/2011/303}{IACR
  2011/303}}\BibitemShut {NoStop}%
\bibitem [{\citenamefont {Horodecki}\ \emph {et~al.}(2008)\citenamefont
  {Horodecki}, \citenamefont {Horodecki}, \citenamefont {Horodecki},
  \citenamefont {Leung},\ and\ \citenamefont {Oppenheim}}]{HHHLO08}%
  \BibitemOpen
  \bibfield  {author} {\bibinfo {author} {\bibnamefont {Horodecki},
  \bibfnamefont {Karol}}, \bibinfo {author} {\bibfnamefont {Micha\l{}}\
  \bibnamefont {Horodecki}}, \bibinfo {author} {\bibfnamefont {Pave\l{}}\
  \bibnamefont {Horodecki}}, \bibinfo {author} {\bibfnamefont {Debbie}\
  \bibnamefont {Leung}}, \ and\ \bibinfo {author} {\bibfnamefont {Jonathan}\
  \bibnamefont {Oppenheim}}} (\bibinfo {year} {2008}),\ \bibfield  {title}
  {\enquote {\bibinfo {title} {Quantum key distribution based on private
  states: Unconditional security over untrusted channels with zero quantum
  capacity},}\ }\href {\doibase 10.1109/TIT.2008.921870} {\bibfield  {journal}
  {\bibinfo  {journal} {IEEE Trans. Inf. Theory}\ }\textbf {\bibinfo {volume}
  {54}}~(\bibinfo {number} {6}),\ \bibinfo {pages} {2604--2620}}\BibitemShut
  {NoStop}%
\bibitem [{\citenamefont {Horodecki}\ and\ \citenamefont
  {Stankiewicz}(2020)}]{HS20}%
  \BibitemOpen
  \bibfield  {author} {\bibinfo {author} {\bibnamefont {Horodecki},
  \bibfnamefont {Karol}}, \ and\ \bibinfo {author} {\bibfnamefont {Maciej}\
  \bibnamefont {Stankiewicz}}} (\bibinfo {year} {2020}),\ \bibfield  {title}
  {\enquote {\bibinfo {title} {Semi-device-independent quantum money},}\ }\href
  {\doibase 10.1088/1367-2630/ab6872} {\bibfield  {journal} {\bibinfo
  {journal} {New J. Phys.}\ }\textbf {\bibinfo {volume} {22}}~(\bibinfo
  {number} {2}),\ \bibinfo {pages} {023007}},\ \Eprint
  {http://arxiv.org/abs/arxiv:1811.10552} {arxiv:1811.10552} \BibitemShut
  {NoStop}%
\bibitem [{\citenamefont {Horodecki}\ \emph {et~al.}(1998)\citenamefont
  {Horodecki}, \citenamefont {Horodecki},\ and\ \citenamefont
  {Horodecki}}]{HHH98}%
  \BibitemOpen
  \bibfield  {author} {\bibinfo {author} {\bibnamefont {Horodecki},
  \bibfnamefont {Micha\l{}}}, \bibinfo {author} {\bibfnamefont {Pawe\l{}}\
  \bibnamefont {Horodecki}}, \ and\ \bibinfo {author} {\bibfnamefont {Ryszard}\
  \bibnamefont {Horodecki}}} (\bibinfo {year} {1998}),\ \bibfield  {title}
  {\enquote {\bibinfo {title} {Mixed-state entanglement and distillation: Is
  there a ``bound'' entanglement in nature?}}\ }\href {\doibase
  10.1103/PhysRevLett.80.5239} {\bibfield  {journal} {\bibinfo  {journal}
  {Phys. Rev. Lett.}\ }\textbf {\bibinfo {volume} {80}},\ \bibinfo {pages}
  {5239--5242}}\BibitemShut {NoStop}%
\bibitem [{\citenamefont {Hwang}(2003)}]{Hwang2003}%
  \BibitemOpen
  \bibfield  {author} {\bibinfo {author} {\bibnamefont {Hwang}, \bibfnamefont
  {Won-Young}}} (\bibinfo {year} {2003}),\ \bibfield  {title} {\enquote
  {\bibinfo {title} {Quantum key distribution with high loss: Toward global
  secure communication},}\ }\href {\doibase 10.1103/PhysRevLett.91.057901}
  {\bibfield  {journal} {\bibinfo  {journal} {Phys. Rev. Lett.}\ }\textbf
  {\bibinfo {volume} {91}},\ \bibinfo {pages} {057901}}\BibitemShut {NoStop}%
\bibitem [{\citenamefont {Inamori}\ \emph {et~al.}(2007)\citenamefont
  {Inamori}, \citenamefont {L{\"u}tkenhaus},\ and\ \citenamefont
  {Mayers}}]{ILM07}%
  \BibitemOpen
  \bibfield  {author} {\bibinfo {author} {\bibnamefont {Inamori}, \bibfnamefont
  {Hitoshi}}, \bibinfo {author} {\bibfnamefont {Norbert}\ \bibnamefont
  {L{\"u}tkenhaus}}, \ and\ \bibinfo {author} {\bibfnamefont {Dominic}\
  \bibnamefont {Mayers}}} (\bibinfo {year} {2007}),\ \bibfield  {title}
  {\enquote {\bibinfo {title} {Unconditional security of practical quantum key
  distribution},}\ }\href {\doibase 10.1140/epjd/e2007-00010-4} {\bibfield
  {journal} {\bibinfo  {journal} {Eur. Phys. J. D}\ }\textbf {\bibinfo {volume}
  {41}}~(\bibinfo {number} {3}),\ \bibinfo {pages} {599--627}},\ \Eprint
  {http://arxiv.org/abs/arXiv:quant-ph/0107017} {arXiv:quant-ph/0107017}
  \BibitemShut {NoStop}%
\bibitem [{\citenamefont {Inoue}\ \emph {et~al.}(2002)\citenamefont {Inoue},
  \citenamefont {Waks},\ and\ \citenamefont {Yamamoto}}]{IWY02}%
  \BibitemOpen
  \bibfield  {author} {\bibinfo {author} {\bibnamefont {Inoue}, \bibfnamefont
  {Kyo}}, \bibinfo {author} {\bibfnamefont {Edo}\ \bibnamefont {Waks}}, \ and\
  \bibinfo {author} {\bibfnamefont {Yoshihisa}\ \bibnamefont {Yamamoto}}}
  (\bibinfo {year} {2002}),\ \bibfield  {title} {\enquote {\bibinfo {title}
  {Differential phase shift quantum key distribution},}\ }\href {\doibase
  10.1103/PhysRevLett.89.037902} {\bibfield  {journal} {\bibinfo  {journal}
  {Phys. Rev. Lett.}\ }\textbf {\bibinfo {volume} {89}},\ \bibinfo {pages}
  {037902}}\BibitemShut {NoStop}%
\bibitem [{\citenamefont {Ishai}\ \emph {et~al.}(2014)\citenamefont {Ishai},
  \citenamefont {Ostrovsky},\ and\ \citenamefont {Zikas}}]{IOZ14}%
  \BibitemOpen
  \bibfield  {author} {\bibinfo {author} {\bibnamefont {Ishai}, \bibfnamefont
  {Yuval}}, \bibinfo {author} {\bibfnamefont {Rafail}\ \bibnamefont
  {Ostrovsky}}, \ and\ \bibinfo {author} {\bibfnamefont {Vassilis}\
  \bibnamefont {Zikas}}} (\bibinfo {year} {2014}),\ \bibfield  {title}
  {\enquote {\bibinfo {title} {Secure multi-party computation with identifiable
  abort},}\ }in\ \href {\doibase 10.1007/978-3-662-44381-1_21} {\emph {\bibinfo
  {booktitle} {Advances in Cryptology -- CRYPTO 2014}}},\ \bibinfo {editor}
  {edited by\ \bibinfo {editor} {\bibfnamefont {Juan~A.}\ \bibnamefont
  {Garay}}\ and\ \bibinfo {editor} {\bibfnamefont {Rosario}\ \bibnamefont
  {Gennaro}}}\ (\bibinfo  {publisher} {Springer})\ pp.\ \bibinfo {pages}
  {369--386},\ \bibinfo {note} {e-Print
  \href{http://eprint.iacr.org/2015/325}{IACR 2015/325}}\BibitemShut {NoStop}%
\bibitem [{\citenamefont {Ishai}\ \emph {et~al.}(2008)\citenamefont {Ishai},
  \citenamefont {Prabhakaran},\ and\ \citenamefont {Sahai}}]{IPS08}%
  \BibitemOpen
  \bibfield  {author} {\bibinfo {author} {\bibnamefont {Ishai}, \bibfnamefont
  {Yuval}}, \bibinfo {author} {\bibfnamefont {Manoj}\ \bibnamefont
  {Prabhakaran}}, \ and\ \bibinfo {author} {\bibfnamefont {Amit}\ \bibnamefont
  {Sahai}}} (\bibinfo {year} {2008}),\ \bibfield  {title} {\enquote {\bibinfo
  {title} {Founding cryptography on oblivious transfer - efficiently},}\ }in\
  \href {\doibase 10.1007/978-3-540-85174-5_32} {\emph {\bibinfo {booktitle}
  {Advances in Cryptology -- CRYPTO 2008}}},\ \bibinfo {series} {LNCS}, Vol.\
  \bibinfo {volume} {5157}\ (\bibinfo  {publisher} {Springer})\ pp.\ \bibinfo
  {pages} {572--591}\BibitemShut {NoStop}%
\bibitem [{\citenamefont {Jost}\ and\ \citenamefont {Maurer}(2018)}]{JM18}%
  \BibitemOpen
  \bibfield  {author} {\bibinfo {author} {\bibnamefont {Jost}, \bibfnamefont
  {Daniel}}, \ and\ \bibinfo {author} {\bibfnamefont {Ueli}\ \bibnamefont
  {Maurer}}} (\bibinfo {year} {2018}),\ \bibfield  {title} {\enquote {\bibinfo
  {title} {Security definitions for hash functions: Combining {UCE} and
  {I}ndifferentiability},}\ }in\ \href@noop {} {\emph {\bibinfo {booktitle}
  {International Conference on Security and Cryptography for Networks -- SCN
  2018}}},\ \bibinfo {series} {LNCS}, Vol.\ \bibinfo {volume} {11035},\
  \bibinfo {editor} {edited by\ \bibinfo {editor} {\bibfnamefont {Dario}\
  \bibnamefont {Catalano}}\ and\ \bibinfo {editor} {\bibfnamefont {Roberto}\
  \bibnamefont {De~Prisco}}}\ (\bibinfo  {publisher} {Springer})\ pp.\ \bibinfo
  {pages} {83--101},\ \bibinfo {note} {e-Print
  \href{http://eprint.iacr.org/2006/281}{IACR 2006/281}}\BibitemShut {NoStop}%
\bibitem [{\citenamefont {Jouguet}\ and\ \citenamefont
  {Kunz-Jacques}(2014)}]{JK14}%
  \BibitemOpen
  \bibfield  {author} {\bibinfo {author} {\bibnamefont {Jouguet}, \bibfnamefont
  {Paul}}, \ and\ \bibinfo {author} {\bibfnamefont {Sebastien}\ \bibnamefont
  {Kunz-Jacques}}} (\bibinfo {year} {2014}),\ \bibfield  {title} {\enquote
  {\bibinfo {title} {High performance error correction for quantum key
  distribution using polar codes},}\ }\href@noop {} {\bibfield  {journal}
  {\bibinfo  {journal} {Quantum Inf. Comput.}\ }\textbf {\bibinfo {volume}
  {14}}~(\bibinfo {number} {3-4}),\ \bibinfo {pages} {329--338}}\BibitemShut
  {NoStop}%
\bibitem [{\citenamefont {Kaniewski}(2015)}]{Kan15}%
  \BibitemOpen
  \bibfield  {author} {\bibinfo {author} {\bibnamefont {Kaniewski},
  \bibfnamefont {J\c{e}drzej}}} (\bibinfo {year} {2015}),\ \emph {\bibinfo
  {title} {Relativistic quantum cryptography}},\ \href@noop {} {Ph.D. thesis}\
  (\bibinfo  {school} {National University of Singapore}),\ \Eprint
  {http://arxiv.org/abs/arXiv:1512.00602} {arXiv:1512.00602} \BibitemShut
  {NoStop}%
\bibitem [{\citenamefont {Kaniewski}\ \emph {et~al.}(2013)\citenamefont
  {Kaniewski}, \citenamefont {Tomamichel}, \citenamefont {H\"anggi},\ and\
  \citenamefont {Wehner}}]{KTHW13}%
  \BibitemOpen
  \bibfield  {author} {\bibinfo {author} {\bibnamefont {Kaniewski},
  \bibfnamefont {J\c{e}drzej}}, \bibinfo {author} {\bibfnamefont {Marco}\
  \bibnamefont {Tomamichel}}, \bibinfo {author} {\bibfnamefont {Esther}\
  \bibnamefont {H\"anggi}}, \ and\ \bibinfo {author} {\bibfnamefont
  {Stephanie}\ \bibnamefont {Wehner}}} (\bibinfo {year} {2013}),\ \bibfield
  {title} {\enquote {\bibinfo {title} {Secure bit commitment from relativistic
  constraints},}\ }\href {\doibase 10.1109/TIT.2013.2247463} {\bibfield
  {journal} {\bibinfo  {journal} {IEEE Transactions on Information Theory}\
  }\textbf {\bibinfo {volume} {59}}~(\bibinfo {number} {7}),\ \bibinfo {pages}
  {4687--4699}},\ \Eprint {http://arxiv.org/abs/arXiv:1206.1740}
  {arXiv:1206.1740} \BibitemShut {NoStop}%
\bibitem [{\citenamefont {Katz}\ and\ \citenamefont {Yung}(2006)}]{KY06}%
  \BibitemOpen
  \bibfield  {author} {\bibinfo {author} {\bibnamefont {Katz}, \bibfnamefont
  {Jonathan}}, \ and\ \bibinfo {author} {\bibfnamefont {Moti}\ \bibnamefont
  {Yung}}} (\bibinfo {year} {2006}),\ \bibfield  {title} {\enquote {\bibinfo
  {title} {Characterization of security notions for probabilistic private-key
  encryption},}\ }\href {\doibase 10.1007/s00145-005-0310-8} {\bibfield
  {journal} {\bibinfo  {journal} {J. Crypt.}\ }\textbf {\bibinfo {volume}
  {19}}~(\bibinfo {number} {1}),\ \bibinfo {pages} {67--95}}\BibitemShut
  {NoStop}%
\bibitem [{\citenamefont {Kent}(1999)}]{Ken99}%
  \BibitemOpen
  \bibfield  {author} {\bibinfo {author} {\bibnamefont {Kent}, \bibfnamefont
  {Adrian}}} (\bibinfo {year} {1999}),\ \bibfield  {title} {\enquote {\bibinfo
  {title} {Unconditionally secure bit commitment},}\ }\href {\doibase
  10.1103/PhysRevLett.83.1447} {\bibfield  {journal} {\bibinfo  {journal}
  {Phys. Rev. Lett.}\ }\textbf {\bibinfo {volume} {83}},\ \bibinfo {pages}
  {1447--1450}},\ \Eprint {http://arxiv.org/abs/arXiv:quant-ph/9810068}
  {arXiv:quant-ph/9810068} \BibitemShut {NoStop}%
\bibitem [{\citenamefont {Kent}(2012)}]{Ken12}%
  \BibitemOpen
  \bibfield  {author} {\bibinfo {author} {\bibnamefont {Kent}, \bibfnamefont
  {Adrian}}} (\bibinfo {year} {2012}),\ \bibfield  {title} {\enquote {\bibinfo
  {title} {Unconditionally secure bit commitment by transmitting measurement
  outcomes},}\ }\href {\doibase 10.1103/PhysRevLett.109.130501} {\bibfield
  {journal} {\bibinfo  {journal} {Phys. Rev. Lett.}\ }\textbf {\bibinfo
  {volume} {109}},\ \bibinfo {pages} {130501}},\ \Eprint
  {http://arxiv.org/abs/arXiv:1108.2879} {arXiv:1108.2879} \BibitemShut
  {NoStop}%
\bibitem [{\citenamefont {Kessler}\ and\ \citenamefont
  {Arnon-Friedman}(2020)}]{KAF20}%
  \BibitemOpen
  \bibfield  {author} {\bibinfo {author} {\bibnamefont {Kessler}, \bibfnamefont
  {Max}}, \ and\ \bibinfo {author} {\bibfnamefont {Rotem}\ \bibnamefont
  {Arnon-Friedman}}} (\bibinfo {year} {2020}),\ \bibfield  {title} {\enquote
  {\bibinfo {title} {Device-independent randomness amplification and
  privatization},}\ }\href {\doibase 10.1109/JSAIT.2020.3012498} {\bibfield
  {journal} {\bibinfo  {journal} {IEEE J. Sel. Areas Inf. Theory}\ }\textbf
  {\bibinfo {volume} {1}}~(\bibinfo {number} {2}),\ \bibinfo {pages}
  {568--584}},\ \Eprint {http://arxiv.org/abs/arXiv:1705.04148}
  {arXiv:1705.04148} \BibitemShut {NoStop}%
\bibitem [{\citenamefont {Koashi}(2004)}]{Koashi04}%
  \BibitemOpen
  \bibfield  {author} {\bibinfo {author} {\bibnamefont {Koashi}, \bibfnamefont
  {Masato}}} (\bibinfo {year} {2004}),\ \bibfield  {title} {\enquote {\bibinfo
  {title} {Unconditional security of coherent-state quantum key distribution
  with a strong phase-reference pulse},}\ }\href {\doibase
  10.1103/PhysRevLett.93.120501} {\bibfield  {journal} {\bibinfo  {journal}
  {Phys. Rev. Lett.}\ }\textbf {\bibinfo {volume} {93}},\ \bibinfo {pages}
  {120501}}\BibitemShut {NoStop}%
\bibitem [{\citenamefont {Koashi}(2009)}]{Koa09}%
  \BibitemOpen
  \bibfield  {author} {\bibinfo {author} {\bibnamefont {Koashi}, \bibfnamefont
  {Masato}}} (\bibinfo {year} {2009}),\ \bibfield  {title} {\enquote {\bibinfo
  {title} {Simple security proof of quantum key distribution based on
  complementarity},}\ }\href {\doibase 10.1088/1367-2630/11/4/045018}
  {\bibfield  {journal} {\bibinfo  {journal} {New J. Phys.}\ }\textbf {\bibinfo
  {volume} {11}}~(\bibinfo {number} {4}),\ \bibinfo {pages}
  {045018}}\BibitemShut {NoStop}%
\bibitem [{\citenamefont {Koashi}\ and\ \citenamefont
  {Winter}(2004)}]{KoashiWinter04}%
  \BibitemOpen
  \bibfield  {author} {\bibinfo {author} {\bibnamefont {Koashi}, \bibfnamefont
  {Masato}}, \ and\ \bibinfo {author} {\bibfnamefont {Andreas}\ \bibnamefont
  {Winter}}} (\bibinfo {year} {2004}),\ \bibfield  {title} {\enquote {\bibinfo
  {title} {Monogamy of quantum entanglement and other correlations},}\ }\href
  {\doibase 10.1103/PhysRevA.69.022309} {\bibfield  {journal} {\bibinfo
  {journal} {Phys. Rev. A}\ }\textbf {\bibinfo {volume} {69}},\ \bibinfo
  {pages} {022309}}\BibitemShut {NoStop}%
\bibitem [{\citenamefont {Kochen}\ and\ \citenamefont
  {Specker}(1967)}]{KocSpe67}%
  \BibitemOpen
  \bibfield  {author} {\bibinfo {author} {\bibnamefont {Kochen}, \bibfnamefont
  {Simon~B}}, \ and\ \bibinfo {author} {\bibfnamefont {Ernst~P.}\ \bibnamefont
  {Specker}}} (\bibinfo {year} {1967}),\ \bibfield  {title} {\enquote {\bibinfo
  {title} {The problem of hidden variables in quantum mechanics},}\ }\href@noop
  {} {\bibfield  {journal} {\bibinfo  {journal} {J. Math. Mech.}\ }\textbf
  {\bibinfo {volume} {17}},\ \bibinfo {pages} {59--87}}\BibitemShut {NoStop}%
\bibitem [{\citenamefont {K\"onig}\ \emph {et~al.}(2005)\citenamefont
  {K\"onig}, \citenamefont {Maurer},\ and\ \citenamefont {Renner}}]{KMR05}%
  \BibitemOpen
  \bibfield  {author} {\bibinfo {author} {\bibnamefont {K\"onig}, \bibfnamefont
  {Robert}}, \bibinfo {author} {\bibfnamefont {Ueli}\ \bibnamefont {Maurer}}, \
  and\ \bibinfo {author} {\bibfnamefont {Renato}\ \bibnamefont {Renner}}}
  (\bibinfo {year} {2005}),\ \bibfield  {title} {\enquote {\bibinfo {title} {On
  the power of quantum memory},}\ }\href {\doibase 10.1109/TIT.2005.850087}
  {\bibfield  {journal} {\bibinfo  {journal} {IEEE Trans. Inf. Theory}\
  }\textbf {\bibinfo {volume} {51}}~(\bibinfo {number} {7}),\ \bibinfo {pages}
  {2391--2401}},\ \Eprint {http://arxiv.org/abs/quant-ph/0305154}
  {quant-ph/0305154} \BibitemShut {NoStop}%
\bibitem [{\citenamefont {K\"onig}\ \emph {et~al.}(2007)\citenamefont
  {K\"onig}, \citenamefont {Renner}, \citenamefont {Bariska},\ and\
  \citenamefont {Maurer}}]{KRBM07}%
  \BibitemOpen
  \bibfield  {author} {\bibinfo {author} {\bibnamefont {K\"onig}, \bibfnamefont
  {Robert}}, \bibinfo {author} {\bibfnamefont {Renato}\ \bibnamefont {Renner}},
  \bibinfo {author} {\bibfnamefont {Andor}\ \bibnamefont {Bariska}}, \ and\
  \bibinfo {author} {\bibfnamefont {Ueli}\ \bibnamefont {Maurer}}} (\bibinfo
  {year} {2007}),\ \bibfield  {title} {\enquote {\bibinfo {title} {Small
  accessible quantum information does not imply security},}\ }\href {\doibase
  10.1103/PhysRevLett.98.140502} {\bibfield  {journal} {\bibinfo  {journal}
  {Phys. Rev. Lett.}\ }\textbf {\bibinfo {volume} {98}},\ \bibinfo {pages}
  {140502}},\ \Eprint {http://arxiv.org/abs/arXiv:quant-ph/0512021}
  {arXiv:quant-ph/0512021} \BibitemShut {NoStop}%
\bibitem [{\citenamefont {K\"onig}\ and\ \citenamefont {Terhal}(2008)}]{KT08}%
  \BibitemOpen
  \bibfield  {author} {\bibinfo {author} {\bibnamefont {K\"onig}, \bibfnamefont
  {Robert}}, \ and\ \bibinfo {author} {\bibfnamefont {Barbara~M.}\ \bibnamefont
  {Terhal}}} (\bibinfo {year} {2008}),\ \bibfield  {title} {\enquote {\bibinfo
  {title} {The bounded-storage model in the presence of a quantum adversary},}\
  }\href {\doibase 10.1109/TIT.2007.913245} {\bibfield  {journal} {\bibinfo
  {journal} {IEEE Trans. Inf. Theory}\ }\textbf {\bibinfo {volume}
  {54}}~(\bibinfo {number} {2}),\ \bibinfo {pages} {749--762}},\ \Eprint
  {http://arxiv.org/abs/arXiv:quant-ph/0608101} {arXiv:quant-ph/0608101}
  \BibitemShut {NoStop}%
\bibitem [{\citenamefont {K\"onig}\ \emph {et~al.}(2012)\citenamefont
  {K\"onig}, \citenamefont {Wehner},\ and\ \citenamefont
  {Wullschleger}}]{KWW12}%
  \BibitemOpen
  \bibfield  {author} {\bibinfo {author} {\bibnamefont {K\"onig}, \bibfnamefont
  {Robert}}, \bibinfo {author} {\bibfnamefont {Stephanie}\ \bibnamefont
  {Wehner}}, \ and\ \bibinfo {author} {\bibfnamefont {J\"urg}\ \bibnamefont
  {Wullschleger}}} (\bibinfo {year} {2012}),\ \bibfield  {title} {\enquote
  {\bibinfo {title} {Unconditional security from noisy quantum storage},}\
  }\href {\doibase 10.1109/TIT.2011.2177772} {\bibfield  {journal} {\bibinfo
  {journal} {IEEE Transactions on Information Theory}\ }\textbf {\bibinfo
  {volume} {58}}~(\bibinfo {number} {3}),\ \bibinfo {pages} {1962--1984}},\
  \Eprint {http://arxiv.org/abs/arXiv:0906.1030} {arXiv:0906.1030} \BibitemShut
  {NoStop}%
\bibitem [{\citenamefont {Kraus}\ \emph {et~al.}(2005)\citenamefont {Kraus},
  \citenamefont {Gisin},\ and\ \citenamefont {Renner}}]{PhysRevLett.95.080501}%
  \BibitemOpen
  \bibfield  {author} {\bibinfo {author} {\bibnamefont {Kraus}, \bibfnamefont
  {Barbara}}, \bibinfo {author} {\bibfnamefont {Nicolas}\ \bibnamefont
  {Gisin}}, \ and\ \bibinfo {author} {\bibfnamefont {Renner}\ \bibnamefont
  {Renner}}} (\bibinfo {year} {2005}),\ \bibfield  {title} {\enquote {\bibinfo
  {title} {Lower and upper bounds on the secret-key rate for quantum key
  distribution protocols using one-way classical communication},}\ }\href
  {\doibase 10.1103/PhysRevLett.95.080501} {\bibfield  {journal} {\bibinfo
  {journal} {Phys. Rev. Lett.}\ }\textbf {\bibinfo {volume} {95}},\ \bibinfo
  {pages} {080501}}\BibitemShut {NoStop}%
\bibitem [{\citenamefont {K{\"u}sters}(2006)}]{Kus06}%
  \BibitemOpen
  \bibfield  {author} {\bibinfo {author} {\bibnamefont {K{\"u}sters},
  \bibfnamefont {Ralf}}} (\bibinfo {year} {2006}),\ \bibfield  {title}
  {\enquote {\bibinfo {title} {Simulation-based security with inexhaustible
  interactive turing machines},}\ }in\ \href {\doibase 10.1109/CSFW.2006.30}
  {\emph {\bibinfo {booktitle} {Proceedings of the 19th IEEE workshop on
  Computer Security Foundations, CSFW~'06}}}\ (\bibinfo  {publisher} {IEEE})\
  pp.\ \bibinfo {pages} {309--320}\BibitemShut {NoStop}%
\bibitem [{\citenamefont {Laneve}\ and\ \citenamefont {del Rio}(2021)}]{LdR21}%
  \BibitemOpen
  \bibfield  {author} {\bibinfo {author} {\bibnamefont {Laneve}, \bibfnamefont
  {Lorenzo}}, \ and\ \bibinfo {author} {\bibfnamefont {L\'idia}\ \bibnamefont
  {del Rio}}} (\bibinfo {year} {2021}),\ \href@noop {} {\enquote {\bibinfo
  {title} {Impossibility of composable oblivious transfer in relativistic
  quantum cryptography},}\ }\bibinfo {howpublished} {e-Print},\ \Eprint
  {http://arxiv.org/abs/arXiv:2106.11200} {arXiv:2106.11200} \BibitemShut
  {NoStop}%
\bibitem [{\citenamefont {Leverrier}\ \emph {et~al.}(2008)\citenamefont
  {Leverrier}, \citenamefont {All\'eaume}, \citenamefont {Boutros},
  \citenamefont {Z\'emor},\ and\ \citenamefont {Grangier}}]{LABZG08}%
  \BibitemOpen
  \bibfield  {author} {\bibinfo {author} {\bibnamefont {Leverrier},
  \bibfnamefont {Anthony}}, \bibinfo {author} {\bibfnamefont {Romain}\
  \bibnamefont {All\'eaume}}, \bibinfo {author} {\bibfnamefont {Joseph}\
  \bibnamefont {Boutros}}, \bibinfo {author} {\bibfnamefont {Gilles}\
  \bibnamefont {Z\'emor}}, \ and\ \bibinfo {author} {\bibfnamefont {Philippe}\
  \bibnamefont {Grangier}}} (\bibinfo {year} {2008}),\ \bibfield  {title}
  {\enquote {\bibinfo {title} {Multidimensional reconciliation for a
  continuous-variable quantum key distribution},}\ }\href {\doibase
  10.1103/PhysRevA.77.042325} {\bibfield  {journal} {\bibinfo  {journal} {Phys.
  Rev. A}\ }\textbf {\bibinfo {volume} {77}},\ \bibinfo {pages}
  {042325}}\BibitemShut {NoStop}%
\bibitem [{\citenamefont {Lim}\ \emph {et~al.}(2014)\citenamefont {Lim},
  \citenamefont {Curty}, \citenamefont {Walenta}, \citenamefont {Xu},\ and\
  \citenamefont {Zbinden}}]{LCWXZ14}%
  \BibitemOpen
  \bibfield  {author} {\bibinfo {author} {\bibnamefont {Lim}, \bibfnamefont
  {Charles Ci~Wen}}, \bibinfo {author} {\bibfnamefont {Marcos}\ \bibnamefont
  {Curty}}, \bibinfo {author} {\bibfnamefont {Nino}\ \bibnamefont {Walenta}},
  \bibinfo {author} {\bibfnamefont {Feihu}\ \bibnamefont {Xu}}, \ and\ \bibinfo
  {author} {\bibfnamefont {Hugo}\ \bibnamefont {Zbinden}}} (\bibinfo {year}
  {2014}),\ \bibfield  {title} {\enquote {\bibinfo {title} {Concise security
  bounds for practical decoy-state quantum key distribution},}\ }\href
  {\doibase 10.1103/PhysRevA.89.022307} {\bibfield  {journal} {\bibinfo
  {journal} {Phys. Rev. A}\ }\textbf {\bibinfo {volume} {89}},\ \bibinfo
  {pages} {022307}}\BibitemShut {NoStop}%
\bibitem [{\citenamefont {Lim}\ \emph {et~al.}(2013)\citenamefont {Lim},
  \citenamefont {Portmann}, \citenamefont {Tomamichel}, \citenamefont
  {Renner},\ and\ \citenamefont {Gisin}}]{LPTRG13}%
  \BibitemOpen
  \bibfield  {author} {\bibinfo {author} {\bibnamefont {Lim}, \bibfnamefont
  {Charles Ci~Wen}}, \bibinfo {author} {\bibfnamefont {Christopher}\
  \bibnamefont {Portmann}}, \bibinfo {author} {\bibfnamefont {Marco}\
  \bibnamefont {Tomamichel}}, \bibinfo {author} {\bibfnamefont {Renato}\
  \bibnamefont {Renner}}, \ and\ \bibinfo {author} {\bibfnamefont {Nicolas}\
  \bibnamefont {Gisin}}} (\bibinfo {year} {2013}),\ \bibfield  {title}
  {\enquote {\bibinfo {title} {Device-independent quantum key distribution with
  local {Bell} test},}\ }\href {\doibase 10.1103/PhysRevX.3.031006} {\bibfield
  {journal} {\bibinfo  {journal} {Phys. Rev. X}\ }\textbf {\bibinfo {volume}
  {3}},\ \bibinfo {pages} {031006}},\ \Eprint
  {http://arxiv.org/abs/arXiv:1208.0023} {arXiv:1208.0023} \BibitemShut
  {NoStop}%
\bibitem [{\citenamefont {Lipinska}\ \emph {et~al.}(2020)\citenamefont
  {Lipinska}, \citenamefont {Ribeiro},\ and\ \citenamefont {Wehner}}]{LRW20}%
  \BibitemOpen
  \bibfield  {author} {\bibinfo {author} {\bibnamefont {Lipinska},
  \bibfnamefont {Victoria}}, \bibinfo {author} {\bibfnamefont {J\'er\'emy}\
  \bibnamefont {Ribeiro}}, \ and\ \bibinfo {author} {\bibfnamefont {Stephanie}\
  \bibnamefont {Wehner}}} (\bibinfo {year} {2020}),\ \bibfield  {title}
  {\enquote {\bibinfo {title} {Secure multiparty quantum computation with few
  qubits},}\ }\href {\doibase 10.1103/PhysRevA.102.022405} {\bibfield
  {journal} {\bibinfo  {journal} {Phys. Rev. A}\ }\textbf {\bibinfo {volume}
  {102}},\ \bibinfo {pages} {022405}},\ \Eprint
  {http://arxiv.org/abs/arXiv:2004.10486} {arXiv:2004.10486} \BibitemShut
  {NoStop}%
\bibitem [{\citenamefont {Liu}\ \emph {et~al.}(2013)\citenamefont {Liu},
  \citenamefont {Chen}, \citenamefont {Wang}, \citenamefont {Liang},
  \citenamefont {Shentu}, \citenamefont {Wang}, \citenamefont {Cui},
  \citenamefont {Yin}, \citenamefont {Liu}, \citenamefont {Li}, \citenamefont
  {Ma}, \citenamefont {Pelc}, \citenamefont {Fejer}, \citenamefont {Peng},
  \citenamefont {Zhang},\ and\ \citenamefont {Pan}}]{Liu13}%
  \BibitemOpen
  \bibfield  {author} {\bibinfo {author} {\bibnamefont {Liu}, \bibfnamefont
  {Yang}}, \bibinfo {author} {\bibfnamefont {Teng-Yun}\ \bibnamefont {Chen}},
  \bibinfo {author} {\bibfnamefont {Liu-Jun}\ \bibnamefont {Wang}}, \bibinfo
  {author} {\bibfnamefont {Hao}\ \bibnamefont {Liang}}, \bibinfo {author}
  {\bibfnamefont {Guo-Liang}\ \bibnamefont {Shentu}}, \bibinfo {author}
  {\bibfnamefont {Jian}\ \bibnamefont {Wang}}, \bibinfo {author} {\bibfnamefont
  {Ke}~\bibnamefont {Cui}}, \bibinfo {author} {\bibfnamefont {Hua-Lei}\
  \bibnamefont {Yin}}, \bibinfo {author} {\bibfnamefont {Nai-Le}\ \bibnamefont
  {Liu}}, \bibinfo {author} {\bibfnamefont {Li}~\bibnamefont {Li}}, \bibinfo
  {author} {\bibfnamefont {Xiongfeng}\ \bibnamefont {Ma}}, \bibinfo {author}
  {\bibfnamefont {Jason~S.}\ \bibnamefont {Pelc}}, \bibinfo {author}
  {\bibfnamefont {M.~M.}\ \bibnamefont {Fejer}}, \bibinfo {author}
  {\bibfnamefont {Cheng-Zhi}\ \bibnamefont {Peng}}, \bibinfo {author}
  {\bibfnamefont {Qiang}\ \bibnamefont {Zhang}}, \ and\ \bibinfo {author}
  {\bibfnamefont {Jian-Wei}\ \bibnamefont {Pan}}} (\bibinfo {year} {2013}),\
  \bibfield  {title} {\enquote {\bibinfo {title} {Experimental
  measurement-device-independent quantum key distribution},}\ }\href {\doibase
  10.1103/PhysRevLett.111.130502} {\bibfield  {journal} {\bibinfo  {journal}
  {Phys. Rev. Lett.}\ }\textbf {\bibinfo {volume} {111}},\ \bibinfo {pages}
  {130502}},\ \Eprint {http://arxiv.org/abs/arXiv:1209.6178} {arXiv:1209.6178}
  \BibitemShut {NoStop}%
\bibitem [{\citenamefont {Liu}(2014)}]{Liu14}%
  \BibitemOpen
  \bibfield  {author} {\bibinfo {author} {\bibnamefont {Liu}, \bibfnamefont
  {Yi-Kai}}} (\bibinfo {year} {2014}),\ \bibfield  {title} {\enquote {\bibinfo
  {title} {Single-shot security for one-time memories in the isolated qubits
  model},}\ }in\ \href {\doibase 10.1007/978-3-662-44381-1_2} {\emph {\bibinfo
  {booktitle} {Advances in Cryptology -- CRYPTO 2014}}},\ \bibinfo {editor}
  {edited by\ \bibinfo {editor} {\bibfnamefont {Juan~A.}\ \bibnamefont
  {Garay}}\ and\ \bibinfo {editor} {\bibfnamefont {Rosario}\ \bibnamefont
  {Gennaro}}}\ (\bibinfo  {publisher} {Springer})\ pp.\ \bibinfo {pages}
  {19--36},\ \Eprint {http://arxiv.org/abs/arXiv:1402.0049} {arXiv:1402.0049}
  \BibitemShut {NoStop}%
\bibitem [{\citenamefont {Liu}(2015)}]{Liu15}%
  \BibitemOpen
  \bibfield  {author} {\bibinfo {author} {\bibnamefont {Liu}, \bibfnamefont
  {Yi-Kai}}} (\bibinfo {year} {2015}),\ \bibfield  {title} {\enquote {\bibinfo
  {title} {Privacy amplification in the isolated qubits model},}\ }in\ \href
  {\doibase 10.1007/978-3-662-46803-6_26} {\emph {\bibinfo {booktitle}
  {Advances in Cryptology -- EUROCRYPT 2015}}},\ \bibinfo {editor} {edited by\
  \bibinfo {editor} {\bibfnamefont {Elisabeth}\ \bibnamefont {Oswald}}\ and\
  \bibinfo {editor} {\bibfnamefont {Marc}\ \bibnamefont {Fischlin}}}\ (\bibinfo
   {publisher} {Springer})\ pp.\ \bibinfo {pages} {785--814},\ \Eprint
  {http://arxiv.org/abs/arxiv:1410.3918} {arxiv:1410.3918} \BibitemShut
  {NoStop}%
\bibitem [{\citenamefont {Lo}(2003)}]{Lo03}%
  \BibitemOpen
  \bibfield  {author} {\bibinfo {author} {\bibnamefont {Lo}, \bibfnamefont
  {Hoi-Kwong}}} (\bibinfo {year} {2003}),\ \bibfield  {title} {\enquote
  {\bibinfo {title} {Method for decoupling error correction from privacy
  amplification},}\ }\href {\doibase 10.1088/1367-2630/5/1/336} {\bibfield
  {journal} {\bibinfo  {journal} {New J. Phys.}\ }\textbf {\bibinfo {volume}
  {5}},\ \bibinfo {pages} {36--36}}\BibitemShut {NoStop}%
\bibitem [{\citenamefont {Lo}\ and\ \citenamefont {Chau}(1999)}]{LoChau99}%
  \BibitemOpen
  \bibfield  {author} {\bibinfo {author} {\bibnamefont {Lo}, \bibfnamefont
  {Hoi-Kwong}}, \ and\ \bibinfo {author} {\bibfnamefont {Hoi~Fung}\
  \bibnamefont {Chau}}} (\bibinfo {year} {1999}),\ \bibfield  {title} {\enquote
  {\bibinfo {title} {Unconditional security of quantum key distribution over
  arbitrarily long distances},}\ }\href {\doibase
  10.1126/science.283.5410.2050} {\bibfield  {journal} {\bibinfo  {journal}
  {Science}\ }\textbf {\bibinfo {volume} {283}}~(\bibinfo {number} {5410}),\
  \bibinfo {pages} {2050--2056}}\BibitemShut {NoStop}%
\bibitem [{\citenamefont {Lo}\ \emph {et~al.}(2012)\citenamefont {Lo},
  \citenamefont {Curty},\ and\ \citenamefont {Qi}}]{LCQ12}%
  \BibitemOpen
  \bibfield  {author} {\bibinfo {author} {\bibnamefont {Lo}, \bibfnamefont
  {Hoi-Kwong}}, \bibinfo {author} {\bibfnamefont {Marcos}\ \bibnamefont
  {Curty}}, \ and\ \bibinfo {author} {\bibfnamefont {Bing}\ \bibnamefont {Qi}}}
  (\bibinfo {year} {2012}),\ \bibfield  {title} {\enquote {\bibinfo {title}
  {Measurement-device-independent quantum key distribution},}\ }\href {\doibase
  10.1103/PhysRevLett.108.130503} {\bibfield  {journal} {\bibinfo  {journal}
  {Phys. Rev. Lett.}\ }\textbf {\bibinfo {volume} {108}},\ \bibinfo {pages}
  {130503}},\ \Eprint {http://arxiv.org/abs/arXiv:1109.1473} {arXiv:1109.1473}
  \BibitemShut {NoStop}%
\bibitem [{\citenamefont {Lo}\ \emph {et~al.}(2005)\citenamefont {Lo},
  \citenamefont {Ma},\ and\ \citenamefont {Chen}}]{Loetal2005}%
  \BibitemOpen
  \bibfield  {author} {\bibinfo {author} {\bibnamefont {Lo}, \bibfnamefont
  {Hoi-Kwong}}, \bibinfo {author} {\bibfnamefont {Xiongfeng}\ \bibnamefont
  {Ma}}, \ and\ \bibinfo {author} {\bibfnamefont {Kai}\ \bibnamefont {Chen}}}
  (\bibinfo {year} {2005}),\ \bibfield  {title} {\enquote {\bibinfo {title}
  {Decoy state quantum key distribution},}\ }\href {\doibase
  10.1103/PhysRevLett.94.230504} {\bibfield  {journal} {\bibinfo  {journal}
  {Phys. Rev. Lett.}\ }\textbf {\bibinfo {volume} {94}},\ \bibinfo {pages}
  {230504}}\BibitemShut {NoStop}%
\bibitem [{\citenamefont {Lucamarini}\ \emph {et~al.}(2018)\citenamefont
  {Lucamarini}, \citenamefont {Yuan}, \citenamefont {Dynes},\ and\
  \citenamefont {Shields}}]{Lucamarinietal}%
  \BibitemOpen
  \bibfield  {author} {\bibinfo {author} {\bibnamefont {Lucamarini},
  \bibfnamefont {Marco}}, \bibinfo {author} {\bibfnamefont {Zhiliang~L.}\
  \bibnamefont {Yuan}}, \bibinfo {author} {\bibfnamefont {James~F.}\
  \bibnamefont {Dynes}}, \ and\ \bibinfo {author} {\bibfnamefont {Andrew~J.}\
  \bibnamefont {Shields}}} (\bibinfo {year} {2018}),\ \bibfield  {title}
  {\enquote {\bibinfo {title} {Overcoming the rate--distance limit of quantum
  key distribution without quantum repeaters},}\ }\href@noop {} {\bibfield
  {journal} {\bibinfo  {journal} {Nature}\ }\textbf {\bibinfo {volume}
  {557}}~(\bibinfo {number} {7705}),\ \bibinfo {pages} {400--403}}\BibitemShut
  {NoStop}%
\bibitem [{\citenamefont {L\"utkenhaus}(2000)}]{Lutkenhaus2000}%
  \BibitemOpen
  \bibfield  {author} {\bibinfo {author} {\bibnamefont {L\"utkenhaus},
  \bibfnamefont {Norbert}}} (\bibinfo {year} {2000}),\ \bibfield  {title}
  {\enquote {\bibinfo {title} {Security against individual attacks for
  realistic quantum key distribution},}\ }\href {\doibase
  10.1103/PhysRevA.61.052304} {\bibfield  {journal} {\bibinfo  {journal} {Phys.
  Rev. A}\ }\textbf {\bibinfo {volume} {61}},\ \bibinfo {pages}
  {052304}}\BibitemShut {NoStop}%
\bibitem [{\citenamefont {Lydersen}\ \emph {et~al.}(2010)\citenamefont
  {Lydersen}, \citenamefont {Wiechers}, \citenamefont {Wittmann}, \citenamefont
  {Elser}, \citenamefont {Skaar},\ and\ \citenamefont {Makarov}}]{LWWESM10}%
  \BibitemOpen
  \bibfield  {author} {\bibinfo {author} {\bibnamefont {Lydersen},
  \bibfnamefont {Lars}}, \bibinfo {author} {\bibfnamefont {Carlos}\
  \bibnamefont {Wiechers}}, \bibinfo {author} {\bibfnamefont {Christoffer}\
  \bibnamefont {Wittmann}}, \bibinfo {author} {\bibfnamefont {Dominique}\
  \bibnamefont {Elser}}, \bibinfo {author} {\bibfnamefont {Johannes}\
  \bibnamefont {Skaar}}, \ and\ \bibinfo {author} {\bibfnamefont {Vadim}\
  \bibnamefont {Makarov}}} (\bibinfo {year} {2010}),\ \bibfield  {title}
  {\enquote {\bibinfo {title} {Hacking commercial quantum cryptography systems
  by tailored bright illumination},}\ }\href {\doibase
  10.1038/nphoton.2010.214} {\bibfield  {journal} {\bibinfo  {journal} {Nat.
  Photonics}\ }\textbf {\bibinfo {volume} {4}}~(\bibinfo {number} {10}),\
  \bibinfo {pages} {686--689}},\ \Eprint {http://arxiv.org/abs/arXiv:1008.4593}
  {arXiv:1008.4593} \BibitemShut {NoStop}%
\bibitem [{\citenamefont {Ma}\ \emph {et~al.}(2019)\citenamefont {Ma},
  \citenamefont {Zhou}, \citenamefont {Yuan},\ and\ \citenamefont
  {Ma}}]{MZYM19}%
  \BibitemOpen
  \bibfield  {author} {\bibinfo {author} {\bibnamefont {Ma}, \bibfnamefont
  {Jiajun}}, \bibinfo {author} {\bibfnamefont {You}\ \bibnamefont {Zhou}},
  \bibinfo {author} {\bibfnamefont {Xiao}\ \bibnamefont {Yuan}}, \ and\
  \bibinfo {author} {\bibfnamefont {Xiongfeng}\ \bibnamefont {Ma}}} (\bibinfo
  {year} {2019}),\ \bibfield  {title} {\enquote {\bibinfo {title} {Operational
  interpretation of coherence in quantum key distribution},}\ }\href {\doibase
  10.1103/PhysRevA.99.062325} {\bibfield  {journal} {\bibinfo  {journal} {Phys.
  Rev. A}\ }\textbf {\bibinfo {volume} {99}},\ \bibinfo {pages} {062325}},\
  \Eprint {http://arxiv.org/abs/arXiv:1810.03267} {arXiv:1810.03267}
  \BibitemShut {NoStop}%
\bibitem [{\citenamefont {Ma}\ and\ \citenamefont {Razavi}(2012)}]{MR12}%
  \BibitemOpen
  \bibfield  {author} {\bibinfo {author} {\bibnamefont {Ma}, \bibfnamefont
  {Xiongfeng}}, \ and\ \bibinfo {author} {\bibfnamefont {Mohsen}\ \bibnamefont
  {Razavi}}} (\bibinfo {year} {2012}),\ \bibfield  {title} {\enquote {\bibinfo
  {title} {Alternative schemes for measurement-device-independent quantum key
  distribution},}\ }\href {\doibase 10.1103/PhysRevA.86.062319} {\bibfield
  {journal} {\bibinfo  {journal} {Phys. Rev. A}\ }\textbf {\bibinfo {volume}
  {86}},\ \bibinfo {pages} {062319}},\ \Eprint
  {http://arxiv.org/abs/arxiv:1204.4856} {arxiv:1204.4856} \BibitemShut
  {NoStop}%
\bibitem [{\citenamefont {Makarov}(2009)}]{Makarov2009}%
  \BibitemOpen
  \bibfield  {author} {\bibinfo {author} {\bibnamefont {Makarov}, \bibfnamefont
  {Vadim}}} (\bibinfo {year} {2009}),\ \bibfield  {title} {\enquote {\bibinfo
  {title} {Controlling passively quenched single photon detectors by bright
  light},}\ }\href {\doibase 10.1088/1367-2630/11/6/065003} {\bibfield
  {journal} {\bibinfo  {journal} {New J. Phys.}\ }\textbf {\bibinfo {volume}
  {11}}~(\bibinfo {number} {6}),\ \bibinfo {pages} {065003}}\BibitemShut
  {NoStop}%
\bibitem [{\citenamefont {Makarov}\ \emph {et~al.}(2006)\citenamefont
  {Makarov}, \citenamefont {Anisimov},\ and\ \citenamefont
  {Skaar}}]{Makarovetal2006}%
  \BibitemOpen
  \bibfield  {author} {\bibinfo {author} {\bibnamefont {Makarov}, \bibfnamefont
  {Vadim}}, \bibinfo {author} {\bibfnamefont {Andrey}\ \bibnamefont
  {Anisimov}}, \ and\ \bibinfo {author} {\bibfnamefont {Johannes}\ \bibnamefont
  {Skaar}}} (\bibinfo {year} {2006}),\ \bibfield  {title} {\enquote {\bibinfo
  {title} {Effects of detector efficiency mismatch on security of quantum
  cryptosystems},}\ }\href {\doibase 10.1103/PhysRevA.74.022313} {\bibfield
  {journal} {\bibinfo  {journal} {Phys. Rev. A}\ }\textbf {\bibinfo {volume}
  {74}},\ \bibinfo {pages} {022313}}\BibitemShut {NoStop}%
\bibitem [{\citenamefont {Mateus}\ \emph {et~al.}(2003)\citenamefont {Mateus},
  \citenamefont {Mitchell},\ and\ \citenamefont {Scedrov}}]{MMS03}%
  \BibitemOpen
  \bibfield  {author} {\bibinfo {author} {\bibnamefont {Mateus}, \bibfnamefont
  {Paulo}}, \bibinfo {author} {\bibfnamefont {John~C.}\ \bibnamefont
  {Mitchell}}, \ and\ \bibinfo {author} {\bibfnamefont {Andre}\ \bibnamefont
  {Scedrov}}} (\bibinfo {year} {2003}),\ \bibfield  {title} {\enquote {\bibinfo
  {title} {Composition of cryptographic protocols in a probabilistic
  polynomial-time process calculus},}\ }in\ \href {\doibase
  10.1007/978-3-540-45187-7_22} {\emph {\bibinfo {booktitle} {{CONCUR} 2003 --
  Concurrency Theory}}},\ \bibinfo {series} {LNCS}, Vol.\ \bibinfo {volume}
  {2761}\ (\bibinfo  {publisher} {Springer})\ pp.\ \bibinfo {pages}
  {327--349}\BibitemShut {NoStop}%
\bibitem [{\citenamefont {Mauerer}\ \emph {et~al.}(2012)\citenamefont
  {Mauerer}, \citenamefont {Portmann},\ and\ \citenamefont {Scholz}}]{MPS12}%
  \BibitemOpen
  \bibfield  {author} {\bibinfo {author} {\bibnamefont {Mauerer}, \bibfnamefont
  {Wolfgang}}, \bibinfo {author} {\bibfnamefont {Christopher}\ \bibnamefont
  {Portmann}}, \ and\ \bibinfo {author} {\bibfnamefont {Volkher~B.}\
  \bibnamefont {Scholz}}} (\bibinfo {year} {2012}),\ \href@noop {} {\enquote
  {\bibinfo {title} {A modular framework for randomness extraction based on
  trevisan's construction},}\ }\bibinfo {howpublished} {e-Print},\ \Eprint
  {http://arxiv.org/abs/arXiv:1212.0520} {arXiv:1212.0520} \BibitemShut
  {NoStop}%
\bibitem [{\citenamefont {Maurer}(1993)}]{Mau93}%
  \BibitemOpen
  \bibfield  {author} {\bibinfo {author} {\bibnamefont {Maurer}, \bibfnamefont
  {Ueli}}} (\bibinfo {year} {1993}),\ \bibfield  {title} {\enquote {\bibinfo
  {title} {Secret key agreement by public discussion},}\ }\href {\doibase
  10.1109/18.256484} {\bibfield  {journal} {\bibinfo  {journal} {IEEE Trans.
  Inf. Theory}\ }\textbf {\bibinfo {volume} {39}}~(\bibinfo {number} {3}),\
  \bibinfo {pages} {733--742}},\ \bibinfo {note} {a preliminary version
  appeared at CRYPTO~'92}\BibitemShut {NoStop}%
\bibitem [{\citenamefont {Maurer}(1994)}]{Mau94}%
  \BibitemOpen
  \bibfield  {author} {\bibinfo {author} {\bibnamefont {Maurer}, \bibfnamefont
  {Ueli}}} (\bibinfo {year} {1994}),\ \enquote {\bibinfo {title} {The strong
  secret key rate of discrete random triples},}\ in\ \href {\doibase
  10.1007/978-1-4615-2694-0_27} {\emph {\bibinfo {booktitle} {Communications
  and Cryptography: Two Sides of One Tapestry}}},\ \bibinfo {series} {The
  Springer International Series in Engineering and Computer Science}, Vol.\
  \bibinfo {volume} {276}\ (\bibinfo  {publisher} {Springer})\ pp.\ \bibinfo
  {pages} {271--285}\BibitemShut {NoStop}%
\bibitem [{\citenamefont {Maurer}(2002)}]{Mau02}%
  \BibitemOpen
  \bibfield  {author} {\bibinfo {author} {\bibnamefont {Maurer}, \bibfnamefont
  {Ueli}}} (\bibinfo {year} {2002}),\ \bibfield  {title} {\enquote {\bibinfo
  {title} {Indistinguishability of random systems},}\ }in\ \href {\doibase
  10.1007/3-540-46035-7_8} {\emph {\bibinfo {booktitle} {Advances in Cryptology
  -- EUROCRYPT 2002}}},\ \bibinfo {series} {LNCS}, Vol.\ \bibinfo {volume}
  {2332}\ (\bibinfo  {publisher} {Springer})\ pp.\ \bibinfo {pages}
  {110--132}\BibitemShut {NoStop}%
\bibitem [{\citenamefont {Maurer}(2012)}]{Mau12}%
  \BibitemOpen
  \bibfield  {author} {\bibinfo {author} {\bibnamefont {Maurer}, \bibfnamefont
  {Ueli}}} (\bibinfo {year} {2012}),\ \bibfield  {title} {\enquote {\bibinfo
  {title} {Constructive cryptography---a new paradigm for security definitions
  and proofs},}\ }in\ \href {\doibase 10.1007/978-3-642-27375-9_3} {\emph
  {\bibinfo {booktitle} {Proceedings of Theory of Security and Applications,
  TOSCA 2011}}},\ \bibinfo {series} {LNCS}, Vol.\ \bibinfo {volume} {6993}\
  (\bibinfo  {publisher} {Springer})\ pp.\ \bibinfo {pages}
  {33--56}\BibitemShut {NoStop}%
\bibitem [{\citenamefont {Maurer}\ \emph {et~al.}(2007)\citenamefont {Maurer},
  \citenamefont {Pietrzak},\ and\ \citenamefont {Renner}}]{MPR07}%
  \BibitemOpen
  \bibfield  {author} {\bibinfo {author} {\bibnamefont {Maurer}, \bibfnamefont
  {Ueli}}, \bibinfo {author} {\bibfnamefont {Krzysztof}\ \bibnamefont
  {Pietrzak}}, \ and\ \bibinfo {author} {\bibfnamefont {Renato}\ \bibnamefont
  {Renner}}} (\bibinfo {year} {2007}),\ \bibfield  {title} {\enquote {\bibinfo
  {title} {Indistinguishability amplification},}\ }in\ \href {\doibase
  10.1007/978-3-540-74143-5_8} {\emph {\bibinfo {booktitle} {Advances in
  Cryptology -- CRYPTO 2007}}},\ \bibinfo {series} {LNCS}, Vol.\ \bibinfo
  {volume} {4622}\ (\bibinfo  {publisher} {Springer})\ pp.\ \bibinfo {pages}
  {130--149}\BibitemShut {NoStop}%
\bibitem [{\citenamefont {Maurer}\ and\ \citenamefont {Renner}(2011)}]{MR11}%
  \BibitemOpen
  \bibfield  {author} {\bibinfo {author} {\bibnamefont {Maurer}, \bibfnamefont
  {Ueli}}, \ and\ \bibinfo {author} {\bibfnamefont {Renato}\ \bibnamefont
  {Renner}}} (\bibinfo {year} {2011}),\ \bibfield  {title} {\enquote {\bibinfo
  {title} {Abstract cryptography},}\ }in\ \href@noop {} {\emph {\bibinfo
  {booktitle} {Proceedings of Innovations in Computer Science, ICS 2011}}}\
  (\bibinfo  {publisher} {Tsinghua University Press})\ pp.\ \bibinfo {pages}
  {1--21}\BibitemShut {NoStop}%
\bibitem [{\citenamefont {Maurer}\ and\ \citenamefont {Renner}(2016)}]{MR16}%
  \BibitemOpen
  \bibfield  {author} {\bibinfo {author} {\bibnamefont {Maurer}, \bibfnamefont
  {Ueli}}, \ and\ \bibinfo {author} {\bibfnamefont {Renato}\ \bibnamefont
  {Renner}}} (\bibinfo {year} {2016}),\ \bibfield  {title} {\enquote {\bibinfo
  {title} {From indifferentiability to constructive cryptography (and back)},}\
  }in\ \href {\doibase 10.1007/978-3-662-53641-4_1} {\emph {\bibinfo
  {booktitle} {Theory of Cryptography, Proceedings of {TCC} 2016-B, Part
  {I}}}},\ \bibinfo {series} {LNCS}, Vol.\ \bibinfo {volume} {9985}\ (\bibinfo
  {publisher} {Springer})\ pp.\ \bibinfo {pages} {3--24},\ \bibinfo {note}
  {e-Print \href{http://eprint.iacr.org/2016/903}{IACR 2016/903}}\BibitemShut
  {NoStop}%
\bibitem [{\citenamefont {Maurer}\ \emph {et~al.}(2012)\citenamefont {Maurer},
  \citenamefont {R\"uedlinger},\ and\ \citenamefont {Tackmann}}]{MRT12}%
  \BibitemOpen
  \bibfield  {author} {\bibinfo {author} {\bibnamefont {Maurer}, \bibfnamefont
  {Ueli}}, \bibinfo {author} {\bibfnamefont {Andreas}\ \bibnamefont
  {R\"uedlinger}}, \ and\ \bibinfo {author} {\bibfnamefont {Bj\"orn}\
  \bibnamefont {Tackmann}}} (\bibinfo {year} {2012}),\ \bibfield  {title}
  {\enquote {\bibinfo {title} {Confidentiality and integrity: A constructive
  perspective},}\ }in\ \href {\doibase 10.1007/978-3-642-28914-9_12} {\emph
  {\bibinfo {booktitle} {Theory of Cryptography, Proceedings of TCC 2012}}},\
  \bibinfo {series} {LNCS}, Vol.\ \bibinfo {volume} {7194},\ \bibinfo {editor}
  {edited by\ \bibinfo {editor} {\bibfnamefont {Ronald}\ \bibnamefont
  {Cramer}}}\ (\bibinfo  {publisher} {Springer})\ pp.\ \bibinfo {pages}
  {209--229}\BibitemShut {NoStop}%
\bibitem [{\citenamefont {Maurer}\ and\ \citenamefont {Wolf}(2000)}]{MW00}%
  \BibitemOpen
  \bibfield  {author} {\bibinfo {author} {\bibnamefont {Maurer}, \bibfnamefont
  {Ueli}}, \ and\ \bibinfo {author} {\bibfnamefont {Stefan}\ \bibnamefont
  {Wolf}}} (\bibinfo {year} {2000}),\ \bibfield  {title} {\enquote {\bibinfo
  {title} {Information-theoretic key agreement: From weak to strong secrecy for
  free},}\ }in\ \href {\doibase 10.1007/3-540-45539-6_24} {\emph {\bibinfo
  {booktitle} {Advances in Cryptology -- EUROCRYPT 2000}}},\ \bibinfo {series}
  {LNCS}, Vol.\ \bibinfo {volume} {1807}\ (\bibinfo  {publisher} {Springer})\
  pp.\ \bibinfo {pages} {351--368}\BibitemShut {NoStop}%
\bibitem [{\citenamefont {Mayers}(1996)}]{May96}%
  \BibitemOpen
  \bibfield  {author} {\bibinfo {author} {\bibnamefont {Mayers}, \bibfnamefont
  {Dominic}}} (\bibinfo {year} {1996}),\ \bibfield  {title} {\enquote {\bibinfo
  {title} {Quantum key distribution and string oblivious transfer in noisy
  channels},}\ }in\ \href {\doibase 10.1007/3-540-68697-5_26} {\emph {\bibinfo
  {booktitle} {Advances in Cryptology -- CRYPTO~'96}}},\ \bibinfo {series}
  {LNCS}, Vol.\ \bibinfo {volume} {1109}\ (\bibinfo  {publisher} {Springer})\
  pp.\ \bibinfo {pages} {343--357},\ \Eprint
  {http://arxiv.org/abs/arXiv:quant-ph/9606003} {arXiv:quant-ph/9606003}
  \BibitemShut {NoStop}%
\bibitem [{\citenamefont {Mayers}(2001)}]{May01}%
  \BibitemOpen
  \bibfield  {author} {\bibinfo {author} {\bibnamefont {Mayers}, \bibfnamefont
  {Dominic}}} (\bibinfo {year} {2001}),\ \bibfield  {title} {\enquote {\bibinfo
  {title} {Unconditional security in quantum cryptography},}\ }\href {\doibase
  10.1145/382780.382781} {\bibfield  {journal} {\bibinfo  {journal} {J. ACM}\
  }\textbf {\bibinfo {volume} {48}}~(\bibinfo {number} {3}),\ \bibinfo {pages}
  {351--406}},\ \Eprint {http://arxiv.org/abs/arXiv:quant-ph/9802025}
  {arXiv:quant-ph/9802025} \BibitemShut {NoStop}%
\bibitem [{\citenamefont {Micali}\ and\ \citenamefont {Rogaway}(1992)}]{MR92}%
  \BibitemOpen
  \bibfield  {author} {\bibinfo {author} {\bibnamefont {Micali}, \bibfnamefont
  {Silvio}}, \ and\ \bibinfo {author} {\bibfnamefont {Phillip}\ \bibnamefont
  {Rogaway}}} (\bibinfo {year} {1992}),\ \bibfield  {title} {\enquote {\bibinfo
  {title} {Secure computation (abstract)},}\ }in\ \href {\doibase
  10.1007/3-540-46766-1_32} {\emph {\bibinfo {booktitle} {Advances in
  Cryptology -- CRYPTO~'91}}},\ \bibinfo {series} {LNCS}, Vol.\ \bibinfo
  {volume} {576}\ (\bibinfo  {publisher} {Springer})\ pp.\ \bibinfo {pages}
  {392--404}\BibitemShut {NoStop}%
\bibitem [{\citenamefont {Miller}\ and\ \citenamefont {Shi}(2014)}]{MS14}%
  \BibitemOpen
  \bibfield  {author} {\bibinfo {author} {\bibnamefont {Miller}, \bibfnamefont
  {Carl}}, \ and\ \bibinfo {author} {\bibfnamefont {Yaoyun}\ \bibnamefont
  {Shi}}} (\bibinfo {year} {2014}),\ \bibfield  {title} {\enquote {\bibinfo
  {title} {Robust protocols for securely expanding randomness and distributing
  keys using untrusted quantum devices},}\ }in\ \href {\doibase
  10.1145/2591796.2591843} {\emph {\bibinfo {booktitle} {Proceedings of the
  46th Symposium on Theory of Computing, STOC~'14}}}\ (\bibinfo  {publisher}
  {ACM})\ pp.\ \bibinfo {pages} {417--426},\ \Eprint
  {http://arxiv.org/abs/arXiv:1402.0489} {arXiv:1402.0489} \BibitemShut
  {NoStop}%
\bibitem [{\citenamefont {Mitchell}\ \emph {et~al.}(2006)\citenamefont
  {Mitchell}, \citenamefont {Ramanathan}, \citenamefont {Scedrov},\ and\
  \citenamefont {Teague}}]{MRST06}%
  \BibitemOpen
  \bibfield  {author} {\bibinfo {author} {\bibnamefont {Mitchell},
  \bibfnamefont {John~C}}, \bibinfo {author} {\bibfnamefont {Ajith}\
  \bibnamefont {Ramanathan}}, \bibinfo {author} {\bibfnamefont {Andre}\
  \bibnamefont {Scedrov}}, \ and\ \bibinfo {author} {\bibfnamefont {Vanessa}\
  \bibnamefont {Teague}}} (\bibinfo {year} {2006}),\ \bibfield  {title}
  {\enquote {\bibinfo {title} {A probabilistic polynomial-time process calculus
  for the analysis of cryptographic protocols},}\ }\href {\doibase
  10.1016/j.tcs.2005.10.044} {\bibfield  {journal} {\bibinfo  {journal} {Theor.
  Comput. Sci.}\ }\textbf {\bibinfo {volume} {353}}~(\bibinfo {number}
  {1–3}),\ \bibinfo {pages} {118--164}}\BibitemShut {NoStop}%
\bibitem [{\citenamefont {Muller}\ \emph {et~al.}(1997)\citenamefont {Muller},
  \citenamefont {Herzog}, \citenamefont {Huttner}, \citenamefont {Tittel},
  \citenamefont {Zbinden},\ and\ \citenamefont {Gisin}}]{Muller97}%
  \BibitemOpen
  \bibfield  {author} {\bibinfo {author} {\bibnamefont {Muller}, \bibfnamefont
  {Antoine}}, \bibinfo {author} {\bibfnamefont {Thomas}\ \bibnamefont
  {Herzog}}, \bibinfo {author} {\bibfnamefont {Bruno}\ \bibnamefont {Huttner}},
  \bibinfo {author} {\bibfnamefont {Woflgang}\ \bibnamefont {Tittel}}, \bibinfo
  {author} {\bibfnamefont {Hugo}\ \bibnamefont {Zbinden}}, \ and\ \bibinfo
  {author} {\bibfnamefont {Nicolas}\ \bibnamefont {Gisin}}} (\bibinfo {year}
  {1997}),\ \bibfield  {title} {\enquote {\bibinfo {title} {``plug and play''
  systems for quantum cryptography},}\ }\href {\doibase
  https://doi.org/10.1063/1.118224} {\bibfield  {journal} {\bibinfo  {journal}
  {Appl. Phys. Lett.}\ }\textbf {\bibinfo {volume} {70}}~(\bibinfo {number}
  {7}),\ \bibinfo {pages} {793--795}}\BibitemShut {NoStop}%
\bibitem [{\citenamefont {M\"uller-Quade}\ and\ \citenamefont
  {Renner}(2009)}]{MR09}%
  \BibitemOpen
  \bibfield  {author} {\bibinfo {author} {\bibnamefont {M\"uller-Quade},
  \bibfnamefont {J\"orn}}, \ and\ \bibinfo {author} {\bibfnamefont {Renato}\
  \bibnamefont {Renner}}} (\bibinfo {year} {2009}),\ \bibfield  {title}
  {\enquote {\bibinfo {title} {Composability in quantum cryptography},}\ }\href
  {\doibase 10.1088/1367-2630/11/8/085006} {\bibfield  {journal} {\bibinfo
  {journal} {New J. Phys.}\ }\textbf {\bibinfo {volume} {11}}~(\bibinfo
  {number} {8}),\ \bibinfo {pages} {085006}},\ \Eprint
  {http://arxiv.org/abs/arXiv:1006.2215} {arXiv:1006.2215} \BibitemShut
  {NoStop}%
\bibitem [{\citenamefont {Nielsen}\ and\ \citenamefont
  {Chuang}(2010)}]{nielsen2010quantum}%
  \BibitemOpen
  \bibfield  {author} {\bibinfo {author} {\bibnamefont {Nielsen}, \bibfnamefont
  {Michael~A}}, \ and\ \bibinfo {author} {\bibfnamefont {Isaac~L}\ \bibnamefont
  {Chuang}}} (\bibinfo {year} {2010}),\ \href@noop {} {\emph {\bibinfo {title}
  {Quantum Computation and Quantum Information}}}\ (\bibinfo  {publisher}
  {Cambridge University Press})\BibitemShut {NoStop}%
\bibitem [{\citenamefont {Nishioka}\ \emph {et~al.}(2002)\citenamefont
  {Nishioka}, \citenamefont {Ishizuka}, \citenamefont {Toshio},\ and\
  \citenamefont {Abe}}]{Nishioka2002}%
  \BibitemOpen
  \bibfield  {author} {\bibinfo {author} {\bibnamefont {Nishioka},
  \bibfnamefont {Tsuyoshi}}, \bibinfo {author} {\bibfnamefont {Hirokazu}\
  \bibnamefont {Ishizuka}}, \bibinfo {author} {\bibnamefont {Toshio}}, \ and\
  \bibinfo {author} {\bibfnamefont {Junichi}\ \bibnamefont {Abe}}} (\bibinfo
  {year} {2002}),\ \bibfield  {title} {\enquote {\bibinfo {title} {``circular
  type'' quantum key distribution},}\ }\href {\doibase 10.1109/68.992616}
  {\bibfield  {journal} {\bibinfo  {journal} {IEEE Photonics Technol. Lett.}\
  }\textbf {\bibinfo {volume} {14}}~(\bibinfo {number} {4}),\ \bibinfo {pages}
  {576--578}},\ \Eprint {http://arxiv.org/abs/arXiv:quant-ph/0106083}
  {arXiv:quant-ph/0106083} \BibitemShut {NoStop}%
\bibitem [{\citenamefont {Ohya}\ and\ \citenamefont {Petz}(1993)}]{OP93}%
  \BibitemOpen
  \bibfield  {author} {\bibinfo {author} {\bibnamefont {Ohya}, \bibfnamefont
  {Masanori}}, \ and\ \bibinfo {author} {\bibfnamefont {D\'enes}\ \bibnamefont
  {Petz}}} (\bibinfo {year} {1993}),\ \href@noop {} {\emph {\bibinfo {title}
  {Quantum Entropy and Its Use}}}\ (\bibinfo  {publisher}
  {Springer})\BibitemShut {NoStop}%
\bibitem [{\citenamefont {Paw\l{}owski}\ and\ \citenamefont
  {Brunner}(2011)}]{PB11}%
  \BibitemOpen
  \bibfield  {author} {\bibinfo {author} {\bibnamefont {Paw\l{}owski},
  \bibfnamefont {Marcin}}, \ and\ \bibinfo {author} {\bibfnamefont {Nicolas}\
  \bibnamefont {Brunner}}} (\bibinfo {year} {2011}),\ \bibfield  {title}
  {\enquote {\bibinfo {title} {Semi-device-independent security of one-way
  quantum key distribution},}\ }\href {\doibase 10.1103/PhysRevA.84.010302}
  {\bibfield  {journal} {\bibinfo  {journal} {Phys. Rev. A}\ }\textbf {\bibinfo
  {volume} {84}},\ \bibinfo {pages} {010302}},\ \Eprint
  {http://arxiv.org/abs/arXiv:1103.4105} {arXiv:1103.4105} \BibitemShut
  {NoStop}%
\bibitem [{\citenamefont {Peres}\ and\ \citenamefont
  {Terno}(2004)}]{Peres_Terno_RMP}%
  \BibitemOpen
  \bibfield  {author} {\bibinfo {author} {\bibnamefont {Peres}, \bibfnamefont
  {Asher}}, \ and\ \bibinfo {author} {\bibfnamefont {Daniel~R.}\ \bibnamefont
  {Terno}}} (\bibinfo {year} {2004}),\ \bibfield  {title} {\enquote {\bibinfo
  {title} {Quantum information and relativity theory},}\ }\href {\doibase
  10.1103/RevModPhys.76.93} {\bibfield  {journal} {\bibinfo  {journal} {Rev.
  Mod. Phys.}\ }\textbf {\bibinfo {volume} {76}},\ \bibinfo {pages}
  {93--123}}\BibitemShut {NoStop}%
\bibitem [{\citenamefont {Pfitzmann}\ and\ \citenamefont
  {Waidner}(2000)}]{PW00}%
  \BibitemOpen
  \bibfield  {author} {\bibinfo {author} {\bibnamefont {Pfitzmann},
  \bibfnamefont {Birgit}}, \ and\ \bibinfo {author} {\bibfnamefont {Michael}\
  \bibnamefont {Waidner}}} (\bibinfo {year} {2000}),\ \bibfield  {title}
  {\enquote {\bibinfo {title} {Composition and integrity preservation of secure
  reactive systems},}\ }in\ \href {\doibase 10.1145/352600.352639} {\emph
  {\bibinfo {booktitle} {Proceedings of the 7th ACM Conference on Computer and
  Communications Security, CSS~'00}}}\ (\bibinfo  {publisher} {ACM})\ pp.\
  \bibinfo {pages} {245--254}\BibitemShut {NoStop}%
\bibitem [{\citenamefont {Pfitzmann}\ and\ \citenamefont
  {Waidner}(2001)}]{PW01}%
  \BibitemOpen
  \bibfield  {author} {\bibinfo {author} {\bibnamefont {Pfitzmann},
  \bibfnamefont {Birgit}}, \ and\ \bibinfo {author} {\bibfnamefont {Michael}\
  \bibnamefont {Waidner}}} (\bibinfo {year} {2001}),\ \bibfield  {title}
  {\enquote {\bibinfo {title} {A model for asynchronous reactive systems and
  its application to secure message transmission},}\ }in\ \href {\doibase
  10.1109/SECPRI.2001.924298} {\emph {\bibinfo {booktitle} {IEEE Symposium on
  Security and Privacy}}}\ (\bibinfo  {publisher} {IEEE})\ pp.\ \bibinfo
  {pages} {184--200}\BibitemShut {NoStop}%
\bibitem [{\citenamefont {Pirandola}\ \emph {et~al.}(2015)\citenamefont
  {Pirandola}, \citenamefont {Ottaviani}, \citenamefont {Spedalieri},
  \citenamefont {Weedbrook}, \citenamefont {Braunstein}, \citenamefont {Lloyd},
  \citenamefont {Gehring}, \citenamefont {Jacobsen},\ and\ \citenamefont
  {Andersen}}]{Pirandola2015}%
  \BibitemOpen
  \bibfield  {author} {\bibinfo {author} {\bibnamefont {Pirandola},
  \bibfnamefont {Stefano}}, \bibinfo {author} {\bibfnamefont {Carlo}\
  \bibnamefont {Ottaviani}}, \bibinfo {author} {\bibfnamefont {Gaetana}\
  \bibnamefont {Spedalieri}}, \bibinfo {author} {\bibfnamefont {Christian}\
  \bibnamefont {Weedbrook}}, \bibinfo {author} {\bibfnamefont {Samuel~L.}\
  \bibnamefont {Braunstein}}, \bibinfo {author} {\bibfnamefont {Seth}\
  \bibnamefont {Lloyd}}, \bibinfo {author} {\bibfnamefont {Tobias}\
  \bibnamefont {Gehring}}, \bibinfo {author} {\bibfnamefont {Christian~S.}\
  \bibnamefont {Jacobsen}}, \ and\ \bibinfo {author} {\bibfnamefont {Ulrik~L.}\
  \bibnamefont {Andersen}}} (\bibinfo {year} {2015}),\ \bibfield  {title}
  {\enquote {\bibinfo {title} {High-rate measurement-device-independent quantum
  cryptography},}\ }\href {\doibase 10.1038/nphoton.2015.83} {\bibfield
  {journal} {\bibinfo  {journal} {Nat. Photonics}\ }\textbf {\bibinfo {volume}
  {9}},\ \bibinfo {pages} {397}}\BibitemShut {NoStop}%
\bibitem [{\citenamefont {Pironio}\ \emph {et~al.}(2009)\citenamefont
  {Pironio}, \citenamefont {Ac\'in}, \citenamefont {Brunner}, \citenamefont
  {Gisin}, \citenamefont {Massar},\ and\ \citenamefont {Scarani}}]{PABGMS09}%
  \BibitemOpen
  \bibfield  {author} {\bibinfo {author} {\bibnamefont {Pironio}, \bibfnamefont
  {Stefano}}, \bibinfo {author} {\bibfnamefont {Antonio}\ \bibnamefont
  {Ac\'in}}, \bibinfo {author} {\bibfnamefont {Nicolas}\ \bibnamefont
  {Brunner}}, \bibinfo {author} {\bibfnamefont {Nicolas}\ \bibnamefont
  {Gisin}}, \bibinfo {author} {\bibfnamefont {Serge}\ \bibnamefont {Massar}}, \
  and\ \bibinfo {author} {\bibfnamefont {Valerio}\ \bibnamefont {Scarani}}}
  (\bibinfo {year} {2009}),\ \bibfield  {title} {\enquote {\bibinfo {title}
  {Device-independent quantum key distribution secure against collective
  attacks},}\ }\href {\doibase 10.1088/1367-2630/11/4/045021} {\bibfield
  {journal} {\bibinfo  {journal} {New J. Phys.}\ }\textbf {\bibinfo {volume}
  {11}}~(\bibinfo {number} {4}),\ \bibinfo {pages} {045021}},\ \Eprint
  {http://arxiv.org/abs/arXiv:0903.4460} {arXiv:0903.4460} \BibitemShut
  {NoStop}%
\bibitem [{\citenamefont {Pironio}\ \emph {et~al.}(2010)\citenamefont
  {Pironio}, \citenamefont {Ac\'in}, \citenamefont {Massar}, \citenamefont
  {de~La~Giroday}, \citenamefont {Matsukevich}, \citenamefont {Maunz},
  \citenamefont {Olmschenk}, \citenamefont {Hayes}, \citenamefont {Luo},\ and\
  \citenamefont {Manning}}]{PAM10}%
  \BibitemOpen
  \bibfield  {author} {\bibinfo {author} {\bibnamefont {Pironio}, \bibfnamefont
  {Stefano}}, \bibinfo {author} {\bibfnamefont {Antonio}\ \bibnamefont
  {Ac\'in}}, \bibinfo {author} {\bibfnamefont {Serge}\ \bibnamefont {Massar}},
  \bibinfo {author} {\bibfnamefont {A~Boyer}\ \bibnamefont {de~La~Giroday}},
  \bibinfo {author} {\bibfnamefont {Dzimitry~N}\ \bibnamefont {Matsukevich}},
  \bibinfo {author} {\bibfnamefont {Peter}\ \bibnamefont {Maunz}}, \bibinfo
  {author} {\bibfnamefont {Steven}\ \bibnamefont {Olmschenk}}, \bibinfo
  {author} {\bibfnamefont {David}\ \bibnamefont {Hayes}}, \bibinfo {author}
  {\bibfnamefont {Le}~\bibnamefont {Luo}}, \ and\ \bibinfo {author}
  {\bibfnamefont {T~Andrew}\ \bibnamefont {Manning}}} (\bibinfo {year}
  {2010}),\ \bibfield  {title} {\enquote {\bibinfo {title} {Random numbers
  certified by {Bell}'s theorem},}\ }\href {\doibase 10.1038/nature09008}
  {\bibfield  {journal} {\bibinfo  {journal} {Nature}\ }\textbf {\bibinfo
  {volume} {464}}~(\bibinfo {number} {7291}),\ \bibinfo {pages} {1021--1024}},\
  \Eprint {http://arxiv.org/abs/arXiv:0911.3427} {arXiv:0911.3427} \BibitemShut
  {NoStop}%
\bibitem [{\citenamefont {Portmann}(2014)}]{Por14}%
  \BibitemOpen
  \bibfield  {author} {\bibinfo {author} {\bibnamefont {Portmann},
  \bibfnamefont {Christopher}}} (\bibinfo {year} {2014}),\ \bibfield  {title}
  {\enquote {\bibinfo {title} {Key recycling in authentication},}\ }\href
  {\doibase 10.1109/TIT.2014.2317312} {\bibfield  {journal} {\bibinfo
  {journal} {IEEE Trans. Inf. Theory}\ }\textbf {\bibinfo {volume}
  {60}}~(\bibinfo {number} {7}),\ \bibinfo {pages} {4383--4396}},\ \Eprint
  {http://arxiv.org/abs/arXiv:1202.1229} {arXiv:1202.1229} \BibitemShut
  {NoStop}%
\bibitem [{\citenamefont {Portmann}(2017{\natexlab{a}})}]{Por17}%
  \BibitemOpen
  \bibfield  {author} {\bibinfo {author} {\bibnamefont {Portmann},
  \bibfnamefont {Christopher}}} (\bibinfo {year} {2017}{\natexlab{a}}),\
  \bibfield  {title} {\enquote {\bibinfo {title} {Quantum authentication with
  key recycling},}\ }in\ \href {\doibase 10.1007/978-3-319-56617-7_12} {\emph
  {\bibinfo {booktitle} {Advances in Cryptology -- {EUROCRYPT} 2017,
  Proceedings, Part {III}}}},\ \bibinfo {series} {LNCS}, Vol.\ \bibinfo
  {volume} {10212}\ (\bibinfo  {publisher} {Springer})\ pp.\ \bibinfo {pages}
  {339--368},\ \Eprint {http://arxiv.org/abs/arXiv:1610.03422}
  {arXiv:1610.03422} \BibitemShut {NoStop}%
\bibitem [{\citenamefont {Portmann}(2017{\natexlab{b}})}]{Portmann2017}%
  \BibitemOpen
  \bibfield  {author} {\bibinfo {author} {\bibnamefont {Portmann},
  \bibfnamefont {Christopher}}} (\bibinfo {year} {2017}{\natexlab{b}}),\
  \href@noop {} {\enquote {\bibinfo {title} {({Quantum}) {Min}-entropy
  resources},}\ }\bibinfo {howpublished} {e-Print},\ \Eprint
  {http://arxiv.org/abs/arXiv:1705.10595} {arXiv:1705.10595} \BibitemShut
  {NoStop}%
\bibitem [{\citenamefont {Portmann}\ \emph {et~al.}(2017)\citenamefont
  {Portmann}, \citenamefont {Matt}, \citenamefont {Maurer}, \citenamefont
  {Renner},\ and\ \citenamefont {Tackmann}}]{PMMRT17}%
  \BibitemOpen
  \bibfield  {author} {\bibinfo {author} {\bibnamefont {Portmann},
  \bibfnamefont {Christopher}}, \bibinfo {author} {\bibfnamefont {Christian}\
  \bibnamefont {Matt}}, \bibinfo {author} {\bibfnamefont {Ueli}\ \bibnamefont
  {Maurer}}, \bibinfo {author} {\bibfnamefont {Renato}\ \bibnamefont {Renner}},
  \ and\ \bibinfo {author} {\bibfnamefont {Bj\"orn}\ \bibnamefont {Tackmann}}}
  (\bibinfo {year} {2017}),\ \bibfield  {title} {\enquote {\bibinfo {title}
  {Causal boxes: Quantum information-processing systems closed under
  composition},}\ }\href {\doibase 10.1109/TIT.2017.2676805} {\bibfield
  {journal} {\bibinfo  {journal} {IEEE Transactions on Information Theory}\
  }\textbf {\bibinfo {volume} {63}}~(\bibinfo {number} {5}),\ \bibinfo {pages}
  {3277--3305}},\ \Eprint {http://arxiv.org/abs/arXiv:1512.02240}
  {arXiv:1512.02240} \BibitemShut {NoStop}%
\bibitem [{\citenamefont {Prokop}(2020)}]{Pro20}%
  \BibitemOpen
  \bibfield  {author} {\bibinfo {author} {\bibnamefont {Prokop}, \bibfnamefont
  {Mil\v{o}s}}} (\bibinfo {year} {2020}),\ \href
  {https://project-archive.inf.ed.ac.uk/ug4/20201685/ug4_proj.pdf} {\enquote
  {\bibinfo {title} {Composable security of quantum bit commitment protocol},}\
  }\bibinfo {howpublished} {e-print}\BibitemShut {NoStop}%
\bibitem [{\citenamefont {Qi}\ \emph {et~al.}(2007)\citenamefont {Qi},
  \citenamefont {Fung}, \citenamefont {Lo},\ and\ \citenamefont
  {Ma}}]{qi2007time}%
  \BibitemOpen
  \bibfield  {author} {\bibinfo {author} {\bibnamefont {Qi}, \bibfnamefont
  {Bing}}, \bibinfo {author} {\bibfnamefont {Chi-Hang~Fred}\ \bibnamefont
  {Fung}}, \bibinfo {author} {\bibfnamefont {Hoi-Kwong}\ \bibnamefont {Lo}}, \
  and\ \bibinfo {author} {\bibfnamefont {Xiongfeng}\ \bibnamefont {Ma}}}
  (\bibinfo {year} {2007}),\ \bibfield  {title} {\enquote {\bibinfo {title}
  {Time-shift attack in practical quantum cryptosystems},}\ }\href@noop {}
  {\bibfield  {journal} {\bibinfo  {journal} {Quantum Inf. Comput.}\ }\textbf
  {\bibinfo {volume} {7}}~(\bibinfo {number} {1}),\ \bibinfo {pages}
  {73--82}}\BibitemShut {NoStop}%
\bibitem [{\citenamefont {Reichardt}\ \emph {et~al.}(2013)\citenamefont
  {Reichardt}, \citenamefont {Unger},\ and\ \citenamefont {Vazirani}}]{RUV13}%
  \BibitemOpen
  \bibfield  {author} {\bibinfo {author} {\bibnamefont {Reichardt},
  \bibfnamefont {Ben~W}}, \bibinfo {author} {\bibfnamefont {Falk}\ \bibnamefont
  {Unger}}, \ and\ \bibinfo {author} {\bibfnamefont {Umesh}\ \bibnamefont
  {Vazirani}}} (\bibinfo {year} {2013}),\ \bibfield  {title} {\enquote
  {\bibinfo {title} {Classical command of quantum systems},}\ }\href {\doibase
  10.1038/nature12035} {\bibfield  {journal} {\bibinfo  {journal} {Nature}\
  }\textbf {\bibinfo {volume} {496}},\ \bibinfo {pages} {456--460}},\ \bibinfo
  {note} {full version available on arXiv},\ \Eprint
  {http://arxiv.org/abs/arXiv:1209.0448} {arXiv:1209.0448} \BibitemShut
  {NoStop}%
\bibitem [{\citenamefont {Renes}(2013)}]{Renes2013}%
  \BibitemOpen
  \bibfield  {author} {\bibinfo {author} {\bibnamefont {Renes}, \bibfnamefont
  {Joseph~M}}} (\bibinfo {year} {2013}),\ \bibfield  {title} {\enquote
  {\bibinfo {title} {The physics of quantum information: Complementarity,
  uncertainty, and entanglement},}\ }\href@noop {} {\bibfield  {journal}
  {\bibinfo  {journal} {Int. J. Quantum Inf.}\ }\textbf {\bibinfo {volume}
  {11}}~(\bibinfo {number} {08}),\ \bibinfo {pages} {1330002}}\BibitemShut
  {NoStop}%
\bibitem [{\citenamefont {Renes}\ and\ \citenamefont
  {Renner}(2012)}]{RenRen12}%
  \BibitemOpen
  \bibfield  {author} {\bibinfo {author} {\bibnamefont {Renes}, \bibfnamefont
  {Joseph~M}}, \ and\ \bibinfo {author} {\bibfnamefont {Renato}\ \bibnamefont
  {Renner}}} (\bibinfo {year} {2012}),\ \bibfield  {title} {\enquote {\bibinfo
  {title} {One-shot classical data compression with quantum side information
  and the distillation of common randomness or secret keys},}\ }\href {\doibase
  10.1109/TIT.2011.2177589} {\bibfield  {journal} {\bibinfo  {journal} {IEEE
  Trans. Inf. Theory}\ }\textbf {\bibinfo {volume} {58}}~(\bibinfo {number}
  {3}),\ \bibinfo {pages} {1985--1991}}\BibitemShut {NoStop}%
\bibitem [{\citenamefont {Renes}\ and\ \citenamefont
  {Renner}(2020)}]{RenesRenner2020}%
  \BibitemOpen
  \bibfield  {author} {\bibinfo {author} {\bibnamefont {Renes}, \bibfnamefont
  {Joseph~M}}, \ and\ \bibinfo {author} {\bibfnamefont {Renato}\ \bibnamefont
  {Renner}}} (\bibinfo {year} {2020}),\ \href@noop {} {\enquote {\bibinfo
  {title} {Are quantum cryptographic security claims vacuous?}}\ }\bibinfo
  {howpublished} {e-Print},\ \Eprint {http://arxiv.org/abs/arXiv:2010.11961}
  {arXiv:2010.11961} \BibitemShut {NoStop}%
\bibitem [{\citenamefont {Renner}(2005)}]{Ren05}%
  \BibitemOpen
  \bibfield  {author} {\bibinfo {author} {\bibnamefont {Renner}, \bibfnamefont
  {Renato}}} (\bibinfo {year} {2005}),\ \emph {\bibinfo {title} {Security of
  Quantum Key Distribution}},\ \href@noop {} {Ph.D. thesis}\ (\bibinfo
  {school} {Swiss Federal Institute of Technology (ETH) Zurich}),\ \Eprint
  {http://arxiv.org/abs/arXiv:quant-ph/0512258} {arXiv:quant-ph/0512258}
  \BibitemShut {NoStop}%
\bibitem [{\citenamefont {Renner}(2007)}]{Ren07}%
  \BibitemOpen
  \bibfield  {author} {\bibinfo {author} {\bibnamefont {Renner}, \bibfnamefont
  {Renato}}} (\bibinfo {year} {2007}),\ \bibfield  {title} {\enquote {\bibinfo
  {title} {Symmetry of large physical systems implies independence of
  subsystems},}\ }\href {\doibase 10.1038/nphys684} {\bibfield  {journal}
  {\bibinfo  {journal} {Nat. Phys.}\ }\textbf {\bibinfo {volume} {3}}~(\bibinfo
  {number} {9}),\ \bibinfo {pages} {645--649}},\ \Eprint
  {http://arxiv.org/abs/arXiv:quant-ph/0703069} {arXiv:quant-ph/0703069}
  \BibitemShut {NoStop}%
\bibitem [{\citenamefont {Renner}\ \emph {et~al.}(2005)\citenamefont {Renner},
  \citenamefont {Gisin},\ and\ \citenamefont {Kraus}}]{RGK05}%
  \BibitemOpen
  \bibfield  {author} {\bibinfo {author} {\bibnamefont {Renner}, \bibfnamefont
  {Renato}}, \bibinfo {author} {\bibfnamefont {Nicolas}\ \bibnamefont {Gisin}},
  \ and\ \bibinfo {author} {\bibfnamefont {Barbara}\ \bibnamefont {Kraus}}}
  (\bibinfo {year} {2005}),\ \bibfield  {title} {\enquote {\bibinfo {title}
  {Information-theoretic security proof for quantum-key-distribution
  protocols},}\ }\href {\doibase 10.1103/PhysRevA.72.012332} {\bibfield
  {journal} {\bibinfo  {journal} {Phys. Rev. A}\ }\textbf {\bibinfo {volume}
  {72}},\ \bibinfo {pages} {012332}},\ \Eprint
  {http://arxiv.org/abs/arXiv:quant-ph/0502064} {arXiv:quant-ph/0502064}
  \BibitemShut {NoStop}%
\bibitem [{\citenamefont {Renner}\ and\ \citenamefont {K\"onig}(2005)}]{RK05}%
  \BibitemOpen
  \bibfield  {author} {\bibinfo {author} {\bibnamefont {Renner}, \bibfnamefont
  {Renato}}, \ and\ \bibinfo {author} {\bibfnamefont {Robert}\ \bibnamefont
  {K\"onig}}} (\bibinfo {year} {2005}),\ \bibfield  {title} {\enquote {\bibinfo
  {title} {Universally composable privacy amplification against quantum
  adversaries},}\ }in\ \href {\doibase 10.1007/978-3-540-30576-7_22} {\emph
  {\bibinfo {booktitle} {Theory of Cryptography, Proceedings of TCC 2005}}},\
  \bibinfo {series} {LNCS}, Vol.\ \bibinfo {volume} {3378},\ \bibinfo {editor}
  {edited by\ \bibinfo {editor} {\bibfnamefont {Joe}\ \bibnamefont {Kilian}}}\
  (\bibinfo  {publisher} {Springer})\ pp.\ \bibinfo {pages} {407--425},\
  \Eprint {http://arxiv.org/abs/arXiv:quant-ph/0403133}
  {arXiv:quant-ph/0403133} \BibitemShut {NoStop}%
\bibitem [{\citenamefont {Renner}\ and\ \citenamefont {Wolf}(2003)}]{RW03}%
  \BibitemOpen
  \bibfield  {author} {\bibinfo {author} {\bibnamefont {Renner}, \bibfnamefont
  {Renato}}, \ and\ \bibinfo {author} {\bibfnamefont {Stefan}\ \bibnamefont
  {Wolf}}} (\bibinfo {year} {2003}),\ \bibfield  {title} {\enquote {\bibinfo
  {title} {Unconditional authenticity and privacy from an arbitrarily weak
  secret},}\ }in\ \href {\doibase 10.1007/978-3-540-45146-4_5} {\emph {\bibinfo
  {booktitle} {Advances in Cryptology -- CRYPTO 2003}}},\ \bibinfo {series}
  {LNCS}, Vol.\ \bibinfo {volume} {2729}\ (\bibinfo  {publisher} {Springer})\
  pp.\ \bibinfo {pages} {78--95}\BibitemShut {NoStop}%
\bibitem [{\citenamefont {Renner}\ and\ \citenamefont {Wolf}(2005)}]{RW05}%
  \BibitemOpen
  \bibfield  {author} {\bibinfo {author} {\bibnamefont {Renner}, \bibfnamefont
  {Renato}}, \ and\ \bibinfo {author} {\bibfnamefont {Stefan}\ \bibnamefont
  {Wolf}}} (\bibinfo {year} {2005}),\ \bibfield  {title} {\enquote {\bibinfo
  {title} {Simple and tight bounds for information reconciliation and privacy
  amplification},}\ }in\ \href {\doibase 10.1007/11593447_11} {\emph {\bibinfo
  {booktitle} {Advances in Cryptology -- ASIACRYPT 2005}}},\ \bibinfo {series}
  {LNCS}, Vol.\ \bibinfo {volume} {3788},\ \bibinfo {editor} {edited by\
  \bibinfo {editor} {\bibfnamefont {Bimal}\ \bibnamefont {Roy}}}\ (\bibinfo
  {publisher} {Springer})\ pp.\ \bibinfo {pages} {199--216}\BibitemShut
  {NoStop}%
\bibitem [{\citenamefont {Renner}\ and\ \citenamefont {Cirac}(2009)}]{RC09}%
  \BibitemOpen
  \bibfield  {author} {\bibinfo {author} {\bibnamefont {Renner}, \bibfnamefont
  {Renner}}, \ and\ \bibinfo {author} {\bibfnamefont {J.~Ignacio}\ \bibnamefont
  {Cirac}}} (\bibinfo {year} {2009}),\ \bibfield  {title} {\enquote {\bibinfo
  {title} {{de Finetti} representation theorem for infinite-dimensional quantum
  systems and applications to quantum cryptography},}\ }\href {\doibase
  10.1103/PhysRevLett.102.110504} {\bibfield  {journal} {\bibinfo  {journal}
  {Phys. Rev. Lett.}\ }\textbf {\bibinfo {volume} {102}},\ \bibinfo {pages}
  {110504}},\ \Eprint {http://arxiv.org/abs/arXiv:0809.2243} {arXiv:0809.2243}
  \BibitemShut {NoStop}%
\bibitem [{\citenamefont {Rivest}\ \emph {et~al.}(1978)\citenamefont {Rivest},
  \citenamefont {Shamir},\ and\ \citenamefont {Adleman}}]{RSA78}%
  \BibitemOpen
  \bibfield  {author} {\bibinfo {author} {\bibnamefont {Rivest}, \bibfnamefont
  {Ronald~L}}, \bibinfo {author} {\bibfnamefont {Adi}\ \bibnamefont {Shamir}},
  \ and\ \bibinfo {author} {\bibfnamefont {Leonard}\ \bibnamefont {Adleman}}}
  (\bibinfo {year} {1978}),\ \bibfield  {title} {\enquote {\bibinfo {title} {A
  method for obtaining digital signatures and public-key cryptosystems},}\
  }\href@noop {} {\bibfield  {journal} {\bibinfo  {journal} {Commun. ACM}\
  }\textbf {\bibinfo {volume} {21}}~(\bibinfo {number} {2}),\ \bibinfo {pages}
  {120--126}}\BibitemShut {NoStop}%
\bibitem [{\citenamefont {Rogaway}(2006)}]{Rog06}%
  \BibitemOpen
  \bibfield  {author} {\bibinfo {author} {\bibnamefont {Rogaway}, \bibfnamefont
  {Phillip}}} (\bibinfo {year} {2006}),\ \bibfield  {title} {\enquote {\bibinfo
  {title} {Formalizing human ignorance},}\ }in\ \href {\doibase
  10.1007/11958239_14} {\emph {\bibinfo {booktitle} {Progress in Cryptology --
  VIETCRYPT 2006}}},\ \bibinfo {series} {LNCS}, Vol.\ \bibinfo {volume} {4341}\
  (\bibinfo  {publisher} {Springer})\ pp.\ \bibinfo {pages} {211--228},\
  \bibinfo {note} {e-Print \href{http://eprint.iacr.org/2006/281}{IACR
  2006/281}}\BibitemShut {NoStop}%
\bibitem [{\citenamefont {Rosenfeld}\ \emph {et~al.}(2017)\citenamefont
  {Rosenfeld}, \citenamefont {Burchardt}, \citenamefont {Garthoff},
  \citenamefont {Redeker}, \citenamefont {Ortegel}, \citenamefont {Rau},\ and\
  \citenamefont {Weinfurter}}]{Rosenfeld}%
  \BibitemOpen
  \bibfield  {author} {\bibinfo {author} {\bibnamefont {Rosenfeld},
  \bibfnamefont {Wenjamin}}, \bibinfo {author} {\bibfnamefont {Daniel}\
  \bibnamefont {Burchardt}}, \bibinfo {author} {\bibfnamefont {Robert}\
  \bibnamefont {Garthoff}}, \bibinfo {author} {\bibfnamefont {Kai}\
  \bibnamefont {Redeker}}, \bibinfo {author} {\bibfnamefont {Norbert}\
  \bibnamefont {Ortegel}}, \bibinfo {author} {\bibfnamefont {Markus}\
  \bibnamefont {Rau}}, \ and\ \bibinfo {author} {\bibfnamefont {Harald}\
  \bibnamefont {Weinfurter}}} (\bibinfo {year} {2017}),\ \bibfield  {title}
  {\enquote {\bibinfo {title} {Event-ready bell test using entangled atoms
  simultaneously closing detection and locality loopholes},}\ }\href {\doibase
  10.1103/PhysRevLett.119.010402} {\bibfield  {journal} {\bibinfo  {journal}
  {Phys. Rev. Lett.}\ }\textbf {\bibinfo {volume} {119}},\ \bibinfo {pages}
  {010402}}\BibitemShut {NoStop}%
\bibitem [{\citenamefont {Rowe}\ \emph {et~al.}(2001)\citenamefont {Rowe},
  \citenamefont {Kielpinski}, \citenamefont {Meyer}, \citenamefont {Sackett},
  \citenamefont {Itano}, \citenamefont {Monroe},\ and\ \citenamefont
  {Wineland}}]{Rowe}%
  \BibitemOpen
  \bibfield  {author} {\bibinfo {author} {\bibnamefont {Rowe}, \bibfnamefont
  {M~A}}, \bibinfo {author} {\bibfnamefont {David}\ \bibnamefont {Kielpinski}},
  \bibinfo {author} {\bibfnamefont {V.}~\bibnamefont {Meyer}}, \bibinfo
  {author} {\bibfnamefont {Charles~A.}\ \bibnamefont {Sackett}}, \bibinfo
  {author} {\bibfnamefont {Wayne~M.}\ \bibnamefont {Itano}}, \bibinfo {author}
  {\bibfnamefont {C.}~\bibnamefont {Monroe}}, \ and\ \bibinfo {author}
  {\bibfnamefont {D.~J.}\ \bibnamefont {Wineland}}} (\bibinfo {year} {2001}),\
  \bibfield  {title} {\enquote {\bibinfo {title} {Experimental violation of a
  bell's inequality with efficient detection},}\ }\href {\doibase
  10.1038/35057215} {\bibfield  {journal} {\bibinfo  {journal} {Nature}\
  }\textbf {\bibinfo {volume} {409}}~(\bibinfo {number} {6822}),\ \bibinfo
  {pages} {791--794}}\BibitemShut {NoStop}%
\bibitem [{\citenamefont {Sasaki}\ \emph {et~al.}(2014)\citenamefont {Sasaki},
  \citenamefont {Yamamoto},\ and\ \citenamefont {Koashi}}]{SYK14}%
  \BibitemOpen
  \bibfield  {author} {\bibinfo {author} {\bibnamefont {Sasaki}, \bibfnamefont
  {Toshihiko}}, \bibinfo {author} {\bibfnamefont {Yoshihisa}\ \bibnamefont
  {Yamamoto}}, \ and\ \bibinfo {author} {\bibfnamefont {Masato}\ \bibnamefont
  {Koashi}}} (\bibinfo {year} {2014}),\ \bibfield  {title} {\enquote {\bibinfo
  {title} {Practical quantum key distribution protocol without monitoring
  signal disturbance},}\ }\href {\doibase 10.1038/nature13303} {\bibfield
  {journal} {\bibinfo  {journal} {Nature}\ }\textbf {\bibinfo {volume} {509}},\
  \bibinfo {pages} {475}}\BibitemShut {NoStop}%
\bibitem [{\citenamefont {Scarani}(2013)}]{Sca13}%
  \BibitemOpen
  \bibfield  {author} {\bibinfo {author} {\bibnamefont {Scarani}, \bibfnamefont
  {Valerio}}} (\bibinfo {year} {2013}),\ \href@noop {} {\enquote {\bibinfo
  {title} {The device-independent outlook on quantum physics (lecture notes on
  the power of {Bell}'s theorem)},}\ }\bibinfo {howpublished} {e-Print},\
  \Eprint {http://arxiv.org/abs/arXiv:1303.3081} {arXiv:1303.3081} \BibitemShut
  {NoStop}%
\bibitem [{\citenamefont {Scarani}\ \emph {et~al.}(2004)\citenamefont
  {Scarani}, \citenamefont {Ac\'{\i}n}, \citenamefont {Ribordy},\ and\
  \citenamefont {Gisin}}]{SARG}%
  \BibitemOpen
  \bibfield  {author} {\bibinfo {author} {\bibnamefont {Scarani}, \bibfnamefont
  {Valerio}}, \bibinfo {author} {\bibfnamefont {Antonio}\ \bibnamefont
  {Ac\'{\i}n}}, \bibinfo {author} {\bibfnamefont {Gr\'egoire}\ \bibnamefont
  {Ribordy}}, \ and\ \bibinfo {author} {\bibfnamefont {Nicolas}\ \bibnamefont
  {Gisin}}} (\bibinfo {year} {2004}),\ \bibfield  {title} {\enquote {\bibinfo
  {title} {Quantum cryptography protocols robust against photon number
  splitting attacks for weak laser pulse implementations},}\ }\href {\doibase
  10.1103/PhysRevLett.92.057901} {\bibfield  {journal} {\bibinfo  {journal}
  {Phys. Rev. Lett.}\ }\textbf {\bibinfo {volume} {92}},\ \bibinfo {pages}
  {057901}}\BibitemShut {NoStop}%
\bibitem [{\citenamefont {Scarani}\ \emph {et~al.}(2009)\citenamefont
  {Scarani}, \citenamefont {Bechmann-Pasquinucci}, \citenamefont {Cerf},
  \citenamefont {Du\ifmmode~\check{s}\else \v{s}\fi{}ek}, \citenamefont
  {L\"utkenhaus},\ and\ \citenamefont {Peev}}]{SBCDLP09}%
  \BibitemOpen
  \bibfield  {author} {\bibinfo {author} {\bibnamefont {Scarani}, \bibfnamefont
  {Valerio}}, \bibinfo {author} {\bibfnamefont {Helle}\ \bibnamefont
  {Bechmann-Pasquinucci}}, \bibinfo {author} {\bibfnamefont {Nicolas~J.}\
  \bibnamefont {Cerf}}, \bibinfo {author} {\bibfnamefont {Miloslav}\
  \bibnamefont {Du\ifmmode~\check{s}\else \v{s}\fi{}ek}}, \bibinfo {author}
  {\bibfnamefont {Norbert}\ \bibnamefont {L\"utkenhaus}}, \ and\ \bibinfo
  {author} {\bibfnamefont {Momtchil}\ \bibnamefont {Peev}}} (\bibinfo {year}
  {2009}),\ \bibfield  {title} {\enquote {\bibinfo {title} {The security of
  practical quantum key distribution},}\ }\href {\doibase
  10.1103/RevModPhys.81.1301} {\bibfield  {journal} {\bibinfo  {journal} {Rev.
  Mod. Phys.}\ }\textbf {\bibinfo {volume} {81}},\ \bibinfo {pages}
  {1301--1350}},\ \Eprint {http://arxiv.org/abs/arXiv:0802.4155}
  {arXiv:0802.4155} \BibitemShut {NoStop}%
\bibitem [{\citenamefont {Scarani}\ and\ \citenamefont {Renner}(2008)}]{SR08}%
  \BibitemOpen
  \bibfield  {author} {\bibinfo {author} {\bibnamefont {Scarani}, \bibfnamefont
  {Valerio}}, \ and\ \bibinfo {author} {\bibfnamefont {Renato}\ \bibnamefont
  {Renner}}} (\bibinfo {year} {2008}),\ \bibfield  {title} {\enquote {\bibinfo
  {title} {Quantum cryptography with finite resources: Unconditional security
  bound for discrete-variable protocols with one-way postprocessing},}\ }\href
  {\doibase 10.1103/PhysRevLett.100.200501} {\bibfield  {journal} {\bibinfo
  {journal} {Phys. Rev. Lett.}\ }\textbf {\bibinfo {volume} {100}},\ \bibinfo
  {pages} {200501}},\ \Eprint {http://arxiv.org/abs/arXiv:0708.0709}
  {arXiv:0708.0709} \BibitemShut {NoStop}%
\bibitem [{\citenamefont {Schaffner}\ \emph {et~al.}(2009)\citenamefont
  {Schaffner}, \citenamefont {Terhal},\ and\ \citenamefont {Wehner}}]{STW09}%
  \BibitemOpen
  \bibfield  {author} {\bibinfo {author} {\bibnamefont {Schaffner},
  \bibfnamefont {Christian}}, \bibinfo {author} {\bibfnamefont {Barbara}\
  \bibnamefont {Terhal}}, \ and\ \bibinfo {author} {\bibfnamefont {Stephanie}\
  \bibnamefont {Wehner}}} (\bibinfo {year} {2009}),\ \bibfield  {title}
  {\enquote {\bibinfo {title} {Robust cryptography in the noisy-quantum-storage
  model},}\ }\href@noop {} {\bibfield  {journal} {\bibinfo  {journal} {Quantum
  Inf. Comput.}\ }\textbf {\bibinfo {volume} {9}}~(\bibinfo {number} {11}),\
  \bibinfo {pages} {963--996}},\ \Eprint {http://arxiv.org/abs/arXiv:0807.1333}
  {arXiv:0807.1333} \BibitemShut {NoStop}%
\bibitem [{\citenamefont {Seiler}\ and\ \citenamefont {Maurer}(2016)}]{SM16}%
  \BibitemOpen
  \bibfield  {author} {\bibinfo {author} {\bibnamefont {Seiler}, \bibfnamefont
  {Gregor}}, \ and\ \bibinfo {author} {\bibfnamefont {Ueli}\ \bibnamefont
  {Maurer}}} (\bibinfo {year} {2016}),\ \bibfield  {title} {\enquote {\bibinfo
  {title} {On the impossibility of information-theoretic composable coin toss
  extension},}\ }in\ \href {\doibase 10.1109/ISIT.2016.7541861} {\emph
  {\bibinfo {booktitle} {Proceedings of the 2016 IEEE International Symposium
  on Information Theory, ISIT 2016}}}\ (\bibinfo  {publisher} {IEEE})\ pp.\
  \bibinfo {pages} {3058--3061}\BibitemShut {NoStop}%
\bibitem [{\citenamefont {Shalm}\ \emph {et~al.}(2015)\citenamefont {Shalm},
  \citenamefont {Meyer-Scott}, \citenamefont {Christensen}, \citenamefont
  {Bierhorst}, \citenamefont {Wayne}, \citenamefont {Stevens}, \citenamefont
  {Gerrits}, \citenamefont {Glancy}, \citenamefont {Hamel}, \citenamefont
  {Allman}, \citenamefont {Coakley}, \citenamefont {Dyer}, \citenamefont
  {Hodge}, \citenamefont {Lita}, \citenamefont {Verma}, \citenamefont
  {Lambrocco}, \citenamefont {Tortorici}, \citenamefont {Migdall},
  \citenamefont {Zhang}, \citenamefont {Kumor}, \citenamefont {Farr},
  \citenamefont {Marsili}, \citenamefont {Shaw}, \citenamefont {Stern},
  \citenamefont {Abell\'an}, \citenamefont {Amaya}, \citenamefont {Pruneri},
  \citenamefont {Jennewein}, \citenamefont {Mitchell}, \citenamefont {Kwiat},
  \citenamefont {Bienfang}, \citenamefont {Mirin}, \citenamefont {Knill},\ and\
  \citenamefont {Nam}}]{Shalm}%
  \BibitemOpen
  \bibfield  {author} {\bibinfo {author} {\bibnamefont {Shalm}, \bibfnamefont
  {Lynden~K}}, \bibinfo {author} {\bibfnamefont {Evan}\ \bibnamefont
  {Meyer-Scott}}, \bibinfo {author} {\bibfnamefont {Bradley~G.}\ \bibnamefont
  {Christensen}}, \bibinfo {author} {\bibfnamefont {Peter}\ \bibnamefont
  {Bierhorst}}, \bibinfo {author} {\bibfnamefont {Michael~A.}\ \bibnamefont
  {Wayne}}, \bibinfo {author} {\bibfnamefont {Martin~J.}\ \bibnamefont
  {Stevens}}, \bibinfo {author} {\bibfnamefont {Thomas}\ \bibnamefont
  {Gerrits}}, \bibinfo {author} {\bibfnamefont {Scott}\ \bibnamefont {Glancy}},
  \bibinfo {author} {\bibfnamefont {Deny~R.}\ \bibnamefont {Hamel}}, \bibinfo
  {author} {\bibfnamefont {Michael~S.}\ \bibnamefont {Allman}}, \bibinfo
  {author} {\bibfnamefont {Kevin~J.}\ \bibnamefont {Coakley}}, \bibinfo
  {author} {\bibfnamefont {Shellee~D.}\ \bibnamefont {Dyer}}, \bibinfo {author}
  {\bibfnamefont {Carson}\ \bibnamefont {Hodge}}, \bibinfo {author}
  {\bibfnamefont {Adriana~E.}\ \bibnamefont {Lita}}, \bibinfo {author}
  {\bibfnamefont {Varun~B.}\ \bibnamefont {Verma}}, \bibinfo {author}
  {\bibfnamefont {Camilla}\ \bibnamefont {Lambrocco}}, \bibinfo {author}
  {\bibfnamefont {Edward}\ \bibnamefont {Tortorici}}, \bibinfo {author}
  {\bibfnamefont {Alan~L.}\ \bibnamefont {Migdall}}, \bibinfo {author}
  {\bibfnamefont {Yanbao}\ \bibnamefont {Zhang}}, \bibinfo {author}
  {\bibfnamefont {Daniel~R.}\ \bibnamefont {Kumor}}, \bibinfo {author}
  {\bibfnamefont {William~H.}\ \bibnamefont {Farr}}, \bibinfo {author}
  {\bibfnamefont {Francesco}\ \bibnamefont {Marsili}}, \bibinfo {author}
  {\bibfnamefont {Matthew~D.}\ \bibnamefont {Shaw}}, \bibinfo {author}
  {\bibfnamefont {Jeffrey~A.}\ \bibnamefont {Stern}}, \bibinfo {author}
  {\bibfnamefont {Carlos}\ \bibnamefont {Abell\'an}}, \bibinfo {author}
  {\bibfnamefont {Waldimar}\ \bibnamefont {Amaya}}, \bibinfo {author}
  {\bibfnamefont {Valerio}\ \bibnamefont {Pruneri}}, \bibinfo {author}
  {\bibfnamefont {Thomas}\ \bibnamefont {Jennewein}}, \bibinfo {author}
  {\bibfnamefont {Morgan~W.}\ \bibnamefont {Mitchell}}, \bibinfo {author}
  {\bibfnamefont {Paul~G.}\ \bibnamefont {Kwiat}}, \bibinfo {author}
  {\bibfnamefont {Joshua~C.}\ \bibnamefont {Bienfang}}, \bibinfo {author}
  {\bibfnamefont {Richard~P.}\ \bibnamefont {Mirin}}, \bibinfo {author}
  {\bibfnamefont {Emanuel}\ \bibnamefont {Knill}}, \ and\ \bibinfo {author}
  {\bibfnamefont {Sae~Woo}\ \bibnamefont {Nam}}} (\bibinfo {year} {2015}),\
  \bibfield  {title} {\enquote {\bibinfo {title} {Strong loophole-free test of
  local realism},}\ }\href {\doibase 10.1103/PhysRevLett.115.250402} {\bibfield
   {journal} {\bibinfo  {journal} {Phys. Rev. Lett.}\ }\textbf {\bibinfo
  {volume} {115}},\ \bibinfo {pages} {250402}}\BibitemShut {NoStop}%
\bibitem [{\citenamefont {Shaltiel}(2004)}]{Shaltiel04}%
  \BibitemOpen
  \bibfield  {author} {\bibinfo {author} {\bibnamefont {Shaltiel},
  \bibfnamefont {Ronen}}} (\bibinfo {year} {2004}),\ \bibfield  {title}
  {\enquote {\bibinfo {title} {Recent developments in explicit constructions of
  extractors},}\ }in\ \href {\doibase 10.1142/9789812562494_0013} {\emph
  {\bibinfo {booktitle} {Current Trends in Theoretical Computer Science: The
  Challenge of the New Century, Vol 1: Algorithms and Complexity}}}\ (\bibinfo
  {publisher} {World Scientific})\ pp.\ \bibinfo {pages} {189--228}\BibitemShut
  {NoStop}%
\bibitem [{\citenamefont {Shannon}(1949)}]{Shannon49}%
  \BibitemOpen
  \bibfield  {author} {\bibinfo {author} {\bibnamefont {Shannon}, \bibfnamefont
  {Claude~E}}} (\bibinfo {year} {1949}),\ \bibfield  {title} {\enquote
  {\bibinfo {title} {Communication theory of secrecy systems},}\ }\href@noop {}
  {\bibfield  {journal} {\bibinfo  {journal} {Bell system technical journal}\
  }\textbf {\bibinfo {volume} {28}}~(\bibinfo {number} {4}),\ \bibinfo {pages}
  {656--715}}\BibitemShut {NoStop}%
\bibitem [{\citenamefont {Sheridan}\ \emph {et~al.}(2010)\citenamefont
  {Sheridan}, \citenamefont {Thinh},\ and\ \citenamefont {Scarani}}]{STS10}%
  \BibitemOpen
  \bibfield  {author} {\bibinfo {author} {\bibnamefont {Sheridan},
  \bibfnamefont {Lana}}, \bibinfo {author} {\bibfnamefont {Phuc~Le}\
  \bibnamefont {Thinh}}, \ and\ \bibinfo {author} {\bibfnamefont {Valerio}\
  \bibnamefont {Scarani}}} (\bibinfo {year} {2010}),\ \bibfield  {title}
  {\enquote {\bibinfo {title} {Finite-key security against coherent attacks in
  quantum key distribution},}\ }\href {\doibase 10.1088/1367-2630/12/12/123019}
  {\bibfield  {journal} {\bibinfo  {journal} {New J. Phys.}\ }\textbf {\bibinfo
  {volume} {12}}~(\bibinfo {number} {12}),\ \bibinfo {pages} {123019}},\
  \Eprint {http://arxiv.org/abs/arXiv:1008.2596} {arXiv:1008.2596} \BibitemShut
  {NoStop}%
\bibitem [{\citenamefont {Shor}(1997)}]{Shor97}%
  \BibitemOpen
  \bibfield  {author} {\bibinfo {author} {\bibnamefont {Shor}, \bibfnamefont
  {Peter~W}}} (\bibinfo {year} {1997}),\ \bibfield  {title} {\enquote {\bibinfo
  {title} {Polynomial-time algorithms for prime factorization and discrete
  logarithms on a quantum computer},}\ }\href {\doibase
  10.1137/S0097539795293172} {\bibfield  {journal} {\bibinfo  {journal} {SIAM
  J. Comput.}\ }\textbf {\bibinfo {volume} {26}}~(\bibinfo {number} {5}),\
  \bibinfo {pages} {1484--1509}}\BibitemShut {NoStop}%
\bibitem [{\citenamefont {Shor}\ and\ \citenamefont {Preskill}(2000)}]{SP00}%
  \BibitemOpen
  \bibfield  {author} {\bibinfo {author} {\bibnamefont {Shor}, \bibfnamefont
  {Peter~W}}, \ and\ \bibinfo {author} {\bibfnamefont {John}\ \bibnamefont
  {Preskill}}} (\bibinfo {year} {2000}),\ \bibfield  {title} {\enquote
  {\bibinfo {title} {Simple proof of security of the {BB84} quantum key
  distribution protocol},}\ }\href {\doibase 10.1103/PhysRevLett.85.441}
  {\bibfield  {journal} {\bibinfo  {journal} {Phys. Rev. Lett.}\ }\textbf
  {\bibinfo {volume} {85}},\ \bibinfo {pages} {441--444}},\ \Eprint
  {http://arxiv.org/abs/arXiv:quant-ph/0003004} {arXiv:quant-ph/0003004}
  \BibitemShut {NoStop}%
\bibitem [{\citenamefont {Simmons}(1985)}]{Sim85}%
  \BibitemOpen
  \bibfield  {author} {\bibinfo {author} {\bibnamefont {Simmons}, \bibfnamefont
  {Gustavus~J}}} (\bibinfo {year} {1985}),\ \bibfield  {title} {\enquote
  {\bibinfo {title} {Authentication theory/coding theory},}\ }in\ \href
  {\doibase 10.1007/3-540-39568-7_32} {\emph {\bibinfo {booktitle} {Advances in
  Cryptology -- CRYPTO~'84}}},\ \bibinfo {series} {LNCS}, Vol.\ \bibinfo
  {volume} {196}\ (\bibinfo  {publisher} {Springer})\ pp.\ \bibinfo {pages}
  {411--431}\BibitemShut {NoStop}%
\bibitem [{\citenamefont {Simmons}(1988)}]{Sim88}%
  \BibitemOpen
  \bibfield  {author} {\bibinfo {author} {\bibnamefont {Simmons}, \bibfnamefont
  {Gustavus~J}}} (\bibinfo {year} {1988}),\ \bibfield  {title} {\enquote
  {\bibinfo {title} {A survey of information authentication},}\ }\href
  {\doibase 10.1109/5.4445} {\bibfield  {journal} {\bibinfo  {journal} {Proc.
  IEEE}\ }\textbf {\bibinfo {volume} {76}}~(\bibinfo {number} {5}),\ \bibinfo
  {pages} {603--620}}\BibitemShut {NoStop}%
\bibitem [{\citenamefont {Steane}(1996)}]{Steane96}%
  \BibitemOpen
  \bibfield  {author} {\bibinfo {author} {\bibnamefont {Steane}, \bibfnamefont
  {Andrew}}} (\bibinfo {year} {1996}),\ \bibfield  {title} {\enquote {\bibinfo
  {title} {Multiple-particle interference and quantum error correction},}\
  }\href@noop {} {\bibfield  {journal} {\bibinfo  {journal} {Proc. R. Soc.
  London, Ser. A}\ }\textbf {\bibinfo {volume} {452}}~(\bibinfo {number}
  {1954}),\ \bibinfo {pages} {2551--2577}}\BibitemShut {NoStop}%
\bibitem [{\citenamefont {Stinson}(1990)}]{Sti90}%
  \BibitemOpen
  \bibfield  {author} {\bibinfo {author} {\bibnamefont {Stinson}, \bibfnamefont
  {Douglas~R}}} (\bibinfo {year} {1990}),\ \bibfield  {title} {\enquote
  {\bibinfo {title} {The combinatorics of authentication and secrecy codes},}\
  }\href {\doibase 10.1007/BF02252868} {\bibfield  {journal} {\bibinfo
  {journal} {J. Crypt.}\ }\textbf {\bibinfo {volume} {2}}~(\bibinfo {number}
  {1}),\ \bibinfo {pages} {23--49}}\BibitemShut {NoStop}%
\bibitem [{\citenamefont {Stinson}(1994)}]{Sti94}%
  \BibitemOpen
  \bibfield  {author} {\bibinfo {author} {\bibnamefont {Stinson}, \bibfnamefont
  {Douglas~R}}} (\bibinfo {year} {1994}),\ \bibfield  {title} {\enquote
  {\bibinfo {title} {Universal hashing and authentication codes},}\ }\href
  {\doibase 10.1007/BF01388651} {\bibfield  {journal} {\bibinfo  {journal}
  {Des. Codes Cryptogr.}\ }\textbf {\bibinfo {volume} {4}}~(\bibinfo {number}
  {3}),\ \bibinfo {pages} {369--380}},\ \bibinfo {note} {a preliminary version
  appeared at CRYPTO~'91}\BibitemShut {NoStop}%
\bibitem [{\citenamefont {Streltsov}\ \emph {et~al.}(2017)\citenamefont
  {Streltsov}, \citenamefont {Adesso},\ and\ \citenamefont {Plenio}}]{SAP17}%
  \BibitemOpen
  \bibfield  {author} {\bibinfo {author} {\bibnamefont {Streltsov},
  \bibfnamefont {Alexander}}, \bibinfo {author} {\bibfnamefont {Gerardo}\
  \bibnamefont {Adesso}}, \ and\ \bibinfo {author} {\bibfnamefont {Martin~B.}\
  \bibnamefont {Plenio}}} (\bibinfo {year} {2017}),\ \bibfield  {title}
  {\enquote {\bibinfo {title} {Colloquium: Quantum coherence as a resource},}\
  }\href {\doibase 10.1103/RevModPhys.89.041003} {\bibfield  {journal}
  {\bibinfo  {journal} {Rev. Mod. Phys.}\ }\textbf {\bibinfo {volume} {89}},\
  \bibinfo {pages} {041003}}\BibitemShut {NoStop}%
\bibitem [{\citenamefont {Stucki}\ \emph {et~al.}(2005)\citenamefont {Stucki},
  \citenamefont {Brunner}, \citenamefont {Gisin}, \citenamefont {Scarani},\
  and\ \citenamefont {Zbinden}}]{SBGSZ05}%
  \BibitemOpen
  \bibfield  {author} {\bibinfo {author} {\bibnamefont {Stucki}, \bibfnamefont
  {Damien}}, \bibinfo {author} {\bibfnamefont {Nicolas}\ \bibnamefont
  {Brunner}}, \bibinfo {author} {\bibfnamefont {Nicolas}\ \bibnamefont
  {Gisin}}, \bibinfo {author} {\bibfnamefont {Valerio}\ \bibnamefont
  {Scarani}}, \ and\ \bibinfo {author} {\bibfnamefont {Hugo}\ \bibnamefont
  {Zbinden}}} (\bibinfo {year} {2005}),\ \bibfield  {title} {\enquote {\bibinfo
  {title} {Fast and simple one-way quantum key distribution},}\ }\href
  {\doibase 10.1063/1.2126792} {\bibfield  {journal} {\bibinfo  {journal}
  {Appl. Phys. Lett.}\ }\textbf {\bibinfo {volume} {87}}~(\bibinfo {number}
  {19}),\ \bibinfo {pages} {194108}},\ \Eprint
  {http://arxiv.org/abs/arXiv:quant-ph/0506097} {arXiv:quant-ph/0506097}
  \BibitemShut {NoStop}%
\bibitem [{\citenamefont {Tamaki}\ \emph {et~al.}(2003)\citenamefont {Tamaki},
  \citenamefont {Koashi},\ and\ \citenamefont {Imoto}}]{Tamaki03}%
  \BibitemOpen
  \bibfield  {author} {\bibinfo {author} {\bibnamefont {Tamaki}, \bibfnamefont
  {Kiyoshi}}, \bibinfo {author} {\bibfnamefont {Masato}\ \bibnamefont
  {Koashi}}, \ and\ \bibinfo {author} {\bibfnamefont {Nobuyuki}\ \bibnamefont
  {Imoto}}} (\bibinfo {year} {2003}),\ \bibfield  {title} {\enquote {\bibinfo
  {title} {Unconditionally secure key distribution based on two nonorthogonal
  states},}\ }\href {\doibase 10.1103/PhysRevLett.90.167904} {\bibfield
  {journal} {\bibinfo  {journal} {Phys. Rev. Lett.}\ }\textbf {\bibinfo
  {volume} {90}},\ \bibinfo {pages} {167904}}\BibitemShut {NoStop}%
\bibitem [{\citenamefont {Tamaki}\ and\ \citenamefont {Lo}(2006)}]{TamakiLo}%
  \BibitemOpen
  \bibfield  {author} {\bibinfo {author} {\bibnamefont {Tamaki}, \bibfnamefont
  {Kiyoshi}}, \ and\ \bibinfo {author} {\bibfnamefont {Hoi-Kwong}\ \bibnamefont
  {Lo}}} (\bibinfo {year} {2006}),\ \bibfield  {title} {\enquote {\bibinfo
  {title} {Unconditionally secure key distillation from multiphotons},}\ }\href
  {\doibase 10.1103/PhysRevA.73.010302} {\bibfield  {journal} {\bibinfo
  {journal} {Phys. Rev. A}\ }\textbf {\bibinfo {volume} {73}},\ \bibinfo
  {pages} {010302}}\BibitemShut {NoStop}%
\bibitem [{\citenamefont {Tamaki}\ \emph {et~al.}(2012)\citenamefont {Tamaki},
  \citenamefont {Lo}, \citenamefont {Fung},\ and\ \citenamefont
  {Qi}}]{TamakiLo12}%
  \BibitemOpen
  \bibfield  {author} {\bibinfo {author} {\bibnamefont {Tamaki}, \bibfnamefont
  {Kiyoshi}}, \bibinfo {author} {\bibfnamefont {Hoi-Kwong}\ \bibnamefont {Lo}},
  \bibinfo {author} {\bibfnamefont {Chi-Hang~Fred}\ \bibnamefont {Fung}}, \
  and\ \bibinfo {author} {\bibfnamefont {Bing}\ \bibnamefont {Qi}}} (\bibinfo
  {year} {2012}),\ \bibfield  {title} {\enquote {\bibinfo {title} {Phase
  encoding schemes for measurement-device-independent quantum key distribution
  with basis-dependent flaw},}\ }\href {\doibase 10.1103/PhysRevA.85.042307}
  {\bibfield  {journal} {\bibinfo  {journal} {Phys. Rev. A}\ }\textbf {\bibinfo
  {volume} {85}},\ \bibinfo {pages} {042307}}\BibitemShut {NoStop}%
\bibitem [{\citenamefont {Tan}\ \emph {et~al.}(2020)\citenamefont {Tan},
  \citenamefont {Lim},\ and\ \citenamefont {Renner}}]{TLR20}%
  \BibitemOpen
  \bibfield  {author} {\bibinfo {author} {\bibnamefont {Tan}, \bibfnamefont
  {Ernest Y-Z}}, \bibinfo {author} {\bibfnamefont {Charles Ci~Wen}\
  \bibnamefont {Lim}}, \ and\ \bibinfo {author} {\bibfnamefont {Renato}\
  \bibnamefont {Renner}}} (\bibinfo {year} {2020}),\ \bibfield  {title}
  {\enquote {\bibinfo {title} {Advantage distillation for device-independent
  quantum key distribution},}\ }\href {\doibase 10.1103/PhysRevLett.124.020502}
  {\bibfield  {journal} {\bibinfo  {journal} {Phys. Rev. Lett.}\ }\textbf
  {\bibinfo {volume} {124}},\ \bibinfo {pages} {020502}},\ \Eprint
  {http://arxiv.org/abs/arXiv:1903.10535} {arXiv:1903.10535} \BibitemShut
  {NoStop}%
\bibitem [{\citenamefont {Tang}\ \emph {et~al.}(2014)\citenamefont {Tang},
  \citenamefont {Yin}, \citenamefont {Chen}, \citenamefont {Liu}, \citenamefont
  {Zhang}, \citenamefont {Jiang}, \citenamefont {Zhang}, \citenamefont {Wang},
  \citenamefont {You}, \citenamefont {Guan}, \citenamefont {Yang},
  \citenamefont {Wang}, \citenamefont {Liang}, \citenamefont {Zhang},
  \citenamefont {Zhou}, \citenamefont {Ma}, \citenamefont {Chen}, \citenamefont
  {Zhang},\ and\ \citenamefont {Pan}}]{Tang2014}%
  \BibitemOpen
  \bibfield  {author} {\bibinfo {author} {\bibnamefont {Tang}, \bibfnamefont
  {Yan-Lin}}, \bibinfo {author} {\bibfnamefont {Hua-Lei}\ \bibnamefont {Yin}},
  \bibinfo {author} {\bibfnamefont {Si-Jing}\ \bibnamefont {Chen}}, \bibinfo
  {author} {\bibfnamefont {Yang}\ \bibnamefont {Liu}}, \bibinfo {author}
  {\bibfnamefont {Wei-Jun}\ \bibnamefont {Zhang}}, \bibinfo {author}
  {\bibfnamefont {Xiao}\ \bibnamefont {Jiang}}, \bibinfo {author}
  {\bibfnamefont {Lu}~\bibnamefont {Zhang}}, \bibinfo {author} {\bibfnamefont
  {Jian}\ \bibnamefont {Wang}}, \bibinfo {author} {\bibfnamefont {Li-Xing}\
  \bibnamefont {You}}, \bibinfo {author} {\bibfnamefont {Jian-Yu}\ \bibnamefont
  {Guan}}, \bibinfo {author} {\bibfnamefont {Dong-Xu}\ \bibnamefont {Yang}},
  \bibinfo {author} {\bibfnamefont {Zhen}\ \bibnamefont {Wang}}, \bibinfo
  {author} {\bibfnamefont {Hao}\ \bibnamefont {Liang}}, \bibinfo {author}
  {\bibfnamefont {Zhen}\ \bibnamefont {Zhang}}, \bibinfo {author}
  {\bibfnamefont {Nan}\ \bibnamefont {Zhou}}, \bibinfo {author} {\bibfnamefont
  {Xiongfeng}\ \bibnamefont {Ma}}, \bibinfo {author} {\bibfnamefont {Teng-Yun}\
  \bibnamefont {Chen}}, \bibinfo {author} {\bibfnamefont {Qiang}\ \bibnamefont
  {Zhang}}, \ and\ \bibinfo {author} {\bibfnamefont {Jian-Wei}\ \bibnamefont
  {Pan}}} (\bibinfo {year} {2014}),\ \bibfield  {title} {\enquote {\bibinfo
  {title} {Measurement-device-independent quantum key distribution over 200
  km},}\ }\href {\doibase 10.1103/PhysRevLett.113.190501} {\bibfield  {journal}
  {\bibinfo  {journal} {Phys. Rev. Lett.}\ }\textbf {\bibinfo {volume} {113}},\
  \bibinfo {pages} {190501}}\BibitemShut {NoStop}%
\bibitem [{\citenamefont {Terhal}(2004)}]{Terhal04}%
  \BibitemOpen
  \bibfield  {author} {\bibinfo {author} {\bibnamefont {Terhal}, \bibfnamefont
  {Barbara~M}}} (\bibinfo {year} {2004}),\ \bibfield  {title} {\enquote
  {\bibinfo {title} {Is entanglement monogamous?}}\ }\href {\doibase
  10.1147/rd.481.0071} {\bibfield  {journal} {\bibinfo  {journal} {IBM J. Res.
  Dev.}\ }\textbf {\bibinfo {volume} {48}}~(\bibinfo {number} {1}),\ \bibinfo
  {pages} {71--78}}\BibitemShut {NoStop}%
\bibitem [{\citenamefont {Thorisson}(2000)}]{Tho00}%
  \BibitemOpen
  \bibfield  {author} {\bibinfo {author} {\bibnamefont {Thorisson},
  \bibfnamefont {Hermann}}} (\bibinfo {year} {2000}),\ \href@noop {} {\emph
  {\bibinfo {title} {Coupling, Stationarity, and Regeneration}}},\ Probability
  and its Applications (New York)\ (\bibinfo  {publisher}
  {Springer})\BibitemShut {NoStop}%
\bibitem [{\citenamefont {Tittel}\ \emph {et~al.}(1998)\citenamefont {Tittel},
  \citenamefont {Brendel}, \citenamefont {Gisin}, \citenamefont {Herzog},
  \citenamefont {Zbinden},\ and\ \citenamefont {Gisin}}]{Tittel}%
  \BibitemOpen
  \bibfield  {author} {\bibinfo {author} {\bibnamefont {Tittel}, \bibfnamefont
  {Wolfgang}}, \bibinfo {author} {\bibfnamefont {Jurgen}\ \bibnamefont
  {Brendel}}, \bibinfo {author} {\bibfnamefont {Bernard}\ \bibnamefont
  {Gisin}}, \bibinfo {author} {\bibfnamefont {Thomas}\ \bibnamefont {Herzog}},
  \bibinfo {author} {\bibfnamefont {Hugo}\ \bibnamefont {Zbinden}}, \ and\
  \bibinfo {author} {\bibfnamefont {Nicolas}\ \bibnamefont {Gisin}}} (\bibinfo
  {year} {1998}),\ \bibfield  {title} {\enquote {\bibinfo {title} {Experimental
  demonstration of quantum correlations over more than 10 km},}\ }\href
  {\doibase 10.1103/PhysRevA.57.3229} {\bibfield  {journal} {\bibinfo
  {journal} {Phys. Rev. A}\ }\textbf {\bibinfo {volume} {57}},\ \bibinfo
  {pages} {3229--3232}}\BibitemShut {NoStop}%
\bibitem [{\citenamefont {Tomamichel}\ and\ \citenamefont
  {Leverrier}(2017)}]{TL17}%
  \BibitemOpen
  \bibfield  {author} {\bibinfo {author} {\bibnamefont {Tomamichel},
  \bibfnamefont {Marco}}, \ and\ \bibinfo {author} {\bibfnamefont {Anthony}\
  \bibnamefont {Leverrier}}} (\bibinfo {year} {2017}),\ \bibfield  {title}
  {\enquote {\bibinfo {title} {A largely self-contained and complete security
  proof for quantum key distribution},}\ }\href {\doibase
  10.22331/q-2017-07-14-14} {\bibfield  {journal} {\bibinfo  {journal}
  {Quantum}\ }\textbf {\bibinfo {volume} {1}},\ \bibinfo {pages} {14}},\
  \Eprint {http://arxiv.org/abs/arXiv:1506.08458} {arXiv:1506.08458}
  \BibitemShut {NoStop}%
\bibitem [{\citenamefont {Tomamichel}\ \emph {et~al.}(2012)\citenamefont
  {Tomamichel}, \citenamefont {Lim}, \citenamefont {Gisin},\ and\ \citenamefont
  {Renner}}]{TLGR12}%
  \BibitemOpen
  \bibfield  {author} {\bibinfo {author} {\bibnamefont {Tomamichel},
  \bibfnamefont {Marco}}, \bibinfo {author} {\bibfnamefont {Charles Ci~Wen}\
  \bibnamefont {Lim}}, \bibinfo {author} {\bibfnamefont {Nicolas}\ \bibnamefont
  {Gisin}}, \ and\ \bibinfo {author} {\bibfnamefont {Renato}\ \bibnamefont
  {Renner}}} (\bibinfo {year} {2012}),\ \bibfield  {title} {\enquote {\bibinfo
  {title} {Tight finite-key analysis for quantum cryptography},}\ }\href
  {\doibase 10.1038/ncomms1631} {\bibfield  {journal} {\bibinfo  {journal}
  {Nat. Commun.}\ }\textbf {\bibinfo {volume} {3}},\ \bibinfo {pages} {634}},\
  \Eprint {http://arxiv.org/abs/arXiv:1103.4130} {arXiv:1103.4130} \BibitemShut
  {NoStop}%
\bibitem [{\citenamefont {Tomamichel}\ and\ \citenamefont
  {Renner}(2011)}]{TR11}%
  \BibitemOpen
  \bibfield  {author} {\bibinfo {author} {\bibnamefont {Tomamichel},
  \bibfnamefont {Marco}}, \ and\ \bibinfo {author} {\bibfnamefont {Renato}\
  \bibnamefont {Renner}}} (\bibinfo {year} {2011}),\ \bibfield  {title}
  {\enquote {\bibinfo {title} {Uncertainty relation for smooth entropies},}\
  }\href {\doibase 10.1103/PhysRevLett.106.110506} {\bibfield  {journal}
  {\bibinfo  {journal} {Phys. Rev. Lett.}\ }\textbf {\bibinfo {volume} {106}},\
  \bibinfo {pages} {110506}},\ \Eprint {http://arxiv.org/abs/arXiv:1009.2015}
  {arXiv:1009.2015} \BibitemShut {NoStop}%
\bibitem [{\citenamefont {Tomamichel}\ \emph {et~al.}(2010)\citenamefont
  {Tomamichel}, \citenamefont {Schaffner}, \citenamefont {Smith},\ and\
  \citenamefont {Renner}}]{TSSR10}%
  \BibitemOpen
  \bibfield  {author} {\bibinfo {author} {\bibnamefont {Tomamichel},
  \bibfnamefont {Marco}}, \bibinfo {author} {\bibfnamefont {Christian}\
  \bibnamefont {Schaffner}}, \bibinfo {author} {\bibfnamefont {Adam}\
  \bibnamefont {Smith}}, \ and\ \bibinfo {author} {\bibfnamefont {Renato}\
  \bibnamefont {Renner}}} (\bibinfo {year} {2010}),\ \bibfield  {title}
  {\enquote {\bibinfo {title} {Leftover hashing against quantum side
  information},}\ }in\ \href {\doibase 10.1109/ISIT.2010.5513652} {\emph
  {\bibinfo {booktitle} {Proceedings of the 2010 IEEE International Symposium
  on Information Theory, ISIT 2010}}}\ (\bibinfo  {publisher} {IEEE})\ pp.\
  \bibinfo {pages} {2703--2707},\ \Eprint
  {http://arxiv.org/abs/arXiv:1002.2436} {arXiv:1002.2436} \BibitemShut
  {NoStop}%
\bibitem [{\citenamefont {Unruh}(2004)}]{Unr04}%
  \BibitemOpen
  \bibfield  {author} {\bibinfo {author} {\bibnamefont {Unruh}, \bibfnamefont
  {Dominique}}} (\bibinfo {year} {2004}),\ \href@noop {} {\enquote {\bibinfo
  {title} {Simulatable security for quantum protocols},}\ }\bibinfo
  {howpublished} {e-Print},\ \Eprint
  {http://arxiv.org/abs/arXiv:quant-ph/0409125} {arXiv:quant-ph/0409125}
  \BibitemShut {NoStop}%
\bibitem [{\citenamefont {Unruh}(2010)}]{Unr10}%
  \BibitemOpen
  \bibfield  {author} {\bibinfo {author} {\bibnamefont {Unruh}, \bibfnamefont
  {Dominique}}} (\bibinfo {year} {2010}),\ \bibfield  {title} {\enquote
  {\bibinfo {title} {Universally composable quantum multi-party computation},}\
  }in\ \href {\doibase 10.1007/978-3-642-13190-5_25} {\emph {\bibinfo
  {booktitle} {Advances in Cryptology -- EUROCRYPT 2010}}},\ \bibinfo {series}
  {LNCS}, Vol.\ \bibinfo {volume} {6110}\ (\bibinfo  {publisher} {Springer})\
  pp.\ \bibinfo {pages} {486--505},\ \Eprint
  {http://arxiv.org/abs/arXiv:0910.2912} {arXiv:0910.2912} \BibitemShut
  {NoStop}%
\bibitem [{\citenamefont {Unruh}(2011)}]{Unr11}%
  \BibitemOpen
  \bibfield  {author} {\bibinfo {author} {\bibnamefont {Unruh}, \bibfnamefont
  {Dominique}}} (\bibinfo {year} {2011}),\ \bibfield  {title} {\enquote
  {\bibinfo {title} {Concurrent composition in the bounded quantum storage
  model},}\ }in\ \href {\doibase 10.1007/978-3-642-20465-4_26} {\emph {\bibinfo
  {booktitle} {Advances in Cryptology -- EUROCRYPT 2011}}},\ \bibinfo {series}
  {LNCS}, Vol.\ \bibinfo {volume} {6632}\ (\bibinfo  {publisher} {Springer})\
  pp.\ \bibinfo {pages} {467--486},\ \bibinfo {note} {e-Print
  \href{http://eprint.iacr.org/2010/229}{IACR 2010/229}}\BibitemShut {NoStop}%
\bibitem [{\citenamefont {Unruh}(2013)}]{Unr13}%
  \BibitemOpen
  \bibfield  {author} {\bibinfo {author} {\bibnamefont {Unruh}, \bibfnamefont
  {Dominique}}} (\bibinfo {year} {2013}),\ \bibfield  {title} {\enquote
  {\bibinfo {title} {Everlasting multi-party computation},}\ }in\ \href
  {\doibase 10.1007/978-3-642-40084-1_22} {\emph {\bibinfo {booktitle}
  {Advances in Cryptology -- CRYPTO 2013}}},\ \bibinfo {series} {LNCS}, Vol.\
  \bibinfo {volume} {8043}\ (\bibinfo  {publisher} {Springer})\ pp.\ \bibinfo
  {pages} {380--397},\ \bibinfo {note} {e-Print
  \href{http://eprint.iacr.org/2012/177}{IACR 2012/177}}\BibitemShut {NoStop}%
\bibitem [{\citenamefont {Unruh}(2014)}]{Unr14}%
  \BibitemOpen
  \bibfield  {author} {\bibinfo {author} {\bibnamefont {Unruh}, \bibfnamefont
  {Dominique}}} (\bibinfo {year} {2014}),\ \bibfield  {title} {\enquote
  {\bibinfo {title} {Quantum position verification in the random oracle
  model},}\ }in\ \href {\doibase 10.1007/978-3-662-44381-1_1} {\emph {\bibinfo
  {booktitle} {Advances in Cryptology -- CRYPTO 2014}}},\ \bibinfo {series}
  {LNCS}, Vol.\ \bibinfo {volume} {8617}\ (\bibinfo  {publisher} {Springer})\
  pp.\ \bibinfo {pages} {1--18},\ \bibinfo {note} {e-Print
  \href{http://eprint.iacr.org/2014/118}{IACR 2014/118}}\BibitemShut {NoStop}%
\bibitem [{\citenamefont {Vakhitov}\ \emph {et~al.}(2001)\citenamefont
  {Vakhitov}, \citenamefont {Makarov},\ and\ \citenamefont
  {Hjelme}}]{Vakhitov2001}%
  \BibitemOpen
  \bibfield  {author} {\bibinfo {author} {\bibnamefont {Vakhitov},
  \bibfnamefont {Artem}}, \bibinfo {author} {\bibfnamefont {Vadim}\
  \bibnamefont {Makarov}}, \ and\ \bibinfo {author} {\bibfnamefont {Dag~R.}\
  \bibnamefont {Hjelme}}} (\bibinfo {year} {2001}),\ \bibfield  {title}
  {\enquote {\bibinfo {title} {Large pulse attack as a method of conventional
  optical eavesdropping in quantum cryptography},}\ }\href {\doibase
  10.1080/09500340108240904} {\bibfield  {journal} {\bibinfo  {journal} {J.
  Mod. Opt.}\ }\textbf {\bibinfo {volume} {48}}~(\bibinfo {number} {13}),\
  \bibinfo {pages} {2023--2038}}\BibitemShut {NoStop}%
\bibitem [{\citenamefont {Vazirani}\ and\ \citenamefont {Vidick}(2012)}]{VV12}%
  \BibitemOpen
  \bibfield  {author} {\bibinfo {author} {\bibnamefont {Vazirani},
  \bibfnamefont {Umesh}}, \ and\ \bibinfo {author} {\bibfnamefont {Thomas}\
  \bibnamefont {Vidick}}} (\bibinfo {year} {2012}),\ \bibfield  {title}
  {\enquote {\bibinfo {title} {Certifiable quantum dice: or, true random number
  generation secure against quantum adversaries},}\ }in\ \href {\doibase
  10.1145/2213977.2213984} {\emph {\bibinfo {booktitle} {Proceedings of the
  44th Symposium on Theory of Computing, STOC~'12}}}\ (\bibinfo  {publisher}
  {ACM})\ pp.\ \bibinfo {pages} {61--76},\ \Eprint
  {http://arxiv.org/abs/arXiv:1111.6054} {arXiv:1111.6054} \BibitemShut
  {NoStop}%
\bibitem [{\citenamefont {Vazirani}\ and\ \citenamefont {Vidick}(2014)}]{VV14}%
  \BibitemOpen
  \bibfield  {author} {\bibinfo {author} {\bibnamefont {Vazirani},
  \bibfnamefont {Umesh}}, \ and\ \bibinfo {author} {\bibfnamefont {Thomas}\
  \bibnamefont {Vidick}}} (\bibinfo {year} {2014}),\ \bibfield  {title}
  {\enquote {\bibinfo {title} {Fully device-independent quantum key
  distribution},}\ }\href {\doibase 10.1103/PhysRevLett.113.140501} {\bibfield
  {journal} {\bibinfo  {journal} {Phys. Rev. Lett.}\ }\textbf {\bibinfo
  {volume} {113}},\ \bibinfo {pages} {140501}},\ \Eprint
  {http://arxiv.org/abs/arXiv:1210.1810} {arXiv:1210.1810} \BibitemShut
  {NoStop}%
\bibitem [{\citenamefont {Vernam}(1926)}]{Vernam26}%
  \BibitemOpen
  \bibfield  {author} {\bibinfo {author} {\bibnamefont {Vernam}, \bibfnamefont
  {Gilbert~S}}} (\bibinfo {year} {1926}),\ \bibfield  {title} {\enquote
  {\bibinfo {title} {Cipher printing telegraph systems for secret wire and
  radio telegraphic communications},}\ }\href@noop {} {\bibfield  {journal}
  {\bibinfo  {journal} {Trans. Am. Inst. Electr. Eng.}\ }\textbf {\bibinfo
  {volume} {XLV}},\ \bibinfo {pages} {295--301}}\BibitemShut {NoStop}%
\bibitem [{\citenamefont {Vilasini}\ \emph {et~al.}(2019)\citenamefont
  {Vilasini}, \citenamefont {Portmann},\ and\ \citenamefont {del
  Rio}}]{VPdR19}%
  \BibitemOpen
  \bibfield  {author} {\bibinfo {author} {\bibnamefont {Vilasini},
  \bibfnamefont {V}}, \bibinfo {author} {\bibfnamefont {Christopher}\
  \bibnamefont {Portmann}}, \ and\ \bibinfo {author} {\bibfnamefont {L\'idia}\
  \bibnamefont {del Rio}}} (\bibinfo {year} {2019}),\ \bibfield  {title}
  {\enquote {\bibinfo {title} {Composable security in relativistic quantum
  cryptography},}\ }\href {\doibase 10.1088/1367-2630/ab0e3b} {\bibfield
  {journal} {\bibinfo  {journal} {New J. Phys.}\ }\textbf {\bibinfo {volume}
  {21}}~(\bibinfo {number} {4}),\ \bibinfo {pages} {043057}},\ \Eprint
  {http://arxiv.org/abs/arXiv:1708.00433} {arXiv:1708.00433} \BibitemShut
  {NoStop}%
\bibitem [{\citenamefont {Wang}(2005)}]{Wang2005}%
  \BibitemOpen
  \bibfield  {author} {\bibinfo {author} {\bibnamefont {Wang}, \bibfnamefont
  {Xiang-Bin}}} (\bibinfo {year} {2005}),\ \bibfield  {title} {\enquote
  {\bibinfo {title} {Beating the photon-number-splitting attack in practical
  quantum cryptography},}\ }\href {\doibase 10.1103/PhysRevLett.94.230503}
  {\bibfield  {journal} {\bibinfo  {journal} {Phys. Rev. Lett.}\ }\textbf
  {\bibinfo {volume} {94}},\ \bibinfo {pages} {230503}},\ \Eprint
  {http://arxiv.org/abs/arxiv:quant-ph/0410075} {arxiv:quant-ph/0410075}
  \BibitemShut {NoStop}%
\bibitem [{\citenamefont {Watrous}(2018)}]{Wat18}%
  \BibitemOpen
  \bibfield  {author} {\bibinfo {author} {\bibnamefont {Watrous}, \bibfnamefont
  {John}}} (\bibinfo {year} {2018}),\ \href {\doibase 10.1017/9781316848142}
  {\emph {\bibinfo {title} {The Theory of Quantum Information}}}\ (\bibinfo
  {publisher} {Cambridge University Press})\ \bibinfo {note} {available at
  \url{http://cs.uwaterloo.ca/~watrous/TQI/}}\BibitemShut {NoStop}%
\bibitem [{\citenamefont {Webb}(2015)}]{Web15}%
  \BibitemOpen
  \bibfield  {author} {\bibinfo {author} {\bibnamefont {Webb}, \bibfnamefont
  {Zak}}} (\bibinfo {year} {2015}),\ \bibfield  {title} {\enquote {\bibinfo
  {title} {The {Clifford} group forms a unitary 3-design},}\ }\href@noop {}
  {\bibfield  {journal} {\bibinfo  {journal} {Quantum Inf. Comput.}\ }\textbf
  {\bibinfo {volume} {16}}~(\bibinfo {number} {15{\&}16}),\ \bibinfo {pages}
  {1379--1400}},\ \Eprint {http://arxiv.org/abs/arXiv:1510.02769}
  {arXiv:1510.02769} \BibitemShut {NoStop}%
\bibitem [{\citenamefont {Wegman}\ and\ \citenamefont {Carter}(1981)}]{WC81}%
  \BibitemOpen
  \bibfield  {author} {\bibinfo {author} {\bibnamefont {Wegman}, \bibfnamefont
  {Mark~N}}, \ and\ \bibinfo {author} {\bibfnamefont {Larry}\ \bibnamefont
  {Carter}}} (\bibinfo {year} {1981}),\ \bibfield  {title} {\enquote {\bibinfo
  {title} {New hash functions and their use in authentication and set
  equality},}\ }\href@noop {} {\bibfield  {journal} {\bibinfo  {journal} {J.
  Comput. Syst. Sci.}\ }\textbf {\bibinfo {volume} {22}}~(\bibinfo {number}
  {3}),\ \bibinfo {pages} {265--279}}\BibitemShut {NoStop}%
\bibitem [{\citenamefont {Wehner}\ \emph {et~al.}(2008)\citenamefont {Wehner},
  \citenamefont {Schaffner},\ and\ \citenamefont {Terhal}}]{WST08}%
  \BibitemOpen
  \bibfield  {author} {\bibinfo {author} {\bibnamefont {Wehner}, \bibfnamefont
  {Stephanie}}, \bibinfo {author} {\bibfnamefont {Christian}\ \bibnamefont
  {Schaffner}}, \ and\ \bibinfo {author} {\bibfnamefont {Barbara~M.}\
  \bibnamefont {Terhal}}} (\bibinfo {year} {2008}),\ \bibfield  {title}
  {\enquote {\bibinfo {title} {Cryptography from noisy storage},}\ }\href
  {\doibase 10.1103/PhysRevLett.100.220502} {\bibfield  {journal} {\bibinfo
  {journal} {Phys. Rev. Lett.}\ }\textbf {\bibinfo {volume} {100}},\ \bibinfo
  {pages} {220502}},\ \Eprint {http://arxiv.org/abs/arXiv:0711.2895}
  {arXiv:0711.2895} \BibitemShut {NoStop}%
\bibitem [{\citenamefont {Weier}\ \emph {et~al.}(2011)\citenamefont {Weier},
  \citenamefont {Krauss}, \citenamefont {Rau}, \citenamefont {F\"{u}rst},
  \citenamefont {Nauerth},\ and\ \citenamefont {Weinfurter}}]{WKRFNW11}%
  \BibitemOpen
  \bibfield  {author} {\bibinfo {author} {\bibnamefont {Weier}, \bibfnamefont
  {Henning}}, \bibinfo {author} {\bibfnamefont {Harald}\ \bibnamefont
  {Krauss}}, \bibinfo {author} {\bibfnamefont {Markus}\ \bibnamefont {Rau}},
  \bibinfo {author} {\bibfnamefont {Martin}\ \bibnamefont {F\"{u}rst}},
  \bibinfo {author} {\bibfnamefont {Sebastian}\ \bibnamefont {Nauerth}}, \ and\
  \bibinfo {author} {\bibfnamefont {Harald}\ \bibnamefont {Weinfurter}}}
  (\bibinfo {year} {2011}),\ \bibfield  {title} {\enquote {\bibinfo {title}
  {Quantum eavesdropping without interception: an attack exploiting the dead
  time of single-photon detectors},}\ }\href {\doibase
  10.1088/1367-2630/13/7/073024} {\bibfield  {journal} {\bibinfo  {journal}
  {New J. Phys.}\ }\textbf {\bibinfo {volume} {13}}~(\bibinfo {number} {7}),\
  \bibinfo {pages} {073024}},\ \Eprint {http://arxiv.org/abs/arXiv:1101.5289}
  {arXiv:1101.5289} \BibitemShut {NoStop}%
\bibitem [{\citenamefont {Weihs}\ \emph {et~al.}(1998)\citenamefont {Weihs},
  \citenamefont {Jennewein}, \citenamefont {Simon}, \citenamefont
  {Weinfurter},\ and\ \citenamefont {Zeilinger}}]{Weihs}%
  \BibitemOpen
  \bibfield  {author} {\bibinfo {author} {\bibnamefont {Weihs}, \bibfnamefont
  {Gregor}}, \bibinfo {author} {\bibfnamefont {Thomas}\ \bibnamefont
  {Jennewein}}, \bibinfo {author} {\bibfnamefont {Christoph}\ \bibnamefont
  {Simon}}, \bibinfo {author} {\bibfnamefont {Harald}\ \bibnamefont
  {Weinfurter}}, \ and\ \bibinfo {author} {\bibfnamefont {Anton}\ \bibnamefont
  {Zeilinger}}} (\bibinfo {year} {1998}),\ \bibfield  {title} {\enquote
  {\bibinfo {title} {Violation of {Bell}'s inequality under strict einstein
  locality conditions},}\ }\href {\doibase 10.1103/PhysRevLett.81.5039}
  {\bibfield  {journal} {\bibinfo  {journal} {Phys. Rev. Lett.}\ }\textbf
  {\bibinfo {volume} {81}},\ \bibinfo {pages} {5039--5043}}\BibitemShut
  {NoStop}%
\bibitem [{\citenamefont {Wiesner}(1983)}]{Wie83}%
  \BibitemOpen
  \bibfield  {author} {\bibinfo {author} {\bibnamefont {Wiesner}, \bibfnamefont
  {Stephen}}} (\bibinfo {year} {1983}),\ \bibfield  {title} {\enquote {\bibinfo
  {title} {Conjugate coding},}\ }\href@noop {} {\bibfield  {journal} {\bibinfo
  {journal} {SIGACT news}\ }\textbf {\bibinfo {volume} {15}}~(\bibinfo {number}
  {1}),\ \bibinfo {pages} {78--88}},\ \bibinfo {note} {original manuscript
  written circa 1969}\BibitemShut {NoStop}%
\bibitem [{\citenamefont {Winter}(2017)}]{Win17}%
  \BibitemOpen
  \bibfield  {author} {\bibinfo {author} {\bibnamefont {Winter}, \bibfnamefont
  {Andreas}}} (\bibinfo {year} {2017}),\ \bibfield  {title} {\enquote {\bibinfo
  {title} {Weak locking capacity of quantum channels can be much larger than
  private capacity},}\ }\href {\doibase 10.1007/s00145-015-9215-3} {\bibfield
  {journal} {\bibinfo  {journal} {J. Crypt.}\ }\textbf {\bibinfo {volume}
  {30}}~(\bibinfo {number} {1}),\ \bibinfo {pages} {1--21}},\ \Eprint
  {http://arxiv.org/abs/arXiv:1403.6361} {arXiv:1403.6361} \BibitemShut
  {NoStop}%
\bibitem [{\citenamefont {Wolf}(1999)}]{Wol99}%
  \BibitemOpen
  \bibfield  {author} {\bibinfo {author} {\bibnamefont {Wolf}, \bibfnamefont
  {Stefan}}} (\bibinfo {year} {1999}),\ \emph {\bibinfo {title}
  {Information-Theoretically and Computationally Secure Key Agreement in
  Cryptography}},\ \href@noop {} {Ph.D. thesis}\ (\bibinfo  {school} {Swiss
  Federal Institute of Technology (ETH) Zurich})\BibitemShut {NoStop}%
\bibitem [{\citenamefont {Wootters}\ and\ \citenamefont
  {Zurek}(1982)}]{Wootters82}%
  \BibitemOpen
  \bibfield  {author} {\bibinfo {author} {\bibnamefont {Wootters},
  \bibfnamefont {William~K}}, \ and\ \bibinfo {author} {\bibfnamefont
  {Wojciech~H.}\ \bibnamefont {Zurek}}} (\bibinfo {year} {1982}),\ \bibfield
  {title} {\enquote {\bibinfo {title} {A single quantum cannot be cloned},}\
  }\href {\doibase 10.1038/299802a0} {\bibfield  {journal} {\bibinfo  {journal}
  {Nature}\ }\textbf {\bibinfo {volume} {299}}~(\bibinfo {number} {5886}),\
  \bibinfo {pages} {802--803}}\BibitemShut {NoStop}%
\bibitem [{\citenamefont {Xu}\ \emph {et~al.}(2010)\citenamefont {Xu},
  \citenamefont {Qi},\ and\ \citenamefont {Lo}}]{XQL10}%
  \BibitemOpen
  \bibfield  {author} {\bibinfo {author} {\bibnamefont {Xu}, \bibfnamefont
  {Feihu}}, \bibinfo {author} {\bibfnamefont {Bing}\ \bibnamefont {Qi}}, \ and\
  \bibinfo {author} {\bibfnamefont {Hoi-Kwong}\ \bibnamefont {Lo}}} (\bibinfo
  {year} {2010}),\ \bibfield  {title} {\enquote {\bibinfo {title} {Experimental
  demonstration of phase-remapping attack in a practical quantum key
  distribution system},}\ }\href {\doibase 10.1088/1367-2630/12/11/113026}
  {\bibfield  {journal} {\bibinfo  {journal} {New J. Phys.}\ }\textbf {\bibinfo
  {volume} {12}}~(\bibinfo {number} {11}),\ \bibinfo {pages}
  {113026}}\BibitemShut {NoStop}%
\bibitem [{\citenamefont {Yin}\ \emph {et~al.}(2016)\citenamefont {Yin},
  \citenamefont {Chen}, \citenamefont {Yu}, \citenamefont {Liu}, \citenamefont
  {You}, \citenamefont {Zhou}, \citenamefont {Chen}, \citenamefont {Mao},
  \citenamefont {Huang}, \citenamefont {Zhang}, \citenamefont {Chen},
  \citenamefont {Li}, \citenamefont {Nolan}, \citenamefont {Zhou},
  \citenamefont {Jiang}, \citenamefont {Wang}, \citenamefont {Zhang},
  \citenamefont {Wang},\ and\ \citenamefont {Pan}}]{Yin2016}%
  \BibitemOpen
  \bibfield  {author} {\bibinfo {author} {\bibnamefont {Yin}, \bibfnamefont
  {Hua-Lei}}, \bibinfo {author} {\bibfnamefont {Teng-Yun}\ \bibnamefont
  {Chen}}, \bibinfo {author} {\bibfnamefont {Zong-Wen}\ \bibnamefont {Yu}},
  \bibinfo {author} {\bibfnamefont {Hui}\ \bibnamefont {Liu}}, \bibinfo
  {author} {\bibfnamefont {Li-Xing}\ \bibnamefont {You}}, \bibinfo {author}
  {\bibfnamefont {Yi-Heng}\ \bibnamefont {Zhou}}, \bibinfo {author}
  {\bibfnamefont {Si-Jing}\ \bibnamefont {Chen}}, \bibinfo {author}
  {\bibfnamefont {Yingqiu}\ \bibnamefont {Mao}}, \bibinfo {author}
  {\bibfnamefont {Ming-Qi}\ \bibnamefont {Huang}}, \bibinfo {author}
  {\bibfnamefont {Wei-Jun}\ \bibnamefont {Zhang}}, \bibinfo {author}
  {\bibfnamefont {Hao}\ \bibnamefont {Chen}}, \bibinfo {author} {\bibfnamefont
  {Ming~Jun}\ \bibnamefont {Li}}, \bibinfo {author} {\bibfnamefont {Daniel}\
  \bibnamefont {Nolan}}, \bibinfo {author} {\bibfnamefont {Fei}\ \bibnamefont
  {Zhou}}, \bibinfo {author} {\bibfnamefont {Xiao}\ \bibnamefont {Jiang}},
  \bibinfo {author} {\bibfnamefont {Zhen}\ \bibnamefont {Wang}}, \bibinfo
  {author} {\bibfnamefont {Qiang}\ \bibnamefont {Zhang}}, \bibinfo {author}
  {\bibfnamefont {Xiang-Bin}\ \bibnamefont {Wang}}, \ and\ \bibinfo {author}
  {\bibfnamefont {Jian-Wei}\ \bibnamefont {Pan}}} (\bibinfo {year} {2016}),\
  \bibfield  {title} {\enquote {\bibinfo {title}
  {Measurement-device-independent quantum key distribution over a 404 km
  optical fiber},}\ }\href {\doibase 10.1103/PhysRevLett.117.190501} {\bibfield
   {journal} {\bibinfo  {journal} {Phys. Rev. Lett.}\ }\textbf {\bibinfo
  {volume} {117}},\ \bibinfo {pages} {190501}},\ \Eprint
  {http://arxiv.org/abs/arXiv:1606.06821} {arXiv:1606.06821} \BibitemShut
  {NoStop}%
\bibitem [{\citenamefont {Yin}\ and\ \citenamefont {Chen}(2019)}]{YinChen}%
  \BibitemOpen
  \bibfield  {author} {\bibinfo {author} {\bibnamefont {Yin}, \bibfnamefont
  {Hua-Lei}}, \ and\ \bibinfo {author} {\bibfnamefont {Zeng-Bing}\ \bibnamefont
  {Chen}}} (\bibinfo {year} {2019}),\ \bibfield  {title} {\enquote {\bibinfo
  {title} {Finite-key analysis for twin-field quantum key distribution with
  composable security},}\ }\href@noop {} {\bibfield  {journal} {\bibinfo
  {journal} {Sci. Rep.}\ }\textbf {\bibinfo {volume} {9}}~(\bibinfo {number}
  {1}),\ \bibinfo {pages} {1--9}}\BibitemShut {NoStop}%
\bibitem [{\citenamefont {Yuan}\ \emph {et~al.}(2010)\citenamefont {Yuan},
  \citenamefont {Dynes},\ and\ \citenamefont {Shields}}]{Yuanetal2010}%
  \BibitemOpen
  \bibfield  {author} {\bibinfo {author} {\bibnamefont {Yuan}, \bibfnamefont
  {Zhiliang}}, \bibinfo {author} {\bibfnamefont {James~F.}\ \bibnamefont
  {Dynes}}, \ and\ \bibinfo {author} {\bibfnamefont {Andrew~J.}\ \bibnamefont
  {Shields}}} (\bibinfo {year} {2010}),\ \bibfield  {title} {\enquote {\bibinfo
  {title} {Avoiding the blinding attack in {QKD}},}\ }\href {\doibase
  10.1038/nphoton.2010.269} {\bibfield  {journal} {\bibinfo  {journal} {Nat.
  Photonics}\ }\textbf {\bibinfo {volume} {4}},\ \bibinfo {pages}
  {800}}\BibitemShut {NoStop}%
\bibitem [{\citenamefont {Zhandry}(2012)}]{Zha12}%
  \BibitemOpen
  \bibfield  {author} {\bibinfo {author} {\bibnamefont {Zhandry}, \bibfnamefont
  {Mark}}} (\bibinfo {year} {2012}),\ \bibfield  {title} {\enquote {\bibinfo
  {title} {How to construct quantum random functions},}\ }in\ \href {\doibase
  10.1109/FOCS.2012.37} {\emph {\bibinfo {booktitle} {Proceedings of the 53rd
  Symposium on Foundations of Computer Science, FOCS~'12}}}\ (\bibinfo
  {publisher} {IEEE})\ pp.\ \bibinfo {pages} {679--687},\ \bibinfo {note}
  {e-Print \href{http://eprint.iacr.org/2012/182}{IACR 2012/182}}\BibitemShut
  {NoStop}%
\bibitem [{\citenamefont {Zhao}\ \emph {et~al.}(2008)\citenamefont {Zhao},
  \citenamefont {Fung}, \citenamefont {Qi}, \citenamefont {Chen},\ and\
  \citenamefont {Lo}}]{Zhaoetal2008}%
  \BibitemOpen
  \bibfield  {author} {\bibinfo {author} {\bibnamefont {Zhao}, \bibfnamefont
  {Yi}}, \bibinfo {author} {\bibfnamefont {Chi-Hang~Fred}\ \bibnamefont
  {Fung}}, \bibinfo {author} {\bibfnamefont {Bing}\ \bibnamefont {Qi}},
  \bibinfo {author} {\bibfnamefont {Christine}\ \bibnamefont {Chen}}, \ and\
  \bibinfo {author} {\bibfnamefont {Hoi-Kwong}\ \bibnamefont {Lo}}} (\bibinfo
  {year} {2008}),\ \bibfield  {title} {\enquote {\bibinfo {title} {Quantum
  hacking: Experimental demonstration of time-shift attack against practical
  quantum-key-distribution systems},}\ }\href {\doibase
  10.1103/PhysRevA.78.042333} {\bibfield  {journal} {\bibinfo  {journal} {Phys.
  Rev. A}\ }\textbf {\bibinfo {volume} {78}},\ \bibinfo {pages}
  {042333}}\BibitemShut {NoStop}%
\bibitem [{\citenamefont {Zhu}(2017)}]{Zhu17}%
  \BibitemOpen
  \bibfield  {author} {\bibinfo {author} {\bibnamefont {Zhu}, \bibfnamefont
  {Huangjun}}} (\bibinfo {year} {2017}),\ \bibfield  {title} {\enquote
  {\bibinfo {title} {Multiqubit {Clifford} groups are unitary 3-designs},}\
  }\href {\doibase 10.1103/PhysRevA.96.062336} {\bibfield  {journal} {\bibinfo
  {journal} {Phys. Rev. A}\ }\textbf {\bibinfo {volume} {96}},\ \bibinfo
  {pages} {062336}},\ \Eprint {http://arxiv.org/abs/arXiv:1510.02619}
  {arXiv:1510.02619} \BibitemShut {NoStop}%
\bibitem [{\citenamefont {Zuckerman}(1990)}]{Zuc90}%
  \BibitemOpen
  \bibfield  {author} {\bibinfo {author} {\bibnamefont {Zuckerman},
  \bibfnamefont {David}}} (\bibinfo {year} {1990}),\ \bibfield  {title}
  {\enquote {\bibinfo {title} {General weak random sources},}\ }in\ \href
  {\doibase 10.1109/FSCS.1990.89574} {\emph {\bibinfo {booktitle} {Proceedings
  of the 31st Symposium on Foundations of Computer Science, FOCS~'90}}}\
  (\bibinfo  {publisher} {IEEE})\ pp.\ \bibinfo {pages} {534--543}\BibitemShut
  {NoStop}%
\end{thebibliography}%

\clearpage
\end{document}


