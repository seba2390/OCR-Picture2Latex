%%
%% This is file `sample-sigconf.tex',
%% generated with the docstrip utility.
%%
%% The original source files were:
%%
%% samples.dtx  (with options: `sigconf')
%% 
%% IMPORTANT NOTICE:
%% 
%% For the copyright see the source file.
%% 
%% Any modified versions of this file must be renamed
%% with new filenames distinct from sample-sigconf.tex.
%% 
%% For distribution of the original source see the terms
%% for copying and modification in the file samples.dtx.
%% 
%% This generated file may be distributed as long as the
%% original source files, as listed above, are part of the
%% same distribution. (The sources need not necessarily be
%% in the same archive or directory.)
%%
%%
%% Commands for TeXCount
%TC:macro \cite [option:text,text]
%TC:macro \citep [option:text,text]
%TC:macro \citet [option:text,text]
%TC:envir table 0 1
%TC:envir table* 0 1
%TC:envir tabular [ignore] word
%TC:envir displaymath 0 word
%TC:envir math 0 word
%TC:envir comment 0 0
%%
%%
%% The first command in your LaTeX source must be the \documentclass command.
\documentclass[sigconf]{acmart}

%%
%% \BibTeX command to typeset BibTeX logo in the docs
\AtBeginDocument{%
  \providecommand\BibTeX{{%
    \normalfont B\kern-0.5em{\scshape i\kern-0.25em b}\kern-0.8em\TeX}}}

\usepackage{CJK}
\usepackage{multirow}
\usepackage{amsmath}
\let\Bbbk\relax
\usepackage{amssymb}
% \usepackage{authblk}
% \fancyhead{}

%% Rights management information.  This information is sent to you
%% when you complete the rights form.  These commands have SAMPLE
%% values in them; it is your responsibility as an author to replace
%% the commands and values with those provided to you when you
%% complete the rights form.
% \setcopyright{acmcopyright}
% \copyrightyear{2021}
% \acmYear{2021}
% \acmDOI{10.1145/1122445.1122456}

%% These commands are for a PROCEEDINGS abstract or paper.
% \acmConference[Woodstock '18]{Woodstock '18: ACM Symposium on Neural
%   Gaze Detection}{June 03--05, 2018}{Woodstock, NY}
% \acmConference[MM '21]{2021 ACM Multimedia Conference}{October 20--24, 2021}{Chengdu, China}
% \acmBooktitle{Woodstock '18: ACM Symposium on Neural Gaze Detection,
%   June 03--05, 2018, Woodstock, NY}
% \acmPrice{15.00}
% \acmISBN{978-1-4503-XXXX-X/18/06}

\copyrightyear{2021}
\acmYear{2021}
\setcopyright{acmcopyright}\acmConference[MM '21]{Proceedings of the 29th ACM International Conference on Multimedia}{October 20--24, 2021}{Virtual Event, China}
\acmBooktitle{Proceedings of the 29th ACM International Conference on Multimedia (MM '21), October 20--24, 2021, Virtual Event, China}
\acmPrice{15.00}
\acmDOI{10.1145/3474085.3475442}
\acmISBN{978-1-4503-8651-7/21/10}

\settopmatter{printacmref=true}
\begin{document}
\fancyhead{}

%%
%% Submission ID.
%% Use this when submitting an article to a sponsored event. You'll
%% receive a unique submission ID from the organizers
%% of the event, and this ID should be used as the parameter to this command.
%%\acmSubmissionID{123-A56-BU3}

%%
%% The majority of ACM publications use numbered citations and
%% references.  The command \citestyle{authoryear} switches to the
%% "author year" style.
%%
%% If you are preparing content for an event
%% sponsored by ACM SIGGRAPH, you must use the "author year" style of
%% citations and references.
%% Uncommenting
%% the next command will enable that style.
%%\citestyle{acmauthoryear}

%%
%% end of the preamble, start of the body of the document source.
% \begin{document}

%%
%% The "title" command has an optional parameter,
%% allowing the author to define a "short title" to be used in page headers.
\title{Neural Free-Viewpoint Performance Rendering under Complex Human-object Interactions}

\address[AFFicrr]{Kamioka Observatory, Institute for Cosmic Ray Research, University of Tokyo, Kamioka, Gifu 506-1205, Japan}
\address[AFFkashiwa]{Research Center for Cosmic Neutrinos, Institute for Cosmic Ray Research, University of Tokyo, Kashiwa, Chiba 277-8582, Japan}
\address[AFFicrronly]{Institute for Cosmic Ray Research, University of Tokyo, Kashiwa, Chiba 277-8582, Japan}
\address[AFFmad]{Department of Theoretical Physics, University Autonoma Madrid, 28049 Madrid, Spain}
\address[AFFbcit]{Department of Physics, British Columbia Institute of Technology, Burnaby, BC, V5G 3H2, Canada }
\address[AFFubc]{Department of Physics and Astronomy, University of British Columbia, Vancouver, BC, V6T1Z4, Canada}
\address[AFFbu]{Department of Physics, Boston University, Boston, MA 02215, USA}
\address[AFFuci]{Department of Physics and Astronomy, University of California, Irvine, Irvine, CA 92697-4575, USA}
\address[AFFcsu]{Department of Physics, California State University, Dominguez Hills, Carson, CA 90747, USA}
\address[AFFcnm]{Institute for Universe and Elementary Particles, Chonnam National University, Gwangju 61186, Korea}
\address[AFFduke]{Department of Physics, Duke University, Durham NC 27708, USA}
\address[AFFllr]{Ecole Polytechnique, IN2P3-CNRS, Laboratoire Leprince-Ringuet, F-91120 Palaiseau, France}
\address[AFFfukuoka]{Junior College, Fukuoka Institute of Technology, Fukuoka, Fukuoka 811-0295, Japan}
\address[AFFgifu]{Department of Physics, Gifu University, Gifu, Gifu 501-1193, Japan}
\address[AFFgist]{GIST College, Gwangju Institute of Science and Technology, Gwangju 500-712, Korea}
\address[AFFuh]{Department of Physics and Astronomy, University of Hawaii, Honolulu, HI 96822, USA}
\address[AFFicl]{Department of Physics, Imperial College London , London, SW7 2AZ, United Kingdom}
\address[AFFifirse]{Institute For Interdisciplinary Research in Science and Education, Quy Nhon 55121, Binh Dinh, Vietnam. }
\address[AFFbari]{Dipartimento Interuniversitario di Fisica, INFN Sezione di Bari and Universit\`a e Politecnico di Bari, I-70125, Bari, Italy}
\address[AFFnapoli]{Dipartimento di Fisica, INFN Sezione di Napoli and Universit\`a di Napoli, I-80126, Napoli, Italy}
\address[AFFpadova]{Dipartimento di Fisica, INFN Sezione di Padova and Universit\`a di Padova, I-35131, Padova, Italy}
\address[AFFroma]{INFN Sezione di Roma and Universit\`a di Roma ``La Sapienza'', I-00185, Roma, Italy}
\address[AFFkeio]{Department of Physics, Keio University, Yokohama, Kanagawa, 223-8522, Japan}
\address[AFFkek]{High Energy Accelerator Research Organization (KEK), Tsukuba, Ibaraki 305-0801, Japan}
\address[AFFkcl]{Department of Physics, King's College London, London, WC2R 2LS, UK }
\address[AFFkobe]{Department of Physics, Kobe University, Kobe, Hyogo 657-8501, Japan}
\address[AFFkyoto]{Department of Physics, Kyoto University, Kyoto, Kyoto 606-8502, Japan}
\address[AFFliv]{Department of Physics, University of Liverpool, Liverpool, L69 7ZE, United Kingdom}
\address[AFFmiyagi]{Department of Physics, Miyagi University of Education, Sendai, Miyagi 980-0845, Japan}
\address[AFFnagoya]{Institute for Space-Earth Environmental Research, Nagoya University, Nagoya, Aichi 464-8602, Japan}
\address[AFFkmi]{Kobayashi-Maskawa Institute for the Origin of Particles and the Universe, Nagoya University, Nagoya, Aichi 464-8602, Japan}
\address[AFFpol]{National Centre For Nuclear Research, 02-093 Warsaw, Poland}
\address[AFFsuny]{Department of Physics and Astronomy, State University of New York at Stony Brook, NY 11794-3800, USA}
\address[AFFokayama]{Department of Physics, Okayama University, Okayama, Okayama 700-8530, Japan}
%%\address[AFFosaka]{Department of Physics, Osaka University, Toyonaka, Osaka 560-0043, Japan}
\address[AFFox]{Department of Physics, Oxford University, Oxford, OX1 3PU, United Kingdom}
%%\address[AFFqmul]{School of Physics and Astronomy, Queen Mary University of London, London, E1 4NS, United Kingdom}
%%\address[AFFregina]{Department of Physics, University of Regina, 3737 Wascana Parkway, Regina, SK, S4SOA2, Canada}
\address[AFFral]{Rutherford Appleton Laboratory, Harwell, Oxford, OX11 0QX, UK }
\address[AFFseoul]{Department of Physics, Seoul National University, Seoul 151-742, Korea}
\address[AFFsheff]{Department of Physics and Astronomy, University of Sheffield, S3 7RH, Sheffield, United Kingdom}
\address[AFFshizuokasc]{Department of Informatics in Social Welfare, Shizuoka University of Welfare, Yaizu, Shizuoka, 425-8611, Japan}
\address[AFFstfc]{STFC, Rutherford Appleton Laboratory, Harwell Oxford, and Daresbury Laboratory, Warrington, OX11 0QX, United Kingdom}
\address[AFFskk]{Department of Physics, Sungkyunkwan University, Suwon 440-746, Korea}
\address[AFFtohoku]{Department of Physics, Tohoku University, Aoba, Sendai 9808578, Japan}
\address[AFFtokai]{Department of Physics, Tokai University, Hiratsuka, Kanagawa 259-1292, Japan}
\address[AFFtokyo]{The University of Tokyo, Bunkyo, Tokyo 113-0033, Japan}
\address[AFFtodai]{Department of Physics, University of Tokyo, Bunkyo, Tokyo 113-0033, Japan}
\address[AFFipmu]{Kavli Institute for the Physics and Mathematics of the Universe (WPI), The University of Tokyo Institutes for Advanced Study, University of Tokyo, Kashiwa, Chiba 277-8583, Japan}
\address[AFFtit]{Department of Physics,Tokyo Institute of Technology, Meguro, Tokyo 152-8551, Japan}
\address[AFFtus]{Department of Physics, Faculty of Science and Technology, Tokyo University of Science, Noda, Chiba 278-8510, Japan}
\address[AFFtoronto]{Department of Physics, University of Toronto, ON, M5S 1A7, Canada }
\address[AFFtriumf]{TRIUMF, 4004 Wesbrook Mall, Vancouver, BC, V6T2A3, Canada }
\address[AFFtsinghua]{Department of Engineering Physics, Tsinghua University, Beijing, 100084, China}
\address[AFFwu]{Faculty of Physics, University of Warsaw, Warsaw, 02-093, Poland}
\address[AFFwarwick]{Department of Physics, University of Warwick, Coventry, CV4 7AL, UK }
\address[AFFwinnipeg]{Department of Physics, University of Winnipeg, MB R3J 3L8, Canada }
\address[AFFynu]{Department of Physics, Yokohama National University, Yokohama, Kanagawa, 240-8501, Japan}




%\AFFicrr
%\AFFkashiwa
%\AFFicrronly
%\AFFmad
%\AFFbu
%\AFFbcit
%\AFFuci
%\AFFcsu
%\AFFcnm
%\AFFduke
%\AFFllr
%\AFFfukuoka
%\AFFgifu
%\AFFgist
%\AFFuh
%\AFFicl
%\AFFbari
%\AFFnapoli
%\AFFpadova
%\AFFroma
%\AFFkeio
%\AFFkek
%\AFFkcl
%\AFFkobe
%\AFFkyoto
%\AFFliv
%\AFFmiyagi
%\AFFnagoya
%\AFFkmi
%\AFFpol
%\AFFsuny
%\AFFokayama
%\AFFox
%\AFFral
%\AFFseoul
%\AFFsheff
%\AFFshizuokasc
%\AFFstfc
%\AFFskk
%\AFFtokai
%\AFFtokyo
%\AFFtodai
%\AFFipmu
%\AFFtit
%\AFFtus
%\AFFtoronto
%\AFFtriumf
%\AFFtsinghua
%\AFFwu
%\AFFwarwick
%\AFFwinnipeg
%\AFFynu


%%%%%%%%%%%%%%%%%%%%%%%%%%%%%%%%%%%%%%%%%%%%%%%%%%%%%%%%%%%%%%%%%%%%
%ICRR
\author[AFFicrr,AFFipmu]{K.~Abe}
%\AFFicrr
%\AFFipmu
\author[AFFicrr]{C.~Bronner}
%\AFFicrr
\author[AFFicrr,AFFipmu]{Y.~Hayato}
%\AFFicrr
%\AFFipmu
\author[AFFicrr]{K.~Hiraide}
%\AFFicrr
\author[AFFicrr,AFFipmu]{M.~Ikeda}
\author[AFFicrr]{S.~Imaizumi}
%\AFFicrr
\author[AFFicrr,AFFipmu]{J.~Kameda}
%\AFFicrr
%\AFFipmu
\author[AFFicrr]{Y.~Kanemura}
%\AFFicrr
\author[AFFicrr]{Y.~Kataoka}
%\AFFicrr
\author[AFFicrr]{S.~Miki}
%\AFFicrr
\author[AFFicrr,AFFipmu]{M.~Miura} 
\author[AFFicrr,AFFipmu]{S.~Moriyama} 
%\AFFicrr
%\AFFipmu
\author[AFFicrr]{Y.~Nagao} 
%\AFFicrr
\author[AFFicrr,AFFipmu]{M.~Nakahata}
%\AFFicrr
%\AFFipmu
\author[AFFicrr,AFFipmu]{S.~Nakayama}
%\AFFicrr
%\AFFipmu
\author[AFFicrr]{T.~Okada}
\author[AFFicrr]{K.~Okamoto}
\author[AFFicrr]{A.~Orii}
\author[AFFicrr]{G.~Pronost}
%\AFFicrr
\author[AFFicrr,AFFipmu]{H.~Sekiya} 
\author[AFFicrr,AFFipmu]{M.~Shiozawa}
%\AFFicrr
%\AFFipmu 
\author[AFFicrr]{Y.~Sonoda}
\author[AFFicrr]{Y.~Suzuki} 
%\AFFicrr
\author[AFFicrr,AFFipmu]{A.~Takeda}
%\AFFicrr
%\AFFipmu
\author[AFFicrr]{Y.~Takemoto}
\author[AFFicrr]{A.~Takenaka}
%\AFFicrr 
\author[AFFicrr]{H.~Tanaka}
%\AFFicrr 
\author[AFFicrr]{S.~Watanabe}
%\AFFicrr
\author[AFFicrr]{T.~Yano}
%\AFFicrr 
%%%%%%%%%%%%%%%%%%%%%%%%%%%%%%%%%%%%%%%%%%%%%%%%%%%%%%%%%%%%%%%%%%%%%
%%Kashiwa
\author[AFFkashiwa]{S.~Han} 
%\AFFkashiwa
\author[AFFkashiwa,AFFipmu]{T.~Kajita} 
%\AFFkashiwa
%\AFFipmu
\author[AFFkashiwa,AFFipmu]{K.~Okumura}
%\AFFkashiwa
%\AFFipmu
\author[AFFkashiwa]{T.~Tashiro}
\author[AFFkashiwa]{J.~Xia}
%\AFFkashiwa

%%%%%%%%%%%%%%%%%%%%%%%%%%%%%%%%%%%%%%%%%%%%%%%%%%%%%%%%%%%%%%%%%%%%%
%%Kashiwa2
\author[AFFicrronly]{G.~D.~Megias}
%\AFFicrronly
%%%%%%%%%%%%%%%%%%%%%%%%%%%%%%%%%%%%%%%%%%%%%%%%%%%%%%%%%%%%%%%%%%%%%
%% Madrid
\author[AFFmad]{D.~Bravo-Bergu\~{n}o}
\author[AFFmad]{L.~Labarga}
\author[AFFmad]{Ll.~Marti}
\author[AFFmad]{B.~Zaldivar}
%\AFFmad
%%%%%%%%%%%%%%%%%%%%%%%%%%%%%%%%%%%%%%%%%%%%%%%%%%%%%%%%%%%%%%%%%%%%%
%% BCIT
\author[AFFbcit,AFFtriumf]{B.~W.~Pointon}
%\AFFbcit

%%%%%%%%%%%%%%%%%%%%%%%%%%%%%%%%%%%%%%%%%%%%%%%%%%%%%%%%%%%%%%%%%%%%%
%%Boston U
\author[AFFbu]{F.~d.~M.~Blaszczyk}
%\AFFbu
\author[AFFbu,AFFipmu]{E.~Kearns}
%\AFFbu
%\AFFipmu
\author[AFFbu]{J.~L.~Raaf}
%\AFFbu
\author[AFFbu,AFFipmu]{J.~L.~Stone}
%\AFFbu
%\AFFipmu
\author[AFFbu]{L.~Wan}
%\AFFbu
\author[AFFbu]{T.~Wester}
%\AFFbu
%%%%%%%%%%%%%%%%%%%%%%%%%%%%%%%%%%%%%%%%%%%%%%%%%%%%%%%%%%%%%%%%%%%%%
%%%%%%%%%%%%%%%%%%%%%%%%%%%%%%%%%%%%%%%%%%%%%%%%%%%%%%%%%%%%%%%%%%%%%
%%Irvine
\author[AFFuci]{J.~Bian}
\author[AFFuci]{N.~J.~Griskevich}
\author[AFFuci]{W.~R.~Kropp\footnote[1]{Deceased.}}
\author[AFFuci]{S.~Locke} 
\author[AFFuci]{S.~Mine} 
%\AFFuci
\author[AFFuci,AFFipmu]{M.~B.~Smy}
\author[AFFuci,AFFipmu]{H.~W.~Sobel} 
%\AFFuci
%\AFFipmu
\author[AFFuci,AFFipmu]{V.~Takhistov}
%\AFFuci
%\AFFipmu

%%%%%%%%%%%%%%%%%%%%%%%%%%%%%%%%%%%%%%%%%%%%%%%%%%%%%%%%%%%%%%%%%%%%%
%%CSU
\author[AFFcsu]{J.~Hill}
%\AFFcsu

%%%%%%%%%%%%%%%%%%%%%%%%%%%%%%%%%%%%%%%%%%%%%%%%%%%%%%%%%%%%%%%%%%%%%
%%Chonnam
\author[AFFcnm]{J.~Y.~Kim}
\author[AFFcnm]{I.~T.~Lim}
\author[AFFcnm]{R.~G.~Park}
%\AFFcnm

%%%%%%%%%%%%%%%%%%%%%%%%%%%%%%%%%%%%%%%%%%%%%%%%%%%%%%%%%%%%%%%%%%%%%
%%Duke
\author[AFFduke]{B.~Bodur}
%\AFFduke
\author[AFFduke,AFFipmu]{K.~Scholberg}
\author[AFFduke,AFFipmu]{C.~W.~Walter}
%\AFFduke
%\AFFipmu

%%%%%%%%%%%%%%%%%%%%%%%%%%%%%%%%%%%%%%%%%%%%%%%%%%%%%%%%%%%%%%%%%%%%%
%%LLR
\author[AFFllr]{L.~Bernard}
\author[AFFllr]{A.~Coffani}
\author[AFFllr]{O.~Drapier}
\author[AFFllr]{S.~El Hedri}
\author[AFFllr]{A.~Giampaolo}
\author[AFFllr]{M.~Gonin}
\author[AFFllr]{Th.~A.~Mueller}
\author[AFFllr]{P.~Paganini}
\author[AFFllr]{B.~Quilain}
%\AFFllr

%%%%%%%%%%%%%%%%%%%%%%%%%%%%%%%%%%%%%%%%%%%%%%%%%%%%%%%%%%%%%%%%%%%%%
%%Fukuoka
\author[AFFfukuoka]{T.~Ishizuka}
%\AFFfukuoka

%%%%%%%%%%%%%%%%%%%%%%%%%%%%%%%%%%%%%%%%%%%%%%%%%%%%%%%%%%%%%%%%%%%%%
%%Gifu U
\author[AFFgifu]{T.~Nakamura}
%\AFFgifu

%%%%%%%%%%%%%%%%%%%%%%%%%%%%%%%%%%%%%%%%%%%%%%%%%%%%%%%%%%%%%%%%%%%%%
%%Gwangju
\author[AFFgist]{J.~S.~Jang}
%\AFFgist

%%%%%%%%%%%%%%%%%%%%%%%%%%%%%%%%%%%%%%%%%%%%%%%%%%%%%%%%%%%%%%%%%%%%%
%%Hawaii U
\author[AFFuh]{J.~G.~Learned} 
%\AFFuh

%%%%%%%%%%%%%%%%%%%%%%%%%%%%%%%%%%%%%%%%%%%%%%%%%%%%%%%%%%%%%%%%%%%%%
%%ICL
\author[AFFicl]{L.~H.~V.~Anthony}
\author[AFFicl]{D.~Martin}
\author[AFFicl]{M.~Scott}
\author[AFFicl]{A.~A.~Sztuc} 
\author[AFFicl]{Y.~Uchida}
%\AFFicl

%%%%%%%%%%%%%%%%%%%%%%%%%%%%%%%%%%%%%%%%%%%%%%%%%%%%%%%%%%%%%%%%%%%%%
%%IFIRSE
\author[AFFifirse]{S.~Cao}
%\AFFifirse

%%%%%%%%%%%%%%%%%%%%%%%%%%%%%%%%%%%%%%%%%%%%%%%%%%%%%%%%%%%%%%%%%%%%%
%%BARI
\author[AFFbari]{V.~Berardi}
\author[AFFbari]{M.~G.~Catanesi}
\author[AFFbari]{E.~Radicioni}
%\AFFbari

%%%%%%%%%%%%%%%%%%%%%%%%%%%%%%%%%%%%%%%%%%%%%%%%%%%%%%%%%%%%%%%%%%%%%
%%NAPOLI
\author[AFFnapoli]{N.~F.~Calabria}
\author[AFFnapoli]{L.~N.~Machado}
\author[AFFnapoli]{G.~De Rosa}
%\AFFnapoli

%%%%%%%%%%%%%%%%%%%%%%%%%%%%%%%%%%%%%%%%%%%%%%%%%%%%%%%%%%%%%%%%%%%%%
%%PADOVA
\author[AFFpadova]{G.~Collazuol}
\author[AFFpadova]{F.~Iacob}
\author[AFFpadova]{M.~Lamoureux}
\author[AFFpadova]{M.~Mattiazzi}
\author[AFFpadova]{N.~Ospina}
%\AFFpadova

%%%%%%%%%%%%%%%%%%%%%%%%%%%%%%%%%%%%%%%%%%%%%%%%%%%%%%%%%%%%%%%%%%%%%
%%Roma
\author[AFFroma]{L.\,Ludovici}
%\AFFroma

%%%%%%%%%%%%%%%%%%%%%%%%%%%%%%%%%%%%%%%%%%%%%%%%%%%%%%%%%%%%%%%%%%%%%
%%Keio
\author[AFFkeio]{Y.~Maekawa}
\author[AFFkeio]{Y.~Nishimura}
%\AFFkeio

%%%%%%%%%%%%%%%%%%%%%%%%%%%%%%%%%%%%%%%%%%%%%%%%%%%%%%%%%%%%%%%%%%%%%
%%KEK
\author[AFFkek]{M.~Friend}
\author[AFFkek]{T.~Hasegawa} 
\author[AFFkek]{T.~Ishida} 
\author[AFFkek]{T.~Kobayashi} 
\author[AFFkek]{M.~Jakkapu}
\author[AFFkek]{T.~Matsubara}
\author[AFFkek]{T.~Nakadaira} 
%\AFFkek 
\author[AFFkek,AFFipmu]{K.~Nakamura}
%\AFFkek 
%\AFFipmu
\author[AFFkek]{Y.~Oyama} 
\author[AFFkek]{K.~Sakashita} 
\author[AFFkek]{T.~Sekiguchi} 
\author[AFFkek]{T.~Tsukamoto}
%\AFFkek 

%%%%%%%%%%%%%%%%%%%%%%%%%%%%%%%%%%%%%%%%%%%%%%%%%%%%%%%%%%%%%%%%%%%%%
%%KCL
\author[AFFkcl]{T.~Boschi}
\author[AFFkcl]{J.~Gao}
\author[AFFkcl]{F.~Di Lodovico}
\author[AFFkcl]{J.~Migenda}
\author[AFFkcl]{M.~Taani}
\author[AFFkcl]{S.~Zsoldos}
%\AFFkcl

%%%%%%%%%%%%%%%%%%%%%%%%%%%%%%%%%%%%%%%%%%%%%%%%%%%%%%%%%%%%%%%%%%%%%
%%Kobe U
\author[AFFkobe]{Y.~Kotsar}
\author[AFFkobe]{Y.~Nakano}
\author[AFFkobe]{H.~Ozaki}
\author[AFFkobe]{T.~Shiozawa}
%\AFFkobe
\author[AFFkobe]{A.~T.~Suzuki}
%\AFFkobe
\author[AFFkobe,AFFipmu]{Y.~Takeuchi}
%\AFFkobe
%\AFFipmu
\author[AFFkobe]{S.~Yamamoto}
%\AFFkobe

%%%%%%%%%%%%%%%%%%%%%%%%%%%%%%%%%%%%%%%%%%%%%%%%%%%%%%%%%%%%%%%%%%%%%
%%Kyoto
\author[AFFkyoto]{A.~Ali}
\author[AFFkyoto]{Y.~Ashida}
\author[AFFkyoto]{J.~Feng}
\author[AFFkyoto]{S.~Hirota}
\author[AFFkyoto]{T.~Kikawa}
\author[AFFkyoto]{M.~Mori}
%\AFFkyoto
\author[AFFkyoto,AFFipmu]{T.~Nakaya}
%\AFFkyoto
%\AFFipmu
\author[AFFkyoto,AFFipmu]{R.~A.~Wendell}
%\AFFkyoto
%\AFFipmu
\author[AFFkyoto]{K.~Yasutome}
%\AFFkyoto

%%%%%%%%%%%%%%%%%%%%%%%%%%%%%%%%%%%%%%%%%%%%%%%%%%%%%%%%%%%%%%%%%%%%%
%%Liverpool
\author[AFFliv]{P.~Fernandez}
\author[AFFliv]{N.~McCauley}
\author[AFFliv]{P.~Mehta}
\author[AFFliv]{K.~M.~Tsui}
%\AFFliv

%%%%%%%%%%%%%%%%%%%%%%%%%%%%%%%%%%%%%%%%%%%%%%%%%%%%%%%%%%%%%%%%%%%%%
%%Miyagi
\author[AFFmiyagi]{Y.~Fukuda}
%\AFFmiyagi

%%%%%%%%%%%%%%%%%%%%%%%%%%%%%%%%%%%%%%%%%%%%%%%%%%%%%%%%%%%%%%%%%%%%%
%%Nagoya
\author[AFFnagoya,AFFkmi]{Y.~Itow}
%\AFFnagoya
%\AFFkmi
\author[AFFnagoya]{H.~Menjo}
\author[AFFnagoya]{T.~Niwa}
\author[AFFnagoya]{K.~Sato}
%\AFFnagoya
\author[AFFnagoya]{M.~Tsukada}
%\AFFnagoya

%%%%%%%%%%%%%%%%%%%%%%%%%%%%%%%%%%%%%%%%%%%%%%%%%%%%%%%%%%%%%%%%%%%%%
%% POLAND
\author[AFFpol]{J.~Lagoda}
\author[AFFpol]{S.~M.~Lakshmi}
\author[AFFpol]{P.~Mijakowski}
\author[AFFpol]{J.~Zalipska}
%\AFFpol

%%%%%%%%%%%%%%%%%%%%%%%%%%%%%%%%%%%%%%%%%%%%%%%%%%%%%%%%%%%%%%%%%%%%%
%%SUNY
\author[AFFsuny]{J.~Jiang}
\author[AFFsuny]{C.~K.~Jung}
\author[AFFsuny]{C.~Vilela}
\author[AFFsuny]{M.~J.~Wilking}
\author[AFFsuny]{C.~Yanagisawa\footnote[2]{also at BMCC/CUNY, Science Department, New York, New York, 1007, USA.}}
%\alt{also at BMCC/CUNY, Science Department, New York, New York, 1007, USA.}
%\AFFsuny

%%%%%%%%%%%%%%%%%%%%%%%%%%%%%%%%%%%%%%%%%%%%%%%%%%%%%%%%%%%%%%%%%%%%%
%%Okayama U
\author[AFFokayama]{K.~Hagiwara}
\author[AFFokayama]{M.~Harada}
\author[AFFokayama]{T.~Horai}
\author[AFFokayama]{H.~Ishino}
\author[AFFokayama]{S.~Ito}
\author[AFFokayama]{F.~Kitagawa}
%\AFFokayama
\author[AFFokayama,AFFipmu]{Y.~Koshio}
%\AFFokayama
%\AFFipmu
\author[AFFokayama]{W.~Ma}
\author[AFFokayama]{N.~Piplani}
\author[AFFokayama]{S.~Sakai}
%\AFFokayama

%%%%%%%%%%%%%%%%%%%%%%%%%%%%%%%%%%%%%%%%%%%%%%%%%%%%%%%%%%%%%%%%%%%%%
%%Oxford
\author[AFFox]{G.~Barr}
\author[AFFox]{D.~Barrow}
%\AFFox
\author[AFFox,AFFipmu]{L.~Cook}
%\AFFox
%\AFFipmu
\author[AFFox,AFFipmu]{A.~Goldsack}
%\AFFox
%\AFFipmu
\author[AFFox]{S.~Samani}
%\AFFox
\author[AFFox,AFFstfc]{D.~Wark}
%\AFFox
%\AFFstfc

%%%%%%%%%%%%%%%%%%%%%%%%%%%%%%%%%%%%%%%%%%%%%%%%%%%%%%%%%%%%%%%%%%%%%
%%RAL
\author[AFFral]{F.~Nova}
%\AFFral


%%%%%%%%%%%%%%%%%%%%%%%%%%%%%%%%%%%%%%%%%%%%%%%%%%%%%%%%%%%%%%%%%%%%%
%%Seoul
\author[AFFseoul]{J.~Y.~Yang}
%\AFFseoul

%%%%%%%%%%%%%%%%%%%%%%%%%%%%%%%%%%%%%%%%%%%%%%%%%%%%%%%%%%%%%%%%%%%%%
%%Sheffield
\author[AFFsheff]{S.~J.~Jenkins}
\author[AFFsheff]{M.~Malek}
\author[AFFsheff]{J.~M.~McElwee}
\author[AFFsheff]{O.~Stone}
\author[AFFsheff]{M.~D.~Thiesse}
\author[AFFsheff]{L.~F.~Thompson}
%\AFFsheff

%%%%%%%%%%%%%%%%%%%%%%%%%%%%%%%%%%%%%%%%%%%%%%%%%%%%%%%%%%%%%%%%%%%%%
%%Shizuoka Seika College
\author[AFFshizuokasc]{H.~Okazawa}
%\AFFshizuokasc

%%%%%%%%%%%%%%%%%%%%%%%%%%%%%%%%%%%%%%%%%%%%%%%%%%%%%%%%%%%%%%%%%%%%%
%%SungKyunKwan
\author[AFFskk]{S.~B.~Kim}
\author[AFFskk]{J.~W.~Seo}
\author[AFFskk]{I.~Yu}
%\AFFskk

%%%%%%%%%%%%%%%%%%%%%%%%%%%%%%%%%%%%%%%%%%%%%%%%%%%%%%%%%%%%%%%%%%%%%
%%Tohoku U
\author[AFFtohoku]{A.~K.~Ichikawa}
\author[AFFtohoku]{K.~Nakamura}
%\AFFtohoku

%%%%%%%%%%%%%%%%%%%%%%%%%%%%%%%%%%%%%%%%%%%%%%%%%%%%%%%%%%%%%%%%%%%%%

%%%%%%%%%%%%%%%%%%%%%%%%%%%%%%%%%%%%%%%%%%%%%%%%%%%%%%%%%%%%%%%%%%%%%
%%Tokai U
\author[AFFtokai]{K.~Nishijima}
%\AFFtokai

%%%%%%%%%%%%%%%%%%%%%%%%%%%%%%%%%%%%%%%%%%%%%%%%%%%%%%%%%%%%%%%%%%%%%
%%Tokyo
\author[AFFtokyo]{M.~Koshiba\footnotemark[1]}
%\alt{Deceased.}
%\AFFtokyo

%%%%%%%%%%%%%%%%%%%%%%%%%%%%%%%%%%%%%%%%%%%%%%%%%%%%%%%%%%%%%%%%%%%%%
%%Tokyo, Department of Physics
\author[AFFtodai]{K.~Iwamoto}
\author[AFFtodai,AFFipmu]{Y.~Nakajima}
\author[AFFtodai]{N.~Ogawa}
%\AFFtodai
\author[AFFtodai,AFFipmu]{M.~Yokoyama}
%\AFFtodai
%\AFFipmu


%%%%%%%%%%%%%%%%%%%%%%%%%%%%%%%%%%%%%%%%%%%%%%%%%%%%%%%%%%%%%%%%%%%%%
%%IPMU

\author[AFFipmu]{K.~Martens}
%\AFFipmu
\author[AFFipmu,AFFuci]{M.~R.~Vagins}
%\AFFipmu
%\AFFuci

%%%%%%%%%%%%%%%%%%%%%%%%%%%%%%%%%%%%%%%%%%%%%%%%%%%%%%%%%%%%%%%%%%%%%
%%TIT
\author[AFFtit]{M.~Kuze}
\author[AFFtit]{S.~Izumiyama}
\author[AFFtit]{T.~Yoshida}
%\AFFtit

%%%%%%%%%%%%%%%%%%%%%%%%%%%%%%%%%%%%%%%%%%%%%%%%%%%%%%%%%%%%%%%%%%%%%
%%TUS
\author[AFFtus]{M.~Inomoto}
\author[AFFtus]{M.~Ishitsuka}
\author[AFFtus]{H.~Ito}
\author[AFFtus]{T.~Kinoshita}
\author[AFFtus]{R.~Matsumoto}
\author[AFFtus]{K.~Ohta}
\author[AFFtus]{M.~Shinoki}
\author[AFFtus]{T.~Suganuma}
%\AFFtus

%%%%%%%%%%%%%%%%%%%%%%%%%%%%%%%%%%%%%%%%%%%%%%%%%%%%%%%%%%%%%%%%%%%%%
%%Toronto
\author[AFFtoronto]{J.~F.~Martin}
\author[AFFtoronto]{H.~A.~Tanaka}
\author[AFFtoronto]{T.~Towstego}
%\AFFtoronto

%%%%%%%%%%%%%%%%%%%%%%%%%%%%%%%%%%%%%%%%%%%%%%%%%%%%%%%%%%%%%%%%%%%%%
%%Triumf
\author[AFFtriumf]{R.~Akutsu}
\author[AFFtriumf]{M.~Hartz}
\author[AFFtriumf]{A.~Konaka}
\author[AFFtriumf]{P.~de Perio}
\author[AFFtriumf]{N.~W.~Prouse}
%\AFFtriumf

%%%%%%%%%%%%%%%%%%%%%%%%%%%%%%%%%%%%%%%%%%%%%%%%%%%%%%%%%%%%%%%%%%%%%
%%Tshinghua U
\author[AFFtsinghua]{S.~Chen}
\author[AFFtsinghua]{B.~D.~Xu}
%\AFFtsinghua

%%%%%%%%%%%%%%%%%%%%%%%%%%%%%%%%%%%%%%%%%%%%%%%%%%%%%%%%%%%%%%%%%%%%%
%%Warsaw
\author[AFFwu]{M.~Posiadala-Zezula}
%\AFFwu

%%%%%%%%%%%%%%%%%%%%%%%%%%%%%%%%%%%%%%%%%%%%%%%%%%%%%%%%%%%%%%%%%%%%%
%%Warwick
\author[AFFwarwick]{D.~Hadley}
\author[AFFwarwick]{M.~O'Flaherty}
\author[AFFwarwick]{B.~Richards}
%\AFFwarwick

%%%%%%%%%%%%%%%%%%%%%%%%%%%%%%%%%%%%%%%%%%%%%%%%%%%%%%%%%%%%%%%%%%%%%
%%Winnipeg
\author[AFFwinnipeg]{B.~Jamieson}
\author[AFFwinnipeg]{J.~Walker}
%\AFFwinnipeg

%%%%%%%%%%%%%%%%%%%%%%%%%%%%%%%%%%%%%%%%%%%%%%%%%%%%%%%%%%%%%%%%%%%%%
%%Yokohama
\author[AFFynu]{A.~Minamino}
\author[AFFynu]{K.~Okamoto}
\author[AFFynu]{G.~Pintaudi}
\author[AFFynu]{S.~Sano}
\author[AFFynu]{R.~Sasaki}
%\AFFynu

%%%%%%%%%%%%%%%%%%%%%%%%%%%%%%%%%%%%%%%%%%%%%%%%%%%%%%%%%%%%%%%%%%%%%
\author{\\
(The Super-Kamiokande Collaboration)
}

% \footnote{$*$ Corresponding author.}

%%
%% The abstract is a short summary of the work to be presented in the
%% article.
% \begin{abstract}
%   A clear and well-documented \LaTeX\ document is presented as an
%   article formatted for publication by ACM in a conference proceedings
%   or journal publication. Based on the ``acmart'' document class, this
%   article presents and explains many of the common variations, as well
%   as many of the formatting elements an author may use in the
%   preparation of the documentation of their work.
% \end{abstract}
\begin{abstract}
    

%that another person, despite having different goals, is analogous to us - we both require food, exert energy and feel similar degrees of pain and pleasure. 
%agents are generally similar - e.g. 
%incur time step penalties, avoid dangerous features, or 
%such that the corresponding optimal action, when viewed through the agent’s own action-value function, matches the action taken by the other agent. 
%our architecture produces an empathetic representation of the other agent's observations that a


% Stephan
\iffalse
We can usually assume others have goals analogous to our own, and this assumption is useful for human cognitive processes like empathy where we can build on similarities between us and others to better understand the behaviour of others. Inspired by empathy we apply this process to multi-agent games by designing a simple and interpretable architecture to model another agent's action-value function. This involves learning an \emph{Imagination Network} to transform the other agent's observed state in order to produce a human-interpretable \emph{empathetic state} which, when presented to the learning agent, produces behaviours that mimic the other agent. Our approach is applicable to multi-agent scenarios consisting of a single learning agent and other (independent) agents acting according to fixed policies. This architecture is particularly beneficial for (but not limited to) algorithms using a composite value or reward function. We show our method produces better performance in multi-agent games, where it robustly estimates the other's model in different environment configurations. Additionally, we show that the empathetic states are human interpretable, and thus verifiable.
\fi

We can usually assume others have goals analogous to our own. This assumption can also, at times, be applied to multi-agent games - e.g. Agent 1's attraction to green pellets is analogous to Agent 2's attraction to red pellets. This ``analogy'' assumption is tied closely to the cognitive process known as empathy. Inspired by empathy, we design a simple and explainable architecture to model another agent's action-value function. This involves learning an \emph{Imagination Network} to transform the other agent's observed state in order to produce a human-interpretable \emph{empathetic state} which, when presented to the learning agent, produces behaviours that mimic the other agent. Our approach is applicable to multi-agent scenarios consisting of a single learning agent and other (independent) agents acting according to fixed policies. This architecture is particularly beneficial for (but not limited to) algorithms using a composite value or reward function. We show our method produces better performance in multi-agent games, where it robustly estimates the other's model in different environment configurations. Additionally, we show that the empathetic states are human interpretable, and thus verifiable.


%Getting along with someone else isn't always easy, but it is vital for a stable and productive society. Communication and understanding are key for smooth social interactions, leading to better cooperation. Luckily for us evolution has bestowed humans and animals the ability to empathise. Empathy tries to understand the feelings and goals of another, by using ourself as a point of reference. How would we feel if we were in a similar situation? What fears and goals do we share? What objects or goals, though not exactly the same, do we feel similarly towards? Despite our biological ability to empathise, understanding another is really tricky and we humans rely on a vast set of information sources to understand the other. These are not limited to verbal and written streams, but also encompass behaviours, body language and subtle verbal and phsyical cues, each of which are small clues to the others hidden state. Now if you thought empathy between humans was hard, try between agents! Or between a human and an agent! With the increasing prevalence of artifically intelligent agents, particularly in the form of robots or virtual agents, their presence in shared environments with other agents or humans will necessesitate the need to understand others. In this work we integrate an empathy-based architecture for modeling `the other' in artificial agents. We hone in on settings where two analagous agents share a space. Of these two, the learning agent (the one we train) tries to models the other `independent agent' by referencing its own rewards and value function. As there is yet no evidence to believe agents have `feelings', we instead utilise our architecture to understand the other's goals (articulate as a reward function). Our proposed architecture is a two stage neural network, the first of which constructs an empathetic representation of the independent agents state, before sending this representation into a copy of its own value function (the second network). In this way, our significant contribution is the ability to generate an empathetic state that is human interpretable, allowing a user to clearly see what features both agents share (those that elicit empathy). In turn, the model can be used to infer the underlying rewards  of the other agent which can be used to elicit cooperative behaviour in the learning agent.

\iffalse
Smooth social interactions are 
%driven by our ability to understand how others feel and behave. This is
enabled via mechanisms such as empathy, which allows us to understand how another is feeling by referencing our own emotions from similar situations. 
%Artificially intelligent systems that learn from interactions with other agents and humans are becoming increasingly common. 
As artificially intelligent systems become more prevalent in applications involving interactions with other agents and humans, it becomes increasingly important that they too exhibit such empathetic qualities. In this work we propose a novel empathy-based approach to enable a learning agent to understand the rewards and values of another agent (the independent agent) by referencing its own reward and value function. 
The empathy mechanism is realised through a two stage neural network architecture, the first of which reconstructs the independent agent's state from the learning agent's perspective, following which it is passed through the learning agent's value function network. 
%We train our learning agent to model the independent agent via an empathy-based architecture. 
%Applied to symmetric games (games in which the goals of each agent are similar, but not necessarily the same), our empathy mechanism allows us to model the independent agent's behaviours by grounding them in the learning agent's own reward and value functions. This in turn enables it to interact with the independent agent in a more context-informed manner.
Through this empathy-based approach, we (1) learn an estimate of the independent agent's reward function that is consistent with the scale of the learning agent's own reward function and (2) produce an empathetic representation of the independent agent's state that is grounded in the learning agent's own value function and experiences. %These aspects allow the learning agent to effectively account for the presence of other agents in its environment.
%the benefits of empathy are the ability to (1) infer a more comparable reward and value function for the independent agent, and (2) the ability to produce an empathetic representation of the independent agent's state as imagined by the learning agent.
%\thommen{Not necessarily human understandable, right? Should we say empathetic state representations?}
%\thommen{Points 1 and 2 are relative to the sympathy paper}\manisha{I don't understand what you mean here}\thommen{I mean that when you say 'more comparable reward and value function..', it automatically implies that this contribution is in relation to the sympathy paper - and at this point, the reader has no context of that paper. Maybe a better way to say it is that with empathy, we can model the independent agent's behaviors by grounding them in the learning agent's own reward and value functions, which may enable it to interact with the independent agent in a more context-informed manner. Specifically, our empathy-based framework (1) learns an estimate of the independent agent's reward function that is consistent with the scale of its own reward function and (2) produces representations of the independent agent's states that are grounded in its own value function and experiences(?). These aspects allow the learning agent to effectively account for the presence of other agents in its environment.} 
%interpretable (can we also say more accurate?) to the learning agent.}
We demonstrate the benefits of these estimates and representations in a variety of environments and show that our learning agent is able to behave considerately towards the independent agent whilst still completing its own task. 
\fi
\end{abstract}

%%
%% The code below is generated by the tool at http://dl.acm.org/ccs.cfm.
%% Please copy and paste the code instead of the example below.
%%
\begin{CCSXML}
<ccs2012>
   <concept>
       <concept_id>10010147.10010371.10010382.10010385</concept_id>
       <concept_desc>Computing methodologies~Image-based rendering</concept_desc>
       <concept_significance>500</concept_significance>
       </concept>
 </ccs2012>
\end{CCSXML}

\ccsdesc[500]{Computing methodologies~Image-based rendering}

%%
%% Keywords. The author(s) should pick words that accurately describe
%% the work being presented. Separate the keywords with commas.
\keywords{Human-object Interaction; Implicit Reconstruction; Dynamic Reconstruction; Neural Rendering}


%% A "teaser" image appears between the author and affiliation
%% information and the body of the document, and typically spans the
%% page.
\begin{teaserfigure}
  \centering
  \vspace{-5pt}
  \includegraphics[width=0.95\textwidth]{figures/teaser}
  \vspace{-10pt}
  \caption{Our approach achieves photo-realistic reconstruction results of human activities in novel views under challenging human-object interactions, using only six RGB cameras. (a) Capture setting. (b) Our results.}
  \label{fig:teaser}
  %\vspace{1pt}
\end{teaserfigure}

%%
%% This command processes the author and affiliation and title
%% information and builds the first part of the formatted document.
\maketitle

% {
% \renewcommand{\thefootnote}{\fnsymbol{footnote}}
% \footnotetext[2]{Corresponding author.}
% % \blfootnote{Corresponding author.}
% }

\begin{CJK}{UTF8}{gbsn}
    \section{Introduction}
\label{sec: intro}

% Operating safely in dynamic environments is crucial for autonomous robots in real-world scenarios. Existing control barrier functions for obstacle avoidance often assume point or circular robots, limiting their applicability to robots with more complex geometries. In this paper, we address this limitation by presenting an analytic approach to compute the distance between a polygonal robot and moving elliptical obstacles in a 2D environment. This distance computation is utilized in constructing a control barrier function for safe control synthesis, enabling the operation of a robot with a more intricate shape. Our proposed approach offers real-time tight elliptical obstacle avoidance for polygon-shaped robots. 

Obstacle avoidance in static and dynamic environments is a central challenge for safe mobile robot autonomy. 

At the planning level, several motion planning algorithms have been developed to provide a feasible path that ensures obstacle avoidance, including prominent approaches like A$^*$~\cite{A_star_planning}, RRT$^*$~\cite{RRT_star}, and their variants~\cite{informed_rrt_star, neural_rrt_star}. These algorithms typically assume that a low-level tracking controller can execute the planned path. However, in dynamic environments where obstacles and conditions change rapidly, reliance on such a controller can be limiting. A significant contribution to the field was made by Khatib \cite{potential-field}, who introduced artificial potential fields to enable collision avoidance during not only the motion planning stage but also the real-time control of a mobile robot. Later, Rimon and Koditschek \cite{navigation-function} developed navigation functions, a particular form of artificial potential functions that guarantees simultaneous collision avoidance and stabilization to a goal configuration.
%. These functions strive to ensure collision avoidance and stabilization towards a goal configuration simultaneously. 
% Meanwhile, Fox \cite{Fox1997TheDW} introduced the dynamics window concept, an influential approach to obstacle avoidance that proactively filters out unsafe control actions. 
In recent years, research has delved into the domain of trajectory generation and optimization, with innovative algorithms proposed for quadrotor safe navigation \cite{mellinger_snap_2011, zhou2019robust, tordesillas2019faster}. In parallel, the rise of learning-based approaches \cite{michels2005high, pfeiffer2018reinforced, loquercio2021learning} has added a new direction to the field, utilizing machine learning to facilitate both planning and real-time obstacle avoidance. Despite their promise, these methods often face challenges in dynamic environments and in providing safety guarantees.


In the field of safe control synthesis, integrating control Lyapunov functions (CLFs) and control barrier functions (CBFs) into a quadratic program (QP) has proven to be a reliable and efficient strategy for formulating safe stabilizing controls across a wide array of robotic tasks \cite{glotfelter2017nonsmooth, grandia_2021_legged, wang2017_aerial}. While CBF-based methodologies have been deployed for obstacle avoidance \cite{srinivasan2020synthesis, Long_learningcbf_ral21, almubarak2022safety, dawson2022learning, abdi2023safe}, such strategies typically simplify the robot as a point or circle and assume static environments when constructing CBFs for control synthesis. Some recent advances have also explored the use of time-varying CBFs to facilitate safe control in dynamic environments \cite{he2021rule, molnar2022safety, hamdipoor2023safe}. However, this concept has yet to be thoroughly investigated in the context of obstacle avoidance for rigid-body robots. For the safe autonomy of robot arms, Koptev \textit{et al}. \cite{Koptev2023_neural_joint_control} introduced a neural network approach to approximate the signed distance function of a robot arm and use it for safe reactive control in dynamic environments. In \cite{Hamatani2020arm}, a CBF construction formula is proposed for a robot arm with a static and circular obstacle. A configuration-aware control approach for the robot arm was proposed in \cite{ding2022configurationaware} by integrating geometric restrictions with CBFs. Thirugnanam \textit{et al}. \cite{discrete_polytope_cbf} introduced a discrete CBF constraint between polytopes and further incorporated the constraint in a model predictive control to enable safe navigation. The authors also extended the formulation for continuous-time systems in \cite{polytopic_cbf} but the CBF computation between polytopes is numerical, requiring a duality-based formulation with non-smooth CBFs. 

\subsubsection*{Notations}

The sets of non-negative real and natural numbers are denoted $\bbR_{\geq 0}$ and $\bbN$. For $N \in \bbN$, $[N] := \{1,2, \dots N\}$. The orientation of a 2D body is denoted by $0 \leq \theta < 2\pi$ for counter-clockwise rotation. We denote the corresponding rotation matrix as 
% \begin{equation}
% \label{eq: rotation}
    $\bfR(\theta) = \begin{bmatrix} \cos \theta & -\sin \theta \\ \sin \theta & \cos \theta \end{bmatrix}.$
% \end{equation}
The configuration of a 2D rigid-body is described by position and orientation, and the space of the positions and orientations in 2D is called the special Euclidean group, denoted as $SE(2)$. Also, we use $\|\bfx\|$ to denote the $L_2$ norm for a vector $\bfx$ and $\otimes$ to denote the Kronecker product. The gradient of a differentiable function $V$ is denoted by $\nabla V$, and its Lie derivative along a vector field $f$ by $\calL_f V  = \nabla V \cdot f$. A continuous function $\alpha: [0,a)\rightarrow [0,\infty )$ is of class $\calK$ if it is strictly increasing and $\alpha(0) = 0$. A continuous function $\alpha:\mathbb{R} \rightarrow \mathbb{R}$ is of extended class $\calK_{\infty}$ if it is of class $\calK$ and $\lim_{r \rightarrow \infty} \alpha(r) = \infty$. Lastly, consider the body-fixed frame of the ellipse $\calE'$. The signed distance function (SDF) of the ellipse $\psi_\calE: \mathbb{R}^2 \to \mathbb{R}$ is defined as 
\begin{equation}
%\label{eq: SDF}
    \psi_\calE(\bfp') 
    = \left\{
    \begin{array}{ll}
        d(\calE',\bfp'), & \text{if } \bfp' \in \calE^c,  \\
        -d(\calE',\bfp'), & \text{if } \bfp' \in \calE,
    \end{array} 
    \right. \notag
\end{equation}
where $d$ is the Euclidean distance. In addition, $\|\nabla \psi_\calE (\bfp')\| = 1$ for all $\bfp'$ except on the boundary of the ellipse and its center of mass, the origin.


\textbf{Contributions}: (i) We present an analytic distance formula in $SE(2)$ for elliptical and polygonal objects, enabling closed-form calculations for distance and its gradient. (ii) We introduce a novel time-varying control barrier function, specifically for rigid-body robots described by one or multiple $SE(2)$ configurations. Its efficacy of ensuring safe autonomy is demonstrated in ground robot navigation and multi-link robot arm problems. 


    
\section{Related Work}

  \begin{figure*}[t]
    \centering
    \includegraphics[width=\linewidth]{figures/pipeline}
    % \vspace{-12pt}
    \vspace{-20pt}
    \caption{The overview of our approach. Given the six RGB stream inputs surrounding the performer and objects, our approach generates high-quality human-object meshes and free-view rendering results. ``DR'' indicates differentiable rendering.
    }
    \vspace{-10pt}
    \label{fig:pipeline}
  \end{figure*}
    
  
\noindent{\textbf{Human Performance Capture.}}
Markerless human performance capture techniques have been widely investigated to achieve human free-viewpoint video or reconstruct the geometry. 
%
The high-end solutions~\cite{motion2fusion,TotalCapture,collet2015high,chen2019tightcap} adopt studio-setup with dense cameras to produce high-quality reconstruction and surface motion, but the synchronized and calibrated multi-camera systems are both difficult to deploy and expensive.
%
The recent low-end approaches~\cite{Xiang_2019_CVPR,LiveCap2019tog,chen2021sportscap, he2021challencap} enable light-weight performance capture under the single-view setup or even hand-held capture setup or drone-based capture setup~\cite{xu2017flycap}.
%
However, these methods require a naked human model or pre-scanned template. 
Volumetric fusion based methods~\cite{newcombe2015CVPR,DoubleFusion,BodyFusion,HybridFusion} enables free-form dynamic reconstruction. But they still suffer from careful and orchestrated motions, especially for a self-scanning process where the performer turns around carefully to obtain complete reconstruction. 
%
\cite{robustfusion} breaks self-scanning constraint by introducing implicit occupancy method.
%
 All these methods suffer from the limited mesh resolution leading to uncanny texturing output. Recent method~\cite{mustafa2020temporally} leverages unsupervised temporally coherent human reconstruction to generate free-viewpoint rendering. It is still hard for this method to get photo-realistic rendering results.
%
Comparably, our approach enables the high-fidelity capture of human-object interactions and eliminates the additional motion constraint under the sparse view RGB camera settings.


\noindent{\textbf{Neural Rendering.}}
The recent progress of differentiable neural rendering brings huge potential for 3D scene modeling and photo-realistic novel view synthesis. Researchers explore various data representations to pursue better performance and characteristics, such as point-clouds~\cite{Wu_2020_CVPR,aliev2019neural,suo2020neural3d}, voxels~\cite{lombardi2019neural}, texture meshes~\cite{thies2019deferred,liu2019neural} or implicit functions~\cite{park2019deepsdf,nerf,meng2021gnerf,chen2021mvsnerf,wang2021mirrornerf,luo2021convolutional}. 
%
However, these methods require inevitable pre-scene training to a new scene.
%
For neural modeling and rendering of dynamic scenes, NHR~\cite{Wu_2020_CVPR} embeds spatial features into sparse dynamic point-clouds, Neural Volumes~\cite{NeuralVolumes} transforms input images into a 3D volume representation by a VAE network.
% 
More recently, \cite{park2020deformable,pumarola2020d,li2020neural,xian2020space,tretschk2020non,peng2021neural,zhang2021editable} extend neural radiance field (NeRF)~\cite{nerf} into the dynamic setting. 
%
They learn a spatial mapping from the canonical scene to the current scene at each time step and regress the canonical radiance field. 
% 
However, for all the dynamic approaches above, dense spatial views or full temporal frames are required in training for high fidelity novel view rendering, leading to deployment difficulty and unacceptable training time overhead. Recent approaches~\cite{peng2021neural} and ~\cite{NeuralHumanFVV2021CVPR} adopt a sparse set of camera views to synthesize photo-realistic novel views of a performer. However, in the scenario of human-object interaction, these methods fail to generate both realistic performers and realistic objects.
Comparably, our approach explores the sparse capture setup and fast generates photo-realistic texture of challenging human-object interaction in novel views.

% \myparagraph{\textbf{Human-object capture}}
\noindent{\textbf{Human-object capture.}}
%
Early high-end work~\cite{collet2015high} captures both human and objects by reconstruction and rendering with dense cameras. 
%
Recently, several works explore the relation between human and scene to estimate 3D human pose and locate human position~\cite{hassan2019resolving,HPS,liu20204d}, naturally place human~\cite{PSI2019,PLACE:3DV:2020,hassan2021populating} or predict human motion~\cite{cao2020long}. 
%
Another related direction~\cite{GRAB:2020,hampali2021handsformer,liu2021semi} models the relationship between hand and objects for generation or capture.
%
PHOSA~\cite{2020phosa_Arrangements} runs human-object capture without any scene- or object-level 3D supervision using constraints to resolve ambiguity. 
However, they only recover the naked human bodies and produce a visually reasonable spatial arrangement.
%
A concurrent close work is RobustFusion(journal)~\cite{su2021robustfusion}. They capture human and objects by volumetric fusion respectively, and track object by Iterative Closest Point (ICP). 
However, their texturing quality is limited by mesh resolution and color representation, and the occluded region is ambiguous in 3D space.
%
Comparably, our approach enables photo-realistic novel view synthesis and accurate human object arrangement in 3D world space
under the human-object interaction for the light-weight sparse RGB settings.
    \section{Overview}
\label{sec:overview}
\subsection{Probing dark energy, gravity, and galaxy formation}
Measuring the anisotropic 3PCF can illuminate two of the most significant challenges in present-day  cosmology: dark energy's fundamental nature, and the complete theory of gravity. Dark energy drives accelerated expansion of the Universe. It constitutes 72\% of the Universe's current energy density, but its fundamental nature remains unknown. Unlike all other known substances, dark energy has negative pressure, and it is not predicted by the Standard Model of particle physics.

An alternative explanation for the accelerated expansion of the Universe is that the most widely accepted theory of gravity, General Relativity (GR), requires modification. Indeed, we already know that GR must be only an approximation to the true theory of gravity because GR cannot be unified with quantum field theory, which governs the interactions of the fundamental particles \cite{Copeland}.

The clustering of galaxies throughout space offers the largest possible laboratory  for pursuing these fundamental questions about the Universe's contents and governing laws. Galaxy clustering is often quantified via correlation functions. These measure the excess of pairs of galaxies as a function of the distance between the galaxies (2-point correlation function, 2PCF) or the excess of triplets as a function of triangle configuration (3PCF) compared to a random distribution \cite{Peebles}.

The 2PCF has been a highly successful tool in the past for exploring both dark energy and gravity.  In particular, 
the Baryon Acoustic Oscillation (BAO) method uses a sharp feature in the 2PCF  as a ``standard ruler'' \cite{Eisenstein1998}. Locating this feature in the 2PCF of galaxy samples from different epochs in the Universe's history enables the mapping of the Universe's expansion over time,  in turn illuminating the nature of dark energy \cite{Weinberg,Alam2016}.

Further, the growth rate of structure can be probed using the { \it{anisotropic}} (direction-dependent) 2PCF. This tracks the excess pairs of galaxies compared to a random distribution as a function of both the separation between the galaxies and the angle between the separation vector and the line of sight to the galaxy pair. 

While the underlying clustering of galaxies is independent of angle to the line of sight, the clustering {\it observed} in a redshift survey is modulated because of Redshift Space Distortions (RSD)~\cite{HamiltonRSD}. RSD occur because galaxies' own (``peculiar'') velocities with respect to the background expansion of the Universe affect our inference of their positions along the line of sight from their redshifts.  

Galaxies' velocities are generated by growth of structure due to gravity, and GR makes a particular prediction for the growth rate. Any measured deviation would offer a vital clue to the modifications required for a complete theory of gravity \cite{Lindergrowth}.

\subsection{Unlocking the anisotropic 3PCF's potential}
In principle, the 3PCF offers a new lever to understand both dark energy and gravity.  
\cite{Sefasutti2006} demonstrated that adding 3PCF information to an analysis of the 2PCF can improve constraints on the cosmological parameters that describe the nature of the universe. 
Measurements of the amount of dark energy growth and the rate of structure (which constrains gravity) can be improved by up to a factor of 2, and biasing, which describes how galaxies trace the underlying matter and is vital for understanding galaxy formation \cite{Fry1994,SERV,GilMarin2}, can be improved by a factor of $~$2 compared to using the 2PCF alone.
Like the 2PCF, the 3PCF has BAO features that can be used as a standard ruler to trace the expansion history \cite{SERSDmodel,SEBaoDetxn}.  And similarly to the anisotropic 2PCF, the anisotropic 3PCF contains valuable information on the growth rate. 

The anisotropic 3PCF depends on two triangle sides, their enclosed angle, and the angle of each side to the line of sight to the galaxy triplet (see Figure \ref{fig:alg_plus_eqn}). Consequently it has much richer structure than the anisotropic 2PCF, and offers many different configurations that all ultimately probe gravity \cite{RampfWong,Scoccimarro} as well as dark energy. 
It has never been measured. Not only would it provide additional information compared to the anisotropic 2PCF, but it would break significant degeneracies between the growth rate and other cosmological parameters that cannot be broken by the anisotropic 2PCF alone.

\begin{figure}
\centering
\includegraphics[width=\columnwidth]{Figures/updated_alg_fig_b.png}
\caption{Representation of a triangle configuration for the 3PCF. Each dot represents a galaxy. The anisotropic 3PCF depends on the vectors $\vec{r}_1$ and $\vec{r}_2$, the relative distances to the primary galaxy, which is at the bottom left vertex. The relevant quantities are the triangle side lengths $r_1$ and $r_2$, the angle between $\vec{r}_1$ and $\vec{r}_2$, and $\vec{r}_1$ and $\vec{r}_2$'s angles to the line of sight (dashed arrow). We expand the angular dependence of the anisotropic 3PCF in the basis of spherical harmonics, with the dependence on triangle side lengths $r_1$ and $r_2$ encoded in the radial coefficient $\zeta^m_{\ell \ell'}(r_1,r_2)$. The panel on the right (taken from~\cite{SE3ptalg}) shows a schematic of the algorithm's output: a coefficient $\zeta_{\ell \ell'}^m$ as a function of the triangle side lengths $r_1$ and $r_2$. The color indicates the number of triangles; red is an excess over a spatially random distribution and blue a deficit. The features are from BAO. 
\label{fig:alg_plus_eqn}}
\end{figure}


Finally, the 3PCF is a highly sensitive measure of how galaxies form. Galaxies are termed a ``biased'' tracer of the underlying dark matter density: they do not follow it with perfect fidelity, but rather form based on local conditions, such as the value of the local gravitational tidal forces or the relative velocity between regular matter and dark matter \cite{McDonald2009, TH2010}. 

 These effects enter at sub-leading order in the 2PCF, meaning one is searching for a small change to a large baseline signal. However, in the 3PCF, these effects are at leading order, making the 3PCF a uniquely powerful probe of galaxy biasing \cite{SERV,SERSDmodel}. 

 
\subsection{The 3PCF is computationally demanding}
Unfortunately, in practice, the 3PCF has rarely been used to constrain models of cosmology: counting all possible triangles formed by galaxy triplets in a modern redshift survey is combinatorially explosive. It naively scales as $\mathcal{O}(N^3)$ compared to $\mathcal{O}(N^2)$ scaling for the 2PCF, where $N$ is the number of galaxies in the survey. The 3PCF and bispectrum (its Fourier-space analog) have had other uses, but given this combinatoric challenge have been restricted to only particular triangle configurations (e.g. isosceles) \cite{McBride, GilMarin1,GilMarin2}.  
Recent work with an algorithm \cite{SE3ptalg} related to the one used in this work has measured many triangle configurations for the spherically-averaged (isotropic) 3PCF \cite{SEBaoDetxn,SERV}, but the full anisotropic dependence of the 3PCF has never been measured. We note that the anisotropic 3PCF contains the isotropic 3PCF as a proper subset. 


Until now no algorithm has existed to measure the full anisotropic 3PCF for all triangles, on large scales, and in million- to billion-galaxy surveys. In this work, we develop and implement an algorithm that measures the anisotropic 3PCF in $\mathcal{O}(N^2)$ time. Galactos, our high-performance parallel computing anisotropic 3PCF code will, for the first time, make the measurement of the 3PCF feasible for modern astronomical surveys. The mathematical framework of the algorithm and other details relevant for cosmology are presented more fully in a companion paper ~\cite{SE3ptAniso}.
Galactos' single node performance has been highly optimized for Intel Xeon Phi, achieving 39\%\ of peak single-node performance with efficient use of vectorization and the full memory hierarchy. 
Galactos presents almost perfect weak- and strong-scaling, and achieves 5.06~PF across 9636 nodes.

The only required input is the 3-D positions of the galaxies, which is already demanded by the 2PCF.  Thus, for zero additional cost in telescope time, our algorithmic and computing advances will yield significant additional insight on the most fundamental questions facing cosmology.











    \section{Method}\label{sec:method}
%

\subsection{Interaction-aware Human-Object Capture}\label{sec:human_capture}
Classical multi-view stereo reconstruction approaches \citep{Furukawa2013,Strecha2008,Newcombe2011,collet2015high} and recent neural rendering approaches \citep{Wu_2020_CVPR,NeuralVolumes,nerf} rely on multi-view dome based setup to achieve high-fidelity reconstruction and rendering results.
%
However, they suffer from both sparse-view inputs and occlusion of objects.
%
To this end, we propose a novel implicit human-object capture scheme to model the mutual influence between human and object from only sparse-view RGB inputs.

\noindent{\textbf{(a) Occlusion-aware Implicit Human Reconstruction.}}
For the human reconstruction, we perform a neural implicit geometry generation to jointly utilize both the pixel-aligned image features and global human motion priors with the aid of an occlusion-aware training data augmentation.
% 

Without dense RGB cameras and depth cameras, traditional multi-view stereo approaches \citep{collet2015high,motion2fusion} and depth-fusion approaches \citep{KinectFusion,UnstructureLan,robustfusion} can hardly reconstruct high-quality human meshes.
%
With implicit function approaches \citep{PIFU_2019ICCV,PIFuHD}, we can generate fine-detailed human meshes with sparse-view RGB inputs.
%
However, the occlusion from human-object interaction can still cause severe artifacts.
%
To end this, we thus utilize the pixel-aligned image features and global human motion priors.

% 
Specifically, we adopt the off-the-shelf instance segmentation approach \citep{Bolya_2019_ICCV} to obtain human and object masks, thus distinguishing the human and object separately from the sparse-view RGB input streams.
%
Meanwhile, we apply the parametric model estimation to provide human motion priors for our implicit human reconstruction.
%
We voxelized the mesh of this estimated human model to represent it with a volume field.

We give both the pixel-aligned image features and global human motion priors in volume representation to two different encoders of our implicit function, as shown in Fig. \ref{fig:pipeline} (a).
%
Different from \cite{2020phosa_Arrangements} with only a single RGB input, we use pixel-aligned image features from the multi-view inputs and concatenate them with our encoded voxel-aligned features.
%
We finally decode the pixel-aligned and voxel-aligned feature to occupancy values with a multilayer perceptron (MLP).

For each query 3D point $P$ on the volume grid, we follow PIFu \citep{saito2019pifu} to formulate the implicit function $f$ as:
\begin{align}
	f( \Phi(P),\Psi(P),Z(P)) & = \sigma : \sigma \in [0.0, 1.0],             \\
	\Phi(P)                  & = \frac{1}{n} \sum_{i}^{n}F_{I_{i}}(\pi_{i}(P)), \\
	\Psi(P)                  & =  G(F_{V},P),
\end{align}
where $p = \pi_{i}(P)$ denotes the projection of 3D point to camera view $i$, $F_{I_{i}}(x)= g(I_{i}(p))$ is the image feature at $p$.
%
$\Psi(P) = G(F_{V},P)$ denotes the voxel aligned features at $P$, $F_{V}$ is the voxel feature.
%
To better deal with occlusion, we introduce an occlusion-aware reconstruction loss to enhance the prediction at the occluded part of human.
% 
It is formulated as:
	\begin{align}
		 & \mathcal{L}_{\sigma} = \lambda_{occ}\sum_{t=1}^T \left\| \sigma_{occ}^{gt} - \sigma_{occ}^{pred} \right\|_2^2 + \lambda_{vis}\sum_{t=1}^T \left\| \sigma_{vis}^{gt} - \sigma_{vis}^{pred} \right\|_2^2.
	\end{align}

% 
Here, $\lambda_{occ}$ and $\lambda_{vis}$ represent the weight of occlusion points and visible points, respectively.
%
$\sigma_{occ}$ and $\sigma_{vis}$ are the training sampling points at the occlusion area and visible area.


\begin{figure}[t]
    \centering
    \includegraphics[width=\linewidth]{figures/data_augmentation}
    \vspace{-10pt}
    \caption{Illustration of our synthetic 3D data with both human and objects.}
    % \vspace{-1mm}
    \vspace{-15pt}
    \label{fig:DataAugmentation}
\end{figure}

\begin{figure*}[t]
	\centering
	\includegraphics[width=\linewidth]{figures/pipeline_net}
	\caption{Illustration of our layered human-object rendering approach, which not only includes a direction-aware neural texture blending scheme to encode the occlusion information explicitly but also adopts a spatial-temporal texture completion for the occluded regions based on the human motion priors.}
	\vspace{-10pt}
	\label{fig:pipeline_net}
\end{figure*}

For the detail of the parametric model estimation, we fit the parametric human model, SMPL \citep{SMPL2015}, to capture occluded human with the predicted 2D keypoints.
%
Specifically, we use Openpose \citep{Openpose} as our joint detector to estimate 2D human keypoints from sparse-view RGB inputs.
%
To estimate the pose/shape parameters of SMPL as our human prior for occluded human, we formulate the energy function $\boldsymbol{E}_{\mathrm{prior}}$ of this optimization as:
\begin{align} \label{eq:opt}
	\boldsymbol{E}_{\mathrm{prior}}(\boldsymbol{\theta}_t, \boldsymbol{\beta}) = \boldsymbol{E}_{\mathrm{2D}} + \lambda_{\mathrm{T}}\boldsymbol{E}_{\mathrm{T}}
\end{align}
% 
Here, $\boldsymbol{E}_{\mathrm{2D}}$ represents the re-projection constraint on 2D keypoints detected from sparse-view RGB inputs, while $\boldsymbol{E}_{\mathrm{T}}$ enforces the final pose and shape to be temporally smooth.
%
$\boldsymbol{\theta}_t$ is the pose parameters of frame $t$, while $\boldsymbol{\beta}$ is the shape parameters.
%
Note that this temporal smoothing enables globally consistent capture during the whole sequence, and benefits the parametric model estimation when some part of the body is gradually occluded.
%
We follow \cite{he2021challencap} to formulate the 2D term $\boldsymbol{E}_{\mathrm{2D}}$ and the temporal term $\boldsymbol{E}_{\mathrm{T}}$ under the sparse-view setting.
%

Moreover, we apply an occlusion-aware data augmentation to reduce the domain gap between our training set and the challenging human-object interaction testing set.
%
Specially, we randomly sample some objects from ShapeNet dataset~\cite{chang2015shapenet}.
%
We then randomly rotate and place them around human before training, as shown in Fig. \ref{fig:DataAugmentation}.
%
By simulating the occlusion of human-object interaction, our network is more robust to occluded human features.

With both the pixel-aligned image features and the statistical human motion priors under this occlusion-aware data augmentation training, our implicit function generates high-quality human meshes with only spare RGB inputs and occlusions from human-object interaction.

\noindent{\textbf{(b) Human-aware Object Tracking.}}
%
For the objects around the human, people recover them from depth maps~\cite{new2011kinect}, implicit fields~\cite{mescheder2019occupancy}, or semantic parts~\cite{chen2018autosweep}. We perform a template-based object alignment for the first frame and human-aware tracking to maintain temporal consistency and prevent the segmentation uncertainty caused by interaction. With the inspiration of PHOSA \cite{2020phosa_Arrangements}, we consider each object as a rigid body mesh.


To faithfully and robustly capture object in 3D space as time going, we introduce a human-aware tracking method.
%
Expressly, we assume objects are rigid bodies and transforming rigidly in the human-object interaction activities.
%
So the object mesh $O_{t}$ at frame $t$ can be represented as: $O_{t} = R_{t}O_{t-1}+T_{t}$.
% 
Based on the soft rasterization rendering~\cite{ravi2020accelerating}, the rotation $R_{t}$ and the translation $T_{t}$ can be naively optimized by comparing $\mathcal{L}_{2}$ norm between the rendered silhouette $S_{t}^{i}$ and object mask $\mathcal{M}o_{t}^{i}$.
%

Human is also an important cue to locate the object position.
%
From the 2D perspective, when objects are occluded by the human at a camera view, the  $\mathcal{L}_{2}$ loss between rendered silhouette and occluded mask will lead to the wrong object location due to the wrong guidelines at the occluded area.
%
So we remove the occluded area affected by human mask $\mathcal{M}h_{t}^{i}$ when computing the $\mathcal{L}_{2}$ loss.
%
From the 3D perspective, human can not interpenetrate an rigid object, so we also add an interpenetration loss $\mathcal{L}_{P}$~\cite{jiang2020mpshape} to regularize optimization. Our total object tracking loss is:
\begin{align}
	\mathcal{L}_{track} = \lambda_{1}\sum_{i=0}^{n}\| \mathcal{B}(\mathcal{M}h_{t}^{i}==0) \odot  S_{t}^{i} - \mathcal{M}o_{t}^{i}  \|  + \lambda_{2}\mathcal{L}_{P},
\end{align}
where n denotes view numbers, $\lambda_{1}$ denotes weight of silhouette loss, $\lambda_{2}$ denotes weight of interpenetration loss, $\mathcal{B}$ represents an binary operation, it returns 0 when the condition is true, else 1.
% 	}

Our implicit human-object capture utilizes both the pixel-aligned image features and global human motion priors with the aid of an occlusion-aware training data augmentation, and captures objects with the template-based alignment and the human-aware tracking to maintain temporal consistency and prevent the segmentation uncertainty caused by interaction. Thus, our approaches can generate high-quality human-object geometry with sparse inputs and occlusions.

\subsection{Layered Human-Object Rendering}\label{sec:rendering}
We introduce a neural human-object rendering pipeline to encode local fine-detailed human geometry and texture features from adjacent input views, so as to produce photo-realistic layered output in the target view, as illustrated in Fig. \ref{fig:pipeline_net}.

\begin{figure*}[t]
	\centering
	\includegraphics[width=\linewidth]{figures/gallery}
	\vspace{-20pt}
	\caption{The geometry and texture results of our proposed approach, which generates photo-realistic rendering results and high fidelity geometry on a various of sequences, such as rolling a box, playing with balls.}
    \vspace{-10pt}
	\label{fig:gallery}
\end{figure*}

\noindent{\textbf{(c) Direction-aware Neural Texture Blending.}} \label{sec:neuralBlending}
%
While traditional image-based rendering approaches always show the artifacts with the sparse-view texture blending, we follow \cite{NeuralHumanFVV2021CVPR} to propose a direction-aware neural texture blending approach to render photo-realistic human in the novel view.
% %
For a novel view image $I_{n}$, the linear combination of two source view $I_{1}$ and $I_{2}$ with blending weight map $W$ is formulate as:
\begin{align}
	I_{n} = W \cdot I_{1} + (1 - W) \cdot I_{2}.
\end{align}
However, in the sparse-view setting, the neural blending approach \citep{NeuralHumanFVV2021CVPR} can still generate unsmooth results. As the reason of these artifacts, the imbalance of angles between two source views with a novel view will lead to the imbalance wrapped image quality.
%

Different from \citet{NeuralHumanFVV2021CVPR}, we thus propose a direction-aware neural texture blending to eliminate such artifacts, as shown in Fig. \ref{fig:pipeline_net}.
%
The direction and angle between the two source views and target view will be an important cue for neural rendering quality, especially under occluded scenarios. 
%
Given novel view depth $D_{n}$ and source view depth $D_{1}$, $D_{2}$, we wrap them to the novel view $D_{1}^{n}$ and $D_{2}^{n}$, then compute the occlusion map $O_{i} = D_{n}- D_{i}^{n} (i=1,2)$.
%
Then, we unproject $D_{i}$ to point-clouds.
%
For each point $P$, we compute the cosine value between $\overrightarrow{c_{i}P}$ and $\overrightarrow{c_{n}P}$ to get angle map $A_{i}$, where $c_{i}$ denotes the optical center of source camera $i$, $c_{n}$ denotes the optical center of novel view camera.
%
Thus, we introduce a direction-aware blending network $\Theta_{DAN}$ to utilize global feature from image and local feature from human geometry to generate the blending weight map $W$, which can be formulated as:
\begin{align}
	W = \Theta_{DAN}(I_{1},O_{1},A_{1},I_{2},O_{2},A_{2}),
\end{align}
% 

\noindent{\textbf{(d) Spatial-temporal Texture Completion.}}
%
While human-object interaction activities consistently lead to occlusion, the missing texture on human, therefore, leads to severe artifacts for free-viewpoint rendering.
%
To end this, we propose a spatial-temporal texture completion method to generate a texture-completed proxy in the canonical human space.
%
We use the temporal and spatial information to complete the missing texture at view $i$ and time $t$ from different times and different views.

Specifically, we first use the non-rigid deformation to register an up-sampled SMPL model (41330 vertices) with the captured human meshes.
%	
Then, for each point on the proxy, we find the nearest visible points along with all views and all frames, then assign an interpolation color to this point.
% 
We thus generate a canonical human space with the fused texture.
%
For the occluded part of human in novel view, we render the texture-completed image and blend it with the neural rendering results in Sec. \ref{sec:neuralBlending} (c).

We utilize a layered human-object rendering strategy to render human-object together with the reconstruction and tracking of object.
%
For each frame, we render human with our novel neural texture blending while rendering objects through a classical graphics pipeline with color correction matrix (CCM).
%
To combine human and object rendering results, we utilize the depth buffer from the geometry of our human-object capture.
%

\noindent{\textbf{Training Strategy.}} To enable our sparse-view neural human performance rendering under human-object interaction, we need to train the direction-aware blending network $\Theta_{DAN}$ properly.
% 

We follow \citet{NeuralHumanFVV2021CVPR} to utilize 1457 scans from the Twindom dataset \cite{Twindom} to train our DAN $\Theta_{DAN}$ properly.
%
Differently, we randomly place the performers inside the virtual camera views and augment this dataset by randomly placing some objects from ShapeNet dataset~\cite{chang2015shapenet}.
%
By simulating the occlusion of human-object interaction, we make our network more robust to occluded human.
%
Our training dataset contains RGB images, depth maps and angle maps for all the views and models.

For the training of our direction-aware blending network $\Theta_{DAN}$, we set out to apply the following learning scheme to enable more robust blending weight learning.
%
The appearance loss function with the perceptual term ~\cite{Johnson2016Perceptual} is to make the blended texture as close as possible to the ground truth, which is formulated as:
\begin{align}
	\left.\mathcal{L}_{r g b}=\frac{1}{n} \sum_{j}^{n}
	\left(
	\left\|I_{r}^{j}-I_{g t}^{j}\right\|_{2}^{2}
% 	\right)
	+\left\|\varphi
	\left(
	I_{r}^{j}
	\right)
	-\varphi
	\left(
	I_{g t}^{j}
	\right)
	\right\|_{2}^{2}
	\right) \right.
\end{align}
where $I_{g t}$ is the ground truth RGB images; $\varphi(\cdot)$ denotes the output features of the third-layer of pre-trained VGG-19.

With the aid of such occlusion analysis, our texturing scheme maps the input adjacent images into a photo-realistic texture output of human-object activities in the target view through efficient blending weight learning, without requiring further per-scene training.

    \section{Experiments}
\label{sec:experiments}
\subsection{Experimental Setup}
We evaluate the effectiveness of our approach on three different domain adaptation datasets: DomainNet~\cite{peng2019moment}, Office-Home~\cite{Venkateswara2017DeepHN} and Office31~\cite{Saenko2010AdaptingVC}. DomainNet ~\cite{peng2019moment} is a large-scale domain adaptation dataset with 345 classes across 6 domains. Following MME ~\cite{Saito2019SemiSupervisedDA}, we use a subset of the dataset containing 126 categories across four domains: Real(R), Clipart(C), Sketch(S), and Painting(P). The performance on DomainNet is evaluated using 7 different combinations out of possible 12 combinations. Office-Home~\cite{Venkateswara2017DeepHN} is another widely used domain adaptation benchmark dataset with 65 classes across four domains: Art(Ar), Product(Pr), Clipart(Cl), and Real (Rl). We perform experiments on all possible combinations of 4 domains. Office31~\cite{Saenko2010AdaptingVC} is a relatively smaller dataset containing just 31 categories of data across three domains- Amazon(A), Dslr(D), Webcam(W). Following prior work ~\cite{Saito2019SemiSupervisedDA, Kim2020AttractPA}, we evaluate our approach on two combinations for the office31 dataset. 

For the fair comparison, we use the data-splits (train, validation, and test splits) released by ~\cite{Saito2019SemiSupervisedDA} on Github \footnote{\url{https://github.com/VisionLearningGroup/SSDA_MME}}. We use the same settings for the benchmark datasets as in the prior work ~\cite{Saito2019SemiSupervisedDA, Kim2020AttractPA}, including the number of labeled samples in the target domain, which are consistent across all experiments.

\subsection{Implementation Details}
Similar to the previous works on SSDA ~\cite{Saito2019SemiSupervisedDA, Kim2020AttractPA, Li2020OnlineMF}, we use Resnet34 and Alexnet as the backbone networks in our paper. We only used VGG for Office31 due to its higher memory requirements. The feature generator model is initialized with ImageNet weights, and the classifier is randomly initialized and has the same architecture as in ~\cite{Saito2019SemiSupervisedDA, Kim2020AttractPA, Li2020OnlineMF}. All our experiments are performed using Pytorch ~\cite{Paszke2019PyTorchAI}.We use an identical set of hyperparameters ($\alpha= 4$, $\beta= 1$ ) across all our experiments other than minibatch size. All the hyperparameters values are decided using validation performance on Product to Art experiments on the Office-Home dataset. We have set $\tau=5$ in our experiments. Each minibatch of size $B$  contains an equal number of source and labeled target examples, while the number of unlabeled target samples is $\mu \times B$. We study the effect of $\mu$ in section \ref{ablation}. Resnet34 experiments are performed with minibatch size, $B= 32$ and Alexnet models are trained with $B= 24$. We use $\mu=4$ for all our experiments. We use SGD optimizer with a momentum of $0.9$ and an initial learning rate of $0.01$ with cosine learning rate decay for all our experiments. Weight decay is set to $0.0005$ for all our models. Other details of the experiments are included in the Appendix.
\subsection{Baselines}
We compare our CLDA framework with previous state-of-the-art SSDA approaches : \textbf{MME} ~\cite{Saito2019SemiSupervisedDA}, \textbf{APE} ~\cite{Kim2020AttractPA}, \textbf{BiAT} ~\cite{Jiang2020BidirectionalAT} , \textbf{UODA} ~\cite{Qin2020Contradictory}, ~\textbf{Meta-MME} ~\cite{Li2020OnlineMF} and ~\textbf{ENT} ~\cite{Grandvalet2004SemisupervisedLB} using the performance reported by these papers. papers. We also included the results from adversarial based baseline methods:
% DANN [12] ,ADR
% 212 [43] and CDANWe also include the results from UDA (Unsupervised Domain Adaptation) based baseline methods using the adversarial approach for the Semi-Supervised Domain Adaptation task: 
\textbf{DANN} ~\cite{Ganin2016DomainAdversarialTO}, \textbf{ADR} ~\cite{Saito2018AdversarialDR} and \textbf{CDAN} ~\cite{Long2018ConditionalAD} as reported in \cite{Saito2019SemiSupervisedDA}. We also provide the \textbf{S+T} results where the model is trained using all the labeled samples across domains.
\begin{table}[!t]
\caption{ \textbf{Performance Comparison in Office-Home.} Numbers show top-1 accuracy values for different domain adaptation scenarios under 3-shot setting using Alexnet and Resnet34 as backbone networks. We have highlighted the best method for each transfer task. CLDA surpasses all the baseline methods in most adaptation scenarios. Our Proposed framework achieves the best average performance among all compared methods.
}
\renewcommand{\arraystretch}{1.2}
\vspace{2mm}
\centering
\label{base_office_home_table}
\resizebox{\columnwidth}{!}{
\begin{tabular}{c|c|cccccccccccc|c}
\specialrule{.1em}{.05em}{.05em}
Net & Method & Rl$\rightarrow$Cl & Rl$\rightarrow$Pr & Rl$\rightarrow$Ar & Pr$\rightarrow$Rl & Pr$\rightarrow$Cl & Pr$\rightarrow$Ar & Ar$\rightarrow$Pl & Ar$\rightarrow$Cl & Ar$\rightarrow$Rl & Cl$\rightarrow$Rl & Cl$\rightarrow$Ar & Cl$\rightarrow$Pr & Mean \\
\hline
\multirow{8}{*}{Alexnet} & S+T & 44.6 & 66.7 & 47.7 & 57.8 & 44.4 & 36.1 & 57.6 & 38.8 & 57.0 & 54.3 & 37.5 & 57.9 & 50.0 \\
 & DANN & 47.2 & 66.7 & 46.6 & 58.1 & 44.4 & 36.1 & 57.2 & 39.8 & 56.6 & 54.3 & 38.6 & 57.9 & 50.3 \\
 & ADR  & 37.8 & 63.5 & 45.4 & 53.5 & 32.5 & 32.2 & 49.5 & 31.8 & 53.4 & 49.7 & 34.2 & 50.4 & 44.5 \\
& CDAN & 36.1 & 62.3 & 42.2 & 52.7 & 28.0 & 27.8 & 48.7 & 28.0 & 51.3 & 41.0 & 26.8 & 49.9 & 41.2 \\
 & ENT & 44.9 & 70.4 & 47.1 & 60.3 & 41.2 & 34.6 & 60.7 & 37.8 & 60.5 & 58.0 & 31.8 & 63.4 & 50.9 \\
 & MME & 51.2 & 73.0 & 50.3 & 61.6 & 47.2 & 40.7 & 63.9 & 43.8 & 61.4 & 59.9 & 44.7 & 64.7 & 55.2 \\
 & Meta-MME & 50.3 & - & - & - & 48.3 & 40.3 & - & 44.5 & - & - & 44.5 & - & - \\
 & BiAT & - & - & - & - & - & - & - & - & - & - & - & - & 56.4 \\
 & APE & \textbf{51.9} & \textbf{74.6} & 51.2 & 61.6 & 47.9 & 42.1 & 65.5 & 44.5 & 60.9 & 58.1 & 44.3 & 64.8 & 55.6 \\
 & \textbf{CLDA}(ours) & 51.5 & 74.1 & \textbf{54.3} & \textbf{67.0} & \textbf{47.9} & \textbf{47.0} & \textbf{65.8} & \textbf{47.4} & \textbf{66.6} & \textbf{64.1} & \textbf{46.8} & \textbf{67.5} & \textbf{58.3} \\
\hline
\hline
\multirow{7}{*}{Resnet34} & S+T & 55.7 & 80.8 & 67.8 & 73.1 & 53.8 & 63.5 & 73.1 & 54.0 & 74.2 & 68.3 & 57.6 & 72.3 & 66.2 \\
 & DANN & 57.3 & 75.5 & 65.2 & 69.2 & 51.8 & 56.6 & 68.3 & 54.7 & 73.8 & 67.1 & 55.1 & 67.5 & 63.5 \\
 & ENT & 62.6 & 85.7 & 70.2 & 79.9 & 60.5 & 63.9 & 79.5 & 61.3 & 79.1 & 76.4 & 64.7 & 79.1 & 71.9 \\
 & MME & 64.6 & 85.5 & 71.3 & 80.1 & 64.6 & 65.5 & 79.0 & 63.6 & 79.7 & 76.6 & 67.2 & 79.3 & 73.1 \\
 & Meta-MME & 65.2 & - & - & - & 64.5 & 66.7 & - & 63.3 & - & - & 67.5 & - & - \\
 & APE & \textbf{66.4} & 86.2 & 73.4 & 82.0 & \textbf{65.2} & 66.1 & 81.1 & \textbf{63.9} & 80.2 & 76.8 & 66.6 & 79.9 & 74.0 \\
 & \textbf{CLDA} (ours) & 66.0 & \textbf{87.6} & \textbf{76.7} & \textbf{82.2} & 63.9 & \textbf{72.4} & \textbf{81.4} & 63.4 & \textbf{81.3} & \textbf{80.3} & \textbf{70.5} & \textbf{80.9} & \textbf{75.5} \\
\specialrule{.1em}{.05em}{.05em}
\end{tabular}}

\vspace{2mm}
\end{table}
% %------------------------------------------------------------------------- 
% %-------------------------------------------------------------------------
\subsection{Results}
Table  ~\ref{base_office_home_table}- ~\ref{base_office_table} show  top-1 accuracies  and mean accuracies for different combination of domain adaptation scenarios for all three datasets in comparison with baseline SSDA methods.

\noindent\textbf{Office-Home.} Table ~\ref{base_office_home_table} contains the results of the Office-Home dataset for 3-shot setting with Alexnet and Resnet34 as backbone networks. Results for the $1$-shot adaptation scenarios are included in the Appendix ~\ref{office_home_1_shot}. 
Our method consistently performs better than the baseline approaches and achieves $58.3\%$  and $75.5\%$ mean accuracy with Alexnet and Resnet34, respectively. Our approach surpasses the state-of-the-art SSDA approaches in most of the adaptation tasks. In some domain adaptation cases, such as Cl $\rightarrow$ Rl, Rl $\rightarrow$ Ar and Pr $\rightarrow$ Ar, we exceeded APE by more than $3\%$.

\noindent\textbf{DomainNet}: Our CLDA approach surpasses the performance of existing SSDA baselines as shown in Table ~\ref{base_domainNet_table}. Using Alexnet backbone, our method improves over BiAT by $5.2\%$ and $4.9\%$ in 1-shot and 3-shot settings, respectively. We obtain similarly improved performance when we switch the neural backbone from Alexnet to Resnet34. With Resnet34 as the backbone, we gain $4.3\%$ and $3.6\%$ over APE in 1-shot and 3-shot settings, respectively. Similar to the Office-Home, our approach surpasses the well-known domain adaptation benchmarks methods in most domain adaptation tasks of the DomainNet dataset. Such consistent improved performance shows that our approach reduces both inter and intra domain discrepancy prevalent in SSDA. 

\noindent\textbf{Office31}: Similar to other datasets, our proposed method with Alexnet and VGG as neural backbone achieves the best performance in both domain adaption scenarios for office31 as shown in Table ~\ref{base_office_table}. Using Alexnet backbone, we beat the APE ~\cite{Kim2020AttractPA} by $3.2\%$ in 3-shot and BiAT by $7.3\%$ in 1-shot settings. We observe similar gains over all the baselines methods with VGG as the neural network backbone. This shows the efficacy of our proposed approach irrespective of the used backbone.







\begin{table}[!t]

\caption{ \textbf{Performance Comparison in DomainNet.} Numbers show Top-1 accuracy values for different domain adaptation scenarios under 1-shot and 3-shot settings using Alexnet and Resnet34 as backbone networks. CLDA achieves better performance than all the baseline methods in most of the domain adaptation tasks. We have highlighted the best approach for each domain adaptation task. Our Proposed framework achieves the best average performance among all compared methods.
}
\renewcommand{\arraystretch}{1.2}
\vspace{2mm}
\label{base_domainNet_table}
\begin{center}{
\resizebox{\columnwidth}{!}{
\begin{tabular}{c|c|cccccccccccccc|cc}
\specialrule{.1em}{.05em}{.05em}
\multirow{2}{*}{Net} & \multirow{2}{*}{Method} & \multicolumn{2}{c}{R$\rightarrow$C} & \multicolumn{2}{c}{R$\rightarrow$P} & \multicolumn{2}{c}{P$\rightarrow$C} & \multicolumn{2}{c}{C$\rightarrow$S} & \multicolumn{2}{c}{S$\rightarrow$P} & \multicolumn{2}{c}{R$\rightarrow$S} & \multicolumn{2}{c|}{P$\rightarrow$R} & \multicolumn{2}{c}{Mean} \\
 & & 1-shot & 3-shot & 1-shot & 3-shot & 1-shot & 3-shot & 1-shot & 3-shot & 1-shot & 3-shot & 1-shot & 3-shot & 1-shot & 3-shot & 1-shot & 3-shot \\ \hline
\multirow{8}{*}{Alexnet} & S+T & 43.3 & 47.1 & 42.4 & 45.0 & 40.1 & 44.9 & 33.6 & 36.4 & 35.7 & 38.4 & 29.1 & 33.3 & 55.8 & 58.7 & 40.0 & 43.4 \\
 & DANN & 43.3 & 46.1 & 41.6 & 43.8 & 39.1 & 41.0 & 35.9 & 36.5 & 36.9 & 38.9 & 32.5 & 33.4 & 53.5 & 57.3 & 40.4 & 42.4 \\
 & ADR      & 43.1 & 46.2 & 41.4 & 44.4 & 39.3 & 43.6 & 32.8 & 36.4 & 33.1 & 38.9 & 29.1 & 32.4 & 55.9 & 57.3 & 39.2 & 42.7 \\
 & CDAN     & 46.3 & 46.8 & 45.7 & 45.0 & 38.3 & 42.3 & 27.5 & 29.5 & 30.2 & 33.7 & 28.8 & 31.3 & 56.7 & 58.7 & 39.1 & 41.0 \\
 & ENT & 37.0 & 45.5 & 35.6 & 42.6 & 26.8 & 40.4 & 18.9 & 31.1 & 15.1 & 29.6 & 18.0 & 29.6 & 52.2 & 60.0 & 29.1 & 39.8 \\
 & MME & 48.9 & 55.6 & 48.0 & 49.0 & 46.7 & 51.7 & 36.3 & 39.4 & 39.4 & 43.0 & 33.3 & 37.9 & 56.8 & 60.7 & 44.2 & 48.2 \\
 & Meta-MME & - & 56.4 & - & 50.2 & & 51.9 & - & 39.6 & - & 43.7 & - & 38.7 & - & 60.7 & - & 48.8 \\
 & BiAT & 54.2 & 58.6 & 49.2 & 50.6 & 44.0 & 52.0 & 37.7 & 41.9 & 39.6 & 42.1 & 37.2 & 42.0 & 56.9 & 58.8 & 45.5 & 49.4 \\
 & APE & 47.7 & 54.6 & 49.0 & 50.5 & 46.9 & 52.1 & 38.5 & 42.6 & 38.5 & 42.2 & 33.8 & 38.7 & 57.5 & 61.4 & 44.6 & 48.9 \\
 & \textbf{CLDA} (ours) & \textbf{56.3} & \textbf{59.9} & \textbf{56.0} & \textbf{57.2} & \textbf{50.8} & \textbf{54.6} & \textbf{42.5} & \textbf{47.3} & \textbf{46.8} & \textbf{51.4} & \textbf{38.0} & \textbf{42.7} & \textbf{64.4} & \textbf{67.0} & \textbf{50.7} & \textbf{54.3} \\ 
\hline
\hline
\multirow{9}{*}{Resnet34} & S+T & 55.6 & 60.0 & 60.6 & 62.2 & 56.8 & 59.4 & 50.8 & 55.0 & 56.0 & 59.5 & 46.3 & 50.1 & 71.8 & 73.9 & 56.9 & 60.0 \\
 & DANN & 58.2 & 59.8 & 61.4 & 62.8 & 56.3 & 59.6 & 52.8 & 55.4 & 57.4 & 59.9 & 52.2 & 54.9 & 70.3 & 72.2 & 58.4 & 60.7 \\
  & ADR      & 57.1 & 60.7 & 61.3 & 61.9 & 57.0 & 60.7 & 51.0 & 54.4 & 56.0 & 59.9 & 49.0 & 51.1 & 72.0 & 74.2 & 57.6 & 60.4 \\
 & CDAN     & 65.0 & 69.0 & 64.9 & 67.3 & 63.7 & 68.4 & 53.1 & 57.8 & 63.4 & 65.3 & 54.5 & 59.0 & 73.2 & 78.5 & 62.5 & 66.5 \\
 & ENT & 65.2 & 71.0 & 65.9 & 69.2 & 65.4 & 71.1 & 54.6 & 60.0 & 59.7 & 62.1 & 52.1 & 61.1 & 75.0 & 78.6 & 62.6 & 67.6 \\
 & MME & 70.0 & 72.2 & 67.7 & 69.7 & 69.0 & 71.7 & 56.3 & 61.8 & 64.8 & 66.8 & 61.0 & 61.9 & 76.1 & 78.5 & 66.4 & 68.9 \\
 & UODA & 72.7 & 75.4 & 70.3 & 71.5 & 69.8 & 73.2 & 60.5 & 64.1 & 66.4 & 69.4 & 62.7 & 64.2 & 77.3 & 80.8 & 68.5 & 71.2 \\
 & Meta-MME & - & 73.5 & - & 70.3 & - & 72.8 & - & 62.8 & - & 68.0 & - & 63.8 & - & 79.2 & - & 70.1 \\
 & BiAT & 73.0 & 74.9 & 68.0 & 68.8 & 71.6 & 74.6 & 57.9 & 61.5 & 63.9 & 67.5 & 58.5 & 62.1 & 77.0 & 78.6 & 67.1 & 69.7 \\
 & APE & 70.4 & 76.6 & 70.8 & 72.1 & \textbf{72.9} & \textbf{76.7} & 56.7 & 63.1 & 64.5 & 66.1 & 63.0 & 67.8 & 76.6 & 79.4 & 67.6 & 71.7 \\
 & \textbf{CLDA} (ours) & \textbf{76.1} & \textbf{77.7} & \textbf{75.1} & \textbf{75.7} & 71.0 & 76.4 & \textbf{63.7} & \textbf{69.7} & \textbf{70.2} & \textbf{73.7} & \textbf{67.1} & \textbf{71.1} & \textbf{80.1} & \textbf{82.9} & \textbf{71.9} & \textbf{75.3} \\ 
 \specialrule{.1em}{.05em}{.05em}
 
\end{tabular}}}
\end{center}
\end{table}
%------------------------------------------------------------------------- 

% %-------------------------------------------------------------------------

\begin{table}[t]
\centering
\caption{ \textbf{Performance Comparison in Office31.} Numbers show Top-1 accuracy values for different domain adaptation scenarios under 1-shot and 3-shot settings using Alexnet and VGG as backbone networks. CLDA outperforms all the baseline approaches in both scenarios. We have highlighted the superior method on each domain adaptation task. Our Proposed framework achieves the best mean accuracy among all baseline methods.
}
\renewcommand{\arraystretch}{1.2}
\label{base_office_table}
\begin{center}{
\resizebox{\columnwidth}{!}{
\begin{tabular}{c|cc|cc|cc||cc|cc|cc}
\specialrule{.1em}{.05em}{.05em}
\multicolumn{7}{c}{Alexnet} & \multicolumn{6}{c}{VGG} \\
\hline
 \multirow{3}{*}{Method} & \multicolumn{2}{c}{W$\rightarrow$A} & \multicolumn{2}{c|}{D$\rightarrow$A} & \multicolumn{2}{c}{Mean} & \multicolumn{2}{c}{W$\rightarrow$A} & \multicolumn{2}{c|}{D$\rightarrow$A} & \multicolumn{2}{c}{Mean}\\
 & 1-shot & 3-shot & 1-shot & 3-shot & 1-shot & 3-shot & 1-shot & 3-shot & 1-shot & 3-shot & 1-shot & 3-shot \\ \hline
 S+T & 50.4 & 61.2 & 50.0 & 62.4 & 50.2 & 61.8 &169.2 &73.2 &68.2 &73.3 &68.7 &73.25 \\
 DANN & 57.0 & 64.4 & 54.5 & 65.2 & 55.8 & 64.8 &69.3 &75.4 &70.4 &74.6 &69.85 &75.0\\
 ADR & 50.2 & 61.2 & 50.9 & 61.4 & 50.6 & 61.3 &69.7 &73.3 &69.2 &74.1 &69.45 &73.7\\
 CDAN & 50.4 & 60.3 & 48.5 & 61.4 & 49.5 & 60.8 &65.9 &74.4 &64.4 &71.4 &65.15 &72.9\\
 ENT & 50.7 & 64.0 & 50.0 & 66.2 & 50.4 & 65.1 &69.1 &75.4 &72.1 &75.1 &70.6 &75.25\\
 MME & 57.2 & 67.3 & 55.8 & 67.8 & 56.5 & 67.6 &73.1 &76.3 &73.6 &\textbf{77.6} &73.35 &76.95\\
BiAT & 57.9 & 68.2 & 54.6 & 68.5 & 56.3 & 68.4 &- &- &- &- &- &- \\
 APE & - & 67.6 & - & 69.0 & - & 68.3 &- &- &- &- &- &-\\
 CLDA & \textbf{64.6} & \textbf{70.5} & \textbf{62.7} & \textbf{72.5} & \textbf{63.6} & \textbf{71.5} &\textbf{76.2} &\textbf{78.6} &\textbf{75.1} &76.7 &\textbf{75.6} &\textbf{77.6} \\
\specialrule{.1em}{.05em}{.05em}
\end{tabular}}}
% \vspace{-2.0mm}
\end{center}
\end{table}


\begin{table}[t]
\small
\begin{center}
\begin{tabular}{ccccc}
\shline
\multirow{2}{*}{{Method}} & {CIFAR-10}&{CIFAR-100} \\
& Acc $\uparrow$(Forget $\downarrow$) & Acc $\uparrow$(Forget $\downarrow$) \\ 
\midrule
baseline & 46.4\std{$\pm$1.2}(36.0\std{$\pm$}2.1) & 18.8\std{$\pm$0.8}(18.5\std{$\pm$}0.7) \\
w/o \methodname & 53.1\std{$\pm$1.4}(24.7\std{$\pm$2.0}) & 19.3\std{$\pm$0.7}(15.9\std{$\pm$0.9}) \\
w/o \dataaugname & 52.0\std{$\pm$1.5}(34.6\std{$\pm$2.4}) & 21.5\std{$\pm$0.5}(16.3\std{$\pm$0.8}) \\ 
\hline
w/o $\mathcal{L}^{\mathrm{new}}_{\mathrm{pro}}$ & 54.8\std{$\pm$1.2}(\textbf{22.1}\std{$\pm$3.0}) & 19.6\std{$\pm$0.8}(19.9\std{$\pm$0.7}) \\
w/o $\mathcal{L}^{\mathrm{seen}}_{\mathrm{pro}}$ & 55.7\std{$\pm$1.4}(25.5\std{$\pm$1.5}) & 20.1\std{$\pm$0.4}(16.2\std{$\pm$0.6}) \\ 
$\mathcal{L}^{\mathrm{seen}}_{\mathrm{pro}}$ w/o $\mathcal{C}^\mathrm{new}$ & 56.2\std{$\pm$1.2}(26.4\std{$\pm$2.3}) & 20.8\std{$\pm$0.6}(17.9\std{$\pm$0.7}) \\ 
\hline
{\frameworkName} (\textbf{ours}) & \textbf{57.8}\std{$\pm$1.1}(23.2\std{$\pm$1.3}) & \textbf{22.7}\std{$\pm$0.7}(\textbf{15.0}\std{$\pm$0.8}) \\ 
\shline 
\end{tabular}
\end{center}
\caption{Ablation studies on CIFAR-10 ($M=0.1k$) and CIFAR-100 ($M=0.5k$). 
``baseline'' means $\mathcal{L}_\mathrm{INS}+\mathcal{L}_\mathrm{CE}$.
``$\mathcal{L}^{\mathrm{seen}}_{\mathrm{pro}}$ w/o $\mathcal{C}^\mathrm{new}$'' means $\mathcal{L}^{\mathrm{seen}}_{\mathrm{pro}}$ do not consider new classes in current task.
}
\label{tab:ablation}
\end{table}
    \section{Conclusion}
We have presented a neural performance rendering system to generate high-quality geometry and photo-realistic textures of human-object interaction activities in novel views using sparse RGB cameras only. 
%
Our layer-wise scene decoupling strategy enables explicit disentanglement of human and object for robust reconstruction and photo-realistic rendering under challenging occlusion caused by interactions. 
%
Specifically, the proposed implicit human-object capture scheme with occlusion-aware human implicit regression and human-aware object tracking enables consistent 4D human-object dynamic geometry reconstruction.
%
Additionally, our layer-wise human-object rendering scheme encodes the occlusion information and human motion priors to provide high-resolution and photo-realistic texture results of interaction activities in the novel views.
%
Extensive experimental results demonstrate the effectiveness of our approach for compelling performance capture and rendering in various challenging scenarios with human-object interactions under the sparse setting.
%
We believe that it is a critical step for dynamic reconstruction under human-object interactions and neural human performance analysis, with many potential applications in VR/AR, entertainment,  human behavior analysis and immersive telepresence.




    \section{Acknowledgments}

This work was supported by NSFC programs (61976138, 61977047), the National Key Research and Development Program (2018YFB2100 500), STCSM (2015F0203-000-06), SHMEC (2019-01-07-00-01-E00003) and Shanghai YangFan Program (21YF1429500).
\end{CJK}


%%
%% The next two lines define the bibliography style to be used, and
%% the bibliography file.
\bibliographystyle{ACM-Reference-Format}
\balance
\bibliography{reference}

% \let\balance\relax
% \nobalance
\newpage
\nobalance
\appendix
\section{More Implementation Details}
We normalize the camera system to $m$ unit and transform it to align the system center with the coordinate origin. Our voxel encoder is a forward 3D CNN like ~\cite{chibane2020implicit}, outputs 1, 16, 32, 32, 32 dimension features from each layer. Our image encoder outputs 128 dimension feature in the final layer. A 1-d depth feature is produced by perspective projection like ~\cite{PIFU_2019ICCV}. We concatenate these features and pass them through a MLP like ~\cite{PIFU_2019ICCV} with dimensions of 242, 1024, 512, 256, 128, 1. For the training data of human reconstruction, we sample points 2 $cm$, 3 $cm$, 6 $cm$ and uniformly in the 3D space. We group the surface samplings, and combine it with uniform sampling using a ratio of 16 : 1. We train the human reconstruction network with Adam optimizer at the batch size of 4. The learning rate is 1e-4 and sampling strategy is (3 $cm$,6 $cm$, uniform) for the first 20 epochs. Then we 
reduce learning rate to 1e-5 and change sampling strategy to (2 $cm$,3 $cm$, uniform) for further 100 epochs. In the object tracking stage, we use silhouette rendering in pytorch3d~\cite{ravi2020accelerating} under perspective projection. The Adam optimizer with learning rate of 1e-2 is used for a 100-iteration tracking optimization. For the neural blending network, we train it with Adam optimizer at the learning rate of 1e-4 and the batch size of 4. We train this network 44,000 iterations.


\section{Camera setting}
Pioneer work ~\cite{NeuralHumanFVV2021CVPR} evaluates the influence of sparse view number in their section 5, and finds that six cameras serve as a good compromise between rendering artifacts and the camera number. Since human-object interaction is even more challenging, we choose ``six'' for stable object tracking, robust human tracking and high-fidelity rendering. Finally, we place the cameras uniformly to minimize the average artifacts in circle rendering.

\section{Efficiency}
Compared with SOTAs, our method doesn't need per-scene training which improves efficiency in practice greatly(save several hours per-scene). However, one promising direction is to add the real-time ability into our framework to enable immersive telepresence under the human-object interaction scenario. Our method takes 0.2s to estimate SMPL, 7s to generate mesh, 7s to track object, and 0.25s to render in single frame. We believe the coarse-to-fine strategy~\cite{NeuralHumanFVV2021CVPR} and the end-to-end 6DoF estimation~\cite{peng2019pvnet} can accelerate human reconstruction and object tracking respectively.

\end{document}
\endinput


% \section{Introduction}
% ACM's consolidated article template, introduced in 2017, provides a
% consistent \LaTeX\ style for use across ACM publications, and
% incorporates accessibility and metadata-extraction functionality
% necessary for future Digital Library endeavors. Numerous ACM and
% SIG-specific \LaTeX\ templates have been examined, and their unique
% features incorporated into this single new template.

% If you are new to publishing with ACM, this document is a valuable
% guide to the process of preparing your work for publication. If you
% have published with ACM before, this document provides insight and
% instruction into more recent changes to the article template.

% The ``\verb|acmart|'' document class can be used to prepare articles
% for any ACM publication --- conference or journal, and for any stage
% of publication, from review to final ``camera-ready'' copy, to the
% author's own version, with {\itshape very} few changes to the source.

% \section{Template Overview}
% As noted in the introduction, the ``\verb|acmart|'' document class can
% be used to prepare many different kinds of documentation --- a
% double-blind initial submission of a full-length technical paper, a
% two-page SIGGRAPH Emerging Technologies abstract, a ``camera-ready''
% journal article, a SIGCHI Extended Abstract, and more --- all by
% selecting the appropriate {\itshape template style} and {\itshape
%   template parameters}.

% This document will explain the major features of the document
% class. For further information, the {\itshape \LaTeX\ User's Guide} is
% available from
% \url{https://www.acm.org/publications/proceedings-template}.

% \subsection{Template Styles}

% The primary parameter given to the ``\verb|acmart|'' document class is
% the {\itshape template style} which corresponds to the kind of publication
% or SIG publishing the work. This parameter is enclosed in square
% brackets and is a part of the {\verb|documentclass|} command:
% \begin{verbatim}
%   \documentclass[STYLE]{acmart}
% \end{verbatim}

% Journals use one of three template styles. All but three ACM journals
% use the {\verb|acmsmall|} template style:
% \begin{itemize}
% \item {\verb|acmsmall|}: The default journal template style.
% \item {\verb|acmlarge|}: Used by JOCCH and TAP.
% \item {\verb|acmtog|}: Used by TOG.
% \end{itemize}

% The majority of conference proceedings documentation will use the {\verb|acmconf|} template style.
% \begin{itemize}
% \item {\verb|acmconf|}: The default proceedings template style.
% \item{\verb|sigchi|}: Used for SIGCHI conference articles.
% \item{\verb|sigchi-a|}: Used for SIGCHI ``Extended Abstract'' articles.
% \item{\verb|sigplan|}: Used for SIGPLAN conference articles.
% \end{itemize}

% \subsection{Template Parameters}

% In addition to specifying the {\itshape template style} to be used in
% formatting your work, there are a number of {\itshape template parameters}
% which modify some part of the applied template style. A complete list
% of these parameters can be found in the {\itshape \LaTeX\ User's Guide.}

% Frequently-used parameters, or combinations of parameters, include:
% \begin{itemize}
% \item {\verb|anonymous,review|}: Suitable for a ``double-blind''
%   conference submission. Anonymizes the work and includes line
%   numbers. Use with the \verb|\acmSubmissionID| command to print the
%   submission's unique ID on each page of the work.
% \item{\verb|authorversion|}: Produces a version of the work suitable
%   for posting by the author.
% \item{\verb|screen|}: Produces colored hyperlinks.
% \end{itemize}

% This document uses the following string as the first command in the
% source file:
% \begin{verbatim}
% \documentclass[sigconf]{acmart}
% \end{verbatim}

% \section{Modifications}

% Modifying the template --- including but not limited to: adjusting
% margins, typeface sizes, line spacing, paragraph and list definitions,
% and the use of the \verb|\vspace| command to manually adjust the
% vertical spacing between elements of your work --- is not allowed.

% {\bfseries Your document will be returned to you for revision if
%   modifications are discovered.}

% \section{Typefaces}

% The ``\verb|acmart|'' document class requires the use of the
% ``Libertine'' typeface family. Your \TeX\ installation should include
% this set of packages. Please do not substitute other typefaces. The
% ``\verb|lmodern|'' and ``\verb|ltimes|'' packages should not be used,
% as they will override the built-in typeface families.

% \section{Title Information}

% The title of your work should use capital letters appropriately -
% \url{https://capitalizemytitle.com/} has useful rules for
% capitalization. Use the {\verb|title|} command to define the title of
% your work. If your work has a subtitle, define it with the
% {\verb|subtitle|} command.  Do not insert line breaks in your title.

% If your title is lengthy, you must define a short version to be used
% in the page headers, to prevent overlapping text. The \verb|title|
% command has a ``short title'' parameter:
% \begin{verbatim}
%   \title[short title]{full title}
% \end{verbatim}

% \section{Authors and Affiliations}

% Each author must be defined separately for accurate metadata
% identification. Multiple authors may share one affiliation. Authors'
% names should not be abbreviated; use full first names wherever
% possible. Include authors' e-mail addresses whenever possible.

% Grouping authors' names or e-mail addresses, or providing an ``e-mail
% alias,'' as shown below, is not acceptable:
% \begin{verbatim}
%   \author{Brooke Aster, David Mehldau}
%   \email{dave,judy,steve@university.edu}
%   \email{firstname.lastname@phillips.org}
% \end{verbatim}

% The \verb|authornote| and \verb|authornotemark| commands allow a note
% to apply to multiple authors --- for example, if the first two authors
% of an article contributed equally to the work.

% If your author list is lengthy, you must define a shortened version of
% the list of authors to be used in the page headers, to prevent
% overlapping text. The following command should be placed just after
% the last \verb|\author{}| definition:
% \begin{verbatim}
%   \renewcommand{\shortauthors}{McCartney, et al.}
% \end{verbatim}
% Omitting this command will force the use of a concatenated list of all
% of the authors' names, which may result in overlapping text in the
% page headers.

% The article template's documentation, available at
% \url{https://www.acm.org/publications/proceedings-template}, has a
% complete explanation of these commands and tips for their effective
% use.

% Note that authors' addresses are mandatory for journal articles.

% \section{Rights Information}

% Authors of any work published by ACM will need to complete a rights
% form. Depending on the kind of work, and the rights management choice
% made by the author, this may be copyright transfer, permission,
% license, or an OA (open access) agreement.

% Regardless of the rights management choice, the author will receive a
% copy of the completed rights form once it has been submitted. This
% form contains \LaTeX\ commands that must be copied into the source
% document. When the document source is compiled, these commands and
% their parameters add formatted text to several areas of the final
% document:
% \begin{itemize}
% \item the ``ACM Reference Format'' text on the first page.
% \item the ``rights management'' text on the first page.
% \item the conference information in the page header(s).
% \end{itemize}

% Rights information is unique to the work; if you are preparing several
% works for an event, make sure to use the correct set of commands with
% each of the works.

% The ACM Reference Format text is required for all articles over one
% page in length, and is optional for one-page articles (abstracts).

% \section{CCS Concepts and User-Defined Keywords}

% Two elements of the ``acmart'' document class provide powerful
% taxonomic tools for you to help readers find your work in an online
% search.

% The ACM Computing Classification System ---
% \url{https://www.acm.org/publications/class-2012} --- is a set of
% classifiers and concepts that describe the computing
% discipline. Authors can select entries from this classification
% system, via \url{https://dl.acm.org/ccs/ccs.cfm}, and generate the
% commands to be included in the \LaTeX\ source.

% User-defined keywords are a comma-separated list of words and phrases
% of the authors' choosing, providing a more flexible way of describing
% the research being presented.

% CCS concepts and user-defined keywords are required for for all
% articles over two pages in length, and are optional for one- and
% two-page articles (or abstracts).

% \section{Sectioning Commands}

% Your work should use standard \LaTeX\ sectioning commands:
% \verb|section|, \verb|subsection|, \verb|subsubsection|, and
% \verb|paragraph|. They should be numbered; do not remove the numbering
% from the commands.

% Simulating a sectioning command by setting the first word or words of
% a paragraph in boldface or italicized text is {\bfseries not allowed.}

% \section{Tables}

% The ``\verb|acmart|'' document class includes the ``\verb|booktabs|''
% package --- \url{https://ctan.org/pkg/booktabs} --- for preparing
% high-quality tables.

% Table captions are placed {\itshape above} the table.

% Because tables cannot be split across pages, the best placement for
% them is typically the top of the page nearest their initial cite.  To
% ensure this proper ``floating'' placement of tables, use the
% environment \textbf{table} to enclose the table's contents and the
% table caption.  The contents of the table itself must go in the
% \textbf{tabular} environment, to be aligned properly in rows and
% columns, with the desired horizontal and vertical rules.  Again,
% detailed instructions on \textbf{tabular} material are found in the
% \textit{\LaTeX\ User's Guide}.

% Immediately following this sentence is the point at which
% Table~\ref{tab:freq} is included in the input file; compare the
% placement of the table here with the table in the printed output of
% this document.

% \begin{table}
%   \caption{Frequency of Special Characters}
%   \label{tab:freq}
%   \begin{tabular}{ccl}
%     \toprule
%     Non-English or Math&Frequency&Comments\\
%     \midrule
%     \O & 1 in 1,000& For Swedish names\\
%     $\pi$ & 1 in 5& Common in math\\
%     \$ & 4 in 5 & Used in business\\
%     $\Psi^2_1$ & 1 in 40,000& Unexplained usage\\
%   \bottomrule
% \end{tabular}
% \end{table}

% To set a wider table, which takes up the whole width of the page's
% live area, use the environment \textbf{table*} to enclose the table's
% contents and the table caption.  As with a single-column table, this
% wide table will ``float'' to a location deemed more
% desirable. Immediately following this sentence is the point at which
% Table~\ref{tab:commands} is included in the input file; again, it is
% instructive to compare the placement of the table here with the table
% in the printed output of this document.

% \begin{table*}
%   \caption{Some Typical Commands}
%   \label{tab:commands}
%   \begin{tabular}{ccl}
%     \toprule
%     Command &A Number & Comments\\
%     \midrule
%     \texttt{{\char'134}author} & 100& Author \\
%     \texttt{{\char'134}table}& 300 & For tables\\
%     \texttt{{\char'134}table*}& 400& For wider tables\\
%     \bottomrule
%   \end{tabular}
% \end{table*}

% Always use midrule to separate table header rows from data rows, and
% use it only for this purpose. This enables assistive technologies to
% recognise table headers and support their users in navigating tables
% more easily.

% \section{Math Equations}
% You may want to display math equations in three distinct styles:
% inline, numbered or non-numbered display.  Each of the three are
% discussed in the next sections.

% \subsection{Inline (In-text) Equations}
% A formula that appears in the running text is called an inline or
% in-text formula.  It is produced by the \textbf{math} environment,
% which can be invoked with the usual
% \texttt{{\char'134}begin\,\ldots{\char'134}end} construction or with
% the short form \texttt{\$\,\ldots\$}. You can use any of the symbols
% and structures, from $\alpha$ to $\omega$, available in
% \LaTeX~\cite{Lamport:LaTeX}; this section will simply show a few
% examples of in-text equations in context. Notice how this equation:
% \begin{math}
%   \lim_{n\rightarrow \infty}x=0
% \end{math},
% set here in in-line math style, looks slightly different when
% set in display style.  (See next section).

% \subsection{Display Equations}
% A numbered display equation---one set off by vertical space from the
% text and centered horizontally---is produced by the \textbf{equation}
% environment. An unnumbered display equation is produced by the
% \textbf{displaymath} environment.

% Again, in either environment, you can use any of the symbols and
% structures available in \LaTeX\@; this section will just give a couple
% of examples of display equations in context.  First, consider the
% equation, shown as an inline equation above:
% \begin{equation}
%   \lim_{n\rightarrow \infty}x=0
% \end{equation}
% Notice how it is formatted somewhat differently in
% the \textbf{displaymath}
% environment.  Now, we'll enter an unnumbered equation:
% \begin{displaymath}
%   \sum_{i=0}^{\infty} x + 1
% \end{displaymath}
% and follow it with another numbered equation:
% \begin{equation}
%   \sum_{i=0}^{\infty}x_i=\int_{0}^{\pi+2} f
% \end{equation}
% just to demonstrate \LaTeX's able handling of numbering.

% \section{Figures}

% The ``\verb|figure|'' environment should be used for figures. One or
% more images can be placed within a figure. If your figure contains
% third-party material, you must clearly identify it as such, as shown
% in the example below.
% \begin{figure}[h]
%   \centering
%   \includegraphics[width=\linewidth]{sample-franklin}
%   \caption{1907 Franklin Model D roadster. Photograph by Harris \&
%     Ewing, Inc. [Public domain], via Wikimedia
%     Commons. (\url{https://goo.gl/VLCRBB}).}
%   \Description{A woman and a girl in white dresses sit in an open car.}
% \end{figure}

% Your figures should contain a caption which describes the figure to
% the reader.

% Figure captions are placed {\itshape below} the figure.

% Every figure should also have a figure description unless it is purely
% decorative. These descriptions convey what’s in the image to someone
% who cannot see it. They are also used by search engine crawlers for
% indexing images, and when images cannot be loaded.

% A figure description must be unformatted plain text less than 2000
% characters long (including spaces).  {\bfseries Figure descriptions
%   should not repeat the figure caption – their purpose is to capture
%   important information that is not already provided in the caption or
%   the main text of the paper.} For figures that convey important and
% complex new information, a short text description may not be
% adequate. More complex alternative descriptions can be placed in an
% appendix and referenced in a short figure description. For example,
% provide a data table capturing the information in a bar chart, or a
% structured list representing a graph.  For additional information
% regarding how best to write figure descriptions and why doing this is
% so important, please see
% \url{https://www.acm.org/publications/taps/describing-figures/}.

% \subsection{The ``Teaser Figure''}

% A ``teaser figure'' is an image, or set of images in one figure, that
% are placed after all author and affiliation information, and before
% the body of the article, spanning the page. If you wish to have such a
% figure in your article, place the command immediately before the
% \verb|\maketitle| command:
% \begin{verbatim}
%   \begin{teaserfigure}
%     \includegraphics[width=\textwidth]{sampleteaser}
%     \caption{figure caption}
%     \Description{figure description}
%   \end{teaserfigure}
% \end{verbatim}

% \section{Citations and Bibliographies}

% The use of \BibTeX\ for the preparation and formatting of one's
% references is strongly recommended. Authors' names should be complete
% --- use full first names (``Donald E. Knuth'') not initials
% (``D. E. Knuth'') --- and the salient identifying features of a
% reference should be included: title, year, volume, number, pages,
% article DOI, etc.

% The bibliography is included in your source document with these two
% commands, placed just before the \verb|\end{document}| command:
% \begin{verbatim}
%   \bibliographystyle{ACM-Reference-Format}
%   \bibliography{bibfile}
% \end{verbatim}
% where ``\verb|bibfile|'' is the name, without the ``\verb|.bib|''
% suffix, of the \BibTeX\ file.

% Citations and references are numbered by default. A small number of
% ACM publications have citations and references formatted in the
% ``author year'' style; for these exceptions, please include this
% command in the {\bfseries preamble} (before the command
% ``\verb|\begin{document}|'') of your \LaTeX\ source:
% \begin{verbatim}
%   \citestyle{acmauthoryear}
% \end{verbatim}

%   Some examples.  A paginated journal article \cite{Abril07}, an
%   enumerated journal article \cite{Cohen07}, a reference to an entire
%   issue \cite{JCohen96}, a monograph (whole book) \cite{Kosiur01}, a
%   monograph/whole book in a series (see 2a in spec. document)
%   \cite{Harel79}, a divisible-book such as an anthology or compilation
%   \cite{Editor00} followed by the same example, however we only output
%   the series if the volume number is given \cite{Editor00a} (so
%   Editor00a's series should NOT be present since it has no vol. no.),
%   a chapter in a divisible book \cite{Spector90}, a chapter in a
%   divisible book in a series \cite{Douglass98}, a multi-volume work as
%   book \cite{Knuth97}, a couple of articles in a proceedings (of a
%   conference, symposium, workshop for example) (paginated proceedings
%   article) \cite{Andler79, Hagerup1993}, a proceedings article with
%   all possible elements \cite{Smith10}, an example of an enumerated
%   proceedings article \cite{VanGundy07}, an informally published work
%   \cite{Harel78}, a couple of preprints \cite{Bornmann2019,
%     AnzarootPBM14}, a doctoral dissertation \cite{Clarkson85}, a
%   master's thesis: \cite{anisi03}, an online document / world wide web
%   resource \cite{Thornburg01, Ablamowicz07, Poker06}, a video game
%   (Case 1) \cite{Obama08} and (Case 2) \cite{Novak03} and \cite{Lee05}
%   and (Case 3) a patent \cite{JoeScientist001}, work accepted for
%   publication \cite{rous08}, 'YYYYb'-test for prolific author
%   \cite{SaeediMEJ10} and \cite{SaeediJETC10}. Other cites might
%   contain 'duplicate' DOI and URLs (some SIAM articles)
%   \cite{Kirschmer:2010:AEI:1958016.1958018}. Boris / Barbara Beeton:
%   multi-volume works as books \cite{MR781536} and \cite{MR781537}. A
%   couple of citations with DOIs:
%   \cite{2004:ITE:1009386.1010128,Kirschmer:2010:AEI:1958016.1958018}. Online
%   citations: \cite{TUGInstmem, Thornburg01, CTANacmart}. Artifacts:
%   \cite{R} and \cite{UMassCitations}.

% \section{Acknowledgments}

% Identification of funding sources and other support, and thanks to
% individuals and groups that assisted in the research and the
% preparation of the work should be included in an acknowledgment
% section, which is placed just before the reference section in your
% document.

% This section has a special environment:
% \begin{verbatim}
%   \begin{acks}
%   ...
%   \end{acks}
% \end{verbatim}
% so that the information contained therein can be more easily collected
% during the article metadata extraction phase, and to ensure
% consistency in the spelling of the section heading.

% Authors should not prepare this section as a numbered or unnumbered {\verb|\section|}; please use the ``{\verb|acks|}'' environment.

% \section{Appendices}

% If your work needs an appendix, add it before the
% ``\verb|\end{document}|'' command at the conclusion of your source
% document.

% Start the appendix with the ``\verb|appendix|'' command:
% \begin{verbatim}
%   \appendix
% \end{verbatim}
% and note that in the appendix, sections are lettered, not
% numbered. This document has two appendices, demonstrating the section
% and subsection identification method.

% \section{SIGCHI Extended Abstracts}

% The ``\verb|sigchi-a|'' template style (available only in \LaTeX\ and
% not in Word) produces a landscape-orientation formatted article, with
% a wide left margin. Three environments are available for use with the
% ``\verb|sigchi-a|'' template style, and produce formatted output in
% the margin:
% \begin{itemize}
% \item {\verb|sidebar|}:  Place formatted text in the margin.
% \item {\verb|marginfigure|}: Place a figure in the margin.
% \item {\verb|margintable|}: Place a table in the margin.
% \end{itemize}

% %%
% %% The acknowledgments section is defined using the "acks" environment
% %% (and NOT an unnumbered section). This ensures the proper
% %% identification of the section in the article metadata, and the
% %% consistent spelling of the heading.
% \begin{acks}
% To Robert, for the bagels and explaining CMYK and color spaces.
% \end{acks}

% %%
% %% The next two lines define the bibliography style to be used, and
% %% the bibliography file.
% \bibliographystyle{ACM-Reference-Format}
% \bibliography{sample-base}

% %%
% %% If your work has an appendix, this is the place to put it.
% \appendix

% \section{Research Methods}

% \subsection{Part One}

% Lorem ipsum dolor sit amet, consectetur adipiscing elit. Morbi
% malesuada, quam in pulvinar varius, metus nunc fermentum urna, id
% sollicitudin purus odio sit amet enim. Aliquam ullamcorper eu ipsum
% vel mollis. Curabitur quis dictum nisl. Phasellus vel semper risus, et
% lacinia dolor. Integer ultricies commodo sem nec semper.

% \subsection{Part Two}

% Etiam commodo feugiat nisl pulvinar pellentesque. Etiam auctor sodales
% ligula, non varius nibh pulvinar semper. Suspendisse nec lectus non
% ipsum convallis congue hendrerit vitae sapien. Donec at laoreet
% eros. Vivamus non purus placerat, scelerisque diam eu, cursus
% ante. Etiam aliquam tortor auctor efficitur mattis.

% \section{Online Resources}

% Nam id fermentum dui. Suspendisse sagittis tortor a nulla mollis, in
% pulvinar ex pretium. Sed interdum orci quis metus euismod, et sagittis
% enim maximus. Vestibulum gravida massa ut felis suscipit
% congue. Quisque mattis elit a risus ultrices commodo venenatis eget
% dui. Etiam sagittis eleifend elementum.

% Nam interdum magna at lectus dignissim, ac dignissim lorem
% rhoncus. Maecenas eu arcu ac neque placerat aliquam. Nunc pulvinar
% massa et mattis lacinia.

% \end{document}
% \endinput
%%
%% End of file `sample-sigconf.tex'.
