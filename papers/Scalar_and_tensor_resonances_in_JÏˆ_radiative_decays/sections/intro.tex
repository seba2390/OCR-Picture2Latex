\section{Introduction}
\label{sec:intro}

The  vast majority of observed mesons can be understood as simple $q\bar q$ bound states, although in principle strong interactions permit a more complex spectrum. 
In a pure Yang-Mills theory, massive gluon bound states
(named ``glueballs") populate the spectrum, as shown for example in lattice calculations.
The lightest glueball is expected to have
$J^{PC} = 0^{++}$, and a mass between 1.5 and 2\gev~\cite{Bali:1993fb,Patel:1986vv,Albanese:1987ds,Michael:1988jr,Sexton:1995kd,Morningstar:1999rf,Szczepaniak:2003mr,Chen:2005mg,Athenodorou:2020ani}. 
An enhanced glueball production is expected in OZI--suppressed processes,  \ie when the the quarks of the initial state annihilate into gluons.
For example, this is the case for central exclusive  production in $pp$ collisions (where mesons are produced by Pomeron---\ie gluon ladder---fusion),
or for \jpsi radiative decays, the $c \bar c$ must annihilate to gluons before hadronizing into the final state.  
In QCD, the mixing between glueballs and $q\bar q$ isoscalar  mesons makes the identification of a glueball candidate challenging, both theoretically and experimentally. The simplest argument for the existence of a glueball component is the presence of a supernumerary state with respect to how many are predicted by the quark model~\cite{Mathieu:2008me,Ochs:2013gi,Llanes-Estrada:2021evz}. 
It is thus of key importance to 
have a precise determination of the number and properties of the resonances seen in data. 

The most recent edition of Particle Data Group (PDG) identifies nine  isoscalar-scalar resonances. The two lightest ones, the \sig and \fzero, have been extensively studied in recent years, and are by now very well established~\cite{Ananthanarayan:2000ht,Colangelo:2001df,GarciaMartin:2011cn,Moussallam:2011zg,Caprini:2005zr,GarciaMartin:2011nna,Pelaez:2015qba}.
Quark model predicts other two scalars below 2\gev, but the three $f_0(1370)$, $f_0(1500)$ and $f_0(1710)$ are observed. This stimulated an intense work to identify one of them as the long-sought glueball~\cite{Chanowitz:1980gu,Amsler:1995td,Amsler:1995tu,Lee:1999kv,Giacosa:2005zt,Giacosa:2005qr,Albaladejo:2008qa,Janowski:2014ppa}. 
The existence of the $f_0(1370)$ is still debated. 
It seems to couple strongly to $4\pi$~\cite{Abele:2001js,Abele:2001pv}, while the analyses of two-body final states led to contradictory results. While some analysis claim to find this resonance in either $\pi \pi \to K \bar K$ or $\eta \eta$ scattering~\cite{Amsler:1992rx,Anisovich:1994bi,Amsler:1995bz,Gaspero:1992gu,Lanaro:1993km,Amsler:1994rv,Cohen:1980cq,Etkin:1982se} other analyses coming from meson-meson reactions do not find it~\cite{Hyams:1973zf,Grayer:1974cr,Hyams:1975mc,Estabrooks:1978de,Adolph:2015tqa}.
The $f_0(1500)$  and  $f_0(1710)$  are instead well established.
They have been determined from $\pi\pi$ production from fixed target experiments~\cite{Hyams:1973zf,Grayer:1974cr,Hyams:1975mc,clas:2017vxx}, and from heavy meson decays~\cite{Ablikim:2013hq,Dobbs:2015dwa,Ablikim:2018izx,dArgent:2017gzv,Lees:2012kxa,Ropertz:2018stk}, with the $f_0(1710)$  coupling mainly to kaon pairs~\cite{Barberis:1999am,Uehara:2013mbo,Ablikim:2018izx}.
Discerning which of the three is (or has the largest component of) the glueball, is an even harder task.
Since photons do not couple directly to gluons, the scarce production of $f_0(1500)$ in $\gamma\gamma$ collisions suggests it may be dominantly a   glueball. On the other hand, arguments based on the chiral suppression of the  perturbative matrix element of a scalar glueball to a $q\bar q$ pair, point to the $f_0(1710)$ as a better candidate~\cite{Chanowitz:2005du,Albaladejo:2008qa}. Although the argument does not necessarily hold nonperturbatively~\cite{Chao:2005si,Chanowitz:2007ma}, it seems to be supported by a quenched Lattice QCD calculation~\cite{Sexton:1995kd}.
The spectrum of scalars above 2\gev is even more confusing. The PDG currently lists $f_0(2020)$, $f_0(2100)$, $f_0(2200)$,  and $f_0(2330)$, but none of them is marked as established. The first one has been 
recently confirmed by a reanalysis of the $B_s^0 \to \jpsi \,\pi\pi$ and $\to \jpsi K \bar K$ decays~\cite{Ropertz:2018stk}.
The $f_0(2100)$ and $f_0(2200)$ appear to decay to only pions or kaons, respectively.
Since their resonance parameters are not dramatically different, they might originate from a single physical resonance (\cf Ref.~\cite{JPAC:2018zyd}).
Finally, the $f_0(2330)$ was seen in $p\bar p$ annihilations fifteen years ago~\cite{Anisovich:2000ut,Bugg:2004rj}, and was recently confirmed by a global reanalysis of reactions where isoscalar-scalar mesons appear~\cite{Sarantsev:2021ein}. 

The isoscalar-tensor sector appears to be better understood.
The $f_2(1270)$ and $f_2'(1525)$  are 
identified as ordinary $u\bar u + d\bar d$ and $s\bar s$ mesons, respectively.
 Indeed, the former couples largely to $\pi \pi$, and  the latter to $K\bar  K$~\cite{GarciaMartin:2011cn,Pelaez:2018qny,Pelaez:2020gnd}. Both resonances are relatively narrow and 
  have also been extracted from lattice QCD with a high degree of accuracy~\cite{Briceno:2017qmb}.\footnote{Alternative interpretations for the $f_2(1270)$ were discussed in~\cite{Molina:2008jw,Gulmez:2016scm,Geng:2016pmf,Du:2018gyn,Molina:2019rai}.}
 The status of  the other four resonances in the mass range up to 2\gev, the $f_2(1810)$, $f_2(1910)$, $f_2(1950)$, and $f_2(2010)$, is not as clear. The $f_2(2010)$ was seen in final states with strangeness only, $K\bar K$ and $\phi\phi$, suggesting the $s\bar s$ assignment. 
The other decay predominantly to multibody channels, making their identification more complicated. Above 2\gev, the PDG reports the $f_2(2150)$ and two more tensors, the  $f_2(2300)$ and the $f_2(2340)$. It is worth noting that a $2^{++}$ glueball is also expected at about $2.5\gev$~\cite{Morningstar:1999rf}. We summarize the status 
 of the isoscalar-scalar and -tensor resonances in Table~\ref{tab:pdgpoles}.

With more of high precision data coming from present and future experiments, including multibody final states,  in order to make further progress in identification of the resonance,   it is necessary to develop adequate amplitude analysis methods. For example,  dispersive techniques that rely on fundamental $S$-matrix principles  have played a key role in determining properties of the lightest scalar resonances~\cite{Caprini:2005zr,DescotesGenon:2006uk,GarciaMartin:2011nna,Hoferichter:2011wk,Moussallam:2011zg,Ditsche:2012fv,Pelaez:2020uiw,Pelaez:2020gnd}.  Their  application, however, has so far been limited to roughly the region below $1\gev$. At higher energies, other  approaches, such as Pad\'e approximants~\cite{Masjuan:2013jha,Masjuan:2014psa,Caprini:2016uxy,Pelaez:2016klv}, Laurent-Pietarinen expansion~\cite{Svarc:2014sqa}, or the Schlessinger point method~\cite{Tripolt:2016cya,Tripolt:2018xeo,Binosi:2019ecz} have been used. However, these methods often require as input an analytic parametrization of the data. But --- unlike men--- not all parameterizations are created equal, and the ones that fulfill as many $S$-matrix principles as possible should be considered more trustworthy. 


\begin{table}
\begin{ruledtabular}
\begin{tabular}{c|ccccc}
 & Mass \mevp& Width \mevp & $\mathcal{B} (f\to\pi\pi)$ ($\%$)  & $\mathcal{B} (f\to K\bar{K})$  ($\%$)& $\mathcal{B} (f \to 4\pi )$  ($\%$) \\ 
\hline
$f_0(1370)$ & $1200$--$1500$& $300$--$500$&$<10$ \cite{Ochs:2013gi}& $35\pm13$ \cite{Bugg:1996ki}& $>72$ \cite{Gaspero:1992gu}\\ 
$f_0(1500)$ &$1506\pm6$ &$112\pm9$ & $34.5\pm2.2$& $8.5\pm 1.0$& $48.9\pm 3.3$\\ 
$f_0(1710)$ & $1704\pm12$&$123\pm18$ &$3.9^{+3.0}_{-2.4}$ \cite{Longacre:1986fh}& $36\pm12$ \cite{Albaladejo:2008qa}& $-$\\ 
$\left[f_0(2020)\right]$ & $1992\pm16$&$442\pm60$ &$-$ & $-$&$-$\\
$\left[f_0(2200)\right]$ & $2187\pm 14$& $207\pm40$&$-$ &$-$ & $-$\\
$\left[f_0(2330)\right]$ & $2324\pm 35$& $195\pm71$&$-$ &$-$ & $-$\\
\hline
$f_2(1270)$ & $1275.5\pm0.8$&$186.7^{+2.2}_{-2.5}$ &$84.2^{+2.9}_{-0.9}$ & $4.6^{+0.5}_{-0.4}$ & $10.4^{+1.6}_{-3.7}$ $^{(a)}$\\
$\left[f_2(1430)\right]$ & $\approx 1430$& $-$ &$-$ & $-$& $-$\\
$f_2'(1525)$ & $1517.4\pm2.5$& $86\pm5$ & $0.83\pm0.16$ &$87.6\pm2.2$ & $-$ \\
$\left[f_2(1565)\right]$ & $1542\pm19$& $122\pm 13$&$-$ & $-$& $-$\\
$\left[f_2(1640)\right]$ & $1639\pm 6$& $99^{+60}_{-40}$& $-$& $-$& $-$\\
$\left[f_2(1810)\right]$ & $1815\pm12$& $197\pm 22$& $21^{+2}_{-3}$ \cite{Longacre:1986fh}&$2\times 0.3^{+1.9}_{-0.2}$ \cite{Longacre:1986fh}&$-$\\
$\left[f_2(1910)\right]$ & $1900\pm9$ $^{(b)}$& $167\pm 21$ $^{(b)}$& $-$& $-$& $-$\\
$f_2(1950)$ & $1936\pm12$& $464\pm24$&$-$ &$-$ &$-$\\
$f_2(2010)$ & $2011^{+62}_{-76}$& $202^{+67}_{-62}$& $-$ & $-$&$-$\\
$\left[f_2(2150)\right]$ & $2157\pm12$& $152\pm30$& $-$& $-$& $-$\\
$\left[f_2(2300)\right]$ & $2297\pm28$& $149\pm41$& $-$& $-$& $-$\\
\end{tabular}
\end{ruledtabular}
\caption{\label{tab:pdgpoles}Summary of scalar and tensor resonances in the $1$--$2.5\gev$ region listed in the PDG~\cite{pdg}. Resonances in square brackets are not well established.  $^{(a)}$ Combination of entries on  $\pi^+\pi^- 2\pi^0$, $2\pi^+2\pi^-$, and $4\pi^0$, errors added linearly due to being asymmetric.
$^{(b)}$ Mass and width from the $\omega\omega$ decay mode.} 
\end{table}
In this paper we extract the scalar and tensor resonances from the partial waves of $\jpsi\to\gamma \pi^0\pi^0$ and $\to \gamma \KSKS$ 
determined by BESIII~\cite{Ablikim:2015umt,Ablikim:2018izx}.
We use a number of different parametrizations that satisfy unitarity and analyticity, in order to put under control large model dependencies. At the energies 
 of interest, the number of available open channels makes the complete rigorous analysis unfeasible. We start by considering the $\pi\pi$ and $K\bar K$ final states only. Implementing unitarity on a subset of available channels does not affect seriously the resonant parameters, provided that resonances are sufficiently separated from each other~\cite{JPAC:2017dbi,JPAC:2018zyd}. This is definitely not the case here. We find that 2-channel fits fail to reproduce some of the details of the resonant peaks and the interference patterns in the regions between nearest resonances. This can bias the pole determination. 
 In addition to $\pi\pi$ and $K\bar K$, the PDG lists at least three  other decay channels for these resonances, \ie $\eta \eta$, $\eta \eta'$, $4 \pi$, with larger coupling to $4\pi$. The channels $\jpsi \to \gamma \,(2\pi^+2\pi^-,\pi^+\pi^-2\pi^0)$ were seen in the experiments done  in the 80s~\cite{Baltrusaitis:1985nd,Bisello:1988as} and later by BES~\cite{Bai:1999mm}. BESIII also measured $J/\psi \to \gamma \eta \eta$~\cite{BESIII:2013qqz}. However these analyses do not provide mass-independent partial wave extractions, and are not comparable in statistics and quality with the most recent ones that we use. For this reason, we decided to add an effective  third channel, which without loss of generality we may interpret as  $\rho\rho$, that is  however, not  constrained by any other  data. 
 Finally, the statistical uncertainties are determined via bootstrap~\cite{recipes,EfroTibs93,Landay:2016cjw}.
 
 
The rest of the paper is organized as follows. A brief description of the data and 
of our selection of the fit region is discussed in Section~\ref{sec:data}. We describe our set of parametrizations in Section~\ref{sec:model}. The 2-channel fits are described in Section~\ref{sec:2charesults}, and in Section~\ref{sec:3charesults} we study the role of the third channel and perform the statistical analysis. 
The summary of results we obtain for the resonant poles are detailed in  Section~\ref{sec:poles} and our conclusions are given in Section~\ref{sec:summary}.

