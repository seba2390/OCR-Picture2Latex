\section{Summary}
\label{sec:summary}
We presented a detailed analysis of the isoscalar-scalar and -tensor resonances in the $1$-$2.5\gev$ mass region. We study the BESIII mass-independent partial waves from $\jpsi \to \gamma \pi^0\pi^0$ and $\to \gamma K_S^0 K_S^0$ radiative decays~\cite{Ablikim:2015umt,Ablikim:2018izx}. 
Data were published in two equivalent solutions in the full kinematic range. However, the region below the $K\bar K$ threshold is not compatible with Watson's theorem expectation, which made us select one of the two solutions, and to restrict to the $1$-$2.5\gev$ mass region.
To assess the model dependence realistically, we explored a large number of amplitude parametrizations that respect the $S$-matrix principles as much as possible, and discuss the results for 14 of them. We first enforce unitarity strictly on the two channels considered, which turns out to be too rigid to describe data, in particular between the resonant peaks. We then extend our models to include a third unconstrained $\rho\rho$ channel, which is known to contribute substantially to the resonances in this region. Fit quality is excellent for all the parametrizations studied. Despite the large systematic uncertainties, we can identify four scalar and three tensor states. 

The four lightest resonances are determined with great accuracy, which allows us to study their couplings. We find that the $f_2(1270)$ and $f_2'(1525)$ couple largely to $\pi\pi$ and $K\bar K$, respectively, as expected by their quark model assignments. The couplings ratios are compatible with the branching fractions reported in the PDG. In the scalar sector, it seems that the $f_0(1710)$ appears in $\jpsi \to \gamma f_0$ more strongly than the $f_0(1500)$. This affinity of the $f_0(1710)$ to the gluon-rich initial state, together with a coupling to $K\bar K$ larger by one order of magnitude, are hints for a sizeable glueball component.

