%============================================
% Ambiguities
%============================================
\section{\boldmath $J/\psi\to \gamma\pi \pi$ Ambiguities}
\label{app:ambi}

\begin{figure}[t]
{\centering
\includegraphics[width=0.4\textwidth]{figures/pipiSalt.pdf} \includegraphics[width=0.4\textwidth]{figures/pipiDalt.pdf} \\
\includegraphics[width=0.4\textwidth]{figures/pipiM2alt.pdf} 
\includegraphics[width=0.4\textwidth]{figures/pipiE3alt.pdf} 
}
\caption{Comparison between the nominal (black) and ambiguous (red) solutions for the intensities extracted in~\cite{Ablikim:2015umt}. The prediction for the latter is shown in green using the relations derived in the experimental paper, and extended below the $K \bar K$ threshold.
}
\label{fig:ambi-intensities}
\end{figure}

\begin{figure}[t]
{\centering
\includegraphics[width=0.4\textwidth]{figures/pipiSDPhalt.pdf}  \\
\includegraphics[width=0.4\textwidth]{figures/pipiM2DPhalt.pdf} 
\includegraphics[width=0.4\textwidth]{figures/pipiE3DPhalt.pdf} 
}
\caption{Comparison between the nominal (black) and ambiguous (red) solutions for the relative phases extracted in~\cite{Ablikim:2015umt}. The prediction for the latter is shown in green using the relations derived in the experimental paper, and extended below the $K \bar K$ threshold.}
\label{fig:ambi-phases}
\end{figure}
As mentioned in Section~\ref{sec:data}, partial wave extractions suffer from ambiguities. 
Specifically, the $J/\psi\to \gamma \h \bar \h$ radiative decays truncated to the $2^{++}$ multipoles, can have two different solutions, related mathematically~\cite{Ablikim:2015umt,Ablikim:2018izx}: in a given energy bin, one can calculate the intensities and relative phases of the four multipoles of one solution from the intensities and relative phases of the four multipoles of the other solution. Below the $K \bar K$ threshold, the experimental papers do not show the ambiguous solution: Watson's theorem is invoked in order to discard one of them. However, as we showed in Section~\ref{sec:data}, Watson's theorem also implies the $0^{++}\,E1$ phase to match the $S$-wave elastic $\pi\pi$ scattering shift, which is not the case. Based on this, and on the fact that the ambiguous solutions in $\pi\pi$ and $K\bar K$ shows some unexpected behaviour in the phases, we decided to focus on the nominal solution, and discard the region below 1\gev.

Nevertheless, we tried to see whether there is a way to make use of these data in the region where the much studied $\sig$ and the $f_0(980)$ appear.
Since the existence of ambiguities is a mathematical fact that does not  depend on unitarity arguments like Watson's theorem, we can calculate the ambiguous solution of $\jpsi \to \gamma \pi^0\pi^0$ below $K\bar K$ threshold and check whether it agrees better with $\pi\pi$ scattering. The exercise is shown in Figs.~\ref{fig:ambi-intensities} and~\ref{fig:ambi-phases}. Since the relative phase of the three $2^{++}$ multipoles is set to zero below the $K \bar K$ threshold, this turns into an underestimation of the errors of the ambiguous solution, that looks very scattered (in particular for the phases) and unusable.

We even tried to proceed in the opposite direction: replacing the measured $0^{++}\,E1$ phase with the known $S$-wave elastic $\pi\pi$ scattering one, we can calculate what would its  ambiguous counterpart be. The result is shown in Fig.~\ref{fig:ambi-phases_new}. This looks closer to the BESIII phase, although with some differences, most notably the sharp rise at $\sim 900\mev$.





\begin{figure}[ht]
{\centering
\includegraphics[width=0.5\textwidth]{figures/deltaSaltelastic.pdf}
}
\caption{Comparison between the nominal BESIII data, the elastic $\pi \pi$ phase shift from~\cite{Pelaez:2019eqa} (solid green band) and the predicted ambiguous partner of the latter.}
\label{fig:ambi-phases_new}
\end{figure}



%============================================
% Gamma dist appendix
%============================================
\section{\boldmath Bootstrap and the $\Gamma$ distribution}
\label{app:gamma}

Bootstrapping has become in the recent past a promising method to assess uncertainties in spectroscopy analyses~\cite{Landay:2016cjw,Pilloni:2016obd,JPAC:2017dbi,JPAC:2018zyd,Molina:2020qpw,Niehus:2020gmf,JPAC:2020umo,Bibrzycki:2021rwh}. In particular it allows one to map the likelihood for a given minimum, which is not accessible through simple error propagation in non-linear problems, or when the number of parameters is very large. Furthermore, this technique can also help us distinguishing between stable ``physical'' poles and spurious ones~\cite{JPAC:2018zyd,Fernandez-Ramirez:2019koa}, whereas simple error propagation would fail to describe in a robust way those uncertainties.

One usually assumes data points to be normally distributed. However, the intensities extracted in~\cite{Ablikim:2015umt,Ablikim:2018izx} are positive defined, and since they are not simple event counts, they are not even Poisson distributed. There are several data points compatible with zero, or even negative values, within $1\sigma$, which is not physical, and no sensible parametrization can reproduce. In this sense using a simple normal distribution to resample the data would produce artifacts in our uncertainties, which would then propagate into the pole errors.

For all intensity data points which are compatible with zero within $2 \sigma$, we assume they follow a $\Gamma$ distribution, having by mean and variance the central value and the error squared. This was used in previous spectroscopy analyses~\cite{Blin:2016dlf}. The distribution is given by
\begin{equation}
H\left(x \mid \mu, \sigma\right)=\theta(x)\,\left(\frac{x \mu}{\sigma^{2}}\right)^{\frac{\mu^{2}}{\sigma^{2}}} \frac{\exp \left(-x \mu / \sigma^{2}\right)}{x\, \Gamma\!\left(\mu^{2} / \sigma^{2}\right)}\,.
\end{equation}
This distribution is positive defined and has light tails as the gaussian, which makes it a good candidate for our purposes. Its mean and variance are $\mu$ and $\sigma^2$.
