\section{Resonant poles}
\label{sec:poles}

\begin{table}[b] 
\caption{List of final pole position and uncertainties resulting from the combination of the 14 different final fits to the data. The errors have been obtained as the variance of the full samples, by assuming that the spread of results for each pole, shown in Fig.~\ref{fig:finalpoles}, resembles a Gaussian distribution.}
\begin{ruledtabular}
\begin{tabular}{c c c c}
$S$-wave  & $\sqrt{s_p}$ \mevp & $D$-wave & $\sqrt{s_p}$ \mevp \\ \hline
\rule[-0.2cm]{-0.1cm}{.55cm} $f_0 (1500)$ &  $(1450 \pm 10) - i (106 \pm 16)/2$  &  $f_2 (1270)$ &  $(1268 \pm 8) - i (201 \pm 11)/2$ \\
\rule[-0.2cm]{-0.1cm}{.55cm} $f_0 (1710)$ &  $(1769 \pm 8) - i (156 \pm 12)/2$  &  $f_2 (1525)$ &  $(1503 \pm 11) - i (84 \pm 15)/2$ \\
\rule[-0.2cm]{-0.1cm}{.55cm} $f_0 (2020)$ &  $(2038 \pm 48) - i (312 \pm 82)/2$  &  $f_2 (1950)$ &  $(1955 \pm 75) - i (350 \pm 113)/2$ \\
\rule[-0.2cm]{-0.1cm}{.55cm} $f_0 (2330)$ &  $(2419 \pm 64) - i (274 \pm 94)/2$  &   &  \\
\end{tabular}
\end{ruledtabular}

\label{tab:polesfinal}
\end{table}


\begin{figure}[t]
\centering
\includegraphics[width=\textwidth]{general_poleD} \\ \includegraphics[width=\textwidth]{general_poleS} 
\caption{Results for the pole positions of the 3-channel fits, superimposed for the 14 models. A point is drawn for each pole found in each of the $O(10^4)$ pseudodatasets  generated by the bootstrap analysis. Colored points represent poles identified as a physical resonance, gray points are spurious. For the physical resonance, gray ellipses show the $68\%$ confidence region of each systematic. Colored ellipses show the average of all 14 systematics, as explained in the text.
}
\label{fig:finalpoles}
\end{figure}
As already discussed, it is not possible to fix a priori the number of poles that appear on the proximal Riemann sheets. In general, there is no one-to-one correspondence between the poles of the amplitude and the number of \KCDD in coupled channel problems. This relation becomes even more complicated because of the additional background polynomial. 
Moreover, the simple left-hand cut parametrizations in $\rho N^J_{ki}(s')$ also tend to generate additional broad poles close to threshold~\cite{JPAC:2017dbi}.
Some of the poles capture the real features of the amplitude, and are associated with the physical resonances. Other poles are mere artefacts of the model implemented, and are unstable upon bootstrap and model variations. Therefore, a sound statistical analysis and a large set of systematic variations are required to filter out the spurious singularities and identify the remaining ones with the physical resonances. 
The pole positions for the systematic variations of amplitudes studied here are plotted in Fig.~\ref{fig:finalpoles}, while the separate plots for each systematic are left in the \nameref{sup:supp-material}. 
For each model, the statistical uncertainties are determined via bootstrap, as explained in Section~\ref{sec:3charesults}. While in~\cite{JPAC:2018zyd} we were able to identify a nominal model, and explored how model variation affected the central values, here the clusters of poles, in particular the heaver ones, move too much to make this strategy feasible. In order to quote 
an average of masses and widths obtained by the 14 models, we calculate the mean and (co)variance of the pole positions among the $14\times O(10^4)$ 
pseudodatasets from the bootstrap analysis for all the models at once. 

In addition to the pole positions, one can extract the residues of the amplitude. The residues of $a_i^J(s)$ can be associated with the couplings of the resonance $f$ to the initial $\jpsi\,\gamma$ and final $\h \bar \h$ states. We remark that we do not include all the possible open channels involved at these energies. However, since the unconstrained third $\rho\rho$ channel  can effectively reabsorb the presence of other channels, we believe that the relative size of the $\pi\pi$ and $K\bar K$ coupling provides reliable information. One can also study the residues of the $D^J(s)^{-1}$ matrix, that are connected to the scattering couplings $\h \bar \h \to f \to \h' \bar \h'$, albeit not rigorously.\footnote{To get the full scattering amplitude, the  $D^J(s)^{-1}$ matrix should be multiplied by the appropriate $N^J(s)$ that satisfies an integral equation that depends on the left-hand cuts of the scattering process. However, since $N^J(s)$ is smooth, we believe it should not affect much the relative size of the couplings, that we discuss here.} Since we are not fitting scattering data, the residues of $D^J(s)^{-1}$ are mostly unconstrained, and have large uncertainties~\cite{Briceno:2021xlc}.

The lightest two $D$-wave poles are very well determined. They correspond to the $f_2(1270)$ and $f_2'(1525)$ resonances, and decay almost elastically to $\pi \pi$ and $K \bar K$ respectively. The $f_2$ peak in $\pi \pi$ and the $f_2'$ peak in $K \bar K$ is very well described by all models. The $f_2'$ lies close to the $\rho\rho$ threshold, so we have to ensure that the poles that form the cluster appear always on the proximal Riemann sheet. For all the models, the $f_2'$ is always centered below this threshold. 
Even though we do not fit scattering data directly, these resonances are so well behaved that the scattering couplings have reasonable ratios:
\begin{align}
f_2(1270): \quad r_{\pi\pi}\Big/\sqrt{r^2_{\pi\pi} + r^2_{K \bar K}} &=82^{+6}_{-8}\% \,, &
f_2'(1525): \quad r_{K\bar K}\Big/\sqrt{r^2_{\pi\pi} + r^2_{K \bar K}}&=95 ^{+3}_{-5}\% \,,
\end{align}
where $r_{\h\bar \h}$ are the absolute values of the residues of the $D^J(s)^{-1}$ matrix in the elastic $\h\bar \h \to \h\bar \h$ channel. These are reasonably close to the PDG estimates~\cite{pdg}. 
Some of the fits produce a second broader cluster in  $D$-wave behind the $f'_2(1525)$. As can be seen in the \nameref{sup:supp-material}, this second pole appears in most of the $K$-matrix parametrizations, often with very large spread, but not in the CDD ones. 
Furthermore, when the pole appears the local $\chi^2$ in that region does not improve. For these reasons, the existence of an additional resonance is not compelling in data.

Moving to the $S$-wave, our result for the $f_0(1500)$ is perfectly compatible with~\cite{Ropertz:2018stk}, even though  we have a $f_0(1710)$  close by, which could easily affect its pole position. The $f_0(1500)$ turns out to be rather narrow, and produces a simple phase motion for the $S$-wave phases. The $f_0(1710)$ is noticeably broader, but nevertheless very well determined. The mass we find for the 
 $f_0(1710)$ is considerably larger than the PDG average, however, it is still compatible with many of the determinations listed in the PDG. All the four scalar resonances we found are roughly compatible with those identified  in~\cite{Sarantsev:2021ein}, although what we call $f_0(1710)$ and $f_0(2020)$ seem to correspond to their $f_0(1770)$ and $f_0(2100)$. 

When comparing the $f_0(1500)$ and $f_0(1710)$ couplings of the full $\jpsi \to \gamma f_0 \to \gamma \h \bar \h$ process, we find that the heavier one couples more strongly to both final states. In particular the coupling of the $f_0(1710)$ 
 to $K \bar K$ is roughly eight times larger than that of the $f_0(1500)$  and roughly three times larger in $\pi\pi$, as can be seen in Fig.~\ref{fig:residues}. It is worth noting that the values of the residues change substantially  under amplitude variations,
 which makes us cautious about strong claims regarding a precise determination of these ratios. However, all determinations agree qualitatively: the heavier resonance is stronger in $\jpsi$ radiative decays, and in particular in the $K \bar K$ channel. As we mentioned in the \nameref{sec:intro}, these arguments favor the interpretation for the $f_0(1710)$ to have a sizeable glueball component. 



\begin{figure}[t]
\centering
\includegraphics[width=0.48\textwidth]{residues_poleD}  \includegraphics[width=0.48\textwidth]{residues_poleS} 
\caption{Left panel: ratio of absolute values of the scattering residues  of $\pi\pi$ and $K\bar K$ final states, for the $f_2(1270)$ against the $f_2'(1525)$. Right panel: ratio of absolute values of decay residues of $f_0(1500)$ and $f_0(1710)$, for $\pi\pi$ against $K\bar K$ final state.A point is drawn for each pole found in each of the $O(10^4)$ pseudodatasets  generated by the bootstrap analysis. Gray ellipses show the $68\%$ confidence region of each systematic. The colored ellipses represents the $68\%$ confidence region of all the systematics at once. Ratios are nongaussian positive-defined quantities, and the results of each systematics scarcely overlap, so this ellipse cannot be taken literally, but nevertheless provides a crude idea of average values, errors and correlations. 
}
\label{fig:residues}
\end{figure}

The final set of poles that can be identified as physical ones is shown in Fig.~\ref{fig:finalpoles}, and the mean values and uncertainties are listed in Table~\ref{tab:polesfinal}. It is worth noting that our poles are compatible with the ones on the BESIII $J/\psi \to \gamma \eta\eta$ decay~\cite{BESIII:2013qqz}, even if we do not include this channel. This supports our choice of including the most relevant high-statistics $\pi \pi$ and $K \bar K$ channels only.

