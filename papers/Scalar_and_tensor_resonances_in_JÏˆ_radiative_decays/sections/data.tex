\section{The Dataset}
\label{sec:data}



\begin{figure}[t]
{\centering
\includegraphics[scale=.8]{diag1} \hspace{0.5cm} \includegraphics[scale=.8]{diag2}}
\caption{Processes contributing to $\jpsi \to \gamma \,\h\bar \h$, with $\h = \pi,K$. Left panel: \jpsi decays through short-range $c\bar c \to \gamma gg$ (blue disk), then resonances are seen emerging from final state interaction (red square). Right panel: \jpsi decays to $V\,\h$ through the short-range $c\bar c \to ggg$ (red square), then the resonance $V$ decays radiatively to $\gamma \bar \h$ (blue disk).}
\label{fig:proces}
\end{figure}

We consider the data from the 
  mass independent analysis of $J/\psi$ 
   radiative decays, 
    $\jpsi \to \gamma\pi^0\pi^0$~\cite{Ablikim:2015umt} and $\to \gamma \KSKS$~\cite{Ablikim:2018izx} 
     by BESIII. 
Bose symmetry requires the two pseudoscalars to have $J^{PC}=\text{(even)}^{++}$; moreover the isospin zero amplitude is dominant.\footnote{In fact, for the $\pi^0\pi^0$ system, isospin one is forbidden, and the decay into isospin two  would be higher order in the isospin breaking. Since $I=2$ has no resonances, there would be no dynamical mechanism that could enhance it. For the \KSKS system, isospin one is allowed, and exhibits a rich resonant structure, that includes, for example, the $a_0(980)$ and the $a_2(1320)$. However, the production of isovector is OZI-suppressed, since the topology  $c\bar c\to \gamma gg$ couples to isoscalars only. }
The mass independent $J=0$ and $J=2$ partial waves are given in the multipole basis~\cite{sebastian:1992xq}. 
  The latter is visible in three different multipoles, $E1$, $M2$, and $E3$.\footnote{We use the standard notation,  $Xj_\gamma$, in which 
   $j_\gamma \ge 1$ is the angular momentum carried by the electromagnetic field, and $X=E$ ($X=M$) if the parities of initial and final state satisfy (or not) $P_\text{in} P_\text{fin} = (-)^{j_\gamma}$. The values allowed for $j_\gamma$ are $|J_\psi - J| \le j_\gamma \le J_\psi + J$.} The three intensities look very similar up to the overall normalization with,  $E1 > M2 > E3$. 
  Considering that the quark model predicts each multipole to scale as $(E_\gamma)^{j_\gamma}$, $E_\gamma$ being the photon energy, the observed 
      hierarchy is consistent with theoretical expectation, at  least close to threshold. 
  Intensities and
phase differences determined with respect to the $2^{++}\,E1$ are given in 15\mev invariant mass bins, from
threshold up to 3\gev. In order to make use of the information on the relative phase, we should analyze simultaneously the $S$- and all the $D$-waves, however, 
 because of the dominance of the lowest multipoles,
   we focus on $0^{++}$ and $2^{++}\,E1$. 

\begin{figure}[t]
{\centering
\includegraphics[width=0.5\textwidth]{figures/deltaScomp.pdf} 
}
\caption{$S$-wave $\pi\pi$ phase shifts in the elastic region. Black empty circles show the $0^{++} - 2^{++}\,E1$  
phase difference of the radiative $\jpsi\to\gamma\pi\pi$ decays by BESIII~\cite{Ablikim:2015umt}. The well known $\pi \pi$ scattering data is shown in green empty squares~\cite{Hyams:1973zf,Grayer:1974cr,Hyams:1975mc,Cohen:1980cq,Kaminski:1996da,Batley:2010zza}. The dispersive fits of~\cite{GarciaMartin:2011cn,Pelaez:2019eqa} are shown in green for the $S$-wave, and blue for (minus) the $D$-wave phase shift, which is basically zero at these energies. Radiative data are incompatible with the dispersive result by more than $9\sigma$, that reduce to $\sim 7\sigma$ if one allows for a constant shift.}
\label{fig:upto1GeV}
\end{figure}
 
\begin{table}[b]
\caption{Branching ratios of resonances appearing in the $\gamma \bar\h$ channel, compared to the total branching ratio. The largest contribution is given by the $\omega$. However, it is removed in~\cite{Ablikim:2015umt} by vetoing the events within 50\mev from the nominal $\omega$ mass.
}
\begin{tabular}{l | c c}
\hline\hline
$\mathcal{B}\!\left(\jpsi \to \gamma \pi^0\pi^0\right)$ & $(11.5 \pm 0.5) \times 10^{-4}$ &\cite{Ablikim:2015umt}\\\hline
$\mathcal{B}\!\left(\jpsi \to \omega\pi^0 \to \gamma \pi^0\pi^0\right)$ & $(3.8 \pm 0.4) \times 10^{-5}$ &\multirow{3}{*}{\cite{pdg}}\\
$\mathcal{B}\!\left(\jpsi \to \rho\pi^0 \to \gamma \pi^0\pi^0\right)$ & $(2.6 \pm 0.5) \times 10^{-6}$ &\\
$\mathcal{B}\!\left(\jpsi \to b_1(1232)\pi^0 \to \gamma \pi^0\pi^0\right)$ & $(3.6 \pm 1.3) \times 10^{-6}$ & \\\hline\hline
$\mathcal{B}\!\left(\jpsi \to \gamma \KSKS\right)$ & $(8.1 \pm 0.4) \times 10^{-4}$ &\cite{Ablikim:2018izx}\\\hline
$\mathcal{B}\!\left(\jpsi \to K^{*}(892)^0 K_S^0 \to \gamma \KSKS\right)$ & $(6.3 \pm 0.6) \times 10^{-6}$ &\multirow{2}{*}{\cite{Ablikim:2018izx}} \\
$\mathcal{B}\!\left(\jpsi \to K_1(1270)^0 K_S^0 \to \gamma \KSKS\right)$ & $(8.5 \pm 2.5) \times 10^{-7}$& \\\hline\hline
\end{tabular}
\label{tab:tchannel}
\end{table}

The dynamics underlying these radiative decays can be represented by the diagrams in Fig.~\ref{fig:proces}. In the left diagram, the \jpsi decay is mediated by the short-range process, for example,  $c\bar c \to \gamma g g$ and resonances originate from rescattering of the two mesons. On the right diagram, the \jpsi decays through another short-range process, {\it e.g.} $c\bar c \to g g g$ to a state containing  an intermediate resonance $V$ and a bachelor meson,  $\h=\pi,K$. The resonance $V$ then decays radiatively to $\gamma \bar \h$.\footnote{Charge conjugation is understood.} 
The latter class of reactions introduce  a nontrivial background to the processes  we are interested in. 
These intermediate resonances appear as peaks in the $\gamma \bar \h$ invariant mass, but their contribution is mostly flat when projected onto the $\h\bar \h$ direction.
Morevover, the region within 50\mev from the dominant  exchange of the of the $\omega$, that appears as a narrow  peak in the $\gamma \bar{h}$ Dalitz plot
 has  been removed from the \pizpiz dataset~\cite{Ablikim:2015umt}. The effect of $K^*(892)$ and $K_1(1270)$ on the \KSKS spectrum was estimated to be negligible~\cite{Ablikim:2018izx}. Indeed,  looking at the branching ratios given in Table~\ref{tab:tchannel}, one can appreciate how small the contribution of these resonances is, even more so when spread over the two-meson invariant mass. While in principle these resonances can still affect the partial waves through 3-body rescattering~\cite{Niecknig:2012sj,*Gan:2020aco,*JPAC:2020umo}, it is expected that these corrections are small for large phase spaces like the ones considered here. 
 We thus restrict the dynamics of the right diagram of Fig.~\ref{fig:proces} to possible heavier resonances that lie outside of the Dalitz plot region.    
The first consequence is that it is expected that  Watson's theorem holds from $\pi\pi$ to $K\bar K$ threshold~\cite{Watson:1952ji}. 
 Specifically, the phase of the $0^{++}\,E1$ multipole of $\jpsi \to \gamma \pi\pi$ should match 
 that of the $S$-wave elastic $\pi\pi$ scattering. The latter is well established~\cite{Hyams:1973zf,Grayer:1974cr,Hyams:1975mc,Cohen:1980cq,Kaminski:1996da,Batley:2010zza,Ananthanarayan:2000ht,Colangelo:2001df,GarciaMartin:2011cn,Moussallam:2011zg,Pelaez:2019eqa} and 
Fig.~\ref{fig:upto1GeV} compares the two. 
Even if one reconsidered the effect of 3-body rescattering, it would be impossible for crossed channel resonances to produce such a fast phase motion, in particular close to the $\pi\pi$ threshold.
Since the focus of this work is on the higher resonances, we shall not consider further  the data below the $K\bar K$ threshold. Moreover, no significant structure appears in data above 2.5\gev. Since the high energy region would require a  different approach~\cite{Bibrzycki:2021rwh}, we also drop it from this analysis. 

The extraction of partial waves from data suffers from Barrelet ambiguities~\cite{Barrelet:1971pw}. For $\jpsi \to \gamma \h\bar \h$ truncated to $J=2$, there are two possible solutions in each  channel, as shown in Appendix~\ref{app:ambi}.
While the nominal ones have a roughly vanishing $2^{++}\,E1-2^{++}\,M2$ phase difference, the alternative solutions display rapid motion, in particular at $\sim 1.5\gev$.  It is well known that the $f_2(1270)$ and $f_2'(1525)$ are mostly elastic and dominate the $\pi\pi$ and $K\bar K$ channels, respectively. Inelasticities contribute to $\lesssim 15\%$ to the width of each resonance. 
In this case, 
Watson's theorem requires that the phase difference between two $2^{++}$ multipoles vanishes in this region.
Hence, the phase motions observed in the alternative solutions are not
justified. These solutions also exhibit phase motion at both low and high masses, where no resonances are expected to contribute. Incidentally, the mass dependent fit of $K\bar K$ in~\cite{Ablikim:2018izx} clearly favors the nominal solution. For these reasons, in our analysis we consider the nominal solutions only.

To summarize, we will perform a coupled-channel analysis of the $0^{++}\,E1$ and $2^{++}\,E1$ intensities and relative phase in the invariant mass region between $1$ and $2.5\gev$ using as data input the nominal solutions. In the following, we will refer to these two multipoles as $S$- and $D$-waves. In total, we fit 606 data points.

