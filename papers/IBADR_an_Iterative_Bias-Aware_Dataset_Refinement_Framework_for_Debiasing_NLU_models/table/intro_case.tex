\begin{table}[tb]
\small
\center
\renewcommand{\arraystretch}{1.1}
\begin{tabularx}{0.45\textwidth}{ l }
\toprule[1pt]
\makecell[l]{\textbf{Premise}:  \textit{The manager knew the tourists supported the author .} \\ 
\textbf{Hypothesis}:  \textit{The manager knew the tourists .} \\  
\textbf{Golden Label}:  \textit{contradiction} \\
\textbf{Predicted Label}:  \textit{\textcolor{red} {entailment}}} \\
\hline
\makecell[l]{\textbf{Premise}:  \textit{The student saw the managers .} \\ 
\textbf{Hypothesis}:  \textit{The managers saw the student .} \\ 
\textbf{Golden Label}:  \textit{contradiction} \\
\textbf{Predicted Label}: \textit{\textcolor{red}{entailment}}} \\
\bottomrule[1pt]
\end{tabularx}
\caption{Two instances from HANS \cite{McCoyPL19}. Both instances contain biased features, which make the dominant model \cite{devlin-etal-2019-bert} unable to accurately predict the relationship between the premise and hypothesis. In the first instance, the hypothesis is a sub-sequence of the premise, while they convey different meanings. In the second one, the premise and hypothesis share identical words, but with swapped objects.}
\label{tab:case_intro}
\end{table} 
