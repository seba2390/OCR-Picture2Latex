\section{The Inert Doublet Model}\label{sec:idm}
\subsection{The model}
The Inert Doublet Model (IDM) \cite{Deshpande:1977rw,Cao:2007rm,Barbieri:2006dq} is an intriguing new physics model that enhances the SM scalar sector by an additional $SU(2)\,\times\,U(1)$ gauge doublet $\phi_D$. Furthermore, it introduces a discrete $\mathbb{Z}_2$ symmetry with the following transformation properties
\begin{equation}\label{eq:symm}
\phi_S\to \phi_S, \,\, \phi_D \to - \phi_D, \,\,
\text{SM} \to \text{SM}.
\end{equation}
In this model the symmetry remains exact. This has important consequences: {\sl (a)} the additional doublet does not acquire a vacuum expectation value (vev) and {\sl (b)} it does not couple to fermions. Therefore, electroweak symmetry breaking proceeds as in the SM. Furthermore, the above symmetry insures that the lightest particle of the so-called dark doublet $\phi_D$ is stable and renders a dark matter candidate.

The potential of the model is given by
\begin{equation}\begin{array}{c}
V=-\frac{1}{2}\left[m_{11}^2(\phi_S^\dagger\phi_S)\!+\! m_{22}^2(\phi_D^\dagger\phi_D)\right]+
\frac{\lambda_1}{2}(\phi_S^\dagger\phi_S)^2\! 
+\!\frac{\lambda_2}{2}(\phi_D^\dagger\phi_D)^2\\[2mm]+\!\lambda_3(\phi_S^\dagger\phi_S)(\phi_D^\dagger\phi_D)\!
\!+\!\lambda_4(\phi_S^\dagger\phi_D)(\phi_D^\dagger\phi_S) +\frac{\lambda_5}{2}\left[(\phi_S^\dagger\phi_D)^2\!
+\!(\phi_D^\dagger\phi_S)^2\right].
\end{array}\label{pot}\end{equation}

After electroweak symmetry breaking, the model features 7 free parameters. We here chose these in the so-called physical basis {\cite{Ilnicka:2015jba}}
\begin{\eqn}\label{eq:physbas}
v,M_h,M_H, M_A, M_{H^{\pm}}, \lam_2, \lam_{345},
\end{\eqn}
where we use $\lam_{345}\,\equiv\,\lam_3+\lam_4+\lam_5$ throughout this work.
The vev $v$ as well as $M_h\,\sim\,125\,\GeV$ are fixed by experimental measurements, leading to a total number of 5 free parameters. We here choose $H$ as the dark matter candidate, which implies $M_{A,\,H^\pm}\,\geq\,M_H$. \footnote{Note that the new scalars in the IDM do not have CP quantum numbers, as they do not couple to fermions. In the subsequent discussion, we can replace $H\,\longleftrightarrow\,A$ if we simultaneously use $\lam_5\,\longleftrightarrow\,-\lam_5$. All phenomenological considerations are identical for these cases.}

The model is subject to a large number of theoretical and experimental constraints. These have been discussed at length e.g. in \cite{Ilnicka:2015jba,Ilnicka:2018def,Dercks:2018wch,Kalinowski:2018ylg,Kalinowski:2020rmb} and will therefore not be repeated here. In the scan for the allowed parameter ranges, we make use of the publicly available tools \texttt{2HDMC} \cite{Eriksson:2009ws},  \HBv5.9.0 \cite{Bechtle:2008jh, Bechtle:2011sb, Bechtle:2013wla,Bechtle:2015pma,Bechtle:2020pkv}, \HSv2.6.0 \cite{Bechtle:2013xfa,Bechtle:2020uwn}, as well as \texttt{micrOMEGAs$\_$5.2.4} \cite{Belanger:2020gnr}. Cross sections are calculated using  {\texttt{Madgraph5}} \cite{Alwall:2011uj} with a UFO input file from \cite{Goudelis:2013uca}\footnote{\label{foot:ufo} Note the official version available at \cite{ufo_idm} exhibits a wrong CKM structure, leading to false results for processes involving electroweak gauge bosons radiated off quark lines. In our implementation, we corrected for this. Our implementation corresponds to the expressions available from \cite{Zyla:2020zbs}.}. We compare to experimental values from GFitter \cite{gfitter,Haller:2018nnx}, as well as results from the Planck \cite{Aghanim:2018eyx} and XENON1T \cite{Aprile:2018dbl} experiments. Direct collider searches as well as agreement with the 125 \GeV coupling strength measurements are implemented via \HB~ and \HS, where we additionally compare to the total width upper limit  \cite{Sirunyan:2019twz} and invisible branching ratio \cite{ATLAS-CONF-2020-052} of $h$. Recast results from a LEP-SUSY search \cite{Lundstrom:2008ai} were also included.{We} refer {the reader} to the above references for more details.

\subsection{Current Status}
The experimental and theoretical constraints lead to a large reduction of the allowed parameter space of the model; in particular, the masses are usually constrained to be quite degenerate, as can be seen {from} figure \ref{fig:massesidm}. This is due to an interplay of electroweak constraints as well as theoretical requirements on the potential. 
\begin{figure}[htb]
\centerline{%
\includegraphics[width=0.48\textwidth]{mamhpm.pdf}
\includegraphics[width=0.48\textwidth]{massdiffrange.pdf}}
\caption{Masses are requested to be quite degenerate after all constraints have been taken into account. {\sl Left:} In the $\lb M_A,\,M_{H^\pm} \rb$ plane (taken from \cite{Ilnicka:2015jba}). {\sl Right:} In the {$\lb M_{H^\pm}-M_H,\,M_A-M_H \rb$} plane (taken from \cite{Kalinowski:2018ylg}).}
\label{fig:massesidm}
\end{figure}
A particularly interesting scenario is the case when $M_H\,\leq\,M_h/2$, which opens up the $h\,\rightarrow\,\text{inv{isible}}$ channel. In such a scenario, there is an interesting interplay between bounds from signal strength measurements, that require $|\lam_{345}|$ to be rather small $\lesssim\,0.3$, and bounds from dark matter relic density, where too low values of that parameter lead to small annihilation cross sections and therefore too large relic density values. The effects of this are shown in figure \ref{fig:lowmh}. In \cite{Ilnicka:2015jba}, it was found that this in general leads to a lower bound of $M_H\,\sim\,50\,\GeV$, although exceptions to this rule were presented in \cite{Kalinowski:2020rmb}. 
\begin{figure}[htb]
\centerline{%
\includegraphics[width=0.48\textwidth]{lam345_v3.pdf}
\includegraphics[width=0.48\textwidth]{lam345_xenon_v3_2.pdf}}
\caption{Interplay of signal strength and relic density constraints in the $\lb M_H,\,\lam_{345}\rb$ plane. {\sl Left:} Using LUX constraints \cite{Akerib:2013tjd}, bounds labelled "excluded from collider data" have been tested using \HB~and \HS~(taken from \cite{Ilnicka:2015jba}). {\sl Right:} Using XENON1T results, with golden points labelling those points that produce exact relic density (taken from \cite{Ilnicka:2018def}).}
\label{fig:lowmh}
\end{figure}
\subsection{Discovery prospects at CLIC}
So far, no publicly available search exists that investigates the IDM parameter space with actual collider data. In \cite{Kalinowski:2018kdn,deBlas:2018mhx}, however, the discovery potential of CLIC was investigated for several benchmark points proposed in \cite{Kalinowski:2018ylg}, for varying center-of-mass energies up to $3\,\TeV$. We investigated both $AH$ and $H^+ H^-$ production with $A\,\rightarrow\,Z\,H$ and $H^\pm\,\rightarrow\,W^\pm H$, where the electroweak gauge bosons subsequently decay leptonically. Event generation was performed using \texttt{WHizard 2.2.8} \cite{Moretti:2001zz,Kilian:2007gr}, with an interface via \texttt{SARAH} \cite{Staub:2015kfa} and \texttt{SPheno 4.0.3} \cite{Porod:2003um,Porod:2011nf} for model implementation. CLIC energy spectra \cite{Linssen:2012hp} were also taken into account.

For the production modes above, we considered leptonic decays of the electroweak gauge bosons. In particular, the investigated final states were
\begin{\eqn*}
e^+\,e^-\,\rightarrow\,\mu^+\mu^-+\slashed{E},\,e^+\,e^-\,\rightarrow\,\mu^\pm\e^\mp+\slashed{E}
\end{\eqn*}
for $HA$ and $H^+\,H^-$ production, respectively. Note however, that in the event generation we did not specify the intermediate states, which means all processes leading to the above signatures were taken into account, including interference between the contributing diagrams. This includes final states where the missing energy can originate from neutrinos in the final state. \\
Event selection was performed using a set of preselection cuts as well as boosted decision trees, as implemented in the TMVA toolkit \cite{Hocker:2007ht}. Results for the discovery reach of CLIC with varying center of mass energies {are shown} in figure \ref{fig:clic}.  We see that in general, production cross sections $\gtrsim\,0.5\,\fb$ seem to be accessible, where best prospects for the considered benchmark points are given for 380 \GeV or 1.5 \TeV center-of-mass energies. {Similarly}, mass sums up to 1 \TeV seem accessible, where in general the $\mu^\pm\,e^\mp$ channel seems to provide a larger discovery range.
\begin{figure}[htb]
\begin{center}%
\caption{Discovery prospects at CLIC for the IDM in $\mu^+\mu^-+\slashed{E}$ {\sl (left)} and $\mu^\pm\,e^\mp+\slashed{E}$ {\sl (right)} final states, as a function of the respective production cross-sections {\sl (top)} and mass sum of the produced particles {\sl (bottom)}. Taken from \cite{Kalinowski:2018kdn}.}
\label{fig:clic}
\end{figure}
Considering {the $H^+\,H^-$ production with the semi-leptonic final state, i.e. with hadronic decay of one of the $W$ bosons}, increases the {corresponding} mass range to about 2 \TeV \cite{Sokolowska:2019xhe,Zarnecki:2020swm,Zarnecki:2020nnw,Klamka:2728552}.

\subsection{Sensitivity comparison at future colliders}
After a dedicated analysis of the IDM benchmarks in the CLIC environment, an important question is whether other current or future collider options provide similar or better discovery prospects. Therefore, for the benchmarks proposed in \cite{Kalinowski:2018ylg,Kalinowski:2018kdn}, production cross sections for a variety of processes have been presented in \cite{Kalinowski:2020rmb}, including VBF-type topologies. Cross sections were calculated using {\texttt{Madgraph5}}. 
%
We list the considered collider types and nominal center-of-mass energies as well as integrated luminosities in table \ref{tab:colls}.
\begin{center}
\begin{table}
\begin{center}
\begin{tabular}{c|c|c|c}
collider&cm energy [\TeV]&$\int\mathcal{L}$&{$\sigma_{_{1000}}$} [\fb]\\ \hline
%LHC now&13&$140\,\fb^{-1}$&7\\
HL-LHC&13/ 14&$3\,\ab^{-1}$&0.33\\
HE-LHC&27&$15\,\ab^{-1}$&0.07\\
FCC-hh&100&$20\,\ab^{-1}$&0.05\\ \hline
ee&3&$5\,\ab^{-1}$&0.2\\
$\mu\mu$&10&$10\,\ab^{-1}$&0.1\\
$\mu\mu$&30&$90\,\ab^{-1}$&0.01
\end{tabular}
\end{center}
\caption{Collider parameters used in the discovery reach {study} performed in \cite{Kalinowski:2020rmb}. Collider specifications have been taken from \cite{ATL-PHYS-PUB-2019-005,Collaboration:2650976,Abada:2019ono,Benedikt:2018csr,Delahaye:2019omf} for the HL-LHC, HE-LHC, FCC-hh and muon collider, respectively. The last column denotes the minimal cross section {required to produce} 1000 events using full target luminosity.}
\label{tab:colls}
\end{table}
\end{center}
We here label a scenario "realistic" when we can expect 1000 events to be produced using target luminosity and center-of-mass energies as specified above. Obviously, more detailed studies, including {both} background {contribution and detector response} simulation, are necessary to assess the actual collider reach.

We consider the following production modes:
\begin{itemize}
\item{}{\bf $pp$ colliders at various center-of-mass energies:}
\begin{eqnarray*}
p\,p&\rightarrow&H\,A,\,H\,H^+,\,H\,H^-,\,A\,H^+,\,A\,H^-,\,H^+\,H^-,\,AA,\\
p\,p&\rightarrow&A\,A\,j\,j,\,H^+\,H^-\,j\,j.
\end{eqnarray*}
The latter two processes are labelled "VBF-like" toplogies, although in practise we include all diagrams that contribute to that specific final state; e.g., to the $A\,A\,j\,j$ final state, also $H^\pm\,A$ production contributes, with subseqent decays $H^\pm\,\rightarrow\,W^\pm\,A$ and hadronic decays of the $W$. Furthermore, all but the $A\,A$ direct pair production are proportional to couplings from the SM electroweak sector (see e.g. \cite{Ilnicka:2015jba}), so in principle these production cross sections are determined by the masses of the pair-produced particles. The $AA$ channel is proportional to 
\begin{\eqn}\label{eqn:l345b}
\bar{\lam}_{345}\,=\,\lam_{345}-2\,\frac{M_H^2-M_A^2}{v^2},
\end{\eqn}
 so dependences here are more involved.
\item{\bf $\mu\mu$ colliders:}\\
At the muon collider, we mainly consider
\begin{\eqn*}
  \mu^+\,\mu^-\,\rightarrow\,\nu_\mu\,\bar{\nu}_\mu A A,\;\;\;
  \mu^+\,\mu^-\,\rightarrow\,\nu_\mu\,\bar{\nu}_\mu H^+ H^-.
\end{\eqn*}
which again corresponds to VBF-like production modes. However, as before, we do not specify intermediate states, so in fact {several} diagrams contribute which not all have a typical VBF topology. See appendix B and C of \cite{Kalinowski:2020rmb} for details.
\end{itemize}
Figures \ref{fig:pppair} and \ref{fig:vbf} show the production cross sections as a function of the mass sum of produced particles for various collider options and production modes. We see clearly that, while {predictions for} direct pair-production {cross sections} at $pp$ colliders exhibit a fall with rising mass-scales for all but the $AA$ pair-production mode, {understanding the behaviour of} the VBF-induced channels {is} less trivial. This can be attributed to the fact that more diagrams contribute. For example, for $AAjj$ we have contributions from $h-$ exchange in the s-channel which are again proportional to $\bar{\lam}_{345}$ given by eqn. (\ref{eqn:l345b}), which can induce large jumps between cross-section predictions for scenarios with similar mass scales. Similar differences can be observed for VBF-type production at $\mu\mu$ colliders; this can be traced back mainly to a fine-tuned cancellation of various contributing diagrams, which is discussed in large detail in \cite{Kalinowski:2020rmb}.

\begin{figure}[htb]
\begin{center}%
\caption{Pair-production cross-section predictions at $pp$ colliders as a function of the sum of produced particle {masses}. {\sl Left:} For {all considered production channels at} 13 \TeV LHC. {\sl Right:} {for selected channels} at 13\,\TeV, 27 \TeV, and 100 \TeV. {Horizontal dashed lines} denote the limit of the cross section at which 1000 events are produced with the respective target luminosity, cf table \ref{tab:colls}. Taken from \cite{Kalinowski:2020rmb}.}
\label{fig:pppair}
\end{figure}
\begin{figure}[htb]
\begin{center}%
\caption{As figure \ref{fig:pppair}, but now considering the VBF-type production mode. {\sl Left:} for pp colliders, where two additional jets are produced and {\sl right:} at $\mu\mu$ colliders. Taken from \cite{Kalinowski:2020rmb}.}
\label{fig:vbf}
\end{figure}
The summary of sensitivities in terms of mass scales is given in table \ref{tab:sens}.
\begin{center}
\begin{table}
\begin{center}
\begin{tabular}{||c||c||c||c||} \hline \hline
{collider}&{all others}& { $AA$} & {$AA$ +VBF}\\ \hline \hline
HL-LHC&1 \TeV&200-600 \GeV& 500-600 \GeV\\
HE-LHC&2 \TeV&400-1400 \GeV&800-1400 \GeV\\
FCC-hh&2 \TeV&600-2000 \GeV&1600-2000 \GeV\\ \hline \hline
CLIC, 3 \TeV&2 \TeV &- &300-600 \GeV\\
$\mu\mu$, 10 \TeV&2 \TeV &-&400-1400 \GeV\\
$\mu\mu$, 30 \TeV&2 \TeV  &-&1800-2000 \GeV \\ \hline \hline
\end{tabular}
\end{center}
\caption{Sensitivity of different collider options specified in table \ref{tab:colls}, using the sensitivity criterium of 1000 generated events in the specific channel. $x-y$ denotes minimal/ maximal mass scales that are reachable. Numbers for CLIC correspond to results from detailed investigations \cite{Kalinowski:2018kdn,deBlas:2018mhx}.}
\label{tab:sens}
\end{table}
\end{center}
We see that especially for $AA$ production the VBF mode at both proton and muon  colliders serves to significantly increase the discovery reach of the respective machine. Using the simple counting criterium above, we can furthermore state that a 27 \TeV proton-proton machine has a similar reach as a 10 \TeV muon collider, while 100 \TeV FCC-hh would correspond to a 30 \TeV muon-muon machine. Obviously, detailed investigations including SM background are needed to give a more realistic estimate of the respective collider reach.
