\section{Two-real-singlet extension}\label{sec:2rs}
\subsection{The model}
The two-real-singlet model (TRSM) is a model that features the extension of the SM scalar sector by two additional real singlets $S,\,X$, which obey a $\mathbb{Z}_2\,\otimes\,\mathbb{Z}_2'$ symmetry. It has been introduced in \cite{Robens:2019kga}, with a first phenomenological study of the model being presented in \cite{Papaefstathiou:2020lyp}. The model is also available in the public tool \texttt{ScannerS}~\cite{Coimbra:2013qq,Costa:2015llh,Muhlleitner:2020wwk}.

The transformation properties under the two discrete symmetries are specified as
\begin{equation}
    \begin{aligned}
        \Ztwo^S:\quad & S\to -S\eqcomma\ X\to X\eqcomma\ \text{SM} \to \text{SM}\eqcomma \\
        \Ztwo^X:\quad & X\to -X\eqcomma\ S \to S\eqcomma\ \text{SM} \to \text{SM}.
    \end{aligned}\label{eq:Z2syms}
\end{equation}
Application of this symmetry reduces the number of possible terms in the potential, such that we obtain 
\begin{equation}
    \begin{aligned}
        V & = \mu_{\Phi}^2 \Phi^\dagger \Phi + \lambda_{\Phi} {(\Phi^\dagger\Phi)}^2
        + \mu_{S}^2 S^2 + \lambda_S S^4
        + \mu_{X}^2 X^2 + \lambda_X X^4                                              \\
          & \quad+ \lambda_{\Phi S} \Phi^\dagger \Phi S^2
        + \lambda_{\Phi X} \Phi^\dagger \Phi X^2
        + \lambda_{SX} S^2 X^2\eqdot
    \end{aligned}\label{eq:TRSMpot}
\end{equation}
So far, we have not specified whether the additional scalar states acquire vevs. In fact, setting one of these to zero opens up the possibility of having a portal-like dark matter scenario. On the other hand, when both additional fields acquire a vev, the above symmetry is softly broken and all scalar fields mix. We here discuss this second scenario. The gauge-eigentstates are then given by
\begin{equation}
    \Phi = \begin{pmatrix} 0\\\frac{\phi_h + v}{\sqrt{2}}\end{pmatrix}\eqcomma\quad
    S = \frac{\phi_S + v_S}{\sqrt{2}}\eqcomma \quad
    X = \frac{\phi_X + v_X}{\sqrt{2}}
    \label{eq:fields}
\end{equation}
Rotation into mass eigenstates is then described by a rotation matrix $R$, with
\begin{equation}
    \begin{pmatrix}
        h_1 \\h_2\\h_3
    \end{pmatrix} = R \begin{pmatrix}
        \phi_h \\\phi_S\\\phi_X
    \end{pmatrix}\eqdot
\end{equation}
where in the following we adapt the convention that 
\begin{\eqn*}
M_1\,\leq\,M_2\,\leq\,M_3.
\end{\eqn*}
The rotation matrix is described via three mixing angles $\theta_{1,2,3}$, with 
\begin{equation}
    R = \begin{pmatrix}
        c_1 c_2             & -s_1 c_2             & -s_2     \\
        s_1 c_3-c_1 s_2 s_3 & c_1 c_3+ s_1 s_2 s_3 & -c_2 s_3 \\
        c_1 s_2 c_3+s_1 s_3 & c_1 s_3-s_1 s_2 c_3  & c_2 c_3
    \end{pmatrix}\eqdot
\end{equation}
with the short-hand notation
\begin{equation}
    s_1\equiv\sin\theta_{hS}\eqcomma\quad s_2\equiv\sin\theta_{hX}\eqcomma\quad s_3\equiv\sin\theta_{SX}\eqcomma\quad c_1\equiv\cos\theta_{hS}\eqcomma\ \ldots
\end{equation}
It is important to note that all interactions to SM particles are inherited through this mixing, with a corresponding scaling factor $\kappa_i\,\equiv\,R_{i1}$ for the mass eigenstate $h_i$.

After electroweak symmetry breaking, the model has in total 9 free parameters; as before, two of these, $v\,\simeq\,246\,\GeV$ and $M_a\,\simeq\,125\,\GeV$, are fixed by the Higgs mass measurement and electroweak precision observables. We then choose as free input parameters
\begin{equation}
    M_b\eqcomma\ M_c\eqcomma\ \theta_{hS}\eqcomma\ \theta_{hX}\eqcomma\ \theta_{SX}\eqcomma\ v_S\eqcomma\ v_X\eqcomma\label{eq:TRSMpars}
\end{equation}
{with $a\neq{}b\neq{}c\in\lbrace1,2,3\rbrace$.}

As before, the model is subject to a large number of theoretical and experimental constraints, which have been presented in detail in \cite{Robens:2019kga} and will not be repeated here.

\subsection{Phenomenology and benchmark planes}
Having three distinct scalar final states, this model allows for interesting scalar-scalar production and decay modes. At $pp$ colliders, we have
\begin{eqnarray*}
pp\,\rightarrow\,h_3\,\rightarrow\,h_1\,h_1;&& 
%\item{}
pp\,\rightarrow\,h_3\,\rightarrow\,h_2\,h_2;\\
%\item{}
pp\,\rightarrow\,h_2\,\rightarrow\,h_1\,h_1;&&
%\item{}
pp\,\rightarrow\,h_3\,\rightarrow\,h_1\,h_2
%\vspace{4mm}
\end{eqnarray*}
with decay modes given by
\begin{center}
$h_2\,\rightarrow\,\text{SM}$; 
%\item{}
$h_2\,\rightarrow\,h_1\,h_1$; $h_1\,\rightarrow\,\text{SM}$
\end{center}
The exact phenomenology depends on the chosen parameter point. While all partial decay widths to SM-like final states are given by the common scaling factors defined above 
\begin{equation}
    \Gamma(h_a\to\text{SM}; M_a) = \kappa_a^2 \cdot \Gamma_\text{tot}(h_\text{SM}; M_a),\label{eq:widthscaling}
\end{equation}
{where $ \Gamma_\text{tot}(h_\text{SM}; M_a)$ denotes the decay width of a SM-like scalar of mass $M_a$,}
partial decays into scalar final states need to be calculated from the new physics parameters in the potential. 

In \cite{Robens:2019kga}, a number of benchmark planes was defined in order to accomodate for production and decay modes in the scalar sector that are currently not investigated by the LHC experiments. We list these in table \ref{tab:benchmarkoverview}.
\begin{center}
\begin{table}
    \centering
%    \begin{tabular}{\textwidth}{sssb}
{\small
\begin{tabular}{cccc}
        \hline \\
        benchmark scenario & $h_{125}$ & target signature      & possible successive decays                          \\
        \hline \\
        \textbf{BP1}       & $h_3$               & $h_{125} \to h_1 h_2$ & $h_2 \to h_1 h_1$ if $M_2 > 2 M_1$                  \\
        \textbf{BP2}       & $h_2$               & $h_3 \to h_1 h_{125}$ & -                                                   \\
        \textbf{BP3}       & $h_1$               & $h_3 \to h_{125} h_2$ & $h_2 \to h_{125}h_{125}$ if $M_2 > \SI{250}{\GeV}$  \\
        \textbf{BP4}       & $h_3$               & $h_2 \to h_1 h_1$     & -                                                   \\
        \textbf{BP5}       & $h_2$               & $h_3 \to h_1 h_1$     & -                                                   \\
        \textbf{BP6}       & $h_1$               & $h_3 \to h_2 h_2$     & $h_2 \to h_{125}h_{125}$ if  $M_2 > \SI{250}{\GeV}$ \\
        \hline
    \end{tabular}
}
    \caption{Overview of the benchmark scenarios: The second column denotes the
    Higgs mass eigenstate that we identify with the observed Higgs boson,
    $h_{125}$, the third column names the targeted decay mode of the resonantly
    produced Higgs state, and the fourth column lists possible relevant
    successive decays of the resulting Higgs states. Taken from \cite{Robens:2019kga}.}\label{tab:benchmarkoverview}
\end{table}
\end{center}

For some of these, especially BP4, BP5, and BP1, relatively high production rates of 60 \pb, 2.5 \pb, and 3 \pb respectively can be achieved at the 13 \TeV LHC. These numbers correspond to rescaled production modes
 \begin{equation}\label{eqn:sigfac}
    \sigma(M_a) = \kappa_a^2 \cdot \sigma_\text{SM} ( M_a).
\end{equation}
followed by factorized decays. $\sigma_\text{SM} ( M_a)$ denotes the NNLO+NNLL production cross section for a SM-like Higgs of mass $M_a$, with numbers taken from \cite{Heinemeyer:2013tqa}. The relatively large rates make these BPs prime targets for current LHC data analyses. In the following, we however concentrate on BP3 and the $hhh$ final state.

\subsection{$hhh$ production in the TRSM}
One interesting scenario within the TRSM is the asymmetric production and subsequent decay
\begin{equation}\label{eq:chain}
p p \,\rightarrow\, h_3\,\rightarrow\,h_2\,h_1\,\rightarrow\,h_1\,h_1
\,h_1,
\end{equation}
where $h_1\,\equiv\,{h}$ is the SM-like scalar. This signature is realized in BP3 and was analysed in detail in \cite{Papaefstathiou:2020lyp}. Input parameters for BP3 are displayed in table \ref{tab:BP3}.
\begin{center}
\begin{table}
\centering
\begin{tabular}{cc}
Parameter & Value \\	
\hline
$M_1$ & $125.09~\rm{GeV}$\\
$M_2$ & $[125,~500]~\rm{GeV}$\\
$M_3$ & $[255,~650]~\rm{GeV}$\\
$\theta_{hS}$ & $-0.129$\\
$\theta_{hX}$ & $0.226$ \\
$\theta_{SX}$ & $-0.899$\\
$v_S$ & $140~\rm{GeV}$\\
$v_X$ & $100~\rm{GeV}$\\ \hline
$\kappa_1$ & $0.966$\\
$\kappa_2$ & $0.094$\\
$\kappa_3$ & $0.239$
\end{tabular}	
\caption{\label{tab:BP3} The numerical values for the independent parameter values of eq.~(\ref{eq:TRSMpars}) that characterise \textbf{BP3}. The Higgs doublet vev, $v$, is fixed to 246~GeV. The $\kappa_i$ values correspond to the rescaling parameters of the SM-like couplings for the respective scalars and are derived quantities.}
\end{table}
\end{center}
We want to briefly comment on the calculation of rates, and differences between these in \cite{Robens:2019kga} and \cite{Papaefstathiou:2020lyp}:
\begin{itemize}
\item{}In \cite{Robens:2019kga}, production cross sections were calculated as specified in eqn. (\ref{eqn:sigfac}), and rates for final states were then derived via multiplication with the corresponding branching ratios. This is a priori a good approach for a first estimate, and incorporates important higher-order effects in the production cross sections.
\item{}In \cite{Papaefstathiou:2020lyp}, on the other hand, we made use of a customized  \texttt{loop\_sm} model implemented in \texttt{MadGraph5\_aMC@NLO} (v2.7.3)~\cite{Alwall:2014hca, Hirschi:2015iia}, and subsequently interfaced to \texttt{HERWIG} (v7.2.1)~\cite{Bahr:2008pv, Gieseke:2011na, Arnold:2012fq, Bellm:2013hwb, Bellm:2015jjp, Bellm:2017bvx, Bellm:2019zci}. This model includes full top and bottom mass effects and calculates production modes at LO, i.e. at the one-loop level. Furthermore, for the process (\ref{eq:chain}) intermediate states were not specified, which guarantees the inclusion of all contributing diagrams as well as interference effects. In particular, in some scenarions contributions from $s$-channel offshell $h_1$ states were on the $\%$ level.
\end{itemize}
Taking this into account, we display in figure \ref{fig:bp3} the production cross section for the ${h}\,h_2$ final state as derived in \cite{Robens:2019kga}. For this final state, production cross sections can reach up to 0.3 \pb.
\begin{figure}[htb]
\begin{center}%
\caption{Production cross section for the ${h}h_2$ final state in BP3 at a 13 \TeV LHC, as a function of $M_2$ and $M_3$. Experimental exclusion bounds stem from searches for $h_{2,3}\,\rightarrow\,V\,V$ from 2016 LHC Run II data \cite{Aaboud:2017rel,Sirunyan:2018qlb,Aaboud:2018bun}. Taken from \cite{Robens:2019kga}.}
\label{fig:bp3}
\end{figure}
Branching ratios for $h_2$ and ${h}\,h_2$ are displayed in figure \ref{fig:BRs}.
\begin{figure}[htb]
\begin{center}%
\caption{ Branching ratios for $h_2$ {\sl (left)} and ${h}\,h_2$ {\sl (right)} in BP3, as a function of $M_2$. Left plot taken from \cite{Robens:2019kga}.}
\label{fig:BRs}
\end{figure}
This particular benchmark plane was chosen such that $h_2\,\rightarrow\,{h}{h}$ becomes dominant as soon as it is kinematically allowed. This induces the predominance of $b\bar{b}b\bar{b}b\bar{b}$ and $b\bar{b}b\bar{b}W^+W^-$ rates over those for four-particle final states.

In \cite{Papaefstathiou:2020lyp}, several benchmark points were selected which were then investigated at the LHC using 14 \TeV center-of-mass energy, where we concentrated in the $b\bar{b}b\bar{b}b\bar{b}$ final state. Those benchmark points as well as significances for an integrated luminosity of $\int\,\mathcal{L}\,=\,300\,\fb^{-1}$ and  $\int\,\mathcal{L}\,=\,3000\,\fb^{-1}$ are shown in table \ref{tab:hhh}.
\begin{center}
\begin{table}
{\small
\begin{center}
\begin{tabular}{c||cc||cc}\\
{\bf $(M_2, M_3)$}& $\sigma(pp\rightarrow h_1 h_1 h_1)$ &
$\sigma(pp\rightarrow 3 b \bar{b})$&$\text{sig}|_{300\rm{fb}^{-1}}$& $\text{sig}|_{3000\rm{fb}^{-1}}$\\
${[\GeV]}$ & ${[\fb]}$  & ${[\fb]}$ & &\\
\hline\hline
$(255, 504)$ & $32.40$ & $6.40$&$2.92$&{  $9.23$}\\
$(263, 455)$ & $50.36$ & $9.95$&{ $4.78$}&{  $15.10 $}\\
$(287, 502)$ & $39.61$ & $7.82$&{  $4.01$} &{  $12.68$}\\
$(290, 454)$ & $49.00$ & $9.68$&{  $5.02$}&{  $15.86 $}\\
$(320, 503)$ & $35.88$& $7.09$& {  $3.76 $}&{  $11.88$}\\
$(264, 504)$ & $37.67$ & $7.44$&{  $3.56 $}&{  $11.27 $}\\
$(280, 455)$& $51.00$ & $10.07$&{  $5.18$} &{  $16.39$}\\
$(300, 475)$&$43.92$& $8.68$&{  $4.64 $}&{  $14.68 $}\\
$(310, 500)$& $37.90$ & $7.49$&{  $4.09 $}&{  $12.94$}\\
$(280, 500)$& $40.26$& $7.95$&{  $4.00 $}&{  $12.65 $}\\
\end{tabular}
\end{center}}
\caption{Benchmark points investigated in \cite{Papaefstathiou:2020lyp}, {leading-order} production cross sections at 14 \TeV, as well as significances for different integrated luminosities.}
\label{tab:hhh}
\end{table}
\end{center}
For details of the analysis as well as SM background simulation, we refer the reader to the above work. We see that several of the benchmark points are in the 4-5 $\sigma$ range already for a relatively low luminosity, and all have significances above the discovery reach after the full run of HL-LHC. We therefore strongly encourage the experimental collaborations to adapt our search stragety, using actual LHC data.

Finally, we can ask whether other channels can not equally constrain the allowed parameter space at the HL-LHC. To this end, we have extrapolated various analyses assessing the heavy Higgs boson prospects of the HL-LHC in final states originating from $h_i \rightarrow h_1 h_1$ \cite{Sirunyan:2018two,Aad:2019uzh}, $h_i \rightarrow ZZ$ \cite{Sirunyan:2018qlb,Cepeda:2019klc} and $h_i \rightarrow W^+W^-$ \cite{Aaboud:2017gsl,ATL-PHYS-PUB-2018-022}, for $i=2,3$, and combined these with extrapolations of results from 13 TeV where appropriate. Details of the extrapolation procedure can be found in Appendix~D of ref.~\cite{Papaefstathiou:2020iag}. The corresponding results are shown in figure \ref{fig:hlothers}.

\begin{center}
\begin{figure}[htb]
\begin{center}%{%
\includegraphics[width=0.48\textwidth]{M2M3_constraints_HLLHC.pdf}
\end{center}
\caption{Constraints on the $\lb M_2,\,M_3 \rb$ plane from extrapolation of other searches at the HL-LHC from extrapolation (see text for details). Taken from \cite{Papaefstathiou:2020lyp}.}
\label{fig:hlothers}
\end{figure}
\end{center}

Especially $ZZ$ final states can probe nearly all of the available parameter space. However, we want to emphasize that these depend on different model parameters than the $h_1\,h_1\,h_1$ final state rates, and therefore these searches can be considered as complementary, testing various parts of the new physics potential. We encourage the LHC experimental collaborations to pursue searches in all possible decay channels.
