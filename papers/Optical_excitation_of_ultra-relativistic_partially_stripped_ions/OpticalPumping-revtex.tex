\section{Exciting ultra-relativistic ions with light}
\label{sec:OptPump}

In high-energy physics, a cross section is often used to describe a scattering process. While in optics the cross section is also used, its application implicitly assumes that the process is investigated in the steady state, when dynamic equilibrium between different processes (e.g., excitation and relaxation) is reached.  Prior to the steady state, however, the system experiences a transient period during which it may undergo significant changes.  The dynamics of this transient evolution depends on many parameters, including incident-light intensity, strength of an atomic transition,
light detuning or excited-state relaxation.  Thereby, over time comparable with the excited-state lifetime $\tau_e$ the population of the excited state, which determines the intensity of the fluorescence from the ions (see below), may continuously increase, eventually reaching its steady-state value, but it may also experience oscillations before finally leveling up at a specific value.  As the frequency and amplitude of these, so-called, Rabi oscillations depend on parameters of incident light, studies of the dynamics of PSI excitation at times shorter than the excited-state lifetime become an important aspect of the GF.

Below, we analyze, both theoretically and numerically, the problem of optical excitation of an energy-dispersed ion bunch by a pulse of light.  By investigating the interaction of a resonant light pulse with a generic closed two-level system, i.e., with a system where levels other than these directly coupled by light are ignored, we analyze the situation which, to the first order, reproduces the GF set-up.  

\subsection{Theoretical model}

We consider an excitation of a two-level atom with semi-resonant light, $\Delta\omega\ll\omega$, where $\omega$ is the light frequency and $\Delta\omega$ is its detuning from the optical transition.  Since the two-level system is considered, there are no dark states, which could prevent the ions from further excitation.  In this system, the atoms are characterized with the excited-state relaxation rate $\gamma_e$, and we also assume that the ground-state lifetime is infinite, $\gamma_g=0$. Finally, the interaction is considered in the rotating-wave approximation, when interaction with only a resonant components of the light field ($\omega\approx\omega_0$) is considered, while the effect of the second (conjugate) frequency component of light, $-\omega$, is neglected.

In order to determine scattering of photons by the atoms, the time-dependent expectation value of the spontaneous-emission operator $\mathcal{F}$ needs to be calculated.  Herein, we calculate the value using the density-matrix formalism
\begin{equation}
	\langle\mathcal{F}\rangle=\textrm{Tr}(\rho\mathcal{F}),
	\label{eq:expF}
\end{equation}
where $\rho$ is the density matrix of the atoms \cite{AuzinshBook2010}.  Evolution of the density matrix can be described using the Liouville equation
\begin{equation}
	\dot{\rho}=\frac{i}{\hbar}[H,\rho]-\frac{1}{2}\left\{\Gamma,\rho\right\},
	\label{eq:Liouv}
\end{equation}
where $H$ is the Hamiltonian of the system, containing the contribution from the Hamiltonian of the unperturbed atoms $H_0$ and the operator $V$, describing their interaction with light. The operator $\Gamma$ describes the relaxation in the system, in particular the relaxation of the excited state due to spontaneous emission.  It can be shown \cite{AuzinshBook2010} that, in the case of a two-level system, the matrix elements of the fluorescence operator are given by
\begin{equation}
	    F^{e}_{g}=\frac{4}{3}\,\frac{\omega_{0}^3}{\hbar c^3}\;\vec{d}_{ge}\cdot\vec{d}_{eg},
	\label{eq:Fge}
\end{equation}
where $\vec{d}_{eg}$ is the electric dipole moment between the ground state $g$ and the excited state $e$.  Because the electric dipole moment is an odd operator, the only nonzero elements of the fluorescence operator $\mathcal{F}$ are at a diagonal.  Moreover, since the fluorescence arises exclusively due to spontaneous emission, and the ground state is relaxation free, the fluorescence of the atoms is proportional to the excited-state population $\rho_{ee}$.  Hence the time-dependent fluorescence operator expectation value is given by
\begin{equation}
	\langle\mathcal{F}\rangle= \frac{3\gamma_e N_{\rm PSI}}{\hbar c^3}\,\rho_{ee},
	\label{eq:FluorescenceFinal}
\end{equation}
where $N_{\rm PSI}$ is the number of partially stripped ions.

Since the only dynamic parameter in Equation~\ref{eq:FluorescenceFinal} is the excited-state population $\rho_{ee}$, henceforth we investigate evolution of the population.  Moreover, normalization of the population, i.e., $\rho_{gg}+\rho_{ee}=1$, where $\rho_{gg}$ is the ground-state population, allows to relate the population $\rho_{ee}$ with the probability of the excited-state occupation.  This provides an intuitive insight into the efficiency of PSI excitation/fluorescence; the higher the excited-state population, the more intense fluorescence from the ions.

The problem of the interaction of classical light with the two-level atom using the Liouville equation is considered in many textbooks (see, for example, Ref.~\cite{AuzinshBook2010}).  The evolution of the density matrix elements are given by
\begin{eqnarray}
        \dot{\rho}_{eg}&=&\left(i\Delta\omega-\frac{\gamma_e}{2}\right)\rho_{eg}+\frac{i\Omega_R}{2}(\rho_{gg}-\rho_{ee}),\label{eq:LiouvilleEquationGeneralFirst}\\ 
        \dot{\rho}_{ge}&=&-\left(i\Delta\omega+\frac{\gamma_e}{2}\right)\rho_{ge}-\frac{i\Omega_R}{2}(\rho_{gg}-\rho_{ee}),\\ 
        \dot{\rho}_{ee}&=&\frac{i\Omega_R}{2}(\rho_{ge}-\rho_{eg})-\gamma_e\rho_{ee}, 
        \label{eq:LiouvilleEquationGeneralLast}
\end{eqnarray}
where $\rho_{eg}$ is the envelope of optical coherence (an amplitude of the superposition between the ground state $g$ and excited state $e$) and 
\begin{equation}
\Omega_R=c\,\sqrt{\frac{6\pi\gamma_e I}{\hbar\omega_0^3}}
\label{eq:RabiFreq}
\end{equation}
is the Rabi frequency, characterizing the coupling strength between light and ions, with $I$ being light intensity.  Solving this set of equations allows one to determine the excited-state population, and hence the number of fluorescence photons.

\subsection{Scattering at steady state}
\label{subsec:SteadyState}

Let us first consider the stationary situation when equilibrium between various processes is achieved, i.e., the steady-state situation.  In such a regime, the left-hand sides of Equations~\ref{eq:LiouvilleEquationGeneralFirst}--\ref{eq:LiouvilleEquationGeneralLast} are equal to zero, $\dot{\rho}=0$, which, through algebraic manipulations of Equations \ref{eq:LiouvilleEquationGeneralLast} allows us to calculate the excited-state population
\begin{equation}
\rho_{ee}=\frac{\Omega_R^2/4}{\Delta\omega^2+\gamma_e^2/4+\Omega_R^2/2}=\frac{\kappa_1/2}{1+4\Delta\tilde{\omega}^2+\kappa_1},
    \label{eq:PopulationSteadyState}
\end{equation}
where $\Delta\tilde{\omega}=\Delta\omega/\gamma_e$ is the normalized detuning and $\kappa_1=2\Omega_R^2/\gamma_e^2$ is the saturation parameter, relating the strength of the light--atom coupling (given by the Rabi frequency $\Omega_R$) to the system's relaxation $\gamma_e$.   In particular, Equation~\ref{eq:PopulationSteadyState} shows that the excited-state population, and hence fluorescence, depends on the light intensity and detuning.  

Comparison of Equation~\ref{eq:PopulationSteadyState} with the classical absorption cross section \cite{Hulst2012Light},
\begin{equation}
    \sigma=\frac{\sigma_0}{1+4\Delta\omega^2/\gamma_e^2+2\Omega_R^2/\gamma_e^2}=\frac{\sigma_0}{1+4\Delta\omega^2/\gamma_t^2}\,,
    \label{eq:AbsorptionCrossSection}
\end{equation}
where $\sigma_0$ is the resonant absorption cross section and $\gamma_t$ is the transition linewidth, reveals similarity between the classical and quantum approach.  In particular, both approaches show that the further the light is detuned from the optical transition, the less efficient the excitation is.  In both approaches scattering also depends on the transition linewidth $\gamma_t$ (full width at half maximum -- FWHM), which in the classical approach is light-intensity independent and is determined by the excited-state relaxation rate $\gamma_e$, $\gamma_t=\gamma_e$.  The difference between the classical and quantum mechanical approach arises at higher light intensities.  In such a case, the quantum-mechanical approach incorporates the saturation effect, which modifies the effective linewidth of the transition, $\gamma_t=\gamma_e \Delta\omega/\sqrt{\Delta\omega^2+\Omega_R^2/2}$ and causes leveling up the efficiency of the excitation at 1/2 for $\Omega_R^2\gg \Delta\omega+\gamma_e^2/4$.  The saturation effect stems from the finite number of atoms that can be excited by light and finite time the excited atom needs to emit the photon (as discussed above, the excited atoms emits a photon with the characteristic time $\tau_e$).  This means that at some point further increase of the incident light intensity does not result in the increase of the number of absorbed/scattered photons as all atoms are already involved into scattering.  The intensity at which the saturation occurs depends on the light detuning, i.e., saturating the transition with off-resonance light requires more intense incident light than it is the case for on-resonance light.
\begin{figure}[!htbp]
    \centering
    \includegraphics[width=0.95\linewidth]{Figures/FluorescenceSteadyk1.pdf}
    \caption{The steady-state excited-state population $\rho_{ee}$, determining the fluorescence of the PSI illuminated with light, versus the saturation parameter $\kappa_1$ for three different normalized detunings: $\Delta\tilde\omega=0$ (red), $\Delta\tilde\omega=1$ (green) and $\Delta\tilde\omega=-5$ (blue).}
    \label{fig:FluorescenceSteadyState}
\end{figure}

The discussion presented above is depicted with the quantum-mechanical results shown in Figure~\ref{fig:FluorescenceSteadyState}, where the excited-state population is plotted against the saturation parameter $\kappa_1$ for three different detunings.  As expected, increasing the light power (saturation parameters) improves the excited-state population and hence the intensity of the scattered light.  While initially the process linearly depends on incident-light intensity, at higher intensity it begins to saturate.  The results also show that saturating the transition with detuned light requires higher light intensity.  This dependence also indicates that saturating moving ions, whose transition frequencies in the laboratory frame (LF) are modified due to the Doppler effect, with a CW light, is more challenging than in the case of motionless ions.

The remaining question is how fast the system reaches the steady state, which can be rephrased into: when the classical approach still adequately describes the atom--light interaction (even if saturation is somehow taken into account in the classical approach).  From Equations~\ref{eq:LiouvilleEquationGeneralFirst}--\ref{eq:LiouvilleEquationGeneralLast} one can generally conclude that the system reaches the steady state for times on the order of $\tau_e$  (more careful analysis reveals that steady-state population is reached at $t\approx 5\tau_e$).  This reveals a fundamental role of spontaneous emission, which acts as a dephasing mechanism for the Rabi oscillations;  initially, all the ions oscillate in phase, but every time spontaneous emission occurs the phase of the ion is randomized. As a result, after the time comparable with the excited-state lifetime, the Rabi phase of majority of the ions is reset and the system reaches the ``incoherent" steady state with a given distribution between the ground and excited state population. Simultaneously, if the interaction time is shorter than $\tau_e$, the classical approach using the cross section does not work and dynamics of the system needs to be evaluated using the quantum-mechanical formalism.  This evolution is discussed in the following section.

\subsection{Dynamics of ion excitation\label{Sec:DynamicsPumping}}

The ultrarelativistic nature of the GF ions results in a significant difference in the flow of time in the ion-rest and laboratory frames.  As a result, the excited-state lifetime in the ion-rest frame (IRF) $\tau_e$ corresponds to the LF excitation time $\tau_e^{\rm LF}$ via
\begin{equation}
    \tau_e^{\rm LF}=\gamma_L\,\tau_e.
\end{equation}
As a consequence, the average path an excited ion propagates in the LF after the excitation is
\begin{equation}
    l^{\rm LF}=c\gamma_L\,\tau_e.
\end{equation}

\begin{figure}[!htbp]
    \centering
    \includegraphics[width=0.95\linewidth]{Figures/Linewidths.pdf}
    \caption{Schematics of spectral characteristics of the system. The red line shows the probability of excitation of the motionless PSI. In a weak-light regime, the probability is given by the Lorentz function whose full width at half maximum (FWHM) is determined by the excited-state relaxation rate $\gamma_e$. The middle orange line corresponds to the spectral profile of the pulse used for the excitation, where shading indicates the frequency range that can be used for the ion excitation. The broadest profile corresponds to the transition line inhomogeneously broadened due to the Doppler effect.}
    \label{fig:SchematicExcitation}
\end{figure}

The ions intended to be used in the GF will be excited at a relatively narrow transition, $\omega_0\gg\gamma_e$.  However, due to the ion energy dispersion 
$\Delta \mathcal{E}/\mathcal{E}=\Delta\gamma_L/\gamma_L$, the transition is inhomogeneously broadened (the Doppler effect).  In the PoP experiment \cite{GF-PoP-LoI:2019}, the Doppler broadening of the transition is 2--4 orders of magnitude larger than its natural width $\gamma_e$, as schematically depicted in Figure~\ref{fig:SchematicExcitation}.  As shown with Equation~\ref{eq:PopulationSteadyState}, to saturate a transition with detuned light requires higher intensities. In fact, to do so for the light detuned by $\Delta\omega$ the intensity should be roughly $4\Delta\omega^2/\gamma_e^2$ times higher. In turn, to saturate the whole Doppler-broadened transition with CW light, the light intensity in the PoP experiment would have to be increased by 4--8 orders of magnitude. On the one hand, this may be difficult, if possible at all, but more importantly, due to the finite interaction region, the ions do not experience the CW light but rather a light pulse of the Fourier-broadened spectrum.  In fact, we exploit this effect to facilitate the interaction and more efficiently excite the ions.  Specifically, we aim at generating pulses, which spectral width coincides with the Doppler-broadened transition of the ions on the bunch
\begin{equation}
    t_p^{\rm IRF}=\frac{2\gamma_L}{\sigma^{\rm LF}_\omega},
\end{equation}
where $\sigma^{\rm LF}_\omega$ is the root-mean-square (rms) pulse width in the lab frame.

The spectral broadening of the pulse significantly beyond the transition natural linewidth has an important consequence.  Specifically, it allows to neglect the relaxation of the ions during the interaction, which significantly simplifies the theoretical description.  In fact, the interaction of light with the relaxation-free two-level atom is a textbook example (see, for example, Ref.\ \cite{Letohov1977Nonlinear}), demonstrating oscillations of the excited-state population $\rho_{ee}(t)$ at the Rabi frequency $\Omega_R$ (the Rabi oscillations).  While dynamics of this coherent, i.e., uninterrupted by spontaneous emission, evolution is harder to determine in the case of pulsed excitation, where the Rabi frequency varies over time, we can generally state that the excited-state population at the time moment $t_1$ is given by $\sin^2\left(\int_0^{t_1}\Omega_R(t)dt\right)$, where the expression under the sine function is the total Rabi phase of the oscillation.  In fact, the dependence may enable mimicking the short-pulse (dynamic) regime even in Monte-Carlo simulations.  This would be the case when one is not interested in describing the whole evolution of the system during the pulse but rather aims at the state of the atoms after transition of the pulse.

From the perspective of the GF, the last consequence of the short length of the pulse is the absence of ion--ion interactions (e.g., collisions) during the light pulse.  Thereby, the PSI in different velocity classes can be treated independently and the problem can be further simplified and the excited-state population of the whole bunch is simply a weighted average over the ions' velocity distribution.  

\begin{figure*}[!htbp]
    \centering
    \includegraphics[width=1\linewidth]{Figures/DynamicsAndEfficiencyOfPulsedPumping.pdf}
    \caption{(Left) The excited-state population of the PSI interacting with the Gaussian light pulse (red trace) of a spectral width coinciding with the atoms' Doppler profile, $\gamma_p^{\rm IRF}=\sigma_\omega^{\rm IRF}$.  Different traces corresponds to different detunings of the light central frequency $\omega$ from the Doppler-shifted resonance frequency: $\Delta\omega/\sigma_\omega^{\rm IRF}=0$ (blue), $\Delta\omega/\sigma_\omega^{\rm IRF}=0.5$ (yellow) and $\Delta\omega/\sigma_\omega^{\rm IRF}=1$ (green).  (Right) The population of the excited states after the pulse (blue dots) along with the number of the PSI in a specific velocity class (red line) versus the normalized detuning.  The results indicate that the PSI-distribution averaged population of the exited state is 70\%. The simulations were performed for the pulse spectrally covering the whole inhomogeneously broadened spectral line and the amplitude of the pulse $\Omega_R^0\approx18000\,\gamma_e$.}
    \label{fig:RabiOscillationsPulsed}
\end{figure*}

The final stage of our discussion concerns the efficiency of the excitation of atoms with different detunings.  In the case of the interaction with the light pulse spectrally coinciding with the Doppler width, the different velocity classes are resonantly excited by appropriate spectral components of the light.  This alleviates the demanding requirement for the light intensity, which may lead to the problems with photoionization or multiphoton excitation.  Moreover, this also provides a better control over the efficiency of the interaction. As shown in Figure~\ref{fig:RabiOscillationsPulsed}, the interaction with a pulse, whose central frequency $\omega$ is detuned from the Doppler-shifted resonant frequency by the rms Doppler width, $\Delta\omega=\sigma_\omega^{\rm LF}$ reduces the excitation efficiency by half. In the case of spectrally narrow light, this case would correspond to very small excitation, unless extremely intense light is used. 

Under the assumption of a short light pulse, the efficiency of the excitation of the whole PSI bunch can be calculated by averaging over the distribution due to the energy dispersion. When the pulse amplitude is chosen in such a way that in-resonance ions undergo half of the Rabi cycle -- the whole population is transferred into the excited state -- and the pulse width coincides with the ions Doppler broadening, $\gamma_p^{\rm IRF}=\sigma_\omega^{\rm IRF}$, the efficiency of excitation of complete ion bunch would reach 70\%.

\subsection{Parameters of envisioned experiments}

Let us now discuss the excitation of PSI in the main three scenarios considered for the GF: lithium-like lead, planned to be used in the PoP experiment in the SPS \cite{GF-PoP-LoI:2019}, hydrogen-like lead, envisioned for the LHC experiment \cite{Bessonov:1995hd}, and lithium-like calcium, considered for optical cooling of accelerator beams \cite{Krasny:2020wgx}.  
The case of the helium-like calcium, discussed in Section~\ref{section.IonChoice},
is not considered here but will be presented in our future studies on other possible applications of the GF.
Table~\ref{table:OpticalPumpingParameters} summarizes the physical parameters based on our assumptions, data from the literature, and results of calculations performed in this work.

\begin{table*}[!htbp]
    \caption{The optical parameters for the planned GF experiments. 
    The relative Gaussian energy spread of $2\times 10^{-4}$ for both the PSI bunch and the laser pulse is assumed 
    in all cases.}
    \centering
    \begin{threeparttable}
    \begin{tabular}{l|c|c|c}
    \hline\hline
    Parameter name & 
    Li-like ${}^{208}_{\phantom{0}82}$Pb$^{79+}$ & 
    Li-like ${}^{40}_{20}$Ca$^{17+}$ & 
    H-like ${}^{208}_{\phantom{0}82}$Pb$^{81+}$ \\
         \hline
         {Electronic transition} & $2s\rightarrow 2p_{1/2}$ &  $2s\rightarrow 3p_{1/2}$ & $1s\rightarrow 2p_{1/2}$ \\
         {Transition energy $\omega_0$ [eV]} & $230.823\,(47)(4) $ \cite{YerokhinSurzhykov:2018a,YerokhinSurzhykov:2018b} & $661.89$ \cite{NIST-ASD} & $75\,280.83\,(26)$ \cite{YerokhinShabaev:2015} \\
         {Excited-state lifetime $\tau_e$ [ps]} & $76.6$ \cite{JohnsonLiuSapirstein:1996} & $0.43$\tnote{(a)} & $3.4\times 10^{-5}\,$\tnote{(a)}  \\
         Excited-state relaxation rate $\gamma_e$ [s$^{-1}$] & $1.3\times 10^{10}$ & $2.3\times 10^{12}$& $3.0\times 10^{16}$ \\
         rms Doppler width $\sigma_\omega$ [s$^{-1}$] & $7.0\times 10^{13}$ & $2.0\times 10^{14}$ & $2.3\times 10^{16}$\\ \hline
         Pulse energy [mJ] & 0.2 and 5.0 & 0.35 and 2.0 & $5.0\,$\tnote{(b)} \\
         LF radiation energy [eV] & 1.2 & 1.6 & 12.6\\
         rms LF pulse length $\tau_p^{\rm LF}$ [ps] & $2.8$ & $2.0$& $500$  \\
         rms\ transverse pulse size $\sigma^p_x=\sigma^p_y$ [m] & $6.5\times 10^{-4}$ & $5.6\times 10^{-4}$ & $2.5\times 10^{-5}$  \\ \hline
         Number of ions per bunch $N_b$ & $9\times 10^7$ & $4\times 10^9$& $9.4\times 10^7$\\
Lorentz relativistic factor of ion $\gamma_L$ & $96.3$ & $205.62$ & $2989$ \\
         rms\ transverse ion beam size [m] & $\sigma_x = 10.5 \times 10^{-4}$ & $\sigma_x=8.0 \times 10^{-4}$ & $\sigma_x = 3.9\times 10^{-5}$ \\
          & $\sigma_y=8.3 \times 10^{-4}$ &  $\sigma_y=5.7 \times 10^{-4}$ & $\sigma_y = 3.9\times 10^{-5}$ \\
         rms\ ion bunch length $\sigma_z$ [m] & $0.06386$ & $0.10$& $0.15$  \\ \hline
         rms IRF pulse spectral width $\gamma_p^{\rm IRF}$ [s$^{-1}$] & $6.9\times 10^{13}$ & $1.4\times 10^{14}$ & $8.4\times 10^{12}$ \\
         Spatio-temporal IRF Rabi amplitude $\Omega_0^{\rm IRF}$ [s$^{-1}$] & $6.2\times 10^{13}$ and $3.1\times 10^{14}$ & $6.7\times 10^{14}$ and $1.6\times 10^{15}$ & $4.8\times 10^{15}$\\
         Characteristic LF excitation distance $x^{\rm LF}$ [m]  & $2.2$ & $0.027$ & $3\times 10^{-5}$ \\ \hline
         Number of photons emitted from the bunch  & $2.0\times 10^6$ and $8.6\times 10^{6}$ & $5.0\times 10^8$ and $5.7\times 10^8$ &  $3.8\times 10^{8}$\\
         \hline\hline
    \end{tabular}
    \begin{tablenotes}\footnotesize
      \item[(a)] This work.
      \item[(b)] Power inaccessible for the current light sources at wavelength of $\approx 100$\,nm, but anticipated in the future.
    \end{tablenotes}
    \end{threeparttable}
    \label{table:OpticalPumpingParameters}
\end{table*}

The ion bunch is characterized with a three-dimensional Gaussian function with the rms widths $\sigma_x$, $\sigma_y$, and $\sigma_z$ in all three directions with the dominant width along $z$ (a cigar-like shape).  In the interaction region, the light-pulse-intensity profile is given by a symmetric Guassian function of a transverse width of $\sigma^p_x=\sigma^p_y$.  In general, the transverse sizes of light pulse and bunch are not matched, which affects the efficiency of the interaction (a number of light accessible ions limited by a geometrical factor).  The efficiency may be further reduced by a non-zero angle between laser and ion beam directions, i.e., imperfect anti-collinear alignment of the two beams.  The excitation of PSI is induced by light pulse, which central frequency $\omega$ is tuned to the center of the Doppler broadened transition $\omega_0$, $\omega=\omega_0$.

All the ions considered in this work are characterized with the energy-level structure of a total angular momentum of 1/2 in the ground state and a total angular momentum of 1/2 in the excited state.  Despite the fact that there are two magnetic sublevels in either of the states, as well as specific selection rules associated with the excitation for a given light polarization, it can be shown that such a system may be effectively reduced to the two-level system discussed above.

In the PoP experiment at the SPS, lithium-like lead ions ($^{208}_{\phantom{0}82}{\rm Pb}^{79+}$) will be used.  With the Lorentz factor $\gamma_L=96.3$ and a transition energy between the two lowest electronic levels of $\sim230\,$eV, one can show that the transition can be induced with a Ti:sapphire laser, emitting infra-red radiation at $1035\,$nm.  Motivated by the technical limitations, we have chosen the pulse of the LF length of 2.8\,ps (rms width), i.e., the pulse spectral width covers 70\% of the bunch Doppler width.  While the interaction efficiency depends on temporal and spectral parameters of the experiment, it is difficult to {\em a priori} determine the parameters of the pulse.  Therefore, we consider two pulse energies: 0.2\,mJ and 5\,mJ.  The first pulse transfers about 35\% of the zero-detuning atoms residing in the center of the bunch, $x=y=0$.  Despite such a high efficiency of excitation in the center, accounting for the beam and bunch profiles significantly modifies the excitation, so that the overall efficiency of the bunch excitation reaches just 2.2\%.  Increasing the pulse energy by a factor of 25, to 5\,mJ, results in roughly one full Rabi cycle experienced by the ions in the bunch center that are tuned to the resonance.  As very few ions of the group are being excited, this may suggest that the overall efficiency of excitation is smaller than in the previous case.  However, because of the experimental geometry (non-zero angle between laser and ion beam directions), increasing the light power results in excitation of the ions that remained in the ground state. In turn, the overall efficiency of the ion excitation rises from 2.2\% to 9.6\%.  The correspond single-bunch scattered-photon number rises from $2.0\times 10^6$ in the first case to $8.9\times 10^6$ in the second case.  The presented results suggest that increasing the pulse energy increases the excitation efficiency.  Indeed, for the energy range between 0 and 2.5\,mJ, the efficiency monotonically increases in nonlinear manner, reaching its maximum of 10.7\% at 2.5\,mJ.  For higher energies, the efficiency drops, which is an indication of complex dynamics of the system, yet the dependence is much weaker (from 2.5\,mJ to 5\,mJ the redaction is from 10.7\% to 9.6\%).  Independently from the actual excitation efficiency, the photons will be emitted in the forward cone within several nanoseconds over a distance of a few meters.

A similar situation is encountered in the case of the lithium-like calcium ion ($^{40}_{20}$Ca$^{17+}$), i.e., the Doppler-broadened transition is several orders of magnitude larger than the natural linewidth.  It should be noted, however, that the difference between linewidths is smaller than in the previous case, which manifests via a larger contribution of the relaxation to the system evolution.  With a Lorentz factor of 205, the PSI may be excited with the light pulse of the carrier wavelength of $771\,$nm, and the pulse length of $2\,$ps allows to spectrally cover 70\% of the Doppler-broadened transition.  Performing the optimization of the pulse energy for given pulse parameters, we have found that below 1\,mJ, the maximum excitation intensity of 12.5\% is achieved for the pulse energy of 0.35\,mJ (about 14.2\% of excitations can be achieved for the pulse energy of 2\,mJ).  This corresponds to about $5.0\times 10^{8}$ ($5.7\times 10^{8}$) photons emitted from each PSI bunch within about $10\,$ps over a distance of about $3\,$mm in the LF.

A different situation is encountered in the case of hydrogen-like lead ions ($^{209}_{\phantom{0}82}{\rm Pb}^{81+}$).  As shown in Table~\ref{table:OpticalPumpingParameters}, the natural linewidth of the transition is comparable with the Doppler width.  Thus, it can be assumed that even with a spectrally narrow pulse all the ions are excited with a comparable efficiency. Specifically, the efficiency drops by a factor of $2$ for the detuning $\gamma_e/2$. Another important difference is the light-induced evolution of the excited-state population, determining the fluorescence of the ions.  In contrast to the previous cases, the pulse is orders of magnitude longer than the excited-state lifetime, thus the system reaches the equilibrium during each instant of the optical pulse. Thereby, the classical cross-section approach can be used in the considered case.  Moreover, the difference in the excited-state lifetime and pulse length suggests that during the light pulse the ions may undergo multiple excitation--emission cycles.  This significantly enhances the photon emission from the ions, even though the efficiency of the excitation is low.  Our calculations show that despite the very low excitation efficiency (at most 0.4\%) with the LF light of the carrier wavelength of 98.46 nm, the pulse energy of 5\,mJ and the pulse rms length of 500\,ps, i.e., the parameters that are not accessible experimentally at the moment but are foreseen in the future, the ions can be excited many times during the pulse.  In turn, under the experimental conditions, each ion can be excited 4 times on average, corresponding to $3.8\times 10^8$ emitted photons for each bunch.  Due to the low saturation, the number of emitted photons would scale linearly with both the length and energy of the pulse, revealing the room for further improvements.

An important question in our consideration is a potential presence of additional energy levels that may trap the ions, making them inaccessible to light. Such levels of energies lower than the energies of the corresponding excited states do not exist in the case of $^{208}_{\phantom{0}82}{\rm Pb}^{79+}$ and $^{208}_{\phantom{0}82}{\rm Pb}^{81+}$ ions. Albeit there is such a level in the $^{40}_{20}{\rm Ca}^{17+}$ ion, in the considered scenario of interaction with a light pulse that is several orders of magnitude shorter than the excited-state lifetime, the spontaneous decay can be neglected, so that this level may not contribute to the dynamics of the interaction. Principally, the problem might be with higher-energy levels present in the ions. However, for the laser-pulse intensities considered in this paper, the probability of multi-photon absorption to these levels is low, so that their effect can also be neglected. In turn, the two-level model well describes the systems under consideration.