\section{Simulations of optical excitation}
\label{sec:MCsim}

As discussed in Subsection~\ref{subsec:SteadyState}, in the steady-state and low-intensity regime, the calculations of the ion-light interactions may be performed using the classical cross section. In particular, numerical simulations of the interaction in the GF can be performed in a similar manner as for light sources based on the inverse Compton scattering~\cite{Sprangle:1992zza}.  In such a case, one only needs to replace the Compton-scattering cross section with the absorption cross section given by Equation~\ref{eq:AbsorptionCrossSection}. We implement this approach using {\sf GF-CAIN} \cite{GF-CAIN}, a Monte Carlo event generator, being a customized (GF-adapted) version of the simulation code {\sf CAIN} \cite{CAIN} developed by K.\ Yokoya {\em et al.} at KEK-Tsukuba, Japan, for the ILC project \cite{ILC}. 

The PSI bunch is characterized in terms of standard high-energy particle beam parameters, such as emittances, beta-functions, etc., while the laser pulse is described by a space-time profile function, e.g.\ the Gaussian distribution.  Since the number of particles in a bunch can be huge, making the simulation of interactions of each individual particle with laser-photons unfeasible due to CPU-time and computer memory limitations, one usually replaces some number of physical particles by the, so-called, macroparticle and performs the actual simulations for a lower number of such macroparticles.  Then, to each macroparticle one assigns a Monte Carlo weight, which is equal to the number of physical particles  it represents.  Of course, the smaller the weight the better, because then the simulations are more precise in terms of systematic effects. 

The simulations proceed in such a way that at the beginning the bunch of macroparticles and a laser pulse are put some distance away in the $z$-direction in the LF, and then as time progresses they pass through each other in discrete time steps.  In each time step, the macroparticles interact with laser photons according to a given probability and, if excited, secondary photons are emitted from them through spontaneous emission.  After a predefined number of steps the simulation is finished.  Then, the outgoing particles can be transformed without interactions to a given value of the $z$-coordinate or the time-coordinate.  In this final position, the space-time coordinates and four-momenta of all outgoing particles are recorded. 

The basic quantity used in the simulation of the interaction of a single macroparticle $m$ ($m=1,\ldots, M$), being in the spatial position $\vec{r}$ and having the momentum $\vec{p}$ in the LF, with laser photons is the scattering probability during a time step $\Delta t_i = t_i - t_{i-1}$, where $t_i$ is the time of the $i$-th step, which is defined as 
\begin{equation}
    P_m(\vec{r},\vec{p},\vec{k},t_i) = \sigma_{\rm abs}(\vec{p},\vec{k})\left(1-\vec{\beta}\cdot\vec{k}/|\vec{k}|\right) n_p(\vec{r},\vec{k},t_i) c\Delta t_i\,.
    \label{eq:ScaProb}
\end{equation}
where $\vec{k}$ is the light wave vector, $\vec{p}$ and $\vec{\beta}$ are the PSI momentum and relativistic velocity, respectively, $n_p(\vec{r},\vec{k},t_i)$ is the local density of the photons, and $\sigma_{\rm abs}(\vec{p},\vec{k})$ is the absorption cross section given by \cite{Bessonov:1995eq}
\begin{equation}
    \sigma_{\rm abs}(\vec{p},\vec{k}) = \frac{\pi r_e c f\gamma_e}{[\gamma_L\omega^{\rm LF}(1-\beta\cos\psi) - \omega_0]^2 \,+ \,\gamma_e^2/4}\,,
    \label{eq:sigabs1}
\end{equation}
where $r_e$ is the classical electron radius, $f$ is the oscillator strength, $\omega^{\rm LF}$ is the irradiated light frequency in the LF, and $\psi$ is the angle between the directions of the light and PSI propagation in this frame.

The above can also be expressed in the following way
\begin{equation}
    \sigma_{\rm abs}(\vec{p},\vec{k}) = \frac{\sigma_0}{1 + 4\tau_e^2 \Delta\omega^2}\,,
    \label{eq:sigabs2}
\end{equation}
where 
\begin{equation}
    \sigma_0 = \frac{\lambda_0^2g_e}{2\pi g_g}\,,
\label{eq:sigma0}
\end{equation}
with $\tau_e=1/\gamma_e$ being the relaxation time of the excited state,  $g_g$ and $g_e$ being the degeneracy factors of the ground state $g$ and the excited state $e$, respectively, detuning $\Delta\omega = \omega - \omega_0$ measured in the IRF, where the IRF light frequency is given by $\omega=\gamma_L(1-\beta\cos\psi)\omega^{\rm LF}$, and $\lambda_0$ being the light central wavelength in the IRF.

For $P_m$ to act as probability, the size of the time step $\Delta t_i$ in Equation~\ref{eq:ScaProb} must be adjusted such that 
\begin{equation}
    0 \le P_m(\vec{r},\vec{p},\vec{k},t_i) \le 1, \; \forall_{m=1,\ldots M}\,.
\label{eq:ProbRange}
\end{equation}
In the simulations, the step size is set in such a way that if, for some macroparticle $m$ and time $t_i$, $P_m$ is larger than 1, the corresponding $\Delta t_i$ is divided into smaller steps until the condition given by Equation~\ref{eq:ProbRange} is fulfilled.

After computing the probability $P_m^i\equiv P_m(\vec{r},\vec{p},\vec{k},t_i)$, a scattering event is sampled using the (von Neumann) acceptance-rejection Monte Carlo method, see, for example, Ref.~\cite{Fishman:1996},
\begin{equation}
    \{0,1\} \ni \,  n_m^i = \int_0^1 dR\;\Theta(P_m^i - R)\,,
    \label{eq:RejMet}
\end{equation}
where $\Theta$ is the step function, i.e.\ a random number $R$ from the uniform distribution on $(0,1)$ is generated, and if $R\le P_m^i$, the event is accepted, otherwise it is rejected.  If the event is accepted, the corresponding macroparticle is marked as {\em excited}, which corresponds to the excitation of the PSI.  The macroparticle ``lives'' in the excited state for a time $\tau$ which is generated from the exponential distribution
\begin{equation}
    \zeta(\tau) = \frac{1}{\tau_e}\, e^{-\tau/\tau_e},\quad \tau \ge 0\,.
    \label{eq:ExpDis}
\end{equation}
While in the excited state, the macroparticle can interact with a laser photon and be deexcited by stimulated emission with the probability
\begin{equation}
    S_m(\vec{r},\vec{p},\vec{k},t_i) = \frac{g_g}{g_e}\,P_m(\vec{r},\vec{p},\vec{k},t_i)\,.
    \label{eq:StimEmProb}
\end{equation}
The stimulated emission event is generated, similarly as above, with the acceptance-rejection Monte Carlo method
\begin{equation}
    \{0,1\} \ni \, k_m^i = \int_0^1 dR\;\Theta(S_m^i - R)\,,
\label{eq:SERM}
\end{equation}
where $S_m^i\equiv S_m(\vec{r},\vec{p},\vec{k},t_i)$.  The corresponding emitted photon is not stored in the event record {as it goes along the laser pulse, and therefore does not reach a detector}.  If the excited ion is not deexcited by the stimulated emission within its lifetime $\tau$, it undergoes the spontaneous emission.  In such a case, the frequency $\omega_1$ as well as the polar $\theta_1$ and azimuthal $\phi_1$ angles of the emitted photon are generated in the ion reference frame (IRF), and then they are Lorentz-transformed to the LF.  The photon frequency is generated from the Lorentzian distribution, as given in Equation~\ref{eq:sigabs2}, while the emission angles are generated according to the angular distribution of fluorescence corresponding to a given atomic transition.  For such a photon, its LF four-momentum and space-time coordinates of the spontaneous emission are stored in the event record.

After the stimulated or spontaneous emission, the ion returns to its atomic ground state and is ready for absorption of another photon.  The whole above procedure is repeated for each macroparticle $m = 1,\ldots, M$ at a given time $t_i$ and is done for all time steps $\Delta t_i, i = 1,\ldots,I$.  The number of the spontaneously emitted photons from the PSI bunch is
\begin{equation}
    N_{\gamma} = \sum_{i=1}^I \sum_{m=1}^M \left(n_m^i - k_m^i\right) \frac{N_b}{M}\,,
    \label{eq:Ngam}
\end{equation}
where $N_b$ is the number of the PSI in the bunch, and $N_b/M$ is the Monte Carlo weight assigned to each macroparticle in the event record.

In a real experiment, the ions in the bunch and the photons in the laser pulse are not monoenergetic, but have some finite energy spread.  In the original {\sf CAIN} program the relative energy spread of particles in a bunch can be set in the input parameters and then the individual particle energy, i.e.\ the PSI Lorentz factor $\gamma_L$ in Equation~\ref{eq:sigabs1}, is generated from an appropriate Gaussian distribution.  On the other hand, the laser pulse  is assumed to be monochromatic.  For the inverse Compton scattering this is not important because the cross section does not depend strongly on the photon energy, but for the resonant absorption the finite energy spread of the laser pulse must be taken into account, particularly for the small linewidth $\gamma_e$ (cf.\ Equation~\ref{eq:sigabs1}).  It can be generated in the LF from the Gaussian distribution 
\begin{equation}
    \mathcal{D}(\omega^{\rm LF}) = \frac{1}{\sqrt{2\pi}\,\sigma_{\omega^{\rm LF}}}\,
    \exp\left[-\frac{(\omega^{\rm LF}-\omega_0^{\rm LF})^2}{2\sigma_{\omega^{\rm LF}}^2}\right]
    \label{eq:LasEspread}
\end{equation}
for a given relative frequency spread rms of a laser pulse $\sigma_{\omega^{\rm LF}}/\omega^{\rm LF}_0$, where $\omega^{\rm LF}_0$ is the central value of the laser-photon pulse frequency, adjusted to the central value of the absorption resonance for the central value of the PSI-bunch energy spread.
 
Alternatively, the photon frequency $\omega$ in IRF can be generated from the Lorentzian distribution of Equation~\ref{eq:sigabs2}, and then the scattering probability can be calculated by replacing the absorption cross section $\sigma_{\rm abs}(\vec{p},\vec{k})$ in Equation~\ref{eq:ScaProb} with the ``spread" cross section
\begin{equation}
    \sigma_{\rm spr}(\vec{p},\vec{k}) = \sigma_0\, \frac{\sqrt{\pi}\,\gamma_e}{2\sqrt{2}\,\sigma_{\omega}}\,
    \exp\left[-\frac{(\omega - \omega_0)^2}{2\sigma_{\omega}^2}\right]\,,
    \label{eq:GaussXsec}
\end{equation}
where $\sigma_{\omega}/\omega_0 = \sigma_{\omega^{\rm LF}}/\omega^{\rm LF}_0$ and $\sigma_0$ is given in Equation~\ref{eq:sigma0}.  In {\sf GF-GAIN} this method is used when $\gamma_e < 2\sqrt{2\ln 2}\,\sigma_{\omega}$,
which improves efficiency of event generation in such cases.  

{\sf GF-CAIN} has been cross-checked with the independent Monte Carlo generators {\sf GF-CMCC} and {\sf GF-Python}, and a good agreement with these programs has been found \cite{Curatolo:2018pza,GF-PoP-LoI:2019}. 

As discussed in the previous section, the above description can be applied reliably only for the H-like lead case of the GF presented in the third column of Table~\ref{table:OpticalPumpingParameters}.  Below we show some numerical results of the Monte Carlo simulations performed with {\sf GF-CAIN} for the input parameters given in the third column of Table~\ref{table:OpticalPumpingParameters}, with the supplementary parameters collected in Table~\ref{table:SimPars}.  The number of the spontaneously emitted photons from the PSI bunch is provided in the last row of Table~\ref{table:SimPars} -- it corresponds to the emission rate of $\sim 2.5$ of photons per ion.  It agrees within a factor of $1.65$ with the result of the semi-analytical calculations presented in the fourth column of Table~\ref{table:OpticalPumpingParameters}.  We have also checked that the emission rate grows linearly with the pulse energy between $0.05$ and $5\,$mJ.  
\begin{table*}[!htbp]
    \centering
    \caption{Some parameters for the {\sf GF-CAIN} simulations and the number of spontaneously emitted photons.}
	\begin{tabular}{lr}\hline\hline
		PSI beam  & ${}^{208}_{~82}\mathrm{Pb}^{81+}$\\
		\hline
		PSI mass $m$ & $193.687\,$GeV/c$^2$ \\
		PSI mean energy $\mathcal{E}$ & $578.9$ TeV \\
        Beta function at the interaction point $\beta_x=\beta_y$ & $0.5\,\mathrm{m}$\\
        Geometric emittance $\epsilon_x=\epsilon_y$ & $3\times 10^{-9}\;\mathrm{m\times rad}$\\
		\hline
		Laser (LF) & FEL (Gaussian)\\
		\hline
		Central wavelength of the laser in the LF $\lambda_0^{\rm LF}$ & $98.46\,$nm  \\
	    Rayleigh length $R_{L,x} = R_{L,y}$ & $7.5\,\mathrm{cm}$\\
		Interaction angle $\psi$ & 0$^\circ$\\
		\hline
 		 Atomic transition  & $1s \rightarrow 2p_{1/2}$\\
		\hline
		On-resonance absorption cross section $\sigma_0$ & $431.7\,$kb\\
		Angular distribution of emitted photons in the IRF $d^2p_1/(d\cos\theta_1 d\phi_1)$ & $1/(4\pi)$\\
		Maximum emitted photon energy in the LF $\hbar\omega_{\gamma}^{\rm max}$ & $450\,$MeV\\
		\hline
		Number of emitted photons per bunch $N_{\gamma}$ & $2.3\times 10^8$\\
		\hline\hline
	\end{tabular}	
\label{table:SimPars}	
\end{table*}

\begin{figure*}[!htbp]
    \centering
    \includegraphics[width=0.49\linewidth]{Figures/Eng-5mJ.png}
    \includegraphics[width=0.49\linewidth]{Figures/Thg-5mJ.png}
    \caption{Distributions of energy (left) and polar angle (right) of the spontaneously emitted photons 
             from the H-like Pb bunch in the GF (in the LF).}
    \label{fig:EnergyAngle}
\end{figure*}

The LF energy and emission-angle distributions of the outgoing photons are presented in Figures~\ref{fig:EnergyAngle} and~\ref{fig:EngThg}.  As discussed above, in the IRF the spontaneous emission is isotropic but its transformation to the LF modifies the emission pattern so that the emitted photons are strongly collimated.  As shown in Fig.~\ref{fig:EnergyAngle} (right panel), for a specific case of the system characterized with the parameters given in Table~\ref{table:SimPars}, most of the photons are emitted within an angle of $1\,$mrad, with the maximum at $\sim 0.25\,$mrad.  At the same time, the number of emitted photons versus the energy is characterized with the uniform distribution [Fig.~\ref{fig:EnergyAngle} (left panel)], which is a result of the isotropic emission in the IRF [$\phi_1\in {\cal U}(0,2\pi)$, $\cos\theta_1 \in {\cal U}(-1,1)$] and the Lorentz boost to the LF: $\omega^{\rm LF} = \gamma_L(1+\beta\cos\theta_1)\omega \Rightarrow dN_{\gamma}/d\omega^{\rm LF} \propto dN_{\gamma}/d\cos\theta_1$.  The difference in the distributions stems from nonlinear dependence between energy and angle, i.e.\ for higher energies, the same energy range corresponds to a smaller angular range than for lower energies, as can be seen in Figure~\ref{fig:EngThg} (left panel).
\begin{figure*}[!htbp]
    \centering
    \includegraphics[width=0.49\linewidth]{Figures/EngThg-5mj.png}
    \includegraphics[width=0.49\linewidth]{Figures/Eng3Thg-5mJ.png}
    \caption{Distributions of energy versus polar angle (left) and energy for three angular upper cut-offs
             of the spontaneously emitted photons from the H-like Pb bunch in the GF (in the LF)}.
    \label{fig:EngThg}
\end{figure*}

The left panel of Figure~\ref{fig:EngThg} shows strong correlations between the energy and angle of the emitted photons. Specifically, the most energetic photons are emitted at smallest angles.  In the right panel of Figure~\ref{fig:EngThg}, we present the photon energy distributions for three angular upper cut-offs: $0.25, 0.5$, and $1\,$mrad.  It shows that the emitted photon energy may be selected by applying angular collimators,
which can be in form of simple circular apertures placed some distance away from the interaction point, such that the sizes of the PSI-beam bunch and the laser pulse can be neglected.  