\documentclass[superscriptaddress,
amsmath,amssymb%,
pra,twocolumn,
floatfix
]{revtex4-2}

\usepackage{graphicx}% Include figure files
\usepackage{xcolor}
\usepackage{threeparttable}
\usepackage[normalem]{ulem}
\usepackage{tabularx}

\bibliographystyle{apsrev4-1}


\begin{document}

\title{Optical excitation of ultra-relativistic partially stripped ions}

\author{Jacek Biero\'n}
\affiliation{Institute of Theoretical Physics, Jagiellonian University, ul. {\L}ojasiewicza 11, 30-348 Krak\'ow, Poland}
\author{Mieczyslaw Witold Krasny}
\affiliation{LPNHE, Sorbonne University, CNRS/IN2P3, Tour 33, RdC, 4, pl.\ Jussieu, 75005 Paris, France}
\affiliation{CERN, Esplanade des Particules 1, 1211 Geneva 23, Switzerland}
\author{Wies{\l}aw P{\l}aczek}
\affiliation{Institute of Applied Computer Science, Jagiellonian University, ul.\ {\L}ojasiewicza 11, \\ ~~30-348 Krak\'ow, Poland}
\author{Szymon Pustelny}
\email{Email: pustelny@uj.edu.pl}
\affiliation{M. Smoluchowski Institute of Physics, Jagiellonian University, ul. {\L}ojasiewicza 11, 30-348 Krak\'ow, Poland}

\begin{abstract}

The Gamma Factory (GF) initiative aims at the construction of a unique experimental tool exploiting resonant interaction of light with ultra-relativistic partially stripped ions (PSI) stored in circular accelerators at CERN. Resonant excitation of high-energy electronic transitions in the ions is achieved through Doppler-boosting (by twice the Lorentz factor; from hundred to several thousand times) of light energy. In order to efficiently excite the ions, and hence generate intense beams of scattered/fluorescent photons, a detailed knowledge of the ions' electronic energy structure and the dynamics of optical excitation is required. Spectroscopic properties of PSI selected for the GF operation, as well as their optical excitation schemes, are investigated.  Two regimes of the ion--light interaction are identified, leading to different dynamics of the excitation.  The efficiency of the ion--light interaction, as well as the number of photons emitted from a single ion bunch, are estimated, both analytically and numerically, for three ions considered for the GF, i.e.~Li-like ${}^{208}_{\phantom{0}82}$Pb$^{79+}$, Li-like ${}^{40}_{20}$Ca$^{17+}$, and H-like ${}^{208}_{\phantom{0}82}$Pb$^{81+}$.

\end{abstract}

\keywords{Gamma Factory, partially stripped ions, ultra-relativistic ions, optical excitation, resonant absorption, Rabi oscillations, electronic transitions in partially stripped ions}

\maketitle

\section{Introduction}
\label{sec:Intro}

The goal of the Gamma Factory (GF) project \cite{Krasny:2015ffb} is to develop new tools for the CERN-based research programme.  They include  \cite{GF-PoP-LoI:2019,Placzek:2019xpw,GammaFactoryWorkingGroup:2020ely,Budker:2020zer,Krasny:2021llv,Budker:2021fts,Nichita:2021iwa}: (1) atomic traps of highly charged atoms; (2) an electron beam for electron--proton collisions in the LHC interaction points; (3) high-intensity photon beams; (4) laser-light-based cooling methods of high-energy hadronic beams; and (5) high-intensity beams of polarized electrons, polarized positrons, polarized muons,  neutrinos, neutrons, and radioactive ions.  It is the first accelerator-technology-based project for which precise and accurate atomic-physics input is indispensible.  Firstly, to maximize the intensity of the GF beams, a very efficient optical excitation of the highly-ionised atoms, stored in the CERN accelerator complex, is of pivotal importance.  Secondly, this project requires high-precision calculations of the atomic energy levels of hydrogen, helium and lithium-like partially stripped ions (PSI) to precisely tune the ion beam energy to resonantly excite specific electronic transitions with Doppler-boosted laser light.  Since linewidths of the atomic resonant excitations are much narrower than frequencies of the transitions, the ``resonance-finding'' procedure, involving tuning of the ion relativistic Lorentz factor $\gamma_{L}$,  will certainly be one of the most difficult operation aspects of the project.  Thirdly, the lifetimes of the excited states must be accurately calculated in order to optimize and control the direction, polarization and energy of fluorescence photons. 
On top of precise knowledge of the static parameters of the PSI, which will be used in the GF research programme, good understanding of the ion--light interaction dynamics is necessary to optimize the parameters of laser pulses as well as to determine spatiotemporal characteristics of ion bunches.

One of the pivotal milestones of the GF project is the development of a theoretical framework for the interaction of light pulses with ion bunches.  In general, the interaction has to be described as a quantum-mechanical process which includes interference of probability amplitudes.  In this scope, its probabilistic description, which does not take into account the quantum superposition of atomic states, is approximate and can only be used in limited context.  At the same time, implementation of the process in terms of probabilistic (classical) observables is tempting as it enables application of Monte-Carlo generators. The Monte-Carlo framework provides an efficient interface between the theoretical framework and existing software tools, used to describe both the individual particles and collective beam dynamics in high-energy storage rings. 

The basic goal of the present paper, being a crucial step for the development of the GF, is to provide a clear assessment under which conditions the probabilistic framework is sufficient and under which conditions it fails.  These initial conditions will be specified by the following parameters: (1) the laser-pulse temporal and spatial shape, e.g.\ its time-dependent energy spectrum and total energy, and (2) the ion-bunch shape and the geometry of interactions. 

The paper is organized as follows. In Section~\ref{section.IonChoice} we investigate: the properties of the Li-like Pb ion which may be suitable for the proof-of-principle experiment (PoP) in the Super Proton Synchrotron (SPS) at CERN \cite{GF-PoP-LoI:2019}; the H-like Pb ion which is considered as a candidate for the GF experiment at the LHC \cite{Bessonov:1995eq}; and Li-like Ca ions which can be used for a high-luminosity option of the LHC with laser-cooled isoscalar ion beams \cite{Krasny:2020wgx,Krasny:2021llv}. We have also included in this section the He-like Ca ion case because of its potential future use for the production of radioactive ion beams \cite{Nichita:2021iwa}.  For the PoP experiment at the SPS, we need an ion which satisfies the following criteria: (1) transition energy of $10\,$--$\,100\,$eV and (2) lifetime of the excited state of $10\,$--$\,100\,$ps.
We have verified that these conditions are fulfilled, both with respect to the energy and the lifetime, for H-like, He-like, Li-like, Ne-like, Mg-like, Ar-like and Kr-like isoelectronic sequences.  We have not investigated Xe-like or Rn-like, nor any other sequence, but it can be done if necessary.  In Section~\ref{sec:OptPump}, we discuss laser-light excitation of PSI in the GF experiments. The calculations are performed for two distinct scenarios: (1) where the Doppler broadening dominates over the natural linewidth of the transition and (2) where these two widths are comparable. These lead to two different dynamics of light absorption and reemission.
Finally, Monte-Carlo simulation of interactions of a PSI bunch with a laser-light pulse is described in Section~\ref{sec:MCsim}.  The conclusions are presented in Section~\ref{sec:Concl}.

\input li-like-Pb-revtex.tex % Ion choice
\input OpticalPumping-revtex.tex % Description of optical pumping of ions
\input Simulations-revtex.tex

\section{Conclusions}
\label{sec:Concl}

We have evaluated and summarized the available literature data on spectroscopic properties of partially stripped ions (Li-like Pb, H-like Pb, He-like Ca and Li-like Ca) considered for the Gamma Factory project. We have demonstrated that lifetimes of relevant excited states of these ions are accurate to a percent level or better, while the corresponding transition energies reach between four and six digit accuracy. These numbers and their accuracies propagate into various parameters that will be necessary in experimental investigations.

Next, we have investigated ion--light interactions in the context of optical excitation, identifying two regimes determined by spectral properties of the ions in the bunch, i.e., the transition linewidth and the Doppler broadening, and the light pulse. We have shown that in the case of the Doppler broadening significantly exceeding the natural linewidth of the transition, efficient excitation needs to be based on Rabi oscillations and a good choice of the pulse total energy and length. We have theoretically motivated that under optimized (not experimental) conditions as much as $70\%$ of the atoms can be excited. We have also shown that in the case of the extremely short lifetimes of the excited state, the atom can undergo multiple excitation--deexcitation cycles, significantly increasing the number of photons emitted from the bunch. By investigating three ions of interest (Li-like ${}^{208}_{\phantom{0}82}$Pb$^{79+}$, Li-like ${}^{40}_{20}$Ca$^{17+}$, and H-like ${}^{208}_{\phantom{0}82}$Pb$^{81+}$), we calculated the efficiency of the excitation for a realistic set of parameters and hence estimated the number of photons scattered by the PSI bunches, but also identified the parameters optimizing the optical excitation and maximizing the number of scattered photons.

Finally, we have described Monte-Carlo simulations considering the process of the optical excitation of the H-like ${}^{208}_{\phantom{0}82}$Pb$^{81+}$ ion in a more realistic scenario.  For the simulations, we have developed the computer code {\sf GF-CAIN} which is currently limited to the steady-state regime, where the classical cross-section formulation can be applied.  Some exemplary results for the Gamma Factory realization at the LHC with the H-like Pb-ion beam have been presented.  The number of emitted photons agrees within a factor of about $1.5$ with semi-analytical calculations performed using the density-matrix formalism.  In our opinion, this agreement is satisfactory given the approximations employed in the latter calculations. 

The presented results are good starting point for experimental activities associated with the laser-light excitation of partially stripped ions in the Gamma Factory.  On the one hand, optical excitations of highly-charged heavy ions will allow to test theoretical calculations, providing access to such fundamental investigations as tests of quantum electrodynamics or violations of discrete symmetries~\cite{Budker2020Atomic}, but also offering means of spin polarization of PSI and studies of collisions of such ions.  On the other hand, the emission of secondary photons from the ions also offers schemes for generating extremely energetic (up to hundreds of MeV) and highly luminous ``light'' beams.  Due to their unique properties, such beams when extracted from production zones and collided with external targets can be used to produce high-intensity polarized electron, positron and muon beams, high-purity neutrino beams as well as high-flux neutron and radioactive-ion beams \cite{GF-PoP-LoI:2019,Placzek:2019xpw,GammaFactoryWorkingGroup:2020ely}.

\medskip
\textbf{Acknowledgements} \par 
SP acknowledges invaluable help and stimulating discussions with Dmitry Budker, Krzysztof Dzier\.z\c{e}ga, Alexey Petrenko, and Simon Rochester.  JB would like to thank Alexander Kramida and Andrey Surzhykov for extensive help in researching and computing spectroscopic data necessary for the Gamma Factory project.  WP acknowledges the fruitful collaboration with Camilla Curatolo.

\medskip

\bibliographystyle{MSP}
\bibliography{GammaFactory}

\end{document}
