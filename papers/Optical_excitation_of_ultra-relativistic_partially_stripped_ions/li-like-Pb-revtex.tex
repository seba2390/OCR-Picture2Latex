\newcommand{\thicksim}{\,\mathbf{\sim}\,}
\section{Energies and lifetimes of atomic excited levels}
\label{section.IonChoice}

Partially stripped ions which are presently considered as the candidates for the GF are:  Li-like Pb, H-like Pb, He-like Ca, and Li-like Ca.  The energies $ E_{ik} $ and lifetimes $ \tau_{ik} $ for selected transitions in these ions (as well as other candidates mentioned above) were computed with the Dirac--Hartree--Fock package {\sf GRASP}~\cite{compas.github,grasp2K:2013,GRASP2018,GrantBook2007,Grant1994}.  These calculations were made in the Dirac--Hartree--Fock model, i.e.~with the Dirac--Coulomb Hamiltonian, with a finite nuclear size modeled as the two-parameter Fermi distribution~\cite{grasp89} and with the leading QED corrections (SE and VP) evaluated perturbatively~\cite{graspMcKenzie1980}.  The electron correlation effects were accounted for through the multiconfiguration variational Complete Active Space approach~\cite{BieronAu2009,Bieron:e-N:2015}.  The calculated values were compared with other data available in the literature, and in each case the most accurate results were selected in Tables~\ref{table.Li-like-Pb}, \ref{table.H-like-Pb}, \ref{table.He-like-Ca}, \ref{table.Li-like-Ca}, and~\ref{table:OpticalPumpingParameters}.

\subsection{Li-like Pb}
\label{section.Li-like-Pb}
The NIST-ASD database~\cite{NIST-ASD} lists 46 references on the subject of Pb$^{79+}$.  Transition energies and lifetimes for the Li-like~Pb ion are collected in Table~\ref{table.Li-like-Pb}.  For each calculation we quote as many digits as provided by the authors.  Only the three latest calculations include the estimates of the uncertainty.  Finally, in Table~\ref{table:OpticalPumpingParameters} we adopted the 2s--2p$_{1/2}$ transition energy
$ 230.823(47)(4) $~eV in Li-like~Pb from Yerokhin and Surzhykov~\cite{YerokhinSurzhykov:2018a,YerokhinSurzhykov:2018b}, and the lifetime for $ 2p_{1/2} $ state of Li-like Pb from Johnson, Liu, and Sapirstein~\cite{JohnsonLiuSapirstein:1996}.  They estimated that their calculated lifetimes are accurate to a fraction of a percent at the neutral end of the isoelectronic sequence, and the accuracy increases at higher $Z$.

\begin{table*}[htbp!]\centering
    \caption{\normalsize{Transition energies $E$ and lifetimes $\tau$ of the 2s--2p$_{1/2}$ and 2s--2p$_{3/2}$ lines in the Li-like Pb ion.}}
    \vspace{0.1cm}
    \begin{tabular}{llllclc}
        \hline\hline
        \multicolumn{2}{c}{2s--2p$_{1/2}$} & 
        \multicolumn{2}{c}{2s--2p$_{3/2}$} \\
        \hline
        $E$ [eV]       &$\tau$ [ps] &  $E$ [eV]       & $\tau$ [fs] & year & method & \multicolumn{1}{c}{reference} \\
        \hline 
        231.374        && 2642.297        && 1990 & MCDF VP SE & \cite{IndelicatoDesclaux} \\
        230.817        && 2641.980        && 1991 & MCDF VP SE & \cite{Kim} \\
        230.16 & 73.96  & 2649.23 & 41     & 1991 & Coul-App HS-core & \cite{TheodosiouCurtisEl-Mekki:1991} \\ 
        230.698        && 2641.989        && 1995 & RCI QED NucPol & \cite{ChenChengJohnsonSapirstein} \\
        231.16 & 76.6   & 2642.39 & 42.22  & 1996 & 3-rd order MBPT & \cite{JohnsonLiuSapirstein:1996} \\
        230.650(30)(22)(29) &&  ---    && 2003 & expt(DR) & \cite{Brandau:2003} \\
        ---           && 2642.26(10)     && 2008 & expt(EBIT) &  \cite{Zhang} \\
        230.68         &&  ---            && 2010 & RCI QED NucPol & \cite{Kozhedub} \\ 
        230.76(4)      && 2642.17(4)      && 2011 & S-m. 2-l. NucPol & \cite{SapirsteinCheng} \\
        230.823(47)(4) && 2642.220(46)(4) && 2018 & RCI QED NucPol & \cite{YerokhinSurzhykov:2018a,YerokhinSurzhykov:2018b} \\
        230.80(5)      && 2642.20(5)      && 2019 & S-m. 2-l. NucPol & \cite{pcSapirsteinCheng} \\
        232 & 76   & 2643 & 42 & 2021 & RCI VP SE & this work \\       
        \hline \hline
        \vspace{0.1cm}
        \end{tabular}
    \label{table.Li-like-Pb}
\end{table*}

\subsection{H-like Pb}
\label{section.H-like-Pb}

\begin{table}[htbp!]\centering
    \caption{\normalsize{Transition energies $E$ and lifetimes $\tau$ of the 1s--2p$_{1/2}$ and 1s--2p$_{3/2}$ transitions in the H-like Pb ion.}}
    \begin{tabular}{lc|lc|ccc}
        \hline\hline
        \multicolumn{2}{c}{1s--2p$_{1/2}$} & 
        \multicolumn{2}{c}{1s--2p$_{3/2}$}  \\
        \hline
        $E$ [eV]       &$\tau$ [as] & $E$ [eV] & $\tau$ [as] & year & reference \\
        \hline 
        75280.47     & ---  &  77934.25     & ---  & 1985 &  \cite{JohnsonSoff:1985} \\
        75279        & ---  &  ---          & ---  & 1997 &  \cite{Beier:1997} \\
        75521        & 33.8 &  78174        & 38.8 & 2003 &  \cite{JitrikBunge:2003} \\
        75280.83(26) & ---  &  77934.59(27) & ---  & 2015 &  \cite{YerokhinShabaev:2015} \\
        75278        & 34.1 &  77935        & 39.2 & 2021 & this work \\
        \hline\hline
    \end{tabular}
    \label{table.H-like-Pb}
\end{table}

The bibliography of spectroscopic properties of hydrogen-like ions lists more then one hundred papers~\cite{NIST-ASD,YerokhinShabaev:2015}.  In the present work (see Table~\ref{table.H-like-Pb}) we have taken into consideration the papers of Johnson and Soff~\cite{JohnsonSoff:1985}, Beier~\textit{et~al.}~\cite{Beier:1997}, Jitrik and Bunge~\cite{JitrikBunge:2003}, and Yerokhin and Shabaev~\cite{YerokhinShabaev:2015}.  Johnson and Soff~\cite{JohnsonSoff:1985} took into account the QED effects (Lamb shift), the effects of the finite nuclear size, reduced mass, nuclear recoil effects, as well as their respective cross-terms.  Beier~\textit{et~al.}~\cite{Beier:1997} took into account the QED effects, the effect of the finite nuclear size, as well as the nuclear recoil effect.  Jitrik and Bunge~\cite{JitrikBunge:2003} evaluated the transition energies and rates for hydrogen-like ions using eigenfunctions of the Dirac Hamiltonian with a point nucleus.  The discrepancy between the values of Jitrik and Bunge~\cite{JitrikBunge:2003} and the values obtained with more elaborate approaches of Johnson and Soff~\cite{JohnsonSoff:1985}, Beier~\textit{et~al.}~\cite{Beier:1997}, and Yerokhin and Shabaev~\cite{YerokhinShabaev:2015} illustrates the contributions of the effects beyond the
the Dirac Hamiltonian with a point nucleus.  Yerokhin and Shabaev~\cite{YerokhinShabaev:2015} took into account the QED effects, the finite nuclear size, the nuclear recoil as well as their respective cross-terms.  In particular, they thoroughly evaluated the two-loop QED correction and the finite nuclear size correction, which constitute the dominant sources of uncertainty for the H-like Pb ion.  All these authors adopted different values of the fine structure constant $\alpha$ which were considered standard at the respective publication dates.  These differences, ranging between seventh up to tenth figure (the current value of \mbox{$\alpha^{-1}$~=~137.035~999~084(21)}~\cite{NIST-Constants}) contributed to several factors involved in the summation of the transition energy and rate.  In Table~\ref{table:OpticalPumpingParameters} we adopted the energy for the transition \mbox{$E(1\mathrm{s}$--$2\mathrm{p}_{1/2})$~=~$75280.83(26)$~eV} in the H-like~Pb ion from Yerokhin and Shabaev~\cite{YerokhinShabaev:2015} and the lifetime (34 attoseconds) calculated in the present work with the Dirac--Coulomb Hamiltonian, with the finite nuclear size and with the leading QED corrections evaluated perturbatively~\cite{graspMcKenzie1980}.
 
\subsection{He-like Ca}
\label{section.He-like-Ca}
The calculations of the transition energies and rates for the He-like Ca ion were performed with the {\sf GRASP} package described at the beginning of Section~\ref{section.IonChoice}.  The results are presented in Table~\ref{table.He-like-Ca} and compared with data available in the literature.  The transition energy calculated by Artemyev~\textit{et~al.}~\cite{Artemyev:2005} is the most reliable among those presented in 
Table~\ref{table.He-like-Ca}.  For the lifetime, one might cautiously adopt $ 6.0(1)$\,ps, i.e.\ an average of the three values calculated by Lin~\textit{et~al.}~\cite{LinJohnsonDalgarno:1977}, Aggarwal~\textit{et~al.}~\cite{Aggarwal:2012} and in the present work, respectively.

\begin{table}[htbp!]\centering
    \caption{\normalsize{Transition energy $E$ and lifetime $\tau$ of the 1s$^2$--1s2p~$^1 \! P_1$ transitions in the He-like Ca ion.}}
    \begin{tabular}{llcc}
    \hline\hline
    \multicolumn{1}{c}{$E$ [eV]\phantom{ene}} & 
    \multicolumn{1}{c}{$\tau$ [fs]\phantom{en}} & year & reference \\
    \hline 
    \phantom{390} 
    ---       & 6.06  & 1977 & \cite{LinJohnsonDalgarno:1977} \\
    3902.3676    & ---   & 1988 & \cite{Drake:1988} \\
    3902.3775(4) & ---   & 2005 & \cite{Artemyev:2005} \\
    3902.2570    & 5.946 & 2012 & \cite{Aggarwal:2012} \\
    3902.2551    & ---   & 2021 & \cite{NIST-ASD} \\
    3902.3351    & 6.09  & 2021 & this work \\
    \hline\hline
\end{tabular}
\label{table.He-like-Ca}
\end{table}

\subsection{Li-like Ca}
\label{section.Li-like-Ca}

Similarly as in the case of He-like Ca, for Li-like Ca ion the transition energies and lifetimes were calculated using {\sf GRASP} package.  The results for four different excitations from the ground {1s$^2$2s} state are presented in Table~\ref{table.Li-like-Ca} and compared with data available in the literature.

\begin{table*}[!htbp]\centering
    \caption{\normalsize{Transition energies $E$ and lifetimes $\tau$ of the 2s--2p$_{1/2}$, 2s--2p$_{3/2}$, 2s--3p$_{1/2}$, 2s--3p$_{3/2}$ transitions in the Li-like Ca ion.}}
    {\small 
    \begin{tabular}{ll|ll|ll|ll|clc}
        \hline
        \hline 
        \multicolumn{2}{c|}{2s--2p$_{1/2}$} & 
        \multicolumn{2}{c|}{2s--2p$_{3/2}$} & 
        \multicolumn{2}{c|}{2s--3p$_{1/2}$} & 
        \multicolumn{2}{c}{2s--3p$_{3/2}$} \\
        \hline
        \phantom{3}$E$~[eV] & $\tau$~[ns] & 
        \phantom{3}$E$~[eV] & $\tau$~[ns] & 
        \phantom{3}$E$~[eV] & $\tau$~[ps] & 
        \phantom{3}$E$~[eV] & $\tau$~[ps] & year & method & ref. \\
        \hline
        &  0.76752 &        & 0.51258  &       &       &       &       & 1991 & Coul-App HS-core & \cite{TheodosiouCurtisEl-Mekki:1991} \\
        \hline
        35.963  & 0.7680    &  41.029 & 0.5123   &      &          &       &            & 1996 & 3-rd order MBPT & {\cite{JohnsonLiuSapirstein:1996}} \\
        \hline
        &          &       &      &  663    &  0.4167  &  663      & 0.4274 & 2002 & R-matrix Breit-Pauli & \cite{Nahar:2002} \\ 
        \hline
        35.962{(1)}  &           & 41.024{(1)} &        &          &          &       &            & 2011 & S-matrix Kohn-Sham &  {\cite{SapirsteinCheng:2011}} \\
        \hline
        &          &       &      &  661.7643 & 0.4282 &  663.2660 & 0.4367  & 2014 & MCDF VP SE & \cite{DengJiangZhang:2014} \\
        \hline
        35.96119{(7\rlap{3)}} &       & 41.02497{(7\rlap{8)}} &     &       &          &       &            & 2018 & RCI QED NucPol & {\cite{YerokhinSurzhykov:2018b}} \\ %a
        \hline
        \multicolumn{1}{l}{35.9625 } & & 
        \multicolumn{1}{l}{41.0286 } & & 
        \multicolumn{1}{l}{661.8896 } & & 
        \multicolumn{1}{l}{663.3403 }  & & 2021 & NIST ASD & \multicolumn{1}{c}{\cite{NIST-ASD}} \\
        \hline
        35.959  & 0.767   &   41.027  & 0.512  &  661.776 & 0.428   &  663.278  & 0.436   & 2021 &  RCI VP SE & this work \\ 
        \hline
        \multicolumn{10}{c}{experiment} \\ 
        \hline
        35.962{(2)}  &          &        &         &       &       &       &       & 1985 &  & \cite{SugarCorliss:1985} \\
        \hline
        &      & 41.029{(2)} &         &       &       &       &       & 1983 &  & \cite{Edlen:1983} \\
        \hline
        \hline
        \vspace{0.1cm}
    \end{tabular}
    }
    \label{table.Li-like-Ca}
\end{table*}

\subsection{Bright future of partially stripped ions in the Gamma Factory}
\label{section.futurePSI}

The Li-like Pb ions will be accelerated and irradiated in the GF PoP experiment at the SPS \cite{GF-PoP-LoI:2019}.  During the experiment, various aspects of the project, including the efficiency of the PSI excitation, will be studied.  This will be the next step in the project which may open means for the GF implementation at the Large Hadron Collider (LHC).  For that experiment, the H-like Pb ion with much larger transition energy is considered \cite{Bessonov:1995eq}.  The transition becomes accessible for existing light sources, such as the Free Electron Laser (FEL), in conjunction with a higher value of the Lorentz factor $\gamma_L$ of the PSI bunches.
The Li-like Ca ions are suggested for the high-luminosity version of the LHC with laser-cooled isoscalar ion beams \cite{Krasny:2020wgx}, while the He-like Ca ions can be used for the potential future production of the radioactive-ion beams at the GF \cite{Nichita:2021iwa}.  In the following section we analyze the interaction of the laser light with the PSI bunch circulating in these two accelerators, investigating different scenarios of the process.