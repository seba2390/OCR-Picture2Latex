% Definitionen f"ur LaTeX2e KOMA Script article
% Verwendet AMS-Fonts: \usepackage{amssymb}
% Verwendet bbm-Font: \usepackage{bbm} 
% Version vom 15.12.2016
\newtheorem{theorem}{Theorem}[section]   
\newtheorem{proposition}[theorem]{Proposition}
\newtheorem{lemma}[theorem]{Lemma}        
\newtheorem{corollary}[theorem]{Corollary}
\newtheorem{definition}[theorem]{Definition}
\newtheorem{hypothesis}[theorem]{Hypothesis}
                                         
%Zahlenmengen unter Verwendung von bbm-Fonts
\newcommand{\N}{\mathbbm{N}}
\newcommand{\Z}{\mathbbm{Z}}
\newcommand{\Q}{\mathbbm{Q}}
\newcommand{\R}{\mathbbm{R}}
\newcommand{\C}{\mathbbm{C}}
\newcommand{\K}{\mathbbm{K}}

\newcommand{\id}{\mathbbm{1}}            % Identitaet

\newcommand{\hilbert}{{\mathcal H}}      % Hilbert-Raum

\DeclareMathOperator{\im}{Im}            % Imaginaerteil
\DeclareMathOperator{\re}{Re}            % Realteil
\DeclareMathOperator{\sign}{sign}        % Signum
\DeclareMathOperator{\supp}{supp}        % Support, Traeger

\DeclareMathOperator{\res}{res}          % Residuum
\DeclareMathOperator{\Sing}{Sing}        % Singularitaeten Menge

\DeclareMathOperator{\dom}{dom}          % domain
\DeclareMathOperator{\ran}{ran}          % Bildbereich, Range
\DeclareMathOperator{\mathspan}{span}    % Lineare Huelle
\DeclareMathOperator{\rank}{rank}        % Rang
\DeclareMathOperator{\diag}{diag}        % Diagonalmatrix
\DeclareMathOperator{\tridiag}{tridiag}  % Tri-Diagonalmatrix
\DeclareMathOperator{\tr}{tr}            % Trace, Spur

\DeclareMathOperator{\grad}{grad}        % Gradient
\DeclareMathOperator{\diverg}{div}       % Divergenz
\DeclareMathOperator{\rot}{rot}          % Rotation

\DeclareMathOperator{\arsinh}{arsinh}    % Area Funktionen
\DeclareMathOperator{\arcosh}{arcosh}
\DeclareMathOperator{\artanh}{artanh}
