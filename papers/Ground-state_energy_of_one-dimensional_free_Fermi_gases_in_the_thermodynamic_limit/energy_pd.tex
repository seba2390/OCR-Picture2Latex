There is a handy formula for $\mathcal{E}(\nu)$ based upon Riesz's integral formula
for spectral projections (or the Dunford integral) and the perturbation determinant.

The resolvents in \eqref{resolvents} are related by Kre\u\i{}n's resolvent formula for additive perturbations
\begin{equation}\label{krein_Omega}
  R_V(z) - R(z) = R(z) \sqrt{|V|} J\Omega(z)\sqrt{|V|} R(z),\ z\in\C\setminus(\sigma(H)\cup\sigma(H_V)) ,
\end{equation}
which holds true whenever the following inverse
\begin{equation}\label{Omega_K}
      \Omega(z) \coloneqq (\id-K(z))^{-1},\ K(z) \coloneqq \sqrt{|V|}R(z)\sqrt{|V|}J ,
\end{equation}
exists. Motivated by stationary scattering theory, see e.g. \cite[3.6.1]{Thirring2002}, we call $\Omega(z)$ the \emph{wave operator}.
The sandwiched resolvent $K(z)$ is called \emph{Birman--Schwinger operator}. We used Rollnik's factorization \eqref{V_polar_decomposition}
since in our applications the operator $R(z)V$ generally will not have the necessary trace class properties.
The wave operator is closely related to the spectrum of the perturbed operator, which is known as Birman--Schwinger principle.

\begin{lemma}\label{energy_omega}
Let $z\in\C\setminus\sigma(H)$. Then, $\Omega(z)$ exists and is bounded if and only if
$z\notin\sigma(H_V)$ in which case the map $z\mapsto\Omega(z)$ is holomorphic with respect 
to the operator norm and we have the formula
\begin{equation}\label{energy_omega01}
  \Omega(z) = \id + \sqrt{|V|} R_V(z) \sqrt{|V|}J .
\end{equation}
\end{lemma}
\begin{proof}
(i) 
Let $z\notin\sigma(H_V)$ which means that $R_V(z)$ exists as a bounded operator. Then,
\begin{equation*}
\begin{split}
\lefteqn{ (\id - \sqrt{|V|}R(z)\sqrt{|V|}J)(\id + \sqrt{|V|}R_V(z)\sqrt{|V|}J) }\\
   & = \id - \sqrt{|V|}R(z)\sqrt{|V|}J + \sqrt{|V|} ( \id - R(z)V ) R_V(z)\sqrt{|V|}J\\
   & = \id - \sqrt{|V|}R(z)\sqrt{|V|}J + \sqrt{|V|}R(z) (z\id - H - V ) R_V(z)\sqrt{|V|}\\
   & = \id .
\end{split}
\end{equation*}
If the factors are interchanged the calcutions are analogous. We conclude that $\Omega(z)$ exists and is bounded.
Formula \eqref{energy_omega01} is obvious.
 
It is well-known that the map $z\mapsto R(z)$ is holomorphic for $z\notin\sigma(H)$.
Since $V$ is bounded a Neumann series argument shows that the map $z\mapsto\Omega(z)$ is holomorphic 
on $\C\setminus( \sigma(H)\cup\sigma(H_V) )$.

(ii)
Let $z\in\sigma(H_V)\subset\R$ and $z_n\in\C\setminus\R$ with $z_n\to z$ for $n\to\infty$.
From (i) we know $\|\Omega(z_n)\|<\infty$.
If $\Omega(z)$ existed as a bounded operator it would follow from (i) that $\Omega(z_n)\to \Omega(z)$. In particular,
$\sup_n \|\Omega(z_n)\| < \infty$.
On the other hand, we have $\|R_V(z_n)\|<\infty$ and $\|R_V(z_n)\|\to\infty$ as $z_n\to z$, which yields a contradiction
via \eqref{energy_omega01}.
\end{proof}

The spectrum of the Birman--Schwinger operator can be described a tad more detailed.

\begin{lemma}\label{spectrum_K}
Let $z\in\C\setminus\R$ and $\varkappa\in\C\setminus\{0\}$ be an eigenvalue of $K(z)$. 
Then $\im(\varkappa)\neq 0$.
\end{lemma}
\begin{proof}
Since $\im(z)\neq 0$ the resolvent $R(z)$ and, hence, the Birman--Schwinger operator is bounded.
Let $\varkappa\in\C\setminus\{0\}$ and let $\varphi\in\hilbert$, $\varphi\neq 0$, such that
\begin{equation*}
  \varkappa\varphi = K(z)\varphi = \sqrt{|V|}R(z)\sqrt{|V|}J \varphi .
\end{equation*}
We multiply by $J$, take scalar products
\begin{equation*}
  \varkappa (\varphi,J\varphi) = (\varphi,J\sqrt{|V|}R(z)\sqrt{|V|}J\varphi) ,
\end{equation*}
and look at the imaginary part
\begin{equation*}
  \im(\varkappa) (\varphi,J\varphi) = -\im(z) \|R(z)\sqrt{|V|}J\varphi\|^2 .
\end{equation*}
The norm on the right-hand side does not vanish since that would imply
$\varkappa\varphi=0$ by the eigenvalue equation which contradicts $\varphi\neq 0$ and 
$\varkappa\neq 0$. Hence, $\im(z)\neq 0$ implies $\im(\varkappa)\neq 0$.
\end{proof}

We note the relevant trace class properties. Because detailed accounts are scattered
throughout the literature we sketch the proofs for the sake of completeness.

\begin{lemma}\label{energy00t}
Let $\sqrt{|V|}R(z_0)\in B_2(\hilbert)$ for some $z_0\in\C\setminus\sigma(H)$.
Then, the following hold true
\begin{enumerate}
\item
$\sqrt{|V|}R(z)\in B_2(\hilbert)$ and $R(z)\sqrt{|V|}\in B_2(\hilbert)$ for all $z\in\C\setminus\sigma(H)$
and the maps $z\mapsto\sqrt{|V|}R(z)$ and $z\mapsto R(z)\sqrt{|V|}$ are continuous with respect to the 
Hilbert--Schmidt norm.
\item
If in addition $\sqrt{|V|}R(z_0)\sqrt{|V|}\in B_1(\hilbert)$ then $\sqrt{|V|}R(z)\sqrt{|V|}\in B_1(\hilbert)$
for all $z\in\C\setminus\sigma(H)$ and the map $z\mapsto\sqrt{|V|}R(z)\sqrt{|V|}$ is holomorphic
with respect to the trace norm.
\item
Furthermore, $R_V(z)-R(z)\in B_1(\hilbert)$ for $z\in\C\setminus(\sigma(H)\cup\sigma(H_V))$ and
the map $z\mapsto R_V(z)-R(z)$ is continuous with respect to the trace norm.
\end{enumerate}
\end{lemma}
\begin{proof}
1.
The first resolvent identity multiplied by $\sqrt{|V|}$ yields
\begin{equation*}
  \sqrt{|V|}R(z) = \sqrt{|V|}R(z_0)\big[\id  - (z-z_0) R(z)\big],\ z_0,z\in\C\setminus\sigma(H) .
\end{equation*}
The right-hand side is the product of the Hilbert--Schmidt operator $\sqrt{|V|}R(z_0)$ 
and a bounded operator. Hence, $\sqrt{|V|}R(z)\in B_2(\hilbert)$. 
Similarly, by the first resolvent identity and H\"older's inequality \eqref{hoelder_inequality}
\begin{equation*}
  \|\sqrt{|V|}R(w) - \sqrt{|V|}R(z)\|_2 \leq |w-z| \|\sqrt{|V|}R(z)\|_2 \|R(w)\| ,
\end{equation*}
which shows Lipschitz-continuity with respect to  the Hilbert--Schmidt norm since the map $w\mapsto\|R(w\|$ is continuous.
Finally, use $R(z)\sqrt{|V|}=(\sqrt{|V|}R(\bar z))^*$ to show the corresponding statements.

2.
As in 1. we obtain via the first resolvent identity
\begin{equation*}
  \sqrt{|V|}R(z)\sqrt{|V|} = \sqrt{|V|}R(z_0)\sqrt{|V|} - (z-z_0) \sqrt{|V|}R(z_0)R(z)\sqrt{|V|},\ 
     z_0,z\in\C\setminus\sigma(H) .
\end{equation*}
The right-hand side is trace class. Furthermore, by H\"older's inequality \eqref{hoelder_inequality}
\begin{equation*}
  \|\sqrt{|V|}R(w)\sqrt{|V|} - \sqrt{|V|}R(z)\sqrt{|V|}\|_1 \leq  |w-z| \| \sqrt{|V|}R(z)\|_2 \|R(w)\sqrt{|V|}\|_2 ,\
     z,w\in\C\setminus\sigma(H),
\end{equation*}
which shows Lipschitz-continuity with respect to the trace norm.  Moreover,
\begin{equation*}
  \frac{1}{z-w}\big[ \sqrt{|V|}R(w)\sqrt{|V|} - \sqrt{|V|}R(z)\sqrt{|V|}\big] 
    = \sqrt{|V|}R(z)R(w)\sqrt{|V|} .
\end{equation*}
Since the right-hand side converges in trace norm to $\sqrt{|V|}R(z)R(z)\sqrt{|V|}$ so does the difference
quotient on the left-hand side.

3.
In Kre\u\i{}n's resolvent formula \eqref{krein_Omega} the right-hand side is the
product of the Hilbert--Schmidt operators $R(z)\sqrt{|V|}J$, $\sqrt{|V|}R(z)$, see part 1.,
and the bounded operator $\Omega(z)$, Lemma \ref{energy_omega},
and thus a trace-class operator. Furthermore, the operators depend continuously on $z$ in the respective
norms and so does their product.
\end{proof}

The applications we have in mind require that the trace-class properties are valid up to the real axis
or in other words for $z\in\sigma(H)$ in Lemma \ref{energy00t}.

\begin{hypothesis}[Limiting absorption principle]\label{h1}
Let $\nu\in\R$.
\begin{enumerate}
\item 
For $s>0$ the Birman--Schwinger operators $\sqrt{|V|}R(\nu+is)\sqrt{|V|}J\in B_1(\hilbert)$ and the limit
$s\to+0$ exists with respect to $B_1(\hilbert)$.
\item 
For $s>0$ the operators $R_V(\nu+is)-R(\nu+is)\in B_1(\hilbert)$ and the limit $s\to+0$ exists 
with respect to $B_1(\hilbert)$.
\end{enumerate}
\end{hypothesis}

Note that by the resolvent property $R(z)^* = R(\bar z)$ Hypothesis \ref{h1} implies the analogous statements
for the lower half-plane, $s<0$, albeit the respective limits need not be the same. 
Now, with Lemma \ref{energy00t} and Hypothesis \ref{h1} in mind, we define
the (modified) perturbation determinant (cf. \cite[(0.9.35)]{Yafaev2010})
\begin{equation}\label{perturbation_determinant}
  \Delta(z) \coloneqq \det(\id - K(z)) .
\end{equation}
It is closely related to Kre\u\i{}n's spectral shift function $\xi$ (see Section \ref{ssf}).
Its behaviour at the non-essential spectrum is analogous to the finite dimensional case.

\begin{lemma}\label{energy00at}
Assume that $R(z)^{\frac{1}{2}}\sqrt{|V|}\in B_2(\hilbert)$ for all $z\in\C\setminus\sigma(H)$.
Let $\lambda\in\C$ be an isolated eigenvalue of $H$ and $H_V$  with finite multiplicities $m,n\in\N_0$, respectively.
Here multiplicity $0$ means that $\lambda$ is a regular value.
Then, for $z\neq 0$ the perturbation determinant can be factorized
\begin{equation*}
  \Delta(z) = \Big(1-\frac{\lambda}{z}\Big)^{n-m} \Delta_\lambda(z)
\end{equation*}
where $\Delta_\lambda(z)$ is continuous and non-vanishing in a neighbourhood of $\lambda$.
\end{lemma}
\begin{proof}
Cf. \cite[IV.3.4]{GohbergKrein1969}. 
Since $R(z)^{\frac{1}{2}}\sqrt{|V|}\in B_2(\hilbert)$ we know that $R(z)^{\frac{1}{2}}VR(z)^{\frac{1}{2}}\in B_1(\hilbert)$ and thus
\begin{equation*}
  \Delta(z) = \det( \id -R(z)^{\frac{1}{2}}VR(z)^{\frac{1}{2}} )
\end{equation*}
where the square root $R(z)^{\frac{1}{2}}$ is defined via the functional calculus.
Determinants of this type were also studied in \cite{GesztesyZinchenko2012}.
Furthermore,
\begin{equation*}
  \id -R(z)^{\frac{1}{2}}VR(z)^{\frac{1}{2}}  = R(z)^{\frac{1}{2}}(z\id - H_V)R(z)^{\frac{1}{2}} .
\end{equation*}
Let $\Pi_\lambda$ be the spectral projection of $H_V$ corresponding to $\lambda$ and $\Pi_\lambda^\perp\coloneqq\id-\Pi_\lambda$. 
From the simple relation
\begin{equation*}\
  z\id-H_V =  (\id- \frac{1}{z}H_V\Pi_\lambda)(z\id-H_V\Pi_\lambda^\perp),\ z\neq 0,
\end{equation*}
we obtain
\begin{equation*}
\begin{split}
  \id -R(z)^{\frac{1}{2}}VR(z)^{\frac{1}{2}}
     & = R(z)^{\frac{1}{2}}(\id-\frac{1}{z}H_V\Pi_\lambda)(z\id-H_V\Pi_\lambda^\perp)R(z)^{\frac{1}{2}}\\
     & = R(z)^{\frac{1}{2}}(\id-\frac{1}{z}H_V\Pi_\lambda)(z\id-H)^{\frac{1}{2}}\times
            R(z)^{\frac{1}{2}}(z\id-H_V\Pi_\lambda^\perp)R(z)^{\frac{1}{2}}\\
     & = ( \id - \frac{\lambda}{z}R(z)^{\frac{1}{2}}\Pi_\lambda(z\id-H)^{\frac{1}{2}} )\times
            R(z)^{\frac{1}{2}}(z\id-H_V\Pi_\lambda^\perp)R(z)^{\frac{1}{2}} .
\end{split}
\end{equation*}
The determinant of the second factor exists. For,
\begin{equation*}
 R(z)^{\frac{1}{2}}(z\id-H_V\Pi_\lambda^\perp)R(z)^{\frac{1}{2}}
    = \id - R(z)^{\frac{1}{2}}VR(z)^{\frac{1}{2}} + \lambda R(z)^{\frac{1}{2}} \Pi_\lambda R(z)^{\frac{1}{2}} .
\end{equation*}
Here, the term with $V$ is trace class by assumption and the term with $\Pi_\lambda$ is a finite rank operator.
Hence, we can write our determinant as
\begin{equation*}
  \Delta(z) = \det\Big(\id-\frac{\lambda}{z}\Pi_\lambda\Big) \det\big[ (z\id-H)^{-\frac{1}{2}}(z\id-H_V\Pi_\lambda^\perp)(z\id-H)^{-\frac{1}{2}} \big] .
\end{equation*}
Let likewise $P_\lambda$ be the spectral projection of $H$ corresponding to $\lambda$ and $P_\lambda^\perp\coloneqq\id-P_\lambda$. Write
\begin{equation*}
  z\id-H = (\id-\frac{1}{z}HP_\lambda)(z\id-HP_\lambda^\perp),\ z\neq 0  .
\end{equation*}
Then,
\begin{equation*}
  R(z) = (\id-\frac{1}{z}HP_\lambda)^{-1} \tilde R(z),\  \tilde R(z) \coloneqq (z\id-HP_\lambda^\perp)^{-1}  ,
\end{equation*}
which implies that
\begin{equation*}
  \tilde R(z)^{\frac{1}{2}}\sqrt{|V|} = (\id-\frac{\lambda}{z}P_\lambda)^{\frac{1}{2}} R(z)^{\frac{1}{2}}\sqrt{|V|}
     \in B_2(\hilbert),\ z\in\C\setminus(\sigma(H)\cup\{0\}) .
\end{equation*}
Furthermore, from
\begin{equation*}
  H_V\Pi_\lambda^\perp = HP_\lambda^\perp + V + \lambda P_\lambda-\lambda\Pi_\lambda
\end{equation*}
we obtain
\begin{equation*}
  \tilde R(z)^{\frac{1}{2}}(z\id-H_V\Pi_\lambda^\perp)\tilde R(z)^{\frac{1}{2}}
    = \id - \tilde R(z)^{\frac{1}{2}}(V+\lambda P_\lambda - \lambda\Pi_\lambda)\tilde R(z)^{\frac{1}{2}} .
\end{equation*}
Since $P_\lambda$ and $\Pi_\lambda$ are finite rank operators we infer that the determinant
\begin{equation*}
   \Delta_\lambda(z) 
      \coloneqq \det [ \tilde R(z)^{\frac{1}{2}}(z\id-H_V\Pi_\lambda^\perp)\tilde R(z)^{\frac{1}{2}} ]
      = \det [ \id - \tilde R(z)^{\frac{1}{2}}(V+\lambda P_\lambda - \lambda\Pi_\lambda)\tilde R(z)^{\frac{1}{2}}]
\end{equation*}
is well-defined. Therefore,
\begin{equation*}
  \Delta(z) = \det[\id-\frac{\lambda}{z}\Pi_\lambda ] \det[\id-\frac{\lambda}{z}P_\lambda]^{-1} \Delta_\lambda(z)  .
\end{equation*}
We study the behaviour for $z\to\lambda$. To begin with,
\begin{equation*}
  \tilde R(z)^{\frac{1}{2}}\sqrt{|V|}
     =   \tilde R(w)^{\frac{1}{2}}\sqrt{|V|}
           + \frac{1}{2} (w-z) \tilde R(z)^{\frac{1}{2}} \int_0^1 \tilde R(tw+(1-t)z)^{\frac{1}{2}}\, dt
                          \times \tilde R(w)^{\frac{1}{2}}\sqrt{|V|}
\end{equation*}
where $w\in\C\setminus\sigma(H)$ such that $tw+(1-t)z\in\C\setminus\sigma(H)$.
Using H\"older's inequality \ref{hoelder_inequality} we conclude that the map 
$z\mapsto\tilde R(z)^{\frac{1}{2}}\sqrt{|V|}\in B_2(\hilbert)$ is continuous in a neighbourhood of $\lambda$. 
Therefore, the map $z\mapsto\Delta_\lambda(z)$ is continuous in that neighbourhood as well.
Finally, one can easily check that $1$ is not an eigenvalue of 
$\tilde R(\lambda)^{\frac{1}{2}}(V+\lambda P_\lambda - \lambda\Pi_\lambda)\tilde R(\lambda)^{\frac{1}{2}}$ and 
hence $\Delta_\lambda(\lambda)\neq 0$. Along with the continuity of $\Delta_\lambda(z)$ that
proves the statement.
\end{proof}

In order to allow the Fermi energy $\nu$ to be in the essential spectrum as well we extend 
the Dunford integral formula for holomorphic functions of self-adjoint operators.

\begin{lemma}\label{dunford01t}
Let the self-adjoint operator $H$ be bounded from below. Let $\nu\in\R$ and assume that 
the spectral family $E_\lambda$ is continuous in a neighbourhood of $\nu$. 
Let $P$ be the spectral projection to the set $\interval[open left]{-\infty}{\nu}$.
Let $\Gamma$ be a closed contour crossing the real axis perpendicularly at $\nu$. Then,
\begin{equation}\label{dunford01t01}
  f(H)P = \int_\Gamma f(z) R(z)\, dz
\end{equation}
where the integral at $\nu$ is to be understood as a Cauchy principal value.
\end{lemma}
\begin{proof}
By the spectral theorem
\begin{equation}\label{dunford01t02}
  f(H)P = \int_{-\infty}^\nu f(\lambda)\, dE_\lambda .
\end{equation}
Since $H$ is bounded from below the interval of integration is actually finite.
We want to replace $f$ via Cauchy's integral formula. For $\lambda\neq\nu$,
\begin{equation}\label{dunford01t03}
  \frac{1}{2\pi i} \int_{\Gamma\setminus\Gamma_\delta} \frac{f(z)}{z-\lambda} \, dz =
\begin{cases}
  f(\lambda) - F_\delta(\lambda) & \lambda \ \text{inside}\ \Gamma , \\
             - F_\delta(\lambda) & \lambda \ \text{outside}\ \Gamma
\end{cases}
\end{equation}
where
\begin{equation*}
   F_\delta(\lambda) =  \frac{1}{2\pi i} \int_{\Gamma_\delta} \frac{f(z)}{z-\lambda} \, dz,\
   \Gamma_\delta \coloneqq \{ z\in\Gamma \mid |z-\nu| \leq \delta \} .
\end{equation*}
By our assumption on $\Gamma$ and with the aid of Cauchy's integral theorem, $\Gamma_\delta$ can be replaced
by a straight line. For the sake of simplicity we assume that it already is one. Hence,
\begin{equation*}
  z\in\Gamma_\delta,\ z(s) = \nu + is,\ -\delta \leq s \leq\delta .
\end{equation*}
By Taylor's formula,
\begin{equation*}
\begin{split}
  F_\delta(\lambda)
    & = \frac{1}{2\pi}\int_{-\delta}^\delta f(\nu+is)\frac{1}{\nu+is-\lambda}\, ds \\
    & = \frac{1}{2\pi} f(\nu) \int_{-\delta}^\delta \frac{1}{\nu+is-\lambda}\, ds +
       \frac{1}{2\pi}  \int_{-\delta}^\delta s f'(\xi(s))\frac{1}{\nu+is-\lambda}\, ds .
\end{split}
\end{equation*}
The first integral can be rewritten
\begin{equation*}
  \int_{-\delta}^\delta \frac{1}{\nu+is-\lambda}\, ds
     =  \int_0^\delta  \frac{\nu-\lambda}{(\nu-\lambda)^2+s^2}\, ds
     =  \int_0^{\frac{\delta}{|\nu-\lambda|}} \frac{1}{1+s^2}\, ds
\end{equation*}
and the second integral can be estimated
\begin{equation*}
 \big|  \int_{-\delta}^\delta s f'(\xi(s))\frac{1}{\nu+is-\lambda}\, ds \big|
   \leq \sup_{z\in\Gamma_\delta}|f'(z)| \int_{-\delta}^\delta \frac{|s|}{((\nu-\lambda)^2+s^2)^{\frac{1}{2}}}\, ds
   \leq 2\delta \sup_{z\in\Gamma_\delta}|f'(z)| .
\end{equation*}
Hence, $|F_\delta(\lambda)|\leq C$ for all $\lambda\in\R$ with some constant $C\geq 0$
and $F_\delta(\lambda)\to 0$ as $\delta\to 0$ for all $\lambda\neq\nu$. We insert \eqref{dunford01t03}
into \eqref{dunford01t02} and obtain
\begin{equation*}
\begin{split}
  f(H)P & = \frac{1}{2\pi i} \int_\R \int_{\Gamma\setminus\Gamma_\delta} \frac{f(z)}{z-\lambda}\, dz\, dE_\lambda +
          \int_\R F_\delta(\lambda)\, dE_\lambda \\
        & = \frac{1}{2\pi i} \int_{\Gamma\setminus\Gamma_\delta} f(z) \int_\R \frac{1}{z-\lambda}\, dE_\lambda\, dz
             +  \int_\R F_\delta(\lambda)\, dE_\lambda\\
        & = \frac{1}{2\pi i} \int_{\Gamma\setminus\Gamma_\delta} f(z) R(z)\, dz
             +  \int_\R F_\delta(\lambda)\, dE_\lambda .
\end{split}
\end{equation*}
By Lebesgue's Theorem, the integral over $F_\delta$ tends to zero as $\delta\to 0$. 
Thus, the first one yields \eqref{dunford01t01}.
\end{proof}

In what follows we will apply Lemmas \ref{energy00at} and \ref{dunford01t} only to operators
with special types of spectra which makes it reasonable to
single out the corresponding assumptions on the Fermi energy $\nu$.

\begin{hypothesis}\label{h2}
Let $\nu\in\R$ and let $P_\nu$ and $\Pi_\nu$ the spectral projections of $H$ and $H_V$ as in \eqref{spectral_projections}.
Then, at least one of the following holds true.
\begin{enumerate}
\item $\dim\ran P_\nu<\infty$ and $\dim\ran\Pi_\nu<\infty$.
\item $P_\lambda$ and $\Pi_\lambda$ are continuous in a neighbourhood of $\nu$.
\end{enumerate}
\end{hypothesis}

The conditions in Hypothesis \ref{h2} are not independent since $\dim\ran P_\nu=0=\dim\ran\Pi_\nu$ in 1.
obviously implies 2. Nor are they exhaustive in that embedded eigenvalues are not included.
Now, everything is at hand to derive the aforementioned formula for the energy difference.

\begin{proposition}\label{energy01t}
Let $H$ and $H_V$ be semi-bounded from below and let $\Gamma_\nu$ be a closed contour in $\C$ 
intersecting the real axis perpendicularly at $\nu\in\R$ and below $\sigma(H)$ and $\sigma(H_V)$ 
(cf. Figure \ref{f_fermi_parabola}, p. \pageref{f_fermi_parabola}).
Assume that Hypothesis \ref{h1} holds true.
Then, the energy difference $\mathcal{E}(\nu)$ from \eqref{energy_difference} satisfies
\begin{equation}\label{energy01t01}
  \mathcal{E}(\nu)
      =   - f(\nu)\xi(\nu) - \frac{1}{2\pi i} \int_{\Gamma_\nu} f'(z) \ln[\Delta(z)]\, dz
\end{equation}
where $f:\C\to\C$ is a holomorphic function. 
\end{proposition}
\begin{proof}
(i)
To begin with, let $\varkappa\in\R$ such that the conditions of Lemma \ref{dunford01t} are satisfied.
Then, we can express the energy difference through a Dunford integral
\begin{equation*}
    f(H_V)\Pi_\varkappa - f(H)P_\varkappa = \frac{1}{2\pi i} \int_{\Gamma_\varkappa} f(z)[ R_V(z) - R(z) ] \, dz  
\end{equation*}
which is a Riemann integral with respect to the operator norm.
By Hypothesis \ref{h1} the integrand is piecewise continuous with respect to the trace norm.
Therefore, the right-hand side is a trace class operator. We, thus, may take the trace and interchange it
with the integration
\begin{equation}\label{energy01t02}
   \mathcal{E}(\varkappa)
    = \tr [f(H_V)\Pi_\varkappa - f(H)P_\varkappa ]
    = \frac{1}{2\pi i} \int_{\Gamma_\varkappa} f(z) \tr [ R_V(z) - R(z) ] \, dz .
\end{equation}
This trace is the logarithmic derivative of the perturbation determinant (see e.g. \cite[(8.1.4)]{Yafaev1992})
\begin{equation*}
  \tr [ R_V(z) - R(z) ] =  \frac{d}{dz} \ln[\Delta(z)] .
\end{equation*}
Thereby, via an integration by parts \eqref{energy01t02} becomes 
\begin{equation}\label{energy01t03}
\begin{split}
  \mathcal{E}(\varkappa)
     & = \frac{1}{2\pi i} \int_{\Gamma_\varkappa} f(z) \frac{d}{dz} \ln[\Delta(z)] \, dz\\
     & = \frac{1}{2\pi i}\lim_{\varepsilon\to+0} [ f(\varkappa-i\varepsilon)\ln[\Delta(\varkappa-i\varepsilon)] 
                                                -f(\nu+i\varepsilon)\ln[\Delta(\varkappa+i\varepsilon)] ]
       - \frac{1}{2\pi i} \int_{\Gamma_\varkappa} f'(z)\ln[\Delta(z)] \, dz\\
     & = - f(\varkappa) \xi(\varkappa) - \frac{1}{2\pi i} \int_{\Gamma_\varkappa} f'(z)\ln[\Delta(z)] \, dz .
\end{split}
\end{equation}
Here we used the continuity of $f$ and Kre\u\i{}n's formula for the spectral shift function \eqref{krein_xi}.

(ii)
If $\nu\in\R$ satisfies the conditions of Lemma \ref{dunford01t} we may choose $\varkappa\coloneqq\nu$ in \eqref{energy01t03}
to prove \eqref{energy01t01}.

(iii)
If $\nu\in\R$ is an isolated eigenvalue of $H$ or $H_V$ we take $\varkappa\coloneqq\nu+\delta$
where $\delta>0$ is so small that $\interval[open left]{\nu}{\nu+\delta}\cap(\sigma(H)\cup\sigma(H_V))=\emptyset$.
In the following considerations we assume that near $\nu$ the contour $\Gamma_\nu$ is a straight line 
\begin{equation*}
   \Gamma_\nu = \tilde\Gamma_\nu \cup \{ z=\nu+is \mid -b\leq s\leq b\},\ b>0,
\end{equation*}
which can be justified by means of standard arguments. Let
\begin{equation*}
  \Gamma_\varkappa \coloneqq \Gamma_{\nu,\delta} 
                 \coloneqq \tilde\Gamma_\nu \cup \Gamma_{\nu,\delta}^{(1)} \cup \Gamma_{\nu,\delta}^{(2)},\
    \Gamma_{\nu,\delta}^{(1)}\coloneqq \{ z = \nu+is \mid \delta \leq |s| \leq b\} ,\ 
    \Gamma_{\nu,\delta}^{(2)}\coloneqq \{ z = \nu+\delta e^{i\vartheta} \mid -\frac{\pi}{2} \leq \vartheta \leq \frac{\pi}{2}\} .
\end{equation*}
The choice of $\delta$ implies $\mathcal{E}(\nu+\delta)=\mathcal{E}(\nu)$ as well as $\xi(\nu+\delta)=\xi(\nu)$ (see \eqref{lifshitz}).
Then, \eqref{energy01t03} and Lemma \ref{energy00at} yield
\begin{equation}\label{energy01t04}
\begin{split}
  \mathcal{E}(\nu) 
    & = -f(\nu+\delta)\xi(\nu) - \frac{1}{2\pi i} \int_{\Gamma_{\nu,\delta}} f'(z)\ln[\Delta(z)]\, dz\\
    & = -f(\nu+\delta)\xi(\nu)
           - \frac{1}{2\pi i}\Big[(n-m)\int_{\Gamma_{\nu,\delta}} f'(z) \ln(1-\frac{\nu}{z})\, dz 
                                    + \int_{\Gamma_{\nu,\delta}} f'(z) \ln[\Delta_\nu(z)]\, dz\Big] .
\end{split}
\end{equation}
In the last integral we may perform the limit $\delta\to 0$ which simply means to replace $\Gamma_{\nu,\delta}$ by $\Gamma_\nu$.
For the remaining integral only the parts over $\Gamma_{\nu,\delta}^{(1)}$ and $\Gamma_{\nu,\delta}^{(2)}$ need a closer look
\begin{equation*}
  \int_{\Gamma_{\nu,\delta}^{(1)}} f'(z) \ln(1- \frac{\nu}{z})\, dz
     = i \int_{\delta\leq |s|\leq b} f'(\nu+is) \ln(1-\frac{\nu}{\nu+is})\, ds
     = i \int_{\frac{1}{b}\leq |s|\leq \frac{1}{\delta}} f'(\nu+\frac{i}{s}) \frac{1}{s^2}\ln(1-\frac{s\nu}{{s\nu+i}})\, ds .
\end{equation*}
The limit $\delta\to 0$ exists since the logarithm is dominated by any power of $s$. Furthermore,
\begin{equation*}
  \int_{\Gamma_{\nu,\delta}^{(2)}} f'(z) \ln(1- \frac{\nu}{z})\, dz
     = i \delta\int_{-\frac{\pi}{2}}^{\frac{\pi}{2}} f'(\nu+\delta e^{i\vartheta}) 
                         \ln(1- \frac{\nu}{\nu+\delta e^{i\vartheta}})e^{i\vartheta}\, d\vartheta \to 0,\ \delta\to 0 .
\end{equation*}
We conclude that
\begin{equation*}
  \lim_{\delta\to 0} \int_{\Gamma_{\nu,\delta}} f'(z) \ln[\Delta(z)]\, dz
     = \int_{\Gamma_\nu} f'(z) \ln[\Delta(z)]\, dz
\end{equation*}
in \eqref{energy01t04}. This and the continuity of $f$ prove \eqref{energy01t01}.
\end{proof}
