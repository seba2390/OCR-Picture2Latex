We will need some analytic properties of the Birman--Schwinger operators
\begin{gather}
  K_L(z)\coloneqq\sqrt{|V|}R_L(z)\sqrt{|V|}J,\label{K_L}\\
  K_\infty(z)\coloneqq\sqrt{|V|}R_\infty(z)\sqrt{|V|}J,\ K_{\infty,L}(z)\coloneqq\sqrt{|V|}R_{\infty,L}(z)\sqrt{|V|}J\label{K_infinity} ,
\end{gather}
see \eqref{V_polar_decomposition} and \eqref{R_L}, \eqref{R_infinity}, \eqref{R_infinity_L},
as well as their boundary values (cf. \eqref{R_boundary_values})
\begin{equation}\label{K_boundary_values}
  K_{\infty,L}^\pm(\nu) \coloneqq \sqrt{|V|}R_{\infty,L}^\pm(\nu)\sqrt{|V|}J,\
  K_\infty^\pm(\nu) \coloneqq \sqrt{|V|}R_\infty^\pm(\nu)\sqrt{|V|}J .
\end{equation}
We will use the simple facts
\begin{equation}\label{exp_properties}
  |e^{\pm i\sqrt{z}u}| = e^{\mp\im(\sqrt{z})u} \ \text{and}\
  |e^{i\sqrt{z}u} - e^{i\sqrt{w}u}| \leq |u| |\sqrt{z}-\sqrt{w}|\int_0^1 e^{\mp u (t \im(\sqrt{z}) + (1-t)\im(\sqrt{w}))}\, dt ,\ u\in\R .
\end{equation}

\begin{lemma}\label{bso01t}
Let $V\in L^1(\R)$ and $z\in\C\setminus\{0\}$. Then the following hold true.
\begin{enumerate}
\item The Birman--Schwinger operators $K_{\infty,L}(z)$ and $K_\infty(z)$ (see \eqref{K_infinity}) are in $B_2(\hilbert)$ with
\begin{equation}\label{bso01t01}
  \| K_{\infty,L}(z)\|_2 \leq \|K_\infty(z)\|_2 \leq \frac{1}{2|\sqrt{z}|}\|V\|_1 .
\end{equation}
In particular, $\sqrt{|V|}R_{\infty,L}(z)\in B_2(\hilbert)$.
\item
If, in addition, $X^nV\in L^1(\R)$ for $n=1,2$ (cf. \eqref{V_decay1}), then for $w\neq 0$
\begin{equation}\label{bso01t01a}
  \|K_{\infty,L}(z) - K_{\infty,L}(w)\|_2
       \leq  \|V\|_{1,2} \frac{1}{|\sqrt{z}|}\Big( 1+\frac{1}{2|w|}\Big)^{\frac{1}{2}} |\sqrt{z}-\sqrt{w}|,\
  \im(\sqrt{z})\cdot\im(\sqrt{w})\geq 0 .
\end{equation}
\item
Let $\chi_L^\perp\coloneqq 1-\chi_L$. Then the operator $K_{\infty,L}(z):\hilbert_\infty\to\hilbert_\infty$ satisfies
\begin{equation}\label{bso01t02}
  \|K_{\infty,L}(z) - K_\infty(z)\|_2 \leq \frac{1}{\sqrt{|z|}} \|V\|_1^{\frac{1}{2}} \|\chi_L^\perp V\|_1^{\frac{1}{2}} .
\end{equation}
\end{enumerate}
\end{lemma}
\begin{proof}
We prove the estimates, essentially, by bounding the respective kernel functions, cf. \eqref{green_infinity}.
Throughout the proof let $\im(\sqrt{z})\geq 0$ and $\im(\sqrt{w})\geq 0$. The other case can be treated in like manner.

1.
The bound \eqref{bso01t01} follows from
\begin{equation}\label{bso01t03}
  |R_\infty(z;x,y)|^2 = \frac{1}{4|z|} e^{-2\im(\sqrt{z})|x-y|}
                 \leq \frac{1}{4|z|} .
\end{equation}
This also implies that $\sqrt{|V|}R_{\infty,L}(z) = \sqrt{|V|}\chi_LR_\infty(z)\chi_L\in B_2(\hilbert)$, cf. \eqref{R_infinity_L}.

2.
We use \eqref{exp_properties} with $u\coloneqq |x-y|\geq 0$
\begin{multline*}
   |R_\infty(z;x,y) - R_\infty(w;x,y)|
     \leq \frac{1}{2}\frac{1}{|\sqrt{z}|}|e^{i\sqrt{z}u}-e^{i\sqrt{w}u}| + \frac{1}{2}|\frac{1}{\sqrt{z}}-\frac{1}{\sqrt{w}}| |e^{i\sqrt{w}u}| \\
     \leq \frac{1}{2|\sqrt{z}|} |\sqrt{z}-\sqrt{w}| ( u + \frac{1}{|\sqrt{w}|} ) 
     \leq \frac{1}{\sqrt{2}}\frac{1}{|\sqrt{z}|} |\sqrt{z}-\sqrt{w}| ( u^2 + \frac{1}{|w|} )^{\frac{1}{2}} .
\end{multline*}
The term with $u^2$ leads to
\begin{equation*}
  \int_{-L}^L\int_{-L}^L |x-y|^2 |V(x)| |V(y)|\, dy \, dx
     = 2\int_{-L}^L\int_{-L}^L x^2 |V(x)| |V(y)|\, dy \, dx 
         - 2\Big[ \int_{-L}^L  x |V(x)| \, dx\Big]^2 .
\end{equation*}
Dropping the rightmost term yields
\begin{equation*}
  \| K_{\infty,L}(z) - K_{\infty,L}(w)\|_2^2
    \leq  \frac{1}{2}\frac{1}{|z|} |\sqrt{z}-\sqrt{w}|^2 \big[ 2 \|X^2V\|_1 \|V\|_1 + \frac{1}{|w|} \|V\|_1^2 \big] .
\end{equation*}
Finally, we simplify the constant via \eqref{V_decay1} and obtain \eqref{bso01t01a}.

3. Furthermore, we have from \eqref{bso01t03}
\begin{equation*}
\begin{split}
  |R_{\infty,L}(z;x,y) - R_\infty(z;x,y)|
    & \leq |(\chi_L(x)-1)R_\infty(z;x,y)\chi_L(y)| + |R_\infty(z;x,y)(\chi_L(y)-1)|\\
    & \leq \frac{1}{2\sqrt{|z|}} \big[  |\chi_L(x)-1| + |\chi_L(y)-1| \big]
\end{split}
\end{equation*}
since $0\leq\chi_L\leq\id$. That implies \eqref{bso01t02}.
\end{proof}

In order to make the Fredholm determinants such as in \eqref{energy01t01} 
well-defined we establish the relevant trace class properties of the Birman--Schwinger operators 
starting with the finite volume operator. 

\begin{lemma}\label{bso02at}
Let $V\in L^1(\R)$. Then, $\sqrt{|V|}R_L(z)\in B_2(\hilbert_L)$ and
$K_L(z)\in B_1(\hilbert_L)$ for all $z\notin\sigma(H_L)$. 
The map $z\mapsto K_L(z)$ is holomorphic with respect to $B_1(\hilbert_L)$ on $\C\setminus\sigma(H_L)$. 
\end{lemma}
\begin{proof}
The spectral representation of the resolvent yields
\begin{equation}\label{bso02at01}
  \sqrt{|V|}R_L(z)\sqrt{|V|} = 
     \sum_{j=1}^\infty \frac{1}{z-\lambda_{j,L}} (\sqrt{|V|}\varphi_{j,L},\cdot)\sqrt{|V|}\varphi_{j,L}.
\end{equation}
From \eqref{H_L_spectrum02} we infer that
\begin{equation}\label{bso02at02}
  \sum_{j=1}^\infty \frac{1}{|z-\lambda_{j,L}|} < \infty \ \text{for}\ z\notin\sigma(H_L) .
\end{equation}
The eigenvectors $\varphi_{j,L}$ have the general form
\begin{equation*}
  \varphi(x) = C_1 e^{i\sqrt{\lambda}x} + C_2 e^{-i\sqrt{\lambda}x}
\end{equation*}
with appropriate constants $C_{1,2}$. For $\lambda>0$ the normalization implies 
\begin{equation*}
  1  = \int_{-L}^L |\varphi(x)|^2 \, dx
     = 2L ( |C_1|^2 + |C_2|^2 ) + 2\re (C_1\bar C_2) \frac{\sin(2L\sqrt{\lambda})}{\sqrt{\lambda}} 
     \geq \frac{2L\sqrt{\lambda}-1}{\sqrt{\lambda}} ( |C_1|^2 + |C_2|^2 ).
\end{equation*}
For $2L\sqrt{\lambda}>1$ this yields an upper bound for $|C_1|^2+|C_2^2|$ and thereby
\begin{equation*}
  |\varphi_{j,L}(x)|^2 \leq 2 (|C_1|^2+|C_2|^2) \leq \frac{2\sqrt{\lambda_{j,L}}}{2L\sqrt{\lambda_{j,L}}-1}\ 
     \text{for}\ 2L\sqrt{\lambda_{j,L}} > 1.
\end{equation*}
By \eqref{H_L_spectrum02} there are only finitely many $\lambda_{j,L}$ with $2L\sqrt{\lambda_{j,L}} \leq 1$
whence $\|\varphi_{j,L}\|_\infty\leq C$, $j\in\N$, with some constant $0\leq C<\infty$, which may depend on $L$, however.
Using $V\in L^1(\R)$ we conclude
\begin{equation}\label{bso02at03}
  \sup_{j\in\N} \|\sqrt{|V|}\varphi_{j,L}\| < \infty .
\end{equation}
With the aid of \eqref{bso02at02} and \eqref{bso02at03} we deduce from \eqref{bso02at01}
that $\sqrt{|V|}R_L(z)\sqrt{|V|}$ is compact since it is the operator norm limit of finite rank operators. 
Let $\varkappa_j\geq 0$ be its singular values. The singular value decomposition gives
\begin{equation*}
  0 \leq \sum_{j=1}^n \varkappa_j
       = \sum_{j=1}^n (f_j, \sqrt{|V|}R_L(z)\sqrt{|V|} g_j)
       = \sum_{k=1}^\infty \frac{1}{z-\lambda_{k,L}}
           \sum_{j=1}^n (f_j,\sqrt{|V|}\varphi_{k,L})(\sqrt{|V|}\varphi_{k,L},g_j)
\end{equation*}
with orthonormal systems $\{f_j\}$ and $\{g_j\}$. By the Cauchy--Schwarz inequality and Bessel's inequality 
\begin{equation*}
\begin{split}
  \sum_{j=1}^n \varkappa_j
    & \leq \sum_{k=1}^\infty \frac{1}{|z-\lambda_{k,L}|}
           \Big[\sum_{j=1}^n |(f_j,\sqrt{|V|}\varphi_{k,L})|^2\Big]^{\frac{1}{2}}
           \Big[\sum_{j=1}^n |(\sqrt{|V|}\varphi_{k,L},g_j)|^2\Big]^{\frac{1}{2}} \\
    & \leq \sum_{k=1}^\infty \frac{1}{|z-\lambda_{k,L}|} \|\sqrt{|V|}\varphi_{k,L}\|^2 \\
    & < \infty
\end{split}
\end{equation*}
which shows the trace class property. For the holomorphy note that
\begin{equation*}
  \sqrt{|V|}R'(z)\sqrt{|V|} = -\sum_{j=1}^\infty \frac{1}{(z-\lambda_{j,L})^2} (\sqrt{|V|}\varphi_{j,L},\cdot)\sqrt{|V|}\varphi_{j,L}
\end{equation*}
is trace class by the same arguments. Then, standard analysis yields the statement.
That $\sqrt{|V|}R_L(z)$ is a Hilbert--Schmidt operator can be shown in like manner.
\end{proof}

The infinite volume Birman--Schwinger operators are treated in two steps (cf. \cite{Froese1997}).

\begin{lemma}\label{bso02t}
Let $V\in L^1(\R)$ and $\im(\sqrt{z})\neq 0$. Then, $K_\infty(z)$ and $K_{\infty,L}(z)$
are trace class and
\begin{equation}\label{bso02t01}
  \|K_{\infty,L}(z)\|_1 \leq \| K_\infty(z)\|_1 \leq \frac{1}{2|\im(\sqrt{z})|} \|V\|_1 .
\end{equation}
\end{lemma}
\begin{proof}
The first inequality in \eqref{bso02t01} follows via H\"older's inequality \eqref{hoelder_inequality}.
In order to prove the second we factor the resolvent into
\begin{equation*}
   R_\infty(z) = ( \sqrt{z}\id - i\nabla)^{-1}(\sqrt{z}\id + i\nabla)^{-1}
\end{equation*}
and start with the first factor, which has the integral kernel
\begin{equation*}
  (\sqrt{z}\id - i\nabla)^{-1}(x,y) = - e^{-i\sqrt{z}x}e^{i\sqrt{z}y} \Theta(y-x),
   \ \im(\sqrt{z})>0 .
\end{equation*}
We compute the Hilbert--Schmidt norm
\begin{equation*}
\begin{split}
\|\sqrt{|V|}(\sqrt{z}\id-i\nabla)^{-1}\|_2^2
     & = \int_\R\int_\R |V(x)| | e^{-i\sqrt{z}x}e^{i\sqrt{z}y} |^2
          \Theta(y-x)\, dy \, dx  \\
     & = \int_\R |V(x)| e^{2\im(\sqrt{z})x} \int_x^\infty e^{-2\im(\sqrt{z})y}\, dy\, dx\\
     & = \frac{1}{2\im(\sqrt{z})}\|V\|_1 .
\end{split}
\end{equation*}
The second factor can be treated likewise. 
For $\im(\sqrt{z})<0$ we obtain the same result since $K_{\infty,L}(\bar{z}) = J K_{\infty,L}(z)^*J$ and $K_{\infty}(\bar{z}) = J K_{\infty}(z)^*J$.

Thus, $K_\infty(z)=\sqrt{|V|}R_\infty(z)\sqrt{|V|}J$ is the product of two Hilbert--Schmidt operators and thereby trace class.
\end{proof}

We prove a trace class limiting absorption principle which, in particular, extends Lemma \ref{bso02t}
to $\im(\sqrt{z})=0$. Our proof combines the use of indicator functions as in \cite[Prop. 5.6]{Simon2005} with a
local resolvent formula from \cite{Froese1997}. Thereby, the Birman--Schwinger operators can, essentially, be written as an infinite 
sum of rank one operators whose trace norm can be computed explicitly
\begin{equation}\label{trace_norm_rank_one}
  \|(g,\cdot)f\|_1 = \|f\| \|g\|,\ f,g\in\hilbert .
\end{equation}
For an alternative approach based upon Mourre-type estimates see \cite[Thm. 6.1]{Sobolev1993}.

\begin{lemma}\label{bso03t} 
Let $V\in L^1(\R)$ satisfy $V\in\ell^{\frac{1}{2}}(L^1(\R))$ and $X^2V\in\ell^{\frac{1}{2}}(L^1(\R))$, see \eqref{birman_solomyak}.
Let $z,w\in\C\setminus\{0\}$ with $\im(\sqrt{z})\cdot\im(\sqrt{w})\geq 0$.
Then, $K_{\infty,L}(z)-K_{\infty,L}(w)\in B_1(\hilbert)$ and
$K_\infty(z)-K_\infty(w)\in B_1(\hilbert)$. Moreover,
\begin{equation}\label{bso03t01}
\begin{split}
    \| K_{\infty,L}(z)-K_{\infty,L}(w)\|_1
     & \leq \| K_\infty(z)-K_\infty(w)\|_1\\
     & \leq \frac{|\sqrt{z}-\sqrt{w}|}{|\sqrt{z}\sqrt{w}|} \Big\{ 
              |\sqrt{z}+\sqrt{w}|\Big[ \llbracket V\rrbracket_{1,\frac{1}{2}} \llbracket X^2V\rrbracket_{1,\frac{1}{2}} + \frac14 \|V\|_1 \Big]
                    + \llbracket V\rrbracket_{1,\frac{1}{2}}^2 \Big\} .
\end{split}
\end{equation}
In particular, the limits
\begin{equation*}
  \lim\limits_{\im(\sqrt{z})\to\pm 0} K_{\infty,L}(z) = K_{\infty,L}^\pm(\nu)\ \text{and}\
  \lim\limits_{\im(\sqrt{z})\to\pm 0} K_\infty(z) = K_\infty^\pm(\nu),\ z=(\sqrt{\nu}+is)^2,
\end{equation*}
exist in trace class. 
\end{lemma}
\begin{proof}
Throughout the proof we assume $\im(\sqrt{z})\geq 0$ and $\im(\sqrt{w})\geq 0$. 
The case $\im(\sqrt{z})\leq 0$ and $\im(\sqrt{w})\leq 0$ follows from the relation $K_{\infty,L}(\bar{z}) = J K_{\infty,L}(z)^*J$ 
and $K_{\infty}(\bar{z}) = J K_{\infty}(z)^*J$.

The first inequality in \eqref{bso03t01} follows via H\"older's inequality \eqref{hoelder_inequality}.
In order to prove the second we write the difference as an infinite sum of trace class operators.
To this end, put $I_j\coloneqq[j,j+1]$, $j\in\Z$, and write
\begin{equation*}
  K_\infty(z) - K_\infty(w) = \sum_{j,k\in\Z} \chi_{I_j}(K_\infty(z) - K_\infty(w))\chi_{I_k} .
\end{equation*}
We distinguish two cases, namely $j\neq k$ and $j=k$.

(a)
Let $j\neq k$. We start with the case $j<k$.
Let $x\in I_j$ and $y\in I_k$. Hence $x\leq y$ and the Green function \eqref{green_infinity} factorizes
which allows us to write it as the sum of four rank one operators
\begin{equation*}
\begin{split}
  \lefteqn{ R_\infty(z;x,y) - R_\infty(w;x,y) }\\
     & = \frac{1}{8i} \big[\frac{1}{\sqrt{z}} + \frac{1}{\sqrt{w}}\big]
          \big[ f(z,w;c_{jk}-x)g(z,w;y-c_{jk}) + g(z,w;c_{jk}-x)f(z,w;y-c_{jk}) \big]  \\
     & \quad + \frac{1}{4i} \big[\frac{1}{\sqrt{z}}-\frac{1}{\sqrt{w}}\big] 
          \big[h(z;c_{jk}-x)h(z;y-c_{jk})+h(w;c_{jk}-x)h(w;y-c_{jk})\big] .
\end{split}
\end{equation*}
The kernel functions are
\begin{equation*}
    f(z,w;u) \coloneqq e^{i\sqrt{z}u} - e^{i\sqrt{w}u},\
    g(z,w;u) \coloneqq e^{i\sqrt{z}u} + e^{i\sqrt{w}u},\
    h(z;u) \coloneqq e^{i\sqrt{z}u} .
\end{equation*}
The auxiliary quantity is to be chosen such that
\begin{equation*}
  c_{jk} \coloneqq 0,\ \text{for}\ j+1\leq 0\leq k\ \text{and}\
     j+1 \leq c_{jk} \leq k \ \text{otherwise} .
\end{equation*}
We conclude that $\chi_{I_j}(K_\infty(z) - K_\infty(w))\chi_{I_k}$ is of finite rank and, thereby, trace class.
We estimate its trace norm via \eqref{trace_norm_rank_one}. To this end, we use \eqref{exp_properties} and obtain
\begin{gather*}
  \|\chi_{I_j}\sqrt{|V|}f(z,w)\|\cdot \|\chi_{I_k}\sqrt{|V|}g(z,w)\|
     \leq 2|\sqrt{z}-\sqrt{w}| \Big[\int_j^{j+1} |V(x)|(c_{jk}-x)^2\, dx \int_k^{k+1} |V(y)|\, dy\Big]^{\frac{1}{2}} ,\\
  \|\chi_{I_j}\sqrt{|V|}g(z,w)\|\cdot \|\chi_{I_k}\sqrt{|V|}f(z,w)\|
     \leq 2|\sqrt{z}-\sqrt{w}| \Big[\int_j^{j+1} |V(x)|\, dx \int_k^{k+1} |V(y)|(y-c_{jk})^2\, dy\Big]^{\frac{1}{2}} .
\end{gather*}
We get rid of $c_{jk}$ by bounding the products of the integrals by
\begin{equation*}
  \alpha_j \coloneqq \int_j^{j+1} |V(x)|\, dx,\ 
  \beta_k \coloneqq \int_k^{k+1}  y^2 |V(y)|\, dy
\end{equation*}
which yields the estimates
\begin{gather*}
  \int_j^{j+1}|V(x)|\, dx \int_k^{k+1} |V(y)| (y-c_{jk})^2\, dy
     \leq \alpha_j\beta_k + \beta_j\alpha_k, \\
  \int_j^{j+1} |V(x)|(c_{jk}-x)^2\, dx \int_k^{k+1}|V(y)|\, dy
     \leq \beta_j\alpha_k  + \alpha_j\beta_k .
\end{gather*}
Here we used
\begin{align*}
  (y-c_{jk})^2 & \leq y^2     & (x-c_{jk})^2 & \leq x^2+y^2 && \text{for}\ 0<j+1\leq k ,\\
  (y-c_{jk})^2 & = y^2        & (x-c_{jk})^2 & = x^2        && \text{for}\ j+1\leq 0 \leq k ,\\
  (y-c_{jk})^2 & \leq y^2+x^2 & (x-c_{jk})^2 & \leq x^2     && \text{for}\ j+1\leq k<0 .
\end{align*}
Since the estimates are symmetric in $j$ and $k$ we obtain the same result for $j>k$.
Obviously, the norms involving $h$ are bounded by $\alpha_j$. Therefore,
\begin{equation*}
  \sum_{j\neq k} \|\chi_{I_j}(K_\infty(z) - K_\infty(w))\chi_{I_k}\|_1
     \leq \frac{1}{2}|\sqrt{z}-\sqrt{w}| \Big|\frac{1}{\sqrt{z}}+\frac{1}{\sqrt{w}}\Big| 
           \sum_{j\neq k} (\alpha_j\beta_k+\beta_j\alpha_k)^{\frac{1}{2}} 
      + \frac{1}{2}\Big|\frac{1}{\sqrt{z}}-\frac{1}{\sqrt{w}}\Big| \sum_{j\neq k} \alpha_j^{\frac{1}{2}}\alpha_k^{\frac{1}{2}} .
\end{equation*}

(b)
Let $j=k$. We use a local resolvent formula which slightly generalizes the one found in \cite[(7.1),(7.2)]{Froese1997}.
For an arbitrary interval $I\coloneqq [a,b]$, $a<b$, and its indicator function $\chi_I$ we have 
\begin{equation}\label{local_resolvent}
  \chi_I ( R_\infty(z) - R_\infty(w) )\chi_I
     = (w-z) \chi_I R_\infty(z)\chi_I R_\infty(w)\chi_I
       + \frac{1}{4i} \big( \frac{1}{\sqrt{z}}-\frac{1}{\sqrt{w}} \big)                    
               \chi_I ( E_a(z,w) + E_b(z,w) )\chi_I .
\end{equation}
Recall that $w,z\neq 0$, $\im(\sqrt{z})\geq 0$, $\im(\sqrt{w})\geq 0$.
The rank one operator $E_c(z,w)$ has the kernel
\begin{equation*}
  E_c(z,w;x,y) \coloneqq e^{i\sqrt{z}|x-c|}e^{i\sqrt{w}|y-c|},\ c\in\R .
\end{equation*}
Note that \eqref{local_resolvent} becomes the usual resolvent formula in the limit $a\to-\infty$, $b\to\infty$.
The product of the resolvents can be estimated by H\"older's inequality \eqref{hoelder_inequality} yielding the
Hilbert--Schmidt norms
\begin{equation*}
  \|\chi_{I_j}\sqrt{|V|}R_\infty(z)\chi_{I_j}\|_2 \leq \frac{1}{2|\sqrt{z}|} \alpha_j^{\frac{1}{2}},\
  \|\chi_{I_j}R_\infty(w)\sqrt{|V|}\chi_{I_j}\|_2 \leq \frac{1}{2|\sqrt{w}|} \alpha_j^{\frac{1}{2}} .
\end{equation*}
Using \eqref{trace_norm_rank_one}
\begin{equation*}
  \| \chi_{I_j} \sqrt{|V|}E_j(z,w)\sqrt{|V|}\chi_{I_j} \|_1 \leq \alpha_j,\   
  \| \chi_{I_j} \sqrt{|V|}E_{j+1}(z,w)\sqrt{|V|}\chi_{I_j}) \|_1 \leq \alpha_j  .
\end{equation*}
For the kernel estimates we used \eqref{exp_properties}. Thus,
\begin{equation*}
  \| \chi_{I_j}( K_\infty(z) - K_\infty(w) )\chi_{I_j}\|_1 
     \leq |w-z| \frac{1}{4|\sqrt{z}||\sqrt{w}|} \alpha_j + \frac{1}{2} \Big| \frac{1}{\sqrt{z}}-\frac{1}{\sqrt{w}}\Big| \alpha_j .
\end{equation*}

(c)
We combine the results of (a) and (b). 
\begin{equation*}
\begin{split}\|
  K_\infty(z) - K_\infty(w) \|_1 
    & \leq \frac{1}{2} |\sqrt{z}-\sqrt{w}| \Big|\frac{1}{\sqrt{z}} 
            + \frac{1}{\sqrt{w}}\Big| \sum_{j\neq k} (\alpha_j\beta_k+\beta_j\alpha_k)^{\frac{1}{2}} 
            + \frac{1}{2}\Big|\frac{1}{\sqrt{z}} - \frac1{\sqrt{w}}\Big| \sum_{j\not=k} \alpha_j^{\frac{1}{2}} \alpha_k^{\frac{1}{2}} \\
    & \quad +\Big[|z-w| \frac{1}{4|\sqrt{z}\sqrt{w}|} +\frac{1}{2} \Big|\frac{1}{\sqrt{z}} - \frac{1}{\sqrt{w}}\Big|\Big]\sum_j\alpha_j \\
    & \leq \frac{1}{2} |\sqrt{z}-\sqrt{w}| \Big|\frac{1}{\sqrt{z}} 
            + \frac1{\sqrt{w}}\Big| \Big[\sum_{j\neq k} \big(\alpha_j^{\frac{1}{2}}\beta_k^{\frac{1}{2}}+\beta_j^{\frac{1}{2}}\alpha_k^{\frac{1}{2}}\big) 
            +\frac{1}{2} \sum_{j}\alpha_j\Big] \\
    & \quad + \Big|\frac{1}{\sqrt{z}} - \frac{1}{\sqrt{w}}\Big|\Big(\sum_j\alpha_j^{\frac{1}{2}}\Big)^2 \\
    & \leq  |\sqrt{z}-\sqrt{w}| \Big|\frac{1}{\sqrt{z}}+\frac{1}{\sqrt{w}}\Big| 
            \big[\llbracket V\rrbracket_{1,\frac{1}{2}} \llbracket X^2 V\rrbracket_{1,\frac{1}{2}} + \frac{1}{4} \|V\|_1 \big] 
              + \Big|\frac{1}{\sqrt{z}}-\frac{1}{\sqrt{w}}\Big| \llbracket V\rrbracket_{1,\frac{1}{2}}^2.
\end{split}
\end{equation*}
See \eqref{birman_solomyak} for the definition of $\llbracket\cdot\rrbracket$.
\end{proof}

The next lemma concerns the Fumi term.

\begin{lemma}\label{bso04t}
Let $V\in L^1(\R)$ satisfy $V\in\ell^{\frac{1}{2}}(L^1(\R))$ and $X^2V\in\ell^{\frac{1}{2}}(L^1(\R))$, see \eqref{birman_solomyak}.
Then, for all $z\in\C\setminus\{0\}$ the operators $K_{\infty,L}(z)$ and $K_\infty(z)$ are trace class.
Moreover, $K_{\infty,L}(z)\to K_\infty(z)$ in $B_1(\hilbert)$ uniformly on 
compact sets $D\subset\C\setminus\{0\}$ as $L\to\infty$.
\end{lemma}
\begin{proof}
The trace class property is a simple consequence of Lemmas \ref{bso02t} and \ref{bso03t}.
To prove convergence we start with
\begin{equation*}
\begin{split}
  \|K_{\infty,L}(z) - K_\infty(z)\|_1
    & \leq \|\chi_L K_\infty(z)\chi_L - \chi_L K_\infty(z)\|_1
          + \|\chi_L K_\infty(z) - K_\infty(z) \|_1 \\
    & \leq \|K_\infty(z)\chi_L^\perp \|_1 + \| \chi_L^\perp K_\infty(z) \|_1 .
\end{split}
\end{equation*}
Obviously, all operators involved are indeed trace class. We bound the trace norm
\begin{equation*}
  \|K_\infty(z)\chi_L^\perp \|_1 \leq \| (K_\infty(z) - K_\infty(w))\chi_L^\perp \|_1 + \| K_\infty(w)\chi_L^\perp\|_1 .
\end{equation*}
Take $\sqrt{w}=\pm i$ for $\im(\sqrt{z})\gtrless 0$. Using the same factorization of the resolvent 
and the Cauchy--Schwarz inequality for the trace norm as in the proof of Lemma \ref{bso02t} we infer that
\begin{equation*}
  \| K_\infty(w) \chi_L^\perp \|_1 \leq \frac{1}{2} \|V\|_1^{\frac{1}{2}} \|V\chi_L^\perp\|_1^{\frac{1}{2}} .
\end{equation*}
The proof of Lemma \ref{bso03t} can be modified to give the bound
\begin{equation*}
\begin{split}
\lefteqn{\| (K_\infty(z) - K_\infty(w))\chi_L^\perp \|_1}\\
   & \leq \frac{1}{2} |\sqrt{z}-\sqrt{w}| \Big|\frac{1}{\sqrt{z}} 
            + \frac{1}{\sqrt{w}}\Big| \sum_{\substack{\chi_{I_k}\cdot\chi_L^\perp\neq 0\\j\neq k}} (\alpha_j\beta_k+\beta_j\alpha_k)^{\frac{1}{2}} 
            + \frac{1}{2}\Big|\frac{1}{\sqrt{z}} - \frac{1}{\sqrt{w}}\Big| 
                 \sum_{\substack{\chi_{I_k}\cdot\chi_L^\perp\neq 0\\j\neq k}} \alpha_j^{\frac{1}{2}} \alpha_k^{\frac{1}{2}} \\
   & \quad + \Big[|z-w| \frac{1}{4|\sqrt{z}\sqrt{w}|} 
           +\frac{1}{2} \Big|\frac{1}{\sqrt{z}} - \frac{1}{\sqrt{w}}\Big|\Big]\sum_{j,\chi_{I_j}\cdot\chi_L^\perp\neq 0}\alpha_j \\
   & \leq |\sqrt{z}-\sqrt{w}| \Big|\frac{1}{\sqrt{z}}+\frac{1}{\sqrt{w}}\Big| 
            \Big[\frac{1}{2}\llbracket V\rrbracket_{1,\frac{1}{2}} \llbracket X^2 V\chi_L^\perp\rrbracket_{1,\frac{1}{2}} 
               + \frac{1}{2}\llbracket X^2 V\rrbracket_{1,\frac{1}{2}}\llbracket V\chi_L^\perp\rrbracket_{1,\frac{1}{2}}  
               + \frac{1}{4} \|V\chi_L^\perp\|_1 \Big]  \\
   & \quad + \Big|\frac{1}{\sqrt{z}}-\frac{1}{\sqrt{w}}\Big| \llbracket V\rrbracket_{1,\frac{1}{2}} \llbracket V\chi_L^\perp\rrbracket_{1,\frac{1}{2}}.
\end{split}
\end{equation*}
The norm $\| \chi_L^\perp K_\infty(z) \|_1$ has the same bound. This and the assumptions on $V$ yield the claimed convergence.
\end{proof}
