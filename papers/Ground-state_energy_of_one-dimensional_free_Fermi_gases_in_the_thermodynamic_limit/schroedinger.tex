From now on we consider concrete Schr\"odinger operators in dimension one.
Let $\hilbert_\infty\coloneqq L^2(\R)$ and let 
\begin{equation}\label{H_infinity}
  H_\infty:\dom(H_\infty)\to\hilbert_\infty,\ H_\infty \coloneqq -\frac{d^2}{dx^2}
\end{equation}
be the free Schr\"odinger operator defined on the whole line. The domain $\dom(H_\infty)$ can be described
with the aid of Sobolev spaces which, however, is not needed herein.
The spectrum is $\sigma(H_\infty)=[0,\infty[$. The resolvent
\begin{equation}\label{R_infinity}
  R_\infty(z)\coloneqq(z\id - H_\infty)^{-1}:\hilbert_\infty\to\hilbert_\infty,\
  z\in\C\setminus\sigma(H_\infty) ,
\end{equation}
has a cut singularity along $\sigma(H_\infty)$.
The corresponding Green function (the kernel of $R_\infty(z)$) reflects this fact. It reads
\begin{equation}\label{green_infinity}
  R_\infty(z;x,y) =
   \mp \frac{i}{2\sqrt{z}} e^{\pm i\sqrt{z}|x-y|},\ \im(\sqrt{z})\gtrless 0,\ x,y\in\R .
\end{equation}
Here, $\sqrt{z}$ is the principal branch of the square root function. This means in particular that
for $\re(z)\geq 0$ the condition $\im(\sqrt{z})\gtrless0$ is equivalent to $\im(z)\gtrless0$.
The Green function can be extended to $\im(\sqrt{z})=0$, $z\neq 0$. We denote the respective boundary values by
\begin{equation}\label{R_boundary_values}
  R_\infty^\pm(\nu;x,y) \coloneqq \lim_{\im(\sqrt{z})\to \pm 0} R_\infty(z;x,y)
                       = \mp \frac{i}{2\sqrt{\nu}} e^{\pm i\sqrt{\nu}|x-y|}
\end{equation}
and write formally
\begin{equation*}
  R_\infty^\pm(\nu) \coloneqq \lim_{\im(\sqrt{z})\to \pm 0} R_\infty(z)
\end{equation*}
for the corresponding unbounded operators. The limit will be studied more carefully below.

We restrict $H_\infty$ to $\hilbert_L\coloneqq L^2(\Lambda_L)$, $\Lambda_L\coloneqq[-L,L]$,
thereby obtaining the maximal operator 
\begin{equation}\label{H_maximal}
  \tilde H:\dom(\tilde H)\to\hilbert_L,\ \tilde H\coloneqq -\frac{d^2}{dx^2},\
  \dom(\tilde H) \coloneqq \{\varphi|_{\Lambda_L} \mid \varphi\in\dom(H_\infty)\} . 
\end{equation}
Integration by parts shows that
\begin{equation*}
  (\varphi,\tilde H\psi) - (\tilde H\varphi,\psi)
  = (\Gamma_1\varphi,\Gamma_2\psi) - (\Gamma_2\varphi,\Gamma_1\psi)
\end{equation*}
where
\begin{equation*}
  \Gamma_1\varphi =
\begin{pmatrix}
  \varphi(L)\\
  \varphi(-L)
\end{pmatrix}
  ,\
  \Gamma_2\varphi =
\begin{pmatrix}
  -\varphi'(L)\\
  \varphi'(-L)
\end{pmatrix} .
\end{equation*}
The deficiency subspace \eqref{abc_deficiency_subspace} is
\begin{equation}\label{deficiency_subspace}
  \mathcal{N}_z = \mathspan\{ \varepsilon_1(z),\varepsilon_2(z)\},\
  \varepsilon_1(z;x) = e^{i\sqrt{z}x},\ \varepsilon_2(z;x) = e^{-i\sqrt{z}x} .
\end{equation}
The boundary conditions are parametrized by the $2\times 2$ matrices $A$ and $B$, see \eqref{abc_condition}.
We define
\begin{equation}\label{H_L}
  H_L\coloneqq\tilde H|_{\dom(H_L)},\
  \dom(H_L) \coloneqq \{ \varphi\in\dom(\tilde H) \mid (A\Gamma_1-B\Gamma_2)\varphi =0 \}
\end{equation}
and note some fundamental properties.

\begin{lemma}\label{H_L_spectrum}
The operator $H_L$ is self-adjoint. Its spectrum $\sigma(H_L)$ consists of eigenvalues
\begin{equation}\label{H_L_spectrum01}
  \sigma(H_L) = \{ \lambda_{j,L} \mid j\in\N\},\ \lambda_{j,L} \leq \lambda_{j+1,L}
\end{equation}
counted with multiplicity, which can be at most two. Furthermore,
\begin{equation}\label{H_L_spectrum02}
  \lambda_{j,L} \geq -c\Big(\frac{1}{L} + \frac{4}{c+1}\Big) + \frac{1}{c+1}\Big(\frac{\pi(j-1)}{2L}\Big)^2,\ j\in\N,
\end{equation}
with some constant $c\geq 0$ depending only on $A$ and $B$.
\end{lemma}
\begin{proof}
The choice $B=0$, i.e. Dirichlet boundary conditions, yields a self-adjoint operator.
We know from Section \ref{abc} that then all self-adjoint realizations are given via matrices $A$ and $B$ that
satisfy \eqref{abc_condition}.

To begin with, we show that $H_L$ is semi-bounded from below and look at the quadratic form
\begin{equation*}
  (\varphi,H_L\varphi)
     = - \int_{-L}^L \bar \varphi(x)\varphi''(x)\, dx
     = (\Gamma_1\varphi,\Gamma_2\varphi) + \int_{-L}^L |\varphi'(x)|^2\, dx,\ \varphi\in\dom(H_L) .
\end{equation*}
We show that
\begin{equation*}
  | (\Gamma_1\varphi,\Gamma_2\varphi) | \leq c ( |\varphi(L)|^2 + |\varphi(-L)|^2 ) ,\ c\geq 0 .
\end{equation*}
For $\rank(B)=0$ this is trivial with $c=0$ and for $\rank(B)=2$ this follows from $\Gamma_2\varphi = B^{-1}A\Gamma_1\varphi$.
The case $\rank(B)=1$ is slightly more difficult. We may choose $A$ and $B$ as
\begin{equation*}
  A =
\begin{pmatrix}
  a_{11} & a_{12} \\
  a_{21} & a_{22}
\end{pmatrix}
  ,\
  B = 
\begin{pmatrix}
  b_{11} & b_{12} \\
  0     & 0 
\end{pmatrix} 
  , \   |a_{21}|^2 + |a_{22}|^2 \neq 0 \neq |b_{11}|^2 + |b_{12}|^2,\   a_{21}\bar b_{11} + a_{22}\bar b_{12} = 0
\end{equation*}
where we used \eqref{abc_condition}. We write out the boundary condition and conclude
\begin{equation*}
\begin{pmatrix}
  \bar\varphi(L) \\ \bar\varphi(-L)
\end{pmatrix}
  = \beta
\begin{pmatrix}
  b_{11} \\ b_{12}
\end{pmatrix}
  ,\ |\beta|^2 = \frac{|\varphi(L)|^2 + |\varphi(-L)|^2}{|b_{11}|^2 + |b_{12}|^2} .
\end{equation*}
Using the boundary condition we obtain
\begin{equation*}
  (\Gamma_1\varphi,\Gamma_2\varphi) 
    = \beta ( - b_{11}\varphi'(L) + b_{12}\varphi'(-L) )
    = \beta ( a_{11}\varphi(L) + a_{12}\varphi(-L) )
\end{equation*}
which proves this case. All in all,
\begin{equation*}
  (\varphi,H_L\varphi) \geq -c ( |\varphi(L)|^2 + |\varphi(-L)|^2 ) + \|\varphi'\|^2 .
\end{equation*}
We estimate the boundary values. To this end,
\begin{equation*}
  |\varphi(L)|^2 = |\varphi(x)|^2 + \int_x^L \frac{d}{dy} |\varphi(y)|^2\, dy
                 = |\varphi(x)|^2 + 2\int_x^L \re(\bar\varphi(y)\varphi'(y))\, dy
                 \leq |\varphi(x)|^2 + 2\|\varphi\| \|\varphi'\| .
\end{equation*}
The other boundary value can be trated likewise. Integrating then yields
\begin{equation*}
  |\varphi(\pm L)|^2 \leq \frac{1}{2L}\|\varphi\|^2 + 2\|\varphi\|\|\varphi'\| 
                     \leq \frac{1}{2L}\|\varphi\|^2 + \frac{1}{\delta}\|\varphi\|^2 + \delta\|\varphi'\|^2
\end{equation*}
with some $\delta>0$. Thus,
\begin{equation*}
  (\varphi,H_L\varphi) 
    \geq -c ( \frac{1}{L}\|\varphi\|^2 + \frac{2}{\delta}\|\varphi\|^2 + 2\delta\|\varphi'\|^2) + \|\varphi'\|^2
     = -c(\frac{1}{L} + \frac{2}{\delta})\|\varphi\|^2 + (1-2c\delta)\|\varphi'\|^2 .
\end{equation*}
A convenient choice is $\delta\coloneqq\frac{1}{2(c+1)}$.
This shows the semi-boundedness. By the variational principle (see e.g. \cite[Thm. XIII.1]{ReedSimon1978})
\begin{equation*}
  \lambda_k \geq -c\Big(\frac{1}{L} + \frac{4}{c+1}\Big) + \frac{1}{c+1}\tilde\lambda_k,\ 
  \tilde\lambda_k=\Big(\frac{\pi(k-1)}{2L}\Big)^2,\ k\in\N,
\end{equation*}
where we used the variational characterization of the $\tilde\lambda_k$, the eigenvalues for Neumann boundary conditions, $A=0$. 
This proves \eqref{H_L_spectrum02} which in turn shows \eqref{H_L_spectrum01}. 
Since $H_L$ is a second order differential operator the eigenvalues' multiplicity can at most be two.
\end{proof}

We will thoroughly study the behaviour of the resolvent
\begin{equation}\label{R_L}
  R_L(z)\coloneqq(z\id- H_L)^{-1}:\hilbert_L\to\hilbert_L,\
  z\in\C\setminus\sigma(H_L) ,
\end{equation}
in the complex plane. 
We investigate the energy difference \eqref{energy_difference} for some fixed $\nu>0$ 
which we call Fermi energy. Because of $\sqrt{z}$ appearing in \eqref{green_infinity},
we choose in \eqref{energy01t01} an integration contour that is composed of parabolas 
(see Fig. \ref{f_fermi_parabola}). 
%%% Figure
%\begin{vchfigure}[b]
%\includegraphics[width=0.75\textwidth]{f_fermi_parabola.pdf}
%\vchcaption{The Fermi parbola in the complex plane.}
%\label{f_fermi_parabola}
%\end{vchfigure}
%%%
%%% Small figure
%\begin{figure}[b]
\begin{figure}[bth]
%\sidecaption
\includegraphics[width=0.75\textwidth]{f_fermi.pdf}
\caption{The Fermi parabola in the complex plane.}
\label{f_fermi_parabola}
\end{figure}
More precisely, for $b>0$ we define $\Gamma_{\nu,b} \coloneqq \Gamma_{\nu,b}^- \cup \Gamma_{\nu,b}^+$ with
\begin{equation}\label{fermi_parabola_b}
    \Gamma_{\nu,b}^- \coloneqq \{ z = (t+ib)^2\mid \sqrt{\nu}\geq t\geq -\sqrt{\nu}\},\
    \Gamma_{\nu,b}^+ \coloneqq \{ z = (\sqrt{\nu}+is)^2\mid -b\leq s\leq b\} .
\end{equation}
Furthermore, we define the Fermi parabola
\begin{equation}\label{fermi_parabola}
    \Gamma_\nu \coloneqq \Gamma_{\nu,\infty}^+ \coloneqq \{ z = (\sqrt{\nu}+is)^2\mid s\in\R \} .
\end{equation}
Let $V:\hilbert_L\to\hilbert_L$ be the multiplication operator
$(V\varphi)(x)\coloneqq V(x)\varphi(x)$, $\varphi\in\hilbert_L$. For simplicity, we assume $V$ to be
bounded, $V\in L^\infty(\R)$. Then, the perturbed operator
\begin{equation}\label{H_VL}
  H_{V,L}:\dom(H_{V,L})\to\hilbert_L,\ H_{V,L} \coloneqq H_L + V,\
\end{equation} 
is self-adjoint and $\dom(H_{V,L})=\dom(H_L)$. Its spectrum $\sigma(H_{V,L})$ consists of eigenvalues,
\begin{equation}\label{H_VL_spectrum}
  \sigma(H_{V,L}) = \{ \mu_{j,L} \mid j\in\N\},\ \mu_{j,L}\leq \mu_{j+1,L},\ \mu_{j,L}\to\infty\ \text{as} \ j\to\infty,
\end{equation}
counted with multiplicity, which can be at most two. This can be shown with the aid of Lemma \ref{H_L_spectrum}.

At times, the potential needs to fall off at infinity sufficiently fast which is expressed via
\begin{equation}\label{V_decay1}
  X^n V\in L^1(\R),\ (X^nV)(x) \coloneqq x^n V(x),\
  \|V\|_{1,n} \coloneqq \max_{k=0,\ldots,n} \|X^k V\|_1 ,\ n\in\N_0 .
\end{equation}
The limiting absorption principle in Lemma \ref{bso03t} requires a slightly more regular behaviour
expressed with the aid of the Birman--Solomyak condition (see e.g. \cite[pp. 38]{Simon2005})
\begin{equation}\label{birman_solomyak}
  \ell^q(L^1(\R)) \coloneqq \{ V \in L^1(\R)\mid \llbracket V\rrbracket_{1,q} < \infty\},\
  \llbracket V\rrbracket_{1,q} \coloneqq \sum_{j\in\Z} \| V\chi_{I_j} \|_1^q,\
  I_j\coloneqq [j,j+1],\ q>0 .
\end{equation}
Note that compactly supported potentials satisfy the Birman--Solomyak condition, i.e. $L^1_0(\R)\subset\ell^q(L^1(\R))$.
Due to the complex integration contour in the integral representation \eqref{energy01t01} 
of the energy difference we need a weighted $L^1$-norm (cf. \cite[(3.25)]{KuettlerOtteSpitzer2014})
\begin{equation}\label{V_L}
  \mathcal{V}_L(s) \coloneqq \int_{-L}^L |V(x)| e^{s|x|} \, dx,\ s\in\R .
\end{equation}
Note that $\mathcal{V}_L$ is differentiable with respect to $s$.
