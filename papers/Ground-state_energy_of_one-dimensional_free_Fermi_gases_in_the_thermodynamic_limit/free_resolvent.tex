We apply the results of Section \ref{abc} to the resolvents of $H_\infty$
and $H_L$, see \eqref{R_infinity} and \eqref{R_L}. First of all, we restrict $R_\infty(z)$ to $\hilbert_L$
which defines both operators on the same Hilbert space.
To this end let $\chi_L$ be the indicator function of $\Lambda_L$. We define
\begin{equation}\label{R_infinity_L}
  R_{\infty,L}(z) \coloneqq \chi_L R_\infty(z)\chi_L,\ R_{\infty,L}^\pm (\nu) \coloneqq \chi_L R_\infty^\pm(\nu)\chi_L ,
\end{equation}
which can be considered both an operator on $L^2(\Lambda_L)$ and on $L^2(\R)$. We use
the same symbol for these operators as the meaning should be clear from the context.
Note that $R_{\infty,L}(z)$ and particularly $R_{\infty,L}^\pm(\nu)$ are bounded operators 
while $R_\infty^\pm(\nu)$ are not since $\nu\in\sigma(H_\infty)$, cf. Lemma \ref{bso01t}.

The operator $R_{\infty,L}(z)$ and the resolvent $R_L(z)$ differ by a rank two operator as we show in Lemma \ref{fr01t}. 
A crucial role is played by the matrix
\begin{equation}\label{fr_bcs}
  U(z) \coloneqq (iA-\sqrt{z}B)^{-1}(iA+\sqrt{z}B)
\end{equation}
with the matrices $A$ and $B$ describing the boundary conditions, see \eqref{H_L} and \eqref{abc_condition}.
In the context of quantum wires, this matrix $U(z)$ is the scattering matrix if $\sqrt{z}$ is real, see \cite[Thm. 2.1]{KostrykinSchrader1999}. 
Since it enters through the boundary conditions we name it \emph{boundary condition scattering matrix}. 
It is studied in more detail in Section \ref{bcs}.

\begin{lemma}\label{fr01t}
For $z\in\C\setminus\sigma(H_L)$ the resolvent $R_L(z)$ of $H_L$ can be decomposed into (cf. \eqref{deficiency_subspace})
\begin{equation}\label{fr01t01}
  R_L(z) = R_{\infty,L}(z) - D_L(z),\ 
  D_L(z) = \frac{1}{2i\sqrt{z}}d_L(z)\sum_{j,k=1}^2 g_{jk}(z) ( \varepsilon_k(\bar z), \cdot )\varepsilon_j(z) .
\end{equation}
Note that $D_L(\bar{z})=(D_L(z))^*$.
The coefficient matrix $G_L(z)\coloneqq (g_{jk}(z))_{j,k=1,2}$ and the scalar prefactor are given through
\begin{equation}\label{fr01t02}
  G_L(z) = ( e^{2i\sqrt{z}L}\id + U(z)\sigma_x )^{-1},\
  d_L(z) \coloneqq e^{2i\sqrt{z}L} ,\
  \im(\sqrt{z})\geq 0,
\end{equation}
with the Pauli matrix $\sigma_x$.
\end{lemma}
\begin{proof}
Since $R_{\infty,L}(z)$ satisfies the resolvent properties \eqref{abc02t01}
(with $\tilde H$ the maximal operator from \eqref{H_maximal}) we may use Proposition \ref{abc02t}.
Computing $\Gamma_{1,2}\varepsilon_{1,2}(z)$ yields the matrix representations
\begin{equation*}
  \Gamma_1|_{\mathcal{N}_z} = e^{i\sqrt{z}L}\id + e^{-i\sqrt{z}L}\sigma_x,\
  \Gamma_2|_{\mathcal{N}_z} = -i\sqrt{z} ( e^{i\sqrt{z}L}\id - e^{-i\sqrt{z}L}\sigma_x )
\end{equation*}
and thus
\begin{equation*}
\begin{split}
  (A\Gamma_1-B\Gamma_2)|_{\mathcal{N}_z}
    & = A(e^{i\sqrt{z}L}\id+e^{-i\sqrt{z}L}\sigma_x) +i\sqrt{z} B(e^{i\sqrt{z}L}\id-e^{-i\sqrt{z}L}\sigma_x) \\
    & = e^{i\sqrt{z}L}(A+i\sqrt{z}B) + e^{-i\sqrt{z}L}(A-i\sqrt{z}B)\sigma_x .
\end{split}
\end{equation*}
Using the formula for the Green function in \eqref{green_infinity} we obtain
\begin{equation*}
  \big[ \Gamma_1R_{\infty,L}(z)\varphi \big]_j = - \frac{i}{2\sqrt{z}}e^{i\sqrt{z}L} (\varepsilon_j(\bar z), \varphi)
  ,\
  \big[ \Gamma_2R_{\infty,L}(z)\varphi \big]_j = -\frac{1}{2} e^{i\sqrt{z}L} (\varepsilon_j(\bar z), \varphi)
  ,\ j=1,2 .
\end{equation*}
Recall, $E(z)_{jk} = (\varepsilon_j(\bar z),\varepsilon_k(z))$. Then, taking $\varphi=\varepsilon_{1,2}(z)$ yields
\begin{equation*}
  \Gamma_1R_{\infty,L}(z)|_{\mathcal{N}_z} = -\frac{i}{2\sqrt{z}}e^{i\sqrt{z}L} E(z),\
  \Gamma_2R_{\infty,L}(z)|_{\mathcal{N}_z} = -\frac{1}{2}e^{i\sqrt{z}L} E(z) .
\end{equation*}
Finally, the equation \eqref{abc02t03} for $\hat D_L(z)$ becomes
\begin{equation*}
  -\frac{1}{2}e^{i\sqrt{z}L} ( \frac{i}{\sqrt{z}}A-B)E(z)
    = ( e^{i\sqrt{z}L}(A+i\sqrt{z}B) + e^{-i\sqrt{z}L}(A-i\sqrt{z}B)\sigma_x ) \hat D_L(z)E(z) .
\end{equation*}
Computing the scalar products $(\varepsilon_j(\bar z),\varepsilon_k(z))$, $j,k=1,2$, we obtain
\begin{equation*}
  E(z) = 2L\id + \frac{1}{\sqrt{z}}\sin(2L\sqrt{z}) \sigma_x
       = 2L ( \id + \frac{1}{w}\sin(w)\sigma_x ),\ w\coloneqq 2L\sqrt{z} .
\end{equation*}
Obviously, $E(z)$ is not invertible if
\begin{equation*}
  0 = \det( \id + \frac{1}{w}\sin(w)\sigma_x ) = 1 - \frac{1}{w^2}\sin^2(w) .
\end{equation*}
Since the right-hand side is an entire holomorphic function of $w$ there are at most countably many solutions 
without any point of accumulation (for more details see \cite{BurnistonSiewert1973a}).
Except for those points we may solve for $\hat D_L(z)$
\begin{equation*}
\begin{split}
  \hat D_L(z) & = -\frac{i}{2\sqrt{z}}( iA-\sqrt{z}B + e^{-2i\sqrt{z}L}(iA+\sqrt{z}B)\sigma_x)^{-1}(iA-\sqrt{z}B)\\
         & = -\frac{i}{2\sqrt{z}} e^{2i\sqrt{z}L} ( e^{2i\sqrt{z}L} \id + U(z)\sigma_x )^{-1}
\end{split}
\end{equation*}
which yields \eqref{fr01t01} and \eqref{fr01t02}. The matrix $G_L(z)$ exists for all $z\in\C\setminus\sigma(H_L)$, 
see Lemma \ref{bcs_special01t}.

Finally, since $R_L(\bar{z}) = (R_L(z))^*$ and $R_{\infty,L}(\bar{z}) = (R_{\infty,L}(z))^*$ we see that $D_L(\bar{z})=(D_L(z))^*$.
\end{proof}
