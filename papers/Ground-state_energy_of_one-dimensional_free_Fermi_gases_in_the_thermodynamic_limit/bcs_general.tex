We consider a somewhat more general situation by allowing for $n\times n$-matrices
instead of the $2\times 2$-matrix in \eqref{fr_bcs}. Also, $\sqrt{z}$ may be replaced by
$w\in\C$. We start with showing that the inverse in \eqref{fr_bcs} exists.

\begin{lemma}\label{bcs_general01t}
The matrix $iA+w B$ is invertible for all $w\in\C$ with $\re(w)\neq 0$ and for all $w\in\C$, $\re(w)=0$,
except a finite number of points $w_1,\ldots,w_m$, $m\leq n$.
\end{lemma}
\begin{proof}
The matrix $iA+w B$ is invertible if and only if its adjoint is invertible.
Let $(iA+wB)^*\varphi=0$. We compute
\begin{equation*}
\begin{split}
  (iA+w B)(iA+w B)^*
    & = AA^* + |w|^2BB^* +i\re(w)(AB^* - BA^*) + \im(w) (AB^*+BA^*) \\
    & = (A+\im(w)B)(A^*+\im(w)B^*) + \re(w)^2 BB^* .
\end{split}
\end{equation*}
Since $\re(w)\neq 0$ we conclude
\begin{equation*}
  (A^* + \im(w) B^*)\varphi = 0,\ B^*\varphi = 0 .
\end{equation*}
Lemma \ref{abc01t} implies that $\varphi=0$. In particular, $\det(iA+wB)\neq 0$ for $\re(w)\neq 0$.
Obviously, the determinant is a polynomial in $w$ of degree at most $n$. Since it does not vanish identically 
there can be at most $n$ zeros.
\end{proof}

The conditions on $A$ and $B$ imply some nice properties of $U$.

\begin{lemma}\label{bcs_general02t}
The matrix $U(w)\coloneqq (iA-wB)^{-1}(iA+wB)$ has an eigenvalue $-1$ if $A$ has an eigenvalue $0$. Likewise,
$U(w)$ has an eigenvalue $1$ if $B$ has an eigenvalue $0$. The respective eigenvectors can be chosen independently of $w$.
Furthermore, $U(w_1)U(w_2)^*=U(w_2)^*U(w_1)$ for all $w_1,w_2\in\C\setminus\{0\}$ and $w_1+\bar w_2\neq 0$. 
In particular, $U(w)$ is normal for $\re(w)\neq 0$. If, in addition, $\im(w)=0$ then $U(w)$ is unitary.
\end{lemma}
\begin{proof}
By Lemma \ref{bcs_general01t} the matrix $iA-wB$ is invertible. We may therefore write
\begin{equation*}
  U(w) = -\id + 2i(iA-wB)^{-1}A = \id + 2w(iA-wB)^{-1}B .
\end{equation*}
This shows the statement on the eigenvalues $\pm 1$.
The eigenvectors of $A$ and $B$ can, obviously, be chosen independently of $w$.
Using \eqref{abc_condition} we compute
\begin{equation*}
\begin{split}
  w_1(U(w_1)+\id)(U(w_2)-\id)^*
   & = 4iw_1\bar w_2 (iA-w_1B)^{-1}AB^*(-iA^*-\bar w_2B^*)^{-1}\\
   & = 4iw_1\bar w_2 (iA-w_1B)^{-1}BA^*(-iA^*-\bar w_2B^*)^{-1}\\
   & = -\bar w_2(U(w_1)-\id)(U(w_2)+\id)^*\
\end{split}
\end{equation*}
which yields
\begin{equation*}
  U(w_1)U(w_2)^* - cU(w_1)+ cU(w_2)^* = \id,\ c\coloneqq\frac{w_1-\bar w_2}{w_1+\bar w_2}
\end{equation*}
and furthermore
\begin{equation*}
  ( U(w_1)+c\id )( U(w_2)^*-c\id ) = (1-c^2)\id .
\end{equation*}
For $c^2\neq 1$ we infer that $U(w_1)+c\id$ is a left-inverse and therefore a right-inverse 
as well since we are working with matrices. Thus
\begin{equation*}
  ( U(w_2)^* - c\id )( U(w_1) + c\id ) = (1-c^2)\id .
\end{equation*}
This implies that $U(w_1)$ and $U(w_2)^*$ commute if $c\neq \pm 1$ or, equivalently, $w_{1,2}\neq 0$.
$U(w)$ is unitary if $c=0$ which is equivalent to $\im(w)=0$.
\end{proof}
