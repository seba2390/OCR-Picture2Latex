We define the wave operators (cf. \eqref{Omega_K}) corresponding to \eqref{K_infinity}
\begin{equation}\label{Omega}
  \Omega_{\infty,L}(z)  \coloneqq (\id -K_{\infty,L}(z))^{-1} : \hilbert_L\to\hilbert_L,\
  \Omega_\infty(z)     \coloneqq (\id -K_\infty(z))^{-1} : \hilbert_\infty\to\hilbert_\infty
\end{equation}
and their boundary values (see \eqref{K_boundary_values})
\begin{equation}\label{Omega_boundary_values}
  \Omega_{\infty,L}^\pm(\nu)  \coloneqq (\id -K_{\infty,L}^\pm(\nu))^{-1},\
  \Omega_\infty^\pm(\nu) \coloneqq (\id -K_\infty^\pm(\nu))^{-1} .
\end{equation}
First of all, we have to ensure that the wave operators exist.

\begin{lemma}\label{wo01t}
Let $V\in L^1(\R)$. Then the following hold true.
\begin{enumerate}
\item 
The wave operator $\Omega_\infty(z)$ exists for all $z\in\C\setminus\interval[open right]{-\frac{1}{4}\|V\|_1^2}{\infty}$.
\item
If, in addition, $V\in\ell^{\frac{1}{2}}(L^1(\R))$ and $X^2V\in\ell^{\frac{1}{2}}(L^1(\R))$ then
$\Omega_\infty(z)$ exists for all $z\in\C\setminus[-\frac{1}{4}\|V\|_1^2, 0]$ and is bounded.
Equivalently, $\det(\id-K_\infty(z))\neq 0$. Moreover, for every compact $D\subset\C\setminus[-\frac{1}{4}\|V\|_1^2, 0]$
\begin{equation}\label{wo01t01}
  \inf_{z\in D} |\det(\id-K_\infty(z))| > 0 .
\end{equation}
\end{enumerate}
\end{lemma}
\begin{proof}
1.
First note that the Birman--Schwinger operator $K_\infty(z)$ is bounded for $z\neq 0$, Lemma \ref{bso01t}.
We distinguish two cases.

(i)
Let $\im(z)\neq 0$. Then $z\notin\sigma(H)\cup\sigma(H_V)$. 
By Lemma \ref{energy_omega} the inverse $\Omega_\infty(z)$ exists and is bounded.

(ii)
Let $\im(z)=0$ and $\re(z)<-\frac{1}{4}\|V\|_1^2$. From Jensen's inequality \eqref{jensen_inequality}
and \eqref{bso01t01}
\begin{equation*}
   \|K_\infty(z)\| \leq \|K_\infty(z)\|_2 \leq \frac{1}{2|\sqrt{z}|}\|V\|_1 < 1 .
\end{equation*}
A Neumann series argument then shows that $\Omega_\infty(z)$ exists and is bounded.

Both in case (i) and (ii) we have $\im(\sqrt{z})\neq 0$ whence $K_\infty(z)\in B_1(\hilbert)$ 
by Lemma \ref{bso02t}. Therefore, the perturbation determinant is well-defined and non-zero
(cf. \cite[Thm. XIII.105]{ReedSimon1978}). 

2.
Assume the Birman--Solomyak condition on $V$ whereby $K_\infty(z)\in B_1(\hilbert)$ for $z\in\interval[open]{0}{\infty}$,
Lemma \ref{bso03t}, and the perturbation determinant is well-defined.
We infer from the Jost--Pais formula (see \eqref{jost_pais01} and \eqref{jost_pais02}) that $|\det(\id-K_\infty(z))| \geq 1$.
Thus, the determinant cannot vanish and the inverse $\Omega(z)$ exists and is bounded 
(cf. \cite[Thm. XIII.105, Thm. XIII.107]{ReedSimon1978}).
By Lemma \ref{bso03t}, the map $K_\infty(z)$ depends continuously on $z$ with respect to $B_1(\hilbert)$ and so does
the determinant $\det(\id-K_\infty(z))$. This implies \eqref{wo01t01}
\end{proof}

The wave operator $\Omega_{\infty,L}(z)$ requires weaker conditions.

\begin{lemma}\label{wo01at}
Let $V\in L^1(\R)$. Then $\Omega_{\infty,L}(z)$ exists for all $z\in\C\setminus[-\frac{1}{4}\|V\|_1^2, 0]$ and is bounded.
\end{lemma}
\begin{proof}
$K_{\infty,L}(z)$ is the Birman--Schwinger operator for the potential $V\chi_L$, which has compact
support and, thus, satisfies the Birman--Solomyak conditions in Lemma \ref{wo01t}.
\end{proof}

In studying the energy difference with the aid of Lemma \ref{wo03t} we need the wave operators restricted, essentially,
to the deficiency subspace (see \eqref{deficiency_subspace}). 
By \ref{Omega_T} this leads to the T-matrix
\begin{equation}\label{Omega_matrix}
  2i\sqrt{z}T_L(z) = ( (\sqrt{|V|}\varepsilon_j(\bar z),J\Omega_{\infty,L}(z)\sqrt{|V|}\varepsilon_k(z)) )_{j,k=1,2},\
   \im(\sqrt{z})\geq 0 .
\end{equation}
For the infinite volume operators this is, in general, true only for $z\in\R$,
\begin{equation*}
  2i\sqrt{\nu}T(\nu) = ( (\sqrt{|V|}\varepsilon_j(\nu),J\Omega_\infty^+(\nu)\sqrt{|V|}\varepsilon_k(\nu)) )_{j,k=1,2}.
\end{equation*}
For the T-matrices and S-matrices see \eqref{Omega_T} and \eqref{s-matrix02}. Note that $z=k^2$ and $S(z)=\mathscr{S}(k)$,
this convention being used for all scattering data.

\begin{lemma}\label{wo02t}
Let $V\in L^1(\R)$. Then, the following hold true.
\begin{enumerate}
\item 
Define $V_-$ as $V_-(x) \coloneqq \min\{ V(x) ,0\}$. Let $z=(\sqrt{\nu}+is)^2$ with $\nu>0$ and $s\geq0$. Then,
\begin{equation} \label{wo02t01}
  \| S_L(z) \|_2 \leq  \sqrt{2}\Big\{ 1+ \frac{1}{2|\sqrt{z}|}\mathcal{V}_L(2s) \exp\Big[ \frac{1}{|\sqrt{z}|}\|V\|_1 \Big] \Big\}
                        \exp\Big[ \frac{1}{2\nu} ( \| V_-\|_1^2 + \sqrt{\nu} \| V_-\|_1 )\Big] .
\end{equation}
Let $z=(t+ib)^2$ with $b\geq \|V_-\|_1$. Then,
\begin{equation}\label{wo02t01a}
  \|T_L(z) \|_2  \leq \frac{\sqrt{2}}{2|\sqrt{z}|}\big[ \|V\|_1 + \mathcal{V}_L(2b) \big]\exp\Big[ \frac{1}{|\sqrt{z}|}\|V\|_1 \Big]
                  \exp\big[ \frac{2}{b}\|V_-\|_1 + \frac{2}{b^2}\|V_-\|_1^2\big] .
\end{equation}
\item
The map $\Gamma_\nu^+\ni z \mapsto S_L(z)$ is continuous at the Fermi energy
\begin{equation}\label{wo02t02}
  \| S_L(z) - S_L(\nu) \|_2 = \| T_L(z) - T_L(\nu) \|_2 \leq C s
\end{equation}
\item
The scattering matrix $S_L(\nu)$ converges to $S(\nu)$ as $L\to\infty$. More precisely,
\begin{equation}\label{wo02t03}
  \| S_L(\nu) - S(\nu) \|_2 = \| T_L(\nu) - T(\nu) \|_2
     \leq \Big(\frac{5}{2}\Big)^{\frac{1}{2}}\frac{1}{\sqrt{\nu}} \int_{|x|\geq L} |V(x)|\, dx \exp\big[\frac{1}{\sqrt{\nu}}\|V\|_1\big].
\end{equation}
\end{enumerate}
\end{lemma}
\begin{proof}
It is convenient to introduce
\begin{equation*}
  s_{0,L}(z) \coloneqq (e_1(\bar z),\chi_L Jf_1(z)),\ 
  s_{1,L}(z) \coloneqq (e_1(\bar z),\chi_L Jf_2(z)),\ s_{2,L}(z) \coloneqq (e_2(\bar z),\chi_LJf_1(z))
\end{equation*}
and the matrix
\begin{equation*}
  \tilde S_L(z) \coloneqq
\begin{pmatrix}
  s_{0,L}(z) & s_{1,L}(z) \\
  s_{2,L}(z) & s_{0,L}(z)
\end{pmatrix} .
\end{equation*}
The relations
\begin{equation}\label{wo02t04}
  r_{j,L}(z) = \frac{1}{2i\sqrt{z}} s_{j,L}(z) t_L(z),\ j=1,2,\
  T_L(z) =  \frac{t_L(z)}{2i\sqrt{z}}\tilde S_L(z)
\end{equation}
follow easily from \eqref{s-matrix01}.

1.
From the Faddeev--Deift--Trubowitz formula \eqref{deift_trubowitz01} we infer that
\begin{equation}\label{deift_trubowitz02}
  |t_L(z)| \leq \prod_{j=1}^n \frac{|u+i(v+\beta_j)|}{|u+i(v-\beta_j)|},\ v\geq 0 .
\end{equation}
We find uniform bounds. Firstly, we maximize with respect to $v$ thereby obtaining
\begin{equation*}
  \frac{|u+i(v+\beta_j)|^2}{|u+i(v-\beta_j)|^2}
    \leq 1 + \frac{2}{u^2} ( \beta_j^2 + \beta_j\sqrt{u^2+\beta_j^2} ) 
    \leq 1 + \frac{2}{u^2} ( 2 \beta_j^2 + \beta_j u ).
\end{equation*}
Secondly, the maximizing with respect to $u$ yields for $v\geq 2\beta_j$, $j=1,\ldots,n$,
\begin{equation*}
  \frac{|u+i(v+\beta_j)|^2}{|u+i(v-\beta_j)|^2}
     = 1 + \frac{4v\beta_j}{u^2+(v-\beta_j)^2}
     \leq 1 + \frac{4\beta_j}{v-\beta_j} + \frac{4\beta_j^2}{(v-\beta_j)^2}
     \leq 1 + \frac{8\beta_j}{v} + \frac{16\beta_j^2}{v^2} .
\end{equation*}
Therefore,
\begin{equation*}
  |t_L(z)| \leq \exp\Big[ \frac{1}{u^2} \sum_{j=1}^n ( 2 \beta_j^2 + \beta_j u ) \Big] ,\
  |t_L(z)| \leq \exp\Big[ \frac{4}{v}\sum_{j=1}^n \beta_j + \frac{8}{v^2}\sum_{j=1}^n \beta_j^2 \Big] .
\end{equation*}
The sums of the $\beta_j$ can be estimated with the aid of a Lieb--Thirring inequality 
(see \cite{Weidl1996}, \cite{HundertmarkLiebThomas1998})
\begin{equation*}
  \big[\sum_{j=1}^n \beta_j^2\big]^{\frac{1}{2}} 
    \leq \sum_{j=1}^n \beta_j 
    \leq \frac{1}{2} \int_\R \chi_L(x) |V_-(x)| \, dx \leq \frac{1}{2} \| V_- \|_1 .
\end{equation*}
The first sum could also be treated directly via an appropriate Lieb--Thirring inequality. However, that
would require an additional condition on $V$. Now,
\begin{equation*}
  |t_L(z)| \leq \exp\big[ \frac{1}{2u^2} ( \| V_-\|_1^2 + u \| V_-\|_1 )\big] ,\
  |t_L(z)| \leq \exp\big[ \frac{2}{v}\|V_-\|_1 + \frac{2}{v^2}\|V_-|_1^2\big] .
\end{equation*}
Using \eqref{stl01t02} we obtain
\begin{gather*}
  |s_{0,L}(z)| \leq \int_\R  |V(x)|\chi_L(x)\, dx \exp\big[ \frac{1}{|\sqrt{z}|}\|V\|_1 \big]
              \leq \|V\|_1 \exp\big[ \frac{1}{|\sqrt{z}|}\|V\|_1 \big], \\
  |s_{j,L}(z)|
    \leq \int_\R e^{-2x\im(\sqrt{z})} |V(x)|\chi_L(x)\, dx \exp\big[ \frac{1}{|\sqrt{z}|}\|V\|_1 \big]
    \leq \mathcal{V}_L(2\im(\sqrt{z})) \exp\big[ \frac{1}{|\sqrt{z}|}\|V\|_1 \big] ,\ j=1,2,
\end{gather*}\
cf. \eqref{V_L}. Finally, put $u=\sqrt{\nu}$ in the first estimate of $t_L(z)$. Then, the first relation in \eqref{wo02t04} 
yields \eqref{wo02t01}. For \eqref{wo02t01a} put $v=b$ in the second estimate of $t_L(z)$ 
and use the second relation in \eqref{wo02t04}.
Note that $\beta_j\leq\frac{1}{2}\|V_-\|_1$ which is a trivial consequence of the Lieb--Thirring inequality.

2.
Using \eqref{stl03t01} we obtain via the mean value theorem
\begin{equation*}
  |m_j(\sqrt{\nu}+is;x) - m_j(\sqrt{\nu};x)| 
    \leq s \frac{2}{\sqrt{\nu}} \exp\big[ \frac{2}{\sqrt{\nu}} \|V\|_1\big] \times
\begin{cases}
   \int_x^\infty (y-x)|V(y)|\, dy & \text{for}\ j=1 , \\
   \int_{-\infty}^x (x-y)|V(y)|\, dy & \text{for}\ j=2 ,
\end{cases}
\end{equation*}
and thus
\begin{align*}
  |s_{0,L}(z) - s_{0,L}(\nu)|
     & \leq s \|XV\|_1\|V\|_1 \frac{2}{\sqrt{\nu}}\exp\big[ \frac{2}{\sqrt{\nu}}\|V\|_1\big] , \\
  |s_{j,L}(z) - s_{j,L}(\nu)| 
     & \leq s \Big\{ \mathcal{V}_L'(2s)\|V\|_1 + \mathcal{V}_L(2s)\|XV\|_1\Big\} \frac{2}{\sqrt{\nu}}\exp\big[\frac{2}{\sqrt{\nu}}\|V\|_1\big] ,\
        j=1,2.
\end{align*}
One can check straightforwardly that
\begin{equation*}
  S_L(z) - S_L(\nu)
    = \frac{t_L(z)}{2i\sqrt{z}}
        \Big\{ \tilde S_L(z) - \tilde S_L(\nu) + \big[ s_{0,L}(z)-s_{0,L}(\nu) - 2i(\sqrt{z}-\sqrt{\nu})\big] T_L(\nu) \Big\}
\end{equation*}
which implies \eqref{wo02t02}.

3.
For $S(\nu)$ define $\tilde S(\nu)$ analogously to $\tilde S_L(\nu)$.
Using  \eqref{stl01t02} we obtain easily
\begin{equation*}
   |s_0(\nu) - s_{0,L}(\nu)|
     = \Big|\frac{1}{2i\sqrt{\nu}} \int_{|x|\geq L} V(x) e^{-i\sqrt{\nu}x}f_1(\nu;x)\, dx\Big|
     \leq \frac{1}{2\sqrt{\nu}} \int_{|x|\geq L} |V(x)| \, dx \exp\big[ \frac{1}{\sqrt{\nu}} \|V\|_1\big]
\end{equation*}
and likewise for $j=1,2$
\begin{equation*}
  |s_j(\nu) - s_{j,L}(\nu)|
     \leq \frac{1}{2\sqrt{\nu}} \int_{|x|\geq L} |V(x)| \, dx \exp\big[ \frac{1}{\sqrt{\nu}} \|V\|_1\big] .
\end{equation*}
The simple relation
\begin{equation*}
  S(\nu) - S_L(\nu)
    = \frac{t(\nu)}{2i\sqrt{\nu}}[ \tilde S(\nu) - \tilde S_L(\nu) + (s_0(\nu)-s_{0,L}(\nu))(S_L(\nu)-\id) ]
\end{equation*}
then yields \eqref{wo02t03}.
\end{proof}

For a finite rank operator $D$ the Fredholm determinant of $\id-D$ reduces to a usual determinant.

\begin{lemma}\label{determinant_finite_rank}
Let $D:\hilbert\to\hilbert$ be an operator of finite rank $n\in\N$ and let $\hat D$ be the corresponding 
$n\times n$ Gram matrix, i.e.
\begin{equation}\label{determinant_finite_rank01}
   D = \sum_{j=1}^n (g_j,\cdot)f_j,\ \hat D\coloneqq((g_j,f_k))_{j,k=1,\ldots,n}
\end{equation}
where $f_j,g_k\in\hilbert$. Then $\sigma(D)\setminus\{0\}=\sigma(\hat D)\setminus\{0\}$.
In particular,
\begin{equation}\label{determinant_finite_rank02}
   \det (\id-D) = \det (\id - \hat D) .
\end{equation}
\end{lemma}
\begin{proof}
Let $\lambda\neq 0$ be an eigenvalue of $D$, i.e. $D\varphi=\lambda\varphi$ for some $\varphi\neq 0$.
Explicitly,
\begin{equation*}
   \lambda\varphi = \sum_{k=1}^n (g_k,\varphi) f_k .
\end{equation*}
We conclude $c\coloneqq( (g_1,\varphi),\ldots,(g_n,\varphi))^T\neq 0$ since otherwise we had $\lambda=0$.
Now, one easily checks $\hat D c=\lambda c$. Conversely, writing out the eigenvalue equation $\lambda c = \hat D c$,
\begin{equation*}
  \lambda c_j = \sum_{k=1}^n (g_j,f_k)c_k = (g_j, \sum_{k=1}^n c_k f_k) = (g_j,\varphi),\ \varphi\coloneqq\sum_{k=1}^n c_kf_k,
\end{equation*}
shows that $\varphi\neq 0$. Now, by simple algebra $D\varphi=\lambda\varphi$.

In the determinants $\det(\id-D)$ and $\det(\id-\hat D)$ a possible eigenvalue $0$ of $D$ or $\hat D$
does not matter which proves \eqref{determinant_finite_rank02}.
\end{proof}

By now we have introduced all objects so that we can finally present the
factorization of the perturbation determinant (cf. \eqref{perturbation_determinant}).

\begin{lemma}\label{wo03t}
The following hold true.
\begin{enumerate}
\item
We have the factorization
\begin{equation}\label{wo03t00}
  \id - K_L(z) = (\id - K_{\infty,L}(z))(\id + \Omega_{\infty,L}(z)\sqrt{|V|}D_L(z)\sqrt{|V|}J) .
\end{equation}
For $\im(z)\neq 0$, the operator $\Omega_{\infty,L}(z)\sqrt{|V|}D_L(z)\sqrt{|V|}J$ has no spectral values
in $\interval[open left]{-\infty}{-1}$.
\item
Let $T_L(z)$ be the T-matrix for the potential $V\chi_L$. Then,
\begin{equation}\label{wo03t01}
  \det(\id-K_L(z))
      = \det(\id-K_{\infty,L}(z)) \det(\id+d_L(z) T_L(z)G_L(z)) .
\end{equation}
Note that $\det(\id-K_L(\bar{z})) = (\det(\id-K_L(z)))^*$, and similarly for the two other determinants on the right-hand side.

For $\im(z)\neq 0$ the $2\times 2$-matrix $d_L(z)T_L(z)G_L(z)$ does not have eigenvalues in $\interval[open left]{-\infty}{-1}$.
\item
With the S-matrix, $S_L(z)=\id+T_L(z)$, we have
\begin{equation}\label{wo03t02}
  \id+d_L(z) T_L(z)G_L(z) = (e^{2iL\sqrt{z}}S_L(z) + U(z)\sigma_x)(e^{2iL\sqrt{z}}\id + U(z)\sigma_x)^{-1} .
\end{equation}
\end{enumerate}
\end{lemma}
\begin{proof}
1.
The decomposition \eqref{fr01t01} in Lemma \ref{fr01t} implies
\begin{equation*}
  \id - K_L(z) = \id - K_{\infty,L}(z) + \sqrt{|V|}D_L(z)\sqrt{|V|}J
               = (\id - K_{\infty,L}(z)) (\id + \Omega_{\infty,L}(z)\sqrt{|V|}D_L(z)\sqrt{|V|}J) .
\end{equation*}
Since $\Omega_{\infty,L}(z)\sqrt{|V|}D_L(z)\sqrt{|V|}J$ is a rank two operator it has only eigenvalues. 
One eigenvalue may be $\varkappa=0\notin\interval[open left]{-\infty}{-1}$. If $\im(z)\neq 0$ and $\im(\varkappa)\neq 0$
the statement is true as well. Hence, the critical case is $\im(z)\neq 0$ and $\im(\varkappa)=0$.
Let $\varphi\neq 0$ such that
\begin{equation*}
  \Omega_{\infty,L}(z)\sqrt{|V|}D_L(z)\sqrt{|V|}J\varphi = \varkappa\varphi .
\end{equation*}
Via the decomposition \eqref{fr01t01} this implies
\begin{equation*}
  ( \varkappa\id + \sqrt{|V|}R_L(z)\sqrt{|V|} )J\varphi = (\varkappa+1)\sqrt{|V|}R_{\infty,L}(z)\sqrt{|V|}J\varphi .
\end{equation*}
We multiply by $J$, take scalar products
\begin{equation*}
  \varkappa (\varphi,J\varphi) + (\varphi,J\sqrt{|V|}R_L(z)\sqrt{|V|}J\varphi)
      = (\varkappa+1)(\varphi,J\sqrt{|V|}R_{\infty,L}(z)\sqrt{|V|}J\varphi) ,
\end{equation*}
and look at the imaginary part
\begin{multline*}
  \im(\varkappa) (\varphi,J\varphi) - \im(z) \|R_L(z)\sqrt{|V|}J\varphi\|^2\\
     = \im(\varkappa)\re((\varphi,J\sqrt{|V|}R_{\infty,L}(z)\sqrt{|V|}J\varphi)) 
      - (\re(\varkappa)+1)\im(z) \|R_{\infty,L}(z)\sqrt{|V|}J\varphi\|^2 .
\end{multline*}
By the assumption on $z$ and $\varkappa$ this implies
\begin{equation*}
  \| R_L(z)\sqrt{|V|}J\varphi\|^2 = (\re(\varkappa)+1) \|R_{\infty,L}(z)\sqrt{|V|}J\varphi\|^2 .
\end{equation*}
The norms do not vanish since that would imply $\sqrt{|V|}J\varphi=0$ and furthermore $\varkappa\varphi=0$ 
by the eigenvalue equation which contradicts $\varkappa\neq 0$ and $\varphi\neq 0$. We conclude
that $\re(\varkappa)+1>0$ which proves the statement on the spectral values.

2.
We take determinants and apply Lemma \ref{determinant_finite_rank} to the finite rank operator 
$\Omega_{\infty,L}(z)\sqrt{|V|}D_L(z)\sqrt{|V|}J$ thereby obtaining
\begin{equation*}
\begin{split}
  \det(\id- K_L(z))
     & = \det(\id-K_{\infty,L}(z)) \det(\id+\Omega_{\infty,L}(z)\sqrt{|V|}D_L(z)\sqrt{|V|}J)\\
     & = \det(\id-K_{\infty,L}(z)) \det(\id+\tilde\Omega_L(z)) .
\end{split}
\end{equation*}
The matrix $\tilde\Omega_L(z)$ has the entries, $j,k=1,2$,
\begin{equation*}
  (\tilde\Omega_L(z))_{jk}
     = \frac{1}{2i\sqrt{z}}d_L(z)\sum_{l=1}^2 (\sqrt{|V|}J\varepsilon_j(\bar z),\Omega_{\infty,L}(z)\sqrt{|V|}\varepsilon_l(z))g_{lk}(z)
     = \frac{1}{2i\sqrt{z}}d_L(z)(\hat\Omega_{\infty,L}(z)G_L(z))_{jk} .
\end{equation*}
This proves \eqref{wo03t01}. Apart from $0$ the matrix $\tilde\Omega_L(z)$ has the same eigenvalues
as the rank two operator in part one, see \ref{determinant_finite_rank}. 

3.
Finally, \eqref{wo03t02} is straightforward to prove.
\end{proof}
