We collect some properties of the spectral shift function 
(see e.g. \cite[Ch.~8]{Yafaev1992}, \cite[Ch.~0~\S~9]{Yafaev2010}).
As our definition we use
\begin{equation}\label{krein_xi}
  \xi(\nu) \coloneqq \frac{1}{\pi} \lim_{y\to +0} \im\ln[\Delta(\nu+iy)] ,
\end{equation}
which is also known as Kre\u\i{}n's formula. If $H$ and $H_V$ are semi-bounded from below and
have no essential spectrum then the spectral shift function can be expressed by Lifshitz's formula
\begin{equation}\label{lifshitz}
  \xi(\nu) = -\tr(\Pi_\nu - P_\nu),\ \nu\in\R .
\end{equation}
Note that in general the difference of the spectral projections is not trace class.

\begin{lemma}\label{ssf01t}
Let $R_V(z)-R(z)\in B_1(\hilbert)$ for $z\notin\sigma(H_V)\cup\sigma(H)$. Then the following hold true.
\begin{enumerate}
\item $\xi\in L^1(\R; (1+|\lambda|)^{-2})$.
\item Assume in addition that there are constants $C\geq 0$ and $\alpha>0$ such that
for some $\re(z)$ and all sufficiently large $\im(z)>0$
\begin{equation*}
  \| R_V(z) - R(z) \|_1 \leq \frac{C}{\im(z)^\alpha} .
\end{equation*}
Then $\xi\in L^1(\R; (1+|\lambda|)^{-\tau-1})$ for all $\tau > \max\{-1, 1-\alpha\}$.
\end{enumerate}
\end{lemma}
\begin{proof}
1. 
See \cite[Thm. 4.1]{SinhaMohapatra1994}.
2. 
See \cite[(8.8.3)]{Yafaev1992}.
\end{proof}

Under certain conditions \eqref{krein_xi} can be inverted in that the perturbation determinant
can be expressed via the spectral shift function.

\begin{lemma}\label{ssf02t}
Assume that $\xi\in L^1(\R;(1+|\lambda|)^{-1})$. Then,
\begin{equation*}
  \ln[\Delta(z)] = \int_{-\infty}^\infty \frac{\xi(\lambda)}{\lambda-z}\, d\lambda,\ \im(z)\neq 0,
\end{equation*}
and $\xi$ is given via \eqref{krein_xi}.
\end{lemma}
\begin{proof}
See \cite[(0.9.38)]{Yafaev2010} and \cite[(0.9.39)]{Yafaev2010}.
\end{proof}

The spectral shift function is related to the scattering matrix (cf. Section \ref{scattering}).

\begin{lemma}[Birman--Kre\u\i{}n formula]\label{ssf03t}
Let $R_V(z)-R(z)\in B_1(\hilbert)$ for $z\notin\sigma(H_V)\cup\sigma(H)$. Then,
\begin{equation}\label{ssf03t01}
  \det(S(\lambda)) = e^{-2\pi i\xi(\lambda)}
\end{equation}
where $S(\lambda)$ is the scattering matrix at energy $\lambda\in\R$ for the operators $H_V$ and $H$.
\end{lemma}
\begin{proof}
See \cite[Thm. 0.9.4]{Yafaev2010}.
\end{proof}
