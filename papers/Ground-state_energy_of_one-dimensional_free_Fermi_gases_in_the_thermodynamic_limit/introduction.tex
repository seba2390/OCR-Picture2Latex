In the seminal papers~\cite{Anderson1967a,Anderson1967} from 1967, Ph.~Anderson studied the
overlap between the ground states of $N$ non-interacting fermions with and without an external potential.
He showed that this overlap vanishes in the thermodynamic limit $N\to\infty$ at the rate $N^{-\gamma}$ and
expressed the orthogonality exponent $\gamma$ in terms of scattering data. 
This effect became known as Anderson's orthogonality catastrophe (AOC).

Beside the ground-state overlap an appropriate quantity to describe the change caused by the potential 
is the difference of the respective ground-state energies.
In 1994, I.~Affleck~\cite{Affleck1997,AffleckLudwig1994} proposed that the coefficient of the 
next-to-leading term of that energy difference in the thermodynamic limit can be identified with $\gamma$. 
His arguments were based upon ideas from conformal field theory.
In this paper we will thoroughly describe the asymptotics of the energy difference for one-dimensional (spin-less) fermions 
and thereby test Affleck's remarkable relation.

The infinite system is approached either by intervals $\Lambda_L\coloneqq [-L,L]$ or by $\Lambda_L \coloneqq [0,L]$. 
On such an interval, the free single-particle Hamiltonian $H_L\coloneqq -\Delta$ acting on the Hilbert space $L^2(\Lambda_L)$ 
(and depending on some boundary conditions, which we suppress in this notation) has eigenvalues 
$\lambda_{1,L}\leq\lambda_{2,L}\leq \ldots$ and (normalized) eigenfunctions $\varphi_{j,L}$. 
The ground state $\Phi_{N,L}$ of $N$ free fermions on $\Lambda_L$ is 
the anti-symmetric tensor product of the single particle (normalized) eigenfunctions $\varphi_{j,L}$. 
Hence, the energy of the ground state $\Phi_{N,L} = \varphi_{1,L} \wedge\cdots\wedge\varphi_{N,L}$ equals
\begin{equation}\label{intro01}
  \lambda^{(N)}_L = \lambda_{1,L}+\cdots+\lambda_{N,L} .
\end{equation}
Similarly, if $H_{V,L} \coloneqq H_L+V$ is another (or perturbed) single-particle Hamiltonian on the same Hilbert space $L^2(\Lambda_L)$ 
with eigenvalues $\mu_{j,L}$ (in ascending order) and eigenfunctions $\psi_{j,L}$ then the new ground state of 
$N$ fermions is $\Psi_{N,L} =\psi_{1,L}\wedge\cdots\wedge\psi_{N,L}$ with energy
\begin{equation}\label{intro02}
  \mu^{(N)}_L = \mu_{1,L}+\cdots+\mu_{N,L}  .
\end{equation}
The overlap mentioned in AOC is the square of the modulus of the scalar product, $|(\Phi_{N,L},\Psi_{N,L})|^2$. 
With growing $N$ and $L$ but with particle density $N/|\Lambda_L|$ converging to some fixed $\rho>0$, this
overlap equals to leading order $N^{-\gamma}$ with $\gamma>0$ as $N\to\infty$. 
It is conjectured (see \cite{Anderson1967a,GebertKuettlerMuellerOtte2016}) that
\begin{equation*}
  \gamma(\nu) = \frac{1}{\pi^2}\|\arcsin|T(\nu)/2|\|_2^2
\end{equation*}
where $T(\nu)\coloneqq S(\nu)-\id$ is the T-matrix and $S(\nu)$ the S-matrix at energy $\nu$. 
AOC has not been studied from a mathematical point until recently. In \cite{GebertKuettlerMuellerOtte2016}, it was proved
that $N^{-\gamma}$ with the above $\gamma$ is indeed an upper bound to the decay of the overlap. A smaller upper bound was proved earlier
in \cite{GebertKuettlerMueller2014} and in spatial dimension one in \cite{KuettlerOtteSpitzer2014} together with a lower bound of the
order $N^{-\gamma'}$ for some $\gamma'>\gamma$.

The difference of the ground-state energies is $\mu^{(N)}_L- \lambda^{(N)}_L$.
We perform the thermodynamic limit in a slightly different manner by fixing some (Fermi) energy $\nu>0$
and instead analyzing the \emph{canonical energy difference}
\begin{equation}\label{intro03}
  \mathcal{E}_L(\nu) \coloneqq \sum_{\mu_{j,L}\leq\nu} f(\mu_{j,L}) - \sum_{\lambda_{j,L}\leq\nu}f(\lambda_{j,L})
\end{equation}
in the limit $L\to\infty$. Here, we allow for a (holomorphic) weight function $f$, the most important case being $f(z)=z$. 
The particle number $N$ is recovered through the largest $\lambda_{N,L}\leq \nu$, see \ref{ssc}. It is easy to see
that $N/L\to\rho>0$ as $N,L\to\infty$.
It turns out that expanding $\mathcal{E}_L(\nu)$ in powers of $1/L$ does not quite cover the actual asymptotic behaviour 
which seems to contradict Affleck's relation when taken too strictly. Rather, leaving aside some technical details
(see \eqref{asymptotics01} for the precise statement) we decompose the energy difference into
\begin{equation*}
  \mathcal{E}_L(\nu) = -f(\nu)\xi_L(\nu) + \mathcal{E}_L^{\text{Fumi}}(\nu) + \mathcal{E}_L^{\text{FSE}}(\nu) .
\end{equation*}
Here, $\xi_L(\nu)$ is Kre\u\i{}n's spectral shift function, see \ref{ssf} and \ref{ssc}.
The so-called \emph{Fumi-term} $\mathcal{E}_L^{\text{Fumi}}(\nu)$ is of order one though with possible lower order correction terms.
The next-to-leading term is $\mathcal{E}_L^{\text{FSE}}(\nu)$, called \emph{finite size energy}, which is of order $1/L$
\begin{equation*}
  \mathcal{E}_L^{\text{FSE}}(\nu) = \mathcal{E}^{\text{FSE}}(\nu)\frac{1}{L} + o(\frac{1}{L}) .
\end{equation*}
According to Affleck, the coefficient $\mathcal{E}^{\text{FSE}}(\nu)$ is supposed to equal the orthogonality exponent $\gamma$.

The paper's main result is a rigorous analysis of the asymptotic behaviour of the Fumi-term and the finite size energy 
as well as explicit expressions for the limit terms. 

\begin{theorem}\label{intro01t}
Let $V\in L^1(\R)$ satisfy the Birman--Solomyak conditions
$V\in\ell^{\frac{1}{2}}(L^1(\R))$ and $X^2V\in\ell^{\frac{1}{2}}(L^1(\R))$, see \eqref{birman_solomyak}.
Then, $\mathcal{E}_L^{\text{Fumi}}(\nu)\to\mathcal{E}^{\text{Fumi}}(\nu)$ as $L\to\infty$ with the Fumi term
\begin{equation*}
  \mathcal{E}^{\text{Fumi}}(\nu)
      = \int_{-\infty}^\nu f'(\lambda)\xi(\lambda)\, d\lambda .
\end{equation*}
Here, $\xi$ is the spectral shift function for the infinite volume system (cf. \eqref{krein_xi}).
\end{theorem}

\begin{theorem}\label{intro02t}
Let $V\in L^1(\R)$ satisfy $X^3V\in L^1(\R)$. 
Assume that $e^{2\pi iL\sqrt{\nu}}\to e^{i\pi\eta}$, $\eta\in\interval[open right]{-1}{1}$, for $L\to\infty$.
Then, in this limit $L\mathcal{E}_L^{\text{FSE}}(\nu)\to\mathcal{E}^{\text{FSE}}(\nu)$ where
\begin{equation*}
  \mathcal{E}^{\text{FSE}}(\nu)
      = \frac{\sqrt{\nu}}{4\pi}f'(\nu) \tr\big[ \arccos^2(\re(e^{i\eta}\sigma_xU(\nu)^*S(\nu))) - \arccos^2(\re(e^{i\eta}\sigma_xU(\nu)^*)) \big]
\end{equation*}
with the boundary condition scattering matrix $U(\nu)$, see \eqref{fr_bcs}, and the Pauli matrix $\sigma_x$.
$S(\nu)$ is the scattering matrix at energy $\nu$, see Section \ref{scattering}).
\end{theorem}

The Fumi term $\mathcal{E}^{\text{Fumi}}(\nu)$ does not depend on the special sequence of system lengths $L$ used for the limit. 
In contrast, the finite size energy does. This subtlety was first noted by M.~Gebert \cite{Gebert2015} who analyzed
the corresponding system on the half-line with $\Lambda_L = [0, L]$ and Dirichlet boundary conditions at both endpoints. 
We return to the half-line model in Section \ref{hl} and recover Gebert's result 
although under slightly different conditions on the potential $V$.

Moreover, while the Fumi term $\mathcal{E}^{\text{Fumi}}(\nu)$ is the same for all boundary conditions
the finite size energy $\mathcal{E}^{\text{FSE}}(\nu)$ is not.
Through the boundary condition scattering matrix $U(\nu)$, see \eqref{fr_bcs},
it depends on the boundary conditions and, therefore, cannot
equal the orthogonality exponent $\gamma$ in general. However, AOC was essentially studied with Dirichlet boundary
conditions and its independence thereof has not yet been established. That leaves Affleck's relation still open.

The first appearance of the relation between the energy difference and the (integral of the) scattering 
data seems to be in Fumi's work~\cite{Fumi1955}, see also \cite{LangerAmbegaokar1961} and \cite[Sec. 4.1F]{Mahan1990}.
Shortly thereafter Fukuda and Newton~\cite{FukudaNewton1956} studied the difference of eigenvalues 
$\mu_{N,L}-\lambda_{N,L}$ and related the limit to scattering data.

Avoiding the thermodynamic limit, Frank, Lewin, Lieb, and Seiringer~\cite{FrankLewinLiebSeiringer2011}
studied the energy difference directly in the infinite-volume system
(cf. \eqref{energy_difference}) and proved quantitative semiclassical bounds reminiscent of Lieb--Thirring bounds.

The proofs of Theorems \ref{intro01t} and \ref{intro02t} are presented in Section \ref{asymptotics}.
We start off by using a generalization of Riesz's integral formula due to Dunford 
to express the energy difference $\mathcal{E}_L(\nu)$ in terms of an integral of the perturbation determinant 
which involves the Birman--Schwinger operator $K_L(z)\coloneqq \sqrt{|V|}R_L(z)\sqrt{|V|}J$, see Proposition \ref{energy01t}.
Here the potential is written in the Rollnik form, $V=\sqrt{|V|}J\sqrt{|V|}$ and 
$R_L(z)$ is the resolvent of the free Schr\"odinger operator on the bounded domain $\Lambda_L$.
The latter can be written such as to separate the infinite volume part from the boundary conditions, 
$R_L(z) = R_{\infty,L}(z) -D_L(z)$.
$R_{\infty,L}(z)$ is the resolvent of the free Schr\"odinger operator on the full domain 
(either $\R$ or $\interval[open right]{0}{\infty}$) restricted to $\Lambda_L$. 
We allow all possible boundary conditions for the Schr\"odinger operator on $\Lambda_L$. 
The operator $D_L(z)$ is trace class and, in dimension one, even a rank two operator, respectively a rank one operator.
An equally important fact is that this representation of $R_L(z)$ eventually yields a natural decomposition of $\mathcal E_L(\nu)$ 
into the Fumi term $\mathcal{E}^{\text{Fumi}}(\nu)$ and the finite size energy $\mathcal{E}^{\text{FSE}}(\nu)$.

Besides the Birman--Schwinger operator $K_L(z)$ corresponding to the (free) Schr\"odinger operator on $\Lambda_L$ 
there is a locally reduced Birman--Schwinger operator $K_{\infty,L}(z)$, 
which stems from the resolvent $R_\infty(z)$ of the (free) Schr\"odinger operator on $\R$, 
respectively $\interval[open right]{0}{\infty}$. 
In Lemma \ref{wo03t} we find an expression of $K_L(z)$ in terms of $K_{\infty,L}(z)$ 
and a factorization of the corresponding perturbation determinants
which allows us to derive the limiting behaviour of the energy difference, see Section \eqref{asymptotics}.

Our mathematical approach via the resolvent is related to the work of Gesztesy and Nichols~\cite[Lemma 3.2]{GesztesyNichols2012},
who proved that
\begin{equation*}
  \det(\id-\sqrt{V}R_L(z)\sqrt{V}) \to \det(\id-\sqrt{V}R_\infty(z)\sqrt{V}) \ \text{as}\ L\to\infty,\
    z\in\C\setminus\interval[open right]{0}{\infty} .
\end{equation*}
Since we are interested in the $1/L$-correction to the leading Fumi-term 
we need more information (uniformly in $z$) than just the above limit of determinants. 
In fact, we need the rate of approach to this limit. 

Although we treat here only one-dimensional systems, our approach via
Riesz's integral formula allows us to treat higher dimensional systems
with non-radial potentials. We plan to pursue this in the future.
