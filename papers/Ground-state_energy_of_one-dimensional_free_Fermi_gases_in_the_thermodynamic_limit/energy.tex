Let $\hilbert$ be a separable complex Hilbert space with scalar product $(\cdot,\cdot)$ which is linear in the second argument. 
We denote by $B_p(\hilbert)$, $1\leq p<\infty$, the Schatten--von Neumann classes with norm $\|\cdot\|_p$.
Furthermore, $B(\hilbert)$ are the bounded operators with operator norm $\|\cdot\|$ occasionally referred to as the $p=\infty$-norm.
These norms satisfy Jensen's inequality
\begin{equation}\label{jensen_inequality}
   \| A\|_q \leq \|A\|_p,\ 1\leq p\leq q\leq \infty
\end{equation}
and H\"older's inequality
\begin{equation}\label{hoelder_inequality}
   \|A B\|_r \leq \|A\|_p\|B\|_q,\ 1\leq r,p,q\leq \infty,\
   \frac{1}{r} = \frac{1}{p}+\frac{1}{q} .
\end{equation}
The only cases needed herein are $p=1,2$, the trace class and Hilbert--Schmidt operators, respectively, 
and $p=\infty$.

Let $H:\dom(H)\to\hilbert$, $\dom(H)\subset\hilbert$, be a self-adjoint operator
and $V:\hilbert\to\hilbert$ be (for simplicity) a bounded symmetric, hence self-adjoint, operator. Its polar decomposition
can be written in a form first used by Rollnik \cite{Rollnik1956} when studying the Lippmann--Schwinger equation (cf. Section \ref{scattering})
\begin{equation}\label{V_polar_decomposition}
  V = \sqrt{|V|}J\sqrt{|V|},\ J^*=J,\ J^2=\id,\ \|J\|=1 ,
\end{equation}
where $\id$ denotes the unity operator. Then, $H_V\coloneqq H+V$ is self-adjoint as well with $\dom(H_V)=\dom(H)$. 
We denote by $\sigma(H)$ and $\sigma(H_V)$ the spectrum of $H$ and $H_V$, respectively, by
\begin{equation}\label{resolvents}
  R(z) \coloneqq (z\id-H)^{-1},\ z\in\C\setminus\sigma(H),\ 
  R_V(z) \coloneqq (z\id- H_V)^{-1},\ z\in\C\setminus\sigma(H_V)
\end{equation}
their resolvents and their spectral families by
\begin{equation}\label{spectral_projections}
  P_\nu \coloneqq \chi_{\interval[open left]{-\infty}{\nu}}(H),\   
  \Pi_\nu \coloneqq \chi_{\interval[open left]{-\infty}{\nu}}(H_V),\ \nu\in\R ,
\end{equation}
where $\chi_I$ is the indicator function of the interval $I\subset\R$.
The paper's central quantity is the energy difference
\begin{equation}\label{energy_difference}
  \mathcal{E}(\nu) \coloneqq \tr \big[f(H_V)\Pi_\nu - f(H)P_\nu\big]
\end{equation}
for some fixed holomorphic function $f:\C\to\C$.
It is well-defined under appropriate conditions on the operators $H$ and $H_V$.
