Because our main focus is on the Fumi term and the finite size energy we will look only
briefly at the spectral shift function $\xi_L(\nu)$ in \eqref{asymptotics01}. 
Since $\xi_L(\nu)\in\Z$ for $L<\infty$ it cannot converge unless it is constant. 
Nonetheless, it is known that at least for certain boundary conditions
$\xi_L(\nu)$ stays bounded as $L\to\infty$ or that even $\xi_L(\nu)=0$ for potentials $V$ being small in an appropriate sense,
see \cite[Sec. 3.5]{KuettlerOtteSpitzer2014}). Furthermore, it was shown in \cite{BorovykMakarov2012} that $\xi_L(\nu)$
is Ces\'aro-convergent to $\xi(\nu)$
\begin{equation*}
  \lim_{L\to\infty} \frac{1}{L} \int_0^L \xi_l(\nu)\, dl = \xi(\nu) .
\end{equation*}
Another way to look at it is via the micro-canonical energy difference. To this end, define
\begin{equation}\label{ssc00}
  M \coloneqq \max\{ k \mid \mu_{k,L} \leq \nu \}, \ 
  N \coloneqq \max\{ k \mid \lambda_{k,L} \leq \nu \} .
\end{equation}
Then, cf. Lemma \ref{H_L_spectrum} and \eqref{H_VL_spectrum},
\begin{equation}\label{ssc01}
  \mathcal{E}^{\text{mc}}_{N,L} \coloneqq \sum_{k=1}^N\mu_{k,L} - \sum_{k=1}^N\lambda_{k,L} ,
\end{equation}
with $f(z)=z$ for simplicity.
It is easily seen that $\mathcal{E}^{\text{mc}}_{N,L}$ is related to $\mathcal{E}_L(\nu)$ from \eqref{energy_difference} via
\begin{equation*}
  \mathcal{E}^{\text{mc}}_{N,L}
     = \mathcal{E}_L(\nu) + \sign(N-M)\sum_{k=\min\{M,N\}+1}^{\max\{M,N\}}\mu_{k,L} .
\end{equation*}
Note that starting with $\mathcal{E}^{\text{mc}}_{N,L}$ one would use the unperturbed eigenvalue $\lambda_{N,L}$ 
to determine $\nu$ and thereby $\mathcal{E}_L(\nu)$. That is why the sum is over the $\mu_{k,L}$ and
not the $\lambda_{k,L}$.
From Lifshitz's formula \eqref{lifshitz} we know that $\xi_L(\nu) = -(M-N)$. Thereby, the spectral shift correction
can be compensated
\begin{equation}\label{ssc02}
  - \nu\xi_L(\nu) + \sign(N-M)\sum_{k=\min\{M,N\}+1}^{\max\{M,N\}}\mu_{k,L} 
     = \sign(N-M)\sum_{k=\min\{M,N\}+1}^{\max\{M,N\}} (\mu_{k,L}-\nu) .
\end{equation}
It can be shown that $\mu_{N,L}-\nu=O(1/L)$ as $N\to\infty,L\to\infty$.
This was first considered in \cite{FukudaNewton1956}. Since $M-N=O(1)$ the term in \eqref{ssc02}
is $O(1/L)$ and does not contribute to the leading Fumi term but to the finite size correction
of the micro-canonical energy difference, $\mathcal{E}^{\text{mc}}_{N,L}$, as $N,L\to\infty$ and $N/(2L)\to\rho$.
