From a simple change of coordinates it can be seen that the
finite size correction $\mathcal{E}_L^{\text{FSE}}(\nu)$ in \eqref{E_L_FSE} is of the order $1/L$.
This will be made precise in Lemma \ref{fse01t}. To begin with,
we determine how $\mathcal{V}_L(s)$ in \eqref{V_L} behaves asymptotically for large $L$. 
The condition $V\in L^1(\R)$ makes $\mathcal{V}_L$ a tad smaller than the exponential function.

\begin{lemma}\label{con01t}
Let $V\in L^1(\R)$ satisfy $X^nV\in L^1(\R)$ with $n\in\N_0$. 
Then, for $s\geq 0$
\begin{equation}\label{con01t01}
  0 \leq e^{-Ls} \mathcal{V}_L^{(n)}(s) \leq \|X^nV\|_1 .
\end{equation}
If $X^{n+\alpha} V\in L^1(\R)$ with $\alpha\geq 0$ then for $s>0$
\begin{equation}\label{con01t02}
  \lim_{L\to\infty} \big[ L^\alpha e^{-Ls} \mathcal{V}_L^{(n)}(s) \big] =0 .
\end{equation}
Moreover,
\begin{equation}\label{con01t03}
  \lim_{L\to\infty} \big[ L^\alpha \int_0^b s^\alpha e^{-Ls} \mathcal{V}_L^{(n)}(s)\, ds \big] = 0 .
\end{equation}
\end{lemma}
\begin{proof}
The bound \eqref{con01t01} follows immediately from definition \eqref{V_L}.
Pick any $0<\delta<1$. For $s\geq 0$
\begin{equation}\label{con01t04}
\begin{split}
 L^\alpha e^{-Ls} \mathcal{V}_L^{(n)}(s) 
   & = L^\alpha e^{-Ls} \int_{0\leq |x|\leq\delta L}|x|^n |V(x)| e^{|x|s}\, dx 
          + L^\alpha\int_{\delta L\leq |x| \leq L} |x|^{n+\alpha} |V(x)| \frac{1}{|x|^\alpha} e^{-(L-|x|)s}\, dx\\
   & \leq L^\alpha e^{-(1-\delta)Ls} \|X^nV\|_1 + \frac{1}{\delta^\alpha} \int_{\delta L\leq |x|} |x|^{n+\alpha} |V(x)|\, dx .
\end{split}
\end{equation}
This implies \eqref{con01t02}. Finally, we integrate \eqref{con01t04}
\begin{equation*}
\begin{split}
  L^\alpha\int_0^b s^\alpha e^{-Ls}\mathcal{V}_L^{(n)}(s)\, ds
    & \leq \int_0^b L^\alpha s^\alpha e^{-(1-\delta)Ls}\, ds \|X^nV\|_1 
          + \frac{1}{\delta^\alpha} \int_0^b s^\alpha \int_{\delta L\leq |x|} |x|^{n+\alpha} |V(x)|\, dx\, ds\\
    & \leq \frac{1}{L}\int_0^\infty s^\alpha e^{-(1-\delta)s}\, ds \|X^nV\|_1 
               + \frac{1}{\delta^\alpha}\frac{b^{\alpha+1}}{\alpha+1} \int_{\delta L\leq |x|} |x|^{n+\alpha} |V(x)|\, dx
\end{split}
\end{equation*}
and obtain \eqref{con01t03}.
\end{proof}

In order to perform the limit in the FSE term we need the inequality for $n\times n$-matrices
\begin{equation}\label{det_difference_matrix}
  |\det(M_1) - \det(M_2)| \leq n \|M_1-M_2\|_2 \max\{\|M_1\|_2^{n-1},\|M_2\|_2^{n-1}\},
\end{equation}
which follows via multilinearity and Hadamard's inequality.

\begin{lemma}\label{fse01t}
Let $V\in L^1(\R)$ such that $X^nV\in L^1(\R)$ for $n=1,2,3$. We assume that $e^{2iL\sqrt{\nu}}\to e^{i\pi\eta}$,
$\eta\in\interval[open right]{-1}{1}$, as $L\to\infty$. 
Then, for all $b>0$ large enough $L\mathcal{E}^{\mathrm{FSE}}_{L,b}(\nu)\to\mathcal{E}^{\mathrm{FSE}}(\nu)$ as $L\to\infty$. Here,
\begin{equation}\label{fse01t01}
  \mathcal{E}^{\mathrm{FSE}}(\nu)
     = - \frac{1}{\pi} \sqrt{\nu} f'(\nu) \int_0^\infty \Big\{\ln\det(\id + d(s) T(\nu) G(s)) + \ln\det(\id + d(s) T(\nu) G(s))^* \Big\}\, ds
\end{equation}
with the T-matrix from \eqref{Omega_T} and with
\begin{equation}\label{fse01t02}
  G(s) \coloneqq  (e^{i\pi\eta}e^{-2s}\id + U(\nu)\sigma_x)^{-1},\ d(s)\coloneqq e^{i\pi\eta}e^{-2s},\ s\geq 0.
\end{equation}
\end{lemma}
\begin{proof}
We treat the two parts $\Gamma_{\nu,b}^-$ and $\Gamma_{\nu,b}^+$ of the Fermi parabola separately.

(a) 
On $\Gamma_{\nu,b}^-$ we use \eqref{det_difference_matrix} with $M_2=\id$ and obtain
\begin{equation*}
  |\det(\id+d_L(z) T_L(z)G_L(z)) - 1|
     \leq 2|d_L(z)| \|T_L(z)\|_2 \|G_L(z)\|_2
     \leq C e^{-2Lb} \mathcal{V}_L(2b) 
\end{equation*}
where we used \ref{wo02t01a} and Lemma \ref{bcs_fermi00t}. Now, Lemma \ref{con01t} shows that
$\det(\id+d_L(z) T_L(z)G_L(z))\to 1$ as $L\to\infty$ uniformly in $z\in\Gamma_{\nu,b}^-$. Hence, the integral
vanishes in this limit. 

(b) 
On $\Gamma_{\nu,b}^+$ we replace $S_L(z)$ and $U(z)$ by $S_L(\nu)$ and $U(\nu)$, respectively, which yields
the difference
\begin{multline*}
  \ln\Big[\frac{\det(e^{2iL\sqrt{z}}S_L(z)\sigma_x U(z)^{-1} + \id)}{\det(e^{2iL\sqrt{z}}\sigma_x U(z)^{-1} + \id)}\Big]
   - \ln\Big[\frac{\det(e^{2iL\sqrt{z}}S_L(\nu)\sigma_x U(\nu)^{-1} + \id)}{\det(e^{2iL\sqrt{z}}\sigma_x U(\nu)^{-1} + \id)}\Big]\\
   = \ln\Big[\frac{\det(e^{2iL\sqrt{z}}S_L(z)\sigma_x U(z)^{-1} + \id)}{\det(e^{2iL\sqrt{z}}S_L(\nu)\sigma_x U(\nu)^{-1} + \id)}\Big]
       - \ln\Big[\frac{\det(e^{2iL\sqrt{z}}\sigma_x U(z)^{-1} + \id)}{\det(e^{2iL\sqrt{z}}\sigma_x U(\nu)^{-1} + \id)}\Big].
\end{multline*}
We rewrite the first term
\begin{equation*}
  \frac{\det(e^{2iL\sqrt{z}}S_L(z)\sigma_x U(z)^{-1} + \id)}{\det(e^{2iL\sqrt{z}}S_L(\nu)\sigma_x U(\nu)^{-1} + \id)}
     = 1 + \frac{\det(e^{2iL\sqrt{z}}S_L(z)\sigma_x U(z)^{-1} + \id) 
                  - \det(e^{2iL\sqrt{z}}S_L(\nu)\sigma_x U(\nu) + \id)}{\det(e^{2iL\sqrt{z}}S_L(\nu)\sigma_x U(\nu)^{-1} + \id)}
\end{equation*}
and estimate the numerator via \eqref{det_difference_matrix}
\begin{equation*}
\begin{split}
  \lefteqn{| \det(e^{2iL\sqrt{z}}S_L(z)\sigma_x U(z)^{-1} + \id) - \det(e^{2iL\sqrt{z}}S_L(\nu)\sigma_x U(\nu)^{-1} + \id) |}\\
    & \leq 2 e^{-2Ls}\|S_L(z)\sigma_x U(z)^{-1} - S_L(\nu)\sigma_x U(\nu)^{-1}\|_2 \\
    & \quad \times \max\{ \|e^{2iL\sqrt{z}}S_L(z)\sigma_x U(z)^{-1}+\id\|_2, \|e^{2iL\sqrt{z}}S_L(\nu)\sigma_x U(\nu)^{-1}+\id\|_2 \} .
\end{split}
\end{equation*}
Now with Lemma \ref{wo02t} (see also Lemma \ref{bcs_fermi01t})
\begin{equation*}
  \|S_L(z)\sigma_x U(z)^{-1} - S_L(\nu)\sigma_x U(\nu)^{-1}\|_2
      \leq \|S_L(z)-S_L(\nu)\|_2 \|U(z)^{-1}\| + \|U(z)^{-1}-U(\nu)^{-1}\|_2 
      \leq C s .
\end{equation*}
The second term can be treated likewise
\begin{equation*}
  \frac{\det(e^{2iL\sqrt{z}}\sigma_x U(z)^{-1} + \id)}{\det(e^{2iL\sqrt{z}}\sigma_x U(\nu)^{-1} + \id)}
     = 1 + \frac{\det(e^{2iL\sqrt{z}}\sigma_x U(z)^{-1} + \id) 
                  - \det(e^{2iL\sqrt{z}}\sigma_x U(\nu)^{-1} + \id)}{\det(e^{2iL\sqrt{z}}\sigma_x U(\nu)^{-1} + \id)} .
\end{equation*}
The numerator can be estimated
\begin{equation*}
\begin{split}
  \lefteqn{|\det(e^{2iL\sqrt{z}}\sigma_x U(z)^{-1} + \id) - \det(e^{2iL\sqrt{z}}\sigma_x U(\nu)^{-1} + \id)|}\\
    & \leq 2 e^{-2Ls}\|U(z)^{-1}-U(\nu)^{-1}\|_2 
         \max\{\|e^{2iL\sqrt{z}}\sigma_x U(z)^{-1} + \id\|_2 ,\|e^{2iL\sqrt{z}}\sigma_x U(\nu)^{-1} + \id\|_2\} \\
    & \leq C e^{-2Ls} s .
\end{split}
\end{equation*}
All in all, Lemma \ref{con01t} proves the statement.
\end{proof}

Next, we cast the integral in \eqref{fse01t01} into a more compact form.

\begin{lemma}\label{fse02t}
The finite size energy \eqref{fse01t01} can be written as (see \eqref{Omega_T}, \eqref{fse01t02})
\begin{equation}\label{fse02t01}
   \mathcal{E}^{\text{FSE}}(\nu)
     = -\frac{\sqrt{\nu}}{2\pi}f'(\nu) \int_0^\infty 
          \ln\big\{\det\big[ \id + ( \cosh(s) + \re(e^{i\eta}\sigma_xU(\nu)^*) )^{-1}\re(e^{i\eta}\sigma_xU(\nu)^*T(\nu)) \big]\big\} \, ds .
\end{equation}
\end{lemma}
\begin{proof}
The integrand in \eqref{fse01t01} is
\begin{gather*}
  \ln\det(\id + d(s) T(\nu) G(s)) + \ln\det(\id + d(s) T(\nu) G(s))^*  
     = \ln\det(F(s)),\\
  F(s) \coloneqq (\id+d(s)T(\nu)G(s))^*(\id+d(s)T(\nu)G(s)) .
\end{gather*}
Note that $T(\nu)^*T(\nu)=-T(\nu)-T(\nu)^*$. Then,
\begin{equation*}
\begin{split}
  F(s) 
   & = \id + \bar d(s) G(s)^*T(\nu)^* + d(s) T(\nu)G(s) + |d(s)|^2 G(s)^*T(\nu)^*T(\nu)G(s) \\
   & = \id + \bar d(s) G(s)^*T(\nu)^*(\id-d(s)G(\nu)) + d(s)(\id-\bar d(s)G(s)^*T(\nu)G(s) \\
   & = \id + G(s)^*\big[ \bar d(s)T(\nu)^*U(\nu)\sigma_x + d(s) \sigma_xU(\nu)^*T(\nu) \big]G(s)
\end{split}
\end{equation*}
where we used $\id-d(s)G(s)= U(\nu)\sigma_xG(s)$. In the determinant we commute the factors
\begin{equation*}
  \det(F(s))
    = \det\big[ \id + G(s)G(s)^*( \bar d(s)T(\nu)^*U(\nu)\sigma_x + d(s) \sigma_xU(\nu)^*T(\nu) ) \big] .
\end{equation*}
Recalling the definition of $d(s)$ we obtain
\begin{equation*}
  G(s)G(s)^* 
   = e^{2s} \big[ (e^{-2s} + e^{2s})\id + e^{-i\eta}U(\nu)\sigma_x + e^{i\eta}\sigma_xU(\nu)^* \big]^{-1}
\end{equation*}
and thereby
\begin{equation*}
  \det(F(s))
   =  \det\big[ \id + ( \cosh(2s) + \re(e^{i\eta}\sigma_xU(\nu)^*) )^{-1}\re(e^{i\eta}\sigma_xU(\nu)^*T(\nu)) \big] .
\end{equation*}
Finally, substitute $s\to s/2$ in the integral to prove \eqref{fse02t01}.
\end{proof}

Surprisingly, the integral in \eqref{fse02t01} can be evaluated.
It is instructive to consider a more general integral.
The method is motivated by that used for integrals of rational functions over the half-line. 
It involves a very simple version of a Riemann--Hilbert problem.

\begin{lemma}\label{integral01t}
Let $H_1$ and $H_2$ be self-adjoint matrices with $H_1>0$ and $H_1+H_2>0$. Let $f(t)\coloneqq\cosh(t)-1$. Then,
\begin{equation}\label{integral01t01}
  \int_0^\infty \ln\big[\det( \id + (f(t)\id+H_1)^{-1}H_2 ) \big] \, dt =
     \frac{1}{2} \tr\big[ \arcosh^2(H_1+H_2-\id) - \arcosh^2(H_1-\id) \big].
\end{equation}
\end{lemma}
\begin{proof}
We start with the more general integral
\begin{equation*}
  I \coloneqq \int_0^\infty \ln\big[\det(\id+(f(t)\id+H_1)^{-1}H_2)\big]\, dt
\end{equation*}
where $f:\R^+\to\R$ is a differentiable function with 
\begin{equation*}
   f'(t)> 0\ \text{for}\ t>0,\
   f(0)=0,\ f(\infty)=\infty,\ \lim_{x\to\infty} f^{-1}(x)/x =0 .
\end{equation*}
We substitute
\begin{equation*}
  t = f^{-1}(x),\ dt = \frac{d}{dx}(f^{-1}(x))\, dx
\end{equation*}
and integrate by parts
\begin{equation*}
\begin{split}
  I  & = \int_0^\infty \ln\big[\det( \id + (x\id+H_1)^{-1}H_2)\big] \frac{d}{dx}( f^{-1}(x) )\, dx\\
     & = - \int_0^\infty \tr\big[ (x\id+H_1+H_2)^{-1} - (x\id+H_1)^{-1} \big] f^{-1}(x)\, dx\\
     & = \int_0^\infty r(x) g(x)\, dx
\end{split}
\end{equation*}
where
\begin{equation*}
   r(x) \coloneqq - \tr\big[ (x\id+H_1+H_2)^{-1} - (x\id+H_1)^{-1} \big],\ g(x)\coloneqq f^{-1}(x).
\end{equation*}
Note that $r$ is a rational function whose poles are the eigenvalues of $-H_1$ and $-H_1-H_2$ which we assumed to be negative. 
Thus, $r$ has no poles on $\R^+$. Integrals of this type can be evaluated if one finds a
holomorphic function $h:\C\setminus\interval[open right]{0}{\infty}\to\C$ satisfying on the cut
\begin{equation*}
  h(x+i0) - h(x-i0) = g(x),\ x\in[0,\infty[ .
\end{equation*}
Then, the keyhole integration contour
\begin{equation*}
  \Gamma_{\varepsilon,R,r} \coloneqq \Gamma_\varepsilon^+ \cup \Gamma_r \cup \Gamma_\varepsilon^- \cup \Gamma_R ,
\end{equation*}
see Fig. \ref{f_keyhole},
%%% Figure
%\begin{vchfigure}[b]
%\includegraphics[width=0.75\textwidth]{f_keyhole.pdf}
%\vchcaption{The keyhole integration contour in the complex plane.}
%\label{f_keyhole}
%\end{vchfigure}
%%%
%%% Small figure
%\begin{figure}[b]
\begin{figure}[bth]
%\sidecaption
\includegraphics[width=0.75\textwidth]{f_keyhole.pdf}
\caption{The keyhole integration contour in the complex plane.}
\label{f_keyhole}
\end{figure}
along with the residue theorem can be used to show that
\begin{equation*}
  I = 2\pi i \sum_{z\in\C\setminus\interval[open right]{0}{\infty}} \res(r(z)h(z)) .
\end{equation*}
In our case,
\begin{equation*}
  f(t) = \cosh(t) - 1,\ 
  g(x)\coloneqq f^{-1}(x) = \arcosh(x+1) .
\end{equation*}
Now, we are left with finding the function $h$ which would generally
require to solve a Riemann--Hilbert problem. Here we can use
the area function itself. The principal branch is given by the formula
\begin{equation*}
  \arcosh(z) = \ln( \pm(z^2-1)^{\frac{1}{2}} + z),\ z\in\C\setminus\interval[open]{-\infty}{1},\ \re(z)\gtrless 0 .
\end{equation*}
It is holomorphic on $\C$ except for $\interval[open left]{-\infty}{1}$ where it satisfies
\begin{equation*}
  \lim_{\varepsilon\to+0} \arcosh(x\pm i\varepsilon) = 
\begin{cases}
  \pm\ln( i(1-x^2)^{\frac{1}{2}} + x )        & -1 < x \leq 1 ,\\
  \pm \pi i + \ln( (x^2-1)^{\frac{1}{2}} - x) & x\leq -1 .
\end{cases}
\end{equation*}
This motivates us to put
\begin{equation*}
  h(z) \coloneqq -\frac{1}{4\pi i} \arcosh^2(-z-1) .
\end{equation*}
The outer minus sign is necessary since the minus sign in the argument swaps the upper and the lower halfplane.
Obviously, $h$ is holomorphic at least on $\C\setminus\interval[open right]{-2}{\infty}$. But
\begin{equation*}
  h(x+i0) - h(x-i0) = 0,\ -2 \leq x < 0 ,
\end{equation*}
shows that it extends to $\C\setminus\interval[open right]{0}{\infty}$. Furthermore,
\begin{equation*}
  h(x+i0) - h(x-i0) 
    = \ln\big( ((x+1)^2-1)^{\frac{1}{2}} + x +1 \big)
    = \arcosh(x+1),\ x\geq 0 .
\end{equation*}
Hence,
\begin{equation*}
\begin{split}
  I & = 2\pi i \sum_{z\in\C\setminus[0,\infty[} \res\Big\{ \tr\big[ (z\id+H_1+H_2)^{-1} - (z\id+H_1)^{-1} \big]
        \frac{1}{4\pi i}\arcosh^2(-z-1)\Big\} \\
    & = \frac{1}{2}  \tr\big[ \arcosh^2(H_1+H_2-\id) - \arcosh^2(H_1-\id) \big] .
\end{split}
\end{equation*}
This proves \eqref{integral01t01}.
\end{proof}

Finally, we can compute the finite size energy.

\begin{proposition}\label{fse03t}
The finite size energy is given by (cf. \eqref{fr_bcs} and \eqref{s-matrix02})
\begin{equation}\label{fse03t01}
   \mathcal{E}^{\text{FSE}}(\nu) = \frac{\sqrt{\nu}}{4\pi}f'(\nu)
     \tr\big[ \arccos^2(\re(e^{i\eta}\sigma_xU(\nu)^*S(\nu))) - \arccos^2(\re(e^{i\eta}\sigma_xU(\nu)^*)) \big] .
\end{equation}
\end{proposition}
\begin{proof}
We define
\begin{equation*}
  H_1 \coloneqq \id + \re(e^{i\eta}\sigma_xU(\nu)^*),\ H_2\coloneqq \re(e^{i\eta}\sigma_xU(\nu)^*T(\nu)) .
\end{equation*}
Note that $0\leq H_1$ and $0\leq H_1+H_2$. Inserting $H_{1,2}$ into Lemma \ref{integral01t} we obtain
\begin{equation*}
\begin{split}
  \lefteqn{\int_0^\infty\ln\det\big[ \id + ( \cosh(s)\id + \re(e^{i\eta}\sigma_xU(\nu)^*))^{-1}\re(e^{i\eta}\sigma_xU(\nu)^*T(\nu)) \big]\, ds}\\
    & = \frac{1}{2}\tr\big[ \arcosh^2(\re(e^{i\eta}\sigma_xU(\nu)^*)+\re(e^{i\eta}\sigma_xU(\nu)^*T(\nu)))
                           - \arcosh^2(\re(e^{i\eta}\sigma_xU(\nu)^*)) \big]\\
    & = \frac{1}{2}\tr\big[ \arcosh^2(\re(e^{i\eta}\sigma_xU(\nu)^*S(\nu)))
                           - \arcosh^2(\re(e^{i\eta}\sigma_xU(\nu)^*)) \big] .
\end{split}
\end{equation*}
Since $U(\nu)$ and $S(\nu)$ are unitary the eigenvalues of
$\re(e^{i\eta}\sigma_xU(\nu)^*)$ and $\re(e^{i\eta}\sigma_xU(\nu)^*S(\nu))$
lie in the interval $[-1,1]$ whereby $\arcosh(z)=\pm i\arccos(z)$. This shows \eqref{fse03t01}.
\end{proof}
