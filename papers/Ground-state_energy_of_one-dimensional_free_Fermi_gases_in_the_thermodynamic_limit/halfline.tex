The study of the asymptotics of the energy difference on the half-line can be pursued along
the same line of ideas as on the whole line. But we do not know how to extend Gebert's methods for the half-line to the whole line.

So, let $\hilbert_\infty\coloneqq L^2(\R^+)$, $\dom(H_\infty)\subset\hilbert$ be dense,
and $H_\infty:\dom(H_\infty)\to\hilbert_\infty$, $H_\infty \coloneqq -\frac{d^2}{dx^2}$
the free Schr\"odinger operator defined on the half-line with boundary condition at the origin
\begin{equation}\label{hl01}
  a\varphi(0) = b\varphi'(0),\ a\bar b=b\bar a,\ |a|^2+|b|^2=1.
\end{equation}
We compute the resolvent $R_\infty(z)\coloneqq (z\id-H_\infty)^{-1}$. The function
\begin{equation}\label{hl02}
  \varepsilon_0(z;x) \coloneqq \frac{ia-\sqrt{z}b}{ia+\sqrt{z}b}e^{i\sqrt{z}x} - e^{-i\sqrt{z}x}
\end{equation}
is a formal solution, i.e. it is not in $\hilbert_\infty$,
to the differential equation $H_\infty \varepsilon_0(z)=z\varepsilon_0(z)$
and satisfies the boundary condition \eqref{hl01} at the origin.
Note the property
\begin{equation}\label{hl03}
  \overline{\varepsilon_0(z)} = - \frac{ia+\sqrt{\bar z}b}{ia-\sqrt{\bar z}b} \varepsilon_0(\bar z) .
\end{equation}
Then, the Green function of $R_\infty(z)$ is
\begin{equation}\label{hl04}
  R_\infty(z;x,y) = \frac{i}{2\sqrt{z}}
\begin{cases}
  \varepsilon_0(z;x)e^{i\sqrt{z}y} & \text{for}\ x\leq y , \\
  e^{i\sqrt{z}x}\varepsilon_0(z;y) & \text{for}\ x\geq y .
\end{cases}
\end{equation}
We restrict the operator $H_\infty$ to $\hilbert_L\coloneqq L^2(\Lambda_L)$, $\Lambda_L\coloneqq [0,L]$, thereby obtaining the maximal operator
$\tilde H:\dom(\tilde H)\to\hilbert_L$, $\tilde H\coloneqq -\frac{d^2}{dx^2}$.
Integration by parts shows that
\begin{equation}\label{hl05}
  (\varphi,\tilde H\psi) - (\tilde H\varphi,\psi)
  = (\Gamma_1\varphi,\Gamma_2\psi) - (\Gamma_2\varphi,\Gamma_1\psi)
    \ \text{with}\ 
  \Gamma_1\varphi = \varphi(L),\ \Gamma_2\varphi = -\varphi'(L) .
\end{equation}
The deficiency subspace is (cf. \eqref{abc_deficiency_subspace}, \eqref{hl02})
\begin{equation}\label{hl06}
  \mathcal{N}_z = \mathspan\{ \varepsilon_0(z) \} .
\end{equation}
The boundary condition at $x=L$ is parametrized by the $1\times 1$-matrices $A$ and $B$.
We define the self-adjoint restriction $H_L\coloneqq\tilde H|_{\dom(H_L)}$ via
\begin{equation}\label{hl07}
  \dom(H_L) \coloneqq \{ \varphi\in\dom(\tilde H) \mid (A\Gamma_1-B\Gamma_2)\varphi =0 \}
\end{equation}
with resolvent
\begin{equation}\label{hl08}
  R_L(z)\coloneqq (z\id- H_L)^{-1}:\hilbert_L\to\hilbert_L,\
  z\in\C\setminus\sigma(H_L) .
\end{equation}
The restrictions of $\Gamma_{1,2}$ to $\mathcal{N}_z$ are just numbers or, more precisely, multiplication operators by thoses numbers
\begin{equation}\label{hl09}
  \Gamma_1 \varepsilon_0(z) = \frac{ia-\sqrt{z}b}{ia+\sqrt{z}b} e^{i\sqrt{z}L} - e^{-i\sqrt{z}L},\
  \Gamma_2 \varepsilon_0(z) = -i\sqrt{z} \big[ \frac{ia-\sqrt{z}b}{ia+\sqrt{z}b} e^{i\sqrt{z}L} + e^{-i\sqrt{z}L} \big] .
\end{equation}
Furthermore, using \eqref{hl03} one obtains
\begin{equation}\label{hl10}
  \Gamma_1R_{\infty,L}(z)\varphi
%   = - \frac{1}{2i\sqrt{z}} e^{i\sqrt{z}L} ( \bar \varepsilon_0(z),\varphi )
    = \frac{1}{2i\sqrt{z}}\frac{ia-\sqrt{z}b}{ia+\sqrt{z}b} e^{i\sqrt{z}L} ( \varepsilon_0(\bar z),\varphi)
  ,\
  \Gamma_2R_{\infty,L}(z)\varphi = -\frac{1}{2}\frac{ia-\sqrt{z}b}{ia+\sqrt{z}b} e^{i\sqrt{z}L} (\varepsilon_0(\bar z),\varphi) .
\end{equation}
Now (cf. \eqref{fr01t02})
\begin{equation}\label{hl11}
  G_L(z) = - \frac{i}{2\sqrt{z}}e^{2i\sqrt{z}L}
    \big[ e^{2i\sqrt{z}L} - \frac{ia+\sqrt{z}b}{ia-\sqrt{z}b}\frac{iA+\sqrt{z}B}{iA-\sqrt{z}B} \big]^{-1} .
\end{equation}
Note that on the half-line the relation between the Fermi energy $\nu$ and the system length $L$ is
\begin{equation*}
  L \sqrt{\nu} = \pi( N+\eta ), \ 0\leq \eta < 1 .
\end{equation*}
The relevant quantity for the finite size energy is (cf. \eqref{fse01t02})
\begin{equation}\label{hl12}
  W(z) : = - e^{-2i\pi\eta}\frac{ia+\sqrt{z}b}{ia-\sqrt{z}b}\frac{iA+\sqrt{z}B}{iA-\sqrt{z}B} ,
\end{equation}
which appears in the analogue of Proposition \ref{fse03t}
\begin{equation}\label{hl13}
   \mathcal{E}^{\text{FSE}}(\nu) = \frac{\sqrt{\nu}}{4\pi}f'(\nu)
     \big[ \arccos^2(\re(W(\nu)^*S(\nu))) - \arccos^2(\re(W(\nu))) \big] .
\end{equation}
We compare \eqref{hl13} with Gebert's result \cite[(1.4), (2.12)]{Gebert2015}
\begin{equation*}
  \mathcal{E}^\text{mc}_{N,L}(\nu)
    = \int_{-\infty}^\nu \xi(\lambda)\, d\lambda 
      + \frac{\sqrt{\nu}\pi}{L} E^\text{mc}_\eta(\nu) + o(\frac{1}{L}),\ 
     E^\text{mc}_\eta(\nu) = (1-2\eta)\xi(\nu) + \xi(\nu)^2 .
\end{equation*} 
To this end, we write
\begin{equation*}
  W(\nu) = e^{i\pi} e^{-2\pi i\eta}  e^{2\pi i\vartheta} ,\ S(\nu) = e^{-2\pi i\xi(\nu)} .
\end{equation*}
The latter is the Birman--Kre\u\i{}n formula (cf. Lemma \ref{ssf03t}). Then,
\begin{equation*}
  \re(W(\nu)^*S(\nu)) = \cos(2\pi( -\frac{1}{2} + \eta  - \vartheta - \xi(\nu) )),\
  \re(W(\nu)) = \cos(2\pi( \frac{1}{2} - \eta  + \vartheta))
\end{equation*}
and furthermore
\begin{multline*}
  \arccos^2(\re(W(\nu)^*S(\nu))) - \arccos^2(\re(W(\nu)) ) \\
    = 4\pi^2 \big[ ( -\frac{1}{2} + \eta  - \vartheta - \xi(\nu) )^2 - (\frac{1}{2} - \eta + \vartheta) \big] 
    = \xi(\nu)^2 + (1-2\eta+2\vartheta) \xi(\nu) .
\end{multline*}
For Dirichlet boundary conditions at $x=0$ and $x=L$ we have $\vartheta=0$ which yields exactly the coefficient
as in \cite{Gebert2015}.
