In a scattering experiment, a plane wave coming from, say, $+\infty$ interacts
with a potential whereby one part is reflected back to $+\infty$ and thus superposes
the incoming wave. Another part moves toward $-\infty$.
This scenario is described by the scattering solution $u_+$ of \eqref{st01}
\begin{equation}\label{stl01}
  u_+(x) \sim
\begin{cases}
  \mathscr{t}_1(k) e^{-ikx}          & \text{for}\ x\to-\infty ,\\
  e^{-ikx} + \mathscr{r_1}(k) e^{ikx} & \text{for}\ x\to\infty ,
\end{cases}
\end{equation}
where $\mathscr{t}_1(k)$ and $\mathscr{r}_1(k)$ are the transmission and reflection coefficient,
respectively. Analogously, $u_-$ describes scattering from $-\infty$
\begin{equation}\label{stl02}
  u_-(x) \sim
\begin{cases}
  \mathscr{t}_2(k) e^{ikx}           & \text{for}\ x\to\infty ,\\
  e^{ikx} + \mathscr{r}_2(k) e^{-ikx} & \text{for}\ x\to-\infty .
\end{cases}
\end{equation}
It can be shown that $\mathscr{t}_1(k)=\mathscr{t}_2(k)\eqqcolon \mathscr{t}(k)$, which is reasonable on physical grounds since the wave
moves through the entire potential. In order to describe the asymptotics in \eqref{stl01} and \eqref{stl02}
we use the so-called Jost solutions $\psi_\pm(k;\cdot)$ of \eqref{st01}
\begin{equation}\label{stl03}
  \psi_+(k;x) \sim e^{ikx},\ \text{for}\ x\to\infty,\
  \psi_-(k;x) \sim e^{-ikx},\ \text{for}\ x\to-\infty .
\end{equation}
Their existence and properties can be obtained via the Lippmann--Schwinger equation
\begin{equation}\label{lippmann_schwinger}
\begin{aligned}
  \psi_+(k;x) & = e^{ikx} + \frac{1}{k}\int_x^\infty \sin(k(y-x)) V(y) \psi_+(k;y)\, dy,\ x\in\R ,\\
  \psi_-(k;x) & = e^{-ikx} + \frac{1}{k}\int_{-\infty}^x \sin(k(x-y)) V(y) \psi_-(k;y)\, dy,\ x\in\R .
\end{aligned}
\end{equation}
In order to study their analytical properties it is more convenient to consider the functions
\begin{equation}\label{stl03_modified}
  m_+(k;x) \coloneqq e^{-ikx}\psi_+(k;x)\ \text{and}\ m_-(k;x)\coloneqq e^{ikx}\psi_-(k;x) . 
\end{equation}
They satisfy the equations
\begin{equation}\label{lippmann_schwinger_modified}
\begin{aligned}
  m_+(k;x) & = 1 + \int_x^\infty D_k(y-x)V(y)m_+(k;y)\, dy,\ x\in\R, \\
  m_-(k;x) & = 1 + \int_{-\infty}^x D_k(x-y)V(y)m_-(k;y)\, dy,\ x\in\R,
\end{aligned}
\end{equation}
with the kernel function
\begin{equation*}
  D_k(x) \coloneqq \int_0^x e^{2iky}\, dy = \frac{1}{2ik} ( e^{2ikx} - 1 ) .
\end{equation*}
We will need a version of Gronwall's lemma.

\begin{lemma}\label{volterra} 
Let $r:\R\to\C$ be bounded. For $k:\R\times\R\to\C$ define 
\begin{equation*}
  \tilde k(x,y) \coloneqq 
\begin{cases}
   \sup_{x\leq t\leq y} |k(t,y)| & \text{for}\ x\leq y , \\
   \tilde k(x,y) = 0          &  \text{otherwise} ,
\end{cases}
\end{equation*}
and assume that $\tilde{k}(x,\cdot)\in L^1(\R)$ for all $x\in\R$. Then, the Volterra equation
\begin{equation*}
  u(x) = r(x) + \int_x^\infty k(x,y) u(y)\, dy
\end{equation*}
has a unique solution $u$ satisfying
\begin{equation*}
  |u(x)-r(x)| \leq \int_x^\infty |r(y_1)| \tilde k(x,y_1)\exp\Big[ \int_x^{y_1} \tilde k(x,y_2)\, dy_2 \Big]\, dy_1.
\end{equation*}
Moreover, if $|r(x)|\leq\tilde r(x)$ where $\tilde r\in C^1(\R)$ is bounded 
and $\lim_{x\to\infty}\tilde r(x)\eqqcolon \tilde r(\infty)$ exists then
\begin{equation*}
  |u(x)-r(x)| \leq \tilde r(\infty)\exp\big[\int_x^\infty \tilde k(x,y)\, dy \big] - \tilde r(x)
               -\int_x^\infty \tilde r'(y_1) \exp\big[ \int_x^{y_1} \tilde k(x,y_2)\, dy_2\big] \, dy_1 .
\end{equation*}
\end{lemma}
\begin{proof}
We only go through the major steps. Unique solvability follows via successive iteration. 
We define $u_0\coloneqq r$ and
\begin{equation*}
  u_{n+1}(x) \coloneqq \int_x^\infty k(x,y) u_n(y)\, dy,\ n\in\N_0 .
\end{equation*}
The solution can then be written as
\begin{equation*}
  u(x) - r(x) = \sum_{n=1}^\infty u_n(x) .
\end{equation*}
In order to ensure uniform convergence we show the estimate
\begin{equation*}
  |u_n(x)| \leq \frac{1}{(n-1)!} \int_x^\infty |r(y_1)|\tilde k(x,y_1) \big[ \int_x^{y_1} \tilde k(x,y_2)\, dy_2\big]^{n-1}\, dy_1,\ n\geq 1.
\end{equation*}
This is obviously true for $n=1$. Now, for $n+1$
\begin{equation*}
\begin{split}
  |u_{n+1}(x)| 
    & \leq \int_x^\infty \tilde k(x,y) |u_n(y)| \, dy \\
    & \leq \frac{1}{(n-1)!}\int_x^\infty \tilde k(x,y_1) \int_{y_1}^\infty |r(y_2)| \tilde k(y_1,y_2)
              \big[\int_{y_1}^{y_2} \tilde k(y_1,y_3)\, dy_3\big]^{n-1}\, dy_2\, dy_1 .
\end{split}
\end{equation*}
Since $x\leq y_1$ we have $\tilde k(y_1,y)\leq \tilde k(x,y)$ and thus
\begin{equation*}
\begin{split}
  |u_{n+1}(x)| 
     & \leq \frac{1}{(n-1)!}\int_x^\infty \tilde k(x,y_1) \int_{y_1}^\infty |r(y_2)| \tilde k(x,y_2)
              \big[\int_{y_1}^{y_2} \tilde k(x,y_3)\, dy_3\big]^{n-1}\, dy_2\, dy_1\\
     & = \frac{1}{(n-1)!} \int_x^\infty |r(y_2)|\tilde k(x,y_2) 
           \int_x^{y_2} \tilde k(x,y_1)\big[ \int_{y_1}^{y_2} \tilde k(x,y_3)\, dy_3\big]^{n-1}\, dy_1\, dy_2\\
     & = - \frac{1}{n!} \int_x^\infty |r(y_2)|\tilde k(x,y_2)
           \int_x^{y_2}\frac{\partial}{\partial y_1} \big[ \int_{y_1}^{y_2}\tilde k(x,y_3)\, dy_3\big]^n \, dy_1\, dy_2\\
     & = \frac{1}{n!} \int_x^\infty |r(y_2)|\tilde k(x,y_2)\big[ \int_x^{y_2}\tilde k(x,y_3)\, dy_3\big]^n\, dy_2 .
\end{split}
\end{equation*}
That proves the bound. Now,
\begin{equation*}
\begin{split}
  |u(x)-r(x)|
    & \leq \sum_{n=1}^\infty |u_n(x)|\\
    & \leq \sum_{n=1}^\infty \frac{1}{(n-1)!} \int_x^\infty r(y_1)\tilde k(x,y_1)\big[ \int_x^{y_1}\tilde k(x,y_2)\, dy_2\big]^{n-1}\, dy_1\\
    &  =   \int_x^\infty r(y_1) \tilde k(x,y_1)\exp\big[\int_x^{y_1}\tilde k(x,y_2)\, dy_2\big]\, dy_1
\end{split}
\end{equation*}
which shows the estimate. If we estimate further $r\leq\tilde r$ the second bound follows simply via an integration by parts.
\end{proof}

We note some fundamental properties of the Jost solutions starting with the $x$-space properties.

\begin{lemma}\label{stl01t}
Let $V\in L^1(\R)$. Then, for all $\im(k)\geq 0$, $k\neq 0$ the Lippmann--Schwinger equations \eqref{lippmann_schwinger_modified} 
have a unique solution that solves \eqref{st01} and has the asymptotics \eqref{stl03}. We have the estimate
\begin{equation}\label{stl01t01}
  |m_+(k;x) - 1 | \leq \exp\big[ \frac{1}{|k|}\int_x^\infty |V(y)|\, dy\big] - 1,\ x\in\R .
\end{equation}
In particular,
\begin{equation}\label{stl01t02}
  |m_+(k;x)| \leq \exp\big[ \frac{1}{|k|} \|V\|_1\big],\ x\in\R .
\end{equation}
\end{lemma}
\begin{proof}
Cf. \cite[2. Lemma 1, (i)]{DeiftTrubowitz1979}.
We use Lemma \ref{volterra}. The estimate
\begin{equation*}
  | D_k(x) | \leq \frac{1}{|k|},\ \im(k)\geq 0,\ k\neq 0,
\end{equation*}
immediately implies \eqref{stl01t01}.
\end{proof}

The restriction $k\neq 0$ in Lemma \ref{stl01t} can be removed if the potential $V$ falls off fast enough.
The bound \eqref{stl01t01} is replaced by, actually, two bounds: one reflecting the correct asymptotic behaviour
at $x=+\infty$ and the other one at $x=-\infty$.

\begin{lemma}\label{stl02t}
Let $V\in L^1(\R)$ satisfy $XV\in L^1(\R)$.
Then, for all $\im(k)\geq 0$ the Lippmann--Schwinger equations \eqref{lippmann_schwinger_modified} 
have a unique solution that solves \eqref{st01} and has the asymptotics \eqref{stl03}. We have the bounds
\begin{align}\label{stl02t01}
  |m_+(k;x) - 1 | & \leq \exp\big[ \int_x^\infty (y-x)|V(y)|\, dy \big] - 1 , \\
\label{stl02t02}
       |m_+(k;x)| & \leq 2 (1+|x|) e^{\|V\|_1 + 2\|XV\|_1} .
\end{align}
\end{lemma}
\begin{proof}
In Lemma \ref{volterra} we use the estimate
\begin{equation*}
  | D_k(x) | \leq |x| ,\ \im(k)\geq 0,
\end{equation*}
which immediately yields \eqref{stl02t01}.
Whereas \eqref{stl02t01} displays the correct behaviour for $x\to+\infty$ the bound behaves like $e^{|x|}$ for $x\to-\infty$.
Therefore, in a second step, we refine our bound. To begin with,
\begin{equation*}
  |m_+(k;x)| \leq \exp\big[ \int_x^\infty y|V(y)|\, dy \big] , \ x\geq 0 .
\end{equation*}
Furthermore,
\begin{equation*}
  |m_+(k;x)|
    \leq 1 + \int_x^\infty y |V(y)| |m_+(k;y)|\, dy - x\int_x^\infty |V(y)||m_+(k;y)|\, dy .
\end{equation*}
The first integral can be bounded uniformly in $x\in\R$
\begin{equation*}
\begin{split}
  \int_x^\infty y |V(y) |m_+(k;y)| \, dy
    & \leq \int_0^\infty y |V(y)| |m_+(k;y)| \, dy\\
    & \leq \int_0^\infty y_1 |V(y_1)| \exp\big[ \int_{y_1}^\infty y_2 |V(y_2)|\, dy_2\big] \, dy_1\\
    & = \exp\big[ \int_0^\infty y |V(y)|\, dy\big] - 1\\
    & \leq e^{\|XV\|_1} .
\end{split}
\end{equation*}
Thereby,
\begin{equation*}
\begin{split}
  \frac{|m_+(k;x)|}{1+|x|} 
     & \leq \frac{1}{1+|x|}\big[ 1 + e^{\|XV\|_1} \big] - \frac{x}{1+|x|} \int_x^\infty (1+|y|)|V(y)| \frac{|m_+(k;y)|}{1+|y|}\, dy \\
     & \leq 2 e^{\|XV\|_1} + \int_x^\infty (1+|y|)|V(y)| \frac{|m_+(k;y)|}{1+|y|}\, dy .
\end{split}
\end{equation*}
Lemma \ref{volterra} implies
\begin{equation*}
  \frac{|m_+(k;x)|}{1+|x|} 
    \leq 2 e^{\|XV\|_1} \exp\big[\int_x^\infty (1+|y|) |V(y)|\, dy\big]
    \leq 2 e^{\|V\|_1 + 2\|XV\|_1} ,
\end{equation*}
which yields \eqref{stl02t02}.
\end{proof}

We study the properties of the Jost solutions as functions of $k$.

\begin{lemma}\label{stl03t}
For each $x\in\R$, $m_+(\cdot,x)$ is analytic for $\im(k) > 0$ and continuous for $\im(k)\geq 0$, $k\neq 0$.
If, in addition, $XV\in L^1(\R)$ then $m(\cdot,x)$ is continuous for all $\im(k)\geq 0$. Furthermore,
\begin{equation}\label{stl03t01}
  |\dot m_+(k;x)| \leq \frac{2}{|k|} \exp\Big[\frac{2}{|k|} \|V\|_1\Big]\,\int_x^\infty (y-x)|V(y)|\, dy 
     ,\ \im(k)\geq 0,\ k\neq 0.
\end{equation}
\end{lemma}
\begin{proof}
See \cite[2. Lemma 1, (v), p.130]{DeiftTrubowitz1979}.
We differentiate \eqref{lippmann_schwinger_modified} by $k$ and obtain
\begin{equation*}
  \dot m_+(k;x) = r(x) + \int_x^\infty D_k(y-x)V(y)\dot m_+(k;y)\, dy,\ 
     r(x)\coloneqq \int_x^\infty \dot D_k(y-x)V(y)m_k(k;y)\, dy .
\end{equation*}
We integrate by parts
\begin{equation*}
  \dot D_k(x) = 2i \int_0^x e^{2iky} y\, dy
              = \frac{1}{k}\big[ e^{2iky}y\big]_0^x - \frac{1}{k}\int_0^x e^{2iky} \, dy
\end{equation*}
and obtain the estimate
\begin{equation*}
  |\dot D_k(x)| \leq \frac{1}{|k|} |x| + \frac{1}{|k|} |x| = \frac{2}{|k|}|x| .
\end{equation*}
Furthermore, using \eqref{stl01t01} we obtain
\begin{equation*}
  |r(x)| \leq \frac{2}{|k|} \int_x^\infty y_1|V(y_1)|\exp\big[ \frac{1}{|k|}\int_{y_1}^\infty |V(y_2)|\, dy_2\big]\, dy_1 \eqqcolon \tilde r(x) .
\end{equation*}
Note that $\tilde r(\infty)=0$. Using $e^x-1 \leq xe^x$ for $x\geq0$ we obtain
\begin{equation*}
\begin{split}
  -\tilde r'(x) 
    & = \frac{2}{|k|} \int_x^\infty |V(y_1)| \exp\big[\frac{1}{|k|}\int_{y_1}^\infty |V(y_2)|\, dy_2\big] \, dy_1\\
    & = 2 \big\{ \exp\big[ \frac{1}{|k|}\int_x^\infty |V(y)|\, dy\big] - 1 \big\}\\
    & \leq \frac{2}{|k|}\int_x^\infty |V(y)|\, dy \,\cdot\,\exp\big[ \frac{1}{|k|}\int_x^\infty |V(y)|\, dy\big] .
\end{split}
\end{equation*}
Since $\tilde k(x,y) \leq \frac{1}{|k|}|V(y)|$ we infer from Lemma \ref{volterra} that
\begin{equation*}
\begin{split}
  |\dot m_+(k;x)|
     & \leq \frac{2}{|k|} \int_x^\infty \int_{y_1}^\infty |V(y_2)|\, dy_2 \exp\big[\frac{1}{|k|}\int_{y_1}^\infty |V(y_3)|\, dy_3\big]
                   \exp\big[\frac{1}{|k|}\int_x^{y_1} |V(y_4)|\, dy_4\big] \, dy_1\\
     & \leq \frac{2}{|k|} \int_x^\infty \int_{y_1}^\infty |V(y_2)|\, dy_2\, dy_1 \exp\big[\frac{2}{|k|}\int_x^\infty |V(y)|\, dy\big] .
\end{split}
\end{equation*}
This implies \eqref{stl03t01}.
\end{proof}                           

The prescribed asymptotics in \eqref{stl01} and \eqref{stl02} imply that
\begin{equation}\label{stl04}
  \mathscr{t}(k)\psi_-(k;x) = \psi_+(-k;x) + \mathscr{r}_1(k) \psi_+(k;x) ,\
  \mathscr{t}(k)\psi_+(k;x) = \psi_-(-k;x) + \mathscr{r}_2(k) \psi_-(k;x) .
\end{equation}
A first consequence, which will be needed below, is a relation between the transmission coefficient and
a Wronski determinant
\begin{equation}\label{wronski01}
\begin{vmatrix}
  \psi_-(k,\cdot)  & \psi_+(k,\cdot) \\
  \psi_-'(k,\cdot) & \psi_+'(k,\cdot)
\end{vmatrix}
  = \frac{2ik}{\mathscr{t}(k)} .
\end{equation}
By computing the Wronski determinants of $u_\pm$ with $\varepsilon_{1,2}$, the solutions \eqref{deficiency_subspace}
of the unperturbed differential equation, one can further show that (cf. \cite[pp. 145,146]{DeiftTrubowitz1979})
\begin{equation}\label{s-matrix01}
\begin{aligned}
  \frac{1}{\mathscr{t}(k)}   & = 1 - \frac{1}{2ik} (e_1(k), Jf_1(k))
                & \frac{\mathscr{r}_1(k)}{\mathscr{t}(k)} & = \frac{1}{2ik} (e_1(k), Jf_2(k))\\
  \frac{\mathscr{r}_2(k)}{\mathscr{t}(k)} & = \frac{1}{2ik} ( e_2(k), Jf_1(k))
                & \frac{1}{\mathscr{t}(k)}   & = 1 - \frac{1}{2ik} ( e_2(k), Jf_2(k))
\end{aligned}
\end{equation}
where
\begin{gather*}
  e_1(k;x)\coloneqq\sqrt{|V(x)|}e^{ikx},\ e_2(k;x)\coloneqq\sqrt{|V(x)|}e^{-ikx},\\
  f_1(k;x)\coloneqq\sqrt{|V(x)|}\psi_+(k;x),\ f_2(k;x)\coloneqq\sqrt{|V(x)|}\psi_-(k;x) .
\end{gather*}
The $f_{1,2}(k)$ satisfy the symmetrized Lippmann--Schwinger equation
\begin{equation}\label{lippmann_schwinger_sym}
  f_1(k) =  e_1(k) + \frac{1}{k} \sqrt{|V|}G_+(k)\sqrt{|V|}J f_1(k),\
  f_2(k) =  e_2(k) + \frac{1}{k} \sqrt{|V|}G_-(k)\sqrt{|V|}J f_2(k) .
\end{equation}
The operators $G_\pm(k)$ are given through their kernels
\begin{equation*}
   G_+(k;x,y) \coloneqq \Theta(y-x) \sin(k(y-x)),\
   G_-(k;x,y) \coloneqq \Theta(x-y) \sin(k(x-y)) .
\end{equation*}
They differ from the resolvent by a rank one operator
\begin{equation*}
\begin{split}
   \sqrt{|V|}R_\infty^+(k^2)\sqrt{|V|}J 
      = - \frac{i}{2k} (Je_1(k),\cdot)e_1(k) + \frac{1}{k} \sqrt{|V|}G_+(k)\sqrt{|V|}J \\
      = - \frac{i}{2k} (Je_2(k),\cdot)e_2(k) + \frac{1}{k} \sqrt{|V|}G_-(k)\sqrt{|V|}J .
\end{split}
\end{equation*}
Thereby, the Lippmann--Schwinger equation can be rewritten as (cf. \eqref{Omega}, \eqref{Omega_boundary_values})
\begin{equation*}
   f_1(k) = ( 1 + \frac{i}{2k} (e_1(k), Jf_1(k)) ) \Omega_\infty^+(k^2)  e_1(k),\
   f_2(k) = ( 1 + \frac{i}{2k} (e_2(k), Jf_2(k)) ) \Omega_\infty^+(k^2)  e_2(k) .
\end{equation*}
Taking scalar products we can express $\mathscr{t}(k)$, $\mathscr{r}_{1,2}(k)$ through the matrix $\hat\Omega_{\infty}^+(k)$ from \eqref{Omega_matrix}.
Recall that $k=\sqrt{\nu}$. With the unitary scattering matrix
\begin{equation}\label{s-matrix02}
  \mathscr{S}(k) \coloneqq
\begin{pmatrix}
   \mathscr{t}(k)   & \mathscr{r}_1(k) \\
   \mathscr{r}_2(k) & \mathscr{t}(k)
\end{pmatrix}
  ,\ 
  |\mathscr{t}(k)|^2 + |\mathscr{r}_1(k)|^2 = 1 = |\mathscr{t}(k)|^2 + |\mathscr{r}_2(k)|^2,\ 
  \bar{\mathscr{t}}(k)\mathscr{r}_1(k) + \bar{\mathscr{r}}_2(k)\mathscr{t}(k) = 0 ,
\end{equation}
we finally obtain
\begin{equation}\label{Omega_T}
  \hat\Omega_\infty^+(k^2) = 2ik\mathscr{T}(k),\ \mathscr{T}(k)\coloneqq  \mathscr{S}(k)-\id,\ k\in\R ,
\end{equation}
where $\mathscr{T}(k)$ is the so-called T-matrix.
The transmission coefficient $\mathscr{t}(k)$ is, essentially, determined by its values for $k\in\R$.

\begin{lemma}[Faddeev--Deift--Trubowitz formula]\label{deift_trubowitz} 
Let $V\in L^1(\R)$ satisfy $X^2V\in L^1(\R)$ and 
let $-\beta_j^2$, $\beta_j>0$, $j=1,\ldots,n$, be the negative eigenvalues of $H_V$. Then,
\begin{equation}\label{deift_trubowitz01}
  \mathscr{t}(k) = \exp\Big[ \frac{1}{\pi i}\int_\R \frac{\ln|\mathscr{t}(u)|}{u-k}\, du \Big]
          \prod_{j=1}^n \frac{k+i\beta_j}{k-i\beta_j} ,\ \im(k)>0 .
\end{equation}
\end{lemma}
\begin{proof}
Cf. \cite[p. 323]{Faddeev1964} and \cite[p. 154]{DeiftTrubowitz1979}. 
Schwarz's integral formula for the half-plane expresses a holomorphic function through its real part.
Apply that to the function $k\mapsto\ln(\mathscr{t}(k))$.
\end{proof}

Moreover, the transmission coefficient can be expressed by the 
perturbation determinant, see \cite{JostPais1951}.

\begin{lemma}[Jost--Pais formula]\label{jost_pais}
Assume that the Birman--Schwinger operator $K(z)\in B_1(\hilbert)$. Then,
\begin{equation}\label{jost_pais01}
  \frac{1}{\mathscr{t}(k)} = \det(\id - K(z)),\ z=k^2,\ k>0 .
\end{equation}
In particular, for $z>0$
\begin{equation}\label{jost_pais02}
  |\det(\id-K(z))| \geq 1 .
\end{equation}
\end{lemma}
\begin{proof}
Following \cite[App. A]{Newton1980} we introduce a coupling parameter and study the function
\begin{equation*}
  D(\alpha) \coloneqq \det(\id - \alpha K),\ 0\leq \alpha\leq 1 .
\end{equation*}
Its logarithmic derivative is (cf. \cite[(1.7.10)]{Yafaev1992})
\begin{equation*}
  \frac{d}{d\alpha}\ln(D(\alpha)) = - \tr[ (\id - \alpha K)^{-1}K ] .
\end{equation*}
Since $K$ is a bounded operator a Neumann series argument shows that the inverse 
exists for $0\leq\alpha\leq\alpha_0$ with some $\alpha_0>0$.
We rewrite the operator
\begin{equation*}
  (\id -\alpha K)^{-1}K 
    = (\id - \alpha \sqrt{|V|}R_\infty\sqrt{|V|}J)^{-1} \sqrt{|V|}R_\infty\sqrt{|V|}J
    = \sqrt{|V|}( k^2\id - H-\alpha V)^{-1}\sqrt{|V|}J .
\end{equation*}
Note that the inverse exists even though it is not a bounded operator. Thus,
\begin{equation*}
  \frac{d}{d\alpha} \ln(D(\alpha)) = - \tr[ \sqrt{|V|}(k^2\id - H - \alpha V)^{-1}\sqrt{|V|}J ] .
\end{equation*}
This trace can be expressed through the Jost solutions. To this end, we need the Green function (cf. \eqref{wronski01})
\begin{equation*}
 ( k^2\id - H - \alpha V)^{-1}(x,y) = \frac{\mathscr{t}_\alpha}{2ik}
\begin{cases}
  \psi_+(x)\psi_-(y) & \text{for}\ x\geq y , \\
  \psi_-(x)\psi_+(y) & \text{for}\ x < y .
\end{cases}
\end{equation*}
Here, $\mathscr{t}_\alpha$ is the transmission coefficient corresponding to the potential $\alpha V$ (instead of $V$ as above). Thus,
\begin{equation*}
  \frac{d}{d\alpha} \ln(D(\alpha))
     = - \frac{\mathscr{t}_\alpha}{2ik} \int_\R \sqrt{|V(x)|}\psi_+(x)\psi_-(x)\sqrt{|V(x)|}J(x)\, dx
     = - \frac{\mathscr{t}_\alpha}{2ik} \int_\R f_1(x)f_2(x) J(x)\, dx .
\end{equation*}
On the other hand, we have
\begin{equation*}
  \frac{1}{\mathscr{t}_\alpha} = 1 - \frac{\alpha}{2ik} (e_1,Jf_1),\ f_{1,2} = e_{1,2} + \alpha\sqrt{|V|}G_\pm \sqrt{|V|}J f_{1,2}
\end{equation*}
and consequently
\begin{equation*}
  - \frac{\dot{\mathscr{t}}_\alpha}{\mathscr{t}_\alpha^2} = - \frac{1}{2ik}(e_1,J(f_1+\alpha\dot f_1))
\end{equation*}
where the dot denotes differentiation with respect to $\alpha$. Differentiating the Lippmann--Schwinger equation
one obtains after some simple calculations
\begin{equation*}
  f_1 + \alpha \dot f_1 = (\id - \alpha\sqrt{|V|}G_+ \sqrt{|V|}J)^{-1} f_1 .
\end{equation*}
Therefore,
\begin{equation*}
  \frac{\dot{\mathscr{t}}_\alpha}{\mathscr{t}_\alpha^2}
    = \frac{1}{2ki} ( e_1, J(\id-\alpha\sqrt{|V|}G_+\sqrt{|V|}J)^{-1}f_1)
    = \frac{1}{2ik} ((\id - \alpha\sqrt{|V|}G_-\sqrt{|V|}J)^{-1}e_1,J f_1)
    = \frac{1}{2ik} ( \bar f_2,Jf_1)
\end{equation*}
where we used $e_1 = \bar e_2$. We conclude that for $0\leq\alpha\leq \alpha_0$,
\begin{equation*}
  \frac{d}{d\alpha}\ln(D(\alpha)) = - \frac{d}{d\alpha}\ln(\mathscr{t}_\alpha) .
\end{equation*}
For $\alpha=0$ both quantities have the same value. Furthermore, since $D(\alpha_0)=\frac{1}{\mathscr{t}_{\alpha_0}}\neq 0$
we can extend the result to $0\leq\alpha\leq 1$ which proves \eqref{jost_pais01}.
The estimate \eqref{jost_pais02} follows immediatetly from $|\mathscr{t(k)}|\leq 1$ for $k>0$.
\end{proof}
