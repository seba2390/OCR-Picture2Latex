On the half-line there cannot be a transmission cofficient since the wave is entirely reflected at the origin. 
Hence, conditions \eqref{stl01} and \eqref{stl02} are to be replaced by (cf. \eqref{hl01})
\begin{equation}\label{sthl01}
  u(x) \sim e^{-ikx} - \frac{ia-kb}{ia+kb} S(k)e^{ikx}\ \text{as}\ x\to\infty,\ a u(0) - bu'(0) =0 ,
\end{equation}
which reads in terms of the Jost solutions
\begin{equation}\label{sthl02}
  u(x) = \psi(-k;x) - \frac{ia-kb}{ia+kb} S(k)\psi(k;x) .
\end{equation}
Conditions \eqref{sthl01} and \eqref{sthl02} have been chosen so that $S(k)=1$ when $V=0$ (cf. \eqref{hl02}).
The Lippmann--Schwinger equation is the same only this time it is restricted to the positive axis
\begin{equation}\label{sthl03}
  \psi(k;x) = e^{ikx} + \frac{1}{k}\int_x^\infty \sin(k(y-x))V(y) \psi(k;y)\, dy,\ x\geq 0 .
\end{equation}
Note that these $\psi$ are the restrictions of the Jost solutions $\psi_+$ defined on the line.
As in Section \ref{stl} it is convenient to work with the symmetrized Lippmann--Schwinger equation
\begin{equation}\label{sthl04}
  f_j = e_j + \frac{1}{k}\sqrt{|V|}G_+\sqrt{|V|} f_j,\ j=1,2,
\end{equation}
with
\begin{equation*}
  f_1(x) \coloneqq \sqrt{|V(x)|}\psi(k;x),\ f_2(x) \coloneqq \sqrt{|V(x)|}\psi(-k;x) .
\end{equation*}
Note that $f_2$ is different from that on the whole line.
Using the boundary conditions one can derive the scattering matrix from \eqref{sthl02}
\begin{equation*}
  S(k) = c \frac{a\psi(-k;0) - b\psi'(-k;0)}{a\psi(k;0)-b\psi'(k;0)},\ 
    c \coloneqq \frac{ia + kb}{ia-kb} = \frac{i\bar a + k\bar b}{i\bar a-k\bar b},\
    |c|^2 = 1 .
\end{equation*}
The values at $0$ can be obtained via the Lippmann--Schwinger equation \eqref{sthl03} and \eqref{sthl04}
\begin{equation*}
  \psi(k;0) = 1 + \frac{1}{k}(e_s,Jf_1),\ \psi'(k;0) = ik - (e_c,Jf_1),\
  \psi(-k;0) = 1 + \frac{1}{k}(e_s,Jf_2),\ \psi'(-k;0) = -ik - (e_c,Jf_2)
\end{equation*}
where
\begin{equation*}
  e_s(x) \coloneqq \sqrt{|V(x)|}\sin(kx),\ e_c(x)\coloneqq\sqrt{|V(x)|}\cos(kx) .
\end{equation*}
We rewrite the scattering matrix
\begin{equation*}
  S(k) = c \frac{ a+ikb + \frac{1}{k}(\bar a e_s+k\bar b e_c,Jf_2) }{ a-ikb + \frac{1}{k}(\bar a e_s+k\bar b e_c,Jf_1)}
       = \frac{i-\frac{1}{2k}(e_0,Jf_2)}{i-\frac{1}{2kc}(e_0,Jf_1)}
\end{equation*}
where (see \eqref{hl02})
\begin{equation*}
  e_0\coloneqq \sqrt{|V|}\varepsilon_0,\ \bar a e_s + k\bar b e_c = - \frac{1}{2}(i\bar a+k\bar b)e_0 .
\end{equation*}
The T-matrix is then
\begin{equation*}
  T(k) = S(k) - 1 = \frac{1}{2ki} \frac{(e_0,Jf_0)}{1 + \frac{i}{2kc}(e_0,Jf_1)},\
  f_0 \coloneqq \frac{1}{c}f_1 - f_2 .
\end{equation*}
By linearity, $f_0$ satisfies the Lippmann--Schwinger equation with $e_0$ instead of $e_{1,2}$.

A look at the kernel functions shows that the resolvent is a rank one perturbation of the Lippmann--Schwinger operator
which yields for the Birman--Schwinger operator
\begin{equation*}
  \sqrt{|V|}R_\infty\sqrt{|V|}J 
     = \frac{i}{2k} (J\bar e_0,\cdot) e_1 + \frac{1}{k}\sqrt{|V|}G_+\sqrt{|V|}J
     = - \frac{i}{2ku} (Je_0,\cdot) e_1 + \frac{1}{k}\sqrt{|V|}G_+\sqrt{|V|}J .
\end{equation*}
Thereby, the Lippmann--Schwinger equation for $f_0$ gives
\begin{equation*}
  f_0 = \Omega_\infty e_0 + \frac{i}{2kc} (e_0,Jf_0)\Omega_\infty e_1 .
\end{equation*}
Taking scalar products one obtains
\begin{equation*}
  (1-\frac{i}{2kc}(e_0,J\Omega_\infty e_1))(e_0,Jf_0) = (e_0,J\Omega_\infty e_0) .
\end{equation*}
Likewise for $f_1$
\begin{equation*}
  f_1 = (1 + \frac{i}{2kc}(e_0,Jf_1)) \Omega_\infty e_1 .
\end{equation*}
Once again, we take scalar products and rearrange the terms 
\begin{equation*}
  (1+\frac{i}{2kc}(e_0,Jf_1))(1-\frac{i}{2kc}(e_0,J\Omega_\infty e_1) = 1 .
\end{equation*}
We use the formulae for $f_0$ and $f_1$ to obtain
\begin{equation*}
  T(k) = \frac{1}{2ki} \frac{(e_0,J\Omega_\infty e_0)}{(1-\frac{i}{2kc}(e_0,J\Omega_\infty e_1))(1 + \frac{i}{2kc}(e_0,Jf_1))}
       = \frac{1}{2ki} (e_0,J\Omega_\infty e_0) .
\end{equation*}
Finally, we have established the relation between the wave operator \eqref{Omega_matrix} and the T-matrix
\begin{equation*}
  T(k) = \frac{1}{2ki} \hat\Omega_\infty(\nu) .
\end{equation*}
Recall that $k=\sqrt{\nu}$.
