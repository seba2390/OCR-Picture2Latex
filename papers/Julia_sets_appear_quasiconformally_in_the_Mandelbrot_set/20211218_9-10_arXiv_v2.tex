\section{Proof of Corollary D}
\paragraph{\bf Proof.}
We recall the setting of Theorem A. Take any small Mandelbrot set
$M_{s_0}$, where $s_0 \ne 0$ is a superattracting parameter and
take any Misiurewicz or parabolic parameter $c_0 \in \partial M$.
Then Theorem A shows that ${\mathcal M}(c_0+\eta)$ appears quasiconformally
in $M$ in a small neighborhood of $c_1 := s_0 \perp c_0$. Now let
$c$ be a parameter which belongs to the quasiconformal image of the 
decoration of ${\mathcal M}(c_0+\eta)$. This means that
$$
  G_c^k(0) \in Y:=\Theta_c(\Gamma_0(c_0+\eta)) = J(f_c) \quad 
  \text{for some} \ k \in \N.
$$
Since $G_c = P_c^{p'}$ for some $p' \in \N$, we have
$P_c^{p'k}(0) \in Y$. On the other hand, 
$Y$ is $f_c$ ($=P_c^p$)-invariant, that is, 
$P_c^{pn}(Y) \subseteq Y$ for every $n$. Then
for a fixed $0 \leq  r < p$, we have
$P_c^{pn+r}(Y) \subseteq P_c^r(Y)$ for every $n$ and each $P_c^r(Y)$ is 
apart from $0$. Therefore for every $i \geq p'k_0$ we have
$P_c^i(Y) \in \bigcup_{r=0}^{p-1} P_c^r(Y)$, which implies that
the orbit of $0$ under the iterate of $P_c$ does not accumulate 
on $0$ itself. Moreover, $P_c$ has no parabolic periodic point
since $Y = J(f_c)$ is a Cantor Julia set of a hyperbolic quadratic-like
map $f_c : U_c' \to U_c$. 
This shows that $P_c$ is semihyperbolic. 
Since Misiurewicz or parabolic parameters are dense in $\partial M$, we can
find decorations in every small neighborhood of every point in $\partial M$. 
Also there are only countably many Misiurewicz parameters. Hence it follows 
that the semihyperbolic parameters which are not Misiurewicz and 
non-hyperbolic are dense in $\partial M$.
\QED











\section{Concluding Remarks}
We have shown that we can see quasiconformal images of some Julia sets
in the Mandelbrot set $M$. But this is not satisfactory, because these 
images are all Cantor sets and disconnected. On the other hand, $M$ is 
connected and so what we have detected is only a small part of the whole
structure of $M$. From computer pictures, it is observed that the points
in these Cantor sets are connected by some complicated filament structures.
This looks like a picture which is obtained from the picture of $K_c$ 
for $c \in {\rm int} (M_{s_1})$ by replacing all small filled Julia sets
with small Mandelbrot sets.
It would be interesting to explain this mathematically.
A similar phenomena as in the quadratic family are observed also in the 
unicritical family $\{ z^d+c \}_{c \in \C}$ by computer pictures. 
These phenomena should
be proved in the same manner as for the quadratic case.

