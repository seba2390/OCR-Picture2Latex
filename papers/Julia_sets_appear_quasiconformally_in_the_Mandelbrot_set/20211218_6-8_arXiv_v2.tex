\setcounter{section}{5}
%%%%%%%%%%%%%%%%%%%%%%
\section{Proof of Theorem B}
Let $M_{s_1}$ be the main Mandelbrot set of the quasiconformal copy
of the decorated Mandelbrot set $\cM(c_0+\eta)$ given in Theorem A. 
Choose any $c \in M$ and set $\sig:=s_1 \perp c \in M_{s_1}$. 
(For example, 
let $c_0$ be the Misiurewicz parameter for which $P_{c_0}^5(0)=P_{c_0}^4(0)$ and $c$ the parameter for Douady's rabbit
as in Figure \ref{nested structure for a filled Julia set}.)


Let $\Phi_{c}:\C \sminus K_{c} \to \C  \sminus \Bar{\D}$ be 
the B\"ottcher coordinate for $P_c$. 
For any $R>1$ with 
$J_{c_0 + \eta} \subset A(R^{-1/2}, R^{1/2})$,
we take the Jordan domains 
$\Omega_1'$ and $\Omega_1$ in $\C$ with $\Omega_1' \Subset \Omega_1$
whose boundaries are the inner and the outer boundaries of
$\Phi_{c}^{-1}(A(R, R^{2}))$.
(That is, we take Douady's radii $\rho'=R^{-1/2}$ and $\rho=R^{1/2}$
in the definition of rescaled Julia set. See section 2.) 
Then $P_{c}:\Omega_1' \to \Omega_1$ is a 
quadratic-like restriction of $P_{c}$, 
and the decorated filled Julia set $\cK_{c}(c_0+\eta)=\cK_{c}(c_0+\eta)_{R^{-1/2},R^{1/2}}$ 
is a compact set in $\Omega_1$.

Now we want to show that for $\sig=s_1 \perp c \in M_{s_1}$
the filled Julia set $K_{\sig}$ contains 
a quasiconformal copy $\cK$ of the model set $\cK_{c}(c_0+\eta)$.
Consider the quadratic-like maps
$f_{\sig}: U_{\sig}' \to U_{\sig}$ and $G_{\sig}: V_{\sig}' \to U_{\sig}$ 
given in the proof of Theorem A.
Since we have $J(f_{\sig}) \subset U_{\sig}  \sminus  \Bar{V_{\sig}'}$,
the set
$$
{\Gamma}:= \bigcup_{m \ge 0} G_{\sig}^{-m}(J(f_{\sig}))
$$
encages the filled Julia set $K(G_{\sig}) \subset V_{\sig}'$.
Let $\cK$ be the union $K(G_{\sig})  \cup \Gamma$, 
which is a compact subset of $U_{\sig}$.
Then the boundary $\partial \cK$ is contained in $\partial K_{\sig}$,
since the set of points that eventually lands on 
a repelling cycle of $f_\sigma$ or $G_\sigma$
is dense in $\partial \cK$.
Hence it is enough to show that there exists a 
quasiconformal map on a domain 
that maps the model set $\cK_{c}(c_0+\eta)$ to $\cK$.

Let $h=h_{\sig}:U_{\sig} \to \C$ be 
a straightening map of $G_{\sig}:V_{\sig}' \to U_{\sig}$.
By setting $\Omega_2:=h(U_{\sig})$ and $\Omega_2':=h(V_{{\sig}}')$,
the map $P_{c}=h \cc G_{\sig} \cc h^{-1}:\Omega_2' \to \Omega_2$
is also a quadratic-like restriction of $P_{c}$
such that 
$h(K(G_{\sig}))=K_{c}$ and 
$h(J(f_\sigma)) \subset h(U_\sigma \smallsetminus \overline{V_\sigma'})
=\Omega_2\smallsetminus\overline{\Omega_2'}$.
By slightly shrinking $\Omega_2$, we may assume that 
the boundaries of $\Omega_2$ and $\Omega_2'$ are smooth Jordan curves.
Since $h$ is quasiconformal, 
it suffices to show that there exists a quasiconformal map 
$H:\Omega_1 \to \Omega_2$  
that maps the model set $\cK_{c}(c_0+\eta)$ 
onto $h(\cK)$.

Now we claim:
\begin{lem}
\label{lem_B-0}
There exists a quasiconformal homeomorphism 
$H:\Bar{\Omega_1} \sminus {\Omega_1'} \to \Bar{\Omega_2} \sminus {\Omega_2'}$
such that
\begin{itemize}
\item
$H$ is equivariant. That is, 
$P_{c}(H(z))=H(P_{c}(z))$ for any  $z \in \partial \Omega_1'$.
\item
$H$ maps 
$J^\ast:=\Phi_{c}^{-1}(J_{c_0 + \eta}\times R^{3/2})$ in the model set onto $h(J(f_{\sig}))$.
\end{itemize}
\end{lem}

\paragraph{\bf Proof.}
Since the boundary components of these annuli are smooth,
we can take a smooth homeomorphism between $\partial \Omega_1$
and $\partial \Omega_2$.
By pulling it back by the action of $P_{c}$,
we have a smooth, equivariant homeomorphism $\psi_0$ between the
boundaries of the closed annuli $\Bar{\Omega_1} \sminus {\Omega_1'}$
and $\Bar{\Omega_2} \sminus {\Omega_2'}$.

Next we consider $J^\ast$.
Recall that $\sigma \in M_{s_1} \subset W \subset \Lambda$,
where $W$ and $\Lambda$ are given in the proof of Theorem A.
Hence there exists a straightening map $\hat{h}:U_\sigma \to \C$ 
that quasiconformally conjugates 
the quadratic-like map $f_{\sig}:U_\sigma' \to U_\sigma$ 
to $P_{\sigma'}:\hat{h}(U_\sigma') \to \hat{h}(U_\sigma)$ for some 
$\sigma' \in \C \smallsetminus M$.
Consider a sequence of homeomorphisms
$$
J^\ast=\Phi_{c}^{-1}(J_{c_0 + \eta}\times R^{3/2})
\stackrel{(1)}{\longrightarrow}
J_{c_0+\eta}
\stackrel{(2)}{\longrightarrow}
J_{\sig'}
\stackrel{(3)}{\longrightarrow}
J(f_{\sig}) 
\stackrel{(4)}{\longrightarrow}
h(J(f_{\sig})), 
$$
where (1) -- (4) are given as follows: 
\begin{enumerate}
\item 
This is just a conformal map $z \mapsto R^{-3/2} \Phi_c(z)$
restricted to $J^\ast$.
\item
Take a simply connected domain $W'$ in $\C \sminus M$ 
containing $c_0+\eta$ and $\sigma'$. 
Then there exists a holomorphic motion of $J_{c_0+\eta}$
over $W'$ that gives a quasiconformal map
on the plane that sends $J_{c_0+\eta}$ to $J_{\sig'}$
by the Bers-Royden theorem (\cite[Thoerem 1]{Bers-Royden 1986}).
\item
This is $(\hat{h}|_{J(f_\sigma)})^{-1}$, which is a 
restriction of a quasiconformal map 
$\hat{h}^{-1}:\hat{h}(U_\sigma) \to U_\sigma$.
\item
This is $h|_{J(f_\sigma)}$,
which is a restriction of the quasiconformal straightening map 
${h}:U_\sigma \to h(U_\sigma)$ of the quadratic-like map 
$G_\sigma:V_\sig' \to U_\sig$.
\end{enumerate}
Hence there exists a neighborhood $D^\ast$ of $J^\ast$
and a quasiconformal map $\psi_1:D^\ast \to \C$ 
that sends $J^\ast$ to $h(J(f_{\sig}))$.
Since $J^\ast$ is a Cantor set,
we may choose $D^\ast$ such that $D^\ast$ is a finite union of smooth Jordan domains
satisfying $D^\ast \Subset \Omega_1 \sminus \overline{\Omega_1'}$ and 
$\psi_1(D^\ast) \Subset \Omega_2  \sminus \overline{\Omega_2'}$. 
Now the sets $\Omega_1  \sminus\overline{\Omega_1' \cup D^\ast}$ 
and $\Omega_2  \sminus \overline{\Omega_2'\cup \psi_1(D^\ast)}$
are multiply connected domains with the same connectivity.
By a standard argument in complex analysis
(see \cite[Chapter 6, Theorem 10]{Ahlfors 1978} for example),
they are conformally equivalent to round annuli with 
concentric circular slits, and there is a 
quasiconformal homeomorphism $\psi_2$ between these domains.
Since each component of $\psi_1(D^\ast)$ is a quasidisk,
we can modify $\psi_2$ such that the boundary correspondence 
agrees with $\psi_0$ and $\psi_1$. 
Hence we obtain a desired quasiconformal homeomorphism $H$
by gluing $\psi_0$, $\psi_1$, and this modified $\psi_2$. 
\QED {\small (Lemma 6.1)}

\medskip
By pulling back the map $H$ given in Lemma \ref{lem_B-0}
by the dynamics of $P_{c}$,
we have a unique homeomorphic extension
$H:\Bar{\Omega_1} \sminus K_{c} \to \Bar{\Omega_2} \sminus K_{c}$
such that $P_{c}(H(z))=H(P_{c}(z))$ for any $\Omega_1' \sminus K_{c}$
and that $H$ maps the decoration of $\cK_{c}(c_0+\eta)$ 
onto $h(\Gamma)$.

We employ the following lemmas. 
(For the proofs, see Lyubich's book in preparation \cite{Lyubich Book}.\footnote{It is still being updated. 
The numbers of subsections below are tentative.})

\begin{lem}[{\cite[\S 41.3]{Lyubich Book}}] %Lemma 41.8
\label{lem_B_1}
Let $f:U' \to U$ be a quadratic-like map with connected Julia set.
Let $W_1 \subset U$ and $W_2 \subset U$ be two open annuli whose
inner boundary is $J(f)$. 
Let $H:W_1 \to W_2$ be an automorphism of $f$,
that is, $f(H(z))=H(f(z))$ on $f^{-1}(W_1)$.
Then $H$ admits a continuous extension to
a map $H:W_1 \cup J(f) \to W_2 \cup J(f)$
identical on the Julia set. 
\end{lem}


\begin{lem}[\bf Bers' Gluing Lemma, {\cite[\S 13.3]{Lyubich Book}}]
\label{lem_Bers}
Let $K$ be a compact set in $\C$ and 
let $\Omega_1$ and $\Omega_2$ 
be neighborhoods of $K$ 
such that there exists two quasiconformal maps
$H_1:\Omega_1  \sminus K \to \C$ and $H_2: \Omega_2 \to \C$
that match on $\partial K$, i.e., the map
$H:\Omega_1 \to \C$ defined by $H(z):=H_1(z)$
for $z \in \Omega_1  \sminus K$ 
and $H(z):=H_2(z)$ for $z \in K$
is continuous. 
Then $H$ is quasiconformal and $\mu_H=\mu_{H_2}$
for almost every $z \in K$.
\end{lem}

Now we apply Lemma \ref{lem_B_1} by regarding $P_{c}$ and $\Omega_j \sminus K_{c}$
as $f$ and $W_j$ for each $j=1,\,2$.
It follows that the restriction $H_1:=H|_{\Omega_1 \sminus K_{c}}$
of the map 
$H:\Bar{\Omega_1} \sminus K_{c} \to \Bar{\Omega_2} \sminus K_{c}$
admits a continuous extension to 
$H_1:\Omega_1 \sminus \mathrm{int}(K_{c}) \to \Omega_2 \sminus \mathrm{int}(K_{c})$
that agrees with the identity map $H_2:=\mathrm{id}:\Omega_2 \to \Omega_2$ on $\partial K_c$.
By Lemma \ref{lem_Bers},
we have a quasiconformal map $H:\Omega_1 \to \Omega_2$ 
such that $H(\cK_{c}(c_0+\eta))=h(\cK)$.
 \QED 













\section{Almost conformal straightenings}
In this section we establish a general formulation
of quadratic-like families 
that generate ``fine" copies of the Mandelbrot set. 

The notation here (for example, the way we use $f_c, U_c', U_c, \ldots$) is different from that in the other sections. 

\begin{lem}[\bf Almost conformal straightening]
\label{lem_almost_conformal}
Fix two positive constants $r$ and $\delta$.
Suppose that for some 
$R>\max\{2(r+\delta), 6\}$ we have a function $u=u_c(z)=u(z,c)$ 
that is holomorphic with $u_c'(0)=0$ and $|u(z,c)|<\delta$
in both $z \in D(R)$ and $c \in D(r)$.
Let 
$$
f_c(z):=z^2 + c+u(z,c), 
$$
$U_c:=D(R)$ and $U_c':=f_c^{-1}(U_c)$.
Then for any $c \in D(r)$
the map
$f_c:U_c' \to U_c$ is a quadratic-like map
 with a critical point $z=0$.
Moreover, it satisfies the following properties 
for sufficiently large $R$:
\begin{enumerate}[\rm (1)]
\item
There exists a family of smooth $(1+O(R^{-1}))$-quasiconformal maps (a tubing) 
$$
\Theta=\{\Theta_c: \Bar{A(R^{1/2},R)} \to \Bar{U_{c}} \sminus {U_{c}'} \}_{c\,\in\, D(r)}
$$
such that 
\begin{itemize}
\item $\Theta_c$ is identity on $\partial D(R)$ for each $c$; 
\item
$\Theta_c$ is equivariant on the boundary, i.e., 
$\Theta_c(z^2)=f_c(\Theta_c(z))$ on $\partial D(R^{1/2})$; and 
\item
for each $z \in \Bar{A(R^{1/2},R)}$
 the map $c \mapsto \Theta_c(z)$ is holomorphic in $c \in D(r)$. 
\end{itemize}
\item
Each $\Theta_c$ induces a straightening map $h_c:U_c \to h(U_c)$
that is uniformly $(1+O(R^{-1}))$-quasiconformal for $c \in D(r)$.
\end{enumerate}
\end{lem}

Thus we obtain an analytic family of quadratic-like maps
 $\bs{f}=\skakko{f_c:U_c' \to U_c}_{c \,\in \,D(r)}$.
(Note that $\bs{f}$ with tubing $\Theta$ 
is not necessarily a Mandelbrot-like family.) 

\paragraph{\bf Proof.}
One can check that $f_c:U_c' \to U_c$ is a quadratic-like map
as in Example 1 of \cite[p.329]{Douady-Hubbard 1985}: 
Indeed, if $w \in \Bar{U_c}=\Bar{D(R)}$ and 
$R > \max\{2(r+\delta), 6\}$, 
then $R^2/4 > 3R/2 >r+\delta +R$ and 
thus the equation $f_c(z)=w$ 
has two solutions in $D(R/2)$ by Rouch\'e's theorem. 
This implies that $\overline{U_c'} \subset D(R/2)$.
By the maximum principle, 
$f_c:U_c' \to U_c$ is a proper branched covering of degree two.
Since $|f_c(0)| \le r+\delta <R/2$,
$f_c(0) \subset U_c$ and hence 
the critical point $0$ is contained in $U_c'$.
Thus the Riemann-Hurwitz formula
implies that $U_c'$ is a topological disk contained in $D(R/2) \Subset U_c$.

Next we construct $\Theta$:
Let $z=g_c(w)$ be the univalent branch of $f_c^{-1}(w)=\sqrt{w-(c+u)}$ 
defined on the disk centered at $w(0):=R$ with radius $R/2$
such that $g_c(R)$ is close to $R^{1/2}$.
(Note that if $|z|=R>6$
we have $|f_c(z)| \ge R^2-(r+\delta) >R^2-R/2>3R/2$.
Thus $D(w(0),R/2) \subset f_c(D(R))$,
and by $f_c(0) \notin D(w(0),R/2)$ 
we obtain such a univalent branch.)
We take the analytic continuation of the branch $g_c$ 
along the curve $w(t):=Re^{it}~(0 \le t < 4\pi)$
that is univalent on each $D(w(t), R/2)$.
Then 
$$
\log z = \log g_c(w)= 
\frac{1}{2}\log w + \frac{1}{2} \log \paren{1-\frac{c+u}{w}},
$$
where $u=u(g_c(w),c)$ on $\Bar{D(w(t), R/2)}$.
Let
$$
\Upsilon_c(w):=\frac{1}{2} \log \paren{1-\frac{c+u}{w}}.
$$
Then we have
$|\Upsilon_c(w)| =O(R^{-1})$ (by taking an appropriate branch)
and hence
$\dfrac{d\Upsilon_c}{dw}(w(t))$ $= O(R^{-2})$
by the Schwarz lemma.
For $t \in [0,2\pi)$,
set 
$$
v_c(t):=\Upsilon_c(w(2t))
$$
such that 
$t \mapsto w(t)=\exp(\log R+ it)$ 
and 
$t \mapsto \exp((\log R)/2+ it+v_c(t))$
parametrize the boundaries of $U_c$ and $U_c'$.
To give a homeomorphism between the closed annuli
$\Bar{A(R^{1/2},R)}$ and
$\Bar{A_c}:=\Bar{U_c} \sminus {U_c'}$, we take their logarithms:
Set $\ell:=(\log R)/2$, and consider 
the rectangle $E:=\{ s+it \st \ell \le s \le 2\ell,  0\le t \le 2\pi\}$. 
Fix a smooth decreasing function $\eta_0:[0,1] \to [0,1]$
such that: $\eta_0(0)=1$; $\eta_0(1)=0$; 
and the $j$-th derivative 
of $\eta_0$ tends to $0$ as $x \to +0$ and as $x \to 1-0$ for any $j$.
Set 
$$
\eta(s):=\eta_0(s/\ell -1),\quad s \in [\ell, 2\ell]. 
$$ 
Then we have $\dfrac{d\eta}{ds}(s) = O(\ell^{-1})$, 
$v_c(t) =O(R^{-1})$, and 
$\dfrac{dv_c}{dt}(t)=O(R^{-1})$.

%Figure 13
\begin{figure}[htbp]
\begin{center}
%\includegraphics[width=.65\textwidth]{fig_log_tubing.eps}
\includegraphics[width=.65\textwidth]{fig_log_tubing.png}
\end{center}
\caption{\small Construction of the tubing $\Theta$ for $\bs{f}$.}
\label{fig_tubing}
\end{figure}
Now consider the smooth map $\theta_c: E \to \C$ defined by
$$
\theta_c(s+it):= s+it + \eta(s)v_c(t).
$$
The map $\theta_c$ is injective for sufficiently large $R$ since 
\begin{align}
&|\theta_c(s+it)-\theta_c(s'+it')| \notag \\ 
\ge& |(s-s') + i(t-t')| 
- |\eta(s)(v_c(t)-v_c(t'))| 
- |(\eta(s)-\eta(s'))v_c(t')|  \notag \\
\ge &
|(s-s') + i(t-t')| 
- O(R^{-1})|t-t'|
- O(\ell^{-1}R^{-1})|s-s'|. \label{eq_injectivity}
\end{align}
The Beltrami coefficient of $\theta_c$ is given by 
$$
\mu_{\theta_c} = 
\frac{(d\eta/ds)\,v_c+\eta \,(dv_c/dt) i}
{2+(d\eta/ds)\, v_c-\eta\, (dv_c/dt) i}
=O(R^{-1}).
$$
Hence $\theta_c$ is an orientation preserving diffeomorphism onto its image
for sufficiently large $R$,
and its maximal dilatation is bounded by $1+O(R^{-1})$.
By observing $\theta_c$
through the exponential function,
we obtain a smooth $(1+O(R^{-1}))$-quasiconformal homeomorphism 
$\Theta_c: \Bar{A(R^{1/2},R)} \to \Bar{A_c}$ 
that fixes the outer boundary
and satisfies 
$\Theta_c(z^2)=f_c(\Theta_c(z))$ on the inner boundary.
Holomorphic dependence of $c \mapsto \Theta_c(z)$
for each fixed $z \in \Bar{A(R^{1/2},R)}$
is obvious by the construction of $\theta_c$.


Finally we construct the straightening map $h_c:U_c \to \C$ 
of $f_c:U_c' \to U_c=D(R)$.
Let us extend $f_c$ to a smooth quasiregular map $F_c:\C \to \C$ by setting
$$
F_c(z):=
\left\{\begin{array}{ll}
f_c(z) 
& \text{if}~z \in U_c',\\[.5em] 
\skakko{\Theta_c^{-1}(z)}^2 
& \text{if}~z \in U_c  \sminus  U_c', ~\text{and}\\[.5em]
z^2  
& \text{if}~z \in \C \sminus U_c.
\end{array}\right.
$$
We define an $F_c$-invariant 
 Beltrami coefficient $\mu_c$ (i.e., $F_c^\ast\mu_c=\mu_c$) by
$$
\mu_c(z):=
\left\{\begin{array}{ll}
0 & 
\text{if}~z \in K(f_c)~\text{or}~ z \in \C \sminus U_c,\\[.5em] 
\dfrac{(F_c)_{\Bar{z}}(z)}{(F_c)_z(z)} 
& \text{if}~z \in U_c  \sminus  U_c',~\text{and}\\[1.2em]
\disp (f_c^n)^\ast \mu_c(z)
& \text{if}~f_c^n(z) \in U_c  \sminus  U_c'~\text{for some}~ n >0,
\end{array}\right.
$$
where 
$$
\disp  (f_c^n)^\ast \mu_c(z)
=\mu_c(f_c^n(z))\frac{\Bar{(f_c^n)'(z)}}{(f_c^n)'(z)}.
$$
Then $\mu_c$ is supported on $\Bar{D(R)}$ 
and it satisfies $\norm{\mu_c}_\infty=O(R^{-1})$.
By existence of the normal solutions of the Beltrami equations \cite[Theorem 4.24]{IT Book}, 
we have a unique $(1+ O(R^{-1}))$-quasiconformal map
$h_c:\C \to \C$ that satisfies the Beltrami equation 
$(h_c)_{\Bar{z}}=\mu_c \cdot (h_c)_z$ a.e.,
$h_c(0)=0$, and  
$(h_c)_z-1 \in L^p(\C)$ for some $p >2$.
(The relation between $R$ and $p$ will be more specified in the next lemma.)
The condition $(h_c)_z-1 \in L^p(\C)$ implies $w=h_c(z)=z+b_c+O(1/z)$ 
as $z \to \infty$ for some constant $b_c \in \C$.

Since $\mu_c$ is $F_c$-invariant, 
the map $P(w):=h_c \cc F_c \cc h_c^{-1}(w)$ is a 
holomorphic map of degree 2 with 
a critical point at $h_c(0)=0$ 
and a superattracting fixed point at $h_c(\infty)=\infty.$
Hence $P(w)$ is a quadratic polynomial.
The expansion of the form $w=h_c(z)=z+b_c+O(1/z)$
implies that we actually have $b_c=0$
and $P(w)$ is of the form 
$P(w)=w^2+\chi(c)=P_{\chi(c)}(w)$. Hence the restriction $h_c|_{U_c}$ 
is our desired straightening map.
\QED

The next lemma shows that the quasiconformal 
map $h_c:\C \to \C$ constructed above is uniformly close 
to the identity on compact sets for sufficiently large $R$:

\begin{lem}\label{lem_estimate_of_h}
Fix any $p>2$ and any compact set $E \subset \C$.
If $R$ is sufficiently large, 
then 
the quasiconformal map
$h_c$ in Lemma \ref{lem_almost_conformal} 
satisfies 
$$
|h_c(z)-z|=O(R^{-1+2/p})
$$
uniformly for each $c \in D(r)$ and $z \in E$.
\end{lem}
\noindent
Indeed, the estimate is valid for any $R \ge C_0p^2$, 
where $C_0$ is a constant independent of $p$.


\paragraph{\bf Proof.}
We have 
$\norm{\mu_c}_\infty \le C/R=:k$ for 
some constant $C$ independent of $c \in D(r)$
by the construction of $h_c$.
By \cite[Theorem 4.24]{IT Book}, 
we have $(h_c)_z-1 \in L^p(\C)$ for any $p>2$ satisfying $k C_p<1$, 
where $C_p$ is the constant that appears in the Calderon-Zygmund inequality
\cite[Proposition 4.22]{IT Book}.
Gaidashev showed in \cite[Lemma 6]{Gaidashev 2007} that $C_p \le \cot^2(\pi/2p)$.
Since $\cot^2(\pi/2p) = (2p/\pi)^2(1+o(1))$ as $p \to \infty$,
the inequality $k C_p \le (C/R) \cot^2(\pi/2p) \le 1/2$ 
is established if we take $R \ge C_0 p^2$ 
for some constant $C_0$ independent of $p>2$. 
By following the proof of \cite[\S 4, Corollary 2]{IT Book}, 
we have 
$$
|h_c(z)-z|
\le 
K_p \cdot \frac{1}{1-kC_p}\norm{\mu_c}_p|z|^{1-2/p}  
$$
for any $z \in \C$, 
where $K_p>0$ is a constant depending only on $p$
and $\norm{\mu_c}_p$ is the $L^p$-norm of $\mu_c$.
Since $|\mu_c|\le C/R$ and $\mu_c$ is supported on $\overline{D(R)}$,
we have $\norm{\mu_c}_p \le (C/R)(\pi R^2)^{1/p} =C \pi^{1/p} R^{-1+2/p}$.
Hence if we take $R \ge C_0 p^2$ such that $kC_p \le 1/2$,
we have 
$$
|h_c(z)-z|= 2 K_p C \pi^{1/p} R^{-1+2/p} |z|^{1-2/p}.
$$
This implies that $|h_c(z)-z| =O( R^{-1+2/p})$
on each compact subset $E$ of $\C$. 
\QED 

\begin{cor}\label{cor_estimate_of_chi}
Fix any $p >2$. 
If $R$ is sufficiently large,
then for each $c \in D(r)$, 
$f_c$ is hybrid equivalent to 
a quadratic polynomial $P_{\chi(c)}(w)=w^2+\chi(c)$ 
with 
$$
|\chi(c)-c| \le \delta+O(R^{-1+2/p}).
$$
\end{cor}

\paragraph{\bf Proof.}
We have $\chi(c)=h_c(f_c(0))$ 
since $h_c$ maps the critical value of $f_c$ 
to that of $P_{\chi(c)}$, 
For each $c \in D(r)$,
$f_c(0) = c+u_c(0)$ is contained in a compact set $\overline{D(r+\delta)}$.
By Lemma \ref{lem_estimate_of_h}, we obtain
$$
\chi(c)=h_c(f_c(0))=c+u_c(0) + O(R^{-1+2/p})
$$
for sufficiently large $R$ and this implies the desired estimate.
\QED 

\medskip

\paragraph{\bf Coordinate changes}
Under the same assumption as in Lemma \ref{lem_almost_conformal},
we assume in addition that  
\begin{enumerate}[(i)]
\item
$\delta<1$ and  $r >4$; and 
\item
$R$ is large enough such that 
${\bs{f}}=\skakko{f_c:U_c' \to U_c}_{c\,\in \,D(r)}$
is an analytic family of quadratic-like maps 
(that may not necessarily be Mandelbrot-like),
 and that (1) and (2) of Lemma \ref{lem_almost_conformal} hold.
\end{enumerate}
By (2) of Lemma \ref{lem_almost_conformal}, 
each $f_c \in \bs{f}$ is hybrid equivalent to 
some quadratic map $P_{\chi(c)}$
by the $(1 + O(R^{-1}))$-quasiconformal straightening $h_c:U_c \to h_c(U_c)$.
We say the map
$$
\chi = \chi_{\bs f}:D(r) \to \C,\quad \chi(c)=h_c(f_c(0))
$$
is the {\it straightening map} of the family ${\bs f}$
associated with the tubing $\Theta=\{\Theta_c\}_{c \,\in\, D(r)}$.
We say the map $(z,c) \mapsto (h_c(z), \chi(c))$
defined on $D(R)\times D(r)$ 
is a {\it (straightening) coordinate change}.

Now we show that the straightening map is quasiconformal with
dilatation arbitrarily close to $1$ if we take sufficiently large $R$
and small $\delta$:

\begin{lem}[\bf Almost conformal straightening of ${\bs f}$]
\label{lem_almost_conformal_M}
If $r$ and $R$ are sufficiently large
and $\delta>0$ is sufficiently small,
then the family ${\bs f}$ is associated with 
a $(1+O(\delta)+O(R^{-1}))$-quasiconformal straightening map
$$
\chi=\chi_{\bs f}:D(r) \to \chi(D(r)) \subset \C
$$
such that
\begin{enumerate}[\rm (1)]
\item $\chi(M_{\bs f})=M$, where $M_{\bs f}$ 
is the connectedness locus of ${\bs f}$;
\item $\chi|_{D(r) \sminus M_{\bs f}}$ is $(1+O(R^{-1}))$-quasiconformal; and
\item $\chi|_{M_{\bs f}}$ extends to a $(1+O(\delta))$-quasiconformal map on the plane.
\end{enumerate}
\end{lem}

\paragraph{\bf Proof.}
By slightly shrinking $r>4$ if necessary, we may assume that 
$f_c$ is defined for  $c \in \partial D(r)$. 
When $c=re^{it}~(0 \le t \le 2 \pi)$, 
we have $|f_c(0)-0|=|c+u(0,c)| \ge r-\delta > 3$ (since $\delta<1$).
Hence as $c$ makes one turn around the origin 
so does $f_c(0)$.
By \cite[p.328]{Douady-Hubbard 1985}, 
$\chi$ gives a homeomorphism between $M_{\bs f}$ and $M$.
Moreover,  
$\chi|_{D(r) \sminus M_{\bs f}}$ is $(1+O(R^{-1}))$-quasiconformal
by \cite[Proposition 20, Lemma in p.327]{Douady-Hubbard 1985},
since each $\Theta_c$ is $(1+O(R^{-1}))$-quasiconformal.

For the dilatation of $\chi|_{M_{\bs f}}$,
we follow the argument of \cite[Lemma 4.2]{McMullen 2000}:
Consider the families 
${\bs f}_t:=\braces{f_{c,t}}_{c\, \in \,D(r)}$
defined for each $t \in \D$,
where
$$
f_{c,t}(z):=z^2+c+\frac{\,t\,}{\delta}\,u(z,c).
$$
By the same argument as above, 
the connectedness locus  $M_{{\bs f}_t}$ of ${\bs f}_t$
is homeomorphic to $M$ by the straightening map 
$$
\chi_t=\chi_{{\bs f}_t}:M_{{\bs f}_t} \to M.
$$
Then the inverse $\phi_t:=\chi_t^{-1}:M \to M_{{\bs f}_t}$
gives a holomorphic family of injections over $\D$.
By Bers and Royden's theorem \cite[Theorem 1]{Bers-Royden 1986},
each of them extends to a $(1+|t|)/(1-|t|)$-quasiconformal map 
$\widetilde{\chi}_t$ on $\C$.
In particular, $\chi|_{M_{\bs f}}=\chi_\delta$ extends 
to a $(1+O(\delta))$-quasiconformal map on $\C$.
Now we apply Lemma \ref{lem_Bers} (Bers' Gluing Lemma)
to $H_1:=\chi|_{D(r) \smallsetminus M_{\bs f}}$ and $H_2=\widetilde{\chi}_\delta$.
Then $H_1$ and $H_2$ are glued along $\partial M$
and the glued map $H:D(r)\to \C$, 
which coincides with $\chi$,
is $(1+O(\delta)+O(R^{-1}))$-quasiconformal.
\QED




\section{Proof of Theorem C}

\paragraph{\bf Idea of the proof.}
The proof follows the argument of Theorem A
and uses the results in the previous section.
Recall that in the proof of Theorem A,
 we construct two families of quadratic-like maps
$\{f_c: U_c' \to U_c\}_{c \, \in \, S \, \cap \,\Lambda}$ (``the first renormalization") 
and 
${\bs G}=\{G_c: V_c' \to U_c\}_{c \, \in \, W}$ (``the second renormalization"),
and we conclude that the small Mandelbrot set
corresponding to the family ${\bs G}$
has a desired decoration.
 
In the following proof of Theorem C, 
we first take a ``thickened" family 
$\widehat{\bs f}
=\{f_c: \widehat{U}_c' \to \widehat{U}_c\}_{c \, \in \, \widehat{\Lambda}}$ 
that contains 
${\bs f}=\{f_c: U_c' \to U_c\}_{c \, \in \, \Lambda}$
as a restriction (in both dynamical and parameter planes), 
such that $U_c \Subset \widehat{U}_c$ and 
the modulus of $\widehat{U}_c\sminus\overline{U_c}$ 
is sufficiently large.
Next we construct another ``thickened" family 
$\widehat{\bs G}=\{G_c: \widehat{V}_c' \to \widehat{U}_c\}_{c \, \in \, \widehat{W}}$
that contains $\bs{G}=\{G_c: V_c' \to U_c\}_{c \, \in \, W}$
with $V_c' \Subset \widehat{V}_c'$.
Then we can apply a slightly modified versions of the lemmas 
in the previous section to the family $\bs{G}$.
Finally we conclude that the small Mandelbrot set
corresponding to the family ${\bs G}$
has a very fine decoration.

\paragraph{\bf Notation. }
We will use a conventional notation: For complex variables $\alpha$ and $\beta$,
by $\alpha \asymp \beta$ we mean $C^{-1}|\al| \le |\beta| \le C|\al|$ for an implicit constant $C>1$. 

\paragraph{\bf First renormalization.}
We start with a result by McMullen \cite[Theorem 3.1]{McMullen 2000}
(see also \cite[Chapter V]{Douady-Hubbard 1985})
applied to (and modified for) the quadratic family:

\begin{lem}[\bf Misiurewicz cascades]
\label{lem_cascade}
For any Misiurewicz parameter $m_0$ 
and any arbitrarily large $r$ and $R$, 
there exist sequences 
$\braces{s_n}_{n \ge 1}$, 
$\braces{p_n}_{n \ge 1}$,
$\braces{t_n}_{n \ge 1}$, 
and $\braces{\delta_n}_{n \ge 1}$
that satisfy the following conditions for each sufficiently large $n$: 
\begin{enumerate}[\rm (a)]
\item
$s_n$ is a superattracting parameter of period $p_n$ 
with $|s_n -m_0| \asymp \mu_0^{-n}$, 
where $\mu_0$ is the multiplier
of the repelling cycle of $P_{m_0}$ 
on which the critical orbit lands.
\item 
$t_n \in \C^\ast$ and $t_n \asymp \mu_0^{-2n}$.
\item 
$\delta_n>0$ and $\delta_n \asymp n \mu_0^{-n}$.
\item
Let $X_n:D_n:=D(s_n,\, r |t_n|) \to D(r)$ be the affine map 
defined by 
$$
C=X_n(c):= \frac{c-s_n}{t_n}.
$$
Then there exists a non-zero holomorphic function 
$c \mapsto \al(c)=\al_c$
defined for $c \in D_n$ such that $\alpha_c \asymp \mu_0^{-n}$ 
and the map
$$
Z = A_c(z) :=\frac{z}{\alpha_c}
$$
conjugates $P_c^{p_n}$ on $D(R \,|\alpha_c|)$ to 
the map $F_{C}:=A_c \cc P_c^{p_n}\cc A_c^{-1}$
of the form
\begin{equation}
F_{C}(Z) 
 = A_c \cc P_c^{p_n} \cc A_c^{-1}(Z)
=Z^2 + C + u(Z,C),
\label{eq_F_C}
\end{equation}
where $u(Z,C) = u_C(Z)$ 
is holomorphic in both $Z \in  D(R)$ and $C \in D(r)$,
and satisfies $u'_C(0) = 0$ and $|u(Z,C)| \le \delta_n$.
\end{enumerate} 
\end{lem}


%Figure 14
\begin{figure}[htbp]
\begin{center}
%\includegraphics[width=.72\textwidth]{fig_coordinate_change.eps}
\includegraphics[width=.65\textwidth]{fig_coordinate_change.png}
\end{center}
\caption{\small 
An affine coordinate change.}
\label{fig_affine_coordinate_change}
\end{figure}

\paragraph{\bf Construction of the family $\widehat{\bs f}$.}
Let us fix arbitrarily small $\varepsilon>0$ 
and $\kappa>0$ as in the statement of Theorem C.
We choose any Misiurewicz parameter 
$m_0$ in $\mathrm{int}(B) \cap \partial M$,
where $B$ is the closed disk given in the statement.
(The Misiurewicz parameters are dense in $\partial M$.)



For any $r$ and $R$ bigger than $4$
(we will replace them with larger ones if necessary,
but it will happen finitely many times in what follows),
by taking a sufficiently large $n$ in Lemma \ref{lem_cascade}
such that $D_n \subset B$
\footnote{More precisely, we fix $r$ first, 
and then take a larger $R$ (and an $n$) if necessary to
apply those lemmas.},
we have an analytic family 
$\braces{F_C: D(R) \to \C}_{C \,\in \,D(r)}$
that satisfies the conditions for 
Lemma \ref{lem_almost_conformal}.
Moreover, its restriction
$$
{\bs F}_n:=
\braces{F_C: F_C^{-1}(D(R)) \to D(R)}_{C \,\in \,D(r)}
$$
is an analytic family of quadratic-like maps
that satisfies the conditions for 
Lemma \ref{lem_almost_conformal_M}.
Hence we have an associated straightening coordinate change
of the form $(Z,C) \mapsto (H_C(Z), \chi_{{\bs F}_n}(C))$.
More precisely, for each $C \in D(r)$, 
$F_C$ is hybrid equivalent to 
$Z \mapsto Z^2+\chi_{{\bs F}_n}(C)$
by a $(1+O(R^{-1}))$-quasiconformal straightening $H_C$
by Lemma \ref{lem_almost_conformal}, 
and $H_C$ satisfies the estimate of Lemma \ref{lem_estimate_of_h}.
By Lemma \ref{lem_almost_conformal_M}, 
the straightening $\chi_{{\bs F}_n}: D(r) \to \C$
is $(1+O(\delta_n)+O(R^{-1}))$-quasiconformal
and satisfies the estimate of Corollary \ref{cor_estimate_of_chi}.
Hence we may assume that $R$ and $n$ are large enough such that
both $Z \mapsto H_C(Z)$ and $C \mapsto \chi_{{\bs F}_n}(C)$
are $(1+\kappa)^{1/2}$-quasiconformal for 
$\kappa >0$ given in the statement.

Let $f_c:=P_c^{p_n}:\widehat{U}_c' \to \widehat{U}_c$
be the pull-back of $F_C:F_C^{-1}(D(R)) \to D(R)$
by the map $(z,c) \mapsto (Z,C)=(A_c(z),X_n(c))$,
which we call the {\it affine coordinate change}.
(See Figure \ref{fig_affine_coordinate_change}.
Note that $\widehat{U}_c=D(R \, |\alpha_c|)$ is a round disk.)
Set 
$$
p:=p_n,\quad s_0:=s_n,\quad \text{and} \quad 
\widehat{\Lambda}:=D_n=D(s_n,\, r |t_n| ).
$$
The quadratic-like family
$$
\widehat{{\bs f}}:=
\big\{f_c:\widehat{U}_c'\to \widehat{U}_c \big\}
_{c \,\in \,\widehat{\Lambda}}
$$
is our first family of renormalizations
whose straightening coordinate change $(z,c) \mapsto (h_c(z),\chi(c))$
is given by
$$
(h_c(z),\chi(c)):= (H_C \cc A_c(z), \chi_{{\bs F}_n} \cc X_n(c)).
$$
Note that both $h_c:\widehat{U}_c \to \C$ 
and $\chi:\widehat{\Lambda} \to \C$
are $(1+\kappa)^{1/2}$-quasiconformal.
By Lemma \ref{lem_estimate_of_h} and Corollary \ref{cor_estimate_of_chi},
if we fix any $p'>2$
and any compact subset $E$ of $U_c$,
then for sufficiently large $R$ we have 
$$
h_c(z) =A_c(z) +O(R^{-1+2/p'})
$$
on $E$ 
and 
$$
\chi(c)= X_n(c)+ O(\delta_n)+O(R^{-1+2/p'})
$$
on $\widehat{\Lambda}$. 
Hence the straightening coordinate change 
$(z,c) \mapsto (h_c(z),\chi(c))$
is very close to the affine coordinate change 
$(z,c) \mapsto (A_c(z),X_n(c))=(z/\al_c,(c-s_n)/t_n)$
if we take sufficiently large $R$ and $n$.


\paragraph{\bf Construction of the family ${\bs f}$.}
Let $\rho>4$ be an arbitrarily large number.
By taking sufficiently large $r$, $R$ and $n$
such that $\rho/R$ is sufficiently small,
we may assume the following:
\begin{itemize}
\item
The set $\Omega(\rho):=\skakko{C \in D(r)\st F_C(0) \in D(\rho)}$
gives a Mandelbrot-like family
$$
\bs{F}_n(\rho):=\{F_C:F_C^{-1}(D(\rho)) \to D(\rho)\}_{C \,\in \,\Omega(\rho)}.
$$
\item 
$D(\rho) \Subset F_C^{-1}(D(R))$ for any $C \in \Omega(\rho)$.
\end{itemize}
Now we define the Mandelbrot-like family
$$
\bs{f}=\{f_c:U_c' \to U_c\}_{c \,\in \,\Lambda}
$$ 
as the pull-back of $\bs{F}_n(\rho)$ by the affine coordinate change 
$(z,c) \mapsto (A_c(z),X_n(c))$ above.
More precisely,
we let 
$U_c:=A_c^{-1}(D(\rho))=D(\rho|\al_c|)$
for each $c \in \widehat{\Lambda}$,
and consider the restriction $f_c:U_c' \to U_c$ of
$f_c:\widehat{U}_c' \to \widehat{U}_c$.
Then we define the subset $\Lambda$ of $\widehat{\Lambda}$ by 
$\Lambda:=X_n^{-1}(\Omega(\rho))$ such that the family $\bs{f}$ above
becomes a Mandelbrot-like family.
Note that we have 
$U_c' \Subset U_c \Subset \widehat{U}_c' \Subset \widehat{U}_c$
for any $c \in \Lambda$.

Let $M_{\bs f} =M_{\bs s_0} = s_0 \perp M$ be the connectedness locus of 
the family ${\bs f}$, which coincides with that of $\widehat{\bs f}$. 
Note that $\bs f$ has the same straightening coordinate change 
$(z,c) \mapsto (h_c(z),\chi(c))$ as $\widehat{\bs f}$
such that $\chi(M_{\bs f})=M$.






\paragraph{\bf Second renormalization.}
For a given Misiurewicz or 
parabolic parameter $c_0$ in the statement of Theorem C, 
we define a Misiurewicz or parabolic parameter $c_1 \in M_{\bs f}$ by
$$
c_1: = \chi^{-1}(c_0) =s_0 \perp c_0.
$$ 
Let $q_{c_1}$ be a repelling or parabolic periodic point of $P_{c_1}$
that belongs to the postcritical set. 
More precisely, when $c_1$ is Misiurewicz, 
the orbit of $0$ 
eventually lands on $q_{c_1}$ that is a repelling periodic point. 
When $c_1$ is parabolic, 
the orbit of $0$ accumulates on $q_{c_1}$ that is a parabolic periodic point.
In both cases, there exists the smallest $m \in \N$ such that 
$f_{c_1}^m(0)$ is contained in the domain $\Omega_{c_1}$ 
of the linearizing coordinate or the attracting Fatou coordinate of $q_{c_1}$. 

The rest of the proof of Theorem C
is divided into Claims 1 to 7 below and their proofs.
For the first two claims,
we may simply apply the argument of
Steps (M1)--(M2) or Steps (P1)--(P2):

\paragraph{\bf Claim 1.}
{\it  
There exists a Jordan domain $\widehat{V}_{c_1}$ 
with $C^1$ boundary and integers $N,\,j \in \N$ 
which satisfy the following:
\begin{enumerate}[\rm (1)]
\item 
$\widehat{V}_{c_1}$ is a connected component of 
$P_{c_1}^{-N}(\widehat{U}_{c_1})$.
\item 
$g_{c_1}:=P_{c_1}^N|_{\widehat{V}_{c_1}}:\widehat{V}_{c_1} \to \widehat{U}_{c_1}$ 
is an isomorphism and 
${f_{c_1}^{j}(\widehat{V}_{c_1})} 
%=\overline{P_{c_1}^{pj}(\widehat{V}_{c_1})} 
\Subset \widehat{U}_{c_1}  \sminus  \Bar{\widehat{U}_{c_1}'}$. 
\item
$\widehat{V}_{c_1} \Subset \Omega_{c_1}$.
Also we can take $\widehat{V}_{c_1}$ 
arbitrarily close to $q_{c_1}$.
\end{enumerate}
}

\medskip

\medskip

\paragraph{\bf Claim 2.}
{\it
There exists 
a Jordan domain $\widehat{W} \Subset \Lambda \sminus M_{\bs f}
~(\subset \widehat{\Lambda} \sminus M_{\bs f})$ 
arbitrarily close to $c_1$ that satisfies the following:
\begin{enumerate}[\rm (1)]
\item
There is a holomorphic motion of $\widehat{V}_{c_1}$
over $\widehat{W}$ that generates 
a family of Jordan domains $\{\widehat{V}_{c}\}_{c \,\in \, \widehat{W}}$
with $C^1$ boundaries
such that for each $c \in \widehat{W}$, 
${f_c^j(\widehat{V}_{c} )} \Subset \widehat{U}_{c} \sminus \overline{\widehat{U}_{c}'}$
and 
$g_c:=P_{c}^{N}|_{\widehat{V}_{c}}:\widehat{V}_{c} \to \widehat{U}_{c}$
is an isomorphism.
\item
There exists an $L$ such that 
$f_{c}^{L}(0)=P_{c}^{pL}(0) \in \widehat{V}_{c}$ for any $c \in \widehat{W}$.
\item 
For $c \in \partial \widehat{W}$, 
we heve $P_{c}^{pL+N}(0) \in \partial  \widehat{U}_c$.
Moreover, when $c$ makes one turn along $\partial \widehat{W}$,
then $P_{c}^{pL+N}(0)$ makes one turn around the origin.
\end{enumerate}
}
Indeed, such a domain $\widehat{W}$ is given by
$$
\widehat{W}
=
\braces{c \in \Lambda \sminus M_{\bs f} 
\st 
f_c^{i}(0) \in \Omega_{c_1}~\text{for $i=m,\,  \,\ldots, L-1$ and}~ 
f_c^L(0) \in \widehat{V}_{c}}.
$$
See Figure \ref{fig_W}. Note that if we take sufficiently large 
$N$ and $j$, we may always assume that 
$\widehat{V}_c \Subset U_c'$ as depicted in Figure \ref{fig_W}.
Moreover, the proof of Lemma 4.1 indicates that
we can choose $\widehat{V}_c$ with arbitrarily small diameter.

%Figure 15
\begin{figure}[htbp]
\begin{center}
%\includegraphics[width=.65\textwidth]{fig_W.eps}
\includegraphics[width=.65\textwidth]{fig_W.png}
\caption{\small
For any $c \in \widehat{W}$,
the Julia set $J(f_c)$ moves only a little from 
$J(f_s)$ of the center $s \in W \Subset \widehat{W}$.
}\label{fig_W}
\end{center}
\end{figure}


\medskip

\paragraph{\bf Definition of the center of $\widehat{W}$ and $W$.}
As in Step (M2) and Step (P2), 
there exists a unique 
superattracting parameter $s \in \widehat{W}$ 
such that $P_{s}^{pL+N}(0)=0$.
We call $s$ the {\it center} of $\widehat{W}$.

By Claim 2, we can find 
a family of Jordan domains $\{V_c\}_{c \, \in \, \widehat{W}}$
with $C^1$ boundaries 
such that $V_c \Subset \widehat{V}_c$
and that $P_c^N:V_c \to U_c$ is an isomorphism for each $c \in \widehat{W}$.
Let $W \Subset \widehat{W}$ be the set of 
$c$ such that $f_c^L(0) \in V_c$.
By the same argument as in Step (M3) and Step (P3),
one can check that both $\widehat{W}$ and $W$ are 
Jordan domains with $C^1$ boundaries.
Note that $s$ is the center of $W$ as well.
(See Figure \ref{fig_W} again.)





Moreover, we have:

\paragraph{\bf Claim 3 (Straightening the center).}
{\it
By choosing $\widehat{V}_c$ in Claim 1 
close enough to $q_{c_1}$,
we can find an $\eta \in D(\vep)$ 
with $c_0 +\eta \notin M$
such that $f_s:\widehat{U}_s' \to \widehat{U}_s$ is 
hybrid equivalent to 
a quadratic-like restriction of $P_{c_0+\eta}$ 
 with $(1+\kappa)^{1/2}$-quasiconformal straightening map.
In particular, $J(f_s)$ is a $(1+\kappa)^{1/2}$-quasiconformal
image of $J_{c_0+\eta}$.
}

\paragraph{\bf Proof.}
By the construction of $\widehat{W}$ in Claim 2
(following Step (M2) or Step (P2)),
we can take $\widehat{W}$ arbitrarily close to $c_1$.
Since the straightening map
$\chi:\widehat{\Lambda} \to \C$
is continuous, 
we have $|\chi(s)-\chi(c_1)| = |\chi(s)-c_0|<\vep$
by taking $s \in \widehat{W}$ close enough to $c_1$.
Set $\eta:=\chi(s)-c_0$.
Then we have
$\chi(s)=c_0+\eta \in \C  \sminus M$ 
since $\widehat{W} \subset \widehat{\Lambda} \sminus M_{\bs f}$.
By the construction of the first renormalization,
$f_s$ is conjugate to $P_{\chi(s)}$ 
by the $(1+\kappa)^{1/2}$-quasiconformal straightening map $h_s$
such that $h_s(J(f_s))=J_{c_0+\eta}$.
\QED

\medskip

\paragraph{\bf Remark.}
Since $h_s(z)= H_{X_n(s)} \cc A_s(z)=z/\al_s + o(1)$,
 $J(f_s)$ is actually an ``almost affine" (even better than ``almost conformal"!)
 copy of $J_{c_0+\eta}$.



\paragraph{\bf Holomorphic motion of the Cantor Julia sets.}
Since 
$\widehat{W} \subset \widehat{\Lambda} \sminus M_{{\bs{f}}}$,
the Julia set $J(f_c)$ for each $c \in \widehat{W}$
is a Cantor set that is a $(1+\kappa)^{1/2}$-quasiconformal image of $J_{\chi(c)}$.
Moreover, the Julia set $J(f_c)$ moves holomorphically for $c \in \widehat{W}$:

\medskip 
\paragraph{\bf Claim 4 (Cantor Julia moves a little).}
{\it
There exists a holomorphic motion $\iota: J(f_s) \times \widehat{W} \to \C$
such that 
$\iota_c(z):=\iota(z,c)$ maps $J(f_s)$ bijectively to 
$J(f_c)$ for each $c \in  \widehat{W}$. 
Moreover, if $R$ is sufficiently large,
then $\iota_c$ extends to a $(1+\kappa)^{1/2}$-quasiconformal homeomorphism on the plane 
for each $c \in W \Subset \widehat{W}$.
}

\medskip 
A direct corollary of Claims 3 and 4 is:

\begin{cor}[\bf Julia appears in Julia]\label{cor_almost_conformal_copy_of_J}
The Julia set $J_c$ of $P_c$ contains a
$(1+\kappa)$-quasiconformal copy of $J_{c_0+\eta}$
for any $c \in W  \Subset \widehat{W}$.
\end{cor}


\paragraph{\bf Proof of Claim 4.}
Since $J(f_c)$ is a hyperbolic set for each $c \in  \widehat{W}$,
 it has a local holomorphic motion near $c$. 
 (See \cite[p.229]{Shishikura 1998}.)
The holomorphic motion extends to that of 
$J(f_s)$ over $\widehat{W}$ 
as in the statement 
since $\widehat{W}$ is simply connected (and isomorphic to $\D$). 

Now we consider the modulus of the annulus $\widehat{W} \sminus \overline{W}$: 
Recall that $A_c(\widehat{U}_c)=D(R)$, where $A_c(z)=z/\al_c$
and $\al_c$ depends holomorphically on $c \in \widehat{W}$ 
(Lemma \ref{lem_cascade}).
By the same argument as in Step (M2) and Step (P2),
for each $\zeta \in D(R)$, the equation
$$
f_{c}^L(0) = (A_c \circ  g_c)^{-1}(\zeta) \quad (\in \widehat{V}_c)
$$ 
with respect to $c$ has a unique solution $c=\check{c}(\zeta) \in \widehat{W}$ and the map $\zeta \mapsto \check{c}(\zeta)$ gives an isomorphism between $D(R)$ to $\widehat{W}$.
In particular, we have $A_c(\widehat{U}_c-\overline{U_c})=A(\rho,R)$
for any $c \in \widehat{W}$ and thus $\widehat{W}-\overline{W}=\check{c}(A(\rho,R))$.
Hence we obtain
$$
\mathrm{mod}(\widehat{W} \sminus \overline{W})
= \mathrm{mod}(A(\rho,R))=\frac{\log (R/\rho)}{2 \pi}.
$$
By taking $R$ relatively larger than $\rho$,
this modulus is arbitrarily large.
Let us choose a uniformization 
$\psi:\widehat{W} \to \D$
such that $\psi(s)=0$.
For an arbitrarily small $\nu>0$,
we may assume that $\psi(W) \subset D(\nu)$
when $\mathrm{mod}(\widehat{W} \sminus \overline{W})$ 
is sufficiently large.
(See \cite[Theorems 2.1 and 2.4]{McMullen 1994}.
Indeed, it is enough to take $R$ such that $\nu \asymp \rho/R$.)
By the Bers-Royden theorem 
(\cite[Theorem 1]{Bers-Royden 1986}), 
each $\iota_c:J(f_s) \to J(f_c)$ extends to a 
$(1 + \nu)/(1 - \nu)$-quasiconformal map on $\C$.
Thus the dilatation is uniformly smaller than 
$(1+\kappa)^{1/2}$ if we choose a sufficiently large $R$.
\QED


\medskip 

\paragraph{\bf Definition of the families $\widehat{\bs G}$ 
and $\bs G$.} 
For each $c \in \widehat{W}$, 
let $\widehat{V}_c'$ be the connected component of 
$f_{c}^{-L}(\widehat{V}_{c})$ 
(or, that of $P_{c}^{-pL-N}(\widehat{U}_{c})$)
containing the critical point $0$. 
We define $G_c:\widehat{V}_c' \to  \widehat{U}_{c}$ 
by the restriction of $P_c^N \cc f_c^L = P_{c}^{pL+N}$ on $\widehat{V}_c'$. 
Then we have a family of quadratic-like maps 
$$
\widehat{\bs G}
:= \{G_c: \widehat{V}'_c \to \widehat{U}_c\}_{c  \,\in \, \widehat{W}}.
$$



Similarly, for each $c \in W$, 
let $V_c'$ be the connected component of 
$f_{c}^{-L}({V}_{c})$ 
(or, that of $P_{c}^{-pL-N}({U}_{c})$)
containing $0$.
Then we have a quadratic-like family 
$$
\bs{G}:= 
\{G_c: {V}_c' \to {U}_c\}_{c  \,\in \, W}.
$$


Note that
both the annuli $\widehat{U}_c \sminus \overline{\widehat{V}_c'}$ 
and $U_c \sminus \overline{V_c'}$ 
contain the Cantor Julia set $J(f_c)$ for each $c \in \widehat{W}$.





\paragraph{\bf Claim 5 (Extending the holomorphic motion).}~
{\it
If $R$ is sufficiently large and relatively larger than $\rho$, 
we have the following extensions of 
the holomorphic motion $\iota$ 
of the Julia set $J(f_s)$ given in Claim 4:
\begin{enumerate}[\rm (1)]
\item
An extension to the holomorphic motion of 
$J(f_s) \cup \partial \widehat{V}_s' \cup \partial U_s
\cup \partial \widehat{U}_s$ over $\widehat{W}$
that is equivariant to the action of 
$G_c:\partial \widehat{V}_c' \to \partial\widehat{U}_c$.
\item
A further extension of {\rm (1)}
to the motion of the closed annulus $\Bar{\widehat{U}_s} \sminus \widehat{V}_s'$ over $\widehat{W}$.  
\item
An extension to the holomorphic motion of 
$J(f_s) \cup \partial {V}_s' \cup \partial U_s$ 
over ${W}$
that is equivariant to the action of 
$G_c:\partial {V}_c' \to \partial{U}_c$.
\item
A further extension of {\rm (3)}
to the motion of
the closed annulus $\Bar{{U}_s} \sminus {V}_s'$ over ${W}$. 
\end{enumerate}
In particular, the quasiconformal map
$\iota_c: \Bar{{U}_s} \sminus {V}_s' \to \Bar{{U}_c} \sminus {V}_c'$
induced by {\rm (4)}
extends to a $(1+\kappa)^{1/2}$-quasiconformal map on the plane 
for each $c \in W$.
}

See Figure \ref{fig_extending_holo_motion}.


%Figure 16
\begin{figure}[htbp]
\begin{center}
%\includegraphics[width=.38\textwidth]{fig_extending_holo_motion.eps}
\includegraphics[width=.38\textwidth]{fig_extending_holo_motion.png}
\end{center}
\caption{\small
Extending the holomorphic motion to the closed annuli.
The motion of $J(f_s)$ is contained in the motion of shadowed annuli.
(Note that the nested annuli are extremely thick in our setting.)
}
\label{fig_extending_holo_motion}
\end{figure}
\medskip

\paragraph{\bf Proof.} 
(1)~
The sets $\partial U_c$, 
$\partial \widehat{V}_c'$, and 
$\partial \widehat{U}_c$ are all images of round circles
by equivariant analytic families of locally conformal injections 
over $\widehat{W}$.
In particular, they never intersect with the Julia set $J(f_c)$
for each $c \in \widehat{W}$.
Hence the extension of $\iota$ to 
$J(f_s) \cup \partial \widehat{V}_s' \cup \partial U_s
\cup \partial \widehat{U}_s$
 over $\widehat{W}$ is straightforward.\\
(2)~
By S\l odkowski's theorem (\cite{Slodkowski 1991}), 
(1) extends to the motion of $\C$,
and its restriction to 
the closed annulus $\Bar{\widehat{U}_s} \sminus \widehat{V}_s'$
is our desired motion.
Note that 
$\iota_c:\Bar{\widehat{U}_s} \sminus \widehat{V}_s'
\to \Bar{\widehat{U}_c} \sminus \widehat{V}_c'$ 
is uniformly $(1+\kappa)^{1/2}$-quasiconformal 
for $c \in W \Subset \widehat{W}$ by taking a sufficiently large $R$
that is relatively larger than $\rho$. (See the proof of Claim 4.) \\
(3)~
Similarly, $\partial V_c'$ is an image of a round circle 
$\partial U_c$ by an analytic family of injections
(where each injection is locally a univalent branch of $G_c^{-1}$) 
for $c \in W$.
Since $V_c' \Subset \widehat{V}_c'$, 
$V_c'$ never intersects with $J(f_c)$
for $c \in W$ and 
we obtain an extension of the motion of $J(f_s)$
to that of $J(f_s) \cup \partial V_s' \cup \partial U_s$ over $W$ 
which satisfies $G_c \cc \iota_c=\iota_c \cc G_s$
on $\partial V_s'$. \\
(4)~
To extend (3) to the closed annulus 
$\Bar{{U}_s} \sminus {V}_s'$, 
we divide the annulus into two annuli 
$\Bar{{U}_s} \sminus \widehat{V}_s'$ and 
$\Bar{\widehat{V}_s'} \sminus {V}_s'$.
The desired motion of $\Bar{{U}_s} \sminus \widehat{V}_s'$ over $W$
is contained in the motion given in (2).
For the annulus $\Bar{\widehat{V}_s'} \sminus {V}_s'$,
we note that the map 
$G_c: \Bar{\widehat{V}_c'} \sminus {V}_c' \to \Bar{\widehat{U}_c} \sminus {U}_c$
is a holomorphic covering of degree two.
Hence we can pull-back the motion of $\Bar{\widehat{U}_s} \sminus {U}_s$ 
over $W$ that is contained in the motion given in (2)
by these covering maps.
More precisely, we can construct  
an analytic family 
$\iota_c:\Bar{\widehat{V}_s'} \sminus \widehat{V}_s' \to 
\Bar{\widehat{V}_c'} \sminus \widehat{V}_c'$ of 
$(1+\kappa)^{1/2}$-quasiconformal maps 
that agrees with the motion of $J(f_s) \cup \partial V_s' \cup \partial U_s$,
by taking a branch of 
$G_c^{-1} \cc \iota_c\cc G_s$
with $\iota_c$ given in (2).
\QED


\medskip


\paragraph{\bf Claim 6 (Decorated tubing).}
{\it 
By taking larger $R$, $r$, and $\rho$ if necessary, 
there exist a $\rho'>0$ and a tubing
$$
\check{\Theta}
:=
\braces{
\check{\Theta}_c: \Bar{A(\check{R}, \check{R}^2)} \to \Bar{U_c} \sminus {V_c'}
}_{c\, \in \,W}
$$
of the family $\bs{G}$ with the following properties:
\begin{enumerate}[\rm (1)]
\item
$\check{R}=\rho/\rho'$ and 
$\Gamma_0(c_0+\eta)=\Gamma_0(c_0+\eta)_{\rho',\rho}$
is contained in $A(\check{R}, \check{R}^2)$. 
\\[-.9em]
\item
$\check{\Theta}_s(\check{z})
=h_s^{-1}\paren{((\rho')^{\,2}/\rho) \cdot \check{z}}$ 
for $\check{z} \in \Gamma_0(c_0+\eta)$
such that $\check{\Theta}_s$ maps $\Gamma_0(c_0+\eta)$ onto $J(f_s)$.
\\[-.7em]
\item 
Each $\check{\Theta}_c:\Bar{A(\check{R}, \check{R}^2)} \to \Bar{U_c} \sminus {V_c'}$
is a $(1+\kappa)$-quasiconformal embedding
that is compatible with the holomorphic motion of 
$\Bar{U_s}\sminus V_s'$ over $W$ given in (2) of Claim 5. 
More precisely, 
we have $\check{\Theta}_c=\iota_c \cc \check{\Theta}_s$
for each $c \in W$, 
where $\iota_c: \Bar{U_s}\sminus V_s' \to \Bar{U_c}\sminus V_c'$ is 
the quasiconformal map 
induced by the motion.
\end{enumerate}
}

We call this tubing $\check{\Theta}$ 
a {\it decorated tubing} of $\bs{G}$.

\paragraph{\bf Proof.}
For each $c \in \widehat{W}$,
the map ${G}_c=P_c^{pL+N}|_{\widehat{V}_c'}$ 
can be decomposed as
${G}_c=Q_c \cc P_0$, where $P_0(z)=z^2$ and
$Q_c:P_0(\widehat{V}_c') \to \widehat{U}_c$ is an isomorphism. 
Let 
$$
\beta_c:=Q_c'(0)
\quad \text{and} \quad 
\gamma_c:=Q_c(0).
$$
Note that $\gamma_c={G}_c(0) \in U_c$ if $c \in W$.

Since $\widehat{U}_c$ and $U_c$ are round disks 
of radii $R|\al_c|$ and $\rho|\al_c|$ respectively, 
we apply the Koebe distortion theorem to 
$Q_c^{-1}:\widehat{U}_c \to P_0(\widehat{V}_c')$
and obtain 
\begin{equation}\label{eq_Q_c_inverse}
z=Q_c^{-1}(w)=\beta_c^{-1}(w-\gamma_c)\paren{1+O(\rho/R)}
\end{equation}
for $w \in U_c$.
Indeed, by the Koebe distortion theorem (\cite[\S 2.3]{D 1983}),
we have 
$$
\abs{\frac{(Q_c^{-1})'(w)}{(Q_c^{-1})'(\gamma_c)}}=1+O(\rho/R)
\quad
\text{and}
\quad
\arg\frac{(Q_c^{-1})'(w)}{(Q_c^{-1})'(\gamma_c)}=O(\rho/R)
$$
for $w \in U_c$.
Hence we have $(Q_c^{-1})'(w)=\beta_c^{-1}(1+O(\rho/R))$ on $U_c$.
By integrating the function $(Q_c^{-1})'(w)-\beta_c^{-1}$ 
along the segment joining $\gamma_c$ to $w$ in $U_c$, 
we obtain
$$
|Q_c^{-1}(w)-\beta_c^{-1}(w-\gamma_c)|
=|w-\gamma_c||\beta_c|^{-1}O(\rho/R)
$$
that is equivalent to \eqref{eq_Q_c_inverse}.


This implies that
$$
G_c(z)=Q_c (z^2)=\gamma_c + \beta_c z^2\, \paren{1+O(\rho/R)}
$$ 
on $V_c'$.
By an affine coordinate change 
$$
\check{z}=\check{A}_c(z):=\beta_c\, z,
$$
we obtain a quadratic-like map $\check{G}_c:\check{V}_c' \to \check{U}_c$
of the form 
\begin{equation}
\check{w}=\check{G}_c(\check{z}):=\check{A}_c \cc G_c \cc \check{A}_c^{-1}(\check{z})
=\beta_c\gamma_c+\check{z}^2 
\paren{1+O(\rho/R)},
\label{eq_G_check}
\end{equation}
where $\check{V}_c':=\check{A}_c(V_c')$ and 
$\check{U}_c:=\check{A}_c(U_c)=D(\rho|\al_c||\beta_c|)$.


Now suppose that $c=s$.
Then the condition $G_s(0)=0$ implies $\gamma_s=0$.
Hence we have 
\begin{equation}\label{eq_G_check_s}
\check{w}=\check{G}_s(\check{z})=\check{z}^2(1+O(\rho/R))
\quad \text{and} \quad
\check{z}=\check{G}_s^{-1}(\check{w})=\sqrt{\check{w}(1+O(\rho/R))}.
\end{equation}
Let $\check{R}:=(\rho|\al_s||\beta_s|)^{1/2}$.
Then $\check{U}_s=D(\check{R}^2)$
and $\partial \check{V}_s'$ is
$C^1$-close to a circle $\partial D(\check{R})$.
In other words, the annulus 
$\mathcal{A}:=\check{U}_s \sminus \overline{\check{V}_s'}$
is close to a round annulus $A(\check{R}, \check{R}^2)$.
Moreover, the annulus $\mathcal{A}$ contains the compact set
$\mathcal{J}:=\check{A}_s(J(f_s))=J(f_s) \times \beta_s$.

Let us define $\rho'>0$ such that $\rho/\rho'=\check{R}$, i.e.,
$$
\rho':=\frac{\rho}{\check{R}}=\paren{\frac{\rho}{|\al_s||\beta_s|}}^{1/2}.
$$
By taking a sufficiently large $N$ in Claim 1,
we may assume that the diameter of $V_s'$ is sufficiently small
(equivalently, $|\beta_s|$ is sufficiently large,
and thus $\rho'$ is sufficiently small) such that
$$
J_{c_0+\eta} \subset A(\rho',\rho). 
$$
Hence the rescaled Julia set  
$$
\mathcal{J}_0:=\Gamma_0(c_0+\eta)_{\rho',\rho}
= J_{c_0+\eta} \times \frac{\rho}{(\rho')^2}
$$
is contained in the annulus $A(\check{R}, \check{R}^2)$. 

\begin{lem}\label{lem_pre_tubing}
There exists a $(1+\kappa)^{1/2}$-quasiconformal map 
$\Psi:\Bar{A(\check{R}, \check{R}^2)} 
\to \Bar{\mathcal{A}}=\Bar{\check{U}_s} \sminus {\check{V}_s'}$
 such that $\Psi(\mathcal{J}_0)=\mathcal{J}$ and 
 $\Psi(\check{z}^2)=\check{G}_s(\Psi(\check{z}))$ 
 for any $\check{z} \in \partial D(\check{R})$
 by taking sufficiently large $R,\,\rho,\,\check{R}$ with sufficiently small $\rho/R$.
\end{lem}



\paragraph{\bf Proof of Lemma \ref{lem_pre_tubing}.}
We will construct such a $\Psi$ for $\partial A(\check{R}, \check{R}^2)$
and for $\mathcal{J}_0$ separately, 
then use the Bers-Royden theorem 
to extend it to $\overline{A(\check{R}, \check{R}^2)}$. 

Let us start with the boundary of the annulus:
By (\ref{eq_G_check_s}), we have 
$$
\log \check{G}_s^{-1}(\check{w})
= \frac{1}{2}\log \check{w}+ \frac{1}{2}\log\paren{1+O(\rho/R)}
= \frac{1}{2}\log \check{w}+ O(\rho/R)
$$
near $\partial \check{U}_s=\partial D(\check{R})$.
Hence for sufficiently large $\check{R}$ and small $\rho/R$,
we may apply the same argument as the proof of 
Lemma \ref{lem_almost_conformal}
by regarding $R$ in Lemma \ref{lem_almost_conformal} as $\check{R}^2$.
Indeed, we let 
$$
\check{\Upsilon}(\check{w}):=  
\log \check{G}_s^{-1}(\check{w})-\frac{1}{2}\log \check{w}
=O(\rho/R)
$$
and $\check{v}(t):=\check{\Upsilon}(\check{w}(2t))$,
where $\check{w}(2t)=\check{R}^2e^{2t i}$
makes two turns along $\partial D(\check{R}^2)$ as $t$ varies from $0$ to $2\pi$.
Set $\check{\ell}:=\log \check{R}$ and $\check{\eta}(s):=\eta_0(s/\check{\ell}-1)$,
where $\eta_0$ is defined in the proof of Lemma \ref{lem_almost_conformal}.
Then the map
$$
\check{\theta}_\xi(s+it):=s+it + \xi \check{\eta}(s)\check{v}(t)
$$
with a parameter $\xi \in \C$ 
is defined for $(s,t) \in [\check{\ell}, 2 \check{\ell}] \times [0,2\pi]$,
and $\check{\theta}_\xi$ is a $(1+O(|\xi|\rho/R))$-quasiconformal map
for sufficiently large $\check{R}$ and small $\rho/R$.
We fix such $\rho, R$ and $\check{R}$,
and obtain a holomorphic family of injections
$\braces{\check{\theta}_\xi}_{\xi}$
with parameter $\xi$ in a disk $D(d_0)$  of radius $d_0 \asymp R/\rho$.
(This bound comes from the estimate like (\ref{eq_injectivity})
that ensures injectivity.)
By observing the motion through the exponential map,
we obtain a holomorphic motion 
$\psi: \partial A(\check{R},\check{R}^2) \times D(d_0) \to \C$
of $\partial A(\check{R},\check{R}^2)$ over $D(d_0)$
with $\psi(\check{z},\xi):=\psi_\xi(\check{z})$.
In particular, by letting $\xi:=1$, 
the map $\Psi:= \psi_1$  
satisfies 
 $\Psi(\check{z}^2)=\check{G}_s(\Psi(\check{z}))$
 by construction.


Next we consider the Julia set:
Let $\Psi:\mathcal{J}_0 \to \mathcal{J}$ be the quasiconformal map
given by composing the four maps
$$
\mathcal{J}_0 
=\Gamma(c_0+\eta)_{\rho',\,\rho}
\stackrel{(1)}{\longrightarrow} 
J_{c_0+\eta} 
\stackrel{(2)}{\longrightarrow} 
J(F_{X_n(s)})
\stackrel{(3)}{\longrightarrow} 
J(f_s)
\stackrel{(4)}{\longrightarrow} 
\mathcal{J},
$$
where  (1) is the affine map $\check{z} \mapsto (\rho/(\rho')^2)^{-1} \check{z}$;
(2) is the inverse of the $(1+O(R^{-1}))$-quasiconformal 
straightening $H:=H_{X_n(s)}$ of $F_{X_n(s)}$ to $P_{c_0+\eta}$;  
(3) is the inverse of the affine map 
${A}_s: z \mapsto z/\al_s$;
and 
(4) 
is the affine map $\check{A}_s:\check{z} \mapsto \beta_s \check{z}$.
The straightening map $H$ in (2) 
extends to a $(1+O(R^{-1}))$-quasiconformal map 
on the plane as in Lemma \ref{lem_almost_conformal}.
Let $\check{\mu}:=\mu_{H^{-1}}$ be the Beltrami coefficient of the {\it inverse} of 
such an extended $H$ with $\norm{\check{\mu}}_\infty=O(R^{-1}).$
Then the Beltrami equation for $\check{\mu}_\xi:=\xi \cdot \check{\mu}$ 
with a complex parameter $\xi$ 
has a solution if $\xi \in D(d_1)$ with $d_1 \asymp R$.
Let $\phi_\xi$ be the unique normalized solution 
such that $\phi_\xi(0)=0$ and $(\phi_\xi)_z-1 \in L^{p'}(\C)$ 
for some $p'>2$.
Then the map 
$\psi_\xi(\check{z})
:=\check{A}_s \cc A_s^{-1} \cc 
\phi_\xi ((\rho/(\rho')^2)^{-1} \check{z})$ 
gives a holomorphic motion
$\psi: \mathcal{J}_0 \times D(d_1) \to \C$
of $\mathcal{J}_0$ over $D(d_1)$ with $\psi(\check{z}, \xi)=\psi_\xi(\check{z})$.
In particular, by letting $\xi:=1$, the map 
\begin{equation}\label{eq_Psi}
\Psi(\check{z}):=\psi_1(\check{z})
=\check{A}_s \cc A_s^{-1} \cc 
H_{X_n(s)}^{-1} ((\rho/(\rho')^2)^{-1} \check{z})
=\check{A}_s \cc h_s^{-1} ((\rho/(\rho')^2)^{-1} \check{z})
\end{equation}
satisfies $\Psi(\mathcal{J}_0) = \mathcal{J}$.


Now by taking $R$ relatively larger than $\rho$,
 we may assume that $d_0 <d_1$.
Let us check that the unified map 
$\psi_\xi: \partial A(\check{R},\check{R}^2) \cup \mathcal{J}_0 \to \C$
gives a holomorphic family of injections for $\xi \in D(d_0)$
if $\check{R}$ is sufficiently large.
Indeed, it is enough to check that the distance between 
$\psi_\xi(\partial A(\check{R},\check{R}^2))$
and $\psi_\xi(\mathcal{J}_0)$ is bounded from below
for $\xi \in D(d_0)$.

Let us fix a constant $0<\sigma<1$ such that 
$J_{c_0+\eta} \subset A(\sigma \rho,\rho)$.
Hence $\dist(0, \mathcal{J}_0) > \sigma \check{R}^2$.
Note that we can replace $\check{R}$ by an arbitrarily larger one 
with only a slight change of $\sigma$, 
because in Claim 1 we can replace $\widehat{V}_{c_1}$ 
by an arbitrarily smaller one 
such that the location of the center $s$ of $\widehat{W}$
changes only a little (relatively to the size of $\widehat{\Lambda}$). 
Hence we may assume that $\check{R}$ is large enough such 
that
$\dist(\partial D(\check{R}), \mathcal{J}_0)
\ge \dist(0, \mathcal{J}_0) -\check{R}
> \check{R}^2(\sigma -1/\check{R})
 \asymp \check{R}^2$.
By taking a sufficiently large $\rho$, 
we have $J_{c_0+\eta} \subset D(\rho/2)$ 
and thus $\dist(\partial D(\check{R}^2), \mathcal{J}_0) \ge \check{R}^2/2$.
Hence we conclude that 
$\dist(\partial A(\check{R},\check{R}^2), \mathcal{J}_0) \asymp \check{R}^2$.

Now suppose that $\xi \in D(d_0)$.
Since $d_0 \asymp R/\rho$,
an explicit calculation shows that 
$\psi_\xi(\check{z})=\check{z}$ on $\partial D(\check{R}^2)$
and 
$\psi_\xi(\check{z})=\check{z}(1+\xi O(\rho/R))$
on $\partial D(\check{R})$.
Hence $\dist(0, \psi_\xi(\partial D(\check{R}))) \asymp \check{R}$.

On the other hand, 
since $\check{\mu}_\xi=O(|\xi| R^{-1}) =O(\rho^{-1})$ for $\xi \in D(d_0)$,
we have $\phi_\xi(z)=z+O(\rho^{-1+2/p'})$
 for some $p' >2$ on $J_{c_0+\eta}$
 (cf. Lemma \ref{lem_estimate_of_h}).
Hence 
$\dist(\mathcal{J}_0, \psi_\xi(\mathcal{J}_0)) 
\asymp O(\rho^{-1+2/p'}) \check{R}^2$ 
for sufficiently large $\rho$.
It follows that if $\rho$, $R$, and $\check{R}$ are sufficiently large
and $\rho/R$ are sufficiently small,
then we have 
\begin{align*}
&\dist(\psi_\xi(\partial A(\check{R},\check{R}^2)),
\psi_\xi(\mathcal{J}_0)) \\
\ge~ & 
\dist(\partial A(\check{R},\check{R}^2),
\mathcal{J}_0)
-\dist(\partial A(\check{R},\check{R}^2),
\psi_\xi(\partial A(\check{R},\check{R}^2)))
 -\dist(\mathcal{J}_0,  \psi_\xi(\mathcal{J}_0))\\
 \asymp~ 
 & \check{R}^2(1 - O(1/\check{R}) -O(\rho^{-1+2/p'})) 
\asymp \check{R}^2 
\end{align*}
for $\xi \in D(d_0)$.

By applying the Bers-Royden theorem
to the holomorphic motion $\psi$ of 
$\partial A(\check{R},\check{R}^2) \cup \mathcal{J}_0$
over $D(d_0)$,
the injection $z \mapsto \psi_1(z)=\psi(z,1)$
extends to a quasiconfomal map on $\C$
whose dilatation is bounded by $1+O(1/d_0)=1+O(\rho/R)$.
It is $(1+\kappa)^{1/2}$-quasiconformal 
 by taking $R$ relatively larger than $\rho$.
Thus the restriction $\Psi$ of $\psi_1$
on $\Bar{A(\check{R}, \check{R}^2)}$
is our desired map. 
\QED ( Lemma \ref{lem_pre_tubing})

 
\medskip

\paragraph{\bf Proof of Claim 6, continued.} 
Let $\check{\Theta}_s:=\check{A}_s^{-1}\cc \Psi$
(hence 
$\check{\Theta}_s(\check{z})=h_s^{-1} ((\rho/(\rho')^2)^{-1} \check{z})$
for $\check{z} \in \Gamma(c_0+\eta)_{\rho',\,\rho}$ by
(\ref{eq_Psi}))
and $\check{\Theta}_c:=\iota_c \cc \check{\Theta}_s$ for $c \in {W}$,
where $\iota_c: \Bar{U_s}  \sminus  {V_s'} \to \Bar{U_c}  \sminus  {V_c'}$
is a $(1+\kappa)^{1/2}$-quasiconformal map given in Claim 5.
Then $\check{\Theta}_c$ is $(1+\kappa)$-quasiconformal for each $c \in W$
with desired properties.
\QED 



\medskip

\paragraph{\bf Almost conformal embedding of the model.}
We finish the proof of Theorem C by the next claim:

\paragraph{\bf Claim 7 (Almost conformal straightening).}
{\it 
The family 
$$
\bs{G}= 
\{G_c: {V}_c' \to {U}_c\}_{c  \,\in \,W} 
$$
is a Mandelbrot-like family whose straightening map
$\chi_{\bs{G}}: W \to \C$ associated with the decorated tubing 
$\check{\Theta}$ is $(1+\kappa)$-quasiconformal. 
Moreover, the inverse of $\chi_{\bs{G}}$ realizes a $(1+\kappa)$-quasiconformal embedding 
of the model $\cM(c_0+\eta)_{\rho',\rho}$.
}

\medskip

\paragraph{\bf Proof.}
By Claim 6, the family $\bs{G}$ is equipped with the decorated 
tubing $\check{\Theta}=\{\check{\Theta}_c\}_{c \, \in \, W}$ 
and hence Mandelbrot-like with connectedness locus $M_{\bs{G}}$
homeomorphic to $M$.
The straightening map $\chi_{\bs{G}}: W \to \C$ associated with 
this tubing is given by $\chi_{\bs{G}}(c):=\check{h}_c(G_c(0))$,
where $\check{h}_c:U_c \to \C$ is the $(1+\kappa)$-quasiconformal 
straightening map of $G_c$ constructed from 
the $(1+\kappa)$-quasiconformal map $\check{\Theta}_c$ 
by the same way as the proof of Lemma \ref{lem_almost_conformal}.

Let us show that the map $\chi_{\bs{G}}$ is $(1+\kappa)$-quasiconformal
by following the argument of Lemma \ref{lem_almost_conformal_M}:
It is $(1+\kappa)$-quasiconformal 
on $W \sminus  M_{\bs{G}}$ because 
$\check{\Theta}_c$ is $(1+\kappa)$-quasiconformal for any $c \in W$.
By Lemma \ref{lem_Bers} (Bers' Gluing Lemma),
it is enough to show that
$\chi_{\bs{G}}: M_{\bs{G}} \to M$
extends to a $(1+\kappa)$-quasiconformal map on $\C$. 

Recall that 
the quadratic-like map $\check{G}_c:\check{V}_c' \to \check{U}_c$
given in (\ref{eq_G_check}) is of the form 
$$
\check{w}=\check{G}_c(\check{z})=
\check{A}_c \cc G_c \cc \check{A}_c^{-1}(\check{z})
=\check{z}^2 +\check{c}+\check{u}(\check{z},\check{c}),
$$
where $\check{c}=\check{X}(c):=\beta_c\gamma_c$ and 
$\check{u}(\check{z},\check{c})=\check{z}^2 \,O(\rho/R)$.
Moreover, $\check{u}(\check{z},\check{c}) =\check{u}_{\check{c}}(\check{z})$
 satisfies $\check{u}_{\check{c}}'(0)=0$,
 and the value 
$$
\check{\delta}:=
\sup \{|\check{u}(\check{z},\check{c})| \st (\check{z},\check{c}) 
\in \overline{D(4)}\times \overline{D(4)}\}
$$
is $O(\rho/R)$.
We may assume that $\rho/R$ is small enough such that $\check{\delta}<1$.
As in the proof of Lemma \ref{lem_almost_conformal_M},
we consider the analytic family
$$
\check{\bs G}(t)=
\skakko{\check{z} \mapsto \check{z}^2+\check{c}
+t \check{u}(\check{z},\check{c})/\check{\delta}}_{\check{c}\, \in \, D(4)}
$$
with parameter $t \in \D$ 
whose connectedness locus $\check{M}(t)$ is homeomorphic to $M$.
Then $\check{\bs G}(\check{\delta})=\{\check{G}_c\}_{c \,\in \, W}$
and the straightening map 
$\check{\chi}:\check{M}(\check{\delta}) \to M$ 
extends to a quasiconformal map on $\C$
with dilatation $1+O(\check{\delta})=1+O(\rho/R)$
by the Bers-Royden theorem.
Hence if $\rho/R$ is sufficiently small, 
$\check{\chi}$ is $(1+\kappa)$-quasiconformal.
Since $\chi_{\bs G}(c) =\check{\chi}(\check{X}(c))$ and
$\check{X}(c)=\beta_c \,\gamma_c$ is holomorphic near $M_{\bf G}$,
we conclude that $\chi_{\bs{G}}: M_{\bs{G}} \to M$
extends to a $(1+\kappa)$-quasiconformal map on $\C$.


As in the proof of Theorem A, the inverse of $\chi_{\bs G}(c)$ 
realizes a $(1+\kappa)$-quasiconformal embedding of 
$\cM(c_0+\eta)_{\rho',\rho}$ into $W$.
\QED

\medskip 

